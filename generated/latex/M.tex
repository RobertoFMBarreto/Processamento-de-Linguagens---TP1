
\begin{itemize}
\item {Proveniência: }
\end{itemize}\documentclass{article}
\usepackage[portuguese]{babel}
\title{M}
\begin{document}
Planta, de flôres vermelhas, do typo da lythraríadas.
\section{Macroglossia}
\begin{itemize}
\item {Grp. gram.:f.}
\end{itemize}
\begin{itemize}
\item {Utilização:Med.}
\end{itemize}
\begin{itemize}
\item {Proveniência:(De \textunderscore macroglosso\textunderscore )}
\end{itemize}
Aumento monstruoso do volume da língua.
\section{Maiorista}
\begin{itemize}
\item {Grp. gram.:m.}
\end{itemize}
\begin{itemize}
\item {Utilização:Neol.}
\end{itemize}
\begin{itemize}
\item {Proveniência:(De \textunderscore maioria\textunderscore )}
\end{itemize}
Partidário do systema eleitoral, que se baseia na maioria dos votos.
\section{Masochismo}
\begin{itemize}
\item {fónica:qui}
\end{itemize}
\begin{itemize}
\item {Grp. gram.:m.}
\end{itemize}
\begin{itemize}
\item {Utilização:Med.}
\end{itemize}
Perversão da sensibilidade genital, caracterizada pelo facto de que o acto sexual, em vez de sêr provocado pelas suas causas habituais, precisa de maus tratos e violências, como a flagellação das nádegas e dos rins, etc.
\section{Masoquismo}
\begin{itemize}
\item {Grp. gram.:m.}
\end{itemize}
\begin{itemize}
\item {Utilização:Med.}
\end{itemize}
Perversão da sensibilidade genital, caracterizada pelo facto de que o acto sexual, em vez de sêr provocado pelas suas causas habituais, precisa de maus tratos e violências, como a flagelação das nádegas e dos rins, etc.
\section{Massinha}
\begin{itemize}
\item {Grp. gram.:f.}
\end{itemize}
Massa miúda para sopa.
\section{Mastite}
\begin{itemize}
\item {Grp. gram.:f.}
\end{itemize}
\begin{itemize}
\item {Utilização:Med.}
\end{itemize}
\begin{itemize}
\item {Proveniência:(Do gr. \textunderscore mastos\textunderscore )}
\end{itemize}
Designação genérica de inflammações nas mamas.
\section{Matias}
\begin{itemize}
\item {Grp. gram.:m.}
\end{itemize}
\begin{itemize}
\item {Utilização:Pop.}
\end{itemize}
\begin{itemize}
\item {Proveniência:(De \textunderscore Mathias\textunderscore , n. p.)}
\end{itemize}
Pateta; palerma.
\section{Matim}
\begin{itemize}
\item {Grp. gram.:m.}
\end{itemize}
\begin{itemize}
\item {Utilização:Ant.}
\end{itemize}
O mesmo que \textunderscore manhã\textunderscore .
(Cp. fr. \textunderscore matin\textunderscore )
\section{Meia-missa}
\begin{itemize}
\item {Grp. gram.:f. Loc.}
\end{itemize}
\begin{itemize}
\item {Utilização:pop.}
\end{itemize}
\textunderscore Não saber meia-missa\textunderscore , ou \textunderscore não saber de missa metade\textunderscore , estar mal informado, conhecer só uma parte do assumpto que se trata. Cf. Júl. Moreira, \textunderscore Estudos\textunderscore , II, 247.
\section{Meio-serviço}
\begin{itemize}
\item {Grp. gram.:m.}
\end{itemize}
Metade das peças de loiça, que constituem ordinariamente um \textunderscore serviço de mesa\textunderscore : \textunderscore adquiriu num leilão meio-serviço\textunderscore .
\section{Menospreçar}
\begin{itemize}
\item {Grp. gram.:v. t.}
\end{itemize}
\begin{itemize}
\item {Utilização:Ant.}
\end{itemize}
O mesmo que \textunderscore menosprezar\textunderscore . Cf. \textunderscore Rev. Lus.\textunderscore  XVI, 8.
\section{Meroça}
\begin{itemize}
\item {Grp. gram.:f.}
\end{itemize}
\begin{itemize}
\item {Utilização:Prov.}
\end{itemize}
\begin{itemize}
\item {Utilização:trasm.}
\end{itemize}
Socalco de vinha, geio, arrêto.
(Relaciona-se com \textunderscore moroiço\textunderscore ?)
\section{Microdulia}
\begin{itemize}
\item {Grp. gram.:f.}
\end{itemize}
\begin{itemize}
\item {Utilização:Neol.}
\end{itemize}
\begin{itemize}
\item {Proveniência:(Do gr. \textunderscore mikros\textunderscore  + \textunderscore douleia\textunderscore )}
\end{itemize}
Exaggerada importância ou consideração, dada a pequenas coisas, a coisas insignificantes.
\section{Micrograma}
\begin{itemize}
\item {Grp. gram.:m.}
\end{itemize}
Milionésima parte da grama. Cf. \textunderscore Decretos\textunderscore  de 19 e 20-IV-911.
\section{Microgramma}
\begin{itemize}
\item {Grp. gram.:m.}
\end{itemize}
Millionésima parte da gramma. Cf. \textunderscore Decretos\textunderscore  de 19 e 20-IV-911.
\section{Microlitro}
\begin{itemize}
\item {Grp. gram.:m.}
\end{itemize}
Millionésima parte do litro. Cf. \textunderscore Decretos\textunderscore  de 19 e 20-IV-911.
\section{Mícron}
\begin{itemize}
\item {Grp. gram.:m.}
\end{itemize}
Unidade de medida, adoptada em micrographia, e correspondente á millésima parte do millímetro. Cf. \textunderscore Decretos\textunderscore  de 19 e 20-IV-911.
(Cp. gr. \textunderscore mikros\textunderscore )
\section{Microrchia}
\begin{itemize}
\item {fónica:qui}
\end{itemize}
\begin{itemize}
\item {Grp. gram.:f.}
\end{itemize}
\begin{itemize}
\item {Utilização:Med.}
\end{itemize}
\begin{itemize}
\item {Proveniência:(Do gr. \textunderscore mikros\textunderscore  + \textunderscore orkhis\textunderscore )}
\end{itemize}
Pequenez excessiva dos testículos.
\section{Microrchidia}
\begin{itemize}
\item {fónica:qui}
\end{itemize}
\begin{itemize}
\item {Grp. gram.:f.}
\end{itemize}
(Fórma afrancesada e injustificável, em vez de \textunderscore microrchia\textunderscore . V. \textunderscore microrchia\textunderscore )
\section{Microrquia}
\begin{itemize}
\item {Grp. gram.:f.}
\end{itemize}
\begin{itemize}
\item {Utilização:Med.}
\end{itemize}
\begin{itemize}
\item {Proveniência:(Do gr. \textunderscore mikros\textunderscore  + \textunderscore orkhis\textunderscore )}
\end{itemize}
Pequenez excessiva dos testículos.
\section{Microrquidia}
\begin{itemize}
\item {Grp. gram.:f.}
\end{itemize}
(Fórma afrancesada e injustificável, em vez de \textunderscore microrchia\textunderscore . V. \textunderscore microrchia\textunderscore )
\section{Micrótomo}
\begin{itemize}
\item {Grp. gram.:m.}
\end{itemize}
\begin{itemize}
\item {Utilização:Med.}
\end{itemize}
\begin{itemize}
\item {Proveniência:(Do gr. \textunderscore mikros\textunderscore  + \textunderscore tome\textunderscore )}
\end{itemize}
Instrumento, com que se cortam, para estudo, as lâminas mais finas de tecido.
\section{Miosalgia}
\begin{itemize}
\item {Grp. gram.:f.}
\end{itemize}
\begin{itemize}
\item {Utilização:Med.}
\end{itemize}
O mesmo que \textunderscore mialgia\textunderscore .
\section{Moestar}
\begin{itemize}
\item {fónica:mo-es}
\end{itemize}
\begin{itemize}
\item {Grp. gram.:v. t.}
\end{itemize}
\begin{itemize}
\item {Utilização:Ant.}
\end{itemize}
O mesmo que \textunderscore admoestar\textunderscore .
\section{Moesteiro}
\begin{itemize}
\item {fónica:mo-es}
\end{itemize}
\begin{itemize}
\item {Grp. gram.:m.}
\end{itemize}
\begin{itemize}
\item {Utilização:Ant.}
\end{itemize}
O mesmo que \textunderscore mosteiro\textunderscore .
\section{Môr}
\begin{itemize}
\item {Grp. gram.:m.}
\end{itemize}
\begin{itemize}
\item {Utilização:ant.}
\end{itemize}
\begin{itemize}
\item {Utilização:Pop.}
\end{itemize}
Causa, motivo:«\textunderscore leva a capa, por môr da chuva.\textunderscore »Cf. \textunderscore Palmeirim de Ingl.\textunderscore , III, 34.
(Aphér. de \textunderscore amor\textunderscore )
\section{Morfosista}
\begin{itemize}
\item {Grp. gram.:m.}
\end{itemize}
\begin{itemize}
\item {Utilização:Neol.}
\end{itemize}
\begin{itemize}
\item {Proveniência:(De \textunderscore morphose\textunderscore )}
\end{itemize}
Aquelle que dá fórma, que reproduz ou que representa:«\textunderscore Brücke e Stratz, morfosistas da beleza humana.\textunderscore »R. Jorge, \textunderscore El Greco\textunderscore , 37.
\section{Morphosista}
\begin{itemize}
\item {Grp. gram.:m.}
\end{itemize}
\begin{itemize}
\item {Utilização:Neol.}
\end{itemize}
\begin{itemize}
\item {Proveniência:(De \textunderscore morphose\textunderscore )}
\end{itemize}
Aquelle que dá fórma, que reproduz ou que representa:«\textunderscore Brücke e Stratz, morphosistas da beleza humana.\textunderscore »R. Jorge, \textunderscore El Greco\textunderscore , 37.
\section{Muco-pus}
\begin{itemize}
\item {Grp. gram.:f.}
\end{itemize}
\begin{itemize}
\item {Utilização:Med.}
\end{itemize}
Muco, que tem a apparência de pus, pela abundância dos licócytos que contém.
\section{Myosalgia}
\begin{itemize}
\item {Grp. gram.:f.}
\end{itemize}
\begin{itemize}
\item {Utilização:Med.}
\end{itemize}
O mesmo que \textunderscore myalgia\textunderscore .
\section{Mudéxar}
\begin{itemize}
\item {Grp. gram.:m.}
\end{itemize}
(V.mudéjar). Cf. R. Jorge, \textunderscore El Greco\textunderscore , 20.
\section{Munichense}
\begin{itemize}
\item {fónica:quen}
\end{itemize}
\begin{itemize}
\item {Grp. gram.:adj.}
\end{itemize}
Relativo a Munich; natural de Munich:«\textunderscore o crítico munichense.\textunderscore »R. Jorge, \textunderscore El Greco\textunderscore , 22.
\section{Mateu}
\begin{itemize}
\item {Grp. gram.:m.}
\end{itemize}
\begin{itemize}
\item {Utilização:Prov.}
\end{itemize}
\begin{itemize}
\item {Utilização:trasm.}
\end{itemize}
Qualquer mata.
\section{Mathíola}
\begin{itemize}
\item {Grp. gram.:f.}
\end{itemize}
\begin{itemize}
\item {Proveniência:(De \textunderscore mathioli\textunderscore , n. p.)}
\end{itemize}
Gêneros de plantas crucíferas.
\section{Mathusalém}
\begin{itemize}
\item {Grp. gram.:m.}
\end{itemize}
\begin{itemize}
\item {Utilização:Fam.}
\end{itemize}
\begin{itemize}
\item {Proveniência:(De \textunderscore Mathusalém\textunderscore , n. p.)}
\end{itemize}
Homem muito velho; macróbio.
\section{Matusalém}
\begin{itemize}
\item {Grp. gram.:m.}
\end{itemize}
\begin{itemize}
\item {Utilização:Fam.}
\end{itemize}
\begin{itemize}
\item {Proveniência:(De \textunderscore Mathusalém\textunderscore , n. p.)}
\end{itemize}
Homem muito velho; macróbio.
\section{Matical}
\begin{itemize}
\item {Grp. gram.:m.}
\end{itemize}
O mesmo que \textunderscore metical\textunderscore :«\textunderscore os maticaes berbericos...\textunderscore »F. Manuel, \textunderscore Apólogos\textunderscore .
\section{Maticina}
\begin{itemize}
\item {Grp. gram.:f.}
\end{itemize}
Princípio amargo, extrahido do mático.
\section{Mático}
\begin{itemize}
\item {Grp. gram.:m.}
\end{itemize}
Árvore piperácea do Peru.
\section{M}
\begin{itemize}
\item {fónica:ême}
\end{itemize}
\begin{itemize}
\item {Grp. gram.:m.}
\end{itemize}
Décima terceira letra do nosso alphabeto.
Em numeração romana, significa \textunderscore mil\textunderscore .
Em grammática, é abrev. de \textunderscore masculino\textunderscore .
No systema legal de medidas, e á direita de um número, designa \textunderscore metro\textunderscore .
\section{Má}
\begin{itemize}
\item {Grp. gram.:adj.}
\end{itemize}
\begin{itemize}
\item {Grp. gram.:F.}
\end{itemize}
(flexão fem. de \textunderscore mau\textunderscore )
Tumor, o mesmo que \textunderscore arrieira\textunderscore .
(Contr. do ant. \textunderscore maa\textunderscore , do lat. \textunderscore mala\textunderscore )
\section{Mâ}
\begin{itemize}
\item {Grp. gram.:f.}
\end{itemize}
Cânhamo da Índia ou de Manila.
\section{Maba}
\begin{itemize}
\item {Grp. gram.:m.}
\end{itemize}
Gênero de plantas ebenáceas da Índia.
\section{Mabaça}
\begin{itemize}
\item {Grp. gram.:m.  e  adj.}
\end{itemize}
\begin{itemize}
\item {Utilização:Bras}
\end{itemize}
Adherente a outro, (falando-se de homem, animal ou fruto).
Gêmeo.
\section{Mabala}
\begin{itemize}
\item {Grp. gram.:f.}
\end{itemize}
Planta leguminosa angolense, (\textunderscore psophocarpus longepedunculatus\textunderscore , Hassk.), de sementes alimentares.
Espécie de tecido de algodão.--Os diccionários dizem erradamente \textunderscore mabata\textunderscore . Cf. B. C. Rubim, \textunderscore Vocab. Bras.\textunderscore 
\section{Mabate}
\begin{itemize}
\item {Grp. gram.:f.}
\end{itemize}
Espécie de carraça africana, que vive nas fendas como os percevejos, e ataca o homem em-quanto êste dorme.
\section{Mabeco}
\begin{itemize}
\item {Grp. gram.:m.}
\end{itemize}
\begin{itemize}
\item {Utilização:T. de Angola}
\end{itemize}
Cão feroz dos matos, (\textunderscore canis mesomelus\textunderscore ?).
\section{Mabeia}
\begin{itemize}
\item {Grp. gram.:f.}
\end{itemize}
Planta euphorbiácea da Guiana.
\section{Mabela}
\begin{itemize}
\item {Grp. gram.:f.}
\end{itemize}
Árvore africana, de bôa madeira para construcções, e com cujas fôlhas os indígenas tecem barretes.
\section{Mabiús}
\begin{itemize}
\item {Grp. gram.:m. pl.}
\end{itemize}
Indígenas brasileiros das margens do Japurá.
\section{Mablemblê}
\begin{itemize}
\item {Grp. gram.:m.}
\end{itemize}
Árvore medicinal da ilha de San-Thomé.
\section{Maboca}
\begin{itemize}
\item {Grp. gram.:f.}
\end{itemize}
Árvore angolense, o mesmo que \textunderscore maboque\textunderscore .
\section{Mabole}
\begin{itemize}
\item {Grp. gram.:m.}
\end{itemize}
Árvore africana, da fam. das loganiáceas, de grandes bagas amarelas, que são comestíveis.
\section{Maboque}
\begin{itemize}
\item {Grp. gram.:m.}
\end{itemize}
(V.mabole)
\section{Mabouiá}
\begin{itemize}
\item {fónica:bou-i}
\end{itemize}
\begin{itemize}
\item {Grp. gram.:f.}
\end{itemize}
\begin{itemize}
\item {Utilização:Bras}
\end{itemize}
Planta medicinal, (\textunderscore morisonia americana\textunderscore ).
\section{Maboulá}
\begin{itemize}
\item {Grp. gram.:f.}
\end{itemize}
Árvore medicinal do Brasil.--É possível que os diccionários dêem \textunderscore mabouiá\textunderscore  por \textunderscore maboulá\textunderscore , ou que os botânicos se tenham equivocado, escrevendo \textunderscore maboulá\textunderscore  por \textunderscore mabouiá\textunderscore .
\section{Mabounga}
\begin{itemize}
\item {Grp. gram.:f.}
\end{itemize}
Grande môsca da África, de reflexos esverdeados.
\section{Mabu}
\begin{itemize}
\item {Grp. gram.:f.}
\end{itemize}
Espécie de palmeira, de fôlhas têxteis, (\textunderscore cyperus tapyrus\textunderscore ). Cf. Capello e Ivens, I, 178.
\section{Mabuinguiri}
\begin{itemize}
\item {Grp. gram.:m.}
\end{itemize}
Árvore esterculiácea de Angola.
\section{Maca}
\begin{itemize}
\item {Grp. gram.:f.}
\end{itemize}
\begin{itemize}
\item {Utilização:Bras. do N}
\end{itemize}
Cama de lona, para descanso dos marinheiros, a bordo.
Espécie de esquife, para transporte de enfermos.
Padiola, com quatro braços, para transporte de mobília ou bagagens.
Saco de coiro, em que se leva roupa e que se amarra á garupa, em viagem.
(Do al.?)
\section{Maca}
\begin{itemize}
\item {Grp. gram.:f.}
\end{itemize}
Ave doméstica do Peru.
\section{Maça}
\begin{itemize}
\item {Grp. gram.:f.}
\end{itemize}
\begin{itemize}
\item {Utilização:Prov.}
\end{itemize}
\begin{itemize}
\item {Utilização:alent.}
\end{itemize}
\begin{itemize}
\item {Utilização:Ant.}
\end{itemize}
\begin{itemize}
\item {Utilização:Geol.}
\end{itemize}
\begin{itemize}
\item {Utilização:Prov.}
\end{itemize}
\begin{itemize}
\item {Utilização:trasm.}
\end{itemize}
\begin{itemize}
\item {Proveniência:(Do lat. \textunderscore matea\textunderscore )}
\end{itemize}
Pau pesado, mais grosso numa extremidade que na outra.
Espécie de pilão cylindrico, usado no serviço de calceteiro.
Clava.
Núcleo central das rodas dos carros, ligado pelos raios ás pinas.
Pau, terminado por uma cabeça ovóide ou esphérica, ás vezes recoberta de pelle, e com que se toca o bombo.
A polpa interior da noz moscada.
Formação eruptiva, desenvolvida irregularmente em qualquer direcção, antes de attingir a superfície do globo. Cf. Gonç. Guimarães, \textunderscore Geol.\textunderscore , 143.
\textunderscore Maça da perna\textunderscore , a coxa.
\section{Maçã}
\begin{itemize}
\item {Grp. gram.:f.}
\end{itemize}
\begin{itemize}
\item {Utilização:Bras}
\end{itemize}
\begin{itemize}
\item {Proveniência:(Do lat. \textunderscore Matiana\textunderscore , n. p.)}
\end{itemize}
Espécie de pomo, fruto da macieira.
Nome de alguns objectos, que têm proximamente o feitio de maçan, como a saliência das faces, a parte da espada, em que se prende o espigão da fôlha, etc.
Variedade de banana, muito apreciada.
Maçaneta da sella.
Fruto do cipreste.
Parte do boi, correspondente á extremidade anterior do esterno.
\section{Macaba}
\begin{itemize}
\item {Grp. gram.:f.}
\end{itemize}
\begin{itemize}
\item {Utilização:Bras}
\end{itemize}
Árvore fructífera dos sertões.
(O mesmo que \textunderscore bacaba\textunderscore  = \textunderscore bacabeira\textunderscore ?)
\section{Macabires}
\begin{itemize}
\item {Grp. gram.:m. pl.}
\end{itemize}
Antigo povo cafreal. Cf. Couto. \textunderscore Déc.\textunderscore , X, c. 14.
\section{Macabra}
\begin{itemize}
\item {Grp. gram.:adj. f.}
\end{itemize}
\begin{itemize}
\item {Proveniência:(Fr. \textunderscore macabre\textunderscore )}
\end{itemize}
Diz-se de uma dança, em que se representava a morte, arrastando pessôas de todas as idades e condições.
\section{Macabrismo}
\begin{itemize}
\item {Grp. gram.:adj.}
\end{itemize}
Qualidade de macabro; diversão macabra. Cf. Camillo, \textunderscore Mar. da Fonte\textunderscore , 142.
\section{Macabro}
\begin{itemize}
\item {Grp. gram.:adj.}
\end{itemize}
Relativo á dança macabra.
Que faz lembrar a dança macabra.
Que desfila lugubremente; fúnebre.
Affeiçoado a coisas tristes. Cf. Camillo, \textunderscore Corja\textunderscore , 263.
(Cp. \textunderscore macabra\textunderscore )
\section{Macaca}
\begin{itemize}
\item {Grp. gram.:f.}
\end{itemize}
\begin{itemize}
\item {Utilização:Pop.}
\end{itemize}
\begin{itemize}
\item {Utilização:Zool.}
\end{itemize}
Fêmea do macaco.
Infelicidade persistente; má sorte.
Mulher feia.
Peixe, espécie de linguado (\textunderscore solea lascaris\textunderscore , Risso).
(Cp. \textunderscore macaco\textunderscore )
\section{Macacascau}
\begin{itemize}
\item {Grp. gram.:f.}
\end{itemize}
\begin{itemize}
\item {Utilização:Bras}
\end{itemize}
Espécie de cacau.
\section{Macacal}
\begin{itemize}
\item {Grp. gram.:adj.}
\end{itemize}
Relativo a macaco:«\textunderscore ...é o Eden macacal na ambundância e recreio\textunderscore ». Castilho, \textunderscore Escav. Poét.\textunderscore , 43.
\section{Macacão}
\begin{itemize}
\item {Grp. gram.:m.}
\end{itemize}
\begin{itemize}
\item {Proveniência:(De \textunderscore macaco\textunderscore )}
\end{itemize}
Homem finório, solerte, manhoso. Cf. Castilho, \textunderscore Fausto\textunderscore , 211.
\section{Macacar}
\textunderscore v. t.\textunderscore  (e der.)
O mesmo que \textunderscore macaquear\textunderscore , etc. Cf. Filinto, X, 118.
\section{Macáçar}
\begin{itemize}
\item {Grp. gram.:m.}
\end{itemize}
\begin{itemize}
\item {Proveniência:(De \textunderscore Macáçar\textunderscore , n. p.)}
\end{itemize}
Idioma das ilhas Celebes.
Espécie de feijão.
Cosmético, em que entra a essência do cravo, e que serve para aromatizar e amaciar o cabello.
\section{Macáçares}
\begin{itemize}
\item {Grp. gram.:m. pl.}
\end{itemize}
Habitantes de Macáçar. Cf. Barros, \textunderscore Déc.\textunderscore  IV, l. VI, c. 25.
\section{Macacaria}
\begin{itemize}
\item {Grp. gram.:f.}
\end{itemize}
Porção de macacos; macaquice. Cf. Camillo, \textunderscore Cancion. Al.\textunderscore , 434.
\section{Macacaúba}
\begin{itemize}
\item {Grp. gram.:f.}
\end{itemize}
Planta leguminosa e vermelha do Brasil.
\section{Macaco}
\begin{itemize}
\item {Grp. gram.:m.}
\end{itemize}
\begin{itemize}
\item {Utilização:Bras}
\end{itemize}
\begin{itemize}
\item {Grp. gram.:Adj.}
\end{itemize}
\begin{itemize}
\item {Utilização:Pop.}
\end{itemize}
\begin{itemize}
\item {Grp. gram.:Loc.}
\end{itemize}
\begin{itemize}
\item {Utilização:burl.}
\end{itemize}
\begin{itemize}
\item {Proveniência:(T. cong.)}
\end{itemize}
Gênero de mammíferos quadrúmanos.
Maquinismo, para levantar grandes pesos.
Bate-estacas.
Peixe de Portugal.
Talão de vara velha de videira.
Pilar, em cuja construcção se empregam só dois tejolos por camada.
Finório.
Feio.
Ordinário; apoucado.
Infeliz.
\textunderscore Morte macaca\textunderscore , morte desastrosa e inglória.
\section{Macacôa}
\begin{itemize}
\item {Grp. gram.:f.}
\end{itemize}
\begin{itemize}
\item {Utilização:Fam.}
\end{itemize}
Doença pouco importante.
\section{Maçacopla}
\begin{itemize}
\item {Grp. gram.:f.}
\end{itemize}
O mesmo que \textunderscore marreta\textunderscore .
(Do cast.)
\section{Macacório}
\begin{itemize}
\item {Grp. gram.:m.}
\end{itemize}
\begin{itemize}
\item {Utilização:Bras. do N}
\end{itemize}
O mesmo que \textunderscore macacão\textunderscore .
\section{Macaçote}
\begin{itemize}
\item {Grp. gram.:m.}
\end{itemize}
O mesmo que \textunderscore barrilheira\textunderscore .
\section{Maçacote}
\begin{itemize}
\item {Grp. gram.:m.}
\end{itemize}
\begin{itemize}
\item {Proveniência:(Do rad. de \textunderscore maço\textunderscore )}
\end{itemize}
Peça de ferro, para encostar a ponta dos pregos, quando ella se arrebita.
Peixe de Portugal.
\section{Maçacroco}
\begin{itemize}
\item {Grp. gram.:m.}
\end{itemize}
(V.maçaroco)
\section{Macacu}
\begin{itemize}
\item {Grp. gram.:m.}
\end{itemize}
Árvore tinctória do Brasil.
\section{Maçacuca}
\begin{itemize}
\item {Grp. gram.:f.}
\end{itemize}
\begin{itemize}
\item {Utilização:Prov.}
\end{itemize}
\begin{itemize}
\item {Proveniência:(De \textunderscore maçan\textunderscore  + \textunderscore cuco\textunderscore )}
\end{itemize}
Fruto da carvalha mansa, differente da bugalha propriamente dita em sêr completamente redondo e, em quanto verde, muito esponjoso.
Bugalhinha.
\section{Maçada}
\begin{itemize}
\item {Grp. gram.:f.}
\end{itemize}
\begin{itemize}
\item {Utilização:Fig.}
\end{itemize}
\begin{itemize}
\item {Utilização:Des.}
\end{itemize}
\begin{itemize}
\item {Proveniência:(De \textunderscore maço\textunderscore  e \textunderscore maça\textunderscore )}
\end{itemize}
Pancada com maço ou maça.
Pancadaria.
Cambôa.
Trapaça no jôgo.
Conluio.
Conversa fastienta e longa; trabalho fastiento.
Mólho ou maço volumoso de objectos atados.
Antiga armação para pescar lampreias. Cf. Viterbo, \textunderscore Elucid.\textunderscore , vb. \textunderscore Couleiro de Fogos\textunderscore .
\section{Macadame}
\begin{itemize}
\item {Grp. gram.:m.}
\end{itemize}
\begin{itemize}
\item {Proveniência:(De \textunderscore Mac-Adam\textunderscore , n. p.)}
\end{itemize}
Systema de pavimento ou calcetamento de ruas ou estradas, em que se emprega granito e saibro, que se recalca com um cylindro.
\section{Macadamização}
\begin{itemize}
\item {Grp. gram.:f.}
\end{itemize}
Acto de macadamizar.
\section{Macadamizar}
\begin{itemize}
\item {Grp. gram.:v. t.}
\end{itemize}
Calcetar, pelo systema de macadame.
\section{Maçadeiro}
\begin{itemize}
\item {Grp. gram.:m.}
\end{itemize}
\begin{itemize}
\item {Utilização:Prov.}
\end{itemize}
\begin{itemize}
\item {Utilização:trasm.}
\end{itemize}
Pedra, sôbre que se maça o linho.
(Cp. gall. \textunderscore mazadeiro\textunderscore )
\section{Maçado}
\begin{itemize}
\item {Grp. gram.:adj.}
\end{itemize}
\begin{itemize}
\item {Utilização:Prov.}
\end{itemize}
\begin{itemize}
\item {Utilização:minh.}
\end{itemize}
\begin{itemize}
\item {Proveniência:(De \textunderscore maçar\textunderscore )}
\end{itemize}
O mesmo que \textunderscore coalhado\textunderscore : \textunderscore leite maçado\textunderscore . (Colhido em Varzim)
\section{Maçador}
\begin{itemize}
\item {Grp. gram.:adj.}
\end{itemize}
\begin{itemize}
\item {Grp. gram.:M.}
\end{itemize}
\begin{itemize}
\item {Utilização:Prov.}
\end{itemize}
\begin{itemize}
\item {Utilização:trasm.}
\end{itemize}
Que maça.
Aquelle que maça.
Maço, com que se bate o linho.
\section{Maçadoria}
\begin{itemize}
\item {Grp. gram.:f.}
\end{itemize}
\begin{itemize}
\item {Utilização:Fam.}
\end{itemize}
\begin{itemize}
\item {Proveniência:(De \textunderscore maçador\textunderscore )}
\end{itemize}
Conversa longa e fastidiosa; grande maçada.
\section{Maçadura}
\begin{itemize}
\item {Grp. gram.:f.}
\end{itemize}
\begin{itemize}
\item {Grp. gram.:Pl.}
\end{itemize}
\begin{itemize}
\item {Utilização:Ant.}
\end{itemize}
\begin{itemize}
\item {Proveniência:(De \textunderscore maçar\textunderscore )}
\end{itemize}
O mesmo que \textunderscore maçada\textunderscore .
Vestígio de pancadas no corpo.
Compressão, feita methodicamente, como meio cirúrgico; maçagem.
Tributo, o mesmo que \textunderscore indícias\textunderscore .
\section{Macaense}
\begin{itemize}
\item {Grp. gram.:m.  e  adj.}
\end{itemize}
O mesmo que \textunderscore macaísta\textunderscore .
\section{Maçagada}
\begin{itemize}
\item {Grp. gram.:f.}
\end{itemize}
\begin{itemize}
\item {Utilização:Bras}
\end{itemize}
\begin{itemize}
\item {Utilização:pop.}
\end{itemize}
Maço grande de papéis, etc.
\section{Maçagem}
\begin{itemize}
\item {Grp. gram.:f.}
\end{itemize}
\begin{itemize}
\item {Utilização:Chapel.}
\end{itemize}
Acto de maçar o linho.
Acto de comprimir methodicamente o corpo ou parte do corpo, para modificar a circulação ou obter vantagens therapêuticas.
Operação de fulista, que consiste em revolver com a sapatilha o chapéu em elaboração, para dar uniformidade e consistência ao pêlo.
\section{Maçagista}
\begin{itemize}
\item {Grp. gram.:m.}
\end{itemize}
\begin{itemize}
\item {Proveniência:(De \textunderscore maçagem\textunderscore )}
\end{itemize}
Aquelle que faz maçagens, especialmente o enfermeiro hospitalar, encarregado dêsse serviço.
\section{Macaíba}
\begin{itemize}
\item {Grp. gram.:f.}
\end{itemize}
Espécie de palmeira, (\textunderscore acroxomia sclerocarpa\textunderscore ).
(Do tupi)
\section{Macaibeira}
\begin{itemize}
\item {fónica:ca-i}
\end{itemize}
\begin{itemize}
\item {Grp. gram.:f.}
\end{itemize}
Espécie de palmeira, (\textunderscore acroxomia sclerocarpa\textunderscore ).
(Do tupi)
\section{Macaio}
\begin{itemize}
\item {Grp. gram.:m.}
\end{itemize}
Antigo tecido de seda e lan.
(Relaciona-se com \textunderscore Macau\textunderscore , n. p.?)
\section{Macaíra}
\begin{itemize}
\item {Grp. gram.:m.}
\end{itemize}
Gênero de peixes ósseos do Atlântico.
\section{Macaísta}
\begin{itemize}
\item {Grp. gram.:m.}
\end{itemize}
\begin{itemize}
\item {Grp. gram.:Adj.}
\end{itemize}
Aquelle que nasceu em Macau.
Relativo a esta cidade.
\section{Macajá}
\begin{itemize}
\item {Grp. gram.:m.}
\end{itemize}
O mesmo que \textunderscore macaibeira\textunderscore .
\section{Macajera}
\begin{itemize}
\item {Grp. gram.:f.}
\end{itemize}
Árvore euphorbiácea do Brasil.
\section{Macajuba}
\begin{itemize}
\item {Grp. gram.:f.}
\end{itemize}
O mesmo que \textunderscore macaibeira\textunderscore .
\section{Macajubeira}
\begin{itemize}
\item {Grp. gram.:f.}
\end{itemize}
O mesmo que \textunderscore macaibeira\textunderscore .
\section{Maçal}
\begin{itemize}
\item {Grp. gram.:m.}
\end{itemize}
\begin{itemize}
\item {Proveniência:(De \textunderscore maça\textunderscore )}
\end{itemize}
Sôro de leite, resultante da batedura do queijo.
\section{Maçala}
\begin{itemize}
\item {Grp. gram.:f.}
\end{itemize}
Árvore conífera de Moçambique; o fruto dessa árvore.
\section{Macalacas}
\begin{itemize}
\item {Grp. gram.:m. pl.}
\end{itemize}
Tríbo africana do Alto Zambeze.
\section{Macaman}
\begin{itemize}
\item {Grp. gram.:m.}
\end{itemize}
\begin{itemize}
\item {Utilização:Bras}
\end{itemize}
Preto fugido.
\section{Macamba}
\begin{itemize}
\item {Grp. gram.:m.  e  f.}
\end{itemize}
\begin{itemize}
\item {Utilização:Bras. do Rio}
\end{itemize}
Nome, com que as quitandeiras designam os seus fregueses.
(Do lundês)
\section{Macambacamba}
\begin{itemize}
\item {Grp. gram.:f.}
\end{itemize}
Corpulenta árvore africana, da fam. das moreáceas, (\textunderscore maclura excelsa\textunderscore , De-Candolle)
\section{Macambas}
\begin{itemize}
\item {Grp. gram.:f. pl.}
\end{itemize}
Fruto brasileiro.
\section{Macambira}
\begin{itemize}
\item {Grp. gram.:f.}
\end{itemize}
\begin{itemize}
\item {Utilização:Bras. do N}
\end{itemize}
Planta bromeliácea.
\section{Macamblalá}
\begin{itemize}
\item {Grp. gram.:m.}
\end{itemize}
Árvore santhomense, de raiz aphrodisíaca.
\section{Macambra}
\begin{itemize}
\item {Grp. gram.:f.}
\end{itemize}
O fruto da macambreira.
\section{Macambrará}
\begin{itemize}
\item {Grp. gram.:m.}
\end{itemize}
O mesmo que \textunderscore macamblalá\textunderscore .
\section{Macambreira}
\begin{itemize}
\item {Grp. gram.:f.}
\end{itemize}
Árvore de Dio.
\section{Macambúzio}
\begin{itemize}
\item {Grp. gram.:adj.}
\end{itemize}
\begin{itemize}
\item {Utilização:Pop.}
\end{itemize}
Carrancudo; embezerrado; triste.
(Cp. \textunderscore embuziar\textunderscore )
\section{Macamecrans}
\begin{itemize}
\item {Grp. gram.:m. pl.}
\end{itemize}
Índios do Brasil, que dominavam nas margens do Tocantins.
\section{Maçan}
\begin{itemize}
\item {Grp. gram.:f.}
\end{itemize}
\begin{itemize}
\item {Utilização:Bras}
\end{itemize}
\begin{itemize}
\item {Proveniência:(Do lat. \textunderscore Matiana\textunderscore , n. p.)}
\end{itemize}
Espécie de pomo, fruto da macieira.
Nome de alguns objectos, que têm proximamente o feitio de maçan, como a saliência das faces, a parte da espada, em que se prende o espigão da fôlha, etc.
Variedade de banana, muito apreciada.
Maçaneta da sella.
Fruto do cipreste.
Parte do boi, correspondente á extremidade anterior do esterno.
\section{Macana}
\begin{itemize}
\item {Grp. gram.:f.}
\end{itemize}
Toucado das mulheres de Ormuz. Cf. Barros, \textunderscore Dec.\textunderscore  III, l. VII, c. 3.
\section{Macaná}
\begin{itemize}
\item {Grp. gram.:m.}
\end{itemize}
\begin{itemize}
\item {Utilização:Bras}
\end{itemize}
Instrumento de guerra, espécie de clava, usada por selvagens.--Neste sentido dizem os diccionários \textunderscore macana\textunderscore . É êrro. Cf. B. Rohan, \textunderscore Vocab. Bras.\textunderscore 
\section{Maçanaria}
\begin{itemize}
\item {Grp. gram.:f.}
\end{itemize}
\begin{itemize}
\item {Utilização:Des.}
\end{itemize}
Pomar de maçans.
\section{Maçanaria}
\begin{itemize}
\item {Grp. gram.:f.}
\end{itemize}
\begin{itemize}
\item {Utilização:Ant.}
\end{itemize}
\begin{itemize}
\item {Proveniência:(De \textunderscore mação\textunderscore ^2)}
\end{itemize}
Obra de pedreiro; alvenaria. Cf. \textunderscore Chrón. dos R. de Bisnaga\textunderscore , 55.
\section{Maçan-de-cravo}
\begin{itemize}
\item {Grp. gram.:f.}
\end{itemize}
Variedade de maçan.
\section{Maçan-de-cuco}
\begin{itemize}
\item {Grp. gram.:f.}
\end{itemize}
\begin{itemize}
\item {Utilização:Prov.}
\end{itemize}
\begin{itemize}
\item {Utilização:beir.}
\end{itemize}
O mesmo que \textunderscore maçacuca\textunderscore .
\section{Maçan-de-pé-comprido}
\begin{itemize}
\item {Grp. gram.:f.}
\end{itemize}
Variedade de maçan.
\section{Maçan-de-prato}
\begin{itemize}
\item {Grp. gram.:f.}
\end{itemize}
Variedade de maçan grande.
\section{Maçan-de-rosa}
\begin{itemize}
\item {Grp. gram.:f.}
\end{itemize}
Variedade de maçan, de casca lustrosa e avermelhada.
\section{Maçan-de-vime}
\begin{itemize}
\item {Grp. gram.:f.}
\end{itemize}
Variedade de maçan.
\section{Maçan-de-caco}
\begin{itemize}
\item {Grp. gram.:f.}
\end{itemize}
Variedade de maçan.
\section{Maçaneta}
\begin{itemize}
\item {fónica:nê}
\end{itemize}
\begin{itemize}
\item {Grp. gram.:f.}
\end{itemize}
\begin{itemize}
\item {Utilização:Prov.}
\end{itemize}
\begin{itemize}
\item {Utilização:beir.}
\end{itemize}
\begin{itemize}
\item {Utilização:Prov.}
\end{itemize}
\begin{itemize}
\item {Proveniência:(De \textunderscore maçan\textunderscore )}
\end{itemize}
Remate esphérico ou pyramidal, para ornamento de certos objectos.
Parte mais alta da cella, na deanteira.

Baqueta de tambor; maça de zabumba.

O mesmo que \textunderscore borla\textunderscore ^1.
\section{Macânico}
\textunderscore adj.\textunderscore  (e der.)
(Forma pop. e ant. de mecânico, etc. Cf. \textunderscore Peregrinação\textunderscore , XCVII)
\section{Maçanilha}
\begin{itemize}
\item {Grp. gram.:f.}
\end{itemize}
\begin{itemize}
\item {Utilização:Prov.}
\end{itemize}
\begin{itemize}
\item {Utilização:alent.}
\end{itemize}
Pequena maçan.
Variedade de azeitona.
(Cast. \textunderscore manzanilla\textunderscore )
\section{Macanjice}
\begin{itemize}
\item {Grp. gram.:f.}
\end{itemize}
Qualidade de macanjo.
\section{Macanjo}
\begin{itemize}
\item {Grp. gram.:m.  e  adj.}
\end{itemize}
\begin{itemize}
\item {Utilização:Gír.}
\end{itemize}
Velhaco.
Pataco falso.
\section{Maçan-pipo}
\begin{itemize}
\item {Grp. gram.:f.}
\end{itemize}
Variedade de maçan.
\section{Maçante}
\begin{itemize}
\item {Grp. gram.:adj.}
\end{itemize}
Que maça, que importuna.
\section{Maçanzeira}
\begin{itemize}
\item {Grp. gram.:f.}
\end{itemize}
\begin{itemize}
\item {Utilização:Bras}
\end{itemize}
O mesmo que \textunderscore macieira\textunderscore .
Árvore fructífera que, no mesmo pé, dá frutos de differente feitio, dos quaes se faz doce. Cf. B. C. Rubim., \textunderscore Vocab. Bras.\textunderscore 
\section{Mação}
\begin{itemize}
\item {Grp. gram.:m.}
\end{itemize}
Maço grande.
\section{Mação}
\begin{itemize}
\item {Grp. gram.:m.}
\end{itemize}
\begin{itemize}
\item {Proveniência:(Fr. \textunderscore maçon\textunderscore )}
\end{itemize}
Membro da maçonaria; pedreiro livre.
\section{Maçapão}
\begin{itemize}
\item {Grp. gram.:m.}
\end{itemize}
Bolo de farinha e ovos com amêndoas.
(Cp. cast. \textunderscore mazapan\textunderscore )
\section{Maçaquara}
\begin{itemize}
\item {Grp. gram.:f.}
\end{itemize}
\begin{itemize}
\item {Utilização:Bras}
\end{itemize}
Certa armadilha, para apanhar camarões em rios ou esteiros.
\section{Macaqueação}
\begin{itemize}
\item {Grp. gram.:f.}
\end{itemize}
Acto ou effeito de macaquear. Cf. O'Neill, \textunderscore Fabulário\textunderscore , 855 e 967.
\section{Macaqueador}
\begin{itemize}
\item {Grp. gram.:m.  e  adj.}
\end{itemize}
O que macaqueia.
\section{Macaquear}
\begin{itemize}
\item {Grp. gram.:v. t.}
\end{itemize}
\begin{itemize}
\item {Utilização:Ext.}
\end{itemize}
Arremedar, como os macacos.
Imitar ridiculamente.
\section{Macaqueiro}
\begin{itemize}
\item {Grp. gram.:adj.}
\end{itemize}
Relativo a macaco; próprio de macaco:«\textunderscore ...carinhas tolas, macaqueiras...\textunderscore »Filinto, V, 14.
\section{Macaquice}
\begin{itemize}
\item {Grp. gram.:f.}
\end{itemize}
\begin{itemize}
\item {Proveniência:(De \textunderscore macaco\textunderscore )}
\end{itemize}
Acto ou effeito de macaquear.
Trejeitos ridículos.
Lisonjas; carinhos interesseiros.
\section{Macaquinhos}
\begin{itemize}
\item {Grp. gram.:m. pl. Loc.}
\end{itemize}
\begin{itemize}
\item {Utilização:fam.}
\end{itemize}
\textunderscore Têr macaquinhos no sótão\textunderscore , ter pouco tino, têr pancada na mola.
Disparatar.
\section{Macar}
\begin{itemize}
\item {Grp. gram.:adv.}
\end{itemize}
\begin{itemize}
\item {Utilização:Ant.}
\end{itemize}
Mau grado; apesar.
(Cp. \textunderscore maguér\textunderscore )
\section{Maçar}
\begin{itemize}
\item {Grp. gram.:v. t.}
\end{itemize}
\begin{itemize}
\item {Utilização:Fig.}
\end{itemize}
Bater com maça ou maço.
Bater.
Pisar.
Enfadar, repisando conversas ou assumptos.
Importunar.
\section{Maçaranduba}
\begin{itemize}
\item {Grp. gram.:f.}
\end{itemize}
\begin{itemize}
\item {Utilização:Bras. do N}
\end{itemize}
O mesmo ou melhor que \textunderscore massaranduba\textunderscore .
Árvore sapotácea do Brasil.
Fruto dessa árvore.
O mesmo que \textunderscore cacete\textunderscore .
\section{Macaranga}
\begin{itemize}
\item {Grp. gram.:f.}
\end{itemize}
Planta euphorbiácea de Madagáscar.
\section{Macaré}
\begin{itemize}
\item {Grp. gram.:m.}
\end{itemize}
O mesmo que \textunderscore macaréu\textunderscore .
\section{Macareno}
\begin{itemize}
\item {Grp. gram.:adj.}
\end{itemize}
\begin{itemize}
\item {Utilização:Gír.}
\end{itemize}
\begin{itemize}
\item {Grp. gram.:M.}
\end{itemize}
\begin{itemize}
\item {Utilização:T. de Coímbra}
\end{itemize}
Mau.

Chapéu amachucado.
Pataco falso.
(Cast. \textunderscore macareno\textunderscore )
\section{Macaréo}
\begin{itemize}
\item {Grp. gram.:m.}
\end{itemize}
Vaga impetuosa que, nalguns rios, precede o comêço da preamar.
(Cp. fr. \textunderscore macrée\textunderscore )
\section{Macaréu}
\begin{itemize}
\item {Grp. gram.:m.}
\end{itemize}
Vaga impetuosa que, nalguns rios, precede o comêço da preamar.
(Cp. fr. \textunderscore macrée\textunderscore )
\section{Maçarica}
\begin{itemize}
\item {Grp. gram.:f.}
\end{itemize}
Lebre pequena, que se esquiva facilmente aos cães. Cf. B. Pato, \textunderscore Livro do Monte\textunderscore , 257.
(Cp. \textunderscore maçarico\textunderscore )
\section{Maçarico}
\begin{itemize}
\item {Grp. gram.:m.}
\end{itemize}
Ave aquática pernalta.
Canudo retorcido, por onde sai a chamma, que há de soldar ou derreter um metal.
Lebracho, com malha branca na testa.
\section{Maçarico-das-rochas}
\begin{itemize}
\item {Grp. gram.:m.}
\end{itemize}
Ave marítima, espécie de maçarico, também chamada \textunderscore lavadeira\textunderscore , (\textunderscore actitis hypoleucos\textunderscore )
\section{Maçarico-gallego}
\begin{itemize}
\item {Grp. gram.:m.}
\end{itemize}
Espécie de pequeno maçarico.
\section{Maçarico-real}
\begin{itemize}
\item {Grp. gram.:m.}
\end{itemize}
Espécie de maçarico, maior que o vulgar.
\section{Macarismo}
\begin{itemize}
\item {Grp. gram.:m.}
\end{itemize}
\begin{itemize}
\item {Proveniência:(Gr. \textunderscore makharismos\textunderscore )}
\end{itemize}
Conjunto de hymnos religiosos, pertencentes ao culto grego.
\section{Maçaroca}
\begin{itemize}
\item {Grp. gram.:f.}
\end{itemize}
Fio, que o fuso enrolou em volta de si.
Espiga de milho.
Feixe.
Rôlo de cabello, em fórma de espiga de milho.
Conjunto de morrões, para uso de artilheiros.
(Do ár.?)
\section{Maçaroco}
\begin{itemize}
\item {fónica:çarô}
\end{itemize}
\begin{itemize}
\item {Grp. gram.:m.}
\end{itemize}
\begin{itemize}
\item {Utilização:Prov.}
\end{itemize}
\begin{itemize}
\item {Grp. gram.:Adj.}
\end{itemize}
\begin{itemize}
\item {Utilização:T. de Turquel}
\end{itemize}
\begin{itemize}
\item {Proveniência:(De \textunderscore maçaroca\textunderscore )}
\end{itemize}
Rôlo de cabello, que se encrespou a ferro.
Pão cru.
Diz-se de uma variedade de milho.
\section{Maçaroqueira}
\begin{itemize}
\item {Grp. gram.:f.}
\end{itemize}
\begin{itemize}
\item {Utilização:Bras}
\end{itemize}
Máquina, que substitue o fuso, para fazer maçarocas.
\section{Maçaroquinha}
\begin{itemize}
\item {Grp. gram.:f.}
\end{itemize}
\begin{itemize}
\item {Proveniência:(De \textunderscore maçaroca\textunderscore )}
\end{itemize}
Resíduos ou desperdícios de algodão nas fábricas, os quaes se applicam na limpeza das carruagens ferroviárias.
\section{Macarrão}
\begin{itemize}
\item {Grp. gram.:m.}
\end{itemize}
\begin{itemize}
\item {Utilização:Náut.}
\end{itemize}
\begin{itemize}
\item {Proveniência:(Do it. \textunderscore maccherone\textunderscore )}
\end{itemize}
Massa de farinha para sopa e outros preparados culinários, fabricada em fórma de tubos compridos.
Moitão pequeno de ferro.
\section{Macarroeiro}
\begin{itemize}
\item {Grp. gram.:m.}
\end{itemize}
Fabricante de macarrão e massas análogas.
\section{Macarrónea}
\begin{itemize}
\item {Grp. gram.:f.}
\end{itemize}
\begin{itemize}
\item {Proveniência:(Do rad. do it. \textunderscore macarone\textunderscore )}
\end{itemize}
Composição literária em gênero macarrónico.
\section{Macarronete}
\begin{itemize}
\item {fónica:nê}
\end{itemize}
\begin{itemize}
\item {Grp. gram.:m.}
\end{itemize}
Macarrão delgado.
\section{Macarronicamente}
\begin{itemize}
\item {Grp. gram.:adv.}
\end{itemize}
De modo macarrónico.
\section{Macarrónico}
\begin{itemize}
\item {Grp. gram.:adj.}
\end{itemize}
Escrito de maneira, que as palavras da língua vulgar apresentam terminações latinas ou de outras línguas.
Relativo á macarrónea.
Que escreve macarrónicamente.
\section{Macarronismo}
\begin{itemize}
\item {Grp. gram.:m.}
\end{itemize}
\begin{itemize}
\item {Proveniência:(De \textunderscore maccarrónea\textunderscore )}
\end{itemize}
Gênero macarrónico.
\section{Macarronista}
\begin{itemize}
\item {Grp. gram.:m.}
\end{itemize}
\begin{itemize}
\item {Proveniência:(De \textunderscore macarrónea\textunderscore )}
\end{itemize}
Aquelle que cultiva o gênero macarrónico.
\section{Macaruba}
\begin{itemize}
\item {Grp. gram.:f.}
\end{itemize}
Planta santhomense.
\section{Maçaruco}
\begin{itemize}
\item {Grp. gram.:m.}
\end{itemize}
Pessôa mal vestida, pandorga. (Colhido no Fundão)
(Talvez infl. de \textunderscore maçaroca\textunderscore )
\section{Macauan}
\begin{itemize}
\item {fónica:ca-u}
\end{itemize}
\begin{itemize}
\item {Grp. gram.:m.}
\end{itemize}
\begin{itemize}
\item {Utilização:Des.}
\end{itemize}
O mesmo que \textunderscore acauan\textunderscore .
\section{Macaúba}
\begin{itemize}
\item {Grp. gram.:f.}
\end{itemize}
O mesmo que \textunderscore macaibeira\textunderscore .
\section{Macavenco}
\begin{itemize}
\item {Grp. gram.:m.  e  adj.}
\end{itemize}
\begin{itemize}
\item {Utilização:T. de Lisbôa}
\end{itemize}
\begin{itemize}
\item {Utilização:T. do Fundão}
\end{itemize}
Esquísito, excêntrico.
Maluco, parvo.
\section{Macaxeira}
\begin{itemize}
\item {Grp. gram.:f.}
\end{itemize}
\begin{itemize}
\item {Utilização:Bras. do N}
\end{itemize}
Mandiva raspada.
\section{Macaxera}
\begin{itemize}
\item {Grp. gram.:f.}
\end{itemize}
O mesmo que \textunderscore macaxeira\textunderscore .
\section{Macdám}
\textunderscore m.\textunderscore  (e der.)
(Fórma incorrecta, mas usual, de \textunderscore macadame\textunderscore , etc. Cf. Camillo, \textunderscore Canc. Al.\textunderscore , 371)
\section{Macdonáldia}
\begin{itemize}
\item {Grp. gram.:f.}
\end{itemize}
\begin{itemize}
\item {Proveniência:(De \textunderscore Macdonald\textunderscore , n. p.)}
\end{itemize}
Gênero de orchídeas da Nova-Holanda.
\section{Mace}
\begin{itemize}
\item {Grp. gram.:m.}
\end{itemize}
Moéda chinesa, correspondente á décima parte do tael.
\section{Mácea}
\begin{itemize}
\item {Grp. gram.:f.}
\end{itemize}
Pia ou gamella, em que comem animaes.
(Contr. de \textunderscore almácega\textunderscore )
\section{Macece}
\begin{itemize}
\item {Grp. gram.:m.}
\end{itemize}
Larva de um lepidóptero diurno, comestível e muito apreciada pelos Indígenas africanos.
\section{Macedo}
\begin{itemize}
\item {fónica:cê}
\end{itemize}
\begin{itemize}
\item {Grp. gram.:m.}
\end{itemize}
Casta de uva branca de Trás-os-Montes.
\section{Macedónia}
\begin{itemize}
\item {Grp. gram.:f.}
\end{itemize}
\begin{itemize}
\item {Utilização:Fig.}
\end{itemize}
\begin{itemize}
\item {Proveniência:(Fr. \textunderscore macedoine\textunderscore )}
\end{itemize}
Iguaria, feita de vários legumes ou frutos.
Amálgama de assumptos ou gêneros numa só composição literária.
\section{Macedoniano}
\begin{itemize}
\item {Grp. gram.:adj.}
\end{itemize}
\begin{itemize}
\item {Utilização:Jur.}
\end{itemize}
\begin{itemize}
\item {Proveniência:(Do lat. \textunderscore Macedo\textunderscore , n. p. de um célebre usurário)}
\end{itemize}
Diz-se de um decreto do senado romano, à cêrca da usura.
\section{Macedónico}
\begin{itemize}
\item {Grp. gram.:adj.}
\end{itemize}
Relativo á Macedónia ou aos seus habitantes.
\section{Macedónio}
\begin{itemize}
\item {Grp. gram.:adj.}
\end{itemize}
\begin{itemize}
\item {Grp. gram.:M.}
\end{itemize}
Relativo á Macedónia.
Habitante da Macedónia.
\section{Macedo-pinto}
\begin{itemize}
\item {Grp. gram.:f.}
\end{itemize}
\begin{itemize}
\item {Proveniência:(De \textunderscore Macedo Pinto\textunderscore , n. p.)}
\end{itemize}
Variedade de pêra granulosa e açucarada.
\section{Mácedo-romano}
\begin{itemize}
\item {Grp. gram.:adj.}
\end{itemize}
\begin{itemize}
\item {Proveniência:(De \textunderscore macedónio\textunderscore  e \textunderscore romano\textunderscore )}
\end{itemize}
Diz-se de um dialecto valáchio.
\section{Macega}
\begin{itemize}
\item {Grp. gram.:f.}
\end{itemize}
\begin{itemize}
\item {Utilização:Bras}
\end{itemize}
Erva damninha, que apparece nas searas.
Campo, em que há muito capim ou pequenos arbustos.
\section{Macegal}
\begin{itemize}
\item {Grp. gram.:m.}
\end{itemize}
Terreno, em que crescem macegas.
\section{Maceió}
\begin{itemize}
\item {Grp. gram.:m.}
\end{itemize}
\begin{itemize}
\item {Utilização:Bras. do N}
\end{itemize}
Lagoeiro, formado no litoral, por effeito das marés ou da água pluvial.
\section{Maceira}
\begin{itemize}
\item {Grp. gram.:f.}
\end{itemize}
O mesmo que \textunderscore macieira\textunderscore .
\section{Maceiro}
\begin{itemize}
\item {Grp. gram.:m.}
\end{itemize}
\begin{itemize}
\item {Proveniência:(Do lat. \textunderscore matiarius\textunderscore )}
\end{itemize}
Bedel ou empregado, que usa como distintivo uma clava ou maça.
\section{Macela}
\begin{itemize}
\item {Grp. gram.:f.}
\end{itemize}
\begin{itemize}
\item {Proveniência:(De \textunderscore maça\textunderscore )}
\end{itemize}
Planta e flôr medicinal, amargosa e aromática.
Camomila.
\section{Macelão}
\begin{itemize}
\item {Grp. gram.:m.}
\end{itemize}
Variedade de macela, o mesmo que \textunderscore amaranto\textunderscore .
\section{Macemutina}
\begin{itemize}
\item {Grp. gram.:f.}
\end{itemize}
O mesmo que \textunderscore macemutino\textunderscore .
\section{Macemutino}
\begin{itemize}
\item {Grp. gram.:m.}
\end{itemize}
Moéda de oiro, moirisca, que foi usada em Espanha.
\section{Macenaria}
\begin{itemize}
\item {Grp. gram.:f.}
\end{itemize}
\begin{itemize}
\item {Utilização:Ant.}
\end{itemize}
O mesmo que \textunderscore marcenaria\textunderscore . Cf. \textunderscore Peregrinação\textunderscore , LXXXIII.
\section{Maceração}
\begin{itemize}
\item {Grp. gram.:f.}
\end{itemize}
\begin{itemize}
\item {Utilização:Fig.}
\end{itemize}
\begin{itemize}
\item {Proveniência:(Lat. \textunderscore maceratio\textunderscore )}
\end{itemize}
Acto ou effeito de macerar.
Operação chímica, que consiste em pôr uma substância sólida num líquido, para que êste se impregne dos princípios solúveis daquella substância.
O producto dessa operação.
Immersão de uma peça anatómica num líquido, para a despir dos tecidos molles ou torná-los transparentes.
Mortificação do corpo, por meio de jejuns, disciplinas, etc.
\section{Macerado}
\begin{itemize}
\item {Grp. gram.:m.}
\end{itemize}
\begin{itemize}
\item {Proveniência:(De \textunderscore macerar\textunderscore )}
\end{itemize}
Resultado da maceração ou infusão a frio.
\section{Maceramento}
\begin{itemize}
\item {Grp. gram.:m.}
\end{itemize}
O mesmo que \textunderscore maceração\textunderscore .
\section{Macerar}
\begin{itemize}
\item {Grp. gram.:v. t.}
\end{itemize}
\begin{itemize}
\item {Utilização:Fig.}
\end{itemize}
\begin{itemize}
\item {Proveniência:(Lat. \textunderscore macerare\textunderscore )}
\end{itemize}
Submeter (uma substância sólida) á acção de um líquido, para que êste se sature de alguns princípios que a constituem, ou para lhe alterar os tecidos ou despojá-la das partes molles.
Amollecer; machucar.
Torturar; mortificar.
\section{Maceria}
\begin{itemize}
\item {Grp. gram.:f.}
\end{itemize}
\begin{itemize}
\item {Proveniência:(Lat. \textunderscore maceria\textunderscore )}
\end{itemize}
Obra do alvenaria, sem barro.
\section{Macesse}
\begin{itemize}
\item {Grp. gram.:m.}
\end{itemize}
Larva de um lepidóptero diurno, comestível e muito apreciada pelos Indígenas africanos.
\section{Maceta}
\begin{itemize}
\item {fónica:cê}
\end{itemize}
\begin{itemize}
\item {Grp. gram.:f.}
\end{itemize}
\begin{itemize}
\item {Grp. gram.:Adj.}
\end{itemize}
\begin{itemize}
\item {Utilização:Bras}
\end{itemize}
Pequena maça de ferro, com um cabo curto de madeira, usada especialmente por canteiros, para bater no escopro, com que trabalham.
Pedra cylíndrica, para moer e desfazer tintas.
Maça, com que se toca o bombo.
Diz-se do cavallo que tem as mãos doentes ou defeituosas.
\section{Maceta}
\begin{itemize}
\item {fónica:cê}
\end{itemize}
\begin{itemize}
\item {Grp. gram.:f.}
\end{itemize}
\begin{itemize}
\item {Utilização:Ant.}
\end{itemize}
Escarrador, cuspideira.
(Cast. \textunderscore maceta\textunderscore )
\section{Macete}
\begin{itemize}
\item {fónica:cê}
\end{itemize}
\begin{itemize}
\item {Grp. gram.:m.}
\end{itemize}
O mesmo que \textunderscore maço\textunderscore .
Pequeno maço do esculptores.
\section{Macha}
\begin{itemize}
\item {Grp. gram.:f.}
\end{itemize}
\begin{itemize}
\item {Utilização:Gír.}
\end{itemize}
Fechadura.
\section{Machacá}
\begin{itemize}
\item {Grp. gram.:m.}
\end{itemize}
\begin{itemize}
\item {Utilização:Bras. do N}
\end{itemize}
Boi mal castrado.
\section{Machacarás}
\begin{itemize}
\item {Grp. gram.:m. pl.}
\end{itemize}
Índios do Brasil, que viviam entre Cerro-Frio e Porto-Seguro.
\section{Machacaz}
\begin{itemize}
\item {Grp. gram.:m.}
\end{itemize}
\begin{itemize}
\item {Utilização:Pleb.}
\end{itemize}
\begin{itemize}
\item {Grp. gram.:M.  e  adj.}
\end{itemize}
\begin{itemize}
\item {Proveniência:(Do rad. de \textunderscore macho\textunderscore )}
\end{itemize}
Indivíduo corpulento e desajeitado.
Finório, espertalhão, machucho.
\section{Machada}
\begin{itemize}
\item {Grp. gram.:f.}
\end{itemize}
Machado pequeno, que se maneja com uma só mão.
(Cp. \textunderscore machado\textunderscore )
\section{Machadada}
\begin{itemize}
\item {Grp. gram.:f.}
\end{itemize}
\begin{itemize}
\item {Proveniência:(De \textunderscore machada\textunderscore  ou \textunderscore machado\textunderscore )}
\end{itemize}
Golpe de machado ou machada.
\section{Machadar}
\begin{itemize}
\item {Grp. gram.:v. i.}
\end{itemize}
Trabalhar com machado.
Dar golpes de machado.
Rachar lenha com machado.
\section{Machadaza}
\begin{itemize}
\item {Grp. gram.:f.}
\end{itemize}
\begin{itemize}
\item {Utilização:Prov.}
\end{itemize}
\begin{itemize}
\item {Utilização:alent.}
\end{itemize}
O mesmo que \textunderscore machadada\textunderscore .
\section{Machadinha}
\begin{itemize}
\item {Grp. gram.:f.}
\end{itemize}
Pequena machada.
Pequeno machado, usado por carniceiros.
\section{Machado}
\begin{itemize}
\item {Grp. gram.:m.}
\end{itemize}
Instrumento cortante, formado de uma espécie de cunha afiada e fixa num cabo de madeira, servindo para rachar troncos, reduzindo-os a lenha, para cortar árvores, apparelhar madeira para a serração, etc.
Instrumento náutico, para picar mastros, mastaréus, viradores ou amarras.
\textunderscore Machado de pedra\textunderscore , nome geral de vários instrumentos de pedra, dos tempos prehistóricos, em forma de cunha, a que o povo dá o nome de \textunderscore pedra de raio\textunderscore , \textunderscore corisco\textunderscore , etc.
\section{Macha-fêmea}
\begin{itemize}
\item {Grp. gram.:f.}
\end{itemize}
\begin{itemize}
\item {Grp. gram.:F.  e  adj.}
\end{itemize}
\begin{itemize}
\item {Proveniência:(De \textunderscore macho\textunderscore  + \textunderscore fêmea\textunderscore )}
\end{itemize}
Espécie de gonzo.
Bisagra.
Hermaphrodita.
\section{Machamba}
\begin{itemize}
\item {Grp. gram.:f.}
\end{itemize}
O mesmo que \textunderscore manchamba\textunderscore .
\section{Machambomba}
\begin{itemize}
\item {Grp. gram.:f.}
\end{itemize}
\begin{itemize}
\item {Utilização:Bras}
\end{itemize}
\begin{itemize}
\item {Proveniência:(De \textunderscore Machambomba\textunderscore , n. p. de uma estação de caminho de ferro, no Brasil)}
\end{itemize}
Carruagem de caminho de ferro, com mais de um pavimento.
\section{Macha-mona}
\begin{itemize}
\item {Grp. gram.:f.}
\end{itemize}
Fruto de uma cucurbitácea africana e americana.
\section{Machão}
\begin{itemize}
\item {Grp. gram.:m.}
\end{itemize}
\begin{itemize}
\item {Utilização:Pleb.}
\end{itemize}
\begin{itemize}
\item {Proveniência:(De \textunderscore macho\textunderscore )}
\end{itemize}
Mulher robusta e de modos grosseiros ou varonis.
\section{Macharrão}
\begin{itemize}
\item {Grp. gram.:m.}
\end{itemize}
Macho grande:«\textunderscore os noivos iam cada qual em macharrão de tremer\textunderscore ». O'Neill, \textunderscore Fabulário\textunderscore , 705.
\section{Macharuim}
\begin{itemize}
\item {Grp. gram.:m.}
\end{itemize}
Fruto indiano. Cf. \textunderscore Chrón. dos R. de Bisnaga\textunderscore , 96.
\section{Machatim}
\begin{itemize}
\item {Grp. gram.:m.}
\end{itemize}
\begin{itemize}
\item {Utilização:Ant.}
\end{itemize}
\begin{itemize}
\item {Grp. gram.:Pl.}
\end{itemize}
\begin{itemize}
\item {Utilização:Ant.}
\end{itemize}
Farçante, pantomimeiro.
Espécie de pantomima, que representava combates, etc.
(Talvez metáth. de \textunderscore matachim\textunderscore , de \textunderscore matar\textunderscore  + \textunderscore chim\textunderscore , se não se relaciona com \textunderscore muchachim\textunderscore )
\section{Macheado}
\begin{itemize}
\item {Grp. gram.:m.}
\end{itemize}
\begin{itemize}
\item {Proveniência:(De \textunderscore machear\textunderscore )}
\end{itemize}
Dobradura do pano em machos.
\section{Machear}
\begin{itemize}
\item {Grp. gram.:v. t.}
\end{itemize}
\begin{itemize}
\item {Utilização:Carp.}
\end{itemize}
Dobrar em machos, (falando-se de artefactos de costura ou trabalhos análogos).
Têr cóito com, (falando-se de animaes).
Encaixar uma peça de madeira numa chanfradura ou fenda de (outra peça).
\section{Machego}
\begin{itemize}
\item {fónica:chê}
\end{itemize}
\begin{itemize}
\item {Grp. gram.:m.}
\end{itemize}
\begin{itemize}
\item {Utilização:Pop.}
\end{itemize}
Macho ordinário.
\section{Macheiro}
\begin{itemize}
\item {Grp. gram.:m.}
\end{itemize}
Sobreiro mais crescido que o chaparro; chaparro.
\section{Machera}
\begin{itemize}
\item {fónica:que}
\end{itemize}
\begin{itemize}
\item {Grp. gram.:f.}
\end{itemize}
\begin{itemize}
\item {Proveniência:(Lat. \textunderscore machaera\textunderscore )}
\end{itemize}
Antiga espada, larga e curta, espécie de sabre.
\section{Machério}
\begin{itemize}
\item {fónica:qué}
\end{itemize}
\begin{itemize}
\item {Grp. gram.:m.}
\end{itemize}
\begin{itemize}
\item {Proveniência:(Gr. \textunderscore makhairion\textunderscore )}
\end{itemize}
Pequeno sabre, antigamente usado pelos Romanos.
\section{Macheróphoro}
\begin{itemize}
\item {fónica:que}
\end{itemize}
\begin{itemize}
\item {Grp. gram.:m.}
\end{itemize}
\begin{itemize}
\item {Proveniência:(Gr. \textunderscore machairophoros\textunderscore )}
\end{itemize}
Soldado, armado de machera.
\section{Machetada}
\begin{itemize}
\item {Grp. gram.:f.}
\end{itemize}
Golpe de machete.
\section{Machete}
\begin{itemize}
\item {fónica:chê}
\end{itemize}
\begin{itemize}
\item {Grp. gram.:m.}
\end{itemize}
\begin{itemize}
\item {Proveniência:(Do rad. de \textunderscore machado\textunderscore )}
\end{itemize}
Sabre de artilheiro, com dois gumes.
Faca de mato.
Viola pequena.
O mesmo que \textunderscore cavaquinho\textunderscore .
\section{Machial}
\begin{itemize}
\item {Grp. gram.:m.}
\end{itemize}
Montado; chaparral.
Lugar inculto, applicado a pastagens.
(Por \textunderscore machieiral\textunderscore , de \textunderscore machieiro\textunderscore )
\section{Machiar}
\begin{itemize}
\item {Grp. gram.:v. i.}
\end{itemize}
Tornar-se máchio; esterilizar-se; degenerar, (falando-se de plantas).
\section{Machiavelicamente}
\begin{itemize}
\item {fónica:qui}
\end{itemize}
\begin{itemize}
\item {Grp. gram.:adv.}
\end{itemize}
De modo \textunderscore machiavélico\textunderscore .
\section{Machiavelice}
\begin{itemize}
\item {fónica:qui}
\end{itemize}
\begin{itemize}
\item {Grp. gram.:f.}
\end{itemize}
Acto ou dito machiavélico.
Manha, ronha.
\section{Machiavélico}
\begin{itemize}
\item {fónica:qui}
\end{itemize}
\begin{itemize}
\item {Grp. gram.:adj.}
\end{itemize}
\begin{itemize}
\item {Utilização:Fig.}
\end{itemize}
Relativo ou semelhante ao machiavelismo.
Astuto; ardiloso; velhaco.
\section{Machiavelismo}
\begin{itemize}
\item {fónica:qui}
\end{itemize}
\begin{itemize}
\item {Grp. gram.:m.}
\end{itemize}
\begin{itemize}
\item {Utilização:Fig.}
\end{itemize}
\begin{itemize}
\item {Proveniência:(De \textunderscore Machiavel\textunderscore , n. p.)}
\end{itemize}
Systema político, preconizado pelo florentino Machiavel, (\textunderscore Machiavello\textunderscore ), e que tem por base a astúcia.
Velhacaria; perfídia.
Alguns mandam lêr \textunderscore makiavelismo\textunderscore , attendendo-se á pronúncia italiana; vulgarmente porém, diz-se \textunderscore maxiavelismo\textunderscore , applicando-se aos der. a mesma pronúncia.
\section{Machiavelista}
\begin{itemize}
\item {fónica:qui}
\end{itemize}
\begin{itemize}
\item {Grp. gram.:adj.}
\end{itemize}
\begin{itemize}
\item {Grp. gram.:M.  e  f.}
\end{itemize}
Machiavélico.
Pessôa, que segue a doutrina de Machiavel.
(Cp. \textunderscore machiavelismo\textunderscore )
\section{Machiavelizar}
\begin{itemize}
\item {fónica:qui}
\end{itemize}
\begin{itemize}
\item {Grp. gram.:v. i.}
\end{itemize}
\begin{itemize}
\item {Proveniência:(De \textunderscore Machiavel\textunderscore , n. p.)}
\end{itemize}
Proceder machiavelicamente.
\section{Machiche}
\begin{itemize}
\item {Grp. gram.:m.}
\end{itemize}
\begin{itemize}
\item {Utilização:Bras}
\end{itemize}
Planta cucurbitácea do Brasil.
Cancan, dança affectada e desenvolta.
\section{Machieiro}
\begin{itemize}
\item {Grp. gram.:m.}
\end{itemize}
\begin{itemize}
\item {Utilização:Prov.}
\end{itemize}
\begin{itemize}
\item {Utilização:alent.}
\end{itemize}
O mesmo que \textunderscore macheiro\textunderscore .
Moita de urze brava.
\section{Machil}
\begin{itemize}
\item {Grp. gram.:m.}
\end{itemize}
\begin{itemize}
\item {Utilização:Des.}
\end{itemize}
O mesmo que \textunderscore manchil\textunderscore . Cf. \textunderscore Anat. Joc.\textunderscore , I, 253.
\section{Machila}
\begin{itemize}
\item {Grp. gram.:f.}
\end{itemize}
Palanquim ou espécie de maca, para transporte de pessôas, na África e na Índia.
(Conc. \textunderscore makila\textunderscore )
\section{Machileiro}
\begin{itemize}
\item {Grp. gram.:m.}
\end{itemize}
Conductor de machila.
\section{Machim}
\begin{itemize}
\item {Grp. gram.:m.}
\end{itemize}
\begin{itemize}
\item {Utilização:Bras. do N}
\end{itemize}
Viola pequena.
Articulação do pé do cavallo.
\section{Machimbombo}
\begin{itemize}
\item {Grp. gram.:m.}
\end{itemize}
\begin{itemize}
\item {Utilização:T. de Lisbôa}
\end{itemize}
Ascensor mecânico, para ladeiras íngremes.
(Cp. \textunderscore Machambomba\textunderscore )
\section{Máchina}
\begin{itemize}
\item {fónica:qui}
\end{itemize}
\textunderscore f.\textunderscore  (e der.)
O mesmo que \textunderscore máquina\textunderscore , etc.
\section{Machinho}
\begin{itemize}
\item {Grp. gram.:m.}
\end{itemize}
\begin{itemize}
\item {Utilização:Mús.}
\end{itemize}
\begin{itemize}
\item {Grp. gram.:Loc.}
\end{itemize}
\begin{itemize}
\item {Utilização:fam.}
\end{itemize}
Espécie de machete, (cobre).
Parte posterior da junta da quartella, nas cavalgaduras.
Espécie de cavaquinho.
\textunderscore Carregar os machinhos\textunderscore , embriagar-se.
\section{Machinhos-pretos}
\begin{itemize}
\item {Grp. gram.:m. pl.}
\end{itemize}
\begin{itemize}
\item {Utilização:Pop.}
\end{itemize}
Botas; sapatos.
\textunderscore Andar nos machinhos-pretos\textunderscore , andar a pé.
\section{Machinhudo}
\begin{itemize}
\item {Grp. gram.:adj.}
\end{itemize}
\begin{itemize}
\item {Proveniência:(De \textunderscore machinho\textunderscore )}
\end{itemize}
Diz-se do animal, que tem muito sensível ou saliente a parte posterior da junta da quartella. Cf. Baganha, \textunderscore Hyg. Pec.\textunderscore , 163.
\section{Máchio}
\begin{itemize}
\item {Grp. gram.:m.}
\end{itemize}
\begin{itemize}
\item {Grp. gram.:Adj.}
\end{itemize}
Doença, que seca os grãos dos cereaes.
Chocho; pêco.
\section{Machío}
\begin{itemize}
\item {Grp. gram.:m.}
\end{itemize}
\begin{itemize}
\item {Proveniência:(De \textunderscore machiar\textunderscore , por \textunderscore machear\textunderscore )}
\end{itemize}
Acto de machear ou ter cóito, (falando-se de animaes).
\section{Machira}
\begin{itemize}
\item {Grp. gram.:f.}
\end{itemize}
O mesmo que \textunderscore machila\textunderscore .
\section{Macho}
\begin{itemize}
\item {Grp. gram.:m.}
\end{itemize}
\begin{itemize}
\item {Utilização:Prov.}
\end{itemize}
\begin{itemize}
\item {Utilização:minh.}
\end{itemize}
\begin{itemize}
\item {Grp. gram.:Adj.}
\end{itemize}
\begin{itemize}
\item {Utilização:Pop.}
\end{itemize}
\begin{itemize}
\item {Proveniência:(Do lat. \textunderscore masculus\textunderscore )}
\end{itemize}
Filho de cavallo e jumenta, e ainda o filho de burro e égua, mas, neste caso, os téchnicos usam \textunderscore mu\textunderscore .
Indivíduo do sexo masculino: \textunderscore o Alexandre teve 5 filhos, isto é, 2 machos e 3 fêmeas\textunderscore .
Dobradura do pano em pregas oppostas.
Parte da dobradiça, que encaixa na fêmea.
Colchete ou parte do colchete, que engancha na fêmea.
Instrumento, que torna côncava a madeira, cortando.
Iró.
A ferragem do leme, que gira nos fusos.
Peça de aço, com que se abrem roscas, dentro de um orifício.
Rabiça do arado.
Travessão de lama, por onde os marnotos passam, da marinha-velha para a marinha-nova.
Que é do sexo ou do gênero masculino; masculino.
Forte, robusto.
Varonil, másculo.
\section{Machôa}
\begin{itemize}
\item {Grp. gram.:f.}
\end{itemize}
\begin{itemize}
\item {Utilização:Pop.}
\end{itemize}
O mesmo que \textunderscore machão\textunderscore .
\section{Machoca}
\begin{itemize}
\item {Grp. gram.:f.}
\end{itemize}
V. \textunderscore manchoca\textunderscore 
\section{Machôco}
\begin{itemize}
\item {Grp. gram.:m.}
\end{itemize}
V. \textunderscore manchoco\textunderscore 
\section{Macho-fêmea}
\begin{itemize}
\item {Grp. gram.:m.}
\end{itemize}
\begin{itemize}
\item {Utilização:Prov.}
\end{itemize}
\begin{itemize}
\item {Utilização:minh.}
\end{itemize}
Instrumento de carpinteiro, para formar saliência ou abrir sulco, em meio do bordo já aplainado da tábua.
\section{Machomharia}
\begin{itemize}
\item {Grp. gram.:f.}
\end{itemize}
\begin{itemize}
\item {Utilização:Ant.}
\end{itemize}
Certo gênero de ornamentos ou lavores de ourivezaria, de feição moirisca. Cf. S. R. Viterbo, \textunderscore Elucid.\textunderscore 
\section{Machorra}
\begin{itemize}
\item {fónica:chô}
\end{itemize}
\begin{itemize}
\item {Grp. gram.:adj. f.}
\end{itemize}
\begin{itemize}
\item {Utilização:Pop.}
\end{itemize}
\begin{itemize}
\item {Proveniência:(Do rad. de \textunderscore macho\textunderscore )}
\end{itemize}
O mesmo que \textunderscore estéril\textunderscore .
\section{Machorro}
\begin{itemize}
\item {fónica:chô}
\end{itemize}
\begin{itemize}
\item {Grp. gram.:m.}
\end{itemize}
\begin{itemize}
\item {Utilização:Prov.}
\end{itemize}
Muro de pedra sêca.
\section{Machuca}
\begin{itemize}
\item {Grp. gram.:f.}
\end{itemize}
Acto ou effeito de machucar.
\section{Machucação}
\begin{itemize}
\item {Grp. gram.:f.}
\end{itemize}
Acto de machucar.
\section{Machucador}
\begin{itemize}
\item {Grp. gram.:m.  e  adj.}
\end{itemize}
O que machuca.
\section{Machucadura}
\begin{itemize}
\item {Grp. gram.:f.}
\end{itemize}
O mesmo que \textunderscore machuca\textunderscore .
\section{Machucão}
\begin{itemize}
\item {Grp. gram.:m.}
\end{itemize}
\begin{itemize}
\item {Utilização:T. de Turquel}
\end{itemize}
Brenha muito cerrada.
\section{Machucar}
\begin{itemize}
\item {Grp. gram.:v. t.}
\end{itemize}
Esmagar (um corpo) com o pêso ou dureza de outro.
Debulhar (cereaes).
Pisar, triturar.
Amachucar.
(Cast. \textunderscore machucar\textunderscore )
\section{Machuca-rolhas}
\begin{itemize}
\item {Grp. gram.:m.}
\end{itemize}
Máquina, para comprimir rôlhas.
\section{Machucho}
\begin{itemize}
\item {Grp. gram.:m.  e  adj.}
\end{itemize}
\begin{itemize}
\item {Utilização:Pop.}
\end{itemize}
\begin{itemize}
\item {Proveniência:(De \textunderscore macho\textunderscore )}
\end{itemize}
Indivíduo rico ou influente.
Astuto, finório, machacaz:«\textunderscore machuchas mestras de tretas.\textunderscore »Castilho.«\textunderscore Cíceros machuchos.\textunderscore »Macedo, \textunderscore Burros\textunderscore , 384.
\section{Machuco}
\begin{itemize}
\item {Grp. gram.:m.}
\end{itemize}
\begin{itemize}
\item {Utilização:Prov.}
\end{itemize}
\begin{itemize}
\item {Utilização:alent.}
\end{itemize}
O mesmo que \textunderscore machuqueiro\textunderscore .
\section{Machuqueiro}
\begin{itemize}
\item {Grp. gram.:m.}
\end{itemize}
\begin{itemize}
\item {Utilização:Prov.}
\end{itemize}
\begin{itemize}
\item {Utilização:alent.}
\end{itemize}
Sobreiro pequeno, com menos de um metro de altura; macheiro.
\section{Machurra}
\begin{itemize}
\item {Grp. gram.:adj. f.}
\end{itemize}
\begin{itemize}
\item {Utilização:Prov.}
\end{itemize}
\begin{itemize}
\item {Utilização:minh.}
\end{itemize}
\begin{itemize}
\item {Grp. gram.:M.}
\end{itemize}
Diz-se da planta, que é tardeira em dar flôr ou fruto.
Mulher durázia e ordinária. Cf. Malheiro, \textunderscore Telles\textunderscore , 336.
(Cp. \textunderscore machorra\textunderscore )
\section{Maciamente}
\begin{itemize}
\item {Grp. gram.:adv.}
\end{itemize}
De modo macio.
Suavemente:«\textunderscore ...correndo maciamente as mãos por sobre uns cothurnos de escócia...\textunderscore »Camillo, \textunderscore Volcões\textunderscore , 132.
\section{Maciar}
\textunderscore v. t.\textunderscore  (e der.)
O mesmo que \textunderscore amaciar\textunderscore , etc. Cf. Filinto, XV, 30.
\section{Maciçaba}
\begin{itemize}
\item {Grp. gram.:m.}
\end{itemize}
\begin{itemize}
\item {Utilização:Bras}
\end{itemize}
O mesmo que \textunderscore penitente\textunderscore .
\section{Maciço}
\begin{itemize}
\item {Grp. gram.:m.}
\end{itemize}
\begin{itemize}
\item {Utilização:Geol.}
\end{itemize}
\begin{itemize}
\item {Grp. gram.:M.  e  adj.}
\end{itemize}
Formação eruptiva de grandes dimensões, desenvolvida irregularmente em qualquer direcção, antes de attingir a superfície do globo. Cf. Gonç. Guimarães, \textunderscore Geol.\textunderscore , 143.
O mesmo ou melhor que \textunderscore massiço\textunderscore .
(Cp. cast. \textunderscore mazizo\textunderscore )
\section{Macicote}
\begin{itemize}
\item {Grp. gram.:m.}
\end{itemize}
(V.massicote)
\section{Macieira}
\begin{itemize}
\item {Grp. gram.:f.}
\end{itemize}
Árvore fructífera, da fam. das rosáceas.
(Por \textunderscore maçãeira\textunderscore , de \textunderscore maçã\textunderscore )
\section{Maciez}
\begin{itemize}
\item {Grp. gram.:f.}
\end{itemize}
Qualidade daquillo que é macio.
\section{Macieza}
\begin{itemize}
\item {Grp. gram.:f.}
\end{itemize}
Qualidade daquillo que é macio.
\section{Macilência}
\begin{itemize}
\item {Grp. gram.:f.}
\end{itemize}
Aspecto daquelle ou daquillo que é macilento.
\section{Macilento}
\begin{itemize}
\item {Grp. gram.:adj.}
\end{itemize}
\begin{itemize}
\item {Proveniência:(Lat. \textunderscore macilentus\textunderscore )}
\end{itemize}
Magro.
Pállido.
Amortecido.
\section{Macina}
\begin{itemize}
\item {Grp. gram.:f.}
\end{itemize}
\begin{itemize}
\item {Proveniência:(Do rad. de \textunderscore macis\textunderscore )}
\end{itemize}
Substância gommosa, extrahida do macis.
\section{Macinho}
\begin{itemize}
\item {Grp. gram.:m.}
\end{itemize}
\begin{itemize}
\item {Utilização:Miner.}
\end{itemize}
\begin{itemize}
\item {Proveniência:(Do it. \textunderscore macigno\textunderscore )}
\end{itemize}
Variedade de arenito, de estructura granitoide, com lâminas de mica e grãos de quartzo e feldspatho.
\section{Macio}
\begin{itemize}
\item {Grp. gram.:adj.}
\end{itemize}
\begin{itemize}
\item {Proveniência:(Do ár. \textunderscore masi\textunderscore )}
\end{itemize}
Brando ao tacto, não áspero; liso; plano; suave; aprazível; agradável.
\section{Macis}
\begin{itemize}
\item {Grp. gram.:m.}
\end{itemize}
Designação vulgar do arillo da noz moscada.
Óleo, extrahido dêsse arillo.
\section{Macla}
\begin{itemize}
\item {Grp. gram.:f.}
\end{itemize}
\begin{itemize}
\item {Utilização:Geol.}
\end{itemize}
\begin{itemize}
\item {Proveniência:(Fr. \textunderscore macle\textunderscore , provavelmente do lat. \textunderscore macula\textunderscore )}
\end{itemize}
Agrupamento regular de crystaes homomorphos da mesma espécie, cada um dos quaes occupa posição invertida a respeito dos indivíduos vizinhos. Cf. Gonç. Guimarães, \textunderscore Geol.\textunderscore  59.
\section{Macleya}
\begin{itemize}
\item {Grp. gram.:f.}
\end{itemize}
\begin{itemize}
\item {Proveniência:(De \textunderscore Macley\textunderscore , n. p.)}
\end{itemize}
Gênero de plantas papaveráceas.
\section{Maclífero}
\begin{itemize}
\item {Grp. gram.:adj.}
\end{itemize}
\begin{itemize}
\item {Utilização:Miner.}
\end{itemize}
\begin{itemize}
\item {Proveniência:(De \textunderscore macla\textunderscore  + lat. \textunderscore ferre\textunderscore )}
\end{itemize}
Diz-se do xisto, que contém macla.
\section{Maclura}
\begin{itemize}
\item {Grp. gram.:f.}
\end{itemize}
\begin{itemize}
\item {Proveniência:(De \textunderscore Maclura\textunderscore , n. p.)}
\end{itemize}
Gênero de plantas moreáceas.
\section{Maco}
\begin{itemize}
\item {Grp. gram.:m.}
\end{itemize}
\begin{itemize}
\item {Utilização:Gír.}
\end{itemize}
Saco.
Dinheiro, bagalhoça.
\section{Maço}
\begin{itemize}
\item {Grp. gram.:m.}
\end{itemize}
\begin{itemize}
\item {Utilização:Bras}
\end{itemize}
Instrumento de madeira, com um pequeno cabo, para uso de escultores, carpinteiros, calceteiros, etc.
Martelo de pau.
Conjunto de papéis ou de outras coisas ligadas, e formando um só volume.
O mesmo que \textunderscore maçada\textunderscore .
(Cp. \textunderscore maça\textunderscore )
\section{Macó}
\begin{itemize}
\item {Grp. gram.:m.}
\end{itemize}
Ave africana, (\textunderscore cobivanellus lateralis\textunderscore , Smith).
\section{Macobío}
\begin{itemize}
\item {Grp. gram.:m.}
\end{itemize}
\begin{itemize}
\item {Utilização:Prov.}
\end{itemize}
\begin{itemize}
\item {Utilização:alent.}
\end{itemize}
Trabalhador do Norte, que no Alentejo se occupa temporariamente na limpeza de herdades ou no fabríco de carvão.
\section{Macobume}
\begin{itemize}
\item {Grp. gram.:m.}
\end{itemize}
O mesmo que \textunderscore vice-rei\textunderscore , entre os antigos Indígenas de Malaca. Cf. Barros, \textunderscore Déc.\textunderscore  II, l. IX, c. 7.
\section{Macoco}
\begin{itemize}
\item {fónica:cô}
\end{itemize}
\begin{itemize}
\item {Grp. gram.:m.}
\end{itemize}
Animal do Congo, talvez espécie de antílope.
\section{Macòcóa}
\begin{itemize}
\item {Grp. gram.:f.}
\end{itemize}
Árvore de Moçambique.
\section{Macola}
\begin{itemize}
\item {Grp. gram.:f.}
\end{itemize}
Fruto granuloso e purgativo dos sertões de Angola.
\section{Macololos}
\begin{itemize}
\item {Grp. gram.:m. pl.}
\end{itemize}
Numerosa tríbo do Alto Zambeze.
\section{Macoma}
\begin{itemize}
\item {Grp. gram.:f.}
\end{itemize}
Fruto da macomeira.
\section{Macomba}
\begin{itemize}
\item {Grp. gram.:f.}
\end{itemize}
O mesmo que \textunderscore macoma\textunderscore .
\section{Macombeira}
\begin{itemize}
\item {Grp. gram.:f.}
\end{itemize}
O mesmo que \textunderscore macomeira\textunderscore .
\section{Macomeira}
\begin{itemize}
\item {Grp. gram.:f.}
\end{itemize}
Palmeira do Brasil.
\section{Maçon}
\begin{itemize}
\item {Grp. gram.:m.}
\end{itemize}
(V. \textunderscore mação\textunderscore ^2)
\section{Maçonaria}
\begin{itemize}
\item {Grp. gram.:f.}
\end{itemize}
\begin{itemize}
\item {Utilização:gal}
\end{itemize}
\begin{itemize}
\item {Utilização:Des.}
\end{itemize}
\begin{itemize}
\item {Utilização:Ant.}
\end{itemize}
\begin{itemize}
\item {Proveniência:(Fr. \textunderscore maçonnerie\textunderscore )}
\end{itemize}
Sociedade secreta e philanthrópica, que usa como sýmbolos os instrumentos de architecto e pedreiro.
Arte de pedreiro.
Obra de talha, de relêvo, bordado a oiro e prata.
\section{Macone}
\begin{itemize}
\item {Grp. gram.:m.}
\end{itemize}
Peixe de Sofala, semelhante á lampreia.
\section{Maçónico}
\begin{itemize}
\item {Grp. gram.:adj.}
\end{itemize}
\begin{itemize}
\item {Grp. gram.:M.}
\end{itemize}
\begin{itemize}
\item {Utilização:Pop.}
\end{itemize}
\begin{itemize}
\item {Proveniência:(De \textunderscore mação\textunderscore ^2)}
\end{itemize}
Relativo á maçonaria.
O mesmo que \textunderscore mação\textunderscore ^2.
\section{Maçonismo}
\begin{itemize}
\item {Grp. gram.:m.}
\end{itemize}
O mesmo que \textunderscore maçonaria\textunderscore . Cf. Macedo, \textunderscore Burros\textunderscore , 234.
\section{Maçonizar}
\begin{itemize}
\item {Grp. gram.:v. t.}
\end{itemize}
Tornar mação ou pedreiro livre. Cf. Macedo, \textunderscore Burros\textunderscore , 338.
\section{Maconta}
\begin{itemize}
\item {Grp. gram.:f.}
\end{itemize}
\begin{itemize}
\item {Utilização:Ant.}
\end{itemize}
Pedaço de cobre, que servia de moéda em Moçambique.--Vem na \textunderscore Ethiópia Oriental\textunderscore , mas é possivel que seja êrro de copista ou de typógrapho, por \textunderscore macoutá\textunderscore .
\section{Maçorral}
\begin{itemize}
\item {Grp. gram.:adj.}
\end{itemize}
O mesmo que \textunderscore mazorral\textunderscore . Cf. \textunderscore Eufrosina\textunderscore , pról.
\section{Maçorro}
\begin{itemize}
\item {fónica:çô}
\end{itemize}
\begin{itemize}
\item {Grp. gram.:adj.}
\end{itemize}
O mesmo que \textunderscore mazorro\textunderscore :«\textunderscore ...as mais maçorras e rijas complicações\textunderscore ». Camillo, \textunderscore Regicida\textunderscore , 111.
\section{Macota}
\begin{itemize}
\item {Grp. gram.:m.}
\end{itemize}
\begin{itemize}
\item {Utilização:Bras}
\end{itemize}
\begin{itemize}
\item {Utilização:T. de Angola}
\end{itemize}
\begin{itemize}
\item {Grp. gram.:Adj.}
\end{itemize}
\begin{itemize}
\item {Utilização:Bras}
\end{itemize}
Homem de prestígio ou influência numa localidade.
Indivíduo importante do séquito dos sobas.--Segundo Capello e Ivens, (I, 49), é cada um dos indivíduos que, com os secúlos, constituem a \textunderscore côrte\textunderscore  do soba. Segundo Serpa Pinto, (I, 141), é o mesmo que secúlo e faz parte de uma espécie de conselho, a que o soba submete as suas deliberações, mas de cuja opinião raramente se importa.
Grande; bom.
Apto, que sabe do seu offício.
Rico: \textunderscore fazendeiro macota\textunderscore .
Formoso: \textunderscore moça macota\textunderscore .
\section{Macote}
\begin{itemize}
\item {Grp. gram.:m.}
\end{itemize}
\begin{itemize}
\item {Utilização:Gír.}
\end{itemize}
\begin{itemize}
\item {Proveniência:(De \textunderscore maco\textunderscore )}
\end{itemize}
Sacola.
\section{Maçote}
\begin{itemize}
\item {Grp. gram.:m.}
\end{itemize}
\begin{itemize}
\item {Utilização:Gír.}
\end{itemize}
\begin{itemize}
\item {Proveniência:(De \textunderscore maço\textunderscore )}
\end{itemize}
Nádegas do homem.
\section{Macouba}
\begin{itemize}
\item {Grp. gram.:f.}
\end{itemize}
\begin{itemize}
\item {Proveniência:(De \textunderscore Macouba\textunderscore , n. p. da região, que produz aquelle tabaco, na Martinica)}
\end{itemize}
Variedade de tabaco, cujo cheiro faz lembrar o das rosas.
\section{Macoutá}
\begin{itemize}
\item {Grp. gram.:f.}
\end{itemize}
Moéda, usada entre algumas tríbos da África.
(Cp. \textunderscore macuta\textunderscore )
\section{Macóvia}
\begin{itemize}
\item {Grp. gram.:f.}
\end{itemize}
\begin{itemize}
\item {Utilização:Gír.}
\end{itemize}
O mesmo que \textunderscore moscóvia\textunderscore ^1.
\section{Macradênia}
\begin{itemize}
\item {Grp. gram.:f.}
\end{itemize}
Gênero de orchídeas.
\section{Macramé}
\begin{itemize}
\item {Grp. gram.:m.}
\end{itemize}
\begin{itemize}
\item {Proveniência:(T. fr.)}
\end{itemize}
Espécie de franja, feita de linha ennodada e semelhante á obra das bolsas de caça.
\section{Macrantho}
\begin{itemize}
\item {Grp. gram.:adj.}
\end{itemize}
\begin{itemize}
\item {Utilização:Bot.}
\end{itemize}
\begin{itemize}
\item {Proveniência:(Do gr. \textunderscore makros\textunderscore  + \textunderscore anthos\textunderscore )}
\end{itemize}
Diz-se das plantas, que têm grandes flôres.
\section{Macranto}
\begin{itemize}
\item {Grp. gram.:adj.}
\end{itemize}
\begin{itemize}
\item {Utilização:Bot.}
\end{itemize}
\begin{itemize}
\item {Proveniência:(Do gr. \textunderscore makros\textunderscore  + \textunderscore anthos\textunderscore )}
\end{itemize}
Diz-se das plantas, que têm grandes flôres.
\section{Má-criação}
\begin{itemize}
\item {Grp. gram.:f.}
\end{itemize}
Qualidade de quem é grosseiro ou incivil.
Acto ou dito incivil; grossaria.
\section{Macro...}
\begin{itemize}
\item {Grp. gram.:pref.}
\end{itemize}
\begin{itemize}
\item {Proveniência:(Do gr. \textunderscore makros\textunderscore )}
\end{itemize}
(designativo de \textunderscore grande\textunderscore )
\section{Macrobia}
\begin{itemize}
\item {Grp. gram.:f.}
\end{itemize}
Estado de quem é macróbio; longevidade, longa vida. Cf. Castilho, \textunderscore Fastos\textunderscore , II, 62.
\section{Macróbio}
\begin{itemize}
\item {Grp. gram.:m.  e  adj.}
\end{itemize}
\begin{itemize}
\item {Grp. gram.:Pl.}
\end{itemize}
\begin{itemize}
\item {Proveniência:(Gr. \textunderscore makrobios\textunderscore )}
\end{itemize}
O que vive muito.
Que tem idade muito avançada.
Antigo povo da Ethiópia, notável pela habitual longevidade dos seus indivíduos.
\section{Macrobiota}
\begin{itemize}
\item {Grp. gram.:m.}
\end{itemize}
\begin{itemize}
\item {Proveniência:(Do gr. \textunderscore makros\textunderscore  + \textunderscore bios\textunderscore )}
\end{itemize}
Designação moderna dos animaes microscópicos, que vivem no musgo e no pó dos telhados, e que precedentemente se chamavam \textunderscore tardígrados\textunderscore .
\section{Macrobiótica}
\begin{itemize}
\item {Grp. gram.:f.}
\end{itemize}
\begin{itemize}
\item {Proveniência:(Do gr. \textunderscore makros\textunderscore  + \textunderscore biotikos\textunderscore )}
\end{itemize}
Parte da hygiene, que se occupa dos meios de prolongar a vida.
\section{Macrocefalia}
\begin{itemize}
\item {Grp. gram.:f.}
\end{itemize}
Desenvolvimento anormal da cabeça ou de uma parte dela.
Qualidade do que é macrocéfalo.
\section{Macrocefálico}
\begin{itemize}
\item {Grp. gram.:adj.}
\end{itemize}
Relativo á macrocefalia.
\section{Macrocéfalo}
\begin{itemize}
\item {Grp. gram.:m.  e  adj.}
\end{itemize}
\begin{itemize}
\item {Proveniência:(Do gr. \textunderscore makros\textunderscore  + \textunderscore kephale\textunderscore )}
\end{itemize}
O que tem anormalmente desenvolvido o encéfalo ou uma parte dele.
\section{Macrocephalia}
\begin{itemize}
\item {Grp. gram.:f.}
\end{itemize}
Desenvolvimento anormal da cabeça ou de uma parte della.
Qualidade do que é macrocéphalo.
\section{Macrocephálico}
\begin{itemize}
\item {Grp. gram.:adj.}
\end{itemize}
Relativo á macrocephalia.
\section{Macrocéphalo}
\begin{itemize}
\item {Grp. gram.:m.  e  adj.}
\end{itemize}
\begin{itemize}
\item {Proveniência:(Do gr. \textunderscore makros\textunderscore  + \textunderscore kephale\textunderscore )}
\end{itemize}
O que tem anormalmente desenvolvido o encéphalo ou uma parte delle.
\section{Macrocerco}
\begin{itemize}
\item {Grp. gram.:adj.}
\end{itemize}
\begin{itemize}
\item {Utilização:Zool.}
\end{itemize}
\begin{itemize}
\item {Grp. gram.:M.}
\end{itemize}
\begin{itemize}
\item {Proveniência:(Do gr. \textunderscore makros\textunderscore  + \textunderscore kerkos\textunderscore )}
\end{itemize}
Que tem cauda longa.
Gênero de aves.
\section{Macrócero}
\begin{itemize}
\item {Grp. gram.:adj.}
\end{itemize}
\begin{itemize}
\item {Utilização:Zool.}
\end{itemize}
\begin{itemize}
\item {Proveniência:(Do gr. \textunderscore makros\textunderscore  + \textunderscore keras\textunderscore )}
\end{itemize}
Que tem cornos longos ou antennas compridas.
\section{Macrochiria}
\begin{itemize}
\item {fónica:qui}
\end{itemize}
\begin{itemize}
\item {Grp. gram.:f.}
\end{itemize}
\begin{itemize}
\item {Proveniência:(De \textunderscore macróchiro\textunderscore )}
\end{itemize}
Monstruosidade, caracterizada pelo comprimento excessivo das mãos.
\section{Macróchiro}
\begin{itemize}
\item {fónica:qui}
\end{itemize}
\begin{itemize}
\item {Grp. gram.:adj.}
\end{itemize}
\begin{itemize}
\item {Proveniência:(Do gr. \textunderscore makros\textunderscore  + \textunderscore kheir\textunderscore )}
\end{itemize}
Que tem grandes mãos.
\section{Macróchloa}
\begin{itemize}
\item {Grp. gram.:f.}
\end{itemize}
\begin{itemize}
\item {Utilização:Bot.}
\end{itemize}
\begin{itemize}
\item {Proveniência:(Do gr. \textunderscore makros\textunderscore  + \textunderscore khloa\textunderscore )}
\end{itemize}
Designação scientífica do baracejo.
\section{Macrócloa}
\begin{itemize}
\item {Grp. gram.:f.}
\end{itemize}
\begin{itemize}
\item {Utilização:Bot.}
\end{itemize}
\begin{itemize}
\item {Proveniência:(Do gr. \textunderscore makros\textunderscore  + \textunderscore khloa\textunderscore )}
\end{itemize}
Designação científica do baracejo.
\section{Macrócomo}
\begin{itemize}
\item {Grp. gram.:adj.}
\end{itemize}
\begin{itemize}
\item {Proveniência:(Do gr. \textunderscore makros\textunderscore  + \textunderscore kome\textunderscore )}
\end{itemize}
Que tem longa cabelleira ou longos filamentos.
\section{Macrocosmo}
\begin{itemize}
\item {Grp. gram.:m.}
\end{itemize}
\begin{itemize}
\item {Proveniência:(Do gr. \textunderscore makros\textunderscore  + \textunderscore kosmos\textunderscore )}
\end{itemize}
O grande mundo, o conjunto de todas as coisas, (em opposição ao microcosmo).
\section{Macro-crystallino}
\begin{itemize}
\item {Grp. gram.:adj.}
\end{itemize}
\begin{itemize}
\item {Utilização:Geol.}
\end{itemize}
Diz-se do estado dos minerais, em que as moléculas não obedecem a nenhuma orientação regular, e cujas propriedades podem observar-se sem o microscópio.
\section{Macrodactilia}
\begin{itemize}
\item {Grp. gram.:f.}
\end{itemize}
\begin{itemize}
\item {Proveniência:(De \textunderscore macrodáctilo\textunderscore )}
\end{itemize}
Monstruosidade, caracterizada pelo excessivo desenvolvimento dos dedos.
\section{Macrodáctilo}
\begin{itemize}
\item {Grp. gram.:adj.}
\end{itemize}
\begin{itemize}
\item {Grp. gram.:M. pl.}
\end{itemize}
\begin{itemize}
\item {Proveniência:(Do gr. \textunderscore makros\textunderscore  + \textunderscore daktulos\textunderscore )}
\end{itemize}
Que tem os dedos muito compridos.
Que tem prolongamentos, em fórma de dedos.
Família de aves pernaltas.
\section{Macrodactylia}
\begin{itemize}
\item {Grp. gram.:f.}
\end{itemize}
\begin{itemize}
\item {Proveniência:(De \textunderscore macrodáctylo\textunderscore )}
\end{itemize}
Monstruosidade, caracterizada pelo excessivo desenvolvimento dos dedos.
\section{Macrodáctylo}
\begin{itemize}
\item {Grp. gram.:adj.}
\end{itemize}
\begin{itemize}
\item {Grp. gram.:M. pl.}
\end{itemize}
\begin{itemize}
\item {Proveniência:(Do gr. \textunderscore makros\textunderscore  + \textunderscore daktulos\textunderscore )}
\end{itemize}
Que tem os dedos muito compridos.
Que tem prolongamentos, em fórma de dedos.
Família de aves pernaltas.
\section{Macrodiagonal}
\begin{itemize}
\item {Grp. gram.:f.}
\end{itemize}
\begin{itemize}
\item {Utilização:Geol.}
\end{itemize}
\begin{itemize}
\item {Proveniência:(De \textunderscore macro...\textunderscore  + \textunderscore diagonal\textunderscore )}
\end{itemize}
Um dos três eixos dos crystaes do systema orthorhômbico.
\section{Macrodoma}
\begin{itemize}
\item {Grp. gram.:f.}
\end{itemize}
\begin{itemize}
\item {Utilização:Neol.}
\end{itemize}
\begin{itemize}
\item {Proveniência:(Do gr. \textunderscore makros\textunderscore  + \textunderscore doma\textunderscore )}
\end{itemize}
Fórma de crystal, equivalente a um prisma transversal, cujo eixo é a macrodiagonal e cuja secção recta é formada pelas arestas culminantes mais curtas das protopyrâmides.
\section{Macroglosso}
\begin{itemize}
\item {Grp. gram.:adj.}
\end{itemize}
\begin{itemize}
\item {Proveniência:(Do gr. \textunderscore makros\textunderscore  + \textunderscore glossa\textunderscore )}
\end{itemize}
Que tem língua muito volumosa ou muito longa.
\section{Macrolóbio}
\begin{itemize}
\item {Grp. gram.:m.}
\end{itemize}
Gênero de plantas leguminosas.
\section{Macrólofo}
\begin{itemize}
\item {Grp. gram.:adj.}
\end{itemize}
\begin{itemize}
\item {Utilização:Zool.}
\end{itemize}
Que tem penacho na cabeça.
\section{Macrologia}
\begin{itemize}
\item {Grp. gram.:f.}
\end{itemize}
\begin{itemize}
\item {Proveniência:(Do gr. \textunderscore makros\textunderscore  + \textunderscore logos\textunderscore )}
\end{itemize}
Estilo diffuso, prolixidade no falar.
\section{Macrólopho}
\begin{itemize}
\item {Grp. gram.:adj.}
\end{itemize}
\begin{itemize}
\item {Utilização:Zool.}
\end{itemize}
Que tem pennacho na cabeça.
\section{Macromelia}
\begin{itemize}
\item {Grp. gram.:f.}
\end{itemize}
\begin{itemize}
\item {Proveniência:(Do gr. \textunderscore makros\textunderscore  + \textunderscore melos\textunderscore )}
\end{itemize}
Desenvolvimento excessivo de qualquer membro.
\section{Macropétalo}
\begin{itemize}
\item {Grp. gram.:adj.}
\end{itemize}
\begin{itemize}
\item {Proveniência:(Do gr. \textunderscore makros\textunderscore  + \textunderscore petalon\textunderscore )}
\end{itemize}
Que tem grandes pétalas.
\section{Macrofilo}
\begin{itemize}
\item {Grp. gram.:adj.}
\end{itemize}
\begin{itemize}
\item {Utilização:Bot.}
\end{itemize}
\begin{itemize}
\item {Proveniência:(Do gr. \textunderscore makros\textunderscore  + \textunderscore phullon\textunderscore )}
\end{itemize}
Diz-se das plantas que têm fôlhas grandes.
\section{Macrophyllo}
\begin{itemize}
\item {Grp. gram.:adj.}
\end{itemize}
\begin{itemize}
\item {Utilização:Bot.}
\end{itemize}
\begin{itemize}
\item {Proveniência:(Do gr. \textunderscore makros\textunderscore  + \textunderscore phullon\textunderscore )}
\end{itemize}
Diz-se das plantas que têm fôlhas grandes.
\section{Macropia}
\begin{itemize}
\item {Grp. gram.:f.}
\end{itemize}
O mesmo que \textunderscore macropsia\textunderscore .
\section{Macropinacóide}
\begin{itemize}
\item {Grp. gram.:m.}
\end{itemize}
\begin{itemize}
\item {Proveniência:(Do gr. \textunderscore makros\textunderscore  + \textunderscore pinax\textunderscore )}
\end{itemize}
Fórma de crystal, limitada por dois planos parallelos entre si e equidistantes do plano de symetria que passa pelo eixo principal e pela macrodiagonal.
\section{Macrópio}
\begin{itemize}
\item {Grp. gram.:adj.}
\end{itemize}
O mesmo que \textunderscore macrópode\textunderscore .
\section{Macropneia}
\begin{itemize}
\item {Grp. gram.:f.}
\end{itemize}
\begin{itemize}
\item {Utilização:Med.}
\end{itemize}
\begin{itemize}
\item {Proveniência:(Do gr. \textunderscore makros\textunderscore  + \textunderscore penein\textunderscore )}
\end{itemize}
Respiração extensa ou lenta.
\section{Macrópode}
\begin{itemize}
\item {Grp. gram.:adj.}
\end{itemize}
\begin{itemize}
\item {Proveniência:(Do gr. \textunderscore makros\textunderscore  + \textunderscore pous\textunderscore , \textunderscore podos\textunderscore )}
\end{itemize}
Que tem longos pés, barbatanas ou pedúnculos.
\section{Macropodia}
\begin{itemize}
\item {Grp. gram.:f.}
\end{itemize}
Qualidade de macrópode.
\section{Macropódio}
\begin{itemize}
\item {Grp. gram.:adj.}
\end{itemize}
O mesmo que \textunderscore macrópode\textunderscore .
\section{Macrópomo}
\begin{itemize}
\item {Grp. gram.:adj.}
\end{itemize}
\begin{itemize}
\item {Proveniência:(Do gr. \textunderscore makros\textunderscore  + \textunderscore poma\textunderscore )}
\end{itemize}
Que tem grandes opérculos.
\section{Macroprosopia}
\begin{itemize}
\item {Grp. gram.:f.}
\end{itemize}
\begin{itemize}
\item {Proveniência:(Do gr. \textunderscore makros\textunderscore  + \textunderscore prosopon\textunderscore )}
\end{itemize}
Monstruosidade, caracterizada pelo desenvolvimento excessivo da face.
\section{Macropsia}
\begin{itemize}
\item {Grp. gram.:f.}
\end{itemize}
\begin{itemize}
\item {Proveniência:(Do gr. \textunderscore makros\textunderscore  + \textunderscore ops\textunderscore )}
\end{itemize}
Estado morbido, em que os objectos da visão aumentam de volume.
\section{Macróptero}
\begin{itemize}
\item {Grp. gram.:adj.}
\end{itemize}
\begin{itemize}
\item {Proveniência:(Do gr. \textunderscore makros\textunderscore  + \textunderscore pteron\textunderscore )}
\end{itemize}
Que tem grandes asas, ou grandes appêndices em fórma de asas, ou grandes barbatanas.
\section{Macroquiria}
\begin{itemize}
\item {Grp. gram.:f.}
\end{itemize}
\begin{itemize}
\item {Proveniência:(De \textunderscore macróquiro\textunderscore )}
\end{itemize}
Monstruosidade, caracterizada pelo comprimento excessivo das mãos.
\section{Macróquiro}
\begin{itemize}
\item {Grp. gram.:adj.}
\end{itemize}
\begin{itemize}
\item {Proveniência:(Do gr. \textunderscore makros\textunderscore  + \textunderscore kheir\textunderscore )}
\end{itemize}
Que tem grandes mãos.
\section{Macrorrhino}
\begin{itemize}
\item {Grp. gram.:m.}
\end{itemize}
\begin{itemize}
\item {Proveniência:(Do gr. \textunderscore makros\textunderscore  + \textunderscore rhin\textunderscore )}
\end{itemize}
Espécie de phoca.
\section{Macrorrhizo}
\begin{itemize}
\item {Grp. gram.:adj.}
\end{itemize}
\begin{itemize}
\item {Proveniência:(Do gr. \textunderscore makros\textunderscore  + \textunderscore rhiza\textunderscore )}
\end{itemize}
Que tem grandes raízes.
\section{Macrorrhynco}
\begin{itemize}
\item {Grp. gram.:adj.}
\end{itemize}
\begin{itemize}
\item {Utilização:Zool.}
\end{itemize}
\begin{itemize}
\item {Proveniência:(Do gr. \textunderscore makros\textunderscore  + \textunderscore rhunkhos\textunderscore )}
\end{itemize}
Que tem bico ou focinho comprido.
\section{Macrorrinco}
\begin{itemize}
\item {Grp. gram.:adj.}
\end{itemize}
\begin{itemize}
\item {Utilização:Zool.}
\end{itemize}
\begin{itemize}
\item {Proveniência:(Do gr. \textunderscore makros\textunderscore  + \textunderscore rhunkhos\textunderscore )}
\end{itemize}
Que tem bico ou focinho comprido.
\section{Macrorrino}
\begin{itemize}
\item {Grp. gram.:m.}
\end{itemize}
\begin{itemize}
\item {Proveniência:(Do gr. \textunderscore makros\textunderscore  + \textunderscore rhin\textunderscore )}
\end{itemize}
Espécie de foca.
\section{Macrorrizo}
\begin{itemize}
\item {Grp. gram.:adj.}
\end{itemize}
\begin{itemize}
\item {Proveniência:(Do gr. \textunderscore makros\textunderscore  + \textunderscore rhiza\textunderscore )}
\end{itemize}
Que tem grandes raízes.
\section{Macroscelia}
\begin{itemize}
\item {Grp. gram.:f.}
\end{itemize}
\begin{itemize}
\item {Proveniência:(Do gr. \textunderscore makros\textunderscore  + \textunderscore skelos\textunderscore )}
\end{itemize}
Desenvolvimento monstruoso das pernas.
\section{Macroscélido}
\begin{itemize}
\item {Grp. gram.:m.}
\end{itemize}
\begin{itemize}
\item {Grp. gram.:Adj.}
\end{itemize}
\begin{itemize}
\item {Proveniência:(Do gr. \textunderscore makros\textunderscore  + \textunderscore kelos\textunderscore )}
\end{itemize}
Animal carnívoro e insectívoro, de focinho alongado.
Diz se dos insectos, que têm muito desenvolvidos os membros posteriores.
\section{Macróscio}
\begin{itemize}
\item {Grp. gram.:adj.}
\end{itemize}
\begin{itemize}
\item {Utilização:Geogr.}
\end{itemize}
\begin{itemize}
\item {Proveniência:(Do gr. \textunderscore makros\textunderscore  + \textunderscore skia\textunderscore , sombra)}
\end{itemize}
Diz-se dos habitantes do Globo, que recebem muito obliquamente os raios do sol, e cujo corpo, ao meio dia, projecta grande sombra.
\section{Macroscópico}
\begin{itemize}
\item {Grp. gram.:adj.}
\end{itemize}
\begin{itemize}
\item {Proveniência:(Do gr. \textunderscore makros\textunderscore  + \textunderscore skopein\textunderscore )}
\end{itemize}
Relativo á observação de coisas grandes. Que é visível, sem o auxílio do microscópio.
\section{Macrosomatia}
\begin{itemize}
\item {fónica:so}
\end{itemize}
\begin{itemize}
\item {Grp. gram.:m.}
\end{itemize}
\begin{itemize}
\item {Proveniência:(Do gr. \textunderscore makros\textunderscore  + \textunderscore soma\textunderscore )}
\end{itemize}
Monstruosidade, caracterizada pela excessiva grossura ou grandeza de todo o corpo.
\section{Macrossomatia}
\begin{itemize}
\item {Grp. gram.:m.}
\end{itemize}
\begin{itemize}
\item {Proveniência:(Do gr. \textunderscore makros\textunderscore  + \textunderscore soma\textunderscore )}
\end{itemize}
Monstruosidade, caracterizada pela excessiva grossura ou grandeza de todo o corpo.
\section{Macrósticho}
\begin{itemize}
\item {fónica:co}
\end{itemize}
\begin{itemize}
\item {Grp. gram.:adj.}
\end{itemize}
\begin{itemize}
\item {Proveniência:(Do gr. \textunderscore makros\textunderscore  + \textunderscore stikhos\textunderscore )}
\end{itemize}
Que está escrito em linhas muito compridas.
\section{Macróstico}
\begin{itemize}
\item {Grp. gram.:adj.}
\end{itemize}
\begin{itemize}
\item {Proveniência:(Do gr. \textunderscore makros\textunderscore  + \textunderscore stikhos\textunderscore )}
\end{itemize}
Que está escrito em linhas muito compridas.
\section{Macróstilo}
\begin{itemize}
\item {Grp. gram.:adj.}
\end{itemize}
\begin{itemize}
\item {Utilização:Bot.}
\end{itemize}
\begin{itemize}
\item {Proveniência:(Do gr. \textunderscore makros\textunderscore  + \textunderscore stule\textunderscore )}
\end{itemize}
Que tem estiletes compridos.
\section{Macróstomos}
\begin{itemize}
\item {Grp. gram.:m. pl.}
\end{itemize}
\begin{itemize}
\item {Proveniência:(Do gr. \textunderscore makros\textunderscore  + \textunderscore stoma\textunderscore )}
\end{itemize}
Família de molluscos, cuja concha não tem opérculos e tem as bordas desunidas.
\section{Macróstylo}
\begin{itemize}
\item {Grp. gram.:adj.}
\end{itemize}
\begin{itemize}
\item {Utilização:Bot.}
\end{itemize}
\begin{itemize}
\item {Proveniência:(Do gr. \textunderscore makros\textunderscore  + \textunderscore stule\textunderscore )}
\end{itemize}
Que tem estiletes compridos.
\section{Macrotársico}
\begin{itemize}
\item {Grp. gram.:adj.}
\end{itemize}
\begin{itemize}
\item {Utilização:Zool.}
\end{itemize}
\begin{itemize}
\item {Proveniência:(De \textunderscore macro...\textunderscore  + \textunderscore tarso\textunderscore )}
\end{itemize}
Que tem tarsos compridos.
\section{Macrozâmia}
\begin{itemize}
\item {Grp. gram.:f.}
\end{itemize}
\begin{itemize}
\item {Utilização:Bot.}
\end{itemize}
\begin{itemize}
\item {Proveniência:(Do gr. \textunderscore makros\textunderscore  + \textunderscore zamia\textunderscore )}
\end{itemize}
Gênero de fêtos.
\section{Macruro}
\begin{itemize}
\item {Grp. gram.:adj.}
\end{itemize}
\begin{itemize}
\item {Grp. gram.:M. pl.}
\end{itemize}
\begin{itemize}
\item {Proveniência:(Do gr. \textunderscore makros\textunderscore  + \textunderscore oura\textunderscore )}
\end{itemize}
Que tem cauda longa.
Família de crustáceos decápodes.
\section{Mactra}
\begin{itemize}
\item {Grp. gram.:f.}
\end{itemize}
Mollusco acéphalo.
\section{Mactráceo}
\begin{itemize}
\item {Grp. gram.:adj.}
\end{itemize}
\begin{itemize}
\item {Grp. gram.:Pl.}
\end{itemize}
Relativo ou semelhante á mactra.
Família de molluscos, que tem por typo a mactra.
\section{Mactrismo}
\begin{itemize}
\item {Grp. gram.:m.}
\end{itemize}
Espécie de dança cómica, entre os antigos Gregos.
\section{Macua}
\begin{itemize}
\item {Grp. gram.:m.}
\end{itemize}
\begin{itemize}
\item {Grp. gram.:Pl.}
\end{itemize}
Idioma da família cafreal, falado entre os Indígenas de Moçambique.
Povo cafreal, nas vizinhanças do lago Chirua.
\section{Macuá}
\begin{itemize}
\item {Grp. gram.:m.}
\end{itemize}
\begin{itemize}
\item {Utilização:Ant.}
\end{itemize}
Pescador, na Índia portuguesa.
\section{Macuácuas}
\begin{itemize}
\item {Grp. gram.:m. pl.}
\end{itemize}
Tríbo cafreal, vizinha dos Vátuas; o mesmo que macuas.
\section{Macuba}
\begin{itemize}
\item {Grp. gram.:f.}
\end{itemize}
O mesmo que \textunderscore macouba\textunderscore .
\section{Macubéa}
\begin{itemize}
\item {Grp. gram.:f.}
\end{itemize}
Árvore gutífera da Guiana.
\section{Macubeia}
\begin{itemize}
\item {Grp. gram.:f.}
\end{itemize}
Árvore gutífera da Guiana.
\section{Macuca}
\begin{itemize}
\item {Grp. gram.:f.}
\end{itemize}
Espécie de pereira silvestre.
\section{Macuca}
\begin{itemize}
\item {Grp. gram.:f.}
\end{itemize}
Moéda, entre os Negros de Angola.
\section{Macucana}
\begin{itemize}
\item {Grp. gram.:f.}
\end{itemize}
Espécie de inambu.
\section{Macuco}
\begin{itemize}
\item {Grp. gram.:m.}
\end{itemize}
\begin{itemize}
\item {Utilização:Bras}
\end{itemize}
Pássaro, também conhecido por \textunderscore melro das rochas\textunderscore , (\textunderscore monticola saxatilis\textunderscore , Lin.).
Árvore africana, esguia e ramosa, de fôlhas simples, dioicas e axillares.
Espécie de mandioca.
\section{Macucu}
\begin{itemize}
\item {Grp. gram.:m.}
\end{itemize}
Planta ilicínea do Brasil.
Ave brasileira.
\section{Macucu-mirim}
\begin{itemize}
\item {Grp. gram.:m.}
\end{itemize}
\begin{itemize}
\item {Utilização:Bras}
\end{itemize}
Árvore, que cresce á beira dos rios, no valle do Amazonas.
\section{Maçudo}
\begin{itemize}
\item {Grp. gram.:adj.}
\end{itemize}
\begin{itemize}
\item {Utilização:Fig.}
\end{itemize}
\begin{itemize}
\item {Proveniência:(Do rad. de \textunderscore maçar\textunderscore )}
\end{itemize}
Que tem fórma de maça.
Maçador, indigesto, monótono, (falando-se de um escrito ou de um discurso).
\section{Macuim}
\begin{itemize}
\item {Grp. gram.:m.}
\end{itemize}
\begin{itemize}
\item {Utilização:Bras}
\end{itemize}
Variedade de mosquitos.
\section{Macujé}
\begin{itemize}
\item {Grp. gram.:f.}
\end{itemize}
Fruta doce do Brasil.
\section{Mácula}
\begin{itemize}
\item {Grp. gram.:f.}
\end{itemize}
\begin{itemize}
\item {Utilização:Fig.}
\end{itemize}
\begin{itemize}
\item {Proveniência:(Lat. \textunderscore macula\textunderscore )}
\end{itemize}
Nódoa.
Labéu.
Infâmia; desdoiro.
\section{Maculado}
\begin{itemize}
\item {Grp. gram.:adj.}
\end{itemize}
\begin{itemize}
\item {Utilização:Bot.}
\end{itemize}
\begin{itemize}
\item {Proveniência:(De \textunderscore macular\textunderscore )}
\end{itemize}
Diz-se das fôlhas, que têm manchas irregulares.
\section{Maculador}
\begin{itemize}
\item {Grp. gram.:adj.}
\end{itemize}
\begin{itemize}
\item {Proveniência:(De \textunderscore macular\textunderscore )}
\end{itemize}
Que causa manchas; que infama; que desacredita; que desdoira.
\section{Macular}
\begin{itemize}
\item {Grp. gram.:v. t.}
\end{itemize}
\begin{itemize}
\item {Utilização:Fig.}
\end{itemize}
\begin{itemize}
\item {Proveniência:(Lat. \textunderscore maculare\textunderscore )}
\end{itemize}
Pôr manchas em; sujar.
Infamar; desdoirar.
\section{Maculatura}
\begin{itemize}
\item {Grp. gram.:f.}
\end{itemize}
\begin{itemize}
\item {Proveniência:(Do rad. de \textunderscore macular\textunderscore )}
\end{itemize}
Fôlha ou fôlhas mal impressas; papel ordinário para embrulho.
\section{Maculável}
\begin{itemize}
\item {Grp. gram.:adj.}
\end{itemize}
Que se póde macular.
Que é susceptível de se ennodoar ou de se emporcalhar.
Que póde peccar ou incorrer em culpas ou defeitos.
\section{Maculiforme}
\begin{itemize}
\item {Grp. gram.:adj.}
\end{itemize}
\begin{itemize}
\item {Proveniência:(De \textunderscore mácula\textunderscore  + \textunderscore fórma\textunderscore )}
\end{itemize}
Que tem fórma de pequena mancha.
\section{Maculirostro}
\begin{itemize}
\item {fónica:rós}
\end{itemize}
\begin{itemize}
\item {Grp. gram.:adj.}
\end{itemize}
\begin{itemize}
\item {Proveniência:(Do lat. \textunderscore macula\textunderscore  + \textunderscore rostrum\textunderscore )}
\end{itemize}
Que tem o bico malhado, (falando-se de aves).
\section{Maculirrostro}
\begin{itemize}
\item {Grp. gram.:adj.}
\end{itemize}
\begin{itemize}
\item {Proveniência:(Do lat. \textunderscore macula\textunderscore  + \textunderscore rostrum\textunderscore )}
\end{itemize}
Que tem o bico malhado, (falando-se de aves).
\section{Macúlo}
\begin{itemize}
\item {Grp. gram.:m.}
\end{itemize}
\begin{itemize}
\item {Utilização:Bras}
\end{itemize}
Espécie de diarrheia, com relaxamento do esphíncter e dilatação da respectiva abertura.
(Talvez do bundo)
\section{Maculoso}
\begin{itemize}
\item {Grp. gram.:adj.}
\end{itemize}
\begin{itemize}
\item {Proveniência:(Lat. \textunderscore maculosus\textunderscore )}
\end{itemize}
O mesmo que \textunderscore maculado\textunderscore .
\section{Macuma}
\begin{itemize}
\item {Grp. gram.:f.}
\end{itemize}
O mesmo que \textunderscore mucama\textunderscore .
\section{Macumã}
\begin{itemize}
\item {Grp. gram.:m.}
\end{itemize}
Substância, extrahida do miolo da palmeira, e empregada como tempêro culinário, entre os Brasileiros.
\section{Macuman}
\begin{itemize}
\item {Grp. gram.:m.}
\end{itemize}
Substância, extrahida do miolo da palmeira, e empregada como tempêro culinário, entre os Brasileiros.
\section{Macumanganhe}
\begin{itemize}
\item {Grp. gram.:m.}
\end{itemize}
Arbusto africano, de caule subterrâneo, fôlhas alternas, cheiro balsâmico, e flôres miúdas, amarelas, muito aromáticas.
\section{Macumans}
\begin{itemize}
\item {Grp. gram.:m. pl.}
\end{itemize}
Indígenas brasileiros da região do Amazonas.
\section{Macumbé}
\begin{itemize}
\item {Grp. gram.:m.}
\end{itemize}
Árvore africana, (\textunderscore swartzia madagascariensis\textunderscore ).
\section{Macumbi}
\begin{itemize}
\item {Grp. gram.:m.}
\end{itemize}
Árvore africana, ornamental e medicinal, (\textunderscore odina-acida\textunderscore , Walp.).
\section{Macuna}
\begin{itemize}
\item {Grp. gram.:f.}
\end{itemize}
\begin{itemize}
\item {Utilização:Bras}
\end{itemize}
Planta hortense e ornamental, (\textunderscore dolichos urens\textunderscore , Lin.).
\section{Macuná}
\begin{itemize}
\item {Grp. gram.:m.}
\end{itemize}
Árvore brasileira, (\textunderscore macuna prurens\textunderscore , De-Cand.).
\section{Macunas}
\begin{itemize}
\item {Grp. gram.:m. pl.}
\end{itemize}
Índios selvagens das margens do Apaporis, no Brasil.
\section{Macundi}
\begin{itemize}
\item {Grp. gram.:m.}
\end{itemize}
Espécie de feijão africano, (\textunderscore vigua sinensis\textunderscore , Endl.).
\section{Macundi-magima}
\begin{itemize}
\item {Grp. gram.:m.}
\end{itemize}
Arbusto africano, da fam. das leguminosas.
\section{Macunhapamba}
\begin{itemize}
\item {Grp. gram.:f.}
\end{itemize}
Espécie de mariposa africana.
\section{Macunis}
\begin{itemize}
\item {Grp. gram.:m. pl.}
\end{itemize}
Antiga tríbo de Minas-Geraes, no Brasil.
\section{Macurá}
\begin{itemize}
\item {Grp. gram.:m.}
\end{itemize}
Dialecto da Guiana inglesa.
\section{Macuripai}
\begin{itemize}
\item {Grp. gram.:m.}
\end{itemize}
Fruto silvestre do Brasil.
\section{Macuru}
\begin{itemize}
\item {Grp. gram.:m.}
\end{itemize}
\begin{itemize}
\item {Utilização:Bras. do N}
\end{itemize}
Baloiço, formado de duas talas, onde se collocam as crianças, que, com o movimento natural das pernas pendentes, fazem baloiçar o apparelho sem perigo de quéda.
(Talvez do tupi)
\section{Macus}
\begin{itemize}
\item {Grp. gram.:m. pl.}
\end{itemize}
Índios selvagens das margens do Apaporis, no Brasil.
\section{Macuta}
\begin{itemize}
\item {Grp. gram.:f.}
\end{itemize}
Moéda de cobre africana, do valor aproximado de 50 reis fracos, ou 30 reis da nossa moéda.
Moéda de prata, de differentes valores, usadas dantes em Angola.
\section{Macuxis}
\begin{itemize}
\item {Grp. gram.:m.}
\end{itemize}
Índios da Guiana brasileira.
\section{Madagascarense}
\begin{itemize}
\item {Grp. gram.:adj.}
\end{itemize}
\begin{itemize}
\item {Grp. gram.:M.}
\end{itemize}
Relativo a Madagáscar.
Habitante de Madagáscar.
\section{Madama}
\begin{itemize}
\item {Grp. gram.:f.}
\end{itemize}
\begin{itemize}
\item {Utilização:Pop.}
\end{itemize}
\begin{itemize}
\item {Proveniência:(Fr. \textunderscore madame\textunderscore )}
\end{itemize}
Senhora:«\textunderscore dez annos consumiu a tal madama.\textunderscore »A. Dinis, \textunderscore Hyssope\textunderscore . Cf. Camillo, \textunderscore Caveira\textunderscore , 459.
Esposa:«\textunderscore o alfaiate foi com a sua madama ao theatro.\textunderscore »
Montículo, espécie de marco de terra, que se deixa em meio de uma escavação, para depois se conhecer a profundidade desta, e que é também designado por \textunderscore dama\textunderscore  e \textunderscore testemunha\textunderscore .
\section{Madameco}
\begin{itemize}
\item {Grp. gram.:m.}
\end{itemize}
\begin{itemize}
\item {Utilização:Ant.}
\end{itemize}
Homem insignificante; bigorrilhas.
O mesmo que \textunderscore badameco\textunderscore .
\section{Madamismo}
\begin{itemize}
\item {Grp. gram.:m.}
\end{itemize}
\begin{itemize}
\item {Utilização:Fam.}
\end{itemize}
\begin{itemize}
\item {Proveniência:(De \textunderscore madama\textunderscore )}
\end{itemize}
Multidão de senhoras; as senhoras.
\section{Madan}
\begin{itemize}
\item {Grp. gram.:m.}
\end{itemize}
Espécie de altar provisório nos templos indianos.
\section{Madanacás}
\begin{itemize}
\item {Grp. gram.:m. pl.}
\end{itemize}
Indígenas brasileiros da região do Amazonas.
\section{Madapolám}
\begin{itemize}
\item {Grp. gram.:m.}
\end{itemize}
\begin{itemize}
\item {Proveniência:(De \textunderscore Madopolám\textunderscore , n. p.)}
\end{itemize}
Tecido branco e consistente, de lan.
\section{Madarose}
\begin{itemize}
\item {Grp. gram.:f.}
\end{itemize}
\begin{itemize}
\item {Proveniência:(Gr. \textunderscore madarosis\textunderscore )}
\end{itemize}
Doença, que faz caír os cabellos, especialmente os cílios.
\section{Madefacção}
\begin{itemize}
\item {Grp. gram.:f.}
\end{itemize}
\begin{itemize}
\item {Proveniência:(Lat. \textunderscore madefactio\textunderscore )}
\end{itemize}
Acto de madeficar.
\section{Madefacto}
\begin{itemize}
\item {Grp. gram.:adj.}
\end{itemize}
\begin{itemize}
\item {Proveniência:(Lat. \textunderscore madefactus\textunderscore )}
\end{itemize}
Que se tornou húmido.
Molhado.
\section{Madeficar}
\begin{itemize}
\item {Grp. gram.:v. t.}
\end{itemize}
\begin{itemize}
\item {Proveniência:(Do lat. \textunderscore madefacere\textunderscore )}
\end{itemize}
Banhar, tornar húmido.
Amollecer (uma substância), na preparação de um medicamento.
\section{Madeira}
\begin{itemize}
\item {Grp. gram.:f.}
\end{itemize}
\begin{itemize}
\item {Proveniência:(Do lat. \textunderscore materia\textunderscore )}
\end{itemize}
Parte lenhosa do tronco, da raíz e dos ramos das plantas.
Corpo lenhoso.
Tábuas, vigas, ripas, caibros, tudo que, sendo lenhoso, se applica a construcções e trabalhos de marcenaria, carpintaria, etc.
\section{Madeira}
\begin{itemize}
\item {Grp. gram.:m.}
\end{itemize}
Vinho generoso da Ilha da Madeira.
\section{Madeirada}
\begin{itemize}
\item {Grp. gram.:f.}
\end{itemize}
Grande porção de madeira.
\section{Madeiral}
\begin{itemize}
\item {Grp. gram.:m.}
\end{itemize}
\begin{itemize}
\item {Utilização:Pop.}
\end{itemize}
Arvoredo, de que se extrahem madeiras.
\section{Madeiramento}
\begin{itemize}
\item {Grp. gram.:m.}
\end{itemize}
\begin{itemize}
\item {Proveniência:(De \textunderscore madeirar\textunderscore )}
\end{itemize}
Porção de madeira.
Madeira, que constitue a armação de uma casa.
\section{Madeirar}
\begin{itemize}
\item {Grp. gram.:v. t.}
\end{itemize}
\begin{itemize}
\item {Grp. gram.:V. i.}
\end{itemize}
Pôr armação de madeira em.
Trabalhar em madeira.
\section{Madeireiro}
\begin{itemize}
\item {Grp. gram.:m.}
\end{itemize}
\begin{itemize}
\item {Utilização:Bras}
\end{itemize}
Negociante de madeiras.
\section{Madeirense}
\begin{itemize}
\item {Grp. gram.:adj.}
\end{itemize}
\begin{itemize}
\item {Grp. gram.:M.}
\end{itemize}
Relativo á ilha da Madeira.
Habitante da Madeira.
\section{Madeiro}
\begin{itemize}
\item {Grp. gram.:m.}
\end{itemize}
\begin{itemize}
\item {Utilização:Pleb.}
\end{itemize}
Peça ou tronco grosso de madeira; trave.
Homem estúpido.
(Cp. \textunderscore madeira\textunderscore ^1)
\section{Madeixa}
\begin{itemize}
\item {Grp. gram.:f.}
\end{itemize}
\begin{itemize}
\item {Utilização:Fig.}
\end{itemize}
Pequena meada.
Porção de fios de seda ou lan, etc.
Negalho.
Porção de cabellos da cabeça; trança.
Marrafa.
(Cast. \textunderscore madeja\textunderscore , do lat. \textunderscore mataxa\textunderscore )
\section{Madepueira}
\begin{itemize}
\item {Grp. gram.:f.}
\end{itemize}
Planta melastomácea do Brasil.
\section{Madgyar}
\begin{itemize}
\item {Grp. gram.:m.  e  adj.}
\end{itemize}
(V.magiar)
\section{Mádi}
\begin{itemize}
\item {Grp. gram.:m.}
\end{itemize}
\begin{itemize}
\item {Proveniência:(De \textunderscore Mádi\textunderscore , n. p.)}
\end{itemize}
Nome de duas plantas synanthéreas.
\section{Mádia}
\begin{itemize}
\item {Grp. gram.:f.}
\end{itemize}
Planta chilena, o mesmo que \textunderscore mádi\textunderscore .
\section{Madiaico}
\begin{itemize}
\item {Grp. gram.:adj.}
\end{itemize}
Diz-se de um ácido, extrahido do óleo de mádi.
\section{Madianita}
\begin{itemize}
\item {Grp. gram.:adj.}
\end{itemize}
\begin{itemize}
\item {Grp. gram.:M.}
\end{itemize}
Relativo a Madian, na Ásia occidental.
Habitante de Madian.
\section{Mádido}
\begin{itemize}
\item {Grp. gram.:adj.}
\end{itemize}
\begin{itemize}
\item {Proveniência:(Lat. \textunderscore madidus\textunderscore )}
\end{itemize}
Humedecido; orvalhado; lento; encharcado.
\section{Madinatura}
\begin{itemize}
\item {Grp. gram.:f.}
\end{itemize}
\begin{itemize}
\item {Utilização:Ant.}
\end{itemize}
Traço ou desenho de um edifício.
\section{Màdinho}
\begin{itemize}
\item {Grp. gram.:m.}
\end{itemize}
\begin{itemize}
\item {Utilização:Prov.}
\end{itemize}
\begin{itemize}
\item {Utilização:dur.}
\end{itemize}
Estadulho, em que se apoia levantado o cabeçalho do carro, para êste se carregar de lenha ou de outros objectos.
\section{Madorna}
\begin{itemize}
\item {Grp. gram.:f.}
\end{itemize}
(V.madorra)
\section{Madorneira}
\begin{itemize}
\item {Grp. gram.:f.}
\end{itemize}
\begin{itemize}
\item {Utilização:T. de Aveiro}
\end{itemize}
Planta agreste, muito resistente.
\section{Madornice}
\begin{itemize}
\item {Grp. gram.:f.}
\end{itemize}
O mesmo que \textunderscore madorra\textunderscore .
\section{Madorra}
\begin{itemize}
\item {Grp. gram.:f.}
\end{itemize}
(Corr. de \textunderscore modorra\textunderscore )
\section{Madorrento}
\begin{itemize}
\item {Grp. gram.:adj.}
\end{itemize}
(V.modorrento)
\section{Madraçal}
\begin{itemize}
\item {Grp. gram.:m.}
\end{itemize}
\begin{itemize}
\item {Utilização:Ant.}
\end{itemize}
\begin{itemize}
\item {Proveniência:(Do ár. \textunderscore madaraça\textunderscore )}
\end{itemize}
Escola de primeiras letras.
Casa de aposentadoria. Cf. Góes, \textunderscore Chrón. de D. Man.\textunderscore , p. IV, c. 79, 58.
\section{Madraçaria}
\begin{itemize}
\item {Grp. gram.:f.}
\end{itemize}
Vida de madraço; ociosidade.
\section{Madraceador}
\begin{itemize}
\item {Grp. gram.:m.  e  adj.}
\end{itemize}
\begin{itemize}
\item {Proveniência:(De \textunderscore madracear\textunderscore )}
\end{itemize}
O que madraceia.
\section{Madracear}
\begin{itemize}
\item {Grp. gram.:v. i.}
\end{itemize}
Têr vida de madraço; vadiar.
\section{Madraceirão}
\begin{itemize}
\item {Grp. gram.:m.  e  adj.}
\end{itemize}
\begin{itemize}
\item {Proveniência:(De \textunderscore madraceiro\textunderscore )}
\end{itemize}
Grande madraço.
\section{Madraceirar}
\begin{itemize}
\item {Grp. gram.:v. i.}
\end{itemize}
\begin{itemize}
\item {Proveniência:(De \textunderscore madraceiro\textunderscore )}
\end{itemize}
O mesmo que \textunderscore madracear\textunderscore . Cf. Filinto, VII, 236.
\section{Madraceiro}
\begin{itemize}
\item {Grp. gram.:adj.}
\end{itemize}
\begin{itemize}
\item {Grp. gram.:M.}
\end{itemize}
Que madraceia.
O mesmo que \textunderscore madraço\textunderscore .
\section{Madracice}
\begin{itemize}
\item {Grp. gram.:f.}
\end{itemize}
Qualidade de quem é madraço; madraçaria.
\section{Madraço}
\begin{itemize}
\item {Grp. gram.:m.  e  adj.}
\end{itemize}
O mesmo que \textunderscore mandrião\textunderscore .
(Contr. de \textunderscore malandraço\textunderscore , de \textunderscore malandro\textunderscore ?)
\section{Madrafan}
\begin{itemize}
\item {Grp. gram.:f.}
\end{itemize}
\begin{itemize}
\item {Proveniência:(T. as.)}
\end{itemize}
Antiga moéda de Cambaia.
\section{Madrafaxão}
\begin{itemize}
\item {Grp. gram.:m.}
\end{itemize}
Antiga moéda de Gôa, equivalente a 1$440 reis.
\section{Madragôa}
\begin{itemize}
\item {Grp. gram.:f.}
\end{itemize}
\begin{itemize}
\item {Utilização:Prov.}
\end{itemize}
(Corr. de \textunderscore madrigôa\textunderscore )
\section{Madragôa}
\begin{itemize}
\item {Grp. gram.:f.}
\end{itemize}
\begin{itemize}
\item {Utilização:T. de Turquel}
\end{itemize}
Mulher ordinária, desalinhada ou extravagante.
\section{Madrasta}
\begin{itemize}
\item {Grp. gram.:f.}
\end{itemize}
\begin{itemize}
\item {Utilização:Fig.}
\end{itemize}
\begin{itemize}
\item {Proveniência:(Do b. lat. \textunderscore matrasta\textunderscore )}
\end{itemize}
Designação da mulher casada, em relação aos filhos que seu marido teve de núpcias anteriores.
Mãe pouco carinhosa; mulher descaroável.
\section{Madre}
\begin{itemize}
\item {Grp. gram.:f.}
\end{itemize}
\begin{itemize}
\item {Utilização:Des.}
\end{itemize}
\begin{itemize}
\item {Utilização:Mús.}
\end{itemize}
\begin{itemize}
\item {Proveniência:(Lat. \textunderscore mater\textunderscore )}
\end{itemize}
Freira.
A regente de um recolhimento.
Útero.
Viga horizontal, em que se assentam barrotes.
Nome de várias peças de um navio.
A parte mais grossa do vinho ou do vinagre, que assenta no fundo das vasilhas.
Primeira parte do nome de algumas plantas.
Mãe.
Fio principal dos bordões, em volta do qual se enrola a fieira.
\section{Madre-cravo}
\begin{itemize}
\item {Grp. gram.:f.}
\end{itemize}
Planta do Brasil, da fam. das compostas.
\section{Madrepérola}
\begin{itemize}
\item {Grp. gram.:f.}
\end{itemize}
\begin{itemize}
\item {Proveniência:(De \textunderscore madre\textunderscore  + \textunderscore pérola\textunderscore )}
\end{itemize}
Mollusco acéphalo lamellibrânchio, (\textunderscore avicula margarita\textunderscore ).
Parte nacarada da concha dêste mollusco.
\section{Madalena}
\begin{itemize}
\item {Grp. gram.:f.}
\end{itemize}
\begin{itemize}
\item {Utilização:Gír.}
\end{itemize}
\begin{itemize}
\item {Proveniência:(De \textunderscore Magdalena\textunderscore , n. p.)}
\end{itemize}
Mulher chorosa e arrependida dos seus erros.
\section{Madaleneta}
\begin{itemize}
\item {fónica:nê}
\end{itemize}
\begin{itemize}
\item {Grp. gram.:f.}
\end{itemize}
\begin{itemize}
\item {Proveniência:(De \textunderscore Magdalena\textunderscore , n. p.)}
\end{itemize}
Mulher, que, tendo vivido desregradamente, se recolhia a um dos conventos que, para gente dessa Classe, havia em Paris, Nápoles, etc.
\section{Madrefilas}
\begin{itemize}
\item {Grp. gram.:f. pl.}
\end{itemize}
\begin{itemize}
\item {Proveniência:(Do lat. \textunderscore mater\textunderscore  + gr. \textunderscore phullon\textunderscore )}
\end{itemize}
Família de madréporas, cujas células são guarnecidas de lâminas.
\section{Madreperolado}
\begin{itemize}
\item {Grp. gram.:adj.}
\end{itemize}
Que tem aspecto de madrepérola.
Revestido de madrepérola.
\section{Madrephyllas}
\begin{itemize}
\item {Grp. gram.:f. pl.}
\end{itemize}
\begin{itemize}
\item {Proveniência:(Do lat. \textunderscore mater\textunderscore  + gr. \textunderscore phullon\textunderscore )}
\end{itemize}
Família de madréporas, cujas céllulas são guarnecidas de lâminas.
\section{Madrépora}
\begin{itemize}
\item {Grp. gram.:f.}
\end{itemize}
\begin{itemize}
\item {Proveniência:(It. \textunderscore madrepora\textunderscore )}
\end{itemize}
Família de polypeiros pedregosos e porosos, (\textunderscore madreporus\textunderscore ).
\section{Madreporários}
\begin{itemize}
\item {Grp. gram.:m. pl.}
\end{itemize}
\begin{itemize}
\item {Proveniência:(De \textunderscore madrépora\textunderscore )}
\end{itemize}
Coraliários de esqueleto calcário.
\section{Madrepórico}
\begin{itemize}
\item {Grp. gram.:adj.}
\end{itemize}
Relativo a madrépora.
\section{Madreporífero}
\begin{itemize}
\item {Grp. gram.:adj.}
\end{itemize}
\begin{itemize}
\item {Proveniência:(De \textunderscore madrépora\textunderscore  + lat. \textunderscore ferre\textunderscore )}
\end{itemize}
Que contém ou produz madréporas.
\section{Madreporiforme}
\begin{itemize}
\item {Grp. gram.:adj.}
\end{itemize}
\begin{itemize}
\item {Proveniência:(De \textunderscore madrépora\textunderscore  + \textunderscore fórma\textunderscore )}
\end{itemize}
Que tem fórma ou aspecto de madrépora.
\section{Madreporita}
\begin{itemize}
\item {Grp. gram.:f.}
\end{itemize}
\begin{itemize}
\item {Proveniência:(De \textunderscore madrépora\textunderscore )}
\end{itemize}
Madrépora fóssil.
\section{Madresilva}
\begin{itemize}
\item {fónica:sil}
\end{itemize}
\begin{itemize}
\item {Grp. gram.:f.}
\end{itemize}
\begin{itemize}
\item {Proveniência:(De \textunderscore madre\textunderscore  + \textunderscore silva\textunderscore )}
\end{itemize}
Gênero de plantas, que contém espécies aromáticas, de ramos flexiveis e compridos, e que serve de typo ás caprifoliáceas.
\section{Madressilva}
\begin{itemize}
\item {Grp. gram.:f.}
\end{itemize}
\begin{itemize}
\item {Proveniência:(De \textunderscore madre\textunderscore  + \textunderscore silva\textunderscore )}
\end{itemize}
Gênero de plantas, que contém espécies aromáticas, de ramos flexiveis e compridos, e que serve de typo ás caprifoliáceas.
\section{Madria}
\begin{itemize}
\item {Grp. gram.:f.}
\end{itemize}
\begin{itemize}
\item {Utilização:Prov.}
\end{itemize}
Encapellamento das ondas; ondas que formam carneirada.
Rêgo de água; açude.
(Por \textunderscore mandria\textunderscore , do lat. \textunderscore mandra\textunderscore , curral)
\section{Madrigal}
\begin{itemize}
\item {Grp. gram.:m.}
\end{itemize}
\begin{itemize}
\item {Utilização:Ext.}
\end{itemize}
\begin{itemize}
\item {Proveniência:(Do lat. \textunderscore matriale\textunderscore )}
\end{itemize}
Antiga composição musical, para vozes, sem acompanhamento.
Pequena composição poética, engenhosa e galante.
Poesia pastoril.
Galanteio; lisonja dirigida a damas.
\section{Madrigalesco}
\begin{itemize}
\item {Grp. gram.:adj.}
\end{itemize}
\begin{itemize}
\item {Grp. gram.:M.}
\end{itemize}
Relativo a madrigal ou ao gênero do madrigal.
Carácter de madrigal; idýllio amoroso.
\section{Madrigalete}
\begin{itemize}
\item {fónica:lê}
\end{itemize}
\begin{itemize}
\item {Grp. gram.:m.}
\end{itemize}
(dem. de \textunderscore madrigal\textunderscore )
\section{Madrigálico}
\begin{itemize}
\item {Grp. gram.:adj.}
\end{itemize}
O mesmo que \textunderscore madrigalesco\textunderscore .
\section{Madrigalista}
\begin{itemize}
\item {Grp. gram.:m.}
\end{itemize}
\begin{itemize}
\item {Proveniência:(De \textunderscore madrigal\textunderscore )}
\end{itemize}
Aquelle que faz madrigaes.
\section{Madrigalizar}
\begin{itemize}
\item {Grp. gram.:v. i.}
\end{itemize}
\begin{itemize}
\item {Utilização:Neol.}
\end{itemize}
\begin{itemize}
\item {Proveniência:(De \textunderscore madrigal\textunderscore )}
\end{itemize}
Fazer madrigaes.
\section{Madrigaz}
\begin{itemize}
\item {Grp. gram.:m.}
\end{itemize}
Homem magro, escaveirado, feio.
(Por \textunderscore magridaz\textunderscore , do rad. de \textunderscore magro\textunderscore )
\section{Madrigôa}
\begin{itemize}
\item {Grp. gram.:f.}
\end{itemize}
\begin{itemize}
\item {Utilização:Prov.}
\end{itemize}
O mesmo que \textunderscore madrigueira\textunderscore .
\section{Madrigueira}
\begin{itemize}
\item {Grp. gram.:f.}
\end{itemize}
\begin{itemize}
\item {Utilização:Fig.}
\end{itemize}
Toca, lura.
Latíbulo.
Esconderijo ou residência de má nota.
(Cast. \textunderscore madriguera\textunderscore )
\section{Madrija}
\begin{itemize}
\item {Grp. gram.:f.}
\end{itemize}
\begin{itemize}
\item {Utilização:Bras}
\end{itemize}
\begin{itemize}
\item {Proveniência:(Do cast. \textunderscore madre\textunderscore )}
\end{itemize}
Nome, que na Baía se dá á baleia mãe.
\section{Madrijo}
\begin{itemize}
\item {Grp. gram.:m.}
\end{itemize}
\begin{itemize}
\item {Utilização:Bras}
\end{itemize}
Baleia fêmea; madrija. Cf. \textunderscore Jorn. do Comm.\textunderscore , do Rio, de 29-VI-900.
\section{Madrileno}
\begin{itemize}
\item {Grp. gram.:adj.}
\end{itemize}
\begin{itemize}
\item {Grp. gram.:M.}
\end{itemize}
Relativo a Madrid.
Habitante de Madrid.
\section{Madrilense}
\begin{itemize}
\item {Grp. gram.:m.  e  adj.}
\end{itemize}
O mesmo que \textunderscore madrilês\textunderscore .
\section{Madrilês}
\begin{itemize}
\item {Grp. gram.:m.  e  adj.}
\end{itemize}
O mesmo que \textunderscore madrileno\textunderscore .
\section{Madrilheira}
\begin{itemize}
\item {Grp. gram.:f.}
\end{itemize}
O mesmo que \textunderscore madrigueira\textunderscore .
Apparelho, para a pesca de peixe miúdo.
(Cast. \textunderscore madrillera\textunderscore )
\section{Madrinha}
\begin{itemize}
\item {Grp. gram.:f.}
\end{itemize}
\begin{itemize}
\item {Utilização:Bras}
\end{itemize}
Mulher, que serve de testemunha nos baptizados, chrismas e casamentos, e que toma aquella designação, em relação ao neóphyto, á noiva e á pessôa chrismada.
Protectora.
Égua, que serve de guia a uma manada de gado muar.
(B. lat. \textunderscore matrina\textunderscore , do lat. \textunderscore mater\textunderscore )
\section{Madriz}
\begin{itemize}
\item {Grp. gram.:f.}
\end{itemize}
Caminho sôbre a maracha das salinas, do lado opposto á pervinca, e pelo qual se leva o sal para a eira.
(Cp. \textunderscore matriz\textunderscore )
\section{Madronho}
\textunderscore m.\textunderscore  (e der.)
O mesmo que \textunderscore medronho\textunderscore , etc. Cf. B. Pereira, \textunderscore Prosódia\textunderscore .
\section{Madrugada}
\begin{itemize}
\item {Grp. gram.:f.}
\end{itemize}
\begin{itemize}
\item {Utilização:Fig.}
\end{itemize}
Acto de madrugar.
Aurora; alvorada.
Precocidade.
\section{Madrugador}
\begin{itemize}
\item {Grp. gram.:adj.}
\end{itemize}
\begin{itemize}
\item {Grp. gram.:M.}
\end{itemize}
\begin{itemize}
\item {Utilização:fig.}
\end{itemize}
\begin{itemize}
\item {Utilização:Ext.}
\end{itemize}
\begin{itemize}
\item {Proveniência:(De \textunderscore madrugar\textunderscore )}
\end{itemize}
Que madruga.
Aquelle que madruga.
Quem é diligente ou antecede outrem em qualquer acção.
\section{Madrugar}
\begin{itemize}
\item {Grp. gram.:v. i.}
\end{itemize}
\begin{itemize}
\item {Proveniência:(Do rad. do lat. \textunderscore maturare\textunderscore )}
\end{itemize}
Levantar-se cedo da cama.
Praticar algum acto antes do tempo próprio.
Anteceder outrem em qualquer coisa.
Manifestar-se muito cedo: \textunderscore madrugaram nelle os vícios\textunderscore .
\section{Maduração}
\begin{itemize}
\item {Grp. gram.:f.}
\end{itemize}
\begin{itemize}
\item {Proveniência:(Lat. \textunderscore maturatío\textunderscore )}
\end{itemize}
Acto ou effeito de amadurecer; maturação.
\section{Madurador}
\begin{itemize}
\item {Grp. gram.:adj.}
\end{itemize}
\begin{itemize}
\item {Proveniência:(De \textunderscore madurar\textunderscore )}
\end{itemize}
Que faz amadurecer.
\section{Madural}
\begin{itemize}
\item {Grp. gram.:adj.}
\end{itemize}
\begin{itemize}
\item {Utilização:Prov.}
\end{itemize}
\begin{itemize}
\item {Utilização:trasm.}
\end{itemize}
\begin{itemize}
\item {Proveniência:(De \textunderscore maduro\textunderscore )}
\end{itemize}
Diz-se de uma variedade de azeitona, também conhecida por \textunderscore negral\textunderscore .
\section{Maduramente}
\begin{itemize}
\item {Grp. gram.:adv.}
\end{itemize}
\begin{itemize}
\item {Proveniência:(De \textunderscore maduro\textunderscore )}
\end{itemize}
Com madureza, com prudência: \textunderscore reflectir maduramente\textunderscore .
\section{Madurar}
\begin{itemize}
\item {Grp. gram.:v. t.}
\end{itemize}
\begin{itemize}
\item {Grp. gram.:V. i.}
\end{itemize}
\begin{itemize}
\item {Utilização:Fig.}
\end{itemize}
\begin{itemize}
\item {Proveniência:(Do lat. \textunderscore maturare\textunderscore )}
\end{itemize}
Tornar maduro; sazonar.
Amadurecer.
Adquirir madureza, prudência, juízo.
\section{Madurecer}
\begin{itemize}
\item {Grp. gram.:v. t.  e  i.}
\end{itemize}
O mesmo que \textunderscore amadurecer\textunderscore .
\section{Madureiro}
\begin{itemize}
\item {Grp. gram.:m.}
\end{itemize}
\begin{itemize}
\item {Proveniência:(De \textunderscore maduro\textunderscore )}
\end{itemize}
Lugar, em que se guardam e acabam de amadurecer as frutas.
\section{Madurez}
\begin{itemize}
\item {Grp. gram.:f.}
\end{itemize}
O mesmo que \textunderscore madureza\textunderscore . Cf. Filinto, III, 212.
\section{Madureza}
\begin{itemize}
\item {Grp. gram.:f.}
\end{itemize}
\begin{itemize}
\item {Utilização:Fig.}
\end{itemize}
\begin{itemize}
\item {Utilização:Gír.}
\end{itemize}
Effeito de madurar.
Qualidade do que é maduro.
Prudência; gravidade.
Mania; patetice.
\textunderscore Exame de madureza\textunderscore , exame final, simultâneo e summário, de disciplinas de um curso secundário, que algumas vezes se tem usado, como preparação para cursos superiores.
\section{Maduridade}
\begin{itemize}
\item {Grp. gram.:f.}
\end{itemize}
O mesmo que \textunderscore madureza\textunderscore . Cf. \textunderscore Techn. Rur.\textunderscore , 77.
\section{Maduridão}
\begin{itemize}
\item {Grp. gram.:f.}
\end{itemize}
\begin{itemize}
\item {Utilização:Ant.}
\end{itemize}
O mesmo que \textunderscore madureza\textunderscore .
\section{Maduro}
\begin{itemize}
\item {Grp. gram.:adj.}
\end{itemize}
\begin{itemize}
\item {Utilização:Fig.}
\end{itemize}
\begin{itemize}
\item {Utilização:Gír.}
\end{itemize}
\begin{itemize}
\item {Proveniência:(Do lat. \textunderscore maturus\textunderscore )}
\end{itemize}
Amadurecido; perfeito: \textunderscore fruta madura\textunderscore .
Prudente; circunspecto.
Que já não é novo: \textunderscore pessôa madura\textunderscore .
Pateta; idiota.
\section{Maduro}
\begin{itemize}
\item {Grp. gram.:m.}
\end{itemize}
\begin{itemize}
\item {Utilização:Bras. do Rio}
\end{itemize}
Bebida fermentada, feita com mel e água.
(Alter. de \textunderscore maluvo\textunderscore )
\section{Mãe}
\begin{itemize}
\item {Grp. gram.:f.}
\end{itemize}
\begin{itemize}
\item {Utilização:Fig.}
\end{itemize}
\begin{itemize}
\item {Utilização:Fig.}
\end{itemize}
\begin{itemize}
\item {Utilização:Gír.}
\end{itemize}
\begin{itemize}
\item {Proveniência:(Do lat. \textunderscore mater\textunderscore )}
\end{itemize}
Mulher, que deu á luz um ou mais filhos.
Relação de parentesco de uma mulher para com seus filhos.
Qualquer fêmea, que teve filho ou filhos.
Madre, bôrra do vinho.
Fonte, origem.
Mulher caridosa e desvelada.
Fechadura.
\section{Mãe-benta}
\begin{itemize}
\item {Grp. gram.:f.}
\end{itemize}
\begin{itemize}
\item {Utilização:Bras}
\end{itemize}
Espécie de bolo pequeno.
\section{Mãe-bôa}
\begin{itemize}
\item {Grp. gram.:f.}
\end{itemize}
\begin{itemize}
\item {Utilização:Bras}
\end{itemize}
Planta medicinal, contra o reumatismo e o beribéri.
\section{Mãe-d'água}
\begin{itemize}
\item {Grp. gram.:f.}
\end{itemize}
\begin{itemize}
\item {Utilização:Bras. do N}
\end{itemize}
Nascente de água.
Reservatório, donde sái água, para casos extraordinários.
Entidade lendária, habitante de rios, lagos e fontes.
\section{Mãe-da-lua}
\begin{itemize}
\item {Grp. gram.:f.}
\end{itemize}
\begin{itemize}
\item {Utilização:Bras. do N}
\end{itemize}
Ave noctívaga e agoireira, cujo canto parece gargalhada.
\section{Mãe-de-balcão}
\begin{itemize}
\item {Grp. gram.:f.}
\end{itemize}
\begin{itemize}
\item {Utilização:Bras}
\end{itemize}
Mulher, que aparta as várias qualidades de açúcar, nos respectivos engenhos.
\section{Mãe-de-família}
\begin{itemize}
\item {Grp. gram.:f.}
\end{itemize}
Mulher casada, que tem filhos.
\section{Mãe-de-famílias}
\begin{itemize}
\item {Grp. gram.:f.}
\end{itemize}
O mesmo que \textunderscore mãe-de-família\textunderscore . Cf. Bernárdez, \textunderscore N. Floresta\textunderscore .
\section{Maéstre}
\begin{itemize}
\item {Grp. gram.:m.}
\end{itemize}
\begin{itemize}
\item {Utilização:Ant.}
\end{itemize}
O mesmo que \textunderscore mestre\textunderscore .
\section{Maestria}
\begin{itemize}
\item {fónica:ma-és}
\end{itemize}
\begin{itemize}
\item {Grp. gram.:f.}
\end{itemize}
\begin{itemize}
\item {Utilização:Ant.}
\end{itemize}
\begin{itemize}
\item {Proveniência:(De \textunderscore maéstre\textunderscore )}
\end{itemize}
Mestria, perícia:«\textunderscore ...com maestria de um Talma\textunderscore ». Camillo, \textunderscore Homem de Brios\textunderscore , 22.
Arte de trovar.
\section{Maestrino}
\begin{itemize}
\item {fónica:ma-e}
\end{itemize}
\begin{itemize}
\item {Grp. gram.:m.}
\end{itemize}
\begin{itemize}
\item {Proveniência:(It. \textunderscore maestrino\textunderscore )}
\end{itemize}
Compositor de música ligeira.
\section{Maéstro}
\begin{itemize}
\item {Grp. gram.:m.}
\end{itemize}
\begin{itemize}
\item {Proveniência:(It. \textunderscore maestro\textunderscore )}
\end{itemize}
Compositor de música.
Aquelle que rege uma orchestra importante.
\section{Maêta}
\begin{itemize}
\item {Grp. gram.:f.}
\end{itemize}
\begin{itemize}
\item {Utilização:Ant.}
\end{itemize}
Coxim; colchão.
\section{Mãezeiro}
\begin{itemize}
\item {Grp. gram.:adj.}
\end{itemize}
\begin{itemize}
\item {Utilização:Prov.}
\end{itemize}
\begin{itemize}
\item {Utilização:trasm.}
\end{itemize}
Que é muito amigo de sua mãe.
\section{Mafahu}
\begin{itemize}
\item {Grp. gram.:m.}
\end{itemize}
Espécie de cajueiro.
\section{Mafamede}
\begin{itemize}
\item {Grp. gram.:m.}
\end{itemize}
\begin{itemize}
\item {Utilização:ant.}
\end{itemize}
\begin{itemize}
\item {Utilização:Pop.}
\end{itemize}
\begin{itemize}
\item {Utilização:Des.}
\end{itemize}
\begin{itemize}
\item {Proveniência:(De \textunderscore Mafamede\textunderscore , n. p.)}
\end{itemize}
O mesmo que \textunderscore moiro\textunderscore . Cf. Sim. Mach., f. 37 v.^o
O mesmo que \textunderscore contador\textunderscore , móvel?:«\textunderscore lá num retrete avisto um mafamede de miúdas gavetinhas.\textunderscore »Filinto, VI, 37.«\textunderscore Levo um mafamede com duas fechaduras.\textunderscore »(De um testamento de 1692)
\section{Mafamético}
\begin{itemize}
\item {Grp. gram.:adj.}
\end{itemize}
Relativo a Mafamede ou a Mafoma.
\section{Mafarrico}
\begin{itemize}
\item {Grp. gram.:m.}
\end{itemize}
\begin{itemize}
\item {Utilização:Pop.}
\end{itemize}
\begin{itemize}
\item {Utilização:Deprec.}
\end{itemize}
\begin{itemize}
\item {Utilização:T. da Bairrada}
\end{itemize}
O diabo.
Habitante de Mafra.
Espécie de jôgo de cartas.
\section{Mafaú}
\begin{itemize}
\item {Grp. gram.:m.}
\end{itemize}
Espécie de cajueiro.
\section{Mafoma}
\begin{itemize}
\item {Grp. gram.:f.}
\end{itemize}
\begin{itemize}
\item {Utilização:Açor}
\end{itemize}
\begin{itemize}
\item {Utilização:ant.}
\end{itemize}
\begin{itemize}
\item {Utilização:Pop.}
\end{itemize}
Esculptura grande e tôsca, que figura homem ou mulher.
Moiro. Cf. Sim. Mach., f. 24, bv.^o.
\section{Mafomista}
\begin{itemize}
\item {Grp. gram.:m.}
\end{itemize}
Sectário de Mafoma.
\section{Mafuca-macoge}
\begin{itemize}
\item {Grp. gram.:m.}
\end{itemize}
Arbusto africano, de fôlhas dispersas, e inflorescência em cachos pedunculados.
\section{Mafuma}
\begin{itemize}
\item {Grp. gram.:f.}
\end{itemize}
O mesmo que \textunderscore mafumeira\textunderscore .
\section{Mafumeira}
\begin{itemize}
\item {Grp. gram.:f.}
\end{itemize}
Árvore africana, (\textunderscore eríodendron anfractuosum\textunderscore , De-Candolle), de que os indígenas fazem dongos e pirogas.
\section{Mafumo}
\begin{itemize}
\item {Grp. gram.:m.}
\end{itemize}
Nome, que os indígenas africanos dão á mafumeira.
\section{Mafunji}
\begin{itemize}
\item {Grp. gram.:m.}
\end{itemize}
Planta trepadeira da ilha de San-Thomé.
\section{Mafureira}
\begin{itemize}
\item {Grp. gram.:f.}
\end{itemize}
\begin{itemize}
\item {Utilização:T. de Moçambique}
\end{itemize}
Árvore meliácea, de cujas sementes se extrái um óleo, com que os Negros temperam suas comidas.
\section{Maga}
\begin{itemize}
\item {Grp. gram.:f.}
\end{itemize}
\begin{itemize}
\item {Proveniência:(De \textunderscore mago\textunderscore )}
\end{itemize}
Mulher, que exerce a arte magica.
O mesmo que \textunderscore feiticeira\textunderscore .
\section{Maga}
\begin{itemize}
\item {Grp. gram.:f.}
\end{itemize}
\begin{itemize}
\item {Utilização:Pesc.}
\end{itemize}
Tripa de sardinha, que serve de isca.
\section{Magabeira}
\begin{itemize}
\item {Grp. gram.:f.}
\end{itemize}
Árvore fructífera do Brasil.
\section{Magaça}
\begin{itemize}
\item {Grp. gram.:f.}
\end{itemize}
Planta campestre, de flôres brancas, amarelas no centro.
O mesmo que \textunderscore magarça\textunderscore .
\section{Magacia}
\begin{itemize}
\item {Grp. gram.:f.}
\end{itemize}
\begin{itemize}
\item {Utilização:Ant.}
\end{itemize}
\begin{itemize}
\item {Proveniência:(De \textunderscore mago\textunderscore )}
\end{itemize}
Arte mágica; feitiçaria.
\section{Magal}
\begin{itemize}
\item {Grp. gram.:m.}
\end{itemize}
\begin{itemize}
\item {Utilização:Gír.}
\end{itemize}
Soldado.
\section{Magala}
\begin{itemize}
\item {Grp. gram.:m.}
\end{itemize}
\begin{itemize}
\item {Utilização:Gír.}
\end{itemize}
O mesmo ou melhor que \textunderscore magal\textunderscore .
\section{Magalânico}
\begin{itemize}
\item {Grp. gram.:adj.}
\end{itemize}
Relativo ao navegador Magalhães. Cf. Rui Barb., \textunderscore Réplica\textunderscore , 157.
\section{Magana}
\begin{itemize}
\item {Grp. gram.:f.}
\end{itemize}
\begin{itemize}
\item {Proveniência:(De \textunderscore magano\textunderscore ^1)}
\end{itemize}
Música antiga.
Mulher jovial, desenvolta.
\section{Maganagem}
\begin{itemize}
\item {Grp. gram.:f.}
\end{itemize}
Grupo de pessôas maganas.
Acto de magano.
\section{Maganagem}
\begin{itemize}
\item {Grp. gram.:f.}
\end{itemize}
\begin{itemize}
\item {Utilização:Prov.}
\end{itemize}
\begin{itemize}
\item {Proveniência:(De \textunderscore maganaz\textunderscore )}
\end{itemize}
Pus, que sái dos abscessos.
\section{Maganão}
\begin{itemize}
\item {Grp. gram.:m.  e  adj.}
\end{itemize}
O que pratíca muitas maganices ou que é muito magano.
\section{Maganaz}
\begin{itemize}
\item {Grp. gram.:m.}
\end{itemize}
\begin{itemize}
\item {Utilização:Prov.}
\end{itemize}
\begin{itemize}
\item {Utilização:alent.}
\end{itemize}
Furúnculo; tumor.
\section{Maganear}
\begin{itemize}
\item {Grp. gram.:v. i.}
\end{itemize}
\begin{itemize}
\item {Proveniência:(De \textunderscore magano\textunderscore ^1)}
\end{itemize}
Fazer ou dizer maganices.
\section{Maganeira}
\begin{itemize}
\item {Grp. gram.:f.}
\end{itemize}
O mesmo que \textunderscore maganice\textunderscore .
\section{Maganeiro}
\begin{itemize}
\item {Grp. gram.:m.}
\end{itemize}
O mesmo que \textunderscore maganeira\textunderscore :«\textunderscore tantos maganeiros eleitoraes\textunderscore ». Carlos de Laet.
\section{Maganento}
\begin{itemize}
\item {Grp. gram.:adj.}
\end{itemize}
\begin{itemize}
\item {Utilização:Des.}
\end{itemize}
\begin{itemize}
\item {Proveniência:(De \textunderscore magano\textunderscore ^1)}
\end{itemize}
Em que há maganice; que faz maganices.
\section{Maganice}
\begin{itemize}
\item {Grp. gram.:f.}
\end{itemize}
Dito ou acto de magano; jovialidade; brincadeira.
\section{Maganjas}
\begin{itemize}
\item {Grp. gram.:m. pl.}
\end{itemize}
Povos da bacia do Zambeze.
\section{Magano}
\begin{itemize}
\item {Grp. gram.:m.  e  adj.}
\end{itemize}
\begin{itemize}
\item {Utilização:Fig.}
\end{itemize}
\begin{itemize}
\item {Proveniência:(Do lat. \textunderscore mango\textunderscore , \textunderscore mangonis\textunderscore ?)}
\end{itemize}
Mariola; indivíduo de baixa extracção.
Negociante de escravos.
Negociante de animaes.
Indivíduo travêsso.
Ardiloso; jovial; desenvolto; engraçado.
\section{Magano}
\begin{itemize}
\item {Grp. gram.:m.}
\end{itemize}
\begin{itemize}
\item {Utilização:Gír.}
\end{itemize}
Relógio.
\section{Magarça}
\begin{itemize}
\item {Grp. gram.:f.}
\end{itemize}
O mesmo que \textunderscore magaça\textunderscore .
\section{Magareb}
\begin{itemize}
\item {Grp. gram.:m.}
\end{itemize}
Oração, que os Persas fazem a Deus, ao pôr do sol. Cf. Barros, \textunderscore Déc.\textunderscore  II, l. X, c. 6.
\section{Magarefe}
\begin{itemize}
\item {Grp. gram.:m.}
\end{itemize}
\begin{itemize}
\item {Utilização:pop.}
\end{itemize}
\begin{itemize}
\item {Utilização:Fig.}
\end{itemize}
\begin{itemize}
\item {Utilização:Prov.}
\end{itemize}
\begin{itemize}
\item {Utilização:minh.}
\end{itemize}
\begin{itemize}
\item {Utilização:beir.}
\end{itemize}
Aquelle que mata e esfola reses.
Mau cirurgião.
Maganão.
Tratante, biltre.
\section{Magarim}
\begin{itemize}
\item {Grp. gram.:m.}
\end{itemize}
Espécie de jasmim da Índia.
\section{Magaris}
\begin{itemize}
\item {Grp. gram.:m. pl.}
\end{itemize}
Tríbo extinta do Alto Amazonas.
\section{Magdaleano}
\begin{itemize}
\item {Grp. gram.:adj.}
\end{itemize}
\begin{itemize}
\item {Utilização:Geol.}
\end{itemize}
Diz-se do terreno, que constitue o último andar da série quaternária, segundo Mortillet.
\section{Magdaleão}
\begin{itemize}
\item {Grp. gram.:m.}
\end{itemize}
\begin{itemize}
\item {Proveniência:(Do gr. \textunderscore magdalia\textunderscore )}
\end{itemize}
Medicamento, enrolado cylindricamente.
\section{Magdalena}
\begin{itemize}
\item {Grp. gram.:f.}
\end{itemize}
\begin{itemize}
\item {Utilização:Gír.}
\end{itemize}
\begin{itemize}
\item {Proveniência:(De \textunderscore Magdalena\textunderscore , n. p.)}
\end{itemize}
Mulher chorosa e arrependida dos seus erros.
\section{Magdaleneta}
\begin{itemize}
\item {Grp. gram.:f.}
\end{itemize}
\begin{itemize}
\item {Proveniência:(De \textunderscore Magdalena\textunderscore , n. p.)}
\end{itemize}
Mulher, que, tendo vivido desregradamente, se recolhia a um dos conventos que, para gente dessa Classe, havia em Paris, Nápoles, etc.
\section{Má-geira}
\begin{itemize}
\item {Grp. gram.:f.}
\end{itemize}
\begin{itemize}
\item {Utilização:Prov.}
\end{itemize}
\begin{itemize}
\item {Utilização:beir.}
\end{itemize}
O mesmo que \textunderscore diabo\textunderscore : \textunderscore vai-te com a má-geira\textunderscore !
\section{Magengro}
\begin{itemize}
\item {Grp. gram.:m.}
\end{itemize}
\begin{itemize}
\item {Utilização:Prov.}
\end{itemize}
O mesmo que \textunderscore fradinho\textunderscore .
\section{Magestade}
\textunderscore f.\textunderscore  (e der.)
(V. \textunderscore majestade\textunderscore , etc.)
\section{Magia}
\begin{itemize}
\item {Grp. gram.:f.}
\end{itemize}
\begin{itemize}
\item {Proveniência:(Do gr. \textunderscore mageia\textunderscore )}
\end{itemize}
Religião dos magos.
Supposta arte de produzir effeitos contra a ordem natural.
Sensação ou sentimento, que se compara aos effeitos da magia.
Fascinação; encanto.
\section{Mágica}
\begin{itemize}
\item {Grp. gram.:f.}
\end{itemize}
\begin{itemize}
\item {Utilização:Fig.}
\end{itemize}
\begin{itemize}
\item {Proveniência:(Lat. \textunderscore magice\textunderscore )}
\end{itemize}
O mesmo que \textunderscore magia\textunderscore .
Peça de theatro, com transformações phantásticas.
Maga.
Planta, semelhante ao barbasco.
Encanto.
\section{Magicar}
\begin{itemize}
\item {Grp. gram.:v. i.}
\end{itemize}
\begin{itemize}
\item {Utilização:Fam.}
\end{itemize}
Scismar muito; andar aprehensivo.
(Cp. \textunderscore mágico\textunderscore )
\section{Mágico}
\begin{itemize}
\item {Grp. gram.:adj.}
\end{itemize}
\begin{itemize}
\item {Utilização:Fig.}
\end{itemize}
\begin{itemize}
\item {Grp. gram.:M.}
\end{itemize}
\begin{itemize}
\item {Utilização:Fig.}
\end{itemize}
\begin{itemize}
\item {Proveniência:(Lat. \textunderscore magicus\textunderscore )}
\end{itemize}
Relativo a magia.
Encantador.
Extraordinário: \textunderscore espectáculo mágico\textunderscore .
Mago; nigromante.
Hypócrita.
Indivíduo misanthropo ou scismático.
Tolo, lunático.
\section{Maginação}
\begin{itemize}
\item {Grp. gram.:f.}
\end{itemize}
\begin{itemize}
\item {Utilização:ant.}
\end{itemize}
\begin{itemize}
\item {Utilização:Pop.}
\end{itemize}
O mesmo que \textunderscore imaginação\textunderscore .
(Aphér. de \textunderscore imaginação\textunderscore )
\section{Magirioba}
\begin{itemize}
\item {Grp. gram.:f.}
\end{itemize}
O mesmo que \textunderscore majerioba\textunderscore .
\section{Magis}
\begin{itemize}
\item {Grp. gram.:m.}
\end{itemize}
Espécie de coiro, empregado em sapataria:«\textunderscore ...pellicas, magis e marroquins...\textunderscore »\textunderscore Inquér. Industr.\textunderscore , 2.^a parte, l. II, 231.
\section{Magismo}
\begin{itemize}
\item {Grp. gram.:m.}
\end{itemize}
\begin{itemize}
\item {Proveniência:(De \textunderscore mago\textunderscore )}
\end{itemize}
Prática da magia; systema dos magos.
\section{Magíster}
\begin{itemize}
\item {Grp. gram.:m.}
\end{itemize}
\begin{itemize}
\item {Utilização:Fam.}
\end{itemize}
\begin{itemize}
\item {Proveniência:(T. lat.)}
\end{itemize}
Mestre; padre-mestre.
Indivíduo sustencioso. Cf. J. Dinís, \textunderscore Morgadinha\textunderscore , 73 e 126.
\section{Magistério}
\begin{itemize}
\item {Grp. gram.:m.}
\end{itemize}
\begin{itemize}
\item {Proveniência:(Lat. \textunderscore magisterium\textunderscore )}
\end{itemize}
Cargo de professor.
Exercício do professorado.
Professorado; classe dos professores: \textunderscore o magistério protestou\textunderscore .
Composto mineral, muitas vezes preparado secretamente nas pharmácias, e a que se attribuíam virtudes superiores.
\section{Magistrado}
\begin{itemize}
\item {Grp. gram.:m.}
\end{itemize}
\begin{itemize}
\item {Utilização:Restrict.}
\end{itemize}
\begin{itemize}
\item {Proveniência:(Lat. \textunderscore magistratus\textunderscore )}
\end{itemize}
Empregado público, que, na esphera administrativa ou judicial, exerce autoridade delegada pela nação ou pelo poder central.
Epítheto dos juízes, delegados do ministério público, governadores civis e administradores de concelho.
\section{Magistral}
\begin{itemize}
\item {Grp. gram.:adj.}
\end{itemize}
\begin{itemize}
\item {Utilização:Fig.}
\end{itemize}
\begin{itemize}
\item {Utilização:Pharm.}
\end{itemize}
\begin{itemize}
\item {Grp. gram.:M.}
\end{itemize}
\begin{itemize}
\item {Proveniência:(Lat. \textunderscore magistralis\textunderscore )}
\end{itemize}
Relativo a mestre.
Perfeito; completo; exemplar: \textunderscore obra magistral\textunderscore .
Diz-se do medicamento, que se prepara na occasião em que é pedido, (em opposição a \textunderscore officinal\textunderscore , que já estava preparado)
Cónego, que tem o ónus do ensino.
\section{Magistralidade}
\begin{itemize}
\item {Grp. gram.:f.}
\end{itemize}
\begin{itemize}
\item {Utilização:Deprec.}
\end{itemize}
\begin{itemize}
\item {Proveniência:(De \textunderscore magistral\textunderscore )}
\end{itemize}
Qualidade de quem é magistrado.
Pedantismo.
\section{Magistralmente}
\begin{itemize}
\item {Grp. gram.:adv.}
\end{itemize}
De modo magistral.
\section{Magistrando}
\begin{itemize}
\item {Grp. gram.:m.}
\end{itemize}
\begin{itemize}
\item {Proveniência:(Lat. \textunderscore magistrandus\textunderscore )}
\end{itemize}
Candidato a mestre.
\section{Magistratura}
\begin{itemize}
\item {Grp. gram.:f.}
\end{itemize}
\begin{itemize}
\item {Proveniência:(Do lat. \textunderscore magistratus\textunderscore )}
\end{itemize}
Dignidade de magistrado; funcções de magistrado.
Classe dos magistrados.
Duração do cargo de magistrado.
\section{Magma}
\begin{itemize}
\item {Grp. gram.:f.}
\end{itemize}
\begin{itemize}
\item {Proveniência:(Gr. \textunderscore magma\textunderscore )}
\end{itemize}
Matéria espêssa, que fica depois de espremidas as partes mais fluidas de uma substância.
\section{Magnanimamente}
\begin{itemize}
\item {Grp. gram.:adv.}
\end{itemize}
De modo magnânimo.
\section{Magnanimento}
\begin{itemize}
\item {Grp. gram.:adj.}
\end{itemize}
\begin{itemize}
\item {Utilização:Ant.}
\end{itemize}
O mesmo que \textunderscore magnânimo\textunderscore .
\section{Magnanimidade}
\begin{itemize}
\item {Grp. gram.:f.}
\end{itemize}
\begin{itemize}
\item {Proveniência:(Lat. \textunderscore magnanimitas\textunderscore )}
\end{itemize}
Qualidade de quem é magnânimo.
Acto próprio de pessôa magnânima.
\section{Magnânimo}
\begin{itemize}
\item {Grp. gram.:adj.}
\end{itemize}
\begin{itemize}
\item {Proveniência:(Lat. \textunderscore magnanimus\textunderscore )}
\end{itemize}
Que tem grandeza de alma.
Bizarro; generoso.
\section{Magnata}
\begin{itemize}
\item {Grp. gram.:m.  e  f.}
\end{itemize}
\begin{itemize}
\item {Utilização:Pop.}
\end{itemize}
Pessôa importante, pessôa grada. Cf. Garrett, \textunderscore Romanceiro\textunderscore , I, 263.
(Cp. \textunderscore magnate\textunderscore )
\section{Magnate}
\begin{itemize}
\item {Grp. gram.:m.}
\end{itemize}
\begin{itemize}
\item {Proveniência:(Lat. \textunderscore magnas\textunderscore , \textunderscore magnatis\textunderscore )}
\end{itemize}
Pessôa importante ou illustre.
\section{Magnatismo}
\begin{itemize}
\item {Grp. gram.:m.}
\end{itemize}
\begin{itemize}
\item {Proveniência:(De \textunderscore magnate\textunderscore )}
\end{itemize}
Poder dos magnates, nos reinos da Polónia e da Hungria.
\section{Magnés}
\begin{itemize}
\item {Grp. gram.:m. pl.}
\end{itemize}
Índios do Brazil, ao norte de Mato-Grosso.
\section{Magnésia}
\begin{itemize}
\item {Grp. gram.:f.}
\end{itemize}
\begin{itemize}
\item {Utilização:Des.}
\end{itemize}
\begin{itemize}
\item {Proveniência:(Do lat. \textunderscore magnes\textunderscore )}
\end{itemize}
Substância alcalina, inodora e insolúvel na água, e que se empréga como purgante, como antídoto, etc.
\textunderscore Magnésia branca\textunderscore , sub-carbonato de magnésia.
\textunderscore Magnésia negra\textunderscore , peróxydo de manganés.
\section{Magnesiano}
\begin{itemize}
\item {Grp. gram.:adj.}
\end{itemize}
Relativo a magnésia.
Que contém magnésia ou que tem por base a magnésia.
\section{Magnésico}
\begin{itemize}
\item {Grp. gram.:adj.}
\end{itemize}
O mesmo que \textunderscore magnesiano\textunderscore .
\section{Magnésio}
\begin{itemize}
\item {Grp. gram.:m.}
\end{itemize}
Metal que, combinado com o oxygênio, produz a magnésia.
\section{Magnesita}
\begin{itemize}
\item {Grp. gram.:f.}
\end{itemize}
\begin{itemize}
\item {Proveniência:(De \textunderscore magnésia\textunderscore )}
\end{itemize}
Mineral, composto de magnésia, sílica e água, conhecido vulgarmente por \textunderscore escuma do mar\textunderscore .
\section{Magnesite}
\begin{itemize}
\item {Grp. gram.:f.}
\end{itemize}
\begin{itemize}
\item {Proveniência:(De \textunderscore magnésia\textunderscore )}
\end{itemize}
Mineral, composto de magnésia, sílica e água, conhecido vulgarmente por \textunderscore escuma do mar\textunderscore .
\section{Magnete}
\begin{itemize}
\item {Grp. gram.:m.}
\end{itemize}
\begin{itemize}
\item {Proveniência:(Do lat. \textunderscore magnes\textunderscore , \textunderscore magnetis\textunderscore )}
\end{itemize}
Minério de ferro, que tem a propriedade de attrahir certos metaes.
Íman.
Peça de ferro ou de aço magnetizada, com a propriedade de attrahir outros metaes.
\section{Magneticamente}
\begin{itemize}
\item {Grp. gram.:adv.}
\end{itemize}
De modo magnético; irresistivelmente.
\section{Magnético}
\begin{itemize}
\item {Grp. gram.:adj.}
\end{itemize}
\begin{itemize}
\item {Utilização:Fig.}
\end{itemize}
\begin{itemize}
\item {Proveniência:(Lat. \textunderscore magneticus\textunderscore )}
\end{itemize}
Relativo ao magnete ou ao magnetismo.
Que tem a propriedade attractiva do magnete.
Attrahente; encantador: \textunderscore olhos magnéticos\textunderscore .
\section{Magnetipolar}
\begin{itemize}
\item {Grp. gram.:adj.}
\end{itemize}
\begin{itemize}
\item {Proveniência:(De \textunderscore magnético\textunderscore  + \textunderscore polar\textunderscore )}
\end{itemize}
Diz-se de uma rocha magnética, em que se manifestam polos.
\section{Magnetismo}
\begin{itemize}
\item {Grp. gram.:m.}
\end{itemize}
\begin{itemize}
\item {Utilização:Fig.}
\end{itemize}
\begin{itemize}
\item {Proveniência:(De \textunderscore magnete\textunderscore )}
\end{itemize}
Poder attractivo do íman sôbre o ferro e o aço, e faculdade que elle tem de se dirigir, de um lado, para o pólo norte, e do outro para o pólo sul.
Influência de um indivíduo sôbre outro ou sôbre certos objectos, exercida pelo auxílio de um fluido particular, chamado flúido magnético, animal ou vítal e pelo esfôrço da vontade.
Arte de magnetizar.
Attracção.
Propriedade de attrahir, de encantar.
\section{Magnetista}
\begin{itemize}
\item {Grp. gram.:m.}
\end{itemize}
Aquelle que se dedica ao estudo e á prática do magnetismo.
\section{Magnetite}
\begin{itemize}
\item {Grp. gram.:f.}
\end{itemize}
\begin{itemize}
\item {Proveniência:(De \textunderscore magnete\textunderscore )}
\end{itemize}
Designação scientífica da pedra íman.
Óxydo de ferro magnético.
\section{Magnetização}
\begin{itemize}
\item {Grp. gram.:f.}
\end{itemize}
Acto ou effeito de magnetizar.
\section{Magnetizador}
\begin{itemize}
\item {Grp. gram.:adj.}
\end{itemize}
\begin{itemize}
\item {Grp. gram.:M.}
\end{itemize}
Que magnetiza.
Aquelle que magnetiza.
\section{Magnetizar}
\begin{itemize}
\item {Grp. gram.:v. t.}
\end{itemize}
\begin{itemize}
\item {Utilização:Fig.}
\end{itemize}
\begin{itemize}
\item {Proveniência:(De \textunderscore magnete\textunderscore )}
\end{itemize}
Communicar o fluido magnético a.
Ter influência sôbre.
Dominar a vontade de.
Attrahir; encantar.
\section{Magnetizável}
\begin{itemize}
\item {Grp. gram.:adj.}
\end{itemize}
Que é susceptível de sêr magnetizado.
\section{Magneto-eléctrico}
\begin{itemize}
\item {Grp. gram.:adj.}
\end{itemize}
O mesmo que \textunderscore electro-magnético\textunderscore .
\section{Magnetogenia}
\begin{itemize}
\item {Grp. gram.:f.}
\end{itemize}
\begin{itemize}
\item {Proveniência:(Do gr. \textunderscore magnes\textunderscore  + \textunderscore genea\textunderscore )}
\end{itemize}
Estudo dos phenómenos magnéticos.
\section{Magnetologia}
\begin{itemize}
\item {Grp. gram.:f.}
\end{itemize}
\begin{itemize}
\item {Proveniência:(Do gr. \textunderscore magnes\textunderscore  + \textunderscore logos\textunderscore )}
\end{itemize}
Tratado á cêrca dos ímans e das suas propriedades.
\section{Magnetológico}
\begin{itemize}
\item {Grp. gram.:adj.}
\end{itemize}
Relativo á magnetologia.
\section{Magnetómetro}
\begin{itemize}
\item {Grp. gram.:m.}
\end{itemize}
\begin{itemize}
\item {Proveniência:(Do gr. \textunderscore magnes\textunderscore  + \textunderscore metron\textunderscore )}
\end{itemize}
Instrumento, para fazer conhecer a fôrça attractiva de um íman.
\section{Magnetotechnia}
\begin{itemize}
\item {Grp. gram.:f.}
\end{itemize}
\begin{itemize}
\item {Proveniência:(Do gr. \textunderscore magnes\textunderscore  + \textunderscore tekhne\textunderscore )}
\end{itemize}
Arte do magnetizador.
\section{Magnetotecnia}
\begin{itemize}
\item {Grp. gram.:f.}
\end{itemize}
\begin{itemize}
\item {Proveniência:(Do gr. \textunderscore magnes\textunderscore  + \textunderscore tekhne\textunderscore )}
\end{itemize}
Arte do magnetizador.
\section{Magnífica}
\begin{itemize}
\item {Grp. gram.:f.}
\end{itemize}
\begin{itemize}
\item {Utilização:Pop.}
\end{itemize}
Oração, que o povo costuma rezar, quando troveja.
(Corr. do lat. \textunderscore magnificat\textunderscore )
\section{Magnificação}
\begin{itemize}
\item {Grp. gram.:f.}
\end{itemize}
\begin{itemize}
\item {Proveniência:(Lat. \textunderscore magnificatio\textunderscore )}
\end{itemize}
Acto ou effeito de magnificar.
\section{Magnificador}
\begin{itemize}
\item {Grp. gram.:m.  e  adj.}
\end{itemize}
O que magnifíca.
\section{Magnificamente}
\begin{itemize}
\item {Grp. gram.:adv.}
\end{itemize}
De modo magnífico.
Excellentemente; ostentosamente.
\section{Magnificar}
\begin{itemize}
\item {Grp. gram.:v. t.}
\end{itemize}
\begin{itemize}
\item {Proveniência:(Lat. \textunderscore magnificare\textunderscore )}
\end{itemize}
Engrandecer, louvando.
Exaltar; glorificar.
Ampliar.
\section{Magnificatório}
\begin{itemize}
\item {Grp. gram.:adj.}
\end{itemize}
\begin{itemize}
\item {Proveniência:(Do lat. \textunderscore magnificatus\textunderscore )}
\end{itemize}
Que magnifíca.
\section{Magnificência}
\begin{itemize}
\item {Grp. gram.:f.}
\end{itemize}
\begin{itemize}
\item {Proveniência:(Lat. \textunderscore magnificentia\textunderscore )}
\end{itemize}
Qualidade de magnificente.
\section{Magnificente}
\begin{itemize}
\item {Grp. gram.:adj.}
\end{itemize}
\begin{itemize}
\item {Proveniência:(Do rad. de \textunderscore magnificência\textunderscore )}
\end{itemize}
Grandioso.
Sumptuoso.
Generoso; liberal.
\section{Magnífico}
\begin{itemize}
\item {Grp. gram.:adj.}
\end{itemize}
\begin{itemize}
\item {Grp. gram.:M.}
\end{itemize}
\begin{itemize}
\item {Proveniência:(Lat. \textunderscore magnificus\textunderscore )}
\end{itemize}
Magnificente.
Muito bom; excellente: \textunderscore jantar magnífico\textunderscore .
Gênero de aves, (\textunderscore paradisoea magnifica\textunderscore , Lin.).
\section{Magniloquência}
\begin{itemize}
\item {fónica:cu-en}
\end{itemize}
\begin{itemize}
\item {Grp. gram.:f.}
\end{itemize}
\begin{itemize}
\item {Proveniência:(Lat. \textunderscore magniloquentia\textunderscore )}
\end{itemize}
Linguagem sublime, pomposa.
\section{Magníloquo}
\begin{itemize}
\item {Grp. gram.:adj.}
\end{itemize}
\begin{itemize}
\item {Proveniência:(Lat. \textunderscore magniloquus\textunderscore )}
\end{itemize}
O mesmo que \textunderscore eloquente\textunderscore .
\section{Magnitude}
\begin{itemize}
\item {Grp. gram.:f.}
\end{itemize}
\begin{itemize}
\item {Utilização:Fig.}
\end{itemize}
\begin{itemize}
\item {Proveniência:(Lat. \textunderscore magnitudo\textunderscore )}
\end{itemize}
Qualidade do que é magno; grandeza.
Importância.
\section{Magno}
\begin{itemize}
\item {Grp. gram.:adj.}
\end{itemize}
\begin{itemize}
\item {Utilização:Poét.}
\end{itemize}
\begin{itemize}
\item {Grp. gram.:M.}
\end{itemize}
\begin{itemize}
\item {Proveniência:(Lat. \textunderscore magnus\textunderscore )}
\end{itemize}
Grande.
Importante.
Concha de cochinilha silvestre.
\section{Magnole}
\begin{itemize}
\item {Grp. gram.:m.}
\end{itemize}
\begin{itemize}
\item {Utilização:T. do Pôrto}
\end{itemize}
O mesmo que \textunderscore nêspera\textunderscore .
\section{Magnólia}
\begin{itemize}
\item {Grp. gram.:f.}
\end{itemize}
\begin{itemize}
\item {Proveniência:(De \textunderscore Magnol\textunderscore , n. p.)}
\end{itemize}
Gênero de árvores, de flôres muito aromáticas.
\section{Magnoliáceas}
\begin{itemize}
\item {Grp. gram.:f.}
\end{itemize}
\begin{itemize}
\item {Proveniência:(De \textunderscore magnoliáceo\textunderscore )}
\end{itemize}
Família de plantas, que tem por typo a magnólia.
\section{Magnoliáceo}
\begin{itemize}
\item {Grp. gram.:adj.}
\end{itemize}
Relativo ou semelhante á magnólia.
\section{Mago}
\begin{itemize}
\item {Grp. gram.:m.}
\end{itemize}
\begin{itemize}
\item {Grp. gram.:Adj.}
\end{itemize}
\begin{itemize}
\item {Proveniência:(Lat. \textunderscore magus\textunderscore )}
\end{itemize}
Antigo sacerdote dos Médos.
Cada uma das três personagens reaes, que foram a Bethlém adorar Jesus recém-nascido.
Feiticeiro.
Encantador; seductor.
Delicioso.
\section{Mágoa}
\begin{itemize}
\item {Grp. gram.:f.}
\end{itemize}
\begin{itemize}
\item {Utilização:Fig.}
\end{itemize}
\begin{itemize}
\item {Proveniência:(Do lat. \textunderscore macula\textunderscore )}
\end{itemize}
Mancha ou nódoa, resultante de contusão; (p. us. neste sentido).
Desgôsto; tristeza.
Impressão, produzida na alma por um facto que desagrada.
\section{Magoadinha}
\begin{itemize}
\item {Grp. gram.:adj.}
\end{itemize}
\begin{itemize}
\item {Proveniência:(De \textunderscore magoado\textunderscore )}
\end{itemize}
Us. na loc. ellíptica e pungitiva das mulheres de Ovar: \textunderscore ai magoadinha de quem!\textunderscore 
\section{Magoado}
\begin{itemize}
\item {Grp. gram.:adj.}
\end{itemize}
\begin{itemize}
\item {Proveniência:(De \textunderscore magoar\textunderscore )}
\end{itemize}
Que revela mágoa: \textunderscore palavras magoadas\textunderscore .
Melindrado; offendido: \textunderscore fiquei magoado com a calúmnia\textunderscore .
\section{Magoar}
\begin{itemize}
\item {Grp. gram.:v. t.}
\end{itemize}
\begin{itemize}
\item {Proveniência:(Do lat. \textunderscore maculare\textunderscore )}
\end{itemize}
Pisar; contundir.
Affligir.
Melindrar; offender.
\section{Magoativo}
\begin{itemize}
\item {Grp. gram.:adj.}
\end{itemize}
\begin{itemize}
\item {Proveniência:(De \textunderscore magoar\textunderscore )}
\end{itemize}
Que produz mágoa:«\textunderscore tristeza, mais que todas magoativa\textunderscore ». Camillo, \textunderscore Amor de Salv.\textunderscore , 3.^a ed., 164.
\section{Magorim}
\begin{itemize}
\item {Grp. gram.:m.}
\end{itemize}
Planta jasmínea, (\textunderscore jasminum sambuc\textunderscore ).
\section{Magosteira}
\begin{itemize}
\item {Grp. gram.:f.}
\end{itemize}
\begin{itemize}
\item {Utilização:Prov.}
\end{itemize}
\begin{itemize}
\item {Utilização:beir.}
\end{itemize}
Lugar, onde ficam resíduos de uvas esmagadas, nas vinhas ou junto dos lagares, por occasião das vindimas.
(Talvez por \textunderscore bagosteira\textunderscore , de \textunderscore bago\textunderscore )
\section{Magote}
\begin{itemize}
\item {Grp. gram.:m.}
\end{itemize}
Grupo de gente.
Rancho; multidão.
Montão, acervo.
(Cp. cast. \textunderscore mogote\textunderscore )
\section{Magoua}
\begin{itemize}
\item {Grp. gram.:m.}
\end{itemize}
Espécie de tinamu, dos mais encorpados.
\section{Magreira}
\begin{itemize}
\item {Grp. gram.:f.}
\end{itemize}
\begin{itemize}
\item {Utilização:Pop.}
\end{itemize}
O mesmo que \textunderscore magreza\textunderscore , por doença.
\section{Magreirote}
\begin{itemize}
\item {Grp. gram.:adj.}
\end{itemize}
\begin{itemize}
\item {Proveniência:(De \textunderscore magreira\textunderscore )}
\end{itemize}
O mesmo que \textunderscore magrete\textunderscore .
\section{Magrete}
\begin{itemize}
\item {fónica:grê}
\end{itemize}
\begin{itemize}
\item {Grp. gram.:adj.}
\end{itemize}
\begin{itemize}
\item {Utilização:Fam.}
\end{itemize}
Um tanto magro.
\section{Magreza}
\begin{itemize}
\item {Grp. gram.:f.}
\end{itemize}
Qualidade ou estado de magro.
\section{Magriça}
\begin{itemize}
\item {Grp. gram.:f.}
\end{itemize}
\begin{itemize}
\item {Utilização:Prov.}
\end{itemize}
\begin{itemize}
\item {Utilização:alent.}
\end{itemize}
Planta, (\textunderscore calluna vulgaris\textunderscore , Salisb.).
O mesmo que \textunderscore queiró\textunderscore .
\section{Magricela}
\begin{itemize}
\item {Grp. gram.:m.}
\end{itemize}
(V.magrizela)
\section{Magricelas}
\begin{itemize}
\item {Grp. gram.:m.}
\end{itemize}
(V.magrizela)
\section{Magriço}
\begin{itemize}
\item {Grp. gram.:m.}
\end{itemize}
\begin{itemize}
\item {Proveniência:(De \textunderscore Magriço\textunderscore , n. p.)}
\end{itemize}
Paladino das damas.
Defensor piegas ou ridículo de coisas fúteis. Cf. Filinto, IV, 215 e 242.
\section{Magriz}
\begin{itemize}
\item {Grp. gram.:m.  e  adj.}
\end{itemize}
\begin{itemize}
\item {Proveniência:(De \textunderscore magro\textunderscore )}
\end{itemize}
Pessôa muito magra:«\textunderscore como usam os magrizes janotas\textunderscore ». Castilho.
\section{Magrizel}
\begin{itemize}
\item {Grp. gram.:m.}
\end{itemize}
Homem magro e descòrado. Cf. Castilho, \textunderscore Fausto\textunderscore , 194.
(Cp. \textunderscore magrizela\textunderscore )
\section{Magrizela}
\begin{itemize}
\item {Grp. gram.:m.  e  f.}
\end{itemize}
\begin{itemize}
\item {Proveniência:(De \textunderscore magriz\textunderscore )}
\end{itemize}
Pessoa magra e descòrada.
\section{Magro}
\begin{itemize}
\item {Grp. gram.:adj.}
\end{itemize}
\begin{itemize}
\item {Utilização:Fig.}
\end{itemize}
\begin{itemize}
\item {Proveniência:(Do lat. \textunderscore macer\textunderscore )}
\end{itemize}
Que tem falta de tecido adiposo: \textunderscore pessôa magra\textunderscore .
Em que há pouca ou nenhuma gordura ou sebo: \textunderscore carne magra\textunderscore .
Pouco rendoso.
Diz-se do tempo ou dos dias, em que é prohibido pela Igreja comer carne.
Dia ou tempo, em que pela Igreja é prohibido comer carne: \textunderscore hoje é dia de magro\textunderscore .
\section{Magrote}
\begin{itemize}
\item {Grp. gram.:adj.}
\end{itemize}
O mesmo que \textunderscore magrete\textunderscore .
\section{Mágua}
\begin{itemize}
\item {Grp. gram.:f.}
\end{itemize}
\begin{itemize}
\item {Utilização:Fig.}
\end{itemize}
\begin{itemize}
\item {Proveniência:(Do lat. \textunderscore macula\textunderscore )}
\end{itemize}
Mancha ou nódoa, resultante de contusão; (p. us. neste sentido).
Desgôsto; tristeza.
Impressão, produzida na alma por um facto que desagrada.
\section{Maguari}
\begin{itemize}
\item {Grp. gram.:m.}
\end{itemize}
\begin{itemize}
\item {Utilização:Bras. do N}
\end{itemize}
Ave pernalta, o mesmo que \textunderscore baguari\textunderscore .
\section{Maguei}
\begin{itemize}
\item {Grp. gram.:m.}
\end{itemize}
O mesmo que \textunderscore pita\textunderscore ^1.
\section{Magueixo}
\begin{itemize}
\item {Grp. gram.:m.}
\end{itemize}
\begin{itemize}
\item {Utilização:T. de Turquel}
\end{itemize}
Alça, que se colloca na deanteira de um carro, carregado de pranchas ou toros compridos, para que não toquem nos bois.
Cada um dos dormentes, que se collocam no leito do carro, para assento de pipas, ou de vasilhas semelhantes.
\section{Maguér}
\begin{itemize}
\item {Grp. gram.:prep.}
\end{itemize}
\begin{itemize}
\item {Utilização:Ant.}
\end{itemize}
\begin{itemize}
\item {Proveniência:(Do fr. \textunderscore malgré\textunderscore )}
\end{itemize}
Apesar de; não obstante.
\section{Maguilho}
\begin{itemize}
\item {Grp. gram.:m.}
\end{itemize}
Macieira brava.
(Cast. \textunderscore maguillo\textunderscore )
\section{Magujo}
\begin{itemize}
\item {Grp. gram.:m.}
\end{itemize}
Instrumento, para extrair das juntas da embarcação a estopa velha.
(Cast. \textunderscore magujo\textunderscore )
\section{Magulhar}
\begin{itemize}
\item {Grp. gram.:v. t.}
\end{itemize}
\begin{itemize}
\item {Utilização:Ant.}
\end{itemize}
\begin{itemize}
\item {Proveniência:(Do lat. \textunderscore maculare\textunderscore )}
\end{itemize}
O mesmo que \textunderscore macular\textunderscore . Cf. Usque, 44 v.^o
\section{Magustal}
\begin{itemize}
\item {Grp. gram.:adj.}
\end{itemize}
Relativo a magusto.
\section{Magusto}
\begin{itemize}
\item {Grp. gram.:m.}
\end{itemize}
\begin{itemize}
\item {Proveniência:(Do lat. (?) + \textunderscore ustus\textunderscore ?)}
\end{itemize}
Fogueira, para assar castanha.
Castanhas assadas em fogueira.
\section{Magiar}
\begin{itemize}
\item {Grp. gram.:adj.}
\end{itemize}
\begin{itemize}
\item {Grp. gram.:M.}
\end{itemize}
Relativo á Hungria ou aos Húngaros.
Língua dos Húngaros.
Habitante da Hungria.
\section{Mahamude}
\begin{itemize}
\item {Grp. gram.:f.}
\end{itemize}
\begin{itemize}
\item {Utilização:Ant.}
\end{itemize}
\begin{itemize}
\item {Proveniência:(T. \textunderscore ár.\textunderscore )}
\end{itemize}
Planta medicinal, o mesmo que \textunderscore escamoneia\textunderscore .
\section{Mahamúdi}
\begin{itemize}
\item {Grp. gram.:m.}
\end{itemize}
\begin{itemize}
\item {Utilização:Ant.}
\end{itemize}
\begin{itemize}
\item {Proveniência:(T. ár., de \textunderscore Mamhud\textunderscore , n. p.)}
\end{itemize}
Moéda de oiro, e de prata, na Turquia e na Índia. Cf. Couto, \textunderscore Década\textunderscore  VII, fol. 191.
\section{Mahometa}
O mesmo que \textunderscore mahometano\textunderscore . Cf. Filinto, \textunderscore Vida de D. Man.\textunderscore , I, 205.
\section{Mahometanismo}
\begin{itemize}
\item {Grp. gram.:m.}
\end{itemize}
O mesmo que \textunderscore mahometismo\textunderscore .
\section{Mahometano}
\begin{itemize}
\item {Grp. gram.:adj.}
\end{itemize}
\begin{itemize}
\item {Grp. gram.:M.}
\end{itemize}
Relativo a Mahomet ou á sua seita.
Sectário de Mahomet.
\section{Mahomético}
\begin{itemize}
\item {Grp. gram.:adj.}
\end{itemize}
(V.mahometano)
\section{Mahometismo}
\begin{itemize}
\item {Grp. gram.:m.}
\end{itemize}
Religião, fundada por Mahomet.
\section{Mahona}
\begin{itemize}
\item {Grp. gram.:f.}
\end{itemize}
Espécie de embarcação antiga.
\section{Mahónia}
\begin{itemize}
\item {Grp. gram.:f.}
\end{itemize}
\begin{itemize}
\item {Proveniência:(De \textunderscore Mahon\textunderscore , n. p.)}
\end{itemize}
Planta berberídea.
\section{Mahuba}
\begin{itemize}
\item {Grp. gram.:f.}
\end{itemize}
Árvore silvestre do Brasil.
\section{Mãi}
\begin{itemize}
\item {Grp. gram.:f.}
\end{itemize}
\begin{itemize}
\item {Utilização:Fig.}
\end{itemize}
\begin{itemize}
\item {Utilização:Fig.}
\end{itemize}
\begin{itemize}
\item {Utilização:Gír.}
\end{itemize}
\begin{itemize}
\item {Proveniência:(Do lat. \textunderscore mater\textunderscore )}
\end{itemize}
Mulher, que deu á luz um ou mais filhos.
Relação de parentesco de uma mulher para com seus filhos.
Qualquer fêmea, que teve filho ou filhos.
Madre, bôrra do vinho.
Fonte, origem.
Mulher caridosa e desvelada.
Fechadura.
\section{Maia}
\begin{itemize}
\item {Grp. gram.:f.}
\end{itemize}
\begin{itemize}
\item {Utilização:Fig.}
\end{itemize}
\begin{itemize}
\item {Utilização:Prov.}
\end{itemize}
\begin{itemize}
\item {Utilização:minh.}
\end{itemize}
\begin{itemize}
\item {Utilização:T. de Turquel. Bot}
\end{itemize}
\begin{itemize}
\item {Utilização:T. da Bairrada}
\end{itemize}
\begin{itemize}
\item {Proveniência:(De \textunderscore Maio\textunderscore )}
\end{itemize}
Antiga festa popular, nos primeiros dias de Maio.
Mulher, que se enfeita com mau gôsto.
Criança, que pede donativos para as maias, (festas).
O mesmo que \textunderscore dedaleira\textunderscore .
Giesta em flôr.
\section{Maiá}
\begin{itemize}
\item {Grp. gram.:m.}
\end{itemize}
Língua holophrástica do Iucatão.
\section{Maiano}
\begin{itemize}
\item {Grp. gram.:m.  e  adj.}
\end{itemize}
O mesmo que \textunderscore maiato\textunderscore ^2.
\section{Maião}
\begin{itemize}
\item {Grp. gram.:m.}
\end{itemize}
\begin{itemize}
\item {Utilização:T. de Aveiro}
\end{itemize}
Um dos dois remeiros dos barcos de pesca, o que trabalha com a pá a bombordo.
\section{Maiata}
\begin{itemize}
\item {Grp. gram.:f.}
\end{itemize}
Mulher natural da Maia. Cf. Camillo, \textunderscore Maria da Fonte\textunderscore , 99.
(Cp. \textunderscore maiato\textunderscore ^2)
\section{Maiato}
\begin{itemize}
\item {Grp. gram.:m.}
\end{itemize}
Árvore silvestre do Brasil.
\section{Maiato}
\begin{itemize}
\item {Grp. gram.:adj.}
\end{itemize}
\begin{itemize}
\item {Grp. gram.:M.}
\end{itemize}
\begin{itemize}
\item {Proveniência:(De \textunderscore Maia\textunderscore , n. p.)}
\end{itemize}
Relativo aos povos da Maia.
Que é natural da Maia.
Homem, que nasceu ou vive na Maia. Cf. Camillo, \textunderscore Corja\textunderscore , 208 e 230. \textunderscore Narcoticos\textunderscore , II, 327.
\section{Maíça}
T. (cit. por Castilho, como fazendo parte da loc. \textunderscore andar com alguém á mal maíça\textunderscore , andar de rixa com alguém. Provavelmente, corr. de \textunderscore maínça\textunderscore )
\section{Maidona}
\begin{itemize}
\item {Grp. gram.:f.}
\end{itemize}
Espécie de feijão de Angola.
\section{Maiêutica}
\begin{itemize}
\item {Grp. gram.:f.}
\end{itemize}
Uma das fórmas pedagógicas do processo socrático.
\section{Maimbu}
\begin{itemize}
\item {Grp. gram.:m.}
\end{itemize}
\begin{itemize}
\item {Utilização:Bras}
\end{itemize}
Planta rasteira e medicinal, que cresce nas praias.
\section{Maimona}
\begin{itemize}
\item {Grp. gram.:f.}
\end{itemize}
\begin{itemize}
\item {Utilização:Ant.}
\end{itemize}
Espécie de trabuco. Cf. \textunderscore Cêrco de Mazagão\textunderscore , 52.
\section{Mainata}
\begin{itemize}
\item {Grp. gram.:m.}
\end{itemize}
O mesmo que \textunderscore mainato\textunderscore .
\section{Mainate}
\begin{itemize}
\item {Grp. gram.:m.}
\end{itemize}
O mesmo que \textunderscore mainato\textunderscore .
\section{Mainato}
\begin{itemize}
\item {Grp. gram.:m.}
\end{itemize}
\begin{itemize}
\item {Utilização:Ant.}
\end{itemize}
Aquelle que lava roupas, na Índia Portuguesa e na China.
(Do tâmil)
\section{Mainça}
\begin{itemize}
\item {Grp. gram.:f.}
\end{itemize}
\begin{itemize}
\item {Proveniência:(Do lat. hyp. \textunderscore manitia\textunderscore , de \textunderscore manus\textunderscore )}
\end{itemize}
Mão-cheia.
Aquillo que cabe na mão.
Remate do fuso.
\section{Mainel}
\begin{itemize}
\item {Grp. gram.:m.}
\end{itemize}
\begin{itemize}
\item {Utilização:Ant.}
\end{itemize}
O mesmo que \textunderscore corrimão\textunderscore .
Pilarete que divide uma fresta verticalmente, sustentando a respectiva bandeira ou laçarias.
\section{Mainibu}
\begin{itemize}
\item {Grp. gram.:m.}
\end{itemize}
Erva rasteira do Brasil.
\section{Maino}
\begin{itemize}
\item {Grp. gram.:adj.}
\end{itemize}
\begin{itemize}
\item {Utilização:T. de Turquel}
\end{itemize}
Calmo, tranquillo.
(Cp. \textunderscore amainar\textunderscore )
\section{Maio}
\begin{itemize}
\item {Grp. gram.:m.}
\end{itemize}
\begin{itemize}
\item {Utilização:Fig.}
\end{itemize}
\begin{itemize}
\item {Utilização:T. da Bairrada}
\end{itemize}
\begin{itemize}
\item {Grp. gram.:Adj.}
\end{itemize}
\begin{itemize}
\item {Proveniência:(Lat. \textunderscore majus\textunderscore )}
\end{itemize}
Quinto mês do anno romano.
Indivíduo, enfeitado com flôres, garrido.
Tempo das flôres.
Planta campestre, de ramos longos e delgados e flôres amarelas.
Relativo a Maio.
Que apparece em Maio.
\section{Maiobá}
\begin{itemize}
\item {Grp. gram.:m.}
\end{itemize}
Arbusto medicinal da Ilha de San-Thomé.
\section{Maiólica}
\begin{itemize}
\item {Grp. gram.:f.}
\end{itemize}
O mesmo ou melhor que \textunderscore majólica\textunderscore .
\section{Maionesa}
\begin{itemize}
\item {fónica:nê}
\end{itemize}
\begin{itemize}
\item {Grp. gram.:f.  e  adj.}
\end{itemize}
O mesmo que \textunderscore baionesa\textunderscore . Cf. Castilho, \textunderscore Avarento\textunderscore , 182.
\section{Maionese}
\begin{itemize}
\item {Grp. gram.:f.}
\end{itemize}
\begin{itemize}
\item {Utilização:Ext.}
\end{itemize}
\begin{itemize}
\item {Utilização:Fig.}
\end{itemize}
\begin{itemize}
\item {Proveniência:(Fr. \textunderscore mayonnaise\textunderscore )}
\end{itemize}
Espécie de môlho frio, composto de azeite, vinagre, sal, pimenta, mostarda e ovos batidos.
Iguaria com aquelle môlho.
Mistura ou confusão de várias coisas.
\section{Maiór}
\begin{itemize}
\item {Grp. gram.:adj.}
\end{itemize}
\begin{itemize}
\item {Grp. gram.:M.}
\end{itemize}
\begin{itemize}
\item {Grp. gram.:Pl.}
\end{itemize}
\begin{itemize}
\item {Proveniência:(Do lat. \textunderscore major\textunderscore )}
\end{itemize}
Que excede outro em grandeza, em espaço, em intensidade ou em número.
Que chegou á idade legal para reger sua pessôa e bens.
* \textunderscore Maior da marca\textunderscore , muito grande, extraordinário, de marca maior. Cf. Camillo, \textunderscore Engeitada\textunderscore .
Indivíduo, que attingiu a maioridade ou que tem mais de 21 annos.
Antepassados.
\section{Maioral}
\begin{itemize}
\item {Grp. gram.:m.}
\end{itemize}
\begin{itemize}
\item {Utilização:Fig.}
\end{itemize}
\begin{itemize}
\item {Proveniência:(De \textunderscore maior\textunderscore )}
\end{itemize}
Chefe.
O maior de todos, (falando-se dos animaes de um rebanho).
\section{Maiorano}
\begin{itemize}
\item {Grp. gram.:m.}
\end{itemize}
Planta malvácea do Brasil.
\section{Maiorca}
\begin{itemize}
\item {Grp. gram.:f.}
\end{itemize}
\begin{itemize}
\item {Proveniência:(De \textunderscore Maiorca\textunderscore , n. p.)}
\end{itemize}
Variedade de pêra.
\section{Maioria}
\begin{itemize}
\item {Grp. gram.:f.}
\end{itemize}
\begin{itemize}
\item {Grp. gram.:Pl.}
\end{itemize}
\begin{itemize}
\item {Utilização:Prov.}
\end{itemize}
\begin{itemize}
\item {Proveniência:(De \textunderscore maior\textunderscore )}
\end{itemize}
O maior número, a maior parte.
Superioridade.
Luvas, gratificação.
\section{Maioridade}
\begin{itemize}
\item {Grp. gram.:f.}
\end{itemize}
\begin{itemize}
\item {Utilização:Ext.}
\end{itemize}
\begin{itemize}
\item {Proveniência:(De \textunderscore maior\textunderscore )}
\end{itemize}
Idade, em que o indivíduo entra no gôzo de direitos civis.
Completo desenvolvimento de uma sociedade.
\section{Maiorino}
\begin{itemize}
\item {Grp. gram.:m.}
\end{itemize}
\begin{itemize}
\item {Utilização:Ant.}
\end{itemize}
\begin{itemize}
\item {Proveniência:(De \textunderscore maior\textunderscore )}
\end{itemize}
Juiz superior, de cujas decisões só havia recurso para o monarcha ou para o governador da província. Cf. Herculano, \textunderscore Bobo\textunderscore , 9.
O mesmo que \textunderscore alcalde\textunderscore . Cf. Herculano, \textunderscore Hist. de Port.\textunderscore , IV, 45.
\section{Maioríssimo}
\begin{itemize}
\item {Grp. gram.:adj.}
\end{itemize}
\begin{itemize}
\item {Proveniência:(De \textunderscore maior\textunderscore )}
\end{itemize}
Muito maior; maior que todos os outros. Cf. Arn. Gama, \textunderscore Últ. Dona\textunderscore , 154.
\section{Maiormente}
\begin{itemize}
\item {Grp. gram.:adv.}
\end{itemize}
O mesmo que \textunderscore mòrmente\textunderscore :«\textunderscore Lamentor houve della dó, maiormente de suas lágrimas.\textunderscore »Bernardim, \textunderscore Saudades\textunderscore . Cf. \textunderscore Eufrosina\textunderscore , 46; \textunderscore Luz e Calor\textunderscore , 21.
\section{Maiorquino}
\begin{itemize}
\item {Grp. gram.:adj.}
\end{itemize}
\begin{itemize}
\item {Grp. gram.:M.}
\end{itemize}
Relativo a Maiorca, nas Baleares.
Habitante de Maiorca; dialecto daquella ilha.
\section{Maiosia}
\begin{itemize}
\item {Grp. gram.:f.}
\end{itemize}
\begin{itemize}
\item {Proveniência:(Do rad. de \textunderscore maio\textunderscore )}
\end{itemize}
Presentes, que os vassallos menores recebiam dos grandes vassallos, e que consistiam talvez em armas, com as quaes se deviam apresentar nos alardos de Maio.
\section{Maios-pequenos}
\begin{itemize}
\item {Grp. gram.:m. pl.}
\end{itemize}
\begin{itemize}
\item {Utilização:Prov.}
\end{itemize}
\begin{itemize}
\item {Utilização:alent.}
\end{itemize}
Espécie de planta, (\textunderscore iris sisynrinchum\textunderscore , Lin.).
\section{Maiozinho}
\begin{itemize}
\item {Grp. gram.:adj.}
\end{itemize}
Relativo a maio; que apparece em maio.
\section{Mais}
\begin{itemize}
\item {Grp. gram.:M.}
\end{itemize}
\begin{itemize}
\item {Grp. gram.:Adj.}
\end{itemize}
\begin{itemize}
\item {Proveniência:(Lat. \textunderscore magis\textunderscore )}
\end{itemize}
\textunderscore adv.,\textunderscore  (designativo de \textunderscore aumento\textunderscore , \textunderscore grandeza\textunderscore  ou \textunderscore comparação\textunderscore )
Também; àlém disso.
Com preferência.
O mesmo que \textunderscore já\textunderscore : \textunderscore não é mais o que foi\textunderscore .
\textunderscore Não mais\textunderscore , nunca mais: \textunderscore não mais o verei\textunderscore .
O restante; o que falta dizer.
O maior número.
Em maior quantidade ou número.
\section{Mais}
\begin{itemize}
\item {Grp. gram.:conj.}
\end{itemize}
\begin{itemize}
\item {Utilização:Des.}
\end{itemize}
O mesmo que \textunderscore mas\textunderscore . Cf. \textunderscore Port. Mon. Hist., Script.\textunderscore , 318 e 328.
\section{Maís}
\begin{itemize}
\item {Grp. gram.:m.}
\end{itemize}
\begin{itemize}
\item {Proveniência:(Fr. \textunderscore maïs\textunderscore )}
\end{itemize}
Variedade de milho graúdo.
\section{Maisena}
\begin{itemize}
\item {fónica:ma-i}
\end{itemize}
\begin{itemize}
\item {Grp. gram.:f.}
\end{itemize}
\begin{itemize}
\item {Proveniência:(De \textunderscore maís\textunderscore )}
\end{itemize}
Farinha fina de milho.
\section{Maisquerer}
\begin{itemize}
\item {Grp. gram.:v. t.}
\end{itemize}
\begin{itemize}
\item {Proveniência:(De \textunderscore mais\textunderscore  + \textunderscore querer\textunderscore )}
\end{itemize}
Preferir; querer mais a.
\section{Maitaca}
\begin{itemize}
\item {Grp. gram.:f.}
\end{itemize}
\begin{itemize}
\item {Utilização:Bras}
\end{itemize}
\begin{itemize}
\item {Utilização:Fig.}
\end{itemize}
Espécie de papagaio verde do Brasil.
Mulher muito faladora.
\section{Maituca}
\begin{itemize}
\item {Grp. gram.:f.}
\end{itemize}
Ave brasileira, nociva aos milharaes.
\section{Maiumbella}
\begin{itemize}
\item {Grp. gram.:f.}
\end{itemize}
(V.himba)
\section{Maiurunas}
\begin{itemize}
\item {Grp. gram.:m. pl.}
\end{itemize}
Tríbo de Índios das margens do Jabari, no Brasil.
\section{Maiúscula}
\begin{itemize}
\item {Grp. gram.:f.}
\end{itemize}
\begin{itemize}
\item {Proveniência:(De \textunderscore maiúsculo\textunderscore )}
\end{itemize}
Letra maiúscula.
\section{Maiúsculo}
\begin{itemize}
\item {Grp. gram.:adj.}
\end{itemize}
\begin{itemize}
\item {Proveniência:(Lat. \textunderscore majusculus\textunderscore )}
\end{itemize}
Diz-se dos caracteres, usados no princípio de nomes próprios ou de nomes communs que eram próprios, no princípio de qualquer discurso, etc.
\section{Majalacórdia}
\begin{itemize}
\item {Grp. gram.:f.}
\end{itemize}
\begin{itemize}
\item {Utilização:ant.}
\end{itemize}
\begin{itemize}
\item {Utilização:Pleb.}
\end{itemize}
O mesmo que \textunderscore misericórdia\textunderscore . Cf. Sim. Mach., 59.
\section{Majangra}
\begin{itemize}
\item {Grp. gram.:m.}
\end{itemize}
\begin{itemize}
\item {Utilização:Prov.}
\end{itemize}
\begin{itemize}
\item {Utilização:trasm.}
\end{itemize}
Rapaz preguiçoso, indolente. (Colhido em Penaguião)
\section{Majarrona}
\begin{itemize}
\item {Grp. gram.:f.}
\end{itemize}
(corr. de \textunderscore bujarrona\textunderscore )
\section{Majerioba}
\begin{itemize}
\item {Grp. gram.:f.}
\end{itemize}
\begin{itemize}
\item {Utilização:Bras}
\end{itemize}
Planta medicinal da região do Amazonas.
(Cp. \textunderscore mangerioba\textunderscore )
\section{Majestade}
\begin{itemize}
\item {Grp. gram.:f.}
\end{itemize}
\begin{itemize}
\item {Proveniência:(Lat. \textunderscore majestas\textunderscore , \textunderscore majestatis\textunderscore )}
\end{itemize}
Apparência de grandeza.
Aspecto solenne.
Sublimidade.
Grandeza que incute respeito.
Poder real.
Título honorifico dos soberanos e suas espôsas.
\section{Majestático}
\begin{itemize}
\item {Grp. gram.:adj.}
\end{itemize}
\begin{itemize}
\item {Proveniência:(Do lat. \textunderscore majestas\textunderscore , \textunderscore majestatis\textunderscore )}
\end{itemize}
Relativo á majestade ou poder supremo.
Majestoso.
\section{Majestosamente}
\begin{itemize}
\item {Grp. gram.:adv.}
\end{itemize}
De modo majestoso; magnificentemente.
\section{Majestoso}
\begin{itemize}
\item {Grp. gram.:adj.}
\end{itemize}
\begin{itemize}
\item {Proveniência:(Do lat. \textunderscore majestas\textunderscore )}
\end{itemize}
Que tem majestade.
Respeitável; augusto.
Sumptuoso.
\section{Majoeira}
\begin{itemize}
\item {Grp. gram.:f.}
\end{itemize}
Rêde fluctuante, espécie de tresmalho.
(Por \textunderscore manjueira\textunderscore , de \textunderscore manjua\textunderscore )
\section{Majólica}
\begin{itemize}
\item {Grp. gram.:f.}
\end{itemize}
\begin{itemize}
\item {Proveniência:(De \textunderscore Maiorca\textunderscore , n. p. &lt;\textunderscore Majorca\textunderscore  &lt; \textunderscore Majolca\textunderscore  &lt; \textunderscore Majólica\textunderscore , que já era conhecida no b. lat.)}
\end{itemize}
Espécie de vaso antigo.
Nome, que se deu na Itália á loiça esmaltada, originária de Maiorca, nas Baleares.
\section{Majór}
\begin{itemize}
\item {Grp. gram.:m.}
\end{itemize}
\begin{itemize}
\item {Utilização:Gír.}
\end{itemize}
\begin{itemize}
\item {Grp. gram.:Adj.}
\end{itemize}
\begin{itemize}
\item {Utilização:T. de Ceilão}
\end{itemize}
Pôsto militar superior a capitão e inferior a tenente-coronel.
Indivíduo, que tem êsse posto.
Pai.
O mesmo que \textunderscore maiór\textunderscore .
\section{Majoração}
\begin{itemize}
\item {Grp. gram.:f.}
\end{itemize}
Acto ou effeito de majorar.
\section{Majorar}
\begin{itemize}
\item {Grp. gram.:v. t.}
\end{itemize}
\begin{itemize}
\item {Utilização:bras}
\end{itemize}
\begin{itemize}
\item {Utilização:Neol.}
\end{itemize}
\begin{itemize}
\item {Proveniência:(Do lat. \textunderscore major\textunderscore )}
\end{itemize}
O mesmo que \textunderscore aumentar\textunderscore ^1.
\section{Majoria}
\begin{itemize}
\item {Grp. gram.:f.}
\end{itemize}
Cargo ou dignidade de majór.
\section{Majovo}
\begin{itemize}
\item {Grp. gram.:m.}
\end{itemize}
(V.manjovo)
\section{Majurunas}
\begin{itemize}
\item {Grp. gram.:m. pl.}
\end{itemize}
Índios do Brasil, nas cabeceiras do Jabari, affluente do Amazonas.
\section{Majus}
\begin{itemize}
\item {Grp. gram.:m. pl.}
\end{itemize}
Designação genérica dos piratas, que enfestavam as costas da península hispanica.
\section{Majuscula}
\begin{itemize}
\item {Grp. gram.:f.}
\end{itemize}
(V.maiúscula)
\section{Majúsculo}
\begin{itemize}
\item {Grp. gram.:adj.}
\end{itemize}
(V.maiúsculo)
\section{Mal}
\begin{itemize}
\item {Grp. gram.:m.}
\end{itemize}
\begin{itemize}
\item {Proveniência:(Lat. \textunderscore malum\textunderscore )}
\end{itemize}
Aquillo que prejudica ou fere.
Aquelle que se oppõe ao bem, á virtude, á probidade, á honra.
Moléstia, achaque: \textunderscore soffrer mal de pelle\textunderscore .
Dôr.
Epidemia.
Calamidade; desgraça.
Prejuízo: \textunderscore fazer mal a alguém\textunderscore .
Mágoa.
O que se diz contra alguém: \textunderscore falam mal de ti\textunderscore .
Inconveniente: \textunderscore não acho nisso mal\textunderscore .
\section{Mal}
\begin{itemize}
\item {Grp. gram.:adv.}
\end{itemize}
\begin{itemize}
\item {Proveniência:(Lat. \textunderscore male\textunderscore )}
\end{itemize}
De modo irregular ou differente do que devia sêr: \textunderscore trabalhar mal\textunderscore .
A custo: \textunderscore mal podia andar\textunderscore .
Pouco, escassamente.
Severamente, com rudez: \textunderscore tratei-o mal\textunderscore .
Logo que: \textunderscore mal cheguei, fui lá\textunderscore .
Gravemente, mortalmente: \textunderscore saiu da refrega mal ferido\textunderscore .
Não.
\section{Mal...}
\begin{itemize}
\item {Grp. gram.:pref.}
\end{itemize}
\begin{itemize}
\item {Proveniência:(De \textunderscore mal\textunderscore ^2)}
\end{itemize}
(designativo de \textunderscore maldade\textunderscore , \textunderscore imperfeição\textunderscore , \textunderscore negação\textunderscore , \textunderscore desgraça\textunderscore , etc.)
\section{Mala}
\begin{itemize}
\item {Grp. gram.:f.}
\end{itemize}
\begin{itemize}
\item {Utilização:Chul.}
\end{itemize}
Saco de coiro ou pano, geralmente fechado com cadeado.
Caixa de madeira, revestida de coiro, lona, etc., destinada ordinariamente para transporte de fato e de outros objectos, em viagem.
O mesmo que \textunderscore estômago\textunderscore .
(B. lat. \textunderscore mala\textunderscore )
\section{Malabar}
\begin{itemize}
\item {Grp. gram.:adj.}
\end{itemize}
\begin{itemize}
\item {Grp. gram.:M.}
\end{itemize}
Relativo á costa occidental da Índia, ou aos habitantes dessa costa.
Habitante do Malabar.
Língua, falada na costa do Malabar e ao norte de Ceilão.
\section{Malabárico}
\begin{itemize}
\item {Grp. gram.:adj.}
\end{itemize}
\begin{itemize}
\item {Grp. gram.:M.}
\end{itemize}
\begin{itemize}
\item {Proveniência:(De \textunderscore Malabar\textunderscore , n. p.)}
\end{itemize}
Relativo ao Malabar.
Língua, falada no Malabar.
\section{Malabarista}
\begin{itemize}
\item {Grp. gram.:m.}
\end{itemize}
\begin{itemize}
\item {Proveniência:(De \textunderscore malabar\textunderscore )}
\end{itemize}
Aquelle que faz jogos malabares e a que os Franceses chamam \textunderscore jongleur\textunderscore .
\section{Malabruto}
\begin{itemize}
\item {Grp. gram.:m.}
\end{itemize}
\begin{itemize}
\item {Utilização:Prov.}
\end{itemize}
\begin{itemize}
\item {Utilização:trasm.}
\end{itemize}
Homem abrutado, labrego, sem tino.
\section{Malaca}
\begin{itemize}
\item {Grp. gram.:f.}
\end{itemize}
\begin{itemize}
\item {Utilização:Bras}
\end{itemize}
Designação genérica de qualquer moléstia.
(Cp. \textunderscore malácia\textunderscore ^1)
\section{Malacacheta}
\begin{itemize}
\item {fónica:chê}
\end{itemize}
\begin{itemize}
\item {Grp. gram.:f.}
\end{itemize}
O mesmo que \textunderscore mica\textunderscore ^1.
\section{Malacafento}
\begin{itemize}
\item {Grp. gram.:adj.}
\end{itemize}
\begin{itemize}
\item {Utilização:Bras}
\end{itemize}
\begin{itemize}
\item {Proveniência:(De \textunderscore malaca\textunderscore )}
\end{itemize}
Adoentado.
\section{Malacara}
\begin{itemize}
\item {Grp. gram.:adj.}
\end{itemize}
\begin{itemize}
\item {Grp. gram.:M.}
\end{itemize}
\begin{itemize}
\item {Utilização:Prov.}
\end{itemize}
\begin{itemize}
\item {Utilização:alent.}
\end{itemize}
\begin{itemize}
\item {Proveniência:(De \textunderscore malus\textunderscore  lat. + \textunderscore cara\textunderscore )}
\end{itemize}
Diz-se do cavallo que tem malha branca, desde a parte anterior da cabeça até ao peito.
Homem de má catadura.
\section{Malacaro}
\begin{itemize}
\item {Grp. gram.:adj.}
\end{itemize}
O mesmo que \textunderscore malacara\textunderscore .
\section{Malacate}
\begin{itemize}
\item {Grp. gram.:m.}
\end{itemize}
Máquina receptora, destinada a utilizar o trabalho muscular dos animaes, transformando-o em movimento circular ou contínuo, directamente applicável aos apparelhos.
Cp. \textunderscore manejo\textunderscore .
(Cast. \textunderscore malacate\textunderscore , sarilho)
\section{Malacênico}
\begin{itemize}
\item {Grp. gram.:adj.}
\end{itemize}
\begin{itemize}
\item {Utilização:Geol.}
\end{itemize}
Diz-se de um dos terrenos da série cenozoica.
\section{Malachadilho}
\begin{itemize}
\item {Grp. gram.:m.}
\end{itemize}
\begin{itemize}
\item {Utilização:Ant.}
\end{itemize}
Espécie de jôgo, em que se empregavam peças semelhantes a pequenas campaínhas:«\textunderscore Vinte e nove peças de prata de lei do reino á maneira de campaínhas, que chamam jôgo de malachadilho, que todas pesam cinco onças e cinco oitavas, com uma bolsa de velludo cramezim em que se mete\textunderscore ». \textunderscore Doc.\textunderscore  da chancell. de D. João III.
\section{Malachite}
\begin{itemize}
\item {fónica:qui}
\end{itemize}
\begin{itemize}
\item {Grp. gram.:f.}
\end{itemize}
\begin{itemize}
\item {Proveniência:(Gr. \textunderscore malakhites\textunderscore )}
\end{itemize}
Pedra preciosa, de côr um pouco semelhante á da malva, e que a natureza apresenta em estalactites sólidas ou opacas.
Espécie de carbonato de cobre natural.
\section{Malácia}
\begin{itemize}
\item {Grp. gram.:f.}
\end{itemize}
\begin{itemize}
\item {Utilização:Fig.}
\end{itemize}
\begin{itemize}
\item {Proveniência:(Do gr. \textunderscore malakia\textunderscore )}
\end{itemize}
Calmaria.
Debilidade; desalento.
Perversão de appetite.
\section{Malácia}
\begin{itemize}
\item {Grp. gram.:f.}
\end{itemize}
\begin{itemize}
\item {Proveniência:(Do gr. \textunderscore malakos\textunderscore )}
\end{itemize}
Gênero de insectos coleópteros pentâmeros.
\section{Malácio}
\begin{itemize}
\item {Grp. gram.:adj.}
\end{itemize}
\begin{itemize}
\item {Grp. gram.:Pl.}
\end{itemize}
Relativo ou semelhante á malácia^2.
Tríbo de insectos, que tem por typo a malácia^2.
\section{Malacoderme}
\begin{itemize}
\item {Grp. gram.:adj.}
\end{itemize}
\begin{itemize}
\item {Utilização:Zool.}
\end{itemize}
\begin{itemize}
\item {Grp. gram.:M.}
\end{itemize}
\begin{itemize}
\item {Proveniência:(Do gr. \textunderscore malakos\textunderscore  + \textunderscore derma\textunderscore )}
\end{itemize}
Que tem pelle molle.
Tríbo de insectos coleópteros.
\section{Malacologia}
\begin{itemize}
\item {Grp. gram.:f.}
\end{itemize}
\begin{itemize}
\item {Proveniência:(Do gr. \textunderscore malakos\textunderscore  + \textunderscore logos\textunderscore )}
\end{itemize}
Tratado á cêrca dos molluscos ou animaes de corpo molle.
\section{Malaconite}
\begin{itemize}
\item {Grp. gram.:f.}
\end{itemize}
Óxydo de cobre natural.
\section{Malacopterigiano}
\begin{itemize}
\item {Grp. gram.:adj.}
\end{itemize}
O mesmo que \textunderscore malacopterígio\textunderscore .
\section{Malacopterígio}
\begin{itemize}
\item {Grp. gram.:M. pl.}
\end{itemize}
\begin{itemize}
\item {Proveniência:(Do gr. \textunderscore malakos\textunderscore  + \textunderscore pterugion\textunderscore )}
\end{itemize}
Grupo de peixes com esqueleto ósseo e barbatanas moles e flexíveis.
\section{Malacopterygiano}
\begin{itemize}
\item {Grp. gram.:adj.}
\end{itemize}
O mesmo que \textunderscore malacopterýgio\textunderscore .
\section{Malacopterýgio}
\begin{itemize}
\item {Grp. gram.:adj.}
\end{itemize}
\begin{itemize}
\item {Grp. gram.:M. pl.}
\end{itemize}
\begin{itemize}
\item {Proveniência:(Do gr. \textunderscore malakos\textunderscore  + \textunderscore pterugion\textunderscore )}
\end{itemize}
Que tem barbatanas molles.
Grupo de peixes com esqueleto ósseo e barbatanas molles e flexíveis.
\section{Malacosarcose}
\begin{itemize}
\item {fónica:sar}
\end{itemize}
\begin{itemize}
\item {Grp. gram.:f.}
\end{itemize}
\begin{itemize}
\item {Utilização:Med.}
\end{itemize}
\begin{itemize}
\item {Proveniência:(Do gr. \textunderscore malakos\textunderscore  + \textunderscore sarx\textunderscore )}
\end{itemize}
Estado de amollecimento no systema muscular.
\section{Malacosoma}
\begin{itemize}
\item {fónica:so}
\end{itemize}
\begin{itemize}
\item {Grp. gram.:f.}
\end{itemize}
\begin{itemize}
\item {Proveniência:(Do gr. \textunderscore malakos\textunderscore  + \textunderscore soma\textunderscore )}
\end{itemize}
Insecto, nocivo ás vinhas.
Doença das videiras, produzida por elle.
\section{Malacosomo}
\begin{itemize}
\item {fónica:so}
\end{itemize}
\begin{itemize}
\item {Grp. gram.:adj.}
\end{itemize}
\begin{itemize}
\item {Utilização:Zool.}
\end{itemize}
\begin{itemize}
\item {Proveniência:(Do gr. \textunderscore malakos\textunderscore  + \textunderscore soma\textunderscore )}
\end{itemize}
Que tem corpo molle.
\section{Malacossarcose}
\begin{itemize}
\item {Grp. gram.:f.}
\end{itemize}
\begin{itemize}
\item {Utilização:Med.}
\end{itemize}
\begin{itemize}
\item {Proveniência:(Do gr. \textunderscore malakos\textunderscore  + \textunderscore sarx\textunderscore )}
\end{itemize}
Estado de amolecimento no sistema muscular.
\section{Malacossoma}
\begin{itemize}
\item {Grp. gram.:f.}
\end{itemize}
\begin{itemize}
\item {Proveniência:(Do gr. \textunderscore malakos\textunderscore  + \textunderscore soma\textunderscore )}
\end{itemize}
Insecto, nocivo ás vinhas.
Doença das videiras, produzida por ele.
\section{Malacossomo}
\begin{itemize}
\item {Grp. gram.:adj.}
\end{itemize}
\begin{itemize}
\item {Utilização:Zool.}
\end{itemize}
\begin{itemize}
\item {Proveniência:(Do gr. \textunderscore malakos\textunderscore  + \textunderscore soma\textunderscore )}
\end{itemize}
Que tem corpo mole.
\section{Malacosteóse}
\begin{itemize}
\item {Grp. gram.:f.}
\end{itemize}
\begin{itemize}
\item {Utilização:Med.}
\end{itemize}
\begin{itemize}
\item {Proveniência:(Do gr. \textunderscore malakos\textunderscore  + \textunderscore osteon\textunderscore )}
\end{itemize}
Amollecimento dos ossos.
\section{Malacozoário}
\begin{itemize}
\item {Grp. gram.:adj.}
\end{itemize}
\begin{itemize}
\item {Utilização:Zool.}
\end{itemize}
\begin{itemize}
\item {Grp. gram.:M. pl.}
\end{itemize}
\begin{itemize}
\item {Proveniência:(Do gr. \textunderscore malakos\textunderscore  + \textunderscore zoarion\textunderscore )}
\end{itemize}
Que não tem membros e cuja pelle é molle e toda contráctil.
Nome, dado por Blainville aos molluscos.
\section{Maláctico}
\begin{itemize}
\item {Grp. gram.:adj.}
\end{itemize}
\begin{itemize}
\item {Utilização:Med.}
\end{itemize}
\begin{itemize}
\item {Utilização:ant.}
\end{itemize}
\begin{itemize}
\item {Proveniência:(Gr. \textunderscore malaktikos\textunderscore )}
\end{itemize}
O mesmo que \textunderscore emolliente\textunderscore .
\section{Malacueco}
\begin{itemize}
\item {Grp. gram.:adj.}
\end{itemize}
\begin{itemize}
\item {Utilização:Fam.}
\end{itemize}
Finório, espertalhão. Cf. Filinto, I, 63.
\section{Malada}
\begin{itemize}
\item {Grp. gram.:f.}
\end{itemize}
\begin{itemize}
\item {Utilização:Ant.}
\end{itemize}
\begin{itemize}
\item {Proveniência:(De \textunderscore malado\textunderscore )}
\end{itemize}
Criada de servir.
\section{Maladia}
\begin{itemize}
\item {Grp. gram.:f.}
\end{itemize}
\begin{itemize}
\item {Utilização:Ant.}
\end{itemize}
\begin{itemize}
\item {Proveniência:(Do rad. do germ. \textunderscore maal\textunderscore )}
\end{itemize}
Solar, terra habitada por vassallos solarengos, sujeitos a encargos feudaes. Cf. Herculano, \textunderscore Hist. de Port.\textunderscore , IV, 480.
\section{Maladia}
\begin{itemize}
\item {Grp. gram.:f.}
\end{itemize}
\begin{itemize}
\item {Utilização:Des.}
\end{itemize}
O mesmo que \textunderscore doença\textunderscore .
(Cp. it. \textunderscore malattia\textunderscore , do lat. \textunderscore malatus\textunderscore )
\section{Maladio}
\begin{itemize}
\item {Grp. gram.:m.  e  adj.}
\end{itemize}
\begin{itemize}
\item {Utilização:Ant.}
\end{itemize}
\begin{itemize}
\item {Proveniência:(Do rad. de \textunderscore maladia\textunderscore ^1)}
\end{itemize}
O que habitava uma maladia e tinha o foro de cavalleiro.
\section{Malado}
\begin{itemize}
\item {Grp. gram.:m.}
\end{itemize}
\begin{itemize}
\item {Utilização:Ant.}
\end{itemize}
\begin{itemize}
\item {Proveniência:(Do rad. de \textunderscore maladia\textunderscore ^2)}
\end{itemize}
Morador em maladia.
\section{Malafaia}
\begin{itemize}
\item {Grp. gram.:m.}
\end{itemize}
\begin{itemize}
\item {Utilização:Gír.}
\end{itemize}
\begin{itemize}
\item {Utilização:Prov.}
\end{itemize}
Melcatrefe.
Aguardente, de fraquíssima graduação, que, no fim das lambicadas de bôa aguardente, se tira para se refinar ou para se misturar com o cadraço.
\section{Malafortunado}
\begin{itemize}
\item {Grp. gram.:adj.}
\end{itemize}
\begin{itemize}
\item {Proveniência:(De \textunderscore mal...\textunderscore  + \textunderscore afortunado\textunderscore )}
\end{itemize}
Não afortunado; desditoso.
\section{Málaga}
\begin{itemize}
\item {Grp. gram.:m.}
\end{itemize}
Vinho procedente de Málaga.
\section{Malagma}
\begin{itemize}
\item {Grp. gram.:m.}
\end{itemize}
\begin{itemize}
\item {Proveniência:(Gr. \textunderscore malagma\textunderscore )}
\end{itemize}
Medicamento tópico, para amollecer os tecidos.
O mesmo que \textunderscore cataplasma\textunderscore .
\section{Malagradecido}
\begin{itemize}
\item {Grp. gram.:adj.}
\end{itemize}
Que não agradece (favores recebidos). Cf. Rui Barb., \textunderscore Réplica\textunderscore , 158.
\section{Malagueiro}
\begin{itemize}
\item {Grp. gram.:m.}
\end{itemize}
\begin{itemize}
\item {Utilização:Ant.}
\end{itemize}
\begin{itemize}
\item {Proveniência:(De \textunderscore Málaga\textunderscore , n. p.?)}
\end{itemize}
O mesmo que \textunderscore fanqueiro\textunderscore .
\section{Malagueiro}
\begin{itemize}
\item {Grp. gram.:m.}
\end{itemize}
Nome de um vegetal, mencionado por Brotero.
O mesmo que \textunderscore lamegueiro\textunderscore ?
\section{Malaguenha}
\begin{itemize}
\item {Grp. gram.:f.}
\end{itemize}
\begin{itemize}
\item {Proveniência:(De \textunderscore malaguenho\textunderscore )}
\end{itemize}
Espécie de canção e música espanhola.
\section{Malaguenho}
\begin{itemize}
\item {Grp. gram.:adj.}
\end{itemize}
\begin{itemize}
\item {Grp. gram.:M.}
\end{itemize}
Relativo a Málaga.
Habitante de Málaga.
(Cast. \textunderscore malagueño\textunderscore )
\section{Malaguês}
\begin{itemize}
\item {Grp. gram.:m.  e  adj.}
\end{itemize}
O mesmo que \textunderscore malaguenho\textunderscore .
\section{Malagueta}
\begin{itemize}
\item {fónica:guê}
\end{itemize}
\begin{itemize}
\item {Grp. gram.:f.}
\end{itemize}
\begin{itemize}
\item {Proveniência:(De \textunderscore Málaga\textunderscore , n. p.)}
\end{itemize}
Espécie de pimenta muito ardente.
Cavilha, que se enfia nos fusos das mesas da amurada de um navio, e dá volta aos cabos de laborar.
Cada um dos raios salientes da roda do leme.
\section{Malaia}
\begin{itemize}
\item {Grp. gram.:f.}
\end{itemize}
\begin{itemize}
\item {Utilização:Ant.}
\end{itemize}
Espécie de saia?:«\textunderscore hũa malaya de pacotilho...\textunderscore »(De um testamento de 1692)
\section{Malaiala}
\begin{itemize}
\item {Grp. gram.:m.}
\end{itemize}
Língua do grupo decânico.
\section{Malaialim}
\begin{itemize}
\item {Grp. gram.:m.}
\end{itemize}
O mesmo que \textunderscore malaiala\textunderscore .
\section{Malaico}
\begin{itemize}
\item {Grp. gram.:m.}
\end{itemize}
Língua dos Malaios.
\section{Malaio}
\begin{itemize}
\item {Grp. gram.:adj.}
\end{itemize}
\begin{itemize}
\item {Grp. gram.:M.}
\end{itemize}
\begin{itemize}
\item {Grp. gram.:Pl.}
\end{itemize}
Relativo á Malásia ou aos Malaios.
Língua dos Malaios.
Uma das grandes raças humanas, que povoam Malaca, Samatra, e a parte da Oceânia, que delles tomou o nome de \textunderscore Malásia\textunderscore .
\section{Malaleuca}
\begin{itemize}
\item {Grp. gram.:f.}
\end{itemize}
Planta myrtácea do Brasil.
\section{Malalis}
\begin{itemize}
\item {Grp. gram.:m. pl.}
\end{itemize}
Tríbos nômadas do Brasil, na cordilheira que fica entre a Baía e Minas-Geraes.
\section{Malamanhado}
\begin{itemize}
\item {Grp. gram.:adj.}
\end{itemize}
\begin{itemize}
\item {Proveniência:(De \textunderscore mal...\textunderscore  + \textunderscore amanhado\textunderscore )}
\end{itemize}
Mal arranjado; mal vestido; desajeitado.
\section{Malambo}
\begin{itemize}
\item {Grp. gram.:m.}
\end{itemize}
Casca febrífuga, procedente da Colômbia.
\section{Malamente}
\begin{itemize}
\item {Grp. gram.:adj.}
\end{itemize}
\begin{itemize}
\item {Utilização:Ant.}
\end{itemize}
O mesmo que \textunderscore mal\textunderscore ^2.
\section{Malampansa}
\begin{itemize}
\item {Grp. gram.:f.}
\end{itemize}
\begin{itemize}
\item {Utilização:Bras. do Rio}
\end{itemize}
O mesmo que \textunderscore manampansa\textunderscore .
\section{Malanca}
\begin{itemize}
\item {Grp. gram.:f.}
\end{itemize}
Espécie de antílope angolense, (\textunderscore hippotragus equinus\textunderscore ).
\section{Malanda}
\begin{itemize}
\item {Grp. gram.:f.}
\end{itemize}
Representação da divindade feminina, no Congo.
\section{Malandante}
\begin{itemize}
\item {Grp. gram.:adj.}
\end{itemize}
\begin{itemize}
\item {Utilização:P. us.}
\end{itemize}
\begin{itemize}
\item {Utilização:Prov.}
\end{itemize}
\begin{itemize}
\item {Utilização:trasm.}
\end{itemize}
\begin{itemize}
\item {Proveniência:(De \textunderscore mal...\textunderscore  + \textunderscore andante\textunderscore )}
\end{itemize}
O mesmo que \textunderscore infeliz\textunderscore .
Mal comportado.
\section{Malandéu}
\begin{itemize}
\item {Grp. gram.:m.}
\end{itemize}
\begin{itemize}
\item {Utilização:Bras}
\end{itemize}
O mesmo que \textunderscore malandro\textunderscore .
(Por \textunderscore malandréu\textunderscore , de \textunderscore malandro\textunderscore )
\section{Malandra}
\begin{itemize}
\item {Grp. gram.:f.}
\end{itemize}
Mulher de infima extracção; vagabunda. Cf. Castilho, \textunderscore Fastos\textunderscore , 333.
(Cp. \textunderscore malandro\textunderscore )
\section{Malandra}
\begin{itemize}
\item {Grp. gram.:f.}
\end{itemize}
Arestim, nas juntas dos joêlhos das cavalgaduras.
(Cp. \textunderscore malandres\textunderscore )
\section{Malandraço}
\begin{itemize}
\item {Grp. gram.:m.}
\end{itemize}
Grande malandro.
\section{Malandragem}
\begin{itemize}
\item {Grp. gram.:f.}
\end{itemize}
Malandrice.
Conjunto de malandros.
\section{Malandrar}
\begin{itemize}
\item {Grp. gram.:v. i.}
\end{itemize}
Têr vida de malandro.
\section{Malandres}
\begin{itemize}
\item {Grp. gram.:m. pl.}
\end{itemize}
\begin{itemize}
\item {Proveniência:(Do lat. \textunderscore malandria\textunderscore )}
\end{itemize}
Ferimento transversal na prega do joêlho de uma cavalgadura.
\section{Malandrete}
\begin{itemize}
\item {fónica:drê}
\end{itemize}
\begin{itemize}
\item {Grp. gram.:m.}
\end{itemize}
O mesmo que \textunderscore malandrim\textunderscore .
\section{Malandrice}
\begin{itemize}
\item {Grp. gram.:f.}
\end{itemize}
Qualidade de quem é malandro.
Acto próprio de malandro.
Vida de malandro.
\section{Malandrim}
\begin{itemize}
\item {Grp. gram.:m.}
\end{itemize}
\begin{itemize}
\item {Utilização:Pop.}
\end{itemize}
\begin{itemize}
\item {Proveniência:(It. \textunderscore malandrino\textunderscore )}
\end{itemize}
Vadio.
Gatuno.
\section{Malandrinice}
\begin{itemize}
\item {Grp. gram.:f.}
\end{itemize}
\begin{itemize}
\item {Proveniência:(De \textunderscore malandrino\textunderscore )}
\end{itemize}
O mesmo que \textunderscore malandrice\textunderscore . Cf. Garrett, \textunderscore D. Branca\textunderscore , 61.
\section{Malandrino}
\begin{itemize}
\item {Grp. gram.:adj.}
\end{itemize}
\begin{itemize}
\item {Utilização:Pleb.}
\end{itemize}
\begin{itemize}
\item {Grp. gram.:M.}
\end{itemize}
\begin{itemize}
\item {Proveniência:(It. \textunderscore malandrino\textunderscore )}
\end{itemize}
Que tem modos ou hábitos de malandrim.
Relativo a malandrim.
Malandrim.
\section{Malandro}
\begin{itemize}
\item {Grp. gram.:m.}
\end{itemize}
\begin{itemize}
\item {Utilização:Pleb.}
\end{itemize}
\begin{itemize}
\item {Grp. gram.:Adj.}
\end{itemize}
\begin{itemize}
\item {Utilização:Prov.}
\end{itemize}
\begin{itemize}
\item {Utilização:alg.}
\end{itemize}
Vadio.
Gatuno.
Preguiçoso.
(Der. regressiva de \textunderscore malandrino\textunderscore , como \textunderscore rosmano\textunderscore  de \textunderscore rosmaninho\textunderscore )
\section{Malanemia}
\begin{itemize}
\item {Grp. gram.:f.}
\end{itemize}
Nome, que alguns africanistas dão á doença do somno.
\section{Malapeira}
\begin{itemize}
\item {Grp. gram.:f.}
\end{itemize}
\begin{itemize}
\item {Utilização:Prov.}
\end{itemize}
Macieira, que dá malápios.
\section{Malápio}
\begin{itemize}
\item {Grp. gram.:m.}
\end{itemize}
(corr. de \textunderscore melápio\textunderscore )
\section{Malápio}
\begin{itemize}
\item {Grp. gram.:adj.}
\end{itemize}
\begin{itemize}
\item {Utilização:Prov.}
\end{itemize}
\begin{itemize}
\item {Utilização:trasm.}
\end{itemize}
Manhoso.
Mandrião.
Larápio.
\section{Mala-posta}
\begin{itemize}
\item {Grp. gram.:f.}
\end{itemize}
Carro, para transporte de malas postaes.
Carro ou diligência que, além das malas postaes, transporta passageiros.
\section{Malaquês}
\begin{itemize}
\item {Grp. gram.:m.}
\end{itemize}
Moéda de prata, mandada cunhar em Malaca por Affonso de Albuquerque. Cf. Barros, \textunderscore Déc.\textunderscore  II, l. XI, c. 6.
\section{Malaqueta}
\begin{itemize}
\item {fónica:quê}
\end{itemize}
\begin{itemize}
\item {Grp. gram.:f.}
\end{itemize}
\begin{itemize}
\item {Utilização:Ant.}
\end{itemize}
Cavilha náutica, o mesmo que \textunderscore malagueta\textunderscore .
\section{Malaquista}
\begin{itemize}
\item {Grp. gram.:m.}
\end{itemize}
Habitante de Malaca.
\section{Malaquite}
\begin{itemize}
\item {Grp. gram.:f.}
\end{itemize}
\begin{itemize}
\item {Proveniência:(Gr. \textunderscore malakhites\textunderscore )}
\end{itemize}
Pedra preciosa, de côr um pouco semelhante á da malva, e que a natureza apresenta em estalactites sólidas ou opacas.
Espécie de carbonato de cobre natural.
\section{Malar}
\begin{itemize}
\item {Grp. gram.:m.}
\end{itemize}
\begin{itemize}
\item {Grp. gram.:Adj.}
\end{itemize}
\begin{itemize}
\item {Proveniência:(Do lat. \textunderscore mala\textunderscore )}
\end{itemize}
Osso, que constitue a parte proeminente da face.
Relativo ao osso malar ou á maçan do rosto.
\section{Malári}
\begin{itemize}
\item {Grp. gram.:m.}
\end{itemize}
Arbusto africano, herbáceo, rasteiro, de fôlhas sem estípulas, e flôres amareladas.
\section{Malária}
\begin{itemize}
\item {Grp. gram.:f.}
\end{itemize}
\begin{itemize}
\item {Proveniência:(T. it. de \textunderscore mala\textunderscore  + \textunderscore ária\textunderscore )}
\end{itemize}
Febre paludosa, intermittente e remittente.
É preferível o t. \textunderscore sezonismo\textunderscore .
\section{Malaricidade}
\begin{itemize}
\item {Grp. gram.:f.}
\end{itemize}
Qualidade de malárico. Cf. R. Jorge, \textunderscore Sezonismo\textunderscore , 15.
\section{Malárico}
\begin{itemize}
\item {Grp. gram.:adj.}
\end{itemize}
Relativo á malária.
\section{Malarífero}
\begin{itemize}
\item {Grp. gram.:adj.}
\end{itemize}
\begin{itemize}
\item {Proveniência:(De \textunderscore malária\textunderscore  + lat. \textunderscore ferre\textunderscore )}
\end{itemize}
Que contém o germe da malária:«\textunderscore águas malaríferas.\textunderscore »R. Jorge, \textunderscore Sezonismo\textunderscore , 12.
\section{Malarígeno}
\begin{itemize}
\item {Grp. gram.:adj.}
\end{itemize}
\begin{itemize}
\item {Proveniência:(De \textunderscore malária\textunderscore  + gr. \textunderscore geenos\textunderscore )}
\end{itemize}
Que produz malária. Cf. R. Jorge, \textunderscore Sezonismo\textunderscore , 13.
\section{Malarina}
\begin{itemize}
\item {Grp. gram.:f.}
\end{itemize}
Medicamento, contra as perturbações do systema nervoso.
\section{Mal-armado}
\begin{itemize}
\item {Grp. gram.:adj.}
\end{itemize}
Diz-se do toiro, que tem hastes defeituosas.
\section{Mal-arrumado}
\begin{itemize}
\item {Grp. gram.:m.}
\end{itemize}
\begin{itemize}
\item {Utilização:Bras}
\end{itemize}
Terreno, coberto de grandes pedaços de rocha, tornando diffícil o trânsito.
\section{Malasado}
\begin{itemize}
\item {Grp. gram.:adj.}
\end{itemize}
\begin{itemize}
\item {Proveniência:(De \textunderscore mal...\textunderscore  + \textunderscore asado\textunderscore )}
\end{itemize}
Desajeitado.
Extravagante:«\textunderscore ...a malasada figura de Pêro Cão.\textunderscore »Garrett, \textunderscore Arco de Sant'Anna\textunderscore , I, 81.
\section{Malasarte}
\begin{itemize}
\item {Grp. gram.:adj.}
\end{itemize}
\begin{itemize}
\item {Grp. gram.:M.}
\end{itemize}
O mesmo que \textunderscore malasado\textunderscore .
Indivíduo desajeitado.
(Naturalmente do nome lendário \textunderscore Pedro de Malas-Artes\textunderscore )
\section{Malascaras}
\begin{itemize}
\item {Grp. gram.:m.}
\end{itemize}
\begin{itemize}
\item {Utilização:Ant.}
\end{itemize}
\begin{itemize}
\item {Utilização:Pop.}
\end{itemize}
Homem carrancudo, trombudo, sorumbático.
(Cp. \textunderscore malacara\textunderscore )
\section{Malassada}
\begin{itemize}
\item {Grp. gram.:f.}
\end{itemize}
\begin{itemize}
\item {Proveniência:(De \textunderscore mal...\textunderscore  + \textunderscore assado\textunderscore )}
\end{itemize}
Ovos, que se batem e se fritam ao mesmo tempo.
\section{Malassombramento}
\begin{itemize}
\item {Grp. gram.:m.}
\end{itemize}
\begin{itemize}
\item {Utilização:Bras}
\end{itemize}
\begin{itemize}
\item {Proveniência:(De \textunderscore mal...\textunderscore  + \textunderscore assombrar\textunderscore )}
\end{itemize}
O mesmo que \textunderscore encantamento\textunderscore .
\section{Malata}
\begin{itemize}
\item {Grp. gram.:f.}
\end{itemize}
\begin{itemize}
\item {Utilização:Prov.}
\end{itemize}
Fêmea do \textunderscore malato\textunderscore ^1.
\section{Malateca}
\begin{itemize}
\item {Grp. gram.:f.}
\end{itemize}
\begin{itemize}
\item {Utilização:Prov.}
\end{itemize}
\begin{itemize}
\item {Utilização:alent.}
\end{itemize}
Pequena herdade.
\section{Malato}
\begin{itemize}
\item {Grp. gram.:m.}
\end{itemize}
\begin{itemize}
\item {Utilização:Prov.}
\end{itemize}
Cordeiro.
Carneiro de meia idade ou de um anno, pouco mais ou menos.
\section{Malato}
\begin{itemize}
\item {Grp. gram.:adj.}
\end{itemize}
\begin{itemize}
\item {Utilização:Ant.}
\end{itemize}
\begin{itemize}
\item {Proveniência:(Do lat. \textunderscore male\textunderscore  + \textunderscore habitus\textunderscore , segundo Cornu)}
\end{itemize}
Adoentado:«\textunderscore Dom Carlos diz que se acha hoje malato.\textunderscore »F. Manuel, \textunderscore Campanha de Euterpe\textunderscore , 116.
\section{Malatosta}
\begin{itemize}
\item {Grp. gram.:f.}
\end{itemize}
\begin{itemize}
\item {Utilização:Ant.}
\end{itemize}
O mesmo ou melhor que \textunderscore maltosta\textunderscore .
\section{Malavença}
\begin{itemize}
\item {Grp. gram.:f.}
\end{itemize}
\begin{itemize}
\item {Utilização:Des.}
\end{itemize}
\begin{itemize}
\item {Proveniência:(De \textunderscore mal...\textunderscore  + \textunderscore avença\textunderscore )}
\end{itemize}
O mesmo que \textunderscore desavença\textunderscore . Cf. Garrett, \textunderscore Camões\textunderscore , 16.
\section{Malaventura}
\begin{itemize}
\item {Grp. gram.:f.}
\end{itemize}
\begin{itemize}
\item {Utilização:P. us.}
\end{itemize}
\begin{itemize}
\item {Proveniência:(Do lat. \textunderscore malus\textunderscore  + \textunderscore venturus\textunderscore )}
\end{itemize}
Desventura.
\section{Malaventurado}
\begin{itemize}
\item {Grp. gram.:adj.}
\end{itemize}
O mesmo que \textunderscore malafortunado\textunderscore .
\section{Malavindo}
\begin{itemize}
\item {Grp. gram.:adj.}
\end{itemize}
\begin{itemize}
\item {Proveniência:(De \textunderscore mal...\textunderscore  + \textunderscore avindo\textunderscore )}
\end{itemize}
O mesmo que \textunderscore desavindo\textunderscore .
\section{Malavinhado}
\begin{itemize}
\item {Grp. gram.:adj.}
\end{itemize}
\begin{itemize}
\item {Utilização:Fig.}
\end{itemize}
\begin{itemize}
\item {Proveniência:(De \textunderscore mal...\textunderscore  + \textunderscore avinhado\textunderscore )}
\end{itemize}
Diz-se das vasilhas, que tornam azêdo o vinho ou que lhe communicam mau gôsto.
Desordeiro, quando bêbedo.
Pessimista.
\section{Malaxação}
\begin{itemize}
\item {fónica:csa}
\end{itemize}
\begin{itemize}
\item {Grp. gram.:f.}
\end{itemize}
\begin{itemize}
\item {Proveniência:(De \textunderscore malaxar\textunderscore )}
\end{itemize}
Espécie de maçagem, destinada a amaciar os tecidos.
\section{Malaxador}
\begin{itemize}
\item {fónica:csa}
\end{itemize}
\begin{itemize}
\item {Grp. gram.:m.}
\end{itemize}
\begin{itemize}
\item {Proveniência:(De \textunderscore malaxar\textunderscore )}
\end{itemize}
Apparelho que, para o fabríco do queijo ou da manteiga, dá consistência á nata, saída da batedeira, supprimindo os poros, como prejudiciaes á conservação daquelle alimento.
\section{Malaxar}
\begin{itemize}
\item {fónica:csar}
\end{itemize}
\begin{itemize}
\item {Grp. gram.:v. t.}
\end{itemize}
\begin{itemize}
\item {Proveniência:(Lat. \textunderscore malaxare\textunderscore )}
\end{itemize}
Amassar, para fazer emplastro.
Dar maçagem em.
Fatigar.
Mexer ou bater muito uma substância, para a tornar compacta: \textunderscore malaxar a manteiga\textunderscore .
\section{Malbaratador}
\begin{itemize}
\item {Grp. gram.:m.  e  adj.}
\end{itemize}
O que malbarata.
\section{Malbaratar}
\begin{itemize}
\item {Grp. gram.:v. t.}
\end{itemize}
\begin{itemize}
\item {Proveniência:(De \textunderscore mal...\textunderscore  + \textunderscore baratar\textunderscore )}
\end{itemize}
Vender com prejuízo.
Gastar mal; dissipar.
\section{Malbaratear}
\begin{itemize}
\item {Grp. gram.:v. t.}
\end{itemize}
\begin{itemize}
\item {Utilização:Bras}
\end{itemize}
O mesmo que \textunderscore malbaratar\textunderscore .
\section{Malbarato}
\begin{itemize}
\item {Grp. gram.:m.}
\end{itemize}
\begin{itemize}
\item {Proveniência:(De \textunderscore mal...\textunderscore  + \textunderscore barato\textunderscore )}
\end{itemize}
Venda que deixa prejuízo.
Depreciação.
Menosprêzo.
\section{Mal-caduco}
\begin{itemize}
\item {Grp. gram.:m.}
\end{itemize}
O mesmo que \textunderscore epilepsia\textunderscore .
\section{Malcasado}
\begin{itemize}
\item {Grp. gram.:adj.}
\end{itemize}
\begin{itemize}
\item {Grp. gram.:M.}
\end{itemize}
\begin{itemize}
\item {Utilização:Bras. de Sergipe}
\end{itemize}
\begin{itemize}
\item {Proveniência:(De \textunderscore mal...\textunderscore  + \textunderscore casado\textunderscore )}
\end{itemize}
Que vive mal com o seu consorte.
Que desposou pessôa de condição inferior.
Espécie de beiju, feito de tapioca e leite de côco.
\section{Malcassá}
\begin{itemize}
\item {Grp. gram.:m.}
\end{itemize}
\begin{itemize}
\item {Utilização:Bras}
\end{itemize}
O mesmo que \textunderscore malcasado\textunderscore , m.
\section{Malcatenho}
\begin{itemize}
\item {Grp. gram.:adj.}
\end{itemize}
Relativo á serra ou região de Malcata.
Diz-se especialmente de uma variedade de gado bovino, quo alli se cria. Cf. \textunderscore Portugal au point de vue agricole\textunderscore , 253.
\section{Malchacava}
\begin{itemize}
\item {Grp. gram.:adj. f.}
\end{itemize}
\begin{itemize}
\item {Utilização:Ant.}
\end{itemize}
(?):«\textunderscore ...enxerga malchacava\textunderscore ». \textunderscore Regimenlo dos panos, de 1590\textunderscore , ms.
\section{Malcheirante}
\begin{itemize}
\item {Grp. gram.:adj.}
\end{itemize}
\begin{itemize}
\item {Utilização:Bras}
\end{itemize}
O mesmo que \textunderscore malcheiroso\textunderscore .
\section{Malcheiroso}
\begin{itemize}
\item {Grp. gram.:adj.}
\end{itemize}
\begin{itemize}
\item {Proveniência:(De \textunderscore mal...\textunderscore  + \textunderscore cheiroso\textunderscore )}
\end{itemize}
Que cheira mal; fedorento.
\section{Malcómia}
\begin{itemize}
\item {Grp. gram.:f.}
\end{itemize}
\begin{itemize}
\item {Proveniência:(De \textunderscore Malcom\textunderscore , n. p.)}
\end{itemize}
Gênero de plantas crucíferas.
\section{Malcomido}
\begin{itemize}
\item {Grp. gram.:adj.}
\end{itemize}
\begin{itemize}
\item {Proveniência:(De \textunderscore mal...\textunderscore  + \textunderscore comido\textunderscore )}
\end{itemize}
Que se alimenta mal; magro, por insufficiência de alimento.
\section{Malconfiar-se}
\begin{itemize}
\item {Grp. gram.:v. p.}
\end{itemize}
Não têr confiança inteira:«\textunderscore e o pai, que se tinha malconfiado na Providência...\textunderscore »Castilho.
\section{Malcontentadiço}
\begin{itemize}
\item {Grp. gram.:adj.}
\end{itemize}
\begin{itemize}
\item {Proveniência:(De \textunderscore mal...\textunderscore  + \textunderscore contentar\textunderscore )}
\end{itemize}
Que se contenta difficilmente.
\section{Malcontente}
\begin{itemize}
\item {Grp. gram.:adj.}
\end{itemize}
\begin{itemize}
\item {Proveniência:(De \textunderscore mal...\textunderscore  + \textunderscore contente\textunderscore )}
\end{itemize}
O mesmo que \textunderscore descontente\textunderscore .
\section{Mal-scorreito}
\begin{itemize}
\item {Grp. gram.:adj.}
\end{itemize}
\begin{itemize}
\item {Utilização:Ant.}
\end{itemize}
Adoentado; achacado.
(Cp. \textunderscore escorreito\textunderscore )
\section{Malcorrente}
\begin{itemize}
\item {Grp. gram.:adj.}
\end{itemize}
\begin{itemize}
\item {Proveniência:(De \textunderscore mal...\textunderscore  + \textunderscore corrente\textunderscore )}
\end{itemize}
Que tem pouca destreza.
Que está amuado, escandalizado.
\section{Malcozer}
\begin{itemize}
\item {Grp. gram.:v. t.}
\end{itemize}
\begin{itemize}
\item {Grp. gram.:V. i.}
\end{itemize}
\begin{itemize}
\item {Proveniência:(De \textunderscore mal...\textunderscore  + \textunderscore cozer\textunderscore )}
\end{itemize}
Cozer mal.
Estar mal cozido.
\section{Malcozinhado}
\begin{itemize}
\item {Grp. gram.:m.}
\end{itemize}
\begin{itemize}
\item {Utilização:Fig.}
\end{itemize}
\begin{itemize}
\item {Grp. gram.:M.}
\end{itemize}
\begin{itemize}
\item {Utilização:Des.}
\end{itemize}
\begin{itemize}
\item {Proveniência:(De \textunderscore mal...\textunderscore  + \textunderscore cozinhado\textunderscore )}
\end{itemize}
Diz-se do comer mal arranjado.
Mal feito; mal dirigido: \textunderscore um drama malcozinhado\textunderscore .
Taberna.
Taberneiro, que frita iscas ou peixe.
Peixe frito ou iscas de taberna.
\section{Malcriadez}
\begin{itemize}
\item {Grp. gram.:f.}
\end{itemize}
\begin{itemize}
\item {Utilização:Pop.}
\end{itemize}
Qualidade de malcriado.
Acção ou dito, próprio de malcriado.
\section{Malcriado}
\begin{itemize}
\item {Grp. gram.:adj.}
\end{itemize}
\begin{itemize}
\item {Utilização:Des.}
\end{itemize}
\begin{itemize}
\item {Proveniência:(De \textunderscore mal...\textunderscore  + \textunderscore criado\textunderscore )}
\end{itemize}
Que não tem bôa educação; descortês.
Magro.
\section{Mal-da-ave-maria}
\begin{itemize}
\item {Grp. gram.:m.}
\end{itemize}
\begin{itemize}
\item {Utilização:Ant.}
\end{itemize}
O mesmo que \textunderscore paralysia\textunderscore .
\section{Maldade}
\begin{itemize}
\item {Grp. gram.:f.}
\end{itemize}
\begin{itemize}
\item {Utilização:Fam.}
\end{itemize}
\begin{itemize}
\item {Proveniência:(Lat. \textunderscore malitas\textunderscore , \textunderscore malitatis\textunderscore )}
\end{itemize}
Qualidade daquelle ou daquillo que é mau.
Acção ruím.
Iniquidade; crueldade.
Travessura; teimosia: \textunderscore maldades de criança\textunderscore .
\section{Mal-de-Bright}
\begin{itemize}
\item {fónica:brait'}
\end{itemize}
\begin{itemize}
\item {Grp. gram.:m.}
\end{itemize}
\begin{itemize}
\item {Utilização:Med.}
\end{itemize}
O mesmo que \textunderscore nephrite\textunderscore .
\section{Mal-de-escancha}
\begin{itemize}
\item {Grp. gram.:m.}
\end{itemize}
\begin{itemize}
\item {Utilização:Bras}
\end{itemize}
O mesmo que \textunderscore quebra-bunda\textunderscore .
\section{Mal-de-fígado}
\begin{itemize}
\item {Grp. gram.:m.}
\end{itemize}
\begin{itemize}
\item {Utilização:T. da Bairrada}
\end{itemize}
O mesmo que \textunderscore morpheia\textunderscore .
\section{Mal-de-fóra}
\begin{itemize}
\item {Grp. gram.:m.}
\end{itemize}
\begin{itemize}
\item {Utilização:Prov.}
\end{itemize}
\begin{itemize}
\item {Utilização:minh.}
\end{itemize}
O mesmo que \textunderscore feitiço\textunderscore .
\section{Mal-de-loanda}
\begin{itemize}
\item {Grp. gram.:m.}
\end{itemize}
O mesmo que \textunderscore escorbuto\textunderscore .
\section{Mal-de-olanda}
\begin{itemize}
\item {Grp. gram.:m.}
\end{itemize}
\begin{itemize}
\item {Utilização:Ant.}
\end{itemize}
O mesmo que \textunderscore mal-de-loanda\textunderscore . Cf. B. Pereira, \textunderscore Prosódia\textunderscore , vb. \textunderscore paradontides\textunderscore .
\section{Mal-de-terra}
\begin{itemize}
\item {Grp. gram.:m.}
\end{itemize}
O mesmo que \textunderscore epilepsia\textunderscore . Cf. Macedo Pinto, \textunderscore Comp. de Viter.\textunderscore , I, 289.
\section{Mal-de-vaso}
\begin{itemize}
\item {Grp. gram.:m.}
\end{itemize}
\begin{itemize}
\item {Utilização:Bras. do S}
\end{itemize}
Ferida cancerosa na raíz do casco das cavalgaduras.
\section{Maldição}
\begin{itemize}
\item {Grp. gram.:f.}
\end{itemize}
\begin{itemize}
\item {Proveniência:(Do lat. \textunderscore maledictio\textunderscore )}
\end{itemize}
Acto ou efeito de amaldiçoar; praga.
\section{Maldicção}
\begin{itemize}
\item {Grp. gram.:f.}
\end{itemize}
\begin{itemize}
\item {Proveniência:(Do lat. \textunderscore maledictio\textunderscore )}
\end{itemize}
Acto ou effeito de amaldiçoar; praga.
\section{Maldiçoar}
\begin{itemize}
\item {Grp. gram.:v. t.}
\end{itemize}
O mesmo que \textunderscore amaldiçoar\textunderscore .
\section{Maldigno}
\begin{itemize}
\item {Grp. gram.:adj.}
\end{itemize}
\begin{itemize}
\item {Utilização:Des.}
\end{itemize}
\begin{itemize}
\item {Proveniência:(De \textunderscore mal...\textunderscore  + \textunderscore digno\textunderscore )}
\end{itemize}
O mesmo que \textunderscore indigno\textunderscore . Cf. Filinto, XIX, 266.
\section{Maldita}
\begin{itemize}
\item {Grp. gram.:f.}
\end{itemize}
\begin{itemize}
\item {Utilização:Pop.}
\end{itemize}
\begin{itemize}
\item {Proveniência:(De \textunderscore maldito\textunderscore )}
\end{itemize}
Impigem rebelde; pústula malígna.
\section{Maldito}
\begin{itemize}
\item {Grp. gram.:adj.}
\end{itemize}
\begin{itemize}
\item {Proveniência:(Do lat. \textunderscore maledictus\textunderscore )}
\end{itemize}
Sinistro; que exerce influência nefasta.
Que tem má índole.
Aborrecido; incômmodo.
\section{Malditoso}
\begin{itemize}
\item {Grp. gram.:adj.}
\end{itemize}
\begin{itemize}
\item {Utilização:P. us.}
\end{itemize}
\begin{itemize}
\item {Proveniência:(De \textunderscore mal...\textunderscore  + \textunderscore ditoso\textunderscore )}
\end{itemize}
Desditoso.
\section{Maldizente}
\begin{itemize}
\item {Grp. gram.:m. ,  f.  e  adj.}
\end{itemize}
\begin{itemize}
\item {Proveniência:(Do lat. \textunderscore maledicens\textunderscore )}
\end{itemize}
Pessôa, que tem má língua ou que fala mal dos outros; diffamador.
\section{Maldizer}
\begin{itemize}
\item {Grp. gram.:v. t.}
\end{itemize}
\begin{itemize}
\item {Grp. gram.:V. i.}
\end{itemize}
\begin{itemize}
\item {Grp. gram.:M.}
\end{itemize}
\begin{itemize}
\item {Proveniência:(Lat. \textunderscore maledicere\textunderscore )}
\end{itemize}
Dizer mal de.
Amaldiçoar; dirigir imprecações a: \textunderscore maldizer a hora em que nasceu\textunderscore .
Dizer mal.
Blasphemar.
Lastimar-se.
Maledicência.
\section{Mal-do-diagalves}
\begin{itemize}
\item {Grp. gram.:m.}
\end{itemize}
O mesmo que \textunderscore maromba\textunderscore ^2. Cf. \textunderscore Portugal au point de vue agr.\textunderscore , 618.
\section{Mal-do-monte}
\begin{itemize}
\item {Grp. gram.:m.}
\end{itemize}
O mesmo que \textunderscore erysipela\textunderscore .
\section{Maldoso}
\begin{itemize}
\item {Grp. gram.:adj.}
\end{itemize}
\begin{itemize}
\item {Utilização:Fig.}
\end{itemize}
Que tem maldade, que é de má índole.
Travesso; malicioso.
(Contr. de \textunderscore maldadoso\textunderscore , de \textunderscore maldade\textunderscore )
\section{Maleante}
\begin{itemize}
\item {Grp. gram.:m.  e  adj.}
\end{itemize}
\begin{itemize}
\item {Utilização:Ant.}
\end{itemize}
Trapaceiro; enganador.
(Cp. \textunderscore maleza\textunderscore )
\section{Malear}
\begin{itemize}
\item {Grp. gram.:v. i.}
\end{itemize}
\begin{itemize}
\item {Utilização:Prov.}
\end{itemize}
\begin{itemize}
\item {Utilização:alg.}
\end{itemize}
\begin{itemize}
\item {Utilização:alent.}
\end{itemize}
Abortar, (tratando-se de animaes); malparir.
(Cast. \textunderscore malear\textunderscore )
\section{Malebranchismo}
\begin{itemize}
\item {Grp. gram.:m.}
\end{itemize}
Doutrina idealista de Malebranche.
\section{Malebranchista}
\begin{itemize}
\item {Grp. gram.:m.}
\end{itemize}
Partidário da philosophia de Malebranche.
\section{Maledicência}
\begin{itemize}
\item {Grp. gram.:f.}
\end{itemize}
\begin{itemize}
\item {Proveniência:(Lat. \textunderscore maledicentia\textunderscore )}
\end{itemize}
Qualidade de quem é maldizente.
Acto de dizer mal; murmuração.
\section{Maledicente}
\begin{itemize}
\item {Grp. gram.:m. ,  f.  e  adj.}
\end{itemize}
\begin{itemize}
\item {Proveniência:(Lat. \textunderscore maledicens\textunderscore )}
\end{itemize}
O mesmo que \textunderscore maldizente\textunderscore .
\section{Malédico}
\begin{itemize}
\item {Grp. gram.:m.  e  adj.}
\end{itemize}
\begin{itemize}
\item {Proveniência:(Lat. \textunderscore maledicus\textunderscore )}
\end{itemize}
O mesmo que \textunderscore maldizente\textunderscore .
\section{Maledo}
\begin{itemize}
\item {fónica:lê}
\end{itemize}
\begin{itemize}
\item {Grp. gram.:m.}
\end{itemize}
Bacello, próprio para vinha de cepa ou de pé curto. Cf. \textunderscore Bibl. da G. do Campo\textunderscore , 323.
\section{Maleficamente}
\begin{itemize}
\item {Grp. gram.:adv.}
\end{itemize}
De modo maléfico.
\section{Maleficência}
\begin{itemize}
\item {Grp. gram.:f.}
\end{itemize}
\begin{itemize}
\item {Proveniência:(Lat. \textunderscore maleficentia\textunderscore )}
\end{itemize}
Qualidade daquelle ou daquillo que é maléfico.
Disposição para fazer mal.
\section{Maleficiar}
\begin{itemize}
\item {Grp. gram.:v. t.}
\end{itemize}
\begin{itemize}
\item {Proveniência:(De \textunderscore malefício\textunderscore )}
\end{itemize}
Fazer mal a.
Prejudicar; damnificar.
\section{Malefício}
\begin{itemize}
\item {Grp. gram.:m.}
\end{itemize}
\begin{itemize}
\item {Proveniência:(Lat. \textunderscore maleficium\textunderscore )}
\end{itemize}
Acto de maleficiar.
Sortilégio.
\section{Maléfico}
\begin{itemize}
\item {Grp. gram.:adj.}
\end{itemize}
\begin{itemize}
\item {Proveniência:(Lat. \textunderscore maleficus\textunderscore )}
\end{itemize}
Disposto para o mal; mal intencionado.
Malévolo; prejudicial.
\section{Maleico}
\begin{itemize}
\item {Grp. gram.:adj.}
\end{itemize}
\begin{itemize}
\item {Proveniência:(Do lat. \textunderscore malum\textunderscore )}
\end{itemize}
Diz-se de um ácido, proveniente da destillação do ácido málico.
\section{Maleiro}
\begin{itemize}
\item {Grp. gram.:m.}
\end{itemize}
Fabricante ou vendedor de malas.
\section{Maleita}
\begin{itemize}
\item {Grp. gram.:f.}
\end{itemize}
\begin{itemize}
\item {Proveniência:(Do lat. \textunderscore male acta\textunderscore . Cp. \textunderscore malato\textunderscore ^2)}
\end{itemize}
Febre intermittente, sezão.
\section{Maleiteira}
\begin{itemize}
\item {Grp. gram.:f.}
\end{itemize}
\begin{itemize}
\item {Proveniência:(De \textunderscore maleita\textunderscore )}
\end{itemize}
Planta euphorbiácea, (\textunderscore euphorbia papillosa\textunderscore ).
\section{Maleiteira-maior}
\begin{itemize}
\item {Grp. gram.:f.}
\end{itemize}
Variedade de maleiteira, (\textunderscore euphorbia characias\textunderscore , Lin.).
\section{Maleitoso}
\begin{itemize}
\item {Grp. gram.:adj.}
\end{itemize}
Que tem maleitas.
Que causa maleitas.
\section{Malembecele}
\begin{itemize}
\item {Grp. gram.:m.}
\end{itemize}
Planta africana, vivaz, de fôlhas oppostas, glabras, e flôres monopétalas infundibuliformes.
\section{Malembos}
\begin{itemize}
\item {Grp. gram.:m. pl.}
\end{itemize}
Cacongos, que habitam sôbre o Chiloango.
\section{Malemo}
\begin{itemize}
\item {Grp. gram.:m.}
\end{itemize}
\begin{itemize}
\item {Utilização:Ant.}
\end{itemize}
Designação antiga de piloto, nos mares da África oriental. Cf. Gaspar Correia, \textunderscore Lendas da Índia\textunderscore .
(Do ár.)
\section{Malencarado}
\begin{itemize}
\item {Grp. gram.:adj.}
\end{itemize}
\begin{itemize}
\item {Proveniência:(De \textunderscore mal...\textunderscore  + \textunderscore encarado\textunderscore )}
\end{itemize}
Que tem má cara; carrancudo.
Que revela má índole.
\section{Malencónico}
\begin{itemize}
\item {Grp. gram.:m.  e  adj.}
\end{itemize}
\begin{itemize}
\item {Utilização:Ant.}
\end{itemize}
O mesmo que \textunderscore melancólico\textunderscore . Cf. \textunderscore Eufrosina\textunderscore , 15 e 25.
\section{Malenconizar}
\textunderscore v. t.\textunderscore  (e der.)
Outra fórma de \textunderscore melancolizar\textunderscore , etc. Cf. \textunderscore Eufrosina\textunderscore ,31.
\section{Malensinado}
\begin{itemize}
\item {Grp. gram.:adj.}
\end{itemize}
\begin{itemize}
\item {Proveniência:(De \textunderscore mal...\textunderscore  + \textunderscore ensinado\textunderscore )}
\end{itemize}
O mesmo que \textunderscore malcriado\textunderscore .
\section{Malentendido}
\begin{itemize}
\item {Grp. gram.:adj.}
\end{itemize}
\begin{itemize}
\item {Grp. gram.:M.}
\end{itemize}
\begin{itemize}
\item {Utilização:Gal}
\end{itemize}
\begin{itemize}
\item {Proveniência:(De \textunderscore mal...\textunderscore  + \textunderscore entendido\textunderscore )}
\end{itemize}
Que entende mal.
Mal interpretado.
Equívoco.
\section{Malentrouxado}
\begin{itemize}
\item {Grp. gram.:adj.}
\end{itemize}
\begin{itemize}
\item {Proveniência:(De \textunderscore mal...\textunderscore  + \textunderscore entrouxado\textunderscore )}
\end{itemize}
Desleixado no vestuário.
\section{Malentrada}
\begin{itemize}
\item {Grp. gram.:f.}
\end{itemize}
\begin{itemize}
\item {Utilização:Ant.}
\end{itemize}
\begin{itemize}
\item {Proveniência:(De \textunderscore mal...\textunderscore  + \textunderscore entrar\textunderscore )}
\end{itemize}
Multa, que o preso pagava ao entrar na cadeia.
\section{Malesso}
\begin{itemize}
\item {Grp. gram.:adj.}
\end{itemize}
\begin{itemize}
\item {Utilização:Prov.}
\end{itemize}
\begin{itemize}
\item {Utilização:beir.}
\end{itemize}
Diz-se do toiro, que tem mau sangue.
Patife, biltre.
\section{Mal-estar}
\begin{itemize}
\item {Grp. gram.:m.}
\end{itemize}
Indisposição phýsica ou moral.
Desassossêgo.
Doença.
Situação incômmoda ou molesta.
Organização defeituosa.
Perturbação no espírito público.
\section{Malestreado}
\begin{itemize}
\item {Grp. gram.:adj.}
\end{itemize}
\begin{itemize}
\item {Proveniência:(De \textunderscore mal...\textunderscore  + \textunderscore estreado\textunderscore )}
\end{itemize}
Que se estreou mal.
Desastrado.
Malfadado.
Que tem má cara.
\section{Maleta}
\begin{itemize}
\item {fónica:lê}
\end{itemize}
\begin{itemize}
\item {Grp. gram.:f.}
\end{itemize}
Pequena mala.
\section{Maleta}
\begin{itemize}
\item {fónica:lê}
\end{itemize}
\begin{itemize}
\item {Grp. gram.:m.}
\end{itemize}
Toireiro sem mérito.
\section{Maleva}
\begin{itemize}
\item {Grp. gram.:f.}
\end{itemize}
\begin{itemize}
\item {Utilização:Ant.}
\end{itemize}
O mesmo que \textunderscore fiança\textunderscore .
\section{Malevolamente}
\begin{itemize}
\item {Grp. gram.:adv.}
\end{itemize}
De modo malévolo.
\section{Malevolência}
\begin{itemize}
\item {Grp. gram.:f.}
\end{itemize}
\begin{itemize}
\item {Proveniência:(Lat. \textunderscore malevolentia\textunderscore )}
\end{itemize}
Qualidade de quem é malevolente.
\section{Malevolente}
\begin{itemize}
\item {Grp. gram.:adj.}
\end{itemize}
\begin{itemize}
\item {Proveniência:(Lat. \textunderscore malevolens\textunderscore )}
\end{itemize}
O mesmo que \textunderscore malévolo\textunderscore .
\section{Malévolo}
\begin{itemize}
\item {Grp. gram.:adj.}
\end{itemize}
\begin{itemize}
\item {Proveniência:(Lat. \textunderscore malevolus\textunderscore )}
\end{itemize}
Que tem má vontade contra alguém.
Que tem má índole; maléfico.
\section{Maleza}
\begin{itemize}
\item {Grp. gram.:f.}
\end{itemize}
\begin{itemize}
\item {Utilização:Ant.}
\end{itemize}
\begin{itemize}
\item {Proveniência:(Do lat. \textunderscore malitia\textunderscore )}
\end{itemize}
Abundância de ervas nocivas ás sementeiras.
Maldade; mal, desgraça.
\section{Malfadadamente}
\begin{itemize}
\item {Grp. gram.:adv.}
\end{itemize}
De modo malfadado; desgraçadamente.
\section{Malfadado}
\begin{itemize}
\item {Grp. gram.:adj.}
\end{itemize}
\begin{itemize}
\item {Grp. gram.:M.}
\end{itemize}
\begin{itemize}
\item {Proveniência:(De \textunderscore malfadar\textunderscore )}
\end{itemize}
Que tem mau fado ou má sorte.
Desgraçado.
Aquelle que é desgraçado.
\section{Maleabilidade}
\begin{itemize}
\item {Grp. gram.:f.}
\end{itemize}
Qualidade do que é maleável.
\section{Maleáceo}
\begin{itemize}
\item {Grp. gram.:adj.}
\end{itemize}
\begin{itemize}
\item {Grp. gram.:M. pl.}
\end{itemize}
\begin{itemize}
\item {Proveniência:(Do lat. \textunderscore malleus\textunderscore )}
\end{itemize}
Semelhante ao martelo.
Família de moluscos acéfalos, que têm fórma de martelo.
\section{Maleador}
\begin{itemize}
\item {Grp. gram.:m.  e  adj.}
\end{itemize}
O que maleia.
\section{Malear}
\begin{itemize}
\item {Grp. gram.:v. t.}
\end{itemize}
\begin{itemize}
\item {Utilização:Fig.}
\end{itemize}
\begin{itemize}
\item {Proveniência:(Do lat. \textunderscore malleus\textunderscore )}
\end{itemize}
Converter em lâminas, distender a martelo, (um metal).
Bater com martelo.
Abrandar.
Tornar susceptível de outro feitio.
Tornar dócil.
\section{Maleável}
\begin{itemize}
\item {Grp. gram.:adj.}
\end{itemize}
\begin{itemize}
\item {Utilização:Fig.}
\end{itemize}
Que se póde malear.
Flexível; dócil.
\section{Maleiforme}
\begin{itemize}
\item {Grp. gram.:adj.}
\end{itemize}
\begin{itemize}
\item {Proveniência:(Do lat. \textunderscore malleus\textunderscore  + \textunderscore forma\textunderscore )}
\end{itemize}
Que tem fórma de martelo.
\section{Maleína}
\begin{itemize}
\item {Grp. gram.:f.}
\end{itemize}
Substância inoculável, que se aplica, á maneira de vaccina, contra o mormo.
\section{Maleinização}
\begin{itemize}
\item {fónica:le-i}
\end{itemize}
\begin{itemize}
\item {Grp. gram.:f.}
\end{itemize}
Acto ou efeito de maleinizar.
\section{Maleinizar}
\begin{itemize}
\item {fónica:le-i}
\end{itemize}
\begin{itemize}
\item {Grp. gram.:v. t.}
\end{itemize}
Aplicar a maleína a. Cf. \textunderscore Decreto\textunderscore  de 22-VII-905.
\section{Maleolar}
\begin{itemize}
\item {Grp. gram.:adj.}
\end{itemize}
Relativo aos maléolos.
\section{Maléolo}
\begin{itemize}
\item {Grp. gram.:m.}
\end{itemize}
\begin{itemize}
\item {Proveniência:(Lat. \textunderscore malleolus\textunderscore )}
\end{itemize}
Saliência óssea do tornozelo.
\section{Malfadar}
\begin{itemize}
\item {Grp. gram.:v. t.}
\end{itemize}
\begin{itemize}
\item {Proveniência:(De \textunderscore mal...\textunderscore  + \textunderscore fadar\textunderscore )}
\end{itemize}
Vaticinar má sorte a.
Tornar infeliz, desgraçar.
\section{Malfalante}
\begin{itemize}
\item {Grp. gram.:adj.}
\end{itemize}
\begin{itemize}
\item {Proveniência:(De \textunderscore mal...\textunderscore  + \textunderscore falante\textunderscore )}
\end{itemize}
O mesmo que \textunderscore maldizente\textunderscore .
\section{Malfario}
\begin{itemize}
\item {Grp. gram.:m.}
\end{itemize}
\begin{itemize}
\item {Utilização:Ant.}
\end{itemize}
\begin{itemize}
\item {Proveniência:(Do it. \textunderscore malfare\textunderscore )}
\end{itemize}
Adultério.
\section{Malfazejo}
\begin{itemize}
\item {Grp. gram.:adj.}
\end{itemize}
\begin{itemize}
\item {Proveniência:(De \textunderscore malfazer\textunderscore )}
\end{itemize}
O mesmo que \textunderscore maléfico\textunderscore .
\section{Malfazer}
\begin{itemize}
\item {Grp. gram.:v. i.}
\end{itemize}
\begin{itemize}
\item {Proveniência:(De \textunderscore mal...\textunderscore  + \textunderscore fazer\textunderscore )}
\end{itemize}
Fazer mal; causar damno.
\section{Malfazente}
\begin{itemize}
\item {Grp. gram.:adj.}
\end{itemize}
O mesmo que \textunderscore malfazejo\textunderscore .
\section{Malfeita}
\begin{itemize}
\item {Grp. gram.:f.}
\end{itemize}
\begin{itemize}
\item {Utilização:Prov.}
\end{itemize}
\begin{itemize}
\item {Utilização:trasm.}
\end{itemize}
\begin{itemize}
\item {Utilização:Deprec.}
\end{itemize}
O mesmo que \textunderscore cara\textunderscore ^1.
\section{Malfeito}
\begin{itemize}
\item {Grp. gram.:adj.}
\end{itemize}
\begin{itemize}
\item {Utilização:Fig.}
\end{itemize}
\begin{itemize}
\item {Grp. gram.:M.}
\end{itemize}
\begin{itemize}
\item {Utilização:Ant.}
\end{itemize}
\begin{itemize}
\item {Proveniência:(De \textunderscore mal...\textunderscore  + \textunderscore feito\textunderscore )}
\end{itemize}
Que não é feito com perfeição.
Deforme.
Mau; maldoso.
Injusto; immerecido.
O mesmo que \textunderscore malfeitoria\textunderscore . Cf. B. Pereira, \textunderscore Prosódia\textunderscore .
\section{Malfeitor}
\begin{itemize}
\item {Grp. gram.:m.}
\end{itemize}
\begin{itemize}
\item {Grp. gram.:Adj.}
\end{itemize}
\begin{itemize}
\item {Proveniência:(Do b. lat. \textunderscore malefactor\textunderscore )}
\end{itemize}
Aquelle que commete crimes ou actos condemnáveis; facínora.
Malfazejo.
\section{Malfeitoria}
\begin{itemize}
\item {Grp. gram.:f.}
\end{itemize}
\begin{itemize}
\item {Proveniência:(Do b. lat. \textunderscore malefacturia\textunderscore )}
\end{itemize}
Malefício; delito.
\section{Malferir}
\begin{itemize}
\item {Grp. gram.:v. t.}
\end{itemize}
\begin{itemize}
\item {Proveniência:(De \textunderscore mal...\textunderscore  + \textunderscore ferir\textunderscore )}
\end{itemize}
Ferir gravemente, de morte.
Tornar renhido, cruento: \textunderscore malferir batalhas\textunderscore .
\section{Malfurada}
\begin{itemize}
\item {Grp. gram.:f.}
\end{itemize}
Arbusto madeirense, (\textunderscore globularia salicina\textunderscore , Lam.).
\section{Malga}
\begin{itemize}
\item {Grp. gram.:f.}
\end{itemize}
\begin{itemize}
\item {Proveniência:(Do lat. \textunderscore madiga\textunderscore )}
\end{itemize}
Tigela vidrada, branca ou de côr.
\section{Malgache}
\begin{itemize}
\item {Grp. gram.:adj.}
\end{itemize}
\begin{itemize}
\item {Grp. gram.:M.}
\end{itemize}
Relativo a Madagáscar, ou aos habitantes desta ilha.
Habitante de Madagáscar.
Língua de Madagáscar.
\section{Malgacho}
\begin{itemize}
\item {Grp. gram.:adj.}
\end{itemize}
\begin{itemize}
\item {Grp. gram.:M.}
\end{itemize}
Relativo a Madagáscar, ou aos habitantes desta ilha.
Habitante de Madagáscar.
Língua de Madagáscar.
\section{Malgastar}
\begin{itemize}
\item {Grp. gram.:v. t.}
\end{itemize}
Desperdiçar, esbanjar, gastar mal:«\textunderscore ...não se perdia nem malgastava nada...\textunderscore »Sousa, \textunderscore Vida do Arceb.\textunderscore , I, 163.
\section{Malgável}
\begin{itemize}
\item {Grp. gram.:adj.}
\end{itemize}
\begin{itemize}
\item {Utilização:Prov.}
\end{itemize}
\begin{itemize}
\item {Utilização:trasm.}
\end{itemize}
\begin{itemize}
\item {Proveniência:(Do lat. hyp. \textunderscore mellicabilis\textunderscore ?)}
\end{itemize}
Amável, carinhoso.
\section{Malgovernar}
\begin{itemize}
\item {Grp. gram.:v. t.}
\end{itemize}
Governar mal. Cf. Latino, \textunderscore Humboldt\textunderscore , 434.
\section{Malgradado}
\begin{itemize}
\item {Grp. gram.:adj.}
\end{itemize}
\begin{itemize}
\item {Proveniência:(De \textunderscore malgrado\textunderscore )}
\end{itemize}
Contrariado.
Contrafeito. Cf. Herculano, \textunderscore Quest. Públ.\textunderscore , II, 5.
\section{Malgrado}
\begin{itemize}
\item {Grp. gram.:m.}
\end{itemize}
Mau grado; má vontade:«\textunderscore mas disporão ellas a vosso malgrado...\textunderscore »Castilho, \textunderscore Palavras de Um Crente\textunderscore , 99. Cf. \textunderscore idem\textunderscore , \textunderscore Metam.\textunderscore , 232.
\section{Malha}
\begin{itemize}
\item {Grp. gram.:f.}
\end{itemize}
\begin{itemize}
\item {Proveniência:(Do lat. \textunderscore macula\textunderscore )}
\end{itemize}
Cada um dos nós ou voltas de um fio ou de qualquer fibra têxtil, entrançados ou tecidos por certos processos.
Abertura entre êsses nós.
Trança de fio de metal, com que se faziam armaduras.
Mancha na pelle dos animaes.
Mancha natural.
Descoloração ou mancha, no conjunto da vegetação de um terreno.
\section{Malha}
\begin{itemize}
\item {Grp. gram.:f.}
\end{itemize}
\begin{itemize}
\item {Utilização:Prov.}
\end{itemize}
\begin{itemize}
\item {Utilização:trasm.}
\end{itemize}
Acto de malhar: \textunderscore a malha do centeio\textunderscore .
Sova; castigo corporal.
O mesmo que \textunderscore maça\textunderscore  ou instrumento de maçar o linho.
\section{Malha}
\begin{itemize}
\item {Grp. gram.:f.}
\end{itemize}
\begin{itemize}
\item {Proveniência:(Do lat. \textunderscore magalia\textunderscore )}
\end{itemize}
O mesmo que \textunderscore choça\textunderscore ^2.
\section{Malha}
\begin{itemize}
\item {Grp. gram.:f.}
\end{itemize}
Antiga moéda, mealha.
Chapa metállica no jôgo do chinquilho e do fito.
O mesmo que \textunderscore chinquilho\textunderscore .
(B. lat. \textunderscore medalea\textunderscore )
\section{Malhaçada}
\begin{itemize}
\item {Grp. gram.:f.}
\end{itemize}
\begin{itemize}
\item {Utilização:Heráld.}
\end{itemize}
\begin{itemize}
\item {Proveniência:(Do rad. de \textunderscore malho\textunderscore )}
\end{itemize}
Maço ou malho nos brasões.
\section{Malhada}
\begin{itemize}
\item {Grp. gram.:f.}
\end{itemize}
\begin{itemize}
\item {Utilização:Ant.}
\end{itemize}
\begin{itemize}
\item {Proveniência:(De \textunderscore malha\textunderscore ^1)}
\end{itemize}
O mesmo que \textunderscore enrêdo\textunderscore .
O mesmo que \textunderscore malhoada\textunderscore .
\section{Malhada}
\begin{itemize}
\item {Grp. gram.:f.}
\end{itemize}
Pancada com malho.
Acto de malhar.
Lugar em que se malha.
\section{Malhada}
\begin{itemize}
\item {Grp. gram.:f.}
\end{itemize}
\begin{itemize}
\item {Utilização:Prov.}
\end{itemize}
\begin{itemize}
\item {Utilização:beir.}
\end{itemize}
\begin{itemize}
\item {Utilização:Ant.}
\end{itemize}
\begin{itemize}
\item {Utilização:Prov.}
\end{itemize}
\begin{itemize}
\item {Utilização:alent.}
\end{itemize}
\begin{itemize}
\item {Utilização:Prov.}
\end{itemize}
\begin{itemize}
\item {Utilização:trasm.}
\end{itemize}
\begin{itemize}
\item {Utilização:Ant.}
\end{itemize}
\begin{itemize}
\item {Proveniência:(De \textunderscore malha\textunderscore ^3)}
\end{itemize}
Cabana de pastores.
Curral de gado.
Rebanho de ovelhas.
Estada de gado lanígero, em terras de semeadura para as estrumar.
Mata de carvalhos, já crescidos, mas ainda não adultos.
Fábrica de cera, constituida por uma cêrca de resguardo para as abelhas e de habitação para o tratador das colmeias.
Lura, toca.
Cama ou casa, em terreno onde há colmeias.
Silha de colmeias, em sítio abrigado.
Terreno cercado, em que se apascenta gado.
\section{Malhadal}
\begin{itemize}
\item {Grp. gram.:m.}
\end{itemize}
\begin{itemize}
\item {Utilização:Marn.}
\end{itemize}
\begin{itemize}
\item {Utilização:Prov.}
\end{itemize}
\begin{itemize}
\item {Utilização:alent.}
\end{itemize}
Espaço, quási sempre cultivado, entre a defensão da marinha e o entraval.
Lugar quási plano, ao lado dos montes ou casaes, onde os pastores fazem malhada ou dormida.
(Cp. \textunderscore malhadil\textunderscore )
\section{Malhadeiro}
\begin{itemize}
\item {Grp. gram.:m.}
\end{itemize}
\begin{itemize}
\item {Grp. gram.:Adj.}
\end{itemize}
\begin{itemize}
\item {Utilização:Ant.}
\end{itemize}
\begin{itemize}
\item {Proveniência:(De \textunderscore malhada\textunderscore ^2)}
\end{itemize}
Instrumento com que se malha trigo.
Aquelle que é malhadiço.
Aquelle que é objecto de motejos.
Malhadiço.
Grosseiro; tôsco.
Pateta; idiota.
\section{Malhadeiro}
\begin{itemize}
\item {Grp. gram.:m.}
\end{itemize}
\begin{itemize}
\item {Proveniência:(De \textunderscore malhada\textunderscore ^3)}
\end{itemize}
Aquelle que trata de colmeias; colmeeiro. Cf. \textunderscore Port. Ant. e Mod.\textunderscore , XI, 1113.
\section{Malhadela}
\begin{itemize}
\item {Grp. gram.:f.}
\end{itemize}
\begin{itemize}
\item {Proveniência:(De \textunderscore malhar\textunderscore ^1)}
\end{itemize}
Obrigação, que alguns foreiros tinham, de dar aos senhorios certos dias de trabalho.
O mesmo que \textunderscore malha\textunderscore ^2.
\section{Malhadiço}
\begin{itemize}
\item {Grp. gram.:adj.}
\end{itemize}
\begin{itemize}
\item {Proveniência:(De \textunderscore malhado\textunderscore ^1)}
\end{itemize}
Habituado a levar pancadas; incorrigível; descarado.
\section{Malhadil}
\begin{itemize}
\item {Grp. gram.:m.}
\end{itemize}
\begin{itemize}
\item {Utilização:Prov.}
\end{itemize}
\begin{itemize}
\item {Utilização:alent.}
\end{itemize}
\begin{itemize}
\item {Proveniência:(De \textunderscore malhada\textunderscore ^3)}
\end{itemize}
Espaço cultivado em meio de charneca.
\section{Malhado}
\begin{itemize}
\item {Grp. gram.:m.}
\end{itemize}
\begin{itemize}
\item {Grp. gram.:M.}
\end{itemize}
\begin{itemize}
\item {Proveniência:(De \textunderscore malha\textunderscore ^1)}
\end{itemize}
Que tem malhas ou manchas.
Sectário do partido constitucional, na linguagem do partido absolutista, em Portugal, (por allusão á bandeira bicolor, e ás calças de xadrez, usadas por muitos constitucionaes)
Arbusto pittosporáceo do Brasil.
\section{Malhado}
\begin{itemize}
\item {Grp. gram.:adj.}
\end{itemize}
\begin{itemize}
\item {Utilização:Bras. do N}
\end{itemize}
\begin{itemize}
\item {Proveniência:(De \textunderscore malhar\textunderscore ^3)}
\end{itemize}
Que está na malhada^3; que está em descanso.
\section{Malhadoiro}
\begin{itemize}
\item {Grp. gram.:m.}
\end{itemize}
\begin{itemize}
\item {Proveniência:(De \textunderscore malhar\textunderscore ^1)}
\end{itemize}
Sitio, em que se malham cereaes.
\section{Malhador}
\begin{itemize}
\item {Grp. gram.:m.}
\end{itemize}
\begin{itemize}
\item {Grp. gram.:Adj.}
\end{itemize}
\begin{itemize}
\item {Proveniência:(Do lat. \textunderscore malleator\textunderscore )}
\end{itemize}
Indivíduo que malha.
Desordeiro; que gosta de bater em qualquer pessôa.
\section{Malhadouro}
\begin{itemize}
\item {Grp. gram.:m.}
\end{itemize}
\begin{itemize}
\item {Proveniência:(De \textunderscore malhar\textunderscore ^1)}
\end{itemize}
Sitio, em que se malham cereaes.
\section{Malhagem}
\begin{itemize}
\item {Grp. gram.:f.}
\end{itemize}
\begin{itemize}
\item {Utilização:Pesc.}
\end{itemize}
\begin{itemize}
\item {Proveniência:(De \textunderscore malha\textunderscore ^1)}
\end{itemize}
Bitola ou padrão das malhas de uma rêde.
\section{Malhal}
\begin{itemize}
\item {Grp. gram.:m.}
\end{itemize}
\begin{itemize}
\item {Utilização:Prov.}
\end{itemize}
\begin{itemize}
\item {Utilização:trasm.}
\end{itemize}
\begin{itemize}
\item {Utilização:T. de Turquel}
\end{itemize}
\begin{itemize}
\item {Proveniência:(Do rad. de \textunderscore malho\textunderscore )}
\end{itemize}
Travéssa de madeira, que assenta no pé da uva, dentro dos lagares, e sôbre a qual carrega a vara do lagar.
Pedaço de barrote, em que os esculptores assentam a pedra em que trabalham.
Baixete.
Calço de madeira; em que assentam as vasilhas na adega.
O mesmo que \textunderscore magueixo\textunderscore .
O mesmo que \textunderscore mangual\textunderscore .
\section{Malhanas}
\begin{itemize}
\item {Grp. gram.:m. pl.}
\end{itemize}
Indígenas do norte do Brasil.
\section{Malhante}
\begin{itemize}
\item {Grp. gram.:m.}
\end{itemize}
\begin{itemize}
\item {Utilização:Açor}
\end{itemize}
\begin{itemize}
\item {Proveniência:(De \textunderscore malhar\textunderscore ^1)}
\end{itemize}
Official de ferreiro, que affeiçoa os pregos com o malho.
\section{Malhão}
\begin{itemize}
\item {Grp. gram.:m.}
\end{itemize}
\begin{itemize}
\item {Utilização:Ant.}
\end{itemize}
\begin{itemize}
\item {Utilização:Prov.}
\end{itemize}
\begin{itemize}
\item {Utilização:trasm.}
\end{itemize}
\begin{itemize}
\item {Utilização:T. de Vimioso}
\end{itemize}
\begin{itemize}
\item {Proveniência:(De \textunderscore malho\textunderscore )}
\end{itemize}
Tiro por alto, no jôgo da bola.
Bola, com que se faz êsse tiro.
Malhal.
Baliza, marco.
Feixe de giestas ou outras plantas, atado com um vincilho e servindo para vedar terras.
A pedra maior do jôgo das nécaras.
\section{Malhão}
\begin{itemize}
\item {Grp. gram.:m.}
\end{itemize}
Espécie de música e dança populares.
\section{Malha-pão}
\begin{itemize}
\item {Grp. gram.:m.}
\end{itemize}
Aquelle em que se bate, como em pão nas eiras. Cf. Castilho, \textunderscore Sabichonas\textunderscore , 152.
\section{Malhar}
\begin{itemize}
\item {Grp. gram.:v. t.}
\end{itemize}
\begin{itemize}
\item {Utilização:Fig.}
\end{itemize}
\begin{itemize}
\item {Grp. gram.:V. i.}
\end{itemize}
\begin{itemize}
\item {Proveniência:(Do lat. \textunderscore malleare\textunderscore )}
\end{itemize}
Bater com malho ou instrumento análogo.
Debulhar nas eiras (cereaes).
Bater, contundir.
Zombar.
Dar pancadas.
\textunderscore Malhar abaixo\textunderscore , cair, despenhar-se. Cf. Camillo, \textunderscore Brasileira\textunderscore , 56.
\section{Malhar}
\begin{itemize}
\item {Grp. gram.:v. i.}
\end{itemize}
\begin{itemize}
\item {Utilização:Prov.}
\end{itemize}
\begin{itemize}
\item {Utilização:alent.}
\end{itemize}
\begin{itemize}
\item {Proveniência:(De \textunderscore malha\textunderscore ^1)}
\end{itemize}
Cair na malha ou na rêde.
\section{Malhar}
\begin{itemize}
\item {Grp. gram.:v. t.}
\end{itemize}
\begin{itemize}
\item {Utilização:Bras. do N}
\end{itemize}
\begin{itemize}
\item {Proveniência:(De \textunderscore malha\textunderscore ^3)}
\end{itemize}
Reunir (gado) em determinado ponto.
\section{Malheirão}
\begin{itemize}
\item {Grp. gram.:m.}
\end{itemize}
\begin{itemize}
\item {Proveniência:(Do rad. de \textunderscore malhar\textunderscore ^1)}
\end{itemize}
Jôgo de rapazes, em que um, sentado nas costas de outro, lhe dá com o cotovelo e com o punho, até que adivinhe quantos dedos da outra mão elle tem abertos.
\section{Malheiro}
\begin{itemize}
\item {Grp. gram.:m.}
\end{itemize}
\begin{itemize}
\item {Utilização:Ant.}
\end{itemize}
\begin{itemize}
\item {Proveniência:(De \textunderscore malha\textunderscore ^1)}
\end{itemize}
Aquelle que fabricava malhas para cotas de malha.
Instrumento, usado na fabricação de rêdes de pesca.
\section{Malhetar}
\begin{itemize}
\item {Grp. gram.:v. t.}
\end{itemize}
Fazer malhetes em.
Encaixar (uma peça de metal ou madeira noutra).
\section{Malhete}
\begin{itemize}
\item {fónica:lhê}
\end{itemize}
\begin{itemize}
\item {Grp. gram.:m.}
\end{itemize}
\begin{itemize}
\item {Utilização:Náut.}
\end{itemize}
\begin{itemize}
\item {Utilização:Prov.}
\end{itemize}
\begin{itemize}
\item {Utilização:trasm.}
\end{itemize}
\begin{itemize}
\item {Proveniência:(De \textunderscore malho\textunderscore  e de \textunderscore malha\textunderscore ^1)}
\end{itemize}
Encaixe, feito na extremidade de duas tábuas, para que se adaptem perfeitamente.
Pequeno malho ou maço, com que, nas sessões maçónicas, um dignitário chama a attenção dos irmãos presentes.
Malha de ferro ou madeira, que liga os óvens de uma enxárcia, num navio.
Cada uma das peças do coucão.
\section{Malhissores}
\begin{itemize}
\item {Grp. gram.:m. pl.}
\end{itemize}
Christãos montanheses da Albânia.
(Do albanês \textunderscore mal'\textunderscore , monte)
\section{Malho}
\begin{itemize}
\item {Grp. gram.:m.}
\end{itemize}
\begin{itemize}
\item {Utilização:Prov.}
\end{itemize}
\begin{itemize}
\item {Utilização:Fig.}
\end{itemize}
\begin{itemize}
\item {Utilização:Fam.}
\end{itemize}
\begin{itemize}
\item {Utilização:Ant.}
\end{itemize}
\begin{itemize}
\item {Proveniência:(Do lat. \textunderscore malleus\textunderscore )}
\end{itemize}
Espécie de martelo de ferro ou de pau, mas sem unhas ou orelhas.
Maço de calceteiro.
Matraca.
O mesmo que \textunderscore mangual\textunderscore .
Pessôa hábil, fina.
Coisa certa, infallivel.
Correia que tem os cascavéis, nas aves empregadas em caça de altanaria.
\section{Malhoada}
\begin{itemize}
\item {Grp. gram.:f.}
\end{itemize}
\begin{itemize}
\item {Utilização:Chul.}
\end{itemize}
\begin{itemize}
\item {Proveniência:(Do rad. de \textunderscore malha\textunderscore ^1)}
\end{itemize}
Conluio; tramóia.
\section{Malhorquino}
\begin{itemize}
\item {Grp. gram.:adj.}
\end{itemize}
O mesmo que \textunderscore maiorquino\textunderscore .
\section{Mal-humorado}
\begin{itemize}
\item {Grp. gram.:adj.}
\end{itemize}
\begin{itemize}
\item {Utilização:Fig.}
\end{itemize}
Que tem maus humores.
Enfermiço; achacado.
Amuado; zangado; intratável.
\section{Máli}
\begin{itemize}
\item {Grp. gram.:m.}
\end{itemize}
\begin{itemize}
\item {Proveniência:(T. afr.)}
\end{itemize}
Pedagogo, entre os Cafres de Quelimane.
\section{Malibundo}
\begin{itemize}
\item {Grp. gram.:m.}
\end{itemize}
O mesmo que \textunderscore maribundo\textunderscore .
\section{Malícia}
\begin{itemize}
\item {Grp. gram.:f.}
\end{itemize}
\begin{itemize}
\item {Utilização:Bras. do N}
\end{itemize}
\begin{itemize}
\item {Proveniência:(Lat. \textunderscore malitia\textunderscore )}
\end{itemize}
Má índole.
Propensão para o mal.
Velhacaria.
Astúcia.
Interpretação maliciosa; ronha.
Brejeirice; dito picante.
O mesmo que \textunderscore sensitiva\textunderscore , planta.
\section{Malícia-de-mulher}
\begin{itemize}
\item {Grp. gram.:f.}
\end{itemize}
\begin{itemize}
\item {Utilização:Bras}
\end{itemize}
\begin{itemize}
\item {Utilização:Bot.}
\end{itemize}
O mesmo que \textunderscore sensitiva\textunderscore .
\section{Maliciar}
\begin{itemize}
\item {Grp. gram.:v. t.}
\end{itemize}
Attribuir malícia a; tomar em mau sentido. Cf. Filinto, \textunderscore Vida de D. Man.\textunderscore , II, 263.
\section{Maliciosamente}
\begin{itemize}
\item {Grp. gram.:adv.}
\end{itemize}
De modo malicioso; arteiramente.
\section{Malicioso}
\begin{itemize}
\item {Grp. gram.:adj.}
\end{itemize}
\begin{itemize}
\item {Proveniência:(Lat. \textunderscore malitiosus\textunderscore )}
\end{itemize}
Que tem malícia; finório; sagaz.
\section{Málico}
\begin{itemize}
\item {Grp. gram.:adj.}
\end{itemize}
\begin{itemize}
\item {Proveniência:(Do lat. \textunderscore malum\textunderscore )}
\end{itemize}
Diz-se de um ácido, que existe em muitos vegetaes.
\section{Maligna}
\begin{itemize}
\item {Grp. gram.:f.}
\end{itemize}
\begin{itemize}
\item {Proveniência:(De \textunderscore maligno\textunderscore )}
\end{itemize}
Febre de mau carácter; typho.
\section{Malignamente}
\begin{itemize}
\item {Grp. gram.:adv.}
\end{itemize}
\begin{itemize}
\item {Proveniência:(De \textunderscore maligno\textunderscore )}
\end{itemize}
Com malícia, com maldade.
\section{Malignante}
\begin{itemize}
\item {Grp. gram.:adj.}
\end{itemize}
\begin{itemize}
\item {Proveniência:(Lat. \textunderscore malignans\textunderscore )}
\end{itemize}
O mesmo que \textunderscore malicioso\textunderscore .
\section{Malignar}
\begin{itemize}
\item {Grp. gram.:v. t.}
\end{itemize}
\begin{itemize}
\item {Grp. gram.:V. i.}
\end{itemize}
\begin{itemize}
\item {Proveniência:(Lat. \textunderscore malignare\textunderscore )}
\end{itemize}
Tornar mau ou maligno; viciar.
Recrudescer, (falando-se de doenças).
\section{Malignidade}
\begin{itemize}
\item {Grp. gram.:f.}
\end{itemize}
\begin{itemize}
\item {Proveniência:(Lat. \textunderscore malignitas\textunderscore )}
\end{itemize}
Qualidade de maligno.
\section{Maligno}
\begin{itemize}
\item {Grp. gram.:adj.}
\end{itemize}
\begin{itemize}
\item {Proveniência:(Lat. \textunderscore malignus\textunderscore )}
\end{itemize}
Tendente para o mal; malicioso.
Nocivo, muito mau; de má qualidade.
\section{Malim}
\begin{itemize}
\item {Grp. gram.:m.}
\end{itemize}
\begin{itemize}
\item {Utilização:Prov.}
\end{itemize}
\begin{itemize}
\item {Proveniência:(De \textunderscore mala\textunderscore )}
\end{itemize}
O mesmo quo \textunderscore molhelha\textunderscore .
\section{Malina}
\begin{itemize}
\item {Grp. gram.:f.}
\end{itemize}
\begin{itemize}
\item {Proveniência:(Lat. \textunderscore malina\textunderscore )}
\end{itemize}
Águas vivas das marés.
\section{Malina}
\begin{itemize}
\item {Grp. gram.:f.}
\end{itemize}
\begin{itemize}
\item {Utilização:Prov.}
\end{itemize}
\begin{itemize}
\item {Utilização:minh.}
\end{itemize}
(Forma pop. de \textunderscore maligna\textunderscore )
Febre de mau carácter.
Mau cheiro.
\section{Malino}
\begin{itemize}
\item {Grp. gram.:adj.}
\end{itemize}
(Fórma ant. de \textunderscore maligno\textunderscore ) Cf. \textunderscore Usque\textunderscore , 51.
\section{Malinense}
\begin{itemize}
\item {Grp. gram.:adj.}
\end{itemize}
\begin{itemize}
\item {Grp. gram.:M.}
\end{itemize}
Relativo á cidade de Malinas, na Bélgica.
Habitante de Malinas.
\section{Má-língua}
\begin{itemize}
\item {Grp. gram.:f.}
\end{itemize}
\begin{itemize}
\item {Grp. gram.:M.  e  f.}
\end{itemize}
Maledicência.
Vício de dizer mal de pessôas e coisas.
Pessôa maldizente.
\section{Malintencionado}
\begin{itemize}
\item {Grp. gram.:adj.}
\end{itemize}
\begin{itemize}
\item {Proveniência:(De \textunderscore mal...\textunderscore  + \textunderscore intencionado\textunderscore )}
\end{itemize}
Que tem más intenções ou má índole.
\section{Malinu}
\begin{itemize}
\item {Grp. gram.:m.}
\end{itemize}
\begin{itemize}
\item {Proveniência:(T. afr.)}
\end{itemize}
O mesmo que \textunderscore doutor\textunderscore , entre os Moiros da África oriental.
\section{Maliolo}
\begin{itemize}
\item {Grp. gram.:m.}
\end{itemize}
\begin{itemize}
\item {Utilização:Ant.}
\end{itemize}
Vinha nova; bacêllo.
\section{Malíssimamente}
\begin{itemize}
\item {Grp. gram.:adv.}
\end{itemize}
De modo malíssimo.
Muito mal.
\section{Malíssimo}
\begin{itemize}
\item {Grp. gram.:adj.}
\end{itemize}
\begin{itemize}
\item {Proveniência:(Do lat. \textunderscore malus\textunderscore )}
\end{itemize}
Muito mau; que é pior que todos:«\textunderscore oh malissimo dos homens!\textunderscore »Filinto, XXII, 77. Cf. Pant. de Aveiro, \textunderscore Itiner.\textunderscore , 20, (2.^a ed.).
\section{Malleabilidade}
\begin{itemize}
\item {Grp. gram.:f.}
\end{itemize}
Qualidade do que é malleável.
\section{Malleáceo}
\begin{itemize}
\item {Grp. gram.:adj.}
\end{itemize}
\begin{itemize}
\item {Grp. gram.:M. pl.}
\end{itemize}
\begin{itemize}
\item {Proveniência:(Do lat. \textunderscore malleus\textunderscore )}
\end{itemize}
Semelhante ao martelo.
Família de molluscos acéphalos, que têm fórma de martelo.
\section{Malleador}
\begin{itemize}
\item {Grp. gram.:m.  e  adj.}
\end{itemize}
O que malleia.
\section{Mallear}
\begin{itemize}
\item {Grp. gram.:v. t.}
\end{itemize}
\begin{itemize}
\item {Utilização:Fig.}
\end{itemize}
\begin{itemize}
\item {Proveniência:(Do lat. \textunderscore malleus\textunderscore )}
\end{itemize}
Converter em lâminas, distender a martello, (um metal).
Bater com martelo.
Abrandar.
Tornar susceptível de outro feitio.
Tornar dócil.
\section{Malleável}
\begin{itemize}
\item {Grp. gram.:adj.}
\end{itemize}
\begin{itemize}
\item {Utilização:Fig.}
\end{itemize}
Que se póde mallear.
Flexível; dócil.
\section{Malleiforme}
\begin{itemize}
\item {Grp. gram.:adj.}
\end{itemize}
\begin{itemize}
\item {Proveniência:(Do lat. \textunderscore malleus\textunderscore  + \textunderscore forma\textunderscore )}
\end{itemize}
Que tem fórma de martelo.
\section{Malleína}
\begin{itemize}
\item {Grp. gram.:f.}
\end{itemize}
Substância inoculável, que se applica, á maneira de vaccina, contra o mormo.
\section{Malleinização}
\begin{itemize}
\item {fónica:le-i}
\end{itemize}
\begin{itemize}
\item {Grp. gram.:f.}
\end{itemize}
Acto ou effeito de malleinizar.
\section{Malleinizar}
\begin{itemize}
\item {fónica:le-i}
\end{itemize}
\begin{itemize}
\item {Grp. gram.:v. t.}
\end{itemize}
Appllicar a malleína a. Cf. \textunderscore Decreto\textunderscore  de 22-VII-905.
\section{Malleolar}
\begin{itemize}
\item {Grp. gram.:adj.}
\end{itemize}
Relativo aos malléolos.
\section{Malléolo}
\begin{itemize}
\item {Grp. gram.:m.}
\end{itemize}
\begin{itemize}
\item {Proveniência:(Lat. \textunderscore malleolus\textunderscore )}
\end{itemize}
Saliência óssea do tornozelo.
\section{Mallogradamente}
\begin{itemize}
\item {Grp. gram.:adv.}
\end{itemize}
\begin{itemize}
\item {Proveniência:(De \textunderscore mal\textunderscore  + \textunderscore lograr\textunderscore )}
\end{itemize}
De modo mallogrado.
Infructuosamente; sem bom êxito.
\section{Mallograr}
\begin{itemize}
\item {Grp. gram.:v. t.}
\end{itemize}
\begin{itemize}
\item {Grp. gram.:V. p.}
\end{itemize}
\begin{itemize}
\item {Proveniência:(De \textunderscore mal...\textunderscore  + \textunderscore lograr\textunderscore )}
\end{itemize}
Inutilizar: \textunderscore a geada mallogrou a sementeira\textunderscore .
Fazer gorar: \textunderscore mallograr projectos\textunderscore .
Fazer desapparecer.
Gorar-se.
Não ir ávante.
Perder-se prematuramente.
Não têr o resultado que se desejava: \textunderscore a conspiração mallogrou-se\textunderscore .
\section{Mallôgro}
\begin{itemize}
\item {Grp. gram.:m.}
\end{itemize}
Effeito de mallograr.
\section{Mallora}
\begin{itemize}
\item {Grp. gram.:f.}
\end{itemize}
\begin{itemize}
\item {Proveniência:(De \textunderscore Mallora\textunderscore , n. p. myth.)}
\end{itemize}
Palmeira de Madagáscar.
\section{Malmajuda}
\begin{itemize}
\item {Grp. gram.:f.}
\end{itemize}
Árvore brasileira.
\section{Malmentinhos}
\begin{itemize}
\item {Grp. gram.:adv.}
\end{itemize}
\begin{itemize}
\item {Utilização:Prov.}
\end{itemize}
\begin{itemize}
\item {Utilização:alg.}
\end{itemize}
De leve; ligeiramente.
\section{Malmequér}
\begin{itemize}
\item {Grp. gram.:m.}
\end{itemize}
\begin{itemize}
\item {Proveniência:(De \textunderscore mal...\textunderscore  + \textunderscore me\textunderscore  + \textunderscore querer\textunderscore )}
\end{itemize}
Nome de várias plantas da fam. das compostas, vulgar em campos e jardins.
\section{Malmequér-da-sécia}
\begin{itemize}
\item {Grp. gram.:f.}
\end{itemize}
\begin{itemize}
\item {Utilização:Bot.}
\end{itemize}
Espécie de margarida, de flôres duplas.
\section{Malmequeres}
\begin{itemize}
\item {Grp. gram.:m.}
\end{itemize}
O mesmo que \textunderscore malmequér\textunderscore .--Us. por Camões.
\section{Malmequerzinho}
\begin{itemize}
\item {Grp. gram.:m.}
\end{itemize}
\begin{itemize}
\item {Proveniência:(De \textunderscore malmequér\textunderscore )}
\end{itemize}
Planta parasita, (\textunderscore epipactis campinaria\textunderscore ).
\section{Malnascido}
\begin{itemize}
\item {Grp. gram.:adj.}
\end{itemize}
\begin{itemize}
\item {Proveniência:(De \textunderscore mal...\textunderscore  + \textunderscore nascido\textunderscore )}
\end{itemize}
Nascido com má sorte; malfadado.
Que tem má índole.
Que é de baixa estirpe.
\section{Malo}
\begin{itemize}
\item {Grp. gram.:adj.}
\end{itemize}
\begin{itemize}
\item {Proveniência:(Lat. \textunderscore malus\textunderscore )}
\end{itemize}
Us. na loc. adv. \textunderscore alto e malo\textunderscore , ao acaso, a esmo, sem escolha:«\textunderscore daremos á mãe e á filha alto e malo senhoria.\textunderscore »Tolentino.
\section{Malóbathro}
\begin{itemize}
\item {Grp. gram.:m.}
\end{itemize}
\begin{itemize}
\item {Proveniência:(Gr. \textunderscore malobathrom\textunderscore )}
\end{itemize}
Designação antiga de uma árvore asiática, talvez o bétel.
\section{Malóbatro}
\begin{itemize}
\item {Grp. gram.:m.}
\end{itemize}
\begin{itemize}
\item {Proveniência:(Gr. \textunderscore malobathrom\textunderscore )}
\end{itemize}
Designação antiga de uma árvore asiática, talvez o bétel.
\section{Maloca}
\begin{itemize}
\item {Grp. gram.:f.}
\end{itemize}
Grande barraca, coberta de palmas sêcas, habitação de Indígenas da América.
Bando de Indigenas do Brasil.
\section{Maloca}
\begin{itemize}
\item {Grp. gram.:f.}
\end{itemize}
\begin{itemize}
\item {Utilização:T. de Pinhel}
\end{itemize}
Chinela.
\section{Maloio}
\begin{itemize}
\item {Grp. gram.:m.}
\end{itemize}
Campónio; aldeão; lapuz.
\section{Malolo}
\begin{itemize}
\item {fónica:lô}
\end{itemize}
\begin{itemize}
\item {Grp. gram.:m.}
\end{itemize}
Árvore africana, (\textunderscore anona senegalensis\textunderscore , Pers.).
\section{Malombada}
\begin{itemize}
\item {Grp. gram.:f.}
\end{itemize}
\begin{itemize}
\item {Utilização:Náut.}
\end{itemize}
Pedaços de cabos velhos, sem prestimo.
\section{Malonato}
\begin{itemize}
\item {Grp. gram.:m.}
\end{itemize}
\begin{itemize}
\item {Utilização:Chím.}
\end{itemize}
\begin{itemize}
\item {Proveniência:(De \textunderscore malónico\textunderscore )}
\end{itemize}
Combinação do ácido malónico com uma base.
\section{Malónico}
\begin{itemize}
\item {Grp. gram.:adj.}
\end{itemize}
\begin{itemize}
\item {Utilização:Chím.}
\end{itemize}
\begin{itemize}
\item {Proveniência:(De \textunderscore málico\textunderscore )}
\end{itemize}
Diz-se de um ácido, derivado do ácido málico por oxydação.
\section{Malopa}
\begin{itemize}
\item {Grp. gram.:f.}
\end{itemize}
Planta malvácea da bacia do Mediterrâneo.
\section{Malora}
\begin{itemize}
\item {Grp. gram.:f.}
\end{itemize}
\begin{itemize}
\item {Proveniência:(De \textunderscore Mallora\textunderscore , n. p. myth.)}
\end{itemize}
Palmeira de Madagáscar.
\section{Malo-russo}
\begin{itemize}
\item {Grp. gram.:m.}
\end{itemize}
O mesmo que \textunderscore ruthênico\textunderscore .
\section{Malotão}
\begin{itemize}
\item {Grp. gram.:m.}
\end{itemize}
\begin{itemize}
\item {Proveniência:(De \textunderscore malote\textunderscore )}
\end{itemize}
Grande mala; grande pacote ou troixa.
\section{Malote}
\begin{itemize}
\item {Grp. gram.:m.}
\end{itemize}
Pequena mala.
Peça de oleado, em que o soldado envolve o capote.
\section{Malouria}
\begin{itemize}
\item {Grp. gram.:f.}
\end{itemize}
\begin{itemize}
\item {Utilização:Ant.}
\end{itemize}
O mesmo que \textunderscore doença\textunderscore . Cf. \textunderscore Cancion. da Vaticana\textunderscore , 1017.
(Relaciona-se com \textunderscore malária\textunderscore ?)
\section{Malparado}
\begin{itemize}
\item {Grp. gram.:adj.}
\end{itemize}
\begin{itemize}
\item {Proveniência:(De \textunderscore malparar\textunderscore )}
\end{itemize}
Que está arriscado a perder-se; pouco seguro.
\section{Malparar}
\begin{itemize}
\item {Grp. gram.:v. t.}
\end{itemize}
\begin{itemize}
\item {Proveniência:(Do lat. \textunderscore male\textunderscore  + \textunderscore parare\textunderscore )}
\end{itemize}
Arriscar; aventurar; sujeitar a mau destino. Cf. Filinto, \textunderscore Vida de D. Man.\textunderscore , III, 225.
\section{Malparida}
\begin{itemize}
\item {Grp. gram.:f.}
\end{itemize}
Mulher que malpariu, que teve abôrto.
\section{Malparir}
\begin{itemize}
\item {Grp. gram.:v. i.}
\end{itemize}
\begin{itemize}
\item {Proveniência:(De \textunderscore mal\textunderscore  + \textunderscore parir\textunderscore )}
\end{itemize}
Têr mau parto.
Abortar:«\textunderscore ...malparindo esta...\textunderscore »Filinto, \textunderscore D. Man.\textunderscore , I, 54.«\textunderscore ...actuadas delle (ribombo), muitas mulheres malparirão.\textunderscore »\textunderscore Id.\textunderscore , \textunderscore ib.\textunderscore , II, 83.
\section{Mal-peccado}
\begin{itemize}
\item {Grp. gram.:adv.}
\end{itemize}
\begin{itemize}
\item {Utilização:Pop.}
\end{itemize}
\begin{itemize}
\item {Grp. gram.:M.}
\end{itemize}
Infelizmente.
Oxalá; praza a Deus.
Infelicidade. Cf. \textunderscore Eufrosina\textunderscore , 47; G. Vicente, \textunderscore Inês Pereira\textunderscore .
\section{Malpica}
\begin{itemize}
\item {Grp. gram.:f.}
\end{itemize}
Planta amaranthácea de Cabo-Verde.
\section{Malpígia}
\begin{itemize}
\item {Grp. gram.:f.}
\end{itemize}
\begin{itemize}
\item {Proveniência:(De \textunderscore Malpighi\textunderscore , n. p.)}
\end{itemize}
Gênero de plantas, muito ramosas, da América.
\section{Malpigiáceas}
\begin{itemize}
\item {Grp. gram.:f. pl.}
\end{itemize}
Família de plantas dicotyledóneas, que tem por typo a malpígia.
\section{Malpinguinho}
\begin{itemize}
\item {Grp. gram.:m.}
\end{itemize}
\begin{itemize}
\item {Utilização:Bras}
\end{itemize}
O mesmo que \textunderscore mapinguim\textunderscore .
\section{Mal-propício}
\begin{itemize}
\item {Grp. gram.:adj.}
\end{itemize}
Impróprio; pouco adequado.
\section{Mal-próprio}
\begin{itemize}
\item {Grp. gram.:adj.}
\end{itemize}
\begin{itemize}
\item {Utilização:Gal}
\end{itemize}
\begin{itemize}
\item {Utilização:Fig.}
\end{itemize}
\begin{itemize}
\item {Proveniência:(Fr. \textunderscore malpropre\textunderscore )}
\end{itemize}
Que não tem propriedade; opposto á propriedade.
Indecoroso.
Opposto ao dever. Cf. Latino, \textunderscore Elogios\textunderscore , 223.
\section{Malquerença}
\begin{itemize}
\item {Grp. gram.:f.}
\end{itemize}
Qualidade de malquerente.
O mesmo que \textunderscore malevolência\textunderscore .
(B. lat. \textunderscore malquerentia\textunderscore )
\section{Malquerente}
\begin{itemize}
\item {Grp. gram.:adj.}
\end{itemize}
\begin{itemize}
\item {Proveniência:(De \textunderscore malquerer\textunderscore )}
\end{itemize}
Que quer mal a outrem.
Malévolo.
Inimigo.
\section{Malquerer}
\begin{itemize}
\item {Grp. gram.:v. t.}
\end{itemize}
\begin{itemize}
\item {Proveniência:(De \textunderscore mal...\textunderscore  + \textunderscore querer\textunderscore )}
\end{itemize}
Querer mal a.
Sêr inimigo de.
\section{Malqueria}
\begin{itemize}
\item {Grp. gram.:f.}
\end{itemize}
\begin{itemize}
\item {Utilização:Ant.}
\end{itemize}
\begin{itemize}
\item {Proveniência:(De \textunderscore malmequerer\textunderscore )}
\end{itemize}
O mesmo que \textunderscore malquerença\textunderscore .
\section{Malquistar}
\begin{itemize}
\item {Grp. gram.:v. t.}
\end{itemize}
Inimizar; tornar malquisto.
\section{Malquisto}
\begin{itemize}
\item {Grp. gram.:adj.}
\end{itemize}
\begin{itemize}
\item {Proveniência:(De \textunderscore mal...\textunderscore  + \textunderscore quisto\textunderscore )}
\end{itemize}
Que adquiriu inimigos.
Que tem má fama.
Antipatico; odiado.
\section{Mal-regido}
\begin{itemize}
\item {Grp. gram.:adj.}
\end{itemize}
Que se governa mal.
\section{Malroupido}
\begin{itemize}
\item {Grp. gram.:m.  e  adj.}
\end{itemize}
\begin{itemize}
\item {Proveniência:(De \textunderscore mal...\textunderscore  + \textunderscore roupa\textunderscore )}
\end{itemize}
O mesmo que \textunderscore maltrapilho\textunderscore .
\section{Mal-sadio}
\begin{itemize}
\item {Grp. gram.:adj.}
\end{itemize}
Que tem pouca saúde; malsão; adoentado:«\textunderscore sou mui franzina e mal-sadia.\textunderscore »Filinto.
\section{Malsão}
\begin{itemize}
\item {Grp. gram.:adj.}
\end{itemize}
\begin{itemize}
\item {Proveniência:(De \textunderscore mal...\textunderscore  + \textunderscore são\textunderscore )}
\end{itemize}
Doentio.
Convalescente; mal curado.
\section{Malsim}
\begin{itemize}
\item {Grp. gram.:m.}
\end{itemize}
\begin{itemize}
\item {Utilização:Ext.}
\end{itemize}
\begin{itemize}
\item {Grp. gram.:Adj.}
\end{itemize}
\begin{itemize}
\item {Proveniência:(Do hebr. \textunderscore meluxin\textunderscore , calumniador?)}
\end{itemize}
Fiscal alfandegário.
Zelador dos regulamentos policiaes.
Beleguim.
Espião; denunciante.
Que denuncía, que espía.
\section{Malsinação}
\begin{itemize}
\item {Grp. gram.:f.}
\end{itemize}
Acto ou effeito de malsinar.
\section{Malsinar}
\begin{itemize}
\item {Grp. gram.:v. t.}
\end{itemize}
\begin{itemize}
\item {Proveniência:(De \textunderscore malsim\textunderscore )}
\end{itemize}
Denunciar, sendo malsim.
Denunciar; calumniar.
Dar má interpretação a.
Agoirar mal de.
\section{Malsoante}
\begin{itemize}
\item {Grp. gram.:adj.}
\end{itemize}
\begin{itemize}
\item {Utilização:Fig.}
\end{itemize}
\begin{itemize}
\item {Proveniência:(De \textunderscore mal...\textunderscore  + \textunderscore soante\textunderscore )}
\end{itemize}
Que sôa mal.
Que escandaliza os ouvidos de pessôas honestas ou religiosas: \textunderscore palavras malsoantes\textunderscore .
\section{Malsoffrido}
\begin{itemize}
\item {Grp. gram.:adj.}
\end{itemize}
\begin{itemize}
\item {Proveniência:(De \textunderscore mal...\textunderscore  + \textunderscore soffrido\textunderscore )}
\end{itemize}
Que não é resignado; que é impaciente.
\section{Malsofrido}
\begin{itemize}
\item {Grp. gram.:adj.}
\end{itemize}
\begin{itemize}
\item {Proveniência:(De \textunderscore mal...\textunderscore  + \textunderscore sofrido\textunderscore )}
\end{itemize}
Que não é resignado; que é impaciente.
\section{Malsonância}
\begin{itemize}
\item {Grp. gram.:f.}
\end{itemize}
Qualidade de malsonante.
\section{Malsonante}
\begin{itemize}
\item {Grp. gram.:adj.}
\end{itemize}
O mesmo que \textunderscore malsoante\textunderscore .
\section{Malt}
\begin{itemize}
\item {Grp. gram.:m.}
\end{itemize}
\begin{itemize}
\item {Proveniência:(Ingl. \textunderscore malt\textunderscore )}
\end{itemize}
Cevada, que se faz germinar e secar, e que serve para a fabricação da cerveja.
\section{Malta}
\begin{itemize}
\item {Grp. gram.:f.}
\end{itemize}
Reunião de gente de baixa condição.
Súcia.
Caterva.
Reunião de trabalhadores, que se transportam juntamente, de um para outro lugar, em procura de trabalhos agrícolas.
Ciganagem.
Vida airada.
Tuna.
\textunderscore Casa de malta\textunderscore , casa onde vivem ou dormem, como em família, vários moços de fretes.
\section{Malta}
\begin{itemize}
\item {Grp. gram.:f.}
\end{itemize}
\begin{itemize}
\item {Proveniência:(Lat. \textunderscore maltha\textunderscore )}
\end{itemize}
Substância, gelatinosa e molle durante o estio, e dura em tempo frio, chamada tambem pez mineral.
\section{Maltagem}
\begin{itemize}
\item {Grp. gram.:f.}
\end{itemize}
Preparação do malt.
\section{Malte}
\begin{itemize}
\item {Grp. gram.:m.}
\end{itemize}
\begin{itemize}
\item {Proveniência:(Ingl. \textunderscore malt\textunderscore )}
\end{itemize}
Cevada, que se faz germinar e secar, e que serve para a fabricação da cerveja.
\section{Maltês}
\begin{itemize}
\item {Grp. gram.:adj.}
\end{itemize}
\begin{itemize}
\item {Grp. gram.:M.}
\end{itemize}
Relativo a Malta ou á Ordem militar de Malta.
Diz-se dos gatos cinzentos.
Cavalleiro de Malta.
Língua de Malta.
\section{Maltês}
\begin{itemize}
\item {Grp. gram.:m.}
\end{itemize}
\begin{itemize}
\item {Grp. gram.:Adj.}
\end{itemize}
\begin{itemize}
\item {Utilização:Prov.}
\end{itemize}
\begin{itemize}
\item {Utilização:trasm.}
\end{itemize}
Trabalhador, que vive em maltas, sem domicílio certo.
Finório.
Mentiroso.
\section{Maltesaria}
\begin{itemize}
\item {Grp. gram.:f.}
\end{itemize}
O mesmo que \textunderscore maltesia\textunderscore .
\section{Maltesia}
\begin{itemize}
\item {Grp. gram.:f.}
\end{itemize}
\begin{itemize}
\item {Proveniência:(De \textunderscore maltês\textunderscore )}
\end{itemize}
Rancho de malteses, trabalhadores.
\section{Maltha}
\begin{itemize}
\item {Grp. gram.:f.}
\end{itemize}
\begin{itemize}
\item {Proveniência:(Lat. \textunderscore maltha\textunderscore )}
\end{itemize}
Substância, gelatinosa e molle durante o estio, e dura em tempo frio, chamada tambem pez mineral.
\section{Malthusianismo}
\begin{itemize}
\item {Grp. gram.:m.}
\end{itemize}
Doutrina de Malthus.--Êste economista inglês, sustentando que a população cresce mais que a producção das substâncias alimentícias, defendeu que se devia restringir por qualquer fórma a reproducção da espécie humana.
\section{Malthusiano}
\begin{itemize}
\item {Grp. gram.:adj.}
\end{itemize}
\begin{itemize}
\item {Grp. gram.:M.}
\end{itemize}
Relativo ao systema económico de Malthus.
Sectário da doutrina de Malthus.
\section{Maltina}
\begin{itemize}
\item {Grp. gram.:f.}
\end{itemize}
Alcalóide, que se extrai do malt.
\section{Maltosta}
\begin{itemize}
\item {Grp. gram.:f.}
\end{itemize}
\begin{itemize}
\item {Proveniência:(Do b. lat. \textunderscore mala tolta\textunderscore )}
\end{itemize}
Imposto de 48 reis por tonel, que pagavam os que no Pôrto embarcavam vinho.
\section{Maltrapido}
\begin{itemize}
\item {Grp. gram.:adj.}
\end{itemize}
\begin{itemize}
\item {Proveniência:(De \textunderscore mal...\textunderscore  + \textunderscore trapo\textunderscore )}
\end{itemize}
Indivíduo mal vestido, esfarrapado; desprezível.
\section{Maltrapilho}
\begin{itemize}
\item {Grp. gram.:m.  e  adj.}
\end{itemize}
\begin{itemize}
\item {Proveniência:(De \textunderscore mal...\textunderscore  + \textunderscore trapo\textunderscore )}
\end{itemize}
Indivíduo mal vestido, esfarrapado; desprezível.
\section{Maltratar}
\begin{itemize}
\item {Grp. gram.:v. t.}
\end{itemize}
\begin{itemize}
\item {Proveniência:(De \textunderscore tratar\textunderscore  + \textunderscore tratar\textunderscore )}
\end{itemize}
Tratar mal: \textunderscore maltratar os animaes\textunderscore .
Receber mal.
Vexar.
Damnificar: \textunderscore o gado maltratou a seara\textunderscore .
Lesar physicamente.
Estragar.
\section{Maltreito}
\begin{itemize}
\item {Grp. gram.:adj.}
\end{itemize}
\begin{itemize}
\item {Utilização:Ant.}
\end{itemize}
O mesmo que \textunderscore maltrido\textunderscore . Cf. \textunderscore Port. Mon. Hist.\textunderscore , \textunderscore Script.\textunderscore , 280.
\section{Maltrido}
\begin{itemize}
\item {Grp. gram.:adj.}
\end{itemize}
\begin{itemize}
\item {Utilização:Ant.}
\end{itemize}
\begin{itemize}
\item {Proveniência:(Do lat. \textunderscore male\textunderscore  + \textunderscore tritus\textunderscore )}
\end{itemize}
Maltratado; ferido.
\section{Maluca}
\begin{itemize}
\item {Grp. gram.:f.}
\end{itemize}
\begin{itemize}
\item {Proveniência:(De \textunderscore maluco\textunderscore )}
\end{itemize}
Mulher sem juízo; doida.
Mulher estouvada ou leviana.
\section{Malucar}
\begin{itemize}
\item {Grp. gram.:v. i.}
\end{itemize}
\begin{itemize}
\item {Proveniência:(De \textunderscore maluco\textunderscore )}
\end{itemize}
Dizer ou praticar maluquices.
Andar pensativo ou scismático.
\section{Maluco}
\begin{itemize}
\item {Grp. gram.:m.}
\end{itemize}
\begin{itemize}
\item {Utilização:Burl.}
\end{itemize}
\begin{itemize}
\item {Grp. gram.:Adj.}
\end{itemize}
Doido, mentecapto.
Indivíduo apalermado.
Doidivanas.
Aquelle que parece doido.
Extravagante.
Moéda portuguesa de bronze, cunhada na ilha Terceira.
Pataco.
Tonto; adoidado; extravagante.
\section{Maludo}
\begin{itemize}
\item {Grp. gram.:adj.}
\end{itemize}
\begin{itemize}
\item {Utilização:Bras. de Minas}
\end{itemize}
O mesmo que \textunderscore valente\textunderscore .
\section{Malundas}
\begin{itemize}
\item {Grp. gram.:m. pl.}
\end{itemize}
Povo sertanejo de Angola.
\section{Malunga}
\begin{itemize}
\item {Grp. gram.:f.}
\end{itemize}
\begin{itemize}
\item {Utilização:T. de Angola}
\end{itemize}
Espécie de manilha.
\section{Malungo}
\begin{itemize}
\item {Grp. gram.:m.}
\end{itemize}
\begin{itemize}
\item {Proveniência:(T. \textunderscore afr.\textunderscore )}
\end{itemize}
Camarada, companheiro.
Nome, com que reciprocamente se designam os Negros, que no mesmo navio saem da África.
\section{Maluqueira}
\begin{itemize}
\item {Grp. gram.:f.}
\end{itemize}
Doença de maluco; maluquice; estado de quem é maluco.
\section{Maluquês}
\begin{itemize}
\item {Grp. gram.:m.}
\end{itemize}
\begin{itemize}
\item {Utilização:Des.}
\end{itemize}
Habitante de Maluco, (hoje \textunderscore Molucas\textunderscore ).
\section{Maluquice}
\begin{itemize}
\item {Grp. gram.:f.}
\end{itemize}
Acto ou dito, próprio de maluco.
Scisma.
\section{Malúrdia}
\begin{itemize}
\item {Grp. gram.:f.}
\end{itemize}
\begin{itemize}
\item {Utilização:Gír.}
\end{itemize}
Mãe.
\section{Mal-usar}
\begin{itemize}
\item {Grp. gram.:v. t.}
\end{itemize}
Usar mal, abusar.
\section{Maluta}
\begin{itemize}
\item {Grp. gram.:f.}
\end{itemize}
\begin{itemize}
\item {Utilização:Pop.}
\end{itemize}
O mesmo que \textunderscore luta\textunderscore : \textunderscore os garotos andavam á maluta\textunderscore .
(Da coalescência de \textunderscore uma\textunderscore  com \textunderscore luta\textunderscore : (\textunderscore u\textunderscore )\textunderscore ma\textunderscore  + \textunderscore luta\textunderscore )
\section{Maluvo}
\begin{itemize}
\item {Grp. gram.:m.}
\end{itemize}
\begin{itemize}
\item {Proveniência:(T. \textunderscore quimb.\textunderscore )}
\end{itemize}
Bebida fermentada, produzida pela seiva açucarada do bordão, e muito apreciada pelos Indígenas africanos, contendo 3 a 5 por cento de álcool.
\section{Malva}
\begin{itemize}
\item {Grp. gram.:f.}
\end{itemize}
\begin{itemize}
\item {Utilização:Pop.}
\end{itemize}
\begin{itemize}
\item {Utilização:Gír.}
\end{itemize}
\begin{itemize}
\item {Grp. gram.:Pl. Loc.}
\end{itemize}
\begin{itemize}
\item {Utilização:fam.}
\end{itemize}
\begin{itemize}
\item {Proveniência:(Lat. \textunderscore malva\textunderscore )}
\end{itemize}
Gênero de plantas emollientes, de que há muitas espécies.
Guarda-chuva.
Chapéu molle.
\textunderscore Ir para as malvas\textunderscore , ir para o cemitério; morrer.
\section{Malváceas}
\begin{itemize}
\item {Grp. gram.:f. pl.}
\end{itemize}
\begin{itemize}
\item {Proveniência:(De \textunderscore malváceo\textunderscore )}
\end{itemize}
Família de plantas, que tem por typo a malva.
\section{Malvaceira}
\begin{itemize}
\item {Grp. gram.:f.}
\end{itemize}
\begin{itemize}
\item {Utilização:Bras}
\end{itemize}
Espécie de malva ornamental.
\section{Malváceo}
\begin{itemize}
\item {Grp. gram.:adj.}
\end{itemize}
\begin{itemize}
\item {Proveniência:(Lat. \textunderscore malvaceus\textunderscore )}
\end{itemize}
Relativo ou semelhante á malva.
\section{Malvadamente}
\begin{itemize}
\item {Grp. gram.:adv.}
\end{itemize}
Á maneira de malvado.
\section{Malvadez}
\begin{itemize}
\item {Grp. gram.:f.}
\end{itemize}
Qualidade ou acto de malvado.
\section{Malvadeza}
\begin{itemize}
\item {fónica:dê}
\end{itemize}
\begin{itemize}
\item {Grp. gram.:f.}
\end{itemize}
\begin{itemize}
\item {Utilização:Prov.}
\end{itemize}
O mesmo que \textunderscore malvadez\textunderscore .
(Colhido na Guarda)
\section{Malvado}
\begin{itemize}
\item {Grp. gram.:m.  e  adj.}
\end{itemize}
\begin{itemize}
\item {Proveniência:(Do lat. \textunderscore male\textunderscore  + \textunderscore levatus\textunderscore )}
\end{itemize}
O que pratica actos infamantes ou criminosos, ou é capaz de os praticar.
\section{Malvaísco}
\begin{itemize}
\item {Grp. gram.:m.}
\end{itemize}
\begin{itemize}
\item {Proveniência:(Do lat. \textunderscore mavaviscus\textunderscore , us. por Apuleio. Bento Pereira registou \textunderscore malvaiscus\textunderscore , cuja authenticidade desconheço)}
\end{itemize}
Planta malvácea, de fôlhas e raízes medicinaes, e de que a altheia é uma espécie.
\section{Malvalistro}
\begin{itemize}
\item {Grp. gram.:m.}
\end{itemize}
Planta malvácea do Brasil.
\section{Malvar}
\begin{itemize}
\item {Grp. gram.:m.}
\end{itemize}
Terreno, em que crescem malvas.
\section{Malva-rosa}
\begin{itemize}
\item {Grp. gram.:f.}
\end{itemize}
Planta malvácea, muito aromática, (\textunderscore altheaa rosea\textunderscore ).
\section{Malvasia}
\begin{itemize}
\item {Grp. gram.:f.}
\end{itemize}
\begin{itemize}
\item {Proveniência:(De \textunderscore Malvasia\textunderscore , n. p.)}
\end{itemize}
Variedade de uva odorífera e doce.
Vinho, feito dessa qualidade de uva.
\section{Malvasia-da-passa}
\begin{itemize}
\item {Grp. gram.:f.}
\end{itemize}
O mesmo que \textunderscore malvasia-fina\textunderscore .
\section{Malvasia-do-bairro}
\begin{itemize}
\item {Grp. gram.:f.}
\end{itemize}
O mesmo que \textunderscore almáfego\textunderscore , em Ourem.
\section{Malvasia-fina}
\begin{itemize}
\item {Grp. gram.:f.}
\end{itemize}
Casta de uva beirôa.
\section{Malvasia-grossa}
\begin{itemize}
\item {Grp. gram.:f.}
\end{itemize}
Casta de uva branca, na região do Doiro e no Alentejo.
\section{Malvasia-miúda}
\begin{itemize}
\item {Grp. gram.:f.}
\end{itemize}
Casta de uva branca, mais pequena que a malvasia-grossa.
\section{Malvasia-penaguiota}
\begin{itemize}
\item {Grp. gram.:f.}
\end{itemize}
\begin{itemize}
\item {Proveniência:(De \textunderscore malvasia\textunderscore  + \textunderscore Penaguião\textunderscore , n. p.)}
\end{itemize}
Casta de uva, na região do Doiro.
\section{Malvasia-penicheira}
\begin{itemize}
\item {Grp. gram.:f.}
\end{itemize}
Casta de uva trasmontana.
\section{Malvasia-tinta}
\begin{itemize}
\item {Grp. gram.:f.}
\end{itemize}
Casta de uva de Penafiel.
\section{Malvasia-vermelha}
\begin{itemize}
\item {Grp. gram.:f.}
\end{itemize}
Casta de uva, cujos bagos, doces e aromáticos, tem côr de rosa muito viva.
\section{Malveiro}
\begin{itemize}
\item {Grp. gram.:m.}
\end{itemize}
\begin{itemize}
\item {Utilização:Prov.}
\end{itemize}
\begin{itemize}
\item {Utilização:trasm.}
\end{itemize}
\begin{itemize}
\item {Proveniência:(De \textunderscore malva\textunderscore )}
\end{itemize}
Sarampo benigno.
\section{Malventuroso}
\begin{itemize}
\item {Grp. gram.:adj.}
\end{itemize}
O mesmo que \textunderscore malaventurado\textunderscore .
\section{Malversão}
\begin{itemize}
\item {Grp. gram.:f.}
\end{itemize}
Acto ou effeito de malversar.
\section{Malversador}
\begin{itemize}
\item {Grp. gram.:m.}
\end{itemize}
Aquelle que malversa.
\section{Malversar}
\begin{itemize}
\item {Grp. gram.:v. t.}
\end{itemize}
\begin{itemize}
\item {Proveniência:(Do lat. \textunderscore male\textunderscore  + \textunderscore versare\textunderscore )}
\end{itemize}
Administrar mal.
Fazer subtracções abusivas de; dilapidar.
\section{Malvis}
\begin{itemize}
\item {Grp. gram.:m.}
\end{itemize}
Pássaro dentirostro, (\textunderscore turdus iliacus\textunderscore ).
(Cp. cast. \textunderscore malvis\textunderscore )
\section{Malvo}
\begin{itemize}
\item {Grp. gram.:m.}
\end{itemize}
\begin{itemize}
\item {Utilização:Bras}
\end{itemize}
Fibras têxteis de certas árvores, (\textunderscore bauhinia\textunderscore ).
\section{Mama}
\begin{itemize}
\item {Grp. gram.:f.}
\end{itemize}
\begin{itemize}
\item {Utilização:Ant.}
\end{itemize}
\begin{itemize}
\item {Proveniência:(Lat. \textunderscore mamma\textunderscore )}
\end{itemize}
Parte glandular do peito da mulher e das fêmeas dos animaes, a qual serve para a secreção e excreção do leite.
Leite, que as crianças sugam do seio da mãe ou da ama.
O mesmo que \textunderscore mama-de-mulher\textunderscore .
\section{Mamã}
\begin{itemize}
\item {Grp. gram.:f.}
\end{itemize}
\begin{itemize}
\item {Utilização:Infant.}
\end{itemize}
\begin{itemize}
\item {Proveniência:(T. onom. da linguagem das crianças, em muitas línguas)}
\end{itemize}
O mesmo que \textunderscore mãe\textunderscore .
\section{Mama-de-cachorra}
\begin{itemize}
\item {Grp. gram.:f.}
\end{itemize}
Árvore fructífera do Brasil, (\textunderscore eugenia formosa\textunderscore , Camb.).
\section{Mamadeira}
\begin{itemize}
\item {Grp. gram.:f.}
\end{itemize}
\begin{itemize}
\item {Proveniência:(De \textunderscore mamar\textunderscore )}
\end{itemize}
Instrumento, para extrahir leite do peito da mulher, ou para amamentar artificialmente crianças.
\section{Mama-de-mulher}
\begin{itemize}
\item {Grp. gram.:f.}
\end{itemize}
\begin{itemize}
\item {Utilização:Ant.}
\end{itemize}
Oiteiro; accidente de terreno com a configuração de um peito de mulher. Cf. \textunderscore Roteiro do Mar-Vermelho\textunderscore , (passim).
\section{Mamado}
\begin{itemize}
\item {Grp. gram.:adj.}
\end{itemize}
\begin{itemize}
\item {Utilização:Chul.}
\end{itemize}
Desapontado, embaçado; desilludido.
\section{Mamadura}
\begin{itemize}
\item {Grp. gram.:f.}
\end{itemize}
Acto de mamar; tempo que dura a amamentação.
\section{Mamãe}
\begin{itemize}
\item {Grp. gram.:f.}
\end{itemize}
\begin{itemize}
\item {Utilização:Bras}
\end{itemize}
O mesmo que \textunderscore maman\textunderscore .
\section{Mamal}
\begin{itemize}
\item {Grp. gram.:adj.}
\end{itemize}
\begin{itemize}
\item {Proveniência:(Lat. \textunderscore mammalis\textunderscore )}
\end{itemize}
O mesmo que \textunderscore mamário\textunderscore .
\section{Mamalhudo}
\begin{itemize}
\item {Grp. gram.:adj.}
\end{itemize}
\begin{itemize}
\item {Utilização:Pleb.}
\end{itemize}
\begin{itemize}
\item {Proveniência:(Do rad. \textunderscore mama\textunderscore )}
\end{itemize}
Que tem grandes mamas.
\section{Mamaltar}
\begin{itemize}
\item {Grp. gram.:m.}
\end{itemize}
\begin{itemize}
\item {Proveniência:(De \textunderscore mama\textunderscore  + \textunderscore altar\textunderscore ?)}
\end{itemize}
O mesmo que \textunderscore mamôa\textunderscore .
\section{Mamaluco}
\begin{itemize}
\item {Grp. gram.:m.}
\end{itemize}
\begin{itemize}
\item {Utilização:Bras}
\end{itemize}
Filho de mulher indígena e de europeu.
(Do tupi \textunderscore mamã\textunderscore , misturar, e \textunderscore ruca\textunderscore , tirar)
\section{Mamaluco}
\begin{itemize}
\item {Grp. gram.:m.}
\end{itemize}
\begin{itemize}
\item {Utilização:Bras}
\end{itemize}
Espécie de árvore de construcção.
\section{Maman}
\begin{itemize}
\item {Grp. gram.:f.}
\end{itemize}
\begin{itemize}
\item {Utilização:Infant.}
\end{itemize}
\begin{itemize}
\item {Proveniência:(T. onom. da linguagem das crianças, em muitas línguas)}
\end{itemize}
O mesmo que \textunderscore mãe\textunderscore .
\section{Mamanga}
\begin{itemize}
\item {Grp. gram.:f.}
\end{itemize}
Planta leguminosa do Brasil, (\textunderscore cassia medica\textunderscore ).
\section{Mamão}
\begin{itemize}
\item {Grp. gram.:adj.}
\end{itemize}
\begin{itemize}
\item {Grp. gram.:M.}
\end{itemize}
\begin{itemize}
\item {Utilização:Prov.}
\end{itemize}
\begin{itemize}
\item {Utilização:Bras. do N}
\end{itemize}
\begin{itemize}
\item {Grp. gram.:M.  e  adj.}
\end{itemize}
\begin{itemize}
\item {Utilização:Prov.}
\end{itemize}
\begin{itemize}
\item {Utilização:minh.}
\end{itemize}
\begin{itemize}
\item {Proveniência:(De \textunderscore mamar\textunderscore )}
\end{itemize}
Que mama muito.
Que depois de desmamado, ainda pede mama.
Aquelle que mama muito ou que ainda quer mama, findo o tempo que se destinou á amamentação.
Burro de um anno.
Rebento, que rouba á planta o suco alimentício.
Fruto do mamoeiro.
O mesmo que \textunderscore mamoeiro\textunderscore .
Bezerro de um anno.
Rês, que ainda mama.
Palerma.
\section{Mamãozeiro}
\begin{itemize}
\item {Grp. gram.:m.}
\end{itemize}
O mesmo que \textunderscore mamoeiro\textunderscore . Cf. Gonç. Dias, 311.
\section{Mamar}
\begin{itemize}
\item {Grp. gram.:v. t.}
\end{itemize}
\begin{itemize}
\item {Utilização:Fig.}
\end{itemize}
\begin{itemize}
\item {Utilização:Pop.}
\end{itemize}
\begin{itemize}
\item {Grp. gram.:V. i.}
\end{itemize}
\begin{itemize}
\item {Proveniência:(De \textunderscore mama\textunderscore )}
\end{itemize}
Sugar (o leite) da mãe ou da ama.
Extorquir.
Enganar; ludibriar.
Comer com soffreguidão, devorar:«\textunderscore a cada jantar mamava um boi.\textunderscore »Macedo, \textunderscore Motim\textunderscore , II, 52.
Sugar o leite materno ou da ama: \textunderscore êste pequeno ainda mama\textunderscore .
\section{Mamário}
\begin{itemize}
\item {Grp. gram.:adj.}
\end{itemize}
\begin{itemize}
\item {Utilização:Bot.}
\end{itemize}
Relativo á mama.
Diz-se de qualquer parte de um vegetal, na qual apparecem elevações ou mamillos.
\section{Mamarracho}
\begin{itemize}
\item {Grp. gram.:m.}
\end{itemize}
\begin{itemize}
\item {Proveniência:(T. cast.)}
\end{itemize}
Mau pintor; pintamonos. Cf. Castilho, \textunderscore D. Quixote\textunderscore , II, 360.
\section{Mamalogia}
\begin{itemize}
\item {Grp. gram.:f.}
\end{itemize}
\begin{itemize}
\item {Proveniência:(Do gr. \textunderscore mamma\textunderscore  + \textunderscore logos\textunderscore )}
\end{itemize}
Estudo tratado á cêrca das mamas.
\section{Mamalógico}
\begin{itemize}
\item {Grp. gram.:adj.}
\end{itemize}
Relativo á mamalogia.
\section{Mamalogista}
\begin{itemize}
\item {Grp. gram.:m.}
\end{itemize}
Aquele que é versado em mamalogia.
\section{Mamaurana}
\begin{itemize}
\item {Grp. gram.:f.}
\end{itemize}
\begin{itemize}
\item {Utilização:Bras. do N}
\end{itemize}
Nome de duas espécies de árvores, cujo alburno filamentoso serve para cordas e para calafetar navios.
\section{Mamba}
\begin{itemize}
\item {Grp. gram.:f.}
\end{itemize}
Grande serpente venenosa da África do Sul.
\section{Mambar}
\begin{itemize}
\item {Grp. gram.:m.}
\end{itemize}
Preto meio civilizado, que volta das colónias portuguesas, para o sertão. Cf. Serpa Pinto, II. 18.
\section{Mambarés}
\begin{itemize}
\item {Grp. gram.:m. pl.}
\end{itemize}
Índios do Brazil, em Mato-Grosso.
\section{Mambisses}
\begin{itemize}
\item {Grp. gram.:m. pl.}
\end{itemize}
Indígenas da ilha de Cuba.
\section{Mambu}
\begin{itemize}
\item {Grp. gram.:m.}
\end{itemize}
\begin{itemize}
\item {Utilização:Ant.}
\end{itemize}
O mesmo que \textunderscore bambu\textunderscore .
\section{Mambuca}
\begin{itemize}
\item {Grp. gram.:f.}
\end{itemize}
Espécie de abelha do Brasil.
\section{Mambude}
\begin{itemize}
\item {Grp. gram.:m.}
\end{itemize}
Árvore angolense.
\section{Mambumbá}
\begin{itemize}
\item {Grp. gram.:m.}
\end{itemize}
Arbusto africano, de propriedades purgativas.
\section{Mambunde}
\begin{itemize}
\item {Grp. gram.:m.}
\end{itemize}
Árvore da Guiné portuguesa.
\section{Mambungo}
\begin{itemize}
\item {Grp. gram.:m.}
\end{itemize}
Árvore de Angola.
\section{Mamé}
\begin{itemize}
\item {Grp. gram.:m.}
\end{itemize}
\begin{itemize}
\item {Utilização:T. de Cabo-Verde}
\end{itemize}
O mesmo que \textunderscore anona\textunderscore ^1.
\section{Mamelado}
\begin{itemize}
\item {Grp. gram.:adj.}
\end{itemize}
Em que há mamelões; constituído por mamelões:«\textunderscore as térmites das clareiras construem montes mamelados...\textunderscore »Serpa Pinto, I, 217.
\section{Mamelão}
\begin{itemize}
\item {Grp. gram.:m.}
\end{itemize}
\begin{itemize}
\item {Utilização:Neol.}
\end{itemize}
\begin{itemize}
\item {Proveniência:(Fr. \textunderscore mamelon\textunderscore )}
\end{itemize}
Montículo insulado.
\section{Mamelonado}
\begin{itemize}
\item {Grp. gram.:adj.}
\end{itemize}
\begin{itemize}
\item {Utilização:bras}
\end{itemize}
\begin{itemize}
\item {Utilização:Neol.}
\end{itemize}
Que tem fórma de mamelão.
\section{Mameluco}
\begin{itemize}
\item {Grp. gram.:m.}
\end{itemize}
\begin{itemize}
\item {Proveniência:(Do ár. \textunderscore mamluk\textunderscore )}
\end{itemize}
Soldado de uma tropa egýpcia, formada primitivamente de escravos.
\section{Mamengás}
\begin{itemize}
\item {Grp. gram.:m. pl.}
\end{itemize}
Indígenas brasileiros da região do Amazonas.
\section{Mamífero}
\begin{itemize}
\item {Grp. gram.:adj.}
\end{itemize}
\begin{itemize}
\item {Grp. gram.:M. pl.}
\end{itemize}
\begin{itemize}
\item {Proveniência:(Do lat. \textunderscore mamma\textunderscore  + \textunderscore ferre\textunderscore )}
\end{itemize}
Que tem mamas.
Classe de animaes vertebrados, que occupa o primeiro lugar na escala zoológica.
\section{Mamiforme}
\begin{itemize}
\item {Grp. gram.:adj.}
\end{itemize}
\begin{itemize}
\item {Proveniência:(Do lat. \textunderscore mamma\textunderscore  + \textunderscore forma\textunderscore )}
\end{itemize}
Que tem fórma de mama.
\section{Mamila}
\begin{itemize}
\item {Grp. gram.:f.}
\end{itemize}
\begin{itemize}
\item {Proveniência:(Lat. \textunderscore mamilla\textunderscore )}
\end{itemize}
Bico do peito.
O mesmo que \textunderscore mâmula\textunderscore .
\section{Mamilão}
\begin{itemize}
\item {Grp. gram.:m.}
\end{itemize}
Termo, proposto, no interesse da língua portuguesa, em substituição do bárbaro \textunderscore mamelão\textunderscore . Cf. Pacheco, \textunderscore Promptuário\textunderscore .
(Cp. \textunderscore mamillo\textunderscore )
\section{Mamilar}
\begin{itemize}
\item {Grp. gram.:m.}
\end{itemize}
\begin{itemize}
\item {Proveniência:(Lat. \textunderscore mamillare\textunderscore )}
\end{itemize}
Espartilho, faixa, ou lenço, com que as mulheres velam o peito.
\section{Mamilar}
\begin{itemize}
\item {Grp. gram.:adj.}
\end{itemize}
Relativo á mamila.
Que tem fórma de mamila.
\section{Mamilária}
\begin{itemize}
\item {Grp. gram.:f.}
\end{itemize}
\begin{itemize}
\item {Proveniência:(De \textunderscore mamilo\textunderscore )}
\end{itemize}
Gênero de cactos.
\section{Mamilho}
\begin{itemize}
\item {Grp. gram.:m.}
\end{itemize}
Proeminência de metal, na superfície interna das bocas de fogo. Cf. Leoni, \textunderscore Diccion. de Artilh.\textunderscore , inédito.
(Cp. \textunderscore mamillo\textunderscore )
\section{Mamilla}
\begin{itemize}
\item {Grp. gram.:f.}
\end{itemize}
\begin{itemize}
\item {Proveniência:(Lat. \textunderscore mamilla\textunderscore )}
\end{itemize}
Bico do peito.
O mesmo que \textunderscore mâmmula\textunderscore .
\section{Mamillão}
\begin{itemize}
\item {Grp. gram.:m.}
\end{itemize}
Termo, proposto, no interesse da língua portuguesa, em substituição do bárbaro \textunderscore mamelão\textunderscore . Cf. Pacheco, \textunderscore Promptuário\textunderscore .
(Cp. \textunderscore mamillo\textunderscore )
\section{Mamillar}
\begin{itemize}
\item {Grp. gram.:m.}
\end{itemize}
\begin{itemize}
\item {Proveniência:(Lat. \textunderscore mamillare\textunderscore )}
\end{itemize}
Espartilho, faixa, ou lenço, com que as mulheres velam o peito.
\section{Mamillar}
\begin{itemize}
\item {Grp. gram.:adj.}
\end{itemize}
Relativo á mamilla.
Que tem fórma de mamilla.
\section{Mamillária}
\begin{itemize}
\item {Grp. gram.:f.}
\end{itemize}
\begin{itemize}
\item {Proveniência:(De \textunderscore mamillo\textunderscore )}
\end{itemize}
Gênero de cactos.
\section{Mamillo}
\begin{itemize}
\item {Grp. gram.:m.}
\end{itemize}
\begin{itemize}
\item {Utilização:Des.}
\end{itemize}
\begin{itemize}
\item {Utilização:Ant.}
\end{itemize}
(V.mamilla)
Pequeno tumor ou borbulha sem cabeça.
Oiteiro, que termina em ponta ou bico. Cf. \textunderscore Roteiro do Mar Vermelho\textunderscore , 143.
\section{Mamilloso}
\begin{itemize}
\item {Grp. gram.:adj.}
\end{itemize}
Que tem mamilla ou fórma de mamilla.
\section{Mamilo}
\begin{itemize}
\item {Grp. gram.:m.}
\end{itemize}
\begin{itemize}
\item {Utilização:Des.}
\end{itemize}
\begin{itemize}
\item {Utilização:Ant.}
\end{itemize}
(V.mamila)
Pequeno tumor ou borbulha sem cabeça.
Oiteiro, que termina em ponta ou bico. Cf. \textunderscore Roteiro do Mar Vermelho\textunderscore , 143.
\section{Mamiloso}
\begin{itemize}
\item {Grp. gram.:adj.}
\end{itemize}
Que tem mamila ou fórma de mamila.
\section{Maminha}
\begin{itemize}
\item {Grp. gram.:f.}
\end{itemize}
\begin{itemize}
\item {Utilização:Ant.}
\end{itemize}
O leite da mama.
\section{Mamirá}
\begin{itemize}
\item {Grp. gram.:m.}
\end{itemize}
\begin{itemize}
\item {Utilização:Bras}
\end{itemize}
Planta medicinal.
\section{Mamite}
\begin{itemize}
\item {Grp. gram.:f.}
\end{itemize}
Inflamação nas mamas.
\section{Mamixi}
\begin{itemize}
\item {Grp. gram.:m.}
\end{itemize}
Fruto silvestre do Brasil.
\section{Mammalogia}
\begin{itemize}
\item {Grp. gram.:f.}
\end{itemize}
\begin{itemize}
\item {Proveniência:(Do gr. \textunderscore mamma\textunderscore  + \textunderscore logos\textunderscore )}
\end{itemize}
Estudo tratado á cêrca das mamas.
\section{Mammalógico}
\begin{itemize}
\item {Grp. gram.:adj.}
\end{itemize}
Relativo á mammalogia.
\section{Mammalogista}
\begin{itemize}
\item {Grp. gram.:m.}
\end{itemize}
Aquelle que é versado em mammalogia.
\section{Mammífero}
\begin{itemize}
\item {Grp. gram.:adj.}
\end{itemize}
\begin{itemize}
\item {Grp. gram.:M. pl.}
\end{itemize}
\begin{itemize}
\item {Proveniência:(Do lat. \textunderscore mamma\textunderscore  + \textunderscore ferre\textunderscore )}
\end{itemize}
Que tem mamas.
Classe de animaes vertebrados, que occupa o primeiro lugar na escala zoológica.
\section{Mammiforme}
\begin{itemize}
\item {Grp. gram.:adj.}
\end{itemize}
\begin{itemize}
\item {Proveniência:(Do lat. \textunderscore mamma\textunderscore  + \textunderscore forma\textunderscore )}
\end{itemize}
Que tem fórma de mama.
\section{Mammologia}
\begin{itemize}
\item {Grp. gram.:f.}
\end{itemize}
O mesmo que \textunderscore mammalogia\textunderscore .
\section{Mammona}
\begin{itemize}
\item {Grp. gram.:m.}
\end{itemize}
\begin{itemize}
\item {Proveniência:(Lat. \textunderscore mammona\textunderscore )}
\end{itemize}
Designação do deus das riquezas na mythologia phenícia e sýria.
Divindade, adorada antigamente em Malaca. Cf. \textunderscore Peregrinação\textunderscore , CXCV.
\section{Mâmmula}
\begin{itemize}
\item {Grp. gram.:f.}
\end{itemize}
\begin{itemize}
\item {Utilização:Hist. nat.}
\end{itemize}
\begin{itemize}
\item {Proveniência:(Lat. \textunderscore mammula\textunderscore )}
\end{itemize}
Protuberância.
\section{Mammuthe}
\begin{itemize}
\item {Grp. gram.:m.}
\end{itemize}
(V.mamute)
\section{Mamologia}
\begin{itemize}
\item {Grp. gram.:f.}
\end{itemize}
O mesmo que \textunderscore mamalogia\textunderscore .
\section{Mamôa}
\begin{itemize}
\item {Grp. gram.:f.}
\end{itemize}
\begin{itemize}
\item {Utilização:Burl.}
\end{itemize}
\begin{itemize}
\item {Proveniência:(Do rad. de \textunderscore mama\textunderscore )}
\end{itemize}
Fruto do mamoeiro.
Oiteiro, de aspecto análogo ao de um seio de mulher.
Montículo artificial ou monumental, de origem prehistórica.
Mama grande.
\section{Mamoca}
\begin{itemize}
\item {Grp. gram.:f.}
\end{itemize}
\begin{itemize}
\item {Utilização:Fam.}
\end{itemize}
Mama, que ainda não attingiu desenvolvimento completo.
\section{Mamoeiro}
\begin{itemize}
\item {Grp. gram.:m.}
\end{itemize}
\begin{itemize}
\item {Proveniência:(De \textunderscore mamôa\textunderscore )}
\end{itemize}
Árvore papaiácea da África e da América.
\section{Mamoínha}
\begin{itemize}
\item {Grp. gram.:f.}
\end{itemize}
\begin{itemize}
\item {Utilização:Prov.}
\end{itemize}
Pequeno oiteiro ou pequena mamôa.
\section{Mamola}
\begin{itemize}
\item {Grp. gram.:f.}
\end{itemize}
\begin{itemize}
\item {Utilização:Prov.}
\end{itemize}
\begin{itemize}
\item {Utilização:trasm.}
\end{itemize}
\begin{itemize}
\item {Proveniência:(De \textunderscore mamar\textunderscore )}
\end{itemize}
Sinecura; coisa de que se tira facilmente grande resultado.
\section{Mamona}
\begin{itemize}
\item {Grp. gram.:f.}
\end{itemize}
\begin{itemize}
\item {Proveniência:(T. \textunderscore bras.\textunderscore )}
\end{itemize}
Semente do rícino; carrapateiro.
\section{Mamona}
\begin{itemize}
\item {Grp. gram.:f.}
\end{itemize}
\begin{itemize}
\item {Proveniência:(De \textunderscore mama\textunderscore )}
\end{itemize}
Peixe dos Açores.
\section{Mamona}
\begin{itemize}
\item {Grp. gram.:m.}
\end{itemize}
\begin{itemize}
\item {Proveniência:(Lat. \textunderscore mammona\textunderscore )}
\end{itemize}
Designação do deus das riquezas na mitologia fenícia e síria.
Divindade, adorada antigamente em Malaca. Cf. \textunderscore Peregrinação\textunderscore , CXCV.
\section{Mamoneira}
\begin{itemize}
\item {Grp. gram.:f.}
\end{itemize}
\begin{itemize}
\item {Proveniência:(De \textunderscore mamona\textunderscore ^1)}
\end{itemize}
Planta euphorbiácea, (\textunderscore ricinus communis\textunderscore ).
\section{Mamoneiro}
\begin{itemize}
\item {Grp. gram.:m.}
\end{itemize}
\begin{itemize}
\item {Proveniência:(De \textunderscore mamona\textunderscore ^1)}
\end{itemize}
Planta euphorbiácea, (\textunderscore palma christi\textunderscore ), o mesmo que \textunderscore carrapateiro\textunderscore  ou \textunderscore rícino\textunderscore .
\section{Mamoninho-bravo}
\begin{itemize}
\item {Grp. gram.:m.}
\end{itemize}
\begin{itemize}
\item {Utilização:Bras}
\end{itemize}
O mesmo que \textunderscore estramónio\textunderscore , (\textunderscore datura stramonium\textunderscore ).
\section{Mamoso}
\begin{itemize}
\item {Grp. gram.:adj.}
\end{itemize}
\begin{itemize}
\item {Utilização:Fig.}
\end{itemize}
\begin{itemize}
\item {Proveniência:(Lat. \textunderscore mammosus\textunderscore )}
\end{itemize}
Que tem mamas; mamudo.
Que tem fórma de mama ou de mamillo.
Arredondado; boleado:«\textunderscore ...o racimo ampli-mamoso...\textunderscore »Castilho, \textunderscore Geórgicas\textunderscore , 81.
\section{Mamota}
\begin{itemize}
\item {Grp. gram.:m.  e  f.}
\end{itemize}
\begin{itemize}
\item {Utilização:Pop.}
\end{itemize}
\begin{itemize}
\item {Utilização:Prov.}
\end{itemize}
\begin{itemize}
\item {Utilização:trasm.}
\end{itemize}
Pessoa atoleimada, aparvalhada.
Castanha cozida e sem sal.
(Cp. \textunderscore mamote\textunderscore )
\section{Mamota}
\begin{itemize}
\item {Grp. gram.:f.}
\end{itemize}
(V.jaracatiá)
\section{Mamote}
\begin{itemize}
\item {Grp. gram.:m.}
\end{itemize}
\begin{itemize}
\item {Utilização:Bras}
\end{itemize}
\begin{itemize}
\item {Grp. gram.:Adj.}
\end{itemize}
\begin{itemize}
\item {Utilização:Ant.}
\end{itemize}
\begin{itemize}
\item {Utilização:Chul.}
\end{itemize}
\begin{itemize}
\item {Proveniência:(De \textunderscore mama\textunderscore )}
\end{itemize}
Bezerro, que ainda mama.
Ridículo; tolo.
\section{Mamoto}
\begin{itemize}
\item {fónica:mô}
\end{itemize}
\begin{itemize}
\item {Grp. gram.:m.}
\end{itemize}
\begin{itemize}
\item {Utilização:Prov.}
\end{itemize}
\begin{itemize}
\item {Utilização:trasm.}
\end{itemize}
Rapaz simples, innocente.
(Cp. \textunderscore mamote\textunderscore )
\section{Mamparra}
\begin{itemize}
\item {Grp. gram.:f.}
\end{itemize}
\begin{itemize}
\item {Grp. gram.:Pl.}
\end{itemize}
\begin{itemize}
\item {Utilização:Bras}
\end{itemize}
\begin{itemize}
\item {Utilização:Bras. do N}
\end{itemize}
Súcia; camaradagem de pândegos ou vadios.
Evasiva; subterfúgio.
Pequeno roubo.
\section{Mampárria}
\begin{itemize}
\item {Grp. gram.:f.}
\end{itemize}
O mesmo que \textunderscore mamparra\textunderscore .
\section{Mampastor}
\begin{itemize}
\item {Grp. gram.:m.}
\end{itemize}
\begin{itemize}
\item {Utilização:Ant.}
\end{itemize}
Juiz delegado, que decidia as causas cíveis.
(Talvez por \textunderscore mampostor\textunderscore , de \textunderscore mão\textunderscore  + \textunderscore pôsto\textunderscore . Cp. \textunderscore mamposteiro\textunderscore )
\section{Mampofa}
\begin{itemize}
\item {Grp. gram.:f.}
\end{itemize}
Arbusto da Guiné, de propriedades antihelmínthicas.
\section{Mamposteiro}
\begin{itemize}
\item {Grp. gram.:m.}
\end{itemize}
\begin{itemize}
\item {Utilização:Ant.}
\end{itemize}
\begin{itemize}
\item {Proveniência:(De \textunderscore mão\textunderscore  + \textunderscore pôsto\textunderscore )}
\end{itemize}
Procurador.
Homem, \textunderscore pôsto\textunderscore  pela \textunderscore mão\textunderscore  de alguém, para algum negócio.
Recebedor de esmolas para cativos.
\section{Mamua}
\begin{itemize}
\item {Grp. gram.:f.}
\end{itemize}
O mesmo que \textunderscore mamôa\textunderscore , oiteiro.
\section{Mamuarana}
\begin{itemize}
\item {Grp. gram.:f.}
\end{itemize}
O mesmo que \textunderscore mamurana\textunderscore .
\section{Mamude}
\begin{itemize}
\item {Grp. gram.:m.}
\end{itemize}
\begin{itemize}
\item {Utilização:Ant.}
\end{itemize}
\begin{itemize}
\item {Proveniência:(De Mamude, n. p.)}
\end{itemize}
Moéda de Surate.
\section{Mamudo}
\begin{itemize}
\item {Grp. gram.:adj.}
\end{itemize}
O mesmo que \textunderscore mamalhudo\textunderscore .
\section{Mamujar}
\begin{itemize}
\item {Grp. gram.:v. i.}
\end{itemize}
\begin{itemize}
\item {Proveniência:(Do rad. de \textunderscore mama\textunderscore )}
\end{itemize}
Mamar aos poucos, com interrupções, sem appetite.
\section{Mâmula}
\begin{itemize}
\item {Grp. gram.:f.}
\end{itemize}
\begin{itemize}
\item {Utilização:Hist. nat.}
\end{itemize}
\begin{itemize}
\item {Proveniência:(Lat. \textunderscore mammula\textunderscore )}
\end{itemize}
Protuberância.
\section{Mamulengos}
\begin{itemize}
\item {Grp. gram.:m. pl.}
\end{itemize}
\begin{itemize}
\item {Utilização:Bras}
\end{itemize}
Divertimento popular, que consiste em representações dramáticas por meio de bonecos; títeres.
\section{Mamunha}
\begin{itemize}
\item {Grp. gram.:f.}
\end{itemize}
Montículo de terra, que cobre sepulturas prehistóricas.
(Cp. \textunderscore mamoínha\textunderscore )
\section{Mamurana}
\begin{itemize}
\item {Grp. gram.:f.}
\end{itemize}
Árvore brasileira, de fibras têxtis.
\section{Mamute}
\begin{itemize}
\item {Grp. gram.:m.}
\end{itemize}
\begin{itemize}
\item {Proveniência:(T. de or. tart.)}
\end{itemize}
Corpulento animal, do gênero elephante e cuja espécie desappareceu, encontrando-se hoje em estado fóssil, sobretudo na Sibéria.
\section{Mana}
\begin{itemize}
\item {Grp. gram.:f.}
\end{itemize}
\begin{itemize}
\item {Utilização:fam.}
\end{itemize}
\begin{itemize}
\item {Utilização:Ant.}
\end{itemize}
\begin{itemize}
\item {Utilização:Ant.}
\end{itemize}
\begin{itemize}
\item {Proveniência:(De \textunderscore mano\textunderscore )}
\end{itemize}
O mesmo que \textunderscore irman\textunderscore .
Tratamento familiar, dado por uma cunhada ou cunhado a outra cunhada.
Tratamento affectuoso, que se dava a uma mulher, sem ideia de parentesco. Cf. G. Vicente, \textunderscore passim\textunderscore .
\section{Manacá}
\begin{itemize}
\item {Grp. gram.:m.}
\end{itemize}
Planta escrofularínea do Brasil, (\textunderscore francisca uniflora\textunderscore ).
\section{Manaça}
\begin{itemize}
\item {Grp. gram.:m.}
\end{itemize}
\begin{itemize}
\item {Utilização:Pop.}
\end{itemize}
Homem indolente, madraço.
\section{Manacan}
\begin{itemize}
\item {Grp. gram.:m.}
\end{itemize}
O mesmo que \textunderscore manacá\textunderscore .
\section{Manada}
\begin{itemize}
\item {Grp. gram.:f.}
\end{itemize}
\begin{itemize}
\item {Utilização:Ant.}
\end{itemize}
\begin{itemize}
\item {Utilização:Bras}
\end{itemize}
\begin{itemize}
\item {Utilização:Prov.}
\end{itemize}
\begin{itemize}
\item {Proveniência:(Do lat. \textunderscore manus\textunderscore )}
\end{itemize}
Rebanho de gado graúdo.
Rancho, grupo de gente.
Magote de trinta a quarenta éguas ou burras, dominadas por um garanhão.
O mesmo que \textunderscore mancheia\textunderscore .
Pequena porção.
\section{Manadeira}
\begin{itemize}
\item {Grp. gram.:f.}
\end{itemize}
\begin{itemize}
\item {Proveniência:(De \textunderscore manar\textunderscore )}
\end{itemize}
O mesmo que \textunderscore manancial\textunderscore .
\section{Manadeiro}
\begin{itemize}
\item {Grp. gram.:m.}
\end{itemize}
\begin{itemize}
\item {Proveniência:(De \textunderscore manar\textunderscore )}
\end{itemize}
O mesmo que \textunderscore manancial\textunderscore .
\section{Manadinha}
\begin{itemize}
\item {Grp. gram.:f.}
\end{itemize}
\begin{itemize}
\item {Utilização:Prov.}
\end{itemize}
\begin{itemize}
\item {Utilização:minh.}
\end{itemize}
\begin{itemize}
\item {Proveniência:(De \textunderscore manada\textunderscore )}
\end{itemize}
Pequena porção, pequena mancheia.
\section{Manadio}
\begin{itemize}
\item {Grp. gram.:adj.}
\end{itemize}
Relativo a manada.
Que anda em manada. Cf. A. Baganha, \textunderscore Hyg. Pec.\textunderscore , 50 e 166.
\section{Manado}
\begin{itemize}
\item {Grp. gram.:m.}
\end{itemize}
\begin{itemize}
\item {Utilização:T. de Cucujães}
\end{itemize}
\begin{itemize}
\item {Proveniência:(Do lat. \textunderscore manus\textunderscore ?)}
\end{itemize}
Aquillo que se leva no regaço ou no colo, especialmente uma criança.
\section{Manageira}
\begin{itemize}
\item {Grp. gram.:f.}
\end{itemize}
\begin{itemize}
\item {Utilização:T. de Aveiro}
\end{itemize}
Mulher, que dirige o trabalho da ceifa.
Mulher, que dirige o grupo das que levam sardinha em pequenas canastras, da praia de S. Jacinto para as povoações do districto.
(Cp. \textunderscore manageiro\textunderscore )
\section{Manageiro}
\begin{itemize}
\item {Grp. gram.:m.}
\end{itemize}
\begin{itemize}
\item {Utilização:Prov.}
\end{itemize}
\begin{itemize}
\item {Utilização:alent.}
\end{itemize}
Aquelle que dirige os trabalhos das ceifas.
Maioral, capataz.
O mesmo que \textunderscore marnoto\textunderscore . Cf. \textunderscore Museu Techn.\textunderscore , 107.
\section{Manaia}
\begin{itemize}
\item {Grp. gram.:f.}
\end{itemize}
\begin{itemize}
\item {Utilização:Prov.}
\end{itemize}
\begin{itemize}
\item {Utilização:dur.}
\end{itemize}
Bragas; calças curtas de varino.
\section{Manaio}
\begin{itemize}
\item {Grp. gram.:adj.}
\end{itemize}
\begin{itemize}
\item {Utilização:Prov.}
\end{itemize}
\begin{itemize}
\item {Utilização:trasm.}
\end{itemize}
Vesgo; pisco.
\section{Manaixo}
\begin{itemize}
\item {Grp. gram.:m.}
\end{itemize}
\begin{itemize}
\item {Utilização:Prov.}
\end{itemize}
\begin{itemize}
\item {Utilização:beir.}
\end{itemize}
Vestuário de mulher, garrido ou vistoso, mas de pouco preço.
(Cp. \textunderscore manaia\textunderscore )
\section{Manajós}
\begin{itemize}
\item {Grp. gram.:m. pl.}
\end{itemize}
Tríbo de Índios do Brasil, oriundos dos Tupinambás.
\section{Manalha}
\begin{itemize}
\item {Grp. gram.:f.}
\end{itemize}
\begin{itemize}
\item {Utilização:Ant.}
\end{itemize}
\begin{itemize}
\item {Proveniência:(De \textunderscore mano\textunderscore )}
\end{itemize}
Bando de estroinas ou tunantes.
Súcia de valdevinos.
\section{Manalvo}
\begin{itemize}
\item {Grp. gram.:adj.}
\end{itemize}
\begin{itemize}
\item {Proveniência:(Do lat. \textunderscore manus\textunderscore  + \textunderscore albus\textunderscore )}
\end{itemize}
Dizia-se do cavallo, que tem manchas brancas nas mãos.
\section{Manampansa}
\begin{itemize}
\item {Grp. gram.:f.}
\end{itemize}
\begin{itemize}
\item {Utilização:Bras. do Rio}
\end{itemize}
Espécie de beiju, feito de massa de mandioca e temperado com açúcar e erva doce.
\section{Manana}
\begin{itemize}
\item {Grp. gram.:f.}
\end{itemize}
Pulseira de varetas de metal, na Lunda.
\section{Manana}
\begin{itemize}
\item {Grp. gram.:m.}
\end{itemize}
Madeira amarela de Taiti.
\section{Manancial}
\begin{itemize}
\item {Grp. gram.:m.}
\end{itemize}
\begin{itemize}
\item {Utilização:Ext.}
\end{itemize}
\begin{itemize}
\item {Grp. gram.:Adj.}
\end{itemize}
Nascente de água; fonte.
Origem; fonte abundante: \textunderscore a Tôrre do Tombo é um manancial de esclarecimentos\textunderscore .
Que mana ou corre incessantemente.
(Cp. cast. \textunderscore manantial\textunderscore )
\section{Manancialmente}
\begin{itemize}
\item {Grp. gram.:adv.}
\end{itemize}
\begin{itemize}
\item {Proveniência:(De \textunderscore manancial\textunderscore )}
\end{itemize}
Sem cessar; perennemente.
\section{Mananguera}
\begin{itemize}
\item {fónica:gu-ê}
\end{itemize}
\begin{itemize}
\item {Grp. gram.:m. ,  f.  e  adj.}
\end{itemize}
\begin{itemize}
\item {Utilização:Bras}
\end{itemize}
Pessôa magra.
\section{Manante}
\begin{itemize}
\item {Grp. gram.:adj.}
\end{itemize}
Que mana:«\textunderscore terra manante leite e mel.\textunderscore »Usque, 8.
\section{Manantéu}
\begin{itemize}
\item {Grp. gram.:m.}
\end{itemize}
Ave, também conhecida por \textunderscore papa-figos\textunderscore .
\section{Manáos}
\begin{itemize}
\item {Grp. gram.:m. pl.}
\end{itemize}
Numerosa tríbo de Índios do Brasil, nas margens do Amazonas.
\section{Manapuçá}
\begin{itemize}
\item {Grp. gram.:m.}
\end{itemize}
\begin{itemize}
\item {Utilização:Bras. do Ceará}
\end{itemize}
Árvore melastomácea do Brasil.
\section{Manápula}
\begin{itemize}
\item {Grp. gram.:f.}
\end{itemize}
\begin{itemize}
\item {Utilização:Pop.}
\end{itemize}
Mão grande e mal feita.
(Corr. de \textunderscore manopla\textunderscore )
\section{Manaquim}
\begin{itemize}
\item {Grp. gram.:m.}
\end{itemize}
Pássaro dentirostro da América, (\textunderscore pipra\textunderscore ).
\section{Manar}
\begin{itemize}
\item {Grp. gram.:v. i.}
\end{itemize}
\begin{itemize}
\item {Utilização:Fig.}
\end{itemize}
\begin{itemize}
\item {Grp. gram.:V. t.}
\end{itemize}
\begin{itemize}
\item {Utilização:Fig.}
\end{itemize}
\begin{itemize}
\item {Proveniência:(Lat. \textunderscore manare\textunderscore )}
\end{itemize}
Correr em abundancia, perennemente.
Fluír; brotar.
Originar-se; provir.
Verter sem cessar (um líquido).
Produzir.
Derramar.
\section{Manata}
\begin{itemize}
\item {Grp. gram.:m.}
\end{itemize}
\begin{itemize}
\item {Utilização:Pop.}
\end{itemize}
Indivíduo casquilho, janota.
Figurão, personagem importante.
(Cp. \textunderscore magnate\textunderscore )
\section{Manatim}
\begin{itemize}
\item {Grp. gram.:m.}
\end{itemize}
Gênero do mammíferos marinhos.
\section{Manatoto}
\begin{itemize}
\item {Grp. gram.:m.}
\end{itemize}
Língua de Timor.
\section{Manauê}
\begin{itemize}
\item {Grp. gram.:m.}
\end{itemize}
\begin{itemize}
\item {Utilização:Bras}
\end{itemize}
Espécie de bolo, feito de fubá de milho, mel, etc.
\section{Manaus}
\begin{itemize}
\item {Grp. gram.:m. pl.}
\end{itemize}
Numerosa tríbo de Índios do Brasil, nas margens do Amazonas.
\section{Manaxo}
\begin{itemize}
\item {Grp. gram.:m.}
\end{itemize}
\begin{itemize}
\item {Utilização:Prov.}
\end{itemize}
\begin{itemize}
\item {Utilização:T. de Resende}
\end{itemize}
O mesmo que \textunderscore manaixo\textunderscore .
Rodilha, farrapo.
\section{Mancal}
\begin{itemize}
\item {Grp. gram.:m.}
\end{itemize}
\begin{itemize}
\item {Grp. gram.:Pl.}
\end{itemize}
\begin{itemize}
\item {Utilização:Ant.}
\end{itemize}
\begin{itemize}
\item {Proveniência:(Do rad. de \textunderscore manco\textunderscore ?)}
\end{itemize}
Pau ferrado para o jôgo do fito.
Peça de ferro, calçada do aço, e que serve de apoio á carapuça em que gira o aguilhão, nos engenhos de açúcar.
Peça de bronze, nos mesmos engenhos.
Jôgo do fito.
\section{Mançanica}
\begin{itemize}
\item {Grp. gram.:f.}
\end{itemize}
Espécie de azeitona, também conhecida por \textunderscore pelle-de-sapo\textunderscore .
\section{Mançanilha}
\begin{itemize}
\item {Grp. gram.:f.}
\end{itemize}
O mesmo ou melhor que \textunderscore mancenilha\textunderscore .
O mesmo que \textunderscore mançanica\textunderscore .
\section{Mancar}
\begin{itemize}
\item {Grp. gram.:v. i.}
\end{itemize}
\begin{itemize}
\item {Grp. gram.:V. t.}
\end{itemize}
\begin{itemize}
\item {Proveniência:(Lat. \textunderscore mancare\textunderscore )}
\end{itemize}
Coxear, manquejar.
Tornar manco.
\section{Mançar}
\begin{itemize}
\item {Grp. gram.:v. i.}
\end{itemize}
\begin{itemize}
\item {Utilização:Ant.}
\end{itemize}
\begin{itemize}
\item {Proveniência:(Fr. \textunderscore manquer\textunderscore )}
\end{itemize}
(inda hoje us. na região de San-Francisco, Brasil)
Fazer falta, faltar.
\section{Mancarra}
\begin{itemize}
\item {Grp. gram.:f.}
\end{itemize}
Amendoim da África occidental.
\section{Manceba}
\begin{itemize}
\item {fónica:cê}
\end{itemize}
\begin{itemize}
\item {Grp. gram.:f.}
\end{itemize}
\begin{itemize}
\item {Utilização:Ant.}
\end{itemize}
\begin{itemize}
\item {Proveniência:(Do b. lat. \textunderscore mancipia\textunderscore )}
\end{itemize}
Mulher amancebada; concubina.
Amásia de clérigo.
\section{Mancebia}
\begin{itemize}
\item {Grp. gram.:f.}
\end{itemize}
\begin{itemize}
\item {Utilização:Ant.}
\end{itemize}
\begin{itemize}
\item {Proveniência:(De \textunderscore mancebo\textunderscore ^2)}
\end{itemize}
Estado de quem vive amancebado.
Vida dissoluta.
Lupanar, alcoice.
\section{Mancebia}
\begin{itemize}
\item {Grp. gram.:f.}
\end{itemize}
\begin{itemize}
\item {Utilização:Des.}
\end{itemize}
\begin{itemize}
\item {Proveniência:(De \textunderscore mancebo\textunderscore ^1)}
\end{itemize}
Juventude; os jóvens. Cf. Filinto, XVIII, 12.
\section{Mancebil}
\begin{itemize}
\item {Grp. gram.:adj.}
\end{itemize}
Relativo a mancebo; juvenil. Cf. Filinto, III, 122.
\section{Mancebo}
\begin{itemize}
\item {Grp. gram.:m.}
\end{itemize}
\begin{itemize}
\item {Utilização:Des.}
\end{itemize}
\begin{itemize}
\item {Utilização:Prov.}
\end{itemize}
\begin{itemize}
\item {Utilização:alent.}
\end{itemize}
\begin{itemize}
\item {Utilização:Prov.}
\end{itemize}
\begin{itemize}
\item {Utilização:alent.}
\end{itemize}
\begin{itemize}
\item {Utilização:Bras}
\end{itemize}
\begin{itemize}
\item {Utilização:Des.}
\end{itemize}
\begin{itemize}
\item {Grp. gram.:Adj.}
\end{itemize}
\begin{itemize}
\item {Proveniência:(Do lat. \textunderscore mancipium\textunderscore )}
\end{itemize}
Indivíduo jóvem; rapaz.
Fasquia, com que se amparam as tábuas, para se pregarem em forros ou lugar alto.
Criado; homem que trabalha por salário.
Pau, que liga a extremidade anterior das chedas ao cabeçalho, para o manter horizontal.
O mesmo que \textunderscore velador\textunderscore .
Tronco, de que se dependuram as balanças de braço, nos mercados e feiras.
Cabide para roupa, formado de uma haste com vários braços.
Amante.
Homem, que tem amásia ou concubina.
O mesmo que \textunderscore juvenil\textunderscore :«\textunderscore ...querem forçar as incrinações mancebas\textunderscore ». \textunderscore Eufrosina\textunderscore , 106.
\section{Mancenilha}
\begin{itemize}
\item {Grp. gram.:f.}
\end{itemize}
Variedade do azeitona.
Mancenilheira.
(Cast. \textunderscore manzanilla\textunderscore )
\section{Mancenilheira}
\begin{itemize}
\item {Grp. gram.:f.}
\end{itemize}
\begin{itemize}
\item {Proveniência:(De \textunderscore mancenilha\textunderscore )}
\end{itemize}
Árvore euphorbiácea, de cujo fruto e tronco se extrái um suco venenoso, e de cuja sombra se diz que é lethífera ou nociva.
\section{Mancha}
\begin{itemize}
\item {Grp. gram.:f.}
\end{itemize}
\begin{itemize}
\item {Utilização:Prov.}
\end{itemize}
\begin{itemize}
\item {Utilização:alent.}
\end{itemize}
\begin{itemize}
\item {Utilização:Bras}
\end{itemize}
\begin{itemize}
\item {Proveniência:(Do lat. \textunderscore macula\textunderscore )}
\end{itemize}
Mácula, nódoa.
Malha.
Labéu na reputação ou na fama: \textunderscore carácter sem mancha\textunderscore .
Pincelada.
Cama do javali.
Mato, que se deixa de pé, como ilhota, em terreno arroteado ou roçado e em que é fácil apanhar caça.
Doença, que ataca o tabaco.
\section{Manchamba}
\begin{itemize}
\item {Grp. gram.:f.}
\end{itemize}
Herdade ou quinta, na região de Lourenço-Marques.
\section{Manchar}
\begin{itemize}
\item {Grp. gram.:v. t.}
\end{itemize}
\begin{itemize}
\item {Utilização:Fig.}
\end{itemize}
Pôr mancha em.
Sujar; emporcalhar: \textunderscore manchar o fato\textunderscore .
Infamar; desacreditar.
Dar pinceladas claras e escuras em, antes de as empastar.
\section{Manchego}
\begin{itemize}
\item {fónica:chê}
\end{itemize}
\begin{itemize}
\item {Grp. gram.:adj.}
\end{itemize}
\begin{itemize}
\item {Grp. gram.:M.}
\end{itemize}
Diz-se do herói de Cervantes, do fidalgo da Mancha, D. Quixote.
\textunderscore Carro de manchego\textunderscore , carro de munição de artilharia.
(Cast. \textunderscore mauchego\textunderscore )
\section{Mancheia}
\begin{itemize}
\item {Grp. gram.:f.}
\end{itemize}
O mesmo ou melhor que \textunderscore mão-cheia\textunderscore .
\section{Manchil}
\begin{itemize}
\item {Grp. gram.:m.}
\end{itemize}
\begin{itemize}
\item {Utilização:Ant.}
\end{itemize}
\begin{itemize}
\item {Proveniência:(Do ár. \textunderscore mangil\textunderscore )}
\end{itemize}
Cutello de carniceiro.
O mesmo que \textunderscore foice\textunderscore .
Antiga arma de guerra.
\section{Manchinha}
\begin{itemize}
\item {Grp. gram.:f.}
\end{itemize}
(dem. de \textunderscore mancheia\textunderscore : \textunderscore uma manchinha de cerejas\textunderscore )
(Por \textunderscore mancheinha\textunderscore , de \textunderscore mancheia\textunderscore )
\section{Mancho}
\begin{itemize}
\item {Grp. gram.:m.}
\end{itemize}
\begin{itemize}
\item {Utilização:Açor}
\end{itemize}
Aquillo que se abrange com a mão.
(Cp. \textunderscore mancheia\textunderscore )
\section{Manchó}
\begin{itemize}
\item {Grp. gram.:m.}
\end{itemize}
\begin{itemize}
\item {Utilização:Prov.}
\end{itemize}
Ave implume.
\section{Manchoca}
\begin{itemize}
\item {Grp. gram.:f.}
\end{itemize}
\begin{itemize}
\item {Utilização:Prov.}
\end{itemize}
\begin{itemize}
\item {Utilização:beir.}
\end{itemize}
\begin{itemize}
\item {Proveniência:(De \textunderscore mancheia\textunderscore )}
\end{itemize}
Pequena porção de vinho, que se fabríca no comêço dos trabalhos da vindima e dos lagares, geralmente destinada aos que se empregam nestes trabalhos.
\section{Manchoco}
\begin{itemize}
\item {fónica:chô}
\end{itemize}
\begin{itemize}
\item {Grp. gram.:m.}
\end{itemize}
\begin{itemize}
\item {Utilização:Prov.}
\end{itemize}
\begin{itemize}
\item {Utilização:beir.}
\end{itemize}
\begin{itemize}
\item {Proveniência:(De \textunderscore mancheia\textunderscore )}
\end{itemize}
Pequena porção; uma mancheia.
\section{Manchoqueira}
\begin{itemize}
\item {Grp. gram.:f.}
\end{itemize}
Acumulação de muitos rebentos de raízes.
(Cp. \textunderscore manchoca\textunderscore )
\section{Manchu}
\begin{itemize}
\item {Grp. gram.:m.}
\end{itemize}
Língua, falada na Manchúria, (Ásia oriental).
\section{Manchua}
\begin{itemize}
\item {Grp. gram.:f.}
\end{itemize}
\begin{itemize}
\item {Utilização:Ant.}
\end{itemize}
Leve embarcação asiática. Cf. \textunderscore Peregr.\textunderscore , XL e CLXXX; \textunderscore Hist. Trág. Marít.\textunderscore , 11.
\section{Mancinela}
\begin{itemize}
\item {Grp. gram.:f.}
\end{itemize}
O mesmo que mancenilha. Cf. Castilho, \textunderscore Sabichonas\textunderscore , 176.
\section{Mancinella}
\begin{itemize}
\item {Grp. gram.:f.}
\end{itemize}
O mesmo que mancenilha. Cf. Castilho, \textunderscore Sabichonas\textunderscore , 176.
\section{Mancipar}
\begin{itemize}
\item {Grp. gram.:V. p.}
\end{itemize}
\begin{itemize}
\item {Proveniência:(Lat. \textunderscore mancipare\textunderscore )}
\end{itemize}
\textunderscore v. t.\textunderscore  (e der.)
(V. \textunderscore emancipar\textunderscore , etc.)
Sujeitar-se, entregar-se: \textunderscore mancipar-se ao serviço de Deus\textunderscore . Cf. M. Bernárdez, \textunderscore N. Floresta\textunderscore , III, 156.
\section{Mancípio}
\begin{itemize}
\item {Grp. gram.:m.}
\end{itemize}
\begin{itemize}
\item {Utilização:Ant.}
\end{itemize}
\begin{itemize}
\item {Utilização:Ext.}
\end{itemize}
\begin{itemize}
\item {Proveniência:(Lat. \textunderscore mancipium\textunderscore )}
\end{itemize}
Escravo.
Indivíduo ou coisa dependente.
\section{Manco}
\begin{itemize}
\item {Grp. gram.:adj.}
\end{itemize}
\begin{itemize}
\item {Utilização:Ext.}
\end{itemize}
\begin{itemize}
\item {Grp. gram.:M.}
\end{itemize}
\begin{itemize}
\item {Proveniência:(Lat. \textunderscore mancus\textunderscore )}
\end{itemize}
A quem falta uma das mãos ou um pé.
Que não póde servir-se de algum dos membros locomotores.
Coxo.
Defeituoso, acanhado, ignorante.
Vagaroso.
Peça curva, que se entalha nos gios da embarcação.
Indivíduo, que é manco.
\section{Mancolitar}
\begin{itemize}
\item {Grp. gram.:v. i.}
\end{itemize}
\begin{itemize}
\item {Utilização:T. da Bairrada}
\end{itemize}
\begin{itemize}
\item {Proveniência:(De \textunderscore mancolitó\textunderscore )}
\end{itemize}
Manquejar, coxear.
\section{Mancolitó}
\begin{itemize}
\item {Grp. gram.:m.}
\end{itemize}
\begin{itemize}
\item {Utilização:T. da Bairrada}
\end{itemize}
O mesmo que \textunderscore manquitó\textunderscore .
\section{Mancommunação}
\begin{itemize}
\item {Grp. gram.:f.}
\end{itemize}
Acto ou effeito de mancommunar.
\section{Mancommunadamente}
\begin{itemize}
\item {Grp. gram.:adv.}
\end{itemize}
\begin{itemize}
\item {Proveniência:(De \textunderscore mancommunar\textunderscore )}
\end{itemize}
De acôrdo.
\section{Mancommunar}
\begin{itemize}
\item {Grp. gram.:v. t.}
\end{itemize}
\begin{itemize}
\item {Proveniência:(De \textunderscore mão\textunderscore  + \textunderscore commum\textunderscore )}
\end{itemize}
Pôr de acôrdo.
Ajustar, combinar.
\section{Mancomunação}
\begin{itemize}
\item {Grp. gram.:f.}
\end{itemize}
Acto ou efeito de mancomunar.
\section{Mancomunadamente}
\begin{itemize}
\item {Grp. gram.:adv.}
\end{itemize}
\begin{itemize}
\item {Proveniência:(De \textunderscore mancomunar\textunderscore )}
\end{itemize}
De acôrdo.
\section{Mancomunar}
\begin{itemize}
\item {Grp. gram.:v. t.}
\end{itemize}
\begin{itemize}
\item {Proveniência:(De \textunderscore mão\textunderscore  + \textunderscore comum\textunderscore )}
\end{itemize}
Pôr de acôrdo.
Ajustar, combinar.
\section{Mancone}
\begin{itemize}
\item {Grp. gram.:m.}
\end{itemize}
Árvore da Guiné, de frutos muito venenosos.
\section{Mancornar}
\begin{itemize}
\item {Grp. gram.:v. i.}
\end{itemize}
\begin{itemize}
\item {Proveniência:(De \textunderscore mão\textunderscore  + \textunderscore corno\textunderscore . Cp. cast. \textunderscore mancornar\textunderscore )}
\end{itemize}
Diz-se do pegador de toiros, quando com as mãos segura os cornos do animal, derribando-o.
\section{Mancuba}
\begin{itemize}
\item {Grp. gram.:m.}
\end{itemize}
Nome que, em alguns pontos de Angola, se dá ao carrapato.
\section{Mancubar}
\textunderscore m.\textunderscore Árvore da Guiné, de raízes medicinaes.
\section{Mancueba}
\begin{itemize}
\item {Grp. gram.:m.}
\end{itemize}
\begin{itemize}
\item {Utilização:Bras}
\end{itemize}
O mesmo que \textunderscore cuba\textunderscore ^2.
\section{Mançupir}
\begin{itemize}
\item {Grp. gram.:v. i.}
\end{itemize}
\begin{itemize}
\item {Utilização:Prov.}
\end{itemize}
\begin{itemize}
\item {Utilização:trasm.}
\end{itemize}
\begin{itemize}
\item {Utilização:pop.}
\end{itemize}
Comer como um alarve, brutalmente.
\section{Manda}
\begin{itemize}
\item {Grp. gram.:f.}
\end{itemize}
\begin{itemize}
\item {Utilização:Ant.}
\end{itemize}
\begin{itemize}
\item {Utilização:Prov.}
\end{itemize}
\begin{itemize}
\item {Utilização:trasm.}
\end{itemize}
\begin{itemize}
\item {Proveniência:(De \textunderscore mandar\textunderscore )}
\end{itemize}
Legado; disposição testamentária.
Referência, chamada.
Peditório, para festas religiosas.
\section{Mandaçaia}
\begin{itemize}
\item {Grp. gram.:f.}
\end{itemize}
\begin{itemize}
\item {Utilização:Bras}
\end{itemize}
Espécie de abelha.
\section{Mandação}
\begin{itemize}
\item {Grp. gram.:f.}
\end{itemize}
\begin{itemize}
\item {Utilização:Ant.}
\end{itemize}
\begin{itemize}
\item {Proveniência:(De \textunderscore mandar\textunderscore )}
\end{itemize}
Districto ou território, em que mandava um rico-homem, com autoridade régia.
\section{Mandaçarre}
\begin{itemize}
\item {Grp. gram.:m.}
\end{itemize}
\begin{itemize}
\item {Proveniência:(T. as.)}
\end{itemize}
Pescador de pérolas.
\section{Mandacaru}
\begin{itemize}
\item {Grp. gram.:m.}
\end{itemize}
Arbusto, da fam. dos cactos, (\textunderscore cercus triangularis\textunderscore ).
\section{Mandachuva}
\begin{itemize}
\item {Grp. gram.:m.}
\end{itemize}
\begin{itemize}
\item {Utilização:Bras}
\end{itemize}
\begin{itemize}
\item {Proveniência:(De \textunderscore mandar\textunderscore  + \textunderscore chuva\textunderscore )}
\end{itemize}
Magnate; pessôa importante.
\section{Mandacuru}
\begin{itemize}
\item {Grp. gram.:m.}
\end{itemize}
\begin{itemize}
\item {Utilização:Bras. do N}
\end{itemize}
O mesmo que \textunderscore mandacaru\textunderscore .
\section{Mandada}
\begin{itemize}
\item {Grp. gram.:f.}
\end{itemize}
\begin{itemize}
\item {Proveniência:(De \textunderscore mandado\textunderscore ^1)}
\end{itemize}
Roda que, nas prensas de engrenagem, faz girar o parafuso, e é subordinada á roda, em que o motor opera.
\section{Mandadeiro}
\begin{itemize}
\item {Grp. gram.:m.}
\end{itemize}
\begin{itemize}
\item {Grp. gram.:Adj.}
\end{itemize}
Aquelle que cumpre mandados ou que leva mensagens.
Relativo a mandado ou ordem: \textunderscore carta mandadeira\textunderscore .
\section{Mandado}
\begin{itemize}
\item {Grp. gram.:adj.}
\end{itemize}
\begin{itemize}
\item {Utilização:fam.}
\end{itemize}
\begin{itemize}
\item {Utilização:Fig.}
\end{itemize}
\textunderscore Pau mandado\textunderscore , pessôa muito obediente, muito dócil.
\section{Mandado}
\begin{itemize}
\item {Grp. gram.:m.}
\end{itemize}
\begin{itemize}
\item {Utilização:Ant.}
\end{itemize}
\begin{itemize}
\item {Proveniência:(Lat. \textunderscore mandatum\textunderscore )}
\end{itemize}
Acto de mandar.
Ordem, determinação imperativa.
Ordem escrita, emanada da autoridade judicial ou administrativa.
Legado.
\section{Mandador}
\begin{itemize}
\item {Grp. gram.:m.  e  adj.}
\end{itemize}
\begin{itemize}
\item {Utilização:Pesc.}
\end{itemize}
\begin{itemize}
\item {Proveniência:(Lat. \textunderscore mandator\textunderscore )}
\end{itemize}
O que manda; o que gosta de mandar.
O encarregado que representa os donos dos cercos ou armações de pesca.
\section{Mandamento}
\begin{itemize}
\item {Grp. gram.:m.}
\end{itemize}
\begin{itemize}
\item {Utilização:Marn.}
\end{itemize}
\begin{itemize}
\item {Utilização:Ant.}
\end{itemize}
\begin{itemize}
\item {Utilização:Ant.}
\end{itemize}
\begin{itemize}
\item {Grp. gram.:Pl.}
\end{itemize}
\begin{itemize}
\item {Utilização:Fam.}
\end{itemize}
Acto ou effeito de mandar.
Mandado.
Voz de commando.
Cada um dos preceitos que constituem o decálogo.
Cada um dos cinco preceitos, formulados no cathecismo cathólico para uso dos fiéis.
O grupo dos caldeiros, sobrecabeceiras, cabeceiras e talhos, nas salinas.
Circunscripção territorial; districto; julgado.
Pastoral de Bispo. Cf. Latino, Hist. Pol., I.
Os dedos da mão, (por allusão ao número dos mandamentos da Igreja).
\section{Mandante}
\begin{itemize}
\item {Grp. gram.:adj.}
\end{itemize}
\begin{itemize}
\item {Grp. gram.:M.}
\end{itemize}
\begin{itemize}
\item {Proveniência:(Lat. \textunderscore mandans\textunderscore )}
\end{itemize}
Que manda; que subordina ou rege.
Aquelle que manda, o que dá ordens.
Aquelle que dirige certos trabalhos.
Aquelle que autoriza outrem ou o incita a certos actos em seu nome.
\section{Mandão}
\begin{itemize}
\item {Grp. gram.:m.}
\end{itemize}
\begin{itemize}
\item {Utilização:Ext.}
\end{itemize}
\begin{itemize}
\item {Proveniência:(De \textunderscore mandar\textunderscore )}
\end{itemize}
Aquelle que manda arrogantemente.
Déspota.
\section{Mandapuá}
\begin{itemize}
\item {Grp. gram.:f.}
\end{itemize}
(V.jabiru)
\section{Mandapuçá}
\begin{itemize}
\item {Grp. gram.:m.}
\end{itemize}
Árvore do Brasil.
Fruta dessa árvore.--Os diccion. dizem erradamente \textunderscore mandapuça\textunderscore .
\section{Mandaque}
\begin{itemize}
\item {Grp. gram.:m.}
\end{itemize}
Planta amaranthácea do Brasil.
\section{Mandar}
\begin{itemize}
\item {Grp. gram.:v. t.}
\end{itemize}
\begin{itemize}
\item {Utilização:Ant.}
\end{itemize}
\begin{itemize}
\item {Grp. gram.:V. i.}
\end{itemize}
\begin{itemize}
\item {Proveniência:(Lat. \textunderscore mandare\textunderscore )}
\end{itemize}
Encarregar de alguma coisa: \textunderscore mandar varrer a casa\textunderscore .
Dar ordem a.
Ordenar.
Exigir: \textunderscore mandar que seu filho lhe obedeça\textunderscore .
Impor, como preceito.
Dirigir.
Governar.
Delegar.
Arremessar: \textunderscore mandar uma pedra\textunderscore .
Enviar.
Transportar.
Deixar em testamento; legar.
Exercer autoridade: \textunderscore os pais mandam nos filhos\textunderscore .
\textunderscore Mandar por\textunderscore , mandar buscar, mandar trazer: \textunderscore deixei as luvas em casa, mas vou mandar por ellas\textunderscore . Cf. Castilho, \textunderscore Camões\textunderscore , 31.
\section{Mandareco}
\begin{itemize}
\item {Grp. gram.:m.}
\end{itemize}
\begin{itemize}
\item {Utilização:Prov.}
\end{itemize}
\begin{itemize}
\item {Utilização:beir.}
\end{itemize}
\begin{itemize}
\item {Proveniência:(De \textunderscore mandar\textunderscore )}
\end{itemize}
Mandado, recado; encommenda.
\section{Mandarete}
\begin{itemize}
\item {fónica:darê}
\end{itemize}
\begin{itemize}
\item {Grp. gram.:m.}
\end{itemize}
\begin{itemize}
\item {Utilização:Prov.}
\end{itemize}
\begin{itemize}
\item {Utilização:beir.}
\end{itemize}
\begin{itemize}
\item {Proveniência:(De \textunderscore mandar\textunderscore )}
\end{itemize}
Moço de recados.
Rapaz, a quem se incumbe qualquer serviço ligeiro fóra de casa; paquete.
\section{Mandarim}
\begin{itemize}
\item {Grp. gram.:m.}
\end{itemize}
\begin{itemize}
\item {Utilização:Fig.}
\end{itemize}
Magistrado chinês.
Mandão.
Dialecto official na China.
(Do hindi \textunderscore mantri\textunderscore )
\section{Mandarinato}
\begin{itemize}
\item {Grp. gram.:m.}
\end{itemize}
\begin{itemize}
\item {Utilização:Fig.}
\end{itemize}
Qualidade ou funcções de mandarim.
Classe superior, classe dos que mandam.
\section{Mandarinete}
\begin{itemize}
\item {fónica:nê}
\end{itemize}
\begin{itemize}
\item {Grp. gram.:m.}
\end{itemize}
Mandarim de categoria inferior.
\section{Mandarinismo}
\begin{itemize}
\item {Grp. gram.:m.}
\end{itemize}
Systema de provas e concursos que, na China, se exigem aos candidatos a graus literários e a empregos públicos.
(Cp. \textunderscore mandarim\textunderscore )
\section{Mandarino}
\begin{itemize}
\item {Grp. gram.:adj.}
\end{itemize}
Diz-se de um dos principaes dialectos chineses, também conhecido por \textunderscore mandarim\textunderscore .
(Cp. \textunderscore mandarim\textunderscore )
\section{Mandatário}
\begin{itemize}
\item {Grp. gram.:m.}
\end{itemize}
\begin{itemize}
\item {Proveniência:(Lat. \textunderscore mandatarius\textunderscore )}
\end{itemize}
Aquelle que recebe mandato.
Executor de mandados.
Delegado, representante.
\section{Mandato}
\begin{itemize}
\item {Grp. gram.:m.}
\end{itemize}
\begin{itemize}
\item {Proveniência:(Lat. \textunderscore mandatum\textunderscore )}
\end{itemize}
Autorização ou procuração, que alguém dá a outrem para, em seu nome, praticar certos actos.
Delegação.
Confiança.
\section{Mandeu}
\begin{itemize}
\item {Grp. gram.:m.}
\end{itemize}
Grupo de línguas da África occidental.
\section{Mandi}
\begin{itemize}
\item {Grp. gram.:m.}
\end{itemize}
\begin{itemize}
\item {Utilização:Bras}
\end{itemize}
Excellente peixe de água doce.
\section{Mandiba}
\begin{itemize}
\item {Grp. gram.:f.}
\end{itemize}
Espécie de mandioca.
\section{Mandibé}
\begin{itemize}
\item {Grp. gram.:m.}
\end{itemize}
Peixe do norte do Brasil.
\section{Mandibi}
\begin{itemize}
\item {Grp. gram.:m.}
\end{itemize}
Arbusto brasileiro, da fam. dos jarros, (\textunderscore arum usum\textunderscore ).
\section{Mandíbula}
\begin{itemize}
\item {Grp. gram.:f.}
\end{itemize}
\begin{itemize}
\item {Proveniência:(Lat. \textunderscore mandibula\textunderscore )}
\end{itemize}
O mesmo que \textunderscore maxilla\textunderscore .
Cada uma das duas partes, em que se divide o bico das aves.
Cada uma das duas peças móveis e duras, collocadas uma á esquerda e outra á direita da boca de certos insectos.
\section{Mandibular}
\begin{itemize}
\item {Grp. gram.:adj.}
\end{itemize}
Relativo a mandibula.
\section{Mandiguera}
\begin{itemize}
\item {Grp. gram.:m.}
\end{itemize}
\begin{itemize}
\item {Utilização:Bras}
\end{itemize}
Leitão enfezado.
\section{Mandígula}
\begin{itemize}
\item {Grp. gram.:f.}
\end{itemize}
\begin{itemize}
\item {Utilização:Gír.}
\end{itemize}
Bebida narcótica.
\section{Mandil}
\begin{itemize}
\item {Grp. gram.:m.}
\end{itemize}
\begin{itemize}
\item {Grp. gram.:Adj.}
\end{itemize}
\begin{itemize}
\item {Utilização:Gír.}
\end{itemize}
Pano grosseiro, de fabricação local, para vestuário de mulheres e principalmente para esfregar ou limpar.
Avental de cozinheiro.
Fazenda, própria para capas, na Índia portuguesa.
Preguiçoso.
(Ár. \textunderscore mandil\textunderscore )
\section{Mandileiro}
\begin{itemize}
\item {Grp. gram.:m.}
\end{itemize}
\begin{itemize}
\item {Utilização:Bras}
\end{itemize}
O mesmo que \textunderscore mandrião\textunderscore .
\section{Mandilete}
\begin{itemize}
\item {fónica:lê}
\end{itemize}
\begin{itemize}
\item {Grp. gram.:m.}
\end{itemize}
\begin{itemize}
\item {Utilização:Prov.}
\end{itemize}
\begin{itemize}
\item {Utilização:trasm.}
\end{itemize}
\begin{itemize}
\item {Proveniência:(Do rad. de \textunderscore mandar\textunderscore )}
\end{itemize}
Pequeno trabalho ou diligência; recado.
\section{Mandim}
\begin{itemize}
\item {Grp. gram.:m.}
\end{itemize}
\begin{itemize}
\item {Utilização:Bras}
\end{itemize}
O mesmo que \textunderscore mandi\textunderscore .
\section{Mandinga}
\begin{itemize}
\item {Grp. gram.:f.}
\end{itemize}
\begin{itemize}
\item {Grp. gram.:M.}
\end{itemize}
\begin{itemize}
\item {Proveniência:(T. \textunderscore afr.\textunderscore )}
\end{itemize}
Acto ou effeito de mandingar.
Feitiçaria.
Difficuldade, embaraço.
Um dos idiomas mais espalhados na África occidental, e pertencente ao grupo mandeu.
\section{Mandingar}
\begin{itemize}
\item {Grp. gram.:v. i.}
\end{itemize}
\begin{itemize}
\item {Proveniência:(De \textunderscore mandinga\textunderscore )}
\end{itemize}
Fazer feitiços a, enfeitiçar.
\section{Mandingas}
\begin{itemize}
\item {Grp. gram.:m. pl.}
\end{itemize}
Uma das tríbos da Guiné.
\section{Mandingo}
\begin{itemize}
\item {Grp. gram.:m.}
\end{itemize}
Língua, o mesmo que \textunderscore mandinga\textunderscore .
\section{Mandingueiro}
\begin{itemize}
\item {Grp. gram.:m.}
\end{itemize}
Aquelle que faz mandingas.
Feiticeiro, bruxo.
\section{Mandioca}
\begin{itemize}
\item {Grp. gram.:f.}
\end{itemize}
\begin{itemize}
\item {Utilização:Pop.}
\end{itemize}
Planta euphorbiácea do Brasil, (\textunderscore jatropha manihot\textunderscore ).
Raíz, de que se faz farinha e polvilho.
Aquillo que se come, aquillo que é comestível.
Acto de comer.
\section{Mandiocaba}
\begin{itemize}
\item {Grp. gram.:f.}
\end{itemize}
Espécie de mandioca.
\section{Mandiocal}
\begin{itemize}
\item {Grp. gram.:m.}
\end{itemize}
\begin{itemize}
\item {Utilização:Bras}
\end{itemize}
Terreno, plantado de mandioca.
\section{Mandioquinha-do-campo}
\begin{itemize}
\item {Grp. gram.:f.}
\end{itemize}
Arbusto bignoniáceo do Brasil.
\section{Mandiva}
\begin{itemize}
\item {Grp. gram.:f.}
\end{itemize}
O mesmo que \textunderscore mandiba\textunderscore .
\section{Mando}
\begin{itemize}
\item {Grp. gram.:m.}
\end{itemize}
Acto ou effeito de mandar.
Commando.
Direito.
Arbítrio.
Autoridade.
\section{Mandó}
\begin{itemize}
\item {Grp. gram.:m.}
\end{itemize}
Música monótona que, em certas festas, cantam as bailadeiras indianas, acompanhando-a com dança. Cf. Th. Ribeiro, \textunderscore Jornadas\textunderscore , II, 77 e 135.
\section{Mandobi}
\begin{itemize}
\item {Grp. gram.:m.}
\end{itemize}
(V.mendobi)
\section{Mandobre}
\begin{itemize}
\item {Grp. gram.:m.}
\end{itemize}
\begin{itemize}
\item {Utilização:Ant.}
\end{itemize}
\begin{itemize}
\item {Proveniência:(De \textunderscore mão\textunderscore  + \textunderscore debre\textunderscore )}
\end{itemize}
Golpe, descarregado com ambas as mãos.
\section{Mandola}
\begin{itemize}
\item {Grp. gram.:f.}
\end{itemize}
O mesmo que \textunderscore mandora\textunderscore .
\section{Mandolim}
\begin{itemize}
\item {Grp. gram.:m.}
\end{itemize}
\begin{itemize}
\item {Proveniência:(It. \textunderscore mandolino\textunderscore )}
\end{itemize}
Instrumento de cordas, espécie de alaúde.
\section{Mandolina}
\begin{itemize}
\item {Grp. gram.:f.}
\end{itemize}
\begin{itemize}
\item {Proveniência:(It. \textunderscore mandolino\textunderscore )}
\end{itemize}
Instrumento de cordas, espécie de alaúde.
\section{Mandolina}
\begin{itemize}
\item {Grp. gram.:f.}
\end{itemize}
O mesmo que \textunderscore bandolina\textunderscore . Cf. Garrett, \textunderscore Viagens\textunderscore , I, 114.
\section{Mandolinata}
\begin{itemize}
\item {Grp. gram.:f.}
\end{itemize}
Peça musical, tocada em mandolina.
\section{Mandolinete}
\begin{itemize}
\item {fónica:nê}
\end{itemize}
\begin{itemize}
\item {Grp. gram.:m.}
\end{itemize}
Instrumento italiano, pequena mandolina.
\section{Mandora}
\begin{itemize}
\item {fónica:dô}
\end{itemize}
\begin{itemize}
\item {Grp. gram.:f.}
\end{itemize}
\begin{itemize}
\item {Proveniência:(It. \textunderscore mandora\textunderscore , do lat. \textunderscore pandura\textunderscore )}
\end{itemize}
Antigo instrumento de cordas, semelhante ao alaúde.
Mandolina ou espécie de mandolina.
O mesmo que \textunderscore bandurra\textunderscore .
\section{Mandovim}
\begin{itemize}
\item {Grp. gram.:m.}
\end{itemize}
\begin{itemize}
\item {Utilização:Ant.}
\end{itemize}
Certo imposto alfandegário, na Índia portuguesa.
\section{Mandraço}
\begin{itemize}
\item {Grp. gram.:m.}
\end{itemize}
O mesmo que \textunderscore madraço\textunderscore .
(Contr. de \textunderscore malandraço\textunderscore , de \textunderscore malandro\textunderscore )
\section{Mandrágora}
\begin{itemize}
\item {Grp. gram.:f.}
\end{itemize}
\begin{itemize}
\item {Proveniência:(Do gr. \textunderscore mandragoras\textunderscore )}
\end{itemize}
Gênero de plantas solâneas.
\section{Mandrana}
\begin{itemize}
\item {Grp. gram.:m.  e  f.}
\end{itemize}
Pessoa, que mandreia ou madraceia.
(Cp. \textunderscore mândria\textunderscore )
\section{Mandranice}
\begin{itemize}
\item {Grp. gram.:f.}
\end{itemize}
Qualidade de mandrana.
\section{Mandrão}
\begin{itemize}
\item {Grp. gram.:m.}
\end{itemize}
Antiga arma de guerra. Cf. \textunderscore Viriato Trág.\textunderscore , VII, 39.
\section{Mândria}
\begin{itemize}
\item {Grp. gram.:f.}
\end{itemize}
\begin{itemize}
\item {Utilização:Fam.}
\end{itemize}
Qualidade de quem é mandrião, ou preguiçoso.
\section{Mandrianar}
\begin{itemize}
\item {Grp. gram.:v. i.}
\end{itemize}
\begin{itemize}
\item {Proveniência:(De \textunderscore mandrião\textunderscore )}
\end{itemize}
O mesmo que \textunderscore mandriar\textunderscore .
\section{Mandrião}
\begin{itemize}
\item {Grp. gram.:adj.}
\end{itemize}
\begin{itemize}
\item {Grp. gram.:M.}
\end{itemize}
\begin{itemize}
\item {Proveniência:(De \textunderscore mândria\textunderscore )}
\end{itemize}
Que é dado á preguiça; indolente.
Indivíduo preguiçoso ou ocioso.
Casaco curto e ligeiro, para uso doméstico de mulheres ou crianças.
Ave aquática, (\textunderscore stercorarius pomatorhinus\textunderscore , Selat.).
\section{Mandriar}
\begin{itemize}
\item {Grp. gram.:v. i.}
\end{itemize}
\begin{itemize}
\item {Proveniência:(De \textunderscore mândria\textunderscore )}
\end{itemize}
Têr vida de mandrião ou madraço.
Sêr preguiçoso.
\section{Mandriice}
\begin{itemize}
\item {Grp. gram.:f.}
\end{itemize}
O mesmo que \textunderscore mândria\textunderscore .
\section{Mandril}
\begin{itemize}
\item {Grp. gram.:m.}
\end{itemize}
\begin{itemize}
\item {Utilização:Med.}
\end{itemize}
Peça cylíndrica, com que, em artilharia, se alisa o olhal do projéctil.
Peça para usar os furos grandes, em certos trabalhos mechânicos.
Haste, mais ou menos rígida, que, introduzida nas sondas flexíveis, serve para lhes dar resistência e guiá-las.
(Cp. cast. \textunderscore mandril\textunderscore )
\section{Mandril}
\begin{itemize}
\item {Grp. gram.:m.}
\end{itemize}
Mammífero da costa da Guiné.
\section{Mandrilagem}
\begin{itemize}
\item {Grp. gram.:f.}
\end{itemize}
Acto de mandrilar.
\section{Mandrilar}
\begin{itemize}
\item {Grp. gram.:v. t.}
\end{itemize}
Alisar com mandril^1.
\section{Mandrilho}
\begin{itemize}
\item {Grp. gram.:m.}
\end{itemize}
\begin{itemize}
\item {Utilização:Bras. do N}
\end{itemize}
Órgão genital do cavallo.
(Por \textunderscore mandril\textunderscore ^1)
\section{Mandu}
\begin{itemize}
\item {Grp. gram.:adj.}
\end{itemize}
\begin{itemize}
\item {Utilização:Bras}
\end{itemize}
Pacóvio; tolo.
\section{Manduba}
\begin{itemize}
\item {Grp. gram.:f.}
\end{itemize}
(V.mandioca)
\section{Mandubi}
\begin{itemize}
\item {Grp. gram.:m.}
\end{itemize}
O mesmo que \textunderscore amendoim\textunderscore .
\section{Mandubi}
\begin{itemize}
\item {Grp. gram.:m.}
\end{itemize}
Peixe do Amazonas.
\section{Manduça}
\begin{itemize}
\item {Grp. gram.:f.}
\end{itemize}
\begin{itemize}
\item {Utilização:Bras}
\end{itemize}
O mesmo que \textunderscore rapadura\textunderscore .
\section{Manducação}
\begin{itemize}
\item {Grp. gram.:f.}
\end{itemize}
Acto de manducar.
\section{Manducar}
\begin{itemize}
\item {Grp. gram.:v. t.}
\end{itemize}
\begin{itemize}
\item {Utilização:Pop.}
\end{itemize}
\begin{itemize}
\item {Grp. gram.:V. i.}
\end{itemize}
\begin{itemize}
\item {Proveniência:(Lat. \textunderscore manducare\textunderscore )}
\end{itemize}
Comer.
Dar ao dente; mastigar.
\section{Manducar}
\begin{itemize}
\item {Grp. gram.:m.}
\end{itemize}
\begin{itemize}
\item {Utilização:T. da Índia port}
\end{itemize}
Colono agrícola, indígena.
\section{Manducativo}
\begin{itemize}
\item {Grp. gram.:adj.}
\end{itemize}
\begin{itemize}
\item {Proveniência:(De \textunderscore manducar\textunderscore )}
\end{itemize}
Relativo a manducação ou a comezainas:«\textunderscore proezas manducativas\textunderscore ». Filinto, IX, 33.
\section{Manducável}
\begin{itemize}
\item {Grp. gram.:adj.}
\end{itemize}
Que se póde manducar; comestível.
\section{Manduco}
\begin{itemize}
\item {Grp. gram.:m.}
\end{itemize}
Árvore medicinal da Guiné.
\section{Manduco}
\begin{itemize}
\item {Grp. gram.:m.}
\end{itemize}
\begin{itemize}
\item {Utilização:T. de Macau}
\end{itemize}
O mesmo que \textunderscore ran\textunderscore .
\section{Manduco}
\begin{itemize}
\item {Grp. gram.:m.}
\end{itemize}
\begin{itemize}
\item {Proveniência:(Lat. \textunderscore manducus\textunderscore )}
\end{itemize}
Manequim, de grandes maxillas e grandes dentes, usado pelos antigos em certas solennidades e em certas comédias.
\section{Mandupitiú}
\begin{itemize}
\item {Grp. gram.:m.}
\end{itemize}
Planta leguminosa do Brasil.
\section{Manduptim}
\begin{itemize}
\item {Grp. gram.:m.}
\end{itemize}
O mesmo que \textunderscore jareré\textunderscore .
\section{Mandureba}
\begin{itemize}
\item {Grp. gram.:f.}
\end{itemize}
\begin{itemize}
\item {Utilização:Bras. do N}
\end{itemize}
Cachaça.
\section{Manduruva}
\begin{itemize}
\item {Grp. gram.:f.}
\end{itemize}
Insecto, que ataca as plantações de tabaco, ao sul do Brasil.
\section{Mane}
\begin{itemize}
\item {Grp. gram.:m.}
\end{itemize}
Antigo pêso de Melinde, correspondente a pouco menos de 1 kilogramma.
\section{Mané}
\begin{itemize}
\item {Grp. gram.:m.}
\end{itemize}
\begin{itemize}
\item {Utilização:Bras}
\end{itemize}
\begin{itemize}
\item {Utilização:T. da Bairrada}
\end{itemize}
Indivíduo inepto, palerma, desleixado.
(Corr. de \textunderscore Manuel\textunderscore , n. p.)
\section{Manéa}
\begin{itemize}
\item {Grp. gram.:f.}
\end{itemize}
\begin{itemize}
\item {Utilização:Bras}
\end{itemize}
Correia de coiro, com que se peia a bêsta.
(Cast. \textunderscore maneu?\textunderscore )
\section{Maneabilidade}
\begin{itemize}
\item {Grp. gram.:f.}
\end{itemize}
Qualidade do que é maneável.
\section{Maneador}
\begin{itemize}
\item {Grp. gram.:m.}
\end{itemize}
\begin{itemize}
\item {Proveniência:(De \textunderscore manear\textunderscore ^1)}
\end{itemize}
Correia do coiro, no freio das bêstas.
\section{Manear}
\begin{itemize}
\item {Grp. gram.:v. t.}
\end{itemize}
\begin{itemize}
\item {Utilização:Bras}
\end{itemize}
Prender com maneia.
\section{Manear}
\begin{itemize}
\item {Grp. gram.:v. t.}
\end{itemize}
\begin{itemize}
\item {Grp. gram.:V. p.}
\end{itemize}
\begin{itemize}
\item {Utilização:T. de Ílhavo}
\end{itemize}
O mesmo que \textunderscore manejar\textunderscore .
Mexer-se, andar depressa, aviar-se.
\section{Maneável}
\begin{itemize}
\item {Grp. gram.:adj.}
\end{itemize}
\begin{itemize}
\item {Utilização:Fig.}
\end{itemize}
\begin{itemize}
\item {Proveniência:(De \textunderscore manear\textunderscore ^2)}
\end{itemize}
Que se pôde manejar.
Que gira bem, que não emperra, (falando-se de uma porta) Cf. Pant. de Aveiro, \textunderscore Itiner.\textunderscore , 238 v.^o, (2.^a ed.).
Dócil; lhano.
\section{Manècôco}
\begin{itemize}
\item {Grp. gram.:m.  e  adj.}
\end{itemize}
\begin{itemize}
\item {Utilização:Fam.}
\end{itemize}
O mesmo que \textunderscore mané\textunderscore . Cf. Camillo, \textunderscore Cav. em Ruínas\textunderscore , 47.
\section{Manega}
\begin{itemize}
\item {fónica:nê}
\end{itemize}
\begin{itemize}
\item {Grp. gram.:f.}
\end{itemize}
Apparelho, com que, na construcção de navios, se faz chegar a tábua ao lugar próprio.
\section{Manegar}
\begin{itemize}
\item {Grp. gram.:v.}
\end{itemize}
\begin{itemize}
\item {Utilização:t. Náut.}
\end{itemize}
Pôr (balizas) de fórma, que cortem a quilha verticalmente em ângulo recto.
\section{Mané-gostoso}
\begin{itemize}
\item {Grp. gram.:m.}
\end{itemize}
\begin{itemize}
\item {Utilização:Bras}
\end{itemize}
Fantoche, títere, franca-tripa.
\section{Maneia}
\begin{itemize}
\item {Grp. gram.:f.}
\end{itemize}
\begin{itemize}
\item {Utilização:Bras}
\end{itemize}
Correia de coiro, com que se peia a bêsta.
(Cast. \textunderscore maneu\textunderscore ?)
\section{Maneio}
\begin{itemize}
\item {Grp. gram.:m.}
\end{itemize}
\begin{itemize}
\item {Utilização:Ant.}
\end{itemize}
\begin{itemize}
\item {Grp. gram.:Pl.}
\end{itemize}
\begin{itemize}
\item {Proveniência:(Do rad. do lat. \textunderscore manus\textunderscore )}
\end{itemize}
Acto de manejar.
Laboração.
Trabalho manual.
Lucro.
Espécie de contribuição industrial.
Tecido adiposo das reses.
\section{Maneira}
\begin{itemize}
\item {Grp. gram.:f.}
\end{itemize}
\begin{itemize}
\item {Utilização:Prov.}
\end{itemize}
\begin{itemize}
\item {Utilização:ant.}
\end{itemize}
\begin{itemize}
\item {Utilização:Prov.}
\end{itemize}
\begin{itemize}
\item {Utilização:minh.}
\end{itemize}
\begin{itemize}
\item {Utilização:T. de Viana}
\end{itemize}
\begin{itemize}
\item {Grp. gram.:Pl.}
\end{itemize}
\begin{itemize}
\item {Proveniência:(Do rad. do lat. \textunderscore manus\textunderscore )}
\end{itemize}
Modo; feitio.
Arte.
Feição.
Opportunidade: \textunderscore não tive maneira de lhe falar\textunderscore .
Aquillo que caracteriza as obras ou trabalhos de um artista ou escritor.
Abertura lateral num vestuário, por onde se mete a mão na algibeira.
Braguilha das calças.
Abertura da saia, no lugar onde se aperta o cós.
Lhaneza, affabilidade no trato: \textunderscore êste homem tem maneiras\textunderscore .
\section{Maneirismo}
\begin{itemize}
\item {Grp. gram.:m.}
\end{itemize}
Processo de maneirista.
\section{Maneirista}
\begin{itemize}
\item {Grp. gram.:m.}
\end{itemize}
\begin{itemize}
\item {Proveniência:(De \textunderscore maneira\textunderscore )}
\end{itemize}
Aquelle que, em pintura, observa processo constante e monótono.
\section{Maneiro}
\begin{itemize}
\item {Grp. gram.:adj.}
\end{itemize}
\begin{itemize}
\item {Proveniência:(Do rad. do lat. \textunderscore manus\textunderscore )}
\end{itemize}
Que se maneja facilmente; portátil; manual; acommodatício.
\section{Maneirosamente}
\begin{itemize}
\item {Grp. gram.:adv.}
\end{itemize}
De modo maneiroso.
\section{Maneiroso}
\begin{itemize}
\item {Grp. gram.:adj.}
\end{itemize}
Dotado de maneiras.
Que tem bôas maneiras.
Delicado, amável.
\section{Manejar}
\begin{itemize}
\item {Grp. gram.:v. t.}
\end{itemize}
\begin{itemize}
\item {Utilização:Fig.}
\end{itemize}
\begin{itemize}
\item {Grp. gram.:V. i.}
\end{itemize}
Executar com as mãos.
Governar com a mão.
Mover com a mão: \textunderscore manejar uma espada\textunderscore .
Trabalhar com: \textunderscore manejar uma enxada\textunderscore .
Dirigir.
Administrar.
Praticar.
Exercer.
Trabalhar com as mãos.
(Cp. cast. \textunderscore manejar\textunderscore )
\section{Manejável}
\begin{itemize}
\item {Grp. gram.:adj.}
\end{itemize}
Que se póde manejar.
\section{Manejo}
\begin{itemize}
\item {Grp. gram.:m.}
\end{itemize}
\begin{itemize}
\item {Grp. gram.:Pl.}
\end{itemize}
\begin{itemize}
\item {Utilização:Pop.}
\end{itemize}
Acto ou effeito de manejar.
Exercício de equitação.
Lugar, em que se exercitam cavallos.
Apparelho, que se adapta a differentes máquinas agricolas movidas por animaes, e que aumenta a intensidade e a velocidade da fôrça dos mesmos animaes.
Manobras militares.
Artimanhas; ardis.
(Cp. it. \textunderscore maneggio\textunderscore , cast. \textunderscore manejo\textunderscore )
\section{Manel}
\begin{itemize}
\item {Grp. gram.:m.}
\end{itemize}
\begin{itemize}
\item {Utilização:Pop.}
\end{itemize}
O mesmo que \textunderscore mané\textunderscore .
\section{Manelo}
\begin{itemize}
\item {fónica:nê}
\end{itemize}
\begin{itemize}
\item {Grp. gram.:m.}
\end{itemize}
\begin{itemize}
\item {Utilização:Prov.}
\end{itemize}
\begin{itemize}
\item {Utilização:trasm.}
\end{itemize}
\begin{itemize}
\item {Utilização:Prov.}
\end{itemize}
\begin{itemize}
\item {Utilização:trasm.}
\end{itemize}
\begin{itemize}
\item {Proveniência:(Do rad. do lat. \textunderscore manus\textunderscore )}
\end{itemize}
Pequena porção de coisas, que se póde abranger na mão; manojo, manípulo.
Pequeno volume de estopa, depois de penteada.
Estriga, enleada na roca.
\section{Manema}
\begin{itemize}
\item {Grp. gram.:m. ,  f.  e  adj.}
\end{itemize}
\begin{itemize}
\item {Utilização:Bras}
\end{itemize}
Pessôa apalermada, pacovia, manècôco.
\section{Manembro}
\begin{itemize}
\item {Grp. gram.:m.}
\end{itemize}
\begin{itemize}
\item {Utilização:Bras}
\end{itemize}
Pateta.
Pacóvio; manema; mané.
\section{Manente}
\begin{itemize}
\item {Grp. gram.:adj.}
\end{itemize}
\begin{itemize}
\item {Proveniência:(Lat. \textunderscore manens\textunderscore )}
\end{itemize}
O mesmo que \textunderscore permanente\textunderscore .
\section{Manequim}
\begin{itemize}
\item {Grp. gram.:m.}
\end{itemize}
\begin{itemize}
\item {Utilização:Fig.}
\end{itemize}
\begin{itemize}
\item {Utilização:Fam.}
\end{itemize}
Boneco, que representa um homem ou um corpo humano, e que serve para estudos artísticos ou scientíficos, ou para trabalhos de costureira ou alfaiate.
Pessôa sem vontade própria; sevandija.
Janota; peralta.
(Talvez do holl. \textunderscore manken\textunderscore )
\section{Maneria}
\begin{itemize}
\item {Grp. gram.:f.}
\end{itemize}
\begin{itemize}
\item {Utilização:Ant.}
\end{itemize}
O mesmo que \textunderscore maninhádego\textunderscore . Cf. Herculano, \textunderscore Hist. de Port.\textunderscore , IV, 295 e 303.
\section{Manério}
\begin{itemize}
\item {Grp. gram.:m.}
\end{itemize}
\begin{itemize}
\item {Utilização:Ant.}
\end{itemize}
\begin{itemize}
\item {Proveniência:(It. \textunderscore maniero\textunderscore )}
\end{itemize}
Gerência, administração; maneio.
\section{Manes}
\begin{itemize}
\item {Grp. gram.:m. pl.}
\end{itemize}
\begin{itemize}
\item {Proveniência:(Lat. \textunderscore manes\textunderscore )}
\end{itemize}
Almas dos mortos.
Divindades infernaes, invocadas pelos Romanos sôbre os túmulos.
\section{Manês}
\begin{itemize}
\item {Grp. gram.:m.}
\end{itemize}
\begin{itemize}
\item {Utilização:Gír.}
\end{itemize}
Homem.
(Cp. ingl. \textunderscore man\textunderscore ?)
\section{Manesa}
\begin{itemize}
\item {fónica:nê}
\end{itemize}
\begin{itemize}
\item {Grp. gram.:f.}
\end{itemize}
\begin{itemize}
\item {Utilização:Gír.}
\end{itemize}
\begin{itemize}
\item {Proveniência:(De \textunderscore manês\textunderscore )}
\end{itemize}
Rameira, rascôa.
Concubina.
Mulher.
Abbadena.
\section{Maneta}
\begin{itemize}
\item {fónica:nê}
\end{itemize}
\begin{itemize}
\item {Grp. gram.:m. ,  f.  e  adj.}
\end{itemize}
\begin{itemize}
\item {Proveniência:(Do rad. do lat. \textunderscore manus\textunderscore )}
\end{itemize}
Pessôa, a quem falta um braço ou que tem uma das mãos cortada ou lesa.
\section{Manfarrico}
\begin{itemize}
\item {Grp. gram.:m.}
\end{itemize}
O mesmo que \textunderscore mafarrico\textunderscore .
\section{Manga}
\begin{itemize}
\item {Grp. gram.:f.}
\end{itemize}
\begin{itemize}
\item {Utilização:Pesc.}
\end{itemize}
\begin{itemize}
\item {Utilização:Fig.}
\end{itemize}
\begin{itemize}
\item {Utilização:Ant.}
\end{itemize}
\begin{itemize}
\item {Proveniência:(Do lat. \textunderscore man(i)ca\textunderscore )}
\end{itemize}
Parte do vestuário, com que se cobre o braço.
Filtro afunilado, para líquidos.
Mangueira^2.
Cada uma das rêdes quadrangulares, no apparelho da pesca da sardinha.
Tromba de água.
Chaminé de candeeiro.
Chocalho grande, choca.
Turba, grupo de gente.
\textunderscore Manga de cavallo\textunderscore , companhia ou esquadrão de cavallaria.
\textunderscore Manga de incandescência\textunderscore , tubo de gaza, impregnado de saes metállicos, que se colloca sôbre uma luz, para a tornar mais brilhante.
\textunderscore Estar em mangas\textunderscore , ou \textunderscore estar em mangas de camisa\textunderscore , estar sem casaco, ou sem outra cobertura sôbre as mangas da camisa.
\section{Manga}
\begin{itemize}
\item {Grp. gram.:f.}
\end{itemize}
\begin{itemize}
\item {Proveniência:(Do mal. \textunderscore manka\textunderscore )}
\end{itemize}
Fruto da mangueira.
A mangueira, planta.
\section{Manga}
\begin{itemize}
\item {Grp. gram.:f.}
\end{itemize}
\begin{itemize}
\item {Utilização:Bras}
\end{itemize}
\begin{itemize}
\item {Utilização:Bras}
\end{itemize}
\begin{itemize}
\item {Utilização:Prov.}
\end{itemize}
\begin{itemize}
\item {Utilização:alent.}
\end{itemize}
\begin{itemize}
\item {Utilização:T. do Ribatejo}
\end{itemize}
\begin{itemize}
\item {Utilização:Mil.}
\end{itemize}
Pastagem cercada, onde se guardam cavallos e bois.
Ramal da estrada de seringueiras.
Prolongamento de uma herdade entre outras.
Caminho cerrado por árvores ou vallados, que conduz ao pátio da herdade.
Os lados immediatos á guarnição.
(Relaciona-se com \textunderscore manga\textunderscore ^1?)
\section{Manga}
\begin{itemize}
\item {Grp. gram.:f.}
\end{itemize}
O mesmo que \textunderscore manganella\textunderscore . Cf. Herculano, \textunderscore Hist. de Port.\textunderscore , I, 376.
\section{Mangaba}
\begin{itemize}
\item {Grp. gram.:f.}
\end{itemize}
Fruto da mangabeira.
A mangabeira.
\section{Mangabal}
\begin{itemize}
\item {Grp. gram.:m.}
\end{itemize}
\begin{itemize}
\item {Proveniência:(De \textunderscore mangaba\textunderscore )}
\end{itemize}
Terreno em que crescem mangabeiras.
\section{Mangabeira}
\begin{itemize}
\item {Grp. gram.:f.}
\end{itemize}
\begin{itemize}
\item {Proveniência:(De \textunderscore mangaba\textunderscore )}
\end{itemize}
Árvore apocýnea do Brasil.
\section{Mangabinha}
\begin{itemize}
\item {Grp. gram.:f.}
\end{itemize}
\begin{itemize}
\item {Utilização:Fam.}
\end{itemize}
(V.mangabeira)
\section{Mangação}
\begin{itemize}
\item {Grp. gram.:f.}
\end{itemize}
Acto de mangar.
Caçoada.
Motejo.
\section{Mangador}
\begin{itemize}
\item {Grp. gram.:m.  e  adj.}
\end{itemize}
O que manga ou gosta de mangar.
\section{Mangagá}
\begin{itemize}
\item {Grp. gram.:adj.}
\end{itemize}
\begin{itemize}
\item {Utilização:Bras}
\end{itemize}
Enorme; muito grande.
\section{Mangaíba}
\begin{itemize}
\item {Grp. gram.:f.}
\end{itemize}
O mesmo que \textunderscore mangaba\textunderscore .
\section{Mangal}
\begin{itemize}
\item {Grp. gram.:m.}
\end{itemize}
\begin{itemize}
\item {Proveniência:(De \textunderscore manga\textunderscore ^2)}
\end{itemize}
Terreno, em que crescem mangueiras.
\section{Mangal}
\begin{itemize}
\item {Grp. gram.:m.}
\end{itemize}
Floresta de mangues.
\section{Mangala}
\begin{itemize}
\item {Grp. gram.:m.}
\end{itemize}
Idioma, falado em Ceilão.
\section{Mangalaça}
\begin{itemize}
\item {Grp. gram.:f.}
\end{itemize}
\begin{itemize}
\item {Utilização:Ext.}
\end{itemize}
Vadiagem.
Mancebia.
\section{Mangalaço}
\begin{itemize}
\item {Grp. gram.:m.}
\end{itemize}
Vadio; tunante; biltre; patife. Cf. Pacheco, \textunderscore Promptuário\textunderscore .
\section{Mangalala}
\begin{itemize}
\item {Grp. gram.:f.}
\end{itemize}
Ramoso arbusto africano, de fôlhas simples e flôres polypétalas, brancas.
\section{Mangalhado}
\begin{itemize}
\item {Grp. gram.:adj.}
\end{itemize}
\begin{itemize}
\item {Utilização:ant.}
\end{itemize}
\begin{itemize}
\item {Utilização:Gír.}
\end{itemize}
O mesmo que \textunderscore preguiçoso\textunderscore .
(Cp. \textunderscore mangalhão\textunderscore )
\section{Mangalhão}
\begin{itemize}
\item {Grp. gram.:m.}
\end{itemize}
\begin{itemize}
\item {Utilização:ant.}
\end{itemize}
\begin{itemize}
\item {Utilização:Fam.}
\end{itemize}
\begin{itemize}
\item {Proveniência:(De \textunderscore manga\textunderscore )}
\end{itemize}
Homem desleixado, que traz as mangas da camisa descaídas sôbre as mãos.
\section{Mangalho}
\begin{itemize}
\item {Grp. gram.:m.}
\end{itemize}
\begin{itemize}
\item {Utilização:Gír.}
\end{itemize}
\begin{itemize}
\item {Proveniência:(De \textunderscore mango\textunderscore ^2)}
\end{itemize}
Pênis grande.
\section{Mangaló}
\begin{itemize}
\item {Grp. gram.:m.}
\end{itemize}
Árvore leguminosa do Brasil.
\section{Mangana}
\begin{itemize}
\item {Grp. gram.:f.}
\end{itemize}
\begin{itemize}
\item {Utilização:Des.}
\end{itemize}
Mangação?«\textunderscore Isso deixo eu para vós, que sois todo hu[~]a mangana\textunderscore ». \textunderscore Eufrosina\textunderscore , 169.
\section{Manganato}
\begin{itemize}
\item {Grp. gram.:m.}
\end{itemize}
\begin{itemize}
\item {Utilização:Chím.}
\end{itemize}
Sal, resultante da combinação do ácido mangânico com uma base.
\section{Manganela}
\begin{itemize}
\item {Grp. gram.:f.}
\end{itemize}
Antiga e pequena máquina de guerra, para arremêsso de projécteis:«\textunderscore ...escorpiões arrojados pelas manganelas de fogo\textunderscore ». Herculano, \textunderscore Bobo\textunderscore , 17.
(B. lat. \textunderscore manganella\textunderscore )
\section{Manganella}
\begin{itemize}
\item {Grp. gram.:f.}
\end{itemize}
Antiga e pequena máquina de guerra, para arremêsso de projécteis:«\textunderscore ...escorpiões arrojados pelas manganellas de fogo\textunderscore ». Herculano, \textunderscore Bobo\textunderscore , 17.
(B. lat. \textunderscore manganella\textunderscore )
\section{Manganés}
\begin{itemize}
\item {Grp. gram.:m.}
\end{itemize}
\begin{itemize}
\item {Proveniência:(Fr. \textunderscore manganèse\textunderscore )}
\end{itemize}
Metal branco, muito friável e duro.
\section{Manganés-do-pântano}
\begin{itemize}
\item {Grp. gram.:m.}
\end{itemize}
Terra molle e leve, composta de óxydos hidratados de manganés, misturados com outros elementos.
\section{Manganesiato}
\begin{itemize}
\item {Grp. gram.:m.}
\end{itemize}
(V.manganato)
\section{Manganésico}
\begin{itemize}
\item {Grp. gram.:adj.}
\end{itemize}
O mesmo que \textunderscore mangânico\textunderscore .
\section{Manganésio}
\begin{itemize}
\item {Grp. gram.:m.}
\end{itemize}
O mesmo que \textunderscore manganés\textunderscore .
\section{Mangangá}
\begin{itemize}
\item {Grp. gram.:m.}
\end{itemize}
\begin{itemize}
\item {Utilização:Bras}
\end{itemize}
\begin{itemize}
\item {Grp. gram.:Adj.}
\end{itemize}
Insecto díptero, cuja mordedura produz calefrios e febre.
Grande peixe marítimo do Brasil.
Grande; muito grande.
\section{Manganesífero}
\begin{itemize}
\item {Grp. gram.:adj.}
\end{itemize}
O mesmo que \textunderscore manganífero\textunderscore .
\section{Manganesita}
\begin{itemize}
\item {Grp. gram.:f.}
\end{itemize}
O mesmo que \textunderscore acerdésia\textunderscore .
\section{Mangangaba}
\begin{itemize}
\item {Grp. gram.:m.}
\end{itemize}
O mesmo que \textunderscore mangangá\textunderscore .
\section{Manganica}
\begin{itemize}
\item {Grp. gram.:f.}
\end{itemize}
O mesmo que \textunderscore manganella\textunderscore . Cf. Herculano, \textunderscore Hist. de Port.\textunderscore , I, 376.
\section{Mangânico}
\begin{itemize}
\item {Grp. gram.:adj.}
\end{itemize}
\begin{itemize}
\item {Proveniência:(De \textunderscore mangano\textunderscore )}
\end{itemize}
Que contém manganés; relativo ao manganés.
\section{Manganídeos}
\begin{itemize}
\item {Grp. gram.:m. pl.}
\end{itemize}
\begin{itemize}
\item {Utilização:Miner.}
\end{itemize}
Família de mineraes, que comprehende o manganés e as suas combinações.
\section{Manganífero}
\begin{itemize}
\item {Grp. gram.:adj.}
\end{itemize}
\begin{itemize}
\item {Proveniência:(De \textunderscore manganés\textunderscore  + lat. \textunderscore ferre\textunderscore )}
\end{itemize}
Que tem ou produz manganés.
\section{Manganilha}
\begin{itemize}
\item {Grp. gram.:f.}
\end{itemize}
\begin{itemize}
\item {Proveniência:(Do rad. de \textunderscore mangar\textunderscore )}
\end{itemize}
Artimanha; lôgro.
\section{Manganita}
\begin{itemize}
\item {Grp. gram.:f.}
\end{itemize}
O mesmo que \textunderscore acerdésia\textunderscore .
\section{Mangânio}
\begin{itemize}
\item {Grp. gram.:m.}
\end{itemize}
O mesmo que \textunderscore mangano\textunderscore .
\section{Manganjas}
\begin{itemize}
\item {Grp. gram.:m. pl.}
\end{itemize}
Povo cafreal do Alto Chire.
\section{Mangano}
\begin{itemize}
\item {Grp. gram.:m.}
\end{itemize}
Nome, que alguns chímicos deram ao manganés.
\section{Manganoso}
\begin{itemize}
\item {Grp. gram.:adj.}
\end{itemize}
\begin{itemize}
\item {Utilização:Chím.}
\end{itemize}
Diz-se do primeiro dos óxydos do manganés.
\section{Mangão}
\begin{itemize}
\item {Grp. gram.:m.}
\end{itemize}
\begin{itemize}
\item {Utilização:Chul.}
\end{itemize}
Aquelle que manga.
\section{Mangar}
\begin{itemize}
\item {Grp. gram.:v. i.}
\end{itemize}
\begin{itemize}
\item {Utilização:Fam.}
\end{itemize}
Fazer escárneo; motejar; zombar.
(Relaciona-se com \textunderscore manguito\textunderscore ^2, gesto obsceno)
\section{Mangará}
\begin{itemize}
\item {Grp. gram.:m.}
\end{itemize}
Túbera ou bolbo, de que nascem certas plantas do Brasil.
\section{Mangará}
\begin{itemize}
\item {Grp. gram.:m.}
\end{itemize}
\begin{itemize}
\item {Utilização:Bras}
\end{itemize}
Ponta terminal da inflorescência da bananeira.
\section{Mangaraíto}
\begin{itemize}
\item {Grp. gram.:m.}
\end{itemize}
\begin{itemize}
\item {Utilização:Bras. do N}
\end{itemize}
\begin{itemize}
\item {Proveniência:(De \textunderscore mangará\textunderscore ^1)}
\end{itemize}
Variedade de raíz comestível.
\section{Mangará-mirim}
\begin{itemize}
\item {Grp. gram.:m.}
\end{itemize}
(V.mangarito)
\section{Mangará-penna}
\begin{itemize}
\item {Grp. gram.:m.}
\end{itemize}
Árvore arácea do Brasil.
\section{Mangarataia}
\begin{itemize}
\item {Grp. gram.:f.}
\end{itemize}
Árvore amómea e medicinal do Brasil.
\section{Mangaraz}
\begin{itemize}
\item {Grp. gram.:f.}
\end{itemize}
Planta arácea do Brasil.
\section{Mangari}
\begin{itemize}
\item {Grp. gram.:m.}
\end{itemize}
\begin{itemize}
\item {Utilização:Bras}
\end{itemize}
Espécie de batata.
(Cp. \textunderscore mangará\textunderscore ^1)
\section{Mangarito}
\begin{itemize}
\item {Grp. gram.:m.}
\end{itemize}
Planta arácea do Brasil, (\textunderscore caladium sagittifolium\textunderscore ).
\section{Mangarobeira}
\begin{itemize}
\item {Grp. gram.:f.}
\end{itemize}
Árvore gutífera do Brasil.
\section{Mangas-de-velludo}
\begin{itemize}
\item {Grp. gram.:f. pl.}
\end{itemize}
Ave longipenne.
\section{Mangaz}
\begin{itemize}
\item {Grp. gram.:adj.}
\end{itemize}
\begin{itemize}
\item {Grp. gram.:M.}
\end{itemize}
\begin{itemize}
\item {Utilização:ant.}
\end{itemize}
\begin{itemize}
\item {Utilização:Chul.}
\end{itemize}
\begin{itemize}
\item {Proveniência:(De \textunderscore mango\textunderscore )}
\end{itemize}
Que, na sua espécie, é grande ou grosso.
Mandrião.
Indivíduo trapalhão, desavergonhado.
\section{Mangedoira}
\begin{itemize}
\item {Grp. gram.:f.}
\end{itemize}
\begin{itemize}
\item {Proveniência:(Do rad. de \textunderscore manjar\textunderscore )}
\end{itemize}
Tabuleiro, em que se dá comida aos animaes nas estrebarias.
\section{Mangedoura}
\begin{itemize}
\item {Grp. gram.:f.}
\end{itemize}
\begin{itemize}
\item {Proveniência:(Do rad. de \textunderscore manjar\textunderscore )}
\end{itemize}
Tabuleiro, em que se dá comida aos animaes nas estrebarias.
\section{Mangelim}
\begin{itemize}
\item {Grp. gram.:m.}
\end{itemize}
\begin{itemize}
\item {Proveniência:(T. as.)}
\end{itemize}
Árvore indiana, (\textunderscore adnanthera pavonina\textunderscore , Lin.), cuja semente era empregada como medida de pêso.
Pêso, com que se avaliavam os diamantes.
\section{Mangengra}
\begin{itemize}
\item {Grp. gram.:f.}
\end{itemize}
O mesmo que \textunderscore mijengra\textunderscore .
(Colhido em Mangualde)
\section{Mangericão}
\begin{itemize}
\item {Grp. gram.:m.}
\end{itemize}
Gênero de plantas odoríferas, da fam. das labiadas.
\section{Mangericar}
\begin{itemize}
\item {Grp. gram.:v. i.}
\end{itemize}
\begin{itemize}
\item {Utilização:Prov.}
\end{itemize}
\begin{itemize}
\item {Utilização:dur.}
\end{itemize}
Tomar a fórma de mangericão, (falando-se de videiras).
\section{Mangerico}
\begin{itemize}
\item {Grp. gram.:m.}
\end{itemize}
O mesmo que \textunderscore mangericão\textunderscore .
\section{Mangerioba}
\begin{itemize}
\item {Grp. gram.:f.}
\end{itemize}
O mesmo que \textunderscore mamanga\textunderscore .
\section{Mangerona}
\begin{itemize}
\item {Grp. gram.:f.}
\end{itemize}
\begin{itemize}
\item {Proveniência:(Do lat. \textunderscore mangyrana\textunderscore )}
\end{itemize}
Planta odorífera, da fam. das labiadas, (\textunderscore origanum manjorana\textunderscore ).
\section{Mangeronas}
\begin{itemize}
\item {Grp. gram.:m. pl.}
\end{itemize}
(V.manjaronas)
\section{Mangífera}
\begin{itemize}
\item {Grp. gram.:f.}
\end{itemize}
\begin{itemize}
\item {Utilização:Bot.}
\end{itemize}
\begin{itemize}
\item {Proveniência:(Do malab. \textunderscore manghas\textunderscore , fruto da mangueira, e lat. \textunderscore ferre\textunderscore )}
\end{itemize}
Designação scientífica da mangueira^2.
\section{Mangil}
\begin{itemize}
\item {Grp. gram.:m.}
\end{itemize}
O mesmo que \textunderscore manchil\textunderscore .
\section{Mangista}
\begin{itemize}
\item {Grp. gram.:f.}
\end{itemize}
Designação commercial da \textunderscore raiva-da-índia\textunderscore .
\section{Mangnela}
\begin{itemize}
\item {Grp. gram.:f.}
\end{itemize}
\begin{itemize}
\item {Utilização:Ant.}
\end{itemize}
O mesmo que \textunderscore manganela\textunderscore .
\section{Mangnella}
\begin{itemize}
\item {Grp. gram.:f.}
\end{itemize}
\begin{itemize}
\item {Utilização:Ant.}
\end{itemize}
O mesmo que \textunderscore manganella\textunderscore .
\section{Mango}
\begin{itemize}
\item {Grp. gram.:m.}
\end{itemize}
\begin{itemize}
\item {Proveniência:(Do lat. \textunderscore manicum\textunderscore )}
\end{itemize}
A parte mais curta do mangual.
Mammífero carnívoro da costa occidental da África.
* Des.
Pênis.
Pequeno peixe africano.
\section{Mangoal}
\begin{itemize}
\item {Grp. gram.:m.}
\end{itemize}
O mesmo que \textunderscore mangual\textunderscore .
\section{Mangoaça}
\begin{itemize}
\item {Grp. gram.:f.}
\end{itemize}
\begin{itemize}
\item {Utilização:Bras. do N}
\end{itemize}
Mangação; escárneo.
(Cp. \textunderscore mangar\textunderscore )
\section{Mangolar}
\begin{itemize}
\item {Grp. gram.:v. i.}
\end{itemize}
\begin{itemize}
\item {Utilização:Bras. do N}
\end{itemize}
O mesmo que \textunderscore mangonar\textunderscore .
\section{Mangona}
\begin{itemize}
\item {Grp. gram.:f.}
\end{itemize}
\begin{itemize}
\item {Utilização:Pop.}
\end{itemize}
\begin{itemize}
\item {Grp. gram.:M.}
\end{itemize}
Preguiça, indolência.
Homem preguiçoso.
\section{Mangonália}
\begin{itemize}
\item {Grp. gram.:f.}
\end{itemize}
O mesmo que \textunderscore manganella\textunderscore . Cf. Herculano, \textunderscore Hist. de Port.\textunderscore , I, 136.
\section{Mangonar}
\begin{itemize}
\item {Grp. gram.:v. i.}
\end{itemize}
\begin{itemize}
\item {Utilização:Pop.}
\end{itemize}
\begin{itemize}
\item {Utilização:ant.}
\end{itemize}
\begin{itemize}
\item {Utilização:Gír.}
\end{itemize}
\begin{itemize}
\item {Proveniência:(De \textunderscore mangona\textunderscore )}
\end{itemize}
Vadiar.
Estar ocioso; têr preguiça.
Fornicar.
\section{Mangosta}
\begin{itemize}
\item {Grp. gram.:f.}
\end{itemize}
Planta, da fam. das clusiáceas.
(Cast. \textunderscore mangosta\textunderscore )
\section{Mangostão}
\begin{itemize}
\item {Grp. gram.:m.}
\end{itemize}
Árvore gutífera da Índia, (\textunderscore garcinia mangostana\textunderscore , Lin.).
Fruto da mesma árvore. Cf. Garcia Orta, \textunderscore Coll.\textunderscore , VIII.
(Do mal.)
\section{Mangostina}
\begin{itemize}
\item {Grp. gram.:f.}
\end{itemize}
\begin{itemize}
\item {Proveniência:(De \textunderscore mangostão\textunderscore )}
\end{itemize}
Substância, que se extrai da goma fornecida pelo fruto do mangostão.
\section{Mangote}
\begin{itemize}
\item {Grp. gram.:m.}
\end{itemize}
\begin{itemize}
\item {Utilização:Náut.}
\end{itemize}
\begin{itemize}
\item {Utilização:Ant.}
\end{itemize}
\begin{itemize}
\item {Proveniência:(De \textunderscore manga\textunderscore ^1)}
\end{itemize}
Parte da armadura, que cobria os braços.
Peça, para ajudar a zonchar.
Gancho, aos lados do cilhão, no qual se apoiam os varaes.
Pequeno feixe? mão-cheia?:«\textunderscore ...quatro mangotes de tabaco\textunderscore ». (De um testamento de 1693)
\section{Mangote}
\begin{itemize}
\item {Grp. gram.:m.}
\end{itemize}
O mesmo que \textunderscore sotilicário\textunderscore . Cf. \textunderscore Roteiro de V. da Gama\textunderscore .
(Do mesmo rad. que \textunderscore mangote\textunderscore ^1?)
\section{Mangra}
\begin{itemize}
\item {Grp. gram.:f.}
\end{itemize}
\begin{itemize}
\item {Utilização:Mad}
\end{itemize}
\begin{itemize}
\item {Utilização:Ant.}
\end{itemize}
Humidade, que os nevoeiros deixam nas espigas do trigo e que as impede de se desenvolverem.
Ferrugem dos trigos.
Humidade ou orvalho, que prejudica os frutos.
O mesmo que oídio.
Qualquer doença. Cf. Sim. Mach., fol. 57.
(Cast. \textunderscore mangla\textunderscore )
\section{Mangrar}
\begin{itemize}
\item {Grp. gram.:v. t.}
\end{itemize}
\begin{itemize}
\item {Grp. gram.:V. i.}
\end{itemize}
\begin{itemize}
\item {Proveniência:(De \textunderscore mangra\textunderscore )}
\end{itemize}
Produzir mangra em.
Tolher o desenvolvimento de.
Inutilizar-se, mallograr-se.
\section{Mangrulho}
\begin{itemize}
\item {Grp. gram.:m.}
\end{itemize}
\begin{itemize}
\item {Utilização:Bras}
\end{itemize}
\begin{itemize}
\item {Proveniência:(T. procedente do Paraguai)}
\end{itemize}
Pôsto militar de observação, em sítio elevado, e formado de madeiras tôscas, acima das quaes se sobe, para observar:«\textunderscore foram inaugurados quatro mangrulhos na Lagôa dos Patos\textunderscore ». \textunderscore Jornal do Comm.\textunderscore , do Rio, de 6-I-909.
\section{Mangu}
\begin{itemize}
\item {Grp. gram.:m.}
\end{itemize}
\begin{itemize}
\item {Proveniência:(T. as.)}
\end{itemize}
Animal carnívoro de Ceilão.
\section{Manguá}
\begin{itemize}
\item {Grp. gram.:m.}
\end{itemize}
\begin{itemize}
\item {Utilização:Bras}
\end{itemize}
Correia, para açoitar animaes.
(Relaciona-se com o port. \textunderscore mangual\textunderscore ?)
\section{Mangual}
\begin{itemize}
\item {Grp. gram.:m.}
\end{itemize}
\begin{itemize}
\item {Proveniência:(Do lat. \textunderscore manualis\textunderscore )}
\end{itemize}
Instrumento, composto de dois paus ligados por uma correia, sendo um curto, grosso e chamado mango, e outro comprido e delgado, servindo de cabo; (é utilizado em debulhar cereaes).
Arma africana, composta de uma haste, a que se ligam vários toros pelo meio.
\section{Mangualada}
\begin{itemize}
\item {Grp. gram.:f.}
\end{itemize}
\begin{itemize}
\item {Utilização:Fig.}
\end{itemize}
Pancada de mangual.
Salto, dado pela cobra.
\section{Manguara}
\begin{itemize}
\item {Grp. gram.:f.}
\end{itemize}
\begin{itemize}
\item {Utilização:Bras}
\end{itemize}
Espécie de bastão, para auxiliar a marcha em terreno escorregadio.
\section{Manguari}
\begin{itemize}
\item {Grp. gram.:m.}
\end{itemize}
\begin{itemize}
\item {Utilização:Bras}
\end{itemize}
O mesmo que \textunderscore galalau\textunderscore .
\section{Manguço}
\begin{itemize}
\item {Grp. gram.:m.}
\end{itemize}
(V. \textunderscore mangusto\textunderscore ^1)
\section{Mangue}
\begin{itemize}
\item {Grp. gram.:m.}
\end{itemize}
\begin{itemize}
\item {Utilização:Bras}
\end{itemize}
Margem lamacenta de portos ou rios, aonde chega a água salgada.
Qualquer planta das que crescem nessa margem.
Floresta, junto ás praias e nas fozes dos rios.
Nome de várias árvores da América e da África, cuja casca é rica em tanino e empregada em curtumes.
(Cp. \textunderscore manga\textunderscore ^2)
\section{Mangue}
\begin{itemize}
\item {Grp. gram.:pron.}
\end{itemize}
\begin{itemize}
\item {Utilização:Gír.}
\end{itemize}
Eu. (Or. ind.)
\section{Manguear}
\begin{itemize}
\item {Grp. gram.:v. t.}
\end{itemize}
\begin{itemize}
\item {Utilização:Bras}
\end{itemize}
\begin{itemize}
\item {Proveniência:(Do rad. de \textunderscore manga\textunderscore ^3)}
\end{itemize}
Guiar a nado ou conduzir (um animal) para a mangueira (curral).
\section{Mangue-da-praia}
\begin{itemize}
\item {Grp. gram.:m.}
\end{itemize}
Árvore santhomense, de madeira roxa.
\section{Mangue-de-obó}
\begin{itemize}
\item {Grp. gram.:m.}
\end{itemize}
Árvore santhomense, de casca amarga.
\section{Mangueira}
\begin{itemize}
\item {Grp. gram.:f.}
\end{itemize}
\begin{itemize}
\item {Proveniência:(De \textunderscore manga\textunderscore ^2)}
\end{itemize}
Gênero de plantas terebintháceas, (\textunderscore mangifera indica\textunderscore ).
\section{Mangueira}
\begin{itemize}
\item {Grp. gram.:f.}
\end{itemize}
\begin{itemize}
\item {Proveniência:(De \textunderscore manga\textunderscore ^1)}
\end{itemize}
Tubo de lona, de coiro ou borracha, para a conducção de ar ou água.
\section{Mangueira}
\begin{itemize}
\item {Grp. gram.:f.}
\end{itemize}
\begin{itemize}
\item {Utilização:Bras. do S}
\end{itemize}
\begin{itemize}
\item {Proveniência:(De \textunderscore manga\textunderscore ^3)}
\end{itemize}
Curral grande.
\section{Mangueira}
\begin{itemize}
\item {fónica:gu-ei}
\end{itemize}
\begin{itemize}
\item {Grp. gram.:f.}
\end{itemize}
\begin{itemize}
\item {Utilização:Prov.}
\end{itemize}
\begin{itemize}
\item {Utilização:Prov.}
\end{itemize}
\begin{itemize}
\item {Utilização:trasm.}
\end{itemize}
\begin{itemize}
\item {Proveniência:(De \textunderscore mango\textunderscore )}
\end{itemize}
O pau maior do mangual.
O mesmo que \textunderscore mangual\textunderscore .
\section{Mangueiral}
\begin{itemize}
\item {Grp. gram.:m.}
\end{itemize}
Terreno, em que crescem mangueiras.
\section{Mangueiro}
\begin{itemize}
\item {Grp. gram.:m.}
\end{itemize}
\begin{itemize}
\item {Utilização:Bras}
\end{itemize}
Curral grande.
(Cp. \textunderscore mangueira\textunderscore ^3)
\section{Manguela}
\begin{itemize}
\item {Grp. gram.:f.}
\end{itemize}
\begin{itemize}
\item {Proveniência:(De \textunderscore manga\textunderscore ^1)}
\end{itemize}
O mesmo que \textunderscore cécum\textunderscore .
\section{Manguella}
\begin{itemize}
\item {Grp. gram.:f.}
\end{itemize}
\begin{itemize}
\item {Proveniência:(De \textunderscore manga\textunderscore ^1)}
\end{itemize}
O mesmo que \textunderscore cécum\textunderscore .
\section{Manguerim}
\begin{itemize}
\item {Grp. gram.:m.}
\end{itemize}
Espécie de embarcação asiática. Cf. Castilho, \textunderscore Fastos\textunderscore , II, 403.
\section{Manguito}
\begin{itemize}
\item {Grp. gram.:m.}
\end{itemize}
\begin{itemize}
\item {Utilização:Prov.}
\end{itemize}
\begin{itemize}
\item {Utilização:minh.}
\end{itemize}
\begin{itemize}
\item {Utilização:Bras. do N}
\end{itemize}
\begin{itemize}
\item {Proveniência:(Do rad. de \textunderscore manga\textunderscore ^1)}
\end{itemize}
Nome antigo do regalo de pelles.
Pequena manga, para enfeite ou resguardo dos punhos.
O mesmo que \textunderscore mitene\textunderscore .
Manga pequena.
\section{Manguito}
\begin{itemize}
\item {Grp. gram.:m.}
\end{itemize}
\begin{itemize}
\item {Proveniência:(De \textunderscore mango\textunderscore )}
\end{itemize}
Gesto obsceno, que consiste em pôr um dos antebraços na curva interna do outro, oscillando com este e com a mão fechada. Cf. Camillo, \textunderscore Brasileira\textunderscore , 207.
\section{Mangula}
\begin{itemize}
\item {Grp. gram.:f.}
\end{itemize}
Espécie de picapau angolense, (\textunderscore dendrobates namaquus\textunderscore ). Cf. Capello e Ivens, I, 326.
\section{Manguna}
\begin{itemize}
\item {Grp. gram.:f.}
\end{itemize}
Ave africana.
O mesmo que \textunderscore mangula\textunderscore ?
\section{Mangus}
\begin{itemize}
\item {Grp. gram.:m.}
\end{itemize}
O mesmo que \textunderscore mangu\textunderscore .
\section{Mangusta}
\begin{itemize}
\item {Grp. gram.:f.}
\end{itemize}
Árvore de Timor.
O mesmo que \textunderscore mangostão\textunderscore .
\section{Mangusto}
\begin{itemize}
\item {Grp. gram.:m.}
\end{itemize}
Animal mammífero e carnívoro da Ásia e da África, (\textunderscore herpertes mungo\textunderscore , ou \textunderscore viverra\textunderscore , \textunderscore ichneumon\textunderscore ), também conhecido no Alentejo, com o nome de \textunderscore saca-rabo\textunderscore .
Mata e come cobras.
\section{Mangusto}
\begin{itemize}
\item {Grp. gram.:m.}
\end{itemize}
O mesmo de \textunderscore mangostão\textunderscore .
\section{Manguxo}
\begin{itemize}
\item {Grp. gram.:m.}
\end{itemize}
\begin{itemize}
\item {Utilização:Bras}
\end{itemize}
Pedúnculo interno da java, o mesmo que \textunderscore bambão\textunderscore .
\section{Manha}
\begin{itemize}
\item {Grp. gram.:f.}
\end{itemize}
\begin{itemize}
\item {Proveniência:(Do lat. \textunderscore machina\textunderscore , seg. Körting)}
\end{itemize}
Destreza, habilidade.
Ardil, astúcia.
Costumes.
\textunderscore Des.\textunderscore : \textunderscore cidadão de bôas manhas\textunderscore .
\section{Manhã}
\begin{itemize}
\item {Grp. gram.:f.}
\end{itemize}
\begin{itemize}
\item {Utilização:Fig.}
\end{itemize}
\begin{itemize}
\item {Proveniência:(De um der. do lat. \textunderscore mane\textunderscore )}
\end{itemize}
Tempo, que medeia entre o nascer do Sol e o meio-dia.
O alvorecer.
Primeiras horas do dia.
Princípio: \textunderscore na manhã da vida\textunderscore .
\section{Manhan}
\begin{itemize}
\item {Grp. gram.:f.}
\end{itemize}
\begin{itemize}
\item {Utilização:Fig.}
\end{itemize}
\begin{itemize}
\item {Proveniência:(De um der. do lat. \textunderscore mane\textunderscore )}
\end{itemize}
Tempo, que medeia entre o nascer do Sol e o meio-dia.
O alvorecer.
Primeiras horas do dia.
Princípio: \textunderscore na manhan da vida\textunderscore .
\section{Manhan-de-páscoa}
\begin{itemize}
\item {Grp. gram.:f.}
\end{itemize}
Planta medicinal e ornamental de Cabo-Verde, (\textunderscore croton mubango\textunderscore , Mill.).
\section{Manhanzinha}
\begin{itemize}
\item {Grp. gram.:f.}
\end{itemize}
Princípio da manhan; madrugada.
\section{Manhificência}
\begin{itemize}
\item {Grp. gram.:f.}
\end{itemize}
\begin{itemize}
\item {Utilização:Ant.}
\end{itemize}
O mesmo que \textunderscore magnificência\textunderscore . Cf. \textunderscore Usque\textunderscore , 15.
\section{Manho}
\begin{itemize}
\item {Grp. gram.:adj.}
\end{itemize}
\begin{itemize}
\item {Utilização:Des.}
\end{itemize}
O mesmo que \textunderscore magno\textunderscore .
\section{Manho}
\begin{itemize}
\item {Grp. gram.:adj.}
\end{itemize}
\begin{itemize}
\item {Utilização:Ant.}
\end{itemize}
Desnorteado, pateta:«\textunderscore ando manho sem saber\textunderscore ». \textunderscore Aulegrafia\textunderscore , 55.
\section{Manhoco}
\begin{itemize}
\item {Grp. gram.:m.}
\end{itemize}
\begin{itemize}
\item {Utilização:Prov.}
\end{itemize}
\begin{itemize}
\item {Utilização:trasm.}
\end{itemize}
Feixe de vides muito apertado.
(Cp. \textunderscore manojo\textunderscore , \textunderscore manolho\textunderscore , \textunderscore manelo\textunderscore )
\section{Manhosamente}
\begin{itemize}
\item {Grp. gram.:adv.}
\end{itemize}
De modo manhoso.
Com manha, com malícia.
\section{Manhosidade}
\begin{itemize}
\item {Grp. gram.:f.}
\end{itemize}
\begin{itemize}
\item {Utilização:Prov.}
\end{itemize}
\begin{itemize}
\item {Utilização:trasm.}
\end{itemize}
Qualidade de manhoso.
\section{Manhoso}
\begin{itemize}
\item {Grp. gram.:adj.}
\end{itemize}
\begin{itemize}
\item {Utilização:Fam.}
\end{itemize}
\begin{itemize}
\item {Proveniência:(Do lat. \textunderscore maniosus\textunderscore )}
\end{itemize}
Que tem manha: \textunderscore rapaz manhoso\textunderscore .
Feito com manha; que revela manha: \textunderscore resposta manhosa\textunderscore .
Hábil; sagaz.
Ordinário, de pouca estimação: \textunderscore pôs na cabeça um chapéu manhoso...\textunderscore 
\section{Manhuça}
\begin{itemize}
\item {Grp. gram.:f.}
\end{itemize}
\begin{itemize}
\item {Utilização:Prov.}
\end{itemize}
\begin{itemize}
\item {Utilização:trasm.}
\end{itemize}
O mesmo que \textunderscore manhuço\textunderscore .
Feixe de estrigas de linho, depois de espadeladas.
\section{Manhuço}
\begin{itemize}
\item {Grp. gram.:m.}
\end{itemize}
\begin{itemize}
\item {Utilização:Prov.}
\end{itemize}
\begin{itemize}
\item {Utilização:trasm.}
\end{itemize}
\begin{itemize}
\item {Proveniência:(Do rad. do lat. \textunderscore manus\textunderscore )}
\end{itemize}
Conjunto de coisas, que se podem abranger na mão sem as esconder.
Pequeno feixe; manelo.
\section{Mani}
\begin{itemize}
\item {Grp. gram.:m.}
\end{itemize}
\begin{itemize}
\item {Utilização:T. da Índia port}
\end{itemize}
Variedade de bambu.
\section{Mani...}
\begin{itemize}
\item {Grp. gram.:pref.}
\end{itemize}
\begin{itemize}
\item {Proveniência:(Do lat. \textunderscore manus\textunderscore )}
\end{itemize}
(designativo de \textunderscore mão\textunderscore )
\section{Mania}
\begin{itemize}
\item {Grp. gram.:f.}
\end{itemize}
\begin{itemize}
\item {Utilização:Fig.}
\end{itemize}
\begin{itemize}
\item {Proveniência:(Lat. \textunderscore mania\textunderscore )}
\end{itemize}
Espécie de loucura, com tendência para a fúria.
Excentricidade.
Esquisitice.
Desejo excessivo.
Mau costume.
Aquillo que se deseja immoderadamente.
\section{Maníaco}
\begin{itemize}
\item {Grp. gram.:m.  e  adj.}
\end{itemize}
\begin{itemize}
\item {Proveniência:(Lat. \textunderscore maniacus\textunderscore )}
\end{itemize}
O que tem mania ou manias.
\section{Maniáco}
\begin{itemize}
\item {Grp. gram.:adj.}
\end{itemize}
\begin{itemize}
\item {Utilização:Prov.}
\end{itemize}
O mesmo que \textunderscore maniáculo\textunderscore .
\section{Maniáculo}
\begin{itemize}
\item {Grp. gram.:adj.}
\end{itemize}
\begin{itemize}
\item {Utilização:Ant.}
\end{itemize}
\begin{itemize}
\item {Proveniência:(Lat. \textunderscore maniaculus\textunderscore , de \textunderscore maniacus\textunderscore . Cf. B. Pereira, \textunderscore Prosódia\textunderscore )}
\end{itemize}
Maníaco.
Demente.
Apatetado.
\section{Maniado}
\begin{itemize}
\item {Grp. gram.:adj.}
\end{itemize}
\begin{itemize}
\item {Utilização:Açor}
\end{itemize}
\begin{itemize}
\item {Proveniência:(De \textunderscore mania\textunderscore )}
\end{itemize}
Que tem pouco juizo; adoidado.
\section{Maniatar}
\begin{itemize}
\item {Grp. gram.:v. t.}
\end{itemize}
\begin{itemize}
\item {Utilização:Fig.}
\end{itemize}
\begin{itemize}
\item {Proveniência:(Do lat. \textunderscore manus\textunderscore  + \textunderscore aptare\textunderscore )}
\end{itemize}
Atar as mãos de.
Tolher os movimentos a.
Constranger.
Tirar a liberdade a.
\section{Manica}
\begin{itemize}
\item {Grp. gram.:f.}
\end{itemize}
\begin{itemize}
\item {Utilização:Bras}
\end{itemize}
\begin{itemize}
\item {Proveniência:(Do lat. \textunderscore manus\textunderscore )}
\end{itemize}
Manícula de sapateiros e correeiros.
Uma das três bolas, com que os camponeses peiam o cavallo ou o boi que foge.
\section{Manicá}
\begin{itemize}
\item {Grp. gram.:f.}
\end{itemize}
Planta acanthácea, (\textunderscore rutilia fertilis\textunderscore ).
\section{Manicaca}
\begin{itemize}
\item {Grp. gram.:m.  e  f.}
\end{itemize}
\begin{itemize}
\item {Utilização:Pop.}
\end{itemize}
\begin{itemize}
\item {Utilização:T. de Turquel}
\end{itemize}
Pessôa fraca, palerma, inhenha.
Cagarola.
Peralvilho.
Velhaquete.
Sujeito de maneiras requebradas.
(Talvez do quimb.)
\section{Manicacas}
\begin{itemize}
\item {Grp. gram.:m.  e  f.}
\end{itemize}
\begin{itemize}
\item {Utilização:Pop.}
\end{itemize}
O mesmo que \textunderscore manicaca\textunderscore .
\section{Manice}
\begin{itemize}
\item {Grp. gram.:f.}
\end{itemize}
\begin{itemize}
\item {Utilização:Des.}
\end{itemize}
\begin{itemize}
\item {Proveniência:(De \textunderscore mano\textunderscore )}
\end{itemize}
Muita intimidade, grande affeição, especialmente entre mulheres.
\section{Manicheísmo}
\begin{itemize}
\item {fónica:que}
\end{itemize}
\begin{itemize}
\item {Grp. gram.:m.}
\end{itemize}
Seita dos manicheus.
\section{Manicheus}
\begin{itemize}
\item {fónica:queus}
\end{itemize}
\begin{itemize}
\item {Grp. gram.:m. pl.}
\end{itemize}
Sectario do heresiarcha do século III, Manes, que attribuía a criação do mundo aos dois princípios oppostos, o do bem e o do mal.
\section{Maniçoba}
\begin{itemize}
\item {Grp. gram.:f.}
\end{itemize}
Planta euphorbiácea do Brasil, (\textunderscore jatropha\textunderscore ), semelhante á mandioca.
Fôlha de mandioca.
Guisado, com fôlhas de mandioca, carne e peixe.
\section{Maniçobal}
\begin{itemize}
\item {Grp. gram.:m.}
\end{itemize}
Mata de maniçobas.
\section{Maniçobeira}
\begin{itemize}
\item {Grp. gram.:f.}
\end{itemize}
O mesmo que \textunderscore maniçoba\textunderscore .
\section{Maniçobeiro}
\begin{itemize}
\item {Grp. gram.:m.}
\end{itemize}
Aquelle que fabríca a borracha da maniçoba.
\section{Manicomial}
\begin{itemize}
\item {Grp. gram.:adj.}
\end{itemize}
Relativo a manicómio: \textunderscore administração manicomial\textunderscore .
\section{Manicómio}
\begin{itemize}
\item {Grp. gram.:m.}
\end{itemize}
\begin{itemize}
\item {Proveniência:(Do gr. \textunderscore mania\textunderscore  + \textunderscore komein\textunderscore )}
\end{itemize}
Hospital de alienados.
\section{Manicora}
\begin{itemize}
\item {Grp. gram.:f.}
\end{itemize}
Ornato architectónico, que representa um animal hýbrido, com cabeça de serpente, tronco globoso e cauda de serpente.
\section{Manicórdio}
\begin{itemize}
\item {Grp. gram.:m.}
\end{itemize}
Instrumento músico, semelhante ao piano, mas mais pequeno.
(Por \textunderscore monocórdio\textunderscore , do gr. \textunderscore monos\textunderscore  + \textunderscore korde\textunderscore )
\section{Manicota}
\begin{itemize}
\item {Grp. gram.:f.}
\end{itemize}
\begin{itemize}
\item {Utilização:Ant.}
\end{itemize}
Peia para as mãos e pés das bêstas.--Parece ler-se \textunderscore maricota\textunderscore  no \textunderscore Diccion. de nomes, vozes e coisas\textunderscore , (ms. da Tôrre do Tombo), mas supponho sêr êrro de escrita, e por isso escrevo \textunderscore manicota\textunderscore , do rad. do lat. \textunderscore manus\textunderscore .
(Cp. \textunderscore maniota\textunderscore )
\section{Manicu}
\begin{itemize}
\item {Grp. gram.:m.}
\end{itemize}
Mammífero marsupial, espécie de sarigueia.
\section{Manicuera}
\begin{itemize}
\item {Grp. gram.:f.}
\end{itemize}
\begin{itemize}
\item {Utilização:Bras. do N}
\end{itemize}
Espécie de mandioca.
Suco doce dessa planta.
\section{Manícula}
\begin{itemize}
\item {Grp. gram.:f.}
\end{itemize}
\begin{itemize}
\item {Proveniência:(Lat. \textunderscore manicula\textunderscore )}
\end{itemize}
Cada um dos membros anteriores dos mammíferos.
Instrumento em fórma de meia lua, de que se servem os correeiros e sapateiros.
\section{Manicuro}
\begin{itemize}
\item {Grp. gram.:m.}
\end{itemize}
\begin{itemize}
\item {Utilização:Neol.}
\end{itemize}
\begin{itemize}
\item {Proveniência:(Do lat. \textunderscore manus\textunderscore  + \textunderscore cura\textunderscore )}
\end{itemize}
Aquelle que se dedica ao tratamento das mãos ou das unhas das mãos:«\textunderscore ...o manicuro que lhe polia as unhas\textunderscore ». Eça.
\section{Manicurto}
\begin{itemize}
\item {Grp. gram.:adj.}
\end{itemize}
\begin{itemize}
\item {Utilização:Fig.}
\end{itemize}
\begin{itemize}
\item {Proveniência:(De \textunderscore manus\textunderscore , lat. + \textunderscore curto\textunderscore )}
\end{itemize}
Que tem mãos curtas.
Sovina, onzeneiro.
\section{Manietar}
\begin{itemize}
\item {Grp. gram.:v. t.}
\end{itemize}
(V.maniatar)
\section{Manifacto}
\begin{itemize}
\item {Grp. gram.:m.}
\end{itemize}
\begin{itemize}
\item {Utilização:Des.}
\end{itemize}
\begin{itemize}
\item {Proveniência:(Do lat. \textunderscore manus\textunderscore  + \textunderscore factus\textunderscore )}
\end{itemize}
O mesmo que \textunderscore manufactura\textunderscore .
\section{Manifestação}
\begin{itemize}
\item {Grp. gram.:f.}
\end{itemize}
\begin{itemize}
\item {Proveniência:(Lat. \textunderscore manifestatio\textunderscore )}
\end{itemize}
Acto ou effeito de manifestar.
Publicidade de sentimentos ou opiniões collectivas.
\section{Manifestador}
\begin{itemize}
\item {Grp. gram.:adj.}
\end{itemize}
\begin{itemize}
\item {Grp. gram.:M.}
\end{itemize}
\begin{itemize}
\item {Proveniência:(Lat. \textunderscore manifestator\textunderscore )}
\end{itemize}
Que manifesta.
Aquelle que manifesta.
\section{Manifestamente}
\begin{itemize}
\item {Grp. gram.:adv.}
\end{itemize}
De modo manifesto.
Claramente.
\section{Manifestante}
\begin{itemize}
\item {Grp. gram.:m.  e  adj.}
\end{itemize}
\begin{itemize}
\item {Proveniência:(Lat. \textunderscore manifestans\textunderscore )}
\end{itemize}
O mesmo que \textunderscore manifestador\textunderscore .
\section{Manifestar}
\begin{itemize}
\item {Grp. gram.:v. t.}
\end{itemize}
\begin{itemize}
\item {Proveniência:(De \textunderscore manifesto\textunderscore )}
\end{itemize}
Tornar manifesto, público, notório.
Apresentar; declarar: \textunderscore manifestar opiniões\textunderscore .
Revelar; dar ao manifesto na alfândega ou noutra estação fiscal.
\section{Manifesto}
\begin{itemize}
\item {Grp. gram.:adj.}
\end{itemize}
\begin{itemize}
\item {Proveniência:(Lat. \textunderscore manifestus\textunderscore )}
\end{itemize}
Patente.
Claro.
Público.
Evidente.
\section{Manifesto}
\begin{itemize}
\item {Grp. gram.:m.}
\end{itemize}
\begin{itemize}
\item {Proveniência:(Lat. \textunderscore manifestum\textunderscore )}
\end{itemize}
Coisa manifestada.
Declaração pública ou solenne das razões que justificam certos actos, ou em que se baseiam certos direitos.
Programma político ou religioso.
Relação, que se dá aos fiscaes da fazenda pública, dos gêneros que se expõem á venda e que são sujeitos a pagamentos de direitos.
\section{Mani-flautista}
\begin{itemize}
\item {Grp. gram.:m.}
\end{itemize}
\begin{itemize}
\item {Proveniência:(De \textunderscore mani...\textunderscore  + \textunderscore flautista\textunderscore )}
\end{itemize}
Aquelle que com as mãos produz sons semelhantes aos da flauta.
\section{Maniforme}
\begin{itemize}
\item {Grp. gram.:adj.}
\end{itemize}
\begin{itemize}
\item {Proveniência:(Do lat. \textunderscore manus\textunderscore  + \textunderscore forma\textunderscore )}
\end{itemize}
Que tem fórma de mão.
\section{Manigância}
\begin{itemize}
\item {Grp. gram.:f.}
\end{itemize}
\begin{itemize}
\item {Utilização:Fam.}
\end{itemize}
\begin{itemize}
\item {Proveniência:(Fr. \textunderscore manigance\textunderscore )}
\end{itemize}
Arte de prestidigitador.
Manobra mysteriosa.
Artes de berliques e berloques.
\section{Manigrepa}
\begin{itemize}
\item {Grp. gram.:f.}
\end{itemize}
\begin{itemize}
\item {Proveniência:(De \textunderscore manigrepo\textunderscore )}
\end{itemize}
Sacerdotisa de pagode chinês.
\section{Manigrepo}
\begin{itemize}
\item {Grp. gram.:m.}
\end{itemize}
Sacerdote chinês. Cf. Herculano, \textunderscore Lendas e Narr.\textunderscore , 264; \textunderscore Peregrinação\textunderscore , etc.
\section{Maniguete}
\begin{itemize}
\item {fónica:guê}
\end{itemize}
\begin{itemize}
\item {Grp. gram.:m.}
\end{itemize}
Semente de uma árvore amomácea, (\textunderscore amomum granumparadisi\textunderscore ).
\section{Maniiba}
\begin{itemize}
\item {Grp. gram.:f.}
\end{itemize}
(V.maniva)
\section{Manilha}
\begin{itemize}
\item {Grp. gram.:f.}
\end{itemize}
Argola, com que se adornam os pulsos e, entre alguns povos, a parte mais delgada das pernas.
Pulseira.
Grilheta.
Elo de cadeia.
Tubo de barro, usado em canalizações.
Nome de algumas cartas, em certos jogos.
Espécie de jôgo de cartas.
(Cast. \textunderscore manilla\textunderscore )
\section{Manilha}
Variedade de tabaco.
(Por \textunderscore manila\textunderscore , de \textunderscore Manila\textunderscore , n. p.)
\section{Manilhar}
\begin{itemize}
\item {Grp. gram.:v. t.}
\end{itemize}
\begin{itemize}
\item {Proveniência:(De \textunderscore manilha\textunderscore ^1)}
\end{itemize}
Ornar com manilhas; canalizar com manilhas.
\section{Manilheiro}
\begin{itemize}
\item {Grp. gram.:m.}
\end{itemize}
Fabricante de manilhas.
Jogador de manilha^1.
\section{Manilúvio}
\begin{itemize}
\item {Grp. gram.:m.}
\end{itemize}
\begin{itemize}
\item {Proveniência:(Do lat. \textunderscore manus\textunderscore  + \textunderscore luere\textunderscore )}
\end{itemize}
Banho ás mãos, de ordinário quente.
\section{Manimbu}
\begin{itemize}
\item {Grp. gram.:m.}
\end{itemize}
Planta gramínea do Brasil.
\section{Maninelo}
\begin{itemize}
\item {Grp. gram.:m.  e  adj.}
\end{itemize}
Idiota; indivíduo effeminado. Cf. \textunderscore Eufrosina\textunderscore , 100 e 158.
\section{Maninha}
\begin{itemize}
\item {Grp. gram.:f.}
\end{itemize}
Uma das matérias primas, empregadas em cordoaria. Cf. \textunderscore Inquér. Industr.\textunderscore , 2.^a parte, l. II, 237.
\section{Maninhádego}
\begin{itemize}
\item {Grp. gram.:m.}
\end{itemize}
\begin{itemize}
\item {Utilização:Ant.}
\end{itemize}
\begin{itemize}
\item {Proveniência:(De \textunderscore maninho\textunderscore )}
\end{itemize}
Tributo, que se pagava aos mosteiros ou á Corôa, e consistia na terça parte dos bens daquelles que, sendo casados, morriam sem deixar descendentes. Cf. Herculano, \textunderscore Hist. de Port.\textunderscore , IV, 295 e 303.
\section{Maninhar}
\begin{itemize}
\item {Grp. gram.:v. t.}
\end{itemize}
\begin{itemize}
\item {Proveniência:(De \textunderscore maninho\textunderscore )}
\end{itemize}
Deixar sem cultura, deixar maninha (uma terra).
\section{Maninhez}
\begin{itemize}
\item {Grp. gram.:f.}
\end{itemize}
Qualidade daquillo que é maninho.
\section{Maninho}
\begin{itemize}
\item {Grp. gram.:adj.}
\end{itemize}
\begin{itemize}
\item {Grp. gram.:M.}
\end{itemize}
\begin{itemize}
\item {Grp. gram.:Pl.}
\end{itemize}
Estéril; que não é prolífico; infecundo: \textunderscore mulher maninha\textunderscore .
Inculto: \textunderscore terras maninhas\textunderscore .
Que é de logradoiro público: \textunderscore aquelle tojal é maninho\textunderscore .
Terra inculta.
Logradoiro público de lenha ou pastagens: \textunderscore os maninhos do concelho\textunderscore .
Bens de alguém fallecido sem deixar filhos.
(Relaciona-se com o lat. \textunderscore manere\textunderscore ?)
\section{Maniota}
\begin{itemize}
\item {Grp. gram.:f.}
\end{itemize}
Peça, para prender a mão dos animaes.
(Cp. cast. \textunderscore maniota\textunderscore )
\section{Manipanço}
\begin{itemize}
\item {Grp. gram.:m.}
\end{itemize}
\begin{itemize}
\item {Utilização:Burl.}
\end{itemize}
Ídolo africano; fetiche.
Indivíduo muito gordo.
(Provavelmente, t. afr.)
\section{Manipo}
\begin{itemize}
\item {Grp. gram.:m.}
\end{itemize}
Planta anacardiácea de Cabo-Verde.
\section{Manipresto}
\begin{itemize}
\item {Grp. gram.:adj.}
\end{itemize}
\begin{itemize}
\item {Proveniência:(De \textunderscore mani...\textunderscore  + \textunderscore presto\textunderscore )}
\end{itemize}
Destro, expedito de mãos; prestímano.
\section{Manipueira}
\begin{itemize}
\item {Grp. gram.:f.}
\end{itemize}
Líquido venenoso, extrahido da mandioca ralada.
(Do tupi \textunderscore manipuera\textunderscore )
\section{Manipulação}
\begin{itemize}
\item {Grp. gram.:f.}
\end{itemize}
Acto ou modo de manipular.
\section{Manipulador}
\begin{itemize}
\item {Grp. gram.:m.}
\end{itemize}
Aquelle que manipula.
Instrumento que, pôsto em movimento pela mão do telegraphista, trasm.tte os sinaes telegráphicos.
\section{Manipular}
\begin{itemize}
\item {Grp. gram.:v. t.}
\end{itemize}
\begin{itemize}
\item {Proveniência:(De \textunderscore manípulo\textunderscore )}
\end{itemize}
Preparar com a mão; preparar com corpos simples (certos medicamentos).
\section{Manipular}
\begin{itemize}
\item {Grp. gram.:m.}
\end{itemize}
\begin{itemize}
\item {Proveniência:(Lat. \textunderscore manipularis\textunderscore )}
\end{itemize}
Cada um dos soldados de um manípulo, entre os Romanos.
\section{Manipulário}
\begin{itemize}
\item {Grp. gram.:m.}
\end{itemize}
\begin{itemize}
\item {Grp. gram.:Adj.}
\end{itemize}
\begin{itemize}
\item {Proveniência:(Lat. \textunderscore manipularius\textunderscore )}
\end{itemize}
Chefe de um manípulo, entre os Romanos.
Relativo aos manipulares.
\section{Maná}
\begin{itemize}
\item {Grp. gram.:m.}
\end{itemize}
\begin{itemize}
\item {Utilização:Fig.}
\end{itemize}
Alimento que, segundo a \textunderscore Bíblia\textunderscore , Deus mandou em fórma de chuva aos Israelitas no deserto.
Suco resinoso de algumas plantas.
Alimento delicioso.
Coisa excelente, vantajosa ou agradável.
Suco concreto de algumas espécies de peixe.
(Do hebr.)
\section{Manípulo}
\begin{itemize}
\item {Grp. gram.:m.}
\end{itemize}
\begin{itemize}
\item {Proveniência:(Lat. \textunderscore manípulus\textunderscore )}
\end{itemize}
Feixe de erva ou qualquer objecto, que a mão póde abranger.
Mão-cheia.
Companhia de 200 homens no exército romano:«\textunderscore ...os manípulos (tinham) vinte e cinco infantes\textunderscore ». \textunderscore Viriat. Trág.\textunderscore , II, 9.
Haste, que servia de bandeira a essas tropas.
Espécie de pequena estola, que pende do braço esquerdo do sacerdote, quando êste diz missa.
\section{Maniputo}
\begin{itemize}
\item {Grp. gram.:m.}
\end{itemize}
Imagem de um deus, entre os Negros da África.
Designação, que alguns povos negros davam ao rei de Portugal.
\section{Maniqueira}
\begin{itemize}
\item {Grp. gram.:f.}
\end{itemize}
Orthogr. errónea, com que os diccion. designam a \textunderscore manicuera\textunderscore .(V.manicuera)
\section{Maniqueísmo}
\begin{itemize}
\item {Grp. gram.:m.}
\end{itemize}
Seita dos maniqueus.
\section{Maniquete}
\begin{itemize}
\item {fónica:quê}
\end{itemize}
\begin{itemize}
\item {Grp. gram.:m.}
\end{itemize}
\begin{itemize}
\item {Proveniência:(Do lat. \textunderscore manica\textunderscore )}
\end{itemize}
Renda, que guarnece a manga das alvas dos padres.
\section{Maniqueus}
\begin{itemize}
\item {Grp. gram.:m. pl.}
\end{itemize}
Sectario do heresiarca do século III, Manes, que atribuía a criação do mundo aos dois princípios opostos, o do bem e o do mal.
\section{Manir}
\begin{itemize}
\item {Grp. gram.:v. i.}
\end{itemize}
\begin{itemize}
\item {Utilização:Prov.}
\end{itemize}
\begin{itemize}
\item {Utilização:trasm.}
\end{itemize}
Infiltrar-se; ressumar; alastrar-se lentamente.
(Relaciona-se com a \textunderscore manar\textunderscore ?)
\section{Maniroto}
\begin{itemize}
\item {fónica:rô}
\end{itemize}
\begin{itemize}
\item {Grp. gram.:adj.}
\end{itemize}
O mesmo que \textunderscore manirroto\textunderscore .
\section{Manirroto}
\begin{itemize}
\item {fónica:rô}
\end{itemize}
\begin{itemize}
\item {Grp. gram.:adj.}
\end{itemize}
\begin{itemize}
\item {Proveniência:(De \textunderscore man...\textunderscore  + \textunderscore roto\textunderscore )}
\end{itemize}
Que gasta muito; que espalha em abundância; perdulário.
\section{Manistérgio}
\begin{itemize}
\item {Grp. gram.:m.}
\end{itemize}
(V.manutérgio)
\section{Manita}
\begin{itemize}
\item {Grp. gram.:m. ,  f.  e  adj.}
\end{itemize}
\begin{itemize}
\item {Grp. gram.:F.}
\end{itemize}
\begin{itemize}
\item {Utilização:Fam.}
\end{itemize}
\begin{itemize}
\item {Proveniência:(Do rad. do lat. \textunderscore manus\textunderscore )}
\end{itemize}
O mesmo que \textunderscore maneta\textunderscore .
Mãozinha, mão pequena.
\section{Manita}
\begin{itemize}
\item {Grp. gram.:f.}
\end{itemize}
\begin{itemize}
\item {Proveniência:(De \textunderscore maná\textunderscore )}
\end{itemize}
Princípio cristalizável, extraido do succo resinoso de algumas plantas.
\section{Manitado}
\begin{itemize}
\item {Grp. gram.:adj.}
\end{itemize}
Que contém manita.
\section{Manitama}
\begin{itemize}
\item {Grp. gram.:f.}
\end{itemize}
\begin{itemize}
\item {Utilização:Chím.}
\end{itemize}
Substância que se prepara, aquecendo a mannita a 200° por alguns minutos.
\section{Manitartárico}
\begin{itemize}
\item {Grp. gram.:adj.}
\end{itemize}
\begin{itemize}
\item {Utilização:Chím.}
\end{itemize}
Diz-se de um ácido, que se prepara, elevando á temperatura de 120° a manita e o ácido tartárico em pesos iguaes.
\section{Manite}
\begin{itemize}
\item {Grp. gram.:f.}
\end{itemize}
Princípio cristalizável, que se extrai do maná. Cf. \textunderscore Techn. Rur.\textunderscore , 20.
O mesmo que \textunderscore manita\textunderscore ^2.
Doença das videiras, de que resulta a transformação da substância-sacarina numa aglomeração de pequeninas hastes, que constituem um fermento especial.
\section{Manítico}
\begin{itemize}
\item {Grp. gram.:adj.}
\end{itemize}
Relativo á manite.
\section{Manitina}
\begin{itemize}
\item {Grp. gram.:f.}
\end{itemize}
Princípio laxativo, extraido do maná.
\section{Manito}
\begin{itemize}
\item {Grp. gram.:m.}
\end{itemize}
O mesmo que \textunderscore manite\textunderscore .
\section{Manitô}
\begin{itemize}
\item {Grp. gram.:m.}
\end{itemize}
O mesmo que \textunderscore manitu\textunderscore . Cf. Gonç. Dias, \textunderscore Poesias\textunderscore , 11, 24 e 276.
\section{Manitu}
\begin{itemize}
\item {Grp. gram.:m.}
\end{itemize}
Uma das divindades dos indígenas da América do Norte.
\section{Maniuara}
\begin{itemize}
\item {fónica:ni-u}
\end{itemize}
\begin{itemize}
\item {Grp. gram.:m.}
\end{itemize}
\begin{itemize}
\item {Utilização:Bras}
\end{itemize}
Variedade de formiga amazónica.
\section{Maniva}
\begin{itemize}
\item {Grp. gram.:f.}
\end{itemize}
Caule de mandioca.
(Do tupi \textunderscore mani\textunderscore  + \textunderscore iva\textunderscore )
\section{Manivela}
\begin{itemize}
\item {Grp. gram.:f.}
\end{itemize}
\begin{itemize}
\item {Proveniência:(Do fr. \textunderscore manivelle\textunderscore ?)}
\end{itemize}
Peça de uma máquina, a que se dá movimento com a mão.
Peça de ferro ou madeira que, sujeita a qualquer fôrça motriz, põe em movimento um engenho ou máquina.
\section{Manivelar}
\begin{itemize}
\item {Grp. gram.:v. i.}
\end{itemize}
\begin{itemize}
\item {Utilização:Fig.}
\end{itemize}
Dar á manivela.
Agenciar.
\section{Manivérsia}
\begin{itemize}
\item {Grp. gram.:f.}
\end{itemize}
\begin{itemize}
\item {Utilização:Pop.}
\end{itemize}
\begin{itemize}
\item {Proveniência:(Do lat. \textunderscore manus\textunderscore  + \textunderscore versus\textunderscore )}
\end{itemize}
Patifaria; tranquibérnia.
\section{Manixeiro}
\begin{itemize}
\item {Grp. gram.:m.}
\end{itemize}
Árvore fructífera do Valle do Amazonas.
\section{Manixi}
\begin{itemize}
\item {Grp. gram.:m.}
\end{itemize}
Fruto do manixeiro.
\section{Manja}
\begin{itemize}
\item {Grp. gram.:f.}
\end{itemize}
\begin{itemize}
\item {Utilização:Prov.}
\end{itemize}
\begin{itemize}
\item {Utilização:alg.}
\end{itemize}
\begin{itemize}
\item {Utilização:Fam.}
\end{itemize}
\begin{itemize}
\item {Proveniência:(De \textunderscore manjar\textunderscore )}
\end{itemize}
Acto de comer; comida.
\section{Manja}
\begin{itemize}
\item {Grp. gram.:f.}
\end{itemize}
\begin{itemize}
\item {Utilização:Bras. do Ceará}
\end{itemize}
Folguedo de crianças, o mesmo que \textunderscore tempo-será\textunderscore .
\section{Manjaco}
\begin{itemize}
\item {Grp. gram.:m.}
\end{itemize}
Língua dos manjacos.
\section{Manjacos}
\begin{itemize}
\item {Grp. gram.:m. pl.}
\end{itemize}
Tríbo da Guiné portuguesa.
\section{Manjadoira}
\begin{itemize}
\item {Grp. gram.:f.}
\end{itemize}
(V.mangedoira)
\section{Manjaleco}
\begin{itemize}
\item {Grp. gram.:m.}
\end{itemize}
\begin{itemize}
\item {Utilização:Bras}
\end{itemize}
Marmanjo.
\section{Manja-léguas}
\begin{itemize}
\item {Grp. gram.:m.}
\end{itemize}
\begin{itemize}
\item {Utilização:Des.}
\end{itemize}
Aquelle que caminha muito.
Andarilho; corredor.
\section{Manjangome}
\begin{itemize}
\item {Grp. gram.:m.}
\end{itemize}
\begin{itemize}
\item {Utilização:Bras}
\end{itemize}
O mesmo que \textunderscore maria-gomes\textunderscore .
\section{Manjar}
\begin{itemize}
\item {Grp. gram.:m.}
\end{itemize}
\begin{itemize}
\item {Utilização:Fig.}
\end{itemize}
\begin{itemize}
\item {Grp. gram.:V. t.}
\end{itemize}
\begin{itemize}
\item {Utilização:Des.}
\end{itemize}
\begin{itemize}
\item {Proveniência:(Fr. \textunderscore manger\textunderscore )}
\end{itemize}
Qualquer substância alimentícia.
Iguaria delicada ou appetitosa.
Aquillo que alimenta ou deleita o espírito.
O mesmo que \textunderscore comer\textunderscore :«\textunderscore já não tinham que manjar\textunderscore ». Romance da nau Catrineta, no \textunderscore Romanceiro\textunderscore  de Garrett.
\section{Manjar-branco}
\begin{itemize}
\item {Grp. gram.:m.}
\end{itemize}
Iguaria ou bolo, que se fazia de arroz e gallinha ou peixe. Hoje, faz-se geralmente de leite e maisena.
\section{Manjar-dos-anjos}
\begin{itemize}
\item {Grp. gram.:m.}
\end{itemize}
Iguaria, feita de leite, açúcar e ovos.
\section{Manjar-grude}
\begin{itemize}
\item {Grp. gram.:m.}
\end{itemize}
\begin{itemize}
\item {Utilização:Bras}
\end{itemize}
Árvore urticácea, (\textunderscore lícania dealbata\textunderscore , Hook.), de fruto comestível e consistência glutinosa na polpa.
\section{Manjar-imperial}
\begin{itemize}
\item {Grp. gram.:m.}
\end{itemize}
Iguaria, em que entram gemmas do ovos, leite e farinha de arroz.
\section{Manjaronas}
\begin{itemize}
\item {Grp. gram.:m. pl.}
\end{itemize}
\begin{itemize}
\item {Utilização:Bras}
\end{itemize}
Tríbo de aborigenes, que habitam no Pará.
\section{Manjar-principal}
\begin{itemize}
\item {Grp. gram.:m.}
\end{itemize}
Iguaria, feita de pão ralado, queijo, gemmas de ovos e leite coalhado.
\section{Manjar-real}
\begin{itemize}
\item {Grp. gram.:m.}
\end{itemize}
Antiga iguaria, feita de gallinha, farinha e amêndoas.
\section{Manjeira}
\begin{itemize}
\item {Grp. gram.:f.}
\end{itemize}
\begin{itemize}
\item {Utilização:T. de Pare -de-Coira}
\end{itemize}
\begin{itemize}
\item {Utilização:des.}
\end{itemize}
Transtôrno, decepção.
(Cp. \textunderscore má-geira\textunderscore )
\section{Manjola}
\begin{itemize}
\item {Grp. gram.:f.}
\end{itemize}
\begin{itemize}
\item {Utilização:Bras}
\end{itemize}
O mesmo que \textunderscore mangual\textunderscore .
\section{Manjorra}
\begin{itemize}
\item {fónica:jô}
\end{itemize}
\begin{itemize}
\item {Grp. gram.:f.}
\end{itemize}
Travessa, fixada no eixo central da nora, e á qual se prende o gado para extracção de água de poço ou cisterna.
(Cp. \textunderscore almanjarra\textunderscore )
\section{Manjovo}
\begin{itemize}
\item {Grp. gram.:m.}
\end{itemize}
O mesmo que \textunderscore munjovo\textunderscore .
\section{Manju}
\begin{itemize}
\item {Grp. gram.:m.}
\end{itemize}
O mesmo que \textunderscore manchu\textunderscore .
\section{Mânjua}
\begin{itemize}
\item {Grp. gram.:f.}
\end{itemize}
\begin{itemize}
\item {Utilização:Prov.}
\end{itemize}
\begin{itemize}
\item {Utilização:Pesc.}
\end{itemize}
\begin{itemize}
\item {Utilização:alg.}
\end{itemize}
\begin{itemize}
\item {Utilização:Ant.}
\end{itemize}
Sardinha.
Alimento.
Pastagem.
(Cp. \textunderscore manjar\textunderscore )
\section{Manjuba}
\begin{itemize}
\item {Grp. gram.:f.}
\end{itemize}
\begin{itemize}
\item {Utilização:Bras. do Rio}
\end{itemize}
Peixe muito miúdo.
\section{Manjuba}
\begin{itemize}
\item {Grp. gram.:f.}
\end{itemize}
\begin{itemize}
\item {Utilização:Bras}
\end{itemize}
Comida.
\section{Manjunda}
\begin{itemize}
\item {Grp. gram.:m.}
\end{itemize}
Designação genérica dos batrácios, entre os Ambuelas.
\section{Manná}
\begin{itemize}
\item {Grp. gram.:m.}
\end{itemize}
\begin{itemize}
\item {Utilização:Fig.}
\end{itemize}
Alimento que, segundo a \textunderscore Bíblia\textunderscore , Deus mandou em fórma de chuva aos Israelitas no deserto.
Suco resinoso de algumas plantas.
Alimento delicioso.
Coisa excellente, vantajosa ou agradável.
Suco concreto de algumas espécies de peixe.
(Do hebr.)
\section{Mannita}
\begin{itemize}
\item {Grp. gram.:f.}
\end{itemize}
\begin{itemize}
\item {Proveniência:(De \textunderscore manná\textunderscore )}
\end{itemize}
Princípio crystallizável, extrahido do succo resinoso de algumas plantas.
\section{Mannitado}
\begin{itemize}
\item {Grp. gram.:adj.}
\end{itemize}
Que contém mannita.
\section{Mannitama}
\begin{itemize}
\item {Grp. gram.:f.}
\end{itemize}
\begin{itemize}
\item {Utilização:Chím.}
\end{itemize}
Substância que se prepara, aquecendo a mannita a 200° por alguns minutos.
\section{Mannitartárico}
\begin{itemize}
\item {Grp. gram.:adj.}
\end{itemize}
\begin{itemize}
\item {Utilização:Chím.}
\end{itemize}
Diz-se de um ácido, que se prepara, elevando á temperatura de 120° a manita e o ácido tartárico em pesos iguaes.
\section{Mannite}
\begin{itemize}
\item {Grp. gram.:f.}
\end{itemize}
Princípio crystallizável, que se extrai do manná. Cf. \textunderscore Techn. Rur.\textunderscore , 20.
O mesmo que \textunderscore mannita\textunderscore .
Doença das videiras, de que resulta a transformação da substância-saccharina numa agglomeração de pequeninas hastes, que constituem um fermento especial.
\section{Mannítico}
\begin{itemize}
\item {Grp. gram.:adj.}
\end{itemize}
Relativo á mannite.
\section{Mannitina}
\begin{itemize}
\item {Grp. gram.:f.}
\end{itemize}
Princípio laxativo, extrahido do manná.
\section{Mannito}
\begin{itemize}
\item {Grp. gram.:m.}
\end{itemize}
O mesmo que \textunderscore mannite\textunderscore .
\section{Mano}
\begin{itemize}
\item {Grp. gram.:m.}
\end{itemize}
\begin{itemize}
\item {Utilização:ant.}
\end{itemize}
\begin{itemize}
\item {Utilização:Fam.}
\end{itemize}
\begin{itemize}
\item {Utilização:Ant.}
\end{itemize}
\begin{itemize}
\item {Grp. gram.:Adj.}
\end{itemize}
O mesmo que \textunderscore irmão\textunderscore .
Tratamento familiar do cunhado ou cunhada para cunhado.
Amigo, vizinho, collega, (na conversação familiar). Cf. \textunderscore Eufrosina\textunderscore .
Muito amigo, intimo:«\textunderscore ...muito mana e acamaradada com...\textunderscore »Camillo, \textunderscore Mulher Fatal\textunderscore , 222.
(Cp. \textunderscore germano\textunderscore )
\section{Mano}
\begin{itemize}
\item {Grp. gram.:f.}
\end{itemize}
\begin{itemize}
\item {Proveniência:(Lat. \textunderscore manus\textunderscore )}
\end{itemize}
O mesmo que \textunderscore mão\textunderscore , em algumas phrases: \textunderscore mano a mano\textunderscore ; \textunderscore de mano a mano\textunderscore .«\textunderscore Pôr manos a lavor.\textunderscore »\textunderscore Eufrosina\textunderscore , 305.
\section{Manobra}
\begin{itemize}
\item {Grp. gram.:f.}
\end{itemize}
\begin{itemize}
\item {Grp. gram.:Pl.}
\end{itemize}
\begin{itemize}
\item {Utilização:Náut.}
\end{itemize}
\begin{itemize}
\item {Proveniência:(Do b. lat. \textunderscore manopera\textunderscore )}
\end{itemize}
Exercício militar.
Acto ou arte de dirigir convenientemente a andadura das embarcações.
Faina de marinheiro; marinharia.
Habilidade, destreza.
Astúcia, ardil.
Meio de illudir ou enganar.
Prestidigitação.
Cabos, com que se governam as velas.
\section{Manobrador}
\begin{itemize}
\item {Grp. gram.:m.}
\end{itemize}
Aquelle que manobra.
\section{Manobrar}
\begin{itemize}
\item {Grp. gram.:v. t.}
\end{itemize}
\begin{itemize}
\item {Grp. gram.:V. i.}
\end{itemize}
Realizar, por meio de manobra.
Encaminhar ou dirigir habilmente: \textunderscore manobrar uma conspiração\textunderscore .
Agenciar.
Governar ou dirigir a andadura de (embarcações).
Fazer com astúcia ou manha: \textunderscore manobrar mexericos\textunderscore .
Fazer exercícios militares.
Fazer qualquer exercício.
Funccionar; trabalhar; lidar.
\section{Manobreiro}
\begin{itemize}
\item {Grp. gram.:m.}
\end{itemize}
Aquelle que dirige manobras militares.
Aquelle que sabe manobrar.
Aquelle que escreve á cêrca de manobras náuticas.
Livro ou arte, que trata de manobras náuticas.
Aquelle que dirige manobras náuticas.
\section{Manobrista}
\begin{itemize}
\item {Grp. gram.:m.}
\end{itemize}
\begin{itemize}
\item {Proveniência:(De \textunderscore manobra\textunderscore )}
\end{itemize}
Aquelle que conhece e pratíca bem as manobras das embarcações.
\section{Manoca}
\begin{itemize}
\item {Grp. gram.:f.}
\end{itemize}
\begin{itemize}
\item {Utilização:Bras}
\end{itemize}
Feixe de fôlhas de tabaco.
Rôlo de tabaco.--É brasileirismo corrente, que já se vai adoptando em Portugal, na indústria do tabaco.
(Poderia vir do fr. \textunderscore manoque\textunderscore , cuja origem os Franceses desconhecem; mas dêste desconhecimento infiro que o fr. \textunderscore manoque\textunderscore  terá vindo do port. \textunderscore manoca\textunderscore , que, nêste caso, será da mesma or. que \textunderscore manhoco\textunderscore  e \textunderscore manolho\textunderscore )
\section{Manocada}
\begin{itemize}
\item {Grp. gram.:f.}
\end{itemize}
\begin{itemize}
\item {Utilização:T. de Alcanena}
\end{itemize}
\begin{itemize}
\item {Proveniência:(De \textunderscore manoca\textunderscore )}
\end{itemize}
Mancheia, manojo, manípulo.
\section{Manocar}
\begin{itemize}
\item {Grp. gram.:v. t.}
\end{itemize}
Formar manocas de (tabaco).
\section{Manógrafo}
\begin{itemize}
\item {Grp. gram.:m.}
\end{itemize}
Apparelho, destinado a traçar photographicamente o diagramma dos motores explosivos ou das máquinas de vapor, de grande velocidade.
\section{Manógrapho}
\begin{itemize}
\item {Grp. gram.:m.}
\end{itemize}
Apparelho, destinado a traçar photographicamente o diagramma dos motores explosivos ou das máquinas de vapor, de grande velocidade.
\section{Manoio}
\begin{itemize}
\item {Grp. gram.:m.}
\end{itemize}
\begin{itemize}
\item {Utilização:Prov.}
\end{itemize}
\begin{itemize}
\item {Utilização:Pesc.}
\end{itemize}
\begin{itemize}
\item {Utilização:alg.}
\end{itemize}
Cada uma das porções iguaes da linha que formam os espinéis.
(Cp. \textunderscore manojo\textunderscore )
\section{Manojeiro}
\begin{itemize}
\item {Grp. gram.:m.}
\end{itemize}
\begin{itemize}
\item {Proveniência:(De \textunderscore manojo\textunderscore )}
\end{itemize}
Aquelle que junta e ata os vellos, espalhados pela tosquia do gado ovelhum.
\section{Manojo}
\begin{itemize}
\item {Grp. gram.:m.}
\end{itemize}
Mólho ou feixe, que se póde abranger com a mão.
(Cast. \textunderscore manojo\textunderscore )
\section{Manola}
\begin{itemize}
\item {Grp. gram.:f.}
\end{itemize}
Rapariga madrilena, de costumes ligeiros e de baixa extracção. Cf. Chagas, \textunderscore Côrte de D. João V\textunderscore , 59.
\section{Manolho}
\begin{itemize}
\item {fónica:nô}
\end{itemize}
\begin{itemize}
\item {Grp. gram.:m.}
\end{itemize}
\begin{itemize}
\item {Proveniência:(Do lat. hyp. \textunderscore manupulum\textunderscore , por \textunderscore manipulum\textunderscore )}
\end{itemize}
O mesmo que \textunderscore manojo\textunderscore .
\section{Manometria}
\begin{itemize}
\item {Grp. gram.:f.}
\end{itemize}
Arte de empregar o manómetro.
\section{Manométrico}
\begin{itemize}
\item {Grp. gram.:adj.}
\end{itemize}
Relativo a manometria.
\section{Manómetro}
\begin{itemize}
\item {Grp. gram.:m.}
\end{itemize}
\begin{itemize}
\item {Proveniência:(Do gr. \textunderscore manos\textunderscore  + \textunderscore metron\textunderscore )}
\end{itemize}
Apparelho de Phýsica, próprio para fazer conhecer a fôrça elástica dos gases e dos vapores.
\section{Manona}
\begin{itemize}
\item {Grp. gram.:f.}
\end{itemize}
\begin{itemize}
\item {Utilização:Prov.}
\end{itemize}
\begin{itemize}
\item {Utilização:trasm.}
\end{itemize}
Figurinha de mulher.
\section{Manopé}
\begin{itemize}
\item {Grp. gram.:m.}
\end{itemize}
Árvore brasileira, própria para construcções.
\section{Manopla}
\begin{itemize}
\item {Grp. gram.:f.}
\end{itemize}
\begin{itemize}
\item {Utilização:Chul.}
\end{itemize}
\begin{itemize}
\item {Proveniência:(Do rad. do lat. \textunderscore manus\textunderscore )}
\end{itemize}
Luva de ferro, que fazia parte das antigas armaduras guerreiras.
Chicote comprido, próprio de cocheiro.
Mão grande e mal feita; manápula.
\section{Manoscópio}
\begin{itemize}
\item {Grp. gram.:m.}
\end{itemize}
\begin{itemize}
\item {Proveniência:(Do gr. \textunderscore manos\textunderscore  + \textunderscore skopein\textunderscore )}
\end{itemize}
Instrumento de Phýsica, para indicar as variações da densidade atmosphérica.
\section{Manotaço}
\begin{itemize}
\item {Grp. gram.:m.}
\end{itemize}
\begin{itemize}
\item {Utilização:Bras. do S}
\end{itemize}
Pancada, que o cavallo dá com a mão, para o lado ou para deante.
(Cast. \textunderscore manotazo\textunderscore )
\section{Manotear}
\begin{itemize}
\item {Grp. gram.:v. t.  e  i.}
\end{itemize}
\begin{itemize}
\item {Utilização:Bras}
\end{itemize}
\begin{itemize}
\item {Proveniência:(T. cast.)}
\end{itemize}
Dar manotaços (o cavallo).
\section{Mano-tolo}
\begin{itemize}
\item {Grp. gram.:m.}
\end{itemize}
\begin{itemize}
\item {Utilização:Bras}
\end{itemize}
Passarinho, também chamado \textunderscore dorminhoco\textunderscore .
\section{Manquadra}
\begin{itemize}
\item {Grp. gram.:f.}
\end{itemize}
O mesmo que \textunderscore mão-quadra\textunderscore .
\section{Manquecer}
\begin{itemize}
\item {Grp. gram.:v. i.}
\end{itemize}
Tornar-se manco.
\section{Manqueira}
\begin{itemize}
\item {Grp. gram.:f.}
\end{itemize}
\begin{itemize}
\item {Utilização:Fig.}
\end{itemize}
Defeito de manco; acto de manquejar.
Defeito; senão.
\section{Manquejante}
\begin{itemize}
\item {Grp. gram.:adj.}
\end{itemize}
Que manqueja.
\section{Manquejar}
\begin{itemize}
\item {Grp. gram.:v. i.}
\end{itemize}
\begin{itemize}
\item {Utilização:Fig.}
\end{itemize}
Estar manco.
Coxear.
Têr defeito.
Andar pouco, (falando-se de embarcações, que não acompanham outras).
\section{Manquês}
\begin{itemize}
\item {Grp. gram.:m.}
\end{itemize}
Dialecto céltico, que se falou na ilha de Man.
\section{Manquitó}
\begin{itemize}
\item {Grp. gram.:m.}
\end{itemize}
\begin{itemize}
\item {Utilização:Pop.}
\end{itemize}
\begin{itemize}
\item {Proveniência:(De \textunderscore manco\textunderscore )}
\end{itemize}
Homem coxo.
\section{Manquitola}
\begin{itemize}
\item {Grp. gram.:m.}
\end{itemize}
\begin{itemize}
\item {Utilização:ant.}
\end{itemize}
\begin{itemize}
\item {Utilização:Pop.}
\end{itemize}
O mesmo que \textunderscore manquitó\textunderscore .
\section{Mansamente}
\begin{itemize}
\item {Grp. gram.:adv.}
\end{itemize}
De modo manso; serenamente; com brandura.
\section{Mansão}
\begin{itemize}
\item {Grp. gram.:f.}
\end{itemize}
\begin{itemize}
\item {Utilização:Fig.}
\end{itemize}
\begin{itemize}
\item {Proveniência:(Lat. \textunderscore mansio\textunderscore )}
\end{itemize}
Morada.
Situação.
\section{Mansarda}
\begin{itemize}
\item {Grp. gram.:f.}
\end{itemize}
\begin{itemize}
\item {Proveniência:(Fr. \textunderscore mansarde\textunderscore , de \textunderscore Mansard\textunderscore , n. p.)}
\end{itemize}
Agua furtada, trapeira.
Morada reles.--Diz Castilho que é gallicismo, introduzido pela ignorância dos traduzidores de novelas.
\section{Mansarrão}
\begin{itemize}
\item {Grp. gram.:m.  e  adj.}
\end{itemize}
\begin{itemize}
\item {Proveniência:(De \textunderscore manso\textunderscore )}
\end{itemize}
O que é muito manso ou sossegado.
Aquelle que tem muita pachorra.
\section{Mansessor}
\begin{itemize}
\item {Grp. gram.:m.}
\end{itemize}
\begin{itemize}
\item {Utilização:Ant.}
\end{itemize}
O mesmo que \textunderscore testamenteiro\textunderscore .
\section{Mansidade}
\begin{itemize}
\item {Grp. gram.:f.}
\end{itemize}
\begin{itemize}
\item {Utilização:Prov.}
\end{itemize}
\begin{itemize}
\item {Utilização:Ant.}
\end{itemize}
O mesmo que \textunderscore mansidão\textunderscore .
Qualidade de sonso: \textunderscore com aquella mansidade, parece que não quebra um prato e deita a prateleira abaixo\textunderscore .
\section{Mansidão}
\begin{itemize}
\item {Grp. gram.:f.}
\end{itemize}
Qualidade ou estado de manso.
Índole pacífica.
Brandura ou suavidade nas palavras ou na voz.
Vagar ou lentidão no falar.
\section{Mansilha}
\begin{itemize}
\item {Grp. gram.:f.}
\end{itemize}
\begin{itemize}
\item {Utilização:Ant.}
\end{itemize}
Azorrague, para amansar ou castigar.
(Por \textunderscore amansilha\textunderscore , de \textunderscore amansar\textunderscore )
\section{Mansinho}
\begin{itemize}
\item {Grp. gram.:adj.}
\end{itemize}
\begin{itemize}
\item {Grp. gram.:Loc. adv.}
\end{itemize}
(dem. de \textunderscore manso\textunderscore )
\textunderscore De mansinho\textunderscore , ao de leve, sem fazer ruido; mansamente.
\section{Mansionário}
\begin{itemize}
\item {Grp. gram.:adj.}
\end{itemize}
\begin{itemize}
\item {Proveniência:(Lat. \textunderscore mansionarius\textunderscore )}
\end{itemize}
Espécie de sacristão, que tinha a seu cargo a guarda da igreja, e residia junto della.
Em algumas cathedraes, beneficiado, de ordem immediatamente inferior á dos cónegos.
\section{Manso}
\begin{itemize}
\item {Grp. gram.:adj.}
\end{itemize}
\begin{itemize}
\item {Grp. gram.:Adv.}
\end{itemize}
\begin{itemize}
\item {Proveniência:(Do lat. \textunderscore mansues\textunderscore )}
\end{itemize}
Que tem gênio brando, pacífico: \textunderscore homem manso\textunderscore .
Sereno, sossegado: \textunderscore lago manso\textunderscore .
Que não faz barulho.
Domesticado: \textunderscore coelho manso\textunderscore .
Cultivado, (falando-se de plantas).
Mansamente; devagar.
\section{Mansobre}
\begin{itemize}
\item {Grp. gram.:m.}
\end{itemize}
\begin{itemize}
\item {Utilização:Ant.}
\end{itemize}
Espécie de verso, em que se repetia a mesma palavra no meio e no fim.--Segundo C. Michaelis, é má leitura do ant. port. \textunderscore mordobre\textunderscore , grande som duplo.
\section{Mansuetíssimo}
\begin{itemize}
\item {Grp. gram.:adj.}
\end{itemize}
\begin{itemize}
\item {Utilização:Des.}
\end{itemize}
\begin{itemize}
\item {Proveniência:(Do lat. \textunderscore mansuetus\textunderscore )}
\end{itemize}
Muito manso.
\section{Mansuetude}
\begin{itemize}
\item {Grp. gram.:f.}
\end{itemize}
\begin{itemize}
\item {Proveniência:(Lat. \textunderscore mansuetudo\textunderscore )}
\end{itemize}
O mesmo que \textunderscore mansidão\textunderscore .
\section{Manta}
\begin{itemize}
\item {Grp. gram.:f.}
\end{itemize}
\begin{itemize}
\item {Utilização:Prov.}
\end{itemize}
\begin{itemize}
\item {Utilização:Fam.}
\end{itemize}
\begin{itemize}
\item {Utilização:Prov.}
\end{itemize}
\begin{itemize}
\item {Utilização:dur.}
\end{itemize}
\begin{itemize}
\item {Utilização:Prov.}
\end{itemize}
Cobertor, especialmente o que se destina á cama.
Lenço de abafar, para cabeça e ombros.
Tira de seda ou de outro tecido, com que se fórma laço ao pescoço, servindo de gravata.
Xairel de lan.
Grande pano, do feitio de um cobertor, e que serve para agasalhar a maior parte do corpo.
Cobrejão.
Rêgo, para plantação de bacêllo.
Terra, que se junta entre dois sulcos parallelos, para a sementeira de várias plantas hortenses: \textunderscore mantas de meloal\textunderscore .
Antiga máquina de guerra.
Parapeito portátil, para resguardar tropas, contra os tiros do inimigo.
Tôldo, em que caem as azeitonas das oliveiras que se varejam.
Cada uma das camadas parallelas da terra, que os cavadores vão formando, ao romper a terra.
Pândega: \textunderscore pintar a manta\textunderscore , andar na pândega.
\textunderscore Manta de toicinho\textunderscore , toicinho da metade de um porco.
O mesmo que \textunderscore cango\textunderscore .
\textunderscore Manta de pêso\textunderscore , espécie de cobertor de lan, grosseiro, que se vende a pêso. (Colhida em Minde)
(Cp. \textunderscore manto\textunderscore )
\section{Manta}
\begin{itemize}
\item {Grp. gram.:f.}
\end{itemize}
\begin{itemize}
\item {Utilização:Mad}
\end{itemize}
\begin{itemize}
\item {Proveniência:(Fr. \textunderscore mante\textunderscore )}
\end{itemize}
Nome vulgar de um crustáceo; coccinela.
O mesmo que \textunderscore milhano\textunderscore .
\section{Manta-de-gato}
\begin{itemize}
\item {Grp. gram.:f.}
\end{itemize}
\begin{itemize}
\item {Utilização:Prov.}
\end{itemize}
\begin{itemize}
\item {Utilização:minh.}
\end{itemize}
O mesmo que \textunderscore faixeiro\textunderscore .
\section{Mantalona}
\begin{itemize}
\item {Grp. gram.:f.}
\end{itemize}
Tecido, com que na Índia se fabricam velas de embarcações.
\section{Mantalote}
\begin{itemize}
\item {Grp. gram.:m.}
\end{itemize}
\begin{itemize}
\item {Utilização:Ant.}
\end{itemize}
\begin{itemize}
\item {Proveniência:(Do rad. de \textunderscore manta\textunderscore ^1?)}
\end{itemize}
Tábua, semelhante á tampa de uma caixa, e que servia de leito.
\section{Mantana}
\begin{itemize}
\item {Grp. gram.:f.}
\end{itemize}
\begin{itemize}
\item {Utilização:Mad}
\end{itemize}
O mesmo que \textunderscore milhano\textunderscore .
\section{Mantão}
\begin{itemize}
\item {Grp. gram.:f.}
\end{itemize}
\begin{itemize}
\item {Utilização:Ant.}
\end{itemize}
\begin{itemize}
\item {Proveniência:(De \textunderscore manto\textunderscore )}
\end{itemize}
Espécie de capote curto. Cf. \textunderscore Dissertações Chronologicas\textunderscore , V, 308.
\section{Mantar}
\begin{itemize}
\item {Grp. gram.:v. t.}
\end{itemize}
Cavar em mantas (a terra), para plantação de bacêllo.
\section{Mantaz}
\begin{itemize}
\item {Grp. gram.:m.}
\end{itemize}
Tecido antigo de Cambaia.
\section{Manteação}
\begin{itemize}
\item {Grp. gram.:f.}
\end{itemize}
Acto ou effeito de mantear.
\section{Manteador}
\begin{itemize}
\item {Grp. gram.:m.  e  adj.}
\end{itemize}
O que manteia.
\section{Mantear}
\begin{itemize}
\item {Grp. gram.:v. t.}
\end{itemize}
\begin{itemize}
\item {Utilização:Fig.}
\end{itemize}
\begin{itemize}
\item {Grp. gram.:V. i.}
\end{itemize}
Pôr e agitar (alguém) sôbre uma manta, e, tomando esta pelas pontas, fazer saltar a pessôa que nella se deitou.
Chamar (o toiro) com a manta ou capa, suspensa na muleta.
Fazer zangar, importunar.
Cavar terra, fazendo manta.
\section{Mantedor}
\begin{itemize}
\item {Grp. gram.:m.}
\end{itemize}
\begin{itemize}
\item {Utilização:Des.}
\end{itemize}
O mesmo ou melhor que \textunderscore mantenedor\textunderscore .
\section{Manteiga}
\begin{itemize}
\item {Grp. gram.:f.}
\end{itemize}
\begin{itemize}
\item {Utilização:Pop.}
\end{itemize}
\begin{itemize}
\item {Proveniência:(Do lat. hyp. \textunderscore nattatica\textunderscore , seg. Cornu)}
\end{itemize}
Substância alimentícia, extrahida da nata do leite.
Substância gordurosa de algumas plantas.
Nome de alguns chloretos metállicos.
Variedade de feijão.
Lisonja; lábia.
Pêra, o mesmo que \textunderscore riscadinha\textunderscore .
\section{Manteigaria}
\begin{itemize}
\item {Grp. gram.:f.}
\end{itemize}
Local ou estabelecimento, onde se vende ou fabrica manteiga.
\section{Manteigoso}
\begin{itemize}
\item {Grp. gram.:adj.}
\end{itemize}
O mesmo que \textunderscore manteiguento\textunderscore .
\section{Manteigueira}
\begin{itemize}
\item {Grp. gram.:f.}
\end{itemize}
Vaso, em que se leva manteiga á mesa.
\section{Manteigueiro}
\begin{itemize}
\item {Grp. gram.:m.}
\end{itemize}
\begin{itemize}
\item {Grp. gram.:Adj.}
\end{itemize}
\begin{itemize}
\item {Utilização:Fam.}
\end{itemize}
\begin{itemize}
\item {Utilização:Pop.}
\end{itemize}
\begin{itemize}
\item {Utilização:Pop.}
\end{itemize}
Fabricante ou vendedor de manteiga.
Que gosta muito de manteiga.
Que lisonjeia.
Que faz meiguices, muita vez interesseiras.
O mesmo que \textunderscore merceeiro\textunderscore ^1.
\section{Manteiguento}
\begin{itemize}
\item {Grp. gram.:adj.}
\end{itemize}
Que tem muita manteiga.
Que tem o sabor da manteiga.
Gorduroso.
\section{Manteiguilha}
\begin{itemize}
\item {Grp. gram.:f.}
\end{itemize}
\begin{itemize}
\item {Proveniência:(De \textunderscore manteiga\textunderscore )}
\end{itemize}
Banha odorífera, em que entram essências de flôres.
\section{Manteiro}
\begin{itemize}
\item {Grp. gram.:m.}
\end{itemize}
\begin{itemize}
\item {Proveniência:(De \textunderscore manta\textunderscore ^1)}
\end{itemize}
Fabricante ou vendedor de mantas.
\section{Mantel}
\begin{itemize}
\item {Grp. gram.:m.}
\end{itemize}
\begin{itemize}
\item {Grp. gram.:Pl.}
\end{itemize}
\begin{itemize}
\item {Utilização:Ant.}
\end{itemize}
\begin{itemize}
\item {Proveniência:(Do lat. \textunderscore mantele\textunderscore )}
\end{itemize}
Toalha de altar ou de mesa.
Roupas de mesa.
\section{Mantelado}
\begin{itemize}
\item {Grp. gram.:adj.}
\end{itemize}
\begin{itemize}
\item {Utilização:Heráld.}
\end{itemize}
\begin{itemize}
\item {Proveniência:(Do rad. de \textunderscore mantel\textunderscore )}
\end{itemize}
Que tem mantelér.
\section{Mantelão}
\begin{itemize}
\item {Grp. gram.:m.}
\end{itemize}
Mantelete grande, usado por monsenhores.
\section{Mantelér}
\begin{itemize}
\item {Grp. gram.:m.}
\end{itemize}
\begin{itemize}
\item {Utilização:Heráld.}
\end{itemize}
\begin{itemize}
\item {Proveniência:(Do rad. de \textunderscore mantel\textunderscore )}
\end{itemize}
Dois meios escudos, oppostos, formados no campo por duas linhas curvas, de cujos extremos uns se ajuntam no alto do escudo, e os outros se afastam, tocando respectivamente a esquerda e a direita do campo.
\section{Manteleta}
\begin{itemize}
\item {fónica:lê}
\end{itemize}
\begin{itemize}
\item {Grp. gram.:f.}
\end{itemize}
\begin{itemize}
\item {Utilização:Prov.}
\end{itemize}
\begin{itemize}
\item {Utilização:minh.}
\end{itemize}
Espécie de lenço grande, com que cobrem a cabeça as mulheres de Castro-Laboreiro.
(Cp. \textunderscore mantelete\textunderscore )
\section{Mantelete}
\begin{itemize}
\item {fónica:lê}
\end{itemize}
\begin{itemize}
\item {Grp. gram.:m.}
\end{itemize}
\begin{itemize}
\item {Proveniência:(De \textunderscore mantel\textunderscore )}
\end{itemize}
Curta vestidura ecclesiástica, que se usa sôbre o roquete.
Pequena capa de senhoras.
Parapeito militar.
Capa curta, com que os cavalleiros cobriam o capacete e o escudo.
\section{Mantém}
\begin{itemize}
\item {Grp. gram.:m.}
\end{itemize}
\begin{itemize}
\item {Proveniência:(Do rad. de \textunderscore mantel\textunderscore )}
\end{itemize}
Toalha de mesa:«\textunderscore levantou-se primeiro que os mantens\textunderscore ». F. Manuel, \textunderscore Apólogos\textunderscore .
\section{Mantena}
\begin{itemize}
\item {Grp. gram.:adj.}
\end{itemize}
\begin{itemize}
\item {Utilização:Bras. de Goiás}
\end{itemize}
Bom; óptimo.
\section{Mantença}
\begin{itemize}
\item {Grp. gram.:f.}
\end{itemize}
\begin{itemize}
\item {Proveniência:(Do rad. de \textunderscore manter\textunderscore )}
\end{itemize}
Aquillo que mantém ou sustenta.
Sustento.
Manutenção.
\section{Mantenedor}
\begin{itemize}
\item {Grp. gram.:m.}
\end{itemize}
\begin{itemize}
\item {Utilização:Ant.}
\end{itemize}
\begin{itemize}
\item {Grp. gram.:Adj.}
\end{itemize}
\begin{itemize}
\item {Proveniência:(T. cast.)}
\end{itemize}
Aquelle que mantém, aquelle que sustenta.
Defensor; campeão.
Cavalleiro principal nos torneios.
Que sustenta, mantém, defende ou protege.
\section{Manter}
\begin{itemize}
\item {Grp. gram.:v. t.}
\end{itemize}
\begin{itemize}
\item {Grp. gram.:V. p.}
\end{itemize}
\begin{itemize}
\item {Proveniência:(Do b. lat. \textunderscore mantenere\textunderscore )}
\end{itemize}
Fornecer alimentos a; sustentar.
Conservar.
Cumprir; observar: \textunderscore manter a lei\textunderscore .
Permanecer, sustentar-se.
\section{Mantes}
\begin{itemize}
\item {Grp. gram.:m. pl.}
\end{itemize}
\begin{itemize}
\item {Proveniência:(Do gr. \textunderscore mantis\textunderscore )}
\end{itemize}
Insectos orthópteros, de thoracete comprido.
\section{Mantéu}
\begin{itemize}
\item {Grp. gram.:m.}
\end{itemize}
\begin{itemize}
\item {Utilização:T. de Tôrres Vedras}
\end{itemize}
Capa com collarinho, usada por frades.
Collarinho encanudado, ou com abas largas pendente.
Saia lisa, sem pregas.
O mesmo que \textunderscore cueiro\textunderscore .
\section{Mantéu}
\begin{itemize}
\item {Grp. gram.:m.}
\end{itemize}
\begin{itemize}
\item {Proveniência:(Lat. \textunderscore manteium\textunderscore )}
\end{itemize}
Lugar, onde se pronunciavam oráculos, entre os antigos.
\section{Manteúdo}
\begin{itemize}
\item {Grp. gram.:adj.}
\end{itemize}
\begin{itemize}
\item {Grp. gram.:M.}
\end{itemize}
O mesmo que \textunderscore mantido\textunderscore : Concubina \textunderscore teúda\textunderscore  e \textunderscore manteúda\textunderscore .
Casta de uva branca do Algarve.
\section{Mantiaria}
\begin{itemize}
\item {Grp. gram.:f.}
\end{itemize}
\begin{itemize}
\item {Proveniência:(Do rad. de \textunderscore mantieiro\textunderscore )}
\end{itemize}
Cargo ou offício de mantieiro.
Objectos, confiados á guarda do mantieiro.
Casa, onde se guardam êsses objectos.
\section{Mântica}
\begin{itemize}
\item {Grp. gram.:f.}
\end{itemize}
\begin{itemize}
\item {Proveniência:(Lat. \textunderscore mantica\textunderscore )}
\end{itemize}
Pequeno saco; alforge.
\section{Manticostumes}
\begin{itemize}
\item {Grp. gram.:m.}
\end{itemize}
Aquillo que mantém os costumes ou as tradições?:«\textunderscore tivemos por manticostumes o teor monárchico\textunderscore ». Cf. Filinto, XXII, 115.
\section{Mantido}
\begin{itemize}
\item {Grp. gram.:adj.}
\end{itemize}
\begin{itemize}
\item {Proveniência:(De \textunderscore manter\textunderscore )}
\end{itemize}
Alimentado; sustentado.
Conservado: \textunderscore costumes, mantidos desde há séculos\textunderscore .
\section{Mantieiro}
\begin{itemize}
\item {Grp. gram.:m.}
\end{itemize}
\begin{itemize}
\item {Utilização:Prov.}
\end{itemize}
\begin{itemize}
\item {Utilização:alent.}
\end{itemize}
\begin{itemize}
\item {Proveniência:(Do rad. de \textunderscore mantel\textunderscore )}
\end{itemize}
Empregado, que tinha a seu cargo a guarda dos manteis na casa real.
Vendedor de água, transportada em burros, com cangalhas e bilhas.
\section{Mantilha}
\begin{itemize}
\item {Grp. gram.:f.}
\end{itemize}
\begin{itemize}
\item {Utilização:Des.}
\end{itemize}
\begin{itemize}
\item {Proveniência:(De \textunderscore manta\textunderscore ^1)}
\end{itemize}
Manto, com que as mulheres cobrem a cabeça.
Bioco; capuz.
Véu de seda ou rendas, que cai em pregas pelas costas, e que é usado principalmente em Espanha.
Faixa infantil; cueiro:«\textunderscore ...saído das mantilhas...\textunderscore »Sousa, \textunderscore Vida do Arceb.\textunderscore , I, 11.
\section{Mantimento}
\begin{itemize}
\item {Grp. gram.:m.}
\end{itemize}
\begin{itemize}
\item {Proveniência:(De \textunderscore manter\textunderscore )}
\end{itemize}
Aquillo que mantém.
Alimento.
Manutenção.
Dispêndio.
\section{Mantissa}
\begin{itemize}
\item {Grp. gram.:f.}
\end{itemize}
\begin{itemize}
\item {Proveniência:(Lat. \textunderscore mantissa\textunderscore )}
\end{itemize}
Parte decimal de um logarithmo.
\section{Manto}
\begin{itemize}
\item {Grp. gram.:m.}
\end{itemize}
\begin{itemize}
\item {Utilização:Fig.}
\end{itemize}
Vestidura larga e sem mangas, para abrigo da cabeça e do tronco.
Véu.
Antiga capa de cauda e roda.
Hábito de algumas freiras.
Aquillo que encobre alguma coisa.
Trevas: \textunderscore o manto da noite\textunderscore .
Parte superior do corpo de alguns animaes, quando ella pela côr se distingue do resto do corpo.
(Cita-se, como or., o lat. \textunderscore mantum\textunderscore , que não passa de mera latinização, e que não mostra a or. lat. do voc.)
\section{Mantó}
\begin{itemize}
\item {Grp. gram.:m.}
\end{itemize}
\begin{itemize}
\item {Utilização:Ant.}
\end{itemize}
\begin{itemize}
\item {Proveniência:(Fr. \textunderscore manteau\textunderscore )}
\end{itemize}
Vestimenta feminina, semelhante ao manto e que as mulheres usavam por cima de outro vestuário.
\section{Mantol}
\begin{itemize}
\item {Grp. gram.:m.}
\end{itemize}
(V.mantó)
\section{Mantuano}
\begin{itemize}
\item {Grp. gram.:adj.}
\end{itemize}
\begin{itemize}
\item {Grp. gram.:M.}
\end{itemize}
\begin{itemize}
\item {Utilização:Restrict.}
\end{itemize}
\begin{itemize}
\item {Proveniência:(De \textunderscore Mântua\textunderscore , n. p.)}
\end{itemize}
Relativo a Mântua.
Habitante do Mântua.
O poeta Vergílio.
\section{Mantulho}
\begin{itemize}
\item {Grp. gram.:m.}
\end{itemize}
\begin{itemize}
\item {Utilização:Prov.}
\end{itemize}
\begin{itemize}
\item {Utilização:alent.}
\end{itemize}
Laçada, que os segadores dão na paveia que conservam na mão esquerda, para que se não espalhe com os golpes immediatos.
\section{Manual}
\begin{itemize}
\item {Grp. gram.:adj.}
\end{itemize}
\begin{itemize}
\item {Proveniência:(Lat. \textunderscore manualis\textunderscore )}
\end{itemize}
Relativo á mão.
Que se faz com a mão: \textunderscore trabalho manual\textunderscore .
Relativo a trabalho de mãos: \textunderscore habilidade manual\textunderscore .
Que se manuseia facilmente: \textunderscore uma história manual\textunderscore .
Que se transporta com facilidade; portátil; leve.
\section{Manual}
\begin{itemize}
\item {Grp. gram.:m.}
\end{itemize}
\begin{itemize}
\item {Proveniência:(Lat. \textunderscore manuale\textunderscore )}
\end{itemize}
Pequeno livro.
Compêndio; summário.
Ritual.
\section{Manubalista}
\begin{itemize}
\item {Grp. gram.:f.}
\end{itemize}
\begin{itemize}
\item {Proveniência:(Do lat. \textunderscore manu\textunderscore  + \textunderscore balista\textunderscore )}
\end{itemize}
Máquina de guerra, que expellia dardos, talvez o mesmo que \textunderscore escorpião\textunderscore .
\section{Manubial}
\begin{itemize}
\item {Grp. gram.:adj.}
\end{itemize}
\begin{itemize}
\item {Proveniência:(Lat. \textunderscore manubialis\textunderscore )}
\end{itemize}
Relativo aos despojos do inimigo.
\section{Manúbrio}
\begin{itemize}
\item {Grp. gram.:m.}
\end{itemize}
\begin{itemize}
\item {Utilização:Ant.}
\end{itemize}
\begin{itemize}
\item {Utilização:Anat.}
\end{itemize}
\begin{itemize}
\item {Proveniência:(Lat. \textunderscore manubrium\textunderscore )}
\end{itemize}
Manivela.
Parte superior do esterno.
\section{Manucodiata}
\begin{itemize}
\item {Grp. gram.:f.}
\end{itemize}
\begin{itemize}
\item {Grp. gram.:M. pl.}
\end{itemize}
Constellação de onze estrêllas, no hemisphério do sul.
Família de aves, que tem por typo o manucódio.
\section{Manucódio}
\begin{itemize}
\item {Grp. gram.:m.}
\end{itemize}
\begin{itemize}
\item {Proveniência:(Do lat. \textunderscore manus\textunderscore  + \textunderscore cauda\textunderscore )}
\end{itemize}
Espécie de ave-do-paraíso.
\section{Manudução}
\begin{itemize}
\item {Grp. gram.:f.}
\end{itemize}
\begin{itemize}
\item {Proveniência:(Do lat. \textunderscore manus\textunderscore  + \textunderscore ductio\textunderscore )}
\end{itemize}
Acto de guiar pela mão.
\section{Manuducção}
\begin{itemize}
\item {Grp. gram.:f.}
\end{itemize}
\begin{itemize}
\item {Proveniência:(Do lat. \textunderscore manus\textunderscore  + \textunderscore ductio\textunderscore )}
\end{itemize}
Acto de guiar pela mão.
\section{Manuductor}
\begin{itemize}
\item {Grp. gram.:m.}
\end{itemize}
\begin{itemize}
\item {Utilização:Ant.}
\end{itemize}
\begin{itemize}
\item {Proveniência:(Do lat. \textunderscore manus\textunderscore  + \textunderscore ductor\textunderscore )}
\end{itemize}
Regente de côro.
\section{Manudutor}
\begin{itemize}
\item {Grp. gram.:m.}
\end{itemize}
\begin{itemize}
\item {Utilização:Ant.}
\end{itemize}
\begin{itemize}
\item {Proveniência:(Do lat. \textunderscore manus\textunderscore  + \textunderscore ductor\textunderscore )}
\end{itemize}
Regente de côro.
\section{Manuê}
\begin{itemize}
\item {Grp. gram.:m.}
\end{itemize}
Iguaria brasileira.
Bolo de milho.
\section{Manuel-cardoso}
\begin{itemize}
\item {Grp. gram.:m.}
\end{itemize}
Arbusto purgativo da ilha de San-Thomé.
\section{Manuel-de-abreu}
\begin{itemize}
\item {Grp. gram.:m.}
\end{itemize}
\begin{itemize}
\item {Utilização:Bras}
\end{itemize}
Espécie de abelha, da côr de canela.
\section{Manuel-de-breu}
\begin{itemize}
\item {Grp. gram.:m.}
\end{itemize}
\begin{itemize}
\item {Utilização:Bras}
\end{itemize}
O mesmo que \textunderscore manuel-de-abreu\textunderscore .
\section{Manuelino}
\begin{itemize}
\item {Grp. gram.:adj.}
\end{itemize}
\begin{itemize}
\item {Proveniência:(De \textunderscore Manuel\textunderscore , n. p.)}
\end{itemize}
Relativo ao rei D. Manuel I, de Portugal, ou ao seu tempo.
Diz-se especialmente de um estilo architectónico, privativo de Portugal, e que é uma transformação do gótico florido, sob a influência das nossas glórias marítimas.
\section{Manufacto}
\begin{itemize}
\item {Grp. gram.:m.}
\end{itemize}
\begin{itemize}
\item {Proveniência:(Lat. \textunderscore manufactus\textunderscore )}
\end{itemize}
O mesmo que \textunderscore artefacto\textunderscore .
\section{Manufactor}
\begin{itemize}
\item {Grp. gram.:m.}
\end{itemize}
\begin{itemize}
\item {Grp. gram.:Adj.}
\end{itemize}
\begin{itemize}
\item {Proveniência:(Do lat. \textunderscore manus\textunderscore  + \textunderscore factor\textunderscore )}
\end{itemize}
Aquelle que manufactura ou faz manufacturar.
Relativo a manufactura.
Manual.
\section{Manufactura}
\begin{itemize}
\item {Grp. gram.:f.}
\end{itemize}
\begin{itemize}
\item {Proveniência:(Do lat. \textunderscore manus\textunderscore  + \textunderscore factura\textunderscore )}
\end{itemize}
Trabalho manual; obra feita á mão.
Grande estabelecimento industrial.
Producto dêsse estabelecimento.
\section{Manufacturar}
\begin{itemize}
\item {Grp. gram.:v. t.}
\end{itemize}
\begin{itemize}
\item {Proveniência:(De \textunderscore manufactura\textunderscore )}
\end{itemize}
Produzir, com trabalho manual; fabricar.
\section{Manufactureiro}
\begin{itemize}
\item {Grp. gram.:adj.}
\end{itemize}
Relativo a manufactura.
\section{Manúlea}
\begin{itemize}
\item {Grp. gram.:f.}
\end{itemize}
\begin{itemize}
\item {Proveniência:(Lat. \textunderscore manulea\textunderscore )}
\end{itemize}
Parte da catapulta, que mantém a corda tensa; manga da catapulta.
\section{Manuma}
\begin{itemize}
\item {Grp. gram.:f.}
\end{itemize}
\begin{itemize}
\item {Utilização:T. de Angola}
\end{itemize}
Caixa, que contém vários objectos e um dente de cada um de vários jagas fallecidos, e que se entrega ao novo jaga, como sýmbolo do poder.
\section{Manumissão}
\begin{itemize}
\item {Grp. gram.:f.}
\end{itemize}
\begin{itemize}
\item {Proveniência:(Lat. \textunderscore manumissio\textunderscore )}
\end{itemize}
Acto ou effeito de manumittir; alforria. Cf. Herculano, \textunderscore Hist. de Port.\textunderscore , III, 307.
\section{Manumisso}
\begin{itemize}
\item {Grp. gram.:m.}
\end{itemize}
\begin{itemize}
\item {Proveniência:(Lat. \textunderscore manumissus\textunderscore )}
\end{itemize}
Aquelle que teve alforria; escravo fôrro. Cf. Herculano, \textunderscore Hist. de Port.\textunderscore , III, 259 e 299.
\section{Manumissor}
\begin{itemize}
\item {Grp. gram.:adj.}
\end{itemize}
\begin{itemize}
\item {Proveniência:(Lat. \textunderscore manumissor\textunderscore )}
\end{itemize}
Aquelle que dá alforria.
\section{Manumitente}
\begin{itemize}
\item {Grp. gram.:adj.}
\end{itemize}
\begin{itemize}
\item {Proveniência:(Do lat. \textunderscore manumittens\textunderscore )}
\end{itemize}
Que manumite ou dá alforria.
\section{Manumitir}
\begin{itemize}
\item {Grp. gram.:v. i.}
\end{itemize}
\begin{itemize}
\item {Proveniência:(Lat. \textunderscore manumittere\textunderscore )}
\end{itemize}
Dar alforria a.
\section{Manumittente}
\begin{itemize}
\item {Grp. gram.:adj.}
\end{itemize}
\begin{itemize}
\item {Proveniência:(Do lat. \textunderscore manumittens\textunderscore )}
\end{itemize}
Que manumitte ou dá alforria.
\section{Manumittir}
\begin{itemize}
\item {Grp. gram.:v. i.}
\end{itemize}
\begin{itemize}
\item {Proveniência:(Lat. \textunderscore manumittere\textunderscore )}
\end{itemize}
Dar alforria a.
\section{Manuschrísti}
\begin{itemize}
\item {Grp. gram.:m.}
\end{itemize}
\begin{itemize}
\item {Proveniência:(Do lat. \textunderscore manus\textunderscore  + \textunderscore Christus\textunderscore , n. p.)}
\end{itemize}
Antigo electuário de açúcar com aljôfre.
\section{Manuscrever}
\begin{itemize}
\item {Grp. gram.:v. t.}
\end{itemize}
\begin{itemize}
\item {Proveniência:(Do lat. \textunderscore manus\textunderscore  + \textunderscore scribere\textunderscore )}
\end{itemize}
Escrever á mão.
\section{Manuscrísti}
\begin{itemize}
\item {Grp. gram.:m.}
\end{itemize}
\begin{itemize}
\item {Proveniência:(Do lat. \textunderscore manus\textunderscore  + \textunderscore Christus\textunderscore , n. p.)}
\end{itemize}
Antigo electuário de açúcar com aljôfre.
\section{Manuscrito}
\begin{itemize}
\item {Grp. gram.:adj.}
\end{itemize}
\begin{itemize}
\item {Grp. gram.:M.}
\end{itemize}
\begin{itemize}
\item {Proveniência:(Do lat. \textunderscore manus\textunderscore  + \textunderscore scriptus\textunderscore )}
\end{itemize}
Que foi escrito á mão: \textunderscore obra manuscrita\textunderscore .
Aquillo que se escreveu á mão.
\section{Manusdei}
\begin{itemize}
\item {Grp. gram.:m.}
\end{itemize}
Antigo emplastro vulnerário.
(Da loc. lat. \textunderscore manus Dei\textunderscore , a mão de Deus)
\section{Manuseação}
\begin{itemize}
\item {Grp. gram.:f.}
\end{itemize}
Acto de manusear. Cf. Castilho, \textunderscore Fastos\textunderscore , I, 321.
\section{Manuseamento}
\begin{itemize}
\item {Grp. gram.:m.}
\end{itemize}
O mesmo que \textunderscore manuseação\textunderscore .
\section{Manusear}
\begin{itemize}
\item {Grp. gram.:v. t.}
\end{itemize}
\begin{itemize}
\item {Proveniência:(Do lat. \textunderscore manus\textunderscore )}
\end{itemize}
Mover com a mão; manejar; folhear: \textunderscore manusear um livro\textunderscore .
Amarrotar.
\section{Manuseio}
\begin{itemize}
\item {Grp. gram.:m.}
\end{itemize}
O mesmo que \textunderscore manuseação\textunderscore .
\section{Manusturbação}
\begin{itemize}
\item {Grp. gram.:f.}
\end{itemize}
(V.masturbação)
\section{Manutenção}
\begin{itemize}
\item {Grp. gram.:f.}
\end{itemize}
\begin{itemize}
\item {Utilização:Neol.}
\end{itemize}
\begin{itemize}
\item {Proveniência:(Do lat. \textunderscore manus\textunderscore  + \textunderscore tenere\textunderscore )}
\end{itemize}
Acto ou effeito de manter.
Gerência, administração: \textunderscore a manutenção de uma fábrica\textunderscore .
Estabelecimento, onde se fabríca pão para as tropas.
\section{Manutenência}
\begin{itemize}
\item {Grp. gram.:f.}
\end{itemize}
O mesmo que \textunderscore manutenção\textunderscore .
\section{Manutenir}
\begin{itemize}
\item {Grp. gram.:v.}
\end{itemize}
\begin{itemize}
\item {Utilização:bras}
\end{itemize}
\begin{itemize}
\item {Utilização:t. Jur.}
\end{itemize}
\begin{itemize}
\item {Proveniência:(Do lat. \textunderscore manus\textunderscore  + \textunderscore tenere\textunderscore )}
\end{itemize}
Conceder (mandado) de manutenção.
\section{Manutenível}
\begin{itemize}
\item {Grp. gram.:adj.}
\end{itemize}
\begin{itemize}
\item {Proveniência:(Do lat. \textunderscore manus\textunderscore  + \textunderscore tenere\textunderscore )}
\end{itemize}
Que se póde manter.
\section{Manutérgio}
\begin{itemize}
\item {Grp. gram.:m.}
\end{itemize}
\begin{itemize}
\item {Proveniência:(Lat. \textunderscore manutergium\textunderscore )}
\end{itemize}
Toalha, com que o sacerdote limpa as mãos, quando se reveste para celebrar a Missa.
\section{Manvio}
\begin{itemize}
\item {Grp. gram.:m.}
\end{itemize}
\begin{itemize}
\item {Utilização:Náut.}
\end{itemize}
Extremidade do cabo chamado chicote.
\section{Manx}
\begin{itemize}
\item {Grp. gram.:m.}
\end{itemize}
Dialecto céltico, o mesmo que \textunderscore manquês\textunderscore .
\section{Manzada}
\begin{itemize}
\item {Grp. gram.:f.}
\end{itemize}
(V.mãozada)
\section{Manzanilha}
\begin{itemize}
\item {Grp. gram.:f.}
\end{itemize}
Variedade de azeitona, o mesmo que \textunderscore mancenílha\textunderscore .
\section{Manzape}
\begin{itemize}
\item {Grp. gram.:m.}
\end{itemize}
\begin{itemize}
\item {Utilização:Bras. do N}
\end{itemize}
Bolo de milho ou de farinha de mandioca.
Bolo mal feito.
\section{Manzari}
\begin{itemize}
\item {Grp. gram.:m.}
\end{itemize}
\begin{itemize}
\item {Proveniência:(T. as.)}
\end{itemize}
Cacho de cocos.
\section{Manzeira}
\begin{itemize}
\item {Grp. gram.:f.}
\end{itemize}
\begin{itemize}
\item {Utilização:Prov.}
\end{itemize}
\begin{itemize}
\item {Utilização:alg.}
\end{itemize}
O maior dos dois paus que constituem o mastucador.
\section{Manzinha}
\begin{itemize}
\item {Grp. gram.:f.}
\end{itemize}
\begin{itemize}
\item {Utilização:Pop.}
\end{itemize}
Mão pequena.
(Por \textunderscore mãozinha\textunderscore )
\section{Manzorra}
\begin{itemize}
\item {Grp. gram.:f.}
\end{itemize}
Mão grande, manápula.
\section{Mão}
\begin{itemize}
\item {Grp. gram.:f.}
\end{itemize}
\begin{itemize}
\item {Utilização:Náut.}
\end{itemize}
\begin{itemize}
\item {Utilização:Náut.}
\end{itemize}
\begin{itemize}
\item {Utilização:Gír.}
\end{itemize}
\begin{itemize}
\item {Utilização:Prov.}
\end{itemize}
\begin{itemize}
\item {Utilização:trasm.}
\end{itemize}
\begin{itemize}
\item {Utilização:Bras. do N}
\end{itemize}
\begin{itemize}
\item {Utilização:Bras. do N}
\end{itemize}
\begin{itemize}
\item {Grp. gram.:Loc. adv.}
\end{itemize}
\begin{itemize}
\item {Utilização:Fam.}
\end{itemize}
\begin{itemize}
\item {Grp. gram.:Loc. adv.}
\end{itemize}
\begin{itemize}
\item {Grp. gram.:Loc. adv.}
\end{itemize}
\begin{itemize}
\item {Grp. gram.:Loc. adv.}
\end{itemize}
\begin{itemize}
\item {Grp. gram.:Loc. adv.}
\end{itemize}
\begin{itemize}
\item {Utilização:Prov.}
\end{itemize}
\begin{itemize}
\item {Utilização:trasm.}
\end{itemize}
\begin{itemize}
\item {Grp. gram.:Pl.}
\end{itemize}
\begin{itemize}
\item {Proveniência:(Do lat. \textunderscore manus\textunderscore )}
\end{itemize}
Parte do corpo humano, a qual, situada na extremidade do braço, serve especialmente para o tacto e para a apprehensão dos objectos.
Extremidade dos membros deanteiros dos quadrúpedes: \textunderscore a mão do cavallo\textunderscore .
Extremidade, depois de cortada de qualquer membro das reses.
Garra de algumas aves.
Posse, domínio: \textunderscore lançar mão de bens alheiroso\textunderscore .
Autoridade.
Influência.
No jôgo, o parceiro que primeiro joga.
Lanço completo de jôgo: \textunderscore ganhar duas mãos\textunderscore .
Gavinha.
Camada de tinta ou cal sôbre uma superficie; demão.
Carda miúda.
Lado direito do cocheiro que guia um carro.
Pequeno feixe ou qualquer objecto que se abrange com a mão: \textunderscore uma mão de nabos\textunderscore .
Modo de fazer as coisas, feição, maneira.
Peça, com que se tritura ou se pisa qualquer coisa no almofariz.
Qualidade ou poder de sêr o primeiro em dizer ou fazer qualquer coisa.
A haste mais curta de um madeiro angular.
Ligação da ponta ou chicote de um cabo com o mesmo cabo.
Parte de um instrumento ou utensílio, por onde elle se segura e se maneja ou se governa: \textunderscore a mão da enxada\textunderscore .
Antigo pêso indiano.
Medida de capacidade em Damão.
Chave.
\textunderscore Mão de papel\textunderscore , cinco cadernos.
\textunderscore Mão de ferro\textunderscore , potência, tyrannia e opressão.
\textunderscore Mão de braseira\textunderscore , pá de ferro, com que se mexe a cinza da braseira, para avivar as brasas.
\textunderscore Mão de Judas\textunderscore , apagador de velas, usado nas igrejas, na semana santa.
\textunderscore Mão de nabos\textunderscore , cinco cabeças de nabos.
\textunderscore Mão morta\textunderscore , mão, que um estranho póde mover á vontade.
\textunderscore Mão de rédea\textunderscore , govêrno do cavallo.
\textunderscore Bens de mão morta\textunderscore , os que pertencem a certas corporações, como confrarias, conventos, etc.
\textunderscore Mão de milho\textunderscore , conjunto de cinco espigas.
\textunderscore Mão de pilão\textunderscore , peça de madeira, com que se tritura qualquer coisa no pilão.
\textunderscore Feito por mão de mestre\textunderscore , bem feito, bem acabado.
\textunderscore Coisa em primeira mão\textunderscore , coisa adquirida directamente de quem a fabricou; coisa que outrem ainda não possuiu.
\textunderscore Coisa em segunda mão\textunderscore , coisa já usada, ou já utilizada por outro ou outros.
\textunderscore Letra de mão\textunderscore , letra manuscrita.
\textunderscore De mão commum\textunderscore , dizia-se o testamento, feito por consortes, um dos quaes ficaria herdeiro universal do que primeiro fallecesse.
\textunderscore Levar mão de\textunderscore , largar:«\textunderscore a desgraça não levava mão delle.\textunderscore »Camillo, \textunderscore Bibl. do Coração\textunderscore , 176.
\textunderscore Ir á mão de\textunderscore , reprehender.
Contrariar. Cf. Pant. de Aveiro, \textunderscore Itiner.\textunderscore , 34 e 62, (2.^a ed.).
\textunderscore Por baixo de mão\textunderscore , ás escondidas. Cf. Camillo, \textunderscore Vinte Hor. de Lit.\textunderscore , 228.
\textunderscore Assentar a mão\textunderscore , têr firmeza ou segurança no que faz.
Bater.
\textunderscore Dar a mão a\textunderscore , auxiliar, proteger.
\textunderscore Dar de mão\textunderscore , \textunderscore a erguer\textunderscore  ou \textunderscore levantar mão de\textunderscore , desviar de si, renunciar, dispensar.
\textunderscore Deitar a mão\textunderscore , apoderar-se; agarrar.
\textunderscore Pedir a mão de\textunderscore , pedir em casamento.
\textunderscore Á mão\textunderscore , perto, ao pé.
\textunderscore De mão em mão\textunderscore , das mãos de um para as mãos de outro, de pessôa para pessôa.
\textunderscore Têr mão\textunderscore , tomar cautela.
Parar.
Amparar alguma coisa.
\textunderscore Fazer mão baixa em\textunderscore , roubar, surripiar.
\textunderscore Numa volta de mão\textunderscore , rapidamente, num abrir e fechar de olhos.
\textunderscore Jogar de mão\textunderscore , sêr o primeiro a jogar.
\textunderscore Mão por baixo, mão por cima\textunderscore , cautelosamente.
\textunderscore Á mão de semear\textunderscore , ao alcance da mão; perto.
\textunderscore Com a mão do gato\textunderscore , surrateiramente.
\textunderscore De mão na ilharga\textunderscore , de modos grosseiros; com ares de regateira.
\textunderscore Vir á mão\textunderscore , vir ás bôas, chegar á razão.
\textunderscore Falar á mão\textunderscore , interromper alguém, objectar:«\textunderscore como ninguém lhe falasse á mão...\textunderscore »Camillo, \textunderscore Filha do Regicida\textunderscore .
\textunderscore Têr mão de\textunderscore , segurar, obstar:«\textunderscore não teve mão de si, que o não atalhasse\textunderscore ». \textunderscore Idem\textunderscore , \textunderscore ib.\textunderscore 
\textunderscore Mão de obra\textunderscore , trabalho manual.
O mesmo que [[bico de obra|bico:1]], pequeno concerto, algum serviço para artista ou operário.
\textunderscore Mãos de anéis\textunderscore , mãos mimosas, delicadas.
\textunderscore Mãos rotas\textunderscore , (m. e f.), pessôa perdulária, dissipadora.
\textunderscore Mãos atadas\textunderscore , (m. e f.), pessôa acanhada.
\textunderscore Mãos limpas\textunderscore , integridade, honradez.
\textunderscore Mãos largas\textunderscore ,(m. e f.), o mesmo que \textunderscore mãos rotas\textunderscore .
\textunderscore Mãos postas\textunderscore , mãos erguidas, juntando-se palma com palma, para rezar ou supplicar.
\textunderscore Com ambas as mãos\textunderscore , da melhor vontade (acceitar).
\textunderscore Estar com as mãos na massa\textunderscore , ou \textunderscore têr entre mãos\textunderscore , estar trabalhando ou estar tratando de.
\textunderscore Lavar as mãos disto\textunderscore  ou \textunderscore daquillo\textunderscore , protestar a sua innocência, não tomar a responsabilidade.
\textunderscore Meter\textunderscore  ou \textunderscore pôr mãos á obra\textunderscore , começá-la com empenho, com bôa vontade.
\textunderscore Meter os pés pelas mãos\textunderscore , confundir-se, não saber o que há de dizer, falar sem tom nem som, disparatar.
\textunderscore Vir ás mãos\textunderscore , lutar, brigar, combater.
\textunderscore Prestar juramento nas mãos de\textunderscore , jurar perante.
\textunderscore Limpo de mãos\textunderscore , honrado, íntegro.--Outras muitas loc. se nos deparam, em que o significado da palavra só se determina pelo sentido e contexto da phrase.
\section{Mão-cheia}
\begin{itemize}
\item {Grp. gram.:f.}
\end{itemize}
\begin{itemize}
\item {Utilização:Fig.}
\end{itemize}
Aquillo que se póde abranger com a mão: \textunderscore uma mão-cheia de sal\textunderscore .
Bôa qualidade, excellência: \textunderscore é pintor de mão-cheia\textunderscore .
\section{Mãochinha}
\begin{itemize}
\item {Grp. gram.:f.}
\end{itemize}
\begin{itemize}
\item {Utilização:Fam.}
\end{itemize}
Pequena porção; pouca coisa. Cf. Castilho, \textunderscore Colloq. Ald.\textunderscore , 177.
(Por \textunderscore mão-cheínha\textunderscore , dem. de \textunderscore mão-cheia\textunderscore )
\section{Mão-de-barca}
\begin{itemize}
\item {Grp. gram.:f.}
\end{itemize}
\begin{itemize}
\item {Utilização:Pesc.}
\end{itemize}
Cabo, que prende a rêde sardinheira ao barco.
\section{Má-olhas}
«\textunderscore ...cazaca, véstia, borjacão, (má-olhas, que se lhe assente em carne)...\textunderscore »Filinto, V, 147.
\section{Maometa}
\begin{itemize}
\item {Grp. gram.:adj.}
\end{itemize}
\begin{itemize}
\item {Utilização:Des.}
\end{itemize}
O mesmo que \textunderscore maometano\textunderscore . Cf. Filinto, \textunderscore Vida de D. Man.\textunderscore , I, 205.
\section{Maometanismo}
\begin{itemize}
\item {Grp. gram.:m.}
\end{itemize}
O mesmo que \textunderscore maometismo\textunderscore .
\section{Maometano}
\begin{itemize}
\item {Grp. gram.:adj.}
\end{itemize}
\begin{itemize}
\item {Grp. gram.:M.}
\end{itemize}
Relativo a Maomet ou á sua seita.
Sectário de Maomet.
\section{Maomético}
\begin{itemize}
\item {Grp. gram.:adj.}
\end{itemize}
(V.maometano)
\section{Maometismo}
\begin{itemize}
\item {Grp. gram.:m.}
\end{itemize}
Religião, fundada por Maomet.
\section{Maôna}
\begin{itemize}
\item {Grp. gram.:f.}
\end{itemize}
Espécie de embarcação antiga.
Antiga embarcação florentina, o mesmo que \textunderscore mahona\textunderscore . Cf. Castilho, \textunderscore Metam.\textunderscore , XXXIII.
\section{Mão-pendente}
\begin{itemize}
\item {Grp. gram.:f.}
\end{itemize}
Offerta para subôrno; peita.
\section{Mão-posta}
\begin{itemize}
\item {Grp. gram.:m.}
\end{itemize}
Prevenção.
Objecto reservado para occasião própria.
Combinação.
\section{Mão-quadra}
\begin{itemize}
\item {Grp. gram.:f.}
\end{itemize}
Mão aberta ou estendida. Cf. Herculano, \textunderscore Hist. de Port.\textunderscore , IV, 366.
\section{Maordomo}
\begin{itemize}
\item {Grp. gram.:m.}
\end{itemize}
\begin{itemize}
\item {Utilização:Ant.}
\end{itemize}
O mesmo que \textunderscore mordómo\textunderscore .
\section{Maóres}
\begin{itemize}
\item {Grp. gram.:m. pl.}
\end{itemize}
Indígenas da Nova-Zelândia.
\section{Mão-tenente}
\begin{itemize}
\item {Grp. gram.:f.}
\end{itemize}
Pouca distância; queima-roupa.
\section{Mão-tente}
\begin{itemize}
\item {Grp. gram.:f.}
\end{itemize}
\begin{itemize}
\item {Grp. gram.:Loc. adv.}
\end{itemize}
(Contr. de \textunderscore mão-tenente\textunderscore )
\textunderscore Á mão-tente\textunderscore , com mão firme, com firmeza.
\section{Mãozada}
\begin{itemize}
\item {Grp. gram.:f.}
\end{itemize}
\begin{itemize}
\item {Utilização:Pop.}
\end{itemize}
\begin{itemize}
\item {Utilização:Prov.}
\end{itemize}
Apêrto de mão, com fôrça.
Porção de coisas, que se abrangem na mão.
\section{Mãozinha}
\begin{itemize}
\item {Grp. gram.:f.}
\end{itemize}
\begin{itemize}
\item {Utilização:T. da Bairrada}
\end{itemize}
\begin{itemize}
\item {Utilização:Prov.}
\end{itemize}
Mão pequena.
Peça, que se engancha no garavato da charrua e serve para virar a leiva.
Travessa de madeira, que fixa á roda da nora a manjorra.
\section{Mãozudo}
\begin{itemize}
\item {Grp. gram.:adj.}
\end{itemize}
\begin{itemize}
\item {Utilização:Chul.}
\end{itemize}
\begin{itemize}
\item {Proveniência:(Do rad. de \textunderscore mão\textunderscore )}
\end{itemize}
Que tem mãos grandes e mal feitas.
\section{Mapa}
\begin{itemize}
\item {Grp. gram.:m.}
\end{itemize}
\begin{itemize}
\item {Grp. gram.:F.}
\end{itemize}
\begin{itemize}
\item {Proveniência:(Lat. \textunderscore mappa\textunderscore )}
\end{itemize}
Delineação de terras ou mares, ou de terras e mares, numa superfície plana.
Carta geográfica.
Relação; lista.
Gênero de plantas euforbiáceas; mapam.
\section{Mapa}
\begin{itemize}
\item {Grp. gram.:m.  e  f.}
\end{itemize}
\begin{itemize}
\item {Utilização:Prov.}
\end{itemize}
\begin{itemize}
\item {Utilização:trasm.}
\end{itemize}
Procedência.
Lugar, donde uma coisa é originária ou onde ella se dá em maior abundância: \textunderscore as margens do Doiro são o mapa do vinho do Porto\textunderscore .
\section{Mapam}
\begin{itemize}
\item {Grp. gram.:m.}
\end{itemize}
Planta euphorbiácea do Brasil.
\section{Mápão}
\begin{itemize}
\item {Grp. gram.:m.}
\end{itemize}
Planta euphorbiácea do Brasil.
\section{Mapará}
\begin{itemize}
\item {Grp. gram.:m.}
\end{itemize}
Peixe saboroso do Tocantins.
\section{Maparajuba}
\begin{itemize}
\item {Grp. gram.:f.}
\end{itemize}
Árvore do Amazonas.
\section{Mapareíba}
\begin{itemize}
\item {Grp. gram.:f.}
\end{itemize}
Planta, variedade de manga vermelha.
\section{Maparis}
\begin{itemize}
\item {Grp. gram.:m. pl.}
\end{itemize}
Índios selvagens das margens do Japurá, no Brasil.
\section{Mapiar}
\begin{itemize}
\item {Grp. gram.:v. i.}
\end{itemize}
\begin{itemize}
\item {Utilização:Bras. de Mato-Grosso}
\end{itemize}
O mesmo que \textunderscore falar\textunderscore .
\section{Mapichi}
\begin{itemize}
\item {Grp. gram.:m.}
\end{itemize}
Planta myrtácea do Brasil.
\section{Mapieninga}
\begin{itemize}
\item {Grp. gram.:f.}
\end{itemize}
Árvore silvestre do Brasil.
\section{Mapila}
\begin{itemize}
\item {Grp. gram.:f.}
\end{itemize}
Espécie de milho miúdo, em Lourenço-Marques.
\section{Mapinguim}
\begin{itemize}
\item {Grp. gram.:m.}
\end{itemize}
\begin{itemize}
\item {Utilização:Bras. do Ceará}
\end{itemize}
Tabaco, importado das províncias do Sul.
\section{Mapinguinho}
\begin{itemize}
\item {Grp. gram.:m.}
\end{itemize}
O mesmo que \textunderscore mapinguim\textunderscore .
\section{Mapira}
\begin{itemize}
\item {Grp. gram.:f.}
\end{itemize}
Espécie de sorgo da Zambézia.--É base da alimentação dos Cafres na Gorongoza e serve-lhes para fabricar uma espécie de cerveja.
\section{Mapirunga}
\begin{itemize}
\item {Grp. gram.:f.}
\end{itemize}
Arbusto myrtáceo do Brasil.
Fruto dêsse arbusto.
\section{Mapixi}
\begin{itemize}
\item {Grp. gram.:m.}
\end{itemize}
Planta mirtácea do Brasil.
\section{Mapoão}
\begin{itemize}
\item {Grp. gram.:m.}
\end{itemize}
\begin{itemize}
\item {Utilização:Bras}
\end{itemize}
Planta venenosa, com cujo suco os Índios ervam as frechas.
\section{Mapole}
\begin{itemize}
\item {Grp. gram.:m.}
\end{itemize}
Fruto de Bié, do tamanho de uma laranja, duríssimo, e cuja parte comestível é um líquido coagulado que tem no interior.
\section{Maponga}
\begin{itemize}
\item {Grp. gram.:f.}
\end{itemize}
\begin{itemize}
\item {Utilização:Bras}
\end{itemize}
Processo de pescar, o mesmo que \textunderscore gaponga\textunderscore .
\section{Mappa}
\begin{itemize}
\item {Grp. gram.:m.}
\end{itemize}
\begin{itemize}
\item {Grp. gram.:F.}
\end{itemize}
\begin{itemize}
\item {Proveniência:(Lat. \textunderscore mappa\textunderscore )}
\end{itemize}
Delineação de terras ou mares, ou de terras e mares, numa superfície plana.
Carta geográphica.
Relação; lista.
Gênero de plantas euphorbiáceas; mappam.
\section{Mappam}
\begin{itemize}
\item {Grp. gram.:m.}
\end{itemize}
Planta euphorbiácea do Brasil.
\section{Mappa-múndi}
\begin{itemize}
\item {Grp. gram.:m.}
\end{itemize}
\begin{itemize}
\item {Proveniência:(Do lat. \textunderscore mappa\textunderscore  + \textunderscore mundus\textunderscore )}
\end{itemize}
Mappa, que representa toda a superfície da Terra.
\section{Mapuás}
\begin{itemize}
\item {Grp. gram.:m. pl.}
\end{itemize}
\begin{itemize}
\item {Utilização:Bras}
\end{itemize}
Aborígenes, que habitaram no Pará.
\section{Mapuca}
\begin{itemize}
\item {Grp. gram.:f.}
\end{itemize}
\begin{itemize}
\item {Utilização:T. de Angola}
\end{itemize}
Espécie de abelha.
\section{Mapunda}
\begin{itemize}
\item {Grp. gram.:f.}
\end{itemize}
Árvore angolense.
\section{Mapuriti}
\begin{itemize}
\item {Grp. gram.:m.}
\end{itemize}
Pequeno quadrúpede de Guiana.
\section{Mapurucuni}
\begin{itemize}
\item {Grp. gram.:m.}
\end{itemize}
Planta medicinal da Guiana inglesa.
\section{Mapurunga}
\begin{itemize}
\item {Grp. gram.:f.}
\end{itemize}
\begin{itemize}
\item {Utilização:Bras}
\end{itemize}
Árvore dos sertões.
\section{Maquarém}
\begin{itemize}
\item {Grp. gram.:m.}
\end{itemize}
\begin{itemize}
\item {Utilização:Ant.}
\end{itemize}
Espécie de ópio.
\section{Maque}
\begin{itemize}
\item {Grp. gram.:m.}
\end{itemize}
Gênero de quadrúmanos nocturnos; o mesmo que \textunderscore lêmur\textunderscore .
\section{Maqueira}
\begin{itemize}
\item {Grp. gram.:f.}
\end{itemize}
\begin{itemize}
\item {Utilização:Bras}
\end{itemize}
Rêde de fibras de tucum, para dormir.
\section{Maqueiro}
\begin{itemize}
\item {Grp. gram.:m.}
\end{itemize}
Cada um dos que conduzem uma maca.
\section{Maqueje}
\begin{itemize}
\item {Grp. gram.:m.}
\end{itemize}
Insecto africano.
\section{Maquera}
\begin{itemize}
\item {Grp. gram.:f.}
\end{itemize}
\begin{itemize}
\item {Proveniência:(Lat. \textunderscore machaera\textunderscore )}
\end{itemize}
Antiga espada, larga e curta, espécie de sabre.
\section{Maquério}
\begin{itemize}
\item {Grp. gram.:m.}
\end{itemize}
\begin{itemize}
\item {Proveniência:(Gr. \textunderscore makhairion\textunderscore )}
\end{itemize}
Pequeno sabre, antigamente usado pelos Romanos.
\section{Maqueróforo}
\begin{itemize}
\item {Grp. gram.:m.}
\end{itemize}
\begin{itemize}
\item {Proveniência:(Gr. \textunderscore machairophoros\textunderscore )}
\end{itemize}
Soldado, armado de maquera.
\section{Maqueta}
\begin{itemize}
\item {fónica:quê}
\end{itemize}
\begin{itemize}
\item {Grp. gram.:f.}
\end{itemize}
\begin{itemize}
\item {Proveniência:(It. \textunderscore macchieta\textunderscore , fr. \textunderscore maquette\textunderscore , propriamente pequena mancha, do lat. \textunderscore macula\textunderscore )}
\end{itemize}
Esbôço de uma estátua ou de outra obra de esculptura, modelado em barro
\textunderscore ou\textunderscore  cera.
\section{Maquia}
\begin{itemize}
\item {Grp. gram.:f.}
\end{itemize}
\begin{itemize}
\item {Utilização:Fig.}
\end{itemize}
\begin{itemize}
\item {Proveniência:(Do ár. \textunderscore maquila\textunderscore )}
\end{itemize}
Porção de cereaes ou de azeitonas, de azeite ou de farinha, que os moleíros e lagareiros recebem, em paga de moêrem porção maior.
Antiga medida de cereaes, equivalente a dois celamins.
Dinheiro; lucro; gorgeta.
\section{Maquiador}
\begin{itemize}
\item {Grp. gram.:m.  e  adj.}
\end{itemize}
O que maquia; o que recebe maquias.
\section{Maquiadura}
\begin{itemize}
\item {Grp. gram.:f.}
\end{itemize}
Acto de maquiar.
\section{Maquiar}
\begin{itemize}
\item {Grp. gram.:v. t.}
\end{itemize}
\begin{itemize}
\item {Utilização:Fig.}
\end{itemize}
\begin{itemize}
\item {Grp. gram.:V. i.}
\end{itemize}
Medir com maquia.
Desfalcar, subtrahir parte de.
Cobrar a maquia nos moínhos ou lagares.
\section{Maquiavelicamente}
\begin{itemize}
\item {Grp. gram.:adv.}
\end{itemize}
De modo \textunderscore maquiavélico\textunderscore .
\section{Maquiavelice}
\begin{itemize}
\item {Grp. gram.:f.}
\end{itemize}
Acto ou dito maquiavélico.
Manha, ronha.
\section{Maquiavélico}
\begin{itemize}
\item {Grp. gram.:adj.}
\end{itemize}
\begin{itemize}
\item {Utilização:Fig.}
\end{itemize}
Relativo ou semelhante ao maquiavelismo.
Astuto; ardiloso; velhaco.
\section{Maquiavelismo}
\begin{itemize}
\item {Grp. gram.:m.}
\end{itemize}
\begin{itemize}
\item {Utilização:Fig.}
\end{itemize}
\begin{itemize}
\item {Proveniência:(De \textunderscore Maquiavel\textunderscore , n. p.)}
\end{itemize}
Sistema político, preconizado pelo florentino Maquiavel, (Machiavello), e que tem por base a astúcia.
Velhacaria; perfídia.
Alguns mandam lêr \textunderscore makiavelismo\textunderscore , atendendo-se á pronúncia italiana; vulgarmente porém, diz-se \textunderscore maxiavelismo\textunderscore , aplicando-se aos der. a mesma pronúncia.
\section{Maquiavelista}
\begin{itemize}
\item {Grp. gram.:adj.}
\end{itemize}
\begin{itemize}
\item {Grp. gram.:M.  e  f.}
\end{itemize}
Maquiavélico.
Pessôa, que segue a doutrina de Maquiavel.
(Cp. \textunderscore maquiavelismo\textunderscore )
\section{Maquiavelizar}
\begin{itemize}
\item {Grp. gram.:v. i.}
\end{itemize}
\begin{itemize}
\item {Proveniência:(De \textunderscore Maquiavel\textunderscore , n. p.)}
\end{itemize}
Proceder maquiavelicamente.
\section{Maquidum}
\begin{itemize}
\item {Grp. gram.:m.}
\end{itemize}
\begin{itemize}
\item {Utilização:Bras}
\end{itemize}
Pequena cadeira.
\section{Maquidura}
\begin{itemize}
\item {Grp. gram.:f.}
\end{itemize}
\begin{itemize}
\item {Utilização:Bras}
\end{itemize}
O mesmo que \textunderscore maquidum\textunderscore .
\section{Maquieiro}
\begin{itemize}
\item {Grp. gram.:m.}
\end{itemize}
\begin{itemize}
\item {Utilização:Prov.}
\end{itemize}
\begin{itemize}
\item {Utilização:trasm.}
\end{itemize}
Medida de capacidade, com que o moleiro tira a sua maquia.
O mesmo que \textunderscore moleiro\textunderscore . Cf. \textunderscore Fenix Renasc.\textunderscore , IV, 6.
Saco pequeno.
\section{Maquieta}
\begin{itemize}
\item {fónica:ê}
\end{itemize}
\begin{itemize}
\item {Grp. gram.:f.}
\end{itemize}
O mesmo que \textunderscore maqueta\textunderscore .
\section{Maquilão}
\begin{itemize}
\item {Grp. gram.:m.}
\end{itemize}
\begin{itemize}
\item {Utilização:Prov.}
\end{itemize}
\begin{itemize}
\item {Proveniência:(Do rad. de \textunderscore maquia\textunderscore  = ár. \textunderscore maquila\textunderscore )}
\end{itemize}
Aquelle que leva ao moinho os cereaes, e a respectiva farinha a casa dos donos.
\section{Maquilhar-se}
\begin{itemize}
\item {Grp. gram.:v. p.}
\end{itemize}
\begin{itemize}
\item {Utilização:Gal}
\end{itemize}
\begin{itemize}
\item {Proveniência:(Do fr. \textunderscore maquiller\textunderscore )}
\end{itemize}
Pintar o rosto; usar cosméticos:«\textunderscore é o homem que se encalamistra, é a dama que se maquilha.\textunderscore »C. Neto, \textunderscore A Bico de Penna\textunderscore , 59.
\section{Maquim}
\begin{itemize}
\item {Grp. gram.:m.}
\end{itemize}
O mesmo que \textunderscore macicote\textunderscore .
\section{Máquina}
\begin{itemize}
\item {Grp. gram.:f.}
\end{itemize}
\begin{itemize}
\item {Utilização:Fig.}
\end{itemize}
\begin{itemize}
\item {Utilização:Prov.}
\end{itemize}
\begin{itemize}
\item {Utilização:trasm.}
\end{itemize}
\begin{itemize}
\item {Proveniência:(Lat. \textunderscore machina\textunderscore )}
\end{itemize}
Apparelho ou instrumento, próprio para communicar movimento ou para aproveitar e pôr em acção um agente natural.
Qualquer instrumento ou utensílio.
Construcção sumptuosa, reveladora de gênio.
Pessôa, que não tem ideias próprias ou que procede automaticamente.
Grande porção: \textunderscore uma máquina de coisas\textunderscore .
\section{Maquinação}
\begin{itemize}
\item {Grp. gram.:f.}
\end{itemize}
\begin{itemize}
\item {Proveniência:(Lat. \textunderscore machinatio\textunderscore )}
\end{itemize}
Acto ou effeito de maquinar.
Trama, conluio.
\section{Maquinador}
\begin{itemize}
\item {Grp. gram.:m.  e  adj.}
\end{itemize}
\begin{itemize}
\item {Proveniência:(Lat. \textunderscore machinator\textunderscore )}
\end{itemize}
O que maquina.
\section{Maquinal}
\begin{itemize}
\item {Grp. gram.:adj.}
\end{itemize}
\begin{itemize}
\item {Utilização:Fig.}
\end{itemize}
\begin{itemize}
\item {Proveniência:(Lat. \textunderscore machinalis\textunderscore )}
\end{itemize}
Relativo a máquinas.
Inconsciente.
Automático.
\section{Maquinalmente}
\begin{itemize}
\item {Grp. gram.:adv.}
\end{itemize}
De modo maquinal.
\section{Maquinar}
\begin{itemize}
\item {Grp. gram.:v. t.}
\end{itemize}
\begin{itemize}
\item {Proveniência:(Lat. \textunderscore maquinari\textunderscore )}
\end{itemize}
Tramar.
Planear (um ardil).
Intentar; engenhar.
\section{Maquinaria}
\begin{itemize}
\item {Grp. gram.:f.}
\end{itemize}
Conjunto de máquinas.
Arte de maquinista.
\section{Maquinde}
\begin{itemize}
\item {Grp. gram.:m.}
\end{itemize}
Insecto africano.
\section{Maquineta}
\begin{itemize}
\item {fónica:nê}
\end{itemize}
\begin{itemize}
\item {Grp. gram.:f.}
\end{itemize}
\begin{itemize}
\item {Utilização:Gír.}
\end{itemize}
\begin{itemize}
\item {Proveniência:(De \textunderscore máquina\textunderscore )}
\end{itemize}
Santuário, sacrário ou pequeno throno, em que se expõe o Sacramento sobre o altar.
Pequeno oratório ou armário envidraçado.
Redoma enfeitada, que contém uma imagem devota.
Cabeça.
\section{Maquinismo}
\begin{itemize}
\item {Grp. gram.:m.}
\end{itemize}
\begin{itemize}
\item {Proveniência:(De \textunderscore máquina\textunderscore )}
\end{itemize}
Arte de maquinista.
Conjunto de máquinas.
Apparelho, para fazer executar movimentos.
Conjunto das peças de um apparelho.
Apparelho, instrumento.
Conjunto das decorações theatraes; scenário.
\section{Maquinista}
\begin{itemize}
\item {Grp. gram.:m.}
\end{itemize}
Aquelle que inventa máquinas.
Aquelle que as constrói ou dirige.
Aquelle que nos theatros é encarregado do scenário ou das decorações.
\section{Maquino}
\begin{itemize}
\item {Grp. gram.:m.}
\end{itemize}
\begin{itemize}
\item {Utilização:Gír.}
\end{itemize}
Ladrão de estrada.
\section{Maquira}
\begin{itemize}
\item {Grp. gram.:f.}
\end{itemize}
Filamento de tucum, com que os Índios do Peru fazem rêdes para dormir.
\section{Maquixos}
\begin{itemize}
\item {Grp. gram.:m. pl.}
\end{itemize}
Povo sertanejo de Angola.
\section{Mar}
\begin{itemize}
\item {Grp. gram.:m.}
\end{itemize}
\begin{itemize}
\item {Utilização:Fig.}
\end{itemize}
\begin{itemize}
\item {Utilização:Ant.}
\end{itemize}
\begin{itemize}
\item {Proveniência:(Lat. \textunderscore mare\textunderscore )}
\end{itemize}
Grande massa de água salgada, que cobre a maior parte da superfície da terra.
Cada uma das grandes partes, em que se divide essa massa.
Grande quantidade.
Abismo; immensidade.
Grandes difficuldades ou tormentas moraes.
Onda grande, vagalhão. Cf. \textunderscore Hist. Trág. Marít.\textunderscore , 50.
\section{Mar}
\begin{itemize}
\item {Grp. gram.:m.}
\end{itemize}
Título, que os Maronitas dão aos seus Bispos. Cf. A. Gouveia, \textunderscore Jornada do Arceb.\textunderscore , I, c. III, 8.
(Ár. \textunderscore mar\textunderscore )
\section{Mará}
\begin{itemize}
\item {Grp. gram.:m.}
\end{itemize}
Mammífero americano, da fam. dos roedores.
\section{Mará}
\begin{itemize}
\item {Grp. gram.:m.}
\end{itemize}
\begin{itemize}
\item {Utilização:Bras}
\end{itemize}
Vergôntea, ramo delgado de árvore.
Vara, para amarrar embarcação, ou para a impellir, ou para lhe retesar a vela.
\section{Marabá}
\begin{itemize}
\item {Grp. gram.:m.  e  f.}
\end{itemize}
\begin{itemize}
\item {Utilização:Bras}
\end{itemize}
Pessôa mestiça de índio e branco.
\section{Marabitanas}
\begin{itemize}
\item {Grp. gram.:m. pl.}
\end{itemize}
Índios do Brasil, nas margens do rio Negro.
O mesmo que \textunderscore marapitanas\textunderscore ?
\section{Marabitino}
\begin{itemize}
\item {Grp. gram.:m.}
\end{itemize}
O mesmo que \textunderscore morabitino\textunderscore .
\section{Marabito}
\begin{itemize}
\item {Grp. gram.:m.}
\end{itemize}
\begin{itemize}
\item {Proveniência:(Do ár. \textunderscore marabit\textunderscore )}
\end{itemize}
(Fórma portuguesa, em vez da afrancesada \textunderscore marabuto\textunderscore )
\section{Marabota}
\begin{itemize}
\item {Grp. gram.:f.}
\end{itemize}
O mesmo que \textunderscore margota\textunderscore .
\section{Marabu}
\begin{itemize}
\item {Grp. gram.:m.}
\end{itemize}
\begin{itemize}
\item {Grp. gram.:Pl.}
\end{itemize}
Espécie de cegonha, (\textunderscore ciconia marabu\textunderscore ).
Ave pernalta de Angola.
Enfeites de pennas do marabu.
Homens, que se dedícam á prática e ensino da vida religiosa, entre os Muçulmanos.
(Cp. \textunderscore marabuto\textunderscore )
\section{Marabumbo}
\begin{itemize}
\item {Grp. gram.:m.}
\end{itemize}
Peixe de Portugal.
\section{Marabuto}
\begin{itemize}
\item {Grp. gram.:m.}
\end{itemize}
\begin{itemize}
\item {Utilização:Gír.}
\end{itemize}
\begin{itemize}
\item {Grp. gram.:Pl.}
\end{itemize}
\begin{itemize}
\item {Proveniência:(Fr. \textunderscore marabout\textunderscore . A fórma portuguesa é \textunderscore marabito\textunderscore . V. \textunderscore marabito\textunderscore )}
\end{itemize}
O mesmo que \textunderscore marabu\textunderscore .
Religioso muçulmano, muito venerado entre os Árabes.
Templo rural, em que êsse religioso faz serviço.
Marinheiro.
O mesmo que \textunderscore almoravides\textunderscore .
\section{Maracá}
\begin{itemize}
\item {Grp. gram.:m.  e  f.}
\end{itemize}
\begin{itemize}
\item {Utilização:Bras}
\end{itemize}
Bálsamo do Peru.
Cabaça sêca e interiormente limpa, que os indígenas do Maranhão agitam nas festas e na guerra, metendo-lhes pedras ou frutos.
Chocalho, com que brincam as crianças.
\section{Maracachão}
\begin{itemize}
\item {Grp. gram.:m.}
\end{itemize}
Apreciada ave dos sertões de Angola.
\section{Maracaiá}
\begin{itemize}
\item {Grp. gram.:m.}
\end{itemize}
O mesmo que \textunderscore maracajá\textunderscore .
\section{Maracajá}
\begin{itemize}
\item {Grp. gram.:m.}
\end{itemize}
Espécie de gato bravo dos sertões brasileiros.
\section{Maracanan}
\begin{itemize}
\item {Grp. gram.:m.}
\end{itemize}
Ave trepadora do Brasil, (\textunderscore conurus\textunderscore ).
\section{Maracanhá}
\begin{itemize}
\item {Grp. gram.:f.}
\end{itemize}
Ave trepadora do Brasil, (\textunderscore conurus\textunderscore ).
\section{Maração}
\begin{itemize}
\item {Grp. gram.:f.}
\end{itemize}
\begin{itemize}
\item {Utilização:Gír.}
\end{itemize}
\begin{itemize}
\item {Proveniência:(De \textunderscore marar\textunderscore )}
\end{itemize}
Morte, assassínio.
\section{Maracatim}
\begin{itemize}
\item {Grp. gram.:m.}
\end{itemize}
\begin{itemize}
\item {Proveniência:(Do guar. \textunderscore maraká\textunderscore )}
\end{itemize}
Pequena embarcação do Pará.
\section{Maracatu}
\begin{itemize}
\item {Grp. gram.:m.}
\end{itemize}
\begin{itemize}
\item {Utilização:Bras}
\end{itemize}
Dança de negros boçaes.
\section{Maracha}
\begin{itemize}
\item {Grp. gram.:f.}
\end{itemize}
\begin{itemize}
\item {Utilização:Prov.}
\end{itemize}
\begin{itemize}
\item {Utilização:Prov.}
\end{itemize}
Pequeno marachão.
Pequeno muro, que divide as peças da salina.
Pequeno muro, que separa os canteiros nas hortas.
Vallado, que fórma rêgo para a água.
\section{Marachão}
\begin{itemize}
\item {Grp. gram.:m.}
\end{itemize}
\begin{itemize}
\item {Proveniência:(De \textunderscore maracha\textunderscore )}
\end{itemize}
Dique; mota.
Recife, restinga.
\section{Marachatim}
\begin{itemize}
\item {Grp. gram.:m.}
\end{itemize}
O mesmo que \textunderscore maracatim\textunderscore .
\section{Marachona}
\begin{itemize}
\item {Grp. gram.:f.}
\end{itemize}
Peixe da Póvoa de Varzim, (\textunderscore blenius gattorugine\textunderscore , Brunn.).
\section{Maracotão}
\begin{itemize}
\item {Grp. gram.:m.}
\end{itemize}
\begin{itemize}
\item {Proveniência:(Do cast. \textunderscore melocoton\textunderscore )}
\end{itemize}
Espécie de pêssego, fruto do maracoteiro.
\section{Maracoteiro}
\begin{itemize}
\item {Grp. gram.:m.}
\end{itemize}
\begin{itemize}
\item {Proveniência:(De \textunderscore maracotão\textunderscore )}
\end{itemize}
Pessegueiro durázio, enxertado em marmeleiro.
\section{Maracoto}
\begin{itemize}
\item {fónica:cô}
\end{itemize}
\begin{itemize}
\item {Grp. gram.:m.}
\end{itemize}
\begin{itemize}
\item {Utilização:Açor}
\end{itemize}
O mesmo que \textunderscore maragota\textunderscore .
\section{Maracu}
\begin{itemize}
\item {Grp. gram.:m.}
\end{itemize}
Formosa árvore do Alto Amazonas.
\section{Maracujá}
\begin{itemize}
\item {Grp. gram.:m.}
\end{itemize}
Gênero de plantas passiflóreas do Brasil, (\textunderscore passifloru edulis\textunderscore , Sims.).
Fruta do maracujazeiro.
O mesmo que \textunderscore martýrio\textunderscore , planta.
(Do tupi)
\section{Maracujazeiro}
\begin{itemize}
\item {Grp. gram.:m.}
\end{itemize}
\begin{itemize}
\item {Utilização:Bras}
\end{itemize}
\begin{itemize}
\item {Proveniência:(De \textunderscore maracujá\textunderscore )}
\end{itemize}
Planta passiflórea.
\section{Maracuta}
\begin{itemize}
\item {Grp. gram.:f.}
\end{itemize}
Moéda de Angola, do valor de 10 reis.
\section{Marafalhas}
\begin{itemize}
\item {Grp. gram.:f. pl.}
\end{itemize}
\begin{itemize}
\item {Utilização:Prov.}
\end{itemize}
\begin{itemize}
\item {Utilização:alent.}
\end{itemize}
O mesmo que \textunderscore maravalhas\textunderscore .
\section{Marafantona}
\begin{itemize}
\item {Grp. gram.:f.}
\end{itemize}
\begin{itemize}
\item {Utilização:T. de Turquel}
\end{itemize}
O mesmo que \textunderscore marafona\textunderscore .
\section{Marafona}
\begin{itemize}
\item {Grp. gram.:f.}
\end{itemize}
\begin{itemize}
\item {Utilização:Pop.}
\end{itemize}
\begin{itemize}
\item {Proveniência:(Do ár. \textunderscore mara-haina\textunderscore )}
\end{itemize}
Boneca de trapos.
Mulher reles, prostituta.
\section{Marafonear}
\begin{itemize}
\item {Grp. gram.:v. i.}
\end{itemize}
Tratar com marafonas.
\section{Marafoneiro}
\begin{itemize}
\item {Grp. gram.:m.}
\end{itemize}
Aquelle que trata ou convive com marafonas.
\section{Maragota}
\begin{itemize}
\item {Grp. gram.:f.}
\end{itemize}
Peixe das costas de Portugal e dos Açores, (\textunderscore labrus bergylta\textunderscore , Ascan.).
\section{Maragote}
\begin{itemize}
\item {Grp. gram.:m.}
\end{itemize}
\begin{itemize}
\item {Utilização:T. de Aveiro}
\end{itemize}
O mesmo que \textunderscore maragota\textunderscore .
\section{Maraguto}
\begin{itemize}
\item {Grp. gram.:adj.}
\end{itemize}
\begin{itemize}
\item {Utilização:Prov.}
\end{itemize}
\begin{itemize}
\item {Utilização:minh.}
\end{itemize}
Bravio, cheio de silvas.
\section{Maraiaíba}
\begin{itemize}
\item {Grp. gram.:f.}
\end{itemize}
Espécie de palmeira do Brasil.
\section{Má-raios}
\begin{itemize}
\item {Grp. gram.:m. pl.}
\end{itemize}
\begin{itemize}
\item {Utilização:Chul.}
\end{itemize}
Muitos raios, (us. em juras ou pragas):«\textunderscore má-raios te partam.\textunderscore »Camillo, \textunderscore Myst. de Lisb.\textunderscore , I, 35.
(Talvez por \textunderscore maus-raios\textunderscore )
\section{Marajá}
\begin{itemize}
\item {Grp. gram.:m.}
\end{itemize}
O mesmo que \textunderscore tucuma\textunderscore .
\section{Marajatina}
\begin{itemize}
\item {Grp. gram.:f.}
\end{itemize}
Bosque de marajás.
\section{Marambá}
\begin{itemize}
\item {Grp. gram.:m.}
\end{itemize}
Árvore do Pará.
\section{Maranata}
\begin{itemize}
\item {Grp. gram.:adj.}
\end{itemize}
Que envolve maldição:«\textunderscore esse tem de ha muito recebido o seu quinhão de anáthemas maranatas\textunderscore ». Herculano, \textunderscore Quest. Públ.\textunderscore , I, 288.
(E referência ao seguinte passo da epístola de San-Paulo aos Corínthios, XVI, 22:«\textunderscore se alguém não ama a Christo, seja anáthema. Maran-Atha\textunderscore »)
\section{Maranduva}
\begin{itemize}
\item {Grp. gram.:f.}
\end{itemize}
\begin{itemize}
\item {Utilização:Bras}
\end{itemize}
Mentira; fábula; conto.
(Corr. do tupi \textunderscore moranduba\textunderscore )
\section{Maranga}
\begin{itemize}
\item {Grp. gram.:f.}
\end{itemize}
Árvore medicinal da Índia.
\section{Marangaba}
\begin{itemize}
\item {Grp. gram.:f.}
\end{itemize}
Planta myrtácea do Brasil.
\section{Marangolar}
\begin{itemize}
\item {Grp. gram.:v. i.}
\end{itemize}
\begin{itemize}
\item {Utilização:Prov.}
\end{itemize}
\begin{itemize}
\item {Utilização:trasm.}
\end{itemize}
Andar na marangolice.
\section{Marangoleiro}
\begin{itemize}
\item {Grp. gram.:adj.}
\end{itemize}
\begin{itemize}
\item {Utilização:Prov.}
\end{itemize}
\begin{itemize}
\item {Utilização:trasm.}
\end{itemize}
Calaceiro; mandrião.
\section{Marangolice}
\begin{itemize}
\item {Grp. gram.:f.}
\end{itemize}
\begin{itemize}
\item {Utilização:Prov.}
\end{itemize}
\begin{itemize}
\item {Utilização:trasm.}
\end{itemize}
Vida de marangoleiro.
\section{Maranha}
\begin{itemize}
\item {Grp. gram.:f.}
\end{itemize}
\begin{itemize}
\item {Utilização:Fig.}
\end{itemize}
Fios ou fibras enredadas.
Teia de lan, antes de apisoada.
Enrêdo.
Negócio intricado.
Astúcia.
(Cast. \textunderscore marana\textunderscore )
\section{Maranhão}
\begin{itemize}
\item {Grp. gram.:m.}
\end{itemize}
\begin{itemize}
\item {Proveniência:(De \textunderscore maranha\textunderscore )}
\end{itemize}
Mentira; peta engenhosa.
\section{Maranhar}
\begin{itemize}
\item {Grp. gram.:v. t.}
\end{itemize}
O mesmo que \textunderscore emmaranhar\textunderscore . Cf. Filinto, XIV, 238.
\section{Maranhense}
\begin{itemize}
\item {Grp. gram.:adj.}
\end{itemize}
\begin{itemize}
\item {Grp. gram.:M.  e  f.}
\end{itemize}
\begin{itemize}
\item {Proveniência:(De \textunderscore Maranhão\textunderscore )}
\end{itemize}
Relativo ao Estado do Maranhão.
Habitante dêsse Estado.
\section{Maranho}
\begin{itemize}
\item {Grp. gram.:m.}
\end{itemize}
\begin{itemize}
\item {Proveniência:(Do rad. de \textunderscore maranha\textunderscore )}
\end{itemize}
Mólho de tripas.
Iguaria, feita de miudezas de carneiro, com arroz, bocados de gallinha, etc.
\section{Maranhões}
\begin{itemize}
\item {Grp. gram.:m. pl.}
\end{itemize}
Habitantes do Maranhão. Cf. F. Manuel, \textunderscore Apólogos\textunderscore , I, 130.
\section{Maranhona}
\begin{itemize}
\item {Grp. gram.:f.}
\end{itemize}
\begin{itemize}
\item {Utilização:Prov.}
\end{itemize}
\begin{itemize}
\item {Utilização:trasm.}
\end{itemize}
Batata grande e branca.
\section{Maranhoso}
\begin{itemize}
\item {Grp. gram.:adj.}
\end{itemize}
\begin{itemize}
\item {Proveniência:(De \textunderscore maranha\textunderscore )}
\end{itemize}
Que diz maranhões, que faz enredos.
Intriguista; mexeriqueiro.
\section{Maranta}
\begin{itemize}
\item {Grp. gram.:m.}
\end{itemize}
Gênero de plantas amómeas, typo da tríbo das marantáceas.
\section{Marantáceas}
\begin{itemize}
\item {Grp. gram.:f. pl.}
\end{itemize}
Tríbo de plantas, de caule herbáceo, com flôres em espigas ou cachos.--A araruta é a fécula do maranta arundináceo.
\section{Marantes}
\begin{itemize}
\item {Grp. gram.:m.}
\end{itemize}
\begin{itemize}
\item {Utilização:Prov.}
\end{itemize}
Ave, o mesmo que \textunderscore papa-figo\textunderscore .
\section{Marão}
\begin{itemize}
\item {Grp. gram.:m.}
\end{itemize}
\begin{itemize}
\item {Utilização:Prov.}
\end{itemize}
\begin{itemize}
\item {Utilização:trasm.}
\end{itemize}
Casa grande.
O mesmo que \textunderscore carneiro\textunderscore ^2.
\section{Marapá}
\begin{itemize}
\item {Grp. gram.:m.}
\end{itemize}
\begin{itemize}
\item {Utilização:Bras. do N}
\end{itemize}
Espécie de pinheiro.
\section{Marapaúba}
\begin{itemize}
\item {Grp. gram.:m.}
\end{itemize}
\begin{itemize}
\item {Utilização:Bras}
\end{itemize}
Árvore leitosa do valle do Amazonas.
\section{Marapião}
\begin{itemize}
\item {Grp. gram.:m.}
\end{itemize}
Grande árvore santhomense, própria para construcções.
\section{Marapinima}
\begin{itemize}
\item {Grp. gram.:f.}
\end{itemize}
Árvore silvestre da região do Amazonas, empregada em marcenaria.
\section{Marapinina}
\begin{itemize}
\item {Grp. gram.:f.}
\end{itemize}
Palavra, registada nos diccionários, talvez erradamente, em vez de \textunderscore marapinima\textunderscore .(V.marapinima)
\section{Marapitanas}
\begin{itemize}
\item {Grp. gram.:m. pl.}
\end{itemize}
\begin{itemize}
\item {Utilização:Bras}
\end{itemize}
Tríbo de aborígenes, que habitou no Pará.
\section{Marapuama}
\begin{itemize}
\item {Grp. gram.:m.}
\end{itemize}
Erva medicinal do Brasil, da fam. das acantháceas.
\section{Maraquitica}
\begin{itemize}
\item {Grp. gram.:f.}
\end{itemize}
\begin{itemize}
\item {Utilização:Bras}
\end{itemize}
Planta medicinal.
\section{Marar}
\begin{itemize}
\item {Grp. gram.:v. t.}
\end{itemize}
\begin{itemize}
\item {Utilização:Gír.}
\end{itemize}
Matar; esfaquear.
(Do calô de Espanha)
\section{Marasca}
\begin{itemize}
\item {Grp. gram.:f.}
\end{itemize}
\begin{itemize}
\item {Proveniência:(It. \textunderscore marasca\textunderscore , por \textunderscore amarasca\textunderscore , amargosa)}
\end{itemize}
Variedade de cereja amargosa, que serve para o fabríco do marasquino.
\section{Marasmar}
\begin{itemize}
\item {Grp. gram.:v. t.}
\end{itemize}
\begin{itemize}
\item {Grp. gram.:V. i.  e  p.}
\end{itemize}
Causar marasmo a.
Caír em marasmo.
\section{Marasmático}
\begin{itemize}
\item {Grp. gram.:adj.}
\end{itemize}
Que tem marasmo; extenuado; apáthico. Cf. Castilho, \textunderscore Fastos\textunderscore , I, 559.
\section{Marasmo}
\begin{itemize}
\item {Grp. gram.:m.}
\end{itemize}
\begin{itemize}
\item {Utilização:Fig.}
\end{itemize}
\begin{itemize}
\item {Proveniência:(Gr. \textunderscore marasmos\textunderscore )}
\end{itemize}
Extenuação, em consequência de uma lesão orgânica.
Fraqueza extrema; atonia.
Magreza excessiva.
Indifferença, apathia moral.
Melancolia.
\section{Marasmódico}
\begin{itemize}
\item {Grp. gram.:adj.}
\end{itemize}
Relativo ou semelhante a marasmo.
\section{Marasquino}
\begin{itemize}
\item {Grp. gram.:m.}
\end{itemize}
\begin{itemize}
\item {Proveniência:(De \textunderscore marasca\textunderscore )}
\end{itemize}
Licor branco, fabricado com marascas ou cerejas azedas.
\section{Marata}
\begin{itemize}
\item {Grp. gram.:f.}
\end{itemize}
O mesmo que \textunderscore bom-vedro\textunderscore .
\section{Marata}
\begin{itemize}
\item {Grp. gram.:m.}
\end{itemize}
\begin{itemize}
\item {Grp. gram.:Pl.}
\end{itemize}
Língua culta, na Índia central.
Habitantes de uma parte da Índia.
\section{Marataná}
\begin{itemize}
\item {Grp. gram.:m.}
\end{itemize}
Árvore da região do Amazonas, própria para construcções.
\section{Marathónio}
\begin{itemize}
\item {Grp. gram.:adj.}
\end{itemize}
\begin{itemize}
\item {Grp. gram.:M.}
\end{itemize}
Relativo a Marathona.
Habitante de Marathona.
\section{Marathro}
\begin{itemize}
\item {Grp. gram.:m.}
\end{itemize}
\begin{itemize}
\item {Proveniência:(Do gr. \textunderscore marathron\textunderscore )}
\end{itemize}
O mesmo que \textunderscore funcho\textunderscore .
\section{Maratónio}
\begin{itemize}
\item {Grp. gram.:adj.}
\end{itemize}
\begin{itemize}
\item {Grp. gram.:M.}
\end{itemize}
Relativo a Maratona.
Habitante de Maratona.
\section{Maratro}
\begin{itemize}
\item {Grp. gram.:m.}
\end{itemize}
\begin{itemize}
\item {Proveniência:(Do gr. \textunderscore marathron\textunderscore )}
\end{itemize}
O mesmo que \textunderscore funcho\textunderscore .
\section{Maratuitica}
\begin{itemize}
\item {Grp. gram.:f.}
\end{itemize}
\begin{itemize}
\item {Utilização:Bras}
\end{itemize}
O mesmo que \textunderscore maraquitica\textunderscore .
\section{Marau}
\begin{itemize}
\item {Grp. gram.:m.}
\end{itemize}
\begin{itemize}
\item {Utilização:Chul.}
\end{itemize}
\begin{itemize}
\item {Proveniência:(Fr. \textunderscore maraud\textunderscore )}
\end{itemize}
Mariola.
Homem finório; espertalhão.
\section{Marauás}
\begin{itemize}
\item {Grp. gram.:m. pl.}
\end{itemize}
Indígenas do norte do Brasil.
\section{Maravalhas}
\begin{itemize}
\item {Grp. gram.:f. pl.}
\end{itemize}
\begin{itemize}
\item {Utilização:Prov.}
\end{itemize}
\begin{itemize}
\item {Utilização:Fig.}
\end{itemize}
Aparas de madeira.
Accendalhas.
Rama de pinheiro, caruma.
Bagatelas.
\section{Maravedi}
\begin{itemize}
\item {Grp. gram.:m.}
\end{itemize}
\begin{itemize}
\item {Proveniência:(Do ár. \textunderscore morabiti\textunderscore )}
\end{itemize}
Antiga moéda gótica, usada em Portugal e Espanha, onde teve geralmente o valor de 27 reis.
\section{Maravediada}
\begin{itemize}
\item {Grp. gram.:f.}
\end{itemize}
\begin{itemize}
\item {Utilização:Ant.}
\end{itemize}
\begin{itemize}
\item {Proveniência:(De \textunderscore maravedi\textunderscore )}
\end{itemize}
Porção de maravedis; dinheirama.
\section{Maravedil}
\begin{itemize}
\item {Grp. gram.:m.}
\end{itemize}
O mesmo que \textunderscore maravedi\textunderscore .
\section{Maravedinada}
\begin{itemize}
\item {Grp. gram.:f.}
\end{itemize}
Antiga medida para grãos, correspondente a pouco mais de 13 fangas.
\section{Maraves}
\begin{itemize}
\item {Grp. gram.:m. pl.}
\end{itemize}
Tríbo cafreal das regiões de Tete e Zumbo.
\section{Maravidi}
\begin{itemize}
\item {Grp. gram.:m.}
\end{itemize}
(V.maravedi)
\section{Maravilha}
\begin{itemize}
\item {Grp. gram.:f.}
\end{itemize}
\begin{itemize}
\item {Proveniência:(Do lat. \textunderscore mirabilia\textunderscore )}
\end{itemize}
Acto ou coisa extraordinária, que produz admiração.
Prodígio.
Coisa milagrosa.
Pessôa, que infunde admiração.
Planta balsamínea, espécie de malmequer.
\section{Maravilhador}
\begin{itemize}
\item {Grp. gram.:adj.}
\end{itemize}
\begin{itemize}
\item {Grp. gram.:M.}
\end{itemize}
Que maravilha ou causa admiração.
Aquelle que causa admiração.
\section{Maravilhar}
\begin{itemize}
\item {Grp. gram.:v. t.}
\end{itemize}
Causar maravilha ou admiração a; encher de espanto.
\section{Maravilho}
\begin{itemize}
\item {Grp. gram.:m.}
\end{itemize}
\begin{itemize}
\item {Utilização:Prov.}
\end{itemize}
\begin{itemize}
\item {Utilização:minh.}
\end{itemize}
Questão, discórdia.
\section{Maravilhosamente}
\begin{itemize}
\item {Grp. gram.:adv.}
\end{itemize}
De modo maravilhoso; admiravelmente; estupendamente.
\section{Maravilhoso}
\begin{itemize}
\item {Grp. gram.:adj.}
\end{itemize}
\begin{itemize}
\item {Grp. gram.:M.}
\end{itemize}
Que maravilha, que causa admiração.
Aquillo que contém maravilha.
Aquillo que é extraordinário.
\section{Marca}
\begin{itemize}
\item {Grp. gram.:f.}
\end{itemize}
\begin{itemize}
\item {Utilização:ant.}
\end{itemize}
\begin{itemize}
\item {Utilização:Gír.}
\end{itemize}
\begin{itemize}
\item {Proveniência:(Lat. \textunderscore marca\textunderscore )}
\end{itemize}
Acto ou effeito de marcar.
Cunho.
Distintivo.
Carimbo; firma.
Categoria.
Grandeza.
Nódoa, produzida por contusão.
Ferrete.
Nota.
Limite.
Tento do jôgo.
Letra ou letras, emblema ou bordado ligeiro, feito com agulha numa peça de roupa.
Botão para calças ou ceroilas.
Antiga moéda portuguesa, em oiro e em prata, do valor de 60 maravedis.
Meretriz.
\textunderscore Grande marca\textunderscore , talento, grande capacidade: \textunderscore advogado de grande marca\textunderscore .
\textunderscore Sêr marca\textunderscore , sêr capaz:«\textunderscore vós sereis marca de me inculcar nesta terra hũa namorada?\textunderscore »\textunderscore Eufrosina\textunderscore , 275.
\textunderscore Passar as marcas\textunderscore , ou \textunderscore passar das marcas\textunderscore , sêr exorbitante, exceder os justos limites.
\section{Marcá}
\begin{itemize}
\item {Grp. gram.:m.}
\end{itemize}
Antiga medida indiana, para azeite e manteiga, equivalente a pouco mais de 4 litros.
\section{Marcação}
\begin{itemize}
\item {Grp. gram.:f.}
\end{itemize}
Acto ou effeito de marcar.
\section{Marca-de-judas}
\begin{itemize}
\item {Grp. gram.:m.  e  f.}
\end{itemize}
\begin{itemize}
\item {Utilização:Pop.}
\end{itemize}
Baixa estatura.
Pessôa de baixa estatura.
\section{Marcado}
\begin{itemize}
\item {Grp. gram.:adj.}
\end{itemize}
\begin{itemize}
\item {Utilização:Fig.}
\end{itemize}
\begin{itemize}
\item {Grp. gram.:M.}
\end{itemize}
\begin{itemize}
\item {Utilização:Bras. do S}
\end{itemize}
Distinto.
Traficante.
Homem da cidade, (na linguagem dos roceiros).
\section{Marcadoiro}
\begin{itemize}
\item {Grp. gram.:adj.}
\end{itemize}
\begin{itemize}
\item {Proveniência:(De \textunderscore marcar\textunderscore )}
\end{itemize}
Diz-se das ligas de oiro ou prata, que estão nas condições legaes, para serem marcadas pelo contraste.
\section{Marcador}
\begin{itemize}
\item {Grp. gram.:m.}
\end{itemize}
\begin{itemize}
\item {Grp. gram.:Adj.}
\end{itemize}
Aquelle que marca.
Pedaço de talagarça, em que as crianças aprendem a marcar ou a bordar.
Que marca.
\section{Marcadouro}
\begin{itemize}
\item {Grp. gram.:adj.}
\end{itemize}
\begin{itemize}
\item {Proveniência:(De \textunderscore marcar\textunderscore )}
\end{itemize}
Diz-se das ligas de oiro ou prata, que estão nas condições legaes, para serem marcadas pelo contraste.
\section{Marçagão}
\begin{itemize}
\item {Grp. gram.:m.}
\end{itemize}
\begin{itemize}
\item {Utilização:Pop.}
\end{itemize}
O mês de Março, quando desabrido e áspero.
\section{Marçalino}
\begin{itemize}
\item {Grp. gram.:adj.}
\end{itemize}
\begin{itemize}
\item {Utilização:Pop.}
\end{itemize}
Relativo ao mês de Março: \textunderscore a lua marçalina\textunderscore .
\section{Marcanaíba}
\begin{itemize}
\item {Grp. gram.:f.}
\end{itemize}
Espécie de aderno.
\section{Marçano}
\begin{itemize}
\item {Grp. gram.:m.}
\end{itemize}
\begin{itemize}
\item {Utilização:Fig.}
\end{itemize}
Aprendiz de caixeiro.
Aprendiz, principiante.
(Por \textunderscore merçano\textunderscore , do lat. \textunderscore merx\textunderscore , \textunderscore mercis\textunderscore )
\section{Marcante}
\begin{itemize}
\item {Grp. gram.:adj.}
\end{itemize}
Que marca.
\section{Marca-pés}
\begin{itemize}
\item {Grp. gram.:m.}
\end{itemize}
\begin{itemize}
\item {Utilização:Bras}
\end{itemize}
Barro, com que se purifica o açúcar.
\section{Marcar}
\begin{itemize}
\item {Grp. gram.:v. t.}
\end{itemize}
\begin{itemize}
\item {Proveniência:(De \textunderscore marca\textunderscore )}
\end{itemize}
Pôr sinal em.
Assinalar.
Determinar, designar.
Calcular.
Firmar.
Fixar; limitar.
Ennodoar.
Bordar a fio de marca.
\section{Marçaria}
\begin{itemize}
\item {Grp. gram.:f.}
\end{itemize}
\begin{itemize}
\item {Utilização:Ant.}
\end{itemize}
O mesmo que \textunderscore mercadoria\textunderscore . Cf. \textunderscore Alvarás\textunderscore  de 22-XI-1498, e 16-XII-1499.
(Cp. \textunderscore marceiro\textunderscore )
\section{Marcassita}
\begin{itemize}
\item {Grp. gram.:f.}
\end{itemize}
\begin{itemize}
\item {Proveniência:(Do ár. \textunderscore marcaxiça\textunderscore )}
\end{itemize}
Crystal cúbico de uma pyrite de ferro sulfurado, empregada como objecto de ornato.
\section{Marcassite}
\begin{itemize}
\item {Grp. gram.:f.}
\end{itemize}
\begin{itemize}
\item {Proveniência:(Do ár. \textunderscore marcaxiça\textunderscore )}
\end{itemize}
Crystal cúbico de uma pyrite de ferro sulfurado, empregada como objecto de ornato.
\section{Marcaureles}
\begin{itemize}
\item {Grp. gram.:m.}
\end{itemize}
\begin{itemize}
\item {Utilização:Prov.}
\end{itemize}
\begin{itemize}
\item {Utilização:trasm.}
\end{itemize}
\begin{itemize}
\item {Utilização:Fam.}
\end{itemize}
O mesmo que \textunderscore dinheiro\textunderscore .
\section{Marcavala}
\begin{itemize}
\item {Grp. gram.:f.}
\end{itemize}
Planta da serra de Sintra.
\section{Marcegão}
\begin{itemize}
\item {Grp. gram.:adj.}
\end{itemize}
(V.marçagão)
\section{Marceiras}
\begin{itemize}
\item {Grp. gram.:f. pl.}
\end{itemize}
\begin{itemize}
\item {Proveniência:(De \textunderscore Março\textunderscore )}
\end{itemize}
Tributo, que se pagava em 1 de Março.
\section{Marceiro}
\begin{itemize}
\item {Grp. gram.:m.}
\end{itemize}
\begin{itemize}
\item {Utilização:Ant.}
\end{itemize}
Lojista, que vende açúcar, arroz e outros gêneros alimentícios; merceeiro.--Moraes, \textunderscore Diccion. da Líng. Port.\textunderscore , não define exactamente a palavra. Cf. F. Borges, \textunderscore Diccion. Jur.\textunderscore 
(Por \textunderscore merceiro\textunderscore  = \textunderscore merceeiro\textunderscore )
\section{Marcela}
\begin{itemize}
\item {Grp. gram.:f.}
\end{itemize}
(Fórma pop. de \textunderscore macela\textunderscore )
\section{Marcella}
\begin{itemize}
\item {Grp. gram.:f.}
\end{itemize}
(Fórma pop. de \textunderscore macela\textunderscore )
\section{Marcelo}
\begin{itemize}
\item {Grp. gram.:m.}
\end{itemize}
Antiga moéda de Veneza. Cf. Pant. de Aveiro, \textunderscore Itiner.\textunderscore , 24 v.^o e 199 v.^o, (2.^a ed.).
\section{Marcenaria}
\begin{itemize}
\item {Grp. gram.:f.}
\end{itemize}
Arte ou obras de marceneiro.
(Cp. \textunderscore marceneiro\textunderscore )
\section{Marceneiro}
\begin{itemize}
\item {Grp. gram.:m.}
\end{itemize}
\begin{itemize}
\item {Proveniência:(Do lat. \textunderscore mercenarius\textunderscore ?)}
\end{itemize}
Fabricante de móveis de madeira, especialmente de móveis tauxiados.
\section{Marceria}
\begin{itemize}
\item {Grp. gram.:f.}
\end{itemize}
\begin{itemize}
\item {Utilização:Ant.}
\end{itemize}
O mesmo que \textunderscore mercearia\textunderscore ^1.
(Cp. \textunderscore marceiro\textunderscore )
\section{Marcescência}
\begin{itemize}
\item {Grp. gram.:f.}
\end{itemize}
Qualidade de marcescente.
\section{Marcescente}
\begin{itemize}
\item {Grp. gram.:adj.}
\end{itemize}
\begin{itemize}
\item {Proveniência:(Lat. \textunderscore marcescens\textunderscore )}
\end{itemize}
Que murcha.
\section{Marcescível}
\begin{itemize}
\item {Grp. gram.:adj.}
\end{itemize}
\begin{itemize}
\item {Proveniência:(Lat. \textunderscore marcescibilis\textunderscore )}
\end{itemize}
Que murcha ou póde murchar.
\section{Marcha}
\begin{itemize}
\item {Grp. gram.:f.}
\end{itemize}
\begin{itemize}
\item {Utilização:Prov.}
\end{itemize}
\begin{itemize}
\item {Utilização:minh.}
\end{itemize}
\begin{itemize}
\item {Grp. gram.:Pl. Loc. adv.}
\end{itemize}
\begin{itemize}
\item {Proveniência:(Fr. \textunderscore marche\textunderscore )}
\end{itemize}
Acto ou effeito de marchar.
Cortejo.
Andamento regular.
Progresso.
Cadência.
Peça musical, para regular o passo de tropas ou de qualquer grupo de gente.
O mesmo que \textunderscore espremedeira\textunderscore .
\textunderscore Marcha surda\textunderscore , marcha sem rumor, silenciosa:«\textunderscore ...e com marcha surda chegou ao amanhecer.\textunderscore »Filinto, \textunderscore D. Man.\textunderscore , II, 223.
\textunderscore A marchas forçadas\textunderscore , marchando com rapidez, sem interrupções successivas.
\section{Marchadela}
\begin{itemize}
\item {Grp. gram.:f.}
\end{itemize}
\begin{itemize}
\item {Utilização:ant.}
\end{itemize}
\begin{itemize}
\item {Utilização:Fam.}
\end{itemize}
Acto de marchar.
\section{Marchador}
\begin{itemize}
\item {Grp. gram.:m.}
\end{itemize}
Apparelho de chapelaria. Cf. \textunderscore Inquér. Industr.\textunderscore , P. II, 1.^o 2.^o, 176.
\section{Marchantaria}
\begin{itemize}
\item {Grp. gram.:f.}
\end{itemize}
Negócio ou profissão de marchante.
\section{Marchante}
\begin{itemize}
\item {Grp. gram.:m.}
\end{itemize}
\begin{itemize}
\item {Utilização:Ant.}
\end{itemize}
\begin{itemize}
\item {Utilização:T. de Viana  e}
\end{itemize}
\begin{itemize}
\item {Utilização:ant.}
\end{itemize}
Negociante de gado para os açougues.
O mesmo que \textunderscore mercador\textunderscore . Cf. G. Vicente.
Dono ou empregado de açougue.
(Alter. de \textunderscore merchante\textunderscore )
\section{Marchar}
\begin{itemize}
\item {Grp. gram.:v. i.}
\end{itemize}
\begin{itemize}
\item {Utilização:Pop.}
\end{itemize}
\begin{itemize}
\item {Proveniência:(De \textunderscore marcha\textunderscore )}
\end{itemize}
Andar, caminhar.
Seguir os trâmites regulares: \textunderscore o negócio vai marchando\textunderscore .
Progredir.
Morrer.
\section{Marche-marche}
\begin{itemize}
\item {Grp. gram.:m.}
\end{itemize}
\begin{itemize}
\item {Grp. gram.:Interj.}
\end{itemize}
O mais rápido passo militar: \textunderscore o regimento partiu a marche-marche\textunderscore .
Voz, com que se ordena aquelle passo.
\section{Marcheta}
\begin{itemize}
\item {fónica:chê}
\end{itemize}
\begin{itemize}
\item {Grp. gram.:f.}
\end{itemize}
Parte do manto, em que se pregam as fitas.
O mesmo que \textunderscore marchete\textunderscore .
\section{Marchetado}
\begin{itemize}
\item {Grp. gram.:m.}
\end{itemize}
\begin{itemize}
\item {Proveniência:(De \textunderscore marchetar\textunderscore )}
\end{itemize}
Obra de marchetaria.
\section{Marchetar}
\begin{itemize}
\item {Grp. gram.:v. t.}
\end{itemize}
\begin{itemize}
\item {Utilização:Fig.}
\end{itemize}
Embutir em; tauxiar.
Matizar.
\section{Marchetaria}
\begin{itemize}
\item {Grp. gram.:f.}
\end{itemize}
\begin{itemize}
\item {Proveniência:(De \textunderscore marchete\textunderscore )}
\end{itemize}
Arte de marchetar.
Obra de embutidos ou de differentes pedaços variegados de madeira preciosa, marfim, madrepérola, etc.
\section{Marchete}
\begin{itemize}
\item {fónica:chê}
\end{itemize}
\begin{itemize}
\item {Grp. gram.:m.}
\end{itemize}
\begin{itemize}
\item {Proveniência:(De \textunderscore marchetar\textunderscore )}
\end{itemize}
Cada uma das peças, que se marchetam ou embutem sôbre a madeira.
\section{Marcheteiro}
\begin{itemize}
\item {Grp. gram.:m.}
\end{itemize}
\begin{itemize}
\item {Proveniência:(De \textunderscore marchete\textunderscore )}
\end{itemize}
Official de marchetaria.
\section{Marcial}
\begin{itemize}
\item {Grp. gram.:adj.}
\end{itemize}
\begin{itemize}
\item {Utilização:Pharm.}
\end{itemize}
\begin{itemize}
\item {Proveniência:(Lat. \textunderscore martialis\textunderscore )}
\end{itemize}
Relativo á guerra; béllico; bellicoso.
Relativo a militares ou a guerreiros: \textunderscore aspecto marcial\textunderscore .
Diz-se dos preparados ferruginosos.
\section{Marciano}
\begin{itemize}
\item {Grp. gram.:adj.}
\end{itemize}
O mesmo que \textunderscore marciático\textunderscore .
\section{Marciático}
\begin{itemize}
\item {Grp. gram.:adj.}
\end{itemize}
\begin{itemize}
\item {Proveniência:(Lat. \textunderscore martiaticus\textunderscore )}
\end{itemize}
Relativo ao planeta Marte.
\section{Márcido}
\begin{itemize}
\item {Grp. gram.:adj.}
\end{itemize}
\begin{itemize}
\item {Proveniência:(Lat. \textunderscore martiaticus\textunderscore )}
\end{itemize}
Relativo ao planeta Marte.
\section{Márcido}
\begin{itemize}
\item {Grp. gram.:adj.}
\end{itemize}
\begin{itemize}
\item {Proveniência:(Lat. \textunderscore marcidus\textunderscore )}
\end{itemize}
Que é murcho; que não tem vigor ou viço.
\section{Márcio}
\begin{itemize}
\item {Grp. gram.:adj.}
\end{itemize}
\begin{itemize}
\item {Proveniência:(Lat. \textunderscore martius\textunderscore )}
\end{itemize}
O mesmo que \textunderscore marcial\textunderscore .
\section{Marco}
\begin{itemize}
\item {Grp. gram.:m.}
\end{itemize}
\begin{itemize}
\item {Utilização:Prov.}
\end{itemize}
\begin{itemize}
\item {Utilização:trasm.}
\end{itemize}
\begin{itemize}
\item {Utilização:Ant.}
\end{itemize}
\begin{itemize}
\item {Utilização:ant.}
\end{itemize}
\begin{itemize}
\item {Utilização:Gír.}
\end{itemize}
Antigo pêso, equivalente a 8 onças.
Pedra oblonga, com que se demarcam terrenos.
Qualquer pedra, de situação natural, e que se aproveita para sinal de limites territoriaes.
Baliza, limite: fronteira.
Unidade monetária na Alemanha, equivalente a 220 reis.
O mesmo que \textunderscore dinheiro\textunderscore .
Talento, capacidade.
Homem.
\textunderscore Marco-postal\textunderscore , pequena construcção, insulada e cylíndrica, que serve para receptáculo público da correspondência, destinada ao correio.
\section{Março}
\begin{itemize}
\item {Grp. gram.:m.}
\end{itemize}
\begin{itemize}
\item {Proveniência:(Do lat. \textunderscore martius\textunderscore )}
\end{itemize}
Terceiro mês do anno romano.
\section{Marco-branco}
\begin{itemize}
\item {Grp. gram.:m.}
\end{itemize}
Moéda hamburguesa, correspondente a 337 reis.
\section{Marco-de-colónia}
\begin{itemize}
\item {Grp. gram.:m.}
\end{itemize}
Pêso prussiano de 234 grammas.
\section{Marcomanos}
\begin{itemize}
\item {Grp. gram.:m. pl.}
\end{itemize}
\begin{itemize}
\item {Proveniência:(Do gót. \textunderscore marka\textunderscore  + \textunderscore man\textunderscore )}
\end{itemize}
Antigo povo da Alemanha do Norte. Cf. \textunderscore Lusíadas\textunderscore , XIII, 78.
\section{Marconigrama}
\begin{itemize}
\item {Grp. gram.:m.}
\end{itemize}
\begin{itemize}
\item {Proveniência:(De \textunderscore Marconí\textunderscore , n. p. + \textunderscore grama\textunderscore )}
\end{itemize}
Comunicação pela telegrafia sem fios.
\section{Marconigramma}
\begin{itemize}
\item {Grp. gram.:m.}
\end{itemize}
\begin{itemize}
\item {Proveniência:(De \textunderscore Marconí\textunderscore , n. p. + \textunderscore gramma\textunderscore )}
\end{itemize}
Communicação pela telegraphia sem fios.
\section{Marco-velho}
\begin{itemize}
\item {Grp. gram.:m.}
\end{itemize}
Antiga moéda portuguesa, de 27 soldos.
\section{Mare}
\begin{itemize}
\item {Grp. gram.:f.}
\end{itemize}
\begin{itemize}
\item {Utilização:Ant.}
\end{itemize}
O mesmo que \textunderscore madre\textunderscore  ou \textunderscore mãe\textunderscore .
\section{Maré}
\begin{itemize}
\item {Grp. gram.:f.}
\end{itemize}
\begin{itemize}
\item {Utilização:Fig.}
\end{itemize}
\begin{itemize}
\item {Utilização:Bras. do Pará}
\end{itemize}
\begin{itemize}
\item {Utilização:Fig.}
\end{itemize}
\begin{itemize}
\item {Proveniência:(Fr. \textunderscore marée\textunderscore )}
\end{itemize}
Movimento das águas do mar, que periodicamente e duas vezes por dia, se elevam e se abaixam alternativamente.
Fluxo e refluxo dos acontecimentos humanos.
Opportunidade.
Distância itinerária, de um ponto a outro, nas viagens fluviaes, que dependem do fluxo ou refluxo da maré.
\textunderscore Maré-do-carvoeiro\textunderscore , opportunidade.
\textunderscore A favor da maré\textunderscore , no sentido ou na direcção, da maré; com o auxílio da maré.
\textunderscore Maré de rosas\textunderscore , situação bonançosa do mar, com tempo próspero para a navegação.
Opportunidade excellente; excellente occasião.
\section{Mareação}
\begin{itemize}
\item {Grp. gram.:f.}
\end{itemize}
Acto ou effeito de marear^2.
\section{Mareagem}
\begin{itemize}
\item {Grp. gram.:f.}
\end{itemize}
Acto ou effeito de marear^1.
Conjunto dos apparelhos, com que se move o navio.
Direcção, que o navio segue.
\section{Mareante}
\begin{itemize}
\item {Grp. gram.:adj.}
\end{itemize}
\begin{itemize}
\item {Grp. gram.:M.}
\end{itemize}
\begin{itemize}
\item {Proveniência:(De \textunderscore marear\textunderscore ^2)}
\end{itemize}
Que mareia.
Navegante; marinheiro.
\section{Marear}
\begin{itemize}
\item {Grp. gram.:v. t.}
\end{itemize}
\begin{itemize}
\item {Utilização:Náut.}
\end{itemize}
\begin{itemize}
\item {Utilização:Prov.}
\end{itemize}
\begin{itemize}
\item {Utilização:trasm.}
\end{itemize}
\begin{itemize}
\item {Grp. gram.:V. i.}
\end{itemize}
Governar (o navio).
Fazer enjoar.
Manchar; tirar o brilho a.
Oxydar.
Deslustrar.
Entontecer (o toiro) com passes de muleta e capote.
O mesmo que \textunderscore orientar\textunderscore .
\textunderscore Marear a vida\textunderscore , governar a vida.
Andar embarcado.
Têr enjôo a bordo.
Perturbar-se:«\textunderscore não me lembres a desgraça..., que mareio e perco a vista.\textunderscore »\textunderscore Luz e Calor\textunderscore , 577.
(B. lat. \textunderscore mareare\textunderscore )
\section{Marear}
\begin{itemize}
\item {Grp. gram.:v. i.}
\end{itemize}
\begin{itemize}
\item {Utilização:Gír.}
\end{itemize}
Assassinar, esfaquear.
(Cp. \textunderscore marar\textunderscore )
\section{Marechal}
\begin{itemize}
\item {Grp. gram.:m.}
\end{itemize}
\begin{itemize}
\item {Proveniência:(Do lat. \textunderscore marescalcus\textunderscore )}
\end{itemize}
Antigo posto superior, no exército.
\section{Marechala}
\begin{itemize}
\item {Grp. gram.:f.}
\end{itemize}
Título da mulher de marechal.
\section{Marechalado}
\begin{itemize}
\item {Grp. gram.:m.}
\end{itemize}
Cargo ou dignidade de marechal.
\section{Marechalato}
\begin{itemize}
\item {Grp. gram.:m.}
\end{itemize}
Cargo ou dignidade de marechal.
\section{Marégrafo}
\begin{itemize}
\item {Grp. gram.:m.}
\end{itemize}
\begin{itemize}
\item {Proveniência:(Do lat. \textunderscore mare\textunderscore  + gr. \textunderscore graphein\textunderscore )}
\end{itemize}
Instrumento, que regista automaticamente a altura das águas do mar.
\section{Marégrapho}
\begin{itemize}
\item {Grp. gram.:m.}
\end{itemize}
\begin{itemize}
\item {Proveniência:(Do lat. \textunderscore mare\textunderscore  + gr. \textunderscore graphein\textunderscore )}
\end{itemize}
Instrumento, que regista automaticamente a altura das águas do mar.
\section{Mareiro}
\begin{itemize}
\item {Grp. gram.:adj.}
\end{itemize}
\begin{itemize}
\item {Grp. gram.:M.}
\end{itemize}
Que sopra do mar, (falando-se do vento).
Propício para a navegação.
Vento do mar.
\section{Marejada}
\begin{itemize}
\item {Grp. gram.:f.}
\end{itemize}
\begin{itemize}
\item {Proveniência:(Do rad. de \textunderscore mar\textunderscore )}
\end{itemize}
Marulho, leve agitação de ondas.
\section{Marejar}
\begin{itemize}
\item {Grp. gram.:v. i.}
\end{itemize}
\begin{itemize}
\item {Utilização:Fig.}
\end{itemize}
\begin{itemize}
\item {Proveniência:(De \textunderscore mar\textunderscore )}
\end{itemize}
Resumar pelos poros um líquido.
Gotejar.
Borbulhar.
\section{Marel}
\begin{itemize}
\item {Grp. gram.:m.  e  adj.}
\end{itemize}
O mesmo que \textunderscore padreador\textunderscore .
\section{Marelante}
\begin{itemize}
\item {Grp. gram.:m.}
\end{itemize}
\begin{itemize}
\item {Utilização:Prov.}
\end{itemize}
Ave, o mesmo que \textunderscore marantes\textunderscore .
\section{Marema}
\begin{itemize}
\item {Grp. gram.:f.}
\end{itemize}
\begin{itemize}
\item {Proveniência:(It. \textunderscore maremma\textunderscore )}
\end{itemize}
Nome que, na Itália, se dá ao terreno situado á beira-mar, inhabitável no estio por causa das emanações deletérias, mas que no inverno fórma abundantes e apreciadas pastagens.
\section{Maremático}
\begin{itemize}
\item {Grp. gram.:adj.}
\end{itemize}
\begin{itemize}
\item {Proveniência:(De \textunderscore marema\textunderscore )}
\end{itemize}
Relativo ás maremas, (especialmente falando-se de febres).
\section{Marémetro}
\begin{itemize}
\item {Grp. gram.:m.}
\end{itemize}
\begin{itemize}
\item {Proveniência:(T. hybr., do lat. \textunderscore mare\textunderscore  + gr. \textunderscore metron\textunderscore )}
\end{itemize}
Instrumento, para medir automaticamente a altura das águas do mar.
\section{Maremma}
\begin{itemize}
\item {Grp. gram.:f.}
\end{itemize}
\begin{itemize}
\item {Proveniência:(It. \textunderscore maremma\textunderscore )}
\end{itemize}
Nome que, na Itália, se dá ao terreno situado á beira-mar, inhabitável no estio por causa das emanações deletérias, mas que no inverno fórma abundantes e apreciadas pastagens.
\section{Maremmático}
\begin{itemize}
\item {Grp. gram.:adj.}
\end{itemize}
\begin{itemize}
\item {Proveniência:(De \textunderscore maremma\textunderscore )}
\end{itemize}
Relativo ás maremmas, (especialmente falando-se de febres).
\section{Maremoto}
\begin{itemize}
\item {Grp. gram.:m.}
\end{itemize}
\begin{itemize}
\item {Proveniência:(Do lat. \textunderscore mare\textunderscore  + \textunderscore motus\textunderscore )}
\end{itemize}
Tremor do mar.
\section{Mareógrafo}
\begin{itemize}
\item {Grp. gram.:m.}
\end{itemize}
(V.marégrafo)
\section{Mareógrapho}
\begin{itemize}
\item {Grp. gram.:m.}
\end{itemize}
(V.marégrapho)
\section{Mareómetro}
\begin{itemize}
\item {Grp. gram.:m.}
\end{itemize}
(V.marémetro)
\section{Mareorama}
\begin{itemize}
\item {Grp. gram.:m.}
\end{itemize}
\begin{itemize}
\item {Utilização:Neol.}
\end{itemize}
Representação óptica dos mares e da navegação.
Panorama do mar.
\section{Mareótico}
\begin{itemize}
\item {Grp. gram.:adj.}
\end{itemize}
\begin{itemize}
\item {Proveniência:(Lat. \textunderscore mareoticus\textunderscore )}
\end{itemize}
Relativo a Mareóta, antiga cidade do Baixo Egypto, notável pelos vinhos produzidos nas suas cercanias:«\textunderscore ...a uva mareótica...\textunderscore »Castilho, \textunderscore Geórgicas\textunderscore , 79.
\section{Maresia}
\begin{itemize}
\item {Grp. gram.:f.}
\end{itemize}
\begin{itemize}
\item {Proveniência:(Do rad. de \textunderscore maré\textunderscore )}
\end{itemize}
Mau cheiro do mar, na vasante.
Marejada.
\section{Mareta}
\begin{itemize}
\item {fónica:marê}
\end{itemize}
\begin{itemize}
\item {Grp. gram.:f.}
\end{itemize}
\begin{itemize}
\item {Proveniência:(De \textunderscore maré\textunderscore )}
\end{itemize}
Pequena onda; onda do rio.--É termo geralmente mal definido nos diccion. Cf. Borges, \textunderscore Diccion. Jur.\textunderscore 
\section{Mareta}
\begin{itemize}
\item {fónica:marê}
\end{itemize}
\begin{itemize}
\item {Grp. gram.:f.}
\end{itemize}
O mesmo que \textunderscore sandará\textunderscore , plantas indiana.
\section{Maretina}
\begin{itemize}
\item {Grp. gram.:f.}
\end{itemize}
Medicamento antipyrético.
\section{Marfado}
\begin{itemize}
\item {Grp. gram.:adj.}
\end{itemize}
\begin{itemize}
\item {Utilização:T. de Ourique}
\end{itemize}
Hydróphobo.
Raivento.
Zangado, desgostoso: \textunderscore aquelle meu vizinho anda marfado commigo\textunderscore .
\section{Marfar}
\begin{itemize}
\item {Grp. gram.:v. t.}
\end{itemize}
Offender; enfurecer.
Causar desgôsto a.
\section{Marfi}
\begin{itemize}
\item {Grp. gram.:m.}
\end{itemize}
\begin{itemize}
\item {Utilização:Ant.}
\end{itemize}
O mesmo que \textunderscore marfim\textunderscore .
\section{Marfim}
\begin{itemize}
\item {Grp. gram.:m.}
\end{itemize}
\begin{itemize}
\item {Proveniência:(Do ár. \textunderscore nabfil\textunderscore ?)}
\end{itemize}
Substância branca e compacta, que constitue os dentes de todos os animaes mammíferos.
Dentes do elephante, antes de transformados em artefactos.
Qualquer trabalho ou obra, feita dos dentes de elephante.
Aquillo que é semelhante ao marfim, na alvura.
Primeiro elemento da denominação de algumas plantas brasileiras.
Nome de uma qualidade de queijo.
\section{Marfim-vegetal}
\begin{itemize}
\item {Grp. gram.:m.}
\end{itemize}
Fruto duríssimo de uma palmeira, que cresce especialmente nas margens do rio Içá, (Brasil)
\section{Marfolhar}
\begin{itemize}
\item {Grp. gram.:v. i.}
\end{itemize}
\begin{itemize}
\item {Utilização:Prov.}
\end{itemize}
\begin{itemize}
\item {Utilização:beir.}
\end{itemize}
Diz-se da seara já crescida, que póde ondear como marfolho.
\section{Marfolho}
\begin{itemize}
\item {fónica:fô}
\end{itemize}
\begin{itemize}
\item {Grp. gram.:m.}
\end{itemize}
\begin{itemize}
\item {Utilização:Prov.}
\end{itemize}
\begin{itemize}
\item {Utilização:beir.}
\end{itemize}
\begin{itemize}
\item {Proveniência:(De \textunderscore mar\textunderscore  + \textunderscore folha\textunderscore )}
\end{itemize}
Seara tenra, mas já crescida e que póde ondear ao sabor do vento. (Colhido no Sabugal)
\section{Marfuz}
\begin{itemize}
\item {Grp. gram.:adj.}
\end{itemize}
\begin{itemize}
\item {Utilização:Ant.}
\end{itemize}
\begin{itemize}
\item {Proveniência:(T. as.)}
\end{itemize}
Mau.
\section{Marga}
\begin{itemize}
\item {Grp. gram.:f.}
\end{itemize}
\begin{itemize}
\item {Proveniência:(Lat. \textunderscore marga\textunderscore )}
\end{itemize}
Espécie de greda, mixto de argilla e calcário, applicavel na olaria ou em adubos de terra.
\section{Margaça}
\begin{itemize}
\item {Grp. gram.:f.}
\end{itemize}
O mesmo que \textunderscore macella\textunderscore , (dizem os diccion).
Na Beira Alta, dá-se aquelle nome a uma planta herbácea, que serve, em rações, para os cavallos e outros animaes.
\section{Margagem}
\begin{itemize}
\item {Grp. gram.:f.}
\end{itemize}
Acto ou effeito de margar.
\section{Margai}
\begin{itemize}
\item {Grp. gram.:m.}
\end{itemize}
Espécie de gato bravo da América do Sul, (\textunderscore felis tigrina\textunderscore , Gmelin).
\section{Margalho}
\begin{itemize}
\item {Grp. gram.:m.}
\end{itemize}
\begin{itemize}
\item {Utilização:T. dos Campos de Coimbra}
\end{itemize}
Lodo ou nateiro, que o rio deixa aos lados, depois das enchentes.
(Provavelmente, de \textunderscore marga\textunderscore )
\section{Margalhudo}
\begin{itemize}
\item {Grp. gram.:adj.}
\end{itemize}
\begin{itemize}
\item {Utilização:des.}
\end{itemize}
\begin{itemize}
\item {Utilização:Pop.}
\end{itemize}
Dizia-se do indivíduo desajeitado, trangalhadanças.
\section{Margar}
\begin{itemize}
\item {Grp. gram.:v. t.}
\end{itemize}
Adubar com marga.
\section{Margarantho}
\begin{itemize}
\item {Grp. gram.:m.}
\end{itemize}
\begin{itemize}
\item {Proveniência:(Do gr. \textunderscore margaros\textunderscore  + \textunderscore anthos\textunderscore )}
\end{itemize}
Gênero de plantas solâneas do México.
\section{Margaranto}
\begin{itemize}
\item {Grp. gram.:m.}
\end{itemize}
\begin{itemize}
\item {Proveniência:(Do gr. \textunderscore margaros\textunderscore  + \textunderscore anthos\textunderscore )}
\end{itemize}
Gênero de plantas solâneas do México.
\section{Margarato}
\begin{itemize}
\item {Grp. gram.:m.}
\end{itemize}
\begin{itemize}
\item {Proveniência:(De \textunderscore margárico\textunderscore )}
\end{itemize}
Combinação do ácido margárico com uma base.
\section{Margárico}
\begin{itemize}
\item {Grp. gram.:adj.}
\end{itemize}
Diz-se de um ácido, contido na margarina.
\section{Margarida}
\begin{itemize}
\item {Grp. gram.:f.}
\end{itemize}
Ave palmípede aquática.
O mesmo que \textunderscore margarita\textunderscore .
\section{Margarina}
\begin{itemize}
\item {Grp. gram.:f.}
\end{itemize}
\begin{itemize}
\item {Proveniência:(Do fr. \textunderscore margarine\textunderscore )}
\end{itemize}
Substância gordurosa, que se extrai de certos óleos e do tecido adiposo dos animaes.
\section{Margarita}
\begin{itemize}
\item {Grp. gram.:f.}
\end{itemize}
\begin{itemize}
\item {Proveniência:(Lat. \textunderscore margarita\textunderscore )}
\end{itemize}
Pérola.
Espécie de mollusco que produz o nácar.
Pedra branca, que contém principalmente silicato de cal e de alumina.
Nome de várias plantas, da fam. das compostas.
Árvore cotonígera do Amazonas.
Gênero de conchas que produzem pérolas.
\section{Margaritáceas}
\begin{itemize}
\item {Grp. gram.:f. pl.}
\end{itemize}
\begin{itemize}
\item {Proveniência:(De \textunderscore margaritáceo\textunderscore )}
\end{itemize}
Família de molluscos bivalves.
\section{Margaritáceo}
\begin{itemize}
\item {Grp. gram.:adj.}
\end{itemize}
\begin{itemize}
\item {Proveniência:(De \textunderscore margarita\textunderscore )}
\end{itemize}
Diz-se dos molluscos que produzem pérolas.
\section{Margaritato}
\begin{itemize}
\item {Grp. gram.:m.}
\end{itemize}
\begin{itemize}
\item {Proveniência:(De \textunderscore margarítico\textunderscore )}
\end{itemize}
Combinação do ácido margarítico com uma base.
\section{Margarítico}
\begin{itemize}
\item {Grp. gram.:adj.}
\end{itemize}
Diz-se de um ácido, que se obtém pela destillação do óleo de rícino.
\section{Margaritífero}
\begin{itemize}
\item {Grp. gram.:adj.}
\end{itemize}
\begin{itemize}
\item {Proveniência:(Lat. \textunderscore margaritiferus\textunderscore )}
\end{itemize}
Que produz pérolas.
\section{Margaritita}
\begin{itemize}
\item {Grp. gram.:f.}
\end{itemize}
\begin{itemize}
\item {Proveniência:(De \textunderscore margarita\textunderscore )}
\end{itemize}
Designação antiquada da pérola fóssil.
\section{Margarona}
\begin{itemize}
\item {Grp. gram.:f.}
\end{itemize}
\begin{itemize}
\item {Utilização:Chím.}
\end{itemize}
Producto sólido, que se fórma durante a destillação sêca do ácido margárico e do margarato de cal.
\section{Marge}
\begin{itemize}
\item {Grp. gram.:f.}
\end{itemize}
O mesmo que \textunderscore margem\textunderscore .
\section{Margear}
\begin{itemize}
\item {Grp. gram.:v. t.}
\end{itemize}
\begin{itemize}
\item {Utilização:Prov.}
\end{itemize}
Ir pela margem de.
Andar ao longo ou ao lado de.
Estar na margem de.
Abrir sulcos com arado em (terra semeada de cereaes), para formar as margens, que devem ficar entre os sulcos.
\section{Margem}
\begin{itemize}
\item {Grp. gram.:f.}
\end{itemize}
\begin{itemize}
\item {Utilização:Fig.}
\end{itemize}
\begin{itemize}
\item {Proveniência:(Do lat. \textunderscore margo\textunderscore , \textunderscore marginis\textunderscore )}
\end{itemize}
A parte branca, em volta de uma página escrita ou impressa, especialmente o espaço em branco de cada um dos dois lados da página.
Borda.
Beira, riba.
Cercadura.
Leira, espaço de terra lavrada, comprehendído entre dois regos.
Ensejo, facilidade: \textunderscore dar margem\textunderscore , dar ensejo ou occasião; dar espaço ou lugar.
\textunderscore Lançar\textunderscore  ou \textunderscore deitar á margem\textunderscore , abandonar; desprezar.
\section{Margido}
\begin{itemize}
\item {Grp. gram.:m.}
\end{itemize}
\begin{itemize}
\item {Utilização:Prov.}
\end{itemize}
\begin{itemize}
\item {Proveniência:(De \textunderscore marge\textunderscore )}
\end{itemize}
\textunderscore Semear de margido\textunderscore , semear em mantas ou margens, separadas por sulcos. Cf. \textunderscore Bibl. da G. do Campo\textunderscore , 275.
\section{Marginal}
\begin{itemize}
\item {Grp. gram.:adj.}
\end{itemize}
\begin{itemize}
\item {Proveniência:(Lat. \textunderscore marginalis\textunderscore )}
\end{itemize}
Relativo a margem: \textunderscore os campos marginaes do Mondego\textunderscore .
\section{Marginar}
\begin{itemize}
\item {Grp. gram.:v. t.}
\end{itemize}
\begin{itemize}
\item {Proveniência:(Lat. \textunderscore marginare\textunderscore )}
\end{itemize}
O mesmo que \textunderscore margear\textunderscore .
Annotar á margem (um livro ou uma fôlha).
\section{Marginário}
\begin{itemize}
\item {Grp. gram.:adj.}
\end{itemize}
\begin{itemize}
\item {Utilização:Bot.}
\end{itemize}
\begin{itemize}
\item {Proveniência:(De \textunderscore marginar\textunderscore )}
\end{itemize}
Diz-se dos septos, formados pelo bordo das válvulas que entram no interior do pericarpo.
\section{Marginatura}
\begin{itemize}
\item {Grp. gram.:f.}
\end{itemize}
\begin{itemize}
\item {Utilização:Bot.}
\end{itemize}
\begin{itemize}
\item {Proveniência:(De \textunderscore marginar\textunderscore )}
\end{itemize}
Estado do órgão vegetal, que é marginado ou marginário.
\section{Marginiforme}
\begin{itemize}
\item {Grp. gram.:adj.}
\end{itemize}
\begin{itemize}
\item {Proveniência:(Do lat. \textunderscore margo\textunderscore , \textunderscore marginis\textunderscore  + \textunderscore forma\textunderscore )}
\end{itemize}
Semelhante a uma cercadura.
\section{Margio}
\begin{itemize}
\item {Grp. gram.:m.}
\end{itemize}
\begin{itemize}
\item {Utilização:Prov.}
\end{itemize}
\begin{itemize}
\item {Utilização:alent.}
\end{itemize}
\begin{itemize}
\item {Proveniência:(De \textunderscore marge\textunderscore ? Cp. \textunderscore margido\textunderscore )}
\end{itemize}
Grande extensão de terra arável.
\section{Margoso}
\begin{itemize}
\item {Grp. gram.:m.}
\end{itemize}
Fruta de Macau, (\textunderscore momordica balsamina\textunderscore ).
\section{Margoso}
\begin{itemize}
\item {Grp. gram.:adj.}
\end{itemize}
Semelhante á marga.
Que contém marga.
\section{Margota}
\begin{itemize}
\item {Grp. gram.:f.}
\end{itemize}
Peixe africano, da fam. dos labroides.
\section{Margrave}
\begin{itemize}
\item {Grp. gram.:m.}
\end{itemize}
\begin{itemize}
\item {Proveniência:(Do al. \textunderscore markgraf\textunderscore )}
\end{itemize}
Antigo chefe civil e militar, que governava as províncias fronteiriças, e cujo título se conservou em alguns principados alemães.
\section{Margraviáceas}
\begin{itemize}
\item {Grp. gram.:f. pl.}
\end{itemize}
Família de plantas parasitas, na América.
\section{Margiricarpo}
\begin{itemize}
\item {Grp. gram.:m.}
\end{itemize}
\begin{itemize}
\item {Proveniência:(Do gr. \textunderscore marguron\textunderscore  + \textunderscore karpos\textunderscore )}
\end{itemize}
Gênero de plantas rosáceas.
\section{Margraviado}
\begin{itemize}
\item {Grp. gram.:m.}
\end{itemize}
Cargo ou dignidade de margrave.
\section{Margraviato}
\begin{itemize}
\item {Grp. gram.:m.}
\end{itemize}
Cargo ou dignidade de margrave.
\section{Margueira}
\begin{itemize}
\item {Grp. gram.:f.}
\end{itemize}
Lugar, em que há marga.
\section{Margueiro}
\begin{itemize}
\item {Grp. gram.:m.}
\end{itemize}
Aquelle que apanha marga.
\section{Margyricarpo}
\begin{itemize}
\item {Grp. gram.:m.}
\end{itemize}
\begin{itemize}
\item {Proveniência:(Do gr. \textunderscore marguron\textunderscore  + \textunderscore karpos\textunderscore )}
\end{itemize}
Gênero de plantas rosáceas.
\section{Mari}
\begin{itemize}
\item {Grp. gram.:m.}
\end{itemize}
Planta leguminosa e medicinal.
\section{Maria}
\begin{itemize}
\item {Grp. gram.:f.}
\end{itemize}
Variedade de pêra.
\section{Maria-antónia}
\begin{itemize}
\item {Grp. gram.:f.}
\end{itemize}
Variedade de pêro.
\section{Maria-congueira}
\begin{itemize}
\item {Grp. gram.:f.}
\end{itemize}
\begin{itemize}
\item {Utilização:Bras. dos arredores do Rio}
\end{itemize}
Jôgo infantil, o mesmo que o jôgo das escondidas.
\section{Maria-da-grade}
\begin{itemize}
\item {Grp. gram.:f.}
\end{itemize}
\begin{itemize}
\item {Utilização:T. de Sangalhos}
\end{itemize}
Mulher fantástica que, segundo a superstição popular, habita nos rios, lagos e poços, attrai as crianças, que se lhe aproximam e afoga-as.--Talvez reminiscências das nymphas pagans.
\section{Maria-farinha}
\begin{itemize}
\item {Grp. gram.:f.}
\end{itemize}
Pequeno caranguejo de Pernambuco.
\section{Maria-fia}
\begin{itemize}
\item {Grp. gram.:f.}
\end{itemize}
Pequeno insecto que, fixando-se pelas antennas em roupa lavada, gira sôbre si, enrolando uma perna em a outra, até que morre.«\textunderscore Fia, fia, maria-fia, três maçarocas por dia\textunderscore ». Loc. pop.
\section{Mariagômbi}
\begin{itemize}
\item {Grp. gram.:m.}
\end{itemize}
\begin{itemize}
\item {Utilização:Bras}
\end{itemize}
Planta portulácea, alimenticia.
\section{Maria-gomes}
\begin{itemize}
\item {Grp. gram.:f.}
\end{itemize}
\begin{itemize}
\item {Utilização:Bras}
\end{itemize}
\begin{itemize}
\item {Utilização:T. da Bairrada}
\end{itemize}
Designação vulgar do mariagômbi.
Casta de uva branca, muito saborosa.
Variedade de maçan.
\section{Marial}
\begin{itemize}
\item {Grp. gram.:adj.}
\end{itemize}
\begin{itemize}
\item {Proveniência:(De \textunderscore Maria\textunderscore , n. p.)}
\end{itemize}
Relativo á Virgem Maria.
\section{Maria-leite}
\begin{itemize}
\item {Grp. gram.:f.}
\end{itemize}
Planta medicinal do Brasil.
\section{Marialva}
\begin{itemize}
\item {Grp. gram.:adj.}
\end{itemize}
\begin{itemize}
\item {Grp. gram.:M.}
\end{itemize}
\begin{itemize}
\item {Utilização:Deprec.}
\end{itemize}
\begin{itemize}
\item {Proveniência:(De \textunderscore Marialva\textunderscore , n. p.)}
\end{itemize}
Relativo ás regras de cavalgar á gineta, estabelecidas pelo marquês de Marialva.
Bom cavalleiro.
Aquelle que gosta de toiros e cavallos e timbra de extravagante e ocioso.
Fadista, que pertence a família distinta.
\section{Maria-molle}
\begin{itemize}
\item {Grp. gram.:f.}
\end{itemize}
\begin{itemize}
\item {Utilização:Bras}
\end{itemize}
O mesmo que \textunderscore umbu\textunderscore ^2.
\section{Maria-mucanguê}
\begin{itemize}
\item {Grp. gram.:f.}
\end{itemize}
\begin{itemize}
\item {Utilização:Bras. do Rio}
\end{itemize}
Divertimento de crianças.
\section{Mariana}
\begin{itemize}
\item {Grp. gram.:f.}
\end{itemize}
Planta solânea do Brasil.
Um dos nomes do pargo-mitrado, (\textunderscore dentex filosus valencianus\textunderscore ), em quanto é novo.
\section{Marianga}
\begin{itemize}
\item {Grp. gram.:f.}
\end{itemize}
Opulenta planta aquática da África.
\section{Mariangombe}
\begin{itemize}
\item {Grp. gram.:m.}
\end{itemize}
Árvore de Angola, (\textunderscore maerua angolensis\textunderscore ).
Provavelmente, o mesmo que \textunderscore mariagômbi\textunderscore .
\section{Marianinha}
\begin{itemize}
\item {Grp. gram.:f.}
\end{itemize}
\begin{itemize}
\item {Utilização:Bras}
\end{itemize}
O mesmo que \textunderscore trapoeraba\textunderscore .
\section{Marianismo}
\begin{itemize}
\item {Grp. gram.:m.}
\end{itemize}
\begin{itemize}
\item {Proveniência:(De \textunderscore mariano\textunderscore )}
\end{itemize}
Tendência para exaltar o culto da Virgem Maria, por uma fórma que excede a própria doutrina da Igreja.
\section{Marianitas}
\begin{itemize}
\item {Grp. gram.:m.}
\end{itemize}
Congregação religiosa, fundada em Bordéus, em 1817.
\section{Marianito}
\begin{itemize}
\item {Grp. gram.:m.}
\end{itemize}
Pequeno papagaio da fronteira occidental do Brasil.
\section{Mariano}
\begin{itemize}
\item {Grp. gram.:adj.}
\end{itemize}
\begin{itemize}
\item {Grp. gram.:M. pl.}
\end{itemize}
\begin{itemize}
\item {Proveniência:(Lat. \textunderscore marianus\textunderscore )}
\end{itemize}
Relativo á Virgem Maria ou ao seu culto: \textunderscore devoções marianas\textunderscore .
Nome de uma Ordem de frades.
\section{Maria-pindu}
\begin{itemize}
\item {Grp. gram.:f.}
\end{itemize}
Ave africana, (\textunderscore nactarina natalensis\textunderscore ).
\section{Maria-preta}
\begin{itemize}
\item {Grp. gram.:f.}
\end{itemize}
Planta brasileira, da fam. das compostas.
Planta cardiacea do Brasil.
\section{Mariaranas}
\begin{itemize}
\item {Grp. gram.:m. pl.}
\end{itemize}
Pequena tríbo de Índios do Pará, nas margens do rio Tefé.
\section{Maria-rosa}
\begin{itemize}
\item {Grp. gram.:f.}
\end{itemize}
\begin{itemize}
\item {Utilização:Bras}
\end{itemize}
Espécie de palmeira.
\section{Maria-segunda}
\begin{itemize}
\item {Grp. gram.:f.}
\end{itemize}
\begin{itemize}
\item {Utilização:T. de Benguela}
\end{itemize}
Missanga encarnada, miúda e de ôlho branco.
(Talvez de \textunderscore Maria Segunda\textunderscore , n. p., porque no tempo da raínha daquelle nome seriam taes missangas enviadas para Benguela)
\section{Maribondo}
\begin{itemize}
\item {Grp. gram.:m.}
\end{itemize}
\begin{itemize}
\item {Utilização:Bras}
\end{itemize}
Espécie de vespão, que faz ninho nos beiraes das casas, e cuja mordedura produz grande ardor.
Designação genérica de vespa, em quási todo o Brasil.
(Cp. \textunderscore maribundo\textunderscore )
\section{Maribunda}
\begin{itemize}
\item {Grp. gram.:f.}
\end{itemize}
Abelha verde e doirada das regiões do Amazonas.
(Cp. \textunderscore maribundo\textunderscore  e \textunderscore maribondo\textunderscore )
\section{Maribundo}
\begin{itemize}
\item {Grp. gram.:m.}
\end{itemize}
Abelha angolense, (\textunderscore pelopaeus spirifex\textunderscore ).
(Do quimb.)
\section{Maricão}
\begin{itemize}
\item {Grp. gram.:m.}
\end{itemize}
\begin{itemize}
\item {Grp. gram.:M.  e  adj.}
\end{itemize}
\begin{itemize}
\item {Utilização:Fam.}
\end{itemize}
\begin{itemize}
\item {Proveniência:(Do rad. de \textunderscore Maria\textunderscore , n. p.)}
\end{itemize}
Homem mulherengo, homem que se occupa de trabalhos próprios de mulheres
Aquelle que tem medo de tudo; cagarola.
\section{Maricas}
\begin{itemize}
\item {Grp. gram.:m.}
\end{itemize}
\begin{itemize}
\item {Grp. gram.:M.  e  adj.}
\end{itemize}
\begin{itemize}
\item {Utilização:Fam.}
\end{itemize}
\begin{itemize}
\item {Proveniência:(Do rad. de \textunderscore Maria\textunderscore , n. p.)}
\end{itemize}
Homem mulherengo, homem que se occupa de trabalhos próprios de mulheres
Aquelle que tem medo de tudo; cagarola.
\section{Mari-cazura}
\begin{itemize}
\item {Grp. gram.:m.}
\end{itemize}
Árvore da Guiana inglesa.
\section{Marichal}
\begin{itemize}
\item {Grp. gram.:m.}
\end{itemize}
O mesmo ou melhor que \textunderscore marechal\textunderscore . Cf. Barros, \textunderscore Déc\textunderscore . II, l. III, c. 9; Filinto, \textunderscore D. Man\textunderscore . II, 132.
\section{Maridagem}
\begin{itemize}
\item {Grp. gram.:f.}
\end{itemize}
O mesmo que \textunderscore maridança\textunderscore .
\section{Maridal}
\begin{itemize}
\item {Grp. gram.:adj.}
\end{itemize}
O mesmo que \textunderscore marital\textunderscore .
\section{Maridança}
\begin{itemize}
\item {Grp. gram.:f.}
\end{itemize}
\begin{itemize}
\item {Utilização:Fig.}
\end{itemize}
Acto ou effeito de maridar.
Vida de casados.
Harmonia, união, entre duas ou mais coisas.
\section{Maridar}
\begin{itemize}
\item {Grp. gram.:v. t.}
\end{itemize}
\begin{itemize}
\item {Proveniência:(Do lat. \textunderscore maritare\textunderscore )}
\end{itemize}
Casar (uma mulher) uni-la em casamento a alguém.
Enlaçar.
\section{Marido}
\begin{itemize}
\item {Grp. gram.:m.}
\end{itemize}
\begin{itemize}
\item {Proveniência:(Do lat. \textunderscore maritus\textunderscore )}
\end{itemize}
Cônjuge do sexo masculino.
Homem, em relação á mulher com quem está ligado por casamento.
\section{Marido-é-dia}
\begin{itemize}
\item {Grp. gram.:m.}
\end{itemize}
\begin{itemize}
\item {Utilização:Bras}
\end{itemize}
Pássaro, cujo canto parece reproduzir o seu nome.
\section{Marifusa}
\begin{itemize}
\item {Grp. gram.:f.}
\end{itemize}
\begin{itemize}
\item {Utilização:Prov.}
\end{itemize}
\begin{itemize}
\item {Utilização:trasm.}
\end{itemize}
Variedade de cogumelos comestíveis.
\section{Marigué}
\begin{itemize}
\item {Grp. gram.:m.}
\end{itemize}
Mosquito do Brasil.
\section{Marilha}
\begin{itemize}
\item {Grp. gram.:f.}
\end{itemize}
O mesmo que \textunderscore amarilha\textunderscore .
\section{Marilho}
\begin{itemize}
\item {Grp. gram.:m.}
\end{itemize}
\begin{itemize}
\item {Utilização:Prov.}
\end{itemize}
\begin{itemize}
\item {Utilização:Veter.}
\end{itemize}
\begin{itemize}
\item {Utilização:alent.}
\end{itemize}
\begin{itemize}
\item {Utilização:extrem.}
\end{itemize}
Falta de evacuação, nos porcos.
(Cp. \textunderscore marilha\textunderscore )
\section{Marim}
\begin{itemize}
\item {Grp. gram.:m.}
\end{itemize}
Antigo pôsto militar e civil, entre os Moiros.
\section{Marimacho}
\begin{itemize}
\item {Grp. gram.:m.}
\end{itemize}
\begin{itemize}
\item {Proveniência:(De \textunderscore Maria\textunderscore , n. p. + \textunderscore macho\textunderscore )}
\end{itemize}
Mulher, que tem o aspecto ou modos próprios de homem.
Virago.
\section{Mari-mari}
\begin{itemize}
\item {Grp. gram.:m.}
\end{itemize}
Planta leguminosa do Brasil.
(Do tupi)
\section{Marimba}
\begin{itemize}
\item {Grp. gram.:f.}
\end{itemize}
Espécie de tambor cafreal.
Instrumento musical, composto de lâminas de vidro ou metal, graduadas em escala, e das quaes se tira o som por meio de baquetas.
(Do quimbundo)
\section{Marimbá}
\begin{itemize}
\item {Grp. gram.:m.}
\end{itemize}
Pequeno mas apreciado peixe do Brasil.
\section{Marimbar}
\begin{itemize}
\item {Grp. gram.:v. i.}
\end{itemize}
\begin{itemize}
\item {Utilização:chul.}
\end{itemize}
\begin{itemize}
\item {Utilização:Fig.}
\end{itemize}
\begin{itemize}
\item {Grp. gram.:V. p.}
\end{itemize}
\begin{itemize}
\item {Utilização:Gír.}
\end{itemize}
\begin{itemize}
\item {Proveniência:(De \textunderscore marimbo\textunderscore )}
\end{itemize}
Ganhar o jôgo do marimbo.
Burlar; enganar.
Não fazer caso.
\section{Marimbau}
\begin{itemize}
\item {Grp. gram.:m.}
\end{itemize}
\begin{itemize}
\item {Utilização:Bras}
\end{itemize}
Peixe marítimo, provavelmente o mesmo que \textunderscore marimbá\textunderscore .
\section{Marimbo}
\begin{itemize}
\item {Grp. gram.:m.}
\end{itemize}
Variedade de jôgo de cartas.
\section{Marimbondo}
\begin{itemize}
\item {Grp. gram.:m.}
\end{itemize}
\begin{itemize}
\item {Utilização:Bras. do N}
\end{itemize}
O mesmo que \textunderscore maribondo\textunderscore .
\section{Marimboque}
\begin{itemize}
\item {Grp. gram.:m.}
\end{itemize}
Árvore ornamental da ilha de San-Thomé.
\section{Marimonda}
\begin{itemize}
\item {Grp. gram.:m.}
\end{itemize}
Macaco, do gênero atele.
\section{Marinas}
\begin{itemize}
\item {Grp. gram.:f. pl.}
\end{itemize}
\begin{itemize}
\item {Proveniência:(Lat. \textunderscore marinus\textunderscore )}
\end{itemize}
As plantas, que nascem e vivem no mar.
\section{Marinha}
\begin{itemize}
\item {Grp. gram.:f.}
\end{itemize}
\begin{itemize}
\item {Utilização:T. de Aveiro}
\end{itemize}
\begin{itemize}
\item {Proveniência:(Do lat. \textunderscore marinus\textunderscore )}
\end{itemize}
Beiramar.
Serviço de marinheiros.
Fôrças navaes, ou navios de guerra, com a sua tripulação e armamento.
Conjunto de navios.
Lugar, em que se recolhe a água do mar, para o fabrico do sal.
Salina.
Desenho ou quadro que representa objectos ou scenas marítimas.
Peixe osteodermo da costa da África.
O mesmo que \textunderscore marinhada\textunderscore .
\section{Marinhada}
\begin{itemize}
\item {Grp. gram.:f.}
\end{itemize}
\begin{itemize}
\item {Utilização:T. de Aveiro}
\end{itemize}
\begin{itemize}
\item {Proveniência:(De \textunderscore marinha\textunderscore )}
\end{itemize}
Ração de um quartilho de vinho a cada pescador das rêdes do cêrco, por cada vez que vai ao mar.
\section{Marinhagem}
\begin{itemize}
\item {Grp. gram.:f.}
\end{itemize}
\begin{itemize}
\item {Proveniência:(De \textunderscore marinhar\textunderscore )}
\end{itemize}
Conjunto de marinheiros.
Arte de navegar.
\section{Marinha-nova}
\begin{itemize}
\item {Grp. gram.:f.}
\end{itemize}
\begin{itemize}
\item {Utilização:Marn.}
\end{itemize}
Parte da marinha ou da salina própriamente dita, comprehendendo os meios das duas filas superiores.
\section{Marinhão}
\begin{itemize}
\item {Grp. gram.:adj.}
\end{itemize}
\begin{itemize}
\item {Proveniência:(De \textunderscore marinha\textunderscore )}
\end{itemize}
Criado no littoral, (falando-se de gado bovino). Cf. A. Baganha, \textunderscore Hyg. Pec.\textunderscore , 207.
Pescador da Murtosa, empregado de ordinário nas campanhas de San-Jacinto, junto da ria de Aveiro. Cf. Rev. \textunderscore Tradição\textunderscore , v, 10.
\section{Marinha-podre}
\begin{itemize}
\item {Grp. gram.:f.}
\end{itemize}
\begin{itemize}
\item {Utilização:Marn.}
\end{itemize}
Marinha ou salina em que nasce água.
\section{Marinhar}
\begin{itemize}
\item {Grp. gram.:v. t.}
\end{itemize}
\begin{itemize}
\item {Grp. gram.:V. i.}
\end{itemize}
\begin{itemize}
\item {Proveniência:(De \textunderscore marinha\textunderscore )}
\end{itemize}
Prover de pessoal náutico.
Governar a manobra de (navios).
Saber navegar.
Trepar: o \textunderscore galo marinhou pela parede acima\textunderscore .
\section{Marinharesco}
\begin{itemize}
\item {fónica:nharês}
\end{itemize}
\begin{itemize}
\item {Grp. gram.:adj.}
\end{itemize}
O mesmo que \textunderscore marinheiresco\textunderscore .
\section{Marinharia}
\begin{itemize}
\item {Grp. gram.:f.}
\end{itemize}
O mesmo que \textunderscore marinhagem\textunderscore .
Arte ou profissão de marinheiro; marinheiraria. Cf. Sousa Viterbo, \textunderscore Trabalhos Náuticos\textunderscore .
\section{Marinhaticamente}
\begin{itemize}
\item {Grp. gram.:adv.}
\end{itemize}
De modo marinhático.
Á maneira de marinheiro.
\section{Marinhático}
\begin{itemize}
\item {Grp. gram.:adj.}
\end{itemize}
\begin{itemize}
\item {Utilização:Ant.}
\end{itemize}
O mesmo que \textunderscore marinheiresco\textunderscore .
\section{Marinha-velha}
\begin{itemize}
\item {Grp. gram.:f.}
\end{itemize}
\begin{itemize}
\item {Utilização:Marn.}
\end{itemize}
Parte da marinha ou da salina propriamente dita, comprehendendo os meios das duas filas inferiores.
\section{Marinheiraria}
\begin{itemize}
\item {Grp. gram.:f.}
\end{itemize}
\begin{itemize}
\item {Proveniência:(De \textunderscore marinheiro\textunderscore )}
\end{itemize}
Parte prática da arte de navegar.
\section{Marinheiraz}
\begin{itemize}
\item {Grp. gram.:m.}
\end{itemize}
\begin{itemize}
\item {Utilização:ant.}
\end{itemize}
\begin{itemize}
\item {Utilização:Burl.}
\end{itemize}
Marinheiro. Cf. Sim. Mach., fol. 28.
\section{Marinheiresco}
\begin{itemize}
\item {fónica:nheirês}
\end{itemize}
\begin{itemize}
\item {Grp. gram.:adj.}
\end{itemize}
Relativo a marinheiro; próprio de marinheiro.
\section{Marinheiro}
\begin{itemize}
\item {Grp. gram.:m.}
\end{itemize}
\begin{itemize}
\item {Utilização:Bot.}
\end{itemize}
\begin{itemize}
\item {Utilização:Zool.}
\end{itemize}
\begin{itemize}
\item {Utilização:Prov.}
\end{itemize}
\begin{itemize}
\item {Utilização:Bras}
\end{itemize}
\begin{itemize}
\item {Utilização:pop.}
\end{itemize}
\begin{itemize}
\item {Grp. gram.:Adj.}
\end{itemize}
\begin{itemize}
\item {Proveniência:(Do b. lat. \textunderscore marinarius\textunderscore )}
\end{itemize}
Aquelle que dirige uma embarcação.
Aquelle que trabalha a bordo; aquelle que executa trabalhos náuticos.
Gênero de plantas meliáceas do Brasil.
Espécie de camarão brasileiro, que marinha pelos mangues.
O mesmo que \textunderscore pica-peixe\textunderscore , ave.
Português; homem branco.
Relativo á marinhagem: \textunderscore vida marinheira\textunderscore .
\section{Marinhesco}
\begin{itemize}
\item {fónica:nhês}
\end{itemize}
\begin{itemize}
\item {Grp. gram.:adj.}
\end{itemize}
Próprio da marinhagem.
(Cast. \textunderscore mariñesco\textunderscore ).
\section{Marinhista}
\begin{itemize}
\item {Grp. gram.:m.}
\end{itemize}
\begin{itemize}
\item {Utilização:Neol.}
\end{itemize}
Pintor de marinhas. Cf. \textunderscore Jorn. do Comm.\textunderscore , do Rio, de 18-X-900.
\section{Marinho}
\begin{itemize}
\item {Grp. gram.:adj.}
\end{itemize}
\begin{itemize}
\item {Proveniência:(Do lat. \textunderscore marinus\textunderscore )}
\end{itemize}
Relativo ao mar: \textunderscore águas marinhas\textunderscore .
Marítimo.
Que existe no mar; que procede do mar; produzido pelo mar: \textunderscore plantas marinhas\textunderscore .
\section{Marinhoto}
\begin{itemize}
\item {fónica:nhó}
\end{itemize}
\begin{itemize}
\item {Grp. gram.:m.  e  adj.}
\end{itemize}
\begin{itemize}
\item {Utilização:T. da Bairrada}
\end{itemize}
Marinheiro das águas costeiras; marinheiro.
O mesmo que \textunderscore marinhão\textunderscore , (gado).
\section{Marinim}
\begin{itemize}
\item {Grp. gram.:m.}
\end{itemize}
\begin{itemize}
\item {Utilização:Bras. do N}
\end{itemize}
Espécie de pequeno mosquito.
\section{Marinismo}
\begin{itemize}
\item {Grp. gram.:m.}
\end{itemize}
\begin{itemize}
\item {Proveniência:(De \textunderscore Marini\textunderscore , n. p.)}
\end{itemize}
Estilo affectado e de mau gôsto, adoptado em Itália pelo poéta Marini, no século XVII, e correspondente ao gongorismo em
Espanha.
\section{Marinista}
\begin{itemize}
\item {Grp. gram.:m.}
\end{itemize}
Sectário do marinísmo.
(Cp. \textunderscore marinismo\textunderscore )
\section{Marino}
\begin{itemize}
\item {Grp. gram.:adj.}
\end{itemize}
O mesmo que \textunderscore marinho\textunderscore .
\section{Marioila}
\begin{itemize}
\item {Grp. gram.:f.}
\end{itemize}
Planta labiada, (\textunderscore thlomis purpurea\textunderscore ).
\section{Mariola}
\begin{itemize}
\item {Grp. gram.:m.}
\end{itemize}
\begin{itemize}
\item {Grp. gram.:Adj.}
\end{itemize}
Moço de fretes.
Homem de recados.
Patife, biltre.
Variedade de pombos, de cabeça pequena.
Que tem mau carácter; infame.
\section{Mariola}
\begin{itemize}
\item {Grp. gram.:f.}
\end{itemize}
\begin{itemize}
\item {Utilização:Prov.}
\end{itemize}
\begin{itemize}
\item {Utilização:minh.}
\end{itemize}
Conjunto de três pedras sobrepostas, que nas serras ínvias indica a direcção da marcha.
\section{Mariolada}
\begin{itemize}
\item {Grp. gram.:f.}
\end{itemize}
Acção ou dito de mariola^1.
\section{Mariolagem}
\begin{itemize}
\item {Grp. gram.:f.}
\end{itemize}
\begin{itemize}
\item {Proveniência:(De \textunderscore mariola\textunderscore ^1)}
\end{itemize}
Acto de mariola; os mariolas.
\section{Mariolar}
\begin{itemize}
\item {Grp. gram.:v. i.}
\end{itemize}
Fazer serviço de mariola.
Têr vida de malandro; vadiar.
\section{Maripa}
\begin{itemize}
\item {Grp. gram.:f.}
\end{itemize}
Palmeira da Guiana.
Planta convolvulácea.
\section{Mariposa}
\begin{itemize}
\item {Grp. gram.:f.}
\end{itemize}
O mesmo que \textunderscore borboleta\textunderscore .
Jóia ou ornato, do feitio de borboleta.--Filinto, I, 217, considera masculina a palavra:«\textunderscore ...esses alados mariposas...\textunderscore »(Cast. \textunderscore mariposa\textunderscore )
\section{Mariposear}
\begin{itemize}
\item {Grp. gram.:v. i.}
\end{itemize}
\begin{itemize}
\item {Utilização:Neol.}
\end{itemize}
O mesmo que \textunderscore borboletear\textunderscore .
(Cast. \textunderscore mariposear\textunderscore )
\section{Mariquice}
\begin{itemize}
\item {Grp. gram.:f.}
\end{itemize}
Qualidade ou acto de maricas. Cf. Eça, \textunderscore P. Basilio\textunderscore , 542.
\section{Mariquina}
\begin{itemize}
\item {Grp. gram.:m.}
\end{itemize}
O mesmo que \textunderscore acarima\textunderscore .
\section{Mariquinhas}
\begin{itemize}
\item {Grp. gram.:f.}
\end{itemize}
Nome, que, em Caminha, se dá ao malmequér branco.
\section{Mariquita}
\begin{itemize}
\item {Grp. gram.:f.}
\end{itemize}
\begin{itemize}
\item {Utilização:Prov.}
\end{itemize}
\begin{itemize}
\item {Utilização:alg.}
\end{itemize}
Sardinha pequena como fôlhas de oliveira.
\section{Maririçó}
\begin{itemize}
\item {Grp. gram.:m.}
\end{itemize}
Planta irídea do Brasil, purgativa.
\section{Marisca}
\begin{itemize}
\item {Grp. gram.:f.  e  adj.}
\end{itemize}
\begin{itemize}
\item {Utilização:Pesc.}
\end{itemize}
\begin{itemize}
\item {Proveniência:(De \textunderscore marisco\textunderscore )}
\end{itemize}
Diz-se de uma truta de água salgada.
\section{Mariscal}
\begin{itemize}
\item {Grp. gram.:m.}
\end{itemize}
\begin{itemize}
\item {Utilização:Ant.}
\end{itemize}
O mesmo que \textunderscore marechal\textunderscore .
(Cast. \textunderscore mariscal\textunderscore ).
\section{Mariscar}
\begin{itemize}
\item {Grp. gram.:v. t.}
\end{itemize}
\begin{itemize}
\item {Grp. gram.:V. i.}
\end{itemize}
\begin{itemize}
\item {Proveniência:(De \textunderscore marisco\textunderscore )}
\end{itemize}
Apanhar (camarão, lagosta ou outros mariscos).
Colhêr mariscos ou insectos, á beira-mar.
\section{Marisco}
\begin{itemize}
\item {Grp. gram.:m.}
\end{itemize}
\begin{itemize}
\item {Grp. gram.:Adj.}
\end{itemize}
Designação genérica de certos crustáceos e molluscos comestíveis.
Diz-se de uma variedade de truta.
(B. lat. \textunderscore mariscus\textunderscore )
\section{Marisco}
\begin{itemize}
\item {Grp. gram.:m.}
\end{itemize}
\begin{itemize}
\item {Utilização:Bras}
\end{itemize}
Espécie de gato bravo.
\section{Marisma}
\begin{itemize}
\item {Grp. gram.:f.}
\end{itemize}
\begin{itemize}
\item {Utilização:Prov.}
\end{itemize}
\begin{itemize}
\item {Utilização:alg.}
\end{itemize}
Sapal ou terreno alagadiço, á beiramar.
Espécie de alga, produzida nesse terreno e que serve para alimento de animaes.
(Alter de \textunderscore marítima\textunderscore )
\section{Marisqueira}
\begin{itemize}
\item {Grp. gram.:f.}
\end{itemize}
Mulher, que vende marisco:«já pregôa o açacal e já canta a \textunderscore marisqueira\textunderscore ». J. Castilho, \textunderscore Manuelinas\textunderscore , 80.
\section{Marisqueiro}
\begin{itemize}
\item {Grp. gram.:m.  e  adj.}
\end{itemize}
\begin{itemize}
\item {Grp. gram.:M.}
\end{itemize}
\begin{itemize}
\item {Utilização:Prov.}
\end{itemize}
\begin{itemize}
\item {Proveniência:(De \textunderscore mariscar\textunderscore )}
\end{itemize}
Aquelle que marisca.
O mesmo que \textunderscore pica-peixe\textunderscore , ave.
\section{Maristas}
\begin{itemize}
\item {Grp. gram.:m. pl.}
\end{itemize}
O mesmo que \textunderscore marianitas\textunderscore .
\section{Marita}
\begin{itemize}
\item {Grp. gram.:f.}
\end{itemize}
\begin{itemize}
\item {Utilização:Des.}
\end{itemize}
\begin{itemize}
\item {Proveniência:(Lat. \textunderscore marita\textunderscore )}
\end{itemize}
Mulher casada; espôsa.
\section{Maritaca}
\begin{itemize}
\item {Grp. gram.:f.}
\end{itemize}
\begin{itemize}
\item {Utilização:Bras. do Rio}
\end{itemize}
Ave, da côr do periquito, mas um pouco maior.
\section{Maritacaca}
\begin{itemize}
\item {Grp. gram.:f.}
\end{itemize}
\begin{itemize}
\item {Utilização:Bras}
\end{itemize}
Pequeno mammífero carnívoro.
\section{Maritágio}
\begin{itemize}
\item {Grp. gram.:m.}
\end{itemize}
Dote ou doação, feita na Idade-Média pelo pai á filha que ia casar. Cf. Herculano, \textunderscore Bobo\textunderscore , 239.
(B. lat. \textunderscore maritagium\textunderscore )
\section{Marital}
\begin{itemize}
\item {Grp. gram.:adj.}
\end{itemize}
\begin{itemize}
\item {Proveniência:(Lat. \textunderscore maritalis\textunderscore )}
\end{itemize}
Relativo a marido ou a matrimonio; conjugal.
\section{Maritalmente}
\begin{itemize}
\item {Grp. gram.:adv.}
\end{itemize}
De modo marital; á maneira de casados: \textunderscore viver maritalmente\textunderscore .
\section{Marítimo}
\begin{itemize}
\item {Grp. gram.:adj.}
\end{itemize}
\begin{itemize}
\item {Grp. gram.:M.}
\end{itemize}
\begin{itemize}
\item {Proveniência:(Lat. \textunderscore maritimus\textunderscore )}
\end{itemize}
Relativo ao mar: \textunderscore vida marítima\textunderscore .
Situado á beira-mar: \textunderscore costa marítima\textunderscore .
Que vive no mar ou á beira-mar.
Que se dedica á navegação: \textunderscore gente marítima\textunderscore .
Naval.
O mesmo que \textunderscore marinheiro\textunderscore .
\section{Marlota}
\begin{itemize}
\item {Grp. gram.:f.}
\end{itemize}
Espécie de capote curto com capuz, em uso entre os Moiros.
(Ár. \textunderscore malota\textunderscore )
\section{Marlotar}
\begin{itemize}
\item {Grp. gram.:v. t.}
\end{itemize}
\begin{itemize}
\item {Proveniência:(De \textunderscore marlota\textunderscore )}
\end{itemize}
Dar aspecto rugoso a.
Amarrotar; enxovalhar.
\section{Marma}
\begin{itemize}
\item {Grp. gram.:f.}
\end{itemize}
\begin{itemize}
\item {Proveniência:(Do cast. \textunderscore merma\textunderscore )}
\end{itemize}
Chapa lisa de ferro, com que se arredonda o vidro nas fábricas.
\section{Marmaco}
\begin{itemize}
\item {Grp. gram.:adj.}
\end{itemize}
\begin{itemize}
\item {Utilização:Prov.}
\end{itemize}
\begin{itemize}
\item {Utilização:alg.}
\end{itemize}
Quente, abafadiço, (falando-se do tempo).
\section{Marmajuda}
\begin{itemize}
\item {Grp. gram.:f.}
\end{itemize}
Planta flacurtiácea, (\textunderscore bixa alagoana\textunderscore ).
\section{Marmanjão}
\begin{itemize}
\item {Grp. gram.:m.}
\end{itemize}
Grande marmanjo; patifório; velhaco. Cf. Júl. Dinis, \textunderscore Morgadinha\textunderscore , 24.
\section{Marmanjaria}
\begin{itemize}
\item {Grp. gram.:f.}
\end{itemize}
Qualidade de marmanjo.
Súcia de marmanjos.
Os marmanjos. Cf. Camillo, \textunderscore Noites de Imomn\textunderscore ., IV, 39,
\section{Marmanjo}
\begin{itemize}
\item {Grp. gram.:m.  e  adj.}
\end{itemize}
\begin{itemize}
\item {Utilização:Chul.}
\end{itemize}
Mariola, tratante; bruto.
\section{Marmar}
\begin{itemize}
\item {Grp. gram.:v. i.}
\end{itemize}
\begin{itemize}
\item {Proveniência:(Do cast. \textunderscore mermar\textunderscore )}
\end{itemize}
Minguar, decrescer.
\section{Marmárico}
\begin{itemize}
\item {Grp. gram.:adj.}
\end{itemize}
Relativo ao mar de Mármara.
\section{Marmasso}
\begin{itemize}
\item {Grp. gram.:adj.}
\end{itemize}
\begin{itemize}
\item {Utilização:Prov.}
\end{itemize}
\begin{itemize}
\item {Utilização:alg.}
\end{itemize}
Quente, abafadiço, (falando-se do tempo).
\section{Marmela}
\begin{itemize}
\item {Grp. gram.:f.}
\end{itemize}
\begin{itemize}
\item {Proveniência:(De \textunderscore marmelo\textunderscore )}
\end{itemize}
Nome, que se dá em Lisbôa á pêra flamenga.
Nome de outra pêra, de pelle menos avermelhada, e que se suppõe produzida por hybridação da pereira com o marmeleiro.
\section{Marmelada}
\begin{itemize}
\item {Grp. gram.:f.}
\end{itemize}
\begin{itemize}
\item {Utilização:Pop.}
\end{itemize}
\begin{itemize}
\item {Utilização:Bras}
\end{itemize}
Doce de marmelo ralado e misturado com calda de açúcar.
Vantagem; pechincha.
Fruto da marmeladeira.
\section{Marmela-de-inverno}
\begin{itemize}
\item {Grp. gram.:f.}
\end{itemize}
Espécie de marmela.
\section{Marmeladeira}
\begin{itemize}
\item {Grp. gram.:f.}
\end{itemize}
\begin{itemize}
\item {Utilização:Bras}
\end{itemize}
Árvore fructifera.
O mesmo que \textunderscore marmeleiro\textunderscore ? Cf. B. C. Rubim. \textunderscore Vocab. Bras.\textunderscore 
\section{Marmela-de-verao}
\begin{itemize}
\item {Grp. gram.:f.}
\end{itemize}
Espécie de marmela.
\section{Marmeleira-da-Índia}
\begin{itemize}
\item {Grp. gram.:f.}
\end{itemize}
Árvore rutácea, de fruto medicinal, (\textunderscore aegle marmelos\textunderscore , Correia).
\section{Marmeleiro}
\begin{itemize}
\item {Grp. gram.:m.}
\end{itemize}
\begin{itemize}
\item {Proveniência:(De \textunderscore marmelo\textunderscore )}
\end{itemize}
Árvore rosácea, de fruto ácido e adstringente.
Varapau, feito da haste de marmeleiro: \textunderscore apanhou uma sova de marmeleiro\textunderscore .
\section{Marmelo}
\begin{itemize}
\item {Grp. gram.:m.}
\end{itemize}
\begin{itemize}
\item {Utilização:Pop.}
\end{itemize}
\begin{itemize}
\item {Utilização:T. do Faial}
\end{itemize}
\begin{itemize}
\item {Proveniência:(Do lat. \textunderscore melimelum\textunderscore )}
\end{itemize}
Fruto do marmeleiro.
Marmeleiro.
Marmanjo, velhaco.
O mesmo que \textunderscore bebedeira\textunderscore .
\section{Marmelos-de-bengala}
\begin{itemize}
\item {Grp. gram.:m.}
\end{itemize}
O mesmo que \textunderscore marmeleira-da-índia\textunderscore .
\section{Marmeluta}
\begin{itemize}
\item {Grp. gram.:f.}
\end{itemize}
\begin{itemize}
\item {Utilização:Ant.}
\end{itemize}
O interior do cérebro? Cf. G. Vicente, I, 269; Moraes, \textunderscore Diccion.\textunderscore 
\section{Marmita}
\begin{itemize}
\item {Grp. gram.:f.}
\end{itemize}
\begin{itemize}
\item {Utilização:Gír.}
\end{itemize}
\begin{itemize}
\item {Proveniência:(Do fr. \textunderscore marmite\textunderscore )}
\end{itemize}
Panela de lata ou de outro metal, com tampa.
Rameira, que sustenta um rufião.
\section{Marmo}
\begin{itemize}
\item {Grp. gram.:adj.}
\end{itemize}
\begin{itemize}
\item {Utilização:Bras. do N}
\end{itemize}
Muito grande, (falando-se de coisas inanimadas).
\section{Marmorário}
\begin{itemize}
\item {Grp. gram.:adj.}
\end{itemize}
\begin{itemize}
\item {Grp. gram.:M.}
\end{itemize}
\begin{itemize}
\item {Proveniência:(Lat. marmorarius)}
\end{itemize}
Relativo a mármore; marmóreo.
Aquelle que trabalha em mármore; marmoreiro.
\section{Mármore}
\begin{itemize}
\item {Grp. gram.:m.}
\end{itemize}
\begin{itemize}
\item {Utilização:Fig.}
\end{itemize}
\begin{itemize}
\item {Proveniência:(Lat. \textunderscore marmor\textunderscore )}
\end{itemize}
Pedra calcária, branca ou de côres, e susceptível de polimento.
Aquillo que frio, duro ou branco como o mármore em geral: \textunderscore corações de mármore\textunderscore .
\section{Marmorear}
\begin{itemize}
\item {Grp. gram.:v. t.}
\end{itemize}
\begin{itemize}
\item {Proveniência:(Do lat. \textunderscore marmorare\textunderscore )}
\end{itemize}
Dar aspecto de mármore a.
\section{Marmoreiro}
\begin{itemize}
\item {Grp. gram.:m.}
\end{itemize}
\begin{itemize}
\item {Utilização:Prov.}
\end{itemize}
\begin{itemize}
\item {Proveniência:(Do lat. \textunderscore marmorarius\textunderscore )}
\end{itemize}
Serrador ou polidor de mármore.
Aquelle que faz esculpturas de mármore.
\section{Marmóreo}
\begin{itemize}
\item {Grp. gram.:adj.}
\end{itemize}
\begin{itemize}
\item {Utilização:Fig.}
\end{itemize}
\begin{itemize}
\item {Proveniência:(Lat. \textunderscore marmoreus\textunderscore )}
\end{itemize}
Relativo ou semelhante ao mármore; feito de mármore: \textunderscore uma cruz marmórea\textunderscore .
Que tem a insensibilidade ou brancura do mármore.
\section{Marmórico}
\begin{itemize}
\item {Grp. gram.:adj.}
\end{itemize}
O mesmo que \textunderscore marmóreo\textunderscore . Cf. \textunderscore Viriato Trág.\textunderscore , VIII, 129.
\section{Marmorista}
\begin{itemize}
\item {Grp. gram.:m.}
\end{itemize}
O mesmo que \textunderscore marmoreiro\textunderscore .
\section{Marmorização}
\begin{itemize}
\item {Grp. gram.:f.}
\end{itemize}
\begin{itemize}
\item {Utilização:Med.}
\end{itemize}
\begin{itemize}
\item {Proveniência:(De \textunderscore marmorizar\textunderscore )}
\end{itemize}
Transformação de um mineral em mármore.
Estado pathológico de um órgão, que apresenta veios ou fios na sua superfície externa.
\section{Marmorizar}
\begin{itemize}
\item {Grp. gram.:v. t.}
\end{itemize}
Transformar em mármore: \textunderscore as pedras que a natureza marmoriza...\textunderscore 
\section{Marmoroso}
\begin{itemize}
\item {Grp. gram.:adj.}
\end{itemize}
\begin{itemize}
\item {Proveniência:(Lat. \textunderscore marmorosus\textunderscore )}
\end{itemize}
O mesmo que \textunderscore marmóreo\textunderscore .
\section{Marmota}
\begin{itemize}
\item {Grp. gram.:f.}
\end{itemize}
\begin{itemize}
\item {Utilização:Des.}
\end{itemize}
\begin{itemize}
\item {Proveniência:(Lat. \textunderscore marmota\textunderscore ?)}
\end{itemize}
Pequeno quadrúpede roedor.
Leirão.
Pequena pescada.
Pequena câmara óptica.
\section{Marmoto}
\begin{itemize}
\item {Grp. gram.:adj.}
\end{itemize}
\begin{itemize}
\item {Utilização:Prov.}
\end{itemize}
\begin{itemize}
\item {Utilização:trasm.}
\end{itemize}
Diz-se de uma espécie de castanheiro quási rebordão.
\section{Marmulano}
\begin{itemize}
\item {Grp. gram.:m.}
\end{itemize}
Planta sapotácea, (\textunderscore sideroxilon marmulano\textunderscore , Lowe).
\section{Marna}
\begin{itemize}
\item {Grp. gram.:f.}
\end{itemize}
\begin{itemize}
\item {Utilização:Gal}
\end{itemize}
\begin{itemize}
\item {Proveniência:(Fr. \textunderscore marne\textunderscore )}
\end{itemize}
O mesmo que \textunderscore marga\textunderscore .
\section{Marneco}
\begin{itemize}
\item {Grp. gram.:m.}
\end{itemize}
\begin{itemize}
\item {Utilização:T. de Avis}
\end{itemize}
O mesmo que \textunderscore marreco\textunderscore .
\section{Marnel}
\begin{itemize}
\item {Grp. gram.:m.}
\end{itemize}
Paúl; terreno alagadiço.
(Por \textunderscore marinel\textunderscore , de \textunderscore marino\textunderscore )
\section{Marnetado}
\begin{itemize}
\item {Grp. gram.:adj.}
\end{itemize}
\begin{itemize}
\item {Utilização:Ant.}
\end{itemize}
\begin{itemize}
\item {Proveniência:(De \textunderscore marnete\textunderscore )}
\end{itemize}
Debruado, guarnecido.
\section{Marnete}
\begin{itemize}
\item {fónica:nê}
\end{itemize}
\begin{itemize}
\item {Grp. gram.:m.}
\end{itemize}
\begin{itemize}
\item {Utilização:Ant.}
\end{itemize}
Espécie de debrum ou guarnição.
\section{Marno}
\begin{itemize}
\item {Grp. gram.:m.}
\end{itemize}
\begin{itemize}
\item {Utilização:Gal}
\end{itemize}
O mesmo que \textunderscore marga\textunderscore .
\section{Marnoceiro}
\begin{itemize}
\item {Grp. gram.:m.}
\end{itemize}
Terreno, coberto de água, alagadiço; paúl.
(Cp. \textunderscore marnota\textunderscore )
\section{Marnoso}
\begin{itemize}
\item {Grp. gram.:adj.}
\end{itemize}
\begin{itemize}
\item {Utilização:Gal}
\end{itemize}
(V. \textunderscore margoso\textunderscore ^2)
\section{Marnota}
\begin{itemize}
\item {Grp. gram.:f.}
\end{itemize}
Terreno, que póde sêr alagado pela água do mar ou de um rio.
Parte da salina, em que se junta a água para o fabrico do sal.
(Cp. \textunderscore marnel\textunderscore )
\section{Marnotagem}
\begin{itemize}
\item {Grp. gram.:f.}
\end{itemize}
Offício ou indústria de marnoto.
\section{Marnotal}
\begin{itemize}
\item {Grp. gram.:adj.}
\end{itemize}
Relativo a marnotos ou a salinas: \textunderscore trabalhos marnotaes\textunderscore .
\section{Marnoteiro}
\begin{itemize}
\item {Grp. gram.:m.}
\end{itemize}
\begin{itemize}
\item {Proveniência:(De \textunderscore marnota\textunderscore )}
\end{itemize}
O mesmo que \textunderscore marnoto\textunderscore .
\section{Marnoto}
\begin{itemize}
\item {fónica:nô}
\end{itemize}
\begin{itemize}
\item {Grp. gram.:m.}
\end{itemize}
Aquelle que trabalha nas salinas.
(Cp. \textunderscore marnota\textunderscore )
\section{Marnuacus}
\begin{itemize}
\item {Grp. gram.:m. pl.}
\end{itemize}
Indígenas do norte do Brasil.
\section{Maro}
\begin{itemize}
\item {Grp. gram.:m.}
\end{itemize}
Planta labiada, medicinal, (\textunderscore teucrium marum\textunderscore ).
\section{Maroba}
\begin{itemize}
\item {fónica:marô}
\end{itemize}
\begin{itemize}
\item {Grp. gram.:f.}
\end{itemize}
Espécie de cereja miúda, vermelha e insipida.
\section{Maroiços}
\begin{itemize}
\item {Grp. gram.:m. pl.}
\end{itemize}
\begin{itemize}
\item {Proveniência:(Do rad. de \textunderscore mar\textunderscore )}
\end{itemize}
Ondas encapelladas.
\section{Marola}
\begin{itemize}
\item {Grp. gram.:f.}
\end{itemize}
\begin{itemize}
\item {Utilização:Prov.}
\end{itemize}
\begin{itemize}
\item {Proveniência:(De \textunderscore maré\textunderscore  ou, antes, de \textunderscore mar\textunderscore )}
\end{itemize}
A agitação ordinária da água do mar. Cf. \textunderscore Museu Techn.\textunderscore , 83.
\section{Maroma}
\begin{itemize}
\item {Grp. gram.:f.}
\end{itemize}
\begin{itemize}
\item {Proveniência:(Do ár. \textunderscore mabrom\textunderscore )}
\end{itemize}
Corda grossa.
Corda, em que se apresentam funâmbulos ou arlequins.
\section{Maromaque}
\begin{itemize}
\item {Grp. gram.:m.}
\end{itemize}
Deus dos Indígenas de Timor.
\section{Maromaque}
\begin{itemize}
\item {Grp. gram.:m.}
\end{itemize}
Antigo tecido de seda e oiro. Cf. \textunderscore Port. Mon. Hist.\textunderscore , \textunderscore Script.\textunderscore , 285.
\section{Maromba}
\begin{itemize}
\item {Grp. gram.:f.}
\end{itemize}
\begin{itemize}
\item {Utilização:Fig.}
\end{itemize}
Vara, com que os funâmbulos ou arlequins mantém o equilíbrio na maroma.
Maroma.
Situação ou posição, que com difficuldade se mantém.
(Cp. \textunderscore maroma\textunderscore )
\section{Maromba}
\begin{itemize}
\item {Grp. gram.:f.}
\end{itemize}
Doença das vinhas do Doiro, que impede a fecundação ou a torna irregular.
\section{Maromba}
\begin{itemize}
\item {Grp. gram.:f.}
\end{itemize}
\begin{itemize}
\item {Utilização:Bras}
\end{itemize}
Manada de bois.
\section{Marombado}
\begin{itemize}
\item {Grp. gram.:adj.}
\end{itemize}
Que esta affectado de maromba^2.
\section{Marombar}
\begin{itemize}
\item {Grp. gram.:v. t.}
\end{itemize}
\begin{itemize}
\item {Proveniência:(De \textunderscore maromba\textunderscore ^2)}
\end{itemize}
Transmittir a bactéria da maromba a. Cf. G. Junqueiro, na \textunderscore Gazeta das Ald.\textunderscore 
\section{Marombeiro}
\begin{itemize}
\item {Grp. gram.:adj.}
\end{itemize}
\begin{itemize}
\item {Utilização:Bras}
\end{itemize}
\begin{itemize}
\item {Proveniência:(De \textunderscore maromba\textunderscore ^1)}
\end{itemize}
Que lisonjeia ou adula, com manha ou por interesse.
\section{Marome}
\begin{itemize}
\item {Grp. gram.:m.}
\end{itemize}
Espécie de músico cafreal.
\section{Maronda}
\begin{itemize}
\item {Grp. gram.:f.}
\end{itemize}
\begin{itemize}
\item {Utilização:T. de Miranda}
\end{itemize}
Ovelha, que se leva ao macho.
\section{Maronês}
\begin{itemize}
\item {Grp. gram.:adj.}
\end{itemize}
Relativo ao Marão.
Criado no Marão, (falando-se de gados). Cf. A. Baganha, \textunderscore Hyg. Pec\textunderscore .
\section{Maronitas}
\begin{itemize}
\item {Grp. gram.:m. pl.}
\end{itemize}
\begin{itemize}
\item {Proveniência:(De \textunderscore Maron\textunderscore , n. p.)}
\end{itemize}
Christãos de Líbano, sujeitos á Turquia.
\section{Marosca}
\begin{itemize}
\item {Grp. gram.:f.}
\end{itemize}
\begin{itemize}
\item {Utilização:Pop.}
\end{itemize}
Enredo; ardil, trapaça.
\section{Marotagem}
\begin{itemize}
\item {Grp. gram.:f.}
\end{itemize}
Maroteira.
Reunião de marotos.
\section{Marotamente}
\begin{itemize}
\item {Grp. gram.:adv.}
\end{itemize}
De modo maroto. Cf. Camillo, \textunderscore Caveira\textunderscore , 63.
\section{Marotear}
\begin{itemize}
\item {Grp. gram.:v. i.}
\end{itemize}
Têr vida de maroto.
\section{Maroteira}
\begin{itemize}
\item {Grp. gram.:f.}
\end{itemize}
Acto próprio de maroto.
Qualidade de quem é maroto.
\section{Marotinho}
\begin{itemize}
\item {Grp. gram.:m.}
\end{itemize}
\begin{itemize}
\item {Utilização:Prov.}
\end{itemize}
Lenço ordinário.
\section{Maroto}
\begin{itemize}
\item {fónica:marô}
\end{itemize}
\begin{itemize}
\item {Grp. gram.:adj.}
\end{itemize}
\begin{itemize}
\item {Grp. gram.:M.}
\end{itemize}
Malicioso; brejeiro.
Lascivo.
Marau, patife.
Homem de baixa extracção.
Casta de uva preta, o mesma que \textunderscore bom-vedro\textunderscore .
Casta de uva branca.
(Cp. fr. \textunderscore maraud\textunderscore )
\section{Marouco}
\begin{itemize}
\item {Grp. gram.:m.}
\end{itemize}
\begin{itemize}
\item {Utilização:Prov.}
\end{itemize}
\begin{itemize}
\item {Utilização:alent.}
\end{itemize}
\begin{itemize}
\item {Proveniência:(Do cast. \textunderscore marueco\textunderscore )}
\end{itemize}
Carneiro velho, pai de manada.
Carneiro, que não é castrado ou que serve para padreação.
\section{Maroufa}
\begin{itemize}
\item {Grp. gram.:f.}
\end{itemize}
\begin{itemize}
\item {Utilização:Prov.}
\end{itemize}
\begin{itemize}
\item {Utilização:trasm.}
\end{itemize}
O mesmo que \textunderscore marouva\textunderscore .
\section{Marouva}
\begin{itemize}
\item {Grp. gram.:f.}
\end{itemize}
Espécie de cereja, o mesmo que \textunderscore maroba\textunderscore .
\section{Marouvaz}
\begin{itemize}
\item {Grp. gram.:m.}
\end{itemize}
\begin{itemize}
\item {Utilização:Prov.}
\end{itemize}
\begin{itemize}
\item {Utilização:alg.}
\end{itemize}
Mariola; tratante.
(Cp. \textunderscore marau\textunderscore )
\section{Marpésio}
\begin{itemize}
\item {Grp. gram.:adj.}
\end{itemize}
\begin{itemize}
\item {Proveniência:(Lat. \textunderscore marpesius\textunderscore )}
\end{itemize}
Relativo ao Marpesso, monte da ilha de Paros. Cf. Filinto, VIII, 47.
\section{Marquês}
\begin{itemize}
\item {Grp. gram.:m.}
\end{itemize}
\begin{itemize}
\item {Proveniência:(Do b. lat. \textunderscore marchensis\textunderscore )}
\end{itemize}
Senhôr, que, antigamente, commandava a guarda das marcas ou fronteiras de um Estado.
Actualmente, título nobiliárchico, immediatamente inferior ao de duque e superior ao de conde.
Casta de uva preta do Minho.
\section{Marquesa}
\begin{itemize}
\item {Grp. gram.:f.}
\end{itemize}
\begin{itemize}
\item {Grp. gram.:Adj.}
\end{itemize}
\begin{itemize}
\item {Proveniência:(De \textunderscore marquês\textunderscore )}
\end{itemize}
Mulher de marquês.
Senhora, que tem marquesado.
Espécie de canapé largo, com assento de palhinha.
Alpendre, que cobre a plataforma, nas estações dos caminhos de ferro.
Pequena construcção envidraçada, annexa ou próxima de maior edificio, e que, geralmente, deita para um jardim.
Diz-se de uma variedade do pêra muito sumarenta e apreciada.
\section{Marquesado}
\begin{itemize}
\item {Grp. gram.:m.}
\end{itemize}
Cargo ou dignidade de marquês.
Terras, que constituíam o domínio e solar de um marquês ou marquesa.
\section{Marquesinha}
\begin{itemize}
\item {Grp. gram.:f.}
\end{itemize}
Pequeno guarda-sol, que as senhoras usavam.
Tôldo, que abriga a tenda dos officiaes, em campanha.
Espécie de alpendre, que faz parto de algumas estações dos caminhos de ferro.
Planta liliácea.
Casta de uva branca do Minho.
Casta de pêra, o mesma que \textunderscore marquesa\textunderscore .
(Cast. \textunderscore marquesina\textunderscore )
\section{Marquesinha-branca-de-inverno}
\begin{itemize}
\item {Grp. gram.:f.}
\end{itemize}
O mesmo que \textunderscore pêra-lemos\textunderscore .
\section{Marquesinho}
\begin{itemize}
\item {Grp. gram.:adj.}
\end{itemize}
\begin{itemize}
\item {Proveniência:(De \textunderscore marquês\textunderscore )}
\end{itemize}
Diz-se de certos palitos, próprios para esgaravatar os dentes.
\section{Marques-loureiro}
\begin{itemize}
\item {Grp. gram.:m.}
\end{itemize}
Espécie de ameixa grande, oval e amarelada.
Variedade de pêssegos saborosos e temporãos.
\section{Marquesota}
\begin{itemize}
\item {Grp. gram.:f.}
\end{itemize}
\begin{itemize}
\item {Utilização:Ant.}
\end{itemize}
Espécie de túbera indiana.
Mantéu, com que se abafava o pescoço.
(Cast. \textunderscore marquesota\textunderscore )
\section{Marquezita}
\begin{itemize}
\item {Grp. gram.:f.}
\end{itemize}
\begin{itemize}
\item {Utilização:Des.}
\end{itemize}
O mesmo que \textunderscore marcassita\textunderscore .
\section{Marra}
\begin{itemize}
\item {Grp. gram.:f.}
\end{itemize}
\begin{itemize}
\item {Utilização:Prov.}
\end{itemize}
\begin{itemize}
\item {Utilização:minh.}
\end{itemize}
\begin{itemize}
\item {Utilização:Prov.}
\end{itemize}
\begin{itemize}
\item {Utilização:beir.}
\end{itemize}
\begin{itemize}
\item {Proveniência:(Lat. \textunderscore marra\textunderscore )}
\end{itemize}
Sacho para mondar.
Rêgo ou valeta, ao lado do caminho.
Clareira, em vinhedos ou olivaes.
Espécie de jôgo de rapazes.
Parte do instrumento cortante, opposta ao gume.
O mesmo que \textunderscore marrão\textunderscore ^2.
Pedra, em que a péla vai bater ou marrar, no respectivo jôgo.
\section{Marrã}
\begin{itemize}
\item {Grp. gram.:f.}
\end{itemize}
\begin{itemize}
\item {Utilização:Bras. do N}
\end{itemize}
\begin{itemize}
\item {Proveniência:(Do cast. \textunderscore marrana\textunderscore )}
\end{itemize}
Porca nova, que deixou do mamar.
Carne fresca de porco.
Ovelha pequena.
\section{Marrã}
\begin{itemize}
\item {Grp. gram.:f.}
\end{itemize}
\begin{itemize}
\item {Utilização:Prov.}
\end{itemize}
Carcunda, corcova.
(Cp. \textunderscore marranica\textunderscore )
\section{Marraco}
\begin{itemize}
\item {Grp. gram.:m.}
\end{itemize}
\begin{itemize}
\item {Proveniência:(De \textunderscore marra\textunderscore )}
\end{itemize}
O mesmo que \textunderscore enxadão\textunderscore .
\section{Marrada}
\begin{itemize}
\item {Grp. gram.:f.}
\end{itemize}
\begin{itemize}
\item {Proveniência:(Do cast. \textunderscore morrada\textunderscore )}
\end{itemize}
Acto de marrar.
\section{Marrada}
\begin{itemize}
\item {Grp. gram.:f.}
\end{itemize}
\begin{itemize}
\item {Utilização:Prov.}
\end{itemize}
\begin{itemize}
\item {Utilização:trasm.}
\end{itemize}
Pedaço de terra em cru, mas coberta de leiva, que o lavrador deixa na arada.
(Cp. \textunderscore marra\textunderscore )
\section{Marrado}
\begin{itemize}
\item {Grp. gram.:adj.}
\end{itemize}
\begin{itemize}
\item {Proveniência:(De \textunderscore marrar\textunderscore )}
\end{itemize}
Diz-se do vinho, que se tolda na vasilha, tornando-se impotável.
\section{Marrafa}
\begin{itemize}
\item {Grp. gram.:f.}
\end{itemize}
\begin{itemize}
\item {Proveniência:(De \textunderscore Maraffi\textunderscore , n. p. de um bailarino it., que viveu em Lisbôa no século XVIII)}
\end{itemize}
Parte do cabello, riçada e caída sobre a testa.
Cada uma das duas partes em que, por meio de uma risca longitudinal se divide o cabello.
\section{Marrafão}
\begin{itemize}
\item {Grp. gram.:adj.}
\end{itemize}
\begin{itemize}
\item {Utilização:Des.}
\end{itemize}
Diz-se do tabaco ordinário.
(Provavelmente, da mesma or. que \textunderscore marrafão\textunderscore ^2, por allusão ao tabaco usado por fadistas)
\section{Marrafão}
\begin{itemize}
\item {Grp. gram.:m.}
\end{itemize}
Fadista, que usa marrafa.
\section{Marrafona}
\begin{itemize}
\item {Grp. gram.:m.}
\end{itemize}
\begin{itemize}
\item {Utilização:T. de Mogofores}
\end{itemize}
O mesmo que \textunderscore maria-da-grade\textunderscore .
\section{Marralhão}
\begin{itemize}
\item {Grp. gram.:m.}
\end{itemize}
\begin{itemize}
\item {Utilização:Pop.}
\end{itemize}
\begin{itemize}
\item {Utilização:T. de Pare -de-Coira}
\end{itemize}
\begin{itemize}
\item {Utilização:des.}
\end{itemize}
Indolente; bonacheirão.
Aquelle que regateia.
(Cp. \textunderscore marralheiro\textunderscore )
\section{Marralhar}
\begin{itemize}
\item {Grp. gram.:v. i.}
\end{itemize}
\begin{itemize}
\item {Utilização:T. de Pare -de-Coira}
\end{itemize}
\begin{itemize}
\item {Utilização:des.}
\end{itemize}
\begin{itemize}
\item {Proveniência:(Do rad. de \textunderscore marralheiro\textunderscore )}
\end{itemize}
Insistir, procurar persuadir alguém.
Sêr marralheiro.
Regatear no preço.
\section{Marralharia}
\begin{itemize}
\item {Grp. gram.:f.}
\end{itemize}
O mesmo que \textunderscore marralhice\textunderscore .
\section{Marralheiro}
\begin{itemize}
\item {Grp. gram.:adj.}
\end{itemize}
\begin{itemize}
\item {Utilização:Prov.}
\end{itemize}
\begin{itemize}
\item {Utilização:Pop.}
\end{itemize}
Que marralha; manhoso.
Que emprega astúcias para convencer ou illudir.
Mau pagador.
Preguiçoso, cábula.
(Cast. \textunderscore marrallero\textunderscore )
\section{Marralhice}
\begin{itemize}
\item {Grp. gram.:f.}
\end{itemize}
\begin{itemize}
\item {Proveniência:(Do rad. de \textunderscore marralheiro\textunderscore )}
\end{itemize}
Astúcia; manha.
Mandriice.
\section{Marran}
\begin{itemize}
\item {Grp. gram.:f.}
\end{itemize}
\begin{itemize}
\item {Utilização:Bras. do N}
\end{itemize}
\begin{itemize}
\item {Proveniência:(Do cast. \textunderscore marrana\textunderscore )}
\end{itemize}
Porca nova, que deixou do mamar.
Carne fresca de porco.
Ovelha pequena.
\section{Marran}
\begin{itemize}
\item {Grp. gram.:f.}
\end{itemize}
\begin{itemize}
\item {Utilização:Prov.}
\end{itemize}
Carcunda, corcova.
(Cp. \textunderscore marranica\textunderscore )
\section{Marrana}
\begin{itemize}
\item {Grp. gram.:f.}
\end{itemize}
\begin{itemize}
\item {Utilização:Prov.}
\end{itemize}
\begin{itemize}
\item {Utilização:beir.}
\end{itemize}
O mesmo que \textunderscore marran\textunderscore ^1.
\section{Marrancha}
\begin{itemize}
\item {Grp. gram.:f.}
\end{itemize}
\begin{itemize}
\item {Utilização:Prov.}
\end{itemize}
\begin{itemize}
\item {Utilização:beir.}
\end{itemize}
\begin{itemize}
\item {Grp. gram.:M.}
\end{itemize}
Carcunda.
Aquelle que tem carcunda ou é corcovado.
(Cp. \textunderscore marran\textunderscore ^2)
\section{Marrancho}
\textunderscore m. Prov. trasm.\textunderscore  O mesmo que \textunderscore porco\textunderscore .
(Cp. \textunderscore marrão\textunderscore ^1)
\section{Marrancho}
\begin{itemize}
\item {Grp. gram.:m.}
\end{itemize}
\begin{itemize}
\item {Utilização:Prov.}
\end{itemize}
\begin{itemize}
\item {Utilização:beir.}
\end{itemize}
Namorado, conversado.
\section{Marraneiro}
\begin{itemize}
\item {Grp. gram.:m.}
\end{itemize}
\begin{itemize}
\item {Utilização:Prov.}
\end{itemize}
Aquelle que vende marran.
(Do \textunderscore marran\textunderscore ^1)
\section{Marranho}
\begin{itemize}
\item {Grp. gram.:m.}
\end{itemize}
\begin{itemize}
\item {Utilização:Ant.}
\end{itemize}
O mesmo que \textunderscore marrano\textunderscore .
\section{Marranica}
\begin{itemize}
\item {Grp. gram.:f.}
\end{itemize}
\begin{itemize}
\item {Utilização:Prov.}
\end{itemize}
\begin{itemize}
\item {Utilização:beir.}
\end{itemize}
\begin{itemize}
\item {Grp. gram.:M.}
\end{itemize}
\begin{itemize}
\item {Proveniência:(De \textunderscore marran\textunderscore ^2)}
\end{itemize}
Carcunda.
Indivíduo gibboso.
\section{Marranita}
\begin{itemize}
\item {Grp. gram.:m.}
\end{itemize}
\begin{itemize}
\item {Utilização:Prov.}
\end{itemize}
\begin{itemize}
\item {Proveniência:(De \textunderscore marran\textunderscore ^2)}
\end{itemize}
Indivíduo, que é carcunda.
\section{Marrano}
\begin{itemize}
\item {Grp. gram.:m.  e  adj.}
\end{itemize}
\begin{itemize}
\item {Grp. gram.:M.}
\end{itemize}
\begin{itemize}
\item {Utilização:Prov.}
\end{itemize}
\begin{itemize}
\item {Utilização:beir.}
\end{itemize}
Designação injuriosa, que se dava aos Moiros e Judeus, talvez por não comerem carne de porco.
Immundo; excomungado.
Porco, já crescido.
(Cast. \textunderscore marrano\textunderscore )
\section{Marrão}
\begin{itemize}
\item {Grp. gram.:m.}
\end{itemize}
\begin{itemize}
\item {Grp. gram.:Adj.}
\end{itemize}
\begin{itemize}
\item {Utilização:Gír.}
\end{itemize}
\begin{itemize}
\item {Utilização:Bras. do Rio Grande do S}
\end{itemize}
Pequeno porco, que deixou de mamar.
Surprehendido em um crime.
Bravio, selvagem.
(Cast. \textunderscore marrano\textunderscore )
\section{Marrão}
\begin{itemize}
\item {Grp. gram.:m.}
\end{itemize}
\begin{itemize}
\item {Proveniência:(Do rad. do \textunderscore marrar\textunderscore )}
\end{itemize}
Grande martelo de ferro, para quebrar pedra principalmente.
\section{Marrão}
\begin{itemize}
\item {Grp. gram.:m.}
\end{itemize}
\begin{itemize}
\item {Utilização:Ant.}
\end{itemize}
Amarrão, grande amarra?:«\textunderscore ...dependurar-nos das tilhas, huns com marrões, outros com cabos\textunderscore ». \textunderscore Hist. Trág. Marit.\textunderscore , 52.
\section{Marrar}
\begin{itemize}
\item {Grp. gram.:v. i.}
\end{itemize}
\begin{itemize}
\item {Utilização:chul.}
\end{itemize}
\begin{itemize}
\item {Utilização:Ext.}
\end{itemize}
Bater com a cornada, (falando-se de animaes cornígeros).
Bater como marrão^2.
Bater com a cabeça.
Encontrar-se de frente com alguém ou com alguma coisa.
Toldar-se (o vinho).
(Cp. \textunderscore marrada\textunderscore ^1)
\section{Marrasquinado}
\begin{itemize}
\item {Grp. gram.:adj.}
\end{itemize}
Que bebeu muito marrasquino; embebedado com marrasquino. Cf. Macedo, \textunderscore Motim\textunderscore , II.
\section{Marrasquino}
\begin{itemize}
\item {Grp. gram.:m.}
\end{itemize}
Licor, feito com cerejas, o mesmo que \textunderscore marasquino\textunderscore .
(Cast. \textunderscore marrasquino\textunderscore , de \textunderscore marrasca\textunderscore , cereja)
\section{Marraxo}
\begin{itemize}
\item {Grp. gram.:m.}
\end{itemize}
\begin{itemize}
\item {Utilização:Des.}
\end{itemize}
\begin{itemize}
\item {Grp. gram.:Adj.}
\end{itemize}
\begin{itemize}
\item {Proveniência:(Do cast. \textunderscore marajo\textunderscore )}
\end{itemize}
Vendedor ambulante de peixe de armação, no Algarve.
Grande tubarão do Mar Índico e do Atlântico.
Gato velho, dócil e brincão.
Marralheiro; velhaco; astuto; matreiro.
\section{Marreca}
\begin{itemize}
\item {Grp. gram.:f.}
\end{itemize}
\begin{itemize}
\item {Grp. gram.:M.  e  f.}
\end{itemize}
Fêmea do marreco.
Corcova; carcunda.
Pessôa corcovada.
\section{Marrecão}
\begin{itemize}
\item {Grp. gram.:m.}
\end{itemize}
Espécie do ganso das regiões do Purus, (Brasil).
\section{Marreco}
\begin{itemize}
\item {Grp. gram.:m.}
\end{itemize}
\begin{itemize}
\item {Grp. gram.:Adj.}
\end{itemize}
\begin{itemize}
\item {Utilização:Des.}
\end{itemize}
Ave palmípede, semelhante ao pato, mas mais pequena.
Corcovado.
Sagaz, astuto.
(Cast. \textunderscore marreco\textunderscore )
\section{Marrela}
\begin{itemize}
\item {Grp. gram.:f.}
\end{itemize}
\begin{itemize}
\item {Utilização:Gír.}
\end{itemize}
Pão.
(Or. ind.)
\section{Marrequinho}
\begin{itemize}
\item {Grp. gram.:m.}
\end{itemize}
Gênero de aves palmípedes.
\section{Marreta}
\begin{itemize}
\item {fónica:rê}
\end{itemize}
\begin{itemize}
\item {Grp. gram.:f.}
\end{itemize}
\begin{itemize}
\item {Utilização:Gír.}
\end{itemize}
\begin{itemize}
\item {Proveniência:(De \textunderscore marrão\textunderscore ^2)}
\end{itemize}
Pequeno marrão.
Sapato.
\section{Marretada}
\begin{itemize}
\item {Grp. gram.:f.}
\end{itemize}
Pancada com marreta.
\section{Marricada}
\begin{itemize}
\item {Grp. gram.:f.}
\end{itemize}
\begin{itemize}
\item {Utilização:T. da Bairrada}
\end{itemize}
\begin{itemize}
\item {Proveniência:(De \textunderscore marricar\textunderscore )}
\end{itemize}
Mossa, produzida no pião pela ferroada de outro.
\section{Marrucate}
\begin{itemize}
\item {Grp. gram.:m.}
\end{itemize}
\begin{itemize}
\item {Utilização:Prov.}
\end{itemize}
\begin{itemize}
\item {Utilização:alent.}
\end{itemize}
Pão de centeio, embebido em água e leite, para os cães e outros animais.
\section{Marricar}
\begin{itemize}
\item {Grp. gram.:v.}
\end{itemize}
\begin{itemize}
\item {Utilização:t. da Bairrada}
\end{itemize}
\begin{itemize}
\item {Proveniência:(De \textunderscore marrar\textunderscore )}
\end{itemize}
Diz-se do pião que, no jôgo da gazola, bate com o bico em (outro pião), dentro de um círculo, que se risca no chão.
\section{Marroada}
\begin{itemize}
\item {Grp. gram.:f.}
\end{itemize}
Pancada com o marrão^2.
\section{Marroada}
\begin{itemize}
\item {Grp. gram.:f.}
\end{itemize}
\begin{itemize}
\item {Proveniência:(De \textunderscore marrão\textunderscore ^1)}
\end{itemize}
Manada ou vara de marrões. Cf. Corvo, \textunderscore Anno na Côrte\textunderscore , III, 39.
\section{Marroio}
\begin{itemize}
\item {Grp. gram.:m.}
\end{itemize}
Gênero de plantas labiadas, (\textunderscore marrubium\textunderscore ).
\section{Marromba}
\begin{itemize}
\item {Grp. gram.:f.}
\end{itemize}
Vestuário gentilico de Gôa.
\section{Marroquim}
\begin{itemize}
\item {Grp. gram.:m.}
\end{itemize}
\begin{itemize}
\item {Proveniência:(De \textunderscore marroquino\textunderscore )}
\end{itemize}
Pelle de cabra ou bode, tinta do lado da flôr e já preparada para artefactos.
\section{Marroquinar}
\begin{itemize}
\item {Grp. gram.:v. t.}
\end{itemize}
Converter em marroquim.
\section{Marroquinaria}
\begin{itemize}
\item {Grp. gram.:f.}
\end{itemize}
Fábrica, em que se prepara o marroquim.
Arte do marroquineiro.
\section{Marroquineiro}
\begin{itemize}
\item {Grp. gram.:m.}
\end{itemize}
\begin{itemize}
\item {Proveniência:(De \textunderscore marroquim\textunderscore )}
\end{itemize}
Aquelle que trabalha marroquinaria.
\section{Marroquino}
\begin{itemize}
\item {Grp. gram.:adj.}
\end{itemize}
\begin{itemize}
\item {Grp. gram.:M.}
\end{itemize}
Relativo a Marrocos.
Individuo de Marrocos.
\section{Marroteiro}
\begin{itemize}
\item {Grp. gram.:m.}
\end{itemize}
Aquelle que dirige o trabalho marnotos.
(Corr. de \textunderscore marnoteiro\textunderscore )
\section{Marroxo}
\begin{itemize}
\item {Grp. gram.:m.}
\end{itemize}
\begin{itemize}
\item {Utilização:Pop.}
\end{itemize}
Resto; rebotalho.
\section{Marruá}
\begin{itemize}
\item {Grp. gram.:m.}
\end{itemize}
\begin{itemize}
\item {Utilização:Bras. do N}
\end{itemize}
Toiro.
Novilho, não domesticado.
(Cp. \textunderscore marruaz\textunderscore )
\section{Marruaz}
\begin{itemize}
\item {Grp. gram.:adj.}
\end{itemize}
\begin{itemize}
\item {Grp. gram.:M.}
\end{itemize}
Teimoso; obstinado.
Embarcação asiática.--Dizem \textunderscore marruaz\textunderscore  os diccion., mas é talvez fórma incorrecta, em vez de \textunderscore marroaz\textunderscore .
Cp. \textunderscore marrão\textunderscore ^2 e \textunderscore marrão\textunderscore ^3.
\section{Marrucar}
\begin{itemize}
\item {Grp. gram.:v. i.}
\end{itemize}
\begin{itemize}
\item {Utilização:Prov.}
\end{itemize}
\begin{itemize}
\item {Utilização:minh.}
\end{itemize}
\begin{itemize}
\item {Proveniência:(De \textunderscore marrar\textunderscore )}
\end{itemize}
Cabecear com somno.
\section{Marrucates}
\begin{itemize}
\item {Grp. gram.:m.}
\end{itemize}
\begin{itemize}
\item {Utilização:Prov.}
\end{itemize}
\begin{itemize}
\item {Utilização:alent.}
\end{itemize}
Pão de centeio, embebido em água e leite, para os cães e outros animais.
\section{Marrucho}
\begin{itemize}
\item {Grp. gram.:m.}
\end{itemize}
\begin{itemize}
\item {Utilização:T. de Turquel}
\end{itemize}
Marrão pequeno, bacoro.
\section{Marrueiro}
\begin{itemize}
\item {Grp. gram.:m.}
\end{itemize}
\begin{itemize}
\item {Utilização:Bras. do N}
\end{itemize}
Indivíduo, habituado a derribar os marruás pela cauda.
\section{Marrufo}
\begin{itemize}
\item {Grp. gram.:m.}
\end{itemize}
\begin{itemize}
\item {Utilização:ant.}
\end{itemize}
\begin{itemize}
\item {Utilização:Pleb.}
\end{itemize}
Frade leigo.
Maroto, patife.
\section{Marrugem}
\begin{itemize}
\item {Grp. gram.:f.}
\end{itemize}
O mesmo que \textunderscore marugem\textunderscore .
\section{Marsala}
\begin{itemize}
\item {Grp. gram.:m.}
\end{itemize}
Apreciado vinho da Sicília. Cf. \textunderscore Tech. Rur.\textunderscore , 65.
\section{Marselhês}
\begin{itemize}
\item {Grp. gram.:m.}
\end{itemize}
\begin{itemize}
\item {Grp. gram.:M.}
\end{itemize}
Relativo a Marselha.
Habitante de Marselha.
\section{Marselhesa}
\begin{itemize}
\item {Grp. gram.:f.}
\end{itemize}
\begin{itemize}
\item {Grp. gram.:Adj.}
\end{itemize}
\begin{itemize}
\item {Proveniência:(De \textunderscore marselhês\textunderscore )}
\end{itemize}
Hymno nacional da França.
Diz-se de uma espécie de telha chata, com um rebordo que a fixa á ripa.
\section{Marsílea}
\begin{itemize}
\item {Grp. gram.:f.}
\end{itemize}
\begin{itemize}
\item {Proveniência:(De \textunderscore Marsigli\textunderscore , n. p.)}
\end{itemize}
Gênero de plantas vivazes, que crescem nas águas estagnadas.
\section{Marsileáceas}
\begin{itemize}
\item {Grp. gram.:f. pl.}
\end{itemize}
\begin{itemize}
\item {Proveniência:(De \textunderscore marcileáceo\textunderscore )}
\end{itemize}
Familia de plantas, que tem por typo a marsílea.
\section{Marsileáceo}
\begin{itemize}
\item {Grp. gram.:adj.}
\end{itemize}
Relativo ou semelhante á marsílea.
\section{Marso}
\begin{itemize}
\item {Grp. gram.:adj.}
\end{itemize}
\begin{itemize}
\item {Grp. gram.:M. pl.}
\end{itemize}
\begin{itemize}
\item {Proveniência:(Do lat. \textunderscore marsi\textunderscore )}
\end{itemize}
Relativo aos Marsos:«\textunderscore fatal conjuro marso\textunderscore ». Castilho, \textunderscore Fastos\textunderscore , III. 107.
Antigo povo do Lácio.
\section{Marsopa}
\begin{itemize}
\item {Grp. gram.:f.}
\end{itemize}
Espécie de cetáceo.
\section{Marsuíno}
\begin{itemize}
\item {Grp. gram.:m.}
\end{itemize}
\begin{itemize}
\item {Proveniência:(Fr. \textunderscore marsouin\textunderscore )}
\end{itemize}
Gênero de cetáceos, parecidos ao golfinho.
\section{Marsúpia}
\begin{itemize}
\item {Grp. gram.:f.}
\end{itemize}
\begin{itemize}
\item {Utilização:Des.}
\end{itemize}
\begin{itemize}
\item {Proveniência:(Do lat. \textunderscore marsupium\textunderscore )}
\end{itemize}
O mesmo que \textunderscore bôlsa\textunderscore . Cf. \textunderscore Fenix Renasc.\textunderscore , IV, 10.
\section{Marsapiaes}
\begin{itemize}
\item {Grp. gram.:m. pl.}
\end{itemize}
\begin{itemize}
\item {Proveniência:(De \textunderscore marsupial\textunderscore )}
\end{itemize}
Familia de mammíferos, caracterizados por uma espécie de bolsa, que as fêmeas têm por baixo do ventre, e na qual recolhem os filhos em-quanto se amamentam.
\section{Marsapiais}
\begin{itemize}
\item {Grp. gram.:m. pl.}
\end{itemize}
\begin{itemize}
\item {Proveniência:(De \textunderscore marsupial\textunderscore )}
\end{itemize}
Familia de mammíferos, caracterizados por uma espécie de bolsa, que as fêmeas têm por baixo do ventre, e na qual recolhem os filhos em-quanto se amamentam.
\section{Marsupial}
\begin{itemize}
\item {Grp. gram.:adj.}
\end{itemize}
\begin{itemize}
\item {Grp. gram.:F.}
\end{itemize}
\begin{itemize}
\item {Proveniência:(Do lat. \textunderscore marsupium\textunderscore )}
\end{itemize}
Que tem fórma de bolsa.
Que tem órgão em fórma de bolsa.
Gênero de molluscos, do grupo das medusas.
\section{Marsupialidade}
\begin{itemize}
\item {Grp. gram.:f.}
\end{itemize}
\begin{itemize}
\item {Proveniência:(De \textunderscore marsupial\textunderscore )}
\end{itemize}
Organização especial dos marsupiaes.
\section{Marsúpios}
\begin{itemize}
\item {Grp. gram.:m. pl.}
\end{itemize}
O mesmo que \textunderscore marsapiaes\textunderscore .
\section{Marta}
\begin{itemize}
\item {Grp. gram.:f.}
\end{itemize}
Gênero de mammíferos, carnívoros e digitigrados.
\section{Marta}
\begin{itemize}
\item {Grp. gram.:f.}
\end{itemize}
\begin{itemize}
\item {Utilização:Prov.}
\end{itemize}
\begin{itemize}
\item {Utilização:trasm.}
\end{itemize}
O mesmo que \textunderscore bebedeira\textunderscore .
\section{Marta}
\begin{itemize}
\item {Grp. gram.:f.}
\end{itemize}
Casta de uva branca americana.
\section{Martagão}
\begin{itemize}
\item {Grp. gram.:m.}
\end{itemize}
Variedade de lírio, (\textunderscore lilium martagon\textunderscore ).
\section{Marte}
\begin{itemize}
\item {Grp. gram.:m.}
\end{itemize}
\begin{itemize}
\item {Utilização:Ext.}
\end{itemize}
\begin{itemize}
\item {Proveniência:(Lat. \textunderscore Mars\textunderscore , \textunderscore Martis\textunderscore )}
\end{itemize}
O deus da guerra, segundo a Mythologia grega e latina.
Planeta, cuja revolução em volta do Sol dura 687 dias.
Guerra.
Homem guerreiro.
\section{Marteirar}
\begin{itemize}
\item {Grp. gram.:v. t.}
\end{itemize}
\begin{itemize}
\item {Utilização:Ant.}
\end{itemize}
\begin{itemize}
\item {Proveniência:(De \textunderscore marteiro\textunderscore )}
\end{itemize}
O mesmo que \textunderscore martyrizar\textunderscore . Cf. G. Vicente, I, 191; Arn. Gama, \textunderscore Ult. Dona\textunderscore , 232.
\section{Marteiro}
\begin{itemize}
\item {Grp. gram.:m.}
\end{itemize}
\begin{itemize}
\item {Utilização:Ant.}
\end{itemize}
O mesmo que \textunderscore martýrio\textunderscore . Cf. G. Vicente, I, 208.
\section{Mártel}
\begin{itemize}
\item {Grp. gram.:m.}
\end{itemize}
\begin{itemize}
\item {Utilização:ant.}
\end{itemize}
\begin{itemize}
\item {Utilização:Pop.}
\end{itemize}
O mesmo que \textunderscore mártyr\textunderscore . Cf. G. Vicente, I, 232.
\section{Martelada}
\begin{itemize}
\item {Grp. gram.:f.}
\end{itemize}
\begin{itemize}
\item {Utilização:Ext.}
\end{itemize}
Pancada com martelo.
Estrondo, semelhante, ao da pancada do martelo.
\section{Martelador}
\begin{itemize}
\item {Grp. gram.:m.}
\end{itemize}
Aquelle que martela.
\section{Martetagem}
\begin{itemize}
\item {Grp. gram.:f.}
\end{itemize}
Acto de \textunderscore martelar\textunderscore .
\section{Martelar}
\begin{itemize}
\item {Grp. gram.:v. t.}
\end{itemize}
\begin{itemize}
\item {Utilização:Fig.}
\end{itemize}
\begin{itemize}
\item {Grp. gram.:V. i.}
\end{itemize}
\begin{itemize}
\item {Utilização:Fig.}
\end{itemize}
Bater com o martelo em.
Importunar.
Causticar; maçar.
Dar marteladas.
Insistir: \textunderscore martelar num assumpto\textunderscore .
\section{Marteleiro}
\begin{itemize}
\item {Grp. gram.:m.}
\end{itemize}
\begin{itemize}
\item {Utilização:Prov.}
\end{itemize}
\begin{itemize}
\item {Utilização:alent.}
\end{itemize}
\begin{itemize}
\item {Utilização:Prov.}
\end{itemize}
Caçador, que raramente acerta o tiro contra a caça, (por allusão á pancada do cão da espingarda sôbre a espoleta).
Aquelle que fabríca vinho a martelo.
\section{Martelejar}
\begin{itemize}
\item {Grp. gram.:v. i.}
\end{itemize}
Dar marteladas.
Soar, como a pancada do martelo.
\section{Martelete}
\begin{itemize}
\item {fónica:lê}
\end{itemize}
\begin{itemize}
\item {Grp. gram.:m.}
\end{itemize}
Pequeno martelo.
Espora moirisca.
\section{Martelinho}
\begin{itemize}
\item {Grp. gram.:m.}
\end{itemize}
\begin{itemize}
\item {Utilização:Gír.}
\end{itemize}
\begin{itemize}
\item {Proveniência:(De \textunderscore martelo\textunderscore )}
\end{itemize}
Copo de meio quartilho.
Pênis.
\section{Martelinhos}
\begin{itemize}
\item {Grp. gram.:m. pl.}
\end{itemize}
\begin{itemize}
\item {Utilização:T. do Pôrto}
\end{itemize}
Variedade de planta, (\textunderscore narcissus cyclamineus\textunderscore , D. C.).
\section{Martello}
\begin{itemize}
\item {Grp. gram.:m.}
\end{itemize}
\begin{itemize}
\item {Utilização:Des.}
\end{itemize}
\begin{itemize}
\item {Utilização:Fig.}
\end{itemize}
\begin{itemize}
\item {Utilização:Bras}
\end{itemize}
\begin{itemize}
\item {Grp. gram.:Loc. adv.}
\end{itemize}
Instrumento de ferro, com cabo geralmente de pau, e destinado a bater ou quebrar, e especialmente a cravar pregos na madeira.
Peça do piano, com que são percutidas as cordas dêste.
Um dos ossos do ouvido.
Peixe, do gênero esqualo.
Mollusco acéphalo da Índia e Austrália.
Parte de certos relógios, que bate as horas ou fracções de horas em campaínha ou sino.
Chave, com que se afinam pianos.
Pessôa, que persegue e procura exterminar um mal.
Larva dos mosquitos, que trasm.ttem a febre amarela.
\textunderscore Vinho a martello\textunderscore , vinho aldrabado, falsificado.
\textunderscore A martello\textunderscore , á fôrça; sem dever sêr.
(B. lat. \textunderscore mariellus\textunderscore )
\section{Martelo}
\begin{itemize}
\item {Grp. gram.:m.}
\end{itemize}
\begin{itemize}
\item {Utilização:Des.}
\end{itemize}
\begin{itemize}
\item {Utilização:Fig.}
\end{itemize}
\begin{itemize}
\item {Utilização:Bras}
\end{itemize}
\begin{itemize}
\item {Grp. gram.:Loc. adv.}
\end{itemize}
Instrumento de ferro, com cabo geralmente de pau, e destinado a bater ou quebrar, e especialmente a cravar pregos na madeira.
Peça do piano, com que são percutidas as cordas dêste.
Um dos ossos do ouvido.
Peixe, do gênero esqualo.
Mollusco acéphalo da Índia e Austrália.
Parte de certos relógios, que bate as horas ou fracções de horas em campaínha ou sino.
Chave, com que se afinam pianos.
Pessôa, que persegue e procura exterminar um mal.
Larva dos mosquitos, que trasm.ttem a febre amarela.
\textunderscore Vinho a martelo\textunderscore , vinho aldrabado, falsificado.
\textunderscore A martelo\textunderscore , á fôrça; sem dever sêr.
(B. lat. \textunderscore mariellus\textunderscore )
\section{Martha}
\begin{itemize}
\item {Grp. gram.:f.}
\end{itemize}
Casta de uva branca americana.
\section{Martim}
\begin{itemize}
\item {Grp. gram.:m.}
\end{itemize}
Casta de uva.
\section{Martimenga}
\begin{itemize}
\item {Grp. gram.:f.}
\end{itemize}
\begin{itemize}
\item {Utilização:Ant.}
\end{itemize}
Carapucínha, usada especialmente por alguns Judeus em Portugal.
\section{Martim-garavato}
\begin{itemize}
\item {Grp. gram.:m.}
\end{itemize}
O mesmo que \textunderscore martim-gravata\textunderscore .
\section{Martim-gil}
\begin{itemize}
\item {Grp. gram.:m.}
\end{itemize}
Variedade de maçan.
\section{Martim-gravata}
\begin{itemize}
\item {Grp. gram.:m.}
\end{itemize}
Espécie de jôgo popular.
\section{Martim-gravato}
\begin{itemize}
\item {Grp. gram.:m.}
\end{itemize}
Espécie de jôgo popular.
\section{Martim-pescador}
\begin{itemize}
\item {Grp. gram.:m.}
\end{itemize}
\begin{itemize}
\item {Utilização:Bras}
\end{itemize}
Ave ribeirinha, o mesmo que \textunderscore rei-pescador\textunderscore .
\section{Martinega}
\begin{itemize}
\item {Grp. gram.:f.}
\end{itemize}
\begin{itemize}
\item {Utilização:Ant.}
\end{itemize}
Tributo, que os proprietários pagavam, no dia de San-Martinho. Cf. Herculano, \textunderscore Hist. de Port\textunderscore ., IV, 418.
\section{Martinete}
\begin{itemize}
\item {fónica:nê}
\end{itemize}
\begin{itemize}
\item {Grp. gram.:m.}
\end{itemize}
\begin{itemize}
\item {Utilização:Mús.}
\end{itemize}
\begin{itemize}
\item {Proveniência:(Fr. \textunderscore martinet\textunderscore )}
\end{itemize}
Grande martelo, movido a vapor ou a água, para bater ferro ou aço.
Espécie de andorinha de asas longas.
Gaivão.
Pennacho.
Enfeite, semelhante ao pennacho dos grous.
Martelo de piano.
A mais pequena das três soalhas da balestilha.
Ponteiro do relógio de sol.
Enfeite de retrós e vidrilho, que imita o pennacho dos grous.
Flôr amaranthácea, avelludada e roxa; o mesmo que \textunderscore crista-gálli\textunderscore ?
Peça do maquinismo dos antigos cravos, na qual se fixava o bico da penna, que fazia vibrar as cordas e que correspondia ao martelo dos pianos.
\section{Martinézia}
\begin{itemize}
\item {Grp. gram.:f.}
\end{itemize}
\begin{itemize}
\item {Proveniência:(De \textunderscore Martinez\textunderscore , n. p.)}
\end{itemize}
Planta ornamental brasileira. Cf. \textunderscore Jorn. do Comm.\textunderscore , do Rio, de 29-V-902.
\section{Martingil}
\begin{itemize}
\item {Grp. gram.:m.}
\end{itemize}
O mesmo que \textunderscore martim-gil\textunderscore .
\section{Martinho}
\begin{itemize}
\item {Grp. gram.:m.}
\end{itemize}
Nome de uma ave indiana. Cf. Th. Ribeiro, \textunderscore Jornadas\textunderscore , II, 143.
\section{Martinho}
\begin{itemize}
\item {Grp. gram.:m.}
\end{itemize}
Carneiro, que vai á frente de um rebanho?:«\textunderscore ...embora níveo seja o martinho da grei...\textunderscore »Castilho, \textunderscore Geórgicas\textunderscore , 193.
\section{Martinho-pescador}
\begin{itemize}
\item {Grp. gram.:m.}
\end{itemize}
(V.martim-pescador)
\section{Martinica}
\begin{itemize}
\item {Grp. gram.:f.}
\end{itemize}
\begin{itemize}
\item {Utilização:Bras. do Maranhão}
\end{itemize}
Calça larga, usada por homem do povo.
\section{Martiniega}
\begin{itemize}
\item {Grp. gram.:f.}
\end{itemize}
\begin{itemize}
\item {Utilização:Ant.}
\end{itemize}
Tributo, que os proprietários pagavam, no dia de San-Martinho. Cf. Herculano, \textunderscore Hist. de Port\textunderscore ., IV, 418.
\section{Mártir}
\begin{itemize}
\item {Grp. gram.:m.  e  f.}
\end{itemize}
\begin{itemize}
\item {Utilização:Ext.}
\end{itemize}
Pessôa, que sofreu tormentos ou a morte por sustentar a fé christan.
Aquele que sofre por causa das suas crenças ou das suas opiniões.
Pessôa, que sofre muito.
Pessôa, que é vítima dos maus tratos de outrem.
(Lat. \textunderscore mártyr\textunderscore ).
\section{Mártire}
\begin{itemize}
\item {Grp. gram.:m.}
\end{itemize}
\begin{itemize}
\item {Utilização:Des.}
\end{itemize}
O mesmo que \textunderscore mártir\textunderscore . Cf. \textunderscore Lusíadas\textunderscore , III, 74.
\section{Martírio}
\begin{itemize}
\item {Grp. gram.:m.}
\end{itemize}
\begin{itemize}
\item {Utilização:Ext.}
\end{itemize}
\begin{itemize}
\item {Utilização:Bot.}
\end{itemize}
\begin{itemize}
\item {Proveniência:(Lat. \textunderscore martyrium\textunderscore )}
\end{itemize}
Sofrimento ou suplício de mártir.
Tormento ou grande sofrimento; grande aflição.
Planta passiflórea, (\textunderscore passiflora caerula\textunderscore , Lin.).
\section{Martirizar}
\begin{itemize}
\item {Grp. gram.:v. i.}
\end{itemize}
\begin{itemize}
\item {Utilização:Ext.}
\end{itemize}
Tornar mártir.
Atormentar, afligir muito.
\section{Martirológio}
\begin{itemize}
\item {Grp. gram.:m.}
\end{itemize}
\begin{itemize}
\item {Proveniência:(Do gr. \textunderscore martur + logion\textunderscore )}
\end{itemize}
Lista dos mártires, com a narração dos seus tormentos.
\section{Martirologista}
\begin{itemize}
\item {Grp. gram.:m.}
\end{itemize}
Autor de um martirológio.
\section{Marto}
\begin{itemize}
\item {Grp. gram.:m.}
\end{itemize}
\begin{itemize}
\item {Utilização:Prov.}
\end{itemize}
\begin{itemize}
\item {Utilização:minh.}
\end{itemize}
Gato bravo.
(Relaciona-se com \textunderscore marta\textunderscore ^1 ?)
\section{Martur}
\begin{itemize}
\item {Grp. gram.:m.}
\end{itemize}
\begin{itemize}
\item {Utilização:Ant.}
\end{itemize}
Espécie de taficira.
\section{Martýnia}
\begin{itemize}
\item {Grp. gram.:f.}
\end{itemize}
\begin{itemize}
\item {Proveniência:(De \textunderscore Martyn\textunderscore , n. p.)}
\end{itemize}
Gênero de plantas gesneráceas.
\section{Mártyr}
\begin{itemize}
\item {Grp. gram.:m.  e  f.}
\end{itemize}
\begin{itemize}
\item {Utilização:Ext.}
\end{itemize}
Pessôa, que soffreu tormentos ou a morte por sustentar a fé christan.
Aquelle que soffre por causa das suas crenças ou das suas opiniões.
Pessôa, que soffre muito.
Pessôa, que é víctima dos maus tratos de outrem.
(Lat. \textunderscore mártyr\textunderscore ).
\section{Mártyre}
\begin{itemize}
\item {Grp. gram.:m.}
\end{itemize}
\begin{itemize}
\item {Utilização:Des.}
\end{itemize}
O mesmo que \textunderscore mártyr\textunderscore . Cf. \textunderscore Lusíadas\textunderscore , III, 74.
\section{Martýrio}
\begin{itemize}
\item {Grp. gram.:m.}
\end{itemize}
\begin{itemize}
\item {Utilização:Ext.}
\end{itemize}
\begin{itemize}
\item {Utilização:Bot.}
\end{itemize}
\begin{itemize}
\item {Proveniência:(Lat. \textunderscore martyrium\textunderscore )}
\end{itemize}
Soffrimento ou supplício de mártyr.
Tormento ou grande soffrimento; grande afflicção.
Planta passiflórea, (\textunderscore passiflora caerula\textunderscore , Lin.).
\section{Martyrizar}
\begin{itemize}
\item {Grp. gram.:v. i.}
\end{itemize}
\begin{itemize}
\item {Utilização:Ext.}
\end{itemize}
Tornar mártyr.
Atormentar, affligir muito.
\section{Martyrológio}
\begin{itemize}
\item {Grp. gram.:m.}
\end{itemize}
\begin{itemize}
\item {Proveniência:(Do gr. \textunderscore martur + logion\textunderscore )}
\end{itemize}
Lista dos mártyres, com a narração dos seus tormentos.
\section{Martyrologista}
\begin{itemize}
\item {Grp. gram.:m.}
\end{itemize}
Autor de um martyrológio.
\section{Maruás}
\begin{itemize}
\item {Grp. gram.:m. pl.}
\end{itemize}
Indígenas do norte do Brasil.
\section{Marubá}
\begin{itemize}
\item {Grp. gram.:m.}
\end{itemize}
Fruto medicinal da quássia do Pará.
\section{Maruca}
\begin{itemize}
\item {Grp. gram.:f.}
\end{itemize}
Planta medicinal da Guiana inglesa.
\section{Marudo}
\begin{itemize}
\item {Grp. gram.:m.}
\end{itemize}
\begin{itemize}
\item {Utilização:ant.}
\end{itemize}
\begin{itemize}
\item {Utilização:Pleb.}
\end{itemize}
O mesmo que \textunderscore marido\textunderscore . Cf. Sim. Mach., f. 82, v.
\section{Marufle}
\begin{itemize}
\item {Grp. gram.:m.}
\end{itemize}
\begin{itemize}
\item {Proveniência:(Fr. \textunderscore maroufle\textunderscore )}
\end{itemize}
Colla muito forte, que os pintores empregam, para reforçar uma tela com outra ou para applicar a mesma tela sôbre madeira ou parede.
\section{Marufo}
\begin{itemize}
\item {Grp. gram.:m.}
\end{itemize}
\begin{itemize}
\item {Utilização:Chul.}
\end{itemize}
O mesmo que \textunderscore maluvo\textunderscore .
\section{Maruge}
\begin{itemize}
\item {Grp. gram.:f.}
\end{itemize}
Planta; o mesmo que \textunderscore murugem\textunderscore .
\section{Marugem}
\begin{itemize}
\item {Grp. gram.:f.}
\end{itemize}
Planta; o mesmo que \textunderscore murugem\textunderscore .
\section{Maruí}
\begin{itemize}
\item {Grp. gram.:m.}
\end{itemize}
Mosquito dos terrenos pantanosos do Brasil.--B. C. Rubim, \textunderscore Vocab. Bras.\textunderscore , acha errada aquella orthogr. e diz que deve escrever-se e dizer-se \textunderscore meruí\textunderscore .
\section{Maruim}
\begin{itemize}
\item {Grp. gram.:m.}
\end{itemize}
Mosquito dos terrenos pantanosos do Brasil.--B. C. Rubim, \textunderscore Vocab. Bras.\textunderscore , acha errada aquella orthogr. e diz que deve escrever-se e dizer-se \textunderscore meruí\textunderscore .
\section{Maruja}
\begin{itemize}
\item {Grp. gram.:f.}
\end{itemize}
O mesmo que \textunderscore marinhagem\textunderscore .
\section{Marujada}
\begin{itemize}
\item {Grp. gram.:f.}
\end{itemize}
Gente do mar; marinhagem.
(De \textunderscore marujo\textunderscore ).
\section{Marujal}
\begin{itemize}
\item {Grp. gram.:adj.}
\end{itemize}
\begin{itemize}
\item {Utilização:P. us.}
\end{itemize}
Relativo a marujo; próprio do marujo. Cf. \textunderscore Agostinheida\textunderscore , 68 e 165.
\section{Marujar}
\begin{itemize}
\item {Grp. gram.:v. i.}
\end{itemize}
\begin{itemize}
\item {Utilização:Prov.}
\end{itemize}
Verdejar com a marugem: \textunderscore  os campos marujam\textunderscore .
\section{Marujinha}
\begin{itemize}
\item {Grp. gram.:f.}
\end{itemize}
Variedade de azeitona.
\section{Marujo}
\begin{itemize}
\item {Grp. gram.:m.}
\end{itemize}
\begin{itemize}
\item {Proveniência:(De \textunderscore mar\textunderscore )}
\end{itemize}
Homem, que trabalha a bordo; marinheiro.
No Algarve, aquelle que tripula um barco pequeno.
\section{Marulhada}
\begin{itemize}
\item {Grp. gram.:f.}
\end{itemize}
O mesmo que \textunderscore marulho\textunderscore .
\section{Marulhado}
\begin{itemize}
\item {Grp. gram.:adj.}
\end{itemize}
\begin{itemize}
\item {Proveniência:(De \textunderscore marulhar\textunderscore )}
\end{itemize}
Tocado ou coberto pelas ondas em marulho.
\section{Marulhar}
\begin{itemize}
\item {Grp. gram.:v. i.  e  pl.}
\end{itemize}
\begin{itemize}
\item {Proveniência:(De \textunderscore marulho\textunderscore )}
\end{itemize}
Agitar-se, formar vagas (o mar).
Imitar o ruido das ondas.
\section{Marulheiro}
\begin{itemize}
\item {Grp. gram.:adj.}
\end{itemize}
Que causa marulho, (falando-se do vento).
\section{Marulho}
\begin{itemize}
\item {Grp. gram.:m.}
\end{itemize}
\begin{itemize}
\item {Utilização:Fig.}
\end{itemize}
\begin{itemize}
\item {Proveniência:(De \textunderscore mar\textunderscore )}
\end{itemize}
Agitação das ondas do mar.
Agitação; barulho; tumulto.
Enjôo do mar.
\section{Marulhoso}
\begin{itemize}
\item {Grp. gram.:adj.}
\end{itemize}
Em que há marulho.
\section{Marulo}
\begin{itemize}
\item {Grp. gram.:m.}
\end{itemize}
\begin{itemize}
\item {Utilização:T. de Pare -de-Coira}
\end{itemize}
\begin{itemize}
\item {Utilização:des.}
\end{itemize}
Indivíduo baixo e gordo.
\section{Marúmia}
\begin{itemize}
\item {Grp. gram.:f.}
\end{itemize}
Gênero de plantas melastomáceas.
\section{Maruorana}
\begin{itemize}
\item {Grp. gram.:f.}
\end{itemize}
Planta malvácea do Pará.
\section{Marupá}
\begin{itemize}
\item {Grp. gram.:m.}
\end{itemize}
Gênero de plantas medicinaes da América, o mesmo que \textunderscore simaruba\textunderscore .
\section{Marupá-mirim}
\begin{itemize}
\item {Grp. gram.:m.}
\end{itemize}
\begin{itemize}
\item {Utilização:Bras}
\end{itemize}
Arbusto amazónico, de raíz medicinal.
\section{Marupaúba}
\begin{itemize}
\item {Grp. gram.:f.}
\end{itemize}
Árvore corpulenta das margens do Amazonas.
\section{Maruru}
\begin{itemize}
\item {Grp. gram.:m.}
\end{itemize}
Planta brasileira, de rhizoma alimenticio.
\section{Maruta}
\begin{itemize}
\item {Grp. gram.:f.}
\end{itemize}
Gênero de plantas, da fam. das compostas.
\section{Maruvané}
\begin{itemize}
\item {Grp. gram.:m.}
\end{itemize}
Tôsco instrumento de cordas em Madagáscar.
\section{Marzapo}
\begin{itemize}
\item {Grp. gram.:m.}
\end{itemize}
\begin{itemize}
\item {Utilização:Prov.}
\end{itemize}
\begin{itemize}
\item {Utilização:Chul.}
\end{itemize}
O mesmo que \textunderscore pênis\textunderscore .
\section{Marzoco}
\begin{itemize}
\item {fónica:zô}
\end{itemize}
\begin{itemize}
\item {Grp. gram.:m.}
\end{itemize}
\begin{itemize}
\item {Utilização:Des.}
\end{itemize}
\begin{itemize}
\item {Proveniência:(It. \textunderscore marzocco\textunderscore )}
\end{itemize}
Bobo; truão.
\section{Mas}
\begin{itemize}
\item {Grp. gram.:conj.}
\end{itemize}
\begin{itemize}
\item {Grp. gram.:M.}
\end{itemize}
\begin{itemize}
\item {Proveniência:(Do lat. \textunderscore magis\textunderscore )}
\end{itemize}
(designativa de \textunderscore opposição\textunderscore  ou \textunderscore restricção\textunderscore )
Difficuldade, obstáculo; defeito: \textunderscore tudo tem o seu mas\textunderscore .
Pôsto que:«\textunderscore ...mas não venha a propósito...\textunderscore »F. Manuel, \textunderscore Carta de Guia\textunderscore , 146.
\section{Màsaldemenos}
\begin{itemize}
\item {Grp. gram.:adv.}
\end{itemize}
\begin{itemize}
\item {Utilização:Ant.}
\end{itemize}
\begin{itemize}
\item {Proveniência:(De \textunderscore mais\textunderscore  + \textunderscore aldemenos\textunderscore )}
\end{itemize}
Mais ou menos.
\section{Masares}
\begin{itemize}
\item {Grp. gram.:m. pl.}
\end{itemize}
\begin{itemize}
\item {Proveniência:(De \textunderscore Masaris\textunderscore , n. p.)}
\end{itemize}
Insectos hymenópteros, notáveis pelo comprimento das antennas.
\section{Mascabar}
\begin{itemize}
\item {Grp. gram.:v. t.}
\end{itemize}
(Corr. de \textunderscore menoscabar\textunderscore )
\section{Mascabo}
\begin{itemize}
\item {Grp. gram.:m.}
\end{itemize}
(Corr. de \textunderscore menoscabo\textunderscore )
\section{Mascador}
\begin{itemize}
\item {Grp. gram.:m.}
\end{itemize}
Aquelle que masca.
\section{Mascar}
\begin{itemize}
\item {Grp. gram.:v. i.}
\end{itemize}
\begin{itemize}
\item {Utilização:fam.}
\end{itemize}
\begin{itemize}
\item {Utilização:Fig.}
\end{itemize}
\begin{itemize}
\item {Grp. gram.:V. i.}
\end{itemize}
\begin{itemize}
\item {Proveniência:(Do lat. \textunderscore masticare\textunderscore )}
\end{itemize}
Mastigar, sem engulir.
Dar a entender, insinuar.
Meditar, planeando.
Repisar (palavras), pronunciando-as indistintamente.
Fingir que se mastiga.
Mover a maxilla, como quem mastiga.
Resmungar, falar por entre dentes.
\section{Máscara}
\begin{itemize}
\item {Grp. gram.:f.}
\end{itemize}
\begin{itemize}
\item {Utilização:Fig.}
\end{itemize}
\begin{itemize}
\item {Grp. gram.:M.  e  f.}
\end{itemize}
\begin{itemize}
\item {Utilização:Bras. de Minas}
\end{itemize}
Artefacto, que representa uma cara ou parte della, e é destinado a cobrir o rosto, para disfarçar a pessoa que o põe.
Peça ou cobertura, com que se resguarda dos golpes a cara, na guerra ou no jogo da esgrima.
Utensílio análogo, para livrar o rosto das picadas das abelhas, quando se faz a cresta das colmeias.
Disfarce.
Pessôa mascarada.
Varanda.
Alpendre.
(Do ár.)
\section{Mascarada}
\begin{itemize}
\item {Grp. gram.:f.}
\end{itemize}
\begin{itemize}
\item {Proveniência:(De \textunderscore máscara\textunderscore )}
\end{itemize}
Grupo de pessôas mascaradas.
Festa, em que apparecem muitas pessôas mascaradas.
\section{Mascarado}
\begin{itemize}
\item {Grp. gram.:m.}
\end{itemize}
\begin{itemize}
\item {Grp. gram.:Adj.}
\end{itemize}
\begin{itemize}
\item {Utilização:Bras}
\end{itemize}
\begin{itemize}
\item {Utilização:Mil.}
\end{itemize}
\begin{itemize}
\item {Proveniência:(De \textunderscore mascarar\textunderscore )}
\end{itemize}
Pessoa mascarada.
Diz-se da bataria, composta de canhões occultos no mato.
\section{Mascarão}
\begin{itemize}
\item {Grp. gram.:m.}
\end{itemize}
\begin{itemize}
\item {Proveniência:(Do rad. de \textunderscore máscara\textunderscore )}
\end{itemize}
Ornato de pedra, em fórma de máscara.
\section{Mascarar}
\begin{itemize}
\item {Grp. gram.:v. t.}
\end{itemize}
\begin{itemize}
\item {Utilização:Fig.}
\end{itemize}
Disfarçar, cobrindo o rosto com máscara.
Disfarçar com máscara e com traje excêntrico ou caprichoso.
Disfarçar; occultar.
Dar falsa apparência a.
\section{Mascarilha}
\begin{itemize}
\item {Grp. gram.:f.}
\end{itemize}
Pequena máscara, que só cobre parte do rosto.
(Cast. \textunderscore mascarilla\textunderscore )
\section{Mascarino}
\begin{itemize}
\item {Grp. gram.:adj.}
\end{itemize}
\begin{itemize}
\item {Utilização:Bot.}
\end{itemize}
\begin{itemize}
\item {Proveniência:(Do rad. de \textunderscore máscara\textunderscore )}
\end{itemize}
Diz-se das flôres e corollas, que têm o aspecto de máscara.
\section{Mascarra}
\begin{itemize}
\item {Grp. gram.:f.}
\end{itemize}
\begin{itemize}
\item {Utilização:Fig.}
\end{itemize}
Mancha, feita com carvão, tinta, etc.
Sujidade.
Labéu; estigma.
(Alter. de \textunderscore máscara\textunderscore )
\section{Mascarrar}
\begin{itemize}
\item {Grp. gram.:v. t.}
\end{itemize}
\begin{itemize}
\item {Utilização:P. us.}
\end{itemize}
\begin{itemize}
\item {Utilização:Fig.}
\end{itemize}
Pôr mascarras em.
Pintar mal, escrever mal.
Deitar borrões em.
Macular, desacreditar.
\section{Masca-tabaco}
\begin{itemize}
\item {Grp. gram.:m.}
\end{itemize}
Peixe de Portugal.
\section{Mascataria}
\begin{itemize}
\item {Grp. gram.:f.}
\end{itemize}
\begin{itemize}
\item {Utilização:Bras}
\end{itemize}
Profissão de mascate.
\section{Mascate}
\begin{itemize}
\item {Grp. gram.:m.}
\end{itemize}
\begin{itemize}
\item {Utilização:Bras}
\end{itemize}
\begin{itemize}
\item {Proveniência:(De \textunderscore Mascate\textunderscore , n. p.)}
\end{itemize}
Vendedor ambulante, de fazendas.
\section{Mascateação}
\begin{itemize}
\item {Grp. gram.:f.}
\end{itemize}
\begin{itemize}
\item {Utilização:Bras}
\end{itemize}
Acto de mascatear; profissão de mascate.
\section{Mascatear}
\begin{itemize}
\item {Grp. gram.:v. i.}
\end{itemize}
\begin{itemize}
\item {Utilização:Bras}
\end{itemize}
\begin{itemize}
\item {Proveniência:(De \textunderscore mascate\textunderscore )}
\end{itemize}
Vender fazendas pelas ruas.
\section{Mascato}
\begin{itemize}
\item {Grp. gram.:m.}
\end{itemize}
Espécie de ganso, (\textunderscore sula alba\textunderscore , Mey.).
\section{Mascavado}
\begin{itemize}
\item {Grp. gram.:adj.}
\end{itemize}
\begin{itemize}
\item {Proveniência:(De \textunderscore mascavar\textunderscore )}
\end{itemize}
Não refinado, (falando-se do açúcar).
\section{Mascavar}
\begin{itemize}
\item {Grp. gram.:v. t.}
\end{itemize}
\begin{itemize}
\item {Utilização:Fig.}
\end{itemize}
Separar e juntar (o açúcar de peór qualidade).
Falsificar; adulterar.
Pronunciar ou escrever, servindo-se de linguagem impura ou incorrecta.
(Alter. de \textunderscore mascabar\textunderscore )
\section{Mascavinho}
\begin{itemize}
\item {Grp. gram.:m.}
\end{itemize}
Açúcar, um pouco mais claro que o mascavo.
\section{Mascavo}
\begin{itemize}
\item {Grp. gram.:m.}
\end{itemize}
\begin{itemize}
\item {Grp. gram.:Adj.}
\end{itemize}
\begin{itemize}
\item {Utilização:Bras}
\end{itemize}
Acto de mascavar.
O mesmo que \textunderscore mascavado\textunderscore , (falando-se de açúcar).
\section{Mascotar}
\begin{itemize}
\item {Grp. gram.:v. t.}
\end{itemize}
Moer com mascoto.
Mascar, mastigar. Cf. D. Bernárdez, \textunderscore Lima\textunderscore , 102; Jazente, II, 119.
\section{Mascote}
\begin{itemize}
\item {Grp. gram.:f.}
\end{itemize}
\begin{itemize}
\item {Utilização:Fam.}
\end{itemize}
\begin{itemize}
\item {Proveniência:(Do n. p. de uma opereta)}
\end{itemize}
Bôa sina, bôa-ventura.
\section{Mascoto}
\begin{itemize}
\item {fónica:cô}
\end{itemize}
\begin{itemize}
\item {Grp. gram.:m.}
\end{itemize}
\begin{itemize}
\item {Utilização:T. de chapeleiro}
\end{itemize}
\begin{itemize}
\item {Utilização:Prov.}
\end{itemize}
\begin{itemize}
\item {Utilização:minh.}
\end{itemize}
Grande martelo, com que, nas fricas de moéda, se reduzem a pó os fragmentos de metal.
Espécie de pisão, em que se opéra a fula.
O mesmo que \textunderscore maço\textunderscore .
\section{Maçambará}
\begin{itemize}
\item {Grp. gram.:m.}
\end{itemize}
Planta gramínea do Brasil.
\section{Maçango}
\begin{itemize}
\item {Grp. gram.:m.}
\end{itemize}
\begin{itemize}
\item {Utilização:T. de Angola}
\end{itemize}
Variedade de milho.
\section{Maçará}
\begin{itemize}
\item {Grp. gram.:m.}
\end{itemize}
\begin{itemize}
\item {Utilização:Bras. do N}
\end{itemize}
Espécie de pari.
\section{Maçarandiba}
\begin{itemize}
\item {Grp. gram.:f.}
\end{itemize}
Gênero de plantas mirtáceas do Brasil.
\section{Maçarandubeira}
\begin{itemize}
\item {Grp. gram.:f.}
\end{itemize}
O mesmo que \textunderscore maçaranduba\textunderscore , árvore.
\section{Maçaruas}
\begin{itemize}
\item {Grp. gram.:m. pl.}
\end{itemize}
Selvagens nômades da África central.
\section{Maciço}
\begin{itemize}
\item {Grp. gram.:adj.}
\end{itemize}
\begin{itemize}
\item {Grp. gram.:M.}
\end{itemize}
Compacto; que não tem cavidades, que não é oco: \textunderscore oiro massiço\textunderscore .
Feito de uma só substância.
Cerrado basto: \textunderscore bosque maciço\textunderscore .
Sólido.
Inabalável.
Importante.
Qualidade do que é compacto.
Coisa compacta.
Grupo de pessôas ou coisas muito juntas: \textunderscore um maciço de plantas\textunderscore .
(Os que preferem \textunderscore massiço\textunderscore , relacionam o t. com \textunderscore massa\textunderscore )
\section{Maçolar}
\begin{itemize}
\item {Grp. gram.:v. t.}
\end{itemize}
\begin{itemize}
\item {Utilização:Des.}
\end{itemize}
Partir, quebrar com pancadas:«\textunderscore O Marquez que foi de Tavora morreu rodado e maçolado vivo.\textunderscore »\textunderscore Collecção da Legisl. Port.\textunderscore , 1756-1762, p. 605. Cf. G. Viana, \textunderscore Apostilas\textunderscore .
\section{Maçonge}
\begin{itemize}
\item {Grp. gram.:m.}
\end{itemize}
Planta africana, medicinal, talvez da fam. das amomáceas.
\section{Mascóvia}
\begin{itemize}
\item {Grp. gram.:f.}
\end{itemize}
O mesmo que \textunderscore moscóvia\textunderscore ^1.
\section{Màsculifloro}
\begin{itemize}
\item {Grp. gram.:adj.}
\end{itemize}
\begin{itemize}
\item {Utilização:Bot.}
\end{itemize}
\begin{itemize}
\item {Proveniência:(Do lat. \textunderscore masculus\textunderscore  + \textunderscore flos\textunderscore , \textunderscore floris\textunderscore )}
\end{itemize}
Que tem flôres masculinas.
\section{Masculinidade}
\begin{itemize}
\item {Grp. gram.:f.}
\end{itemize}
\begin{itemize}
\item {Proveniência:(Lat. \textunderscore masculinitas\textunderscore )}
\end{itemize}
Qualidade do masculino; qualidade de másculo; virilidade.
\section{Masculinismo}
\begin{itemize}
\item {Grp. gram.:m.}
\end{itemize}
\begin{itemize}
\item {Utilização:Neol.}
\end{itemize}
Qualidade de masculino.
\section{Masculinizar}
\begin{itemize}
\item {Grp. gram.:v. t.}
\end{itemize}
\begin{itemize}
\item {Utilização:Fig.}
\end{itemize}
Tornar masculino.
Attribuir gênero masculino a.
Dar fórma masculina a.
Dar a apparência de sexo masculino a.
\section{Masculino}
\begin{itemize}
\item {Grp. gram.:adj.}
\end{itemize}
\begin{itemize}
\item {Utilização:Fig.}
\end{itemize}
\begin{itemize}
\item {Utilização:Gram.}
\end{itemize}
\begin{itemize}
\item {Proveniência:(Lat. \textunderscore masculinus\textunderscore )}
\end{itemize}
Que é do sexo dos animaes machos.
Relativo ao macho.
Varonil, másculo, enérgico.
Diz-se das palavras ou dos nomes e do gênero de palavras ou nomes, que, pela sua terminação ou pela sua concordância, designam seres que são masculinos ou se consideram taes, embora não tenham sexo.
\section{Másculo}
\begin{itemize}
\item {Grp. gram.:adj.}
\end{itemize}
\begin{itemize}
\item {Utilização:Ext.}
\end{itemize}
\begin{itemize}
\item {Proveniência:(Lat. \textunderscore masculus\textunderscore )}
\end{itemize}
Relativo ao homem ou a animal macho.
Viril; enérgico.
\section{Masdeísmo}
\begin{itemize}
\item {Grp. gram.:m.}
\end{itemize}
Religião de Zoroastro.
(Do zenda \textunderscore masda\textunderscore )
\section{Masdevália}
\begin{itemize}
\item {Grp. gram.:f.}
\end{itemize}
Gênero de orquídeas do Brasil.
\section{Masdevállia}
\begin{itemize}
\item {Grp. gram.:f.}
\end{itemize}
Gênero de orchídeas do Brasil.
\section{Masgar}
\begin{itemize}
\item {Grp. gram.:v. t.}
\end{itemize}
\begin{itemize}
\item {Utilização:Prov.}
\end{itemize}
\begin{itemize}
\item {Utilização:minh.}
\end{itemize}
O mesmo que \textunderscore esmagar\textunderscore , por metáth.
\section{Masmarro}
\begin{itemize}
\item {Grp. gram.:m.}
\end{itemize}
\begin{itemize}
\item {Utilização:Chul.}
\end{itemize}
Frade leigo.
Ermitão, de hábito talar.
Donato de frades.
Fradalhão; frade interesseiro.
Marmanjo.
\section{Masmorra}
\begin{itemize}
\item {fónica:mô}
\end{itemize}
\begin{itemize}
\item {Grp. gram.:f.}
\end{itemize}
\begin{itemize}
\item {Utilização:fam.}
\end{itemize}
\begin{itemize}
\item {Utilização:Fig.}
\end{itemize}
Primitivamente, celleiro subterrâneo, que também servia de cárcere entre os Moiros.
Prisão subterrânea.
Lugar ou aposento sombrio e triste.
(Cast. \textunderscore mazmorra\textunderscore , do ár. \textunderscore matmure\textunderscore ).
\section{Masmorreiro}
\begin{itemize}
\item {Grp. gram.:m.}
\end{itemize}
\begin{itemize}
\item {Utilização:Des.}
\end{itemize}
\begin{itemize}
\item {Proveniência:(De \textunderscore masmorra\textunderscore )}
\end{itemize}
Carcereiro.
\section{Masquir}
\begin{itemize}
\item {Grp. gram.:v. t.}
\end{itemize}
\begin{itemize}
\item {Utilização:Gír.}
\end{itemize}
Mastigar.
\section{Másrio}
\begin{itemize}
\item {Grp. gram.:m.}
\end{itemize}
\begin{itemize}
\item {Proveniência:(De um t. ár., que significa \textunderscore Egypto\textunderscore )}
\end{itemize}
Mineral, descoberto em 1890, no Egypto.
\section{Masrita}
\begin{itemize}
\item {Grp. gram.:f.}
\end{itemize}
Composição mineral, em que entra o másrio, água, alumina, vários óxydos, etc.
\section{Másrium}
\begin{itemize}
\item {Grp. gram.:m.}
\end{itemize}
\begin{itemize}
\item {Proveniência:(De um t. ár., que significa \textunderscore Egypto\textunderscore )}
\end{itemize}
Mineral, descoberto em 1890, no Egypto.
\section{Massa}
\begin{itemize}
\item {Grp. gram.:f.}
\end{itemize}
\begin{itemize}
\item {Utilização:Chul.}
\end{itemize}
\begin{itemize}
\item {Utilização:Bras}
\end{itemize}
\begin{itemize}
\item {Grp. gram.:Loc. adv.}
\end{itemize}
\begin{itemize}
\item {Grp. gram.:Pl.}
\end{itemize}
\begin{itemize}
\item {Proveniência:(Lat. \textunderscore massa\textunderscore )}
\end{itemize}
Mistura de farinha com um líquido, formando pasta: \textunderscore massa de centeio\textunderscore .
Aquillo que tem fórma ou semelhança de farinha empastada.
Substância molle ou simplesmente pulverizada: \textunderscore massa de tomates\textunderscore .
O todo, cujas partes são da mesma natureza.
Conjunto das partes que constituem um todo.
Totalidade: \textunderscore a massa da população\textunderscore .
Um corpo compacto: \textunderscore caiu, como massa inerte\textunderscore .
Multidão de povo.
Matéria que constitue um corpo.
O conjunto de fórças militares, congregadas como um todo.
Substância informe.
Pecúlio, formado por deducções no salário do soldado, para vestuário o outras despesas.
Dinheiro.
Mandioca ralada.
\textunderscore Estar com as mãos na massa\textunderscore , estar tratando, estar-se occupando, de certo assumpto.
\textunderscore Em massa\textunderscore , no conjunto; na totalidade; totalmente.
Agglomeração de povo: \textunderscore falar ás massas\textunderscore .
O povo; a população.
\section{Massa}
\begin{itemize}
\item {Grp. gram.:m.}
\end{itemize}
\begin{itemize}
\item {Proveniência:(De \textunderscore Massa\textunderscore , n. p.)}
\end{itemize}
Vinho muito apreciado das vizinhanças de Nápoles.
\section{Massacrar}
\begin{itemize}
\item {Grp. gram.:v. t.}
\end{itemize}
\begin{itemize}
\item {Utilização:Gal}
\end{itemize}
\begin{itemize}
\item {Proveniência:(De \textunderscore massacre\textunderscore )}
\end{itemize}
Matar cruelmente.
\section{Massacre}
\begin{itemize}
\item {Grp. gram.:m.}
\end{itemize}
\begin{itemize}
\item {Utilização:Gal}
\end{itemize}
\begin{itemize}
\item {Proveniência:(Fr. \textunderscore massacre\textunderscore )}
\end{itemize}
Carnificina.
Matança cruel de várias pessôas, que se não podem defender ou que se defendem mal.--Este t. e seus der. constituem gallicismos dispensaveis, não empregados até agora pelos bons cultores da língua.
\section{Massacroco}
\begin{itemize}
\item {fónica:crô}
\end{itemize}
\begin{itemize}
\item {Grp. gram.:m.}
\end{itemize}
\begin{itemize}
\item {Utilização:Ant.}
\end{itemize}
Canudo, tecido de cabello, para ornato de cabelleira.
\section{Massagada}
\begin{itemize}
\item {Grp. gram.:f.}
\end{itemize}
\begin{itemize}
\item {Utilização:Pop.}
\end{itemize}
\begin{itemize}
\item {Proveniência:(Do rad. do \textunderscore massa\textunderscore )}
\end{itemize}
Grande confusão do coisas; salgalhada; mistifório.
\section{Massagem}
\begin{itemize}
\item {Grp. gram.:f.}
\end{itemize}
(V.maçagem)
\section{Masságetas}
\begin{itemize}
\item {Grp. gram.:m. pl.}
\end{itemize}
\begin{itemize}
\item {Proveniência:(Lat. \textunderscore massagetae\textunderscore )}
\end{itemize}
Antigo povo seýthio das vizinhanças do Cáucaso.
\section{Massambala}
\begin{itemize}
\item {Grp. gram.:f.}
\end{itemize}
Espécie de sorgho angolense.
\section{Massambará}
\begin{itemize}
\item {Grp. gram.:m.}
\end{itemize}
Planta gramínea do Brasil.
\section{Massame}
\begin{itemize}
\item {Grp. gram.:m.}
\end{itemize}
\begin{itemize}
\item {Proveniência:(De \textunderscore massa\textunderscore ^1)}
\end{itemize}
Leito dos poços, formado do pedra e argamassa.
Cordoalha de navio.
\section{Massamorda}
\begin{itemize}
\item {fónica:môr}
\end{itemize}
\begin{itemize}
\item {Grp. gram.:f.}
\end{itemize}
Açorda.
Comida mal feita.
Salsada; mixórdia; massagada.
Fragmemtos de biscoitos ou de bolos, que a bordo so dão ás aves. Cf. M. de Aguiar, \textunderscore Diccion. de Marinha\textunderscore .
\section{Massango}
\begin{itemize}
\item {Grp. gram.:m.}
\end{itemize}
\begin{itemize}
\item {Utilização:T. de Angola}
\end{itemize}
Variedade de milho.
\section{Massapé}
\begin{itemize}
\item {Grp. gram.:m.}
\end{itemize}
\begin{itemize}
\item {Utilização:Bras}
\end{itemize}
\begin{itemize}
\item {Proveniência:(De \textunderscore massa\textunderscore  + \textunderscore pé\textunderscore )}
\end{itemize}
Terra fértil, em consequência dos álcalis que nella abundam.
Terreno lamacento, atoleiro.
Pozzolana dos Açores.
Erva medicinal do Brasil.
Caule do beijoim.
\section{Massará}
\begin{itemize}
\item {Grp. gram.:m.}
\end{itemize}
\begin{itemize}
\item {Utilização:Bras. do N}
\end{itemize}
Espécie de pari.
\section{Massarana}
\begin{itemize}
\item {Grp. gram.:f.}
\end{itemize}
O mesmo que \textunderscore mussurana\textunderscore .
\section{Massarandiba}
\begin{itemize}
\item {Grp. gram.:f.}
\end{itemize}
Gênero de plantas myrtáceas do Brasil.
\section{Massaranduba}
\begin{itemize}
\item {Grp. gram.:f.}
\end{itemize}
\begin{itemize}
\item {Utilização:Bras. do N}
\end{itemize}
Árvore sapotácea do Brasil.
Fruto dessa árvore.
O mesmo que \textunderscore cacete\textunderscore .
\section{Massarandubeira}
\begin{itemize}
\item {Grp. gram.:f.}
\end{itemize}
O mesmo que \textunderscore massaranduba\textunderscore , árvore.
\section{Massaroco}
\begin{itemize}
\item {fónica:sarô}
\end{itemize}
\begin{itemize}
\item {Grp. gram.:m.}
\end{itemize}
\begin{itemize}
\item {Proveniência:(Do rad. de \textunderscore massa\textunderscore )}
\end{itemize}
Porção de fermento, com que se leveda o pão.
Planta borraginácea, (\textunderscore echium candicans\textunderscore , Lin.).
\section{Massaruas}
\begin{itemize}
\item {Grp. gram.:m. pl.}
\end{itemize}
Selvagens nômades da África central.
\section{Masseira}
\begin{itemize}
\item {Grp. gram.:f.}
\end{itemize}
\begin{itemize}
\item {Utilização:T. de Viana}
\end{itemize}
\begin{itemize}
\item {Utilização:Constr.}
\end{itemize}
\begin{itemize}
\item {Proveniência:(De \textunderscore massa\textunderscore )}
\end{itemize}
Grande tabuleiro, em que se amassa a farinha, para o fabrico do pão.
Arteza.
Calha ou cale, por onde corre a água que cai dos alcatruzes.
Pequena embarcação, em fórma de taboleiro, usada por pescadores da costa do norte de Portugal.
Pequena cuba de madeira, que substitue o cesto da vindima.
\textunderscore Tecto de masseira\textunderscore , o mesmo que \textunderscore tecto aconchado\textunderscore .
\section{Masseirão}
\begin{itemize}
\item {Grp. gram.:m.}
\end{itemize}
\begin{itemize}
\item {Utilização:Prov.}
\end{itemize}
\begin{itemize}
\item {Utilização:alent.}
\end{itemize}
Espécie de alguidar, em qne os sapateiros humedecem o cerol e o cabedal e em que se dá alimento a cevados e gallináceos.
O mesmo que \textunderscore barranhão\textunderscore . Cf. Rev. \textunderscore Tradição\textunderscore , II, 11.
\section{Massenda-senda}
\begin{itemize}
\item {Grp. gram.:f.}
\end{itemize}
Árvore de Cabinda, cuja madeira é empregada em coronhas, mobília, etc.
\section{Masséter}
\begin{itemize}
\item {Grp. gram.:m.}
\end{itemize}
\begin{itemize}
\item {Utilização:Anat.}
\end{itemize}
\begin{itemize}
\item {Proveniência:(Do gr. \textunderscore masseter\textunderscore )}
\end{itemize}
Músculo, que, serve para o movimento da maxilla, na mastigação.
\section{Masseterino}
\begin{itemize}
\item {Grp. gram.:adj.}
\end{itemize}
Relativo ao masséter.
\section{Mássia}
\begin{itemize}
\item {Grp. gram.:f.}
\end{itemize}
\begin{itemize}
\item {Utilização:Ant.}
\end{itemize}
Casa rústica para gente do campo.
(Relaciona-se com o lat. \textunderscore mansio\textunderscore ?)
\section{Massiço}
\begin{itemize}
\item {Grp. gram.:adj.}
\end{itemize}
\begin{itemize}
\item {Grp. gram.:M.}
\end{itemize}
Compacto; que não tem cavidades, que não é oco: \textunderscore oiro massiço\textunderscore .
Feito de uma só substância.
Cerrado basto: \textunderscore bosque massiço\textunderscore .
Sólido.
Inabalável.
Importante.
Qualidade do que é compacto.
Coisa compacta.
Grupo de pessôas ou coisas muito juntas: \textunderscore um massiço de plantas\textunderscore .
(Cp. \textunderscore maciço\textunderscore ^1, que é talvez preferível. Os que preferem \textunderscore massiço\textunderscore , relacionam o t. com \textunderscore massa\textunderscore )
\section{Massicote}
\begin{itemize}
\item {Grp. gram.:m.}
\end{itemize}
\begin{itemize}
\item {Proveniência:(Do fr. \textunderscore massicot\textunderscore )}
\end{itemize}
Protóxydo de chumbo.
\section{Massilha}
\begin{itemize}
\item {Grp. gram.:f.}
\end{itemize}
\begin{itemize}
\item {Proveniência:(De \textunderscore massa\textunderscore )}
\end{itemize}
Polme, feito de papel ou de outras substâncias.
\section{Massinha}
\begin{itemize}
\item {Grp. gram.:f.}
\end{itemize}
O mesmo que \textunderscore massilha\textunderscore .
\section{Massolar}
\begin{itemize}
\item {Grp. gram.:v. t.}
\end{itemize}
\begin{itemize}
\item {Utilização:Des.}
\end{itemize}
Partir, quebrar com pancadas:«\textunderscore O Marquez que foi de Tavora morreu rodado e massolado vivo.\textunderscore »\textunderscore Collecção da Legisl. Port.\textunderscore , 1756-1762, p. 605. Cf. G. Viana, \textunderscore Apostilas\textunderscore .
\section{Massoleimão}
\begin{itemize}
\item {Grp. gram.:m.}
\end{itemize}
\begin{itemize}
\item {Utilização:Ant.}
\end{itemize}
O mesmo que \textunderscore muçulmano\textunderscore . Cf. \textunderscore Peregrinação\textunderscore , XXVII.
\section{Massonge}
\begin{itemize}
\item {Grp. gram.:m.}
\end{itemize}
Planta africana, medicinal, talvez da fam. das amomáceas.
\section{Massorá}
\begin{itemize}
\item {Grp. gram.:f.}
\end{itemize}
\begin{itemize}
\item {Proveniência:(Hebr. \textunderscore massorah\textunderscore )}
\end{itemize}
Trabalho critico á cêrca da \textunderscore Biblia\textunderscore , feito por doutores judeus.
\section{Massoreta}
\begin{itemize}
\item {fónica:sorê}
\end{itemize}
\begin{itemize}
\item {Grp. gram.:m.}
\end{itemize}
Cada um dos que collaboraram na massorá.
\section{Massua}
\begin{itemize}
\item {Grp. gram.:f.}
\end{itemize}
\begin{itemize}
\item {Utilização:Ant.}
\end{itemize}
O mesmo que \textunderscore massuca\textunderscore .
\section{Massuca}
\begin{itemize}
\item {Grp. gram.:f.}
\end{itemize}
\begin{itemize}
\item {Utilização:Ant.}
\end{itemize}
\begin{itemize}
\item {Proveniência:(De \textunderscore massuco\textunderscore )}
\end{itemize}
Pedaço do ferro, não purificado.
Pequeno molho de linho.
\section{Massuco}
\begin{itemize}
\item {Grp. gram.:adj.}
\end{itemize}
\begin{itemize}
\item {Utilização:Ant.}
\end{itemize}
\begin{itemize}
\item {Grp. gram.:M.}
\end{itemize}
\begin{itemize}
\item {Proveniência:(De \textunderscore massa\textunderscore )}
\end{itemize}
O mesmo que \textunderscore maciço\textunderscore ^1.
O mesmo que \textunderscore massuca\textunderscore .
\section{Massucote}
\begin{itemize}
\item {Grp. gram.:m.}
\end{itemize}
\begin{itemize}
\item {Utilização:Serralh.}
\end{itemize}
Ferramenta para encostar rebites.
\section{Massudo}
\begin{itemize}
\item {Grp. gram.:adj.}
\end{itemize}
Que tem aspecto de massa.
Cheio.
Compacto.
Grosso; grosseiro.
\section{Mastaréo}
\begin{itemize}
\item {Grp. gram.:m.}
\end{itemize}
\begin{itemize}
\item {Utilização:Náut.}
\end{itemize}
\begin{itemize}
\item {Proveniência:(De mastro)}
\end{itemize}
Pequeno mastro supplementar.
\section{Mastaréu}
\begin{itemize}
\item {Grp. gram.:m.}
\end{itemize}
\begin{itemize}
\item {Utilização:Náut.}
\end{itemize}
\begin{itemize}
\item {Proveniência:(De mastro)}
\end{itemize}
Pequeno mastro suplementar.
\section{Mastica}
\begin{itemize}
\item {Grp. gram.:f.}
\end{itemize}
\begin{itemize}
\item {Utilização:P. us.}
\end{itemize}
O mesmo que \textunderscore mastique\textunderscore .
\section{Masticária}
\begin{itemize}
\item {Grp. gram.:f.}
\end{itemize}
Planta da serra de Sintra.
O mesmo que \textunderscore matricária\textunderscore ? Cf. Juromenha, \textunderscore Cintra Pinturesca\textunderscore .
\section{Masticatório}
\begin{itemize}
\item {Grp. gram.:m.}
\end{itemize}
O mesmo que \textunderscore mastigatório\textunderscore .
\section{Mastiche}
\begin{itemize}
\item {Grp. gram.:m.}
\end{itemize}
O mesmo que \textunderscore mastique\textunderscore . Cf. Moraes, \textunderscore Diccion.\textunderscore , vb. \textunderscore almécega\textunderscore .
\section{Mastigação}
\begin{itemize}
\item {Grp. gram.:f.}
\end{itemize}
\begin{itemize}
\item {Proveniência:(Do lat. \textunderscore masticatio\textunderscore )}
\end{itemize}
Acto ou effeito de mastigar.
\section{Mastigada}
\begin{itemize}
\item {Grp. gram.:f.}
\end{itemize}
\begin{itemize}
\item {Utilização:Prov.}
\end{itemize}
\begin{itemize}
\item {Utilização:Prov.}
\end{itemize}
\begin{itemize}
\item {Utilização:minh.}
\end{itemize}
\begin{itemize}
\item {Proveniência:(De \textunderscore mastigar\textunderscore )}
\end{itemize}
Carnificina, mortandade. Cf. Camillo. \textunderscore Brasileira\textunderscore , 276.
Confusão; balbúrdia.
\section{Mastigado}
\begin{itemize}
\item {Grp. gram.:m.}
\end{itemize}
\begin{itemize}
\item {Grp. gram.:Adj.}
\end{itemize}
\begin{itemize}
\item {Utilização:Fig.}
\end{itemize}
\begin{itemize}
\item {Proveniência:(De \textunderscore mastigar\textunderscore )}
\end{itemize}
Aquillo que se mastigou e que, em vez do se deglutir, se expelliu da bôca.
Bem preparado, bem planeado.
\section{Mastigadoiro}
\begin{itemize}
\item {Grp. gram.:m.}
\end{itemize}
Espécie de freio, que facilita aos cavallos a mastigação.
(Do \textunderscore mastigar\textunderscore )
\section{Mastigador}
\begin{itemize}
\item {Grp. gram.:m.  e  adj.}
\end{itemize}
O que mastiga.
\section{Mastigadouro}
\begin{itemize}
\item {Grp. gram.:m.}
\end{itemize}
Espécie de freio, que facilita aos cavallos a mastigação.
(Do \textunderscore mastigar\textunderscore )
\section{Mastigar}
\begin{itemize}
\item {Grp. gram.:v. t.}
\end{itemize}
\begin{itemize}
\item {Utilização:Fig.}
\end{itemize}
\begin{itemize}
\item {Proveniência:(Do lat. \textunderscore maslicare\textunderscore )}
\end{itemize}
Triturar com os dentes.
Apertar com os dentes.
Ponderar, examinar.
Pronunciar indistintamente, dizer por entre os dentes; resmungar.
\section{Mastigatório}
\begin{itemize}
\item {Grp. gram.:m.}
\end{itemize}
\begin{itemize}
\item {Proveniência:(Do lat. \textunderscore masticatorium\textunderscore )}
\end{itemize}
Aquillo que se mastiga, para promover salivação.
\section{Mastigo}
\begin{itemize}
\item {Grp. gram.:m.}
\end{itemize}
(V.mastique)
\section{Mastigóforo}
\begin{itemize}
\item {Grp. gram.:m.}
\end{itemize}
\begin{itemize}
\item {Proveniência:(Lat. \textunderscore mastigophoros\textunderscore )}
\end{itemize}
Funcionário público que, armado do chicote, era encarregado de manter a ordem nos teatros gregos e romanos.
\section{Mastigóphoro}
\begin{itemize}
\item {Grp. gram.:m.}
\end{itemize}
\begin{itemize}
\item {Proveniência:(Lat. \textunderscore mastigophoros\textunderscore )}
\end{itemize}
Funccionário público que, armado do chicote, era encarregado de manter a ordem nos theatros gregos e romanos.
\section{Mastil}
\begin{itemize}
\item {Grp. gram.:m.}
\end{itemize}
\begin{itemize}
\item {Utilização:Des.}
\end{itemize}
Pequeno mastro de navio; mastro.
(Cp. cast. \textunderscore mastil\textunderscore )
\section{Mastilina}
\begin{itemize}
\item {Grp. gram.:f.}
\end{itemize}
Princípio constitutivo do mastique.
\section{Mastim}
\begin{itemize}
\item {Grp. gram.:m.}
\end{itemize}
\begin{itemize}
\item {Utilização:Ext.}
\end{itemize}
\begin{itemize}
\item {Utilização:Fig.}
\end{itemize}
\begin{itemize}
\item {Proveniência:(Do it. \textunderscore mastino\textunderscore )}
\end{itemize}
Cão para guarda de gado.
Cão bulhento.
Pessôa, que tem má língua.
Agente policial.
\section{Mastique}
\begin{itemize}
\item {Grp. gram.:m.}
\end{itemize}
\begin{itemize}
\item {Proveniência:(Fr. \textunderscore mastic\textunderscore , do lat. \textunderscore mastiche\textunderscore )}
\end{itemize}
O mesmo que \textunderscore almécega\textunderscore .
\section{Masto}
\begin{itemize}
\item {Grp. gram.:m.}
\end{itemize}
\begin{itemize}
\item {Utilização:Ant.}
\end{itemize}
\begin{itemize}
\item {Proveniência:(Do al. \textunderscore mast\textunderscore )}
\end{itemize}
O mesmo que \textunderscore mastro\textunderscore . Cf. Filinto. \textunderscore D. Man.\textunderscore , I, 307.
\section{Mastodinia}
\begin{itemize}
\item {Grp. gram.:f.}
\end{itemize}
\begin{itemize}
\item {Proveniência:(Do gr. \textunderscore mastos\textunderscore  + \textunderscore odune\textunderscore )}
\end{itemize}
Dôr nas glândulas mamárias.
\section{Mastodonte}
\begin{itemize}
\item {Grp. gram.:m.}
\end{itemize}
\begin{itemize}
\item {Proveniência:(Do gr. \textunderscore mastos\textunderscore  + \textunderscore odous\textunderscore , \textunderscore odontos\textunderscore )}
\end{itemize}
Corpulento animal fóssil, de estructura análoga á do elephante.
\section{Mastodôntico}
\begin{itemize}
\item {Grp. gram.:adj.}
\end{itemize}
\begin{itemize}
\item {Utilização:Fig.}
\end{itemize}
Relativo ao mastodonte.
Corpulento, agigantado.
\section{Mastodynia}
\begin{itemize}
\item {Grp. gram.:f.}
\end{itemize}
\begin{itemize}
\item {Proveniência:(Do gr. \textunderscore mastos\textunderscore  + \textunderscore odune\textunderscore )}
\end{itemize}
Dôr nas glândulas mamárias.
\section{Mastóide}
\begin{itemize}
\item {Grp. gram.:adj.}
\end{itemize}
O mesmo que \textunderscore mastoídeo\textunderscore .
\section{Mastoídeo}
\begin{itemize}
\item {Grp. gram.:adj.}
\end{itemize}
\begin{itemize}
\item {Utilização:Anat.}
\end{itemize}
\begin{itemize}
\item {Proveniência:(Do gr. \textunderscore mastos\textunderscore  + \textunderscore eidos\textunderscore )}
\end{itemize}
Que tem fórma de mamillo.
\textunderscore Apóphyse mastoídea\textunderscore , apóphyse, situada na parte póstero-inferior do osso temporal.
\section{Mastoidite}
\begin{itemize}
\item {Grp. gram.:f.}
\end{itemize}
Inflammação da apóphyse mastoidea.
\section{Mastologia}
\begin{itemize}
\item {Grp. gram.:f.}
\end{itemize}
O mesmo que \textunderscore mastozoologia\textunderscore .
\section{Mastoquino}
\begin{itemize}
\item {Grp. gram.:m.}
\end{itemize}
\begin{itemize}
\item {Proveniência:(Fr. \textunderscore mastoquin\textunderscore )}
\end{itemize}
Navalha curta, usada pela marinhagem.
\section{Mastozoário}
\begin{itemize}
\item {Grp. gram.:adj.}
\end{itemize}
\begin{itemize}
\item {Utilização:Zool.}
\end{itemize}
\begin{itemize}
\item {Grp. gram.:M. pl.}
\end{itemize}
\begin{itemize}
\item {Proveniência:(Do gr. \textunderscore mastos\textunderscore  + \textunderscore zoon\textunderscore )}
\end{itemize}
Que tem mamas.
O mesmo que [[mammíferos|mammífero]].
\section{Mastozoologia}
\begin{itemize}
\item {Grp. gram.:f.}
\end{itemize}
\begin{itemize}
\item {Proveniência:(Do gr. \textunderscore mastos\textunderscore  + \textunderscore zoon\textunderscore  + \textunderscore logos\textunderscore )}
\end{itemize}
Tratado dos mammíferos.
\section{Mastozoótico}
\begin{itemize}
\item {Grp. gram.:adj.}
\end{itemize}
\begin{itemize}
\item {Proveniência:(Do gr. \textunderscore mastos\textunderscore  + \textunderscore zoon\textunderscore )}
\end{itemize}
Diz-se de um terreno, que contém restos fósseis de mammíferos.
\section{Mastreação}
\begin{itemize}
\item {Grp. gram.:f.}
\end{itemize}
Acto de mastrear.
Conjunto mastros de uma embarcação.
\section{Mastrear}
\begin{itemize}
\item {Grp. gram.:v. t.}
\end{itemize}
Pôr mastros em (navios).
\section{Mastro}
\begin{itemize}
\item {Grp. gram.:m.}
\end{itemize}
Peça comprida de madeira, que sustenta as velas das embarcações.
Madeiro alto, para uso de gymnastas.
Madeiro alto e esguio, que se reveste de folhagem ou flôres, para ladear as ruas ou lugares, em que passa uma procissão ou um cortejo festivo.
Árvore da ilha de San-Thomé.
(Cp. \textunderscore masto\textunderscore )
\section{Mastro-real}
\begin{itemize}
\item {Grp. gram.:m.}
\end{itemize}
\begin{itemize}
\item {Utilização:Náut.}
\end{itemize}
Designação, que compete a cada um dos três mastros,--o grande, o do traquete ou da prôa e o da mezena.
\section{Mastruço}
\begin{itemize}
\item {Grp. gram.:m.}
\end{itemize}
Planta crucífera e medicinal.
(Cf. \textunderscore masturço\textunderscore )
\section{Mastucador}
\begin{itemize}
\item {Grp. gram.:m.}
\end{itemize}
\begin{itemize}
\item {Utilização:Prov.}
\end{itemize}
\begin{itemize}
\item {Utilização:alg.}
\end{itemize}
Espécie de mangual, para mastucar.
\section{Mastucar}
\begin{itemize}
\item {Grp. gram.:v. t.}
\end{itemize}
\begin{itemize}
\item {Utilização:Prov.}
\end{itemize}
\begin{itemize}
\item {Utilização:alg.}
\end{itemize}
Debulhar (centeio), por fórma que a palha fique inteiriça.
(Cp. \textunderscore machucar\textunderscore )
\section{Mastura}
\textunderscore f.\textunderscore  (e der.)
Fórma pop. e corrente na Beira, em vez de \textunderscore mistura\textunderscore , etc.
\section{Masturbação}
\begin{itemize}
\item {Grp. gram.:f.}
\end{itemize}
Acto de masturbar.
\section{Masturbador}
\begin{itemize}
\item {Grp. gram.:m.}
\end{itemize}
\begin{itemize}
\item {Proveniência:(Lat. \textunderscore masturbator\textunderscore )}
\end{itemize}
Aquelle que masturba.
\section{Masturbar}
\begin{itemize}
\item {Grp. gram.:v. i.}
\end{itemize}
\begin{itemize}
\item {Proveniência:(Lat. \textunderscore masturbare\textunderscore )}
\end{itemize}
Têr certos prazeres solitários, nocivos á saude.
\section{Masturço}
\begin{itemize}
\item {Grp. gram.:m.}
\end{itemize}
\begin{itemize}
\item {Proveniência:(Do lat. \textunderscore masturtium\textunderscore )}
\end{itemize}
O mesmo que \textunderscore mastruço\textunderscore . Cf. \textunderscore Aulegrafia\textunderscore , 135.
\section{Masulipatão}
\begin{itemize}
\item {Grp. gram.:m.}
\end{itemize}
\begin{itemize}
\item {Proveniência:(De \textunderscore Masulipatão\textunderscore , n. p.)}
\end{itemize}
Tecido, com desenhos primorosos, fabricado na Índia
\section{Mata}
\begin{itemize}
\item {Grp. gram.:f.}
\end{itemize}
\begin{itemize}
\item {Utilização:Fig.}
\end{itemize}
\begin{itemize}
\item {Utilização:Gír.}
\end{itemize}
\begin{itemize}
\item {Utilização:Ant.}
\end{itemize}
\begin{itemize}
\item {Proveniência:(Do lat. \textunderscore matta\textunderscore , esteira?)}
\end{itemize}
Terreno, em que crescem árvores silvestres.
Bosque; arvoredo.
Grande porção de hastes ou de objectos análogos.
Lugar, onde se vende fato velho.
Árvore ou arbusto. Cf. Bernárdez, \textunderscore Luz e Calor\textunderscore , 471.
\section{Mata}
\begin{itemize}
\item {Grp. gram.:f.}
\end{itemize}
\begin{itemize}
\item {Utilização:Bras. do S}
\end{itemize}
O mesmo que \textunderscore matadura\textunderscore .
\section{Màtabicho}
\begin{itemize}
\item {Grp. gram.:m.}
\end{itemize}
\begin{itemize}
\item {Utilização:Bras}
\end{itemize}
\begin{itemize}
\item {Proveniência:(De \textunderscore matar\textunderscore  + \textunderscore bicho\textunderscore )}
\end{itemize}
Um gole, que se toma, de qualquer bebida alcoólica.
Na África port., presente, mimo.
\section{Mata-boi}
\begin{itemize}
\item {Grp. gram.:m.}
\end{itemize}
\begin{itemize}
\item {Utilização:Bras. do S}
\end{itemize}
\begin{itemize}
\item {Utilização:Prov.}
\end{itemize}
\begin{itemize}
\item {Utilização:alent.}
\end{itemize}
Tira de coiro, que une o eixo ao leito das carrêtas.
Cavilha que, com a ajuda do apeiro, liga o cabeçalho á canga, e que, nos carros de muares, é de ferro, e de madeira nos carros de bois.
\section{Mata-borrão}
\begin{itemize}
\item {Grp. gram.:m.}
\end{itemize}
Papel, destinado a absorver tinta ou outro líquido.
\section{Mata-cães}
\begin{itemize}
\item {Grp. gram.:m.}
\end{itemize}
\begin{itemize}
\item {Utilização:Fig.}
\end{itemize}
\begin{itemize}
\item {Utilização:Ant.}
\end{itemize}
Preparado venenoso, para matar cães.
Indivíduo ocioso, vadio.
Galeria saliente nos velhos castellos e sobre as antigas portas das cidades, com aberturas, pelas quaes se arremessavam pedras ou outros projécteis, para impedir a aproximação dos inimigos.
\section{Mata-cana}
\begin{itemize}
\item {Grp. gram.:f.}
\end{itemize}
\begin{itemize}
\item {Utilização:Bras}
\end{itemize}
Planta escrofuiarínea e medicinal.
\section{Matacão}
\begin{itemize}
\item {Grp. gram.:m.}
\end{itemize}
\begin{itemize}
\item {Utilização:Fig.}
\end{itemize}
Pedra pequena.
Planta venenosa, da fam. das ranunculáceas.
Naco; grande fatia ou pedaço.
\section{Matação}
\begin{itemize}
\item {Grp. gram.:f.}
\end{itemize}
\begin{itemize}
\item {Utilização:Ant.}
\end{itemize}
\begin{itemize}
\item {Utilização:Fig.}
\end{itemize}
\begin{itemize}
\item {Proveniência:(De \textunderscore matar\textunderscore )}
\end{itemize}
Quantia, com que se pagava um aluguér ou renda.
Grande cuidado.
Azáfama.
Afflicção; apoquentação.
\section{Matacardo}
\begin{itemize}
\item {Grp. gram.:m.}
\end{itemize}
Nome de uma planta, mencionada por Brotero.
\section{Mata-cavallo}
\begin{itemize}
\item {Grp. gram.:m.}
\end{itemize}
Planta borragínea.
\section{Mata-cavallos}
\begin{itemize}
\item {Grp. gram.:m.}
\end{itemize}
\begin{itemize}
\item {Utilização:pop.}
\end{itemize}
\begin{itemize}
\item {Utilização:Us. na loc. adv.}
\end{itemize}
\textunderscore a mata-cavallos\textunderscore , á desfilada, a galope; apressadamente.
\section{Mataco}
\begin{itemize}
\item {Grp. gram.:m.}
\end{itemize}
\begin{itemize}
\item {Utilização:Bras}
\end{itemize}
Nádegas.
(Or. afr.)
\section{Matacões}
\begin{itemize}
\item {Grp. gram.:m. pl.}
\end{itemize}
Barba, em fórma de suíças.
\section{Matadeira}
\begin{itemize}
\item {Grp. gram.:f.}
\end{itemize}
\begin{itemize}
\item {Utilização:T. de Viana}
\end{itemize}
Mulher, que prepara e vende carne de porco, geralmente em tabernas.
(Cp. \textunderscore matador\textunderscore )
\section{Matadeiro}
\begin{itemize}
\item {Grp. gram.:m.}
\end{itemize}
\begin{itemize}
\item {Utilização:Des.}
\end{itemize}
O mesmo que \textunderscore matadoiro\textunderscore . Us. por Vieira.
\section{Matado}
\begin{itemize}
\item {Grp. gram.:adj.}
\end{itemize}
\begin{itemize}
\item {Utilização:Bras. do S}
\end{itemize}
\begin{itemize}
\item {Proveniência:(De \textunderscore mata\textunderscore ^2)}
\end{itemize}
Que tem mataduras, (falando-se do cavallo).
\section{Matadoiro}
\begin{itemize}
\item {Grp. gram.:m.}
\end{itemize}
\begin{itemize}
\item {Proveniência:(De \textunderscore matar\textunderscore )}
\end{itemize}
Lugar, onde se abatem as reses para consumo público.
Grande mortandade; carnificina.
Lugar muito insalubre.
\section{Matador}
\begin{itemize}
\item {Grp. gram.:adj.}
\end{itemize}
\begin{itemize}
\item {Grp. gram.:M.}
\end{itemize}
\begin{itemize}
\item {Utilização:Prov.}
\end{itemize}
\begin{itemize}
\item {Utilização:Fig.}
\end{itemize}
\begin{itemize}
\item {Grp. gram.:Pl.}
\end{itemize}
\begin{itemize}
\item {Utilização:Pop.}
\end{itemize}
Que mata.
Aquelle que mata.
Individuo, encarregado de matar porcos.
Pessôa, que importuna ou que enfada.
As três cartas de maior valor e os trunfos immediatos, pela ordem dos seus valores, no voltarete.
Tudo que é preciso para determinados fins.
\section{Matadouro}
\begin{itemize}
\item {Grp. gram.:m.}
\end{itemize}
\begin{itemize}
\item {Proveniência:(De \textunderscore matar\textunderscore )}
\end{itemize}
Lugar, onde se abatem as reses para consumo público.
Grande mortandade; carnificina.
Lugar muito insalubre.
\section{Matadura}
\begin{itemize}
\item {Grp. gram.:f.}
\end{itemize}
\begin{itemize}
\item {Utilização:Fig.}
\end{itemize}
Pequena ferida, feita na pelle de uma calvagadura pelo roçar dos arreios.
Defeito moral.
(Cp. cast. \textunderscore matadura\textunderscore )
\section{Mata-flôres}
\begin{itemize}
\item {Grp. gram.:m. pl.}
\end{itemize}
\begin{itemize}
\item {Utilização:Prov.}
\end{itemize}
\begin{itemize}
\item {Utilização:minh.}
\end{itemize}
Atilhos da rêde sardinheira.
\section{Mata-fome}
\begin{itemize}
\item {Grp. gram.:m.}
\end{itemize}
\begin{itemize}
\item {Utilização:Bras}
\end{itemize}
Espécie de mandioca.
O mesmo que \textunderscore camapu\textunderscore .
\section{Matagal}
\begin{itemize}
\item {Grp. gram.:m.}
\end{itemize}
\begin{itemize}
\item {Utilização:Fig.}
\end{itemize}
\begin{itemize}
\item {Proveniência:(Do rad. de \textunderscore mata\textunderscore ^1)}
\end{itemize}
Bosque espêsso e grande.
Terreno coberto de plantas bravas.
Conjunto de coisas densas ou erriçádas.
\section{Matagem}
\begin{itemize}
\item {Grp. gram.:f.}
\end{itemize}
\begin{itemize}
\item {Utilização:Prov.}
\end{itemize}
\begin{itemize}
\item {Utilização:alent.}
\end{itemize}
\begin{itemize}
\item {Proveniência:(De \textunderscore mato\textunderscore )}
\end{itemize}
Contrato, que fabricante de carvão vegetal faz com o dono da lenha, a quem paga certa quantia, correspondente a cada saca de carvão.
\section{Matagoso}
\begin{itemize}
\item {Grp. gram.:adj.}
\end{itemize}
\begin{itemize}
\item {Proveniência:(Do rad. de \textunderscore matagal\textunderscore )}
\end{itemize}
Coberto de plantas silvestres.
\section{Matahambre}
\begin{itemize}
\item {Grp. gram.:f.}
\end{itemize}
\begin{itemize}
\item {Utilização:Bras. do S}
\end{itemize}
\begin{itemize}
\item {Proveniência:(Do cast. \textunderscore matar\textunderscore  + \textunderscore hambre\textunderscore )}
\end{itemize}
Carne das reses, extrahida das costellas.
\section{Mataime}
\begin{itemize}
\item {Grp. gram.:m.}
\end{itemize}
O mesmo que \textunderscore matame\textunderscore .
\section{Mata-juntas}
\begin{itemize}
\item {Grp. gram.:m.}
\end{itemize}
\begin{itemize}
\item {Utilização:Prov.}
\end{itemize}
Fasquia ou régua, que se prega a um dos lados da porta, para tapar a fisga ou intervallo, que essa porta deixa, ao fechar-se.
\section{Matal}
\begin{itemize}
\item {Grp. gram.:m.}
\end{itemize}
\begin{itemize}
\item {Utilização:Ant.}
\end{itemize}
\begin{itemize}
\item {Proveniência:(De \textunderscore mata\textunderscore ^1)}
\end{itemize}
O mesmo que \textunderscore matagal\textunderscore .
\section{Mata-leopardos}
\begin{itemize}
\item {Grp. gram.:m.}
\end{itemize}
Espécie de acónito.
\section{Mata-lobos}
\begin{itemize}
\item {Grp. gram.:m.}
\end{itemize}
Planta venenosa, da fam. das ranunculáceas.
\section{Matalotado}
\begin{itemize}
\item {Grp. gram.:adj.}
\end{itemize}
Provido de matolagem.
(De \textunderscore matalote\textunderscore ).
\section{Matalotagem}
\begin{itemize}
\item {Grp. gram.:f.}
\end{itemize}
\begin{itemize}
\item {Utilização:Ext.}
\end{itemize}
\begin{itemize}
\item {Utilização:Fig.}
\end{itemize}
\begin{itemize}
\item {Proveniência:(De \textunderscore matalote\textunderscore )}
\end{itemize}
Provisão de mantimentos para a marinhagem ou para outras pessôas que embarcam, fazendo camaradagem ou rancho.
Provisão de mantimentos.
Amálgama; montão de coisas confusas.
\section{Matalote}
\begin{itemize}
\item {Grp. gram.:m.}
\end{itemize}
\begin{itemize}
\item {Utilização:Ant.}
\end{itemize}
Marinheiro.
Camarada a bordo.
Camarada ou companheiro de serviço.
Navio, que precede outro e lhe serve de baliza para as manobras.
Embarcação ordinária.
Marau, brejeiro.
(Cp. cast. \textunderscore matalote\textunderscore )
\section{Matamatá}
\begin{itemize}
\item {Grp. gram.:m.}
\end{itemize}
Espécie de tartaruga, (\textunderscore testudo fimbria\textunderscore ), indígena de Caiena.
Árvore myrtácea.--B. C. Rubim, \textunderscore Voc. Bras.\textunderscore , manda lêr \textunderscore mata-mata\textunderscore .
\section{Matambu}
\begin{itemize}
\item {Grp. gram.:m.}
\end{itemize}
Árvore silvestre do Brasil, de bôa madeira para trabalhos de carpintaria.
\section{Matame}
\begin{itemize}
\item {Grp. gram.:m.}
\end{itemize}
\begin{itemize}
\item {Utilização:Bras}
\end{itemize}
Córtes angulares na extremidade de toalhas, de lençoes, camisas do mulher, etc.
\section{Mata-me-embora}
\begin{itemize}
\item {Grp. gram.:m.}
\end{itemize}
\begin{itemize}
\item {Utilização:Bras}
\end{itemize}
Espécie de capim.
O mesmo que \textunderscore mate-me-embora\textunderscore .
\section{Matamingo}
\begin{itemize}
\item {Grp. gram.:m.}
\end{itemize}
\begin{itemize}
\item {Utilização:Ant.}
\end{itemize}
Missanga, contas miúdas de vidro.
\section{Mata-moiro}
\begin{itemize}
\item {Grp. gram.:m.}
\end{itemize}
Valentão; fanfarrão; ferrabraz.
\section{Mata-moiros}
\begin{itemize}
\item {Grp. gram.:m.}
\end{itemize}
Valentão; fanfarrão; ferrabraz.
\section{Matamorra}
\begin{itemize}
\item {Grp. gram.:f.}
\end{itemize}
\begin{itemize}
\item {Utilização:Des.}
\end{itemize}
\begin{itemize}
\item {Proveniência:(Do ár. \textunderscore matmora\textunderscore )}
\end{itemize}
O mesmo que \textunderscore masmorra\textunderscore .
Entre os Moiros, celleiro ou tulha subterrânea. Cp. \textunderscore masmorra\textunderscore .
\section{Matança}
\begin{itemize}
\item {Grp. gram.:f.}
\end{itemize}
\begin{itemize}
\item {Utilização:fam.}
\end{itemize}
\begin{itemize}
\item {Utilização:Fig.}
\end{itemize}
\begin{itemize}
\item {Proveniência:(Do b. lat. \textunderscore mactancia\textunderscore )}
\end{itemize}
Acto do matar.
Carnificína, morticínio.
Acto do abater reses para consumo.
Trabalho persitente; afan, matação.
\section{Mata-negro}
\begin{itemize}
\item {Grp. gram.:f.}
\end{itemize}
Espécie de mandioca.
\section{Matante}
\begin{itemize}
\item {Grp. gram.:m.}
\end{itemize}
\begin{itemize}
\item {Utilização:Ant.}
\end{itemize}
\begin{itemize}
\item {Proveniência:(De \textunderscore matar\textunderscore )}
\end{itemize}
Valentão; facínora. Cf. \textunderscore Aulegrafia\textunderscore , 116.
\section{Matante}
\begin{itemize}
\item {Grp. gram.:m.}
\end{itemize}
Espécie de peixe, que abunda na ria de Aveiro.
\section{Matanteria}
\begin{itemize}
\item {Grp. gram.:f.}
\end{itemize}
\begin{itemize}
\item {Utilização:Ant.}
\end{itemize}
\begin{itemize}
\item {Proveniência:(De \textunderscore matante\textunderscore ^1)}
\end{itemize}
Súcia de matantes.
\section{Matapá}
\begin{itemize}
\item {Grp. gram.:m.}
\end{itemize}
\begin{itemize}
\item {Utilização:Bras}
\end{itemize}
Iguaria; o mesmo que \textunderscore vatapá\textunderscore .
\section{Matapasto}
\begin{itemize}
\item {Grp. gram.:m.}
\end{itemize}
Nome de algumas plantas brasileiras, da fam. das leguminosas, e que crescem rasteiras.
\section{Mata-pau}
\begin{itemize}
\item {Grp. gram.:f.}
\end{itemize}
Planta clusiácea do Brasil.
\section{Matapi}
\begin{itemize}
\item {Grp. gram.:m.}
\end{itemize}
\begin{itemize}
\item {Utilização:Bras. do N}
\end{itemize}
Espécie de nassa oblonga.
\section{Mata-piolhos}
\begin{itemize}
\item {Grp. gram.:m.}
\end{itemize}
\begin{itemize}
\item {Utilização:Chul.}
\end{itemize}
Dedo pollegar.
\section{Matapulga}
\begin{itemize}
\item {Grp. gram.:f.}
\end{itemize}
\begin{itemize}
\item {Utilização:Prov.}
\end{itemize}
\begin{itemize}
\item {Utilização:trasm.}
\end{itemize}
Planta, (\textunderscore odontites tenuifolia\textunderscore , Pers.), de que o povo faz vassoiras.
\section{Matar}
\begin{itemize}
\item {Grp. gram.:v. i.}
\end{itemize}
\begin{itemize}
\item {Grp. gram.:Loc.}
\end{itemize}
\begin{itemize}
\item {Utilização:Pop.}
\end{itemize}
\begin{itemize}
\item {Grp. gram.:V. p.}
\end{itemize}
\begin{itemize}
\item {Proveniência:(Do lat. \textunderscore mactare\textunderscore )}
\end{itemize}
Causar a morte a: \textunderscore matar um homem\textunderscore .
Extinguir: \textunderscore matar a fome\textunderscore .
Expungir.
Destruir.
Causar afflicção a.
Importunar; enfadar; cansar.
Tornar sêco: \textunderscore o sol matou as plantas\textunderscore .
Vencer.
Abater (reses) para consumo público.
\textunderscore Matar o bicho\textunderscore , tomar, em jejum, qualquer bebida, ás vezes acompanhada de um pouco de comida.
Suicidar-se.
Empregar grande diligência, afadigar-se; sacrificar-se.
\section{Matarana}
\begin{itemize}
\item {Grp. gram.:f.}
\end{itemize}
\begin{itemize}
\item {Utilização:Bras}
\end{itemize}
Planta amomácea.
Maça de pau rijo, esquinada na parte mais grossa e aguçada na outra extremidade.
\section{Matarão}
\begin{itemize}
\item {Grp. gram.:m.}
\end{itemize}
\begin{itemize}
\item {Utilização:Bras}
\end{itemize}
O mesmo que \textunderscore curandeiro\textunderscore .
\section{Mata-ratos}
\begin{itemize}
\item {Grp. gram.:adj.}
\end{itemize}
\begin{itemize}
\item {Grp. gram.:M.}
\end{itemize}
\begin{itemize}
\item {Utilização:Chul.}
\end{itemize}
Próprio para matar ratos.
Veneno, que mata ratos.
Vinho ordinário.
\section{Matari}
\begin{itemize}
\item {Grp. gram.:m.}
\end{itemize}
Fruto silvestre do Brasil.
\section{Matariz}
\begin{itemize}
\item {Grp. gram.:m.}
\end{itemize}
\begin{itemize}
\item {Utilização:Ant.}
\end{itemize}
Brigão, homem rixoso, desordeiro. Cf. \textunderscore Viriato Trág.\textunderscore , XIV, 71.
\section{Matarotilho}
\begin{itemize}
\item {Grp. gram.:m.}
\end{itemize}
\begin{itemize}
\item {Utilização:Prov.}
\end{itemize}
\begin{itemize}
\item {Utilização:trasm.}
\end{itemize}
Rapagão vadio, estroina, tunante.
\section{Mataru}
\begin{itemize}
\item {Grp. gram.:m.}
\end{itemize}
\begin{itemize}
\item {Utilização:Bras}
\end{itemize}
Vaso de barro para fabricação de azeite de peixe.
\section{Matasano}
\begin{itemize}
\item {fónica:sa}
\end{itemize}
\begin{itemize}
\item {Grp. gram.:m.}
\end{itemize}
O mesmo ou melhor que \textunderscore mata-sanos\textunderscore . Cf. Castilho, \textunderscore Fausto\textunderscore , 76.
\section{Mata-sanos}
\begin{itemize}
\item {Grp. gram.:m.}
\end{itemize}
Curandeiro; medico inhábil. Cf. Macedo, \textunderscore Burros\textunderscore , 198.
(Cast. \textunderscore matasanos\textunderscore )
\section{Mata-são}
\begin{itemize}
\item {Grp. gram.:m.}
\end{itemize}
\begin{itemize}
\item {Utilização:Prov.}
\end{itemize}
\begin{itemize}
\item {Utilização:trasm.}
\end{itemize}
Médico ruím; mata-sanos.
\section{Mata-sete}
\begin{itemize}
\item {Grp. gram.:m.}
\end{itemize}
O mesmo que \textunderscore mata-moiro\textunderscore ; fanfarrão.
\section{Matassa}
\begin{itemize}
\item {Grp. gram.:f.}
\end{itemize}
Seda, antes de fiada.
Seda crua.
\section{Matassano}
\begin{itemize}
\item {Grp. gram.:m.}
\end{itemize}
O mesmo ou melhor que \textunderscore mata-sanos\textunderscore . Cf. Castilho, \textunderscore Fausto\textunderscore , 76.
\section{Matatarana}
\begin{itemize}
\item {Grp. gram.:f.}
\end{itemize}
(V.matarana)
\section{Matataúba}
\begin{itemize}
\item {Grp. gram.:f.}
\end{itemize}
\begin{itemize}
\item {Utilização:Bras}
\end{itemize}
Árvore silvestre, de que se faz carvão.
Ave, o mesmo que \textunderscore sabacuím\textunderscore .
\section{Mate}
\begin{itemize}
\item {Grp. gram.:m.}
\end{itemize}
Lance, no jôgo de xadrez.
Ponto de meia, em que de uma vez se apanham duas malhas, para as estreitar ou fechar.
Remate, perfeição:«\textunderscore ...dá-lhe ás suas virginaes esquivanças o derradeiro mate.\textunderscore »Castilho, \textunderscore Metam.\textunderscore , 304.
(Do persa)
\section{Mate}
\begin{itemize}
\item {Grp. gram.:adj.}
\end{itemize}
\begin{itemize}
\item {Proveniência:(Do al. \textunderscore mast\textunderscore ?)}
\end{itemize}
Diz-se dos metaes, e de certas côres, que não têm brilho ou o perderam.
Fosco; embaciado.
Trigueiro-claro.
\section{Mate}
\begin{itemize}
\item {Grp. gram.:m.}
\end{itemize}
O mesmo que \textunderscore congonha\textunderscore , (\textunderscore ilex paraguariensis\textunderscore , S. Hil.), de que se faz uma espécie de chá, chamado \textunderscore chá mate\textunderscore .
\section{Mate}
\begin{itemize}
\item {Grp. gram.:m.}
\end{itemize}
\begin{itemize}
\item {Utilização:Ant.}
\end{itemize}
Espécie de quilate, com que se avaliava a pureza do oiro em Malaca.
\section{Matear}
\begin{itemize}
\item {Grp. gram.:v. i.}
\end{itemize}
Tomar chá mate^3.
\section{Mateba}
\begin{itemize}
\item {Grp. gram.:f.}
\end{itemize}
Árvore africana, de fibras têxteis, (\textunderscore hiphoene guinensís\textunderscore ).
\section{Matebeira}
\begin{itemize}
\item {Grp. gram.:f.}
\end{itemize}
Árvore de Cabinda, provavelmente o mesmo que \textunderscore mateba\textunderscore .
\section{Matebeles}
\begin{itemize}
\item {Grp. gram.:m. pl.}
\end{itemize}
Numerosa tríbo da África austro-central, também conhecida por \textunderscore zulos\textunderscore .
\section{Mateira}
\begin{itemize}
\item {Grp. gram.:f.}
\end{itemize}
O mesmo que \textunderscore matagal\textunderscore . Cf. Camillo, \textunderscore Doze Casam.\textunderscore , 225.
\section{Mateiro}
\begin{itemize}
\item {Grp. gram.:m.}
\end{itemize}
\begin{itemize}
\item {Utilização:Bras}
\end{itemize}
\begin{itemize}
\item {Proveniência:(De \textunderscore mata\textunderscore ^1)}
\end{itemize}
Guarda de matas.
Explorador de matas; aquelle que se norteia pelas matas, sem bússola.
Aquelle que abre estradas de seringueiras; aquelle quo explora o seringal.
\section{Mateiro}
\begin{itemize}
\item {Grp. gram.:m.}
\end{itemize}
\begin{itemize}
\item {Utilização:T. de Ílhavo}
\end{itemize}
\begin{itemize}
\item {Grp. gram.:m.}
\end{itemize}
\begin{itemize}
\item {Proveniência:(De \textunderscore mato\textunderscore )}
\end{itemize}
Aquelle que tira o estrume dos curraes e sentinas, e o carrega em cestos para os carros.
O mesmo que \textunderscore mateira\textunderscore .
\section{Mateiró}
\begin{itemize}
\item {Grp. gram.:m.}
\end{itemize}
\begin{itemize}
\item {Utilização:T. do Fundão}
\end{itemize}
Pau, atravessado ao fundo da rabiça do arado.
(Cp. \textunderscore teiró\textunderscore )
\section{Matejar}
\begin{itemize}
\item {Grp. gram.:v. i.}
\end{itemize}
Andar no mato; cortar lenha no mato.
\section{Matemática}
\begin{itemize}
\item {Grp. gram.:f.}
\end{itemize}
\begin{itemize}
\item {Grp. gram.:Pl.}
\end{itemize}
\begin{itemize}
\item {Proveniência:(Lat. \textunderscore mathema\textunderscore )}
\end{itemize}
Ciência que tem por objecto os números, as figuras e os intervenientes.
Conjunto das ciências, em que usa as teorias dos números.
\section{Matematicamente}
\begin{itemize}
\item {Grp. gram.:adv.}
\end{itemize}
\begin{itemize}
\item {Utilização:Fig.}
\end{itemize}
De modo matemático.
Segundo as regras da matemática.
Rigorosamente, exactamente, à matemático.
\section{Matemático}
\begin{itemize}
\item {Grp. gram.:adj.}
\end{itemize}
\begin{itemize}
\item {Utilização:Fig.}
\end{itemize}
\begin{itemize}
\item {Grp. gram.:M.}
\end{itemize}
\begin{itemize}
\item {Proveniência:(Lat. \textunderscore mathematicu\textunderscore )}
\end{itemize}
Relativo á matemática.
Muito rigoroso, exacto.
Aquele que é versado em Matemática.
\section{Mate-me-embora}
\begin{itemize}
\item {Grp. gram.:m.}
\end{itemize}
\begin{itemize}
\item {Utilização:Bras}
\end{itemize}
Planta gramínea, medicinal.
\section{Matenda}
\begin{itemize}
\item {Grp. gram.:f.}
\end{itemize}
Árvore angolense do Cazengo.
\section{Mateologia}
\begin{itemize}
\item {Grp. gram.:f.}
\end{itemize}
\begin{itemize}
\item {Proveniência:(Lat. \textunderscore mataeologia\textunderscore )}
\end{itemize}
Estudo inútil de assumptos superiores ao alcance do entendimento humano.
\section{Mateológico}
\begin{itemize}
\item {Grp. gram.:adj.}
\end{itemize}
Relativo á mateologia.
\section{Mateologista}
\begin{itemize}
\item {Grp. gram.:m.}
\end{itemize}
Aquelle que se entrega á mateologia.
\section{Mateólogo}
\begin{itemize}
\item {Grp. gram.:m.}
\end{itemize}
Aquelle que é perito em mateologia.
\section{Mateotechnia}
\begin{itemize}
\item {Grp. gram.:f.}
\end{itemize}
\begin{itemize}
\item {Proveniência:(Lat. \textunderscore mataeotechnia\textunderscore )}
\end{itemize}
Sciência van, phantástica.
\section{Mateotecnia}
\begin{itemize}
\item {Grp. gram.:f.}
\end{itemize}
\begin{itemize}
\item {Proveniência:(Lat. \textunderscore mataeotechnia\textunderscore )}
\end{itemize}
Ciência vã, fantástica.
\section{Matere}
\begin{itemize}
\item {Grp. gram.:f.}
\end{itemize}
\begin{itemize}
\item {Proveniência:(Lat. \textunderscore materis\textunderscore )}
\end{itemize}
Espécie de lança, usada pelos Celtas.
\section{Matéria}
\begin{itemize}
\item {Grp. gram.:f.}
\end{itemize}
\begin{itemize}
\item {Proveniência:(Lat. \textunderscore materia\textunderscore )}
\end{itemize}
Tudo que é palpável e tem corpo e fórma.
Substância, susceptível de receber determinada fórma.
Substância, em que actua determinado agente.
Pus, que se fórma nas feridas.
Objecto, assumpto.
Opportunidade.
Causa.
Pretexto.
Aquillo que se oppõe ás elevadas concepções de um espírito.
\section{Material}
\begin{itemize}
\item {Grp. gram.:adj.}
\end{itemize}
\begin{itemize}
\item {Grp. gram.:M.}
\end{itemize}
\begin{itemize}
\item {Proveniência:(Lat. \textunderscore materialis\textunderscore )}
\end{itemize}
Relativo á matéria.
Que não é espiritual.
Grosseiro: \textunderscore gozos materiaes\textunderscore .
Aquillo que é relativo á matéria.
Conjunto dos objectos que constituem ou formam uma obra, uma construcção, etc.
Mobiliário ou conjunto de utensílios de uma escola, de um estabelecimento, etc.
Armamento ou petrechos militares.
\section{Materialão}
\begin{itemize}
\item {Grp. gram.:m.  e  adj.}
\end{itemize}
\begin{itemize}
\item {Proveniência:(De \textunderscore material\textunderscore )}
\end{itemize}
Indivíduo grosseiramente materialista; bestial.
\section{Materialeira}
\begin{itemize}
\item {Grp. gram.:f.}
\end{itemize}
\begin{itemize}
\item {Utilização:Burl.}
\end{itemize}
\begin{itemize}
\item {Proveniência:(De \textunderscore material\textunderscore )}
\end{itemize}
Coisa material, grosseira.
\section{Materialidade}
\begin{itemize}
\item {Grp. gram.:f.}
\end{itemize}
Qualidade do que é material, estupidez; sentimentos vis.
\section{Materialismo}
\begin{itemize}
\item {Grp. gram.:m.}
\end{itemize}
Systema dos que entendem que tudo é matéria e que não há substância immaterial.
(Do \textunderscore material\textunderscore )
\section{Materialista}
\begin{itemize}
\item {Grp. gram.:m., f. e adj.}
\end{itemize}
\begin{itemize}
\item {Proveniência:(De \textunderscore material\textunderscore )}
\end{itemize}
Pessoa, que é partidária do materialismo.
\section{Materialista}
\begin{itemize}
\item {Grp. gram.:m.}
\end{itemize}
\begin{itemize}
\item {Utilização:Bras. do S}
\end{itemize}
\begin{itemize}
\item {Utilização:Burl.}
\end{itemize}
Mercador de materiaes de construcção.
(Cp. \textunderscore materialista\textunderscore ^1)
\section{Materialístico}
\begin{itemize}
\item {Grp. gram.:adj.}
\end{itemize}
Relativo a materialista.
Próprio de materialista^1.
\section{Materialização}
\begin{itemize}
\item {Grp. gram.:f.}
\end{itemize}
Acto ou effeito de materializar.
\section{Materializador}
\begin{itemize}
\item {Grp. gram.:adj.}
\end{itemize}
O mesmo que \textunderscore materializante\textunderscore .
\section{Materializante}
\begin{itemize}
\item {Grp. gram.:adj.}
\end{itemize}
Que materializa. Cf. Ortigão, \textunderscore Hollanda\textunderscore , 265.
\section{Materializar}
\begin{itemize}
\item {Grp. gram.:v. t.}
\end{itemize}
Considerar material; attribuir qualidades da matéria a.
Tornar estúpido; embrutecer.
\section{Materialmente}
\begin{itemize}
\item {Grp. gram.:adv.}
\end{itemize}
De modo material.
Sob o ponto de vista material; em relação á matéria.
Physicamente.
\section{Maternal}
\begin{itemize}
\item {Grp. gram.:adj.}
\end{itemize}
O mesmo que \textunderscore materno\textunderscore .
\section{Maternalmente}
\begin{itemize}
\item {Grp. gram.:adv.}
\end{itemize}
De modo maternal; á semelhança das mães ou do affecto das mães.
\section{Maternamente}
\begin{itemize}
\item {Grp. gram.:adv.}
\end{itemize}
O mesmo que \textunderscore maternalmente\textunderscore .
\section{Maternidade}
\begin{itemize}
\item {Grp. gram.:f.}
\end{itemize}
Estado ou qualidade de mãe.
Tratamento, dado ás religiosas que têm o titulo de mater.
(De \textunderscore materno\textunderscore ).
\section{Materno}
\begin{itemize}
\item {Grp. gram.:adj.}
\end{itemize}
\begin{itemize}
\item {Utilização:Fig.}
\end{itemize}
\begin{itemize}
\item {Proveniência:(Lat. \textunderscore maternus\textunderscore )}
\end{itemize}
Relativo a mãe.
Próprio de mãe: \textunderscore amor materno\textunderscore .
Relativo á pátria.
Affectuoso; carinhoso.
\section{Matete}
\begin{itemize}
\item {Grp. gram.:m.}
\end{itemize}
\begin{itemize}
\item {Utilização:T. de Angola}
\end{itemize}
Papas de mandioca.
Certas de mel.
\section{Mateto}
\begin{itemize}
\item {Utilização:Prov.}
\end{itemize}
\begin{itemize}
\item {Utilização:trasm.}
\end{itemize}
Qualquer mata.
\section{Mateva}
\begin{itemize}
\item {Grp. gram.:f.}
\end{itemize}
(V.mateba)
\section{Mathambre}
\begin{itemize}
\item {Grp. gram.:m.}
\end{itemize}
\begin{itemize}
\item {Utilização:Bras. do N}
\end{itemize}
O mesmo que \textunderscore matahambre\textunderscore .
\section{Mathemática}
\begin{itemize}
\item {Grp. gram.:f.}
\end{itemize}
\begin{itemize}
\item {Grp. gram.:Pl.}
\end{itemize}
\begin{itemize}
\item {Proveniência:(Lat. \textunderscore mathema\textunderscore )}
\end{itemize}
Sciência que tem por objecto os números, as figuras e os intervenientes.
Conjunto das sciências, em que usa as teorias dos números.
\section{Mathematicamente}
\begin{itemize}
\item {Grp. gram.:adv.}
\end{itemize}
\begin{itemize}
\item {Utilização:Fig.}
\end{itemize}
De modo mathemático.
Segundo as regras da mathemática.
Rigorosamente, exactamente, à mathemático.
\section{Mathemático}
\begin{itemize}
\item {Grp. gram.:adj.}
\end{itemize}
\begin{itemize}
\item {Utilização:Fig.}
\end{itemize}
\begin{itemize}
\item {Grp. gram.:M.}
\end{itemize}
\begin{itemize}
\item {Proveniência:(Lat. \textunderscore mathematicu\textunderscore )}
\end{itemize}
Relativo á mathemática.
Muito rigoroso, exacto.
Aquelle que é versado em Mathemática.
\section{Mathias}
\begin{itemize}
\item {Grp. gram.:m.}
\end{itemize}
Planta brasileira, da fam. das compostas.
\section{Matias}
\begin{itemize}
\item {Grp. gram.:m.}
\end{itemize}
Planta brasileira, da fam. das compostas.
\section{Matias}
\begin{itemize}
\item {Grp. gram.:m.}
\end{itemize}
\begin{itemize}
\item {Utilização:Pop.}
\end{itemize}
\begin{itemize}
\item {Proveniência:(De \textunderscore Mathias\textunderscore , n. p.)}
\end{itemize}
Pateta; palerma.
\section{Mátia}
\begin{itemize}
\item {Grp. gram.:f.}
\end{itemize}
Gênero do plantas borragineas.
\section{Matilha}
\begin{itemize}
\item {Grp. gram.:f.}
\end{itemize}
\begin{itemize}
\item {Utilização:deprec.}
\end{itemize}
\begin{itemize}
\item {Utilização:Fig.}
\end{itemize}
Grupo de cães de caça.
Súcia; corja.
Ajuntamento de maldizentes.
(Parece alter, de um hypoth. \textunderscore motilha\textunderscore , do b. lat. \textunderscore mota\textunderscore , do lat. \textunderscore movere\textunderscore . No b. lat., \textunderscore mota-canum\textunderscore  era o que hoje dizemos \textunderscore matilha de cães\textunderscore )
\section{Matilheiro}
\begin{itemize}
\item {Grp. gram.:m.}
\end{itemize}
\begin{itemize}
\item {Proveniência:(De \textunderscore matilha\textunderscore )}
\end{itemize}
Aquelle que, para a caça, leva galgos á trela.
Aquelle que ensina podengos para a caça.
\section{Matinada}
\begin{itemize}
\item {Grp. gram.:f.}
\end{itemize}
\begin{itemize}
\item {Proveniência:(De \textunderscore matinas\textunderscore )}
\end{itemize}
Madrugada; acto de madrugar.
Canto de matinas.
Estrondo; ruído:«\textunderscore e imdo todos com grande matynada de folguar...\textunderscore »\textunderscore Alvará\textunderscore  de D. Sebast., in \textunderscore Rev. Lus.\textunderscore , XV, 143.
Festa ou espectáculo matinal ou que se faz antes da noite, e que os Franceses e os francesistas chamam \textunderscore matinée\textunderscore .
\section{Matinal}
\begin{itemize}
\item {Grp. gram.:adj.}
\end{itemize}
O mesmo que \textunderscore matutino\textunderscore .
(Cp. \textunderscore matinas\textunderscore )
\section{Marinar}
\begin{itemize}
\item {Grp. gram.:v. i.}
\end{itemize}
\begin{itemize}
\item {Utilização:Fig.}
\end{itemize}
\begin{itemize}
\item {Grp. gram.:V. i.}
\end{itemize}
\begin{itemize}
\item {Proveniência:(Do lat. hypoth. \textunderscore matinus\textunderscore , contr. do lat. \textunderscore matutinus\textunderscore )}
\end{itemize}
Fazer acordar de manhan.
Despertar.
Adestrar.
Insistir em convencer.
Madrugar.
Cantar matinas.
\section{Matinário}
\begin{itemize}
\item {Grp. gram.:m.}
\end{itemize}
Cantor de matinas.
Cantochanista.
\section{Matinas}
\begin{itemize}
\item {Grp. gram.:f. pl.}
\end{itemize}
Primeira parte do offício divino que os padres rezam.
(Cp. \textunderscore matinal\textunderscore )
\section{Matinca}
\begin{itemize}
\item {Grp. gram.:m.}
\end{itemize}
\begin{itemize}
\item {Utilização:Prov.}
\end{itemize}
\begin{itemize}
\item {Utilização:alent.}
\end{itemize}
Espécie de melão vulgar.
\section{Matineiro}
\begin{itemize}
\item {Grp. gram.:m.  e  adj.}
\end{itemize}
\begin{itemize}
\item {Utilização:Ant.}
\end{itemize}
Dizia-se do livro, por onde se rezavam matinas.
\section{Matintaperera}
\begin{itemize}
\item {Grp. gram.:f.}
\end{itemize}
\begin{itemize}
\item {Utilização:Bras. do N}
\end{itemize}
\begin{itemize}
\item {Proveniência:(T. onom.)}
\end{itemize}
Nome de uma ave, que canta de noite.
\section{Matiri}
\begin{itemize}
\item {Grp. gram.:m.}
\end{itemize}
\begin{itemize}
\item {Utilização:Bras. do N}
\end{itemize}
Espécie de saco, feito de fibras de tucum.
\section{Matiz}
\begin{itemize}
\item {Grp. gram.:m.}
\end{itemize}
\begin{itemize}
\item {Utilização:Ext.}
\end{itemize}
\begin{itemize}
\item {Utilização:Fig.}
\end{itemize}
Combinação de côres diversas, num tecido, pintura, paisagem, etc.
Côr mimosa de alguns objectos da natureza.
Gradação de côres.
Colorido no estilo.
Côr politica.
Facção.
(Cast. \textunderscore matiz\textunderscore )
\section{Matização}
\begin{itemize}
\item {Grp. gram.:f.}
\end{itemize}
Acto ou effeito de \textunderscore matizar\textunderscore .
\section{Matizar}
\begin{itemize}
\item {Grp. gram.:v. t.}
\end{itemize}
\begin{itemize}
\item {Utilização:Fig.}
\end{itemize}
\begin{itemize}
\item {Proveniência:(De \textunderscore matiz\textunderscore )}
\end{itemize}
Variar (côres).
Dar côres diversas a.
Adornar.
\section{Matmorra}
\begin{itemize}
\item {fónica:mô}
\end{itemize}
\begin{itemize}
\item {Grp. gram.:f.}
\end{itemize}
O mesmo que \textunderscore matamorra\textunderscore . Cf. Goes, \textunderscore Chrón. de D. Man.\textunderscore , p. III, c. 71.
\section{Mato}
\begin{itemize}
\item {Grp. gram.:m.}
\end{itemize}
\begin{itemize}
\item {Grp. gram.:Loc.}
\end{itemize}
\begin{itemize}
\item {Utilização:bras. do N}
\end{itemize}
Terreno inculto, em que crescem plantas agrestes.
Conjunto de pequenas plantas agrestes.
\textunderscore Botar no mato\textunderscore , deitar fora; desperdiçar.
(Cp. \textunderscore mata\textunderscore ^1)
\section{Mato-bom}
\begin{itemize}
\item {Grp. gram.:m.}
\end{itemize}
\begin{itemize}
\item {Utilização:Bras}
\end{itemize}
Mato, cujo desenvolvimento denuncía a fertilidade do terreno.
\section{Mato-branco}
\begin{itemize}
\item {Grp. gram.:m.}
\end{itemize}
Planta de Cabo-Verde.
\section{Mato-grossense}
\begin{itemize}
\item {Grp. gram.:adj.}
\end{itemize}
\begin{itemize}
\item {Utilização:Bras}
\end{itemize}
\begin{itemize}
\item {Grp. gram.:M.  e  f.}
\end{itemize}
Relativo ao Estado do Mato-Grosso.
Habitante dêsse Estado.
\section{Matolão}
\begin{itemize}
\item {Grp. gram.:m.}
\end{itemize}
\begin{itemize}
\item {Utilização:Bras. do N}
\end{itemize}
Espécie de alforge de coiro, em que os sertanejos conduzem roupa e utensilios de viagem.
(Talvez metáth. de \textunderscore malolão\textunderscore , \textunderscore de mala\textunderscore )
\section{Mato-mau}
\begin{itemize}
\item {Grp. gram.:m.}
\end{itemize}
\begin{itemize}
\item {Utilização:Bras}
\end{itemize}
O mesmo que \textunderscore caíva\textunderscore .
\section{Matombo}
\begin{itemize}
\item {Grp. gram.:m.}
\end{itemize}
\begin{itemize}
\item {Utilização:Bras}
\end{itemize}
Cova, em que se planta de estaca a mandioca.
\section{Matomozumos}
\begin{itemize}
\item {Grp. gram.:m. pl.}
\end{itemize}
Uma das categorias, em que se divide o séquito do soba dos Jingas.
\section{Matonice}
\begin{itemize}
\item {Grp. gram.:f.}
\end{itemize}
Espécie de febre perniciosa de Lourenço-Marques.
\section{Matoninha}
\begin{itemize}
\item {Grp. gram.:f.}
\end{itemize}
\begin{itemize}
\item {Utilização:Prov.}
\end{itemize}
O mesmo que \textunderscore abibe\textunderscore .
\section{Matorral}
\begin{itemize}
\item {Grp. gram.:m.}
\end{itemize}
\begin{itemize}
\item {Utilização:T. da Bairrada}
\end{itemize}
\begin{itemize}
\item {Proveniência:(De um hypoth. \textunderscore matorro\textunderscore , mato grande)}
\end{itemize}
Terreno coberto de mato espêsso e alto.
\section{Matoso}
\begin{itemize}
\item {Grp. gram.:adj.}
\end{itemize}
Coberto de mato; em que há mato.
\section{Matraca}
\begin{itemize}
\item {Grp. gram.:f.}
\end{itemize}
\begin{itemize}
\item {Utilização:Fig.}
\end{itemize}
\begin{itemize}
\item {Grp. gram.:Loc.}
\end{itemize}
\begin{itemize}
\item {Utilização:ant.}
\end{itemize}
\begin{itemize}
\item {Proveniência:(Do ár. \textunderscore mitraca\textunderscore )}
\end{itemize}
Instrumento de madeira, formado de tabuínhas movediças, que se agitam para fazer barulho.
Motejo; troça; vaias.
\textunderscore Dar matraca\textunderscore , dar pateada. Cf. B. Pereira, \textunderscore Prosódia\textunderscore , vb. \textunderscore obstrepito\textunderscore .
\section{Matracar}
\begin{itemize}
\item {Grp. gram.:v. i.}
\end{itemize}
\begin{itemize}
\item {Utilização:Prov.}
\end{itemize}
\begin{itemize}
\item {Utilização:beir.}
\end{itemize}
\begin{itemize}
\item {Utilização:Fig.}
\end{itemize}
\begin{itemize}
\item {Grp. gram.:V. t.}
\end{itemize}
\begin{itemize}
\item {Proveniência:(De \textunderscore matraca\textunderscore )}
\end{itemize}
Bater com fôrça á porta de uma casa, para chamar a attenção e para que de dentro abram a mesma porta.
Insistir nalguma coisa, impertinentemente.
Matraquear.
Repetir monotonamente.
Enfadar, maçar.
\section{Matral}
\begin{itemize}
\item {Grp. gram.:adj.}
\end{itemize}
\begin{itemize}
\item {Grp. gram.:M. pl.}
\end{itemize}
Relativo a mãe. Cf. Castilho, \textunderscore Fastos\textunderscore , III, 145.
O mesmo que \textunderscore matrália\textunderscore .
\section{Matrália}
\begin{itemize}
\item {Grp. gram.:f.}
\end{itemize}
\begin{itemize}
\item {Proveniência:(Lat. \textunderscore matralia\textunderscore )}
\end{itemize}
Antiga festa, que os Romanos celebravam a 11 de Junho, em honra da deusa Matuta, (\textunderscore mater Matuta\textunderscore ), que presidia ao romper da manhan.
\section{Matraqueado}
\begin{itemize}
\item {Grp. gram.:adj.}
\end{itemize}
Experiente; experimentado; matreiro.
\section{Matraqueador}
\begin{itemize}
\item {Grp. gram.:m.}
\end{itemize}
Aquelle que matraqueia.
\section{Matraquear}
\begin{itemize}
\item {Grp. gram.:v. t.}
\end{itemize}
\begin{itemize}
\item {Utilização:Pop.}
\end{itemize}
\begin{itemize}
\item {Proveniência:(De \textunderscore matraca\textunderscore )}
\end{itemize}
Dirigir vaias a; apupar.
Amotinar.
Ensinar; habituar, tornar experiente.
\section{Matraxi}
\begin{itemize}
\item {Grp. gram.:m.}
\end{itemize}
\begin{itemize}
\item {Utilização:Ant.}
\end{itemize}
\begin{itemize}
\item {Proveniência:(T. ár.)}
\end{itemize}
Aguadeiro, entre os Turcos:«\textunderscore andão continuamente homens pela rua a que chamam matraxis, com odres cheios de água...\textunderscore »Godinho, \textunderscore Viagem da Índia\textunderscore , L. I, c. XXV, 161.
\section{Matraz}
\begin{itemize}
\item {Grp. gram.:m.}
\end{itemize}
Retorta, vaso de vidro, usado em operações chímicas.
(Cp. cast. \textunderscore matraz\textunderscore , it. \textunderscore matraccio\textunderscore )
\section{Matreirice}
\begin{itemize}
\item {Grp. gram.:f.}
\end{itemize}
Qualidade de matreiro.
\section{Matreiro}
\begin{itemize}
\item {Grp. gram.:adj.}
\end{itemize}
\begin{itemize}
\item {Grp. gram.:M.}
\end{itemize}
\begin{itemize}
\item {Utilização:Bras. do N}
\end{itemize}
Astuto; matraqueado; que tem grande experiência.
Espécie de veado.
(Cast. \textunderscore matrero\textunderscore )
\section{Mátri}
\begin{itemize}
\item {Grp. gram.:m.}
\end{itemize}
Planta da ilha de San-Thomé, semelhante á celga.
\section{Mátria}
\begin{itemize}
\item {Grp. gram.:f.}
\end{itemize}
\begin{itemize}
\item {Proveniência:(Do lat. \textunderscore mater\textunderscore )}
\end{itemize}
Substituição arbitrária do t. \textunderscore pátria\textunderscore :«\textunderscore Se a pátria se derivara da terra, que é a mãe que nos cria, havia-se de chamar mátria.\textunderscore »Vieira, VI, 288.
\section{Matriarca}
\begin{itemize}
\item {Grp. gram.:f.}
\end{itemize}
\begin{itemize}
\item {Proveniência:(T. hybr., do lat. \textunderscore mater\textunderscore , + gr. \textunderscore arkhe\textunderscore )}
\end{itemize}
A mulher, segundo o sistema sociológico dos que a consideram base da família. Cf. Camillo, \textunderscore Gen. Carlos Rib.\textunderscore , 58.
\section{Matriarcado}
\begin{itemize}
\item {Grp. gram.:m.}
\end{itemize}
\begin{itemize}
\item {Proveniência:(De \textunderscore matriarca\textunderscore )}
\end{itemize}
Primeiro tipo da organização social, em que a mulher é a base da família, trasm.tindo-se a herança pela linha colateral uterina.
\section{Matriarcha}
\begin{itemize}
\item {fónica:ca}
\end{itemize}
\begin{itemize}
\item {Grp. gram.:f.}
\end{itemize}
\begin{itemize}
\item {Proveniência:(T. hybr., do lat. \textunderscore mater\textunderscore , + gr. \textunderscore arkhe\textunderscore )}
\end{itemize}
A mulher, segundo o systema sociológico dos que a consideram base da família. Cf. Camillo, \textunderscore Gen. Carlos Rib.\textunderscore , 58.
\section{Matriarchado}
\begin{itemize}
\item {fónica:ca}
\end{itemize}
\begin{itemize}
\item {Grp. gram.:m.}
\end{itemize}
\begin{itemize}
\item {Proveniência:(De \textunderscore matriarcha\textunderscore )}
\end{itemize}
Primeiro typo da organização social, em que a mulher é a base da família, trasm.ttindo-se a herança pela linha collateral uterina.
\section{Matricária}
\begin{itemize}
\item {Grp. gram.:f.}
\end{itemize}
\begin{itemize}
\item {Proveniência:(Lat. \textunderscore matricaria\textunderscore )}
\end{itemize}
Nome de algumas plantas da fam. das compostas.
\section{Matricida}
\begin{itemize}
\item {Grp. gram.:m. ,  f.  e  adj.}
\end{itemize}
\begin{itemize}
\item {Proveniência:(Lat. \textunderscore matricida\textunderscore )}
\end{itemize}
Pessôa, que commeteu matricidio.
\section{Matricídio}
\begin{itemize}
\item {Grp. gram.:m.}
\end{itemize}
\begin{itemize}
\item {Proveniência:(Lat. \textunderscore matricidium\textunderscore )}
\end{itemize}
Crime de quem mata sua própria mãe.
\section{Matrícula}
\begin{itemize}
\item {Grp. gram.:f.}
\end{itemize}
\begin{itemize}
\item {Proveniência:(Lat. \textunderscore matricula\textunderscore )}
\end{itemize}
Relação de pessôas, sujeitas a certos serviços ou encargos.
Acto de matricular.
Propina ou quantia, paga por quem se matricúla numa escola.
\section{Matriculado}
\begin{itemize}
\item {Grp. gram.:adj.}
\end{itemize}
\begin{itemize}
\item {Utilização:Pop.}
\end{itemize}
\begin{itemize}
\item {Proveniência:(De \textunderscore matricular\textunderscore )}
\end{itemize}
Experiente, matraqueado.
\section{Matricular}
\begin{itemize}
\item {Grp. gram.:v. t.}
\end{itemize}
Inscrever em matrícula.
\section{Matrimonial}
\begin{itemize}
\item {Grp. gram.:adj.}
\end{itemize}
\begin{itemize}
\item {Proveniência:(Lat. \textunderscore matrimonialis\textunderscore )}
\end{itemize}
Relativo a matrimónio.
\section{Matrimonialmente}
\begin{itemize}
\item {Grp. gram.:adv.}
\end{itemize}
De modo matrimonial; á maneira de cônjuges.
\section{Matrimoniamento}
\begin{itemize}
\item {Grp. gram.:m.}
\end{itemize}
Acto de matrimoniar.
O mesmo que \textunderscore casamento\textunderscore . Cf. Camillo, \textunderscore Sc. da Foz\textunderscore , 108.
\section{Matrimoniar}
\begin{itemize}
\item {Grp. gram.:v. t.}
\end{itemize}
Ligar pelo matrimónio; casar.
\section{Matrimónio}
\begin{itemize}
\item {Grp. gram.:m.}
\end{itemize}
\begin{itemize}
\item {Proveniência:(Lat. \textunderscore matrimonium\textunderscore )}
\end{itemize}
União legítima de homem com mulher; casamento.
\section{Matrinchan}
\begin{itemize}
\item {Grp. gram.:m.}
\end{itemize}
\begin{itemize}
\item {Utilização:Bras}
\end{itemize}
Peixe do Purus.
\section{Matrindingue}
\begin{itemize}
\item {Grp. gram.:m.}
\end{itemize}
Espécie de gafanhoto de Ambaca.
\section{Mátrio}
\begin{itemize}
\item {Grp. gram.:adj.}
\end{itemize}
\begin{itemize}
\item {Utilização:Neol.}
\end{itemize}
Relativo a mãe:«\textunderscore o mátrio poder.\textunderscore »Th. Braga, \textunderscore Modernas Ideias\textunderscore , II, 486.
(Palavra, criada por analogia de \textunderscore pátrio\textunderscore , lat. \textunderscore patrius\textunderscore )
\section{Matritense}
\begin{itemize}
\item {Grp. gram.:adj.}
\end{itemize}
\begin{itemize}
\item {Proveniência:(Do b. lat. \textunderscore Matritum\textunderscore , n. p.)}
\end{itemize}
O mesmo que \textunderscore madrileno\textunderscore .
\section{Matriz}
\begin{itemize}
\item {Grp. gram.:f.}
\end{itemize}
\begin{itemize}
\item {Grp. gram.:Adj.}
\end{itemize}
\begin{itemize}
\item {Proveniência:(Lat. \textunderscore matrix\textunderscore )}
\end{itemize}
Órgão da mulher e da fêmea dos mammíferos, em que se gera o féto.
Madre, útero.
Lugar onde alguma coisa se gera.
Manancial.
Molde, para fundição de caracteres typográphicos.
Arrolamento de prédios ou de pessôas sujeitas a contribuição.
Barca, usada no Doiro.
Que é fonte ou origem.
Diz-se da igreja, que tem jurisdicção ou superioridade em relação a outras igrejas ou a todas as capellas de uma dada circunscripção.
Diz-se de uma língua, de que se formaram outras.
Superior; principal.
Fórma, correspondente ao fr. \textunderscore cliché\textunderscore , em photographia, e que torna desnecessário, em português, o uso daquelle termo estrangeiro.
\section{Matroca}
\begin{itemize}
\item {Grp. gram.:f.}
\end{itemize}
Us. na loc. \textunderscore á matroca\textunderscore , significando \textunderscore ao acaso\textunderscore , \textunderscore á tôa\textunderscore .
\section{Matroco}
\begin{itemize}
\item {fónica:trô}
\end{itemize}
\begin{itemize}
\item {Grp. gram.:m.}
\end{itemize}
\begin{itemize}
\item {Utilização:Prov.}
\end{itemize}
\begin{itemize}
\item {Utilização:beir.}
\end{itemize}
\begin{itemize}
\item {Utilização:T. da Bairrada}
\end{itemize}
\begin{itemize}
\item {Grp. gram.:Adj.}
\end{itemize}
Indivíduo desajeitado, baixo e gordo.
Milhano grande.
Matreiro, finório.
Ronso.
\section{Matrona}
\begin{itemize}
\item {Grp. gram.:f.}
\end{itemize}
\begin{itemize}
\item {Utilização:Fam.}
\end{itemize}
\begin{itemize}
\item {Proveniência:(Lat. \textunderscore matrona\textunderscore )}
\end{itemize}
Mulher respeitável, por idade, estado ou procedimento exemplar.
Virago.
\section{Matronaça}
\begin{itemize}
\item {Grp. gram.:f.}
\end{itemize}
\begin{itemize}
\item {Utilização:Fam.}
\end{itemize}
\begin{itemize}
\item {Proveniência:(De \textunderscore matrona\textunderscore )}
\end{itemize}
Mulher gorda e corpulenta.
\section{Matronal}
\begin{itemize}
\item {Grp. gram.:adj.}
\end{itemize}
\begin{itemize}
\item {Proveniência:(Lat. \textunderscore matronalis\textunderscore )}
\end{itemize}
Relativo a matrona.
\section{Matronaria}
\begin{itemize}
\item {Grp. gram.:f.}
\end{itemize}
Qualidade de matrona.
As matronas. Cf. F. Manuel, \textunderscore Carta de Guia\textunderscore , 133.
\section{Matrucadela}
\begin{itemize}
\item {Grp. gram.:f.}
\end{itemize}
Acto de \textunderscore matrucar\textunderscore .
\section{Matrucar}
\begin{itemize}
\item {Grp. gram.:v. t.}
\end{itemize}
\begin{itemize}
\item {Utilização:Prov.}
\end{itemize}
\begin{itemize}
\item {Utilização:trasm.}
\end{itemize}
Pisar; amachucar.
\section{Matruz}
\begin{itemize}
\item {Grp. gram.:m.}
\end{itemize}
\begin{itemize}
\item {Utilização:Bras}
\end{itemize}
O mesmo que \textunderscore erva-formigueira\textunderscore .
(Relaciona-se com \textunderscore mastruço\textunderscore ?)
\section{Máttia}
\begin{itemize}
\item {Grp. gram.:f.}
\end{itemize}
Gênero do plantas borragineas.
\section{Mátula}
\begin{itemize}
\item {Grp. gram.:f.}
\end{itemize}
\begin{itemize}
\item {Proveniência:(Lat. \textunderscore matula\textunderscore )}
\end{itemize}
Designação antiga do vaso, em que se urina.
\section{Matula}
\begin{itemize}
\item {Grp. gram.:f.}
\end{itemize}
\begin{itemize}
\item {Utilização:Pop.}
\end{itemize}
\begin{itemize}
\item {Grp. gram.:M.}
\end{itemize}
\begin{itemize}
\item {Utilização:T. de Gaia}
\end{itemize}
Súcia, corja; matulagem.
Aquelle que trabalha em armazens de vinho, trasfegando, lotando, etc.
(Cp. \textunderscore matulagem\textunderscore )
\section{Matula}
\begin{itemize}
\item {Grp. gram.:f.}
\end{itemize}
\begin{itemize}
\item {Utilização:Bras}
\end{itemize}
Farnel, alforge com comida.
(É possível que tenha a mesma or. que \textunderscore matula\textunderscore ^1. Cp. \textunderscore matalotagem\textunderscore , no sentido de \textunderscore provisões\textunderscore )
\section{Matula}
\begin{itemize}
\item {Grp. gram.:f.}
\end{itemize}
\begin{itemize}
\item {Utilização:Ant.}
\end{itemize}
\begin{itemize}
\item {Utilização:Prov.}
\end{itemize}
\begin{itemize}
\item {Utilização:alent.}
\end{itemize}
Torcida de candeeiro. Cf. \textunderscore Roteiro de V. da Gama\textunderscore , Nunes do Leão, etc.
Trapo, embebido em azeite, para acender o lume.
(Cp. \textunderscore medulla\textunderscore )
\section{Matulagem}
\begin{itemize}
\item {Grp. gram.:f.}
\end{itemize}
Bando de vadios; os vadios.
Vida de vadio; vadiagem.
\section{Matulão}
\begin{itemize}
\item {Grp. gram.:m.}
\end{itemize}
\begin{itemize}
\item {Proveniência:(De \textunderscore matula\textunderscore ^1)}
\end{itemize}
Vadio, estroina.
Rapagão.
\section{Matulaz}
\begin{itemize}
\item {Grp. gram.:m.}
\end{itemize}
\begin{itemize}
\item {Utilização:Des.}
\end{itemize}
O mesmo que \textunderscore matulão\textunderscore .
\section{Matulo}
\begin{itemize}
\item {Grp. gram.:m.}
\end{itemize}
\begin{itemize}
\item {Utilização:Ant.}
\end{itemize}
\begin{itemize}
\item {Utilização:Pop.}
\end{itemize}
Homem grosseiro.
Vadio.
(Cp. \textunderscore matula\textunderscore ^1)
\section{Matulo}
\begin{itemize}
\item {Grp. gram.:m.}
\end{itemize}
\begin{itemize}
\item {Utilização:Ant.}
\end{itemize}
O mesmo que \textunderscore matula\textunderscore ^3.
\section{Matumbo}
\begin{itemize}
\item {Grp. gram.:m.}
\end{itemize}
\begin{itemize}
\item {Utilização:Bras. do N}
\end{itemize}
Cova, o mesmo que \textunderscore matombo\textunderscore .
\section{Matundo}
\begin{itemize}
\item {Grp. gram.:m.}
\end{itemize}
Representação do deus masculino, no Congo.
\section{Matungada}
\begin{itemize}
\item {Grp. gram.:f.}
\end{itemize}
Porção de matungos.
\section{Matungo}
\begin{itemize}
\item {Grp. gram.:m.  e  adj.}
\end{itemize}
\begin{itemize}
\item {Utilização:Bras. do S}
\end{itemize}
Diz-se do cavallo velho ou inútil.
\section{Matupá}
\begin{itemize}
\item {Grp. gram.:m.}
\end{itemize}
\begin{itemize}
\item {Utilização:Bras}
\end{itemize}
\begin{itemize}
\item {Proveniência:(T. tupi)}
\end{itemize}
Grupo compacto de capim aquático, á beira dos rios e lagos.
\section{Maturação}
\begin{itemize}
\item {Grp. gram.:f.}
\end{itemize}
\begin{itemize}
\item {Proveniência:(Lat. \textunderscore maturatio\textunderscore )}
\end{itemize}
Estado do que se acha amadurecido; acto de maturar.
\section{Maturado}
\begin{itemize}
\item {Grp. gram.:adj.}
\end{itemize}
\begin{itemize}
\item {Utilização:Fig.}
\end{itemize}
Sazonado.
Que se tornou circunspecto pelo estudo ou pela experiência.
\section{Maturar}
\begin{itemize}
\item {Grp. gram.:v. i.}
\end{itemize}
\begin{itemize}
\item {Grp. gram.:V. i.  e  p.}
\end{itemize}
\begin{itemize}
\item {Utilização:Fig.}
\end{itemize}
\begin{itemize}
\item {Proveniência:(Lat. \textunderscore maturare\textunderscore )}
\end{itemize}
Tornar maduro.
Amadurecer.
Tornar-se sisudo ou circunspecto, pelo andar dos annos, pelo estudo ou pela experiência.
\section{Maturativo}
\begin{itemize}
\item {Grp. gram.:adj.}
\end{itemize}
\begin{itemize}
\item {Proveniência:(De \textunderscore maturar\textunderscore )}
\end{itemize}
Que auxilia ou produz a maturação.
Que promove a suppuração.
\section{Maturauás}
\begin{itemize}
\item {Grp. gram.:m. pl.}
\end{itemize}
Indígenas do norte do Brasil.
\section{Maturescência}
\begin{itemize}
\item {Grp. gram.:f.}
\end{itemize}
\begin{itemize}
\item {Proveniência:(Do lat. \textunderscore maturescere\textunderscore )}
\end{itemize}
Qualidade ou estado do que é maduro.
\section{Maturi}
\begin{itemize}
\item {Grp. gram.:m.}
\end{itemize}
O mesmo que \textunderscore muturi\textunderscore .
\section{Maturidade}
\begin{itemize}
\item {Grp. gram.:f.}
\end{itemize}
\begin{itemize}
\item {Utilização:Fig.}
\end{itemize}
\begin{itemize}
\item {Utilização:Bras}
\end{itemize}
\begin{itemize}
\item {Proveniência:(Lat. \textunderscore maturitas\textunderscore )}
\end{itemize}
O mesmo que \textunderscore madureza\textunderscore .
Idade madura.
Perfeição.
\textunderscore Exame de maturidade\textunderscore , exame de madureza.
\section{Maturo}
\begin{itemize}
\item {Grp. gram.:adj.}
\end{itemize}
\begin{itemize}
\item {Utilização:Ant.}
\end{itemize}
\begin{itemize}
\item {Proveniência:(Lat. \textunderscore maturus\textunderscore )}
\end{itemize}
O mesmo que \textunderscore maduro\textunderscore ^1.
\section{Maturrangas}
\begin{itemize}
\item {Grp. gram.:f. pl.}
\end{itemize}
\begin{itemize}
\item {Utilização:Prov.}
\end{itemize}
\begin{itemize}
\item {Utilização:trasm.}
\end{itemize}
\textunderscore Dar nas maturrangas\textunderscore , descobrir as manhas.
Tocar no ponto vulnerável.
Achar a solução de uma difficuldade.
(Cp. \textunderscore maturrango\textunderscore )
\section{Maturrango}
\begin{itemize}
\item {Grp. gram.:adj.}
\end{itemize}
\begin{itemize}
\item {Utilização:Bras. do S}
\end{itemize}
\begin{itemize}
\item {Proveniência:(De \textunderscore maturrão\textunderscore )}
\end{itemize}
Mau cavalleiro.
\section{Maturrão}
\begin{itemize}
\item {Grp. gram.:m.}
\end{itemize}
\begin{itemize}
\item {Utilização:Bras}
\end{itemize}
Bêsta velha, aleijada ou cega.
(Cp. \textunderscore matungo\textunderscore )
\section{Maturrengo}
\begin{itemize}
\item {Grp. gram.:adj.}
\end{itemize}
\begin{itemize}
\item {Utilização:Bras. do S}
\end{itemize}
O mesmo que \textunderscore maturrango\textunderscore .
\section{Matutação}
\begin{itemize}
\item {Grp. gram.:f.}
\end{itemize}
\begin{itemize}
\item {Utilização:Fam.}
\end{itemize}
Acto de matutar; scisma.
\section{Matuta-e-meia}
\begin{itemize}
\item {Grp. gram.:f.}
\end{itemize}
(Outra fórma da loc. \textunderscore uma tuta-e-meia\textunderscore . Cp. \textunderscore tuta-e-meia\textunderscore )
\section{Matutalimoi}
\begin{itemize}
\item {Grp. gram.:m.}
\end{itemize}
Espécie de escaravelho africano.
\section{Matutar}
\begin{itemize}
\item {Grp. gram.:v. i.}
\end{itemize}
\begin{itemize}
\item {Utilização:Chul.}
\end{itemize}
Meditar; reflectir.
Têr uma ideia fixa; scismar.
(De \textunderscore matuto\textunderscore ).
\section{Matutice}
\begin{itemize}
\item {Grp. gram.:f.}
\end{itemize}
Maneiras ou actos próprios de matuto.
\section{Matutinal}
\begin{itemize}
\item {Grp. gram.:adj.}
\end{itemize}
O mesmo que \textunderscore matutino\textunderscore .
\section{Matutinário}
\begin{itemize}
\item {Grp. gram.:m.}
\end{itemize}
\begin{itemize}
\item {Proveniência:(De \textunderscore matutino\textunderscore )}
\end{itemize}
Livro de matinas; matineiro.
\section{Matutino}
\begin{itemize}
\item {Grp. gram.:adj.}
\end{itemize}
\begin{itemize}
\item {Grp. gram.:M. pl.}
\end{itemize}
\begin{itemize}
\item {Utilização:Ant.}
\end{itemize}
\begin{itemize}
\item {Proveniência:(Lat. \textunderscore matutinus\textunderscore )}
\end{itemize}
Relativo á manhan: \textunderscore horas matutinas\textunderscore .
Que apparece de manhan: \textunderscore orvalho matutino\textunderscore .
Que se faz ou acontce de manhan.
Madrugador.
O mesmo que \textunderscore matinas\textunderscore .
\section{Matuto}
\begin{itemize}
\item {Grp. gram.:adj.}
\end{itemize}
\begin{itemize}
\item {Utilização:Bras. do N}
\end{itemize}
\begin{itemize}
\item {Utilização:Fig.}
\end{itemize}
\begin{itemize}
\item {Utilização:Fam.}
\end{itemize}
\begin{itemize}
\item {Grp. gram.:M.}
\end{itemize}
\begin{itemize}
\item {Utilização:Bras. do N}
\end{itemize}
\begin{itemize}
\item {Proveniência:(De \textunderscore mato\textunderscore )}
\end{itemize}
Sertanejo; que vive no mato.
Acanhado, tímido.
Maníaco, scismático.
Matreiro; manhoso; finório.
Provinciano.
Roceiro.
Homem ignorante.
\section{Matuto}
\begin{itemize}
\item {Grp. gram.:m.}
\end{itemize}
Árvore africana, de fôlhas simples, pecioladas, e flôres hermaphroditas.
\section{Mau}
\begin{itemize}
\item {Grp. gram.:adj.}
\end{itemize}
\begin{itemize}
\item {Grp. gram.:M.}
\end{itemize}
\begin{itemize}
\item {Grp. gram.:Interj.}
\end{itemize}
\begin{itemize}
\item {Proveniência:(Do lat. \textunderscore malus\textunderscore )}
\end{itemize}
Que é desagradável ou nocivo: \textunderscore tempo mau\textunderscore ; \textunderscore olhar mau\textunderscore .
Que não cumpre os seus deveres: \textunderscore mau filho\textunderscore .
Mal feito: \textunderscore êsse trabalho está mau\textunderscore .
Estragado.
Que offerece obstáculos.
Funesto: \textunderscore mau negócio\textunderscore .
Opposto ao direito ou á razão.
Inconveniente: \textunderscore mau expediente\textunderscore .
Injusto.
Ingrato.
Que faz travessuras: \textunderscore este pequeno é muito mau\textunderscore .
Desastrado; que não tem a conveniente perícia: \textunderscore mau cozinheiro\textunderscore .
Que não presta.
Aquillo que é mau.
Indivíduo de má índole, de maus costumes.
(designativa de reprovação ou descontentamento).
\textunderscore Alto e mau\textunderscore , o mesmo que \textunderscore alto-e-malo\textunderscore .
\section{Mauaiás}
\begin{itemize}
\item {Grp. gram.:m. pl.}
\end{itemize}
Indígenas do norte do Brasil.
\section{Maúba}
\begin{itemize}
\item {Grp. gram.:f.}
\end{itemize}
Árvore silvestre do Brasil.
\section{Mauresco}
\begin{itemize}
\item {fónica:maurês}
\end{itemize}
\begin{itemize}
\item {Grp. gram.:adj.}
\end{itemize}
Relativo a San-Mauro: \textunderscore convento mauresco\textunderscore . Cf. Herculano, \textunderscore Opúsc.\textunderscore , III, 70.
\section{Maués}
\begin{itemize}
\item {Grp. gram.:m. pl.}
\end{itemize}
Indígenas do norte do Brasil.
\section{Maueza}
\begin{itemize}
\item {fónica:ê}
\end{itemize}
\begin{itemize}
\item {Grp. gram.:f.}
\end{itemize}
\begin{itemize}
\item {Utilização:Bras. de Minas}
\end{itemize}
O mesmo que \textunderscore maldade\textunderscore : \textunderscore que maueza de criança!\textunderscore 
\section{Mauíndo}
\begin{itemize}
\item {Grp. gram.:m.}
\end{itemize}
\begin{itemize}
\item {Utilização:T. de Angola}
\end{itemize}
Insecto, que ataca os pés da gente, causando soffrimento, (\textunderscore pulex penetrans\textunderscore ).
\section{Mauís}
\begin{itemize}
\item {Grp. gram.:m. pl.}
\end{itemize}
Indígena do norte do Brasil.
\section{Maújo}
\begin{itemize}
\item {Grp. gram.:m.}
\end{itemize}
Instrumento de calafate, para tirar estôpa das fendas.
\section{Maumetano}
\begin{itemize}
\item {Grp. gram.:m.  e  adj.}
\end{itemize}
O mesmo que \textunderscore mahometano\textunderscore . Cf. \textunderscore Lusíadas\textunderscore , IX, 8 e 12.
\section{Maúnça}
\begin{itemize}
\item {Grp. gram.:f.}
\end{itemize}
\begin{itemize}
\item {Proveniência:(Do b. lat. hyp. \textunderscore manutia\textunderscore , de \textunderscore manus\textunderscore )}
\end{itemize}
Mão-cheia; manipulo.
Pequeno feixe que póde abranger-se com a mão.
\section{Maupataz}
\begin{itemize}
\item {Grp. gram.:m.}
\end{itemize}
Árvore de Cabo-Verde.
\section{Maurá}
\begin{itemize}
\item {Grp. gram.:m.}
\end{itemize}
Árvore indiana, (\textunderscore bassia latifolia\textunderscore ).
Bebida alcoólica, muito vulgar na Índia portuguesa.
\section{Mauriense}
\begin{itemize}
\item {Grp. gram.:adj.}
\end{itemize}
Relativo a San-Mauro: \textunderscore convento mauriense\textunderscore . Cf. Herculano, \textunderscore Opúsc.\textunderscore , III, 70.
\section{Maurino}
\begin{itemize}
\item {Grp. gram.:adj.}
\end{itemize}
Relativo á congregação religiosa de San-Bruno:«\textunderscore ...a celebrada e eruditíssima congregação maurina.\textunderscore »Latino, \textunderscore Elogios\textunderscore , 18.
\section{Mauritânia}
\begin{itemize}
\item {Grp. gram.:f.}
\end{itemize}
Planta caryophyllácea, (\textunderscore diantus barbatus\textunderscore , Lin.).
\section{Mauritano}
\begin{itemize}
\item {Grp. gram.:adj.}
\end{itemize}
\begin{itemize}
\item {Grp. gram.:M.}
\end{itemize}
\begin{itemize}
\item {Proveniência:(Lat. \textunderscore mauritanus\textunderscore )}
\end{itemize}
Relativo á Mauritânia.
Habitante da Mauritânia; moiro.
\section{Mauro}
\begin{itemize}
\item {Grp. gram.:m.}
\end{itemize}
\begin{itemize}
\item {Utilização:Ant.}
\end{itemize}
\begin{itemize}
\item {Grp. gram.:M.  e  adj.}
\end{itemize}
\begin{itemize}
\item {Utilização:Poét.}
\end{itemize}
O mesmo que \textunderscore maravedi\textunderscore .
O mesmo que moiro.
\section{Mausoléo}
\begin{itemize}
\item {Grp. gram.:m.}
\end{itemize}
\begin{itemize}
\item {Utilização:Ext.}
\end{itemize}
\begin{itemize}
\item {Proveniência:(Lat. \textunderscore mausoleum\textunderscore )}
\end{itemize}
Sepulcro de Mausolo.
Sepulcro sumptuoso.
\section{Mausoléu}
\begin{itemize}
\item {Grp. gram.:m.}
\end{itemize}
\begin{itemize}
\item {Utilização:Ext.}
\end{itemize}
\begin{itemize}
\item {Proveniência:(Lat. \textunderscore mausoleum\textunderscore )}
\end{itemize}
Sepulcro de Mausolo.
Sepulcro sumptuoso.
\section{Mávia}
\begin{itemize}
\item {Grp. gram.:m.}
\end{itemize}
Língua de algumas tribos cafreaes.
\section{Maviosamente}
\begin{itemize}
\item {Grp. gram.:adv.}
\end{itemize}
De modo mavioso; com suavidade; deliciosamente.
\section{Maviosidade}
\begin{itemize}
\item {Grp. gram.:f.}
\end{itemize}
Qualidade de mavioso.
\section{Mavioso}
\begin{itemize}
\item {Grp. gram.:adj.}
\end{itemize}
Affectuoso; compassivo.
Agradável aos sentidos.
Suave.
Terno; enternecedor.
(Por \textunderscore amavioso\textunderscore , de \textunderscore amavios\textunderscore )
\section{Mavórcio}
\begin{itemize}
\item {Grp. gram.:adj.}
\end{itemize}
\begin{itemize}
\item {Proveniência:(Lat. \textunderscore mavorlius\textunderscore )}
\end{itemize}
Relativo a Marte; bellicoso.
Aguerrido.
\section{Mavuvi}
\begin{itemize}
\item {Grp. gram.:m.}
\end{itemize}
Espécie de aranha africana.
\section{Maxicote}
\begin{itemize}
\item {Grp. gram.:m.}
\end{itemize}
Argamassa, feita de areia, cal, terra e água. Cf. Júl. Moreira, \textunderscore Estudos da Ling. Port.\textunderscore , I, 191.
(Cp. \textunderscore massicote\textunderscore )
\section{Maxila}
\begin{itemize}
\item {fónica:csi}
\end{itemize}
\begin{itemize}
\item {Grp. gram.:f.}
\end{itemize}
\begin{itemize}
\item {Proveniência:(Lat. \textunderscore maxilla\textunderscore )}
\end{itemize}
Cada uma das peças ósseas, era que estão inseridos os dentes dos animaes vertebrados; queixada.
\section{Maxilar}
\begin{itemize}
\item {fónica:csi}
\end{itemize}
\begin{itemize}
\item {Grp. gram.:adj.}
\end{itemize}
\begin{itemize}
\item {Proveniência:(Lat. \textunderscore maxillaris\textunderscore )}
\end{itemize}
Relativo á maxila.
\section{Maxilária}
\begin{itemize}
\item {fónica:csi}
\end{itemize}
\begin{itemize}
\item {Grp. gram.:f.}
\end{itemize}
\begin{itemize}
\item {Proveniência:(De \textunderscore maxila\textunderscore )}
\end{itemize}
Gênero de orquídeas.
\section{Maxilite}
\begin{itemize}
\item {fónica:csi}
\end{itemize}
\begin{itemize}
\item {Grp. gram.:f.}
\end{itemize}
Inflamação das maxilas.
\section{Maxilla}
\begin{itemize}
\item {fónica:csi}
\end{itemize}
\begin{itemize}
\item {Grp. gram.:f.}
\end{itemize}
\begin{itemize}
\item {Proveniência:(Lat. \textunderscore maxilla\textunderscore )}
\end{itemize}
Cada uma das peças ósseas, era que estão inseridos os dentes dos animaes vertebrados; queixada.
\section{Maxillar}
\begin{itemize}
\item {fónica:csi}
\end{itemize}
\begin{itemize}
\item {Grp. gram.:adj.}
\end{itemize}
\begin{itemize}
\item {Proveniência:(Lat. \textunderscore maxillaris\textunderscore )}
\end{itemize}
Relativo á maxilla.
\section{Maxillária}
\begin{itemize}
\item {fónica:csi}
\end{itemize}
\begin{itemize}
\item {Grp. gram.:f.}
\end{itemize}
\begin{itemize}
\item {Proveniência:(De \textunderscore maxilla\textunderscore )}
\end{itemize}
Gênero de orchídeas.
\section{Maxillite}
\begin{itemize}
\item {fónica:csi}
\end{itemize}
\begin{itemize}
\item {Grp. gram.:f.}
\end{itemize}
Inflammação das maxillas.
\section{Maxillo-dental}
\begin{itemize}
\item {Grp. gram.:adj.}
\end{itemize}
\begin{itemize}
\item {Proveniência:(De \textunderscore maxilla\textunderscore  + \textunderscore dente\textunderscore )}
\end{itemize}
Relativo a dentes e maxillas.
\section{Maxillo-dentário}
\begin{itemize}
\item {Grp. gram.:adj.}
\end{itemize}
\begin{itemize}
\item {Proveniência:(De \textunderscore maxilla\textunderscore  + \textunderscore dente\textunderscore )}
\end{itemize}
Relativo a dentes e maxillas.
\section{Maxillo-labial}
\begin{itemize}
\item {Grp. gram.:adj.}
\end{itemize}
\begin{itemize}
\item {Proveniência:(De \textunderscore maxilla\textunderscore  + \textunderscore lábio\textunderscore )}
\end{itemize}
Relativo a lábios e maxilla.
\section{Maxillo-muscular}
\begin{itemize}
\item {Grp. gram.:adj.}
\end{itemize}
\begin{itemize}
\item {Proveniência:(De \textunderscore maxilla\textunderscore  + \textunderscore muscular\textunderscore )}
\end{itemize}
Relativo aos músculos da maxilla.
\section{Maxilloso}
\begin{itemize}
\item {fónica:csi}
\end{itemize}
\begin{itemize}
\item {Grp. gram.:adj.}
\end{itemize}
Que tem grandes maxillas.
\section{Maxiloso}
\begin{itemize}
\item {fónica:csi}
\end{itemize}
\begin{itemize}
\item {Grp. gram.:adj.}
\end{itemize}
Que tem grandes maxilas.
\section{Maxim}
\begin{itemize}
\item {Grp. gram.:m.}
\end{itemize}
\begin{itemize}
\item {Utilização:T. de Angola}
\end{itemize}
Longa faca para cortar erva.
\section{Máxima}
\begin{itemize}
\item {fónica:cï}
\end{itemize}
\begin{itemize}
\item {Grp. gram.:f.}
\end{itemize}
\begin{itemize}
\item {Proveniência:(Do lat. \textunderscore maximus\textunderscore )}
\end{itemize}
Axioma; sentença moral.
Brocardo; conceito.
Nota musical, com o valor de oito semi-breves.
\section{Maximamente}
\begin{itemize}
\item {fónica:ci}
\end{itemize}
\begin{itemize}
\item {Grp. gram.:adv.}
\end{itemize}
Principalmente; de modo máximo.
\section{Maximário}
\begin{itemize}
\item {fónica:ci}
\end{itemize}
\begin{itemize}
\item {Grp. gram.:m.}
\end{itemize}
\begin{itemize}
\item {Utilização:Neol.}
\end{itemize}
Collecção de máximas.
\section{Maximiliana}
\begin{itemize}
\item {fónica:ci}
\end{itemize}
\begin{itemize}
\item {Grp. gram.:f.}
\end{itemize}
Planeta telescópico, descoberto em 1862.
\section{Maximita}
\begin{itemize}
\item {Grp. gram.:f.}
\end{itemize}
Novo explosivo, inventado por Maxim, fabricante de metralhadoras. Cf. \textunderscore Jorn. do Comm.\textunderscore , do Rio, de 4-VI-901.
\section{Maximite}
\begin{itemize}
\item {Grp. gram.:f.}
\end{itemize}
Novo explosivo, inventado por Maxim, fabricante de metralhadoras. Cf. \textunderscore Jorn. do Comm.\textunderscore , do Rio, de 4-VI-901.
\section{Máximo}
\begin{itemize}
\item {fónica:ci}
\end{itemize}
\begin{itemize}
\item {Grp. gram.:adj.}
\end{itemize}
\begin{itemize}
\item {Grp. gram.:M.}
\end{itemize}
\begin{itemize}
\item {Proveniência:(Lat. \textunderscore maximus\textunderscore )}
\end{itemize}
Que é maior que todos.
Que está acima de todos os da sua espécie ou gênero.
Que é o mais alto; excelso.
Aquillo que é maior e mais alto ou mais intenso.
O ponto mais alto, a que póde subir um preço, um objecto, qualquer coisa.
\section{Maxinjes}
\begin{itemize}
\item {Grp. gram.:m. pl.}
\end{itemize}
Povo sertanejo de Angola.
\section{Maxixe}
\begin{itemize}
\item {Grp. gram.:m.}
\end{itemize}
\begin{itemize}
\item {Utilização:Bras}
\end{itemize}
Fruto de uma planta cucurbitácea.
\section{Maxixe}
\begin{itemize}
\item {Grp. gram.:m.}
\end{itemize}
\begin{itemize}
\item {Utilização:Bras}
\end{itemize}
Espécie de batuque.
\section{Maxixeiro}
\begin{itemize}
\item {Grp. gram.:m.}
\end{itemize}
Planta brasileira, que produz o maxixe^1.
\section{Maxona}
\begin{itemize}
\item {Grp. gram.:m.}
\end{itemize}
Língua falada no centro da África meridional.
\section{Maxoxolo}
\begin{itemize}
\item {Grp. gram.:m.}
\end{itemize}
Ave africana.
\section{Maxueira}
\begin{itemize}
\item {Grp. gram.:adj.}
\end{itemize}
\textunderscore Ambar maxueira\textunderscore , o âmbar pardo.
\section{Maynardina}
\begin{itemize}
\item {Grp. gram.:f.}
\end{itemize}
\begin{itemize}
\item {Proveniência:(De \textunderscore Maynard\textunderscore , n. p.)}
\end{itemize}
Preparação pharmacêutica, para a extracção de callos.
\section{Maz}
\begin{itemize}
\item {Grp. gram.:m.}
\end{itemize}
Antigo pêso de Malaca. Cf. \textunderscore Peregrinação\textunderscore , XXV.
Quantia equivalente a 50 reis, em tempo de Fernão Mendes Pinto. Cf. \textunderscore Peregrinação\textunderscore , LXXXIX e XCI.
\section{Mazá}
\begin{itemize}
\item {Grp. gram.:m.}
\end{itemize}
Verme anelídeo, espécie de sanguesuga, que habita nalgumas lagôas.
\section{Mazagania}
\begin{itemize}
\item {Grp. gram.:f.}
\end{itemize}
\begin{itemize}
\item {Utilização:Ant.}
\end{itemize}
Companhia de soldados moiros:«\textunderscore após elle vinha o alcaide com sua mazagania...\textunderscore »Goes. \textunderscore Chrón. de D. Man.\textunderscore , p. IV, c. XLIV.
(Do ár.)
\section{Mazama}
\begin{itemize}
\item {Grp. gram.:m.}
\end{itemize}
Espécie de veado americano.
\section{Mazan}
\begin{itemize}
\item {Grp. gram.:m.}
\end{itemize}
O mesmo que \textunderscore mazane\textunderscore .
\section{Mazane}
\begin{itemize}
\item {Grp. gram.:m.}
\end{itemize}
\begin{itemize}
\item {Utilização:T. da Índia port}
\end{itemize}
Fundador de um pagode.
Cada um dos membros da família dêsse fundador.
Cada um dos membros da mazania. Cf. Th. Ribeiro, \textunderscore Jornadas\textunderscore , II, 113.
\section{Mazania}
\begin{itemize}
\item {Grp. gram.:f.}
\end{itemize}
\begin{itemize}
\item {Utilização:T. da Índia port}
\end{itemize}
Corporação fabriqueira ou administrativa de um pagode ou templo hindu.
\section{Mazanigas}
\begin{itemize}
\item {Grp. gram.:m. pl.}
\end{itemize}
Tribo africana da costa de Moçambique, uma das grandes divisões dos Ajanas.
\section{Mazanza}
\begin{itemize}
\item {Grp. gram.:m.  e  f.}
\end{itemize}
\begin{itemize}
\item {Utilização:Bras}
\end{itemize}
Pessôa indolente, apalermada.
\section{Mazar}
\begin{itemize}
\item {Grp. gram.:m.}
\end{itemize}
\begin{itemize}
\item {Utilização:Obsol.}
\end{itemize}
O mesmo que \textunderscore madrepérola\textunderscore , segundo se julgou antigamente.
\section{Mazarize}
\begin{itemize}
\item {Grp. gram.:m.}
\end{itemize}
Tejolo grande, empregado em construcções de abóbadas, no Alentejo.
\section{Mazela}
\begin{itemize}
\item {Grp. gram.:f.}
\end{itemize}
\begin{itemize}
\item {Utilização:Fam.}
\end{itemize}
\begin{itemize}
\item {Utilização:Fig.}
\end{itemize}
\begin{itemize}
\item {Proveniência:(Do lat. \textunderscore macella\textunderscore )}
\end{itemize}
Matadura.
Ferida.
Enfermidade.
Tudo que aflige.
Mancha na reputação; labéu.
\section{Mazelar}
\begin{itemize}
\item {Grp. gram.:v. t.}
\end{itemize}
\begin{itemize}
\item {Utilização:Fam.}
\end{itemize}
\begin{itemize}
\item {Utilização:Fig.}
\end{itemize}
Encher de mazelas, chagar.
Molestar.
Infamar, desacreditar.
\section{Mazelento}
\begin{itemize}
\item {Grp. gram.:adj.}
\end{itemize}
Que tem mazelas; achacado. Cf. Garrett, \textunderscore Fábulas\textunderscore , 71.
\section{Mazella}
\begin{itemize}
\item {Grp. gram.:f.}
\end{itemize}
\begin{itemize}
\item {Utilização:Fam.}
\end{itemize}
\begin{itemize}
\item {Utilização:Fig.}
\end{itemize}
\begin{itemize}
\item {Proveniência:(Do lat. \textunderscore macella\textunderscore )}
\end{itemize}
Matadura.
Ferida.
Enfermidade.
Tudo que afflige.
Mancha na reputação; labéu.
\section{Mazellar}
\begin{itemize}
\item {Grp. gram.:v. t.}
\end{itemize}
\begin{itemize}
\item {Utilização:Fam.}
\end{itemize}
\begin{itemize}
\item {Utilização:Fig.}
\end{itemize}
Encher de mazellas, chagar.
Molestar.
Infamar, desacreditar.
\section{Mazellento}
\begin{itemize}
\item {Grp. gram.:adj.}
\end{itemize}
Que tem mazellas; achacado. Cf. Garrett, \textunderscore Fábulas\textunderscore , 71.
\section{Mazera}
\begin{itemize}
\item {Grp. gram.:f.}
\end{itemize}
\begin{itemize}
\item {Utilização:Ant.}
\end{itemize}
Espécie de taficira.
\section{Mazombice}
\begin{itemize}
\item {Grp. gram.:f.}
\end{itemize}
\begin{itemize}
\item {Utilização:Fam.}
\end{itemize}
Qualidade de mazombo ou sorumbático; mau humor.
\section{Mazombo}
\begin{itemize}
\item {Grp. gram.:m.}
\end{itemize}
\begin{itemize}
\item {Utilização:Bras}
\end{itemize}
\begin{itemize}
\item {Utilização:Deprec.}
\end{itemize}
\begin{itemize}
\item {Grp. gram.:m.  e  adj.}
\end{itemize}
\begin{itemize}
\item {Utilização:Fig.}
\end{itemize}
Aquelle que nasce de pessôas europeias no Brasil.
Indivíduo tristonho; mal humorado, sorumbático.
(Talvez t. afr.)
\section{Mazorral}
\begin{itemize}
\item {Grp. gram.:adj.}
\end{itemize}
Grosseiro; incivil.
(Cast. \textunderscore mazorral\textunderscore )
\section{Mazorro}
\begin{itemize}
\item {fónica:zô}
\end{itemize}
\begin{itemize}
\item {Grp. gram.:adj.}
\end{itemize}
\begin{itemize}
\item {Grp. gram.:M.}
\end{itemize}
Mazorral.
Preguiçoso.
Sorumbático.
Indivíduo mazorral.
\section{Mazurca}
\begin{itemize}
\item {Grp. gram.:f.}
\end{itemize}
Dança nacional da Polónia, a três tempos, em movimento moderado.
\section{Mazurcar}
\begin{itemize}
\item {Grp. gram.:v. i.}
\end{itemize}
Dançar mazurca.
\section{Me}
\begin{itemize}
\item {Grp. gram.:pron.}
\end{itemize}
\begin{itemize}
\item {Proveniência:(Lat. \textunderscore me\textunderscore )}
\end{itemize}
A mim.--É um dos casos do pron. \textunderscore eu\textunderscore , e emprega-se geralmente como complemento directo ou terminativo.
\section{Meã}
\begin{itemize}
\item {Grp. gram.:f.}
\end{itemize}
\begin{itemize}
\item {Utilização:Prov.}
\end{itemize}
\begin{itemize}
\item {Utilização:beir.}
\end{itemize}
Anilha de coiro, que liga o mango ao pírtigo do mangual.
Variedade de azeitona, o mesmo que \textunderscore negral\textunderscore  ou \textunderscore madural\textunderscore .
\section{Meaça}
\begin{itemize}
\item {Grp. gram.:f.}
\end{itemize}
\begin{itemize}
\item {Utilização:Ant.}
\end{itemize}
O mesmo que \textunderscore ameaça\textunderscore .
\section{Meação}
\begin{itemize}
\item {Grp. gram.:f.}
\end{itemize}
\begin{itemize}
\item {Proveniência:(De \textunderscore mear\textunderscore )}
\end{itemize}
Divisão em duas partes.
\section{Meaçar}
\begin{itemize}
\item {Grp. gram.:v. t.}
\end{itemize}
\begin{itemize}
\item {Utilização:Ant.}
\end{itemize}
O mesmo que \textunderscore ameaçar\textunderscore .
\section{Meaco}
\begin{itemize}
\item {Grp. gram.:m.}
\end{itemize}
Tolda de algumas embarcações asiáticas.
\section{Meada}
\begin{itemize}
\item {Grp. gram.:f.}
\end{itemize}
\begin{itemize}
\item {Utilização:Fig.}
\end{itemize}
\begin{itemize}
\item {Proveniência:(De \textunderscore mear\textunderscore )}
\end{itemize}
Porção de fios dobados.
Intriga; mexerico; enrêdo.
\section{Meadade}
\begin{itemize}
\item {Grp. gram.:f.}
\end{itemize}
\begin{itemize}
\item {Utilização:Ant.}
\end{itemize}
\begin{itemize}
\item {Proveniência:(De \textunderscore meio\textunderscore )}
\end{itemize}
O mesmo que \textunderscore metade\textunderscore .
\section{Meadeira}
\begin{itemize}
\item {Grp. gram.:f.}
\end{itemize}
\begin{itemize}
\item {Utilização:Bras}
\end{itemize}
\begin{itemize}
\item {Utilização:Prov.}
\end{itemize}
\begin{itemize}
\item {Utilização:trasm.}
\end{itemize}
\begin{itemize}
\item {Proveniência:(De \textunderscore meada\textunderscore )}
\end{itemize}
Máquina para fazer meadas.
O mesmo que \textunderscore dobadoira\textunderscore . (Colhido em Alijó)
\section{Meadinha-de-oiro}
\begin{itemize}
\item {Grp. gram.:f.}
\end{itemize}
Espécie de jôgo popular.
\section{Meado}
\begin{itemize}
\item {Grp. gram.:m.}
\end{itemize}
\begin{itemize}
\item {Utilização:Prov.}
\end{itemize}
\begin{itemize}
\item {Grp. gram.:Adj.}
\end{itemize}
\begin{itemize}
\item {Proveniência:(De \textunderscore mear\textunderscore )}
\end{itemize}
Meio, parte média: \textunderscore no meado de Abril\textunderscore .
Mistura de centeio e milho, em grão.
Dividido ao meio; que chegou ao meio: \textunderscore meado do século xv...\textunderscore 
\section{Mealha}
\begin{itemize}
\item {Grp. gram.:f.}
\end{itemize}
\begin{itemize}
\item {Utilização:Fig.}
\end{itemize}
\begin{itemize}
\item {Proveniência:(Do b. lat. \textunderscore medalia\textunderscore )}
\end{itemize}
Moéda, constituída por cada uma das duas metades de um dinheiro partido.--Também se chamava por isso \textunderscore meio-dinheiro\textunderscore  e depois \textunderscore meio-ceitil\textunderscore , quando o nome \textunderscore ceitil\textunderscore  substituiu o nome \textunderscore dinheiro\textunderscore .
Pequena porção de qualquer coisa; migalha.
\section{Mealharia}
\begin{itemize}
\item {Grp. gram.:f.}
\end{itemize}
\begin{itemize}
\item {Proveniência:(De \textunderscore mealha\textunderscore )}
\end{itemize}
Antigo imposto, que os vendedores de Lisbôa pagavam á câmara municipal, pelo lugar que occupavam nos mercados.
\section{Mealheiro}
\begin{itemize}
\item {Grp. gram.:m.}
\end{itemize}
\begin{itemize}
\item {Grp. gram.:Adj.}
\end{itemize}
Conjunto de mealhas.
Pecúlio.
Caixinha ou cofre com uma pequena fenda, por onde se deita o dinheiro que se deseja acumular no mesmo cofre.
Que consta de mealhas ou de pouco dinheiro.
\section{Mealho}
\begin{itemize}
\item {Grp. gram.:m.}
\end{itemize}
\begin{itemize}
\item {Utilização:Prov.}
\end{itemize}
Placa de ferro, numa cavidade do orreiro.
\section{Mean}
\begin{itemize}
\item {Grp. gram.:f.}
\end{itemize}
\begin{itemize}
\item {Utilização:Prov.}
\end{itemize}
\begin{itemize}
\item {Utilização:beir.}
\end{itemize}
Anilha de coiro, que liga o mango ao pírtigo do mangual.
Variedade de azeitona, o mesmo que \textunderscore negral\textunderscore  ou \textunderscore madural\textunderscore .
\section{Meana}
\begin{itemize}
\item {Grp. gram.:f.}
\end{itemize}
\begin{itemize}
\item {Utilização:Ant.}
\end{itemize}
O mesmo que \textunderscore minhana\textunderscore . Cf. \textunderscore Port. Mon. Hist.\textunderscore , \textunderscore Script.\textunderscore , 286.
\section{Meandrar}
\begin{itemize}
\item {Grp. gram.:v. i.}
\end{itemize}
Formar meandro; serpear. Cf. Camillo, \textunderscore Mem. do Cárc.\textunderscore , II.
\section{Meândrico}
\begin{itemize}
\item {Grp. gram.:adj.}
\end{itemize}
\begin{itemize}
\item {Utilização:Des.}
\end{itemize}
\begin{itemize}
\item {Proveniência:(Do lat. \textunderscore maeandricus\textunderscore )}
\end{itemize}
Que tem meandros; emmaranhado; diffícil de comprehender.
\section{Meandrina}
\begin{itemize}
\item {Grp. gram.:f.}
\end{itemize}
Planta submarina, de \textunderscore fôlhas\textunderscore  zebradas.
\section{Meandro}
\begin{itemize}
\item {Grp. gram.:m.}
\end{itemize}
\begin{itemize}
\item {Utilização:Fig.}
\end{itemize}
\begin{itemize}
\item {Proveniência:(Do lat. \textunderscore maeandrus\textunderscore )}
\end{itemize}
Sinuosidade; ambages.
Enrêdo, intriga.
\section{Meandroso}
\begin{itemize}
\item {Grp. gram.:adj.}
\end{itemize}
Em que há meandros.
Intricádo; meândrico. Cf. Filinto, V, 134.
\section{Meanmente}
\begin{itemize}
\item {Grp. gram.:adv.}
\end{itemize}
\begin{itemize}
\item {Proveniência:(De \textunderscore meão\textunderscore )}
\end{itemize}
Medianamente.
\section{Meano}
\begin{itemize}
\item {Grp. gram.:adj.}
\end{itemize}
Diz-se do toiro, que tem branco o pêlo dos órgãos reproductores.
(Contr. de \textunderscore mediano\textunderscore )
\section{Meante}
\begin{itemize}
\item {Grp. gram.:adj.}
\end{itemize}
\begin{itemize}
\item {Utilização:Pop.}
\end{itemize}
Dividido ao meio.
Que em meio:«\textunderscore em janeiro, mele obreiro; mês meante, que não ante\textunderscore ».
(Prolóq. pop.)
\section{Meão}
\begin{itemize}
\item {Grp. gram.:adj.}
\end{itemize}
\begin{itemize}
\item {Grp. gram.:M.}
\end{itemize}
\begin{itemize}
\item {Proveniência:(Do lat. \textunderscore medeanus\textunderscore )}
\end{itemize}
Que está no meio.
Interposto, intermediário.
Que não é grande nem pequeno; mediano.
Medíocre.
Que occupa posição média, entre a classe nobre e a plebeia.
Peça central do tampo das vasilhas.
Peça central da roda dos carros, na qual se embebe o eixo e se assentam as cambas.
\section{Mear}
\begin{itemize}
\item {Grp. gram.:v. t.}
\end{itemize}
\begin{itemize}
\item {Grp. gram.:v. i.}
\end{itemize}
Dividir ao meio.
Chegar ao meio.
\section{Meato}
\begin{itemize}
\item {Grp. gram.:m.}
\end{itemize}
\begin{itemize}
\item {Proveniência:(Lat. \textunderscore meatus\textunderscore )}
\end{itemize}
Pequeno canal; abertura.
Caminho.
Orifício externo de um canal.
Intervallo, que dá passagem.
\section{Mebaar}
\begin{itemize}
\item {Grp. gram.:m.}
\end{itemize}
Peixe melacopterýgio do Japão.
\section{Meca}
\begin{itemize}
\item {Grp. gram.:f.}
\end{itemize}
\begin{itemize}
\item {Utilização:Chul.}
\end{itemize}
\begin{itemize}
\item {Proveniência:(Lat. \textunderscore moecha\textunderscore )}
\end{itemize}
Mulher brejeira. Cf. O'Neill, \textunderscore Fabulário\textunderscore , 503.
\section{Mecânica}
\begin{itemize}
\item {Grp. gram.:f.}
\end{itemize}
\begin{itemize}
\item {Proveniência:(De \textunderscore mecânico\textunderscore )}
\end{itemize}
Ciência, que estuda as fôrças motoras, as leis do equilíbrio e do movimento, e bem assim a teoria da acção das máquinas.
Conjunto de máquinas de um estabelecimento.
Aplicação dos princípios de uma arte ou ciência.
\section{Mecanicamente}
\begin{itemize}
\item {Grp. gram.:adv.}
\end{itemize}
De modo mecânico.
\section{Mecanicismo}
\begin{itemize}
\item {Grp. gram.:m.}
\end{itemize}
\begin{itemize}
\item {Utilização:Med.}
\end{itemize}
Explicação biológica dos fenómenos vitaes, pelas leis da Mecânica dos corpos inorgânicos.
\section{Mecanicista}
\begin{itemize}
\item {Grp. gram.:m.}
\end{itemize}
\begin{itemize}
\item {Utilização:Bras}
\end{itemize}
Aquele que, profissionalmente, se ocupa de Mecânica.
\section{Mecânico}
\begin{itemize}
\item {Grp. gram.:Adj.}
\end{itemize}
\begin{itemize}
\item {Utilização:Fig.}
\end{itemize}
\begin{itemize}
\item {Grp. gram.:M.}
\end{itemize}
\begin{itemize}
\item {Utilização:Fam.}
\end{itemize}
\begin{itemize}
\item {Proveniência:(Lat. \textunderscore mechanicus\textunderscore )}
\end{itemize}
Relativo á Mecânica.
Maquinal; automático.
Aquele que é versado ou trabalha em Mecânica.
Trabalhador, ou indivíduo de inferior condição.
\section{Mecanismo}
\begin{itemize}
\item {Grp. gram.:m.}
\end{itemize}
\begin{itemize}
\item {Utilização:Fig.}
\end{itemize}
\begin{itemize}
\item {Proveniência:(Lat. \textunderscore mechanisma\textunderscore )}
\end{itemize}
Disposição das partes constitutivas de uma máquina.
Maquinismo.
Organização de um todo.
Parte material ou externa da linguagem, independentemente do sentido das palavras.
\section{Mecanista}
\begin{itemize}
\item {Grp. gram.:m.}
\end{itemize}
\begin{itemize}
\item {Utilização:P. us.}
\end{itemize}
O mesmo que \textunderscore matemático\textunderscore .
(Cp. \textunderscore mecanismo\textunderscore )
\section{Mecanoterapia}
\begin{itemize}
\item {Grp. gram.:f.}
\end{itemize}
\begin{itemize}
\item {Proveniência:(Do gr. \textunderscore mekhane\textunderscore  + \textunderscore therapeia\textunderscore )}
\end{itemize}
Emprêgo da maçagem, como meio terapêutico.
\section{Mecanoterápico}
\begin{itemize}
\item {Grp. gram.:adj.}
\end{itemize}
Relativo á mecanoterapia.
\section{Mecanotherapia}
\begin{itemize}
\item {Grp. gram.:f.}
\end{itemize}
\begin{itemize}
\item {Proveniência:(Do gr. \textunderscore mekhane\textunderscore  + \textunderscore therapeia\textunderscore )}
\end{itemize}
Emprêgo da maçagem, como meio therapêutico.
\section{Mecanotherápico}
\begin{itemize}
\item {Grp. gram.:adj.}
\end{itemize}
Relativo á mecanotherapia.
\section{Mecas}
\begin{itemize}
\item {Grp. gram.:f. pl.}
\end{itemize}
\begin{itemize}
\item {Utilização:T. de Lagoaça}
\end{itemize}
O jôgo das nécaras.
\section{Meças}
\begin{itemize}
\item {Grp. gram.:f. pl.}
\end{itemize}
\begin{itemize}
\item {Proveniência:(De \textunderscore meça\textunderscore , conjunctivo do v. \textunderscore medir\textunderscore )}
\end{itemize}
Acto de medir; confronto: \textunderscore peço meças\textunderscore .
\section{Mecate}
\begin{itemize}
\item {Grp. gram.:m.}
\end{itemize}
\begin{itemize}
\item {Utilização:minh}
\end{itemize}
\begin{itemize}
\item {Utilização:Pesc.}
\end{itemize}
Tralha inferior, de esparto, na rêde sardinheira.
\section{Mècatrefe}
\begin{itemize}
\item {Grp. gram.:m.}
\end{itemize}
O mesmo que \textunderscore melcatrefe\textunderscore .
\section{Mêcê}
\begin{itemize}
\item {Utilização:Pleb. do Bras.}
\end{itemize}
(Corr. de \textunderscore você\textunderscore , ou contr. \textunderscore vossemecê\textunderscore )
\section{Mecedura}
\begin{itemize}
\item {Grp. gram.:f.}
\end{itemize}
\begin{itemize}
\item {Utilização:Ant.}
\end{itemize}
O mesmo que \textunderscore medidagem\textunderscore .
(Cp. \textunderscore meças\textunderscore )
\section{Mecenas}
\begin{itemize}
\item {Grp. gram.:m.}
\end{itemize}
\begin{itemize}
\item {Utilização:Fig.}
\end{itemize}
\begin{itemize}
\item {Proveniência:(De \textunderscore Mecenas\textunderscore , n. p.)}
\end{itemize}
Protector das letras ou dos sábios.
\section{Mecha}
\begin{itemize}
\item {Grp. gram.:f.}
\end{itemize}
\begin{itemize}
\item {Utilização:Náut.}
\end{itemize}
\begin{itemize}
\item {Utilização:Fam.}
\end{itemize}
\begin{itemize}
\item {Utilização:Açor}
\end{itemize}
\begin{itemize}
\item {Proveniência:(Do fr. \textunderscore mèche\textunderscore )}
\end{itemize}
Pedaço de papel ou pano, embebido em enxôfre e que serve especialmente para defumar pipas ou tonéis.
Torcida.
Substância combustível, para communicar fogo.
Rastilho.
Fios, que se colado sôbre uma chaga, para que esta se não feche.
Lardo.
Saliência de uma tábua, para que esta se encaixe noutra.
Parte inferior do mastro, por onde eelle se encaixa na carlinga.
Parte superior do mastro, no topo do calcês.
Maçada.
O mesmo que \textunderscore phósphoro\textunderscore .
\section{Machagem}
\begin{itemize}
\item {Grp. gram.:f.}
\end{itemize}
Acto ou effeito de mechar.
\section{Mechânica}
\begin{itemize}
\item {fónica:câ}
\end{itemize}
\begin{itemize}
\item {Grp. gram.:f.}
\end{itemize}
\begin{itemize}
\item {Proveniência:(De \textunderscore mecânico\textunderscore )}
\end{itemize}
Sciência, que estuda as fôrças motoras, as leis do equilíbrio e do movimento, e bem assim a theoria da acção das máquinas.
Conjunto de máquinas de um estabelecimento.
Applicação dos princípios de uma arte ou sciência.
\section{Mechanicamente}
\begin{itemize}
\item {fónica:câ}
\end{itemize}
\begin{itemize}
\item {Grp. gram.:adv.}
\end{itemize}
De modo mecânico.
\section{Mechanicismo}
\begin{itemize}
\item {fónica:câ}
\end{itemize}
\begin{itemize}
\item {Grp. gram.:m.}
\end{itemize}
\begin{itemize}
\item {Utilização:Med.}
\end{itemize}
Explicação biológica dos phenómenos vitaes, pelas leis da Mecânica dos corpos inorgânicos.
\section{Mechanicista}
\begin{itemize}
\item {fónica:câ}
\end{itemize}
\begin{itemize}
\item {Grp. gram.:m.}
\end{itemize}
\begin{itemize}
\item {Utilização:Bras}
\end{itemize}
Aquelle que, profissionalmente, se occupa de Mecânica.
\section{Mechânico}
\begin{itemize}
\item {fónica:câ}
\end{itemize}
\begin{itemize}
\item {Grp. gram.:Adj.}
\end{itemize}
\begin{itemize}
\item {Utilização:Fig.}
\end{itemize}
\begin{itemize}
\item {Grp. gram.:M.}
\end{itemize}
\begin{itemize}
\item {Utilização:Fam.}
\end{itemize}
\begin{itemize}
\item {Proveniência:(Lat. \textunderscore mechanicus\textunderscore )}
\end{itemize}
Relativo á Mecânica.
Maquinal; automático.
Aquelle que é versado ou trabalha em Mecânica.
Trabalhador, ou indivíduo de inferior condição.
\section{Mechanismo}
\begin{itemize}
\item {fónica:câ}
\end{itemize}
\begin{itemize}
\item {Grp. gram.:m.}
\end{itemize}
\begin{itemize}
\item {Utilização:Fig.}
\end{itemize}
\begin{itemize}
\item {Proveniência:(Lat. \textunderscore mechanisma\textunderscore )}
\end{itemize}
Disposição das partes constitutivas de uma máquina.
Maquinismo.
Organização de um todo.
Parte material ou externa da linguagem, independentemente do sentido das palavras.
\section{Mechanista}
\begin{itemize}
\item {fónica:ca}
\end{itemize}
\begin{itemize}
\item {Grp. gram.:m.}
\end{itemize}
\begin{itemize}
\item {Utilização:P. us.}
\end{itemize}
O mesmo que \textunderscore mathemático\textunderscore .
(Cp. \textunderscore mecanismo\textunderscore )
\section{Mechar}
\begin{itemize}
\item {Grp. gram.:v. t.}
\end{itemize}
Defumar com mecha.
Commumicar fogo a.
Encaixar por meio de mecha.
\section{Mècheiro}
\begin{itemize}
\item {Grp. gram.:m.}
\end{itemize}
Bico de candeeiro ou candeia, em que se introduz a mecha.
Aquelle que faz mechas.
\section{Mechoacão}
\begin{itemize}
\item {Grp. gram.:m.}
\end{itemize}
Planta convolvulácea do México.
\section{Mécia}
\begin{itemize}
\item {Grp. gram.:f.}
\end{itemize}
\begin{itemize}
\item {Proveniência:(De \textunderscore Mécia\textunderscore , n. p.)}
\end{itemize}
Variedade de excellente pêra de inverno.
\section{Meco}
\begin{itemize}
\item {Grp. gram.:m.}
\end{itemize}
\begin{itemize}
\item {Utilização:Gír.}
\end{itemize}
\begin{itemize}
\item {Utilização:Prov.}
\end{itemize}
\begin{itemize}
\item {Utilização:dur.}
\end{itemize}
\begin{itemize}
\item {Proveniência:(Lat. \textunderscore moechus\textunderscore )}
\end{itemize}
Qualquer indivíduo; sujeito; typo. Homem libertino, atrevido.
Espartalhão.
Pauzinho, que se põe de pé, no jogo da patela.
\section{Mecoedondo}
\begin{itemize}
\item {fónica:co-e}
\end{itemize}
\begin{itemize}
\item {Grp. gram.:m.}
\end{itemize}
Árvore angolense, de fibras têxteis.
\section{Mecómetro}
\begin{itemize}
\item {Grp. gram.:m.}
\end{itemize}
\begin{itemize}
\item {Proveniência:(Do gr. \textunderscore mekos\textunderscore  + \textunderscore metron\textunderscore )}
\end{itemize}
Instrumento cirúrgico, em fórma de compasso, para medir o comprimento do féto.
\section{Mecónico}
\begin{itemize}
\item {Grp. gram.:adj.}
\end{itemize}
\begin{itemize}
\item {Proveniência:(Do gr. \textunderscore mekon\textunderscore , papoila)}
\end{itemize}
Diz-se do um ácido, descoberto no ópio.
\section{Meconina}
\begin{itemize}
\item {Grp. gram.:f.}
\end{itemize}
\begin{itemize}
\item {Proveniência:(Do gr. \textunderscore mekon\textunderscore )}
\end{itemize}
Substância crystallina, que se extrai do ópio.
\section{Mecónio}
\begin{itemize}
\item {Grp. gram.:m.}
\end{itemize}
\begin{itemize}
\item {Proveniência:(Gr. \textunderscore mekonion\textunderscore )}
\end{itemize}
Matéria escura ou esverdeada e viscosa, que se accumula nos intestinos do feto durante a gestação e que é evacuada logo que elle nasce.
\section{Mecópode}
\begin{itemize}
\item {Grp. gram.:adj.}
\end{itemize}
\begin{itemize}
\item {Grp. gram.:Pl.}
\end{itemize}
\begin{itemize}
\item {Proveniência:(Do gr. \textunderscore mekos\textunderscore  + \textunderscore pous\textunderscore , \textunderscore podos\textunderscore )}
\end{itemize}
Que tem pés compridos.
Gênero de insectos orthópteros.
\section{Mecru}
\begin{itemize}
\item {Grp. gram.:m.}
\end{itemize}
Planta canácea do Brasil.
\section{Méda}
\begin{itemize}
\item {Grp. gram.:f.}
\end{itemize}
\begin{itemize}
\item {Utilização:Fig.}
\end{itemize}
\begin{itemize}
\item {Proveniência:(Do lat. \textunderscore meta\textunderscore )}
\end{itemize}
Montão de feixes de trigo, palha, etc, sobrepostos de maneira que constituam proximamente um cone.
Montão; agrupamento de objectos da mesma espécie.
\section{Mêda}
\begin{itemize}
\item {Grp. gram.:f.}
\end{itemize}
\begin{itemize}
\item {Utilização:Fig.}
\end{itemize}
\begin{itemize}
\item {Proveniência:(Do lat. \textunderscore meta\textunderscore )}
\end{itemize}
Montão de feixes de trigo, palha, etc, sobrepostos de maneira que constituam proximamente um cone.
Montão; agrupamento de objectos da mesma espécie.
\section{Medalha}
\begin{itemize}
\item {Grp. gram.:f.}
\end{itemize}
\begin{itemize}
\item {Proveniência:(Do lat. \textunderscore metallia\textunderscore )}
\end{itemize}
Peça mettállica, geralmente redonda, com a representação de personagem ou successo notável e respectiva inscripção.
Insígnia de Ordem honorífica.
Nome genérico das moédas dos povos antigos.
Peça, que representa assumpto de devoção, e ordinariamente benzida pelo Papa ou algum dignitário ecclesiástico.
Prêmio que, em concursos ou exposições, se confere aos que se distinguem nessas exposições ou concursos.
Caixinha ornamental, formada de duas tampas e geralmente de oiro, que as mulheres põem ao pescoço, ou que serve de berloque na corrente do relógio.
Baixo relevo architectónico, que representa personagem célebre ou acto memorável.
\section{Medalhão}
\begin{itemize}
\item {Grp. gram.:m.}
\end{itemize}
\begin{itemize}
\item {Utilização:Fig.}
\end{itemize}
Medalha grande.
Baixo relevo, oval ou circular.
Caixinha circular ou oval, geralmente com uma das faces de vidro, na qual se contém um retrato, cabellos, etc.
Homem importante, figurão. Cf. Castilho, \textunderscore Fastos\textunderscore , I,348.
\section{Medalhar}
\begin{itemize}
\item {Grp. gram.:v. t.}
\end{itemize}
Gravar em medalha; commemorar por meio de medalha.
\section{Medalhário}
\begin{itemize}
\item {Grp. gram.:m.}
\end{itemize}
Lugar, em que se guardam medalhas, methodicamente dispostas.
Collecção de medalhas.
Fabricante de medalhas.
\section{Medalheiro}
\begin{itemize}
\item {Grp. gram.:m.}
\end{itemize}
Lugar, em que se guardam medalhas, methodicamente dispostas.
Collecção de medalhas.
Fabricante de medalhas.
\section{Medalhista}
\begin{itemize}
\item {Grp. gram.:m.}
\end{itemize}
Aquelle que é versado no estudo de medalhas.
Colleccionador de medalhas.
\section{Medalhística}
\begin{itemize}
\item {Grp. gram.:f.}
\end{itemize}
\begin{itemize}
\item {Utilização:Neol.}
\end{itemize}
\begin{itemize}
\item {Proveniência:(De \textunderscore medalhista\textunderscore )}
\end{itemize}
Estudo de medalhas.
\section{Medam}
\begin{itemize}
\item {Grp. gram.:m.}
\end{itemize}
\begin{itemize}
\item {Proveniência:(Do cast. \textunderscore medano\textunderscore )}
\end{itemize}
O mesmo que \textunderscore médo\textunderscore ^1.
\section{Medanismo}
\begin{itemize}
\item {Grp. gram.:m.}
\end{itemize}
\begin{itemize}
\item {Utilização:Neol.}
\end{itemize}
\begin{itemize}
\item {Proveniência:(De \textunderscore Médan\textunderscore , n. p.)}
\end{itemize}
Escola literária de Zola.
\section{Médão}
\begin{itemize}
\item {Grp. gram.:m.}
\end{itemize}
\begin{itemize}
\item {Proveniência:(Do cast. \textunderscore medano\textunderscore )}
\end{itemize}
O mesmo que \textunderscore médo\textunderscore ^1.
\section{Medeira}
\begin{itemize}
\item {Grp. gram.:f.}
\end{itemize}
\begin{itemize}
\item {Utilização:Prov.}
\end{itemize}
\begin{itemize}
\item {Utilização:beir.}
\end{itemize}
\begin{itemize}
\item {Utilização:Chul.}
\end{itemize}
\begin{itemize}
\item {Proveniência:(De \textunderscore mêdo\textunderscore )}
\end{itemize}
Grande mêdo, grande susto.
\section{Medeiro}
\begin{itemize}
\item {Grp. gram.:m.}
\end{itemize}
\begin{itemize}
\item {Utilização:Prov.}
\end{itemize}
\begin{itemize}
\item {Utilização:minh.}
\end{itemize}
Lugar, onde há medas de milho; meda.
\section{Medeixes}
\begin{itemize}
\item {Grp. gram.:m. pl.}
\end{itemize}
\begin{itemize}
\item {Utilização:Bras}
\end{itemize}
\begin{itemize}
\item {Proveniência:(De \textunderscore me\textunderscore  + \textunderscore deixes\textunderscore , de \textunderscore deixar\textunderscore )}
\end{itemize}
Desdém, desprêzo pela pessôa que nos procura.
\section{Medês}
\begin{itemize}
\item {Grp. gram.:adj.}
\end{itemize}
\begin{itemize}
\item {Utilização:Prov.}
\end{itemize}
\begin{itemize}
\item {Utilização:ant.}
\end{itemize}
\begin{itemize}
\item {Grp. gram.:Pl.}
\end{itemize}
\begin{itemize}
\item {Proveniência:(Do it. \textunderscore medesimo\textunderscore )}
\end{itemize}
Mesmo; até.
Medeses.
\section{Média}
\begin{itemize}
\item {Grp. gram.:f.}
\end{itemize}
\begin{itemize}
\item {Proveniência:(De \textunderscore médio\textunderscore )}
\end{itemize}
Quociente da divisão do total de differentes quantidades pelo número destas.
Termo médio.
\section{Mediação}
\begin{itemize}
\item {Grp. gram.:f.}
\end{itemize}
\begin{itemize}
\item {Proveniência:(Lat. \textunderscore mediatia\textunderscore )}
\end{itemize}
Acto ou effeito de mediar.
Intercessão; intervenção.
Momento da culminação de um astro.
Divisão de um versículo ou de um psalmo em duas partes, uma das quaes é cantada pelos indivíduos que formam um lado do côro, e a outra pelos que formam o lado opposto.
\section{Mediador}
\begin{itemize}
\item {Grp. gram.:m.  e  adj.}
\end{itemize}
\begin{itemize}
\item {Proveniência:(Lat. \textunderscore mediator\textunderscore )}
\end{itemize}
O que intervém; medianeiro.
Árbitro.
\section{Medial}
\begin{itemize}
\item {Grp. gram.:adj.}
\end{itemize}
\begin{itemize}
\item {Utilização:Gram.}
\end{itemize}
\begin{itemize}
\item {Grp. gram.:F.}
\end{itemize}
\begin{itemize}
\item {Proveniência:(Lat. \textunderscore medialis\textunderscore )}
\end{itemize}
Diz-se das letras, que estão no meio de uma palavra.
Letra medial.
\section{Medianamente}
\begin{itemize}
\item {Grp. gram.:adv.}
\end{itemize}
De modo mediano.
Mediocremente.
Nem muito nem pouco.
\section{Medianeira}
\begin{itemize}
\item {Grp. gram.:f.}
\end{itemize}
\begin{itemize}
\item {Utilização:Fig.}
\end{itemize}
\begin{itemize}
\item {Proveniência:(De \textunderscore medianeiro\textunderscore )}
\end{itemize}
Intercessora.
Alcoviteira.
\section{Medianeiro}
\begin{itemize}
\item {Grp. gram.:m.  e  adj.}
\end{itemize}
\begin{itemize}
\item {Proveniência:(De \textunderscore mediano\textunderscore )}
\end{itemize}
O mesmo que \textunderscore mediador\textunderscore .
\section{Mediania}
\begin{itemize}
\item {Grp. gram.:f.}
\end{itemize}
\begin{itemize}
\item {Utilização:Náut.}
\end{itemize}
\begin{itemize}
\item {Utilização:Fig.}
\end{itemize}
Qualidade de mediano.
Termo médio.
Meio termo, entre a riqueza e a pobreza.
Espaço, comprehendido entre as sicordas de popa á prôa.
Moderação.
\section{Medianido}
\begin{itemize}
\item {Grp. gram.:m.}
\end{itemize}
\begin{itemize}
\item {Utilização:Ant.}
\end{itemize}
\begin{itemize}
\item {Proveniência:(Do b. lat. \textunderscore medianelum\textunderscore )}
\end{itemize}
Raia ou extremo entre dois municípios. Cf. Herculano, \textunderscore Hisl. de Port.\textunderscore , II, 428.
Reunião dos magistrados e respectivos officiaes de dois concelhos no estremo commum dos respectivos territórios. Cf. Herculano, \textunderscore Hist. de Port.\textunderscore , IV, 194.
\section{Medianímico}
\begin{itemize}
\item {Grp. gram.:adj.}
\end{itemize}
\begin{itemize}
\item {Utilização:Espir.}
\end{itemize}
\begin{itemize}
\item {Proveniência:(Do lat. \textunderscore medius\textunderscore  + \textunderscore animus\textunderscore )}
\end{itemize}
Que tem a qualidade ou a faculdade de médium.
Relativo aos médiums.
\section{Medianimidade}
\begin{itemize}
\item {Grp. gram.:f.}
\end{itemize}
Estado ou propriedade de medianímico.
\section{Medianiz}
\begin{itemize}
\item {Grp. gram.:f.}
\end{itemize}
\begin{itemize}
\item {Proveniência:(De \textunderscore mediano\textunderscore )}
\end{itemize}
Espaço em branco, entre as páginas de fôlhas impressas.
\section{Mediano}
\begin{itemize}
\item {Grp. gram.:adj.}
\end{itemize}
\begin{itemize}
\item {Proveniência:(Lat. \textunderscore medinnus\textunderscore )}
\end{itemize}
Que está no meio.
Que está entre dois extremos.
Medíocre.
Meão.
Moderado.
\section{Mediante}
\begin{itemize}
\item {Grp. gram.:adj.}
\end{itemize}
\begin{itemize}
\item {Grp. gram.:Prep.}
\end{itemize}
\begin{itemize}
\item {Grp. gram.:M.}
\end{itemize}
\begin{itemize}
\item {Proveniência:(Lat. \textunderscore medians\textunderscore , \textunderscore mediantis\textunderscore )}
\end{itemize}
Que medeia.
Que intervém.
Por meio de, por intervenção de.
Tempo decorrido entre dois factos ou duas épocas. Cf. Filinto, XX, 71.
\section{Medião}
\begin{itemize}
\item {Grp. gram.:m.}
\end{itemize}
\begin{itemize}
\item {Proveniência:(Gr. \textunderscore medion\textunderscore )}
\end{itemize}
Planta campanulácea, com que os antigos tratavam os mênstruos.
\section{Mediar}
\begin{itemize}
\item {Grp. gram.:v. t.}
\end{itemize}
\begin{itemize}
\item {Grp. gram.:V. i.}
\end{itemize}
\begin{itemize}
\item {Proveniência:(Lat. \textunderscore mediare\textunderscore )}
\end{itemize}
Dividir ao meio.
Intervir á cêrca de.
Estar no meio.
Estar ou decorrer entre dois pontos ou duas épocas.
Intervir.
\section{Mediastinite}
\begin{itemize}
\item {Grp. gram.:f.}
\end{itemize}
Inflammação do tecido laminoso do mediastino.
\section{Mediastino}
\begin{itemize}
\item {Grp. gram.:m.}
\end{itemize}
\begin{itemize}
\item {Utilização:Anat.}
\end{itemize}
\begin{itemize}
\item {Utilização:Bot.}
\end{itemize}
\begin{itemize}
\item {Grp. gram.:Adj.}
\end{itemize}
\begin{itemize}
\item {Proveniência:(Lat. \textunderscore mediastinus\textunderscore )}
\end{itemize}
Espaço entre cada uma das pleuras e a columna vertebral.
Divisão transversal, que separa em duas partes o fruto das plantas crucíferas.
Relativo ao mediastino.
\section{Mediatamente}
\begin{itemize}
\item {Grp. gram.:adv.}
\end{itemize}
De modo mediato.
Indirectamente; com interposição de alguém ou de alguma coisa.
Com demora. Cf. Castilho, \textunderscore Felic. pela Agr.\textunderscore , 173.
\section{Mediatário}
\begin{itemize}
\item {Grp. gram.:m.  e  adj.}
\end{itemize}
\begin{itemize}
\item {Proveniência:(De \textunderscore mediato\textunderscore )}
\end{itemize}
O mesmo que \textunderscore medianeiro\textunderscore .
\section{Mediatização}
\begin{itemize}
\item {Grp. gram.:f.}
\end{itemize}
Acto de \textunderscore mediatizar\textunderscore .
\section{Mediatizar}
\begin{itemize}
\item {Grp. gram.:v. t.}
\end{itemize}
\begin{itemize}
\item {Proveniência:(De \textunderscore mediato\textunderscore )}
\end{itemize}
Fazer que não dependa immediatamente do poder supremo de uma confederação, para não depender senão mediatamente, (falando-se especialmente dos príncipes ou Estados do império alemão).
\section{Mediato}
\begin{itemize}
\item {Grp. gram.:adj.}
\end{itemize}
\begin{itemize}
\item {Proveniência:(Lat. \textunderscore mediatus\textunderscore )}
\end{itemize}
Indirecto; que está em relação com outras coisas, por intermédio de uma terceira.
\section{Mediator}
\begin{itemize}
\item {Grp. gram.:m.}
\end{itemize}
O mesmo que \textunderscore mediador\textunderscore . Cf. Vieira, VI, 26.
\section{Médica}
\begin{itemize}
\item {Grp. gram.:f.}
\end{itemize}
\begin{itemize}
\item {Proveniência:(De \textunderscore médico\textunderscore )}
\end{itemize}
Mulher, que exerce a Medicina ou que tem diploma de médico.
\section{Médica}
\begin{itemize}
\item {Grp. gram.:f.}
\end{itemize}
\begin{itemize}
\item {Proveniência:(Lat. \textunderscore medica\textunderscore )}
\end{itemize}
Espécie de trevo; luzerna.
\section{Medicação}
\begin{itemize}
\item {Grp. gram.:f.}
\end{itemize}
\begin{itemize}
\item {Proveniência:(Lat. \textunderscore medicatio\textunderscore )}
\end{itemize}
Acto de medicar; tratamento therapêutico.
\section{Medicagem-dos-pastos}
\begin{itemize}
\item {Grp. gram.:f.}
\end{itemize}
Espécie de luzerna.
(Cp. \textunderscore médica\textunderscore ^2)
\section{Medicamentação}
\begin{itemize}
\item {Grp. gram.:f.}
\end{itemize}
Acto de medicamentar.
Medicação; tratamento médico.
\section{Medicamentar}
\begin{itemize}
\item {Grp. gram.:v. t.}
\end{itemize}
Dar medicamentos a; medicar.
\section{Mèdicamente}
\begin{itemize}
\item {Grp. gram.:adv.}
\end{itemize}
\begin{itemize}
\item {Proveniência:(De \textunderscore médico\textunderscore )}
\end{itemize}
Segundo os preceitos da Medicina.
\section{Medicamento}
\begin{itemize}
\item {Grp. gram.:m.}
\end{itemize}
\begin{itemize}
\item {Proveniência:(Lat. \textunderscore medicamentum\textunderscore )}
\end{itemize}
Substância, que se applica externa ou internamente, para restabelecer a saúde; remédio.
\section{Medicamentoso}
\begin{itemize}
\item {Grp. gram.:adj.}
\end{itemize}
Que tem propriedades de medicamento.
\section{Medicando}
\begin{itemize}
\item {Grp. gram.:m.  e  adj.}
\end{itemize}
\begin{itemize}
\item {Proveniência:(De \textunderscore medicar\textunderscore )}
\end{itemize}
O que vai sêr medicado ou mèdicamente tratado.
\section{Medicante}
\begin{itemize}
\item {Grp. gram.:adj.}
\end{itemize}
Que medíca.
\section{Medição}
\begin{itemize}
\item {Grp. gram.:f.}
\end{itemize}
Acto ou effeito de medir.
\section{Medicar}
\begin{itemize}
\item {Grp. gram.:v. t.}
\end{itemize}
\begin{itemize}
\item {Proveniência:(Lat. \textunderscore medicare\textunderscore )}
\end{itemize}
Tratar com medicamentos; dirigir o curativo de.
\section{Medicastro}
\begin{itemize}
\item {Grp. gram.:m.}
\end{itemize}
\begin{itemize}
\item {Proveniência:(De \textunderscore médico\textunderscore )}
\end{itemize}
Aquelle que faz curativos, sem diploma nem aptidões de médico; curandeiro.
\section{Medicatriz}
\begin{itemize}
\item {Grp. gram.:adj. f.}
\end{itemize}
\begin{itemize}
\item {Proveniência:(Do lat. hypoth. \textunderscore medicatrix\textunderscore )}
\end{itemize}
Que tem a propriedade de curar; que tem a natureza de medicamento: \textunderscore substância medicatriz\textunderscore .
\section{Medicável}
\begin{itemize}
\item {Grp. gram.:adj.}
\end{itemize}
\begin{itemize}
\item {Proveniência:(Lat. \textunderscore medicabilis\textunderscore )}
\end{itemize}
Que se póde medicar.
\section{Medicina}
\begin{itemize}
\item {Grp. gram.:f.}
\end{itemize}
\begin{itemize}
\item {Utilização:Fig.}
\end{itemize}
\begin{itemize}
\item {Utilização:Gír.}
\end{itemize}
\begin{itemize}
\item {Proveniência:(Lat. \textunderscore medicina\textunderscore )}
\end{itemize}
Arte, que tem por objecto a cura das doenças e a conservação da saúde, baseando-se na Pathologia ou sciência das enfermidades.
Systema medicinal.
Medicamento.
Aquillo que remedeia um mal.
O mesmo que \textunderscore conselho\textunderscore , aviso.
\section{Medicinal}
\begin{itemize}
\item {Grp. gram.:adj.}
\end{itemize}
\begin{itemize}
\item {Proveniência:(Lat. \textunderscore medicinalis\textunderscore )}
\end{itemize}
Relativo á Medicina; que serve de remédio; que cura; que se applica contra doenças: \textunderscore plantas medicinaes\textunderscore . \textunderscore Fig.\textunderscore 
Que remedeia qualquer mal moral.
\section{Medicinalmente}
\begin{itemize}
\item {Grp. gram.:adv.}
\end{itemize}
De modo medicinal.
\section{Medicinar}
\begin{itemize}
\item {Grp. gram.:v. t.  e  p.}
\end{itemize}
\begin{itemize}
\item {Proveniência:(De \textunderscore medicina\textunderscore )}
\end{itemize}
O mesmo que \textunderscore medicar\textunderscore .
\section{Medicineiro}
\begin{itemize}
\item {Grp. gram.:m.}
\end{itemize}
\begin{itemize}
\item {Proveniência:(De \textunderscore medicina\textunderscore )}
\end{itemize}
Arbusto medicinal, da fam. das euphorbiáceas.
\section{Médico}
\begin{itemize}
\item {Grp. gram.:m.}
\end{itemize}
\begin{itemize}
\item {Utilização:Fig.}
\end{itemize}
\begin{itemize}
\item {Grp. gram.:Adj.}
\end{itemize}
Aquelle que é versado em Medicina.
Aquelle que a exerce; clínico.
Aquelle que cursou uma faculdade ou escola de Medicina, recebendo os graus respectivos.
Aquillo que póde restabelecer a saúde.
\textunderscore Médico espiritual\textunderscore , confessor ou director de consciência.
O mesmo que \textunderscore medicinal\textunderscore : \textunderscore progressos médicos\textunderscore .
(Lat. \textunderscore medicus\textunderscore ).
\section{Médico}
\begin{itemize}
\item {Grp. gram.:adj.}
\end{itemize}
\begin{itemize}
\item {Proveniência:(De \textunderscore médo\textunderscore ^2)}
\end{itemize}
Relativo á Média ou aos \textunderscore Medos\textunderscore .
\section{Médico-mania}
\begin{itemize}
\item {Grp. gram.:f.}
\end{itemize}
Mania de curar ou de exercer funcções de médico.
\section{Médico-maníaco}
\begin{itemize}
\item {Grp. gram.:m.  e  adj.}
\end{itemize}
O que tem a médico-mania.
\section{Médico-psychológico}
\begin{itemize}
\item {Grp. gram.:adj.}
\end{itemize}
Diz-se dos estudos, relativos á Psychologia, feitos por meio da observação das doenças cerebraes.
\section{Medida}
\begin{itemize}
\item {Grp. gram.:f.}
\end{itemize}
\begin{itemize}
\item {Utilização:Bras. do N}
\end{itemize}
\begin{itemize}
\item {Proveniência:(Do lat. \textunderscore metita\textunderscore )}
\end{itemize}
Grandeza determinada, que serve de base para avaliar outras grandezas: \textunderscore o metro é medida de extensão\textunderscore .
Acto de medir.
Extensão ou grandeza que se póde calcular.
Vaso, de grandeza determinada, para avaliar a quantidade de certos objectos.
Régua ou qualquer tira graduada, com que se tomam certas medidas.
Grau.
Bitola.
Proporção: \textunderscore na medida das minhas forças\textunderscore .
Baliza.
Alcance.
Cálculo.
Regra; ordem, bôa disposição.
Moderação.
Predisposições; projecto.
Compasso musical.
Porção de dez litros.
\section{Medidagem}
\begin{itemize}
\item {Grp. gram.:f.}
\end{itemize}
\begin{itemize}
\item {Proveniência:(De \textunderscore medir\textunderscore )}
\end{itemize}
Aquillo que, de um objecto medido, pertence ao medidor.
Fôro antigo, que se pagava ao medidor do pão e do vinho.
\section{Medidamente}
\begin{itemize}
\item {Grp. gram.:adv.}
\end{itemize}
\begin{itemize}
\item {Proveniência:(De \textunderscore medido\textunderscore )}
\end{itemize}
Moderadamente; com proporção.
\section{Medideira}
\begin{itemize}
\item {Grp. gram.:f.}
\end{itemize}
\begin{itemize}
\item {Utilização:Ant.}
\end{itemize}
Mulher que, em Lisbôa, media o trigo aos compradores.
\section{Medidor}
\begin{itemize}
\item {Grp. gram.:adj.}
\end{itemize}
\begin{itemize}
\item {Grp. gram.:M.}
\end{itemize}
Que mede.
Aquelle que mede.
\section{Medieval}
\begin{itemize}
\item {Grp. gram.:adj.}
\end{itemize}
O mesmo que \textunderscore medievo\textunderscore .
\section{Mediévico}
\begin{itemize}
\item {Grp. gram.:adj.}
\end{itemize}
O mesmo que \textunderscore medievo\textunderscore .
\section{Medievista}
\begin{itemize}
\item {Grp. gram.:m.}
\end{itemize}
Aquelle que é versado em assumptos da Ídade-Média.
(Do \textunderscore medievo\textunderscore )
\section{Medievo}
\begin{itemize}
\item {Grp. gram.:adj.}
\end{itemize}
\begin{itemize}
\item {Proveniência:(Do lat. \textunderscore medium\textunderscore  + \textunderscore aevum\textunderscore )}
\end{itemize}
Relativo á Idade-Média.
\section{Mediista}
\begin{itemize}
\item {Grp. gram.:m.}
\end{itemize}
\begin{itemize}
\item {Proveniência:(De \textunderscore médio\textunderscore )}
\end{itemize}
Partidário da sciência média, em Theologia.
\section{Medimarémetro}
\begin{itemize}
\item {Grp. gram.:m.}
\end{itemize}
\begin{itemize}
\item {Proveniência:(De \textunderscore médio\textunderscore  + \textunderscore mar\textunderscore  + gr. \textunderscore metron\textunderscore )}
\end{itemize}
Instrumento, com que determina o nível médio do mar.
\section{Medimno}
\begin{itemize}
\item {Grp. gram.:m.}
\end{itemize}
\begin{itemize}
\item {Proveniência:(Lat. \textunderscore medimnum\textunderscore )}
\end{itemize}
Medida grega, correspondente a seis módios romanos.
\section{Medina}
\begin{itemize}
\item {Grp. gram.:f.}
\end{itemize}
Espécie de tecido antigo.
\section{Medínila}
\begin{itemize}
\item {Grp. gram.:f.}
\end{itemize}
\begin{itemize}
\item {Proveniência:(De \textunderscore Medinilla\textunderscore , n. p.)}
\end{itemize}
Gênero de plantas melastomáceas, com 75 espécies na Índia, Filippinas, etc.
\section{Médio}
\begin{itemize}
\item {Grp. gram.:adj.}
\end{itemize}
\begin{itemize}
\item {Proveniência:(Lat. \textunderscore medius\textunderscore )}
\end{itemize}
Que está no meio ou entre dois pontos.
Que occupa situação intermediária.
Que separa duas coisas.
Que exprime o meio termo.
Que occupa o meio entre duas extremidades: \textunderscore temperatura média\textunderscore .
\section{Mediocracia}
\begin{itemize}
\item {Grp. gram.:f.}
\end{itemize}
\begin{itemize}
\item {Utilização:Neol.}
\end{itemize}
\begin{itemize}
\item {Proveniência:(T. hybr., do lat. \textunderscore medius\textunderscore  + gr. \textunderscore krateia\textunderscore )}
\end{itemize}
Predomínio social das classes médias; burguesia.
\section{Mediocremente}
\begin{itemize}
\item {Grp. gram.:adv.}
\end{itemize}
De modo medíocre; pouco.
\section{Medíocre}
\begin{itemize}
\item {Grp. gram.:adj.}
\end{itemize}
\begin{itemize}
\item {Grp. gram.:m.}
\end{itemize}
\begin{itemize}
\item {Proveniência:(Lat. \textunderscore mediocris\textunderscore )}
\end{itemize}
Médio ou mediano.
Que não é bom e mau: \textunderscore estuante medíocre\textunderscore .
Que não é grande nem pequeno, ou que está entre pequeno e grande.
Ordinário, insignificante: \textunderscore um jantar medíocre\textunderscore .
Aquelle que é medíocre ou tem pouco merecimento.
Aquillo que é medíocre.
\section{Mediocricidade}
\begin{itemize}
\item {Grp. gram.:f.}
\end{itemize}
Qualidade ou estado de medíocre.
Falta de mérito; vulgaridade: \textunderscore a mediocricidade de um livro\textunderscore .
Pessôa ou coisa, que tem pouco merecimento.
\section{Medioterreno}
\begin{itemize}
\item {Grp. gram.:m.}
\end{itemize}
\begin{itemize}
\item {Utilização:Ant.}
\end{itemize}
O mesmo que \textunderscore mediterrâneo\textunderscore .
\section{Medir}
\begin{itemize}
\item {Grp. gram.:v. t.}
\end{itemize}
\begin{itemize}
\item {Grp. gram.:V. p.}
\end{itemize}
\begin{itemize}
\item {Proveniência:(Do lat. \textunderscore metiri\textunderscore )}
\end{itemize}
Avaliar ou determinar a grandeza ou quantidade de: \textunderscore medir azeite\textunderscore .
Ajustar, regular.
Têr a extensão de: \textunderscore êste campo mede 2 hectares\textunderscore .
Podenrar: \textunderscore medir as circunstâncias\textunderscore .
Percorrer,
Olhar, provocando.
Commedir, moderar.
Rivalizar.
Bater-se com alguém; arcar.
\section{Meditabundo}
\begin{itemize}
\item {Grp. gram.:adj.}
\end{itemize}
\begin{itemize}
\item {Proveniência:(Lat. \textunderscore meditabundus\textunderscore )}
\end{itemize}
Que medita; melancólico.
\section{Meditação}
\begin{itemize}
\item {Grp. gram.:f.}
\end{itemize}
\begin{itemize}
\item {Proveniência:(Lat. \textunderscore meditatio\textunderscore )}
\end{itemize}
Acto ou effeito de meditar; contemplação.
Oração mental.
\section{Meditador}
\begin{itemize}
\item {Grp. gram.:m.  e  adj.}
\end{itemize}
\begin{itemize}
\item {Proveniência:(Lat. \textunderscore meditator\textunderscore )}
\end{itemize}
O que medita.
\section{Meditar}
\begin{itemize}
\item {Grp. gram.:v. t.}
\end{itemize}
\begin{itemize}
\item {Grp. gram.:V. i.}
\end{itemize}
\begin{itemize}
\item {Grp. gram.:M.}
\end{itemize}
\begin{itemize}
\item {Proveniência:(Lat. \textunderscore meditari\textunderscore )}
\end{itemize}
Ponderar, considerar.
Pensar sôbre.
Projectar: \textunderscore meditar uma viagem\textunderscore .
Reflectir, pensar.
Meditação.
\section{Meditativo}
\begin{itemize}
\item {Grp. gram.:adj.}
\end{itemize}
\begin{itemize}
\item {Proveniência:(Lat. \textunderscore meditativus\textunderscore )}
\end{itemize}
Meditabundo.
Próprio de quem medita.
\section{Meditável}
\begin{itemize}
\item {Grp. gram.:adj.}
\end{itemize}
\begin{itemize}
\item {Proveniência:(De \textunderscore meditar\textunderscore )}
\end{itemize}
Que deve sêr meditado.
\section{Mediterrâneo}
\begin{itemize}
\item {Grp. gram.:adj.}
\end{itemize}
\begin{itemize}
\item {Grp. gram.:M.}
\end{itemize}
\begin{itemize}
\item {Proveniência:(Lat. \textunderscore mediterraneus\textunderscore )}
\end{itemize}
Que está situado entre terras.
Relativo ao Mediterrâneo ou aos países que o cercam.
Mar interior.
\section{Mediterrânico}
\begin{itemize}
\item {Grp. gram.:adj.}
\end{itemize}
Relativo ao Mediterrâneo.
Situado á beira do Mediterrâneo. Cf. Castilho, \textunderscore Fastos\textunderscore , II, 402 e 403.
\section{Médium}
\begin{itemize}
\item {Grp. gram.:m.}
\end{itemize}
\begin{itemize}
\item {Proveniência:(Lat. \textunderscore medium\textunderscore )}
\end{itemize}
Supposto intermediário entre os vivos e as almas dos mortos.
Pl. \textunderscore médiums\textunderscore .
\section{Mediúmnico}
\begin{itemize}
\item {Grp. gram.:adj.}
\end{itemize}
\begin{itemize}
\item {Utilização:Espir.}
\end{itemize}
(V.medianímico)
\section{Medíumnidade}
\begin{itemize}
\item {Grp. gram.:f.}
\end{itemize}
\begin{itemize}
\item {Utilização:Espir.}
\end{itemize}
(V.medianimidade)
\section{Medivalvular}
\begin{itemize}
\item {Grp. gram.:adj.}
\end{itemize}
\begin{itemize}
\item {Utilização:Bot.}
\end{itemize}
\begin{itemize}
\item {Proveniência:(De \textunderscore médio\textunderscore  + \textunderscore válvula\textunderscore )}
\end{itemize}
Diz-se dos septos, que partem do meio das válvulas para o eixo dos frutos, como succede nas liliáceas.
\section{Medível}
\begin{itemize}
\item {Grp. gram.:adj.}
\end{itemize}
Que se póde medir.
\section{Mêdo}
\begin{itemize}
\item {Grp. gram.:m.}
\end{itemize}
\begin{itemize}
\item {Utilização:Pop.}
\end{itemize}
\begin{itemize}
\item {Proveniência:(Lat. \textunderscore metus\textunderscore )}
\end{itemize}
Temor ou susto, resultante da ideia de umperigo real ou apparente, ou causado pela presença de objecto perigoso ou estranho.
Receio.
Fantasma.
Alma do outro mundo.
\section{Médo}
\begin{itemize}
\item {Grp. gram.:m.}
\end{itemize}
Monte de areia, mais ou menos prolongado, e formado pelo vento, nas vizinhanças do mar; duna:«\textunderscore ...sobiu por hũs medos de areia.\textunderscore »\textunderscore Jorn. de África\textunderscore , X.
(Cp. \textunderscore médão\textunderscore )
\section{Médo}
\begin{itemize}
\item {Grp. gram.:adj.}
\end{itemize}
\begin{itemize}
\item {Grp. gram.:M.}
\end{itemize}
Relativo a Média.
Habitante da Média.
\section{Medo-báctrio}
\begin{itemize}
\item {Grp. gram.:adj.}
\end{itemize}
Relativo aos Médos e Báctrios.
\section{Medocho}
\begin{itemize}
\item {fónica:dô}
\end{itemize}
\begin{itemize}
\item {Grp. gram.:m.}
\end{itemize}
O mesmo que \textunderscore medoiço\textunderscore .
\section{Medoiço}
\begin{itemize}
\item {Grp. gram.:m.}
\end{itemize}
\begin{itemize}
\item {Utilização:Prov.}
\end{itemize}
\begin{itemize}
\item {Utilização:trasm.}
\end{itemize}
Mêda de centeio.
\section{Medol}
\begin{itemize}
\item {Grp. gram.:m.}
\end{itemize}
Árvore de Damão.
\section{Medonhamente}
\begin{itemize}
\item {Grp. gram.:adv.}
\end{itemize}
De modo medonho, de modo assustador.
\section{Medonho}
\begin{itemize}
\item {Grp. gram.:adj.}
\end{itemize}
\begin{itemize}
\item {Grp. gram.:M.}
\end{itemize}
\begin{itemize}
\item {Proveniência:(De \textunderscore mêdo\textunderscore )}
\end{itemize}
Que causa mêdo.
Funesto.
Hediondo.
O mesmo que \textunderscore mandrião\textunderscore , ave.
\section{Médo-persa}
\begin{itemize}
\item {Grp. gram.:adj.}
\end{itemize}
Relativo a Médos e Persas.
\section{Medorro}
\begin{itemize}
\item {fónica:dô}
\end{itemize}
\begin{itemize}
\item {Grp. gram.:m.}
\end{itemize}
\begin{itemize}
\item {Utilização:Prov.}
\end{itemize}
Grande médo.
Oiteiro.
\section{Medra}
\begin{itemize}
\item {Grp. gram.:f.}
\end{itemize}
Acto ou effeito de medrar.
\section{Medrança}
\begin{itemize}
\item {Grp. gram.:f.}
\end{itemize}
\begin{itemize}
\item {Utilização:Prov.}
\end{itemize}
\begin{itemize}
\item {Utilização:trasm.}
\end{itemize}
\begin{itemize}
\item {Proveniência:(De \textunderscore medrar\textunderscore )}
\end{itemize}
Estado daquelle ou daquillo que esta medrando ou crescendo.
Acto de crescer ou medrar.
O mesmo que \textunderscore medra\textunderscore .
Tumor na pelle dos bois, onde se cria um bicho negro: \textunderscore o bicho da medrança\textunderscore .
\section{Medrançoso}
\begin{itemize}
\item {Grp. gram.:adj.}
\end{itemize}
Que tem medrança ou vai medrando. Cf. Castilho, \textunderscore Fastos\textunderscore , I, 308; \textunderscore Outono\textunderscore , p. XIII.
\section{Medrar}
\begin{itemize}
\item {Grp. gram.:v. t.}
\end{itemize}
\begin{itemize}
\item {Grp. gram.:V. i.}
\end{itemize}
\begin{itemize}
\item {Proveniência:(Do lat. \textunderscore meliorare\textunderscore , seg. Cornu)}
\end{itemize}
Desenvolver.
Fazer crescer.
Melhorar.
Aumentar a fortuna de.
Desenvolver-se; crescer.
\section{Medrica}
\begin{itemize}
\item {Grp. gram.:m.  e  f.}
\end{itemize}
O mesmo que \textunderscore medrincas\textunderscore .
\section{Medricas}
\begin{itemize}
\item {Grp. gram.:m.  e  f.}
\end{itemize}
O mesmo que \textunderscore medrincas\textunderscore .
\section{Medrincas}
\begin{itemize}
\item {Grp. gram.:m.  e  f.}
\end{itemize}
\begin{itemize}
\item {Utilização:Fam.}
\end{itemize}
Pessôa, que tem mêdo de qualquer coisa; cagarola.
(Cp. \textunderscore medroso\textunderscore )
\section{Medrio}
\begin{itemize}
\item {Grp. gram.:m.}
\end{itemize}
O mesmo que \textunderscore medra\textunderscore .
\section{Medronhal}
\begin{itemize}
\item {Grp. gram.:m.}
\end{itemize}
Lugar, onde crescem medronhos.
\section{Medronheiro}
\begin{itemize}
\item {Grp. gram.:m.}
\end{itemize}
\begin{itemize}
\item {Proveniência:(De \textunderscore medronho\textunderscore )}
\end{itemize}
Arvore ericácea, de fruto semelhante ao morango.
\section{Medronho}
\begin{itemize}
\item {Grp. gram.:m.}
\end{itemize}
\begin{itemize}
\item {Proveniência:(Do cast. \textunderscore madroño\textunderscore )}
\end{itemize}
Fruto do medronheiro.
\section{Medroso}
\begin{itemize}
\item {Grp. gram.:adj.}
\end{itemize}
Que tem mêdo.
Acanhado; tímido.
Receoso.
Medonho.
(Cp. cast. \textunderscore medroso\textunderscore , talvez do lat. \textunderscore meticulosus\textunderscore )
\section{Medrozan}
\begin{itemize}
\item {Grp. gram.:f.}
\end{itemize}
\begin{itemize}
\item {Utilização:Ant.}
\end{itemize}
\begin{itemize}
\item {Proveniência:(Do ar. \textunderscore madruzon\textunderscore )}
\end{itemize}
Sutura craniana.
\section{Medula}
\begin{itemize}
\item {Grp. gram.:f.}
\end{itemize}
\begin{itemize}
\item {Utilização:Fig.}
\end{itemize}
\begin{itemize}
\item {Utilização:Ant.}
\end{itemize}
\begin{itemize}
\item {Proveniência:(Lat. \textunderscore medulla\textunderscore )}
\end{itemize}
Substância amarelada ou avermelhada, contida na cavidade dos ossos longos, nas cavidades celulares das extremidades dêsses ossos, no díploe dos ossos chatos.
Miôlo, contido no caule das plantas dicotyledóneas.
\textunderscore Espinhal medula\textunderscore , substância, que occupa o canal formado pelos arcos das vértebras e que forma o encéfalo.
A parte mais íntima.
Aquilo que é essencial.
Torcida de candeeiro, o mesmo que \textunderscore matula\textunderscore 3.
\section{Medulante}
\begin{itemize}
\item {Grp. gram.:adj.}
\end{itemize}
\begin{itemize}
\item {Utilização:Des.}
\end{itemize}
\begin{itemize}
\item {Proveniência:(De \textunderscore medular\textunderscore ^2)}
\end{itemize}
Que percorre a medula ou medulas.
\section{Medular}
\begin{itemize}
\item {Grp. gram.:adj.}
\end{itemize}
\begin{itemize}
\item {Proveniência:(Lat. \textunderscore medullaris\textunderscore )}
\end{itemize}
Relativo á medula.
\section{Medular}
\begin{itemize}
\item {Grp. gram.:v. i.}
\end{itemize}
\begin{itemize}
\item {Utilização:Des.}
\end{itemize}
Percorrer as medulas.
\section{Medulina}
\begin{itemize}
\item {Grp. gram.:f.}
\end{itemize}
Princípio imediato, que se acha nas paredes das células da medula dos vegetaes.
\section{Medulite}
\begin{itemize}
\item {Grp. gram.:f.}
\end{itemize}
Inflamação da medula dos ossos.
\section{Meduloso}
\begin{itemize}
\item {Grp. gram.:adj.}
\end{itemize}
\begin{itemize}
\item {Proveniência:(De \textunderscore medula\textunderscore )}
\end{itemize}
Que tem canal medular.
Cuja substância interior é mais branda que a superfície externa.
Diz-se da maneira de bem esculpir ou pintar um objecto flexível e macio.
Aveludado.
\section{Medulla}
\begin{itemize}
\item {Grp. gram.:f.}
\end{itemize}
\begin{itemize}
\item {Utilização:Fig.}
\end{itemize}
\begin{itemize}
\item {Utilização:Ant.}
\end{itemize}
\begin{itemize}
\item {Proveniência:(Lat. \textunderscore medulla\textunderscore )}
\end{itemize}
Substância amarelada ou avermelhada, contida na cavidade dos ossos longos, nas cavidades cellulares das extremidades dêsses ossos, no díploe dos ossos chatos.
Miôlo, contido no caule das plantas dicotyledóneas.
\textunderscore Espinhal medulla\textunderscore , substância, que occupa o canal formado pelos arcos das vértebras e que forma o encéphalo.
A parte mais íntima.
Aquillo que é essencial.
Torcida de candeeiro, o mesmo que \textunderscore matula\textunderscore 3.
\section{Medullante}
\begin{itemize}
\item {Grp. gram.:adj.}
\end{itemize}
\begin{itemize}
\item {Utilização:Des.}
\end{itemize}
\begin{itemize}
\item {Proveniência:(De \textunderscore medullar\textunderscore ^2)}
\end{itemize}
Que percorre a medulla ou medullas.
\section{Medullar}
\begin{itemize}
\item {Grp. gram.:adj.}
\end{itemize}
\begin{itemize}
\item {Proveniência:(Lat. \textunderscore medullaris\textunderscore )}
\end{itemize}
Relativo á medulla.
\section{Medullar}
\begin{itemize}
\item {Grp. gram.:v. i.}
\end{itemize}
\begin{itemize}
\item {Utilização:Des.}
\end{itemize}
Percorrer as medullas.
\section{Medullina}
\begin{itemize}
\item {Grp. gram.:f.}
\end{itemize}
Princípio immediato, que se acha nas paredes das céllulas da medulla dos vegetaes.
\section{Medullite}
\begin{itemize}
\item {Grp. gram.:f.}
\end{itemize}
Inflammação da medulla dos ossos.
\section{Medulloso}
\begin{itemize}
\item {Grp. gram.:adj.}
\end{itemize}
\begin{itemize}
\item {Proveniência:(De \textunderscore medulla\textunderscore )}
\end{itemize}
Que tem canal medullar.
Cuja substância interior é mais branda que a superfície externa.
Diz-se da maneira de bem esculpir ou pintar um objecto flexível e macio.
Avelludado.
\section{Medunha}
\begin{itemize}
\item {Grp. gram.:f.}
\end{itemize}
\begin{itemize}
\item {Utilização:Gír.}
\end{itemize}
Dedos.
\section{Medusa}
\begin{itemize}
\item {Grp. gram.:f.}
\end{itemize}
\begin{itemize}
\item {Proveniência:(Lat. \textunderscore Medusa\textunderscore , n. p.)}
\end{itemize}
Designação scientífica da alforreca.
Variedade de borboleta europeia.
\section{Medusário}
\begin{itemize}
\item {Grp. gram.:adj.}
\end{itemize}
\begin{itemize}
\item {Proveniência:(De \textunderscore medusa\textunderscore )}
\end{itemize}
Semelhante á alforreca.
\section{Meduseu}
\begin{itemize}
\item {Grp. gram.:adj.}
\end{itemize}
\begin{itemize}
\item {Utilização:Poét.}
\end{itemize}
\begin{itemize}
\item {Proveniência:(Lat. \textunderscore medusaeus\textunderscore )}
\end{itemize}
Relativo a Medusa.
Diz-se do cavallo, que a Mythologia chama Pégaso e que nasceu do sangue de Medusa.
Diz-se da fonte, que se chama Hippocrene, o que nasceu no lugar onde Pégaso deu uma patada.
\section{Meeiro}
\begin{itemize}
\item {Grp. gram.:adj.}
\end{itemize}
\begin{itemize}
\item {Grp. gram.:M.}
\end{itemize}
Que tem de sêr dividido ao meio.
Partível em dois quinhões iguaes.
Aquelle que tem metade em certos bens ou interesses.
\section{Meetingueiro}
\begin{itemize}
\item {fónica:mi}
\end{itemize}
\begin{itemize}
\item {Grp. gram.:m.}
\end{itemize}
\begin{itemize}
\item {Utilização:Pop.}
\end{itemize}
\begin{itemize}
\item {Proveniência:(Do ingl. \textunderscore meeting\textunderscore )}
\end{itemize}
Aquelle que frequenta comícios ou costuma falar nêlles.
\section{Mega}
\begin{itemize}
\item {Grp. gram.:f.}
\end{itemize}
\begin{itemize}
\item {Utilização:T. de Lisbôa}
\end{itemize}
Mosquito mais pequeno que o trombeteiro; muchão.
(Cp. \textunderscore melga\textunderscore )
\section{Mega...}
\begin{itemize}
\item {Grp. gram.:pref.}
\end{itemize}
\begin{itemize}
\item {Proveniência:(Do gr. \textunderscore megas\textunderscore )}
\end{itemize}
(design. de \textunderscore grandeza\textunderscore )
\section{Megalantho}
\begin{itemize}
\item {Grp. gram.:adj.}
\end{itemize}
\begin{itemize}
\item {Proveniência:(Do gr. \textunderscore megas\textunderscore  + \textunderscore anthos\textunderscore )}
\end{itemize}
Que tem flôres grandes.
\section{Megalanthropogenesia}
\begin{itemize}
\item {Grp. gram.:f.}
\end{itemize}
\begin{itemize}
\item {Proveniência:(Do gr. \textunderscore magos\textunderscore  + \textunderscore anthropos\textunderscore  + \textunderscore genesis\textunderscore )}
\end{itemize}
Supposta arte de procriar homens de gênio.
\section{Megalanto}
\begin{itemize}
\item {Grp. gram.:adj.}
\end{itemize}
\begin{itemize}
\item {Proveniência:(Do gr. \textunderscore megas\textunderscore  + \textunderscore anthos\textunderscore )}
\end{itemize}
Que tem flôres grandes.
\section{Megalantropogenesia}
\begin{itemize}
\item {Grp. gram.:f.}
\end{itemize}
\begin{itemize}
\item {Proveniência:(Do gr. \textunderscore magos\textunderscore  + \textunderscore anthropos\textunderscore  + \textunderscore genesis\textunderscore )}
\end{itemize}
Suposta arte de procriar homens de gênio.
\section{Megalegoria}
\begin{itemize}
\item {Grp. gram.:f.}
\end{itemize}
\begin{itemize}
\item {Proveniência:(Gr. \textunderscore megalegoria\textunderscore )}
\end{itemize}
Estilo pomposo, magnificente.
\section{Megalésio}
\begin{itemize}
\item {Grp. gram.:adj.}
\end{itemize}
\begin{itemize}
\item {Proveniência:(De \textunderscore Mégale\textunderscore , sobrenome de Cybele)}
\end{itemize}
Diz-se dos jogos ou festas, que os Romanos celebravam em honra de Cybele.
\section{Megalino}
\begin{itemize}
\item {Grp. gram.:m.}
\end{itemize}
\begin{itemize}
\item {Proveniência:(Lat. \textunderscore megalium\textunderscore )}
\end{itemize}
Perfume delicioso, feito de óleo de avelan da Índia, ou de bálsamo, de cana da Arábia, da junca, da canela, etc.
\section{Megálio}
\begin{itemize}
\item {Grp. gram.:m.}
\end{itemize}
\begin{itemize}
\item {Proveniência:(Lat. \textunderscore megalium\textunderscore )}
\end{itemize}
Perfume delicioso, feito de óleo de avelan da Índia, ou de bálsamo, de cana da Arábia, da junca, da canela, etc.
\section{Megalíthico}
\begin{itemize}
\item {Grp. gram.:adj.}
\end{itemize}
\begin{itemize}
\item {Proveniência:(Do gr. \textunderscore megas\textunderscore , \textunderscore megalos\textunderscore  + \textunderscore lithos\textunderscore )}
\end{itemize}
Que é feito de grandes pedras.
\section{Megalítico}
\begin{itemize}
\item {Grp. gram.:adj.}
\end{itemize}
\begin{itemize}
\item {Proveniência:(Do gr. \textunderscore megas\textunderscore , \textunderscore megalos\textunderscore  + \textunderscore lithos\textunderscore )}
\end{itemize}
Que é feito de grandes pedras.
\section{Megalocefalia}
\begin{itemize}
\item {Grp. gram.:f.}
\end{itemize}
Qualidade de megalocéfalo.
\section{Megalocéfalo}
\begin{itemize}
\item {Grp. gram.:adj.}
\end{itemize}
\begin{itemize}
\item {Proveniência:(Do gr. \textunderscore megalos\textunderscore  + \textunderscore kephale\textunderscore )}
\end{itemize}
Que tem cabeça excessivamente grande.
\section{Megalocele}
\begin{itemize}
\item {Grp. gram.:m.}
\end{itemize}
\begin{itemize}
\item {Utilização:Med.}
\end{itemize}
\begin{itemize}
\item {Proveniência:(Do gr. \textunderscore megas\textunderscore , \textunderscore megalos\textunderscore  + \textunderscore kele\textunderscore )}
\end{itemize}
Aumento do volume do ventre.
\section{Megalocephalia}
\begin{itemize}
\item {Grp. gram.:f.}
\end{itemize}
Qualidade de megalocéphalo.
\section{Megalocéphalo}
\begin{itemize}
\item {Grp. gram.:adj.}
\end{itemize}
\begin{itemize}
\item {Proveniência:(Do gr. \textunderscore megalos\textunderscore  + \textunderscore kephale\textunderscore )}
\end{itemize}
Que tem cabeça excessivamente grande.
\section{Megalofonia}
\begin{itemize}
\item {Grp. gram.:f.}
\end{itemize}
\begin{itemize}
\item {Utilização:Med.}
\end{itemize}
\begin{itemize}
\item {Proveniência:(Do gr. \textunderscore megas\textunderscore , \textunderscore megalos\textunderscore  + \textunderscore phone\textunderscore )}
\end{itemize}
Aumento da voz, em certos doentes.
\section{Megaloftalmo}
\begin{itemize}
\item {Grp. gram.:m.}
\end{itemize}
\begin{itemize}
\item {Proveniência:(Do gr. \textunderscore megalos\textunderscore  + \textunderscore ophtalmos\textunderscore )}
\end{itemize}
O mesmo que \textunderscore buftalmo\textunderscore .
\section{Megalógono}
\begin{itemize}
\item {Grp. gram.:adj.}
\end{itemize}
\begin{itemize}
\item {Utilização:Miner.}
\end{itemize}
\begin{itemize}
\item {Proveniência:(Do gr. \textunderscore megalos\textunderscore  + \textunderscore gonos\textunderscore )}
\end{itemize}
Diz-se dos crystaes, cujas faces formam entre si ângulos muito obtusos.
\section{Megalografia}
\begin{itemize}
\item {Grp. gram.:f.}
\end{itemize}
\begin{itemize}
\item {Proveniência:(Do gr. \textunderscore megalos\textunderscore  + \textunderscore graphein\textunderscore )}
\end{itemize}
Descripção dos factos grandiosos.
Desenho ou pintura dêsses factos.
\section{Megalographia}
\begin{itemize}
\item {Grp. gram.:f.}
\end{itemize}
\begin{itemize}
\item {Proveniência:(Do gr. \textunderscore megalos\textunderscore  + \textunderscore graphein\textunderscore )}
\end{itemize}
Descripção dos factos grandiosos.
Desenho ou pintura dêsses factos.
\section{Megalomania}
\begin{itemize}
\item {Grp. gram.:f.}
\end{itemize}
\begin{itemize}
\item {Utilização:Neol.}
\end{itemize}
\begin{itemize}
\item {Proveniência:(Do gr. \textunderscore megas\textunderscore , \textunderscore megalos\textunderscore  e \textunderscore mania\textunderscore )}
\end{itemize}
Mania das grandezas.
\section{Megalophonia}
\begin{itemize}
\item {Grp. gram.:f.}
\end{itemize}
\begin{itemize}
\item {Utilização:Med.}
\end{itemize}
\begin{itemize}
\item {Proveniência:(Do gr. \textunderscore megas\textunderscore , \textunderscore megalos\textunderscore  + \textunderscore phone\textunderscore )}
\end{itemize}
Aumento da voz, em certos doentes.
\section{Megalophtalmo}
\begin{itemize}
\item {Grp. gram.:m.}
\end{itemize}
\begin{itemize}
\item {Proveniência:(Do gr. \textunderscore megalos\textunderscore  + \textunderscore ophtalmos\textunderscore )}
\end{itemize}
O mesmo que \textunderscore buphthalmo\textunderscore .
\section{Megalopia}
\begin{itemize}
\item {Grp. gram.:f.}
\end{itemize}
\begin{itemize}
\item {Proveniência:(Do gr. \textunderscore megale\textunderscore  + \textunderscore ops\textunderscore )}
\end{itemize}
O mesmo que \textunderscore macropia\textunderscore .
\section{Megalópico}
\begin{itemize}
\item {Grp. gram.:adj.}
\end{itemize}
Relativo á megalopia.
\section{Megalóporo}
\begin{itemize}
\item {Grp. gram.:adj.}
\end{itemize}
\begin{itemize}
\item {Proveniência:(Do gr. \textunderscore megale\textunderscore  + \textunderscore poros\textunderscore )}
\end{itemize}
Que tem grandes poros.
\section{Megalosáurio}
\begin{itemize}
\item {fónica:sau}
\end{itemize}
\begin{itemize}
\item {Grp. gram.:m.}
\end{itemize}
\begin{itemize}
\item {Proveniência:(Do gr. \textunderscore megale\textunderscore  + \textunderscore sauros\textunderscore )}
\end{itemize}
Espécie de grande lagarto fóssil.
\section{Megalosplenia}
\begin{itemize}
\item {Grp. gram.:f.}
\end{itemize}
\begin{itemize}
\item {Utilização:Med.}
\end{itemize}
\begin{itemize}
\item {Proveniência:(Do gr. \textunderscore megale\textunderscore  + \textunderscore splen\textunderscore )}
\end{itemize}
Aumento do volume do fígado, sem endurecimento.
\section{Megalossáurio}
\begin{itemize}
\item {Grp. gram.:m.}
\end{itemize}
\begin{itemize}
\item {Proveniência:(Do gr. \textunderscore megale\textunderscore  + \textunderscore sauros\textunderscore )}
\end{itemize}
Espécie de grande lagarto fóssil.
\section{Megalostilo}
\begin{itemize}
\item {Grp. gram.:m.}
\end{itemize}
\begin{itemize}
\item {Proveniência:(Do gr. \textunderscore megas\textunderscore , \textunderscore megalos\textunderscore  + \textunderscore stulos\textunderscore )}
\end{itemize}
Espécie de escaravelho.
\section{Mególastomo}
\begin{itemize}
\item {Grp. gram.:adj.}
\end{itemize}
\begin{itemize}
\item {Utilização:Hist. Nat.}
\end{itemize}
\begin{itemize}
\item {Proveniência:(Do gr. \textunderscore megas\textunderscore , \textunderscore megalos\textunderscore  + \textunderscore stoma\textunderscore )}
\end{itemize}
Que tem boca grande.
\section{Megalostylo}
\begin{itemize}
\item {Grp. gram.:m.}
\end{itemize}
\begin{itemize}
\item {Proveniência:(Do gr. \textunderscore megas\textunderscore , \textunderscore megalos\textunderscore  + \textunderscore stulos\textunderscore )}
\end{itemize}
Espécie de escaravelho.
\section{Megâmetro}
\begin{itemize}
\item {Grp. gram.:m.}
\end{itemize}
\begin{itemize}
\item {Proveniência:(Do gr. \textunderscore megas\textunderscore  + \textunderscore metron\textunderscore )}
\end{itemize}
Instrumento, para medir as distâncias angulares entre os astros.
Instrumento, para determinar longitudes marítimas.
\section{Megárico}
\begin{itemize}
\item {Grp. gram.:adj.}
\end{itemize}
\begin{itemize}
\item {Utilização:Fig.}
\end{itemize}
\begin{itemize}
\item {Proveniência:(De \textunderscore Mégara\textunderscore , n. p.)}
\end{itemize}
Relativo a uma escola philosóphica, fundada em Mégara por Euclides, escola que duvidava do testemunho dos sentidos, admittia a unidade absoluta e considerava o sêr e o bem como idênticos.
Sardónico; próprio de incrédulo:«\textunderscore ...riso megárico\textunderscore ». \textunderscore Eufrosina\textunderscore , 24.
\section{Megascópio}
\begin{itemize}
\item {Grp. gram.:m.}
\end{itemize}
\begin{itemize}
\item {Proveniência:(Do gr. \textunderscore megas\textunderscore  + \textunderscore skopein\textunderscore )}
\end{itemize}
Instrumento de óptica, para se obterem cópias aumentadas de pequenos quadros ou de outros objectos.
\section{Megasemo}
\begin{itemize}
\item {fónica:se}
\end{itemize}
\begin{itemize}
\item {Grp. gram.:m.  e  adj.}
\end{itemize}
Que tem grande orbita ocular, (segundo a classificação anthropológica de Broca).
\section{Megassemo}
\begin{itemize}
\item {Grp. gram.:m.  e  adj.}
\end{itemize}
Que tem grande orbita ocular, (segundo a classificação anthropológica de Broca).
\section{Megastáchia}
\begin{itemize}
\item {fónica:qui}
\end{itemize}
\begin{itemize}
\item {Grp. gram.:f.}
\end{itemize}
Gênero de plantas gramineas.
\section{Megastáquia}
\begin{itemize}
\item {Grp. gram.:f.}
\end{itemize}
Gênero de plantas gramineas.
\section{Megatério}
\begin{itemize}
\item {Grp. gram.:m.}
\end{itemize}
\begin{itemize}
\item {Proveniência:(Do gr. \textunderscore megas\textunderscore  + \textunderscore therion\textunderscore )}
\end{itemize}
Grande mamífero, de que apenas se conhecem ossos fósseis.
\section{Megathério}
\begin{itemize}
\item {Grp. gram.:m.}
\end{itemize}
\begin{itemize}
\item {Proveniência:(Do gr. \textunderscore megas\textunderscore  + \textunderscore therion\textunderscore )}
\end{itemize}
Grande mammífero, de que apenas se conhecem ossos fósseis.
\section{Megengra}
\begin{itemize}
\item {Grp. gram.:f.}
\end{itemize}
Pássaro conirostro, (\textunderscore parus major\textunderscore , Lin.).
\section{Megera}
\begin{itemize}
\item {Grp. gram.:f.}
\end{itemize}
\begin{itemize}
\item {Utilização:Fig.}
\end{itemize}
\begin{itemize}
\item {Proveniência:(De \textunderscore Megera\textunderscore , n. p.)}
\end{itemize}
Gênero de serpentes.
Gênero de borboletas.
Mulher de má índole, cruel.
Mãe desnaturada.
\section{Megistocéfalo}
\begin{itemize}
\item {Grp. gram.:adj.}
\end{itemize}
\begin{itemize}
\item {Proveniência:(Do gr. \textunderscore megistos\textunderscore  + \textunderscore kephale\textunderscore )}
\end{itemize}
Que tem a cabeça enorme.
\section{Megistocéphalo}
\begin{itemize}
\item {Grp. gram.:adj.}
\end{itemize}
\begin{itemize}
\item {Proveniência:(Do gr. \textunderscore megistos\textunderscore  + \textunderscore kephale\textunderscore )}
\end{itemize}
Que tem a cabeça enorme.
\section{Meia}
\begin{itemize}
\item {Grp. gram.:f.}
\end{itemize}
\begin{itemize}
\item {Proveniência:(De \textunderscore meio\textunderscore )}
\end{itemize}
Tecido de malha, para cobrir o pé e parte da perna.
Qualquer peça de vestuário, feita com ponto de malha.
Antiga medida português para líquidos, ainda hoje conhecida na Bairrada, e que corresponde a 6 quartilhos.
\section{Meia-cana}
\begin{itemize}
\item {Grp. gram.:f.}
\end{itemize}
Lima, em fórma de cana, fendida ao meio longitudinalmente.
Ferramenta do mesmo feitio, usada por carpinteiros e marceneiros.
Estria, caneladura.
\section{Meia-canha}
\begin{itemize}
\item {Grp. gram.:f.}
\end{itemize}
\begin{itemize}
\item {Utilização:Bras. do S}
\end{itemize}
Bailado popular, espécie de fandango.
\section{Meia-cara}
\begin{itemize}
\item {Grp. gram.:m.}
\end{itemize}
\begin{itemize}
\item {Utilização:Bras}
\end{itemize}
Escravo, importado por contrabando.
\section{Meia-colubrina}
\begin{itemize}
\item {Grp. gram.:f.}
\end{itemize}
Antiga peça de artilharia.
\section{Meia-corôa}
\begin{itemize}
\item {Grp. gram.:f.}
\end{itemize}
Moeda portuguesa de prata, do valor de 500 reis.
\section{Meiadeiro}
\begin{itemize}
\item {Grp. gram.:m.}
\end{itemize}
\begin{itemize}
\item {Utilização:Ant.}
\end{itemize}
\begin{itemize}
\item {Proveniência:(De \textunderscore meiar\textunderscore )}
\end{itemize}
Aquelle a quem pertence metade de qualquer coisa.
\section{Meia-esquadria}
\begin{itemize}
\item {Grp. gram.:f.}
\end{itemize}
Linha, que divide ao meio um ângulo recto.
Metade da esquadria.
\section{Meiagoó}
\begin{itemize}
\item {Grp. gram.:m.}
\end{itemize}
\begin{itemize}
\item {Utilização:Ant.}
\end{itemize}
O meio de qualquer coisa.
\section{Meia-idade}
\begin{itemize}
\item {Grp. gram.:f.}
\end{itemize}
A idade de quarenta annos proximamente: \textunderscore homem de meia-idade\textunderscore .
A Idade-Média.
\section{Meiaído}
\begin{itemize}
\item {Grp. gram.:m.}
\end{itemize}
\begin{itemize}
\item {Utilização:Ant.}
\end{itemize}
Raia, fronteira, limite.
(De \textunderscore meio\textunderscore ).
\section{Meia-laranja}
\begin{itemize}
\item {Grp. gram.:f.}
\end{itemize}
Escotilha, que dá serventia ás ante-câmaras dos navios.
Qualquer lugar em fórma de semi-círculo.
\section{Meia-lona}
\begin{itemize}
\item {Grp. gram.:f.}
\end{itemize}
Tecido grosso de linho.
\section{Meia-lua}
\begin{itemize}
\item {Grp. gram.:f.}
\end{itemize}
\begin{itemize}
\item {Utilização:Náut.}
\end{itemize}
\begin{itemize}
\item {Utilização:Fort.}
\end{itemize}
Phase da lua, em que esta apresenta um semi-círculo luminoso.
Crescente.
Semi-círculo.
Aquillo que tem a fórma de semi-círculo.
Embarcação de pesca costeira, de fundo chato e terminada em bico á prôa e á popa.
Instrumento, para picar carne.
Peça de metal ou madeira, que substitue a lona do leme, nos escaleres.
Revelim, obra em fórma de meia-lua ou cunha, na contra-escarpa de uma praça de guerra, para defesa de uma porta ou cortina.
\section{Meia-moirisca}
\begin{itemize}
\item {Grp. gram.:f.}
\end{itemize}
\textunderscore Telhado á meia-moirisca\textunderscore , telhado, em que as carreiras são alternadamente moiriscadas e de valladio.
(Cp. \textunderscore moiriscada\textunderscore .)
\section{Meia-murça}
\begin{itemize}
\item {Grp. gram.:f.}
\end{itemize}
Espécie de lima, cuja serrilha ou picado é um pouco menos fino que o da \textunderscore murça\textunderscore ^2.
\section{Meia-nau}
\begin{itemize}
\item {Grp. gram.:f.}
\end{itemize}
Linha mediana e longitudinal do navio, igualmente afastada das duas amuradas.
\section{Meia-noite}
\begin{itemize}
\item {Grp. gram.:f.}
\end{itemize}
Hora ou momento, que divide a noite em duas partes iguaes.
\section{Meiante}
\begin{itemize}
\item {Grp. gram.:adj.}
\end{itemize}
\begin{itemize}
\item {Utilização:Ant.}
\end{itemize}
\begin{itemize}
\item {Proveniência:(De \textunderscore meiar\textunderscore )}
\end{itemize}
Dizia-se do homem de meia-idade.
\section{Meiar}
\begin{itemize}
\item {Grp. gram.:v. t.}
\end{itemize}
(V.mear).
\section{Meia-rotunda}
\begin{itemize}
\item {Grp. gram.:f.}
\end{itemize}
Construcção semi-circular por dentro e por fóra.
\section{Meias}
\begin{itemize}
\item {Grp. gram.:f. pl.}
\end{itemize}
Contrato, em que se dividem igualmente lucros e perdas por duas partes contratantes.
Espécie de arrendamento rural, em que o arrendatário faz as despezas da cultura o divide igualmente os lucros entre si e o senhorio.
Contrato, em que um porco, uma vaca ou outro animal, é cedido pelos donos a quem o sustente e crie, dividindo-se ao depois igualmente o producto da venda do animal pelo dono e pelo criador.
(De \textunderscore meio\textunderscore ).
\section{Meias-partidas}
\begin{itemize}
\item {Grp. gram.:f. pl.}
\end{itemize}
Termos médios, entre os rumos da rosa dos ventos.
\section{Meiatade}
\begin{itemize}
\item {Grp. gram.:f.}
\end{itemize}
\begin{itemize}
\item {Utilização:Ant.}
\end{itemize}
O mesmo que \textunderscore metade\textunderscore .
\section{Meia-tinta}
\begin{itemize}
\item {Grp. gram.:f.}
\end{itemize}
Graduação de côres.
Côr intermédia á luz e á sombra.
\section{Meidado}
\begin{itemize}
\item {Grp. gram.:adj.}
\end{itemize}
\begin{itemize}
\item {Utilização:Ant.}
\end{itemize}
Dividido ao meio.
(Por \textunderscore mediado\textunderscore , de \textunderscore mediar\textunderscore )
\section{Meieiro}
\begin{itemize}
\item {Grp. gram.:adj.}
\end{itemize}
\begin{itemize}
\item {Utilização:T. da Bairrada}
\end{itemize}
\begin{itemize}
\item {Grp. gram.:M.}
\end{itemize}
\begin{itemize}
\item {Utilização:T. da Bairrada}
\end{itemize}
Diz-se de uma localidade, parte da qual pertence a uma freguesia, pertencendo a outra parte a outra freguesia.
O dedo máximo da mão.
(Cp. \textunderscore meeiro\textunderscore )
\section{Meiga}
\begin{itemize}
\item {Grp. gram.:f.}
\end{itemize}
\begin{itemize}
\item {Utilização:Ant.}
\end{itemize}
Aprêço, estima?:«\textunderscore quam pouca meiga faço nesses gostos...\textunderscore »\textunderscore Eufrosina\textunderscore , 176.
(Relaciona-se com \textunderscore meigo\textunderscore ?)
\section{Meigamente}
\begin{itemize}
\item {Grp. gram.:adv.}
\end{itemize}
De modo meigo: com meiguice; ternamente.
\section{Meigar}
\begin{itemize}
\item {Grp. gram.:v. t.}
\end{itemize}
O mesmo que \textunderscore ameigar\textunderscore . Cf. Filinto, XIII, 218.
\section{Meigengro}
\begin{itemize}
\item {Grp. gram.:adj.}
\end{itemize}
\begin{itemize}
\item {Utilização:Des.}
\end{itemize}
Pêco, chôco, (falando-se de frutos)
\section{Meigo}
\begin{itemize}
\item {Grp. gram.:adj.}
\end{itemize}
\begin{itemize}
\item {Proveniência:(Do lat. \textunderscore magicus\textunderscore )}
\end{itemize}
Amável; terno, carinhoso; bondoso.
Suave.
\section{Meiguice}
\begin{itemize}
\item {Grp. gram.:f.}
\end{itemize}
\begin{itemize}
\item {Grp. gram.:Pl.}
\end{itemize}
Qualidade de meigo; carinho, ternura.
Carícias.
Palavras affectuosas, para attrahir a benevolência de alguém.
\section{Meiguiceiro}
\begin{itemize}
\item {Grp. gram.:adj.}
\end{itemize}
Que tem meiguice: carinhoso; melioiro.
\section{Meimendro}
\begin{itemize}
\item {Grp. gram.:m.}
\end{itemize}
\begin{itemize}
\item {Proveniência:(Do lat. \textunderscore milimindrum\textunderscore )}
\end{itemize}
Planta solânea, medicinal.
\section{Meiminho}
\begin{itemize}
\item {Grp. gram.:m.  e  adj.}
\end{itemize}
(V.mindinho)
\section{Meinim}
\begin{itemize}
\item {Grp. gram.:m.}
\end{itemize}
Espécie de tecido antigo. Cf. Arn. Gama, \textunderscore Bailio\textunderscore , 9; \textunderscore Filho do Baldaia\textunderscore , 43.
\section{Meio}
\begin{itemize}
\item {Grp. gram.:adj.}
\end{itemize}
\begin{itemize}
\item {Grp. gram.:M.}
\end{itemize}
\begin{itemize}
\item {Grp. gram.:Loc. adv.}
\end{itemize}
\begin{itemize}
\item {Grp. gram.:Pl.}
\end{itemize}
\begin{itemize}
\item {Proveniência:(Do lat. \textunderscore medius\textunderscore )}
\end{itemize}
Que indica metade ou a primeira metade de alguma coisa: \textunderscore meio metro\textunderscore .
Médio.
Ponto equidistante de dois extremos.
Centro.
Qualquer ponto interior a uma peripheria.
Condição: \textunderscore dar se bem com a gente do seu meio\textunderscore .
Intervenção.
Maneira: \textunderscore não tenho meio de o convencer\textunderscore .
Pessôa ou coisa intermediária.
Aquillo que estabelece communicação.
Possibilidade.
Cada uma das ordens, em que se dividem os talhos das salinas.
Ambiente, em que se realizam certos phenómenos.
\textunderscore Neste em meio\textunderscore , entretanto, entrementes. Cf. Filinto, \textunderscore D. Man.\textunderscore , I, 144; Camillo, \textunderscore Filha do Beg. Adv.\textunderscore 
Por metade; em metade, proximamente; um tanto; não inteiramente:« \textunderscore estava meio deitada.\textunderscore »Camillo, \textunderscore Noites de Lamego\textunderscore , 63.
Haveres.
Recursos para subsistência: \textunderscore pessôa que dispõe de meios\textunderscore .
\section{Meio-bôrdo}
\begin{itemize}
\item {Grp. gram.:m.}
\end{itemize}
\begin{itemize}
\item {Utilização:Gír.}
\end{itemize}
Facada.
\section{Meio-branco}
\begin{itemize}
\item {Grp. gram.:m.}
\end{itemize}
Antiga moéda portuguesa, \textunderscore meio\textunderscore  real \textunderscore branco\textunderscore , do valor de três ceitis.
\section{Meio-busto}
\begin{itemize}
\item {Grp. gram.:m.}
\end{itemize}
Retrato ou effígie, em que só se representa a cabeça e o pescoço.
\section{Meio-corpo}
\begin{itemize}
\item {Grp. gram.:m.}
\end{itemize}
Parte superior de uma figura humana, desde a cintura.
\section{Meio-dia}
\begin{itemize}
\item {Grp. gram.:m.}
\end{itemize}
Hora ou momento, que divide em duas partes iguaes o dia alumiado.
Cada uma dessas duas partes.
O Sul.
\section{Meio-fio}
\begin{itemize}
\item {Grp. gram.:m.}
\end{itemize}
\begin{itemize}
\item {Utilização:Náut.}
\end{itemize}
\begin{itemize}
\item {Utilização:Prov.}
\end{itemize}
\begin{itemize}
\item {Utilização:Carp.}
\end{itemize}
Anteparo, que, no porão, vai da popa á prôa, para equilibrar a carga, impedindo-a de ir de um lado para outro.
Instrumento de carpinteiro.
Chanfradura, no batente da porta ou em caixilhos.
\section{Meiógono}
\begin{itemize}
\item {Grp. gram.:adj.}
\end{itemize}
\begin{itemize}
\item {Utilização:Miner.}
\end{itemize}
\begin{itemize}
\item {Proveniência:(Do gr. \textunderscore meion\textunderscore  + \textunderscore gonos\textunderscore )}
\end{itemize}
Diz-se da substância crystallizada em prismas, cujas faces se inclinam de maneira, que os ângulos formados por ellas vão successivamente deminuindo.
\section{Meio-grosso}
\begin{itemize}
\item {Grp. gram.:m.  e  adj.}
\end{itemize}
Certa qualidade de rapé.
\section{Meior}
\begin{itemize}
\item {Grp. gram.:adj.}
\end{itemize}
\begin{itemize}
\item {Utilização:Ant.}
\end{itemize}
O mesmo que \textunderscore menór\textunderscore .
\section{Meio-relêvo}
\begin{itemize}
\item {Grp. gram.:m.}
\end{itemize}
Figura ou ornato, em que metade do vulto resai de um plano, no sentido da espessura.
\section{Meio-rufo}
\begin{itemize}
\item {Grp. gram.:m.}
\end{itemize}
Espécie de lima, de serrilha ou picado menos grosso que o do rufo^4.
\section{Meios-bastos}
\begin{itemize}
\item {Grp. gram.:m. pl.}
\end{itemize}
\begin{itemize}
\item {Utilização:Pesc.}
\end{itemize}
Rêde do apparelho do arrastar para terra, a qual se liga com o saco.
\section{Meios-meínhos}
\begin{itemize}
\item {Grp. gram.:m. pl.}
\end{itemize}
\begin{itemize}
\item {Utilização:Pesc.}
\end{itemize}
Rêde do saco, ligada aos meios-bastos.
\section{Meio-soprano}
\begin{itemize}
\item {Grp. gram.:m.}
\end{itemize}
\begin{itemize}
\item {Utilização:Mús.}
\end{itemize}
\begin{itemize}
\item {Grp. gram.:F.}
\end{itemize}
Gênero de voz feminina, intermediário ao soprano e ao contralto.
Cantora, que tem essa voz.
\section{Meio-tom}
\begin{itemize}
\item {Grp. gram.:m.}
\end{itemize}
O intervallo mais curto, empregado em música.
\section{Meiri}
\begin{itemize}
\item {Grp. gram.:m.}
\end{itemize}
Planta brasileira, de raiz alimentícia.
\section{Meirinhado}
\begin{itemize}
\item {Grp. gram.:m.}
\end{itemize}
\begin{itemize}
\item {Proveniência:(De \textunderscore meirinho\textunderscore )}
\end{itemize}
Cargo de meirinhos.
Território, sujeito á jurisdicção dos antigos magistrados, chamados meirinhos.
\section{Meirinhar}
\begin{itemize}
\item {Grp. gram.:v. i.}
\end{itemize}
Exercer o offício de meirinho.
\section{Meirinho}
\begin{itemize}
\item {Grp. gram.:m.}
\end{itemize}
\begin{itemize}
\item {Grp. gram.:Adj.}
\end{itemize}
Antigo empregado judicial, correspondente ao moderno official de diligências.
Beleguim.
Antigo magistrado, que, por nomeação real, governava amplamente uma comarca ou um território.
Casta de uva preta da Beira-Alta.
Diz-se do gado que, de verão, pasta nas montanhas, descendo no inverno ás planícies.
E diz-se da lan dêste gado. Cf. G. Vicente, I, 175.
(Contr. de \textunderscore maiorinho\textunderscore , dem. de \textunderscore maior\textunderscore )
\section{Meiru-de-preto}
\begin{itemize}
\item {Grp. gram.:m.}
\end{itemize}
Planta anonácea do Brasil.
\section{Meisnéria}
\begin{itemize}
\item {Grp. gram.:f.}
\end{itemize}
\begin{itemize}
\item {Proveniência:(De \textunderscore Meisner\textunderscore , n. p.)}
\end{itemize}
Gênero de plantas melastomáceas.
\section{Meisom}
\begin{itemize}
\item {Grp. gram.:f.}
\end{itemize}
\begin{itemize}
\item {Utilização:gal}
\end{itemize}
\begin{itemize}
\item {Utilização:Ant.}
\end{itemize}
\begin{itemize}
\item {Proveniência:(Fr. \textunderscore maison\textunderscore )}
\end{itemize}
O mesmo que \textunderscore casa\textunderscore .
\section{Meitade}
\begin{itemize}
\item {Grp. gram.:f.}
\end{itemize}
\begin{itemize}
\item {Utilização:Pop.}
\end{itemize}
\begin{itemize}
\item {Proveniência:(Do lat. \textunderscore medietas\textunderscore , \textunderscore medietatis\textunderscore )}
\end{itemize}
O mesmo ou melhor que \textunderscore metade\textunderscore .
\section{Meitega}
\begin{itemize}
\item {Grp. gram.:f.}
\end{itemize}
\begin{itemize}
\item {Utilização:Ant.}
\end{itemize}
Almôço ou refeição, que se dava aos cobradores dos foros reaes.
O mesmo que \textunderscore almeitiga\textunderscore . Cf. \textunderscore Elucidário\textunderscore .
\section{Meixil}
\begin{itemize}
\item {Grp. gram.:m.}
\end{itemize}
\begin{itemize}
\item {Utilização:Prov.}
\end{itemize}
\begin{itemize}
\item {Utilização:minh.}
\end{itemize}
Travessa, que prende a rabiça ás aivecas do arado.
(Cp. \textunderscore mexilho\textunderscore )
\section{Meixilho}
\begin{itemize}
\item {Grp. gram.:m.}
\end{itemize}
\begin{itemize}
\item {Utilização:Prov.}
\end{itemize}
\begin{itemize}
\item {Utilização:trasm.}
\end{itemize}
O mesmo que \textunderscore meixil\textunderscore .
\section{Meizinha}
\textunderscore f.\textunderscore  (e der.) \textunderscore Ant.\textunderscore 
O mesmo que \textunderscore mèzinha\textunderscore , etc. Cf. Usque, \textunderscore passim\textunderscore .
\section{Mel}
\begin{itemize}
\item {Grp. gram.:m.}
\end{itemize}
\begin{itemize}
\item {Utilização:Bras}
\end{itemize}
\begin{itemize}
\item {Utilização:Fig.}
\end{itemize}
\begin{itemize}
\item {Grp. gram.:Pl.}
\end{itemize}
\begin{itemize}
\item {Proveniência:(Lat. \textunderscore mel\textunderscore )}
\end{itemize}
Substância doce, que as abelhas formam como suco das flôres e que depositam em alvéolos apropriados.
Substância análoga, formada por outros insectos, como a que se encontra no canal medullar das silvas sêcas.
Calda de açúcar, destillada das fôrmas nos respectivos engenhos.
Doçura, suavidade.
\textunderscore Meles\textunderscore  e \textunderscore méis\textunderscore :«\textunderscore ...e eu premia aos panaes os melles abundantes\textunderscore ». Castilho, \textunderscore Geórgicas\textunderscore , 241.«\textunderscore ...ao deus os méis contentam\textunderscore ». Castilho, \textunderscore Fastos\textunderscore , II, 87. Cf. Castilho, \textunderscore Geórgicas\textunderscore , 19, 127 e 245.
\section{Mela}
\begin{itemize}
\item {Grp. gram.:f.}
\end{itemize}
\begin{itemize}
\item {Utilização:Fig.}
\end{itemize}
\begin{itemize}
\item {Utilização:Bras}
\end{itemize}
\begin{itemize}
\item {Utilização:Bras. do S}
\end{itemize}
\begin{itemize}
\item {Utilização:Prov.}
\end{itemize}
\begin{itemize}
\item {Utilização:trasm.}
\end{itemize}
Doença dos vegetaos, que lhes impede a medrança, e torna pecos os frutos.
Doença.
Falta de vigor; cachexia.
O mesmo que \textunderscore bebedeira\textunderscore .
Sova, tunda.
Falha, mossa.
\section{Melaceiro}
\begin{itemize}
\item {Grp. gram.:m.}
\end{itemize}
Vendedor de melaço.
\section{Melácico}
\begin{itemize}
\item {Grp. gram.:adj.}
\end{itemize}
Diz-se de um ácido, que se encontra no melaço.
\section{Melaço}
\begin{itemize}
\item {Grp. gram.:m.}
\end{itemize}
\begin{itemize}
\item {Proveniência:(De \textunderscore mel\textunderscore )}
\end{itemize}
Líquido viscoso, espécie de fezes, que ficam da crystallização do açúcar.
\section{Meladermia}
\begin{itemize}
\item {Grp. gram.:f.}
\end{itemize}
\begin{itemize}
\item {Utilização:Med.}
\end{itemize}
\begin{itemize}
\item {Proveniência:(Do gr. \textunderscore melas\textunderscore  + \textunderscore derma\textunderscore )}
\end{itemize}
Côr escura, que apparece na pelle, em virtude de certas moléstias.
\section{Meladinha}
\begin{itemize}
\item {Grp. gram.:f.}
\end{itemize}
\begin{itemize}
\item {Utilização:Bras. do N}
\end{itemize}
\begin{itemize}
\item {Proveniência:(De \textunderscore mel\textunderscore )}
\end{itemize}
Gênero de plantas labiadas do Brasil, (\textunderscore peltodon radicans\textunderscore ).
Bebida, feita de aguardente e mel.
\section{Meladinho}
\begin{itemize}
\item {Grp. gram.:adj.}
\end{itemize}
\begin{itemize}
\item {Utilização:Prov.}
\end{itemize}
\begin{itemize}
\item {Utilização:minh.}
\end{itemize}
Muito melado, muito magro, muito enfezado.
\section{Melado}
\begin{itemize}
\item {Grp. gram.:adj.}
\end{itemize}
\begin{itemize}
\item {Utilização:Des.}
\end{itemize}
\begin{itemize}
\item {Utilização:Bras}
\end{itemize}
\begin{itemize}
\item {Utilização:Prov.}
\end{itemize}
\begin{itemize}
\item {Utilização:minh.}
\end{itemize}
\begin{itemize}
\item {Grp. gram.:M.}
\end{itemize}
\begin{itemize}
\item {Utilização:Des.}
\end{itemize}
Que é da côr do mel.
Adocicado, doce como o mel.
Diz-se do cavallo, que tem a pelle e o pêlo amarelos.
Magro, rachítico.
Caldo da cana do açúcar, limpo na caldeira e pouco grosso.
Menino órfão de collégio, a quem os gaiatos perguntavam se era o que caíu na talha do mel.
\section{Melado}
\begin{itemize}
\item {Grp. gram.:adj.}
\end{itemize}
\begin{itemize}
\item {Utilização:Bras. do N}
\end{itemize}
\begin{itemize}
\item {Proveniência:(De \textunderscore mela\textunderscore )}
\end{itemize}
Chôco.
Pêco.
Que tem mossas ou falhas no gume.
Bêbedo.
\section{Meladura}
\begin{itemize}
\item {Grp. gram.:f.}
\end{itemize}
\begin{itemize}
\item {Proveniência:(De \textunderscore melar\textunderscore ^1)}
\end{itemize}
Caldeirada de sumo da cana de açúcar.
\section{Melafírico}
\begin{itemize}
\item {Grp. gram.:adj.}
\end{itemize}
Relativo ao meláfiro.
\section{Meláfiro}
\begin{itemize}
\item {Grp. gram.:m.}
\end{itemize}
\begin{itemize}
\item {Proveniência:(Do gr. \textunderscore melas\textunderscore  e de \textunderscore pórphyro\textunderscore )}
\end{itemize}
Pórfiro negro, que é um diabásico com olivina.
\section{Melagastro}
\begin{itemize}
\item {Grp. gram.:adj.}
\end{itemize}
\begin{itemize}
\item {Utilização:Zool.}
\end{itemize}
\begin{itemize}
\item {Proveniência:(Do gr. \textunderscore melas\textunderscore  + \textunderscore gaster\textunderscore )}
\end{itemize}
Diz-se do animal, que tem o ventre negro.
\section{Melaína}
\begin{itemize}
\item {Grp. gram.:f.}
\end{itemize}
\begin{itemize}
\item {Proveniência:(Do gr. \textunderscore melas\textunderscore )}
\end{itemize}
Matéria negra, que segregam os molluscos cephalópodes.
Pigmento da pelle dos Negros.
\section{Melaleuca}
\begin{itemize}
\item {Grp. gram.:f.}
\end{itemize}
\begin{itemize}
\item {Proveniência:(Do gr. \textunderscore melas\textunderscore  + \textunderscore leukos\textunderscore )}
\end{itemize}
Gênero de plantas myrtáceas.
\section{Melam}
\begin{itemize}
\item {Grp. gram.:m.}
\end{itemize}
\begin{itemize}
\item {Utilização:Chím.}
\end{itemize}
Substância chímica, branca, descoberta nos resíduos insolúveis que se obtêm, destilando-se a mistura de 1 parte de sulfo-cyaneto de potássio com 2 partes de sal ammoníaco.
\section{Melambo}
\begin{itemize}
\item {Grp. gram.:m.}
\end{itemize}
Árvore magnoliácea do Brasil.
Casca resinosa e amarga dessa árvore.
\section{Melampirina}
\begin{itemize}
\item {Grp. gram.:f.}
\end{itemize}
Princípio, extraido do melampiro.
\section{Melampiro}
\begin{itemize}
\item {Grp. gram.:m.}
\end{itemize}
\begin{itemize}
\item {Proveniência:(Gr. \textunderscore melampuros\textunderscore )}
\end{itemize}
Planta escrofularínea, parasita dos trigaes.
\section{Melampódio}
\begin{itemize}
\item {Grp. gram.:m.}
\end{itemize}
\begin{itemize}
\item {Proveniência:(Do gr. \textunderscore melamos\textunderscore  + \textunderscore pous\textunderscore , \textunderscore podos\textunderscore )}
\end{itemize}
Gênero do ervas americanas.
\section{Melampreáceas}
\begin{itemize}
\item {Grp. gram.:f. pl.}
\end{itemize}
\begin{itemize}
\item {Utilização:Bot.}
\end{itemize}
O mesmo que \textunderscore escrofularíneas\textunderscore .
\section{Melampyrina}
\begin{itemize}
\item {Grp. gram.:f.}
\end{itemize}
Princípio, extrahido do melampyro.
\section{Melampyro}
\begin{itemize}
\item {Grp. gram.:m.}
\end{itemize}
\begin{itemize}
\item {Proveniência:(Gr. \textunderscore melampuros\textunderscore )}
\end{itemize}
Planta escrofularínea, parasita dos trigaes.
\section{Melanagogo}
\begin{itemize}
\item {Grp. gram.:m.  e  adj.}
\end{itemize}
\begin{itemize}
\item {Utilização:Med.}
\end{itemize}
\begin{itemize}
\item {Proveniência:(Do gr. \textunderscore melas\textunderscore  + \textunderscore agogos\textunderscore )}
\end{itemize}
Diz-se do medicamento, que se suppõe têr a propriedade de fazer expellir os humores negros ou a atrabílis.
\section{Melanantho}
\begin{itemize}
\item {Grp. gram.:adj.}
\end{itemize}
\begin{itemize}
\item {Proveniência:(Do gr. \textunderscore melanos\textunderscore  + \textunderscore anthos\textunderscore )}
\end{itemize}
Que tem flôres negras.
\section{Melananto}
\begin{itemize}
\item {Grp. gram.:adj.}
\end{itemize}
\begin{itemize}
\item {Proveniência:(Do gr. \textunderscore melanos\textunderscore  + \textunderscore anthos\textunderscore )}
\end{itemize}
Que tem flôres negras.
\section{Melanchlorose}
\begin{itemize}
\item {Grp. gram.:f.}
\end{itemize}
\begin{itemize}
\item {Utilização:Med.}
\end{itemize}
\begin{itemize}
\item {Proveniência:(Do gr. \textunderscore melas\textunderscore  + \textunderscore khorosis\textunderscore )}
\end{itemize}
Icterícia, que dá á pelle uma côr amarela denegrida ou esverdeada.
\section{Melancia}
\begin{itemize}
\item {Grp. gram.:f.}
\end{itemize}
Planta cucurbitácea.
Fruto dessa planta.
Variedade de maçan.
(Refl. de \textunderscore melão\textunderscore )
\section{Melancial}
\begin{itemize}
\item {Grp. gram.:m.}
\end{itemize}
Terreno, em que crescem melancias.
Producção de melancias.
\section{Melancieira}
\begin{itemize}
\item {Grp. gram.:f.}
\end{itemize}
Melancia, planta.
Vendedora de melancias.
\section{Melanclorose}
\begin{itemize}
\item {Grp. gram.:f.}
\end{itemize}
\begin{itemize}
\item {Utilização:Med.}
\end{itemize}
\begin{itemize}
\item {Proveniência:(Do gr. \textunderscore melas\textunderscore  + \textunderscore khorosis\textunderscore )}
\end{itemize}
Icterícia, que dá á pele uma côr amarela denegrida ou esverdeada.
\section{Melancolia}
\begin{itemize}
\item {Grp. gram.:f.}
\end{itemize}
\begin{itemize}
\item {Proveniência:(Gr. \textunderscore melankholia\textunderscore )}
\end{itemize}
Tristeza; desgosto.
Doença mental, acompanhada de tristeza e apprehensões; hypocondria.
\section{Melancolicamente}
\begin{itemize}
\item {Grp. gram.:adv.}
\end{itemize}
De modo melancólico.
Com tristeza; tristemente.
\section{Melancólico}
\begin{itemize}
\item {Grp. gram.:adj.}
\end{itemize}
\begin{itemize}
\item {Proveniência:(Lat. \textunderscore melancholicus\textunderscore )}
\end{itemize}
Que tem melancolia.
Triste.
Que infunde melancolia.
\section{Melancolizador}
\begin{itemize}
\item {Grp. gram.:adj.}
\end{itemize}
\begin{itemize}
\item {Proveniência:(De \textunderscore melancolizar\textunderscore )}
\end{itemize}
Que causa melancolia.
\section{Melancolizar}
\begin{itemize}
\item {Grp. gram.:v. t.}
\end{itemize}
\begin{itemize}
\item {Proveniência:(De \textunderscore melancolia\textunderscore )}
\end{itemize}
Tornar melancólico.
\section{Melancónico}
\begin{itemize}
\item {Grp. gram.:adj.}
\end{itemize}
\begin{itemize}
\item {Utilização:Ant.}
\end{itemize}
O mesmo que \textunderscore melancólico\textunderscore . Cf. \textunderscore Aulegrafia\textunderscore , 10.
\section{Melancónio}
\begin{itemize}
\item {Grp. gram.:m.}
\end{itemize}
Gênero de cogumelos.
\section{Melanconizar}
\begin{itemize}
\item {Grp. gram.:v. t.}
\end{itemize}
\begin{itemize}
\item {Utilização:Ant.}
\end{itemize}
\begin{itemize}
\item {Proveniência:(De \textunderscore melancónico\textunderscore )}
\end{itemize}
O mesmo que \textunderscore melancolizar\textunderscore . Cf. \textunderscore Viriato Trág\textunderscore ., III, 33.
\section{Melancrânide}
\begin{itemize}
\item {Grp. gram.:f.}
\end{itemize}
Gênero de plantas cyperáceas.
\section{Melândria}
\begin{itemize}
\item {Grp. gram.:f.}
\end{itemize}
\begin{itemize}
\item {Proveniência:(Do gr. \textunderscore melas\textunderscore  + \textunderscore aner\textunderscore , \textunderscore andros\textunderscore )}
\end{itemize}
Gênero de insectos coleópteros heterómeros.
\section{Melanemia}
\begin{itemize}
\item {Grp. gram.:f.}
\end{itemize}
\begin{itemize}
\item {Utilização:Med.}
\end{itemize}
\begin{itemize}
\item {Proveniência:(Do gr. \textunderscore melanos\textunderscore  + \textunderscore haima\textunderscore )}
\end{itemize}
Estado mórbido, em que o sangue apresenta o carácter de venoso nos systemas arterial o capillar.
\section{Melanemo}
\begin{itemize}
\item {Grp. gram.:m.}
\end{itemize}
\begin{itemize}
\item {Proveniência:(Do gr. \textunderscore melanos\textunderscore  + \textunderscore haima\textunderscore )}
\end{itemize}
Matéria negra, vomitada dejectada pelos doentes de febre amarela.
\section{Melanésios}
\begin{itemize}
\item {Grp. gram.:m. pl.}
\end{itemize}
Selvagens da Oceânia.
\section{Melânia}
\begin{itemize}
\item {Grp. gram.:f.}
\end{itemize}
\begin{itemize}
\item {Utilização:Des.}
\end{itemize}
Qualidade daquillo que é sombrio ou escuro.
Espécie de tecido ondeado, de lan ou seda, próprio para decorações. Cf. Camillo, \textunderscore Caveira\textunderscore , 449, 451, 452 e 453.
\section{Melânico}
\begin{itemize}
\item {Grp. gram.:adj.}
\end{itemize}
\begin{itemize}
\item {Proveniência:(Do gr. \textunderscore melas\textunderscore )}
\end{itemize}
Diz-se de um ácido da urina.
\section{Melanina}
\begin{itemize}
\item {Grp. gram.:f.}
\end{itemize}
O mesmo que \textunderscore melaína\textunderscore .
\section{Melânios}
\begin{itemize}
\item {Grp. gram.:m. pl.}
\end{itemize}
\begin{itemize}
\item {Proveniência:(Do gr. \textunderscore melas\textunderscore )}
\end{itemize}
Família de molluscos gasterópodes.
\section{Melanipa}
\begin{itemize}
\item {Grp. gram.:f.}
\end{itemize}
Gênero de insectos lepidópteros nocturnos.
Gênero de falenas.
\section{Melanippa}
\begin{itemize}
\item {Grp. gram.:f.}
\end{itemize}
Gênero de insectos lepidópteros nocturnos.
Gênero de phalenas.
\section{Melanismo}
\begin{itemize}
\item {Grp. gram.:m.}
\end{itemize}
\begin{itemize}
\item {Utilização:Med.}
\end{itemize}
\begin{itemize}
\item {Proveniência:(Do gr. \textunderscore melas\textunderscore )}
\end{itemize}
Anomalia, caracterizada pela côr, accidentalmente negra ou escura, no pêlo dos animaes.
\section{Melanita}
\begin{itemize}
\item {Grp. gram.:f.}
\end{itemize}
\begin{itemize}
\item {Proveniência:(Do gr. \textunderscore melas\textunderscore )}
\end{itemize}
Mineral escuro, que se acha entre matérias vulcânicas.
\section{Melanizar}
\begin{itemize}
\item {Grp. gram.:v. t.}
\end{itemize}
\begin{itemize}
\item {Utilização:Neol.}
\end{itemize}
\begin{itemize}
\item {Proveniência:(Do gr. \textunderscore melas\textunderscore )}
\end{itemize}
Tornar escuro.
\section{Melanocaroíta}
\begin{itemize}
\item {Grp. gram.:f.}
\end{itemize}
\begin{itemize}
\item {Proveniência:(Do gr. \textunderscore melas\textunderscore  + \textunderscore khroa\textunderscore )}
\end{itemize}
Mineral da Sibéria, de côr violácea ou avermelhada, e que é uma espécie de chumbo chromatado.
\section{Melanocarpo}
\begin{itemize}
\item {Grp. gram.:adj.}
\end{itemize}
\begin{itemize}
\item {Utilização:Bot.}
\end{itemize}
\begin{itemize}
\item {Proveniência:(Do gr. \textunderscore melas\textunderscore  + \textunderscore karpos\textunderscore )}
\end{itemize}
Que dá frutos negros.
\section{Melanocéfalo}
\begin{itemize}
\item {Grp. gram.:adj.}
\end{itemize}
\begin{itemize}
\item {Utilização:Zool.}
\end{itemize}
\begin{itemize}
\item {Proveniência:(Do gr. \textunderscore melas\textunderscore  + \textunderscore kephale\textunderscore )}
\end{itemize}
Que tem cabeça negra.
\section{Melanocéphalo}
\begin{itemize}
\item {Grp. gram.:adj.}
\end{itemize}
\begin{itemize}
\item {Utilização:Zool.}
\end{itemize}
\begin{itemize}
\item {Proveniência:(Do gr. \textunderscore melas\textunderscore  + \textunderscore kephale\textunderscore )}
\end{itemize}
Que tem cabeça negra.
\section{Melanoceraso}
\begin{itemize}
\item {Grp. gram.:m.}
\end{itemize}
Designação antiga da belladona.
\section{Melanócero}
\begin{itemize}
\item {Grp. gram.:adj.}
\end{itemize}
\begin{itemize}
\item {Utilização:Zool.}
\end{itemize}
\begin{itemize}
\item {Proveniência:(Do gr. \textunderscore melas\textunderscore  + \textunderscore keras\textunderscore )}
\end{itemize}
Que tem negros os cornos ou as antennas.
\section{Melanócomo}
\begin{itemize}
\item {Grp. gram.:adj.}
\end{itemize}
\begin{itemize}
\item {Utilização:Hist. Nat.}
\end{itemize}
\begin{itemize}
\item {Proveniência:(Do gr. \textunderscore melas\textunderscore , \textunderscore metanos\textunderscore  + \textunderscore kome\textunderscore )}
\end{itemize}
Que tem o cabello ou pêlo escuro.
\section{Melanocroíta}
\begin{itemize}
\item {Grp. gram.:f.}
\end{itemize}
\begin{itemize}
\item {Proveniência:(Do gr. \textunderscore melas\textunderscore  + \textunderscore khroa\textunderscore )}
\end{itemize}
Mineral da Sibéria, de côr violácea ou avermelhada, e que é uma espécie de chumbo chromatado.
\section{Melanodermia}
\begin{itemize}
\item {Grp. gram.:f.}
\end{itemize}
Pigmentação mórbida da pelle.
\section{Melanoftalmo}
\begin{itemize}
\item {Grp. gram.:adj.}
\end{itemize}
\begin{itemize}
\item {Proveniência:(Do gr. \textunderscore melas\textunderscore  + \textunderscore ophthalmos\textunderscore )}
\end{itemize}
Que tem olhos negros.
Que tem manchas rodeadas de um círculo negro, figurando um olho.
\section{Melano-gállico}
\begin{itemize}
\item {Grp. gram.:adj.}
\end{itemize}
Diz-se de um ácido, que é o resíduo da destillação dos ácidos tânnico, gállico o pyrogállico.
\section{Melanografita}
\begin{itemize}
\item {Grp. gram.:f.}
\end{itemize}
\begin{itemize}
\item {Utilização:Miner.}
\end{itemize}
\begin{itemize}
\item {Proveniência:(Do gr. \textunderscore melas\textunderscore  + \textunderscore graphein\textunderscore )}
\end{itemize}
Pedra, que apresenta traços escuros, semelhando desenhos.
\section{Melanographita}
\begin{itemize}
\item {Grp. gram.:f.}
\end{itemize}
\begin{itemize}
\item {Utilização:Miner.}
\end{itemize}
\begin{itemize}
\item {Proveniência:(Do gr. \textunderscore melas\textunderscore  + \textunderscore graphein\textunderscore )}
\end{itemize}
Pedra, que apresenta traços escuros, semelhando desenhos.
\section{Melanoma}
\begin{itemize}
\item {Grp. gram.:m.}
\end{itemize}
\begin{itemize}
\item {Proveniência:(Do gr. \textunderscore melas\textunderscore , \textunderscore metanos\textunderscore )}
\end{itemize}
Tumor pigmentoso.
\section{Melanope}
\begin{itemize}
\item {Grp. gram.:adj.}
\end{itemize}
\begin{itemize}
\item {Utilização:Zool.}
\end{itemize}
\begin{itemize}
\item {Proveniência:(Do gr. \textunderscore melas\textunderscore  + \textunderscore ops\textunderscore )}
\end{itemize}
Que tem olhos negros.
\section{Melanophthalmo}
\begin{itemize}
\item {Grp. gram.:adj.}
\end{itemize}
\begin{itemize}
\item {Proveniência:(Do gr. \textunderscore melas\textunderscore  + \textunderscore ophthalmos\textunderscore )}
\end{itemize}
Que tem olhos negros.
Que tem manchas rodeadas de um círculo negro, figurando um olho.
\section{Melanóptero}
\begin{itemize}
\item {Grp. gram.:adj.}
\end{itemize}
\begin{itemize}
\item {Utilização:Zool.}
\end{itemize}
\begin{itemize}
\item {Proveniência:(Do gr. \textunderscore melas\textunderscore  + \textunderscore pteron\textunderscore )}
\end{itemize}
Que tem asas ou elytros negros.
\section{Melanoquina}
\begin{itemize}
\item {Grp. gram.:f.}
\end{itemize}
\begin{itemize}
\item {Utilização:Chím.}
\end{itemize}
Producto da decomposição da quinina pelo chloro.
\section{Melanose}
\begin{itemize}
\item {Grp. gram.:f.}
\end{itemize}
\begin{itemize}
\item {Proveniência:(Gr. \textunderscore melanosis\textunderscore )}
\end{itemize}
Tecido negro e anormal, que se desenvolve no corpo.
Cogumelo microscópico, que ataca as videiras americanas.
\section{Melanospermo}
\begin{itemize}
\item {Grp. gram.:adj.}
\end{itemize}
\begin{itemize}
\item {Utilização:Bot.}
\end{itemize}
\begin{itemize}
\item {Proveniência:(Do gr. \textunderscore melas\textunderscore , \textunderscore melanos\textunderscore  + \textunderscore sperma\textunderscore )}
\end{itemize}
Diz-se dos vegetaes, cujas sementes são negras.
\section{Melanóstola}
\begin{itemize}
\item {Grp. gram.:f.}
\end{itemize}
\begin{itemize}
\item {Proveniência:(Do gr. \textunderscore melas\textunderscore  + \textunderscore stole\textunderscore )}
\end{itemize}
Gênero de insectos coleópteros heterómenos.
\section{Melanóstomo}
\begin{itemize}
\item {Grp. gram.:adj.}
\end{itemize}
\begin{itemize}
\item {Utilização:Zool.}
\end{itemize}
\begin{itemize}
\item {Proveniência:(Do gr. \textunderscore melas\textunderscore  + \textunderscore stoma\textunderscore )}
\end{itemize}
Que tem boca negra.
\section{Malanótico}
\begin{itemize}
\item {Grp. gram.:m.  e  adj.}
\end{itemize}
O que tem melanose.
\section{Melanótrico}
\begin{itemize}
\item {Grp. gram.:adj.}
\end{itemize}
\begin{itemize}
\item {Proveniência:(Do gr. \textunderscore melas\textunderscore  + \textunderscore trikhos\textunderscore )}
\end{itemize}
Que tem cabellos pretos.
\section{Melanoxantho}
\begin{itemize}
\item {Grp. gram.:m.}
\end{itemize}
\begin{itemize}
\item {Proveniência:(Do gr. \textunderscore melas\textunderscore  + \textunderscore xanthos\textunderscore )}
\end{itemize}
Gênero de insectos coleópteros pentâmeros.
\section{Melanoxanto}
\begin{itemize}
\item {fónica:csan}
\end{itemize}
\begin{itemize}
\item {Grp. gram.:m.}
\end{itemize}
\begin{itemize}
\item {Proveniência:(Do gr. \textunderscore melas\textunderscore  + \textunderscore xanthos\textunderscore )}
\end{itemize}
Gênero de insectos coleópteros pentâmeros.
\section{Melanóxilo}
\begin{itemize}
\item {fónica:csi}
\end{itemize}
\begin{itemize}
\item {Grp. gram.:m.}
\end{itemize}
\begin{itemize}
\item {Proveniência:(Do gr. \textunderscore melas\textunderscore  + \textunderscore xulon\textunderscore )}
\end{itemize}
Gênero de plantas leguminosas, cuja madeira é preta.
\section{Melanóxylo}
\begin{itemize}
\item {Grp. gram.:m.}
\end{itemize}
\begin{itemize}
\item {Proveniência:(Do gr. \textunderscore melas\textunderscore  + \textunderscore xulon\textunderscore )}
\end{itemize}
Gênero de plantas leguminosas, cuja madeira é preta.
\section{Melantáceas}
\begin{itemize}
\item {Grp. gram.:f. pl.}
\end{itemize}
(V.colchicáceas)
\section{Melantáceo}
\begin{itemize}
\item {Grp. gram.:adj.}
\end{itemize}
Relativo ou semelhante ao melanto.
\section{Melântemo}
\begin{itemize}
\item {Grp. gram.:m.}
\end{itemize}
\begin{itemize}
\item {Proveniência:(Gr. \textunderscore melanthemon\textunderscore )}
\end{itemize}
Designação antiga da camomila.
\section{Melantera}
\begin{itemize}
\item {Grp. gram.:f.}
\end{itemize}
\begin{itemize}
\item {Proveniência:(Do gr. \textunderscore melas\textunderscore  + \textunderscore antheros\textunderscore )}
\end{itemize}
Gênero de plantas, da fam. das compostas.
\section{Melantéria}
\begin{itemize}
\item {Grp. gram.:f.}
\end{itemize}
\begin{itemize}
\item {Proveniência:(Gr. \textunderscore melanteria\textunderscore )}
\end{itemize}
Espécie de pez, usado por cordoeiros, entre os antigos.
Espécie de greda, com que se tingia de negro o calçado.
\section{Melanterita}
\begin{itemize}
\item {Grp. gram.:f.}
\end{itemize}
Xisto negro, com que se póde desenhar.
\section{Melantesa}
\begin{itemize}
\item {Grp. gram.:f.}
\end{itemize}
Gênero de plantas euforbiáceas.
(Cp. \textunderscore melantho\textunderscore )
\section{Melantháceas}
\begin{itemize}
\item {Grp. gram.:f. pl.}
\end{itemize}
(V.colchicáceas)
\section{Melanthâceo}
\begin{itemize}
\item {Grp. gram.:adj.}
\end{itemize}
Relativo ou semelhante ao melantho.
\section{Melânthemo}
\begin{itemize}
\item {Grp. gram.:m.}
\end{itemize}
\begin{itemize}
\item {Proveniência:(Gr. \textunderscore melanthemon\textunderscore )}
\end{itemize}
Designação antiga da camomila.
\section{Melanthera}
\begin{itemize}
\item {Grp. gram.:f.}
\end{itemize}
\begin{itemize}
\item {Proveniência:(Do gr. \textunderscore melas\textunderscore  + \textunderscore antheros\textunderscore )}
\end{itemize}
Gênero de plantas, da fam. das compostas.
\section{Melantherita}
\begin{itemize}
\item {Grp. gram.:f.}
\end{itemize}
Xisto negro, com que se póde desenhar.
\section{Melanthesa}
\begin{itemize}
\item {Grp. gram.:f.}
\end{itemize}
Gênero de plantas euphorbiáceas.
(Cp. \textunderscore melantho\textunderscore )
\section{Melânthia}
\begin{itemize}
\item {Grp. gram.:f.}
\end{itemize}
Gênero de insectos lepidópleros nocturnos.
(Cp. \textunderscore melantho\textunderscore )
\section{Melânthio}
\begin{itemize}
\item {Grp. gram.:m.}
\end{itemize}
Planta, o mesmo que \textunderscore melantho\textunderscore .
\section{Melantho}
\begin{itemize}
\item {Grp. gram.:m.}
\end{itemize}
\begin{itemize}
\item {Proveniência:(Do gr. \textunderscore melas\textunderscore  + \textunderscore anthos\textunderscore )}
\end{itemize}
Planta, de raiz bulbosa, do Cabo da Bôa-Esperança.
Gênero de insectos coleópteros pentâmeros.
\section{Melântia}
\begin{itemize}
\item {Grp. gram.:f.}
\end{itemize}
Gênero de insectos lepidópleros nocturnos.
(Cp. \textunderscore melanto\textunderscore )
\section{Melântio}
\begin{itemize}
\item {Grp. gram.:m.}
\end{itemize}
Planta, o mesmo que \textunderscore melanto\textunderscore .
\section{Melanto}
\begin{itemize}
\item {Grp. gram.:m.}
\end{itemize}
\begin{itemize}
\item {Proveniência:(Do gr. \textunderscore melas\textunderscore  + \textunderscore anthos\textunderscore )}
\end{itemize}
Planta, de raiz bulbosa, do Cabo da Bôa-Esperança.
Gênero de insectos coleópteros pentâmeros.
\section{Melanuria}
\begin{itemize}
\item {Grp. gram.:f.}
\end{itemize}
\begin{itemize}
\item {Utilização:Med.}
\end{itemize}
\begin{itemize}
\item {Proveniência:(Do gr. \textunderscore melas\textunderscore  + \textunderscore ouron\textunderscore )}
\end{itemize}
Emissão de urina preta, denegrida ou azulada.
\section{Melanurina}
\begin{itemize}
\item {Grp. gram.:f.}
\end{itemize}
\begin{itemize}
\item {Utilização:Med.}
\end{itemize}
\begin{itemize}
\item {Proveniência:(Do gr. \textunderscore melas\textunderscore  + \textunderscore ourou\textunderscore )}
\end{itemize}
Substância negra, que se acha na urina de certos enfermos.
\section{Melanuro}
\begin{itemize}
\item {Grp. gram.:adj.}
\end{itemize}
\begin{itemize}
\item {Utilização:Zool.}
\end{itemize}
\begin{itemize}
\item {Proveniência:(Do gr. \textunderscore melas\textunderscore  + \textunderscore oura\textunderscore )}
\end{itemize}
Que tem cauda negra.
\section{Melanzela}
\begin{itemize}
\item {Grp. gram.:f.}
\end{itemize}
Planta trepadeira da ilha de San-Thomé.
\section{Melão}
\begin{itemize}
\item {Grp. gram.:m.}
\end{itemize}
\begin{itemize}
\item {Utilização:Ext.}
\end{itemize}
\begin{itemize}
\item {Proveniência:(Lat. \textunderscore melo\textunderscore )}
\end{itemize}
Fruto do meloeiro.
O meloeiro.
\section{Melão-da-Índia}
\begin{itemize}
\item {Grp. gram.:m.}
\end{itemize}
O mesmo que \textunderscore pateca\textunderscore .
\section{Melaphýrico}
\begin{itemize}
\item {Grp. gram.:adj.}
\end{itemize}
Relativo ao meláphyro.
\section{Meláphyro}
\begin{itemize}
\item {Grp. gram.:m.}
\end{itemize}
\begin{itemize}
\item {Proveniência:(Do gr. \textunderscore melas\textunderscore  e de \textunderscore pórphyro\textunderscore )}
\end{itemize}
Pórphyro negro, que é um diabásico com olivina.
\section{Melápio}
\begin{itemize}
\item {Grp. gram.:m.}
\end{itemize}
\begin{itemize}
\item {Proveniência:(De \textunderscore mel\textunderscore )}
\end{itemize}
Variedade de pero doce.
\section{Meláptero}
\begin{itemize}
\item {Grp. gram.:adj.}
\end{itemize}
\begin{itemize}
\item {Utilização:Zool.}
\end{itemize}
\begin{itemize}
\item {Proveniência:(Do gr. \textunderscore melas\textunderscore  + \textunderscore pteron\textunderscore )}
\end{itemize}
Que tem asas ou barbatanas negras.
\section{Melar}
\begin{itemize}
\item {Grp. gram.:v. t.}
\end{itemize}
\begin{itemize}
\item {Grp. gram.:V. i.}
\end{itemize}
\begin{itemize}
\item {Utilização:Bras. de Piauí}
\end{itemize}
Adoçar com mel.
Untar ou cobrir de mel.
Dar côr de mel a.
Procurar o mel das abelhas do mato.
\section{Melar}
\begin{itemize}
\item {Grp. gram.:v. t.}
\end{itemize}
\begin{itemize}
\item {Grp. gram.:V. i.}
\end{itemize}
\begin{itemize}
\item {Grp. gram.:V. p.}
\end{itemize}
\begin{itemize}
\item {Utilização:Bras}
\end{itemize}
Produzir mela em.
Fazer mossas em; cortar, retalhar: \textunderscore se percebemos esta embrulhada, melem-nos\textunderscore .«\textunderscore Me melem, se eu o entendo\textunderscore ».
Têr mela.
Tornar-se pêco, chocho.
Embebedar-se.
\section{Melasicterícia}
\begin{itemize}
\item {Grp. gram.:f.}
\end{itemize}
\begin{itemize}
\item {Utilização:Med.}
\end{itemize}
\begin{itemize}
\item {Proveniência:(Do gr. \textunderscore melas\textunderscore  + \textunderscore ikteros\textunderscore )}
\end{itemize}
Icterícia, que faz negra a pelle.
\section{Melasmo}
\begin{itemize}
\item {Grp. gram.:m.}
\end{itemize}
\begin{itemize}
\item {Utilização:Med.}
\end{itemize}
\begin{itemize}
\item {Proveniência:(Do gr. \textunderscore melas\textunderscore )}
\end{itemize}
Mancha escura, que apparece principalmente nas pernas dos velhos.
\section{Melasomos}
\begin{itemize}
\item {fónica:so}
\end{itemize}
\begin{itemize}
\item {Grp. gram.:m. pl.}
\end{itemize}
\begin{itemize}
\item {Proveniência:(Do gr. \textunderscore melas\textunderscore  + \textunderscore somos\textunderscore )}
\end{itemize}
Família de insectos coleópteros, que comprehende os coleópteros que têm corpo escuro e cinzento.
\section{Melassomos}
\begin{itemize}
\item {Grp. gram.:m. pl.}
\end{itemize}
\begin{itemize}
\item {Proveniência:(Do gr. \textunderscore melas\textunderscore  + \textunderscore somos\textunderscore )}
\end{itemize}
Família de insectos coleópteros, que comprehende os coleópteros que têm corpo escuro e cinzento.
\section{Melastomáceas}
\begin{itemize}
\item {Grp. gram.:f. pl.}
\end{itemize}
\begin{itemize}
\item {Grp. gram.:f.}
\end{itemize}
\begin{itemize}
\item {Utilização:Med.}
\end{itemize}
\begin{itemize}
\item {Proveniência:(De \textunderscore melastomáceo\textunderscore )}
\end{itemize}
Famíl* Melastrophia,
Atrophia de um membro.
(Do gr. \textunderscore melos\textunderscore  + \textunderscore a\textunderscore  + \textunderscore trephein\textunderscore )ia de plantas, que tem por typo o melástomo.
\section{Melastomáceo}
\begin{itemize}
\item {Grp. gram.:adj.}
\end{itemize}
Relativo ou semelhante ao melástomo.
\section{Melástomo}
\begin{itemize}
\item {Grp. gram.:m.}
\end{itemize}
\begin{itemize}
\item {Proveniência:(Do gr. \textunderscore melas\textunderscore  + \textunderscore slomá\textunderscore )}
\end{itemize}
Gênero de plantas da Ásia tropical.
\section{Melastrofia}
\begin{itemize}
\item {Grp. gram.:f.}
\end{itemize}
\begin{itemize}
\item {Utilização:Med.}
\end{itemize}
\begin{itemize}
\item {Proveniência:(Do gr. \textunderscore melos\textunderscore  + \textunderscore a\textunderscore  + \textunderscore trephein\textunderscore )}
\end{itemize}
Atrofia de um membro.
\section{Melastrophia}
\begin{itemize}
\item {Grp. gram.:f.}
\end{itemize}
\begin{itemize}
\item {Utilização:Med.}
\end{itemize}
\begin{itemize}
\item {Proveniência:(Do gr. \textunderscore melos\textunderscore  + \textunderscore a\textunderscore  + \textunderscore trephein\textunderscore )}
\end{itemize}
Atrophia de um membro.
\section{Melaxanto}
\begin{itemize}
\item {fónica:csan}
\end{itemize}
\begin{itemize}
\item {Grp. gram.:adj.}
\end{itemize}
\begin{itemize}
\item {Proveniência:(Do gr. \textunderscore melas\textunderscore  + \textunderscore xanthos\textunderscore )}
\end{itemize}
Que é amarelo e negro.
\section{Melato}
\begin{itemize}
\item {Grp. gram.:m.}
\end{itemize}
\begin{itemize}
\item {Proveniência:(Do lat. \textunderscore mel\textunderscore , \textunderscore mellis\textunderscore )}
\end{itemize}
Sal, produzido pela combinação do ácido mélico com uma base.
\section{Melca}
\begin{itemize}
\item {Grp. gram.:f.}
\end{itemize}
Pequeno peixe marítimo, o mesmo que \textunderscore melga\textunderscore .
\section{Melcatrefe}
\begin{itemize}
\item {Grp. gram.:m.  e  adj.}
\end{itemize}
\begin{itemize}
\item {Utilização:Chul.}
\end{itemize}
Indivíduo insignificante; vadio; biltre.
\section{Melchior}
\begin{itemize}
\item {Grp. gram.:m.}
\end{itemize}
\begin{itemize}
\item {Utilização:Bras}
\end{itemize}
O mesmo que \textunderscore belchior\textunderscore .
\section{Mel-de-dedo}
\begin{itemize}
\item {Grp. gram.:m.}
\end{itemize}
\begin{itemize}
\item {Utilização:Bras}
\end{itemize}
Variedade de mel, que, sendo saboroso para se comer com mistura de outra coisa, não adoça as substâncias a que se junta. Cf. B. C. Rubim, \textunderscore Voc. Bras.\textunderscore 
\section{Mel-do-tanque}
\begin{itemize}
\item {Grp. gram.:m.}
\end{itemize}
\begin{itemize}
\item {Utilização:Bras}
\end{itemize}
O mesmo que \textunderscore melaço\textunderscore .
\section{Meleagre}
\begin{itemize}
\item {Grp. gram.:m.}
\end{itemize}
Planta bulbosa.
\section{Meleágride}
\begin{itemize}
\item {Grp. gram.:f.}
\end{itemize}
\begin{itemize}
\item {Grp. gram.:Pl.}
\end{itemize}
\begin{itemize}
\item {Proveniência:(Gr. \textunderscore meleagrides\textunderscore )}
\end{itemize}
Nome que os antigos deram á gallinha da Índia.
Família da ordem das gallináceas, que tem por typo a pintada ou gallinha da Índia.
\section{Meleagro}
\begin{itemize}
\item {Grp. gram.:m.}
\end{itemize}
Gênero de molluscos.
Espécie de borboleta.
\section{Meleante}
\begin{itemize}
\item {Grp. gram.:m.}
\end{itemize}
Malandro; patife; libertino; vadio.
(Cp. cast. \textunderscore maleante\textunderscore )
\section{Melecas}
\begin{itemize}
\item {Grp. gram.:m.}
\end{itemize}
\begin{itemize}
\item {Proveniência:(De \textunderscore Melecas\textunderscore , n. p.)}
\end{itemize}
Qualidade de pão macio e fofo, que se fabrica nos arredores de Lisboa.
\section{Meleia}
\begin{itemize}
\item {Grp. gram.:f.}
\end{itemize}
\begin{itemize}
\item {Utilização:Prov.}
\end{itemize}
\begin{itemize}
\item {Utilização:trasm.}
\end{itemize}
Espécie de almofada, que se põe sobre a cabeça dos bois, antes de os jungir, descaíndo-lhes sobre a testa á maneira de franja.
(Por \textunderscore melena\textunderscore ^1)
\section{Meleiro}
\begin{itemize}
\item {Grp. gram.:m.}
\end{itemize}
\begin{itemize}
\item {Grp. gram.:m.}
\end{itemize}
\begin{itemize}
\item {Utilização:Prov.}
\end{itemize}
\begin{itemize}
\item {Utilização:trasm.}
\end{itemize}
Negociante de mel.
Vendedor de mel.
\section{Melena}
\begin{itemize}
\item {Grp. gram.:f.}
\end{itemize}
\begin{itemize}
\item {Utilização:Prov.}
\end{itemize}
\begin{itemize}
\item {Utilização:trasm.}
\end{itemize}
Cabello comprido.
Cabello desgrenhado.
Gadelha.
Parte da crina do cavallo, pendendo da cabeça sobre a fronte.
O mesmo que \textunderscore meleia\textunderscore .
\section{Melena}
\begin{itemize}
\item {Grp. gram.:f.}
\end{itemize}
\begin{itemize}
\item {Proveniência:(Do gr. \textunderscore melaina\textunderscore )}
\end{itemize}
Vómito de matérias negras, acompanhado de dejecções da mesma côr, e precedido de cólica súbita e intensa. Cp. \textunderscore melanemo\textunderscore .
\section{Méleo}
\begin{itemize}
\item {Grp. gram.:adj.}
\end{itemize}
\begin{itemize}
\item {Utilização:Poét.}
\end{itemize}
\begin{itemize}
\item {Proveniência:(Lat. \textunderscore melleus\textunderscore )}
\end{itemize}
Doce, melífluo.
\section{Melez}
\begin{itemize}
\item {Grp. gram.:m.}
\end{itemize}
(V. \textunderscore molhelha\textunderscore ^1)
\section{Melfa}
\begin{itemize}
\item {Grp. gram.:m.}
\end{itemize}
\begin{itemize}
\item {Utilização:T. de Turquel}
\end{itemize}
Mosquito, o mesmo que \textunderscore melga\textunderscore .
\section{Melfo}
\begin{itemize}
\item {Grp. gram.:adj.}
\end{itemize}
\begin{itemize}
\item {Utilização:T. de Turquel}
\end{itemize}
Diz-se de alguns animaes domésticos, que têm os beiços defeituosamente retrahidos, ficando os dentes a descoberto.
(Relaciona-se com \textunderscore belfo\textunderscore ?)
\section{Melfurado}
\begin{itemize}
\item {Grp. gram.:m.}
\end{itemize}
(V.milfurada)O mesmo que \textunderscore luzerna\textunderscore ^1, (\textunderscore medicago silvestris\textunderscore ).
\section{Melga}
\begin{itemize}
\item {Grp. gram.:f.}
\end{itemize}
Espécie de mosquito, que se encontra em terrenos pantanosos.
Pequeno peixe, do feitio da raia.
O mesmo que \textunderscore melga-dos-prados\textunderscore .
\section{Melga-dos-prados}
\begin{itemize}
\item {Grp. gram.:f.}
\end{itemize}
Planta forragínea, o mesmo que \textunderscore alfafa\textunderscore , (\textunderscore medicago sativa\textunderscore , Linn.).
\section{Melgaço}
\begin{itemize}
\item {Grp. gram.:adj.}
\end{itemize}
\begin{itemize}
\item {Utilização:Bras. do N}
\end{itemize}
Loiro; arruívado.
\section{Melgo}
\begin{itemize}
\item {Grp. gram.:m.  e  adj.}
\end{itemize}
\begin{itemize}
\item {Utilização:Prov.}
\end{itemize}
\begin{itemize}
\item {Utilização:trasm.}
\end{itemize}
O mesmo que \textunderscore gêmeo\textunderscore .
(Contr. de \textunderscore gemelgo\textunderscore )
\section{Melgotão}
\begin{itemize}
\item {Grp. gram.:m.}
\end{itemize}
\begin{itemize}
\item {Utilização:Prov.}
\end{itemize}
\begin{itemize}
\item {Utilização:trasm.}
\end{itemize}
Designação vulgar do pêssego.
(Cp. \textunderscore maracotão\textunderscore )
\section{Melgotoeiro}
\begin{itemize}
\item {Grp. gram.:m.}
\end{itemize}
\begin{itemize}
\item {Utilização:Prov.}
\end{itemize}
\begin{itemize}
\item {Utilização:trasm.}
\end{itemize}
\begin{itemize}
\item {Proveniência:(De \textunderscore melgotão\textunderscore )}
\end{itemize}
Variedade de pessegueiro.
\section{Melgueira}
\begin{itemize}
\item {Grp. gram.:f.}
\end{itemize}
\begin{itemize}
\item {Utilização:Fig.}
\end{itemize}
\begin{itemize}
\item {Utilização:Chul.}
\end{itemize}
\begin{itemize}
\item {Utilização:Prov.}
\end{itemize}
\begin{itemize}
\item {Utilização:minh.}
\end{itemize}
\begin{itemize}
\item {Utilização:ant.}
\end{itemize}
\begin{itemize}
\item {Proveniência:(De \textunderscore mel\textunderscore )}
\end{itemize}
Cortiço com favos de mel.
Dinheiro, que se junta ás occultas.
Pechincha.
Gôzo tranquillo.
Patifaria.
Maquinação dolorosa.
\section{Melharuco}
\begin{itemize}
\item {Grp. gram.:m.}
\end{itemize}
(Corr. de \textunderscore abelharuco\textunderscore )
\section{Melhór}
\begin{itemize}
\item {Grp. gram.:adj.}
\end{itemize}
\begin{itemize}
\item {Grp. gram.:M.}
\end{itemize}
\begin{itemize}
\item {Grp. gram.:Adv.}
\end{itemize}
\begin{itemize}
\item {Grp. gram.:Interj.}
\end{itemize}
\begin{itemize}
\item {Proveniência:(Do lat. \textunderscore melior\textunderscore )}
\end{itemize}
Que é mais bom.
Superior a outro em boas qualidades: \textunderscore o filho é melhor que o pai\textunderscore .
Aquillo que tem melhor qualidade que tudo mais.
Aquillo que é acertado ou sensato: \textunderscore o melhor é não curar das vidas alheias\textunderscore .
De maneira superior; superiormente; com mais acêrto: \textunderscore agora, pensas melhor\textunderscore .
Mais, em maior número.
(designativa do satisfação)
\textunderscore Levar a melhor\textunderscore , têr vantagem, sair triumphante:« \textunderscore tinha levado a melhor na luta.\textunderscore »Camillo, \textunderscore Enjeitada\textunderscore , 36.
\section{Melhora}
\begin{itemize}
\item {Grp. gram.:f.}
\end{itemize}
Acto ou effeito de melhorar; melhoria.
\section{Melhoração}
\begin{itemize}
\item {Grp. gram.:f.}
\end{itemize}
\begin{itemize}
\item {Proveniência:(Do b. lat. \textunderscore melioratio\textunderscore )}
\end{itemize}
O mesmo que \textunderscore melhora\textunderscore .
\section{Melhoradamente}
\begin{itemize}
\item {Grp. gram.:adv.}
\end{itemize}
\begin{itemize}
\item {Proveniência:(De \textunderscore melhorado\textunderscore )}
\end{itemize}
Com melhoria.
\section{Melhorado}
\begin{itemize}
\item {Grp. gram.:adj.}
\end{itemize}
\begin{itemize}
\item {Proveniência:(Do lat. \textunderscore melioratus\textunderscore )}
\end{itemize}
Que se tornou melhor; corrigido; aperfeiçoado: \textunderscore nova edição, melhorada\textunderscore .
\section{Melhorador}
\begin{itemize}
\item {Grp. gram.:m.  e  adj.}
\end{itemize}
\begin{itemize}
\item {Proveniência:(De \textunderscore melhorar\textunderscore )}
\end{itemize}
O que torna melhor ou faz melhoramentos.
\section{Melhoramento}
\begin{itemize}
\item {Grp. gram.:m.}
\end{itemize}
\begin{itemize}
\item {Proveniência:(De \textunderscore melhorar\textunderscore )}
\end{itemize}
Melhora; bem-feitoria.
Adiantamento.
\section{Melhorança}
\begin{itemize}
\item {Grp. gram.:f.}
\end{itemize}
\begin{itemize}
\item {Utilização:Des.}
\end{itemize}
O mesmo que \textunderscore melhora\textunderscore . Cf. D. Bernárdez, \textunderscore Lima\textunderscore , 97.
\section{Melhorar}
\begin{itemize}
\item {Grp. gram.:v. t.}
\end{itemize}
\begin{itemize}
\item {Grp. gram.:V. i.}
\end{itemize}
\begin{itemize}
\item {Proveniência:(De \textunderscore melhor\textunderscore )}
\end{itemize}
Tornar melhor.
Tornar superior.
Fazer próspero.
Alliviar.
Restituír a saúde a.
Tornar convalescente.
Aperfeiçoar.
Tornar-se melhor.
Entrar em convalescença.
Adquirir vantagens ou aumentos.
Abrandar-se; suavizar-se: o \textunderscore tempo melhorou\textunderscore .
\section{Melhorativo}
\begin{itemize}
\item {Grp. gram.:adj.}
\end{itemize}
\begin{itemize}
\item {Utilização:Neol.}
\end{itemize}
\begin{itemize}
\item {Proveniência:(De \textunderscore melhorar\textunderscore )}
\end{itemize}
Que encerra melhoria ou conceito favorável, por opposiçâo a \textunderscore pejorativo\textunderscore : \textunderscore accepção melhorativa\textunderscore . Cf. \textunderscore País\textunderscore , do Rio, IV-902. Cp. \textunderscore meliorativo\textunderscore .
\section{Melhoria}
\begin{itemize}
\item {Grp. gram.:f.}
\end{itemize}
\begin{itemize}
\item {Grp. gram.:Adv.}
\end{itemize}
\begin{itemize}
\item {Utilização:Ant.}
\end{itemize}
\begin{itemize}
\item {Proveniência:(De \textunderscore melhor\textunderscore )}
\end{itemize}
Transição para melhor estado.
Melhor estado.
Superioridade.
Bem-feitoria: \textunderscore fazer melhorias num prédio\textunderscore .
Casta de uva do Minho, de que há duas variedades, branca e preta.
Um tanto mais, mais alguma coisa.
\section{Melhorio}
\begin{itemize}
\item {Grp. gram.:m.}
\end{itemize}
\begin{itemize}
\item {Utilização:Pop.}
\end{itemize}
Casta de uva, também conhecida por \textunderscore melhoria\textunderscore .
A melhor parte, o escol: \textunderscore o melhorio da assembleia\textunderscore .
\section{Melhormente}
\begin{itemize}
\item {Grp. gram.:adv.}
\end{itemize}
Em melhores condições; de melhor modo:«\textunderscore para que melhormente se aprecie a obra...\textunderscore »Castilho, \textunderscore Sonho\textunderscore , notas.
\section{Mélia}
\begin{itemize}
\item {Grp. gram.:f.}
\end{itemize}
\begin{itemize}
\item {Proveniência:(Gr. \textunderscore melia\textunderscore )}
\end{itemize}
Gênero do árvores, o mesmo que \textunderscore azederaco\textunderscore .
\section{Meliáceas}
\begin{itemize}
\item {Grp. gram.:f. pl.}
\end{itemize}
\begin{itemize}
\item {Proveniência:(De \textunderscore meliáceo\textunderscore )}
\end{itemize}
Família de plantas, que tem por typo a mélia.
\section{Meliáceo}
\begin{itemize}
\item {Grp. gram.:adj.}
\end{itemize}
Relativo ou semelhante á mélia.
\section{Meliana}
\begin{itemize}
\item {Grp. gram.:f.  e  adj.}
\end{itemize}
Qualidade de terra, usada pelos pintores, para a conservação da tinta nos quadros.
\section{Meliante}
\begin{itemize}
\item {Grp. gram.:m.}
\end{itemize}
Malandro; patife; libertino; vadio.
(Cp. cast. \textunderscore maleante\textunderscore )
\section{Meliantho}
\begin{itemize}
\item {Grp. gram.:m.}
\end{itemize}
\begin{itemize}
\item {Proveniência:(Do gr. \textunderscore meli\textunderscore  + \textunderscore anthos\textunderscore )}
\end{itemize}
Planta, originária da África.
\section{Melianto}
\begin{itemize}
\item {Grp. gram.:m.}
\end{itemize}
\begin{itemize}
\item {Proveniência:(Do gr. \textunderscore meli\textunderscore  + \textunderscore anthos\textunderscore )}
\end{itemize}
Planta, originária da África.
\section{Mélica}
\begin{itemize}
\item {Grp. gram.:f.}
\end{itemize}
\begin{itemize}
\item {Proveniência:(T. it.)}
\end{itemize}
Gênero de plantas gramíneas.
\section{Melicéris}
\begin{itemize}
\item {Grp. gram.:m.}
\end{itemize}
\begin{itemize}
\item {Proveniência:(Gr. \textunderscore melikeris\textunderscore )}
\end{itemize}
Espécie de tumôr cistoso, formado por um líquido amarelado, que tem a consistência do mel.
\section{Melícia}
\begin{itemize}
\item {Grp. gram.:f.}
\end{itemize}
\begin{itemize}
\item {Proveniência:(De \textunderscore mel\textunderscore )}
\end{itemize}
Murcella doce, que contém uma mistura de amêndoas pisadas, com banha de porco, açúcar, canela, etc.
\section{Mélico}
\begin{itemize}
\item {Grp. gram.:adj.}
\end{itemize}
\begin{itemize}
\item {Proveniência:(Lat. \textunderscore melicus\textunderscore )}
\end{itemize}
Musical.
Harmonioso.
Suave; melodioso: \textunderscore voz mélica\textunderscore .
\section{Mélico}
\begin{itemize}
\item {Grp. gram.:adj.}
\end{itemize}
\begin{itemize}
\item {Utilização:Chím.}
\end{itemize}
\begin{itemize}
\item {Proveniência:(Do lat. \textunderscore mel\textunderscore , \textunderscore mellis\textunderscore )}
\end{itemize}
Relativo a mel.
Doce; méleo.
Diz-se de um ácido, que é o hidrato de cálcio.
\section{Melido}
\begin{itemize}
\item {Grp. gram.:m.}
\end{itemize}
O mesmo que \textunderscore molhelha\textunderscore ^1.
\section{Melieiro}
\begin{itemize}
\item {Grp. gram.:adj.}
\end{itemize}
\begin{itemize}
\item {Proveniência:(De \textunderscore mel\textunderscore )}
\end{itemize}
Carinhoso, meigo.
Lisonjeiro por interesse; que tem lábia.
\section{Melifagídeas}
\begin{itemize}
\item {Grp. gram.:f. pl.}
\end{itemize}
Família de aves, que tem por tipo o melifago.
\section{Melifago}
\begin{itemize}
\item {Grp. gram.:m.}
\end{itemize}
\begin{itemize}
\item {Proveniência:(Do gr. \textunderscore meli\textunderscore  + \textunderscore phagein\textunderscore )}
\end{itemize}
Gênero de aves anisodáctilas.
\section{Melífero}
\begin{itemize}
\item {Grp. gram.:adj.}
\end{itemize}
\begin{itemize}
\item {Proveniência:(Do lat. \textunderscore mel\textunderscore , \textunderscore mellis\textunderscore  + \textunderscore ferre\textunderscore )}
\end{itemize}
Que produz mel.
\section{Melificação}
\begin{itemize}
\item {Grp. gram.:f.}
\end{itemize}
Acto ou efeito de melificar.
\section{Melificador}
\begin{itemize}
\item {Grp. gram.:m.}
\end{itemize}
\begin{itemize}
\item {Proveniência:(De \textunderscore melificar\textunderscore )}
\end{itemize}
Vaso, em que se aquecem os favos, para que êstes larguem o mel.
\section{Melificar}
\begin{itemize}
\item {Grp. gram.:v. t.}
\end{itemize}
\begin{itemize}
\item {Grp. gram.:V. i.}
\end{itemize}
\begin{itemize}
\item {Proveniência:(Lat. \textunderscore mellificare\textunderscore )}
\end{itemize}
Converter em mel; adoçar.
Fabricar mel.
\section{Melífico}
\begin{itemize}
\item {Grp. gram.:adj.}
\end{itemize}
\begin{itemize}
\item {Utilização:Fig.}
\end{itemize}
\begin{itemize}
\item {Proveniência:(Lat. \textunderscore mellificus\textunderscore )}
\end{itemize}
Melífero.
Relativo a mel.
Que tem a natureza do mel.
Doce.
\section{Melifluentar}
\begin{itemize}
\item {Grp. gram.:v. i.}
\end{itemize}
\begin{itemize}
\item {Proveniência:(De \textunderscore melífluo\textunderscore )}
\end{itemize}
Tornar melifluo, doce, suave. Cf. Castilho, \textunderscore Sabichonas\textunderscore , 228.
\section{Melifluidade}
\begin{itemize}
\item {fónica:flu-i}
\end{itemize}
\begin{itemize}
\item {Grp. gram.:f.}
\end{itemize}
Qualidade do que é melífluo.
Suavidade; doçura.
\section{Melífluo}
\begin{itemize}
\item {Grp. gram.:adj.}
\end{itemize}
\begin{itemize}
\item {Utilização:Fig.}
\end{itemize}
\begin{itemize}
\item {Proveniência:(Lat. \textunderscore mellifluus\textunderscore )}
\end{itemize}
Que corre como o mel.
Suave.
Harmonioso.
Que tem voz branda ou doce.
\section{Meliloto}
\begin{itemize}
\item {Grp. gram.:m.}
\end{itemize}
\begin{itemize}
\item {Proveniência:(Gr. \textunderscore melilotos\textunderscore )}
\end{itemize}
Trevo de cheiro.
\section{Melimba}
\begin{itemize}
\item {Grp. gram.:f.}
\end{itemize}
Árvore de Cabinda, própria para tabuado.
\section{Melindano}
\begin{itemize}
\item {Grp. gram.:adj.}
\end{itemize}
Relativo a Melinde. Cf. \textunderscore Lusiadas\textunderscore , VI, 92.
\section{Melindrar}
\begin{itemize}
\item {Grp. gram.:v. t.}
\end{itemize}
\begin{itemize}
\item {Proveniência:(De \textunderscore melindre\textunderscore )}
\end{itemize}
Tornar melindroso.
Offender o melindre de.
Maguar; escandalizar.
\section{Melindre}
\begin{itemize}
\item {Grp. gram.:m.}
\end{itemize}
\begin{itemize}
\item {Utilização:Fig.}
\end{itemize}
\begin{itemize}
\item {Proveniência:(De \textunderscore mel\textunderscore ?)}
\end{itemize}
Bolo, em que entra o mel.
Planta delicada, (\textunderscore balsamina vulgaris\textunderscore ).
Delicadeza no trato.
Recato.
Susceptibilidade; facilidade, com que alguém amua ou se julga offendido.
\section{Melindrice}
\begin{itemize}
\item {Grp. gram.:f.}
\end{itemize}
\begin{itemize}
\item {Utilização:Bras}
\end{itemize}
O mesmo que \textunderscore melindrismo\textunderscore .
\section{Melindrismo}
\begin{itemize}
\item {Grp. gram.:m.}
\end{itemize}
\begin{itemize}
\item {Utilização:Bras}
\end{itemize}
\begin{itemize}
\item {Utilização:Neol.}
\end{itemize}
\begin{itemize}
\item {Proveniência:(De \textunderscore melindrar\textunderscore )}
\end{itemize}
Qualidade de quem se melindra facilmente.
\section{Melindrosamente}
\begin{itemize}
\item {Grp. gram.:adv.}
\end{itemize}
De modo melindroso.
Com delicadeza, com escrúpulo: \textunderscore tratar melindrosamente um assumpto\textunderscore .
\section{Melindroso}
\begin{itemize}
\item {Grp. gram.:adj.}
\end{itemize}
\begin{itemize}
\item {Proveniência:(De \textunderscore melindre\textunderscore )}
\end{itemize}
Que tem melindre.
Escrupuloso.
Muito delicado.
Susceptível; facilmente impressionável.
Débil; innocente.
Arriscado: \textunderscore empresa melindrosa\textunderscore .
\section{Melingrar}
\begin{itemize}
\item {Grp. gram.:v. i.}
\end{itemize}
\begin{itemize}
\item {Utilização:Prov.}
\end{itemize}
\begin{itemize}
\item {Utilização:minh.}
\end{itemize}
Entreter o tempo, fingindo que se trabalha ou que se come, e gastá-lo afinal em trejeitos ou momices.
\section{Melinite}
\begin{itemize}
\item {Grp. gram.:f.}
\end{itemize}
Substância explosiva, recentemente descoberta em França, e de acção mais enérgica que a da dynamite.
Variedade de opala.
\section{Meliorativo}
\begin{itemize}
\item {Grp. gram.:adj.}
\end{itemize}
\begin{itemize}
\item {Proveniência:(Do lat. \textunderscore melior\textunderscore )}
\end{itemize}
Que envolve ou designa melhoria, por opposição a \textunderscore pejorativo\textunderscore : \textunderscore sentido meliorativo\textunderscore .
\section{Meliphagídeas}
\begin{itemize}
\item {Grp. gram.:f. pl.}
\end{itemize}
Família de aves, que tem por typo o meliphago.
\section{Meliphago}
\begin{itemize}
\item {Grp. gram.:m.}
\end{itemize}
\begin{itemize}
\item {Proveniência:(Do gr. \textunderscore meli\textunderscore  + \textunderscore phagein\textunderscore )}
\end{itemize}
Gênero de aves anisodáctylas.
\section{Melipona}
\begin{itemize}
\item {Grp. gram.:f.}
\end{itemize}
\begin{itemize}
\item {Proveniência:(Do gr. \textunderscore meli\textunderscore , mel, e \textunderscore ponos\textunderscore , trabalho)}
\end{itemize}
Gênero de insectos hymnópteros, da família das abelhas.
\section{Melismático}
\begin{itemize}
\item {Grp. gram.:adj.}
\end{itemize}
\begin{itemize}
\item {Utilização:Mús.}
\end{itemize}
\begin{itemize}
\item {Proveniência:(Do gr. \textunderscore melisma\textunderscore )}
\end{itemize}
Diz-se do canto, em que uma só syllaba é entoada com diversas notas.
\section{Melissa}
\begin{itemize}
\item {Grp. gram.:f.}
\end{itemize}
\begin{itemize}
\item {Proveniência:(Do gr. \textunderscore melissa\textunderscore )}
\end{itemize}
O mesmo que \textunderscore erva-cidreira\textunderscore .
Designação scientifica da abelha.
\section{Melissografia}
\begin{itemize}
\item {Grp. gram.:f.}
\end{itemize}
\begin{itemize}
\item {Proveniência:(Do gr. \textunderscore melissa\textunderscore  + \textunderscore graphein\textunderscore )}
\end{itemize}
Tratado á cerca das abelhas.
Descripção dos costumes das abelhas.
\section{Melissográfico}
\begin{itemize}
\item {Grp. gram.:adj.}
\end{itemize}
Relativo á melissografia.
\section{Melissographia}
\begin{itemize}
\item {Grp. gram.:f.}
\end{itemize}
\begin{itemize}
\item {Proveniência:(Do gr. \textunderscore melissa\textunderscore  + \textunderscore graphein\textunderscore )}
\end{itemize}
Tratado á cerca das abelhas.
Descripção dos costumes das abelhas.
\section{Melissográphico}
\begin{itemize}
\item {Grp. gram.:adj.}
\end{itemize}
Relativo á melissographia.
\section{Melissugo}
\begin{itemize}
\item {Grp. gram.:adj.}
\end{itemize}
\begin{itemize}
\item {Proveniência:(Do lat. \textunderscore mel\textunderscore , \textunderscore mellis\textunderscore  + \textunderscore sugere\textunderscore )}
\end{itemize}
Que suga o suco das flôres.
\section{Melita}
\begin{itemize}
\item {Grp. gram.:f.}
\end{itemize}
Gênero de plantas labiadas.
Gênero de crustáceos.
\section{Melitato}
\begin{itemize}
\item {Grp. gram.:m.}
\end{itemize}
O mesmo que \textunderscore melato\textunderscore .
\section{Melite}
\begin{itemize}
\item {Grp. gram.:adj.}
\end{itemize}
\begin{itemize}
\item {Proveniência:(Do lat. \textunderscore mel\textunderscore , \textunderscore mellis\textunderscore )}
\end{itemize}
Espécie de mineral carbonado, pedra amarelada.
\section{Melithrepto}
\begin{itemize}
\item {Grp. gram.:m.}
\end{itemize}
\begin{itemize}
\item {Proveniência:(Do gr. \textunderscore meli\textunderscore  + \textunderscore threptos\textunderscore )}
\end{itemize}
Gênero de aves anisodáctylas.
\section{Melítico}
\begin{itemize}
\item {Grp. gram.:adj.}
\end{itemize}
Diz-se de um ácido, que se extrai da melite.
\section{Melito}
\begin{itemize}
\item {Grp. gram.:m.}
\end{itemize}
\begin{itemize}
\item {Proveniência:(Do gr. \textunderscore meli\textunderscore , \textunderscore melitos\textunderscore )}
\end{itemize}
Designação genérica dos medicamentos, em que entra mel.
\section{Melitose}
\begin{itemize}
\item {Grp. gram.:f.}
\end{itemize}
\begin{itemize}
\item {Proveniência:(Do gr. \textunderscore meli\textunderscore , \textunderscore melitos\textunderscore )}
\end{itemize}
Exsudação açucarada de algumas espécies de eucalyptos da Austrália.
\section{Melitrepto}
\begin{itemize}
\item {Grp. gram.:m.}
\end{itemize}
\begin{itemize}
\item {Proveniência:(Do gr. \textunderscore meli\textunderscore  + \textunderscore threptos\textunderscore )}
\end{itemize}
Gênero de aves anisodáctilas.
\section{Meliturgia}
\begin{itemize}
\item {Grp. gram.:f.}
\end{itemize}
\begin{itemize}
\item {Proveniência:(Gr. \textunderscore meltourgía\textunderscore )}
\end{itemize}
A indústria das abelhas.
\section{Melituria}
\begin{itemize}
\item {Grp. gram.:f.}
\end{itemize}
\begin{itemize}
\item {Proveniência:(Do gr. \textunderscore meli\textunderscore , \textunderscore melitos\textunderscore  + \textunderscore ouron\textunderscore )}
\end{itemize}
Estado mórbido de quem expelle urina açucarada.
Diabete açucarada.
\section{Melívoro}
\begin{itemize}
\item {Grp. gram.:adj.}
\end{itemize}
\begin{itemize}
\item {Proveniência:(Do lat. \textunderscore mel\textunderscore , \textunderscore mellis\textunderscore  + \textunderscore vorare\textunderscore )}
\end{itemize}
Que se alimenta de \textunderscore mel\textunderscore .
\section{Mellato}
\begin{itemize}
\item {Grp. gram.:m.}
\end{itemize}
\begin{itemize}
\item {Proveniência:(Do lat. \textunderscore mel\textunderscore , \textunderscore mellis\textunderscore )}
\end{itemize}
Sal, produzido pela combinação do ácido méllico com uma base.
\section{Melleiro}
\begin{itemize}
\item {Grp. gram.:m.}
\end{itemize}
\begin{itemize}
\item {Utilização:Prov.}
\end{itemize}
\begin{itemize}
\item {Utilização:trasm.}
\end{itemize}
Vendedor de mel.
\section{Mélleo}
\begin{itemize}
\item {Grp. gram.:adj.}
\end{itemize}
\begin{itemize}
\item {Utilização:Poét.}
\end{itemize}
\begin{itemize}
\item {Proveniência:(Lat. \textunderscore melleus\textunderscore )}
\end{itemize}
Doce, mellífluo.
\section{Méllico}
\begin{itemize}
\item {Grp. gram.:adj.}
\end{itemize}
\begin{itemize}
\item {Utilização:Chím.}
\end{itemize}
\begin{itemize}
\item {Proveniência:(Do lat. \textunderscore mel\textunderscore , \textunderscore mellis\textunderscore )}
\end{itemize}
Relativo a mel.
Doce; mélleo.
Diz-se de um ácido, que é o hydrato de cálcio.
\section{Mellífero}
\begin{itemize}
\item {Grp. gram.:adj.}
\end{itemize}
\begin{itemize}
\item {Proveniência:(Do lat. \textunderscore mel\textunderscore , \textunderscore mellis\textunderscore  + \textunderscore ferre\textunderscore )}
\end{itemize}
Que produz mel.
\section{Mellificação}
\begin{itemize}
\item {Grp. gram.:f.}
\end{itemize}
Acto ou effeito de mellificar.
\section{Mellificador}
\begin{itemize}
\item {Grp. gram.:m.}
\end{itemize}
\begin{itemize}
\item {Proveniência:(De \textunderscore mellificar\textunderscore )}
\end{itemize}
Vaso, em que se aquecem os favos, para que êstes larguem o mel.
\section{Mellificar}
\begin{itemize}
\item {Grp. gram.:v. t.}
\end{itemize}
\begin{itemize}
\item {Grp. gram.:V. i.}
\end{itemize}
\begin{itemize}
\item {Proveniência:(Lat. \textunderscore mellificare\textunderscore )}
\end{itemize}
Converter em mel; adoçar.
Fabricar mel.
\section{Mellífico}
\begin{itemize}
\item {Grp. gram.:adj.}
\end{itemize}
\begin{itemize}
\item {Utilização:Fig.}
\end{itemize}
\begin{itemize}
\item {Proveniência:(Lat. \textunderscore mellificus\textunderscore )}
\end{itemize}
Mellífero.
Relativo a mel.
Que tem a natureza do mel.
Doce.
\section{Mellifluentar}
\begin{itemize}
\item {Grp. gram.:v. i.}
\end{itemize}
\begin{itemize}
\item {Proveniência:(De \textunderscore mellífluo\textunderscore )}
\end{itemize}
Tornar mellifluo, doce, suave. Cf. Castilho, \textunderscore Sabichonas\textunderscore , 228.
\section{Mellifluidade}
\begin{itemize}
\item {Grp. gram.:f.}
\end{itemize}
Qualidade do que é mellífluo.
Suavidade; doçura.
\section{Mellífluo}
\begin{itemize}
\item {Grp. gram.:adj.}
\end{itemize}
\begin{itemize}
\item {Utilização:Fig.}
\end{itemize}
\begin{itemize}
\item {Proveniência:(Lat. \textunderscore mellifluus\textunderscore )}
\end{itemize}
Que corre como o mel.
Suave.
Harmonioso.
Que tem voz branda ou doce.
\section{Mellisugo}
\begin{itemize}
\item {fónica:su}
\end{itemize}
\begin{itemize}
\item {Grp. gram.:adj.}
\end{itemize}
\begin{itemize}
\item {Proveniência:(Do lat. \textunderscore mel\textunderscore , \textunderscore mellis\textunderscore  + \textunderscore sugere\textunderscore )}
\end{itemize}
Que suga o suco das flôres.
\section{Mellita}
\begin{itemize}
\item {Grp. gram.:adj.}
\end{itemize}
\begin{itemize}
\item {Proveniência:(Do lat. \textunderscore mel\textunderscore , \textunderscore mellis\textunderscore )}
\end{itemize}
Espécie de mineral carbonado, pedra amarelada.
\section{Mellithato}
\begin{itemize}
\item {Grp. gram.:m.}
\end{itemize}
O mesmo que \textunderscore mellato\textunderscore .
\section{Mellítico}
\begin{itemize}
\item {Grp. gram.:adj.}
\end{itemize}
Diz-se de um ácido, que se extrai da mellita.
\section{Mellívoro}
\begin{itemize}
\item {Grp. gram.:adj.}
\end{itemize}
\begin{itemize}
\item {Proveniência:(Do lat. \textunderscore mel\textunderscore , \textunderscore mellis\textunderscore  + \textunderscore vorare\textunderscore )}
\end{itemize}
Que se alimenta de \textunderscore mel\textunderscore .
\section{Mellóchia}
\begin{itemize}
\item {fónica:qui}
\end{itemize}
\begin{itemize}
\item {Grp. gram.:f.}
\end{itemize}
Gênero de plantas buthneriáceas da América.
\section{Melloso}
\begin{itemize}
\item {Grp. gram.:adj.}
\end{itemize}
\begin{itemize}
\item {Proveniência:(Lat. \textunderscore mellosus\textunderscore )}
\end{itemize}
Doce; semelhante ao mel.
\section{Mellúria}
\begin{itemize}
\item {Grp. gram.:f.}
\end{itemize}
\begin{itemize}
\item {Utilização:Pop.}
\end{itemize}
\begin{itemize}
\item {Utilização:Prov.}
\end{itemize}
\begin{itemize}
\item {Utilização:minh.}
\end{itemize}
\begin{itemize}
\item {Proveniência:(Do lat. \textunderscore mel\textunderscore , \textunderscore mellis\textunderscore )}
\end{itemize}
Qualidade daquelle ou daquillo que é mellífluo, suave.
Suavidade, doçura.
Primícias do mel, que se dão ao párocho. Cf. Camillo, \textunderscore Maria da Fonte\textunderscore , 33.
\section{Melmosa}
\begin{itemize}
\item {Grp. gram.:f.}
\end{itemize}
\begin{itemize}
\item {Utilização:Prov.}
\end{itemize}
\begin{itemize}
\item {Utilização:trasm.}
\end{itemize}
Rapariga de aspecto lacrimoso ou angustiado.
Rapariga que, por effeito de constipação, traz o nariz húmido e os olhos meio cerrados.
\section{Melo}
\begin{itemize}
\item {Grp. gram.:m.}
\end{itemize}
Peixe da Póvoa de Varzim, (\textunderscore berix decadactylus\textunderscore , Cuv.).
\section{Melôa}
\begin{itemize}
\item {Grp. gram.:f.}
\end{itemize}
\begin{itemize}
\item {Grp. gram.:Adj.}
\end{itemize}
\begin{itemize}
\item {Proveniência:(De \textunderscore melão\textunderscore )}
\end{itemize}
Grande melão.
Pequeno melão arredondado.
Diz-se de uma abóbora grande, que tem o feitio do melão.
\section{Meloal}
\begin{itemize}
\item {Grp. gram.:m.}
\end{itemize}
\begin{itemize}
\item {Proveniência:(De \textunderscore melão\textunderscore )}
\end{itemize}
Terreno, em que crescem meloeiros; melancial.
\section{Melocacto}
\begin{itemize}
\item {Grp. gram.:m.}
\end{itemize}
\begin{itemize}
\item {Proveniência:(Do lat. \textunderscore melo\textunderscore  + \textunderscore cactus\textunderscore )}
\end{itemize}
Gênero de cactos.
\section{Melocotão}
\begin{itemize}
\item {Grp. gram.:m.}
\end{itemize}
O mesmo que \textunderscore maracotão\textunderscore .
(Cast. \textunderscore melocoton\textunderscore )
\section{Melodia}
\begin{itemize}
\item {Grp. gram.:f.}
\end{itemize}
\begin{itemize}
\item {Utilização:Ext.}
\end{itemize}
\begin{itemize}
\item {Proveniência:(Gr. \textunderscore melodia\textunderscore )}
\end{itemize}
Série de sons, de que resulta um canto regular e agradável.
Conjunto de sons successivos, que formam uma ou mais phrases musicaes.
Qualidade de um canto agradável.
Peça musical, suave, para uma só voz ou para um coro unísono.
Ária.
Suavidade no cantar, no falar ou no escrever.
Aquillo que é agradável ao ouvido.
\section{Melodiar}
\begin{itemize}
\item {Grp. gram.:v. t.}
\end{itemize}
\begin{itemize}
\item {Proveniência:(De \textunderscore melodia\textunderscore )}
\end{itemize}
Tornar melodioso; cantar suavemente.
\section{Melódica}
\begin{itemize}
\item {Grp. gram.:f.}
\end{itemize}
\begin{itemize}
\item {Proveniência:(De \textunderscore melódico\textunderscore )}
\end{itemize}
Instrumento musical, cujos sons são produzidos pelo attrito de umas pontas de metal sôbre um cylindro de aço.
Theoria da melodia.
\section{Melódico}
\begin{itemize}
\item {Grp. gram.:adj.}
\end{itemize}
\begin{itemize}
\item {Proveniência:(Lat. \textunderscore melodicus\textunderscore )}
\end{itemize}
Relativo á melodia; melodioso.
\section{Melodino}
\begin{itemize}
\item {Grp. gram.:m.}
\end{itemize}
\begin{itemize}
\item {Proveniência:(Do gr. \textunderscore melon\textunderscore  + \textunderscore dinos\textunderscore )}
\end{itemize}
Gênero de plantas apocýneas.
\section{Melodiosamente}
\begin{itemize}
\item {Grp. gram.:adv.}
\end{itemize}
De modo melodioso.
\section{Melodioso}
\begin{itemize}
\item {Grp. gram.:adj.}
\end{itemize}
Em que há melodia; agradável, suave.
\section{Melodista}
\begin{itemize}
\item {Grp. gram.:m.}
\end{itemize}
Aquelle que faz melodias; compositor de melodias.
\section{Melodizar}
\begin{itemize}
\item {Grp. gram.:v. t.}
\end{itemize}
\begin{itemize}
\item {Utilização:Neol.}
\end{itemize}
\begin{itemize}
\item {Proveniência:(De \textunderscore melódico\textunderscore )}
\end{itemize}
Tornar melodioso; suavizar o som de; melodiar.
\section{Melodrama}
\begin{itemize}
\item {Grp. gram.:m.}
\end{itemize}
\begin{itemize}
\item {Utilização:Ant.}
\end{itemize}
\begin{itemize}
\item {Proveniência:(Do gr. \textunderscore melos\textunderscore  + \textunderscore drama\textunderscore )}
\end{itemize}
Peça dramática, de situações violentas e sentimentos exaggerados.
Espécie de drama, em que o diálogo era interrompido por música instrumental.
\section{Melodramático}
\begin{itemize}
\item {Grp. gram.:adj.}
\end{itemize}
Relativo ao melodrama.
Que tem propriedades de melodrama.
\section{Meloeiro}
\begin{itemize}
\item {Grp. gram.:m.}
\end{itemize}
\begin{itemize}
\item {Proveniência:(De \textunderscore melão\textunderscore )}
\end{itemize}
Planta cucurbitácea e hortense.
\section{Méloes}
\begin{itemize}
\item {Grp. gram.:m. pl.}
\end{itemize}
Gênero de insectos coleópteros.
\section{Melófilo}
\begin{itemize}
\item {Grp. gram.:adj.}
\end{itemize}
\begin{itemize}
\item {Proveniência:(Do gr. \textunderscore meios\textunderscore  + \textunderscore philos\textunderscore )}
\end{itemize}
Que gosta de música.
\section{Melofone}
\begin{itemize}
\item {Grp. gram.:m.}
\end{itemize}
O mesmo que \textunderscore melofono\textunderscore .
\section{Melofónio}
\begin{itemize}
\item {Grp. gram.:m.}
\end{itemize}
\begin{itemize}
\item {Proveniência:(Do gr. \textunderscore melos\textunderscore  + \textunderscore phone\textunderscore )}
\end{itemize}
Instrumento de sopro, um pouco semelhante no feitio á guitarra.
\section{Melofono}
\begin{itemize}
\item {Grp. gram.:m.}
\end{itemize}
\begin{itemize}
\item {Proveniência:(Do gr. \textunderscore melos\textunderscore  + \textunderscore phone\textunderscore )}
\end{itemize}
Instrumento de sopro, um pouco semelhante no feitio á guitarra.
\section{Melóforo}
\begin{itemize}
\item {Grp. gram.:m.}
\end{itemize}
\begin{itemize}
\item {Proveniência:(Do gr. \textunderscore melos\textunderscore  + \textunderscore phoros\textunderscore )}
\end{itemize}
Espécie de lampião, que serve de estante, em cujas faces se aplicam papéis transparentes, com música escrita, para que esta se execute de noite, ao ar livre.
\section{Melografia}
\begin{itemize}
\item {Grp. gram.:f.}
\end{itemize}
\begin{itemize}
\item {Utilização:Des.}
\end{itemize}
\begin{itemize}
\item {Proveniência:(De \textunderscore melógrafo\textunderscore )}
\end{itemize}
Arte da melodia.
\section{Melograficamente}
\begin{itemize}
\item {Grp. gram.:adv.}
\end{itemize}
De modo melográfico.
Segundo os preceitos da melografia.
\section{Melográfico}
\begin{itemize}
\item {Grp. gram.:adj.}
\end{itemize}
Relativo á melografia.
\section{Melógrafo}
\begin{itemize}
\item {Grp. gram.:m.}
\end{itemize}
\begin{itemize}
\item {Proveniência:(Do gr. \textunderscore melos\textunderscore  + \textunderscore graphein\textunderscore )}
\end{itemize}
Aquele que exerce a melografia.
Aparelho, que reproduz, escritos ou gravados, os sons de um piano.
\section{Melographia}
\begin{itemize}
\item {Grp. gram.:f.}
\end{itemize}
\begin{itemize}
\item {Utilização:Des.}
\end{itemize}
\begin{itemize}
\item {Proveniência:(De \textunderscore melógrapho\textunderscore )}
\end{itemize}
Arte da melodia.
\section{Melographicamente}
\begin{itemize}
\item {Grp. gram.:adv.}
\end{itemize}
De modo melográphico.
Segundo os preceitos da melographia.
\section{Melográphico}
\begin{itemize}
\item {Grp. gram.:adj.}
\end{itemize}
Relativo á melographia.
\section{Melógrapho}
\begin{itemize}
\item {Grp. gram.:m.}
\end{itemize}
\begin{itemize}
\item {Proveniência:(Do gr. \textunderscore melos\textunderscore  + \textunderscore graphein\textunderscore )}
\end{itemize}
Aquelle que exerce a melographia.
Apparelho, que reproduz, escritos ou gravados, os sons de um piano.
\section{Melolonta}
\begin{itemize}
\item {Grp. gram.:f.}
\end{itemize}
Espécie de insecto, (\textunderscore melolonta vulgaris\textunderscore , Fabr., ou \textunderscore escarabeus melolonta\textunderscore , Lin.), damninho ás plantas, especialmente aos viveiros de videiras e ao tabaco.
É também conhecido por \textunderscore besoiro\textunderscore .
\section{Melomania}
\begin{itemize}
\item {Grp. gram.:f.}
\end{itemize}
\begin{itemize}
\item {Proveniência:(Do gr. \textunderscore melos\textunderscore  + \textunderscore mania\textunderscore )}
\end{itemize}
Paixão pela música.
\section{Melomaníaco}
\begin{itemize}
\item {Grp. gram.:adj.}
\end{itemize}
\begin{itemize}
\item {Proveniência:(De \textunderscore melomania\textunderscore )}
\end{itemize}
Que tem paixão pela música.
\section{Melómano}
\begin{itemize}
\item {Grp. gram.:adj.}
\end{itemize}
(V.melomaníaco)
\section{Melombe}
\begin{itemize}
\item {Grp. gram.:m.}
\end{itemize}
Pássaro dentírostro da África.
\section{Melombeanganza}
\begin{itemize}
\item {Grp. gram.:m.}
\end{itemize}
Pássaro dentírostro da África occidental.
\section{Melomelia}
\begin{itemize}
\item {Grp. gram.:f.}
\end{itemize}
Qualidade de melómelo.
\section{Melómelo}
\begin{itemize}
\item {Grp. gram.:m.}
\end{itemize}
\begin{itemize}
\item {Proveniência:(Do gr. \textunderscore melos\textunderscore )}
\end{itemize}
Monstro, que tem membros supplementares inseridos nos membros normaes.
\section{Melonídeo}
\begin{itemize}
\item {Grp. gram.:adj.}
\end{itemize}
\begin{itemize}
\item {Utilização:Bot.}
\end{itemize}
\begin{itemize}
\item {Proveniência:(Do gr. \textunderscore melon\textunderscore  + \textunderscore eidos\textunderscore )}
\end{itemize}
Diz-se do fruto, proveniente de muitos ovários, ligados com o cálice.
\section{Meloniforme}
\begin{itemize}
\item {Grp. gram.:adj.}
\end{itemize}
\begin{itemize}
\item {Utilização:Bot.}
\end{itemize}
\begin{itemize}
\item {Proveniência:(Do lat. \textunderscore melo\textunderscore  + \textunderscore forma\textunderscore )}
\end{itemize}
Que tem fórma semelhante á do melão.
\section{Melonita}
\begin{itemize}
\item {Grp. gram.:f.}
\end{itemize}
\begin{itemize}
\item {Utilização:Miner.}
\end{itemize}
\begin{itemize}
\item {Proveniência:(Do lat. \textunderscore melo\textunderscore )}
\end{itemize}
Pedra globulosa, com a fórma de melão.
\section{Melope}
\begin{itemize}
\item {Grp. gram.:m.}
\end{itemize}
Peixe variegado, do gênero dos labros, (\textunderscore labrus melops\textunderscore ).
\section{Melopeia}
\begin{itemize}
\item {Grp. gram.:f.}
\end{itemize}
\begin{itemize}
\item {Utilização:Ext.}
\end{itemize}
\begin{itemize}
\item {Proveniência:(Gr. \textunderscore melopoiia\textunderscore )}
\end{itemize}
Peça musical, que serve de acompanhamento a uma recitação.
Arte de fazer acompanhamentos musicaes.
Suavidade.
Declamação agradável ao ouvido.
\section{Melóphilo}
\begin{itemize}
\item {Grp. gram.:adj.}
\end{itemize}
\begin{itemize}
\item {Proveniência:(Do gr. \textunderscore meios\textunderscore  + \textunderscore philos\textunderscore )}
\end{itemize}
Que gosta de música.
\section{Melophone}
\begin{itemize}
\item {Grp. gram.:m.}
\end{itemize}
O mesmo que \textunderscore melophono\textunderscore .
\section{Melophono}
\begin{itemize}
\item {Grp. gram.:m.}
\end{itemize}
\begin{itemize}
\item {Proveniência:(Do gr. \textunderscore melos\textunderscore  + \textunderscore phone\textunderscore )}
\end{itemize}
Instrumento de sopro, um pouco semelhante no feitio á guitarra.
\section{Melóphoro}
\begin{itemize}
\item {Grp. gram.:m.}
\end{itemize}
\begin{itemize}
\item {Proveniência:(Do gr. \textunderscore melos\textunderscore  + \textunderscore phoros\textunderscore )}
\end{itemize}
Espécie de lampião, que serve de estante, em cujas faces se applicam papéis transparentes, com música escrita, para que esta se execute de noite, ao ar livre.
\section{Meloplastia}
\begin{itemize}
\item {Grp. gram.:f.}
\end{itemize}
\begin{itemize}
\item {Proveniência:(De gr. \textunderscore melon\textunderscore  + \textunderscore plassein\textunderscore )}
\end{itemize}
Operação cirúrgica, com que se restaura a face, corroída por uma chaga ou por uma ulceração.
\section{Meloplasto}
\begin{itemize}
\item {Grp. gram.:m.}
\end{itemize}
Fragmento de pauta musical, com uma linha supplementar superior e outra inferior, adoptado era várias escolas de música.
\section{Melóquia}
\begin{itemize}
\item {Grp. gram.:f.}
\end{itemize}
Gênero de plantas butneriáceas da América.
\section{Melose}
\begin{itemize}
\item {Grp. gram.:f.}
\end{itemize}
\begin{itemize}
\item {Utilização:Med.}
\end{itemize}
\begin{itemize}
\item {Proveniência:(Do gr. \textunderscore mele\textunderscore )}
\end{itemize}
Acto de explorar com a sonda.
\section{Meloso}
\begin{itemize}
\item {Grp. gram.:adj.}
\end{itemize}
\begin{itemize}
\item {Proveniência:(Lat. \textunderscore mellosus\textunderscore )}
\end{itemize}
Doce; semelhante ao mel.
\section{Melote}
\begin{itemize}
\item {Grp. gram.:m.}
\end{itemize}
\begin{itemize}
\item {Proveniência:(Do gr. \textunderscore melon\textunderscore )}
\end{itemize}
Pelle de carneiro com a lan.
\section{Meloterapeuta}
\begin{itemize}
\item {Grp. gram.:m.}
\end{itemize}
Aquele que exerce a meloterapêutica.
\section{Meloterapêutica}
\begin{itemize}
\item {Grp. gram.:f.}
\end{itemize}
O mesmo que \textunderscore meloterapia\textunderscore .
\section{Meloterapia}
\begin{itemize}
\item {Grp. gram.:f.}
\end{itemize}
\begin{itemize}
\item {Proveniência:(Do gr. \textunderscore melos\textunderscore  + \textunderscore therapeia\textunderscore )}
\end{itemize}
Tratamento médico, por meio da música.
\section{Meloterápico}
\begin{itemize}
\item {Grp. gram.:adj.}
\end{itemize}
Relativo a meloterapia.
\section{Melotherapeuta}
\begin{itemize}
\item {Grp. gram.:m.}
\end{itemize}
Aquelle que exerce a melotherapêutica.
\section{Melotherapêutica}
\begin{itemize}
\item {Grp. gram.:f.}
\end{itemize}
O mesmo que \textunderscore melotherapia\textunderscore .
\section{Melotherapia}
\begin{itemize}
\item {Grp. gram.:f.}
\end{itemize}
\begin{itemize}
\item {Proveniência:(Do gr. \textunderscore melos\textunderscore  + \textunderscore therapeia\textunderscore )}
\end{itemize}
Tratamento médico, por meio da música.
\section{Melotherápico}
\begin{itemize}
\item {Grp. gram.:adj.}
\end{itemize}
Relativo a melotherapia.
\section{Melpómene}
\begin{itemize}
\item {Grp. gram.:f.}
\end{itemize}
\begin{itemize}
\item {Proveniência:(De \textunderscore Melpómene\textunderscore , n. p.)}
\end{itemize}
Planeta telescópico, descoberto em 1852.
\section{Melquetrefe}
\begin{itemize}
\item {Grp. gram.:m.}
\end{itemize}
O mesmo que \textunderscore melcatrefe\textunderscore . Cf. Filinto, IV, 213.
\section{Melrinho-da-giesta}
\begin{itemize}
\item {Grp. gram.:m.}
\end{itemize}
\begin{itemize}
\item {Utilização:Mad}
\end{itemize}
O mesmo que \textunderscore tinge-burro\textunderscore .
\section{Melrinho-da-serra}
\begin{itemize}
\item {Grp. gram.:m.}
\end{itemize}
\begin{itemize}
\item {Utilização:Mad}
\end{itemize}
O mesmo que \textunderscore abibe\textunderscore .
\section{Melrinho-das-ribeiras}
\begin{itemize}
\item {Grp. gram.:m.}
\end{itemize}
\begin{itemize}
\item {Utilização:Mad}
\end{itemize}
O mesmo que \textunderscore lavandisca\textunderscore .
\section{Melrinho-das-urzes}
\begin{itemize}
\item {Grp. gram.:m.}
\end{itemize}
\begin{itemize}
\item {Utilização:Mad}
\end{itemize}
O mesmo que \textunderscore abibe\textunderscore .
\section{Melrinho-de-nossa-senhora}
\begin{itemize}
\item {Grp. gram.:m.}
\end{itemize}
O mesmo que \textunderscore carreiró\textunderscore .
\section{Melrinho-de-nosso-senhor}
\begin{itemize}
\item {Grp. gram.:m.}
\end{itemize}
\begin{itemize}
\item {Utilização:Mad}
\end{itemize}
O mesmo que \textunderscore carreiró\textunderscore .
\section{Melrinho-de-papo-vermelho}
\begin{itemize}
\item {Grp. gram.:m.}
\end{itemize}
\begin{itemize}
\item {Utilização:Mad}
\end{itemize}
O mesmo que \textunderscore pintarroxo\textunderscore .
\section{Melrinho-do-mato}
\begin{itemize}
\item {Grp. gram.:m.}
\end{itemize}
\begin{itemize}
\item {Utilização:Mad}
\end{itemize}
O mesmo que \textunderscore tinge-burro\textunderscore .
\section{Melrinho-dos-pereiros}
\begin{itemize}
\item {Grp. gram.:m.}
\end{itemize}
\begin{itemize}
\item {Utilização:Mad}
\end{itemize}
O mesmo que \textunderscore abibe\textunderscore . Cf. Schmitz, \textunderscore Die Vögel\textunderscore , \textunderscore Madeira's\textunderscore .
\section{Melro}
\begin{itemize}
\item {Grp. gram.:m.}
\end{itemize}
\begin{itemize}
\item {Utilização:Fam.}
\end{itemize}
Pássaro dentirostro, (\textunderscore turdus merula\textunderscore ).
Homem finório, espertalhão.
Peixe de Portugal.
(Methát. de \textunderscore merlo\textunderscore , do lat. \textunderscore merulus\textunderscore )
\section{Mélroa}
\begin{itemize}
\item {Grp. gram.:f.}
\end{itemize}
Fêmea do melro.
\section{Melroado}
\begin{itemize}
\item {Grp. gram.:adj.}
\end{itemize}
Diz-se do cavallo, que tem a côr escura do melro.
\section{Melro-azul}
\begin{itemize}
\item {Grp. gram.:m.}
\end{itemize}
Passarinho, o mesmo que \textunderscore solitário\textunderscore .
\section{Melro-de-água}
\begin{itemize}
\item {Grp. gram.:m.}
\end{itemize}
O mesmo que \textunderscore melro-peixeiro\textunderscore .
\section{Melro-de-nossa-senhora}
\begin{itemize}
\item {Grp. gram.:m.}
\end{itemize}
\begin{itemize}
\item {Utilização:Mad}
\end{itemize}
O mesmo que \textunderscore pintasilgo\textunderscore .
\section{Melro-do-rancho}
\begin{itemize}
\item {Grp. gram.:m.}
\end{itemize}
\begin{itemize}
\item {Utilização:Mad}
\end{itemize}
O mesmo que \textunderscore pardal-francês\textunderscore .
\section{Melro-peixeiro}
\begin{itemize}
\item {Grp. gram.:m.}
\end{itemize}
O mesmo que \textunderscore tordo-marinho\textunderscore .
\section{Mélton}
\begin{itemize}
\item {Grp. gram.:m.}
\end{itemize}
Espécie de tecido de lan, fabricado em Inglaterra.
\section{Melúria}
\begin{itemize}
\item {Grp. gram.:f.}
\end{itemize}
\begin{itemize}
\item {Utilização:Pop.}
\end{itemize}
Lamentação habitual ou astuciosa.
(Por \textunderscore malúria\textunderscore , metáth. de \textunderscore lamúria\textunderscore )
\section{Melúria}
\begin{itemize}
\item {Grp. gram.:m.  e  f.}
\end{itemize}
\begin{itemize}
\item {Utilização:Pop.}
\end{itemize}
\begin{itemize}
\item {Proveniência:(De \textunderscore mel\textunderscore . Cp. \textunderscore mellúria\textunderscore )}
\end{itemize}
Pessoa dissimulada, melieira.
\section{Melúria}
\begin{itemize}
\item {Grp. gram.:f.}
\end{itemize}
\begin{itemize}
\item {Utilização:Pop.}
\end{itemize}
\begin{itemize}
\item {Utilização:Prov.}
\end{itemize}
\begin{itemize}
\item {Utilização:minh.}
\end{itemize}
\begin{itemize}
\item {Proveniência:(Do lat. \textunderscore mel\textunderscore , \textunderscore mellis\textunderscore )}
\end{itemize}
Qualidade daquele ou daquilo que é melífluo, suave.
Suavidade, doçura.
Primícias do mel, que se dão ao pároco. Cf. Camillo, \textunderscore Maria da Fonte\textunderscore , 33.
\section{Melusina}
\begin{itemize}
\item {Grp. gram.:f.}
\end{itemize}
\begin{itemize}
\item {Utilização:Heráld.}
\end{itemize}
Figura nua, meio mulher e meio serpente, de cabellos desgrenhados e banhando-se ou mirando-se numa piscina.
(Do gallês \textunderscore melusine\textunderscore , mulher que canta)
\section{Melzina}
\begin{itemize}
\item {Grp. gram.:f.}
\end{itemize}
\begin{itemize}
\item {Utilização:T. de Miranda}
\end{itemize}
O mesmo que \textunderscore mèzinha\textunderscore .
\section{Memactérias}
\begin{itemize}
\item {Grp. gram.:f. pl.}
\end{itemize}
Festas, que os Athenienses celebravam em honra de Júpiter, offerecendo sacrifícios para obter a salubridade pública.
(Cp. \textunderscore memactério\textunderscore )
\section{Memactério}
\begin{itemize}
\item {Grp. gram.:m.}
\end{itemize}
\begin{itemize}
\item {Proveniência:(Gr. \textunderscore maimakterion\textunderscore )}
\end{itemize}
Undécimo mês do anno áttico, em que os Athenienses celebravam as memactérias.
\section{Membeca}
\begin{itemize}
\item {Grp. gram.:adj.}
\end{itemize}
\begin{itemize}
\item {Utilização:Bras}
\end{itemize}
\begin{itemize}
\item {Proveniência:(T. tupi)}
\end{itemize}
Brando; tenro; molle.
\section{Membi}
\begin{itemize}
\item {Grp. gram.:m.}
\end{itemize}
\begin{itemize}
\item {Utilização:Bras}
\end{itemize}
Espécie de bambu.
\section{Membiapara}
\begin{itemize}
\item {Grp. gram.:f.}
\end{itemize}
\begin{itemize}
\item {Utilização:Bras}
\end{itemize}
Clarim de guerra, entre os Puris.
\section{Membóia-xió}
\begin{itemize}
\item {Grp. gram.:m.}
\end{itemize}
Árvore; espécie de taboca.
\section{Membrado}
\begin{itemize}
\item {Grp. gram.:adj.}
\end{itemize}
\begin{itemize}
\item {Utilização:Heráld.}
\end{itemize}
\begin{itemize}
\item {Proveniência:(De \textunderscore membro\textunderscore )}
\end{itemize}
Diz-se das aves, representadas nos escudos, com pernas de qualquer esmalte.
\section{Membrana}
\begin{itemize}
\item {Grp. gram.:f.}
\end{itemize}
\begin{itemize}
\item {Proveniência:(Lat. \textunderscore membrana\textunderscore )}
\end{itemize}
Tecido orgânico, mais ou menos laminoso, que envolve certos órgãos ou segrega certos líquidos.
Pellícula, que reveste certos órgãos vegetaes.
Pellícula.
\section{Membranáceo}
\begin{itemize}
\item {Grp. gram.:adj.}
\end{itemize}
\begin{itemize}
\item {Utilização:Bot.}
\end{itemize}
\begin{itemize}
\item {Proveniência:(Lat. \textunderscore membranaceus\textunderscore )}
\end{itemize}
Que tem a fórma ou consistência de membrana.
\section{Membraniforme}
\begin{itemize}
\item {Grp. gram.:adj.}
\end{itemize}
\begin{itemize}
\item {Proveniência:(De \textunderscore membrana\textunderscore  + \textunderscore fórma\textunderscore )}
\end{itemize}
Que tem fórma de membrana.
\section{Membranoso}
\begin{itemize}
\item {Grp. gram.:adj.}
\end{itemize}
Que tem membrana ou a natureza delia.
\section{Membrânula}
\begin{itemize}
\item {Grp. gram.:f.}
\end{itemize}
\begin{itemize}
\item {Utilização:Bot.}
\end{itemize}
Pequena membrana.
\section{Membro}
\begin{itemize}
\item {Grp. gram.:m.}
\end{itemize}
\begin{itemize}
\item {Utilização:Ant.}
\end{itemize}
\begin{itemize}
\item {Proveniência:(Lat. \textunderscore membrum\textunderscore )}
\end{itemize}
Parte appendicular do corpo do homem e do animal, com a qual se exercem movimentos.
Indivíduo, que faz parte de uma corporação.
Parte de uma nação ou de uma província.
Cada uma das partes de uma construcção.
Parte de uma phrase ou de um período, com sentido parcial.
Cada uma das partes de uma equação algébrica, separadas pelo sinal de igualdade.
Parte de um todo, em quanto reunida a elle.
Espécie de moeda antiga.
\section{Membrudo}
\begin{itemize}
\item {Grp. gram.:adj.}
\end{itemize}
\begin{itemize}
\item {Utilização:Fig.}
\end{itemize}
Que tem membros grandes e vigorosos.
Vigoroso.
\section{Membura}
\begin{itemize}
\item {Grp. gram.:f.}
\end{itemize}
\begin{itemize}
\item {Utilização:Bras. do N}
\end{itemize}
Cada um dos paus, que formam os extremos lateraes da jangada.
\section{Memecíleas}
\begin{itemize}
\item {Grp. gram.:f. pl.}
\end{itemize}
Família de plantas dicotiledóneas, originárias das regiões tropicaes. Cf. De-Candolle.
\section{Memecýleas}
\begin{itemize}
\item {Grp. gram.:f. pl.}
\end{itemize}
Família de plantas dicotyledóneas, originárias das regiões tropicaes. Cf. De-Candolle.
\section{Meminho}
\begin{itemize}
\item {Grp. gram.:adj.}
\end{itemize}
O mesmo que \textunderscore meiminho\textunderscore . Cf. Filinto, IV, 392.
\section{Memoração}
\begin{itemize}
\item {Grp. gram.:f.}
\end{itemize}
\begin{itemize}
\item {Proveniência:(Lat. \textunderscore memoratio\textunderscore )}
\end{itemize}
Acto ou effeito de memorar; commemoração.
\section{Memorando}
\begin{itemize}
\item {Grp. gram.:adj.}
\end{itemize}
\begin{itemize}
\item {Proveniência:(Lat. \textunderscore memorandus\textunderscore )}
\end{itemize}
Digno de memória, memorável.
\section{Memorandum}
\begin{itemize}
\item {fónica:memorândun}
\end{itemize}
\begin{itemize}
\item {Grp. gram.:m.}
\end{itemize}
\begin{itemize}
\item {Proveniência:(T. lat.)}
\end{itemize}
Livrinho do lembranças.
Participação ou aviso por escrito.
Nota diplomática de uma nação para outra, sôbre o estado de uma questão.
\section{Memorar}
\begin{itemize}
\item {Grp. gram.:v. i.}
\end{itemize}
\begin{itemize}
\item {Proveniência:(Lat. \textunderscore memorare\textunderscore )}
\end{itemize}
Trazer á memória; tornar lembrado.
Commemorar.
\section{Memorativo}
\begin{itemize}
\item {Grp. gram.:adj.}
\end{itemize}
\begin{itemize}
\item {Proveniência:(Lat. \textunderscore memorativus\textunderscore )}
\end{itemize}
O mesmo que \textunderscore commemorativo\textunderscore .
\section{Memorável}
\begin{itemize}
\item {Grp. gram.:adj.}
\end{itemize}
\begin{itemize}
\item {Utilização:Ext.}
\end{itemize}
\begin{itemize}
\item {Proveniência:(Lat. \textunderscore memorabilis\textunderscore )}
\end{itemize}
Digno de ficar na memória.
Notável; célebre.
\section{Memória}
\begin{itemize}
\item {Grp. gram.:f.}
\end{itemize}
\begin{itemize}
\item {Proveniência:(Lat. \textunderscore memoria\textunderscore )}
\end{itemize}
Faculdade de conservar ideias ou noções de objectos: \textunderscore conservar na memória\textunderscore .
Lembrança, reminiscência: \textunderscore memórias do passado\textunderscore .
Celebridade.
Reputação: \textunderscore deixou bôa memória\textunderscore .
Monumento commemorativo de pessoa célebre ou de sucesso notável: \textunderscore construiu-se uma memória no Buçaco\textunderscore .
Relação: \textunderscore escrever a memória de um naufrágio\textunderscore .
Apontamento para lembrança.
Memorial.
Vestígio.
Aquillo que serve de lembrança.
Dissertação.
Exposição summária de um successo, de um pedido, de uma reclamação, etc.
\section{Memorial}
\begin{itemize}
\item {Grp. gram.:m.}
\end{itemize}
\begin{itemize}
\item {Grp. gram.:Adj.}
\end{itemize}
\begin{itemize}
\item {Proveniência:(Lat. \textunderscore memoralis\textunderscore )}
\end{itemize}
Livrinho de lembranças.
Petição escrita.
Lembrança.
Memorável.
\section{Memoralista}
\begin{itemize}
\item {Grp. gram.:m.}
\end{itemize}
\begin{itemize}
\item {Proveniência:(De \textunderscore memorial\textunderscore )}
\end{itemize}
Aquelle que escreve memórias.
\section{Memorião}
\begin{itemize}
\item {Grp. gram.:m.}
\end{itemize}
\begin{itemize}
\item {Utilização:Fam.}
\end{itemize}
\begin{itemize}
\item {Proveniência:(De \textunderscore memória\textunderscore )}
\end{itemize}
Bôa memória; facilidade em decorar.
\section{Memoriar}
\begin{itemize}
\item {Grp. gram.:v. t.}
\end{itemize}
Reduzir a uma memória ou relação.
Fazer uma memória sôbre.
Inscrever. Cf. Castilho, \textunderscore D. Quixote\textunderscore , II, 80.
\section{Memorista}
\begin{itemize}
\item {Grp. gram.:m.}
\end{itemize}
\begin{itemize}
\item {Proveniência:(De \textunderscore memória\textunderscore )}
\end{itemize}
Autor do dissertações acadêmicas.
\section{Mempastor}
\begin{itemize}
\item {Grp. gram.:m.}
\end{itemize}
\begin{itemize}
\item {Utilização:Ant.}
\end{itemize}
O mesmo que \textunderscore mamposteiro\textunderscore . Cf. \textunderscore Port. Ant. e Mod.\textunderscore 
\section{Memphita}
\begin{itemize}
\item {Grp. gram.:adj.}
\end{itemize}
\begin{itemize}
\item {Proveniência:(Lat. \textunderscore memphites\textunderscore )}
\end{itemize}
Relativo a Mêmphis.
Diz-se especialmente do período da história da Arte, caracterizado pelos monumentos funerários de Mêmphis.
\section{Memphítico}
\begin{itemize}
\item {Grp. gram.:adj.}
\end{itemize}
O mesmo que \textunderscore memphita\textunderscore .
\section{Mênade}
\begin{itemize}
\item {Grp. gram.:f.}
\end{itemize}
\begin{itemize}
\item {Proveniência:(Lat. \textunderscore maenades\textunderscore )}
\end{itemize}
Sacerdotisa de Baccho; bacchante.
\section{Menagem}
\begin{itemize}
\item {Grp. gram.:f.}
\end{itemize}
\begin{itemize}
\item {Utilização:Ant.}
\end{itemize}
Homenagem, (des. neste sentido).
Prisão, fóra do cárcere ou sob a palavra do preso.
Promessa feita ou palavra dada, sôbre o cumprimento de uma cláusula ou contrato.
\textunderscore Tôrre de menagem\textunderscore , a tôrre principal de uma fortaleza.
(Aphér. de \textunderscore homenagem\textunderscore )
\section{Menálio}
\begin{itemize}
\item {Grp. gram.:adj.}
\end{itemize}
\begin{itemize}
\item {Utilização:Poét.}
\end{itemize}
\begin{itemize}
\item {Proveniência:(Lat. \textunderscore maenalius\textunderscore )}
\end{itemize}
Relativo ao monte Mênalo.
Bucólico, pastoril.
\section{Menancabos}
\begin{itemize}
\item {Grp. gram.:m. pl.}
\end{itemize}
Antigo povo da Malásia. Cf. \textunderscore Peregrinação\textunderscore , XVI.
\section{Menar}
\begin{itemize}
\item {Grp. gram.:v. t.}
\end{itemize}
\begin{itemize}
\item {Utilização:T. de Cucujães}
\end{itemize}
Espreitar.
Observar: \textunderscore demorei-me a menar a loja nora do Gregório\textunderscore .
\section{Menção}
\begin{itemize}
\item {Grp. gram.:f.}
\end{itemize}
\begin{itemize}
\item {Utilização:Pop.}
\end{itemize}
\begin{itemize}
\item {Proveniência:(Lat. \textunderscore mentio\textunderscore )}
\end{itemize}
Referência.
Registo.
Inscripção.
Lembrança por incidente.
Tenção, gestos de quem se dispõe para praticar um acto: \textunderscore fez menção de me bater\textunderscore .
\section{Mencionar}
\begin{itemize}
\item {Grp. gram.:v. t.}
\end{itemize}
\begin{itemize}
\item {Proveniência:(Do lat. \textunderscore mentio\textunderscore )}
\end{itemize}
Fazer menção de; expor, referir.
\section{Mençonha}
\begin{itemize}
\item {Grp. gram.:f.}
\end{itemize}
\begin{itemize}
\item {Utilização:Ant.}
\end{itemize}
\begin{itemize}
\item {Proveniência:(It. \textunderscore menzogna\textunderscore )}
\end{itemize}
O mesmo que \textunderscore mentira\textunderscore .
\section{Mencumbió}
\begin{itemize}
\item {Grp. gram.:m.}
\end{itemize}
Árvore da Índia portuguesa.
\section{Mendace}
\begin{itemize}
\item {Grp. gram.:adj.}
\end{itemize}
O mesmo que \textunderscore mendaz\textunderscore .
\section{Mendacidade}
\begin{itemize}
\item {Grp. gram.:f.}
\end{itemize}
\begin{itemize}
\item {Proveniência:(Lat. \textunderscore mendacitas\textunderscore )}
\end{itemize}
Qualidade de quem é mendaz.
\section{Mendáculo}
\begin{itemize}
\item {Grp. gram.:m.}
\end{itemize}
\begin{itemize}
\item {Utilização:Bras}
\end{itemize}
Defeito moral; mancha.
(Talvez de \textunderscore mendaz\textunderscore )
\section{Mendaz}
\begin{itemize}
\item {Grp. gram.:adj.}
\end{itemize}
\begin{itemize}
\item {Proveniência:(Lat. \textunderscore mendax\textunderscore )}
\end{itemize}
Mentiroso; falso.
\section{Mendésio}
\begin{itemize}
\item {Grp. gram.:m.}
\end{itemize}
\begin{itemize}
\item {Proveniência:(Lat. \textunderscore mendesius\textunderscore )}
\end{itemize}
Unguento cheiroso, feito de óleo de amêndoas amargas do Egypto.
\section{Mendicância}
\begin{itemize}
\item {Grp. gram.:f.}
\end{itemize}
\begin{itemize}
\item {Proveniência:(De \textunderscore mendicante\textunderscore )}
\end{itemize}
O mesmo que \textunderscore mendicidade\textunderscore .
\section{Mendicante}
\begin{itemize}
\item {Grp. gram.:m.  e  adj.}
\end{itemize}
\begin{itemize}
\item {Utilização:Prov.}
\end{itemize}
\begin{itemize}
\item {Utilização:trasm.}
\end{itemize}
\begin{itemize}
\item {Proveniência:(Lat. \textunderscore mendicans\textunderscore )}
\end{itemize}
O que mendiga.
Ocioso, vadio.
\section{Mendicidade}
\begin{itemize}
\item {Grp. gram.:f.}
\end{itemize}
\begin{itemize}
\item {Proveniência:(Lat. \textunderscore mendicitas\textunderscore )}
\end{itemize}
Qualidade de quem é mendigo.
Acto de mendigar.
Classe dos mendigos; os mendigos: \textunderscore asilo de mendicidade\textunderscore .
\section{Mendigação}
\begin{itemize}
\item {Grp. gram.:f.}
\end{itemize}
\begin{itemize}
\item {Proveniência:(Lat. \textunderscore mendicatio\textunderscore )}
\end{itemize}
Acto de mendigar.
\section{Mendigagem}
\begin{itemize}
\item {Grp. gram.:f.}
\end{itemize}
Vida de mendigo, mendicidade.
Os mendigos. Cf. Alv. Mendes, \textunderscore Discursos\textunderscore , 280.
\section{Mendigar}
\begin{itemize}
\item {Grp. gram.:v. t.}
\end{itemize}
\begin{itemize}
\item {Utilização:Fig.}
\end{itemize}
\begin{itemize}
\item {Grp. gram.:V. i.}
\end{itemize}
\begin{itemize}
\item {Proveniência:(Do lat. \textunderscore mendicare\textunderscore )}
\end{itemize}
Pedir por esmola.
Solicitar com humildade: \textunderscore mendigar um emprego\textunderscore .
Procurar entre pessôas ou coisas estranhas.
Pedir esmola, viver de esmolas.
\section{Mendigaria}
\begin{itemize}
\item {Grp. gram.:f.}
\end{itemize}
\begin{itemize}
\item {Utilização:Ant.}
\end{itemize}
Mendicidade. Cf. \textunderscore Eufrosina\textunderscore , 44.
\section{Mendigo}
\begin{itemize}
\item {Grp. gram.:m.}
\end{itemize}
\begin{itemize}
\item {Proveniência:(Do lat. \textunderscore mendicus\textunderscore )}
\end{itemize}
Aquelle que pede esmola para viver.
Pedinte.
\section{Mendiguez}
\begin{itemize}
\item {Grp. gram.:f.}
\end{itemize}
\begin{itemize}
\item {Utilização:P. us.}
\end{itemize}
Estado ou qualidade de mendigo.
\section{Mendinho}
\begin{itemize}
\item {Grp. gram.:adj.}
\end{itemize}
O mesmo que \textunderscore mindinho\textunderscore .
\section{Mendobi}
\begin{itemize}
\item {Grp. gram.:m.}
\end{itemize}
O mesmo que \textunderscore amendoim\textunderscore .
\section{Mendobim}
\begin{itemize}
\item {Grp. gram.:m.}
\end{itemize}
O mesmo que \textunderscore amendoim\textunderscore .
\section{Mêndola}
\begin{itemize}
\item {Grp. gram.:f.}
\end{itemize}
Espécie de arenque.
\section{Mendorim}
\begin{itemize}
\item {Grp. gram.:m.}
\end{itemize}
\begin{itemize}
\item {Utilização:Bras}
\end{itemize}
Pequena abelha avermelhada.
\section{Mendos}
\begin{itemize}
\item {Grp. gram.:m. pl.}
\end{itemize}
Indígenas do norte do Brasil.
\section{Mendosa}
\begin{itemize}
\item {Grp. gram.:adj. f.}
\end{itemize}
\begin{itemize}
\item {Utilização:Anat.}
\end{itemize}
\begin{itemize}
\item {Proveniência:(Lat. \textunderscore mendosa\textunderscore )}
\end{itemize}
Dizia-se de cada uma das falsas costellas, isto é, das não articuladas com o esterno.
\section{Mendrugo}
\begin{itemize}
\item {Grp. gram.:m.}
\end{itemize}
\begin{itemize}
\item {Utilização:Des.}
\end{itemize}
Pedaço de pão, que se dá ao pobre. Cf. Roquete, \textunderscore Diction. Port. Franç.\textunderscore 
\section{Mendrulho}
\begin{itemize}
\item {Grp. gram.:m.}
\end{itemize}
\begin{itemize}
\item {Utilização:Prov.}
\end{itemize}
\begin{itemize}
\item {Utilização:minh.}
\end{itemize}
Pessôa suja e desajeitada.
\section{Mendubi}
\begin{itemize}
\item {Grp. gram.:m.}
\end{itemize}
O mesmo que \textunderscore mendobim\textunderscore .
\section{Meneador}
\begin{itemize}
\item {Grp. gram.:m.  e  adj.}
\end{itemize}
O que meneia.
\section{Meneamento}
\begin{itemize}
\item {Grp. gram.:m.}
\end{itemize}
Acto ou effeito de \textunderscore menear\textunderscore .
\section{Menear}
\begin{itemize}
\item {Grp. gram.:v. t.}
\end{itemize}
Mover de um lado para outro.
Saracotear; manejar.
(Alter. de manear)
\section{Meneável}
\begin{itemize}
\item {Grp. gram.:adj.}
\end{itemize}
\begin{itemize}
\item {Utilização:Fig.}
\end{itemize}
\begin{itemize}
\item {Proveniência:(De \textunderscore menear\textunderscore )}
\end{itemize}
Que se póde menear.
O mesmo que \textunderscore flexível\textunderscore .
\section{Menecombió}
\begin{itemize}
\item {Grp. gram.:m.}
\end{itemize}
O mesmo que \textunderscore mencumbió\textunderscore .
\section{Meneio}
\begin{itemize}
\item {Grp. gram.:m.}
\end{itemize}
\begin{itemize}
\item {Utilização:Fig.}
\end{itemize}
\begin{itemize}
\item {Proveniência:(De \textunderscore menear\textunderscore )}
\end{itemize}
O mesmo que \textunderscore meneamento\textunderscore ; gesto.
Manejo, astúcia.
Mão de obra.
Preparo.
Applicação.
\section{Menemenebanta}
\begin{itemize}
\item {Grp. gram.:f.}
\end{itemize}
Árvore medicinal da Guiné.
\section{Menencoria}
\begin{itemize}
\item {Grp. gram.:f.}
\end{itemize}
\begin{itemize}
\item {Utilização:Ant.}
\end{itemize}
O mesmo que \textunderscore melancolia\textunderscore .
\section{Menêo}
\begin{itemize}
\item {Grp. gram.:m.}
\end{itemize}
\begin{itemize}
\item {Utilização:Ant.}
\end{itemize}
O mesmo que \textunderscore meneio\textunderscore .
\section{Menesa}
\begin{itemize}
\item {fónica:nê}
\end{itemize}
\begin{itemize}
\item {Grp. gram.:f.}
\end{itemize}
\begin{itemize}
\item {Utilização:Gír.}
\end{itemize}
O mesmo que \textunderscore manesa\textunderscore .
\section{Menestrel}
\begin{itemize}
\item {Grp. gram.:m.}
\end{itemize}
\begin{itemize}
\item {Proveniência:(Do lat. hypoth. \textunderscore mínistrellus\textunderscore )}
\end{itemize}
Poéta medieval.
Trovador; músico.
\section{Menfita}
\begin{itemize}
\item {Grp. gram.:adj.}
\end{itemize}
\begin{itemize}
\item {Proveniência:(Lat. \textunderscore memphites\textunderscore )}
\end{itemize}
Relativo a Mêmphis.
Diz-se especialmente do período da história da Arte, caracterizado pelos monumentos funerários de Mêmphis.
\section{Mengengra}
\begin{itemize}
\item {Grp. gram.:f.}
\end{itemize}
O mesmo que \textunderscore megengra\textunderscore .
\section{Mengo}
\begin{itemize}
\item {Grp. gram.:m.}
\end{itemize}
A lan, que a esfarrapadeira deixa apta para se laborar, nas fábricas de lanifícios.
\section{Mêngoa}
\textunderscore f.\textunderscore  (e der.)
(Fórma ant. de \textunderscore míngua\textunderscore , etc.)
\section{Menha}
\begin{itemize}
\item {Grp. gram.:f.}
\end{itemize}
\begin{itemize}
\item {Utilização:T. da África port}
\end{itemize}
\begin{itemize}
\item {Proveniência:(T. quimbundo)}
\end{itemize}
O mesmo que \textunderscore água\textunderscore ^1. Cf. \textunderscore Diccion. de nomes, Vozes e Coisas\textunderscore , ms. da Tôrre do Tombo.
\section{Menhir}
\begin{itemize}
\item {Grp. gram.:m.}
\end{itemize}
Grande pedra, fixada verticalmente no solo, em tempos remotíssimos, e cuja applicação ou significação é ainda desconhecida.
(Do baixo-bretão \textunderscore men\textunderscore  + \textunderscore hir\textunderscore )
\section{Meni}
\begin{itemize}
\item {Grp. gram.:m.}
\end{itemize}
\begin{itemize}
\item {Utilização:Ant.}
\end{itemize}
Baêta ou pano ordinário, de que as mulheres faziam mantilhas.
\section{Meniano}
\begin{itemize}
\item {Grp. gram.:m.}
\end{itemize}
\begin{itemize}
\item {Proveniência:(Lat. \textunderscore maenianum\textunderscore )}
\end{itemize}
Pequeno terraço, varanda ou balcão, á frente de edifícios, usado principalmente em Itália.
Cada uma das ordens de degraus nos circos romanos.
\section{Menianthina}
\begin{itemize}
\item {Grp. gram.:f.}
\end{itemize}
O mesmo que \textunderscore dahlina\textunderscore .--Assim escreve Caminhoá, \textunderscore Bot. Ger. e Med.\textunderscore , reproduzindo o êrro de Linneu, que chamou \textunderscore meniantho\textunderscore  ao que devia chamar \textunderscore mynianlho\textunderscore .
\section{Meniantina}
\begin{itemize}
\item {Grp. gram.:f.}
\end{itemize}
O mesmo que \textunderscore dahlina\textunderscore .--Assim escreve Caminhoá, \textunderscore Bot. Ger. e Med.\textunderscore , reproduzindo o êrro de Linneu, que chamou \textunderscore meniantho\textunderscore  ao que devia chamar \textunderscore mynianlho\textunderscore .
\section{Ménidos}
\begin{itemize}
\item {Grp. gram.:m. pl.}
\end{itemize}
Família de peixes acanthopterýgios.
\section{Menigrepo}
\begin{itemize}
\item {Grp. gram.:m.}
\end{itemize}
(V.manigrepo)
\section{Menilita}
\begin{itemize}
\item {Grp. gram.:f.}
\end{itemize}
Variedade de opala.
(Metáth. de \textunderscore melinita\textunderscore ? Cp. \textunderscore melinite\textunderscore )
\section{Menim}
\begin{itemize}
\item {Grp. gram.:m.}
\end{itemize}
\begin{itemize}
\item {Utilização:Ant.}
\end{itemize}
Espécie de tecido:«\textunderscore ...calças de menim\textunderscore ». J. Pedro Ribeiro, \textunderscore Dissert.\textunderscore , \textunderscore Chronol.\textunderscore , V, 308.
\section{Menina}
\begin{itemize}
\item {Grp. gram.:f.}
\end{itemize}
\begin{itemize}
\item {Utilização:Gír.}
\end{itemize}
\begin{itemize}
\item {Utilização:Fam.}
\end{itemize}
\begin{itemize}
\item {Grp. gram.:Adj.}
\end{itemize}
\begin{itemize}
\item {Proveniência:(De \textunderscore menino\textunderscore )}
\end{itemize}
Criança do sexo feminino.
Mulher, nova, delicada e de bôa educação.
Tratamento affectuoso que, em família, se dá as pessoas do sexo feminino, crianças e adultas.
Chave.
\textunderscore Menina do olho\textunderscore , a pupilla ocular.
\textunderscore Menina de cinco olhos\textunderscore , palmatória.
Diz-se de uma espécie de abóbora.
\section{Menina-casadoira}
\begin{itemize}
\item {Grp. gram.:f.}
\end{itemize}
\begin{itemize}
\item {Utilização:Prov.}
\end{itemize}
\begin{itemize}
\item {Utilização:alent.}
\end{itemize}
Espécie de dança de roda.
\section{Menineiro}
\begin{itemize}
\item {Grp. gram.:adj.}
\end{itemize}
Que tem aparência de menino.
Pueril.
Que gosta de crianças ou que é muito carinhoso para crianças.
\section{Meninez}
\begin{itemize}
\item {Grp. gram.:f.}
\end{itemize}
O mesmo que \textunderscore meninice\textunderscore .
\section{Meninges}
\begin{itemize}
\item {Grp. gram.:f. pl.}
\end{itemize}
\begin{itemize}
\item {Utilização:Anat.}
\end{itemize}
\begin{itemize}
\item {Proveniência:(Do lat. \textunderscore meninx\textunderscore , \textunderscore meningis\textunderscore )}
\end{itemize}
As três membranas, que envolvem o apparelho cérebro-espinal.
\section{Meningina}
\begin{itemize}
\item {Grp. gram.:f.}
\end{itemize}
\begin{itemize}
\item {Proveniência:(De \textunderscore meninges\textunderscore )}
\end{itemize}
Nome, que se deu collectivamente a duas das meninges, a arachnoide e a pia-máter, que foram consideradas como uma só membrana, formada de dois foliolos.
\section{Meningite}
\begin{itemize}
\item {Grp. gram.:f.}
\end{itemize}
Inflammação das meninges, especialmente de arachnoide.
\section{Meningocele}
\begin{itemize}
\item {Grp. gram.:m.}
\end{itemize}
\begin{itemize}
\item {Proveniência:(Do gr. \textunderscore meninx\textunderscore  + \textunderscore kele\textunderscore )}
\end{itemize}
Tumor craniano, formado por hérnia da pia-máter.
\section{Meningo-cephalite}
\begin{itemize}
\item {Grp. gram.:f.}
\end{itemize}
O mesmo que \textunderscore meningococco\textunderscore .
\section{Meningococco}
\begin{itemize}
\item {Grp. gram.:m.}
\end{itemize}
Micróbio da meningite.
\section{Meningococo}
\begin{itemize}
\item {Grp. gram.:m.}
\end{itemize}
Micróbio da meningite.
\section{Meningo-encephalite}
\begin{itemize}
\item {Grp. gram.:f.}
\end{itemize}
Inflammação simultânea das meninges e da massa encephálica.
\section{Meningofilaz}
\begin{itemize}
\item {Grp. gram.:m.}
\end{itemize}
\begin{itemize}
\item {Proveniência:(Gr. \textunderscore meningophulax\textunderscore )}
\end{itemize}
Antigo instrumento, com que se protegiam as meninges, quando se fazia a trepanação.
\section{Meniago-gástrico}
\begin{itemize}
\item {Grp. gram.:adj.}
\end{itemize}
Nome, que se deu ás febres biliosas ou gástricas.
\section{Meningo-myalite}
\begin{itemize}
\item {Grp. gram.:f.}
\end{itemize}
Inflammação da medulla espinhal e dos seus invólucros.
\section{Meningophylaz}
\begin{itemize}
\item {Grp. gram.:m.}
\end{itemize}
\begin{itemize}
\item {Proveniência:(Gr. \textunderscore meningophulax\textunderscore )}
\end{itemize}
Antigo instrumento, com que se protegiam as meninges, quando se fazia a trepanação.
\section{Meningose}
\begin{itemize}
\item {Grp. gram.:f.}
\end{itemize}
\begin{itemize}
\item {Utilização:Anat.}
\end{itemize}
\begin{itemize}
\item {Proveniência:(Do lat. \textunderscore meninx\textunderscore )}
\end{itemize}
União de dois ossos por meio de ligamentos em forma de membrana.
\section{Meninho}
\begin{itemize}
\item {Grp. gram.:m.}
\end{itemize}
\begin{itemize}
\item {Utilização:Ant.}
\end{itemize}
O mesmo que \textunderscore menino\textunderscore .
\section{Meninice}
\begin{itemize}
\item {Grp. gram.:f.}
\end{itemize}
Idade ou qualidade de quem é menino.
Modos ou actos próprios de menino.
\section{Meninil}
\begin{itemize}
\item {Grp. gram.:adj.}
\end{itemize}
\begin{itemize}
\item {Utilização:P. us.}
\end{itemize}
Próprio de menino; infantil; acriançado.
\section{Menino}
\begin{itemize}
\item {Grp. gram.:m.}
\end{itemize}
\begin{itemize}
\item {Utilização:Ext.}
\end{itemize}
\begin{itemize}
\item {Utilização:Irón.}
\end{itemize}
\begin{itemize}
\item {Proveniência:(Do lat. \textunderscore minimus\textunderscore , de \textunderscore minor\textunderscore ? Já se aventurou a or. do cast. \textunderscore mi niño\textunderscore . Cp. \textunderscore meninho\textunderscore , e a forma des. \textunderscore minino\textunderscore )}
\end{itemize}
Criança do sexo masculino.
Tratamento affectuoso entre parentes ou amigos, ainda que adultos.
Indivíduo finório, espertalhão: \textunderscore aquillo que é menino\textunderscore .
\section{Meninó}
\begin{itemize}
\item {Grp. gram.:m.}
\end{itemize}
\begin{itemize}
\item {Utilização:Fam.}
\end{itemize}
Indivíduo finório, espertalhão.
(Alter. de \textunderscore menino\textunderscore )
\section{Menĩo}
\begin{itemize}
\item {Grp. gram.:m.}
\end{itemize}
Fórma obsoleta de \textunderscore menino\textunderscore . Cf. Frei Fortun., \textunderscore Inéditos\textunderscore , 310.
\section{Menisco}
\begin{itemize}
\item {Grp. gram.:f.}
\end{itemize}
\begin{itemize}
\item {Proveniência:(Do gr. \textunderscore meniskos\textunderscore )}
\end{itemize}
Vidro lenticular.
Superfície curva de um líquido contido num tubo capillar.
Figura geométrica com um lado convexo e outro côncavo.
\section{Meniscóide}
\begin{itemize}
\item {Grp. gram.:adj.}
\end{itemize}
\begin{itemize}
\item {Proveniência:(Do gr. \textunderscore meniskos\textunderscore  + \textunderscore eidos\textunderscore )}
\end{itemize}
Que tem fórma de menisco.
\section{Meniscoídeo}
\begin{itemize}
\item {Grp. gram.:adj.}
\end{itemize}
O mesmo que \textunderscore meniscóide\textunderscore .
\section{Menispermáceas}
\begin{itemize}
\item {Grp. gram.:f. pl.}
\end{itemize}
\begin{itemize}
\item {Proveniência:(De \textunderscore menispermáceo\textunderscore )}
\end{itemize}
Família de plantas, que tem por typo o menispermo.
\section{Menispermáceo}
\begin{itemize}
\item {Grp. gram.:adj.}
\end{itemize}
Relativo ou semelhante ao menispermo.
\section{Menispérmeas}
\begin{itemize}
\item {Grp. gram.:f. pl.}
\end{itemize}
\begin{itemize}
\item {Proveniência:(De \textunderscore menispérmeo\textunderscore )}
\end{itemize}
O mesmo que \textunderscore menispermáceas\textunderscore .
Tríbo de menispermáceas.
\section{Menispérmeo}
\begin{itemize}
\item {Grp. gram.:adj.}
\end{itemize}
O mesmo que \textunderscore menispermáceo\textunderscore .
\section{Menispermínico}
\begin{itemize}
\item {Grp. gram.:adj.}
\end{itemize}
Diz-se de um ácido, extrahido do uma espécie de menispermo, (\textunderscore menispermum coccalus\textunderscore ).
\section{Menispermo}
\begin{itemize}
\item {Grp. gram.:m.}
\end{itemize}
\begin{itemize}
\item {Proveniência:(Do gr. \textunderscore mene\textunderscore  + \textunderscore sperma\textunderscore )}
\end{itemize}
Gênero de plantas medicinaes, trepadeiras e sarmentosas.
\section{Menispermóides}
\begin{itemize}
\item {Grp. gram.:f. pl.}
\end{itemize}
(V.menispermáceas)
\section{Menistre}
\begin{itemize}
\item {Grp. gram.:m.}
\end{itemize}
\begin{itemize}
\item {Utilização:Ant.}
\end{itemize}
\begin{itemize}
\item {Grp. gram.:F.}
\end{itemize}
O mesmo que \textunderscore menestrel\textunderscore .
Instrumento de menestréis, ou charamela:«\textunderscore ...atabales e menistres altas que tangiam\textunderscore ». G. de Resende.
\section{Menistril}
\begin{itemize}
\item {Grp. gram.:m.}
\end{itemize}
\begin{itemize}
\item {Utilização:Ant.}
\end{itemize}
O mesmo que \textunderscore menestrel\textunderscore .
\section{Meno}
\begin{itemize}
\item {Grp. gram.:m.}
\end{itemize}
\begin{itemize}
\item {Utilização:Ant.}
\end{itemize}
Bugio grande.
\section{Menológio}
\begin{itemize}
\item {Grp. gram.:m.}
\end{itemize}
\begin{itemize}
\item {Proveniência:(Lat. \textunderscore menologium\textunderscore )}
\end{itemize}
Descripção ou tratado dos meses, entre os differentes povos.
Catálogo dos mártyres, na Igreja grega.
\section{Menopausa}
\begin{itemize}
\item {Grp. gram.:f.}
\end{itemize}
\begin{itemize}
\item {Utilização:Physiol.}
\end{itemize}
\begin{itemize}
\item {Proveniência:(Do gr. \textunderscore men\textunderscore , mês, e \textunderscore pausis\textunderscore  = lat. \textunderscore pausa\textunderscore , interrupção, cessação)}
\end{itemize}
Cessação do catamênio ou das regras; idade critica da mulher.
\section{Menór}
\begin{itemize}
\item {Grp. gram.:adj. comp.}
\end{itemize}
\begin{itemize}
\item {Grp. gram.:Pl.}
\end{itemize}
\begin{itemize}
\item {Grp. gram.:M.  e  f.}
\end{itemize}
\begin{itemize}
\item {Grp. gram.:Pl.}
\end{itemize}
\begin{itemize}
\item {Utilização:Prov.}
\end{itemize}
\begin{itemize}
\item {Utilização:trasm.}
\end{itemize}
\begin{itemize}
\item {Utilização:Des.}
\end{itemize}
\begin{itemize}
\item {Grp. gram.:Loc. adv.}
\end{itemize}
\begin{itemize}
\item {Proveniência:(Lat. \textunderscore minor\textunderscore )}
\end{itemize}
Mais pequeno.
Inferior.
Que ainda não attingiu a maioridade.
Diz-se dos hábitos ou peças de vestuário, que se usam só por baixo de outras roupas, taes como a camisa, as ceroilas, a anágua.
Pessôa, que ainda não chegou á maioridade.
\textunderscore Frade menor\textunderscore , o mesmo que \textunderscore franciscano\textunderscore .
O mesmo que \textunderscore ceroilas\textunderscore .
Descendentes.
Minúcias.
\textunderscore Por menóres\textunderscore , minuciosamente:«\textunderscore contou por menores o que se passava\textunderscore ». Camillo, \textunderscore Volcoens\textunderscore , 134.
\section{Menoretas}
\begin{itemize}
\item {fónica:norê}
\end{itemize}
\begin{itemize}
\item {Grp. gram.:f. pl.}
\end{itemize}
\begin{itemize}
\item {Proveniência:(De \textunderscore menór\textunderscore )}
\end{itemize}
Religiosas de Santa-Clara, da Ordem de San-Francisco, o patriarcha \textunderscore menór\textunderscore , como êlle se appellidava.
\section{Menoridade}
\begin{itemize}
\item {Grp. gram.:f.}
\end{itemize}
\begin{itemize}
\item {Utilização:Fig.}
\end{itemize}
\begin{itemize}
\item {Proveniência:(De \textunderscore menór\textunderscore )}
\end{itemize}
Período da vida até aos vinte e um annos, em que a lei portuguesa reconhece no indivíduo a faculdade de reger sua pessôa e bens.
Minoria.
A parte ou quantidade mais pequena de um todo.
\section{Menorista}
\begin{itemize}
\item {Grp. gram.:m.}
\end{itemize}
Clérigo de ordens menóres.
\section{Menorita}
\begin{itemize}
\item {Grp. gram.:m.}
\end{itemize}
\begin{itemize}
\item {Proveniência:(De \textunderscore menór\textunderscore . Cp. \textunderscore menoretas\textunderscore )}
\end{itemize}
Religioso franciscano.
\section{Menorragia}
\begin{itemize}
\item {Grp. gram.:f.}
\end{itemize}
\begin{itemize}
\item {Proveniência:(Do gr. \textunderscore men\textunderscore  + \textunderscore rhagein\textunderscore )}
\end{itemize}
Excesso de fluxo menstrual.
\section{Menorrágico}
\begin{itemize}
\item {Grp. gram.:adj.}
\end{itemize}
Relativo á menorragia.
\section{Menorrhagia}
\begin{itemize}
\item {Grp. gram.:f.}
\end{itemize}
\begin{itemize}
\item {Proveniência:(Do gr. \textunderscore men\textunderscore  + \textunderscore rhagein\textunderscore )}
\end{itemize}
Excesso de fluxo menstrual.
\section{Menorreia}
\begin{itemize}
\item {Grp. gram.:f.}
\end{itemize}
\begin{itemize}
\item {Proveniência:(Do gr. \textunderscore men\textunderscore  + \textunderscore rhein\textunderscore )}
\end{itemize}
Mênstruo.
\section{Menorrhágico}
\begin{itemize}
\item {Grp. gram.:adj.}
\end{itemize}
Relativo á menorrhagia.
\section{Menorrheia}
\begin{itemize}
\item {Grp. gram.:f.}
\end{itemize}
\begin{itemize}
\item {Proveniência:(Do gr. \textunderscore men\textunderscore  + \textunderscore rhein\textunderscore )}
\end{itemize}
Mênstruo.
\section{Menos}
\begin{itemize}
\item {Grp. gram.:adv. comp.}
\end{itemize}
\begin{itemize}
\item {Grp. gram.:Prep.}
\end{itemize}
\begin{itemize}
\item {Grp. gram.:Adj.}
\end{itemize}
\begin{itemize}
\item {Utilização:ant.}
\end{itemize}
\begin{itemize}
\item {Grp. gram.:M.}
\end{itemize}
Inferiormente em quantidade, número ou condição: \textunderscore agora soffres menos\textunderscore .
Em menor número, quantidade ou posição.
Excepto: \textunderscore todos, menos êlle\textunderscore .
Menór. Cf. G. Vicente, I, 108.
Aquillo que é inferior, mais baixo ou mínimo: \textunderscore mas isso é o menos\textunderscore .
(Lat. minus).
\section{Menos...}
\begin{itemize}
\item {Grp. gram.:pref. átono}
\end{itemize}
(designativo de \textunderscore inferioridade\textunderscore )
\section{Menoscobador}
\begin{itemize}
\item {Grp. gram.:m.  e  adj.}
\end{itemize}
O que menoscaba.
\section{Menoscabar}
\begin{itemize}
\item {Grp. gram.:v. t.}
\end{itemize}
\begin{itemize}
\item {Utilização:Fig.}
\end{itemize}
Tornar imperfeito; deixar incompleto.
Depreciar; desprezar; desdoirar.
\section{Menoscabo}
\begin{itemize}
\item {Grp. gram.:m.}
\end{itemize}
Acto ou effeito do menoscabar.
\section{Menospreciar}
\begin{itemize}
\item {Grp. gram.:v. t.}
\end{itemize}
\begin{itemize}
\item {Utilização:Des.}
\end{itemize}
O mesmo que \textunderscore menosprezar\textunderscore .
\section{Menospreço}
\begin{itemize}
\item {Grp. gram.:m.}
\end{itemize}
\begin{itemize}
\item {Utilização:Bras}
\end{itemize}
O mesmo que \textunderscore menosprêzo\textunderscore .
\section{Menosprezador}
\begin{itemize}
\item {Grp. gram.:m.  e  adj.}
\end{itemize}
O que menospreza.
\section{Menosprezar}
\begin{itemize}
\item {Grp. gram.:v. t.}
\end{itemize}
\begin{itemize}
\item {Proveniência:(De \textunderscore menos...\textunderscore  + \textunderscore prezar\textunderscore )}
\end{itemize}
Têr em pouca conta; depreciar; desprezar.
\section{Menosprezível}
\begin{itemize}
\item {Grp. gram.:adj.}
\end{itemize}
\begin{itemize}
\item {Proveniência:(De \textunderscore menosprezar\textunderscore )}
\end{itemize}
Que é digno de desprêzo, desprezível.
\section{Menosprêzo}
\begin{itemize}
\item {Grp. gram.:m.}
\end{itemize}
Acto ou effeito de menosprezar; desprêzo; desdém.
\section{Menostasia}
\begin{itemize}
\item {Grp. gram.:f.}
\end{itemize}
\begin{itemize}
\item {Proveniência:(Do gr. \textunderscore men\textunderscore  + \textunderscore stasis\textunderscore )}
\end{itemize}
Retenção ou suppresão do mênstruo.
\section{Mensa}
\begin{itemize}
\item {Grp. gram.:f.}
\end{itemize}
\begin{itemize}
\item {Utilização:Ant.}
\end{itemize}
\begin{itemize}
\item {Proveniência:(Lat. \textunderscore mensa\textunderscore )}
\end{itemize}
O mesmo que \textunderscore mesa\textunderscore ^1.
\section{Mensageira}
\begin{itemize}
\item {Grp. gram.:f.  e  adj.}
\end{itemize}
Diz-se a pessôa ou coisa, que leva mensagem ou que annuncia ou que presagia: \textunderscore nuvem mensageira de tempestade\textunderscore . \textunderscore A pomba, mensageira da paz\textunderscore .
(De \textunderscore mensageiro\textunderscore ).
\section{Mensageiro}
\begin{itemize}
\item {Grp. gram.:m.  e  adj.}
\end{itemize}
\begin{itemize}
\item {Proveniência:(De \textunderscore mensagem\textunderscore )}
\end{itemize}
O que leva mensagem.
Annunciador.
Aquillo que envolve preságio.
Aquelle que faz preságio.
\section{Mensagem}
\begin{itemize}
\item {Grp. gram.:f.}
\end{itemize}
\begin{itemize}
\item {Proveniência:(Do b. lat. \textunderscore missaticum\textunderscore )}
\end{itemize}
Notícia, communicada verbalmente.
Recado.
Discurso escrito, que um presidente de república envia ao parlamento.
Felicitação ou discurso escrito, dirigido a uma autoridade.
Communicação official entre as câmaras legislativas, ou entre o poder legislativo e o executivo.
\section{Mensal}
\begin{itemize}
\item {Grp. gram.:adj.}
\end{itemize}
\begin{itemize}
\item {Proveniência:(Lat. \textunderscore mensualis\textunderscore )}
\end{itemize}
Relativo a mês.
Que se realiza de mês a mês: \textunderscore publicação mensal\textunderscore .
Que dura um mês.
\section{Mensalidade}
\begin{itemize}
\item {Grp. gram.:f.}
\end{itemize}
\begin{itemize}
\item {Proveniência:(De \textunderscore mensal\textunderscore )}
\end{itemize}
Quantia de dinheiro relativo a um mês.
Mesada.
\section{Mensalmente}
\begin{itemize}
\item {Grp. gram.:adv.}
\end{itemize}
\begin{itemize}
\item {Proveniência:(De \textunderscore mensal\textunderscore )}
\end{itemize}
Em cada mês.
Uma vez por mês.
\section{Mensário}
\begin{itemize}
\item {Grp. gram.:m.}
\end{itemize}
\begin{itemize}
\item {Utilização:Neol.}
\end{itemize}
\begin{itemize}
\item {Proveniência:(Do lat. \textunderscore mensis\textunderscore )}
\end{itemize}
Periódico, que se publica de mês a mês: \textunderscore a«Arte», mensário de literatura...\textunderscore 
\section{Mensário}
\begin{itemize}
\item {Grp. gram.:adj.}
\end{itemize}
\begin{itemize}
\item {Proveniência:(Lat. \textunderscore mensarius\textunderscore )}
\end{itemize}
Relativo á mesa ou ao que se come á mesa.
\section{Mensório}
\begin{itemize}
\item {Grp. gram.:m.}
\end{itemize}
\begin{itemize}
\item {Utilização:Ant.}
\end{itemize}
\begin{itemize}
\item {Proveniência:(Do lat. \textunderscore mensa\textunderscore )}
\end{itemize}
Tudo que pertence a ornatos, alfaias e pertences da mesa, como toalha, loiça, garfos, etc.
\section{Mênstruação}
\begin{itemize}
\item {Grp. gram.:f.}
\end{itemize}
\begin{itemize}
\item {Proveniência:(De \textunderscore menstruo\textunderscore )}
\end{itemize}
O menstruo.
Tempo que dura o fluxo menstrual.
\section{Menstruada}
\begin{itemize}
\item {Grp. gram.:adj. f.}
\end{itemize}
\begin{itemize}
\item {Proveniência:(De \textunderscore mênstruo\textunderscore )}
\end{itemize}
Diz-se da mulher, que está com o mênstruo ou que o tem regularmente.
\section{Menstruado}
\begin{itemize}
\item {Grp. gram.:adj.}
\end{itemize}
\begin{itemize}
\item {Utilização:Des.}
\end{itemize}
Manchado por mênstruo:«\textunderscore ...obras mais immundas, que os panos menstruados\textunderscore ». \textunderscore Luz e Calor\textunderscore , 543.
\section{Menstrual}
\begin{itemize}
\item {Grp. gram.:adj.}
\end{itemize}
\begin{itemize}
\item {Proveniência:(Lat. \textunderscore menstrualis\textunderscore )}
\end{itemize}
Relativo ao mênstruo.
\section{Mênstruo}
\begin{itemize}
\item {Grp. gram.:m.}
\end{itemize}
\begin{itemize}
\item {Proveniência:(Lat. \textunderscore menstruus\textunderscore )}
\end{itemize}
Evacuação sanguínea e periódica, proveniente do útero.
Líquido dissolvente, com que se extrahem de um sólido os princípios activos contidos neste.
\section{Mênsula}
\begin{itemize}
\item {Grp. gram.:f.}
\end{itemize}
\begin{itemize}
\item {Proveniência:(Lat. \textunderscore mensula\textunderscore )}
\end{itemize}
O mesmo que \textunderscore mísula\textunderscore .
\section{Mensura}
\begin{itemize}
\item {Grp. gram.:f.}
\end{itemize}
\begin{itemize}
\item {Utilização:Des.}
\end{itemize}
\begin{itemize}
\item {Utilização:Des.}
\end{itemize}
\begin{itemize}
\item {Proveniência:(Lat. \textunderscore mensura\textunderscore )}
\end{itemize}
O mesmo que \textunderscore medida\textunderscore .

Compasso musical.
\section{Mensurabilidade}
\begin{itemize}
\item {Grp. gram.:f.}
\end{itemize}
\begin{itemize}
\item {Proveniência:(Do lat. \textunderscore mensurabilis\textunderscore )}
\end{itemize}
Qualidade do que é mensurável.
\section{Mensuração}
\begin{itemize}
\item {Grp. gram.:f.}
\end{itemize}
\begin{itemize}
\item {Utilização:Des.}
\end{itemize}
\begin{itemize}
\item {Proveniência:(Lat. \textunderscore mensuratio\textunderscore )}
\end{itemize}
Acto de medir.
\section{Mensurador}
\begin{itemize}
\item {Grp. gram.:adj.}
\end{itemize}
\begin{itemize}
\item {Grp. gram.:M.}
\end{itemize}
Que mensura: \textunderscore a ampulheta mensuradora do tempo\textunderscore .
Funccionário, encarregado da medição o identificação dos criminosos, nos postos anthropométricos.
\section{Mensural}
\begin{itemize}
\item {Grp. gram.:adj.}
\end{itemize}
\begin{itemize}
\item {Utilização:Mús.}
\end{itemize}
Que tem mensura ou compasso.
\section{Mensuralista}
\begin{itemize}
\item {Grp. gram.:m.}
\end{itemize}
\begin{itemize}
\item {Proveniência:(Do lat. \textunderscore mensura\textunderscore )}
\end{itemize}
Compositor musical, na Idade-Média.
\section{Mensurar}
\begin{itemize}
\item {Grp. gram.:v. t.}
\end{itemize}
\begin{itemize}
\item {Proveniência:(Lat. \textunderscore mensurare\textunderscore )}
\end{itemize}
Determinar a medida de; medir.
\section{Mensurável}
\begin{itemize}
\item {Grp. gram.:adj.}
\end{itemize}
\begin{itemize}
\item {Proveniência:(Lat. \textunderscore mensurabilis\textunderscore )}
\end{itemize}
Que se póde medir.
\section{Menta}
\begin{itemize}
\item {Proveniência:(Lat. \textunderscore menta\textunderscore )}
\end{itemize}
\textunderscore f.\textunderscore , (e der.)
O mesmo que \textunderscore mentha\textunderscore , etc.
\section{Mentado}
\begin{itemize}
\item {Grp. gram.:adj.}
\end{itemize}
\begin{itemize}
\item {Proveniência:(De \textunderscore mente\textunderscore . Cp. \textunderscore ementado\textunderscore )}
\end{itemize}
Lembrado, recordado.
\section{Mentagra}
\begin{itemize}
\item {Grp. gram.:f.}
\end{itemize}
\begin{itemize}
\item {Proveniência:(Lat. \textunderscore mentagra\textunderscore )}
\end{itemize}
Impigem na barba.
\section{Mental}
\begin{itemize}
\item {Grp. gram.:adj.}
\end{itemize}
Relativo á mente; intellectual.
Espiritual: oração \textunderscore mental\textunderscore .
\section{Mental}
\begin{itemize}
\item {Grp. gram.:adj.}
\end{itemize}
Relativo ao mento.
\section{Mentalidade}
\begin{itemize}
\item {Grp. gram.:f.}
\end{itemize}
\begin{itemize}
\item {Utilização:Neol.}
\end{itemize}
\begin{itemize}
\item {Proveniência:(De \textunderscore mental\textunderscore )}
\end{itemize}
Qualidade de mental.
A mente.
Movimento intellectual.
Estado psichológico.
\section{Mentalmente}
\begin{itemize}
\item {Grp. gram.:adv.}
\end{itemize}
De modo mental; em espírito; intelectualmente.
\section{Mentar}
\begin{itemize}
\item {Grp. gram.:v. t.}
\end{itemize}
\begin{itemize}
\item {Utilização:Ant.}
\end{itemize}
O mesmo que \textunderscore ementar\textunderscore . Cf. \textunderscore Eufrosina\textunderscore , 302; \textunderscore Aulegrafia\textunderscore , 27.
\section{Mefistofelicamente}
\begin{itemize}
\item {Grp. gram.:adv.}
\end{itemize}
De modo mefistofélico.
Sarcasticamente.
\section{Mefistofélico}
\begin{itemize}
\item {Grp. gram.:adj.}
\end{itemize}
\begin{itemize}
\item {Proveniência:(De \textunderscore Mephistópheles\textunderscore , n. p.)}
\end{itemize}
Próprio de Mephistópheles.
Diabólico.
Sarcástico: \textunderscore riso mefistofélico\textunderscore .
\section{Mefítico}
\begin{itemize}
\item {Grp. gram.:adj.}
\end{itemize}
\begin{itemize}
\item {Proveniência:(Lat. \textunderscore mephiticus\textunderscore )}
\end{itemize}
Que tem exalações nocivas á saúde.
Pestilencial; fétido; podre.
\section{Mefitismo}
\begin{itemize}
\item {Grp. gram.:m.}
\end{itemize}
\begin{itemize}
\item {Proveniência:(Do lat. \textunderscore mephitis\textunderscore )}
\end{itemize}
Qualidade de mefítico.
Doença ou estado mórbido, resultante de exalações mefíticas.
Impaludismo.
\section{Mentastro}
\begin{itemize}
\item {Grp. gram.:m.}
\end{itemize}
Planta synanthérea, medicinal, espécie de hortelan, (\textunderscore ageratum conyzoides\textunderscore , ou \textunderscore mentha rotundifolia\textunderscore , Lin.). (Lat. \textunderscore mentastrum\textunderscore )
\section{Mente}
\begin{itemize}
\item {Grp. gram.:f.}
\end{itemize}
\begin{itemize}
\item {Grp. gram.:Loc. adv.}
\end{itemize}
\begin{itemize}
\item {Grp. gram.:Pl. Loc. adv.}
\end{itemize}
\begin{itemize}
\item {Proveniência:(Lat. \textunderscore mens\textunderscore , \textunderscore mentis\textunderscore )}
\end{itemize}
Intelligência.
Alma, espirito.
Tenção, disposição, imaginação.
Intuito.
\textunderscore De bôa mente\textunderscore , de bôa vontade.
\textunderscore Têr mentes\textunderscore , fazer reparo, dar fé.
\section{Mentecapto}
\begin{itemize}
\item {Grp. gram.:adj.}
\end{itemize}
\begin{itemize}
\item {Proveniência:(Do lat. \textunderscore mens\textunderscore  + \textunderscore captus\textunderscore )}
\end{itemize}
Que não faz uso da razão.
Alienado; idiota; néscio.
\section{Mentes}
\begin{itemize}
\item {Grp. gram.:conj.  e  adv.}
\end{itemize}
\begin{itemize}
\item {Utilização:Ant.}
\end{itemize}
\begin{itemize}
\item {Grp. gram.:M.}
\end{itemize}
\begin{itemize}
\item {Proveniência:(De \textunderscore mente\textunderscore )}
\end{itemize}
Em-quanto; entretanto.
Attenção, consideração: \textunderscore têr em mentes a gravidade da culpa\textunderscore .
\section{Mentes}
\begin{itemize}
\item {Grp. gram.:m.}
\end{itemize}
\begin{itemize}
\item {Proveniência:(De \textunderscore mentir\textunderscore )}
\end{itemize}
Arte de chamar mentiroso. Cf. Herculano, \textunderscore Hist. de Port.\textunderscore , IV, 461.
\section{Mentha}
\begin{itemize}
\item {Grp. gram.:f.}
\end{itemize}
\begin{itemize}
\item {Proveniência:(Lat. \textunderscore mentha\textunderscore , ou \textunderscore menta\textunderscore )}
\end{itemize}
Designação scientifica de várias espécies de hortelan.
\section{Menthol}
\begin{itemize}
\item {Grp. gram.:m.}
\end{itemize}
\begin{itemize}
\item {Proveniência:(Do lat. \textunderscore mentha\textunderscore )}
\end{itemize}
Extracto da essência da hortelan-pimenta.
\section{Mentholado}
\begin{itemize}
\item {Grp. gram.:adj.}
\end{itemize}
Que contém menthol.
\section{Menthólico}
\begin{itemize}
\item {Grp. gram.:adj.}
\end{itemize}
Relativo ao menthol.
\section{Menthylo}
\begin{itemize}
\item {Grp. gram.:m.}
\end{itemize}
\begin{itemize}
\item {Utilização:Chím.}
\end{itemize}
\begin{itemize}
\item {Proveniência:(Do gr. \textunderscore mintha\textunderscore  + \textunderscore khule\textunderscore )}
\end{itemize}
Radical do álcool menthólico.
\section{Mentideiro}
\begin{itemize}
\item {Grp. gram.:adj.}
\end{itemize}
\begin{itemize}
\item {Utilização:Ant.}
\end{itemize}
\begin{itemize}
\item {Grp. gram.:M.}
\end{itemize}
\begin{itemize}
\item {Utilização:Neol.}
\end{itemize}
O mesmo que \textunderscore mentiroso\textunderscore .
Lugar, onde se inventam casos e boatos: \textunderscore diz-se nos mentideiros de Lisbôa que o ministro vai pedir a demissão\textunderscore .
(Cast. \textunderscore mentidero\textunderscore )
\section{Mentido}
\begin{itemize}
\item {Grp. gram.:adj.}
\end{itemize}
\begin{itemize}
\item {Proveniência:(Do lat. \textunderscore mentilus\textunderscore )}
\end{itemize}
Falso; vão: \textunderscore promessas mentidas\textunderscore .
Illusório.
\section{Mentigem}
\begin{itemize}
\item {Grp. gram.:f.}
\end{itemize}
\begin{itemize}
\item {Proveniência:(Do lat. \textunderscore mentigo\textunderscore )}
\end{itemize}
Doença cutânea, espécie de ronha, que dá nos cordeiros.
\section{Mentigo}
\begin{itemize}
\item {Grp. gram.:m.}
\end{itemize}
(V.mentigem)
\section{Mentilo}
\begin{itemize}
\item {Grp. gram.:m.}
\end{itemize}
\begin{itemize}
\item {Utilização:Chím.}
\end{itemize}
\begin{itemize}
\item {Proveniência:(Do gr. \textunderscore mintha\textunderscore  + \textunderscore khule\textunderscore )}
\end{itemize}
Radical do álcool mentólico.
\section{Mentir}
\begin{itemize}
\item {Grp. gram.:v. i.}
\end{itemize}
\begin{itemize}
\item {Utilização:Constr.}
\end{itemize}
\begin{itemize}
\item {Proveniência:(Lat. \textunderscore mentiri\textunderscore )}
\end{itemize}
Apresentar como verdade o que é falsidade.
Errar.
Degenerar.
Faltar a um dever, a um compromisso.
Não têr effeito, falhar: \textunderscore mentiu-me a esperança\textunderscore .
Induzir alguém em êrro ou engano.
Diz-se de uma peça de madeira, que, por êrro de, medida, não entra ou não assenta bem no lugar a que se destina.
\section{Mentira}
\begin{itemize}
\item {Grp. gram.:f.}
\end{itemize}
Acto de mentir.
Fraude; falsidade; engano.
Juízo errado.
Persuasão falsa.
(B. lat. \textunderscore mentira\textunderscore )
\section{Mentirada}
\begin{itemize}
\item {Grp. gram.:f.}
\end{itemize}
O mesmo que \textunderscore mentirola\textunderscore . Cf. Castilho, \textunderscore D. Quixote\textunderscore , I, 23.
\section{Mentireiro}
\begin{itemize}
\item {Grp. gram.:adj.}
\end{itemize}
O mesmo que \textunderscore mentiroso\textunderscore .
\section{Mentirola}
\begin{itemize}
\item {Grp. gram.:f.}
\end{itemize}
\begin{itemize}
\item {Proveniência:(De \textunderscore mentira\textunderscore )}
\end{itemize}
Mentira inoffensiva.
Peta; galga.
\section{Mentirosamente}
\begin{itemize}
\item {Grp. gram.:adv.}
\end{itemize}
De modo mentiroso; com engano.
\section{Mentiroso}
\begin{itemize}
\item {Grp. gram.:adj.}
\end{itemize}
\begin{itemize}
\item {Proveniência:(De \textunderscore mentira\textunderscore )}
\end{itemize}
Que diz ou costuma dizer mentiras.
Opposto á verdade; falso; enganoso.
\section{Mento}
\begin{itemize}
\item {Grp. gram.:m.}
\end{itemize}
\begin{itemize}
\item {Proveniência:(Lat. \textunderscore mentum\textunderscore )}
\end{itemize}
Parte do rosto, correspondente á maxilla inferior.
Queixo.
Saliência carnuda, por baixo do beiço inferior dos animaes.
Cimalha.
\section{Mentol}
\begin{itemize}
\item {Grp. gram.:m.}
\end{itemize}
\begin{itemize}
\item {Proveniência:(Do lat. \textunderscore mentha\textunderscore )}
\end{itemize}
Extracto da essência da hortelan-pimenta.
\section{Mento-labial}
\begin{itemize}
\item {Grp. gram.:adj.}
\end{itemize}
\begin{itemize}
\item {Utilização:Anat.}
\end{itemize}
Diz-se de um músculo, que vai do queixo ao lábio inferior.
\section{Mentolado}
\begin{itemize}
\item {Grp. gram.:adj.}
\end{itemize}
Que contém mentol.
\section{Mentólico}
\begin{itemize}
\item {Grp. gram.:adj.}
\end{itemize}
Relativo ao mentol.
\section{Mentor}
\begin{itemize}
\item {Grp. gram.:m.}
\end{itemize}
\begin{itemize}
\item {Proveniência:(De \textunderscore Mentor\textunderscore , n. p.)}
\end{itemize}
Pessôa, que aconselha ou ensina outra; guia.
\section{Mentraste}
\begin{itemize}
\item {Grp. gram.:m.}
\end{itemize}
(V.mentastro)
\section{Mentrasto}
\begin{itemize}
\item {Grp. gram.:m.}
\end{itemize}
(V.mentastro)
\section{Mentre}
\begin{itemize}
\item {Grp. gram.:conj.}
\end{itemize}
\begin{itemize}
\item {Utilização:Ant.}
\end{itemize}
O mesmo que \textunderscore mentres\textunderscore . Cf. \textunderscore Port. Mon. Hist.\textunderscore , \textunderscore Script.\textunderscore , 254.
\section{Mentres}
\begin{itemize}
\item {Grp. gram.:conj.  e  adv.}
\end{itemize}
\begin{itemize}
\item {Utilização:Ant.}
\end{itemize}
O mesmo que \textunderscore mentes\textunderscore ^1.
(Cp. cast. ant. \textunderscore mientre\textunderscore )
\section{Mentrusto}
\begin{itemize}
\item {Grp. gram.:m.}
\end{itemize}
\begin{itemize}
\item {Utilização:Bras}
\end{itemize}
Planta medicinal.
O mesmo que \textunderscore mentruz\textunderscore ?--Talvêz êrro typográphico da obra de B. C. Rubim, \textunderscore Vocab. Bras.\textunderscore , em vez de \textunderscore mentrasto\textunderscore .
\section{Mentruz}
\begin{itemize}
\item {Grp. gram.:m.}
\end{itemize}
O mesmo que \textunderscore matruz\textunderscore .
\section{Mentual}
\begin{itemize}
\item {Grp. gram.:adj.}
\end{itemize}
\begin{itemize}
\item {Utilização:Anat.}
\end{itemize}
Relativo ao mento.
\section{Mentulagra}
\begin{itemize}
\item {Grp. gram.:f.}
\end{itemize}
\begin{itemize}
\item {Utilização:Med.}
\end{itemize}
\begin{itemize}
\item {Utilização:ant.}
\end{itemize}
\begin{itemize}
\item {Proveniência:(T. hýbr., do lat. \textunderscore mentula\textunderscore  + gr. \textunderscore agra\textunderscore )}
\end{itemize}
Doença no pênis.
\section{Mentzélia}
\begin{itemize}
\item {Grp. gram.:f.}
\end{itemize}
\begin{itemize}
\item {Proveniência:(De \textunderscore Mentzel\textunderscore , n. p.)}
\end{itemize}
Gênero de plantas da América tropical.
\section{Mentzeliáceas}
\begin{itemize}
\item {Grp. gram.:f. pl.}
\end{itemize}
\begin{itemize}
\item {Proveniência:(De \textunderscore mentzeliáceo\textunderscore )}
\end{itemize}
Família de plantas, que tem por typo a mentzélia; loáseas.
\section{Mentzeliáceo}
\begin{itemize}
\item {Grp. gram.:adj.}
\end{itemize}
Relativo ou semelhante á \textunderscore mentzélia\textunderscore .
\section{Menudência}
\begin{itemize}
\item {Grp. gram.:f.}
\end{itemize}
O mesmo ou melhor que \textunderscore minudência\textunderscore . Cf. \textunderscore Luz e Calor\textunderscore , 87.
\section{Menza}
\begin{itemize}
\item {Grp. gram.:f.}
\end{itemize}
\begin{itemize}
\item {Utilização:t. de Lisbôa}
\end{itemize}
\begin{itemize}
\item {Utilização:Ant.}
\end{itemize}
O mesmo que \textunderscore mesa\textunderscore ^1. Cf. \textunderscore Peregrinação\textunderscore , passim; Tenreiro, XIV; etc.
\section{Meo}
\begin{itemize}
\item {Grp. gram.:m.  e  adj.}
\end{itemize}
\begin{itemize}
\item {Utilização:Ant.}
\end{itemize}
O mesmo que \textunderscore meio\textunderscore .
\section{Meógo}
\begin{itemize}
\item {Grp. gram.:m.}
\end{itemize}
\begin{itemize}
\item {Utilização:Ant.}
\end{itemize}
O mesmo que \textunderscore meiagoó\textunderscore .
\section{Meolo}
\begin{itemize}
\item {fónica:ô}
\end{itemize}
\begin{itemize}
\item {Grp. gram.:m.}
\end{itemize}
\begin{itemize}
\item {Utilização:Prov.}
\end{itemize}
\begin{itemize}
\item {Utilização:beir.}
\end{itemize}
Peça central da roda dos carros, o mesmo que \textunderscore meão\textunderscore .
(Cp. \textunderscore meão\textunderscore )
\section{Méor}
\begin{itemize}
\item {Grp. gram.:adj.}
\end{itemize}
\begin{itemize}
\item {Utilização:Ant.}
\end{itemize}
O mesmo que \textunderscore menór\textunderscore : \textunderscore ...era frei Luys de Raz ministro dos frades méores\textunderscore .
\section{Méos}
\begin{itemize}
\item {Grp. gram.:adv.}
\end{itemize}
\begin{itemize}
\item {Utilização:Ant.}
\end{itemize}
O mesmo que \textunderscore menos\textunderscore . Cf. Frei Fortun., \textunderscore Inéd.\textunderscore , 310.
\section{Meótes}
\begin{itemize}
\item {Grp. gram.:m. pl.}
\end{itemize}
\begin{itemize}
\item {Utilização:Prov.}
\end{itemize}
\begin{itemize}
\item {Utilização:açor}
\end{itemize}
\begin{itemize}
\item {Proveniência:(De \textunderscore meia\textunderscore )}
\end{itemize}
O mesmo que [[peúgas|peúga]].
\section{Mephistophelicamente}
\begin{itemize}
\item {Grp. gram.:adv.}
\end{itemize}
De modo mephistophélico.
Sarcasticamente.
\section{Mephistophélico}
\begin{itemize}
\item {Grp. gram.:adj.}
\end{itemize}
\begin{itemize}
\item {Proveniência:(De \textunderscore Mephistópheles\textunderscore , n. p.)}
\end{itemize}
Próprio de Mephistópheles.
Diabólico.
Sarcástico: \textunderscore riso mephistophélico\textunderscore .
\section{Mephítico}
\begin{itemize}
\item {Grp. gram.:adj.}
\end{itemize}
\begin{itemize}
\item {Proveniência:(Lat. \textunderscore mephiticus\textunderscore )}
\end{itemize}
Que tem exhalações nocivas á saúde.
Pestilencial; fétido; podre.
\section{Mephitismo}
\begin{itemize}
\item {Grp. gram.:m.}
\end{itemize}
\begin{itemize}
\item {Proveniência:(Do lat. \textunderscore mephitis\textunderscore )}
\end{itemize}
Qualidade de mephítico.
Doença ou estado mórbido, resultante de exhalações mephíticas.
Impaludismo.
\section{Meporis}
\begin{itemize}
\item {Grp. gram.:m. pl.}
\end{itemize}
Indígenas do norte do Brasil.
O mesmo que \textunderscore mepurus\textunderscore ? Cf. Lour. Amazonas, \textunderscore Diccion. Topogr.\textunderscore 
\section{Mepurus}
\begin{itemize}
\item {Grp. gram.:m. pl.}
\end{itemize}
Tribo de aborígenes do Pará.
\section{Mequens}
\begin{itemize}
\item {Grp. gram.:m. pl.}
\end{itemize}
Tribo de índios, nas margens do rio Mequém, ao norte de Mato-Grosso, no Brasil.
\section{Mequetrefe}
\begin{itemize}
\item {Grp. gram.:m.}
\end{itemize}
\begin{itemize}
\item {Utilização:Chul.}
\end{itemize}
\begin{itemize}
\item {Proveniência:(Do ár. \textunderscore moiatrefe\textunderscore , petulante)}
\end{itemize}
Indivíduo metediço, que intervém onde não é chamado.
\section{Méra}
\begin{itemize}
\item {Grp. gram.:f.}
\end{itemize}
Líquido medicamentoso, proveniente da destillação do zimbro.
\section{Mêra}
\begin{itemize}
\item {Grp. gram.:f.}
\end{itemize}
\begin{itemize}
\item {Utilização:Prov.}
\end{itemize}
\begin{itemize}
\item {Utilização:trasm.}
\end{itemize}
Resina das árvores.
(Alter, phonética de \textunderscore méra\textunderscore ?)
\section{Meramente}
\begin{itemize}
\item {Grp. gram.:adv.}
\end{itemize}
\begin{itemize}
\item {Proveniência:(De \textunderscore mero\textunderscore )}
\end{itemize}
Simplesmente; unicamente.
\section{Merapinina}
\begin{itemize}
\item {Grp. gram.:f.}
\end{itemize}
\begin{itemize}
\item {Utilização:Bras}
\end{itemize}
Árvore silvestre, de madeira listrada.
\section{Merarchia}
\begin{itemize}
\item {fónica:qui}
\end{itemize}
\begin{itemize}
\item {Grp. gram.:f.}
\end{itemize}
\begin{itemize}
\item {Proveniência:(Do gr. \textunderscore meros\textunderscore  + \textunderscore arkhe\textunderscore )}
\end{itemize}
Formatura de 2048 homens ou duas chiliarchias, na phalange macedónica.
\section{Meraró}
\begin{itemize}
\item {Grp. gram.:m.}
\end{itemize}
\begin{itemize}
\item {Utilização:Bras}
\end{itemize}
Planta medicinal.
\section{Merarquia}
\begin{itemize}
\item {Grp. gram.:f.}
\end{itemize}
\begin{itemize}
\item {Proveniência:(Do gr. \textunderscore meros\textunderscore  + \textunderscore arkhe\textunderscore )}
\end{itemize}
Formatura de 2048 homens ou duas quiliarquias, na falange macedónica.
\section{Merca}
\begin{itemize}
\item {Grp. gram.:f.}
\end{itemize}
\begin{itemize}
\item {Utilização:Pop.}
\end{itemize}
Acto de mercar.
Aquillo que se mercou: \textunderscore trouxe da feira muitas mercas\textunderscore .
\section{Mercaço}
\begin{itemize}
\item {Grp. gram.:m.}
\end{itemize}
Uma das divisórias, nas armações fixas da pesca.
\section{Mercadeiro}
\begin{itemize}
\item {Grp. gram.:m.}
\end{itemize}
\begin{itemize}
\item {Utilização:Ant.}
\end{itemize}
O mesmo que \textunderscore mercador\textunderscore .
\section{Mercadejar}
\begin{itemize}
\item {Grp. gram.:v. i.}
\end{itemize}
\begin{itemize}
\item {Proveniência:(De \textunderscore mercado\textunderscore )}
\end{itemize}
Commerciar.
Traficar; auferir ganhos illícitos.
\section{Mercadejável}
\begin{itemize}
\item {Grp. gram.:adj.}
\end{itemize}
Que se póde \textunderscore mercadejar\textunderscore . Cf. \textunderscore Techn. Rur.\textunderscore , 467.
\section{Mercado}
\begin{itemize}
\item {Grp. gram.:m.}
\end{itemize}
\begin{itemize}
\item {Proveniência:(Do lat. \textunderscore mercatus\textunderscore )}
\end{itemize}
Lugar, onde se vendem gêneros alimentícios e outros.
Povoação, em que há grande movimento commercial.
Centro de commércio.
O commércio: \textunderscore o mercado dos livros afroixa no verão\textunderscore .
\section{Mercador}
\begin{itemize}
\item {Grp. gram.:m.}
\end{itemize}
\begin{itemize}
\item {Proveniência:(Do lat. \textunderscore mercator\textunderscore )}
\end{itemize}
Aquelle que merca, para vender a retalho.
Negociante de panos.
\section{Mercadoria}
\begin{itemize}
\item {Grp. gram.:f.}
\end{itemize}
\begin{itemize}
\item {Proveniência:(De \textunderscore mercador\textunderscore )}
\end{itemize}
Profissão de mercador.
Aquillo que é o objecto de compra e venda.
Aquillo que se comprou e que se expõe á venda.
\section{Merca-honra}
\begin{itemize}
\item {Grp. gram.:m.  e  f.}
\end{itemize}
O mesmo ou melhor que \textunderscore merca-honras\textunderscore . Cf. Garrett, \textunderscore Filippa\textunderscore , 36.
\section{Merca-honras}
\begin{itemize}
\item {Grp. gram.:m.  e  f.}
\end{itemize}
Pessôa, que trafica com a honra alheia.
\section{Mercancia}
\begin{itemize}
\item {Grp. gram.:f.}
\end{itemize}
O mesmo que \textunderscore mercadoria\textunderscore .
Acto de mercanciar.
(Cp. \textunderscore mercante\textunderscore )
\section{Mercanciar}
\begin{itemize}
\item {Grp. gram.:v. i.}
\end{itemize}
\begin{itemize}
\item {Proveniência:(De \textunderscore mercancia\textunderscore )}
\end{itemize}
O mesmo que \textunderscore mercadejar\textunderscore .
\section{Mercante}
\begin{itemize}
\item {Grp. gram.:m.}
\end{itemize}
\begin{itemize}
\item {Grp. gram.:adj.}
\end{itemize}
\begin{itemize}
\item {Proveniência:(De \textunderscore mercar\textunderscore )}
\end{itemize}
Mercador.
Relativo ao commércio ou ao movimento commercial: \textunderscore marinha mercante\textunderscore .
\section{Mercantel}
\begin{itemize}
\item {Grp. gram.:m.}
\end{itemize}
\begin{itemize}
\item {Utilização:T. de Aveiro}
\end{itemize}
\begin{itemize}
\item {Proveniência:(De \textunderscore mercante\textunderscore )}
\end{itemize}
Negociante de sardinha, que, estabelecido em palheiro, á beira da ria, na Torreira, a compra aos pescadores e a transporta em bateira sua para Aveiro e Pardelhas.
\section{Mercantil}
\begin{itemize}
\item {Grp. gram.:adj.}
\end{itemize}
\begin{itemize}
\item {Utilização:Fig.}
\end{itemize}
\begin{itemize}
\item {Proveniência:(De \textunderscore mercante\textunderscore )}
\end{itemize}
Relativo a mercadores ou a mercadorias:«\textunderscore naus mercantis...\textunderscore »Filinto. \textunderscore D. Man.\textunderscore , II, 35.
Que pratíca o commércio.
Interesseiro.
Ambicioso.
\section{Mercantilagem}
\begin{itemize}
\item {Grp. gram.:f.}
\end{itemize}
O mesmo que \textunderscore mercantilismo\textunderscore . Cf. Camillo, \textunderscore Narcót.\textunderscore , II, 10.
\section{Mercantilidade}
\begin{itemize}
\item {Grp. gram.:f.}
\end{itemize}
Qualidade de mercantil.
\section{Mercantilismo}
\begin{itemize}
\item {Grp. gram.:m.}
\end{itemize}
\begin{itemize}
\item {Proveniência:(De \textunderscore mercantil\textunderscore )}
\end{itemize}
Propensão para subordinar tudo ao commércio, ao interesse, ao ganho.
Predomínio do interesse ou do espírito mercantil.
\section{Mercantilizar}
\begin{itemize}
\item {Grp. gram.:v. i.}
\end{itemize}
\begin{itemize}
\item {Utilização:Neol.}
\end{itemize}
\begin{itemize}
\item {Proveniência:(De \textunderscore mercantil\textunderscore )}
\end{itemize}
Fazer transacções mercantis.
Exercer o commércio.
\section{Mercantilmente}
\begin{itemize}
\item {Grp. gram.:adv.}
\end{itemize}
De modo mercantil.
\section{Mercantismo}
\begin{itemize}
\item {Grp. gram.:m.}
\end{itemize}
(V.mercantilismo)
\section{Mercar}
\begin{itemize}
\item {Grp. gram.:v. t.}
\end{itemize}
\begin{itemize}
\item {Utilização:Bras. da Baía}
\end{itemize}
\begin{itemize}
\item {Utilização:Fig.}
\end{itemize}
\begin{itemize}
\item {Proveniência:(Lat. \textunderscore mercari\textunderscore )}
\end{itemize}
Comprar para vender.
Adquirir por dinheiro, comprar.
Apregoar para vender: \textunderscore vão mercando laranjas, ali na rua\textunderscore .
Conseguir com trabalho.
\section{Mercar}
\begin{itemize}
\item {Grp. gram.:m.}
\end{itemize}
Antiga medida da África oriental portuguesa.
\section{Mercaria}
\begin{itemize}
\item {Grp. gram.:f.}
\end{itemize}
Profissão de mercador.
Depósito de mercadorias.
(De \textunderscore mercar\textunderscore ^1).
\section{Mercatório}
\begin{itemize}
\item {Grp. gram.:adj.}
\end{itemize}
\begin{itemize}
\item {Proveniência:(Lat. \textunderscore mervatorius\textunderscore )}
\end{itemize}
O mesmo que \textunderscore mercantil\textunderscore .
\section{Merca-tudo}
\begin{itemize}
\item {Grp. gram.:m.  e  adj.}
\end{itemize}
O que faz negócio de tudo; ferro-velho.
\section{Mercadura}
\begin{itemize}
\item {Grp. gram.:f.}
\end{itemize}
\begin{itemize}
\item {Utilização:Des.}
\end{itemize}
\begin{itemize}
\item {Proveniência:(Lat. \textunderscore mercatura\textunderscore )}
\end{itemize}
Arte de commerciar.
Commércio, negócio.
\section{Mercável}
\begin{itemize}
\item {Grp. gram.:adj.}
\end{itemize}
\begin{itemize}
\item {Proveniência:(De \textunderscore mercar\textunderscore )}
\end{itemize}
Que se póde mercar.
Que se póde commerciar; vendível.
\section{Mercer}
\begin{itemize}
\item {Grp. gram.:f.}
\end{itemize}
\begin{itemize}
\item {Utilização:Des.}
\end{itemize}
\begin{itemize}
\item {Proveniência:(Do lat. \textunderscore merx\textunderscore , \textunderscore mercis\textunderscore )}
\end{itemize}
O mesmo que \textunderscore mercadoria\textunderscore . Cf. Filinto, IX, 224.
\section{Mercê}
\begin{itemize}
\item {Grp. gram.:f.}
\end{itemize}
\begin{itemize}
\item {Proveniência:(Do lat. \textunderscore mercer\textunderscore , \textunderscore mercedis\textunderscore )}
\end{itemize}
Aquillo que se dá ou paga, em retribuição de um trabalho.
Provimento num cargo público.
Concessão de título honorífico.
Benefício; favor: \textunderscore faça-me a mercê de me ouvir\textunderscore .
Benignidade.
Indulto.
Arbítrio, capricho.
\textunderscore Vossa mercê\textunderscore , tratamento que se dava a pessôa de ceremónia, e que depois se contrahiu tratamento vulgar \textunderscore vossemecê\textunderscore .
\section{Mercéa}
\begin{itemize}
\item {Grp. gram.:f.}
\end{itemize}
\begin{itemize}
\item {Utilização:Ant.}
\end{itemize}
O mesmo que \textunderscore mercê\textunderscore . Cf. G. Vicente, \textunderscore Inês Pereira\textunderscore .
\section{Merceano}
\begin{itemize}
\item {Grp. gram.:m.}
\end{itemize}
\begin{itemize}
\item {Utilização:Ant.}
\end{itemize}
\begin{itemize}
\item {Proveniência:(De \textunderscore mercê\textunderscore )}
\end{itemize}
Serventuário; indivídu assalariado.
\section{Mercearia}
\begin{itemize}
\item {Grp. gram.:f.}
\end{itemize}
\begin{itemize}
\item {Proveniência:(De \textunderscore mercê\textunderscore )}
\end{itemize}
Commércio de pouco valor, ou loja onde se faz este commércio, (des. neste sentido).
Loja, em que se vendem géneros alimentícios e quaesquer especiarias.
\section{Mercearia}
\begin{itemize}
\item {Grp. gram.:f.}
\end{itemize}
\begin{itemize}
\item {Utilização:Ant.}
\end{itemize}
\begin{itemize}
\item {Utilização:Ant.}
\end{itemize}
\begin{itemize}
\item {Proveniência:(De \textunderscore mercê\textunderscore )}
\end{itemize}
Obrigação de fazer certas obras ou práticas religiosas por alma de algum defunto ou pela conservação de alguém.
Albergaria, asilo.
\section{Mercedes!}
\begin{itemize}
\item {fónica:cê}
\end{itemize}
\begin{itemize}
\item {Grp. gram.:interj.}
\end{itemize}
\begin{itemize}
\item {Utilização:Prov.}
\end{itemize}
\begin{itemize}
\item {Utilização:minh.}
\end{itemize}
Viva muitos annos! Deus lhe dê saúde!
(Contr. de \textunderscore merecêdes\textunderscore , forma ant. de \textunderscore mereceis\textunderscore ? Ou relaciona-se com o cast. \textunderscore mercedes\textunderscore , mercês?)
\section{Merceeira}
\begin{itemize}
\item {Grp. gram.:f.}
\end{itemize}
Mulher de merceeiro.
Dona de mercearia ou tenda.
\section{Merceeira}
\begin{itemize}
\item {Grp. gram.:f.}
\end{itemize}
Mulher que, fazendo parte de uma communidade, recebia certa pensão ou moradia, obrigando-se a certos encargos espirituaes.
(De \textunderscore merceeiro\textunderscore ^2).
\section{Merceeiro}
\begin{itemize}
\item {Grp. gram.:m.}
\end{itemize}
\begin{itemize}
\item {Proveniência:(De \textunderscore mercê\textunderscore )}
\end{itemize}
Individuo, que tem mercearia ou tenda; tendeiro.
\section{Merceeiro}
\begin{itemize}
\item {Grp. gram.:m.}
\end{itemize}
\begin{itemize}
\item {Utilização:Ant.}
\end{itemize}
\begin{itemize}
\item {Proveniência:(De \textunderscore mercê\textunderscore )}
\end{itemize}
Indivíduo, a quem se dava pensão ou casa, com certos encargos espirituaes.
\section{Mercenário}
\begin{itemize}
\item {Grp. gram.:adj.}
\end{itemize}
\begin{itemize}
\item {Grp. gram.:M.}
\end{itemize}
\begin{itemize}
\item {Proveniência:(Lat. \textunderscore mercenarius\textunderscore )}
\end{itemize}
Que trabalha por soldada ou estipêndio.
Interesseiro.
Aquelle que serve ou trabalha por estipêndio.
Frade da Ordem da Senhora das Mercês.
\section{Mercenarismo}
\begin{itemize}
\item {Grp. gram.:m.}
\end{itemize}
\begin{itemize}
\item {Proveniência:(De \textunderscore mercenário\textunderscore )}
\end{itemize}
Espírito mercenário ou interesseiro.
\section{Merceologia}
\begin{itemize}
\item {Grp. gram.:f.}
\end{itemize}
\begin{itemize}
\item {Proveniência:(Do lat. \textunderscore merx\textunderscore , \textunderscore mercis\textunderscore  + gr. \textunderscore logos\textunderscore )}
\end{itemize}
Parte da sciência do commércio, que se occupa especialmente da compra e venda.
\section{Merchandia}
\begin{itemize}
\item {Grp. gram.:f.}
\end{itemize}
\begin{itemize}
\item {Utilização:Ant.}
\end{itemize}
O mesmo que \textunderscore mercadoria\textunderscore .
(Cp. \textunderscore merchante\textunderscore )
\section{Merchante}
\begin{itemize}
\item {Grp. gram.:m.}
\end{itemize}
\begin{itemize}
\item {Utilização:Ant.}
\end{itemize}
Aquelle que vende carne no açougue; marchante. Cf. Fern. Mendes Pinto, \textunderscore Peregr.\textunderscore , CVII.
(Provavelmente, do lat. \textunderscore mercans\textunderscore )
\section{Mércia}
\begin{itemize}
\item {Grp. gram.:f.}
\end{itemize}
\begin{itemize}
\item {Utilização:Chul.}
\end{itemize}
\begin{itemize}
\item {Proveniência:(De \textunderscore mercê\textunderscore )}
\end{itemize}
Negócio ou trato clandestino.
\section{Mercieira}
\begin{itemize}
\item {Grp. gram.:f.}
\end{itemize}
(V. \textunderscore merceeira\textunderscore ^1)
\section{Mercieiro}
\begin{itemize}
\item {Grp. gram.:m.}
\end{itemize}
(V. \textunderscore merceeiro\textunderscore ^1)
\section{Mercurial}
\begin{itemize}
\item {Grp. gram.:adj.}
\end{itemize}
\begin{itemize}
\item {Grp. gram.:M.}
\end{itemize}
\begin{itemize}
\item {Grp. gram.:F.}
\end{itemize}
\begin{itemize}
\item {Utilização:Fam.}
\end{itemize}
\begin{itemize}
\item {Proveniência:(Lat. \textunderscore mercurialis\textunderscore )}
\end{itemize}
Que é composto de mercúrio.
Medicamento, em que entra o mercúrio.
Planta euphorbiácea, (\textunderscore mercurialis amam\textunderscore ).
Reprehensão.
\section{Mercurialismo}
\begin{itemize}
\item {Grp. gram.:m.}
\end{itemize}
\begin{itemize}
\item {Proveniência:(De \textunderscore mercurial\textunderscore )}
\end{itemize}
Estado mórbido, resultante do abuso do mercúrio.
\section{Mercurializar}
\begin{itemize}
\item {Grp. gram.:v. t.}
\end{itemize}
\begin{itemize}
\item {Proveniência:(De \textunderscore mercurial\textunderscore )}
\end{itemize}
Causar mercurialismo a.
\section{Mercúrio}
\begin{itemize}
\item {Grp. gram.:m.}
\end{itemize}
\begin{itemize}
\item {Utilização:Fig.}
\end{itemize}
\begin{itemize}
\item {Proveniência:(De \textunderscore Mercúrio\textunderscore , n. p.)}
\end{itemize}
Substância metállica, fluida na temperatura ordinária, também conhecida por azougue.
Planeta, o mais próximo do Sol.
Medianeiro de negócios amorosos, mensageiro de amor.
Preparação pharmacêutica, em que entra o mercúrio.
\section{Mercurioso}
\begin{itemize}
\item {Grp. gram.:adj.}
\end{itemize}
\begin{itemize}
\item {Utilização:Pharm.}
\end{itemize}
Em que entra mercúrio; composto com mercúrio.
\section{Mercuroso}
\begin{itemize}
\item {Grp. gram.:adj.}
\end{itemize}
Forma incorrecta, por \textunderscore mercurioso\textunderscore . Cf. \textunderscore Pharmacopeia Port.\textunderscore 
\section{Mercuzan}
\begin{itemize}
\item {Grp. gram.:f.}
\end{itemize}
\begin{itemize}
\item {Utilização:Ant.}
\end{itemize}
O mesmo que \textunderscore medrozan\textunderscore .
\section{Merda}
\begin{itemize}
\item {Grp. gram.:f.}
\end{itemize}
\begin{itemize}
\item {Utilização:Pleb.}
\end{itemize}
\begin{itemize}
\item {Utilização:Ext.}
\end{itemize}
\begin{itemize}
\item {Grp. gram.:Interj.}
\end{itemize}
\begin{itemize}
\item {Proveniência:(Lat. \textunderscore merda\textunderscore )}
\end{itemize}
Matérias fecaes.
Fezes, que o intestino expelle normalmente pelo ânus.
Excremento.
Dejecto.
Porcaria.
(design. de repulsão ou desprezo, em conversações plebeias)
\section{Merdalha}
\begin{itemize}
\item {Grp. gram.:f.}
\end{itemize}
\begin{itemize}
\item {Utilização:Pleb.}
\end{itemize}
\begin{itemize}
\item {Proveniência:(De \textunderscore merda\textunderscore )}
\end{itemize}
Gente ordinária, escória, ralé.
\section{Merdeiro}
\begin{itemize}
\item {Grp. gram.:adj.}
\end{itemize}
\begin{itemize}
\item {Utilização:Pleb.}
\end{itemize}
\begin{itemize}
\item {Grp. gram.:M.}
\end{itemize}
\begin{itemize}
\item {Utilização:T. de Aveiro}
\end{itemize}
\begin{itemize}
\item {Utilização:T. da Bairrada}
\end{itemize}
Relativo a merda. Cf. G. Vicente, I, 224.
Aquelle que negocia em excremento humano, para adubos.
Espécie de moscardo, que vive nos excrementos.
\section{Merdice}
\begin{itemize}
\item {Grp. gram.:f.}
\end{itemize}
\begin{itemize}
\item {Utilização:Pleb.}
\end{itemize}
\begin{itemize}
\item {Proveniência:(De \textunderscore merda\textunderscore )}
\end{itemize}
Coisa suja ou de pouco valor; porcaria.
Acção indigna.
\section{Merdícola}
\begin{itemize}
\item {Grp. gram.:adj.}
\end{itemize}
\begin{itemize}
\item {Utilização:Zool.}
\end{itemize}
\begin{itemize}
\item {Proveniência:(Do lat. \textunderscore merda\textunderscore  + \textunderscore colere\textunderscore )}
\end{itemize}
Que constróe o ninho com excremento de bêstas.
\section{Merdilheiro}
\begin{itemize}
\item {Grp. gram.:m.}
\end{itemize}
\begin{itemize}
\item {Utilização:Prov.}
\end{itemize}
\begin{itemize}
\item {Utilização:Pleb.}
\end{itemize}
Homem nojento, biltre.
Rapaz sujo e malcriado.
(Por \textunderscore merdalheiro\textunderscore , de \textunderscore merdalha\textunderscore )
\section{Merdívoro}
\begin{itemize}
\item {Grp. gram.:adj.}
\end{itemize}
\begin{itemize}
\item {Proveniência:(Do lat. \textunderscore merda\textunderscore  + \textunderscore vorare\textunderscore )}
\end{itemize}
Diz-se dos insectos, que se nutrem de excrementos.
\section{Mere}
\begin{itemize}
\item {Grp. gram.:m.}
\end{itemize}
\begin{itemize}
\item {Utilização:Neol.}
\end{itemize}
Primeiro official municipal nas communas francesas. Cf. Castilho, \textunderscore Colloq. Ald.\textunderscore , 390.
(Aportuguesamento do fr. \textunderscore maire\textunderscore , do lat. \textunderscore major\textunderscore )
\section{Merecedor}
\begin{itemize}
\item {Grp. gram.:adj.}
\end{itemize}
Que merece alguma coisa; digno.
\section{Merecer}
\begin{itemize}
\item {Grp. gram.:v. t.}
\end{itemize}
\begin{itemize}
\item {Utilização:T. de Turquel}
\end{itemize}
\begin{itemize}
\item {Grp. gram.:V. i.}
\end{itemize}
\begin{itemize}
\item {Proveniência:(Do lat. \textunderscore merere\textunderscore )}
\end{itemize}
Sêr digno de.
Têr direito a: \textunderscore merecer elogio\textunderscore .
Granjear.
Pagar com serviços: \textunderscore o operário comprou um saco de milho ao fazendeiro, com a condição de lho ir merecer\textunderscore .
Tornar-se merecedor.
\section{Merecidamente}
\begin{itemize}
\item {Grp. gram.:adv.}
\end{itemize}
De modo merecido; com justiça; sem favor.
\section{Merecido}
\begin{itemize}
\item {Grp. gram.:adj.}
\end{itemize}
\begin{itemize}
\item {Proveniência:(De \textunderscore merecer\textunderscore )}
\end{itemize}
Devido; justo: \textunderscore louvor merecido\textunderscore .
\section{Merecimento}
\begin{itemize}
\item {Grp. gram.:m.}
\end{itemize}
\begin{itemize}
\item {Proveniência:(De \textunderscore merecer\textunderscore )}
\end{itemize}
Qualidade que torna alguém digno de premio ou castigo.
Qualidade, que torna alguém digno de aprêço: \textunderscore êste rapaz tem merecimento\textunderscore .
Importância, superioridade.
Habilitações.
\section{Merênchyma}
\begin{itemize}
\item {fónica:qui}
\end{itemize}
\begin{itemize}
\item {Grp. gram.:m.}
\end{itemize}
\begin{itemize}
\item {Utilização:Bot.}
\end{itemize}
\begin{itemize}
\item {Proveniência:(Do gr. \textunderscore meros\textunderscore  + \textunderscore enkhuma\textunderscore )}
\end{itemize}
Variedade de tecido utricular vegetal, caracterizado pela forma espheroidal e pela froixa união dos utrículos constituintes.
\section{Merencório}
\begin{itemize}
\item {Grp. gram.:adj.}
\end{itemize}
(Fórma des. de \textunderscore melancólico\textunderscore )
\section{Merenda}
\begin{itemize}
\item {Grp. gram.:f.}
\end{itemize}
\begin{itemize}
\item {Utilização:Bras}
\end{itemize}
\begin{itemize}
\item {Proveniência:(Lat. \textunderscore merenda\textunderscore )}
\end{itemize}
Ligeira refeição entre o jantar e a ceia.
Comezaina fóra de horas, de noite.
Foro antigo, que os caseiros pagavam aos senhorios, quando tomavam conta dos prazos.
\section{Merendal}
\begin{itemize}
\item {Grp. gram.:m.}
\end{itemize}
\begin{itemize}
\item {Utilização:Ant.}
\end{itemize}
Metade de um bragal, ou medida de três varas e meia.
Espécie de pano ordinário.
O mesmo que \textunderscore merenda\textunderscore .
\section{Merendar}
\begin{itemize}
\item {Grp. gram.:v. t.}
\end{itemize}
\begin{itemize}
\item {Grp. gram.:V. i.}
\end{itemize}
\begin{itemize}
\item {Proveniência:(Do b. lat. \textunderscore merendare\textunderscore )}
\end{itemize}
Comer á hora da merenda: \textunderscore merendar laranjas\textunderscore .
Comer a merenda: \textunderscore já merendaste\textunderscore ?
\section{Merendeira}
\begin{itemize}
\item {Grp. gram.:f.}
\end{itemize}
Pão pequeno, próprio para merendas.
\section{Merendeiro}
\begin{itemize}
\item {Grp. gram.:m.}
\end{itemize}
\begin{itemize}
\item {Grp. gram.:Adj.}
\end{itemize}
\begin{itemize}
\item {Proveniência:(Do lat. \textunderscore merendarius\textunderscore )}
\end{itemize}
Merendeira.
Pessôa, que merenda habitualmente.
Pedaço de barro, com que se fabrica cada telha.
Cesto, em que vai a merenda para fóra de casa.
Diz-se do cesto, em que se leva a merenda, e do pão pequeno, destinado a merenda.
\section{Merendera}
\begin{itemize}
\item {Grp. gram.:f.}
\end{itemize}
Gênero de plantas colchicáceas.
(Cast. \textunderscore merendera\textunderscore )
\section{Merendéreas}
\begin{itemize}
\item {Grp. gram.:f. pl.}
\end{itemize}
\begin{itemize}
\item {Proveniência:(De \textunderscore merendera\textunderscore )}
\end{itemize}
O mesmo que \textunderscore colchicáceas\textunderscore .
\section{Merendiba}
\begin{itemize}
\item {Grp. gram.:f.}
\end{itemize}
Planta silvestre do Brasil, (\textunderscore terminalia merendiba\textunderscore ).
\section{Merendona}
\begin{itemize}
\item {Grp. gram.:f.}
\end{itemize}
\begin{itemize}
\item {Utilização:Fam.}
\end{itemize}
Grande merenda.
\section{Merenducar}
\begin{itemize}
\item {Grp. gram.:v. i.}
\end{itemize}
\begin{itemize}
\item {Utilização:Prov.}
\end{itemize}
\begin{itemize}
\item {Utilização:trasm.}
\end{itemize}
Comer a merenda.
\section{Merengue}
\begin{itemize}
\item {Grp. gram.:m.}
\end{itemize}
Bolo de claras de ovos com açúcar.
\section{Merênquima}
\begin{itemize}
\item {Grp. gram.:m.}
\end{itemize}
\begin{itemize}
\item {Utilização:Bot.}
\end{itemize}
\begin{itemize}
\item {Proveniência:(Do gr. \textunderscore meros\textunderscore  + \textunderscore enkhuma\textunderscore )}
\end{itemize}
Variedade de tecido utricular vegetal, caracterizado pela forma esferoidal e pela froixa união dos utrículos constituintes.
\section{Merepe}
\begin{itemize}
\item {Grp. gram.:m.}
\end{itemize}
\begin{itemize}
\item {Utilização:Bras. do N}
\end{itemize}
Jôgo de cartas, em que se arrisca pouco dinheiro.
\section{Meretrice}
\begin{itemize}
\item {Grp. gram.:f.}
\end{itemize}
\begin{itemize}
\item {Utilização:Ant.}
\end{itemize}
O mesmo que \textunderscore meretriz\textunderscore . Cf. Vieira, IX, 251 e 259.
\section{Meretrício}
\begin{itemize}
\item {Grp. gram.:adj.}
\end{itemize}
\begin{itemize}
\item {Grp. gram.:M.}
\end{itemize}
\begin{itemize}
\item {Utilização:Neol.}
\end{itemize}
\begin{itemize}
\item {Proveniência:(Lat. \textunderscore meretricius\textunderscore )}
\end{itemize}
Relativo a meretriz.
Profissão de meretriz; prostituição.
\section{Meretrícula}
\begin{itemize}
\item {Grp. gram.:f.}
\end{itemize}
\begin{itemize}
\item {Proveniência:(Lat. \textunderscore meretricula\textunderscore )}
\end{itemize}
Meretriz não adulta.
\section{Meretriz}
\begin{itemize}
\item {Grp. gram.:f.}
\end{itemize}
\begin{itemize}
\item {Proveniência:(Lat. \textunderscore meretrix\textunderscore )}
\end{itemize}
Mulher pública.
Rameira; marafona.
\section{Merganso}
\begin{itemize}
\item {Grp. gram.:m.}
\end{itemize}
\begin{itemize}
\item {Proveniência:(Do lat. \textunderscore mergus\textunderscore  + al. \textunderscore gans\textunderscore )}
\end{itemize}
Ave palmípede, (\textunderscore mergus serrator\textunderscore , Lin.).
\section{Mergulha}
\begin{itemize}
\item {Grp. gram.:f.}
\end{itemize}
\begin{itemize}
\item {Utilização:Prov.}
\end{itemize}
\begin{itemize}
\item {Proveniência:(De \textunderscore mergulhar\textunderscore )}
\end{itemize}
O mesmo que \textunderscore mergulhia\textunderscore .
\section{Mergulhador}
\begin{itemize}
\item {Grp. gram.:m.  e  adj.}
\end{itemize}
\begin{itemize}
\item {Grp. gram.:M.}
\end{itemize}
\begin{itemize}
\item {Utilização:bras}
\end{itemize}
O que mergulha.
Homem, que trabalha debaixo de água.
Pescador de pérolas.
Ave, semelhante á gaivota.
\section{Mergulhante}
\begin{itemize}
\item {Grp. gram.:adj.}
\end{itemize}
\begin{itemize}
\item {Utilização:Bras}
\end{itemize}
\begin{itemize}
\item {Utilização:Neol.}
\end{itemize}
Que mergulha.
\section{Mergulhão}
\begin{itemize}
\item {Grp. gram.:m.}
\end{itemize}
\begin{itemize}
\item {Grp. gram.:Adj.}
\end{itemize}
\begin{itemize}
\item {Proveniência:(De \textunderscore mergulhar\textunderscore )}
\end{itemize}
Vara das videiras, que se mergulha na terra, ficando a ponta de fóra.
Mergulhia.
Ave palmípede, (\textunderscore colymbus\textunderscore ).
Ave pernalta (\textunderscore podiceps\textunderscore ).
Que mergulha (falando-se do uma espécie de ganso). Cf. Filinto, XIII, 273.
\section{Mergulhar}
\begin{itemize}
\item {Grp. gram.:v. t.}
\end{itemize}
\begin{itemize}
\item {Utilização:Ext.}
\end{itemize}
\begin{itemize}
\item {Grp. gram.:V. i.}
\end{itemize}
\begin{itemize}
\item {Utilização:Ext.}
\end{itemize}
\begin{itemize}
\item {Utilização:Fig.}
\end{itemize}
\begin{itemize}
\item {Proveniência:(Do lat. hyp. \textunderscore merguliare\textunderscore , de \textunderscore mergulus\textunderscore , dem. de \textunderscore mergus\textunderscore )}
\end{itemize}
Meter debaixo da água.
Afundar na água.
Entranhar.
Enterrar (o mergulhão, vide).
Occultar-se dentro da água; afundar-se nella; immergir.
Entranhar-se.
Desapparecer.
\section{Mergulhia}
\begin{itemize}
\item {Grp. gram.:f.}
\end{itemize}
Acto de enterrar o mergulhão da vide, para reproducção da videira; mergulhão.
(Do \textunderscore mergulhar\textunderscore )
\section{Mergulho}
\begin{itemize}
\item {Grp. gram.:m.}
\end{itemize}
\begin{itemize}
\item {Grp. gram.:Loc. adv.}
\end{itemize}
Acto de mergulhar.
Mergulhão, ave.
O mesmo que \textunderscore mergulhia\textunderscore  (de vide).
\textunderscore De mergulho\textunderscore , mergulhando, indo ao fundo da água.
\section{Mèria}
\begin{itemize}
\item {Grp. gram.:f.}
\end{itemize}
\begin{itemize}
\item {Utilização:Neol.}
\end{itemize}
Repartição ou casa, onde funcciona o mere. Cf. Castilho, \textunderscore Colloq. Ald\textunderscore ., 390.
(Aportuguesamento do fr. \textunderscore mairie\textunderscore ).
\section{Mericarpo}
\begin{itemize}
\item {Grp. gram.:m.}
\end{itemize}
\begin{itemize}
\item {Utilização:Bot.}
\end{itemize}
\begin{itemize}
\item {Proveniência:(Do gr. \textunderscore meros\textunderscore  + \textunderscore karpos\textunderscore )}
\end{itemize}
Parte de um fruto, separada naturalmente no sentido longitudinal e contendo uma só semente. Cf. De-Candolle.
\section{Merícico}
\begin{itemize}
\item {Grp. gram.:adj.}
\end{itemize}
\begin{itemize}
\item {Utilização:Hist. Nat.}
\end{itemize}
Relativo á mastigação dos alimentos que voltaram do estômago á boca.
(Cp. \textunderscore mericismo\textunderscore )
\section{Mericismo}
\begin{itemize}
\item {Grp. gram.:m.}
\end{itemize}
\begin{itemize}
\item {Proveniência:(Gr. \textunderscore merukismos\textunderscore )}
\end{itemize}
Estado mórbido, em que os alimentos voltam do estômago á boca, para de novo serem mastigados, o que é facto normal entre os ruminantes.
\section{Meridiana}
\begin{itemize}
\item {Grp. gram.:f.}
\end{itemize}
\begin{itemize}
\item {Utilização:Bras}
\end{itemize}
\begin{itemize}
\item {Proveniência:(De \textunderscore meridiano\textunderscore )}
\end{itemize}
Intersecção do plano do meridiano com o plano do horizonte, ou com outro plano qualquer.
Relógio de sol.
O mesmo que \textunderscore sesta\textunderscore .
\section{Meridiano}
\begin{itemize}
\item {Grp. gram.:m.}
\end{itemize}
\begin{itemize}
\item {Utilização:Ant.}
\end{itemize}
\begin{itemize}
\item {Grp. gram.:Adj.}
\end{itemize}
\begin{itemize}
\item {Proveniência:(Lat. \textunderscore meridianus\textunderscore )}
\end{itemize}
Círculo máximo, que passa pelos polos, pelo zenithe e pelo nadir, e corta o equador em ângulos rectos.
Gladiador, que combatia á hora do meio-dia.
Relativo ao meridiano; relativo ao meio-dia.
Diz-se da luneta, que serve para observações meridianas.
\section{Merídio}
\begin{itemize}
\item {Grp. gram.:adj.}
\end{itemize}
\begin{itemize}
\item {Proveniência:(Do lat. \textunderscore meridies\textunderscore )}
\end{itemize}
Relativo ao meio-dia.
Meridional.
\section{Meridional}
\begin{itemize}
\item {Grp. gram.:adj.}
\end{itemize}
\begin{itemize}
\item {Grp. gram.:M.}
\end{itemize}
\begin{itemize}
\item {Proveniência:(Lat. \textunderscore meridionalis\textunderscore )}
\end{itemize}
Que está do lado do Sul.
Habitante das regiões do Sul.
\section{Merídios}
\begin{itemize}
\item {Grp. gram.:m. pl.}
\end{itemize}
\begin{itemize}
\item {Proveniência:(Do gr. \textunderscore meris\textunderscore  + \textunderscore eidos\textunderscore )}
\end{itemize}
Segmentos, mais ou menos heterónomos, em que se póde dividir o corpo de um animal, e resultantes da reunião de plastídios.
\section{Merifela}
\begin{itemize}
\item {Grp. gram.:f.}
\end{itemize}
Espécie de melro, (\textunderscore monticola cyanus\textunderscore  ou \textunderscore turdus cyanus\textunderscore , Lin.).
\section{Merinaque}
\begin{itemize}
\item {Grp. gram.:m.}
\end{itemize}
\begin{itemize}
\item {Proveniência:(Do cast. \textunderscore miriñaque\textunderscore )}
\end{itemize}
Saia, enfunada por arcos ou varas flexíveis; saia de balão.
\section{Merindiba}
\begin{itemize}
\item {Grp. gram.:f.}
\end{itemize}
\begin{itemize}
\item {Utilização:Bras}
\end{itemize}
O mesmo que \textunderscore merendiba\textunderscore .
\section{Meringalha}
\begin{itemize}
\item {Grp. gram.:f.}
\end{itemize}
\begin{itemize}
\item {Utilização:Prov.}
\end{itemize}
\begin{itemize}
\item {Utilização:trasm.}
\end{itemize}
O mesmo que \textunderscore bengala\textunderscore .
\section{Merino}
\begin{itemize}
\item {Grp. gram.:adj.}
\end{itemize}
\begin{itemize}
\item {Grp. gram.:M.}
\end{itemize}
Relativo á raça dos carneiros, chamados merinos.
Espécie de carneiro espanhol, cuja lan é muito fina e apreciada.
Tecido dessa lan.
(Cast. \textunderscore merino\textunderscore )
\section{Merinó}
\begin{itemize}
\item {Grp. gram.:m.}
\end{itemize}
\begin{itemize}
\item {Utilização:Bras}
\end{itemize}
O mesmo que \textunderscore merino\textunderscore .
\section{Merió}
\begin{itemize}
\item {Grp. gram.:m.}
\end{itemize}
Árvore da Índia portuguesa.
\section{Merioba}
\begin{itemize}
\item {Grp. gram.:f.}
\end{itemize}
\begin{itemize}
\item {Utilização:Bras}
\end{itemize}
Planta medicinal.
\section{Merisma}
\begin{itemize}
\item {Grp. gram.:m.}
\end{itemize}
\begin{itemize}
\item {Proveniência:(Do gr. \textunderscore meros\textunderscore )}
\end{itemize}
Divisão de um assumpto em partes distintas.
\section{Merismático}
\begin{itemize}
\item {Grp. gram.:adj.}
\end{itemize}
\begin{itemize}
\item {Proveniência:(Do gr. \textunderscore merisma\textunderscore )}
\end{itemize}
Diz-se da multiplicação ou reproducção, que se realiza pela divisão das céllulas ou dos organismos.
\section{Meristema}
\begin{itemize}
\item {Grp. gram.:m.}
\end{itemize}
\begin{itemize}
\item {Utilização:Bot.}
\end{itemize}
Tecido vivo, não differenciado ainda.
(Palavra mal formada, do gr. \textunderscore merizein\textunderscore  + \textunderscore stema\textunderscore )
\section{Merital}
\begin{itemize}
\item {Grp. gram.:m.}
\end{itemize}
\begin{itemize}
\item {Utilização:Bot.}
\end{itemize}
\begin{itemize}
\item {Proveniência:(Do gr. \textunderscore meris\textunderscore  + \textunderscore thallos\textunderscore )}
\end{itemize}
Distância entre os nós das plantas ou entre duas inserções de fôlhas num ramo.
\section{Meritalo}
\begin{itemize}
\item {Grp. gram.:m.}
\end{itemize}
O mesmo que \textunderscore merital\textunderscore .
\section{Mèritamente}
\begin{itemize}
\item {Grp. gram.:adv.}
\end{itemize}
O mesmo que \textunderscore merecidamente\textunderscore . Cf. Pant. de Aveiro, \textunderscore Itiner.\textunderscore , 46 v.^1 e 236, (2.^a ed.).
\section{Merithal}
\begin{itemize}
\item {Grp. gram.:m.}
\end{itemize}
\begin{itemize}
\item {Utilização:Bot.}
\end{itemize}
\begin{itemize}
\item {Proveniência:(Do gr. \textunderscore meris\textunderscore  + \textunderscore thallos\textunderscore )}
\end{itemize}
Distância entre os nós das plantas ou entre duas inserções de fôlhas num ramo.
\section{Merithallo}
\begin{itemize}
\item {Grp. gram.:m.}
\end{itemize}
O mesmo que \textunderscore merithal\textunderscore .
\section{Meriti}
\begin{itemize}
\item {Grp. gram.:m.}
\end{itemize}
\begin{itemize}
\item {Utilização:Bras. do N}
\end{itemize}
Espécie de palmeira, muito vulgar no valle do Amazona.
\section{Meritíssimo}
\begin{itemize}
\item {Grp. gram.:adj.}
\end{itemize}
\begin{itemize}
\item {Proveniência:(Lat. \textunderscore meritisimus\textunderscore )}
\end{itemize}
Muito digno.
\section{Mérito}
\begin{itemize}
\item {Grp. gram.:m.}
\end{itemize}
\begin{itemize}
\item {Proveniência:(Lat. \textunderscore meritum\textunderscore )}
\end{itemize}
Merecimento; aptidão; superioridade.
Bom serviço no desempenho de quaesquer funcções.
\section{Meritoriamente}
\begin{itemize}
\item {Grp. gram.:adv.}
\end{itemize}
De modo meritório; com mérito.
\section{Meritório}
\begin{itemize}
\item {Grp. gram.:adv.}
\end{itemize}
\begin{itemize}
\item {Proveniência:(Lat. \textunderscore meritorius\textunderscore )}
\end{itemize}
Digno de louvor.
Que merece prêmio.
\section{Meritoso}
\begin{itemize}
\item {Grp. gram.:adj.}
\end{itemize}
\begin{itemize}
\item {Utilização:P. us.}
\end{itemize}
Em que há mérito; que merece apreço ou gabos.
\section{Merlão}
\begin{itemize}
\item {Grp. gram.:m.}
\end{itemize}
\begin{itemize}
\item {Proveniência:(Fr. \textunderscore merlon\textunderscore )}
\end{itemize}
Intervallo dentado, nas ameias de uma fortaleza.
\section{Merlim}
\begin{itemize}
\item {Grp. gram.:m.}
\end{itemize}
\begin{itemize}
\item {Utilização:Fig.}
\end{itemize}
\begin{itemize}
\item {Utilização:Des.}
\end{itemize}
\begin{itemize}
\item {Proveniência:(Fr. \textunderscore merlin\textunderscore )}
\end{itemize}
Mialhar, com que se forram os cabos de navios.
Qualquer tecido ralo e engomado, como a tarlatana.
Espertalhão.
Espécie de maço ou martelo, com que se abatiam os bois no matadoiro.
Machado para partir lenha.
\section{Merlo}
\begin{itemize}
\item {Grp. gram.:m.}
\end{itemize}
\begin{itemize}
\item {Proveniência:(Do lat. \textunderscore merulus\textunderscore )}
\end{itemize}
O mesmo que \textunderscore melro\textunderscore .
\section{Merma}
\begin{itemize}
\item {Grp. gram.:f.}
\end{itemize}
\begin{itemize}
\item {Utilização:Prov.}
\end{itemize}
\begin{itemize}
\item {Utilização:alg.}
\end{itemize}
\begin{itemize}
\item {Grp. gram.:Pl.}
\end{itemize}
\begin{itemize}
\item {Utilização:Açor}
\end{itemize}
Peixe, semelhante ao bonito, mas impróprio para a alimentação.
Coisas sem valor, pequenas coisas.
(Talvez de \textunderscore mermar\textunderscore )
\section{Mermar}
\begin{itemize}
\item {Grp. gram.:v. t.}
\end{itemize}
\begin{itemize}
\item {Utilização:Des.}
\end{itemize}
\begin{itemize}
\item {Proveniência:(Do lat. hyp. \textunderscore minimare\textunderscore , seg. Car. Michaëlis)}
\end{itemize}
Tornar mais pequeno; deminuir.
Cercear.
\section{Mero}
\begin{itemize}
\item {Grp. gram.:adj.}
\end{itemize}
\begin{itemize}
\item {Grp. gram.:M.}
\end{itemize}
\begin{itemize}
\item {Proveniência:(Lat. \textunderscore merus\textunderscore )}
\end{itemize}
Simples; genuíno; sem mistura: \textunderscore mentir por mera diversão\textunderscore .
Peixe percoide, (\textunderscore seranus gigas\textunderscore ).
\section{Meroblástico}
\begin{itemize}
\item {Grp. gram.:adj.}
\end{itemize}
\begin{itemize}
\item {Utilização:Physiol.}
\end{itemize}
Diz-se do ovo, em que o vitello de nutrição, muito abundante, está separado do vitello de formação, nos pássaros, peixes,etc.
\section{Merocele}
\begin{itemize}
\item {Grp. gram.:m.}
\end{itemize}
\begin{itemize}
\item {Proveniência:(Do gr. \textunderscore meros\textunderscore  + \textunderscore kele\textunderscore )}
\end{itemize}
Hérnia crural.
\section{Merologia}
\begin{itemize}
\item {Grp. gram.:f.}
\end{itemize}
\begin{itemize}
\item {Proveniência:(Do gr. \textunderscore meros\textunderscore  + \textunderscore logos\textunderscore )}
\end{itemize}
Tratado elementar.
\section{Merougo}
\begin{itemize}
\item {Grp. gram.:m.}
\end{itemize}
\begin{itemize}
\item {Utilização:Prov.}
\end{itemize}
\begin{itemize}
\item {Utilização:trasm.}
\end{itemize}
Pessôa immunda.
Pessôa envergonhada.
\section{Mertolengo}
\begin{itemize}
\item {Grp. gram.:adj.}
\end{itemize}
\begin{itemize}
\item {Proveniência:(De \textunderscore Mértola\textunderscore , n. p.)}
\end{itemize}
Criado no concelho de Mértola (falando-se de gado). Cf. A. Baganha, \textunderscore Hyg. Pec.\textunderscore , 205 e 206.
\section{Merotomia}
\begin{itemize}
\item {Grp. gram.:f.}
\end{itemize}
\begin{itemize}
\item {Utilização:Physiol.}
\end{itemize}
\begin{itemize}
\item {Proveniência:(Do gr. \textunderscore meros\textunderscore  + \textunderscore tome\textunderscore )}
\end{itemize}
Secção de uma céllula viva, para se estudarem as modificações ulteriores dos fragmentos.
\section{Merouço}
\begin{itemize}
\item {Grp. gram.:m.}
\end{itemize}
\begin{itemize}
\item {Utilização:Prov.}
\end{itemize}
\begin{itemize}
\item {Utilização:minh.}
\end{itemize}
Quantidade de excrementos de gente.
(Por moroiço)
\section{Merosena}
\begin{itemize}
\item {fónica:csê}
\end{itemize}
\begin{itemize}
\item {Grp. gram.:f.}
\end{itemize}
Espécie de biotite, de côr verde.
\section{Meru}
\begin{itemize}
\item {Grp. gram.:m.}
\end{itemize}
Animal, semelhante a um burro, mas com cornos e patas fendidas, do qual fala a \textunderscore Ethiópia Oriental\textunderscore .
\section{Meru}
\begin{itemize}
\item {Grp. gram.:m.}
\end{itemize}
Planta amomácea do Brasil.
\section{Meruanha}
\begin{itemize}
\item {Grp. gram.:f.}
\end{itemize}
\begin{itemize}
\item {Utilização:Bras. do N}
\end{itemize}
Môsca da estação invernosa.
\section{Meruf}
\begin{itemize}
\item {Grp. gram.:m.}
\end{itemize}
O mesmo que \textunderscore maruí\textunderscore . Cf. Rui Barb., \textunderscore Réplica\textunderscore , 158.
\section{Merufo}
\begin{itemize}
\item {Grp. gram.:m.}
\end{itemize}
\begin{itemize}
\item {Utilização:Prov.}
\end{itemize}
\begin{itemize}
\item {Utilização:trasm.}
\end{itemize}
Topête de rapazote presumido.
\section{Meruge}
\begin{itemize}
\item {Grp. gram.:f.}
\end{itemize}
O mesmo que \textunderscore murugem\textunderscore .
\section{Merugem}
\begin{itemize}
\item {Grp. gram.:f.}
\end{itemize}
O mesmo que \textunderscore murugem\textunderscore .
\section{Meruja}
\begin{itemize}
\item {Grp. gram.:f.}
\end{itemize}
\begin{itemize}
\item {Utilização:Prov.}
\end{itemize}
\begin{itemize}
\item {Utilização:trasm.}
\end{itemize}
Chuvisco; acto de \textunderscore merujar\textunderscore .
\section{Merujar}
\begin{itemize}
\item {Grp. gram.:v. i.}
\end{itemize}
\begin{itemize}
\item {Utilização:Prov.}
\end{itemize}
\begin{itemize}
\item {Utilização:trasm.}
\end{itemize}
\begin{itemize}
\item {Grp. gram.:V. t.}
\end{itemize}
\begin{itemize}
\item {Utilização:Prov.}
\end{itemize}
\begin{itemize}
\item {Utilização:beir.}
\end{itemize}
Chuviscar.
Regar com água permanente, o mesmo que \textunderscore limar\textunderscore ^3.
\section{Merujinha}
\begin{itemize}
\item {Grp. gram.:f.}
\end{itemize}
\begin{itemize}
\item {Utilização:Prov.}
\end{itemize}
\begin{itemize}
\item {Utilização:trasm.}
\end{itemize}
\begin{itemize}
\item {Proveniência:(De \textunderscore meruja\textunderscore )}
\end{itemize}
Chuva miúda, chuvisco.
\section{Mérula}
\begin{itemize}
\item {Grp. gram.:f.}
\end{itemize}
\begin{itemize}
\item {Proveniência:(Lat. \textunderscore merula\textunderscore )}
\end{itemize}
Antiga máquina hydráulica, em que o movimento da água produzia sons semelhantes á voz do melro.
\section{Mérula}
\begin{itemize}
\item {Grp. gram.:f.}
\end{itemize}
\begin{itemize}
\item {Utilização:Prov.}
\end{itemize}
O mesmo que \textunderscore melro\textunderscore .
\section{Merunhar}
\begin{itemize}
\item {Grp. gram.:v. i.}
\end{itemize}
\begin{itemize}
\item {Utilização:T. do Fundão}
\end{itemize}
O mesmo que \textunderscore merujar\textunderscore .
\section{Meruxinga}
\begin{itemize}
\item {Grp. gram.:f.}
\end{itemize}
\begin{itemize}
\item {Utilização:Bras. do N}
\end{itemize}
Espécie de môsca pequena.
\section{Merýcico}
\begin{itemize}
\item {Grp. gram.:adj.}
\end{itemize}
\begin{itemize}
\item {Utilização:Hist. Nat.}
\end{itemize}
Relativo á mastigação dos alimentos que voltaram do estômago á boca.
(Cp. \textunderscore merycismo\textunderscore )
\section{Merycismo}
\begin{itemize}
\item {Grp. gram.:m.}
\end{itemize}
\begin{itemize}
\item {Proveniência:(Gr. \textunderscore merukismos\textunderscore )}
\end{itemize}
Estado mórbido, em que os alimentos voltam do estômago á boca, para de novo serem mastigados, o que é facto normal entre os ruminantes.
\section{Mês}
\begin{itemize}
\item {Grp. gram.:m.}
\end{itemize}
\begin{itemize}
\item {Proveniência:(Do lat. \textunderscore mensis\textunderscore )}
\end{itemize}
Uma das doze partes do anno, cada uma das quaes tem 30 ou 31 dias, excepto Fevereiro, que tem 28 nos annos ordinários e 29 nos bissextos.
Espaço de 30 dias.
Mênstruo.
Soldada, que se paga mensalmente.
\section{Mesa}
\begin{itemize}
\item {Grp. gram.:f.}
\end{itemize}
\begin{itemize}
\item {Utilização:Fig.}
\end{itemize}
\begin{itemize}
\item {Proveniência:(Do lat. \textunderscore mensa\textunderscore )}
\end{itemize}
Prancha ou pranchas, sustentadas por pés e com várias applicações.
Superfície lisa e horizontal.
Conjunto do presidente e secretários de uma assembleia.
Conjunto de indivíduos que dirigem uma associação.
Quantia ou bolo, que se põe na mesa, para sêr levantado pelo jogador que ganha.
Parte superior dos fechos da espingarda, em que bate a boca do cão.
Espaço plano, em que se empilha barro amassado.
Leito (de um carro).
Grade ou altar (para communhão).
Alimentação: \textunderscore tem alli cama e mesa\textunderscore .
Modo como se vive, relativamente a alimentos.
\textunderscore Mesa de pé de gallo\textunderscore , mesa, geralmente redonda, com um só pé central, tripartido na base.
\section{Mesa}
\begin{itemize}
\item {Grp. gram.:f.}
\end{itemize}
\begin{itemize}
\item {Utilização:Ant.}
\end{itemize}
Vara de videira.
\section{Mesada}
\begin{itemize}
\item {Grp. gram.:f.}
\end{itemize}
Quantia, que se paga em cada mês ou de mês a mês.
\section{Mesagra}
\begin{itemize}
\item {Grp. gram.:f.}
\end{itemize}
O mesmo que \textunderscore bisagra\textunderscore . Cf. \textunderscore Viriato Trág.\textunderscore , XV, 25.
\section{Mesânculo}
\begin{itemize}
\item {Grp. gram.:m.}
\end{itemize}
\begin{itemize}
\item {Proveniência:(Lat. \textunderscore mesanculon\textunderscore )}
\end{itemize}
Lança antiga, no meio da qual se prendia uma correia.
\section{Mesaraico}
\begin{itemize}
\item {Grp. gram.:adj.}
\end{itemize}
\begin{itemize}
\item {Proveniência:(Gr. \textunderscore mesaraikos\textunderscore )}
\end{itemize}
O mesmo que \textunderscore mesentérico\textunderscore .
\section{Mesário}
\begin{itemize}
\item {Grp. gram.:m.}
\end{itemize}
\begin{itemize}
\item {Proveniência:(Do lat. \textunderscore mensarius\textunderscore )}
\end{itemize}
Membro da mesa de uma corporação, especialmente nas confrarias.
\section{Mesaticefalia}
\begin{itemize}
\item {Grp. gram.:f.}
\end{itemize}
Qualidade de mesaticéfalo.
\section{Mesaticéfalo}
\begin{itemize}
\item {Grp. gram.:adj.}
\end{itemize}
\begin{itemize}
\item {Proveniência:(Do gr. \textunderscore mesatos\textunderscore  + \textunderscore kephale\textunderscore )}
\end{itemize}
Diz-se do crânio que, pela sua configuração, occupa o meio termo entre o dolichocéfalo e o braquicéfalo.
\section{Mesaticephalia}
\begin{itemize}
\item {Grp. gram.:f.}
\end{itemize}
Qualidade de mesaticéphalo.
\section{Mesaticéphalo}
\begin{itemize}
\item {Grp. gram.:adj.}
\end{itemize}
\begin{itemize}
\item {Proveniência:(Do gr. \textunderscore mesatos\textunderscore  + \textunderscore kephale\textunderscore )}
\end{itemize}
Diz-se do crânio que, pela sua configuração, occupa o meio termo entre o dolichocéphalo e o brachycéphalo.
\section{Mesáulio}
\begin{itemize}
\item {Grp. gram.:m.}
\end{itemize}
\begin{itemize}
\item {Proveniência:(Lat. \textunderscore mesaulos\textunderscore )}
\end{itemize}
Pátio, entre dois corpos de um edifício grego, ou entre dois muros.
\section{Mesaulo}
\begin{itemize}
\item {Grp. gram.:m.}
\end{itemize}
\begin{itemize}
\item {Proveniência:(Lat. \textunderscore mesaulos\textunderscore )}
\end{itemize}
Pátio, entre dois corpos de um edifício grego, ou entre dois muros.
\section{Mẽscabar}
\begin{itemize}
\item {Grp. gram.:v. i.}
\end{itemize}
\begin{itemize}
\item {Utilização:Des.}
\end{itemize}
O mesmo que \textunderscore menoscabar\textunderscore :« \textunderscore ...só o podia deslustrar e mẽscabar...\textunderscore »Sousa, \textunderscore Vida do Arceb.\textunderscore , II, 46.
\section{Mescal}
\begin{itemize}
\item {Grp. gram.:m.}
\end{itemize}
Cacto mexicano, pequeno e espinhoso, que, comido, produz visões extraordinárias. Cf. \textunderscore Jorn. do Comm.\textunderscore , do Rio, de 9-1-903.
\section{Mescambilha}
\begin{itemize}
\item {Grp. gram.:f.}
\end{itemize}
\begin{itemize}
\item {Utilização:Prov.}
\end{itemize}
Tramóia, trapaça.
Intriga.
\section{Mescambilheiro}
\begin{itemize}
\item {Grp. gram.:m.  e  adj.}
\end{itemize}
\begin{itemize}
\item {Utilização:Prov.}
\end{itemize}
\begin{itemize}
\item {Utilização:T. do Fundão}
\end{itemize}
Mexeriqueiro.
Intriguista.
O mesmo que \textunderscore biqueiro\textunderscore .
\section{Mescão}
\begin{itemize}
\item {Grp. gram.:m.}
\end{itemize}
\begin{itemize}
\item {Utilização:Ant.}
\end{itemize}
Indivíduo de raça mista ou de má raça. Cf. Arn. Gama, \textunderscore Bailio\textunderscore , 48.
(Cp. \textunderscore mescar\textunderscore )
\section{Mescar}
\begin{itemize}
\item {Grp. gram.:v. t.}
\end{itemize}
\begin{itemize}
\item {Utilização:Ant.}
\end{itemize}
Mesclar, misturar.
(Cast. \textunderscore mescar\textunderscore )
\section{Méscia}
\begin{itemize}
\item {Grp. gram.:f.}
\end{itemize}
Peça do lagar de azeite, que empurra a azeitona para o caminho da galga.
\section{Mescla}
\begin{itemize}
\item {Grp. gram.:f.}
\end{itemize}
\begin{itemize}
\item {Utilização:Fig.}
\end{itemize}
\begin{itemize}
\item {Proveniência:(De \textunderscore mesclar\textunderscore )}
\end{itemize}
Coisa mesclada.
Impurezas.
Tecido de várias côres.
Mistura de tintas variegadas.
Agrupamento.
\section{Mesclado}
\begin{itemize}
\item {Grp. gram.:adj.}
\end{itemize}
Variegado; misturado, amalgamado.
\section{Mesclar}
\begin{itemize}
\item {Grp. gram.:v. t.}
\end{itemize}
\begin{itemize}
\item {Proveniência:(Do b. lat. \textunderscore misculare\textunderscore )}
\end{itemize}
Misturar; amalgamar; encorporar; ligar.
Misturar (o sangue) pelo casamento de pessôas de raças diversas.
\section{Mesello}
\begin{itemize}
\item {Grp. gram.:adj.}
\end{itemize}
\begin{itemize}
\item {Utilização:Prov.}
\end{itemize}
\begin{itemize}
\item {Utilização:alg.}
\end{itemize}
\begin{itemize}
\item {Proveniência:(Do lat. \textunderscore misellus\textunderscore )}
\end{itemize}
Triste; que tem ar compungido.
\section{Meselo}
\begin{itemize}
\item {Grp. gram.:adj.}
\end{itemize}
\begin{itemize}
\item {Utilização:Prov.}
\end{itemize}
\begin{itemize}
\item {Utilização:alg.}
\end{itemize}
\begin{itemize}
\item {Proveniência:(Do lat. \textunderscore misellus\textunderscore )}
\end{itemize}
Triste; que tem ar compungido.
\section{Mesembriantêmeas}
\begin{itemize}
\item {Grp. gram.:f. pl.}
\end{itemize}
\begin{itemize}
\item {Proveniência:(De \textunderscore mesembriantémeo\textunderscore )}
\end{itemize}
Família de plantas, que tem por tipo o género ficóide.
\section{Mesembriantêmeo}
\begin{itemize}
\item {Grp. gram.:adj.}
\end{itemize}
Relativo ou semelhante ao mesembriântemo.
\section{Mesembriântemo}
\begin{itemize}
\item {Grp. gram.:m.}
\end{itemize}
\begin{itemize}
\item {Proveniência:(Do gr. \textunderscore mesembria\textunderscore  + \textunderscore anthema\textunderscore )}
\end{itemize}
Designação científica do género ficóide.
\section{Mesembriantéreas}
\begin{itemize}
\item {Grp. gram.:f. pl.}
\end{itemize}
É assim que a \textunderscore Glossotogia Botânica\textunderscore  de Benevides chama as mesembriantêmeas. Se não é erro, há substituição propositada do gr. \textunderscore anthema\textunderscore  por \textunderscore antheros\textunderscore .
\section{Mesembrianthêmeas}
\begin{itemize}
\item {Grp. gram.:f. pl.}
\end{itemize}
\begin{itemize}
\item {Proveniência:(De \textunderscore mesembrianthémeo\textunderscore )}
\end{itemize}
Família de plantas, que tem por typo o género ficóide.
\section{Mesembrianthêmeo}
\begin{itemize}
\item {Grp. gram.:adj.}
\end{itemize}
Relativo ou semelhante ao mesembriânthemo.
\section{Mesembriânthemo}
\begin{itemize}
\item {Grp. gram.:m.}
\end{itemize}
\begin{itemize}
\item {Proveniência:(Do gr. \textunderscore mesembria\textunderscore  + \textunderscore anthema\textunderscore )}
\end{itemize}
Designação scientífica do género ficóide.
\section{Mesembryanthéreas}
\begin{itemize}
\item {Grp. gram.:f. pl.}
\end{itemize}
É assim que a \textunderscore Glossotogia Botânica\textunderscore  de Benevides chama as mesembryanthêmeas. Se não é erro, há substituição propositada do gr. \textunderscore anthema\textunderscore  por \textunderscore antheros\textunderscore .
\section{Mesentérico}
\begin{itemize}
\item {Grp. gram.:adj.}
\end{itemize}
Relativo ao mesentério.
Que ataca o mesentério ou que se manifesta no mesentério: \textunderscore tísica mesentérica\textunderscore .
\section{Mesentério}
\begin{itemize}
\item {Grp. gram.:m.}
\end{itemize}
\begin{itemize}
\item {Proveniência:(Gr. \textunderscore mesenterion\textunderscore )}
\end{itemize}
Membrana, em que estão suspensos os intestinos, e que lhes dá mobilidade.
\section{Mesenterite}
\begin{itemize}
\item {Grp. gram.:f.}
\end{itemize}
Inflammação do mesentério.
\section{Meseta}
\begin{itemize}
\item {fónica:zê}
\end{itemize}
\begin{itemize}
\item {Grp. gram.:f.}
\end{itemize}
\begin{itemize}
\item {Utilização:Geogr.}
\end{itemize}
\begin{itemize}
\item {Utilização:Geol.}
\end{itemize}
\begin{itemize}
\item {Proveniência:(De \textunderscore mesa\textunderscore )}
\end{itemize}
Pequeno planalto.
A parte da península ibérica, que está emersa desde o princípio do período secundário.
\section{Mésico}
\begin{itemize}
\item {Grp. gram.:adj.}
\end{itemize}
\begin{itemize}
\item {Utilização:P. us.}
\end{itemize}
O mesmo que \textunderscore mesológico\textunderscore . Cf. Júl. Ribeiro, \textunderscore Estudos Philol.\textunderscore 
\section{Mesma}
\begin{itemize}
\item {Grp. gram.:f.}
\end{itemize}
\begin{itemize}
\item {Proveniência:(De \textunderscore mesmo\textunderscore )}
\end{itemize}
O mesmo estado, as mesmas circunstancias.
Estado daquillo ou daquelle que não soffreu alterações: \textunderscore o doente continua na mesma\textunderscore .
Us. também na loc. pop. \textunderscore na mesma da hora\textunderscore , no mesmo instante. Cf. Lobo, \textunderscore Auto do Nascimento\textunderscore .
\section{Mesmamente}
\begin{itemize}
\item {Grp. gram.:adv.}
\end{itemize}
\begin{itemize}
\item {Proveniência:(De \textunderscore mesmo\textunderscore )}
\end{itemize}
Do mesmo modo; sem alteração; identicamente.
\section{Mesmeidade}
\begin{itemize}
\item {Grp. gram.:f.}
\end{itemize}
Qualidade daquillo ou daquelle que é o mesmo que outro.
Qualidade do que é idêntico, identidade. Cf. Camillo, \textunderscore Esqueleto\textunderscore , 85.
\section{Mesmeriano}
\begin{itemize}
\item {Grp. gram.:adj.}
\end{itemize}
\begin{itemize}
\item {Grp. gram.:M.}
\end{itemize}
Relativo ao mesmerismo.
Sectário do mesmerismo.
\section{Mesmérico}
\begin{itemize}
\item {Grp. gram.:adj.}
\end{itemize}
O mesmo que \textunderscore mesmeriano\textunderscore .
\section{Mesmerismo}
\begin{itemize}
\item {Grp. gram.:m.}
\end{itemize}
\begin{itemize}
\item {Proveniência:(De \textunderscore Mesmer\textunderscore , n. p.)}
\end{itemize}
Doutrina de Mesmer sôbre o magnetismo animal; magnetismo animal.
\section{Mesmerista}
\begin{itemize}
\item {Grp. gram.:m.}
\end{itemize}
Partidário do mesmerismo.
\section{Mesmice}
\begin{itemize}
\item {Grp. gram.:f.}
\end{itemize}
\begin{itemize}
\item {Utilização:Neol.}
\end{itemize}
\begin{itemize}
\item {Proveniência:(De \textunderscore mesmo\textunderscore )}
\end{itemize}
Qualidade daquelle ou daquillo que em tudo é o mesmo que outro.
Falta de variedade. Cf. Eça de Queirós, no periódico \textunderscore Illustração\textunderscore .
\section{Mesmissimamente}
\begin{itemize}
\item {Grp. gram.:adv.}
\end{itemize}
De modo mesmíssimo.
Sem differença ou alteração nenhuma.
\section{Mesmíssimo}
\begin{itemize}
\item {Grp. gram.:adj.}
\end{itemize}
Que é perfeitamente o mesmo; absolutamente idêntico.
\section{Mesmo}
\begin{itemize}
\item {Grp. gram.:adj.}
\end{itemize}
\begin{itemize}
\item {Grp. gram.:M.}
\end{itemize}
\begin{itemize}
\item {Grp. gram.:Adv.}
\end{itemize}
\begin{itemize}
\item {Proveniência:(Do lat. hyp. \textunderscore semetipsissimus\textunderscore , que, por haplologia, daria \textunderscore mesissimus\textunderscore , seg. Cornu)}
\end{itemize}
Que é como outra coisa; idêntico.
Que não é outro.
Semelhante.
Que não soffreu alteração: \textunderscore estás sempre o mesmo rapaz\textunderscore .
Que é o proprio: \textunderscore êste é mesmo, de que falámos\textunderscore .
A mesma coisa: \textunderscore a mim succedeu-me o mesmo\textunderscore .
Aquillo que é indifferente, ou que não importa: \textunderscore mintas ou não mintas, isso é o mesmo\textunderscore .
Com exactidão; precisamente; até: \textunderscore mesmo depois de morto...\textunderscore --Há quem duvide da vernaculidade dêste adv.; Filinto porém não o enjeitou:«\textunderscore ...qualquer coisa, mesmo a ti nociva.\textunderscore »Filinto, II, 73; VIII, 60; XVII, 201;
XIX, 182 e 218; XIX, 54, 68 e 121; Latino, \textunderscore Hist. Pol.\textunderscore , I, 43; Rebello, \textunderscore Mocidade\textunderscore , etc.
\section{Mesnada}
\begin{itemize}
\item {Grp. gram.:f.}
\end{itemize}
\begin{itemize}
\item {Utilização:Ant.}
\end{itemize}
\begin{itemize}
\item {Utilização:Prov.}
\end{itemize}
\begin{itemize}
\item {Utilização:beir.}
\end{itemize}
Porção de soldados assalariados.
Tropa mercenária.
Porção de comida, que alguém leva furtada ou contra a vontade de algum dos donos da casa donde sai. (Colhido no Fundão)
(B. lat. \textunderscore mesnada\textunderscore )
\section{Mesnadaria}
\begin{itemize}
\item {Grp. gram.:f.}
\end{itemize}
\begin{itemize}
\item {Proveniência:(De \textunderscore mesnada\textunderscore )}
\end{itemize}
O sôldo do mesnadeiro.
\section{Mesnadeiro}
\begin{itemize}
\item {Grp. gram.:m.}
\end{itemize}
\begin{itemize}
\item {Proveniência:(Do b. lat. \textunderscore maisnadarius\textunderscore . Cp. \textunderscore mesnada\textunderscore )}
\end{itemize}
Soldado de mesnada; chefe de mesnada.
\section{Mesnaderia}
\begin{itemize}
\item {Grp. gram.:f.}
\end{itemize}
O mesmo que \textunderscore mesnadaria\textunderscore .
\section{Mesnado}
\begin{itemize}
\item {Grp. gram.:m.}
\end{itemize}
\begin{itemize}
\item {Utilização:Ant.}
\end{itemize}
O mesmo que \textunderscore mesnada\textunderscore .
\section{Meso}
\begin{itemize}
\item {Grp. gram.:m.}
\end{itemize}
\begin{itemize}
\item {Utilização:Anat.}
\end{itemize}
\begin{itemize}
\item {Proveniência:(Gr. \textunderscore mesos\textunderscore )}
\end{itemize}
Ligamento peritoneal entre a parede e alguma víscera.
\section{Meso...}
\begin{itemize}
\item {Grp. gram.:pref.}
\end{itemize}
\begin{itemize}
\item {Proveniência:(Do gr. \textunderscore mesos\textunderscore )}
\end{itemize}
(designativo de \textunderscore médio\textunderscore , \textunderscore meio\textunderscore ).
\section{Mesocarpal}
\begin{itemize}
\item {Grp. gram.:adj.}
\end{itemize}
Relativo ao mesocarpo, próprio do mesocarpo.
\section{Mesocárpico}
\begin{itemize}
\item {Grp. gram.:adj.}
\end{itemize}
Que diz respeito ao mesocarpo.
\section{Mesocarpo}
\begin{itemize}
\item {Grp. gram.:m.}
\end{itemize}
\begin{itemize}
\item {Utilização:Bot.}
\end{itemize}
\begin{itemize}
\item {Utilização:Anat.}
\end{itemize}
\begin{itemize}
\item {Proveniência:(Do gr. \textunderscore mesos\textunderscore  + \textunderscore karpos\textunderscore )}
\end{itemize}
Substância carnuda, entre a epiderme e a pellícula interior de certos frutos.
Miolo do fruto.
Série inferior dos ossos o corpo.
\section{Mesocécum}
\begin{itemize}
\item {Grp. gram.:m.}
\end{itemize}
\begin{itemize}
\item {Utilização:Anat.}
\end{itemize}
\begin{itemize}
\item {Proveniência:(De \textunderscore meso...\textunderscore  + \textunderscore cécum\textunderscore )}
\end{itemize}
Dobra, que o peritonéu fórma as vezes na parte posterior do cécum.
\section{Mesocefalia}
\begin{itemize}
\item {Grp. gram.:f.}
\end{itemize}
Estado de quem tem mesocéfalo.
\section{Mesocefalite}
\begin{itemize}
\item {Grp. gram.:f.}
\end{itemize}
Inflamação do mesocéfalo.
\section{Mesocéfalo}
\begin{itemize}
\item {Grp. gram.:m.}
\end{itemize}
\begin{itemize}
\item {Proveniência:(Do gr. \textunderscore mesos\textunderscore  + \textunderscore kephale\textunderscore )}
\end{itemize}
Protuberância, na parte inferior e média do cérebro.
\section{Mesocephalia}
\begin{itemize}
\item {Grp. gram.:f.}
\end{itemize}
Estado de quem tem mesocéphalo.
\section{Mesocephalite}
\begin{itemize}
\item {Grp. gram.:f.}
\end{itemize}
Inflammação do mesocéphalo.
\section{Mesocéphalo}
\begin{itemize}
\item {Grp. gram.:m.}
\end{itemize}
\begin{itemize}
\item {Proveniência:(Do gr. \textunderscore mesos\textunderscore  + \textunderscore kephale\textunderscore )}
\end{itemize}
Protuberância, na parte inferior e média do cérebro.
\section{Mesoclasto}
\begin{itemize}
\item {Grp. gram.:adj.}
\end{itemize}
\begin{itemize}
\item {Proveniência:(Do gr. \textunderscore mesos\textunderscore  + \textunderscore klao\textunderscore )}
\end{itemize}
Dizia-se do verso hexâmetro, grego ou latino, no meio do qual havia falta de uma quantidade métrica.
\section{Mesâclise}
\begin{itemize}
\item {Grp. gram.:f.}
\end{itemize}
\begin{itemize}
\item {Utilização:Gram.}
\end{itemize}
Interposição de variações pronominaes aos verbos: \textunderscore louvar-te-ei; dir-me-ão\textunderscore .
Tmese.
(Do. gr. \textunderscore mesos\textunderscore  + \textunderscore klasis\textunderscore )
\section{Mesocólico}
\begin{itemize}
\item {Grp. gram.:adj.}
\end{itemize}
\begin{itemize}
\item {Utilização:Med.}
\end{itemize}
Relativo ao mesocólon: \textunderscore hérnia mesocólica\textunderscore .
\section{Mesocolo}
\begin{itemize}
\item {Grp. gram.:m.}
\end{itemize}
\begin{itemize}
\item {Proveniência:(Do gr. \textunderscore mesos\textunderscore  + \textunderscore kolon\textunderscore )}
\end{itemize}
Cada uma das pregas do peritonéu.
\section{Mesocólon}
\begin{itemize}
\item {Grp. gram.:m.}
\end{itemize}
\begin{itemize}
\item {Proveniência:(Do gr. \textunderscore mesos\textunderscore  + \textunderscore kolon\textunderscore )}
\end{itemize}
Cada uma das pregas do peritonéu.
\section{Mesocracia}
\begin{itemize}
\item {Grp. gram.:f.}
\end{itemize}
\begin{itemize}
\item {Utilização:Neol.}
\end{itemize}
\begin{itemize}
\item {Proveniência:(Do gr. \textunderscore mesos\textunderscore  + \textunderscore krateia\textunderscore )}
\end{itemize}
Govêrno, exercido ou influenciado pelas classes médias ou pela burguesia.
\section{Mesocrânio}
\begin{itemize}
\item {Grp. gram.:m.}
\end{itemize}
\begin{itemize}
\item {Proveniência:(Do gr. \textunderscore mesos\textunderscore  + \textunderscore kranion\textunderscore )}
\end{itemize}
O meio da testa.
\section{Mesocrático}
\begin{itemize}
\item {Grp. gram.:adj.}
\end{itemize}
Relativo á mesocracia.
\section{Mesocuneiforme}
\begin{itemize}
\item {Grp. gram.:adj.}
\end{itemize}
\begin{itemize}
\item {Proveniência:(De \textunderscore meso...\textunderscore  + \textunderscore cuneiforme\textunderscore )}
\end{itemize}
Diz-se do osso cuneiforme, que fica em meio dos três que estão alinhados transversalmente no peito do pé.
\section{Mesode}
\begin{itemize}
\item {Grp. gram.:f.}
\end{itemize}
\begin{itemize}
\item {Proveniência:(De \textunderscore meso...\textunderscore  + \textunderscore ode\textunderscore )}
\end{itemize}
Trecho de canto, entre a estrophe e a antístropho, na poesia antiga.
\section{Mesoderme}
\begin{itemize}
\item {Grp. gram.:f.}
\end{itemize}
\begin{itemize}
\item {Utilização:Bot.}
\end{itemize}
\begin{itemize}
\item {Proveniência:(Do gr. \textunderscore mesos\textunderscore  + \textunderscore derma\textunderscore )}
\end{itemize}
Parte da casca, entre a camada tuberosa e o invólucro herbáceo.
\section{Mesodesmo}
\begin{itemize}
\item {Grp. gram.:m.}
\end{itemize}
\begin{itemize}
\item {Proveniência:(Do gr. \textunderscore mesos\textunderscore  + \textunderscore desmos\textunderscore )}
\end{itemize}
Gênero de molluscos bivalves.
\section{Mesodiscal}
\begin{itemize}
\item {Grp. gram.:adj.}
\end{itemize}
\begin{itemize}
\item {Utilização:Bot.}
\end{itemize}
\begin{itemize}
\item {Proveniência:(De \textunderscore meso...\textunderscore  + \textunderscore disco\textunderscore )}
\end{itemize}
Diz-se da inserção dos estames, quando estes estão na face superior do disco.
\section{Mesofalange}
\begin{itemize}
\item {Grp. gram.:f.}
\end{itemize}
\begin{itemize}
\item {Utilização:Anat.}
\end{itemize}
\begin{itemize}
\item {Proveniência:(De \textunderscore meso...\textunderscore  + \textunderscore phalange\textunderscore )}
\end{itemize}
Peça média do dedo.
\section{Mesofalangeal}
\begin{itemize}
\item {Grp. gram.:adj.}
\end{itemize}
Relativo á mesofalange.
\section{Mesofilo}
\begin{itemize}
\item {Grp. gram.:m.}
\end{itemize}
\begin{itemize}
\item {Utilização:Bot.}
\end{itemize}
\begin{itemize}
\item {Proveniência:(Do gr. \textunderscore mesos\textunderscore  + \textunderscore phullon\textunderscore )}
\end{itemize}
Parte média da folha; parênquima.
\section{Mesófito}
\begin{itemize}
\item {Grp. gram.:m.}
\end{itemize}
\begin{itemize}
\item {Utilização:Anat.}
\end{itemize}
\begin{itemize}
\item {Proveniência:(Do gr. \textunderscore mesos\textunderscore  + \textunderscore phuton\textunderscore )}
\end{itemize}
Linha divisória entre a haste e a raíz da planta.
\section{Mesofragma}
\begin{itemize}
\item {Grp. gram.:m.}
\end{itemize}
\begin{itemize}
\item {Proveniência:(Do gr. \textunderscore mesos\textunderscore  + \textunderscore phragma\textunderscore )}
\end{itemize}
Divisão interior do tórax dos insectos.
\section{Mesófrio}
\begin{itemize}
\item {Grp. gram.:m.}
\end{itemize}
\begin{itemize}
\item {Utilização:Anat.}
\end{itemize}
\begin{itemize}
\item {Proveniência:(Do gr. \textunderscore mesos\textunderscore  + \textunderscore ophrus\textunderscore )}
\end{itemize}
Parte do rosto, situada entre as duas sobrancelhas.
\section{Mesogastro}
\begin{itemize}
\item {Grp. gram.:m.}
\end{itemize}
\begin{itemize}
\item {Utilização:Anat.}
\end{itemize}
\begin{itemize}
\item {Proveniência:(Do gr. \textunderscore mesos\textunderscore  + \textunderscore gaster\textunderscore )}
\end{itemize}
Região média do abdome, ou região intermédia ás regiões epigástrica e hypogástrica.
\section{Mesojurássico}
\begin{itemize}
\item {Grp. gram.:adj.}
\end{itemize}
\begin{itemize}
\item {Utilização:Geol.}
\end{itemize}
\begin{itemize}
\item {Proveniência:(De \textunderscore meso...\textunderscore  + \textunderscore jurássico\textunderscore )}
\end{itemize}
Relativo ao terreno jurássico médio.
\section{Mesolábio}
\begin{itemize}
\item {Grp. gram.:m.}
\end{itemize}
\begin{itemize}
\item {Proveniência:(Do gr. \textunderscore mesos\textunderscore  + \textunderscore labein\textunderscore )}
\end{itemize}
Antigo instrumento geométrico, destinado a achar mechanicamente duas médias proporcionaes, que não podiam sêr achadas simetricamente.
\section{Mesolíthico}
\begin{itemize}
\item {Grp. gram.:adj.}
\end{itemize}
\begin{itemize}
\item {Proveniência:(Do gr. \textunderscore mesos\textunderscore  + \textunderscore lithos\textunderscore )}
\end{itemize}
Diz-se do período prehistórico, em que se usavam promiscuamente instrumentos de pedra polida e de pedra lascada.
\section{Mesolítico}
\begin{itemize}
\item {Grp. gram.:adj.}
\end{itemize}
\begin{itemize}
\item {Proveniência:(Do gr. \textunderscore mesos\textunderscore  + \textunderscore lithos\textunderscore )}
\end{itemize}
Diz-se do período prehistórico, em que se usavam promiscuamente instrumentos de pedra polida e de pedra lascada.
\section{Mesolóbulo}
\begin{itemize}
\item {Grp. gram.:m.}
\end{itemize}
\begin{itemize}
\item {Proveniência:(De \textunderscore mesò...\textunderscore  + \textunderscore lóbulo\textunderscore )}
\end{itemize}
Parte callosa, entre os dois hemisférios de cérebro.
\section{Mesologarithmo}
\begin{itemize}
\item {Grp. gram.:m.}
\end{itemize}
\begin{itemize}
\item {Utilização:Mathem.}
\end{itemize}
\begin{itemize}
\item {Proveniência:(De \textunderscore meso...\textunderscore  + \textunderscore logarithmo\textunderscore )}
\end{itemize}
Designação antiquada do logarithmo dos cosenos e das cotangentes.
\section{Mesologaritmo}
\begin{itemize}
\item {Grp. gram.:m.}
\end{itemize}
\begin{itemize}
\item {Utilização:Mathem.}
\end{itemize}
\begin{itemize}
\item {Proveniência:(De \textunderscore meso...\textunderscore  + \textunderscore logarithmo\textunderscore )}
\end{itemize}
Designação antiquada do logarithmo dos cosenos e das cotangentes.
\section{Mesologia}
\begin{itemize}
\item {Grp. gram.:f.}
\end{itemize}
\begin{itemize}
\item {Proveniência:(Do gr. \textunderscore mesos\textunderscore  + \textunderscore logos\textunderscore )}
\end{itemize}
Sciência, que tem por objecto as relações entre os seres e o seu meio ou ambiente.
\section{Mesológico}
\begin{itemize}
\item {Grp. gram.:adj.}
\end{itemize}
Relativo á Mesologia.
\section{Mesómacro}
\begin{itemize}
\item {Grp. gram.:adj.}
\end{itemize}
\begin{itemize}
\item {Proveniência:(Do gr. \textunderscore mesos\textunderscore  + \textunderscore makros\textunderscore )}
\end{itemize}
Dizia-se, na poesia antiga, de um pé de verso, composto de duas breves, uma longa e mais outras duas breves.
\section{Mesomeria}
\begin{itemize}
\item {Grp. gram.:f.}
\end{itemize}
\begin{itemize}
\item {Utilização:Anat.}
\end{itemize}
\begin{itemize}
\item {Proveniência:(Do gr. \textunderscore mesos\textunderscore  + \textunderscore meros\textunderscore )}
\end{itemize}
A parte do corpo, situada entre as coxas.
\section{Mesómetro}
\begin{itemize}
\item {Grp. gram.:m.}
\end{itemize}
\begin{itemize}
\item {Utilização:Anat.}
\end{itemize}
\begin{itemize}
\item {Proveniência:(Do gr. \textunderscore mesos\textunderscore  + \textunderscore metra\textunderscore )}
\end{itemize}
Dobra peritoneal, que liga o útero ás paredes abdominaes.
\section{Mesophalange}
\begin{itemize}
\item {Grp. gram.:f.}
\end{itemize}
\begin{itemize}
\item {Utilização:Anat.}
\end{itemize}
\begin{itemize}
\item {Proveniência:(De \textunderscore meso...\textunderscore  + \textunderscore phalange\textunderscore )}
\end{itemize}
Peça média do dedo.
\section{Mesophalangeal}
\begin{itemize}
\item {Grp. gram.:adj.}
\end{itemize}
Relativo á mesophalange.
\section{Mesophragma}
\begin{itemize}
\item {Grp. gram.:m.}
\end{itemize}
\begin{itemize}
\item {Proveniência:(Do gr. \textunderscore mesos\textunderscore  + \textunderscore phragma\textunderscore )}
\end{itemize}
Divisão interior do thórax dos insectos.
\section{Mesóphryo}
\begin{itemize}
\item {Grp. gram.:m.}
\end{itemize}
\begin{itemize}
\item {Utilização:Anat.}
\end{itemize}
\begin{itemize}
\item {Proveniência:(Do gr. \textunderscore mesos\textunderscore  + \textunderscore ophrus\textunderscore )}
\end{itemize}
Parte do rosto, situada entre as duas sobrancelhas.
\section{Mesophyllo}
\begin{itemize}
\item {Grp. gram.:m.}
\end{itemize}
\begin{itemize}
\item {Utilização:Bot.}
\end{itemize}
\begin{itemize}
\item {Proveniência:(Do gr. \textunderscore mesos\textunderscore  + \textunderscore phullon\textunderscore )}
\end{itemize}
Parte média da folha; parênchyma.
\section{Mesóphyto}
\begin{itemize}
\item {Grp. gram.:m.}
\end{itemize}
\begin{itemize}
\item {Utilização:Anat.}
\end{itemize}
\begin{itemize}
\item {Proveniência:(Do gr. \textunderscore mesos\textunderscore  + \textunderscore phuton\textunderscore )}
\end{itemize}
Linha divisória entre a haste e a raíz da planta.
\section{Mesopiteco}
\begin{itemize}
\item {Grp. gram.:m.}
\end{itemize}
\begin{itemize}
\item {Proveniência:(Do gr. \textunderscore mesos\textunderscore  + \textunderscore pithekos\textunderscore )}
\end{itemize}
Gênero de macacos fósseis.
\section{Mesopitheco}
\begin{itemize}
\item {Grp. gram.:m.}
\end{itemize}
\begin{itemize}
\item {Proveniência:(Do gr. \textunderscore mesos\textunderscore  + \textunderscore pithekos\textunderscore )}
\end{itemize}
Gênero de macacos fósseis.
\section{Mesopotâmia}
\begin{itemize}
\item {Grp. gram.:f.}
\end{itemize}
Região, situada entre rios. Cf. Filinto, \textunderscore D. Man.\textunderscore , I, 180.
(Do. gr. \textunderscore mesos\textunderscore  + \textunderscore potamos\textunderscore )
\section{Mesopotâmico}
\begin{itemize}
\item {Grp. gram.:adj.}
\end{itemize}
Relativo á Mesopotâmia: \textunderscore civilisação mesopotâmica\textunderscore .
\section{Mesor}
\begin{itemize}
\item {Grp. gram.:m.}
\end{itemize}
\begin{itemize}
\item {Utilização:Ant.}
\end{itemize}
O mesmo que \textunderscore salmão\textunderscore .
\section{Meso-recto}
\begin{itemize}
\item {Grp. gram.:m.}
\end{itemize}
\begin{itemize}
\item {Proveniência:(De \textunderscore meso...\textunderscore  + \textunderscore recto\textunderscore )}
\end{itemize}
Expansão do peritonéu, que mantém o recto na sua posição natural.
\section{Mesorrhinia}
\begin{itemize}
\item {Grp. gram.:f.}
\end{itemize}
Qualidade de mesorrhínio.
\section{Mesorrhínio}
\begin{itemize}
\item {Grp. gram.:adj.}
\end{itemize}
\begin{itemize}
\item {Utilização:Anat.}
\end{itemize}
\begin{itemize}
\item {Proveniência:(Do gr. \textunderscore mesos\textunderscore  + \textunderscore rhinos\textunderscore )}
\end{itemize}
Cujo esqueleto nasal tem dimensão média.
\section{Mesorrhino}
\begin{itemize}
\item {Grp. gram.:adj.}
\end{itemize}
O mesmo que \textunderscore mesorrhínio\textunderscore . Cf. Ed. Burnay, \textunderscore Craneol.\textunderscore , 143.
\section{Mesorrinia}
\begin{itemize}
\item {Grp. gram.:f.}
\end{itemize}
Qualidade de mesorrínio.
\section{Mesorrínio}
\begin{itemize}
\item {Grp. gram.:adj.}
\end{itemize}
\begin{itemize}
\item {Utilização:Anat.}
\end{itemize}
\begin{itemize}
\item {Proveniência:(Do gr. \textunderscore mesos\textunderscore  + \textunderscore rhinos\textunderscore )}
\end{itemize}
Cujo esqueleto nasal tem dimensão média.
\section{Mesorrino}
\begin{itemize}
\item {Grp. gram.:adj.}
\end{itemize}
O mesmo que \textunderscore mesorrínio\textunderscore . Cf. Ed. Burnay, \textunderscore Craneol.\textunderscore , 143.
\section{Mesosemo}
\begin{itemize}
\item {fónica:sê}
\end{itemize}
\begin{itemize}
\item {Grp. gram.:adj.}
\end{itemize}
\begin{itemize}
\item {Utilização:Anthrop.}
\end{itemize}
Cuja órbita ocular é de medianas dimensões, segundo a classificação anthropológica de Broca.
\section{Mesossemo}
\begin{itemize}
\item {Grp. gram.:adj.}
\end{itemize}
\begin{itemize}
\item {Utilização:Anthrop.}
\end{itemize}
Cuja órbita ocular é de medianas dimensões, segundo a classificação antropológica de Broca.
\section{Mesosterno}
\begin{itemize}
\item {Grp. gram.:m.}
\end{itemize}
\begin{itemize}
\item {Proveniência:(De \textunderscore meso...\textunderscore  + \textunderscore esterno\textunderscore )}
\end{itemize}
O corpo ou parte média do esterno.
\section{Mesotenar}
\begin{itemize}
\item {Grp. gram.:m.}
\end{itemize}
\begin{itemize}
\item {Utilização:Anat.}
\end{itemize}
\begin{itemize}
\item {Proveniência:(De \textunderscore mesò...\textunderscore  + \textunderscore thenar\textunderscore )}
\end{itemize}
Músculo, que aproxima da palma da mão o dedo polegar.
\section{Mesotério}
\begin{itemize}
\item {Grp. gram.:m.}
\end{itemize}
\begin{itemize}
\item {Proveniência:(Do gr. \textunderscore mesos\textunderscore  + \textunderscore therion\textunderscore )}
\end{itemize}
Animal fóssil, descoberto nos pampas, perto de Buenos-Aires.
\section{Mesothenar}
\begin{itemize}
\item {Grp. gram.:m.}
\end{itemize}
\begin{itemize}
\item {Utilização:Anat.}
\end{itemize}
\begin{itemize}
\item {Proveniência:(De \textunderscore mesò...\textunderscore  + \textunderscore thenar\textunderscore )}
\end{itemize}
Músculo, que aproxima da palma da mão o dedo pollegar.
\section{Mesothério}
\begin{itemize}
\item {Grp. gram.:m.}
\end{itemize}
\begin{itemize}
\item {Proveniência:(Do gr. \textunderscore mesos\textunderscore  + \textunderscore therion\textunderscore )}
\end{itemize}
Animal fóssil, descoberto nos pampas, perto de Buenos-Aires.
\section{Mesothórax}
\begin{itemize}
\item {Grp. gram.:m.}
\end{itemize}
\begin{itemize}
\item {Proveniência:(De \textunderscore meso...\textunderscore  + \textunderscore thórax\textunderscore )}
\end{itemize}
Parte dos insectos, que sustém as asas superiores e as patas intermédias.
\section{Mesótipe}
\begin{itemize}
\item {Grp. gram.:f.}
\end{itemize}
Minério dos Açores. Cf. Flaviense, \textunderscore Diccion. Geogr.\textunderscore 
\section{Mesotórax}
\begin{itemize}
\item {Grp. gram.:m.}
\end{itemize}
\begin{itemize}
\item {Proveniência:(De \textunderscore meso...\textunderscore  + \textunderscore thórax\textunderscore )}
\end{itemize}
Parte dos insectos, que sustém as asas superiores e as patas intermédias.
\section{Mesótype}
\begin{itemize}
\item {Grp. gram.:f.}
\end{itemize}
Minério dos Açores. Cf. Flaviense, \textunderscore Diccion. Geogr.\textunderscore 
\section{Mesozeugma}
\begin{itemize}
\item {Grp. gram.:m.}
\end{itemize}
\begin{itemize}
\item {Proveniência:(De \textunderscore meso...\textunderscore  + \textunderscore zeugma\textunderscore )}
\end{itemize}
Espécie de zeugma, quando a palavra subentendida está no meio de outra phrase.
\section{Mesozoico}
\begin{itemize}
\item {Grp. gram.:adj.}
\end{itemize}
\begin{itemize}
\item {Proveniência:(Do gr. \textunderscore mesos\textunderscore  + \textunderscore zoon\textunderscore )}
\end{itemize}
Diz-se, em Geologia, dos terrenos mais recentes entre os secundários.
\section{Mesquindade}
\begin{itemize}
\item {Grp. gram.:f.}
\end{itemize}
\begin{itemize}
\item {Utilização:Ant.}
\end{itemize}
O mesmo que \textunderscore mesquinhez\textunderscore . Cf. Fernão Lopes.
\section{Mesquinhamente}
\begin{itemize}
\item {Grp. gram.:adv.}
\end{itemize}
De modo mesquinho.
\section{Mesquinhar}
\begin{itemize}
\item {Grp. gram.:v. t.}
\end{itemize}
\begin{itemize}
\item {Proveniência:(De \textunderscore mesquinho\textunderscore )}
\end{itemize}
Recusar por mesquinhez; regatear.
Julgar mesquinho ou infeliz:«\textunderscore e em vez de rendermos a Deos muitas graças, nos mesquinhamos e entristecemos.\textunderscore »\textunderscore Luz e Calor\textunderscore , 303.
\section{Mesquinharia}
\begin{itemize}
\item {Grp. gram.:f.}
\end{itemize}
Qualidade de mesquinho.
Desdita.
Insignificância.
Sovinice.
\section{Mesquinhez}
\begin{itemize}
\item {Grp. gram.:f.}
\end{itemize}
Qualidade de mesquinho.
Desdita.
Insignificância.
Sovinice.
\section{Mesquinho}
\begin{itemize}
\item {Grp. gram.:adj.}
\end{itemize}
\begin{itemize}
\item {Utilização:Bras}
\end{itemize}
\begin{itemize}
\item {Grp. gram.:M.}
\end{itemize}
\begin{itemize}
\item {Proveniência:(Do cast. \textunderscore mezquino\textunderscore )}
\end{itemize}
Privado do que é necessário.
Pobre.
Insignificante.
Infeliz.
Estéril.
Avarento; miserável.
Que não consente o freio, (falando-se do cavallo).
Indivíduo dasgraçado:«\textunderscore a mísera e mesquinha, que depois de ser morta foi rainha.\textunderscore »\textunderscore Lusíadas\textunderscore , III.
Avarento.
\section{Mesquino}
\begin{itemize}
\item {Grp. gram.:m.}
\end{itemize}
\begin{itemize}
\item {Utilização:Ant.}
\end{itemize}
Servo, que trabalhava na herdade do seu senhor.
(Cp. \textunderscore mesquinho\textunderscore )
\section{Mesquita}
\begin{itemize}
\item {Grp. gram.:f.}
\end{itemize}
\begin{itemize}
\item {Proveniência:(Do ár. \textunderscore mesqid\textunderscore )}
\end{itemize}
Templo dos Mahometanos.
\section{Messageiro}
\begin{itemize}
\item {Grp. gram.:m.  e  adj.}
\end{itemize}
\begin{itemize}
\item {Utilização:Des.}
\end{itemize}
O mesmo que \textunderscore mensageiro\textunderscore :«\textunderscore messageira lucifera do dia...\textunderscore »Filinto, X, 5.
(Cp. fr. \textunderscore messager\textunderscore )
\section{Messagem}
\begin{itemize}
\item {Grp. gram.:f.}
\end{itemize}
\begin{itemize}
\item {Utilização:Ant.}
\end{itemize}
O mesmo ou melhor que \textunderscore mensagem\textunderscore . Cf. \textunderscore Port. Mon. Hist.\textunderscore , \textunderscore Script.\textunderscore , 281.
\section{Messajaria}
\begin{itemize}
\item {Grp. gram.:f.}
\end{itemize}
\begin{itemize}
\item {Utilização:Ant.}
\end{itemize}
O mesmo que \textunderscore mensagem\textunderscore .
\section{Messalina}
\begin{itemize}
\item {Grp. gram.:f.}
\end{itemize}
\begin{itemize}
\item {Utilização:Fig.}
\end{itemize}
\begin{itemize}
\item {Proveniência:(De \textunderscore Messalina\textunderscore , n. p.)}
\end{itemize}
Mulher, extremamente lasciva e dissoluta.
\section{Messangeiro}
\begin{itemize}
\item {Grp. gram.:m.}
\end{itemize}
\begin{itemize}
\item {Utilização:T. de Ceilão}
\end{itemize}
O mesmo que \textunderscore mensageiro\textunderscore .
\section{Messar}
\begin{itemize}
\item {Grp. gram.:v. t.}
\end{itemize}
\begin{itemize}
\item {Utilização:Ant.}
\end{itemize}
Puxar injuriosamente as barbas a. Cf. Figaniére, \textunderscore G. Ansures\textunderscore .
\section{Messar-se}
\begin{itemize}
\item {Grp. gram.:v. p.}
\end{itemize}
\begin{itemize}
\item {Utilização:Prov.}
\end{itemize}
\begin{itemize}
\item {Utilização:beir.}
\end{itemize}
Ferir-se naturalmente nos refegos dos tecidos orgânicos, por effeito da gordura ou da delicadeza da epiderme, (falando-se especialmente das crianças, quando se lhes escoría naturalmente o pescoço ou as virilhas).
(Talvez alter. de \textunderscore mossar\textunderscore , de \textunderscore mossa\textunderscore )
\section{Messe}
\begin{itemize}
\item {Grp. gram.:f.}
\end{itemize}
\begin{itemize}
\item {Utilização:Fig.}
\end{itemize}
\begin{itemize}
\item {Proveniência:(Lat. \textunderscore messis\textunderscore )}
\end{itemize}
Seara, em estado de se ceifar.
Ceifa.
Acquisição.
Conquista; lucro.
Conversão de almas.
\section{Messiádego}
\begin{itemize}
\item {Grp. gram.:m.}
\end{itemize}
O mesmo que \textunderscore messiado\textunderscore . Cf. Heitor Pinto, \textunderscore Imagem da V. Chr.\textunderscore 
\section{Messiado}
\begin{itemize}
\item {Grp. gram.:m.}
\end{itemize}
Missão ou funcções de Messias.
\section{Messiânico}
\begin{itemize}
\item {Grp. gram.:adj.}
\end{itemize}
Relativo ao Messias.
\section{Messianismo}
\begin{itemize}
\item {Grp. gram.:m.}
\end{itemize}
Crença na vinda do Messias.
\section{Messianista}
\begin{itemize}
\item {Grp. gram.:m.}
\end{itemize}
Sectário do messianismo.
\section{Messias}
\begin{itemize}
\item {Grp. gram.:m.}
\end{itemize}
\begin{itemize}
\item {Utilização:Ext.}
\end{itemize}
\begin{itemize}
\item {Utilização:Fig.}
\end{itemize}
\begin{itemize}
\item {Proveniência:(Lat. \textunderscore messias\textunderscore )}
\end{itemize}
O redemptor promettido no \textunderscore Antigo Testamento\textunderscore .
Pessôa, esperada ansiosamente.
Reformador social.
\section{Messidor}
\begin{itemize}
\item {Grp. gram.:m.}
\end{itemize}
\begin{itemize}
\item {Proveniência:(Do lat. \textunderscore messis\textunderscore  + gr. \textunderscore doron\textunderscore )}
\end{itemize}
Décimo mês do calendário da primeira república francesa, o qual começava a 19 ou 20 de Junho.
\section{Messório}
\begin{itemize}
\item {Grp. gram.:adj.}
\end{itemize}
\begin{itemize}
\item {Proveniência:(Lat. \textunderscore messorius\textunderscore )}
\end{itemize}
Que ceifa ou recolhe cereaes. Cf. Garção, II, 21.
\section{Mesteiral}
\begin{itemize}
\item {Grp. gram.:m.}
\end{itemize}
\begin{itemize}
\item {Utilização:Ant.}
\end{itemize}
\begin{itemize}
\item {Proveniência:(De \textunderscore mestér\textunderscore )}
\end{itemize}
O mesmo que \textunderscore artífice\textunderscore .
\section{Mesteireiro}
\begin{itemize}
\item {Grp. gram.:m.}
\end{itemize}
O mesmo que \textunderscore mesteiral\textunderscore . Cf. Castilho, \textunderscore Sonho\textunderscore , notas.
\section{Mesteiroso}
\begin{itemize}
\item {Grp. gram.:adj.}
\end{itemize}
\begin{itemize}
\item {Utilização:Ant.}
\end{itemize}
\begin{itemize}
\item {Proveniência:(De \textunderscore mestér\textunderscore )}
\end{itemize}
Que mesteiral.
Necessitado.
Preciso.
\section{Mestér}
\begin{itemize}
\item {Grp. gram.:m.}
\end{itemize}
\begin{itemize}
\item {Utilização:Ant.}
\end{itemize}
\begin{itemize}
\item {Proveniência:(Do lat. \textunderscore ministerium\textunderscore )}
\end{itemize}
Offício, arte manual.
Mesteiral.
O mesmo que \textunderscore mister\textunderscore .
Cada um dos quatro homens que, no século XVII, havia no Senado de Lisbôa, os quaes o representavam perante os officios mecânicos, taxavam salários ou preço dos productos dos offícios, regulavam certas eleições, etc. Cf. Bluteau.
\section{Mestiçagem}
\begin{itemize}
\item {Grp. gram.:f.}
\end{itemize}
Acto ou effeito de \textunderscore mestiçar-se\textunderscore .
\section{Mestiçamento}
\begin{itemize}
\item {Grp. gram.:m.}
\end{itemize}
Acto ou effeito de \textunderscore mestiçar-se\textunderscore .
\section{Mestiçar-se}
\begin{itemize}
\item {Grp. gram.:v. p.}
\end{itemize}
\begin{itemize}
\item {Proveniência:(De \textunderscore mestiço\textunderscore )}
\end{itemize}
Diz-se das raças, ou dos indivíduos de uma raça, que se cruzam com os de outra, procriando mestiços. Cf. João Ribeiro, \textunderscore Diccion. Gram.\textunderscore , 64.
\section{Mestiço}
\begin{itemize}
\item {Grp. gram.:m.  e  adj.}
\end{itemize}
Indivíduo, cujos pais são, entre si, de raça differente.
(Por \textunderscore mistiço\textunderscore , de \textunderscore misto\textunderscore  = \textunderscore mixto\textunderscore ).
\section{Mesto}
\begin{itemize}
\item {Grp. gram.:adj.}
\end{itemize}
\begin{itemize}
\item {Utilização:Poét.}
\end{itemize}
\begin{itemize}
\item {Proveniência:(Lat. \textunderscore moestus\textunderscore )}
\end{itemize}
Triste.
Que causa tristeza.
\section{Mestra}
\begin{itemize}
\item {Grp. gram.:f.}
\end{itemize}
\begin{itemize}
\item {Proveniência:(De \textunderscore mestre\textunderscore )}
\end{itemize}
Mulher, que ensina.
Professora.
Qualquer coisa que, sendo do gênero feminino, faculta ensinamentos úteis: \textunderscore a experiência é grande mestra\textunderscore .
\section{Mestraça}
\begin{itemize}
\item {Grp. gram.:f.}
\end{itemize}
\begin{itemize}
\item {Proveniência:(De \textunderscore mestraço\textunderscore )}
\end{itemize}
Mulher, que sabe do seu offício; mulher hábil. Cf. \textunderscore Agostinheida\textunderscore , 83.
\section{Mestraço}
\begin{itemize}
\item {Grp. gram.:m.}
\end{itemize}
\begin{itemize}
\item {Proveniência:(De \textunderscore mestre\textunderscore )}
\end{itemize}
Aquelle que sabe muito do seu offício; mestre hábil.
\section{Mestrado}
\begin{itemize}
\item {Grp. gram.:m.}
\end{itemize}
Dignidade de mestre, numa Ordem militar.
Exercicio dessa dignidade.
\section{Mestral}
\begin{itemize}
\item {Grp. gram.:adj.}
\end{itemize}
\begin{itemize}
\item {Proveniência:(Do lat. \textunderscore magistralis\textunderscore )}
\end{itemize}
Relativo a mestrado.
\section{Mestrança}
\begin{itemize}
\item {Grp. gram.:f.}
\end{itemize}
\begin{itemize}
\item {Utilização:Pop.}
\end{itemize}
\begin{itemize}
\item {Proveniência:(De \textunderscore mestre\textunderscore )}
\end{itemize}
Local das officinas do material de guerra.
Depósito de material para embarcações.
Conjunto dos mestres de um arsenal, quando reunidos para vistoria ou inspecção.
Conjunto dos indivíduos mais graduados, ou mais considerados, de uma arte ou corporação.
\section{Mestrão}
\begin{itemize}
\item {Grp. gram.:m.}
\end{itemize}
\begin{itemize}
\item {Utilização:Pop.}
\end{itemize}
O mesmo que \textunderscore mestraço\textunderscore .
\section{Mestras}
\begin{itemize}
\item {Grp. gram.:f. pl.}
\end{itemize}
\begin{itemize}
\item {Utilização:Des.}
\end{itemize}
Porções de terreno, que os trabalhadores deixam por cavar, para se facilitar a medição do trabalho feito.
(Corr. de \textunderscore meta\textunderscore ?)
\section{Mestre}
\begin{itemize}
\item {Grp. gram.:m.}
\end{itemize}
\begin{itemize}
\item {Utilização:Bras}
\end{itemize}
\begin{itemize}
\item {Utilização:Ant.}
\end{itemize}
\begin{itemize}
\item {Grp. gram.:Adj.}
\end{itemize}
\begin{itemize}
\item {Proveniência:(Do lat. \textunderscore magister\textunderscore )}
\end{itemize}
Homem, que ensina.
Professor.
Aquelle que é versado numa arte ou sciência: \textunderscore os mestres da língua\textunderscore .
Aquelle que tem qualquer superioridade.
Artífice, que dirige outros ou que trabalha por sua conta: o \textunderscore mestre barbeiro\textunderscore .
Chefe de fábrica.
Aquelle que fiscaliza o apparelho e velame, a bordo.
Aquelle que commanda uma pequena embarcação.
Indivíduo, que na maçonaria tem o terceiro grau.
Cão adestrado na caça:«\textunderscore solta os mestres, ó perreiro, que a caça vai começar\textunderscore ». Araújo Porto-Alegre.
Director espiritual; confessor.
Que tem vantagem ou occupa posição superior, em relação a outrem.
Grande, extraordinário: \textunderscore um escândalo mestre\textunderscore .
\section{Mestrear}
\begin{itemize}
\item {Grp. gram.:v. i.}
\end{itemize}
\begin{itemize}
\item {Utilização:Bras}
\end{itemize}
\begin{itemize}
\item {Utilização:Neol.}
\end{itemize}
Fazer de mestre, falar como mestre.
\section{Mestre-de-dança}
\begin{itemize}
\item {Grp. gram.:m.}
\end{itemize}
Designação vulgar de um compasso especial de pontas duplas.
\section{Mestre-escola}
\begin{itemize}
\item {Grp. gram.:m.}
\end{itemize}
Professor de instrucção primária.
Dignidade inferior, em cabidos.--Expressão preferível, ou mais portuguesa, é \textunderscore mestre de meninos\textunderscore  ou \textunderscore de primeiras letras\textunderscore .
(Contr. de \textunderscore mestre-de-escola\textunderscore )
\section{Mestre-escolado}
\begin{itemize}
\item {Grp. gram.:m.}
\end{itemize}
\begin{itemize}
\item {Utilização:Des.}
\end{itemize}
Cargo do mestre-escola.
\section{Mestre-sala}
\begin{itemize}
\item {Grp. gram.:m.}
\end{itemize}
Empregado da casa real, que dirigia o ceremonial nas recepções do paço e noutros actos solennes.
Aquelle que dirige um baile público.
(Contr. de \textunderscore mestre-de-sala\textunderscore )
\section{Mestria}
\begin{itemize}
\item {Grp. gram.:f.}
\end{itemize}
\begin{itemize}
\item {Proveniência:(De \textunderscore mestre\textunderscore )}
\end{itemize}
Grande saber; perícia.
\section{Mestrona}
\begin{itemize}
\item {Grp. gram.:f.}
\end{itemize}
\begin{itemize}
\item {Utilização:irón.}
\end{itemize}
\begin{itemize}
\item {Utilização:Fam.}
\end{itemize}
\begin{itemize}
\item {Proveniência:(De \textunderscore mestra\textunderscore )}
\end{itemize}
Sabichona; doutora.
\section{Mestrunço}
\begin{itemize}
\item {Grp. gram.:m.}
\end{itemize}
\begin{itemize}
\item {Utilização:Prov.}
\end{itemize}
\begin{itemize}
\item {Utilização:beir.}
\end{itemize}
Pessôa, que não serve para nada.
Estafermo.
Mostrengo.
(Por \textunderscore monstrunco\textunderscore , de \textunderscore monstro\textunderscore )
\section{Mesua}
\begin{itemize}
\item {Grp. gram.:f.}
\end{itemize}
\begin{itemize}
\item {Utilização:Ant.}
\end{itemize}
Escolta; acompanhamento.
\section{Mesuada}
\begin{itemize}
\item {Grp. gram.:f.}
\end{itemize}
\begin{itemize}
\item {Utilização:Ant.}
\end{itemize}
O mesmo que \textunderscore mesua\textunderscore .
\section{Mesura}
\begin{itemize}
\item {Grp. gram.:f.}
\end{itemize}
\begin{itemize}
\item {Utilização:Ant.}
\end{itemize}
\begin{itemize}
\item {Utilização:Ant.}
\end{itemize}
\begin{itemize}
\item {Proveniência:(Do lat. \textunderscore mensura\textunderscore )}
\end{itemize}
Reverência; cortesia.
Medida.
Generosidade; magnanimidade.
\section{Mesuradamente}
\begin{itemize}
\item {Grp. gram.:adv.}
\end{itemize}
De modo mesurado.
Commedidamente; com prudência.
\section{Mesurado}
\begin{itemize}
\item {Grp. gram.:adj.}
\end{itemize}
Commedido; circunspecto; prudente.
\section{Mesurado}
\begin{itemize}
\item {Grp. gram.:adj.}
\end{itemize}
\begin{itemize}
\item {Proveniência:(De \textunderscore mesura\textunderscore )}
\end{itemize}
Reverenciado.
Que faz mesuras; mesureiro. Cf. Filinto, IX, 110.
\section{Mesurar}
\begin{itemize}
\item {Grp. gram.:v. i.}
\end{itemize}
\begin{itemize}
\item {Grp. gram.:V. t.}
\end{itemize}
\begin{itemize}
\item {Utilização:Ant.}
\end{itemize}
Dirigir cumprimentos; fazer mesuras.
Medir.
\section{Mesureiro}
\begin{itemize}
\item {Grp. gram.:adj.}
\end{itemize}
\begin{itemize}
\item {Utilização:Fig.}
\end{itemize}
Que gosta de fazer mesuras.
Servil; adulador.
\section{Mesurice}
\begin{itemize}
\item {Grp. gram.:f.}
\end{itemize}
\begin{itemize}
\item {Proveniência:(De \textunderscore mesura\textunderscore )}
\end{itemize}
Qualidade de mesureiro.
\section{Meta}
\begin{itemize}
\item {Grp. gram.:f.}
\end{itemize}
\begin{itemize}
\item {Utilização:Fig.}
\end{itemize}
\begin{itemize}
\item {Proveniência:(Lat. \textunderscore meta\textunderscore )}
\end{itemize}
Marco; limite.
Barreira.
Alvo, mira.
Arena.
\section{Meta...}
\begin{itemize}
\item {Grp. gram.:pref.}
\end{itemize}
(designativo de \textunderscore transformação\textunderscore , \textunderscore além de\textunderscore , etc.).
(Da prep. gr. \textunderscore meta\textunderscore )
\section{Metá}
\begin{itemize}
\item {Grp. gram.:f.}
\end{itemize}
\begin{itemize}
\item {Utilização:Ant.}
\end{itemize}
O mesmo que \textunderscore metade\textunderscore . Cf. G. Vicente.
\section{Metábole}
\begin{itemize}
\item {Grp. gram.:f.}
\end{itemize}
\begin{itemize}
\item {Utilização:Rhet.}
\end{itemize}
\begin{itemize}
\item {Proveniência:(Lat. \textunderscore metahole\textunderscore )}
\end{itemize}
Alteração nas palavras ou nas phrases.
\section{Metabolelogia}
\begin{itemize}
\item {Grp. gram.:f.}
\end{itemize}
\begin{itemize}
\item {Utilização:Med.}
\end{itemize}
\begin{itemize}
\item {Proveniência:(Do gr. \textunderscore metabole\textunderscore  + \textunderscore logos\textunderscore )}
\end{itemize}
Descripções das mudanças ou alterações, que ocorrem durante uma doença.
\section{Metabólico}
\begin{itemize}
\item {Grp. gram.:adj.}
\end{itemize}
\begin{itemize}
\item {Utilização:Chím.}
\end{itemize}
\begin{itemize}
\item {Proveniência:(De \textunderscore metábole\textunderscore )}
\end{itemize}
Que constitue mudança de natureza.
Relativo a mudança de natureza dos corpos.
\section{Metabolismo}
\begin{itemize}
\item {Grp. gram.:m.}
\end{itemize}
\begin{itemize}
\item {Utilização:Chím.}
\end{itemize}
\begin{itemize}
\item {Proveniência:(De \textunderscore metábole\textunderscore )}
\end{itemize}
Mudança da natureza molecular dos corpos.
\section{Metábolo}
\begin{itemize}
\item {Grp. gram.:adj.}
\end{itemize}
Diz-se dos insectos, que soffrem mudanças ou metamorphoses.
(Cp. \textunderscore metábole\textunderscore )
\section{Metacarpal}
\begin{itemize}
\item {Grp. gram.:m.}
\end{itemize}
Peça, correspondente ao metacarpo, no typo ideal da mão.
\section{Metacarpiano}
\begin{itemize}
\item {Grp. gram.:adj.}
\end{itemize}
Relativo ao metacarpo.
\section{Metacárpico}
\begin{itemize}
\item {Grp. gram.:adj.}
\end{itemize}
Relativo ao metacarpo.
\section{Metacarpo}
\begin{itemize}
\item {Grp. gram.:m.}
\end{itemize}
\begin{itemize}
\item {Utilização:Anat.}
\end{itemize}
\begin{itemize}
\item {Proveniência:(Do gr. \textunderscore meta\textunderscore  + \textunderscore karpos\textunderscore )}
\end{itemize}
Parte da mão, entre o carpo e os dedos.
\section{Metacêntrico}
\begin{itemize}
\item {Grp. gram.:adj.}
\end{itemize}
\begin{itemize}
\item {Utilização:Náut.}
\end{itemize}
\begin{itemize}
\item {Proveniência:(De \textunderscore metacentro\textunderscore )}
\end{itemize}
Diz-se da curva, formada pela reunião dos metacentros correspondente a todas as inclinações possíveis de um navio.
\section{Metacentro}
\begin{itemize}
\item {Grp. gram.:m.}
\end{itemize}
\begin{itemize}
\item {Proveniência:(De \textunderscore meta...\textunderscore  + \textunderscore centro\textunderscore )}
\end{itemize}
Centro da gravidade de um corpo fluctuante.
\section{Metacetona}
\begin{itemize}
\item {Grp. gram.:f.}
\end{itemize}
\begin{itemize}
\item {Utilização:Chím.}
\end{itemize}
Substância, obtida pela acção da cal sôbre o açúcar o o amido, a uma temperatura elevada.
\section{Metachromatismo}
\begin{itemize}
\item {Grp. gram.:m.}
\end{itemize}
\begin{itemize}
\item {Proveniência:(Do gr. \textunderscore meta\textunderscore  + \textunderscore khroma\textunderscore )}
\end{itemize}
Mudança de côr, que se observa nos pêlos, nas pennas ou na pelle dos animaes, segundo a idade ou differentes condições mórbidas.
\section{Metachronismo}
\begin{itemize}
\item {Grp. gram.:m.}
\end{itemize}
\begin{itemize}
\item {Proveniência:(Do gr. \textunderscore meta\textunderscore  + \textunderscore khronos\textunderscore )}
\end{itemize}
Êrro de data, que consiste em collocar um acontecimento num tempo posterior ao verdadeiro.--Definem assim, pouco mais ou menos, os diccion. portugueses. Littré define diversamente, dando ao pref. \textunderscore meta\textunderscore  a planificação de \textunderscore anterior\textunderscore , mas a sua definição briga, pelo menos, com a que se dá de \textunderscore metazoário\textunderscore .
\section{Metacismo}
\begin{itemize}
\item {Grp. gram.:m.}
\end{itemize}
\begin{itemize}
\item {Proveniência:(Lat. \textunderscore melacismus\textunderscore )}
\end{itemize}
Repetição frequente da letra \textunderscore m\textunderscore .
\section{Metacromatismo}
\begin{itemize}
\item {Grp. gram.:m.}
\end{itemize}
\begin{itemize}
\item {Proveniência:(Do gr. \textunderscore meta\textunderscore  + \textunderscore khroma\textunderscore )}
\end{itemize}
Mudança de côr, que se observa nos pêlos, nas penas ou na pele dos animaes, segundo a idade ou diferentes condições mórbidas.
\section{Metacronismo}
\begin{itemize}
\item {Grp. gram.:m.}
\end{itemize}
\begin{itemize}
\item {Proveniência:(Do gr. \textunderscore meta\textunderscore  + \textunderscore khronos\textunderscore )}
\end{itemize}
Êrro de data, que consiste em colocar um acontecimento num tempo posterior ao verdadeiro.--Definem assim, pouco mais ou menos, os dicion. portugueses. Littré define diversamente, dando ao pref. \textunderscore meta\textunderscore  a planificação de \textunderscore anterior\textunderscore , mas a sua definição briga, pelo menos, com a que se dá de \textunderscore metazoário\textunderscore .
\section{Metade}
\begin{itemize}
\item {Grp. gram.:f.}
\end{itemize}
\begin{itemize}
\item {Utilização:Ext.}
\end{itemize}
\begin{itemize}
\item {Utilização:Fam.}
\end{itemize}
\begin{itemize}
\item {Proveniência:(Lat. \textunderscore medietas\textunderscore )}
\end{itemize}
Cada uma das duas partes iguaes, em que se divide um todo.
Parte, proximamente igual a metade.
Espôsa, em relação ao marido: \textunderscore a minha cara metade\textunderscore .
\section{Metafisga}
\begin{itemize}
\item {Grp. gram.:f.}
\end{itemize}
\begin{itemize}
\item {Utilização:Prov.}
\end{itemize}
Astúcia; esperteza.
(Por \textunderscore metaphysica\textunderscore )
\section{Metagalato}
\begin{itemize}
\item {Grp. gram.:m.}
\end{itemize}
\begin{itemize}
\item {Proveniência:(De \textunderscore metagálico\textunderscore )}
\end{itemize}
Combinação do ácido metagálico com uma base salificável.
\section{Metagálico}
\begin{itemize}
\item {Grp. gram.:adj.}
\end{itemize}
\begin{itemize}
\item {Proveniência:(De \textunderscore meta...\textunderscore  + \textunderscore gálico\textunderscore ^2)}
\end{itemize}
Produzido pela acção do fogo sôbre o ácido gálico.
\section{Metagallato}
\begin{itemize}
\item {Grp. gram.:m.}
\end{itemize}
\begin{itemize}
\item {Proveniência:(De \textunderscore metagállico\textunderscore )}
\end{itemize}
Combinação do ácido metagállico com uma base salificável.
\section{Metagállico}
\begin{itemize}
\item {Grp. gram.:adj.}
\end{itemize}
\begin{itemize}
\item {Proveniência:(De \textunderscore meta...\textunderscore  + \textunderscore gállico\textunderscore ^2)}
\end{itemize}
Produzido pela acção do fogo sôbre o ácido gállico.
\section{Metagênese}
\begin{itemize}
\item {Grp. gram.:f.}
\end{itemize}
O mesmo que \textunderscore metagênesis\textunderscore .
\section{Metagênesis}
\begin{itemize}
\item {Grp. gram.:m.  e  f.}
\end{itemize}
\begin{itemize}
\item {Utilização:Bot.}
\end{itemize}
\begin{itemize}
\item {Proveniência:(Do gr. \textunderscore meta\textunderscore  + \textunderscore genesis\textunderscore )}
\end{itemize}
Evolução de cerdo seres vegetaes, que, sem attingirem completo desenvolvimento e sem terem ainda órgãos de reproducção, dão origem a outros seres e morrem, ficando os novos seres encarregados da reproducção dos primeiros.
\section{Metagenético}
\begin{itemize}
\item {Grp. gram.:adj.}
\end{itemize}
Relativo ao metagênesis.
\section{Metagítnias}
\begin{itemize}
\item {Grp. gram.:f. pl.}
\end{itemize}
\begin{itemize}
\item {Proveniência:(Do gr. \textunderscore Metageitnion\textunderscore , epítheto de Apollo)}
\end{itemize}
Festas athenienses em honra de Apollo.
\section{Metagítnio}
\begin{itemize}
\item {Grp. gram.:m.}
\end{itemize}
Segundo mês do anno áttico, no qual se celebravam as metagítnias e que dellas recebeu o nome.
\section{Metagoge}
\begin{itemize}
\item {Grp. gram.:f.}
\end{itemize}
\begin{itemize}
\item {Proveniência:(Gr. \textunderscore metagoge\textunderscore )}
\end{itemize}
Figura de rhetórica, com que se attribuem sentimentos a coisas inanimadas.
\section{Metagrafar}
\begin{itemize}
\item {Grp. gram.:v. t.}
\end{itemize}
\begin{itemize}
\item {Utilização:Neol.}
\end{itemize}
\begin{itemize}
\item {Proveniência:(Do gr. \textunderscore meta\textunderscore  + \textunderscore graphein\textunderscore )}
\end{itemize}
O mesmo que \textunderscore transcrever\textunderscore .
\section{Metagrama}
\begin{itemize}
\item {Grp. gram.:m.}
\end{itemize}
\begin{itemize}
\item {Proveniência:(Do gr. \textunderscore meta\textunderscore  + \textunderscore  gramma\textunderscore )}
\end{itemize}
O mesmo que \textunderscore metaplasmo\textunderscore .
\section{Metagramma}
\begin{itemize}
\item {Grp. gram.:m.}
\end{itemize}
\begin{itemize}
\item {Proveniência:(Do gr. \textunderscore meta\textunderscore  + \textunderscore  gramma\textunderscore )}
\end{itemize}
O mesmo que \textunderscore metaplasmo\textunderscore .
\section{Metagraphar}
\begin{itemize}
\item {Grp. gram.:v. t.}
\end{itemize}
\begin{itemize}
\item {Utilização:Neol.}
\end{itemize}
\begin{itemize}
\item {Proveniência:(Do gr. \textunderscore meta\textunderscore  + \textunderscore graphein\textunderscore )}
\end{itemize}
O mesmo que \textunderscore transcrever\textunderscore .
\section{Metal}
\begin{itemize}
\item {Grp. gram.:m.}
\end{itemize}
\begin{itemize}
\item {Utilização:Heráld.}
\end{itemize}
\begin{itemize}
\item {Utilização:Fig.}
\end{itemize}
\begin{itemize}
\item {Proveniência:(Lat. \textunderscore metallum\textunderscore )}
\end{itemize}
Qualquer corpo mineral, opaco, pesado, que a natureza apresenta entre substâncias terrosas ou associado a ellas.
Côr branca ou amarela, no campo do escudo.
Dinheiro: \textunderscore pagar em metal\textunderscore .
Som, timbre: \textunderscore metal de voz\textunderscore .
\section{Metafalange}
\begin{itemize}
\item {Grp. gram.:f.}
\end{itemize}
\begin{itemize}
\item {Utilização:Anat.}
\end{itemize}
\begin{itemize}
\item {Proveniência:(De \textunderscore meta...\textunderscore  + \textunderscore phalange\textunderscore )}
\end{itemize}
Peça distal do dedo, ou falangeta.
\section{Metafalangeal}
\begin{itemize}
\item {Grp. gram.:adj.}
\end{itemize}
Relativo á metafalange.
\section{Metafisica}
\begin{itemize}
\item {Grp. gram.:f.}
\end{itemize}
\begin{itemize}
\item {Utilização:Fig.}
\end{itemize}
\begin{itemize}
\item {Proveniência:(De \textunderscore metafisico\textunderscore )}
\end{itemize}
Doutrina da essência das coisas.
Inventário sistemático de todos os conhecimentos provenientes da razão pura.
Ciência dos princípios.
Teoria das ideias.
Subtileza ou transcendência no discorrer.
\section{Metafisicamente}
\begin{itemize}
\item {Grp. gram.:adv.}
\end{itemize}
De modo metafísico.
\section{Metafisicar}
\begin{itemize}
\item {Grp. gram.:v. t.}
\end{itemize}
Tornar metafísico:«\textunderscore ...tudo que se póde metafisicar.\textunderscore »\textunderscore Arte de Furtar\textunderscore , 275.
\section{Metafisicismo}
\begin{itemize}
\item {Grp. gram.:m.}
\end{itemize}
A influência ou o domínio da Metafísica; o requinte da Metafísica.
\section{Metafísico}
\begin{itemize}
\item {Grp. gram.:adj.}
\end{itemize}
\begin{itemize}
\item {Utilização:Fig.}
\end{itemize}
\begin{itemize}
\item {Grp. gram.:M.}
\end{itemize}
\begin{itemize}
\item {Utilização:Fig.}
\end{itemize}
\begin{itemize}
\item {Proveniência:(Gr. \textunderscore metaphusikos\textunderscore )}
\end{itemize}
Relativo á Metafísica.
Transcendente.
Nebuloso.
Que é de difícil compreensão.
Aquele que é versado em Metafísica.
Aquele que tem teorias ou argumentos muito nebulosos.
\section{Metafonia}
\begin{itemize}
\item {Grp. gram.:f.}
\end{itemize}
\begin{itemize}
\item {Utilização:Gram.}
\end{itemize}
Influência de uma vogal final átona sôbre a vogal radical tónica, como em \textunderscore dêvo\textunderscore , comparado com \textunderscore déves\textunderscore .
\section{Metafónico}
\begin{itemize}
\item {Grp. gram.:adj.}
\end{itemize}
Relativo á metafonia.
\section{Metáfora}
\begin{itemize}
\item {Grp. gram.:f.}
\end{itemize}
\begin{itemize}
\item {Utilização:Restrict.}
\end{itemize}
\begin{itemize}
\item {Proveniência:(Lat. \textunderscore metaphora\textunderscore )}
\end{itemize}
Primitivamente, o mesmo que \textunderscore tropo\textunderscore .
Figura de Retórica, em que a significação natural de uma palavra é substituída por outra, que lhe não é aplicável, senão por comparação subentendida: \textunderscore a luz do espírito; a flôr dos anos\textunderscore .
\section{Metaforicamente}
\begin{itemize}
\item {Grp. gram.:adv.}
\end{itemize}
De modo metafórico.
Com emprego do metáfora.
\section{Metafórico}
\begin{itemize}
\item {Grp. gram.:adj.}
\end{itemize}
Relativo á metáfora; figurado: \textunderscore estilo metafórico\textunderscore .
\section{Metaforista}
\begin{itemize}
\item {Grp. gram.:m.}
\end{itemize}
\begin{itemize}
\item {Grp. gram.:Pl.}
\end{itemize}
\begin{itemize}
\item {Proveniência:(De \textunderscore metáfora\textunderscore )}
\end{itemize}
Aquele que emprega metáforas.
Herejes, que consideravam metáfora o dogma da presença real de Cristo.
\section{Metaforizar}
\begin{itemize}
\item {Grp. gram.:v. t.}
\end{itemize}
\begin{itemize}
\item {Proveniência:(De \textunderscore metáfora\textunderscore )}
\end{itemize}
Exprimir metaforicamente. Cf. Camillo, \textunderscore Mar. da Fonte\textunderscore , 163 e 412.
\section{Metafosfato}
\begin{itemize}
\item {Grp. gram.:m.}
\end{itemize}
Sal, produzido pela combinação do ácido metafosfórico com uma base.
(Cp. \textunderscore metafosfórico\textunderscore )
\section{Metafosfórico}
\begin{itemize}
\item {Grp. gram.:adj.}
\end{itemize}
\begin{itemize}
\item {Proveniência:(De \textunderscore meta...\textunderscore  + \textunderscore fosfórico\textunderscore )}
\end{itemize}
Diz-se de um dos ácidos do fósforo.
\section{Metáfrase}
\begin{itemize}
\item {Grp. gram.:f.}
\end{itemize}
\begin{itemize}
\item {Proveniência:(Lat. \textunderscore metaphrasis\textunderscore )}
\end{itemize}
Comentário ou interpretação simples ou natural de uma frase figurada ou de um escritor original.
O mesmo que \textunderscore paráfrase\textunderscore .
\section{Metafrasta}
\begin{itemize}
\item {Grp. gram.:m.}
\end{itemize}
\begin{itemize}
\item {Proveniência:(Gr. \textunderscore metaphrastes\textunderscore )}
\end{itemize}
Aquele que faz metáfrases.
\section{Metafrástico}
\begin{itemize}
\item {Grp. gram.:adj.}
\end{itemize}
\begin{itemize}
\item {Proveniência:(Gr. \textunderscore metaphrastikos\textunderscore )}
\end{itemize}
Relativo á metáfrase.
Interpretado literalmente.
\section{Metalbumina}
\begin{itemize}
\item {Grp. gram.:f.}
\end{itemize}
\begin{itemize}
\item {Proveniência:(De \textunderscore metal\textunderscore  + \textunderscore albumina\textunderscore )}
\end{itemize}
Variedade de albumina, que se encontra nas exsudações hydrópicas.
\section{Metalepse}
\begin{itemize}
\item {Grp. gram.:f.}
\end{itemize}
\begin{itemize}
\item {Proveniência:(Lat. \textunderscore metalepsis\textunderscore )}
\end{itemize}
Figura de rhetórica, em que se toma o antecedente pelo consequente, e vice-versa.
\section{Metalépsia}
\begin{itemize}
\item {Grp. gram.:f.}
\end{itemize}
\begin{itemize}
\item {Utilização:Chím.}
\end{itemize}
Theoria das substituições.
(Cp. \textunderscore metalepse\textunderscore )
\section{Metaléptico}
\begin{itemize}
\item {Grp. gram.:adj.}
\end{itemize}
Relativo á metalepse; em que há metalepse.
\section{Metalescência}
\begin{itemize}
\item {Grp. gram.:f.}
\end{itemize}
Qualidade de metalescente.
\section{Metalescente}
\begin{itemize}
\item {Grp. gram.:adj.}
\end{itemize}
\begin{itemize}
\item {Utilização:P. us.}
\end{itemize}
\begin{itemize}
\item {Proveniência:(Do lat. \textunderscore meiallum\textunderscore )}
\end{itemize}
Cuja superfície apresenta côres metálicas.
\section{Metalicidade}
\begin{itemize}
\item {Grp. gram.:f.}
\end{itemize}
Qualidade de metálico.
Conjunto de propriedades que caracterizam um metal.
\section{Metálico}
\begin{itemize}
\item {Grp. gram.:adj.}
\end{itemize}
\begin{itemize}
\item {Grp. gram.:M.}
\end{itemize}
\begin{itemize}
\item {Proveniência:(Lat. \textunderscore metallicus\textunderscore )}
\end{itemize}
Relativo ao metal; feito de metal.
Dinheiro em metal sonante.
\section{Metalífero}
\begin{itemize}
\item {Grp. gram.:adj.}
\end{itemize}
\begin{itemize}
\item {Proveniência:(Lat. \textunderscore metallifer\textunderscore )}
\end{itemize}
Que contém metal: \textunderscore terrenos metalíferos\textunderscore .
\section{Metalificação}
\begin{itemize}
\item {Grp. gram.:f.}
\end{itemize}
\begin{itemize}
\item {Proveniência:(Do lat. \textunderscore mettallum\textunderscore  + \textunderscore facere\textunderscore )}
\end{itemize}
Acto ou efeito de reduzir uma substância ao estado metálico.
Formação natural dos metaes na terra.
\section{Metaliforme}
\begin{itemize}
\item {Grp. gram.:adj.}
\end{itemize}
\begin{itemize}
\item {Proveniência:(Do lat. \textunderscore metallum\textunderscore  + \textunderscore forma\textunderscore )}
\end{itemize}
Que tem aparência de metal.
\section{Metalímneo}
\begin{itemize}
\item {Grp. gram.:adj.}
\end{itemize}
\begin{itemize}
\item {Utilização:Geol.}
\end{itemize}
\begin{itemize}
\item {Proveniência:(Do gr. \textunderscore meta\textunderscore  + \textunderscore limne\textunderscore )}
\end{itemize}
Diz-se dos depósitos do água doce, que só appareceram depois da formação do calcário marinho.
\section{Metalino}
\begin{itemize}
\item {Grp. gram.:adj.}
\end{itemize}
\begin{itemize}
\item {Proveniência:(Do lat. \textunderscore metallum\textunderscore )}
\end{itemize}
Que tem côr ou aparência metálica; metálico.
\section{Metalismo}
\begin{itemize}
\item {Grp. gram.:m.}
\end{itemize}
\begin{itemize}
\item {Proveniência:(Do lat. \textunderscore metallum\textunderscore )}
\end{itemize}
Em Economia Política, a representação do dinheiro por metal cunhado.
\section{Metalista}
\begin{itemize}
\item {Grp. gram.:m.}
\end{itemize}
\begin{itemize}
\item {Proveniência:(Do lat. \textunderscore metallum\textunderscore )}
\end{itemize}
Homem perito em metalurgia.
Engenheiro de minas.
\section{Metalização}
\begin{itemize}
\item {Grp. gram.:f.}
\end{itemize}
Acto ou efeito de \textunderscore metalizar\textunderscore .
\section{Metalizar}
\begin{itemize}
\item {Grp. gram.:v. i.}
\end{itemize}
\begin{itemize}
\item {Proveniência:(Do lat. \textunderscore metallum\textunderscore )}
\end{itemize}
Tornar puro (um metal).
Transformar em metal.
Guarnecer ou cobrir com uma ligeira capa de metal.
\section{Metallescência}
\begin{itemize}
\item {Grp. gram.:f.}
\end{itemize}
Qualidade de metallescente.
\section{Metallescente}
\begin{itemize}
\item {Grp. gram.:adj.}
\end{itemize}
\begin{itemize}
\item {Utilização:P. us.}
\end{itemize}
\begin{itemize}
\item {Proveniência:(Do lat. \textunderscore meiallum\textunderscore )}
\end{itemize}
Cuja superfície apresenta côres metállicas.
\section{Metallicidade}
\begin{itemize}
\item {Grp. gram.:f.}
\end{itemize}
Qualidade de metállico.
Conjunto de propriedades que caracterizam um metal.
\section{Metállico}
\begin{itemize}
\item {Grp. gram.:adj.}
\end{itemize}
\begin{itemize}
\item {Grp. gram.:M.}
\end{itemize}
\begin{itemize}
\item {Proveniência:(Lat. \textunderscore metallicus\textunderscore )}
\end{itemize}
Relativo ao metal; feito de metal.
Dinheiro em metal sonante.
\section{Metallífero}
\begin{itemize}
\item {Grp. gram.:adj.}
\end{itemize}
\begin{itemize}
\item {Proveniência:(Lat. \textunderscore metallifer\textunderscore )}
\end{itemize}
Que contém metal: \textunderscore terrenos metallíferos\textunderscore .
\section{Metallificação}
\begin{itemize}
\item {Grp. gram.:f.}
\end{itemize}
\begin{itemize}
\item {Proveniência:(Do lat. \textunderscore mettallum\textunderscore  + \textunderscore facere\textunderscore )}
\end{itemize}
Acto ou effeito de reduzir uma substância ao estado metállico.
Formação natural dos metaes na terra.
\section{Metalliforme}
\begin{itemize}
\item {Grp. gram.:adj.}
\end{itemize}
\begin{itemize}
\item {Proveniência:(Do lat. \textunderscore metallum\textunderscore  + \textunderscore forma\textunderscore )}
\end{itemize}
Que tem apparência de metal.
\section{Metallino}
\begin{itemize}
\item {Grp. gram.:adj.}
\end{itemize}
\begin{itemize}
\item {Proveniência:(Do lat. \textunderscore metallum\textunderscore )}
\end{itemize}
Que tem côr ou apparência metállica; metállico.
\section{Metallismo}
\begin{itemize}
\item {Grp. gram.:m.}
\end{itemize}
\begin{itemize}
\item {Proveniência:(Do lat. \textunderscore metallum\textunderscore )}
\end{itemize}
Em Economia Política, a representação do dinheiro por metal cunhado.
\section{Metallista}
\begin{itemize}
\item {Grp. gram.:m.}
\end{itemize}
\begin{itemize}
\item {Proveniência:(Do lat. \textunderscore metallum\textunderscore )}
\end{itemize}
Homem perito em metallurgia.
Engenheiro de minas.
\section{Metallização}
\begin{itemize}
\item {Grp. gram.:f.}
\end{itemize}
Acto ou effeito de \textunderscore metallizar\textunderscore .
\section{Metallizar}
\begin{itemize}
\item {Grp. gram.:v. i.}
\end{itemize}
\begin{itemize}
\item {Proveniência:(Do lat. \textunderscore metallum\textunderscore )}
\end{itemize}
Tornar puro (um metal).
Transformar em metal.
Guarnecer ou cobrir com uma ligeira capa de metal.
\section{Metallochímica}
\begin{itemize}
\item {fónica:qui}
\end{itemize}
\begin{itemize}
\item {Grp. gram.:f.}
\end{itemize}
Parte da Chímica, em que se trata dos metaes.
\section{Metallochímico}
\begin{itemize}
\item {fónica:qui}
\end{itemize}
\begin{itemize}
\item {Grp. gram.:adj.}
\end{itemize}
Relativo á metallochímica.
\section{Metallographia}
\begin{itemize}
\item {Grp. gram.:f.}
\end{itemize}
Descripção dos metaes.
Sciência, que tem por objecto os metaes.
(Cp. \textunderscore metallógrapho\textunderscore )
\section{Metallográphico}
\begin{itemize}
\item {Grp. gram.:adj.}
\end{itemize}
Relativo á metallographia.
\section{Metallógrapho}
\begin{itemize}
\item {Grp. gram.:m.}
\end{itemize}
\begin{itemize}
\item {Proveniência:(Do gr. \textunderscore metallon\textunderscore  + \textunderscore graphein\textunderscore )}
\end{itemize}
Aquelle que se occupa de metallographia.
\section{Metallóide}
\begin{itemize}
\item {Grp. gram.:adj.}
\end{itemize}
\begin{itemize}
\item {Grp. gram.:M.}
\end{itemize}
\begin{itemize}
\item {Proveniência:(Do gr. \textunderscore metallon\textunderscore  + \textunderscore eidos\textunderscore )}
\end{itemize}
Semelhante a um metal pelas suas propriedades ou brilho.
Qualquer corpo simples que não tem todos os caracteres physicos dos metaes propriamente ditos.
\section{Metallologia}
\begin{itemize}
\item {Grp. gram.:f.}
\end{itemize}
\begin{itemize}
\item {Proveniência:(Do gr. \textunderscore metallon\textunderscore  + \textunderscore logos\textunderscore )}
\end{itemize}
O mesmo ou melhor que \textunderscore Mineralogia\textunderscore .
\section{Metallológico}
\begin{itemize}
\item {Grp. gram.:adj.}
\end{itemize}
Relativo á Metallologia.
\section{Metallologista}
\begin{itemize}
\item {Grp. gram.:m.}
\end{itemize}
Aquelle que trata scientificamente da Metallologia.
\section{Metallophobia}
\begin{itemize}
\item {Grp. gram.:f.}
\end{itemize}
\begin{itemize}
\item {Proveniência:(Do gr. \textunderscore metallon\textunderscore  + \textunderscore phobein\textunderscore )}
\end{itemize}
Repugnância mórbida a tocar em metaes.
\section{Metallotherapia}
\begin{itemize}
\item {Grp. gram.:f.}
\end{itemize}
\begin{itemize}
\item {Proveniência:(Do gr. \textunderscore metallon\textunderscore  + \textunderscore therapeia\textunderscore )}
\end{itemize}
Systema do tratamento medicinal, que consiste em applicar sôbre a pelle certas placas metállicas.
\section{Metallurgia}
\begin{itemize}
\item {Grp. gram.:f.}
\end{itemize}
\begin{itemize}
\item {Proveniência:(Gr. \textunderscore metallurgia\textunderscore )}
\end{itemize}
Arte de extrahir os metaes da terra e de os purificar.
\section{Metallúrgico}
\begin{itemize}
\item {Grp. gram.:adj.}
\end{itemize}
\begin{itemize}
\item {Grp. gram.:M.}
\end{itemize}
Relativo á metallurgia.
O mesmo que \textunderscore metallurgista\textunderscore .
\section{Metallurgista}
\begin{itemize}
\item {Grp. gram.:m.}
\end{itemize}
Aquelle que se occupa da metallurgia.
\section{Metalofobia}
\begin{itemize}
\item {Grp. gram.:f.}
\end{itemize}
\begin{itemize}
\item {Proveniência:(Do gr. \textunderscore metallon\textunderscore  + \textunderscore phobein\textunderscore )}
\end{itemize}
Repugnância mórbida a tocar em metaes.
\section{Metalografia}
\begin{itemize}
\item {Grp. gram.:f.}
\end{itemize}
Descripção dos metaes.
Ciência, que tem por objecto os metaes.
(Cp. \textunderscore metalógrafo\textunderscore )
\section{Metalográfico}
\begin{itemize}
\item {Grp. gram.:adj.}
\end{itemize}
Relativo á metalografia.
\section{Metalógrafo}
\begin{itemize}
\item {Grp. gram.:m.}
\end{itemize}
\begin{itemize}
\item {Proveniência:(Do gr. \textunderscore metallon\textunderscore  + \textunderscore graphein\textunderscore )}
\end{itemize}
Aquele que se ocupa de metalografia.
\section{Metalóide}
\begin{itemize}
\item {Grp. gram.:adj.}
\end{itemize}
\begin{itemize}
\item {Grp. gram.:M.}
\end{itemize}
\begin{itemize}
\item {Proveniência:(Do gr. \textunderscore metallon\textunderscore  + \textunderscore eidos\textunderscore )}
\end{itemize}
Semelhante a um metal pelas suas propriedades ou brilho.
Qualquer corpo simples que não tem todos os caracteres fisicos dos metaes propriamente ditos.
\section{Metalologia}
\begin{itemize}
\item {Grp. gram.:f.}
\end{itemize}
\begin{itemize}
\item {Proveniência:(Do gr. \textunderscore metallon\textunderscore  + \textunderscore logos\textunderscore )}
\end{itemize}
O mesmo ou melhor que \textunderscore Mineralogia\textunderscore .
\section{Metalológico}
\begin{itemize}
\item {Grp. gram.:adj.}
\end{itemize}
Relativo á Metalologia.
\section{Metalologista}
\begin{itemize}
\item {Grp. gram.:m.}
\end{itemize}
Aquele que trata cientificamente da Metalologia.
\section{Metaloquímica}
\begin{itemize}
\item {Grp. gram.:f.}
\end{itemize}
Parte da Química, em que se trata dos metaes.
\section{Metaloquímico}
\begin{itemize}
\item {Grp. gram.:adj.}
\end{itemize}
Relativo á metaloquímica.
\section{Metaloterapia}
\begin{itemize}
\item {Grp. gram.:f.}
\end{itemize}
\begin{itemize}
\item {Proveniência:(Do gr. \textunderscore metallon\textunderscore  + \textunderscore therapeia\textunderscore )}
\end{itemize}
Sistema do tratamento medicinal, que consiste em aplicar sôbre a pele certas placas metálicas.
\section{Metalurgia}
\begin{itemize}
\item {Grp. gram.:f.}
\end{itemize}
\begin{itemize}
\item {Proveniência:(Gr. \textunderscore metallurgia\textunderscore )}
\end{itemize}
Arte de extrair os metaes da terra e de os purificar.
\section{Metalúrgico}
\begin{itemize}
\item {Grp. gram.:adj.}
\end{itemize}
\begin{itemize}
\item {Grp. gram.:M.}
\end{itemize}
Relativo á metalurgia.
O mesmo que \textunderscore metalurgista\textunderscore .
\section{Metalurgista}
\begin{itemize}
\item {Grp. gram.:m.}
\end{itemize}
Aquele que se ocupa da metalurgia.
\section{Metameria}
\begin{itemize}
\item {Grp. gram.:f.}
\end{itemize}
Qualidade de metâmero.
\section{Metâmero}
\begin{itemize}
\item {Grp. gram.:M.}
\end{itemize}
\begin{itemize}
\item {Proveniência:(Do gr. \textunderscore meta\textunderscore  + \textunderscore meros\textunderscore )}
\end{itemize}
Diz-se de um corpo, que é isómero de outro.
Cada um dos anéis de um verme.
\section{Metamilênio}
\begin{itemize}
\item {Grp. gram.:m.}
\end{itemize}
\begin{itemize}
\item {Utilização:Chím.}
\end{itemize}
\begin{itemize}
\item {Proveniência:(De \textunderscore meta...\textunderscore  + \textunderscore amylênio\textunderscore )}
\end{itemize}
Producto da decomposição do amilênio pela destilação.
\section{Metamomo}
\begin{itemize}
\item {Grp. gram.:m.}
\end{itemize}
Espaço entre dentículos, em architectura.--Vejo a palavra num diccion. de architectura, mas cuido que terá havido êrro de escrita, em vez de \textunderscore metátomo\textunderscore . Cp. \textunderscore metátomo\textunderscore .
\section{Metamórfico}
\begin{itemize}
\item {Grp. gram.:adj.}
\end{itemize}
\begin{itemize}
\item {Proveniência:(Do gr. \textunderscore meta\textunderscore  + \textunderscore morphe\textunderscore )}
\end{itemize}
Relativo ás metamorfosos dos insectos.
Relativo a rochas, que se supõem alteradas por causas plutónicas.
\section{Metamorfismo}
\begin{itemize}
\item {Grp. gram.:m.}
\end{itemize}
\begin{itemize}
\item {Proveniência:(Do gr. \textunderscore meta\textunderscore  + \textunderscore morphe\textunderscore )}
\end{itemize}
Teoria da transformação dos terrenos pela acção do calor.
Faculdade de transformar-se; transformação.
\section{Metamorfopse}
\begin{itemize}
\item {Grp. gram.:f.}
\end{itemize}
\begin{itemize}
\item {Proveniência:(Do gr. \textunderscore metamorphosis\textunderscore  + \textunderscore ops\textunderscore )}
\end{itemize}
Estado mórbido dos que vêem os objectos deformados.
\section{Metamorfopsia}
\begin{itemize}
\item {Grp. gram.:f.}
\end{itemize}
\begin{itemize}
\item {Proveniência:(Do gr. \textunderscore metamorphosis\textunderscore  + \textunderscore ops\textunderscore )}
\end{itemize}
Estado mórbido dos que vêem os objectos deformados.
\section{Metamorfóptico}
\begin{itemize}
\item {Grp. gram.:adj.}
\end{itemize}
Relativo a metamorfopse; que sofre metamorfopse.
\section{Metamorfose}
\begin{itemize}
\item {Grp. gram.:f.}
\end{itemize}
\begin{itemize}
\item {Utilização:Bot.}
\end{itemize}
\begin{itemize}
\item {Proveniência:(Lat. \textunderscore metamorphosis\textunderscore )}
\end{itemize}
Transformação de um objecto noutro, operada pelos deuses, segundo a crença dos Pagãos.
Transformação de substâncias, operada por causas naturaes.
Mudança, a que estão sujeitos os insectos e os batrácios e que os faz passar por estados muito diferentes.
Mudança, manifestada por pessôas, no vestir, no carácter, na fortuna, nos costumes, etc.
Mudança, transformação.
Planta malvácea de Cabo-Verde, (\textunderscore hibíscus notabilis\textunderscore , Lin).
\section{Metamorfosear}
\begin{itemize}
\item {Grp. gram.:v. t.}
\end{itemize}
\begin{itemize}
\item {Proveniência:(De \textunderscore metamorfose\textunderscore )}
\end{itemize}
Transformar; mudar a fórma de.
Alterar os caracteres de.
\section{Metamórphico}
\begin{itemize}
\item {Grp. gram.:adj.}
\end{itemize}
\begin{itemize}
\item {Proveniência:(Do gr. \textunderscore meta\textunderscore  + \textunderscore morphe\textunderscore )}
\end{itemize}
Relativo ás metamorphosos dos insectos.
Relativo a rochas, que se suppõem alteradas por causas plutónicas.
\section{Metamorphismo}
\begin{itemize}
\item {Grp. gram.:m.}
\end{itemize}
\begin{itemize}
\item {Proveniência:(Do gr. \textunderscore meta\textunderscore  + \textunderscore morphe\textunderscore )}
\end{itemize}
Theoria da transformação dos terrenos pela acção do calor.
Faculdade de transformar-se; transformação.
\section{Metamorphopse}
\begin{itemize}
\item {Grp. gram.:f.}
\end{itemize}
\begin{itemize}
\item {Proveniência:(Do gr. \textunderscore metamorphosis\textunderscore  + \textunderscore ops\textunderscore )}
\end{itemize}
Estado mórbido dos que vêem os objectos deformados.
\section{Metamorphopsia}
\begin{itemize}
\item {Grp. gram.:f.}
\end{itemize}
\begin{itemize}
\item {Proveniência:(Do gr. \textunderscore metamorphosis\textunderscore  + \textunderscore ops\textunderscore )}
\end{itemize}
Estado mórbido dos que vêem os objectos deformados.
\section{Metamorphóptico}
\begin{itemize}
\item {Grp. gram.:adj.}
\end{itemize}
Relativo a metamorphopse; que soffre metamorphopse.
\section{Metamorphose}
\begin{itemize}
\item {Grp. gram.:f.}
\end{itemize}
\begin{itemize}
\item {Utilização:Bot.}
\end{itemize}
\begin{itemize}
\item {Proveniência:(Lat. \textunderscore metamorphosis\textunderscore )}
\end{itemize}
Transformação de um objecto noutro, operada pelos deuses, segundo a crença dos Pagãos.
Transformação de substâncias, operada por causas naturaes.
Mudança, a que estão sujeitos os insectos e os batrácios e que os faz passar por estados muito differentes.
Mudança, manifestada por pessôas, no vestir, no carácter, na fortuna, nos costumes, etc.
Mudança, transformação.
Planta malvácea de Cabo-Verde, (\textunderscore hibíscus notabilis\textunderscore , Lin).
\section{Metamorphosear}
\begin{itemize}
\item {Grp. gram.:v. t.}
\end{itemize}
\begin{itemize}
\item {Proveniência:(De \textunderscore metamorphose\textunderscore )}
\end{itemize}
Transformar; mudar a fórma de.
Alterar os caracteres de.
\section{Metamylênio}
\begin{itemize}
\item {Grp. gram.:m.}
\end{itemize}
\begin{itemize}
\item {Utilização:Chím.}
\end{itemize}
\begin{itemize}
\item {Proveniência:(De \textunderscore meta...\textunderscore  + \textunderscore amylênio\textunderscore )}
\end{itemize}
Producto da decomposição do amylênio pela destillação.
\section{Metapectico}
\begin{itemize}
\item {Grp. gram.:adj.}
\end{itemize}
\begin{itemize}
\item {Proveniência:(De \textunderscore meta...\textunderscore  + \textunderscore péctico\textunderscore )}
\end{itemize}
Diz-se de um ácido, que se fórma á custa da pectina, exposta muitos dias ao ar, ou posta era contacto com a pectose. Cf. \textunderscore Techn. Rur.\textunderscore , 20.
\section{Metapectina}
\begin{itemize}
\item {Grp. gram.:f.}
\end{itemize}
\begin{itemize}
\item {Utilização:Chím.}
\end{itemize}
\begin{itemize}
\item {Proveniência:(De \textunderscore meta...\textunderscore  + \textunderscore pectina\textunderscore )}
\end{itemize}
Corpo isómero com a pectina e a parapectina.
\section{Metaphalange}
\begin{itemize}
\item {Grp. gram.:f.}
\end{itemize}
\begin{itemize}
\item {Utilização:Anat.}
\end{itemize}
\begin{itemize}
\item {Proveniência:(De \textunderscore meta...\textunderscore  + \textunderscore phalange\textunderscore )}
\end{itemize}
Peça distal do dedo, ou phalangeta.
\section{Metaphalangeal}
\begin{itemize}
\item {Grp. gram.:adj.}
\end{itemize}
Relativo á metaphalange.
\section{Metaphonia}
\begin{itemize}
\item {Grp. gram.:f.}
\end{itemize}
\begin{itemize}
\item {Utilização:Gram.}
\end{itemize}
Influência de uma vogal final átona sôbre a vogal radical tónica, como em \textunderscore dêvo\textunderscore , comparado com \textunderscore déves\textunderscore .
\section{Metaphónico}
\begin{itemize}
\item {Grp. gram.:adj.}
\end{itemize}
Relativo á metaphonia.
\section{Metáphora}
\begin{itemize}
\item {Grp. gram.:f.}
\end{itemize}
\begin{itemize}
\item {Utilização:Restrict.}
\end{itemize}
\begin{itemize}
\item {Proveniência:(Lat. \textunderscore metaphora\textunderscore )}
\end{itemize}
Primitivamente, o mesmo que \textunderscore tropo\textunderscore .
Figura de Rhetórica, em que a significação natural de uma palavra é substituída por outra, que lhe não é applicável, senão por comparação subentendida: \textunderscore a luz do espírito; a flôr dos annos\textunderscore .
\section{Metaphoricamente}
\begin{itemize}
\item {Grp. gram.:adv.}
\end{itemize}
De modo metaphórico.
Com emprego do metáphora.
\section{Metaphórico}
\begin{itemize}
\item {Grp. gram.:adj.}
\end{itemize}
Relativo á metáphora; figurado: \textunderscore estilo metaphórico\textunderscore .
\section{Metaphorista}
\begin{itemize}
\item {Grp. gram.:m.}
\end{itemize}
\begin{itemize}
\item {Grp. gram.:Pl.}
\end{itemize}
\begin{itemize}
\item {Proveniência:(De \textunderscore metáphora\textunderscore )}
\end{itemize}
Aquelle que emprega metáphoras.
Herejes, que consideravam metáphora o dogma da presença real de Christo.
\section{Metaphorizar}
\begin{itemize}
\item {Grp. gram.:v. t.}
\end{itemize}
\begin{itemize}
\item {Proveniência:(De \textunderscore metáphora\textunderscore )}
\end{itemize}
Exprimir metaphoricamente. Cf. Camillo, \textunderscore Mar. da Fonte\textunderscore , 163 e 412.
\section{Metaphosphato}
\begin{itemize}
\item {Grp. gram.:m.}
\end{itemize}
Sal, produzido pela combinação do ácido metaphosphórico com uma base.
(Cp. \textunderscore metaphosphórico\textunderscore )
\section{Metaphosphórico}
\begin{itemize}
\item {Grp. gram.:adj.}
\end{itemize}
\begin{itemize}
\item {Proveniência:(De \textunderscore meta...\textunderscore  + \textunderscore phosphórico\textunderscore )}
\end{itemize}
Diz-se de um dos ácidos do phósphoro.
\section{Metáphrase}
\begin{itemize}
\item {Grp. gram.:f.}
\end{itemize}
\begin{itemize}
\item {Proveniência:(Lat. \textunderscore metaphrasis\textunderscore )}
\end{itemize}
Commentário ou interpretação simples ou natural de uma phrase figurada ou de um escritor original.
O mesmo que \textunderscore paráphrase\textunderscore .
\section{Metaphrasta}
\begin{itemize}
\item {Grp. gram.:m.}
\end{itemize}
\begin{itemize}
\item {Proveniência:(Gr. \textunderscore metaphrastes\textunderscore )}
\end{itemize}
Aquelle que faz metáphrases.
\section{Metaphrástico}
\begin{itemize}
\item {Grp. gram.:adj.}
\end{itemize}
\begin{itemize}
\item {Proveniência:(Gr. \textunderscore metaphrastikos\textunderscore )}
\end{itemize}
Relativo á metáphrase.
Interpretado literalmente.
\section{Metaphysica}
\begin{itemize}
\item {Grp. gram.:f.}
\end{itemize}
\begin{itemize}
\item {Utilização:Fig.}
\end{itemize}
\begin{itemize}
\item {Proveniência:(De \textunderscore metaphysico\textunderscore )}
\end{itemize}
Doutrina da essência das coisas.
Inventário systemático de todos os conhecimentos provenientes da razão pura.
Sciência dos princípios.
Theoria das ideias.
Subtileza ou transcendência no discorrer.
\section{Metaphysicamente}
\begin{itemize}
\item {Grp. gram.:adv.}
\end{itemize}
De modo metaphýsico.
\section{Metaphysicar}
\begin{itemize}
\item {Grp. gram.:v. t.}
\end{itemize}
Tornar metaphýsico:«\textunderscore ...tudo que se póde metaphysicar.\textunderscore »\textunderscore Arte de Furtar\textunderscore , 275.
\section{Metaphysicismo}
\begin{itemize}
\item {Grp. gram.:m.}
\end{itemize}
A influência ou o domínio da Metaphýsica; o requinte da Metaphýsica.
\section{Metaphýsico}
\begin{itemize}
\item {Grp. gram.:adj.}
\end{itemize}
\begin{itemize}
\item {Utilização:Fig.}
\end{itemize}
\begin{itemize}
\item {Grp. gram.:M.}
\end{itemize}
\begin{itemize}
\item {Utilização:Fig.}
\end{itemize}
\begin{itemize}
\item {Proveniência:(Gr. \textunderscore metaphusikos\textunderscore )}
\end{itemize}
Relativo á Metaphýsica.
Transcendente.
Nebuloso.
Que é de diffícil comprehensão.
Aquelle que é versado em Metaphýsica.
Aquelle que tem theorias ou argumentos muito nebulosos.
\section{Metaplasma}
\begin{itemize}
\item {Grp. gram.:f.}
\end{itemize}
O mesmo que \textunderscore metaplasmo\textunderscore .
\section{Metaplasmo}
\begin{itemize}
\item {Grp. gram.:m.}
\end{itemize}
\begin{itemize}
\item {Utilização:Gram.}
\end{itemize}
\begin{itemize}
\item {Proveniência:(Lat. \textunderscore metaplasmus\textunderscore )}
\end{itemize}
Alteração na estructura das palavras, tirando ou accrescentaudo ou alterando letras.
\section{Metaplástico}
\begin{itemize}
\item {Grp. gram.:adj.}
\end{itemize}
\begin{itemize}
\item {Proveniência:(Do gr. \textunderscore meta\textunderscore  + \textunderscore plassein\textunderscore )}
\end{itemize}
Relativo ao metaplasmo; em que há metaplasmo.
\section{Metapsíquica}
\begin{itemize}
\item {Grp. gram.:f.}
\end{itemize}
\begin{itemize}
\item {Proveniência:(De \textunderscore meta...\textunderscore  + \textunderscore psýchico\textunderscore )}
\end{itemize}
Estudo dos fenómenos psíquicos anormaes, como a clarividência, a telepatia, a visão dupla.
\section{Metapsýchica}
\begin{itemize}
\item {fónica:qui}
\end{itemize}
\begin{itemize}
\item {Grp. gram.:f.}
\end{itemize}
\begin{itemize}
\item {Proveniência:(De \textunderscore meta...\textunderscore  + \textunderscore psýchico\textunderscore )}
\end{itemize}
Estudo dos phenómenos psýchicos anormaes, como a clarividência, a telepathia, a visão dupla.
\section{Metaptose}
\begin{itemize}
\item {Grp. gram.:f.}
\end{itemize}
\begin{itemize}
\item {Utilização:Med.}
\end{itemize}
\begin{itemize}
\item {Proveniência:(Do gr. \textunderscore meta\textunderscore  + \textunderscore ptosis\textunderscore )}
\end{itemize}
Mudança na séde ou na fórma de uma doença.
\section{Metara}
\begin{itemize}
\item {Grp. gram.:f.}
\end{itemize}
\begin{itemize}
\item {Utilização:Bras}
\end{itemize}
Rodela de pedra, que os Tupinambás usavam no beiço inferior, previamente furado desde a infância.
O mesmo que \textunderscore botoque\textunderscore .
\section{Metástase}
\begin{itemize}
\item {Grp. gram.:f.}
\end{itemize}
\begin{itemize}
\item {Proveniência:(Gr. \textunderscore metastasis\textunderscore )}
\end{itemize}
Figura de rhetórica, em que um orador lança á conta de outrem as coisas a que elle se refere.
Alteração de uma doença, quanto á fórma ou quanto á séde.
\section{Metane}
\begin{itemize}
\item {Grp. gram.:m.}
\end{itemize}
O mesmo que \textunderscore metânio\textunderscore .
\section{Metânio}
\begin{itemize}
\item {Grp. gram.:m.}
\end{itemize}
Gás incolor, produzido por substâncias em putrefacção, e chamado também \textunderscore gás dos pântanos\textunderscore , \textunderscore formena\textunderscore , etc.
\section{Metano}
\begin{itemize}
\item {Grp. gram.:m.}
\end{itemize}
O mesmo que \textunderscore metânio\textunderscore .
\section{Metassíncrise}
\begin{itemize}
\item {Grp. gram.:f.}
\end{itemize}
\begin{itemize}
\item {Utilização:Med.}
\end{itemize}
\begin{itemize}
\item {Utilização:Ant.}
\end{itemize}
\begin{itemize}
\item {Proveniência:(Do gr. \textunderscore meta\textunderscore  + \textunderscore suncrisis\textunderscore )}
\end{itemize}
Renovação ou regeneração de qualquer parte do corpo ou do corpo todo.
\section{Metassincrítico}
\begin{itemize}
\item {Grp. gram.:m.  e  adj.}
\end{itemize}
Dizia-se dos medicamentos, a que se atribuía a metassíncrise.
\section{Metastático}
\begin{itemize}
\item {Grp. gram.:adj.}
\end{itemize}
\begin{itemize}
\item {Proveniência:(Gr. \textunderscore metastatikos\textunderscore )}
\end{itemize}
Relativo á metástase.
\section{Metasterno}
\begin{itemize}
\item {Grp. gram.:m.}
\end{itemize}
\begin{itemize}
\item {Proveniência:(De \textunderscore meta...\textunderscore  + \textunderscore esterno\textunderscore )}
\end{itemize}
A ponta ou extremidade superior do esterno.
\section{Metasýncrise}
\begin{itemize}
\item {fónica:sin}
\end{itemize}
\begin{itemize}
\item {Grp. gram.:f.}
\end{itemize}
\begin{itemize}
\item {Utilização:Med.}
\end{itemize}
\begin{itemize}
\item {Utilização:Ant.}
\end{itemize}
\begin{itemize}
\item {Proveniência:(Do gr. \textunderscore meta\textunderscore  + \textunderscore suncrisis\textunderscore )}
\end{itemize}
Renovação ou regeneração de qualquer parte do corpo ou do corpo todo.
\section{Metasyncrítico}
\begin{itemize}
\item {fónica:sin}
\end{itemize}
\begin{itemize}
\item {Grp. gram.:m.  e  adj.}
\end{itemize}
Dizia-se dos medicamentos, a que se attribuía a metasýncrise.
\section{Metatarsiano}
\begin{itemize}
\item {Grp. gram.:adj.}
\end{itemize}
Relativo ao metatarso.
\section{Metatársico}
\begin{itemize}
\item {Grp. gram.:adj.}
\end{itemize}
Relativo ao metatarso.
\section{Metatarso}
\begin{itemize}
\item {Grp. gram.:m.}
\end{itemize}
\begin{itemize}
\item {Proveniência:(De \textunderscore meta...\textunderscore  + \textunderscore tarso\textunderscore )}
\end{itemize}
Parte do pé, entre o tarso e os dedos.
\section{Metátese}
\begin{itemize}
\item {Grp. gram.:f.}
\end{itemize}
\begin{itemize}
\item {Proveniência:(Lat. \textunderscore metathesis\textunderscore )}
\end{itemize}
Transposição das letras, numa palavra.
Transposição dos termos do um raciocínio.
Operação cirúrgica, com que se transporta de um lugar a causa do uma doença para outro, em que é menos nociva.
\section{Metatético}
\begin{itemize}
\item {Grp. gram.:adj.}
\end{itemize}
Relativo á metátese.
Em que há metátese: \textunderscore pederneira é forma metatética de pedreneira\textunderscore .
\section{Metáthese}
\begin{itemize}
\item {Grp. gram.:f.}
\end{itemize}
\begin{itemize}
\item {Proveniência:(Lat. \textunderscore metathesis\textunderscore )}
\end{itemize}
Transposição das letras, numa palavra.
Transposição dos termos do um raciocínio.
Operação cirúrgica, com que se transporta de um lugar a causa do uma doença para outro, em que é menos nociva.
\section{Metathético}
\begin{itemize}
\item {Grp. gram.:adj.}
\end{itemize}
Relativo á metáthese.
Em que há metáthese: \textunderscore pederneira é forma metathética de pedreneira\textunderscore .
\section{Metathórax}
\begin{itemize}
\item {Grp. gram.:m.}
\end{itemize}
\begin{itemize}
\item {Proveniência:(Do gr. \textunderscore meta\textunderscore  + \textunderscore thorax\textunderscore )}
\end{itemize}
Segmento posterior do thórax dos insectos.
\section{Metatipia}
\begin{itemize}
\item {Grp. gram.:f.}
\end{itemize}
\begin{itemize}
\item {Proveniência:(De \textunderscore meta\textunderscore  + \textunderscore typo\textunderscore )}
\end{itemize}
Mudança de tipo, em a natureza vegetal ou animal.
\section{Metátomo}
\begin{itemize}
\item {Grp. gram.:m.}
\end{itemize}
\begin{itemize}
\item {Utilização:Constr.}
\end{itemize}
\begin{itemize}
\item {Proveniência:(Do gr. \textunderscore meta\textunderscore  + \textunderscore tome\textunderscore )}
\end{itemize}
Espaço entre dois dentículos de uma cornija.
\section{Metator}
\begin{itemize}
\item {Grp. gram.:m.}
\end{itemize}
Espécie de engenheiro das legiões romanas. Cf. Herculano, \textunderscore Eurico\textunderscore , 233.
(Lat. \textunderscore metator\textunderscore ).
\section{Metatórax}
\begin{itemize}
\item {Grp. gram.:m.}
\end{itemize}
\begin{itemize}
\item {Proveniência:(Do gr. \textunderscore meta\textunderscore  + \textunderscore thorax\textunderscore )}
\end{itemize}
Segmento posterior do tórax dos insectos.
\section{Metatypia}
\begin{itemize}
\item {Grp. gram.:f.}
\end{itemize}
\begin{itemize}
\item {Proveniência:(De \textunderscore meta\textunderscore  + \textunderscore typo\textunderscore )}
\end{itemize}
Mudança de typo, em a natureza vegetal ou animal.
\section{Metaxita}
\begin{itemize}
\item {fónica:csi}
\end{itemize}
\begin{itemize}
\item {Grp. gram.:f.}
\end{itemize}
Composto natural de sílica e magnésia.
\section{Metazoário}
\begin{itemize}
\item {Grp. gram.:adj.}
\end{itemize}
\begin{itemize}
\item {Utilização:Geol.}
\end{itemize}
\begin{itemize}
\item {Grp. gram.:M. pl.}
\end{itemize}
\begin{itemize}
\item {Proveniência:(Do gr. \textunderscore meta\textunderscore  + \textunderscore zoon\textunderscore )}
\end{itemize}
Posterior á apparição dos animaes.
Terrenos, posteriores á apparição dos animaes.
\section{Metazoico}
\begin{itemize}
\item {Grp. gram.:adj.}
\end{itemize}
\begin{itemize}
\item {Utilização:Geol.}
\end{itemize}
\begin{itemize}
\item {Proveniência:(Do gr. \textunderscore meta\textunderscore  + \textunderscore zoon\textunderscore )}
\end{itemize}
Diz-se do terreno, que formou depois do apparecimento dos animaes; metazoário.
\section{Mete}
\begin{itemize}
\item {Grp. gram.:m.}
\end{itemize}
Árvore angolonse de Caconda.
\section{Meteal}
\begin{itemize}
\item {Grp. gram.:m.}
\end{itemize}
\begin{itemize}
\item {Utilização:Ant.}
\end{itemize}
O mesmo que \textunderscore metical\textunderscore .
\section{Meteco}
\begin{itemize}
\item {Grp. gram.:m.}
\end{itemize}
\begin{itemize}
\item {Proveniência:(Lat. \textunderscore metoecos\textunderscore )}
\end{itemize}
Assim se chamava o estranjeiro, domiciliado em Athenas
\section{Metediço}
\begin{itemize}
\item {Grp. gram.:adj.}
\end{itemize}
\begin{itemize}
\item {Proveniência:(De \textunderscore meter\textunderscore )}
\end{itemize}
Que intervém em todos os assumptos, a que não é chamado; intrometido.
\section{Metedor}
\begin{itemize}
\item {Grp. gram.:m.}
\end{itemize}
\begin{itemize}
\item {Utilização:Náut.}
\end{itemize}
\begin{itemize}
\item {Utilização:T. de Aveiro}
\end{itemize}
\begin{itemize}
\item {Proveniência:(De \textunderscore meter\textunderscore )}
\end{itemize}
Pano, que se enrola no mastro para o preservar da humidade.
Remador, immediato ao revezeiro.
\section{Metemerino}
\begin{itemize}
\item {Grp. gram.:adj.}
\end{itemize}
\begin{itemize}
\item {Utilização:Med.}
\end{itemize}
\begin{itemize}
\item {Proveniência:(Gr. \textunderscore methemerinos\textunderscore )}
\end{itemize}
Diz-se da febre, cujos accessos se repetem todos os dias.
\section{Metempsicose}
\begin{itemize}
\item {Grp. gram.:f.}
\end{itemize}
\begin{itemize}
\item {Proveniência:(Lat. \textunderscore metempsycosis\textunderscore )}
\end{itemize}
Teoria da trasm.gração da alma, de um corpo para outro.
Passagem da alma, de um corpo para outro.
\section{Metempsycose}
\begin{itemize}
\item {Grp. gram.:f.}
\end{itemize}
\begin{itemize}
\item {Proveniência:(Lat. \textunderscore metempsycosis\textunderscore )}
\end{itemize}
Theoria da trasm.gração da alma, de um corpo para outro.
Passagem da alma, de um corpo para outro.
\section{Metemptose}
\begin{itemize}
\item {Grp. gram.:f.}
\end{itemize}
\begin{itemize}
\item {Utilização:Astron.}
\end{itemize}
\begin{itemize}
\item {Proveniência:(Do gr. \textunderscore meta\textunderscore  + \textunderscore emptosis\textunderscore )}
\end{itemize}
Equação solar dos novilúnios, para que elles não cheguem um dia mais tarde.
\section{Metena}
\begin{itemize}
\item {Grp. gram.:f.}
\end{itemize}
\begin{itemize}
\item {Utilização:T. da Figueira-da-Foz}
\end{itemize}
Medida de lenha.
(Relaciona-se com o lat. \textunderscore metirí\textunderscore ? ou com o lat. \textunderscore meta\textunderscore ?)
\section{Meteoricamente}
\begin{itemize}
\item {Grp. gram.:adv.}
\end{itemize}
\begin{itemize}
\item {Proveniência:(De \textunderscore meteórico\textunderscore )}
\end{itemize}
Á semelhança dos meteóros.
\section{Meteórico}
\begin{itemize}
\item {Grp. gram.:adj.}
\end{itemize}
Relativo a meteóro.
Produzido por meteóros.
Que depende do estado atmosphérico.
\section{Meteorismo}
\begin{itemize}
\item {Grp. gram.:m.}
\end{itemize}
\begin{itemize}
\item {Utilização:Med.}
\end{itemize}
\begin{itemize}
\item {Proveniência:(De \textunderscore meteóro\textunderscore )}
\end{itemize}
Tumetacção do ventre, pela accumulação de um gás interior.
\section{Meteorita}
\begin{itemize}
\item {Grp. gram.:f.}
\end{itemize}
O mesmo que \textunderscore meteorito\textunderscore .
\section{Meteorite}
\begin{itemize}
\item {Grp. gram.:f.}
\end{itemize}
O mesmo que \textunderscore meteorito\textunderscore .
\section{Meteorito}
\begin{itemize}
\item {Grp. gram.:m.}
\end{itemize}
\begin{itemize}
\item {Proveniência:(De \textunderscore meteóro\textunderscore )}
\end{itemize}
Pequeno corpo, que se move fóra da atmosphera, nos espaços intercósmicos.
\section{Meteorização}
\begin{itemize}
\item {Grp. gram.:f.}
\end{itemize}
Acto de \textunderscore meteorizar\textunderscore .
\section{Meteorizar}
\begin{itemize}
\item {Grp. gram.:v. t.}
\end{itemize}
\begin{itemize}
\item {Proveniência:(Do gr. \textunderscore meteorizein\textunderscore )}
\end{itemize}
Tornar inchado por flatuosidades (o ventre).
\section{Meteóro}
\begin{itemize}
\item {Grp. gram.:m.}
\end{itemize}
\begin{itemize}
\item {Utilização:Restrict.}
\end{itemize}
\begin{itemize}
\item {Proveniência:(Gr. meteoros)}
\end{itemize}
Qualquer phenómeno atmosphérico.
Apparição brilhante e de curta duração.
Estrêlla cadente.
\section{Meteorografia}
\begin{itemize}
\item {Grp. gram.:f.}
\end{itemize}
Descripção dos meteóros.
(Cp. \textunderscore meteorógrafo\textunderscore )
\section{Meteorográfico}
\begin{itemize}
\item {Grp. gram.:adj.}
\end{itemize}
Relativo á meteorografia.
\section{Meteorógrafo}
\begin{itemize}
\item {Grp. gram.:m.}
\end{itemize}
\begin{itemize}
\item {Proveniência:(Do gr. \textunderscore meteoros\textunderscore  + \textunderscore graphein\textunderscore )}
\end{itemize}
Instrumento, para observações meteorológicas.
Aquele que escreve á cêrca de meteóros.
\section{Meteorographia}
\begin{itemize}
\item {Grp. gram.:f.}
\end{itemize}
Descripção dos meteóros.
(Cp. \textunderscore meteorógrapho\textunderscore )
\section{Meteorográphico}
\begin{itemize}
\item {Grp. gram.:adj.}
\end{itemize}
Relativo á meteorographia.
\section{Meteorógrapho}
\begin{itemize}
\item {Grp. gram.:m.}
\end{itemize}
\begin{itemize}
\item {Proveniência:(Do gr. \textunderscore meteoros\textunderscore  + \textunderscore graphein\textunderscore )}
\end{itemize}
Instrumento, para observações meteorológicas.
Aquelle que escreve á cêrca de meteóros.
\section{Meteorólitho}
\begin{itemize}
\item {Grp. gram.:m.}
\end{itemize}
\begin{itemize}
\item {Proveniência:(Do gr. \textunderscore meteoros\textunderscore  + \textunderscore lithos\textunderscore )}
\end{itemize}
O mesmo que \textunderscore aerólitho\textunderscore .
\section{Meteorólito}
\begin{itemize}
\item {Grp. gram.:m.}
\end{itemize}
\begin{itemize}
\item {Proveniência:(Do gr. \textunderscore meteoros\textunderscore  + \textunderscore lithos\textunderscore )}
\end{itemize}
O mesmo que \textunderscore aerólito\textunderscore .
\section{Meteorologia}
\begin{itemize}
\item {Grp. gram.:f.}
\end{itemize}
\begin{itemize}
\item {Proveniência:(Do gr. \textunderscore meteoros\textunderscore  + \textunderscore logos\textunderscore )}
\end{itemize}
Sciência, que trata dos meteóros ou phenómenos atmosphéricos.
\section{Meteorológico}
\begin{itemize}
\item {Grp. gram.:adj.}
\end{itemize}
Relativo á Meteorologia.
\section{Meteorologista}
\begin{itemize}
\item {Grp. gram.:m.}
\end{itemize}
Aquelle que é versado em Meteorologia.
\section{Meteoromancia}
\begin{itemize}
\item {Grp. gram.:f.}
\end{itemize}
\begin{itemize}
\item {Proveniência:(Do gr. \textunderscore meteoros\textunderscore  + \textunderscore manteia\textunderscore )}
\end{itemize}
Pretendida adivinhação, por meio de meteóros.
\section{Meteoronomia}
\begin{itemize}
\item {Grp. gram.:f.}
\end{itemize}
\begin{itemize}
\item {Proveniência:(Do gr. \textunderscore meteoros\textunderscore  + \textunderscore nomos\textunderscore )}
\end{itemize}
Investigação das leis dos meteóros.
\section{Meteoroscópio}
\begin{itemize}
\item {Grp. gram.:m.}
\end{itemize}
\begin{itemize}
\item {Proveniência:(Do gr. \textunderscore meteoros\textunderscore  + \textunderscore skopein\textunderscore )}
\end{itemize}
Instrumento, para observações meteorológicas.
\section{Meter}
\begin{itemize}
\item {Grp. gram.:v. t.}
\end{itemize}
\begin{itemize}
\item {Utilização:Prov.}
\end{itemize}
\begin{itemize}
\item {Utilização:minh.}
\end{itemize}
\begin{itemize}
\item {Proveniência:(Do lat. \textunderscore mittere\textunderscore )}
\end{itemize}
Pôr dentro: \textunderscore meter o pão no forno\textunderscore .
Fazer entrar: \textunderscore meter um pau no chão\textunderscore .
Collocar.
Insinuar; infundir: \textunderscore meter medo\textunderscore .
Abranger.
Induzir.
Guardar.
Pôr de permeio.
Reduzir a menor espaço.
\textunderscore Meter os dedos pelos olhos\textunderscore , pretender negar o que é evidente.
\textunderscore Meter feira\textunderscore , dar nas vistas, (falando-se de coisas).
\textunderscore Meter num chinelo\textunderscore , suplantar.
\textunderscore Meter dente em\textunderscore , perceber (coisas diffíceis).
\textunderscore Meter-se nas encolhas\textunderscore , retrahir-se, calar-se.
\section{Methane}
\begin{itemize}
\item {Grp. gram.:m.}
\end{itemize}
O mesmo que \textunderscore methânio\textunderscore .
\section{Methânio}
\begin{itemize}
\item {Grp. gram.:m.}
\end{itemize}
Gás incolor, produzido por substâncias em putrefacção, e chamado também \textunderscore gás dos pântanos\textunderscore , \textunderscore formena\textunderscore , etc.
\section{Methano}
\begin{itemize}
\item {Grp. gram.:m.}
\end{itemize}
O mesmo que \textunderscore methânio\textunderscore .
\section{Metheal}
\begin{itemize}
\item {Grp. gram.:m.}
\end{itemize}
\begin{itemize}
\item {Utilização:Ant.}
\end{itemize}
O mesmo que \textunderscore metical\textunderscore .
\section{Methemerino}
\begin{itemize}
\item {Grp. gram.:adj.}
\end{itemize}
\begin{itemize}
\item {Utilização:Med.}
\end{itemize}
\begin{itemize}
\item {Proveniência:(Gr. \textunderscore methemerinos\textunderscore )}
\end{itemize}
Diz-se da febre, cujos accessos se repetem todos os dias.
\section{Methiónico}
\begin{itemize}
\item {Grp. gram.:adj.}
\end{itemize}
\begin{itemize}
\item {Proveniência:(Do gr. \textunderscore meta\textunderscore  + \textunderscore heion\textunderscore )}
\end{itemize}
Diz-se de um ácido, que se obtém, sujeitando o éther á acção do ácido sulfúrico.
\section{Methodicamente}
\begin{itemize}
\item {Grp. gram.:adv.}
\end{itemize}
De modo methódico.
Com méthodo.
Com circunspecção.
\section{Methódico}
\begin{itemize}
\item {Grp. gram.:adj.}
\end{itemize}
\begin{itemize}
\item {Utilização:Fig.}
\end{itemize}
\begin{itemize}
\item {Proveniência:(Lat. \textunderscore methodicus\textunderscore )}
\end{itemize}
Relativo a méthodo.
Em que há méthodo.
Commedido; circunspecto.
\section{Methodificar}
\begin{itemize}
\item {Grp. gram.:v. t.}
\end{itemize}
O mesmo que \textunderscore methodizar\textunderscore .
\section{Methodismo}
\begin{itemize}
\item {Grp. gram.:m.}
\end{itemize}
\begin{itemize}
\item {Proveniência:(De \textunderscore méthodo\textunderscore )}
\end{itemize}
Doutrina da seita dos Methodistas.
\section{Methodista}
\begin{itemize}
\item {Grp. gram.:m.  e  f.}
\end{itemize}
\begin{itemize}
\item {Proveniência:(De \textunderscore méthodo\textunderscore )}
\end{itemize}
Pessôa, que segue rigorosamente certo méthodo.
Rotineiro.
Membro de uma seita protestante, que ostenta grande austeridade.
\section{Methodização}
\begin{itemize}
\item {Grp. gram.:f.}
\end{itemize}
Acto de \textunderscore methodizar\textunderscore .
\section{Methodizar}
\begin{itemize}
\item {Grp. gram.:v. t.}
\end{itemize}
Tornar methódico.
Regularizar.
\section{Méthodo}
\begin{itemize}
\item {Grp. gram.:m.}
\end{itemize}
\begin{itemize}
\item {Utilização:Fig.}
\end{itemize}
\begin{itemize}
\item {Proveniência:(Lat. \textunderscore methodus\textunderscore )}
\end{itemize}
Conjunto de processos racionaes, para fazer qualquer coisa ou obter qualquer fim theórico ou prático.
Modo de proceder.
Classificação de diversos seres, segundo os caracteres que os aproximam.
Tratado elementar.
Prudência; circunspecção.
Modo judicioso de proceder: \textunderscore trabalhar com méthodo\textunderscore .
\section{Methodologia}
\begin{itemize}
\item {Grp. gram.:f.}
\end{itemize}
\begin{itemize}
\item {Proveniência:(Do gr. \textunderscore methodos\textunderscore  + \textunderscore logos\textunderscore )}
\end{itemize}
Tratado dos méthodos.
Arte de dirigir o espírito na investigação da verdade.
\section{Methodológico}
\begin{itemize}
\item {Grp. gram.:adj.}
\end{itemize}
Relativo a methodologia.
\section{Methomania}
\begin{itemize}
\item {Grp. gram.:f.}
\end{itemize}
\begin{itemize}
\item {Utilização:Med.}
\end{itemize}
\begin{itemize}
\item {Proveniência:(Do gr. \textunderscore methe\textunderscore  + \textunderscore mania\textunderscore )}
\end{itemize}
Desejo irresistível de bebidas espirituosas ou fermentadas.
\section{Methónica}
\begin{itemize}
\item {Grp. gram.:f.}
\end{itemize}
Gênero de plantas liliáceas.
(Do gr.)
\section{Methyl}
\begin{itemize}
\item {Grp. gram.:m.}
\end{itemize}
O mesmo que \textunderscore methylo\textunderscore . Cf. \textunderscore Diár. do Govêrno\textunderscore , de 13-XI-901.
\section{Methylena}
\begin{itemize}
\item {Grp. gram.:f.}
\end{itemize}
O mesmo que \textunderscore methylene\textunderscore .
\section{Methylene}
\begin{itemize}
\item {Grp. gram.:m.}
\end{itemize}
\begin{itemize}
\item {Utilização:Chím.}
\end{itemize}
\begin{itemize}
\item {Proveniência:(De \textunderscore methylo\textunderscore )}
\end{itemize}
Espírito de madeira, adoptado como agente para a desnaturação dos álcooes. Cf. \textunderscore Diár. do Govêrno\textunderscore , de 13-XI-901.
\section{Methýlico}
\begin{itemize}
\item {Grp. gram.:adj.}
\end{itemize}
\begin{itemize}
\item {Proveniência:(De \textunderscore methylo\textunderscore )}
\end{itemize}
Diz-se dos ácidos análogos ao ácido vínico, entrando o álcool methýlico, em vez do álcool ordinário.
\section{Methylo}
\begin{itemize}
\item {Grp. gram.:m.}
\end{itemize}
\begin{itemize}
\item {Utilização:Chím.}
\end{itemize}
\begin{itemize}
\item {Proveniência:(Do gr. \textunderscore methe\textunderscore )}
\end{itemize}
Radical do éther methýlico.
\section{Methymneu}
\begin{itemize}
\item {Grp. gram.:adj.}
\end{itemize}
\begin{itemize}
\item {Proveniência:(Lat. \textunderscore methymnaeus\textunderscore )}
\end{itemize}
Relativo a Methymna, cidade lésbia, afamada pelo seu excellente vinho:«\textunderscore ...a cepa methymnea, em Lesbos afamada\textunderscore ». Castilho, \textunderscore Geórgicas\textunderscore , 79.
\section{Metiba}
\begin{itemize}
\item {Grp. gram.:f.}
\end{itemize}
Árvore de Moçambique.
\section{Metical}
\begin{itemize}
\item {Grp. gram.:m.}
\end{itemize}
\begin{itemize}
\item {Proveniência:(Do ár. \textunderscore mitcal\textunderscore )}
\end{itemize}
Antiga moéda africana, ainda hoje usada em Marrocos e correspondente a 881 reis.
Antigo e pequeno pêso de Ormuz.
\section{Metição}
\begin{itemize}
\item {Grp. gram.:f.}
\end{itemize}
Acto de meter. Cf. \textunderscore Techn. Rur.\textunderscore , 552.
\section{Meticulosamente}
\begin{itemize}
\item {Grp. gram.:adv.}
\end{itemize}
De modo meticuloso.
Com escrúpulo.
\section{Meticulosidade}
\begin{itemize}
\item {Grp. gram.:f.}
\end{itemize}
Qualidade de meticuloso.
Timidez.
\section{Meticuloso}
\begin{itemize}
\item {Grp. gram.:adj.}
\end{itemize}
\begin{itemize}
\item {Proveniência:(Lat. \textunderscore meticulosus\textunderscore )}
\end{itemize}
Escrupuloso.
Cauteloso.
Medroso, tímido; receoso.
\section{Metida}
\begin{itemize}
\item {Grp. gram.:f.}
\end{itemize}
\begin{itemize}
\item {Utilização:Prov.}
\end{itemize}
\begin{itemize}
\item {Utilização:minh.}
\end{itemize}
\begin{itemize}
\item {Proveniência:(De \textunderscore metido\textunderscore )}
\end{itemize}
Grande carrada de pedra.
\section{Metido}
\begin{itemize}
\item {Grp. gram.:adj.}
\end{itemize}
\begin{itemize}
\item {Proveniência:(De \textunderscore meter\textunderscore )}
\end{itemize}
Intrometido; familiarizado.
\section{Metil}
\begin{itemize}
\item {Grp. gram.:m.}
\end{itemize}
O mesmo que \textunderscore metilo\textunderscore . Cf. \textunderscore Diár. do Govêrno\textunderscore , de 13-XI-901.
\section{Metilena}
\begin{itemize}
\item {Grp. gram.:f.}
\end{itemize}
O mesmo que \textunderscore metilene\textunderscore .
\section{Metilene}
\begin{itemize}
\item {Grp. gram.:m.}
\end{itemize}
\begin{itemize}
\item {Utilização:Chím.}
\end{itemize}
\begin{itemize}
\item {Proveniência:(De \textunderscore metilo\textunderscore )}
\end{itemize}
Espírito de madeira, adoptado como agente para a desnaturação dos álcooes. Cf. \textunderscore Diár. do Govêrno\textunderscore , de 13-XI-901.
\section{Metílico}
\begin{itemize}
\item {Grp. gram.:adj.}
\end{itemize}
\begin{itemize}
\item {Proveniência:(De \textunderscore metilo\textunderscore )}
\end{itemize}
Diz-se dos ácidos análogos ao ácido vínico, entrando o álcool metílico, em vez do álcool ordinário.
\section{Metilo}
\begin{itemize}
\item {Grp. gram.:m.}
\end{itemize}
\begin{itemize}
\item {Utilização:Chím.}
\end{itemize}
\begin{itemize}
\item {Proveniência:(Do gr. \textunderscore methe\textunderscore )}
\end{itemize}
Radical do éter metílico.
\section{Metinás}
\begin{itemize}
\item {Grp. gram.:m. pl.}
\end{itemize}
Indígenas do norte do Brasil.
\section{Metins}
\begin{itemize}
\item {Grp. gram.:m. pl.}
\end{itemize}
\begin{itemize}
\item {Utilização:Ant.}
\end{itemize}
Enfeite africano, o mesmo que \textunderscore mites\textunderscore .
\section{Metiónico}
\begin{itemize}
\item {Grp. gram.:adj.}
\end{itemize}
\begin{itemize}
\item {Proveniência:(Do gr. \textunderscore meta\textunderscore  + \textunderscore heion\textunderscore )}
\end{itemize}
Diz-se de um ácido, que se obtém, sujeitando o éter á acção do ácido sulfúrico.
\section{Metocho}
\begin{itemize}
\item {fónica:tô}
\end{itemize}
\begin{itemize}
\item {Grp. gram.:m.}
\end{itemize}
O mesmo que \textunderscore metátomo\textunderscore .
\section{Metodicamente}
\begin{itemize}
\item {Grp. gram.:adv.}
\end{itemize}
De modo metódico.
Com método.
Com circunspecção.
\section{Metódico}
\begin{itemize}
\item {Grp. gram.:adj.}
\end{itemize}
\begin{itemize}
\item {Utilização:Fig.}
\end{itemize}
\begin{itemize}
\item {Proveniência:(Lat. \textunderscore methodicus\textunderscore )}
\end{itemize}
Relativo a método.
Em que há método.
Comedido; circunspecto.
\section{Metodificar}
\begin{itemize}
\item {Grp. gram.:v. t.}
\end{itemize}
O mesmo que \textunderscore metodizar\textunderscore .
\section{Metodismo}
\begin{itemize}
\item {Grp. gram.:m.}
\end{itemize}
\begin{itemize}
\item {Proveniência:(De \textunderscore método\textunderscore )}
\end{itemize}
Doutrina da seita dos Metodistas.
\section{Metodista}
\begin{itemize}
\item {Grp. gram.:m.  e  f.}
\end{itemize}
\begin{itemize}
\item {Proveniência:(De \textunderscore método\textunderscore )}
\end{itemize}
Pessôa, que segue rigorosamente certo método.
Rotineiro.
Membro de uma seita protestante, que ostenta grande austeridade.
\section{Metodização}
\begin{itemize}
\item {Grp. gram.:f.}
\end{itemize}
Acto de \textunderscore metodizar\textunderscore .
\section{Metodizar}
\begin{itemize}
\item {Grp. gram.:v. t.}
\end{itemize}
Tornar metódico.
Regularizar.
\section{Método}
\begin{itemize}
\item {Grp. gram.:m.}
\end{itemize}
\begin{itemize}
\item {Utilização:Fig.}
\end{itemize}
\begin{itemize}
\item {Proveniência:(Lat. \textunderscore methodus\textunderscore )}
\end{itemize}
Conjunto de processos racionaes, para fazer qualquer coisa ou obter qualquer fim teórico ou prático.
Modo de proceder.
Classificação de diversos seres, segundo os caracteres que os aproximam.
Tratado elementar.
Prudência; circunspecção.
Modo judicioso de proceder: \textunderscore trabalhar com método\textunderscore .
\section{Metodologia}
\begin{itemize}
\item {Grp. gram.:f.}
\end{itemize}
\begin{itemize}
\item {Proveniência:(Do gr. \textunderscore methodos\textunderscore  + \textunderscore logos\textunderscore )}
\end{itemize}
Tratado dos métodos.
Arte de dirigir o espírito na investigação da verdade.
\section{Metodológico}
\begin{itemize}
\item {Grp. gram.:adj.}
\end{itemize}
Relativo a metodologia.
\section{Metoita}
\begin{itemize}
\item {Grp. gram.:f.}
\end{itemize}
\begin{itemize}
\item {Utilização:Prov.}
\end{itemize}
\begin{itemize}
\item {Utilização:trasm.}
\end{itemize}
\begin{itemize}
\item {Utilização:Chul.}
\end{itemize}
Cabeça da gente.
(Cp. \textunderscore toutiço\textunderscore )
\section{Metomania}
\begin{itemize}
\item {Grp. gram.:f.}
\end{itemize}
\begin{itemize}
\item {Utilização:Med.}
\end{itemize}
\begin{itemize}
\item {Proveniência:(Do gr. \textunderscore methe\textunderscore  + \textunderscore mania\textunderscore )}
\end{itemize}
Desejo irresistível de bebidas espirituosas ou fermentadas.
\section{Metónica}
\begin{itemize}
\item {Grp. gram.:f.}
\end{itemize}
Gênero de plantas liliáceas.
(Do gr.)
\section{Metonímia}
\begin{itemize}
\item {Grp. gram.:f.}
\end{itemize}
\begin{itemize}
\item {Proveniência:(Lat. \textunderscore metonymia\textunderscore )}
\end{itemize}
Figura de Retórica, com que se emprega um termo por outro, cuja significação aquele indica.
\section{Metonímico}
\begin{itemize}
\item {Grp. gram.:adj.}
\end{itemize}
\begin{itemize}
\item {Proveniência:(Lat. \textunderscore metonymicus\textunderscore )}
\end{itemize}
Relativo á metonímia.
\section{Metonomásia}
\begin{itemize}
\item {Grp. gram.:f.}
\end{itemize}
\begin{itemize}
\item {Proveniência:(Do gr. \textunderscore meta\textunderscore  + \textunderscore onoma\textunderscore )}
\end{itemize}
Mudança ou disfarce de um nome, por meio de traducção, como se um indivíduo, chamado \textunderscore Coelho Júnior\textunderscore , assinasse \textunderscore Petit Lapin\textunderscore , ou como se outro, chamado \textunderscore Carvalho\textunderscore , assinasse \textunderscore Quercus\textunderscore .
\section{Metonýmia}
\begin{itemize}
\item {Grp. gram.:f.}
\end{itemize}
\begin{itemize}
\item {Proveniência:(Lat. \textunderscore metonymia\textunderscore )}
\end{itemize}
Figura de Rhetórica, com que se emprega um termo por outro, cuja significação aquelle indica.
\section{Metonýmico}
\begin{itemize}
\item {Grp. gram.:adj.}
\end{itemize}
\begin{itemize}
\item {Proveniência:(Lat. \textunderscore metonymicus\textunderscore )}
\end{itemize}
Relativo á metonýmia.
\section{Métopa}
\begin{itemize}
\item {Grp. gram.:f.}
\end{itemize}
\begin{itemize}
\item {Proveniência:(Lat. \textunderscore metopa\textunderscore )}
\end{itemize}
Intervallo quadrado, entre os triglyphos do friso dórico.
\section{Metopagia}
\begin{itemize}
\item {Grp. gram.:f.}
\end{itemize}
Estado ou qualidade de metópago.
\section{Metópago}
\begin{itemize}
\item {Grp. gram.:m.  e  adj.}
\end{itemize}
\begin{itemize}
\item {Proveniência:(Do gr. \textunderscore metopon\textunderscore  + \textunderscore pagein\textunderscore )}
\end{itemize}
Monstro, formado de dois indivíduos, de umbigos distintos e cabeças reunidas.
\section{Métope}
\begin{itemize}
\item {Grp. gram.:f.}
\end{itemize}
(V.métopa)
\section{Metópico}
\begin{itemize}
\item {Grp. gram.:adj.}
\end{itemize}
Relativo ao metópion.
\section{Metópio}
\begin{itemize}
\item {Grp. gram.:m.}
\end{itemize}
\begin{itemize}
\item {Utilização:Anat.}
\end{itemize}
\begin{itemize}
\item {Proveniência:(Lat. \textunderscore metopium\textunderscore )}
\end{itemize}
Ponto, situado na linha média da fronte, entre as duas bossas frontaes.
\section{Metópion}
\begin{itemize}
\item {Grp. gram.:m.}
\end{itemize}
\begin{itemize}
\item {Utilização:Anat.}
\end{itemize}
\begin{itemize}
\item {Proveniência:(Lat. \textunderscore metopium\textunderscore )}
\end{itemize}
Ponto, situado na linha média da fronte, entre as duas bossas frontaes.
\section{Metoposcopia}
\begin{itemize}
\item {Grp. gram.:f.}
\end{itemize}
Arte de adivinhar, pelos traços da physionomia, o que succede a alguém.
(Cp. \textunderscore metopóscopo\textunderscore )
\section{Metoposcópico}
\begin{itemize}
\item {Grp. gram.:adj.}
\end{itemize}
Relativo á metoposcopia.
\section{Metopóscopo}
\begin{itemize}
\item {Grp. gram.:m.}
\end{itemize}
\begin{itemize}
\item {Proveniência:(Lat. \textunderscore metoscopos\textunderscore )}
\end{itemize}
Aquelle que pratíca a metoposcopia.
\section{Metose}
\begin{itemize}
\item {Grp. gram.:f.}
\end{itemize}
\begin{itemize}
\item {Utilização:Med.}
\end{itemize}
Contracção da pupilla ocular.
\section{Metralgia}
\begin{itemize}
\item {Grp. gram.:f.}
\end{itemize}
\begin{itemize}
\item {Proveniência:(Do gr. \textunderscore metra\textunderscore  + \textunderscore algos\textunderscore )}
\end{itemize}
Dôr no útero.
\section{Metrálgico}
\begin{itemize}
\item {Grp. gram.:adj.}
\end{itemize}
Relativo a metralgia.
\section{Metralha}
\begin{itemize}
\item {Grp. gram.:f.}
\end{itemize}
\begin{itemize}
\item {Utilização:Fig.}
\end{itemize}
\begin{itemize}
\item {Proveniência:(Do fr. \textunderscore mitraille\textunderscore )}
\end{itemize}
Balas de ferro.
Pedaços de ferro, cacos, etc., com que se carregam projécteis ocos.
Grande porção; conjunto de coisas; amálgama.
\section{Metralhada}
\begin{itemize}
\item {Grp. gram.:f.}
\end{itemize}
Tiro de metralha.
\section{Metralhador}
\begin{itemize}
\item {Grp. gram.:m.  e  adj.}
\end{itemize}
O que metralha.
\section{Metralhadora}
\begin{itemize}
\item {Grp. gram.:f.}
\end{itemize}
\begin{itemize}
\item {Proveniência:(De \textunderscore metralhador\textunderscore )}
\end{itemize}
Máquina de guerra, que dispara muitos projécteis ao mesmo tempo.
\section{Metralhar}
\begin{itemize}
\item {Grp. gram.:v. t.}
\end{itemize}
Ferir ou atacar com tiros de metralha.
\section{Metreta}
\begin{itemize}
\item {Grp. gram.:f.}
\end{itemize}
\begin{itemize}
\item {Proveniência:(Lat. \textunderscore metreta\textunderscore )}
\end{itemize}
Grande medida para líquidos, usada pelos Athenienses, e talvez equivalente a cêrca de 40 litros.
Vasilha, que podia conter uma metreta de qualquer líquido.
\section{Métrica}
\begin{itemize}
\item {Grp. gram.:f.}
\end{itemize}
\begin{itemize}
\item {Proveniência:(De \textunderscore métrico\textunderscore )}
\end{itemize}
Arte de medir versos. Cf. Latino, \textunderscore Elogios\textunderscore , 98 e 302.
\section{Metricamente}
\begin{itemize}
\item {Grp. gram.:adv.}
\end{itemize}
De modo métrico; com medida.
\section{Métrico}
\begin{itemize}
\item {Grp. gram.:adj.}
\end{itemize}
\begin{itemize}
\item {Proveniência:(Lat. \textunderscore metricus\textunderscore )}
\end{itemize}
Relativo ao metro: \textunderscore systema métrico dos pesos e medidas\textunderscore .
Relativo a versos: \textunderscore sýllabas métricas\textunderscore .
Que está em verso: \textunderscore uma obra métrica\textunderscore .
Relativo á metrificação: \textunderscore sýllabas métricas\textunderscore .
\section{Metrificação}
\begin{itemize}
\item {Grp. gram.:f.}
\end{itemize}
Acto ou effeito de metrificar.
\section{Metrificador}
\begin{itemize}
\item {Grp. gram.:adj.}
\end{itemize}
\begin{itemize}
\item {Grp. gram.:M.}
\end{itemize}
Que metrifica.
Aquelle que metrifica; versejador.
\section{Metrificância}
\begin{itemize}
\item {Grp. gram.:f.}
\end{itemize}
Arte ou qualidade de metrificante:«\textunderscore ...se, assim como se deo á predica, se houvera dado á metrificância...\textunderscore »Filinto, VIII, 247.
\section{Metrificante}
\begin{itemize}
\item {Grp. gram.:m.}
\end{itemize}
\begin{itemize}
\item {Utilização:Deprec.}
\end{itemize}
Aquelle que metrifica; metrificador. Cf. Filinto, XIII, 70; XVIII, 109.
\section{Metrificar}
\begin{itemize}
\item {Grp. gram.:v. t.}
\end{itemize}
\begin{itemize}
\item {Grp. gram.:V. i.}
\end{itemize}
\begin{itemize}
\item {Proveniência:(Do lat. \textunderscore metrum\textunderscore  + \textunderscore facere\textunderscore )}
\end{itemize}
Pôr em verso, reduzir a verso.
Fazer versos, versejar.
\section{Metrífluo}
\begin{itemize}
\item {Grp. gram.:adj.}
\end{itemize}
\begin{itemize}
\item {Utilização:Poét.}
\end{itemize}
\begin{itemize}
\item {Proveniência:(Do lat. \textunderscore metrum\textunderscore  + \textunderscore fluere\textunderscore )}
\end{itemize}
Diz-se de Apollo, como deus da poesia.
\section{Metriopathia}
\begin{itemize}
\item {Grp. gram.:f.}
\end{itemize}
\begin{itemize}
\item {Proveniência:(Gr. \textunderscore metriopatheia\textunderscore )}
\end{itemize}
Disposição, com que, segundo os philósophos scépticos, se moderam as paixões.
\section{Metriopatia}
\begin{itemize}
\item {Grp. gram.:f.}
\end{itemize}
\begin{itemize}
\item {Proveniência:(Gr. \textunderscore metriopatheia\textunderscore )}
\end{itemize}
Disposição, com que, segundo os filósophos cépticos, se moderam as paixões.
\section{Metrite}
\begin{itemize}
\item {Grp. gram.:f.}
\end{itemize}
\begin{itemize}
\item {Proveniência:(Do gr. \textunderscore metra\textunderscore )}
\end{itemize}
Inflammação do útero.
\section{Metro}
\begin{itemize}
\item {Grp. gram.:m.}
\end{itemize}
\begin{itemize}
\item {Proveniência:(Lat. \textunderscore metrum\textunderscore )}
\end{itemize}
Medida de verso.
Conjunto dos pés ou sýllabas que constituem um verso.
Rythmo.
Unidade fundamental das medidas comprehendidas no systema métrico.
\section{Metróbata}
\begin{itemize}
\item {Grp. gram.:m.}
\end{itemize}
\begin{itemize}
\item {Proveniência:(Do gr. \textunderscore metron\textunderscore  + \textunderscore bates\textunderscore )}
\end{itemize}
Antigo instrumento, com que se regulava o passo da infantaria.
\section{Metrocampsia}
\begin{itemize}
\item {Grp. gram.:f.}
\end{itemize}
\begin{itemize}
\item {Utilização:Med.}
\end{itemize}
\begin{itemize}
\item {Proveniência:(Do gr. \textunderscore metra\textunderscore  + \textunderscore kamptein\textunderscore )}
\end{itemize}
Inflexão da madre.
\section{Metrocele}
\begin{itemize}
\item {Grp. gram.:m.}
\end{itemize}
\begin{itemize}
\item {Utilização:Med.}
\end{itemize}
\begin{itemize}
\item {Proveniência:(Do gr. \textunderscore metra\textunderscore  + \textunderscore kele\textunderscore )}
\end{itemize}
Hernia, formada pela madre.
\section{Metrodinia}
\begin{itemize}
\item {Grp. gram.:f.}
\end{itemize}
\begin{itemize}
\item {Utilização:Med.}
\end{itemize}
\begin{itemize}
\item {Proveniência:(Do gr. \textunderscore metra\textunderscore  + \textunderscore odune\textunderscore )}
\end{itemize}
Dôr no útero.
\section{Metrodynia}
\begin{itemize}
\item {Grp. gram.:f.}
\end{itemize}
\begin{itemize}
\item {Utilização:Med.}
\end{itemize}
\begin{itemize}
\item {Proveniência:(Do gr. \textunderscore metra\textunderscore  + \textunderscore odune\textunderscore )}
\end{itemize}
Dôr no útero.
\section{Metroflebite}
\begin{itemize}
\item {Grp. gram.:f.}
\end{itemize}
Inflamação das veias uterinas.
\section{Metrofotografia}
\begin{itemize}
\item {Grp. gram.:f.}
\end{itemize}
Método de levantamentos fotográficos, reconhecido pelo Congresso Internacional de Fotografia, de Paris, em 1889.
\section{Metrografia}
\begin{itemize}
\item {Grp. gram.:f.}
\end{itemize}
Tratado á cêrca dos pesos e medidas.
(Cp. \textunderscore metrógrafo\textunderscore ^1)
\section{Metrografia}
\begin{itemize}
\item {Grp. gram.:f.}
\end{itemize}
Descripção do útero.
(Cp. \textunderscore metrógrafo\textunderscore ^2)
\section{Metrógrafo}
\begin{itemize}
\item {Grp. gram.:m.}
\end{itemize}
\begin{itemize}
\item {Proveniência:(Do gr. \textunderscore metron\textunderscore  + \textunderscore graphein\textunderscore )}
\end{itemize}
Aquele que escreve á cêrca de pesos e medidas.
\section{Metrógrafo}
\begin{itemize}
\item {Grp. gram.:m.}
\end{itemize}
\begin{itemize}
\item {Proveniência:(Do gr. \textunderscore metra\textunderscore  + \textunderscore graphein\textunderscore )}
\end{itemize}
Aquele que escreve á cêrca do útero ou das doenças dêste órgão.
\section{Metrographia}
\begin{itemize}
\item {Grp. gram.:f.}
\end{itemize}
Tratado á cêrca dos pesos e medidas.
(Cp. \textunderscore metrógrapho\textunderscore ^1)
\section{Metrographia}
\begin{itemize}
\item {Grp. gram.:f.}
\end{itemize}
Descripção do útero.
(Cp. \textunderscore metrógrapho\textunderscore ^2)
\section{Metrógrapho}
\begin{itemize}
\item {Grp. gram.:m.}
\end{itemize}
\begin{itemize}
\item {Proveniência:(Do gr. \textunderscore metron\textunderscore  + \textunderscore graphein\textunderscore )}
\end{itemize}
Aquelle que escreve á cêrca de pesos e medidas.
\section{Metrógrapho}
\begin{itemize}
\item {Grp. gram.:m.}
\end{itemize}
\begin{itemize}
\item {Proveniência:(Do gr. \textunderscore metra\textunderscore  + \textunderscore graphein\textunderscore )}
\end{itemize}
Aquelle que escreve á cêrca do útero ou das doenças dêste órgão.
\section{Metrologia}
\begin{itemize}
\item {Grp. gram.:f.}
\end{itemize}
\begin{itemize}
\item {Proveniência:(Do gr. \textunderscore metron\textunderscore  + \textunderscore logos\textunderscore )}
\end{itemize}
Conhecimento dos pesos e medidas de todos os povos, antigos e modernos.
\section{Metrológico}
\begin{itemize}
\item {Grp. gram.:adj.}
\end{itemize}
Relativo á metrologia.
\section{Metrologista}
\begin{itemize}
\item {Grp. gram.:m.}
\end{itemize}
Aquelle que escreveu alguma obra sôbre metrologia.
Aquelle que faz investigações metrológicas.
\section{Metroloxia}
\begin{itemize}
\item {fónica:csi}
\end{itemize}
\begin{itemize}
\item {Grp. gram.:f.}
\end{itemize}
\begin{itemize}
\item {Proveniência:(Do gr. \textunderscore metra\textunderscore  + \textunderscore loxos\textunderscore )}
\end{itemize}
Obliquidade do útero.
\section{Metromania}
\begin{itemize}
\item {Grp. gram.:f.}
\end{itemize}
\begin{itemize}
\item {Utilização:Med.}
\end{itemize}
\begin{itemize}
\item {Proveniência:(Do gr. \textunderscore metra\textunderscore  + \textunderscore mania\textunderscore )}
\end{itemize}
Furor uterino.
\section{Metromania}
\begin{itemize}
\item {Grp. gram.:f.}
\end{itemize}
\begin{itemize}
\item {Proveniência:(De \textunderscore metro\textunderscore  + \textunderscore mania\textunderscore )}
\end{itemize}
Mania de fazer versos.
\section{Metromaníaco}
\begin{itemize}
\item {Grp. gram.:adj.}
\end{itemize}
Que tem metromania^2.
\section{Metrómano}
\begin{itemize}
\item {Grp. gram.:adj.}
\end{itemize}
Que tem metromania^2.
\section{Metrómetro}
\begin{itemize}
\item {Grp. gram.:m.}
\end{itemize}
O mesmo que \textunderscore metrónomo\textunderscore .
\section{Metronímico}
\begin{itemize}
\item {Grp. gram.:adj.}
\end{itemize}
Relativo ao metrónomo.
\section{Metrónomo}
\begin{itemize}
\item {Grp. gram.:m.}
\end{itemize}
\begin{itemize}
\item {Proveniência:(Do gr. \textunderscore metron\textunderscore  + \textunderscore nomos\textunderscore )}
\end{itemize}
Instrumento mecânico, cuja parte principal é um pêndulo, e que indica o grau de movimento, em que se deve executar uma peça de música.
\section{Metroperitonite}
\begin{itemize}
\item {Grp. gram.:f.}
\end{itemize}
Inflammação do útero e do peritonéu.
\section{Metrophlebite}
\begin{itemize}
\item {Grp. gram.:f.}
\end{itemize}
Inflammação das veias uterinas.
\section{Metrophotographia}
\begin{itemize}
\item {Grp. gram.:f.}
\end{itemize}
Méthodo de levantamentos photográphicos, reconhecido pelo Congresso Internacional de Fotografia, de Paris, em 1889.
\section{Metrópole}
\begin{itemize}
\item {Grp. gram.:f.}
\end{itemize}
\begin{itemize}
\item {Proveniência:(Lat. \textunderscore metropolis\textunderscore )}
\end{itemize}
Cidade principal ou capital de uma província ou de um Estado.
Cidade, com séde archiepiscopal.
Nação, considerada em relação ás suas colónias.
Centro de civilização ou de commercio; empório.
\section{Metrópoli}
\begin{itemize}
\item {Grp. gram.:f.}
\end{itemize}
Outra fórma de \textunderscore metrópole\textunderscore . Cf. Filinto, \textunderscore D. Man.\textunderscore , I, 245; \textunderscore Peregrinação\textunderscore , XLV.
\section{Metropólipo}
\begin{itemize}
\item {Grp. gram.:m.}
\end{itemize}
\begin{itemize}
\item {Utilização:Med.}
\end{itemize}
\begin{itemize}
\item {Proveniência:(Do gr. \textunderscore metra\textunderscore  + \textunderscore polupos\textunderscore )}
\end{itemize}
Pólipo do útero.
\section{Metropolita}
\begin{itemize}
\item {Grp. gram.:f.}
\end{itemize}
\begin{itemize}
\item {Proveniência:(Lat. \textunderscore metropolita\textunderscore )}
\end{itemize}
Prelado metropolitano.
\section{Metropolitano}
\begin{itemize}
\item {Grp. gram.:adj.}
\end{itemize}
\begin{itemize}
\item {Grp. gram.:M.}
\end{itemize}
\begin{itemize}
\item {Proveniência:(Lat. \textunderscore metropolitanus\textunderscore )}
\end{itemize}
Relativo á metrópole.
Prelado de metrópole, considerado em relação aos prelados que lhe são suffragâneos.
\section{Metropolítico}
\begin{itemize}
\item {Grp. gram.:adj.}
\end{itemize}
Relativo a metropolita. Cf. Herculano, \textunderscore Reacção Ultram.\textunderscore , 6 e 39.
\section{Metropólypo}
\begin{itemize}
\item {Grp. gram.:m.}
\end{itemize}
\begin{itemize}
\item {Utilização:Med.}
\end{itemize}
\begin{itemize}
\item {Proveniência:(Do gr. \textunderscore metra\textunderscore  + \textunderscore polupos\textunderscore )}
\end{itemize}
Pólypo do útero.
\section{Metroptose}
\begin{itemize}
\item {Grp. gram.:f.}
\end{itemize}
\begin{itemize}
\item {Utilização:Med.}
\end{itemize}
\begin{itemize}
\item {Proveniência:(Do gr. \textunderscore metra\textunderscore  + \textunderscore ptosis\textunderscore )}
\end{itemize}
Quéda ou descida do útero.
\section{Metrorragia}
\begin{itemize}
\item {Grp. gram.:f.}
\end{itemize}
\begin{itemize}
\item {Proveniência:(Do gr. \textunderscore metra\textunderscore  + \textunderscore rhagein\textunderscore )}
\end{itemize}
Hemorragia do útero.
\section{Metrorrexia}
\begin{itemize}
\item {fónica:csi}
\end{itemize}
\begin{itemize}
\item {Grp. gram.:f.}
\end{itemize}
\begin{itemize}
\item {Utilização:Med.}
\end{itemize}
\begin{itemize}
\item {Proveniência:(Do gr. \textunderscore metra\textunderscore  + \textunderscore rhexis\textunderscore )}
\end{itemize}
Ruptura da madre.
\section{Metrorrhagia}
\begin{itemize}
\item {Grp. gram.:f.}
\end{itemize}
\begin{itemize}
\item {Proveniência:(Do gr. \textunderscore metra\textunderscore  + \textunderscore rhagein\textunderscore )}
\end{itemize}
Hemorragia do útero.
\section{Metrorrhexia}
\begin{itemize}
\item {fónica:csi}
\end{itemize}
\begin{itemize}
\item {Grp. gram.:f.}
\end{itemize}
\begin{itemize}
\item {Utilização:Med.}
\end{itemize}
\begin{itemize}
\item {Proveniência:(Do gr. \textunderscore metra\textunderscore  + \textunderscore rhexis\textunderscore )}
\end{itemize}
Ruptura da madre.
\section{Metrosídero}
\begin{itemize}
\item {fónica:si}
\end{itemize}
\begin{itemize}
\item {Grp. gram.:m.}
\end{itemize}
\begin{itemize}
\item {Utilização:Bot.}
\end{itemize}
Gênero de plantas myrtáceas; o mesmo que \textunderscore pennacheiro\textunderscore .
\section{Metrossídero}
\begin{itemize}
\item {Grp. gram.:m.}
\end{itemize}
\begin{itemize}
\item {Utilização:Bot.}
\end{itemize}
Gênero de plantas mirtáceas; o mesmo que \textunderscore penacheiro\textunderscore .
\section{Metrotomia}
\begin{itemize}
\item {Grp. gram.:f.}
\end{itemize}
\begin{itemize}
\item {Proveniência:(Do gr. \textunderscore metra\textunderscore  + \textunderscore tome\textunderscore )}
\end{itemize}
Incisão do útero; operação cesariana.
\section{Metudo}
\begin{itemize}
\item {Utilização:ant.}
\end{itemize}
O mesmo que \textunderscore metido\textunderscore .
(Part. de \textunderscore meter\textunderscore )
\section{Metuendo}
\begin{itemize}
\item {Grp. gram.:adj.}
\end{itemize}
\begin{itemize}
\item {Proveniência:(Lat. \textunderscore metuendus\textunderscore )}
\end{itemize}
Que põe medo; terrivel.
\section{Metxina}
\begin{itemize}
\item {Grp. gram.:f.}
\end{itemize}
Árvore do Congo.
\section{Meu}
\begin{itemize}
\item {Grp. gram.:adj.}
\end{itemize}
\begin{itemize}
\item {Proveniência:(Lat. \textunderscore meus\textunderscore )}
\end{itemize}
(designativo da \textunderscore posse\textunderscore  que tem a pessôa que fala)
Relativo a mim.
Êsse, aquelle.
O tal.
\section{Meúdo}
\textunderscore adj.\textunderscore  (e der.)
O mesmo ou melhor que \textunderscore miúdo\textunderscore , etc.
\section{Meúl}
\begin{itemize}
\item {Grp. gram.:m.}
\end{itemize}
O mesmo que \textunderscore meão\textunderscore  do carro.
(Cp. \textunderscore meolo\textunderscore )
\section{Meúle}
\begin{itemize}
\item {Grp. gram.:m.}
\end{itemize}
O mesmo que \textunderscore meúl\textunderscore .
\section{Meúlo}
\begin{itemize}
\item {Grp. gram.:m.}
\end{itemize}
O mesmo que \textunderscore meúl\textunderscore .
\section{Meutanga}
\begin{itemize}
\item {fónica:me-u}
\end{itemize}
\begin{itemize}
\item {Grp. gram.:f.}
\end{itemize}
Planta chinesa, de flôres semelhantes á rosa.
\section{Mexão}
\begin{itemize}
\item {Grp. gram.:m.}
\end{itemize}
\begin{itemize}
\item {Utilização:Prov.}
\end{itemize}
\begin{itemize}
\item {Utilização:minh.}
\end{itemize}
\begin{itemize}
\item {Proveniência:(De \textunderscore mexer\textunderscore )}
\end{itemize}
Colhér de pau, para mexer papas.
\section{Mexediço}
\begin{itemize}
\item {Grp. gram.:adj.}
\end{itemize}
Que se mexe muito; movediço; inquieto.
\section{Mexedor}
\begin{itemize}
\item {Grp. gram.:adj.}
\end{itemize}
\begin{itemize}
\item {Grp. gram.:M.}
\end{itemize}
\begin{itemize}
\item {Utilização:Fig.}
\end{itemize}
Que mexe.
Objecto com que se mexe.
Intriguista.
\section{Mexedura}
\begin{itemize}
\item {Grp. gram.:f.}
\end{itemize}
Acto ou effeito de mexer.
\section{Mexelhão}
\begin{itemize}
\item {Grp. gram.:m.  e  adj.}
\end{itemize}
\begin{itemize}
\item {Utilização:Fam.}
\end{itemize}
\begin{itemize}
\item {Proveniência:(De \textunderscore mexer\textunderscore )}
\end{itemize}
Indivíduo, que mexe muito nos objectos; buliçoso.
Traquinas, travesso.
Metediço.
\section{Mexelhar}
\begin{itemize}
\item {Grp. gram.:v. i.}
\end{itemize}
\begin{itemize}
\item {Utilização:Fam.}
\end{itemize}
Sêr mexelhão.
Mexerucar.
\section{Mexer}
\begin{itemize}
\item {Grp. gram.:v. t.}
\end{itemize}
\begin{itemize}
\item {Grp. gram.:V. i.}
\end{itemize}
\begin{itemize}
\item {Proveniência:(Do lat. \textunderscore miscere\textunderscore )}
\end{itemize}
Dar movimento a: \textunderscore mexer as pernas\textunderscore .
Agitar.
Deslocar: \textunderscore mexer uma pedra\textunderscore .
Revolver: \textunderscore mexer o caldo\textunderscore .
Misturar; confundir.
Bulir, tocar.
\section{Mexericada}
\begin{itemize}
\item {Grp. gram.:f.}
\end{itemize}
O mesmo que \textunderscore mexerico\textunderscore .
\section{Mexericar}
\begin{itemize}
\item {Grp. gram.:v. t.}
\end{itemize}
\begin{itemize}
\item {Grp. gram.:V. i.}
\end{itemize}
\begin{itemize}
\item {Proveniência:(De \textunderscore mexer\textunderscore )}
\end{itemize}
Narrar astutamente e em segrêdo, com o fim de malquistar e intrigar.
Fazer intrigas; promover discórdias, desavenças.
\section{Mexerico}
\begin{itemize}
\item {Grp. gram.:m.}
\end{itemize}
Acto de mexericar.
Enrêdo; bisbilhotice; intriga.
\section{Mexeriqueira}
\begin{itemize}
\item {Grp. gram.:f.}
\end{itemize}
\begin{itemize}
\item {Proveniência:(De \textunderscore mexeriqueiro\textunderscore )}
\end{itemize}
Mulher, que faz mexericos; mulher intriguista.
\section{Mexeriqueiro}
\begin{itemize}
\item {Grp. gram.:m.}
\end{itemize}
\begin{itemize}
\item {Grp. gram.:Adj.}
\end{itemize}
Intriguista, bisbilhoteiro.
Que mexerica.
\section{Mexeriquice}
\begin{itemize}
\item {Grp. gram.:f.}
\end{itemize}
Acção de mexericar. Cf. Garrett, \textunderscore Viagens\textunderscore , 12.
\section{Mexeroto}
\begin{itemize}
\item {fónica:xerô}
\end{itemize}
\begin{itemize}
\item {Grp. gram.:adj.}
\end{itemize}
\begin{itemize}
\item {Utilização:Prov.}
\end{itemize}
\begin{itemize}
\item {Utilização:trasm.}
\end{itemize}
\begin{itemize}
\item {Proveniência:(De \textunderscore mexer\textunderscore )}
\end{itemize}
Inquieto; buliçoso. (Colhido em Valpaços)
\section{Mexerucar}
\begin{itemize}
\item {Grp. gram.:v. t.}
\end{itemize}
\begin{itemize}
\item {Utilização:Pop.}
\end{itemize}
O mesmo que \textunderscore mexer\textunderscore .
\section{Mexerufada}
\begin{itemize}
\item {Grp. gram.:f.}
\end{itemize}
\begin{itemize}
\item {Utilização:Pop.}
\end{itemize}
Comida de porcos.
Mixórdia.
Garrafada de remédio.
(Alter. de \textunderscore moxinifada\textunderscore , por infl. de \textunderscore mexer\textunderscore )
\section{Mexicana}
\begin{itemize}
\item {Grp. gram.:f.}
\end{itemize}
\begin{itemize}
\item {Proveniência:(De \textunderscore mexicano\textunderscore )}
\end{itemize}
Moéda mexicana, de prata, equivalente a 820 reis aproximadamente.
\section{Mexicano}
\begin{itemize}
\item {Grp. gram.:adj.}
\end{itemize}
\begin{itemize}
\item {Grp. gram.:M.}
\end{itemize}
Relativo ao México.
Habitante do México.
\section{México}
\begin{itemize}
\item {Grp. gram.:m.}
\end{itemize}
\begin{itemize}
\item {Proveniência:(De \textunderscore México\textunderscore , n. p.)}
\end{itemize}
Variedade de tabaco.
\section{Mexida}
\begin{itemize}
\item {Grp. gram.:f.}
\end{itemize}
\begin{itemize}
\item {Proveniência:(De \textunderscore mexido\textunderscore )}
\end{itemize}
Confusão; desordem; discórdia; reboliço.
Mixórdia.
\section{Mexido}
\begin{itemize}
\item {Grp. gram.:adj.}
\end{itemize}
\begin{itemize}
\item {Utilização:Ext.}
\end{itemize}
\begin{itemize}
\item {Grp. gram.:M.}
\end{itemize}
\begin{itemize}
\item {Utilização:Prov.}
\end{itemize}
\begin{itemize}
\item {Utilização:minh.}
\end{itemize}
\begin{itemize}
\item {Grp. gram.:Pl.}
\end{itemize}
\begin{itemize}
\item {Proveniência:(De \textunderscore mexer\textunderscore )}
\end{itemize}
Revolvido.
Agitado.
Inquieto.
Doce, que se prepara para a ceia da véspera do Natal e é feito de calda de açúcar, pão ralado, mel e casca de limão.
Intrigas.
Saracoteio, em certas danças.
\section{Mexilhão}
\begin{itemize}
\item {Grp. gram.:m.}
\end{itemize}
Gênero de molluscos, (\textunderscore mytilus\textunderscore ).
(Cp. cast. \textunderscore mejillón\textunderscore )
\section{Mexilho}
\begin{itemize}
\item {Grp. gram.:m.}
\end{itemize}
\begin{itemize}
\item {Proveniência:(De \textunderscore mexer\textunderscore )}
\end{itemize}
Barra de ferro, que prende a aiveca ao teiró, regulando-lhes o maior ou menor afastamento.
\section{Mexilhoeira}
\begin{itemize}
\item {Grp. gram.:f.}
\end{itemize}
Lugar, onde se criam mexilhões:«\textunderscore as grandes mexilhoeiras naturaes desappareceram\textunderscore ». \textunderscore Bol. da Soc. de Geogr.\textunderscore , XXX, 268.
\section{Mexoalho}
\begin{itemize}
\item {Grp. gram.:m.}
\end{itemize}
Porção de caranguejos ou plantas em putrefacção, para adubos de terrenos.
\section{Mexoeira}
\begin{itemize}
\item {Grp. gram.:f.}
\end{itemize}
Robusta árvore africana, da fam. das gramíneas, (\textunderscore penincellaria spicatta\textunderscore ).
Sementes miúdas e comestíveis dessa árvore.
\section{Mexonada}
\begin{itemize}
\item {Grp. gram.:f.}
\end{itemize}
\begin{itemize}
\item {Utilização:Ant.}
\end{itemize}
O mesmo que \textunderscore mexida\textunderscore .
\section{Mexorofada}
\begin{itemize}
\item {Grp. gram.:f.}
\end{itemize}
O mesmo que \textunderscore mexerufada\textunderscore .
\section{Mexuar}
\begin{itemize}
\item {Grp. gram.:m.}
\end{itemize}
\begin{itemize}
\item {Utilização:Ant.}
\end{itemize}
\begin{itemize}
\item {Proveniência:(T. ár.)}
\end{itemize}
Sala, onde o rei moiro dava audiência e onde se justiçavam os condemnados:«\textunderscore ...forão presos e levados ao mexuar...\textunderscore »Jer. de Mendonça, \textunderscore Jornada de África\textunderscore , l. III, c. IV, 158.
\section{Mexuda}
\begin{itemize}
\item {Grp. gram.:f.}
\end{itemize}
\begin{itemize}
\item {Utilização:Prov.}
\end{itemize}
\begin{itemize}
\item {Utilização:beir.}
\end{itemize}
\begin{itemize}
\item {Proveniência:(De \textunderscore mexudo\textunderscore )}
\end{itemize}
Papas de milho.
\section{Mexudo}
\begin{itemize}
\item {Grp. gram.:adj.}
\end{itemize}
\begin{itemize}
\item {Utilização:Ant.}
\end{itemize}
O mesmo que \textunderscore mexido\textunderscore .
\section{Mexueira}
\begin{itemize}
\item {Grp. gram.:f.}
\end{itemize}
Robusta árvore africana, da fam. das gramíneas, (\textunderscore penincellaria spicatta\textunderscore ).
Sementes miúdas e comestíveis dessa árvore.
\section{Mezanino}
\begin{itemize}
\item {Grp. gram.:m.}
\end{itemize}
\begin{itemize}
\item {Utilização:Constr.}
\end{itemize}
\begin{itemize}
\item {Proveniência:(It. \textunderscore mezzanino\textunderscore )}
\end{itemize}
Andar pouco elevado, entre dois andares altos.
Janela de maior largura que altura.
\section{Mezena}
\begin{itemize}
\item {Grp. gram.:f.}
\end{itemize}
\begin{itemize}
\item {Utilização:Náut.}
\end{itemize}
\begin{itemize}
\item {Proveniência:(Do it. \textunderscore mezzana\textunderscore )}
\end{itemize}
Vela, que se enverga na carangueja do mastro de ré.
\section{Mezengro}
\begin{itemize}
\item {Grp. gram.:m.}
\end{itemize}
\begin{itemize}
\item {Utilização:Prov.}
\end{itemize}
O mesmo que \textunderscore megengra\textunderscore .
\section{Mezereão}
\begin{itemize}
\item {Grp. gram.:m.}
\end{itemize}
Gênero de plantas thymeliáceas, (\textunderscore daphne mezerium\textunderscore , Lin.).
\section{Mezereína}
\begin{itemize}
\item {Grp. gram.:f.}
\end{itemize}
\begin{itemize}
\item {Utilização:Chím.}
\end{itemize}
Princípio activo do mezereão.
\section{Mezeréu-menór}
\begin{itemize}
\item {Grp. gram.:m.}
\end{itemize}
Planta thymeliácea, também conhecida por \textunderscore lauréola macha\textunderscore .
\section{Mèzinha}
\begin{itemize}
\item {Grp. gram.:f.}
\end{itemize}
\begin{itemize}
\item {Utilização:Pop.}
\end{itemize}
\begin{itemize}
\item {Proveniência:(De \textunderscore medicina\textunderscore )}
\end{itemize}
Líquido para clister.
Qualquer medicamento.
\section{Mèzinhadoiro}
\begin{itemize}
\item {Grp. gram.:m.}
\end{itemize}
\begin{itemize}
\item {Utilização:Ant.}
\end{itemize}
\begin{itemize}
\item {Proveniência:(De \textunderscore mèzinhar\textunderscore )}
\end{itemize}
Foragem, que se pagava para despesas de enfermaria.
\section{Mèzinhadouro}
\begin{itemize}
\item {Grp. gram.:m.}
\end{itemize}
\begin{itemize}
\item {Utilização:Ant.}
\end{itemize}
\begin{itemize}
\item {Proveniência:(De \textunderscore mèzinhar\textunderscore )}
\end{itemize}
Foragem, que se pagava para despesas de enfermaria.
\section{Mèzinhar}
\begin{itemize}
\item {Grp. gram.:v. t.}
\end{itemize}
\begin{itemize}
\item {Utilização:Pop.}
\end{itemize}
Applicar mèzinhas a.
Medicar.
\section{Mèzinheira}
\begin{itemize}
\item {Grp. gram.:f.}
\end{itemize}
\begin{itemize}
\item {Proveniência:(De \textunderscore mèzinheiro\textunderscore )}
\end{itemize}
Mulher, que applica mèzinhas.
Mulher achacadiça, sempre preoccupada com mèzinhas.
\section{Mèzinheiro}
\begin{itemize}
\item {Grp. gram.:m.}
\end{itemize}
Aquelle que faz ou applica mèzinhas.
Curandeiro.
O que anda sempre a medicar-se.
\section{Mèzinhice}
\begin{itemize}
\item {Grp. gram.:f.}
\end{itemize}
\begin{itemize}
\item {Utilização:Pop.}
\end{itemize}
\begin{itemize}
\item {Proveniência:(De \textunderscore mèzinha\textunderscore )}
\end{itemize}
Remédio caseiro; remédio ou práticas de curandeiro.
\section{Mezquinho}
\begin{itemize}
\item {Grp. gram.:adj.}
\end{itemize}
\begin{itemize}
\item {Utilização:Bras}
\end{itemize}
\begin{itemize}
\item {Grp. gram.:M.}
\end{itemize}
\begin{itemize}
\item {Proveniência:(Do cast. \textunderscore mezquino\textunderscore )}
\end{itemize}
Privado do que é necessário.
Pobre.
Insignificante.
Infeliz.
Estéril.
Avarento; miserável.
Que não consente o freio, (falando-se do cavallo).
Indivíduo dasgraçado:«\textunderscore a mísera e mesquinha, que depois de ser morta foi rainha.\textunderscore »\textunderscore Lusíadas\textunderscore , III.
Avarento.
\section{Mezzanino}
\begin{itemize}
\item {Grp. gram.:m.}
\end{itemize}
\begin{itemize}
\item {Utilização:Constr.}
\end{itemize}
\begin{itemize}
\item {Proveniência:(It. \textunderscore mezzanino\textunderscore )}
\end{itemize}
Andar pouco elevado, entre dois andares altos.
Janela de maior largura que altura.
\section{Mi}
\begin{itemize}
\item {Grp. gram.:m.}
\end{itemize}
Terceira nota da escala musical.
Sinal representativo desta nota.
Primeira corda do violino; quarta corda do rabecão.
(Da 1.^a sýllaba do lat. \textunderscore mira\textunderscore , aproveitada por G. de Arezzo, com as 1.^{as} sýllabas de algumas palavras de um hymno religioso, para a formação da antiga escala musical)
\section{Mi}
\begin{itemize}
\item {Grp. gram.:pron.}
\end{itemize}
(Fórma ant. de \textunderscore mim\textunderscore )
\section{Mi}
\begin{itemize}
\item {Grp. gram.:pron.}
\end{itemize}
\begin{itemize}
\item {Utilização:Ant.}
\end{itemize}
O mesmo que \textunderscore meu\textunderscore .
(Cast. \textunderscore mi\textunderscore )
\section{Miada}
\begin{itemize}
\item {Grp. gram.:f.}
\end{itemize}
\begin{itemize}
\item {Proveniência:(De \textunderscore miar\textunderscore )}
\end{itemize}
O miar de muitos gatos.
\section{Miadela}
\begin{itemize}
\item {Grp. gram.:f.}
\end{itemize}
\begin{itemize}
\item {Proveniência:(De \textunderscore miar\textunderscore )}
\end{itemize}
Grito do gato.
\section{Miado}
\begin{itemize}
\item {Grp. gram.:m.}
\end{itemize}
\begin{itemize}
\item {Proveniência:(De \textunderscore miar\textunderscore )}
\end{itemize}
Grito do gato.
\section{Miador}
\begin{itemize}
\item {Grp. gram.:m.  e  adj.}
\end{itemize}
O que mia muito.
\section{Miadura}
\begin{itemize}
\item {Grp. gram.:f.}
\end{itemize}
\begin{itemize}
\item {Proveniência:(De \textunderscore miar\textunderscore )}
\end{itemize}
Série de miados.
\section{Mialhar}
\begin{itemize}
\item {Grp. gram.:m.}
\end{itemize}
Fio de amarras velhas, com que se fazem lambazes.
Lambazes ou vassoiras, para enxugar qualquer superfície molhada do navio.
Cordel.
\section{Miapia}
\begin{itemize}
\item {Grp. gram.:f.}
\end{itemize}
Pássaro fissirostro da África occidental.
\section{Miapiata}
\begin{itemize}
\item {Grp. gram.:f.}
\end{itemize}
\begin{itemize}
\item {Utilização:Bras}
\end{itemize}
O mesmo que \textunderscore beiju\textunderscore .
\section{Miapiúlo}
\begin{itemize}
\item {Grp. gram.:m.}
\end{itemize}
\begin{itemize}
\item {Proveniência:(T. afr.)}
\end{itemize}
Reptil ophídio de Catumbela.
\section{Miar}
\begin{itemize}
\item {Grp. gram.:v. i.}
\end{itemize}
\begin{itemize}
\item {Utilização:Gír.}
\end{itemize}
Dar mios.
Gritar.
\section{Miasma}
\begin{itemize}
\item {Grp. gram.:m.}
\end{itemize}
\begin{itemize}
\item {Proveniência:(Lat. \textunderscore miasma\textunderscore )}
\end{itemize}
Emanação mephítica.
Emanação, procedente de moléstias contagiosas.
\section{Miasmático}
\begin{itemize}
\item {Grp. gram.:adj.}
\end{itemize}
Que produz miasmas.
Resultante de miasmas.
\section{Miau}
\begin{itemize}
\item {Grp. gram.:m.}
\end{itemize}
\begin{itemize}
\item {Utilização:Fam.}
\end{itemize}
\begin{itemize}
\item {Utilização:Infant.}
\end{itemize}
\begin{itemize}
\item {Proveniência:(T. onom.)}
\end{itemize}
Voz do gato.
O gato.
\section{Miba}
\begin{itemize}
\item {Grp. gram.:f.}
\end{itemize}
\begin{itemize}
\item {Utilização:Ant.}
\end{itemize}
Xarope de marmelo.
(Ár. \textunderscore mibah\textunderscore )
\section{Mica}
\begin{itemize}
\item {Grp. gram.:f.}
\end{itemize}
\begin{itemize}
\item {Proveniência:(Lat. \textunderscore mica\textunderscore )}
\end{itemize}
Pequena porção; bocado; migalha.
Pedra, composta de lâminas finas, com brilho metállico, doirado, argentado ou bronzeado.
\section{Mica}
\begin{itemize}
\item {Grp. gram.:f.}
\end{itemize}
\begin{itemize}
\item {Utilização:T. de Caminha}
\end{itemize}
O mesmo que \textunderscore cabra\textunderscore ^1.
\section{Micáceo}
\begin{itemize}
\item {Grp. gram.:adj.}
\end{itemize}
\begin{itemize}
\item {Proveniência:(De \textunderscore mica\textunderscore ^1)}
\end{itemize}
Que contém mica ou é da natureza della.
Que tem apparência de mica.
Diz-se dos órgãos vegetaes, em que há pellículas com a apparência da mica.
\section{Micado}
\begin{itemize}
\item {Grp. gram.:m.}
\end{itemize}
Antigamente, título da suprema autoridade religiosa, no Japão.
Hoje, título do soberano japonês.
\section{Micante}
\begin{itemize}
\item {Grp. gram.:adj.}
\end{itemize}
\begin{itemize}
\item {Utilização:Poét.}
\end{itemize}
\begin{itemize}
\item {Proveniência:(Lat. \textunderscore micans\textunderscore )}
\end{itemize}
Brilhante.
\section{Micar}
\begin{itemize}
\item {Grp. gram.:v. i.}
\end{itemize}
\begin{itemize}
\item {Proveniência:(De \textunderscore mico\textunderscore )}
\end{itemize}
Fazer mico ou cêrco, no jôgo.
\section{Micaxisto}
\begin{itemize}
\item {Grp. gram.:m.}
\end{itemize}
\begin{itemize}
\item {Proveniência:(De \textunderscore mica\textunderscore ^1 + \textunderscore xisto\textunderscore )}
\end{itemize}
Espécie de rocha, cujos elementos essenciaes são o quartzo e a mica.
\section{Micção}
\begin{itemize}
\item {Grp. gram.:f.}
\end{itemize}
\begin{itemize}
\item {Proveniência:(Do lat. \textunderscore mictio\textunderscore )}
\end{itemize}
Acto de urinar.
\section{Micendeira}
\begin{itemize}
\item {Grp. gram.:f.}
\end{itemize}
Nome que, nalguns pontos de Angola, se dá ao sycómoro.
\section{Micha}
\begin{itemize}
\item {Grp. gram.:f.}
\end{itemize}
\begin{itemize}
\item {Proveniência:(Fr. \textunderscore miche\textunderscore )}
\end{itemize}
Fatia de pão, fabricado de farinhas diversas e misturadas.
\section{Michela}
\begin{itemize}
\item {Grp. gram.:f.}
\end{itemize}
\begin{itemize}
\item {Utilização:Pleb.}
\end{itemize}
O mesmo que \textunderscore meretriz\textunderscore . Cf. Camillo, \textunderscore Maria da Fonte\textunderscore , 37.
\section{Michelos}
\begin{itemize}
\item {Grp. gram.:m. pl.}
\end{itemize}
\begin{itemize}
\item {Utilização:Náut.}
\end{itemize}
Fios grossos, com que se liga a amarra ao cabo de alar.
\section{Micho}
\begin{itemize}
\item {Grp. gram.:m.}
\end{itemize}
\begin{itemize}
\item {Utilização:deprec.}
\end{itemize}
\begin{itemize}
\item {Utilização:Ant.}
\end{itemize}
Súcio, sujeito; typo:«\textunderscore ...indo-me o micho a parar á tábola...\textunderscore »F. Manuel, \textunderscore Apólogos\textunderscore . Cf. G. Vicente, I, 178.
\section{Micho}
\begin{itemize}
\item {Grp. gram.:m.}
\end{itemize}
O mesmo que \textunderscore micha\textunderscore .
\section{Micholi}
\begin{itemize}
\item {Grp. gram.:m.}
\end{itemize}
Peixe marítimo, ordinário, do Brasil.
\section{Miciriri}
\begin{itemize}
\item {Grp. gram.:m.}
\end{itemize}
Espécie de erva africana.
\section{Miclas}
\begin{itemize}
\item {Grp. gram.:m.  e  f.}
\end{itemize}
\begin{itemize}
\item {Utilização:Prov.}
\end{itemize}
\begin{itemize}
\item {Utilização:trasm.}
\end{itemize}
Pessôa doente.
\section{Mico}
\begin{itemize}
\item {Grp. gram.:m.}
\end{itemize}
\begin{itemize}
\item {Utilização:Fig.}
\end{itemize}
\begin{itemize}
\item {Utilização:Prov.}
\end{itemize}
\begin{itemize}
\item {Utilização:minh.}
\end{itemize}
Pequeno macaco do Brasil.
O mesmo que \textunderscore cêrco\textunderscore , no jôgo de asar.
Pessôa de aspecto grutesco.
O diabo.
\section{Micocó}
\begin{itemize}
\item {Grp. gram.:m.}
\end{itemize}
Planta aromática da ilha de San-Thomé.
\section{Micondó}
\begin{itemize}
\item {Grp. gram.:m.}
\end{itemize}
Nome santhomense do imbondeiro.
\section{Micónia}
\begin{itemize}
\item {Grp. gram.:f.}
\end{itemize}
\begin{itemize}
\item {Proveniência:(De \textunderscore Micon\textunderscore , n. p. cast.)}
\end{itemize}
Gênero de plantas melastomáceas.
\section{Micro...}
\begin{itemize}
\item {Grp. gram.:pref.}
\end{itemize}
\begin{itemize}
\item {Proveniência:(Do gr. \textunderscore mikros\textunderscore )}
\end{itemize}
(designativo de \textunderscore pequenez\textunderscore )
\section{Microacústico}
\begin{itemize}
\item {Grp. gram.:adj.}
\end{itemize}
\begin{itemize}
\item {Proveniência:(De \textunderscore micro...\textunderscore  + \textunderscore acústico\textunderscore )}
\end{itemize}
Diz-se dos instrumentos que reforçam os sons.
\section{Microbial}
\begin{itemize}
\item {Grp. gram.:adj.}
\end{itemize}
Relativo a micróbio.
\section{Microbiano}
\begin{itemize}
\item {Grp. gram.:adj.}
\end{itemize}
O mesmo que \textunderscore microbial\textunderscore .
\section{Microbicida}
\begin{itemize}
\item {Grp. gram.:m.  e  adj.}
\end{itemize}
\begin{itemize}
\item {Proveniência:(De \textunderscore micróbio\textunderscore  + lat. \textunderscore caedere\textunderscore )}
\end{itemize}
Aquillo que serve para destruir micróbios.
\section{Micróbico}
\begin{itemize}
\item {Grp. gram.:adj.}
\end{itemize}
O mesmo que \textunderscore microbial\textunderscore .
\section{Micróbio}
\begin{itemize}
\item {Grp. gram.:m.}
\end{itemize}
\begin{itemize}
\item {Utilização:Med.}
\end{itemize}
\begin{itemize}
\item {Proveniência:(Do gr. \textunderscore mikros\textunderscore  + \textunderscore bios\textunderscore )}
\end{itemize}
Sêr microscópico, vivo, que habita no ar e na água.
Sêr unicellular, pertencente ao grupo das algas, e apenas visível com o auxilio de grande ampliação microscópica.
Bacillo; bactéria.
\section{Microbiologia}
\begin{itemize}
\item {Grp. gram.:f.}
\end{itemize}
Estudo ou tratado dos micróbios.
(Cp. \textunderscore microbiólogo\textunderscore )
\section{Microbiológico}
\begin{itemize}
\item {Grp. gram.:adj.}
\end{itemize}
Relativo á microbiologia.
\section{Microbiologista}
\begin{itemize}
\item {Grp. gram.:m.}
\end{itemize}
\begin{itemize}
\item {Proveniência:(De \textunderscore micróbio\textunderscore  + gr. \textunderscore logos\textunderscore )}
\end{itemize}
Tratadista de microbiologia.
\section{Microbiólogo}
\begin{itemize}
\item {Grp. gram.:m.}
\end{itemize}
\begin{itemize}
\item {Proveniência:(De \textunderscore micróbio\textunderscore  + gr. \textunderscore logos\textunderscore )}
\end{itemize}
Tratadista de microbiologia.
\section{Microbista}
\begin{itemize}
\item {Grp. gram.:m.}
\end{itemize}
(V.microbiologista)
\section{Microcefalia}
\begin{itemize}
\item {Grp. gram.:f.}
\end{itemize}
Qualidade de microcéfalo.
\section{Microcefálico}
\begin{itemize}
\item {Grp. gram.:adj.}
\end{itemize}
Relativo a microcefalia.
\section{Microcéfalo}
\begin{itemize}
\item {Grp. gram.:m.  e  adj.}
\end{itemize}
\begin{itemize}
\item {Proveniência:(Do gr. \textunderscore mikros\textunderscore  + \textunderscore kephale\textunderscore )}
\end{itemize}
Aquele que tem a cabeça pequena ou a massa encefálica muito deminuta.
Que tem flôres reunidas em pequenos capítulos, (falando-se de plantas).
Idiota; pouco inteligente.
\section{Microcephalia}
\begin{itemize}
\item {Grp. gram.:f.}
\end{itemize}
Qualidade de microcéphalo.
\section{Microcephálico}
\begin{itemize}
\item {Grp. gram.:adj.}
\end{itemize}
Relativo a microcephalia.
\section{Microcéphalo}
\begin{itemize}
\item {Grp. gram.:m.  e  adj.}
\end{itemize}
\begin{itemize}
\item {Proveniência:(Do gr. \textunderscore mikros\textunderscore  + \textunderscore kephale\textunderscore )}
\end{itemize}
Aquelle que tem a cabeça pequena ou a massa encephálica muito deminuta.
Que tem flôres reunidas em pequenos capítulos, (falando-se de plantas).
Idiota; pouco intelligente.
\section{Micrócero}
\begin{itemize}
\item {Grp. gram.:adj.}
\end{itemize}
\begin{itemize}
\item {Utilização:Zool.}
\end{itemize}
\begin{itemize}
\item {Proveniência:(Do gr. \textunderscore mikros\textunderscore  + \textunderscore keras\textunderscore )}
\end{itemize}
Que tem antennas curtas.
\section{Microchímica}
\begin{itemize}
\item {fónica:qui}
\end{itemize}
\begin{itemize}
\item {Grp. gram.:f.}
\end{itemize}
\begin{itemize}
\item {Proveniência:(De \textunderscore micro...\textunderscore  + \textunderscore Chímica\textunderscore )}
\end{itemize}
Emprêgo do microscópio, para verificar os caracteres dos princípios chímicos, cujos crystaes não podem sêr observados a ôlho nu.
\section{Microchímico}
\begin{itemize}
\item {fónica:quí}
\end{itemize}
\begin{itemize}
\item {Grp. gram.:adj.}
\end{itemize}
Relativo á microchímica.
\section{Microclina}
\begin{itemize}
\item {Grp. gram.:f.}
\end{itemize}
Feldspatho potássico, que crystalliza no systema triclínico.
\section{Micrococco}
\begin{itemize}
\item {Grp. gram.:m.}
\end{itemize}
O mesmo ou melhor que \textunderscore microcoques\textunderscore .
\section{Micrococo}
\begin{itemize}
\item {Grp. gram.:m.}
\end{itemize}
O mesmo ou melhor que \textunderscore microcoques\textunderscore .
\section{Microcoques}
\begin{itemize}
\item {Grp. gram.:m. Pl.}
\end{itemize}
Bactérias ou indivíduos microscópicos, da extremidade inferior da escala vegetal, e cujo comprimento abrange apenas algumas centésimas millésimas de millímetro.
(Voc. mal formado, do gr. \textunderscore mikros\textunderscore  + fr. \textunderscore coque\textunderscore , invólucro)
\section{Microcósmico}
\begin{itemize}
\item {Grp. gram.:adj.}
\end{itemize}
Relativo ao microcosmo.
\section{Microcosmo}
\begin{itemize}
\item {Grp. gram.:m.}
\end{itemize}
\begin{itemize}
\item {Proveniência:(Do gr. \textunderscore mikros\textunderscore  + \textunderscore kosmos\textunderscore )}
\end{itemize}
Pequeno mundo.
O homem, segundo alguns philósophos.
\section{Microcosmologia}
\begin{itemize}
\item {Grp. gram.:f.}
\end{itemize}
\begin{itemize}
\item {Proveniência:(De \textunderscore microcosmo\textunderscore )}
\end{itemize}
Descripção do corpo humano.
\section{Micro-crystallino}
\begin{itemize}
\item {Grp. gram.:adj.}
\end{itemize}
\begin{itemize}
\item {Utilização:Geol.}
\end{itemize}
Diz-se do estado dos mineraes, em que as moléculas não obedecem a nenhuma orientação regular, e cujas propriedades só se pódem observar com o microscópio.
\section{Microdáctilo}
\begin{itemize}
\item {Grp. gram.:adj.}
\end{itemize}
\begin{itemize}
\item {Utilização:Zool.}
\end{itemize}
\begin{itemize}
\item {Proveniência:(Do gr. \textunderscore mikros\textunderscore  + \textunderscore dactulos\textunderscore )}
\end{itemize}
Que tem dedos curtos.
\section{Microdáctylo}
\begin{itemize}
\item {Grp. gram.:adj.}
\end{itemize}
\begin{itemize}
\item {Utilização:Zool.}
\end{itemize}
\begin{itemize}
\item {Proveniência:(Do gr. \textunderscore mikros\textunderscore  + \textunderscore dactulos\textunderscore )}
\end{itemize}
Que tem dedos curtos.
\section{Microdonte}
\begin{itemize}
\item {Grp. gram.:adj.}
\end{itemize}
\begin{itemize}
\item {Proveniência:(Do gr. \textunderscore mikros\textunderscore  + \textunderscore odous\textunderscore , \textunderscore odontos\textunderscore )}
\end{itemize}
Que tem dentes pequenos.
\section{Microfilo}
\begin{itemize}
\item {Grp. gram.:adj.}
\end{itemize}
\begin{itemize}
\item {Utilização:Bot.}
\end{itemize}
\begin{itemize}
\item {Proveniência:(Do gr. \textunderscore mikros\textunderscore  + \textunderscore phullon\textunderscore )}
\end{itemize}
Que tem fôlhas pequenas.
\section{Micrófita}
\begin{itemize}
\item {Grp. gram.:f.}
\end{itemize}
\begin{itemize}
\item {Proveniência:(Do gr. \textunderscore mikros\textunderscore  + \textunderscore phuton\textunderscore )}
\end{itemize}
Vegetal extremamente pequeno.
\section{Microfítico}
\begin{itemize}
\item {Grp. gram.:adj.}
\end{itemize}
Relativo ás micrófitas.
\section{Micrófito}
\begin{itemize}
\item {Grp. gram.:m.}
\end{itemize}
O mesmo que \textunderscore micrófita\textunderscore .
\section{Microfonia}
\begin{itemize}
\item {Grp. gram.:f.}
\end{itemize}
\begin{itemize}
\item {Proveniência:(Do gr. \textunderscore mikros\textunderscore  + \textunderscore phone\textunderscore )}
\end{itemize}
Fraqueza da voz.
\section{Microfónico}
\begin{itemize}
\item {Grp. gram.:adj.}
\end{itemize}
\begin{itemize}
\item {Proveniência:(De \textunderscore microfonia\textunderscore )}
\end{itemize}
Que tem voz fraca.
Que torna fraco um som.
\section{Microfónio}
\begin{itemize}
\item {Grp. gram.:m.}
\end{itemize}
\begin{itemize}
\item {Proveniência:(Do gr. \textunderscore mikros\textunderscore  + \textunderscore phone\textunderscore )}
\end{itemize}
Aparelho, inventado por Hughes e destinado a aumentar a intensidade do som.
\section{Microfonógrafo}
\begin{itemize}
\item {Grp. gram.:m.}
\end{itemize}
\begin{itemize}
\item {Proveniência:(Do gr. \textunderscore mikros\textunderscore  + \textunderscore phone\textunderscore  + \textunderscore graphein\textunderscore )}
\end{itemize}
Novo aparelho, inventado em 1897, para tornar percebidos os sons mais tênues, e empregado no tratamento da surdez. Cf. \textunderscore micróphono\textunderscore .
\section{Microftalmo}
\begin{itemize}
\item {Grp. gram.:m.}
\end{itemize}
\begin{itemize}
\item {Proveniência:(Do gr. \textunderscore mikros\textunderscore  + \textunderscore ophtalmos\textunderscore )}
\end{itemize}
Estado do ôlho, cujo volume é inferior ao normal.
\section{Microglosso}
\begin{itemize}
\item {Grp. gram.:adj.}
\end{itemize}
\begin{itemize}
\item {Grp. gram.:M. pl.}
\end{itemize}
\begin{itemize}
\item {Proveniência:(Do gr. \textunderscore mikros\textunderscore  + \textunderscore glossa\textunderscore )}
\end{itemize}
Que tem a língua curta.
Gênero de aves da família dos papagaios.
\section{Micrognatho}
\begin{itemize}
\item {Grp. gram.:adj.}
\end{itemize}
\begin{itemize}
\item {Proveniência:(Do gr. \textunderscore mikros\textunderscore  + \textunderscore gnathos\textunderscore )}
\end{itemize}
Que tem pequenas maxillas.
\section{Micrognato}
\begin{itemize}
\item {Grp. gram.:adj.}
\end{itemize}
\begin{itemize}
\item {Proveniência:(Do gr. \textunderscore mikros\textunderscore  + \textunderscore gnathos\textunderscore )}
\end{itemize}
Que tem pequenas maxilas.
\section{Micrografia}
\begin{itemize}
\item {Grp. gram.:f.}
\end{itemize}
Descripção dos objectos estudados com o auxílio do microscópio.
Tudo que é relativo ao emprêgo do microscópio.
(Cp. \textunderscore micrógrafo\textunderscore )
\section{Micrográfico}
\begin{itemize}
\item {Grp. gram.:adj.}
\end{itemize}
Relativo á micrografia.
\section{Micrógrafo}
\begin{itemize}
\item {Grp. gram.:m.}
\end{itemize}
\begin{itemize}
\item {Proveniência:(Do gr. \textunderscore mikros\textunderscore  + \textunderscore graphein\textunderscore )}
\end{itemize}
Aquele que é versado em micrografia.
\section{Microgranito}
\begin{itemize}
\item {Grp. gram.:m.}
\end{itemize}
\begin{itemize}
\item {Utilização:Geol.}
\end{itemize}
\begin{itemize}
\item {Proveniência:(De \textunderscore micro...\textunderscore  + \textunderscore granito\textunderscore )}
\end{itemize}
Granito compacto, em que, a ôlho nu, se não descobrem os seus elementos componentes.--É rocha privativa dos terrenos eruptivos.
\section{Micrographia}
\begin{itemize}
\item {Grp. gram.:f.}
\end{itemize}
Descripção dos objectos estudados com o auxílio do microscópio.
Tudo que é relativo ao emprêgo do microscópio.
(Cp. \textunderscore micrógrapho\textunderscore )
\section{Micrográphico}
\begin{itemize}
\item {Grp. gram.:adj.}
\end{itemize}
Relativo á micrographia.
\section{Micrógrapho}
\begin{itemize}
\item {Grp. gram.:m.}
\end{itemize}
\begin{itemize}
\item {Proveniência:(Do gr. \textunderscore mikros\textunderscore  + \textunderscore graphein\textunderscore )}
\end{itemize}
Aquelle que é versado em micrographia.
\section{Microlena}
\begin{itemize}
\item {Grp. gram.:f.}
\end{itemize}
\begin{itemize}
\item {Proveniência:(Do gr. \textunderscore mikros\textunderscore  + \textunderscore laina\textunderscore )}
\end{itemize}
Gênero de gramíneas da Nova-Hollanda.
\section{Microlépide}
\begin{itemize}
\item {Grp. gram.:f.}
\end{itemize}
Gênero de plantas melastomáceas.
\section{Microlepidóptero}
\begin{itemize}
\item {Grp. gram.:m.}
\end{itemize}
\begin{itemize}
\item {Proveniência:(De \textunderscore micro...\textunderscore  + \textunderscore lepidóptero\textunderscore )}
\end{itemize}
Pequeno lepidóptero.
\section{Microlícia}
\begin{itemize}
\item {Grp. gram.:f.}
\end{itemize}
Gênero de plantas melastomáceas.
\section{Microlíthico}
\begin{itemize}
\item {Grp. gram.:adj.}
\end{itemize}
Relativo ao micrólitho.
\section{Micrólitho}
\begin{itemize}
\item {Grp. gram.:m.}
\end{itemize}
\begin{itemize}
\item {Utilização:Geol.}
\end{itemize}
\begin{itemize}
\item {Proveniência:(Do gr. \textunderscore mikros\textunderscore  + \textunderscore lithos\textunderscore )}
\end{itemize}
Fórmade crystal, microscópica e prismatoide.
\section{Microlítico}
\begin{itemize}
\item {Grp. gram.:adj.}
\end{itemize}
Relativo ao micrólito.
\section{Micrólito}
\begin{itemize}
\item {Grp. gram.:m.}
\end{itemize}
\begin{itemize}
\item {Utilização:Geol.}
\end{itemize}
\begin{itemize}
\item {Proveniência:(Do gr. \textunderscore mikros\textunderscore  + \textunderscore lithos\textunderscore )}
\end{itemize}
Fórma de cristal, microscópica e prismatoide.
\section{Micrologia}
\begin{itemize}
\item {Grp. gram.:f.}
\end{itemize}
Tratado á cêrca de objectos microscópicos.
Discurso froixo ou sem colorido.
(Cp. \textunderscore micrólogo\textunderscore )
\section{Micrológico}
\begin{itemize}
\item {Grp. gram.:adj.}
\end{itemize}
Relativo a micrologia.
\section{Micrólogo}
\begin{itemize}
\item {Grp. gram.:m.}
\end{itemize}
\begin{itemize}
\item {Proveniência:(Do gr. \textunderscore mikros\textunderscore  + \textunderscore logos\textunderscore )}
\end{itemize}
Aquelle que é versado em micrologia.
Aquelle que se importa muito de bagatelas.
Pequeno discurso.
\section{Micrómato}
\begin{itemize}
\item {Grp. gram.:adj.}
\end{itemize}
\begin{itemize}
\item {Proveniência:(Do gr. \textunderscore mikros\textunderscore  + \textunderscore omma\textunderscore )}
\end{itemize}
Que tem pequenos olhos.
\section{Micrómega}
\begin{itemize}
\item {Grp. gram.:m.}
\end{itemize}
\begin{itemize}
\item {Proveniência:(Do gr. \textunderscore mikros\textunderscore  + \textunderscore megas\textunderscore )}
\end{itemize}
Gênero de plantas cryptogâmicas.
\section{Micrómegas}
\begin{itemize}
\item {Grp. gram.:m.}
\end{itemize}
\begin{itemize}
\item {Proveniência:(De \textunderscore Micrómegas\textunderscore , n. p. do protagonistade um conto de Voltaire. Cp. \textunderscore micrómega\textunderscore )}
\end{itemize}
Anão. Cf. O'Neill, \textunderscore Fabulário\textunderscore , 802.
\section{Micrómego}
\begin{itemize}
\item {Grp. gram.:m.}
\end{itemize}
Instrumento de Mathemática, de 15 graus, para medir terra.
(Cp. \textunderscore micrómega\textunderscore )
\section{Micromelia}
\begin{itemize}
\item {Grp. gram.:f.}
\end{itemize}
\begin{itemize}
\item {Proveniência:(Do gr. \textunderscore mikros\textunderscore  + \textunderscore melos\textunderscore )}
\end{itemize}
Monstruosidade, caracterizada pela excessiva pequenez de algum membro.
\section{Microméria}
\begin{itemize}
\item {Grp. gram.:f.}
\end{itemize}
\begin{itemize}
\item {Proveniência:(Do gr. \textunderscore mikros\textunderscore  + \textunderscore meros\textunderscore )}
\end{itemize}
Gênero de plantas labiadas.
\section{Micrómero}
\begin{itemize}
\item {Grp. gram.:adj.}
\end{itemize}
\begin{itemize}
\item {Proveniência:(Do gr. \textunderscore mikros\textunderscore  + \textunderscore meros\textunderscore )}
\end{itemize}
Que é delgado em todos os membros e appêndices.
\section{Micrómetra}
\begin{itemize}
\item {Grp. gram.:m.}
\end{itemize}
Aquelle que é versado em micrometria.
\section{Micrometria}
\begin{itemize}
\item {Grp. gram.:f.}
\end{itemize}
Applicação do micrómetro; arte de o usar.
\section{Micrometricamente}
\begin{itemize}
\item {Grp. gram.:adv.}
\end{itemize}
De modo micrométrico; por meio de micrómetro.
\section{Micrométrico}
\begin{itemize}
\item {Grp. gram.:adj.}
\end{itemize}
Relativo ao micrómetro.
\section{Micrómetro}
\begin{itemize}
\item {Grp. gram.:m.}
\end{itemize}
\begin{itemize}
\item {Proveniência:(Do gr. \textunderscore mikros\textunderscore  + \textunderscore metron\textunderscore )}
\end{itemize}
Instrumento, para medir a grandeza dos objectos observados pelo microscópio.
Instrumento, para medir pequenas dimensões.
Instrumento, para medir o diâmetro apparente dos astros.
\section{Micromicetos}
\begin{itemize}
\item {Grp. gram.:m. pl.}
\end{itemize}
\begin{itemize}
\item {Proveniência:(Do gr. \textunderscore mikros\textunderscore  + \textunderscore mukes\textunderscore )}
\end{itemize}
Plantas criptogâmicas, que fermentam as bebidas alcoólicas.
Leveduras.
\section{Micromilímetro}
\begin{itemize}
\item {Grp. gram.:m.}
\end{itemize}
Milésima parte do milímetro.
\section{Micromillímetro}
\begin{itemize}
\item {Grp. gram.:m.}
\end{itemize}
Millésima parte do millímetro.
\section{Micrómmato}
\begin{itemize}
\item {Grp. gram.:adj.}
\end{itemize}
\begin{itemize}
\item {Proveniência:(Do gr. \textunderscore mikros\textunderscore  + \textunderscore omma\textunderscore )}
\end{itemize}
Que tem pequenos olhos.
\section{Micromorfite}
\begin{itemize}
\item {Grp. gram.:f.}
\end{itemize}
\begin{itemize}
\item {Utilização:Geol.}
\end{itemize}
\begin{itemize}
\item {Proveniência:(Do gr. \textunderscore mikros\textunderscore  + \textunderscore morphe\textunderscore )}
\end{itemize}
Fórma rudimentar de cristal, sem regularidade geométrica.
\section{Micromorfítico}
\begin{itemize}
\item {Grp. gram.:adj.}
\end{itemize}
Relativo á micromorfite.
\section{Micromorphite}
\begin{itemize}
\item {Grp. gram.:f.}
\end{itemize}
\begin{itemize}
\item {Utilização:Geol.}
\end{itemize}
\begin{itemize}
\item {Proveniência:(Do gr. \textunderscore mikros\textunderscore  + \textunderscore morphe\textunderscore )}
\end{itemize}
Fórma rudimentar de crystal, sem regularidade geométrica.
\section{Micromorphítico}
\begin{itemize}
\item {Grp. gram.:adj.}
\end{itemize}
Relativo á micromorphite.
\section{Micromotoscópio}
\begin{itemize}
\item {Grp. gram.:m.}
\end{itemize}
\begin{itemize}
\item {Proveniência:(Do gr. \textunderscore mikros\textunderscore  + \textunderscore motos\textunderscore  + \textunderscore skopein\textunderscore )}
\end{itemize}
Apparelho, recentemente inventado, (1897), e que serve para a applicação do cinematógrapho á photographia microscópica e instantânea.
\section{Micromycetos}
\begin{itemize}
\item {Grp. gram.:m. pl.}
\end{itemize}
\begin{itemize}
\item {Proveniência:(Do gr. \textunderscore mikros\textunderscore  + \textunderscore mukes\textunderscore )}
\end{itemize}
Plantas cryptogâmicas, que fermentam as bebidas alcoólicas.
Leveduras.
\section{Micronemo}
\begin{itemize}
\item {Grp. gram.:adj.}
\end{itemize}
\begin{itemize}
\item {Proveniência:(Do gr. \textunderscore mikros\textunderscore  + \textunderscore nema\textunderscore )}
\end{itemize}
Que tem tentáculos muito pequenos.
\section{Microonte}
\begin{itemize}
\item {Grp. gram.:m.}
\end{itemize}
Quadrúpede fóssil, o mais pequeno dos ruminantes.
\section{Microorganismo}
\begin{itemize}
\item {Grp. gram.:m.}
\end{itemize}
\begin{itemize}
\item {Proveniência:(De \textunderscore micro...\textunderscore  + \textunderscore organismo\textunderscore )}
\end{itemize}
Organismo excessivamente pequeno.
Micróbio.
\section{Micropathologia}
\begin{itemize}
\item {Grp. gram.:f.}
\end{itemize}
\begin{itemize}
\item {Proveniência:(Do gr. \textunderscore mikros\textunderscore  + \textunderscore pathos\textunderscore  + \textunderscore logos\textunderscore )}
\end{itemize}
Pathologia baseada na microbiologia.
\section{Micropathológico}
\begin{itemize}
\item {Grp. gram.:adj.}
\end{itemize}
Relativo á micropathologia.
\section{Micropatologia}
\begin{itemize}
\item {Grp. gram.:f.}
\end{itemize}
\begin{itemize}
\item {Proveniência:(Do gr. \textunderscore mikros\textunderscore  + \textunderscore pathos\textunderscore  + \textunderscore logos\textunderscore )}
\end{itemize}
Patologia baseada na microbiologia.
\section{Micropatológico}
\begin{itemize}
\item {Grp. gram.:adj.}
\end{itemize}
Relativo á micropatologia.
\section{Micrópera}
\begin{itemize}
\item {Grp. gram.:f.}
\end{itemize}
Gênero de orchídeas.
\section{Micropétalo}
\begin{itemize}
\item {Grp. gram.:adj.}
\end{itemize}
\begin{itemize}
\item {Utilização:Bot.}
\end{itemize}
\begin{itemize}
\item {Proveniência:(De \textunderscore micro...\textunderscore  + \textunderscore pétala\textunderscore )}
\end{itemize}
Que tem pétalas pequenas.
\section{Microphonia}
\begin{itemize}
\item {Grp. gram.:f.}
\end{itemize}
\begin{itemize}
\item {Proveniência:(Do gr. \textunderscore mikros\textunderscore  + \textunderscore phone\textunderscore )}
\end{itemize}
Fraqueza da voz.
\section{Microphónico}
\begin{itemize}
\item {Grp. gram.:adj.}
\end{itemize}
\begin{itemize}
\item {Proveniência:(De \textunderscore microphonia\textunderscore )}
\end{itemize}
Que tem voz fraca. Que torna fraco um som.
\section{Microphónio}
\begin{itemize}
\item {Grp. gram.:m.}
\end{itemize}
\begin{itemize}
\item {Proveniência:(Do gr. \textunderscore mikros\textunderscore  + \textunderscore phone\textunderscore )}
\end{itemize}
Apparelho, inventado por Hughes e destinado a aumentar a intensidade do som.
\section{Microphonógrapho}
\begin{itemize}
\item {Grp. gram.:m.}
\end{itemize}
\begin{itemize}
\item {Proveniência:(Do gr. \textunderscore mikros\textunderscore  + \textunderscore phone\textunderscore  + \textunderscore graphein\textunderscore )}
\end{itemize}
Novo apparelho, inventado em 1897, para tornar percebidos os sons mais tênues, e empregado no tratamento da surdez. Cf. \textunderscore micróphono\textunderscore .
\section{Microphtalmo}
\begin{itemize}
\item {Grp. gram.:m.}
\end{itemize}
\begin{itemize}
\item {Proveniência:(Do gr. \textunderscore mikros\textunderscore  + \textunderscore ophtalmos\textunderscore )}
\end{itemize}
Estado do ôlho, cujo volume é inferior ao normal.
\section{Microphyllo}
\begin{itemize}
\item {Grp. gram.:adj.}
\end{itemize}
\begin{itemize}
\item {Utilização:Bot.}
\end{itemize}
\begin{itemize}
\item {Proveniência:(Do gr. \textunderscore mikros\textunderscore  + \textunderscore phullon\textunderscore )}
\end{itemize}
Que tem fôlhas pequenas.
\section{Micróphyta}
\begin{itemize}
\item {Grp. gram.:f.}
\end{itemize}
\begin{itemize}
\item {Proveniência:(Do gr. \textunderscore mikros\textunderscore  + \textunderscore phuton\textunderscore )}
\end{itemize}
Vegetal extremamente pequeno.
\section{Microphýtico}
\begin{itemize}
\item {Grp. gram.:adj.}
\end{itemize}
Relativo ás micróphytas.
\section{Micróphyto}
\begin{itemize}
\item {Grp. gram.:m.}
\end{itemize}
O mesmo que \textunderscore micróphyta\textunderscore .
\section{Micrópila}
\begin{itemize}
\item {Grp. gram.:f.}
\end{itemize}
O mesmo que \textunderscore micrópilo\textunderscore .
\section{Micrópilo}
\begin{itemize}
\item {Grp. gram.:m.}
\end{itemize}
\begin{itemize}
\item {Utilização:Bot.}
\end{itemize}
\begin{itemize}
\item {Proveniência:(Do gr. \textunderscore mikros\textunderscore  + \textunderscore pule\textunderscore )}
\end{itemize}
Pequena abertura, por onde o óvulo vegetal recebe a acção fecundante do pólen.
\section{Micrópogo}
\begin{itemize}
\item {Grp. gram.:m.}
\end{itemize}
Gênero de peixes acantopterígios.
\section{Micróporo}
\begin{itemize}
\item {Grp. gram.:adj.}
\end{itemize}
\begin{itemize}
\item {Proveniência:(De \textunderscore micro...\textunderscore  + \textunderscore poro\textunderscore )}
\end{itemize}
Que tem poros excessivamente pequenos.
\section{Micropsia}
\begin{itemize}
\item {Grp. gram.:f.}
\end{itemize}
\begin{itemize}
\item {Proveniência:(Do gr. \textunderscore mikros\textunderscore  + \textunderscore opsis\textunderscore )}
\end{itemize}
Alteração na vista, que faz que os objectos pareçam mais pequenos do que realmente são.
\section{Micropsiquia}
\begin{itemize}
\item {Grp. gram.:f.}
\end{itemize}
\begin{itemize}
\item {Proveniência:(Do gr. \textunderscore mikros\textunderscore  + \textunderscore psukhe\textunderscore )}
\end{itemize}
Fraqueza de espírito; pusilanimidade.
\section{Micropsychia}
\begin{itemize}
\item {fónica:qui}
\end{itemize}
\begin{itemize}
\item {Grp. gram.:f.}
\end{itemize}
\begin{itemize}
\item {Proveniência:(Do gr. \textunderscore mikros\textunderscore  + \textunderscore psukhe\textunderscore )}
\end{itemize}
Fraqueza de espírito; pusillanimidade.
\section{Micróptero}
\begin{itemize}
\item {Grp. gram.:adj.}
\end{itemize}
\begin{itemize}
\item {Proveniência:(Do gr. \textunderscore mikros\textunderscore  + \textunderscore pteron\textunderscore )}
\end{itemize}
Que tem pequenas asas ou barbatanas.
\section{Micropterígio}
\begin{itemize}
\item {Grp. gram.:adj.}
\end{itemize}
\begin{itemize}
\item {Proveniência:(Do gr. \textunderscore mikros\textunderscore  + \textunderscore pterux\textunderscore )}
\end{itemize}
Que tem pequenas barbatanas.
\section{Micropterýgio}
\begin{itemize}
\item {Grp. gram.:adj.}
\end{itemize}
\begin{itemize}
\item {Proveniência:(Do gr. \textunderscore mikros\textunderscore  + \textunderscore pterux\textunderscore )}
\end{itemize}
Que tem pequenas barbatanas.
\section{Micrópyla}
\begin{itemize}
\item {Grp. gram.:f.}
\end{itemize}
O mesmo que \textunderscore micrópylo\textunderscore .
\section{Micrópylo}
\begin{itemize}
\item {Grp. gram.:m.}
\end{itemize}
\begin{itemize}
\item {Utilização:Bot.}
\end{itemize}
\begin{itemize}
\item {Proveniência:(Do gr. \textunderscore mikros\textunderscore  + \textunderscore pule\textunderscore )}
\end{itemize}
Pequena abertura, por onde o óvulo vegetal recebe a acção fecundante do póllen.
\section{Microquímica}
\begin{itemize}
\item {Grp. gram.:f.}
\end{itemize}
\begin{itemize}
\item {Proveniência:(De \textunderscore micro...\textunderscore  + \textunderscore Química\textunderscore )}
\end{itemize}
Emprêgo do microscópio, para verificar os caracteres dos princípios químicos, cujos cristaes não podem sêr observados a ôlho nu.
\section{Microquímico}
\begin{itemize}
\item {Grp. gram.:adj.}
\end{itemize}
Relativo á microquímica.
\section{Microsaco}
\begin{itemize}
\item {fónica:sa}
\end{itemize}
\begin{itemize}
\item {Grp. gram.:m.}
\end{itemize}
Gênero de orchídeas.
\section{Microscopia}
\begin{itemize}
\item {Grp. gram.:f.}
\end{itemize}
\begin{itemize}
\item {Proveniência:(De \textunderscore microscópio\textunderscore )}
\end{itemize}
Arte de usar o microscópio.
Conjunto dos estudos microscópicos.
\section{Microscópico}
\begin{itemize}
\item {Grp. gram.:adj.}
\end{itemize}
\begin{itemize}
\item {Utilização:Fig.}
\end{itemize}
\begin{itemize}
\item {Proveniência:(De \textunderscore microscópio\textunderscore )}
\end{itemize}
Feito com o auxílio do microscópio: \textunderscore exame microscópico\textunderscore .
Que só póde sêr visto por meio do microscópio; pequeníssimo: \textunderscore animal microscópico\textunderscore .
Que tem vista penetrante.
\section{Microscópio}
\begin{itemize}
\item {Grp. gram.:m.}
\end{itemize}
\begin{itemize}
\item {Utilização:Fig.}
\end{itemize}
\begin{itemize}
\item {Proveniência:(Do gr. \textunderscore mikros\textunderscore  + \textunderscore skopein\textunderscore )}
\end{itemize}
Instrumento, para amplificar e representar próximos os objectos que por elle se observam.
Aquillo que amplifica as coisas abstractas, intellectuaes ou moraes.
Constellação meridional.
\section{Microscopista}
\begin{itemize}
\item {Grp. gram.:m. ,  f.  e  adj.}
\end{itemize}
\begin{itemize}
\item {Proveniência:(De \textunderscore microscópio\textunderscore )}
\end{itemize}
Pessôa, que se occupa de observações microscópicas.
\section{Microsemo}
\begin{itemize}
\item {fónica:sê}
\end{itemize}
\begin{itemize}
\item {Grp. gram.:m.}
\end{itemize}
O mesmo que \textunderscore mesosemo\textunderscore .
\section{Microsficto}
\begin{itemize}
\item {Grp. gram.:adj.}
\end{itemize}
\begin{itemize}
\item {Utilização:Med.}
\end{itemize}
Que tem o pulso fraco.
\section{Micro-sísmico}
\begin{itemize}
\item {Grp. gram.:adj.}
\end{itemize}
\begin{itemize}
\item {Utilização:Hist. Nat.}
\end{itemize}
Relativo ás vibrações, que agitam as camadas superficiaes do globo terrestre.
\section{Microsomatia}
\begin{itemize}
\item {fónica:so}
\end{itemize}
\begin{itemize}
\item {Grp. gram.:f.}
\end{itemize}
\begin{itemize}
\item {Proveniência:(Do gr. \textunderscore mikros\textunderscore  + \textunderscore soma\textunderscore )}
\end{itemize}
Monstruosidade, caracterizada pela excessiva pequenez de todo o corpo.
\section{Microsomático}
\begin{itemize}
\item {fónica:so}
\end{itemize}
\begin{itemize}
\item {Grp. gram.:adj.}
\end{itemize}
Em que há microsomatia.
\section{Micrósomo}
\begin{itemize}
\item {fónica:so}
\end{itemize}
\begin{itemize}
\item {Grp. gram.:adj.}
\end{itemize}
\begin{itemize}
\item {Proveniência:(Do gr. \textunderscore mikros\textunderscore  + \textunderscore soma\textunderscore )}
\end{itemize}
Que tem o corpo muito pequeno; microsomático.
\section{Microspermo}
\begin{itemize}
\item {Grp. gram.:m.}
\end{itemize}
Gênero de plantas, da fam. das compostas.
\section{Microsphycto}
\begin{itemize}
\item {Grp. gram.:adj.}
\end{itemize}
\begin{itemize}
\item {Utilização:Med.}
\end{itemize}
Que tem o pulso fraco.
\section{Micrósporo}
\begin{itemize}
\item {Grp. gram.:m.}
\end{itemize}
\begin{itemize}
\item {Grp. gram.:Adj.}
\end{itemize}
\begin{itemize}
\item {Proveniência:(Do gr. \textunderscore mikros\textunderscore  + \textunderscore spora\textunderscore )}
\end{itemize}
Pequeno esporo.
Esporo natural.
Que tem pequenos esporos.
\section{Microssaco}
\begin{itemize}
\item {Grp. gram.:m.}
\end{itemize}
Gênero de orquídeas.
\section{Microssemo}
\begin{itemize}
\item {Grp. gram.:m.}
\end{itemize}
O mesmo que \textunderscore mesosemo\textunderscore .
\section{Microssomatia}
\begin{itemize}
\item {Grp. gram.:f.}
\end{itemize}
\begin{itemize}
\item {Proveniência:(Do gr. \textunderscore mikros\textunderscore  + \textunderscore soma\textunderscore )}
\end{itemize}
Monstruosidade, caracterizada pela excessiva pequenez de todo o corpo.
\section{Microssomático}
\begin{itemize}
\item {Grp. gram.:adj.}
\end{itemize}
Em que há microsomatia.
\section{Micróssomo}
\begin{itemize}
\item {Grp. gram.:adj.}
\end{itemize}
\begin{itemize}
\item {Proveniência:(Do gr. \textunderscore mikros\textunderscore  + \textunderscore soma\textunderscore )}
\end{itemize}
Que tem o corpo muito pequeno; microsomático.
\section{Microstema}
\begin{itemize}
\item {Grp. gram.:f.}
\end{itemize}
Gênero de plantas asclepiádeas.
\section{Micróstomo}
\begin{itemize}
\item {Grp. gram.:adj.}
\end{itemize}
\begin{itemize}
\item {Utilização:Zool.}
\end{itemize}
\begin{itemize}
\item {Proveniência:(Do gr. \textunderscore mikros\textunderscore  + \textunderscore stoma\textunderscore )}
\end{itemize}
Que tem bôca ou abertura muito pequena.
\section{Micrótide}
\begin{itemize}
\item {Grp. gram.:f.}
\end{itemize}
Gênero de orchídeas.
\section{Microtina}
\begin{itemize}
\item {Grp. gram.:f.}
\end{itemize}
Espécie de rocha feldspáthica.
\section{Microtópide}
\begin{itemize}
\item {Grp. gram.:f.}
\end{itemize}
Gênero de plantas celastríneas.
\section{Microzoário}
\begin{itemize}
\item {Grp. gram.:m.}
\end{itemize}
\begin{itemize}
\item {Proveniência:(Do gr. \textunderscore mikros\textunderscore  + \textunderscore zoarion\textunderscore )}
\end{itemize}
Animálculo, que só se póde observar com o auxílio do microscópio.
Infusório.
\section{Microzoonito}
\begin{itemize}
\item {Grp. gram.:m.}
\end{itemize}
\begin{itemize}
\item {Proveniência:(Do gr. \textunderscore mikros\textunderscore  + \textunderscore zoon\textunderscore )}
\end{itemize}
O mesmo que \textunderscore microzoário\textunderscore .
\section{Micrura}
\begin{itemize}
\item {Grp. gram.:f.}
\end{itemize}
Gênero de helminthos.
(Cp. \textunderscore micruro\textunderscore )
\section{Micruro}
\begin{itemize}
\item {Grp. gram.:adj.}
\end{itemize}
\begin{itemize}
\item {Utilização:Zool.}
\end{itemize}
\begin{itemize}
\item {Proveniência:(Do gr. \textunderscore mikros\textunderscore  + \textunderscore oura\textunderscore )}
\end{itemize}
Que tem cauda curta.
\section{Micto}
\begin{itemize}
\item {Grp. gram.:m.}
\end{itemize}
\begin{itemize}
\item {Proveniência:(Lat. \textunderscore mictus\textunderscore )}
\end{itemize}
O mesmo que \textunderscore micção\textunderscore .
\section{Mictório}
\begin{itemize}
\item {Grp. gram.:adj.}
\end{itemize}
\begin{itemize}
\item {Grp. gram.:M.}
\end{itemize}
\begin{itemize}
\item {Proveniência:(Lat. \textunderscore mictorius\textunderscore )}
\end{itemize}
Que promove a micção.
Diurético.
Lugar, onde se urina; urinol.
\section{Mictorição}
\begin{itemize}
\item {Grp. gram.:f.}
\end{itemize}
\begin{itemize}
\item {Proveniência:(Do lat. \textunderscore micturire\textunderscore )}
\end{itemize}
Frequente necessidade de urinar.
\section{Micuim}
\begin{itemize}
\item {Grp. gram.:m.}
\end{itemize}
\begin{itemize}
\item {Utilização:Bras. do N}
\end{itemize}
\begin{itemize}
\item {Utilização:Prov.}
\end{itemize}
\begin{itemize}
\item {Utilização:minh.}
\end{itemize}
Insecto encarnado, cuja mordedura produz comichão incômmoda.
Mulher, ridiculamente feia.
\section{Micunco}
\begin{itemize}
\item {Grp. gram.:m.}
\end{itemize}
\begin{itemize}
\item {Utilização:Prov.}
\end{itemize}
\begin{itemize}
\item {Utilização:minh.}
\end{itemize}
Homem muito feio e velho.
\section{Mida}
\begin{itemize}
\item {Grp. gram.:f.}
\end{itemize}
Gênero de plantas santaláceas.
\section{Middletonita}
\begin{itemize}
\item {Grp. gram.:f.}
\end{itemize}
\begin{itemize}
\item {Proveniência:(De \textunderscore Middleton\textunderscore , n. p.)}
\end{itemize}
Mineral, que se encontra nas minas carboníferas de Newcastle, em Inglaterra.
\section{Mideia}
\begin{itemize}
\item {Grp. gram.:f.}
\end{itemize}
\begin{itemize}
\item {Utilização:Gír.}
\end{itemize}
Cabeça.
\section{Mienguelecas}
\begin{itemize}
\item {Grp. gram.:f.}
\end{itemize}
Espécie de esparregado de fôlhas de abóbora e mandioca, em uso nas tríbos de Angola.
\section{Miérsia}
\begin{itemize}
\item {Grp. gram.:f.}
\end{itemize}
Gênero de plantas liliáceas.
\section{Mifongo}
\begin{itemize}
\item {Grp. gram.:m.}
\end{itemize}
Arbusto africano, de fôlhas simples, alternas, coriáceas, e frutos monospermos, semelhantes a ameixas.
\section{Miga}
\begin{itemize}
\item {Grp. gram.:f.}
\end{itemize}
\begin{itemize}
\item {Grp. gram.:Pl.}
\end{itemize}
\begin{itemize}
\item {Proveniência:(Do lat. \textunderscore mica\textunderscore )}
\end{itemize}
Espécie de búzio.
Sopas de pão, açorda.
\section{Migado}
\begin{itemize}
\item {Grp. gram.:m.}
\end{itemize}
\begin{itemize}
\item {Utilização:Prov.}
\end{itemize}
\begin{itemize}
\item {Utilização:trasm.}
\end{itemize}
\begin{itemize}
\item {Proveniência:(De \textunderscore migar\textunderscore )}
\end{itemize}
Pão, que se migou e que fica no fundo da malga do caldo.
\section{Migalha}
\begin{itemize}
\item {Grp. gram.:f.}
\end{itemize}
\begin{itemize}
\item {Utilização:Ext.}
\end{itemize}
\begin{itemize}
\item {Grp. gram.:M.}
\end{itemize}
\begin{itemize}
\item {Utilização:Prov.}
\end{itemize}
\begin{itemize}
\item {Utilização:trasm.}
\end{itemize}
\begin{itemize}
\item {Grp. gram.:Pl.}
\end{itemize}
\begin{itemize}
\item {Proveniência:(Do b. lat. \textunderscore micalia\textunderscore , de \textunderscore mica\textunderscore )}
\end{itemize}
Pequeno fragmento do pão, de bolos ou de outro alimento farináceo.
Pequena porção.
Indivíduo sovina, forreta.
Sobejos.
Aquillo que é supérfluo ou se despreza por sêr de pouca monta.
\section{Migalhar}
\begin{itemize}
\item {Grp. gram.:v. t.}
\end{itemize}
O mesmo que \textunderscore esmigalhar\textunderscore .
\section{Migalheiro}
\begin{itemize}
\item {Grp. gram.:adj.}
\end{itemize}
\begin{itemize}
\item {Utilização:P. us.}
\end{itemize}
\begin{itemize}
\item {Utilização:Des.}
\end{itemize}
\begin{itemize}
\item {Grp. gram.:M.}
\end{itemize}
\begin{itemize}
\item {Proveniência:(De \textunderscore migalha\textunderscore )}
\end{itemize}
Que se occupa de pequenas coisas.
Que se prende com miudezas.
Que repara em tudo.
Sovina; mofino.
(Corr. de \textunderscore mealheiro\textunderscore )
\section{Migalhice}
\begin{itemize}
\item {Grp. gram.:f.}
\end{itemize}
\begin{itemize}
\item {Proveniência:(De \textunderscore migalha\textunderscore )}
\end{itemize}
Bagatela, insignificância.
\section{Migalho}
\begin{itemize}
\item {Grp. gram.:m.}
\end{itemize}
O mesmo que \textunderscore migalha\textunderscore ; bocadinho; partícula. Cf. Camillo, \textunderscore Myst. de Lisb.\textunderscore , I, 246; \textunderscore Cav. em Ruínas\textunderscore , 168.
\section{Migar}
\begin{itemize}
\item {Grp. gram.:v. t.}
\end{itemize}
\begin{itemize}
\item {Proveniência:(De \textunderscore miga\textunderscore )}
\end{itemize}
Partir em migalhas; cortar em bocadinhos.
\section{Migma}
\begin{itemize}
\item {Grp. gram.:m.}
\end{itemize}
\begin{itemize}
\item {Utilização:P. us.}
\end{itemize}
\begin{itemize}
\item {Proveniência:(Lat. \textunderscore migma\textunderscore )}
\end{itemize}
Mistura de viandas; fricassé. Cf. Latino, \textunderscore Or. da Corôa\textunderscore , LXXXIII.
\section{Migo}
\begin{itemize}
\item {Utilização:Obsol.}
\end{itemize}
\begin{itemize}
\item {Proveniência:(Do lat. \textunderscore mecum\textunderscore )}
\end{itemize}
(variação do pron. \textunderscore eu\textunderscore , antepondo-se-lhe a partícula \textunderscore com\textunderscore )
O mesmo que \textunderscore commigo\textunderscore . Cf. Alfredo Gomes, \textunderscore Gramm.\textunderscore , 168.
\section{Migração}
\begin{itemize}
\item {Grp. gram.:f.}
\end{itemize}
\begin{itemize}
\item {Proveniência:(Lat. \textunderscore migratio\textunderscore )}
\end{itemize}
Acto de passar de um país para outro, (falando-se de um povo ou de uma grande multidão de gente).
Viagens periódicas ou irregulares, feitas por certas espécies de animaes.
\section{Migrainina}
\begin{itemize}
\item {Grp. gram.:f.}
\end{itemize}
\begin{itemize}
\item {Proveniência:(Do fr. \textunderscore migraine\textunderscore )}
\end{itemize}
Substância, recentemente descoberta, e apresentada como producto chímico, mas que é talvez a simples mistura da cafeína e da antipyrina.
\section{Migrante}
\begin{itemize}
\item {Grp. gram.:adj.}
\end{itemize}
\begin{itemize}
\item {Proveniência:(Lat. \textunderscore migrans\textunderscore )}
\end{itemize}
Que muda de país.
\section{Migratório}
\begin{itemize}
\item {Grp. gram.:adj.}
\end{itemize}
\begin{itemize}
\item {Proveniência:(Do lat. \textunderscore migrare\textunderscore )}
\end{itemize}
Relativo a migração.
\section{Miguelangelesco}
\begin{itemize}
\item {fónica:lês}
\end{itemize}
\begin{itemize}
\item {Grp. gram.:adj.}
\end{itemize}
Relativo ao pintor Miguel Ângelo.
\section{Miguelismo}
\begin{itemize}
\item {Grp. gram.:m.}
\end{itemize}
\begin{itemize}
\item {Proveniência:(De \textunderscore Miguel\textunderscore , n. p.)}
\end{itemize}
Partido político de D. Miguel de Bragança.
Os miguelistas.
\section{Miguelista}
\begin{itemize}
\item {Grp. gram.:m.}
\end{itemize}
\begin{itemize}
\item {Proveniência:(De \textunderscore Miguel\textunderscore , n. p.)}
\end{itemize}
Partidário do rei D. Miguel, em Portugal.
\section{Mija}
\begin{itemize}
\item {Grp. gram.:f.}
\end{itemize}
\begin{itemize}
\item {Utilização:pleb.}
\end{itemize}
\begin{itemize}
\item {Utilização:Infant.}
\end{itemize}
Urina.
Acto de mijar.
\section{Mija-cão}
\begin{itemize}
\item {Grp. gram.:m.}
\end{itemize}
\begin{itemize}
\item {Utilização:Bras}
\end{itemize}
Cogumelo, em fórma de guarda-sol, e que nasce onde há bosta ou urina de animaes.
\section{Mijaceiro}
\begin{itemize}
\item {Grp. gram.:m.}
\end{itemize}
\begin{itemize}
\item {Utilização:Prov.}
\end{itemize}
\begin{itemize}
\item {Utilização:trasm.}
\end{itemize}
Chuva miúda, meruje.
\section{Mijada}
\begin{itemize}
\item {Grp. gram.:f.}
\end{itemize}
\begin{itemize}
\item {Utilização:Pleb.}
\end{itemize}
\begin{itemize}
\item {Proveniência:(De \textunderscore mijar\textunderscore )}
\end{itemize}
Mija.
\section{Mijadeira}
\begin{itemize}
\item {Grp. gram.:f.}
\end{itemize}
\begin{itemize}
\item {Utilização:Prov.}
\end{itemize}
\begin{itemize}
\item {Proveniência:(De \textunderscore mijar\textunderscore )}
\end{itemize}
Designação vulgar do \textunderscore androsemo\textunderscore .
\section{Mijadeiro}
\begin{itemize}
\item {Grp. gram.:m.}
\end{itemize}
\begin{itemize}
\item {Proveniência:(De \textunderscore mijar\textunderscore )}
\end{itemize}
Urinol.
Lugar, destinado para nelle se urinar.
\section{Mijadela}
\begin{itemize}
\item {Grp. gram.:f.}
\end{itemize}
\begin{itemize}
\item {Proveniência:(De \textunderscore mijar\textunderscore )}
\end{itemize}
Jacto de urina.
Mancha, produzida por urina.
\section{Mijadoiro}
\begin{itemize}
\item {Grp. gram.:m.}
\end{itemize}
O mesmo que \textunderscore mijadeiro\textunderscore .
\section{Mijados}
\begin{itemize}
\item {Grp. gram.:m. pl.}
\end{itemize}
\begin{itemize}
\item {Proveniência:(De \textunderscore mijar\textunderscore )}
\end{itemize}
Designação depreciativa dos partidários de Passos Manuel, (de quem se dizia que soffria da bexiga e molhava as calças, quando urinava).
\section{Mijadouro}
\begin{itemize}
\item {Grp. gram.:m.}
\end{itemize}
O mesmo que \textunderscore mijadeiro\textunderscore .
\section{Mija-mansinho}
\begin{itemize}
\item {Grp. gram.:m.  e  adj.}
\end{itemize}
\begin{itemize}
\item {Utilização:Pleb.}
\end{itemize}
Indivíduo sonso ou dissimulado.
\section{Mija-manso}
\begin{itemize}
\item {Grp. gram.:adj.}
\end{itemize}
O mesmo que \textunderscore mija-mansinho\textunderscore .
\section{Mijanágua}
\begin{itemize}
\item {Grp. gram.:m.}
\end{itemize}
\begin{itemize}
\item {Utilização:Ant.}
\end{itemize}
\begin{itemize}
\item {Utilização:Chul.}
\end{itemize}
\begin{itemize}
\item {Proveniência:(De \textunderscore mijar\textunderscore  + \textunderscore na\textunderscore  + \textunderscore água\textunderscore )}
\end{itemize}
Homem embarcadiço; marinheiro.
\section{Mijanceira}
\begin{itemize}
\item {Grp. gram.:f.}
\end{itemize}
\begin{itemize}
\item {Utilização:Des.}
\end{itemize}
Grande porção de mijo.
\section{Mijão}
\begin{itemize}
\item {Grp. gram.:m.  e  adj.}
\end{itemize}
\begin{itemize}
\item {Utilização:Pleb.}
\end{itemize}
\begin{itemize}
\item {Grp. gram.:Adj.}
\end{itemize}
\begin{itemize}
\item {Utilização:T. de Setúbal}
\end{itemize}
\begin{itemize}
\item {Utilização:T. de Turquel}
\end{itemize}
Aquelle que mija muitas vezes.
Diz-se do vento noroéste, porque traz aguaceiros. Cf. Rev. \textunderscore Tradição\textunderscore , v, 12.
Chocho, (falando-se do pinhão de casca).
\section{Mijar}
\begin{itemize}
\item {Grp. gram.:v. t.}
\end{itemize}
\begin{itemize}
\item {Utilização:Pleb.}
\end{itemize}
\begin{itemize}
\item {Utilização:Fig.}
\end{itemize}
\begin{itemize}
\item {Utilização:Prov.}
\end{itemize}
\begin{itemize}
\item {Utilização:Burl.}
\end{itemize}
\begin{itemize}
\item {Grp. gram.:V. i.}
\end{itemize}
\begin{itemize}
\item {Grp. gram.:V. p.}
\end{itemize}
\begin{itemize}
\item {Utilização:Fig.}
\end{itemize}
\begin{itemize}
\item {Proveniência:(Do lat. \textunderscore mingere\textunderscore )}
\end{itemize}
Expellir pela uretra ou pela vagina: \textunderscore mijar sangue\textunderscore .
Tratar com desprêzo.
\textunderscore Mijar ossos\textunderscore , parir.
Urinar.
Urinar involuntariamente.
Molhar-se com urina.
Têr medo.
\section{Mijarete}
\begin{itemize}
\item {fónica:jarê}
\end{itemize}
\begin{itemize}
\item {Grp. gram.:m.}
\end{itemize}
\begin{itemize}
\item {Utilização:Pleb.}
\end{itemize}
\begin{itemize}
\item {Proveniência:(De \textunderscore mijar\textunderscore )}
\end{itemize}
Porção de pólvora amassada, que fórma uma espécie de jacto ou de esguicho quando arde.
Urinol.
\section{Mija-vinagre}
\begin{itemize}
\item {Grp. gram.:m.}
\end{itemize}
Substância esponjosa, que o mar expelle na vazante.
\section{Mijengra}
\begin{itemize}
\item {Grp. gram.:f.}
\end{itemize}
O mesmo ou melhor que \textunderscore megengra\textunderscore .
\section{Mijina}
\begin{itemize}
\item {Grp. gram.:f.}
\end{itemize}
\begin{itemize}
\item {Utilização:Prov.}
\end{itemize}
\begin{itemize}
\item {Utilização:minh.}
\end{itemize}
O mesmo que \textunderscore mijo\textunderscore .
\section{Mijo}
\begin{itemize}
\item {Grp. gram.:m.}
\end{itemize}
\begin{itemize}
\item {Utilização:Pleb.}
\end{itemize}
\begin{itemize}
\item {Proveniência:(De \textunderscore mijar\textunderscore )}
\end{itemize}
O mesmo que \textunderscore urina\textunderscore .
\section{Mijoca}
\begin{itemize}
\item {Grp. gram.:f.}
\end{itemize}
\begin{itemize}
\item {Utilização:Pop.}
\end{itemize}
\begin{itemize}
\item {Proveniência:(De \textunderscore mijo\textunderscore )}
\end{itemize}
Bebida reles: \textunderscore isto não é vinho que se beba, é uma mijoca indecente\textunderscore .
\section{Mijolo}
\begin{itemize}
\item {Grp. gram.:m.}
\end{itemize}
\begin{itemize}
\item {Utilização:Bras. do N}
\end{itemize}
O mesmo que \textunderscore munjolo\textunderscore .
\section{Mijona}
\begin{itemize}
\item {Grp. gram.:adj.}
\end{itemize}
\begin{itemize}
\item {Proveniência:(De \textunderscore mijão\textunderscore )}
\end{itemize}
Diz-se de uma espécie de uva, de bagos molles e sabor desagradável.
\section{Mijote}
\begin{itemize}
\item {Grp. gram.:m.}
\end{itemize}
\begin{itemize}
\item {Utilização:Pop.}
\end{itemize}
\begin{itemize}
\item {Proveniência:(De \textunderscore mijar\textunderscore )}
\end{itemize}
O mesmo que \textunderscore cagarola\textunderscore .
\section{Mijuí}
\begin{itemize}
\item {Grp. gram.:m.}
\end{itemize}
\begin{itemize}
\item {Utilização:Bras}
\end{itemize}
Pequena abelha preta.
\section{Mil}
\begin{itemize}
\item {Grp. gram.:adj.}
\end{itemize}
\begin{itemize}
\item {Utilização:Ext.}
\end{itemize}
\begin{itemize}
\item {Proveniência:(Lat. \textunderscore mille\textunderscore )}
\end{itemize}
Déz vezes cem.
Muitos, em quantidade indeterminada: \textunderscore repeti-lho mil vezes\textunderscore .
\section{Milagre}
\begin{itemize}
\item {Grp. gram.:m.}
\end{itemize}
\begin{itemize}
\item {Utilização:Pop.}
\end{itemize}
\begin{itemize}
\item {Proveniência:(Do lat. \textunderscore miraculum\textunderscore )}
\end{itemize}
Aquillo que causa admiração.
Aquillo que é sobrenatural.
Prodígio.
Sucesso extraordinário.
Figura de madeira ou de cera, que se leva aos santos, em cumprimento de um voto.
\section{Milagreira}
\begin{itemize}
\item {Grp. gram.:f.}
\end{itemize}
\begin{itemize}
\item {Utilização:Deprec.}
\end{itemize}
Invenção estupenda; coisa nunca vista.
(Cp. \textunderscore milagreiro\textunderscore )
\section{Milagreiro}
\begin{itemize}
\item {Grp. gram.:m.  e  adj.}
\end{itemize}
\begin{itemize}
\item {Proveniência:(De \textunderscore milagre\textunderscore )}
\end{itemize}
Aquelle que crê facilmente em milagres.
Aquelle que pratíca milagres, ou se inculca como tal.
\section{Milagrento}
\begin{itemize}
\item {Grp. gram.:adj.}
\end{itemize}
\begin{itemize}
\item {Utilização:Chul.}
\end{itemize}
O mesmo que \textunderscore milagreiro\textunderscore .
\section{Milagrório}
\begin{itemize}
\item {Grp. gram.:m.}
\end{itemize}
\begin{itemize}
\item {Utilização:Fam.}
\end{itemize}
Milagre fingido, ridículo.
\section{Milagrosa}
\begin{itemize}
\item {Grp. gram.:f.}
\end{itemize}
\begin{itemize}
\item {Utilização:Bras}
\end{itemize}
Espécie de mandioca.
\section{Milagrosamente}
\begin{itemize}
\item {Grp. gram.:adv.}
\end{itemize}
De modo milagroso; por milagre.
Prodigiosamente.
\section{Milagroso}
\begin{itemize}
\item {Grp. gram.:adj.}
\end{itemize}
\begin{itemize}
\item {Proveniência:(De \textunderscore milagre\textunderscore )}
\end{itemize}
Que faz milagres.
A quem se attribuem milagres: \textunderscore um santo milagroso\textunderscore .
Estupendo, maravilhoso.
\section{Milanária}
\begin{itemize}
\item {Grp. gram.:f.}
\end{itemize}
Aquela que tem mil anos:«\textunderscore ...chegaria a abraçar a milanaria bela\textunderscore ». Castilho, \textunderscore Geórgicas\textunderscore , 298.
(Talvez seja êrro tipográfico, em vez de \textunderscore milenária\textunderscore ; se o não é, e como Castilho perfilhava a geminação etimológica das consoantes, deveria escrever \textunderscore milanária\textunderscore , do lat. \textunderscore mille\textunderscore  + \textunderscore annus\textunderscore )
\section{Milando}
\begin{itemize}
\item {Grp. gram.:m.}
\end{itemize}
\begin{itemize}
\item {Utilização:T. da Áfr. or. port}
\end{itemize}
Litígio, demanda.
\section{Milanês}
\begin{itemize}
\item {Grp. gram.:adj.}
\end{itemize}
\begin{itemize}
\item {Grp. gram.:M.}
\end{itemize}
\begin{itemize}
\item {Proveniência:(Do it. \textunderscore Milano\textunderscore , n. p.)}
\end{itemize}
Relativo a Milão.
Habitante de Milão.
\section{Milanesa}
\begin{itemize}
\item {Grp. gram.:f.}
\end{itemize}
\begin{itemize}
\item {Proveniência:(De \textunderscore milanês\textunderscore )}
\end{itemize}
Tecido antigo:«\textunderscore Deixo um guarda-pé de milanesa de seda...\textunderscore »(De um testamento do séc. XVII)
\section{Milano}
\begin{itemize}
\item {Grp. gram.:m.}
\end{itemize}
\begin{itemize}
\item {Utilização:Mad}
\end{itemize}
O mesmo que milhano.
\section{Milão}
\begin{itemize}
\item {Grp. gram.:m.}
\end{itemize}
Tecido de linho, fabricado na cidade do mesmo nome.
\section{Milara}
\begin{itemize}
\item {Grp. gram.:f.}
\end{itemize}
\begin{itemize}
\item {Utilização:Ant.}
\end{itemize}
Espécie de mandil, na Índia.
\section{Milavo}
\begin{itemize}
\item {Grp. gram.:m.}
\end{itemize}
\begin{itemize}
\item {Utilização:Neol.}
\end{itemize}
\begin{itemize}
\item {Proveniência:(De \textunderscore mil\textunderscore  + \textunderscore avo\textunderscore )}
\end{itemize}
Centésima parte.--O milavo do moderno escudo português corresponde ao real, que tem sido a nossa unidade monetária.
\section{Milde}
\begin{itemize}
\item {Grp. gram.:m.}
\end{itemize}
\begin{itemize}
\item {Utilização:Prov.}
\end{itemize}
\begin{itemize}
\item {Utilização:trasm.}
\end{itemize}
A mão do mangual.
\section{Míldio}
\begin{itemize}
\item {Grp. gram.:m.}
\end{itemize}
\begin{itemize}
\item {Proveniência:(Do ingl. \textunderscore mildew\textunderscore )}
\end{itemize}
Doença das videiras, que lhes ataca os órgãos verdes, mormente as fôlhas.--(Também se tem escrito \textunderscore mildiú\textunderscore ).
\section{Mile}
\begin{itemize}
\item {Grp. gram.:m.}
\end{itemize}
\begin{itemize}
\item {Utilização:Prov.}
\end{itemize}
\begin{itemize}
\item {Utilização:trasm.}
\end{itemize}
O mesmo que \textunderscore meúle\textunderscore .
\section{Miledi}
\begin{itemize}
\item {Grp. gram.:f.}
\end{itemize}
Dama inglesa.
(Aportuguesamento, feito por Filinto, do ingl. \textunderscore milady\textunderscore )
\section{Milefólio}
\begin{itemize}
\item {Grp. gram.:m.}
\end{itemize}
\begin{itemize}
\item {Proveniência:(Lat. \textunderscore millefollium\textunderscore )}
\end{itemize}
Planta da fam. das compostas, (\textunderscore achillea millefolium\textunderscore ), o mesmo que \textunderscore mil-em-rama\textunderscore , ou \textunderscore mil-fôlhas\textunderscore , ou \textunderscore milfolhada\textunderscore .
\section{Mileglana}
\begin{itemize}
\item {Grp. gram.:f.}
\end{itemize}
Planta, da serra de Sintra.
(Provavelmente, da mesma or. que \textunderscore milligran\textunderscore )
\section{Mil-em-rama}
\begin{itemize}
\item {Grp. gram.:f.}
\end{itemize}
Planta medicinal, da fam. das compostas, (\textunderscore achillea millefolia\textunderscore ).
\section{Milenar}
\begin{itemize}
\item {Grp. gram.:adj.}
\end{itemize}
\begin{itemize}
\item {Utilização:Neol.}
\end{itemize}
O mesmo que \textunderscore miliário\textunderscore ^1. Cf. C. Netto, \textunderscore Saldunes\textunderscore .
\section{Milenário}
\begin{itemize}
\item {Grp. gram.:adj.}
\end{itemize}
\begin{itemize}
\item {Grp. gram.:M.}
\end{itemize}
\begin{itemize}
\item {Proveniência:(Lat. \textunderscore millenarius\textunderscore )}
\end{itemize}
Relativo ao milhar.
Que tem mil anos.
Espaço de mil anos.
Sectário cristão, que sustentava que, depois do dia de juízo, os escolhidos ficariam mil anos sôbre a terra, gozando todas as delícias.
Aquele que, entre os Cristãos da Idade-Média, acreditava que o mundo acabaria no ano 1000.
\section{Milenarismo}
\begin{itemize}
\item {Grp. gram.:m.}
\end{itemize}
\begin{itemize}
\item {Proveniência:(De \textunderscore milenário\textunderscore )}
\end{itemize}
Sistema dos que sustentavam que o mundo acabaria no anno mil da era cristan.
\section{Milênia}
\begin{itemize}
\item {Grp. gram.:f.}
\end{itemize}
Espécie de batalhão godo, que compreendia duas quingentarias. Cf. C. Aires, \textunderscore Hist. do Exérc. Port.\textunderscore 
\section{Milênio}
\begin{itemize}
\item {Grp. gram.:m.}
\end{itemize}
\begin{itemize}
\item {Proveniência:(Lat. \textunderscore milleni\textunderscore )}
\end{itemize}
Espaço de mil anos.
\section{Milépede}
\begin{itemize}
\item {Grp. gram.:m.}
\end{itemize}
\begin{itemize}
\item {Proveniência:(Fr. \textunderscore millepieds\textunderscore )}
\end{itemize}
Fórma bárbara com que alguns designam o bicho-de-conta.
\section{Milépora}
\begin{itemize}
\item {Grp. gram.:f.}
\end{itemize}
\begin{itemize}
\item {Proveniência:(Do lat. \textunderscore mille\textunderscore  + \textunderscore porus\textunderscore )}
\end{itemize}
Gênero de polipeiros, cuja superfície é cavada por grande porção de poros.
\section{Mileporáceo}
\begin{itemize}
\item {Grp. gram.:adj.}
\end{itemize}
Semelhante á milépora.
\section{Milésima}
\begin{itemize}
\item {Grp. gram.:f.}
\end{itemize}
\begin{itemize}
\item {Proveniência:(De \textunderscore milésimo\textunderscore )}
\end{itemize}
Cada uma das mil partes, em que se divide um todo.
\section{Milésio}
\begin{itemize}
\item {Grp. gram.:adj.}
\end{itemize}
\begin{itemize}
\item {Grp. gram.:M.}
\end{itemize}
\begin{itemize}
\item {Proveniência:(Lat. \textunderscore milesius\textunderscore )}
\end{itemize}
Relativo a Mileto.
Habitante de Mileto. Cf. Latino, \textunderscore Or. da Corôa\textunderscore , CLXX.
\section{Mil-flôres}
\begin{itemize}
\item {Grp. gram.:m.}
\end{itemize}
\begin{itemize}
\item {Grp. gram.:Adj.}
\end{itemize}
Essência de muitas espécies de flôres.
Diz-se do cavallo, que tem o pêlo mesclado de branco e vermelho.
\section{Milfolhada}
\begin{itemize}
\item {Grp. gram.:f.}
\end{itemize}
O mesmo que \textunderscore millefólio\textunderscore . Cf. \textunderscore Regul. dos Preços dos Medicamentos\textunderscore .
\section{Mil-fôlhas}
\begin{itemize}
\item {Grp. gram.:f.}
\end{itemize}
O mesmo que \textunderscore millefólio\textunderscore .
Variedade de pêra beirôa.
\section{Milfurada}
\begin{itemize}
\item {Grp. gram.:f.}
\end{itemize}
Planta hypericácea, (\textunderscore hypericum perforatum\textunderscore , Lin.).
O mesmo que \textunderscore hypericão\textunderscore .
\section{Milfurado}
\begin{itemize}
\item {Grp. gram.:adj.}
\end{itemize}
\begin{itemize}
\item {Proveniência:(De \textunderscore mil\textunderscore  + \textunderscore furado\textunderscore )}
\end{itemize}
Que tem muitos furos; esburacado.
\section{Mil-gamenho}
\begin{itemize}
\item {Grp. gram.:adj.}
\end{itemize}
\begin{itemize}
\item {Utilização:Des.}
\end{itemize}
Mui gamenho. Cf. Filinto, VII, 42.
\section{Milgrada}
\begin{itemize}
\item {Grp. gram.:f.}
\end{itemize}
\begin{itemize}
\item {Utilização:Prov.}
\end{itemize}
\begin{itemize}
\item {Utilização:trasm.}
\end{itemize}
O mesmo que \textunderscore roman\textunderscore ^1.
(Contr. de \textunderscore mil\textunderscore  + \textunderscore granada\textunderscore , de \textunderscore granar\textunderscore . Cp. \textunderscore milligran\textunderscore )
\section{Milgreira}
\begin{itemize}
\item {Grp. gram.:f.}
\end{itemize}
\begin{itemize}
\item {Utilização:Prov.}
\end{itemize}
\begin{itemize}
\item {Utilização:trasm.}
\end{itemize}
O mesmo que \textunderscore milgrada\textunderscore .
\section{Milha}
\begin{itemize}
\item {Grp. gram.:f.}
\end{itemize}
\begin{itemize}
\item {Proveniência:(Do lat. \textunderscore millia\textunderscore )}
\end{itemize}
Medida itinerária.
Extensão de mil passos geométricos.
Extensão marítima de 1852 metros.
\section{Milha}
\begin{itemize}
\item {Grp. gram.:adj. f.}
\end{itemize}
\begin{itemize}
\item {Proveniência:(De \textunderscore milho\textunderscore )}
\end{itemize}
Diz-se da palha e da farinha de milho.
\section{Milhã}
\begin{itemize}
\item {Grp. gram.:f.}
\end{itemize}
\begin{itemize}
\item {Proveniência:(De \textunderscore milho\textunderscore )}
\end{itemize}
Planta gramínea e rasteira, que cresce principalmente entre os milheiraes.
\section{Milhaça}
\begin{itemize}
\item {Grp. gram.:f.}
\end{itemize}
\begin{itemize}
\item {Utilização:Des.}
\end{itemize}
\begin{itemize}
\item {Proveniência:(Do b. lat. \textunderscore miliacea\textunderscore )}
\end{itemize}
Farinha de milho.
\section{Milhado}
\begin{itemize}
\item {Grp. gram.:adj.}
\end{itemize}
\begin{itemize}
\item {Utilização:Bras. do S}
\end{itemize}
\begin{itemize}
\item {Utilização:Fig.}
\end{itemize}
Diz-se do animal, que adoece, por têr comido muito milho.
Ébrio.
\section{Milhafre}
\begin{itemize}
\item {Grp. gram.:m.}
\end{itemize}
\begin{itemize}
\item {Utilização:Pop.}
\end{itemize}
\begin{itemize}
\item {Utilização:Gír.}
\end{itemize}
\begin{itemize}
\item {Proveniência:(Do lat. \textunderscore milvus\textunderscore , segundo a opinião corrente; vejo-lhe porém mais parentesco com o lat. \textunderscore millio\textunderscore , milhano, que poderia juntar-se a \textunderscore afer\textunderscore ,--\textunderscore millio-afer\textunderscore ,--e produzir \textunderscore milhafre\textunderscore )}
\end{itemize}
Ave de rapina, (\textunderscore milvus regalis\textunderscore ).
Gavião.
Larápio; ratoneiro.
Mil reis.
\section{Milhagem}
\begin{itemize}
\item {Grp. gram.:f.}
\end{itemize}
\begin{itemize}
\item {Utilização:Bras}
\end{itemize}
Contagem das milhas.
\section{Milhal}
\begin{itemize}
\item {Grp. gram.:m.}
\end{itemize}
Terreno, em que cresce milho; milheiral.
\section{Milhan}
\begin{itemize}
\item {Grp. gram.:f.}
\end{itemize}
\begin{itemize}
\item {Proveniência:(De \textunderscore milho\textunderscore )}
\end{itemize}
Planta gramínea e rasteira, que cresce principalmente entre os milheiraes.
\section{Milhan-de-pendão}
\begin{itemize}
\item {Grp. gram.:f.}
\end{itemize}
Planta, (\textunderscore digitaria sanguinalis\textunderscore , Lin.).
\section{Milhan-do-norte}
\begin{itemize}
\item {Grp. gram.:f.}
\end{itemize}
Forragem, (\textunderscore panicum altissimum\textunderscore ).
\section{Milhaneiro}
\begin{itemize}
\item {Grp. gram.:adj.}
\end{itemize}
Que caça milhanos.
\section{Milhano}
\begin{itemize}
\item {Grp. gram.:m.}
\end{itemize}
O mesmo que \textunderscore milhafre\textunderscore .
\section{Milhano-das-tôrres}
\begin{itemize}
\item {Grp. gram.:m.}
\end{itemize}
Ave, o mesmo que \textunderscore peneireiro\textunderscore .
\section{Milhano-de-asa-redonda}
\begin{itemize}
\item {Grp. gram.:m.}
\end{itemize}
O mesmo que \textunderscore mioto\textunderscore .
\section{Milhão}
\begin{itemize}
\item {Grp. gram.:m.}
\end{itemize}
\begin{itemize}
\item {Utilização:Ext.}
\end{itemize}
\begin{itemize}
\item {Proveniência:(De \textunderscore mil\textunderscore )}
\end{itemize}
Mil vezes mil.
Somma de 400 contos de reis, equivalente a um milhão de cruzados.
Número muito considerável, mas indeterminado.
\section{Milhão}
\begin{itemize}
\item {Grp. gram.:m.}
\end{itemize}
\begin{itemize}
\item {Utilização:Ant.}
\end{itemize}
Milho de cana muito alta e grão muito graúdo.
Milho miúdo.
\section{Milhar}
\begin{itemize}
\item {Grp. gram.:m.}
\end{itemize}
\begin{itemize}
\item {Utilização:Ext.}
\end{itemize}
\begin{itemize}
\item {Proveniência:(Do lat. \textunderscore milliarius\textunderscore )}
\end{itemize}
Quantidade, que abrange déz centenas.
O número mil.
O quarto lugar, da direita para a esquerda, numa série de algarismos.
Grande número, grande quantidade.
\section{Milhar}
\begin{itemize}
\item {Grp. gram.:m.}
\end{itemize}
\begin{itemize}
\item {Proveniência:(De \textunderscore milho\textunderscore )}
\end{itemize}
O mesmo que \textunderscore milhal\textunderscore .
\section{Mílhara}
\begin{itemize}
\item {Grp. gram.:f.}
\end{itemize}
\begin{itemize}
\item {Utilização:Prov.}
\end{itemize}
\begin{itemize}
\item {Utilização:beir.}
\end{itemize}
\begin{itemize}
\item {Proveniência:(De \textunderscore milho\textunderscore )}
\end{itemize}
Papas de farinha de milho miúdo, fervidas com leite.
O mesmo que \textunderscore mílharas\textunderscore .
\section{Milharada}
\begin{itemize}
\item {Grp. gram.:f.}
\end{itemize}
\begin{itemize}
\item {Utilização:T. de Avis}
\end{itemize}
\begin{itemize}
\item {Proveniência:(Do b. lat. \textunderscore miliarata\textunderscore )}
\end{itemize}
O mesmo que \textunderscore milheiral\textunderscore .
Porção de espigas de milho, que se levam para a eira.
\section{Milharado}
\begin{itemize}
\item {Grp. gram.:adj.}
\end{itemize}
Que tem mílharas.
\section{Milharal}
\begin{itemize}
\item {Grp. gram.:m.}
\end{itemize}
Mais usado, mas menos correcto que \textunderscore milheiral\textunderscore .(V.milheiral)
\section{Milharão}
\begin{itemize}
\item {Grp. gram.:m.}
\end{itemize}
O mesmo que \textunderscore abelharuco\textunderscore .
\section{Milharaque}
\begin{itemize}
\item {Grp. gram.:m.}
\end{itemize}
\begin{itemize}
\item {Utilização:Prov.}
\end{itemize}
\begin{itemize}
\item {Proveniência:(De \textunderscore milho\textunderscore )}
\end{itemize}
Fritura, feita de farinha de milho com abóbora, ovos, pimenta e sal, usada especialmente nas consoadas do Natal. Cf. \textunderscore Bibl. da G. do Campo\textunderscore , 437.
\section{Mílharas}
\begin{itemize}
\item {Grp. gram.:f. pl.}
\end{itemize}
\begin{itemize}
\item {Proveniência:(De \textunderscore milho\textunderscore . Cp. \textunderscore mílhara\textunderscore )}
\end{itemize}
Substância granulosa dos ovos dos peixes.
Substância granulosa do interior dos figos.
\section{Milharós}
\begin{itemize}
\item {Grp. gram.:m.}
\end{itemize}
Pássaro syndáctylo, (\textunderscore merops apiaster\textunderscore ), o mesmo que \textunderscore melharuco\textunderscore .
\section{Milharuco}
\begin{itemize}
\item {Grp. gram.:m.}
\end{itemize}
\begin{itemize}
\item {Utilização:T. da Bairrada}
\end{itemize}
O mesmo que \textunderscore melharuco\textunderscore .
\section{Milhear}
\begin{itemize}
\item {Grp. gram.:adj.}
\end{itemize}
O mesmo que \textunderscore miliar\textunderscore ^1.
\section{Milheira}
\begin{itemize}
\item {Grp. gram.:f.}
\end{itemize}
\begin{itemize}
\item {Proveniência:(De \textunderscore milho\textunderscore )}
\end{itemize}
Passarinho conirostro, de asas verdes e cabeça amarela.
Milhan.
\section{Milheirada}
\begin{itemize}
\item {Grp. gram.:f.}
\end{itemize}
\begin{itemize}
\item {Utilização:Pop.}
\end{itemize}
O mesmo que \textunderscore milheiral\textunderscore .
\section{Milheira-galante}
\begin{itemize}
\item {Grp. gram.:f.}
\end{itemize}
Nome que, nalguns pontos da Beira, se dá ao pintasilgo.
\section{Milheiral}
\begin{itemize}
\item {Grp. gram.:m.}
\end{itemize}
\begin{itemize}
\item {Proveniência:(De \textunderscore milheiro\textunderscore ^2)}
\end{itemize}
Terreno, semeado de milho ou em que crescem milheiros.
\section{Milheirão}
\begin{itemize}
\item {Grp. gram.:m.}
\end{itemize}
(V.milharão)
\section{Milheiriça}
\begin{itemize}
\item {Grp. gram.:f.}
\end{itemize}
O mesmo que \textunderscore milheira\textunderscore , ave.
\section{Milheirinha}
\begin{itemize}
\item {Grp. gram.:f.}
\end{itemize}
\begin{itemize}
\item {Utilização:Prov.}
\end{itemize}
\begin{itemize}
\item {Utilização:dur.}
\end{itemize}
O mesmo que \textunderscore pintarroxo\textunderscore .
\section{Milheirita}
\begin{itemize}
\item {Grp. gram.:f.}
\end{itemize}
\begin{itemize}
\item {Utilização:T. da Bairrada}
\end{itemize}
O mesmo que \textunderscore milheirinha\textunderscore .
\section{Milheiro}
\begin{itemize}
\item {Grp. gram.:m.}
\end{itemize}
\begin{itemize}
\item {Proveniência:(Do lat. \textunderscore milliarius\textunderscore )}
\end{itemize}
O mesmo que \textunderscore milhar\textunderscore ^1.
\section{Milheiro}
\begin{itemize}
\item {Grp. gram.:m.}
\end{itemize}
\begin{itemize}
\item {Utilização:Prov.}
\end{itemize}
\begin{itemize}
\item {Utilização:minh.}
\end{itemize}
Planta, que dá milho.
Cana dessa planta.
Bago de milho.
Espécie de uva preta.
Milheiriça.
Pintarroxo.
\section{Milheiró}
\begin{itemize}
\item {Grp. gram.:m.}
\end{itemize}
\begin{itemize}
\item {Utilização:Mad}
\end{itemize}
Variedade de uva preta, o mesmo que \textunderscore milheiro\textunderscore .
O mesmo que \textunderscore pintasilgo\textunderscore .
\section{Milhém}
\begin{itemize}
\item {Grp. gram.:f.}
\end{itemize}
O mesmo que \textunderscore milhan\textunderscore .
\section{Milhenta}
\begin{itemize}
\item {Grp. gram.:adj.}
\end{itemize}
Designação burlesca ou infantil de um número muito elevado ou superior a mil.«\textunderscore ...as mil milhentas palavras dessa mesma Biblia.\textunderscore »Filinto, III, 101.
(Cp. \textunderscore milhar\textunderscore ^1)
\section{Mílheras}
\begin{itemize}
\item {Grp. gram.:f. pl.}
\end{itemize}
O mesmo que \textunderscore mílharas\textunderscore .
\section{Milhereu}
\begin{itemize}
\item {Grp. gram.:m.}
\end{itemize}
O mesmo que \textunderscore pernilongo\textunderscore , ave.
\section{Milherite}
\begin{itemize}
\item {Grp. gram.:f.}
\end{itemize}
Sulfato de níquel mineral.
\section{Milhete}
\begin{itemize}
\item {fónica:lhê}
\end{itemize}
\begin{itemize}
\item {Grp. gram.:m.}
\end{itemize}
Variedade de milho, de grão muito miúdo.
\section{Milheu}
\begin{itemize}
\item {Grp. gram.:m.}
\end{itemize}
Espécie de pano antigo, que vinha de fóra do reino.
\section{Milho}
\begin{itemize}
\item {Grp. gram.:m.}
\end{itemize}
\begin{itemize}
\item {Utilização:Chul.}
\end{itemize}
\begin{itemize}
\item {Grp. gram.:Pl.}
\end{itemize}
\begin{itemize}
\item {Proveniência:(Do lat. \textunderscore milium\textunderscore )}
\end{itemize}
Gênero de plantas gramíneas, (\textunderscore zea-mays\textunderscore ).
Grão, produzido por essa planta.
Qualquer quantidade de grãos dessa planta.
Dinheiro.
Farinha de milhão, grossa e mal moída, para papas; carolos.
\section{Milhoal}
\begin{itemize}
\item {Grp. gram.:m.}
\end{itemize}
\begin{itemize}
\item {Utilização:Prov.}
\end{itemize}
O mesmo que \textunderscore milheiral\textunderscore .
\section{Milho-cozido}
\begin{itemize}
\item {Grp. gram.:m.}
\end{itemize}
Árvore brasileira, própria para construcções.
\section{Milho-das-vassoiras}
\begin{itemize}
\item {Grp. gram.:m.}
\end{itemize}
Espécie de sorgo, de cana delgada e alta, com bandeira e sem espiga, que serve especialmente para vassoiras.
\section{Mil-homens}
\begin{itemize}
\item {Grp. gram.:m.}
\end{itemize}
\begin{itemize}
\item {Utilização:Pop.}
\end{itemize}
Homem de pequena estatura, que tem bazófia de valentão.
Planta brasileira, também conhecida por \textunderscore jarrinha\textunderscore .
Planta trepadeira, que cresce á beira dos rios e em terrenos pantanosos.
\section{Milhorde}
\begin{itemize}
\item {Grp. gram.:adj.}
\end{itemize}
\begin{itemize}
\item {Utilização:Prov.}
\end{itemize}
\begin{itemize}
\item {Utilização:trasm.}
\end{itemize}
\begin{itemize}
\item {Proveniência:(Do ingl. \textunderscore milord\textunderscore )}
\end{itemize}
Rico.
Ocioso.
\section{Miliáceas}
\begin{itemize}
\item {Grp. gram.:f. pl.}
\end{itemize}
\begin{itemize}
\item {Proveniência:(De \textunderscore miliáceo\textunderscore )}
\end{itemize}
Tríbo de plantas, que têm por typo o milho.
\section{Miliáceo}
\begin{itemize}
\item {Grp. gram.:adj.}
\end{itemize}
\begin{itemize}
\item {Proveniência:(Lat. \textunderscore miliaceus\textunderscore )}
\end{itemize}
Relativo ou semelhante ao milho.
\section{Miliar}
\begin{itemize}
\item {Grp. gram.:adj.}
\end{itemize}
\begin{itemize}
\item {Proveniência:(Lat. \textunderscore miliarius\textunderscore )}
\end{itemize}
Que tém fórma de grão de milho.
Que tem pequenas dimensões, (falando-se de um animal).
\section{Miliarésio}
\begin{itemize}
\item {Grp. gram.:m.}
\end{itemize}
Moéda romana, o mesmo que \textunderscore denário\textunderscore . Cf. Castilho, \textunderscore Fastos\textunderscore , I, XLIV.
\section{Miliário}
\begin{itemize}
\item {Grp. gram.:adj.}
\end{itemize}
O mesmo que \textunderscore miliar\textunderscore ^1.
\section{Milícia}
\begin{itemize}
\item {Grp. gram.:f.}
\end{itemize}
\begin{itemize}
\item {Utilização:Ext.}
\end{itemize}
\begin{itemize}
\item {Utilização:Pop.}
\end{itemize}
\begin{itemize}
\item {Grp. gram.:Pl.}
\end{itemize}
\begin{itemize}
\item {Proveniência:(Lat. \textunderscore militia\textunderscore )}
\end{itemize}
Vida ou disciplina militar.
Funcções militares.
Exercício da guerra.
Fôrça militar de um país.
Corporação disciplinada.
Os militares.
Corpos de tropas de segunda linha, que auxiliavam os de primeira ou os regimentos.
\section{Miliciano}
\begin{itemize}
\item {Grp. gram.:adj.}
\end{itemize}
\begin{itemize}
\item {Grp. gram.:M.}
\end{itemize}
Relativo a milícia.
Soldado de milícias.
\section{Milíolo}
\begin{itemize}
\item {Grp. gram.:m.}
\end{itemize}
\begin{itemize}
\item {Proveniência:(Lat. hyp. \textunderscore miliolum\textunderscore , dem. de \textunderscore milium\textunderscore ?)}
\end{itemize}
Espécie de foraminíferos. Cf. Fil. Simões, \textunderscore Beiramar\textunderscore , 170.
\section{Militança}
\begin{itemize}
\item {Grp. gram.:f.}
\end{itemize}
\begin{itemize}
\item {Utilização:Pop.}
\end{itemize}
\begin{itemize}
\item {Proveniência:(De \textunderscore militar\textunderscore )}
\end{itemize}
A profissão militar.
Os militares.
\section{Militante}
\begin{itemize}
\item {Grp. gram.:adj.}
\end{itemize}
\begin{itemize}
\item {Grp. gram.:M.}
\end{itemize}
\begin{itemize}
\item {Utilização:Ant.}
\end{itemize}
\begin{itemize}
\item {Proveniência:(Lat. \textunderscore militans\textunderscore )}
\end{itemize}
Que milita; que está em exercício; que funcciona.
Soldado, guerreiro.
\section{Militar}
\begin{itemize}
\item {Grp. gram.:adj.}
\end{itemize}
\begin{itemize}
\item {Grp. gram.:M.}
\end{itemize}
\begin{itemize}
\item {Proveniência:(Lat. \textunderscore militaris\textunderscore )}
\end{itemize}
Relativo á guerra, á milícia, ás tropas.
Pertencente ao exército; que segue a carreira das armas.
Indivíduo, que segue a carreira das armas; soldado.
\section{Militar}
\begin{itemize}
\item {Grp. gram.:v. i.}
\end{itemize}
\begin{itemize}
\item {Proveniência:(Lat. \textunderscore militare\textunderscore )}
\end{itemize}
Seguir a carreira das armas.
Combater.
Oppor-se.
Sêr membro de um partido, ou partidário de uma doutrina.
\section{Militarão}
\begin{itemize}
\item {Grp. gram.:m.}
\end{itemize}
\begin{itemize}
\item {Utilização:Pop.}
\end{itemize}
Militar rude e autoritário. Cf. J. de Deus, \textunderscore Campo de Flôres\textunderscore , 280.
\section{Militarismo}
\begin{itemize}
\item {Grp. gram.:m.}
\end{itemize}
\begin{itemize}
\item {Proveniência:(De \textunderscore militar\textunderscore ^1)}
\end{itemize}
Systema político, em que predominam elementos militares.
A milícia.
\section{Militarista}
\begin{itemize}
\item {Grp. gram.:adj.}
\end{itemize}
\begin{itemize}
\item {Grp. gram.:M.}
\end{itemize}
\begin{itemize}
\item {Proveniência:(De \textunderscore militar\textunderscore ^1)}
\end{itemize}
Relativo ao militarismo.
Partidário do militarismo. Cf. A. Cândido, \textunderscore Philos. Pol.\textunderscore , 72.
\section{Militarização}
\begin{itemize}
\item {Grp. gram.:f.}
\end{itemize}
Acto ou effeito de militarizar.
\section{Militarizar}
\begin{itemize}
\item {Grp. gram.:v. t.}
\end{itemize}
Tornar militar; dar feição militar a.
\section{Militarmente}
\begin{itemize}
\item {Grp. gram.:adv.}
\end{itemize}
De modo militar; com disciplina rigorosa.
\section{Mílite}
\begin{itemize}
\item {Grp. gram.:m.}
\end{itemize}
\begin{itemize}
\item {Utilização:Poét.}
\end{itemize}
\begin{itemize}
\item {Proveniência:(Do lat. \textunderscore miles\textunderscore , \textunderscore militis\textunderscore )}
\end{itemize}
O mesmo que \textunderscore soldado\textunderscore .
\section{Militofobia}
\begin{itemize}
\item {Grp. gram.:f.}
\end{itemize}
\begin{itemize}
\item {Utilização:Neol.}
\end{itemize}
\begin{itemize}
\item {Proveniência:(T. hybr., do lat. \textunderscore miles\textunderscore  + gr. \textunderscore phobos\textunderscore )}
\end{itemize}
Aversão á vida militar.
Ódio aos militares. Cf. Camillo, \textunderscore Mar. da Fonte\textunderscore , 144 e 148.
\section{Militophobia}
\begin{itemize}
\item {Grp. gram.:f.}
\end{itemize}
\begin{itemize}
\item {Utilização:Neol.}
\end{itemize}
\begin{itemize}
\item {Proveniência:(T. hybr., do lat. \textunderscore miles\textunderscore  + gr. \textunderscore phobos\textunderscore )}
\end{itemize}
Aversão á vida militar.
Ódio aos militares. Cf. Camillo, \textunderscore Mar. da Fonte\textunderscore , 144 e 148.
\section{Miliúsia}
\begin{itemize}
\item {Grp. gram.:f.}
\end{itemize}
Gênero de plantas anonáceas.
\section{Millanária}
\begin{itemize}
\item {Grp. gram.:f.}
\end{itemize}
Aquella que tem mil annos:«\textunderscore ...chegaria a abraçar a millanaria bella\textunderscore ». Castilho, \textunderscore Geórgicas\textunderscore , 298.--(Talvez seja êrro typográphico, em vez de \textunderscore millenária\textunderscore ; se o não é, e como Castilho perfilhava a geminação etymológica das consoantes, deveria escrever \textunderscore millannária\textunderscore , do lat. \textunderscore mille\textunderscore  + \textunderscore annus\textunderscore )
\section{Millefólio}
\begin{itemize}
\item {Grp. gram.:m.}
\end{itemize}
\begin{itemize}
\item {Proveniência:(Lat. \textunderscore millefollium\textunderscore )}
\end{itemize}
Planta da fam. das compostas, (\textunderscore achillea millefolium\textunderscore ), o mesmo que \textunderscore mil-em-rama\textunderscore , ou \textunderscore mil-fôlhas\textunderscore , ou \textunderscore milfolhada\textunderscore .
\section{Milleglana}
\begin{itemize}
\item {Grp. gram.:f.}
\end{itemize}
Planta, da serra de Sintra.
(Provavelmente, da mesma or. que \textunderscore milligran\textunderscore )
\section{Millenar}
\begin{itemize}
\item {Grp. gram.:adj.}
\end{itemize}
\begin{itemize}
\item {Utilização:Neol.}
\end{itemize}
O mesmo que \textunderscore milliário\textunderscore . Cf. C. Netto, \textunderscore Saldunes\textunderscore .
\section{Millenário}
\begin{itemize}
\item {Grp. gram.:adj.}
\end{itemize}
\begin{itemize}
\item {Grp. gram.:M.}
\end{itemize}
\begin{itemize}
\item {Proveniência:(Lat. \textunderscore millenarius\textunderscore )}
\end{itemize}
Relativo ao milhar.
Que tem mil annos.
Espaço de mil annos.
Sectário christão, que sustentava que, depois do dia de juízo, os escolhidos ficariam mil annos sôbre a terra, gozando todas as delícias.
Aquelle que, entre os Christãos da Idade-Média, acreditava que o mundo acabaria no anno 1000.
\section{Millenarismo}
\begin{itemize}
\item {Grp. gram.:m.}
\end{itemize}
\begin{itemize}
\item {Proveniência:(De \textunderscore millenário\textunderscore )}
\end{itemize}
Systema dos que sustentavam que o mundo acabaria no anno mil da era christan.
\section{Millênia}
\begin{itemize}
\item {Grp. gram.:f.}
\end{itemize}
Espécie de batalhão godo, que comprehendia duas quingentarias. Cf. C. Aires, \textunderscore Hist. do Exérc. Port.\textunderscore 
\section{Millênio}
\begin{itemize}
\item {Grp. gram.:m.}
\end{itemize}
\begin{itemize}
\item {Proveniência:(Lat. \textunderscore milleni\textunderscore )}
\end{itemize}
Espaço de mil annos.
\section{Millépede}
\begin{itemize}
\item {Grp. gram.:m.}
\end{itemize}
(Fórma bárbara com que alguns designam o bicho-de-conta. Fr. \textunderscore millepieds\textunderscore )
\section{Millépora}
\begin{itemize}
\item {Grp. gram.:f.}
\end{itemize}
\begin{itemize}
\item {Proveniência:(Do lat. \textunderscore mille\textunderscore  + \textunderscore porus\textunderscore )}
\end{itemize}
Gênero de polypeiros, cuja superfície é cavada por grande porção de poros.
\section{Milleporáceo}
\begin{itemize}
\item {Grp. gram.:adj.}
\end{itemize}
Semelhante á millépora.
\section{Millésima}
\begin{itemize}
\item {Grp. gram.:f.}
\end{itemize}
\begin{itemize}
\item {Proveniência:(De \textunderscore millésimo\textunderscore )}
\end{itemize}
Cada uma das mil partes, em que se divide um todo.
\section{Milésimo}
\begin{itemize}
\item {Grp. gram.:adj.}
\end{itemize}
\begin{itemize}
\item {Grp. gram.:M.}
\end{itemize}
\begin{itemize}
\item {Proveniência:(Lat. \textunderscore millesimus\textunderscore )}
\end{itemize}
Diz-se da última coisa numa série de mil.
Milésima.
\section{Milétia}
\begin{itemize}
\item {Grp. gram.:f.}
\end{itemize}
Gênero de plantas leguminosas.
\section{Milfose}
\begin{itemize}
\item {Grp. gram.:f.}
\end{itemize}
\begin{itemize}
\item {Utilização:Med.}
\end{itemize}
\begin{itemize}
\item {Proveniência:(Gr. \textunderscore milphosis\textunderscore )}
\end{itemize}
Quéda dos cílios, sem doença das pálpebras.
\section{Miliar}
\begin{itemize}
\item {Grp. gram.:m.}
\end{itemize}
\begin{itemize}
\item {Utilização:Neol.}
\end{itemize}
\begin{itemize}
\item {Proveniência:(Fr. \textunderscore milliard\textunderscore )}
\end{itemize}
Déz vezes cem milhões.
Bilião.
\section{Miliare}
\begin{itemize}
\item {Grp. gram.:m.}
\end{itemize}
\begin{itemize}
\item {Proveniência:(De \textunderscore milli...\textunderscore  + \textunderscore are\textunderscore )}
\end{itemize}
A milésima parte do are.
\section{Miliário}
\begin{itemize}
\item {Grp. gram.:adj.}
\end{itemize}
\begin{itemize}
\item {Proveniência:(Lat. \textunderscore milliarius\textunderscore )}
\end{itemize}
Relativo a milhas.
Que marca distâncias numa estrada: \textunderscore os marcos miliários dos Romanos\textunderscore .
Que assinala na história uma época ou data memorável.
\section{Miligrama}
\begin{itemize}
\item {Grp. gram.:m.}
\end{itemize}
\begin{itemize}
\item {Proveniência:(De \textunderscore milli...\textunderscore  + \textunderscore gramma\textunderscore )}
\end{itemize}
A milésima parte do grama.
\section{Miligran}
\begin{itemize}
\item {Grp. gram.:f.}
\end{itemize}
\begin{itemize}
\item {Utilização:Prov.}
\end{itemize}
\begin{itemize}
\item {Utilização:trasm.}
\end{itemize}
\begin{itemize}
\item {Proveniência:(Do lat. \textunderscore mille\textunderscore  + \textunderscore granus\textunderscore )}
\end{itemize}
O mesmo que \textunderscore milgrada\textunderscore .
\section{Miligrana}
\begin{itemize}
\item {Utilização:T. de Miranda}
\end{itemize}
O mesmo que \textunderscore milgrada\textunderscore .
\section{Mililitro}
\begin{itemize}
\item {Grp. gram.:m.}
\end{itemize}
\begin{itemize}
\item {Proveniência:(De \textunderscore milli...\textunderscore  + \textunderscore litro\textunderscore )}
\end{itemize}
A milésima parte do litro.
\section{Milímetro}
\begin{itemize}
\item {Grp. gram.:m.}
\end{itemize}
\begin{itemize}
\item {Proveniência:(De \textunderscore milli...\textunderscore  + \textunderscore metro\textunderscore )}
\end{itemize}
A milésima parte do metro.
\section{Mílimo}
\begin{itemize}
\item {Grp. gram.:m.}
\end{itemize}
\begin{itemize}
\item {Utilização:Des.}
\end{itemize}
\begin{itemize}
\item {Proveniência:(Do lat. \textunderscore mille\textunderscore )}
\end{itemize}
A milésima parte, de um todo.
\section{Milímodo}
\begin{itemize}
\item {Grp. gram.:adj.}
\end{itemize}
\begin{itemize}
\item {Proveniência:(Lat. \textunderscore millimodus\textunderscore )}
\end{itemize}
Que se realiza de mil maneiras.
Infinitamente variado.
\section{Milionário}
\begin{itemize}
\item {Grp. gram.:m.  e  adj.}
\end{itemize}
\begin{itemize}
\item {Proveniência:(De \textunderscore milhão\textunderscore ^1)}
\end{itemize}
Aquele que possue milhões.
Muito rico.
\section{Milionésima}
\begin{itemize}
\item {Grp. gram.:f.}
\end{itemize}
\begin{itemize}
\item {Proveniência:(De \textunderscore milionésimo\textunderscore )}
\end{itemize}
Cada uma de um milhão de partes, em que se divide um todo.
\section{Milionésimo}
\begin{itemize}
\item {Grp. gram.:adj.}
\end{itemize}
\begin{itemize}
\item {Grp. gram.:M.}
\end{itemize}
\begin{itemize}
\item {Proveniência:(De \textunderscore milhão\textunderscore )}
\end{itemize}
Que ocupa o último lugar numa série de um milhão de coisas.
Milionésima.
\section{Milionocracia}
\begin{itemize}
\item {Grp. gram.:f.}
\end{itemize}
\begin{itemize}
\item {Utilização:bras}
\end{itemize}
\begin{itemize}
\item {Utilização:Neol.}
\end{itemize}
O mesmo que \textunderscore plutocracia\textunderscore .
\section{Milípede}
\begin{itemize}
\item {Grp. gram.:adj.}
\end{itemize}
\begin{itemize}
\item {Proveniência:(Do lat. \textunderscore mille\textunderscore  + \textunderscore pes\textunderscore , \textunderscore pedis\textunderscore )}
\end{itemize}
Que tem muitos pés; miriápode.
\section{Milistere}
\begin{itemize}
\item {Grp. gram.:m.}
\end{itemize}
\begin{itemize}
\item {Proveniência:(De \textunderscore milli...\textunderscore  + \textunderscore estere\textunderscore )}
\end{itemize}
A milésima parte de um estere.
\section{Millésimo}
\begin{itemize}
\item {Grp. gram.:adj.}
\end{itemize}
\begin{itemize}
\item {Grp. gram.:M.}
\end{itemize}
\begin{itemize}
\item {Proveniência:(Lat. \textunderscore millesimus\textunderscore )}
\end{itemize}
Diz-se da última coisa numa série de mil.
Millésima.
\section{Millétia}
\begin{itemize}
\item {Grp. gram.:f.}
\end{itemize}
Gênero de plantas leguminosas.
\section{Milli...}
\begin{itemize}
\item {Grp. gram.:pref.}
\end{itemize}
(designativo, no systema métrico, da millésima parte da quantidade a que se junta)
\section{Milliar}
\begin{itemize}
\item {Grp. gram.:m.}
\end{itemize}
\begin{itemize}
\item {Utilização:Neol.}
\end{itemize}
\begin{itemize}
\item {Proveniência:(Fr. \textunderscore milliard\textunderscore )}
\end{itemize}
Déz vezes cem milhões.
Billião.
\section{Milliare}
\begin{itemize}
\item {Grp. gram.:m.}
\end{itemize}
\begin{itemize}
\item {Proveniência:(De \textunderscore milli...\textunderscore  + \textunderscore are\textunderscore )}
\end{itemize}
A millésima parte do are.
\section{Milliário}
\begin{itemize}
\item {Grp. gram.:adj.}
\end{itemize}
\begin{itemize}
\item {Proveniência:(Lat. \textunderscore milliarius\textunderscore )}
\end{itemize}
Relativo a milhas.
Que marca distâncias numa estrada: \textunderscore os marcos milliários dos Romanos\textunderscore .
Que assinala na história uma época ou data memorável.
\section{Milligramma}
\begin{itemize}
\item {Grp. gram.:m.}
\end{itemize}
\begin{itemize}
\item {Proveniência:(De \textunderscore milli...\textunderscore  + \textunderscore gramma\textunderscore )}
\end{itemize}
A millésima parte do gramma.
\section{Milligran}
\begin{itemize}
\item {Grp. gram.:f.}
\end{itemize}
\begin{itemize}
\item {Utilização:Prov.}
\end{itemize}
\begin{itemize}
\item {Utilização:trasm.}
\end{itemize}
\begin{itemize}
\item {Proveniência:(Do lat. \textunderscore mille\textunderscore  + \textunderscore granus\textunderscore )}
\end{itemize}
O mesmo que \textunderscore milgrada\textunderscore .
\section{Milligrana}
\begin{itemize}
\item {Utilização:T. de Miranda}
\end{itemize}
O mesmo que \textunderscore milgrada\textunderscore .
\section{Millilitro}
\begin{itemize}
\item {Grp. gram.:m.}
\end{itemize}
\begin{itemize}
\item {Proveniência:(De \textunderscore milli...\textunderscore  + \textunderscore litro\textunderscore )}
\end{itemize}
A millésima parte do litro.
\section{Millímetro}
\begin{itemize}
\item {Grp. gram.:m.}
\end{itemize}
\begin{itemize}
\item {Proveniência:(De \textunderscore milli...\textunderscore  + \textunderscore metro\textunderscore )}
\end{itemize}
A millésima parte do metro.
\section{Míllimo}
\begin{itemize}
\item {Grp. gram.:m.}
\end{itemize}
\begin{itemize}
\item {Utilização:Des.}
\end{itemize}
\begin{itemize}
\item {Proveniência:(Do lat. \textunderscore mille\textunderscore )}
\end{itemize}
A millésima parte, de um todo.
\section{Millímodo}
\begin{itemize}
\item {Grp. gram.:adj.}
\end{itemize}
\begin{itemize}
\item {Proveniência:(Lat. \textunderscore millimodus\textunderscore )}
\end{itemize}
Que se realiza de mil maneiras.
Infinitamente variado.
\section{Mil-lindo}
\begin{itemize}
\item {Grp. gram.:adj.}
\end{itemize}
\begin{itemize}
\item {Utilização:Des.}
\end{itemize}
Muito lindo. Cf. Filinto, VII, 42.
\section{Millionário}
\begin{itemize}
\item {Grp. gram.:m.  e  adj.}
\end{itemize}
\begin{itemize}
\item {Proveniência:(De \textunderscore milhão\textunderscore ^1)}
\end{itemize}
Aquelle que possue milhões.
Muito rico.
\section{Millionésima}
\begin{itemize}
\item {Grp. gram.:f.}
\end{itemize}
\begin{itemize}
\item {Proveniência:(De \textunderscore millionésimo\textunderscore )}
\end{itemize}
Cada uma de um milhão de partes, em que se divide um todo.
\section{Millionésimo}
\begin{itemize}
\item {Grp. gram.:adj.}
\end{itemize}
\begin{itemize}
\item {Grp. gram.:M.}
\end{itemize}
\begin{itemize}
\item {Proveniência:(De \textunderscore milhão\textunderscore )}
\end{itemize}
Que occupa o último lugar numa série de um milhão de coisas.
Millionésima.
\section{Millionocracia}
\begin{itemize}
\item {Grp. gram.:f.}
\end{itemize}
\begin{itemize}
\item {Utilização:bras}
\end{itemize}
\begin{itemize}
\item {Utilização:Neol.}
\end{itemize}
O mesmo que \textunderscore plutocracia\textunderscore .
\section{Millípede}
\begin{itemize}
\item {Grp. gram.:adj.}
\end{itemize}
\begin{itemize}
\item {Proveniência:(Do lat. \textunderscore mille\textunderscore  + \textunderscore pes\textunderscore , \textunderscore pedis\textunderscore )}
\end{itemize}
Que tem muitos pés; miyriápode.
\section{Millistere}
\begin{itemize}
\item {Grp. gram.:m.}
\end{itemize}
\begin{itemize}
\item {Proveniência:(De \textunderscore milli...\textunderscore  + \textunderscore estere\textunderscore )}
\end{itemize}
A millésima parte de um estere.
\section{Milococo}
\begin{itemize}
\item {Grp. gram.:m.}
\end{itemize}
Milho miúdo da África.
\section{Milola}
\begin{itemize}
\item {Grp. gram.:f.}
\end{itemize}
Arvoreta malvácea de Moçambique, (\textunderscore hibiscus tiliaceus\textunderscore , Lin.).
\section{Milòló}
\begin{itemize}
\item {Grp. gram.:m.}
\end{itemize}
Planta anonácea, (\textunderscore anona reticulata\textunderscore ).
\section{Milombe}
\begin{itemize}
\item {Grp. gram.:m.}
\end{itemize}
Ave africana, (\textunderscore lamppocolius acuticaudus\textunderscore , Buc.), de plumagem verde-bronze-escura, com reflexos azulados, bico levemente curvo, comprido e luzidio.
\section{Milongas}
\begin{itemize}
\item {Grp. gram.:f.}
\end{itemize}
\begin{itemize}
\item {Utilização:Bras}
\end{itemize}
Mexericos; intrigas.
(Do quimbundo)
\section{Milongo}
\begin{itemize}
\item {Grp. gram.:m.}
\end{itemize}
Nome, com que os sertanejos de Angola designam qualquer medicamento.
\section{Milongueiro}
\begin{itemize}
\item {Grp. gram.:m.}
\end{itemize}
Aquelle que, entre os Negros da África, faz ou applica os milongos.
\section{Milphose}
\begin{itemize}
\item {Grp. gram.:f.}
\end{itemize}
\begin{itemize}
\item {Utilização:Med.}
\end{itemize}
\begin{itemize}
\item {Proveniência:(Gr. \textunderscore milphosis\textunderscore )}
\end{itemize}
Quéda dos cílios, sem doença das pálpebras.
\section{Miltónia}
\begin{itemize}
\item {Grp. gram.:f.}
\end{itemize}
\begin{itemize}
\item {Proveniência:(De \textunderscore Milton\textunderscore , n. p.)}
\end{itemize}
Gênero de orchídeas.
\section{Miltoniano}
\begin{itemize}
\item {Grp. gram.:adj.}
\end{itemize}
Relativo a Milton ou ao seu estilo.
Parecido ao estilo de Milton.
\section{Milu}
\begin{itemize}
\item {Grp. gram.:m.  e  f.}
\end{itemize}
Ave gallinácea da América.
\section{Milvina}
\begin{itemize}
\item {Grp. gram.:f.}
\end{itemize}
\begin{itemize}
\item {Proveniência:(Lat. \textunderscore milvina\textunderscore )}
\end{itemize}
Espécie de frauta, de sons agudíssimos, conhecida entre os Romanos.
\section{Milvíneas}
\begin{itemize}
\item {Grp. gram.:f. pl.}
\end{itemize}
Família de aves, a que pertence o mílvio.
\section{Mílvio}
\begin{itemize}
\item {Grp. gram.:m.}
\end{itemize}
\begin{itemize}
\item {Utilização:Poét.}
\end{itemize}
\begin{itemize}
\item {Proveniência:(Lat. \textunderscore milvius\textunderscore )}
\end{itemize}
Milhafre, milhano. Cf. Castilho, Fastos, II, 93.
\section{Mim}
\begin{itemize}
\item {Proveniência:(Do lat. \textunderscore mihi\textunderscore )}
\end{itemize}
Variação do pron. \textunderscore eu\textunderscore , quando êste é precedido de preposição.
\section{Mima}
\begin{itemize}
\item {Grp. gram.:f.}
\end{itemize}
\begin{itemize}
\item {Proveniência:(Lat. \textunderscore mima\textunderscore . Cp. \textunderscore mimo\textunderscore ^2)}
\end{itemize}
A mulhér que representa comédias burlescas, servindo-se especialmente do gesto para imitar caracteres ridículos ou baixos.
\section{Mimalhice}
\begin{itemize}
\item {Grp. gram.:f.}
\end{itemize}
Qualidade ou acto de mimalho. Cf. Camillo, \textunderscore Brasileira\textunderscore , 298.
\section{Mimalho}
\begin{itemize}
\item {Grp. gram.:m. e adj.}
\end{itemize}
Aquelle que tem muito mimo; piegas.
\section{Mimança}
\begin{itemize}
\item {Grp. gram.:f.}
\end{itemize}
\begin{itemize}
\item {Utilização:T. da Bairrada}
\end{itemize}
Muito mimo.
Ousadia.
(Cp. \textunderscore mimanço\textunderscore )
\section{Mimanço}
\begin{itemize}
\item {Grp. gram.:m. e adj.}
\end{itemize}
O mesmo que \textunderscore mimalho\textunderscore .
\section{Mimansa}
\begin{itemize}
\item {Grp. gram.:f.}
\end{itemize}
Doutrina ou escola philosóphica da Índia.
\section{Mimar}
\begin{itemize}
\item {Grp. gram.:v. t.}
\end{itemize}
\begin{itemize}
\item {Proveniência:(De \textunderscore mimo\textunderscore ^2)}
\end{itemize}
Exprimir por gestos; falar por mímica.
\section{Mimar}
\begin{itemize}
\item {Grp. gram.:v. t.}
\end{itemize}
O mesmo que \textunderscore amimar\textunderscore .
\section{Mimese}
\begin{itemize}
\item {Grp. gram.:f.}
\end{itemize}
\begin{itemize}
\item {Proveniência:(Gr. \textunderscore mimesis\textunderscore )}
\end{itemize}
Figura de Rhetórica, em que o orador imita o gesto ou a voz de outrem.
\section{Mimete}
\begin{itemize}
\item {Grp. gram.:m.}
\end{itemize}
Gênero de plantas proteáceas.
\section{Mimetesa}
\begin{itemize}
\item {Grp. gram.:f.}
\end{itemize}
Variedade de arseniato de chumbo.
\section{Mimetismo}
\begin{itemize}
\item {Grp. gram.:m.}
\end{itemize}
\begin{itemize}
\item {Proveniência:(Do gr. \textunderscore mimetes\textunderscore , imitador)}
\end{itemize}
Tendência de vários animaes a tomar a côr e a configuração dos objectos, em cujo meio vivem.
\section{Mimiambo}
\begin{itemize}
\item {Grp. gram.:m. e adj.}
\end{itemize}
Espécie de verso livre que os mimos repetiam nas suas farsas.
(Cp. \textunderscore mimo\textunderscore ^2 + \textunderscore jambo\textunderscore )
\section{Mímica}
\begin{itemize}
\item {Grp. gram.:f.}
\end{itemize}
\begin{itemize}
\item {Proveniência:(De \textunderscore mímico\textunderscore )}
\end{itemize}
Arte de exprimir o pensamento por meio de gestos.
O mesmo que \textunderscore gesticulação\textunderscore .
\section{Mimicamente}
\begin{itemize}
\item {Grp. gram.:adv.}
\end{itemize}
De modo mímico.
Por gestos.
\section{Mimicar}
\begin{itemize}
\item {Grp. gram.:v. t.  e  i.}
\end{itemize}
\begin{itemize}
\item {Utilização:bras}
\end{itemize}
\begin{itemize}
\item {Utilização:Neol.}
\end{itemize}
\begin{itemize}
\item {Proveniência:(De \textunderscore mímica\textunderscore )}
\end{itemize}
Exprimir por gestos.
Gesticular.
\section{Mímico}
\begin{itemize}
\item {Grp. gram.:adj.}
\end{itemize}
\begin{itemize}
\item {Proveniência:(Lat. \textunderscore mimicus\textunderscore )}
\end{itemize}
Relativo á mímica ou á gesticulação: \textunderscore representação mímica\textunderscore .
Que exprime as suas ideias por meio de gestos.
\section{Mimo}
\begin{itemize}
\item {Grp. gram.:m.}
\end{itemize}
Coisa pequena e delicada, que se offerece ou se dá.
Offerenda; presente.
Meiguices, afago, carinho.
Suavidade.
Primor.
Delicadeza.
Belleza.
(Cp. cast. \textunderscore mimo\textunderscore )
\section{Mimo}
\begin{itemize}
\item {Grp. gram.:m.}
\end{itemize}
\begin{itemize}
\item {Utilização:Ant.}
\end{itemize}
\begin{itemize}
\item {Proveniência:(Lat. \textunderscore mimus\textunderscore )}
\end{itemize}
Actor, que representava peças familiares ou burlescas.
Representação burlesca.
Pequeno drama familiar, no dialecto syracusano.
\section{Mimo-de-vênus}
\begin{itemize}
\item {Grp. gram.:m.}
\end{itemize}
Planta brasileira, (\textunderscore hibiscus rosa sinensis\textunderscore ).
\section{Mimodrama}
\begin{itemize}
\item {Grp. gram.:m.}
\end{itemize}
\begin{itemize}
\item {Proveniência:(Do gr. \textunderscore mimos\textunderscore  + \textunderscore drama\textunderscore )}
\end{itemize}
Acção dramática, representada em pantomima.
\section{Mimografia}
\begin{itemize}
\item {Grp. gram.:f.}
\end{itemize}
\begin{itemize}
\item {Proveniência:(Do gr. \textunderscore mimos\textunderscore  + \textunderscore graphein\textunderscore )}
\end{itemize}
Tratado á cêrca dos mímicos ou da mímica.
\section{Mimographia}
\begin{itemize}
\item {Grp. gram.:f.}
\end{itemize}
\begin{itemize}
\item {Proveniência:(Do gr. \textunderscore mimos\textunderscore  + \textunderscore graphein\textunderscore )}
\end{itemize}
Tratado á cêrca dos mímicos ou da mímica.
\section{Mimologia}
\begin{itemize}
\item {Grp. gram.:f.}
\end{itemize}
Imitação da voz ou das locuções habituaes de uma pessôa.
(Cp. \textunderscore mimólogo\textunderscore )
\section{Mimológico}
\begin{itemize}
\item {Grp. gram.:adj.}
\end{itemize}
Relativo á mimologia.
\section{Mimologismo}
\begin{itemize}
\item {Grp. gram.:m.}
\end{itemize}
\begin{itemize}
\item {Proveniência:(De \textunderscore mimólogo\textunderscore )}
\end{itemize}
Palavra, formada por mimologia.
Onomatopeia.
\section{Mimólogo}
\begin{itemize}
\item {Grp. gram.:m.}
\end{itemize}
\begin{itemize}
\item {Proveniência:(Do gr. \textunderscore mimos\textunderscore  + \textunderscore logos\textunderscore )}
\end{itemize}
Aquelle que imita a voz ou a pronúncia de outro.
Aquelle que é versado em mimologia.
\section{Mimo-no-caco}
\begin{itemize}
\item {Grp. gram.:m.}
\end{itemize}
\begin{itemize}
\item {Utilização:T. da Bairrada}
\end{itemize}
\begin{itemize}
\item {Utilização:fam.}
\end{itemize}
O mesmo que \textunderscore mimalho\textunderscore .
Choramingas.
\section{Mimoplástico}
\begin{itemize}
\item {Grp. gram.:adj.}
\end{itemize}
\begin{itemize}
\item {Proveniência:(De \textunderscore mimo\textunderscore ^2 + \textunderscore plástico\textunderscore )}
\end{itemize}
Diz-se dos quadros vivos, especialmente dos que representam a Paixão de Christo.
\section{Mimopórfiro}
\begin{itemize}
\item {Grp. gram.:m.}
\end{itemize}
Variedade de rocha, parecida ao pórfiro.
\section{Mimopórphyro}
\begin{itemize}
\item {Grp. gram.:m.}
\end{itemize}
Variedade de rocha, parecida ao pórphyro.
\section{Mimosa}
\begin{itemize}
\item {Grp. gram.:f.}
\end{itemize}
\begin{itemize}
\item {Utilização:Gír.}
\end{itemize}
\begin{itemize}
\item {Proveniência:(De \textunderscore mimoso\textunderscore )}
\end{itemize}
Gênero de plantas leguminosas, a que pertence a sensitiva.
Camisa.
\section{Mimosamente}
\begin{itemize}
\item {Grp. gram.:adv.}
\end{itemize}
De modo mimoso; delicadamente; com mimo; com primor.
\section{Mimosear}
\begin{itemize}
\item {Grp. gram.:v. t.}
\end{itemize}
\begin{itemize}
\item {Proveniência:(De \textunderscore mimoso\textunderscore )}
\end{itemize}
Amimar.
Dar presentes a; presentear.
Obsequiar.
\section{Mimóseas}
\begin{itemize}
\item {Grp. gram.:f. pl.}
\end{itemize}
\begin{itemize}
\item {Proveniência:(De \textunderscore mimóseo\textunderscore )}
\end{itemize}
Família de plantas leguminosas, que tem por typo a mimosa.
\section{Mimóseo}
\begin{itemize}
\item {Grp. gram.:adj.}
\end{itemize}
Relativo ou semelhante á mimosa.
\section{Mimoso}
\begin{itemize}
\item {Grp. gram.:adj.}
\end{itemize}
\begin{itemize}
\item {Grp. gram.:M.}
\end{itemize}
\begin{itemize}
\item {Grp. gram.:M.}
\end{itemize}
\begin{itemize}
\item {Utilização:Gír.}
\end{itemize}
\begin{itemize}
\item {Utilização:Bras}
\end{itemize}
Que tem mimo.
Habituado a meiguices.
Delicado: \textunderscore poesia mimosa\textunderscore .
Sensível.
Carinhoso.
Brando.
Excellente.
Favorito; favorecido.
Aquelle que é feliz ou favorecido: \textunderscore os mimosos da fortuna\textunderscore .
Chapéu fino.
Espécie de forragem, no Ceará.
\section{Mímulo}
\begin{itemize}
\item {Grp. gram.:m.}
\end{itemize}
\begin{itemize}
\item {Proveniência:(Lat. bot. \textunderscore mimulus\textunderscore )}
\end{itemize}
Gênero de plantas escrofularíneas.
\section{Mina}
\begin{itemize}
\item {Grp. gram.:f.}
\end{itemize}
\begin{itemize}
\item {Utilização:Fig.}
\end{itemize}
\begin{itemize}
\item {Proveniência:(Do lat. \textunderscore miniaria\textunderscore ?)}
\end{itemize}
Cavidade artificial na terra, para se extrahirem metaes, combustiveis, liquidos, etc.
Veio mineral, no seio da terra.
Nascente de água.
Cavidade, cheia de pólvora, para que, explodindo, destrua tudo que há por cima.
Caminho subterrâneo, por onde os sitiantes de uma cidade ou praça procuram penetrar por baixo das muralhas ou trincheiras.
Manancial de riquezas.
Grandes vantagens.
Preciosidade.
\section{Mina}
\begin{itemize}
\item {Grp. gram.:f.}
\end{itemize}
\begin{itemize}
\item {Proveniência:(Lat. mina)}
\end{itemize}
Pêso e moéda entre os Gregos.
Antiga medida agrária dos Gregos e Romanos.
\section{Mina}
\begin{itemize}
\item {Grp. gram.:f.}
\end{itemize}
Gênero de plantas de jardim.
Balsamina?
\section{Minacíssimo}
\begin{itemize}
\item {Grp. gram.:adj.}
\end{itemize}
\begin{itemize}
\item {Proveniência:(Do lat. \textunderscore minax\textunderscore , \textunderscore minacis\textunderscore )}
\end{itemize}
Muito minaz; que ameaça gravemente. Cf. Camillo, \textunderscore Corja\textunderscore , 240.
\section{Minana}
\begin{itemize}
\item {Grp. gram.:f.}
\end{itemize}
Planta onagrariácea do Brasil e da África.
\section{Minante}
\begin{itemize}
\item {Grp. gram.:m.}
\end{itemize}
\begin{itemize}
\item {Utilização:T. de Âncora, no Minho}
\end{itemize}
\begin{itemize}
\item {Proveniência:(De \textunderscore minar\textunderscore )}
\end{itemize}
Porco.
Animal.
\section{Minar}
\begin{itemize}
\item {Grp. gram.:v. t.}
\end{itemize}
\begin{itemize}
\item {Utilização:Fig.}
\end{itemize}
\begin{itemize}
\item {Grp. gram.:V. i.}
\end{itemize}
\begin{itemize}
\item {Proveniência:(De \textunderscore mina\textunderscore ^1)}
\end{itemize}
Escavar, para extrahir da terra, metaes, líquidos, etc.
Abrir cavidade por baixo de.
Invadir occultamente.
Cavar.
Consumir; corroer.
Amofinar.
Prejudicar ás occultas: \textunderscore minar o crédito de alguém\textunderscore .
Diffundir-se, lavrar.
\section{Minarete}
\begin{itemize}
\item {fónica:narê}
\end{itemize}
\begin{itemize}
\item {Grp. gram.:m.}
\end{itemize}
\begin{itemize}
\item {Utilização:Gal}
\end{itemize}
\begin{itemize}
\item {Proveniência:(Fr. \textunderscore minaret\textunderscore )}
\end{itemize}
Pequena tôrre, de três ou quatro andares e balcões salientes, junto ás mesquitas.--A fórma portuguêsa é \textunderscore almenara\textunderscore .
\section{Minaz}
\begin{itemize}
\item {Grp. gram.:adj.}
\end{itemize}
\begin{itemize}
\item {Utilização:Poét.}
\end{itemize}
\begin{itemize}
\item {Proveniência:(Lat. \textunderscore minax\textunderscore )}
\end{itemize}
Ameaçador.
\section{Mincção}
\begin{itemize}
\item {Grp. gram.:m.}
\end{itemize}
\begin{itemize}
\item {Proveniência:(Lat. \textunderscore minctio\textunderscore )}
\end{itemize}
O mesmo que \textunderscore micção\textunderscore .
\section{Míncio}
\begin{itemize}
\item {Grp. gram.:m.}
\end{itemize}
\begin{itemize}
\item {Utilização:Ant.}
\end{itemize}
O mesmo que \textunderscore lutuosa\textunderscore .
\section{Mindanaus}
\begin{itemize}
\item {Grp. gram.:m. pl.}
\end{itemize}
Habitantes de Mindanau. Cf. \textunderscore Peregrinação\textunderscore , CXLIII.
\section{Mindinho}
\begin{itemize}
\item {Grp. gram.:m.  e  adj.}
\end{itemize}
\begin{itemize}
\item {Utilização:Pop.}
\end{itemize}
O dedo mínimo.
(Corr. de \textunderscore mínimo\textunderscore ?)
\section{Mindrico}
\begin{itemize}
\item {Grp. gram.:m.}
\end{itemize}
\begin{itemize}
\item {Utilização:T. do Cadaval}
\end{itemize}
\begin{itemize}
\item {Proveniência:(De \textunderscore Mindre\textunderscore , por \textunderscore Minde\textunderscore )}
\end{itemize}
Habitante de Minde.
\section{Minduba}
\begin{itemize}
\item {Grp. gram.:f.}
\end{itemize}
\begin{itemize}
\item {Utilização:Bras}
\end{itemize}
Aguardente, cachaça.
\section{Mineira}
\begin{itemize}
\item {Grp. gram.:f.}
\end{itemize}
\begin{itemize}
\item {Proveniência:(De \textunderscore mineiro\textunderscore )}
\end{itemize}
Terreno abundante de minério; mina.
\section{Mineiro}
\begin{itemize}
\item {Grp. gram.:adj.}
\end{itemize}
\begin{itemize}
\item {Grp. gram.:M.}
\end{itemize}
Relativo a mina: \textunderscore indústria mineira\textunderscore .
Em que há minas.
Aquelle que trabalha em minas.
Aquelle que possue minas.
\section{Mineiro}
\begin{itemize}
\item {Grp. gram.:adj.}
\end{itemize}
\begin{itemize}
\item {Utilização:Bras}
\end{itemize}
\begin{itemize}
\item {Grp. gram.:M.}
\end{itemize}
\begin{itemize}
\item {Proveniência:(De \textunderscore Minas-Geraes\textunderscore , n. p.)}
\end{itemize}
Relativo ao Estado de Minas-Geraes.
Habitante dêsse Estado.
\section{Mineração}
\begin{itemize}
\item {Grp. gram.:f.}
\end{itemize}
\begin{itemize}
\item {Proveniência:(De \textunderscore minerar\textunderscore )}
\end{itemize}
Exploração de minas.
Purificação do minério.
\section{Minerador}
\begin{itemize}
\item {Grp. gram.:m.}
\end{itemize}
\begin{itemize}
\item {Proveniência:(De \textunderscore minerar\textunderscore )}
\end{itemize}
Aquelle que trabalha em mineração. Cf. \textunderscore Jorn. do Comm.\textunderscore , do Rio, de 17-XII-904.
\section{Mineral}
\begin{itemize}
\item {Grp. gram.:m.}
\end{itemize}
\begin{itemize}
\item {Grp. gram.:Adj.}
\end{itemize}
Qualquer substância inorgânica, que se encontra no interior ou na superfície da terra, como metaes, pedras, combustíveis.
Relativo aos mineraes: \textunderscore o reino mineral\textunderscore .
(B. lat. \textunderscore minerale\textunderscore )
\section{Mineralização}
\begin{itemize}
\item {Grp. gram.:f.}
\end{itemize}
\begin{itemize}
\item {Proveniência:(De \textunderscore mineralizar\textunderscore )}
\end{itemize}
Transformação dos metaes em mineraes.
Combinação de substâncias metállicas com águas de nascentes.
\section{Mineralizador}
\begin{itemize}
\item {Grp. gram.:m.}
\end{itemize}
\begin{itemize}
\item {Grp. gram.:Adj.}
\end{itemize}
\begin{itemize}
\item {Proveniência:(De \textunderscore mineralizar\textunderscore )}
\end{itemize}
Substância, que mineraliza outra, isto é, que a faz passar do estado de metal ao de mineral.
Diz-se das substâncias que mineralizam, especialmente do enxôfre e do oxygênio, corpos que, combinando-se com metaes, os alteram profundamente, fazendo-os passar ao estado de mineraes.
\section{Mineralizante}
\begin{itemize}
\item {Grp. gram.:m.  e  adj.}
\end{itemize}
O mesmo que \textunderscore mineralizador\textunderscore .
\section{Mineralizar}
\begin{itemize}
\item {Grp. gram.:v. t.}
\end{itemize}
\begin{itemize}
\item {Grp. gram.:V. i.}
\end{itemize}
\begin{itemize}
\item {Proveniência:(De \textunderscore mineral\textunderscore )}
\end{itemize}
Transformar em mineral ou minério.
Procurar mineraes na terra.
\section{Mineralogia}
\begin{itemize}
\item {Grp. gram.:f.}
\end{itemize}
\begin{itemize}
\item {Proveniência:(De \textunderscore mineral\textunderscore  + gr. \textunderscore logos\textunderscore )}
\end{itemize}
Parte da História Natural, que se occupa dos mineraes.
\section{Mineralogicamente}
\begin{itemize}
\item {Grp. gram.:adv.}
\end{itemize}
\begin{itemize}
\item {Proveniência:(De \textunderscore mineralógico\textunderscore )}
\end{itemize}
Em linguagem mineralógica.
\section{Mineralógico}
\begin{itemize}
\item {Grp. gram.:adj.}
\end{itemize}
Relativo á Mineralogia.
\section{Mineralogista}
\begin{itemize}
\item {Grp. gram.:m.}
\end{itemize}
Aquelle que é versado em Mineralogia.
\section{Mineralurgia}
\begin{itemize}
\item {Grp. gram.:f.}
\end{itemize}
\begin{itemize}
\item {Proveniência:(De \textunderscore mineral\textunderscore  + gr. \textunderscore ergon\textunderscore )}
\end{itemize}
Arte, que trata das applicações dos metaes e ensina a tirar delles a maior utilidade.
\section{Mineralúrgico}
\begin{itemize}
\item {Grp. gram.:adj.}
\end{itemize}
Relativo á mineralurgia.
\section{Minerar}
\begin{itemize}
\item {Grp. gram.:v. t.}
\end{itemize}
\begin{itemize}
\item {Grp. gram.:V. i.}
\end{itemize}
Explorar (mina).
Extrahir de mina.
Trabalhar em minas.
\section{Minério}
\begin{itemize}
\item {Grp. gram.:m.}
\end{itemize}
\begin{itemize}
\item {Proveniência:(De \textunderscore mina\textunderscore )}
\end{itemize}
Mineral, que se extrái da mina, misturado com terra.
Qualquer substância metallífera.
\section{Minerografia}
\begin{itemize}
\item {Grp. gram.:f.}
\end{itemize}
\begin{itemize}
\item {Proveniência:(De \textunderscore minerógrafo\textunderscore )}
\end{itemize}
Descripção dos mineraes.
\section{Minerográfico}
\begin{itemize}
\item {Grp. gram.:adj.}
\end{itemize}
Relativo á minerografia.
\section{Minerógrafo}
\begin{itemize}
\item {Grp. gram.:m.}
\end{itemize}
\begin{itemize}
\item {Proveniência:(De \textunderscore mineral\textunderscore  + gr. \textunderscore graphein\textunderscore )}
\end{itemize}
Aquele que é versado em minerografia.
\section{Minerographia}
\begin{itemize}
\item {Grp. gram.:f.}
\end{itemize}
\begin{itemize}
\item {Proveniência:(De \textunderscore minerógrapho\textunderscore )}
\end{itemize}
Descripção dos mineraes.
\section{Minerográphico}
\begin{itemize}
\item {Grp. gram.:adj.}
\end{itemize}
Relativo á minerographia.
\section{Minerógrapho}
\begin{itemize}
\item {Grp. gram.:m.}
\end{itemize}
\begin{itemize}
\item {Proveniência:(De \textunderscore mineral\textunderscore  + gr. \textunderscore graphein\textunderscore )}
\end{itemize}
Aquelle que é versado em minerographia.
\section{Minerval}
\begin{itemize}
\item {Grp. gram.:adj.}
\end{itemize}
\begin{itemize}
\item {Grp. gram.:m.}
\end{itemize}
Relativo a Minerva.
Retribuição que, em algumas escolas de vários países, é paga aos professores pelos alumnos externos.
\section{Minervas}
\begin{itemize}
\item {Grp. gram.:f. pl.}
\end{itemize}
\begin{itemize}
\item {Utilização:T. do Fundão}
\end{itemize}
Gracejos lisonjeiros.
Galanteios, finezas.
\section{Minestra}
\begin{itemize}
\item {Grp. gram.:f.}
\end{itemize}
\begin{itemize}
\item {Utilização:Bras}
\end{itemize}
Artifício ou jeito, com que se procura obter certas coisas.
\section{Minestre}
\begin{itemize}
\item {Grp. gram.:m.}
\end{itemize}
\begin{itemize}
\item {Utilização:Bras}
\end{itemize}
Indivíduo jeitoso nos meios que emprega para conseguir qualquer coisa.
\section{Minga}
\begin{itemize}
\item {Grp. gram.:f.}
\end{itemize}
\begin{itemize}
\item {Utilização:Pop.}
\end{itemize}
\begin{itemize}
\item {Grp. gram.:Loc.}
\end{itemize}
\begin{itemize}
\item {Utilização:pop.}
\end{itemize}
O mesmo que \textunderscore míngua\textunderscore .
\textunderscore Isso não faz minga\textunderscore , isso não importa, isso não tem valor.
\section{Mingacho}
\begin{itemize}
\item {Grp. gram.:m.}
\end{itemize}
Cabaço com água, em que os pescadores conservam vivos por algum tempo os peixes de água doce.
\section{Mingança}
\begin{itemize}
\item {Grp. gram.:f.}
\end{itemize}
\begin{itemize}
\item {Utilização:Prov.}
\end{itemize}
\begin{itemize}
\item {Utilização:minh.}
\end{itemize}
\begin{itemize}
\item {Proveniência:(De \textunderscore mingar\textunderscore )}
\end{itemize}
Falta de fiado nos teares.
\section{Mingar}
\begin{itemize}
\item {Grp. gram.:v. i.}
\end{itemize}
\begin{itemize}
\item {Utilização:Pop.}
\end{itemize}
O mesmo que \textunderscore minguar\textunderscore :«\textunderscore ...Duas fontes, uma das quaes não cresce nem minga.\textunderscore »\textunderscore Rev. de Guimarães\textunderscore , vol. XV, 160.
\section{Mingau}
\begin{itemize}
\item {Grp. gram.:m.}
\end{itemize}
\begin{itemize}
\item {Utilização:Bras}
\end{itemize}
\begin{itemize}
\item {Proveniência:(Do guar. \textunderscore migau\textunderscore )}
\end{itemize}
Papas de farinha.
\section{Mingo}
\begin{itemize}
\item {Grp. gram.:m.}
\end{itemize}
\begin{itemize}
\item {Utilização:Prov.}
\end{itemize}
\begin{itemize}
\item {Utilização:alent.}
\end{itemize}
Bóla encarnada do bilhar.
(Cast. \textunderscore mingo\textunderscore )
\section{Mingola}
\begin{itemize}
\item {Grp. gram.:adj.}
\end{itemize}
\begin{itemize}
\item {Utilização:Ant.}
\end{itemize}
O mesmo que \textunderscore mendicante\textunderscore  (frade). Cf. Pant. de Aveiro, \textunderscore Itiner.\textunderscore , 244 v.^o, (2.^a ed.).
\section{Mingolas}
\begin{itemize}
\item {Grp. gram.:m.}
\end{itemize}
\begin{itemize}
\item {Utilização:Bras}
\end{itemize}
Homem avarento.
\section{Mingornilha}
\begin{itemize}
\item {Grp. gram.:f.}
\end{itemize}
\begin{itemize}
\item {Utilização:T. de Alcanena}
\end{itemize}
Usado na loc. adv. \textunderscore de mingornilha\textunderscore , de alto abaixo.
\section{Mingorra}
\begin{itemize}
\item {fónica:gô}
\end{itemize}
\begin{itemize}
\item {Grp. gram.:f.}
\end{itemize}
\begin{itemize}
\item {Utilização:Prov.}
\end{itemize}
\begin{itemize}
\item {Utilização:alent.}
\end{itemize}
Membro viril da criança.
(Cp. lat. \textunderscore mingere\textunderscore )
\section{Mingrélio}
\begin{itemize}
\item {Grp. gram.:m.}
\end{itemize}
Uma das línguas bárbaras do Cáucaso.
\section{Mingrólio}
\begin{itemize}
\item {Grp. gram.:m.}
\end{itemize}
Espécie de flôr escura:«\textunderscore ...vaso... carregado de mingrólios, essas flôres escuras...\textunderscore »Castilho, \textunderscore Mil e um Myst.\textunderscore , 228.
\section{Mingu}
\begin{itemize}
\item {Grp. gram.:m.}
\end{itemize}
Árvore silvestre do Brasil.
\section{Míngua}
\begin{itemize}
\item {Grp. gram.:f.}
\end{itemize}
\begin{itemize}
\item {Proveniência:(De \textunderscore minguar\textunderscore )}
\end{itemize}
Falta do necessário.
Escassez; deminuição.
Defeito.
\section{Minguá}
\begin{itemize}
\item {Grp. gram.:m.}
\end{itemize}
\begin{itemize}
\item {Utilização:Bras}
\end{itemize}
Ave marinha.
\section{Minguadamente}
\begin{itemize}
\item {Grp. gram.:adv.}
\end{itemize}
De modo minguado; escassamente.
\section{Minguado}
\begin{itemize}
\item {Grp. gram.:adj.}
\end{itemize}
\begin{itemize}
\item {Proveniência:(De \textunderscore minguar\textunderscore )}
\end{itemize}
Que carece do necessário; privado.
Escasso; limitado.
Desditoso.
\section{Minguador}
\begin{itemize}
\item {Grp. gram.:adj.}
\end{itemize}
Que mingua.
\section{Minguamento}
\begin{itemize}
\item {Grp. gram.:m.}
\end{itemize}
Acto de minguar.
\section{Minguante}
\begin{itemize}
\item {Grp. gram.:adj.}
\end{itemize}
\begin{itemize}
\item {Grp. gram.:M.}
\end{itemize}
Que mingúa.
Quarto minguante.
Deminuição; decadência.
\section{Minguar}
\begin{itemize}
\item {Grp. gram.:v. i.}
\end{itemize}
\begin{itemize}
\item {Proveniência:(Lat. hypoth. \textunderscore minuicare\textunderscore )}
\end{itemize}
Tornar-se menor, deminuír.
Escassear.
Declinar; decaír.
\section{Minha}
\begin{itemize}
\item {Grp. gram.:F.}
\end{itemize}
\begin{itemize}
\item {Proveniência:(Do obsol. \textunderscore mia\textunderscore )}
\end{itemize}
Flexão \textunderscore fem.\textunderscore  do pron. \textunderscore meu\textunderscore .
Variedade de pêra.
\section{Minhafre}
\begin{itemize}
\item {Grp. gram.:m.}
\end{itemize}
\begin{itemize}
\item {Utilização:Prov.}
\end{itemize}
\begin{itemize}
\item {Utilização:trasm.}
\end{itemize}
O mesmo que \textunderscore milhafre\textunderscore .
\section{Minha-minha}
\begin{itemize}
\item {Grp. gram.:f.}
\end{itemize}
Raíz de uma árvore de Angola.
\section{Minhana}
\begin{itemize}
\item {Grp. gram.:f.}
\end{itemize}
\begin{itemize}
\item {Utilização:Ant.}
\end{itemize}
O mesmo que \textunderscore menina\textunderscore ? Cf. \textunderscore Port. Mon. Hist.\textunderscore , \textunderscore Script.\textunderscore , 271, 316, 321, 323, 324 e 325.--Encontra-se o vocab., applicado a filhas de ricos-homens, e parece designar tratamento carinhoso. Os diccionaristas, os philólogos, os antiquários, inda o não explicaram. Cp. entretanto o port. \textunderscore minhão\textunderscore ; o fr. \textunderscore mignon\textunderscore ; o ant. alt. alemão \textunderscore mimne\textunderscore , (amor, objecto amado); o mod. alt. alemão \textunderscore mimne\textunderscore , (amor); o irlandês \textunderscore mion\textunderscore  e \textunderscore mian\textunderscore , (amor); e vejam-se as expressões dos \textunderscore Script.\textunderscore : \textunderscore minhana dona Oraca\textunderscore ; \textunderscore minhana dona Tareja...\textunderscore  Fazem lembrar o fr. \textunderscore mademoiselle\textunderscore , anteposto a nomes; \textunderscore M.^{lle} Marie...\textunderscore 
\section{Minhão}
\begin{itemize}
\item {Grp. gram.:m.}
\end{itemize}
\begin{itemize}
\item {Utilização:Ant.}
\end{itemize}
\begin{itemize}
\item {Proveniência:(Fr. \textunderscore mignon\textunderscore )}
\end{itemize}
Menino querido, de relações íntimas.
\section{Minheiro}
\begin{itemize}
\item {Grp. gram.:adj.}
\end{itemize}
\begin{itemize}
\item {Utilização:Prov.}
\end{itemize}
\begin{itemize}
\item {Proveniência:(De \textunderscore Minho\textunderscore , n. p.)}
\end{itemize}
Diz-se de uma espécie de pão, também chamado \textunderscore pão ralo\textunderscore .
\section{Minhoca}
\begin{itemize}
\item {Grp. gram.:f.}
\end{itemize}
\begin{itemize}
\item {Grp. gram.:Pl.}
\end{itemize}
\begin{itemize}
\item {Utilização:Fam.}
\end{itemize}
\begin{itemize}
\item {Utilização:Gír.}
\end{itemize}
Verme anélido, (\textunderscore lumbricus terrestris\textunderscore ).
Manias; crendices.
Sopas de macarrão.
(Quimb. \textunderscore minhoca\textunderscore )
\section{Minhocada}
\begin{itemize}
\item {Grp. gram.:f.}
\end{itemize}
\begin{itemize}
\item {Utilização:Pesc.}
\end{itemize}
Engôdo para pescar enguias, feito de molho de minhocas.
\section{Minhocão}
\begin{itemize}
\item {Grp. gram.:m.}
\end{itemize}
\begin{itemize}
\item {Utilização:Bras}
\end{itemize}
\begin{itemize}
\item {Proveniência:(De \textunderscore minhoca\textunderscore ?)}
\end{itemize}
Amphíbio das lagôas do centro do Brasil.
\section{Minhonete}
\begin{itemize}
\item {Grp. gram.:f.}
\end{itemize}
\begin{itemize}
\item {Proveniência:(Fr. \textunderscore mignonnette\textunderscore )}
\end{itemize}
Planta resedácea, (\textunderscore reseda odorata\textunderscore ).
\section{Minhoteira}
\begin{itemize}
\item {Grp. gram.:f.}
\end{itemize}
\begin{itemize}
\item {Utilização:Ant.}
\end{itemize}
\begin{itemize}
\item {Proveniência:(De \textunderscore minhoto\textunderscore ?)}
\end{itemize}
Pequena ponte de madeira.
Propriedade ou fazenda, dividida por uma ponte?:«\textunderscore deixo a minha minhoteira, que tem hũ tanque com duas terras de fóra.\textunderscore »(De um testamento de 1692)
\section{Minhotismo}
\begin{itemize}
\item {Grp. gram.:m.}
\end{itemize}
\begin{itemize}
\item {Proveniência:(De \textunderscore minhoto\textunderscore )}
\end{itemize}
Palavra ou locução privativa do Minho.
\section{Minhoto}
\begin{itemize}
\item {fónica:nhô}
\end{itemize}
\begin{itemize}
\item {Grp. gram.:adj.}
\end{itemize}
\begin{itemize}
\item {Grp. gram.:M.}
\end{itemize}
Relativo ao Minho: \textunderscore costume minhoto\textunderscore .
Peça, formada de dois triângulos de pau, que se embebem em madeira raxada, para que não abra mais.
Habitante ou indivíduo natural do Minho.
O mesmo que \textunderscore papa-pintos\textunderscore :«\textunderscore gaviões e minhotos.\textunderscore »Vieira, VI, 298.
\section{Miniatura}
\begin{itemize}
\item {Grp. gram.:f.}
\end{itemize}
\begin{itemize}
\item {Proveniência:(Do lat. \textunderscore miniatus\textunderscore )}
\end{itemize}
Letra vermelha, traçada com mínio, e posta no princípio dos capítulos ou dos parágraphos de manuscritos antigos.
Letra pintada ou ornada de qualquer côr.
Espécie de pintura delicada, em pequeno ponto.
Qualquer coisa em ponto pequeno; resumo.
\section{Miniaturar}
\begin{itemize}
\item {Grp. gram.:v. t.}
\end{itemize}
Pintar em miniaturas.
Descrever por miúdo.
\section{Miniaturista}
\begin{itemize}
\item {Grp. gram.:m. ,  f.  e  adj.}
\end{itemize}
Pessôa, que faz miniaturas.
\section{Mínima}
\begin{itemize}
\item {Grp. gram.:f.}
\end{itemize}
\begin{itemize}
\item {Proveniência:(De \textunderscore mínimo\textunderscore )}
\end{itemize}
Nota musical, do valor de metade da semibreve.
\section{Mínimo}
\begin{itemize}
\item {Grp. gram.:adj.}
\end{itemize}
\begin{itemize}
\item {Grp. gram.:M.}
\end{itemize}
\begin{itemize}
\item {Proveniência:(Lat. \textunderscore minimus\textunderscore )}
\end{itemize}
Que é o mais pequeno: \textunderscore preço mínimo\textunderscore .
A mais pequena porção de uma coisa.
Dedo mínimo.
\section{Minino}
\begin{itemize}
\item {Grp. gram.:m.}
\end{itemize}
\begin{itemize}
\item {Utilização:Des.}
\end{itemize}
O mesmo que \textunderscore menino\textunderscore . Cf. Usque, f. 15; \textunderscore Eufrosina\textunderscore , 89 e 117; \textunderscore Luz e Calor\textunderscore , passim. \textunderscore Lusíadas\textunderscore , II, 36; III, 125; IV, 3 e 92; IX, 30 e 35.
\section{Mínio}
\begin{itemize}
\item {Grp. gram.:m.}
\end{itemize}
\begin{itemize}
\item {Proveniência:(Lat. \textunderscore minium\textunderscore )}
\end{itemize}
Designação vulgar do deutóxydo de chumbo, também conhecido por \textunderscore cinábrio\textunderscore  ou \textunderscore vermelhão\textunderscore .
\section{Ministerial}
\begin{itemize}
\item {Grp. gram.:adj.}
\end{itemize}
\begin{itemize}
\item {Grp. gram.:M.}
\end{itemize}
\begin{itemize}
\item {Proveniência:(Lat. \textunderscore ministerialis\textunderscore )}
\end{itemize}
Relativo a Ministério.
Feito por Ministros ou proveniente delles: \textunderscore resolução ministerial\textunderscore .
Que defende ou segue o partido ou os actos de um Ministério.
Partidário ou defensor de um Ministério ou Govêrno.
\section{Ministerialismo}
\begin{itemize}
\item {Grp. gram.:m.}
\end{itemize}
\begin{itemize}
\item {Proveniência:(De \textunderscore ministerial\textunderscore )}
\end{itemize}
Systema ou opinião dos que defendem incondicionalmente os Ministros ou o Govêrno.
\section{Ministerialista}
\begin{itemize}
\item {Grp. gram.:m.}
\end{itemize}
\begin{itemize}
\item {Proveniência:(De \textunderscore ministerial\textunderscore )}
\end{itemize}
Aquelle que defende incondicionalmente os Ministros ou o Govêrno.
\section{Ministerialmente}
\begin{itemize}
\item {Grp. gram.:adv.}
\end{itemize}
De modo ministerial.
Á maneira de ministro de Estado.
\section{Ministério}
\begin{itemize}
\item {Grp. gram.:m.}
\end{itemize}
\begin{itemize}
\item {Proveniência:(Lat. \textunderscore ministerium\textunderscore )}
\end{itemize}
Mestér.
Profissão manual, officio.
Cargo.
Conjunto dos Ministros, que constituem o poder executivo com o chefe do Estado.
Secretaria de Estado: \textunderscore o Ministério da Justiça\textunderscore .
\textunderscore Ministério público\textunderscore , a magistratura que, junto dos tribunaes, vela pela execução das leis e manutenção da ordem.
\section{Ministra}
\begin{itemize}
\item {Grp. gram.:f.}
\end{itemize}
\begin{itemize}
\item {Proveniência:(De \textunderscore ministro\textunderscore )}
\end{itemize}
Pessôa do sexo feminino, ou coisa do gênero feminino, que concorre para determinado fim.
Roda, por onde se passava a comida, da cozinha para os refeitórios dos conventos.
Utensílio de madeira, com um entalhe, em que os sapateiros embebem o fio da faca, para o resguardar de qualquer contacto.
Mulher de Ministro ou de Embaixador.
\section{Ministra}
\begin{itemize}
\item {Grp. gram.:f.}
\end{itemize}
\begin{itemize}
\item {Proveniência:(Do it. \textunderscore minestra\textunderscore )}
\end{itemize}
Espécie de sopa italiana:«\textunderscore as ministras italianas foram caldo de gallinha\textunderscore ». Vieira, \textunderscore Carta CXIII\textunderscore .
\section{Ministraço}
\begin{itemize}
\item {Grp. gram.:m.}
\end{itemize}
\begin{itemize}
\item {Utilização:deprec.}
\end{itemize}
\begin{itemize}
\item {Utilização:Pop.}
\end{itemize}
O mesmo que \textunderscore Ministro\textunderscore .
\section{Ministrado}
\begin{itemize}
\item {Grp. gram.:m.}
\end{itemize}
Cargo de Ministro. Cf. \textunderscore Vita Christi\textunderscore , (annexo).
\section{Ministrador}
\begin{itemize}
\item {Grp. gram.:m.  e  adj.}
\end{itemize}
\begin{itemize}
\item {Utilização:Ant.}
\end{itemize}
\begin{itemize}
\item {Proveniência:(Lat. \textunderscore ministrator\textunderscore )}
\end{itemize}
O que ministra.
Administrador.
\section{Ministrante}
\begin{itemize}
\item {Grp. gram.:m.  e  adj.}
\end{itemize}
\begin{itemize}
\item {Proveniência:(Lat. \textunderscore ministrans\textunderscore )}
\end{itemize}
Aquelle que ministra.
Aquelle que exerce algum ministério ou cargo.
\section{Ministrar}
\begin{itemize}
\item {Grp. gram.:v. t.}
\end{itemize}
\begin{itemize}
\item {Proveniência:(Lat. \textunderscore ministrare\textunderscore )}
\end{itemize}
Prestar; fornecer: \textunderscore ministrar recursos\textunderscore .
Apresentar.
Servir.
Administrar.
Inspirar: \textunderscore ministrar rancores\textunderscore .
\section{Ministraria}
\begin{itemize}
\item {Grp. gram.:f.}
\end{itemize}
\begin{itemize}
\item {Utilização:Des.}
\end{itemize}
Cargo de Ministro.
\section{Ministre}
\begin{itemize}
\item {Grp. gram.:m.}
\end{itemize}
\begin{itemize}
\item {Utilização:Ant.}
\end{itemize}
O mesmo que \textunderscore menestrel\textunderscore .
\section{Ministrice}
\begin{itemize}
\item {Grp. gram.:f.}
\end{itemize}
\begin{itemize}
\item {Utilização:deprec.}
\end{itemize}
\begin{itemize}
\item {Utilização:Pop.}
\end{itemize}
Exercício das funcções de Ministro de Estado.
\section{Ministril}
\begin{itemize}
\item {Grp. gram.:m.}
\end{itemize}
\begin{itemize}
\item {Utilização:Ant.}
\end{itemize}
Tocador de instrumentos de sôpro.
(Cp. \textunderscore menestrel\textunderscore )
\section{Ministro}
\begin{itemize}
\item {Grp. gram.:m.}
\end{itemize}
\begin{itemize}
\item {Proveniência:(Do lat. \textunderscore minister\textunderscore )}
\end{itemize}
Aquelle que tem um cargo, ou está incumbido de uma funcção.
Auxiliar.
Executor.
Membro de um Ministério ou chefe de uma Secretaria de Estado.
Secretário de Estado.
Enviado de um Govêrno junto do Govêrno de outra nação.
Padre.
Passarinho canoro, azul-ferrete.
\section{Minjolo}
\begin{itemize}
\item {Grp. gram.:m.}
\end{itemize}
O mesmo que \textunderscore munjolo\textunderscore .
\section{Minoração}
\begin{itemize}
\item {Grp. gram.:f.}
\end{itemize}
\begin{itemize}
\item {Proveniência:(Lat. \textunderscore minoratio\textunderscore )}
\end{itemize}
Acto ou effeito de minorar.
\section{Minorar}
\begin{itemize}
\item {Grp. gram.:v. t.}
\end{itemize}
\begin{itemize}
\item {Proveniência:(Lat. \textunderscore minorare\textunderscore )}
\end{itemize}
Tornar menor.
Abrandar, suavizar: \textunderscore o remédio minorou-lhe as dores\textunderscore .
\section{Minorativamente}
\begin{itemize}
\item {Grp. gram.:adv.}
\end{itemize}
De modo minorativo; suavemente.
\section{Minorativo}
\begin{itemize}
\item {Grp. gram.:adj.}
\end{itemize}
\begin{itemize}
\item {Grp. gram.:M.}
\end{itemize}
Suave, (falando-se de purgantes).
Que minora.
Laxante.
\section{Minoria}
\begin{itemize}
\item {Grp. gram.:f.}
\end{itemize}
\begin{itemize}
\item {Proveniência:(Do lat. \textunderscore minor\textunderscore )}
\end{itemize}
Inferioridade em número.
A parte menos numerosa do uma corporação deliberativa, e que sustenta ideias oppostas ás do maior número.
\section{Minoridade}
\begin{itemize}
\item {Grp. gram.:f.}
\end{itemize}
O mesmo que \textunderscore menoridade\textunderscore . Cf. Herculano, \textunderscore Hist. de Port.\textunderscore , I, 100.
\section{Minorista}
\begin{itemize}
\item {Grp. gram.:m.}
\end{itemize}
Clérigo de Ordens menores.
\section{Menoritas}
\begin{itemize}
\item {Grp. gram.:m. pl.}
\end{itemize}
O mesmo que \textunderscore menoretas\textunderscore . Cf. Herculano, \textunderscore Hist. de Port.\textunderscore , II, 234 e 236.
\section{Minorquino}
\begin{itemize}
\item {Grp. gram.:adj.}
\end{itemize}
\begin{itemize}
\item {Grp. gram.:M.}
\end{itemize}
Relativo á ilha de Minorca.
Habitante de Minorca.
\section{Minotaurização}
\begin{itemize}
\item {Grp. gram.:f.}
\end{itemize}
Acto ou effeito de minotaurizar.
\section{Minotaurizado}
\begin{itemize}
\item {Grp. gram.:adj.}
\end{itemize}
Tornado semelhante ao Minotauro da Mythologia; corneado. Cf. Camillo, \textunderscore Vinho do Porto\textunderscore , 64.
\section{Minautorizar}
\begin{itemize}
\item {Grp. gram.:v. t.}
\end{itemize}
Tornar semelhante a Minotauro; pôr cornos em. Cf. Camillo, \textunderscore Cav. em Ruínas\textunderscore , 186.
\section{Minotauro}
\begin{itemize}
\item {Grp. gram.:m.}
\end{itemize}
\begin{itemize}
\item {Proveniência:(De \textunderscore Minotauro\textunderscore , n. p. myth.)}
\end{itemize}
Indivíduo, a quem a mulher é infiel; cabrão.
\section{Minuano}
\begin{itemize}
\item {Grp. gram.:m.}
\end{itemize}
\begin{itemize}
\item {Utilização:Bras. do S}
\end{itemize}
\begin{itemize}
\item {Grp. gram.:Pl.}
\end{itemize}
Vento frio e sêco que sopra de Léste.
Antiga nação de Índios do Brasil, nas margens da lagôa dos Patos.
\section{Minúcia}
\begin{itemize}
\item {Grp. gram.:f.}
\end{itemize}
\begin{itemize}
\item {Proveniência:(Lat. \textunderscore minutia\textunderscore )}
\end{itemize}
Coisa muito miúda.
Bagatela; insignificância.
Particularidade.
\section{Minuciosamente}
\begin{itemize}
\item {Grp. gram.:adv.}
\end{itemize}
De modo minucioso; por miúdo: \textunderscore narrar minuciosamente\textunderscore .
\section{Minuciosidade}
\begin{itemize}
\item {Grp. gram.:f.}
\end{itemize}
Qualidade de minucioso; minúcia, pormenór.
\section{Minucioso}
\begin{itemize}
\item {Grp. gram.:adj.}
\end{itemize}
Que trata de minúcias.
Narrado circunstanciadamente ou por miúdo.
Feito escrupulosamente, com toda a attenção.
\section{Minudência}
\begin{itemize}
\item {Grp. gram.:f.}
\end{itemize}
\begin{itemize}
\item {Utilização:Fig.}
\end{itemize}
\begin{itemize}
\item {Proveniência:(Do lat. \textunderscore minutus\textunderscore )}
\end{itemize}
O mesmo que \textunderscore minúcia\textunderscore .
Observação escrupulosa, exame attento.
\section{Minudenciar}
\begin{itemize}
\item {Grp. gram.:v. t.}
\end{itemize}
\begin{itemize}
\item {Proveniência:(De \textunderscore minudência\textunderscore )}
\end{itemize}
Expor minuciosamente. Cf. \textunderscore Techn. Rur.\textunderscore , 596.
\section{Minudencioso}
\begin{itemize}
\item {Grp. gram.:adj.}
\end{itemize}
Que emprega minudências; em que há minudências; minucioso. Cf. Camillo, \textunderscore Noites de Insómn.\textunderscore , II, 89.
\section{Minudente}
\begin{itemize}
\item {Grp. gram.:adj.}
\end{itemize}
\begin{itemize}
\item {Proveniência:(De \textunderscore minudência\textunderscore )}
\end{itemize}
O mesmo que \textunderscore minudencioso\textunderscore . Cf. Ortigão, \textunderscore Praias\textunderscore , 17 e 67.
\section{Minuendo}
\begin{itemize}
\item {Grp. gram.:adj.}
\end{itemize}
\begin{itemize}
\item {Utilização:Bras}
\end{itemize}
O mesmo que \textunderscore deminuendo\textunderscore .
\section{Minuête}
\begin{itemize}
\item {Grp. gram.:m.}
\end{itemize}
\begin{itemize}
\item {Grp. gram.:Pl.}
\end{itemize}
\begin{itemize}
\item {Utilização:Prov.}
\end{itemize}
\begin{itemize}
\item {Utilização:trasm.}
\end{itemize}
\begin{itemize}
\item {Proveniência:(It. \textunderscore minuetto\textunderscore )}
\end{itemize}
Antiga dança, elegante e simples.
Música, que acompanhava essa dança.
Trecho musical, em compasso ternário e andamento vagaroso.
Negaças.
\section{Minuir}
\begin{itemize}
\item {Grp. gram.:adj.}
\end{itemize}
\begin{itemize}
\item {Proveniência:(Lat. \textunderscore minuere\textunderscore )}
\end{itemize}
O mesmo que \textunderscore deminuir\textunderscore .
\section{Minúsculo}
\begin{itemize}
\item {Grp. gram.:adj.}
\end{itemize}
\begin{itemize}
\item {Proveniência:(Lat. \textunderscore minusculus\textunderscore )}
\end{itemize}
Pequeno; que tem pequena extensão ou pequena fórma: \textunderscore letras minúsculas\textunderscore .
Que tem pequeno valor; insignificante: \textunderscore questões minúsculas\textunderscore .
\section{Minuta}
\begin{itemize}
\item {Grp. gram.:f.}
\end{itemize}
\begin{itemize}
\item {Proveniência:(Do lat. \textunderscore minutus\textunderscore )}
\end{itemize}
Rascunho.
Primeira redacção de um documento ou de qualquer escrito.
No levantamento do plantas, desenho traçado á vista do terreno.
\section{Minutador}
\begin{itemize}
\item {Grp. gram.:m.  e  adj.}
\end{itemize}
O que minuta.
\section{Minutar}
\begin{itemize}
\item {Grp. gram.:v. t.}
\end{itemize}
Fazer ou ditar a minuta de: \textunderscore minutar um requerimento\textunderscore .
\section{Minutíssimo}
\begin{itemize}
\item {Grp. gram.:adj.}
\end{itemize}
\begin{itemize}
\item {Proveniência:(Do lat. \textunderscore minutus\textunderscore )}
\end{itemize}
Muito miúdo.
Muito minucioso. Cf. \textunderscore Panorama\textunderscore , III, 77.
\section{Minuto}
\begin{itemize}
\item {Grp. gram.:m.}
\end{itemize}
\begin{itemize}
\item {Grp. gram.:Adj.}
\end{itemize}
\begin{itemize}
\item {Proveniência:(Lat. \textunderscore minutus\textunderscore )}
\end{itemize}
Sexagésima parte de uma hora.
Sexagésima parte do um grau.
Curto espaço de tempo; momento, instante.
Subdivisão da cabeça humana, em pintura, para se regularem as proporções de uma figura.
A 12.^a, a 18.^a ou a 30.^a parte do módulo, em architectura.
Moéda judaica, de pouco valor, usada em tempo de Christo. Cf. \textunderscore Luz e Calor\textunderscore , 353.
O mesmo que \textunderscore deminuto\textunderscore , reduzido, muito pequeno.
\section{Minutor}
\begin{itemize}
\item {Grp. gram.:m.}
\end{itemize}
Aquelle que faz as minutas na chancellaria apostólica.
\section{Miniantina}
\begin{itemize}
\item {Grp. gram.:f.}
\end{itemize}
Substância, extraida do minianto.
\section{Minianto}
\begin{itemize}
\item {Grp. gram.:m.}
\end{itemize}
\begin{itemize}
\item {Proveniência:(Lat. \textunderscore minyanthes\textunderscore )}
\end{itemize}
Trevo aquático, (\textunderscore trifolium fibrinum\textunderscore ).
\section{Minyanthina}
\begin{itemize}
\item {Grp. gram.:f.}
\end{itemize}
Substância, extrahida do myniantho.
\section{Minyantho}
\begin{itemize}
\item {Grp. gram.:m.}
\end{itemize}
\begin{itemize}
\item {Proveniência:(Lat. \textunderscore minyanthes\textunderscore )}
\end{itemize}
Trevo aquático, (\textunderscore trifolium fibrinum\textunderscore ).
\section{Mio}
\begin{itemize}
\item {Grp. gram.:m.}
\end{itemize}
\begin{itemize}
\item {Proveniência:(T. \textunderscore onom.\textunderscore )}
\end{itemize}
Grito do gato; miadela.
\section{Mioca}
\begin{itemize}
\item {Grp. gram.:f.}
\end{itemize}
\begin{itemize}
\item {Utilização:Prov.}
\end{itemize}
\begin{itemize}
\item {Utilização:trasm.}
\end{itemize}
O mesmo que \textunderscore minhoca\textunderscore .
\section{Miocênico}
\begin{itemize}
\item {Grp. gram.:adj.}
\end{itemize}
O mesmo que \textunderscore mioceno\textunderscore .
\section{Mioceno}
\begin{itemize}
\item {Grp. gram.:adj.}
\end{itemize}
\begin{itemize}
\item {Utilização:Geol.}
\end{itemize}
\begin{itemize}
\item {Proveniência:(Do gr. \textunderscore meion\textunderscore  + \textunderscore kainos\textunderscore )}
\end{itemize}
Diz-se do terreno fossilífero, sobreposto ao eoceno, e que contem menor porção de conchas recentes do que o plioceno.
\section{Miógono}
\begin{itemize}
\item {Grp. gram.:adj.}
\end{itemize}
\begin{itemize}
\item {Utilização:Miner.}
\end{itemize}
\begin{itemize}
\item {Proveniência:(Do gr. \textunderscore meion\textunderscore  + \textunderscore gonos\textunderscore )}
\end{itemize}
Diz-se da substância crystallizada em prismas, cujas faces se inclinam de maneira, que os ângulos formados por ellas vão successivamente deminuindo.
\section{Miolada}
\begin{itemize}
\item {Grp. gram.:f.}
\end{itemize}
\begin{itemize}
\item {Utilização:Pop.}
\end{itemize}
Miolos.
Preparado culinário, em que entram miolos.
\section{Mioleira}
\begin{itemize}
\item {Grp. gram.:f.}
\end{itemize}
\begin{itemize}
\item {Utilização:Pop.}
\end{itemize}
\begin{itemize}
\item {Utilização:Fig.}
\end{itemize}
Miolos.
Tino, juizo.
\section{Miolha}
\begin{itemize}
\item {Grp. gram.:f.}
\end{itemize}
\begin{itemize}
\item {Utilização:T. de Miranda}
\end{itemize}
O mesmo que \textunderscore medulla\textunderscore .
\section{Miolo}
\begin{itemize}
\item {fónica:ô}
\end{itemize}
\begin{itemize}
\item {Grp. gram.:m.}
\end{itemize}
\begin{itemize}
\item {Utilização:Ext.}
\end{itemize}
\begin{itemize}
\item {Utilização:Fig.}
\end{itemize}
\begin{itemize}
\item {Proveniência:(Do lat. \textunderscore medulla\textunderscore )}
\end{itemize}
Parte do pão, que fica dentro da côdea.
A parte interior de alguns frutos.
Polpa; medulla.
Cérebro.
A parte interior de alguma coisa.
A essência, o principal.
Juizo.
\section{Mioloso}
\begin{itemize}
\item {Grp. gram.:adj.}
\end{itemize}
\begin{itemize}
\item {Proveniência:(De \textunderscore miolo\textunderscore )}
\end{itemize}
Abundante em medulla, (falando-se de vegetaes).
\section{Mioludo}
\begin{itemize}
\item {Grp. gram.:adj.}
\end{itemize}
O mesmo que \textunderscore mioloso\textunderscore .
\section{Mioprasia}
\begin{itemize}
\item {Grp. gram.:f.}
\end{itemize}
\begin{itemize}
\item {Utilização:Physiol.}
\end{itemize}
Inferioridade funccional de um órgão.
\section{Mioto}
\begin{itemize}
\item {fónica:ô}
\end{itemize}
\begin{itemize}
\item {Grp. gram.:m.}
\end{itemize}
Nome de algumas espécies de milhanos.
\section{Mique}
\begin{itemize}
\item {Grp. gram.:m.  ou  pron.}
\end{itemize}
\begin{itemize}
\item {Utilização:Ant.}
\end{itemize}
Cp. \textunderscore chique\textunderscore ^2?
\section{Miquelete}
\begin{itemize}
\item {fónica:lê}
\end{itemize}
\begin{itemize}
\item {Grp. gram.:m.}
\end{itemize}
Antigamente, bandido dos Pyrenéus.
Hoje, soldado da guarda dos governadores das províncias de Espanha.
(Cast. \textunderscore miquelete\textunderscore )
\section{Mir}
\begin{itemize}
\item {Grp. gram.:m.}
\end{itemize}
\begin{itemize}
\item {Utilização:Ant.}
\end{itemize}
O mesmo que \textunderscore emir\textunderscore . Cf. J. Sousa, \textunderscore Vestígios\textunderscore .
\section{Mira}
\begin{itemize}
\item {Grp. gram.:f.}
\end{itemize}
Appêndice metállico, na extremidade do cano de algumas armas do fogo, e pelo qual se dirige a vista ao alvo.
Acto de mirar.
Desejo.
Íntuito, fim: \textunderscore com a mira do interesse\textunderscore .
Instrumento de Mathemática. Cf. \textunderscore Jorn.-do-Comm.\textunderscore , do Rio, de 7-VI-902.
\section{Mirabanda}
\begin{itemize}
\item {Grp. gram.:f.}
\end{itemize}
Espécie de moscardo brasileiro.
\section{Mirabela}
\begin{itemize}
\item {Grp. gram.:f.}
\end{itemize}
\begin{itemize}
\item {Proveniência:(Do fr. \textunderscore mirabelle\textunderscore )}
\end{itemize}
Planta chenopodiacea, (\textunderscore chenopodiums coparia\textunderscore ).
\section{Mirabella}
\begin{itemize}
\item {Grp. gram.:f.}
\end{itemize}
\begin{itemize}
\item {Proveniência:(Do fr. \textunderscore mirabelle\textunderscore )}
\end{itemize}
Planta chenopodiacea, (\textunderscore chenopodiums coparia\textunderscore ).
\section{Mirabolâneas}
\begin{itemize}
\item {Grp. gram.:f. pl.}
\end{itemize}
\begin{itemize}
\item {Utilização:Bot.}
\end{itemize}
\begin{itemize}
\item {Proveniência:(De \textunderscore mirabólano\textunderscore )}
\end{itemize}
O mesmo que \textunderscore combretáceas\textunderscore .
\section{Mirabólano}
\begin{itemize}
\item {Grp. gram.:m.}
\end{itemize}
Fruto medicinal, (\textunderscore terminalis belerica\textunderscore , Roxb.).
(Cp. \textunderscore mirobálano\textunderscore )
\section{Mirabolante}
\begin{itemize}
\item {Grp. gram.:adj.}
\end{itemize}
\begin{itemize}
\item {Utilização:Deprec.}
\end{itemize}
\begin{itemize}
\item {Utilização:Neol.}
\end{itemize}
Ridiculamente vistoso; espalhafatoso. Cf. Camillo, \textunderscore Ratazzi\textunderscore , 25.
\section{Miraculosamente}
\begin{itemize}
\item {Grp. gram.:adv.}
\end{itemize}
De modo miraculoso; por milagre.
\section{Miraculoso}
\begin{itemize}
\item {Grp. gram.:adj.}
\end{itemize}
\begin{itemize}
\item {Proveniência:(Do lat. \textunderscore miraculum\textunderscore )}
\end{itemize}
O mesmo que \textunderscore milagroso\textunderscore .
\section{Mirada}
\begin{itemize}
\item {Grp. gram.:f.}
\end{itemize}
Acto de mirar; olhar. Cf. Latino, \textunderscore Humboldt\textunderscore , 352.
\section{Miradoiro}
\begin{itemize}
\item {Grp. gram.:m.}
\end{itemize}
O mesmo que \textunderscore mirante\textunderscore .
\section{Miradouro}
\begin{itemize}
\item {Grp. gram.:m.}
\end{itemize}
O mesmo que \textunderscore mirante\textunderscore .
\section{Mirães}
\begin{itemize}
\item {Grp. gram.:m. pl.}
\end{itemize}
\begin{itemize}
\item {Utilização:T. de Aveiro}
\end{itemize}
Habitantes de Mira. Cf. Rev. \textunderscore Tradição\textunderscore , V, 10.
\section{Miragaia}
\begin{itemize}
\item {Grp. gram.:f.}
\end{itemize}
\begin{itemize}
\item {Utilização:Bras}
\end{itemize}
Peixe semelhante ao bacalhau.
\section{Miragaio}
\begin{itemize}
\item {Grp. gram.:m.}
\end{itemize}
\begin{itemize}
\item {Utilização:Chul.}
\end{itemize}
Homem finório, espertalhão.
\section{Miragem}
\begin{itemize}
\item {Grp. gram.:f.}
\end{itemize}
\begin{itemize}
\item {Utilização:Fig.}
\end{itemize}
\begin{itemize}
\item {Proveniência:(De \textunderscore mirar\textunderscore )}
\end{itemize}
Phenómeno de refracção, em que os objectos apresentam duas imagens, uma directa e outra invertida.
Illusão.
Engano dos sentidos.
Decepção.
\section{Miralmuminim}
\begin{itemize}
\item {Grp. gram.:m.}
\end{itemize}
O mesmo ou melhor que \textunderscore miramolim\textunderscore . Cf. \textunderscore Lusiadas\textunderscore ; Vieira, II, 49.
\section{Miramar}
\begin{itemize}
\item {Grp. gram.:m.}
\end{itemize}
Mirante, que deita para o mar. Cf. \textunderscore Port. Ant. e Mod.\textunderscore , vb. \textunderscore Leça da Palmeira\textunderscore .
\section{Miramento}
\begin{itemize}
\item {Grp. gram.:m.}
\end{itemize}
Acto de mirar; attenção.
\section{Miramolim}
\begin{itemize}
\item {Grp. gram.:m.}
\end{itemize}
Califa, ou chefe de crentes, entre os Muçulmanos.
(Corr. do ar. \textunderscore emir-almuminin\textunderscore )
\section{Miranas}
\begin{itemize}
\item {Grp. gram.:m. pl.}
\end{itemize}
O mesmo que \textunderscore miranhas\textunderscore .
\section{Mirandense}
\begin{itemize}
\item {Grp. gram.:adj.}
\end{itemize}
Relativo á cidade do Miranda; mirandês. Cf. Camillo, \textunderscore Quéda\textunderscore , 19.
\section{Mirandês}
\begin{itemize}
\item {Grp. gram.:adj.}
\end{itemize}
\begin{itemize}
\item {Grp. gram.:M.}
\end{itemize}
Relativo á cidade de Miranda.
Habitante de Miranda.
Dialecto, falado no termo de Miranda. Cf. G. Vianna, \textunderscore Classificação das Línguas\textunderscore , 10.
\section{Mirandum}
\begin{itemize}
\item {Grp. gram.:m.}
\end{itemize}
\begin{itemize}
\item {Utilização:T. de Miranda}
\end{itemize}
Espécie de dança.
\section{Miranhas}
\begin{itemize}
\item {Grp. gram.:m. pl.}
\end{itemize}
Índios selvagens das margens do Japurá, no Brasil.
\section{Mirante}
\begin{itemize}
\item {Grp. gram.:m.}
\end{itemize}
\begin{itemize}
\item {Proveniência:(De \textunderscore mirar\textunderscore )}
\end{itemize}
Ponto elevado, donde se descobre largo horizonte.
Pequena, mas elevada construcção, para gôzo de largas perspectivas.
\section{Mirão}
\begin{itemize}
\item {Grp. gram.:m.}
\end{itemize}
\begin{itemize}
\item {Utilização:Pop.}
\end{itemize}
Espectador do jôgo.
Aquelle que mira, que observa. Cf. Filinto, XII, 97.
(Talvez do cast. \textunderscore mirón\textunderscore )
\section{Mira-ôlho}
\begin{itemize}
\item {Grp. gram.:adj.}
\end{itemize}
\begin{itemize}
\item {Grp. gram.:M.}
\end{itemize}
Appetitoso, de bom aspecto.
Variedade do pêssego.
\section{Miraonde}
\begin{itemize}
\item {Grp. gram.:m.}
\end{itemize}
Árvore angolense.
\section{Mirapuama}
\begin{itemize}
\item {Grp. gram.:f.}
\end{itemize}
\begin{itemize}
\item {Utilização:Bras}
\end{itemize}
O mesmo que \textunderscore marapuama\textunderscore .
\section{Mirar}
\begin{itemize}
\item {Grp. gram.:v. t.}
\end{itemize}
\begin{itemize}
\item {Grp. gram.:V. i.}
\end{itemize}
\begin{itemize}
\item {Grp. gram.:V. p.}
\end{itemize}
\begin{itemize}
\item {Proveniência:(Lat. \textunderscore mirari\textunderscore )}
\end{itemize}
Fitar a vista em.
Encarar.
Avistar.
Espreitar.
Appetecer; aspirar a.
Apontar uma arma.
Têr em vista, formar plano.
Olhar, estar voltado para certo lado.
Rever-se, contemplar-se num espelho.
\section{Mirasol}
\begin{itemize}
\item {fónica:sol}
\end{itemize}
\begin{itemize}
\item {Grp. gram.:m.}
\end{itemize}
Planta da serra do Sintra.
\section{Mirassol}
\begin{itemize}
\item {Grp. gram.:m.}
\end{itemize}
Planta da serra do Sintra.
\section{Miratoá}
\begin{itemize}
\item {Grp. gram.:m.}
\end{itemize}
\begin{itemize}
\item {Utilização:Bras. do Amazonas}
\end{itemize}
Árvore preciosa para construcção de canôas.
\section{Mirgadeira}
\begin{itemize}
\item {Grp. gram.:f.}
\end{itemize}
\begin{itemize}
\item {Utilização:Prov.}
\end{itemize}
\begin{itemize}
\item {Utilização:trasm.}
\end{itemize}
Árvore, que dá mirgans; romanzeira.
\section{Mirgan}
\begin{itemize}
\item {Grp. gram.:f.}
\end{itemize}
\begin{itemize}
\item {Utilização:Prov.}
\end{itemize}
\begin{itemize}
\item {Utilização:trasm.}
\end{itemize}
O mesmo que \textunderscore milligran\textunderscore .
\section{Miri}
\begin{itemize}
\item {Grp. gram.:f.}
\end{itemize}
Planta sapotácea do Brasil.
\section{Miri}
\begin{itemize}
\item {Grp. gram.:m.}
\end{itemize}
Espécie de papagaio da região do Amazonas.
\section{Mirificamente}
\begin{itemize}
\item {Grp. gram.:adv.}
\end{itemize}
De modo mirífico.
\section{Mirificar}
\begin{itemize}
\item {Grp. gram.:v. t.}
\end{itemize}
\begin{itemize}
\item {Proveniência:(Lat. \textunderscore mirificare\textunderscore )}
\end{itemize}
Tornar mirífico, admirável.
Causar admiração a.
\section{Mirífico}
\begin{itemize}
\item {Grp. gram.:adj.}
\end{itemize}
\begin{itemize}
\item {Proveniência:(Lat. \textunderscore mirificus\textunderscore )}
\end{itemize}
Admirável; maravilhoso; excellente.
\section{Mirim}
\begin{itemize}
\item {Grp. gram.:m.}
\end{itemize}
\begin{itemize}
\item {Proveniência:(T. tupi)}
\end{itemize}
Planta brasileira, provavelmente o mesmo que \textunderscore miri\textunderscore ^1.
\section{...Mirim}
\begin{itemize}
\item {Grp. gram.:suf. adj.}
\end{itemize}
\begin{itemize}
\item {Utilização:Bras}
\end{itemize}
Designativo de \textunderscore pequeno\textunderscore .
\section{Mirindiba}
\begin{itemize}
\item {Grp. gram.:f.}
\end{itemize}
Árvore combretácea do Brasil.
\section{Mirinzal}
\begin{itemize}
\item {Grp. gram.:m.}
\end{itemize}
\begin{itemize}
\item {Utilização:Bras}
\end{itemize}
Matagal, em que predomina o mirim.
\section{Mirmidão}
\begin{itemize}
\item {Grp. gram.:m.}
\end{itemize}
\begin{itemize}
\item {Proveniência:(Do lat. \textunderscore Myrmidones\textunderscore , n. p.)}
\end{itemize}
Ajudante de cozinheiro.
Companheiro.
\section{Mirmilão}
\begin{itemize}
\item {Grp. gram.:m.}
\end{itemize}
\begin{itemize}
\item {Proveniência:(Lat. \textunderscore mirmillo\textunderscore )}
\end{itemize}
Gladiador, que trazia a figura de um peixe no elmo.
\section{Mirmillão}
\begin{itemize}
\item {Grp. gram.:m.}
\end{itemize}
\begin{itemize}
\item {Proveniência:(Lat. \textunderscore mirmillo\textunderscore )}
\end{itemize}
Gladiador, que trazia a figura de um peixe no elmo.
\section{Mirobalâneas}
\begin{itemize}
\item {Grp. gram.:f. pl.}
\end{itemize}
\begin{itemize}
\item {Utilização:Bot.}
\end{itemize}
O mesmo ou melhor que \textunderscore mirabolâneas\textunderscore . Cf. Benevides, \textunderscore Gloss. Bot.\textunderscore 
\section{Mirobálano}
\begin{itemize}
\item {Grp. gram.:m.}
\end{itemize}
\begin{itemize}
\item {Proveniência:(Lat. \textunderscore myrobalanum\textunderscore )}
\end{itemize}
O mesmo ou melhor que \textunderscore mirabólano\textunderscore .
Designação genérica de vários frutos sêcos, procedentes da Índia, e que se aplicavam unicamente em preparações farmacêuticas.
Cp. \textunderscore mirabólano\textunderscore .
\section{Mirolho}
\begin{itemize}
\item {fónica:mirô}
\end{itemize}
\begin{itemize}
\item {Grp. gram.:m.  e  adj.}
\end{itemize}
\begin{itemize}
\item {Utilização:Prov.}
\end{itemize}
\begin{itemize}
\item {Utilização:trasm.}
\end{itemize}
\begin{itemize}
\item {Proveniência:(De \textunderscore mirar\textunderscore  + \textunderscore ôlho\textunderscore )}
\end{itemize}
Indivíduo vesgo.
\section{Mirone}
\begin{itemize}
\item {Grp. gram.:m.}
\end{itemize}
\begin{itemize}
\item {Utilização:Fam.}
\end{itemize}
Espectador.
Aquelle que, sem jogar, observa o andamento de um jôgo; mirão. Cf. Filinto, XIII, 174.(V.mirão)
\section{Miroró}
\begin{itemize}
\item {Grp. gram.:m.}
\end{itemize}
\begin{itemize}
\item {Utilização:Bras}
\end{itemize}
Nome de duas espécies de peixe, grande e pequena, (\textunderscore miroró-mirim\textunderscore  e \textunderscore miroró-açum\textunderscore ).
\section{Mirra}
\begin{itemize}
\item {Grp. gram.:f.}
\end{itemize}
\begin{itemize}
\item {Proveniência:(Lat. \textunderscore myrrha\textunderscore )}
\end{itemize}
Planta terebinthácea das cercanias do Mar-Vermelho.
Goma resinosa desta planta.
\section{Mirra}
\begin{itemize}
\item {Grp. gram.:m.}
\end{itemize}
\begin{itemize}
\item {Utilização:Fam.}
\end{itemize}
\begin{itemize}
\item {Utilização:Fig.}
\end{itemize}
\begin{itemize}
\item {Proveniência:(De \textunderscore mirrar\textunderscore ^2)}
\end{itemize}
Magrizela.
Homem mesquinho, avarento.
\section{Mirrado}
\begin{itemize}
\item {Grp. gram.:adj.}
\end{itemize}
\begin{itemize}
\item {Proveniência:(De \textunderscore mirrar\textunderscore ^2)}
\end{itemize}
Murcho; sêco: \textunderscore flores mirradas\textunderscore .
Magro; definhado: \textunderscore cara mirrada\textunderscore .
\section{Mirrador}
\begin{itemize}
\item {Grp. gram.:adj.}
\end{itemize}
Que mirra.
\section{Mirrar}
\begin{itemize}
\item {Grp. gram.:v. t.}
\end{itemize}
Preparar com mirra^1.
\section{Mirrar}
\begin{itemize}
\item {Grp. gram.:v. t.}
\end{itemize}
\begin{itemize}
\item {Grp. gram.:V. i.}
\end{itemize}
Tornar sêco, definhado.
Tornar magro.
Gastar.
Secar-se, perder o viço.
Perder a energia.
Humilhar-se.
Desapparecer.
Fugir, esconder-se. Cf. Garrett, \textunderscore Arco\textunderscore , II, 203.
\section{Mirrastes}
\begin{itemize}
\item {Grp. gram.:m. pl.}
\end{itemize}
Môlho de amêndoas pisadas.
\section{Mírreo}
\begin{itemize}
\item {Grp. gram.:adj.}
\end{itemize}
\begin{itemize}
\item {Utilização:Poét.}
\end{itemize}
\begin{itemize}
\item {Proveniência:(Lat. \textunderscore myrrheus\textunderscore )}
\end{itemize}
Perfumado de mirra.
\section{Mirro}
\begin{itemize}
\item {Grp. gram.:adj.}
\end{itemize}
\begin{itemize}
\item {Utilização:Des.}
\end{itemize}
\begin{itemize}
\item {Proveniência:(De \textunderscore mirrar\textunderscore ^2)}
\end{itemize}
O mesmo que \textunderscore mirrado\textunderscore .
Sêco.
Esgotado:«\textunderscore ...tendes a bolsa mirra.\textunderscore »Castilho, \textunderscore Fastos\textunderscore , II, 95;«\textunderscore já a bolsa andava mirra.\textunderscore »Filinto, VII, 214.
\section{Mírtil}
\begin{itemize}
\item {Grp. gram.:m.}
\end{itemize}
Insecto lepidóptero, (\textunderscore satyrus janire\textunderscore ).
\section{Misanthropia}
\begin{itemize}
\item {Grp. gram.:f.}
\end{itemize}
\begin{itemize}
\item {Utilização:Pop.}
\end{itemize}
Qualidade de quem é misanthropo.
Melancolia.
\section{Misanthrópico}
\begin{itemize}
\item {Grp. gram.:adj.}
\end{itemize}
Relativo á misanthropia.
\section{Misanthropo}
\begin{itemize}
\item {fónica:trô}
\end{itemize}
\begin{itemize}
\item {Grp. gram.:m.}
\end{itemize}
\begin{itemize}
\item {Utilização:Pop.}
\end{itemize}
\begin{itemize}
\item {Grp. gram.:Adj.}
\end{itemize}
\begin{itemize}
\item {Proveniência:(Gr. \textunderscore misanthropos\textunderscore )}
\end{itemize}
Aquelle que tem ódio á sociedade.
Aquelle que evita convivência.
Homem melancólico.
Misanthrópico.
\section{Misantropia}
\begin{itemize}
\item {Grp. gram.:f.}
\end{itemize}
\begin{itemize}
\item {Utilização:Pop.}
\end{itemize}
Qualidade de quem é misantropo.
Melancolia.
\section{Misantrópico}
\begin{itemize}
\item {Grp. gram.:adj.}
\end{itemize}
Relativo á misantropia.
\section{Misantropo}
\begin{itemize}
\item {fónica:trô}
\end{itemize}
\begin{itemize}
\item {Grp. gram.:m.}
\end{itemize}
\begin{itemize}
\item {Utilização:Pop.}
\end{itemize}
\begin{itemize}
\item {Grp. gram.:Adj.}
\end{itemize}
\begin{itemize}
\item {Proveniência:(Gr. \textunderscore misanthropos\textunderscore )}
\end{itemize}
Aquele que tem ódio á sociedade.
Aquele que evita convivência.
Homem melancólico.
Misantrópico.
\section{Miscambilha}
\textunderscore f. T. de Lanhoso\textunderscore , (e der.)
(V. \textunderscore mescambilha\textunderscore , etc.)
\section{Miscandilhas}
\begin{itemize}
\item {Grp. gram.:f. pl.}
\end{itemize}
\begin{itemize}
\item {Utilização:Prov.}
\end{itemize}
\begin{itemize}
\item {Utilização:trasm.}
\end{itemize}
Ninharias, bagatelas.
(Relaciona-se com \textunderscore mescambilha\textunderscore ?)
\section{Míscaro}
\begin{itemize}
\item {Grp. gram.:m.}
\end{itemize}
Espécie de cogumelo amarelo e comestível, que nasce habitualmente nos pinheiraes.
(Cast. \textunderscore miscalo\textunderscore )
\section{Miscar-se}
\begin{itemize}
\item {Grp. gram.:v. p.}
\end{itemize}
\begin{itemize}
\item {Utilização:Gír.}
\end{itemize}
Safar-se; esgueirar-se; escapulir-se.
\section{Miscelânea}
\begin{itemize}
\item {Grp. gram.:f.}
\end{itemize}
\begin{itemize}
\item {Utilização:Fig.}
\end{itemize}
\begin{itemize}
\item {Proveniência:(Lat. \textunderscore miscellanea\textunderscore )}
\end{itemize}
Compilação de várias peças literárias.
Confusão; mistifório.
\section{Miscellânea}
\begin{itemize}
\item {Grp. gram.:f.}
\end{itemize}
\begin{itemize}
\item {Utilização:Fig.}
\end{itemize}
\begin{itemize}
\item {Proveniência:(Lat. \textunderscore miscellanea\textunderscore )}
\end{itemize}
Compilação de várias peças literárias.
Confusão; mistifório.
\section{Mester}
\begin{itemize}
\item {Grp. gram.:m.}
\end{itemize}
\begin{itemize}
\item {Proveniência:(Do lat. \textunderscore ministerium\textunderscore )}
\end{itemize}
Urgência; precisão.
Aquillo que é forçoso:«\textunderscore faz-se mester muita caridade\textunderscore ». Camillo, \textunderscore Retr. de Ricard.\textunderscore , 48.
\section{Miscibilidade}
\begin{itemize}
\item {Grp. gram.:f.}
\end{itemize}
Qualidade daquillo que é miscível.
\section{Miscível}
\begin{itemize}
\item {Grp. gram.:adj.}
\end{itemize}
\begin{itemize}
\item {Proveniência:(Do lat. \textunderscore miscere\textunderscore )}
\end{itemize}
Que se póde misturar.
\section{Miscrar}
\textunderscore v. t.\textunderscore  (e der.)
O mesmo que \textunderscore mesclar\textunderscore , etc. Cf. Sous. Monteiro, \textunderscore Elog. de Latino\textunderscore .
\section{Miscro}
\begin{itemize}
\item {Grp. gram.:m.}
\end{itemize}
\begin{itemize}
\item {Utilização:Prov.}
\end{itemize}
\begin{itemize}
\item {Utilização:trasm.}
\end{itemize}
O mesmo que \textunderscore míscaro\textunderscore .
\section{Mísera}
\begin{itemize}
\item {Grp. gram.:f.}
\end{itemize}
\begin{itemize}
\item {Proveniência:(De \textunderscore mísero\textunderscore )}
\end{itemize}
Mulher infeliz:«\textunderscore a mísera e mesquinha...\textunderscore »\textunderscore Lusíadas\textunderscore .
\section{Miserabilismo}
\begin{itemize}
\item {Grp. gram.:m.}
\end{itemize}
\begin{itemize}
\item {Utilização:Neol.}
\end{itemize}
\begin{itemize}
\item {Proveniência:(Do lat. \textunderscore miserabilis\textunderscore )}
\end{itemize}
Estado de miserável. Cf. Alves Mendes, \textunderscore Discursos\textunderscore , 279.
\section{Miserabilíssimo}
\begin{itemize}
\item {Grp. gram.:adj.}
\end{itemize}
\begin{itemize}
\item {Proveniência:(Do lat. \textunderscore miserabilis\textunderscore )}
\end{itemize}
Muito miserável.
\section{Miseração}
\begin{itemize}
\item {Grp. gram.:f.}
\end{itemize}
\begin{itemize}
\item {Proveniência:(Lat. \textunderscore miseratio\textunderscore )}
\end{itemize}
O mesmo que \textunderscore commiseração\textunderscore .
\section{Miserado}
\begin{itemize}
\item {Grp. gram.:m.}
\end{itemize}
\begin{itemize}
\item {Utilização:Ant.}
\end{itemize}
\begin{itemize}
\item {Proveniência:(De \textunderscore miserar\textunderscore )}
\end{itemize}
Desacreditado, calumniado. Cf. \textunderscore Port. Mon. Hist.\textunderscore , \textunderscore Script.\textunderscore , 282.
\section{Miseramente}
\begin{itemize}
\item {Grp. gram.:adv.}
\end{itemize}
\begin{itemize}
\item {Proveniência:(De \textunderscore misero\textunderscore )}
\end{itemize}
O mesmo que \textunderscore miseravelmente\textunderscore .
\section{Miserandamente}
\begin{itemize}
\item {Grp. gram.:adv.}
\end{itemize}
De modo miserando. Cf. Camillo, \textunderscore Cav. em Ruinas\textunderscore , 82.
\section{Miserando}
\begin{itemize}
\item {Grp. gram.:adj.}
\end{itemize}
\begin{itemize}
\item {Proveniência:(Lat. \textunderscore miserandus\textunderscore )}
\end{itemize}
Digno de commiseração; lastimável; deplorável.
\section{Miserar}
\begin{itemize}
\item {Grp. gram.:v. t.}
\end{itemize}
\begin{itemize}
\item {Utilização:Neol.}
\end{itemize}
Tornar mísero, desgraçar:«\textunderscore trabalhou o mau fado em miserar-me.\textunderscore »Filinto, VIII, 74. Cf. Sousa, \textunderscore Vida do Arceb.\textunderscore , III, 258.
\section{Miserável}
\begin{itemize}
\item {Grp. gram.:adj.}
\end{itemize}
\begin{itemize}
\item {Grp. gram.:M.  e  f.}
\end{itemize}
\begin{itemize}
\item {Proveniência:(Lat. \textunderscore miserabilis\textunderscore )}
\end{itemize}
Digno de compaixão; lastimoso; miserando.
Desprezível.
Malvado.
Avarento.
Pessôa desgraçada; indigente.
Pessôa infame, digna de aversão.
Avarento.
\section{Miseravelmente}
\begin{itemize}
\item {Grp. gram.:adv.}
\end{itemize}
De modo miserável.
\section{Miserere}
\begin{itemize}
\item {Grp. gram.:m.}
\end{itemize}
\begin{itemize}
\item {Utilização:Med.}
\end{itemize}
\begin{itemize}
\item {Proveniência:(T. lat.)}
\end{itemize}
Designação de um psalmo, que começa por esta palavra.
Peça de cantochão ou música, composta sôbre as palavras daquelle psalmo.
Volvo.
\section{Miséria}
\begin{itemize}
\item {Grp. gram.:f.}
\end{itemize}
\begin{itemize}
\item {Proveniência:(Lat. \textunderscore miseria\textunderscore )}
\end{itemize}
Estado que inspira compaixão.
Indigência; penúria: \textunderscore há muita miséria por êsses bêcos\textunderscore .
Estado vergonhoso, indecoroso.
Avareza.
Insignificância, bagatela: \textunderscore êste prédio custou uma miséria\textunderscore .
Imperfeição, própria da natureza do homem ou das suas obras.
\section{Miséria}
\begin{itemize}
\item {Grp. gram.:f.}
\end{itemize}
Árvore de Cabo-Verde.
\section{Misericórdia}
\begin{itemize}
\item {Grp. gram.:f.}
\end{itemize}
\begin{itemize}
\item {Utilização:Fig.}
\end{itemize}
\begin{itemize}
\item {Proveniência:(Lat. \textunderscore misericordia\textunderscore )}
\end{itemize}
Compaixão, despertada pela miséria alheia.
Perdão.
Instituição de piedade e caridade.
Punhal, que os cavalleiros usavam, do lado opposto ao da espada, e com que matavam o adversário derribado, se êste não pedia misericórdia.
Exclamação de quem pede soccôrro ou compaixão.
\textunderscore Bandeira da misericórdia\textunderscore , pessôa sempre disposta a desculpar quaesquer erros ou defeitos de outrem.
\section{Misericordiador}
\begin{itemize}
\item {Grp. gram.:adj.}
\end{itemize}
\begin{itemize}
\item {Utilização:P. us.}
\end{itemize}
O mesmo que \textunderscore misericordioso\textunderscore .
\section{Misericordiosamente}
\begin{itemize}
\item {Grp. gram.:adv.}
\end{itemize}
De modo misericordioso; compassivamente.
\section{Misericordioso}
\begin{itemize}
\item {Grp. gram.:adj.}
\end{itemize}
\begin{itemize}
\item {Grp. gram.:M.}
\end{itemize}
\begin{itemize}
\item {Proveniência:(Do lat. \textunderscore misericords\textunderscore )}
\end{itemize}
Que tem misericórdia; compassivo.
Aquelle que perdôa o mal que lhe fazem.
\section{Mísero}
\begin{itemize}
\item {Grp. gram.:adj.}
\end{itemize}
\begin{itemize}
\item {Utilização:Fig.}
\end{itemize}
\begin{itemize}
\item {Grp. gram.:M.}
\end{itemize}
\begin{itemize}
\item {Proveniência:(Lat. \textunderscore miser\textunderscore )}
\end{itemize}
Desgraçado, miserável.
Escasso, avarento.
Indivíduo infeliz.
\section{Misérrimo}
\begin{itemize}
\item {Grp. gram.:adj.}
\end{itemize}
\begin{itemize}
\item {Proveniência:(Lat. \textunderscore miserrimus\textunderscore )}
\end{itemize}
Muito mísero.
\section{Misofobia}
\begin{itemize}
\item {Grp. gram.:f.}
\end{itemize}
\begin{itemize}
\item {Proveniência:(Do gr. \textunderscore misein\textunderscore  + \textunderscore phobein\textunderscore )}
\end{itemize}
Medo mórbido dos contactos.
\section{Misófobo}
\begin{itemize}
\item {Grp. gram.:m.}
\end{itemize}
Aquele que tem misofobia.
\section{Misogamia}
\begin{itemize}
\item {Grp. gram.:f.}
\end{itemize}
Qualidade ou estado de misógamo.
\section{Misógamo}
\begin{itemize}
\item {Grp. gram.:m.}
\end{itemize}
\begin{itemize}
\item {Proveniência:(Do gr. \textunderscore misein\textunderscore  + \textunderscore gamos\textunderscore )}
\end{itemize}
Aquelle que tem horror ao casamento.
\section{Misoginia}
\begin{itemize}
\item {Grp. gram.:f.}
\end{itemize}
\begin{itemize}
\item {Utilização:Med.}
\end{itemize}
Repulsão mórbida do homem para as relações sexuaes.
(Cp. \textunderscore misógino\textunderscore )
\section{Misógino}
\begin{itemize}
\item {Grp. gram.:adj.}
\end{itemize}
\begin{itemize}
\item {Proveniência:(Do gr. \textunderscore misein\textunderscore  + \textunderscore gune\textunderscore )}
\end{itemize}
Que tem misoginia.
\section{Misogynia}
\begin{itemize}
\item {Grp. gram.:f.}
\end{itemize}
\begin{itemize}
\item {Utilização:Med.}
\end{itemize}
Repulsão mórbida do homem para as relações sexuaes.
(Cp. \textunderscore misógyno\textunderscore )
\section{Misógyno}
\begin{itemize}
\item {Grp. gram.:adj.}
\end{itemize}
\begin{itemize}
\item {Proveniência:(Do gr. \textunderscore misein\textunderscore  + \textunderscore gune\textunderscore )}
\end{itemize}
Que tem misogynia.
\section{Misologia}
\begin{itemize}
\item {Grp. gram.:f.}
\end{itemize}
\begin{itemize}
\item {Proveniência:(Do gr. \textunderscore misein\textunderscore  + \textunderscore logos\textunderscore )}
\end{itemize}
Aversão ao raciocinio, ás sciências.
\section{Misólogo}
\begin{itemize}
\item {Grp. gram.:m.}
\end{itemize}
Aquelle que tem misologia.
\section{Misoneico}
\begin{itemize}
\item {Grp. gram.:adj.}
\end{itemize}
\begin{itemize}
\item {Utilização:Neol.}
\end{itemize}
Hostil a invenções ou novidades.
(Cp. \textunderscore misoneísmo\textunderscore )
\section{Misoneísmo}
\begin{itemize}
\item {Grp. gram.:m.}
\end{itemize}
\begin{itemize}
\item {Utilização:Neol.}
\end{itemize}
\begin{itemize}
\item {Proveniência:(Do gr. \textunderscore misein\textunderscore  + \textunderscore neos\textunderscore )}
\end{itemize}
Aversão a tudo que é novo.
\section{Misoneísta}
\begin{itemize}
\item {Grp. gram.:m.  e  adj.}
\end{itemize}
Partidário do misoneísmo.
\section{Misopedia}
\begin{itemize}
\item {Grp. gram.:f.}
\end{itemize}
\begin{itemize}
\item {Proveniência:(Do gr. \textunderscore misein\textunderscore  + \textunderscore pais\textunderscore , \textunderscore paidos\textunderscore )}
\end{itemize}
Ódio mórbido ás crianças ou aos próprios filhos.
\section{Misophobia}
\begin{itemize}
\item {Grp. gram.:f.}
\end{itemize}
\begin{itemize}
\item {Proveniência:(Do gr. \textunderscore misein\textunderscore  + \textunderscore phobein\textunderscore )}
\end{itemize}
Medo mórbido dos contactos.
\section{Misóphobo}
\begin{itemize}
\item {Grp. gram.:m.}
\end{itemize}
Aquelle que tem misophobia.
\section{Misosophia}
\begin{itemize}
\item {fónica:zosso}
\end{itemize}
\begin{itemize}
\item {Grp. gram.:f.}
\end{itemize}
\begin{itemize}
\item {Proveniência:(Do gr. \textunderscore misein\textunderscore  + \textunderscore sophos\textunderscore )}
\end{itemize}
Aversão ao saber; o mesmo que \textunderscore misologia\textunderscore .
\section{Misósopho}
\begin{itemize}
\item {fónica:zosso}
\end{itemize}
\begin{itemize}
\item {Grp. gram.:m.}
\end{itemize}
Aquelle que tem misosophia.
\section{Misossofia}
\begin{itemize}
\item {Grp. gram.:f.}
\end{itemize}
\begin{itemize}
\item {Proveniência:(Do gr. \textunderscore misein\textunderscore  + \textunderscore sophos\textunderscore )}
\end{itemize}
Aversão ao saber; o mesmo que \textunderscore misologia\textunderscore .
\section{Misóssofo}
\begin{itemize}
\item {Grp. gram.:m.}
\end{itemize}
Aquele que tem misosofia.
\section{Misraim}
\begin{itemize}
\item {Grp. gram.:m.}
\end{itemize}
Rito egýpcio da maçonaria.
\section{Missa}
\begin{itemize}
\item {Grp. gram.:f.}
\end{itemize}
\begin{itemize}
\item {Utilização:Ant.}
\end{itemize}
\begin{itemize}
\item {Utilização:Ant.}
\end{itemize}
\begin{itemize}
\item {Grp. gram.:Loc.}
\end{itemize}
\begin{itemize}
\item {Utilização:ant.}
\end{itemize}
\begin{itemize}
\item {Grp. gram.:Loc.}
\end{itemize}
\begin{itemize}
\item {Utilização:ant.}
\end{itemize}
\begin{itemize}
\item {Grp. gram.:Loc.}
\end{itemize}
\begin{itemize}
\item {Utilização:des.}
\end{itemize}
\begin{itemize}
\item {Grp. gram.:Loc.}
\end{itemize}
\begin{itemize}
\item {Utilização:des.}
\end{itemize}
\begin{itemize}
\item {Grp. gram.:Loc.}
\end{itemize}
\begin{itemize}
\item {Utilização:ant.}
\end{itemize}
\begin{itemize}
\item {Grp. gram.:Loc.}
\end{itemize}
\begin{itemize}
\item {Utilização:ant.}
\end{itemize}
\begin{itemize}
\item {Proveniência:(Lat. \textunderscore missa\textunderscore )}
\end{itemize}
Acto solenne, com que a Igreja celebra o sacrifício de Christo pelos homens.
Qualquer festividade religiosa.
Romaria ou feira, que se faz juntamente com uma festa religiosa.
\textunderscore Missa dos pobres\textunderscore , esmola que se repartia pelos pobres, nos adros das igrejas.
\textunderscore Missa psaltério\textunderscore , conjunto de psalmos e orações, com que, em épocas de interdicção ecclesiástica, se substituía o sacrifício da missa.
\textunderscore Missa calada\textunderscore , o mesmo que \textunderscore missa rezada\textunderscore  ou celebrada sem canto.
\textunderscore Missa alta\textunderscore , a missa que se celebrava com um vagaroso e delicado canto.
\textunderscore Missa chan\textunderscore , o mesmo que missa rezada.
\textunderscore Missa officiada\textunderscore  ou \textunderscore official\textunderscore , missa cantada e solenne.
\textunderscore Missa pedida\textunderscore , aquella que se retribue com dinheiro, pedido de porta em porta, em cumprimento de voto ou promessa.
\textunderscore Missa grande\textunderscore , missa solenne, missa de pontifical. Cf. \textunderscore Hyssope\textunderscore , 14 e 25.
\textunderscore Missa nova\textunderscore , a primeira missa que um padre celebra.
\textunderscore Missa sêca\textunderscore , missa em que não há consumpção da hóstia e vinho consagrados:«\textunderscore á vista da qual me revesti e logo disse missa sêca na nau.\textunderscore »\textunderscore Ethiópia Or.\textunderscore , II, 366.
\section{Missacantante}
\begin{itemize}
\item {Grp. gram.:m.}
\end{itemize}
Clérigo que celebra ou canta missa pela primeira vez:«\textunderscore ...quis honrar a festa e o missacantante...\textunderscore »Sousa, \textunderscore Vida do Arceb.\textunderscore , I, 572.
\section{Missagra}
\begin{itemize}
\item {Grp. gram.:f.}
\end{itemize}
Garlindéu; bisagra.
\section{Missal}
\begin{itemize}
\item {Grp. gram.:m.}
\end{itemize}
Livro, que contém as orações da missa e outras.
Variedade de caracteres typográphicos.
(B. lat. \textunderscore missale\textunderscore )
\section{Missalo}
\begin{itemize}
\item {Grp. gram.:m.}
\end{itemize}
\begin{itemize}
\item {Utilização:T. de Angola}
\end{itemize}
Espécie de peneira cylíndrica, destinada á fuba.
\section{Missanga}
\begin{itemize}
\item {Grp. gram.:f.}
\end{itemize}
\begin{itemize}
\item {Proveniência:(T. cafr.)}
\end{itemize}
Contas miúdas e variegadas, de vidro.
Ornato, feito dessas contas.
Variedade de caractéres typográphicos muito miúdos.
Miudezas, bagatelas, bugigangas.
\section{Missão}
\begin{itemize}
\item {Grp. gram.:f.}
\end{itemize}
\begin{itemize}
\item {Grp. gram.:M.}
\end{itemize}
\begin{itemize}
\item {Utilização:Ant.}
\end{itemize}
\begin{itemize}
\item {Proveniência:(Lat. \textunderscore missio\textunderscore )}
\end{itemize}
Acto de mandar.
Incumbência.
Commissão diplomática.
Espécie de sermão.
Os missionários.
Compromisso.
Enviado, postilhão, correio.
\section{Missar}
\begin{itemize}
\item {Grp. gram.:v. i.}
\end{itemize}
\begin{itemize}
\item {Utilização:Fam.}
\end{itemize}
\begin{itemize}
\item {Grp. gram.:V. t.}
\end{itemize}
\begin{itemize}
\item {Utilização:Ant.}
\end{itemize}
Dizer ou ouvir missa.
Dizer missa por alma de.
\section{Misseiro}
\begin{itemize}
\item {Grp. gram.:m.  e  adj.}
\end{itemize}
Aquelle que é muito devoto de missas.
\section{Missício}
\begin{itemize}
\item {Grp. gram.:m.}
\end{itemize}
\begin{itemize}
\item {Proveniência:(Lat. \textunderscore missicius\textunderscore )}
\end{itemize}
Soldado reformado, entre os Romanos, ou soldado a quem se dava baixa do serviço militar.
\section{Míssil}
\begin{itemize}
\item {Grp. gram.:adj.}
\end{itemize}
\begin{itemize}
\item {Proveniência:(Lat. \textunderscore missilis\textunderscore )}
\end{itemize}
Próprio para sêr arremessado.
\section{Missionar}
\begin{itemize}
\item {Grp. gram.:v. t.}
\end{itemize}
\begin{itemize}
\item {Grp. gram.:V. i.}
\end{itemize}
\begin{itemize}
\item {Proveniência:(Do lat. \textunderscore missio\textunderscore )}
\end{itemize}
Prègar a fé a; cathechizar.
Fazer missões, prègar.
\section{Missionário}
\begin{itemize}
\item {Grp. gram.:m.}
\end{itemize}
\begin{itemize}
\item {Utilização:Ext.}
\end{itemize}
\begin{itemize}
\item {Proveniência:(Do lat. \textunderscore missio\textunderscore )}
\end{itemize}
Aquelle que missiona.
Propagandista.
\section{Missionarismo}
\begin{itemize}
\item {Grp. gram.:m.}
\end{itemize}
\begin{itemize}
\item {Proveniência:(De \textunderscore missionário\textunderscore )}
\end{itemize}
Funcções de missionário; apostolização. Cf. Arn. Gama, \textunderscore Ult. Dona\textunderscore , 482.
\section{Missioneiro}
\begin{itemize}
\item {Grp. gram.:m.}
\end{itemize}
\begin{itemize}
\item {Utilização:Bras. do S}
\end{itemize}
\begin{itemize}
\item {Proveniência:(Do lat. \textunderscore missio\textunderscore )}
\end{itemize}
Indígena ou habitante das regiões, onde se estabeleceram as antigas missões jesuíticas.
\section{Missiva}
\begin{itemize}
\item {Grp. gram.:f.}
\end{itemize}
\begin{itemize}
\item {Proveniência:(De \textunderscore missivo\textunderscore )}
\end{itemize}
Carta; epístola.
Bilhete, que se manda a alguém.
\section{Missivista}
\begin{itemize}
\item {Grp. gram.:m.}
\end{itemize}
\begin{itemize}
\item {Utilização:bras}
\end{itemize}
\begin{itemize}
\item {Utilização:Neol.}
\end{itemize}
Portador de missiva.
\section{Missivo}
\begin{itemize}
\item {Grp. gram.:adj.}
\end{itemize}
\begin{itemize}
\item {Proveniência:(Do lat. \textunderscore missus\textunderscore )}
\end{itemize}
Que se remete; que se expede.
Que se despede ou se arremessa.
\section{Missoilo}
\begin{itemize}
\item {Grp. gram.:m.}
\end{itemize}
\begin{itemize}
\item {Utilização:Prov.}
\end{itemize}
\begin{itemize}
\item {Utilização:trasm.}
\end{itemize}
Pequeno saco de farinha.
Pequena fornada.
Criança de collo, gorda.
\section{Missoira}
\begin{itemize}
\item {Grp. gram.:f.}
\end{itemize}
\begin{itemize}
\item {Utilização:Des.}
\end{itemize}
Corda, com que se arroja para baixo a vela da embarcação.
(Por \textunderscore missora\textunderscore , do lat. \textunderscore missor\textunderscore )
\section{Missongo}
\begin{itemize}
\item {Grp. gram.:m.}
\end{itemize}
Cada um dos indivíduos subalternos do séquito dos Sobas angolenses. Cf. Capello e Ivens, I, 173.
\section{Missório}
\begin{itemize}
\item {Grp. gram.:adj.}
\end{itemize}
\begin{itemize}
\item {Proveniência:(De \textunderscore missa\textunderscore )}
\end{itemize}
Diz-se do encargo, imposto sôbre os possuidores de certos bens ou capellas, de mandar dizer missas por alma do instituidor de certa capella ou vínculo.
\section{Missuri}
\begin{itemize}
\item {Grp. gram.:m.}
\end{itemize}
\begin{itemize}
\item {Proveniência:(De \textunderscore Missouri\textunderscore , n. p.)}
\end{itemize}
Variedade de tabaco.
\section{Missuris}
\begin{itemize}
\item {Grp. gram.:m. pl.}
\end{itemize}
Tríbo de Índios da América do norte.
(Cp. \textunderscore missuri\textunderscore )
\section{Mistão}
\begin{itemize}
\item {Grp. gram.:m.}
\end{itemize}
\begin{itemize}
\item {Proveniência:(Lat. \textunderscore mistio\textunderscore )}
\end{itemize}
Preparação de sebo e azeite, com que os gravadores cobrem, na chapa, os lugares que querem poupar á água-forte.
Mistura. Cf. Garcia Orta, \textunderscore Colloq.\textunderscore  I.
\section{Misteiroso}
\begin{itemize}
\item {Grp. gram.:adj.}
\end{itemize}
O mesmo que \textunderscore mesteiroso\textunderscore .
\section{Mistela}
\begin{itemize}
\item {Grp. gram.:f.}
\end{itemize}
\begin{itemize}
\item {Utilização:Pop.}
\end{itemize}
\begin{itemize}
\item {Proveniência:(De \textunderscore misto\textunderscore )}
\end{itemize}
Bebida, composta de vinho, água, açúcar e canela.
Agua-pé.
Comida ou bebida, em que entram vários ingredientes, e que não tem sabor agradável.
Mistifório.
\section{Mister}
\begin{itemize}
\item {Grp. gram.:m.}
\end{itemize}
\begin{itemize}
\item {Proveniência:(Do lat. \textunderscore ministerium\textunderscore )}
\end{itemize}
Urgência; precisão.
Aquillo que é forçoso:«\textunderscore faz-se mister muita caridade\textunderscore ». Camillo, \textunderscore Retr. de Ricard.\textunderscore , 48.
\section{Mistério}
\begin{itemize}
\item {Grp. gram.:m.}
\end{itemize}
\begin{itemize}
\item {Utilização:Açor}
\end{itemize}
\begin{itemize}
\item {Utilização:Pop.}
\end{itemize}
Tracto de terreno coberto de lavas, provenientes de erupções posteriores á colonização, e que apresenta ainda os vestígios dos estragos que causaram.
(Relaciona-se com \textunderscore mystério\textunderscore ?)
\section{Misterioso}
\begin{itemize}
\item {Grp. gram.:adj.}
\end{itemize}
\begin{itemize}
\item {Utilização:Ant.}
\end{itemize}
\begin{itemize}
\item {Proveniência:(De \textunderscore mister\textunderscore )}
\end{itemize}
Preciso, necessário.
\section{Místico}
\begin{itemize}
\item {Grp. gram.:adj.}
\end{itemize}
\begin{itemize}
\item {Utilização:P. us.}
\end{itemize}
\begin{itemize}
\item {Proveniência:(Do lat. \textunderscore mistus\textunderscore )}
\end{itemize}
Misto; misturado.
Annexo.
\section{Místico}
\begin{itemize}
\item {Grp. gram.:adj.}
\end{itemize}
\begin{itemize}
\item {Utilização:Gír.}
\end{itemize}
Acordado.
Muito bom: \textunderscore pitéu místico\textunderscore .
Perfeito, guapo: \textunderscore moçoila mística\textunderscore .
(Do caló \textunderscore mistó\textunderscore , bom)
\section{Mistíco}
\begin{itemize}
\item {Grp. gram.:m.}
\end{itemize}
Embarcação longa e estreita, usada nas costas da Espanha e da Grécia.
(Do turco \textunderscore mistigo\textunderscore )
\section{Mistiço}
\begin{itemize}
\item {Proveniência:(De \textunderscore misto\textunderscore )}
\end{itemize}
\textunderscore m.\textunderscore  e \textunderscore adj.\textunderscore  (e der.)
O mesmo ou melhor que \textunderscore mestiço\textunderscore , etc.
Que tem mistura de branco e negro.
Mesclado. Cf. Filinto, \textunderscore Vida de D. Man.\textunderscore , III, 183.
\section{Mistifório}
\begin{itemize}
\item {Grp. gram.:m.}
\end{itemize}
\begin{itemize}
\item {Utilização:Fam.}
\end{itemize}
\begin{itemize}
\item {Proveniência:(De \textunderscore mixti-fóri\textunderscore , expressão latina, relativa ao foro jurídico, simultaneamente civil e canónico)}
\end{itemize}
Miscellânea; salsada.
Confusão de coisas ou pessôas.
\section{Mistilíneo}
\begin{itemize}
\item {Grp. gram.:adj.}
\end{itemize}
\begin{itemize}
\item {Proveniência:(Do lat. \textunderscore mistus\textunderscore  + \textunderscore linea\textunderscore )}
\end{itemize}
Formado em parte por linhas curvas e em parte por linhas rectas.
\section{Mistilingue}
\begin{itemize}
\item {Grp. gram.:adj.}
\end{itemize}
\begin{itemize}
\item {Proveniência:(De \textunderscore misto\textunderscore  + \textunderscore língua\textunderscore )}
\end{itemize}
Relativo a várias línguas. Cf. Filinto, III, 278.
\section{Mistinérveo}
\begin{itemize}
\item {Grp. gram.:adj.}
\end{itemize}
\begin{itemize}
\item {Utilização:Bot.}
\end{itemize}
\begin{itemize}
\item {Proveniência:(Do lat. \textunderscore mistus\textunderscore  + \textunderscore nervus\textunderscore )}
\end{itemize}
Diz-se das fôlhas, cujas nervuras se dirigem em vários sentidos.
\section{Misto}
\begin{itemize}
\item {Grp. gram.:adj.}
\end{itemize}
\begin{itemize}
\item {Grp. gram.:M.}
\end{itemize}
\begin{itemize}
\item {Proveniência:(Lat. \textunderscore mistus\textunderscore )}
\end{itemize}
Mesclado; confuso; misturado.
Resultante da mistura de duas ou mais coisas.
Conjunto; mistura.
Refeição de pão e vinho, que os frades de San-Bento e de San-Bernardo tomavam, antes de ir para o côro.
\section{Misto}
\begin{itemize}
\item {Grp. gram.:adj.}
\end{itemize}
\begin{itemize}
\item {Utilização:Gír.}
\end{itemize}
Bom.
(Caló de Espanha, \textunderscore mistó\textunderscore )
\section{Mistral}
\begin{itemize}
\item {Grp. gram.:m.}
\end{itemize}
Vento do Nordeste, no Mediterrâneo.
(Ant. provn. \textunderscore maestral\textunderscore )
\section{Mistura}
\begin{itemize}
\item {Grp. gram.:f.}
\end{itemize}
\begin{itemize}
\item {Utilização:Prov.}
\end{itemize}
\begin{itemize}
\item {Utilização:alent.}
\end{itemize}
\begin{itemize}
\item {Utilização:Bras. de Minas}
\end{itemize}
\begin{itemize}
\item {Proveniência:(Lat. \textunderscore mistura\textunderscore )}
\end{itemize}
Acto ou effeito de misturar.
União de substâncias, que conservam as suas propriedades específicas.
Agua-pé.
\textunderscore Café com mistura\textunderscore , o que se toma com pão ou bolos.
\section{Misturada}
\begin{itemize}
\item {Grp. gram.:f.}
\end{itemize}
\begin{itemize}
\item {Utilização:T. de Alcobaça}
\end{itemize}
\begin{itemize}
\item {Proveniência:(De \textunderscore misturar\textunderscore )}
\end{itemize}
Mistura, mistifório, miscellânea.
Caldo de feijão com couves.
\section{Misturadeira}
\begin{itemize}
\item {Grp. gram.:f.}
\end{itemize}
Um dos maquinismos de chapeleiro. Cf. \textunderscore Inquér. Industr.\textunderscore , P. II, l.^o 2.^o, 176.
\section{Misturar}
\begin{itemize}
\item {Grp. gram.:v. t.}
\end{itemize}
\begin{itemize}
\item {Proveniência:(De \textunderscore mistura\textunderscore )}
\end{itemize}
Confundir, juntar (coisas differentes).
Unir, ligar.
Cruzar.
\section{Misturável}
\begin{itemize}
\item {Grp. gram.:adj.}
\end{itemize}
Que se póde misturar.
\section{Mistureiro}
\begin{itemize}
\item {Grp. gram.:m.}
\end{itemize}
\begin{itemize}
\item {Proveniência:(De \textunderscore misturar\textunderscore )}
\end{itemize}
Traficante, que faz misturas fraudulentas em gêneros alimentícios.
\section{Mísula}
\begin{itemize}
\item {Grp. gram.:f.}
\end{itemize}
\begin{itemize}
\item {Utilização:Náut.}
\end{itemize}
\begin{itemize}
\item {Proveniência:(Do lat. \textunderscore mensula\textunderscore )}
\end{itemize}
Ornato, que resái de uma superfície, geralmente vertical, e que sustenta um vaso, um busto, um arco, etc.
Curva, em que assenta a varanda da popa, nos navios de alto bôrdo.
\section{Mitene}
\begin{itemize}
\item {Grp. gram.:f.}
\end{itemize}
\begin{itemize}
\item {Proveniência:(Do fr. \textunderscore mitaine\textunderscore )}
\end{itemize}
Luva que, cobrindo a mão, deixa descobertos os dedos; punhete.
\section{Mites}
\begin{itemize}
\item {Grp. gram.:m. pl.}
\end{itemize}
\begin{itemize}
\item {Utilização:Ant.}
\end{itemize}
Fíos de contas de vidro ou de barro vidrado, de que as mulheres fazem gargantilhas, entre os Cafres, e que corriam como moéda em Moçambique.
\section{Mithridático}
\begin{itemize}
\item {Grp. gram.:adj.}
\end{itemize}
Relativo ao mithridato.
\section{Mithridatismo}
\begin{itemize}
\item {Grp. gram.:m.}
\end{itemize}
\begin{itemize}
\item {Proveniência:(De \textunderscore mithridato\textunderscore )}
\end{itemize}
Immunidade contra os venenos, adquirida pela absorpção repetida de pequenas doses dos mesmos, gradualmente aumentadas.
\section{Mithridato}
\begin{itemize}
\item {Grp. gram.:m.}
\end{itemize}
\begin{itemize}
\item {Proveniência:(Gr. \textunderscore mithridatos\textunderscore )}
\end{itemize}
Espécie de teriaga ou de antídoto.
\section{Mitical}
\begin{itemize}
\item {Grp. gram.:m.}
\end{itemize}
O mesmo que \textunderscore metical\textunderscore .
\section{Mitigação}
\begin{itemize}
\item {Grp. gram.:f.}
\end{itemize}
\begin{itemize}
\item {Proveniência:(Lat. \textunderscore mitigatio\textunderscore )}
\end{itemize}
Acto ou effeito de mitigar.
\section{Mitigador}
\begin{itemize}
\item {Grp. gram.:m.  e  adj.}
\end{itemize}
O que mitiga.
\section{Mitigar}
\begin{itemize}
\item {Grp. gram.:v. t.}
\end{itemize}
\begin{itemize}
\item {Proveniência:(Lat. \textunderscore mitigare\textunderscore )}
\end{itemize}
Tornar brando, manso.
Alliviar; suavizar; acalmar: \textunderscore mitigar a sêde\textunderscore .
Deminuír.
\section{Mitigativo}
\begin{itemize}
\item {Grp. gram.:adj.}
\end{itemize}
\begin{itemize}
\item {Proveniência:(Lat. \textunderscore mitigativus\textunderscore )}
\end{itemize}
Que mitiga.
\section{Mitigável}
\begin{itemize}
\item {Grp. gram.:adj.}
\end{itemize}
Que se póde mitigar.
\section{Mitonde}
\begin{itemize}
\item {Grp. gram.:m.}
\end{itemize}
(V.pau-dos-feiticeiros)
\section{Mitra}
\begin{itemize}
\item {Grp. gram.:f.}
\end{itemize}
\begin{itemize}
\item {Utilização:Fig.}
\end{itemize}
\begin{itemize}
\item {Utilização:Prov.}
\end{itemize}
\begin{itemize}
\item {Utilização:trasm.}
\end{itemize}
\begin{itemize}
\item {Utilização:Pop.}
\end{itemize}
\begin{itemize}
\item {Grp. gram.:M.}
\end{itemize}
\begin{itemize}
\item {Utilização:Gír.}
\end{itemize}
\begin{itemize}
\item {Proveniência:(Gr. \textunderscore mitra\textunderscore )}
\end{itemize}
Cobertura para a cabeça, usada entre os Persas, Egýpcios, Árabes, etc.
Barrete de fórma cónica, fendido na parte superior, e que em certas solennidades é usado por Bispos, Arcebispos e Cardeaes.
O poder pontifício.
Dignidade ou jurisdicção de um prelado ecclesiástico.
Carapuço de papel, que se collocava na cabeça dos condemnados da Inquisição.
Mollusco gasterópode, cuja cabeça tem fórma de mitra.
Gênero de arachnídeos.
O mesmo que \textunderscore carapuça\textunderscore .
O sobrecu das aves.
Coêlho.
\section{Meudamente}
\begin{itemize}
\item {fónica:mi-u}
\end{itemize}
\begin{itemize}
\item {Grp. gram.:adv.}
\end{itemize}
Em bocadinhos, em miuçalhas.
Minuciosamente, por miüdo.
Cuidadosamente.
\section{Meudar}
\begin{itemize}
\item {fónica:mi-u}
\end{itemize}
\textunderscore v. t.\textunderscore  (e der.)
O mesmo que \textunderscore amiudar\textunderscore ^1, etc. Cf. Filinto, IX, 207.
\section{Meudas}
\begin{itemize}
\item {fónica:mi-u}
\end{itemize}
\begin{itemize}
\item {Grp. gram.:f. pl.}
\end{itemize}
\begin{itemize}
\item {Proveniência:(De \textunderscore miúdo\textunderscore )}
\end{itemize}
Lucros, que vêm pouco a pouco, como ás pinguinhas.
Pequenas mas numerosas dádivas.
\section{Meudear}
\begin{itemize}
\item {fónica:mi-u}
\end{itemize}
\begin{itemize}
\item {Grp. gram.:v. t.}
\end{itemize}
\begin{itemize}
\item {Proveniência:(De \textunderscore miúdo\textunderscore )}
\end{itemize}
Narrar minuciosamente.
Esmiuçar.
\section{Meudeiro}
\begin{itemize}
\item {fónica:mi-u}
\end{itemize}
\begin{itemize}
\item {Grp. gram.:m.}
\end{itemize}
\begin{itemize}
\item {Utilização:Prov.}
\end{itemize}
\begin{itemize}
\item {Utilização:beir.}
\end{itemize}
\begin{itemize}
\item {Grp. gram.:Adj.}
\end{itemize}
\begin{itemize}
\item {Utilização:Açor}
\end{itemize}
\begin{itemize}
\item {Proveniência:(De \textunderscore miúdo\textunderscore )}
\end{itemize}
Espécie de saca de rede, para a pesca de peixe miúdo.
Diz-se do homem exigente, impertinente.
\section{Meudeza}
\begin{itemize}
\item {fónica:mi-u}
\end{itemize}
\begin{itemize}
\item {Grp. gram.:f.}
\end{itemize}
\begin{itemize}
\item {Utilização:Fig.}
\end{itemize}
\begin{itemize}
\item {Grp. gram.:Pl.}
\end{itemize}
Qualidade do que é miúdo, pequeno, delicado.
Rigor; cuidado no exame ou observação de alguma coisa.
Mesquinharia.
Minúcias, pormenores.
Insignificâncias; bugigangas.
Vísceras de alguns animaes.
Carne de venda ambulante, em retalhos: \textunderscore miudezas de vaca\textunderscore .
Casta de uva da Bairrada.
\section{Mitrado}
\begin{itemize}
\item {Grp. gram.:adj.}
\end{itemize}
Que tem mitra ou direito de a usar.
Que tem na cabeça um ornato natural, semelhante a uma mitra, (falando-se de alguns animaes).
\section{Mitral}
\begin{itemize}
\item {Grp. gram.:adj.}
\end{itemize}
\begin{itemize}
\item {Utilização:Anat.}
\end{itemize}
\begin{itemize}
\item {Proveniência:(De \textunderscore mitra\textunderscore )}
\end{itemize}
O mesmo que \textunderscore mitriforme\textunderscore .
Diz-se da válvula que, collocada no ventrículo esquerdo do coração, cérca a abertura de communicação dêste ventrículo com a aurícula correspondente.
\section{Mitrar}
\begin{itemize}
\item {Grp. gram.:v. t.}
\end{itemize}
Pôr mitra em:«\textunderscore Nem terão de me aspar com sambenito, nem mitrar com carocha.\textunderscore »Filinto, II, 214.
\section{Mitrasacmo}
\begin{itemize}
\item {Grp. gram.:m.}
\end{itemize}
\begin{itemize}
\item {Proveniência:(Do gr. \textunderscore mitra\textunderscore  + \textunderscore akme\textunderscore )}
\end{itemize}
Gênero de plantas herbáceas da Nova Holanda, semelhante á genciana.
\section{Mitridático}
\begin{itemize}
\item {Grp. gram.:adj.}
\end{itemize}
Relativo ao mitridato.
\section{Mitridatismo}
\begin{itemize}
\item {Grp. gram.:m.}
\end{itemize}
\begin{itemize}
\item {Proveniência:(De \textunderscore mitridato\textunderscore )}
\end{itemize}
Imunidade contra os venenos, adquirida pela absorpção repetida de pequenas doses dos mesmos, gradualmente aumentadas.
\section{Mitridato}
\begin{itemize}
\item {Grp. gram.:m.}
\end{itemize}
\begin{itemize}
\item {Proveniência:(Gr. \textunderscore mithridatos\textunderscore )}
\end{itemize}
Espécie de teriaga ou de antídoto.
\section{Mitriforme}
\begin{itemize}
\item {Grp. gram.:adj.}
\end{itemize}
\begin{itemize}
\item {Proveniência:(De \textunderscore mitra\textunderscore  + \textunderscore fórma\textunderscore )}
\end{itemize}
Que tem fórma de mitra.
\section{Mitro}
\begin{itemize}
\item {Grp. gram.:m.}
\end{itemize}
\begin{itemize}
\item {Utilização:Ant.}
\end{itemize}
Manípulo dos sacerdotes.
\section{Miúça}
\begin{itemize}
\item {Grp. gram.:f.}
\end{itemize}
\begin{itemize}
\item {Grp. gram.:Pl.}
\end{itemize}
\begin{itemize}
\item {Proveniência:(Do lat. \textunderscore minutia\textunderscore )}
\end{itemize}
Miuçalha.
Antigos dízimos, que se pagavam á Igreja, em gêneros por miúdo.
\section{Miuçalha}
\begin{itemize}
\item {Grp. gram.:f.}
\end{itemize}
\begin{itemize}
\item {Proveniência:(De \textunderscore miúça\textunderscore )}
\end{itemize}
Pequena porção.
Pequeno fragmento.
Conjunto de coisas miúdas de pouco préstimo.
\section{Miuçalho}
\begin{itemize}
\item {Grp. gram.:m.}
\end{itemize}
O mesmo que \textunderscore miuçalha\textunderscore . Cf. Filinto, II, 136.
\section{Miudamente}
\begin{itemize}
\item {Grp. gram.:adv.}
\end{itemize}
Em bocadinhos, em miuçalhas.
Minuciosamente, por miüdo.
Cuidadosamente.
\section{Miudar}
\textunderscore v. t.\textunderscore  (e der.)
O mesmo que \textunderscore amiudar\textunderscore ^1, etc. Cf. Filinto, IX, 207.
\section{Miúdas}
\begin{itemize}
\item {Grp. gram.:f. pl.}
\end{itemize}
\begin{itemize}
\item {Proveniência:(De \textunderscore miúdo\textunderscore )}
\end{itemize}
Lucros, que vêm pouco a pouco, como ás pinguinhas.
Pequenas mas numerosas dádivas.
\section{Miúde}
\begin{itemize}
\item {Grp. gram.:adv.}
\end{itemize}
(Cp. \textunderscore amiúde\textunderscore )
\section{Miudear}
\begin{itemize}
\item {Grp. gram.:v. t.}
\end{itemize}
\begin{itemize}
\item {Proveniência:(De \textunderscore miúdo\textunderscore )}
\end{itemize}
Narrar minuciosamente.
Esmiuçar.
\section{Miudeiro}
\begin{itemize}
\item {Grp. gram.:m.}
\end{itemize}
\begin{itemize}
\item {Utilização:Prov.}
\end{itemize}
\begin{itemize}
\item {Utilização:beir.}
\end{itemize}
\begin{itemize}
\item {Grp. gram.:Adj.}
\end{itemize}
\begin{itemize}
\item {Utilização:Açor}
\end{itemize}
\begin{itemize}
\item {Proveniência:(De \textunderscore miúdo\textunderscore )}
\end{itemize}
Espécie de saca de rede, para a pesca de peixe miúdo.
Diz-se do homem exigente, impertinente.
\section{Miudeza}
\begin{itemize}
\item {Grp. gram.:f.}
\end{itemize}
\begin{itemize}
\item {Utilização:Fig.}
\end{itemize}
\begin{itemize}
\item {Grp. gram.:Pl.}
\end{itemize}
Qualidade do que é miúdo, pequeno, delicado.
Rigor; cuidado no exame ou observação de alguma coisa.
Mesquinharia.
Minúcias, pormenores.
Insignificâncias; bugigangas.
Vísceras de alguns animaes.
Carne de venda ambulante, em retalhos: \textunderscore miudezas de vaca\textunderscore .
Casta de uva da Bairrada.
\section{Miúdo}
\begin{itemize}
\item {Grp. gram.:adj.}
\end{itemize}
\begin{itemize}
\item {Utilização:Fig.}
\end{itemize}
\begin{itemize}
\item {Grp. gram.:M.}
\end{itemize}
\begin{itemize}
\item {Grp. gram.:Loc. adv.}
\end{itemize}
\begin{itemize}
\item {Grp. gram.:Loc. adv.}
\end{itemize}
\begin{itemize}
\item {Grp. gram.:Loc. adv.}
\end{itemize}
\begin{itemize}
\item {Grp. gram.:Pl.}
\end{itemize}
\begin{itemize}
\item {Utilização:Prov.}
\end{itemize}
\begin{itemize}
\item {Utilização:trasm.}
\end{itemize}
\begin{itemize}
\item {Proveniência:(Do lat. \textunderscore minutus\textunderscore )}
\end{itemize}
Deminuto, de mui pequenas dimensões: \textunderscore feijões miúdos\textunderscore .
Frequente.
Minucioso.
Escrupuloso.
Mesquinho, sovina.
Travadoiro.
\textunderscore Por miúdo\textunderscore  ou \textunderscore por miúdos\textunderscore .
Minuciosamente, com todos os pormenores:«\textunderscore referindo por miúdos...\textunderscore »Camillo, \textunderscore Filha do Reg.\textunderscore 
\textunderscore A miúdo\textunderscore , (V. \textunderscore amiúde\textunderscore ).
\textunderscore Pelo miúdo\textunderscore , o mesmo que \textunderscore por miúdo\textunderscore . Cf. Camillo, \textunderscore Enjeitada\textunderscore , 24.
Dinheiro em moédas de pouco valor: \textunderscore trocar uma libra em miúdos\textunderscore .
Miudezas de animaes.
Partículas.
Gravetos.
\section{Miul}
\begin{itemize}
\item {Grp. gram.:m.}
\end{itemize}
O mesmo ou melhor que \textunderscore meul\textunderscore .
\section{Miúlo}
\begin{itemize}
\item {Grp. gram.:m.}
\end{itemize}
O mesmo ou melhor que \textunderscore meul\textunderscore .
\section{Miunça}
\begin{itemize}
\item {Grp. gram.:f.}
\end{itemize}
\begin{itemize}
\item {Utilização:Pop.}
\end{itemize}
O mesmo que \textunderscore miúça\textunderscore .
\section{Miuva}
\begin{itemize}
\item {fónica:mi-u}
\end{itemize}
\begin{itemize}
\item {Grp. gram.:f.}
\end{itemize}
Planta melastomácea, medicinal, do Brasil.
\section{Mixira}
\begin{itemize}
\item {Grp. gram.:f.}
\end{itemize}
\begin{itemize}
\item {Utilização:Bras}
\end{itemize}
\begin{itemize}
\item {Proveniência:(T. tupi)}
\end{itemize}
Chouriço delgado, feito com a carne do peixe-boi.
Conserva de carne ou peixe, em azeite de tartaruga.
\section{Mixórdia}
\begin{itemize}
\item {Grp. gram.:f.}
\end{itemize}
\begin{itemize}
\item {Utilização:Pop.}
\end{itemize}
\begin{itemize}
\item {Proveniência:(Do lat. \textunderscore miscere\textunderscore )}
\end{itemize}
Salsada.
Mistifório.
Confusão, embrulhada.
\section{Mixorofada}
\begin{itemize}
\item {Grp. gram.:f.}
\end{itemize}
\begin{itemize}
\item {Utilização:Pop.}
\end{itemize}
O mesmo que \textunderscore mixórdia\textunderscore , ou \textunderscore mexerufada\textunderscore .
\section{Mixtiárabe}
\begin{itemize}
\item {Proveniência:(De \textunderscore mixto\textunderscore  + \textunderscore árabe\textunderscore )}
\end{itemize}
\textunderscore m.\textunderscore  e \textunderscore adj.\textunderscore  (e der.)
(V. \textunderscore moçárabe\textunderscore , etc.) Cf. Herculano, \textunderscore Hist. de Port.\textunderscore , I, 54.
\section{Mixto}
\begin{itemize}
\item {Proveniência:(Lat. \textunderscore mixtus\textunderscore  = \textunderscore mistus\textunderscore )}
\end{itemize}
\textunderscore adj.\textunderscore  (e der.)
O mesmo que \textunderscore misto\textunderscore ^1, etc.
\section{Mizena}
\begin{itemize}
\item {Grp. gram.:f.}
\end{itemize}
\begin{itemize}
\item {Utilização:Pesc.}
\end{itemize}
Uma das rêdes nos apparelhos de arrastar.
\section{Mnemónica}
\begin{itemize}
\item {Grp. gram.:f.}
\end{itemize}
\begin{itemize}
\item {Proveniência:(De \textunderscore mnemónico\textunderscore )}
\end{itemize}
Arte, que facilita as operações da memória.
\section{Mnemónico}
\begin{itemize}
\item {Grp. gram.:adj.}
\end{itemize}
\begin{itemize}
\item {Proveniência:(Gr. \textunderscore mnemonikos\textunderscore )}
\end{itemize}
Relativo á memória.
Conforme aos preceitos da mnemónica.
Que facilmente se retém na memória: \textunderscore um número mnemónico\textunderscore .
\section{Mnemonização}
\begin{itemize}
\item {Grp. gram.:f.}
\end{itemize}
Acto de mnemonizar.
\section{Mnemonizar}
\begin{itemize}
\item {Grp. gram.:v. t.}
\end{itemize}
Tornar mnemónico.
\section{Mnemonizavel}
\begin{itemize}
\item {Grp. gram.:adj.}
\end{itemize}
Que se póde mnemonizar.
Que facilmente se póde fixar bem na memória.
\section{Mnemotechnia}
\begin{itemize}
\item {Grp. gram.:f.}
\end{itemize}
\begin{itemize}
\item {Proveniência:(Do gr. \textunderscore mneme\textunderscore  + \textunderscore tekhne\textunderscore )}
\end{itemize}
Arte de aumentar a memória.
\section{Mnemotéchnico}
\begin{itemize}
\item {Grp. gram.:adj.}
\end{itemize}
Relativo á mnemotechnia.
\section{Mnemotecnia}
\begin{itemize}
\item {Grp. gram.:f.}
\end{itemize}
\begin{itemize}
\item {Proveniência:(Do gr. \textunderscore mneme\textunderscore  + \textunderscore tekhne\textunderscore )}
\end{itemize}
Arte de aumentar a memória.
\section{Mnemotécnico}
\begin{itemize}
\item {Grp. gram.:adj.}
\end{itemize}
Relativo á mnemotecnia.
\section{Mniaro}
\begin{itemize}
\item {Grp. gram.:m.}
\end{itemize}
Gênero de plantas caryophylláceas da Oceânia.
\section{Mó}
\begin{itemize}
\item {Grp. gram.:f.}
\end{itemize}
\begin{itemize}
\item {Utilização:Prov.}
\end{itemize}
\begin{itemize}
\item {Utilização:trasm.}
\end{itemize}
\begin{itemize}
\item {Grp. gram.:Loc.}
\end{itemize}
\begin{itemize}
\item {Utilização:Bras. do N}
\end{itemize}
\begin{itemize}
\item {Proveniência:(Lat. \textunderscore mola\textunderscore )}
\end{itemize}
Pedra redonda e chata, com que se tritura os cereaes no moínho, ou a azeitona nos lagares.
Pedra, em que se amolam instrumentos cortantes ou perfurantes.
Dente queixal ou molar.
\textunderscore Fazer mó\textunderscore , fazer que a boiada descreva espiral, para tomar em seguida a direcção desejada.
\section{Mó}
\begin{itemize}
\item {Proveniência:(Do lat. \textunderscore moles\textunderscore )}
\end{itemize}
\textunderscore f.\textunderscore  (\textunderscore m.\textunderscore  em Trás-os-Montes)
Grande quantidade.
Grande ajuntamento.
\section{Môa}
\begin{itemize}
\item {Grp. gram.:m.}
\end{itemize}
Pássaro gigante da Nova-Zelândia.
\section{Moabitas}
\begin{itemize}
\item {Grp. gram.:m. pl.}
\end{itemize}
\begin{itemize}
\item {Proveniência:(De \textunderscore Moab\textunderscore , n. p.)}
\end{itemize}
Antigo povo da margem oriental do Mar-Morto.
Nome, que em Portugal se deu aos Moiros residentes em África, para os differençar dos residentes ou naturalizados na Espanha.
\section{Moado}
\begin{itemize}
\item {Grp. gram.:m.}
\end{itemize}
\begin{itemize}
\item {Utilização:Prov.}
\end{itemize}
O resto do caldo, no fundo da malga, com pão migado.
\section{Moafa}
\begin{itemize}
\item {Grp. gram.:f.}
\end{itemize}
\begin{itemize}
\item {Utilização:Pop.}
\end{itemize}
\begin{itemize}
\item {Utilização:Prov.}
\end{itemize}
\begin{itemize}
\item {Utilização:trasm.}
\end{itemize}
\begin{itemize}
\item {Utilização:Chul.}
\end{itemize}
\begin{itemize}
\item {Grp. gram.:Pl.}
\end{itemize}
\begin{itemize}
\item {Utilização:Prov.}
\end{itemize}
Borracheira, embriaguez:«\textunderscore ...porque era padre-mestre de moafas\textunderscore ». Garrett, \textunderscore Falar Verdade\textunderscore .
O mesmo que \textunderscore dinheiro\textunderscore .
Gaifonas, esgares, trejeitos.
(Talvez t. cafreal)
\section{Moageiro}
\begin{itemize}
\item {Grp. gram.:m.}
\end{itemize}
Aquelle que tem fábrica de moagem.
\section{Moagem}
\begin{itemize}
\item {Grp. gram.:f.}
\end{itemize}
Acto de moer.
Moedura, porção de grão ou de azeitona, que se póde moer de cada vez.
A indústria de moageiro.
\section{Moaica}
\begin{itemize}
\item {Grp. gram.:f.}
\end{itemize}
Gênero de palmeiras.
\section{Moal}
\begin{itemize}
\item {Grp. gram.:m.}
\end{itemize}
\begin{itemize}
\item {Utilização:Ant.}
\end{itemize}
O mesmo que \textunderscore mangual\textunderscore .
\section{Moametano}
\begin{itemize}
\item {Grp. gram.:m.  e  adj.}
\end{itemize}
O mesmo ou melhor que \textunderscore mahometano\textunderscore .
\section{Moandim}
\begin{itemize}
\item {Grp. gram.:m.}
\end{itemize}
Planta de San-Thomé.
\section{Moanha}
\begin{itemize}
\item {Grp. gram.:f.}
\end{itemize}
\begin{itemize}
\item {Utilização:Prov.}
\end{itemize}
\begin{itemize}
\item {Utilização:trasm.}
\end{itemize}
O mesmo que \textunderscore caruma\textunderscore . (Colhido em Mesão-Frio)
\section{Moansa}
\begin{itemize}
\item {Grp. gram.:m.}
\end{itemize}
\begin{itemize}
\item {Proveniência:(T. afr.)}
\end{itemize}
Director espiritual dos Negros do Congo.
\section{Mobato}
\begin{itemize}
\item {Grp. gram.:m.}
\end{itemize}
Reptil ophíbio da África occidental.
\section{Mobele}
\begin{itemize}
\item {Grp. gram.:m.}
\end{itemize}
Planta africana, espécie de sorgo, de cujos grãos se faz uma espécie de cerveja.
\section{Mobica}
\begin{itemize}
\item {Grp. gram.:m.  e  f.}
\end{itemize}
\begin{itemize}
\item {Utilização:Bras}
\end{itemize}
Indivíduo, que deixou de sêr escravo.
(Cp. quimbundo \textunderscore m'bica\textunderscore )
\section{Móbil}
\begin{itemize}
\item {Grp. gram.:adj.}
\end{itemize}
\begin{itemize}
\item {Grp. gram.:M.}
\end{itemize}
\begin{itemize}
\item {Proveniência:(Lat. \textunderscore mobilis\textunderscore )}
\end{itemize}
O mesmo que \textunderscore móvel\textunderscore .
Causa; motor.
\section{Mobilação}
\begin{itemize}
\item {Grp. gram.:f.}
\end{itemize}
Acto de mobilar.
\section{Mobilador}
\begin{itemize}
\item {Grp. gram.:adj.}
\end{itemize}
\begin{itemize}
\item {Grp. gram.:M.}
\end{itemize}
Que mobila.
Aquelle que mobíla.
\section{Mobilamento}
\begin{itemize}
\item {Grp. gram.:m.}
\end{itemize}
O mesmo que \textunderscore mobilação\textunderscore .
\section{Mobilar}
\begin{itemize}
\item {Grp. gram.:v. t.}
\end{itemize}
\begin{itemize}
\item {Proveniência:(De \textunderscore móbil\textunderscore )}
\end{itemize}
Guarnecer de mobília: \textunderscore mobilar uma sala\textunderscore .
Fornecer móveis para.
\section{Móbile}
\begin{itemize}
\item {Grp. gram.:m.  e  adj.}
\end{itemize}
O mesmo que \textunderscore móbil\textunderscore .
\section{Mobília}
\begin{itemize}
\item {Grp. gram.:f.}
\end{itemize}
\begin{itemize}
\item {Proveniência:(Lat. \textunderscore mobilia\textunderscore )}
\end{itemize}
Objectos móveis, para uso ou ornato interior de uma casa.
\section{Mobiliário}
\begin{itemize}
\item {Grp. gram.:adj.}
\end{itemize}
\begin{itemize}
\item {Grp. gram.:M.}
\end{itemize}
Relativo a mobília, ou a bens móveis.
Que tem a natureza de bens móveis.
Conjunto de móveis; mobília.
\section{Mobilidade}
\begin{itemize}
\item {Grp. gram.:f.}
\end{itemize}
\begin{itemize}
\item {Utilização:Fig.}
\end{itemize}
\begin{itemize}
\item {Proveniência:(Do lat. \textunderscore mobilitas\textunderscore )}
\end{itemize}
Qualidade ou propriedade do que é móvel ou do que obedece ás leis do movimento.
Volubilidade; inconstância.
\section{Mobilismo}
\begin{itemize}
\item {Grp. gram.:m.}
\end{itemize}
\begin{itemize}
\item {Proveniência:(De \textunderscore móbil\textunderscore )}
\end{itemize}
Systema de apicultura, por meio de quadros móveis.
\section{Mobilista}
\begin{itemize}
\item {Grp. gram.:adj.}
\end{itemize}
\begin{itemize}
\item {Grp. gram.:M.}
\end{itemize}
Relativo ao mobilismo.
Apicultor, que emprega o mobilismo.
\section{Mobilização}
\begin{itemize}
\item {Grp. gram.:f.}
\end{itemize}
Acto de mobilizar.
\section{Mobilizar}
\begin{itemize}
\item {Grp. gram.:v. t.}
\end{itemize}
\begin{itemize}
\item {Proveniência:(De \textunderscore móbil\textunderscore )}
\end{itemize}
Dar movimento a.
Pôr em movimento ou passar do estado de paz para o de guerra (tropas)
\section{Mobilizável}
\begin{itemize}
\item {Grp. gram.:adj.}
\end{itemize}
Que se póde mobilizar.
\section{Mobilo}
\begin{itemize}
\item {Grp. gram.:m.}
\end{itemize}
Trepadeira angolense, medicinal, de frutos amarelos.
\section{Mobiro}
\begin{itemize}
\item {Grp. gram.:m.}
\end{itemize}
O mesmo que \textunderscore mobilo\textunderscore .
\section{Mobula}
\begin{itemize}
\item {Grp. gram.:f.}
\end{itemize}
Árvore da África meridional, cujo fruto tem a consistência da manteiga e o aroma da baunilha.
\section{Moca}
\begin{itemize}
\item {Grp. gram.:f.}
\end{itemize}
\begin{itemize}
\item {Utilização:Bras}
\end{itemize}
\begin{itemize}
\item {Utilização:Gír.}
\end{itemize}
Zombaria.
Peta.
Tolice.
Traição.
\section{Moca}
\begin{itemize}
\item {Grp. gram.:f.}
\end{itemize}
\begin{itemize}
\item {Utilização:Pop.}
\end{itemize}
Cacete, maça.
Clava.
\section{Moca}
\begin{itemize}
\item {Grp. gram.:m.}
\end{itemize}
\begin{itemize}
\item {Proveniência:(De \textunderscore Moca\textunderscore , n. p.)}
\end{itemize}
Variedade de café, muito apreciado.
\section{Moça}
\begin{itemize}
\item {fónica:mô}
\end{itemize}
\begin{itemize}
\item {Grp. gram.:f.}
\end{itemize}
\begin{itemize}
\item {Utilização:Chul. lisb.}
\end{itemize}
\begin{itemize}
\item {Utilização:Prov.}
\end{itemize}
\begin{itemize}
\item {Utilização:Bras. do Amazonas}
\end{itemize}
\begin{itemize}
\item {Proveniência:(De \textunderscore moço\textunderscore )}
\end{itemize}
Pessôa nova do sexo feminino.
Rapariga.
Mulher pública, meretriz.
O mesmo que \textunderscore criada\textunderscore  ou \textunderscore serva\textunderscore .
O mesmo que \textunderscore amásia\textunderscore .
\section{Moça-bonita}
\begin{itemize}
\item {Grp. gram.:f.}
\end{itemize}
\begin{itemize}
\item {Utilização:Bras. de Minas}
\end{itemize}
Erva medicinal, desinfectante e diurética.
\section{Moça-branca}
\begin{itemize}
\item {Grp. gram.:f.}
\end{itemize}
\begin{itemize}
\item {Utilização:Bras}
\end{itemize}
Pequena abelha, quási branca.
\section{Mòcada}
\begin{itemize}
\item {Grp. gram.:f.}
\end{itemize}
Pancada com moca^2.
\section{Moçada}
\begin{itemize}
\item {Grp. gram.:f.}
\end{itemize}
\begin{itemize}
\item {Utilização:T. de Mértola}
\end{itemize}
\begin{itemize}
\item {Proveniência:(De \textunderscore moço\textunderscore )}
\end{itemize}
O mesmo que \textunderscore rapaziada\textunderscore .
\section{Mocadão}
\begin{itemize}
\item {Grp. gram.:m.}
\end{itemize}
\begin{itemize}
\item {Utilização:Ant.}
\end{itemize}
\begin{itemize}
\item {Proveniência:(Do ár. \textunderscore mocadam\textunderscore )}
\end{itemize}
Arraes ou patrão de barco, na Índia portuguesa.
\section{Moçalhão}
\begin{itemize}
\item {Grp. gram.:m.}
\end{itemize}
O mesmo que \textunderscore mocetão\textunderscore .
\section{Mocamau}
\begin{itemize}
\item {Grp. gram.:m.}
\end{itemize}
\begin{itemize}
\item {Utilização:Bras}
\end{itemize}
O mesmo que \textunderscore mocambeiro\textunderscore .
\section{Moçambaz}
\begin{itemize}
\item {Grp. gram.:m.}
\end{itemize}
Nome que em Moçambique se dava ao pombeiro^2. Cf. Ficalho, \textunderscore Plantas Úteis da Áfr.\textunderscore 
\section{Mocambeiro}
\begin{itemize}
\item {Grp. gram.:m.}
\end{itemize}
\begin{itemize}
\item {Utilização:Bras}
\end{itemize}
\begin{itemize}
\item {Grp. gram.:Adj.}
\end{itemize}
\begin{itemize}
\item {Utilização:Bras}
\end{itemize}
Escravo fugitivo ou refugiado em mocambo.
Diz-se do gado, que se esconde nos mocambos.
\section{Moçambicano}
\begin{itemize}
\item {Grp. gram.:adj.}
\end{itemize}
\begin{itemize}
\item {Grp. gram.:M.}
\end{itemize}
Relativo a Moçambique.
Indígena ou habitante de Moçambique.
\section{Mocambo}
\begin{itemize}
\item {Grp. gram.:m.}
\end{itemize}
\begin{itemize}
\item {Utilização:Bras}
\end{itemize}
\begin{itemize}
\item {Utilização:Ext.}
\end{itemize}
Choça, em que os pretos se abrigam, quando fogem para o mato.
Choça.
Grande moita, onde se esconde o gado nos sertões.
Habitação ou abrigo de quem vigia a lavoira.
(Talvez t. afr., da mesma or. que \textunderscore mocamo\textunderscore )
\section{Mocamo}
\begin{itemize}
\item {Grp. gram.:m.}
\end{itemize}
\begin{itemize}
\item {Proveniência:(T. ár.)}
\end{itemize}
Mesquita ou lugar sagrado, entre os Moiros.
\section{Mocan}
\begin{itemize}
\item {Grp. gram.:m.}
\end{itemize}
O mesmo que \textunderscore mocano\textunderscore .
\section{Mocanco}
\begin{itemize}
\item {Grp. gram.:adj.}
\end{itemize}
O mesmo que \textunderscore mocanqueiro\textunderscore .
\section{Mocano}
\begin{itemize}
\item {Grp. gram.:m.}
\end{itemize}
Árvore madeirense, (\textunderscore pittosporum coriaceum\textunderscore , Ait.), de 6 a 8 metros de altura. Cf. \textunderscore Bol. da Soc. de Geogr.\textunderscore , XXX, 610.
\section{Mocanqueiro}
\begin{itemize}
\item {Grp. gram.:adj.}
\end{itemize}
\begin{itemize}
\item {Utilização:Fam.}
\end{itemize}
O mesmo que \textunderscore moquenco\textunderscore .
\section{Mocanquice}
\begin{itemize}
\item {Grp. gram.:f.}
\end{itemize}
\begin{itemize}
\item {Utilização:Fam.}
\end{itemize}
\begin{itemize}
\item {Proveniência:(De \textunderscore mocanco\textunderscore )}
\end{itemize}
Carícias, lábia; momice.
\section{Moção}
\begin{itemize}
\item {Grp. gram.:f.}
\end{itemize}
\begin{itemize}
\item {Utilização:Fig.}
\end{itemize}
\begin{itemize}
\item {Proveniência:(Do lat. \textunderscore motio\textunderscore )}
\end{itemize}
Acto ou effeito de mover.
Commoção.
Espécie de proposta, numa assembleia, sôbre o estado de uma questão, ou sôbre qualquer incidente relativo ao funccionamento ou aos trabalhos da assembleia.
\section{Mocar}
\begin{itemize}
\item {Grp. gram.:v. t.}
\end{itemize}
\begin{itemize}
\item {Utilização:Gír.}
\end{itemize}
\begin{itemize}
\item {Proveniência:(De \textunderscore moca\textunderscore ^1)}
\end{itemize}
Enganar, atraiçoar.
\section{Moçar}
\begin{itemize}
\item {Grp. gram.:m.}
\end{itemize}
\begin{itemize}
\item {Utilização:Ant.}
\end{itemize}
Montão de pedras; ruínas de habitação; pardieiro.
\section{Moçárabe}
\begin{itemize}
\item {Grp. gram.:m.  e  adj.}
\end{itemize}
\begin{itemize}
\item {Proveniência:(Do ár. \textunderscore mustarabi\textunderscore , estrangeiro)}
\end{itemize}
Christão da Espanha, submetido aos Muçulmanos.
Descendente de Christãos da Espanha, que estavam sujeitos aos Muçulmanos.
\section{Moçarábico}
\begin{itemize}
\item {Grp. gram.:adj.}
\end{itemize}
Relativo aos Moçárabes.
\section{Moçarabismo}
\begin{itemize}
\item {Grp. gram.:m.}
\end{itemize}
\begin{itemize}
\item {Proveniência:(De \textunderscore moçárabe\textunderscore )}
\end{itemize}
Situação política e social dos Moçarabes.
Os Moçarabes. Cf. Herculano, \textunderscore Opúsc.\textunderscore , III, 290.
\section{Mocarraria}
\begin{itemize}
\item {Grp. gram.:f.}
\end{itemize}
\begin{itemize}
\item {Utilização:Ant.}
\end{itemize}
Presente, que os reis de Ormuz faziam aos soberanos de certas terras, para que não impedissem o commércio a estranhos.
\section{Moçasso}
\begin{itemize}
\item {Grp. gram.:m.}
\end{itemize}
Árvore africana, de fôlhas simples e flôres hermaphroditas, muito vulgar nas margens do Lovo.
\section{Mocassó}
\begin{itemize}
\item {Grp. gram.:m.}
\end{itemize}
\begin{itemize}
\item {Utilização:T. da Índia port}
\end{itemize}
Terreno que, em compensação de serviços públicos, foi doado pela autoridade soberana, e que se constituiu propriedade trasm.ssível a herdeiros.
\section{Mocedo}
\begin{itemize}
\item {fónica:cê}
\end{itemize}
\begin{itemize}
\item {Grp. gram.:m.}
\end{itemize}
Porção de moças ou raparigas; raparigada. Cf. Castilho, \textunderscore Fausto\textunderscore , 347.
\section{Mocelinha}
\begin{itemize}
\item {Grp. gram.:f.}
\end{itemize}
\begin{itemize}
\item {Utilização:Ant.}
\end{itemize}
\begin{itemize}
\item {Proveniência:(De \textunderscore moça\textunderscore )}
\end{itemize}
Mocinha, menina.
\section{Mocerengue}
\begin{itemize}
\item {Grp. gram.:m.}
\end{itemize}
\begin{itemize}
\item {Utilização:Bras}
\end{itemize}
Gênero de árvores silvestres.
\section{Mocerenguçu}
\begin{itemize}
\item {Grp. gram.:m.}
\end{itemize}
\begin{itemize}
\item {Utilização:Bras}
\end{itemize}
Espécie de mocerengue.
\section{Mocetão}
\begin{itemize}
\item {Grp. gram.:m.}
\end{itemize}
\begin{itemize}
\item {Proveniência:(De \textunderscore moço\textunderscore )}
\end{itemize}
Rapagão; rapaz forte e bem parecido.
\section{Mocetona}
\begin{itemize}
\item {Grp. gram.:f.}
\end{itemize}
Rapariga ou moça robusta e bem parecida ou formosa.
\section{Mochachim}
\begin{itemize}
\item {Grp. gram.:m.}
\end{itemize}
\begin{itemize}
\item {Utilização:Ant.}
\end{itemize}
\begin{itemize}
\item {Utilização:Des.}
\end{itemize}
Espécie de dança popular.
Homem ridículo; bobo. Cf. Filinto, IX, 111; S. Viterbo, \textunderscore Artes e Artistas\textunderscore ; F. Manuel, \textunderscore Fidalgo Aprendiz\textunderscore .
\section{Mochaco}
\begin{itemize}
\item {Grp. gram.:m.}
\end{itemize}
\begin{itemize}
\item {Utilização:Bras. do S}
\end{itemize}
\begin{itemize}
\item {Proveniência:(De \textunderscore mocho\textunderscore )}
\end{itemize}
Espécie de cambão, em que descansa o cabeçalho do carro.
\section{Mochadura}
\begin{itemize}
\item {Grp. gram.:f.}
\end{itemize}
Acto de mochar.
\section{Mocha-mocha}
\begin{itemize}
\item {Grp. gram.:f.}
\end{itemize}
Fruto da África central.
\section{Mochar}
\begin{itemize}
\item {Grp. gram.:v. t.}
\end{itemize}
Tornar mocho, cortar um membro a.
\section{Mocharra}
\begin{itemize}
\item {Grp. gram.:f.}
\end{itemize}
\begin{itemize}
\item {Proveniência:(De \textunderscore mocho\textunderscore )}
\end{itemize}
Pequeno peixe do Guadiana.
\section{Mochau}
\begin{itemize}
\item {Grp. gram.:m.}
\end{itemize}
Planta venenosa do sul da África, variedade de trepadeira.
\section{Mocheta}
\begin{itemize}
\item {fónica:chê}
\end{itemize}
\begin{itemize}
\item {Grp. gram.:f.}
\end{itemize}
Filete, listel.
\section{Mochichos}
\begin{itemize}
\item {Grp. gram.:m. pl.}
\end{itemize}
\begin{itemize}
\item {Utilização:Prov.}
\end{itemize}
Acenos, macaquices, esgares.
\section{Mochico}
\begin{itemize}
\item {Grp. gram.:m.}
\end{itemize}
\begin{itemize}
\item {Utilização:Prov.}
\end{itemize}
\begin{itemize}
\item {Utilização:alg.}
\end{itemize}
Rapazote.
(Por \textunderscore mocico\textunderscore , de \textunderscore moço\textunderscore )
\section{Mochila}
\begin{itemize}
\item {Grp. gram.:f.}
\end{itemize}
\begin{itemize}
\item {Utilização:Ext.}
\end{itemize}
\begin{itemize}
\item {Utilização:Fig.}
\end{itemize}
\begin{itemize}
\item {Utilização:Bras. do N}
\end{itemize}
\begin{itemize}
\item {Utilização:Bras. do N}
\end{itemize}
\begin{itemize}
\item {Grp. gram.:M.}
\end{itemize}
Espécie de saco, em que os soldados levam ás costas roupa e outros objectos.
Gualdrapa; saco de viagem.
Corcova, carcunda.
Saco de coiro ou de lona, em que se dá ração ás cavalgaduras.
Pequeno saco, em que se mete a cabeça do cabrito, para que não mame.
Soldado? homem carcunda?:«\textunderscore ...um mochila brejeiro...\textunderscore »Filinto, V, 131.
(Cast. \textunderscore mochila\textunderscore )
\section{Mochileta}
\begin{itemize}
\item {fónica:lê}
\end{itemize}
\begin{itemize}
\item {Grp. gram.:f.}
\end{itemize}
Pequena mochila.
\section{Mochilo}
\begin{itemize}
\item {Grp. gram.:m.}
\end{itemize}
\begin{itemize}
\item {Utilização:Prov.}
\end{itemize}
\begin{itemize}
\item {Utilização:trasm.}
\end{itemize}
Saco pequeno. (Colhido em Valpaços)
(Cp. \textunderscore mochila\textunderscore )
\section{Mochinete}
\begin{itemize}
\item {fónica:nê}
\end{itemize}
\begin{itemize}
\item {Grp. gram.:m.}
\end{itemize}
\begin{itemize}
\item {Utilização:Prov.}
\end{itemize}
\begin{itemize}
\item {Utilização:trasm.}
\end{itemize}
Murro; lambada.
\section{Mochinga}
\begin{itemize}
\item {Grp. gram.:f.}
\end{itemize}
\begin{itemize}
\item {Utilização:T. de Turquel}
\end{itemize}
Castigo leve.
Zurzidela.
\section{Mocho}
\begin{itemize}
\item {fónica:mô}
\end{itemize}
\begin{itemize}
\item {Grp. gram.:adj.}
\end{itemize}
\begin{itemize}
\item {Utilização:Ext.}
\end{itemize}
\begin{itemize}
\item {Utilização:Prov.}
\end{itemize}
\begin{itemize}
\item {Utilização:trasm.}
\end{itemize}
\begin{itemize}
\item {Utilização:Náut.}
\end{itemize}
\begin{itemize}
\item {Grp. gram.:M.}
\end{itemize}
\begin{itemize}
\item {Utilização:Fig.}
\end{itemize}
\begin{itemize}
\item {Utilização:Prov.}
\end{itemize}
\begin{itemize}
\item {Grp. gram.:Loc.}
\end{itemize}
\begin{itemize}
\item {Utilização:Loc. de Alcanena.}
\end{itemize}
\begin{itemize}
\item {Proveniência:(Do lat. \textunderscore mutilus\textunderscore )}
\end{itemize}
Diz-se do animal, que não tem cornos, porque lhos cortaram, ou porque nasceu sem êlles, devendo-os têr.
Que tem falta de algum membro.
Que não tem grãos ou sementes: \textunderscore ervilha mocha\textunderscore .
Que perdeu os ramos, (falando-se de uma árvore), ou que perdeu os mastros, (falando-se de um navio).
Diz-se de uma espécie de trigo molle.
\textunderscore Mastaréu mocho\textunderscore , diz-se o mastaréu em que a borla fica quási em cima das últimas encapelladuras.
Ave nocturna, (\textunderscore strix otus\textunderscore ).
Misanthropo; homem sorumbático.
Banco, de assento quadrado e sem encôsto.
Canilha de ferro, na extremidade do eixo do carro, para segurar a roda.
\textunderscore Ou cuco ou mocho\textunderscore , ou uma coisa ou outra.
\section{Mocholi}
\begin{itemize}
\item {Grp. gram.:m.}
\end{itemize}
\begin{itemize}
\item {Utilização:Bras}
\end{itemize}
Pequeno peixe, muito apreciado.
\section{Mochuluchulu}
\begin{itemize}
\item {Grp. gram.:m.}
\end{itemize}
Árvore africana, de frutos comestíveis.
\section{Mociço}
\begin{itemize}
\item {Grp. gram.:adj.}
\end{itemize}
(V. \textunderscore maciço\textunderscore ^1)
\section{Mocidade}
\begin{itemize}
\item {Grp. gram.:f.}
\end{itemize}
\begin{itemize}
\item {Utilização:Fig.}
\end{itemize}
Estado de quem é moço.
Idade ou frescor de quem é moço.
Os que são moços: \textunderscore a mocidade illude-se\textunderscore .
Falta de reflexão, imprudência.
\section{Mocitaíba}
\begin{itemize}
\item {Grp. gram.:f.}
\end{itemize}
\begin{itemize}
\item {Utilização:Bras}
\end{itemize}
Gênero de árvores leguminosas.
\section{Mocitaibuçu}
\begin{itemize}
\item {fónica:ta-i}
\end{itemize}
\begin{itemize}
\item {Grp. gram.:m.}
\end{itemize}
Espécie de mocitaíba.
\section{Moco}
\begin{itemize}
\item {Grp. gram.:m.}
\end{itemize}
\begin{itemize}
\item {Utilização:Gír.}
\end{itemize}
Pateta; idiota.
\section{Mocó}
\begin{itemize}
\item {Grp. gram.:m.}
\end{itemize}
\begin{itemize}
\item {Utilização:Bras}
\end{itemize}
Saco de pelles de animaes.
Animal roedor, semelhante ao coêlho, mas maior e sem orelhas nem cauda.
\section{Moço}
\begin{itemize}
\item {fónica:mô}
\end{itemize}
\begin{itemize}
\item {Grp. gram.:adj.}
\end{itemize}
\begin{itemize}
\item {Utilização:Fig.}
\end{itemize}
\begin{itemize}
\item {Grp. gram.:M.}
\end{itemize}
\begin{itemize}
\item {Utilização:Bras}
\end{itemize}
\begin{itemize}
\item {Proveniência:(Do lat. \textunderscore mustus\textunderscore )}
\end{itemize}
Jovem.
Que já não é criança e ainda não é adulto.
Inexperiente; imprudente.
Rapaz; mancebo.
Aquelle que está na idade juvenil.
Criado, serviçal.
Menino ou rapaz branco.
\section{Mocôa}
\begin{itemize}
\item {Grp. gram.:f.}
\end{itemize}
Resina americana, de que os indígenas fazem um verniz semelhante ao charão.
\section{Mococó}
\begin{itemize}
\item {Grp. gram.:m.}
\end{itemize}
Mammífero africano, espécie de lêmure.
\section{Moçôco}
\begin{itemize}
\item {Grp. gram.:m.}
\end{itemize}
\begin{itemize}
\item {Utilização:Ant.}
\end{itemize}
Menino do côro.
\section{Mococona}
\begin{itemize}
\item {Grp. gram.:f.}
\end{itemize}
\begin{itemize}
\item {Utilização:Bras}
\end{itemize}
Fruto comestível de uma leguminosa, espécie de engá.
\section{Moço-de-forcado}
\begin{itemize}
\item {Grp. gram.:m.}
\end{itemize}
Indivíduo, que péga os toiros, depois de lidados pelos bandarilheiros ou cavalleiros.
\section{Moçoila}
\begin{itemize}
\item {Grp. gram.:f.}
\end{itemize}
\begin{itemize}
\item {Proveniência:(De \textunderscore moça\textunderscore )}
\end{itemize}
Raparigota; rapariga esforçada.
\section{Moconis}
\begin{itemize}
\item {Grp. gram.:m. pl.}
\end{itemize}
\begin{itemize}
\item {Utilização:Bras}
\end{itemize}
Tríbo de aborígenes.
\section{Mocori}
\begin{itemize}
\item {Grp. gram.:m.}
\end{itemize}
Árvore silvestre do Brasil.
\section{Mocororó}
\begin{itemize}
\item {Grp. gram.:m.}
\end{itemize}
\begin{itemize}
\item {Utilização:Bras. do N}
\end{itemize}
Suco fermentado do acaju.
Nome commum a várias bebidas refrigerantes.
\section{Mocotó}
\begin{itemize}
\item {Grp. gram.:m.}
\end{itemize}
\begin{itemize}
\item {Utilização:Bras}
\end{itemize}
\begin{itemize}
\item {Utilização:Prov.}
\end{itemize}
\begin{itemize}
\item {Utilização:trasm.}
\end{itemize}
Mão de vaca.
Planta silvestre, da fam. das acantháceas.
Espécie de sapo.
Homem velho, muito gordo e pesado.
\section{Moçuaquim}
\begin{itemize}
\item {Grp. gram.:m.}
\end{itemize}
\begin{itemize}
\item {Proveniência:(Do ár. \textunderscore misuaque\textunderscore )}
\end{itemize}
Planta medicinal e dentífrica.
\section{Mocuba}
\begin{itemize}
\item {Grp. gram.:f.}
\end{itemize}
Gênero de árvores silvestres do Brasil.
\section{Mocubuçu}
\begin{itemize}
\item {Grp. gram.:m.}
\end{itemize}
\begin{itemize}
\item {Utilização:Bras}
\end{itemize}
Espécie de mocuba.
\section{Mocujé}
\begin{itemize}
\item {Grp. gram.:m.}
\end{itemize}
\begin{itemize}
\item {Utilização:Bras}
\end{itemize}
Árvore fructífera dos sertões.
\section{Mocunhambele}
\begin{itemize}
\item {Grp. gram.:m.}
\end{itemize}
Árvore angolense.
\section{Moda}
\begin{itemize}
\item {Grp. gram.:f.}
\end{itemize}
\begin{itemize}
\item {Utilização:Prov.}
\end{itemize}
\begin{itemize}
\item {Utilização:minh.}
\end{itemize}
\begin{itemize}
\item {Proveniência:(Fr. \textunderscore mode\textunderscore . Cp. lat. \textunderscore modus\textunderscore )}
\end{itemize}
Maneira; costume, uso geral.
Uso, que depende do capricho.
Fantasia.
Ária, cantiga.
Pão migado, que assenta no fundo da malga do caldo.
\section{Modal}
\begin{itemize}
\item {Grp. gram.:adj.}
\end{itemize}
\begin{itemize}
\item {Utilização:Gram.}
\end{itemize}
\begin{itemize}
\item {Proveniência:(De \textunderscore modo\textunderscore )}
\end{itemize}
Relativo á modalidade.
Relativo ao modo particular de fazer alguma coisa.
Diz-se das proposições, que encerram condição ou restricção.
\section{Modalidade}
\begin{itemize}
\item {Grp. gram.:f.}
\end{itemize}
\begin{itemize}
\item {Proveniência:(De \textunderscore modal\textunderscore )}
\end{itemize}
Modo de sêr.
Restrícção de certas proposições.
\section{Modão}
\begin{itemize}
\item {Grp. gram.:m.}
\end{itemize}
(?):«\textunderscore e tendo modão de aranha...\textunderscore »G. Vicente, I, 261.
\section{Modelação}
\begin{itemize}
\item {Grp. gram.:f.}
\end{itemize}
Acto de modelar.
\section{Modelador}
\begin{itemize}
\item {Grp. gram.:m.  e  adj.}
\end{itemize}
O que modela.
\section{Modelagem}
\begin{itemize}
\item {Grp. gram.:f.}
\end{itemize}
O mesmo que \textunderscore modelação\textunderscore .
\section{Modelar}
\begin{itemize}
\item {Grp. gram.:v. t.}
\end{itemize}
\begin{itemize}
\item {Grp. gram.:V. p.}
\end{itemize}
\begin{itemize}
\item {Proveniência:(Do lat. \textunderscore modulari\textunderscore )}
\end{itemize}
Representar por meio de um modêlo.
Fazer o modêlo de.
Reproduzir exactamente, em pintura.
Ajustar-se a; envolver, contornando, e deixando conhecer as fórmas de: \textunderscore aquelle vestido modela-a completamente\textunderscore .
Planear, delinear.
Tomar como modêlo.
\section{Modelar}
\begin{itemize}
\item {Grp. gram.:adj.}
\end{itemize}
Que serve de modêlo.
Exemplar; perfeito.
\section{Modêlo}
\begin{itemize}
\item {Grp. gram.:m.}
\end{itemize}
\begin{itemize}
\item {Utilização:Fig.}
\end{itemize}
\begin{itemize}
\item {Proveniência:(De \textunderscore modelar\textunderscore )}
\end{itemize}
O mesmo que \textunderscore molde\textunderscore .
Imagem, que se pretende reproduzir em esculptura.
Typo.
Aquillo que serve de exemplo.
Pessôa exemplar, perfeita.
Objecto bem feito, que póde servir de norma.
Pessôa que serve para estudo prático de pintores ou esculptores: \textunderscore este pintor escolhe bem os seus modêlos\textunderscore .
\section{Moderação}
\begin{itemize}
\item {Grp. gram.:f.}
\end{itemize}
\begin{itemize}
\item {Proveniência:(Lat. \textunderscore moderatio\textunderscore )}
\end{itemize}
Acto ou effeito de moderar.
Attenuação.
Afroixamento.
Deminuição: \textunderscore a moderação do frio\textunderscore .
Circunspecção; prudência: \textunderscore proceder com moderação\textunderscore .
Meio termo; mediania.
\section{Moderadamente}
\begin{itemize}
\item {Grp. gram.:adv.}
\end{itemize}
De modo moderado; com moderação; discretamente; prudentemente.
\section{Moderado}
\begin{itemize}
\item {Grp. gram.:adj.}
\end{itemize}
\begin{itemize}
\item {Proveniência:(Lat. \textunderscore moderatus\textunderscore )}
\end{itemize}
Discreto; prudente.
Razoável.
Em que não há exaggêro.
Que observa as conveniências.
Attenuado; reduzido; limitado.
Medíocre.
\section{Moderador}
\begin{itemize}
\item {Grp. gram.:adj.}
\end{itemize}
\begin{itemize}
\item {Grp. gram.:M.}
\end{itemize}
\begin{itemize}
\item {Proveniência:(Lat. \textunderscore moderator\textunderscore )}
\end{itemize}
Que modera; que attenua.
Que modifica; que abranda.
Que reduz ou restringe.
Aquelle que modera.
\section{Moderante}
\begin{itemize}
\item {Grp. gram.:adj.}
\end{itemize}
\begin{itemize}
\item {Proveniência:(Lat. \textunderscore moderans\textunderscore )}
\end{itemize}
Que modera.
\section{Moderantismo}
\begin{itemize}
\item {Grp. gram.:m.}
\end{itemize}
\begin{itemize}
\item {Proveniência:(De \textunderscore moderante\textunderscore )}
\end{itemize}
Qualidade ou acto de sêr moderado em opiniões ou procedimento.
Ideias moderadas em política.
\section{Moderar}
\begin{itemize}
\item {Grp. gram.:v. t.}
\end{itemize}
\begin{itemize}
\item {Grp. gram.:V. p.}
\end{itemize}
\begin{itemize}
\item {Proveniência:(Lat. \textunderscore moderare\textunderscore )}
\end{itemize}
Pôr no meio termo entre os extremos.
Ajustar aos convenientes limites.
Acommodar ás conveniências.
Conter, reprimir.
Tornar menor, menos intenso.
Sêr commedido, prudente; não commeter excessos.
\section{Moderativo}
\begin{itemize}
\item {Grp. gram.:adj.}
\end{itemize}
\begin{itemize}
\item {Proveniência:(De \textunderscore moderar\textunderscore )}
\end{itemize}
Que modera; moderável.
\section{Moderável}
\begin{itemize}
\item {Grp. gram.:adj.}
\end{itemize}
\begin{itemize}
\item {Proveniência:(Lat. \textunderscore moderabilis\textunderscore )}
\end{itemize}
Que se póde moderar.
\section{Modernadamente}
\begin{itemize}
\item {Grp. gram.:adv.}
\end{itemize}
No tempo moderno; nos últimos tempos.
Na actualidade, hodiernamente.
\section{Modernar}
\textunderscore v. t.\textunderscore  (e der.)
O mesmo que \textunderscore amodernar\textunderscore , etc.
\section{Modernice}
\begin{itemize}
\item {Grp. gram.:f.}
\end{itemize}
\begin{itemize}
\item {Proveniência:(De \textunderscore moderno\textunderscore ^1)}
\end{itemize}
Afêrro a coisas modernas.
Uso exaggerado de coisas novas.
\section{Modernidade}
\begin{itemize}
\item {Grp. gram.:f.}
\end{itemize}
Estado ou qualidade daquillo que é moderno.
\section{Modernismo}
\begin{itemize}
\item {Grp. gram.:m.}
\end{itemize}
O mesmo que \textunderscore modernice\textunderscore .
Opinião e partido dos Cathólicos modernistas.
\section{Modernista}
\begin{itemize}
\item {Grp. gram.:m.  e  f.}
\end{itemize}
\begin{itemize}
\item {Grp. gram.:M.  e  adj.}
\end{itemize}
\begin{itemize}
\item {Proveniência:(De \textunderscore moderno\textunderscore ^1)}
\end{itemize}
Pessôa apaixonada pelo modernismo.
Diz-se dos Cathólicos, que pretendem a reforma do Catholicismo, de acôrdo com as condições da sociedade moderna.
\section{Modernização}
\begin{itemize}
\item {Grp. gram.:f.}
\end{itemize}
Acto de modernizar.
\section{Modernizar}
\begin{itemize}
\item {Grp. gram.:v. t.}
\end{itemize}
Tornar moderno, acommodar aos usos modernos.
\section{Moderno}
\begin{itemize}
\item {Grp. gram.:adj.}
\end{itemize}
\begin{itemize}
\item {Grp. gram.:M. pl.}
\end{itemize}
\begin{itemize}
\item {Proveniência:(Lat. \textunderscore modernus\textunderscore )}
\end{itemize}
Relativo aos tempos mais próximos.
Recente; usado desde pouco tempo.
Actual, hodierno.
Os homens de hoje.
\section{Moderno}
\begin{itemize}
\item {Grp. gram.:adj.}
\end{itemize}
\begin{itemize}
\item {Utilização:Açor}
\end{itemize}
\begin{itemize}
\item {Utilização:Prov.}
\end{itemize}
\begin{itemize}
\item {Utilização:beir.}
\end{itemize}
\begin{itemize}
\item {Proveniência:(De \textunderscore modo\textunderscore ? Cp. \textunderscore moderar\textunderscore )}
\end{itemize}
Moderado, brando.
Sossegado.
Calado.
\section{Modestaço}
\begin{itemize}
\item {Grp. gram.:adj.}
\end{itemize}
\begin{itemize}
\item {Proveniência:(De \textunderscore modesto\textunderscore )}
\end{itemize}
Que alardeia ou figura modéstia. Cf. Camillo, \textunderscore Cav. em Ruínas\textunderscore , 113.
\section{Modestamente}
\begin{itemize}
\item {Grp. gram.:adv.}
\end{itemize}
De modo modesto.
Sem vaidade; sem luxo; sem ostentação.
Com pudôr.
\section{Modéstia}
\begin{itemize}
\item {Grp. gram.:f.}
\end{itemize}
\begin{itemize}
\item {Proveniência:(Lat. \textunderscore modestia\textunderscore )}
\end{itemize}
Qualidade de modesto.
Desambição.
Simplicidade.
\section{Modesto}
\begin{itemize}
\item {Grp. gram.:adj.}
\end{itemize}
\begin{itemize}
\item {Proveniência:(Lat. \textunderscore modestus\textunderscore )}
\end{itemize}
Moderado nos desejos ou aspirações.
Que procede ou se apresenta despretensiosamente, sem vaidade.
Commedido.
Decente, pudico, honesto: \textunderscore usar traje modesto\textunderscore .
Medíocre; parco.
Que revela poucos haveres.
Opposto ao luxo.
\section{Modicamente}
\begin{itemize}
\item {Grp. gram.:adv.}
\end{itemize}
De modo módico; mediocremente.
Com muita economia.
\section{Modicar}
\begin{itemize}
\item {Grp. gram.:v. t.}
\end{itemize}
Tornar módico; moderar, limitar.
\section{Modicidade}
\begin{itemize}
\item {Grp. gram.:f.}
\end{itemize}
\begin{itemize}
\item {Proveniência:(Lat. \textunderscore modicitas\textunderscore )}
\end{itemize}
Qualidade do que é módico.
\section{Módico}
\begin{itemize}
\item {Grp. gram.:adj.}
\end{itemize}
\begin{itemize}
\item {Proveniência:(Lat. \textunderscore modicus\textunderscore )}
\end{itemize}
Exíguo, pequeno, insignificante: \textunderscore preços módicos\textunderscore .
Modesto.
Económico, parco.
\section{Modificação}
\begin{itemize}
\item {Grp. gram.:f.}
\end{itemize}
\begin{itemize}
\item {Proveniência:(Lat. \textunderscore modificatio\textunderscore )}
\end{itemize}
Acto ou effeito de modificar.
\section{Modificador}
\begin{itemize}
\item {Grp. gram.:m.  e  adj.}
\end{itemize}
\begin{itemize}
\item {Proveniência:(Lat. \textunderscore modificator\textunderscore )}
\end{itemize}
O que modifica.
\section{Modificar}
\begin{itemize}
\item {Grp. gram.:v. t.}
\end{itemize}
\begin{itemize}
\item {Proveniência:(Lat. \textunderscore modificare\textunderscore )}
\end{itemize}
Moderar; refrear.
Dar fórma nova a.
Alterar o modo de sêr a.
Alterar, restringindo ou ampliando.
\section{Modificativo}
\begin{itemize}
\item {Grp. gram.:adj.}
\end{itemize}
Que modifica.
\section{Modilhão}
\begin{itemize}
\item {Grp. gram.:m.}
\end{itemize}
\begin{itemize}
\item {Proveniência:(Do it. \textunderscore modiglione\textunderscore )}
\end{itemize}
Ornato architectónico, em fórma de um S invertido.
\section{Modilhar}
\begin{itemize}
\item {Grp. gram.:v. i.}
\end{itemize}
Cantar modilhos.
\section{Modilho}
\begin{itemize}
\item {Grp. gram.:m.}
\end{itemize}
\begin{itemize}
\item {Grp. gram.:Adj.}
\end{itemize}
\begin{itemize}
\item {Proveniência:(De \textunderscore moda\textunderscore )}
\end{itemize}
Música ligeira, ária.
Que observa exaggeradamente as modas.
\section{Modinatura}
\begin{itemize}
\item {Grp. gram.:f.}
\end{itemize}
Conjunto das differentes molduras de uma construcção, segundo o carácter das ordens de architectura.
(Cp. it. \textunderscore modanatura\textunderscore )
\section{Modinha}
\begin{itemize}
\item {Grp. gram.:f.}
\end{itemize}
\begin{itemize}
\item {Utilização:Bras}
\end{itemize}
O mesmo que \textunderscore modilho\textunderscore .
Cantiga sentimental ou triste.--No Brasil, a cantiga alegre é o lundu.
\section{Modinho}
\begin{itemize}
\item {Grp. gram.:adv.}
\end{itemize}
\begin{itemize}
\item {Utilização:Prov.}
\end{itemize}
\begin{itemize}
\item {Utilização:minh.}
\end{itemize}
\begin{itemize}
\item {Proveniência:(De \textunderscore modo\textunderscore )}
\end{itemize}
Com cuidado, com cautela; devagar; com jeito, com modos.
\section{Módio}
\begin{itemize}
\item {Grp. gram.:m.}
\end{itemize}
\begin{itemize}
\item {Proveniência:(Lat. \textunderscore modius\textunderscore )}
\end{itemize}
Antiga medida de capacidade, entre os Romanos, que equivalia proximamente ao alqueire.
Antiga medida agrária, também conhecida por \textunderscore mina\textunderscore .
\section{Modíola}
\begin{itemize}
\item {Grp. gram.:f.}
\end{itemize}
Gênero de plantas malváceas.
\section{Modíolo}
\begin{itemize}
\item {Grp. gram.:m.}
\end{itemize}
\begin{itemize}
\item {Utilização:Constr.}
\end{itemize}
\begin{itemize}
\item {Proveniência:(Lat. \textunderscore modiolus\textunderscore )}
\end{itemize}
Espaço entre os modilhões.
\section{Modismo}
\begin{itemize}
\item {Grp. gram.:m.}
\end{itemize}
\begin{itemize}
\item {Proveniência:(De \textunderscore modo\textunderscore )}
\end{itemize}
Modo de falar, admittido pelo uso, mas que parece opposto ás regras grammaticaes da respectiva língua; idiotismo.
\section{Modista}
\begin{itemize}
\item {Grp. gram.:f.}
\end{itemize}
\begin{itemize}
\item {Proveniência:(De \textunderscore moda\textunderscore )}
\end{itemize}
Mulher, que tem por offício fazer vestuários de senhoras e crianças, ou que dirige a feitura dêlles.
\section{Modisto}
\begin{itemize}
\item {Grp. gram.:m.}
\end{itemize}
\begin{itemize}
\item {Utilização:Neol.}
\end{itemize}
Marido ou companheiro de modista.
\section{Modo}
\begin{itemize}
\item {Grp. gram.:m.}
\end{itemize}
\begin{itemize}
\item {Utilização:Gram.}
\end{itemize}
\begin{itemize}
\item {Utilização:Mús.}
\end{itemize}
\begin{itemize}
\item {Utilização:Prov.}
\end{itemize}
\begin{itemize}
\item {Utilização:minh.}
\end{itemize}
\begin{itemize}
\item {Grp. gram.:Loc. adv.}
\end{itemize}
\begin{itemize}
\item {Grp. gram.:Loc. adv.}
\end{itemize}
\begin{itemize}
\item {Grp. gram.:Pl.}
\end{itemize}
\begin{itemize}
\item {Grp. gram.:Loc. conj.}
\end{itemize}
\begin{itemize}
\item {Utilização:Fam.}
\end{itemize}
\begin{itemize}
\item {Proveniência:(Lat. \textunderscore modus\textunderscore )}
\end{itemize}
Maneira de sêr.
Fórma, méthodo.
Maneira particular de fazer qualquer coisa.
Fórma de dizer.
Qualidade.
Habilidade.
Prática.
Cada uma das fórmas verbaes, que se empregam, não para exprimir os tempos, mas os differentes pontos de vista ou as differentes maneiras, sob que se considera a acção.
Cada uma das differentes fórmas, sob que se apresenta uma substância.
Ordem, por que se sucedem, na escala diatónia, os tons e os meios tons.
Pão migado, que assenta no fundo da malga do caldo, e que também se diz \textunderscore moda\textunderscore .
Maneira de tratar, trato social: \textunderscore o Reinaldo tem bom modo\textunderscore .
\textunderscore A modo\textunderscore , devagar, com jeito. Cf. Castilho, \textunderscore Avarento\textunderscore , 69, 85 e 179.
\textunderscore Sôbre modo\textunderscore , extraordinariamente; com excesso.
Maneira de viver ou de tratar com os outros.
Moderação; discrição.
\textunderscore A modos que\textunderscore , parece que.
\section{Modorra}
\begin{itemize}
\item {fónica:dô}
\end{itemize}
\begin{itemize}
\item {Grp. gram.:f.}
\end{itemize}
\begin{itemize}
\item {Utilização:Fig.}
\end{itemize}
\begin{itemize}
\item {Utilização:Náut.}
\end{itemize}
\begin{itemize}
\item {Utilização:ant.}
\end{itemize}
Prostração mórbida; grande vontade de dormir.
Somnolência.
Doença do gado ovelhum.
Apathia; indolência; insensibilidade.
Terceira vigia da noite. Cf. Franc. Manuel, \textunderscore Epanáf.\textunderscore ; \textunderscore Rot. de D. João de Castro\textunderscore , (\textunderscore passim\textunderscore ), etc.
(Cast. \textunderscore modorra\textunderscore )
\section{Modorra}
\begin{itemize}
\item {fónica:dô}
\end{itemize}
\begin{itemize}
\item {Grp. gram.:f.}
\end{itemize}
\begin{itemize}
\item {Utilização:Ant.}
\end{itemize}
Monte de pedras miúdas ou de cascalho.
Túmulo romano, (\textunderscore ager sepulcralis\textunderscore ).
(Por \textunderscore medorra\textunderscore , de \textunderscore médo\textunderscore )
\section{Modorral}
\begin{itemize}
\item {Grp. gram.:adj.}
\end{itemize}
Que produz modorra^1.
\section{Modorrar}
\begin{itemize}
\item {Grp. gram.:v. t.}
\end{itemize}
\begin{itemize}
\item {Grp. gram.:V. i.}
\end{itemize}
Tornar somnolento; atordoar.
Estar em modorra.
\section{Modorrento}
\begin{itemize}
\item {Grp. gram.:adj.}
\end{itemize}
\begin{itemize}
\item {Utilização:Fig.}
\end{itemize}
Que tem modorra^1.
Estúpido.
\section{Modorro}
\begin{itemize}
\item {fónica:dô}
\end{itemize}
\begin{itemize}
\item {Grp. gram.:adj.}
\end{itemize}
O mesmo que \textunderscore modorrento\textunderscore .
\section{Modulação}
\begin{itemize}
\item {Grp. gram.:f.}
\end{itemize}
\begin{itemize}
\item {Utilização:Mús.}
\end{itemize}
\begin{itemize}
\item {Utilização:Gram.}
\end{itemize}
\begin{itemize}
\item {Utilização:Fig.}
\end{itemize}
\begin{itemize}
\item {Proveniência:(Lat. \textunderscore modulatio\textunderscore )}
\end{itemize}
Acto ou effeito de modular.
Passagem de um modo ou tom para outro.
Facilidade ou habilidade, com que se realiza essa passagem.
Cada um dos differentes valores, que se pódem dar ás vogaes \textunderscore a\textunderscore , \textunderscore e\textunderscore , \textunderscore o\textunderscore .
Melodia, suavidade.
\section{Modulador}
\begin{itemize}
\item {Grp. gram.:m.  e  adj.}
\end{itemize}
\begin{itemize}
\item {Proveniência:(Lat. \textunderscore modulator\textunderscore )}
\end{itemize}
O que modula.
\section{Modulagem}
\begin{itemize}
\item {Grp. gram.:f.}
\end{itemize}
O mesmo que \textunderscore modulação\textunderscore .
\section{Modular}
\begin{itemize}
\item {Grp. gram.:v. t.}
\end{itemize}
\begin{itemize}
\item {Proveniência:(Lat. \textunderscore modulari\textunderscore )}
\end{itemize}
Cantar ou tocar, variando de tom, segundo as regras da harmonia.
Dizer, cantar ou recitar melodiosamente.
\section{Módulo}
\begin{itemize}
\item {Grp. gram.:m.}
\end{itemize}
\begin{itemize}
\item {Utilização:Ext.}
\end{itemize}
\begin{itemize}
\item {Utilização:Mathem.}
\end{itemize}
\begin{itemize}
\item {Proveniência:(Lat. \textunderscore modulus\textunderscore )}
\end{itemize}
Medida proporcional, para avaliar as construcções architectónicas, ou partes de uma construcção, nas relações que estas devem conservar entre si.
Diâmetro de medalha.
Quantidade, que se toma como unidade de qualquer medida.
Tudo que serve para medir.
Modulação.
Quantidade, pela qual é preciso multiplicar os logarithmos de certo systema, para obter os logarithmos correspondentes em outro systema.
\section{Módulo}
\begin{itemize}
\item {Grp. gram.:adj.}
\end{itemize}
\begin{itemize}
\item {Proveniência:(De \textunderscore modular\textunderscore )}
\end{itemize}
Melodioso.
\section{Modumbiro}
\begin{itemize}
\item {Grp. gram.:m.}
\end{itemize}
Árvore de Angola.
\section{Moeção}
\begin{itemize}
\item {fónica:mo-e}
\end{itemize}
\begin{itemize}
\item {Grp. gram.:f.}
\end{itemize}
\begin{itemize}
\item {Utilização:Neol.}
\end{itemize}
O mesmo que \textunderscore moedura\textunderscore . Cf. \textunderscore Portugalia\textunderscore , I, 387.
\section{Moéda}
\begin{itemize}
\item {Grp. gram.:f.}
\end{itemize}
\begin{itemize}
\item {Utilização:Fig.}
\end{itemize}
\begin{itemize}
\item {Proveniência:(Do lat. \textunderscore moneta\textunderscore )}
\end{itemize}
Peça de metal ou de outra substância, cunhada por autoridade soberana, e representativa do valor dos objectos que por ella se trocam.
Nome privativo de uma antiga moéda portuguesa, do valor de 4.800 reis.
Estabelecimento, onde se fabríca moéda por conta do Estado: \textunderscore aquelle rapaz é empregado na Moéda\textunderscore .
\textunderscore Papel moéda\textunderscore , papel que, por determinação official, serve de moéda.
Tudo que tem valor moral ou intellectual.
\textunderscore Pagar na mesma moéda\textunderscore , corresponder, retorquir no mesmo tom; proceder em conformidade da acção recebida.
\section{Moedagem}
\begin{itemize}
\item {fónica:mo-e}
\end{itemize}
\begin{itemize}
\item {Grp. gram.:f.}
\end{itemize}
Arte de fabricar moéda.
Aquillo que se paga pela fabricação da moéda.
\section{Moedeira}
\begin{itemize}
\item {fónica:mo-e}
\end{itemize}
\begin{itemize}
\item {Grp. gram.:f.}
\end{itemize}
\begin{itemize}
\item {Utilização:Fig.}
\end{itemize}
\begin{itemize}
\item {Proveniência:(De \textunderscore moer\textunderscore )}
\end{itemize}
Instrumento, para moer o esmalte, em ourivezaria.
Fadiga, cansaço; trabalho que fatiga.
\section{Moedeiro}
\begin{itemize}
\item {fónica:mo-e}
\end{itemize}
\begin{itemize}
\item {Grp. gram.:m.}
\end{itemize}
Aquelle que fabríca moéda.
\section{Moedela}
\begin{itemize}
\item {fónica:mo-e}
\end{itemize}
\begin{itemize}
\item {Grp. gram.:f.}
\end{itemize}
\begin{itemize}
\item {Proveniência:(De \textunderscore moer\textunderscore )}
\end{itemize}
Acto de ser moído com pancadas; sova.
\section{Moedor}
\begin{itemize}
\item {fónica:mo-e}
\end{itemize}
\begin{itemize}
\item {Grp. gram.:m.  e  adj.}
\end{itemize}
Aquelle que mói; impertinente, importuno.
\section{Moedura}
\begin{itemize}
\item {fónica:mo-e}
\end{itemize}
\begin{itemize}
\item {Grp. gram.:f.}
\end{itemize}
\begin{itemize}
\item {Utilização:Prov.}
\end{itemize}
\begin{itemize}
\item {Utilização:Prov.}
\end{itemize}
\begin{itemize}
\item {Proveniência:(Do lat. \textunderscore molitura\textunderscore )}
\end{itemize}
Acto ou effeito de moer; moagem.
Porção de azeitonas, que entram, de cada vez, na vasa, para se moerem.
Pancadaria; sova, tunda.
\section{Moéga}
\begin{itemize}
\item {Grp. gram.:f.}
\end{itemize}
\begin{itemize}
\item {Utilização:Bras}
\end{itemize}
\begin{itemize}
\item {Proveniência:(De \textunderscore moer\textunderscore )}
\end{itemize}
Peça de moínho, o mesmo que \textunderscore canoira\textunderscore .
Um dos depósitos do trapiche.
\section{Moeira}
\begin{itemize}
\item {Grp. gram.:f.}
\end{itemize}
\begin{itemize}
\item {Proveniência:(De \textunderscore mão\textunderscore )}
\end{itemize}
Um dos dois cabos, que seguram as extremidades do eixo do círcio.
Cabo do mangoal, ou mango.
\section{Moeiro}
\begin{itemize}
\item {Grp. gram.:m.}
\end{itemize}
Utensílio de marnoteiro, do feitio da espada.
(Cp. \textunderscore moeira\textunderscore )
\section{Moéla}
\begin{itemize}
\item {Grp. gram.:f.}
\end{itemize}
\begin{itemize}
\item {Proveniência:(De \textunderscore moer\textunderscore )}
\end{itemize}
Terceiro estômago das aves.
\section{Moéla}
\begin{itemize}
\item {Grp. gram.:f.}
\end{itemize}
\begin{itemize}
\item {Utilização:Ant.}
\end{itemize}
\begin{itemize}
\item {Proveniência:(Fr. \textunderscore moelle\textunderscore )}
\end{itemize}
Medulla, miôlo.
\section{Moélha}
\begin{itemize}
\item {Grp. gram.:f.}
\end{itemize}
\begin{itemize}
\item {Utilização:Ant.}
\end{itemize}
O mesmo que \textunderscore moéda\textunderscore .
\section{Moenda}
\begin{itemize}
\item {Grp. gram.:f.}
\end{itemize}
\begin{itemize}
\item {Utilização:Ext.}
\end{itemize}
\begin{itemize}
\item {Utilização:T. da Bairrada}
\end{itemize}
\begin{itemize}
\item {Proveniência:(De \textunderscore moer\textunderscore )}
\end{itemize}
Peça que mói; mó.
Acto de moer ou triturar.
Retribuição do moleiro em gêneros; maquia.
Moínho.
A taleigada que se dá a moer.
\section{Moendeira}
\begin{itemize}
\item {fónica:mo-en}
\end{itemize}
\begin{itemize}
\item {Grp. gram.:f.}
\end{itemize}
\begin{itemize}
\item {Proveniência:(De \textunderscore moendeiro\textunderscore )}
\end{itemize}
Mulher, que tem moenda.
Mulher de moleiro.
\section{Moendeiro}
\begin{itemize}
\item {fónica:mo-en}
\end{itemize}
\begin{itemize}
\item {Grp. gram.:m.}
\end{itemize}
Moleiro; dono de moenda.
\section{Moengo}
\begin{itemize}
\item {Grp. gram.:f.}
\end{itemize}
(V.moenda)
\section{Moente}
\begin{itemize}
\item {fónica:mo-en}
\end{itemize}
\begin{itemize}
\item {Grp. gram.:adj.}
\end{itemize}
\begin{itemize}
\item {Utilização:Fig.}
\end{itemize}
\begin{itemize}
\item {Grp. gram.:M.}
\end{itemize}
Que mói.
Prompto para qualquer uso ou applicação:«\textunderscore o que vos eu digo darvoloey moente e corrente.\textunderscore »\textunderscore Aulegrafia\textunderscore , 10.
Cavilha ou pequena peça cylíndrica, que gira dentro de um orifício circular.
\section{Moenza}
\begin{itemize}
\item {fónica:mo-en}
\end{itemize}
\begin{itemize}
\item {Grp. gram.:f.}
\end{itemize}
\begin{itemize}
\item {Utilização:Bras}
\end{itemize}
Árvore silvestre, cuja madeira se emprega em tamancos e canôas.
\section{Moêr}
\begin{itemize}
\item {Grp. gram.:v. t.}
\end{itemize}
\begin{itemize}
\item {Utilização:Fig.}
\end{itemize}
\begin{itemize}
\item {Proveniência:(Do lat. \textunderscore molere\textunderscore )}
\end{itemize}
Triturar, esmagar ou reduzir a pó.
Desfazer em partículas como pó.
Comprimir, para tirar o suco de.
Mastigar.
Meditar demoradamente.
Fatigar com trabalho.
Importunar, molestar.
Pisar, derrear com pancadas.
Repisar, repetir.
\section{Moêta}
\begin{itemize}
\item {Grp. gram.:f.}
\end{itemize}
Espécie de tenaz, com dois grandes cabos de madeira, usada em agricultura para escardear. Cf. F. Lapa, \textunderscore Phýs. e Chím.\textunderscore 
(Cf. fr. \textunderscore moettes\textunderscore )
\section{Mofa}
\begin{itemize}
\item {Grp. gram.:f.}
\end{itemize}
Motejo, zombaria.
Objecto de escárneo.
(Do alto al. méd. \textunderscore muppen\textunderscore , arreganhar os dentes por zombaria)
\section{Mofador}
\begin{itemize}
\item {Grp. gram.:adj.}
\end{itemize}
\begin{itemize}
\item {Grp. gram.:M.}
\end{itemize}
\begin{itemize}
\item {Proveniência:(De \textunderscore mofar\textunderscore ^2)}
\end{itemize}
Que mofa.
Que envolve ou significa mofa.
Que zomba.
Aquelle que mofa.
\section{Mofar}
\begin{itemize}
\item {Grp. gram.:v. t.}
\end{itemize}
\begin{itemize}
\item {Grp. gram.:V. i.}
\end{itemize}
Pôr môfo em.
Criar môfo.
\section{Mofar}
\begin{itemize}
\item {Grp. gram.:v. i.}
\end{itemize}
\begin{itemize}
\item {Grp. gram.:V. t.}
\end{itemize}
Fazer mofa, escarnecer, motejar.
Fazer mofa de; troçar:«\textunderscore casquilhos que nos móffão...\textunderscore »Filinto, I, 44.
\section{Mofatra}
\begin{itemize}
\item {Grp. gram.:f.}
\end{itemize}
Trapaça; burla.
Transacção fraudulenta.
(Cast. \textunderscore mohatra\textunderscore , do ár.)
\section{Mofatrão}
\begin{itemize}
\item {Grp. gram.:m.}
\end{itemize}
Aquelle que pratica mofatras.
\section{Mofedo}
\begin{itemize}
\item {fónica:fê}
\end{itemize}
\begin{itemize}
\item {Grp. gram.:m.}
\end{itemize}
\begin{itemize}
\item {Utilização:Prov.}
\end{itemize}
\begin{itemize}
\item {Utilização:beir.}
\end{itemize}
\begin{itemize}
\item {Proveniência:(De \textunderscore môfo\textunderscore )}
\end{itemize}
Excesso de ramagem, que prejudica o desenvolvimento da árvore. (Colhido no Fundão)
\section{Mofendo}
\begin{itemize}
\item {Grp. gram.:adj.}
\end{itemize}
Mofento?:«\textunderscore ...autores dos mofendos attentados.\textunderscore »O'Neill, \textunderscore Fabulário\textunderscore , 38.
\section{Mofento}
\begin{itemize}
\item {Grp. gram.:adj.}
\end{itemize}
\begin{itemize}
\item {Utilização:Fig.}
\end{itemize}
Que tem môfo.
Funesto; que causa infelicidade.
\section{Mofetta}
\begin{itemize}
\item {Grp. gram.:f.}
\end{itemize}
\begin{itemize}
\item {Utilização:Geol.}
\end{itemize}
\begin{itemize}
\item {Proveniência:(T. it.)}
\end{itemize}
Manifestação attenuada da actividade vulcânica, consistindo apenas em exhalações accidentaes ou permanentes de hydrocarbonetos gasosos e anhydrido carbónico.
\section{Mofina}
\begin{itemize}
\item {Grp. gram.:f.}
\end{itemize}
\begin{itemize}
\item {Utilização:Fig.}
\end{itemize}
\begin{itemize}
\item {Utilização:Bras}
\end{itemize}
\begin{itemize}
\item {Proveniência:(De \textunderscore mofino\textunderscore )}
\end{itemize}
Infelicidade.
Mulher infeliz.
Mulher acanhada, tacanha.
Mulher turbulenta.
Avareza.
Artigo anónymo e diffamatório.
\section{Mofinamente}
\begin{itemize}
\item {Grp. gram.:adv.}
\end{itemize}
\begin{itemize}
\item {Proveniência:(De \textunderscore mofino\textunderscore )}
\end{itemize}
Desgraçadamente.
Com mesquinhez.
\section{Mofineiro}
\begin{itemize}
\item {Grp. gram.:m.}
\end{itemize}
\begin{itemize}
\item {Utilização:Bras}
\end{itemize}
Aquelle que escreve artigos anónymos e diffamatórios ou mofinos.
\section{Mofinento}
\begin{itemize}
\item {Grp. gram.:adj.}
\end{itemize}
\begin{itemize}
\item {Proveniência:(De \textunderscore mofina\textunderscore )}
\end{itemize}
Que tem modos de mofino.
Inditoso; aziago. Cf. Camillo, \textunderscore Noites de Insómn.\textunderscore , X, 43.
\section{Mofineza}
\begin{itemize}
\item {Grp. gram.:f.}
\end{itemize}
\begin{itemize}
\item {Utilização:Des.}
\end{itemize}
Qualidade do que é mofino.
\section{Mofino}
\begin{itemize}
\item {Grp. gram.:adj.}
\end{itemize}
\begin{itemize}
\item {Grp. gram.:M.}
\end{itemize}
Infeliz.
Acanhado.
Avarento.
Turbulento.
Aquelle que é mofino.
(Cp. cast. \textunderscore mohino\textunderscore )
\section{Môfo}
\begin{itemize}
\item {Grp. gram.:m.}
\end{itemize}
\begin{itemize}
\item {Utilização:Fam.}
\end{itemize}
\begin{itemize}
\item {Utilização:Prov.}
\end{itemize}
\begin{itemize}
\item {Utilização:trasm.}
\end{itemize}
\begin{itemize}
\item {Proveniência:(Do neerl. \textunderscore muf\textunderscore , bolorento)}
\end{itemize}
Vegetação cryptogâmica, desenvolvida sôbre objectos húmidos, e conhecida vulgarmente por \textunderscore bolor\textunderscore .
Bafio.
Vantagem gratuita, borla.
\textunderscore Fazer môfo\textunderscore , fazer má cara, mostrar pouca vontade (de comer, principalmente)
\section{Mofoso}
\begin{itemize}
\item {Grp. gram.:adj.}
\end{itemize}
O mesmo que \textunderscore mofento\textunderscore .
\section{Mofti}
\begin{itemize}
\item {Grp. gram.:m.}
\end{itemize}
O mesmo ou melhor que \textunderscore mufti\textunderscore .
\section{Mofumbo}
\begin{itemize}
\item {Grp. gram.:m.}
\end{itemize}
Planta leguminosa do Brasil.
\section{Mofungo}
\begin{itemize}
\item {Grp. gram.:m.}
\end{itemize}
Planta amaranthácea do Brasil.
\section{Mogaininha}
\begin{itemize}
\item {Grp. gram.:f.}
\end{itemize}
\begin{itemize}
\item {Utilização:T. de Caminha}
\end{itemize}
O mesmo que \textunderscore fagulha\textunderscore .
\section{Moganga}
\begin{itemize}
\item {Grp. gram.:f.  e  adj.}
\end{itemize}
Variedade de abóbora menina.
\section{Mogango}
\begin{itemize}
\item {Grp. gram.:m.}
\end{itemize}
\begin{itemize}
\item {Utilização:T. do Redondo}
\end{itemize}
O mesmo que \textunderscore moganga\textunderscore .
\section{Mogango}
\begin{itemize}
\item {Grp. gram.:m.}
\end{itemize}
\begin{itemize}
\item {Utilização:Bras. do N}
\end{itemize}
Trejeito, esgar.
\section{Mogangueiro}
\begin{itemize}
\item {Grp. gram.:m.}
\end{itemize}
O mesmo que \textunderscore moquenqueiro\textunderscore .
\section{Moganguice}
\begin{itemize}
\item {Grp. gram.:f.}
\end{itemize}
\begin{itemize}
\item {Proveniência:(Do ár. \textunderscore gonj\textunderscore , segundo Dozy)}
\end{itemize}
O mesmo ou melhor que \textunderscore moquenquice\textunderscore .
\section{Moganguista}
\begin{itemize}
\item {Grp. gram.:m.  e  adj.}
\end{itemize}
\begin{itemize}
\item {Utilização:Bras. do N}
\end{itemize}
O que faz mogangos ou trejeitos.
\section{Mogão}
\begin{itemize}
\item {Grp. gram.:adj.}
\end{itemize}
Diz-se do toiro, cujas hastes não têm pontas.
\section{Mogarabil}
\begin{itemize}
\item {Grp. gram.:m.}
\end{itemize}
\begin{itemize}
\item {Utilização:Ant.}
\end{itemize}
Negociante; mercador.
\section{Mogataces}
\begin{itemize}
\item {Grp. gram.:m. pl.}
\end{itemize}
Soldados de cavallaria, indígenas, que constituem guarnição de alguns presídios espanhóes em África.
\section{Mogenifada}
\begin{itemize}
\item {Grp. gram.:m.}
\end{itemize}
\begin{itemize}
\item {Utilização:Ant.}
\end{itemize}
O mesmo que \textunderscore moxinifada\textunderscore :«\textunderscore fazem lá umas mogenifadas de misturadas, de águas, de óleos...\textunderscore »A. Ferreira, \textunderscore Cioso\textunderscore , act. III, sc. 1.
\section{Mógi}
\begin{itemize}
\item {Grp. gram.:m.}
\end{itemize}
Arbusto africano, trepador.
\section{Mogiganga}
\begin{itemize}
\item {Grp. gram.:f.}
\end{itemize}
Dança burlesca.
Bugiganga.
Momices.
(Cast. \textunderscore mojiganga\textunderscore )
\section{Mogigrafia}
\begin{itemize}
\item {Grp. gram.:f.}
\end{itemize}
\begin{itemize}
\item {Utilização:Med.}
\end{itemize}
\begin{itemize}
\item {Proveniência:(Do gr. \textunderscore mogis\textunderscore  + \textunderscore graphein\textunderscore )}
\end{itemize}
Dificuldade ou impossibilidade, que têm certos músculos dos dedos polegar e indicador, de segurar e dirigir a penna de escrever.
\section{Mogigraphia}
\begin{itemize}
\item {Grp. gram.:f.}
\end{itemize}
\begin{itemize}
\item {Utilização:Med.}
\end{itemize}
\begin{itemize}
\item {Proveniência:(Do gr. \textunderscore mogis\textunderscore  + \textunderscore graphein\textunderscore )}
\end{itemize}
Difficuldade ou impossibilidade, que têm certos músculos dos dedos pollegar e indicador, de segurar e dirigir a penna de escrever.
\section{Mogilalismo}
\begin{itemize}
\item {Grp. gram.:m.}
\end{itemize}
\begin{itemize}
\item {Utilização:Med.}
\end{itemize}
\begin{itemize}
\item {Proveniência:(Do gr. \textunderscore mogis\textunderscore , difficilmente, e \textunderscore lalein\textunderscore , falar)}
\end{itemize}
Vício prosódico ou gaguez na pronúncia do \textunderscore p\textunderscore  e do \textunderscore b\textunderscore .
Difficuldade em articular as palavras; gaguez.
\section{Mogislalismo}
\begin{itemize}
\item {Grp. gram.:m.}
\end{itemize}
\begin{itemize}
\item {Utilização:Med.}
\end{itemize}
\begin{itemize}
\item {Proveniência:(Do gr. \textunderscore mogis\textunderscore , difficilmente, e \textunderscore lalein\textunderscore , falar)}
\end{itemize}
Vício prosódico ou gaguez na pronúncia do \textunderscore p\textunderscore  e do \textunderscore b\textunderscore .
Difficuldade em articular as palavras; gaguez.
\section{Mogno}
\begin{itemize}
\item {Grp. gram.:m.}
\end{itemize}
Árvore cedrelácea da América tropical, cuja madeira é muito conhecida e apreciada em marcenaria.
O mesmo que \textunderscore acaju\textunderscore .
(Talvez do ingl. \textunderscore mahogony\textunderscore , de or. amer.)
\section{Mogo}
\begin{itemize}
\item {Grp. gram.:m.}
\end{itemize}
\begin{itemize}
\item {Proveniência:(Do lat. \textunderscore monachus\textunderscore  &lt; \textunderscore móago\textunderscore  &lt; \textunderscore móogo\textunderscore  &lt; \textunderscore mogo\textunderscore )}
\end{itemize}
Marco, para limite ou extrema de terrenos.
\section{Mogol}
\begin{itemize}
\item {Grp. gram.:m.  e  adj.}
\end{itemize}
\begin{itemize}
\item {Proveniência:(T. turco)}
\end{itemize}
O mesmo que \textunderscore mongol\textunderscore , (falando-se especialmente da dominação mongólica, na Índia setentrional).
\textunderscore Grão mogol\textunderscore , título do imperador de Mogol.
\section{Mógono}
\begin{itemize}
\item {Grp. gram.:m.}
\end{itemize}
Árvore cedrelácea da América tropical, cuja madeira é muito conhecida e apreciada em marcenaria.
O mesmo que \textunderscore acaju\textunderscore .
(Talvez do ingl. \textunderscore mahogony\textunderscore , de or. amer.)
\section{Mógono}
\begin{itemize}
\item {Grp. gram.:m.}
\end{itemize}
(Fórma preferível a \textunderscore mogno\textunderscore )
\section{Mogór}
\begin{itemize}
\item {Grp. gram.:m.  e  adj.}
\end{itemize}
O mesmo que \textunderscore mongol\textunderscore .
\section{Mogóres}
\begin{itemize}
\item {Grp. gram.:m. pl.}
\end{itemize}
\begin{itemize}
\item {Utilização:Ant.}
\end{itemize}
Povos do império Mogol.
\section{Mogorim}
\begin{itemize}
\item {Grp. gram.:adj.}
\end{itemize}
\begin{itemize}
\item {Utilização:Bras}
\end{itemize}
Espécie de rosa branca, muito aromática.
\section{Mogosigue}
\begin{itemize}
\item {Grp. gram.:m.}
\end{itemize}
Árvore angolense, da fam. das verbenáceas.
\section{Mogueixo}
\begin{itemize}
\item {Grp. gram.:m.}
\end{itemize}
\begin{itemize}
\item {Utilização:T. de Avis}
\end{itemize}
\begin{itemize}
\item {Proveniência:(De \textunderscore mogo\textunderscore )}
\end{itemize}
Pequena pedra.
\section{Moguino}
\begin{itemize}
\item {Grp. gram.:adj.}
\end{itemize}
\begin{itemize}
\item {Utilização:Prov.}
\end{itemize}
\begin{itemize}
\item {Utilização:alent.}
\end{itemize}
Diz-se do animal, que tem a aresta do cachaço ou a região da crina um pouco inclinada para o lado.
\section{Mohádi}
\begin{itemize}
\item {Grp. gram.:m.}
\end{itemize}
Sacerdote muçulmano. Cf. Herculano, \textunderscore Hist. de Port.\textunderscore 
\section{Mohametano}
\begin{itemize}
\item {Grp. gram.:m.  e  adj.}
\end{itemize}
\begin{itemize}
\item {Proveniência:(De \textunderscore Mohamet\textunderscore , n. p.)}
\end{itemize}
O mesmo que \textunderscore mahometano\textunderscore . Cf. Herculano, \textunderscore Eurico\textunderscore , 93 e 115.
\section{Mohametismo}
\begin{itemize}
\item {Grp. gram.:m.}
\end{itemize}
O mesmo que \textunderscore mahometismo\textunderscore . Cf. Herculano, \textunderscore Hist. de Port.\textunderscore , III, 177.
\section{Moharrão}
\begin{itemize}
\item {Grp. gram.:m.}
\end{itemize}
Primeiro mês do anno mahometano, mês em que é prohibido pegar em armas e fazer guerra. Cf. \textunderscore Monarchia Lusit.\textunderscore , t. II, 271.
(Do ár.)
\section{Mohatra}
\begin{itemize}
\item {Grp. gram.:m.}
\end{itemize}
\begin{itemize}
\item {Utilização:Des.}
\end{itemize}
O mesmo que \textunderscore mofatra\textunderscore . Cf. F. Borges, \textunderscore Diccion. Jur.\textunderscore 
\section{Mohipua}
\begin{itemize}
\item {Grp. gram.:f.}
\end{itemize}
Árvore de Angola.
\section{Mohógono}
\begin{itemize}
\item {Grp. gram.:m.}
\end{itemize}
(V.mogno)
\section{Móia}
\begin{itemize}
\item {Grp. gram.:f.}
\end{itemize}
\begin{itemize}
\item {Utilização:T. de Pare -de-Coira}
\end{itemize}
\begin{itemize}
\item {Utilização:des.}
\end{itemize}
\begin{itemize}
\item {Proveniência:(De \textunderscore moer\textunderscore )}
\end{itemize}
Sova, tareia, tunda.
\section{Moiação}
\begin{itemize}
\item {Grp. gram.:f.}
\end{itemize}
Antiga pensão de frutos, equivalente a certo número de moios ou a parte de um moio.
\section{Moião}
\begin{itemize}
\item {Grp. gram.:m.}
\end{itemize}
\begin{itemize}
\item {Utilização:Prov.}
\end{itemize}
\begin{itemize}
\item {Utilização:minh.}
\end{itemize}
O mesmo que \textunderscore meão\textunderscore  (do carro).
\section{Moição}
\begin{itemize}
\item {fónica:moi}
\end{itemize}
\begin{itemize}
\item {Grp. gram.:f.}
\end{itemize}
\begin{itemize}
\item {Utilização:Prov.}
\end{itemize}
\begin{itemize}
\item {Utilização:alg.}
\end{itemize}
O mesmo que \textunderscore moedeira\textunderscore .
\section{Moico}
\begin{itemize}
\item {Grp. gram.:adj.}
\end{itemize}
\begin{itemize}
\item {Utilização:Prov.}
\end{itemize}
\begin{itemize}
\item {Utilização:trasm.}
\end{itemize}
Diz-se do boi, a que falta um dos galhos ou ambos.
(Cp. \textunderscore mocho\textunderscore )
\section{Moiçó}
\begin{itemize}
\item {Grp. gram.:f.}
\end{itemize}
\begin{itemize}
\item {Utilização:Prov.}
\end{itemize}
O mesmo que \textunderscore moéla\textunderscore ^1.
\section{Moído}
\begin{itemize}
\item {Grp. gram.:adj.}
\end{itemize}
\begin{itemize}
\item {Proveniência:(De \textunderscore moer\textunderscore )}
\end{itemize}
Cansado, fatigado; importunado.
\section{Moiene}
\begin{itemize}
\item {Grp. gram.:adj.}
\end{itemize}
\begin{itemize}
\item {Utilização:Gír.}
\end{itemize}
Meu.
\section{Moimbaimbai}
\begin{itemize}
\item {Grp. gram.:m.}
\end{itemize}
Árvore da África meridional.
\section{Moimento}
\begin{itemize}
\item {Grp. gram.:m.}
\end{itemize}
\begin{itemize}
\item {Utilização:Ext.}
\end{itemize}
\begin{itemize}
\item {Proveniência:(Do lat. \textunderscore monimentum\textunderscore )}
\end{itemize}
Monumento fúnebre.
Monumento em honra de alguém.
\section{Moìmento}
\begin{itemize}
\item {fónica:mo-i}
\end{itemize}
\begin{itemize}
\item {Grp. gram.:m.}
\end{itemize}
\begin{itemize}
\item {Utilização:Fig.}
\end{itemize}
\begin{itemize}
\item {Proveniência:(De \textunderscore moer\textunderscore )}
\end{itemize}
O mesmo que \textunderscore moedura\textunderscore .
Abatimento de fôrças; prostração.
\section{Móina}
\begin{itemize}
\item {Grp. gram.:f.}
\end{itemize}
\begin{itemize}
\item {Utilização:Prov.}
\end{itemize}
\begin{itemize}
\item {Utilização:T. de Turquel}
\end{itemize}
\begin{itemize}
\item {Utilização:Gír.}
\end{itemize}
\begin{itemize}
\item {Grp. gram.:M.}
\end{itemize}
\begin{itemize}
\item {Utilização:Gír. de Lisbôa.}
\end{itemize}
Subscripção com pequenas quantias.
Vida airada.
\textunderscore Andar á moina\textunderscore , pedir esmola.
Vadio, que frequenta os tribunaes e explora com conselhos ou promessas os ingênuos que alli têm negócios pendentes.
(Cp. \textunderscore moinar\textunderscore )
\section{Moinante}
\begin{itemize}
\item {Grp. gram.:adj.}
\end{itemize}
\begin{itemize}
\item {Proveniência:(De \textunderscore móina\textunderscore )}
\end{itemize}
Brincalhão; festeiro.
Mandrião; malandro, vadio.
\section{Moinar}
\begin{itemize}
\item {Grp. gram.:v. i.}
\end{itemize}
\begin{itemize}
\item {Utilização:Gír.}
\end{itemize}
Dormir.
\section{Moínha}
\begin{itemize}
\item {Grp. gram.:f.}
\end{itemize}
\begin{itemize}
\item {Utilização:Fig.}
\end{itemize}
\begin{itemize}
\item {Utilização:Fam.}
\end{itemize}
\begin{itemize}
\item {Utilização:Prov.}
\end{itemize}
\begin{itemize}
\item {Utilização:dur.}
\end{itemize}
\begin{itemize}
\item {Proveniência:(De \textunderscore moer\textunderscore )}
\end{itemize}
Fragmentos de palha, que ficam na eira, em que se debulharam cereaes; alimpadura dos cereaes.
Pó, a que se reduz uma substância sêca ou triturada.
Repetição enfadonha de palavras ou actos.
Dôr fraca, mas aturada, nos dentes.
O mesmo que \textunderscore caruma\textunderscore .
\section{Moinhar}
\begin{itemize}
\item {fónica:mo-i}
\end{itemize}
\begin{itemize}
\item {Grp. gram.:v. i.}
\end{itemize}
\begin{itemize}
\item {Utilização:Neol.}
\end{itemize}
Agitar ou mover as velas (o moínho); molinhar:«\textunderscore os moínhos moinhando ao sol...\textunderscore »Ortigão, \textunderscore Holanda\textunderscore , 123.
\section{Moinheira}
\begin{itemize}
\item {fónica:mo-i}
\end{itemize}
\begin{itemize}
\item {Grp. gram.:f.}
\end{itemize}
\begin{itemize}
\item {Utilização:Ant.}
\end{itemize}
O mesmo que \textunderscore moínho\textunderscore ; molinheira.
\section{Moínho}
\begin{itemize}
\item {Grp. gram.:m.}
\end{itemize}
\begin{itemize}
\item {Utilização:Fig.}
\end{itemize}
\begin{itemize}
\item {Proveniência:(Do lat. \textunderscore molinum\textunderscore )}
\end{itemize}
Engenho para moer cereaes.
Lagar, onde se mói azeitona.
Máquina, com que se tritura qualquer coisa.
Porção de azeitona, que se mói de uma vez.
Azenha.
Pessôa, que come muito.
\section{Moipua}
\begin{itemize}
\item {fónica:mo-i}
\end{itemize}
\begin{itemize}
\item {Grp. gram.:f.}
\end{itemize}
Árvore de Angola.
\section{Moio}
\begin{itemize}
\item {Grp. gram.:m.}
\end{itemize}
\begin{itemize}
\item {Utilização:Prov.}
\end{itemize}
\begin{itemize}
\item {Proveniência:(Do b. lat. \textunderscore modigus\textunderscore )}
\end{itemize}
Antiga medida de capacidade, equivalente a 60 alqueires.
O número de sessenta: \textunderscore já tenho um moio de annos\textunderscore .
\section{Moira}
\begin{itemize}
\item {Grp. gram.:f.}
\end{itemize}
Chinela de cordovão, geralmente clara. Cf. Camillo, \textunderscore Doze Casam.\textunderscore , 17.
Mulher, de procedência moirisca: \textunderscore as moiras encantadas\textunderscore .
\section{Moira}
\begin{itemize}
\item {Grp. gram.:f.}
\end{itemize}
\begin{itemize}
\item {Utilização:Prov.}
\end{itemize}
\begin{itemize}
\item {Utilização:trasm.}
\end{itemize}
\begin{itemize}
\item {Utilização:Açor}
\end{itemize}
\begin{itemize}
\item {Proveniência:(Do lat. \textunderscore muria\textunderscore )}
\end{itemize}
O mesmo que \textunderscore salga\textunderscore ^1 ou \textunderscore salmoira\textunderscore .
Chouriço de sangue; tabafeia.
Espécie de caranguejo pequeno.
\section{Moiradela}
\begin{itemize}
\item {Grp. gram.:f.}
\end{itemize}
Acto de moirar.
\section{Moiradoiro}
\begin{itemize}
\item {Grp. gram.:m.}
\end{itemize}
\begin{itemize}
\item {Proveniência:(De \textunderscore moirar\textunderscore )}
\end{itemize}
Um dos compartimentos das salinas. Cf. \textunderscore Museu Techn.\textunderscore , 79.
\section{Moirajaca}
\begin{itemize}
\item {Grp. gram.:f.}
\end{itemize}
\begin{itemize}
\item {Utilização:Açor}
\end{itemize}
Espécie de pequeno caranguejo.
(Cp. \textunderscore moira\textunderscore ^2)
\section{Moiral}
\begin{itemize}
\item {Grp. gram.:m.}
\end{itemize}
\begin{itemize}
\item {Utilização:Prov.}
\end{itemize}
O mesmo que \textunderscore maioral\textunderscore .
\section{Moirama}
\begin{itemize}
\item {Grp. gram.:f.}
\end{itemize}
Terra de Moiros.
Grande porção de Moiros; os Moiros.
\section{Moirão}
\begin{itemize}
\item {Grp. gram.:m.}
\end{itemize}
\begin{itemize}
\item {Utilização:T. da Bairrada}
\end{itemize}
\begin{itemize}
\item {Utilização:Prov.}
\end{itemize}
\begin{itemize}
\item {Utilização:trasm.}
\end{itemize}
Cada uma das varas grossas, que se fixam verticalmente, na formação de estacadas.
Estaca, em que se empa a videira.
Cada um dos esteios que sustentam a vêrga da chaminé.
O mesmo que \textunderscore trasfogueiro\textunderscore .
\section{Moirar}
\begin{itemize}
\item {Grp. gram.:v. i.}
\end{itemize}
\begin{itemize}
\item {Proveniência:(De \textunderscore moira\textunderscore ^2)}
\end{itemize}
Depor o sal na borda dos caldeirões, (falando-se da água salgada das marinhas).
\section{Moirar}
\begin{itemize}
\item {Grp. gram.:v. i.}
\end{itemize}
Tornar-se Moiro.
Praticar o culto do islamismo.
Trajar ou proceder como os Moiros. Cf. Serpa, \textunderscore Solaus\textunderscore , 154.
\section{Moiraria}
\begin{itemize}
\item {Grp. gram.:f.}
\end{itemize}
Bairro, em que habitavam Moiros.
\section{Moiras}
\begin{itemize}
\item {Grp. gram.:f. pl.}
\end{itemize}
\begin{itemize}
\item {Utilização:Prov.}
\end{itemize}
\begin{itemize}
\item {Utilização:alg.}
\end{itemize}
Papas de milho, com calda de murcellas, acabadas de arranjar.
\section{Moirejado}
\begin{itemize}
\item {Grp. gram.:adj.}
\end{itemize}
\begin{itemize}
\item {Proveniência:(De \textunderscore moirejar\textunderscore )}
\end{itemize}
Conseguido á custa de muito trabalho.
\section{Moirejar}
\begin{itemize}
\item {Grp. gram.:v. i.}
\end{itemize}
\begin{itemize}
\item {Proveniência:(De \textunderscore moiro\textunderscore )}
\end{itemize}
Trabalhar constantemente; lidar.
Lutar pela vida.
\section{Moiresco}
\begin{itemize}
\item {fónica:moirês}
\end{itemize}
\begin{itemize}
\item {Grp. gram.:adj.}
\end{itemize}
\begin{itemize}
\item {Grp. gram.:M. pl.}
\end{itemize}
\begin{itemize}
\item {Proveniência:(De \textunderscore moiro\textunderscore )}
\end{itemize}
Que é da Moirama.
Relativo a Moiros.
Ornatos de ourivezaria.
\section{Moirisca}
\begin{itemize}
\item {Grp. gram.:f.  e  adj.}
\end{itemize}
Variedade de uva preta do Doiro, semelhante á periquita.
\section{Moirisca}
\begin{itemize}
\item {Grp. gram.:f.}
\end{itemize}
\begin{itemize}
\item {Utilização:Açor}
\end{itemize}
\begin{itemize}
\item {Proveniência:(De \textunderscore moirisco\textunderscore )}
\end{itemize}
Pantomima, representação ao ar livre, em trajes apropriados ao assumpto.
Antiga dança de Moiros, verdadeiros ou fingidos.
Espécie de chinela.
\section{Moiriscada}
\begin{itemize}
\item {Grp. gram.:f.}
\end{itemize}
\begin{itemize}
\item {Utilização:Açor}
\end{itemize}
Canção dramática, popular.
\section{Moirisco}
\begin{itemize}
\item {Grp. gram.:adj.}
\end{itemize}
\begin{itemize}
\item {Grp. gram.:M.}
\end{itemize}
Moiresco.
Moiro.
Variedade de uva, o mesmo que \textunderscore moirisca\textunderscore .
Variedade de trigo rijo.
Indivíduo da Moirama. Cf. Filinto, \textunderscore D. Man.\textunderscore , I, 23.
\section{Moirisco-branco}
\begin{itemize}
\item {Grp. gram.:m.}
\end{itemize}
Variedade de uva moirisca.
\section{Moirisma}
\begin{itemize}
\item {Grp. gram.:f.}
\end{itemize}
Religião dos Moiros.
Terra de Moiros.
Moirama.
(Cast. \textunderscore morisma\textunderscore )
\section{Moirismo}
\begin{itemize}
\item {Grp. gram.:m.}
\end{itemize}
\begin{itemize}
\item {Utilização:P. us.}
\end{itemize}
Os Moiros.
\section{Moiro}
\begin{itemize}
\item {Grp. gram.:adj.}
\end{itemize}
\begin{itemize}
\item {Utilização:Bras}
\end{itemize}
\begin{itemize}
\item {Grp. gram.:M.}
\end{itemize}
\begin{itemize}
\item {Utilização:Ext.}
\end{itemize}
\begin{itemize}
\item {Utilização:Fam.}
\end{itemize}
\begin{itemize}
\item {Utilização:Pop.}
\end{itemize}
\begin{itemize}
\item {Proveniência:(Do lat. \textunderscore maurus\textunderscore )}
\end{itemize}
Relativo aos Moiros.
Moirisco.
\textunderscore Cavallo moirisco\textunderscore , cavallo escuro, mesclado de branco.
\textunderscore Chouriço moiro\textunderscore , espécie de morcella, feita com sangue de porco, vinho branco, etc.
Habitante de Mauritânia.
Sarraceno.
Idólatra, infiel.
Espécie de jôgo popular.
Espécie de peixe, da ria de Aveiro.
Homem, que trabalha muito, que labuta constantemente: \textunderscore trabalhar como um moiro\textunderscore ; \textunderscore é um moiro de trabalho\textunderscore .
\textunderscore Vinho moiro\textunderscore , vinho puro, sem mistura de água, em contraposição a \textunderscore vinho christão\textunderscore  ou \textunderscore vinho baptizado\textunderscore , vinho em que misturaram água.
\section{Moiroiço}
\begin{itemize}
\item {Grp. gram.:m.}
\end{itemize}
O mesmo que \textunderscore moroiço\textunderscore .
\section{Moirouço}
\begin{itemize}
\item {Grp. gram.:m.}
\end{itemize}
O mesmo que \textunderscore moroiço\textunderscore .
\section{Moisaico}
\begin{itemize}
\item {Grp. gram.:adj.}
\end{itemize}
Relativo a Moisés.
\section{Moiseísmo}
\begin{itemize}
\item {Grp. gram.:m.}
\end{itemize}
A religião de Moisés.
\section{Moiseísta}
\begin{itemize}
\item {Grp. gram.:m.}
\end{itemize}
Sectário do moiseísmo.
\section{Moiseístico}
\begin{itemize}
\item {Grp. gram.:adj.}
\end{itemize}
Relativo aos moiseístas.
\section{Moisém}
\begin{itemize}
\item {Grp. gram.:m.}
\end{itemize}
\begin{itemize}
\item {Utilização:Ant.}
\end{itemize}
Intimação judicial para comparecimento em dia certo.
\section{Moita}
\begin{itemize}
\item {Grp. gram.:f.}
\end{itemize}
\begin{itemize}
\item {Utilização:Prov.}
\end{itemize}
Conjunto espêsso de plantas arborescentes.
Conjunto de castanheiros novos, que nasceram e cresceram bastos e que geralmente se applicam a corras de cesteiro e a varas com que se derruba a azeitona.
(Os outros diccion. relacionam o t. com \textunderscore mata\textunderscore ; creio porém que, assim como o lat. \textunderscore multus\textunderscore  deu o port. \textunderscore muito\textunderscore , o lat. \textunderscore multa\textunderscore , de \textunderscore multus\textunderscore , podia dar o port. \textunderscore moita\textunderscore )
\section{Moita!}
\begin{itemize}
\item {Grp. gram.:interj.}
\end{itemize}
Designa que nada se respondeu, quando se esperava ou pedia resposta:«\textunderscore interrogaram-no sôbre todos os pontos, mas elle... moita!\textunderscore »Cf. Castilho, \textunderscore Fausto\textunderscore , 310.
\section{Moita-carrasco!}
\begin{itemize}
\item {Grp. gram.:interj.}
\end{itemize}
O mesmo que \textunderscore moita!\textunderscore ^2
\section{Moitão}
\begin{itemize}
\item {Grp. gram.:m.}
\end{itemize}
Peça metállica ou de madeira, em fórma de ellipse, e atravessada por um eixo, cercada de goivadura, onde se introduz uma alça, e destinada a levantar pesos.
Cadernal.--A bordo, há mais de uma espécie de moitões.
(Cp. cast. \textunderscore moutón\textunderscore )
\section{Moitão}
\begin{itemize}
\item {Grp. gram.:m.}
\end{itemize}
O mesmo que \textunderscore moitedo\textunderscore .
\section{Moitedo}
\begin{itemize}
\item {fónica:tê}
\end{itemize}
\begin{itemize}
\item {Grp. gram.:m.}
\end{itemize}
Lugar onde há moitas.
\section{Moiteira}
\begin{itemize}
\item {Grp. gram.:f.}
\end{itemize}
Moita extensa.
\section{Mojangué}
\begin{itemize}
\item {Grp. gram.:m.}
\end{itemize}
Iguaria brasileira, em que entra milho verde. Cf. Rebello, \textunderscore Mocidade\textunderscore , I, 98; III, 50.
\section{Mojica}
\begin{itemize}
\item {Grp. gram.:f.}
\end{itemize}
\begin{itemize}
\item {Utilização:Bras}
\end{itemize}
Modo de engrossar um caldo com qualquer fécula.
(Do tupi \textunderscore moagica\textunderscore )
\section{Mojicar}
\begin{itemize}
\item {Grp. gram.:v. t.}
\end{itemize}
\begin{itemize}
\item {Utilização:Bras}
\end{itemize}
\begin{itemize}
\item {Proveniência:(De \textunderscore mojica\textunderscore )}
\end{itemize}
Engrossar (caldo) com qualquer fécula.
\section{Mojiganga}
\begin{itemize}
\item {Grp. gram.:f.}
\end{itemize}
Dança burlesca.
Bugiganga.
Momices.
(Cast. \textunderscore mojiganga\textunderscore )
\section{Mola}
\begin{itemize}
\item {Grp. gram.:f.}
\end{itemize}
\begin{itemize}
\item {Utilização:Fig.}
\end{itemize}
\begin{itemize}
\item {Utilização:Gír.}
\end{itemize}
\begin{itemize}
\item {Proveniência:(It. \textunderscore molla\textunderscore , do lat. \textunderscore mollis\textunderscore )}
\end{itemize}
Lâmina de metal, com que se dá impulso ou resistência a qualquer peça.
Tudo que concorre para um movimento ou para um fim.
Gênero de peixes sem espinha, (\textunderscore tetrodon mola\textunderscore ).
Arco de arame, com que os marceneiros apertam peças que se grudam.
Cabeça, intelligência.
\section{Mola}
\begin{itemize}
\item {Grp. gram.:f.}
\end{itemize}
\begin{itemize}
\item {Proveniência:(Lat. \textunderscore mola\textunderscore )}
\end{itemize}
Carne informe, gerada no ventre das mulheres.
Bolo de farinha de grãos de trigo torrado, usado nos sacrifícios da antiga Roma.
\section{Molachino}
\begin{itemize}
\item {fónica:qui}
\end{itemize}
\begin{itemize}
\item {Grp. gram.:m.}
\end{itemize}
\begin{itemize}
\item {Utilização:Ant.}
\end{itemize}
Menino de côro.
Sacristão.
(Alter. de \textunderscore monachino\textunderscore )
\section{Molada}
\begin{itemize}
\item {Grp. gram.:f.}
\end{itemize}
\begin{itemize}
\item {Proveniência:(Do lat. \textunderscore mola\textunderscore )}
\end{itemize}
Porção de tinta, que se pisa de uma vez na moleta.
Água, contida na caixa, em que gira a pedra de amolar.
\section{Molagem}
\begin{itemize}
\item {Grp. gram.:f.}
\end{itemize}
\begin{itemize}
\item {Proveniência:(De \textunderscore mola\textunderscore ^2)}
\end{itemize}
Vantagem gratuita, môfo, borla.
\section{Molambeiro}
\begin{itemize}
\item {Grp. gram.:m.}
\end{itemize}
\begin{itemize}
\item {Utilização:T. de Moçambique}
\end{itemize}
O mesmo que \textunderscore imbondeiro\textunderscore .
\section{Molambo}
\begin{itemize}
\item {Grp. gram.:m.}
\end{itemize}
\begin{itemize}
\item {Utilização:Bras}
\end{itemize}
\begin{itemize}
\item {Grp. gram.:Loc.}
\end{itemize}
\begin{itemize}
\item {Utilização:burl.}
\end{itemize}
Farrapo; rodilha.
Vestido velho ou esfarrapado.
\textunderscore Estender o molambo\textunderscore , espichar a canela, morrer. Cf. Filinto, XIII, p. 309.
\section{Molambudo}
\begin{itemize}
\item {Grp. gram.:adj.}
\end{itemize}
\begin{itemize}
\item {Utilização:Bras}
\end{itemize}
\begin{itemize}
\item {Proveniência:(De \textunderscore molambo\textunderscore )}
\end{itemize}
Esfarrapado.
\section{Molaquino}
\begin{itemize}
\item {Grp. gram.:m.}
\end{itemize}
\begin{itemize}
\item {Utilização:Ant.}
\end{itemize}
Menino de côro.
Sacristão.
(Alter. de \textunderscore monachino\textunderscore )
\section{Molar}
\begin{itemize}
\item {Grp. gram.:adj.}
\end{itemize}
\begin{itemize}
\item {Proveniência:(Lat. \textunderscore molaris\textunderscore )}
\end{itemize}
Próprio para moêr ou triturar: \textunderscore dentes molares\textunderscore , que são o mesmo que \textunderscore dentes queixaes\textunderscore .
Que se mói facilmente.
Diz-se da amêndoa, que se póde partir com os dentes.
\section{Molariforme}
\begin{itemize}
\item {Grp. gram.:adj.}
\end{itemize}
\begin{itemize}
\item {Utilização:Bot.}
\end{itemize}
\begin{itemize}
\item {Proveniência:(Do lat. \textunderscore molaris\textunderscore  + \textunderscore forma\textunderscore )}
\end{itemize}
Que tem fórma de dente molar.
Diz-se de certos cogumelos, que têm a superfície coberta de uma espécie de dentes.
\section{Molarinha}
\begin{itemize}
\item {Grp. gram.:f.}
\end{itemize}
Erva, o mesmo que \textunderscore fumária\textunderscore .
Casta de uva.
\section{Molarinho}
\begin{itemize}
\item {Grp. gram.:adj.}
\end{itemize}
Diz-se de uma variedade de tojo.
\section{Moldação}
\begin{itemize}
\item {Grp. gram.:f.}
\end{itemize}
Acto ou effeito de moldar.
Acto de tirar a fórma de um objecto, estendendo sôbre êlle uma substância flexível que, depois de endurecida, conserva na sua cavidade os contornos daquelle objecto, podendo por êlles fundir-se ou formar-se objecto igual.
\section{Moldado}
\begin{itemize}
\item {Grp. gram.:m.}
\end{itemize}
\begin{itemize}
\item {Proveniência:(De \textunderscore moldar\textunderscore )}
\end{itemize}
Obra de moldura.
\section{Moldador}
\begin{itemize}
\item {Grp. gram.:m.}
\end{itemize}
\begin{itemize}
\item {Proveniência:(De \textunderscore moldar\textunderscore )}
\end{itemize}
Aquelle que faz moldes para fundição.
Instrumento de entalhador, para ornar as molduras em madeira rija.
\section{Moldagem}
\begin{itemize}
\item {Grp. gram.:f.}
\end{itemize}
O mesmo que \textunderscore moldação\textunderscore .
Um dos gêneros de esculptura.
\section{Moldar}
\begin{itemize}
\item {Grp. gram.:v. t.}
\end{itemize}
\begin{itemize}
\item {Utilização:Fig.}
\end{itemize}
Formar os moldes de.
Acommodar ao molde.
Fundir, vazando no molde.
Dar fórma ou contornos a.
Formar; adaptar; conformar.
(Por \textunderscore moledar\textunderscore , metáth. de \textunderscore modelar\textunderscore )
\section{Moldávia}
\begin{itemize}
\item {Grp. gram.:f.}
\end{itemize}
\begin{itemize}
\item {Proveniência:(De \textunderscore Moldávia\textunderscore , n. p.)}
\end{itemize}
Planta labiada, espécie de erva-cidreira.
\section{Molde}
\begin{itemize}
\item {Grp. gram.:m.}
\end{itemize}
\begin{itemize}
\item {Utilização:Fig.}
\end{itemize}
\begin{itemize}
\item {Grp. gram.:Loc. adv.}
\end{itemize}
Modêlo oco, formado de diversas peças reunidas, para nêlle se fundirem obras de metal.
Modelo, feito de qualquer substância, pelo qual se talha ou se fórma alguma coisa.
Norma.
Exemplo.
Caixa da matriz, para a fundição de caracteres typográphicos.
\textunderscore De molde\textunderscore , opportunamente, a propósito:«\textunderscore e aqui vem de molde avisar o leitor...\textunderscore »Camillo, \textunderscore Retr. de Ricard.\textunderscore , 206.
(Cp. cast. \textunderscore molde\textunderscore , do lat. \textunderscore modulus\textunderscore )
\section{Moldina}
\begin{itemize}
\item {Grp. gram.:f.}
\end{itemize}
\begin{itemize}
\item {Utilização:Neol.}
\end{itemize}
Substância, com que se faz moldagem.
\section{Moldura}
\begin{itemize}
\item {Grp. gram.:f.}
\end{itemize}
Ornato saliente em obras de architectura.
Caixilho de metal, de madeira ou de outra substância, para guarnecer quadros, estampas, etc.
(Cp. cast. \textunderscore moldura\textunderscore )
\section{Molduragem}
\begin{itemize}
\item {Grp. gram.:f.}
\end{itemize}
Acto de moldurar.
Conjunto de molduras, que adornam uma peça de architectura.
\section{Moldurar}
\begin{itemize}
\item {Grp. gram.:v. t.}
\end{itemize}
Ornar de moldura; guarnecer, encaixilhar em moldura; emmoldurar.
\section{Moldureiro}
\begin{itemize}
\item {Grp. gram.:m.}
\end{itemize}
Fabricante de molduras ou aquelle que guarnece com molduras.
\section{Mole}
\begin{itemize}
\item {Grp. gram.:M.}
\end{itemize}
\begin{itemize}
\item {Utilização:Ant.}
\end{itemize}
\begin{itemize}
\item {Proveniência:(Lat. \textunderscore moles\textunderscore )}
\end{itemize}
Volume enorme.
Colosso.
Grande porção.
Massa informe.
Construcção de grandes proporções.
O mesmo que \textunderscore molhe\textunderscore . Cf. B. Pereira, \textunderscore Prosódia\textunderscore , vb. \textunderscore progressus\textunderscore .
\section{Moleca}
\begin{itemize}
\item {Grp. gram.:f.}
\end{itemize}
\begin{itemize}
\item {Utilização:Bras}
\end{itemize}
Menina negra.
(Cp. \textunderscore moleque\textunderscore ^1)
\section{Molecada}
\begin{itemize}
\item {Grp. gram.:f.}
\end{itemize}
\begin{itemize}
\item {Utilização:Bras}
\end{itemize}
Bando de moleques^1.
\section{Molecagem}
\begin{itemize}
\item {Grp. gram.:f.}
\end{itemize}
\begin{itemize}
\item {Utilização:Bras}
\end{itemize}
Acto próprio de moleque^1; acto mau.
\section{Molecão}
\begin{itemize}
\item {Grp. gram.:m.}
\end{itemize}
\begin{itemize}
\item {Utilização:Bras}
\end{itemize}
Moleque taludo, encorpado.
\section{Molecar}
\begin{itemize}
\item {Grp. gram.:v. i.}
\end{itemize}
\begin{itemize}
\item {Utilização:Bras}
\end{itemize}
Proceder ou divertir-se como moleque^1.
\section{Molecote}
\begin{itemize}
\item {Grp. gram.:m.}
\end{itemize}
O mesmo que \textunderscore molecão\textunderscore .
\section{Molécula}
\begin{itemize}
\item {Grp. gram.:f.}
\end{itemize}
\begin{itemize}
\item {Utilização:Fig.}
\end{itemize}
\begin{itemize}
\item {Proveniência:(Fr. \textunderscore molécule\textunderscore )}
\end{itemize}
Pequenina parte de um corpo.
A mais pequena parte de um corpo, a qual póde existir em liberdade.
Parte deminuta de um todo.
\section{Molecular}
\begin{itemize}
\item {Grp. gram.:adj.}
\end{itemize}
Que tem moléculas.
Relativo a moléculas.
\section{Moledo}
\begin{itemize}
\item {fónica:lê}
\end{itemize}
\begin{itemize}
\item {Grp. gram.:m.}
\end{itemize}
O mesmo que \textunderscore moledro\textunderscore .
\section{Moledro}
\begin{itemize}
\item {fónica:lê}
\end{itemize}
\begin{itemize}
\item {Grp. gram.:m.}
\end{itemize}
\begin{itemize}
\item {Utilização:Prov.}
\end{itemize}
\begin{itemize}
\item {Utilização:alg.}
\end{itemize}
Grande pedra.
Monte de pedras.
\section{Molangueirão}
\begin{itemize}
\item {Grp. gram.:m.  e  adj.}
\end{itemize}
\begin{itemize}
\item {Utilização:Pop.}
\end{itemize}
Indivíduo froixo, indolente, sem energia.
(Cp. \textunderscore molengueiro\textunderscore )
\section{Molanqueiro}
\begin{itemize}
\item {Grp. gram.:m.  e  adj.}
\end{itemize}
\begin{itemize}
\item {Utilização:Pop.}
\end{itemize}
O mesmo que \textunderscore molangueirão\textunderscore . Cf. Camillo, \textunderscore Eusébio\textunderscore , 239.
\section{Molar}
\begin{itemize}
\item {Grp. gram.:adj.}
\end{itemize}
\begin{itemize}
\item {Grp. gram.:F.  e  adj.}
\end{itemize}
\begin{itemize}
\item {Utilização:Prov.}
\end{itemize}
\begin{itemize}
\item {Utilização:alent.}
\end{itemize}
\begin{itemize}
\item {Proveniência:(De \textunderscore mole\textunderscore )}
\end{itemize}
Que tem casca pouco dura; mole.
Diz-se do mato brando, em terra areenta.
Diz-se de uma variedade de azeitona, também conhecida por \textunderscore negral\textunderscore  ou \textunderscore mean\textunderscore .
Diz-se de uma variedade de noz.
\section{Molarinho}
\begin{itemize}
\item {Grp. gram.:adj.}
\end{itemize}
\begin{itemize}
\item {Utilização:Prov.}
\end{itemize}
\begin{itemize}
\item {Utilização:minh.}
\end{itemize}
\begin{itemize}
\item {Proveniência:(De \textunderscore molar\textunderscore )}
\end{itemize}
Tenro: \textunderscore erva molarinha\textunderscore .
Que tem pêlo macio: \textunderscore cão molarinho\textunderscore .
\section{Molasso}
\begin{itemize}
\item {Grp. gram.:m.}
\end{itemize}
\begin{itemize}
\item {Grp. gram.:Pl.}
\end{itemize}
\begin{itemize}
\item {Proveniência:(Fr. \textunderscore mollasse\textunderscore )}
\end{itemize}
Espécie de arenito calcário ou margoso.
Vermes intestinaes, cujo corpo é formado de uma substância gelatinosa, branda e transparente.
\section{Mole}
\begin{itemize}
\item {Grp. gram.:adj.}
\end{itemize}
\begin{itemize}
\item {Utilização:Fig.}
\end{itemize}
\begin{itemize}
\item {Utilização:Prov.}
\end{itemize}
\begin{itemize}
\item {Utilização:minh.}
\end{itemize}
\begin{itemize}
\item {Grp. gram.:Adv.}
\end{itemize}
\begin{itemize}
\item {Proveniência:(Lat. \textunderscore mollis\textunderscore )}
\end{itemize}
Brando.
Que cede a qualquer pressão sem se desfazer: \textunderscore pão mole\textunderscore .
Indolente, preguiçoso.
Que não tem energia.
Que produz enfraquecimento.
Froixo, sem colorido.
Diz-se do vinho, que está fermentando e que ainda se não envasilhou.
\textunderscore Mole-mole\textunderscore , pouco a pouco.
\section{Moledo}
\begin{itemize}
\item {fónica:lê}
\end{itemize}
\begin{itemize}
\item {Grp. gram.:m.}
\end{itemize}
\begin{itemize}
\item {Utilização:Prov.}
\end{itemize}
Terreno mole, o mesmo que \textunderscore molo\textunderscore .
\section{Molego}
\begin{itemize}
\item {fónica:lê}
\end{itemize}
\begin{itemize}
\item {Grp. gram.:m.}
\end{itemize}
\begin{itemize}
\item {Utilização:Prov.}
\end{itemize}
\begin{itemize}
\item {Utilização:trasm.}
\end{itemize}
\begin{itemize}
\item {Proveniência:(De \textunderscore mole\textunderscore )}
\end{itemize}
Pão de trigo, dividido em quartos. (Colhido em Valpaços)
\section{Moleguim}
\begin{itemize}
\item {Grp. gram.:m.}
\end{itemize}
\begin{itemize}
\item {Utilização:Prov.}
\end{itemize}
\begin{itemize}
\item {Utilização:alent.}
\end{itemize}
O mesmo que \textunderscore molhelha\textunderscore ^1.
\section{Moleia}
\begin{itemize}
\item {Grp. gram.:f.}
\end{itemize}
\begin{itemize}
\item {Utilização:Prov.}
\end{itemize}
O mesmo que \textunderscore molhelha\textunderscore ^1.
\section{Moleira}
\begin{itemize}
\item {Grp. gram.:f.}
\end{itemize}
\begin{itemize}
\item {Utilização:Ant.}
\end{itemize}
Dona de moínho.
Mulher que, por offício, leva cereaes ao moínho, conduzindo depois a farinha a casa dos seus fregueses.
Mulher de moleiro.
O mesmo que \textunderscore moínho\textunderscore .
\section{Moleira}
\begin{itemize}
\item {Grp. gram.:f.}
\end{itemize}
\begin{itemize}
\item {Utilização:Fam.}
\end{itemize}
\begin{itemize}
\item {Utilização:Ext.}
\end{itemize}
\begin{itemize}
\item {Grp. gram.:Loc.}
\end{itemize}
\begin{itemize}
\item {Utilização:pop.}
\end{itemize}
\begin{itemize}
\item {Proveniência:(De \textunderscore mole\textunderscore )}
\end{itemize}
Fontanela, correspondente á sutura coronal, em-quanto se não completa a ossificação.
Abóbada do crânio.
\textunderscore Pôr o sal na moleira a alguém\textunderscore , criar-lhe obstaculos, dar-lhe cuidados.
\section{Moleirão}
\begin{itemize}
\item {Grp. gram.:m.  e  adj.}
\end{itemize}
\begin{itemize}
\item {Utilização:Bras}
\end{itemize}
\begin{itemize}
\item {Proveniência:(De \textunderscore mole\textunderscore )}
\end{itemize}
O mesmo que \textunderscore molengão\textunderscore .
\section{Moleirinha}
\begin{itemize}
\item {Grp. gram.:f.}
\end{itemize}
\begin{itemize}
\item {Grp. gram.:Adj.}
\end{itemize}
\begin{itemize}
\item {Utilização:Prov.}
\end{itemize}
O mesmo que \textunderscore moleira\textunderscore ^1.
Diz-se de uma casta de oliveira e de azeitona, também conhecida por \textunderscore negral\textunderscore .
\section{Moleirinha}
\begin{itemize}
\item {Grp. gram.:f.}
\end{itemize}
\begin{itemize}
\item {Utilização:Prov.}
\end{itemize}
Borboleta, bôa nova.
O mesmo que \textunderscore erva-moleirinha\textunderscore .
\section{Moleirinho}
\begin{itemize}
\item {Grp. gram.:m.}
\end{itemize}
Peixe de Portugal.
\section{Moleirinho}
\begin{itemize}
\item {Grp. gram.:adj.}
\end{itemize}
\begin{itemize}
\item {Utilização:Marn.}
\end{itemize}
\begin{itemize}
\item {Proveniência:(De \textunderscore mole\textunderscore )}
\end{itemize}
Argiloso.
\section{Moleiro}
\begin{itemize}
\item {Grp. gram.:m.}
\end{itemize}
\begin{itemize}
\item {Proveniência:(Do ant. \textunderscore monlário\textunderscore , por \textunderscore molnário\textunderscore , do lat. \textunderscore molinarius\textunderscore )}
\end{itemize}
Dono de moínho.
Aquelle que se occupa em trabalhos de moínho.
O mesmo que \textunderscore mandrião\textunderscore , ave.
Espécie de papagaio da região do Amazonas.
\section{Moleirona}
\begin{itemize}
\item {Grp. gram.:f.}
\end{itemize}
\begin{itemize}
\item {Proveniência:(De \textunderscore moleirão\textunderscore )}
\end{itemize}
Mulher indolente, preguiçosa.
\section{Moleja}
\begin{itemize}
\item {Grp. gram.:f.}
\end{itemize}
\begin{itemize}
\item {Utilização:Pop.}
\end{itemize}
\begin{itemize}
\item {Utilização:Prov.}
\end{itemize}
\begin{itemize}
\item {Utilização:alg.}
\end{itemize}
Glândula carnosa no corpo dos animaes, mormente na parte inferior do pescoço do gado vacum.
Excremento de aves.
O pâncreas nas reses.
Sarrabulho.
(Cp. cast. \textunderscore molleja\textunderscore )
\section{Molendário}
\begin{itemize}
\item {Grp. gram.:adj.}
\end{itemize}
\begin{itemize}
\item {Proveniência:(Lat. \textunderscore molendarius\textunderscore )}
\end{itemize}
Relativo a moagem. Cf. Castilho, \textunderscore Fastos\textunderscore , III, 481.
\section{Moleque}
\begin{itemize}
\item {Grp. gram.:m.}
\end{itemize}
Rapaz preto.
Criado preto, de pouca idade.
(Quimbundo \textunderscore moleque\textunderscore )
\section{Moleque}
\begin{itemize}
\item {Grp. gram.:m.}
\end{itemize}
\begin{itemize}
\item {Utilização:Bras}
\end{itemize}
Barra de íman, para separar do oiro em pó as partículas de ferro que êlle tem misturadas.
\section{Moleque}
\begin{itemize}
\item {Grp. gram.:m.}
\end{itemize}
\begin{itemize}
\item {Utilização:Gír.}
\end{itemize}
\begin{itemize}
\item {Proveniência:(De \textunderscore mola\textunderscore ^1)}
\end{itemize}
Bofetão.
\section{Molequear}
\begin{itemize}
\item {Grp. gram.:v. i.}
\end{itemize}
\begin{itemize}
\item {Utilização:Bras}
\end{itemize}
Agarotar-se; proceder como moleque ou garoto.
\section{Moleque-de-assentar}
\begin{itemize}
\item {Grp. gram.:m.}
\end{itemize}
\begin{itemize}
\item {Utilização:Bras}
\end{itemize}
Pau grosso, que serve de rasoira, para igualar o açúcar dentro das caixas, nos respectivos engenhos.
\section{Molequeira}
\begin{itemize}
\item {Grp. gram.:f.}
\end{itemize}
O mesmo que \textunderscore molecagem\textunderscore .
\section{Molestador}
\begin{itemize}
\item {Grp. gram.:adj.}
\end{itemize}
\begin{itemize}
\item {Grp. gram.:M.}
\end{itemize}
Que molesta.
Aquelle que molesta.
\section{Molestamente}
\begin{itemize}
\item {Grp. gram.:adv.}
\end{itemize}
De modo molesto; incommodamente.
\section{Molestamento}
\begin{itemize}
\item {Grp. gram.:m.}
\end{itemize}
Acto ou effeito de molestar.
\section{Molestar}
\begin{itemize}
\item {Grp. gram.:v. t.}
\end{itemize}
\begin{itemize}
\item {Proveniência:(Lat. \textunderscore molestare\textunderscore )}
\end{itemize}
Sêr molesto a.
Enfadar.
Incommodar.
Atormentar.
Causar moléstia a.
Maltratar.
Opprimir; offender.
\section{Moleste}
\begin{itemize}
\item {Grp. gram.:m.}
\end{itemize}
\begin{itemize}
\item {Utilização:Açor}
\end{itemize}
\begin{itemize}
\item {Proveniência:(De \textunderscore molestar\textunderscore )}
\end{itemize}
Prejuízo, damno, mal.
\textunderscore Não faz moleste\textunderscore , não faz mal, não importa.
\section{Moléstia}
\begin{itemize}
\item {Grp. gram.:f.}
\end{itemize}
\begin{itemize}
\item {Proveniência:(Lat. \textunderscore molestia\textunderscore )}
\end{itemize}
Estado penoso.
Incómmodo phýsico ou moral; mal-estar; inquietação.--Toma-se indevidamente como synónymo de doença ou enfermidade.
\section{Molesto}
\begin{itemize}
\item {Grp. gram.:adj.}
\end{itemize}
\begin{itemize}
\item {Proveniência:(Lat. \textunderscore molestus\textunderscore )}
\end{itemize}
Que enfada, que incommoda.
Árduo, trabalhoso; prejudicial.
O mesmo que \textunderscore doente\textunderscore . Cf. Filinto, I, 53.
\section{Molestoso}
\begin{itemize}
\item {Grp. gram.:adj.}
\end{itemize}
\begin{itemize}
\item {Utilização:Des.}
\end{itemize}
O mesmo que \textunderscore molesto\textunderscore .
\section{Moleta}
\begin{itemize}
\item {fónica:lê}
\end{itemize}
\begin{itemize}
\item {Grp. gram.:f.}
\end{itemize}
\begin{itemize}
\item {Utilização:Prov.}
\end{itemize}
\begin{itemize}
\item {Utilização:alg.}
\end{itemize}
\begin{itemize}
\item {Utilização:Heráld.}
\end{itemize}
\begin{itemize}
\item {Proveniência:(Do lat. \textunderscore mola\textunderscore )}
\end{itemize}
Utensílio de mármore, em que se pisam e móem tintas.
Pequena mó.
Figura, em fórma de estrêlla e vazada no centro.
\section{Moletos}
\begin{itemize}
\item {fónica:lê}
\end{itemize}
\begin{itemize}
\item {Grp. gram.:m. pl.}
\end{itemize}
\begin{itemize}
\item {Utilização:Gír.}
\end{itemize}
Pés.
Dedos.
Cp. \textunderscore moleta\textunderscore , se é que não vem de \textunderscore muleta\textunderscore , devendo, neste caso, escrever-se \textunderscore muletos\textunderscore .
\section{Mólha}
\begin{itemize}
\item {Grp. gram.:f.}
\end{itemize}
O mesmo que \textunderscore molhadela\textunderscore .
\section{Môlha}
\begin{itemize}
\item {Grp. gram.:f.}
\end{itemize}
O mesmo que \textunderscore molhadela\textunderscore .
\section{Molha}
\begin{itemize}
\item {fónica:mô}
\end{itemize}
\begin{itemize}
\item {Grp. gram.:f.}
\end{itemize}
\begin{itemize}
\item {Utilização:Prov.}
\end{itemize}
\begin{itemize}
\item {Utilização:trasm.}
\end{itemize}
\begin{itemize}
\item {Utilização:Prov.}
\end{itemize}
\begin{itemize}
\item {Utilização:açor}
\end{itemize}
Porção de grãos verdes de cevada, que os rapazes juntam na copa do chapéu, e dahi os comem.
Jôgo do pião, em que só ganha quem toca com o seu pião numa pouca de saliva, cuspida dentro de um círculo, que serve de alvo.
(Cp. \textunderscore molhagem\textunderscore )
\section{Molhaça}
\begin{itemize}
\item {Grp. gram.:f.}
\end{itemize}
\begin{itemize}
\item {Utilização:ant.}
\end{itemize}
\begin{itemize}
\item {Utilização:Fam.}
\end{itemize}
Qualquer iguaria com grande porção de môlho.
\section{Mòlhada}
\begin{itemize}
\item {Grp. gram.:f.}
\end{itemize}
Grande mólho ou feixe.
Porção de mólhos.
Mólho, que se leva nos braços.
\section{Molhadela}
\begin{itemize}
\item {Grp. gram.:f.}
\end{itemize}
Acto ou effeito de molhar; banho.
\section{Molhado}
\begin{itemize}
\item {Grp. gram.:m.}
\end{itemize}
\begin{itemize}
\item {Grp. gram.:Pl.}
\end{itemize}
\begin{itemize}
\item {Utilização:Bras}
\end{itemize}
Lugar, humedecido por um líquido que nelle cahiu ou se entornou.
Vinho, azeite ou outras substâncias líquidas que se vendem nas mercearias: \textunderscore tem loja de secos e molhados\textunderscore .
\section{Molhadura}
\begin{itemize}
\item {Grp. gram.:f.}
\end{itemize}
\begin{itemize}
\item {Utilização:pop.}
\end{itemize}
\begin{itemize}
\item {Utilização:Fig.}
\end{itemize}
O mesmo que \textunderscore molhadela\textunderscore .
Gratificação.
Gorgeta, que se dá ao alfaiate ou a outro artífice, que leva a casa do freguês obra feita de encommenda ou que conclue o trabalho, de que estava incumbido.
\section{Molhagem}
\begin{itemize}
\item {Grp. gram.:f.}
\end{itemize}
\begin{itemize}
\item {Proveniência:(De \textunderscore molhar\textunderscore )}
\end{itemize}
Acto de pôr em água a cevada em grão, para que êste germine e sirva para o fabríco da cerveja.
\section{Molhamento}
\begin{itemize}
\item {Grp. gram.:m.}
\end{itemize}
\begin{itemize}
\item {Proveniência:(De \textunderscore molhar\textunderscore )}
\end{itemize}
Molhadela; immersão.
\section{Molhança}
\begin{itemize}
\item {Grp. gram.:f.}
\end{itemize}
Grande porção de môlho.
\section{Molhanga}
\begin{itemize}
\item {Grp. gram.:f.}
\end{itemize}
\begin{itemize}
\item {Proveniência:(De \textunderscore môlho\textunderscore )}
\end{itemize}
Molhança; caldivana.
\section{Molhar}
\begin{itemize}
\item {Grp. gram.:v. t.}
\end{itemize}
\begin{itemize}
\item {Utilização:Fig.}
\end{itemize}
\begin{itemize}
\item {Proveniência:(Do lat. \textunderscore molliare\textunderscore )}
\end{itemize}
Repassar de líquido.
Embeber em água ou noutro líquido.
Humedecer, cobrir de líquido: \textunderscore molhar os pés\textunderscore .
\textunderscore Molhar a sopa\textunderscore , têr quinhão, têr parte.
\section{Molhe}
\begin{itemize}
\item {Grp. gram.:m.}
\end{itemize}
\begin{itemize}
\item {Proveniência:(Do lat. \textunderscore moles\textunderscore )}
\end{itemize}
Paredão, em fórma de caes, para abrigo de navios.
\section{Molheira}
\begin{itemize}
\item {Grp. gram.:f.}
\end{itemize}
Vaso, em que, á mesa, se servem môlhos.
\section{Molheiro}
\begin{itemize}
\item {Grp. gram.:m.}
\end{itemize}
O mesmo que \textunderscore molheira\textunderscore .
\section{Molhelha}
\begin{itemize}
\item {fónica:lhê}
\end{itemize}
\begin{itemize}
\item {Grp. gram.:f.}
\end{itemize}
\begin{itemize}
\item {Grp. gram.:Pl.}
\end{itemize}
\begin{itemize}
\item {Utilização:Náut.}
\end{itemize}
\begin{itemize}
\item {Proveniência:(Do lat. \textunderscore monilia\textunderscore . Cp. \textunderscore monelha\textunderscore )}
\end{itemize}
Espécie de almofada, em que assenta a canga que junge os bois.
Chinguiço.
Malim.
Estofos, com que se forram as peças de madeira em que laboram os cabos de navios.
\section{Molhelha}
\begin{itemize}
\item {fónica:lhê}
\end{itemize}
\begin{itemize}
\item {Grp. gram.:f.}
\end{itemize}
\begin{itemize}
\item {Utilização:Prov.}
\end{itemize}
\begin{itemize}
\item {Utilização:beir.}
\end{itemize}
O mesmo que \textunderscore faneca\textunderscore  (da castanha).
(Por \textunderscore mollelha\textunderscore , de \textunderscore molle\textunderscore )
\section{Molhelheiro}
\begin{itemize}
\item {Grp. gram.:m.}
\end{itemize}
\begin{itemize}
\item {Proveniência:(De \textunderscore molhelha\textunderscore ^1)}
\end{itemize}
Aquelle que faz ou concerta molhelhas.
\section{Molhe-molhe}
\begin{itemize}
\item {Grp. gram.:m.}
\end{itemize}
\begin{itemize}
\item {Proveniência:(De \textunderscore molhar\textunderscore )}
\end{itemize}
Chuva miúda, molinheiro.
\section{Molhér}
\begin{itemize}
\item {Grp. gram.:f.}
\end{itemize}
\begin{itemize}
\item {Utilização:P. us.}
\end{itemize}
O mesmo que \textunderscore mulhér\textunderscore .
\section{Molhida}
\begin{itemize}
\item {Grp. gram.:f.}
\end{itemize}
\begin{itemize}
\item {Utilização:Prov.}
\end{itemize}
\begin{itemize}
\item {Utilização:trasm.}
\end{itemize}
O mesmo que \textunderscore molida\textunderscore .
\section{Mólho}
\begin{itemize}
\item {Grp. gram.:m.}
\end{itemize}
Feixe.
Mão-cheia.
Paveia.
(Contr. de \textunderscore manolho\textunderscore . Cp. \textunderscore manolho\textunderscore )
\section{Môlho}
\begin{itemize}
\item {Grp. gram.:m.}
\end{itemize}
\begin{itemize}
\item {Proveniência:(Do lat. \textunderscore mollis\textunderscore )}
\end{itemize}
Espécie de caldo, em que se refogam iguarias ou que se junta a estas.
Água, ou qualquer líquido, em que se immerge alguma substância, para amollecer ou para perder o sal que contém.
\section{Moli}
\begin{itemize}
\item {Grp. gram.:m.}
\end{itemize}
\begin{itemize}
\item {Utilização:T. da Índia port}
\end{itemize}
Vestidura leve, com que as bailadeiras encobrem o seio.
\section{Moliana}
\begin{itemize}
\item {Grp. gram.:f.}
\end{itemize}
\begin{itemize}
\item {Utilização:Pop.}
\end{itemize}
Reprehensão; sarabanda.
\section{Molição}
\begin{itemize}
\item {Grp. gram.:f.}
\end{itemize}
\begin{itemize}
\item {Proveniência:(Lat. \textunderscore molitio\textunderscore )}
\end{itemize}
Grande esfôrço para a consecução de um fim.
\section{Molida}
\begin{itemize}
\item {Grp. gram.:f.}
\end{itemize}
\begin{itemize}
\item {Utilização:Prov.}
\end{itemize}
\begin{itemize}
\item {Utilização:trasm.}
\end{itemize}
O mesmo que \textunderscore molhelha\textunderscore ^1.
\section{Molieresco}
\begin{itemize}
\item {fónica:erês}
\end{itemize}
\begin{itemize}
\item {Grp. gram.:adj.}
\end{itemize}
Relativo a Molière.
Que imita o estilo ou gôsto literário de Molière.
\section{Molime}
\begin{itemize}
\item {Grp. gram.:m.}
\end{itemize}
\begin{itemize}
\item {Proveniência:(Lat. \textunderscore molimen\textunderscore )}
\end{itemize}
Molição.
Fôrça impulsiva de um corpo em movimento.
Aquillo que impulsiona.
\section{Molímen}
\begin{itemize}
\item {Grp. gram.:m.}
\end{itemize}
\begin{itemize}
\item {Proveniência:(Lat. \textunderscore molimen\textunderscore )}
\end{itemize}
Molição.
Fôrça impulsiva de um corpo em movimento.
Aquillo que impulsiona.
\section{Molina}
\begin{itemize}
\item {Grp. gram.:f.}
\end{itemize}
Tecido de lan, fabricado em Molina, (Aragão).
\section{Molineta}
\begin{itemize}
\item {fónica:nê}
\end{itemize}
\begin{itemize}
\item {Grp. gram.:f.}
\end{itemize}
\begin{itemize}
\item {Utilização:Prov.}
\end{itemize}
\begin{itemize}
\item {Utilização:alg.}
\end{itemize}
\begin{itemize}
\item {Proveniência:(Do cast. \textunderscore molino\textunderscore )}
\end{itemize}
Moínho caseiro.
\section{Molinete}
\begin{itemize}
\item {fónica:nê}
\end{itemize}
\begin{itemize}
\item {Grp. gram.:m.}
\end{itemize}
\begin{itemize}
\item {Proveniência:(Fr. \textunderscore moulinet\textunderscore )}
\end{itemize}
Espécie de cabrestante, que sustenta a âncora, na prôa dos navios pequenos.
Ventilador nas vidraças.
Passe de muleta, que o toireiro executa, firmando-se nos calcanhares e dando uma volta rápida em frente do toiro.
Cruzamento de paus, que giravam sôbre um pião, para impedir que pelas portas das antigas fortalezas se entrasse de tropel.
\section{Molinha}
\begin{itemize}
\item {Grp. gram.:f.}
\end{itemize}
\begin{itemize}
\item {Proveniência:(De \textunderscore molinhar\textunderscore )}
\end{itemize}
O mesmo que \textunderscore molhe-molhe\textunderscore .
Chuvisco.
\section{Molinhar}
\begin{itemize}
\item {Grp. gram.:v. t.}
\end{itemize}
\begin{itemize}
\item {Grp. gram.:V. i.}
\end{itemize}
\begin{itemize}
\item {Proveniência:(Do lat. \textunderscore molinus\textunderscore )}
\end{itemize}
Moêr aos poucos, frequentemente.
Funccionar o (moínho).
Caír molinha, ou chuva miúda.
\section{Molinheira}
\begin{itemize}
\item {Grp. gram.:f.}
\end{itemize}
\begin{itemize}
\item {Proveniência:(De \textunderscore molinheiro\textunderscore )}
\end{itemize}
Grande moínho.
Molinha persistente.
\section{Molinheiro}
\begin{itemize}
\item {Grp. gram.:m.}
\end{itemize}
\begin{itemize}
\item {Proveniência:(Do lat. \textunderscore molinarius\textunderscore )}
\end{itemize}
O mesmo que \textunderscore molinha\textunderscore ^1.
\section{Molinhoso}
\begin{itemize}
\item {Grp. gram.:adj.}
\end{itemize}
Em que há molinha: \textunderscore dia molinhoso\textunderscore .
\section{Molínia}
\begin{itemize}
\item {Grp. gram.:f.}
\end{itemize}
\begin{itemize}
\item {Proveniência:(De \textunderscore Molina\textunderscore , n. p.)}
\end{itemize}
Gênero de plantas gramíneas.
\section{Molinilho}
\begin{itemize}
\item {Grp. gram.:m.}
\end{itemize}
\begin{itemize}
\item {Proveniência:(Do lat. \textunderscore molinus\textunderscore )}
\end{itemize}
Pequeno moínho, a que se dá movimento com a mão.
Círculo dentado, com que se bate o chocolate.
\section{Molinismo}
\begin{itemize}
\item {Grp. gram.:m.}
\end{itemize}
Doutrina de Molina, sôbre o acôrdo do livre arbítrio com a graça.
\section{Molinista}
\begin{itemize}
\item {Grp. gram.:m.}
\end{itemize}
Partidário de Molina, em Theologia.
\section{Molinote}
\begin{itemize}
\item {Grp. gram.:m.}
\end{itemize}
\begin{itemize}
\item {Proveniência:(Do lat. \textunderscore molinus\textunderscore )}
\end{itemize}
Moenda de cana de açúcar.
Cabrestante, que se arma nos engenhos de açúcar, para os fazer trabalhar por meio de bêstas, quando falta água para êsse effeito.
\section{Mólio}
\begin{itemize}
\item {Grp. gram.:m.}
\end{itemize}
Árvore da Índia portuguesa.
\section{Mollangueirão}
\begin{itemize}
\item {Grp. gram.:m.  e  adj.}
\end{itemize}
\begin{itemize}
\item {Utilização:Pop.}
\end{itemize}
Indivíduo froixo, indolente, sem energia.
(Cp. \textunderscore mollengueiro\textunderscore )
\section{Mollanqueiro}
\begin{itemize}
\item {Grp. gram.:m.  e  adj.}
\end{itemize}
\begin{itemize}
\item {Utilização:Pop.}
\end{itemize}
O mesmo que \textunderscore mollangueirão\textunderscore . Cf. Camillo, \textunderscore Eusébio\textunderscore , 239.
\section{Mollar}
\begin{itemize}
\item {Grp. gram.:adj.}
\end{itemize}
\begin{itemize}
\item {Grp. gram.:F.  e  adj.}
\end{itemize}
\begin{itemize}
\item {Utilização:Prov.}
\end{itemize}
\begin{itemize}
\item {Utilização:alent.}
\end{itemize}
\begin{itemize}
\item {Proveniência:(De \textunderscore molle\textunderscore )}
\end{itemize}
Que tem casca pouco dura; molle.
Diz-se do mato brando, em terra areenta.
Diz-se de uma variedade de azeitona, também conhecida por \textunderscore negral\textunderscore  ou \textunderscore mean\textunderscore .
Diz-se de uma variedade de noz.
\section{Mollar-grosso}
\begin{itemize}
\item {Grp. gram.:m.}
\end{itemize}
Variedade de uva.
\section{Mollarinho}
\begin{itemize}
\item {Grp. gram.:adj.}
\end{itemize}
\begin{itemize}
\item {Utilização:Prov.}
\end{itemize}
\begin{itemize}
\item {Utilização:minh.}
\end{itemize}
\begin{itemize}
\item {Proveniência:(De \textunderscore mollar\textunderscore )}
\end{itemize}
Tenro: \textunderscore erva mollarinha\textunderscore .
Que tem pêlo macio: \textunderscore cão mollarinho\textunderscore .
\section{Mollasso}
\begin{itemize}
\item {Grp. gram.:m.}
\end{itemize}
\begin{itemize}
\item {Grp. gram.:Pl.}
\end{itemize}
\begin{itemize}
\item {Proveniência:(Fr. \textunderscore mollasse\textunderscore )}
\end{itemize}
Espécie de arenito calcário ou margoso.
Vermes intestinaes, cujo corpo é formado de uma substância gelatinosa, branda e transparente.
\section{Molle}
\begin{itemize}
\item {Grp. gram.:adj.}
\end{itemize}
\begin{itemize}
\item {Utilização:Fig.}
\end{itemize}
\begin{itemize}
\item {Utilização:Prov.}
\end{itemize}
\begin{itemize}
\item {Utilização:minh.}
\end{itemize}
\begin{itemize}
\item {Grp. gram.:Adv.}
\end{itemize}
\begin{itemize}
\item {Proveniência:(Lat. \textunderscore mollis\textunderscore )}
\end{itemize}
Brando.
Que cede a qualquer pressão sem se desfazer: \textunderscore pão molle\textunderscore .
Indolente, preguiçoso.
Que não tem energia.
Que produz enfraquecimento.
Froixo, sem colorido.
Diz-se do vinho, que está fermentando e que ainda se não envasilhou.
\textunderscore Molle-molle\textunderscore , pouco a pouco.
\section{Molledo}
\begin{itemize}
\item {Grp. gram.:m.}
\end{itemize}
\begin{itemize}
\item {Utilização:Prov.}
\end{itemize}
Terreno molle, o mesmo que \textunderscore mollo\textunderscore .
\section{Mollego}
\begin{itemize}
\item {Grp. gram.:m.}
\end{itemize}
\begin{itemize}
\item {Utilização:Prov.}
\end{itemize}
\begin{itemize}
\item {Utilização:trasm.}
\end{itemize}
\begin{itemize}
\item {Proveniência:(De \textunderscore molle\textunderscore )}
\end{itemize}
Pão de trigo, dividido em quartos. (Colhido em Valpaços)
\section{Molleguim}
\begin{itemize}
\item {Grp. gram.:m.}
\end{itemize}
\begin{itemize}
\item {Utilização:Prov.}
\end{itemize}
\begin{itemize}
\item {Utilização:alent.}
\end{itemize}
O mesmo que \textunderscore molhelha\textunderscore ^1.
\section{Molleia}
\begin{itemize}
\item {Grp. gram.:f.}
\end{itemize}
\begin{itemize}
\item {Utilização:Prov.}
\end{itemize}
O mesmo que \textunderscore molhelha\textunderscore ^1.
\section{Molleira}
\begin{itemize}
\item {Grp. gram.:f.}
\end{itemize}
\begin{itemize}
\item {Utilização:Fam.}
\end{itemize}
\begin{itemize}
\item {Utilização:Ext.}
\end{itemize}
\begin{itemize}
\item {Grp. gram.:Loc.}
\end{itemize}
\begin{itemize}
\item {Utilização:pop.}
\end{itemize}
\begin{itemize}
\item {Proveniência:(De \textunderscore molle\textunderscore )}
\end{itemize}
Fontanella, correspondente á sutura coronal, em-quanto se não completa a ossificação.
Abóbada do crânio.
\textunderscore Pôr o sal na molleira a alguém\textunderscore , criar-lhe obstaculos, dar-lhe cuidados.
\section{Molleirão}
\begin{itemize}
\item {Grp. gram.:m.  e  adj.}
\end{itemize}
\begin{itemize}
\item {Utilização:Bras}
\end{itemize}
\begin{itemize}
\item {Proveniência:(De \textunderscore molle\textunderscore )}
\end{itemize}
O mesmo que \textunderscore mollengão\textunderscore .
\section{Molleirinha}
\begin{itemize}
\item {Grp. gram.:f.}
\end{itemize}
\begin{itemize}
\item {Grp. gram.:Adj.}
\end{itemize}
\begin{itemize}
\item {Utilização:Prov.}
\end{itemize}
O mesmo que \textunderscore molleira\textunderscore .
Diz-se de uma casta de oliveira e de azeitona, também conhecida por \textunderscore negral\textunderscore .
\section{Molleirinho}
\begin{itemize}
\item {Grp. gram.:adj.}
\end{itemize}
\begin{itemize}
\item {Utilização:Marn.}
\end{itemize}
\begin{itemize}
\item {Proveniência:(De \textunderscore molle\textunderscore )}
\end{itemize}
Argilloso.
\section{Molleirona}
\begin{itemize}
\item {Grp. gram.:f.}
\end{itemize}
\begin{itemize}
\item {Proveniência:(De \textunderscore molleirão\textunderscore )}
\end{itemize}
Mulher indolente, preguiçosa.
\section{Molleja}
\begin{itemize}
\item {Grp. gram.:f.}
\end{itemize}
\begin{itemize}
\item {Utilização:Pop.}
\end{itemize}
\begin{itemize}
\item {Utilização:Prov.}
\end{itemize}
\begin{itemize}
\item {Utilização:alg.}
\end{itemize}
Glândula carnosa no corpo dos animaes, mormente na parte inferior do pescoço do gado vacum.
Excremento de aves.
O pâncreas nas reses.
Sarrabulho.
(Cp. cast. \textunderscore molleja\textunderscore )
\section{Molemente}
\begin{itemize}
\item {Grp. gram.:adv.}
\end{itemize}
\begin{itemize}
\item {Proveniência:(De \textunderscore mole\textunderscore )}
\end{itemize}
Com moleza; lentamente; com preguiça.
Deleitosamente.
\section{Molenga}
\begin{itemize}
\item {Grp. gram.:m.  e  adj.}
\end{itemize}
Indivíduo muito mole, indolente, preguiçoso; fracalhão.
\section{Molengão}
\begin{itemize}
\item {Grp. gram.:m.  e  adj.}
\end{itemize}
Indivíduo muito molenga.
\section{Molengar}
\begin{itemize}
\item {Grp. gram.:v. i.}
\end{itemize}
Andar molenga. Cf. Filinto, XII, 159.
\section{Molengueiro}
\begin{itemize}
\item {Grp. gram.:adj.}
\end{itemize}
\begin{itemize}
\item {Proveniência:(De \textunderscore molenga\textunderscore )}
\end{itemize}
O mesmo que \textunderscore molanqueiro\textunderscore .
\section{Moletão}
\begin{itemize}
\item {Grp. gram.:m.}
\end{itemize}
\begin{itemize}
\item {Proveniência:(De \textunderscore mole\textunderscore )}
\end{itemize}
Estôfo macio de lan ou de algodão.
\section{Molete}
\begin{itemize}
\item {fónica:lê}
\end{itemize}
\begin{itemize}
\item {Grp. gram.:m.}
\end{itemize}
\begin{itemize}
\item {Utilização:Prov.}
\end{itemize}
Pão pequeno e mole, de trigo.
(Cp. cast. \textunderscore molleta\textunderscore )
\section{Moleza}
\begin{itemize}
\item {Grp. gram.:f.}
\end{itemize}
\begin{itemize}
\item {Proveniência:(Do lat. \textunderscore mollitia\textunderscore )}
\end{itemize}
Qualidade do que é mole.
Falta de fôrças.
Falta de ânimo.
Falta de colorido, em obras de arte.
Languidez, volúpia.
\section{Móli}
\begin{itemize}
\item {Grp. gram.:m.}
\end{itemize}
\begin{itemize}
\item {Proveniência:(Do gr. \textunderscore molu\textunderscore )}
\end{itemize}
Espécie de alho, (\textunderscore allium moly\textunderscore ).
\section{Molibdatado}
\begin{itemize}
\item {Grp. gram.:adj.}
\end{itemize}
\begin{itemize}
\item {Utilização:Miner.}
\end{itemize}
Convertido em molibdato.
\section{Molibdato}
\begin{itemize}
\item {Grp. gram.:m.}
\end{itemize}
\begin{itemize}
\item {Utilização:Chím.}
\end{itemize}
Designação genérica dos saes neutros, formados pela união do ácido molíbdico com as bases.
(Cp. \textunderscore molíbdico\textunderscore )
\section{Molibdene}
\begin{itemize}
\item {Grp. gram.:f.}
\end{itemize}
O mesmo que \textunderscore molibdeno\textunderscore .
\section{Molibdênio}
\begin{itemize}
\item {Grp. gram.:m.}
\end{itemize}
O mesmo que \textunderscore molibdeno\textunderscore .
\section{Molibdenita}
\begin{itemize}
\item {Grp. gram.:f.}
\end{itemize}
Sulfureto de molibdeno.
\section{Molibdenite}
\begin{itemize}
\item {Grp. gram.:f.}
\end{itemize}
O mesmo que \textunderscore molibdenita\textunderscore .
\section{Molibdeno}
\begin{itemize}
\item {Grp. gram.:m.}
\end{itemize}
\begin{itemize}
\item {Proveniência:(Do gr. \textunderscore molubdaina\textunderscore )}
\end{itemize}
Metal sólido, branco, malleável, quási infusível.
\section{Molíbdico}
\begin{itemize}
\item {Grp. gram.:adj.}
\end{itemize}
Diz-se de um ácido, que é o segundo grau da oxigenação do molibdeno.
\section{Molibdina}
\begin{itemize}
\item {Grp. gram.:f.}
\end{itemize}
Óxido de molibdeno natural.
\section{Molibdita}
\begin{itemize}
\item {Grp. gram.:f.}
\end{itemize}
\begin{itemize}
\item {Proveniência:(Do gr. \textunderscore molubdos\textunderscore )}
\end{itemize}
Mineral, que contém partículas de chumbo.
\section{Molibdomancia}
\begin{itemize}
\item {Grp. gram.:f.}
\end{itemize}
\begin{itemize}
\item {Proveniência:(Do gr. \textunderscore molubdos\textunderscore  + \textunderscore manteia\textunderscore )}
\end{itemize}
Suposta arte de adivinhar, por meio de chumbo derretido.
\section{Moliceiro}
\begin{itemize}
\item {Grp. gram.:m.}
\end{itemize}
\begin{itemize}
\item {Grp. gram.:Adj.}
\end{itemize}
Aquele que se emprega na apanha de moliço ou sargaço; sargaceiro.
Diz-se do barco, em que se transporta moliço.
\section{Molícia}
\begin{itemize}
\item {Grp. gram.:f.}
\end{itemize}
\begin{itemize}
\item {Proveniência:(Lat. \textunderscore mollitia\textunderscore )}
\end{itemize}
O mesmo que \textunderscore moleza\textunderscore .
\section{Molície}
\begin{itemize}
\item {Grp. gram.:f.}
\end{itemize}
(V.molícia)
\section{Moliço}
\begin{itemize}
\item {Grp. gram.:m.}
\end{itemize}
\begin{itemize}
\item {Utilização:Prov.}
\end{itemize}
\begin{itemize}
\item {Utilização:minh.}
\end{itemize}
\begin{itemize}
\item {Utilização:T. da Bairrada}
\end{itemize}
\begin{itemize}
\item {Proveniência:(Do lat. hyp. \textunderscore mollicium\textunderscore )}
\end{itemize}
Colmo, em cobertura de choupanas.
Limos e outras plantas aquáticas, que se colhem para adubos de terras.
Caruma sêca.
O mesmo que \textunderscore trote\textunderscore ^2.
Espécie de mato pouco áspero.
Planta aquática.
\section{Molídia}
\begin{itemize}
\item {Grp. gram.:f.}
\end{itemize}
\begin{itemize}
\item {Utilização:Prov.}
\end{itemize}
\begin{itemize}
\item {Utilização:beir.}
\end{itemize}
\begin{itemize}
\item {Proveniência:(De \textunderscore mole\textunderscore )}
\end{itemize}
Rodilha ou sogra, que as mulheres usam na cabeça, por baixo da vasilha, canastra, mólho ou outro pêso que transportam.
\section{Molificação}
\begin{itemize}
\item {Grp. gram.:f.}
\end{itemize}
Acto ou efeito de molificar.
Qualidade do que molifica.
\section{Molificante}
\begin{itemize}
\item {Grp. gram.:adj.}
\end{itemize}
\begin{itemize}
\item {Proveniência:(Lat. \textunderscore mollificans\textunderscore )}
\end{itemize}
Que molifica.
\section{Molificar}
\begin{itemize}
\item {Grp. gram.:v. t.}
\end{itemize}
\begin{itemize}
\item {Utilização:Fig.}
\end{itemize}
\begin{itemize}
\item {Proveniência:(Lat. \textunderscore mollificare\textunderscore )}
\end{itemize}
Tornar mole, amolecer.
Aplacar; suavizar; amansar.
\section{Molificativo}
\begin{itemize}
\item {Grp. gram.:adj.}
\end{itemize}
\begin{itemize}
\item {Proveniência:(De \textunderscore molificar\textunderscore )}
\end{itemize}
O mesmo que \textunderscore emoliente\textunderscore .
\section{Molificável}
\begin{itemize}
\item {Grp. gram.:adj.}
\end{itemize}
Que se póde molificar.
\section{Molifónio}
\begin{itemize}
\item {Grp. gram.:m.}
\end{itemize}
\begin{itemize}
\item {Proveniência:(T. hybr., do lat. \textunderscore mollis\textunderscore  + gr. \textunderscore phone\textunderscore )}
\end{itemize}
Aparelho, destinado a enfraquecer os sons do piano, para se poder estudar, sem incômodo dos vizinhos.
\section{Molim}
\begin{itemize}
\item {Grp. gram.:f.  e  adj.}
\end{itemize}
\begin{itemize}
\item {Utilização:Prov.}
\end{itemize}
\begin{itemize}
\item {Utilização:alg.}
\end{itemize}
\begin{itemize}
\item {Utilização:alent.}
\end{itemize}
\begin{itemize}
\item {Proveniência:(De \textunderscore mole\textunderscore )}
\end{itemize}
Variedade de uva branca.
Chumaço, em que assenta a canga ou o cangalho.
Cp. \textunderscore malim\textunderscore  e \textunderscore molhelha\textunderscore .
\section{Molinha}
\begin{itemize}
\item {Grp. gram.:f.}
\end{itemize}
\begin{itemize}
\item {Proveniência:(De \textunderscore mole\textunderscore )}
\end{itemize}
Variedade de uva branca, sumarenta.
\section{Molinhã}
\begin{itemize}
\item {Grp. gram.:f.}
\end{itemize}
\begin{itemize}
\item {Proveniência:(De \textunderscore molinha\textunderscore )}
\end{itemize}
Casta de uva.
\section{Molinhan}
\begin{itemize}
\item {Grp. gram.:f.}
\end{itemize}
\begin{itemize}
\item {Proveniência:(De \textunderscore molinha\textunderscore )}
\end{itemize}
Casta de uva.
\section{Molípede}
\begin{itemize}
\item {Grp. gram.:adj.}
\end{itemize}
\begin{itemize}
\item {Utilização:Zool.}
\end{itemize}
\begin{itemize}
\item {Proveniência:(Do lat. \textunderscore mollis\textunderscore  + \textunderscore pes\textunderscore )}
\end{itemize}
Que tem pés moles ou brandos.
\section{Molitivo}
\begin{itemize}
\item {Grp. gram.:m.  e  adj.}
\end{itemize}
O mesmo que \textunderscore emoliente\textunderscore .
(Cp. lat. \textunderscore mollitia\textunderscore )
\section{Molito}
\begin{itemize}
\item {Grp. gram.:adj.}
\end{itemize}
\begin{itemize}
\item {Utilização:Bras}
\end{itemize}
\begin{itemize}
\item {Proveniência:(De \textunderscore mole\textunderscore )}
\end{itemize}
Froixo, indolente.
Lascivo.
\section{Mollemente}
\begin{itemize}
\item {Grp. gram.:adv.}
\end{itemize}
\begin{itemize}
\item {Proveniência:(De \textunderscore molle\textunderscore )}
\end{itemize}
Com molleza; lentamente; com preguiça.
Deleitosamente.
\section{Mollenga}
\begin{itemize}
\item {Grp. gram.:m.  e  adj.}
\end{itemize}
Indivíduo muito molle, indolente, preguiçoso; fracalhão.
\section{Mollengão}
\begin{itemize}
\item {Grp. gram.:m.  e  adj.}
\end{itemize}
Indivíduo muito mollenga.
\section{Mollengar}
\begin{itemize}
\item {Grp. gram.:v. i.}
\end{itemize}
Andar mollenga. Cf. Filinto, XII, 159.
\section{Mollengueiro}
\begin{itemize}
\item {Grp. gram.:adj.}
\end{itemize}
\begin{itemize}
\item {Proveniência:(De \textunderscore mollenga\textunderscore )}
\end{itemize}
O mesmo que \textunderscore mollanqueiro\textunderscore .
\section{Molletão}
\begin{itemize}
\item {Grp. gram.:m.}
\end{itemize}
\begin{itemize}
\item {Proveniência:(De \textunderscore molle\textunderscore )}
\end{itemize}
Estôfo macio de lan ou de algodão.
\section{Mollete}
\begin{itemize}
\item {fónica:lê}
\end{itemize}
\begin{itemize}
\item {Grp. gram.:m.}
\end{itemize}
\begin{itemize}
\item {Utilização:Prov.}
\end{itemize}
Pão pequeno e molle, de trigo.
(Cp cast. \textunderscore molleta\textunderscore )
\section{Molleza}
\begin{itemize}
\item {Grp. gram.:f.}
\end{itemize}
\begin{itemize}
\item {Proveniência:(Do lat. \textunderscore mollitia\textunderscore )}
\end{itemize}
Qualidade do que é molle.
Falta de fôrças.
Falta de ânimo.
Falta de colorido, em obras de arte.
Languidez, volúpia.
\section{Molliceiro}
\begin{itemize}
\item {Grp. gram.:m.}
\end{itemize}
\begin{itemize}
\item {Grp. gram.:Adj.}
\end{itemize}
Aquelle que se emprega na apanha de molliço ou sargaço; sargaceiro.
Diz-se do barco, em que se transporta molliço.
\section{Mollícia}
\begin{itemize}
\item {Grp. gram.:f.}
\end{itemize}
\begin{itemize}
\item {Proveniência:(Lat. \textunderscore mollitia\textunderscore )}
\end{itemize}
O mesmo que \textunderscore molleza\textunderscore .
\section{Mollície}
\begin{itemize}
\item {Grp. gram.:f.}
\end{itemize}
(V.mollícia)
\section{Molliço}
\begin{itemize}
\item {Grp. gram.:m.}
\end{itemize}
\begin{itemize}
\item {Utilização:Prov.}
\end{itemize}
\begin{itemize}
\item {Utilização:minh.}
\end{itemize}
\begin{itemize}
\item {Utilização:T. da Bairrada}
\end{itemize}
\begin{itemize}
\item {Proveniência:(Do lat. hyp. \textunderscore mollicium\textunderscore )}
\end{itemize}
Colmo, em cobertura de choupanas.
Limos e outras plantas aquáticas, que se colhem para adubos de terras.
Caruma sêca.
O mesmo que \textunderscore trote\textunderscore ^2.
Espécie de mato pouco áspero.
Planta aquática.
\section{Mollídia}
\begin{itemize}
\item {Grp. gram.:f.}
\end{itemize}
\begin{itemize}
\item {Utilização:Prov.}
\end{itemize}
\begin{itemize}
\item {Utilização:beir.}
\end{itemize}
\begin{itemize}
\item {Proveniência:(De \textunderscore molle\textunderscore )}
\end{itemize}
Rodilha ou sogra, que as mulheres usam na cabeça, por baixo da vasilha, canastra, mólho ou outro pêso que transportam.
\section{Mollificação}
\begin{itemize}
\item {Grp. gram.:f.}
\end{itemize}
Acto ou effeito de mollificar.
Qualidade do que mollifica.
\section{Mollificante}
\begin{itemize}
\item {Grp. gram.:adj.}
\end{itemize}
\begin{itemize}
\item {Proveniência:(Lat. \textunderscore mollificans\textunderscore )}
\end{itemize}
Que mollifica.
\section{Mollificar}
\begin{itemize}
\item {Grp. gram.:v. t.}
\end{itemize}
\begin{itemize}
\item {Utilização:Fig.}
\end{itemize}
\begin{itemize}
\item {Proveniência:(Lat. \textunderscore mollificare\textunderscore )}
\end{itemize}
Tornar molle, ammollecer.
Applacar; suavizar; amansar.
\section{Mollificativo}
\begin{itemize}
\item {Grp. gram.:adj.}
\end{itemize}
\begin{itemize}
\item {Proveniência:(De \textunderscore mollificar\textunderscore )}
\end{itemize}
O mesmo que \textunderscore emolliente\textunderscore .
\section{Mollificável}
\begin{itemize}
\item {Grp. gram.:adj.}
\end{itemize}
Que se póde mollificar.
\section{Mollim}
\begin{itemize}
\item {Grp. gram.:f.  e  adj.}
\end{itemize}
\begin{itemize}
\item {Utilização:Prov.}
\end{itemize}
\begin{itemize}
\item {Utilização:alg.}
\end{itemize}
\begin{itemize}
\item {Utilização:alent.}
\end{itemize}
\begin{itemize}
\item {Proveniência:(De \textunderscore molle\textunderscore )}
\end{itemize}
Variedade de uva branca.
Chumaço, em que assenta a canga ou o cangalho.
Cp. \textunderscore malim\textunderscore  e \textunderscore molhelha\textunderscore .
\section{Mollinha}
\begin{itemize}
\item {Grp. gram.:f.}
\end{itemize}
\begin{itemize}
\item {Proveniência:(De \textunderscore molle\textunderscore )}
\end{itemize}
Variedade de uva branca, sumarenta.
\section{Mollinhan}
\begin{itemize}
\item {Grp. gram.:f.}
\end{itemize}
\begin{itemize}
\item {Proveniência:(De \textunderscore mollinha\textunderscore )}
\end{itemize}
Casta de uva.
\section{Mollípede}
\begin{itemize}
\item {Grp. gram.:adj.}
\end{itemize}
\begin{itemize}
\item {Utilização:Zool.}
\end{itemize}
\begin{itemize}
\item {Proveniência:(Do lat. \textunderscore mollis\textunderscore  + \textunderscore pes\textunderscore )}
\end{itemize}
Que tem pés molles ou brandos.
\section{Molliphónio}
\begin{itemize}
\item {Grp. gram.:m.}
\end{itemize}
\begin{itemize}
\item {Proveniência:(T. hybr., do lat. \textunderscore mollis\textunderscore  + gr. \textunderscore phone\textunderscore )}
\end{itemize}
Apparelho, destinado a enfraquecer os sons do piano, para se poder estudar, sem incômmodo dos vizinhos.
\section{Mollitivo}
\begin{itemize}
\item {Grp. gram.:m.  e  adj.}
\end{itemize}
O mesmo que \textunderscore emolliente\textunderscore .
(Cp. lat. \textunderscore mollitia\textunderscore )
\section{Mollito}
\begin{itemize}
\item {Grp. gram.:adj.}
\end{itemize}
\begin{itemize}
\item {Utilização:Bras}
\end{itemize}
\begin{itemize}
\item {Proveniência:(De \textunderscore molle\textunderscore )}
\end{itemize}
Froixo, indolente.
Lascivo.
\section{Mollo}
\begin{itemize}
\item {Grp. gram.:m.}
\end{itemize}
\begin{itemize}
\item {Utilização:Prov.}
\end{itemize}
Terreno molle.
\section{Mollongó}
\begin{itemize}
\item {Grp. gram.:m.}
\end{itemize}
\begin{itemize}
\item {Utilização:Bras. do N}
\end{itemize}
O mesmo que \textunderscore molleirão\textunderscore .
(Infl. de \textunderscore mollenga\textunderscore ?)
\section{Mollosso}
\begin{itemize}
\item {fónica:lô}
\end{itemize}
\begin{itemize}
\item {Grp. gram.:m.}
\end{itemize}
Grande árvore africana, de fôlhas compostas, glabras, lustrosas, e flôres brancas em longos cachos.
\section{Mollucella}
\begin{itemize}
\item {Grp. gram.:f.}
\end{itemize}
\begin{itemize}
\item {Proveniência:(Do lat. \textunderscore mollis\textunderscore )}
\end{itemize}
Gênero de plantas labiadas.
\section{Mollugem}
\begin{itemize}
\item {Grp. gram.:f.}
\end{itemize}
\begin{itemize}
\item {Proveniência:(Lat. \textunderscore mollugo\textunderscore )}
\end{itemize}
Planta, o mesmo que \textunderscore solda\textunderscore ^1.
\section{Mollura}
\begin{itemize}
\item {Grp. gram.:f.}
\end{itemize}
\begin{itemize}
\item {Utilização:Ant.}
\end{itemize}
\begin{itemize}
\item {Proveniência:(De \textunderscore molle\textunderscore )}
\end{itemize}
Orvalho copioso, que ammollece o solo.
\section{Mollúria}
\begin{itemize}
\item {Grp. gram.:f.}
\end{itemize}
\begin{itemize}
\item {Grp. gram.:M.}
\end{itemize}
\begin{itemize}
\item {Utilização:Pop.}
\end{itemize}
\begin{itemize}
\item {Proveniência:(De \textunderscore molle\textunderscore )}
\end{itemize}
Molleza.
Relento; mollura.
Homem tímido, inepto, inhenho.
\section{Mollusco}
\begin{itemize}
\item {Grp. gram.:m.}
\end{itemize}
\begin{itemize}
\item {Proveniência:(Lat. \textunderscore mollusca\textunderscore )}
\end{itemize}
Nome dos animaes sem vértebras, que formam uma das ramificações do reino animal, e que comprehendem seis classes: cephalópodes, esterópodes, gasterópodes, acéphalos, brachiópodes e cirrhópodes.
\section{Molluscóides}
\begin{itemize}
\item {Grp. gram.:m. pl.}
\end{itemize}
\begin{itemize}
\item {Proveniência:(De \textunderscore mollusco\textunderscore  + gr. \textunderscore eidos\textunderscore )}
\end{itemize}
Designação, adoptada por alguns naturalistas para os brachiópodes e bryozoários.
\section{Mollúsculo}
\begin{itemize}
\item {Grp. gram.:m.}
\end{itemize}
Pequeno mollusco.
\section{Molo}
\begin{itemize}
\item {Grp. gram.:m.}
\end{itemize}
\begin{itemize}
\item {Utilização:Ant.}
\end{itemize}
Carregação de navio; frete. Cf. Pant. de Aveiro, \textunderscore Itiner.\textunderscore , 35, (2.^a ed.)
(Talvez de \textunderscore mole\textunderscore )
\section{Molo}
\begin{itemize}
\item {Grp. gram.:m.}
\end{itemize}
\begin{itemize}
\item {Utilização:Prov.}
\end{itemize}
Terreno mole.
\section{Molongó}
\begin{itemize}
\item {Grp. gram.:m.}
\end{itemize}
\begin{itemize}
\item {Utilização:Bras. do N}
\end{itemize}
O mesmo que \textunderscore moleirão\textunderscore .
(Infl. de \textunderscore molenga\textunderscore ?)
\section{Molosso}
\begin{itemize}
\item {fónica:lô}
\end{itemize}
\begin{itemize}
\item {Grp. gram.:m.}
\end{itemize}
\begin{itemize}
\item {Utilização:Fig.}
\end{itemize}
\begin{itemize}
\item {Proveniência:(Lat. \textunderscore molossus\textunderscore )}
\end{itemize}
Espécie de cão, que os antigos empregavam na caça e na guarda de gados.
Pé de verso, de três sýllabas longas.
Indivíduo turbulento, valentão.
\section{Molosso}
\begin{itemize}
\item {fónica:lô}
\end{itemize}
\begin{itemize}
\item {Grp. gram.:m.}
\end{itemize}
Grande árvore africana, de fôlhas compostas, glabras, lustrosas, e flôres brancas em longos cachos.
\section{Molucela}
\begin{itemize}
\item {Grp. gram.:f.}
\end{itemize}
\begin{itemize}
\item {Proveniência:(Do lat. \textunderscore mollis\textunderscore )}
\end{itemize}
Gênero de plantas labiadas.
\section{Molugem}
\begin{itemize}
\item {Grp. gram.:f.}
\end{itemize}
\begin{itemize}
\item {Proveniência:(Lat. \textunderscore mollugo\textunderscore )}
\end{itemize}
Planta, o mesmo que \textunderscore solda\textunderscore ^1.
\section{Molulo}
\begin{itemize}
\item {Grp. gram.:m.}
\end{itemize}
Arbusto africano, de casca amarga e medicinal.
\section{Molungo}
\begin{itemize}
\item {Grp. gram.:m.}
\end{itemize}
Árvore africana, da fam. das leguminosas, (\textunderscore erythrina suberifera\textunderscore , Welw.).
\section{Molura}
\begin{itemize}
\item {Grp. gram.:f.}
\end{itemize}
\begin{itemize}
\item {Utilização:Ant.}
\end{itemize}
\begin{itemize}
\item {Proveniência:(De \textunderscore mole\textunderscore )}
\end{itemize}
Orvalho copioso, que amolece o solo.
\section{Molúria}
\begin{itemize}
\item {Grp. gram.:f.}
\end{itemize}
\begin{itemize}
\item {Grp. gram.:M.}
\end{itemize}
\begin{itemize}
\item {Utilização:Pop.}
\end{itemize}
\begin{itemize}
\item {Proveniência:(De \textunderscore mole\textunderscore )}
\end{itemize}
Moleza.
Relento; molura.
Homem tímido, inepto, inhenho.
\section{Molusco}
\begin{itemize}
\item {Grp. gram.:m.}
\end{itemize}
\begin{itemize}
\item {Proveniência:(Lat. \textunderscore mollusca\textunderscore )}
\end{itemize}
Nome dos animaes sem vértebras, que formam uma das ramificações do reino animal, e que compreendem seis classes: cefalópodes, esterópodes, gasterópodes, acéfalos, braquiópodes e cirrópodes.
\section{Moluscóides}
\begin{itemize}
\item {Grp. gram.:m. pl.}
\end{itemize}
\begin{itemize}
\item {Proveniência:(De \textunderscore molusco\textunderscore  + gr. \textunderscore eidos\textunderscore )}
\end{itemize}
Designação, adoptada por alguns naturalistas para os braquiópodes e briozoários.
\section{Molúsculo}
\begin{itemize}
\item {Grp. gram.:m.}
\end{itemize}
Pequeno molusco.
\section{Móly}
\begin{itemize}
\item {Grp. gram.:m.}
\end{itemize}
\begin{itemize}
\item {Proveniência:(Do gr. \textunderscore molu\textunderscore )}
\end{itemize}
Espécie de alho, (\textunderscore allium moly\textunderscore ).
\section{Molybdatado}
\begin{itemize}
\item {Grp. gram.:adj.}
\end{itemize}
\begin{itemize}
\item {Utilização:Miner.}
\end{itemize}
Convertido em molybdato.
\section{Molybdato}
\begin{itemize}
\item {Grp. gram.:m.}
\end{itemize}
\begin{itemize}
\item {Utilização:Chím.}
\end{itemize}
Designação genérica dos saes neutros, formados pela união do ácido molýbdico com as bases.
(Cp. \textunderscore molýbdico\textunderscore )
\section{Molybdene}
\begin{itemize}
\item {Grp. gram.:f.}
\end{itemize}
O mesmo que \textunderscore molybdeno\textunderscore .
\section{Molybdênio}
\begin{itemize}
\item {Grp. gram.:m.}
\end{itemize}
O mesmo que \textunderscore molybdeno\textunderscore .
\section{Molybdenita}
\begin{itemize}
\item {Grp. gram.:f.}
\end{itemize}
Sulfureto de molybdeno.
\section{Molybdenite}
\begin{itemize}
\item {Grp. gram.:f.}
\end{itemize}
O mesmo que \textunderscore molybdenita\textunderscore .
\section{Molybdeno}
\begin{itemize}
\item {Grp. gram.:m.}
\end{itemize}
\begin{itemize}
\item {Proveniência:(Do gr. \textunderscore molubdaina\textunderscore )}
\end{itemize}
Metal sólido, branco, malleável, quási infusível.
\section{Molýbdico}
\begin{itemize}
\item {Grp. gram.:adj.}
\end{itemize}
Diz-se de um ácido, que é o segundo grau da oxygenação do molybdeno.
\section{Molybdina}
\begin{itemize}
\item {Grp. gram.:f.}
\end{itemize}
Óxydo de molybdeno natural.
\section{Molybdita}
\begin{itemize}
\item {Grp. gram.:f.}
\end{itemize}
\begin{itemize}
\item {Proveniência:(Do gr. \textunderscore molubdos\textunderscore )}
\end{itemize}
Mineral, que contém partículas de chumbo.
\section{Molybdomancia}
\begin{itemize}
\item {Grp. gram.:f.}
\end{itemize}
\begin{itemize}
\item {Proveniência:(Do gr. \textunderscore molubdos\textunderscore  + \textunderscore manteia\textunderscore )}
\end{itemize}
Supposta arte de adivinhar, por meio de chumbo derretido.
\section{Momanás}
\begin{itemize}
\item {Grp. gram.:m. pl.}
\end{itemize}
Tríbo de Índios no Pará.
\section{Mômaro}
\begin{itemize}
\item {Grp. gram.:m.}
\end{itemize}
\begin{itemize}
\item {Utilização:Ant.}
\end{itemize}
Bobo, jogral, histrião. Cf. Sabugosa, \textunderscore Donas\textunderscore , I, 32; \textunderscore Michaëlis\textunderscore , ed. crit. do \textunderscore Cancioneiro da Ajuda\textunderscore , I, 758.
(Gall. \textunderscore momaro\textunderscore )
\section{Mombaca}
\begin{itemize}
\item {Grp. gram.:f.}
\end{itemize}
Fruto acre, que no Brasil se emprega como adubo culinário.
\section{Mombina}
\begin{itemize}
\item {Grp. gram.:f.}
\end{itemize}
(V.imbu)
\section{Mombiú}
\begin{itemize}
\item {Grp. gram.:m.}
\end{itemize}
Árvore de Angola.
\section{Mombóia-xió}
\begin{itemize}
\item {Grp. gram.:f.}
\end{itemize}
\begin{itemize}
\item {Utilização:Bras. do N}
\end{itemize}
Espécie de gaita, usada pelos caboclos.
\section{Mombutos}
\begin{itemize}
\item {Grp. gram.:m. pl.}
\end{itemize}
Povos africanos do Sudão.
\section{Momentaneamente}
\begin{itemize}
\item {Grp. gram.:adv.}
\end{itemize}
De modo momentâneo; rapidamente.
Neste momento; naquelle momento.
\section{Momentâneo}
\begin{itemize}
\item {Grp. gram.:adj.}
\end{itemize}
\begin{itemize}
\item {Proveniência:(Lat. \textunderscore momentaneus\textunderscore )}
\end{itemize}
Que só dura um momento; instantâneo.
Rápido.
Transitório.
\section{Momentão}
\begin{itemize}
\item {Grp. gram.:m.}
\end{itemize}
Processo de enxertia de videiras, devendo a videira, que fórma o garfo, estar perto daquella que se quere enxertar.
\section{Momento}
\begin{itemize}
\item {Grp. gram.:m.}
\end{itemize}
\begin{itemize}
\item {Grp. gram.:Loc. adv.}
\end{itemize}
\begin{itemize}
\item {Grp. gram.:Loc. adv.}
\end{itemize}
\begin{itemize}
\item {Proveniência:(Lat. \textunderscore momentum\textunderscore )}
\end{itemize}
Espaço pequeníssimo, mas indeterminado, de tempo.
Occasião asada; opportunidade.
A occasião ou o tempo, em que alguma coisa se faz ou succede.
Circunstância.
\textunderscore Estar em momento\textunderscore , estar em dia, estar bem informado:«\textunderscore custava-lhe muito não estar em momento com os successos.\textunderscore »Camillo, \textunderscore Onde está a Felicidade\textunderscore ? 279.
\textunderscore Num momento\textunderscore , logo, imediatamente.
\textunderscore De momento a momento\textunderscore , sem intervallo, successivamente.
\section{Momento}
\begin{itemize}
\item {Grp. gram.:m.}
\end{itemize}
\begin{itemize}
\item {Utilização:T. de Mecânica}
\end{itemize}
\begin{itemize}
\item {Proveniência:(Lat. \textunderscore momentum\textunderscore , no sentido de impulso)}
\end{itemize}
Producto de um braço de alavanca pela fôrça que se lhe applica perpendicularmente.
Producto de qualquer fôrça por uma distância.
Producto de uma massa por uma velocidade ou uma quantidade de movimento.
\section{Momento}
\begin{itemize}
\item {Grp. gram.:adj.}
\end{itemize}
\begin{itemize}
\item {Proveniência:(De \textunderscore momo\textunderscore )}
\end{itemize}
Que faz momices. Cf. Filinto, V, 132.
\section{Momentoso}
\begin{itemize}
\item {Grp. gram.:adj.}
\end{itemize}
\begin{itemize}
\item {Proveniência:(De \textunderscore momento\textunderscore ^1)}
\end{itemize}
Grave, importante: \textunderscore assumpto momentoso\textunderscore .
\section{Mómia}
\begin{itemize}
\item {Grp. gram.:f.}
\end{itemize}
\begin{itemize}
\item {Utilização:Ant.}
\end{itemize}
O mesmo que \textunderscore múmia\textunderscore .
\section{Momice}
\begin{itemize}
\item {Grp. gram.:f.}
\end{itemize}
\begin{itemize}
\item {Utilização:Fig.}
\end{itemize}
\begin{itemize}
\item {Proveniência:(De \textunderscore momo\textunderscore )}
\end{itemize}
Esgares; careta.
Disfarce; hypocrisia.
\section{Momo}
\begin{itemize}
\item {Grp. gram.:m.}
\end{itemize}
\begin{itemize}
\item {Utilização:Ant.}
\end{itemize}
\begin{itemize}
\item {Utilização:Fig.}
\end{itemize}
\begin{itemize}
\item {Proveniência:(De \textunderscore Momo\textunderscore , n. p. myth.)}
\end{itemize}
Momice.
Representação mímica.
Farsa satírica.
Actor dessa farça.
Escárneo.
\section{Momórdica}
\begin{itemize}
\item {Grp. gram.:f.}
\end{itemize}
\begin{itemize}
\item {Utilização:Bot.}
\end{itemize}
Designação scientífica da balsamina.
Gênero de plantas cucurbitáceas.
\section{Momota}
\begin{itemize}
\item {Grp. gram.:f.}
\end{itemize}
Pássaro dentirostro da América.
\section{Momposteiro}
\begin{itemize}
\item {Grp. gram.:m.}
\end{itemize}
\begin{itemize}
\item {Utilização:Ant.}
\end{itemize}
O mesmo que \textunderscore mamposteiro\textunderscore .--Em Lisbôa, na estrada da \textunderscore Penha\textunderscore , lê-se o termo numa inscripção antiga, feita em azulejos, sôbre a padieira de uma porta.
\section{Mona}
\begin{itemize}
\item {Grp. gram.:f.}
\end{itemize}
\begin{itemize}
\item {Utilização:Chul.}
\end{itemize}
\begin{itemize}
\item {Utilização:Fam.}
\end{itemize}
\begin{itemize}
\item {Utilização:Prov.}
\end{itemize}
\begin{itemize}
\item {Utilização:alg.}
\end{itemize}
\begin{itemize}
\item {Proveniência:(De \textunderscore mono\textunderscore )}
\end{itemize}
Fêmea do mono.
Bebedeira, carraspana.
Cabeça.
Amuo.
Boneca de trapos.
O mesmo que \textunderscore bioba\textunderscore .
\section{Mona}
\begin{itemize}
\item {Grp. gram.:f.}
\end{itemize}
Armadura de ferro, usada debaixo do calção por picadores de toiros, como resguardo contra as cornadas.
\section{Monacagem}
\begin{itemize}
\item {Grp. gram.:f.}
\end{itemize}
\begin{itemize}
\item {Utilização:Deprec.}
\end{itemize}
\begin{itemize}
\item {Proveniência:(Do lat. \textunderscore monachus\textunderscore )}
\end{itemize}
Os monges.
\section{Monacal}
\begin{itemize}
\item {Grp. gram.:adj.}
\end{itemize}
\begin{itemize}
\item {Proveniência:(Do lat. \textunderscore monachus\textunderscore )}
\end{itemize}
Relativo a monge ou á vida dos conventos.
\section{Monacalmente}
\begin{itemize}
\item {Grp. gram.:adv.}
\end{itemize}
De modo monacal.
Á maneira dos conventos ou dos monges.
\section{Monacantho}
\begin{itemize}
\item {Grp. gram.:adj.}
\end{itemize}
\begin{itemize}
\item {Proveniência:(Do gr. \textunderscore monos\textunderscore  + \textunderscore akantha\textunderscore )}
\end{itemize}
Que tem uma só espinha.
\section{Monacanto}
\begin{itemize}
\item {Grp. gram.:adj.}
\end{itemize}
\begin{itemize}
\item {Proveniência:(Do gr. \textunderscore monos\textunderscore  + \textunderscore akantha\textunderscore )}
\end{itemize}
Que tem uma só espinha.
\section{Monacato}
\begin{itemize}
\item {Grp. gram.:m.}
\end{itemize}
\begin{itemize}
\item {Proveniência:(Do lat. \textunderscore monachus\textunderscore )}
\end{itemize}
Estado ou vida de monge.
\section{Monacetina}
\begin{itemize}
\item {Grp. gram.:f.}
\end{itemize}
\begin{itemize}
\item {Utilização:Chím.}
\end{itemize}
\begin{itemize}
\item {Proveniência:(Do gr. \textunderscore monos\textunderscore  + lat. \textunderscore acetum\textunderscore )}
\end{itemize}
Líquido neutro, que cheira a éther.
\section{Monachagem}
\begin{itemize}
\item {fónica:ca}
\end{itemize}
\begin{itemize}
\item {Grp. gram.:f.}
\end{itemize}
\begin{itemize}
\item {Utilização:Deprec.}
\end{itemize}
\begin{itemize}
\item {Proveniência:(Do lat. \textunderscore monachus\textunderscore )}
\end{itemize}
Os monges.
\section{Monachal}
\begin{itemize}
\item {fónica:cal}
\end{itemize}
\begin{itemize}
\item {Grp. gram.:adj.}
\end{itemize}
\begin{itemize}
\item {Proveniência:(Do lat. \textunderscore monachus\textunderscore )}
\end{itemize}
Relativo a monge ou á vida dos conventos.
\section{Monachalmente}
\begin{itemize}
\item {fónica:cal}
\end{itemize}
\begin{itemize}
\item {Grp. gram.:adv.}
\end{itemize}
De modo monachal.
Á maneira dos conventos ou dos monges.
\section{Monachato}
\begin{itemize}
\item {fónica:ca}
\end{itemize}
\begin{itemize}
\item {Grp. gram.:m.}
\end{itemize}
\begin{itemize}
\item {Proveniência:(Do lat. \textunderscore monachus\textunderscore )}
\end{itemize}
Estado ou vida de monge.
\section{Monachino}
\begin{itemize}
\item {fónica:qui}
\end{itemize}
\begin{itemize}
\item {Grp. gram.:m.}
\end{itemize}
\begin{itemize}
\item {Utilização:Ant.}
\end{itemize}
\begin{itemize}
\item {Proveniência:(Lat. \textunderscore monachinus\textunderscore )}
\end{itemize}
Sacristão; menino de côro.
\section{Monachismo}
\begin{itemize}
\item {fónica:quis}
\end{itemize}
\begin{itemize}
\item {Grp. gram.:m.}
\end{itemize}
O mesmo que \textunderscore monachato\textunderscore .
\section{Mónaco}
\begin{itemize}
\item {Grp. gram.:m.}
\end{itemize}
Pequena moéda de cobre do principado de Mónaco.
\section{Monacórdio}
\begin{itemize}
\item {Grp. gram.:m.}
\end{itemize}
O mesmo que \textunderscore monocórdio\textunderscore .
\section{Monáda}
\begin{itemize}
\item {Grp. gram.:f.}
\end{itemize}
\begin{itemize}
\item {Proveniência:(De \textunderscore mono\textunderscore )}
\end{itemize}
Macaquices; esgares; trejeitos.
Porção de monos.
\section{Mónada}
\begin{itemize}
\item {Grp. gram.:f.}
\end{itemize}
\begin{itemize}
\item {Utilização:Philos.}
\end{itemize}
\begin{itemize}
\item {Proveniência:(Lat. \textunderscore monas\textunderscore , \textunderscore monadis\textunderscore )}
\end{itemize}
Elementos das coisas, ou substâncias simples, incorruptíveis, nascidas com a criação, inacessíveis a toda a influência externa, mas sujeitas a mudanças internas, de que resulta a percepção, (na theoria de Leibnitz).
Unidade perfeita que, segundo a philosophia de Pythágoras, compreende o espírito e a matéria, sem separação alguma, e constítue o próprio Deus.
Número, formado de uma só figura, como 1, 2, 3, etc.
Gênero de animáculos microscópicos.
\section{Monadário}
\begin{itemize}
\item {Grp. gram.:adj.}
\end{itemize}
\begin{itemize}
\item {Grp. gram.:M. pl.}
\end{itemize}
Relativo a mónada.
Pequeno como as mónadas.
Família de animáculos, que tem por typo o gênero mónada.
\section{Mónade}
\begin{itemize}
\item {Grp. gram.:f.}
\end{itemize}
O mesmo ou melhor que \textunderscore mónada\textunderscore .
\section{Monadelfia}
\begin{itemize}
\item {Grp. gram.:f.}
\end{itemize}
\begin{itemize}
\item {Utilização:Bot.}
\end{itemize}
\begin{itemize}
\item {Proveniência:(Do gr. \textunderscore monos\textunderscore  + \textunderscore adelphos\textunderscore )}
\end{itemize}
União dos estames da flôr, formando um só feixe.
\section{Monadelfo}
\begin{itemize}
\item {Grp. gram.:adj.}
\end{itemize}
\begin{itemize}
\item {Utilização:Bot.}
\end{itemize}
\begin{itemize}
\item {Proveniência:(Do gr. \textunderscore monos\textunderscore  + \textunderscore adelphos\textunderscore )}
\end{itemize}
Que tem os estames reunidos num só fascículo.
\section{Monadelphia}
\begin{itemize}
\item {Grp. gram.:f.}
\end{itemize}
\begin{itemize}
\item {Utilização:Bot.}
\end{itemize}
\begin{itemize}
\item {Proveniência:(Do gr. \textunderscore monos\textunderscore  + \textunderscore adelphos\textunderscore )}
\end{itemize}
União dos estames da flôr, formando um só feixe.
\section{Monadelpho}
\begin{itemize}
\item {Grp. gram.:adj.}
\end{itemize}
\begin{itemize}
\item {Utilização:Bot.}
\end{itemize}
\begin{itemize}
\item {Proveniência:(Do gr. \textunderscore monos\textunderscore  + \textunderscore adelphos\textunderscore )}
\end{itemize}
Que tem os estames reunidos num só fascículo.
\section{Monadênia}
\begin{itemize}
\item {Grp. gram.:f.}
\end{itemize}
\begin{itemize}
\item {Proveniência:(Do gr. \textunderscore monos\textunderscore  + \textunderscore aden\textunderscore )}
\end{itemize}
Planta herbácea, da fam. das orchídeas.
\section{Monádio-lentilha}
\begin{itemize}
\item {Grp. gram.:m.}
\end{itemize}
Bactéria, que tem apenas uma céllula e, como meio locomotor, uma só celha.
(Cp. \textunderscore mónada\textunderscore )
\section{Monadismo}
\begin{itemize}
\item {Grp. gram.:m.}
\end{itemize}
Systema philosóphico, segundo o qual, o universo é um conjunto de mónadas.
\section{Monadista}
\begin{itemize}
\item {Grp. gram.:m.}
\end{itemize}
\begin{itemize}
\item {Proveniência:(De \textunderscore mónada\textunderscore )}
\end{itemize}
Sectário do monadismo.
\section{Monadologia}
\begin{itemize}
\item {Grp. gram.:f.}
\end{itemize}
\begin{itemize}
\item {Proveniência:(Do gr. \textunderscore monas\textunderscore  + \textunderscore logos\textunderscore )}
\end{itemize}
Systema de Leibnitz, á cêrca das mónadas.
\section{Monadológico}
\begin{itemize}
\item {Grp. gram.:adj.}
\end{itemize}
Relativo á monadologia.
\section{Monailo}
\begin{itemize}
\item {Grp. gram.:m.}
\end{itemize}
\begin{itemize}
\item {Utilização:Prov.}
\end{itemize}
Gaita, com que os castradores de porcos se annunciam pelos povoados.
(Corr. de \textunderscore monaulo\textunderscore )
\section{Monamido}
\begin{itemize}
\item {Grp. gram.:m.}
\end{itemize}
\begin{itemize}
\item {Utilização:Chím.}
\end{itemize}
Corpo, resultante da perda de uma ou duas moléculas de água do oxalato de ammoníaco ácido.
\section{Monaminas}
\begin{itemize}
\item {Grp. gram.:f. pl.}
\end{itemize}
\begin{itemize}
\item {Utilização:Chím.}
\end{itemize}
\begin{itemize}
\item {Proveniência:(De \textunderscore monos\textunderscore  gr. + \textunderscore aminas\textunderscore )}
\end{itemize}
Aminas, derivadas de uma molécula de ammoníaco.
\section{Monandria}
\begin{itemize}
\item {Grp. gram.:f.}
\end{itemize}
\begin{itemize}
\item {Proveniência:(De \textunderscore monandro\textunderscore )}
\end{itemize}
Primeira classe do systema botânico de Linneu.
\section{Monândrico}
\begin{itemize}
\item {Grp. gram.:adj.}
\end{itemize}
O mesmo que \textunderscore monandro\textunderscore .
\section{Monandro}
\begin{itemize}
\item {Grp. gram.:adj.}
\end{itemize}
\begin{itemize}
\item {Utilização:Bot.}
\end{itemize}
\begin{itemize}
\item {Proveniência:(Do gr. \textunderscore monos\textunderscore  + \textunderscore aner\textunderscore , \textunderscore andros\textunderscore )}
\end{itemize}
Que tem um só estame.
\section{Monangama}
\begin{itemize}
\item {Grp. gram.:f.}
\end{itemize}
\begin{itemize}
\item {Proveniência:(T. lund.)}
\end{itemize}
Árvore africana, de fôlhas inteiras, coriáceas, e flôres miúdas, brancas, inodoras.
\section{Monantero}
\begin{itemize}
\item {Grp. gram.:adj.}
\end{itemize}
\begin{itemize}
\item {Utilização:Bot.}
\end{itemize}
\begin{itemize}
\item {Proveniência:(Do gr. \textunderscore monos\textunderscore  + \textunderscore antheros\textunderscore )}
\end{itemize}
Que tem uma só antera.
\section{Monanthero}
\begin{itemize}
\item {Grp. gram.:adj.}
\end{itemize}
\begin{itemize}
\item {Utilização:Bot.}
\end{itemize}
\begin{itemize}
\item {Proveniência:(Do gr. \textunderscore monos\textunderscore  + \textunderscore antheros\textunderscore )}
\end{itemize}
Que tem uma só anthera.
\section{Monantho}
\begin{itemize}
\item {Grp. gram.:adj.}
\end{itemize}
\begin{itemize}
\item {Utilização:Bot.}
\end{itemize}
\begin{itemize}
\item {Proveniência:(Do gr. \textunderscore monos\textunderscore  + \textunderscore anthos\textunderscore )}
\end{itemize}
Que tem só uma flôr.
Que tem flôres solitárias.
\section{Monanthropia}
\begin{itemize}
\item {Grp. gram.:f.}
\end{itemize}
\begin{itemize}
\item {Proveniência:(Do gr. \textunderscore monos\textunderscore  + \textunderscore anthropos\textunderscore )}
\end{itemize}
Systema antropológico, que só admitte originariamente uma raça de homens.
\section{Monanto}
\begin{itemize}
\item {Grp. gram.:adj.}
\end{itemize}
\begin{itemize}
\item {Utilização:Bot.}
\end{itemize}
\begin{itemize}
\item {Proveniência:(Do gr. \textunderscore monos\textunderscore  + \textunderscore anthos\textunderscore )}
\end{itemize}
Que tem só uma flôr.
Que tem flôres solitárias.
\section{Monantropia}
\begin{itemize}
\item {Grp. gram.:f.}
\end{itemize}
\begin{itemize}
\item {Proveniência:(Do gr. \textunderscore monos\textunderscore  + \textunderscore anthropos\textunderscore )}
\end{itemize}
Sistema antropológico, que só admite originariamente uma raça de homens.
\section{Monaquino}
\begin{itemize}
\item {Grp. gram.:m.}
\end{itemize}
\begin{itemize}
\item {Utilização:Ant.}
\end{itemize}
\begin{itemize}
\item {Proveniência:(Lat. \textunderscore monachinus\textunderscore )}
\end{itemize}
Sacristão; menino de côro.
\section{Monaquismo}
\begin{itemize}
\item {Grp. gram.:m.}
\end{itemize}
O mesmo que \textunderscore monacato\textunderscore .
\section{Monarca}
\begin{itemize}
\item {Grp. gram.:m.}
\end{itemize}
\begin{itemize}
\item {Utilização:Fig.}
\end{itemize}
\begin{itemize}
\item {Utilização:Bras. do S}
\end{itemize}
\begin{itemize}
\item {Grp. gram.:Adj.}
\end{itemize}
\begin{itemize}
\item {Utilização:Bras. do N}
\end{itemize}
\begin{itemize}
\item {Proveniência:(Lat. \textunderscore monarcha\textunderscore )}
\end{itemize}
Chefe supremo, vitalício e, geralmente, hereditário, de uma nação ou Estado.
Soberano.
Pessôa ou coisa que domina.
Camponês rude e armado.
Cavalo, que anda com garbo.
Gaúcho, que anda com garbo.
Muito grande: \textunderscore um navio monarca\textunderscore .
\section{Monarcha}
\begin{itemize}
\item {fónica:ca}
\end{itemize}
\begin{itemize}
\item {Grp. gram.:m.}
\end{itemize}
\begin{itemize}
\item {Utilização:Fig.}
\end{itemize}
\begin{itemize}
\item {Utilização:Bras. do S}
\end{itemize}
\begin{itemize}
\item {Grp. gram.:Adj.}
\end{itemize}
\begin{itemize}
\item {Utilização:Bras. do N}
\end{itemize}
\begin{itemize}
\item {Proveniência:(Lat. \textunderscore monarcha\textunderscore )}
\end{itemize}
Chefe supremo, vitalício e, geralmente, hereditário, de uma nação ou Estado.
Soberano.
Pessôa ou coisa que domina.
Camponês rude e armado.
Cavallo, que anda com garbo.
Gaúcho, que anda com garbo.
Muito grande: \textunderscore um navio monarcha\textunderscore .
\section{Monarchia}
\begin{itemize}
\item {fónica:qui}
\end{itemize}
\begin{itemize}
\item {Grp. gram.:f.}
\end{itemize}
\begin{itemize}
\item {Proveniência:(Lat. \textunderscore monarchia\textunderscore )}
\end{itemize}
Govêrno supremo de um Estado, exercido por um monarca.
Estado, cujo chefe é monarca.
\section{Monarchiar}
\begin{itemize}
\item {fónica:qui}
\end{itemize}
\begin{itemize}
\item {Grp. gram.:v. i.}
\end{itemize}
\begin{itemize}
\item {Utilização:P. us.}
\end{itemize}
\begin{itemize}
\item {Utilização:Ext.}
\end{itemize}
\begin{itemize}
\item {Proveniência:(De \textunderscore monarchia\textunderscore )}
\end{itemize}
Desempenhar as funcções de monarcha.
Dominar, têr império.
\section{Monarchicamente}
\begin{itemize}
\item {fónica:qui}
\end{itemize}
\begin{itemize}
\item {Grp. gram.:adv.}
\end{itemize}
De modo monárchico.
Á semelhança de monarcha.
Segundo o systema monárchico.
\section{Monárchico}
\begin{itemize}
\item {fónica:qui}
\end{itemize}
\begin{itemize}
\item {Grp. gram.:adj.}
\end{itemize}
\begin{itemize}
\item {Grp. gram.:M.}
\end{itemize}
\begin{itemize}
\item {Proveniência:(Lat. \textunderscore monarchicus\textunderscore )}
\end{itemize}
Relativo a monarcha ou á monarchia.
O mesmo que \textunderscore monarchista\textunderscore .
\section{Monarchismo}
\begin{itemize}
\item {fónica:quis}
\end{itemize}
\begin{itemize}
\item {Grp. gram.:m.}
\end{itemize}
\begin{itemize}
\item {Proveniência:(De \textunderscore monarcha\textunderscore )}
\end{itemize}
Systema político dos monarchistas.
\section{Monarchista}
\begin{itemize}
\item {fónica:quis}
\end{itemize}
\begin{itemize}
\item {Grp. gram.:m.}
\end{itemize}
\begin{itemize}
\item {Proveniência:(De \textunderscore monarcha\textunderscore )}
\end{itemize}
Sectário da monarchia ou do systema monárchico.
\section{Monarcholatria}
\begin{itemize}
\item {fónica:co}
\end{itemize}
\begin{itemize}
\item {Grp. gram.:f.}
\end{itemize}
\begin{itemize}
\item {Proveniência:(Do gr. \textunderscore monos\textunderscore  + \textunderscore arckhein\textunderscore  + \textunderscore latreia\textunderscore )}
\end{itemize}
Adoração dos monarchas, muito praticada pelos antigos povos asiáticos.
\section{Monarchóphago}
\begin{itemize}
\item {fónica:có}
\end{itemize}
\begin{itemize}
\item {Grp. gram.:adj.}
\end{itemize}
\begin{itemize}
\item {Utilização:Fig.}
\end{itemize}
\begin{itemize}
\item {Proveniência:(Do gr. \textunderscore monarkhes\textunderscore  + \textunderscore phagein\textunderscore )}
\end{itemize}
Que desejaria tragar os monarchas, pelo ódio que lhes tem. Cf. Camilo, \textunderscore Narcóticos\textunderscore , I, 286.
\section{Monarcófago}
\begin{itemize}
\item {Grp. gram.:adj.}
\end{itemize}
\begin{itemize}
\item {Utilização:Fig.}
\end{itemize}
\begin{itemize}
\item {Proveniência:(Do gr. \textunderscore monarkhes\textunderscore  + \textunderscore phagein\textunderscore )}
\end{itemize}
Que desejaria tragar os monarcas, pelo ódio que lhes tem. Cf. Camilo, \textunderscore Narcóticos\textunderscore , I, 286.
\section{Monarcolatria}
\begin{itemize}
\item {Grp. gram.:f.}
\end{itemize}
\begin{itemize}
\item {Proveniência:(Do gr. \textunderscore monos\textunderscore  + \textunderscore arckhein\textunderscore  + \textunderscore latreia\textunderscore )}
\end{itemize}
Adoração dos monarchas, muito praticada pelos antigos povos asiáticos.
\section{Monarda}
\begin{itemize}
\item {Grp. gram.:f.}
\end{itemize}
\begin{itemize}
\item {Proveniência:(De \textunderscore Monardez\textunderscore , n. p.)}
\end{itemize}
Gênero de plantas, o mais considerável da fam. das labiadas.
\section{Monaria}
\begin{itemize}
\item {Grp. gram.:f.}
\end{itemize}
O mesmo que \textunderscore monáda\textunderscore ; chusma de monos.
\section{Monarquia}
\begin{itemize}
\item {Grp. gram.:f.}
\end{itemize}
\begin{itemize}
\item {Proveniência:(Lat. \textunderscore monarchia\textunderscore )}
\end{itemize}
Governo supremo de um Estado, exercido por um monarca.
Estado, cujo chefe é monarca.
\section{Monarquiar}
\begin{itemize}
\item {Grp. gram.:v. i.}
\end{itemize}
\begin{itemize}
\item {Utilização:P. us.}
\end{itemize}
\begin{itemize}
\item {Utilização:Ext.}
\end{itemize}
\begin{itemize}
\item {Proveniência:(De \textunderscore monarquia\textunderscore )}
\end{itemize}
Desempenhar as funções de monarca.
Dominar, têr império.
\section{Monarquicamente}
\begin{itemize}
\item {Grp. gram.:adv.}
\end{itemize}
De modo monárquico.
Á semelhança de monarca.
Segundo o sistema monárquico.
\section{Monárquico}
\begin{itemize}
\item {Grp. gram.:adj.}
\end{itemize}
\begin{itemize}
\item {Grp. gram.:M.}
\end{itemize}
\begin{itemize}
\item {Proveniência:(Lat. \textunderscore monarchicus\textunderscore )}
\end{itemize}
Relativo a monarca ou á monarquia.
O mesmo que \textunderscore monarquista\textunderscore .
\section{Monarquismo}
\begin{itemize}
\item {Grp. gram.:m.}
\end{itemize}
\begin{itemize}
\item {Proveniência:(De \textunderscore monarca\textunderscore )}
\end{itemize}
Sistema político dos monarquistas.
\section{Monarquista}
\begin{itemize}
\item {Grp. gram.:m.}
\end{itemize}
\begin{itemize}
\item {Proveniência:(De \textunderscore monarca\textunderscore )}
\end{itemize}
Sectário da monarquia ou do sistema monárquico.
\section{Monastical}
\begin{itemize}
\item {Grp. gram.:adj.}
\end{itemize}
O mesmo que \textunderscore monástico\textunderscore . Cf. Herculano, \textunderscore Bobo\textunderscore , 42.
\section{Monasticamente}
\begin{itemize}
\item {Grp. gram.:adv.}
\end{itemize}
De modo monástico; á maneira de monge.
\section{Monástico}
\begin{itemize}
\item {Grp. gram.:adj.}
\end{itemize}
\begin{itemize}
\item {Proveniência:(Gr. \textunderscore monastikos\textunderscore )}
\end{itemize}
O mesmo que \textunderscore monachal\textunderscore .
\section{Monaulo}
\begin{itemize}
\item {Grp. gram.:m.}
\end{itemize}
\begin{itemize}
\item {Proveniência:(Gr. \textunderscore monaulos\textunderscore )}
\end{itemize}
Espécie de frauta, usada entre os antigos.
\section{Monazilho}
\begin{itemize}
\item {Grp. gram.:m.}
\end{itemize}
\begin{itemize}
\item {Utilização:Ant.}
\end{itemize}
Menino de côro; sacristão. Cf. \textunderscore Anat. Joc.\textunderscore , I, 41.
(Cast. \textunderscore monacillo\textunderscore )
\section{Monazite}
\begin{itemize}
\item {Grp. gram.:f.}
\end{itemize}
\begin{itemize}
\item {Utilização:Miner.}
\end{itemize}
Areia brasileira, que se emprega no fabrico das mangas de incandescência e que provém da decomposição de um mineral, que se encontra nas areias amarelas.
\section{Monazítico}
\begin{itemize}
\item {Grp. gram.:adj.}
\end{itemize}
Relativo a monazite; em que há monazite.
\section{Moncalho}
\begin{itemize}
\item {Grp. gram.:m.}
\end{itemize}
\begin{itemize}
\item {Utilização:Prov.}
\end{itemize}
\begin{itemize}
\item {Utilização:minh.}
\end{itemize}
\begin{itemize}
\item {Proveniência:(De \textunderscore monco\textunderscore ?)}
\end{itemize}
Farrapo.
Porção de trapos sujos.
Mulher suja e mal vestida.
\section{Monção}
\begin{itemize}
\item {Grp. gram.:f.}
\end{itemize}
\begin{itemize}
\item {Utilização:Fig.}
\end{itemize}
\begin{itemize}
\item {Proveniência:(Do ár. \textunderscore mausin\textunderscore )}
\end{itemize}
Época ou vento favorável á navegação.
Bôa occasião, ensejo.
\section{Moncar}
\begin{itemize}
\item {Grp. gram.:v. i.}
\end{itemize}
Limpar o monco, assoar-se.
\section{Monco}
\begin{itemize}
\item {Grp. gram.:m.}
\end{itemize}
\begin{itemize}
\item {Proveniência:(Lat. \textunderscore mucus\textunderscore )}
\end{itemize}
Humor espêsso, segregado pela mucosa do nariz; ranho.
\textunderscore Monco de peru\textunderscore , excrescência carnosa, que se estende ou pende sôbre o bico do peru.
\section{Moncóne}
\begin{itemize}
\item {Grp. gram.:m.}
\end{itemize}
Árvore de Angola.
\section{Moncoso}
\begin{itemize}
\item {Grp. gram.:adj.}
\end{itemize}
\begin{itemize}
\item {Utilização:Fig.}
\end{itemize}
\begin{itemize}
\item {Grp. gram.:M.}
\end{itemize}
\begin{itemize}
\item {Utilização:Gír.}
\end{itemize}
Que tem monco; ranhoso.
Desprezível; immundo.
Lenço de assoar.
\section{Monda}
\begin{itemize}
\item {Grp. gram.:f.}
\end{itemize}
Acto de mondar.
Tempo próprio para mondar.
Erva nociva ás sementeiras.
Pequeno pão, que se dava de esmola, á porta dos conventos, e que era feito de toda a farinha.
\section{Mondadeira}
\begin{itemize}
\item {Grp. gram.:f.}
\end{itemize}
\begin{itemize}
\item {Proveniência:(De \textunderscore mondar\textunderscore )}
\end{itemize}
Mulher, que monda, que trabalha nas mondas.
\section{Mondadeiro}
\begin{itemize}
\item {Grp. gram.:m.}
\end{itemize}
\begin{itemize}
\item {Proveniência:(De \textunderscore mondar\textunderscore )}
\end{itemize}
Aquelle que trabalha na monda.
\section{Mondadentes}
\begin{itemize}
\item {Grp. gram.:m.}
\end{itemize}
\begin{itemize}
\item {Utilização:Ant.}
\end{itemize}
Palito de limpar dentes.
(Cast. \textunderscore mondadientes\textunderscore )
\section{Mondador}
\begin{itemize}
\item {Grp. gram.:adj.}
\end{itemize}
\begin{itemize}
\item {Grp. gram.:M.}
\end{itemize}
\begin{itemize}
\item {Proveniência:(Do lat. \textunderscore mundator\textunderscore )}
\end{itemize}
Que monda.
O mesmo que \textunderscore mondadeiro\textunderscore .
Utensílio, empregado na monda.
\section{Mondadura}
\begin{itemize}
\item {Grp. gram.:f.}
\end{itemize}
\begin{itemize}
\item {Proveniência:(De \textunderscore mondar\textunderscore )}
\end{itemize}
Monda; erva mondada.
\section{Mondágide}
\begin{itemize}
\item {Grp. gram.:f.}
\end{itemize}
\begin{itemize}
\item {Proveniência:(T. inventado, á imitação de \textunderscore tágide\textunderscore )}
\end{itemize}
Nympha de Mondego:«\textunderscore ...as formosas mondágides me escutam.\textunderscore »Castilho, \textunderscore Primavera\textunderscore , 243.
\section{Mondar}
\begin{itemize}
\item {Grp. gram.:v. t.}
\end{itemize}
\begin{itemize}
\item {Utilização:Fig.}
\end{itemize}
\begin{itemize}
\item {Proveniência:(Do lat. \textunderscore mundare\textunderscore )}
\end{itemize}
Arrancar nos campos (a erva que prejudica os cereaes).
Desramar.
Cortar (ramos secos ou supérfluos).
Limpar de ervas damninhas.
Limpar ou expurgar do que é nocivo ou supérfluo.
Emendar, corrigir.
\section{Mondé}
\begin{itemize}
\item {Grp. gram.:m.}
\end{itemize}
O mesmo que \textunderscore mundé\textunderscore .
\section{Mondéu}
\begin{itemize}
\item {Grp. gram.:m.}
\end{itemize}
O mesmo que \textunderscore mundé\textunderscore .
\section{Mondéu}
\begin{itemize}
\item {Grp. gram.:m.}
\end{itemize}
\begin{itemize}
\item {Utilização:Pesc.}
\end{itemize}
Cêrco fixo de rêdes, que remata estacadas, na pesca fluvial, e usado ao norte do país.
(Cp. \textunderscore mundéu\textunderscore )
\section{Mondilho}
\begin{itemize}
\item {Grp. gram.:m.}
\end{itemize}
\begin{itemize}
\item {Utilização:T. de Sanfins}
\end{itemize}
O mesmo que \textunderscore caruma\textunderscore  ou \textunderscore gravalha\textunderscore .
\section{Mondina}
\begin{itemize}
\item {Grp. gram.:f.}
\end{itemize}
\begin{itemize}
\item {Utilização:Miner.}
\end{itemize}
Substância pétrea, que se encontra nas minas de estanho.
\section{Mondina}
\begin{itemize}
\item {Grp. gram.:f.}
\end{itemize}
\begin{itemize}
\item {Utilização:Prov.}
\end{itemize}
\begin{itemize}
\item {Proveniência:(De \textunderscore mondino\textunderscore )}
\end{itemize}
Mulher, que trabalha na monda de cereaes.
\section{Mondino}
\begin{itemize}
\item {Grp. gram.:m.}
\end{itemize}
\begin{itemize}
\item {Utilização:Prov.}
\end{itemize}
\begin{itemize}
\item {Proveniência:(De \textunderscore monda\textunderscore )}
\end{itemize}
Homem que trabalha na monda de cereaes.
\section{Mondo}
\begin{itemize}
\item {Grp. gram.:adj.}
\end{itemize}
\begin{itemize}
\item {Utilização:Ant.}
\end{itemize}
\begin{itemize}
\item {Proveniência:(Do lat. \textunderscore mundus\textunderscore )}
\end{itemize}
O mesmo que \textunderscore limpo\textunderscore . Cf. G. Vicente, \textunderscore Auto do Cler. da Beira\textunderscore .
\section{Mondolim}
\begin{itemize}
\item {Grp. gram.:m.}
\end{itemize}
Doença, que ataca a raíz das palmeiras, impedindo que os cocos se desenvolvam e amadureçam.
(Do conc.)
\section{Mondombes}
\begin{itemize}
\item {Grp. gram.:m. pl.}
\end{itemize}
Tríbo nómade de Angola.
\section{Mondonga}
\begin{itemize}
\item {Grp. gram.:f.}
\end{itemize}
Mulher suja e desmazelada.
(Cp. \textunderscore mondongo\textunderscore )
\section{Mondongo}
\begin{itemize}
\item {Grp. gram.:m.}
\end{itemize}
\begin{itemize}
\item {Utilização:Bras. do Pará}
\end{itemize}
\begin{itemize}
\item {Proveniência:(Do cast. \textunderscore almondengo\textunderscore . Cp. \textunderscore almôndega\textunderscore )}
\end{itemize}
Intestinos miúdos de alguns animaes.
Pessôa suja e desmazelada.
Terreno baixo, cheio de atoleiros e coberto geralmente de plantas palustres.
\section{Mondongueira}
\begin{itemize}
\item {Grp. gram.:f.}
\end{itemize}
O mesmo que \textunderscore mondonga\textunderscore .
Criada de servir.
\section{Mondongueiro}
\begin{itemize}
\item {Grp. gram.:m.}
\end{itemize}
\begin{itemize}
\item {Utilização:Fig.}
\end{itemize}
\begin{itemize}
\item {Utilização:Prov.}
\end{itemize}
\begin{itemize}
\item {Utilização:trasm.}
\end{itemize}
\begin{itemize}
\item {Proveniência:(De \textunderscore mondongo\textunderscore )}
\end{itemize}
Vendedor de fígado ou de outros intestinos de reses; fressureiro.
Aquelle que se occupa em mesteres sórdidos.
Indivíduo namorador e mollangueiro ou preguiçoso. Cf. Deusdado, \textunderscore Escorços Trasm.\textunderscore , 335.
\section{Monduaí}
\begin{itemize}
\item {Grp. gram.:m.}
\end{itemize}
Arvore silvestre.
\section{Monduí}
\begin{itemize}
\item {Grp. gram.:m.}
\end{itemize}
\begin{itemize}
\item {Utilização:Bras}
\end{itemize}
Arvore silvestre.
\section{Mondururu}
\begin{itemize}
\item {Grp. gram.:m.}
\end{itemize}
\begin{itemize}
\item {Utilização:Bras}
\end{itemize}
Gênero de árvores silvestres.
\section{Monécia}
\begin{itemize}
\item {Grp. gram.:f.}
\end{itemize}
\begin{itemize}
\item {Utilização:Bot.}
\end{itemize}
\begin{itemize}
\item {Proveniência:(Do gr. \textunderscore monos\textunderscore  + \textunderscore oikia\textunderscore )}
\end{itemize}
Classe das plantas que, na mesma haste, têm separadas as flôres masculinas e as femininas.
\section{Monécico}
\begin{itemize}
\item {Grp. gram.:adj.}
\end{itemize}
Relativo á monécia.
\section{Monelha}
\begin{itemize}
\item {fónica:nê}
\end{itemize}
\begin{itemize}
\item {Grp. gram.:f.}
\end{itemize}
\begin{itemize}
\item {Utilização:Náut.}
\end{itemize}
\begin{itemize}
\item {Utilização:Prov.}
\end{itemize}
\begin{itemize}
\item {Utilização:minh.}
\end{itemize}
\begin{itemize}
\item {Proveniência:(Do lat. \textunderscore monilia\textunderscore )}
\end{itemize}
Corda, com que se reforçam os mastros, cingindo-os.
O mesmo que \textunderscore molhelha\textunderscore ^1.
\section{Monêmero}
\begin{itemize}
\item {Grp. gram.:m.}
\end{itemize}
\begin{itemize}
\item {Proveniência:(Do gr. \textunderscore monos\textunderscore  + \textunderscore hemera\textunderscore )}
\end{itemize}
Espécie de collýrio antigo, a que se attribuía a virtude de curar num dia quaesquer doenças de olhos.
\section{Monentelo}
\begin{itemize}
\item {Grp. gram.:m.}
\end{itemize}
Gênero de plantas, da fam. das compostas.
\section{Monera}
\begin{itemize}
\item {Grp. gram.:f.}
\end{itemize}
\begin{itemize}
\item {Proveniência:(Gr. \textunderscore moneres\textunderscore )}
\end{itemize}
Organismo rudimentar, que representa a transição do reino vegetal para o animal.
\section{Monere}
\begin{itemize}
\item {Grp. gram.:f.}
\end{itemize}
O mesmo que \textunderscore monera\textunderscore .
\section{Monerom}
\begin{itemize}
\item {Grp. gram.:m.}
\end{itemize}
\begin{itemize}
\item {Proveniência:(De \textunderscore Moneron\textunderscore , n. p.)}
\end{itemize}
Moéda francesa de cobre, fabricada no tempo da Revolução, quando escasseava a prata.
\section{Monésia}
\begin{itemize}
\item {Grp. gram.:f.}
\end{itemize}
Casca medicinal de buranhém.
O mesmo que \textunderscore buranhém\textunderscore .
\section{Monesteirol}
\begin{itemize}
\item {Grp. gram.:m.}
\end{itemize}
\begin{itemize}
\item {Utilização:Ant.}
\end{itemize}
Pequeno mosteiro.
(Cp. cast. \textunderscore monestério\textunderscore )
\section{Moneta}
\begin{itemize}
\item {fónica:nê}
\end{itemize}
\begin{itemize}
\item {Grp. gram.:f.}
\end{itemize}
\begin{itemize}
\item {Utilização:Náut.}
\end{itemize}
\begin{itemize}
\item {Proveniência:(Do b. lat. \textunderscore boneta\textunderscore )}
\end{itemize}
Pequena vela ou tira de pano, que se põe por baixo dos papafigos, para aproveitar o bom tempo.
\section{Monetário}
\begin{itemize}
\item {Grp. gram.:adj.}
\end{itemize}
\begin{itemize}
\item {Grp. gram.:M.}
\end{itemize}
\begin{itemize}
\item {Proveniência:(Lat. \textunderscore monetarius\textunderscore )}
\end{itemize}
Relativo á moéda.
Collecção de moédas.
Livro, que tem gravuras de moédas.
Numismata.
\section{Monete}
\begin{itemize}
\item {fónica:nê}
\end{itemize}
\begin{itemize}
\item {Grp. gram.:m.}
\end{itemize}
Gadelha, farripa.
Caracol de cabello, que faz parte do penteado das senhoras.
\section{Monetizar}
\begin{itemize}
\item {Proveniência:(Do lat. \textunderscore moneta\textunderscore )}
\end{itemize}
\textunderscore v. t.\textunderscore  (e der.)
O mesmo que \textunderscore amoedar\textunderscore .
\section{Monezilho}
\begin{itemize}
\item {Grp. gram.:m.}
\end{itemize}
\begin{itemize}
\item {Utilização:Ant.}
\end{itemize}
Menino do côro, o mesmo que \textunderscore monazilho\textunderscore .
\section{Monferir}
\begin{itemize}
\item {Grp. gram.:v. t.}
\end{itemize}
\begin{itemize}
\item {Utilização:Ant.}
\end{itemize}
\begin{itemize}
\item {Proveniência:(De \textunderscore mão\textunderscore  + \textunderscore ferir\textunderscore )}
\end{itemize}
Pôr marca de ferro nas mãos ou pernas de (gado).
\section{Monfis}
\begin{itemize}
\item {Grp. gram.:m. pl.}
\end{itemize}
Nome, que se deu na Espanha aos Moiros que viviam da pilhagem e do roubo.
\section{Monge}
\begin{itemize}
\item {Grp. gram.:m.}
\end{itemize}
\begin{itemize}
\item {Utilização:Pop.}
\end{itemize}
\begin{itemize}
\item {Proveniência:(Do lat. \textunderscore monachus\textunderscore , por intermédio de uma fórma francesa. Cf. G. Viana, \textunderscore Apostilas\textunderscore , vb. \textunderscore mogo\textunderscore )}
\end{itemize}
Religioso ou frade de mosteiro.
Anachoreta.
Misantropo, homem pouco sociável.
\section{Monge-do-mar}
\begin{itemize}
\item {Grp. gram.:m.}
\end{itemize}
\begin{itemize}
\item {Utilização:Zool.}
\end{itemize}
Amphíbio, da família das phocas.
\section{Monger}
\begin{itemize}
\item {Grp. gram.:v. t.}
\end{itemize}
\begin{itemize}
\item {Utilização:Prov.}
\end{itemize}
\begin{itemize}
\item {Utilização:trasm.}
\end{itemize}
O mesmo que \textunderscore mungir\textunderscore .
\section{Mongi}
\begin{itemize}
\item {Grp. gram.:m.}
\end{itemize}
O mesmo que \textunderscore mongil\textunderscore ^1.
\section{Mongia}
\begin{itemize}
\item {Grp. gram.:f.}
\end{itemize}
\begin{itemize}
\item {Utilização:Ant.}
\end{itemize}
Convento; casa de monges. Cf. \textunderscore Port. Mon. Hist.\textunderscore , \textunderscore Script.\textunderscore , 242.
\section{Mongil}
\begin{itemize}
\item {Grp. gram.:m.}
\end{itemize}
\begin{itemize}
\item {Utilização:Ant.}
\end{itemize}
\begin{itemize}
\item {Proveniência:(De \textunderscore monja\textunderscore )}
\end{itemize}
Hábito de monja.
Túnica talar para mulheres.
Vestimenta de luto, para mulher, não viúva.
\section{Mongil}
\begin{itemize}
\item {Grp. gram.:adj.}
\end{itemize}
Diz-se de uma variedade de trigo rijo.
\section{Mongiloto}
\begin{itemize}
\item {Grp. gram.:m.}
\end{itemize}
\begin{itemize}
\item {Utilização:Ant.}
\end{itemize}
Espécie de procurador de causas perdidas, na China. Cf. \textunderscore Peregrinação\textunderscore , XCIX.
\section{Mongírio}
\begin{itemize}
\item {Grp. gram.:adj.}
\end{itemize}
\begin{itemize}
\item {Utilização:Gír.}
\end{itemize}
Animoso, valente.
\section{Môngoa}
\begin{itemize}
\item {Grp. gram.:f.}
\end{itemize}
\begin{itemize}
\item {Utilização:T. de Angola}
\end{itemize}
O mesmo que \textunderscore sal\textunderscore .
\section{Mongoiós}
\begin{itemize}
\item {Grp. gram.:m. pl.}
\end{itemize}
\begin{itemize}
\item {Utilização:Bras}
\end{itemize}
Tribo de aborígenes da Baía.
\section{Mongol}
\begin{itemize}
\item {Grp. gram.:adj.}
\end{itemize}
\begin{itemize}
\item {Grp. gram.:M.}
\end{itemize}
O mesmo que \textunderscore mongólico\textunderscore .
Habitante da Mongólia.
Antiga língua dos Mongóes.
\section{Mongólico}
\begin{itemize}
\item {Grp. gram.:adj.}
\end{itemize}
Relativo á Mongólia ou aos Mongóes.
\section{Mongolista}
\begin{itemize}
\item {Grp. gram.:m.}
\end{itemize}
Philólogo, que estuda especialmente o mongol.
\section{Mongolo}
\begin{itemize}
\item {Grp. gram.:m.}
\end{itemize}
Árvore de Angola.
\section{Mongolóide}
\begin{itemize}
\item {Grp. gram.:adj.}
\end{itemize}
\begin{itemize}
\item {Proveniência:(De \textunderscore mongol\textunderscore  + gr. \textunderscore eidos\textunderscore )}
\end{itemize}
Próprio da raça mongol.
Semelhante ao typo da raça mongol.
\section{Mongoose}
\begin{itemize}
\item {Grp. gram.:m.}
\end{itemize}
Espécie de raposa de Moçambique.
\section{Mongu}
\begin{itemize}
\item {Grp. gram.:m.}
\end{itemize}
\begin{itemize}
\item {Utilização:Zool.}
\end{itemize}
Sub-gênero de mammíferos quadrúmanos, do gênero maque.
\section{Monguba}
\begin{itemize}
\item {Grp. gram.:f.}
\end{itemize}
Árvore das regiões do Amazonas.--Lopes Mendes, nas \textunderscore Cartas da América\textunderscore , fala da \textunderscore monguba\textunderscore  e da \textunderscore mongubá\textunderscore  ou \textunderscore mongabá\textunderscore ; provavelmente, com aquelles nomes, quis referir-se a uma só árvore. O \textunderscore Diccion. Topogr.\textunderscore  de Lourenço Amazonas diz \textunderscore monguba\textunderscore .
\section{Monha}
\begin{itemize}
\item {Grp. gram.:f.}
\end{itemize}
Laço de fitas, com que se enfeita o pescoço dos toiros, nas corridas.
Roseta, usada por toireiros, na parte posterior da cabeça.
Manequim de cabelleireiro ou de modista.
(Cast. \textunderscore moña\textunderscore )
\section{Monhé}
\begin{itemize}
\item {Grp. gram.:m.}
\end{itemize}
\begin{itemize}
\item {Utilização:T. da Áfr. Or. Port}
\end{itemize}
Mestiço de árabe e negro.
\section{Monho}
\begin{itemize}
\item {Grp. gram.:m.}
\end{itemize}
Pequeno chinó de senhoras.
Laço de fita, com que as senhoras atam ou adornam o cabello.
(Cast. \textunderscore moño\textunderscore )
\section{Móni}
\begin{itemize}
\item {Grp. gram.:m.}
\end{itemize}
\begin{itemize}
\item {Utilização:Gír.}
\end{itemize}
\begin{itemize}
\item {Proveniência:(Do ingl. \textunderscore money\textunderscore )}
\end{itemize}
Dinheiro.
\section{Monje}
\begin{itemize}
\item {Grp. gram.:m.}
\end{itemize}
\begin{itemize}
\item {Utilização:Pop.}
\end{itemize}
\begin{itemize}
\item {Proveniência:(Do lat. \textunderscore monachus\textunderscore , por intermédio de uma fórma francesa. Cf. G. Viana, \textunderscore Apostilas\textunderscore , vb. \textunderscore mogo\textunderscore )}
\end{itemize}
Religioso ou frade de mosteiro.
Anachoreta.
Misantropo, homem pouco sociável.
\section{Monjil}
\begin{itemize}
\item {Grp. gram.:m.}
\end{itemize}
\begin{itemize}
\item {Utilização:Ant.}
\end{itemize}
\begin{itemize}
\item {Proveniência:(De \textunderscore monja\textunderscore )}
\end{itemize}
Hábito de monja.
Túnica talar para mulheres.
Vestimenta de luto, para mulher, não viúva.
\section{Mónica}
\begin{itemize}
\item {Grp. gram.:f.}
\end{itemize}
\begin{itemize}
\item {Utilização:Açor}
\end{itemize}
\begin{itemize}
\item {Utilização:Bras}
\end{itemize}
O mesmo que \textunderscore nêspera\textunderscore .
Espécie de mandioca.
\section{Mónicas}
\begin{itemize}
\item {Grp. gram.:f. pl.}
\end{itemize}
Freiras de Santa-Mónica ou de Santo-Agostinho, como as que viveram no convento de Odivelas. Cf. Camillo, \textunderscore Caveira\textunderscore , 454 e 455.
\section{Moniço}
\begin{itemize}
\item {Grp. gram.:m.}
\end{itemize}
\begin{itemize}
\item {Utilização:Prov.}
\end{itemize}
\begin{itemize}
\item {Utilização:dur.}
\end{itemize}
Caruma sêca.
\section{Monilha}
\begin{itemize}
\item {Grp. gram.:f.}
\end{itemize}
\begin{itemize}
\item {Utilização:Prov.}
\end{itemize}
\begin{itemize}
\item {Utilização:alent.}
\end{itemize}
\begin{itemize}
\item {Proveniência:(Do lat. \textunderscore monilia\textunderscore )}
\end{itemize}
O mesmo que \textunderscore molhelha\textunderscore ^1.
\section{Moniliforme}
\begin{itemize}
\item {Grp. gram.:adj.}
\end{itemize}
\begin{itemize}
\item {Proveniência:(Do lat. \textunderscore monile\textunderscore  + \textunderscore forma\textunderscore )}
\end{itemize}
Que tem fórma de rosário ou collar.
\section{Monímia}
\begin{itemize}
\item {Grp. gram.:f.}
\end{itemize}
Gênero de plantas de Madagáscar.
\section{Monimiáceas}
\begin{itemize}
\item {Grp. gram.:f. pl.}
\end{itemize}
\begin{itemize}
\item {Proveniência:(De \textunderscore monimiáceo\textunderscore )}
\end{itemize}
Família de plantas, que tem por typo a monímia.
\section{Monimiáceo}
\begin{itemize}
\item {Grp. gram.:adj.}
\end{itemize}
Relativo ou semelhante á monímia.
\section{Monir}
\begin{itemize}
\item {Grp. gram.:v. t.}
\end{itemize}
\begin{itemize}
\item {Utilização:Ant.}
\end{itemize}
\begin{itemize}
\item {Proveniência:(Do lat. \textunderscore monere\textunderscore )}
\end{itemize}
Avisar, para vir depor sôbre a matéria de uma monitória.
\section{Monismo}
\begin{itemize}
\item {Grp. gram.:m.}
\end{itemize}
\begin{itemize}
\item {Proveniência:(Do gr. \textunderscore monos\textunderscore )}
\end{itemize}
Concepção dynâmica da unidade de todas as fôrças phýsicas, isto é, das fôrças da natureza, reduzidas ao phenómeno do movimento,--movimento atómico, molecular, sideral, cellular, physiológico, anímico, social. Cf. Th. Braga, \textunderscore Mod. Ideias\textunderscore , 232; Latino, \textunderscore Elogios\textunderscore , 168.
\section{Monista}
\begin{itemize}
\item {Grp. gram.:adj.}
\end{itemize}
\begin{itemize}
\item {Grp. gram.:M.}
\end{itemize}
O mesmo que \textunderscore monístico\textunderscore .
Partidário do monismo.
\section{Monístico}
\begin{itemize}
\item {Grp. gram.:adj.}
\end{itemize}
Relativo ao monismo.
Relativo á concepção dynâmica das forças phýsicas.
(Cp. \textunderscore monismo\textunderscore )
\section{Mónita}
\begin{itemize}
\item {Grp. gram.:f.}
\end{itemize}
\begin{itemize}
\item {Proveniência:(Lat. \textunderscore monita\textunderscore )}
\end{itemize}
Aviso, advertência.
\section{Monitor}
\begin{itemize}
\item {Grp. gram.:m.}
\end{itemize}
\begin{itemize}
\item {Proveniência:(Lat. \textunderscore monitor\textunderscore )}
\end{itemize}
Aquelle que admoésta.
Empregado ou estudante, que numa escola toma conta de uma classe de alumnos, para os dirigir no estudo.
Prefeito; decurião.
Gênero de reptís sáurios.
Espécie de navio de guerra.
\section{Monitora}
\begin{itemize}
\item {Grp. gram.:f.}
\end{itemize}
Mulher, que numa escola é encarregada de dirigir uma classe de alumnos.
(Cp. \textunderscore monitor\textunderscore )
\section{Monitória}
\begin{itemize}
\item {Grp. gram.:f.}
\end{itemize}
\begin{itemize}
\item {Utilização:Fam.}
\end{itemize}
\begin{itemize}
\item {Proveniência:(Lat. \textunderscore monitoria\textunderscore )}
\end{itemize}
Aviso, em que se convida o público a ir declarar o que souber, á cêrca de um crime.
Conselho.
Repreensão.
\section{Monitorial}
\begin{itemize}
\item {Grp. gram.:adj.}
\end{itemize}
Relativo á monitória.
\section{Monja}
\begin{itemize}
\item {Grp. gram.:f.}
\end{itemize}
\begin{itemize}
\item {Proveniência:(De \textunderscore monge\textunderscore )}
\end{itemize}
Freira de mosteiro.
\section{Monjal}
\begin{itemize}
\item {Grp. gram.:adj.}
\end{itemize}
Relativo a monge ou a monja.
Feito por mão de monja:«\textunderscore as morcellas monjaes...\textunderscore »Filinto, IX, 159.
\section{Monjolo}
\begin{itemize}
\item {Grp. gram.:m.}
\end{itemize}
\begin{itemize}
\item {Utilização:Bras}
\end{itemize}
Nome de uma planta.
Novilho, bezerro.
\section{Monleiro}
\begin{itemize}
\item {Grp. gram.:m.}
\end{itemize}
\begin{itemize}
\item {Utilização:Ant.}
\end{itemize}
O mesmo que \textunderscore moleiro\textunderscore .
\section{Mono}
\begin{itemize}
\item {Grp. gram.:m.}
\end{itemize}
\begin{itemize}
\item {Utilização:Fig.}
\end{itemize}
\begin{itemize}
\item {Utilização:Chul.}
\end{itemize}
\begin{itemize}
\item {Grp. gram.:Adj.}
\end{itemize}
\begin{itemize}
\item {Utilização:Prov.}
\end{itemize}
\begin{itemize}
\item {Utilização:trasm.}
\end{itemize}
Macaco, bugio.
Indivíduo feio, estúpido.
Macambúzio.
Burla; fraude.
Mercadoria, que se conserva muito tempo num estabelecimento, sem que se venda ou a procurem.
Boneco de trapos.
Relativo a macaco:«\textunderscore ...flôr das raças monas.\textunderscore »Castilho.
Semsaborão, macambúzio.
Que não tem chifres: \textunderscore cabra mona\textunderscore .
(Cast. \textunderscore moño\textunderscore )
\section{Mono...}
\begin{itemize}
\item {Grp. gram.:pref.}
\end{itemize}
\begin{itemize}
\item {Proveniência:(Do gr. \textunderscore monos\textunderscore )}
\end{itemize}
(designativo de \textunderscore unidade\textunderscore )
\section{Monoatómico}
\begin{itemize}
\item {Grp. gram.:adj.}
\end{itemize}
\begin{itemize}
\item {Utilização:Chím.}
\end{itemize}
\begin{itemize}
\item {Proveniência:(De \textunderscore mono\textunderscore  + \textunderscore atómico\textunderscore )}
\end{itemize}
Formado pela combinação de um equivalente de oxygênio, e de um equivalente de outro corpo simples.
\section{Monoaxífero}
\begin{itemize}
\item {fónica:csi}
\end{itemize}
\begin{itemize}
\item {Grp. gram.:adj.}
\end{itemize}
\begin{itemize}
\item {Utilização:Bot.}
\end{itemize}
\begin{itemize}
\item {Proveniência:(Do gr. \textunderscore monos\textunderscore  + lat. \textunderscore axis\textunderscore  + \textunderscore ferre\textunderscore )}
\end{itemize}
Diz-se da inflorescência, quando apresenta um só disco.
\section{Monobafia}
\begin{itemize}
\item {Grp. gram.:f.}
\end{itemize}
\begin{itemize}
\item {Proveniência:(Do gr. \textunderscore monos\textunderscore  + \textunderscore baphein\textunderscore )}
\end{itemize}
Estado daquilo que apresenta uma só côr.
\section{Monobaphia}
\begin{itemize}
\item {Grp. gram.:f.}
\end{itemize}
\begin{itemize}
\item {Proveniência:(Do gr. \textunderscore monos\textunderscore  + \textunderscore baphein\textunderscore )}
\end{itemize}
Estado daquillo que apresenta uma só côr.
\section{Monobásico}
\begin{itemize}
\item {Grp. gram.:adj.}
\end{itemize}
\begin{itemize}
\item {Utilização:Chím.}
\end{itemize}
\begin{itemize}
\item {Proveniência:(Do gr. \textunderscore monos\textunderscore  + \textunderscore basis\textunderscore )}
\end{itemize}
Diz-se dos saes, que só contêm um equivalente de base.
Diz-se das fórmulas pharmacêuticas, que têm só uma base.
\section{Monoblepsia}
\begin{itemize}
\item {Grp. gram.:f.}
\end{itemize}
\begin{itemize}
\item {Proveniência:(Do gr. \textunderscore monos\textunderscore  + \textunderscore blepsis\textunderscore )}
\end{itemize}
Doença, em que a visão só é perfeita, fechando-se um dos olhos.
\section{Monocário}
\begin{itemize}
\item {Grp. gram.:m.}
\end{itemize}
Gênero de plantas melastomáceas.
\section{Monocarpelar}
\begin{itemize}
\item {Grp. gram.:adj.}
\end{itemize}
\begin{itemize}
\item {Utilização:Bot.}
\end{itemize}
\begin{itemize}
\item {Proveniência:(De \textunderscore mono...\textunderscore  + \textunderscore carpela\textunderscore )}
\end{itemize}
Que tem uma só carpela.
\section{Monocarpellar}
\begin{itemize}
\item {Grp. gram.:adj.}
\end{itemize}
\begin{itemize}
\item {Utilização:Bot.}
\end{itemize}
\begin{itemize}
\item {Proveniência:(De \textunderscore mono...\textunderscore  + \textunderscore carpella\textunderscore )}
\end{itemize}
Que tem uma só carpella.
\section{Monocarpiano}
\begin{itemize}
\item {Grp. gram.:adj.}
\end{itemize}
O mesmo que \textunderscore monocárpico\textunderscore .
\section{Monocárpico}
\begin{itemize}
\item {Grp. gram.:adj.}
\end{itemize}
\begin{itemize}
\item {Utilização:Bot.}
\end{itemize}
\begin{itemize}
\item {Proveniência:(De \textunderscore monocarpo\textunderscore )}
\end{itemize}
Que dá flôr e fruto, só por uma vez, como o trigo, a cenoira, etc.
\section{Monocarpo}
\begin{itemize}
\item {Grp. gram.:adj.}
\end{itemize}
\begin{itemize}
\item {Proveniência:(Do gr. \textunderscore monos\textunderscore  + \textunderscore karpos\textunderscore )}
\end{itemize}
Que tem só um fruto.
\section{Monocefalia}
\begin{itemize}
\item {Grp. gram.:f.}
\end{itemize}
Monstruosidade de monocéfalo.
\section{Monocéfalo}
\begin{itemize}
\item {Grp. gram.:adj.}
\end{itemize}
\begin{itemize}
\item {Proveniência:(Do gr. \textunderscore monos\textunderscore  + \textunderscore kephale\textunderscore )}
\end{itemize}
Que tem só uma cabeça, (falando-se de monstros, que nascem com os corpos intimamente ligados).
\section{Monocellular}
\begin{itemize}
\item {Grp. gram.:adj.}
\end{itemize}
\begin{itemize}
\item {Utilização:Hist. Nat.}
\end{itemize}
\begin{itemize}
\item {Proveniência:(De \textunderscore mono...\textunderscore  + \textunderscore céllula\textunderscore )}
\end{itemize}
Diz-se do organismo rudimentar com uma só céllula.
\section{Monocelular}
\begin{itemize}
\item {Grp. gram.:adj.}
\end{itemize}
\begin{itemize}
\item {Utilização:Hist. Nat.}
\end{itemize}
\begin{itemize}
\item {Proveniência:(De \textunderscore mono...\textunderscore  + \textunderscore célula\textunderscore )}
\end{itemize}
Diz-se do organismo rudimentar com uma só célula.
\section{Monocephalia}
\begin{itemize}
\item {Grp. gram.:f.}
\end{itemize}
Monstruosidade de monocéphalo.
\section{Monocéphalo}
\begin{itemize}
\item {Grp. gram.:adj.}
\end{itemize}
\begin{itemize}
\item {Proveniência:(Do gr. \textunderscore monos\textunderscore  + \textunderscore kephale\textunderscore )}
\end{itemize}
Que tem só uma cabeça, (falando-se de monstros, que nascem com os corpos intimamente ligados).
\section{Monócero}
\begin{itemize}
\item {Grp. gram.:adj.}
\end{itemize}
\begin{itemize}
\item {Utilização:Zool.}
\end{itemize}
\begin{itemize}
\item {Utilização:Bot.}
\end{itemize}
\begin{itemize}
\item {Proveniência:(Do gr. \textunderscore monos\textunderscore  + \textunderscore keras\textunderscore )}
\end{itemize}
Que tem um corno só.
Que tem um só prolongamento, em fórma de corno.
\section{Monoceronte}
\begin{itemize}
\item {Grp. gram.:m.}
\end{itemize}
\begin{itemize}
\item {Proveniência:(Do gr. \textunderscore monos\textunderscore  + \textunderscore keras\textunderscore )}
\end{itemize}
O mesmo que \textunderscore unicórneo\textunderscore .
\section{Monochiro}
\begin{itemize}
\item {fónica:qui}
\end{itemize}
\begin{itemize}
\item {Grp. gram.:m.}
\end{itemize}
\begin{itemize}
\item {Proveniência:(Do gr. \textunderscore monos\textunderscore  + \textunderscore kheir\textunderscore )}
\end{itemize}
Gênero de peixes chatos.
\section{Monochlamýdeas}
\begin{itemize}
\item {Grp. gram.:f. pl.}
\end{itemize}
\begin{itemize}
\item {Proveniência:(De \textunderscore monochlamýdeo\textunderscore )}
\end{itemize}
Uma das classes das plantas dicotyledóneas.
\section{Monochlamýdeo}
\begin{itemize}
\item {Grp. gram.:adj.}
\end{itemize}
\begin{itemize}
\item {Utilização:Bot.}
\end{itemize}
\begin{itemize}
\item {Proveniência:(Do gr. \textunderscore monos\textunderscore  + \textunderscore khlamus\textunderscore )}
\end{itemize}
Que tem um periantho, como a tulipa.
\section{Monociclista}
\begin{itemize}
\item {Grp. gram.:m.  e  f.}
\end{itemize}
Pessôa, que anda em monociclo.
\section{Monociclo}
\begin{itemize}
\item {Grp. gram.:m.}
\end{itemize}
\begin{itemize}
\item {Proveniência:(Do gr. \textunderscore monos\textunderscore  + \textunderscore kuclos\textunderscore )}
\end{itemize}
Velocípede de uma só roda, usado hoje apenas por acrobatas.
\section{Monoclamídeas}
\begin{itemize}
\item {Grp. gram.:f. pl.}
\end{itemize}
\begin{itemize}
\item {Proveniência:(De \textunderscore monoclamídeo\textunderscore )}
\end{itemize}
Uma das classes das plantas dicotiledóneas.
\section{Monoclamídeo}
\begin{itemize}
\item {Grp. gram.:adj.}
\end{itemize}
\begin{itemize}
\item {Utilização:Bot.}
\end{itemize}
\begin{itemize}
\item {Proveniência:(Do gr. \textunderscore monos\textunderscore  + \textunderscore khlamus\textunderscore )}
\end{itemize}
Que tem um perianto, como a tulipa.
\section{Monochromático}
\begin{itemize}
\item {Grp. gram.:adj.}
\end{itemize}
\begin{itemize}
\item {Proveniência:(De \textunderscore monochromo\textunderscore )}
\end{itemize}
Que é pintado com uma só côr.
\section{Monochromo}
\begin{itemize}
\item {Grp. gram.:adj.}
\end{itemize}
\begin{itemize}
\item {Proveniência:(Do gr. \textunderscore monos\textunderscore  + \textunderscore khroma\textunderscore )}
\end{itemize}
Que tem só uma côr.
\section{Monoclínico}
\begin{itemize}
\item {Grp. gram.:adj.}
\end{itemize}
\begin{itemize}
\item {Utilização:Geol.}
\end{itemize}
\begin{itemize}
\item {Proveniência:(Do gr. \textunderscore monos\textunderscore  + \textunderscore kline\textunderscore )}
\end{itemize}
Diz-se do systema crystallográphico, caracterizado por três eixos desiguaes e oblíquos.
\section{Monóclino}
\begin{itemize}
\item {Grp. gram.:adj.}
\end{itemize}
\begin{itemize}
\item {Proveniência:(Do gr. \textunderscore monos\textunderscore  + \textunderscore kline\textunderscore )}
\end{itemize}
Diz-se dos vegetaes, que reunem os dois sexos na mesma flôr.
\section{Monococo}
\begin{itemize}
\item {fónica:cô}
\end{itemize}
\begin{itemize}
\item {Grp. gram.:adj.}
\end{itemize}
Diz-se de uma variedade de trigo.
\section{Monocórdio}
\begin{itemize}
\item {Grp. gram.:m.}
\end{itemize}
\begin{itemize}
\item {Proveniência:(Gr. \textunderscore monokhordon\textunderscore )}
\end{itemize}
Instrumento musical, usado pelos Gregos, e que tinha uma só corda.
Instrumento de uma só corda, para conhecer os differentes intervallos dos sons.
\section{Monocotilar}
\begin{itemize}
\item {Grp. gram.:adj.}
\end{itemize}
\begin{itemize}
\item {Utilização:Zool.}
\end{itemize}
\begin{itemize}
\item {Proveniência:(Do gr. \textunderscore monos\textunderscore  + \textunderscore kutule\textunderscore )}
\end{itemize}
Que tem uma só tromba ou sugadoiro.
\section{Monocotilários}
\begin{itemize}
\item {Grp. gram.:m. pl.}
\end{itemize}
Família de helmintos.
(Cp. \textunderscore monocotilar\textunderscore )
\section{Monocotiledóneas}
\begin{itemize}
\item {Grp. gram.:f. pl.}
\end{itemize}
\begin{itemize}
\item {Proveniência:(De \textunderscore monocotiledóneo\textunderscore )}
\end{itemize}
Uma das três grandes ramificações do reino vegetal, caracterizada por têr na semente um só cotilédone.
\section{Monocotiledóneo}
\begin{itemize}
\item {Grp. gram.:adj.}
\end{itemize}
\begin{itemize}
\item {Proveniência:(De \textunderscore mono...\textunderscore  + \textunderscore cotilédone\textunderscore )}
\end{itemize}
Que tem um só cotilédone.
\section{Monocótilo}
\begin{itemize}
\item {Grp. gram.:adj.}
\end{itemize}
\begin{itemize}
\item {Proveniência:(Do gr. \textunderscore monos\textunderscore  + \textunderscore kotule\textunderscore )}
\end{itemize}
O mesmo que \textunderscore monocotiledóneo\textunderscore .
\section{Monocotylar}
\begin{itemize}
\item {Grp. gram.:adj.}
\end{itemize}
\begin{itemize}
\item {Utilização:Zool.}
\end{itemize}
\begin{itemize}
\item {Proveniência:(Do gr. \textunderscore monos\textunderscore  + \textunderscore kutule\textunderscore )}
\end{itemize}
Que tem uma só tromba ou sugadoiro.
\section{Monocotylários}
\begin{itemize}
\item {Grp. gram.:m. pl.}
\end{itemize}
Família de helminthos.
(Cp. \textunderscore monocotylar\textunderscore )
\section{Monocotyledóneas}
\begin{itemize}
\item {Grp. gram.:f. pl.}
\end{itemize}
\begin{itemize}
\item {Proveniência:(De \textunderscore monocotyledóneo\textunderscore )}
\end{itemize}
Uma das três grandes ramificações do reino vegetal, caracterizada por têr na semente um só cotylédone.
\section{Monocotyledóneo}
\begin{itemize}
\item {Grp. gram.:adj.}
\end{itemize}
\begin{itemize}
\item {Proveniência:(De \textunderscore mono...\textunderscore  + \textunderscore cotylédone\textunderscore )}
\end{itemize}
Que tem um só cotylédone.
\section{Monocótylo}
\begin{itemize}
\item {Grp. gram.:adj.}
\end{itemize}
\begin{itemize}
\item {Proveniência:(Do gr. \textunderscore monos\textunderscore  + \textunderscore kotule\textunderscore )}
\end{itemize}
O mesmo que \textunderscore monocotyledóneo\textunderscore .
\section{Monocromático}
\begin{itemize}
\item {Grp. gram.:adj.}
\end{itemize}
\begin{itemize}
\item {Proveniência:(De \textunderscore monocromo\textunderscore )}
\end{itemize}
Que é pintado com uma só côr.
\section{Monocromo}
\begin{itemize}
\item {Grp. gram.:adj.}
\end{itemize}
\begin{itemize}
\item {Proveniência:(Do gr. \textunderscore monos\textunderscore  + \textunderscore khroma\textunderscore )}
\end{itemize}
Que tem só uma côr.
\section{Monoculizar}
\begin{itemize}
\item {Grp. gram.:v. t.}
\end{itemize}
\begin{itemize}
\item {Utilização:Neol.}
\end{itemize}
Observar com o monóculo. Cf. Ortigão, \textunderscore Holanda\textunderscore .
\section{Monóculo}
\begin{itemize}
\item {Grp. gram.:adj.}
\end{itemize}
\begin{itemize}
\item {Grp. gram.:M.}
\end{itemize}
\begin{itemize}
\item {Proveniência:(Do gr. \textunderscore monos\textunderscore  + lat. \textunderscore oculus\textunderscore )}
\end{itemize}
Que tem só um ôlho.
Luneta de um só vidro.
\section{Monocultura}
\begin{itemize}
\item {Grp. gram.:f.}
\end{itemize}
\begin{itemize}
\item {Utilização:Neol.}
\end{itemize}
\begin{itemize}
\item {Proveniência:(De \textunderscore mono...\textunderscore  + \textunderscore cultura\textunderscore )}
\end{itemize}
Cultura de uma só especialidade agrícola.
\section{Monocyclista}
\begin{itemize}
\item {Grp. gram.:m.  e  f.}
\end{itemize}
Pessôa, que anda em monocyclo.
\section{Monocyclo}
\begin{itemize}
\item {Grp. gram.:m.}
\end{itemize}
\begin{itemize}
\item {Proveniência:(Do gr. \textunderscore monos\textunderscore  + \textunderscore kuclos\textunderscore )}
\end{itemize}
Velocípede de uma só roda, usado hoje apenas por acrobatas.
\section{Monodáctilo}
\begin{itemize}
\item {Grp. gram.:adj.}
\end{itemize}
\begin{itemize}
\item {Proveniência:(Do gr. \textunderscore monos\textunderscore  + \textunderscore daktulos\textunderscore )}
\end{itemize}
Que tem só um dedo.
\section{Monodáctylo}
\begin{itemize}
\item {Grp. gram.:adj.}
\end{itemize}
\begin{itemize}
\item {Proveniência:(Do gr. \textunderscore monos\textunderscore  + \textunderscore daktulos\textunderscore )}
\end{itemize}
Que tem só um dedo.
\section{Monodelfo}
\begin{itemize}
\item {Grp. gram.:adj.}
\end{itemize}
\begin{itemize}
\item {Proveniência:(Do gr. \textunderscore monos\textunderscore  + \textunderscore delphus\textunderscore )}
\end{itemize}
Diz-se da classe de animaes, em que o féto se desenvolve completamente na matriz.
\section{Monodelpho}
\begin{itemize}
\item {Grp. gram.:adj.}
\end{itemize}
\begin{itemize}
\item {Proveniência:(Do gr. \textunderscore monos\textunderscore  + \textunderscore delphus\textunderscore )}
\end{itemize}
Diz-se da classe de animaes, em que o féto se desenvolve completamente na matriz.
\section{Monódia}
\begin{itemize}
\item {Grp. gram.:f.}
\end{itemize}
\begin{itemize}
\item {Proveniência:(Do gr. \textunderscore monos\textunderscore  + \textunderscore ode\textunderscore )}
\end{itemize}
Monólogos das antigas tragédias.
Canção plangente, executada por uma só voz.
\section{Monódico}
\begin{itemize}
\item {Grp. gram.:adj.}
\end{itemize}
Relativo á monódia.
\section{Monodimétrico}
\begin{itemize}
\item {Grp. gram.:adj.}
\end{itemize}
\begin{itemize}
\item {Utilização:Mathem.}
\end{itemize}
Diz-se da projecção axonométrica, quando seis dos três eixos principaes têm inclinação igual.
\section{Monodonte}
\begin{itemize}
\item {Grp. gram.:adj.}
\end{itemize}
\begin{itemize}
\item {Proveniência:(Do gr. \textunderscore monos\textunderscore  + \textunderscore odous\textunderscore , \textunderscore odontos\textunderscore )}
\end{itemize}
Que tem só um dente.
\section{Monodrama}
\begin{itemize}
\item {Grp. gram.:m.}
\end{itemize}
\begin{itemize}
\item {Utilização:Mús.}
\end{itemize}
\begin{itemize}
\item {Proveniência:(De \textunderscore mono...\textunderscore  + \textunderscore drama\textunderscore )}
\end{itemize}
Drama, cantado por uma só personagem.
Scena dramática, em monólogo.
\section{Monogamia}
\begin{itemize}
\item {Grp. gram.:f.}
\end{itemize}
\begin{itemize}
\item {Proveniência:(De \textunderscore monógamo\textunderscore )}
\end{itemize}
Estado conjugal, em que o marido é monógamo.
União de certos animaes, em que os individuos dos dois sexos convivem só aos pares.
Qualidade das plantas, cujas flores são separadas e distintas.
\section{Monogâmico}
\begin{itemize}
\item {Grp. gram.:adj.}
\end{itemize}
Relativo á monogamia.
\section{Monogamista}
\begin{itemize}
\item {Grp. gram.:m.  e  adj.}
\end{itemize}
Partidário da monogamia ou do estado conjugal, em que o marido tem uma só espôsa.
\section{Monógamo}
\begin{itemize}
\item {Grp. gram.:m.  e  adj.}
\end{itemize}
\begin{itemize}
\item {Utilização:Bot.}
\end{itemize}
\begin{itemize}
\item {Utilização:Chím.}
\end{itemize}
\begin{itemize}
\item {Proveniência:(Do gr. \textunderscore monos\textunderscore  + \textunderscore gamos\textunderscore )}
\end{itemize}
O que tem só uma espôsa.
O animal que se acasala com uma só fêmea.
Cujas flores têm o mesmo sexo.
Diz-se dos corpos, cujas combinações se realizam na relação de um só equivalente dos corpos que se unem.
\section{Monogástrico}
\begin{itemize}
\item {Grp. gram.:adj.}
\end{itemize}
\begin{itemize}
\item {Proveniência:(Do gr. \textunderscore monos\textunderscore  + \textunderscore gaster\textunderscore )}
\end{itemize}
Que tem só um estômago.
\section{Monogenésico}
\begin{itemize}
\item {Grp. gram.:adj.}
\end{itemize}
\begin{itemize}
\item {Proveniência:(Do gr. \textunderscore monos\textunderscore  + \textunderscore genesis\textunderscore )}
\end{itemize}
Que tem só uma fórma de reproducção, por meio de ovos ou óvulos.
\section{Monogenia}
\begin{itemize}
\item {Grp. gram.:f.}
\end{itemize}
\begin{itemize}
\item {Proveniência:(Do gr. \textunderscore monos\textunderscore  + \textunderscore genea\textunderscore )}
\end{itemize}
Modo de geração, que consiste em produzir, por meio de um corpo organizado, uma parte que se separa logo, constituíndo novo indivíduo.
\section{Monogênico}
\begin{itemize}
\item {Grp. gram.:adj.}
\end{itemize}
Relativo á monogenia.
\section{Monogênio}
\begin{itemize}
\item {Grp. gram.:adj.}
\end{itemize}
Diz-se dos animaes que, pertencendo a gêneros differentes, têm todavia tal semelhança entre si, que parecem do mesmo gênero.
(Cp. \textunderscore monogenia\textunderscore )
\section{Monogenismo}
\begin{itemize}
\item {Grp. gram.:m.}
\end{itemize}
\begin{itemize}
\item {Proveniência:(De \textunderscore monogenia\textunderscore )}
\end{itemize}
Systema anthropológico, que considera todos os homens provenientes de um só tronco ou de uma só origem.
\section{Monogenista}
\begin{itemize}
\item {Grp. gram.:m.}
\end{itemize}
\begin{itemize}
\item {Proveniência:(De \textunderscore monogenia\textunderscore )}
\end{itemize}
Sectário do monogenismo.
\section{Monógeno}
\begin{itemize}
\item {Grp. gram.:adj.}
\end{itemize}
O mesmo ou melhor que \textunderscore monogênio\textunderscore .
\section{Monoginia}
\begin{itemize}
\item {Grp. gram.:f.}
\end{itemize}
\begin{itemize}
\item {Utilização:Bot.}
\end{itemize}
\begin{itemize}
\item {Proveniência:(De \textunderscore monógino\textunderscore )}
\end{itemize}
Ordem do sistema sexual de Linneu, em que se compreende as plantas monóginas.
\section{Monógino}
\begin{itemize}
\item {Grp. gram.:adj.}
\end{itemize}
\begin{itemize}
\item {Proveniência:(Do gr. \textunderscore monos\textunderscore  + \textunderscore gune\textunderscore )}
\end{itemize}
Diz-se dos vegetaes, cuja flôr tem só um pistilo.
\section{Monografia}
\begin{itemize}
\item {Grp. gram.:f.}
\end{itemize}
Dissertação ou tratado á cêrca de um ponto particular, de uma ciência, arte, etc.
Descripção de um só gênero ou espécie de animaes ou vegetaes.
(Cp. \textunderscore monógrafo\textunderscore )
\section{Monográfico}
\begin{itemize}
\item {Grp. gram.:adj.}
\end{itemize}
Relativo á monografia.
\section{Monografista}
\begin{itemize}
\item {Grp. gram.:m.}
\end{itemize}
Autor de uma monografia.
Monógrafo.
\section{Monógrafo}
\begin{itemize}
\item {Grp. gram.:adj.}
\end{itemize}
\begin{itemize}
\item {Grp. gram.:M.}
\end{itemize}
\begin{itemize}
\item {Proveniência:(Do gr. \textunderscore monos\textunderscore  + \textunderscore graphein\textunderscore )}
\end{itemize}
Que trata de um só objecto.
Autor de uma monografia.
\section{Monograma}
\begin{itemize}
\item {Grp. gram.:m.}
\end{itemize}
\begin{itemize}
\item {Proveniência:(Do gr. \textunderscore monos\textunderscore  + \textunderscore gramma\textunderscore )}
\end{itemize}
Reunião ou entrelaçamento de duas ou mais letras.
\section{Monogramático}
\begin{itemize}
\item {Grp. gram.:adj.}
\end{itemize}
Relativo ao monograma.
\section{Monogramista}
\begin{itemize}
\item {Grp. gram.:m.}
\end{itemize}
\begin{itemize}
\item {Proveniência:(De \textunderscore monograma\textunderscore )}
\end{itemize}
Aquele que faz monogramas.
Artista, que não assigna as suas obras com o nome por extenso, mas sim com um monograma, uma abreviatura ou iniciaes.
\section{Monogramma}
\begin{itemize}
\item {Grp. gram.:m.}
\end{itemize}
\begin{itemize}
\item {Proveniência:(Do gr. \textunderscore monos\textunderscore  + \textunderscore gramma\textunderscore )}
\end{itemize}
Reunião ou entrelaçamento de duas ou mais letras.
\section{Monogrammático}
\begin{itemize}
\item {Grp. gram.:adj.}
\end{itemize}
Relativo ao monogramma.
\section{Monogrammista}
\begin{itemize}
\item {Grp. gram.:m.}
\end{itemize}
\begin{itemize}
\item {Proveniência:(De \textunderscore monogramma\textunderscore )}
\end{itemize}
Aquelle que faz monogrammas.
Artista, que não assigna as suas obras com o nome por extenso, mas sim com um monogramma, uma abreviatura ou iniciaes.
\section{Monogrammo}
\begin{itemize}
\item {Grp. gram.:adj.}
\end{itemize}
\begin{itemize}
\item {Utilização:Philos.}
\end{itemize}
Diz-se do trabalho de pintura, composto só de linhas ou contornos.
Incorpóreo, que não é palpável.
(Cp. \textunderscore monogramma\textunderscore )
\section{Monogramo}
\begin{itemize}
\item {Grp. gram.:adj.}
\end{itemize}
\begin{itemize}
\item {Utilização:Philos.}
\end{itemize}
Diz-se do trabalho de pintura, composto só de linhas ou contornos.
Incorpóreo, que não é palpável.
(Cp. \textunderscore monograma\textunderscore )
\section{Monographia}
\begin{itemize}
\item {Grp. gram.:f.}
\end{itemize}
Dissertação ou tratado á cêrca de um ponto particular, de uma sciência, arte, etc.
Descripção de um só gênero ou espécie de animaes ou vegetaes.
(Cp. \textunderscore monógrapho\textunderscore )
\section{Monográphico}
\begin{itemize}
\item {Grp. gram.:adj.}
\end{itemize}
Relativo á monographia.
\section{Monographista}
\begin{itemize}
\item {Grp. gram.:m.}
\end{itemize}
Autor de uma monografia.
Monógrapho.
\section{Monógrapho}
\begin{itemize}
\item {Grp. gram.:adj.}
\end{itemize}
\begin{itemize}
\item {Grp. gram.:M.}
\end{itemize}
\begin{itemize}
\item {Proveniência:(Do gr. \textunderscore monos\textunderscore  + \textunderscore graphein\textunderscore )}
\end{itemize}
Que trata de um só objecto.
Autor de uma monographia.
\section{Monogynia}
\begin{itemize}
\item {Grp. gram.:f.}
\end{itemize}
\begin{itemize}
\item {Utilização:Bot.}
\end{itemize}
\begin{itemize}
\item {Proveniência:(De \textunderscore monógyno\textunderscore )}
\end{itemize}
Ordem do systema sexual de Linneu, em que se comprehende as plantas monógynas.
\section{Monógyno}
\begin{itemize}
\item {Grp. gram.:adj.}
\end{itemize}
\begin{itemize}
\item {Proveniência:(Do gr. \textunderscore monos\textunderscore  + \textunderscore gune\textunderscore )}
\end{itemize}
Diz-se dos vegetaes, cuja flôr tem só um pistillo.
\section{Monohydratado}
\begin{itemize}
\item {Grp. gram.:adj.}
\end{itemize}
Que se acha no estado de monohydrato.
\section{Monohydrato}
\begin{itemize}
\item {Grp. gram.:m.}
\end{itemize}
\begin{itemize}
\item {Proveniência:(De \textunderscore mono...\textunderscore  + \textunderscore hydrato\textunderscore )}
\end{itemize}
Primeiro hydrato de uma substância que fórma muitos.
\section{Monohýdrico}
\begin{itemize}
\item {Grp. gram.:adj.}
\end{itemize}
\begin{itemize}
\item {Utilização:Chím.}
\end{itemize}
\begin{itemize}
\item {Proveniência:(De \textunderscore mono...\textunderscore  + \textunderscore hydrico\textunderscore )}
\end{itemize}
Diz-se de um composto, que tem uma proporção de hydrogênio por uma de outro componente.
\section{Monóhylo}
\begin{itemize}
\item {Grp. gram.:adj.}
\end{itemize}
\begin{itemize}
\item {Utilização:Zool.}
\end{itemize}
\begin{itemize}
\item {Proveniência:(Do gr. \textunderscore monos\textunderscore  + \textunderscore hule\textunderscore )}
\end{itemize}
Que tem corpo formado de uma só massa homogênea.
\section{Monohypogynia}
\begin{itemize}
\item {Grp. gram.:f.}
\end{itemize}
\begin{itemize}
\item {Utilização:Bot.}
\end{itemize}
Estado de monohypógyno.
\section{Monohypógyno}
\begin{itemize}
\item {Grp. gram.:adj.}
\end{itemize}
\begin{itemize}
\item {Utilização:Bot.}
\end{itemize}
Diz-se de uma planta monocotyledónea, que tem estames hypógynos.
\section{Monoïdratado}
\begin{itemize}
\item {Grp. gram.:adj.}
\end{itemize}
Que se acha no estado de monohidrato.
\section{Monóicia}
\begin{itemize}
\item {Grp. gram.:f.}
\end{itemize}
\begin{itemize}
\item {Utilização:Bot.}
\end{itemize}
(V. \textunderscore monécia\textunderscore , que é fórma preferível)
\section{Monoico}
\begin{itemize}
\item {Grp. gram.:adj.}
\end{itemize}
\begin{itemize}
\item {Utilização:Bot.}
\end{itemize}
(V.monécico)
\section{Monoïdrato}
\begin{itemize}
\item {Grp. gram.:m.}
\end{itemize}
\begin{itemize}
\item {Proveniência:(De \textunderscore mono...\textunderscore  + \textunderscore hidrato\textunderscore )}
\end{itemize}
Primeiro hidrato de uma substância que fórma muitos.
\section{Monoídrico}
\begin{itemize}
\item {Grp. gram.:adj.}
\end{itemize}
\begin{itemize}
\item {Utilização:Chím.}
\end{itemize}
\begin{itemize}
\item {Proveniência:(De \textunderscore mono...\textunderscore  + \textunderscore hidrico\textunderscore )}
\end{itemize}
Diz-se de um composto, que tem uma proporção de hidrogênio por uma de outro componente.
\section{Monóilo}
\begin{itemize}
\item {fónica:nó-i}
\end{itemize}
\begin{itemize}
\item {Grp. gram.:adj.}
\end{itemize}
\begin{itemize}
\item {Utilização:Zool.}
\end{itemize}
\begin{itemize}
\item {Proveniência:(Do gr. \textunderscore monos\textunderscore  + \textunderscore hule\textunderscore )}
\end{itemize}
Que tem corpo formado de uma só massa homogênea.
\section{Monoipoginia}
\begin{itemize}
\item {fónica:no-i}
\end{itemize}
\begin{itemize}
\item {Grp. gram.:f.}
\end{itemize}
\begin{itemize}
\item {Utilização:Bot.}
\end{itemize}
Estado de monohipógino.
\section{Monoipógino}
\begin{itemize}
\item {fónica:no-i}
\end{itemize}
\begin{itemize}
\item {Grp. gram.:adj.}
\end{itemize}
\begin{itemize}
\item {Utilização:Bot.}
\end{itemize}
Diz-se de uma planta monocotiledónea, que tem estames hipóginos.
\section{Monoleína}
\begin{itemize}
\item {Grp. gram.:f.}
\end{itemize}
\begin{itemize}
\item {Proveniência:(De \textunderscore mono...\textunderscore  + \textunderscore oleína\textunderscore )}
\end{itemize}
Combinação do ácido oleico com a glycerina.
\section{Monolépido}
\begin{itemize}
\item {Grp. gram.:adj.}
\end{itemize}
\begin{itemize}
\item {Utilização:Zool.}
\end{itemize}
\begin{itemize}
\item {Proveniência:(Do gr. \textunderscore monos\textunderscore  + \textunderscore lepis\textunderscore )}
\end{itemize}
Que tem uma só escama.
\section{Monolíthico}
\begin{itemize}
\item {Grp. gram.:adj.}
\end{itemize}
Relativo a monólitho.
Semelhante a um monólitho.
\section{Monólitho}
\begin{itemize}
\item {Grp. gram.:m.}
\end{itemize}
\begin{itemize}
\item {Proveniência:(Do gr. \textunderscore monos\textunderscore  + \textunderscore lithos\textunderscore )}
\end{itemize}
Pedra de grandes dimensões.
Obra ou monumento, feito de uma só pedra.
\section{Monolítico}
\begin{itemize}
\item {Grp. gram.:adj.}
\end{itemize}
Relativo a monólito.
Semelhante a um monólito.
\section{Monólito}
\begin{itemize}
\item {Grp. gram.:m.}
\end{itemize}
\begin{itemize}
\item {Proveniência:(Do gr. \textunderscore monos\textunderscore  + \textunderscore lithos\textunderscore )}
\end{itemize}
Pedra de grandes dimensões.
Obra ou monumento, feito de uma só pedra.
\section{Monologar}
\begin{itemize}
\item {Grp. gram.:v. i.}
\end{itemize}
\begin{itemize}
\item {Grp. gram.:V. t.}
\end{itemize}
Recitar monólogos.
Falar consigo só.
Dizer, de si para si.
\section{Monologia}
\begin{itemize}
\item {Grp. gram.:f.}
\end{itemize}
\begin{itemize}
\item {Utilização:P. us.}
\end{itemize}
Costume de falar a sós.
\section{Monológico}
\begin{itemize}
\item {Grp. gram.:adj.}
\end{itemize}
Relativo a monólogo.
\section{Monólogo}
\begin{itemize}
\item {Grp. gram.:m.}
\end{itemize}
\begin{itemize}
\item {Proveniência:(Do gr. \textunderscore monos\textunderscore  + \textunderscore logos\textunderscore )}
\end{itemize}
Peça theatral ou scena, em que apparece e fala um só actôr.
Solilóquio.
\section{Monomachia}
\begin{itemize}
\item {fónica:qui}
\end{itemize}
\begin{itemize}
\item {Grp. gram.:f.}
\end{itemize}
\begin{itemize}
\item {Utilização:Jur.}
\end{itemize}
\begin{itemize}
\item {Utilização:ant.}
\end{itemize}
\begin{itemize}
\item {Proveniência:(Gr. \textunderscore monomakhia\textunderscore )}
\end{itemize}
Combate entre dois homens.
Meio de prova judiciária, por meio do duello.
\section{Monomania}
\begin{itemize}
\item {Grp. gram.:f.}
\end{itemize}
\begin{itemize}
\item {Proveniência:(Do gr. \textunderscore monos\textunderscore  + \textunderscore mania\textunderscore )}
\end{itemize}
Loucura ou mania, caracterizada por o doente têr uma ideia fixa.
\section{Monomaníaco}
\begin{itemize}
\item {Grp. gram.:m.  e  adj.}
\end{itemize}
Aquelle que soffre monomania.
\section{Monomaquia}
\begin{itemize}
\item {Grp. gram.:f.}
\end{itemize}
\begin{itemize}
\item {Utilização:Jur.}
\end{itemize}
\begin{itemize}
\item {Utilização:ant.}
\end{itemize}
\begin{itemize}
\item {Proveniência:(Gr. \textunderscore monomakhia\textunderscore )}
\end{itemize}
Combate entre dois homens.
Meio de prova judiciária, por meio do duelo.
\section{Monómero}
\begin{itemize}
\item {Grp. gram.:adj.}
\end{itemize}
\begin{itemize}
\item {Utilização:Zool.}
\end{itemize}
\begin{itemize}
\item {Grp. gram.:M. pl.}
\end{itemize}
\begin{itemize}
\item {Proveniência:(Do gr. \textunderscore monos\textunderscore  + \textunderscore meros\textunderscore )}
\end{itemize}
Diz-se dos insectos, cujos tarsos têm uma só articulação.
Secção de insectos coleópteros.
\section{Monometálico}
\begin{itemize}
\item {Grp. gram.:adj.}
\end{itemize}
\begin{itemize}
\item {Proveniência:(De \textunderscore mono...\textunderscore  + \textunderscore metálico\textunderscore )}
\end{itemize}
Relativo ao monometalismo.
\section{Monometállico}
\begin{itemize}
\item {Grp. gram.:adj.}
\end{itemize}
\begin{itemize}
\item {Proveniência:(De \textunderscore mono...\textunderscore  + \textunderscore metállico\textunderscore )}
\end{itemize}
Relativo ao monometallismo.
\section{Monofilo}
\begin{itemize}
\item {Grp. gram.:adj.}
\end{itemize}
\begin{itemize}
\item {Proveniência:(Do gr. \textunderscore monos\textunderscore  + \textunderscore phullon\textunderscore )}
\end{itemize}
Formado de uma só peça, ou que tem uma só fólha, (falando-se de vegetaes).
\section{Monofiodontes}
\begin{itemize}
\item {Grp. gram.:m. pl.}
\end{itemize}
\begin{itemize}
\item {Proveniência:(Do gr. \textunderscore mono\textunderscore  + \textunderscore phu\textunderscore  + \textunderscore odous\textunderscore , \textunderscore odontos\textunderscore )}
\end{itemize}
Animaes, que têm uma só dentição.
\section{Monofisismo}
\begin{itemize}
\item {Grp. gram.:m.}
\end{itemize}
Doutrina dos que admitiam uma só natureza em Jesus-Cristo.
(Cp. \textunderscore monofisita\textunderscore )
\section{Monofisita}
\begin{itemize}
\item {Grp. gram.:adj.}
\end{itemize}
\begin{itemize}
\item {Grp. gram.:M.}
\end{itemize}
\begin{itemize}
\item {Proveniência:(Gr. \textunderscore monophysites\textunderscore )}
\end{itemize}
Relativo ao monofisismo.
Sectário do monofisismo.
\section{Monofobia}
\begin{itemize}
\item {Grp. gram.:f.}
\end{itemize}
\begin{itemize}
\item {Proveniência:(Do gr. \textunderscore monos\textunderscore  + \textunderscore phobein\textunderscore )}
\end{itemize}
Mêdo mórbido da solidão.
\section{Monofobo}
\begin{itemize}
\item {Grp. gram.:m.}
\end{itemize}
Aquele que sofre monofobia.
\section{Monofonia}
\begin{itemize}
\item {Grp. gram.:f.}
\end{itemize}
\begin{itemize}
\item {Utilização:Mús.}
\end{itemize}
\begin{itemize}
\item {Proveniência:(Do gr. \textunderscore monos\textunderscore  + \textunderscore phone\textunderscore )}
\end{itemize}
Producção de um som único.
\section{Monofónico}
\begin{itemize}
\item {Grp. gram.:adj.}
\end{itemize}
Relativo á monofonia.
\section{Monoftalmo}
\begin{itemize}
\item {Grp. gram.:adj.}
\end{itemize}
\begin{itemize}
\item {Proveniência:(Do gr. \textunderscore monos\textunderscore  + \textunderscore ophtalmos\textunderscore )}
\end{itemize}
Que nasce com um só ôlho.
O mesmo que \textunderscore ciclopia\textunderscore .
\section{Monometalismo}
\begin{itemize}
\item {Grp. gram.:m.}
\end{itemize}
\begin{itemize}
\item {Proveniência:(De \textunderscore mono...\textunderscore  + \textunderscore metal\textunderscore )}
\end{itemize}
Sistema económico, segundo o qual a moéda deveria sêr representada por metal de uma só espécie.
\section{Monometalista}
\begin{itemize}
\item {Grp. gram.:m.}
\end{itemize}
Partidário do monometalismo.
\section{Monometallismo}
\begin{itemize}
\item {Grp. gram.:m.}
\end{itemize}
\begin{itemize}
\item {Proveniência:(De \textunderscore mono...\textunderscore  + \textunderscore metal\textunderscore )}
\end{itemize}
Systema económico, segundo o qual a moéda deveria sêr representada por metal de uma só espécie.
\section{Monometallista}
\begin{itemize}
\item {Grp. gram.:m.}
\end{itemize}
Partidário do monometallismo.
\section{Monométrico}
\begin{itemize}
\item {Grp. gram.:adj.}
\end{itemize}
Relativo ao monómetro.
Formado de versos de uma só medida: \textunderscore poema monómetrico\textunderscore .
\section{Monómetro}
\begin{itemize}
\item {Grp. gram.:m.}
\end{itemize}
\begin{itemize}
\item {Proveniência:(Do gr. \textunderscore monos\textunderscore  + \textunderscore metron\textunderscore )}
\end{itemize}
Poema, em que ha só uma espécie de versos.
\section{Monomiários}
\begin{itemize}
\item {Grp. gram.:m. pl.}
\end{itemize}
\begin{itemize}
\item {Proveniência:(Do gr. \textunderscore monos\textunderscore , + \textunderscore mus\textunderscore , \textunderscore muos\textunderscore )}
\end{itemize}
Ordem de moluscos acéfalos.
\section{Monomiceto}
\begin{itemize}
\item {Grp. gram.:m.}
\end{itemize}
Gênero de polypeiros.
\section{Monómio}
\begin{itemize}
\item {Grp. gram.:m.}
\end{itemize}
\begin{itemize}
\item {Proveniência:(Do gr. \textunderscore monos\textunderscore  + \textunderscore nomos\textunderscore )}
\end{itemize}
Quantidade algébrica, entre cujas partes não há sinal interposto de addição ou de subtracção.
\section{Monomocaia}
\begin{itemize}
\item {Grp. gram.:m.}
\end{itemize}
\begin{itemize}
\item {Utilização:T. de Moçambique}
\end{itemize}
Espécie de cyclone.
\section{Monomphalia}
\begin{itemize}
\item {Grp. gram.:f.}
\end{itemize}
Estado de \textunderscore monomphálio\textunderscore .
\section{Monomphálico}
\begin{itemize}
\item {Grp. gram.:adj.}
\end{itemize}
\begin{itemize}
\item {Proveniência:(Do gr. \textunderscore monos\textunderscore  + \textunderscore omphalos\textunderscore )}
\end{itemize}
Diz-se do monstro, formado de dois indivíduos quási completos, com um umbigo commum.
\section{Monomphálio}
\begin{itemize}
\item {Grp. gram.:m.}
\end{itemize}
Monstro monomphálico.
\section{Monomyários}
\begin{itemize}
\item {Grp. gram.:m. pl.}
\end{itemize}
\begin{itemize}
\item {Proveniência:(Do gr. \textunderscore monos\textunderscore , \textunderscore mus\textunderscore , \textunderscore muos\textunderscore )}
\end{itemize}
Ordem de molluscos acéphalos.
\section{Mononeuro}
\begin{itemize}
\item {Grp. gram.:adj.}
\end{itemize}
\begin{itemize}
\item {Utilização:Zool.}
\end{itemize}
\begin{itemize}
\item {Proveniência:(Do gr. \textunderscore monos\textunderscore  + \textunderscore neuron\textunderscore )}
\end{itemize}
Que tem um só systema nervoso.
\section{Mononfalia}
\begin{itemize}
\item {Grp. gram.:f.}
\end{itemize}
Estado de \textunderscore mononfálio\textunderscore .
\section{Mononfálico}
\begin{itemize}
\item {Grp. gram.:adj.}
\end{itemize}
\begin{itemize}
\item {Proveniência:(Do gr. \textunderscore monos\textunderscore  + \textunderscore omphalos\textunderscore )}
\end{itemize}
Diz-se do monstro, formado de dois indivíduos quási completos, com um umbigo commum.
\section{Mononfálio}
\begin{itemize}
\item {Grp. gram.:m.}
\end{itemize}
Monstro mononfálico.
\section{Monónico}
\begin{itemize}
\item {Grp. gram.:m.}
\end{itemize}
Gênero de insectos coleópteros tetrâmeros.
\section{Monónimo}
\begin{itemize}
\item {Grp. gram.:adj.}
\end{itemize}
\begin{itemize}
\item {Proveniência:(Do gr. \textunderscore monos\textunderscore  + \textunderscore onuma\textunderscore )}
\end{itemize}
Que exprime uma só ideia: \textunderscore termo monónimo\textunderscore . Cf. M. Maciel, \textunderscore Gram.\textunderscore 
\section{Monónymo}
\begin{itemize}
\item {Grp. gram.:adj.}
\end{itemize}
\begin{itemize}
\item {Proveniência:(Do gr. \textunderscore monos\textunderscore  + \textunderscore onuma\textunderscore )}
\end{itemize}
Que exprime uma só ideia: \textunderscore termo monónymo\textunderscore . Cf. M. Maciel, \textunderscore Gram.\textunderscore 
\section{Monopegia}
\begin{itemize}
\item {Grp. gram.:f.}
\end{itemize}
\begin{itemize}
\item {Utilização:Med.}
\end{itemize}
\begin{itemize}
\item {Proveniência:(Do gr. \textunderscore monos\textunderscore  + \textunderscore pegeis\textunderscore )}
\end{itemize}
Dôr de cabeça, que occupa uma parte muito circunscrita.
\section{Monoperiantado}
\begin{itemize}
\item {Grp. gram.:adj.}
\end{itemize}
\begin{itemize}
\item {Proveniência:(De \textunderscore mono...\textunderscore  + \textunderscore perianto\textunderscore )}
\end{itemize}
Que tem um só perianto.
\section{Monoperiânteo}
\begin{itemize}
\item {Grp. gram.:adj.}
\end{itemize}
O mesmo que \textunderscore monoperiantado\textunderscore .
\section{Monoperianthado}
\begin{itemize}
\item {Grp. gram.:adj.}
\end{itemize}
\begin{itemize}
\item {Proveniência:(De \textunderscore mono...\textunderscore  + \textunderscore periantho\textunderscore )}
\end{itemize}
Que tem um só periantho.
\section{Monoperiântheo}
\begin{itemize}
\item {Grp. gram.:adj.}
\end{itemize}
O mesmo que \textunderscore monoperianthado\textunderscore .
\section{Monoperiginia}
\begin{itemize}
\item {Grp. gram.:f.}
\end{itemize}
Estado de monoperígino.
\section{Monoperígino}
\begin{itemize}
\item {Grp. gram.:adj.}
\end{itemize}
\begin{itemize}
\item {Utilização:Bot.}
\end{itemize}
Diz-se da planta monocotiledónea, que tem estames períginos.
\section{Monoperigynia}
\begin{itemize}
\item {Grp. gram.:f.}
\end{itemize}
Estado de monoperígyno.
\section{Monoperígyno}
\begin{itemize}
\item {Grp. gram.:adj.}
\end{itemize}
\begin{itemize}
\item {Utilização:Bot.}
\end{itemize}
Diz-se da planta monocotyledónea, que tem estames perígynos.
\section{Monopétalo}
\begin{itemize}
\item {Grp. gram.:adj.}
\end{itemize}
\begin{itemize}
\item {Proveniência:(De \textunderscore mono...\textunderscore  + \textunderscore pétala\textunderscore )}
\end{itemize}
Que tem só uma pétala.
\section{Monophobia}
\begin{itemize}
\item {Grp. gram.:f.}
\end{itemize}
\begin{itemize}
\item {Proveniência:(Do gr. \textunderscore monos\textunderscore  + \textunderscore phobein\textunderscore )}
\end{itemize}
Mêdo mórbido da solidão.
\section{Monóphobo}
\begin{itemize}
\item {Grp. gram.:m.}
\end{itemize}
Aquelle que soffre monophobia.
\section{Monophonia}
\begin{itemize}
\item {Grp. gram.:f.}
\end{itemize}
\begin{itemize}
\item {Utilização:Mús.}
\end{itemize}
\begin{itemize}
\item {Proveniência:(Do gr. \textunderscore monos\textunderscore  + \textunderscore phone\textunderscore )}
\end{itemize}
Producção de um som único.
\section{Monophónico}
\begin{itemize}
\item {Grp. gram.:adj.}
\end{itemize}
Relativo á monophonia.
\section{Monophtalmo}
\begin{itemize}
\item {Grp. gram.:adj.}
\end{itemize}
\begin{itemize}
\item {Proveniência:(Do gr. \textunderscore monos\textunderscore  + \textunderscore ophtalmos\textunderscore )}
\end{itemize}
Que nasce com um só ôlho.
O mesmo que \textunderscore cyclopia\textunderscore .
\section{Monophyllo}
\begin{itemize}
\item {Grp. gram.:adj.}
\end{itemize}
\begin{itemize}
\item {Proveniência:(Do gr. \textunderscore monos\textunderscore  + \textunderscore phullon\textunderscore )}
\end{itemize}
Formado de uma só peça, ou que tem uma só fólha, (falando-se de vegetaes).
\section{Monophyodontes}
\begin{itemize}
\item {Grp. gram.:m. pl.}
\end{itemize}
\begin{itemize}
\item {Proveniência:(Do gr. \textunderscore mono\textunderscore  + \textunderscore phu\textunderscore  + \textunderscore odous\textunderscore , \textunderscore odontos\textunderscore )}
\end{itemize}
Animaes, que têm uma só dentição.
\section{Monophysismo}
\begin{itemize}
\item {Grp. gram.:m.}
\end{itemize}
Doutrina dos que admittiam uma só natureza em Jesus-Christo.
(Cp. \textunderscore monophysita\textunderscore )
\section{Monophysita}
\begin{itemize}
\item {Grp. gram.:adj.}
\end{itemize}
\begin{itemize}
\item {Grp. gram.:M.}
\end{itemize}
\begin{itemize}
\item {Proveniência:(Gr. \textunderscore monophysites\textunderscore )}
\end{itemize}
Relativo ao monophysismo.
Sectário do monophysismo.
\section{Monófito}
\begin{itemize}
\item {Grp. gram.:adj.}
\end{itemize}
\begin{itemize}
\item {Proveniência:(Do gr. \textunderscore monos\textunderscore  + \textunderscore phuton\textunderscore )}
\end{itemize}
Que abrange uma só espécie, (falando-se de gêneros de plantas).
\section{Monóphyto}
\begin{itemize}
\item {Grp. gram.:adj.}
\end{itemize}
\begin{itemize}
\item {Proveniência:(Do gr. \textunderscore monos\textunderscore  + \textunderscore phuton\textunderscore )}
\end{itemize}
Que abrange uma só espécie, (falando-se de gêneros de plantas).
\section{Monoplano}
\begin{itemize}
\item {Grp. gram.:m.}
\end{itemize}
\begin{itemize}
\item {Proveniência:(De \textunderscore mono...\textunderscore  + \textunderscore plano\textunderscore )}
\end{itemize}
Espécie de aeroplano, de um só plano ou lâmina.
\section{Monoplástico}
\begin{itemize}
\item {Grp. gram.:adj.}
\end{itemize}
\begin{itemize}
\item {Proveniência:(De \textunderscore mono...\textunderscore  + \textunderscore plástico\textunderscore )}
\end{itemize}
Feito de uma só peça.
\section{Monoplegia}
\begin{itemize}
\item {Grp. gram.:f.}
\end{itemize}
\begin{itemize}
\item {Utilização:Med.}
\end{itemize}
\begin{itemize}
\item {Proveniência:(Do gr. \textunderscore monos\textunderscore  + \textunderscore plassein\textunderscore )}
\end{itemize}
Paralysia de um só membro ou grupo muscular.
\section{Monopleurobrânchios}
\begin{itemize}
\item {fónica:qui}
\end{itemize}
\begin{itemize}
\item {Grp. gram.:m. pl.}
\end{itemize}
Ordem de molluscos malacozoários.
\section{Monopleurobrânquios}
\begin{itemize}
\item {Grp. gram.:m. pl.}
\end{itemize}
Ordem de moluscos malacozoários.
\section{Monopneumóneo}
\begin{itemize}
\item {Grp. gram.:adj.}
\end{itemize}
\begin{itemize}
\item {Proveniência:(Do gr. \textunderscore monos\textunderscore  + \textunderscore pneumon\textunderscore )}
\end{itemize}
Que tem um só pulmão ou um só saco pulmonar.
\section{Monópode}
\begin{itemize}
\item {Grp. gram.:adj.}
\end{itemize}
\begin{itemize}
\item {Utilização:Zool.}
\end{itemize}
\begin{itemize}
\item {Proveniência:(Do gr. \textunderscore monos\textunderscore  + \textunderscore pous\textunderscore , \textunderscore podos\textunderscore )}
\end{itemize}
Que tem um só pé.
\section{Monopodia}
\begin{itemize}
\item {Grp. gram.:f.}
\end{itemize}
Qualidade do que é monópode.
\section{Monopódio}
\begin{itemize}
\item {Grp. gram.:m.}
\end{itemize}
\begin{itemize}
\item {Proveniência:(Lat. \textunderscore monopodium\textunderscore )}
\end{itemize}
Mesa com um só pé.
\section{Monopódio}
\begin{itemize}
\item {Grp. gram.:adj.}
\end{itemize}
O mesmo que \textunderscore monópode\textunderscore .
\section{Monopólico}
\begin{itemize}
\item {Grp. gram.:adj.}
\end{itemize}
\begin{itemize}
\item {Utilização:Ant.}
\end{itemize}
Relativo a monopólio.
\section{Monopólio}
\begin{itemize}
\item {Grp. gram.:m.}
\end{itemize}
\begin{itemize}
\item {Proveniência:(Lat. \textunderscore monopolium\textunderscore )}
\end{itemize}
Tráfico exclusivo.
Exploração exclusiva de um negócio ou de uma indústria, em virtude de um privilégio.
Privilégio para a prática dessa indústria ou negócio.
Posse exclusiva.
Açambarcamento de mercadorias, para serem vendidas pelo preço que mais conviér ao vendedor.
\section{Monopolista}
\begin{itemize}
\item {Grp. gram.:m.}
\end{itemize}
Aquelle que monopoliza; aquelle que tem monopólio.
\section{Monopolização}
\begin{itemize}
\item {Grp. gram.:f.}
\end{itemize}
Acto ou effeito de monopolizar.
\section{Monopolizador}
\begin{itemize}
\item {Grp. gram.:m.  e  adj.}
\end{itemize}
O que monopoliza.
\section{Monopolizar}
\begin{itemize}
\item {Grp. gram.:v. t.}
\end{itemize}
Fazer monopólio de.
Açambarcar.
Possuír exclusivamente.
Explorar abusivamente, vendendo sem competidor.
\section{Monopse}
\begin{itemize}
\item {Grp. gram.:adj.}
\end{itemize}
\begin{itemize}
\item {Proveniência:(Do gr. \textunderscore monos\textunderscore  + \textunderscore opsis\textunderscore )}
\end{itemize}
Que tem só um ôlho.
\section{Monopsia}
\begin{itemize}
\item {Grp. gram.:f.}
\end{itemize}
Estado de monopso.
\section{Monopso}
\begin{itemize}
\item {Grp. gram.:adj.}
\end{itemize}
(V.monopse)
\section{Monóptero}
\begin{itemize}
\item {Grp. gram.:adj.}
\end{itemize}
\begin{itemize}
\item {Utilização:Zool.}
\end{itemize}
\begin{itemize}
\item {Utilização:Constr.}
\end{itemize}
\begin{itemize}
\item {Grp. gram.:M.}
\end{itemize}
\begin{itemize}
\item {Proveniência:(Do gr. \textunderscore monos\textunderscore  + \textunderscore pteron\textunderscore )}
\end{itemize}
Que tem só uma barbatana ou uma só asa.
Que é sustentado por uma só ordem de columnas, sem paredes.
Templo monóptero.
Peixe ou ave monóptera.
\section{Monoptoto}
\begin{itemize}
\item {Grp. gram.:m.}
\end{itemize}
\begin{itemize}
\item {Utilização:Gram.}
\end{itemize}
\begin{itemize}
\item {Proveniência:(Do gr. \textunderscore monos\textunderscore  + \textunderscore ptotos\textunderscore )}
\end{itemize}
Palavra grega ou latina, que tem a mesma desinência em todos os casos, como o lat. \textunderscore cornu\textunderscore  em o singular.
\section{Monoquilho}
\begin{itemize}
\item {Grp. gram.:m.}
\end{itemize}
Ganho do bolo, ao voltarete, pelo parceiro que deu codilho na mão anterior.
(Cast. \textunderscore moquillo\textunderscore )
\section{Monoquiro}
\begin{itemize}
\item {Grp. gram.:m.}
\end{itemize}
\begin{itemize}
\item {Proveniência:(Do gr. \textunderscore monos\textunderscore  + \textunderscore kheir\textunderscore )}
\end{itemize}
Gênero de peixes chatos.
\section{Monórchido}
\begin{itemize}
\item {fónica:qui}
\end{itemize}
\begin{itemize}
\item {Grp. gram.:adj.}
\end{itemize}
\begin{itemize}
\item {Utilização:Bot.}
\end{itemize}
\begin{itemize}
\item {Proveniência:(Do gr. \textunderscore monos\textunderscore  + \textunderscore orkhis\textunderscore )}
\end{itemize}
Que tem um só testículo.
Diz-se da planta que, pelo menos na apparência, tem um só tubérculo.
\section{Monoregional}
\begin{itemize}
\item {fónica:re}
\end{itemize}
\begin{itemize}
\item {Grp. gram.:adj.}
\end{itemize}
\begin{itemize}
\item {Utilização:Neol.}
\end{itemize}
\begin{itemize}
\item {Proveniência:(De \textunderscore mono...\textunderscore  + \textunderscore regional\textunderscore )}
\end{itemize}
Diz-se da doença, que se manifesta numa só região do organismo.
\section{Monórquido}
\begin{itemize}
\item {Grp. gram.:adj.}
\end{itemize}
\begin{itemize}
\item {Utilização:Bot.}
\end{itemize}
\begin{itemize}
\item {Proveniência:(Do gr. \textunderscore monos\textunderscore  + \textunderscore orkhis\textunderscore )}
\end{itemize}
Que tem um só testículo.
Diz-se da planta que, pelo menos na aparência, tem um só tubérculo.
\section{Monorregional}
\begin{itemize}
\item {Grp. gram.:adj.}
\end{itemize}
\begin{itemize}
\item {Utilização:Neol.}
\end{itemize}
\begin{itemize}
\item {Proveniência:(De \textunderscore mono...\textunderscore  + \textunderscore regional\textunderscore )}
\end{itemize}
Diz-se da doença, que se manifesta numa só região do organismo.
\section{Monorrimo}
\begin{itemize}
\item {Grp. gram.:adj.}
\end{itemize}
\begin{itemize}
\item {Proveniência:(Fr. \textunderscore monorrime\textunderscore )}
\end{itemize}
Diz-se das composições poéticas, cujos versos têm todos a mesma rima.
\section{Monorritmia}
\begin{itemize}
\item {Grp. gram.:f.}
\end{itemize}
Qualidade de monorítmico.
\section{Monorrítmico}
\begin{itemize}
\item {Grp. gram.:adj.}
\end{itemize}
\begin{itemize}
\item {Utilização:Mús.}
\end{itemize}
\begin{itemize}
\item {Proveniência:(De \textunderscore mono...\textunderscore  + \textunderscore rítmico\textunderscore )}
\end{itemize}
Que tem ritmo uniforme ou pouca variedade de ritmo.
\section{Monorythmia}
\begin{itemize}
\item {fónica:ri}
\end{itemize}
\begin{itemize}
\item {Grp. gram.:f.}
\end{itemize}
Qualidade de monorýthmico.
\section{Monorýthmico}
\begin{itemize}
\item {fónica:ri}
\end{itemize}
\begin{itemize}
\item {Grp. gram.:adj.}
\end{itemize}
\begin{itemize}
\item {Utilização:Mús.}
\end{itemize}
\begin{itemize}
\item {Proveniência:(De \textunderscore mono...\textunderscore  + \textunderscore rýthmico\textunderscore )}
\end{itemize}
Que tem rythmo uniforme ou pouca variedade de rythmo.
\section{Mono-sábio}
\begin{itemize}
\item {Grp. gram.:m.}
\end{itemize}
Moço que, nas praças de toiros em Espanha, trata dos cavallos, ajuda os picadores a montar, etc.
\section{Monosépalo}
\begin{itemize}
\item {fónica:sé}
\end{itemize}
\begin{itemize}
\item {Grp. gram.:adj.}
\end{itemize}
\begin{itemize}
\item {Proveniência:(De \textunderscore mono...\textunderscore  + \textunderscore sépala\textunderscore )}
\end{itemize}
O mesmo que \textunderscore monophyllo\textunderscore .
\section{Monoseriado}
\begin{itemize}
\item {fónica:se}
\end{itemize}
\begin{itemize}
\item {Grp. gram.:adj.}
\end{itemize}
\begin{itemize}
\item {Proveniência:(De \textunderscore mono...\textunderscore  + \textunderscore série\textunderscore )}
\end{itemize}
Que fórma uma só série.
\section{Monositia}
\begin{itemize}
\item {Grp. gram.:f.}
\end{itemize}
\begin{itemize}
\item {Proveniência:(Gr. \textunderscore monositia\textunderscore )}
\end{itemize}
Hábito de tomar uma só refeição em cada dia.
\section{Monosomo}
\begin{itemize}
\item {fónica:sô}
\end{itemize}
\begin{itemize}
\item {Grp. gram.:adj.}
\end{itemize}
\begin{itemize}
\item {Proveniência:(Do gr. \textunderscore monos\textunderscore  + \textunderscore soma\textunderscore )}
\end{itemize}
Diz-se de dois monstros, que têm um só corpo.
\section{Monospermo}
\begin{itemize}
\item {Grp. gram.:adj.}
\end{itemize}
\begin{itemize}
\item {Utilização:Bot.}
\end{itemize}
\begin{itemize}
\item {Proveniência:(Do gr. \textunderscore monos\textunderscore  + \textunderscore sperma\textunderscore )}
\end{itemize}
Que contém uma só semente.
\section{Monósporo}
\begin{itemize}
\item {Grp. gram.:adj.}
\end{itemize}
\begin{itemize}
\item {Utilização:Bot.}
\end{itemize}
\begin{itemize}
\item {Proveniência:(Do gr. \textunderscore monos\textunderscore  + \textunderscore spora\textunderscore )}
\end{itemize}
Que tem um só corpo reproductor.
\section{Monossépalo}
\begin{itemize}
\item {Grp. gram.:adj.}
\end{itemize}
\begin{itemize}
\item {Proveniência:(De \textunderscore mono...\textunderscore  + \textunderscore sépala\textunderscore )}
\end{itemize}
O mesmo que \textunderscore monofilo\textunderscore .
\section{Monosseriado}
\begin{itemize}
\item {Grp. gram.:adj.}
\end{itemize}
\begin{itemize}
\item {Proveniência:(De \textunderscore mono...\textunderscore  + \textunderscore série\textunderscore )}
\end{itemize}
Que fórma uma só série.
\section{Monossilábico}
\begin{itemize}
\item {Grp. gram.:adj.}
\end{itemize}
\begin{itemize}
\item {Proveniência:(De \textunderscore monosílabo\textunderscore )}
\end{itemize}
Formado de uma só sílaba.
Formado de palavras que constam de uma só sílaba.
\section{Monossilabismo}
\begin{itemize}
\item {Grp. gram.:m.}
\end{itemize}
Estado das línguas, em que as raízes são sempre monosílabos. Cf. Latino, \textunderscore Elogios\textunderscore , 86.
Costume dos que falam por monosílabos.
\section{Monossílabo}
\begin{itemize}
\item {Grp. gram.:adj.}
\end{itemize}
\begin{itemize}
\item {Grp. gram.:M.}
\end{itemize}
\begin{itemize}
\item {Grp. gram.:Pl.}
\end{itemize}
\begin{itemize}
\item {Proveniência:(Do gr. \textunderscore monos\textunderscore  + \textunderscore sullabe\textunderscore )}
\end{itemize}
O mesmo que \textunderscore monossilábico\textunderscore .
Palavra, que tem só uma sílaba.
Meias palavras, expressões incompletas.
\section{Monossomo}
\begin{itemize}
\item {Grp. gram.:adj.}
\end{itemize}
\begin{itemize}
\item {Proveniência:(Do gr. \textunderscore monos\textunderscore  + \textunderscore soma\textunderscore )}
\end{itemize}
Diz-se de dois monstros, que têm um só corpo.
\section{Monossulfureto}
\begin{itemize}
\item {fónica:furê}
\end{itemize}
\begin{itemize}
\item {Grp. gram.:m.}
\end{itemize}
\begin{itemize}
\item {Utilização:Chím.}
\end{itemize}
\begin{itemize}
\item {Proveniência:(De \textunderscore mono...\textunderscore  + \textunderscore sulfureto\textunderscore )}
\end{itemize}
Composto binário, que contém um equivalente de enxôfre.
\section{Monóstico}
\begin{itemize}
\item {Grp. gram.:adj.}
\end{itemize}
\begin{itemize}
\item {Grp. gram.:M.}
\end{itemize}
\begin{itemize}
\item {Proveniência:(Gr. \textunderscore monostikhos\textunderscore )}
\end{itemize}
Que consta de um só verso.
Epigramma ou inscripção de um só verso.
\section{Monostigmatia}
\begin{itemize}
\item {Grp. gram.:f.}
\end{itemize}
\begin{itemize}
\item {Utilização:Bot.}
\end{itemize}
\begin{itemize}
\item {Proveniência:(De \textunderscore mono...\textunderscore  + \textunderscore estigma\textunderscore )}
\end{itemize}
Conjunto ou estado das plantas, que tem um só estigma.
\section{Monostilo}
\begin{itemize}
\item {Proveniência:(Do gr. \textunderscore monos\textunderscore  + \textunderscore stulos\textunderscore )}
\end{itemize}
Diz-se do ovário, que tem um só estilete.
\section{Monóstrofe}
\begin{itemize}
\item {Grp. gram.:f.}
\end{itemize}
Composição poética de uma só estrofe.
(Cp. \textunderscore monóstrofo\textunderscore )
\section{Monóstrofo}
\begin{itemize}
\item {Grp. gram.:adj.}
\end{itemize}
\begin{itemize}
\item {Proveniência:(Gr. \textunderscore monostrophos\textunderscore )}
\end{itemize}
Que consta de uma só estrofe.
\section{Monóstrophe}
\begin{itemize}
\item {Grp. gram.:f.}
\end{itemize}
Composição poética de uma só estrophe.
(Cp. \textunderscore monóstropho\textunderscore )
\section{Monóstropho}
\begin{itemize}
\item {Grp. gram.:adj.}
\end{itemize}
\begin{itemize}
\item {Proveniência:(Gr. \textunderscore monostrophos\textunderscore )}
\end{itemize}
Que consta de uma só estrophe.
\section{Monosulfureto}
\begin{itemize}
\item {fónica:sul}
\end{itemize}
\begin{itemize}
\item {Grp. gram.:m.}
\end{itemize}
\begin{itemize}
\item {Utilização:Chím.}
\end{itemize}
\begin{itemize}
\item {Proveniência:(De \textunderscore mono...\textunderscore  + \textunderscore sulfureto\textunderscore )}
\end{itemize}
Composto binário, que contém um equivalente de enxôfre.
\section{Monostylo}
\begin{itemize}
\item {Grp. gram.:adj.}
\end{itemize}
\begin{itemize}
\item {Utilização:Bot.}
\end{itemize}
\begin{itemize}
\item {Proveniência:(Do gr. \textunderscore monos\textunderscore  + \textunderscore stulos\textunderscore )}
\end{itemize}
Diz-se do ovário, que tem um só estilete.
\section{Monosyllábico}
\begin{itemize}
\item {fónica:si}
\end{itemize}
\begin{itemize}
\item {Grp. gram.:adj.}
\end{itemize}
\begin{itemize}
\item {Proveniência:(De \textunderscore monosýllabo\textunderscore )}
\end{itemize}
Formado de uma só sýllaba.
Formado de palavras que constam de uma só sýllaba.
\section{Monosyllabismo}
\begin{itemize}
\item {fónica:si}
\end{itemize}
\begin{itemize}
\item {Grp. gram.:m.}
\end{itemize}
Estado das línguas, em que as raízes são sempre monosýllabos. Cf. Latino, \textunderscore Elogios\textunderscore , 86.
Costume dos que fallam por monosýllabos.
\section{Monosýllabo}
\begin{itemize}
\item {fónica:si}
\end{itemize}
\begin{itemize}
\item {Grp. gram.:adj.}
\end{itemize}
\begin{itemize}
\item {Grp. gram.:M.}
\end{itemize}
\begin{itemize}
\item {Grp. gram.:Pl.}
\end{itemize}
\begin{itemize}
\item {Proveniência:(Do gr. \textunderscore monos\textunderscore  + \textunderscore sullabe\textunderscore )}
\end{itemize}
O mesmo que \textunderscore monosyllábico\textunderscore .
Palavra, que tem só uma sýllaba.
Meias palavras, expressões incompletas.
\section{Monotálamo}
\begin{itemize}
\item {Grp. gram.:adj.}
\end{itemize}
\begin{itemize}
\item {Proveniência:(Do gr. \textunderscore monos\textunderscore  + \textunderscore thalamos\textunderscore )}
\end{itemize}
Diz-se das conchas, que têm uma só cavidade.
\section{Monoteico}
\begin{itemize}
\item {Grp. gram.:adj.}
\end{itemize}
Relativo ao monoteísmo.
\section{Monoteísmo}
\begin{itemize}
\item {Grp. gram.:m.}
\end{itemize}
\begin{itemize}
\item {Proveniência:(Do gr. \textunderscore monos\textunderscore  + \textunderscore theos\textunderscore )}
\end{itemize}
Adoração de um só Deus.
Sistema dos que admitem a existência de um Deus único.
\section{Monoteísta}
\begin{itemize}
\item {Grp. gram.:m. ,  f.  e  adj.}
\end{itemize}
\begin{itemize}
\item {Proveniência:(De \textunderscore mono...\textunderscore  + \textunderscore teísta\textunderscore )}
\end{itemize}
Pessôa, que adora um só Deus.
Sectário do monoteísmo.
\section{Monoteístico}
\begin{itemize}
\item {Grp. gram.:adj.}
\end{itemize}
Relativo ao monoteísmo; monoteico. Cf. Castilho, \textunderscore Fastos\textunderscore , III, 209.
\section{Monotelismo}
\begin{itemize}
\item {Grp. gram.:m.}
\end{itemize}
(V.monotelitismo)
\section{Monotelita}
\begin{itemize}
\item {Grp. gram.:m.}
\end{itemize}
\begin{itemize}
\item {Proveniência:(Do gr. \textunderscore monos\textunderscore  + \textunderscore thelo\textunderscore )}
\end{itemize}
Sectário do monotelitismo.
\section{Monotelitismo}
\begin{itemize}
\item {Grp. gram.:m.}
\end{itemize}
\begin{itemize}
\item {Proveniência:(De \textunderscore monotelita\textunderscore )}
\end{itemize}
Sistema religioso dos que admitem duas naturezas em Cristo e uma só vontade.
\section{Monothálamo}
\begin{itemize}
\item {Grp. gram.:adj.}
\end{itemize}
\begin{itemize}
\item {Proveniência:(Do gr. \textunderscore monos\textunderscore  + \textunderscore thalamos\textunderscore )}
\end{itemize}
Diz-se das conchas, que têm uma só cavidade.
\section{Monotheico}
\begin{itemize}
\item {Grp. gram.:adj.}
\end{itemize}
Relativo ao monoteísmo.
\section{Monotheísmo}
\begin{itemize}
\item {Grp. gram.:m.}
\end{itemize}
\begin{itemize}
\item {Proveniência:(Do gr. \textunderscore monos\textunderscore  + \textunderscore theos\textunderscore )}
\end{itemize}
Adoração de um só Deus.
Systema dos que admittem a existência de um Deus único.
\section{Monotheísta}
\begin{itemize}
\item {Grp. gram.:m. ,  f.  e  adj.}
\end{itemize}
\begin{itemize}
\item {Proveniência:(De \textunderscore mono...\textunderscore  + \textunderscore theísta\textunderscore )}
\end{itemize}
Pessôa, que adora um só Deus.
Sectário do monotheísmo.
\section{Monotheístico}
\begin{itemize}
\item {Grp. gram.:adj.}
\end{itemize}
Relativo ao monotheísmo; monotheico. Cf. Castilho, \textunderscore Fastos\textunderscore , III, 209.
\section{Monothelismo}
\begin{itemize}
\item {Grp. gram.:m.}
\end{itemize}
(V.monothelitismo)
\section{Monothelita}
\begin{itemize}
\item {Grp. gram.:m.}
\end{itemize}
\begin{itemize}
\item {Proveniência:(Do gr. \textunderscore monos\textunderscore  + \textunderscore thelo\textunderscore )}
\end{itemize}
Sectário do monothelitismo.
\section{Monothelitismo}
\begin{itemize}
\item {Grp. gram.:m.}
\end{itemize}
\begin{itemize}
\item {Proveniência:(De \textunderscore monothelita\textunderscore )}
\end{itemize}
Systema religioso dos que admittem duas naturezas em Christo e uma só vontade.
\section{Monothiónico}
\begin{itemize}
\item {Grp. gram.:adj.}
\end{itemize}
\begin{itemize}
\item {Utilização:Chím.}
\end{itemize}
\begin{itemize}
\item {Proveniência:(Do gr. \textunderscore monos\textunderscore  + \textunderscore thion\textunderscore )}
\end{itemize}
Diz-se dos ácidos do enxôfre, que não têm senão um equivalente do radical, como o ácido sulfúrico e o sulfuroso.
\section{Monothongo}
\begin{itemize}
\item {Grp. gram.:m.}
\end{itemize}
\begin{itemize}
\item {Utilização:Gram.}
\end{itemize}
\begin{itemize}
\item {Proveniência:(Do gr. \textunderscore monos\textunderscore  + \textunderscore phtongos\textunderscore )}
\end{itemize}
Grupo vocálico, que representa um só som, por sêr insonora a primeira letra dêsse grupo, como em guerra, quinta.
\section{Monothyro}
\begin{itemize}
\item {Grp. gram.:adj.}
\end{itemize}
\begin{itemize}
\item {Proveniência:(Do gr. \textunderscore monos\textunderscore  + \textunderscore thura\textunderscore )}
\end{itemize}
Diz-se das conchas que têm só uma válvula.
\section{Monotiónico}
\begin{itemize}
\item {Grp. gram.:adj.}
\end{itemize}
\begin{itemize}
\item {Utilização:Chím.}
\end{itemize}
\begin{itemize}
\item {Proveniência:(Do gr. \textunderscore monos\textunderscore  + \textunderscore thion\textunderscore )}
\end{itemize}
Diz-se dos ácidos do enxôfre, que não têm senão um equivalente do radical, como o ácido sulfúrico e o sulfuroso.
\section{Monótipo}
\begin{itemize}
\item {Grp. gram.:adj.}
\end{itemize}
\begin{itemize}
\item {Utilização:Bot.}
\end{itemize}
\begin{itemize}
\item {Proveniência:(De \textunderscore mono...\textunderscore  + \textunderscore typo\textunderscore )}
\end{itemize}
Que, considerado nas suas diversas espécies, apresenta insignificantes modificações.
Que abrange várias espécies, ligadas por muitas relações.
\section{Monotiro}
\begin{itemize}
\item {Grp. gram.:adj.}
\end{itemize}
\begin{itemize}
\item {Proveniência:(Do gr. \textunderscore monos\textunderscore  + \textunderscore thura\textunderscore )}
\end{itemize}
Diz-se das conchas que têm só uma válvula.
\section{Monotomitos}
\begin{itemize}
\item {Grp. gram.:m. pl.}
\end{itemize}
Grupo de insectos coleópteros tetrâmeros.
\section{Monotongo}
\begin{itemize}
\item {Grp. gram.:m.}
\end{itemize}
\begin{itemize}
\item {Utilização:Gram.}
\end{itemize}
\begin{itemize}
\item {Proveniência:(Do gr. \textunderscore monos\textunderscore  + \textunderscore phtongos\textunderscore )}
\end{itemize}
Grupo vocálico, que representa um só som, por sêr insonora a primeira letra dêsse grupo, como em guerra, quinta.
\section{Monotonia}
\begin{itemize}
\item {Grp. gram.:f.}
\end{itemize}
Qualidade do que é monótono.
Ausência de variedade.
Persistência de determinadas condições ou circunstâncias.
Uniformidade ou falta de gradação, na distribuição das tintas de um quadro ou nas côres de uma perspectiva.
\section{Monótono}
\begin{itemize}
\item {Grp. gram.:adj.}
\end{itemize}
\begin{itemize}
\item {Proveniência:(Do gr. \textunderscore monos\textunderscore  + \textunderscore tonos\textunderscore )}
\end{itemize}
Que está sempre no mesmo tom.
Que não tem variação.
Uniforme: \textunderscore vida monótona\textunderscore .
Que não offerece gradação nas côres.
Enfadonho: \textunderscore conversa monótona\textunderscore .
\section{Monotrematos}
\begin{itemize}
\item {Grp. gram.:m. pl.}
\end{itemize}
Animaes que são monotremos.(V.monotremo)
\section{Monotreme}
\begin{itemize}
\item {Grp. gram.:m.  e  adj.}
\end{itemize}
O mesmo que \textunderscore monotremo\textunderscore .
\section{Monotremo}
\begin{itemize}
\item {Grp. gram.:adj.}
\end{itemize}
\begin{itemize}
\item {Utilização:Zool.}
\end{itemize}
\begin{itemize}
\item {Grp. gram.:M. pl.}
\end{itemize}
\begin{itemize}
\item {Proveniência:(Do gr. \textunderscore monos\textunderscore  + \textunderscore trema\textunderscore )}
\end{itemize}
Que tem uma só abertura para todas as excreções.
Grupo de mammíferos, que são monotremos.
\section{Monotríglifo}
\begin{itemize}
\item {Grp. gram.:m.}
\end{itemize}
\begin{itemize}
\item {Utilização:Constr.}
\end{itemize}
\begin{itemize}
\item {Proveniência:(Do gr. \textunderscore monos\textunderscore  + \textunderscore trigluphos\textunderscore )}
\end{itemize}
Espaço de um tríglifo e duas métopas, entre duas colunas de ordem dórica.
\section{Monotríglypho}
\begin{itemize}
\item {Grp. gram.:m.}
\end{itemize}
\begin{itemize}
\item {Utilização:Constr.}
\end{itemize}
\begin{itemize}
\item {Proveniência:(Do gr. \textunderscore monos\textunderscore  + \textunderscore trigluphos\textunderscore )}
\end{itemize}
Espaço de um tríglypho e duas métopas, entre duas columnas de ordem dórica.
\section{Monótropa}
\begin{itemize}
\item {Grp. gram.:f.}
\end{itemize}
\begin{itemize}
\item {Proveniência:(Do gr. \textunderscore monos\textunderscore  + \textunderscore tropos\textunderscore )}
\end{itemize}
Gênero de plantas parasitas, que vivem nas raízes das árvores.
\section{Monotrópeas}
\begin{itemize}
\item {Grp. gram.:f. pl.}
\end{itemize}
\begin{itemize}
\item {Proveniência:(De \textunderscore monotrópeo\textunderscore )}
\end{itemize}
Família de plantas, que tem por typo a monótropa.
\section{Monotrópeo}
\begin{itemize}
\item {Grp. gram.:adj.}
\end{itemize}
Relativo ou semelhante á monótropa.
\section{Monótypo}
\begin{itemize}
\item {Grp. gram.:adj.}
\end{itemize}
\begin{itemize}
\item {Utilização:Bot.}
\end{itemize}
\begin{itemize}
\item {Proveniência:(De \textunderscore mono...\textunderscore  + \textunderscore typo\textunderscore )}
\end{itemize}
Que, considerado nas suas diversas espécies, apresenta insignificantes modificações.
Que abrange várias espécies, ligadas por muitas relações.
\section{Monóxilo}
\begin{itemize}
\item {fónica:csi}
\end{itemize}
\begin{itemize}
\item {Grp. gram.:adj.}
\end{itemize}
\begin{itemize}
\item {Grp. gram.:M.}
\end{itemize}
\begin{itemize}
\item {Proveniência:(Do gr. \textunderscore monos\textunderscore  + \textunderscore xulon\textunderscore )}
\end{itemize}
Feito de uma só peça de madeira.
Barco inteiriço, de uma só peça; piroga.
\section{Monóxylo}
\begin{itemize}
\item {fónica:csi}
\end{itemize}
\begin{itemize}
\item {Grp. gram.:adj.}
\end{itemize}
\begin{itemize}
\item {Grp. gram.:M.}
\end{itemize}
\begin{itemize}
\item {Proveniência:(Do gr. \textunderscore monos\textunderscore  + \textunderscore xulon\textunderscore )}
\end{itemize}
Feito de uma só peça de madeira.
Barco inteiriço, de uma só peça; piroga.
\section{Monozoicidade}
\begin{itemize}
\item {Grp. gram.:f.}
\end{itemize}
Qualidade de monozoico.
\section{Monozoico}
\begin{itemize}
\item {Grp. gram.:adj.}
\end{itemize}
\begin{itemize}
\item {Proveniência:(Do gr. \textunderscore monos\textunderscore  + \textunderscore zoon\textunderscore )}
\end{itemize}
Diz-se dos animaes, que têm vida individual e insulada.
\section{Monquilho}
\begin{itemize}
\item {Grp. gram.:m.}
\end{itemize}
\begin{itemize}
\item {Proveniência:(De \textunderscore monco\textunderscore )}
\end{itemize}
Moléstia do gado lanígero.
Esgana, doença de cães.
\section{Monquilho}
\begin{itemize}
\item {Grp. gram.:m.}
\end{itemize}
\begin{itemize}
\item {Utilização:Prov.}
\end{itemize}
\begin{itemize}
\item {Utilização:beir.}
\end{itemize}
O mesmo ou melhor que \textunderscore monoquilho\textunderscore .
\section{Monroeano}
\begin{itemize}
\item {Grp. gram.:adj.}
\end{itemize}
O mesmo que \textunderscore monroíno\textunderscore .
\section{Monroelatria}
\begin{itemize}
\item {Grp. gram.:f.}
\end{itemize}
Adhesão calorosa á doutrina de Monroe, ao monroísmo. Cf. Rui Barb., \textunderscore Cartas de Ingl.\textunderscore , 41.
\section{Monroíno}
\begin{itemize}
\item {Grp. gram.:adj.}
\end{itemize}
Relativo á doutrina de Monroe. Cf. Rui Barb., \textunderscore Cartas de Ingl.\textunderscore , 41.
\section{Monroísmo}
\begin{itemize}
\item {Grp. gram.:m.}
\end{itemize}
\begin{itemize}
\item {Proveniência:(De \textunderscore Monroe\textunderscore , n. p.)}
\end{itemize}
Doutrina dos que repellem a intervenção europeia em assumptos políticos da América.
\section{Monroísta}
\begin{itemize}
\item {Grp. gram.:m.}
\end{itemize}
Partidário do monroísmo.
\section{Monsenhor}
\begin{itemize}
\item {Grp. gram.:m.}
\end{itemize}
\begin{itemize}
\item {Utilização:Bras}
\end{itemize}
\begin{itemize}
\item {Proveniência:(It. \textunderscore monsignore\textunderscore )}
\end{itemize}
Título honorífico, que o Papa concede aos seus camareiros, a alguns prelados, e, fóra da Itália, a alguns ecclesiásticos.
Título, que nalgumas nações se dava aos príncipes e a outros indivíduos de procedência real.
O mesmo que \textunderscore chrysânthemo\textunderscore .
\section{Monsenhorado}
\begin{itemize}
\item {Grp. gram.:m.}
\end{itemize}
Dignidade de monsenhor.
\section{Monsior}
\begin{itemize}
\item {Grp. gram.:m.}
\end{itemize}
\begin{itemize}
\item {Utilização:Ant.}
\end{itemize}
\begin{itemize}
\item {Proveniência:(Fr. \textunderscore monsieur\textunderscore )}
\end{itemize}
Homem illustre.
\section{Monso}
\begin{itemize}
\item {Grp. gram.:adj.}
\end{itemize}
\begin{itemize}
\item {Utilização:Prov.}
\end{itemize}
\begin{itemize}
\item {Utilização:trasm.}
\end{itemize}
Indolente, songa.
\section{Monstera}
\begin{itemize}
\item {Grp. gram.:f.}
\end{itemize}
Gênero de plantas aráceas.
\section{Monstro}
\begin{itemize}
\item {Grp. gram.:m.}
\end{itemize}
\begin{itemize}
\item {Utilização:Ext.}
\end{itemize}
\begin{itemize}
\item {Utilização:Fig.}
\end{itemize}
\begin{itemize}
\item {Utilização:Fam.}
\end{itemize}
\begin{itemize}
\item {Proveniência:(Lat. \textunderscore monstrum\textunderscore )}
\end{itemize}
Corpo organizado, animal ou vegetal, que apresenta conformação anómala em todas ou em algumas das suas partes.
Sêr, imaginado pela Mythologia, de conformação extravagante, como o Dragão, o Minotauro, etc.
Animal de grandeza descommunal.
Figura colossal.
Aquillo que causa pavor.
Pessôa cruel, desnaturada, ou notável por qualquer vício, levado ao excesso.
Assombro, prodigio.
Mostrengo, pessôa inerte, indolente; estafermo.
\section{Monstruário}
\begin{itemize}
\item {Grp. gram.:m.}
\end{itemize}
(V.mostruário)
\section{Monstruosamente}
\begin{itemize}
\item {Grp. gram.:adv.}
\end{itemize}
De modo monstruoso.
\section{Monstruosidade}
\begin{itemize}
\item {Grp. gram.:f.}
\end{itemize}
Qualidade do que é monstruoso.
Coisa extraordinária.
Coisa abominável.
Assombro; monstro.
\section{Monstruoso}
\begin{itemize}
\item {Grp. gram.:adj.}
\end{itemize}
\begin{itemize}
\item {Proveniência:(Lat. \textunderscore monstruosus\textunderscore )}
\end{itemize}
Que tem a conformação de um monstro.
Contrário ás leis naturaes: \textunderscore paixões monstruosas\textunderscore .
Assombroso, prodigioso.
Enorme: \textunderscore riqueza monstruosa\textunderscore .
Que excede quanto se possa imaginar de mau: \textunderscore crime monstruoso\textunderscore .
Excessivamente feio.
\section{Monta}
\begin{itemize}
\item {Grp. gram.:f.}
\end{itemize}
\begin{itemize}
\item {Utilização:Ant.}
\end{itemize}
\begin{itemize}
\item {Proveniência:(De \textunderscore montar\textunderscore )}
\end{itemize}
Somma.
Importância.
Estimação.
Custo.
Lanço, nos leilões.
Quinhão, sorte.
\section{Monta-cargas}
\begin{itemize}
\item {Grp. gram.:m.}
\end{itemize}
Apparelho, para carregar peças de artilharia.
\section{Montada}
\begin{itemize}
\item {Grp. gram.:f.}
\end{itemize}
Acto de montar.
Elevação nas cambas do freio, para que a língua do cavallo passe facilmente debaixo delle.
Cavalgadura montada.
Uma pessôa, com outra a cavalleiro.
\section{Montádego}
\begin{itemize}
\item {Grp. gram.:m.}
\end{itemize}
\begin{itemize}
\item {Proveniência:(Do b. lat. \textunderscore montaticum\textunderscore )}
\end{itemize}
Imposto, que se pagava, por os gados pastarem nos montes de certos concelhos ou senhorios.
\section{Montádigo}
\begin{itemize}
\item {Grp. gram.:m.}
\end{itemize}
O mesmo ou melhor que \textunderscore montádego\textunderscore . Cf. S. R. Viterbo, \textunderscore Elucidário\textunderscore .
\section{Montado}
\begin{itemize}
\item {Grp. gram.:m.}
\end{itemize}
\begin{itemize}
\item {Grp. gram.:Adj.}
\end{itemize}
\begin{itemize}
\item {Utilização:Bras}
\end{itemize}
\begin{itemize}
\item {Proveniência:(Do b. lat. \textunderscore montatus\textunderscore )}
\end{itemize}
Terreno, em que crescem principalmente sobreiros ou azinheiros, e em que póde pastar gado suíno.
Aquillo que se paga ao dono de tal terreno, pela engorda de porcos que alli vão pastar.
Diz-se do animal amontado ou bravio.
\section{Montado}
\begin{itemize}
\item {Grp. gram.:adj.}
\end{itemize}
\begin{itemize}
\item {Proveniência:(De \textunderscore montar\textunderscore )}
\end{itemize}
Pôsto sôbre o cavallo.
Collocado, á maneira de cavalleiro: \textunderscore montado numa cana\textunderscore ; \textunderscore montado no muro\textunderscore .
\section{Montagem}
\begin{itemize}
\item {Grp. gram.:f.}
\end{itemize}
Acto ou effeito de montar.
Preparação das peças de um maquinismo, para que êste funccione.
\section{Montan-do-outono}
\begin{itemize}
\item {Grp. gram.:f.}
\end{itemize}
Planta ranunculácea.
\section{Montanha}
\begin{itemize}
\item {Grp. gram.:f.}
\end{itemize}
\begin{itemize}
\item {Proveniência:(Do lat. hypoth. \textunderscore montaneus\textunderscore )}
\end{itemize}
Série de montes.
Grande elevação de alguma coisa: \textunderscore uma montanha de papéis\textunderscore .
Grande volume.
\section{Montanhão}
\begin{itemize}
\item {Grp. gram.:m.}
\end{itemize}
\begin{itemize}
\item {Utilização:Prov.}
\end{itemize}
\begin{itemize}
\item {Utilização:minh.}
\end{itemize}
Camponês, que vive na montanha; montanheiro; serrano.
\section{Montanhaque}
\begin{itemize}
\item {Grp. gram.:m.}
\end{itemize}
\begin{itemize}
\item {Proveniência:(Fr. \textunderscore montagnac\textunderscore )}
\end{itemize}
Espécie de tecido de lan.
\section{Montanheira}
\begin{itemize}
\item {Grp. gram.:f.}
\end{itemize}
\begin{itemize}
\item {Proveniência:(De \textunderscore montanha\textunderscore )}
\end{itemize}
Montado.
Ceva de porcos, por meio de bolotas, num montado.
\section{Montanheiro}
\begin{itemize}
\item {Grp. gram.:m.}
\end{itemize}
\begin{itemize}
\item {Grp. gram.:Adj.}
\end{itemize}
\begin{itemize}
\item {Utilização:Prov.}
\end{itemize}
\begin{itemize}
\item {Utilização:alent.}
\end{itemize}
\begin{itemize}
\item {Proveniência:(De \textunderscore montanha\textunderscore )}
\end{itemize}
Camponês algarvio ou alentejano, que vive no monte ou casal da herdade.
Pastor da montanha.
O mesmo que \textunderscore montanhês\textunderscore .
Sertanejo.
Diz-se do porco, que começa a criar-se com a bolota que cai, de Outubro a Março ou Abril. Cf. Rev. \textunderscore Tradição\textunderscore , V, 146.
Diz-se dos bácoros, nascidos no verão. Cf. \textunderscore Port. au point de vue agr.\textunderscore , 308.
\section{Montanhento}
\begin{itemize}
\item {Grp. gram.:adj.}
\end{itemize}
Em que há montanhas. Cf. Pacheco, \textunderscore Promptuário\textunderscore .
\section{Montanhês}
\begin{itemize}
\item {Grp. gram.:adj.}
\end{itemize}
\begin{itemize}
\item {Grp. gram.:M.}
\end{itemize}
\begin{itemize}
\item {Proveniência:(De \textunderscore montanha\textunderscore )}
\end{itemize}
Que vive nas montanhas.
Relativo a quem vive nas montanhas.
Montanhoso.
Aquelle que vive nas montanhas.
\section{Montanhesco}
\begin{itemize}
\item {fónica:nhês}
\end{itemize}
\begin{itemize}
\item {Grp. gram.:adj.}
\end{itemize}
Relativo a montanha; silvestre; alpestre.
\section{Montanhoso}
\begin{itemize}
\item {Grp. gram.:adj.}
\end{itemize}
\begin{itemize}
\item {Proveniência:(De \textunderscore montanha\textunderscore )}
\end{itemize}
Em que há muitas montanhas: \textunderscore região montanhosa\textunderscore .
\section{Montanismo}
\begin{itemize}
\item {Grp. gram.:m.}
\end{itemize}
Seita dos Montanistas.
\section{Montanistas}
\begin{itemize}
\item {Grp. gram.:m. pl.}
\end{itemize}
\begin{itemize}
\item {Proveniência:(De \textunderscore Montano\textunderscore , n. p.)}
\end{itemize}
Herejes, que sustentavam serem sacrílegas e profanas as segundas núpcias.
\section{Montanística}
\begin{itemize}
\item {Grp. gram.:f.}
\end{itemize}
\begin{itemize}
\item {Proveniência:(De \textunderscore montanístico\textunderscore )}
\end{itemize}
Tratado sôbre a extracção e fusão dos metaes.
\section{Montanístico}
\begin{itemize}
\item {Grp. gram.:adj.}
\end{itemize}
\begin{itemize}
\item {Proveniência:(De \textunderscore montano\textunderscore )}
\end{itemize}
Relativo á extracção e fusão dos metaes.
\section{Montano}
\begin{itemize}
\item {Grp. gram.:adj.}
\end{itemize}
\begin{itemize}
\item {Proveniência:(Lat. \textunderscore montanus\textunderscore )}
\end{itemize}
Montanhês.
Montanhoso.
Rude.
\section{Montante}
\begin{itemize}
\item {Grp. gram.:m.}
\end{itemize}
\begin{itemize}
\item {Grp. gram.:Loc. adv.}
\end{itemize}
\begin{itemize}
\item {Grp. gram.:Adj.}
\end{itemize}
\begin{itemize}
\item {Proveniência:(De \textunderscore montar\textunderscore )}
\end{itemize}
Grande espada antiga, que se brandia com ambas as mãos.
Importância, somma.
Cada uma das hastes verticaes do estere.
\textunderscore A montante\textunderscore , para o lado da nascente (de um rio).
Que se eleva.
\section{Montão}
\begin{itemize}
\item {Grp. gram.:m.}
\end{itemize}
\begin{itemize}
\item {Proveniência:(De \textunderscore monte\textunderscore )}
\end{itemize}
Accumulação desordenada.
Conjunto de coisas empilhadas; acervo.
\section{Montar}
\begin{itemize}
\item {Grp. gram.:v. t.}
\end{itemize}
\begin{itemize}
\item {Grp. gram.:V. i.}
\end{itemize}
\begin{itemize}
\item {Utilização:Ant.}
\end{itemize}
\begin{itemize}
\item {Utilização:Ant.}
\end{itemize}
\begin{itemize}
\item {Grp. gram.:V. p.}
\end{itemize}
\begin{itemize}
\item {Proveniência:(Do b. lat. \textunderscore montare\textunderscore )}
\end{itemize}
Collocar-se sôbre (uma cavalgadura).
Collocar sôbre.
Fornecer o que é preciso a; estabelecer: \textunderscore montar uma fábrica\textunderscore .
Sêr capaz de conter ou abranger.
Pôr-se a cavallo.
Têr importância, importar: \textunderscore que monta isso\textunderscore ?
Dar lanço em leilão.
Importar ou valer; attingir.
Elevar-se a certa quantia ou valor: \textunderscore a despesa montou a déz contos\textunderscore .
Apascentar ou fazer lenha em montes.
Pôr-se a cavallo.
Collocar-se sôbre alguma coisa, como sôbre um cavallo.
\section{Montaraz}
\begin{itemize}
\item {Grp. gram.:adj.}
\end{itemize}
\begin{itemize}
\item {Grp. gram.:M.}
\end{itemize}
Montanhesco; silvestre.
Guarda de matas, coiteiro.
Chefe de boieiros, maioral de gados.
(Cast. \textunderscore montaraz\textunderscore )
\section{Montareco}
\begin{itemize}
\item {Grp. gram.:m.}
\end{itemize}
\begin{itemize}
\item {Utilização:Prov.}
\end{itemize}
\begin{itemize}
\item {Utilização:alent.}
\end{itemize}
Monte pequeno.
\section{Montaria}
\begin{itemize}
\item {Grp. gram.:f.}
\end{itemize}
\begin{itemize}
\item {Utilização:Fig.}
\end{itemize}
\begin{itemize}
\item {Proveniência:(Do b. lat. \textunderscore montaria\textunderscore )}
\end{itemize}
Lugar, onde se corre caça grossa.
Acto de correr essa caça.
Monteada.
Offício de monteiro.
Arte de caçar.
Agrupamento de monteiros.
Assuada; perseguição, feita por muita gente.
\section{Montaria}
\begin{itemize}
\item {Grp. gram.:f.}
\end{itemize}
\begin{itemize}
\item {Utilização:Bras}
\end{itemize}
\begin{itemize}
\item {Proveniência:(De \textunderscore montar\textunderscore )}
\end{itemize}
Provisão de cavallos para o exército; remonta.
O mesmo que \textunderscore cavalgadura\textunderscore .
\section{Montaria}
\begin{itemize}
\item {Grp. gram.:f.}
\end{itemize}
\begin{itemize}
\item {Utilização:Bras}
\end{itemize}
Canôa ligeira, construida de um só madeiro, com uma ou duas falcas de cada lado.
\section{Montarico}
\begin{itemize}
\item {Grp. gram.:m.}
\end{itemize}
\begin{itemize}
\item {Utilização:Prov.}
\end{itemize}
\begin{itemize}
\item {Utilização:alent.}
\end{itemize}
\begin{itemize}
\item {Proveniência:(De \textunderscore monte\textunderscore )}
\end{itemize}
Grupo de três a seis casitas baixas, habitadas por guardas, pastores ou caseiros.
\section{Montático}
\begin{itemize}
\item {Grp. gram.:m.}
\end{itemize}
O mesmo que \textunderscore montádigo\textunderscore . Cf. Herculano, \textunderscore Hist. de Port.\textunderscore , IV, 145 e 146.
\section{Monte}
\begin{itemize}
\item {Grp. gram.:m.}
\end{itemize}
\begin{itemize}
\item {Utilização:Prov.}
\end{itemize}
\begin{itemize}
\item {Utilização:alent.}
\end{itemize}
\begin{itemize}
\item {Grp. gram.:Loc. adv.}
\end{itemize}
\begin{itemize}
\item {Grp. gram.:Pl.}
\end{itemize}
\begin{itemize}
\item {Utilização:Ant.}
\end{itemize}
\begin{itemize}
\item {Proveniência:(Lat. \textunderscore mons\textunderscore , \textunderscore montis\textunderscore )}
\end{itemize}
Grande massa de terra e de rocha, elevada acima do terreno que a rodeia.
Porção.
Acervo, ajuntamento: \textunderscore um monte de sal\textunderscore .
Grande volume.
Conjunto dos bens de uma herança.
Quinhão de uma herança.
Espécie de jôgo de asar, também conhecido por \textunderscore jôgo de parar\textunderscore .
Montado.
Casal da herdade.
\textunderscore De monte a monte\textunderscore , diz-se da corrente, que se avolumou enormemente:«\textunderscore mas a torrente ia tão de monte a monte...\textunderscore »Camillo, \textunderscore Senhor do P. de Ninães\textunderscore , 164.
Montanha; cordilheira.
Caçada.
\section{Montéa}
\begin{itemize}
\item {Grp. gram.:f.}
\end{itemize}
\begin{itemize}
\item {Proveniência:(De \textunderscore monte\textunderscore )}
\end{itemize}
Esbôço ou planta de uma construcção, indicando as respectivas elevações.
Espaço, occupado por um edificio.
\section{Monteada}
\begin{itemize}
\item {Grp. gram.:f.}
\end{itemize}
O mesmo que \textunderscore montaria\textunderscore ^1, caçada nos montes.
Busca de caça grossa.
\section{Monteador}
\begin{itemize}
\item {Grp. gram.:m.}
\end{itemize}
\begin{itemize}
\item {Utilização:Prov.}
\end{itemize}
\begin{itemize}
\item {Utilização:minh.}
\end{itemize}
\begin{itemize}
\item {Proveniência:(De \textunderscore montear\textunderscore )}
\end{itemize}
Aquelle que caça nos montes.
Monteiro.
Caçador que, acompanhado de cães, bate o mato, para que a caça vá ter ás esperas.
\section{Montear}
\begin{itemize}
\item {Grp. gram.:v. t.}
\end{itemize}
\begin{itemize}
\item {Grp. gram.:V. i.}
\end{itemize}
\begin{itemize}
\item {Proveniência:(De \textunderscore monte\textunderscore )}
\end{itemize}
Caçar no monte.
Pôr em monte, amontoar.
Andar á caça nos montes.
\section{Montear}
\begin{itemize}
\item {Grp. gram.:v. t.}
\end{itemize}
\begin{itemize}
\item {Proveniência:(De \textunderscore monteia\textunderscore )}
\end{itemize}
Fazer a monteia de.
\section{Montearia}
\begin{itemize}
\item {Grp. gram.:f.}
\end{itemize}
O mesmo que \textunderscore montaria\textunderscore ^1.
\section{Monteia}
\begin{itemize}
\item {Grp. gram.:f.}
\end{itemize}
\begin{itemize}
\item {Proveniência:(De \textunderscore monte\textunderscore )}
\end{itemize}
Esbôço ou planta de uma construcção, indicando as respectivas elevações.
Espaço, occupado por um edificio.
\section{Monteira}
\begin{itemize}
\item {Grp. gram.:f.}
\end{itemize}
\begin{itemize}
\item {Utilização:Gír.}
\end{itemize}
\begin{itemize}
\item {Proveniência:(De \textunderscore monte\textunderscore )}
\end{itemize}
Mulher, que caça nos montes.
Variedade de carapuça de pano.
Prisão.
\section{Monteiria}
\begin{itemize}
\item {Grp. gram.:f.}
\end{itemize}
\begin{itemize}
\item {Proveniência:(De \textunderscore monteiro\textunderscore )}
\end{itemize}
Cargo de monteiro.
Parte, que pertencia aos monteiros, na multa que pagavam os que iam caçar ás coitadas.
\section{Monteiro}
\begin{itemize}
\item {Grp. gram.:adj.}
\end{itemize}
\begin{itemize}
\item {Grp. gram.:M.}
\end{itemize}
\begin{itemize}
\item {Proveniência:(Do b. lat. \textunderscore montarius\textunderscore )}
\end{itemize}
Relativo a monteiros.
Próprio para montear^1.
Aquelle que caça nos montes.
Guarda de mata, coiteiro.
\section{Montenegrino}
\begin{itemize}
\item {Grp. gram.:adj.}
\end{itemize}
\begin{itemize}
\item {Grp. gram.:M.}
\end{itemize}
Relativo a Montenegro.
Habitante de Montenegro.
\section{Monte-pio}
\begin{itemize}
\item {Grp. gram.:m.}
\end{itemize}
\begin{itemize}
\item {Proveniência:(De \textunderscore monte\textunderscore  + \textunderscore pio\textunderscore )}
\end{itemize}
Associação, em que cada membro, mediante uma prestação mensal e outras condições, adquire certos direitos, como o de sêr subsidiado em certos casos, deixar por morte pensão a sua família, etc.
\section{Monteria}
\begin{itemize}
\item {Grp. gram.:f.}
\end{itemize}
(V. \textunderscore montaria\textunderscore ^1)
\section{Montês}
\begin{itemize}
\item {Grp. gram.:adj.}
\end{itemize}
Que cresce ou vive nos montes.
Montanhês; bravío; rústico:«\textunderscore comi dessa fruta, amargosa, montesa.\textunderscore »G. Vicente, I, 317.--Não obstante êste exemplo, o adjectivo considera-se epiceno: \textunderscore gato montês\textunderscore , \textunderscore cabra montês\textunderscore .
\section{Montes-de-oiro}
\begin{itemize}
\item {Grp. gram.:m. pl.}
\end{itemize}
Planta, da tribo das heliântheas, (\textunderscore helianthus multiflorus\textunderscore , Lin.). Cf. P. Coutinho, \textunderscore Flora\textunderscore , 624.
\section{Montesinho}
\begin{itemize}
\item {Grp. gram.:adj.}
\end{itemize}
Montanhês; montês; silvestre.
(Cp. \textunderscore montesino\textunderscore )
\section{Montesino}
\begin{itemize}
\item {Grp. gram.:adj.}
\end{itemize}
\begin{itemize}
\item {Proveniência:(Do b. lat. \textunderscore montesinus\textunderscore )}
\end{itemize}
O mesmo que \textunderscore montesinho\textunderscore .
\section{Monteu}
\begin{itemize}
\item {Grp. gram.:m.}
\end{itemize}
Espécie de magistrado judicial, na China. Cf. \textunderscore Peregrinação\textunderscore , CIII.
\section{Montevideano}
\begin{itemize}
\item {Grp. gram.:adj.}
\end{itemize}
\begin{itemize}
\item {Grp. gram.:M.}
\end{itemize}
Relativo a Montevideu.
Habitante de Montevideu.
\section{Montícola}
\begin{itemize}
\item {Grp. gram.:adj.}
\end{itemize}
\begin{itemize}
\item {Proveniência:(Lat. \textunderscore monticola\textunderscore )}
\end{itemize}
Que vive ou cresce nos montes.
\section{Montículo}
\begin{itemize}
\item {Grp. gram.:m.}
\end{itemize}
\begin{itemize}
\item {Proveniência:(Lat. \textunderscore monticulus\textunderscore )}
\end{itemize}
Pequeno monte; oiteiro; cômoro.
\section{Montígeno}
\begin{itemize}
\item {Grp. gram.:adj.}
\end{itemize}
\begin{itemize}
\item {Proveniência:(De \textunderscore monte\textunderscore  + gr. \textunderscore genes\textunderscore )}
\end{itemize}
Produzido nos montes.
\section{Montijo}
\begin{itemize}
\item {Grp. gram.:m.}
\end{itemize}
\begin{itemize}
\item {Utilização:Prov.}
\end{itemize}
\begin{itemize}
\item {Utilização:alent.}
\end{itemize}
\begin{itemize}
\item {Proveniência:(De \textunderscore monte\textunderscore )}
\end{itemize}
Montículo em fórma de pyrâmide cónica.
\section{Montilhão}
\begin{itemize}
\item {Grp. gram.:m.}
\end{itemize}
\begin{itemize}
\item {Utilização:T. de Esposende}
\end{itemize}
\begin{itemize}
\item {Proveniência:(De um hyp. \textunderscore montilho\textunderscore  = \textunderscore montijo\textunderscore )}
\end{itemize}
O mesmo que \textunderscore mamôa\textunderscore .
\section{Montíneas}
\begin{itemize}
\item {Grp. gram.:f. pl.}
\end{itemize}
Tríbo de plantas onagrárias. Cf. De-Candolle.
\section{Montívago}
\begin{itemize}
\item {Grp. gram.:adj.}
\end{itemize}
\begin{itemize}
\item {Proveniência:(Lat. \textunderscore montivagus\textunderscore )}
\end{itemize}
Que vagueia pelos montes.
\section{Montoso}
\begin{itemize}
\item {Grp. gram.:adj.}
\end{itemize}
O mesmo que \textunderscore montuoso\textunderscore . Cf. Fern. Pereira, \textunderscore Caça de Altan.\textunderscore , parte II, cp. I.
\section{Montra}
\begin{itemize}
\item {Grp. gram.:f.}
\end{itemize}
\begin{itemize}
\item {Proveniência:(Fr. \textunderscore montre\textunderscore )}
\end{itemize}
Gallicismo inútil, com que se designa a vitrina de estabelecimento commercial, o mostrador, e a caixa com tampo de vidro, dentro da qual se expõem á vista mercadorias.
V. \textunderscore mostruário\textunderscore .
\section{Montuoso}
\begin{itemize}
\item {Grp. gram.:adj.}
\end{itemize}
\begin{itemize}
\item {Utilização:Fig.}
\end{itemize}
\begin{itemize}
\item {Proveniência:(Lat. \textunderscore montuosus\textunderscore )}
\end{itemize}
O mesmo que \textunderscore montanhoso\textunderscore .
Que não é plano, que tem superfície accidentada.
\section{Montureira}
\begin{itemize}
\item {Grp. gram.:f.}
\end{itemize}
O mesmo que \textunderscore monturo\textunderscore .
\section{Montureiro}
\begin{itemize}
\item {Grp. gram.:m.}
\end{itemize}
\begin{itemize}
\item {Proveniência:(De \textunderscore monturo\textunderscore )}
\end{itemize}
Aquelle que busca nos monturos quaesquer objectos, de que possa tirar proveito.
Gandaieiro; trapeiro.
\section{Monturo}
\begin{itemize}
\item {Grp. gram.:m.}
\end{itemize}
\begin{itemize}
\item {Utilização:Fig.}
\end{itemize}
\begin{itemize}
\item {Proveniência:(De \textunderscore monte\textunderscore )}
\end{itemize}
Logar, onde se depositam dejecções ou immundicies.
Acervo de lixo, de estêrco, etc.
Montão de coisas repugnantes ou vis.
\section{Monumental}
\begin{itemize}
\item {Grp. gram.:adj.}
\end{itemize}
\begin{itemize}
\item {Proveniência:(Lat. \textunderscore monumentalis\textunderscore )}
\end{itemize}
Relativo a monumento.
Grandioso, magnifico: \textunderscore poema monumental\textunderscore .
Extraordinário, enorme: \textunderscore uma enchente monumental\textunderscore .
\section{Monumentalizar}
\begin{itemize}
\item {Grp. gram.:v. t.}
\end{itemize}
\begin{itemize}
\item {Utilização:Neol.}
\end{itemize}
Dar carácter monumental a. Cf. Alves Mendes, \textunderscore Discursos\textunderscore , 100.
\section{Monumentalmente}
\begin{itemize}
\item {Grp. gram.:adv.}
\end{itemize}
\begin{itemize}
\item {Proveniência:(De \textunderscore monumental\textunderscore )}
\end{itemize}
Por meio de monumentos.
\section{Monumento}
\begin{itemize}
\item {Grp. gram.:m.}
\end{itemize}
\begin{itemize}
\item {Proveniência:(Lat. \textunderscore monumentum\textunderscore )}
\end{itemize}
Obra ou construcção, feita para trasm.ttir á posteridade a memória de facto ou personagem notável: \textunderscore o monumento da Praça dos Restauradores\textunderscore .
Edifício, admirável por sua construcção, antiguidade ou valia dos factos que relembra: \textunderscore o monumento dos Jerónimos em Belém\textunderscore .
Mausoléu.
Obra notável: \textunderscore a Divina Comédia é um monumento\textunderscore .
Recordação, memória.
\section{Monumentoso}
\begin{itemize}
\item {Grp. gram.:adj.}
\end{itemize}
\begin{itemize}
\item {Utilização:P. us.}
\end{itemize}
O mesmo que \textunderscore monumental\textunderscore .
\section{Monvanas}
\begin{itemize}
\item {Grp. gram.:m. pl.}
\end{itemize}
Indígenas do norte do Brasil.
\section{Monvedro}
\begin{itemize}
\item {Grp. gram.:m.}
\end{itemize}
O mesmo que \textunderscore bom-vedro\textunderscore .
\section{Moónia}
\begin{itemize}
\item {Grp. gram.:f.}
\end{itemize}
Gênero de plantas, da fam. das compostas.
\section{Mopani}
\begin{itemize}
\item {Grp. gram.:m.}
\end{itemize}
Árvore resinosa da África do Sul.
\section{Moplas}
\begin{itemize}
\item {Grp. gram.:m. pl.}
\end{itemize}
Islamitas do Malabar, de raça malaia.
\section{Moponga}
\begin{itemize}
\item {Grp. gram.:f.}
\end{itemize}
\begin{itemize}
\item {Utilização:Bras. do N}
\end{itemize}
Modo de pescar, batendo a água com os braços, para que o peixe remonte o rio até o lugar da rêde.
\section{Moque}
\begin{itemize}
\item {Grp. gram.:m.}
\end{itemize}
Tríbuto, que os Moiros pagavam em Portugal e que abrangia a quarentena dos frutos.
\section{Moquear}
\begin{itemize}
\item {Grp. gram.:v. t.}
\end{itemize}
\begin{itemize}
\item {Utilização:Bras}
\end{itemize}
\begin{itemize}
\item {Proveniência:(De \textunderscore moquém\textunderscore )}
\end{itemize}
Secar (carne) sôbre o moquém.
Passar pelo fogo (a carne), para se não damnificar.
\section{Moqueca}
\begin{itemize}
\item {Grp. gram.:f.}
\end{itemize}
\begin{itemize}
\item {Utilização:Bras}
\end{itemize}
\begin{itemize}
\item {Proveniência:(Do guar. \textunderscore poque\textunderscore  + \textunderscore mboqué\textunderscore )}
\end{itemize}
Guisado de peixe ou marisco, temperado com côco.
\section{Moquém}
\begin{itemize}
\item {Grp. gram.:m.}
\end{itemize}
\begin{itemize}
\item {Utilização:Bras}
\end{itemize}
\begin{itemize}
\item {Proveniência:(Do guar. \textunderscore mocae\textunderscore )}
\end{itemize}
Grade alta, em que se moqueia a carne.
Grelha.
Utensílio, com que se assa alguma coisa.
\section{Moquenca}
\begin{itemize}
\item {Grp. gram.:f.}
\end{itemize}
\begin{itemize}
\item {Utilização:Prov.}
\end{itemize}
\begin{itemize}
\item {Utilização:alg.}
\end{itemize}
Guisado de carne de vaca, com vinagre, alhos, pimenta, etc.
Haveres, cabedal.
(Relaciona-se com \textunderscore moqueca\textunderscore ?)
\section{Moquenco}
\begin{itemize}
\item {Grp. gram.:m.  e  adj.}
\end{itemize}
\begin{itemize}
\item {Utilização:Chul.}
\end{itemize}
Aquelle que faz moquenquices.
Indolente, preguiçoso.
(Cast. \textunderscore macuenco\textunderscore . Cp. \textunderscore moganguice\textunderscore )
\section{Moquenqueiro}
\begin{itemize}
\item {Grp. gram.:m.  e  adj.}
\end{itemize}
O mesmo que \textunderscore moquenco\textunderscore .
\section{Moquenquice}
\begin{itemize}
\item {Grp. gram.:f.}
\end{itemize}
\begin{itemize}
\item {Proveniência:(De \textunderscore moquenco\textunderscore )}
\end{itemize}
Momice, lábia.
Preguiça.
\section{Moqueta}
\begin{itemize}
\item {fónica:quê}
\end{itemize}
\begin{itemize}
\item {Grp. gram.:f.}
\end{itemize}
\begin{itemize}
\item {Proveniência:(Fr. \textunderscore moquette\textunderscore )}
\end{itemize}
Tecido de lan, fabricado em França, e próprio para estofos e alcatifas.
\section{Moquideira}
\begin{itemize}
\item {Grp. gram.:f.}
\end{itemize}
\begin{itemize}
\item {Utilização:Gír.}
\end{itemize}
\begin{itemize}
\item {Proveniência:(De \textunderscore moquir\textunderscore )}
\end{itemize}
Bôca.
\section{Moquilho}
\begin{itemize}
\item {Grp. gram.:m.}
\end{itemize}
\begin{itemize}
\item {Utilização:Veter.}
\end{itemize}
Esgana dos cães, o mesmo que \textunderscore monquilho\textunderscore ^1.
\section{Moquínia}
\begin{itemize}
\item {Grp. gram.:f.}
\end{itemize}
Gênero de plantas, da fam. das compostas.
\section{Moquir}
\begin{itemize}
\item {Grp. gram.:v. i.}
\end{itemize}
\begin{itemize}
\item {Utilização:Gír.}
\end{itemize}
Comer.
\section{Mór}
\begin{itemize}
\item {Grp. gram.:adj.}
\end{itemize}
O mesmo que \textunderscore maiór\textunderscore .
(Contr. \textunderscore maór\textunderscore , fórma primitiva de \textunderscore maior\textunderscore )
\section{Mora}
\begin{itemize}
\item {Grp. gram.:f.}
\end{itemize}
\begin{itemize}
\item {Proveniência:(Lat. \textunderscore mora\textunderscore )}
\end{itemize}
Demora, delonga.
Alargamento de um prazo, para se restituír ou pagar alguma coisa.
\section{Mora}
\begin{itemize}
\item {Grp. gram.:f.}
\end{itemize}
\begin{itemize}
\item {Utilização:Pop.}
\end{itemize}
\begin{itemize}
\item {Proveniência:(Lat. \textunderscore mora\textunderscore , pl. de \textunderscore morum\textunderscore )}
\end{itemize}
O mesmo que \textunderscore amora\textunderscore .
\section{Morabitinada}
\begin{itemize}
\item {Grp. gram.:f.}
\end{itemize}
\begin{itemize}
\item {Proveniência:(De \textunderscore morabitino\textunderscore )}
\end{itemize}
Porção de morabitinos.
Antiga medida de cereaes.
\section{Morabitino}
\begin{itemize}
\item {Grp. gram.:m.}
\end{itemize}
O mesmo que \textunderscore maravedi\textunderscore . Cf. Herculano, \textunderscore Hist. de Port.\textunderscore , I, 26 e 513; III, 325 e 330.
\section{Morabito}
\begin{itemize}
\item {Grp. gram.:m.}
\end{itemize}
Asceta mahometano.
(Fórma preferível a \textunderscore marabuto\textunderscore )
\section{Moráceas}
\begin{itemize}
\item {Grp. gram.:f. pl.}
\end{itemize}
O mesmo que \textunderscore moreáceas\textunderscore .
\section{Morada}
\begin{itemize}
\item {Grp. gram.:f.}
\end{itemize}
\begin{itemize}
\item {Utilização:Fig.}
\end{itemize}
\begin{itemize}
\item {Proveniência:(De \textunderscore morar\textunderscore )}
\end{itemize}
Lugar, onde se mora.
Casa de habitação.
Habitação, domicílio.
Estada.
Lugar, em que uma coisa está habitualmente.
\section{Moradia}
\begin{itemize}
\item {Grp. gram.:f.}
\end{itemize}
\begin{itemize}
\item {Proveniência:(Do b. lat. \textunderscore moratela\textunderscore )}
\end{itemize}
O mesmo que \textunderscore morada\textunderscore .
Certa pensão, que se dava aos fidalgos e ainda hoje a alguns empregados menores, para despesas de habitação.
\section{Moradilho}
\begin{itemize}
\item {Grp. gram.:m.}
\end{itemize}
\begin{itemize}
\item {Proveniência:(De \textunderscore morado\textunderscore )}
\end{itemize}
Qualidade de madeira de côr pardo-violeta.
\section{Morado}
\begin{itemize}
\item {Grp. gram.:adj.}
\end{itemize}
\begin{itemize}
\item {Proveniência:(De \textunderscore mora\textunderscore ^2)}
\end{itemize}
Que é da côr da amora. Cf. Pant. de Aveiro, \textunderscore Itiner.\textunderscore , 76 v.^o, (2.^a ed.).
\section{Morador}
\begin{itemize}
\item {Grp. gram.:adj.}
\end{itemize}
\begin{itemize}
\item {Grp. gram.:M.}
\end{itemize}
\begin{itemize}
\item {Proveniência:(Lat. \textunderscore morator\textunderscore )}
\end{itemize}
Que mora.
Aquelle que mora.
Habitante.
Vizinho; inquilino.
\section{Moraís}
\begin{itemize}
\item {Grp. gram.:m.}
\end{itemize}
Medida de capacidade, usada na Índia.
Mês que, no calendário árabe, corresponde ao nosso Agosto.
\section{Moral}
\begin{itemize}
\item {Grp. gram.:adj.}
\end{itemize}
\begin{itemize}
\item {Grp. gram.:F.}
\end{itemize}
\begin{itemize}
\item {Grp. gram.:M.}
\end{itemize}
\begin{itemize}
\item {Proveniência:(Lat. \textunderscore moralis\textunderscore )}
\end{itemize}
Relativo aos bons costumes: \textunderscore comportamento moral\textunderscore .
Que tem bons costumes.
Relativo ao domínio da alma ou da intelligência, (em opposição a phýsico ou material).
Parte da Philosophia, que trata dos costumes ou dos deveres do homem para com os seus semelhantes e para consigo.
Conjunto das nossas faculdades moraes.
O que há de moralidade em qualquer coisa.
\section{Moralidade}
\begin{itemize}
\item {Grp. gram.:f.}
\end{itemize}
\begin{itemize}
\item {Proveniência:(Lat. \textunderscore moralitas\textunderscore )}
\end{itemize}
Qualidade daquillo que é moral.
Doutrina moral.
Reflexão moral.
O intuito moral de certas fábulas ou narrativas.
\section{Moralismo}
\begin{itemize}
\item {Grp. gram.:m.}
\end{itemize}
Systema philosóphico, que trata exclusivamente da moral.
\section{Moralista}
\begin{itemize}
\item {Grp. gram.:m.  e  adj.}
\end{itemize}
Aquelle que escreve sôbre moral.
Aquelle que preconiza preceitos moraes.
\section{Moralização}
\begin{itemize}
\item {Grp. gram.:f.}
\end{itemize}
Acto ou effeito de moralizar.
\section{Moralizador}
\begin{itemize}
\item {Grp. gram.:adj.}
\end{itemize}
\begin{itemize}
\item {Grp. gram.:M.}
\end{itemize}
Que moraliza.
Que contribue para os bons costumes.
Que encerra ou preconiza doutrinas sans: \textunderscore livro moralizador\textunderscore .
Que dá bons exemplos: \textunderscore costumes moralizadores\textunderscore .
Aquelle que moraliza.
Aquelle que aconselha os bons costumes.
\section{Moralizar}
\begin{itemize}
\item {Grp. gram.:v. t.}
\end{itemize}
\begin{itemize}
\item {Grp. gram.:V. i.}
\end{itemize}
Tornar conforme aos princípios da moral.
Corrigir.
Infundir ideias sans em.
Interpretar moralmente: \textunderscore moralizar um discurso\textunderscore .
Fazer reflexões moraes.
\section{Moralmente}
\begin{itemize}
\item {Grp. gram.:adv.}
\end{itemize}
De modo moral.
De acôrdo com a moral.
Em sentido espiritual.
Relativamente ao domínio da alma ou da intelligência, (por opposição a \textunderscore physicamente\textunderscore  ou \textunderscore materialmente\textunderscore ).
Quási absolutamente; muito provavelmente: \textunderscore isso é moralmente impossivel\textunderscore .
\section{Moranga}
\begin{itemize}
\item {Grp. gram.:f.}
\end{itemize}
\begin{itemize}
\item {Utilização:Bras}
\end{itemize}
Planta medicinal.
Morango bravo.
(Cp. \textunderscore morango\textunderscore )
\section{Morangal}
\begin{itemize}
\item {Grp. gram.:m.}
\end{itemize}
\begin{itemize}
\item {Proveniência:(De \textunderscore morango\textunderscore )}
\end{itemize}
Terreno, onde crescem morangueiros.
\section{Morangar}
\begin{itemize}
\item {Grp. gram.:v. i.}
\end{itemize}
\begin{itemize}
\item {Utilização:Prov.}
\end{itemize}
\begin{itemize}
\item {Utilização:trasm.}
\end{itemize}
Trabalhar pouco, fingindo que faz alguma coisa.
(Provavelmente por \textunderscore mollengar\textunderscore , de \textunderscore mollenga\textunderscore )
\section{Morango}
\begin{itemize}
\item {Grp. gram.:m.}
\end{itemize}
\begin{itemize}
\item {Grp. gram.:Adj.}
\end{itemize}
\begin{itemize}
\item {Proveniência:(Do lat. hyp. \textunderscore moranicus\textunderscore , de \textunderscore mora\textunderscore , amora)}
\end{itemize}
Fruto dos morangueiros, semelhante á amora.
Morangueiro.
Diz-se de uma variedade de abóbora.
\section{Morangueiro}
\begin{itemize}
\item {Grp. gram.:m.}
\end{itemize}
Planta rosácea, (\textunderscore fragaria vesca\textunderscore ).
Vendedor de morangos.
\section{Morangueiro}
\begin{itemize}
\item {Grp. gram.:m.}
\end{itemize}
\begin{itemize}
\item {Utilização:Prov.}
\end{itemize}
\begin{itemize}
\item {Utilização:trasm.}
\end{itemize}
Aquelle que passa a vida a morangar.
\section{Morão}
\begin{itemize}
\item {Grp. gram.:m.}
\end{itemize}
\begin{itemize}
\item {Proveniência:(Do lat. \textunderscore morum\textunderscore . Cp. \textunderscore moreto\textunderscore ^1)}
\end{itemize}
Casta de uva preta do Algarve.
\section{Morar}
\begin{itemize}
\item {Grp. gram.:v. i.}
\end{itemize}
\begin{itemize}
\item {Utilização:Fig.}
\end{itemize}
\begin{itemize}
\item {Utilização:Prov.}
\end{itemize}
\begin{itemize}
\item {Utilização:alent.}
\end{itemize}
\begin{itemize}
\item {Proveniência:(Do b. lat. \textunderscore morare\textunderscore )}
\end{itemize}
Habitar, residir.
Permanecer, achar-se, estar.
Sêr criado ou criada, estar ao serviço effectivo de alguém.
\section{Morar}
\begin{itemize}
\item {Grp. gram.:v. i.}
\end{itemize}
\begin{itemize}
\item {Utilização:Prov.}
\end{itemize}
\begin{itemize}
\item {Utilização:trasm.}
\end{itemize}
Brincar.
(Por \textunderscore amorar\textunderscore , de \textunderscore amôr\textunderscore ?)
\section{Morato}
\begin{itemize}
\item {Grp. gram.:adj.}
\end{itemize}
\begin{itemize}
\item {Proveniência:(Lat. \textunderscore moratus\textunderscore )}
\end{itemize}
Bem organizado.
Em que se descrevem bem os caracteres. Cf. Filinto, XVIII, 94.
\section{Moratória}
\begin{itemize}
\item {Grp. gram.:f.}
\end{itemize}
\begin{itemize}
\item {Proveniência:(De \textunderscore moratório\textunderscore )}
\end{itemize}
Dilação, concedida pelo crèdor ao devedor, para pagamento da dívida.
\section{Moratório}
\begin{itemize}
\item {Grp. gram.:adj.}
\end{itemize}
\begin{itemize}
\item {Proveniência:(Lat. \textunderscore moratórius\textunderscore )}
\end{itemize}
Que envolve demora ou dilação; dilatório.
\section{Morávios}
\begin{itemize}
\item {Grp. gram.:m. pl.}
\end{itemize}
\begin{itemize}
\item {Proveniência:(De \textunderscore Morávia\textunderscore , n. p.)}
\end{itemize}
Seita christan, que rejeita o Purgatório e a adoração dos santos, e prega a absoluta fraternidade dos homens, sem distincção de raças.
Também se dizem \textunderscore irmãos morários\textunderscore .
\section{Morbidez}
\begin{itemize}
\item {Grp. gram.:f.}
\end{itemize}
Estado daquillo que é mórbido.
Enfraquecimento doentio.
Quebramento de fôrças.
Languidez.
Delicadeza ou suavidade nas côres de um retrato ou de uma esculptura.
\section{Morbideza}
\begin{itemize}
\item {Grp. gram.:f.}
\end{itemize}
O mesmo que \textunderscore morbidez\textunderscore , mas pouco vernáculo.
\section{Mórbido}
\begin{itemize}
\item {Grp. gram.:adj.}
\end{itemize}
\begin{itemize}
\item {Proveniência:(Lat. \textunderscore morbidus\textunderscore )}
\end{itemize}
Enfermo.
Relativo a doença.
Enfermiço.
Doentio.
Lânguido; froixo.
Enervante.
Em pintura ou esculptura, delicado, suave.
\section{Morbífico}
\begin{itemize}
\item {Grp. gram.:adj.}
\end{itemize}
\begin{itemize}
\item {Proveniência:(Lat. \textunderscore morbificus\textunderscore )}
\end{itemize}
Que causa doença.
Insalubre.
\section{Morbígeno}
\begin{itemize}
\item {Grp. gram.:adj.}
\end{itemize}
O mesmo que \textunderscore morbígero\textunderscore .
\section{Morbígero}
\begin{itemize}
\item {Grp. gram.:adj.}
\end{itemize}
\begin{itemize}
\item {Proveniência:(Do lat. \textunderscore morbus\textunderscore  + \textunderscore gerere\textunderscore )}
\end{itemize}
O mesmo que \textunderscore morbífico\textunderscore .
\section{Morbilidade}
\begin{itemize}
\item {Grp. gram.:f.}
\end{itemize}
\begin{itemize}
\item {Utilização:Neol.}
\end{itemize}
Relação entre os casos de moléstia e o número de habitantes de uma dada agglomeração humana. Cf. R. Jorge, \textunderscore Censo dos Tuberculosos\textunderscore .
(Cp. it. \textunderscore morbilitá\textunderscore )
\section{Morbiparo}
\begin{itemize}
\item {Grp. gram.:adj.}
\end{itemize}
\begin{itemize}
\item {Proveniência:(Do lat. \textunderscore morbus\textunderscore  + \textunderscore parere\textunderscore )}
\end{itemize}
Que produz doenças.
\section{Morbo}
\begin{itemize}
\item {Grp. gram.:m.}
\end{itemize}
\begin{itemize}
\item {Proveniência:(Lat. \textunderscore morbus\textunderscore )}
\end{itemize}
Doença; estado pathológico.
\section{Morbosidade}
\begin{itemize}
\item {Grp. gram.:f.}
\end{itemize}
Qualidade de morboso.
\section{Morboso}
\begin{itemize}
\item {Grp. gram.:adj.}
\end{itemize}
\begin{itemize}
\item {Proveniência:(Lat. \textunderscore morbosus\textunderscore )}
\end{itemize}
Mórbido; morbífico.
\section{Mórbus}
\begin{itemize}
\item {Grp. gram.:m.}
\end{itemize}
(V.morbo)
\section{Mórca}
\begin{itemize}
\item {Grp. gram.:f.}
\end{itemize}
Pequeno peixe do rio Minho.
Rêde de malha miúda, prohibida pelos regulamentos da pesca.
\section{Môrca}
\begin{itemize}
\item {Grp. gram.:f.}
\end{itemize}
\begin{itemize}
\item {Utilização:Prov.}
\end{itemize}
\begin{itemize}
\item {Utilização:minh.}
\end{itemize}
Lagarta esverdeada, que se cria nas couves.
\section{Morcão}
\begin{itemize}
\item {Grp. gram.:m.}
\end{itemize}
\begin{itemize}
\item {Utilização:Prov.}
\end{itemize}
\begin{itemize}
\item {Utilização:minh.}
\end{itemize}
Grande môrca.
\section{Morcão}
\begin{itemize}
\item {Grp. gram.:m.}
\end{itemize}
Indivíduo indolente e taciturno, inhenho ou aparvalhado.
\section{Môrcas}
\begin{itemize}
\item {Grp. gram.:m.}
\end{itemize}
\begin{itemize}
\item {Utilização:Pop.}
\end{itemize}
Indivíduo indolente e taciturno, inhenho ou aparvalhado.
\section{Morcegal}
\begin{itemize}
\item {Grp. gram.:adj.}
\end{itemize}
Relativo a morcego, próprio de morcego:«\textunderscore ...da morcegal caterva.\textunderscore »Macedo, \textunderscore Burros\textunderscore , 315.
\section{Morcegar}
\begin{itemize}
\item {Grp. gram.:v. t.}
\end{itemize}
\begin{itemize}
\item {Utilização:Bras}
\end{itemize}
\begin{itemize}
\item {Utilização:chul. de Pernambuco.}
\end{itemize}
Subir e descer de um combóio ou bonde, em movimento.
\section{Morcego}
\begin{itemize}
\item {fónica:cê}
\end{itemize}
\begin{itemize}
\item {Grp. gram.:m.}
\end{itemize}
\begin{itemize}
\item {Utilização:Burl.}
\end{itemize}
\begin{itemize}
\item {Utilização:burl.}
\end{itemize}
\begin{itemize}
\item {Utilização:Ant.}
\end{itemize}
\begin{itemize}
\item {Proveniência:(Do lat. \textunderscore mus\textunderscore , \textunderscore muris\textunderscore  + \textunderscore coecus\textunderscore )}
\end{itemize}
Gênero de mammíferos nocturnos, (\textunderscore vespertilio\textunderscore ).
Pessôa, que só de noite sái de casa.
Soldado de ronda.
Peixe de Portugal.
\section{Morcegueiro}
\begin{itemize}
\item {Grp. gram.:m.}
\end{itemize}
Árvore rutácea da Índia portuguesa, (\textunderscore ficus asiela\textunderscore , Roxb.).
\section{Morcela}
\begin{itemize}
\item {Grp. gram.:f.}
\end{itemize}
Espécie de chouriço, em que entra como elemento principal o sangue de porco.
Chouriço doce.
(Cp. \textunderscore morcilha\textunderscore )
\section{Morcella}
\begin{itemize}
\item {Grp. gram.:f.}
\end{itemize}
Espécie de chouriço, em que entra como elemento principal o sangue de porco.
Chouriço doce.
(Cp. \textunderscore morcilha\textunderscore )
\section{Morchão}
\begin{itemize}
\item {Grp. gram.:m.}
\end{itemize}
\begin{itemize}
\item {Utilização:Prov.}
\end{itemize}
\begin{itemize}
\item {Utilização:trasm.}
\end{itemize}
O mesmo que \textunderscore muchão\textunderscore .
\section{Morchetar}
\begin{itemize}
\item {Grp. gram.:v. t.}
\end{itemize}
\begin{itemize}
\item {Utilização:Prov.}
\end{itemize}
\begin{itemize}
\item {Utilização:trasm.}
\end{itemize}
Depenicar; belliscar.
\section{Morcilha}
\begin{itemize}
\item {Grp. gram.:f.}
\end{itemize}
\begin{itemize}
\item {Utilização:açor}
\end{itemize}
\begin{itemize}
\item {Utilização:Bras. do S}
\end{itemize}
O mesmo que \textunderscore morcella\textunderscore .
(Cast. \textunderscore morcilla\textunderscore )
\section{Morcos-diabos}
\begin{itemize}
\item {Grp. gram.:f.}
\end{itemize}
Planta da serra de Sintra.
(Alter. de \textunderscore mór\textunderscore  + \textunderscore co'os\textunderscore  + \textunderscore diabos\textunderscore ?)
\section{Mordaça}
\begin{itemize}
\item {Grp. gram.:f.}
\end{itemize}
\begin{itemize}
\item {Utilização:Fig.}
\end{itemize}
Objecto, com que se tapa a bôca de alguém, para que não fale nem grite.
Açamo.
Repressão de liberdade de falar ou de escrever: \textunderscore aquillo não é uma lei de imprensa, é uma mordaça\textunderscore .
(Cast. \textunderscore mordaza\textunderscore )
\section{Mordacidade}
\begin{itemize}
\item {Grp. gram.:f.}
\end{itemize}
\begin{itemize}
\item {Proveniência:(Lat. \textunderscore mordacitas\textunderscore )}
\end{itemize}
Qualidade de mordaz.
Propriedade de corrosivo.
Sabôr picante.
Maledicência.
Qualidade de crítico exaggeradamente severo ou injusto.
\section{Mordanga}
\begin{itemize}
\item {Grp. gram.:f.}
\end{itemize}
O mesmo que \textunderscore mordango\textunderscore .
\section{Mordango}
\begin{itemize}
\item {Grp. gram.:m.}
\end{itemize}
Espécie de tamboril, na Índia portuguesa. Cf. Th. Ribeiro, \textunderscore Jornadas\textunderscore , II, 208.
(Do conc.)
\section{Mordangueiro}
\begin{itemize}
\item {Grp. gram.:m.}
\end{itemize}
Tocador de mordango.
\section{Mordângui}
\begin{itemize}
\item {Grp. gram.:m.}
\end{itemize}
O mesmo que \textunderscore mordango\textunderscore . Cf. Th. Ribeiro, \textunderscore Jornadas\textunderscore , II, 102.
\section{Mordaz}
\begin{itemize}
\item {Grp. gram.:adj.}
\end{itemize}
\begin{itemize}
\item {Proveniência:(Lat. \textunderscore mordax\textunderscore )}
\end{itemize}
Que morde.
Que corrói.
Picante.
Pungente.
Satýrico; maledicente.
\section{Mordazmente}
\begin{itemize}
\item {Grp. gram.:adv.}
\end{itemize}
De modo mordaz; com má língua; com azedume; acrimoniosamente.
\section{Mordedela}
\begin{itemize}
\item {Grp. gram.:f.}
\end{itemize}
O mesmo que \textunderscore mordedura\textunderscore .
\section{Mordedoiro}
\begin{itemize}
\item {Grp. gram.:m.}
\end{itemize}
\begin{itemize}
\item {Utilização:Náut.}
\end{itemize}
\begin{itemize}
\item {Proveniência:(De \textunderscore morder\textunderscore )}
\end{itemize}
Apparelho, para suster a amarra da âncora, quando corre pelo escovém.
\section{Mordedor}
\begin{itemize}
\item {Grp. gram.:m.  e  adj.}
\end{itemize}
O que morde.
\section{Mordedouro}
\begin{itemize}
\item {Grp. gram.:m.}
\end{itemize}
\begin{itemize}
\item {Utilização:Náut.}
\end{itemize}
\begin{itemize}
\item {Proveniência:(De \textunderscore morder\textunderscore )}
\end{itemize}
Apparelho, para suster a amarra da âncora, quando corre pelo escovém.
\section{Mordedura}
\begin{itemize}
\item {Grp. gram.:f.}
\end{itemize}
\begin{itemize}
\item {Utilização:Fig.}
\end{itemize}
Acto ou effeito de morder.
Vestígio de dentada.
Vestígio doloroso.
Offensa.
\section{Mordelas}
\begin{itemize}
\item {Grp. gram.:f. pl.}
\end{itemize}
Gênero de insectos heterómeros.
\section{Mordellas}
\begin{itemize}
\item {Grp. gram.:f. pl.}
\end{itemize}
Gênero de insectos heterómeros.
\section{Mordente}
\begin{itemize}
\item {Grp. gram.:adj.}
\end{itemize}
\begin{itemize}
\item {Grp. gram.:M.}
\end{itemize}
\begin{itemize}
\item {Proveniência:(Do lat. \textunderscore mordens\textunderscore , \textunderscore mordentis\textunderscore )}
\end{itemize}
Que morde.
Que arranha.
Mordaz.
Provocante.
Preparação de tinta, para cobrir objectos que se pretendem doirar.
Preparação, para se fixarem as côres.
Instrumento, com que o typógrapho marca as linhas que vai copiando.
Ornamento melódico, composto de duas notas rápidas, como um trílo de curta duração.
\section{Morder}
\begin{itemize}
\item {Grp. gram.:v. i.}
\end{itemize}
\begin{itemize}
\item {Grp. gram.:V. i.}
\end{itemize}
\begin{itemize}
\item {Grp. gram.:V. p.}
\end{itemize}
\begin{itemize}
\item {Utilização:Fig.}
\end{itemize}
\begin{itemize}
\item {Proveniência:(Lat. \textunderscore mordere\textunderscore )}
\end{itemize}
Apertar ou ferir com os dentes: \textunderscore morder os beiços\textunderscore .
Dar dentadas em.
Cercear; corroer.
Affligir; torturar.
Tostar, queimar.
Embeber-se em.
Incitar, estimular.
Murmurar de.
Criticar com azedume.
\textunderscore Morder a areia\textunderscore , enterrar-se na areia.
\textunderscore Morder o pó\textunderscore , cair morto no chão.
Ficar vencido e morto.
Dar dentadas; cravar os dentes: \textunderscore aquelle cão morde\textunderscore .
Tomar o gôsto ou sabor.
Sêr picante.
Causar ardor: \textunderscore êste emplasto morde\textunderscore .
Causar comichão.
Sentir comichão ou prurido: \textunderscore morde-lhe o corpo\textunderscore .
Murmurar, falar mal de alguém ou de alguma coisa, ou com malevolência.
Irritar-se.
Despeitar-se.
Sentir o estímulo de algum sentimento condemnável: \textunderscore morder-se de inveja\textunderscore .
\section{Mordexi}
\begin{itemize}
\item {Grp. gram.:m.}
\end{itemize}
\begin{itemize}
\item {Proveniência:(T. pracrítico. Cp. conc. \textunderscore modaxi\textunderscore , marata \textunderscore modxi\textunderscore )}
\end{itemize}
O mesmo que \textunderscore cólera-morbo\textunderscore , na Índia port. Cp. Camillo, \textunderscore Narcót.\textunderscore , II, 131; Orta, \textunderscore Colóquios\textunderscore .
\section{Mordexim}
\begin{itemize}
\item {Grp. gram.:m.}
\end{itemize}
\begin{itemize}
\item {Proveniência:(T. pracrítico. Cp. conc. \textunderscore modaxi\textunderscore , marata \textunderscore modxi\textunderscore )}
\end{itemize}
O mesmo que \textunderscore cólera-morbo\textunderscore , na Índia port. Cp. Camillo, \textunderscore Narcót.\textunderscore , II, 131; Orta, \textunderscore Colóquios\textunderscore .
\section{Mordexinado}
\begin{itemize}
\item {Grp. gram.:adj.}
\end{itemize}
Aquelle que foi atacado de mordexim. Cf. Camillo, \textunderscore Narcót.\textunderscore , II, 131.
\section{Mordezada}
\begin{itemize}
\item {Grp. gram.:f.}
\end{itemize}
\begin{itemize}
\item {Utilização:Prov.}
\end{itemize}
\begin{itemize}
\item {Utilização:alg.}
\end{itemize}
O mesmo que \textunderscore mordedura\textunderscore .
\section{Mordicação}
\begin{itemize}
\item {Grp. gram.:f.}
\end{itemize}
\begin{itemize}
\item {Proveniência:(Lat. \textunderscore mordicatio\textunderscore )}
\end{itemize}
Acto ou effeito de mordicar.
Sensação, que os líquidos acres ou corrosivos produzem no corpo.
Acção dêsses líquidos.
\section{Mordicadela}
\begin{itemize}
\item {Grp. gram.:f.}
\end{itemize}
O mesmo que \textunderscore mordicação\textunderscore .
\section{Mordicante}
\begin{itemize}
\item {Grp. gram.:adj.}
\end{itemize}
\begin{itemize}
\item {Proveniência:(Lat. \textunderscore mordicans\textunderscore )}
\end{itemize}
Que mordica; que produz mordicação.
\section{Mordicar}
\begin{itemize}
\item {Grp. gram.:v. t.}
\end{itemize}
\begin{itemize}
\item {Utilização:Ext.}
\end{itemize}
\begin{itemize}
\item {Proveniência:(Lat. \textunderscore mordicare\textunderscore )}
\end{itemize}
Morder levemente, repetidas vezes.
Morder.
Pungir.
Estimular.
\section{Mordicativo}
\begin{itemize}
\item {Grp. gram.:adj.}
\end{itemize}
\begin{itemize}
\item {Proveniência:(Lat. \textunderscore mordicativus\textunderscore )}
\end{itemize}
O mesmo que \textunderscore mordicante\textunderscore .
\section{Mordico}
\begin{itemize}
\item {Grp. gram.:m.}
\end{itemize}
\begin{itemize}
\item {Utilização:Prov.}
\end{itemize}
\begin{itemize}
\item {Proveniência:(De \textunderscore mordo\textunderscore )}
\end{itemize}
Lanche, piqueta.
\section{Mordidela}
\begin{itemize}
\item {Grp. gram.:f.}
\end{itemize}
O mesmo que \textunderscore mordedela\textunderscore . Us. por Castilho.
\section{Mordido}
\begin{itemize}
\item {Grp. gram.:adj.}
\end{itemize}
\begin{itemize}
\item {Utilização:Náut.}
\end{itemize}
Que deixa intervallos ou fórma saliências, (falando-se das voltas de um cordão ou cabo náutico, dadas noutro cabo ou noutro objecto).
\section{Mordímano}
\begin{itemize}
\item {Grp. gram.:m.}
\end{itemize}
Peixe de Portugal.
\section{Mordimento}
\begin{itemize}
\item {Grp. gram.:m.}
\end{itemize}
\begin{itemize}
\item {Utilização:Ext.}
\end{itemize}
\begin{itemize}
\item {Proveniência:(De \textunderscore morder\textunderscore )}
\end{itemize}
O mesmo que \textunderscore mordedura\textunderscore .
Remorso.
\section{Mordiscar}
\textunderscore v. t.\textunderscore  (e der.)
O mesmo que \textunderscore mordicar\textunderscore :«\textunderscore ...meio charuto mordiscado...\textunderscore »Camillo, \textunderscore Corja\textunderscore , 171.
\section{Mordo}
\begin{itemize}
\item {fónica:môr}
\end{itemize}
\begin{itemize}
\item {Grp. gram.:m.}
\end{itemize}
\begin{itemize}
\item {Utilização:Prov.}
\end{itemize}
\begin{itemize}
\item {Utilização:trasm.}
\end{itemize}
\begin{itemize}
\item {Proveniência:(De \textunderscore morder\textunderscore )}
\end{itemize}
Pequena porção de qualquer coisa, que se dá ao animal para o desaguar.
Cibo; bocado.
\section{Mor-dobre}
\begin{itemize}
\item {Grp. gram.:m.}
\end{itemize}
\begin{itemize}
\item {Utilização:Ant.}
\end{itemize}
Transformação ou variação de uma palavra, especialmente verbo, formando duplicação de som semelhante:«\textunderscore se tu vês eu vejo\textunderscore ». Car. Míchaelis, \textunderscore Geschichte der Port. Litter.\textunderscore , 196.
\section{Mordomado}
\begin{itemize}
\item {Grp. gram.:m.}
\end{itemize}
\begin{itemize}
\item {Proveniência:(Do b. lat. \textunderscore maiordomatus\textunderscore )}
\end{itemize}
Mordomia.
Tempo, que dura a mordomia.
Imposto, que pagavam os que tinham mordómo.
\section{Mordomar}
\begin{itemize}
\item {Grp. gram.:v. t.}
\end{itemize}
\begin{itemize}
\item {Grp. gram.:V. i.}
\end{itemize}
Administrar como mordómo.
Exercer funcções de mordómo.
\section{Mordomear}
\begin{itemize}
\item {Grp. gram.:v. t.}
\end{itemize}
O mesmo que \textunderscore mordomar\textunderscore :«\textunderscore ...essa fazemda que feitoriza e mordomea é toda de Deos...\textunderscore »Sousa, \textunderscore Vida do Arceb.\textunderscore , I, 200.
\section{Mordomia}
\begin{itemize}
\item {Grp. gram.:f.}
\end{itemize}
Offício de mordómo.
\section{Mordomice}
\begin{itemize}
\item {Grp. gram.:f.}
\end{itemize}
\begin{itemize}
\item {Utilização:Deprec.}
\end{itemize}
O mesmo que \textunderscore mordomia\textunderscore .
\section{Mordomizar}
\begin{itemize}
\item {Grp. gram.:v. t.}
\end{itemize}
O mesmo que \textunderscore mordomar\textunderscore . Cf. Camillo, \textunderscore Mulher Fatal\textunderscore , 64.
\section{Mordómo}
\begin{itemize}
\item {Grp. gram.:m.}
\end{itemize}
\begin{itemize}
\item {Proveniência:(Do lat. \textunderscore maior\textunderscore  + \textunderscore domus\textunderscore )}
\end{itemize}
Aquelle que administra casa ou estabelecimento de outrem.
O encarregado de preparar e dirigir uma festa de igreja.
Aquelle que administra bens de confrarias ou irmandades.
Antigo official de justiça, encarregado de citações e execuções.
\textunderscore Mordomo-mór\textunderscore , funccionário da antiga casa real, encarregado de superintender nas despesas da mesma casa.
\section{Morduínio}
\begin{itemize}
\item {Grp. gram.:m.}
\end{itemize}
Língua uralo-altaica, vernácula na Rússia.
\section{Morduíno}
\begin{itemize}
\item {Grp. gram.:m.}
\end{itemize}
Língua uralo-altaica, vernácula na Rússia.
\section{Moreáceas}
\begin{itemize}
\item {Grp. gram.:f. pl.}
\end{itemize}
\begin{itemize}
\item {Proveniência:(De \textunderscore moreáceo\textunderscore )}
\end{itemize}
Família de plantas, que tem por typo a amoreira.
\section{Moreáceo}
\begin{itemize}
\item {Grp. gram.:adj.}
\end{itemize}
\begin{itemize}
\item {Proveniência:(Do lat. \textunderscore morum\textunderscore )}
\end{itemize}
Relativo ou semelhante á amoreira.
\section{Moreão}
\begin{itemize}
\item {Grp. gram.:m.}
\end{itemize}
\begin{itemize}
\item {Proveniência:(De \textunderscore moreia\textunderscore ^2)}
\end{itemize}
Peixe dos Açores.
\section{Moreia}
\begin{itemize}
\item {Grp. gram.:f.}
\end{itemize}
\begin{itemize}
\item {Utilização:Prov.}
\end{itemize}
\begin{itemize}
\item {Utilização:trasm.}
\end{itemize}
\begin{itemize}
\item {Utilização:Prov.}
\end{itemize}
\begin{itemize}
\item {Utilização:alg.}
\end{itemize}
\begin{itemize}
\item {Utilização:Ant.}
\end{itemize}
Grupo de feixes de trigo ou de outro cereal, verticalmente collocados na terra, com as espigas para cima.
O mesmo que \textunderscore mêda\textunderscore .
Monte de estrume, de paus, de qualquer coisa.
Feixe de mato, que no inverno se cobre de terra e que se queima no verão, para que a sua cinza sirva de adubo ás terras, em que se semeiam cereaes.
O mesmo que \textunderscore carrada\textunderscore .
(Relaciona-se talvez com o b. lat. \textunderscore mora\textunderscore , columna, pilar, ou antes com o b. lat. \textunderscore murea\textunderscore , assento de pedra, se não tem a mesma or. que \textunderscore moroiço\textunderscore )
\section{Moreia}
\begin{itemize}
\item {Grp. gram.:f.}
\end{itemize}
\begin{itemize}
\item {Proveniência:(Do gr. \textunderscore muraina\textunderscore )}
\end{itemize}
Gênero de peixes anguilliformes.
\section{Moreira}
\begin{itemize}
\item {Grp. gram.:f.}
\end{itemize}
\begin{itemize}
\item {Utilização:Bras}
\end{itemize}
\begin{itemize}
\item {Utilização:Pop.}
\end{itemize}
O mesmo que \textunderscore tatajuba\textunderscore .
O mesmo que \textunderscore amoreira\textunderscore .
\section{Moreiredo}
\begin{itemize}
\item {fónica:eirê}
\end{itemize}
\begin{itemize}
\item {Grp. gram.:m.}
\end{itemize}
\begin{itemize}
\item {Utilização:Des.}
\end{itemize}
\begin{itemize}
\item {Proveniência:(De \textunderscore morar\textunderscore ?)}
\end{itemize}
O mesmo que \textunderscore lapedo\textunderscore .
\section{Moreiredo}
\begin{itemize}
\item {fónica:eirê}
\end{itemize}
\begin{itemize}
\item {Grp. gram.:m.}
\end{itemize}
\begin{itemize}
\item {Proveniência:(De \textunderscore moreira\textunderscore )}
\end{itemize}
Lugar, onde crescem amoreiras.
\section{Morélia}
\begin{itemize}
\item {Grp. gram.:f.}
\end{itemize}
Gênero de plantas rubiáceas.
Gênero de reptís ophídios.
\section{Morena}
\begin{itemize}
\item {Grp. gram.:f.}
\end{itemize}
Mulher trigueira.
Variedade de maçan.
(Cp. \textunderscore moreno\textunderscore )
\section{Morena}
\begin{itemize}
\item {Grp. gram.:f.}
\end{itemize}
\begin{itemize}
\item {Utilização:Geol.}
\end{itemize}
Acervo de pedras, que as geleiras, descendo, foram acumulando aos lados e na sua extremidade inferior.
(B. lat. \textunderscore morena\textunderscore )
\section{Morenado}
\begin{itemize}
\item {Grp. gram.:adj.}
\end{itemize}
\begin{itemize}
\item {Utilização:Poét.}
\end{itemize}
Que se fez moreno.
\section{Morênia}
\begin{itemize}
\item {Grp. gram.:f.}
\end{itemize}
Gênero de palmeiras.
\section{Moreno}
\begin{itemize}
\item {Grp. gram.:m.  e  adj.}
\end{itemize}
\begin{itemize}
\item {Grp. gram.:M.}
\end{itemize}
\begin{itemize}
\item {Utilização:Prov.}
\end{itemize}
\begin{itemize}
\item {Utilização:trasm.}
\end{itemize}
Aquelle que tem côr trigueira.
Pó negro, que se deposita nas fráguas e que é uma mistura de limalha de ferro e de pó de carvão.
(Cast. \textunderscore moreno\textunderscore , de \textunderscore moro\textunderscore , moiro)
\section{Moreóta}
\begin{itemize}
\item {Grp. gram.:m.}
\end{itemize}
Habitante da península da Moreia.
\section{Morerenga}
\begin{itemize}
\item {Grp. gram.:f.}
\end{itemize}
\begin{itemize}
\item {Utilização:Bras}
\end{itemize}
Árvore silvestre.
\section{Morétia}
\begin{itemize}
\item {Grp. gram.:f.}
\end{itemize}
Gênero de plantas crucíferas.
\section{Moreto}
\begin{itemize}
\item {fónica:morê}
\end{itemize}
\begin{itemize}
\item {Grp. gram.:m.}
\end{itemize}
\begin{itemize}
\item {Proveniência:(Do lat. \textunderscore morum\textunderscore )}
\end{itemize}
Variedade de uva preta.
\section{Moreto}
\begin{itemize}
\item {fónica:morê}
\end{itemize}
\begin{itemize}
\item {Grp. gram.:m.}
\end{itemize}
\begin{itemize}
\item {Proveniência:(Lat. \textunderscore moretum\textunderscore )}
\end{itemize}
Iguaria de camponeses, entre os antigos Romanos, composta de arruda, alho, vinagre, azeite, etc. Cf. Castilho, \textunderscore Fastos\textunderscore , II, 145.
\section{Morexim}
\begin{itemize}
\item {Grp. gram.:m.}
\end{itemize}
O mesmo que \textunderscore mordexim\textunderscore .
\section{Morfanho}
\begin{itemize}
\item {Grp. gram.:adj.}
\end{itemize}
O mesmo que \textunderscore fanhoso\textunderscore .
\section{Morgada}
\begin{itemize}
\item {Grp. gram.:f.}
\end{itemize}
Mulher, que é ou foi de morgado.
Senhora de bens, que constituem um morgado.
\section{Morgadelho}
\begin{itemize}
\item {fónica:dê}
\end{itemize}
\begin{itemize}
\item {Grp. gram.:m.}
\end{itemize}
O mesmo que \textunderscore morgadete\textunderscore . Cf. Eça, \textunderscore P. Amaro\textunderscore , 461.
\section{Morgadeta}
\begin{itemize}
\item {fónica:dê}
\end{itemize}
\begin{itemize}
\item {Grp. gram.:f.}
\end{itemize}
\begin{itemize}
\item {Utilização:Deprec.}
\end{itemize}
Morgada, de poucos rendimentos. Cf. Castilho, \textunderscore Sabichonas\textunderscore , 75.
\section{Morgadete}
\begin{itemize}
\item {fónica:dê}
\end{itemize}
\begin{itemize}
\item {Grp. gram.:m.}
\end{itemize}
\begin{itemize}
\item {Utilização:Deprec.}
\end{itemize}
Morgado, que tem poucos rendimentos.
\section{Morgadilho}
\begin{itemize}
\item {Grp. gram.:m.}
\end{itemize}
\begin{itemize}
\item {Utilização:T. de Bragança}
\end{itemize}
O mesmo que \textunderscore peneira\textunderscore .
\section{Morgadio}
\begin{itemize}
\item {Grp. gram.:adj.}
\end{itemize}
\begin{itemize}
\item {Grp. gram.:M.}
\end{itemize}
Relativo a morgado.
Qualidade de morgado.
Bens de morgado.
\section{Morgado}
\begin{itemize}
\item {Grp. gram.:m.}
\end{itemize}
\begin{itemize}
\item {Utilização:Ext.}
\end{itemize}
\begin{itemize}
\item {Utilização:Fig.}
\end{itemize}
\begin{itemize}
\item {Grp. gram.:Pl.}
\end{itemize}
\begin{itemize}
\item {Proveniência:(Do b. lat. \textunderscore maioricatus\textunderscore )}
\end{itemize}
Propriedade vinculada, ou conjunto de bens vinculados, que não podiam alienar-se ou dividir-se, e que geralmente, por morte do possuidor, pertenciam ao filho primogênito.
Possuidor dêsses bens.
Filho primogênito ou herdeiro de possuidor de bens vinculados.
Filho mais velho; filho único.
Coisa muito rendosa.
Espécie de pastéis.
\section{Morganaticamente}
\begin{itemize}
\item {Grp. gram.:adv.}
\end{itemize}
De modo morganático.
\section{Morganático}
\begin{itemize}
\item {Grp. gram.:adj.}
\end{itemize}
\begin{itemize}
\item {Proveniência:(Do al. \textunderscore morgen\textunderscore , manhan, por allusão ás horas matutinas, preferidas para casamento de pessôas de condição desigual?)}
\end{itemize}
Diz-se do casamento, contrahido por um principe com uma mulher de condição inferior.
\section{Morganho}
\begin{itemize}
\item {Grp. gram.:m.}
\end{itemize}
\begin{itemize}
\item {Utilização:Prov.}
\end{itemize}
\begin{itemize}
\item {Utilização:minh.}
\end{itemize}
Ajuntamento, grupo.
Porção de coisas: \textunderscore avistei no adro um morganho de gente\textunderscore .
\section{Morganho}
\begin{itemize}
\item {Grp. gram.:m.}
\end{itemize}
\begin{itemize}
\item {Utilização:Prov.}
\end{itemize}
\begin{itemize}
\item {Utilização:trasm.}
\end{itemize}
Indivíduo sem valor; môrcas.
(Cp. \textunderscore morcão\textunderscore ^1)
\section{Morganiça}
\begin{itemize}
\item {Grp. gram.:f.}
\end{itemize}
\begin{itemize}
\item {Utilização:T. de Turquel}
\end{itemize}
Espécie de urze.
\section{Morgue}
\begin{itemize}
\item {Grp. gram.:f.}
\end{itemize}
\begin{itemize}
\item {Utilização:Gal}
\end{itemize}
\begin{itemize}
\item {Proveniência:(Fr. \textunderscore morgue\textunderscore )}
\end{itemize}
O mesmo que \textunderscore necrotério\textunderscore .
\section{Moribundo}
\begin{itemize}
\item {Grp. gram.:m.  e  adj.}
\end{itemize}
\begin{itemize}
\item {Proveniência:(Lat. \textunderscore moribundus\textunderscore )}
\end{itemize}
Aquelle que está quási a morrer.
Que vae acabar.
Que vai desapparecer, (falando-se do sol).
Que está quási a perder o brilho, a seiva: \textunderscore flôres moribundas\textunderscore .
Amortecido: \textunderscore luz moribunda\textunderscore .
\section{Morfeia}
\begin{itemize}
\item {Grp. gram.:f.}
\end{itemize}
Designação antiga de uma doença cutânea, mal classificada; elefantíase.
(B. lat. \textunderscore morphea\textunderscore )
\section{Morfético}
\begin{itemize}
\item {Grp. gram.:adj.}
\end{itemize}
Relativo a Morfeu.
Relativo ao sono. Cf. Camillo, \textunderscore Mar. da Fonte\textunderscore , 129.
\section{Morfetina}
\begin{itemize}
\item {Grp. gram.:f.}
\end{itemize}
\begin{itemize}
\item {Utilização:Chím.}
\end{itemize}
Produto da morfina, sujeita á acção do peróxido de cobre e ácido sulfúrico.
\section{Mórfico}
\begin{itemize}
\item {Grp. gram.:adj.}
\end{itemize}
\begin{itemize}
\item {Proveniência:(Do gr. \textunderscore morphe\textunderscore )}
\end{itemize}
Relativo á fórma ou ás manifestações externas do pensamento ou do sentimento. Cf. Th. Braga, \textunderscore Mod. Id.\textunderscore , I, 372.
\section{Morfina}
\begin{itemize}
\item {Grp. gram.:f.}
\end{itemize}
\begin{itemize}
\item {Proveniência:(De \textunderscore Morpheu\textunderscore , n. p.)}
\end{itemize}
Alcali vegetal, de qualidades narcóticas, que se encontra no ópio do comércio e na papoila.
\section{Morfínico}
\begin{itemize}
\item {Grp. gram.:adj.}
\end{itemize}
Relativo a morfina: \textunderscore sal morfínico\textunderscore .
\section{Morfinismo}
\begin{itemize}
\item {Grp. gram.:m.}
\end{itemize}
\begin{itemize}
\item {Proveniência:(De \textunderscore morfina\textunderscore )}
\end{itemize}
Abuso da morfina, para excitar a imaginação ou acalmar dores.
\section{Morfinização}
\begin{itemize}
\item {Grp. gram.:f.}
\end{itemize}
Acto ou efeito de morfinizar.
\section{Morfinizar}
\begin{itemize}
\item {Grp. gram.:v. t.}
\end{itemize}
Aplicar morfina a.
\section{Morfinomania}
\begin{itemize}
\item {Grp. gram.:f.}
\end{itemize}
\begin{itemize}
\item {Proveniência:(De \textunderscore morfina\textunderscore  + \textunderscore mania\textunderscore )}
\end{itemize}
Paixão pela morfina.
Emprêgo habitual desta substância.
\section{Morfno}
\begin{itemize}
\item {Grp. gram.:m.}
\end{itemize}
\begin{itemize}
\item {Proveniência:(Lat. \textunderscore morphnos\textunderscore )}
\end{itemize}
Designação antiga de uma espécie de águia, que se encontrava principalmente no mar.
\section{Morfogenia}
\begin{itemize}
\item {Grp. gram.:f.}
\end{itemize}
\begin{itemize}
\item {Proveniência:(Do gr. \textunderscore morphe\textunderscore  + \textunderscore genea\textunderscore )}
\end{itemize}
Producção da fórma.
Disposição, que as moléculas tomam, na composição de um corpo.
\section{Morfogênico}
\begin{itemize}
\item {Grp. gram.:adj.}
\end{itemize}
Relativo á morfogenia.
\section{Morfologia}
\begin{itemize}
\item {Grp. gram.:f.}
\end{itemize}
\begin{itemize}
\item {Proveniência:(Do gr. \textunderscore morphe\textunderscore  + \textunderscore logos\textunderscore )}
\end{itemize}
Tratado das fórmas, que a matéria póde assumir.
Parte da Gramática, que se occupa da fórma e transformação das palavras.
\section{Morfologicamente}
\begin{itemize}
\item {Grp. gram.:adv.}
\end{itemize}
De modo morfológico.
Segundo as regras da morfologia.
\section{Morfológico}
\begin{itemize}
\item {Grp. gram.:adj.}
\end{itemize}
Relativo á morfologia.
\section{Morfologista}
\begin{itemize}
\item {Grp. gram.:m.}
\end{itemize}
Aquele que se ocupa ou trata cientificamente da morfologia. Cf. Latino, \textunderscore Or. da Corôa\textunderscore , CCVII.
\section{Morfólogo}
\begin{itemize}
\item {Grp. gram.:m.}
\end{itemize}
O mesmo que \textunderscore morfologista\textunderscore .
Aquele que é perito em morfologia. Cf. Camillo, \textunderscore Maria da Fonte\textunderscore , 351.
\section{Morfose}
\begin{itemize}
\item {Grp. gram.:f.}
\end{itemize}
\begin{itemize}
\item {Proveniência:(Gr. \textunderscore morphosis\textunderscore )}
\end{itemize}
Acto de tomar fórma.
Acto de formar ou de dar fórma.
\section{Morfostequia}
\begin{itemize}
\item {Grp. gram.:f.}
\end{itemize}
\begin{itemize}
\item {Proveniência:(Do gr. \textunderscore morphe\textunderscore  + \textunderscore stoikheion\textunderscore )}
\end{itemize}
O mesmo que \textunderscore micromorfite\textunderscore .
\section{Morfozoário}
\begin{itemize}
\item {Grp. gram.:m.}
\end{itemize}
\begin{itemize}
\item {Proveniência:(Do gr. \textunderscore morphe\textunderscore  + \textunderscore zoon\textunderscore )}
\end{itemize}
Qualquer animal, cuja fórma está bem determinada.
\section{Moricândia}
\begin{itemize}
\item {Grp. gram.:f.}
\end{itemize}
\begin{itemize}
\item {Proveniência:(De \textunderscore Moricand\textunderscore , n. p.)}
\end{itemize}
Planta crucífera.
\section{Mórico}
\begin{itemize}
\item {Grp. gram.:adj.}
\end{itemize}
\begin{itemize}
\item {Proveniência:(Do lat. hypoth. \textunderscore moricus\textunderscore )}
\end{itemize}
Diz-se de um ácido, extrahido de casca da amoreira branca, (\textunderscore morus alba\textunderscore , Lin.).
\section{Morigeração}
\begin{itemize}
\item {Grp. gram.:f.}
\end{itemize}
\begin{itemize}
\item {Proveniência:(Lat. \textunderscore morigeratio\textunderscore )}
\end{itemize}
Acto ou effeito de morigerar.
Moralização; bons costumes.
\section{Morigerado}
\begin{itemize}
\item {Grp. gram.:adj.}
\end{itemize}
\begin{itemize}
\item {Proveniência:(De \textunderscore morigerar\textunderscore )}
\end{itemize}
Que tem bons costumes ou vida exemplar.
\section{Morigerar}
\begin{itemize}
\item {Grp. gram.:v. i.}
\end{itemize}
\begin{itemize}
\item {Grp. gram.:V. p.}
\end{itemize}
\begin{itemize}
\item {Proveniência:(Lat. \textunderscore morigerari\textunderscore )}
\end{itemize}
Modificar os costumes de.
Dar bons costumes a.
Ensinar; educar.
Adquirir bons costumes.
\section{Morígero}
\begin{itemize}
\item {Grp. gram.:adj.}
\end{itemize}
\begin{itemize}
\item {Utilização:Poét.}
\end{itemize}
\begin{itemize}
\item {Proveniência:(Lat. \textunderscore morigerus\textunderscore )}
\end{itemize}
O mesmo que \textunderscore morigerado\textunderscore .
\section{Moril}
\begin{itemize}
\item {Grp. gram.:m.}
\end{itemize}
\begin{itemize}
\item {Utilização:Prov.}
\end{itemize}
O mesmo que \textunderscore morilho\textunderscore .
\section{Morilhão}
\begin{itemize}
\item {Grp. gram.:m.}
\end{itemize}
\begin{itemize}
\item {Proveniência:(Do lat. \textunderscore morum\textunderscore )}
\end{itemize}
Espécie de pulgão, que ataca as favas e outros vegetaes.
\section{Morilho}
\begin{itemize}
\item {Grp. gram.:m.}
\end{itemize}
\begin{itemize}
\item {Utilização:Prov.}
\end{itemize}
\begin{itemize}
\item {Utilização:trasm.}
\end{itemize}
Pedra ou peça de ferro, em que se apoia a lenha que arde na cozinha, e que separa da borralheira o lar; trasfogueiro.
(Cast. \textunderscore morillo\textunderscore )
\section{Morim}
\begin{itemize}
\item {Grp. gram.:m.}
\end{itemize}
\begin{itemize}
\item {Utilização:Prov.}
\end{itemize}
Pano branco e fino de algodão, também conhecido por \textunderscore pano patente\textunderscore .
\section{Morimo}
\begin{itemize}
\item {Grp. gram.:m.}
\end{itemize}
Gênero de insectos coleópteros, longicórneos.
\section{Morina}
\begin{itemize}
\item {Grp. gram.:f.}
\end{itemize}
\begin{itemize}
\item {Proveniência:(Do lat. \textunderscore morum\textunderscore )}
\end{itemize}
Substância còrante da amoreira.
Gênero de plantas dipsáceas.
\section{Morinda}
\begin{itemize}
\item {Grp. gram.:f.}
\end{itemize}
Gênero de plantas rubiáceas, (\textunderscore morinda citrifolia\textunderscore , Lin.).
\section{Moríndeas}
\begin{itemize}
\item {Grp. gram.:f. pl.}
\end{itemize}
Sub-tribo de plantas, que tem por typo a morinda.
\section{Moríneas}
\begin{itemize}
\item {Grp. gram.:f. pl.}
\end{itemize}
Tribo de plantas, que tem por typo a morina.
\section{Morinelo}
\begin{itemize}
\item {Grp. gram.:m.}
\end{itemize}
Ave pernalta do norte da Europa, que emigra para o sul.
(Relaciona-se talvez com o b. lat. \textunderscore murinus\textunderscore , que tem côr de rato)
\section{Moringa}
\begin{itemize}
\item {Grp. gram.:f.}
\end{itemize}
\begin{itemize}
\item {Utilização:Bras}
\end{itemize}
O mesmo que \textunderscore moringue\textunderscore .
\section{Moringo}
\begin{itemize}
\item {Grp. gram.:m.}
\end{itemize}
O mesmo que \textunderscore moringue\textunderscore .
\section{Moringue}
\begin{itemize}
\item {Grp. gram.:m.}
\end{itemize}
\begin{itemize}
\item {Proveniência:(T. bras.)}
\end{itemize}
Bilha para água, que tem superiormente uma asa, e um gargalo em cada extremidade desta.
\section{Moringueiro}
\begin{itemize}
\item {Grp. gram.:m.}
\end{itemize}
Bella árvore indiana, (\textunderscore moringa pterygosperna\textunderscore , Gaërtn).
\section{Morintânico}
\begin{itemize}
\item {Grp. gram.:adj.}
\end{itemize}
Diz-se de um ácido, que se acha na madeira da amoreira, juntamente com o ácido mórico.
\section{Morintânnico}
\begin{itemize}
\item {Grp. gram.:adj.}
\end{itemize}
Diz-se de um ácido, que se acha na madeira da amoreira, juntamente com o ácido mórico.
\section{Morioplastia}
\begin{itemize}
\item {Grp. gram.:f.}
\end{itemize}
\begin{itemize}
\item {Proveniência:(Do gr. \textunderscore morion\textunderscore  + \textunderscore plassein\textunderscore )}
\end{itemize}
Substituição cirúrgica de qualquer parte dos nossos órgãos.
\section{Moriquino}
\begin{itemize}
\item {Grp. gram.:m.}
\end{itemize}
Antiga moéda moirisca, usada em Espanha.
(B. lat. \textunderscore morikinus\textunderscore )
\section{Morisco}
\begin{itemize}
\item {Grp. gram.:m.}
\end{itemize}
\begin{itemize}
\item {Utilização:Prov.}
\end{itemize}
\begin{itemize}
\item {Utilização:beir.}
\end{itemize}
Nome, com que o boieiro ou carreiro chama e incita o mais escuro dos bois, que vai guiando: \textunderscore eh! cá morisco!\textunderscore 
(Cp. \textunderscore moreno\textunderscore )
\section{Morissica}
\begin{itemize}
\item {Grp. gram.:f.}
\end{itemize}
\begin{itemize}
\item {Utilização:Prov.}
\end{itemize}
Mosquito, melga, trombeteiro.
\section{Mormacento}
\begin{itemize}
\item {Grp. gram.:adj.}
\end{itemize}
\begin{itemize}
\item {Utilização:Fig.}
\end{itemize}
Semelhante ao mormo.
Quente e húmido, (falando-se do tempo).
\section{Mormaço}
\begin{itemize}
\item {Grp. gram.:m.}
\end{itemize}
Tempo mormacento.
(Talvez de \textunderscore bruma\textunderscore , visto que no Pico e no Faial dizem \textunderscore brumaço\textunderscore , de que \textunderscore mormaço\textunderscore  seria alter. explicável)
\section{Mòrmente}
\begin{itemize}
\item {Grp. gram.:adv.}
\end{itemize}
\begin{itemize}
\item {Proveniência:(De \textunderscore mòr\textunderscore )}
\end{itemize}
O mesmo que \textunderscore principalmente\textunderscore .
\section{Mormiro}
\begin{itemize}
\item {Grp. gram.:m.}
\end{itemize}
\begin{itemize}
\item {Proveniência:(Lat. \textunderscore mormyr\textunderscore )}
\end{itemize}
Gênero de peixes do Nilo e do Mar-Negro.
\section{Mormo}
\begin{itemize}
\item {Grp. gram.:m.}
\end{itemize}
\begin{itemize}
\item {Proveniência:(Do lat. \textunderscore morbus\textunderscore )}
\end{itemize}
Doença do gado cavallar e asinino, que consiste na inflammação da membrana pituitária, com corrimento de pus pelas vias nasaes.
\section{Mormo}
\begin{itemize}
\item {Grp. gram.:m.}
\end{itemize}
\begin{itemize}
\item {Utilização:açor}
\end{itemize}
\begin{itemize}
\item {Utilização:Pop.}
\end{itemize}
Azêmola, bêsta ordinária.
(Corr. de \textunderscore mono\textunderscore ? Ou relaciona-se com \textunderscore mormo\textunderscore ^1?)
\section{Mormonismo}
\begin{itemize}
\item {Grp. gram.:m.}
\end{itemize}
\begin{itemize}
\item {Proveniência:(De \textunderscore Mormon\textunderscore , n. p.)}
\end{itemize}
Doutrina social e religiosa, materialista, utilitária e polygâmica, preconizada na América.
\section{Mormonista}
\begin{itemize}
\item {Grp. gram.:m.}
\end{itemize}
Partidário do mormonismo.
\section{Mormons}
\begin{itemize}
\item {Grp. gram.:m. pl.}
\end{itemize}
\begin{itemize}
\item {Proveniência:(De \textunderscore Mormon\textunderscore , n. p.)}
\end{itemize}
Conjunto dos sectários do mormonismo.
\section{Mormoso}
\begin{itemize}
\item {Grp. gram.:adj.}
\end{itemize}
Que tem \textunderscore mormo\textunderscore ^1.
\section{Mormyro}
\begin{itemize}
\item {Grp. gram.:m.}
\end{itemize}
\begin{itemize}
\item {Proveniência:(Lat. \textunderscore mormyr\textunderscore )}
\end{itemize}
Gênero de peixes do Nilo e do Mar-Negro.
\section{Mornal}
\begin{itemize}
\item {Grp. gram.:m.}
\end{itemize}
\begin{itemize}
\item {Utilização:Prov.}
\end{itemize}
\begin{itemize}
\item {Utilização:trasm.}
\end{itemize}
\begin{itemize}
\item {Proveniência:(De \textunderscore morno\textunderscore ?)}
\end{itemize}
Meda de cereaes.
\section{Mornar}
\begin{itemize}
\item {Grp. gram.:v. t.}
\end{itemize}
O mesmo que \textunderscore amornar\textunderscore .
\section{Mornidão}
\begin{itemize}
\item {Grp. gram.:f.}
\end{itemize}
\begin{itemize}
\item {Utilização:Fig.}
\end{itemize}
\begin{itemize}
\item {Proveniência:(De \textunderscore morno\textunderscore )}
\end{itemize}
Estado do que é morno.
Qualidade do que é froixo, ou falto de energia.
\section{Morno}
\begin{itemize}
\item {Grp. gram.:adj.}
\end{itemize}
\begin{itemize}
\item {Utilização:Fig.}
\end{itemize}
\begin{itemize}
\item {Grp. gram.:Loc.}
\end{itemize}
\begin{itemize}
\item {Utilização:fam.}
\end{itemize}
\begin{itemize}
\item {Proveniência:(Do gót. \textunderscore maurnan\textunderscore )}
\end{itemize}
Pouco quente; tépido.
Que não tem energia.
Sereno.
Insípido, monótono.
\textunderscore Águas mornas\textunderscore , palliativos, que entretêm uma espectativa; meios, que se empregam, para adiar uma solução definitiva.--A flexão fem. lê-se \textunderscore mórna\textunderscore .
\section{Mornura}
\begin{itemize}
\item {Grp. gram.:f.}
\end{itemize}
\begin{itemize}
\item {Utilização:Prov.}
\end{itemize}
O mesmo que \textunderscore mornidão\textunderscore .
\section{Moroba}
\begin{itemize}
\item {Grp. gram.:f.}
\end{itemize}
\begin{itemize}
\item {Utilização:Bras}
\end{itemize}
Peixe fluvial.
\section{Morobixada}
\begin{itemize}
\item {Grp. gram.:m.}
\end{itemize}
O mesmo que \textunderscore tuxaua\textunderscore .
\section{Moroda}
\begin{itemize}
\item {Grp. gram.:f.}
\end{itemize}
Terra, destinada á cultura dos coqueiros, na Índia portuguesa.
(Do conc.)
\section{Moroiço}
\begin{itemize}
\item {Grp. gram.:m.}
\end{itemize}
\begin{itemize}
\item {Utilização:Prov.}
\end{itemize}
\begin{itemize}
\item {Utilização:beir.}
\end{itemize}
\begin{itemize}
\item {Proveniência:(Do vasc. \textunderscore mulço\textunderscore ?)}
\end{itemize}
Montão; ruma.
Montículo.
Monte de pedras.
\section{Moroixo}
\begin{itemize}
\item {Grp. gram.:m.}
\end{itemize}
\begin{itemize}
\item {Utilização:Prov.}
\end{itemize}
\begin{itemize}
\item {Utilização:dur.}
\end{itemize}
O mesmo que \textunderscore moroiço\textunderscore .
\section{Mororó}
\begin{itemize}
\item {Grp. gram.:m.}
\end{itemize}
Espécie bauhínia do Ceará. Cf. \textunderscore Jorn. do Comm.\textunderscore , do Rio, de 16-XI-900.
\section{Morosamente}
\begin{itemize}
\item {Grp. gram.:adv.}
\end{itemize}
De modo moroso; vagarosamente; com lentidão.
\section{Morosidade}
\begin{itemize}
\item {Grp. gram.:f.}
\end{itemize}
Qualidade do que é moroso; lentidão; vagar.
Froixidão.
\section{Moroso}
\begin{itemize}
\item {Grp. gram.:adj.}
\end{itemize}
\begin{itemize}
\item {Proveniência:(Do lat. \textunderscore mora\textunderscore )}
\end{itemize}
Que anda ou procede com lentidão; tardio.
Diffícil de fazer; demorado.
\section{Morotinga}
\begin{itemize}
\item {Grp. gram.:adj.}
\end{itemize}
O mesmo que \textunderscore tinga\textunderscore .
\section{Morouço}
\begin{itemize}
\item {Grp. gram.:m.}
\end{itemize}
\begin{itemize}
\item {Utilização:Prov.}
\end{itemize}
\begin{itemize}
\item {Utilização:beir.}
\end{itemize}
\begin{itemize}
\item {Proveniência:(Do vasc. \textunderscore mulço\textunderscore ?)}
\end{itemize}
Montão; ruma.
Montículo.
Monte de pedras.
\section{Morphéa}
\begin{itemize}
\item {Grp. gram.:f.}
\end{itemize}
Designação antiga de uma doença cutânea, mal classificada; elephantíase.
(B. lat. \textunderscore morphea\textunderscore )
\section{Morpheia}
\begin{itemize}
\item {Grp. gram.:f.}
\end{itemize}
Designação antiga de uma doença cutânea, mal classificada; elephantíase.
(B. lat. \textunderscore morphea\textunderscore )
\section{Morphético}
\begin{itemize}
\item {Grp. gram.:adj.}
\end{itemize}
Relativo a Morpheu.
Relativo ao somno. Cf. Camillo, \textunderscore Mar. da Fonte\textunderscore , 129.
\section{Morphetina}
\begin{itemize}
\item {Grp. gram.:f.}
\end{itemize}
\begin{itemize}
\item {Utilização:Chím.}
\end{itemize}
Producto da morphina, sujeita á acção do peróxydo de cobre e ácido sulfúrico.
\section{Mórphico}
\begin{itemize}
\item {Grp. gram.:adj.}
\end{itemize}
\begin{itemize}
\item {Proveniência:(Do gr. \textunderscore morphe\textunderscore )}
\end{itemize}
Relativo á fórma ou ás manifestações externas do pensamento ou do sentimento. Cf. Th. Braga, \textunderscore Mod. Id.\textunderscore , I, 372.
\section{Morphina}
\begin{itemize}
\item {Grp. gram.:f.}
\end{itemize}
\begin{itemize}
\item {Proveniência:(De \textunderscore Morpheu\textunderscore , n. p.)}
\end{itemize}
Alcali vegetal, de qualidades narcóticas, que se encontra no ópio do commércio e na papoila.
\section{Morphínico}
\begin{itemize}
\item {Grp. gram.:adj.}
\end{itemize}
Relativo a morphina: \textunderscore sal morphínico\textunderscore .
\section{Morphinismo}
\begin{itemize}
\item {Grp. gram.:m.}
\end{itemize}
\begin{itemize}
\item {Proveniência:(De \textunderscore morphina\textunderscore )}
\end{itemize}
Abuso da morphina, para excitar a imaginação ou acalmar dores.
\section{Morphinização}
\begin{itemize}
\item {Grp. gram.:f.}
\end{itemize}
Acto ou effeito de morphinizar.
\section{Morphinizar}
\begin{itemize}
\item {Grp. gram.:v. t.}
\end{itemize}
Applicar morphina a.
\section{Morphinomania}
\begin{itemize}
\item {Grp. gram.:f.}
\end{itemize}
\begin{itemize}
\item {Proveniência:(De \textunderscore morphina\textunderscore  + \textunderscore mania\textunderscore )}
\end{itemize}
Paixão pela morphina.
Emprêgo habitual desta substância.
\section{Morphno}
\begin{itemize}
\item {Grp. gram.:m.}
\end{itemize}
\begin{itemize}
\item {Proveniência:(Lat. \textunderscore morphnos\textunderscore )}
\end{itemize}
Designação antiga de uma espécie de águia, que se encontrava principalmente no mar.
\section{Morphogenia}
\begin{itemize}
\item {Grp. gram.:f.}
\end{itemize}
\begin{itemize}
\item {Proveniência:(Do gr. \textunderscore morphe\textunderscore  + \textunderscore genea\textunderscore )}
\end{itemize}
Producção da fórma.
Disposição, que as moléculas tomam, na composição de um corpo.
\section{Morphogênico}
\begin{itemize}
\item {Grp. gram.:adj.}
\end{itemize}
Relativo á morphogenia.
\section{Morphologia}
\begin{itemize}
\item {Grp. gram.:f.}
\end{itemize}
\begin{itemize}
\item {Proveniência:(Do gr. \textunderscore morphe\textunderscore  + \textunderscore logos\textunderscore )}
\end{itemize}
Tratado das fórmas, que a matéria póde assumir.
Parte da Grammática, que se occupa da fórma e transformação das palavras.
\section{Morphologicamente}
\begin{itemize}
\item {Grp. gram.:adv.}
\end{itemize}
De modo morphológico.
Segundo as regras da morphologia.
\section{Morphológico}
\begin{itemize}
\item {Grp. gram.:adj.}
\end{itemize}
Relativo á morphologia.
\section{Morphologista}
\begin{itemize}
\item {Grp. gram.:m.}
\end{itemize}
Aquelle que se occupa ou trata scientificamente da morphologia. Cf. Latino, \textunderscore Or. da Corôa\textunderscore , CCVII.
\section{Morphólogo}
\begin{itemize}
\item {Grp. gram.:m.}
\end{itemize}
O mesmo que \textunderscore morphologista\textunderscore .
Aquelle que é perito em morphologia. Cf. Camillo, \textunderscore Maria da Fonte\textunderscore , 351.
\section{Morphose}
\begin{itemize}
\item {Grp. gram.:f.}
\end{itemize}
\begin{itemize}
\item {Proveniência:(Gr. \textunderscore morphosis\textunderscore )}
\end{itemize}
Acto de tomar fórma.
Acto de formar ou de dar fórma.
\section{Morphostechia}
\begin{itemize}
\item {fónica:qui}
\end{itemize}
\begin{itemize}
\item {Grp. gram.:f.}
\end{itemize}
\begin{itemize}
\item {Proveniência:(Do gr. \textunderscore morphe\textunderscore  + \textunderscore stoikheion\textunderscore )}
\end{itemize}
O mesmo que \textunderscore micromorphite\textunderscore .
\section{Morphozoário}
\begin{itemize}
\item {Grp. gram.:m.}
\end{itemize}
\begin{itemize}
\item {Proveniência:(Do gr. \textunderscore morphe\textunderscore  + \textunderscore zoon\textunderscore )}
\end{itemize}
Qualquer animal, cuja fórma está bem determinada.
\section{Morra!}
\begin{itemize}
\item {fónica:mô}
\end{itemize}
\begin{itemize}
\item {Grp. gram.:interj.}
\end{itemize}
\begin{itemize}
\item {Grp. gram.:M.}
\end{itemize}
\begin{itemize}
\item {Proveniência:(De \textunderscore morrer\textunderscore )}
\end{itemize}
(que mostra o desejo de que alguma coisa acabe ou de que alguém seja morto)
O grito ou a voz, com que se pronuncia aquella interj.: \textunderscore ouviram-se alguns morras\textunderscore .
\section{Morraca}
\begin{itemize}
\item {Grp. gram.:f.}
\end{itemize}
\begin{itemize}
\item {Proveniência:(De \textunderscore morrão\textunderscore )}
\end{itemize}
Espécie de isca, feita de trapos, para accender lume.
\section{Morraça}
\begin{itemize}
\item {Grp. gram.:f.}
\end{itemize}
\begin{itemize}
\item {Utilização:Prov.}
\end{itemize}
\begin{itemize}
\item {Utilização:alg.}
\end{itemize}
\begin{itemize}
\item {Utilização:Prov.}
\end{itemize}
\begin{itemize}
\item {Utilização:beir.}
\end{itemize}
\begin{itemize}
\item {Utilização:Chul.}
\end{itemize}
Erva para alimento de gado.
Terra, que as enxurradas cobrem de uma espécie do vegetação, e que serve como de estrume.
Vegetaes nascidos no rio; moliço; rapeira.
Vinho de má qualidade.
(Por \textunderscore moraça\textunderscore , do lat. \textunderscore morum\textunderscore ?)
\section{Morraçal}
\begin{itemize}
\item {Grp. gram.:m.}
\end{itemize}
Terreno, em que há morraça.
\section{Morraçar}
\begin{itemize}
\item {Grp. gram.:v. i.}
\end{itemize}
\begin{itemize}
\item {Utilização:Prov.}
\end{itemize}
\begin{itemize}
\item {Utilização:alent.}
\end{itemize}
Caír chuva miúda.
\section{Morraceira}
\begin{itemize}
\item {Grp. gram.:f.}
\end{itemize}
\begin{itemize}
\item {Utilização:Prov.}
\end{itemize}
\begin{itemize}
\item {Utilização:minh.}
\end{itemize}
\begin{itemize}
\item {Proveniência:(De \textunderscore morraça\textunderscore )}
\end{itemize}
O mesmo que \textunderscore mouchão\textunderscore .
\section{Morralana}
\begin{itemize}
\item {Grp. gram.:f.}
\end{itemize}
Árvore da África central.
\section{Morrão}
\begin{itemize}
\item {Grp. gram.:m.}
\end{itemize}
Pedaço de corda, que se accende numa extremidade, para communicar fogo ás peças de artilharia.
Extremidade carbonizada de torcida ou mecha.
Grão, que apodrece na espiga, antes de amadurecer.
(Talvez da mesma or. que \textunderscore morraça\textunderscore )
\section{Morraria}
\begin{itemize}
\item {Grp. gram.:f.}
\end{itemize}
Série de morros.
\section{Morrediço}
\begin{itemize}
\item {Grp. gram.:adj.}
\end{itemize}
Que vai morrer; que está a acabar.
Amortecido: \textunderscore luz morrediça\textunderscore .
\section{Morredio}
\begin{itemize}
\item {Grp. gram.:adj.}
\end{itemize}
\begin{itemize}
\item {Utilização:Prov.}
\end{itemize}
\begin{itemize}
\item {Utilização:alent.}
\end{itemize}
Diz-se do animal, que morreu de morte natural e que a gente pobre aproveita para comer.
Mortezinho.
\section{Morredoiro}
\begin{itemize}
\item {Grp. gram.:adj.}
\end{itemize}
\begin{itemize}
\item {Grp. gram.:M.}
\end{itemize}
\begin{itemize}
\item {Proveniência:(Do lat. \textunderscore moriturus\textunderscore )}
\end{itemize}
Morrediço.
Decrépito.
Transitório: \textunderscore alegrias morredoiras\textunderscore .
Frágil.
Mortal.
Lugar doentio ou miasmático, em que há muitos óbitos.
\section{Morredor}
\begin{itemize}
\item {Grp. gram.:adj.}
\end{itemize}
O mesmo que \textunderscore morredoiro\textunderscore .
\section{Morredouro}
\begin{itemize}
\item {Grp. gram.:adj.}
\end{itemize}
\begin{itemize}
\item {Grp. gram.:M.}
\end{itemize}
\begin{itemize}
\item {Proveniência:(Do lat. \textunderscore moriturus\textunderscore )}
\end{itemize}
Morrediço.
Decrépito.
Transitório: \textunderscore alegrias morredoUras\textunderscore .
Frágil.
Mortal.
Lugar doentio ou miasmático, em que há muitos óbitos.
\section{Morrente}
\begin{itemize}
\item {Grp. gram.:adj.}
\end{itemize}
\begin{itemize}
\item {Utilização:Gal}
\end{itemize}
\begin{itemize}
\item {Proveniência:(Do fr. \textunderscore mourant\textunderscore )}
\end{itemize}
Que está morrendo; morrediço; moribundo.
\section{Morrer}
\begin{itemize}
\item {Grp. gram.:v. i.}
\end{itemize}
\begin{itemize}
\item {Utilização:Fig.}
\end{itemize}
\begin{itemize}
\item {Grp. gram.:V. p.}
\end{itemize}
\begin{itemize}
\item {Grp. gram.:M.}
\end{itemize}
\begin{itemize}
\item {Proveniência:(Do lat. \textunderscore morire\textunderscore )}
\end{itemize}
Deixar do viver.
Fallecer; finar-se.
Extinguir-se: \textunderscore já o Sol morria...\textunderscore 
Acabar: \textunderscore morreram aquelles amores\textunderscore .
Interromper-se, não chegar a concluir-se.
Soffrer muito.
Desaguar.
Desapparecer da memória.
Têr grande affeição a alguma coisa ou a alguém: \textunderscore a Corina morre por novenas\textunderscore .
A mesma sign. do \textunderscore v. i.\textunderscore  Cf. Castilho, \textunderscore Geórgicas\textunderscore , 207.
Morte.
\section{Morrhuol}
\begin{itemize}
\item {Grp. gram.:m.}
\end{itemize}
\begin{itemize}
\item {Utilização:Pharm.}
\end{itemize}
Extracto alcoólico do óleo do fígado de bacalhau.
\section{Morrião}
\begin{itemize}
\item {Grp. gram.:m.}
\end{itemize}
Planta primulácea, (\textunderscore anagallis arvensis\textunderscore ).
Antigo capacete sem viseira, com tope enfeitado.
(Cast. \textunderscore morrión\textunderscore )
\section{Morrinha}
\begin{itemize}
\item {Grp. gram.:f.}
\end{itemize}
\begin{itemize}
\item {Utilização:Ext.}
\end{itemize}
\begin{itemize}
\item {Utilização:Pop.}
\end{itemize}
\begin{itemize}
\item {Utilização:Bras}
\end{itemize}
\begin{itemize}
\item {Proveniência:(Do b. lat. \textunderscore morina\textunderscore )}
\end{itemize}
Sarna epidêmica do gado.
Gafeira.
Doença epidêmica dos gados.
Ligeira enfermidade.
Mau cheiro, exhalado por pessôa ou animal.
\section{Morrinha}
\begin{itemize}
\item {Grp. gram.:f.}
\end{itemize}
\begin{itemize}
\item {Utilização:Prov.}
\end{itemize}
\begin{itemize}
\item {Utilização:beir.}
\end{itemize}
O mesmo que \textunderscore molinha\textunderscore ^1.
\section{Morrinhento}
\begin{itemize}
\item {Grp. gram.:adj.}
\end{itemize}
\begin{itemize}
\item {Utilização:Pop.}
\end{itemize}
Que tem morrinha.
Enfraquecido; morredoiro; achacadiço.
\section{Morrinhoso}
\begin{itemize}
\item {Grp. gram.:adj.}
\end{itemize}
O mesmo que \textunderscore morrinhento\textunderscore .
\section{Morro}
\begin{itemize}
\item {fónica:mô}
\end{itemize}
\begin{itemize}
\item {Grp. gram.:m.}
\end{itemize}
Monte, pouco elevado.
Oiteiro.
Pedreira.
(Cast. \textunderscore morro\textunderscore )
\section{Morrudo}
\begin{itemize}
\item {Grp. gram.:adj.}
\end{itemize}
\begin{itemize}
\item {Utilização:Bras. do S}
\end{itemize}
\begin{itemize}
\item {Proveniência:(De morro)}
\end{itemize}
Muito alto ou alongado.
\section{Morruol}
\begin{itemize}
\item {Grp. gram.:m.}
\end{itemize}
\begin{itemize}
\item {Utilização:Pharm.}
\end{itemize}
Extracto alcoólico do óleo do fígado de bacalhau.
\section{Morsa}
\begin{itemize}
\item {Grp. gram.:f.}
\end{itemize}
O mesmo que \textunderscore cavallo-marinho\textunderscore .
\section{Morsegão}
\begin{itemize}
\item {Grp. gram.:m.}
\end{itemize}
\begin{itemize}
\item {Proveniência:(De \textunderscore morsegar\textunderscore )}
\end{itemize}
Bocado, que se arranca, com os dentes.
Beliscão.
\section{Morsegar}
\begin{itemize}
\item {Grp. gram.:v. t.}
\end{itemize}
\begin{itemize}
\item {Proveniência:(Lat. \textunderscore morsicare\textunderscore )}
\end{itemize}
Arrancar ou partir com os dentes.
Mordicar.
Fazer mossa em.
\section{Morso}
\begin{itemize}
\item {Grp. gram.:m.}
\end{itemize}
\begin{itemize}
\item {Utilização:Poét.}
\end{itemize}
\begin{itemize}
\item {Proveniência:(Lat. \textunderscore morsus\textunderscore )}
\end{itemize}
O mesmo que \textunderscore mordedura\textunderscore .
\section{Morso-diabólico}
\begin{itemize}
\item {Grp. gram.:m.}
\end{itemize}
Planta dipsácea, (\textunderscore succisa praemorsa\textunderscore , Gilib.).
\section{Morsó}
\begin{itemize}
\item {Grp. gram.:f.}
\end{itemize}
(V.moiçó)
\section{Morta}
\begin{itemize}
\item {Grp. gram.:f.}
\end{itemize}
Mulher defunta; cadáver de mulher.
(Cp. \textunderscore morto\textunderscore )
\section{Mortaço}
\begin{itemize}
\item {Grp. gram.:m.}
\end{itemize}
\begin{itemize}
\item {Utilização:Ant.}
\end{itemize}
\begin{itemize}
\item {Utilização:Pop.}
\end{itemize}
O mesmo que \textunderscore mortandade\textunderscore .
\section{Mortadela}
\begin{itemize}
\item {Grp. gram.:f.}
\end{itemize}
\begin{itemize}
\item {Proveniência:(It. \textunderscore mortadella\textunderscore )}
\end{itemize}
Grande chouriço, fabricado na Itália.
\section{Mortadella}
\begin{itemize}
\item {Grp. gram.:f.}
\end{itemize}
\begin{itemize}
\item {Proveniência:(It. \textunderscore mortadella\textunderscore )}
\end{itemize}
Grande chouriço, fabricado na Itália.
\section{Mortagem}
\begin{itemize}
\item {Grp. gram.:f.}
\end{itemize}
\begin{itemize}
\item {Proveniência:(De \textunderscore morto\textunderscore )}
\end{itemize}
Chanfradura, na extremidade de uma peça do madeira, para receber o topo de outra peça.
\section{Mortágua}
\begin{itemize}
\item {Grp. gram.:f.}
\end{itemize}
\begin{itemize}
\item {Proveniência:(De \textunderscore Mortágua\textunderscore , n. p.)}
\end{itemize}
Variedade de videira.
Uva dessa videira.
\section{Mortal}
\begin{itemize}
\item {Grp. gram.:adj.}
\end{itemize}
\begin{itemize}
\item {Grp. gram.:M.}
\end{itemize}
\begin{itemize}
\item {Grp. gram.:Pl.}
\end{itemize}
\begin{itemize}
\item {Proveniência:(Lat. \textunderscore mortalis\textunderscore )}
\end{itemize}
Sujeito á morte: \textunderscore todo homem é mortal\textunderscore .
Que produz morte: \textunderscore golpe mortal\textunderscore .
Moribundo, morredoiro.
Passageiro, transitório.
Figadal; profundo: \textunderscore ódio mortal\textunderscore .
Insupportável.
O homem.
A humanidade.
\section{Mortalha}
\begin{itemize}
\item {Grp. gram.:f.}
\end{itemize}
\begin{itemize}
\item {Utilização:Prov.}
\end{itemize}
\begin{itemize}
\item {Grp. gram.:Pl.}
\end{itemize}
\begin{itemize}
\item {Utilização:Ant.}
\end{itemize}
\begin{itemize}
\item {Proveniência:(Do lat. \textunderscore mortualia\textunderscore )}
\end{itemize}
Vestidura, em que se envolve o cadáver que vai sêr sepultado.
Pequena tira de papel, em que se embrulha tabaco com que se faz o cigarro.
Espécie de vestidura talar branca, que certos penitentes levam nas procissões, em cumprimento de voto.
Exéquias, funeral.
\section{Mortalidade}
\begin{itemize}
\item {Grp. gram.:f.}
\end{itemize}
\begin{itemize}
\item {Proveniência:(Lat. \textunderscore mortalitas\textunderscore )}
\end{itemize}
Qualidade do que é mortal.
Obituário.
Mortandade, carnificina.
\section{Mortalmente}
\begin{itemize}
\item {Grp. gram.:adv.}
\end{itemize}
De modo mortal; de maneira que póde causar morte; de modo que a morte póde facilmente sobrevir: \textunderscore ferido mortalmente\textunderscore .
\section{Mortandade}
\begin{itemize}
\item {Grp. gram.:f.}
\end{itemize}
Mortalidade.
Matança, carnificina.
(Cp. \textunderscore mortalidade\textunderscore )
\section{Morte}
\begin{itemize}
\item {Grp. gram.:f.}
\end{itemize}
\begin{itemize}
\item {Grp. gram.:Loc. adv.}
\end{itemize}
\begin{itemize}
\item {Grp. gram.:Loc.}
\end{itemize}
\begin{itemize}
\item {Utilização:fam.}
\end{itemize}
\begin{itemize}
\item {Grp. gram.:Loc.}
\end{itemize}
\begin{itemize}
\item {Utilização:fam.}
\end{itemize}
\begin{itemize}
\item {Proveniência:(Lat. \textunderscore mors\textunderscore , \textunderscore mortis\textunderscore )}
\end{itemize}
Acto de morrer.
Fim da vida animal ou vegetal.
Termo, fim.
Destruição.
\textunderscore De morte\textunderscore , mortalmente, com ódio profundo; rancorosamente.
\textunderscore Estar pela hora\textunderscore  ou \textunderscore horas da morte\textunderscore , sêr muito caro:«\textunderscore pago-vos duas canadas de vinho maduro e mais elle está pelas horas da morte.\textunderscore »Camillo, \textunderscore Bruxa\textunderscore , 2.^a p., c. V.
\textunderscore Morrer de morte macaca\textunderscore , morrer indecorosamente.
\textunderscore De má morte\textunderscore , de má índole, de mau carácter:«\textunderscore criticos de má morte\textunderscore ». Filinto, I, 43.
\section{Mortecor}
\begin{itemize}
\item {Grp. gram.:f.}
\end{itemize}
\begin{itemize}
\item {Proveniência:(De \textunderscore morte\textunderscore  + \textunderscore côr\textunderscore )}
\end{itemize}
Primeiras côres em obras de pintura.
\section{Morteira}
\begin{itemize}
\item {Grp. gram.:f.}
\end{itemize}
Variedade de uva preta, a mesma que \textunderscore bom-vedro\textunderscore .--Inclino-me a que deveria escrever-se \textunderscore murteira\textunderscore , de \textunderscore Murteira\textunderscore , n. p., se não de \textunderscore murta\textunderscore .
\section{Morteirada}
\begin{itemize}
\item {Grp. gram.:f.}
\end{itemize}
Tiro de morteiro.
\section{Morteirete}
\begin{itemize}
\item {fónica:teirê}
\end{itemize}
\begin{itemize}
\item {Grp. gram.:m.}
\end{itemize}
\begin{itemize}
\item {Proveniência:(De \textunderscore morteiro\textunderscore ^1)}
\end{itemize}
Antiga e pequena peça de artilharia.
\section{Morteiro}
\begin{itemize}
\item {Grp. gram.:m.}
\end{itemize}
\begin{itemize}
\item {Proveniência:(Do fr. \textunderscore mortier\textunderscore )}
\end{itemize}
Canhão curto, de bôca larga.
Pequena peça de ferro, que se ataca de pólvora, para dar tiros ou fazer explosão festiva.
\section{Morteiro}
\begin{itemize}
\item {Grp. gram.:m.}
\end{itemize}
\begin{itemize}
\item {Utilização:Ant.}
\end{itemize}
\begin{itemize}
\item {Proveniência:(Do lat. \textunderscore mortarium\textunderscore )}
\end{itemize}
Aquillo que se pisa no almofariz.
Almofariz.
Caixa de metal, em que se colloca a agulha de marear.
\section{Morte-luz}
\begin{itemize}
\item {Grp. gram.:f.}
\end{itemize}
O mesmo que \textunderscore mortecor\textunderscore . Cf. Garrett, \textunderscore Flor. sem Fruto\textunderscore , 196.
\section{Mortezinho}
\begin{itemize}
\item {Grp. gram.:m.}
\end{itemize}
\begin{itemize}
\item {Utilização:Ant.}
\end{itemize}
\begin{itemize}
\item {Grp. gram.:Adj.}
\end{itemize}
\begin{itemize}
\item {Utilização:Ant.}
\end{itemize}
\begin{itemize}
\item {Proveniência:(Do lat. \textunderscore morticinus\textunderscore )}
\end{itemize}
Cadáver.
Morto naturalmente ou sem violência.
Relativo ao animal, que morreu de morte natural, (falando-se da carne).
Cp. \textunderscore morredio\textunderscore .
\section{Mortical}
\begin{itemize}
\item {Grp. gram.:m.}
\end{itemize}
Moéda marroquina.
\section{Morticidade}
\begin{itemize}
\item {Grp. gram.:f.}
\end{itemize}
\begin{itemize}
\item {Utilização:Ant.}
\end{itemize}
Epidemia, que causa muitas mortes.
(Cp. \textunderscore morticínio\textunderscore )
\section{Morticínio}
\begin{itemize}
\item {Grp. gram.:m.}
\end{itemize}
\begin{itemize}
\item {Proveniência:(Do lat. \textunderscore morticinum\textunderscore )}
\end{itemize}
Matança, carnificina.
\section{Mortiço}
\begin{itemize}
\item {Grp. gram.:adj.}
\end{itemize}
\begin{itemize}
\item {Proveniência:(De \textunderscore morto\textunderscore )}
\end{itemize}
Morrediço.
Que está prestes a apagar-se: \textunderscore uma luz mortiça\textunderscore .
Desanimado.
\section{Mortífero}
\begin{itemize}
\item {Grp. gram.:adj.}
\end{itemize}
\begin{itemize}
\item {Proveniência:(Lat. \textunderscore mortifer\textunderscore )}
\end{itemize}
Que causa morte: \textunderscore a lança mortífera\textunderscore .
Mortal.
\section{Mortificação}
\begin{itemize}
\item {Grp. gram.:f.}
\end{itemize}
\begin{itemize}
\item {Proveniência:(Lat. \textunderscore mortificatio\textunderscore )}
\end{itemize}
Acto ou effeito de mortificar.
Afflicção; tormento.
Paralysia parcial.
\section{Mortificado}
\begin{itemize}
\item {Grp. gram.:adj.}
\end{itemize}
\begin{itemize}
\item {Proveniência:(De \textunderscore mortificar\textunderscore )}
\end{itemize}
Apoquentado; atormentado.
\section{Mortificador}
\begin{itemize}
\item {Grp. gram.:m.  e  adj.}
\end{itemize}
\begin{itemize}
\item {Proveniência:(Lat. \textunderscore mortificator\textunderscore )}
\end{itemize}
O que mortifica.
\section{Mortificante}
\begin{itemize}
\item {Grp. gram.:adj.}
\end{itemize}
\begin{itemize}
\item {Proveniência:(Lat. \textunderscore mortificans\textunderscore )}
\end{itemize}
Que mortifica.
\section{Mortificar}
\begin{itemize}
\item {Grp. gram.:v. t.}
\end{itemize}
\begin{itemize}
\item {Utilização:Fig.}
\end{itemize}
\begin{itemize}
\item {Proveniência:(Lat. \textunderscore mortificare\textunderscore )}
\end{itemize}
Deminuir a vitalidade de (alguma parte do corpo).
Atormentar; apoquentar.
Macerar com penitências; torturar.
\section{Mortificativo}
\begin{itemize}
\item {Grp. gram.:adj.}
\end{itemize}
O mesmo que \textunderscore mortificante\textunderscore .
\section{Mortilha}
\begin{itemize}
\item {Grp. gram.:f.}
\end{itemize}
\begin{itemize}
\item {Utilização:Prov.}
\end{itemize}
\begin{itemize}
\item {Proveniência:(De \textunderscore morte\textunderscore )}
\end{itemize}
O mesmo que \textunderscore matança\textunderscore  (de porcos).
\section{Mortinatalidade}
\begin{itemize}
\item {Grp. gram.:f.}
\end{itemize}
\begin{itemize}
\item {Utilização:Neol.}
\end{itemize}
\begin{itemize}
\item {Proveniência:(De \textunderscore morto\textunderscore  + \textunderscore natalidade\textunderscore )}
\end{itemize}
Conjunto dos indivíduos que nascem mortos.
\section{Mortindade}
\begin{itemize}
\item {Grp. gram.:f.}
\end{itemize}
\begin{itemize}
\item {Utilização:Ant.}
\end{itemize}
O mesmo que \textunderscore mortandade\textunderscore .
\section{Morto}
\begin{itemize}
\item {Grp. gram.:adj.}
\end{itemize}
\begin{itemize}
\item {Grp. gram.:M.}
\end{itemize}
Que deixou de viver.
Defunto.
Murcho ou sêco (falando-se de vegetaes).
Extinto: \textunderscore os mortos regimes políticos\textunderscore .
Desvanecido.
Esquecido.
Que desappareceu.
Acabado.
Profundamente dominado ou possuído (por um sentimento, por um desejo, etc.): \textunderscore morto de inveja\textunderscore .
Insensível; paralysado; em que não há movimento: \textunderscore tem um braço morto\textunderscore .
Inexpressivo.
Inerte.
Inútil.
Extremamente fatigado.
Que caíu em desuso.
Aquelle que morreu.
Aquelle que foi morto por causa externa.
Cadáver humano.
(Lat. \textunderscore mortuus\textunderscore ).
\section{Mortodas}
\begin{itemize}
\item {Grp. gram.:f. pl.}
\end{itemize}
Pérolas falsas, que se empregam no commércio com os Negros do Senegal e da Guiné.
\section{Mortona}
\begin{itemize}
\item {Grp. gram.:f.}
\end{itemize}
\begin{itemize}
\item {Utilização:Bras}
\end{itemize}
Apparelho, formado por correntes e por um carro especial, que corre sôbre longarinas em plano inclinado até ao mar, onde prende a embarcação que precisa consêrto, conduzindo-a para terra.
\section{Mortório}
\begin{itemize}
\item {Grp. gram.:m.}
\end{itemize}
\begin{itemize}
\item {Utilização:Fig.}
\end{itemize}
\begin{itemize}
\item {Proveniência:(De \textunderscore morto\textunderscore )}
\end{itemize}
Funeral.
Préstito fúnebre.
Parte das sementeiras, em que a semente não chegou a germinar.
Esquecimento, desuso.
C.p. \textunderscore mortuório\textunderscore .
\section{Mortualha}
\begin{itemize}
\item {Grp. gram.:f.}
\end{itemize}
\begin{itemize}
\item {Proveniência:(Do lat. \textunderscore mortualia\textunderscore )}
\end{itemize}
Grande porção de cadáveres.
Funeral.
\section{Mortuárias}
\begin{itemize}
\item {Grp. gram.:f. pl.}
\end{itemize}
O mesmo que \textunderscore mortulhas\textunderscore . Cf. Pacheco, \textunderscore Promptuário\textunderscore .
\section{Mortuário}
\begin{itemize}
\item {Grp. gram.:adj.}
\end{itemize}
\begin{itemize}
\item {Proveniência:(Lat. \textunderscore mortuarius\textunderscore )}
\end{itemize}
Relativo á morte ou aos mortos; fúnebre.
\section{Mortulhas}
\begin{itemize}
\item {Grp. gram.:f. pl.}
\end{itemize}
\begin{itemize}
\item {Proveniência:(De \textunderscore morto\textunderscore )}
\end{itemize}
Aquillo que dos bens de um defunto se pagava á Igreja.
\section{Mortulho}
\begin{itemize}
\item {Grp. gram.:m.}
\end{itemize}
\begin{itemize}
\item {Utilização:Pop.}
\end{itemize}
O mesmo que \textunderscore mortuório\textunderscore .
\section{Mortuório}
\begin{itemize}
\item {Grp. gram.:m.}
\end{itemize}
\begin{itemize}
\item {Proveniência:(Do lat. \textunderscore mortuus\textunderscore )}
\end{itemize}
Funeral; exéquias.
\section{Mortuoso}
\begin{itemize}
\item {Grp. gram.:adj.}
\end{itemize}
\begin{itemize}
\item {Proveniência:(Lat. \textunderscore mortuosus\textunderscore )}
\end{itemize}
O mesmo que \textunderscore cadavérico\textunderscore . Cf. Filinto, IV, 201.
\section{Morturas}
\begin{itemize}
\item {Grp. gram.:f. pl.}
\end{itemize}
\begin{itemize}
\item {Utilização:Ant.}
\end{itemize}
O mesmo que \textunderscore mortulhas\textunderscore .
\section{Morubixaba}
\begin{itemize}
\item {Grp. gram.:m.}
\end{itemize}
\begin{itemize}
\item {Utilização:Bras}
\end{itemize}
Chefe ou cacique de povoação de Indígenas.
\section{Mórula}
\begin{itemize}
\item {Grp. gram.:f.}
\end{itemize}
\begin{itemize}
\item {Proveniência:(Lat. \textunderscore morula\textunderscore )}
\end{itemize}
Pequena demora.
\section{Mórula}
\begin{itemize}
\item {Grp. gram.:f.}
\end{itemize}
\begin{itemize}
\item {Utilização:Hist. Nat.}
\end{itemize}
\begin{itemize}
\item {Proveniência:(Lat. \textunderscore morula\textunderscore ?)}
\end{itemize}
Aggregado de corpúsculos, resultante da segmentação do óvulo fecundado.
\section{Morula}
\begin{itemize}
\item {Grp. gram.:f.}
\end{itemize}
Árvore da África meridional, de frutos comestíveis, dos quaes se fabríca uma espécie de cerveja ao sul do Zambeze.
\section{Moruoni}
\begin{itemize}
\item {Grp. gram.:m.}
\end{itemize}
Canora ave indiana. Cf. Th. Ribeiro, \textunderscore Jornadas\textunderscore , II, 143.
\section{Morxama}
\begin{itemize}
\item {Grp. gram.:f.}
\end{itemize}
\begin{itemize}
\item {Utilização:Des.}
\end{itemize}
Pelle de carne de vaca com gordura.
\section{Morzello}
\begin{itemize}
\item {fónica:zê}
\end{itemize}
\begin{itemize}
\item {Grp. gram.:adj.}
\end{itemize}
\begin{itemize}
\item {Grp. gram.:M.}
\end{itemize}
Diz-se do cavallo preto, côr de amora.
Cavallo preto.
(Cast. \textunderscore morcillo\textunderscore )
\section{Morzelo}
\begin{itemize}
\item {fónica:zê}
\end{itemize}
\begin{itemize}
\item {Grp. gram.:adj.}
\end{itemize}
\begin{itemize}
\item {Grp. gram.:M.}
\end{itemize}
Diz-se do cavalo preto, côr de amora.
Cavalo preto.
(Cast. \textunderscore morcillo\textunderscore )
\section{Mosa}
\begin{itemize}
\item {Grp. gram.:f.}
\end{itemize}
Corça grande da América.
\section{Mosaico}
\begin{itemize}
\item {Grp. gram.:m.}
\end{itemize}
\begin{itemize}
\item {Utilização:Fig.}
\end{itemize}
\begin{itemize}
\item {Grp. gram.:Adj.}
\end{itemize}
Pavimento, feito de ladrilhos variegados.
Embutido, de pequenas pedras ou de outras peças, dispostas de fórma que, pela disposição das suas várias côres, dão a apparência de desenho.
Qualquer obra ou artefacto, composto de partes, visivelmente distíntas.
Miscellânea.
Feito de mosaico ou á maneira de mosaico.
(B. lat. \textunderscore mosaicum\textunderscore )
\section{Mosaísta}
\begin{itemize}
\item {Grp. gram.:m. ,  f.  e  adj.}
\end{itemize}
Pessôa, que trabalha em obras de mosaico.
(Cp. \textunderscore mosaico\textunderscore )
\section{Mosárabe}
\begin{itemize}
\item {Grp. gram.:m.  e  adj.}
\end{itemize}
(V. \textunderscore moçárabe\textunderscore , etc.)
\section{Môsca}
\begin{itemize}
\item {Grp. gram.:f.}
\end{itemize}
\begin{itemize}
\item {Utilização:Fig.}
\end{itemize}
\begin{itemize}
\item {Utilização:T. das Caldas da Raínha}
\end{itemize}
\begin{itemize}
\item {Utilização:Prov.}
\end{itemize}
\begin{itemize}
\item {Utilização:trasm.}
\end{itemize}
\begin{itemize}
\item {Utilização:Chul.}
\end{itemize}
\begin{itemize}
\item {Grp. gram.:Loc.}
\end{itemize}
\begin{itemize}
\item {Utilização:fam.}
\end{itemize}
\begin{itemize}
\item {Proveniência:(Do lat. \textunderscore musca\textunderscore )}
\end{itemize}
Gênero de insectos dípteros, que tem por typo a môsca vulgar ou môsca doméstica.
Pessôa importuna.
Pequena porção de cabellos, que alguns homens deixam crescer insuladamente por baixo do lábio inferior.
Sinal preto, usado como enfeite no rosto de algumas damas.
Pontos fortes, com que se rematam certas costuras, especialmente as casas dos botões.
Vara de videira que, torcida na ponta, se empa atando-a ao pé.
Qualquer objecto que tem a apparência de môsca.
Jôgo de cartas, entre três a seis jogadores.
O mesmo que \textunderscore dinheiro\textunderscore .
\textunderscore Asa de môsca\textunderscore , espécie de prego, o mesmo que \textunderscore faísco\textunderscore .
\textunderscore Môsca volante\textunderscore , ponto, mancha ou filamento que, em certos estados mórbidos da vista, parece collocar-se ou surgir no campo da visão.
\textunderscore Cavallo mosca\textunderscore , cavallo de pequena estatura, mas nutrido e ligeiro, o mesmo que \textunderscore mosquete\textunderscore ^2.
\textunderscore Estar ás môscas\textunderscore , estar desocupado; vazio: \textunderscore o theatro ás moscas\textunderscore .
\section{Moscada}
\begin{itemize}
\item {Grp. gram.:f.}
\end{itemize}
\begin{itemize}
\item {Proveniência:(De \textunderscore moscado\textunderscore )}
\end{itemize}
Fruto da moscadeira.
\section{Moscadeira}
\begin{itemize}
\item {Grp. gram.:f.}
\end{itemize}
\begin{itemize}
\item {Proveniência:(De \textunderscore moscada\textunderscore )}
\end{itemize}
Árvore myristicácea, (\textunderscore myristica officinalis\textunderscore ).
\section{Moscadeiro}
\begin{itemize}
\item {Grp. gram.:m.}
\end{itemize}
Utensílio, em fórma de vassoira ou de abano, para enxotar môscas.
\section{Moscado}
\begin{itemize}
\item {Grp. gram.:adj.}
\end{itemize}
\begin{itemize}
\item {Proveniência:(Do lat. \textunderscore muscatus\textunderscore )}
\end{itemize}
Almiscarado; aromático.
\section{Môsca-morta}
\begin{itemize}
\item {Grp. gram.:f.}
\end{itemize}
O mesmo que \textunderscore mosquinha-morta\textunderscore .
\section{Moscão}
\begin{itemize}
\item {Grp. gram.:m.}
\end{itemize}
\begin{itemize}
\item {Utilização:Fig.}
\end{itemize}
Grande môsca.
Pessôa sonsa.
\section{Moscar}
\begin{itemize}
\item {Grp. gram.:v. i.}
\end{itemize}
\begin{itemize}
\item {Utilização:Fig.}
\end{itemize}
Fugir das môscas, como o gado.
Sumir-se, desapparecer.
\section{Moscardo}
\begin{itemize}
\item {Grp. gram.:m.}
\end{itemize}
\begin{itemize}
\item {Utilização:Gír.}
\end{itemize}
\begin{itemize}
\item {Proveniência:(De \textunderscore môsca\textunderscore )}
\end{itemize}
Tavão; moscão.
Bofetão.
\section{Moscardo-fusco}
\begin{itemize}
\item {Grp. gram.:m.}
\end{itemize}
Designação vulgar de uma variedade de orchídeas portuguesas, (\textunderscore ophrys fusca\textunderscore , Link.). Cf. Estac. Veiga, \textunderscore Orchíd. de Portugal\textunderscore .
\section{Moscaria}
\begin{itemize}
\item {Grp. gram.:f.}
\end{itemize}
\begin{itemize}
\item {Utilização:Fam.}
\end{itemize}
Grande quantidade de môscas.
\section{Moscata}
\begin{itemize}
\item {Grp. gram.:f.}
\end{itemize}
Gênero de pólypos.
\section{Moscatel}
\begin{itemize}
\item {Grp. gram.:m.  e  adj.}
\end{itemize}
Variedade de uva, muito apreciada e de que há várias espécies, como \textunderscore moscatel preto\textunderscore  ou \textunderscore tinto\textunderscore , \textunderscore moscatel de Jesus\textunderscore , \textunderscore moscatel do Doiro\textunderscore , \textunderscore moscatel roixo\textunderscore , \textunderscore moscatel branco\textunderscore .
Vinho dessa uva.
Variedade de figo.
Variedade de pêra de verão, de cheiro almiscarado.
Variedade de maçan.
(Cp. cast. \textunderscore moscatel\textunderscore )
\section{Moscatelina}
\begin{itemize}
\item {Grp. gram.:f.}
\end{itemize}
\begin{itemize}
\item {Proveniência:(Do lat. \textunderscore muscatus\textunderscore . Cp. \textunderscore moscado\textunderscore )}
\end{itemize}
Gênero de plantas araliáceas.
\section{Moschos}
\begin{itemize}
\item {fónica:cos}
\end{itemize}
\begin{itemize}
\item {Grp. gram.:m. pl.}
\end{itemize}
\begin{itemize}
\item {Proveniência:(Lat. \textunderscore moschi\textunderscore )}
\end{itemize}
Antigo povo oriental, entre o Cáspio e o Mar-Negro:«\textunderscore ...Ruthenos, Moscos e Licónios...\textunderscore »\textunderscore Lusíadas\textunderscore , III, 11.
\section{Môsco}
\begin{itemize}
\item {Grp. gram.:m.}
\end{itemize}
\begin{itemize}
\item {Utilização:Gír.}
\end{itemize}
Roubo hábil, engenhoso, por meio de assalto a uma casa: \textunderscore ontem foi preso um gatuno de môsco.\textunderscore 
(Cp. \textunderscore mosqueiro\textunderscore )
\section{Môsco}
\begin{itemize}
\item {Grp. gram.:m.}
\end{itemize}
Môsca pequena; mosquito.
(Cp. \textunderscore môsca\textunderscore )
\section{Moscos}
\begin{itemize}
\item {fónica:cos}
\end{itemize}
\begin{itemize}
\item {Grp. gram.:m. pl.}
\end{itemize}
\begin{itemize}
\item {Proveniência:(Lat. \textunderscore moschi\textunderscore )}
\end{itemize}
Antigo povo oriental, entre o Cáspio e o Mar-Negro:«\textunderscore ...Ruthenos, Moscos e Licónios...\textunderscore »\textunderscore Lusíadas\textunderscore , III, 11.
\section{Moscou}
\begin{itemize}
\item {Grp. gram.:m.}
\end{itemize}
\begin{itemize}
\item {Proveniência:(De \textunderscore Moscou\textunderscore , n. p.)}
\end{itemize}
Espécie de tecido encorpado e fino, para fato do homem especialmente.
\section{Moscóvia}
\begin{itemize}
\item {Grp. gram.:f.}
\end{itemize}
\begin{itemize}
\item {Proveniência:(De \textunderscore Moscóvia\textunderscore , n. p., correspondente á fórma francesa, \textunderscore Moscou\textunderscore , já usual entre nós)}
\end{itemize}
Coiro, preparado na Rússia, e que serve para cobrir bahus, cadeiras, etc.
\section{Moscovita}
\begin{itemize}
\item {Grp. gram.:m.  e  adj.}
\end{itemize}
O mesmo que \textunderscore russo\textunderscore ^2.
(Cp. \textunderscore Moscóvia\textunderscore ^1)
\section{Moscovite}
\begin{itemize}
\item {Grp. gram.:f.}
\end{itemize}
\begin{itemize}
\item {Utilização:Geol.}
\end{itemize}
Uma das espécies mais communs da mica, também conhecida por \textunderscore vidro de Moscóvia\textunderscore .
\section{Moscovítico}
\begin{itemize}
\item {Grp. gram.:adj.}
\end{itemize}
Que contém moscovite.
\section{Mosequins}
\begin{itemize}
\item {Grp. gram.:m. pl.}
\end{itemize}
\begin{itemize}
\item {Utilização:Ant.}
\end{itemize}
O mesmo que [[borzeguins|borzeguim]].
\section{Mosimagão}
\begin{itemize}
\item {Grp. gram.:m.}
\end{itemize}
Festa da purificação, entre os Índios, durante a qual todos devem lavar-se nos lagos ou tanques sagrados.
\section{Moslém}
\begin{itemize}
\item {Grp. gram.:m.}
\end{itemize}
\begin{itemize}
\item {Proveniência:(Do ár. \textunderscore moslim\textunderscore )}
\end{itemize}
O mesmo que \textunderscore muçulmano\textunderscore . Cf. Herculano, \textunderscore Hist. de Port.\textunderscore , l. VII, p. I.
\section{Moslêmico}
\begin{itemize}
\item {Grp. gram.:adj.}
\end{itemize}
\begin{itemize}
\item {Proveniência:(De \textunderscore moslém\textunderscore )}
\end{itemize}
Relativo aos Muçulmanos. Cf. Herculano, \textunderscore Opúsc\textunderscore , III, 210.
\section{Moslemita}
\begin{itemize}
\item {Grp. gram.:m.}
\end{itemize}
\begin{itemize}
\item {Proveniência:(De \textunderscore moslém\textunderscore )}
\end{itemize}
Renegado, que deixou o Christianismo para abraçar o Mahometísmo.
\section{Mosocosa}
\begin{itemize}
\item {Grp. gram.:f.}
\end{itemize}
Grande árvore africana, de bôa madeira para construcções navaes.
\section{Mosocoso}
\begin{itemize}
\item {Grp. gram.:m.}
\end{itemize}
Grande árvore africana, de bôa madeira para construcções navaes.
\section{Mosqueado}
\begin{itemize}
\item {Grp. gram.:adj.}
\end{itemize}
\begin{itemize}
\item {Proveniência:(De \textunderscore mosquear\textunderscore )}
\end{itemize}
Que tem malhas escuras; sarapintado.
\section{Mosquear}
\begin{itemize}
\item {Grp. gram.:v. t.}
\end{itemize}
\begin{itemize}
\item {Proveniência:(De \textunderscore môsca\textunderscore )}
\end{itemize}
Salpicar de pintas ou manchas.
\section{Mosquedo}
\begin{itemize}
\item {fónica:quê}
\end{itemize}
\begin{itemize}
\item {Grp. gram.:m.}
\end{itemize}
Grande quantidade de môscas; moscaria.
\section{Mosqueia}
\begin{itemize}
\item {Grp. gram.:f.}
\end{itemize}
\begin{itemize}
\item {Utilização:Ant.}
\end{itemize}
\begin{itemize}
\item {Proveniência:(Fr. \textunderscore mosquée\textunderscore )}
\end{itemize}
O mesmo que \textunderscore mesquita\textunderscore . Cf. \textunderscore Jornada do Arceb. de Gôa\textunderscore .
\section{Mosqueiro}
\begin{itemize}
\item {Grp. gram.:m.}
\end{itemize}
\begin{itemize}
\item {Utilização:Gír.}
\end{itemize}
\begin{itemize}
\item {Utilização:Bras. do N}
\end{itemize}
\begin{itemize}
\item {Grp. gram.:Adj.}
\end{itemize}
\begin{itemize}
\item {Proveniência:(Do lat. \textunderscore muscarius\textunderscore )}
\end{itemize}
Lugar, onde há muitas môscas.
Utensílio ou qualquer objecto, para apanhar ou afugentar môscas.
Cobertura de arame ou de outra substância, para evitar o contacto das môscas.
O mesmo que \textunderscore mosquedo\textunderscore .
Casa.
Hospedaria reles, o mesmo que \textunderscore frege\textunderscore .
Inquieto, por causa das môscas, (falando-se de gado).
\section{Mósquem-se}
\begin{itemize}
\item {Grp. gram.:m.}
\end{itemize}
Espécie de jôgo popular.
\section{Mosqueta}
\begin{itemize}
\item {fónica:quê}
\end{itemize}
\begin{itemize}
\item {Grp. gram.:f.}
\end{itemize}
Espécie de rosa branca, de cheiro almiscarado, (\textunderscore rosa moschata\textunderscore ).
(Cast. \textunderscore mosqueta\textunderscore )
\section{Mosquetaço}
\begin{itemize}
\item {Grp. gram.:m.}
\end{itemize}
Tiro de mosquete.
\section{Mosquetada}
\begin{itemize}
\item {Grp. gram.:f.}
\end{itemize}
\begin{itemize}
\item {Proveniência:(De \textunderscore mosquete\textunderscore )}
\end{itemize}
O mesmo que \textunderscore mosquetaço\textunderscore .
Ferida que elle produz.
\section{Mosquetão}
\begin{itemize}
\item {Grp. gram.:m.}
\end{itemize}
\begin{itemize}
\item {Proveniência:(De \textunderscore môsca\textunderscore )}
\end{itemize}
Peça metállica, com que se prendem os relógios de algibeira á respectiva cadeia.
\section{Mosquetaria}
\begin{itemize}
\item {Grp. gram.:f.}
\end{itemize}
\begin{itemize}
\item {Utilização:Ext.}
\end{itemize}
\begin{itemize}
\item {Utilização:Prov.}
\end{itemize}
\begin{itemize}
\item {Utilização:trasm.}
\end{itemize}
Grande porção de mosquetes, de mosqueteiros ou de tiros de mosquete.
Tiros de espingarda, pistola ou de arma semelhante.
O mesmo que \textunderscore bofetada\textunderscore .
\section{Mosquete}
\begin{itemize}
\item {fónica:quê}
\end{itemize}
\begin{itemize}
\item {Grp. gram.:m.}
\end{itemize}
\begin{itemize}
\item {Utilização:Fam.}
\end{itemize}
Arma de fogo, do feitio da espingarda, mas mais pesada.
Tabefe, dado com as costas da mão; bofetada.
(Cp. cast. \textunderscore mosquete\textunderscore )
\section{Mosquete}
\begin{itemize}
\item {fónica:quê}
\end{itemize}
\begin{itemize}
\item {Grp. gram.:m.}
\end{itemize}
\begin{itemize}
\item {Utilização:Bras}
\end{itemize}
\begin{itemize}
\item {Proveniência:(De \textunderscore môsca\textunderscore )}
\end{itemize}
Cavallo de pequena estatura.
\section{Mosquetear}
\begin{itemize}
\item {Grp. gram.:v. t.}
\end{itemize}
\begin{itemize}
\item {Grp. gram.:V. i.}
\end{itemize}
Disparar (tiros de mosquete).
Dar tiros de mosquete.
\section{Mosqueteiro}
\begin{itemize}
\item {Grp. gram.:m.}
\end{itemize}
Soldado, armado de mosquete.
\section{Mosquinha-morta}
\begin{itemize}
\item {Grp. gram.:m.  e  f.}
\end{itemize}
Indivíduo sonso, que parece que não quebra um prato, e deita a prateleira abaixo. Cf. Camillo, \textunderscore Maria da Fonte\textunderscore , 97.
\section{Mosquir}
\begin{itemize}
\item {Grp. gram.:v. i.}
\end{itemize}
\begin{itemize}
\item {Utilização:Gír.}
\end{itemize}
O mesmo que \textunderscore moquir\textunderscore .
\section{Mosquitar}
\begin{itemize}
\item {Grp. gram.:v. i.}
\end{itemize}
\begin{itemize}
\item {Utilização:Prov.}
\end{itemize}
\begin{itemize}
\item {Utilização:beir.}
\end{itemize}
\begin{itemize}
\item {Proveniência:(De \textunderscore mosquito\textunderscore )}
\end{itemize}
O mesmo que \textunderscore moscar\textunderscore .
\section{Mosquiteiro}
\begin{itemize}
\item {Grp. gram.:m.}
\end{itemize}
Cortinado ou rêde, que resguarda dos mosquitos.
Mosqueiro.
\section{Mosquitinho}
\begin{itemize}
\item {Grp. gram.:m.}
\end{itemize}
\begin{itemize}
\item {Utilização:Bras}
\end{itemize}
\begin{itemize}
\item {Proveniência:(De \textunderscore mosquito\textunderscore )}
\end{itemize}
Abelha negra e pequena, que faz casa no chão.
\section{Mosquito}
\begin{itemize}
\item {Grp. gram.:m.}
\end{itemize}
\begin{itemize}
\item {Proveniência:(De \textunderscore môsca\textunderscore )}
\end{itemize}
Gênero de insectos dípteros, (\textunderscore culex\textunderscore ).
\section{Mosquitos}
\begin{itemize}
\item {Grp. gram.:m. pl.}
\end{itemize}
Indígenas da Mosquitia, na América central, a léste do Guatemala, os quaes formam um estado índio, sob o patronato da Inglaterra.
\section{Mossa}
\begin{itemize}
\item {Grp. gram.:m.}
\end{itemize}
\begin{itemize}
\item {Utilização:Fig.}
\end{itemize}
Vestígio de uma pancada ou pressão.
Cavidade nos dentes dos paus da canga.
Cavidade ou golpe, no fio de uma lâmina ou em qualquer superfície acutangular ou boleada.
Impressão moral.
(Por \textunderscore morsa\textunderscore , do lat. \textunderscore morsus\textunderscore )
\section{Mossar}
\begin{itemize}
\item {Grp. gram.:v. t.}
\end{itemize}
\begin{itemize}
\item {Utilização:Prov.}
\end{itemize}
\begin{itemize}
\item {Utilização:beir.}
\end{itemize}
Torcer e amachucar com as mãos (o linho, depois de maçado e antes de espadelado, para o limpar das praganas ou arestas).
(Relaciona-se com \textunderscore mossa\textunderscore ?)
\section{Mossassa}
\begin{itemize}
\item {Grp. gram.:f.}
\end{itemize}
\begin{itemize}
\item {Utilização:T. de Moçambique}
\end{itemize}
Residência de régulo ou de xeque.
Povoação, onde reside o régulo.
\section{Mossecar}
\begin{itemize}
\item {Grp. gram.:v. t.}
\end{itemize}
\begin{itemize}
\item {Utilização:T. de Lamego}
\end{itemize}
O mesmo que \textunderscore mossegar\textunderscore .
\section{Mossegar}
\textunderscore v. t.\textunderscore  (e der.)
O mesmo que \textunderscore morsegar\textunderscore , etc.
\section{Mosseguejos}
\begin{itemize}
\item {Grp. gram.:m. pl.}
\end{itemize}
Antigo povo cafreal das vizinhanças de Melinde. Cf. Couto, \textunderscore Déc.\textunderscore 
\section{Mossém}
\begin{itemize}
\item {Grp. gram.:m.}
\end{itemize}
Título, que se dava aos nobres de segunda classe, no antigo reino de Aragão.
\section{Mossiço}
\begin{itemize}
\item {Grp. gram.:adj.}
\end{itemize}
(V.massiço). Cf. Sousa, \textunderscore Vida do Arceb.\textunderscore , I, 173.
\section{Mossolemano}
\begin{itemize}
\item {Grp. gram.:m.  e  adj.}
\end{itemize}
O mesmo que \textunderscore muçulmano\textunderscore .
(Cp. \textunderscore moslém\textunderscore )
\section{Mossosso}
\begin{itemize}
\item {Grp. gram.:f.}
\end{itemize}
Pequena e elegante árvore africana, (\textunderscore entada abyssinica\textunderscore , Ol.).
\section{Mostacha}
\begin{itemize}
\item {Grp. gram.:adj. f.}
\end{itemize}
Diz-se da cera que é um cosmético para alisar e empastar o bigode. Cf. Eça, \textunderscore P. Basilio\textunderscore , 49 e 498.
(Cp. \textunderscore mostacho\textunderscore )
\section{Mostacho}
\begin{itemize}
\item {Grp. gram.:m.}
\end{itemize}
\begin{itemize}
\item {Utilização:Des.}
\end{itemize}
O mesmo que \textunderscore bigode\textunderscore ^1. Cf. \textunderscore Anat. Joc.\textunderscore , I, 7.
Bigode postiço.
Anel de cabello postiço.
(Cast. \textunderscore mostacho\textunderscore )
\section{Mostaço}
\begin{itemize}
\item {Grp. gram.:m.}
\end{itemize}
Grande quantidade de mosto.
\section{Mostajada}
\begin{itemize}
\item {Grp. gram.:f.}
\end{itemize}
\begin{itemize}
\item {Utilização:Prov.}
\end{itemize}
\begin{itemize}
\item {Utilização:beir.}
\end{itemize}
Doce de mostajo.
\section{Mostajeiro}
\begin{itemize}
\item {Grp. gram.:m.}
\end{itemize}
\begin{itemize}
\item {Utilização:Prov.}
\end{itemize}
\begin{itemize}
\item {Utilização:beir.}
\end{itemize}
\begin{itemize}
\item {Proveniência:(De \textunderscore mostajo\textunderscore )}
\end{itemize}
Planta rosácea, de talo liso e ramoso, que é a mostardeira preta, (\textunderscore pirus latifolia\textunderscore , Pers.).
\section{Mostajo}
\begin{itemize}
\item {Grp. gram.:m.}
\end{itemize}
\begin{itemize}
\item {Utilização:Prov.}
\end{itemize}
\begin{itemize}
\item {Utilização:beir.}
\end{itemize}
Fruto do mostajeiro.
(Cast. \textunderscore mostajo\textunderscore )
\section{Mostaquel}
\begin{itemize}
\item {Grp. gram.:m.}
\end{itemize}
\begin{itemize}
\item {Utilização:Prov.}
\end{itemize}
\begin{itemize}
\item {Utilização:trasm.}
\end{itemize}
O mesmo que \textunderscore moscatel\textunderscore .
(Metáth. de \textunderscore moscatel\textunderscore )
\section{Mostárabe}
\begin{itemize}
\item {Grp. gram.:m.  e  adj.}
\end{itemize}
(V. \textunderscore moçárabe\textunderscore , etc.). Cf. Herculano, \textunderscore Hist. de Port.\textunderscore , I, 54.
\section{Mostarda}
\begin{itemize}
\item {Grp. gram.:f.}
\end{itemize}
\begin{itemize}
\item {Utilização:Fig.}
\end{itemize}
\begin{itemize}
\item {Grp. gram.:Loc.}
\end{itemize}
\begin{itemize}
\item {Utilização:fam.}
\end{itemize}
Semente de mostardeira.
Mostardeira.
Farinha daquella semente.
Môlho, que se faz com esta farinha.
Estímulo.
\textunderscore Subir a mostarda ao nariz\textunderscore , haver accesso de fúria.
(Cp. it. \textunderscore mostarda\textunderscore )
\section{Mostardal}
\begin{itemize}
\item {Grp. gram.:m.}
\end{itemize}
\begin{itemize}
\item {Proveniência:(De \textunderscore mostarda\textunderscore )}
\end{itemize}
Terreno, onde crescem mostardeiras.
\section{Mostardeira}
\begin{itemize}
\item {Grp. gram.:f.}
\end{itemize}
\begin{itemize}
\item {Proveniência:(De \textunderscore mostarda\textunderscore )}
\end{itemize}
Gênero de plantas crucíferas.
Vaso, em que se serve na mesa o môlho de mostarda ou farinha da semente da mostardeira.
\section{Mostardeiro}
\begin{itemize}
\item {Grp. gram.:m.}
\end{itemize}
Vendedor de mostarda.
Vaso, o mesmo que \textunderscore mostardeira\textunderscore .
\section{Mosteia}
\begin{itemize}
\item {Grp. gram.:f.}
\end{itemize}
\begin{itemize}
\item {Utilização:Ant.}
\end{itemize}
Carro minhoto, para transporte de cereaes, e serviços de lavoira.
Carrada ou feixe de palha.--Santa-R. Viterbo manda lêr \textunderscore móstea\textunderscore .
\section{Mosteira}
\begin{itemize}
\item {Grp. gram.:adj. f.}
\end{itemize}
Diz-se vulgarmente da antiga linguagem crioulo-portuguesa de Dio.
\section{Mosteiro}
\begin{itemize}
\item {Grp. gram.:m.}
\end{itemize}
\begin{itemize}
\item {Proveniência:(Do lat. \textunderscore monasterium\textunderscore )}
\end{itemize}
Habitação de monges ou monjas; convento.
Igreja, junto da qual havia uma família, obrigada a esmolar e a hospedar frades, sacerdotes ou peregrinos.
\section{Mosteiró}
\begin{itemize}
\item {Grp. gram.:m.}
\end{itemize}
\begin{itemize}
\item {Utilização:Ant.}
\end{itemize}
Pequeno mosteiro.
(Cp. \textunderscore monesteirol\textunderscore )
\section{Mostífero}
\begin{itemize}
\item {Grp. gram.:adj.}
\end{itemize}
\begin{itemize}
\item {Proveniência:(Do lat. \textunderscore mustum\textunderscore  + \textunderscore ferre\textunderscore )}
\end{itemize}
Que produz mosto; em que há mosto.
\section{Mostil}
\begin{itemize}
\item {Grp. gram.:m.}
\end{itemize}
\begin{itemize}
\item {Utilização:Ant.}
\end{itemize}
O mesmo que \textunderscore artífice\textunderscore .
\section{Mostímetro}
\begin{itemize}
\item {Grp. gram.:m.}
\end{itemize}
O mesmo que \textunderscore gleucómetro\textunderscore .
\section{Mosto}
\begin{itemize}
\item {Grp. gram.:m.}
\end{itemize}
\begin{itemize}
\item {Proveniência:(Do lat. \textunderscore mustum\textunderscore )}
\end{itemize}
Sumo das uvas, antes de acabar a fermentação.
Suco, em fermentação, de qualquer fruta que contenha açúcar.
\section{Mostra}
\begin{itemize}
\item {Grp. gram.:f.}
\end{itemize}
\begin{itemize}
\item {Grp. gram.:Pl.}
\end{itemize}
Acto ou effeito de mostrar.
Gesto, actos exteriores; apparências.
\section{Mostrado}
\begin{itemize}
\item {Grp. gram.:adj.}
\end{itemize}
Exposto; patente.
\section{Mostrador}
\begin{itemize}
\item {Grp. gram.:adj.}
\end{itemize}
\begin{itemize}
\item {Grp. gram.:M.}
\end{itemize}
\begin{itemize}
\item {Utilização:Ant.}
\end{itemize}
\begin{itemize}
\item {Proveniência:(De \textunderscore mostrar\textunderscore )}
\end{itemize}
Que mostra.
O mesmo que \textunderscore indicador\textunderscore , (falando-se do dedo dêste nome).
Parte do relógio, em que estão indicadas as horas e as suas fracções.
Mesa ou balcão, em que nos estabelecimentos se expõem as mercadorias, ou em que se confeccionam certas drogas.
O mesmo que \textunderscore denunciante\textunderscore . Cf. F. Manuel, \textunderscore Carta de Guia\textunderscore , 9.
\section{Mostrança}
\begin{itemize}
\item {Grp. gram.:f.}
\end{itemize}
\begin{itemize}
\item {Utilização:Ant.}
\end{itemize}
\begin{itemize}
\item {Proveniência:(De \textunderscore mostrar\textunderscore )}
\end{itemize}
Apparência, exterioridade:«\textunderscore sôbre aspas fazem mostrança as quinas...\textunderscore »J. R. Sá, \textunderscore Hist. Geneal.\textunderscore , V.
\section{Mostrar}
\begin{itemize}
\item {Grp. gram.:v. t.}
\end{itemize}
\begin{itemize}
\item {Proveniência:(Lat. \textunderscore mostrare\textunderscore )}
\end{itemize}
Fazer vêr; expor á vista; apresentar.
Dar sinal de; significar: \textunderscore mostrar que soffre\textunderscore .
Demonstrar.
Apparentar.
\section{Mostrenga}
\begin{itemize}
\item {Grp. gram.:f.}
\end{itemize}
Mulher desajeitada, estafermo. Cf. Camillo, \textunderscore Maria da Fonte\textunderscore , 62.
\section{Mostrengo}
\begin{itemize}
\item {Grp. gram.:m.}
\end{itemize}
\begin{itemize}
\item {Proveniência:(De \textunderscore monstro\textunderscore )}
\end{itemize}
Pessôa desajeitada, gorda e feia.
Estafermo.
Pessôas que não tem modo do vida.
\section{Mostroiço}
\begin{itemize}
\item {Grp. gram.:m.}
\end{itemize}
\begin{itemize}
\item {Utilização:Prov.}
\end{itemize}
\begin{itemize}
\item {Utilização:beir.}
\end{itemize}
O mesmo que \textunderscore mostrunço\textunderscore .
\section{Mostruário}
\begin{itemize}
\item {Grp. gram.:m.}
\end{itemize}
\begin{itemize}
\item {Utilização:Neol.}
\end{itemize}
Lugar ou móvel, em que se expõem mercadorias ao público.
Mostrador.
Vitrina.
(Cp. \textunderscore mostrar\textunderscore )
\section{Mostrunço}
\begin{itemize}
\item {Grp. gram.:m.}
\end{itemize}
O mesmo ou melhor que \textunderscore mestrunço\textunderscore .
\section{Mota}
\begin{itemize}
\item {Grp. gram.:f.}
\end{itemize}
\begin{itemize}
\item {Utilização:T. da Azambuja}
\end{itemize}
Atêrro, com que se resguarda de inundações um campo, campos ou lugar.
Terra, que se ajunta em volta do tronco da árvore, para resguardar do calor as raízes.
O mesmo que \textunderscore arribana\textunderscore  ou curral de bois.
(B. lat. \textunderscore motta\textunderscore . Cp. cast. \textunderscore mota\textunderscore )
\section{Motacila}
\begin{itemize}
\item {Grp. gram.:f.}
\end{itemize}
\begin{itemize}
\item {Proveniência:(Lat. \textunderscore motacilla\textunderscore )}
\end{itemize}
O mesmo que \textunderscore lavandisca\textunderscore .
Ave canora dos sertões de Angola.
\section{Motacilinos}
\begin{itemize}
\item {Grp. gram.:m. pl.}
\end{itemize}
\begin{itemize}
\item {Proveniência:(De \textunderscore motacila\textunderscore )}
\end{itemize}
Grupo de pássaros dentirostros, que tem por tipo a lavandisca.
\section{Motacilla}
\begin{itemize}
\item {Grp. gram.:f.}
\end{itemize}
\begin{itemize}
\item {Proveniência:(Lat. \textunderscore motacilla\textunderscore )}
\end{itemize}
O mesmo que \textunderscore lavandisca\textunderscore .
Ave canora dos sertões de Angola.
\section{Motacillinos}
\begin{itemize}
\item {Grp. gram.:m. pl.}
\end{itemize}
\begin{itemize}
\item {Proveniência:(De \textunderscore motacilla\textunderscore )}
\end{itemize}
Grupo de pássaros dentirostros, que tem por typo a lavandisca.
\section{Motamo}
\begin{itemize}
\item {Grp. gram.:m.}
\end{itemize}
Ornato antigo. Cf. Castanheda, \textunderscore Descobr. da Índia\textunderscore , l. III, c. 42.
\section{Motano}
\begin{itemize}
\item {Grp. gram.:m.}
\end{itemize}
\begin{itemize}
\item {Utilização:Des.}
\end{itemize}
\begin{itemize}
\item {Proveniência:(De \textunderscore mota\textunderscore , se não é corr. de \textunderscore montano\textunderscore )}
\end{itemize}
Feixe de vides, ainda por atar. Cf. Filinto, XII, 31.
Lenha miúda ou paveias de mato.
\section{Motava}
\begin{itemize}
\item {Grp. gram.:f.}
\end{itemize}
Porção de fios de contas de barro vidrado, que corriam como moéda em Moçambique e equivaliam a vinte lipotes.
\section{Mote}
\begin{itemize}
\item {Grp. gram.:m.}
\end{itemize}
\begin{itemize}
\item {Utilização:Ant.}
\end{itemize}
\begin{itemize}
\item {Proveniência:(Fr. \textunderscore mot\textunderscore )}
\end{itemize}
Sentença, exposta em um ou mais versos, para servir de thema a uma estrophe ou estrophes, cujos versos finaes são os daquella sentença.
Epígraphe.
Motejo.
O mesmo que \textunderscore moto\textunderscore ^1.
\section{Motejador}
\begin{itemize}
\item {Grp. gram.:m.  e  adj.}
\end{itemize}
O que moteja.
\section{Motejar}
\begin{itemize}
\item {Grp. gram.:v. t.}
\end{itemize}
\begin{itemize}
\item {Grp. gram.:V. i.}
\end{itemize}
Fazer motejo de.
Escarnecer.
Censurar.
Dizer gracejos.
Fazer escárneo.
Satirizar.
\section{Motejar}
\begin{itemize}
\item {Grp. gram.:v. i.}
\end{itemize}
Fazer motes; dar mote para glosas. Cf. Camillo, \textunderscore Noites de Insómn.\textunderscore , IV, 92.
\section{Motejo}
\begin{itemize}
\item {Grp. gram.:m.}
\end{itemize}
\begin{itemize}
\item {Proveniência:(Do it. \textunderscore motteggio\textunderscore )}
\end{itemize}
Zombaria; gracejo.
Censura.
Dito picante.
\section{Moteneteiro}
\begin{itemize}
\item {Grp. gram.:m.  e  adj.}
\end{itemize}
\begin{itemize}
\item {Utilização:Prov.}
\end{itemize}
\begin{itemize}
\item {Utilização:trasm.}
\end{itemize}
Indivíduo que, só muito rogado, é que se resolve a acceitar qualquer coisa de comer.
\section{Moteno}
\begin{itemize}
\item {Grp. gram.:m.}
\end{itemize}
Feixe de ramos de pinheiro, empregado em aquecer os fornos de cozer pão. Cf. \textunderscore Gazeta das Ald.\textunderscore , n.^o 106.--Se não é êrro typographico, é outra fórma de \textunderscore motano\textunderscore .
\section{Motete}
\begin{itemize}
\item {fónica:tê}
\end{itemize}
\begin{itemize}
\item {Grp. gram.:m.}
\end{itemize}
\begin{itemize}
\item {Proveniência:(Do it. \textunderscore motteto\textunderscore )}
\end{itemize}
Motejo.
Trecho de música religiosa com letra.
Composição poética, para sêr cantada; cantiga.
\section{Motete}
\begin{itemize}
\item {fónica:tê}
\end{itemize}
\begin{itemize}
\item {Grp. gram.:m.}
\end{itemize}
Planta cucurbitácea do Brasil.
\section{Moteteiro}
\begin{itemize}
\item {Grp. gram.:m.  e  adj.}
\end{itemize}
\begin{itemize}
\item {Utilização:P. us.}
\end{itemize}
\begin{itemize}
\item {Proveniência:(De \textunderscore motete\textunderscore ^1)}
\end{itemize}
Aquelle que diz motetes.
\section{Motezuma}
\begin{itemize}
\item {Grp. gram.:f.}
\end{itemize}
\begin{itemize}
\item {Proveniência:(De \textunderscore Montezuma\textunderscore , n. p.)}
\end{itemize}
Gênero de plantas esterculiáceas.
\section{Moti}
\begin{itemize}
\item {Grp. gram.:m.}
\end{itemize}
\begin{itemize}
\item {Utilização:Ant.}
\end{itemize}
\begin{itemize}
\item {Proveniência:(Do conc. \textunderscore moti\textunderscore , pérole)}
\end{itemize}
Pingente de metal, que algumas mulheres asiáticas traziam na asa esquerda do nariz.
\section{Motilidade}
\begin{itemize}
\item {Grp. gram.:f.}
\end{itemize}
\begin{itemize}
\item {Proveniência:(Do lat. \textunderscore motus\textunderscore )}
\end{itemize}
Faculdade de mover.
Fôrça motriz.
\section{Motim}
\begin{itemize}
\item {Grp. gram.:m.}
\end{itemize}
Barulho.
Desordem; revolta.
Estrondo; fragor.
(Cp. cast. \textunderscore motin\textunderscore )
\section{Motinada}
\begin{itemize}
\item {Grp. gram.:f.}
\end{itemize}
Motim, alvorôto.
\section{Motinoso}
\begin{itemize}
\item {Grp. gram.:adj.}
\end{itemize}
Relativo a motim.
\section{Motira}
\begin{itemize}
\item {Grp. gram.:f.}
\end{itemize}
\begin{itemize}
\item {Utilização:Ant.}
\end{itemize}
Tranca ou fecho, com que se segura uma porta.
\section{Motivação}
\begin{itemize}
\item {Grp. gram.:f.}
\end{itemize}
Acto de motivar.
Exposição de motivos ou causas. Cf. Camillo, \textunderscore Narcót.\textunderscore , II, 253.
\section{Motivado}
\begin{itemize}
\item {Grp. gram.:adj.}
\end{itemize}
\begin{itemize}
\item {Proveniência:(De \textunderscore motivar\textunderscore )}
\end{itemize}
Causado; determinado.
\section{Motivador}
\begin{itemize}
\item {Grp. gram.:m.  e  adj.}
\end{itemize}
O que motiva.
\section{Motivar}
\begin{itemize}
\item {Grp. gram.:v. t.}
\end{itemize}
Dar motivo a.
Causar; occasionar.
Expor o motivo de.
\section{Motivo}
\begin{itemize}
\item {Grp. gram.:adj.}
\end{itemize}
\begin{itemize}
\item {Grp. gram.:M.}
\end{itemize}
\begin{itemize}
\item {Proveniência:(Lat. \textunderscore motivus\textunderscore )}
\end{itemize}
Que pode fazer mover.
Motor.
Que determina ou causa alguma coisa.
Causa; razão.
Fim, com que se faz alguma coisa.
Intuito, escopo.
Phrase predominante em qualquer composição musical.
\section{Moto}
\begin{itemize}
\item {Grp. gram.:m.}
\end{itemize}
Divisa de cavalleiros antigos.
Sinal, que um artista põe na sua obra, para indicar que é feita por êlle.
(Cp. \textunderscore mote\textunderscore )
\section{Moto}
\begin{itemize}
\item {Grp. gram.:m.}
\end{itemize}
\begin{itemize}
\item {Proveniência:(Lat. \textunderscore motus\textunderscore )}
\end{itemize}
Movimento.
Giro.
Andamento musical, mais ou menos rápido.
\section{Môto}
\begin{itemize}
\item {Grp. gram.:m.}
\end{itemize}
Planta leguminosa de Dio.
\section{Motocicleta}
\begin{itemize}
\item {fónica:clê}
\end{itemize}
\begin{itemize}
\item {Grp. gram.:f.}
\end{itemize}
\begin{itemize}
\item {Proveniência:(De \textunderscore moto\textunderscore ^2 + \textunderscore ciclo\textunderscore )}
\end{itemize}
Bicicleta com motor eléctrico.
\section{Motocycleta}
\begin{itemize}
\item {Grp. gram.:f.}
\end{itemize}
\begin{itemize}
\item {Proveniência:(De \textunderscore moto\textunderscore ^2 + \textunderscore cyclo\textunderscore )}
\end{itemize}
Bicycleta com motor eléctrico.
\section{Motolelom}
\begin{itemize}
\item {Grp. gram.:m.}
\end{itemize}
Árvore do Congo.
\section{Motor}
\begin{itemize}
\item {Grp. gram.:m.}
\end{itemize}
\begin{itemize}
\item {Grp. gram.:Adj.}
\end{itemize}
\begin{itemize}
\item {Proveniência:(Lat. \textunderscore motor\textunderscore )}
\end{itemize}
Pessôa ou coisa que faz mover ou que dá impulso.
Aquillo que induz ou aconselha.
Tudo que dá movimento a um maquinismo.
Que faz mover.
Que determina, que é causa.
\section{Motorneiro}
\begin{itemize}
\item {Grp. gram.:m.}
\end{itemize}
\begin{itemize}
\item {Utilização:Bras}
\end{itemize}
\begin{itemize}
\item {Utilização:T. do Rio de Janeiro}
\end{itemize}
Indivíduo, que dirige um motor.
Empregado, que dirige o bonde.
(Má der. de \textunderscore motor\textunderscore )
\section{Motoscópio}
\begin{itemize}
\item {Grp. gram.:m.}
\end{itemize}
\begin{itemize}
\item {Utilização:bras}
\end{itemize}
\begin{itemize}
\item {Utilização:Neol.}
\end{itemize}
\begin{itemize}
\item {Proveniência:(T. hýbr., do lat. \textunderscore motus\textunderscore  + gr. \textunderscore skopein\textunderscore )}
\end{itemize}
Espécie de cynematógrapho.
\section{Motreco}
\begin{itemize}
\item {Grp. gram.:m.}
\end{itemize}
\begin{itemize}
\item {Utilização:Pop.}
\end{itemize}
Pedaço, bocado.
\section{Motrete}
\begin{itemize}
\item {fónica:trê}
\end{itemize}
\begin{itemize}
\item {Grp. gram.:m.}
\end{itemize}
\begin{itemize}
\item {Utilização:Ant.}
\end{itemize}
Pedaço; naco:«\textunderscore sem motrete de pão, nem fome para o comer\textunderscore ». G. Vicente, I, 261.
(Cp. \textunderscore motreco\textunderscore )
\section{Motricidade}
\begin{itemize}
\item {Grp. gram.:f.}
\end{itemize}
Qualidade da força motriz.
\section{Motril}
\begin{itemize}
\item {Grp. gram.:m.}
\end{itemize}
\begin{itemize}
\item {Utilização:Prov.}
\end{itemize}
\begin{itemize}
\item {Utilização:trasm.}
\end{itemize}
Ajudante de escritório.
Fiel de feitos.
Criado desprezivel.
(Cast. \textunderscore motril\textunderscore )
\section{Motriz}
\begin{itemize}
\item {Grp. gram.:f.  e  adj.}
\end{itemize}
Coisa ou fôrça que dá movimento.
(Fem. de \textunderscore motor\textunderscore )
\section{Motuca}
\begin{itemize}
\item {Grp. gram.:f.}
\end{itemize}
\begin{itemize}
\item {Utilização:Bras. do N}
\end{itemize}
Espécie de moscardo grande, que persegue os gados, e cuja mordedura é muito dolorosa.
\section{Motucal}
\begin{itemize}
\item {Grp. gram.:m.}
\end{itemize}
Lugar, onde há muitas motucas.
\section{Motula}
\begin{itemize}
\item {Grp. gram.:f.}
\end{itemize}
\begin{itemize}
\item {Utilização:Ant.}
\end{itemize}
Torcida de candeia ou de candeeiro.
(Por \textunderscore medula\textunderscore )
\section{Motulla}
\begin{itemize}
\item {Grp. gram.:f.}
\end{itemize}
\begin{itemize}
\item {Utilização:Ant.}
\end{itemize}
Torcida de candeia ou de candeeiro.
(Por \textunderscore medulla\textunderscore )
\section{Motum}
\begin{itemize}
\item {Grp. gram.:m.}
\end{itemize}
\begin{itemize}
\item {Utilização:Bras}
\end{itemize}
Grande ave comestível, do tamanho do peru.
\section{Mótu-próprio}
\begin{itemize}
\item {Grp. gram.:m.}
\end{itemize}
Espontaneidade; vontade própria.
(Loc. lat.)
\section{Moução}
\begin{itemize}
\item {Grp. gram.:f.}
\end{itemize}
\begin{itemize}
\item {Utilização:Ant.}
\end{itemize}
O mesmo que \textunderscore monção\textunderscore ^1.
\section{Moucarrão}
\begin{itemize}
\item {Grp. gram.:adj.}
\end{itemize}
Muito mouco. Cf. \textunderscore Eufrosina\textunderscore , 208.
\section{Moucarrice}
\begin{itemize}
\item {Grp. gram.:f.}
\end{itemize}
\begin{itemize}
\item {Utilização:Chul.}
\end{itemize}
O mesmo que \textunderscore mouquice\textunderscore .
\section{Mouchão}
\begin{itemize}
\item {Grp. gram.:m.}
\end{itemize}
Pequena porção de terreno arborizado, nas lezírias, ou ilhota em meio de um rio.
\section{Mouchar}
\begin{itemize}
\item {Grp. gram.:v. t.}
\end{itemize}
\begin{itemize}
\item {Utilização:Ant.}
\end{itemize}
Formar moucho; mutilar. Cf. B. Pereira, \textunderscore Prosódia\textunderscore , vb. \textunderscore decurto\textunderscore .
\section{Moucho}
\begin{itemize}
\item {Grp. gram.:m.}
\end{itemize}
Fórma antiga de \textunderscore mocho\textunderscore . Cf. B. Pereira, \textunderscore Prosódia\textunderscore .
\section{Mouco}
\begin{itemize}
\item {Grp. gram.:adj.}
\end{itemize}
\begin{itemize}
\item {Utilização:T. da Covilhan}
\end{itemize}
\begin{itemize}
\item {Proveniência:(Do lat. \textunderscore Malchus\textunderscore , n. p.)}
\end{itemize}
Que não ouve nada; surdo.
Que ouve pouco ou mal.
Parvo, maluco.
\section{Mounúchio}
\begin{itemize}
\item {fónica:qui}
\end{itemize}
\begin{itemize}
\item {Grp. gram.:m.}
\end{itemize}
\begin{itemize}
\item {Proveniência:(Gr. \textunderscore mounukhion\textunderscore )}
\end{itemize}
Décimo mês do anno áttico, consagrado a Diana.
\section{Mounúquio}
\begin{itemize}
\item {Grp. gram.:m.}
\end{itemize}
\begin{itemize}
\item {Proveniência:(Gr. \textunderscore mounukhion\textunderscore )}
\end{itemize}
Décimo mês do ano ático, consagrado a Diana.
\section{Mouquice}
\begin{itemize}
\item {Grp. gram.:f.}
\end{itemize}
Estado de quem é mouco.
\section{Mouquidão}
\begin{itemize}
\item {Grp. gram.:f.}
\end{itemize}
O mesmo que \textunderscore mouquice\textunderscore .
\section{Moura}
\begin{itemize}
\item {Grp. gram.:f.}
\end{itemize}
Chinela de cordovão, geralmente clara. Cf. Camillo, \textunderscore Doze Casam.\textunderscore , 17.
Mulher, de procedência mourisca: \textunderscore as mouras encantadas\textunderscore .
\section{Moura}
\begin{itemize}
\item {Grp. gram.:f.}
\end{itemize}
\begin{itemize}
\item {Utilização:Prov.}
\end{itemize}
\begin{itemize}
\item {Utilização:trasm.}
\end{itemize}
\begin{itemize}
\item {Utilização:Açor}
\end{itemize}
\begin{itemize}
\item {Proveniência:(Do lat. \textunderscore muria\textunderscore )}
\end{itemize}
O mesmo que \textunderscore salga\textunderscore ^1 ou \textunderscore salmoura\textunderscore .
Chouriço de sangue; tabafeia.
Espécie de caranguejo pequeno.
\section{Mouradela}
\begin{itemize}
\item {Grp. gram.:f.}
\end{itemize}
Acto de mourar.
\section{Mouradouro}
\begin{itemize}
\item {Grp. gram.:m.}
\end{itemize}
\begin{itemize}
\item {Proveniência:(De \textunderscore mourar\textunderscore )}
\end{itemize}
Um dos compartimentos das salinas. Cf. \textunderscore Museu Techn.\textunderscore , 79.
\section{Mourama}
\begin{itemize}
\item {Grp. gram.:f.}
\end{itemize}
Terra de Mouros.
Grande porção de Mouros; os Mouros.
\section{Mourão}
\begin{itemize}
\item {Grp. gram.:m.}
\end{itemize}
Gênero de myriápodes, semelhante ao bicho de conta.
\section{Mourão}
\begin{itemize}
\item {Grp. gram.:m.}
\end{itemize}
\begin{itemize}
\item {Utilização:Prov.}
\end{itemize}
\begin{itemize}
\item {Utilização:beir.}
\end{itemize}
Pedra, que separa da lareira a pilheira.
(Por \textunderscore murão\textunderscore , de \textunderscore muro\textunderscore ?)
\section{Mourão}
\begin{itemize}
\item {Grp. gram.:m.}
\end{itemize}
O cavalleiro que vae á esquerda, no jôgo das canas.
\section{Mourão}
\begin{itemize}
\item {Grp. gram.:m.}
\end{itemize}
\begin{itemize}
\item {Utilização:Prov.}
\end{itemize}
\begin{itemize}
\item {Utilização:trasm.}
\end{itemize}
Planta crucífera, de flôr amarela, e que nasce nas vinhos e searas.
\section{Mourar}
\begin{itemize}
\item {Grp. gram.:v. i.}
\end{itemize}
\begin{itemize}
\item {Proveniência:(De \textunderscore moura\textunderscore ^2)}
\end{itemize}
Depor o sal na borda dos caldeirões, (falando-se da água salgada das marinhas).
\section{Mourar}
\begin{itemize}
\item {Grp. gram.:v. i.}
\end{itemize}
Tornar-se Mouro.
Praticar o culto do islamismo.
Trajar ou proceder como os Mouros. Cf. Serpa, \textunderscore Solaus\textunderscore , 154.
\section{Mouraria}
\begin{itemize}
\item {Grp. gram.:f.}
\end{itemize}
Bairro, em que habitavam Mouros.
\section{Mouras}
\begin{itemize}
\item {Grp. gram.:f. pl.}
\end{itemize}
\begin{itemize}
\item {Utilização:Prov.}
\end{itemize}
\begin{itemize}
\item {Utilização:alg.}
\end{itemize}
Papas de milho, com calda de murcellas, acabadas de arranjar.
\section{Mouraz}
\begin{itemize}
\item {Grp. gram.:m.}
\end{itemize}
\begin{itemize}
\item {Utilização:Ant.}
\end{itemize}
\begin{itemize}
\item {Utilização:Deprec.}
\end{itemize}
O mesmo que \textunderscore mouro\textunderscore . Cf. Sim. Mach., \textunderscore Com. de Dio\textunderscore .
\section{Mourázio}
\begin{itemize}
\item {Grp. gram.:m.}
\end{itemize}
\begin{itemize}
\item {Utilização:Ant.}
\end{itemize}
\begin{itemize}
\item {Utilização:Deprec.}
\end{itemize}
O mesmo que \textunderscore mouraz\textunderscore .
\section{Mourejado}
\begin{itemize}
\item {Grp. gram.:adj.}
\end{itemize}
\begin{itemize}
\item {Proveniência:(De \textunderscore mourejar\textunderscore )}
\end{itemize}
Conseguido á custa de muito trabalho.
\section{Mourejar}
\begin{itemize}
\item {Grp. gram.:v. i.}
\end{itemize}
\begin{itemize}
\item {Proveniência:(De \textunderscore mouro\textunderscore )}
\end{itemize}
Trabalhar constantemente; lidar.
Lutar pela vida.
\section{Mourera}
\begin{itemize}
\item {Grp. gram.:f.}
\end{itemize}
\begin{itemize}
\item {Utilização:Bot.}
\end{itemize}
Gênero de podostemáceas do Brasil. Cf. \textunderscore Jorn.-do-Comm.\textunderscore , do Rio, de 12-XI-901.
\section{Mouresco}
\begin{itemize}
\item {fónica:mourês}
\end{itemize}
\begin{itemize}
\item {Grp. gram.:adj.}
\end{itemize}
\begin{itemize}
\item {Grp. gram.:M. pl.}
\end{itemize}
\begin{itemize}
\item {Proveniência:(De \textunderscore mouro\textunderscore )}
\end{itemize}
Que é da Mourama.
Relativo a Mouros.
Ornatos de ourivezaria.
\section{Mourete}
\begin{itemize}
\item {fónica:mourê}
\end{itemize}
\begin{itemize}
\item {Grp. gram.:m.}
\end{itemize}
(Dem. de \textunderscore mouro\textunderscore ). Cf. Sim. Mach., fol. 32.
\section{Mourisca}
\begin{itemize}
\item {Grp. gram.:f.  e  adj.}
\end{itemize}
Variedade de uva preta do Douro, semelhante á periquita.
\section{Mourisca}
\begin{itemize}
\item {Grp. gram.:f.}
\end{itemize}
\begin{itemize}
\item {Utilização:Açor}
\end{itemize}
\begin{itemize}
\item {Proveniência:(De \textunderscore mourisco\textunderscore )}
\end{itemize}
Pantomima, representação ao ar livre, em trajes apropriados ao assumpto.
Antiga dança de Mouros, verdadeiros ou fingidos.
\section{Mouriscada}
\begin{itemize}
\item {Grp. gram.:f.}
\end{itemize}
\begin{itemize}
\item {Utilização:Açor}
\end{itemize}
Canção dramática, popular.
\section{Mourisco}
\begin{itemize}
\item {Grp. gram.:adj.}
\end{itemize}
\begin{itemize}
\item {Grp. gram.:M.}
\end{itemize}
Mouresco.
Mouro.
Variedade de uva, o mesmo que \textunderscore mourisca\textunderscore .
Variedade de trigo rijo.
Indivíduo da Mourama. Cf. Filinto, \textunderscore D. Man.\textunderscore , I, 23.
\section{Mourisma}
\begin{itemize}
\item {Grp. gram.:f.}
\end{itemize}
Religião dos Mouros.
Terra de Mouros.
Mourama.
(Cast. \textunderscore morisma\textunderscore )
\section{Mourismo}
\begin{itemize}
\item {Grp. gram.:m.}
\end{itemize}
\begin{itemize}
\item {Utilização:P. us.}
\end{itemize}
Os Mouros.
\section{Mouro}
\begin{itemize}
\item {Grp. gram.:adj.}
\end{itemize}
\begin{itemize}
\item {Utilização:Bras}
\end{itemize}
\begin{itemize}
\item {Grp. gram.:M.}
\end{itemize}
\begin{itemize}
\item {Utilização:Ext.}
\end{itemize}
\begin{itemize}
\item {Utilização:Fam.}
\end{itemize}
\begin{itemize}
\item {Utilização:Pop.}
\end{itemize}
\begin{itemize}
\item {Proveniência:(Do lat. \textunderscore maurus\textunderscore )}
\end{itemize}
Relativo aos Mouros.
Mourisco.
\textunderscore Cavallo mourisco\textunderscore , cavallo escuro, mesclado de branco.
\textunderscore Chouriço mouro\textunderscore , espécie de morcella, feita com sangue de porco, vinho branco, etc.
Habitante de Mauritânia.
Sarraceno.
Idólatra, infiel.
Espécie de jôgo popular.
Espécie de peixe, da ria de Aveiro.
Homem, que trabalha muito, que labuta constantemente: \textunderscore trabalhar como um mouro\textunderscore ; \textunderscore é um mouro de trabalho\textunderscore .
\textunderscore Vinho mouro\textunderscore , vinho puro, sem mistura de água, em contraposição a \textunderscore vinho cristão\textunderscore  ou \textunderscore vinho baptizado\textunderscore , vinho em que misturaram água.
\section{Mousinha}
\begin{itemize}
\item {Grp. gram.:adj. f.}
\end{itemize}
Dizia-se de uma espécie de pêra. Cf. \textunderscore Aulegrafia\textunderscore , 92.
\section{Mouta}
\begin{itemize}
\item {Grp. gram.:f.}
\end{itemize}
\begin{itemize}
\item {Utilização:Prov.}
\end{itemize}
Conjunto espêsso de plantas arborescentes.
Conjunto de castanheiros novos, que nasceram e cresceram bastos e que geralmente se applicam a corras de cesteiro e a varas com que se derruba a azeitona.
(Os outros diccion. relacionam o t. com \textunderscore mata\textunderscore ; creio porém que, assim como o lat. \textunderscore multus\textunderscore  deu o port. \textunderscore muito\textunderscore , o lat. \textunderscore multa\textunderscore , de \textunderscore multus\textunderscore , podia dar o port. \textunderscore moita\textunderscore )
\section{Mouta}
\begin{itemize}
\item {Grp. gram.:f.}
\end{itemize}
Espécie de seda crua de Bengala.
\section{Moutão}
\begin{itemize}
\item {Grp. gram.:m.}
\end{itemize}
Peça metállica ou de madeira, em fórma de ellipse, e atravessada por um eixo, cercada de goivadura, onde se introduz uma alça, e destinada a levantar pesos.
Cadernal.--A bordo, há mais de uma espécie de moutões.
(Cp. cast. \textunderscore moutón\textunderscore )
\section{Moutão}
\begin{itemize}
\item {Grp. gram.:m.}
\end{itemize}
O mesmo que \textunderscore moutedo\textunderscore .
\section{Moutedo}
\begin{itemize}
\item {fónica:tê}
\end{itemize}
\begin{itemize}
\item {Grp. gram.:m.}
\end{itemize}
Lugar onde há moutas.
\section{Mouteira}
\begin{itemize}
\item {Grp. gram.:f.}
\end{itemize}
Mouta extensa.
\section{Movediço}
\begin{itemize}
\item {Grp. gram.:adj.}
\end{itemize}
\begin{itemize}
\item {Utilização:Fig.}
\end{itemize}
Que se move facilmente.
Que tem pouca firmeza.
Instável; que não está fixo.
Portátil.
Volúvel.
\section{Movedor}
\begin{itemize}
\item {Grp. gram.:m.  e  adj.}
\end{itemize}
Aquelle ou aquillo que move; motor.
\section{Móvel}
\begin{itemize}
\item {Grp. gram.:adj.}
\end{itemize}
\begin{itemize}
\item {Grp. gram.:M.}
\end{itemize}
\begin{itemize}
\item {Grp. gram.:Pl.}
\end{itemize}
\begin{itemize}
\item {Utilização:Heráld.}
\end{itemize}
\begin{itemize}
\item {Proveniência:(Do lat. \textunderscore mobilis\textunderscore )}
\end{itemize}
Que se póde mover.
Movediço.
Causa, motivo; móbil.
Objecto de mobília.
Projéctil.
Todos os objectos materiaes, que não são bens immóveis, e todos os direitos inherentes a êsses objectos.
Objectos, que se collocam do campo do escudo e que também se chamam \textunderscore figuras\textunderscore  ou \textunderscore peças\textunderscore . (Us. também no sing) Cf. L. Ribeiro, \textunderscore Armaria\textunderscore , 69.
\section{Movente}
\begin{itemize}
\item {Grp. gram.:adj.}
\end{itemize}
\begin{itemize}
\item {Utilização:Heráld.}
\end{itemize}
\begin{itemize}
\item {Utilização:Heráld.}
\end{itemize}
\begin{itemize}
\item {Proveniência:(Do lat. \textunderscore movens\textunderscore , \textunderscore moventis\textunderscore )}
\end{itemize}
Que move.
Diz-se de algumas peças, que parecem saír do centro ou dos ângulos do escudo; móvel.
Diz-se da figura que, sendo tangente, por um dos lados, do chefe da ponta, parece que, da esquerda ou da direita, rompe de qualquer daquelles pontos para o centro ou para a parte opposta do campo.
\section{Mover}
\begin{itemize}
\item {Grp. gram.:v. t.}
\end{itemize}
\begin{itemize}
\item {Utilização:Fig.}
\end{itemize}
\begin{itemize}
\item {Utilização:Ant.}
\end{itemize}
\begin{itemize}
\item {Grp. gram.:V. i.}
\end{itemize}
\begin{itemize}
\item {Proveniência:(Lat. \textunderscore movere\textunderscore )}
\end{itemize}
Dar movimento a.
Deslocar.
Agitar.
Mexer.
Menear.
Occasionar; causar.
Estimular.
Influir em.
Alterar.
Intentar.
Commover.
Dar á luz, abortando.
Pôr-se em movimento.
Têr móvito.
\section{Movido}
\begin{itemize}
\item {Grp. gram.:adj.}
\end{itemize}
\begin{itemize}
\item {Proveniência:(De \textunderscore mover\textunderscore )}
\end{itemize}
Impellido.
Occasionado.
\section{Movimentação}
\begin{itemize}
\item {Grp. gram.:f.}
\end{itemize}
Acto do movimentar; movimento.
\section{Movimentar}
\begin{itemize}
\item {Grp. gram.:v. t.}
\end{itemize}
\begin{itemize}
\item {Utilização:Neol.}
\end{itemize}
Dar movimento a.
Pôr em movimento.
Fazer agitar em várias direcções.
\section{Movimento}
\begin{itemize}
\item {Grp. gram.:m.}
\end{itemize}
\begin{itemize}
\item {Utilização:Fig.}
\end{itemize}
\begin{itemize}
\item {Proveniência:(Lat. \textunderscore movimentum\textunderscore )}
\end{itemize}
Passagem de um corpo, ou de alguma das suas partes, de um para o outro lugar, ou de uma para outra posição.
Deslocação.
Variação, que se dá em certas quantidades.
Agitação de pessôas, que se movem em vários sentidos.
Andamento musical.
Evolução das ideias.
Transformação social.
Marcha dos corpos celestes.
Marcha de tropas.
Animação; variedade.
\section{Móvito}
\begin{itemize}
\item {Grp. gram.:m.}
\end{itemize}
\begin{itemize}
\item {Proveniência:(De \textunderscore mover\textunderscore )}
\end{itemize}
Parto prematuro; parto por abôrto.
\section{Movível}
\begin{itemize}
\item {Grp. gram.:adj.}
\end{itemize}
Que se póde mover; móvel.
\section{Mói}
\begin{itemize}
\item {Grp. gram.:m.}
\end{itemize}
Árvore da Índia portuguesa.
\section{Moxama}
\begin{itemize}
\item {Grp. gram.:f.}
\end{itemize}
Peixe sêco e salgado.
Tira sêca de lombo de atum.
(Cast. \textunderscore mojama\textunderscore )
\section{Moxameiro}
\begin{itemize}
\item {Grp. gram.:m.}
\end{itemize}
Aquelle que prepara ou vende moxama.
Lugar, onde se prepara a moxama.
\section{Moxara}
\begin{itemize}
\item {Grp. gram.:f.}
\end{itemize}
\begin{itemize}
\item {Utilização:Ant.}
\end{itemize}
Tença, na Índia portuguesa.
\section{Moxingueiro}
\begin{itemize}
\item {Grp. gram.:m.}
\end{itemize}
\begin{itemize}
\item {Utilização:Gír.}
\end{itemize}
\begin{itemize}
\item {Utilização:Prov.}
\end{itemize}
Juíz de uma prisão.
O preso, que faz a limpeza da cadeia.
\section{Moxinifada}
\begin{itemize}
\item {Grp. gram.:f.}
\end{itemize}
\begin{itemize}
\item {Utilização:Burl.}
\end{itemize}
\begin{itemize}
\item {Proveniência:(Do ár. \textunderscore mohchi\textunderscore ?)}
\end{itemize}
Salsada; miscellânea.
Mistifório.
Mistura de substâncias, que constituem um preparado pharmacêutico.
\section{Moxos}
\begin{itemize}
\item {Grp. gram.:m. pl.}
\end{itemize}
Índios selvagens, na fronteira oriental da Bolívia.
\section{Moxurunfada}
\begin{itemize}
\item {Grp. gram.:f.}
\end{itemize}
(V.moxinifada)
\section{Móy}
\begin{itemize}
\item {Grp. gram.:m.}
\end{itemize}
Árvore da Índia portuguesa.
\section{Moyér}
\begin{itemize}
\item {Grp. gram.:f.}
\end{itemize}
\begin{itemize}
\item {Utilização:Ant.}
\end{itemize}
O mesmo que \textunderscore mulhér\textunderscore .
\section{Mozabitas}
\begin{itemize}
\item {Grp. gram.:m. pl.}
\end{itemize}
Raça mesclada de Turcos e Moiros, que habita na região meridional da Berberia.
\section{Mozárabe}
\textunderscore m.\textunderscore  e \textunderscore adj.\textunderscore  (e der.)
(V. \textunderscore moçárabe\textunderscore , etc.)
\section{Mozés}
\begin{itemize}
\item {Grp. gram.:m.}
\end{itemize}
Árvore brasileira, de flôr branca em fórma de pincel.
\section{Mozeta}
\begin{itemize}
\item {fónica:zê}
\end{itemize}
\begin{itemize}
\item {Grp. gram.:f.}
\end{itemize}
Murça ecclesiástica; murça prelatícia.
(Cast. \textunderscore museta\textunderscore )
\section{Mozina}
\begin{itemize}
\item {Grp. gram.:f.}
\end{itemize}
Gênero de plantas euphorbiáceas.
\section{Mòzinha}
\begin{itemize}
\item {Grp. gram.:adj. f.}
\end{itemize}
Dizia-se de uma espécie de ameixa. Cf. F. Manuel, \textunderscore Apólogos\textunderscore .
\section{Mozmodiz}
\begin{itemize}
\item {Grp. gram.:m.}
\end{itemize}
Moéda portuguesa do tempo de Affonso Henriques, equivalente a meio morabitino.
\section{Mozom}
\begin{itemize}
\item {Grp. gram.:m.}
\end{itemize}
\begin{itemize}
\item {Utilização:Ant.}
\end{itemize}
Guindaste; roldana.
\section{Ms.}
(Abrev. de \textunderscore manuscrito\textunderscore )
\section{Mu}
\begin{itemize}
\item {Grp. gram.:m.}
\end{itemize}
\begin{itemize}
\item {Grp. gram.:Loc.}
\end{itemize}
\begin{itemize}
\item {Utilização:des.}
\end{itemize}
\begin{itemize}
\item {Proveniência:(Do lat. \textunderscore mulus\textunderscore )}
\end{itemize}
Filho de burro e égua, ou ainda o filho de cavallo e burra, se bem que, neste último caso, os téchnicos só dizem \textunderscore macho\textunderscore .
\textunderscore Tomar o mu\textunderscore , amuar, desconfiar. Cf. Gil Vicente.
\section{Mu}
\begin{itemize}
\item {Grp. gram.:m.}
\end{itemize}
Nome da letra, que no alphabeto grego corresponde ao nosso \textunderscore m\textunderscore .
\section{Mua}
\begin{itemize}
\item {Grp. gram.:f.}
\end{itemize}
\begin{itemize}
\item {Utilização:Ant.}
\end{itemize}
\begin{itemize}
\item {Proveniência:(De \textunderscore mu\textunderscore )}
\end{itemize}
O mesmo que \textunderscore mula\textunderscore ^1.
\section{Muaca}
\begin{itemize}
\item {Grp. gram.:m.}
\end{itemize}
Peixe do rio Cuanza, (\textunderscore hemichromis angolensis\textunderscore ).
\section{Muacara}
\begin{itemize}
\item {Grp. gram.:f.}
\end{itemize}
\begin{itemize}
\item {Utilização:Bras}
\end{itemize}
Pimenta vermelha.
\section{Muafo}
\begin{itemize}
\item {Grp. gram.:m.}
\end{itemize}
\begin{itemize}
\item {Utilização:Bras. do N}
\end{itemize}
Troixa de roupa.
\section{Muagi}
\begin{itemize}
\item {Grp. gram.:m.}
\end{itemize}
O mesmo que \textunderscore muavi\textunderscore .
\section{Muaianiampale}
\begin{itemize}
\item {Grp. gram.:m.}
\end{itemize}
Árvore angolense.
\section{Mualape}
\begin{itemize}
\item {Grp. gram.:m.}
\end{itemize}
Árvore angolense.
\section{Muamba}
\begin{itemize}
\item {Grp. gram.:f.}
\end{itemize}
\begin{itemize}
\item {Utilização:Bras. do N}
\end{itemize}
Velhacaria; fraude.
Patranha.
Compra e venda de objectos furtados.
(Cp. \textunderscore muambo\textunderscore ^1)
\section{Muamba}
\begin{itemize}
\item {Grp. gram.:f.}
\end{itemize}
Espécie de canastra para transporte, em África.
\section{Muambeiro}
\begin{itemize}
\item {Grp. gram.:m.}
\end{itemize}
\begin{itemize}
\item {Utilização:Bras. do N}
\end{itemize}
\begin{itemize}
\item {Proveniência:(De \textunderscore muamba\textunderscore )}
\end{itemize}
Velhaco; homem fraudulento.
Aquelle que negocia illicitamente em objectos furtados.
\section{Muambo}
\begin{itemize}
\item {Grp. gram.:m.}
\end{itemize}
Doutrina secreta, que faz parte da educação cafreal em terras de Quelimane.
\section{Muambo}
\begin{itemize}
\item {Grp. gram.:m.}
\end{itemize}
O mesmo que \textunderscore mungo\textunderscore ^1.
\section{Muance}
\begin{itemize}
\item {Grp. gram.:m.}
\end{itemize}
Grande árvore de Angola.
\section{Muandiu}
\begin{itemize}
\item {Grp. gram.:m.}
\end{itemize}
Árvore da ilha de San-Thomé.
\section{Muane}
\begin{itemize}
\item {Grp. gram.:m.}
\end{itemize}
Árvore angolense do Cazengo.
\section{Muanga}
\begin{itemize}
\item {Grp. gram.:f.}
\end{itemize}
Juramento religioso, entre os Cafres de Inhambane.
\section{Muango}
\begin{itemize}
\item {Grp. gram.:m.}
\end{itemize}
Ave africana, (\textunderscore maristeus olivaceus\textunderscore , Vieill.).
\section{Muanhi}
\begin{itemize}
\item {Grp. gram.:m.}
\end{itemize}
Arbusto africano, de caule contorcido, fôlhas oppostas em cruz, e flôres em umbellas axillares.
\section{Muanjolo}
\begin{itemize}
\item {Grp. gram.:m.}
\end{itemize}
Árvore angolense.
\section{Muanza}
\begin{itemize}
\item {Grp. gram.:f.}
\end{itemize}
Árvore africana, de casca tinctória.
\section{Muar}
\begin{itemize}
\item {Grp. gram.:adj.}
\end{itemize}
\begin{itemize}
\item {Grp. gram.:M.  e  f.}
\end{itemize}
\begin{itemize}
\item {Proveniência:(Do lat. \textunderscore mularis\textunderscore )}
\end{itemize}
Que é da raça dos mus.
Bêsta muar; mula.
\section{Muári}
\begin{itemize}
\item {Grp. gram.:f.}
\end{itemize}
Mulher principal do soba, na Lunda.
\section{Muas}
\begin{itemize}
\item {Grp. gram.:m.}
\end{itemize}
Pêso grego, correspondente a 1500 grammas.
\section{Muave}
\begin{itemize}
\item {Grp. gram.:m.}
\end{itemize}
Planta venenosa da África portuguesa, também chamada \textunderscore pau-dos-feiticeiros\textunderscore , e usada nas \textunderscore provas judiciaes\textunderscore  de algumas tríbos, fazendo-a comer ao accusado. Cf. Ficalho, \textunderscore Plantas Úteis\textunderscore , 161; Capello e Ivens, I, 349.
\section{Muavi}
\begin{itemize}
\item {Grp. gram.:m.}
\end{itemize}
Planta venenosa da África portuguesa, também chamada \textunderscore pau-dos-feiticeiros\textunderscore , e usada nas \textunderscore provas judiciaes\textunderscore  de algumas tríbos, fazendo-a comer ao accusado. Cf. Ficalho, \textunderscore Plantas Úteis\textunderscore , 161; Capello e Ivens, I, 349.
\section{Mubafo}
\begin{itemize}
\item {Grp. gram.:m.}
\end{itemize}
Majestosa e utilíssima árvore intertropical, (\textunderscore canarium-mubafo\textunderscore , Fic.), que produz bellos frutos roxos e comestíveis, e transuda resina balsâmica.
\section{Mubaladongo}
\begin{itemize}
\item {Grp. gram.:m.}
\end{itemize}
Árvore de Angola.
\section{Mubandongo}
\begin{itemize}
\item {Grp. gram.:m.}
\end{itemize}
Árvore angolense.
\section{Mubanga}
\begin{itemize}
\item {Grp. gram.:f.}
\end{itemize}
O mesmo que \textunderscore mubango\textunderscore .
\section{Mubango}
\begin{itemize}
\item {Grp. gram.:m.}
\end{itemize}
Árvore africana, ornamental, cujas fôlhas mudam de côr com as estações, e cujos frutos são semelhantes a nozes, (\textunderscore croton-mubango\textunderscore , Mul.).
\section{Mubangolule}
\begin{itemize}
\item {Grp. gram.:m.}
\end{itemize}
Árvore de Angola.
\section{Mubanja}
\begin{itemize}
\item {Grp. gram.:f.}
\end{itemize}
Árvore angolense.
\section{Mubanqui}
\begin{itemize}
\item {Grp. gram.:m.}
\end{itemize}
Árvore de Angola.
\section{Mube}
\begin{itemize}
\item {Grp. gram.:m.}
\end{itemize}
Árvore combretácea de Angola.
\section{Mubela}
\begin{itemize}
\item {Grp. gram.:f.}
\end{itemize}
Árvore de Angola.
\section{Mubongo}
\begin{itemize}
\item {Grp. gram.:m.}
\end{itemize}
Árvore angolense.
\section{Mubota}
\begin{itemize}
\item {Grp. gram.:f.}
\end{itemize}
Pequena árvore intertropical, (\textunderscore thalamiflora\textunderscore , De Cand.), que transuda goma alaranjada, e dá flôres miúdas com 5 pétalas brancas, raiadas de roxo.
\section{Mubula}
\begin{itemize}
\item {Grp. gram.:f.}
\end{itemize}
Árvore de Angola.
\section{Mubumbo}
\begin{itemize}
\item {Grp. gram.:m.}
\end{itemize}
Árvore angolense.
\section{Mubunda}
\begin{itemize}
\item {Grp. gram.:f.}
\end{itemize}
Árvore de Angola.
\section{Muçá}
\begin{itemize}
\item {Grp. gram.:m.}
\end{itemize}
\begin{itemize}
\item {Utilização:Bras. do N}
\end{itemize}
Excellente fruto silvestre, amarelo ou côr de rosa.
\section{Mucaça-mucumbi}
\begin{itemize}
\item {Grp. gram.:m.}
\end{itemize}
Árvore melliácea de Angola, (\textunderscore carapa procera\textunderscore , De Cand.).
\section{Mucadiquinho}
\begin{itemize}
\item {Grp. gram.:m.}
\end{itemize}
\begin{itemize}
\item {Utilização:Bras. de Minas}
\end{itemize}
Pedaço.
Um pouco.
Bocadinho.
\section{Muca-encaca}
\begin{itemize}
\item {Grp. gram.:f.}
\end{itemize}
Árvore do Congo.
\section{Mucajá}
\begin{itemize}
\item {Grp. gram.:m.}
\end{itemize}
Árvore silvestre do Brasil.
\section{Mucaje}
\begin{itemize}
\item {Grp. gram.:m.}
\end{itemize}
Grande árvore combretácea de Angola.
\section{Mucajé}
\begin{itemize}
\item {Grp. gram.:m.}
\end{itemize}
Fruto muito agradável dos sertões da Baía.
O mesmo que \textunderscore mucujê\textunderscore ?
\section{Mucala}
\begin{itemize}
\item {Grp. gram.:f.}
\end{itemize}
Árvore africana, de amplas fôlhas trilobadas e serreadas, e pequenas flôres amarelas e aromáticas.
(Do lund.)
\section{Mucala-cala}
\begin{itemize}
\item {Grp. gram.:f.}
\end{itemize}
Árvore africana de fôlhas inteiras, verde-amareladas, e flôres hermaphroditas, côr de castanha.
\section{Mucalate}
\begin{itemize}
\item {Grp. gram.:m.}
\end{itemize}
Árvore de Angola.
\section{Mucalina}
\begin{itemize}
\item {Grp. gram.:f.}
\end{itemize}
\begin{itemize}
\item {Utilização:T. de Ceilão}
\end{itemize}
O mesmo que \textunderscore devesa\textunderscore .
\section{Mucama}
\begin{itemize}
\item {Grp. gram.:f.}
\end{itemize}
Criada ou escrava, que na África e no Brasil acompanha a cadeirinha, em que a sua senhora sái a passeio.
Criada negra.
\section{Mucamba}
\begin{itemize}
\item {Grp. gram.:f.}
\end{itemize}
O mesmo que \textunderscore mucama\textunderscore .
\section{Mucamba}
\begin{itemize}
\item {Grp. gram.:f.}
\end{itemize}
Árvore da ilha de San-Thomé, (\textunderscore morus excelsa\textunderscore ).
\section{Mucambacamba}
\begin{itemize}
\item {Grp. gram.:f.}
\end{itemize}
Árvore angolense, o mesmo que \textunderscore mucamba\textunderscore ^2.
\section{Muçambé}
\begin{itemize}
\item {Grp. gram.:m.}
\end{itemize}
\begin{itemize}
\item {Utilização:Bras}
\end{itemize}
Planta medicinal.
\section{Mucamuca}
\begin{itemize}
\item {Grp. gram.:f.}
\end{itemize}
Espécie de loireiro do Peru.
\section{Mucanda}
\begin{itemize}
\item {Grp. gram.:f.}
\end{itemize}
Habitação dos Tus, na África.
\section{Mucanda}
\begin{itemize}
\item {Grp. gram.:f.}
\end{itemize}
\begin{itemize}
\item {Utilização:T. de Ambaca}
\end{itemize}
O mesmo que \textunderscore carta\textunderscore ^1, escrita, a lápis ou tinta, usada por negociantes e representativa de uma quantia.
\section{Mucanda}
\begin{itemize}
\item {Grp. gram.:f.}
\end{itemize}
O mesmo que \textunderscore mucando\textunderscore .
\section{Mucando}
\begin{itemize}
\item {Grp. gram.:m.}
\end{itemize}
Árvore angolense.
\section{Mucangala}
\begin{itemize}
\item {Grp. gram.:f.}
\end{itemize}
Árvore angolense.
\section{Mucange}
\begin{itemize}
\item {Grp. gram.:m.}
\end{itemize}
Árvore de Angola.
\section{Mucano}
\begin{itemize}
\item {Grp. gram.:m.}
\end{itemize}
\begin{itemize}
\item {Utilização:T. de Angola}
\end{itemize}
O mesmo que \textunderscore multa\textunderscore .
\section{Mucaraanga}
\begin{itemize}
\item {Grp. gram.:f.}
\end{itemize}
Arbusto de Moçambique, cujas cinzas os pretos utilizam como condimento.
\section{Mucarati}
\begin{itemize}
\item {Grp. gram.:m.}
\end{itemize}
Árvore tortursa dos sertões de Angola.
\section{Mucaro}
\begin{itemize}
\item {Grp. gram.:m.}
\end{itemize}
\begin{itemize}
\item {Utilização:Ant.}
\end{itemize}
O mesmo que \textunderscore almocreve\textunderscore . Cf. Pant. de Aveiro, \textunderscore Itiner.\textunderscore , 280, (2.^a ed.).
\section{Mucata}
\begin{itemize}
\item {Grp. gram.:m.}
\end{itemize}
\begin{itemize}
\item {Utilização:T. da Zambézia}
\end{itemize}
Cabo de tropa.
\section{Mucato}
\begin{itemize}
\item {Grp. gram.:m.}
\end{itemize}
\begin{itemize}
\item {Utilização:Chím.}
\end{itemize}
Sal, produzido pela combinação do ácido múcico com uma base salificável.
\section{Mucece}
\begin{itemize}
\item {Grp. gram.:m.}
\end{itemize}
Árvore de Angola.
\section{Mucedíneas}
\begin{itemize}
\item {Grp. gram.:f. pl.}
\end{itemize}
O mesmo que \textunderscore mucedíneos\textunderscore .
\section{Mucedíneos}
\begin{itemize}
\item {Grp. gram.:m. pl.}
\end{itemize}
\begin{itemize}
\item {Proveniência:(Do lat. \textunderscore mucedo\textunderscore )}
\end{itemize}
Família de cogumelos.
\section{Mucete}
\begin{itemize}
\item {Grp. gram.:m.}
\end{itemize}
Espécie de palhoça com estrado, para depósito de cereaes, entre algumas tríbos de Angola.
\section{Mucha}
\begin{itemize}
\item {Grp. gram.:f.}
\end{itemize}
Árvore de Angola.
\section{Mucha}
\begin{itemize}
\item {Utilização:T. de Angola}
\end{itemize}
Cylindro de palha, em que os sertanejos angolenses conservam o sal.
\section{Muchacha}
\begin{itemize}
\item {Grp. gram.:f.}
\end{itemize}
\begin{itemize}
\item {Utilização:Fam.}
\end{itemize}
Rapariga ladina.
Rapariga.
(Cp. \textunderscore muchacho\textunderscore )
\section{Muchacharia}
\begin{itemize}
\item {Grp. gram.:f.}
\end{itemize}
\begin{itemize}
\item {Utilização:Fam.}
\end{itemize}
\begin{itemize}
\item {Proveniência:(De \textunderscore muchacho\textunderscore )}
\end{itemize}
Grande porção de rapazes.
\section{Muchachim}
\begin{itemize}
\item {Grp. gram.:m.}
\end{itemize}
\begin{itemize}
\item {Utilização:Ant.}
\end{itemize}
\begin{itemize}
\item {Proveniência:(De \textunderscore muchacho\textunderscore )}
\end{itemize}
Rapaz, que ia nas procissões, vestido de panos variegados.
\section{Muchacho}
\begin{itemize}
\item {Grp. gram.:m.}
\end{itemize}
\begin{itemize}
\item {Utilização:Fam.}
\end{itemize}
\begin{itemize}
\item {Utilização:Bras}
\end{itemize}
\begin{itemize}
\item {Utilização:Prov.}
\end{itemize}
\begin{itemize}
\item {Utilização:trasm.}
\end{itemize}
Rapaz; garoto.
Pontalete, que sustenta horizontalmente o cabeçalho do carro, quando êste está parado.
Apparelho, para tirar dos tonéis o vinho, que já não chega á torneira.
(Cast. \textunderscore muchacho\textunderscore )
\section{Muchaço}
\begin{itemize}
\item {Grp. gram.:m.}
\end{itemize}
\begin{itemize}
\item {Utilização:Ant.}
\end{itemize}
O mesmo que \textunderscore muchacho\textunderscore . Cp. G. Vicente, \textunderscore Inês Pereira\textunderscore .
\section{Muchão}
\begin{itemize}
\item {Grp. gram.:m.}
\end{itemize}
\begin{itemize}
\item {Proveniência:(Do lat. \textunderscore mustio\textunderscore , \textunderscore mustionis\textunderscore , mosquito do vinho?)}
\end{itemize}
Insecto, espécie de mosquito, que frequenta os lugares húmidos, especialmente os lagares e adegas.
Trombeteiro.
\section{Mucharinga}
\begin{itemize}
\item {Grp. gram.:f.}
\end{itemize}
Dança popular em Abrantes, usada em vésperas da festa da Senhora da Piedade.
(Cp. \textunderscore mochachim\textunderscore )
\section{Muchém}
\begin{itemize}
\item {Grp. gram.:m.}
\end{itemize}
Nome que, na África oriental portuguesa, se dá ao salalé.
Montículo, construido pelo muchém.
\section{Muchenche}
\begin{itemize}
\item {Grp. gram.:m.}
\end{itemize}
Fruto da África central.
\section{Muchicar}
\begin{itemize}
\item {Grp. gram.:v. t.}
\end{itemize}
\begin{itemize}
\item {Utilização:Prov.}
\end{itemize}
\begin{itemize}
\item {Utilização:minh.}
\end{itemize}
O mesmo que \textunderscore morchetar\textunderscore .
\section{Muchiche}
\begin{itemize}
\item {Grp. gram.:m.}
\end{itemize}
Árvore africana, de fôlhas compostas e flôres papilionáceas, muito aromáticas.
\section{Muchinta}
\begin{itemize}
\item {Grp. gram.:f.}
\end{itemize}
Arbusto africano, herbáceo, de fôlhas pubescentes e flôres hermaphroditas, brancas.
\section{Muchir}
\begin{itemize}
\item {Grp. gram.:m.}
\end{itemize}
Título do marechal das tropas da Turquia.
\section{Muchis}
\begin{itemize}
\item {Grp. gram.:m. pl.}
\end{itemize}
Nome de dois paus, com que algumas tríbos bellicosas da África occidental, batendo-os compassadamente, conduzem os gados que roubam. Cf. Capello e Ivens, I, 35.
\section{Muchochamento}
\begin{itemize}
\item {Grp. gram.:m.}
\end{itemize}
Acto de muchochar.
\section{Muchochar}
\begin{itemize}
\item {Grp. gram.:v. i.}
\end{itemize}
\begin{itemize}
\item {Utilização:Bras}
\end{itemize}
\begin{itemize}
\item {Proveniência:(De \textunderscore muchocho\textunderscore )}
\end{itemize}
Dar beijos; fazer carícias.
\section{Muchocho}
\begin{itemize}
\item {fónica:chô}
\end{itemize}
\begin{itemize}
\item {Grp. gram.:m.}
\end{itemize}
\begin{itemize}
\item {Utilização:Bras}
\end{itemize}
\begin{itemize}
\item {Utilização:Bras. da Baía}
\end{itemize}
Beijo; carícia.
Estalo ou chiado, que se dá com a bôca, aspirando o ar e tendo a ponta da língua applicada á parte anterior do palato.
É sinal de desdem ou de contrariedade.
(Cp. \textunderscore chocho\textunderscore ^2)
\section{Mucica}
\begin{itemize}
\item {Grp. gram.:f.}
\end{itemize}
\begin{itemize}
\item {Utilização:Bras}
\end{itemize}
Empuxão, que o pescador dá á linha, quando sente que o peixe mordeu a isca.
\textunderscore Fazer mucica\textunderscore , puxar o boi pela cauda, para o derribar.
(Do tupi \textunderscore aimocic\textunderscore )
\section{Mucicão}
\begin{itemize}
\item {Grp. gram.:m.}
\end{itemize}
\begin{itemize}
\item {Utilização:Bras. do N}
\end{itemize}
\begin{itemize}
\item {Proveniência:(De \textunderscore mucica\textunderscore )}
\end{itemize}
Safanão.
Bofetada.
\section{Múcico}
\begin{itemize}
\item {Grp. gram.:adj.}
\end{itemize}
\begin{itemize}
\item {Proveniência:(Do lat. \textunderscore mucus\textunderscore )}
\end{itemize}
Diz-se de um ácido, produzido pela acção do ácido nítrico sôbre as gomas e o açúcar de leite.
\section{Mucilagem}
\begin{itemize}
\item {Grp. gram.:f.}
\end{itemize}
\begin{itemize}
\item {Proveniência:(Lat. \textunderscore mucilago\textunderscore )}
\end{itemize}
Substância gomosa e nutriente dos vegetaes.
Líquido espêsso e gomoso.
\section{Mucilaginoso}
\begin{itemize}
\item {Grp. gram.:adj.}
\end{itemize}
Diz-se das plantas, que contém mucilagem.
Que participa da natureza da mucilagem.
\section{Mucina}
\begin{itemize}
\item {Grp. gram.:f.}
\end{itemize}
\begin{itemize}
\item {Proveniência:(Do lat. \textunderscore mucus\textunderscore )}
\end{itemize}
Substância mucilaginosa, que se encontra com o glútten dos vegetaes.
\section{Mucíparo}
\begin{itemize}
\item {Grp. gram.:adj.}
\end{itemize}
\begin{itemize}
\item {Proveniência:(Do lat. \textunderscore mucus\textunderscore  + \textunderscore parere\textunderscore )}
\end{itemize}
Que produz muco.
\section{Mucito}
\begin{itemize}
\item {Grp. gram.:m.}
\end{itemize}
\begin{itemize}
\item {Utilização:Chím.}
\end{itemize}
\begin{itemize}
\item {Proveniência:(Do lat. \textunderscore mucus\textunderscore )}
\end{itemize}
Sal, produzido pela combinação do ácido mucoso com differentes bases alcalinas, térreas ou metállicas.
\section{Mucívoro}
\begin{itemize}
\item {Grp. gram.:adj.}
\end{itemize}
\begin{itemize}
\item {Proveniência:(Do lat. \textunderscore mucus\textunderscore  + \textunderscore vorare\textunderscore )}
\end{itemize}
Que se alimenta de mucosidades.
\section{Muco}
\begin{itemize}
\item {Grp. gram.:m.}
\end{itemize}
\begin{itemize}
\item {Proveniência:(Lat. \textunderscore mucus\textunderscore )}
\end{itemize}
Humidade das mucosas do nariz.
Monco.
Qualquer humor viscoso, segregado de membranas mucosas.
Árvore malvácea.
\section{Mucoáli}
\begin{itemize}
\item {Grp. gram.:m.}
\end{itemize}
\begin{itemize}
\item {Utilização:T. de Angola}
\end{itemize}
Espécie de cutello.
\section{Mucoco}
\begin{itemize}
\item {Grp. gram.:m.}
\end{itemize}
Espécie de carneiro africano.
\section{Mucoco}
\begin{itemize}
\item {Grp. gram.:m.}
\end{itemize}
Árvore de Angola, (\textunderscore cissampelos pareira\textunderscore , Lin.).
\section{Mucocolo}
\begin{itemize}
\item {Grp. gram.:m.}
\end{itemize}
(V.mucuta-veado)
\section{Mucogênio}
\begin{itemize}
\item {Grp. gram.:m.}
\end{itemize}
Medicamento purgativo.
\section{Mucol}
\begin{itemize}
\item {Grp. gram.:m.}
\end{itemize}
Mucilagem, considerada como excipiente, em pharmácia.
(Cp. \textunderscore múcico\textunderscore )
\section{Mucolito}
\begin{itemize}
\item {Grp. gram.:m.}
\end{itemize}
\begin{itemize}
\item {Proveniência:(De \textunderscore mucol\textunderscore )}
\end{itemize}
Mucilagem medicinal.
\section{Mucolo}
\begin{itemize}
\item {Grp. gram.:m.}
\end{itemize}
Arbusto africano, sarmentoso, da fam. das ampelídeas.
\section{Mucômbia}
\begin{itemize}
\item {Grp. gram.:f.}
\end{itemize}
Pássaro dentirostro.
\section{Mucombo}
\begin{itemize}
\item {Grp. gram.:m.}
\end{itemize}
Árvore angolense.
\section{Muçondo}
\begin{itemize}
\item {Grp. gram.:m.}
\end{itemize}
Árvore anacardiácea de Angola e Moçambique, (\textunderscore spondias microcarpa\textunderscore , Rich.).
\section{Mucondute}
\begin{itemize}
\item {Grp. gram.:m.}
\end{itemize}
Grande árvore africana, de bôa madeira para construcções navaes.
\section{Muçongue-alambo}
\begin{itemize}
\item {Grp. gram.:m.}
\end{itemize}
Árvore angolense de Cazengo, (\textunderscore acácia siberiana\textunderscore ).
\section{Mucor}
\begin{itemize}
\item {Grp. gram.:m.}
\end{itemize}
Gênero de cogumelos, typo das mucedíneas.
\section{Mucóreas}
\begin{itemize}
\item {Grp. gram.:f. pl.}
\end{itemize}
\begin{itemize}
\item {Utilização:Bot.}
\end{itemize}
\begin{itemize}
\item {Proveniência:(De \textunderscore mucor\textunderscore )}
\end{itemize}
Tríbo de mucedíneas.
\section{Mucoricori}
\begin{itemize}
\item {Grp. gram.:m.}
\end{itemize}
Pássaro conirostro, (\textunderscore colius castanotus\textunderscore ).
\section{Mucosa}
\begin{itemize}
\item {Grp. gram.:f.}
\end{itemize}
\begin{itemize}
\item {Utilização:Anat.}
\end{itemize}
\begin{itemize}
\item {Proveniência:(De \textunderscore mucoso\textunderscore )}
\end{itemize}
Membrana, que segrega muco.
\section{Mucosidade}
\begin{itemize}
\item {Grp. gram.:f.}
\end{itemize}
O mesmo que \textunderscore muco\textunderscore .
\section{Mucoso}
\begin{itemize}
\item {Grp. gram.:adj.}
\end{itemize}
\begin{itemize}
\item {Proveniência:(Lat. \textunderscore mucosus\textunderscore )}
\end{itemize}
Que produz muco, mucíparo.
Que tem a natureza do muco.
Diz-se da febre, que acompanha a irritação das mucosas.
\section{Mucostito}
\begin{itemize}
\item {Grp. gram.:m.}
\end{itemize}
O mesmo que \textunderscore cataplasma\textunderscore .
\section{Mucouco}
\begin{itemize}
\item {Grp. gram.:m.}
\end{itemize}
Ave africana, (\textunderscore centropus monachus\textunderscore , Rüp.).
\section{Mucro}
\begin{itemize}
\item {Grp. gram.:m.}
\end{itemize}
\begin{itemize}
\item {Utilização:Anat.}
\end{itemize}
\begin{itemize}
\item {Proveniência:(Lat. \textunderscore mucro\textunderscore )}
\end{itemize}
A extremidade xifoide do esterno.
\section{Múcron}
\begin{itemize}
\item {Grp. gram.:m.}
\end{itemize}
\begin{itemize}
\item {Utilização:Anat.}
\end{itemize}
\begin{itemize}
\item {Proveniência:(Lat. \textunderscore mucro\textunderscore )}
\end{itemize}
A extremidade xyphoide do esterno.
\section{Mucronado}
\begin{itemize}
\item {Grp. gram.:adj.}
\end{itemize}
\begin{itemize}
\item {Proveniência:(Lat. \textunderscore mucronatus.\textunderscore )}
\end{itemize}
Diz-se do órgão vegetal, que termina em ponta aguda e direita.
\section{Mucroxesse}
\begin{itemize}
\item {Grp. gram.:m.}
\end{itemize}
Árvore de Moçambique.
\section{Muçu}
\begin{itemize}
\item {Grp. gram.:m.}
\end{itemize}
Peixe do norte do Brasil.
\section{Mucual}
\begin{itemize}
\item {Grp. gram.:m.}
\end{itemize}
Arma branca de alguns povos africanos.
\section{Muçuan}
\begin{itemize}
\item {Grp. gram.:m.}
\end{itemize}
Espécie de cágado da região do Amazonas.
\section{Mucuco}
\begin{itemize}
\item {Grp. gram.:m.}
\end{itemize}
Espécie de cuco africano.
O mesmo que \textunderscore mucouco\textunderscore ?
\section{Mucuco}
\begin{itemize}
\item {Grp. gram.:m.}
\end{itemize}
Árvore de Angola.
O mesmo que \textunderscore mucoco\textunderscore ^2?
\section{Mucuço}
\begin{itemize}
\item {Grp. gram.:m.}
\end{itemize}
Árvore angolense.
\section{Mucucu}
\begin{itemize}
\item {Grp. gram.:m.}
\end{itemize}
Árvore de Moçambique.
\section{Mucuíba}
\begin{itemize}
\item {Grp. gram.:f.}
\end{itemize}
Árvore brasileira, de cujo fruto se extrai um óleo medicinal.
\section{Mucuím}
\begin{itemize}
\item {Grp. gram.:m.}
\end{itemize}
Parasita microscópico, que se encontra nos terrenos alagadiços do Brasil, e que se introduz na pelle do corpo humano.
\section{Mucuio}
\begin{itemize}
\item {Grp. gram.:m.}
\end{itemize}
Árvore angolense.
\section{Mucujê}
\begin{itemize}
\item {Grp. gram.:m.}
\end{itemize}
\begin{itemize}
\item {Utilização:Bras}
\end{itemize}
Árvore apocýnea do Brasil.
Fruta dessa árvore.
(Do tupi)
\section{Muçulmanismo}
\begin{itemize}
\item {Grp. gram.:m.}
\end{itemize}
\begin{itemize}
\item {Proveniência:(De \textunderscore muçulmano\textunderscore )}
\end{itemize}
Religião muçulmana; islamismo.
\section{Muçulmano}
\begin{itemize}
\item {Grp. gram.:adj.}
\end{itemize}
\begin{itemize}
\item {Grp. gram.:M.}
\end{itemize}
\begin{itemize}
\item {Proveniência:(Do ár. \textunderscore moslim\textunderscore  + pers. \textunderscore an\textunderscore )}
\end{itemize}
Relativo ao muçulmanismo, ou aos séctários de Mahomet.
Sectário da religião de Mahomet.
\section{Muculongoto}
\begin{itemize}
\item {Grp. gram.:m.}
\end{itemize}
Árvore de Angola.
\section{Muculuvende}
\begin{itemize}
\item {Grp. gram.:m.}
\end{itemize}
Árvore angolense.
\section{Muçum}
\begin{itemize}
\item {Grp. gram.:m.}
\end{itemize}
\begin{itemize}
\item {Utilização:Bras}
\end{itemize}
\begin{itemize}
\item {Grp. gram.:m.}
\end{itemize}
Espécie de enguia.
Peixe acanthopterýgio, da fam. dos escômbridas.
\section{Mucum}
\begin{itemize}
\item {Grp. gram.:m.}
\end{itemize}
Peixe acanthopterýgio, da fam. dos escômbridas.
\section{Mucuma}
\begin{itemize}
\item {Grp. gram.:m.}
\end{itemize}
Árvore de Angola.
\section{Mucuman}
\begin{itemize}
\item {Grp. gram.:f.}
\end{itemize}
Planta leguminosa do Brasil. Cf. \textunderscore Diár. do Congresso\textunderscore , de 11-X-900.
\section{Mucumbi-bambi}
\begin{itemize}
\item {Grp. gram.:m.}
\end{itemize}
Árvore angolense.
\section{Mucumbli}
\begin{itemize}
\item {Grp. gram.:m.}
\end{itemize}
Grande árvore medicinal da ilha de San-Thomé.
\section{Mucumbu}
\begin{itemize}
\item {Grp. gram.:m.}
\end{itemize}
\begin{itemize}
\item {Utilização:Bras. do N}
\end{itemize}
A parte da cauda do boi, que não é coberta pelas sedas.
\section{Mucuna}
\begin{itemize}
\item {Grp. gram.:f.}
\end{itemize}
Planta africana e brasileira, (\textunderscore mucuna pruriens\textunderscore ), cujas fôlhas são cobertas de um pêlo que produz comichão ao tocar-se-lhe.--Outros dizem, e talvez melhor, \textunderscore mucuná\textunderscore .
\section{Mucuná}
\begin{itemize}
\item {Grp. gram.:f.}
\end{itemize}
Nome que alguns dão á mucuna. Cf. B. C. Rubim, \textunderscore Vocab. Bras.\textunderscore 
\section{Mucunan}
\begin{itemize}
\item {Grp. gram.:f.}
\end{itemize}
O mesmo que \textunderscore mucuna\textunderscore .
\section{Mucundulo}
\begin{itemize}
\item {Grp. gram.:m.}
\end{itemize}
Árvore angolense.
\section{Mucunga}
\begin{itemize}
\item {Grp. gram.:f.}
\end{itemize}
Grande peixe escamoso da África.
\section{Mucungo}
\begin{itemize}
\item {Grp. gram.:m.}
\end{itemize}
\begin{itemize}
\item {Proveniência:(T. lund.)}
\end{itemize}
Pequena árvore africana, de fôlhas alternas, serreadas, e flôres hermaphroditas, polysépalas.
\section{Mucungungo}
\begin{itemize}
\item {Grp. gram.:m.}
\end{itemize}
Ave africana, espécie de calau.
\section{Mucunhambambe}
\begin{itemize}
\item {Grp. gram.:m.}
\end{itemize}
Árvore de Angola.
\section{Mucunzá}
\begin{itemize}
\item {Grp. gram.:m.}
\end{itemize}
\begin{itemize}
\item {Utilização:Bras. do S}
\end{itemize}
O mesmo que \textunderscore canjica\textunderscore .
\section{Mucuoca}
\begin{itemize}
\item {Grp. gram.:m.}
\end{itemize}
\begin{itemize}
\item {Utilização:Bras do N}
\end{itemize}
Tapume, feito de paus nos riachos, para alisar um pouco a corrente e facilitar a pesca.
(Do tupi \textunderscore mocooca\textunderscore )
\section{Mucura}
\begin{itemize}
\item {Grp. gram.:f.}
\end{itemize}
\begin{itemize}
\item {Utilização:Bras}
\end{itemize}
Planta gramínea.
Quadrúpede marsupial.
Prisão, enxovia.
\section{Mucuracaá}
\begin{itemize}
\item {Grp. gram.:m.}
\end{itemize}
Erva medicinal da região do Amazonas.
\section{Muçurana}
\begin{itemize}
\item {Grp. gram.:f.}
\end{itemize}
\begin{itemize}
\item {Utilização:Bras}
\end{itemize}
Cobra não venenosa, (\textunderscore rachidelus brasili\textunderscore ), que se nutre de cobras venenosas.
\section{Mucuri}
\begin{itemize}
\item {Grp. gram.:m.}
\end{itemize}
Árvore brasileira, (\textunderscore platonia\textunderscore ), própria para construcções.
\section{Mucuri}
\begin{itemize}
\item {Grp. gram.:m.}
\end{itemize}
Arbusto africano, na extremidade de cujas radículas há uns tubérculos esponjosos, ensopados num líquido, que mata a sêde.
\section{Mucuro}
\begin{itemize}
\item {Grp. gram.:m.}
\end{itemize}
Árvore angolense de Cazengo.
\section{Mucurocas}
\begin{itemize}
\item {Grp. gram.:m. pl.}
\end{itemize}
Povos de raça cafreal, em Angola.
\section{Mucurro}
\begin{itemize}
\item {Grp. gram.:m.}
\end{itemize}
\begin{itemize}
\item {Utilização:T. de Moçambique}
\end{itemize}
Pequeno esteiro ou fio de água, que corta as terras, produzido por enchentes ou grandes chuvas.
\section{Mucurulumbia}
\begin{itemize}
\item {Grp. gram.:m.}
\end{itemize}
Insecto africano, espécie de \textunderscore louva-a-deus\textunderscore .
\section{Mucusso}
\begin{itemize}
\item {Grp. gram.:m.}
\end{itemize}
Frondosa árvore angolense, da fam. das artocárpeas, (\textunderscore ficus mucuso\textunderscore , Welw.), de fôlhas cordiformes e coriáceas.
\section{Mucuta}
\begin{itemize}
\item {Grp. gram.:f.}
\end{itemize}
\begin{itemize}
\item {Utilização:Bras. de Minas}
\end{itemize}
O mesmo que \textunderscore embornal\textunderscore .
\section{Mucutaia}
\begin{itemize}
\item {Grp. gram.:f.}
\end{itemize}
Planta laurínea do Brasil.
\section{Mucuta-veado}
\begin{itemize}
\item {Grp. gram.:m.}
\end{itemize}
Planta sarmentosa, africana, da fam. das ampelídeas.
\section{Muda}
\begin{itemize}
\item {Grp. gram.:f.}
\end{itemize}
\begin{itemize}
\item {Utilização:Hortic.}
\end{itemize}
\begin{itemize}
\item {Utilização:Ant.}
\end{itemize}
\begin{itemize}
\item {Proveniência:(De \textunderscore mudar\textunderscore )}
\end{itemize}
Acto ou effeito de mudar.
Mudança.
Lugar, em que descansam os animaes, que hão de substituír os que chegam cansados, nas grandes jornadas.
Acto de substituír, uns por outros, êsses animaes.
Renovação do pêlo, pelle ou pennas de certos animaes.
Época dessa renovação.
Planta, tirada do viveiro para plantação definitiva.
Idade da puberdade. Cf. B. Pereira, \textunderscore Prosódia\textunderscore , vb. \textunderscore hirquillio\textunderscore .
\section{Muda}
\begin{itemize}
\item {Grp. gram.:f.}
\end{itemize}
\begin{itemize}
\item {Utilização:Gír.}
\end{itemize}
Mulher, que, por defeito orgânico, está privada de falar.
Mulher calada ou taciturna.
A consciência.
(Cp. \textunderscore mudo\textunderscore )
\section{Mudada}
\begin{itemize}
\item {Grp. gram.:f.}
\end{itemize}
\begin{itemize}
\item {Utilização:Ant.}
\end{itemize}
O mesmo que \textunderscore mudança\textunderscore .
\section{Mudadiço}
\begin{itemize}
\item {Grp. gram.:adj.}
\end{itemize}
O mesmo que \textunderscore mudável\textunderscore .
\section{Mudado}
\begin{itemize}
\item {Grp. gram.:adj.}
\end{itemize}
\begin{itemize}
\item {Proveniência:(De \textunderscore mudar\textunderscore )}
\end{itemize}
Transformado; differente.
\section{Mudador}
\begin{itemize}
\item {Grp. gram.:m.  e  adj.}
\end{itemize}
\begin{itemize}
\item {Grp. gram.:M.}
\end{itemize}
\begin{itemize}
\item {Utilização:Bras. do S}
\end{itemize}
Aquelle ou aquillo que muda ou causa mudança.
Lugar, protegido por pedras, arroios ou matos, onde, á míngua de curral, se reunem os cavallos, para descanso, em quanto os cavalleiros utilizam os cavallos descansados.
\section{Mudamente}
\begin{itemize}
\item {Grp. gram.:adv.}
\end{itemize}
\begin{itemize}
\item {Proveniência:(De \textunderscore mudo\textunderscore )}
\end{itemize}
Em silêncio; sem barulho.
\section{Mudamento}
\begin{itemize}
\item {Grp. gram.:m.}
\end{itemize}
\begin{itemize}
\item {Utilização:Des.}
\end{itemize}
O mesmo que \textunderscore mudança\textunderscore .
\section{Mudança}
\begin{itemize}
\item {Grp. gram.:f.}
\end{itemize}
Acto ou effeito de mudar.
\section{Mudar}
\begin{itemize}
\item {Grp. gram.:v. t.}
\end{itemize}
\begin{itemize}
\item {Grp. gram.:V. i.}
\end{itemize}
\begin{itemize}
\item {Proveniência:(Do lat. \textunderscore mutare\textunderscore )}
\end{itemize}
Mover de um lugar.
Deslocar.
Desviar.
Substituír.
Alterar.
Variar.
Transformar.
Ir viver noutro lugar.
Transformar-se.
Seguir outro rumo.
\section{Mudável}
\begin{itemize}
\item {Grp. gram.:adj.}
\end{itemize}
\begin{itemize}
\item {Utilização:Fig.}
\end{itemize}
\begin{itemize}
\item {Proveniência:(Lat. \textunderscore mutabilis\textunderscore )}
\end{itemize}
Que se póde mudar.
Sujeito a mudança.
Volúvel.
\section{Mudavelmente}
\begin{itemize}
\item {Grp. gram.:adv.}
\end{itemize}
De modo mudável.
\section{Mudbage}
\begin{itemize}
\item {Grp. gram.:f.}
\end{itemize}
\begin{itemize}
\item {Utilização:Ant.}
\end{itemize}
Tela preciosa, usada em paramentos ecclesiásticos.
\section{Mude}
\begin{itemize}
\item {Grp. gram.:f.}
\end{itemize}
Espécie de tecido chinês, fabricado com a casca de certa árvore.
\section{Mudéjar}
\begin{itemize}
\item {Grp. gram.:m.}
\end{itemize}
\begin{itemize}
\item {Grp. gram.:M. pl.}
\end{itemize}
\begin{itemize}
\item {Grp. gram.:Adj.}
\end{itemize}
\begin{itemize}
\item {Proveniência:(Do ár. \textunderscore mudejjan\textunderscore )}
\end{itemize}
Ornato architectónico de linhas rectas entrelaçadas, que têm por directriz figuras geométricas.
Designação arábica dos Moiros da Espanha, avassallados pelos Christãos.
Relativo aos mudéjares; moirisco.
Feito ao gôsto moirisco.
\section{Mudeloaquime}
\begin{itemize}
\item {Grp. gram.:m.}
\end{itemize}
Elegante arbusto africano, de fôlhas ásperas, dispostas em espigas.
\section{Mudez}
\begin{itemize}
\item {Grp. gram.:f.}
\end{itemize}
Qualidade ou estado de quem é mudo.
Impossibilidade de falar.
Serenidade, silêncio: \textunderscore na mudez da noite\textunderscore .
\section{Mudeza}
\begin{itemize}
\item {Grp. gram.:f.}
\end{itemize}
\begin{itemize}
\item {Utilização:Des.}
\end{itemize}
O mesmo que \textunderscore mudez\textunderscore .
\section{Mudiangila}
\begin{itemize}
\item {Grp. gram.:f.}
\end{itemize}
Arbusto bis-annual, africano, de caule quadrangular, fistuloso, fôlhas simples, e flôres em corymbos terminaes.
\section{Mudiangombo}
\begin{itemize}
\item {Grp. gram.:m.}
\end{itemize}
(V.mudiangila)
\section{Mudianona}
\begin{itemize}
\item {Grp. gram.:f.}
\end{itemize}
Árvore africana, de fôlhas simples e flôres amarelo-esverdeadas.
\section{Mudibirum}
\begin{itemize}
\item {Grp. gram.:m.}
\end{itemize}
Árvore angolense.
\section{Mudo}
\begin{itemize}
\item {Grp. gram.:adj.}
\end{itemize}
\begin{itemize}
\item {Utilização:Fig.}
\end{itemize}
\begin{itemize}
\item {Grp. gram.:M.}
\end{itemize}
\begin{itemize}
\item {Proveniência:(Do lat. \textunderscore mutus\textunderscore )}
\end{itemize}
Que está impossibilitado de falar por defeito orgânico ou por um accidente.
Calado: \textunderscore em-quanto falei, manteve-se mudo\textunderscore .
Silencioso; sereno.
Taciturno.
Que não tem o dom da fala.
Que se manifesta por qualquer modo que não seja o da fala.
Aquelle que está privado do dom da palavra.
Espécie de jôgo popular.
\section{Mudubim}
\begin{itemize}
\item {Grp. gram.:m.}
\end{itemize}
\begin{itemize}
\item {Utilização:Bras. do N}
\end{itemize}
O mesmo que \textunderscore amendoim\textunderscore .
\section{Mudulo}
\begin{itemize}
\item {Grp. gram.:m.}
\end{itemize}
Árvore de Angola.
\section{Mueia}
\begin{itemize}
\item {Grp. gram.:f.}
\end{itemize}
Árvore intertropical, (\textunderscore terminalia angolensis\textunderscore , Welw.).
\section{Mueiraquetan}
\begin{itemize}
\item {Grp. gram.:f.}
\end{itemize}
\begin{itemize}
\item {Utilização:Bras}
\end{itemize}
O mesmo que \textunderscore saussurite\textunderscore .
\section{Muele}
\begin{itemize}
\item {Grp. gram.:m.}
\end{itemize}
Ave gallinácea da África occidental.
\section{Muele-muele-branco}
\begin{itemize}
\item {Grp. gram.:m.}
\end{itemize}
Arbusto aphrodisíaco da ilha de San-Thomé.
\section{Mueles}
\begin{itemize}
\item {Grp. gram.:m.}
\end{itemize}
\begin{itemize}
\item {Utilização:Gír. do Pôrto.}
\end{itemize}
O mesmo que \textunderscore rapé\textunderscore .
\section{Muembrige}
\begin{itemize}
\item {Grp. gram.:f.}
\end{itemize}
Pequena árvore africana, de fôlhas alternas, ásperas, pubescentes, e flôres axillares em grandes espigas.
\section{Muenda}
\begin{itemize}
\item {Grp. gram.:f.}
\end{itemize}
Árvore de Angola.
\section{Muene}
\begin{itemize}
\item {Grp. gram.:m.}
\end{itemize}
Designação indígena de \textunderscore senhor\textunderscore , em terras portuguesas da África occidental. Cf. Capello e Ivens, I, 140.
\section{Muene-caria}
\begin{itemize}
\item {Grp. gram.:m.}
\end{itemize}
\begin{itemize}
\item {Utilização:T. de Angola}
\end{itemize}
Espécie de ministro de soba, encarregado especialmente das relações com estranhos.
\section{Mueneputo}
\begin{itemize}
\item {Grp. gram.:m.}
\end{itemize}
\begin{itemize}
\item {Utilização:T. de Angola}
\end{itemize}
Grande senhor.(V.maniputo)
\section{Mueniche}
\begin{itemize}
\item {Grp. gram.:m.}
\end{itemize}
Título de soba dos Jingas.
\section{Muenie}
\begin{itemize}
\item {Grp. gram.:m.}
\end{itemize}
Árvore de Angola.
\section{Muenque}
\begin{itemize}
\item {Grp. gram.:m.}
\end{itemize}
Árvore angolense.
\section{Mueratinge}
\begin{itemize}
\item {Grp. gram.:f.}
\end{itemize}
Árvore das regiões do Amazonas.
\section{Muere}
\begin{itemize}
\item {Grp. gram.:m.}
\end{itemize}
Árvore angolense.
\section{Mufé}
\begin{itemize}
\item {Grp. gram.:m.}
\end{itemize}
Grande árvore angolense.
\section{Mufirompepo}
\begin{itemize}
\item {Grp. gram.:m.}
\end{itemize}
Árvore angolense, de fôlhas extremamente movediças.
\section{Mufita}
\begin{itemize}
\item {Grp. gram.:f.}
\end{itemize}
Arbusto africano, de fôlhas simples, e flôres axillares em corymbos.
\section{Mufla}
\begin{itemize}
\item {Grp. gram.:f.}
\end{itemize}
\begin{itemize}
\item {Proveniência:(Fr. \textunderscore moufle\textunderscore )}
\end{itemize}
Ornato, em fórma de focinho de animal.
\section{Mufla}
\begin{itemize}
\item {Grp. gram.:f.}
\end{itemize}
\begin{itemize}
\item {Utilização:Chím.}
\end{itemize}
Vaso de barro, que serve para se sujeitarem certos corpos á acção do fogo, sem que a chamma lhes toque.
Cobertura de barro, com alguns furos, que serve para certas forjas.
(Cast. \textunderscore mufla\textunderscore )
\section{Mufti}
\begin{itemize}
\item {Grp. gram.:m.}
\end{itemize}
Chefe religioso muçulmano, que resolve em última instância as controvérsias em matéria civil e religiosa.
(Ár. \textunderscore mufti\textunderscore )
\section{Mufufuta}
\begin{itemize}
\item {Grp. gram.:f.}
\end{itemize}
Árvore africana, da fam. das leguminosas, que transuda goma alambreada, muito agglutinativa, (\textunderscore albizzia angolensis\textunderscore , Welw.).
\section{Mufuí}
\begin{itemize}
\item {Grp. gram.:m.}
\end{itemize}
Árvore angolense.
\section{Mufulanfula}
\begin{itemize}
\item {Grp. gram.:f.}
\end{itemize}
\begin{itemize}
\item {Proveniência:(T. lund.)}
\end{itemize}
Árvore africana, de fôlhas digitadas, oppostas, e flôres axillares, com o cálice côr de canela.
\section{Mufuma}
\begin{itemize}
\item {Grp. gram.:f.}
\end{itemize}
Nome africano da \textunderscore mafumeira\textunderscore .
\section{Mugalati}
\begin{itemize}
\item {Grp. gram.:m.}
\end{itemize}
Árvore de Angola.
\section{Mugambo}
\begin{itemize}
\item {Grp. gram.:m.}
\end{itemize}
Árvore de Angola.
\section{Mugande}
\begin{itemize}
\item {Grp. gram.:m.}
\end{itemize}
Árvore de Angola.
\section{Mugangalas}
\begin{itemize}
\item {Grp. gram.:m. pl.}
\end{itemize}
Tríbo de Hotentotes em Angola.
\section{Mugangue}
\begin{itemize}
\item {Grp. gram.:m.}
\end{itemize}
Pássaro africano, de asas preto-amareladas.
\section{Muge}
\begin{itemize}
\item {Grp. gram.:f.}
\end{itemize}
O mesmo que \textunderscore mugem\textunderscore .
\section{Muge}
\begin{itemize}
\item {Grp. gram.:m.}
\end{itemize}
Corrente de metal, usada ao pescoço como adôrno, na Índia e na África.
Cinto de metal ou missanga, usado por alguns Negros da Afríca oriental.
\section{Mugeira}
\begin{itemize}
\item {Grp. gram.:f.}
\end{itemize}
Espécie de rêde, para pescar a mugem.
Barco, empregado nessa pesca; o mesmo que \textunderscore saveiro\textunderscore .
\section{Mugeiro}
\begin{itemize}
\item {Grp. gram.:m.}
\end{itemize}
Espécie de águia, que pesca mugem, e é também chamada pesqueira.
\section{Mugem}
\begin{itemize}
\item {Grp. gram.:f.}
\end{itemize}
\begin{itemize}
\item {Proveniência:(Do lat. \textunderscore mugil\textunderscore )}
\end{itemize}
Gênero de peixes mugiloides.
\section{Muggletonianos}
\begin{itemize}
\item {Grp. gram.:m. pl.}
\end{itemize}
Sectários de Muggleton, que, na Inglaterra, no século XVII, pretendia passar por propheta e negava o dogma da Trindade, sustentando que o Padre Eterno, e não o Filho, é que tinha encarnado, deixando a Elias o govêrno do céu.
\section{Mugia}
\begin{itemize}
\item {Grp. gram.:f.}
\end{itemize}
Apparelho africano, para pescar peixe; espécie de nassa.
\section{Mugido}
\begin{itemize}
\item {Grp. gram.:m.}
\end{itemize}
\begin{itemize}
\item {Proveniência:(De \textunderscore mugir\textunderscore ^1)}
\end{itemize}
Voz do boi ou dos animaes bovídeos.
\section{Mugidor}
\begin{itemize}
\item {Grp. gram.:adj.}
\end{itemize}
\begin{itemize}
\item {Proveniência:(Lat. \textunderscore mugitor\textunderscore )}
\end{itemize}
Que muge.
\section{Mugiganga}
\begin{itemize}
\item {Grp. gram.:f.}
\end{itemize}
(V.bugiganga)
\section{Múgil}
\begin{itemize}
\item {Grp. gram.:m.}
\end{itemize}
\begin{itemize}
\item {Proveniência:(Lat. \textunderscore mugil\textunderscore )}
\end{itemize}
Gênero de peixes, a que pertence a muge e a tainha.
\section{Mugiloides}
\begin{itemize}
\item {Grp. gram.:m. pl.}
\end{itemize}
\begin{itemize}
\item {Proveniência:(Do lat. \textunderscore mugil\textunderscore  + gr. \textunderscore eidos\textunderscore )}
\end{itemize}
Família de peixes acanthopterýgios, que tem por typo a muge.
\section{Muginge}
\begin{itemize}
\item {Grp. gram.:m.}
\end{itemize}
Árvore de Angola e de San-Thomé.
\section{Muginha}
\begin{itemize}
\item {Grp. gram.:f.}
\end{itemize}
Nome do algodão, entre indígenas da África.
\section{Mugir}
\begin{itemize}
\item {Grp. gram.:v. i.}
\end{itemize}
\begin{itemize}
\item {Utilização:Fig.}
\end{itemize}
\begin{itemize}
\item {Proveniência:(Lat. \textunderscore mugire\textunderscore )}
\end{itemize}
Dar mugidos.
Berrar.
Bramir, estrondear, (falando-se do mar, do vento, etc.).
\section{Mugir}
\begin{itemize}
\item {Grp. gram.:v. t.}
\end{itemize}
(Corr. de \textunderscore mungir\textunderscore ). Cf. \textunderscore Luz e Calor\textunderscore , 410.
\section{Mugo}
\begin{itemize}
\item {Grp. gram.:m.}
\end{itemize}
Planta indiana, (\textunderscore phaseolus radiatus\textunderscore ).
(Do concani)
\section{Mugondo}
\begin{itemize}
\item {Grp. gram.:m.}
\end{itemize}
Árvore de Angola, no Bumbo.
\section{Mugongo}
\begin{itemize}
\item {Grp. gram.:m.}
\end{itemize}
Árvore de Angola.
O mesmo que \textunderscore mugondo\textunderscore ?
\section{Mugre}
\begin{itemize}
\item {Grp. gram.:m.}
\end{itemize}
Ferrugem dos metaes:«\textunderscore vou-me ao sino e raspo o mugre.\textunderscore »Castilho, \textunderscore Escav. Poét.\textunderscore , 261. Cf. F. Manuel, \textunderscore Apólogos\textunderscore ; Filinto, 224.
(Cp. cast. \textunderscore mugre\textunderscore , immundície)
\section{Mugueira}
\begin{itemize}
\item {Grp. gram.:f.}
\end{itemize}
(V.taínha)(Cp. \textunderscore múgil\textunderscore )
\section{Mugumbire}
\begin{itemize}
\item {Grp. gram.:m.}
\end{itemize}
Árvore de Angola.
\section{Mugunge}
\begin{itemize}
\item {Grp. gram.:m.}
\end{itemize}
Árvore de Angola, no Bumbo.
\section{Mugunzá}
\begin{itemize}
\item {Grp. gram.:m.}
\end{itemize}
\begin{itemize}
\item {Utilização:Bras. do N}
\end{itemize}
Milho cozido.
\section{Muhambo}
\begin{itemize}
\item {Grp. gram.:m.}
\end{itemize}
O mesmo que \textunderscore mungo\textunderscore ^1, ou antes \textunderscore muambo\textunderscore .
\section{Mui}
\begin{itemize}
\item {Grp. gram.:adv.}
\end{itemize}
O mesmo que \textunderscore muito\textunderscore ^1, com a differença de que só se emprega antes de adjectivos e advérbios.
(Apócope de \textunderscore muito\textunderscore )
\section{Mui}
\begin{itemize}
\item {Grp. gram.:m.}
\end{itemize}
\begin{itemize}
\item {Utilização:T. de Macau}
\end{itemize}
Espécie de ameixa. Cf. \textunderscore Ásia Sínica\textunderscore , 60.
\section{Muia-á-muia}
\begin{itemize}
\item {Grp. gram.:f.}
\end{itemize}
Robusta árvore africana, de fôlhas sempre verdes, dispersas na extremidade dos ramos, e frutos em folículos escarlates, dehiscentes.
\section{Muicanzo}
\begin{itemize}
\item {Grp. gram.:m.}
\end{itemize}
Bairro de vassallos, nas senzalas dos sobas, em Angola.
\section{Muigem}
\begin{itemize}
\item {Grp. gram.:f.}
\end{itemize}
\begin{itemize}
\item {Utilização:Ant.}
\end{itemize}
O mesmo que \textunderscore mugem\textunderscore . Cf. \textunderscore Peregrinação\textunderscore , LX.
\section{Muílo}
\begin{itemize}
\item {Grp. gram.:m.}
\end{itemize}
Árvore de Angola.
\section{Mui-muito}
\begin{itemize}
\item {Grp. gram.:adv.}
\end{itemize}
\begin{itemize}
\item {Utilização:Ant.}
\end{itemize}
O mesmo que muitíssimo:«\textunderscore mui-muito me espanto eu.\textunderscore »G. Vicente, \textunderscore Auto da Lusit.\textunderscore 
\section{Muinda}
\begin{itemize}
\item {Grp. gram.:f.}
\end{itemize}
Árvore medicinal da ilha de San-Thomé.
\section{Muindo}
\begin{itemize}
\item {Grp. gram.:m.}
\end{itemize}
O mesmo que \textunderscore muinda\textunderscore .
\section{Muinge}
\begin{itemize}
\item {Grp. gram.:m.}
\end{itemize}
Árvore de Angola, (\textunderscore ximenea americana\textunderscore , Lin.).
\section{Muinzique}
\begin{itemize}
\item {Grp. gram.:m.}
\end{itemize}
Arbusto africano, da fam. das leguminosas.
\section{Muiracatiara}
\begin{itemize}
\item {fónica:mu-i}
\end{itemize}
\begin{itemize}
\item {Grp. gram.:f.}
\end{itemize}
Árvore brasileira, (\textunderscore centrolobium\textunderscore ).
\section{Muirajanara}
\begin{itemize}
\item {fónica:mu-i}
\end{itemize}
\begin{itemize}
\item {Grp. gram.:f.}
\end{itemize}
Árvore brasileira, empregada em construcções.
\section{Muirapinima}
\begin{itemize}
\item {fónica:mu-i}
\end{itemize}
\begin{itemize}
\item {Grp. gram.:f.}
\end{itemize}
Árvore artocárpea do Brasil, (\textunderscore brosimum aubletii\textunderscore ).
\section{Muirapiranga}
\begin{itemize}
\item {fónica:mu-i}
\end{itemize}
\begin{itemize}
\item {Grp. gram.:f.}
\end{itemize}
Árvore leguminosa do Brasil.
\section{Muirapiririca}
\begin{itemize}
\item {fónica:mu-i}
\end{itemize}
\begin{itemize}
\item {Grp. gram.:f.}
\end{itemize}
Árvore brasileira, empregada em construcções.
\section{Muirapuama}
\begin{itemize}
\item {Grp. gram.:f.}
\end{itemize}
\begin{itemize}
\item {Utilização:Bras}
\end{itemize}
O mesmo que \textunderscore marapuama\textunderscore .
\section{Muiratinga}
\begin{itemize}
\item {Grp. gram.:f.}
\end{itemize}
\begin{itemize}
\item {Utilização:Bras}
\end{itemize}
Árvore equatorial do Brasil.
\section{Muiteramá}
\begin{itemize}
\item {Grp. gram.:adv.}
\end{itemize}
Com muito má hora; embora. Cp. G. Vicente, \textunderscore Inês Pereira\textunderscore .
\section{Muitíssimo}
\begin{itemize}
\item {Grp. gram.:adv.}
\end{itemize}
\begin{itemize}
\item {Proveniência:(De \textunderscore muito\textunderscore )}
\end{itemize}
Em alto grau.
\section{Muito}
\begin{itemize}
\item {Grp. gram.:adj.}
\end{itemize}
\begin{itemize}
\item {Grp. gram.:M.}
\end{itemize}
\begin{itemize}
\item {Proveniência:(Do lat. \textunderscore multus\textunderscore )}
\end{itemize}
Que é em grande número ou abundância: \textunderscore muito dinheiro\textunderscore .
O mesmo que \textunderscore grande\textunderscore :«\textunderscore ...muito número de gente cega...\textunderscore »Sousa, \textunderscore Vida do Arceb.\textunderscore  I, 105.
Grande quantidade; grande valor.
\section{Muito}
\begin{itemize}
\item {Grp. gram.:adv.}
\end{itemize}
\begin{itemize}
\item {Grp. gram.:Loc. adv.}
\end{itemize}
\begin{itemize}
\item {Grp. gram.:Loc. adv.}
\end{itemize}
Com excesso.
Abundantemente.
Profundamente; em alto grau; intensamente: \textunderscore soffrer muito\textunderscore .
Com fôrça.
\textunderscore Mais que muito\textunderscore , extraordinariamente, no mais alto grau. Cf. Camillo, \textunderscore Retr. de Ricard.\textunderscore , 285 e 286.
\textunderscore De muito\textunderscore , há muito tempo, ou havia muito tempo:«\textunderscore ...que a trazia muito, alvoroçada\textunderscore ». Camillo, \textunderscore Enjeitada\textunderscore , 63.
\section{Muízas}
\begin{itemize}
\item {Grp. gram.:m. pl.}
\end{itemize}
Povo africano, talvez representante dos \textunderscore Muzimbas\textunderscore .
\section{Mujanguê}
\begin{itemize}
\item {fónica:gu-ê}
\end{itemize}
\begin{itemize}
\item {Grp. gram.:m.}
\end{itemize}
\begin{itemize}
\item {Utilização:Bras. do N}
\end{itemize}
Espécie de massa, feita de ovos de tartaruga, para sêr desfeita em água e beber-se.
\section{Mujingue}
\begin{itemize}
\item {Grp. gram.:m.}
\end{itemize}
Árvore angolense.
\section{Mujoroso}
\begin{itemize}
\item {Grp. gram.:m.}
\end{itemize}
Árvore burserácea africana, (\textunderscore commiphora edulis\textunderscore , Engl.), de frutos comestíveis. Cf. Ficalho, \textunderscore Plantas Út. da Áfr.\textunderscore 
\section{Mula}
\begin{itemize}
\item {Grp. gram.:f.}
\end{itemize}
\begin{itemize}
\item {Utilização:Pop.}
\end{itemize}
Fêmea do mu.
Pessôa ruím, que tem más manhas.
(Cp. \textunderscore mulo\textunderscore )
\section{Mula}
\begin{itemize}
\item {Grp. gram.:f.}
\end{itemize}
\begin{itemize}
\item {Utilização:Marn.}
\end{itemize}
Monte de sal, em fórma de prisma de secção triangular, terminando em dois meios cónes.
\section{Mulabá}
\begin{itemize}
\item {Grp. gram.:m.}
\end{itemize}
\begin{itemize}
\item {Utilização:T. de Angoche}
\end{itemize}
O mesmo que \textunderscore imbondeiro\textunderscore .
\section{Mulada}
\begin{itemize}
\item {Grp. gram.:f.}
\end{itemize}
\begin{itemize}
\item {Utilização:Bras. do S}
\end{itemize}
\begin{itemize}
\item {Proveniência:(De \textunderscore mula\textunderscore ^1)}
\end{itemize}
Manada de mulas.
\section{Muladar}
\begin{itemize}
\item {Grp. gram.:m.}
\end{itemize}
\begin{itemize}
\item {Utilização:Fig.}
\end{itemize}
\begin{itemize}
\item {Proveniência:(De \textunderscore mula\textunderscore ^1, dizem os diccion., mas não me parece acceitável tal derivação; provavelmente, \textunderscore muladar\textunderscore  é a simples metáth. de \textunderscore muradal\textunderscore . Cp. \textunderscore muradal\textunderscore )}
\end{itemize}
Monturo; esterqueira.
Tudo que suja ou ennodôa.
\section{Mulage}
\begin{itemize}
\item {Grp. gram.:m.}
\end{itemize}
(V.pau-dos-feiticeiros)
\section{Mulambalai}
\begin{itemize}
\item {Grp. gram.:m.}
\end{itemize}
Árvore angolense.
\section{Mulambe}
\begin{itemize}
\item {Grp. gram.:m.}
\end{itemize}
Árvore de Angola.
\section{Mulambó}
\begin{itemize}
\item {Grp. gram.:m.}
\end{itemize}
Árvore da Índia portuguesa.
\section{Mulana}
\begin{itemize}
\item {Grp. gram.:m.}
\end{itemize}
\begin{itemize}
\item {Utilização:Ant.}
\end{itemize}
\begin{itemize}
\item {Proveniência:(T. ár.)}
\end{itemize}
Espécie de ministro da justiça, entre os Moiros. Cf. Fern. M. Pinto, \textunderscore Peregr.\textunderscore , c. III.
\section{Mulata}
\begin{itemize}
\item {Grp. gram.:adj. f.}
\end{itemize}
\begin{itemize}
\item {Grp. gram.:F.}
\end{itemize}
\begin{itemize}
\item {Utilização:Ant.}
\end{itemize}
Diz-se de uma variedade de batata, mimosa e roxa, preferida especialmente para assar.
Pessôa do sexo feminino, procedente de pai branco e mulher preta ou vice-versa.
O mesmo que \textunderscore mula\textunderscore ^1. Cf. G. Vicente, \textunderscore Clér. da Beira\textunderscore .
(Cp. \textunderscore mulato\textunderscore ^1)
\section{Mulataria}
\begin{itemize}
\item {Grp. gram.:f.}
\end{itemize}
Chusma de mulatos. Cf. Camillo, \textunderscore Narcót.\textunderscore , II, 9.
\section{Mulateira}
\begin{itemize}
\item {Grp. gram.:f.}
\end{itemize}
Burra que se dá á cobrição por cavallo, para a producção de mus ou muares.
(Cp. \textunderscore mulateiro\textunderscore )
\section{Mulateiro}
\begin{itemize}
\item {Grp. gram.:m.}
\end{itemize}
\begin{itemize}
\item {Utilização:Bot.}
\end{itemize}
Burro de cobrição de éguas, para a producção de mus ou muares.
Árvore amazónica, também conhecida por \textunderscore pau-mulato\textunderscore .
(Cp. \textunderscore muleteiro\textunderscore )
\section{Mulatete}
\begin{itemize}
\item {fónica:tê}
\end{itemize}
\begin{itemize}
\item {Grp. gram.:m.}
\end{itemize}
\begin{itemize}
\item {Utilização:Bras}
\end{itemize}
Pequeno mulato. Cf. \textunderscore Jorn.-do-Comm.\textunderscore , do Rio, de 1-I-905.
\section{Mulatinho}
\begin{itemize}
\item {Grp. gram.:m.}
\end{itemize}
\begin{itemize}
\item {Utilização:Prov.}
\end{itemize}
\begin{itemize}
\item {Utilização:trasm.}
\end{itemize}
Mulato pequeno.
Variedade de feijão.
Passarinho implume, ainda no ninho.
\section{Mulato}
\begin{itemize}
\item {Grp. gram.:m.}
\end{itemize}
\begin{itemize}
\item {Utilização:Ext.}
\end{itemize}
\begin{itemize}
\item {Proveniência:(De \textunderscore mulo\textunderscore )}
\end{itemize}
O mesmo que \textunderscore mu\textunderscore ^1.
Aquelle que procede de pai branco e mãe preta ou viceversa.
Aquelle que é escuro; trigueiro. Cf. G. Vicente, \textunderscore Clér. da Beira\textunderscore .
\section{Mulato}
\begin{itemize}
\item {Grp. gram.:m.}
\end{itemize}
Variedade de pêssegos grandes, no distrito de Leiria.
Casta de figueira.
\section{Mulato-velho}
\begin{itemize}
\item {Grp. gram.:m.}
\end{itemize}
\begin{itemize}
\item {Utilização:Bras. do Rio}
\end{itemize}
O mesmo que \textunderscore patureba\textunderscore ^1.
\section{Muldera}
\begin{itemize}
\item {Grp. gram.:f.}
\end{itemize}
Gênero de plantas piperáceas.
\section{Mule}
\begin{itemize}
\item {Grp. gram.:m.}
\end{itemize}
Um dos aparelhos das fábricas de fiação.
\section{Mulela}
\begin{itemize}
\item {Grp. gram.:f.}
\end{itemize}
Árvore angolense, chamada também pelos indígenas \textunderscore pau-de-manteiga\textunderscore .
\section{Mulelame}
\begin{itemize}
\item {Grp. gram.:m.}
\end{itemize}
Pequena árvore burserácea de Angola.
\section{Mulemba}
\begin{itemize}
\item {Grp. gram.:f.}
\end{itemize}
Espécie de grande figueira de Angola, (\textunderscore ficus psilopoga\textunderscore ), de frutos comestíveis e raízes medicinaes.
\section{Mulembare}
\begin{itemize}
\item {Grp. gram.:m.}
\end{itemize}
Árvore angolense.
\section{Mulembérgia}
\begin{itemize}
\item {Grp. gram.:f.}
\end{itemize}
Gênero de plantas gramíneas.
\section{Mulembo}
\begin{itemize}
\item {Grp. gram.:m.}
\end{itemize}
Árvore de Angola.
\section{Mulembuge}
\begin{itemize}
\item {Grp. gram.:m.}
\end{itemize}
Arbusto africano, sarmentoso e trepador.
\section{Mulende}
\begin{itemize}
\item {Grp. gram.:m.}
\end{itemize}
Grande e vistosa árvore africana, de fôlhas simples, pilosas, macias, com 5 lóbos acuminados.
\section{Mulenga}
\begin{itemize}
\item {Grp. gram.:f.}
\end{itemize}
Árvore africana.
\section{Mulengue}
\begin{itemize}
\item {Grp. gram.:m.}
\end{itemize}
Arbusto africano, annual, de fôlhas simples, glabras, e flôres axillares em espigas delgadas.
\section{Múleo}
\begin{itemize}
\item {Grp. gram.:m.}
\end{itemize}
\begin{itemize}
\item {Proveniência:(Lat. \textunderscore mulleus\textunderscore )}
\end{itemize}
Calçado vermelho, de que usavam os reis de Alba-Longa e, depois, os patrícios romanos.
\section{Muleta}
\begin{itemize}
\item {fónica:lê}
\end{itemize}
\begin{itemize}
\item {Grp. gram.:m.}
\end{itemize}
\begin{itemize}
\item {Utilização:T. de Buarcos}
\end{itemize}
\begin{itemize}
\item {Utilização:Fig.}
\end{itemize}
Pau, em que se apoiam os coxos ou aleijados das pernas.
Pau, em que o toireiro suspende a capa, para chamar o toiro que quere matar.
Manivela de realejo.
Pequena embarcação, com que se faz pescaria, fóra da barra de Lisbôa.
Forcado de ferro, com que se empurra o barco, do lado da prôa.
Apoio, amparo.
(Cast. \textunderscore muleta\textunderscore )
\section{Muletada}
\begin{itemize}
\item {Grp. gram.:f.}
\end{itemize}
Manada de gado muar.
(Cast. \textunderscore muletada\textunderscore )
\section{Muleteiro}
\begin{itemize}
\item {Grp. gram.:m.}
\end{itemize}
Aquelle que trata de mulas; arreeiro.
(Cast. \textunderscore muletero\textunderscore )
\section{Muletim}
\begin{itemize}
\item {Grp. gram.:m.}
\end{itemize}
Vela da embarcação que se chama muleta.
\section{Muleu}
\begin{itemize}
\item {Grp. gram.:m.}
\end{itemize}
Árvore angolense.
\section{Mulhe-mulhe}
\begin{itemize}
\item {Grp. gram.:m.}
\end{itemize}
\begin{itemize}
\item {Utilização:Ant.}
\end{itemize}
O mesmo que \textunderscore molhe-molhe\textunderscore ; molinha.
\section{Mulhér}
\begin{itemize}
\item {Grp. gram.:f.}
\end{itemize}
\begin{itemize}
\item {Utilização:Fam.}
\end{itemize}
\begin{itemize}
\item {Utilização:Fig.}
\end{itemize}
\begin{itemize}
\item {Utilização:Pop.}
\end{itemize}
\begin{itemize}
\item {Proveniência:(Do lat. \textunderscore mulier\textunderscore )}
\end{itemize}
Pessôa do sexo feminino, depois da puberdade.
Espôsa: \textunderscore minha mulhér está doente\textunderscore .
Pessôa do sexo feminino, pertencente ás classes inferiores da sociedade: \textunderscore vão alli duas mulheres\textunderscore .
Homem mulherengo.
Espécie de jôgo popular.
\section{Mulheraça}
\begin{itemize}
\item {Grp. gram.:f.}
\end{itemize}
Mulher alta e robusta.
\section{Mulherão}
\begin{itemize}
\item {Grp. gram.:m.}
\end{itemize}
O mesmo que \textunderscore mulheraça\textunderscore .
\section{Mulhereiro}
\begin{itemize}
\item {Grp. gram.:adj.}
\end{itemize}
\begin{itemize}
\item {Proveniência:(Do lat. \textunderscore mulierarius\textunderscore )}
\end{itemize}
O mesmo que \textunderscore mulherengo\textunderscore .
\section{Mulherengo}
\begin{itemize}
\item {Grp. gram.:m.  e  adj.}
\end{itemize}
Homem apaixonado por mulheres, ou que se occupa em mesteres próprios de mulheres.
Maricas.
\section{Mulherento}
\begin{itemize}
\item {Grp. gram.:adj.}
\end{itemize}
\begin{itemize}
\item {Utilização:Ant.}
\end{itemize}
O mesmo que \textunderscore mulherengo\textunderscore .
\section{Mulher-frágil}
\begin{itemize}
\item {Grp. gram.:f.}
\end{itemize}
Planta africana, apparentemente robusta mas frágil, de fôlhas simples e oppostas, e flôres axillares em cachos.
\section{Mulhericas}
\begin{itemize}
\item {Grp. gram.:m.}
\end{itemize}
\begin{itemize}
\item {Utilização:Fam.}
\end{itemize}
O mesmo que \textunderscore maricas\textunderscore .
\section{Mulhericídio}
\begin{itemize}
\item {Grp. gram.:m.}
\end{itemize}
\begin{itemize}
\item {Utilização:P. us.}
\end{itemize}
\begin{itemize}
\item {Proveniência:(Do lat. \textunderscore mulier\textunderscore  + \textunderscore caedere\textunderscore )}
\end{itemize}
Assassínio de mulher. Cf. Castilho, \textunderscore Mil e Um Myst.\textunderscore , 237.
\section{Mulherico}
\begin{itemize}
\item {Grp. gram.:adj.}
\end{itemize}
\begin{itemize}
\item {Proveniência:(De \textunderscore mulher\textunderscore )}
\end{itemize}
Effeminado.
Fraco.
Cobarde.
\section{Mulherigo}
\begin{itemize}
\item {Grp. gram.:m.}
\end{itemize}
Homem mulherico, effeminado. Cf. Camillo, \textunderscore Vinho do Porto\textunderscore , 66.
\section{Mulheril}
\begin{itemize}
\item {Grp. gram.:adj.}
\end{itemize}
Relativo a mulher, próprio de mulheres.
Mulherengo.
\section{Mulherilmente}
\begin{itemize}
\item {Grp. gram.:adv.}
\end{itemize}
De modo mulheril.
\section{Mulherinha}
\begin{itemize}
\item {Grp. gram.:f.}
\end{itemize}
\begin{itemize}
\item {Utilização:Fam.}
\end{itemize}
Mulher ordinária.
Mulher, que se porta mal.
Mexeriqueira.
\section{Mulherio}
\begin{itemize}
\item {Grp. gram.:m.}
\end{itemize}
\begin{itemize}
\item {Utilização:Pop.}
\end{itemize}
As mulheres.
Grande quantidade de mulheres.
\section{Mulhermente}
\begin{itemize}
\item {Grp. gram.:adv.}
\end{itemize}
Á maneira de mulher.
Feminilmente.
Pusillanimemente.--Termo inventado por Filinto, não me consta que tenha sido reproduzido, nem o merece talvez:«\textunderscore o que é viver apprende, sem deixar quebrantar-te mulhermente\textunderscore ». Filinto, IV, 95.
\section{Mulherum}
\begin{itemize}
\item {Grp. gram.:m.}
\end{itemize}
\begin{itemize}
\item {Utilização:Prov.}
\end{itemize}
\begin{itemize}
\item {Utilização:alg.}
\end{itemize}
O mesmo que \textunderscore mulherio\textunderscore .
\section{Muliado}
\begin{itemize}
\item {Grp. gram.:adj.}
\end{itemize}
\begin{itemize}
\item {Utilização:Fig.}
\end{itemize}
\begin{itemize}
\item {Proveniência:(De \textunderscore mula\textunderscore )}
\end{itemize}
Monstruoso.
Hýbrido.
Opposto ao que deve sêr, ao que é conveniente.
\section{Mulíebre}
\begin{itemize}
\item {Grp. gram.:adj.}
\end{itemize}
\begin{itemize}
\item {Proveniência:(Lat. \textunderscore muliebris\textunderscore )}
\end{itemize}
O mesmo que \textunderscore mulheril\textunderscore .
\section{Mulilampepo}
\begin{itemize}
\item {Grp. gram.:m.}
\end{itemize}
Árvore de Angola.
\section{Mulle}
\begin{itemize}
\item {Grp. gram.:m.}
\end{itemize}
Um dos apparelhos das fábricas de fiação.
\section{Múlleo}
\begin{itemize}
\item {Grp. gram.:m.}
\end{itemize}
\begin{itemize}
\item {Proveniência:(Lat. \textunderscore mulleus\textunderscore )}
\end{itemize}
Calçado vermelho, de que usavam os reis de Alba-Longa e, depois, os patrícios romanos.
\section{Múllera}
\begin{itemize}
\item {Grp. gram.:f.}
\end{itemize}
\begin{itemize}
\item {Proveniência:(De \textunderscore Müller\textunderscore , n. p.)}
\end{itemize}
Gênero de plantas leguminosas.
\section{Mulo}
\begin{itemize}
\item {Grp. gram.:m.}
\end{itemize}
\begin{itemize}
\item {Proveniência:(Lat. \textunderscore mulus\textunderscore )}
\end{itemize}
O mesmo que \textunderscore mu\textunderscore ^1.
\section{Mulola}
\begin{itemize}
\item {Grp. gram.:f.}
\end{itemize}
\begin{itemize}
\item {Utilização:T. da Áfr. Or. Port}
\end{itemize}
Areal extenso.
\section{Mulolo}
\begin{itemize}
\item {Grp. gram.:m.}
\end{itemize}
Pequena árvore africana, irregular, (\textunderscore bauhinia reticulata\textunderscore ).
\section{Mulombe}
\begin{itemize}
\item {Grp. gram.:m.}
\end{itemize}
Pássaro africano, de vivíssimas côres brilhantes, (\textunderscore lamprocolius acuticaudus\textunderscore ).
\section{Mulondo}
\begin{itemize}
\item {Grp. gram.:m.}
\end{itemize}
Árvore angolense, de frutos comestíveis.
\section{Mulondolonda}
\begin{itemize}
\item {Grp. gram.:f.}
\end{itemize}
Árvore de Angola.
\section{Mulope}
\begin{itemize}
\item {Grp. gram.:m.}
\end{itemize}
Árvore de Timor.
\section{Mulsa}
\begin{itemize}
\item {Grp. gram.:f.}
\end{itemize}
\begin{itemize}
\item {Proveniência:(Lat. \textunderscore mulsum\textunderscore )}
\end{itemize}
O mesmo que \textunderscore hydromel\textunderscore .
\section{Mulso}
\begin{itemize}
\item {Grp. gram.:m.}
\end{itemize}
\begin{itemize}
\item {Proveniência:(Lat. \textunderscore mulsum\textunderscore )}
\end{itemize}
O mesmo que \textunderscore hydromel\textunderscore .
\section{Multa}
\begin{itemize}
\item {Grp. gram.:f.}
\end{itemize}
\begin{itemize}
\item {Proveniência:(Lat. \textunderscore mulcta\textunderscore )}
\end{itemize}
Acto ou effeito de multar.
Cóima.
Pena pecuniária.
Pena, condemnação.
\section{Multangular}
\begin{itemize}
\item {Grp. gram.:adj.}
\end{itemize}
\begin{itemize}
\item {Utilização:Mathem.}
\end{itemize}
\begin{itemize}
\item {Proveniência:(Do lat. \textunderscore multus\textunderscore  + \textunderscore angulus\textunderscore )}
\end{itemize}
Diz-se da figura que tem mais de quatro lados. Cp. \textunderscore multiangular\textunderscore .
\section{Multar}
\begin{itemize}
\item {Grp. gram.:v. t.}
\end{itemize}
\begin{itemize}
\item {Utilização:Fig.}
\end{itemize}
\begin{itemize}
\item {Proveniência:(Lat. \textunderscore mulctare\textunderscore  ou \textunderscore multare\textunderscore )}
\end{itemize}
Impor pena pecuniária ou outra qualquer a.
Condemnar.
\section{Multi...}
\begin{itemize}
\item {Grp. gram.:pref.}
\end{itemize}
\begin{itemize}
\item {Proveniência:(Do lat. \textunderscore multus\textunderscore  e \textunderscore multum\textunderscore )}
\end{itemize}
(designativo de \textunderscore muito\textunderscore )
\section{Multiangular}
\begin{itemize}
\item {Grp. gram.:adj.}
\end{itemize}
\begin{itemize}
\item {Proveniência:(De \textunderscore multi...\textunderscore  + \textunderscore angular\textunderscore )}
\end{itemize}
Que tem muitos ângulos.
\section{Multiaxífero}
\begin{itemize}
\item {fónica:csi}
\end{itemize}
\begin{itemize}
\item {Grp. gram.:adj.}
\end{itemize}
\begin{itemize}
\item {Utilização:Bot.}
\end{itemize}
\begin{itemize}
\item {Proveniência:(Do lat. \textunderscore multus\textunderscore  + \textunderscore axis\textunderscore  + \textunderscore ferre\textunderscore )}
\end{itemize}
Que tem muitos eixos.
\section{Multicapsular}
\begin{itemize}
\item {Grp. gram.:adj.}
\end{itemize}
\begin{itemize}
\item {Utilização:Bot.}
\end{itemize}
\begin{itemize}
\item {Proveniência:(De \textunderscore multi...\textunderscore  + \textunderscore capsular\textunderscore )}
\end{itemize}
Diz-se do fruto, que tem muitas cápsulas.
\section{Multicaudo}
\begin{itemize}
\item {Grp. gram.:adj.}
\end{itemize}
\begin{itemize}
\item {Utilização:Bot.}
\end{itemize}
\begin{itemize}
\item {Proveniência:(De \textunderscore multi...\textunderscore  + \textunderscore cauda\textunderscore )}
\end{itemize}
Que tem muitos prolongamentos em fórma de cauda.
\section{Multicaule}
\begin{itemize}
\item {Grp. gram.:adj.}
\end{itemize}
\begin{itemize}
\item {Proveniência:(De \textunderscore multi...\textunderscore  + \textunderscore caule\textunderscore )}
\end{itemize}
Diz-se do vegetal, de cuja raíz sáem muitos caules.
\section{Multicellular}
\begin{itemize}
\item {Grp. gram.:adj.}
\end{itemize}
\begin{itemize}
\item {Utilização:Bot.}
\end{itemize}
\begin{itemize}
\item {Proveniência:(Do lat. \textunderscore multus\textunderscore  + \textunderscore cellula\textunderscore )}
\end{itemize}
Que tem muitas céllulas.
\section{Multicelular}
\begin{itemize}
\item {Grp. gram.:adj.}
\end{itemize}
\begin{itemize}
\item {Utilização:Bot.}
\end{itemize}
\begin{itemize}
\item {Proveniência:(Do lat. \textunderscore multus\textunderscore  + \textunderscore cellula\textunderscore )}
\end{itemize}
Que tem muitas células.
\section{Multicolor}
\begin{itemize}
\item {Grp. gram.:adj.}
\end{itemize}
\begin{itemize}
\item {Proveniência:(Lat. \textunderscore multicolor\textunderscore )}
\end{itemize}
Que tem muitas côres.
\section{Multicor}
\begin{itemize}
\item {Grp. gram.:adj.}
\end{itemize}
O mesmo que \textunderscore multicolor\textunderscore .
\section{Multidão}
\begin{itemize}
\item {Grp. gram.:f.}
\end{itemize}
\begin{itemize}
\item {Proveniência:(Lat. \textunderscore multitudo\textunderscore )}
\end{itemize}
Grande número ou ajuntamento de pessôas ou coisas.
Montão.
Abundância.
Populacho; o povo.
\section{Multiface}
\begin{itemize}
\item {Grp. gram.:adj.}
\end{itemize}
\begin{itemize}
\item {Utilização:Neol.}
\end{itemize}
\begin{itemize}
\item {Proveniência:(Do lat. \textunderscore multus\textunderscore  + \textunderscore facies\textunderscore )}
\end{itemize}
Que tem muitas faces ou muitos aspectos.
Que applica a sua actividade a vários assumptos.
\section{Multifário}
\begin{itemize}
\item {Grp. gram.:adj.}
\end{itemize}
\begin{itemize}
\item {Proveniência:(Lat. \textunderscore multifarius\textunderscore )}
\end{itemize}
Variado; que offerece vários aspectos.
\section{Multífido}
\begin{itemize}
\item {Grp. gram.:adj.}
\end{itemize}
\begin{itemize}
\item {Proveniência:(Lat. \textunderscore multifidus\textunderscore )}
\end{itemize}
Fendido em muitas partes.
\section{Multifloro}
\begin{itemize}
\item {Grp. gram.:adj.}
\end{itemize}
\begin{itemize}
\item {Utilização:Bot.}
\end{itemize}
\begin{itemize}
\item {Proveniência:(Do lat. \textunderscore multus\textunderscore  + \textunderscore flos\textunderscore )}
\end{itemize}
Que tem muitas flôres.
\section{Multífluo}
\begin{itemize}
\item {Grp. gram.:adj.}
\end{itemize}
\begin{itemize}
\item {Proveniência:(Lat. \textunderscore multifluus\textunderscore )}
\end{itemize}
Que flue abundantemente.
\section{Multifoliado}
\begin{itemize}
\item {Grp. gram.:adj.}
\end{itemize}
\begin{itemize}
\item {Utilização:Bot.}
\end{itemize}
\begin{itemize}
\item {Proveniência:(Do lat. \textunderscore multus\textunderscore  + \textunderscore folium\textunderscore )}
\end{itemize}
Que tem muitos folíolos.
\section{Multiforme}
\begin{itemize}
\item {Grp. gram.:adj.}
\end{itemize}
\begin{itemize}
\item {Proveniência:(Lat. \textunderscore multiformis\textunderscore )}
\end{itemize}
Que tem muitas fórmas; que se manifesta de diversas maneiras.
\section{Multifuro}
\begin{itemize}
\item {Grp. gram.:adj.}
\end{itemize}
\begin{itemize}
\item {Proveniência:(De \textunderscore multi...\textunderscore  + \textunderscore furo\textunderscore )}
\end{itemize}
Que tem muitos furos:«\textunderscore ...flautista, que o multifuro tubo inchar parece.\textunderscore »Filinto, V, 100.
\section{Multígeno}
\begin{itemize}
\item {Grp. gram.:adj.}
\end{itemize}
\begin{itemize}
\item {Proveniência:(Lat. \textunderscore multigenus\textunderscore )}
\end{itemize}
Que abrange differentes gêneros ou espécies.
\section{Multilátero}
\begin{itemize}
\item {Grp. gram.:adj.}
\end{itemize}
\begin{itemize}
\item {Utilização:Mathem.}
\end{itemize}
\begin{itemize}
\item {Proveniência:(Do lat. \textunderscore multus\textunderscore  + \textunderscore latus\textunderscore )}
\end{itemize}
Diz-se da figura, que tem mais de quatro lados.
\section{Multilobado}
\begin{itemize}
\item {Grp. gram.:adj.}
\end{itemize}
\begin{itemize}
\item {Proveniência:(De \textunderscore multi...\textunderscore  + \textunderscore lobado\textunderscore )}
\end{itemize}
Que é por natureza dividido em muitos lóbulos.
\section{Multilocular}
\begin{itemize}
\item {Grp. gram.:adj.}
\end{itemize}
\begin{itemize}
\item {Proveniência:(De \textunderscore multi...\textunderscore  + \textunderscore locular\textunderscore )}
\end{itemize}
Que tem muitos lóculos.
\section{Multíloquo}
\begin{itemize}
\item {Grp. gram.:adj.}
\end{itemize}
\begin{itemize}
\item {Proveniência:(Lat. \textunderscore multiloquus\textunderscore )}
\end{itemize}
Que fala muito.
\section{Multilustroso}
\begin{itemize}
\item {Grp. gram.:adj.}
\end{itemize}
\begin{itemize}
\item {Proveniência:(De \textunderscore multi...\textunderscore  + \textunderscore lustroso\textunderscore )}
\end{itemize}
Que tem muito brilho. Cf. Filinto, IV, 101.
\section{Multimâmio}
\begin{itemize}
\item {Grp. gram.:adj.}
\end{itemize}
\begin{itemize}
\item {Utilização:Zool.}
\end{itemize}
\begin{itemize}
\item {Proveniência:(Do lat. \textunderscore multus\textunderscore  + \textunderscore mamma\textunderscore )}
\end{itemize}
Que tem mais de duas mamas ou tetas.
\section{Multimâmmio}
\begin{itemize}
\item {Grp. gram.:adj.}
\end{itemize}
\begin{itemize}
\item {Utilização:Zool.}
\end{itemize}
\begin{itemize}
\item {Proveniência:(Do lat. \textunderscore multus\textunderscore  + \textunderscore mamma\textunderscore )}
\end{itemize}
Que tem mais de duas mamas ou tetas.
\section{Multimilionário}
\begin{itemize}
\item {Grp. gram.:adj.}
\end{itemize}
\begin{itemize}
\item {Utilização:Neol.}
\end{itemize}
\begin{itemize}
\item {Proveniência:(De \textunderscore multi...\textunderscore  + \textunderscore milionário\textunderscore )}
\end{itemize}
Que possue muitos milhões; que é muitissimo rico.
\section{Multimillionário}
\begin{itemize}
\item {Grp. gram.:adj.}
\end{itemize}
\begin{itemize}
\item {Utilização:Neol.}
\end{itemize}
\begin{itemize}
\item {Proveniência:(De \textunderscore multi...\textunderscore  + \textunderscore millionário\textunderscore )}
\end{itemize}
Que possue muitos milhões; que é muitissimo rico.
\section{Multímodo}
\begin{itemize}
\item {Grp. gram.:adj.}
\end{itemize}
\begin{itemize}
\item {Proveniência:(Lat. \textunderscore multimodus\textunderscore )}
\end{itemize}
Multiforme; multifário.
\section{Multinérveo}
\begin{itemize}
\item {Grp. gram.:adj.}
\end{itemize}
\begin{itemize}
\item {Proveniência:(De \textunderscore multi...\textunderscore  + \textunderscore nérveo\textunderscore )}
\end{itemize}
Que tem muitas nervuras, (falando-se das fôlhas das plantas).
\section{Multinérvia}
\begin{itemize}
\item {Grp. gram.:f.}
\end{itemize}
\begin{itemize}
\item {Utilização:Bot.}
\end{itemize}
Designação antiga da \textunderscore tanchagem\textunderscore .
\section{Multinómio}
\begin{itemize}
\item {Grp. gram.:m.}
\end{itemize}
O mesmo que \textunderscore polynómio\textunderscore .
\section{Multiparidade}
\begin{itemize}
\item {Grp. gram.:f.}
\end{itemize}
Qualidade de multíparo.
\section{Multíparo}
\begin{itemize}
\item {Grp. gram.:adj.}
\end{itemize}
\begin{itemize}
\item {Proveniência:(Do lat. \textunderscore multus\textunderscore  + \textunderscore parere\textunderscore )}
\end{itemize}
Que póde dar á luz, de um só parto, muitos filhos, (falando-se de certas fêmeas de animaes).
\section{Multipartido}
\begin{itemize}
\item {Grp. gram.:adj.}
\end{itemize}
\begin{itemize}
\item {Proveniência:(De \textunderscore multi...\textunderscore  + \textunderscore partido\textunderscore )}
\end{itemize}
Diz-se do órgão vegetal, dividido em grande número de tiras.
\section{Multípede}
\begin{itemize}
\item {Grp. gram.:adj.}
\end{itemize}
\begin{itemize}
\item {Proveniência:(Do lat. \textunderscore multus\textunderscore  + \textunderscore pes\textunderscore , \textunderscore pedis\textunderscore )}
\end{itemize}
Que tem muitos pés.
\section{Multipétalo}
\begin{itemize}
\item {Grp. gram.:adj.}
\end{itemize}
\begin{itemize}
\item {Proveniência:(De \textunderscore multi...\textunderscore  + \textunderscore pétala\textunderscore )}
\end{itemize}
O mesmo que \textunderscore polypétalo\textunderscore .
\section{Multiplicação}
\begin{itemize}
\item {Grp. gram.:f.}
\end{itemize}
\begin{itemize}
\item {Proveniência:(Lat. \textunderscore multiplicatio\textunderscore )}
\end{itemize}
Acto ou effeito de multiplicar.
Reproducção: \textunderscore a multiplicação da espécie humana\textunderscore .
Operação arithmética, em que um número chamado \textunderscore multiplicando\textunderscore , se repete tantas vezes quantas são as unidades de outro, chamado \textunderscore multiplicador\textunderscore .
\section{Multiplicadamente}
\begin{itemize}
\item {Grp. gram.:adv.}
\end{itemize}
\begin{itemize}
\item {Proveniência:(De \textunderscore multiplicar\textunderscore )}
\end{itemize}
Com multiplicação.
\section{Multiplicador}
\begin{itemize}
\item {Grp. gram.:m.}
\end{itemize}
\begin{itemize}
\item {Grp. gram.:Adj.}
\end{itemize}
\begin{itemize}
\item {Proveniência:(Lat. \textunderscore multiplicator\textunderscore )}
\end{itemize}
Número, que designa quantas vezes se há de tomar outro como parcella.
Vidro, que representa simultaneamente muitas imagens de um só objecto.
Que multiplica.
\section{Multiplicando}
\begin{itemize}
\item {Grp. gram.:m.}
\end{itemize}
\begin{itemize}
\item {Proveniência:(Lat. \textunderscore multiplicandus\textunderscore )}
\end{itemize}
Número que, na operação arithmética da multiplicação, se há de tomar tantas vezes, quantas as unidades do multiplicador.
\section{Multiplicar}
\begin{itemize}
\item {Grp. gram.:v. t.}
\end{itemize}
\begin{itemize}
\item {Grp. gram.:V. i.}
\end{itemize}
\begin{itemize}
\item {Proveniência:(Lat. \textunderscore multiplicare\textunderscore )}
\end{itemize}
Aumentar o número de.
Produzir em grande quantidade.
Repetir: \textunderscore multiplicar pedidos\textunderscore .
Repetir (um número como parcella) tantas vezes, quantas as unidades de outro.
Propagar-se: \textunderscore crescei e multiplicai, disse o Criador\textunderscore .
Aumentar em número.
Fazer a operação arithmética da multiplicação.
\section{Multiplicativo}
\begin{itemize}
\item {Grp. gram.:adj.}
\end{itemize}
\begin{itemize}
\item {Proveniência:(Lat. \textunderscore multiplicativus\textunderscore )}
\end{itemize}
Que multiplica ou serve para multiplicar.
\section{Multiplicável}
\begin{itemize}
\item {Grp. gram.:adj.}
\end{itemize}
\begin{itemize}
\item {Proveniência:(Lat. \textunderscore multiplicabilis\textunderscore )}
\end{itemize}
Que se pode multiplicar.
\section{Multíplice}
\begin{itemize}
\item {Grp. gram.:adj.}
\end{itemize}
\begin{itemize}
\item {Proveniência:(Lat. \textunderscore multiplex\textunderscore )}
\end{itemize}
Variado; complexo.
Que se manifesta de vários modos.
Copioso.
Que não é único.
\section{Multiplicidade}
\begin{itemize}
\item {Grp. gram.:f.}
\end{itemize}
\begin{itemize}
\item {Proveniência:(Lat. \textunderscore multiplicitas\textunderscore )}
\end{itemize}
Qualidade de multíplice.
Grande número; abundância.
\section{Múltiplo}
\begin{itemize}
\item {Grp. gram.:adj.}
\end{itemize}
\begin{itemize}
\item {Utilização:Gram.}
\end{itemize}
\begin{itemize}
\item {Utilização:Bot.}
\end{itemize}
\begin{itemize}
\item {Grp. gram.:M.}
\end{itemize}
\begin{itemize}
\item {Proveniência:(Lat. \textunderscore multiplus\textunderscore )}
\end{itemize}
Que abrange muitas coisas, que não é simples, que não é único.
Diz-se, em Arithmética, de um número que póde dividir-se por outro, sem deixar resto.
Em Geometria, diz-se do ponto commum, por onde passam muitos ramos de uma só curva.
Diz se do sujeito, que abrange objectos differentes.
Diz-se do fruto, que é composto de muitas carpellas insuladas.
Diz-se do ovário, formado de muitas carpellas livres.
Diz-se do echo, que repete os mesmos sons muitas vezes successivamente.
Número múltiplo.
\section{Multipolar}
\begin{itemize}
\item {Grp. gram.:adj.}
\end{itemize}
\begin{itemize}
\item {Utilização:Zool.}
\end{itemize}
\begin{itemize}
\item {Proveniência:(De \textunderscore multi...\textunderscore  + \textunderscore polar\textunderscore )}
\end{itemize}
Diz-se da céllula nervosa, que tem vários prolongamentos. Cf. Max. Lemos, \textunderscore Zool.\textunderscore , 19.
\section{Multipontuado}
\begin{itemize}
\item {Grp. gram.:adj.}
\end{itemize}
\begin{itemize}
\item {Proveniência:(Lat. \textunderscore multi...\textunderscore  + \textunderscore pontuado\textunderscore )}
\end{itemize}
Que tem muitos pontos ou pintas.
Mosqueado.
\section{Multipotente}
\begin{itemize}
\item {Grp. gram.:adj.}
\end{itemize}
\begin{itemize}
\item {Proveniência:(Lat. \textunderscore multipotens\textunderscore )}
\end{itemize}
Que póde muito.
\section{Multisciente}
\begin{itemize}
\item {Grp. gram.:adj.}
\end{itemize}
\begin{itemize}
\item {Proveniência:(Do lat. \textunderscore multum\textunderscore  + \textunderscore sciens\textunderscore )}
\end{itemize}
Que sabe muito; mui sabedor.
\section{Multíscio}
\begin{itemize}
\item {Grp. gram.:adj.}
\end{itemize}
\begin{itemize}
\item {Proveniência:(Lat. \textunderscore multiscius\textunderscore )}
\end{itemize}
O mesmo que \textunderscore multisciente\textunderscore .
\section{Multisecular}
\begin{itemize}
\item {fónica:se}
\end{itemize}
\begin{itemize}
\item {Grp. gram.:adj.}
\end{itemize}
\begin{itemize}
\item {Proveniência:(Do lat. \textunderscore multus\textunderscore  + \textunderscore saeculum\textunderscore )}
\end{itemize}
Que tem muitos séculos.
\section{Multísono}
\begin{itemize}
\item {fónica:so}
\end{itemize}
\begin{itemize}
\item {Grp. gram.:adj.}
\end{itemize}
\begin{itemize}
\item {Proveniência:(Lat. \textunderscore multisonus\textunderscore )}
\end{itemize}
Que produz muitos ou variados sons.
\section{Multissecular}
\begin{itemize}
\item {Grp. gram.:adj.}
\end{itemize}
\begin{itemize}
\item {Proveniência:(Do lat. \textunderscore multus\textunderscore  + \textunderscore saeculum\textunderscore )}
\end{itemize}
Que tem muitos séculos.
\section{Multissono}
\begin{itemize}
\item {Grp. gram.:adj.}
\end{itemize}
\begin{itemize}
\item {Proveniência:(Lat. \textunderscore multisonus\textunderscore )}
\end{itemize}
Que produz muitos ou variados sons.
\section{Multitubular}
\begin{itemize}
\item {Grp. gram.:adj.}
\end{itemize}
\begin{itemize}
\item {Proveniência:(De \textunderscore multi...\textunderscore  + \textunderscore tubular\textunderscore )}
\end{itemize}
Que tem muitos tubos, (falando-se especialmente de caldeiras de navios a vapor).
\section{Multiungulado}
\begin{itemize}
\item {Grp. gram.:adj.}
\end{itemize}
\begin{itemize}
\item {Proveniência:(De \textunderscore multi...\textunderscore  + \textunderscore ungulado\textunderscore )}
\end{itemize}
Diz-se do animal, que tem mais de dois cascos em cada pé.
\section{Multívago}
\begin{itemize}
\item {Grp. gram.:adj.}
\end{itemize}
\begin{itemize}
\item {Proveniência:(Lat. \textunderscore multivagus\textunderscore )}
\end{itemize}
Que anda de uma para outra parte.
Vagabundo.
Que anda sempre.
\section{Multivalve}
\begin{itemize}
\item {Grp. gram.:adj.}
\end{itemize}
\begin{itemize}
\item {Proveniência:(De \textunderscore multi...\textunderscore  + \textunderscore valva\textunderscore )}
\end{itemize}
Que tem muitas valvas.
\section{Multivalvular}
\begin{itemize}
\item {Grp. gram.:adj.}
\end{itemize}
\begin{itemize}
\item {Utilização:Bot.}
\end{itemize}
\begin{itemize}
\item {Proveniência:(De \textunderscore multi\textunderscore  + \textunderscore válvula\textunderscore )}
\end{itemize}
Que tem muitas válvulas.
\section{Multívio}
\begin{itemize}
\item {Grp. gram.:adj.}
\end{itemize}
\begin{itemize}
\item {Proveniência:(Lat. \textunderscore multivius\textunderscore )}
\end{itemize}
Que apresenta muitos caminhos.
\section{Multívolo}
\begin{itemize}
\item {Grp. gram.:adj.}
\end{itemize}
\begin{itemize}
\item {Proveniência:(Lat. \textunderscore multivolus\textunderscore )}
\end{itemize}
Que quere muitas coisas ao mesmo tempo; exigente; ambicioso.
\section{Muluanda}
\begin{itemize}
\item {Grp. gram.:f.}
\end{itemize}
Arbusto africano.
\section{Mulucolo}
\begin{itemize}
\item {Grp. gram.:m.}
\end{itemize}
Árvore de Angola.
\section{Mulumba}
\begin{itemize}
\item {Grp. gram.:f.}
\end{itemize}
Grande árvore africana, leguminosa, (\textunderscore pterocarpus melliferus\textunderscore ).
\section{Mulumbuaco}
\begin{itemize}
\item {Grp. gram.:m.}
\end{itemize}
Árvore angolense.
\section{Mulumbuantanga}
\begin{itemize}
\item {Grp. gram.:f.}
\end{itemize}
Árvore angolense.
\section{Mulungo}
\begin{itemize}
\item {Grp. gram.:m.}
\end{itemize}
Formosa árvore africana, (\textunderscore erythrína suberifera\textunderscore , Bak.), notável pelas suas espigas de flôres escarlates e pela sua reputação medicinal.
O mesmo que \textunderscore mulungu\textunderscore ?
\section{Mulungu}
\begin{itemize}
\item {Grp. gram.:m.}
\end{itemize}
Árvore leguminosa do Brasil.
O mesmo que \textunderscore mulungo\textunderscore ?
\section{Mumbaca}
\begin{itemize}
\item {Grp. gram.:f.}
\end{itemize}
\begin{itemize}
\item {Utilização:Bras}
\end{itemize}
Espécie de palmeira.
\section{Mumbamba}
\begin{itemize}
\item {Grp. gram.:f.}
\end{itemize}
Árvore angolense.
\section{Mumbanda}
\begin{itemize}
\item {Grp. gram.:f.}
\end{itemize}
O mesmo que \textunderscore mucama\textunderscore .
\section{Mumbangululo}
\begin{itemize}
\item {Grp. gram.:m.}
\end{itemize}
Árvore angolense.
\section{Mumbavo}
\begin{itemize}
\item {Grp. gram.:m.}
\end{itemize}
\begin{itemize}
\item {Utilização:Bras}
\end{itemize}
O mesmo que \textunderscore xerimbabo\textunderscore .
\section{Mumbé}
\begin{itemize}
\item {Grp. gram.:m.}
\end{itemize}
Árvore de Angola.
\section{Mumbica}
\begin{itemize}
\item {Grp. gram.:m.}
\end{itemize}
\begin{itemize}
\item {Utilização:Bras. do Ceará}
\end{itemize}
Bezerro magro, de um anno.
\section{Mumbimba}
\begin{itemize}
\item {Grp. gram.:f.}
\end{itemize}
Árvore de Angola.
\section{Mumbiri}
\begin{itemize}
\item {Grp. gram.:m.}
\end{itemize}
Árvore angolense.
\section{Mumbos}
\begin{itemize}
\item {Grp. gram.:m. pl.}
\end{itemize}
Povo cafreal das vizinhanças de Tete. Cf. Couto, \textunderscore Déc.\textunderscore 
\section{Mumbuca}
\begin{itemize}
\item {Grp. gram.:f.}
\end{itemize}
\begin{itemize}
\item {Utilização:Bras}
\end{itemize}
Abelha grande e negra.
\section{Mumbula}
\begin{itemize}
\item {Grp. gram.:f.}
\end{itemize}
Árvore de Angola.
\section{Mumbulo}
\begin{itemize}
\item {Grp. gram.:m.}
\end{itemize}
Bella árvore africana, ramosa, de fôlhas simples na extremidade dos ramos, e frutos dispostos em grupos de bagas fulvas.
\section{Mumbungurulu}
\begin{itemize}
\item {Grp. gram.:m.}
\end{itemize}
Árvore de Angola.
\section{Múmia}
\begin{itemize}
\item {Grp. gram.:f.}
\end{itemize}
\begin{itemize}
\item {Utilização:Ext.}
\end{itemize}
\begin{itemize}
\item {Utilização:Fig.}
\end{itemize}
Corpo embalsamado pelos antigos Egýpcios e descoberto nas sepulturas do Egýpto.
Cadáver embalsamado por processo análogo ao dos Egýpcios.
Cadáver desecado e embalsamado.
Pessôa muito magra ou descarnada.
(Ar. \textunderscore mumia\textunderscore )
\section{Mumificação}
\begin{itemize}
\item {Grp. gram.:f.}
\end{itemize}
Acto ou effeito de mumificar.
Estado de múmia.
\section{Mumificador}
\begin{itemize}
\item {Grp. gram.:adj.}
\end{itemize}
Que mumifica.
\section{Mumificante}
\begin{itemize}
\item {Grp. gram.:adj.}
\end{itemize}
O mesmo que \textunderscore mumificador\textunderscore .
\section{Mumificar}
\begin{itemize}
\item {Grp. gram.:v. t.}
\end{itemize}
\begin{itemize}
\item {Grp. gram.:V. i.  e  p.}
\end{itemize}
\begin{itemize}
\item {Utilização:Fig.}
\end{itemize}
\begin{itemize}
\item {Proveniência:(De \textunderscore múmia\textunderscore  + lat. \textunderscore facere\textunderscore )}
\end{itemize}
Converter em múmia.
Emmagrecer.
Atrophiar-se intellectualmente.
\section{Mumificável}
\begin{itemize}
\item {Grp. gram.:adj.}
\end{itemize}
Que se póde mumificar.
\section{Mumizar}
\begin{itemize}
\item {Grp. gram.:v. t.}
\end{itemize}
O mesmo que \textunderscore mumificar\textunderscore .
\section{Mumo}
\begin{itemize}
\item {Grp. gram.:m.}
\end{itemize}
Árvore de Angola.
\section{Mumonamona}
\begin{itemize}
\item {Grp. gram.:f.}
\end{itemize}
Árvore de Angola.
\section{Mumono}
\begin{itemize}
\item {Grp. gram.:m.}
\end{itemize}
Árvore angolense.
\section{Mumpapa}
\begin{itemize}
\item {Grp. gram.:f.}
\end{itemize}
Árvore de Angola.
\section{Mumpepo}
\begin{itemize}
\item {Grp. gram.:m.}
\end{itemize}
Árvore de Angola.
\section{Mumpeque}
\begin{itemize}
\item {Grp. gram.:m.}
\end{itemize}
Árvore de Angola.
\section{Mumpingué}
\begin{itemize}
\item {Grp. gram.:m.}
\end{itemize}
Árvore africana, de madeira preta como o ébano.
Ébano do Senegal, no commércio francês.
\section{Mumpuma}
\begin{itemize}
\item {Grp. gram.:f.}
\end{itemize}
Árvore de Angola.
\section{Mumuca}
\begin{itemize}
\item {Grp. gram.:f.}
\end{itemize}
\begin{itemize}
\item {Utilização:Bras}
\end{itemize}
Ente phantástico, que se invoca para meter medo ás crianças quando choram.
\section{Mumungu}
\begin{itemize}
\item {Grp. gram.:m.}
\end{itemize}
Árvore de Angola.
\section{Munan}
\begin{itemize}
\item {Grp. gram.:f.}
\end{itemize}
\begin{itemize}
\item {Utilização:Bras. dos sertões}
\end{itemize}
O mesmo que \textunderscore égua\textunderscore .
\section{Munango-munguena}
\begin{itemize}
\item {Grp. gram.:m.}
\end{itemize}
Arbusto intertropical, de fôlhas lisas, glabras, e flores inodoras.
\section{Munanos}
\begin{itemize}
\item {Grp. gram.:m. pl.}
\end{itemize}
Povos de raça cafreal, em Angola.
\section{Munchamba}
\begin{itemize}
\item {Grp. gram.:m.}
\end{itemize}
Ave africana, (\textunderscore myrmecocichla nigro\textunderscore , Vieill.).
\section{Munchica}
\begin{itemize}
\item {Grp. gram.:f.}
\end{itemize}
Espécie de jôgo popular.
\section{Muncoto}
\begin{itemize}
\item {Grp. gram.:m.}
\end{itemize}
Árvore do Congo.
\section{Muncurio}
\begin{itemize}
\item {Grp. gram.:m.}
\end{itemize}
Árvore de Angola.
\section{Munda}
\begin{itemize}
\item {Grp. gram.:f.}
\end{itemize}
\begin{itemize}
\item {Utilização:T. da Áfr. Occid. Port}
\end{itemize}
O mesmo que \textunderscore morro\textunderscore .
(Cp. \textunderscore monte\textunderscore . Cf. G. Viana, \textunderscore Apostilas\textunderscore )
\section{Mundaçó}
\begin{itemize}
\item {Grp. gram.:m.}
\end{itemize}
Espécie de barrete indiano. Cf. Th. Ribeiro, \textunderscore Jornadas\textunderscore , II, 77.
\section{Mundana}
\begin{itemize}
\item {Grp. gram.:f.}
\end{itemize}
Mulher dissoluta; prostituta.
(Cp. \textunderscore mundano\textunderscore )
\section{Mundanal}
\begin{itemize}
\item {Grp. gram.:adj.}
\end{itemize}
O mesmo que \textunderscore mundano\textunderscore .
\section{Mundanalidade}
\begin{itemize}
\item {Grp. gram.:f.}
\end{itemize}
O mesmo que \textunderscore mundanidade\textunderscore .
\section{Mundanalmente}
\begin{itemize}
\item {Grp. gram.:adv.}
\end{itemize}
De modo mundano.
\section{Mundanamente}
\begin{itemize}
\item {Grp. gram.:adv.}
\end{itemize}
De modo mundano.
\section{Mundanário}
\begin{itemize}
\item {Grp. gram.:adj.}
\end{itemize}
O mesmo que \textunderscore mundano\textunderscore .
\section{Mundaneidade}
\begin{itemize}
\item {Grp. gram.:f.}
\end{itemize}
(V.mundanidade)
\section{Mundanidade}
\begin{itemize}
\item {Grp. gram.:f.}
\end{itemize}
\begin{itemize}
\item {Utilização:Ext.}
\end{itemize}
Qualidade do que é mundano.
Tudo que é relativo ao mundo ou que não é espiritual.
Tendência para os gozos materiaes.
Vida desregrada.
\section{Mundanismo}
\begin{itemize}
\item {Grp. gram.:m.}
\end{itemize}
\begin{itemize}
\item {Proveniência:(De \textunderscore mundano\textunderscore )}
\end{itemize}
Vida mundana.
Systema ou hábito dos que só procuram gozos materiaes.
\section{Mundano}
\begin{itemize}
\item {Grp. gram.:adj.}
\end{itemize}
\begin{itemize}
\item {Proveniência:(Lat. \textunderscore mundanus\textunderscore )}
\end{itemize}
Relativo ao mundo, (encarado êste pelo lado material e transitório).
Dado a gozos materiaes.
\section{Mundanoso}
\begin{itemize}
\item {Grp. gram.:adj.}
\end{itemize}
\begin{itemize}
\item {Utilização:P. us.}
\end{itemize}
O mesmo que \textunderscore mundano\textunderscore .
\section{Mundas}
\begin{itemize}
\item {Grp. gram.:m. pl.}
\end{itemize}
Tríbo indiana.
\section{Mundaú}
\begin{itemize}
\item {Grp. gram.:m.}
\end{itemize}
Planta euphorbiácea do Brasil.
\section{Mundável}
\begin{itemize}
\item {Grp. gram.:adj.}
\end{itemize}
\begin{itemize}
\item {Utilização:Ant.}
\end{itemize}
O mesmo que \textunderscore mundano\textunderscore ; dissoluto, libertino.
\section{Mundé}
\begin{itemize}
\item {Grp. gram.:m.}
\end{itemize}
O mesmo que \textunderscore mundéu\textunderscore .
\section{Mundembas}
\begin{itemize}
\item {Grp. gram.:m. pl.}
\end{itemize}
Tríbo cafreal.
\section{Mundéo}
\begin{itemize}
\item {Grp. gram.:m.}
\end{itemize}
Armadilha, com que no Brasil se apanham animaes.
(Do tupi \textunderscore mondé\textunderscore )
\section{Mundequetes}
\begin{itemize}
\item {Grp. gram.:m. pl.}
\end{itemize}
Povo africano que, parece, habitava junto ás nascentes do Zaire. Cf. Barros, \textunderscore Déc.\textunderscore , I e III, c. 9.
\section{Mundéu}
\begin{itemize}
\item {Grp. gram.:m.}
\end{itemize}
Armadilha, com que no Brasil se apanham animaes.
(Do tupi \textunderscore mondé\textunderscore )
\section{Múndia}
\begin{itemize}
\item {Grp. gram.:f.}
\end{itemize}
Gênero de plantas polygaleáceas.
\section{Mundiaíla}
\begin{itemize}
\item {Grp. gram.:f.}
\end{itemize}
Árvore angolense, cuja madeira tem propriedades análogas ás do buxo.
\section{Mundial}
\begin{itemize}
\item {Grp. gram.:adj.}
\end{itemize}
\begin{itemize}
\item {Proveniência:(Lat. \textunderscore mundialis\textunderscore )}
\end{itemize}
Relativo ao mundo; geral.
\section{Mundianhoca}
\begin{itemize}
\item {Grp. gram.:f.}
\end{itemize}
(V.fedegosa)
\section{Mundice}
\begin{itemize}
\item {Grp. gram.:f.}
\end{itemize}
O mesmo que \textunderscore mundícia\textunderscore ^1. Cf. Camillo, \textunderscore Noites de Insómn.\textunderscore  VII, 26.
\section{Mundice}
\begin{itemize}
\item {Grp. gram.:f.}
\end{itemize}
\begin{itemize}
\item {Utilização:Prov.}
\end{itemize}
Os porcos.
Rebanho de cabras e ovelhas. Cf. Camillo, \textunderscore Noites de Insómn.\textunderscore , VII, 26.
(Aphérese de \textunderscore immundície\textunderscore ?)
\section{Mundícia}
\begin{itemize}
\item {Grp. gram.:adj.}
\end{itemize}
\begin{itemize}
\item {Proveniência:(Lat. \textunderscore munditia\textunderscore )}
\end{itemize}
Limpeza; asseio.
Amor ao asseio.
\section{Mundícia}
\begin{itemize}
\item {Grp. gram.:f.}
\end{itemize}
\begin{itemize}
\item {Utilização:Prov.}
\end{itemize}
Os porcos.
Rebanho de cabras e ovelhas. Cf. Camillo, \textunderscore Noites de Insómn.\textunderscore , VII, 26.
(Aphérese de \textunderscore immundície\textunderscore ?)
\section{Mundície}
\begin{itemize}
\item {Grp. gram.:f.}
\end{itemize}
\begin{itemize}
\item {Proveniência:(Lat. \textunderscore mundities\textunderscore )}
\end{itemize}
O mesmo que \textunderscore mundícia\textunderscore ^1.
\section{Mundificação}
\begin{itemize}
\item {Grp. gram.:f.}
\end{itemize}
Acto ou effeito de mundificar.
\section{Mundificante}
\begin{itemize}
\item {Grp. gram.:adj.}
\end{itemize}
\begin{itemize}
\item {Proveniência:(Lat. \textunderscore mundificans\textunderscore )}
\end{itemize}
Que mundifica.
\section{Mundificar}
\begin{itemize}
\item {Grp. gram.:v. t.}
\end{itemize}
\begin{itemize}
\item {Utilização:Fig.}
\end{itemize}
\begin{itemize}
\item {Proveniência:(Lat. \textunderscore mundificare\textunderscore )}
\end{itemize}
Limpar; assear.
Purificar.
\section{Mundificativo}
\begin{itemize}
\item {Grp. gram.:adj.}
\end{itemize}
\begin{itemize}
\item {Proveniência:(De \textunderscore mundificar\textunderscore )}
\end{itemize}
O mesmo que \textunderscore mundificante\textunderscore .
\section{Mundinóvi}
\begin{itemize}
\item {Grp. gram.:m.}
\end{itemize}
Espécie de cosmorama ou càmara óptica.
\section{Mundo}
\begin{itemize}
\item {Grp. gram.:m.}
\end{itemize}
\begin{itemize}
\item {Utilização:Prov.}
\end{itemize}
\begin{itemize}
\item {Utilização:trasm.}
\end{itemize}
\begin{itemize}
\item {Grp. gram.:Adj.}
\end{itemize}
\begin{itemize}
\item {Proveniência:(Lat. \textunderscore mundus\textunderscore )}
\end{itemize}
Conjunto de espaço, corpos e seres, que a vista humana póde abranger.
Conjunto dos astros, a que o Sol serve de centro.
Universo.
Globo terrestre.
Cada um dos dois grandes continentes, o antigo e o novo.
Humanidade.
A maioria da humanidade: \textunderscore endireitar o mundo\textunderscore .
Tudo que o desejo ou a intelligência póde abranger.
A vida presente.
Esphera armilar.
Categoria social: \textunderscore o mundo operário\textunderscore .
Conjunto de seres ou de phenómenos, que constituem um todo.
Os prazeres materiaes da vida.
\textunderscore O outro mundo\textunderscore , a vida de além-túmulo; vida eterna.
\textunderscore Pôr-se no mundo\textunderscore , fugir.
Mundificado, limpo; puro.
\section{Mundombes}
\begin{itemize}
\item {Grp. gram.:m. pl.}
\end{itemize}
Povos de raça cafreal, em Angola.
\section{Mundondo}
\begin{itemize}
\item {Grp. gram.:m.}
\end{itemize}
Planta angolense, trepadeira, de grandes dimensões.
\section{Mundongama}
\begin{itemize}
\item {Grp. gram.:f.}
\end{itemize}
Árvore de Angola.
\section{Mundungu}
\begin{itemize}
\item {Grp. gram.:m.}
\end{itemize}
Árvore de Angola.
\section{Munduri}
\begin{itemize}
\item {Grp. gram.:m.}
\end{itemize}
\begin{itemize}
\item {Utilização:Bras. do N}
\end{itemize}
Espécie de abelha.
\section{Mundurucús}
\begin{itemize}
\item {Grp. gram.:m. pl.}
\end{itemize}
Índios do Pará, na margem esquerda do Tapajós.
\section{Munerário}
\begin{itemize}
\item {Grp. gram.:m.}
\end{itemize}
\begin{itemize}
\item {Proveniência:(Lat. \textunderscore munerarius\textunderscore )}
\end{itemize}
Aquelle que dava espectáculos de gladiadores, entre os Romanos.
\section{Munga}
\begin{itemize}
\item {Grp. gram.:f.}
\end{itemize}
\begin{itemize}
\item {Utilização:Ant.}
\end{itemize}
O mesmo que \textunderscore monja\textunderscore .
\section{Mungai}
\begin{itemize}
\item {Grp. gram.:m.}
\end{itemize}
O mesmo que \textunderscore mungaia\textunderscore .
\section{Mungaia}
\begin{itemize}
\item {Grp. gram.:f.}
\end{itemize}
Grande árvore de Angola, de madeira forte e resistente.
\section{Munganga}
\begin{itemize}
\item {Grp. gram.:f.}
\end{itemize}
\begin{itemize}
\item {Utilização:Bras. do N}
\end{itemize}
Esgares; caretas; momice.
\section{Mungange}
\begin{itemize}
\item {Grp. gram.:m.}
\end{itemize}
Nome de duas aves de Angola.
\section{Mungare}
\begin{itemize}
\item {Grp. gram.:m.}
\end{itemize}
Grande árvore moçambicana, de que se faz bom tabuado.
\section{Munga-zuigazi}
\begin{itemize}
\item {Grp. gram.:f.}
\end{itemize}
Trepadeira moçambicana.
\section{Mungida}
\begin{itemize}
\item {Grp. gram.:f.}
\end{itemize}
\begin{itemize}
\item {Proveniência:(De \textunderscore mungir\textunderscore )}
\end{itemize}
O mesmo que \textunderscore mungidura\textunderscore . Cf. Castilho, \textunderscore Geórg.\textunderscore  195.
\section{Mungidura}
\begin{itemize}
\item {Grp. gram.:f.}
\end{itemize}
Acto de mungir.
Porção de leite mungido.
\section{Mungil}
\begin{itemize}
\item {Grp. gram.:m.}
\end{itemize}
O mesmo que \textunderscore mongil\textunderscore ^1.
\section{Mungimento}
\begin{itemize}
\item {Grp. gram.:m.}
\end{itemize}
Acto de mungir.
\section{Munginge}
\begin{itemize}
\item {Grp. gram.:m.}
\end{itemize}
Árvore angolense.
\section{Mungir}
\begin{itemize}
\item {Grp. gram.:v. t.}
\end{itemize}
\begin{itemize}
\item {Utilização:Fig.}
\end{itemize}
\begin{itemize}
\item {Proveniência:(Lat. \textunderscore mulgere\textunderscore )}
\end{itemize}
Extrahir das tetas (leite).
Ordenhar.
Explorar.
Despejar.
\section{Mungo}
\begin{itemize}
\item {Grp. gram.:m.}
\end{itemize}
Formosa árvore africana, (\textunderscore nauclea stipulosa\textunderscore , De-Cand.; segundo outros, \textunderscore nauclea bracteosa\textunderscore ).
Não será o mesmo que \textunderscore mungo\textunderscore ^2.
\section{Mungo}
\begin{itemize}
\item {Grp. gram.:m.}
\end{itemize}
Árvore indiana leguminosa, (\textunderscore phaseolus mungo\textunderscore , Lin.).
Fruto dessa árvore. Cf. \textunderscore Livro dos Pesos da Ymdia\textunderscore , 24; Garcia Orta, \textunderscore Coll.\textunderscore  XXXVI.
\section{Mungo}
\begin{itemize}
\item {Grp. gram.:m.}
\end{itemize}
O mesmo que \textunderscore mengo\textunderscore .
\section{Mungolo}
\begin{itemize}
\item {Grp. gram.:m.}
\end{itemize}
Árvore angolense, empregada em armação de telhados e andaimes.
\section{Mungombei}
\begin{itemize}
\item {Grp. gram.:m.}
\end{itemize}
Árvore de Angola.
\section{Mungondo}
\begin{itemize}
\item {Grp. gram.:m.}
\end{itemize}
Ave africana, (\textunderscore bradyornis ater\textunderscore ).
\section{Mungororo}
\begin{itemize}
\item {Grp. gram.:m.}
\end{itemize}
Árvore fructífera de Moçambique.
\section{Mungu}
\begin{itemize}
\item {Grp. gram.:m.}
\end{itemize}
Árvore angolense.
Provavelmente o mesmo que \textunderscore mungo\textunderscore ^1.
\section{Munguai}
\begin{itemize}
\item {Grp. gram.:m.}
\end{itemize}
Árvore de Angola.
(O mesmo que \textunderscore mungai\textunderscore ?)
\section{Mungubeira}
\begin{itemize}
\item {Grp. gram.:f.}
\end{itemize}
Árvore bombácea do Brasil.
\section{Mungue}
\begin{itemize}
\item {Grp. gram.:m.}
\end{itemize}
Arbusto africano, sarmentoso, talvez da fam. das convolvuláceas.
\section{Munguengue}
\begin{itemize}
\item {Grp. gram.:m.}
\end{itemize}
\begin{itemize}
\item {Proveniência:(T. de Angola)}
\end{itemize}
Árvore anacardiácea, de frutos comestíveis, (\textunderscore spondias lutea\textunderscore , Lin.).
\section{Mungulfe}
\begin{itemize}
\item {Grp. gram.:m.}
\end{itemize}
Pequena árvore africana, da fam. das leguminosas.
\section{Mungumbi}
\begin{itemize}
\item {Grp. gram.:m.}
\end{itemize}
Árvore de Angola.
\section{Mungundo}
\begin{itemize}
\item {Grp. gram.:m.}
\end{itemize}
Árvore angolense, (\textunderscore symphonia globulifera\textunderscore , Lin.).
\section{Mungunzá}
\begin{itemize}
\item {Grp. gram.:m.}
\end{itemize}
\begin{itemize}
\item {Utilização:Bras. do N}
\end{itemize}
Papas de grãos de milho, que se cozem inteiros.
\section{Munguzá}
\begin{itemize}
\item {Grp. gram.:m.}
\end{itemize}
\begin{itemize}
\item {Utilização:Bras. do N}
\end{itemize}
Papas de grãos de milho, que se cozem inteiros.
\section{Munha}
\begin{itemize}
\item {Grp. gram.:f.}
\end{itemize}
\begin{itemize}
\item {Utilização:Prov.}
\end{itemize}
Caruma sêca.
O mesmo que \textunderscore munho\textunderscore .
\section{Munhambe}
\begin{itemize}
\item {Grp. gram.:m.}
\end{itemize}
Árvore angolense.
\section{Munhande}
\begin{itemize}
\item {Grp. gram.:m.}
\end{itemize}
Árvore de Angola.
\section{Munhanecas}
\begin{itemize}
\item {Grp. gram.:m. pl.}
\end{itemize}
Povos de raça cafreal, em Angola.
\section{Munhango}
\begin{itemize}
\item {Grp. gram.:m.}
\end{itemize}
Árvore de Angola.
\section{Munhangolo}
\begin{itemize}
\item {Grp. gram.:m.}
\end{itemize}
Arbusto africano, cujo fruto é semelhante a um pequeno morango.
\section{Munhanoca}
\begin{itemize}
\item {Grp. gram.:f.}
\end{itemize}
Designação vulgar, na África portuguesa, do \textunderscore fedegoso\textunderscore , planta.
\section{Munhão}
\begin{itemize}
\item {Grp. gram.:m.}
\end{itemize}
Eixo, quási a meio do comprimento de uma peça de artilharia, para que esta se possa abaixar ou levantar, segundo a conveniência da pontaria.
(Cast. \textunderscore muñon\textunderscore )
\section{Munheca}
\begin{itemize}
\item {Grp. gram.:f.}
\end{itemize}
Parte do corpo, em que a mão se liga ao braço; pulso.
(Cast. \textunderscore muñeca\textunderscore )
\section{Munhere}
\begin{itemize}
\item {Grp. gram.:m.}
\end{itemize}
Árvore angolense, espécie de acácia.
\section{Munhime}
\begin{itemize}
\item {Grp. gram.:m.}
\end{itemize}
Árvore angolense.
\section{Munho}
\begin{itemize}
\item {Grp. gram.:m.}
\end{itemize}
\begin{itemize}
\item {Utilização:Prov.}
\end{itemize}
\begin{itemize}
\item {Utilização:minh.}
\end{itemize}
\begin{itemize}
\item {Utilização:Prov.}
\end{itemize}
O mesmo que \textunderscore moínho\textunderscore .
Pellículas, que envolvem os grãos de milho na espiga.
\section{Munhoneira}
\begin{itemize}
\item {Grp. gram.:f.}
\end{itemize}
\begin{itemize}
\item {Proveniência:(De \textunderscore munhão\textunderscore )}
\end{itemize}
Encaixe, em que assenta o munhão.
\section{Múni}
\begin{itemize}
\item {Grp. gram.:m.}
\end{itemize}
Designação genérica do homem piedoso e sábio, entre os Índios.
\section{Munição}
\begin{itemize}
\item {Grp. gram.:f.}
\end{itemize}
\begin{itemize}
\item {Utilização:Fig.}
\end{itemize}
\begin{itemize}
\item {Proveniência:(Lat. \textunderscore munitio\textunderscore )}
\end{itemize}
Fortificação de uma praça.
Defesa.
Preservativo.
Provisão de alimentos ou do que é preciso a uma porção de tropas.
Chumbo para a caça dos pássaros.
\textunderscore Pão de munição\textunderscore , pão grosseiro, para ração de soldados.
\section{Munichis}
\begin{itemize}
\item {Grp. gram.:m.}
\end{itemize}
Tríbo de Índios americanos.
\section{Municiamento}
\begin{itemize}
\item {Grp. gram.:m.}
\end{itemize}
Acto ou effeito de municionar.
\section{Municionar}
\begin{itemize}
\item {Grp. gram.:v. t.}
\end{itemize}
\begin{itemize}
\item {Proveniência:(Do lat. \textunderscore munitio\textunderscore )}
\end{itemize}
Prover de munições de qualquer espécie.
\section{Municionário}
\begin{itemize}
\item {Grp. gram.:m.}
\end{itemize}
\begin{itemize}
\item {Proveniência:(Do lat. \textunderscore munitio\textunderscore )}
\end{itemize}
O encarregado de municionar tropas.
\section{Munício}
\begin{itemize}
\item {Grp. gram.:m.}
\end{itemize}
\begin{itemize}
\item {Utilização:Pop.}
\end{itemize}
Pão ordinário, que faz parte do rancho dos soldados. Cf. Camillo, \textunderscore Brasileira\textunderscore , 188.
(Cp. \textunderscore munição\textunderscore )
\section{Municipal}
\begin{itemize}
\item {Grp. gram.:adj.}
\end{itemize}
\begin{itemize}
\item {Grp. gram.:M.}
\end{itemize}
\begin{itemize}
\item {Utilização:Pop.}
\end{itemize}
\begin{itemize}
\item {Grp. gram.:F.}
\end{itemize}
\begin{itemize}
\item {Utilização:Pop.}
\end{itemize}
\begin{itemize}
\item {Proveniência:(Lat. \textunderscore municipalis\textunderscore )}
\end{itemize}
Relativo a município: \textunderscore câmara municipal\textunderscore .
Soldado da guarda municipal.
Corpo de tropas, que constituía a guarda municipal.
\section{Municipalense}
\begin{itemize}
\item {Grp. gram.:adj.}
\end{itemize}
\begin{itemize}
\item {Utilização:P. us.}
\end{itemize}
O mesmo que \textunderscore municipal\textunderscore .
\section{Municipalidade}
\begin{itemize}
\item {Grp. gram.:f.}
\end{itemize}
\begin{itemize}
\item {Proveniência:(De \textunderscore municipal\textunderscore )}
\end{itemize}
Conjunto dos indivíduos, eleitos para gerir os negócios municipaes de interesse colectivo.
Vereação.
Município.
Local, onde celebram suas sessões os vereadores.
Funccionalismo inferior, dependente da vereação.
\section{Municipalismo}
\begin{itemize}
\item {Grp. gram.:m.}
\end{itemize}
\begin{itemize}
\item {Proveniência:(De \textunderscore municipal\textunderscore )}
\end{itemize}
Systema de administração, que attende especialmente á organização e prerogativas dos municípios.
Descentralização da administração pública, em favor dos municípios. Cf. Herculano, \textunderscore Hist. de Port.\textunderscore , III, 222; IV, 16.
\section{Municipalista}
\begin{itemize}
\item {Grp. gram.:adj.}
\end{itemize}
\begin{itemize}
\item {Grp. gram.:M.}
\end{itemize}
\begin{itemize}
\item {Proveniência:(De \textunderscore municipal\textunderscore )}
\end{itemize}
Relativo ao municipalismo.
Partidário do municipalismo.
\section{Municipalmente}
\begin{itemize}
\item {Grp. gram.:adv.}
\end{itemize}
\begin{itemize}
\item {Proveniência:(De \textunderscore municipal\textunderscore )}
\end{itemize}
Conforme aos usos municipaes.
\section{Munícipe}
\begin{itemize}
\item {Grp. gram.:m.  e  adj.}
\end{itemize}
\begin{itemize}
\item {Proveniência:(Lat. \textunderscore municeps\textunderscore )}
\end{itemize}
Cada um dos cidadãos de um município.
\section{Município}
\begin{itemize}
\item {Grp. gram.:m.}
\end{itemize}
\begin{itemize}
\item {Proveniência:(Lat. \textunderscore municipium\textunderscore )}
\end{itemize}
Cada uma das circunscripções territoriaes, em que se exerce a jurisdicção de uma vereação.
Concelho.
Habitantes de um concelho.
\section{Munificência}
\begin{itemize}
\item {Grp. gram.:f.}
\end{itemize}
\begin{itemize}
\item {Proveniência:(Lat. \textunderscore munificentia\textunderscore )}
\end{itemize}
Qualidade de munificente.
Generosidade.
\section{Munificente}
\begin{itemize}
\item {Grp. gram.:adj.}
\end{itemize}
\begin{itemize}
\item {Proveniência:(Lat. \textunderscore munificens\textunderscore )}
\end{itemize}
Generoso.
Bizarro, magnânimo.
\section{Munífico}
\begin{itemize}
\item {Grp. gram.:adj.}
\end{itemize}
\begin{itemize}
\item {Proveniência:(Lat. \textunderscore munificus\textunderscore )}
\end{itemize}
O mesmo que \textunderscore munificente\textunderscore .
\section{Muningo}
\begin{itemize}
\item {Grp. gram.:m.}
\end{itemize}
Ave nocturna de rapina, em África.
\section{Muninhé}
\begin{itemize}
\item {Grp. gram.:m.}
\end{itemize}
Árvore angolense, espécie de espinheiro.
\section{Munir}
\begin{itemize}
\item {Grp. gram.:v. t.}
\end{itemize}
\begin{itemize}
\item {Proveniência:(Lat. \textunderscore munire\textunderscore )}
\end{itemize}
Defender, fortificar.
Acautelar.
Prover do necessário.
Abastecer de munições.
\section{Munjojos}
\begin{itemize}
\item {Grp. gram.:m. pl.}
\end{itemize}
Tríbo, de origem árabe, predominante no país dos Namarraes, em Moçambique.
\section{Munjolo}
\begin{itemize}
\item {Grp. gram.:m.}
\end{itemize}
\begin{itemize}
\item {Utilização:Bras. do S}
\end{itemize}
\begin{itemize}
\item {Utilização:Bras. do N}
\end{itemize}
\begin{itemize}
\item {Utilização:Bras. do Rio}
\end{itemize}
Máquina agrícola, com que se limpa o milho, tornando-o idóneo para a fabricação da farinha.
Bezerrinho.
Árvore leguminosa.
\section{Munjomba}
\begin{itemize}
\item {Grp. gram.:f.}
\end{itemize}
Árvore de Angola.
\section{Munjovo}
\begin{itemize}
\item {Grp. gram.:m.}
\end{itemize}
Ornato guerreiro, usado á cintura pelos indígenas de Moçambique, e formado de filamentos vegetaes ou de pelles, e de caudas de manguços.
\section{Munjue}
\begin{itemize}
\item {Grp. gram.:m.}
\end{itemize}
Árvore angolense.
\section{Munombumba}
\begin{itemize}
\item {Grp. gram.:f.}
\end{itemize}
Árvore de Angola.
\section{Munongo}
\begin{itemize}
\item {Grp. gram.:m.}
\end{itemize}
Árvore angolense.
\section{Munquia}
\begin{itemize}
\item {Grp. gram.:f.}
\end{itemize}
Árvore de Angola.
\section{Munquir}
\begin{itemize}
\item {Grp. gram.:v. i.}
\end{itemize}
\begin{itemize}
\item {Utilização:Prov.}
\end{itemize}
Mastigar e comer sem abrir a bôca.
(Cp. \textunderscore moquir\textunderscore )
\section{Muntalandonga}
\begin{itemize}
\item {Grp. gram.:m.}
\end{itemize}
Nova espécie de reptis, descoberta nos rios Luando e Cuanza, (\textunderscore euprepes ivensis\textunderscore ).
\section{Muntinta}
\begin{itemize}
\item {Grp. gram.:f.}
\end{itemize}
Árvore angolense.
\section{Munto}
\begin{itemize}
\item {Grp. gram.:adj.  e  adv.}
\end{itemize}
\begin{itemize}
\item {Utilização:Pop.}
\end{itemize}
O mesmo que \textunderscore muito\textunderscore ^1.
\section{Munto}
\begin{itemize}
\item {Grp. gram.:m.}
\end{itemize}
Árvore de Angola.
\section{Muntumbilo}
\begin{itemize}
\item {Grp. gram.:m.}
\end{itemize}
Árvore angolense.
\section{Muntumbiri}
\begin{itemize}
\item {Grp. gram.:m.}
\end{itemize}
Árvore angolense.
\section{Munumucaia}
\begin{itemize}
\item {Grp. gram.:m.}
\end{itemize}
\begin{itemize}
\item {Utilização:T. da Áfr. Or. Port}
\end{itemize}
O mesmo que \textunderscore tufão\textunderscore ^1.
\section{Munupiú}
\begin{itemize}
\item {Grp. gram.:m.}
\end{itemize}
Planta euphorbiácea do Brasil, (\textunderscore sapium\textunderscore ).
\section{Munuru}
\begin{itemize}
\item {Grp. gram.:m.}
\end{itemize}
Árvore das regiões do Amazonas.
\section{Múnus}
\begin{itemize}
\item {Grp. gram.:m.}
\end{itemize}
\begin{itemize}
\item {Proveniência:(Lat. \textunderscore munus\textunderscore )}
\end{itemize}
Encargo.
Emprêgo.
Funcções que um indivíduo tem de exercer.
\section{Munzóni}
\begin{itemize}
\item {Grp. gram.:m.}
\end{itemize}
Ave pernalta da África.
\section{Munzuá}
\begin{itemize}
\item {Grp. gram.:m.}
\end{itemize}
\begin{itemize}
\item {Utilização:Bras}
\end{itemize}
Espécie de nassa afunilada, feita de fasquias de tacuara.
\section{Muolo}
\begin{itemize}
\item {Grp. gram.:m.}
\end{itemize}
Árvore angolense.
\section{Muondojola}
\begin{itemize}
\item {Grp. gram.:f.}
\end{itemize}
\begin{itemize}
\item {Proveniência:(T. lund.)}
\end{itemize}
Arbusto africano, de fôlhas oppostas, e flôres em grandes cachos.
\section{Muonumucaia}
\begin{itemize}
\item {Grp. gram.:m.}
\end{itemize}
\begin{itemize}
\item {Utilização:T. da Áfr. Or. Port}
\end{itemize}
O mesmo que \textunderscore tufão\textunderscore ^1.
\section{Mupa}
\begin{itemize}
\item {Grp. gram.:f.}
\end{itemize}
\begin{itemize}
\item {Utilização:T. afr}
\end{itemize}
O mesmo que \textunderscore alpondras\textunderscore .
\section{Mupaco}
\begin{itemize}
\item {Grp. gram.:m.}
\end{itemize}
Nome angolense do \textunderscore pau-ferro\textunderscore .
\section{Mupalaia}
\begin{itemize}
\item {Grp. gram.:f.}
\end{itemize}
Árvore angolense, cuja sombra, segundo affirmam os Indígenas, faz dormir.
\section{Mupanda}
\begin{itemize}
\item {Grp. gram.:f.}
\end{itemize}
Árvore de Angola, cuja casca se emprega no curtimento de coiros.
\section{Mupandambale}
\begin{itemize}
\item {Grp. gram.:m.}
\end{itemize}
O mesmo que \textunderscore mupandambar\textunderscore .
\section{Mupandambar}
\begin{itemize}
\item {Grp. gram.:m.}
\end{itemize}
Árvore de Angola.
\section{Mupandolola}
\begin{itemize}
\item {Grp. gram.:m.}
\end{itemize}
Árvore angolense.
\section{Mupandopando}
\begin{itemize}
\item {Grp. gram.:m.}
\end{itemize}
Pequena árvore inter-tropical.
\section{Mupanduambire}
\begin{itemize}
\item {Grp. gram.:m.}
\end{itemize}
Árvore de Angola.
\section{Mupapa}
\begin{itemize}
\item {Grp. gram.:f.}
\end{itemize}
Árvore de Angola.
\section{Mupapala}
\begin{itemize}
\item {Grp. gram.:f.}
\end{itemize}
Árvore de Angola.
\section{Muparala}
\begin{itemize}
\item {Grp. gram.:f.}
\end{itemize}
Árvore angolense.
\section{Mupeixe}
\begin{itemize}
\item {Grp. gram.:m.}
\end{itemize}
Pequena árvore africana, tortuosa, de fôlhas simples, verde-amareladas, e flôres monécicas.
\section{Mupeque}
\begin{itemize}
\item {Grp. gram.:m.}
\end{itemize}
Árvore angolense, cujo fruto, depois de cozido, fornece um óleo, com que os Indígenas untam o corpo.
\section{Mupicar}
\begin{itemize}
\item {Grp. gram.:v. i.}
\end{itemize}
\begin{itemize}
\item {Utilização:Bras}
\end{itemize}
Remar com ligeireza.
(Do tupi \textunderscore mupica\textunderscore )
\section{Mupinga-ombôa}
\begin{itemize}
\item {Grp. gram.:f.}
\end{itemize}
Árvore angolense.
\section{Muponguluve}
\begin{itemize}
\item {Grp. gram.:m.}
\end{itemize}
Árvore de Angola.
\section{Mupumbes}
\begin{itemize}
\item {Grp. gram.:m. pl.}
\end{itemize}
Tríbo cafreal.
\section{Mupupo}
\begin{itemize}
\item {Grp. gram.:m.}
\end{itemize}
Árvore de Angola.
\section{Muqueca}
\begin{itemize}
\item {Grp. gram.:f.}
\end{itemize}
\begin{itemize}
\item {Utilização:Ant.}
\end{itemize}
Montículo de terra, em que se planta uma estaca, até criar raízes.
\section{Muquende}
\begin{itemize}
\item {Grp. gram.:m.}
\end{itemize}
Ave de rapina, diurna, na África.
\section{Muquengue}
\begin{itemize}
\item {Grp. gram.:m.}
\end{itemize}
Árvore angolense.
\section{Muquequeta}
\begin{itemize}
\item {fónica:quê}
\end{itemize}
\begin{itemize}
\item {Grp. gram.:f.}
\end{itemize}
Árvore angolense, espécie de espinheiro.
\section{Muquequete}
\begin{itemize}
\item {fónica:quê}
\end{itemize}
\begin{itemize}
\item {Grp. gram.:m.}
\end{itemize}
O mesmo que \textunderscore muquequeta\textunderscore .
\section{Muquete}
\begin{itemize}
\item {fónica:quê}
\end{itemize}
\begin{itemize}
\item {Grp. gram.:m.}
\end{itemize}
Árvore de Angola.
Talvez o mesmo que \textunderscore muquequete\textunderscore .
\section{Muquice}
\begin{itemize}
\item {Grp. gram.:m.}
\end{itemize}
\begin{itemize}
\item {Utilização:T. de Angola}
\end{itemize}
Pó, resultante da incineração de certas fôlhas, e com o qual, entre algumas tríbos, se besunta o peito e os braços das mulheres fracas.
\section{Muquiche}
\begin{itemize}
\item {Grp. gram.:m.}
\end{itemize}
Arbusto africano, da fam. das labiadas.
\section{Muquindo}
\begin{itemize}
\item {Grp. gram.:m.}
\end{itemize}
Habitação, que para si constrói o salalé.
\section{Muquirana}
\begin{itemize}
\item {Grp. gram.:f.}
\end{itemize}
\begin{itemize}
\item {Utilização:Bras}
\end{itemize}
Piolho da roupa.
\section{Mura}
\begin{itemize}
\item {Grp. gram.:m.}
\end{itemize}
Antiga medida da Índia portuguesa, correspondente a 735 litros.
\section{Mura}
\begin{itemize}
\item {Grp. gram.:f.}
\end{itemize}
\begin{itemize}
\item {Utilização:Prov.}
\end{itemize}
Acto de \textunderscore murar^2\textunderscore : \textunderscore o gato está na mura\textunderscore .
\section{Mura}
\begin{itemize}
\item {Grp. gram.:f.}
\end{itemize}
\begin{itemize}
\item {Utilização:Prov.}
\end{itemize}
\begin{itemize}
\item {Utilização:minh.}
\end{itemize}
Pinta negra, produzida pelo excesso do calor do lume.
\section{Muraçanga}
\begin{itemize}
\item {Grp. gram.:f.}
\end{itemize}
\begin{itemize}
\item {Utilização:Bras}
\end{itemize}
O mesmo que \textunderscore buraçanga\textunderscore .
\section{Murada}
\begin{itemize}
\item {Grp. gram.:f.}
\end{itemize}
\begin{itemize}
\item {Utilização:Pesc.}
\end{itemize}
\begin{itemize}
\item {Proveniência:(De \textunderscore muro\textunderscore )}
\end{itemize}
Fiada de malhas em toda a largura da rêde.
\section{Muradal}
\begin{itemize}
\item {Grp. gram.:m.}
\end{itemize}
\begin{itemize}
\item {Utilização:Ant.}
\end{itemize}
\begin{itemize}
\item {Proveniência:(De \textunderscore muro\textunderscore )}
\end{itemize}
Montão de caliça, de entulho ou de coisas análogas.

Residência.
\section{Muradoiro}
\begin{itemize}
\item {Grp. gram.:m.}
\end{itemize}
\begin{itemize}
\item {Utilização:Ant.}
\end{itemize}
\begin{itemize}
\item {Proveniência:(De \textunderscore murar\textunderscore ^1)}
\end{itemize}
Estaca arqueada, com que se abrem os lagrimaes nas barachas das marinhas.
Muro, parede.
\section{Muradouro}
\begin{itemize}
\item {Grp. gram.:m.}
\end{itemize}
\begin{itemize}
\item {Utilização:Ant.}
\end{itemize}
\begin{itemize}
\item {Proveniência:(De \textunderscore murar\textunderscore ^1)}
\end{itemize}
Estaca arqueada, com que se abrem os lagrimaes nas barachas das marinhas.
Muro, parede.
\section{Murador}
\begin{itemize}
\item {Grp. gram.:m.  e  adj.}
\end{itemize}
\begin{itemize}
\item {Proveniência:(De \textunderscore murar\textunderscore ^2)}
\end{itemize}
Diz-se do gato que apanha ou caça ratos.
\section{Muragem}
\begin{itemize}
\item {Grp. gram.:f.}
\end{itemize}
\begin{itemize}
\item {Proveniência:(De \textunderscore murar\textunderscore )}
\end{itemize}
Antigo imposto, que se pagava para a reedificação, concerto e conservação de muralhas e monumentos públicos.
\section{Murajuba}
\begin{itemize}
\item {Grp. gram.:f.}
\end{itemize}
Espécie de papagaio da região do Amazonas.
\section{Mural}
\begin{itemize}
\item {Grp. gram.:adj.}
\end{itemize}
\begin{itemize}
\item {Proveniência:(Lat. \textunderscore muralis\textunderscore )}
\end{itemize}
Relativo a muro.
\section{Muralha}
\begin{itemize}
\item {Grp. gram.:f.}
\end{itemize}
\begin{itemize}
\item {Utilização:Ext.}
\end{itemize}
\begin{itemize}
\item {Proveniência:(Lat. \textunderscore muralia\textunderscore )}
\end{itemize}
Muro, que guarnece uma fortaleza ou praça de armas.
Grande muro, paredão.
Qualquer guarnição ou sebe, que defende, resguarda, ou separa como a parede.
Cinta óssea, que fórma a parte externa do pé do cavallo, e que também se chama parede ou taipa.
\section{Muralhado}
\begin{itemize}
\item {Grp. gram.:adj.}
\end{itemize}
Cercado de muralhas; fortificado.
Encerrado como dentro de muralhas.
\section{Muralhar}
\begin{itemize}
\item {Grp. gram.:v. t.}
\end{itemize}
Cercar de muralhas; servir de muralha a:«\textunderscore ...as serras muralhavam o horizonte\textunderscore ». C. Neto, \textunderscore Saldunes\textunderscore .
\section{Muramento}
\begin{itemize}
\item {Grp. gram.:m.}
\end{itemize}
Acto ou effeito de \textunderscore murar\textunderscore ^2.
Muralha; fortificação. Cf. Camillo, \textunderscore Quéda\textunderscore , 156.
\section{Murangane}
\begin{itemize}
\item {Grp. gram.:m.}
\end{itemize}
Árvore fructífera de Moçambique.
\section{Murapiranga}
\begin{itemize}
\item {Grp. gram.:f.}
\end{itemize}
\begin{itemize}
\item {Utilização:Bras}
\end{itemize}
Árvore silvestre, de bôa madeira para construcções.
\section{Muraqueteca}
\begin{itemize}
\item {Grp. gram.:f.}
\end{itemize}
Cipó medicinal da região do Amazonas.
\section{Murar}
\begin{itemize}
\item {Grp. gram.:v. t.}
\end{itemize}
\begin{itemize}
\item {Proveniência:(Lat. \textunderscore murare\textunderscore )}
\end{itemize}
Cercar de muro ou muros.
Fortificar.
Defender contra assaltos.
\section{Murar}
\begin{itemize}
\item {Grp. gram.:v. t.}
\end{itemize}
\begin{itemize}
\item {Grp. gram.:V. i.}
\end{itemize}
\begin{itemize}
\item {Proveniência:(Do lat. \textunderscore mus\textunderscore , \textunderscore muris\textunderscore )}
\end{itemize}
Espiar ou espreitar (ratos), para os caçar, (falando-se do gato).
Caçar ratos.
\section{Muras}
\begin{itemize}
\item {Grp. gram.:m. pl.}
\end{itemize}
Índios do Brasil, que habitaram nas margens do Madeira.
\section{Murça}
\begin{itemize}
\item {Grp. gram.:f.}
\end{itemize}
\begin{itemize}
\item {Proveniência:(Do al. \textunderscore mutze\textunderscore )}
\end{itemize}
Vestimenta de côr, em fórma de cabeção, usada pelos cónegos, em cima da sobrepelliz.
\section{Murça}
\begin{itemize}
\item {Grp. gram.:f.}
\end{itemize}
Espécie de lima, com serrilha ou picado fino.
\section{Murceiro}
\begin{itemize}
\item {Grp. gram.:m.}
\end{itemize}
Fabricante ou vendedor de murças.
\section{Murcha}
\begin{itemize}
\item {Grp. gram.:f.}
\end{itemize}
Acto ou effeito de murchar.
\section{Murchar}
\begin{itemize}
\item {Grp. gram.:v. t.}
\end{itemize}
\begin{itemize}
\item {Utilização:Fig.}
\end{itemize}
\begin{itemize}
\item {Grp. gram.:V. i.}
\end{itemize}
Tornar murcho.
Privar da frescura ou víço.
Tirar a fôrça ou a intensidade a.
Perder o viço ou a frescura.
Tornar-se triste.
Perder a belleza, a côr ou o brilho.
\section{Murchecer}
\begin{itemize}
\item {Grp. gram.:v. t.  e  i.}
\end{itemize}
O mesmo que \textunderscore emmurchecer\textunderscore .
\section{Murchecível}
\begin{itemize}
\item {Grp. gram.:adj.}
\end{itemize}
Que póde murchecer.
\section{Murchidão}
\begin{itemize}
\item {Grp. gram.:f.}
\end{itemize}
\begin{itemize}
\item {Proveniência:(De \textunderscore murcho\textunderscore )}
\end{itemize}
Estado daquillo que murchou.
\section{Murcho}
\begin{itemize}
\item {Grp. gram.:adj.}
\end{itemize}
\begin{itemize}
\item {Utilização:Fig.}
\end{itemize}
\begin{itemize}
\item {Proveniência:(Do lat. hyp. \textunderscore murculus\textunderscore , dem. de \textunderscore murcus\textunderscore ?)}
\end{itemize}
Que perdeu a frescura, o viço, a côr ou a belleza: \textunderscore flôres murchas\textunderscore .
Que perdeu a fôrça ou a energia.
Triste; pensativo.
\section{Murchoso}
\begin{itemize}
\item {Grp. gram.:adj.}
\end{itemize}
\begin{itemize}
\item {Utilização:Bot.}
\end{itemize}
\begin{itemize}
\item {Proveniência:(De \textunderscore murchar\textunderscore )}
\end{itemize}
O mesmo que \textunderscore marcescente\textunderscore .
\section{Murciana}
\begin{itemize}
\item {Grp. gram.:f.  e  adj.}
\end{itemize}
\begin{itemize}
\item {Proveniência:(De \textunderscore murciano\textunderscore )}
\end{itemize}
Diz-se de uma espécie de couve.
\section{Murciano}
\begin{itemize}
\item {Grp. gram.:adj.}
\end{itemize}
Relativo a Múrcia.
\section{Murco}
\begin{itemize}
\item {Grp. gram.:m.}
\end{itemize}
\begin{itemize}
\item {Proveniência:(Lat. \textunderscore murcus\textunderscore )}
\end{itemize}
Indivíduo que, entre os Romanos, cortava o dedo pollegar, para se eximir do serviço militar.
\section{Murdanga}
\begin{itemize}
\item {Grp. gram.:f.}
\end{itemize}
O mesmo que \textunderscore mordango\textunderscore . Cf. L. Mendes, \textunderscore Índia Port.\textunderscore 
\section{Murear}
\begin{itemize}
\item {Grp. gram.:v. t.}
\end{itemize}
\begin{itemize}
\item {Utilização:Prov.}
\end{itemize}
\begin{itemize}
\item {Utilização:trasm.}
\end{itemize}
Cercar de muro, murar^1.
\section{Muregona}
\begin{itemize}
\item {Grp. gram.:f.}
\end{itemize}
\begin{itemize}
\item {Utilização:Pesc.}
\end{itemize}
Armadilha de vêrga, em fórma espheróide.
\section{Mureira}
\begin{itemize}
\item {Grp. gram.:f.}
\end{itemize}
\begin{itemize}
\item {Proveniência:(De \textunderscore muro\textunderscore )}
\end{itemize}
Montão de estrume, ordinariamente ao pé de um muro; estrumeira.
\section{Muremuré}
\begin{itemize}
\item {Grp. gram.:m.}
\end{itemize}
O mesmo que \textunderscore murmuré\textunderscore .
\section{Mures}
\begin{itemize}
\item {Grp. gram.:m. pl.}
\end{itemize}
\begin{itemize}
\item {Utilização:Ant.}
\end{itemize}
\begin{itemize}
\item {Proveniência:(Lat. \textunderscore mus\textunderscore , \textunderscore muris\textunderscore )}
\end{itemize}
Ratos.
\section{Murganho}
\begin{itemize}
\item {Grp. gram.:m.}
\end{itemize}
\begin{itemize}
\item {Utilização:Prov.}
\end{itemize}
\begin{itemize}
\item {Proveniência:(Do lat. hyp. \textunderscore muricaneus\textunderscore )}
\end{itemize}
Rato pequeno, acastanhado.
Criança enfèzada, franzina.
\section{Murgeira}
\begin{itemize}
\item {Grp. gram.:f.}
\end{itemize}
Rêde de arrastar para terra, usada na ria de Aveiro.
(Por \textunderscore mugeira\textunderscore , de \textunderscore mugem\textunderscore ?)
\section{Múria}
\begin{itemize}
\item {Grp. gram.:f.}
\end{itemize}
\begin{itemize}
\item {Proveniência:(Lat. \textunderscore muria\textunderscore )}
\end{itemize}
Salmoira, feita do pingo do atum.
\section{Múria-á-pembe}
\begin{itemize}
\item {Grp. gram.:m.}
\end{itemize}
Árvore africana, de fôlhas sempre verdes, inteiras, oblongas, e inflorescência em cachos pendentes.
\section{Múria-candombe}
\begin{itemize}
\item {Grp. gram.:m.}
\end{itemize}
(V.mudiangila)
\section{Muriaçu}
\begin{itemize}
\item {Grp. gram.:m.}
\end{itemize}
\begin{itemize}
\item {Utilização:Bras}
\end{itemize}
O mesmo que \textunderscore muriciaçu\textunderscore .
\section{Muria-maembe}
\begin{itemize}
\item {Grp. gram.:m.}
\end{itemize}
Arbusto africano, de fôlhas simples, cuja infusão é applicada pelos indígenas na cura de feridas siphylíticas.
\section{Muriambambe}
\begin{itemize}
\item {Grp. gram.:m.}
\end{itemize}
Nome, que os indígenas da África portuguesa dão ao cafezeiro.
\section{Murianganga}
\begin{itemize}
\item {Grp. gram.:f.}
\end{itemize}
Árvore angolense de Caconda.
\section{Muriangombe}
\begin{itemize}
\item {Grp. gram.:m.}
\end{itemize}
Arvoreta capparidácea de Angola, (\textunderscore maerua angolensis\textunderscore , De-Cand.).
\section{Muriatado}
\begin{itemize}
\item {Grp. gram.:adj.}
\end{itemize}
\begin{itemize}
\item {Utilização:Chím.}
\end{itemize}
Dizia-se de uma base combinada com o ácido muriático.
\section{Muriático}
\begin{itemize}
\item {Grp. gram.:adj.}
\end{itemize}
\begin{itemize}
\item {Proveniência:(Lat. \textunderscore muriaticus\textunderscore )}
\end{itemize}
Diz-se de um ácido, formado de hydrogênio e chloro.
O mesmo que \textunderscore chlorhýdrico\textunderscore .
\section{Muriato}
\begin{itemize}
\item {Grp. gram.:m.}
\end{itemize}
\begin{itemize}
\item {Utilização:Chím.}
\end{itemize}
\begin{itemize}
\item {Proveniência:(Do lat. \textunderscore muria\textunderscore )}
\end{itemize}
Designação antiga do chlorhydrato.
\section{Muribixaba}
\begin{itemize}
\item {Grp. gram.:m.}
\end{itemize}
\begin{itemize}
\item {Utilização:Bras. do N}
\end{itemize}
Chefe indígena, entre alguns selvagens do Amazonas.
\section{Múrice}
\begin{itemize}
\item {Grp. gram.:m.}
\end{itemize}
\begin{itemize}
\item {Utilização:Poét.}
\end{itemize}
\begin{itemize}
\item {Proveniência:(Lat. \textunderscore murex\textunderscore )}
\end{itemize}
Mollusco gasterópode, purpurífero.
Púrpura.
\section{Murícea}
\begin{itemize}
\item {Grp. gram.:f.}
\end{itemize}
Gênero de polypeiros.
(Cp. \textunderscore múrice\textunderscore )
\section{Murici}
\begin{itemize}
\item {Grp. gram.:m.}
\end{itemize}
Gênero de plantas malpigiáceas do Brasil.
\section{Muriciaçu}
\begin{itemize}
\item {Grp. gram.:m.}
\end{itemize}
\begin{itemize}
\item {Utilização:Bras}
\end{itemize}
Espécie de murici.
\section{Muricida}
\begin{itemize}
\item {Grp. gram.:adj.}
\end{itemize}
\begin{itemize}
\item {Proveniência:(Do lat. \textunderscore mus\textunderscore , \textunderscore muris\textunderscore  + \textunderscore caedere\textunderscore )}
\end{itemize}
Que mata os ratos: \textunderscore pós muricidas\textunderscore .
\section{Muricito}
\begin{itemize}
\item {Grp. gram.:m.}
\end{itemize}
Espécie de múrice fóssil.
\section{Muriçoca}
\begin{itemize}
\item {Grp. gram.:f.}
\end{itemize}
Insecto do Brasil, (\textunderscore steogomia fasciata\textunderscore ).
\section{Murídeo}
\begin{itemize}
\item {Grp. gram.:adj.}
\end{itemize}
\begin{itemize}
\item {Grp. gram.:M. pl.}
\end{itemize}
\begin{itemize}
\item {Proveniência:(Do lat. \textunderscore mus\textunderscore , \textunderscore muris\textunderscore  + gr. \textunderscore eidos\textunderscore )}
\end{itemize}
Relativo ou semelhante ao rato.
Família de mammíferos roedores, que abrange os ratos.
\section{Murilahonde}
\begin{itemize}
\item {Grp. gram.:m.}
\end{itemize}
Pequena árvore angolense, de seiva encarnada.
\section{Murili}
\begin{itemize}
\item {Grp. gram.:m.}
\end{itemize}
O mesmo que \textunderscore muriti\textunderscore .--Talvez escrita errónea, tendo-se trocado o \textunderscore t\textunderscore  por \textunderscore l\textunderscore .
\section{Murino}
\begin{itemize}
\item {Grp. gram.:adj.}
\end{itemize}
\begin{itemize}
\item {Proveniência:(Lat. \textunderscore murinus\textunderscore )}
\end{itemize}
Relativo a ratos: \textunderscore propagação murina\textunderscore .
\section{Muriosulfato}
\begin{itemize}
\item {Grp. gram.:m.}
\end{itemize}
\begin{itemize}
\item {Utilização:Chím.}
\end{itemize}
Sal, produzido pela dissolução do estanho no ácido sulfúrico e no ácido chlorhýdrico.
\section{Muriosulfúrico}
\begin{itemize}
\item {Grp. gram.:adj.}
\end{itemize}
Dizia-se da solução do estanho no ácido sulfúrico e no ácido muriático, a qual serve para tingir de escarlate.
\section{Muriti}
\begin{itemize}
\item {Grp. gram.:m.}
\end{itemize}
Gênero de plantas brasileiras, (\textunderscore mauritia\textunderscore ); buriti.
\section{Muritim}
\begin{itemize}
\item {Grp. gram.:m.}
\end{itemize}
\begin{itemize}
\item {Utilização:Bras}
\end{itemize}
O mesmo que \textunderscore muriti\textunderscore .
\section{Muritinzal}
\begin{itemize}
\item {Grp. gram.:m.}
\end{itemize}
\begin{itemize}
\item {Utilização:Bras}
\end{itemize}
O mesmo que \textunderscore buritizal\textunderscore .
\section{Murmulhante}
\begin{itemize}
\item {Grp. gram.:adj.}
\end{itemize}
\begin{itemize}
\item {Utilização:bras}
\end{itemize}
\begin{itemize}
\item {Utilização:Neol.}
\end{itemize}
Que murmulha.
\section{Murmulhar}
\begin{itemize}
\item {Grp. gram.:v. t.}
\end{itemize}
\begin{itemize}
\item {Utilização:bras}
\end{itemize}
\begin{itemize}
\item {Utilização:Neol.}
\end{itemize}
\begin{itemize}
\item {Proveniência:(De \textunderscore murmulho\textunderscore )}
\end{itemize}
Ramalhar (a árvore).
Rumorejar.
\section{Murmulho}
\begin{itemize}
\item {Grp. gram.:m.}
\end{itemize}
Murmúrio das ondas.
O ramalhar das árvores.
(Corr. de \textunderscore murmúrio\textunderscore )
\section{Múrmur}
\begin{itemize}
\item {Grp. gram.:m.}
\end{itemize}
(V.múrmure)
\section{Murmuração}
\begin{itemize}
\item {Grp. gram.:f.}
\end{itemize}
\begin{itemize}
\item {Proveniência:(Lat. \textunderscore murmuratio\textunderscore )}
\end{itemize}
Acto de murmurar; maledicência.
\section{Murmurador}
\begin{itemize}
\item {Grp. gram.:adj.}
\end{itemize}
\begin{itemize}
\item {Grp. gram.:M.}
\end{itemize}
Que produz murmúrio; que murmura.
Que diz mal do próximo, que diffama, que tem má língua.
Aquelle que é diffamador ou maldizente.
\section{Murmurante}
\begin{itemize}
\item {Grp. gram.:adj.}
\end{itemize}
\begin{itemize}
\item {Proveniência:(Lat. \textunderscore murmurans\textunderscore )}
\end{itemize}
Que produz murmúrio; que murmura.
\section{Murmurar}
\begin{itemize}
\item {Grp. gram.:v. t.}
\end{itemize}
\begin{itemize}
\item {Grp. gram.:V. i.}
\end{itemize}
\begin{itemize}
\item {Proveniência:(Lat. \textunderscore murmurare\textunderscore )}
\end{itemize}
Emittir (som leve, froixo).
Dizer em voz baixa; segredar.
Produzir murmúrio ou sussurro.
Queixar-se.
Falar mal de alguém ou de alguma coisa.
Conversar, diffamando ou desacreditando.
\section{Murmurativo}
\begin{itemize}
\item {Grp. gram.:adj.}
\end{itemize}
\begin{itemize}
\item {Proveniência:(De \textunderscore murmurar\textunderscore )}
\end{itemize}
Murmurante; que envolve murmuração.
\section{Múrmure}
\begin{itemize}
\item {Grp. gram.:m.}
\end{itemize}
\begin{itemize}
\item {Utilização:P. us.}
\end{itemize}
\begin{itemize}
\item {Proveniência:(Lat. \textunderscore murmur\textunderscore )}
\end{itemize}
O mesmo que \textunderscore murmúrio\textunderscore .
Ruido das ondas ou água corrente. Cf. Pant. de Aveiro, \textunderscore Itiner.\textunderscore , 31, (3.^a ed.).
\section{Murmuré}
\begin{itemize}
\item {Grp. gram.:m.}
\end{itemize}
Instrumento dos Índios do Brasil, feito de ossos de defunto.
\section{Murmurejar}
\begin{itemize}
\item {Grp. gram.:v. i.}
\end{itemize}
\begin{itemize}
\item {Proveniência:(De \textunderscore múrmuro\textunderscore )}
\end{itemize}
Produzir murmúrio; rumorejar. Cf. Arn. Gama, \textunderscore Segr. do Abb.\textunderscore , 325.
\section{Murmurejo}
\begin{itemize}
\item {Grp. gram.:m.}
\end{itemize}
Acto de murmurejar.
\section{Murmurinho}
\begin{itemize}
\item {Grp. gram.:m.}
\end{itemize}
\begin{itemize}
\item {Proveniência:(Do lat. \textunderscore murmurillum\textunderscore )}
\end{itemize}
Sussurro de vozes simultâneas.
Ruído brando das águas, das fôlhas, etc.
Som confuso, murmúrio.
\section{Murmúrio}
\begin{itemize}
\item {Grp. gram.:m.}
\end{itemize}
\begin{itemize}
\item {Proveniência:(Do lat. \textunderscore murmurium\textunderscore )}
\end{itemize}
Ruído de água corrente, das ondas do mar, das fôlhas agitadas pela aragem, etc.
Som de muitas vozes juntas.
Palavras, pronunciadas em voz baixa; murmuração.
\section{Murmúro}
\begin{itemize}
\item {Grp. gram.:m.}
\end{itemize}
(V.murmúrio)
\section{Múrmuro}
\begin{itemize}
\item {Grp. gram.:adj.}
\end{itemize}
\begin{itemize}
\item {Utilização:Poét.}
\end{itemize}
O mesmo que \textunderscore murmurante\textunderscore .
\section{Murmuroso}
\begin{itemize}
\item {Grp. gram.:adj.}
\end{itemize}
\begin{itemize}
\item {Proveniência:(Lat. \textunderscore murmuriosus\textunderscore )}
\end{itemize}
Que murmura muito; que faz ruído prolongado: \textunderscore as árvores murmurosas\textunderscore .
Múrmuro.
\section{Murnau}
\begin{itemize}
\item {Grp. gram.:m.}
\end{itemize}
Planta medicinal da Guiana inglesa.
\section{Muro}
\begin{itemize}
\item {Grp. gram.:m.}
\end{itemize}
\begin{itemize}
\item {Utilização:Fig.}
\end{itemize}
\begin{itemize}
\item {Utilização:Bras}
\end{itemize}
\begin{itemize}
\item {Proveniência:(Lat. \textunderscore murus\textunderscore )}
\end{itemize}
Construcção de pedra ou de tejolos, etc., própria para vedar qualquer terreno ou recinto, ou para fortificar um lugar, uma praça, etc.
Construcção análoga, para separar terrenos ou recintos.
Qualquer construcção para defesa, separação ou resguardo.
Defesa.
O mesmo que \textunderscore murada\textunderscore .
Lugar cerrado, para guardar colmeias.
\section{Muro}
\begin{itemize}
\item {Grp. gram.:m.}
\end{itemize}
\begin{itemize}
\item {Utilização:Prov.}
\end{itemize}
\begin{itemize}
\item {Utilização:trasm.}
\end{itemize}
O mesmo que \textunderscore rato\textunderscore ^1.
(Cp. lat. \textunderscore mus\textunderscore , \textunderscore muris\textunderscore )
\section{Murquir}
\begin{itemize}
\item {Grp. gram.:v. i.}
\end{itemize}
\begin{itemize}
\item {Utilização:Prov.}
\end{itemize}
\begin{itemize}
\item {Utilização:trasm.}
\end{itemize}
O mesmo que \textunderscore munquir\textunderscore .
\section{Murra}
\begin{itemize}
\item {Grp. gram.:f.}
\end{itemize}
Mancha, que o fogo produz na pelle, quando êlle se aproxima muito do corpo.
(Cp. \textunderscore mura\textunderscore ^3)
\section{Murra}
\begin{itemize}
\item {Grp. gram.:f.}
\end{itemize}
\begin{itemize}
\item {Proveniência:(Lat. \textunderscore murrha\textunderscore )}
\end{itemize}
Substância mineral, de que se faziam copos para beber.
\section{Murraça}
\begin{itemize}
\item {Grp. gram.:f.}
\end{itemize}
\begin{itemize}
\item {Utilização:Pop.}
\end{itemize}
O mesmo que \textunderscore murro\textunderscore .
\section{Murraco}
\begin{itemize}
\item {Grp. gram.:m.}
\end{itemize}
\begin{itemize}
\item {Utilização:Prov.}
\end{itemize}
\begin{itemize}
\item {Utilização:minh.}
\end{itemize}
Casca de vidoeiro, enrolada e sêca.
Brandão de vidoeiro.
(Cp. \textunderscore morrão\textunderscore )
\section{Murrha}
\begin{itemize}
\item {Grp. gram.:f.}
\end{itemize}
\begin{itemize}
\item {Proveniência:(Lat. \textunderscore murrha\textunderscore )}
\end{itemize}
Substância mineral, de que se faziam copos para beber.
\section{Murrhino}
\begin{itemize}
\item {Grp. gram.:adj.}
\end{itemize}
\begin{itemize}
\item {Proveniência:(Lat. \textunderscore murrhinus\textunderscore )}
\end{itemize}
Feito de murrha, ou relativo a murrha.
\section{Murrino}
\begin{itemize}
\item {Grp. gram.:adj.}
\end{itemize}
\begin{itemize}
\item {Proveniência:(Lat. \textunderscore murrhinus\textunderscore )}
\end{itemize}
Feito de murra, ou relativo a murra.
\section{Murro}
\begin{itemize}
\item {Grp. gram.:m.}
\end{itemize}
Pancada, dada com a mão fechada; sôco.
\section{Murrondo}
\begin{itemize}
\item {Grp. gram.:m.}
\end{itemize}
Árvore de Moçambique.
\section{Murta}
\begin{itemize}
\item {Grp. gram.:f.}
\end{itemize}
\begin{itemize}
\item {Proveniência:(Lat. \textunderscore murta\textunderscore )}
\end{itemize}
Gênero de plantas, que serve de typo ás myrtáceas.
\section{Murtal}
\begin{itemize}
\item {Grp. gram.:m.}
\end{itemize}
Terreno, onde crescem murtas.
\section{Murtefuge}
\begin{itemize}
\item {Grp. gram.:m.}
\end{itemize}
Gênero de peixes acanthopterýgios, (\textunderscore blennius ruber\textunderscore ).
\section{Murteira}
\begin{itemize}
\item {Grp. gram.:f.}
\end{itemize}
\begin{itemize}
\item {Proveniência:(Do b. lat. \textunderscore murtaria\textunderscore )}
\end{itemize}
O mesmo que \textunderscore murta\textunderscore .
\section{Murteiro}
\begin{itemize}
\item {Grp. gram.:m.}
\end{itemize}
\begin{itemize}
\item {Utilização:Des.}
\end{itemize}
O mesmo que \textunderscore murteira\textunderscore , Cf. Filinto, X, 141.
Variedade de uva.
\section{Murtilla}
\begin{itemize}
\item {Grp. gram.:f.}
\end{itemize}
\begin{itemize}
\item {Proveniência:(De \textunderscore murta\textunderscore )}
\end{itemize}
Árvore fructífera do Rio-Grande-do-Sul.
\section{Murtinha}
\begin{itemize}
\item {Grp. gram.:f.}
\end{itemize}
Casta de uva de Cascaes.
(Cp. \textunderscore murteiro\textunderscore )
\section{Murtinheira}
\begin{itemize}
\item {Grp. gram.:f.}
\end{itemize}
Planta, que dá murtinhos.
O mesmo que \textunderscore murta\textunderscore .
\section{Murtinho}
\begin{itemize}
\item {Grp. gram.:m.}
\end{itemize}
Baga de murta.
Gênero de plantas myrtáceas do Brasil.
\section{Murtoseira}
\begin{itemize}
\item {Grp. gram.:f.}
\end{itemize}
\begin{itemize}
\item {Proveniência:(De \textunderscore Murtosa\textunderscore , n. p.)}
\end{itemize}
Embarcação da ria de Aveiro.
\section{Muru}
\begin{itemize}
\item {Grp. gram.:m.}
\end{itemize}
Planta canácea.
\section{Muruca}
\begin{itemize}
\item {Grp. gram.:f.}
\end{itemize}
Árvore de Angola.
\section{Muruci}
\begin{itemize}
\item {Grp. gram.:m.}
\end{itemize}
O mesmo que \textunderscore murici\textunderscore .
\section{Murocoça}
\begin{itemize}
\item {Grp. gram.:f.}
\end{itemize}
\begin{itemize}
\item {Utilização:Bras. do N}
\end{itemize}
Mosquito, espécie de \textunderscore carapaná\textunderscore .
\section{Murucu}
\begin{itemize}
\item {Grp. gram.:m.}
\end{itemize}
\begin{itemize}
\item {Utilização:Bras}
\end{itemize}
Espécie de lança de pau vermelho, com a ponta de outra madeira e ervada.--É usada pelos Muras, como arma de guerra.
\section{Muruçuca}
\begin{itemize}
\item {Grp. gram.:f.}
\end{itemize}
Árvore silvestre do Brasil.
\section{Murucucu}
\begin{itemize}
\item {Grp. gram.:m.}
\end{itemize}
Árvore do norte do Brasil, muito applicada em construcções.
\section{Murucujá}
\begin{itemize}
\item {Grp. gram.:m.}
\end{itemize}
\begin{itemize}
\item {Utilização:Bras}
\end{itemize}
\begin{itemize}
\item {Utilização:ant.}
\end{itemize}
O mesmo que \textunderscore maracujá\textunderscore .
\section{Murucututu}
\begin{itemize}
\item {Grp. gram.:m.}
\end{itemize}
Ave nocturna da região do Amazonas.
\section{Murugem}
\begin{itemize}
\item {Grp. gram.:f.}
\end{itemize}
\begin{itemize}
\item {Proveniência:(Do lat. \textunderscore mus\textunderscore , \textunderscore muris\textunderscore )}
\end{itemize}
Planta borragínea, (\textunderscore myosotis intermedia\textunderscore ).
\section{Murumo}
\begin{itemize}
\item {Grp. gram.:m.}
\end{itemize}
Palmeira africana, de que se extrai um líquido vinoso e uma substância açucarada.
\section{Murumuru}
\begin{itemize}
\item {Grp. gram.:m.}
\end{itemize}
Espécie de palmeira do norte do Brasil.
\section{Murumuxaua}
\begin{itemize}
\item {Grp. gram.:m.}
\end{itemize}
\begin{itemize}
\item {Utilização:Bras}
\end{itemize}
O mesmo que \textunderscore tuxaua\textunderscore .
\section{Murundu}
\begin{itemize}
\item {Grp. gram.:m.}
\end{itemize}
\begin{itemize}
\item {Utilização:Bras. do Rio}
\end{itemize}
Montão de coisas.
(Do quimbundo \textunderscore mulundu\textunderscore )
\section{Murungu}
\begin{itemize}
\item {Grp. gram.:m.}
\end{itemize}
Árvore leguminosa do Brasil.
\section{Murupamiri}
\begin{itemize}
\item {Grp. gram.:m.}
\end{itemize}
Árvore medicinal da região do Amazonas.
\section{Murupaúba}
\begin{itemize}
\item {Grp. gram.:f.}
\end{itemize}
Árvore silvestre do Brasil, provavelmente o mesmo que \textunderscore mulungu\textunderscore .
\section{Murupi}
\begin{itemize}
\item {Grp. gram.:m.}
\end{itemize}
Espécie de pimenta do Brasil.
\section{Mururé}
\begin{itemize}
\item {Grp. gram.:m.}
\end{itemize}
\begin{itemize}
\item {Utilização:Bras. do N}
\end{itemize}
\begin{itemize}
\item {Utilização:Bras}
\end{itemize}
Planta nympheácea do valle do Amazonas, (\textunderscore nymphaea alba\textunderscore ).
Ilhota fluctuante, que acompanha a corrente dos grandes rios do norte do Brasil.
\section{Mururu}
\begin{itemize}
\item {Grp. gram.:m.}
\end{itemize}
Planta urticácea do Brasil.
\section{Mururu}
\begin{itemize}
\item {Grp. gram.:m.}
\end{itemize}
\begin{itemize}
\item {Utilização:Bras. do N}
\end{itemize}
Achaque.
Moléstia intermittente.
\section{Muruti}
\begin{itemize}
\item {Grp. gram.:m.}
\end{itemize}
Fruto do murutijeiro.
\section{Murutijeiro}
\begin{itemize}
\item {Grp. gram.:m.}
\end{itemize}
Palmeira silvestre do Brasil.
\section{Muruxaba}
\begin{itemize}
\item {Grp. gram.:f.}
\end{itemize}
\begin{itemize}
\item {Utilização:Bras}
\end{itemize}
Mulata, de mau comportamento.
\section{Muruxaua}
\begin{itemize}
\item {Grp. gram.:m.}
\end{itemize}
\begin{itemize}
\item {Utilização:Bras}
\end{itemize}
(Contr. de \textunderscore murumuxaua\textunderscore )
\section{Murviedro}
\begin{itemize}
\item {Grp. gram.:m.}
\end{itemize}
Variedade de videira brasileira.
\section{Murzá}
\begin{itemize}
\item {Grp. gram.:m.}
\end{itemize}
Nome, com que entre os Turcos se designam as pessôas nobres.
(Cp. pers. \textunderscore mirzá\textunderscore )
\section{Murzela}
\begin{itemize}
\item {Grp. gram.:f.}
\end{itemize}
Planta euforbiácea, também conhecida por \textunderscore goéla de pato\textunderscore .
(Da mesma or. que \textunderscore morzello\textunderscore ?)
\section{Murzella}
\begin{itemize}
\item {Grp. gram.:f.}
\end{itemize}
Planta euphorbiácea, também conhecida por \textunderscore goéla de pato\textunderscore .
(Da mesma or. que \textunderscore morzello\textunderscore ?)
\section{Musa}
\begin{itemize}
\item {Grp. gram.:f.}
\end{itemize}
\begin{itemize}
\item {Proveniência:(Lat. \textunderscore musa\textunderscore )}
\end{itemize}
Cada uma das nove deusas, que presidiam ás artes liberaes.
Divindade, que se suppunha inspirar a poesia.
Tudo que póde inspirar um poeta.
A poesia.
Inspiração poética.
\section{Musa}
\begin{itemize}
\item {Grp. gram.:f.}
\end{itemize}
Espécie de bananeira asiática.
O mesmo que \textunderscore banana\textunderscore . Cf. Pant. de Aveiro, \textunderscore Itinerar.\textunderscore , 32 v.^o, (2.^a ed.).
(Talvez de \textunderscore Musa\textunderscore , n. p. de um médico romano)
\section{Musaça}
\begin{itemize}
\item {Grp. gram.:f.}
\end{itemize}
Árvore araliácea de Angola.
\section{Musáceas}
\begin{itemize}
\item {Grp. gram.:f. pl.}
\end{itemize}
\begin{itemize}
\item {Proveniência:(De \textunderscore musáceo\textunderscore )}
\end{itemize}
Família de plantas, que têm por typo a bananeira.
\section{Musáceo}
\begin{itemize}
\item {Grp. gram.:adj.}
\end{itemize}
\begin{itemize}
\item {Proveniência:(De \textunderscore musa\textunderscore ^2)}
\end{itemize}
Relativo ou semelhante á bananeira.
\section{Musal}
\begin{itemize}
\item {Grp. gram.:adj.}
\end{itemize}
\begin{itemize}
\item {Proveniência:(De \textunderscore musa\textunderscore ^1)}
\end{itemize}
Relativo a musas.
\section{Musalengue}
\begin{itemize}
\item {Grp. gram.:m.}
\end{itemize}
Árvore angolense de Cazengo.
\section{Musambisambi}
\begin{itemize}
\item {Grp. gram.:m.}
\end{itemize}
Árvore do Congo.
\section{Musanda}
\begin{itemize}
\item {Grp. gram.:f.}
\end{itemize}
Árvore do Congo.
\section{Musango}
\begin{itemize}
\item {Grp. gram.:m.}
\end{itemize}
Pássaro conirostro da África.
\section{Musaranho}
\begin{itemize}
\item {Grp. gram.:m.}
\end{itemize}
\begin{itemize}
\item {Proveniência:(Lat. \textunderscore musaraneus\textunderscore )}
\end{itemize}
Gênero de mammíferos, que se alimentam de insectos, e a que pertence o musaranho vulgar, (\textunderscore sorex araneus\textunderscore , Lin.).
\section{Musaria}
\begin{itemize}
\item {Grp. gram.:f.}
\end{itemize}
\begin{itemize}
\item {Utilização:Ant.}
\end{itemize}
Tudo que diz respeito a bens de alma e anniversários religiosos.
\section{Musassa}
\begin{itemize}
\item {Grp. gram.:f.}
\end{itemize}
Árvore araliácea de Angola.
\section{Muscadínea}
\begin{itemize}
\item {Grp. gram.:f.}
\end{itemize}
\begin{itemize}
\item {Proveniência:(Do b. lat. \textunderscore muschatus\textunderscore )}
\end{itemize}
Espécie de videira americana, denominada em Botânica \textunderscore vitis rotundifolia\textunderscore ; uma das duas grandes secções, em que Planchon dividiu as videiras.
\section{Muscardina}
\begin{itemize}
\item {Grp. gram.:f.}
\end{itemize}
\begin{itemize}
\item {Proveniência:(Do b. lat. \textunderscore muschatus\textunderscore )}
\end{itemize}
Doença contagiosa dos bichos da seda.
\section{Muscardínico}
\begin{itemize}
\item {Grp. gram.:adj.}
\end{itemize}
Relativo á muscardina.
Doente de muscardina.
\section{Muscari}
\begin{itemize}
\item {Grp. gram.:m.}
\end{itemize}
\begin{itemize}
\item {Proveniência:(Do gr. \textunderscore moskhos\textunderscore )}
\end{itemize}
Gênero de plantas liliáceas, de que há duas espécies.
Jacinto bravo.
\section{Muscarina}
\begin{itemize}
\item {Grp. gram.:f.}
\end{itemize}
Alcalóide, extrahido de uma espécie de agárico e que tem a propriedade de contrahir a pupilla ocular.
\section{Muscícola}
\begin{itemize}
\item {Grp. gram.:adj.}
\end{itemize}
\begin{itemize}
\item {Proveniência:(Do lat. \textunderscore museus\textunderscore  + \textunderscore colere\textunderscore )}
\end{itemize}
Que vive ou vegeta nos musgos.
\section{Muscíneas}
\begin{itemize}
\item {Grp. gram.:f. pl.}
\end{itemize}
\begin{itemize}
\item {Proveniência:(De \textunderscore muscíneo\textunderscore )}
\end{itemize}
Divisão de plantas cryptogâmicas, que comprehende os musgos e as hepáticas.
\section{Muscíneo}
\begin{itemize}
\item {Grp. gram.:adj.}
\end{itemize}
\begin{itemize}
\item {Proveniência:(Do lat. \textunderscore muscus\textunderscore )}
\end{itemize}
Relativo ou semelhante aos musgos.
\section{Muscívoro}
\begin{itemize}
\item {Grp. gram.:adj.}
\end{itemize}
\begin{itemize}
\item {Proveniência:(Do lat. \textunderscore musca\textunderscore  + \textunderscore vorare\textunderscore )}
\end{itemize}
Que se alimenta de môscas.
\section{Muscologia}
\begin{itemize}
\item {Grp. gram.:f.}
\end{itemize}
\begin{itemize}
\item {Proveniência:(Do lat. \textunderscore muscus\textunderscore  + gr. \textunderscore logos\textunderscore )}
\end{itemize}
Tratado dos musgos.
\section{Muscoso}
\begin{itemize}
\item {Grp. gram.:adj.}
\end{itemize}
O mesmo que \textunderscore musgoso\textunderscore .
\section{Musculação}
\begin{itemize}
\item {Grp. gram.:f.}
\end{itemize}
Exercício dos músculos.
\section{Musculado}
\begin{itemize}
\item {Grp. gram.:adj.}
\end{itemize}
Que tem músculos.
Cujos músculos se distinguem nitidamente, (em obras de pintura).
\section{Muscular}
\begin{itemize}
\item {Grp. gram.:adj.}
\end{itemize}
Relativo aos músculos.
Próprio dos músculos: \textunderscore fôrça muscular\textunderscore .
\section{Muscularmente}
\begin{itemize}
\item {Grp. gram.:adj.}
\end{itemize}
\begin{itemize}
\item {Proveniência:(De \textunderscore muscular\textunderscore )}
\end{itemize}
Por meio dos músculos.
Sob o ponto de vista muscular; quanto aos músculos.
\section{Musculatura}
\begin{itemize}
\item {Grp. gram.:f.}
\end{itemize}
\begin{itemize}
\item {Proveniência:(De \textunderscore muscular\textunderscore )}
\end{itemize}
Conjunto dos músculos do corpo humano.
Vigor dos músculos.
Maneira de representar os músculos, em pintura.
\section{Musculina}
\begin{itemize}
\item {Grp. gram.:f.}
\end{itemize}
\begin{itemize}
\item {Utilização:Chím.}
\end{itemize}
\begin{itemize}
\item {Proveniência:(De \textunderscore músculo\textunderscore )}
\end{itemize}
Preparação de carne crua de vaca, sem gordura.
Substância, que se encontra unicamente no tecido muscular.
\section{Músculo}
\begin{itemize}
\item {Grp. gram.:m.}
\end{itemize}
\begin{itemize}
\item {Utilização:Anat.}
\end{itemize}
\begin{itemize}
\item {Proveniência:(Lat. \textunderscore musculus\textunderscore )}
\end{itemize}
Órgão carnudo, formado pela reunião de muitas fibras, e que serve para operar movimentos, sob a influência da vontade ou de uma excitação orgânica ou mecânica.
Antiga máquina de guerra, para proteger os sitiantes.
\section{Musculodérmico}
\begin{itemize}
\item {Grp. gram.:adj.}
\end{itemize}
\begin{itemize}
\item {Proveniência:(De \textunderscore musculoso\textunderscore  + \textunderscore dérmico\textunderscore )}
\end{itemize}
Relativo á derme e ao systema muscular subjacente.
\section{Musculosidade}
\begin{itemize}
\item {Grp. gram.:f.}
\end{itemize}
Qualidade de musculoso; musculatura.
\section{Musculoso}
\begin{itemize}
\item {Grp. gram.:adj.}
\end{itemize}
\begin{itemize}
\item {Utilização:Ext.}
\end{itemize}
\begin{itemize}
\item {Proveniência:(Lat. \textunderscore musculosus\textunderscore )}
\end{itemize}
Musculado; que tem músculos fortes.
Que tem a natureza dos músculos.
Vigoroso.
\section{Musendesende}
\begin{itemize}
\item {Grp. gram.:m.}
\end{itemize}
Árvore do Congo.
\section{Musequere}
\begin{itemize}
\item {Grp. gram.:m.}
\end{itemize}
Árvore fructífera de Moçambique.
\section{Museu}
\begin{itemize}
\item {Grp. gram.:m.}
\end{itemize}
\begin{itemize}
\item {Utilização:Ext.}
\end{itemize}
\begin{itemize}
\item {Utilização:Fig.}
\end{itemize}
\begin{itemize}
\item {Proveniência:(Do lat. \textunderscore museum\textunderscore )}
\end{itemize}
Templo das musas, (accepção p. us.).
Lugar ou edifício, em que se estudam artes, sciências, etc.
Lugar, destinado não só a estudo, mas principalmente á reunião dos monumentos de bellas artes e sciências, dos objectos antigos, etc.

Reunião de coisas várias, variedade; miscellânea.
\section{Musgar}
\begin{itemize}
\item {Grp. gram.:v. t.}
\end{itemize}
\begin{itemize}
\item {Utilização:Prov.}
\end{itemize}
\begin{itemize}
\item {Utilização:alent.}
\end{itemize}
Queimar, chamuscar, o pêlo de (porco morto).
(Cp. \textunderscore chamuscar\textunderscore )
\section{Musgo}
\begin{itemize}
\item {Grp. gram.:m.}
\end{itemize}
\begin{itemize}
\item {Proveniência:(Lat. \textunderscore muscus\textunderscore )}
\end{itemize}
Gênero de plantas cryptogâmicas, annuaes ou vivazes.
\section{Musgo}
\begin{itemize}
\item {Grp. gram.:m.}
\end{itemize}
\begin{itemize}
\item {Utilização:Ant.}
\end{itemize}
O mesmo que \textunderscore músculo\textunderscore .
\section{Musgo}
\begin{itemize}
\item {Grp. gram.:m.}
\end{itemize}
\begin{itemize}
\item {Utilização:Ant.}
\end{itemize}
Talvez o mesmo que calção:«\textunderscore ...um daquelles soldados veteranos com pelote pero joelho, uns musgos cortados...\textunderscore »Couto, \textunderscore Déc.\textunderscore 
\section{Musgo}
\begin{itemize}
\item {Grp. gram.:adj.}
\end{itemize}
Diz-se dos animaes, especialmente das ovelhas e carneiros, que têm as orelhas muito pequenas: \textunderscore aquella ovelha é musga\textunderscore .
(Cp. lat. \textunderscore mus\textunderscore  e \textunderscore muricus\textunderscore )
\section{Musgoso}
\begin{itemize}
\item {Grp. gram.:adj.}
\end{itemize}
\begin{itemize}
\item {Proveniência:(Do lat. \textunderscore muscosus\textunderscore )}
\end{itemize}
Que produz musgo.
Coberto de musgo: \textunderscore parede musgosa\textunderscore .
Semelhante ao musgo.
\section{Musguenta}
\begin{itemize}
\item {Grp. gram.:f.  e  adj.}
\end{itemize}
\begin{itemize}
\item {Proveniência:(De \textunderscore musguento\textunderscore )}
\end{itemize}
Variedade de uva preta.
\section{Musguento}
\begin{itemize}
\item {Grp. gram.:adj.}
\end{itemize}
O mesmo que \textunderscore musgoso\textunderscore .
\section{Música}
\begin{itemize}
\item {Grp. gram.:f.}
\end{itemize}
\begin{itemize}
\item {Utilização:Fig.}
\end{itemize}
\begin{itemize}
\item {Utilização:Prov.}
\end{itemize}
\begin{itemize}
\item {Utilização:trasm.}
\end{itemize}
\begin{itemize}
\item {Utilização:Chul.}
\end{itemize}
\begin{itemize}
\item {Proveniência:(Lat. \textunderscore musica\textunderscore )}
\end{itemize}
Resultado da combinação de sons ou arte de os combinar, para que produzam effeito agradável.
Producto desta arte.
Execução de uma composição musical, por meio da voz ou de instrumentos.
Conjunto ou corporação de músicos.
Philarmónica; orchestra.
Qualquer conjunto de sons.
Em sentido primitivo, tudo que dizia respeito ás musas, tudo que dava ideia de coisa agradável ou bem disposta.

O mesmo que \textunderscore dinheiro\textunderscore .
\section{Musical}
\begin{itemize}
\item {Grp. gram.:adj.}
\end{itemize}
Relativo á música.
\section{Musicalmente}
\begin{itemize}
\item {Grp. gram.:adv.}
\end{itemize}
De modo musical.
\section{Musicante}
\begin{itemize}
\item {Grp. gram.:adj.}
\end{itemize}
\begin{itemize}
\item {Utilização:P. us.}
\end{itemize}
Que toca ou canta música. Cf. Garrett, \textunderscore Frei Luís de S.\textunderscore , 222.
\section{Musicar}
\begin{itemize}
\item {Grp. gram.:v. i.}
\end{itemize}
\begin{itemize}
\item {Proveniência:(De \textunderscore música\textunderscore )}
\end{itemize}
Cantar, trautear; tocar instrumento músico.
\section{Musicata}
\begin{itemize}
\item {Grp. gram.:f.}
\end{itemize}
\begin{itemize}
\item {Utilização:Fam.}
\end{itemize}
\begin{itemize}
\item {Proveniência:(De \textunderscore música\textunderscore )}
\end{itemize}
Fanfarra; philarmónica.
Execução de uma peça musical.
\section{Musicista}
\begin{itemize}
\item {Grp. gram.:m.}
\end{itemize}
\begin{itemize}
\item {Utilização:bras}
\end{itemize}
\begin{itemize}
\item {Utilização:Neol.}
\end{itemize}
Apreciador ou amador de música. Cf. \textunderscore Jornal-do-Comm.\textunderscore , do Rio, de 26-XI-900; \textunderscore Jorn.-do-Recife\textunderscore , de 22-IX-900.
\section{Músico}
\begin{itemize}
\item {Grp. gram.:adj.}
\end{itemize}
\begin{itemize}
\item {Grp. gram.:M.}
\end{itemize}
\begin{itemize}
\item {Utilização:Escol.}
\end{itemize}
\begin{itemize}
\item {Proveniência:(Lat. \textunderscore musicus\textunderscore )}
\end{itemize}
O mesmo que musical:«\textunderscore ...viola mais música e suave\textunderscore ». Sousa, \textunderscore Vida do Arceb.\textunderscore , 188.
Agradável ao ouvido.
Aquelle que professa a arte musical, compondo peças, ou tocando, ou cantando.
Aquelle que faz parte de uma philarmónica, orchestra ou banda regimental.
Estudante não premiado.
\section{Musicografia}
\begin{itemize}
\item {Grp. gram.:f.}
\end{itemize}
Tratado á cêrca da música.
Arte de escrever música.
(Cp. \textunderscore musicógrafo\textunderscore )
\section{Musicógrafo}
\begin{itemize}
\item {Grp. gram.:m.}
\end{itemize}
\begin{itemize}
\item {Proveniência:(Do gr. \textunderscore musike\textunderscore  + \textunderscore graphein\textunderscore )}
\end{itemize}
Aquele que escreve á cêrca da arte musical.
Instrumento, para escrever música.
\section{Musicographia}
\begin{itemize}
\item {Grp. gram.:f.}
\end{itemize}
Tratado á cêrca da música.
Arte de escrever música.
(Cp. \textunderscore musicógrapho\textunderscore )
\section{Musicógrapho}
\begin{itemize}
\item {Grp. gram.:m.}
\end{itemize}
\begin{itemize}
\item {Proveniência:(Do gr. \textunderscore musike\textunderscore  + \textunderscore graphein\textunderscore )}
\end{itemize}
Aquelle que escreve á cêrca da arte musical.
Instrumento, para escrever música.
\section{Musicologia}
\begin{itemize}
\item {Grp. gram.:f.}
\end{itemize}
Arte da música.
(Cp. \textunderscore musicólogo\textunderscore )
\section{Musicólogo}
\begin{itemize}
\item {Grp. gram.:m.}
\end{itemize}
\begin{itemize}
\item {Proveniência:(Do gr. \textunderscore musike\textunderscore  + \textunderscore logos\textunderscore )}
\end{itemize}
Aquelle que discorre sôbre a arte musical; aquelle que literariamente se occupa de música.
\section{Musicomania}
\begin{itemize}
\item {Grp. gram.:f.}
\end{itemize}
\begin{itemize}
\item {Proveniência:(De \textunderscore música\textunderscore  + \textunderscore mania\textunderscore )}
\end{itemize}
Paixão pela música.
Loucura, caracterizada por uma excessiva paixão pela música.
\section{Musicomaníaco}
\begin{itemize}
\item {Grp. gram.:m.}
\end{itemize}
Aquelle que tem musicomania.
\section{Musicómano}
\begin{itemize}
\item {Grp. gram.:m.}
\end{itemize}
Aquelle que tem musicomania.
\section{Musicoterapia}
\begin{itemize}
\item {Grp. gram.:f.}
\end{itemize}
\begin{itemize}
\item {Proveniência:(Do gr. \textunderscore musike\textunderscore  + \textunderscore therapeia\textunderscore )}
\end{itemize}
Tratamento terapêutico por meio da música.
\section{Musicoterápico}
\begin{itemize}
\item {Grp. gram.:adj.}
\end{itemize}
Relativo á musicoterapia.
\section{Musicotherapia}
\begin{itemize}
\item {Grp. gram.:f.}
\end{itemize}
\begin{itemize}
\item {Proveniência:(Do gr. \textunderscore musike\textunderscore  + \textunderscore therapeia\textunderscore )}
\end{itemize}
Tratamento therapêutico por meio da música.
\section{Musicotherápico}
\begin{itemize}
\item {Grp. gram.:adj.}
\end{itemize}
Relativo á musicotherapia.
\section{Musiquear}
\begin{itemize}
\item {Grp. gram.:v. i.}
\end{itemize}
O mesmo que \textunderscore musicar\textunderscore .
\section{Musiqueta}
\begin{itemize}
\item {fónica:quê}
\end{itemize}
\begin{itemize}
\item {Grp. gram.:f.}
\end{itemize}
Pequeno trecho de música. Cf. \textunderscore Filodemo\textunderscore , IV, 2.
\section{Musiquia}
\begin{itemize}
\item {Grp. gram.:f.}
\end{itemize}
\begin{itemize}
\item {Utilização:Ant.}
\end{itemize}
O mesmo que \textunderscore musicata\textunderscore .
\section{Musiquim}
\begin{itemize}
\item {Grp. gram.:m.}
\end{itemize}
\begin{itemize}
\item {Utilização:Pop.}
\end{itemize}
Músico ordinário, pouco hábil.
\section{Muslemia}
\begin{itemize}
\item {Grp. gram.:f.}
\end{itemize}
\begin{itemize}
\item {Proveniência:(De \textunderscore muslemo\textunderscore )}
\end{itemize}
O mesmo que \textunderscore moirisma\textunderscore .
\section{Muslemo}
\begin{itemize}
\item {Grp. gram.:adj.}
\end{itemize}
\begin{itemize}
\item {Utilização:Fig.}
\end{itemize}
O mesmo que \textunderscore muslim\textunderscore . Cf. Viterbo, \textunderscore Elucid.\textunderscore , vb. \textunderscore muzlemo\textunderscore .
Rude, grosseiro; rústico; incivil.
(Cp. \textunderscore muslim\textunderscore )
\section{Muslim}
\begin{itemize}
\item {Grp. gram.:adj.}
\end{itemize}
O mesmo que \textunderscore muçulmano\textunderscore .
(Cp. \textunderscore moslém\textunderscore )
\section{Musoso}
\begin{itemize}
\item {Grp. gram.:m.}
\end{itemize}
Pequena árvore angolense, de casca medicinal.
\section{Mussache}
\begin{itemize}
\item {Grp. gram.:m.}
\end{itemize}
Arbusto africano, (\textunderscore thalamiflora\textunderscore , De-Cand.), de fôlhas semelhantes ás do damasqueiro.
\section{Mussala}
\begin{itemize}
\item {Grp. gram.:f.}
\end{itemize}
Árvore africana, da fam. das leguminosas.
\section{Mussale}
\begin{itemize}
\item {Grp. gram.:f.}
\end{itemize}
Peneira africana, em fórma de jarro.
\section{Mussalo}
\begin{itemize}
\item {Grp. gram.:m.}
\end{itemize}
Árvore intertropical, leguminosa.
O mesmo que \textunderscore mussala\textunderscore ?
\section{Mussamba}
\begin{itemize}
\item {Grp. gram.:f.}
\end{itemize}
Árvore africana, com grandes camadas de liber, impregnado de matéria còrante.
\section{Mussamba}
\begin{itemize}
\item {Grp. gram.:f.}
\end{itemize}
Instrumento, semelhante á puita, e usado entro os Indígenas de San-Thomé.
\section{Mussambé}
\begin{itemize}
\item {Grp. gram.:m.}
\end{itemize}
Planta capparidácea do Brasil.
\section{Mussambo}
\begin{itemize}
\item {Grp. gram.:m.}
\end{itemize}
Enfeite de metal, com que na Lunda se aperta a parte inferior das tranças dos cabellos.
\section{Mussanda}
\begin{itemize}
\item {Grp. gram.:f.}
\end{itemize}
Árvore africana, de fôlhas compostas e alternas, e flôres brancas em grandes espigas.
\section{Mussandala}
\begin{itemize}
\item {Grp. gram.:f.}
\end{itemize}
Planta herbácea africana, da fam. das papaveráceas, (\textunderscore argemona mexicana\textunderscore ).
\section{Mussande}
\begin{itemize}
\item {Grp. gram.:m.}
\end{itemize}
Trapo, com que os Lundeses resguardam as partes pudendas.
\section{Mussandeira-sangue}
\begin{itemize}
\item {Grp. gram.:f.}
\end{itemize}
Arbusto africano, herbáceo, da fam. das leguminosas.
\section{Mussangará}
\begin{itemize}
\item {Grp. gram.:m.}
\end{itemize}
Árvore de Moçambique.
\section{Mussanhi}
\begin{itemize}
\item {Grp. gram.:m.}
\end{itemize}
Árvore de Angola.
\section{Mussão}
\begin{itemize}
\item {Grp. gram.:m.}
\end{itemize}
Personagem que, junto de alguns sobas, na África Occidental, exerce funcções de ministro. Cf. Camillo, \textunderscore Noites de Insómn.\textunderscore , I, 69.
\section{Mussapo}
\begin{itemize}
\item {Grp. gram.:m.}
\end{itemize}
Pequena árvore africana, muito ramosa.
\section{Mussegueia}
\begin{itemize}
\item {Grp. gram.:f.}
\end{itemize}
Planta africana, da fam. das cucurbitáceas.
\section{Musselemano}
\begin{itemize}
\item {Grp. gram.:adj.}
\end{itemize}
(V.muçulmano). Cf. Latino, \textunderscore Elogios\textunderscore , 58 e 59.
\section{Musselina}
\begin{itemize}
\item {Grp. gram.:f.}
\end{itemize}
\begin{itemize}
\item {Utilização:Ext.}
\end{itemize}
Tecido leve e transparente, de algodão; cassa.
Estofo de lan ou seda, muito leve.
(Cp. cast. \textunderscore muselina\textunderscore )
\section{Mussemba}
\begin{itemize}
\item {Grp. gram.:f.}
\end{itemize}
Árvore intertropical, da fam. das leguminosas, (\textunderscore albizzia coriarea\textunderscore , Welw.?), de fôlhas glabras e flôres hermaphroditas.
\section{Mussesse}
\begin{itemize}
\item {Grp. gram.:m.}
\end{itemize}
Árvore africana, de amplas fôlhas decompostas.
\section{Mussitação}
\begin{itemize}
\item {Grp. gram.:f.}
\end{itemize}
\begin{itemize}
\item {Proveniência:(Lat. \textunderscore mussitatio\textunderscore )}
\end{itemize}
Movimento automático dos lábios, produzindo murmúrio ou som confuso.
\section{Mussitar}
\begin{itemize}
\item {Grp. gram.:v. i.}
\end{itemize}
\begin{itemize}
\item {Utilização:Ant.}
\end{itemize}
\begin{itemize}
\item {Proveniência:(Lat. \textunderscore mussitare\textunderscore )}
\end{itemize}
Falar em voz baixa.
Murmurar; cochichar.
\section{Mussó}
\begin{itemize}
\item {Grp. gram.:m.}
\end{itemize}
\begin{itemize}
\item {Utilização:T. da Índia port}
\end{itemize}
Pilão, para descascar arroz.
\section{Mussoco}
\begin{itemize}
\item {Grp. gram.:m.}
\end{itemize}
Tributo, pago ao governo português pelos colonos indígenas de Moçambique.
Tributo, que pagam aos seus régulos os pretos dos districtos centraes de Moçambique.
\section{Mussombe}
\begin{itemize}
\item {Grp. gram.:m.}
\end{itemize}
Árvore africana, muito copada, de fôlhas simples, lanceoladas, e flôres miúdas, polysépalas.
\section{Mussombo}
\begin{itemize}
\item {Grp. gram.:m.}
\end{itemize}
Designação vulgar, na Lunda, de uma robusta árvore, de fôlhas alternas, lustrosas, e frutos de pericarpo escarlate.
\section{Mussorongos}
\begin{itemize}
\item {Grp. gram.:m. pl.}
\end{itemize}
Congueses, que habitam ao lado esquerdo do Zaire.
\section{Mussu}
\begin{itemize}
\item {Grp. gram.:m.}
\end{itemize}
O mesmo que \textunderscore mussum\textunderscore .
\section{Mussuco}
\begin{itemize}
\item {Grp. gram.:m.}
\end{itemize}
Árvore de Angola.
\section{Mussuesso}
\begin{itemize}
\item {Grp. gram.:m.}
\end{itemize}
Ave africana, (\textunderscore passer diffusus\textunderscore , Smith).
\section{Mussulmano}
\textunderscore m.\textunderscore  e \textunderscore adj.\textunderscore  (e der.)
(V. \textunderscore muçulmano\textunderscore , que é melhor orthogr.)
\section{Mussum}
\begin{itemize}
\item {Grp. gram.:m.}
\end{itemize}
Espécie de enguia do Brasil.
\section{Mussumba}
\begin{itemize}
\item {Grp. gram.:f.}
\end{itemize}
Arbusto africano, esguio, de grandes fôlhas alternas e estipuladas, e flôres hermaphroditas.
\section{Mussununga}
\begin{itemize}
\item {Grp. gram.:f.}
\end{itemize}
\begin{itemize}
\item {Utilização:Bras}
\end{itemize}
Terreno arenoso e húmido.
\section{Mussunda}
\begin{itemize}
\item {Grp. gram.:f.}
\end{itemize}
Árvore da ilha de San-Thomé.
\section{Mussurana}
\begin{itemize}
\item {Grp. gram.:f.}
\end{itemize}
Corda, com que os Indígenas do Brasil atavam os prisioneiros.
\section{Mustelídeo}
\begin{itemize}
\item {Grp. gram.:adj.}
\end{itemize}
\begin{itemize}
\item {Grp. gram.:M. pl.}
\end{itemize}
\begin{itemize}
\item {Proveniência:(Do lat. \textunderscore mustela\textunderscore  + gr. \textunderscore eidos\textunderscore )}
\end{itemize}
Relativo ou semelhante á doninha.
Família de mammíferos carnívoros, que tem por typo a doninha.
\section{Mustelino}
\begin{itemize}
\item {Grp. gram.:adj.}
\end{itemize}
O mesmo que \textunderscore mustelídeo\textunderscore .
\section{Musteriano}
\begin{itemize}
\item {Grp. gram.:adj.}
\end{itemize}
\begin{itemize}
\item {Utilização:Geol.}
\end{itemize}
O mesmo que \textunderscore glaciário\textunderscore , segundo Mortillet, se bem que alguns o confundem com \textunderscore préglaciário\textunderscore .
\section{Mustímetro}
\begin{itemize}
\item {Grp. gram.:m.}
\end{itemize}
\begin{itemize}
\item {Proveniência:(T. hýbr., do lat. \textunderscore mustum\textunderscore  + gr. \textunderscore metron\textunderscore )}
\end{itemize}
O mesmo ou melhor que \textunderscore mostímetro\textunderscore .
\section{Musuar}
\begin{itemize}
\item {Grp. gram.:m.}
\end{itemize}
Nassa de arame, usada nos ríos do norte do país.
\section{Musuno}
\begin{itemize}
\item {Grp. gram.:m.}
\end{itemize}
Arbusto angolense, trepador, semelhante ás silvas do nosso país.
\section{Musurana}
\begin{itemize}
\item {Grp. gram.:f.}
\end{itemize}
O mesmo que \textunderscore mussurana\textunderscore . Cf. Gonç. Dias, \textunderscore Poesias\textunderscore , 24.
\section{Muta}
\begin{itemize}
\item {Grp. gram.:m.}
\end{itemize}
Árvore angolense, de cuja lenha, friccionada com madeira mais dura, os Gentios fazem lume.
\section{Mutá}
\begin{itemize}
\item {Grp. gram.:m.}
\end{itemize}
\begin{itemize}
\item {Utilização:Bras}
\end{itemize}
\begin{itemize}
\item {Proveniência:(T. tupi)}
\end{itemize}
Estrado, construido no mato, e no qual o caçador se colloca á espera da caça.
\section{Mutabeia}
\begin{itemize}
\item {Grp. gram.:f.}
\end{itemize}
Gênero de plantas ebenáceas.
\section{Mutabilidade}
\begin{itemize}
\item {Grp. gram.:f.}
\end{itemize}
\begin{itemize}
\item {Proveniência:(Lat. \textunderscore mutabilitas\textunderscore )}
\end{itemize}
Qualidade do que é mudável; instabilidade; volubilidade.
\section{Mutação}
\begin{itemize}
\item {Grp. gram.:f.}
\end{itemize}
\begin{itemize}
\item {Proveniência:(Lat. \textunderscore mutatio\textunderscore )}
\end{itemize}
Mudança, alteração e substituição.
Volubilidade, inconstância.
Mudança de scenário, nos theatros.
\section{Mutacismo}
\begin{itemize}
\item {Grp. gram.:m.}
\end{itemize}
Repetição abusiva da letra \textunderscore m\textunderscore  em muitas palavras da mesma phrase.
(Do nome da letra gr. \textunderscore mu\textunderscore )
\section{Mutala-mema}
\begin{itemize}
\item {Grp. gram.:f.}
\end{itemize}
(V.olha-a-água)
\section{Mutala-menha}
\begin{itemize}
\item {Grp. gram.:f.}
\end{itemize}
Formosa árvore angolense, (\textunderscore lonchocarpus sericeus\textunderscore ; segundo outros, \textunderscore milletia drastica\textunderscore ).
\section{Mutalara}
\begin{itemize}
\item {Grp. gram.:f.}
\end{itemize}
Arbusto fructífero de Moçambique.
\section{Mutali-cumbi}
\begin{itemize}
\item {Grp. gram.:m.}
\end{itemize}
Árvore de Angola.
\section{Mutamba}
\begin{itemize}
\item {Grp. gram.:f.}
\end{itemize}
Árvore byttneriácea do Brasil.
O mesmo que \textunderscore mucungo\textunderscore .
\section{Mutambo}
\begin{itemize}
\item {Grp. gram.:m.}
\end{itemize}
O mesmo que \textunderscore mutamba\textunderscore .
\section{Mutá-mutá}
\begin{itemize}
\item {Grp. gram.:f.}
\end{itemize}
Planta trepadeira do Brasil.
\section{Mutan}
\begin{itemize}
\item {Grp. gram.:m.}
\end{itemize}
\begin{itemize}
\item {Utilização:Bras. do N}
\end{itemize}
Palanque de caçador, que espera a caça no mato.
O mesmo que \textunderscore mutá\textunderscore .
\section{Mutança}
\begin{itemize}
\item {Grp. gram.:f.}
\end{itemize}
\begin{itemize}
\item {Utilização:Mús.}
\end{itemize}
Troca de nomes, que no solfejo antigo só se effectuava com seis notas, quando o canto excedia os limites de uma deducção, mudando de propriedade.
(Cp. lat. \textunderscore mutare\textunderscore )
\section{Mutanos}
\begin{itemize}
\item {Grp. gram.:m. pl.}
\end{itemize}
\begin{itemize}
\item {Utilização:Prov.}
\end{itemize}
Mólho de tojo ou de ramos de pinho.
(Cp. \textunderscore motano\textunderscore , que parece melhor orthogr.)
\section{Mutanta}
\begin{itemize}
\item {Grp. gram.:f.}
\end{itemize}
Robusta árvore africana, de tronco bojudo, fôlhas simples, e flôres amarelas.
\section{Mutatório}
\begin{itemize}
\item {Grp. gram.:adj.}
\end{itemize}
\begin{itemize}
\item {Proveniência:(Lat. \textunderscore mutatorius\textunderscore )}
\end{itemize}
Que muda.
Que serve para fazer mudança.
\section{Mutável}
\begin{itemize}
\item {Grp. gram.:adj.}
\end{itemize}
\begin{itemize}
\item {Proveniência:(Lat. \textunderscore mutabilis\textunderscore )}
\end{itemize}
O mesmo que \textunderscore mudável\textunderscore .
\section{Mutazilitas}
\begin{itemize}
\item {Grp. gram.:m. pl.}
\end{itemize}
Seita religiosa, que admitte uma mansão intermediária ao inferno e ao paraíso, mansão differente do purgatório christão.
\section{Mutelatete}
\begin{itemize}
\item {Grp. gram.:m.}
\end{itemize}
Árvore angolense.
\section{Mutele}
\begin{itemize}
\item {Grp. gram.:m.}
\end{itemize}
Árvore africana, elegante e copada, de fôlhas simples, e frutos globosos, dehiscentes.
\section{Mutelina}
\begin{itemize}
\item {Grp. gram.:f.}
\end{itemize}
\begin{itemize}
\item {Proveniência:(De \textunderscore Mutel\textunderscore , n. p.)}
\end{itemize}
Planta umbellífera, (\textunderscore meum mutellina\textunderscore , Gaertner).
\section{Mutenga}
\begin{itemize}
\item {Grp. gram.:f.}
\end{itemize}
Árvore de Angola.
\section{Mutenge}
\begin{itemize}
\item {Grp. gram.:m.}
\end{itemize}
Árvore angolense.
O mesmo que \textunderscore mutenga\textunderscore ?
\section{Mutepa}
\begin{itemize}
\item {Grp. gram.:f.}
\end{itemize}
\begin{itemize}
\item {Proveniência:(T. lund.)}
\end{itemize}
Árvore africana, de seiva leitosa, fôlhas cordiformes e frutos syncarpados.
\section{Mutete}
\begin{itemize}
\item {Grp. gram.:m.}
\end{itemize}
Árvore silvestre, intertropical, (\textunderscore haronga madagascariensis\textunderscore ?).
\section{Mutete}
\begin{itemize}
\item {Grp. gram.:m.}
\end{itemize}
Árvore africana, (\textunderscore pterocarpus erinaceus\textunderscore , Bak.).
\section{Mutia}
\begin{itemize}
\item {Grp. gram.:f.}
\end{itemize}
Formosa árvore angolense, de fôlhas pretas, e cujo fruto tem um caroço, de que se extrai bom azeite.
\section{Mutialare}
\begin{itemize}
\item {Grp. gram.:m.}
\end{itemize}
Árvore angolense.
\section{Mutiati}
\begin{itemize}
\item {Grp. gram.:m.}
\end{itemize}
Árvore de Angola, no Bumbo.
\section{Mutico}
\begin{itemize}
\item {Grp. gram.:m.}
\end{itemize}
Ave insectívora, que vive junto dos rios, em Angola.
\section{Mutil}
\begin{itemize}
\item {Grp. gram.:f. pl.}
\end{itemize}
Insectos hymenópteros.
\section{Mutilação}
\begin{itemize}
\item {Grp. gram.:f.}
\end{itemize}
\begin{itemize}
\item {Proveniência:(Lat. \textunderscore mutilatio\textunderscore )}
\end{itemize}
Acto ou effeito de mutilar.
\section{Mutilado}
\begin{itemize}
\item {Grp. gram.:m.}
\end{itemize}
\begin{itemize}
\item {Proveniência:(De \textunderscore mutilar\textunderscore )}
\end{itemize}
Aquelle a quem falta um membro.
\section{Mutilador}
\begin{itemize}
\item {Grp. gram.:m.  e  adj.}
\end{itemize}
\begin{itemize}
\item {Proveniência:(Lat. \textunderscore mutilator\textunderscore )}
\end{itemize}
O que mutila.
\section{Mutilão}
\begin{itemize}
\item {Grp. gram.:m.}
\end{itemize}
(?)«\textunderscore Nos mutilões das escadas acostumam trazer senhos falções...\textunderscore »Fern. Oliveira, \textunderscore Arte da Guerra do Mar\textunderscore , fol. 47, v.^o
\section{Mutilar}
\begin{itemize}
\item {Grp. gram.:v. t.}
\end{itemize}
\begin{itemize}
\item {Utilização:Fig.}
\end{itemize}
\begin{itemize}
\item {Proveniência:(Lat. \textunderscore mutilare\textunderscore )}
\end{itemize}
Privar de algum membro.
Cortar (um membro do corpo).
Cortar qualquer membro ou parte.
Desramar.
Truncar.
Destruir parte de.
Depreciar, amesquinhar.
Deturpar: \textunderscore mutilar um texto\textunderscore .
\section{Mútilo}
\begin{itemize}
\item {Grp. gram.:m.  e  adj.}
\end{itemize}
\begin{itemize}
\item {Proveniência:(Lat. \textunderscore mutilus\textunderscore )}
\end{itemize}
Mutilado.
\section{Mutirão}
\begin{itemize}
\item {Grp. gram.:m.}
\end{itemize}
\begin{itemize}
\item {Utilização:Bras}
\end{itemize}
O mesmo que \textunderscore mutirom\textunderscore .
\section{Mutirom}
\begin{itemize}
\item {Grp. gram.:m.}
\end{itemize}
\begin{itemize}
\item {Utilização:Bras}
\end{itemize}
O mesmo que \textunderscore muxirão\textunderscore .
\section{Mutirum}
\begin{itemize}
\item {Grp. gram.:m.}
\end{itemize}
\begin{itemize}
\item {Utilização:Bras. do N}
\end{itemize}
O mesmo que \textunderscore muxirão\textunderscore .
\section{Mutísia}
\begin{itemize}
\item {Grp. gram.:f.}
\end{itemize}
Planta trepadeira do Brasil.
\section{Mutismo}
\begin{itemize}
\item {Grp. gram.:m.}
\end{itemize}
\begin{itemize}
\item {Proveniência:(Do lat. \textunderscore mutus\textunderscore )}
\end{itemize}
Qualidade de mudo; mudez; taciturnidade.
\section{Mutôa}
\begin{itemize}
\item {Grp. gram.:f.}
\end{itemize}
Árvore de Moçambique.
\section{Mutóbue}
\begin{itemize}
\item {Grp. gram.:m.}
\end{itemize}
Árvore moçambicana, própria para almadias, tabuado, etc.
\section{Mutom}
\begin{itemize}
\item {Grp. gram.:m.}
\end{itemize}
Pássaro azul das regiões do Amazonas.
\section{Mutomboti}
\begin{itemize}
\item {Grp. gram.:m.}
\end{itemize}
Árvore de Angola.
\section{Mutom-utom}
\begin{itemize}
\item {Grp. gram.:m.}
\end{itemize}
Arbusto dos pântanos africanos, (\textunderscore dissolis grandiflora\textunderscore ).
\section{Mutona}
\begin{itemize}
\item {Grp. gram.:f.}
\end{itemize}
Árvore de Angola, de madeira forte e de muitas applicações.
\section{Mutonatona}
\begin{itemize}
\item {Grp. gram.:f.}
\end{itemize}
Árvore angolense.
\section{Mutongatonga}
\begin{itemize}
\item {Grp. gram.:f.}
\end{itemize}
Árvore de Moçambique.
O mesmo que \textunderscore mutonatona\textunderscore ?
\section{Mutongolo}
\begin{itemize}
\item {Grp. gram.:m.}
\end{itemize}
Árvore de Angola.
\section{Mutonto}
\begin{itemize}
\item {Grp. gram.:f.}
\end{itemize}
Árvore africana. Cf. Capello e Ivens, \textunderscore De Angola\textunderscore , I, 383.
\section{Mutontola}
\begin{itemize}
\item {Grp. gram.:f.}
\end{itemize}
Árvore angolense.
\section{Mutonuti}
\begin{itemize}
\item {Grp. gram.:m.}
\end{itemize}
Árvore de Angola.
\section{Mutopa}
\begin{itemize}
\item {Grp. gram.:f.}
\end{itemize}
Cachimbo africano, feito de uma cabaça de collo alto.
\section{Mutove}
\begin{itemize}
\item {Grp. gram.:m.}
\end{itemize}
Grande árvore fructífera de Moçambique, espécie de figueira.
\section{Mutra}
\begin{itemize}
\item {Grp. gram.:f.}
\end{itemize}
\begin{itemize}
\item {Utilização:Ant.}
\end{itemize}
O mesmo que \textunderscore sinete\textunderscore . Cf. \textunderscore Peregrinação\textunderscore , 36 e 177.
\section{Mutrar}
\begin{itemize}
\item {Grp. gram.:v. t.}
\end{itemize}
\begin{itemize}
\item {Utilização:Ant.}
\end{itemize}
Pôr mutra em. Cf. \textunderscore Peregrinação\textunderscore , LXXXVII.
\section{Mutuação}
\begin{itemize}
\item {Grp. gram.:f.}
\end{itemize}
\begin{itemize}
\item {Proveniência:(Lat. \textunderscore mutuatio\textunderscore )}
\end{itemize}
Acto ou effeito de mutuar.
Permutação, troca.
Acto de dar ou tomar emprestado.
Empréstimo.
\section{Mutuador}
\begin{itemize}
\item {Grp. gram.:m.}
\end{itemize}
Aquelle que mutua ou empresta. Cf. Castilho, \textunderscore Avarento\textunderscore , 95.
\section{Mutual}
\begin{itemize}
\item {Grp. gram.:adj.}
\end{itemize}
\begin{itemize}
\item {Utilização:Des.}
\end{itemize}
(V.mútuo)
\section{Mutualidade}
\begin{itemize}
\item {Grp. gram.:f.}
\end{itemize}
\begin{itemize}
\item {Proveniência:(De \textunderscore mutual\textunderscore )}
\end{itemize}
Estado do que é mútuo ou recíproco.
Permutação.
\section{Mutualista}
\begin{itemize}
\item {Grp. gram.:m.  e  f.}
\end{itemize}
\begin{itemize}
\item {Grp. gram.:Adj.}
\end{itemize}
\begin{itemize}
\item {Proveniência:(De \textunderscore mutual\textunderscore )}
\end{itemize}
Pessôa, que faz parte de uma companhia de seguros ou de soccorros mútuos.
Relativo a soccorros mútuos: \textunderscore o congresso mutualista de Marselha em 1886\textunderscore ...
\section{Miagro}
\begin{itemize}
\item {Grp. gram.:m.}
\end{itemize}
Planta crucífera.
\section{Mialgia}
\begin{itemize}
\item {Grp. gram.:f.}
\end{itemize}
\begin{itemize}
\item {Proveniência:(Do gr. \textunderscore mus\textunderscore  + \textunderscore algos\textunderscore )}
\end{itemize}
Dôr nos músculos.
\section{Micélio}
\begin{itemize}
\item {Grp. gram.:m.}
\end{itemize}
\begin{itemize}
\item {Proveniência:(Do gr. \textunderscore mukes\textunderscore , cogumelo)}
\end{itemize}
Parte filamentosa do talo do fungo.
Filamentos que, partindo das raízes do míldio, se insinuam entre as paredes das células das fôlhas da videira.
\section{Micênico}
\begin{itemize}
\item {Grp. gram.:adj.}
\end{itemize}
Relativo a Micenas: \textunderscore a arte micênica em Portugal\textunderscore .
\section{Micetographia}
\begin{itemize}
\item {Grp. gram.:f.}
\end{itemize}
\begin{itemize}
\item {Proveniência:(Do gr. \textunderscore mukes\textunderscore  + \textunderscore graphein\textunderscore )}
\end{itemize}
Descripção ou história dos cogumelos.
\section{Micetográphico}
\begin{itemize}
\item {Grp. gram.:adj.}
\end{itemize}
Relativo á micetographia.
\section{Micetologia}
\begin{itemize}
\item {Grp. gram.:f.}
\end{itemize}
O mesmo que \textunderscore micologia\textunderscore .
\section{Micetoma}
\begin{itemize}
\item {Grp. gram.:m.}
\end{itemize}
\begin{itemize}
\item {Utilização:Med.}
\end{itemize}
\begin{itemize}
\item {Proveniência:(Do gr. \textunderscore mukhes\textunderscore )}
\end{itemize}
Moléstia parasitária.
\section{Micetóphagos}
\begin{itemize}
\item {Grp. gram.:m. pl.}
\end{itemize}
\begin{itemize}
\item {Proveniência:(Do gr. \textunderscore mukes\textunderscore  + \textunderscore phagein\textunderscore )}
\end{itemize}
Gênero de insectos coleópteros.
\section{Micogenia}
\begin{itemize}
\item {Grp. gram.:f.}
\end{itemize}
\begin{itemize}
\item {Proveniência:(Do gr. \textunderscore mukes\textunderscore  + \textunderscore genos\textunderscore )}
\end{itemize}
Producção de mucedíneos.
\section{Micogênico}
\begin{itemize}
\item {Grp. gram.:adj.}
\end{itemize}
Relativo á micogenia.
Que produz mucedíneos.
\section{Micologia}
\begin{itemize}
\item {Grp. gram.:f.}
\end{itemize}
Tratado dos cogumelos.
(Cp. \textunderscore micólogo\textunderscore )
\section{Micologista}
\begin{itemize}
\item {Grp. gram.:m.  e  adj.}
\end{itemize}
\begin{itemize}
\item {Proveniência:(Do gr. \textunderscore mukes\textunderscore  + \textunderscore logos\textunderscore )}
\end{itemize}
Aquele que é versado em micologia ou trata desta ciência.
\section{Micólogo}
\begin{itemize}
\item {Grp. gram.:m.  e  adj.}
\end{itemize}
\begin{itemize}
\item {Proveniência:(Do gr. \textunderscore mukes\textunderscore  + \textunderscore logos\textunderscore )}
\end{itemize}
Aquele que é versado em micologia ou trata desta ciência.
\section{Micose}
\begin{itemize}
\item {Grp. gram.:f.}
\end{itemize}
\begin{itemize}
\item {Proveniência:(Do gr. \textunderscore mukes\textunderscore )}
\end{itemize}
Saliência esponjosa, em fórma de cogumelo.
Princípio açucarado da cravagem do centeio.
\section{Micótico}
\begin{itemize}
\item {Grp. gram.:adj.}
\end{itemize}
\begin{itemize}
\item {Proveniência:(Do gr. \textunderscore mukes\textunderscore , cogumelo)}
\end{itemize}
Que tem natureza microbiana.
\section{Mictéria}
\begin{itemize}
\item {Grp. gram.:f.}
\end{itemize}
\begin{itemize}
\item {Proveniência:(Do gr. \textunderscore mukter\textunderscore )}
\end{itemize}
Gênero de aves pernaltas.
\section{Micterismo}
\begin{itemize}
\item {Grp. gram.:f.}
\end{itemize}
\begin{itemize}
\item {Utilização:Neol.}
\end{itemize}
\begin{itemize}
\item {Proveniência:(Gr. \textunderscore mukterismos\textunderscore )}
\end{itemize}
Zombaria.
Carranca, má catadura. Cf. Pacheco, \textunderscore Promptuário\textunderscore .
\section{Midríase}
\begin{itemize}
\item {Grp. gram.:f.}
\end{itemize}
\begin{itemize}
\item {Proveniência:(Gr. \textunderscore mudriasis\textunderscore )}
\end{itemize}
Paralisia da íris.
Dilatação da pupila ocular.
\section{Midriático}
\begin{itemize}
\item {Grp. gram.:adj.}
\end{itemize}
Relativo á midríase.
Que produz a dilatação da pupila.
\section{Midrol}
\begin{itemize}
\item {Grp. gram.:m.}
\end{itemize}
Composto químico, de aplicação midriática.
\section{Mielastenia}
\begin{itemize}
\item {Grp. gram.:f.}
\end{itemize}
\begin{itemize}
\item {Utilização:Med.}
\end{itemize}
\begin{itemize}
\item {Proveniência:(Do gr. \textunderscore muelos\textunderscore  + \textunderscore a\textunderscore  priv. + \textunderscore sthenos\textunderscore )}
\end{itemize}
Fraqueza da medula espinhal, impotência da medula. Cf. Sousa Martins, \textunderscore Nosographia\textunderscore .
\section{Mielencefálico}
\begin{itemize}
\item {Grp. gram.:adj.}
\end{itemize}
Relativo ao mielencéfalo.
\section{Mielencéfalo}
\begin{itemize}
\item {Grp. gram.:m.}
\end{itemize}
\begin{itemize}
\item {Proveniência:(Do gr. \textunderscore muelos\textunderscore , e \textunderscore encéphalo\textunderscore )}
\end{itemize}
Conjunto dos órgãos, que formam o sistema nervoso central, e compreendem o encéfalo e a medula espinhal.
\section{Mielina}
\begin{itemize}
\item {Grp. gram.:f.}
\end{itemize}
\begin{itemize}
\item {Proveniência:(Do gr. \textunderscore muelos\textunderscore )}
\end{itemize}
Substância medular, contida nos tubos nervosos.
\section{Mielite}
\begin{itemize}
\item {Grp. gram.:f.}
\end{itemize}
\begin{itemize}
\item {Utilização:Geol.}
\end{itemize}
\begin{itemize}
\item {Proveniência:(Do gr. \textunderscore muelos\textunderscore )}
\end{itemize}
Inflamação da medula espinhal.
Uma das variedades mais importantes do caulim.
\section{Mielócito}
\begin{itemize}
\item {Grp. gram.:m.}
\end{itemize}
\begin{itemize}
\item {Proveniência:(Do gr. \textunderscore muelos\textunderscore  + \textunderscore kutos\textunderscore )}
\end{itemize}
Elementos da substância pardacenta do sistema encéfalo-raquídio.
\section{Mielóide}
\begin{itemize}
\item {Grp. gram.:adj.}
\end{itemize}
\begin{itemize}
\item {Proveniência:(Do gr. \textunderscore muelos\textunderscore  + \textunderscore eidos\textunderscore )}
\end{itemize}
Relativo á medula dos ossos.
Parecido á medula dos ossos.
\section{Mieloma}
\begin{itemize}
\item {Grp. gram.:m.}
\end{itemize}
\begin{itemize}
\item {Utilização:Med.}
\end{itemize}
\begin{itemize}
\item {Proveniência:(Do gr. \textunderscore muelos\textunderscore )}
\end{itemize}
Tumor medular.
\section{Mielomacia}
\begin{itemize}
\item {Grp. gram.:f.}
\end{itemize}
\begin{itemize}
\item {Utilização:Med.}
\end{itemize}
\begin{itemize}
\item {Proveniência:(Do gr. \textunderscore muelos\textunderscore  + \textunderscore malakia\textunderscore )}
\end{itemize}
Amolecimento da espinhal-medula.
\section{Mielopatia}
\begin{itemize}
\item {Grp. gram.:f.}
\end{itemize}
\begin{itemize}
\item {Utilização:Med.}
\end{itemize}
\begin{itemize}
\item {Proveniência:(Do gr. \textunderscore muelos\textunderscore  + \textunderscore pathos\textunderscore )}
\end{itemize}
Afecção da medula-espinhal.
\section{Mielosclerose}
\begin{itemize}
\item {Grp. gram.:f.}
\end{itemize}
\begin{itemize}
\item {Utilização:Med.}
\end{itemize}
Esclerose da medula espinhal.
\section{Mígala}
\begin{itemize}
\item {Grp. gram.:f.}
\end{itemize}
\begin{itemize}
\item {Proveniência:(Gr. \textunderscore mugale\textunderscore )}
\end{itemize}
Grande aranha, vulgar ao sul da Europa, e cuja mordedura produz inchação.
\section{Migálidas}
\begin{itemize}
\item {Grp. gram.:m. pl.}
\end{itemize}
\begin{itemize}
\item {Utilização:Zool.}
\end{itemize}
\begin{itemize}
\item {Proveniência:(Do gr. \textunderscore mugale\textunderscore  + \textunderscore eidos\textunderscore )}
\end{itemize}
Família de aranhas.
\section{Miginda}
\begin{itemize}
\item {Grp. gram.:f.}
\end{itemize}
\begin{itemize}
\item {Proveniência:(De \textunderscore Mygind\textunderscore , n. p. al.)}
\end{itemize}
Gênero de plantas celastríneas.
\section{Miíase}
\begin{itemize}
\item {Grp. gram.:f.}
\end{itemize}
\begin{itemize}
\item {Utilização:Med.}
\end{itemize}
\begin{itemize}
\item {Proveniência:(Do gr. \textunderscore muia\textunderscore , môsca)}
\end{itemize}
Doença, causada pela larva de certos dípteros.
\section{Miiocéfalo}
\begin{itemize}
\item {Grp. gram.:m.}
\end{itemize}
\begin{itemize}
\item {Utilização:Med.}
\end{itemize}
\begin{itemize}
\item {Proveniência:(Do gr. \textunderscore muia\textunderscore  + \textunderscore kephale\textunderscore )}
\end{itemize}
Espécie de estafiloma, que forma um pequenino tumor arredondado e escuro.
\section{Miiodopsia}
\begin{itemize}
\item {Grp. gram.:f.}
\end{itemize}
\begin{itemize}
\item {Utilização:Med.}
\end{itemize}
\begin{itemize}
\item {Proveniência:(Do gr. \textunderscore muiodes\textunderscore )}
\end{itemize}
Turvação visual, a que se dá vulgarmente o nome de \textunderscore môscas volantes\textunderscore .
\section{Miiologia}
\begin{itemize}
\item {Grp. gram.:f.}
\end{itemize}
\begin{itemize}
\item {Proveniência:(Do gr. \textunderscore muia\textunderscore  + \textunderscore logos\textunderscore )}
\end{itemize}
Tratado ou descripção das môscas.
\section{Miiológico}
\begin{itemize}
\item {Grp. gram.:adj.}
\end{itemize}
Relativo á miiologia.
\section{Miite}
\begin{itemize}
\item {Grp. gram.:f.}
\end{itemize}
\begin{itemize}
\item {Proveniência:(Do gr. \textunderscore mus\textunderscore )}
\end{itemize}
Inflamação nos músculos.
\section{Milabres}
\begin{itemize}
\item {Grp. gram.:m. pl.}
\end{itemize}
Insectos coleópteros, da fam. dos vesicantes.
\section{Míleo}
\begin{itemize}
\item {Grp. gram.:adj.}
\end{itemize}
\begin{itemize}
\item {Utilização:Anat.}
\end{itemize}
\begin{itemize}
\item {Proveniência:(Do gr. \textunderscore mulos\textunderscore )}
\end{itemize}
Relativo aos dentes molares.
\section{Miloglosso}
\begin{itemize}
\item {Grp. gram.:m.}
\end{itemize}
\begin{itemize}
\item {Proveniência:(Do gr. \textunderscore mulos\textunderscore  + \textunderscore glossa\textunderscore )}
\end{itemize}
Conjunto de fibras musculares, que vão da maxila inferior á faringe, passando por baixo dos dentes molares.
\section{Milóide}
\begin{itemize}
\item {Grp. gram.:adj.}
\end{itemize}
O mesmo que \textunderscore míleo\textunderscore .
\section{Milorde}
\begin{itemize}
\item {Grp. gram.:m.}
\end{itemize}
\begin{itemize}
\item {Utilização:Burl.}
\end{itemize}
\begin{itemize}
\item {Proveniência:(Ingl. \textunderscore mylord\textunderscore )}
\end{itemize}
Espécie de cabriolé de quatro rodas.
Que tem aparência de rico e nobre.
Figurão.
\section{Miôa}
\begin{itemize}
\item {Grp. gram.:f.}
\end{itemize}
\begin{itemize}
\item {Proveniência:(Do lat. \textunderscore myo\textunderscore )}
\end{itemize}
Molusco acéfalo.
\section{Miocárdio}
\begin{itemize}
\item {Grp. gram.:m.}
\end{itemize}
\begin{itemize}
\item {Utilização:Anat.}
\end{itemize}
\begin{itemize}
\item {Proveniência:(Do gr. \textunderscore mus\textunderscore  + \textunderscore kardia\textunderscore )}
\end{itemize}
A parte muscular do coração.
\section{Miocardite}
\begin{itemize}
\item {Grp. gram.:f.}
\end{itemize}
\begin{itemize}
\item {Proveniência:(Do gr. \textunderscore mus\textunderscore  + \textunderscore kardia\textunderscore )}
\end{itemize}
Inflamação da substância muscular do coração.
\section{Miocele}
\begin{itemize}
\item {Grp. gram.:m.}
\end{itemize}
\begin{itemize}
\item {Proveniência:(Do gr. \textunderscore mus\textunderscore  + \textunderscore kele\textunderscore )}
\end{itemize}
Tumor muscular.
\section{Miocelite}
\begin{itemize}
\item {Grp. gram.:f.}
\end{itemize}
\begin{itemize}
\item {Utilização:Med.}
\end{itemize}
\begin{itemize}
\item {Proveniência:(Do gr. \textunderscore mus\textunderscore  + \textunderscore kóslia\textunderscore )}
\end{itemize}
Inflamação dos músculos do baixo ventre.
\section{Miodinia}
\begin{itemize}
\item {Grp. gram.:f.}
\end{itemize}
\begin{itemize}
\item {Proveniência:(Do gr. \textunderscore mus\textunderscore  + \textunderscore odune\textunderscore )}
\end{itemize}
Dôr nos músculos; reumatismo muscular.
\section{Miografia}
\begin{itemize}
\item {Grp. gram.:f.}
\end{itemize}
Descripção dos músculos.
(Cp. \textunderscore miógrafo\textunderscore )
\section{Miográfico}
\begin{itemize}
\item {Grp. gram.:adj.}
\end{itemize}
Relativo á miografia.
\section{Miógrafo}
\begin{itemize}
\item {Grp. gram.:m.}
\end{itemize}
\begin{itemize}
\item {Proveniência:(Do gr. \textunderscore mus\textunderscore  + \textunderscore graphein\textunderscore )}
\end{itemize}
Instrumento, que representa graficamente a contracção dos músculos.
\section{Mioide}
\begin{itemize}
\item {Grp. gram.:adj.}
\end{itemize}
\begin{itemize}
\item {Proveniência:(Do gr. \textunderscore mus\textunderscore  + \textunderscore eidos\textunderscore )}
\end{itemize}
Diz-se do tumor formado de fibras musculares.
\section{Miolema}
\begin{itemize}
\item {Grp. gram.:f.}
\end{itemize}
\begin{itemize}
\item {Utilização:Anat.}
\end{itemize}
\begin{itemize}
\item {Proveniência:(Do gr. \textunderscore mus\textunderscore  + \textunderscore lemma\textunderscore )}
\end{itemize}
Tubo transparente, que contém fibrilhas musculares.
\section{Miólise}
\begin{itemize}
\item {Grp. gram.:f.}
\end{itemize}
\begin{itemize}
\item {Utilização:Med.}
\end{itemize}
\begin{itemize}
\item {Proveniência:(Do gr. \textunderscore mus\textunderscore  + \textunderscore lusis\textunderscore )}
\end{itemize}
Resolução da fibra muscular em seus elementos constitutivos.
\section{Miologia}
\begin{itemize}
\item {Grp. gram.:f.}
\end{itemize}
\begin{itemize}
\item {Proveniência:(Do gr. \textunderscore mus\textunderscore  + \textunderscore logos\textunderscore )}
\end{itemize}
O mesmo que \textunderscore miografia\textunderscore .
\section{Miológico}
\begin{itemize}
\item {Grp. gram.:adj.}
\end{itemize}
Relativo á miologia.
\section{Mioma}
\begin{itemize}
\item {Grp. gram.:m.}
\end{itemize}
\begin{itemize}
\item {Utilização:Med.}
\end{itemize}
\begin{itemize}
\item {Proveniência:(Do gr. \textunderscore mus\textunderscore , \textunderscore muos\textunderscore )}
\end{itemize}
Tumor, formado principalmente de tecido muscular.
\section{Miomalacia}
\begin{itemize}
\item {Grp. gram.:f.}
\end{itemize}
\begin{itemize}
\item {Utilização:Med.}
\end{itemize}
\begin{itemize}
\item {Proveniência:(Do gr. \textunderscore mus\textunderscore  + \textunderscore malakia\textunderscore )}
\end{itemize}
Amolecimento dos músculos.
\section{Miomérico}
\begin{itemize}
\item {Grp. gram.:adj.}
\end{itemize}
Relativo ao miómero.
\section{Miomério}
\begin{itemize}
\item {Grp. gram.:m.}
\end{itemize}
O mesmo que \textunderscore miómero\textunderscore .
\section{Miómero}
\begin{itemize}
\item {Grp. gram.:m.}
\end{itemize}
\begin{itemize}
\item {Proveniência:(Do gr. \textunderscore mus\textunderscore , \textunderscore muos\textunderscore  + \textunderscore meros\textunderscore )}
\end{itemize}
Parte muscular do metâmero.
\section{Miómetro}
\begin{itemize}
\item {Grp. gram.:m.}
\end{itemize}
\begin{itemize}
\item {Proveniência:(Do gr. \textunderscore mus\textunderscore  + \textunderscore metron\textunderscore )}
\end{itemize}
Instrumento, imaginado para medir o encurtamento dos músculos no ôlho estrábico.
\section{Mutualistamente}
\begin{itemize}
\item {Grp. gram.:adv.}
\end{itemize}
De modo mutualista. Cf. Oliv. Martins, \textunderscore Port. nos Mares\textunderscore , 243.
\section{Mutuamente}
\begin{itemize}
\item {Grp. gram.:adv.}
\end{itemize}
De modo mútuo; de parte a parte, reciprocamente.
\section{Mutuante}
\begin{itemize}
\item {Grp. gram.:m. ,  f.  e  adj.}
\end{itemize}
\begin{itemize}
\item {Proveniência:(Lat. \textunderscore mutuans\textunderscore )}
\end{itemize}
Pessôa, que mutúa.
Indivíduo, que dá por empréstimo.
\section{Mutuar}
\begin{itemize}
\item {Grp. gram.:v. t.}
\end{itemize}
\begin{itemize}
\item {Proveniência:(Lat. \textunderscore mutuari\textunderscore )}
\end{itemize}
Trocar entre si, (falando-se de mais de um indivíduo ou de collectividades).
Permutar.
Dar de empréstimo; tomar (como empréstimo).
\section{Mutuário}
\begin{itemize}
\item {Grp. gram.:m.}
\end{itemize}
\begin{itemize}
\item {Proveniência:(Lat. \textunderscore mutuarius\textunderscore )}
\end{itemize}
Aquelle que recebe alguma coisa por empréstimo.
\section{Mutuatário}
\begin{itemize}
\item {Grp. gram.:m.}
\end{itemize}
O mesmo que \textunderscore mutuário\textunderscore .
\section{Mutuca}
\begin{itemize}
\item {Grp. gram.:f.}
\end{itemize}
Espécie de môsca da região do Amazonas.
\section{Mutucuna}
\begin{itemize}
\item {Grp. gram.:f.}
\end{itemize}
Espécie de môsca da região do Amozonas.
\section{Mutuge}
\begin{itemize}
\item {Grp. gram.:m.}
\end{itemize}
Árvore angolense, (\textunderscore miristica angolensis\textunderscore , Welw.).
\section{Mutula}
\begin{itemize}
\item {Grp. gram.:f.}
\end{itemize}
\begin{itemize}
\item {Utilização:T. de Angola}
\end{itemize}
Espécie de leito, formado por quatro forquilhas que sustentam duas varas, sôbre que se atravessam paus recobertos por uma esteira. Cf. Capello e Ivens, I, 295.
\section{Mútulo}
\begin{itemize}
\item {Grp. gram.:m.}
\end{itemize}
\begin{itemize}
\item {Proveniência:(Lat. \textunderscore mutulus\textunderscore )}
\end{itemize}
Modilhão quadrado, em cornija de ordem dórica.
\section{Mutum}
\begin{itemize}
\item {Grp. gram.:m.}
\end{itemize}
Ave gallinácea do Brasil.
\section{Mutumbu}
\begin{itemize}
\item {Grp. gram.:m.}
\end{itemize}
Árvore moçambicana, cuja madeira se emprega em almadias, tabuado, etc.
\section{Mutunda}
\begin{itemize}
\item {Grp. gram.:f.}
\end{itemize}
O mesmo que \textunderscore mutundo\textunderscore .
\section{Mutundo}
\begin{itemize}
\item {Grp. gram.:m.}
\end{itemize}
(V.cangalulo)
\section{Mutune}
\begin{itemize}
\item {Grp. gram.:m.}
\end{itemize}
Árvore medicinal de Angola, o mesmo que \textunderscore cabul\textunderscore .
Nome de outra árvore igualmente, hypericínea, (\textunderscore haronga madagascariensis\textunderscore , Chois.).
\section{Mutunge}
\begin{itemize}
\item {Grp. gram.:m.}
\end{itemize}
Árvore de Angola, (\textunderscore haronga madagascariensis\textunderscore ), o mesmo que \textunderscore mutune\textunderscore .
\section{Mutungo}
\begin{itemize}
\item {Grp. gram.:m.}
\end{itemize}
Pequena árvore africana, de fôlhas simples, glabras, brevemente estipuladas, e flôres hermaphroditas.
\section{Mutunte}
\begin{itemize}
\item {Grp. gram.:m.}
\end{itemize}
Árvore de Angola.
\section{Mútuo}
\begin{itemize}
\item {Grp. gram.:adj.}
\end{itemize}
\begin{itemize}
\item {Grp. gram.:M.}
\end{itemize}
\begin{itemize}
\item {Proveniência:(Lat. \textunderscore mutuus\textunderscore )}
\end{itemize}
Que se permuta entre duas ou mais pessôas ou coisas; recíproco: \textunderscore soccorros mútuos\textunderscore .
Empréstimo; permutação.
Reciprocidade.
Contrato, em que se empresta um objecto, que deve sêr restituído no mesmo gênero, quantidade e qualidade.
\section{Mutuque}
\begin{itemize}
\item {Grp. gram.:m.}
\end{itemize}
Arbusto angolense.
\section{Mutuqueiro}
\begin{itemize}
\item {Grp. gram.:m.}
\end{itemize}
\begin{itemize}
\item {Utilização:Bras. do N}
\end{itemize}
Lugar, onde há muitas mutucas.
Grande quantidade de mutucas.
\section{Muturi}
\begin{itemize}
\item {Grp. gram.:m.}
\end{itemize}
\begin{itemize}
\item {Utilização:Bras}
\end{itemize}
Castanha de caju, que se emprega como adubo em vários guisados, em-quanto é verde.
\section{Muturicus}
\begin{itemize}
\item {Grp. gram.:m. pl.}
\end{itemize}
\begin{itemize}
\item {Utilização:Bras}
\end{itemize}
Tríbo de aborígenes do Pará.
\section{Muturuti}
\begin{itemize}
\item {Grp. gram.:m.}
\end{itemize}
Arvore de Angola.
\section{Mututi}
\begin{itemize}
\item {Grp. gram.:m.}
\end{itemize}
Árvore da região do Amazonas.
\section{Mututu}
\begin{itemize}
\item {Grp. gram.:m.}
\end{itemize}
Pequena árvore esterculiácea de Angola.
\section{Muungo}
\begin{itemize}
\item {Grp. gram.:m.}
\end{itemize}
Nome angolense da teca.
\section{Muvandi}
\begin{itemize}
\item {Grp. gram.:m.}
\end{itemize}
Árvore do Congo.
\section{Muviú}
\begin{itemize}
\item {Grp. gram.:m.}
\end{itemize}
Arvoreta angolense, que cresce na areia e entre pedras.
\section{Muvovo}
\begin{itemize}
\item {Grp. gram.:m.}
\end{itemize}
Árvore angolense.
\section{Muvuga}
\begin{itemize}
\item {Grp. gram.:f.}
\end{itemize}
Formosa árvore africana, de fôlhas simples, alternas, glabras, com cachos de flôres polysépalas, branco-amareladas.
\section{Muxara}
\begin{itemize}
\item {Grp. gram.:f.}
\end{itemize}
\begin{itemize}
\item {Utilização:Ant.}
\end{itemize}
\begin{itemize}
\item {Proveniência:(T. as.)}
\end{itemize}
Agasalho, asylo.
\section{Muxaxa}
\begin{itemize}
\item {Grp. gram.:f.}
\end{itemize}
Árvore angolense.
\section{Muxiba}
\begin{itemize}
\item {Grp. gram.:f.}
\end{itemize}
\begin{itemize}
\item {Utilização:Bras}
\end{itemize}
Carne magra; pelhancas.
(Talvez do quimbundo)
\section{Muxicongos}
\begin{itemize}
\item {Grp. gram.:m. pl.}
\end{itemize}
Congueses, que habitam ao sul de San-Salvador do Congo.
\section{Muxiloxilo}
\begin{itemize}
\item {Grp. gram.:m.}
\end{itemize}
Árvore robusta, da fam. das verbenáceas, (\textunderscore vitex cuniata\textunderscore ?).
\section{Muxinga}
\begin{itemize}
\item {Grp. gram.:f.}
\end{itemize}
\begin{itemize}
\item {Utilização:Bras}
\end{itemize}
Sova; tunda.
Azorrague.
(Quimb. \textunderscore muxinga\textunderscore )
\section{Muxirão}
\begin{itemize}
\item {Grp. gram.:m.}
\end{itemize}
\begin{itemize}
\item {Utilização:Bras}
\end{itemize}
Auxílio, a que se prestam reciprocamente, durante um dia, os pequenos agricultores, no tempo das plantações e colheitas.
(Do tupi)
\section{Muxiri}
\begin{itemize}
\item {Grp. gram.:m.}
\end{itemize}
Arbusto leguminoso de Angola, (\textunderscore eriosema muxiria\textunderscore , Baker).
\section{Muxirom}
\begin{itemize}
\item {Grp. gram.:m.}
\end{itemize}
\begin{itemize}
\item {Utilização:Bras}
\end{itemize}
O mesmo que \textunderscore muxirão\textunderscore .
\section{Muxuango}
\begin{itemize}
\item {Grp. gram.:m.}
\end{itemize}
\begin{itemize}
\item {Utilização:Bras}
\end{itemize}
O mesmo que \textunderscore caipira\textunderscore .
\section{Muxuri}
\begin{itemize}
\item {Grp. gram.:m.}
\end{itemize}
Árvore tinctória da região do Amazonas.
\section{Muzangala-cachico}
\begin{itemize}
\item {Grp. gram.:m.}
\end{itemize}
Arbusto africano, de caule subterrâneo, herbáceo, de flôres solitárias e corolla amarela.
\section{Muzeba}
\begin{itemize}
\item {Grp. gram.:f.}
\end{itemize}
Árvore africana, da fam. das leguminosas.
\section{Muzemba}
\begin{itemize}
\item {Grp. gram.:f.}
\end{itemize}
Grande árvore africana, (\textunderscore albizzia coriaria\textunderscore , Welw.).
\section{Muzenza}
\begin{itemize}
\item {Grp. gram.:f.}
\end{itemize}
Arbusto africano, muito espinhoso, de fôlhas miúdas e flôres côr de rosa.
\section{Muzimbas}
\begin{itemize}
\item {Grp. gram.:m. pl.}
\end{itemize}
Antigo povo selvagem que, procedendo do centro da África, invadiu no século XVI a costa oriental, desde Moçambique a Melinde. Cf. Couto, \textunderscore Déc.\textunderscore 
\section{Muzimo}
\begin{itemize}
\item {Grp. gram.:m.}
\end{itemize}
\begin{itemize}
\item {Utilização:T. da Áfr. Or. Port}
\end{itemize}
Adivinho; feiticeiro.
\section{Muzombe}
\begin{itemize}
\item {Grp. gram.:m.}
\end{itemize}
(V.mussombo)
\section{Múzua}
\begin{itemize}
\item {Grp. gram.:f.}
\end{itemize}
Trepadeira africana.
\section{Muzuco}
\begin{itemize}
\item {Grp. gram.:m.}
\end{itemize}
Árvore intertropical, da fam. das leguminosas.
\section{Muzuemba}
\begin{itemize}
\item {Grp. gram.:f.}
\end{itemize}
Árvore angolense, de casca adstringente.
Não será o mesmo que \textunderscore muzemba\textunderscore ?
\section{Muzumba}
\begin{itemize}
\item {Grp. gram.:f.}
\end{itemize}
Pequena árvore leguminosa de Angola, (\textunderscore milletia versicolor\textunderscore , Welw.).
\section{Muzumbo}
\begin{itemize}
\item {Grp. gram.:m.}
\end{itemize}
O mesmo que \textunderscore muzumba\textunderscore .
\section{Muzuna}
\begin{itemize}
\item {Grp. gram.:f.}
\end{itemize}
Moéda de prata, em Marrocos.
\section{Muzungo}
\begin{itemize}
\item {Grp. gram.:m.}
\end{itemize}
O mesmo que \textunderscore português\textunderscore , entre os indígenas do Lui, em África.
\section{Muzungo}
\begin{itemize}
\item {Grp. gram.:m.}
\end{itemize}
Pequena árvore africana, (\textunderscore piptadenia africana\textunderscore , Hook.).
\section{Muzuzo}
\begin{itemize}
\item {Grp. gram.:m.}
\end{itemize}
Grande serpente angolense, (\textunderscore siminophis bicolor\textunderscore ).
\section{Muzuzuros}
\begin{itemize}
\item {Grp. gram.:m. pl.}
\end{itemize}
Uma das tríbos cafreaes de Tete e Zumbo.
\section{Myagro}
\begin{itemize}
\item {Grp. gram.:m.}
\end{itemize}
Planta crucífera.
\section{Myalgia}
\begin{itemize}
\item {Grp. gram.:f.}
\end{itemize}
\begin{itemize}
\item {Proveniência:(Do gr. \textunderscore mus\textunderscore  + \textunderscore algos\textunderscore )}
\end{itemize}
Dôr nos músculos.
\section{Mycélio}
\begin{itemize}
\item {Grp. gram.:m.}
\end{itemize}
\begin{itemize}
\item {Proveniência:(Do gr. \textunderscore mukes\textunderscore , cogumelo)}
\end{itemize}
Parte filamentosa do thallo do fungo.
Filamentos que, partindo das raízes do míldio, se insinuam entre as paredes das céllulas das fôlhas da videira.
\section{Mycênico}
\begin{itemize}
\item {Grp. gram.:adj.}
\end{itemize}
Relativo a Mycenas: \textunderscore a arte mycênica em Portugal\textunderscore .
\section{Mycetographia}
\begin{itemize}
\item {Grp. gram.:f.}
\end{itemize}
\begin{itemize}
\item {Proveniência:(Do gr. \textunderscore mukes\textunderscore  + \textunderscore graphein\textunderscore )}
\end{itemize}
Descripção ou história dos cogumelos.
\section{Mycetográphico}
\begin{itemize}
\item {Grp. gram.:adj.}
\end{itemize}
Relativo á mycetographia.
\section{Mycetologia}
\begin{itemize}
\item {Grp. gram.:f.}
\end{itemize}
O mesmo que \textunderscore mycologia\textunderscore .
\section{Mycetoma}
\begin{itemize}
\item {Grp. gram.:m.}
\end{itemize}
\begin{itemize}
\item {Utilização:Med.}
\end{itemize}
\begin{itemize}
\item {Proveniência:(Do gr. \textunderscore mukhes\textunderscore )}
\end{itemize}
Moléstia parasitária.
\section{Mycetóphagos}
\begin{itemize}
\item {Grp. gram.:m. pl.}
\end{itemize}
\begin{itemize}
\item {Proveniência:(Do gr. \textunderscore mukes\textunderscore  + \textunderscore phagein\textunderscore )}
\end{itemize}
Gênero de insectos coleópteros.
\section{Mycogenia}
\begin{itemize}
\item {Grp. gram.:f.}
\end{itemize}
\begin{itemize}
\item {Proveniência:(Do gr. \textunderscore mukes\textunderscore  + \textunderscore genos\textunderscore )}
\end{itemize}
Producção de mucedíneos.
\section{Mycogênico}
\begin{itemize}
\item {Grp. gram.:adj.}
\end{itemize}
Relativo á mycogenia.
Que produz mucedíneos.
\section{Mycologia}
\begin{itemize}
\item {Grp. gram.:f.}
\end{itemize}
Tratado dos cogumelos.
(Cp. \textunderscore mycólogo\textunderscore )
\section{Mycologista}
\begin{itemize}
\item {Grp. gram.:m.  e  adj.}
\end{itemize}
\begin{itemize}
\item {Proveniência:(Do gr. \textunderscore mukes\textunderscore  + \textunderscore logos\textunderscore )}
\end{itemize}
Aquelle que é versado em mycologia ou trata desta sciência.
\section{Mycólogo}
\begin{itemize}
\item {Grp. gram.:m.  e  adj.}
\end{itemize}
\begin{itemize}
\item {Proveniência:(Do gr. \textunderscore mukes\textunderscore  + \textunderscore logos\textunderscore )}
\end{itemize}
Aquelle que é versado em mycologia ou trata desta sciência.
\section{Mycose}
\begin{itemize}
\item {Grp. gram.:f.}
\end{itemize}
\begin{itemize}
\item {Proveniência:(Do gr. \textunderscore mukes\textunderscore )}
\end{itemize}
Saliência esponjosa, em fórma de cogumelo.
Princípio açucarado da cravagem do centeio.
\section{Mycótico}
\begin{itemize}
\item {Grp. gram.:adj.}
\end{itemize}
\begin{itemize}
\item {Proveniência:(Do gr. \textunderscore mukes\textunderscore , cogumelo)}
\end{itemize}
Que tem natureza microbiana.
\section{Myctéria}
\begin{itemize}
\item {Grp. gram.:f.}
\end{itemize}
\begin{itemize}
\item {Proveniência:(Do gr. \textunderscore mukter\textunderscore )}
\end{itemize}
Gênero de aves pernaltas.
\section{Mycterismo}
\begin{itemize}
\item {Grp. gram.:f.}
\end{itemize}
\begin{itemize}
\item {Utilização:Neol.}
\end{itemize}
\begin{itemize}
\item {Proveniência:(Gr. \textunderscore mukterismos\textunderscore )}
\end{itemize}
Zombaria.
Carranca, má catadura. Cf. Pacheco, \textunderscore Promptuário\textunderscore .
\section{Mydríase}
\begin{itemize}
\item {Grp. gram.:f.}
\end{itemize}
\begin{itemize}
\item {Proveniência:(Gr. \textunderscore mudriasis\textunderscore )}
\end{itemize}
Paralysia da íris.
Dilatação da pupilla ocular.
\section{Mydriático}
\begin{itemize}
\item {Grp. gram.:adj.}
\end{itemize}
Relativo á mydríase.
Que produz a dilatação da pupilla.
\section{Mydrol}
\begin{itemize}
\item {Grp. gram.:m.}
\end{itemize}
Composto chímico, de applicação mydriática.
\section{Myelasthenia}
\begin{itemize}
\item {Grp. gram.:f.}
\end{itemize}
\begin{itemize}
\item {Utilização:Med.}
\end{itemize}
\begin{itemize}
\item {Proveniência:(Do gr. \textunderscore muelos\textunderscore  + \textunderscore a\textunderscore  priv. + \textunderscore sthenos\textunderscore )}
\end{itemize}
Fraqueza da medulla espinhal, impotência da medulla. Cf. Sousa Martins, \textunderscore Nosographia\textunderscore .
\section{Myelatelia}
\begin{itemize}
\item {Grp. gram.:f.}
\end{itemize}
\begin{itemize}
\item {Utilização:Med.}
\end{itemize}
\begin{itemize}
\item {Proveniência:(Do gr. \textunderscore muelos\textunderscore  + \textunderscore ateles\textunderscore )}
\end{itemize}
Desenvolvimento incompleto da medulla espinhal.
\section{Myelencephálico}
\begin{itemize}
\item {Grp. gram.:adj.}
\end{itemize}
Relativo ao myelencéfalo.
\section{Myelencéphalo}
\begin{itemize}
\item {Grp. gram.:m.}
\end{itemize}
\begin{itemize}
\item {Proveniência:(Do gr. \textunderscore muelos\textunderscore , e \textunderscore encéphalo\textunderscore )}
\end{itemize}
Conjunto dos órgãos, que formam o systema nervoso central, e comprehendem o encéphalo e a medulla espinhal.
\section{Myelina}
\begin{itemize}
\item {Grp. gram.:f.}
\end{itemize}
\begin{itemize}
\item {Proveniência:(Do gr. \textunderscore muelos\textunderscore )}
\end{itemize}
Substância medullar, contida nos tubos nervosos.
\section{Myelite}
\begin{itemize}
\item {Grp. gram.:f.}
\end{itemize}
\begin{itemize}
\item {Utilização:Geol.}
\end{itemize}
\begin{itemize}
\item {Proveniência:(Do gr. \textunderscore muelos\textunderscore )}
\end{itemize}
Inflammação da medulla espinhal.
Uma das variedades mais importantes do caulim.
\section{Myelócyto}
\begin{itemize}
\item {Grp. gram.:m.}
\end{itemize}
\begin{itemize}
\item {Proveniência:(Do gr. \textunderscore muelos\textunderscore  + \textunderscore kutos\textunderscore )}
\end{itemize}
Elementos da substância pardacenta do systema encéphalo-rachídio.
\section{Myelóide}
\begin{itemize}
\item {Grp. gram.:adj.}
\end{itemize}
\begin{itemize}
\item {Proveniência:(Do gr. \textunderscore muelos\textunderscore  + \textunderscore eidos\textunderscore )}
\end{itemize}
Relativo á medulla dos ossos.
Parecido á medulla dos ossos.
\section{Myeloma}
\begin{itemize}
\item {Grp. gram.:m.}
\end{itemize}
\begin{itemize}
\item {Utilização:Med.}
\end{itemize}
\begin{itemize}
\item {Proveniência:(Do gr. \textunderscore muelos\textunderscore )}
\end{itemize}
Tumor medullar.
\section{Myelomacia}
\begin{itemize}
\item {Grp. gram.:f.}
\end{itemize}
\begin{itemize}
\item {Utilização:Med.}
\end{itemize}
\begin{itemize}
\item {Proveniência:(Do gr. \textunderscore muelos\textunderscore  + \textunderscore malakia\textunderscore )}
\end{itemize}
Amollecimento da espinhal-medulla.
\section{Myelopathia}
\begin{itemize}
\item {Grp. gram.:f.}
\end{itemize}
\begin{itemize}
\item {Utilização:Med.}
\end{itemize}
\begin{itemize}
\item {Proveniência:(Do gr. \textunderscore muelos\textunderscore  + \textunderscore pathos\textunderscore )}
\end{itemize}
Affecção da medulla-espinhal.
\section{Myelo-sarcoma}
\begin{itemize}
\item {Grp. gram.:m.}
\end{itemize}
\begin{itemize}
\item {Proveniência:(Do gr. \textunderscore muelos\textunderscore  + \textunderscore sarkoma\textunderscore )}
\end{itemize}
Sarcoma da medulla dos ossos.
\section{Myelosclerose}
\begin{itemize}
\item {Grp. gram.:f.}
\end{itemize}
\begin{itemize}
\item {Utilização:Med.}
\end{itemize}
Esclerose da medulla espinhal.
\section{Mýgala}
\begin{itemize}
\item {Grp. gram.:f.}
\end{itemize}
\begin{itemize}
\item {Proveniência:(Gr. \textunderscore mugale\textunderscore )}
\end{itemize}
Grande aranha, vulgar ao sul da Europa, e cuja mordedura produz inchação.
\section{Mygálidas}
\begin{itemize}
\item {Grp. gram.:m. pl.}
\end{itemize}
\begin{itemize}
\item {Utilização:Zool.}
\end{itemize}
\begin{itemize}
\item {Proveniência:(Do gr. \textunderscore mugale\textunderscore  + \textunderscore eidos\textunderscore )}
\end{itemize}
Família de aranhas.
\section{Myginda}
\begin{itemize}
\item {Grp. gram.:f.}
\end{itemize}
\begin{itemize}
\item {Proveniência:(De \textunderscore Mygind\textunderscore , n. p. al.)}
\end{itemize}
Gênero de plantas celastríneas.
\section{Myíase}
\begin{itemize}
\item {Grp. gram.:f.}
\end{itemize}
\begin{itemize}
\item {Utilização:Med.}
\end{itemize}
\begin{itemize}
\item {Proveniência:(Do gr. \textunderscore muia\textunderscore , môsca)}
\end{itemize}
Doença, causada pela larva de certos dípteros.
\section{Myiocéphalo}
\begin{itemize}
\item {Grp. gram.:m.}
\end{itemize}
\begin{itemize}
\item {Utilização:Med.}
\end{itemize}
\begin{itemize}
\item {Proveniência:(Do gr. \textunderscore muia\textunderscore  + \textunderscore kephale\textunderscore )}
\end{itemize}
Espécie de estaphyloma, que forma um pequenino tumor arredondado e escuro.
\section{Myiodopsia}
\begin{itemize}
\item {Grp. gram.:f.}
\end{itemize}
\begin{itemize}
\item {Utilização:Med.}
\end{itemize}
\begin{itemize}
\item {Proveniência:(Do gr. \textunderscore muiodes\textunderscore )}
\end{itemize}
Turvação visual, a que se dá vulgarmente o nome de \textunderscore môscas volantes\textunderscore .
\section{Myiologia}
\begin{itemize}
\item {Grp. gram.:f.}
\end{itemize}
\begin{itemize}
\item {Proveniência:(Do gr. \textunderscore muia\textunderscore  + \textunderscore logos\textunderscore )}
\end{itemize}
Tratado ou descripção das môscas.
\section{Myiológico}
\begin{itemize}
\item {Grp. gram.:adj.}
\end{itemize}
Relativo á myiologia.
\section{Myite}
\begin{itemize}
\item {Grp. gram.:f.}
\end{itemize}
\begin{itemize}
\item {Proveniência:(Do gr. \textunderscore mus\textunderscore )}
\end{itemize}
Inflammação nos músculos.
\section{Mylabres}
\begin{itemize}
\item {Grp. gram.:m. pl.}
\end{itemize}
Insectos coleópteros, da fam. dos vesicantes.
\section{Mýleo}
\begin{itemize}
\item {Grp. gram.:adj.}
\end{itemize}
\begin{itemize}
\item {Utilização:Anat.}
\end{itemize}
\begin{itemize}
\item {Proveniência:(Do gr. \textunderscore mulos\textunderscore )}
\end{itemize}
Relativo aos dentes molares.
\section{Myloglosso}
\begin{itemize}
\item {Grp. gram.:m.}
\end{itemize}
\begin{itemize}
\item {Proveniência:(Do gr. \textunderscore mulos\textunderscore  + \textunderscore glossa\textunderscore )}
\end{itemize}
Conjunto de fibras musculares, que vão da maxilla inferior á pharynge, passando por baixo dos dentes molares.
\section{Mylóide}
\begin{itemize}
\item {Grp. gram.:adj.}
\end{itemize}
O mesmo que \textunderscore mýleo\textunderscore .
\section{Mylord}
\begin{itemize}
\item {Grp. gram.:m.}
\end{itemize}
\begin{itemize}
\item {Utilização:Burl.}
\end{itemize}
\begin{itemize}
\item {Proveniência:(Ingl. \textunderscore mylord\textunderscore )}
\end{itemize}
Espécie de cabriolé de quatro rodas.
Que tem apparência de rico e nobre.
Figurão.
\section{Myôa}
\begin{itemize}
\item {Grp. gram.:f.}
\end{itemize}
\begin{itemize}
\item {Proveniência:(Do lat. \textunderscore myo\textunderscore )}
\end{itemize}
Mollusco acéphalo.
\section{Myocárdio}
\begin{itemize}
\item {Grp. gram.:m.}
\end{itemize}
\begin{itemize}
\item {Utilização:Anat.}
\end{itemize}
\begin{itemize}
\item {Proveniência:(Do gr. \textunderscore mus\textunderscore  + \textunderscore kardia\textunderscore )}
\end{itemize}
A parte muscular do coração.
\section{Myocardite}
\begin{itemize}
\item {Grp. gram.:f.}
\end{itemize}
\begin{itemize}
\item {Proveniência:(Do gr. \textunderscore mus\textunderscore  + \textunderscore kardia\textunderscore )}
\end{itemize}
Inflammação da substância muscular do coração.
\section{Myocele}
\begin{itemize}
\item {Grp. gram.:m.}
\end{itemize}
\begin{itemize}
\item {Proveniência:(Do gr. \textunderscore mus\textunderscore  + \textunderscore kele\textunderscore )}
\end{itemize}
Tumor muscular.
\section{Myocelite}
\begin{itemize}
\item {Grp. gram.:f.}
\end{itemize}
\begin{itemize}
\item {Utilização:Med.}
\end{itemize}
\begin{itemize}
\item {Proveniência:(Do gr. \textunderscore mus\textunderscore  + \textunderscore kóslia\textunderscore )}
\end{itemize}
Inflammação dos músculos do baixo ventre.
\section{Myodynia}
\begin{itemize}
\item {Grp. gram.:f.}
\end{itemize}
\begin{itemize}
\item {Proveniência:(Do gr. \textunderscore mus\textunderscore  + \textunderscore odune\textunderscore )}
\end{itemize}
Dôr nos músculos; rheumatismo muscular.
\section{Myographia}
\begin{itemize}
\item {Grp. gram.:f.}
\end{itemize}
Descripção dos músculos.
(Cp. \textunderscore myógrapho\textunderscore )
\section{Myográphico}
\begin{itemize}
\item {Grp. gram.:adj.}
\end{itemize}
Relativo á myographia.
\section{Myógrapho}
\begin{itemize}
\item {Grp. gram.:m.}
\end{itemize}
\begin{itemize}
\item {Proveniência:(Do gr. \textunderscore mus\textunderscore  + \textunderscore graphein\textunderscore )}
\end{itemize}
Instrumento, que representa graphicamente a contracção dos músculos.
\section{Myoide}
\begin{itemize}
\item {Grp. gram.:adj.}
\end{itemize}
\begin{itemize}
\item {Proveniência:(Do gr. \textunderscore mus\textunderscore  + \textunderscore eidos\textunderscore )}
\end{itemize}
Diz-se do tumor formado de fibras musculares.
\section{Myolemma}
\begin{itemize}
\item {Grp. gram.:f.}
\end{itemize}
\begin{itemize}
\item {Utilização:Anat.}
\end{itemize}
\begin{itemize}
\item {Proveniência:(Do gr. \textunderscore mus\textunderscore  + \textunderscore lemma\textunderscore )}
\end{itemize}
Tubo transparente, que contém fibrilhas musculares.
\section{Myólise}
\begin{itemize}
\item {Grp. gram.:f.}
\end{itemize}
\begin{itemize}
\item {Utilização:Med.}
\end{itemize}
\begin{itemize}
\item {Proveniência:(Do gr. \textunderscore mus\textunderscore  + \textunderscore lusis\textunderscore )}
\end{itemize}
Resolução da fibra muscular em seus elementos constitutivos.
\section{Myologia}
\begin{itemize}
\item {Grp. gram.:f.}
\end{itemize}
\begin{itemize}
\item {Proveniência:(Do gr. \textunderscore mus\textunderscore  + \textunderscore logos\textunderscore )}
\end{itemize}
O mesmo que \textunderscore myographia\textunderscore .
\section{Myológico}
\begin{itemize}
\item {Grp. gram.:adj.}
\end{itemize}
Relativo á myologia.
\section{Myoma}
\begin{itemize}
\item {Grp. gram.:m.}
\end{itemize}
\begin{itemize}
\item {Utilização:Med.}
\end{itemize}
\begin{itemize}
\item {Proveniência:(Do gr. \textunderscore mus\textunderscore , \textunderscore muos\textunderscore )}
\end{itemize}
Tumor, formado principalmente de tecido muscular.
\section{Myomalacia}
\begin{itemize}
\item {Grp. gram.:f.}
\end{itemize}
\begin{itemize}
\item {Utilização:Med.}
\end{itemize}
\begin{itemize}
\item {Proveniência:(Do gr. \textunderscore mus\textunderscore  + \textunderscore malakia\textunderscore )}
\end{itemize}
Amollecimento dos músculos.
\section{Myomérico}
\begin{itemize}
\item {Grp. gram.:adj.}
\end{itemize}
Relativo ao myómero.
\section{Myomério}
\begin{itemize}
\item {Grp. gram.:m.}
\end{itemize}
O mesmo que \textunderscore myómero\textunderscore .
\section{Myómero}
\begin{itemize}
\item {Grp. gram.:m.}
\end{itemize}
\begin{itemize}
\item {Proveniência:(Do gr. \textunderscore mus\textunderscore , \textunderscore muos\textunderscore  + \textunderscore meros\textunderscore )}
\end{itemize}
Parte muscular do metâmero.
\section{Myómetro}
\begin{itemize}
\item {Grp. gram.:m.}
\end{itemize}
\begin{itemize}
\item {Proveniência:(Do gr. \textunderscore mus\textunderscore  + \textunderscore metron\textunderscore )}
\end{itemize}
Instrumento, imaginado para medir o encurtamento dos músculos no ôlho estrábico.
\section{Mióparo}
\begin{itemize}
\item {Grp. gram.:m.}
\end{itemize}
\begin{itemize}
\item {Proveniência:(Gr. \textunderscore muoparon\textunderscore )}
\end{itemize}
Espécie de navio antigo e ligeiro, usado por piratas.
\section{Miopas}
\begin{itemize}
\item {Grp. gram.:f. pl.}
\end{itemize}
\begin{itemize}
\item {Proveniência:(Do gr. \textunderscore muia\textunderscore  + \textunderscore ops\textunderscore )}
\end{itemize}
Gênero de insectos dípteros.
\section{Miopathia}
\begin{itemize}
\item {Grp. gram.:f.}
\end{itemize}
\begin{itemize}
\item {Utilização:Med.}
\end{itemize}
\begin{itemize}
\item {Proveniência:(Do gr. \textunderscore mus\textunderscore , \textunderscore muos\textunderscore  + \textunderscore pathos\textunderscore )}
\end{itemize}
Qualquer affecção muscular.
\section{Míope}
\begin{itemize}
\item {Grp. gram.:m.  e  f.}
\end{itemize}
\begin{itemize}
\item {Utilização:Fig.}
\end{itemize}
\begin{itemize}
\item {Grp. gram.:Adj.}
\end{itemize}
\begin{itemize}
\item {Proveniência:(Do gr. \textunderscore muops\textunderscore )}
\end{itemize}
Pessôa, que tem a vista muito curta, ou que sofre miopia.
Pessôa pouco inteligente ou perspicaz.
Que sofre miopia.--A pronúncia exacta é \textunderscore miópe\textunderscore , mas não se usa.
\section{Miopia}
\begin{itemize}
\item {Grp. gram.:f.}
\end{itemize}
\begin{itemize}
\item {Utilização:Fig.}
\end{itemize}
\begin{itemize}
\item {Proveniência:(Do gr. \textunderscore muopia\textunderscore )}
\end{itemize}
Imperfeição da vista, que só permitte vêr os objectos a pequena distância do ôlho.
Vista curta.
Falta de perspicácia.
\section{Mioplasma}
\begin{itemize}
\item {Grp. gram.:m.}
\end{itemize}
\begin{itemize}
\item {Proveniência:(Do gr. \textunderscore mus\textunderscore  + \textunderscore plassein\textunderscore )}
\end{itemize}
Plasma muscular.
\section{Mioporíneas}
\begin{itemize}
\item {Grp. gram.:f. pl.}
\end{itemize}
\begin{itemize}
\item {Proveniência:(De \textunderscore mioporíneo\textunderscore )}
\end{itemize}
Família de plantas, que têm por tipo o mióporo.
\section{Mioporíneo}
\begin{itemize}
\item {Grp. gram.:adj.}
\end{itemize}
Relativo ou semelhante ao mióporo.
\section{Mióporo}
\begin{itemize}
\item {Grp. gram.:m.}
\end{itemize}
\begin{itemize}
\item {Proveniência:(Do gr. \textunderscore muia\textunderscore  + \textunderscore poros\textunderscore )}
\end{itemize}
Gênero de arbustos, originários da Nova-Holanda, (\textunderscore myoporum acuminatum\textunderscore , Brown.).
\section{Miopresbita}
\begin{itemize}
\item {Grp. gram.:m.  e  adj.}
\end{itemize}
\begin{itemize}
\item {Utilização:Med.}
\end{itemize}
\begin{itemize}
\item {Proveniência:(De \textunderscore míope\textunderscore  + \textunderscore presbita\textunderscore )}
\end{itemize}
Diz-se do indivíduo, que é míope de um ôlho e presbita do outro.
\section{Miose}
\begin{itemize}
\item {Grp. gram.:f.}
\end{itemize}
\begin{itemize}
\item {Proveniência:(Do gr. \textunderscore muein\textunderscore )}
\end{itemize}
Contracção permanente da pupila.
\section{Miosite}
\begin{itemize}
\item {Grp. gram.:f.}
\end{itemize}
\begin{itemize}
\item {Utilização:Med.}
\end{itemize}
\begin{itemize}
\item {Proveniência:(Do gr. \textunderscore mus\textunderscore , \textunderscore muos\textunderscore )}
\end{itemize}
Inflamação dos músculos.
\section{Miosota}
\begin{itemize}
\item {Grp. gram.:f.}
\end{itemize}
\begin{itemize}
\item {Proveniência:(Lat. \textunderscore myosota\textunderscore )}
\end{itemize}
Gênero de plantas borragíneas, cuja espécie principal, de flôres miúdas e azues, é o \textunderscore forget-me-not\textunderscore  dos inglêses, que corresponde a \textunderscore não-te-esqueças\textunderscore .
\section{Miosótis}
\begin{itemize}
\item {Grp. gram.:f.}
\end{itemize}
(V.miosota)
\section{Miosuro}
\begin{itemize}
\item {Grp. gram.:m.}
\end{itemize}
\begin{itemize}
\item {Proveniência:(Do gr. \textunderscore muos\textunderscore  + \textunderscore oura\textunderscore )}
\end{itemize}
Planta ranunculácea.
\section{Miospasia}
\begin{itemize}
\item {Grp. gram.:f.}
\end{itemize}
\begin{itemize}
\item {Utilização:Med.}
\end{itemize}
\begin{itemize}
\item {Proveniência:(Do gr. \textunderscore mus\textunderscore  + \textunderscore spasis\textunderscore )}
\end{itemize}
Qualquer moléstia nervosa, que se traduz em espasmos.
\section{Miótico}
\begin{itemize}
\item {Grp. gram.:adj.}
\end{itemize}
Relativo á miose.
\section{Miotomia}
\begin{itemize}
\item {Grp. gram.:f.}
\end{itemize}
\begin{itemize}
\item {Proveniência:(Do gr. \textunderscore mus\textunderscore  + \textunderscore tome\textunderscore )}
\end{itemize}
Operação, que tem por objecto a dissecção dos músculos.
\section{Miotómico}
\begin{itemize}
\item {Grp. gram.:adj.}
\end{itemize}
Relativo á miotomia.
\section{Miótomo}
\begin{itemize}
\item {Grp. gram.:m.}
\end{itemize}
\begin{itemize}
\item {Proveniência:(Do gr. \textunderscore mus\textunderscore , \textunderscore muos\textunderscore  + \textunderscore tome\textunderscore )}
\end{itemize}
Instrumento cirúrgico, para fazer incisão num músculo debaixo da conjunctiva.
\section{Míria...}
\begin{itemize}
\item {Grp. gram.:f. pref.}
\end{itemize}
\begin{itemize}
\item {Proveniência:(Do gr. \textunderscore murioi\textunderscore )}
\end{itemize}
(designativo de \textunderscore déz vezes mil\textunderscore )
\section{Miríade}
\begin{itemize}
\item {Grp. gram.:f.}
\end{itemize}
\begin{itemize}
\item {Utilização:Fig.}
\end{itemize}
\begin{itemize}
\item {Proveniência:(Gr. \textunderscore murias\textunderscore )}
\end{itemize}
Número de déz mil.
Grande quantidade, quantidade indefinida.
\section{Miriagramma}
\begin{itemize}
\item {Grp. gram.:m.}
\end{itemize}
\begin{itemize}
\item {Proveniência:(De \textunderscore míria...\textunderscore  + \textunderscore gramma\textunderscore )}
\end{itemize}
Pêso de déz mil grammas.
\section{Mirialitro}
\begin{itemize}
\item {Grp. gram.:m.}
\end{itemize}
\begin{itemize}
\item {Proveniência:(De \textunderscore míria...\textunderscore  + \textunderscore litro\textunderscore )}
\end{itemize}
Número de déz mil litros.
\section{Miriâmetro}
\begin{itemize}
\item {Grp. gram.:m.}
\end{itemize}
\begin{itemize}
\item {Proveniência:(De \textunderscore míria...\textunderscore  + \textunderscore metro\textunderscore )}
\end{itemize}
Medida itinerária, de déz mil metros.
\section{Miriana}
\begin{itemize}
\item {Grp. gram.:f.}
\end{itemize}
Animal radiário das grandes profundidades do Oceano.
\section{Miriantho}
\begin{itemize}
\item {Grp. gram.:m.}
\end{itemize}
\begin{itemize}
\item {Proveniência:(Do gr. \textunderscore murias\textunderscore  + \textunderscore anthos\textunderscore )}
\end{itemize}
Árvore cucurbitácea da África.
\section{Miriápode}
\begin{itemize}
\item {Grp. gram.:adj.}
\end{itemize}
\begin{itemize}
\item {Grp. gram.:M. pl.}
\end{itemize}
\begin{itemize}
\item {Proveniência:(Do gr. \textunderscore murias\textunderscore  + \textunderscore pous\textunderscore , \textunderscore podos\textunderscore )}
\end{itemize}
Que tem muitos pés.
Classe de insectos ápteros, que se distinguem por grande número de pés.
\section{Miriare}
\begin{itemize}
\item {Grp. gram.:m.}
\end{itemize}
\begin{itemize}
\item {Proveniência:(De \textunderscore míria...\textunderscore  + \textunderscore are\textunderscore )}
\end{itemize}
Extensão de déz mil ares ou de um quilómetro quadrado.
\section{Mírica}
\begin{itemize}
\item {Grp. gram.:m.}
\end{itemize}
\begin{itemize}
\item {Proveniência:(Do gr. \textunderscore murike\textunderscore )}
\end{itemize}
Gênero único de plantas dicotiledóneas da Ásia e da América, também conhecido por \textunderscore tamargueira\textunderscore .
\section{Miricáceas}
\begin{itemize}
\item {Grp. gram.:f. pl.}
\end{itemize}
Família de plantas, que tem por tipo o mírica.
\section{Miringite}
\begin{itemize}
\item {Grp. gram.:f.}
\end{itemize}
\begin{itemize}
\item {Utilização:Med.}
\end{itemize}
Inchação no ouvido.
\section{Miriodésmeas}
\begin{itemize}
\item {Grp. gram.:f. pl.}
\end{itemize}
\begin{itemize}
\item {Utilização:Bot.}
\end{itemize}
\begin{itemize}
\item {Proveniência:(Do gr. \textunderscore murioi\textunderscore  + \textunderscore desme\textunderscore )}
\end{itemize}
Tríbo de algas fucáceas.
\section{Mirógono}
\begin{itemize}
\item {Grp. gram.:m.}
\end{itemize}
\begin{itemize}
\item {Proveniência:(Do gr. \textunderscore murias\textunderscore  + \textunderscore gonos\textunderscore )}
\end{itemize}
Suposto poligono de déz mil lados.
\section{Mirioftalmo}
\begin{itemize}
\item {Grp. gram.:adj.}
\end{itemize}
\begin{itemize}
\item {Utilização:Zool.}
\end{itemize}
\begin{itemize}
\item {Proveniência:(Do gr. \textunderscore murias\textunderscore  + \textunderscore ophtalmos\textunderscore )}
\end{itemize}
Que tem grande quantidade de olhos.
\section{Miriofilo}
\begin{itemize}
\item {Grp. gram.:m.}
\end{itemize}
\begin{itemize}
\item {Proveniência:(Do gr. \textunderscore murias\textunderscore  + \textunderscore phullon\textunderscore )}
\end{itemize}
Gênero de plantas halorágeas.
\section{Miristicáceas}
\begin{itemize}
\item {Grp. gram.:f. pl.}
\end{itemize}
\begin{itemize}
\item {Proveniência:(Do gr. \textunderscore muristikos\textunderscore )}
\end{itemize}
Família de plantas, que tem por tipo a moscadeira.
\section{Mírmecas}
\begin{itemize}
\item {Grp. gram.:m. pl.}
\end{itemize}
\begin{itemize}
\item {Proveniência:(Do gr. \textunderscore murmex\textunderscore )}
\end{itemize}
Insectos himenópteros, com antennas cerdosas, abdome redondo e aguilhão.
\section{Mirmecófago}
\begin{itemize}
\item {Grp. gram.:adj.}
\end{itemize}
\begin{itemize}
\item {Utilização:Zool.}
\end{itemize}
\begin{itemize}
\item {Proveniência:(Do gr. \textunderscore murmex\textunderscore  + \textunderscore phagein\textunderscore )}
\end{itemize}
Que se alimenta de formigas.
\section{Mirmecologia}
\begin{itemize}
\item {Grp. gram.:f.}
\end{itemize}
\begin{itemize}
\item {Proveniência:(Do gr. \textunderscore murmex\textunderscore  + \textunderscore logos\textunderscore )}
\end{itemize}
Tratado á cêrca das formigas.
\section{Mirmecológico}
\begin{itemize}
\item {Grp. gram.:adj.}
\end{itemize}
Relativo á mirmecologia.
\section{Mirmeleão}
\begin{itemize}
\item {Grp. gram.:m.}
\end{itemize}
\begin{itemize}
\item {Proveniência:(Do gr. \textunderscore murmex\textunderscore  + \textunderscore leon\textunderscore )}
\end{itemize}
Gênero de insectos neurópteros.
\section{Mirmeleoniano}
\begin{itemize}
\item {Grp. gram.:adj.}
\end{itemize}
\begin{itemize}
\item {Proveniência:(De \textunderscore mirmeleão\textunderscore )}
\end{itemize}
Diz-se de certos insectos neurópteros.
\section{Mironato}
\begin{itemize}
\item {Grp. gram.:m.}
\end{itemize}
Gênero de saes, formados pelo ácido mirónico com uma base.
(Cp. \textunderscore mirónico\textunderscore )
\section{Mirónico}
\begin{itemize}
\item {Grp. gram.:adj.}
\end{itemize}
\begin{itemize}
\item {Proveniência:(Do gr. \textunderscore muron\textunderscore , perfume)}
\end{itemize}
Diz-se de um ácido cristalizável, que é um dos princípios da mostarda.
\section{Mirosina}
\begin{itemize}
\item {Grp. gram.:f.}
\end{itemize}
\begin{itemize}
\item {Proveniência:(Do gr. \textunderscore muron\textunderscore )}
\end{itemize}
Substância albuminóide, que produz a essência de mostarda preta.
\section{Mirospermina}
\begin{itemize}
\item {Grp. gram.:f.}
\end{itemize}
Essência, extraida de mirospermo.
\section{Mirospermo}
\begin{itemize}
\item {Grp. gram.:m.}
\end{itemize}
\begin{itemize}
\item {Proveniência:(Do gr. \textunderscore muron\textunderscore  + \textunderscore sperma\textunderscore )}
\end{itemize}
Árvore leguminosa, que produz o bálsamo do Peru.
\section{Miroxilina}
\begin{itemize}
\item {Grp. gram.:f.}
\end{itemize}
\begin{itemize}
\item {Proveniência:(De \textunderscore miroxilo\textunderscore )}
\end{itemize}
Substância insolúvel, contida na essência do bálsamo do Peru.
\section{Miróxilo}
\begin{itemize}
\item {fónica:csi}
\end{itemize}
\begin{itemize}
\item {Grp. gram.:m.}
\end{itemize}
\begin{itemize}
\item {Proveniência:(Do gr. \textunderscore muron\textunderscore  + \textunderscore xulon\textunderscore )}
\end{itemize}
Gênero de plantas leguminosas, a que pertencem as árvores que dão os bálsamos do Peru e de Tolu.
\section{Myóparo}
\begin{itemize}
\item {Grp. gram.:m.}
\end{itemize}
\begin{itemize}
\item {Proveniência:(Gr. \textunderscore muoparon\textunderscore )}
\end{itemize}
Espécie de navio antigo e ligeiro, usado por piratas.
\section{Myopas}
\begin{itemize}
\item {Grp. gram.:f. pl.}
\end{itemize}
\begin{itemize}
\item {Proveniência:(Do gr. \textunderscore muia\textunderscore  + \textunderscore ops\textunderscore )}
\end{itemize}
Gênero de insectos dípteros.
\section{Myopathia}
\begin{itemize}
\item {Grp. gram.:f.}
\end{itemize}
\begin{itemize}
\item {Utilização:Med.}
\end{itemize}
\begin{itemize}
\item {Proveniência:(Do gr. \textunderscore mus\textunderscore , \textunderscore muos\textunderscore  + \textunderscore pathos\textunderscore )}
\end{itemize}
Qualquer affecção muscular.
\section{Mýope}
\begin{itemize}
\item {Grp. gram.:m.  e  f.}
\end{itemize}
\begin{itemize}
\item {Utilização:Fig.}
\end{itemize}
\begin{itemize}
\item {Grp. gram.:Adj.}
\end{itemize}
\begin{itemize}
\item {Proveniência:(Do gr. \textunderscore muops\textunderscore )}
\end{itemize}
Pessôa, que tem a vista muito curta, ou que soffre myopia.
Pessôa pouco intelligente ou perspicaz.
Que soffre myopia.--A pronúncia exacta é \textunderscore miópe\textunderscore , mas não se usa.
\section{Myopia}
\begin{itemize}
\item {Grp. gram.:f.}
\end{itemize}
\begin{itemize}
\item {Utilização:Fig.}
\end{itemize}
\begin{itemize}
\item {Proveniência:(Do gr. \textunderscore muopia\textunderscore )}
\end{itemize}
Imperfeição da vista, que só permitte vêr os objectos a pequena distância do ôlho.
Vista curta.
Falta de perspicácia.
\section{Myoplasma}
\begin{itemize}
\item {Grp. gram.:m.}
\end{itemize}
\begin{itemize}
\item {Proveniência:(Do gr. \textunderscore mus\textunderscore  + \textunderscore plassein\textunderscore )}
\end{itemize}
Plasma muscular.
\section{Myoporíneas}
\begin{itemize}
\item {Grp. gram.:f. pl.}
\end{itemize}
\begin{itemize}
\item {Proveniência:(De \textunderscore myoporíneo\textunderscore )}
\end{itemize}
Família de plantas, que têm por typo o myóporo.
\section{Myoporíneo}
\begin{itemize}
\item {Grp. gram.:adj.}
\end{itemize}
Relativo ou semelhante ao myóporo.
\section{Myóporo}
\begin{itemize}
\item {Grp. gram.:m.}
\end{itemize}
\begin{itemize}
\item {Proveniência:(Do gr. \textunderscore muia\textunderscore  + \textunderscore poros\textunderscore )}
\end{itemize}
Gênero de arbustos, originários da Nova-Holanda, (\textunderscore myoporum acuminatum\textunderscore , Brown.).
\section{Myopresbyta}
\begin{itemize}
\item {Grp. gram.:m.  e  adj.}
\end{itemize}
\begin{itemize}
\item {Utilização:Med.}
\end{itemize}
\begin{itemize}
\item {Proveniência:(De \textunderscore mýope\textunderscore  + \textunderscore presbyta\textunderscore )}
\end{itemize}
Diz-se do indivíduo, que é mýope de um ôlho e presbyta do outro.
\section{Myose}
\begin{itemize}
\item {Grp. gram.:f.}
\end{itemize}
\begin{itemize}
\item {Proveniência:(Do gr. \textunderscore muein\textunderscore )}
\end{itemize}
Contracção permanente da pupilla.
\section{Myosite}
\begin{itemize}
\item {Grp. gram.:f.}
\end{itemize}
\begin{itemize}
\item {Utilização:Med.}
\end{itemize}
\begin{itemize}
\item {Proveniência:(Do gr. \textunderscore mus\textunderscore , \textunderscore muos\textunderscore )}
\end{itemize}
Inflammação dos músculos.
\section{Myosota}
\begin{itemize}
\item {Grp. gram.:f.}
\end{itemize}
\begin{itemize}
\item {Proveniência:(Lat. \textunderscore myosota\textunderscore )}
\end{itemize}
Gênero de plantas borragíneas, cuja espécie principal, de flôres miúdas e azues, é o \textunderscore forget-me-not\textunderscore  dos inglêses, que corresponde a \textunderscore não-te-esqueças\textunderscore .
\section{Myosótis}
\begin{itemize}
\item {Grp. gram.:f.}
\end{itemize}
(V.myosota)
\section{Myosuro}
\begin{itemize}
\item {Grp. gram.:m.}
\end{itemize}
\begin{itemize}
\item {Proveniência:(Do gr. \textunderscore muos\textunderscore  + \textunderscore oura\textunderscore )}
\end{itemize}
Planta ranunculácea.
\section{Myospasia}
\begin{itemize}
\item {Grp. gram.:f.}
\end{itemize}
\begin{itemize}
\item {Utilização:Med.}
\end{itemize}
\begin{itemize}
\item {Proveniência:(Do gr. \textunderscore mus\textunderscore  + \textunderscore spasis\textunderscore )}
\end{itemize}
Qualquer moléstia nervosa, que se traduz em espasmos.
\section{Myótico}
\begin{itemize}
\item {Grp. gram.:adj.}
\end{itemize}
Relativo á myose.
\section{Myotomia}
\begin{itemize}
\item {Grp. gram.:f.}
\end{itemize}
\begin{itemize}
\item {Proveniência:(Do gr. \textunderscore mus\textunderscore  + \textunderscore tome\textunderscore )}
\end{itemize}
Operação, que tem por objecto a dissecção dos músculos.
\section{Myotómico}
\begin{itemize}
\item {Grp. gram.:adj.}
\end{itemize}
Relativo á myotomia.
\section{Myótomo}
\begin{itemize}
\item {Grp. gram.:m.}
\end{itemize}
\begin{itemize}
\item {Proveniência:(Do gr. \textunderscore mus\textunderscore , \textunderscore muos\textunderscore  + \textunderscore tome\textunderscore )}
\end{itemize}
Instrumento cirúrgico, para fazer incisão num músculo debaixo da conjunctiva.
\section{Mýria...}
\begin{itemize}
\item {Grp. gram.:f. pref.}
\end{itemize}
(designativo de \textunderscore déz vezes mil\textunderscore )
(Do. gr. \textunderscore murioi\textunderscore )
\section{Myríade}
\begin{itemize}
\item {Grp. gram.:f.}
\end{itemize}
\begin{itemize}
\item {Utilização:Fig.}
\end{itemize}
\begin{itemize}
\item {Proveniência:(Gr. \textunderscore murias\textunderscore )}
\end{itemize}
Número de déz mil.
Grande quantidade, quantidade indefinida.
\section{Myriagramma}
\begin{itemize}
\item {Grp. gram.:m.}
\end{itemize}
\begin{itemize}
\item {Proveniência:(De \textunderscore mýria...\textunderscore  + \textunderscore gramma\textunderscore )}
\end{itemize}
Pêso de déz mil grammas.
\section{Myrialitro}
\begin{itemize}
\item {Grp. gram.:m.}
\end{itemize}
\begin{itemize}
\item {Proveniência:(De \textunderscore mýria...\textunderscore  + \textunderscore litro\textunderscore )}
\end{itemize}
Número de déz mil litros.
\section{Myriâmetro}
\begin{itemize}
\item {Grp. gram.:m.}
\end{itemize}
\begin{itemize}
\item {Proveniência:(De \textunderscore mýria...\textunderscore  + \textunderscore metro\textunderscore )}
\end{itemize}
Medida itinerária, de déz mil metros.
\section{Myriana}
\begin{itemize}
\item {Grp. gram.:f.}
\end{itemize}
Animal radiário das grandes profundidades do Oceano.
\section{Myriantho}
\begin{itemize}
\item {Grp. gram.:m.}
\end{itemize}
\begin{itemize}
\item {Proveniência:(Do gr. \textunderscore murias\textunderscore  + \textunderscore anthos\textunderscore )}
\end{itemize}
Árvore cucurbitácea da África.
\section{Myriápode}
\begin{itemize}
\item {Grp. gram.:adj.}
\end{itemize}
\begin{itemize}
\item {Grp. gram.:M. pl.}
\end{itemize}
\begin{itemize}
\item {Proveniência:(Do gr. \textunderscore murias\textunderscore  + \textunderscore pous\textunderscore , \textunderscore podos\textunderscore )}
\end{itemize}
Que tem muitos pés.
Classe de insectos ápteros, que se distinguem por grande número de pés.
\section{Myriare}
\begin{itemize}
\item {Grp. gram.:m.}
\end{itemize}
\begin{itemize}
\item {Proveniência:(De \textunderscore mýria...\textunderscore  + \textunderscore are\textunderscore )}
\end{itemize}
Extensão de déz mil ares ou de um quilómetro quadrado.
\section{Mýrica}
\begin{itemize}
\item {Grp. gram.:m.}
\end{itemize}
\begin{itemize}
\item {Proveniência:(Do gr. \textunderscore murike\textunderscore )}
\end{itemize}
Gênero único de plantas dicotyledóneas da Ásia e da América, também conhecido por \textunderscore tamargueira\textunderscore .
\section{Myricáceas}
\begin{itemize}
\item {Grp. gram.:f. pl.}
\end{itemize}
Família de plantas, que tem por typo o mýrica.
\section{Myringite}
\begin{itemize}
\item {Grp. gram.:f.}
\end{itemize}
\begin{itemize}
\item {Utilização:Med.}
\end{itemize}
Inchação no ouvido.
\section{Myriodésmeas}
\begin{itemize}
\item {Grp. gram.:f. pl.}
\end{itemize}
\begin{itemize}
\item {Utilização:Bot.}
\end{itemize}
\begin{itemize}
\item {Proveniência:(Do gr. \textunderscore murioi\textunderscore  + \textunderscore desme\textunderscore )}
\end{itemize}
Tríbo de algas fucáceas.
\section{Myriógono}
\begin{itemize}
\item {Grp. gram.:m.}
\end{itemize}
\begin{itemize}
\item {Proveniência:(Do gr. \textunderscore murias\textunderscore  + \textunderscore gonos\textunderscore )}
\end{itemize}
Supposto polygono de déz mil lados.
\section{Myriophtalmo}
\begin{itemize}
\item {Grp. gram.:adj.}
\end{itemize}
\begin{itemize}
\item {Utilização:Zool.}
\end{itemize}
\begin{itemize}
\item {Proveniência:(Do gr. \textunderscore murias\textunderscore  + \textunderscore ophtalmos\textunderscore )}
\end{itemize}
Que tem grande quantidade de olhos.
\section{Myriophyllo}
\begin{itemize}
\item {Grp. gram.:m.}
\end{itemize}
\begin{itemize}
\item {Proveniência:(Do gr. \textunderscore murias\textunderscore  + \textunderscore phullon\textunderscore )}
\end{itemize}
Gênero de plantas halorágeas.
\section{Myriópode}
\begin{itemize}
\item {Grp. gram.:adj.}
\end{itemize}
O mesmo que \textunderscore myriápode\textunderscore , e orthogr. mais correcta.
\section{Myristicáceas}
\begin{itemize}
\item {Grp. gram.:f. pl.}
\end{itemize}
\begin{itemize}
\item {Proveniência:(Do gr. \textunderscore muristikos\textunderscore )}
\end{itemize}
Família de plantas, que tem por typo a moscadeira.
\section{Mýrmecas}
\begin{itemize}
\item {Grp. gram.:m. pl.}
\end{itemize}
\begin{itemize}
\item {Proveniência:(Do gr. \textunderscore murmex\textunderscore )}
\end{itemize}
Insectos hymenópteros, com antennas cerdosas, abdome redondo e aguilhão.
\section{Myrmécio}
\begin{itemize}
\item {Grp. gram.:m.}
\end{itemize}
\begin{itemize}
\item {Utilização:Med.}
\end{itemize}
\begin{itemize}
\item {Proveniência:(Gr. \textunderscore murmekion\textunderscore )}
\end{itemize}
Espécie de verruga, que apparece principalmente na palma da mão e na planta do pé.
\section{Myrmecologia}
\begin{itemize}
\item {Grp. gram.:f.}
\end{itemize}
\begin{itemize}
\item {Proveniência:(Do gr. \textunderscore murmex\textunderscore  + \textunderscore logos\textunderscore )}
\end{itemize}
Tratado á cêrca das formigas.
\section{Myrmecológico}
\begin{itemize}
\item {Grp. gram.:adj.}
\end{itemize}
Relativo á myrmecologia.
\section{Myrmecóphago}
\begin{itemize}
\item {Grp. gram.:adj.}
\end{itemize}
\begin{itemize}
\item {Utilização:Zool.}
\end{itemize}
\begin{itemize}
\item {Proveniência:(Do gr. \textunderscore murmex\textunderscore  + \textunderscore phagein\textunderscore )}
\end{itemize}
Que se alimenta de formigas.
\section{Myrmeleão}
\begin{itemize}
\item {Grp. gram.:m.}
\end{itemize}
\begin{itemize}
\item {Proveniência:(Do gr. \textunderscore murmex\textunderscore  + \textunderscore leon\textunderscore )}
\end{itemize}
Gênero de insectos neurópteros.
\section{Myrmeleoniano}
\begin{itemize}
\item {Grp. gram.:adj.}
\end{itemize}
\begin{itemize}
\item {Proveniência:(De \textunderscore myrmeleão\textunderscore )}
\end{itemize}
Diz-se de certos insectos neurópteros.
\section{Myrobálano}
\begin{itemize}
\item {Grp. gram.:m.}
\end{itemize}
\begin{itemize}
\item {Proveniência:(Lat. \textunderscore myrobalanum\textunderscore )}
\end{itemize}
Designação genérica de vários frutos sêcos, procedentes da Índia, e que se applicavam unicamente em preparações pharmacêuticas.
Cp. \textunderscore mirabólano\textunderscore  e \textunderscore mirobálano\textunderscore .
\section{Myronato}
\begin{itemize}
\item {Grp. gram.:m.}
\end{itemize}
Gênero de saes, formados pelo ácido myrónico com uma base.
(Cp. \textunderscore myrónico\textunderscore )
\section{Myrónico}
\begin{itemize}
\item {Grp. gram.:adj.}
\end{itemize}
\begin{itemize}
\item {Proveniência:(Do gr. \textunderscore muron\textunderscore , perfume)}
\end{itemize}
Diz-se de um ácido crystallizável, que é um dos princípios da mostarda.
\section{Myrosina}
\begin{itemize}
\item {Grp. gram.:f.}
\end{itemize}
\begin{itemize}
\item {Proveniência:(Do gr. \textunderscore muron\textunderscore )}
\end{itemize}
Substância albuminóide, que produz a essência de mostarda preta.
\section{Mýrospermina}
\begin{itemize}
\item {Grp. gram.:f.}
\end{itemize}
Essência, extrahida de mirospermo.
\section{Myrospermo}
\begin{itemize}
\item {Grp. gram.:m.}
\end{itemize}
\begin{itemize}
\item {Proveniência:(Do gr. \textunderscore muron\textunderscore  + \textunderscore sperma\textunderscore )}
\end{itemize}
Árvore leguminosa, que produz o bálsamo do Peru.
\section{Myroxilina}
\begin{itemize}
\item {fónica:csi}
\end{itemize}
\begin{itemize}
\item {Grp. gram.:f.}
\end{itemize}
\begin{itemize}
\item {Proveniência:(De \textunderscore myroxylo\textunderscore )}
\end{itemize}
Substância insolúvel, contida na essência do bálsamo do Peru.
\section{Myróxilo}
\begin{itemize}
\item {fónica:csi}
\end{itemize}
\begin{itemize}
\item {Grp. gram.:m.}
\end{itemize}
\begin{itemize}
\item {Proveniência:(Do gr. \textunderscore muron\textunderscore  + \textunderscore xulon\textunderscore )}
\end{itemize}
Gênero de plantas leguminosas, a que pertencem as árvores que dão os bálsamos do Peru e de Tolu.
\section{Myrrha}
\begin{itemize}
\item {Grp. gram.:f.}
\end{itemize}
\begin{itemize}
\item {Proveniência:(Lat. \textunderscore myrrha\textunderscore )}
\end{itemize}
Planta e resina, o mesmo que \textunderscore mirra\textunderscore ^1.
Planta terebinthácea das cercanias do Mar-Vermelho.
Goma resinosa desta planta.
\section{Mirrínio}
\begin{itemize}
\item {Grp. gram.:m.}
\end{itemize}
Gênero de plantas mirtáceas.
\section{Mirrite}
\begin{itemize}
\item {Grp. gram.:f.}
\end{itemize}
Ágata amarela.
\section{Mirsina}
\begin{itemize}
\item {Grp. gram.:f.}
\end{itemize}
\begin{itemize}
\item {Proveniência:(Do gr. \textunderscore mursine\textunderscore )}
\end{itemize}
Gênero de plantas tropicaes.
\section{Mirsineáceas}
\begin{itemize}
\item {Grp. gram.:f. pl.}
\end{itemize}
\begin{itemize}
\item {Proveniência:(De \textunderscore mirsineáceo\textunderscore )}
\end{itemize}
Fam. de plantas, que tem por tipo a mirsina.
\section{Mirsineáceo}
\begin{itemize}
\item {Grp. gram.:adj.}
\end{itemize}
Relativo ou semelhante á mirsina.
\section{Mirsíneas}
\begin{itemize}
\item {Grp. gram.:f. pl.}
\end{itemize}
O mesmo que \textunderscore mirsineáceas\textunderscore .
\section{Mirsifilo}
\begin{itemize}
\item {Grp. gram.:m.}
\end{itemize}
\begin{itemize}
\item {Proveniência:(Do gr. \textunderscore mursine\textunderscore  + \textunderscore phullon\textunderscore )}
\end{itemize}
Gênero de plantas liliáceas.
\section{Mirtáceas}
\begin{itemize}
\item {Grp. gram.:f. pl.}
\end{itemize}
\begin{itemize}
\item {Proveniência:(De \textunderscore mirto\textunderscore )}
\end{itemize}
Família de plantas, que tem por tipo a murta.
\section{Mírteas}
\begin{itemize}
\item {Grp. gram.:f. pl.}
\end{itemize}
\begin{itemize}
\item {Utilização:Bot.}
\end{itemize}
\begin{itemize}
\item {Proveniência:(De \textunderscore mírteo\textunderscore )}
\end{itemize}
Tríbo de mirtáceas.
\section{Mirtedo}
\begin{itemize}
\item {fónica:tê}
\end{itemize}
\begin{itemize}
\item {Grp. gram.:m.}
\end{itemize}
\begin{itemize}
\item {Proveniência:(Do lat. \textunderscore myrtetum\textunderscore )}
\end{itemize}
Lugar, onde crescem mirtos. Cf. Castilho, \textunderscore Geórg.\textunderscore , 123.
\section{Mírteo}
\begin{itemize}
\item {Grp. gram.:adj.}
\end{itemize}
\begin{itemize}
\item {Proveniência:(Lat. \textunderscore myrteus\textunderscore )}
\end{itemize}
Relativo a murta ou a mirto.
Feito de murta.
Em que cresce a murta.
\section{Mirtifloras}
\begin{itemize}
\item {Grp. gram.:f. pl.}
\end{itemize}
\begin{itemize}
\item {Proveniência:(Do lat. \textunderscore myrtus\textunderscore  + \textunderscore flos\textunderscore , \textunderscore floris\textunderscore .)}
\end{itemize}
Ordem de plantas, que compreende as mirtáceas e as punicáceas.
\section{Mirtiforme}
\begin{itemize}
\item {Grp. gram.:adj.}
\end{itemize}
\begin{itemize}
\item {Proveniência:(De \textunderscore myrto\textunderscore  + \textunderscore fórma\textunderscore )}
\end{itemize}
Semelhante á folha do mirto.
\section{Mirtilo}
\begin{itemize}
\item {Grp. gram.:m.}
\end{itemize}
Designação específica de uma espécie de murtinho, (\textunderscore vaccinium myrtillus\textunderscore ).
\section{Mirtíneas}
\begin{itemize}
\item {Grp. gram.:f. pl.}
\end{itemize}
(V.mirtáceas)
\section{Mirto}
\begin{itemize}
\item {Grp. gram.:m.}
\end{itemize}
\begin{itemize}
\item {Proveniência:(Lat. \textunderscore myrtus\textunderscore )}
\end{itemize}
O mesmo que \textunderscore murta\textunderscore .
\section{Mirtóide}
\begin{itemize}
\item {Grp. gram.:adj.}
\end{itemize}
O mesmo que \textunderscore mirtoídeo\textunderscore .
\section{Mirtoídeo}
\begin{itemize}
\item {Grp. gram.:adj.}
\end{itemize}
\begin{itemize}
\item {Proveniência:(Do gr. \textunderscore murtos\textunderscore  + \textunderscore eidos\textunderscore )}
\end{itemize}
Semelhante ao mirto.
\section{Mirtoso}
\begin{itemize}
\item {Grp. gram.:adj.}
\end{itemize}
\begin{itemize}
\item {Proveniência:(Lat. \textunderscore myrtuosus\textunderscore )}
\end{itemize}
Que tem mirto; em que há mirto.
\section{Mista}
\begin{itemize}
\item {Grp. gram.:m.}
\end{itemize}
\begin{itemize}
\item {Proveniência:(Lat. \textunderscore mysta\textunderscore )}
\end{itemize}
Noviço, no antigo culto de Ceres.
Sacerdote, iniciado nos pequenos mistérios de Elêusis.
\section{Mistagogia}
\begin{itemize}
\item {Grp. gram.:f.}
\end{itemize}
Iniciação nos mistérios de uma religião.
(Cp. \textunderscore mistagogo\textunderscore )
\section{Mistagogo}
\begin{itemize}
\item {fónica:gô}
\end{itemize}
\begin{itemize}
\item {Grp. gram.:m.}
\end{itemize}
\begin{itemize}
\item {Utilização:Ext.}
\end{itemize}
\begin{itemize}
\item {Proveniência:(Do gr. \textunderscore mustes\textunderscore  + \textunderscore aqein\textunderscore )}
\end{itemize}
Sacerdote, que iniciava nos mistérios da religião.
Iniciador, mentor.
\section{Mistério}
\begin{itemize}
\item {Grp. gram.:m.}
\end{itemize}
\begin{itemize}
\item {Grp. gram.:Pl.}
\end{itemize}
\begin{itemize}
\item {Utilização:Açor}
\end{itemize}
\begin{itemize}
\item {Proveniência:(Lat. \textunderscore mysterium\textunderscore )}
\end{itemize}
Culto secreto no politeísmo.
Tudo que na religião cristan se apresenta como objecto de fé, e que é impenetrável á razão humana.
Tudo que tem causa oculta.
Aquilo que é incompreensivel.
Segrêdo.
Reserva, cautela.
Composição dramática na Idade-média, cujo assumpto era tirado da sagrada escriptura ou da vida dos santos.
Lavas a descoberto, que tornam pedregoso o solo.
\section{Misteriosamente}
\begin{itemize}
\item {Grp. gram.:adv.}
\end{itemize}
De modo misterioso.
Em segrêdo.
\section{Misterioso}
\begin{itemize}
\item {Grp. gram.:adj.}
\end{itemize}
\begin{itemize}
\item {Grp. gram.:M.}
\end{itemize}
\begin{itemize}
\item {Proveniência:(De \textunderscore mistério\textunderscore )}
\end{itemize}
Em que há mistério; inexplicável.
Que tem modos enigmáticos, que faz segrêdo de coisas insignificantes.
Qualidade do que é misterioso.
\section{Mística}
\begin{itemize}
\item {Grp. gram.:f.}
\end{itemize}
Estudo das coisas divinas ou espirituaes.
(Cp. \textunderscore místico\textunderscore ^1)
\section{Misticamente}
\begin{itemize}
\item {Grp. gram.:adv.}
\end{itemize}
De modo místico.
\section{Misticidade}
\begin{itemize}
\item {Grp. gram.:f.}
\end{itemize}
Qualidade de místico; misticismo.
\section{Misticismo}
\begin{itemize}
\item {Grp. gram.:m.}
\end{itemize}
\begin{itemize}
\item {Proveniência:(De \textunderscore místico\textunderscore )}
\end{itemize}
Crença religiosa ou filosófica, que admite comunicação occulta entre o homem e a divindade.
Devoção religiosa.
Tendência, para acreditar no sobrenatural.
\section{Místico}
\begin{itemize}
\item {Grp. gram.:adj.}
\end{itemize}
\begin{itemize}
\item {Utilização:Pop.}
\end{itemize}
\begin{itemize}
\item {Grp. gram.:M.}
\end{itemize}
\begin{itemize}
\item {Proveniência:(Gr. \textunderscore mustikos\textunderscore )}
\end{itemize}
Que contém o carácter de alegoria, (falando-se de coisas religiosas).
Misterioso.
Relativo á vida espiritual.
Relativo á devoção religiosa; devoto: \textunderscore vida mística\textunderscore .
Bem feito.
Saboroso.
Catita, peralta.
Aquele que professa o misticismo.
Aquele que é muito devoto.
Aquele que escreve á cêrca do misticismo.
\section{Mistificação}
\begin{itemize}
\item {Grp. gram.:f.}
\end{itemize}
Acto ou efeito de mistificar.
\section{Mistificado}
\begin{itemize}
\item {Grp. gram.:adj.}
\end{itemize}
\begin{itemize}
\item {Proveniência:(De \textunderscore mistificar\textunderscore )}
\end{itemize}
Iludido; burlado.
\section{Mistificador}
\begin{itemize}
\item {Grp. gram.:m.  e  adj.}
\end{itemize}
O que mistifica.
\section{Mistificante}
\begin{itemize}
\item {Grp. gram.:adj.}
\end{itemize}
Que mistifica.
\section{Mistificar}
\begin{itemize}
\item {Grp. gram.:v. t.}
\end{itemize}
\begin{itemize}
\item {Proveniência:(Do fr. \textunderscore mystifier\textunderscore )}
\end{itemize}
Abusar da credulidade de.
Burlar, lograr.
\section{Mistro}
\begin{itemize}
\item {Grp. gram.:m.}
\end{itemize}
\begin{itemize}
\item {Proveniência:(Lat. \textunderscore mystrum\textunderscore )}
\end{itemize}
Medida para líquidos, entre os antigos Gregos, que também tinham uma medida agrária do mesmo nome.
\section{Mitacismo}
\begin{itemize}
\item {Grp. gram.:m.}
\end{itemize}
O mesmo ou melhor que \textunderscore mutacismo\textunderscore .
\section{Mitaes}
\begin{itemize}
\item {Grp. gram.:m. pl.}
\end{itemize}
\begin{itemize}
\item {Utilização:Ant.}
\end{itemize}
O mesmo que \textunderscore mites\textunderscore .
\section{Mithicamente}
\begin{itemize}
\item {Grp. gram.:adv.}
\end{itemize}
De modo míthico; á maneira de mitho; fabulosamente.
\section{Míthico}
\begin{itemize}
\item {Grp. gram.:adj.}
\end{itemize}
\begin{itemize}
\item {Proveniência:(Lat. \textunderscore mythicus\textunderscore )}
\end{itemize}
Relativo aos mithos ou que é de natureza dêles; fabuloso.
\section{Mithificação}
\begin{itemize}
\item {Grp. gram.:f.}
\end{itemize}
Acto de mithificar.
\section{Mithificar}
\begin{itemize}
\item {Grp. gram.:v. t.}
\end{itemize}
\begin{itemize}
\item {Proveniência:(Do lat. \textunderscore mythos\textunderscore  + \textunderscore facere\textunderscore )}
\end{itemize}
Converter em mitho; tornar míthico. Cf. Th. Braga, \textunderscore Modernas Ideias\textunderscore , I, 366.
\section{Mithismo}
\begin{itemize}
\item {Grp. gram.:m.}
\end{itemize}
\begin{itemize}
\item {Utilização:Neol.}
\end{itemize}
Sciência dos mithos.
\section{Mitho}
\begin{itemize}
\item {Grp. gram.:m.}
\end{itemize}
\begin{itemize}
\item {Utilização:Fig.}
\end{itemize}
\begin{itemize}
\item {Proveniência:(Do lat. \textunderscore mythus\textunderscore )}
\end{itemize}
Passagem ou particularidade da fábula.
Exposição simbólica de um facto; fábula.
Coisa inacreditável, que não tem realidade.
Enigma.
\section{Mithographia}
\begin{itemize}
\item {Grp. gram.:f.}
\end{itemize}
Descripção de mithos.
(Cp. \textunderscore mithógrapho\textunderscore )
\section{Mithográphico}
\begin{itemize}
\item {Grp. gram.:adj.}
\end{itemize}
Relativo á mithographia.
\section{Mithógrapho}
\begin{itemize}
\item {Grp. gram.:m.}
\end{itemize}
\begin{itemize}
\item {Proveniência:(Do gr. \textunderscore muthos\textunderscore  + \textunderscore graphein\textunderscore )}
\end{itemize}
Aquele que escreve á cêrca dos mithos.
\section{Mithologia}
\begin{itemize}
\item {Grp. gram.:f.}
\end{itemize}
História das divindades do paganismo.
Conjunto de fábulas.
Explicação dos mithos.
(Cp. \textunderscore mithólogo\textunderscore )
\section{Mithologicamente}
\begin{itemize}
\item {Grp. gram.:adv.}
\end{itemize}
De modo mithológico.
\section{Mithológico}
\begin{itemize}
\item {Grp. gram.:adj.}
\end{itemize}
Relativo á Mithologia.
\section{Mithologismo}
\begin{itemize}
\item {Grp. gram.:m.}
\end{itemize}
O mesmo que \textunderscore mithismo\textunderscore .
\section{Mithologista}
\begin{itemize}
\item {Grp. gram.:m.}
\end{itemize}
\begin{itemize}
\item {Proveniência:(Do gr. \textunderscore mutos\textunderscore  + \textunderscore logos\textunderscore )}
\end{itemize}
Aquele que é versado em Mithologia ou escreve a respeito dela.
\section{Mithólogo}
\begin{itemize}
\item {Grp. gram.:m.}
\end{itemize}
\begin{itemize}
\item {Proveniência:(Do gr. \textunderscore mutos\textunderscore  + \textunderscore logos\textunderscore )}
\end{itemize}
Aquele que é versado em Mithologia ou escreve a respeito dela.
\section{Mitilicultura}
\begin{itemize}
\item {Grp. gram.:f.}
\end{itemize}
\begin{itemize}
\item {Proveniência:(Do lat. \textunderscore mytilus\textunderscore  + \textunderscore cultura\textunderscore )}
\end{itemize}
Arte de criar e multiplicar os mexilhões.
Processo para a fecundação artificial dos mexilhões.
\section{Mitilóides}
\begin{itemize}
\item {Grp. gram.:m. pl.}
\end{itemize}
\begin{itemize}
\item {Proveniência:(Do lat. \textunderscore mytilus\textunderscore  + gr. \textunderscore eidos\textunderscore )}
\end{itemize}
Família de moluscos, que tem por tipo o mexilhão.
\section{Mitilotoxina}
\begin{itemize}
\item {fónica:csi}
\end{itemize}
\begin{itemize}
\item {Grp. gram.:f.}
\end{itemize}
\begin{itemize}
\item {Proveniência:(Do gr. \textunderscore mytilos\textunderscore  + \textunderscore toxikos\textunderscore )}
\end{itemize}
Ptomaína tóxica, que se extrai de certos mexilhões.
\section{Miúro}
\begin{itemize}
\item {Grp. gram.:adj.}
\end{itemize}
\begin{itemize}
\item {Proveniência:(Do gr. \textunderscore mus\textunderscore  + \textunderscore oura\textunderscore )}
\end{itemize}
Diz-se do pulso, que enfraquece progressivamente.
\section{Miva}
\begin{itemize}
\item {Grp. gram.:f.}
\end{itemize}
Preparado farmacêutico, espécie de geleia, em que entra sumo de frutos e suco de carne.
\section{Mixa}
\begin{itemize}
\item {fónica:csa}
\end{itemize}
\begin{itemize}
\item {Grp. gram.:f.}
\end{itemize}
\begin{itemize}
\item {Proveniência:(Do gr. \textunderscore muxa\textunderscore )}
\end{itemize}
Parte superior da mandíbula das aves.
\section{Mixedema}
\begin{itemize}
\item {fónica:cse}
\end{itemize}
\begin{itemize}
\item {Grp. gram.:m.}
\end{itemize}
\begin{itemize}
\item {Utilização:Med.}
\end{itemize}
\begin{itemize}
\item {Proveniência:(Do gr. \textunderscore muxa\textunderscore  + \textunderscore oidema\textunderscore )}
\end{itemize}
Doença nervosa, caracterizada pela tumefacção dos tegumentos, perturbação intelectual e atrofia do corpo tiroídeo.
\section{Mixoma}
\begin{itemize}
\item {fónica:cso}
\end{itemize}
\begin{itemize}
\item {Grp. gram.:m.}
\end{itemize}
\begin{itemize}
\item {Proveniência:(Do gr. \textunderscore muxa\textunderscore )}
\end{itemize}
Tumor do tecido mucoso.
\section{Mixomycetos}
\begin{itemize}
\item {fónica:cso}
\end{itemize}
\begin{itemize}
\item {Grp. gram.:m. pl.}
\end{itemize}
\begin{itemize}
\item {Proveniência:(Do gr. \textunderscore muxa\textunderscore  + \textunderscore mukes\textunderscore )}
\end{itemize}
Nome de uma ordem de cogumelos.
\section{Mizocéphalo}
\begin{itemize}
\item {Grp. gram.:adj.}
\end{itemize}
\begin{itemize}
\item {Utilização:Zool.}
\end{itemize}
\begin{itemize}
\item {Proveniência:(Do gr. \textunderscore muzein\textunderscore  + \textunderscore kephale\textunderscore )}
\end{itemize}
Que tem cabeça em fórma de ventosa ou sugadoiro.
\section{Myrrhínio}
\begin{itemize}
\item {Grp. gram.:m.}
\end{itemize}
Gênero de plantas myrtáceas.
\section{Myrrhite}
\begin{itemize}
\item {Grp. gram.:f.}
\end{itemize}
Ágata amarela.
\section{Myrsina}
\begin{itemize}
\item {Grp. gram.:f.}
\end{itemize}
\begin{itemize}
\item {Proveniência:(Do gr. \textunderscore mursine\textunderscore )}
\end{itemize}
Gênero de plantas tropicaes.
\section{Myrsineáceas}
\begin{itemize}
\item {Grp. gram.:f. pl.}
\end{itemize}
\begin{itemize}
\item {Proveniência:(De \textunderscore myrsineáceo\textunderscore )}
\end{itemize}
Fam. de plantas, que tem por typo a myrsina.
\section{Myrsineáceo}
\begin{itemize}
\item {Grp. gram.:adj.}
\end{itemize}
Relativo ou semelhante á myrsina.
\section{Myrsíneas}
\begin{itemize}
\item {Grp. gram.:f. pl.}
\end{itemize}
O mesmo que \textunderscore myrsineáceas\textunderscore .
\section{Myrsiphyllo}
\begin{itemize}
\item {Grp. gram.:m.}
\end{itemize}
\begin{itemize}
\item {Proveniência:(Do gr. \textunderscore mursine\textunderscore  + \textunderscore phullon\textunderscore )}
\end{itemize}
Gênero de plantas liliáceas.
\section{Myrtáceas}
\begin{itemize}
\item {Grp. gram.:f. pl.}
\end{itemize}
\begin{itemize}
\item {Proveniência:(De \textunderscore myrto\textunderscore )}
\end{itemize}
Família de plantas, que tem por typo a murta.
\section{Mýrteas}
\begin{itemize}
\item {Grp. gram.:f. pl.}
\end{itemize}
\begin{itemize}
\item {Utilização:Bot.}
\end{itemize}
\begin{itemize}
\item {Proveniência:(De \textunderscore mýrteo\textunderscore )}
\end{itemize}
Tríbo de myrtáceas.
\section{Myrtedo}
\begin{itemize}
\item {fónica:tê}
\end{itemize}
\begin{itemize}
\item {Grp. gram.:m.}
\end{itemize}
\begin{itemize}
\item {Proveniência:(Do lat. \textunderscore myrtetum\textunderscore )}
\end{itemize}
Lugar, onde crescem myrtos. Cf. Castilho, \textunderscore Geórg.\textunderscore , 123.
\section{Mýrteo}
\begin{itemize}
\item {Grp. gram.:adj.}
\end{itemize}
\begin{itemize}
\item {Proveniência:(Lat. \textunderscore myrteus\textunderscore )}
\end{itemize}
Relativo a murta ou a myrto.
Feito de murta.
Em que cresce a murta.
\section{Myrtifloras}
\begin{itemize}
\item {Grp. gram.:f. pl.}
\end{itemize}
\begin{itemize}
\item {Proveniência:(Do lat. \textunderscore myrtus\textunderscore  + \textunderscore flos\textunderscore , \textunderscore floris.\textunderscore )}
\end{itemize}
Ordem de plantas, que comprehende as myrtáceas e as punicáceas.
\section{Myrtiforme}
\begin{itemize}
\item {Grp. gram.:adj.}
\end{itemize}
\begin{itemize}
\item {Proveniência:(De \textunderscore myrto\textunderscore  + \textunderscore fórma\textunderscore )}
\end{itemize}
Semelhante á folha do myrto.
\section{Myrtillo}
\begin{itemize}
\item {Grp. gram.:m.}
\end{itemize}
Designação específica de uma espécie de murtinho, (\textunderscore vaccinium myrtillus\textunderscore ).
\section{Myrtíneas}
\begin{itemize}
\item {Grp. gram.:f. pl.}
\end{itemize}
(V.myrtáceas)
\section{Myrto}
\begin{itemize}
\item {Grp. gram.:m.}
\end{itemize}
\begin{itemize}
\item {Proveniência:(Lat. \textunderscore myrtus\textunderscore )}
\end{itemize}
O mesmo que \textunderscore murta\textunderscore .
\section{Myrtóide}
\begin{itemize}
\item {Grp. gram.:adj.}
\end{itemize}
O mesmo que \textunderscore myrtoídeo\textunderscore .
\section{Myrtoídeo}
\begin{itemize}
\item {Grp. gram.:adj.}
\end{itemize}
\begin{itemize}
\item {Proveniência:(Do gr. \textunderscore murtos\textunderscore  + \textunderscore eidos\textunderscore )}
\end{itemize}
Semelhante ao myrto.
\section{Myrtoso}
\begin{itemize}
\item {Grp. gram.:adj.}
\end{itemize}
\begin{itemize}
\item {Proveniência:(Lat. \textunderscore myrtuosus\textunderscore )}
\end{itemize}
Que tem myrto; em que há myrto.
\section{Mysta}
\begin{itemize}
\item {Grp. gram.:m.}
\end{itemize}
\begin{itemize}
\item {Proveniência:(Lat. \textunderscore mysta\textunderscore )}
\end{itemize}
Noviço, no antigo culto de Ceres.
Sacerdote, iniciado nos pequenos mystérios de Elêusis.
\section{Mystagogia}
\begin{itemize}
\item {Grp. gram.:f.}
\end{itemize}
Iniciação nos mystérios de uma religião.
(Cp. \textunderscore mystagogo\textunderscore )
\section{Mystagogo}
\begin{itemize}
\item {fónica:gô}
\end{itemize}
\begin{itemize}
\item {Grp. gram.:m.}
\end{itemize}
\begin{itemize}
\item {Utilização:Ext.}
\end{itemize}
\begin{itemize}
\item {Proveniência:(Do gr. \textunderscore mustes\textunderscore  + \textunderscore aqein\textunderscore )}
\end{itemize}
Sacerdote, que iniciava nos mystérios da religião.
Iniciador, mentor.
\section{Mystério}
\begin{itemize}
\item {Grp. gram.:m.}
\end{itemize}
\begin{itemize}
\item {Grp. gram.:Pl.}
\end{itemize}
\begin{itemize}
\item {Utilização:Açor}
\end{itemize}
\begin{itemize}
\item {Proveniência:(Lat. \textunderscore mysterium\textunderscore )}
\end{itemize}
Culto secreto no polytheísmo.
Tudo que na religião christan se apresenta como objecto de fé, e que é impenetrável á razão humana.
Tudo que tem causa occulta.
Aquillo que é incomprehensivel.
Segrêdo.
Reserva, cautela.
Composição dramática na Idade-média, cujo assumpto era tirado da sagrada escriptura ou da vida dos santos.
Lavas a descoberto, que tornam pedregoso o solo.
\section{Mysteriosamente}
\begin{itemize}
\item {Grp. gram.:adv.}
\end{itemize}
De modo mysterioso.
Em segrêdo.
\section{Mysterioso}
\begin{itemize}
\item {Grp. gram.:adj.}
\end{itemize}
\begin{itemize}
\item {Grp. gram.:M.}
\end{itemize}
\begin{itemize}
\item {Proveniência:(De \textunderscore mystério\textunderscore )}
\end{itemize}
Em que há mystério; inexplicável.
Que tem modos enigmáticos, que faz segrêdo de coisas insignificantes.
Qualidade do que é mysterioso.
\section{Mýstica}
\begin{itemize}
\item {Grp. gram.:f.}
\end{itemize}
Estudo das coisas divinas ou espirituaes.
(Cp. \textunderscore mýstico\textunderscore )
\section{Mysticamente}
\begin{itemize}
\item {Grp. gram.:adv.}
\end{itemize}
De modo mýstico.
\section{Mysticidade}
\begin{itemize}
\item {Grp. gram.:f.}
\end{itemize}
Qualidade de mýstico; mysticismo.
\section{Mysticismo}
\begin{itemize}
\item {Grp. gram.:m.}
\end{itemize}
\begin{itemize}
\item {Proveniência:(De \textunderscore mýstico\textunderscore )}
\end{itemize}
Crença religiosa ou philosóphica, que admitte communicação occulta entre o homem e a divindade.
Devoção religiosa.
Tendência, para acreditar no sobrenatural.
\section{Mýstico}
\begin{itemize}
\item {Grp. gram.:adj.}
\end{itemize}
\begin{itemize}
\item {Utilização:Pop.}
\end{itemize}
\begin{itemize}
\item {Grp. gram.:M.}
\end{itemize}
\begin{itemize}
\item {Proveniência:(Gr. \textunderscore mustikos\textunderscore )}
\end{itemize}
Que contém o carácter de allegoria, (falando-se de coisas religiosas).
Mysterioso.
Relativo á vida espiritual.
Relativo á devoção religiosa; devoto: \textunderscore vida mýstica\textunderscore .
Bem feito.
Saboroso.
Catita, peralta.
Aquelle que professa o mysticismo.
Aquelle que é muito devoto.
Aquelle que escreve á cêrca do mysticismo.
\section{Mystificação}
\begin{itemize}
\item {Grp. gram.:f.}
\end{itemize}
Acto ou effeito de mystificar.
\section{Mystificado}
\begin{itemize}
\item {Grp. gram.:adj.}
\end{itemize}
\begin{itemize}
\item {Proveniência:(De \textunderscore mystificar\textunderscore )}
\end{itemize}
Illudido; burlado.
\section{Mystificador}
\begin{itemize}
\item {Grp. gram.:m.  e  adj.}
\end{itemize}
O que mystifica.
\section{Mystificante}
\begin{itemize}
\item {Grp. gram.:adj.}
\end{itemize}
Que mystifica.
\section{Mystificar}
\begin{itemize}
\item {Grp. gram.:v. t.}
\end{itemize}
\begin{itemize}
\item {Proveniência:(Do fr. \textunderscore mystifier\textunderscore )}
\end{itemize}
Abusar da credulidade de.
Burlar, lograr.
\section{Mystro}
\begin{itemize}
\item {Grp. gram.:m.}
\end{itemize}
\begin{itemize}
\item {Proveniência:(Lat. \textunderscore mystrum\textunderscore )}
\end{itemize}
Medida para líquidos, entre os antigos Gregos, que também tinham uma medida agrária do mesmo nome.
\section{Mytacismo}
\begin{itemize}
\item {Grp. gram.:m.}
\end{itemize}
O mesmo ou melhor que \textunderscore mutacismo\textunderscore .
\section{Mytaes}
\begin{itemize}
\item {Grp. gram.:m. pl.}
\end{itemize}
\begin{itemize}
\item {Utilização:Ant.}
\end{itemize}
O mesmo que \textunderscore mites\textunderscore .
\section{Mythicamente}
\begin{itemize}
\item {Grp. gram.:adv.}
\end{itemize}
De modo mýthico; á maneira de mytho; fabulosamente.
\section{Mýthico}
\begin{itemize}
\item {Grp. gram.:adj.}
\end{itemize}
\begin{itemize}
\item {Proveniência:(Lat. \textunderscore mythicus\textunderscore )}
\end{itemize}
Relativo aos mythos ou que é de natureza dêlles; fabuloso.
\section{Mythificação}
\begin{itemize}
\item {Grp. gram.:f.}
\end{itemize}
Acto de \textunderscore mythificar\textunderscore .
\section{Mythificar}
\begin{itemize}
\item {Grp. gram.:v. t.}
\end{itemize}
\begin{itemize}
\item {Proveniência:(Do lat. \textunderscore mythos\textunderscore  + \textunderscore facere\textunderscore )}
\end{itemize}
Converter em mytho; tornar mýthico. Cf. Th. Braga, \textunderscore Modernas Ideias\textunderscore , I, 366.
\section{Mythismo}
\begin{itemize}
\item {Grp. gram.:m.}
\end{itemize}
\begin{itemize}
\item {Utilização:Neol.}
\end{itemize}
Sciência dos mythos.
\section{Mytho}
\begin{itemize}
\item {Grp. gram.:m.}
\end{itemize}
\begin{itemize}
\item {Utilização:Fig.}
\end{itemize}
\begin{itemize}
\item {Proveniência:(Do lat. \textunderscore mythus\textunderscore )}
\end{itemize}
Passagem ou particularidade da fábula.
Exposição symbólica de um facto; fábula.
Coisa inacreditável, que não tem realidade.
Enigma.
\section{Mythographia}
\begin{itemize}
\item {Grp. gram.:f.}
\end{itemize}
Descripção de mythos.
(Cp. \textunderscore mythógrapho\textunderscore )
\section{Mythográphico}
\begin{itemize}
\item {Grp. gram.:adj.}
\end{itemize}
Relativo á mythographia.
\section{Mythógrapho}
\begin{itemize}
\item {Grp. gram.:m.}
\end{itemize}
\begin{itemize}
\item {Proveniência:(Do gr. \textunderscore muthos\textunderscore  + \textunderscore graphein\textunderscore )}
\end{itemize}
Aquelle que escreve á cêrca dos mythos.
\section{Mythologia}
\begin{itemize}
\item {Grp. gram.:f.}
\end{itemize}
História das divindades do paganismo.
Conjunto de fábulas.
Explicação dos mythos.
(Cp. \textunderscore mythólogo\textunderscore )
\section{Mythologicamente}
\begin{itemize}
\item {Grp. gram.:adv.}
\end{itemize}
De modo mythológico.
\section{Mythológico}
\begin{itemize}
\item {Grp. gram.:adj.}
\end{itemize}
Relativo á Mythologia.
\section{Mythologismo}
\begin{itemize}
\item {Grp. gram.:m.}
\end{itemize}
O mesmo que \textunderscore mythismo\textunderscore .
\section{Mythologista}
\begin{itemize}
\item {Grp. gram.:m.}
\end{itemize}
\begin{itemize}
\item {Proveniência:(Do gr. \textunderscore mutos\textunderscore  + \textunderscore logos\textunderscore )}
\end{itemize}
Aquelle que é versado em Mythologia ou escreve a respeito della.
\section{Mythólogo}
\begin{itemize}
\item {Grp. gram.:m.}
\end{itemize}
\begin{itemize}
\item {Proveniência:(Do gr. \textunderscore mutos\textunderscore  + \textunderscore logos\textunderscore )}
\end{itemize}
Aquelle que é versado em Mythologia ou escreve a respeito della.
\section{Mytilicultura}
\begin{itemize}
\item {Grp. gram.:f.}
\end{itemize}
\begin{itemize}
\item {Proveniência:(Do lat. \textunderscore mytilus\textunderscore  + \textunderscore cultura\textunderscore )}
\end{itemize}
Arte de criar e multiplicar os mexilhões.
Processo para a fecundação artificial dos mexilhões.
\section{Mytilóides}
\begin{itemize}
\item {Grp. gram.:m. pl.}
\end{itemize}
\begin{itemize}
\item {Proveniência:(Do lat. \textunderscore mytilus\textunderscore  + gr. \textunderscore eidos\textunderscore )}
\end{itemize}
Família de molluscos, que tem por typo o mexilhão.
\section{Mytilotoxina}
\begin{itemize}
\item {Grp. gram.:f.}
\end{itemize}
\begin{itemize}
\item {Proveniência:(Do gr. \textunderscore mytilos\textunderscore  + \textunderscore toxikos\textunderscore )}
\end{itemize}
Ptomaína tóxica, que se extrai de certos mexilhões.
\section{Myúro}
\begin{itemize}
\item {Grp. gram.:adj.}
\end{itemize}
\begin{itemize}
\item {Proveniência:(Do gr. \textunderscore mus\textunderscore  + \textunderscore oura\textunderscore )}
\end{itemize}
Diz-se do pulso, que enfraquece progressivamente.
\section{Myva}
\begin{itemize}
\item {Grp. gram.:f.}
\end{itemize}
Preparado pharmacêutico, espécie de geleia, em que entra sumo de frutos e suco de carne.
\section{Myxa}
\begin{itemize}
\item {Grp. gram.:f.}
\end{itemize}
\begin{itemize}
\item {Proveniência:(Do gr. \textunderscore muxa\textunderscore )}
\end{itemize}
Parte superior da mandíbula das aves.
\section{Myxedema}
\begin{itemize}
\item {Grp. gram.:m.}
\end{itemize}
\begin{itemize}
\item {Utilização:Med.}
\end{itemize}
\begin{itemize}
\item {Proveniência:(Do gr. \textunderscore muxa\textunderscore  + \textunderscore oidema\textunderscore )}
\end{itemize}
Doença nervosa, caracterizada pela tumefacção dos tegumentos, perturbação intellectual e atrophia do corpo thyroídeo.
\section{Myxoma}
\begin{itemize}
\item {Grp. gram.:m.}
\end{itemize}
\begin{itemize}
\item {Proveniência:(Do gr. \textunderscore muxa\textunderscore )}
\end{itemize}
Tumor do tecido mucoso.
\section{Myxomycetos}
\begin{itemize}
\item {Grp. gram.:m. pl.}
\end{itemize}
\begin{itemize}
\item {Proveniência:(Do gr. \textunderscore muxa\textunderscore  + \textunderscore mukes\textunderscore )}
\end{itemize}
Nome de uma ordem de cogumelos.
\section{Myzocéphalo}
\begin{itemize}
\item {Grp. gram.:adj.}
\end{itemize}
\begin{itemize}
\item {Utilização:Zool.}
\end{itemize}
\begin{itemize}
\item {Proveniência:(Do gr. \textunderscore muzein\textunderscore  + \textunderscore kephale\textunderscore )}
\end{itemize}
\end{document}