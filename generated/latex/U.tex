\documentclass{article}
\usepackage[portuguese]{babel}
\title{U}
\begin{document}
O mesmo que \textunderscore etrusco\textunderscore .
\section{U}
\begin{itemize}
\item {Grp. gram.:m.}
\end{itemize}
\begin{itemize}
\item {Grp. gram.:Adj.}
\end{itemize}
Vigésima primeira letra do alphabeto português.
Que numa série de 21 occupa o último lugar.
Na escrita antiga, há \textunderscore u\textunderscore  vogal e \textunderscore u\textunderscore  consoante, lendo-se êste como \textunderscore v\textunderscore .
\section{U}
\begin{itemize}
\item {Grp. gram.:adj.}
\end{itemize}
\begin{itemize}
\item {Utilização:Ant.}
\end{itemize}
\begin{itemize}
\item {Proveniência:(Do fr. \textunderscore ou\textunderscore )}
\end{itemize}
O mesmo que \textunderscore onde\textunderscore .
\section{Ua}
\begin{itemize}
\item {Grp. gram.:adj. f.}
\end{itemize}
O mesmo que \textunderscore uma\textunderscore .
\section{Ũa}
\begin{itemize}
\item {Grp. gram.:adj. f.}
\end{itemize}
\begin{itemize}
\item {Utilização:Ant.}
\end{itemize}
O mesmo que \textunderscore uma\textunderscore .
\section{Uaçaçu}
\begin{itemize}
\item {Grp. gram.:m.}
\end{itemize}
\begin{itemize}
\item {Utilização:Bras. do N}
\end{itemize}
Espécie de palmeira.
\section{Uaçaí}
\begin{itemize}
\item {Grp. gram.:m.}
\end{itemize}
\begin{itemize}
\item {Utilização:Bras}
\end{itemize}
Espécie de palmeira.
\section{Uacapu}
\begin{itemize}
\item {Grp. gram.:m.}
\end{itemize}
\begin{itemize}
\item {Utilização:Bras}
\end{itemize}
Árvore silvestre, de madeira duríssima, impenetrável á serra.
\section{Uacapurana}
\begin{itemize}
\item {Grp. gram.:f.}
\end{itemize}
\begin{itemize}
\item {Utilização:Bras}
\end{itemize}
Árvore tinctória das regiões do Amazonas.
\section{Uacarauás}
\begin{itemize}
\item {Grp. gram.:m. pl.}
\end{itemize}
Indígenas do Norte do Brasil.
\section{Uacari}
\begin{itemize}
\item {Grp. gram.:m.}
\end{itemize}
\begin{itemize}
\item {Utilização:Bras}
\end{itemize}
Espécie de macaco.
Peixe do Norte do Brasil.
\section{Uacari-guaçau}
\begin{itemize}
\item {Grp. gram.:m.}
\end{itemize}
Peixe, da fam. dos silúridas, do Brasil.
\section{Uaçu}
\begin{itemize}
\item {Grp. gram.:adj.}
\end{itemize}
O mesmo que \textunderscore guaçu\textunderscore .
\section{Uacuman}
\begin{itemize}
\item {Grp. gram.:m.}
\end{itemize}
\begin{itemize}
\item {Utilização:Bras}
\end{itemize}
\begin{itemize}
\item {Proveniência:(T. tupi)}
\end{itemize}
Espécie de palmeira.
\section{Uacurau}
\begin{itemize}
\item {Grp. gram.:m.}
\end{itemize}
\begin{itemize}
\item {Utilização:Bras}
\end{itemize}
Ave nocturna, das regiões do Amazonas.
\section{Uadadá}
\begin{itemize}
\item {Grp. gram.:m.}
\end{itemize}
Planta tuberculosa do Norte do Brasil.
\section{Uaipi}
\begin{itemize}
\item {Grp. gram.:m.}
\end{itemize}
Planta tuberculosa do Norte do Brasil.
\section{Uaiumanas}
\begin{itemize}
\item {Grp. gram.:m. pl.}
\end{itemize}
Indígenas do Norte do Brasil.
\section{Uaiumás}
\begin{itemize}
\item {Grp. gram.:m. pl.}
\end{itemize}
Indígenas do Norte do Brasil.
\section{Uaiupés}
\begin{itemize}
\item {Grp. gram.:m. pl.}
\end{itemize}
Tríbo de Índios do Amazonas, hoje civilizada, que constitue a população da villa de Ega.
\section{Uaiurus}
\begin{itemize}
\item {Grp. gram.:m. pl.}
\end{itemize}
Indígenas do Norte do Brasil.
\section{Uajará}
\begin{itemize}
\item {Grp. gram.:m.}
\end{itemize}
Fruto silvestre do Brasil.
\section{Uajuru}
\begin{itemize}
\item {Grp. gram.:m.}
\end{itemize}
\begin{itemize}
\item {Utilização:Bras. do N}
\end{itemize}
O mesmo que \textunderscore guajeru\textunderscore .
\section{Ualua}
\begin{itemize}
\item {Grp. gram.:f.}
\end{itemize}
Espécie de cerveja, o mesmo que \textunderscore quimbombo\textunderscore .
\section{Uamamis}
\begin{itemize}
\item {Grp. gram.:m. pl.}
\end{itemize}
Tribo de Índios do Brasil, de que há representantes na villa de Alvedos.
\section{Uambé}
\begin{itemize}
\item {Grp. gram.:m.}
\end{itemize}
Cipó do Brasil, também conhecido por \textunderscore ambé\textunderscore , (\textunderscore philodendron ambe\textunderscore ).
\section{Uamiri}
\begin{itemize}
\item {Grp. gram.:m.}
\end{itemize}
\begin{itemize}
\item {Utilização:Bras}
\end{itemize}
Pequena frexa de selvagens.
(Alter. do tupi \textunderscore uibamirim\textunderscore )
\section{Uanambé}
\begin{itemize}
\item {Grp. gram.:m.}
\end{itemize}
Ave brasileira, do tamanho de uma pomba e de côr azul ferrete.
\section{Uananaus}
\begin{itemize}
\item {Grp. gram.:m. pl.}
\end{itemize}
Indígenas do Norte do Brasil.
\section{Uanhaes}
\begin{itemize}
\item {Grp. gram.:m. pl.}
\end{itemize}
Povo cafreal, ao sul de Moçambique.
\section{Uanhais}
\begin{itemize}
\item {Grp. gram.:m. pl.}
\end{itemize}
Povo cafreal, ao sul de Moçambique.
\section{Uanhi}
\begin{itemize}
\item {Grp. gram.:m.}
\end{itemize}
Nome, que na África portuguesa se dá ao pygargo. Cf. Serpa Pinto, II, 85.
\section{Uaniás}
\begin{itemize}
\item {Grp. gram.:m. pl.}
\end{itemize}
Indígenas do Norte do Brasil.
\section{Uapés}
\begin{itemize}
\item {Grp. gram.:m. pl.}
\end{itemize}
\begin{itemize}
\item {Utilização:Bras}
\end{itemize}
Tríbo de Índios, nas margens do rio do mesmo nome, affluente do Rio Negro.
\section{Uapixanas}
\begin{itemize}
\item {Grp. gram.:m. pl.}
\end{itemize}
Indígenas do Norte do Brasil.
\section{Uaraná}
\begin{itemize}
\item {Grp. gram.:m.}
\end{itemize}
O mesmo que \textunderscore guaraná\textunderscore . Cf. \textunderscore Pharmacopeia Port.\textunderscore 
\section{Uaraicus}
\begin{itemize}
\item {Grp. gram.:m. pl.}
\end{itemize}
Indígenas do Norte do Brasil.
\section{Uaranacuacenas}
\begin{itemize}
\item {Grp. gram.:m. pl.}
\end{itemize}
\begin{itemize}
\item {Utilização:Bras}
\end{itemize}
Tríbo de indígenas do Pará.
\section{Uariá}
\begin{itemize}
\item {Grp. gram.:m.}
\end{itemize}
\begin{itemize}
\item {Utilização:Bras}
\end{itemize}
Planta, de tubérculos farináceos e comestíveis.
\section{Uariquena}
\begin{itemize}
\item {Grp. gram.:f.}
\end{itemize}
\begin{itemize}
\item {Utilização:Bras}
\end{itemize}
Espécie de pimenta vermelha.
\section{Uariquina}
\begin{itemize}
\item {Grp. gram.:f.}
\end{itemize}
\begin{itemize}
\item {Utilização:Bras}
\end{itemize}
O mesmo que \textunderscore uariquena\textunderscore .
\section{Uarubé}
\begin{itemize}
\item {Grp. gram.:m.}
\end{itemize}
\begin{itemize}
\item {Utilização:Bras}
\end{itemize}
Suco de massa de mandioca.
\section{Uaruman}
\begin{itemize}
\item {Grp. gram.:m.}
\end{itemize}
\begin{itemize}
\item {Utilização:Bras. do N}
\end{itemize}
Palmeira, o mesmo que \textunderscore guaruman\textunderscore .
\section{Uarurembóia}
\begin{itemize}
\item {Grp. gram.:f.}
\end{itemize}
\begin{itemize}
\item {Utilização:Bras}
\end{itemize}
Arbusto medicinal das regiões do Amazonas.
\section{Uassassu}
\begin{itemize}
\item {Grp. gram.:m.}
\end{itemize}
\begin{itemize}
\item {Utilização:Bras. do N}
\end{itemize}
Espécie de palmeira.
\section{Uassu}
\begin{itemize}
\item {Grp. gram.:adj.}
\end{itemize}
O mesmo que \textunderscore guaçu\textunderscore .
\section{Uatapu}
\begin{itemize}
\item {Grp. gram.:m.}
\end{itemize}
\begin{itemize}
\item {Utilização:Bras. do N}
\end{itemize}
Buzina, com que os Índios suppõem attrahir o peixe.
(Do tupi)
\section{Uatite}
\begin{itemize}
\item {Grp. gram.:f.}
\end{itemize}
\begin{itemize}
\item {Proveniência:(Do fr. \textunderscore ouate\textunderscore )}
\end{itemize}
Substância mineral, que se apresenta sob a fórma de algodão em rama, e é um hydróxydo de manganés baritífero, que contém óxydo de ferro e carbóne.
\section{Uatito}
\begin{itemize}
\item {Grp. gram.:f.}
\end{itemize}
\begin{itemize}
\item {Proveniência:(Do fr. \textunderscore ouate\textunderscore )}
\end{itemize}
Substância mineral, que se apresenta sob a fórma de algodão em rama, e é um hydróxydo de manganés baritífero, que contém óxydo de ferro e carbóne.
\section{Uaturá}
\begin{itemize}
\item {Grp. gram.:m.}
\end{itemize}
\begin{itemize}
\item {Utilização:Bras. do N}
\end{itemize}
Cesto, o mesmo que \textunderscore aturá\textunderscore .
\section{Uauaçu}
\begin{itemize}
\item {Grp. gram.:m.}
\end{itemize}
\begin{itemize}
\item {Utilização:Bras. do N}
\end{itemize}
O mesmo que \textunderscore coqueiro\textunderscore ^1.
\section{Uauçu}
\begin{itemize}
\item {Grp. gram.:m.}
\end{itemize}
Planta leguminosa do Brasil.
O mesmo que \textunderscore uauaçu\textunderscore , provavelmente, e o mesmo que \textunderscore uassu\textunderscore .
\section{Uauíra}
\begin{itemize}
\item {Grp. gram.:f.}
\end{itemize}
Gênio das águas, entre os indígenas do Tocantins.
\section{Uaupés}
\begin{itemize}
\item {Grp. gram.:m. pl.}
\end{itemize}
\begin{itemize}
\item {Utilização:Bras}
\end{itemize}
Tríbo de aborígenes do Pará, provavelmente o mesmo que \textunderscore Uapés\textunderscore .
\section{Uavaona}
\begin{itemize}
\item {Grp. gram.:f.}
\end{itemize}
\begin{itemize}
\item {Utilização:Bras}
\end{itemize}
Árvore fructífera dos sertões.
\section{Ubá}
\begin{itemize}
\item {Grp. gram.:f.}
\end{itemize}
\begin{itemize}
\item {Utilização:Bras}
\end{itemize}
Planta amomácea.
Canôa, de uma só peça de madeira, sem quilha e sem bojo.
\section{Ubaaçu}
\begin{itemize}
\item {Grp. gram.:m.}
\end{itemize}
O mesmo que \textunderscore pau-pereira\textunderscore .
\section{Ubacaba}
\begin{itemize}
\item {Grp. gram.:m.}
\end{itemize}
Planta myrtácea do Brasil.
\section{Ubacaia}
\begin{itemize}
\item {Grp. gram.:f.}
\end{itemize}
O mesmo que \textunderscore cana-de-macaco\textunderscore .
\section{Ubaia}
\begin{itemize}
\item {Grp. gram.:f.}
\end{itemize}
O fruto da ubaia-muchama.
O mesmo que \textunderscore ubaia-muchama\textunderscore .
\section{Ubaia-muchama}
\begin{itemize}
\item {Grp. gram.:f.}
\end{itemize}
Arbusto myrtáceo da América.
\section{Ubaia-muxama}
\begin{itemize}
\item {Grp. gram.:f.}
\end{itemize}
Arbusto myrtáceo da América.
\section{Ubaína}
\begin{itemize}
\item {Grp. gram.:f.}
\end{itemize}
Alcalóide da ubaia.
\section{Ubango}
\begin{itemize}
\item {Grp. gram.:m.}
\end{itemize}
Pássaro dentirostro da África.
\section{Ubarana}
\begin{itemize}
\item {Grp. gram.:f.}
\end{itemize}
Peixe marítimo, ordinário, do Brasil.
\section{Ubatan}
\begin{itemize}
\item {Grp. gram.:m.}
\end{itemize}
\begin{itemize}
\item {Utilização:Bras}
\end{itemize}
Árvore terebinthácea dos sertões.
\section{Uberana}
\begin{itemize}
\item {Grp. gram.:f.}
\end{itemize}
O mesmo que \textunderscore ubarana\textunderscore .
\section{Uberdade}
\begin{itemize}
\item {Grp. gram.:f.}
\end{itemize}
\begin{itemize}
\item {Proveniência:(Do lat. \textunderscore ubertas\textunderscore )}
\end{itemize}
Qualidade de úbere.
Abundância.
Fertilidade; fecundidade.
Opulência; fartura.
\section{Úbere}
\begin{itemize}
\item {Grp. gram.:adj.}
\end{itemize}
\begin{itemize}
\item {Grp. gram.:M.}
\end{itemize}
\begin{itemize}
\item {Proveniência:(Lat. \textunderscore uber\textunderscore )}
\end{itemize}
Fértil; abundante; farto.
Têta, glândula mamal.
\section{Ubérrimo}
\begin{itemize}
\item {Grp. gram.:adj.}
\end{itemize}
\begin{itemize}
\item {Proveniência:(Lat. \textunderscore uberrirmus\textunderscore )}
\end{itemize}
Muito úbere, muito abundante.
\section{Ubertoso}
\begin{itemize}
\item {Grp. gram.:adj.}
\end{itemize}
\begin{itemize}
\item {Utilização:Poét.}
\end{itemize}
O mesmo que \textunderscore úbere\textunderscore .
(Cp. lat. \textunderscore ubertus\textunderscore )
\section{Ubi}
\begin{itemize}
\item {Grp. gram.:m.}
\end{itemize}
\begin{itemize}
\item {Utilização:Bras}
\end{itemize}
Gênero de palmeiras.
\section{Ubianganga}
\begin{itemize}
\item {Grp. gram.:f.}
\end{itemize}
Espécie de corvo africano.
\section{Ubim}
\begin{itemize}
\item {Grp. gram.:m.}
\end{itemize}
\begin{itemize}
\item {Utilização:Bras}
\end{itemize}
Gênero de palmeiras.
\section{Ubim-mirim}
\begin{itemize}
\item {Grp. gram.:m.}
\end{itemize}
\begin{itemize}
\item {Utilização:Bras. do N}
\end{itemize}
Espécie de palmeira.
\section{Ubim-uaçu}
\begin{itemize}
\item {Grp. gram.:m.}
\end{itemize}
\begin{itemize}
\item {Utilização:Bras. do N}
\end{itemize}
Espécie de palmeira, cujas fôlhas servem para cobrir casas.
\section{Ubiquação}
\begin{itemize}
\item {Grp. gram.:f.}
\end{itemize}
\begin{itemize}
\item {Proveniência:(Do lat. \textunderscore ubique\textunderscore )}
\end{itemize}
Propriedade ou estado do que é ubíquo.
\section{Ubiquidade}
\begin{itemize}
\item {fónica:qu-i}
\end{itemize}
\begin{itemize}
\item {Grp. gram.:f.}
\end{itemize}
\begin{itemize}
\item {Proveniência:(Do lat. \textunderscore ubique\textunderscore )}
\end{itemize}
Propriedade ou estado do que é ubíquo.
\section{Ubiquista}
\begin{itemize}
\item {fónica:qu-is}
\end{itemize}
\begin{itemize}
\item {Grp. gram.:m.}
\end{itemize}
\begin{itemize}
\item {Proveniência:(De \textunderscore ubíquo\textunderscore )}
\end{itemize}
Membro de uma seita lutherana, segundo a qual o corpo de Christo está presente na Eucharistia, não em virtude da transubstanciação, mas porque êlle está em toda a parte.
\section{Ubiquitário}
\begin{itemize}
\item {fónica:qu-i}
\end{itemize}
\begin{itemize}
\item {Grp. gram.:m.}
\end{itemize}
O mesmo que \textunderscore ubiquista\textunderscore .
\section{Ubíquo}
\begin{itemize}
\item {Grp. gram.:adj.}
\end{itemize}
\begin{itemize}
\item {Proveniência:(Do lat. \textunderscore ubique\textunderscore )}
\end{itemize}
Que está ao mesmo tempo em toda a parte.
\section{Ubiracicá}
\begin{itemize}
\item {Grp. gram.:f.}
\end{itemize}
O mesmo que \textunderscore icica\textunderscore .
\section{Ubirarema}
\begin{itemize}
\item {Grp. gram.:f.}
\end{itemize}
O mesmo que \textunderscore ibirarema\textunderscore .
\section{Ubre}
\begin{itemize}
\item {Grp. gram.:m.}
\end{itemize}
(V.úbere)
\section{Ubuçu}
\begin{itemize}
\item {Grp. gram.:m.}
\end{itemize}
\begin{itemize}
\item {Utilização:Bras}
\end{itemize}
Espécie de coqueiro.
\section{Ubussu}
\begin{itemize}
\item {Grp. gram.:m.}
\end{itemize}
\begin{itemize}
\item {Utilização:Bras}
\end{itemize}
Espécie de coqueiro.
\section{Uca}
\begin{itemize}
\item {Grp. gram.:f.}
\end{itemize}
Gênero de plantas gramíneas.
\section{Ucá}
\begin{itemize}
\item {Grp. gram.:m.}
\end{itemize}
\begin{itemize}
\item {Proveniência:(Do conc. \textunderscore hukha\textunderscore )}
\end{itemize}
Cachimbo, usado por Banianes.
\section{Uçá}
\begin{itemize}
\item {Grp. gram.:m.}
\end{itemize}
\begin{itemize}
\item {Utilização:Bras}
\end{itemize}
Espécie de formiga.
\section{Ucace}
\begin{itemize}
\item {Grp. gram.:m.}
\end{itemize}
Decreto do Imperador da Rússia.
(Do russo \textunderscore ukás\textunderscore , ordenação)
\section{Ucanha}
\begin{itemize}
\item {Grp. gram.:f.}
\end{itemize}
Fruta de Moçambique.
\section{Ucha}
\begin{itemize}
\item {Grp. gram.:Loc.}
\end{itemize}
\begin{itemize}
\item {Utilização:pop.}
\end{itemize}
\textunderscore f.\textunderscore  (e der.)
O mesmo que \textunderscore hucha\textunderscore , etc.
\textunderscore Ficar á ucha\textunderscore , ficar sem nada, ficar a chuchar no dedo.
\section{Ucha}
\begin{itemize}
\item {Grp. gram.:f.}
\end{itemize}
\begin{itemize}
\item {Utilização:Prov.}
\end{itemize}
\begin{itemize}
\item {Proveniência:(De \textunderscore uscla\textunderscore , por \textunderscore ústula\textunderscore , do lat. \textunderscore ustus\textunderscore )}
\end{itemize}
Queimada de urze.
\section{Ucubu}
\begin{itemize}
\item {Grp. gram.:m.}
\end{itemize}
Árvore da ilha de San-Thomé.
\section{Ucui}
\begin{itemize}
\item {Grp. gram.:m.}
\end{itemize}
Planta antifebril da Guiné.
\section{Ucuuba}
\begin{itemize}
\item {Grp. gram.:f.}
\end{itemize}
Árvore myristicácea do Brasil, (\textunderscore myristica officínalis\textunderscore ).
\section{Udo}
\begin{itemize}
\item {Grp. gram.:m.}
\end{itemize}
Coisa nenhuma:«\textunderscore ...não deixei na matéria udo nem miúdo\textunderscore ». Castilho, \textunderscore Felic. pela Agr.\textunderscore 
(Aphérese de \textunderscore graúdo\textunderscore ?)
\section{Udometria}
\begin{itemize}
\item {Grp. gram.:f.}
\end{itemize}
Emprêgo do udómetro.
\section{Udométrico}
\begin{itemize}
\item {Grp. gram.:adj.}
\end{itemize}
Relativo á udometria.
\section{Udómetro}
\begin{itemize}
\item {Grp. gram.:m.}
\end{itemize}
\begin{itemize}
\item {Proveniência:(Do lat. \textunderscore udus\textunderscore  + gr. \textunderscore metron\textunderscore )}
\end{itemize}
O mesmo que \textunderscore pluviómetro\textunderscore .
\section{Udora}
\begin{itemize}
\item {Grp. gram.:f.}
\end{itemize}
\begin{itemize}
\item {Proveniência:(Do gr. \textunderscore udor\textunderscore )}
\end{itemize}
Gênero de crustáceos decápodes.
Espécie de alga americana.
\section{Udótea}
\begin{itemize}
\item {Grp. gram.:f.}
\end{itemize}
Gênero de plantas phýceas.
\section{Uerequenas}
\begin{itemize}
\item {Grp. gram.:f. pl.}
\end{itemize}
\begin{itemize}
\item {Utilização:Bras}
\end{itemize}
Tríbo de aborígenes do Pará.
\section{Uerimás}
\begin{itemize}
\item {Grp. gram.:m. pl.}
\end{itemize}
Índios selvagens das margens do Apaporis, no Brasil.
\section{Ufa, á}
\begin{itemize}
\item {Grp. gram.:loc. adv.}
\end{itemize}
\begin{itemize}
\item {Proveniência:(It. \textunderscore uffa\textunderscore )}
\end{itemize}
Abundantemente; á larga.
Á custa alheia.
\section{Ufá!}
\begin{itemize}
\item {Grp. gram.:interj.}
\end{itemize}
\begin{itemize}
\item {Proveniência:(Do fr. \textunderscore ouf\textunderscore )}
\end{itemize}
(designativa de \textunderscore admiração\textunderscore , \textunderscore ironia\textunderscore , \textunderscore cansaço\textunderscore )
\section{Ufanamente}
\begin{itemize}
\item {Grp. gram.:adv.}
\end{itemize}
De modo ufano; com ufania; com vanglória.
\section{Ufanar}
\begin{itemize}
\item {Grp. gram.:v. t.}
\end{itemize}
Tornar ufano.
Regosijar; causar vaidade em.
\section{Ufania}
\begin{itemize}
\item {Grp. gram.:f.}
\end{itemize}
Qualidade do que é ufano.
Vanglória; vaidade; ostentação.
\section{Ufano}
\begin{itemize}
\item {Grp. gram.:adj.}
\end{itemize}
\begin{itemize}
\item {Proveniência:(De \textunderscore ufa\textunderscore )}
\end{itemize}
Que se orgulha de alguma coisa; que se vangloría.
Jactancioso; vaidoso.
Satisfeito de si próprio.
Ostentoso; bizarro.
\section{Ufanoso}
\begin{itemize}
\item {Grp. gram.:adj.}
\end{itemize}
\begin{itemize}
\item {Proveniência:(De \textunderscore ufano\textunderscore )}
\end{itemize}
Que tem ufania; envaidado; ufano.
\section{Uga}
\begin{itemize}
\item {Grp. gram.:f.}
\end{itemize}
O mesmo que \textunderscore ujamanta\textunderscore . Cf. Bluteau, \textunderscore Vocab.\textunderscore 
\section{Uga!}
\begin{itemize}
\item {Grp. gram.:interj.}
\end{itemize}
\begin{itemize}
\item {Utilização:Prov.}
\end{itemize}
\begin{itemize}
\item {Utilização:trasm.}
\end{itemize}
Àvante! para a frente!
\section{Uga}
\begin{itemize}
\item {Grp. gram.:f.}
\end{itemize}
\begin{itemize}
\item {Utilização:Gír.}
\end{itemize}
\textunderscore Fazer a uga\textunderscore , continuar.
(Cp. \textunderscore ugalhar\textunderscore )
\section{Ugalha}
\begin{itemize}
\item {Grp. gram.:f.}
\end{itemize}
\begin{itemize}
\item {Utilização:Pop.}
\end{itemize}
O mesmo que \textunderscore igualha\textunderscore :«\textunderscore busque-lhe da sua ugalha o pai vaqueiro á novilha\textunderscore ». F. Manuel.
\section{Ugalhar}
\begin{itemize}
\item {Grp. gram.:v.}
\end{itemize}
\begin{itemize}
\item {Utilização:t. Marn.}
\end{itemize}
O mesmo que \textunderscore apancar\textunderscore .
(Corr. do \textunderscore igualar\textunderscore )
\section{Ugalho}
\begin{itemize}
\item {Grp. gram.:m.}
\end{itemize}
\begin{itemize}
\item {Proveniência:(De \textunderscore ugalhar\textunderscore )}
\end{itemize}
Espécie de ancinho ou varredoiro, nas salinas.
\section{Ugar}
\begin{itemize}
\item {Grp. gram.:v. t.}
\end{itemize}
\begin{itemize}
\item {Utilização:Gír.}
\end{itemize}
Gritar; dar alarma.
\section{Ugar}
\begin{itemize}
\item {Grp. gram.:v. t.}
\end{itemize}
\begin{itemize}
\item {Utilização:Prov.}
\end{itemize}
O mesmo que \textunderscore igualar\textunderscore .
Apertar (mólhos).
(Por \textunderscore igar\textunderscore  = \textunderscore igaar\textunderscore , contr. de \textunderscore igualar\textunderscore )
\section{Uge}
\begin{itemize}
\item {Grp. gram.:m.}
\end{itemize}
Peixe, o mesmo que \textunderscore ujo\textunderscore .
\section{Ugerbo}
\begin{itemize}
\item {Grp. gram.:m.}
\end{itemize}
Outra fórma de \textunderscore ogervão\textunderscore . Cf. S. Costa, \textunderscore Hist. das Pl. Méd.\textunderscore 
\section{Uginas}
\begin{itemize}
\item {Grp. gram.:m. pl.}
\end{itemize}
Indígenas do Norte do Brasil.
\section{Ugonotos}
\begin{itemize}
\item {Grp. gram.:m. pl.}
\end{itemize}
(V.huguenotes). Cf. Sousa, \textunderscore Vida do Arceb.\textunderscore , I, 387.
\section{Úgrico}
\begin{itemize}
\item {Grp. gram.:m.}
\end{itemize}
Grupo de línguas uralo-altaicas.
Uma dessas línguas.
\section{Ugro...}
\begin{itemize}
\item {Grp. gram.:pref.}
\end{itemize}
(designativo do \textunderscore úgrico\textunderscore )
\section{Ugro-finlandês}
\begin{itemize}
\item {Grp. gram.:adj.}
\end{itemize}
Diz-se de um grupo de línguas uralo-altaicas.
\section{Uhlano}
\begin{itemize}
\item {Grp. gram.:m.}
\end{itemize}
\begin{itemize}
\item {Proveniência:(Al. \textunderscore uhlan\textunderscore , do polaco \textunderscore ula\textunderscore , lança)}
\end{itemize}
Cavalleiro, armado de lança, no exército austriaco e alemão.
\section{Ui!}
\begin{itemize}
\item {Grp. gram.:interj.}
\end{itemize}
(designativa de \textunderscore surpresa\textunderscore , \textunderscore admiração\textunderscore , \textunderscore repugnância\textunderscore  ou \textunderscore dôr\textunderscore )
\section{Uíaás}
\begin{itemize}
\item {Grp. gram.:m. pl.}
\end{itemize}
\begin{itemize}
\item {Utilização:Bras}
\end{itemize}
Tribo de aborígenes de Mato-Grosso.
\section{Uiacima}
\begin{itemize}
\item {Grp. gram.:f.}
\end{itemize}
\begin{itemize}
\item {Utilização:Bras}
\end{itemize}
Árvore dos sertões.
\section{Uiaeira}
\begin{itemize}
\item {Grp. gram.:f.}
\end{itemize}
\begin{itemize}
\item {Utilização:Bras}
\end{itemize}
Árvore fructífera dos sertões.
\section{Uiara}
\begin{itemize}
\item {Grp. gram.:f.}
\end{itemize}
\begin{itemize}
\item {Utilização:Bras. do N}
\end{itemize}
\begin{itemize}
\item {Proveniência:(T. tupi)}
\end{itemize}
O mesmo que \textunderscore mãe-d'água\textunderscore .
\section{Uigúrico}
\begin{itemize}
\item {Grp. gram.:m.}
\end{itemize}
Língua uralo-altaica.
\section{Uinarana}
\begin{itemize}
\item {Grp. gram.:f.}
\end{itemize}
Peixe do Norte do Brasil.
\section{Uiqué}
\begin{itemize}
\item {Grp. gram.:m.}
\end{itemize}
Fruto comestível, das regiões do Amazonas, (\textunderscore lucuma mammosa\textunderscore , Gaertn.).
\section{Uirari}
\begin{itemize}
\item {Grp. gram.:m.}
\end{itemize}
O mesmo que \textunderscore curare\textunderscore .
\section{Uistiti}
\begin{itemize}
\item {Grp. gram.:m.}
\end{itemize}
Variedade de macaco.
\section{Uiti}
\begin{itemize}
\item {Grp. gram.:m.}
\end{itemize}
O mesmo que \textunderscore oiti\textunderscore .
\section{Uivador}
\begin{itemize}
\item {Grp. gram.:m.  e  adj.}
\end{itemize}
O que uiva.
\section{Uivante}
\begin{itemize}
\item {Grp. gram.:adj.}
\end{itemize}
Que uiva. Cf. Dom. Vieira, vb. \textunderscore coruja\textunderscore .
\section{Uivar}
\begin{itemize}
\item {Grp. gram.:v. i.}
\end{itemize}
\begin{itemize}
\item {Utilização:Fig.}
\end{itemize}
\begin{itemize}
\item {Proveniência:(Do lat. \textunderscore ululare\textunderscore ?)}
\end{itemize}
Dar uivos.
Bravejar, vociferar.
\section{Uivo}
\begin{itemize}
\item {Grp. gram.:m.}
\end{itemize}
\begin{itemize}
\item {Utilização:Fig.}
\end{itemize}
\begin{itemize}
\item {Proveniência:(De \textunderscore uivar\textunderscore )}
\end{itemize}
Voz de lobo e de outras feras.
Grito prolongado e lamentoso do cão.
Acto de vociferar.
\section{Ujamanta}
\begin{itemize}
\item {Grp. gram.:f.}
\end{itemize}
O mesmo ou melhór que \textunderscore jamanta\textunderscore .
\section{Ujica}
\begin{itemize}
\item {Grp. gram.:f.}
\end{itemize}
\begin{itemize}
\item {Utilização:Bras}
\end{itemize}
Espécie de quitute.
\section{Ujo}
\begin{itemize}
\item {Grp. gram.:m.}
\end{itemize}
Nome vulgar de uma ave de rapina, espécie de águia, também conhecida por \textunderscore corujão\textunderscore .
Pequeno peixe, em fórma de raia.
\section{Ula}
\begin{itemize}
\item {Grp. gram.:f.}
\end{itemize}
\begin{itemize}
\item {Utilização:Prov.}
\end{itemize}
\begin{itemize}
\item {Utilização:alg.}
\end{itemize}
O mesmo que \textunderscore fula-fula\textunderscore .
(Contr. de \textunderscore fula\textunderscore ?)
\section{Ula}
(Cp. \textunderscore ulo\textunderscore ^1)
\section{Ulano}
\begin{itemize}
\item {Grp. gram.:m.}
\end{itemize}
\begin{itemize}
\item {Proveniência:(Al. \textunderscore uhlan\textunderscore , do polaco \textunderscore ula\textunderscore , lança)}
\end{itemize}
Cavaleiro, armado de lança, no exército austriaco e alemão.
\section{Ulas}
(Cp. \textunderscore ulo\textunderscore ^1)
\section{Úlcera}
\begin{itemize}
\item {Grp. gram.:f.}
\end{itemize}
\begin{itemize}
\item {Proveniência:(Lat. \textunderscore ulcera\textunderscore , pl. de \textunderscore ulcus\textunderscore , \textunderscore ulceris\textunderscore )}
\end{itemize}
Ferida ou chaga antiga, cuja cicatrização é pouco provável.
Alteração do tecido lenhoso das árvores.
\section{Ulceração}
\begin{itemize}
\item {Grp. gram.:f.}
\end{itemize}
\begin{itemize}
\item {Proveniência:(Do lat. \textunderscore ulceratio\textunderscore )}
\end{itemize}
Acto ou effeito de ulcerar.
\section{Ulcerar}
\begin{itemize}
\item {Grp. gram.:v. t.}
\end{itemize}
\begin{itemize}
\item {Utilização:Fig.}
\end{itemize}
\begin{itemize}
\item {Grp. gram.:V. i.  e  p.}
\end{itemize}
\begin{itemize}
\item {Proveniência:(Lat. \textunderscore ulcerare\textunderscore )}
\end{itemize}
Causar úlcera em.
Atormentar; magoar.
Alterar, corromper.
Adquirir úlcera; cobrir-se de úlceras.
\section{Ulcerativo}
\begin{itemize}
\item {Grp. gram.:adj.}
\end{itemize}
\begin{itemize}
\item {Proveniência:(De \textunderscore ulcerar\textunderscore )}
\end{itemize}
Relativo a úlcera.
Que ulcéra.
\section{Ulceróide}
\begin{itemize}
\item {Grp. gram.:adj.}
\end{itemize}
\begin{itemize}
\item {Proveniência:(De \textunderscore úlcera\textunderscore  + gr. \textunderscore eidos\textunderscore )}
\end{itemize}
Semelhante a uma úlcera.
\section{Ulceroso}
\begin{itemize}
\item {Grp. gram.:adj.}
\end{itemize}
\begin{itemize}
\item {Proveniência:(Lat. \textunderscore ulcerosus\textunderscore )}
\end{itemize}
Que tem úlceras.
Que é da natureza da úlcera.
\section{Uleda}
\begin{itemize}
\item {Grp. gram.:f.}
\end{itemize}
Gênero de insectos coleópteros heterómeros.
\section{Ulemás}
\begin{itemize}
\item {Grp. gram.:m. pl.}
\end{itemize}
Sábios ou doutores da lei, entre os Árabes e Turcos.
(Ar. \textunderscore ulema\textunderscore )
\section{Ulfilano}
\begin{itemize}
\item {Grp. gram.:adj.}
\end{itemize}
Diz-se de uma espécie de letras góticas, cuja invenção se attribue ao Bispo Ulfilas.
\section{Ulídia}
\begin{itemize}
\item {Grp. gram.:f.}
\end{itemize}
Gênero de insectos dípteros.
\section{Uliginário}
\begin{itemize}
\item {Grp. gram.:adj.}
\end{itemize}
\begin{itemize}
\item {Utilização:Bot.}
\end{itemize}
\begin{itemize}
\item {Proveniência:(Do lat. \textunderscore uligo\textunderscore , \textunderscore uliginis\textunderscore )}
\end{itemize}
Que cresce em lugares húmidos.
\section{Uliginoso}
\begin{itemize}
\item {Grp. gram.:adj.}
\end{itemize}
\begin{itemize}
\item {Proveniência:(Lat. \textunderscore uliginosus\textunderscore )}
\end{itemize}
Lamacento; pantanoso.
Diz-se dos vegetaes que crescem em terrenos pantanosos.
\section{Ulissiponense}
\begin{itemize}
\item {Grp. gram.:adj.}
\end{itemize}
O mesmo que \textunderscore olisiponense\textunderscore . Cf. Herculano.
\section{Ulite}
\begin{itemize}
\item {Grp. gram.:f.}
\end{itemize}
\begin{itemize}
\item {Proveniência:(Do gr. \textunderscore oulon\textunderscore )}
\end{itemize}
Inflammação da membrana mucosa das gengivas.
\section{Ulluco}
\begin{itemize}
\item {Grp. gram.:m.}
\end{itemize}
Gênero de plantas portuláceas da Bolívia e do Peru, (\textunderscore ullucus tuberosus\textunderscore , Collas).
\section{Ulmáceas}
\begin{itemize}
\item {Grp. gram.:f. pl.}
\end{itemize}
Família de plantas, que tem por typo o ulmo.
(Fem. pl. de \textunderscore ulmáceo\textunderscore )
\section{Ulmáceo}
\begin{itemize}
\item {Grp. gram.:adj.}
\end{itemize}
Relativo ou semelhante ao ulmo.
\section{Ulmarena}
\begin{itemize}
\item {Grp. gram.:f.}
\end{itemize}
\begin{itemize}
\item {Proveniência:(De \textunderscore ulmária\textunderscore )}
\end{itemize}
Preparação pharmacêutica, contra o rheumatismo articular.
\section{Ulmária}
\begin{itemize}
\item {Grp. gram.:f.}
\end{itemize}
\begin{itemize}
\item {Proveniência:(De \textunderscore ulmo\textunderscore )}
\end{itemize}
Planta rosácea, (\textunderscore spiraea ulmaria\textunderscore , Lin.)
\section{Ulmárico}
\begin{itemize}
\item {Grp. gram.:adj.}
\end{itemize}
Diz-se de um ácido, que é a ulmarina.
\section{Ulmarina}
\begin{itemize}
\item {Grp. gram.:f.}
\end{itemize}
\begin{itemize}
\item {Utilização:Chím.}
\end{itemize}
Pó crystallino, que se extrái da ulmária.
\section{Ulmato}
\begin{itemize}
\item {Grp. gram.:m.}
\end{itemize}
\begin{itemize}
\item {Proveniência:(De \textunderscore ulmo\textunderscore )}
\end{itemize}
Sal, resultante da combinação do ácido úlmico com uma base.
\section{Ulmeira}
\begin{itemize}
\item {Grp. gram.:f.}
\end{itemize}
O mesmo que \textunderscore ulmária\textunderscore .
\section{Ulmeiro}
\begin{itemize}
\item {Grp. gram.:m.}
\end{itemize}
O mesmo que \textunderscore ulmo\textunderscore .
\section{Úlmico}
\begin{itemize}
\item {Grp. gram.:adj.}
\end{itemize}
\begin{itemize}
\item {Proveniência:(De \textunderscore ulmo\textunderscore )}
\end{itemize}
Diz-se de um ácido, que existe no terriço e casca do ulmo.
\section{Ulmina}
\begin{itemize}
\item {Grp. gram.:f.}
\end{itemize}
\begin{itemize}
\item {Proveniência:(De \textunderscore ulmo\textunderscore )}
\end{itemize}
Um dos productos da decomposição da cellulose.
\section{Ulmo}
\begin{itemize}
\item {Grp. gram.:m.}
\end{itemize}
(V.olmo)
\section{Ulna}
\begin{itemize}
\item {Grp. gram.:f.}
\end{itemize}
\begin{itemize}
\item {Proveniência:(Lat. \textunderscore ulna\textunderscore )}
\end{itemize}
Nome antigo do cúbito.
Espécie de medida antiga, equivalente a uma braça.
\section{Ulnal}
\begin{itemize}
\item {Grp. gram.:adj.}
\end{itemize}
O mesmo que \textunderscore ulnário\textunderscore .
\section{Ulnário}
\begin{itemize}
\item {Grp. gram.:adj.}
\end{itemize}
\begin{itemize}
\item {Proveniência:(De \textunderscore ulna\textunderscore )}
\end{itemize}
Relativo ao cúbito.
\section{Ulo}
(juncção do ant. adv. \textunderscore u\textunderscore , onde, e do art. \textunderscore lo\textunderscore ):«\textunderscore ulo\textunderscore  ser a autoridade de fidalgo?»Sousa, \textunderscore Vida do Arceb.\textunderscore , II, 68. E assim \textunderscore ulos\textunderscore  = \textunderscore onde os\textunderscore ; \textunderscore ula\textunderscore  = \textunderscore onde a\textunderscore ; \textunderscore ulas\textunderscore  = \textunderscore onde as\textunderscore .
\section{Ulo}
\begin{itemize}
\item {Grp. gram.:m.}
\end{itemize}
\begin{itemize}
\item {Utilização:Bras}
\end{itemize}
Grito; gemido.
\section{Ulóboro}
\begin{itemize}
\item {Grp. gram.:m.}
\end{itemize}
Gênero de arachnídeos.
\section{Ulodendro}
\begin{itemize}
\item {Grp. gram.:m.}
\end{itemize}
\begin{itemize}
\item {Proveniência:(Do gr. \textunderscore oulos\textunderscore  + \textunderscore dendron\textunderscore )}
\end{itemize}
Gênero de vegetaes fósseis.
\section{Ulofobia}
\begin{itemize}
\item {Grp. gram.:f.}
\end{itemize}
Mania, caracterizada por certa aversão aos próprios filhos.
\section{Ulojanja}
\begin{itemize}
\item {Grp. gram.:f.}
\end{itemize}
Pássaro conirostro africano.
\section{Uloncia}
\begin{itemize}
\item {Grp. gram.:f.}
\end{itemize}
\begin{itemize}
\item {Proveniência:(Do gr. \textunderscore oulon\textunderscore  + \textunderscore onkos\textunderscore )}
\end{itemize}
Inchação das gengivas.
\section{Ulophobia}
\begin{itemize}
\item {Grp. gram.:f.}
\end{itemize}
Mania, caracterizada por certa aversão aos próprios filhos.
\section{Ulorragia}
\begin{itemize}
\item {Grp. gram.:f.}
\end{itemize}
\begin{itemize}
\item {Proveniência:(Do gr. \textunderscore oulon\textunderscore  + \textunderscore regnumi\textunderscore )}
\end{itemize}
Hemorragia nas gengivas.
\section{Ulorrhagia}
\begin{itemize}
\item {Grp. gram.:f.}
\end{itemize}
\begin{itemize}
\item {Proveniência:(Do gr. \textunderscore oulon\textunderscore  + \textunderscore regnumi\textunderscore )}
\end{itemize}
Hemorrhagia nas gengivas.
\section{Ulos}
(Cp. \textunderscore ulo\textunderscore ^1)
\section{Ulosomo}
\begin{itemize}
\item {fónica:sô}
\end{itemize}
\begin{itemize}
\item {Grp. gram.:m.}
\end{itemize}
\begin{itemize}
\item {Proveniência:(Do gr. \textunderscore oulos\textunderscore  + \textunderscore soma\textunderscore )}
\end{itemize}
Gênero de insectos coleópteros tetrâmeros.
\section{Ulossomo}
\begin{itemize}
\item {Grp. gram.:m.}
\end{itemize}
\begin{itemize}
\item {Proveniência:(Do gr. \textunderscore oulos\textunderscore  + \textunderscore soma\textunderscore )}
\end{itemize}
Gênero de insectos coleópteros tetrâmeros.
\section{Ulótricho}
\begin{itemize}
\item {fónica:co}
\end{itemize}
\begin{itemize}
\item {Grp. gram.:adj.}
\end{itemize}
\begin{itemize}
\item {Grp. gram.:M.}
\end{itemize}
\begin{itemize}
\item {Grp. gram.:Pl.}
\end{itemize}
\begin{itemize}
\item {Proveniência:(Do gr. \textunderscore oulotrikhos\textunderscore )}
\end{itemize}
Que tem cabellos crespos.
Que tem crespos os appêndices ciliares.
Gênero de algas filamentosas.
Homens de cabello lanoso ou crespo, que constituem a primeira divisão da espécie humana, no systema de F. Müller.
\section{Ulótrico}
\begin{itemize}
\item {Grp. gram.:adj.}
\end{itemize}
\begin{itemize}
\item {Grp. gram.:M.}
\end{itemize}
\begin{itemize}
\item {Grp. gram.:Pl.}
\end{itemize}
\begin{itemize}
\item {Proveniência:(Do gr. \textunderscore oulotrikhos\textunderscore )}
\end{itemize}
Que tem cabelos crespos.
Que tem crespos os apêndices ciliares.
Gênero de algas filamentosas.
Homens de cabelo lanoso ou crespo, que constituem a primeira divisão da espécie humana, no sistema de F. Müller.
\section{Ulterior}
\begin{itemize}
\item {Grp. gram.:adj.}
\end{itemize}
\begin{itemize}
\item {Proveniência:(Lat. \textunderscore ulterior\textunderscore )}
\end{itemize}
Situado além.
Que está ou succede depois.
Que chega depois.
\section{Ulterioridade}
\begin{itemize}
\item {Grp. gram.:f.}
\end{itemize}
Qualidade do que é ulterior.
\section{Ulteriormente}
\begin{itemize}
\item {Grp. gram.:adv.}
\end{itemize}
\begin{itemize}
\item {Proveniência:(De \textunderscore ulterior\textunderscore )}
\end{itemize}
Depois.
Em lugar ulterior; ultimamente.
\section{Última}
\begin{itemize}
\item {Grp. gram.:f.}
\end{itemize}
(V.últimas)
\section{Ultimação}
\begin{itemize}
\item {Grp. gram.:f.}
\end{itemize}
Acto de ultimar.
Aperfeiçoamento.
\section{Ultimadamente}
\begin{itemize}
\item {Grp. gram.:adv.}
\end{itemize}
\begin{itemize}
\item {Proveniência:(De \textunderscore ultimado\textunderscore )}
\end{itemize}
O mesmo que \textunderscore ultimamente\textunderscore .
Por último.
Até o ponto extremo.
\section{Ultimado}
\begin{itemize}
\item {Grp. gram.:adj.}
\end{itemize}
\begin{itemize}
\item {Proveniência:(De \textunderscore ultimar\textunderscore )}
\end{itemize}
Concluído; acabado.
\section{Ultimador}
\begin{itemize}
\item {Grp. gram.:m.}
\end{itemize}
Apparelho, com que se ultimam certas operações, em fábricas de tecidos. Cf. \textunderscore Inquér. Industr.\textunderscore , p. II, V. III, 89 e 91.
\section{Ultimamente}
\begin{itemize}
\item {Grp. gram.:adv.}
\end{itemize}
\begin{itemize}
\item {Proveniência:(De \textunderscore último\textunderscore )}
\end{itemize}
Por último; posteriormente.
Há pouco: \textunderscore ultimamente, encontrei-o no Chiado\textunderscore .
Nos últimos tempos, nos tempos mais próximos de agora: \textunderscore ultimamente, tem tido desgostos\textunderscore .
\section{Ultimar}
\begin{itemize}
\item {Grp. gram.:v. t.}
\end{itemize}
\begin{itemize}
\item {Proveniência:(De \textunderscore último\textunderscore )}
\end{itemize}
Terminar; concluír.
\section{Últimas}
\begin{itemize}
\item {Grp. gram.:f. pl.}
\end{itemize}
\begin{itemize}
\item {Proveniência:(De \textunderscore último\textunderscore )}
\end{itemize}
O ponto extremo.
A extrema miséria.
Lance decisivo.
Hora final da vida: \textunderscore o doente está nas últimas\textunderscore .
\section{Ultimato}
\begin{itemize}
\item {Grp. gram.:m.}
\end{itemize}
\begin{itemize}
\item {Proveniência:(De \textunderscore último\textunderscore )}
\end{itemize}
Nome, que se deu ás últimas moléculas, a que um corpo póde sêr reduzido.
O mesmo ou melhor que \textunderscore ultimatum\textunderscore .
\section{Ultimatum}
\begin{itemize}
\item {fónica:má}
\end{itemize}
\begin{itemize}
\item {Grp. gram.:m.}
\end{itemize}
\begin{itemize}
\item {Utilização:Ext.}
\end{itemize}
\begin{itemize}
\item {Proveniência:(T. lat.)}
\end{itemize}
Últimas condições, que um Estado apresenta a outro, e de cuja acceitação depende o não se declarar guerra.
Resolução final e irrevogável.
\section{Último}
\begin{itemize}
\item {Grp. gram.:adj.}
\end{itemize}
\begin{itemize}
\item {Grp. gram.:M.}
\end{itemize}
\begin{itemize}
\item {Proveniência:(Lat. \textunderscore ultimus\textunderscore )}
\end{itemize}
Que está ou vem depois de todos os outros: \textunderscore último monarcha de uma dynastia\textunderscore .
O mais moderno, na ordem chronológica: \textunderscore em Setembro último\textunderscore .
Que é o mais recente: \textunderscore as últimas notícias da guerra\textunderscore .
Derradeiro.
Precedente.
Que está no lugar menos importante.
Que é o mais insignificante.
Final: \textunderscore no último instante\textunderscore .
Restante.
Extremo: \textunderscore está na última miséria\textunderscore .
Ínfimo: \textunderscore as últimas camadas sociaes\textunderscore .
Gravíssimo.
Decisivo.
Aquelle ou aquillo que vem ou está depois de todos os outros.
Aquelle ou aquillo que occupa a posição mais humilde ou ínfima.
O que é piór que todos.
\section{Ultimogênito}
\begin{itemize}
\item {Grp. gram.:m.}
\end{itemize}
\begin{itemize}
\item {Proveniência:(Do lat. \textunderscore ultimus\textunderscore  + \textunderscore genitus\textunderscore )}
\end{itemize}
O filho mais novo. Cf. Rui Barb., \textunderscore Réplica\textunderscore , 158.
\section{Ultor}
\begin{itemize}
\item {Grp. gram.:m.}
\end{itemize}
Aquelle que vinga; vingador. Cf. Rui Barb., \textunderscore Réplica\textunderscore , 158.
\section{Ultra...}
\begin{itemize}
\item {Grp. gram.:pref.}
\end{itemize}
\begin{itemize}
\item {Proveniência:(Lat. \textunderscore ultra\textunderscore )}
\end{itemize}
(designativo de \textunderscore além\textunderscore  ou de \textunderscore excesso\textunderscore )
\section{Ultraexistência}
\begin{itemize}
\item {Grp. gram.:f.}
\end{itemize}
\begin{itemize}
\item {Utilização:Des.}
\end{itemize}
Qualidade de ultraexistente.
\section{Ultraexistente}
\begin{itemize}
\item {Grp. gram.:adj.}
\end{itemize}
\begin{itemize}
\item {Proveniência:(De \textunderscore ultraexistir\textunderscore )}
\end{itemize}
Que existe àlém da morte.
\section{Ultraexistir}
\begin{itemize}
\item {Grp. gram.:v. i.}
\end{itemize}
\begin{itemize}
\item {Utilização:Des.}
\end{itemize}
\begin{itemize}
\item {Proveniência:(De \textunderscore ultra...\textunderscore  + \textunderscore existir\textunderscore )}
\end{itemize}
Existir àlém da morte.
Sobreviver ao corpo, (falando-se da alma)
\section{Ultrajador}
\begin{itemize}
\item {Grp. gram.:m.  e  adj.}
\end{itemize}
O que ultraja.
\section{Ultrajante}
\begin{itemize}
\item {Grp. gram.:adj.}
\end{itemize}
Que ultraja.
\section{Ultrajar}
\begin{itemize}
\item {Grp. gram.:v. t.}
\end{itemize}
\begin{itemize}
\item {Proveniência:(De \textunderscore ultraje\textunderscore )}
\end{itemize}
Insultar, afrontar.
Offender a dignidade de.
Diffamar.
\section{Ultraje}
\begin{itemize}
\item {Grp. gram.:m.}
\end{itemize}
\begin{itemize}
\item {Proveniência:(Do lat. hyp. \textunderscore ultraticum\textunderscore )}
\end{itemize}
Acto ou effeito de ultrajar.
Afronta.
\section{Ultrajoso}
\begin{itemize}
\item {Grp. gram.:adj.}
\end{itemize}
O mesmo que \textunderscore ultrajante\textunderscore .
\section{Ultraliberal}
\begin{itemize}
\item {Grp. gram.:m.  e  adj.}
\end{itemize}
\begin{itemize}
\item {Proveniência:(De \textunderscore ultra...\textunderscore  + \textunderscore liberal\textunderscore )}
\end{itemize}
Excessivamente liberal.
\section{Ultraliberalismo}
\begin{itemize}
\item {Grp. gram.:m.}
\end{itemize}
\begin{itemize}
\item {Proveniência:(De \textunderscore ultra...\textunderscore  + \textunderscore liberalismo\textunderscore )}
\end{itemize}
Liberalismo exaggerado.
\section{Ultramar}
\begin{itemize}
\item {Grp. gram.:m.}
\end{itemize}
\begin{itemize}
\item {Proveniência:(De \textunderscore ultra...\textunderscore  + \textunderscore mar\textunderscore )}
\end{itemize}
Região ou regiões além do mar.
Possessões ultramarinas: \textunderscore os governadores do Ultramar\textunderscore .
Tinta azul, extrahida do lápis-lazúli.
\section{Ultramarino}
\begin{itemize}
\item {Grp. gram.:adj.}
\end{itemize}
Relativo ao ultramar: \textunderscore questões ultramarinas\textunderscore .
Situado no ultramar.
\section{Ultramonárchico}
\begin{itemize}
\item {fónica:qui}
\end{itemize}
\begin{itemize}
\item {Grp. gram.:adj.}
\end{itemize}
\begin{itemize}
\item {Proveniência:(De \textunderscore ultra...\textunderscore  + \textunderscore monárchico\textunderscore )}
\end{itemize}
Excessivamente monárchico. Cf. Latino, \textunderscore Elogios\textunderscore , 195 e 336.
\section{Ultramonárquico}
\begin{itemize}
\item {Grp. gram.:adj.}
\end{itemize}
\begin{itemize}
\item {Proveniência:(De \textunderscore ultra...\textunderscore  + \textunderscore monárchico\textunderscore )}
\end{itemize}
Excessivamente monárquico. Cf. Latino, \textunderscore Elogios\textunderscore , 195 e 336.
\section{Ultramontanismo}
\begin{itemize}
\item {Grp. gram.:m.}
\end{itemize}
\begin{itemize}
\item {Proveniência:(De \textunderscore ultramontano\textunderscore )}
\end{itemize}
Systema dos Ultramontanos.
\section{Ultramontano}
\begin{itemize}
\item {Grp. gram.:adj.}
\end{itemize}
\begin{itemize}
\item {Grp. gram.:M.}
\end{itemize}
\begin{itemize}
\item {Proveniência:(De \textunderscore ultra...\textunderscore  + \textunderscore monte\textunderscore )}
\end{itemize}
Trasmontano.
Relativo aos princípios da côrte de Roma, quanto ao poder ecclesiástico.
Aquelle que sustenta ou defende o poder absoluto do Papa, na ordem espiritual e temporal.
\section{Ultramundano}
\begin{itemize}
\item {Grp. gram.:adj.}
\end{itemize}
\begin{itemize}
\item {Proveniência:(De \textunderscore ultra...\textunderscore  + \textunderscore mundano\textunderscore )}
\end{itemize}
Excessivamente mundano. Cf. Latino, \textunderscore Humboldt\textunderscore , 522.
\section{Ultraoceânico}
\begin{itemize}
\item {Grp. gram.:adj.}
\end{itemize}
\begin{itemize}
\item {Proveniência:(De \textunderscore ultra...\textunderscore  + \textunderscore oceano\textunderscore )}
\end{itemize}
Situado àlém do Oceano.
\section{Ultraparodoxal}
\begin{itemize}
\item {fónica:csal}
\end{itemize}
\begin{itemize}
\item {Grp. gram.:adj.}
\end{itemize}
\begin{itemize}
\item {Proveniência:(De \textunderscore ultra...\textunderscore  + \textunderscore parodoxal\textunderscore )}
\end{itemize}
Excessivamente parodoxal. Cf. Latino, \textunderscore Elogios\textunderscore , 62.
\section{Ultrapassar}
\begin{itemize}
\item {Grp. gram.:v. i.}
\end{itemize}
\begin{itemize}
\item {Proveniência:(De \textunderscore ultra...\textunderscore  + \textunderscore passar\textunderscore )}
\end{itemize}
Passar além de.
Transpor.
Exceder o limite de: \textunderscore ultrapassar a paciência de alguém\textunderscore .
\section{Ultrarealismo}
\begin{itemize}
\item {fónica:re}
\end{itemize}
\begin{itemize}
\item {Grp. gram.:m.}
\end{itemize}
\begin{itemize}
\item {Proveniência:(De \textunderscore ultra...\textunderscore  + \textunderscore realismo\textunderscore )}
\end{itemize}
Systema ou opinião dos ultrarealistas.
\section{Ultrarealista}
\begin{itemize}
\item {fónica:re}
\end{itemize}
\begin{itemize}
\item {Grp. gram.:adj.}
\end{itemize}
\begin{itemize}
\item {Grp. gram.:M.}
\end{itemize}
\begin{itemize}
\item {Proveniência:(De \textunderscore ultra...\textunderscore  + \textunderscore realista\textunderscore )}
\end{itemize}
Relativo ao ultrarealismo.
Partidário do despotismo ou do poder absoluto dos monarchas.
\section{Ultrarrealismo}
\begin{itemize}
\item {Grp. gram.:m.}
\end{itemize}
\begin{itemize}
\item {Proveniência:(De \textunderscore ultra...\textunderscore  + \textunderscore realismo\textunderscore )}
\end{itemize}
Sistema ou opinião dos ultrarrealistas.
\section{Ultrarrealista}
\begin{itemize}
\item {Grp. gram.:adj.}
\end{itemize}
\begin{itemize}
\item {Grp. gram.:M.}
\end{itemize}
\begin{itemize}
\item {Proveniência:(De \textunderscore ultra...\textunderscore  + \textunderscore realista\textunderscore )}
\end{itemize}
Relativo ao ultrarrealismo.
Partidário do despotismo ou do poder absoluto dos monarcas.
\section{Ultrasensível}
\begin{itemize}
\item {fónica:sen}
\end{itemize}
\begin{itemize}
\item {Grp. gram.:adj.}
\end{itemize}
\begin{itemize}
\item {Proveniência:(De \textunderscore ultra...\textunderscore  + \textunderscore sensível\textunderscore )}
\end{itemize}
Extraordinariamente sensível.
\section{Ultrassensível}
\begin{itemize}
\item {Grp. gram.:adj.}
\end{itemize}
\begin{itemize}
\item {Proveniência:(De \textunderscore ultra...\textunderscore  + \textunderscore sensível\textunderscore )}
\end{itemize}
Extraordinariamente sensível.
\section{Ultrazodiacal}
\begin{itemize}
\item {Grp. gram.:adj.}
\end{itemize}
\begin{itemize}
\item {Proveniência:(De \textunderscore ultra...\textunderscore  + \textunderscore zodiacal\textunderscore )}
\end{itemize}
Diz-se dos planetas, cuja órbita não está comprehendida na largura do Zodíaco.
\section{Ultrice}
\begin{itemize}
\item {Grp. gram.:f.}
\end{itemize}
\begin{itemize}
\item {Grp. gram.:F.}
\end{itemize}
\begin{itemize}
\item {Proveniência:(Lat. \textunderscore ultrix\textunderscore )}
\end{itemize}
Que vinga.
Mulhér vingadora, mulhér que se vinga.
\section{Ultriz}
\begin{itemize}
\item {Grp. gram.:adj. f.}
\end{itemize}
\begin{itemize}
\item {Utilização:Poét.}
\end{itemize}
\begin{itemize}
\item {Grp. gram.:F.}
\end{itemize}
\begin{itemize}
\item {Proveniência:(Lat. \textunderscore ultrix\textunderscore )}
\end{itemize}
Que vinga.
Mulhér vingadora, mulhér que se vinga.
\section{Ultróneo}
\begin{itemize}
\item {Grp. gram.:adj.}
\end{itemize}
\begin{itemize}
\item {Utilização:Des.}
\end{itemize}
\begin{itemize}
\item {Proveniência:(Lat. \textunderscore ultroneus\textunderscore )}
\end{itemize}
Espontâneo.
\section{Uluco}
\begin{itemize}
\item {Grp. gram.:m.}
\end{itemize}
Gênero de plantas portuláceas da Bolívia e do Peru, (\textunderscore ullucus tuberosus\textunderscore , Collas).
\section{Ululação}
\begin{itemize}
\item {Grp. gram.:f.}
\end{itemize}
\begin{itemize}
\item {Proveniência:(Do lat. \textunderscore ululatio\textunderscore )}
\end{itemize}
Acto ou effeito de ulular.
\section{Ululador}
\begin{itemize}
\item {Grp. gram.:m.  e  adj.}
\end{itemize}
O que ulula.
\section{Ululante}
\begin{itemize}
\item {Grp. gram.:adj.}
\end{itemize}
\begin{itemize}
\item {Proveniência:(Lat. \textunderscore ululans\textunderscore )}
\end{itemize}
Que ulula; lamentoso.
\section{Ulular}
\begin{itemize}
\item {Grp. gram.:v. i.}
\end{itemize}
\begin{itemize}
\item {Utilização:Fig.}
\end{itemize}
\begin{itemize}
\item {Grp. gram.:V. t.}
\end{itemize}
\begin{itemize}
\item {Grp. gram.:M.}
\end{itemize}
\begin{itemize}
\item {Proveniência:(Lat. \textunderscore ululare\textunderscore )}
\end{itemize}
Soltar voz triste e lamentosa.
Uivar; gannir.
Gritar afflictivamente.
Queixar-se, gritando.
Exprimir, gritando lamentosamente.
Proferir, berrando.
Bradar.
Ululação.
\section{Ululato}
\begin{itemize}
\item {Grp. gram.:m.}
\end{itemize}
O mesmo que \textunderscore ululação\textunderscore . Cf. Júl. Ribeiro, \textunderscore Padre Belch.\textunderscore , 137.
\section{Ulva}
\begin{itemize}
\item {Grp. gram.:f.}
\end{itemize}
\begin{itemize}
\item {Proveniência:(Lat. \textunderscore ulva\textunderscore )}
\end{itemize}
Gênero de algas, que nascem nos paues e á beira de águas estagnadas.
\section{Ulváceas}
\begin{itemize}
\item {Grp. gram.:f. pl.}
\end{itemize}
Família de plantas, que tem tem por typo a ulva.
\section{Ulvina}
\begin{itemize}
\item {Grp. gram.:f.}
\end{itemize}
\begin{itemize}
\item {Proveniência:(Do lat. \textunderscore ulva\textunderscore )}
\end{itemize}
Gênero de algas, de que há quatro espécies.
\section{Um}
\begin{itemize}
\item {Grp. gram.:adj.}
\end{itemize}
\begin{itemize}
\item {Grp. gram.:Art.}
\end{itemize}
\begin{itemize}
\item {Grp. gram.:M.}
\end{itemize}
\begin{itemize}
\item {Utilização:Prov.}
\end{itemize}
\begin{itemize}
\item {Utilização:alg.}
\end{itemize}
\begin{itemize}
\item {Proveniência:(Do lat. \textunderscore unus\textunderscore )}
\end{itemize}
Diz-se do número cardinal, que exprime uma só unidade: \textunderscore um homem\textunderscore .
Que é o primeiro de todos os números.
Uno; único.
Que se distingue de todos os outros.
Contínuo, indivisível.
Qualquer.
Algum; certo.
Algarismo, que representa o número um.
Homem sem préstimo.
\section{Um}
\begin{itemize}
\item {Grp. gram.:m.}
\end{itemize}
Árvore de Damão, (\textunderscore guatteria cerasoides\textunderscore ).
\section{Uma}
\begin{itemize}
\item {Grp. gram.:adj.}
\end{itemize}
\begin{itemize}
\item {Grp. gram.:Loc. adv.}
\end{itemize}
\begin{itemize}
\item {Proveniência:(Do lat. \textunderscore una\textunderscore )}
\end{itemize}
(Fem. de \textunderscore um\textunderscore )
\textunderscore Á uma\textunderscore , juntamente; ao mesmo tempo.
Por um lado.
\section{Umans}
\begin{itemize}
\item {Grp. gram.:m. pl.}
\end{itemize}
Antiga tríbo de Índios de Pernambuco.
\section{Umar}
\begin{itemize}
\item {Grp. gram.:v. i.}
\end{itemize}
\begin{itemize}
\item {Utilização:Prov.}
\end{itemize}
\begin{itemize}
\item {Utilização:minh.}
\end{itemize}
Ganhar umidade e estragar-se, (falando-se da madeira). (Colhido em Barcelos)
(Cp. \textunderscore úmido\textunderscore )
\section{Umari}
\begin{itemize}
\item {Grp. gram.:m.}
\end{itemize}
Nome de duas plantas leguminosas do Brasil.
\section{Umauás}
\begin{itemize}
\item {Grp. gram.:m. pl.}
\end{itemize}
Indígenas do Norte do Brasil.
\section{Umbala}
\begin{itemize}
\item {Grp. gram.:f.}
\end{itemize}
\begin{itemize}
\item {Utilização:T. da África Port}
\end{itemize}
O mesmo que \textunderscore libata\textunderscore .
\section{Umbamba}
\begin{itemize}
\item {Grp. gram.:f.}
\end{itemize}
Espécie de palmeira do Brasil.
\section{Umbaru}
\begin{itemize}
\item {Grp. gram.:m.}
\end{itemize}
Planta meliácea do Brasil.
\section{Umbaúba}
\begin{itemize}
\item {Grp. gram.:f.}
\end{itemize}
Árvore urticácea, (\textunderscore cecropia palmata\textunderscore ).
\section{Umbela}
\begin{itemize}
\item {Grp. gram.:f.}
\end{itemize}
\begin{itemize}
\item {Utilização:Bot.}
\end{itemize}
\begin{itemize}
\item {Proveniência:(Lat. \textunderscore umbella\textunderscore )}
\end{itemize}
Guarda-sol; sombrinha.
Pequeno pállio redondo.
Inflorescência, formada por eixos que, partindo do mesmo ponto, chegam á mesma altura, produzindo uma superfície convexa, á semelhança de um guarda-chuva, como na flôr das assembleias, (plantas).
\section{Umbeladas}
\begin{itemize}
\item {Grp. gram.:f. pl.}
\end{itemize}
O mesmo que \textunderscore umbelíferas\textunderscore .
\section{Umbelado}
\begin{itemize}
\item {Grp. gram.:adj.}
\end{itemize}
\begin{itemize}
\item {Proveniência:(De \textunderscore umbela\textunderscore )}
\end{itemize}
O mesmo que \textunderscore umbelífero\textunderscore . Cf. Júl. Dinis, \textunderscore Morgadinha\textunderscore , 33.
\section{Umbelíferas}
\begin{itemize}
\item {Grp. gram.:f. pl.}
\end{itemize}
Família de plantas dicotiledóneas, cuja inflorescência toma a fórma de umbela.
(Fem. pl. de \textunderscore umbelífero\textunderscore )
\section{Umbella}
\begin{itemize}
\item {Grp. gram.:f.}
\end{itemize}
\begin{itemize}
\item {Utilização:Bot.}
\end{itemize}
\begin{itemize}
\item {Proveniência:(Lat. \textunderscore umbella\textunderscore )}
\end{itemize}
Guarda-sol; sombrinha.
Pequeno pállio redondo.
Inflorescência, formada por eixos que, partindo do mesmo ponto, chegam á mesma altura, produzindo uma superfície convexa, á semelhança de um guarda-chuva, como na flôr das assembleias, (plantas).
\section{Umbelladas}
\begin{itemize}
\item {Grp. gram.:f. pl.}
\end{itemize}
O mesmo que \textunderscore umbellíferas\textunderscore .
\section{Umbellado}
\begin{itemize}
\item {Grp. gram.:adj.}
\end{itemize}
\begin{itemize}
\item {Proveniência:(De \textunderscore umbella\textunderscore )}
\end{itemize}
O mesmo que \textunderscore umbellífero\textunderscore . Cf. Júl. Dinis, \textunderscore Morgadinha\textunderscore , 33.
\section{Umbellíferas}
\begin{itemize}
\item {Grp. gram.:f. pl.}
\end{itemize}
Família de plantas dicotyledóneas, cuja inflorescência toma a fórma de umbella.
(Fem. pl. de \textunderscore umbellífero\textunderscore )
\section{Umbelífero}
\begin{itemize}
\item {Grp. gram.:adj.}
\end{itemize}
\begin{itemize}
\item {Utilização:Bot.}
\end{itemize}
\begin{itemize}
\item {Proveniência:(Do lat. \textunderscore umbella\textunderscore  + \textunderscore ferre\textunderscore )}
\end{itemize}
Que tem umbela.
Que tem flôres dispostas em umbela.
\section{Umbellífero}
\begin{itemize}
\item {Grp. gram.:adj.}
\end{itemize}
\begin{itemize}
\item {Utilização:Bot.}
\end{itemize}
\begin{itemize}
\item {Proveniência:(Do lat. \textunderscore umbella\textunderscore  + \textunderscore ferre\textunderscore )}
\end{itemize}
Que tem umbella.
Que tem flôres dispostas em umbella.
\section{Umbéllula}
\begin{itemize}
\item {Grp. gram.:f.}
\end{itemize}
Umbella pequena.
\section{Umbellulária}
\begin{itemize}
\item {Grp. gram.:f.}
\end{itemize}
\begin{itemize}
\item {Proveniência:(De \textunderscore umbéllula\textunderscore )}
\end{itemize}
Animal ou grupo de animaes, que têm um caule commum e fixo, como as plantas, o qual termina superiormente numa inflorescência em capítulo.--É da Groenlândia. Cf. Caminhoá, \textunderscore Bot. Ger. e Med.\textunderscore 
\section{Umbélula}
\begin{itemize}
\item {Grp. gram.:f.}
\end{itemize}
Umbela pequena.
\section{Umbelulária}
\begin{itemize}
\item {Grp. gram.:f.}
\end{itemize}
\begin{itemize}
\item {Proveniência:(De \textunderscore umbélula\textunderscore )}
\end{itemize}
Animal ou grupo de animaes, que têm um caule comum e fixo, como as plantas, o qual termina superiormente numa inflorescência em capítulo.--É da Groenlândia. Cf. Caminhoá, \textunderscore Bot. Ger. e Med.\textunderscore 
\section{Umbigada}
\begin{itemize}
\item {Grp. gram.:f.}
\end{itemize}
Pancada com o umbigo ou com a barriga.
Região do umbigo.
\section{Umbigo}
\begin{itemize}
\item {Grp. gram.:m.}
\end{itemize}
\begin{itemize}
\item {Utilização:Anat.}
\end{itemize}
\begin{itemize}
\item {Utilização:Ext.}
\end{itemize}
\begin{itemize}
\item {Proveniência:(Do lat. \textunderscore umbilicus\textunderscore )}
\end{itemize}
Cicatriz deprimida ou saliente, resultante do córte do cordão umbilical.
Centro.
\section{Umbigo-de-freira}
\begin{itemize}
\item {Grp. gram.:m.}
\end{itemize}
\begin{itemize}
\item {Utilização:Bras}
\end{itemize}
Espécie de biscoitos doces, que se servem ao chá.
\section{Umbigo-de-vênus}
\begin{itemize}
\item {Grp. gram.:m.}
\end{itemize}
\begin{itemize}
\item {Utilização:Bot.}
\end{itemize}
O mesmo que \textunderscore coucelo\textunderscore .
\section{Umbilicado}
\begin{itemize}
\item {Grp. gram.:adj.}
\end{itemize}
\begin{itemize}
\item {Proveniência:(Do lat. \textunderscore umbilicatus\textunderscore )}
\end{itemize}
Semelhante ao umbigo.
\section{Umbilical}
\begin{itemize}
\item {Grp. gram.:adj.}
\end{itemize}
\begin{itemize}
\item {Proveniência:(Do lat. \textunderscore umbilicus\textunderscore )}
\end{itemize}
Relativo ao umbigo.
Diz-se do cordão, que liga o féto á placenta e que, durante a gestação, lhe transmitte os sucos nutritivos.
\section{Umbla}
\begin{itemize}
\item {Grp. gram.:f.}
\end{itemize}
\begin{itemize}
\item {Proveniência:(Do fr. \textunderscore umbre\textunderscore )}
\end{itemize}
Espécie de salmão.
\section{Umblina}
\begin{itemize}
\item {Grp. gram.:f.}
\end{itemize}
\begin{itemize}
\item {Proveniência:(De \textunderscore umbla\textunderscore )}
\end{itemize}
Peixe esquamodermo.
\section{Umblo}
\begin{itemize}
\item {Grp. gram.:m.}
\end{itemize}
Árvore da ilha de San-Thomé.
\section{Umbraculíferas}
\begin{itemize}
\item {Grp. gram.:f. pl.}
\end{itemize}
Ordem de plantas, que comprehende as umbellíferas, as araliáceas e as cornuáceas.
(Fem. pl. de \textunderscore umbraculífero\textunderscore )
\section{Umbraculífero}
\begin{itemize}
\item {Grp. gram.:adj.}
\end{itemize}
\begin{itemize}
\item {Utilização:Hist. Nat.}
\end{itemize}
\begin{itemize}
\item {Proveniência:(Do lat. \textunderscore umbraculum\textunderscore  + \textunderscore ferre\textunderscore )}
\end{itemize}
Que tem órgão em fórma de umbella.
\section{Umbraculiforme}
\begin{itemize}
\item {Grp. gram.:adj.}
\end{itemize}
\begin{itemize}
\item {Utilização:Bot.}
\end{itemize}
\begin{itemize}
\item {Proveniência:(Do lat. \textunderscore umbraculum\textunderscore  + \textunderscore forma\textunderscore )}
\end{itemize}
Que tem fórma de umbella.
\section{Umbráculo}
\begin{itemize}
\item {Grp. gram.:m.}
\end{itemize}
\begin{itemize}
\item {Utilização:Bot.}
\end{itemize}
\begin{itemize}
\item {Proveniência:(Lat. \textunderscore umbraculum\textunderscore )}
\end{itemize}
Espécie de disco, que corôa o pedúnculo de algumas plantas cryptogâmicas.
\section{Umbral}
\begin{itemize}
\item {Grp. gram.:m.}
\end{itemize}
O mesmo ou melhor que \textunderscore humbral\textunderscore . Cf. Herculano, \textunderscore Cistér\textunderscore , 52.
\section{Umbrático}
\begin{itemize}
\item {Grp. gram.:adj.}
\end{itemize}
\begin{itemize}
\item {Utilização:Poét.}
\end{itemize}
\begin{itemize}
\item {Proveniência:(Lat. \textunderscore umbraticus\textunderscore )}
\end{itemize}
Relativo a sombra.
Que se deleita com a sombra ou que a procura.
Obscuro.
Imaginário.
\section{Umbratícola}
\begin{itemize}
\item {Grp. gram.:adj.}
\end{itemize}
\begin{itemize}
\item {Utilização:Hist. Nat.}
\end{itemize}
\begin{itemize}
\item {Proveniência:(Do lat. \textunderscore umbraticus\textunderscore  + \textunderscore colere\textunderscore )}
\end{itemize}
Que vive ou cresce em lugares sombrios.
\section{Umbrátil}
\begin{itemize}
\item {Grp. gram.:adj.}
\end{itemize}
\begin{itemize}
\item {Proveniência:(Lat. \textunderscore umbratilis\textunderscore )}
\end{itemize}
Umbrático.
Fantástico; allegórico.
\section{Umbrela}
\begin{itemize}
\item {Grp. gram.:f.}
\end{itemize}
O mesmo que \textunderscore umbela\textunderscore .
Gênero de moluscos gasterópodes.
\section{Umbrelado}
\begin{itemize}
\item {Grp. gram.:adj.}
\end{itemize}
O mesmo que \textunderscore umbelífero\textunderscore .
\section{Umbrella}
\begin{itemize}
\item {Grp. gram.:f.}
\end{itemize}
O mesmo que \textunderscore umbella\textunderscore .
Gênero de molluscos gasterópodes.
\section{Umbrellado}
\begin{itemize}
\item {Grp. gram.:adj.}
\end{itemize}
O mesmo que \textunderscore umbellífero\textunderscore .
\section{Umbreta}
\begin{itemize}
\item {fónica:brê}
\end{itemize}
\begin{itemize}
\item {Grp. gram.:f.}
\end{itemize}
Gênero de aves pernaltas.
\section{Umbria}
\begin{itemize}
\item {Grp. gram.:f.}
\end{itemize}
\begin{itemize}
\item {Utilização:Poét.}
\end{itemize}
\begin{itemize}
\item {Utilização:Prov.}
\end{itemize}
\begin{itemize}
\item {Utilização:alent.}
\end{itemize}
Lugar sombrio.
Vertente occidental de um monte.
Vertente setentrional de um monte, onde o mato cresce mais.--Os diccion. trazem a palavra como proparoxýtona; considero-a porém paroxýtona, não só porque assim o é em cast., mas principalmente porque é derivada e tem um suff. que, exprimindo \textunderscore qualidade\textunderscore , nunca é átono.
(Cast. \textunderscore umbría\textunderscore )
\section{Úmbrico}
\begin{itemize}
\item {Grp. gram.:adj.}
\end{itemize}
\begin{itemize}
\item {Grp. gram.:M.}
\end{itemize}
\begin{itemize}
\item {Proveniência:(De \textunderscore Úmbria\textunderscore , n. p.)}
\end{itemize}
Relativo a Úmbria ou aos Umbros.
Antigo dialecto do ramo itálico.
\section{Umbrícola}
\begin{itemize}
\item {Grp. gram.:adj.}
\end{itemize}
\begin{itemize}
\item {Proveniência:(Lat. \textunderscore umbricola\textunderscore )}
\end{itemize}
Que vive nas sombras.
\section{Umbrífero}
\begin{itemize}
\item {Grp. gram.:adj.}
\end{itemize}
\begin{itemize}
\item {Proveniência:(Lat. \textunderscore umbrifer\textunderscore )}
\end{itemize}
Sombrio, umbroso.
\section{Umbrinos}
\begin{itemize}
\item {Grp. gram.:m. Pl.}
\end{itemize}
\begin{itemize}
\item {Proveniência:(Do lat. \textunderscore umbra\textunderscore )}
\end{itemize}
Gênero de peixes, semelhantes ás percas, (\textunderscore umbrina communis\textunderscore ).
\section{Umbro}
\begin{itemize}
\item {Grp. gram.:m.}
\end{itemize}
\begin{itemize}
\item {Proveniência:(Do lat. \textunderscore umber\textunderscore )}
\end{itemize}
Cão para caçar veados.
O dialecto úmbrico.
\section{Umbror}
\begin{itemize}
\item {Grp. gram.:m.}
\end{itemize}
\begin{itemize}
\item {Proveniência:(Do lat. \textunderscore umbra\textunderscore )}
\end{itemize}
Conjunto de sombras ou nuvens. Cf. Camillo, \textunderscore Cancion. Al.\textunderscore , 143.
\section{Umbros}
\begin{itemize}
\item {Grp. gram.:m. Pl.}
\end{itemize}
\begin{itemize}
\item {Proveniência:(Lat. \textunderscore umbri\textunderscore )}
\end{itemize}
Antigo povo italiano, que habitou entre o Tibre e o Adriático.
\section{Umbroso}
\begin{itemize}
\item {Grp. gram.:adj.}
\end{itemize}
\begin{itemize}
\item {Proveniência:(Lat. \textunderscore umbrosus\textunderscore )}
\end{itemize}
Que tem ou produz sombra.
Sombrio, copado; escuro.
\section{Umbu}
\begin{itemize}
\item {Grp. gram.:m.}
\end{itemize}
O mesmo que \textunderscore imbu\textunderscore .
\section{Umbu}
\begin{itemize}
\item {Grp. gram.:m.}
\end{itemize}
\begin{itemize}
\item {Utilização:Bras. do N}
\end{itemize}
Grande árvore phytollácea da América.
Fruto do imbuzeiro.
\section{Umbula}
\begin{itemize}
\item {Grp. gram.:f.}
\end{itemize}
Árvore angolense da Caconda.
\section{Umbuzada}
\begin{itemize}
\item {Grp. gram.:f.}
\end{itemize}
O mesmo que \textunderscore imbuzada\textunderscore .
\section{Umbuzeiro}
\begin{itemize}
\item {Grp. gram.:m.}
\end{itemize}
\begin{itemize}
\item {Utilização:Bras}
\end{itemize}
Árvore, o mesmo que \textunderscore umbu\textunderscore ^2.
\section{Ume}
\begin{itemize}
\item {Grp. gram.:adj.}
\end{itemize}
\begin{itemize}
\item {Grp. gram.:M.}
\end{itemize}
\begin{itemize}
\item {Proveniência:(De \textunderscore alume\textunderscore )}
\end{itemize}
Diz-se de uma pedra, que é um sulphato de alumina e potassa.
O mesmo que \textunderscore alúmen\textunderscore .
Cp. \textunderscore pedra-ume\textunderscore .
\section{Úmero}
\begin{itemize}
\item {Grp. gram.:m.}
\end{itemize}
\begin{itemize}
\item {Proveniência:(Lat. \textunderscore humerus\textunderscore , ou \textunderscore umerus\textunderscore )}
\end{itemize}
\begin{itemize}
\item {Grp. gram.:m.}
\end{itemize}
\begin{itemize}
\item {Proveniência:(Lat. \textunderscore umerus\textunderscore )}
\end{itemize}
Parte do braço, comprehendida entre o cotovelo e a espádua.
Fórma preferível a \textunderscore húmero\textunderscore .
\section{Úmido}
\begin{itemize}
\item {Proveniência:(Lat. \textunderscore umidus\textunderscore )}
\end{itemize}
\begin{itemize}
\item {Grp. gram.:adj.}
\end{itemize}
\begin{itemize}
\item {Proveniência:(Lat. \textunderscore humidus\textunderscore , ou \textunderscore umidus\textunderscore )}
\end{itemize}
\textunderscore adj.\textunderscore  (e der.)
O mesmo ou melhór que \textunderscore húmido\textunderscore , etc.
Levemente molhado; que tem a natureza da água.
Aquoso; impregnando de vapores aquosos.
\section{Umiri}
\begin{itemize}
\item {Grp. gram.:m.}
\end{itemize}
Planta meliácea do Brasil, (\textunderscore umirium balsamiferum\textunderscore ).
Nome de um óleo, extrahido da casca dessa árvore.
\section{Umpada}
\begin{itemize}
\item {Grp. gram.:f.}
\end{itemize}
Árvore angolense.
\section{Umperevu}
\begin{itemize}
\item {Grp. gram.:m.}
\end{itemize}
Árvore de Moçambique.
\section{Unanimar}
\begin{itemize}
\item {Grp. gram.:v. t.}
\end{itemize}
Tornar unânime.
\section{Unânime}
\begin{itemize}
\item {Grp. gram.:adj.}
\end{itemize}
\begin{itemize}
\item {Proveniência:(Lat. \textunderscore unanimus\textunderscore )}
\end{itemize}
Que tem o mesmo sentimento ou a mesma opinião que outrem.
Relativo a todos, geral.
Proveniente de acôrdo commum; concorde: \textunderscore por votação unânime\textunderscore .
\section{Unanimemente}
\begin{itemize}
\item {Grp. gram.:adv.}
\end{itemize}
De modo unânime.
\section{Unanimidade}
\begin{itemize}
\item {Grp. gram.:f.}
\end{itemize}
\begin{itemize}
\item {Proveniência:(Do lat. \textunderscore unanimitas\textunderscore )}
\end{itemize}
Qualidade do que é unânime.
Conformidade de opinião ou de voto.
\section{Unau}
\begin{itemize}
\item {Grp. gram.:m.}
\end{itemize}
Mammífero tardígrado, da América do Sul.
\section{Unça}
\begin{itemize}
\item {Grp. gram.:adj. f.}
\end{itemize}
\begin{itemize}
\item {Utilização:T. de Alcanena}
\end{itemize}
Diz-se de uma erva aromática, silvestre, de fôlhas miúdas, que se emprega na curtimenta da azeitona.
\section{Unção}
\begin{itemize}
\item {Grp. gram.:f.}
\end{itemize}
\begin{itemize}
\item {Utilização:Fig.}
\end{itemize}
\begin{itemize}
\item {Proveniência:(Do lat. \textunderscore unctio\textunderscore )}
\end{itemize}
Acto ou effeito de ungir ou untar.
Sentimento de piedade.
Doçura commovente na expressão.
Modo insinuante de dizer.
\section{Uncção}
\begin{itemize}
\item {Grp. gram.:f.}
\end{itemize}
\begin{itemize}
\item {Utilização:Fig.}
\end{itemize}
\begin{itemize}
\item {Proveniência:(Do lat. \textunderscore unctio\textunderscore )}
\end{itemize}
Acto ou effeito de ungir ou untar.
Sentimento de piedade.
Doçura commovente na expressão.
Modo insinuante de dizer.
\section{Úncia}
\begin{itemize}
\item {Grp. gram.:f.}
\end{itemize}
\begin{itemize}
\item {Proveniência:(Lat. \textunderscore uneia\textunderscore )}
\end{itemize}
O mesmo que \textunderscore pollegada\textunderscore . Cf. Castilho, \textunderscore Fastos\textunderscore , 353 e 387.
\section{Uncial}
\begin{itemize}
\item {Grp. gram.:adj.}
\end{itemize}
\begin{itemize}
\item {Proveniência:(Lat. \textunderscore uncialis\textunderscore )}
\end{itemize}
Dizia-se das letras ou caracteres maiúsculos, em que eram escritos os textos ecclesiásticos medievaes até o século XI.--Os minúsculos começaram a usar-so no século X, reservando-se as letras unciaes para os títulos das obras e dos capítulos. A Escritura \textunderscore semi-uncial\textunderscore  era mais pequena que a uncial ordinária. Cf. Mélida, \textunderscore Vocab. de Term.\textunderscore 
\section{Unciário}
\begin{itemize}
\item {Grp. gram.:adj.}
\end{itemize}
\begin{itemize}
\item {Proveniência:(Lat. \textunderscore unciarius\textunderscore )}
\end{itemize}
Que tinha direito á duodécima parte de uma herança, segundo a jurisprudência romana.
\section{Unciforme}
\begin{itemize}
\item {Grp. gram.:adj.}
\end{itemize}
\begin{itemize}
\item {Grp. gram.:M.}
\end{itemize}
\begin{itemize}
\item {Utilização:Anat.}
\end{itemize}
\begin{itemize}
\item {Proveniência:(Do lat. \textunderscore uncus\textunderscore  + \textunderscore forma\textunderscore )}
\end{itemize}
Que tem fórma de gancho.
O quarto osso da segunda série do corpo.
\section{Uncinado}
\begin{itemize}
\item {Grp. gram.:adj.}
\end{itemize}
\begin{itemize}
\item {Proveniência:(Do lat. \textunderscore uncinatus\textunderscore )}
\end{itemize}
Que tem unha.
Que tem fórma de unha ou de garra.
Que termina em unha.
\section{Uncínia}
\begin{itemize}
\item {Grp. gram.:f.}
\end{itemize}
Gênero do plantas cyperáceas.
(Cp. \textunderscore uncinado\textunderscore )
\section{Uncirostro}
\begin{itemize}
\item {fónica:rós}
\end{itemize}
\begin{itemize}
\item {Grp. gram.:adj.}
\end{itemize}
\begin{itemize}
\item {Grp. gram.:M. Pl.}
\end{itemize}
\begin{itemize}
\item {Proveniência:(Do lat. \textunderscore uncus\textunderscore  + \textunderscore rostrum\textunderscore )}
\end{itemize}
Que tem bico curvo, em fórma de unha.
Família de aves pernaltas, de bico adunco.
\section{Uncirrostro}
\begin{itemize}
\item {Grp. gram.:adj.}
\end{itemize}
\begin{itemize}
\item {Grp. gram.:M. Pl.}
\end{itemize}
\begin{itemize}
\item {Proveniência:(Do lat. \textunderscore uncus\textunderscore  + \textunderscore rostrum\textunderscore )}
\end{itemize}
Que tem bico curvo, em fórma de unha.
Família de aves pernaltas, de bico adunco.
\section{Unctório}
\begin{itemize}
\item {Grp. gram.:m.}
\end{itemize}
\begin{itemize}
\item {Proveniência:(Lat. \textunderscore unctorium\textunderscore )}
\end{itemize}
Compartimento ou sala, nas casa de banhos entre os Romanos, na qual os banhistas se friccionavam com perfumes, e onde êstes se guardavam.
\section{Unctuosamente}
\begin{itemize}
\item {Grp. gram.:adv.}
\end{itemize}
\begin{itemize}
\item {Utilização:Fig.}
\end{itemize}
De modo unctuoso.
Com uncção.
Suavemente; mellifluamente.
\section{Unctuosidade}
\begin{itemize}
\item {Grp. gram.:f.}
\end{itemize}
Estado ou qualidade de unctuoso.
Qualidade de gorduroso ou escorregadio.
\section{Unctuoso}
\begin{itemize}
\item {Grp. gram.:adj.}
\end{itemize}
\begin{itemize}
\item {Utilização:Fig.}
\end{itemize}
\begin{itemize}
\item {Proveniência:(Lat. \textunderscore unctuosus\textunderscore )}
\end{itemize}
Em que há unto ou gordura.
Gorduroso.
Lubrificado, escorregadio.
Que dá ao tacto a impressão das substâncias gordurosas.
Relativo a iguarias, em que predominam môlhos e carnes:«\textunderscore dava cama e mesa unctuosa aos missionários...\textunderscore »Camillo, \textunderscore Brasileira\textunderscore , 237.
Suave; amorável; meigo; mellífluo.
\section{Undação}
\begin{itemize}
\item {Grp. gram.:f.}
\end{itemize}
\begin{itemize}
\item {Proveniência:(Do lat. \textunderscore undatio\textunderscore )}
\end{itemize}
Corrente de rio; inundação; enchente.
\section{Undai}
\begin{itemize}
\item {Grp. gram.:m.}
\end{itemize}
Planta angolense, (\textunderscore gardenia Jovis tonantis\textunderscore ).
\section{Undalo}
\begin{itemize}
\item {Grp. gram.:m.}
\end{itemize}
Pássaro dentirostro africano.
\section{Undante}
\begin{itemize}
\item {Grp. gram.:adj.}
\end{itemize}
\begin{itemize}
\item {Proveniência:(Lat. \textunderscore undans\textunderscore )}
\end{itemize}
Que fórma ondas; ondeante.
Que tem ou leva muita água.
\section{Unde}
\begin{itemize}
\item {Grp. gram.:adv.}
\end{itemize}
\begin{itemize}
\item {Utilização:Obsol.}
\end{itemize}
\begin{itemize}
\item {Proveniência:(Lat. \textunderscore unde\textunderscore )}
\end{itemize}
Portanto, por consequência.
\section{Undecágono}
\begin{itemize}
\item {Grp. gram.:m.}
\end{itemize}
O mesmo que \textunderscore hendecágono\textunderscore .
\section{Undecêmviro}
\begin{itemize}
\item {Grp. gram.:m.}
\end{itemize}
\begin{itemize}
\item {Proveniência:(Do lat. \textunderscore undecim\textunderscore  + \textunderscore vir\textunderscore )}
\end{itemize}
Cada um dos onze magistrados athenienses, a quem cumpria conduzir ao patíbulo os condemnados á pena última.
\section{Undecênviro}
\begin{itemize}
\item {Grp. gram.:m.}
\end{itemize}
\begin{itemize}
\item {Proveniência:(Do lat. \textunderscore undecim\textunderscore  + \textunderscore vir\textunderscore )}
\end{itemize}
Cada um dos onze magistrados atenienses, a quem cumpria conduzir ao patíbulo os condenados á pena última.
\section{Undecimanos}
\begin{itemize}
\item {Grp. gram.:m. pl.}
\end{itemize}
O mesmo que \textunderscore undecumanos\textunderscore .
\section{Undécimo}
\begin{itemize}
\item {Grp. gram.:adj.}
\end{itemize}
\begin{itemize}
\item {Grp. gram.:M.}
\end{itemize}
\begin{itemize}
\item {Proveniência:(Lat. \textunderscore undecimus\textunderscore )}
\end{itemize}
Que numa série de onze occupa o último lugar.
A undécima parte.
\section{Undecumanos}
\begin{itemize}
\item {Grp. gram.:m. pl.}
\end{itemize}
\begin{itemize}
\item {Proveniência:(Lat. \textunderscore undecumani\textunderscore )}
\end{itemize}
Soldados da 11.^a legião, entre os antigos Romanos.
\section{Undécuplo}
\begin{itemize}
\item {Grp. gram.:adj.}
\end{itemize}
\begin{itemize}
\item {Grp. gram.:M.}
\end{itemize}
\begin{itemize}
\item {Proveniência:(Lat. hyp. \textunderscore undecuplus\textunderscore )}
\end{itemize}
Diz-se de uma quantidade, que é onze vezes maiór que outra com que se compara.
Quantidade, onze vezes maiór que outra.
\section{Undeira}
\begin{itemize}
\item {Grp. gram.:f.}
\end{itemize}
Árvore da Índia Portuguesa, o mesmo que \textunderscore ponaca\textunderscore .
\section{Undevicesimano}
\begin{itemize}
\item {Grp. gram.:m.}
\end{itemize}
\begin{itemize}
\item {Proveniência:(Lat. \textunderscore undevicesimanus\textunderscore )}
\end{itemize}
Soldado da 19.^a legião, entre os Romanos antigos.
\section{Undevicésimo}
\begin{itemize}
\item {Grp. gram.:adj.}
\end{itemize}
\begin{itemize}
\item {Proveniência:(Lat. \textunderscore undevicesimus\textunderscore )}
\end{itemize}
O mesmo que \textunderscore décimonono\textunderscore .
\section{Undícola}
\begin{itemize}
\item {Grp. gram.:m. ,  f.  e  adj.}
\end{itemize}
\begin{itemize}
\item {Utilização:P. us.}
\end{itemize}
\begin{itemize}
\item {Proveniência:(Lat. \textunderscore undicola\textunderscore )}
\end{itemize}
Que vive nas águas.
\section{Undífero}
\begin{itemize}
\item {Grp. gram.:adj.}
\end{itemize}
\begin{itemize}
\item {Proveniência:(Do lat. \textunderscore unda\textunderscore  + \textunderscore ferre\textunderscore )}
\end{itemize}
Que tem ondas.
Que contém águas. Cf. Filinto, X, 7.
\section{Undiflavo}
\begin{itemize}
\item {Grp. gram.:adj.}
\end{itemize}
\begin{itemize}
\item {Utilização:Poét.}
\end{itemize}
\begin{itemize}
\item {Proveniência:(Do lat. \textunderscore unda\textunderscore  + \textunderscore flavus\textunderscore )}
\end{itemize}
Que tem ondas côr de oiro ou reflexos áureos.
\section{Undífluo}
\begin{itemize}
\item {Grp. gram.:adj.}
\end{itemize}
\begin{itemize}
\item {Proveniência:(Lat. \textunderscore undifluus\textunderscore )}
\end{itemize}
Que corre em ondas. Cf. Filinto, XVI, 168.
\section{Undísono}
\begin{itemize}
\item {fónica:so}
\end{itemize}
\begin{itemize}
\item {Grp. gram.:adj.}
\end{itemize}
\begin{itemize}
\item {Utilização:Poét.}
\end{itemize}
\begin{itemize}
\item {Proveniência:(Lat. \textunderscore undisonus\textunderscore )}
\end{itemize}
Que sôa como as ondas agitadas.
\section{Undíssono}
\begin{itemize}
\item {Grp. gram.:adj.}
\end{itemize}
\begin{itemize}
\item {Utilização:Poét.}
\end{itemize}
\begin{itemize}
\item {Proveniência:(Lat. \textunderscore undisonus\textunderscore )}
\end{itemize}
Que sôa como as ondas agitadas.
\section{Undívago}
\begin{itemize}
\item {Grp. gram.:adj.}
\end{itemize}
\begin{itemize}
\item {Utilização:Poét.}
\end{itemize}
\begin{itemize}
\item {Proveniência:(Lat. \textunderscore undivagus\textunderscore )}
\end{itemize}
Que anda sôbre as ondas.
\section{Undo}
\begin{itemize}
\item {Grp. gram.:m.}
\end{itemize}
Grande peixe africano. Cf. Serpa Pinto, I, 299.
\section{Undoso}
\begin{itemize}
\item {Grp. gram.:adj.}
\end{itemize}
\begin{itemize}
\item {Proveniência:(Lat. \textunderscore undosus\textunderscore )}
\end{itemize}
Em que há ondas; que fórma ondas; ondeante.
\section{Undular}
\begin{itemize}
\item {Grp. gram.:v. i.}
\end{itemize}
O mesmo que \textunderscore ondular\textunderscore :«\textunderscore mocinho entretanto, de coma a undular...\textunderscore »Castilho, \textunderscore Lir. de Anacr.\textunderscore , 37.
\section{Undulosamente}
\begin{itemize}
\item {Grp. gram.:adv.}
\end{itemize}
De modo unduloso; á maneira de ondas.
\section{Unduloso}
\begin{itemize}
\item {Grp. gram.:adj.}
\end{itemize}
\begin{itemize}
\item {Proveniência:(De \textunderscore undular\textunderscore )}
\end{itemize}
O mesmo que \textunderscore ondeante\textunderscore .
\section{Únea-golina}
\begin{itemize}
\item {Grp. gram.:f.}
\end{itemize}
Árvore da ilha de San-Thomé.
\section{Ungã}
\begin{itemize}
\item {Grp. gram.:m.}
\end{itemize}
Tambor de honra, usado no Daomé.
\section{Ungan}
\begin{itemize}
\item {Grp. gram.:m.}
\end{itemize}
Tambor de honra, usado no Daomé.
\section{Ungido}
\begin{itemize}
\item {Grp. gram.:adj.}
\end{itemize}
\begin{itemize}
\item {Grp. gram.:M.}
\end{itemize}
\begin{itemize}
\item {Proveniência:(De \textunderscore ungir\textunderscore )}
\end{itemize}
Untado.
Aquelle que se ungiu.
\section{Ungir}
\begin{itemize}
\item {Grp. gram.:v. t.}
\end{itemize}
\begin{itemize}
\item {Utilização:Fig.}
\end{itemize}
\begin{itemize}
\item {Proveniência:(Lat. \textunderscore ungere\textunderscore )}
\end{itemize}
Untar com substância oleosa.
Friccionar com uma substância gorda.
Applicar óleos consagrados a.
Sagrar; purificar.
Influir em, com palavras insinuantes ou affectivas.
\section{Unguari}
\begin{itemize}
\item {Grp. gram.:m.}
\end{itemize}
Espécie de perdiz africana.
\section{Ungueal}
\begin{itemize}
\item {Grp. gram.:adj.}
\end{itemize}
\begin{itemize}
\item {Proveniência:(Do lat. \textunderscore unguis\textunderscore )}
\end{itemize}
Relativo á unha.
\section{Unguebe}
\begin{itemize}
\item {Grp. gram.:m.}
\end{itemize}
Árvore angolense de Caconda.
\section{Unguentáceo}
\begin{itemize}
\item {fónica:gu-en}
\end{itemize}
\begin{itemize}
\item {Grp. gram.:adj.}
\end{itemize}
Relativo ou semelhante a unguento.
\section{Unguentário}
\begin{itemize}
\item {fónica:gu-en}
\end{itemize}
\begin{itemize}
\item {Grp. gram.:adj.}
\end{itemize}
\begin{itemize}
\item {Grp. gram.:M.}
\end{itemize}
\begin{itemize}
\item {Utilização:Ant.}
\end{itemize}
\begin{itemize}
\item {Proveniência:(Lat. \textunderscore unguentarius\textunderscore )}
\end{itemize}
O mesmo que \textunderscore unguentáceo\textunderscore .
O mesmo que \textunderscore perfumista\textunderscore .
Vaso para unguentos. Cf. \textunderscore Archeol. Port.\textunderscore , XIV, 57.
\section{Unguento}
\begin{itemize}
\item {fónica:gu-en}
\end{itemize}
\begin{itemize}
\item {Grp. gram.:m.}
\end{itemize}
\begin{itemize}
\item {Proveniência:(Lat. \textunderscore unguentum\textunderscore )}
\end{itemize}
Medicamento para uso externo, pouco conistente, e que tem por base uma substância gorda.
Designação antiga de certas drogas ou essências, com que se perfumava o corpo.
\section{Ungui}
\begin{itemize}
\item {fónica:gu-i}
\end{itemize}
\begin{itemize}
\item {Grp. gram.:m.}
\end{itemize}
\begin{itemize}
\item {Utilização:Bras}
\end{itemize}
Iguaria de farinha de pau, feijões, etc.
\section{Unguiculado}
\begin{itemize}
\item {fónica:gu-i}
\end{itemize}
\begin{itemize}
\item {Grp. gram.:adj.}
\end{itemize}
\begin{itemize}
\item {Utilização:Bot.}
\end{itemize}
\begin{itemize}
\item {Utilização:Zool.}
\end{itemize}
\begin{itemize}
\item {Proveniência:(Do lat. \textunderscore unguicula\textunderscore )}
\end{itemize}
Que termina em fórma de unha, (falando-se das pétalas).
Diz-se dos mammíferos, que têm unhas.
\section{Unguífero}
\begin{itemize}
\item {fónica:gu-i}
\end{itemize}
\begin{itemize}
\item {Grp. gram.:adj.}
\end{itemize}
\begin{itemize}
\item {Proveniência:(Lat. \textunderscore unguifer\textunderscore )}
\end{itemize}
Que tem unha.
\section{Unguiforme}
\begin{itemize}
\item {fónica:gu-i}
\end{itemize}
\begin{itemize}
\item {Grp. gram.:adj.}
\end{itemize}
\begin{itemize}
\item {Proveniência:(Do lat. \textunderscore unguis\textunderscore  + \textunderscore forma\textunderscore )}
\end{itemize}
Que tem fórma de unha.
\section{Unguinoso}
\begin{itemize}
\item {fónica:gu-i}
\end{itemize}
\begin{itemize}
\item {Grp. gram.:adj.}
\end{itemize}
\begin{itemize}
\item {Proveniência:(Lat. \textunderscore unguinosus\textunderscore )}
\end{itemize}
Gordurento, oleoso.
\section{Únguis}
\begin{itemize}
\item {Grp. gram.:m.}
\end{itemize}
\begin{itemize}
\item {Proveniência:(Lat. \textunderscore unguis\textunderscore )}
\end{itemize}
Pequeno osso, semelhante a unha e situado na parte anterò-interior da órbita ocular.
\section{Úngula}
\begin{itemize}
\item {Grp. gram.:f.}
\end{itemize}
\begin{itemize}
\item {Proveniência:(Lat. \textunderscore ungula\textunderscore )}
\end{itemize}
Saliência membranosa do ângulo interno do ôlho.
\section{Ungulado}
\begin{itemize}
\item {Grp. gram.:adj.}
\end{itemize}
\begin{itemize}
\item {Proveniência:(Do lat. \textunderscore ungulatus\textunderscore )}
\end{itemize}
Diz-se dos animaes, que têm unhas.
\section{Unha}
\begin{itemize}
\item {Grp. gram.:f.}
\end{itemize}
\begin{itemize}
\item {Grp. gram.:Loc. adv.}
\end{itemize}
\begin{itemize}
\item {Grp. gram.:Loc. interj.}
\end{itemize}
\begin{itemize}
\item {Grp. gram.:Pl.}
\end{itemize}
\begin{itemize}
\item {Utilização:Fam.}
\end{itemize}
\begin{itemize}
\item {Grp. gram.:Loc. adv.}
\end{itemize}
\begin{itemize}
\item {Proveniência:(Do lat. \textunderscore ungula\textunderscore )}
\end{itemize}
Lâmina córnea, um pouco transparente em geral, que reveste a extremidade dorsal dos dedos.
Garra.
Casco dos pachydermes e ruminantes.
Extremidade curva do pé dos insectos.
Nome de diversos opérculos das conchas.
Úngula.
Callosidade no dorso das bêstas; pisadura, produzida nas cavalgaduras pelos arreios.
Pé do caranguejo.
Pedaço de cepa ou do tronco da videira, que vai prêso ao pé do bacêllo que se cortou.
Parte recurva ou ponteaguda de alguns instrumentos ou de outros objectos.
\textunderscore Por uma unha negra\textunderscore , por um triz.
\textunderscore Unha com carne\textunderscore , pessôa, que é da maior intimidade de outra.
\textunderscore Têr na unha\textunderscore , estar na posse de, têr em seu poder.
\textunderscore Enterrar a unha\textunderscore , vender muito caro.
\textunderscore Dar á unha\textunderscore , trabalhar afincadamente, com muito cuidado.
\textunderscore Unha\textunderscore  ou \textunderscore unhas de fome\textunderscore  \textunderscore m.\textunderscore  e \textunderscore f.\textunderscore  pessôa muito avarenta.
\textunderscore Unha do martelo\textunderscore , a parte opposta á cabeça do martelo; dente do martelo; orelha do martello.
\textunderscore Á unha\textunderscore ! (para estimular, nas praças de toiros, os moços de forcado a fazerem pégas, ou para incitar a vias de facto indivíduos que estão altercando)
A mão; domínio, poder.
\textunderscore A unhas de cavallo\textunderscore , a toda a pressa.
\textunderscore Untar as unhas de\textunderscore , peitar, subornar.
\textunderscore Têr unhas na palma da mão\textunderscore , têr o hábito de furtar.
\section{Unhaca}
\begin{itemize}
\item {Grp. gram.:m.  e  f.}
\end{itemize}
\begin{itemize}
\item {Utilização:Burl.}
\end{itemize}
\begin{itemize}
\item {Proveniência:(De \textunderscore unha\textunderscore )}
\end{itemize}
Pessôa sovina, somítica.
Pessôa íntima, muito amiga.
\section{Unhada}
\begin{itemize}
\item {Grp. gram.:f.}
\end{itemize}
Traço, arranhadura ou ferimento feito com unha.
\section{Unha-de-anta}
\begin{itemize}
\item {Grp. gram.:f.}
\end{itemize}
\begin{itemize}
\item {Utilização:Bras}
\end{itemize}
O mesmo que \textunderscore unha-de-vaca\textunderscore .
\section{Unha-de-asno}
\begin{itemize}
\item {Grp. gram.:f.}
\end{itemize}
Nome de várias plantas medicinaes.
\section{Unha-de-boi}
\begin{itemize}
\item {Grp. gram.:f.}
\end{itemize}
\begin{itemize}
\item {Utilização:Bras}
\end{itemize}
Nome de várias plantas medicinaes.
\section{Unha-de-cavallo}
\begin{itemize}
\item {Grp. gram.:f.}
\end{itemize}
Um dos nomes vulgares da tussilagem.
\section{Unha-de-gato}
\begin{itemize}
\item {Grp. gram.:f.}
\end{itemize}
\begin{itemize}
\item {Utilização:Bras}
\end{itemize}
Arbusto espinhoso do Zaire, também conhecido em Piauí, (Brasil).
O mesmo que \textunderscore ancinho\textunderscore .
\section{Unha-de-vaca}
\begin{itemize}
\item {Grp. gram.:f.}
\end{itemize}
\begin{itemize}
\item {Utilização:Bras}
\end{itemize}
Planta cesalpínea, medicinal.
\section{Unha-de-velha}
\begin{itemize}
\item {Grp. gram.:f.}
\end{itemize}
\begin{itemize}
\item {Utilização:Bras}
\end{itemize}
Espécie de concha longa e descòrada.
\section{Unhador}
\begin{itemize}
\item {Grp. gram.:m.  e  adj.}
\end{itemize}
\begin{itemize}
\item {Proveniência:(De \textunderscore unhar\textunderscore )}
\end{itemize}
O que unha bacellos.
\section{Unhagata}
\begin{itemize}
\item {Grp. gram.:f.}
\end{itemize}
Planta, o mesmo que \textunderscore resta-boi\textunderscore .
\section{Unhame}
\begin{itemize}
\item {Grp. gram.:m.}
\end{itemize}
\begin{itemize}
\item {Utilização:Des.}
\end{itemize}
O mesmo que \textunderscore inhame\textunderscore . Cf. B. Pereira, \textunderscore Prosódia\textunderscore , vb. \textunderscore cyamus\textunderscore .
\section{Unhamento}
\begin{itemize}
\item {Grp. gram.:m.}
\end{itemize}
Acto ou effeito de unhar.
A parte unhada do bacêllo.
\section{Unhante}
\begin{itemize}
\item {Grp. gram.:m.}
\end{itemize}
\begin{itemize}
\item {Utilização:Gír.}
\end{itemize}
\begin{itemize}
\item {Proveniência:(De \textunderscore unhar\textunderscore )}
\end{itemize}
Veado novo.
Pescador que, na ria de Aveiro, apanha enguias á mão.
Pessôa que rouba, que deita a unha ao que não é seu.
\section{Unhão}
\begin{itemize}
\item {Grp. gram.:m.}
\end{itemize}
\begin{itemize}
\item {Utilização:Náut.}
\end{itemize}
\begin{itemize}
\item {Proveniência:(De \textunderscore unha\textunderscore )}
\end{itemize}
Acto ou effeito de entrançar um cabo partido, ligando com fio novo as partes separadas.
Nó, com que se peiam os chicotes de um cabo, que quebrou accidentalmente.
\section{Unhão}
\begin{itemize}
\item {Grp. gram.:m.}
\end{itemize}
\begin{itemize}
\item {Proveniência:(De \textunderscore Unhão\textunderscore , n. p.)}
\end{itemize}
Variedade de maçan.
\section{Unhar}
\begin{itemize}
\item {Grp. gram.:v. t.}
\end{itemize}
\begin{itemize}
\item {Grp. gram.:V. i.}
\end{itemize}
\begin{itemize}
\item {Utilização:Prov.}
\end{itemize}
\begin{itemize}
\item {Utilização:minh.}
\end{itemize}
\begin{itemize}
\item {Proveniência:(De \textunderscore unha\textunderscore )}
\end{itemize}
Riscar ou ferir com as unhas.
Arranhar.
Aferrar (âncoras).
Collocar na manta (o bacêllo), aconchegando-o com terra no lugar onde há de lançar raízes.
Diz-se da pedra, que assenta no chão, por fórma que é diffícil submeter-lhe o alvião, para a erguer.
\section{Unhas}
\begin{itemize}
\item {Grp. gram.:m.}
\end{itemize}
\begin{itemize}
\item {Utilização:Fam.}
\end{itemize}
Indivíduo somítico, sovina.
\section{Unhas-de-fome}
\begin{itemize}
\item {Grp. gram.:m.}
\end{itemize}
O mesmo que \textunderscore unhas\textunderscore . Cf. Castilho, II, 348.
\section{Unheira}
\begin{itemize}
\item {Grp. gram.:f.}
\end{itemize}
\begin{itemize}
\item {Utilização:Bras. do S}
\end{itemize}
Matadura incurável, ao lado do fio do lombo dos cavallos e proveniente do mau uso dos lombilhos.
\section{Unheiro}
\begin{itemize}
\item {Grp. gram.:m.}
\end{itemize}
\begin{itemize}
\item {Proveniência:(De \textunderscore unha\textunderscore )}
\end{itemize}
Tumor ou inflammação, entre a unha e o dedo.
\section{Unheirudo}
\begin{itemize}
\item {Grp. gram.:adj.}
\end{itemize}
\begin{itemize}
\item {Utilização:Bras. do S}
\end{itemize}
Que soffre unheira.
\section{Unheta}
\begin{itemize}
\item {fónica:nhê}
\end{itemize}
\begin{itemize}
\item {Grp. gram.:f.}
\end{itemize}
Nome de várias peças para tornear metaes.
\section{Unho}
\begin{itemize}
\item {Grp. gram.:m.}
\end{itemize}
Acto de unhar (bacêllo).
\section{Uni...}
\begin{itemize}
\item {Grp. gram.:pref.}
\end{itemize}
\begin{itemize}
\item {Proveniência:(Do lat. \textunderscore unus\textunderscore )}
\end{itemize}
(designativo do \textunderscore um\textunderscore )
\section{Unialado}
\begin{itemize}
\item {Grp. gram.:adj.}
\end{itemize}
\begin{itemize}
\item {Proveniência:(De \textunderscore uni...\textunderscore  + \textunderscore alado\textunderscore )}
\end{itemize}
Que tem só uma asa.
\section{Uniangular}
\begin{itemize}
\item {Grp. gram.:adj.}
\end{itemize}
\begin{itemize}
\item {Proveniência:(De \textunderscore uni...\textunderscore  + \textunderscore angular\textunderscore )}
\end{itemize}
Que tem só um ângulo.
\section{União}
\begin{itemize}
\item {Grp. gram.:f.}
\end{itemize}
\begin{itemize}
\item {Proveniência:(Do lat. \textunderscore unio\textunderscore )}
\end{itemize}
Acto ou effeito de unir.
Juncção de duas coisas ou pessôas.
Juncção, adhesão, contacto.
Alliança; casamento.
Cóito de animais.
Concórdia; pacto.
Confederação.
Conjunto de differentes Estados que, gozando certa autonomia administrativa, estão sobordinados todos a uma administração ou govêrno central.
Esfôrço moral ou intellectual com que os mýsticos procuram unir-se á ideia ou objecto que lhes occupa a mente.
\textunderscore Traço de união\textunderscore , o mesmo que \textunderscore hýphen\textunderscore .
\section{Uniarticulado}
\begin{itemize}
\item {Grp. gram.:adj.}
\end{itemize}
\begin{itemize}
\item {Utilização:Zool.}
\end{itemize}
\begin{itemize}
\item {Proveniência:(De \textunderscore uni...\textunderscore  + \textunderscore articulado\textunderscore )}
\end{itemize}
Que tem só uma articulação.
\section{Uniaxial}
\begin{itemize}
\item {fónica:csi}
\end{itemize}
\begin{itemize}
\item {Grp. gram.:adj.}
\end{itemize}
\begin{itemize}
\item {Proveniência:(Do lat. \textunderscore unus\textunderscore  + \textunderscore axis\textunderscore )}
\end{itemize}
Que tem um só eixo.
\section{Unicamente}
\begin{itemize}
\item {Grp. gram.:adv.}
\end{itemize}
De modo único; simplesmente; somente.
\section{Unicapsular}
\begin{itemize}
\item {Grp. gram.:adj.}
\end{itemize}
\begin{itemize}
\item {Utilização:Bot.}
\end{itemize}
\begin{itemize}
\item {Proveniência:(De \textunderscore uni...\textunderscore  + \textunderscore capsular\textunderscore )}
\end{itemize}
Que tem só uma cápsula.
\section{Unicaule}
\begin{itemize}
\item {Grp. gram.:adj.}
\end{itemize}
\begin{itemize}
\item {Utilização:Bot.}
\end{itemize}
\begin{itemize}
\item {Proveniência:(De \textunderscore uni...\textunderscore  + \textunderscore caule\textunderscore )}
\end{itemize}
Que tem só um caule.
\section{Unicellular}
\begin{itemize}
\item {Grp. gram.:adj.}
\end{itemize}
\begin{itemize}
\item {Utilização:Bot.}
\end{itemize}
\begin{itemize}
\item {Proveniência:(De \textunderscore uni...\textunderscore  + \textunderscore cellular\textunderscore )}
\end{itemize}
Que tem uma só céllula, ou que é formado de uma só céllula.
\section{Unicelular}
\begin{itemize}
\item {Grp. gram.:adj.}
\end{itemize}
\begin{itemize}
\item {Utilização:Bot.}
\end{itemize}
\begin{itemize}
\item {Proveniência:(De \textunderscore uni...\textunderscore  + \textunderscore celular\textunderscore )}
\end{itemize}
Que tem uma só célula, ou que é formado de uma só célula.
\section{Unichroísmo}
\begin{itemize}
\item {Grp. gram.:m.}
\end{itemize}
\begin{itemize}
\item {Proveniência:(De \textunderscore uni...\textunderscore  + gr. \textunderscore khroa\textunderscore )}
\end{itemize}
Propriedade, que alguns mineraes têm, de apresentar sempre a mesma côr, seja qual fôr a direcção dos raios luminosos que nelles incidem.
\section{Unichroísta}
\begin{itemize}
\item {Grp. gram.:adj.}
\end{itemize}
\begin{itemize}
\item {Proveniência:(De \textunderscore uni...\textunderscore  + gr. \textunderscore khroa\textunderscore )}
\end{itemize}
Que tem a propriedade do unichroísmo.
\section{Unicidade}
\begin{itemize}
\item {Grp. gram.:f.}
\end{itemize}
\begin{itemize}
\item {Utilização:Neol.}
\end{itemize}
Estado ou qualidade daquillo que é único.
\section{Unicismo}
\begin{itemize}
\item {Grp. gram.:m.}
\end{itemize}
\begin{itemize}
\item {Proveniência:(De \textunderscore único\textunderscore )}
\end{itemize}
Doutrina médica, de que os accidentes syphilíticos são determinados por um vírus único.
\section{Unicista}
\begin{itemize}
\item {Grp. gram.:m. ,  f.  e  adj.}
\end{itemize}
\begin{itemize}
\item {Proveniência:(De \textunderscore único\textunderscore )}
\end{itemize}
Pessôa, que segue o unicismo.
\section{Único}
\begin{itemize}
\item {Grp. gram.:adj.}
\end{itemize}
\begin{itemize}
\item {Proveniência:(Lat. \textunderscore unicus\textunderscore )}
\end{itemize}
Que é só um.
De cuja qualidade ou natureza não há outro.
Exclusivo; excepcional.
A que nada se póde comparar.
Que não tem semelhante.
Superior a todos os demais.
\section{Unicolor}
\begin{itemize}
\item {Grp. gram.:adj.}
\end{itemize}
\begin{itemize}
\item {Proveniência:(De \textunderscore uni...\textunderscore  + \textunderscore color\textunderscore )}
\end{itemize}
Que tem só uma côr.
\section{Unicorne}
\begin{itemize}
\item {Grp. gram.:adj.}
\end{itemize}
\begin{itemize}
\item {Grp. gram.:M.}
\end{itemize}
\begin{itemize}
\item {Proveniência:(Lat. \textunderscore unicornis\textunderscore )}
\end{itemize}
Que tem só uma ponta ou corno.
Unicórnio.
\section{Unicórneo}
\begin{itemize}
\item {Grp. gram.:m.  e  adj.}
\end{itemize}
\begin{itemize}
\item {Grp. gram.:M.}
\end{itemize}
\begin{itemize}
\item {Proveniência:(De \textunderscore unicorne\textunderscore )}
\end{itemize}
Que tem só um corno ou ponta.
Espécie de rhinoceronte.
\section{Unicórnio}
\begin{itemize}
\item {Grp. gram.:adj.}
\end{itemize}
\begin{itemize}
\item {Grp. gram.:M.}
\end{itemize}
\begin{itemize}
\item {Proveniência:(De \textunderscore unicorne\textunderscore )}
\end{itemize}
Que tem só um corno ou ponta.
Espécie de rhinoceronte.
\section{Unicroísmo}
\begin{itemize}
\item {Grp. gram.:m.}
\end{itemize}
\begin{itemize}
\item {Proveniência:(De \textunderscore uni...\textunderscore  + gr. \textunderscore khroa\textunderscore )}
\end{itemize}
Propriedade, que alguns mineraes têm, de apresentar sempre a mesma côr, seja qual fôr a direcção dos raios luminosos que neles incidem.
\section{Unicroísta}
\begin{itemize}
\item {Grp. gram.:adj.}
\end{itemize}
\begin{itemize}
\item {Proveniência:(De \textunderscore uni...\textunderscore  + gr. \textunderscore khroa\textunderscore )}
\end{itemize}
Que tem a propriedade do unicroísmo.
\section{Unicúspide}
\begin{itemize}
\item {Grp. gram.:adj.}
\end{itemize}
\begin{itemize}
\item {Proveniência:(De \textunderscore uni...\textunderscore  + \textunderscore cúspide\textunderscore )}
\end{itemize}
Que tem só uma ponta.
\section{Unidade}
\begin{itemize}
\item {Grp. gram.:f.}
\end{itemize}
\begin{itemize}
\item {Utilização:Eccles.}
\end{itemize}
\begin{itemize}
\item {Proveniência:(Do lat. \textunderscore unitas\textunderscore )}
\end{itemize}
Quantidade, tomada arbitrariamente, para servir de termo de comparação a quantidades da mesma espécie.
Princípio da numeração.
O número um.
Número inteiro, inferior a déz.
Número inferior de uma série.
Qualidade do que é único ou uno, ou do que não é partível.
Reunião de seres individuaes, considerados nas suas relações recíprocas ou caracteres communs.
União.
Mónada, na philosophia do Leibnitz.
Coordenação das partes de um trabalho literário ou artístico.
Acção collectiva, tendente a um fim único.
Uniformidade.
Profissão da mesma fé e obediência aos mesmos chefes.
\textunderscore Unidade táctica\textunderscore , corpo de soldados, destinados a manobrar juntos, nas circunstâncias em que outros corpos manobram também juntamente: \textunderscore como o batalhão é a unidade táctica da infantaria...\textunderscore 
\section{Unidamente}
\begin{itemize}
\item {Grp. gram.:adv.}
\end{itemize}
\begin{itemize}
\item {Proveniência:(De \textunderscore unido\textunderscore )}
\end{itemize}
Com união; estreitamente.
\section{Unido}
\begin{itemize}
\item {Grp. gram.:adj.}
\end{itemize}
\begin{itemize}
\item {Proveniência:(De \textunderscore unir\textunderscore )}
\end{itemize}
Que se uniu.
Que está em contacto.
Junto.
Ligado.
\section{Unificação}
\begin{itemize}
\item {Grp. gram.:f.}
\end{itemize}
Acto ou effeito de unificar.
\section{Unificar}
\begin{itemize}
\item {Grp. gram.:v. t.}
\end{itemize}
\begin{itemize}
\item {Proveniência:(Do lat. \textunderscore unus\textunderscore  + \textunderscore facere\textunderscore )}
\end{itemize}
Reunir num só corpo ou num todo.
Tornar uno.
\section{Unifloro}
\begin{itemize}
\item {Grp. gram.:adj.}
\end{itemize}
\begin{itemize}
\item {Proveniência:(Do lat. \textunderscore unus\textunderscore  + \textunderscore flos\textunderscore , \textunderscore floris\textunderscore )}
\end{itemize}
Que tem só uma flôr.
\section{Unifoliado}
\begin{itemize}
\item {Grp. gram.:adj.}
\end{itemize}
\begin{itemize}
\item {Proveniência:(De \textunderscore uni...\textunderscore  + \textunderscore foliado\textunderscore )}
\end{itemize}
Que tem só uma fôlha.
\section{Unifólio}
\begin{itemize}
\item {Grp. gram.:adj.}
\end{itemize}
\begin{itemize}
\item {Proveniência:(Do lat. \textunderscore unus\textunderscore  + \textunderscore folium\textunderscore )}
\end{itemize}
O mesmo que \textunderscore unifoliado\textunderscore .
\section{Uniformador}
\begin{itemize}
\item {Grp. gram.:adj.}
\end{itemize}
Que uniforma. Cf. Herculano, \textunderscore Opúsc.\textunderscore , IV, 200.
\section{Uniformar}
\begin{itemize}
\item {Grp. gram.:v. t.  e  p.}
\end{itemize}
O mesmo que \textunderscore uniformizar\textunderscore .
\section{Uniforme}
\begin{itemize}
\item {Grp. gram.:adj.}
\end{itemize}
\begin{itemize}
\item {Grp. gram.:M.}
\end{itemize}
\begin{itemize}
\item {Proveniência:(Lat. \textunderscore uniformis\textunderscore )}
\end{itemize}
Que tem só uma fórma.
Que não tem variedade.
Semelhante.
Cujas partes são semelhantes ou idênticas; idêntico.
Farda ou vestuário, feito segundo um modelo commum, para uma corporação ou uma classe.
\section{Uniformemente}
\begin{itemize}
\item {Grp. gram.:adv.}
\end{itemize}
De modo uniforme.
Da mesma maneira.
Unanimemente; sem divergência.
Com igualdade.
\section{Uniformidade}
\begin{itemize}
\item {Grp. gram.:f.}
\end{itemize}
\begin{itemize}
\item {Proveniência:(Do lat. \textunderscore uniformitas\textunderscore )}
\end{itemize}
Qualidade do que é uniforme.
Monotonia.
Coherência.
\section{Uniformização}
\begin{itemize}
\item {Grp. gram.:f.}
\end{itemize}
Acto ou effeito de uniformizar.
\section{Uniformizar}
\begin{itemize}
\item {Grp. gram.:v. t.}
\end{itemize}
\begin{itemize}
\item {Grp. gram.:V. p.}
\end{itemize}
Tornar uniforme.
Fazer vestir de uniforme.
Trajar uniforme.
\section{Unigamia}
\begin{itemize}
\item {Grp. gram.:f.}
\end{itemize}
Estado de unígamo.
\section{Unígamo}
\begin{itemize}
\item {Grp. gram.:m.}
\end{itemize}
\begin{itemize}
\item {Proveniência:(Do lat. \textunderscore unus\textunderscore  + gr. \textunderscore gamos\textunderscore )}
\end{itemize}
O mesmo que \textunderscore monógamo\textunderscore .
\section{Unigênito}
\begin{itemize}
\item {Grp. gram.:adj.}
\end{itemize}
\begin{itemize}
\item {Grp. gram.:M.}
\end{itemize}
\begin{itemize}
\item {Proveniência:(Lat. \textunderscore unigenitus\textunderscore )}
\end{itemize}
Único que foi gerado por seus pais.
Filho único.
Christo.
\section{Unijugado}
\begin{itemize}
\item {Grp. gram.:adj.}
\end{itemize}
\begin{itemize}
\item {Proveniência:(Do lat. \textunderscore unus\textunderscore  + \textunderscore jugum\textunderscore )}
\end{itemize}
Que fórma só um par.
\section{Unilabiado}
\begin{itemize}
\item {Grp. gram.:adj.}
\end{itemize}
\begin{itemize}
\item {Utilização:Bot.}
\end{itemize}
\begin{itemize}
\item {Proveniência:(De \textunderscore uni...\textunderscore  + \textunderscore labiado\textunderscore )}
\end{itemize}
Que tem só um lábio ou lóbulo principal, (falando-se de corollas).
\section{Unilateral}
\begin{itemize}
\item {Grp. gram.:adj.}
\end{itemize}
\begin{itemize}
\item {Utilização:Jur.}
\end{itemize}
\begin{itemize}
\item {Proveniência:(De \textunderscore uni...\textunderscore  + \textunderscore lateral\textunderscore )}
\end{itemize}
Situado de um só lado.
Que se inclina para um lado só.
Diz-se do contrato, em que só uma das partes se obriga para com outra, sem que esta contraia obrigação alguma para com aquella.
\section{Unilingue}
\begin{itemize}
\item {Grp. gram.:adj.}
\end{itemize}
\begin{itemize}
\item {Proveniência:(Do lat. \textunderscore unus\textunderscore  + \textunderscore lingua\textunderscore )}
\end{itemize}
Que está escrito numa só língua.
\section{Unilobado}
\begin{itemize}
\item {Grp. gram.:adj.}
\end{itemize}
\begin{itemize}
\item {Proveniência:(De \textunderscore uni...\textunderscore  + \textunderscore lobulado\textunderscore )}
\end{itemize}
Que tém só um lóbulo.
\section{Unilobulado}
\begin{itemize}
\item {Grp. gram.:adj.}
\end{itemize}
\begin{itemize}
\item {Proveniência:(De \textunderscore uni...\textunderscore  + \textunderscore lobulado\textunderscore )}
\end{itemize}
Que tem só um lóbulo.
\section{Unilocular}
\begin{itemize}
\item {Grp. gram.:adj.}
\end{itemize}
\begin{itemize}
\item {Utilização:Hist. Nat.}
\end{itemize}
\begin{itemize}
\item {Proveniência:(De \textunderscore uni...\textunderscore  + \textunderscore locular\textunderscore )}
\end{itemize}
Que tem só uma cavidade, ou cuja cavidade não tem separações interiores.
\section{Uníloquo}
\begin{itemize}
\item {Grp. gram.:adj.}
\end{itemize}
\begin{itemize}
\item {Proveniência:(Do lat. \textunderscore unus\textunderscore  + \textunderscore loqui\textunderscore )}
\end{itemize}
Que exprime o sentir ou a vontade de uma pessôa só.
\section{Unimetalismo}
\begin{itemize}
\item {Grp. gram.:m.}
\end{itemize}
\begin{itemize}
\item {Proveniência:(Do lat. \textunderscore unus\textunderscore  + \textunderscore metallum\textunderscore )}
\end{itemize}
Sistema de um só metal, para moéda.
\section{Unimetalista}
\begin{itemize}
\item {Grp. gram.:m.  e  adj.}
\end{itemize}
Partidário do unimetalismo.
\section{Unimetallismo}
\begin{itemize}
\item {Grp. gram.:m.}
\end{itemize}
\begin{itemize}
\item {Proveniência:(Do lat. \textunderscore unus\textunderscore  + \textunderscore metallum\textunderscore )}
\end{itemize}
Systema de um só metal, para moéda.
\section{Unimetallista}
\begin{itemize}
\item {Grp. gram.:m.  e  adj.}
\end{itemize}
Partidário do unimetallismo.
\section{Uninervado}
\begin{itemize}
\item {Grp. gram.:adj.}
\end{itemize}
\begin{itemize}
\item {Utilização:Bot.}
\end{itemize}
\begin{itemize}
\item {Proveniência:(De \textunderscore uni...\textunderscore  + \textunderscore nervo\textunderscore )}
\end{itemize}
Que tem só uma nervura, sem ramificações, como as fôlhas do teixo, do pinheiro, etc.
\section{Uninominal}
\begin{itemize}
\item {Grp. gram.:adj.}
\end{itemize}
\begin{itemize}
\item {Proveniência:(Do lat. \textunderscore unus\textunderscore  + \textunderscore nomen\textunderscore )}
\end{itemize}
Relativo a um nome só: \textunderscore votação uninominal\textunderscore .
Que encerra um só nome: \textunderscore lista uninominal\textunderscore .
\section{Uniôa}
\begin{itemize}
\item {Grp. gram.:f.}
\end{itemize}
Mollusco acéphalo.
\section{Unioculado}
\begin{itemize}
\item {Grp. gram.:adj.}
\end{itemize}
\begin{itemize}
\item {Proveniência:(De \textunderscore uni...\textunderscore  + \textunderscore oculado\textunderscore )}
\end{itemize}
Que tem só um ôlho.
\section{Unionista}
\begin{itemize}
\item {Grp. gram.:m. ,  f.  e  adj.}
\end{itemize}
\begin{itemize}
\item {Proveniência:(Do lat. \textunderscore unio\textunderscore , \textunderscore unionis\textunderscore )}
\end{itemize}
Pessôa, que faz parte de uma união política.
Partidário de uma confederação.
\section{Uníparo}
\begin{itemize}
\item {Grp. gram.:adj.}
\end{itemize}
\begin{itemize}
\item {Proveniência:(Do lat. \textunderscore unus\textunderscore  + \textunderscore p[-a]rere\textunderscore )}
\end{itemize}
Diz-se das fêmeas, que só parem um filho de cada vez.
\section{Unipedal}
\begin{itemize}
\item {Grp. gram.:adj.}
\end{itemize}
\begin{itemize}
\item {Proveniência:(De \textunderscore uni...\textunderscore  + \textunderscore pedal\textunderscore )}
\end{itemize}
Que tem só um pé.
Relativo a um só pé.
\section{Unipessoal}
\begin{itemize}
\item {Grp. gram.:adj.}
\end{itemize}
\begin{itemize}
\item {Utilização:Gram.}
\end{itemize}
\begin{itemize}
\item {Proveniência:(De \textunderscore uni...\textunderscore  + \textunderscore pessoal\textunderscore )}
\end{itemize}
Relativo a uma só pessôa.
Que consta de uma só pessôa.
O mesmo que \textunderscore impessoal\textunderscore , (falando-se dos verbos).
\section{Unipessoalmente}
\begin{itemize}
\item {Grp. gram.:adv.}
\end{itemize}
De modo unipessoal.
\section{Unipètalado}
\begin{itemize}
\item {Grp. gram.:adj.}
\end{itemize}
O mesmo que \textunderscore unipétalo\textunderscore .
\section{Unipétalo}
\begin{itemize}
\item {Grp. gram.:adj.}
\end{itemize}
\begin{itemize}
\item {Proveniência:(De \textunderscore uni...\textunderscore  + \textunderscore pétala\textunderscore )}
\end{itemize}
Que tem só uma pétala.
\section{Unipolar}
\begin{itemize}
\item {Grp. gram.:adj.}
\end{itemize}
\begin{itemize}
\item {Utilização:Phýs.}
\end{itemize}
\begin{itemize}
\item {Proveniência:(De \textunderscore uni...\textunderscore  + \textunderscore polar\textunderscore )}
\end{itemize}
Que tem só um pólo.
Diz-se dos fios de uma pilha, que conduzem só uma electricidade.
\section{Unipolaridade}
\begin{itemize}
\item {Grp. gram.:f.}
\end{itemize}
\begin{itemize}
\item {Proveniência:(De \textunderscore uni...\textunderscore  + \textunderscore polaridade\textunderscore )}
\end{itemize}
Estado de um corpo unipolar.
\section{Unir}
\begin{itemize}
\item {Grp. gram.:v. t.}
\end{itemize}
\begin{itemize}
\item {Grp. gram.:V. i.}
\end{itemize}
\begin{itemize}
\item {Proveniência:(Lat. \textunderscore unire\textunderscore )}
\end{itemize}
Unificar.
Juntar num só.
Juntar.
Aproximar.
Ligar.
Aggregar.
Reunir.
Estabelecer communicação entre.
Combinar.
Aconchegar.
Possuir conjuntamente.
Ligar pelo amor ou pelo casamento.
Conciliar, estabelecer acôrdo entre.
Adherir, ligar-se.
\section{Unirefringente}
\begin{itemize}
\item {fónica:re}
\end{itemize}
\begin{itemize}
\item {Grp. gram.:adj.}
\end{itemize}
\begin{itemize}
\item {Utilização:Phýs.}
\end{itemize}
\begin{itemize}
\item {Proveniência:(De \textunderscore uni...\textunderscore  + \textunderscore refringente\textunderscore )}
\end{itemize}
Diz-se dos corpos ou substâncias, em que a luz, refractando-se, produz uma só imagem, como no vidro.
\section{Unireme}
\begin{itemize}
\item {fónica:rê}
\end{itemize}
\begin{itemize}
\item {Grp. gram.:adj.}
\end{itemize}
\begin{itemize}
\item {Proveniência:(Do lat. \textunderscore unus\textunderscore  + \textunderscore remus\textunderscore )}
\end{itemize}
Que tem um só remo. Cf. Castilho, \textunderscore Fastos\textunderscore , II, 413.
\section{Unirrefringente}
\begin{itemize}
\item {Grp. gram.:adj.}
\end{itemize}
\begin{itemize}
\item {Utilização:Phýs.}
\end{itemize}
\begin{itemize}
\item {Proveniência:(De \textunderscore uni...\textunderscore  + \textunderscore refringente\textunderscore )}
\end{itemize}
Diz-se dos corpos ou substâncias, em que a luz, refractando-se, produz uma só imagem, como no vidro.
\section{Unirreme}
\begin{itemize}
\item {Grp. gram.:adj.}
\end{itemize}
\begin{itemize}
\item {Proveniência:(Do lat. \textunderscore unus\textunderscore  + \textunderscore remus\textunderscore )}
\end{itemize}
Que tem um só remo. Cf. Castilho, \textunderscore Fastos\textunderscore , II, 413.
\section{Unisexuado}
\begin{itemize}
\item {fónica:se,csu}
\end{itemize}
\begin{itemize}
\item {Grp. gram.:adj.}
\end{itemize}
\begin{itemize}
\item {Utilização:Bot.}
\end{itemize}
\begin{itemize}
\item {Proveniência:(De \textunderscore uni...\textunderscore  + \textunderscore sexual\textunderscore )}
\end{itemize}
Que tem só um sexo.
Que tem só estames ou só pistillos.
\section{Unisexual}
\begin{itemize}
\item {fónica:se,csu}
\end{itemize}
\begin{itemize}
\item {Grp. gram.:adj.}
\end{itemize}
\begin{itemize}
\item {Utilização:Bot.}
\end{itemize}
\begin{itemize}
\item {Proveniência:(De \textunderscore uni...\textunderscore  + \textunderscore sexual\textunderscore )}
\end{itemize}
Que tem só um sexo.
Que tem só estames ou só pistillos.
\section{Unisonamente}
\begin{itemize}
\item {fónica:so}
\end{itemize}
\begin{itemize}
\item {Grp. gram.:adv.}
\end{itemize}
De modo unísono.
\section{Unisonância}
\begin{itemize}
\item {fónica:so}
\end{itemize}
\begin{itemize}
\item {Grp. gram.:f.}
\end{itemize}
\begin{itemize}
\item {Proveniência:(De \textunderscore unisonante\textunderscore )}
\end{itemize}
Qualidade do que é unísono.
Conjunto de sons unísonos.
Melodia.
Monotonia.
\section{Unisonante}
\begin{itemize}
\item {fónica:so}
\end{itemize}
\begin{itemize}
\item {Grp. gram.:adj.}
\end{itemize}
\begin{itemize}
\item {Proveniência:(De \textunderscore uni...\textunderscore  + \textunderscore sonante\textunderscore )}
\end{itemize}
Unísono.
Que se póde executar unísono.
\section{Unisonantemente}
\begin{itemize}
\item {fónica:so}
\end{itemize}
\begin{itemize}
\item {Grp. gram.:adv.}
\end{itemize}
De modo unisonante.
\section{Unísono}
\begin{itemize}
\item {fónica:so}
\end{itemize}
\begin{itemize}
\item {Grp. gram.:adj.}
\end{itemize}
\begin{itemize}
\item {Grp. gram.:M.}
\end{itemize}
\begin{itemize}
\item {Proveniência:(Lat. \textunderscore unisonus\textunderscore )}
\end{itemize}
Diz-se do acorde de vozes ou instrumentos, que faz ouvir um som único ou semelhante.
Que tem som igual ao de outro.
Semelhante no som.
Concorde.
Conjunto de sons, cuja entonação é absolutamente a mesma.
\section{Unissexuado}
\begin{itemize}
\item {fónica:csu}
\end{itemize}
\begin{itemize}
\item {Grp. gram.:adj.}
\end{itemize}
\begin{itemize}
\item {Utilização:Bot.}
\end{itemize}
\begin{itemize}
\item {Proveniência:(De \textunderscore uni...\textunderscore  + \textunderscore sexual\textunderscore )}
\end{itemize}
Que tem só um sexo.
Que tem só estames ou só pistillos.
\section{Unissexual}
\begin{itemize}
\item {fónica:csu}
\end{itemize}
\begin{itemize}
\item {Grp. gram.:adj.}
\end{itemize}
\begin{itemize}
\item {Utilização:Bot.}
\end{itemize}
\begin{itemize}
\item {Proveniência:(De \textunderscore uni...\textunderscore  + \textunderscore sexual\textunderscore )}
\end{itemize}
Que tem só um sexo.
Que tem só estames ou só pistillos.
\section{Unissonamente}
\begin{itemize}
\item {Grp. gram.:adv.}
\end{itemize}
De modo uníssono.
\section{Unissonância}
\begin{itemize}
\item {Grp. gram.:f.}
\end{itemize}
\begin{itemize}
\item {Proveniência:(De \textunderscore unissonante\textunderscore )}
\end{itemize}
Qualidade do que é uníssono.
Conjunto de sons uníssonos.
Melodia.
Monotonia.
\section{Unissonante}
\begin{itemize}
\item {Grp. gram.:adj.}
\end{itemize}
\begin{itemize}
\item {Proveniência:(De \textunderscore uni...\textunderscore  + \textunderscore sonante\textunderscore )}
\end{itemize}
Uníssono.
Que se póde executar uníssono.
\section{Unissonantemente}
\begin{itemize}
\item {Grp. gram.:adv.}
\end{itemize}
De modo unissonante.
\section{Uníssono}
\begin{itemize}
\item {Grp. gram.:adj.}
\end{itemize}
\begin{itemize}
\item {Grp. gram.:M.}
\end{itemize}
\begin{itemize}
\item {Proveniência:(Lat. \textunderscore unisonus\textunderscore )}
\end{itemize}
Diz-se do acorde de vozes ou instrumentos, que faz ouvir um som único ou semelhante.
Que tem som igual ao de outro.
Semelhante no som.
Concorde.
Conjunto de sons, cuja entonação é absolutamente a mesma.
\section{Unitário}
\begin{itemize}
\item {Grp. gram.:adj.}
\end{itemize}
\begin{itemize}
\item {Grp. gram.:M.}
\end{itemize}
\begin{itemize}
\item {Proveniência:(Do lat. \textunderscore unitas\textunderscore )}
\end{itemize}
Relativo á unidade.
Relativo á unidade política de um país.
Que tem o carácter de unidade.
Que se não subdivide em zoonitos, (falando-se de animaes).
Sectário de um systema theológico, em que domina a unidade.
\section{Unitarismo}
\begin{itemize}
\item {Grp. gram.:m.}
\end{itemize}
\begin{itemize}
\item {Proveniência:(De \textunderscore unitário\textunderscore )}
\end{itemize}
Systema unitário.
\section{Unitarista}
\begin{itemize}
\item {Grp. gram.:m.  e  adj.}
\end{itemize}
Partidário do unitarismo.
\section{Unitivo}
\begin{itemize}
\item {Grp. gram.:adj.}
\end{itemize}
\begin{itemize}
\item {Proveniência:(Lat. \textunderscore unitivus\textunderscore )}
\end{itemize}
Próprio para unir ou para se unir.
\section{Univalve}
\begin{itemize}
\item {Grp. gram.:adj.}
\end{itemize}
\begin{itemize}
\item {Utilização:Hist. Nat.}
\end{itemize}
\begin{itemize}
\item {Proveniência:(De \textunderscore uni...\textunderscore  + \textunderscore valva\textunderscore )}
\end{itemize}
Que se abre de um só lado, (falando-se de frutos).
Formado de uma só peça, (falando-se das conchas ou molluscos).
\section{Univalvular}
\begin{itemize}
\item {Grp. gram.:adj.}
\end{itemize}
\begin{itemize}
\item {Utilização:Bot.}
\end{itemize}
\begin{itemize}
\item {Proveniência:(De \textunderscore uni...\textunderscore  + \textunderscore válvula\textunderscore )}
\end{itemize}
Que tem uma só válvula, como os follículos.
\section{Universal}
\begin{itemize}
\item {Grp. gram.:adj.}
\end{itemize}
\begin{itemize}
\item {Grp. gram.:M.}
\end{itemize}
\begin{itemize}
\item {Proveniência:(Lat. \textunderscore universalis\textunderscore )}
\end{itemize}
Que abrange tudo, ou que se estende a tudo ou por toda a parte.
Que provém de todos.
Que tem o carácter de generalidade abstracta.
Que tem capacidade ou aptidão para tudo.
Aquillo que é universal.
\section{Universalidade}
\begin{itemize}
\item {Grp. gram.:f.}
\end{itemize}
\begin{itemize}
\item {Proveniência:(Do lat. \textunderscore universalitas\textunderscore )}
\end{itemize}
Qualidade do que é universal; totalidade.
\section{Universalismo}
\begin{itemize}
\item {Grp. gram.:m.}
\end{itemize}
\begin{itemize}
\item {Proveniência:(De \textunderscore universal\textunderscore )}
\end{itemize}
Tendência ou esfôrço para universalizar uma ideia ou uma obra.
Cosmopolitismo.
Opinião dos que não reconhecem maiór autoridade que o assentimento universal.
\section{Universalista}
\begin{itemize}
\item {Grp. gram.:adj.}
\end{itemize}
Que se dedica a universalizar uma ideia ou uma obra.
(Cp. \textunderscore universalismo\textunderscore )
\section{Universalização}
\begin{itemize}
\item {Grp. gram.:f.}
\end{itemize}
Acto ou effeito de universalizar.
\section{Universalizar}
\begin{itemize}
\item {Grp. gram.:v. t.}
\end{itemize}
Tornar universal; generalizar.
\section{Universalmente}
\begin{itemize}
\item {Grp. gram.:adv.}
\end{itemize}
De modo universal.
Em toda a parte; em todo o mundo.
\section{Universidade}
\begin{itemize}
\item {Grp. gram.:f.}
\end{itemize}
\begin{itemize}
\item {Proveniência:(Do lat. \textunderscore universitas\textunderscore )}
\end{itemize}
O mesmo que \textunderscore universalidade\textunderscore :«\textunderscore ...a universidade de todas as cousas.\textunderscore »\textunderscore Luz e Calor\textunderscore .
Conjunto de escolas, em que se professam sciências.
Estabelecimento escolar de Coimbra, em que se ensina a Theologia, o Direito, a Mathemática, a Philosophia e a Medicina.
Edifício, em que se professam estas faculdades.
Corporação docente dessas escolas.
Em França, toda a corporação docente do país, escolhida pelo Estado e dirigida pelo Ministro da Instrucção Pública.
\section{Universitário}
\begin{itemize}
\item {Grp. gram.:adj.}
\end{itemize}
\begin{itemize}
\item {Grp. gram.:M.}
\end{itemize}
\begin{itemize}
\item {Proveniência:(Do lat. \textunderscore universitas\textunderscore )}
\end{itemize}
Relativo á universidade.
Universal.
Professor de uma universidade.
\section{Universo}
\begin{itemize}
\item {Grp. gram.:adj.}
\end{itemize}
\begin{itemize}
\item {Grp. gram.:M.}
\end{itemize}
\begin{itemize}
\item {Utilização:Fig.}
\end{itemize}
\begin{itemize}
\item {Proveniência:(Lat. \textunderscore universus\textunderscore )}
\end{itemize}
O mesmo que \textunderscore universal\textunderscore .
Conjunto de todos os corpos ou astros, disseminados pelo espaço illimitado.
O systema solar.
O mundo.
A Terra.
A maior parte da Terra.
Os habitantes da Terra.
A Sociedade.
Um todo.
Domínio moral ou material, em relação ao universo.
\section{Univocação}
\begin{itemize}
\item {Grp. gram.:f.}
\end{itemize}
\begin{itemize}
\item {Proveniência:(Do lat. \textunderscore univocatio\textunderscore )}
\end{itemize}
Qualidade do que é unívoco.
\section{Univocamente}
\begin{itemize}
\item {Grp. gram.:adv.}
\end{itemize}
De modo unívoco.
Com causa unívoca.
\section{Unívoco}
\begin{itemize}
\item {Grp. gram.:adj.}
\end{itemize}
\begin{itemize}
\item {Proveniência:(Lat. \textunderscore univocus\textunderscore )}
\end{itemize}
Que se applica a muitas coisas do mesmo gênero e da mesma ou differente espécie.
Que só se póde interpretar de uma fórma.
Que tem a mesma natureza.
Que tem o mesmo som.
Homónymo.
\section{Uno}
\begin{itemize}
\item {Grp. gram.:adj.}
\end{itemize}
\begin{itemize}
\item {Proveniência:(Lat. \textunderscore unus\textunderscore )}
\end{itemize}
Um; singular; único no seu gênero ou espécie: \textunderscore Deus é uno\textunderscore .
\section{Unóculo}
\begin{itemize}
\item {Grp. gram.:m.  e  adj.}
\end{itemize}
\begin{itemize}
\item {Proveniência:(Lat. \textunderscore unoculus\textunderscore )}
\end{itemize}
O que tem só um ôlho.
\section{Unógatos}
\begin{itemize}
\item {Grp. gram.:m. pl.}
\end{itemize}
\begin{itemize}
\item {Utilização:Zool.}
\end{itemize}
Sétima classe de insectos, no systema de Fabrício.
\section{Únsia}
\begin{itemize}
\item {Grp. gram.:f.}
\end{itemize}
Árvore angolense de Caconda.
\section{Untadela}
\begin{itemize}
\item {Grp. gram.:f.}
\end{itemize}
O mesmo que \textunderscore untura\textunderscore .
\section{Untador}
\begin{itemize}
\item {Grp. gram.:m.  e  adj.}
\end{itemize}
O que unta.
\section{Untadura}
\begin{itemize}
\item {Grp. gram.:f.}
\end{itemize}
O mesmo que \textunderscore untura\textunderscore .
\section{Untar}
\begin{itemize}
\item {Grp. gram.:v. t.}
\end{itemize}
\begin{itemize}
\item {Proveniência:(De \textunderscore unto\textunderscore )}
\end{itemize}
Applicar unto a.
Esfregar com unto.
Cobrir de unto; besuntar.
\section{Unteiro}
\begin{itemize}
\item {Grp. gram.:m.}
\end{itemize}
\begin{itemize}
\item {Utilização:T. da Bairrada}
\end{itemize}
\begin{itemize}
\item {Proveniência:(De \textunderscore unto\textunderscore )}
\end{itemize}
Vaso, que usam os carreiros e em que elles trazem um preparado, com que untam os eixos dos carros.
\section{Unto}
\begin{itemize}
\item {Grp. gram.:m.}
\end{itemize}
\begin{itemize}
\item {Grp. gram.:Pl. Loc. adv.}
\end{itemize}
\begin{itemize}
\item {Proveniência:(Lat. \textunderscore unctus\textunderscore )}
\end{itemize}
Gordura ou banha de porco; gordura.
Óleo.
\textunderscore Ir aos untos\textunderscore , bater, dar pancadas. Cf. G. Braga, \textunderscore Mal da Delfina\textunderscore , 4.
\section{Untre}
\begin{itemize}
\item {Grp. gram.:prep.}
\end{itemize}
\begin{itemize}
\item {Utilização:Ant.}
\end{itemize}
O mesmo que \textunderscore entre\textunderscore . Cf. S. R. Viterbo, \textunderscore Elucidário\textunderscore .
\section{Untué-de-obó}
\begin{itemize}
\item {Grp. gram.:m.}
\end{itemize}
Grande árvore santhomense, própria para construcções.
\section{Untuém}
\begin{itemize}
\item {Grp. gram.:m.}
\end{itemize}
O mesmo que \textunderscore untué-de-obó\textunderscore .
Planta sapotácea, que produz um látex, semelhante á guta-percha.
\section{Untuosamente}
\begin{itemize}
\item {Grp. gram.:adv.}
\end{itemize}
\begin{itemize}
\item {Utilização:Fig.}
\end{itemize}
De modo untuoso.
Com uncção.
Suavemente; melifluamente.
\section{Untuosidade}
\begin{itemize}
\item {Grp. gram.:f.}
\end{itemize}
Estado ou qualidade de untuoso.
Qualidade de gorduroso ou escorregadio.
\section{Untuoso}
\begin{itemize}
\item {Grp. gram.:adj.}
\end{itemize}
\begin{itemize}
\item {Utilização:Fig.}
\end{itemize}
\begin{itemize}
\item {Proveniência:(Lat. \textunderscore unctuosus\textunderscore )}
\end{itemize}
Em que há unto ou gordura.
Gorduroso.
Lubrificado, escorregadio.
Que dá ao tacto a impressão das substâncias gordurosas.
Relativo a iguarias, em que predominam môlhos e carnes:«\textunderscore dava cama e mesa untuosa aos missionários...\textunderscore »Camillo, \textunderscore Brasileira\textunderscore , 237.
Suave; amorável; meigo; melífluo.
\section{Untura}
\begin{itemize}
\item {Grp. gram.:f.}
\end{itemize}
\begin{itemize}
\item {Utilização:Fig.}
\end{itemize}
\begin{itemize}
\item {Proveniência:(Lat. \textunderscore unctura\textunderscore )}
\end{itemize}
Acto ou effeito de untar.
Unto.
Unguento.
Substância medicinal, para fomentações.
Ligeiras noções, conhecimento superficial.
\section{Upa}
\begin{itemize}
\item {Grp. gram.:f.}
\end{itemize}
\begin{itemize}
\item {Grp. gram.:Interj.}
\end{itemize}
\begin{itemize}
\item {Utilização:Ext.}
\end{itemize}
\begin{itemize}
\item {Proveniência:(Do ingl. \textunderscore up\textunderscore )}
\end{itemize}
Salto brusco do cavallo; corcôvo.
(própria para incitar um animal a levantar-se ou a subir).
Exprime o acto de se levantar alguém com difficuldade ou de erguer nos braços uma criança.
\section{Upanda}
\begin{itemize}
\item {Grp. gram.:f.}
\end{itemize}
Nome, com que os sertanejos de Angola designam qualquer demanda ou pendência. Cf. Capello e Ivens, I, 173.
\section{Upanda}
\begin{itemize}
\item {Grp. gram.:m.}
\end{itemize}
Medida africana, correspondente a 2 jardas. Cf. Capello e Ivens, I, 6.
\section{Upar}
\begin{itemize}
\item {Grp. gram.:v. i.}
\end{itemize}
\begin{itemize}
\item {Utilização:Bras}
\end{itemize}
Diz-se da bêsta, que dá upas, ou pequenos saltos, erguendo as ancas.
\section{Upas}
\begin{itemize}
\item {Grp. gram.:m.}
\end{itemize}
\begin{itemize}
\item {Proveniência:(Do jav. \textunderscore upas\textunderscore )}
\end{itemize}
Substância venenosa, com que os habitantes das ilhas de Sonda ervam as suas frechas.
\section{Upas-tienté}
\begin{itemize}
\item {Grp. gram.:m.}
\end{itemize}
Planta loganiácea medicinal do Brasil.
\section{Upiúba}
\begin{itemize}
\item {Grp. gram.:f.}
\end{itemize}
\begin{itemize}
\item {Utilização:Bras}
\end{itemize}
Árvore das regiões do Amazonas, própria para construcções.
\section{Uplot}
\begin{itemize}
\item {Grp. gram.:m.}
\end{itemize}
\begin{itemize}
\item {Utilização:Ant.}
\end{itemize}
\begin{itemize}
\item {Proveniência:(T. do Guzerate)}
\end{itemize}
O mesmo que \textunderscore pucho\textunderscore .
\section{Upo}
\begin{itemize}
\item {Grp. gram.:m.}
\end{itemize}
Espécie de beleguim na China. Cf. \textunderscore Peregrinação\textunderscore , LXXXIX.
\section{Úpsilo}
\begin{itemize}
\item {Grp. gram.:m.}
\end{itemize}
O mesmo ou melhór que \textunderscore úpsilon\textunderscore .
\section{Upsilóide}
\begin{itemize}
\item {Grp. gram.:f.}
\end{itemize}
\begin{itemize}
\item {Utilização:Anat.}
\end{itemize}
\begin{itemize}
\item {Proveniência:(Do gr. \textunderscore úpsilon\textunderscore  + \textunderscore eidos\textunderscore )}
\end{itemize}
Sutura do crânio, em fórma de Y.
\section{Úpsilon}
\begin{itemize}
\item {Grp. gram.:m.}
\end{itemize}
Nome da letra grega, que uns representam por \textunderscore u\textunderscore  e outros por \textunderscore i\textunderscore  ou \textunderscore y\textunderscore .
Designação da letra \textunderscore y\textunderscore .
\section{Uptioto}
\begin{itemize}
\item {Grp. gram.:m.}
\end{itemize}
Gênero de arachnídeos.
\section{Uqueté}
\begin{itemize}
\item {Grp. gram.:m.}
\end{itemize}
Gênero de arbustos medicinaes da ilha de San-Thomé.
\section{Uqueté-de-água}
\begin{itemize}
\item {Grp. gram.:m.}
\end{itemize}
Espécie de planta medicinal da ilha de San-Thomé.
\section{Uqueté-de-obó}
\begin{itemize}
\item {Grp. gram.:m.}
\end{itemize}
Espécie de árvore medicinal da ilha do San-Thomé.
\section{Ura}
\begin{itemize}
\item {Grp. gram.:f.}
\end{itemize}
\begin{itemize}
\item {Utilização:Bras. do N}
\end{itemize}
\begin{itemize}
\item {Proveniência:(T. tupi)}
\end{itemize}
Larva, que se cria nas feridas dos animaes.
\section{Uraca}
\begin{itemize}
\item {Grp. gram.:f.}
\end{itemize}
Vinho de cachos de palmeira.
\section{Úracho}
\begin{itemize}
\item {fónica:co}
\end{itemize}
\begin{itemize}
\item {Grp. gram.:m.}
\end{itemize}
\begin{itemize}
\item {Utilização:Anat.}
\end{itemize}
\begin{itemize}
\item {Proveniência:(Do gr. \textunderscore ourakhos\textunderscore )}
\end{itemize}
Cordão, que vai da bexiga ao umbigo do féto.
\section{Úraco}
\begin{itemize}
\item {Grp. gram.:m.}
\end{itemize}
\begin{itemize}
\item {Utilização:Anat.}
\end{itemize}
\begin{itemize}
\item {Proveniência:(Do gr. \textunderscore ourakhos\textunderscore )}
\end{itemize}
Cordão, que vai da bexiga ao umbigo do féto.
\section{Uraconisa}
\begin{itemize}
\item {Grp. gram.:f.}
\end{itemize}
\begin{itemize}
\item {Utilização:Miner.}
\end{itemize}
Peróxydo de urânio.
\section{Uracrasia}
\begin{itemize}
\item {Grp. gram.:f.}
\end{itemize}
\begin{itemize}
\item {Utilização:Med.}
\end{itemize}
\begin{itemize}
\item {Proveniência:(Do gr. \textunderscore oura\textunderscore  + \textunderscore krasis\textunderscore )}
\end{itemize}
Incontinência de urinas.
\section{Uracupa}
\begin{itemize}
\item {Grp. gram.:f.}
\end{itemize}
\begin{itemize}
\item {Utilização:Bras}
\end{itemize}
O mesmo que \textunderscore aracupa\textunderscore .
\section{Urago}
\begin{itemize}
\item {Grp. gram.:m.}
\end{itemize}
\begin{itemize}
\item {Proveniência:(Do gr. \textunderscore ouragos\textunderscore )}
\end{itemize}
Commandante da última linha, no antigo exército grego.
\section{Uraliano}
\begin{itemize}
\item {Grp. gram.:adj.}
\end{itemize}
Relativo aos montes Urales ou aos seus habitadores.
\section{Uralite}
\begin{itemize}
\item {Grp. gram.:f.}
\end{itemize}
Novo material de construcção, formado de cimento e amianto, próprio para proteger paredes e tectos, contra o calor e a humidade.
\section{Uralo...}
\begin{itemize}
\item {Grp. gram.:pref.}
\end{itemize}
(designativo da região dos montes Urales)
\section{Uralo-altaico}
\begin{itemize}
\item {Grp. gram.:adj.}
\end{itemize}
Relativo aos Urales e ao Altai.
Diz-se especialmente das línguas relativas aos povos que demoram entre os Urales e o Altai.
\section{Uranar}
\begin{itemize}
\item {Grp. gram.:v. t.}
\end{itemize}
Misturar ou combinar com urânio.
\section{Uranato}
\begin{itemize}
\item {Grp. gram.:m.}
\end{itemize}
\begin{itemize}
\item {Proveniência:(De \textunderscore urânio\textunderscore )}
\end{itemize}
Sal, resultante da combinação do óxydo urânico com uma base.
\section{Urânia}
\begin{itemize}
\item {Grp. gram.:f.}
\end{itemize}
Formoso lepidóptero nocturno de Madagáscar e das costas da Índia.
Planeta telescópico, descoberto em 1854.
Planta ornamental do Brasil. Cf. \textunderscore Jorn.-do-Comm.\textunderscore , do Rio, de 29-V-902.
\section{Urânico}
\begin{itemize}
\item {Grp. gram.:adj.}
\end{itemize}
Diz-se do óxydo e dos saes, produzidos pelo urânio.
\section{Urânio}
\begin{itemize}
\item {Grp. gram.:m.}
\end{itemize}
\begin{itemize}
\item {Proveniência:(De \textunderscore Úrano\textunderscore , n. p.)}
\end{itemize}
Corpo simples e metállico.
\section{Uranismo}
\begin{itemize}
\item {Grp. gram.:m.}
\end{itemize}
Inversão sexual.
Homosexualidade, perversão que arrasta o indivíduo para outro do mesmo sexo.
\section{Uranista}
\begin{itemize}
\item {Grp. gram.:m.  e  f.}
\end{itemize}
Pessôa, que tem a perversão do uranismo.
\section{Uranita}
\begin{itemize}
\item {Grp. gram.:f.}
\end{itemize}
Phosphato de urânio e de cal.
\section{Uranite}
\begin{itemize}
\item {Grp. gram.:f.}
\end{itemize}
Fosfato de urânio e de cal.
\section{Uranito}
\begin{itemize}
\item {Grp. gram.:m.}
\end{itemize}
O mesmo ou melhór que \textunderscore uranita\textunderscore .
\section{Úrano}
\begin{itemize}
\item {Grp. gram.:m.}
\end{itemize}
\begin{itemize}
\item {Proveniência:(Gr. \textunderscore ouranos\textunderscore )}
\end{itemize}
Nome do planeta mais distante de nós.
\section{Uranocro}
\begin{itemize}
\item {Grp. gram.:m.}
\end{itemize}
Óxydo de urânio.
\section{Uranognosia}
\begin{itemize}
\item {Grp. gram.:f.}
\end{itemize}
\begin{itemize}
\item {Proveniência:(Do gr. \textunderscore ouranos\textunderscore  + \textunderscore gnosis\textunderscore )}
\end{itemize}
O mesmo que \textunderscore Astronomia\textunderscore .
\section{Uranognóstico}
\begin{itemize}
\item {Grp. gram.:adj.}
\end{itemize}
Relativo á Uranognosia.
\section{Uranografia}
\begin{itemize}
\item {Grp. gram.:f.}
\end{itemize}
\begin{itemize}
\item {Proveniência:(De \textunderscore uranógrafo\textunderscore )}
\end{itemize}
Descripção do céu.
\section{Uranográfico}
\begin{itemize}
\item {Grp. gram.:adj.}
\end{itemize}
Relativo á Uranografia.
\section{Uranógrafo}
\begin{itemize}
\item {Grp. gram.:m.}
\end{itemize}
\begin{itemize}
\item {Proveniência:(Do gr. \textunderscore ouranos\textunderscore  + \textunderscore graphein\textunderscore )}
\end{itemize}
Aquele que é versado em Uranografia.
Astrónomo.
\section{Uranographia}
\begin{itemize}
\item {Grp. gram.:f.}
\end{itemize}
\begin{itemize}
\item {Proveniência:(De \textunderscore uranógrapho\textunderscore )}
\end{itemize}
Descripção do céu.
\section{Uranográphico}
\begin{itemize}
\item {Grp. gram.:adj.}
\end{itemize}
Relativo á Uranographia.
\section{Uranógrapho}
\begin{itemize}
\item {Grp. gram.:m.}
\end{itemize}
\begin{itemize}
\item {Proveniência:(Do gr. \textunderscore ouranos\textunderscore  + \textunderscore graphein\textunderscore )}
\end{itemize}
Aquelle que é versado em Uranographia.
Astrónomo.
\section{Uranólitho}
\begin{itemize}
\item {Grp. gram.:m.}
\end{itemize}
\begin{itemize}
\item {Proveniência:(Do gr. \textunderscore ouranos\textunderscore  + \textunderscore lithos\textunderscore )}
\end{itemize}
O mesmo que \textunderscore aerólitho\textunderscore .
\section{Uranólito}
\begin{itemize}
\item {Grp. gram.:m.}
\end{itemize}
\begin{itemize}
\item {Proveniência:(Do gr. \textunderscore ouranos\textunderscore  + \textunderscore lithos\textunderscore )}
\end{itemize}
O mesmo que \textunderscore aerólito\textunderscore .
\section{Uranologia}
\textunderscore f.\textunderscore  (e der.)
O mesmo que \textunderscore Uranographia\textunderscore , etc.
Estudo do estado dos céus, nas diversas épocas da idade da Terra.
\section{Uranológico}
\begin{itemize}
\item {Grp. gram.:adj.}
\end{itemize}
Relativo á Uranologia.
\section{Uranometria}
\begin{itemize}
\item {Grp. gram.:f.}
\end{itemize}
Applicação do uranómetro.
\section{Uranométrico}
\begin{itemize}
\item {Grp. gram.:adj.}
\end{itemize}
Relativo á uranometria.
\section{Uranómetro}
\begin{itemize}
\item {Grp. gram.:m.}
\end{itemize}
\begin{itemize}
\item {Proveniência:(Do gr. \textunderscore ouranos\textunderscore  + \textunderscore metron\textunderscore )}
\end{itemize}
Instrumento, com que se medem as distâncias celestes.
\section{Uranoplastia}
\begin{itemize}
\item {Grp. gram.:f.}
\end{itemize}
\begin{itemize}
\item {Utilização:Med.}
\end{itemize}
\begin{itemize}
\item {Proveniência:(Do gr. \textunderscore ouranos\textunderscore  + \textunderscore plassein\textunderscore )}
\end{itemize}
Restauração do véu palatino.
\section{Uranorama}
\begin{itemize}
\item {Grp. gram.:m.}
\end{itemize}
\begin{itemize}
\item {Proveniência:(Do gr. \textunderscore ouranos\textunderscore  + \textunderscore horama\textunderscore )}
\end{itemize}
Vista do céu, ou exposição do systema planetário, por meio de um globo móvel.
\section{Uranoscopia}
\begin{itemize}
\item {Grp. gram.:f.}
\end{itemize}
\begin{itemize}
\item {Proveniência:(Do gr. \textunderscore ouranos\textunderscore  + \textunderscore skopein\textunderscore )}
\end{itemize}
O mesmo que \textunderscore Astrologia\textunderscore .
\section{Uranoso}
\begin{itemize}
\item {Grp. gram.:adj.}
\end{itemize}
\begin{itemize}
\item {Utilização:Miner.}
\end{itemize}
Diz-se do primeiro óxydo e dos saes de urânio.
\section{Uranosteoplastia}
\begin{itemize}
\item {Grp. gram.:f.}
\end{itemize}
\begin{itemize}
\item {Proveniência:(Do gr. \textunderscore ouranos\textunderscore  + \textunderscore osteon\textunderscore  + \textunderscore plassein\textunderscore )}
\end{itemize}
O mesmo que \textunderscore uranoplastia\textunderscore .
\section{Urantera}
\begin{itemize}
\item {Grp. gram.:f.}
\end{itemize}
\begin{itemize}
\item {Proveniência:(Do gr. \textunderscore oura\textunderscore  + \textunderscore antheros\textunderscore )}
\end{itemize}
Gênero de plantas melastomáceas.
\section{Uranthera}
\begin{itemize}
\item {Grp. gram.:f.}
\end{itemize}
\begin{itemize}
\item {Proveniência:(Do gr. \textunderscore oura\textunderscore  + \textunderscore antheros\textunderscore )}
\end{itemize}
Gênero de plantas melastomáceas.
\section{Urao}
\begin{itemize}
\item {Grp. gram.:m.}
\end{itemize}
Espécie de soda natural de algumas minas do México e de Venezuela.
\section{Uraptérice}
\begin{itemize}
\item {Grp. gram.:m.}
\end{itemize}
Gênero de insectos lepidópteros nocturnos.
\section{Uraptéryce}
\begin{itemize}
\item {Grp. gram.:m.}
\end{itemize}
Gênero de insectos lepidópteros nocturnos.
\section{Urapuru}
\begin{itemize}
\item {Grp. gram.:m.}
\end{itemize}
Pássaro canoro do Brasil.
\section{Uraquitan}
\begin{itemize}
\item {Grp. gram.:m.}
\end{itemize}
\begin{itemize}
\item {Utilização:Bras}
\end{itemize}
Variedade de pedra verde, brilhante e estimada.
\section{Urária}
\begin{itemize}
\item {Grp. gram.:f.}
\end{itemize}
Gênero de plantas leguminosas.
\section{Uraricus}
\begin{itemize}
\item {Grp. gram.:m. pl.}
\end{itemize}
Indígenas do norte do Brasil.
\section{Urato}
\begin{itemize}
\item {Grp. gram.:m.}
\end{itemize}
\begin{itemize}
\item {Proveniência:(Do rad. de \textunderscore urina\textunderscore )}
\end{itemize}
Sal, resultante da combinação ácido úrico com uma base.
\section{Uraúna}
\begin{itemize}
\item {Grp. gram.:f.}
\end{itemize}
Ave do Brasil.
\section{Urbanamente}
\begin{itemize}
\item {Grp. gram.:adv.}
\end{itemize}
De modo urbano; com delicadeza; com polidez.
\section{Urbaniciano}
\begin{itemize}
\item {Grp. gram.:adj.}
\end{itemize}
\begin{itemize}
\item {Proveniência:(Lat. \textunderscore urbanicianus\textunderscore )}
\end{itemize}
Dizia-se do soldado, que fazia parte da guarnição do Roma, depois do tempo de Augusto.
\section{Urbanidade}
\begin{itemize}
\item {Grp. gram.:f.}
\end{itemize}
\begin{itemize}
\item {Proveniência:(Do lat. \textunderscore urbanitas\textunderscore )}
\end{itemize}
Qualidade do que é urbano.
Civilidade.
Cortesia; affabilidade.
\section{Urbanita}
\begin{itemize}
\item {Grp. gram.:m. ,  f.  e  adj.}
\end{itemize}
\begin{itemize}
\item {Proveniência:(De \textunderscore urbano\textunderscore )}
\end{itemize}
Pessôa, que reside numa cidade.
\section{Urbanizar}
\begin{itemize}
\item {Grp. gram.:v. t.}
\end{itemize}
\begin{itemize}
\item {Utilização:Ext.}
\end{itemize}
Tornar urbano.
Civilizar.
\section{Urbano}
\begin{itemize}
\item {Grp. gram.:adj.}
\end{itemize}
\begin{itemize}
\item {Utilização:Fig.}
\end{itemize}
\begin{itemize}
\item {Grp. gram.:M.}
\end{itemize}
\begin{itemize}
\item {Utilização:Bras}
\end{itemize}
\begin{itemize}
\item {Proveniência:(Lat. \textunderscore urbanus\textunderscore )}
\end{itemize}
Relativo á cidade: \textunderscore a população urbana\textunderscore .
Affável; cortês; civilizado.
Diz-se dos prédios, próprios para habitação, em opposição a prédios rústicos ou prédios para cultivar.
Guarda urbano, polícia civil.
\section{Urbeque}
\begin{itemize}
\item {Grp. gram.:m.}
\end{itemize}
Insecto parasito das árvores fructíferas, que lhes ataca os gomos, (\textunderscore rhynchites betuleti\textunderscore ). Cf. P. Moraes, \textunderscore Zool. Elem.\textunderscore , 732.
\section{Urca}
\begin{itemize}
\item {Grp. gram.:f.}
\end{itemize}
\begin{itemize}
\item {Utilização:Pop.}
\end{itemize}
\begin{itemize}
\item {Grp. gram.:Adj.}
\end{itemize}
\begin{itemize}
\item {Utilização:Bras}
\end{itemize}
Antiga embarcação portuguesa, muito larga.
Mulhér gorda e feia.
Grande, enorme.
(Cp. \textunderscore urco\textunderscore )
\section{Urca}
\begin{itemize}
\item {Grp. gram.:f.}
\end{itemize}
\begin{itemize}
\item {Utilização:Prov.}
\end{itemize}
\begin{itemize}
\item {Utilização:beir.}
\end{itemize}
Pequeno pássaro do campo. (Colhido em Sátão)
\section{Urcela}
\begin{itemize}
\item {Grp. gram.:f.}
\end{itemize}
\begin{itemize}
\item {Utilização:T. da Bairrada}
\end{itemize}
Cada uma das peças de madeira, que se erguem sôbre um dos lados do lagar de vinho e entre as quaes há uma travessa que serve de eixo á vara do lagar.
\section{Urcéola}
\begin{itemize}
\item {Grp. gram.:f.}
\end{itemize}
O mesmo que \textunderscore urcéolo\textunderscore .
\section{Urceolado}
\begin{itemize}
\item {Grp. gram.:adj.}
\end{itemize}
\begin{itemize}
\item {Utilização:Bot.}
\end{itemize}
\begin{itemize}
\item {Proveniência:(De \textunderscore urcéolo\textunderscore )}
\end{itemize}
Diz-se de um órgão vegetal, bojudo na parte média, apertado no orifício e dilatado no limbo.
\section{Urceolar}
\begin{itemize}
\item {Grp. gram.:adj.}
\end{itemize}
O mesmo que \textunderscore urceolado\textunderscore .
\section{Urceolífero}
\begin{itemize}
\item {Grp. gram.:adj.}
\end{itemize}
\begin{itemize}
\item {Utilização:Bot.}
\end{itemize}
\begin{itemize}
\item {Proveniência:(Do lat. \textunderscore urceolus\textunderscore  + \textunderscore ferre\textunderscore )}
\end{itemize}
Que tem urcéolos.
\section{Urcéolo}
\begin{itemize}
\item {Grp. gram.:m.}
\end{itemize}
\begin{itemize}
\item {Proveniência:(Lat. \textunderscore urceolus\textunderscore )}
\end{itemize}
Órgão vegetal, em fórma de copo ou tigelinha.
\section{Urchila}
\begin{itemize}
\item {Grp. gram.:f.}
\end{itemize}
Substância vegetal, roxa, usada em Pintura.
(Cast. \textunderscore urchilla\textunderscore )
\section{Urchilla}
\begin{itemize}
\item {Grp. gram.:f.}
\end{itemize}
Substância vegetal, roxa, usada em Pintura.
(Cast. \textunderscore urchilla\textunderscore )
\section{Urco}
\begin{itemize}
\item {Grp. gram.:m.}
\end{itemize}
\begin{itemize}
\item {Grp. gram.:Adj.}
\end{itemize}
\begin{itemize}
\item {Utilização:Bras. do N}
\end{itemize}
Cavallo forte e corpulento, também conhecido por \textunderscore frisão\textunderscore .
Grande; formidável.
\section{Urdição}
\begin{itemize}
\item {Grp. gram.:f.}
\end{itemize}
O mesmo que \textunderscore urdidura\textunderscore . Cf. \textunderscore Inquér. Industr.\textunderscore , p. II, V. III, 75.
\section{Urdideira}
\begin{itemize}
\item {Grp. gram.:f.  e  adj.}
\end{itemize}
\begin{itemize}
\item {Grp. gram.:F.}
\end{itemize}
\begin{itemize}
\item {Proveniência:(De \textunderscore urdir\textunderscore )}
\end{itemize}
Diz-se da mulhér que urde ou tece.
Conjunto de duas peças parallelas e verticaes, guarnecidas de pregos de madeira, em que se urdem os ramos da teia.--Na Beira-Baixa, em vez de pregos de madeira, também se usam ganchos de ferro, cravados em parede.
\section{Urdidor}
\begin{itemize}
\item {Grp. gram.:m.  e  adj.}
\end{itemize}
\begin{itemize}
\item {Grp. gram.:M.}
\end{itemize}
O que urde.
Caixa baixa, com casas em que estão os novelos, donde se tiram os fios que formam o ramo da teia.
\section{Urdidura}
\begin{itemize}
\item {Grp. gram.:f.}
\end{itemize}
Acto ou effeito de urdir.
\section{Urdim}
\begin{itemize}
\item {Grp. gram.:m.}
\end{itemize}
\begin{itemize}
\item {Utilização:P. us.}
\end{itemize}
O mesmo que \textunderscore urdimento\textunderscore .
\section{Urdimaças}
\begin{itemize}
\item {Grp. gram.:m.  e  f.}
\end{itemize}
\begin{itemize}
\item {Utilização:Pop.}
\end{itemize}
\begin{itemize}
\item {Grp. gram.:Pl.}
\end{itemize}
\begin{itemize}
\item {Proveniência:(De \textunderscore urdir\textunderscore )}
\end{itemize}
Pessôa intriguista e mexeriqueira.
Enredos, intrigas.
\section{Urdimalas}
\begin{itemize}
\item {Grp. gram.:m.  e  f.}
\end{itemize}
O mesmo que \textunderscore urdimaças\textunderscore .
\section{Urdimento}
\begin{itemize}
\item {Grp. gram.:m.}
\end{itemize}
O mesmo que \textunderscore urdidura\textunderscore .
Travejamento do tecto dos palcos, e os sótãos que lhe ficam por cima.
\section{Urdir}
\begin{itemize}
\item {Grp. gram.:v. t.}
\end{itemize}
\begin{itemize}
\item {Utilização:Fig.}
\end{itemize}
\begin{itemize}
\item {Proveniência:(Do lat. \textunderscore ordiri\textunderscore )}
\end{itemize}
Pôr por ordem ou dispor (os fios da teia), para se fazer o tecido.
Enredar; intrigar.
Maquinar.
\section{Urdu}
\begin{itemize}
\item {Grp. gram.:m.}
\end{itemize}
Língua moderna da Índia, mesclada de estranjeirismos árabes e persas.
O mesmo que \textunderscore indostano\textunderscore .
\section{Urdume}
\begin{itemize}
\item {Grp. gram.:m.}
\end{itemize}
O mesmo que \textunderscore urdidura\textunderscore .
\section{Uréa}
\begin{itemize}
\item {Grp. gram.:f.}
\end{itemize}
Um dos princípios immediatos da urina.
(Cp. \textunderscore urina\textunderscore )
\section{Uredíneo}
\begin{itemize}
\item {Grp. gram.:adj.}
\end{itemize}
\begin{itemize}
\item {Grp. gram.:F. pl.}
\end{itemize}
Relativo ao uredo^2, ou parecido com elle.
Família de cogumelos, que têm por typo o uredo^2.
\section{Uredo}
\begin{itemize}
\item {fónica:urê}
\end{itemize}
\begin{itemize}
\item {Grp. gram.:m.}
\end{itemize}
\begin{itemize}
\item {Proveniência:(Lat. \textunderscore uredo\textunderscore )}
\end{itemize}
Comichão, ardor.
\section{Uredo}
\begin{itemize}
\item {fónica:urê}
\end{itemize}
\begin{itemize}
\item {Grp. gram.:m.}
\end{itemize}
Cogumelo na urina de certos oxalúricos.
(Cp. \textunderscore urina\textunderscore )
\section{Ureia}
\begin{itemize}
\item {Grp. gram.:f.}
\end{itemize}
Um dos princípios immediatos da urina.
(Cp. \textunderscore urina\textunderscore )
\section{Urélia}
\begin{itemize}
\item {Grp. gram.:f.}
\end{itemize}
Gênero de insectos dípteros.
\section{Uremia}
\begin{itemize}
\item {Grp. gram.:f.}
\end{itemize}
\begin{itemize}
\item {Proveniência:(Do gr. \textunderscore ouron\textunderscore  + \textunderscore haima\textunderscore )}
\end{itemize}
Estado mórbido, resultante talvez da retenção da ureia no sangue.
\section{Urêmico}
\begin{itemize}
\item {Grp. gram.:adj.}
\end{itemize}
Relativo á uremia.
\section{Urente}
\begin{itemize}
\item {Grp. gram.:adj.}
\end{itemize}
\begin{itemize}
\item {Proveniência:(Lat. \textunderscore urens\textunderscore )}
\end{itemize}
Que queima; ardente.
\section{Urentemente}
\begin{itemize}
\item {Grp. gram.:adv.}
\end{itemize}
\begin{itemize}
\item {Utilização:P. us.}
\end{itemize}
De modo urente.
Ardentemente.
\section{Ureómetro}
\begin{itemize}
\item {Grp. gram.:m.}
\end{itemize}
\begin{itemize}
\item {Utilização:Med.}
\end{itemize}
\begin{itemize}
\item {Proveniência:(De \textunderscore ureia\textunderscore  + gr. \textunderscore metron\textunderscore )}
\end{itemize}
Instrumento, para a dosagem da ureia.
\section{Uretana}
\begin{itemize}
\item {Grp. gram.:f.}
\end{itemize}
\begin{itemize}
\item {Utilização:Chím.}
\end{itemize}
Carbonato anhydro de ammoníaco e de gás oleificante.
\section{Uréter}
\begin{itemize}
\item {Grp. gram.:m.}
\end{itemize}
\begin{itemize}
\item {Proveniência:(Gr. \textunderscore oureter\textunderscore )}
\end{itemize}
Cada um dos canaes membranosos, que conduzem a urina, dos rins para a bexiga.
Pl. \textunderscore uretéres\textunderscore , como de \textunderscore carácter caractéres\textunderscore .
\section{Urétera}
\begin{itemize}
\item {Grp. gram.:f.}
\end{itemize}
(V.urethra)
\section{Ureteralgia}
\begin{itemize}
\item {Grp. gram.:f.}
\end{itemize}
\begin{itemize}
\item {Proveniência:(Do gr. \textunderscore oureter\textunderscore  + \textunderscore algos\textunderscore )}
\end{itemize}
Dôr nos ureteres.
\section{Ureterálgico}
\begin{itemize}
\item {Grp. gram.:adj.}
\end{itemize}
Relativo á ureteralgia.
\section{Uretére}
\begin{itemize}
\item {Grp. gram.:m.}
\end{itemize}
Fórma e pronúncia incorrecta, em vez de \textunderscore uréter\textunderscore .(V.uréter)
\section{Uretérico}
\begin{itemize}
\item {Grp. gram.:adj.}
\end{itemize}
Relativo ao uréter.
\section{Ureterite}
\begin{itemize}
\item {Grp. gram.:f.}
\end{itemize}
\begin{itemize}
\item {Proveniência:(De \textunderscore uréter\textunderscore )}
\end{itemize}
Inflammação dos ureteres.
\section{Ureterolithíase}
\begin{itemize}
\item {Grp. gram.:f.}
\end{itemize}
\begin{itemize}
\item {Utilização:Med.}
\end{itemize}
\begin{itemize}
\item {Proveniência:(Do gr. \textunderscore oureter\textunderscore  + \textunderscore lithos\textunderscore )}
\end{itemize}
Retenção de cálculos nos ureteres.
\section{Ureterolíthico}
\begin{itemize}
\item {Grp. gram.:adj.}
\end{itemize}
Relativo á ureterolithíase.
Produzido pela presença de cálculos nos ureteres.
\section{Ureterolitíase}
\begin{itemize}
\item {Grp. gram.:f.}
\end{itemize}
\begin{itemize}
\item {Utilização:Med.}
\end{itemize}
\begin{itemize}
\item {Proveniência:(Do gr. \textunderscore oureter\textunderscore  + \textunderscore lithos\textunderscore )}
\end{itemize}
Retenção de cálculos nos ureteres.
\section{Ureterolítico}
\begin{itemize}
\item {Grp. gram.:adj.}
\end{itemize}
Relativo á ureterolitíase.
Produzido pela presença de cálculos nos ureteres.
\section{Ureterostomático}
\begin{itemize}
\item {Grp. gram.:adj.}
\end{itemize}
\begin{itemize}
\item {Utilização:Anat.}
\end{itemize}
\begin{itemize}
\item {Utilização:Med.}
\end{itemize}
\begin{itemize}
\item {Proveniência:(Do gr. \textunderscore oureter\textunderscore  + \textunderscore stoma\textunderscore )}
\end{itemize}
Relativo ao orifício dos ureteres.
Causado pela obstrucção do orifício dos ureteres na bexiga.
\section{Urethra}
\begin{itemize}
\item {Grp. gram.:f.}
\end{itemize}
\begin{itemize}
\item {Utilização:Anat.}
\end{itemize}
\begin{itemize}
\item {Proveniência:(Lat. \textunderscore urethra\textunderscore )}
\end{itemize}
Canal excretor da urina.
\section{Urethral}
\begin{itemize}
\item {Grp. gram.:adj.}
\end{itemize}
Relativo á urethra.
\section{Urethralgia}
\begin{itemize}
\item {Grp. gram.:f.}
\end{itemize}
\begin{itemize}
\item {Proveniência:(Do gr. \textunderscore ourethra\textunderscore  + \textunderscore algos\textunderscore )}
\end{itemize}
Dôr na urethra.
\section{Urethrálgico}
\begin{itemize}
\item {Grp. gram.:adj.}
\end{itemize}
Relativo á urethralgia.
\section{Urethrelmínthico}
\begin{itemize}
\item {Grp. gram.:adj.}
\end{itemize}
\begin{itemize}
\item {Utilização:Med.}
\end{itemize}
\begin{itemize}
\item {Proveniência:(De \textunderscore urethra\textunderscore  + \textunderscore helmintho\textunderscore )}
\end{itemize}
Causado pela presença de vermes na urethra.
\section{Uréthrico}
\begin{itemize}
\item {Grp. gram.:adj.}
\end{itemize}
O mesmo que \textunderscore urethral\textunderscore .
\section{Urethrite}
\begin{itemize}
\item {Grp. gram.:f.}
\end{itemize}
Inflammação da urethra.
\section{Urethrocystomia}
\begin{itemize}
\item {Grp. gram.:f.}
\end{itemize}
\begin{itemize}
\item {Utilização:Cir.}
\end{itemize}
Operação da talha.
\section{Urethrolíthico}
\begin{itemize}
\item {Grp. gram.:adj.}
\end{itemize}
\begin{itemize}
\item {Utilização:Med.}
\end{itemize}
\begin{itemize}
\item {Proveniência:(Do gr. \textunderscore ourethra\textunderscore  + \textunderscore lithos\textunderscore )}
\end{itemize}
Causado pela presença de cálculos na urethra.
\section{Urethrophraxia}
\begin{itemize}
\item {fónica:csi}
\end{itemize}
\begin{itemize}
\item {Grp. gram.:f.}
\end{itemize}
\begin{itemize}
\item {Proveniência:(Do gr. \textunderscore ourethra\textunderscore  + \textunderscore phrassein\textunderscore )}
\end{itemize}
Obstrucção da urethra.
\section{Urethroplástica}
\begin{itemize}
\item {Grp. gram.:f.}
\end{itemize}
\begin{itemize}
\item {Proveniência:(Do gr. \textunderscore ourethra\textunderscore  + \textunderscore plassein\textunderscore )}
\end{itemize}
Operação cirúrgica, com que se substitue uma parte perdida da substância da urethra.
\section{Urethrópyco}
\begin{itemize}
\item {Grp. gram.:adj.}
\end{itemize}
\begin{itemize}
\item {Utilização:Med.}
\end{itemize}
\begin{itemize}
\item {Proveniência:(Do gr. \textunderscore ourethra\textunderscore  + \textunderscore puon\textunderscore )}
\end{itemize}
Causado pela accumulação de pus na urethra.
\section{Urethrorragia}
\begin{itemize}
\item {Grp. gram.:f.}
\end{itemize}
\begin{itemize}
\item {Proveniência:(Do gr. \textunderscore ourethra\textunderscore  + \textunderscore regnumi\textunderscore )}
\end{itemize}
Derramamento de sangue pela urethra.
\section{Urethrorraphia}
\begin{itemize}
\item {Grp. gram.:f.}
\end{itemize}
\begin{itemize}
\item {Proveniência:(Do gr. \textunderscore ourethra\textunderscore  + \textunderscore rhaphe\textunderscore )}
\end{itemize}
Sutura de uma fenda da urethra.
\section{Urethrorrheia}
\begin{itemize}
\item {Grp. gram.:f.}
\end{itemize}
\begin{itemize}
\item {Proveniência:(Do gr. \textunderscore ourethra\textunderscore  + \textunderscore rhein\textunderscore )}
\end{itemize}
Fluxo ou corrimento pela urethra.
\section{Urethroscopia}
\begin{itemize}
\item {Grp. gram.:f.}
\end{itemize}
Observação da urethra por meio do urethroscópio.
\section{Urethroscópio}
\begin{itemize}
\item {Grp. gram.:m.}
\end{itemize}
\begin{itemize}
\item {Proveniência:(Do gr. \textunderscore ourethra\textunderscore  + \textunderscore skopein\textunderscore )}
\end{itemize}
Instrumento cirúrgico, para fazer observações no interior da urethra.
\section{Urethróscopo}
\begin{itemize}
\item {Grp. gram.:m.}
\end{itemize}
(V.urethroscópio)
\section{Urethrostenia}
\begin{itemize}
\item {Grp. gram.:f.}
\end{itemize}
\begin{itemize}
\item {Proveniência:(Do gr. \textunderscore ourethra\textunderscore  + \textunderscore stenos\textunderscore )}
\end{itemize}
Apêrto de urethra.
\section{Urethrostênico}
\begin{itemize}
\item {Grp. gram.:adj.}
\end{itemize}
Relativo á urethrostenia.
Que soffre urethrostenia.
\section{Urethrothrombóide}
\begin{itemize}
\item {Grp. gram.:adj.}
\end{itemize}
\begin{itemize}
\item {Utilização:Med.}
\end{itemize}
Causado pela presença de grumos de sangue na urethra.
\section{Urethrotomia}
\begin{itemize}
\item {Grp. gram.:f.}
\end{itemize}
\begin{itemize}
\item {Proveniência:(De \textunderscore urethrótomo\textunderscore )}
\end{itemize}
Incisão na urethra.
\section{Urethrótomo}
\begin{itemize}
\item {Grp. gram.:m.}
\end{itemize}
\begin{itemize}
\item {Proveniência:(Do gr. \textunderscore ourethra\textunderscore  + \textunderscore tome\textunderscore )}
\end{itemize}
Instrumento, para fazer incisões na urethra.
\section{Urethrovesical}
\begin{itemize}
\item {Grp. gram.:adj.}
\end{itemize}
\begin{itemize}
\item {Utilização:Anat.}
\end{itemize}
\begin{itemize}
\item {Proveniência:(De \textunderscore urethra\textunderscore  + \textunderscore vesical\textunderscore )}
\end{itemize}
Relativo á urethra e á bexiga.
\section{Urético}
\begin{itemize}
\item {Grp. gram.:adj.}
\end{itemize}
\begin{itemize}
\item {Proveniência:(Do gr. \textunderscore ouron\textunderscore )}
\end{itemize}
Relativo á urina; diurético.
Diz-se de qualquer enfermidade do canal excretor da urina e, especialmente, de uma espécie de febres, complicada de diabete.
\section{Uretilana}
\begin{itemize}
\item {Grp. gram.:f.}
\end{itemize}
\begin{itemize}
\item {Utilização:Chím.}
\end{itemize}
Carbonato anhydro de ammoníaco e de methylena.
\section{Uretra}
\begin{itemize}
\item {Grp. gram.:f.}
\end{itemize}
\begin{itemize}
\item {Utilização:Anat.}
\end{itemize}
\begin{itemize}
\item {Proveniência:(Lat. \textunderscore urethra\textunderscore )}
\end{itemize}
Canal excretor da urina.
\section{Uretral}
\begin{itemize}
\item {Grp. gram.:adj.}
\end{itemize}
Relativo á uretra.
\section{Uretralgia}
\begin{itemize}
\item {Grp. gram.:f.}
\end{itemize}
\begin{itemize}
\item {Proveniência:(Do gr. \textunderscore ourethra\textunderscore  + \textunderscore algos\textunderscore )}
\end{itemize}
Dôr na uretra.
\section{Uretrálgico}
\begin{itemize}
\item {Grp. gram.:adj.}
\end{itemize}
Relativo á uretralgia.
\section{Uretrelmíntico}
\begin{itemize}
\item {Grp. gram.:adj.}
\end{itemize}
\begin{itemize}
\item {Utilização:Med.}
\end{itemize}
\begin{itemize}
\item {Proveniência:(De \textunderscore uretra\textunderscore  + \textunderscore helminto\textunderscore )}
\end{itemize}
Causado pela presença de vermes na uretra.
\section{Urétrico}
\begin{itemize}
\item {Grp. gram.:adj.}
\end{itemize}
O mesmo que \textunderscore uretral\textunderscore .
\section{Uretrite}
\begin{itemize}
\item {Grp. gram.:f.}
\end{itemize}
Inflamação da uretra.
\section{Uretrocistomia}
\begin{itemize}
\item {Grp. gram.:f.}
\end{itemize}
\begin{itemize}
\item {Utilização:Cir.}
\end{itemize}
Operação da talha.
\section{Uretrofraxia}
\begin{itemize}
\item {fónica:csi}
\end{itemize}
\begin{itemize}
\item {Grp. gram.:f.}
\end{itemize}
\begin{itemize}
\item {Proveniência:(Do gr. \textunderscore ourethra\textunderscore  + \textunderscore phrassein\textunderscore )}
\end{itemize}
Obstrucção da uretra.
\section{Uretrolítico}
\begin{itemize}
\item {Grp. gram.:adj.}
\end{itemize}
\begin{itemize}
\item {Utilização:Med.}
\end{itemize}
\begin{itemize}
\item {Proveniência:(Do gr. \textunderscore ourethra\textunderscore  + \textunderscore lithos\textunderscore )}
\end{itemize}
Causado pela presença de cálculos na uretra.
\section{Uretrópico}
\begin{itemize}
\item {Grp. gram.:adj.}
\end{itemize}
\begin{itemize}
\item {Utilização:Med.}
\end{itemize}
\begin{itemize}
\item {Proveniência:(Do gr. \textunderscore ourethra\textunderscore  + \textunderscore puon\textunderscore )}
\end{itemize}
Causado pela acumulação de pus na uretra.
\section{Uretroplástica}
\begin{itemize}
\item {Grp. gram.:f.}
\end{itemize}
\begin{itemize}
\item {Proveniência:(Do gr. \textunderscore ourethra\textunderscore  + \textunderscore plassein\textunderscore )}
\end{itemize}
Operação cirúrgica, com que se substitue uma parte perdida da substância da uretra.
\section{Uretrorragia}
\begin{itemize}
\item {Grp. gram.:f.}
\end{itemize}
\begin{itemize}
\item {Proveniência:(Do gr. \textunderscore ourethra\textunderscore  + \textunderscore regnumi\textunderscore )}
\end{itemize}
Derramamento de sangue pela uretra.
\section{Uretrorrafia}
\begin{itemize}
\item {Grp. gram.:f.}
\end{itemize}
\begin{itemize}
\item {Proveniência:(Do gr. \textunderscore ourethra\textunderscore  + \textunderscore rhaphe\textunderscore )}
\end{itemize}
Sutura de uma fenda da uretra.
\section{Uretrorreia}
\begin{itemize}
\item {Grp. gram.:f.}
\end{itemize}
\begin{itemize}
\item {Proveniência:(Do gr. \textunderscore ourethra\textunderscore  + \textunderscore rhein\textunderscore )}
\end{itemize}
Fluxo ou corrimento pela uretra.
\section{Uretroscopia}
\begin{itemize}
\item {Grp. gram.:f.}
\end{itemize}
Observação da uretra por meio do uretroscópio.
\section{Uretroscópio}
\begin{itemize}
\item {Grp. gram.:m.}
\end{itemize}
\begin{itemize}
\item {Proveniência:(Do gr. \textunderscore ourethra\textunderscore  + \textunderscore skopein\textunderscore )}
\end{itemize}
Instrumento cirúrgico, para fazer observações no interior da uretra.
\section{Uretróscopo}
\begin{itemize}
\item {Grp. gram.:m.}
\end{itemize}
(V.uretroscópio)
\section{Uretrostenia}
\begin{itemize}
\item {Grp. gram.:f.}
\end{itemize}
\begin{itemize}
\item {Proveniência:(Do gr. \textunderscore ourethra\textunderscore  + \textunderscore stenos\textunderscore )}
\end{itemize}
Apêrto de uretra.
\section{Uretrostênico}
\begin{itemize}
\item {Grp. gram.:adj.}
\end{itemize}
Relativo á uretrostenia.
Que sofre uretrostenia.
\section{Uretrotomia}
\begin{itemize}
\item {Grp. gram.:f.}
\end{itemize}
\begin{itemize}
\item {Proveniência:(De \textunderscore urethrótomo\textunderscore )}
\end{itemize}
Incisão na uretra.
\section{Uretrótomo}
\begin{itemize}
\item {Grp. gram.:m.}
\end{itemize}
\begin{itemize}
\item {Proveniência:(Do gr. \textunderscore ourethra\textunderscore  + \textunderscore tome\textunderscore )}
\end{itemize}
Instrumento, para fazer incisões na uretra.
\section{Uretrotrombóide}
\begin{itemize}
\item {Grp. gram.:adj.}
\end{itemize}
\begin{itemize}
\item {Utilização:Med.}
\end{itemize}
Causado pela presença de grumos de sangue na uretra.
\section{Uretrovesical}
\begin{itemize}
\item {Grp. gram.:adj.}
\end{itemize}
\begin{itemize}
\item {Utilização:Anat.}
\end{itemize}
\begin{itemize}
\item {Proveniência:(De \textunderscore urethra\textunderscore  + \textunderscore vesical\textunderscore )}
\end{itemize}
Relativo á uretra e á bexiga.
\section{Urge}
\begin{itemize}
\item {Grp. gram.:m.}
\end{itemize}
(V.uge)
\section{Urge}
\textunderscore f.\textunderscore  (e der.) \textunderscore Prov. minh.\textunderscore 
O mesmo que \textunderscore urze\textunderscore , etc.
\section{Urgebão}
\begin{itemize}
\item {Grp. gram.:m.}
\end{itemize}
\begin{itemize}
\item {Proveniência:(Do gr. \textunderscore hiera\textunderscore  + \textunderscore botane\textunderscore )}
\end{itemize}
Planta verbenácea.
\section{Urgência}
\begin{itemize}
\item {Grp. gram.:f.}
\end{itemize}
\begin{itemize}
\item {Proveniência:(Lat. \textunderscore urgentia\textunderscore )}
\end{itemize}
Qualidade do que é urgente; pressa.
\section{Urgente}
\begin{itemize}
\item {Grp. gram.:adj.}
\end{itemize}
\begin{itemize}
\item {Proveniência:(Lat. \textunderscore urgens\textunderscore )}
\end{itemize}
Que urge.
Que é preciso fazer-se rapidamente.
Indispensável.
Imminente.
\section{Urgentemente}
\begin{itemize}
\item {Grp. gram.:adv.}
\end{itemize}
De modo urgente.
\section{Úrgico}
\begin{itemize}
\item {Grp. gram.:adj.}
\end{itemize}
\begin{itemize}
\item {Utilização:Burl.}
\end{itemize}
O mesmo que \textunderscore urgente\textunderscore .
\section{Urgir}
\begin{itemize}
\item {Grp. gram.:v. i.}
\end{itemize}
\begin{itemize}
\item {Grp. gram.:V. t.}
\end{itemize}
\begin{itemize}
\item {Proveniência:(Lat. \textunderscore urgere\textunderscore )}
\end{itemize}
Sêr necessário sem demora; não permitir demora.
Instar.
Impellir, perseguir de perto.
Comprimir, impellindo.
\section{Urgueira}
\begin{itemize}
\item {Grp. gram.:f.}
\end{itemize}
\begin{itemize}
\item {Utilização:Prov.}
\end{itemize}
O mesmo que \textunderscore urze\textunderscore . Cf. Camillo, \textunderscore Doze Casam.\textunderscore , 120.
(Cp. \textunderscore urzeira\textunderscore )
\section{Uricana}
\begin{itemize}
\item {Grp. gram.:f.}
\end{itemize}
\begin{itemize}
\item {Utilização:Bras}
\end{itemize}
Espécie de palmeira.
\section{Uricemia}
\begin{itemize}
\item {Grp. gram.:f.}
\end{itemize}
\begin{itemize}
\item {Proveniência:(De \textunderscore úrico\textunderscore  + gr. \textunderscore haima\textunderscore )}
\end{itemize}
Estado mórbido, manifestado por excesso de ácido úrico na urina.
\section{Úrico}
\begin{itemize}
\item {Grp. gram.:adj.}
\end{itemize}
Diz se de um ácido, contido nas urinas.
Resultante da ureia.
\section{Urida}
\begin{itemize}
\item {Grp. gram.:f.}
\end{itemize}
Planta indiana, (\textunderscore phaseolus max\textunderscore ).
\section{Urim}
\begin{itemize}
\item {Grp. gram.:m.}
\end{itemize}
Adôrno, que o supremo sacerdote dos Judeus punha ao peito, quando tinha de consultar a Deus, nos casos mais graves de interesse público.
\section{Urina}
\begin{itemize}
\item {Grp. gram.:f.}
\end{itemize}
\begin{itemize}
\item {Proveniência:(Lat. \textunderscore urina\textunderscore )}
\end{itemize}
Líquido excrementício, segregado pelos rins, donde corre pelos ureteres para a bexiga.
\section{Urinação}
\begin{itemize}
\item {Grp. gram.:f.}
\end{itemize}
Acto ou effeito de urinar.
\section{Urinanás}
\begin{itemize}
\item {Grp. gram.:m. Pl.}
\end{itemize}
Indígenas do Norte do Brasil.
\section{Urinar}
\begin{itemize}
\item {Grp. gram.:v. i.}
\end{itemize}
\begin{itemize}
\item {Grp. gram.:V. t.}
\end{itemize}
\begin{itemize}
\item {Utilização:Fam.}
\end{itemize}
Expellir urina pela via natural.
Expellir com urina ou de mistura com urina: \textunderscore urinar sangue\textunderscore .
Sujar com urina, expellir urina sôbre: \textunderscore urinar a cama\textunderscore .
\section{Urinário}
\begin{itemize}
\item {Grp. gram.:adj.}
\end{itemize}
Relativo á urina.
\section{Urinatório}
\begin{itemize}
\item {Grp. gram.:m.}
\end{itemize}
\begin{itemize}
\item {Utilização:Neol.}
\end{itemize}
O mesmo que \textunderscore urinol\textunderscore .
\section{Urinífero}
\begin{itemize}
\item {Grp. gram.:adj.}
\end{itemize}
\begin{itemize}
\item {Proveniência:(Do lat. \textunderscore urina\textunderscore  + \textunderscore ferre\textunderscore )}
\end{itemize}
Que contém urina.
Que conduz urina.
\section{Uriníparo}
\begin{itemize}
\item {Grp. gram.:adj.}
\end{itemize}
\begin{itemize}
\item {Proveniência:(Do lat. \textunderscore urina\textunderscore  + \textunderscore p[-a]rere\textunderscore )}
\end{itemize}
Que produz urina.
\section{Urinol}
\begin{itemize}
\item {Grp. gram.:m.}
\end{itemize}
\begin{itemize}
\item {Proveniência:(De \textunderscore urina\textunderscore )}
\end{itemize}
Vaso ou lugar, preparado para nêlle se urinar; mictório.
\section{Urinoso}
\begin{itemize}
\item {Grp. gram.:adj.}
\end{itemize}
O mesmo que \textunderscore urinário\textunderscore .
\section{Uriunduba}
\begin{itemize}
\item {Grp. gram.:f.}
\end{itemize}
O mesmo que \textunderscore aroeira\textunderscore .
\section{Urivi}
\begin{itemize}
\item {Grp. gram.:m.}
\end{itemize}
Armadilha, com que os Ganguelas apanham lebres e pequenos antílopes.
\section{Urjal}
\begin{itemize}
\item {Grp. gram.:f.}
\end{itemize}
Variedade de figueira algarvia.
\section{Urjamanta}
\begin{itemize}
\item {Grp. gram.:f.}
\end{itemize}
O mesmo que \textunderscore ujamanta\textunderscore .
\section{Urmeiro}
\begin{itemize}
\item {Grp. gram.:m.}
\end{itemize}
\begin{itemize}
\item {Utilização:Ant.}
\end{itemize}
O mesmo que \textunderscore ulmeiro\textunderscore .
\section{Urna}
\begin{itemize}
\item {Grp. gram.:f.}
\end{itemize}
\begin{itemize}
\item {Utilização:Bot.}
\end{itemize}
\begin{itemize}
\item {Utilização:Chul.}
\end{itemize}
\begin{itemize}
\item {Proveniência:(Lat. \textunderscore urna\textunderscore )}
\end{itemize}
Vaso para água, entre os antigos.
Vaso, em que se guardava a cinza dos mortos.
Vaso ou objecto análogo, em que se recolhem os votos, num acto eleitoral ou os números de uma lotaria, rifa, etc.
Vaso em fórma de urna antiga.
Espécie de cápsula, coberta por um opérculo.
Chapéu alto.
\section{Urnário}
\begin{itemize}
\item {Grp. gram.:adj.}
\end{itemize}
\begin{itemize}
\item {Grp. gram.:M.}
\end{itemize}
\begin{itemize}
\item {Utilização:Bot.}
\end{itemize}
\begin{itemize}
\item {Proveniência:(Lat. \textunderscore urnarium\textunderscore )}
\end{itemize}
Relativo ou semelhante a urna.
Receptáculo da semente de alguns fungos e musgos.
Mesa, sôbre que os Romanos assentavam as vasilhas da água.
\section{Urnígero}
\begin{itemize}
\item {Grp. gram.:adj.}
\end{itemize}
\begin{itemize}
\item {Utilização:Bot.}
\end{itemize}
\begin{itemize}
\item {Proveniência:(Do lat. \textunderscore urna\textunderscore  + \textunderscore gerere\textunderscore )}
\end{itemize}
Que tem urna, ou cápsula em fórma de urna.
\section{Urningo}
\begin{itemize}
\item {Grp. gram.:m.}
\end{itemize}
\begin{itemize}
\item {Utilização:Med.}
\end{itemize}
Homem, que tem a aberração sexual de procurar os indivíduos do seu sexo para a satisfação de prazezes sensuaes.
Fanchono.
(Desconheço a razão do termo, aventado pela Medicina italiana)
\section{Uro}
\begin{itemize}
\item {Grp. gram.:m.}
\end{itemize}
\begin{itemize}
\item {Proveniência:(Lat. \textunderscore urus\textunderscore )}
\end{itemize}
Espécie de boi selvagem.
\section{Uró}
\begin{itemize}
\item {Grp. gram.:m.}
\end{itemize}
Árvore da Índia Portuguesa.
\section{Uro...}
\begin{itemize}
\item {Grp. gram.:pref.}
\end{itemize}
\begin{itemize}
\item {Proveniência:(Do gr. \textunderscore ouron\textunderscore )}
\end{itemize}
(designativo de \textunderscore urina\textunderscore )
\section{Uro...}
\begin{itemize}
\item {Grp. gram.:pref.}
\end{itemize}
\begin{itemize}
\item {Proveniência:(Do gr. \textunderscore oura\textunderscore )}
\end{itemize}
(designativo de \textunderscore cauda\textunderscore )
\section{Urobenzoato}
\begin{itemize}
\item {Grp. gram.:m.}
\end{itemize}
\begin{itemize}
\item {Utilização:Chím.}
\end{itemize}
Sal, produzido pela combinação do ácido urobenzóico com uma base.
\section{Urobenzóico}
\begin{itemize}
\item {Grp. gram.:adj.}
\end{itemize}
\begin{itemize}
\item {Utilização:Chím.}
\end{itemize}
Diz-se de um ácido, análogo ao úrico, e que existe na urina dos animaes herbívoros em geral.
\section{Urobilina}
\begin{itemize}
\item {Grp. gram.:f.}
\end{itemize}
O mesmo que \textunderscore urochromo\textunderscore .
\section{Urobrânchio}
\begin{itemize}
\item {fónica:qui}
\end{itemize}
\begin{itemize}
\item {Grp. gram.:adj.}
\end{itemize}
\begin{itemize}
\item {Proveniência:(De \textunderscore uro\textunderscore ^2... + \textunderscore brânchias\textunderscore )}
\end{itemize}
Que tem as brânchias perto da cauda.
\section{Urobrânquio}
\begin{itemize}
\item {Grp. gram.:adj.}
\end{itemize}
\begin{itemize}
\item {Proveniência:(De \textunderscore uro\textunderscore ^2... + \textunderscore brânquias\textunderscore )}
\end{itemize}
Que tem as brânquias perto da cauda.
\section{Urocele}
\begin{itemize}
\item {Grp. gram.:m.}
\end{itemize}
\begin{itemize}
\item {Utilização:Med.}
\end{itemize}
\begin{itemize}
\item {Proveniência:(Do gr. \textunderscore ouron\textunderscore  + \textunderscore kele\textunderscore )}
\end{itemize}
Infiltração da urina no escroto.
\section{Urochlena}
\begin{itemize}
\item {Grp. gram.:f.}
\end{itemize}
Gênero de plantas gramíneas.
\section{Uróchloa}
\begin{itemize}
\item {Grp. gram.:f.}
\end{itemize}
\begin{itemize}
\item {Proveniência:(Do gr. \textunderscore oura\textunderscore  + \textunderscore khlon\textunderscore )}
\end{itemize}
Gênero de plantas gramíneas.
\section{Urochroma}
\begin{itemize}
\item {Grp. gram.:m.}
\end{itemize}
O mesmo ou melhór que \textunderscore urochromo\textunderscore .
\section{Urochromo}
\begin{itemize}
\item {Grp. gram.:m.}
\end{itemize}
\begin{itemize}
\item {Proveniência:(Do gr. \textunderscore ouron\textunderscore  + \textunderscore khroma\textunderscore )}
\end{itemize}
Substância còrante da urina.
\section{Urocianina}
\begin{itemize}
\item {Grp. gram.:f.}
\end{itemize}
\begin{itemize}
\item {Utilização:Chím.}
\end{itemize}
\begin{itemize}
\item {Proveniência:(Do gr. \textunderscore ouron\textunderscore  + \textunderscore kuon\textunderscore )}
\end{itemize}
Princípio imediato, mas acidental, da urina.
\section{Urocistite}
\begin{itemize}
\item {Grp. gram.:f.}
\end{itemize}
\begin{itemize}
\item {Utilização:Med.}
\end{itemize}
\begin{itemize}
\item {Proveniência:(Do gr. \textunderscore ouron\textunderscore  + \textunderscore kustis\textunderscore )}
\end{itemize}
Inflamação da bexiga urinária.
\section{Uroclena}
\begin{itemize}
\item {Grp. gram.:f.}
\end{itemize}
Gênero de plantas gramíneas.
\section{Urócloa}
\begin{itemize}
\item {Grp. gram.:f.}
\end{itemize}
\begin{itemize}
\item {Proveniência:(Do gr. \textunderscore oura\textunderscore  + \textunderscore khlon\textunderscore )}
\end{itemize}
Gênero de plantas gramíneas.
\section{Urócopo}
\begin{itemize}
\item {Grp. gram.:m.}
\end{itemize}
Gênero de insectos coleópteros clavicórneos.
\section{Urocrisia}
\begin{itemize}
\item {Grp. gram.:f.}
\end{itemize}
\begin{itemize}
\item {Utilização:Med.}
\end{itemize}
\begin{itemize}
\item {Proveniência:(Do gr. \textunderscore ouron\textunderscore  + \textunderscore crisis\textunderscore )}
\end{itemize}
Diagnóstico, feito pelo exame das urinas.
\section{Urocrítico}
\begin{itemize}
\item {Grp. gram.:adj.}
\end{itemize}
Relativo á urocrisia.
\section{Urocroma}
\begin{itemize}
\item {Grp. gram.:m.}
\end{itemize}
O mesmo ou melhór que \textunderscore urocromo\textunderscore .
\section{Urocromo}
\begin{itemize}
\item {Grp. gram.:m.}
\end{itemize}
\begin{itemize}
\item {Proveniência:(Do gr. \textunderscore ouron\textunderscore  + \textunderscore khroma\textunderscore )}
\end{itemize}
Substância còrante da urina.
\section{Urocyanina}
\begin{itemize}
\item {Grp. gram.:f.}
\end{itemize}
\begin{itemize}
\item {Utilização:Chím.}
\end{itemize}
\begin{itemize}
\item {Proveniência:(Do gr. \textunderscore ouron\textunderscore  + \textunderscore kuon\textunderscore )}
\end{itemize}
Princípio immediato, mas accidental, da urina.
\section{Urocystite}
\begin{itemize}
\item {Grp. gram.:f.}
\end{itemize}
\begin{itemize}
\item {Utilização:Med.}
\end{itemize}
\begin{itemize}
\item {Proveniência:(Do gr. \textunderscore ouron\textunderscore  + \textunderscore kustis\textunderscore )}
\end{itemize}
Inflamação da bexiga urinária.
\section{Uródeos}
\begin{itemize}
\item {Grp. gram.:m. pl.}
\end{itemize}
\begin{itemize}
\item {Utilização:Zool.}
\end{itemize}
\begin{itemize}
\item {Proveniência:(Do gr. \textunderscore oura\textunderscore  + \textunderscore eidos\textunderscore )}
\end{itemize}
Família de animaes microscópicos, cujo corpo termina por appêndice em fórma de cauda.
\section{Urodelo}
\begin{itemize}
\item {fónica:dê}
\end{itemize}
\begin{itemize}
\item {Grp. gram.:m.}
\end{itemize}
\begin{itemize}
\item {Utilização:Zool.}
\end{itemize}
\begin{itemize}
\item {Grp. gram.:M.}
\end{itemize}
\begin{itemize}
\item {Grp. gram.:Pl.}
\end{itemize}
\begin{itemize}
\item {Proveniência:(Do gr. \textunderscore oura\textunderscore  + \textunderscore delos\textunderscore )}
\end{itemize}
Que tem cauda muito visível.
Batrácio, que perde as brânchias e que conserva a cauda em-quanto existe.
Ordem de batrácios, a que pertence a salamandra.
\section{Urodiálise}
\begin{itemize}
\item {Grp. gram.:f.}
\end{itemize}
\begin{itemize}
\item {Utilização:Med.}
\end{itemize}
\begin{itemize}
\item {Proveniência:(Do gr. \textunderscore ouron\textunderscore  + \textunderscore dialusis\textunderscore )}
\end{itemize}
Supressão da urina.
\section{Urodiályse}
\begin{itemize}
\item {Grp. gram.:f.}
\end{itemize}
\begin{itemize}
\item {Utilização:Med.}
\end{itemize}
\begin{itemize}
\item {Proveniência:(Do gr. \textunderscore ouron\textunderscore  + \textunderscore dialusis\textunderscore )}
\end{itemize}
Suppressão da urina.
\section{Urodinia}
\begin{itemize}
\item {Grp. gram.:f.}
\end{itemize}
\begin{itemize}
\item {Proveniência:(Do gr. \textunderscore ouron\textunderscore  + \textunderscore odune\textunderscore )}
\end{itemize}
Dôr, sentida quando se urina.
\section{Urodonte}
\begin{itemize}
\item {Grp. gram.:m.}
\end{itemize}
\begin{itemize}
\item {Proveniência:(Do gr. \textunderscore oura\textunderscore  + \textunderscore odous\textunderscore )}
\end{itemize}
Gênero de insectos coleópteros tetrâmeros.
\section{Urodrimia}
\begin{itemize}
\item {Grp. gram.:f.}
\end{itemize}
\begin{itemize}
\item {Utilização:Med.}
\end{itemize}
Acrimónia da urina.
\section{Urodynia}
\begin{itemize}
\item {Grp. gram.:f.}
\end{itemize}
\begin{itemize}
\item {Proveniência:(Do gr. \textunderscore ouron\textunderscore  + \textunderscore odune\textunderscore )}
\end{itemize}
Dôr, sentida quando se urina.
\section{Uroeritrina}
\begin{itemize}
\item {Grp. gram.:f.}
\end{itemize}
\begin{itemize}
\item {Proveniência:(Do gr. \textunderscore ouron\textunderscore  + \textunderscore eruthros\textunderscore )}
\end{itemize}
Matéria còrante, vermelha, da urina.
\section{Uroerythrina}
\begin{itemize}
\item {Grp. gram.:f.}
\end{itemize}
\begin{itemize}
\item {Proveniência:(Do gr. \textunderscore ouron\textunderscore  + \textunderscore eruthros\textunderscore )}
\end{itemize}
Matéria còrante, vermelha, da urina.
\section{Urofilo}
\begin{itemize}
\item {Grp. gram.:m.}
\end{itemize}
\begin{itemize}
\item {Proveniência:(Do gr. \textunderscore oura\textunderscore  + \textunderscore phullon\textunderscore )}
\end{itemize}
Gênero de plantas rubiáceas.
\section{Urófora}
\begin{itemize}
\item {Grp. gram.:f.}
\end{itemize}
\begin{itemize}
\item {Proveniência:(Do gr. \textunderscore oura\textunderscore  + \textunderscore phoros\textunderscore )}
\end{itemize}
Gênero de insectos dípteros.
Gênero de insectos hemípteros.
\section{Urohyal}
\begin{itemize}
\item {Grp. gram.:m.}
\end{itemize}
\begin{itemize}
\item {Utilização:Anat.}
\end{itemize}
\begin{itemize}
\item {Proveniência:(De \textunderscore uro\textunderscore ^2... + \textunderscore hyal\textunderscore )}
\end{itemize}
Peça, situada detrás do entohyal.
\section{Uroial}
\begin{itemize}
\item {fónica:o-i}
\end{itemize}
\begin{itemize}
\item {Grp. gram.:m.}
\end{itemize}
\begin{itemize}
\item {Utilização:Anat.}
\end{itemize}
\begin{itemize}
\item {Proveniência:(De \textunderscore uro\textunderscore ^2... + \textunderscore hyal\textunderscore )}
\end{itemize}
Peça, situada detrás do entoial.
\section{Urolíthico}
\begin{itemize}
\item {Grp. gram.:adj.}
\end{itemize}
\begin{itemize}
\item {Utilização:Chím.}
\end{itemize}
\begin{itemize}
\item {Proveniência:(Do gr. \textunderscore ouron\textunderscore  + \textunderscore lithos\textunderscore )}
\end{itemize}
Epítheto, que se dá, ás vezes, ao ácido úrico, porque se encontra em muitos cálculos urinários.
\section{Urólitho}
\begin{itemize}
\item {Grp. gram.:m.}
\end{itemize}
\begin{itemize}
\item {Utilização:Med.}
\end{itemize}
\begin{itemize}
\item {Proveniência:(Do gr. \textunderscore ouron\textunderscore  + \textunderscore lithos\textunderscore )}
\end{itemize}
Cálculo urinário.
\section{Urolítico}
\begin{itemize}
\item {Grp. gram.:adj.}
\end{itemize}
\begin{itemize}
\item {Utilização:Chím.}
\end{itemize}
\begin{itemize}
\item {Proveniência:(Do gr. \textunderscore ouron\textunderscore  + \textunderscore lithos\textunderscore )}
\end{itemize}
Epíteto, que se dá, ás vezes, ao ácido úrico, porque se encontra em muitos cálculos urinários.
\section{Urólito}
\begin{itemize}
\item {Grp. gram.:m.}
\end{itemize}
\begin{itemize}
\item {Utilização:Med.}
\end{itemize}
\begin{itemize}
\item {Proveniência:(Do gr. \textunderscore ouron\textunderscore  + \textunderscore lithos\textunderscore )}
\end{itemize}
Cálculo urinário.
\section{Urologia}
\begin{itemize}
\item {Grp. gram.:f.}
\end{itemize}
\begin{itemize}
\item {Proveniência:(Do gr. \textunderscore ouron\textunderscore  + \textunderscore logos\textunderscore )}
\end{itemize}
Tratado da urina, da sua constituição e das suas alterações mórbidas.
\section{Urológico}
\begin{itemize}
\item {Grp. gram.:adj.}
\end{itemize}
Relativo á urologia.
\section{Uromancia}
\begin{itemize}
\item {Grp. gram.:f.}
\end{itemize}
\begin{itemize}
\item {Proveniência:(Do gr. \textunderscore ouron\textunderscore  + \textunderscore manteia\textunderscore )}
\end{itemize}
Arte de conhecer as enfermidades pelo aspecto da urina.
\section{Uromelanina}
\begin{itemize}
\item {Grp. gram.:f.}
\end{itemize}
\begin{itemize}
\item {Proveniência:(Do gr. \textunderscore ouron\textunderscore  + \textunderscore melas\textunderscore )}
\end{itemize}
O mesmo que \textunderscore índigo\textunderscore .
\section{Uromeila}
\begin{itemize}
\item {Grp. gram.:f.}
\end{itemize}
Monstruosidade de urómelo.
\section{Uromeliano}
\begin{itemize}
\item {Grp. gram.:adj.}
\end{itemize}
Relativo ao urómelo ou á uromelía.
\section{Uromélico}
\begin{itemize}
\item {Grp. gram.:adj.}
\end{itemize}
Que apresenta os caracteres do urómelo.
\section{Urómelo}
\begin{itemize}
\item {Grp. gram.:m.}
\end{itemize}
\begin{itemize}
\item {Proveniência:(Do gr. \textunderscore oura\textunderscore  + \textunderscore melos\textunderscore )}
\end{itemize}
Monstro, cujos membros estão reunidos num só, terminado por um pé.
\section{Urómetro}
\begin{itemize}
\item {Grp. gram.:m.}
\end{itemize}
\begin{itemize}
\item {Utilização:Med.}
\end{itemize}
\begin{itemize}
\item {Proveniência:(Do gr. \textunderscore ouron\textunderscore  + \textunderscore metron\textunderscore )}
\end{itemize}
Instrumento, que deixa conhecer o pêso específico da urina.
\section{Uropéstide}
\begin{itemize}
\item {Grp. gram.:m.}
\end{itemize}
Gênero de reptís ophídios.
\section{Uropétalo}
\begin{itemize}
\item {Grp. gram.:m.}
\end{itemize}
\begin{itemize}
\item {Proveniência:(Do gr. \textunderscore oura\textunderscore  + \textunderscore petalon\textunderscore )}
\end{itemize}
Gênero de plantas liliáceas.
\section{Uróphora}
\begin{itemize}
\item {Grp. gram.:f.}
\end{itemize}
\begin{itemize}
\item {Proveniência:(Do gr. \textunderscore oura\textunderscore  + \textunderscore phoros\textunderscore )}
\end{itemize}
Gênero de insectos dípteros.
Gênero de insectos hemípteros.
\section{Urophyllo}
\begin{itemize}
\item {Grp. gram.:m.}
\end{itemize}
\begin{itemize}
\item {Proveniência:(Do gr. \textunderscore oura\textunderscore  + \textunderscore phullon\textunderscore )}
\end{itemize}
Gênero de plantas rubiáceas.
\section{Uropigial}
\begin{itemize}
\item {Grp. gram.:adj.}
\end{itemize}
Relativo ao uropígio.
\section{Uropígio}
\begin{itemize}
\item {Grp. gram.:m.}
\end{itemize}
\begin{itemize}
\item {Proveniência:(Do gr. \textunderscore oura\textunderscore  + \textunderscore puge\textunderscore )}
\end{itemize}
Saliência triangular sôbre as vértebras inferiores das aves, e da qual nascem as pennas da cauda.
\section{Uropoése}
\begin{itemize}
\item {Grp. gram.:f.}
\end{itemize}
\begin{itemize}
\item {Proveniência:(Do gr. \textunderscore ouron\textunderscore  + \textunderscore poiesis\textunderscore )}
\end{itemize}
Producção da urina.
\section{Uropoético}
\begin{itemize}
\item {Grp. gram.:adj.}
\end{itemize}
Relativo á uropoése.
Que favorece ou promove a uropoése.
\section{Uroplania}
\begin{itemize}
\item {Grp. gram.:f.}
\end{itemize}
\begin{itemize}
\item {Utilização:Med.}
\end{itemize}
\begin{itemize}
\item {Proveniência:(Do gr. \textunderscore ouron\textunderscore  + \textunderscore plane\textunderscore )}
\end{itemize}
Apparecimento da urina em qualquer parte do corpo, onde ella é anómala.
\section{Uropo}
\begin{itemize}
\item {Grp. gram.:m.}
\end{itemize}
Gênero de insectos lepidópteros nocturnos.
(Cp. \textunderscore urópodes\textunderscore )
\section{Urópoda}
\begin{itemize}
\item {Grp. gram.:f.}
\end{itemize}
Gênero de arachnídeos.
(Cp. \textunderscore urópodes\textunderscore )
\section{Urópodes}
\begin{itemize}
\item {Grp. gram.:m. pl.}
\end{itemize}
\begin{itemize}
\item {Proveniência:(Do gr. \textunderscore oura\textunderscore  + \textunderscore pous\textunderscore , \textunderscore podos\textunderscore )}
\end{itemize}
Família de aves palmípedes, cujos pés estão tanto para trás, que faz parecer que essas aves andam sôbre a cauda.
\section{Urópteros}
\begin{itemize}
\item {Grp. gram.:m. pl.}
\end{itemize}
\begin{itemize}
\item {Utilização:Zool.}
\end{itemize}
\begin{itemize}
\item {Proveniência:(Do gr. \textunderscore oura\textunderscore  + \textunderscore pteron\textunderscore )}
\end{itemize}
Família de crustáceos amphípodes.
\section{Uropygial}
\begin{itemize}
\item {Grp. gram.:adj.}
\end{itemize}
Relativo ao uropýgio.
\section{Uropýgio}
\begin{itemize}
\item {Grp. gram.:m.}
\end{itemize}
\begin{itemize}
\item {Proveniência:(Do gr. \textunderscore oura\textunderscore  + \textunderscore puge\textunderscore )}
\end{itemize}
Saliência triangular sôbre as vértebras inferiores das aves, e da qual nascem as pennas da cauda.
\section{Uroque}
\begin{itemize}
\item {Grp. gram.:m.}
\end{itemize}
O mesmo ou melhor que \textunderscore auroque\textunderscore .
(Cp. \textunderscore uro\textunderscore )
\section{Urorragia}
\begin{itemize}
\item {Grp. gram.:f.}
\end{itemize}
Designação imprópria da urorrheia.
\section{Urorreia}
\begin{itemize}
\item {Grp. gram.:f.}
\end{itemize}
\begin{itemize}
\item {Utilização:Med.}
\end{itemize}
\begin{itemize}
\item {Proveniência:(Do gr. \textunderscore ouron\textunderscore  + \textunderscore rhein\textunderscore )}
\end{itemize}
Fluxo abundante de urina; diabete.
\section{Urorrhagia}
\begin{itemize}
\item {Grp. gram.:f.}
\end{itemize}
Designação imprópria da urorrheia.
\section{Urorrheia}
\begin{itemize}
\item {Grp. gram.:f.}
\end{itemize}
\begin{itemize}
\item {Utilização:Med.}
\end{itemize}
\begin{itemize}
\item {Proveniência:(Do gr. \textunderscore ouron\textunderscore  + \textunderscore rhein\textunderscore )}
\end{itemize}
Fluxo abundante de urina; diabete.
\section{Uroscopia}
\begin{itemize}
\item {Grp. gram.:f.}
\end{itemize}
\begin{itemize}
\item {Proveniência:(Do gr. \textunderscore ouron\textunderscore  + \textunderscore skopein\textunderscore )}
\end{itemize}
Exame das urinas.
\section{Uroscópico}
\begin{itemize}
\item {Grp. gram.:adj.}
\end{itemize}
Relativo á uroscopia.
\section{Uróscopo}
\begin{itemize}
\item {Grp. gram.:adj.}
\end{itemize}
\begin{itemize}
\item {Proveniência:(Do gr. \textunderscore ouron\textunderscore  + \textunderscore skopein\textunderscore )}
\end{itemize}
Diz-se do médico, que observa as urinas de um enfermo, para que, combinada essa observação com outros phenómenos, possa, estabelecer o diagnóstico.
\section{Urose}
\begin{itemize}
\item {Grp. gram.:f.}
\end{itemize}
\begin{itemize}
\item {Proveniência:(Do gr. \textunderscore ouron\textunderscore )}
\end{itemize}
Qualquer doença das vias urinárias.
\section{Urosemiologia}
\begin{itemize}
\item {fónica:se}
\end{itemize}
\begin{itemize}
\item {Grp. gram.:f.}
\end{itemize}
\begin{itemize}
\item {Proveniência:(De \textunderscore uro...\textunderscore  + \textunderscore semeiologia\textunderscore )}
\end{itemize}
Tratado dos symptomas das doenças urinárias. Cf. Verg. Machado, \textunderscore Urosemeiologia\textunderscore .
\section{Urosemiológico}
\begin{itemize}
\item {fónica:se}
\end{itemize}
\begin{itemize}
\item {Grp. gram.:adj.}
\end{itemize}
Relativo á urosemeiologia.
\section{Uroseptina}
\begin{itemize}
\item {fónica:sé}
\end{itemize}
\begin{itemize}
\item {Grp. gram.:f.}
\end{itemize}
\begin{itemize}
\item {Proveniência:(Do gr. \textunderscore ouron\textunderscore  + \textunderscore septikos\textunderscore )}
\end{itemize}
Preparação antiséptica, em que entra urotropina, piperasina, benzoato de soda e benzoato de lithina.
\section{Urospermo}
\begin{itemize}
\item {Grp. gram.:m.}
\end{itemize}
\begin{itemize}
\item {Proveniência:(Do gr. \textunderscore oura\textunderscore  + \textunderscore sperma\textunderscore )}
\end{itemize}
Gênero de plantas synanthéreas.
\section{Urossemiologia}
\begin{itemize}
\item {Grp. gram.:f.}
\end{itemize}
\begin{itemize}
\item {Proveniência:(De \textunderscore uro...\textunderscore  + \textunderscore semeiologia\textunderscore )}
\end{itemize}
Tratado dos symptomas das doenças urinárias. Cf. Verg. Machado, \textunderscore Urosemeiologia\textunderscore .
\section{Urossemiológico}
\begin{itemize}
\item {Grp. gram.:adj.}
\end{itemize}
Relativo á urossemeiologia.
\section{Urosseptina}
\begin{itemize}
\item {Grp. gram.:f.}
\end{itemize}
\begin{itemize}
\item {Proveniência:(Do gr. \textunderscore ouron\textunderscore  + \textunderscore septikos\textunderscore )}
\end{itemize}
Preparação antiséptica, em que entra urotropina, piperasina, benzoato de soda e benzoato de lithina.
\section{Urotropina}
\begin{itemize}
\item {Grp. gram.:f.}
\end{itemize}
Medicamento diurético, para a diátese úrica.
\section{Uroxanthina}
\begin{itemize}
\item {fónica:csan}
\end{itemize}
\begin{itemize}
\item {Grp. gram.:f.}
\end{itemize}
\begin{itemize}
\item {Utilização:Chím.}
\end{itemize}
\begin{itemize}
\item {Proveniência:(Do gr. \textunderscore ouron\textunderscore  + \textunderscore xanthos\textunderscore )}
\end{itemize}
Matéria còrante, amarela, da urina.
\section{Uroxantina}
\begin{itemize}
\item {fónica:csan}
\end{itemize}
\begin{itemize}
\item {Grp. gram.:f.}
\end{itemize}
\begin{itemize}
\item {Utilização:Chím.}
\end{itemize}
\begin{itemize}
\item {Proveniência:(Do gr. \textunderscore ouron\textunderscore  + \textunderscore xanthos\textunderscore )}
\end{itemize}
Matéria còrante, amarela, da urina.
\section{Urraca}
\begin{itemize}
\item {Grp. gram.:f.}
\end{itemize}
\begin{itemize}
\item {Utilização:Náut.}
\end{itemize}
\begin{itemize}
\item {Utilização:Pop.}
\end{itemize}
\begin{itemize}
\item {Proveniência:(De \textunderscore Urraca\textunderscore , n. p.)}
\end{itemize}
Apparelho das velas do estai, entre os mastros.
O mesmo que \textunderscore pêga\textunderscore ^1.
\section{Urrar}
\begin{itemize}
\item {Grp. gram.:v. i.}
\end{itemize}
\begin{itemize}
\item {Grp. gram.:V. t.}
\end{itemize}
\begin{itemize}
\item {Proveniência:(It. \textunderscore urlare\textunderscore )}
\end{itemize}
Dar urros; rugir.
Exprimir, á maneira de urro.
\section{Urreiro}
\begin{itemize}
\item {Grp. gram.:m.}
\end{itemize}
\begin{itemize}
\item {Utilização:Prov.}
\end{itemize}
\begin{itemize}
\item {Utilização:dur.}
\end{itemize}
(V.orreiro)
\section{Urrhodina}
\begin{itemize}
\item {Grp. gram.:f.}
\end{itemize}
O mesmo que \textunderscore urochromo\textunderscore .
\section{Urro}
\begin{itemize}
\item {Grp. gram.:m.}
\end{itemize}
\begin{itemize}
\item {Utilização:Fig.}
\end{itemize}
\begin{itemize}
\item {Proveniência:(De \textunderscore urrar\textunderscore )}
\end{itemize}
Rugido ou voz forte de algumas feras.
Berro.
\section{Urrodina}
\begin{itemize}
\item {Grp. gram.:f.}
\end{itemize}
O mesmo que \textunderscore urocromo\textunderscore .
\section{Urrosacina}
\begin{itemize}
\item {Grp. gram.:f.}
\end{itemize}
\begin{itemize}
\item {Utilização:Chím.}
\end{itemize}
\begin{itemize}
\item {Proveniência:(Do rad. de \textunderscore urina\textunderscore  e do lat. \textunderscore rosaceus\textunderscore )}
\end{itemize}
Substância orgânica, que só se dissolve em grande porção de água, e cuja côr varía do rosado para o encarnado tirante a escuro.
\section{Ursa}
\begin{itemize}
\item {Grp. gram.:f.}
\end{itemize}
\begin{itemize}
\item {Proveniência:(Lat. \textunderscore ursa\textunderscore )}
\end{itemize}
Fêmea do urso.
Nome de duas constellações boreaes, que se distinguem por Ursa-maior e Ursa-menor.
\section{Urselo}
\begin{itemize}
\item {fónica:sê}
\end{itemize}
\begin{itemize}
\item {Grp. gram.:m.}
\end{itemize}
\begin{itemize}
\item {Utilização:Prov.}
\end{itemize}
\begin{itemize}
\item {Utilização:trasm.}
\end{itemize}
Urso pequeno.
\section{Ursídeo}
\begin{itemize}
\item {Grp. gram.:adj.}
\end{itemize}
\begin{itemize}
\item {Grp. gram.:M. pl.}
\end{itemize}
\begin{itemize}
\item {Proveniência:(De \textunderscore urso\textunderscore  + gr. \textunderscore eidos\textunderscore )}
\end{itemize}
Relativo ou semelhante ao urso.
Família de mammíferos, a que pertence o urso.
\section{Ursino}
\begin{itemize}
\item {Grp. gram.:adj.}
\end{itemize}
\begin{itemize}
\item {Proveniência:(Lat. \textunderscore ursinus\textunderscore )}
\end{itemize}
Relativo ao urso.
\section{Urso}
\begin{itemize}
\item {Grp. gram.:m.}
\end{itemize}
\begin{itemize}
\item {Utilização:Fig.}
\end{itemize}
\begin{itemize}
\item {Utilização:Escol.}
\end{itemize}
\begin{itemize}
\item {Utilização:Fam.}
\end{itemize}
\begin{itemize}
\item {Utilização:Bras. da Baía}
\end{itemize}
\begin{itemize}
\item {Proveniência:(Lat. \textunderscore ursus\textunderscore )}
\end{itemize}
Gênero de animaes carnívoros.
Homem pouco sociável, intratável.
Homem feio.
Estudante premiado ou distinto.
Indivíduo que é objecto de zombaria.
Mandatário de assassínios.
\section{Ursulina}
\begin{itemize}
\item {Grp. gram.:f.}
\end{itemize}
\begin{itemize}
\item {Grp. gram.:Pl.}
\end{itemize}
Nome de freiras, que se encarregavam especialmente da instrucção de meninas e pertenciam á Ordem de Santo Agostinho.
Convento de ursulinas.
(Do nome de Santa \textunderscore Úrsula\textunderscore )
\section{Urticação}
\begin{itemize}
\item {Grp. gram.:f.}
\end{itemize}
\begin{itemize}
\item {Proveniência:(Do lat. \textunderscore urtica\textunderscore )}
\end{itemize}
Acto de flagellar a pelle, para a excitar.
\section{Urticáceas}
\begin{itemize}
\item {Grp. gram.:f. pl.}
\end{itemize}
Família de plantas, que tem por typo a urtiga.
\section{Urticante}
\begin{itemize}
\item {Grp. gram.:adj.}
\end{itemize}
\begin{itemize}
\item {Proveniência:(De \textunderscore urticar\textunderscore )}
\end{itemize}
Que produz sensação análoga á das urtigas sôbre a pelle.
\section{Urticar}
\begin{itemize}
\item {Grp. gram.:v. t.}
\end{itemize}
\begin{itemize}
\item {Proveniência:(Lat. \textunderscore urticare\textunderscore )}
\end{itemize}
Produzir sensação análoga á das urtigas sôbre (a pelle). Cf. Fil. Simões, \textunderscore Cartas da Beir.\textunderscore , 206.
\section{Urticária}
\begin{itemize}
\item {Grp. gram.:f.}
\end{itemize}
\begin{itemize}
\item {Utilização:Med.}
\end{itemize}
\begin{itemize}
\item {Proveniência:(Lat. \textunderscore urticaria\textunderscore )}
\end{itemize}
Inflammação, caracterizada por manchas, raramente persistentes, e que produz um prurido semelhante ao que a urtiga produz sôbre a pelle.
\section{Urtíceas}
\begin{itemize}
\item {Grp. gram.:f. pl.}
\end{itemize}
(V.urticáceas)
\section{Urticifoliado}
\begin{itemize}
\item {Grp. gram.:adj.}
\end{itemize}
\begin{itemize}
\item {Utilização:Bot.}
\end{itemize}
\begin{itemize}
\item {Proveniência:(Do lat. \textunderscore urtica\textunderscore  + \textunderscore folium\textunderscore )}
\end{itemize}
Que tem fôlhas parecidas com as da urtiga.
\section{Urticíneas}
\begin{itemize}
\item {Grp. gram.:f. pl.}
\end{itemize}
\begin{itemize}
\item {Proveniência:(Do lat. \textunderscore urtica\textunderscore )}
\end{itemize}
Ordem de plantas, que contém as urticáceas e outras famílias.
\section{Urtiga}
\begin{itemize}
\item {Grp. gram.:f.}
\end{itemize}
\begin{itemize}
\item {Proveniência:(Do lat. \textunderscore urtica\textunderscore )}
\end{itemize}
Gênero de plantas bravas, em que se distingue a urtiga commum, cuja haste e cujas fôlhas produzem na pelle um ardor especial.
Peixe dos Açores.
\section{Urtiga-branca}
\begin{itemize}
\item {Grp. gram.:f.}
\end{itemize}
O mesmo que \textunderscore urtiga-morta\textunderscore .
\section{Urtigação}
\begin{itemize}
\item {Grp. gram.:f.}
\end{itemize}
Acto de urtigar.
\section{Urtiga-da-china}
\begin{itemize}
\item {Grp. gram.:f.}
\end{itemize}
Nome de várias plantas urticáceas, de filamentos têxteis.
\section{Urtiga-de-cipó}
\begin{itemize}
\item {Grp. gram.:f.}
\end{itemize}
Planta euphorbiácea do Brasil.
\section{Urtiga-de-espinho}
\begin{itemize}
\item {Grp. gram.:f.}
\end{itemize}
Planta escrofularínea do Brasil.
\section{Urtiga-do-mar}
\begin{itemize}
\item {Grp. gram.:f.}
\end{itemize}
Nome da alforreca e de outros acalephos, bem como de alguns pólypos.
\section{Urtiga-do-papel}
\begin{itemize}
\item {Grp. gram.:f.}
\end{itemize}
Planta têxtil urticácea.
\section{Urtigal}
\begin{itemize}
\item {Grp. gram.:m.}
\end{itemize}
Lugar, onde crescem urtigas.
\section{Urtiga-morta}
\begin{itemize}
\item {Grp. gram.:f.}
\end{itemize}
Planta labiada, também conhecida por \textunderscore lâmio-branco\textunderscore .
\section{Urtigão}
\begin{itemize}
\item {Grp. gram.:m.}
\end{itemize}
Espécie de urtiga, (\textunderscore urtica dioica\textunderscore , Lin.).
\section{Urtigar}
\begin{itemize}
\item {Grp. gram.:v. t.}
\end{itemize}
\begin{itemize}
\item {Proveniência:(Do lat. \textunderscore urticare\textunderscore )}
\end{itemize}
Picar ou flagellar com urtigas.
\section{Urtigueira}
\begin{itemize}
\item {Grp. gram.:f.}
\end{itemize}
O mesmo que \textunderscore urtigal\textunderscore .
Urtiga grande.
\section{Uru}
\begin{itemize}
\item {Grp. gram.:m.}
\end{itemize}
\begin{itemize}
\item {Utilização:Bras}
\end{itemize}
\begin{itemize}
\item {Proveniência:(T. tupi)}
\end{itemize}
Ave gallinácea do Brasil.
\section{Uru}
\begin{itemize}
\item {Grp. gram.:m.}
\end{itemize}
Cesto de palha, em que os Indígenas do Brasil guardam cachimbos, tabaco, anzóis, e outros objectos.
\section{Urubá}
\begin{itemize}
\item {Grp. gram.:f.}
\end{itemize}
Planta amarantácea do Brasil.
\section{Urubu}
\begin{itemize}
\item {Grp. gram.:m.}
\end{itemize}
\begin{itemize}
\item {Utilização:Bras}
\end{itemize}
\begin{itemize}
\item {Utilização:Fig.}
\end{itemize}
\begin{itemize}
\item {Utilização:Bras}
\end{itemize}
\begin{itemize}
\item {Utilização:Bras}
\end{itemize}
\begin{itemize}
\item {Utilização:Bras}
\end{itemize}
Pequeno abutre da América.
Nome de outras aves de rapina.
Grande usurário; financeiro, que enriquece illicitamente.
Espécie de mandioca, de raíz curta e grossa.
Árvore silvestre, que dá tinta roxa.
Serviçal, que acompanha os enterros, de tocha na mão.
\section{Urubu-cáa}
\begin{itemize}
\item {Grp. gram.:m.}
\end{itemize}
Planta aristolóchia do Brasil.
\section{Urubu-rei}
\begin{itemize}
\item {Grp. gram.:m.}
\end{itemize}
\begin{itemize}
\item {Utilização:Bras}
\end{itemize}
Espécie de ave grande, formosa e rara.
\section{Urubus}
\begin{itemize}
\item {Grp. gram.:m. Pl.}
\end{itemize}
Indígenas do Norte do Brasil.
\section{Urucabaca}
\begin{itemize}
\item {Grp. gram.:f.}
\end{itemize}
\begin{itemize}
\item {Utilização:Bras. riograndense}
\end{itemize}
Caiporismo; má sorte.
\section{Urucaiana}
\begin{itemize}
\item {Grp. gram.:f.}
\end{itemize}
\begin{itemize}
\item {Utilização:Bras}
\end{itemize}
Espécie de onça.
\section{Urucari}
\begin{itemize}
\item {Grp. gram.:m.}
\end{itemize}
Espécie de palmeira do Brasil.
Fruto dessa árvore.
Caroço dêsse fruto, que se queima para defumar o leite da siringueira.
\section{Urucatu}
\begin{itemize}
\item {Grp. gram.:m.}
\end{itemize}
Planta amaryllídea do Brasil.
\section{Urucongo}
\begin{itemize}
\item {Grp. gram.:m.}
\end{itemize}
\begin{itemize}
\item {Utilização:Bras}
\end{itemize}
O mesmo que \textunderscore urucungo\textunderscore .
\section{Urucu}
\begin{itemize}
\item {Grp. gram.:m.}
\end{itemize}
\begin{itemize}
\item {Utilização:Bras}
\end{itemize}
Substância tinctorial do urucueiro.
O mesmo que \textunderscore urucueiro\textunderscore .
\section{Uruçu}
\begin{itemize}
\item {Grp. gram.:m.}
\end{itemize}
\begin{itemize}
\item {Utilização:Bras}
\end{itemize}
Grande abelha avermelhada e inoffensiva.
\section{Urucuana}
\begin{itemize}
\item {Grp. gram.:f.}
\end{itemize}
Árvore euphorbiácea do Brasil.
\section{Urucubaca}
\begin{itemize}
\item {Grp. gram.:f.}
\end{itemize}
\begin{itemize}
\item {Utilização:Bras. do N}
\end{itemize}
Feitiçaria.
\section{Uruçuca}
\begin{itemize}
\item {Grp. gram.:f.}
\end{itemize}
\begin{itemize}
\item {Utilização:Bras}
\end{itemize}
Árvore silvestre.
\section{Urucueiro}
\begin{itemize}
\item {Grp. gram.:m.}
\end{itemize}
\begin{itemize}
\item {Proveniência:(De \textunderscore urucu\textunderscore )}
\end{itemize}
Arbusto brasileiro, cuja semente é revestida de uma polpa avermelhada a que se chama urucu.
\section{Uruçuí}
\begin{itemize}
\item {Grp. gram.:m.}
\end{itemize}
\begin{itemize}
\item {Utilização:Bras}
\end{itemize}
Pequenina abelha amarela.
(Dem. de \textunderscore uruçu\textunderscore )
\section{Urucungo}
\begin{itemize}
\item {Grp. gram.:m.}
\end{itemize}
\begin{itemize}
\item {Utilização:Bras}
\end{itemize}
Grosseiro instrumento músico, usado pelos Negros.
\section{Urucurana}
\begin{itemize}
\item {Grp. gram.:f.}
\end{itemize}
\begin{itemize}
\item {Utilização:Bras}
\end{itemize}
Árvore silvestre, de bôa madeira para construcções.
\section{Urucuranis}
\begin{itemize}
\item {Grp. gram.:m. Pl.}
\end{itemize}
Tríbo de Índios do Brasil, em Mato-Grosso.
\section{Urucuri}
\begin{itemize}
\item {Grp. gram.:m.}
\end{itemize}
Palmeira, o mesmo que \textunderscore urucari\textunderscore .
\section{Urucutufus}
\begin{itemize}
\item {Grp. gram.:m. Pl.}
\end{itemize}
\begin{itemize}
\item {Utilização:Bras}
\end{itemize}
Índios das margens de um affluente da Madeira.
\section{Urucuuba}
\begin{itemize}
\item {Grp. gram.:f.}
\end{itemize}
O mesmo que \textunderscore urucueiro\textunderscore .
\section{Urucuzeiro}
\begin{itemize}
\item {Grp. gram.:m.}
\end{itemize}
O mesmo que \textunderscore urucueiro\textunderscore .
\section{Uruguaiano}
\begin{itemize}
\item {Grp. gram.:adj.}
\end{itemize}
\begin{itemize}
\item {Grp. gram.:M.}
\end{itemize}
Relativo ao Uruguai.
Habitante do Uruguai.
\section{Uruguaio}
\begin{itemize}
\item {Grp. gram.:m.  e  adj.}
\end{itemize}
\begin{itemize}
\item {Utilização:Bras}
\end{itemize}
O mesmo que \textunderscore uruguaiano\textunderscore .
\section{Uruiauara}
\begin{itemize}
\item {Grp. gram.:f.}
\end{itemize}
\begin{itemize}
\item {Utilização:Bras}
\end{itemize}
Espécie de onça.
\section{Urumbamba}
\begin{itemize}
\item {Grp. gram.:f.}
\end{itemize}
\begin{itemize}
\item {Utilização:Bras}
\end{itemize}
Espécie de palmeira.
\section{Urumbeba}
\begin{itemize}
\item {Grp. gram.:f.}
\end{itemize}
O mesmo que \textunderscore cumbeba\textunderscore .
\section{Urumbebal}
\begin{itemize}
\item {Grp. gram.:m.}
\end{itemize}
Terreno, plantado de urumbebas.
\section{Urumutum}
\begin{itemize}
\item {Grp. gram.:m.}
\end{itemize}
\begin{itemize}
\item {Utilização:Bras}
\end{itemize}
Ave gallinácea da América.
(Do tupi)
\section{Urupás}
\begin{itemize}
\item {Grp. gram.:m. Pl.}
\end{itemize}
Indígenas do Norte do Brasil.
\section{Urupé}
\begin{itemize}
\item {Grp. gram.:m.}
\end{itemize}
\begin{itemize}
\item {Utilização:Bras}
\end{itemize}
Espécie de cogumelo.
\section{Urupema}
\begin{itemize}
\item {Grp. gram.:f.}
\end{itemize}
\begin{itemize}
\item {Utilização:Bras}
\end{itemize}
Espécie de joeira de palha de cana, para peneirar farinha de mandioca.
\section{Urupemba}
\begin{itemize}
\item {Grp. gram.:f.}
\end{itemize}
O mesmo que \textunderscore urupema\textunderscore .
\section{Ururau}
\begin{itemize}
\item {Grp. gram.:m.}
\end{itemize}
\begin{itemize}
\item {Utilização:Bras}
\end{itemize}
Espécie de lagarto voraz.
(Alter. do tupi \textunderscore ururá\textunderscore )
\section{Ururi}
\begin{itemize}
\item {Grp. gram.:m.}
\end{itemize}
Fruto silvestre do Brasil.
\section{Úrus}
\begin{itemize}
\item {Grp. gram.:m.}
\end{itemize}
(V.uro)
\section{Urussacanga}
\begin{itemize}
\item {Grp. gram.:m.}
\end{itemize}
\begin{itemize}
\item {Utilização:Bras}
\end{itemize}
Grande cesto, o mesmo que \textunderscore aturá\textunderscore .
\section{Urutago}
\begin{itemize}
\item {Grp. gram.:m.}
\end{itemize}
\begin{itemize}
\item {Utilização:Bras}
\end{itemize}
O mesmo que \textunderscore urutau\textunderscore ? Cf. Júl. Ribeiro, \textunderscore Padre Belch.\textunderscore , 168:«\textunderscore ...o regougo de um urutago...\textunderscore »
\section{Urutau}
\begin{itemize}
\item {Grp. gram.:m.}
\end{itemize}
\begin{itemize}
\item {Utilização:Bras}
\end{itemize}
\begin{itemize}
\item {Proveniência:(T. tupi)}
\end{itemize}
Ave nocturna de rapina.
\section{Urutu}
\begin{itemize}
\item {Grp. gram.:m.}
\end{itemize}
\begin{itemize}
\item {Utilização:Bras}
\end{itemize}
Espécie de cobra muito venenosa, (\textunderscore lachesus alternatus\textunderscore ).
\section{Uruxi}
\begin{itemize}
\item {Grp. gram.:m.}
\end{itemize}
\begin{itemize}
\item {Utilização:Ant.}
\end{itemize}
Espécie de verniz do Japão.
\section{Urvílea}
\begin{itemize}
\item {Grp. gram.:f.}
\end{itemize}
\begin{itemize}
\item {Proveniência:(De \textunderscore Urville\textunderscore , n. p.)}
\end{itemize}
Gênero de plantas sapindáceas.
\section{Urzal}
\begin{itemize}
\item {Grp. gram.:m.}
\end{itemize}
Terreno, onde crescem urzes.
Matagal pouco crescido.
\section{Urze}
\begin{itemize}
\item {Grp. gram.:f.}
\end{itemize}
\begin{itemize}
\item {Proveniência:(Do lat. \textunderscore ulex\textunderscore )}
\end{itemize}
Planta ericínia.
Torga, queiró.
Espécie de uva do Doiro.
Árvore açoreana, (\textunderscore erica azorica\textunderscore ).
Casta de uva branca da região do Doiro.
\section{Urze-branca}
\begin{itemize}
\item {Grp. gram.:f.}
\end{itemize}
Espécie de urze arbustiva, (\textunderscore erica arborea\textunderscore , Lin.).
\section{Urze-das-vassoiras}
\begin{itemize}
\item {Grp. gram.:f.}
\end{itemize}
Planta ericácea, (\textunderscore erica scoparia\textunderscore , Lin.).
\section{Urzeira}
\begin{itemize}
\item {Grp. gram.:f.}
\end{itemize}
O mesmo que \textunderscore urzal\textunderscore .
\section{Urzeiro}
\begin{itemize}
\item {Grp. gram.:m.}
\end{itemize}
Urze arbustiva.
\section{Urzela}
\begin{itemize}
\item {Grp. gram.:f.}
\end{itemize}
\begin{itemize}
\item {Proveniência:(Do it. \textunderscore oricello\textunderscore . Cp. cast. \textunderscore orchilla\textunderscore )}
\end{itemize}
Espécie de líchen tinctorial.
\section{Urzelina}
\begin{itemize}
\item {Grp. gram.:f.}
\end{itemize}
\begin{itemize}
\item {Utilização:Açor}
\end{itemize}
Terreno, semeado de urzela.
\section{Urzella}
\begin{itemize}
\item {Grp. gram.:f.}
\end{itemize}
\begin{itemize}
\item {Proveniência:(Do it. \textunderscore oricello\textunderscore . Cp. cast. \textunderscore orchilla\textunderscore )}
\end{itemize}
Espécie de líchen tinctorial.
\section{Urzellina}
\begin{itemize}
\item {Grp. gram.:f.}
\end{itemize}
\begin{itemize}
\item {Utilização:Açor}
\end{itemize}
Terreno, semeado de urzella.
\section{Urzibelha}
\begin{itemize}
\item {fónica:bê}
\end{itemize}
\begin{itemize}
\item {Grp. gram.:f.}
\end{itemize}
\begin{itemize}
\item {Utilização:Prov.}
\end{itemize}
\begin{itemize}
\item {Utilização:trasm.}
\end{itemize}
Arbusto, o mesmo que \textunderscore chaguarço\textunderscore .
(Affinidade com \textunderscore urgebão\textunderscore ?)
\section{Usado}
\begin{itemize}
\item {Grp. gram.:adj.}
\end{itemize}
\begin{itemize}
\item {Proveniência:(De \textunderscore usar\textunderscore )}
\end{itemize}
Acostumado.
Deteriorado, gasto.
\section{Usagem}
\begin{itemize}
\item {Grp. gram.:f.}
\end{itemize}
\begin{itemize}
\item {Utilização:Des.}
\end{itemize}
\begin{itemize}
\item {Proveniência:(De \textunderscore usar\textunderscore )}
\end{itemize}
Uso.
Direito, baseado no uso.
\section{Usai-della}
\begin{itemize}
\item {Grp. gram.:f.}
\end{itemize}
\begin{itemize}
\item {Utilização:Açor}
\end{itemize}
O mesmo que \textunderscore erva-formigueira\textunderscore .
\section{Usança}
\begin{itemize}
\item {Grp. gram.:f.}
\end{itemize}
\begin{itemize}
\item {Proveniência:(De \textunderscore usar\textunderscore )}
\end{itemize}
Uso.
Hábito antigo.
Costumeira.
\section{Usar}
\begin{itemize}
\item {Grp. gram.:v. t.}
\end{itemize}
\begin{itemize}
\item {Grp. gram.:V. i.}
\end{itemize}
\begin{itemize}
\item {Proveniência:(De \textunderscore uso\textunderscore )}
\end{itemize}
Praticar: \textunderscore usar um offício\textunderscore .
Têr por costume: \textunderscore usar mentir\textunderscore .
Empregar.
Vestir, trajar.
Trazer por hábito.
Deteriorar; cotiar.
Estar acostumado.
Servir-se.
\section{Usável}
\begin{itemize}
\item {Grp. gram.:adj.}
\end{itemize}
\begin{itemize}
\item {Utilização:Ant.}
\end{itemize}
Que se póde usar.
O mesmo que \textunderscore usual\textunderscore .
\section{Useiro}
\begin{itemize}
\item {Grp. gram.:adj.}
\end{itemize}
\begin{itemize}
\item {Proveniência:(De \textunderscore uso\textunderscore )}
\end{itemize}
Que costuma fazer alguma coisa.
Que tem por uso alguma coisa.
\section{Usitar}
\begin{itemize}
\item {Grp. gram.:v. t.}
\end{itemize}
\begin{itemize}
\item {Proveniência:(Lat. \textunderscore usitare\textunderscore )}
\end{itemize}
O mesmo que \textunderscore usar\textunderscore ; empregar com frequência. Cf. Castilho, \textunderscore Camões.\textunderscore 
\section{Uçu}
\begin{itemize}
\item {Grp. gram.:adj.}
\end{itemize}
O mesmo que \textunderscore guassu\textunderscore  ou \textunderscore guaçu\textunderscore .
\section{Usmar}
\textunderscore v. t.\textunderscore  (e der.)
Corr. trasm. de \textunderscore esmar\textunderscore , etc.
\section{Usmeira}
\begin{itemize}
\item {Grp. gram.:adj. f.}
\end{itemize}
\begin{itemize}
\item {Utilização:Prov.}
\end{itemize}
\begin{itemize}
\item {Utilização:trasm.}
\end{itemize}
\begin{itemize}
\item {Proveniência:(De \textunderscore usmar\textunderscore ?)}
\end{itemize}
Diz-se da mulhér, que é useira em qualquer coisa.
\section{Úsnea}
\begin{itemize}
\item {Grp. gram.:f.}
\end{itemize}
\begin{itemize}
\item {Proveniência:(Do ár. \textunderscore ashnah\textunderscore )}
\end{itemize}
Gênero de líchens tinctoriaes; penugem.
\section{Uso}
\begin{itemize}
\item {Grp. gram.:m.}
\end{itemize}
\begin{itemize}
\item {Proveniência:(Lat. \textunderscore usus\textunderscore )}
\end{itemize}
Acto ou effeito de usar.
Moda.
Emprêgo de qualquer coisa; applicação; serviço: \textunderscore esta loiça tem muito uso\textunderscore .
Cotio.
Experiência.
\section{Ussa}
\begin{itemize}
\item {Grp. gram.:f.}
\end{itemize}
Planta africana, herbácea, ornamental, de fôlhas simples, levemente recortadas, e flôres vermelhas, inodoras.
\section{Ússar}
\begin{itemize}
\item {Grp. gram.:m.}
\end{itemize}
O mesmo ou melhór que \textunderscore hússar\textunderscore . Cf. Rui Barb., \textunderscore Réplica\textunderscore , 158.
\section{Ussia}
\begin{itemize}
\item {Grp. gram.:f.}
\end{itemize}
(V.adussia)
\section{Usso}
\begin{itemize}
\item {Grp. gram.:m.}
\end{itemize}
\begin{itemize}
\item {Utilização:Ant.}
\end{itemize}
O mesmo que \textunderscore urso\textunderscore . Cf. \textunderscore Eufrosina\textunderscore , 126; Usque, 17.
\section{Ussu}
\begin{itemize}
\item {Grp. gram.:adj.}
\end{itemize}
O mesmo que \textunderscore guassu\textunderscore  ou \textunderscore guaçu\textunderscore .
\section{Ussubi}
\begin{itemize}
\item {Grp. gram.:m.}
\end{itemize}
Árvore da ilha de San-Thomé.
\section{Ustão}
\begin{itemize}
\item {Grp. gram.:f.}
\end{itemize}
\begin{itemize}
\item {Proveniência:(Do lat. \textunderscore ustio\textunderscore )}
\end{itemize}
Acto ou effeito de queimar.
Cauterização.
Combustão.
\section{Uste}
\begin{itemize}
\item {Grp. gram.:m.}
\end{itemize}
Us. na loc. proverbial: \textunderscore quem quer uste, que lhe custe\textunderscore , isto é, quem quer riquezas, que labute. Cf. Sim. Mach., f. 68, v.^o
\section{Ustéria}
\begin{itemize}
\item {Grp. gram.:f.}
\end{itemize}
Gênero de plantas liliáceas.
\section{Ustório}
\begin{itemize}
\item {Grp. gram.:adj.}
\end{itemize}
\begin{itemize}
\item {Proveniência:(Do lat. \textunderscore ustor\textunderscore )}
\end{itemize}
Que queima; que facilita a queimadura.
\section{Ustrina}
\begin{itemize}
\item {Grp. gram.:f.}
\end{itemize}
\begin{itemize}
\item {Proveniência:(Lat. \textunderscore ustrina\textunderscore )}
\end{itemize}
Lugar, onde os Romanos queimavam os cadáveres, na occasião do funeral.
\section{Ustulação}
\begin{itemize}
\item {Grp. gram.:f.}
\end{itemize}
\begin{itemize}
\item {Proveniência:(Lat. \textunderscore ustulatio\textunderscore )}
\end{itemize}
Acto ou effeito de ustular.
\section{Ustular}
\begin{itemize}
\item {Grp. gram.:v. t.}
\end{itemize}
\begin{itemize}
\item {Proveniência:(Lat. \textunderscore ustulare\textunderscore )}
\end{itemize}
Queimar ligeiramente.
Secar ao fôgo.
\section{Usual}
\begin{itemize}
\item {Grp. gram.:adj.}
\end{itemize}
\begin{itemize}
\item {Proveniência:(Lat. \textunderscore usualis\textunderscore )}
\end{itemize}
Que se usa geralmente; habitual; frequente.
\section{Usualmente}
\begin{itemize}
\item {Grp. gram.:adv.}
\end{itemize}
De modo usual; commummente; em geral; vulgarmente.
\section{Usuano}
\begin{itemize}
\item {Grp. gram.:adj.}
\end{itemize}
(?)«\textunderscore ...da usuana da Serpente que lá no Brasil serve de carruagem...\textunderscore »\textunderscore Anat. Joc.\textunderscore  II, 445.
\section{Usuário}
\begin{itemize}
\item {Grp. gram.:m.  e  adj.}
\end{itemize}
\begin{itemize}
\item {Proveniência:(Lat. \textunderscore usuarius\textunderscore )}
\end{itemize}
O que possue ou frue alguma coisa por direito que provém do uso.
Que serve para nosso uso.
Dizia-se do escravo, de que só se tinha o uso e não a propriedade.
\section{Usucapião}
\begin{itemize}
\item {Grp. gram.:m.}
\end{itemize}
\begin{itemize}
\item {Utilização:Jur.}
\end{itemize}
\begin{itemize}
\item {Proveniência:(Lat. \textunderscore usucapio\textunderscore )}
\end{itemize}
Modo antigo de adquirir propriedade, pela posse pacífica durante certo tempo.
Espécie de prescripção.
\section{Usucapiente}
\begin{itemize}
\item {Grp. gram.:m.  e  adj.}
\end{itemize}
\begin{itemize}
\item {Proveniência:(Lat. \textunderscore usucapiens\textunderscore )}
\end{itemize}
O que usucapiu.
\section{Usucapir}
\begin{itemize}
\item {Grp. gram.:v.}
\end{itemize}
\begin{itemize}
\item {Utilização:t. Jur.}
\end{itemize}
\begin{itemize}
\item {Proveniência:(Lat. \textunderscore usucapere\textunderscore )}
\end{itemize}
Adquirir pelo usucapião.
\section{Usucapto}
\begin{itemize}
\item {Grp. gram.:adj.}
\end{itemize}
\begin{itemize}
\item {Proveniência:(Lat. \textunderscore usucaptus\textunderscore )}
\end{itemize}
Adquirido por usucapião.
\section{Usufructueiro}
\begin{itemize}
\item {Grp. gram.:adj.}
\end{itemize}
O mesmo que \textunderscore usufructuário\textunderscore . Cf. Camillo, \textunderscore Sc. da Foz\textunderscore , 198.
\section{Usufructo}
\begin{itemize}
\item {Grp. gram.:m.}
\end{itemize}
\begin{itemize}
\item {Proveniência:(Lat. \textunderscore usufructus\textunderscore )}
\end{itemize}
Acto ou effeito de usufruir; aquillo que se usufrue.
Direito, proveniente do usufructo; fruição.
\section{Usufructuar}
\begin{itemize}
\item {Grp. gram.:v. t.}
\end{itemize}
\begin{itemize}
\item {Proveniência:(De \textunderscore usufructo\textunderscore )}
\end{itemize}
O mesmo que \textunderscore usufruir\textunderscore .
\section{Usufructuário}
\begin{itemize}
\item {Grp. gram.:adj.}
\end{itemize}
\begin{itemize}
\item {Grp. gram.:M.}
\end{itemize}
\begin{itemize}
\item {Proveniência:(Lat. \textunderscore usufructuarius\textunderscore )}
\end{itemize}
Relativo ao usufructo.
Aquelle que usufrue.
\section{Usufruição}
\begin{itemize}
\item {fónica:fru-i}
\end{itemize}
\begin{itemize}
\item {Grp. gram.:f.}
\end{itemize}
Acto ou effeito de usufruir.
\section{Usufrutueiro}
\begin{itemize}
\item {Grp. gram.:adj.}
\end{itemize}
O mesmo que \textunderscore usufrutuário\textunderscore . Cf. Camillo, \textunderscore Sc. da Foz\textunderscore , 198.
\section{Usufruto}
\begin{itemize}
\item {Grp. gram.:m.}
\end{itemize}
\begin{itemize}
\item {Proveniência:(Lat. \textunderscore usufructus\textunderscore )}
\end{itemize}
Acto ou efeito de usufruir; aquilo que se usufrue.
Direito, proveniente do usufruto; fruição.
\section{Usufrutuar}
\begin{itemize}
\item {Grp. gram.:v. t.}
\end{itemize}
\begin{itemize}
\item {Proveniência:(De \textunderscore usufruto\textunderscore )}
\end{itemize}
O mesmo que \textunderscore usufruir\textunderscore .
\section{Usufrutuário}
\begin{itemize}
\item {Grp. gram.:adj.}
\end{itemize}
\begin{itemize}
\item {Grp. gram.:M.}
\end{itemize}
\begin{itemize}
\item {Proveniência:(Lat. \textunderscore usufructuarius\textunderscore )}
\end{itemize}
Relativo ao usufruto.
Aquele que usufrue.
\section{Usufruir}
\begin{itemize}
\item {Grp. gram.:v. t.}
\end{itemize}
\begin{itemize}
\item {Proveniência:(Do lat. \textunderscore usus\textunderscore  + \textunderscore frui\textunderscore )}
\end{itemize}
Têr a posse e o gôzo de (alguma coisa que se não póde alienar ou destruír).
\section{Usufruto}
\textunderscore m.\textunderscore  (e der.)
O mesmo ou melhór que \textunderscore usufructo\textunderscore , etc.
\section{Usura}
\begin{itemize}
\item {Grp. gram.:f.}
\end{itemize}
\begin{itemize}
\item {Proveniência:(Lat. \textunderscore usura\textunderscore )}
\end{itemize}
Juro de um capital.
Juro de dinheiro que se emprestou.
Contrato de empréstimo, com a cláusula do pagamento de juros por parte do devedor.
Juro excessivo; lucro exaggerado.
\section{Usurar}
\begin{itemize}
\item {Grp. gram.:v. i.}
\end{itemize}
\begin{itemize}
\item {Utilização:Des.}
\end{itemize}
Emprestar dinheiro ou outras coisas com usura.
\section{Usurariamente}
\begin{itemize}
\item {Grp. gram.:adv.}
\end{itemize}
\begin{itemize}
\item {Proveniência:(De \textunderscore usurário\textunderscore )}
\end{itemize}
Com usura.
\section{Usurário}
\begin{itemize}
\item {Grp. gram.:adj.}
\end{itemize}
\begin{itemize}
\item {Grp. gram.:M.}
\end{itemize}
\begin{itemize}
\item {Utilização:Pop.}
\end{itemize}
\begin{itemize}
\item {Proveniência:(Lat. \textunderscore usurarius\textunderscore )}
\end{itemize}
Que empresta com juro excessivo.
Que tem o carácter da usura ou é acompanhado por ella.
Aquelle que empresta com usura ou com juro excessivo.
Agiota; avarento.
\section{Usureiro}
\begin{itemize}
\item {Grp. gram.:m.  e  adj.}
\end{itemize}
(V.usurário). Cf. \textunderscore Luz e Calor\textunderscore , 51.
\section{Usurpação}
\begin{itemize}
\item {Grp. gram.:f.}
\end{itemize}
\begin{itemize}
\item {Proveniência:(Do lat. \textunderscore usurpatio\textunderscore )}
\end{itemize}
Acto ou effeito de usurpar.
\section{Usurpador}
\begin{itemize}
\item {Grp. gram.:m.  e  adj.}
\end{itemize}
\begin{itemize}
\item {Proveniência:(Lat. \textunderscore usurpator\textunderscore )}
\end{itemize}
O que usurpa; intruso.
\section{Usurpar}
\begin{itemize}
\item {Grp. gram.:v. t.}
\end{itemize}
\begin{itemize}
\item {Proveniência:(Lat. \textunderscore usurpare\textunderscore )}
\end{itemize}
Apoderar-se violentamente de.
Adquirir fraudulentamente.
Obter sem direito: \textunderscore usurpar um throno\textunderscore .
\section{Ut}
\begin{itemize}
\item {fónica:ud'}
\end{itemize}
\begin{itemize}
\item {Grp. gram.:m.}
\end{itemize}
\begin{itemize}
\item {Utilização:Ant.}
\end{itemize}
Primeira nota da escala musical, hoje substituída por \textunderscore dó\textunderscore .
(Cp. \textunderscore fá\textunderscore )
\section{Utar}
\begin{itemize}
\item {Grp. gram.:v. t.}
\end{itemize}
(V.outar)
\section{Utata}
\begin{itemize}
\item {Grp. gram.:f.}
\end{itemize}
Árvore angolense de Caconda.
\section{Utena}
\begin{itemize}
\item {Grp. gram.:f.}
\end{itemize}
Pássaro dentirostro africano.
\section{Utênsil}
\begin{itemize}
\item {Grp. gram.:m.}
\end{itemize}
\begin{itemize}
\item {Proveniência:(Lat. \textunderscore utensilis\textunderscore )}
\end{itemize}
O mesmo que \textunderscore utensílio\textunderscore . Cf. Filinto, IX, 72.
\section{Utensílio}
\begin{itemize}
\item {Grp. gram.:m.}
\end{itemize}
\begin{itemize}
\item {Proveniência:(Do lat. hyp. \textunderscore utensilium\textunderscore )}
\end{itemize}
Qualquer instrumento de trabalho, de que se serve um artista ou um industrial.
Objecto, que serve de meio ou instrumento para se fazer qualquer coisa: \textunderscore utensílios de cozinha\textunderscore .
\section{Utente}
\begin{itemize}
\item {Grp. gram.:adj.}
\end{itemize}
\begin{itemize}
\item {Proveniência:(Lat. \textunderscore utens\textunderscore )}
\end{itemize}
Que usa.
\section{Uteralgia}
\begin{itemize}
\item {Grp. gram.:f.}
\end{itemize}
\begin{itemize}
\item {Proveniência:(De \textunderscore útero\textunderscore  + gr. \textunderscore algus\textunderscore )}
\end{itemize}
Dôr no útero.
\section{Uteremia}
\begin{itemize}
\item {Grp. gram.:f.}
\end{itemize}
\begin{itemize}
\item {Proveniência:(De \textunderscore útero\textunderscore  + gr. \textunderscore haima\textunderscore )}
\end{itemize}
Congestão sanguínea do útero.
\section{Uterino}
\begin{itemize}
\item {Grp. gram.:adj.}
\end{itemize}
\begin{itemize}
\item {Proveniência:(Lat. \textunderscore uterinus\textunderscore )}
\end{itemize}
Relativo a útero.
\section{Útero}
\begin{itemize}
\item {Grp. gram.:m.}
\end{itemize}
\begin{itemize}
\item {Proveniência:(Lat. \textunderscore uterus\textunderscore )}
\end{itemize}
Órgão, em que se gera o féto dos mammíferos; madre.
\section{Uteróceps}
\begin{itemize}
\item {Grp. gram.:m.}
\end{itemize}
\begin{itemize}
\item {Utilização:Cir.}
\end{itemize}
\begin{itemize}
\item {Proveniência:(Do lat. \textunderscore uterus\textunderscore  + \textunderscore capere\textunderscore )}
\end{itemize}
Instrumento, com que se apprehende o collo do útero.
\section{Uteromania}
\begin{itemize}
\item {Grp. gram.:f.}
\end{itemize}
\begin{itemize}
\item {Proveniência:(De \textunderscore útero\textunderscore  + \textunderscore mania\textunderscore )}
\end{itemize}
O mesmo que \textunderscore nymphomania\textunderscore .
\section{Utero-placentário}
\begin{itemize}
\item {Grp. gram.:adj.}
\end{itemize}
\begin{itemize}
\item {Utilização:Anat.}
\end{itemize}
Relativo ao útero e á placenta.
\section{Uterorragia}
\begin{itemize}
\item {Grp. gram.:f.}
\end{itemize}
\begin{itemize}
\item {Proveniência:(De \textunderscore útero\textunderscore  + gr. \textunderscore regnumi\textunderscore )}
\end{itemize}
O mesmo que \textunderscore metrorragia\textunderscore .
\section{Uterorrhagia}
\begin{itemize}
\item {Grp. gram.:f.}
\end{itemize}
\begin{itemize}
\item {Proveniência:(De \textunderscore útero\textunderscore  + gr. \textunderscore regnumi\textunderscore )}
\end{itemize}
O mesmo que \textunderscore metrorrhagia\textunderscore .
\section{Uteroscopia}
\begin{itemize}
\item {Grp. gram.:f.}
\end{itemize}
\begin{itemize}
\item {Proveniência:(De \textunderscore útero\textunderscore  + gr. \textunderscore skopein\textunderscore )}
\end{itemize}
Observação do útero, por meio de instrumentos apropriados.
\section{Uterostomátomo}
\begin{itemize}
\item {Grp. gram.:m.}
\end{itemize}
\begin{itemize}
\item {Utilização:Cir.}
\end{itemize}
\begin{itemize}
\item {Proveniência:(T. hybr., do lat. \textunderscore uterus\textunderscore  + gr. \textunderscore stoma\textunderscore  + \textunderscore tome\textunderscore )}
\end{itemize}
Instrumento, para a incisão dos bordos do collo do útero, quando se manifestam convulsões na occasião do parto.
\section{Uterotomia}
\begin{itemize}
\item {Grp. gram.:f.}
\end{itemize}
\begin{itemize}
\item {Proveniência:(De \textunderscore uterótomo\textunderscore )}
\end{itemize}
Incisão do collo do útero.
\section{Uterotómico}
\begin{itemize}
\item {Grp. gram.:adj.}
\end{itemize}
Relativo á uterotomia.
\section{Uterótomo}
\begin{itemize}
\item {Grp. gram.:m.}
\end{itemize}
\begin{itemize}
\item {Proveniência:(De \textunderscore útero\textunderscore  + gr. \textunderscore tome\textunderscore )}
\end{itemize}
Instrumento, com que se opera a uterotomia.
\section{Utero-vaginal}
\begin{itemize}
\item {Grp. gram.:adj.}
\end{itemize}
\begin{itemize}
\item {Utilização:Anat.}
\end{itemize}
Relativo ao útero e á vagina.
\section{Uteruéria}
\begin{itemize}
\item {Grp. gram.:f.}
\end{itemize}
Gênero de plantas capparídeas.
\section{Uticense}
\begin{itemize}
\item {Grp. gram.:adj.}
\end{itemize}
\begin{itemize}
\item {Grp. gram.:M.}
\end{itemize}
\begin{itemize}
\item {Proveniência:(Lat. \textunderscore uticensis\textunderscore )}
\end{itemize}
Relativo a Útica.
Habitante de Útica.--Além da cidade africana célebre pelo suicídio de Catão, há hoje outra cidade do mesmo nome na América no Norte. Cf. \textunderscore Eufrosina\textunderscore , 98.
\section{Útil}
\begin{itemize}
\item {Grp. gram.:adj.}
\end{itemize}
\begin{itemize}
\item {Grp. gram.:M.}
\end{itemize}
\begin{itemize}
\item {Utilização:Des.}
\end{itemize}
\begin{itemize}
\item {Proveniência:(Lat. \textunderscore utilis\textunderscore )}
\end{itemize}
Que póde têr algum uso, ou que serve para alguma coisa.
Vantajoso.
Em que se póde trabalhar, (falando-se de certos dias).
Determinado por lei.
Utilidade; aquillo que é útil:«\textunderscore ...que não havia uteis que equivalessem aos riscos...\textunderscore »Filinto, \textunderscore D. Man.\textunderscore , I, 48.
\section{Utilidade}
\begin{itemize}
\item {Grp. gram.:f.}
\end{itemize}
\begin{itemize}
\item {Proveniência:(Do lat. \textunderscore utilitas\textunderscore )}
\end{itemize}
Qualidade do que é útil.
Serventia.
Vantagem.
Pessôa ou coisa útil.
\section{Utilitariamente}
\begin{itemize}
\item {Grp. gram.:adv.}
\end{itemize}
De modo utilitário.
Com feição prática ou positiva.
Por interesse.
\section{Utilitário}
\begin{itemize}
\item {Grp. gram.:adj.}
\end{itemize}
\begin{itemize}
\item {Grp. gram.:M.}
\end{itemize}
\begin{itemize}
\item {Proveniência:(Do lat. \textunderscore utilitas\textunderscore )}
\end{itemize}
Relativo a utilidade.
Aquelle que tem a utilidade ou o interesse como fim principal dos seus actos.
\section{Utilitarismo}
\begin{itemize}
\item {Grp. gram.:m.}
\end{itemize}
Systema dos utilitários.
\section{Utilitarista}
\begin{itemize}
\item {Grp. gram.:m.}
\end{itemize}
Partidário do utilitarismo.
\section{Utilização}
\begin{itemize}
\item {Grp. gram.:f.}
\end{itemize}
Acto ou effeito de utilizar.
\section{Utilizar}
\begin{itemize}
\item {Grp. gram.:v. t.}
\end{itemize}
\begin{itemize}
\item {Grp. gram.:V. i.}
\end{itemize}
\begin{itemize}
\item {Grp. gram.:V. p.}
\end{itemize}
Tornar útil.
Aproveitar.
Empregar com vantagem.
Sêr útil: \textunderscore a má lingua não utiliza a ninguém\textunderscore .
Lançar mão; tirar proveito: \textunderscore utilizar-se da ignorância alheia\textunderscore .
\section{Utilizável}
\begin{itemize}
\item {Grp. gram.:adj.}
\end{itemize}
Que se póde utilizar.
\section{Utilmente}
\begin{itemize}
\item {Grp. gram.:adv.}
\end{itemize}
De modo útil; com vantagem; com interesse.
\section{Utopia}
\begin{itemize}
\item {Grp. gram.:f.}
\end{itemize}
\begin{itemize}
\item {Utilização:Fig.}
\end{itemize}
\begin{itemize}
\item {Utilização:Ext.}
\end{itemize}
\begin{itemize}
\item {Proveniência:(Do gr. ou + \textunderscore topos\textunderscore )}
\end{itemize}
País imaginário, em que tudo está organizado da melhór fórma.
Plano de govêrno, de que resultaria a felicidade pública, se êlle fosse exequível.
Projecto imaginário.
Systema ou ideia irrealizável; fantasia.
\section{Utópico}
\begin{itemize}
\item {Grp. gram.:adj.}
\end{itemize}
Relativo a utopia.
\section{Utopista}
\begin{itemize}
\item {Grp. gram.:adj.}
\end{itemize}
\begin{itemize}
\item {Grp. gram.:M.  e  f.}
\end{itemize}
\begin{itemize}
\item {Proveniência:(De \textunderscore utopia\textunderscore )}
\end{itemize}
O mesmo que \textunderscore utópico\textunderscore .
Pessôa, que fórma ou defende utopias.
\section{Utopístico}
\begin{itemize}
\item {Grp. gram.:adj.}
\end{itemize}
Próprio de utopista.
\section{Utota}
\begin{itemize}
\item {Grp. gram.:f.}
\end{itemize}
Árvore africana, de tronco tortuoso, fôlhas sempre verdes, glabras, lisas, e flôres completas, gamopétalas.
\section{Utre}
\begin{itemize}
\item {Grp. gram.:m.}
\end{itemize}
(Fórma pop. de \textunderscore útero\textunderscore )
\section{Utricular}
\begin{itemize}
\item {Grp. gram.:adj.}
\end{itemize}
Semelhante a um utrículo.
\section{Utriculária}
\begin{itemize}
\item {Grp. gram.:f.}
\end{itemize}
\begin{itemize}
\item {Proveniência:(De \textunderscore utrículo\textunderscore )}
\end{itemize}
Gênero de plantas herbáceas e aquáticas.
\section{Utriculariáceas}
\begin{itemize}
\item {Grp. gram.:f. pl.}
\end{itemize}
Família de plantas, que têm por typo a utriculária.
(Fem. \textunderscore pl.\textunderscore  de \textunderscore utriculariáceo\textunderscore )
\section{Utriculariáceo}
\begin{itemize}
\item {Grp. gram.:adj.}
\end{itemize}
Relativo ou semelhante á utriculária.
\section{Utriculariadas}
\begin{itemize}
\item {Grp. gram.:f. pl.}
\end{itemize}
(V.utriculariáceas)
\section{Utriculariforme}
\begin{itemize}
\item {Grp. gram.:adj.}
\end{itemize}
O mesmo que \textunderscore utricular\textunderscore .
\section{Utricularíneas}
\begin{itemize}
\item {Grp. gram.:f. pl.}
\end{itemize}
O mesmo que \textunderscore utriculariáceas\textunderscore .
\section{Utriculário}
\begin{itemize}
\item {Grp. gram.:m.}
\end{itemize}
\begin{itemize}
\item {Proveniência:(Lat. \textunderscore utricularius\textunderscore )}
\end{itemize}
Nome, que se deu, entre os antigos, ao tocador de cornamusa ou de gaita de folles.
\section{Utrículo}
\begin{itemize}
\item {Grp. gram.:m.}
\end{itemize}
\begin{itemize}
\item {Utilização:Bot.}
\end{itemize}
\begin{itemize}
\item {Utilização:Anat.}
\end{itemize}
\begin{itemize}
\item {Utilização:Bot.}
\end{itemize}
\begin{itemize}
\item {Proveniência:(Lat. \textunderscore utriculus\textunderscore )}
\end{itemize}
Pequeno saco.
Cada uma das céllulas de tecido cellular dos vegetaes.
A maior porção de labyrintho membranoso do ouvido.
Cavidade dos órgãos pollínicos.
\section{Utriculoso}
\begin{itemize}
\item {Grp. gram.:adj.}
\end{itemize}
Que tem utrículos.
\section{Utriforme}
\begin{itemize}
\item {Grp. gram.:adj.}
\end{itemize}
\begin{itemize}
\item {Proveniência:(Do lat. \textunderscore uter\textunderscore  + \textunderscore forma\textunderscore )}
\end{itemize}
Que tem fórma de odre.
\section{Utuaba}
\begin{itemize}
\item {Grp. gram.:f.}
\end{itemize}
Planta meliácea do Brasil.
\section{Utuapoca}
\begin{itemize}
\item {Grp. gram.:f.}
\end{itemize}
Planta meliácea do Brasil.
\section{Utuaúba}
\begin{itemize}
\item {Grp. gram.:f.}
\end{itemize}
O mesmo que \textunderscore utuaba\textunderscore .
\section{Uuçango}
\begin{itemize}
\item {Grp. gram.:m.}
\end{itemize}
Árvore de Angola.
\section{Uussango}
\begin{itemize}
\item {Grp. gram.:m.}
\end{itemize}
Árvore de Angola.
\section{Uva}
\begin{itemize}
\item {Grp. gram.:f.}
\end{itemize}
\begin{itemize}
\item {Proveniência:(Lat. \textunderscore uva\textunderscore )}
\end{itemize}
Nome do bago, que é o fruto da videira.
Conjunto dêsses frutos, constituindo um cacho.
Conjunto dos frutos de uma videira.
Designação genérica dos fructos das vinhas: \textunderscore este anno, a uva soffreu muito com o calor\textunderscore .
\section{Uva-açu}
\begin{itemize}
\item {Grp. gram.:f.}
\end{itemize}
Planta palmácea, da América meridional, (\textunderscore manicaria saccifera\textunderscore ).
\section{Uva-aia}
\begin{itemize}
\item {Grp. gram.:f.}
\end{itemize}
\begin{itemize}
\item {Utilização:Bras}
\end{itemize}
Árvore fructífera, (\textunderscore eugenia arrabidae\textunderscore , Berg.).
\section{Uvaça}
\begin{itemize}
\item {Grp. gram.:f.}
\end{itemize}
Grande porção de uvas.
\section{Uva-crespa}
\begin{itemize}
\item {Grp. gram.:f.}
\end{itemize}
O mesmo que \textunderscore groselheira\textunderscore .
\section{Uvacupari}
\begin{itemize}
\item {Grp. gram.:f.}
\end{itemize}
\begin{itemize}
\item {Utilização:Bras}
\end{itemize}
Árvore fructífera dos sertões.
\section{Uvada}
\begin{itemize}
\item {Grp. gram.:f.}
\end{itemize}
Conserva de uvas.
\section{Uva-da-promissão}
\begin{itemize}
\item {Grp. gram.:f.}
\end{itemize}
Variedade de uva branca.
\section{Uva-da-promissão-roxa}
\begin{itemize}
\item {Grp. gram.:f.}
\end{itemize}
Casta de uva de Castello de Vide. Cf. \textunderscore Rev. Agron.\textunderscore , I, 18.
\section{Uva-de-cão}
\begin{itemize}
\item {Grp. gram.:f.}
\end{itemize}
Planta solânea, (\textunderscore solanum dulcamara\textunderscore ).
Variedade de uva miúda e azêda.
\section{Uva-de-cheiro}
\begin{itemize}
\item {Grp. gram.:f.}
\end{itemize}
Designação popular da uva isabel.
(Cp. \textunderscore isabel\textunderscore )
\section{Uva-de-gallo}
\begin{itemize}
\item {Grp. gram.:m.}
\end{itemize}
O mesmo que \textunderscore coração-de-gallo\textunderscore .
\section{Uva-de-joão-paes}
\begin{itemize}
\item {Grp. gram.:f.}
\end{itemize}
O mesmo que \textunderscore escabellado\textunderscore .
\section{Uva-de-obó}
\begin{itemize}
\item {Grp. gram.:f.}
\end{itemize}
Planta medicinal da ilha de San-Thomé.
\section{Uva-de-rei}
\begin{itemize}
\item {Grp. gram.:f.}
\end{itemize}
Casta de uva trasmontana.
\section{Uva-de-urso}
\begin{itemize}
\item {Grp. gram.:f.}
\end{itemize}
Planta, o mesmo que \textunderscore uva-ursina\textunderscore .
\section{Uva-do-inferno}
\begin{itemize}
\item {Grp. gram.:f.}
\end{itemize}
Variedade de uva minhota, muito resistente e branca.
\section{Uva-do-inverno}
\begin{itemize}
\item {Grp. gram.:f.}
\end{itemize}
Variedade de uva minhota, muito resistente e branca.
\section{Uva-do-mato}
\begin{itemize}
\item {Grp. gram.:f.}
\end{itemize}
Planta cordiácea, (\textunderscore cordia argentea\textunderscore ).
\section{Uva-do-monte}
\begin{itemize}
\item {Grp. gram.:f.}
\end{itemize}
Árvore ericácea, o mesmo que \textunderscore arando\textunderscore .
\section{Uva-do-nascimento}
\begin{itemize}
\item {Grp. gram.:f.}
\end{itemize}
Casta de uva Penafiel. Cf. \textunderscore Rev. Agron.\textunderscore , I, 18.
\section{Uva-espim}
\begin{itemize}
\item {Grp. gram.:f.}
\end{itemize}
Planta berberídea, (\textunderscore berberis vulgaris\textunderscore ).
\section{Uva-espinha}
\begin{itemize}
\item {Grp. gram.:f.}
\end{itemize}
O mesmo que \textunderscore groselheira\textunderscore .
\section{Uva-gorda}
\begin{itemize}
\item {Grp. gram.:f.}
\end{itemize}
Casta de uva do districto de Leiria.
\section{Uvaia}
\begin{itemize}
\item {Grp. gram.:f.}
\end{itemize}
\begin{itemize}
\item {Utilização:Bras}
\end{itemize}
Fruta da uvaieira.
(Do tupi)
\section{Uvaieira}
\begin{itemize}
\item {Grp. gram.:f.}
\end{itemize}
\begin{itemize}
\item {Utilização:Bras}
\end{itemize}
\begin{itemize}
\item {Proveniência:(De \textunderscore uvaia\textunderscore )}
\end{itemize}
Planta myrtácea da América, (\textunderscore eugenia uvaia\textunderscore ).
\section{Uval}
\begin{itemize}
\item {Grp. gram.:adj.}
\end{itemize}
\begin{itemize}
\item {Grp. gram.:M.}
\end{itemize}
\begin{itemize}
\item {Utilização:Pop.}
\end{itemize}
Relativo á uva.
Tumores hemorrhoidaes.
\section{Uvalha}
\begin{itemize}
\item {Grp. gram.:f.}
\end{itemize}
\begin{itemize}
\item {Proveniência:(De \textunderscore uva\textunderscore )}
\end{itemize}
Planta myrtácea.
\section{Uvalheira}
\begin{itemize}
\item {Grp. gram.:f.}
\end{itemize}
O mesmo que \textunderscore uvalha\textunderscore .
\section{Uva-maçan}
\begin{itemize}
\item {Grp. gram.:f.}
\end{itemize}
Casta de uva de Azeitão. Cf. \textunderscore Rev. Agron.\textunderscore , I, 18.
\section{Uvapiritica}
\begin{itemize}
\item {Grp. gram.:f.}
\end{itemize}
\begin{itemize}
\item {Utilização:Bras}
\end{itemize}
Planta, semelhante ao morangueiro.
\section{Uva-purama}
\begin{itemize}
\item {Grp. gram.:f.}
\end{itemize}
Planta myrtácea, (\textunderscore myrtus racemosa\textunderscore ).
\section{Uva-rara}
\begin{itemize}
\item {Grp. gram.:f.}
\end{itemize}
Casta de uva brasileira.
\section{Uva-rei}
\begin{itemize}
\item {Grp. gram.:f.}
\end{itemize}
Variedade de uva, o mesmo que \textunderscore moirisco\textunderscore .
\section{Uvária}
\begin{itemize}
\item {Grp. gram.:f.}
\end{itemize}
\begin{itemize}
\item {Proveniência:(De \textunderscore uva\textunderscore )}
\end{itemize}
Gênero de plantas anonáceas, cujo fruto é febrífugo.
\section{Uvarovito}
\begin{itemize}
\item {Grp. gram.:m.}
\end{itemize}
\begin{itemize}
\item {Utilização:Miner.}
\end{itemize}
Variedade de granada verde, que se encontra na Sibéria.
\section{Uva-ursina}
\begin{itemize}
\item {Grp. gram.:f.}
\end{itemize}
Planta ericácea, (\textunderscore arctostaphylos uva-ursi\textunderscore , Lin.).
\section{Úvea}
\begin{itemize}
\item {Grp. gram.:f.}
\end{itemize}
\begin{itemize}
\item {Utilização:Ant.}
\end{itemize}
\begin{itemize}
\item {Proveniência:(Do lat. \textunderscore uva\textunderscore )}
\end{itemize}
Conjunto das partes do ôlho, representadas pela choróide, pela íris e pelos processos ciliares.
Choróide, ou a face posterior da íris.
\section{Uvedália}
\begin{itemize}
\item {Grp. gram.:f.}
\end{itemize}
Gênero de plantas escrofularíneas.
\section{Uveira}
\begin{itemize}
\item {Grp. gram.:f.}
\end{itemize}
\begin{itemize}
\item {Proveniência:(De \textunderscore uva\textunderscore )}
\end{itemize}
Árvore, a que se prendem braços de videira.
\section{Uveíta}
\begin{itemize}
\item {Grp. gram.:m.}
\end{itemize}
Membro de uma Ordem de austeros anachoretas muçulmanos, fundada em 667.
\section{Uveíte}
\begin{itemize}
\item {Grp. gram.:f.}
\end{itemize}
Inflammação da úvea.
\section{Uviar}
\begin{itemize}
\item {Grp. gram.:v. i.}
\end{itemize}
\begin{itemize}
\item {Utilização:Prov.}
\end{itemize}
\begin{itemize}
\item {Utilização:trasm.}
\end{itemize}
O mesmo que \textunderscore uivar\textunderscore .
(Metáth. de \textunderscore uivar\textunderscore )
\section{Úvico}
\begin{itemize}
\item {Grp. gram.:adj.}
\end{itemize}
\begin{itemize}
\item {Proveniência:(De \textunderscore uva\textunderscore )}
\end{itemize}
O mesmo que \textunderscore tartárico\textunderscore ^2.
\section{Úvido}
\begin{itemize}
\item {Grp. gram.:adj.}
\end{itemize}
\begin{itemize}
\item {Proveniência:(Lat. \textunderscore uvidus\textunderscore )}
\end{itemize}
O mesmo que \textunderscore húmido\textunderscore .
\section{Uvífero}
\begin{itemize}
\item {Grp. gram.:adj.}
\end{itemize}
\begin{itemize}
\item {Proveniência:(Lat. \textunderscore uvifer\textunderscore )}
\end{itemize}
Que dá frutos semelhantes ao cacho de uvas.
\section{Uviforme}
\begin{itemize}
\item {Grp. gram.:adj.}
\end{itemize}
\begin{itemize}
\item {Proveniência:(Do lat. \textunderscore uva\textunderscore  + \textunderscore forma\textunderscore )}
\end{itemize}
Semelhante a um bago de uva.
\section{Uvre}
\begin{itemize}
\item {Grp. gram.:m.}
\end{itemize}
\begin{itemize}
\item {Utilização:Ant.}
\end{itemize}
O mesmo que \textunderscore úbere\textunderscore . Cf. B. Pereira, \textunderscore Prosódia\textunderscore , vb. \textunderscore sumen\textunderscore .
\section{Úvula}
\begin{itemize}
\item {Grp. gram.:f.}
\end{itemize}
\begin{itemize}
\item {Utilização:Anat.}
\end{itemize}
\begin{itemize}
\item {Proveniência:(Lat. \textunderscore uvula\textunderscore )}
\end{itemize}
Saliência cónica, na parte posterior do véu palatino.
\section{Uvular}
\begin{itemize}
\item {Grp. gram.:adj.}
\end{itemize}
Relativo á úvula.
\section{Uvulária}
\begin{itemize}
\item {Grp. gram.:f.}
\end{itemize}
\begin{itemize}
\item {Proveniência:(De \textunderscore úvula\textunderscore )}
\end{itemize}
Gênero de plantas melantháceas.
\section{Uvulário}
\begin{itemize}
\item {Grp. gram.:adj.}
\end{itemize}
O mesmo que \textunderscore uvular\textunderscore .
\section{Uvuliforme}
\begin{itemize}
\item {Grp. gram.:adj.}
\end{itemize}
\begin{itemize}
\item {Proveniência:(De \textunderscore úvula\textunderscore  + \textunderscore fórma\textunderscore )}
\end{itemize}
Semelhante á úvula.
\section{Uvulite}
\begin{itemize}
\item {Grp. gram.:f.}
\end{itemize}
\begin{itemize}
\item {Utilização:Med.}
\end{itemize}
Inflammação na úvula.
\section{Uxi}
\begin{itemize}
\item {Grp. gram.:m.}
\end{itemize}
Grande árvore rosácea.
Designação de vários frutos silvestres do Norte do Brasil.
\section{Uxim}
\begin{itemize}
\item {Grp. gram.:m.}
\end{itemize}
Variedade de chá.
\section{Uxirana-da-várzea}
\begin{itemize}
\item {Grp. gram.:f.}
\end{itemize}
\begin{itemize}
\item {Utilização:Bras. do N}
\end{itemize}
Árvore para construcções.
\section{Uxoricida}
\begin{itemize}
\item {fónica:cso}
\end{itemize}
\begin{itemize}
\item {Grp. gram.:m.}
\end{itemize}
Aquelle que assassinou sua mulhér. Cf. Camillo, \textunderscore Corja\textunderscore , 167.
(Cp. \textunderscore uxoricídio\textunderscore )
\section{Uxoricídio}
\begin{itemize}
\item {fónica:csi}
\end{itemize}
\begin{itemize}
\item {Grp. gram.:m.}
\end{itemize}
\begin{itemize}
\item {Proveniência:(Do lat. \textunderscore uxor\textunderscore  + \textunderscore caedere\textunderscore )}
\end{itemize}
Assassínio de uma mulhér, commetido por seu marido. Cf. Camillo, \textunderscore Caveira\textunderscore , 472.
\section{Uxório}
\begin{itemize}
\item {fónica:csó}
\end{itemize}
\begin{itemize}
\item {Grp. gram.:adj.}
\end{itemize}
\begin{itemize}
\item {Proveniência:(Lat. \textunderscore uxorius\textunderscore )}
\end{itemize}
Relativo a mulhér casada.
\section{Uxte!}
\begin{itemize}
\item {Grp. gram.:interj.}
\end{itemize}
\begin{itemize}
\item {Utilização:Ant.}
\end{itemize}
Apre! arreda! t'arrenego! Cf. G. Vicente, I, 267.
(É affim de \textunderscore uste\textunderscore ?)
\section{Uzagre}
\begin{itemize}
\item {Grp. gram.:m.}
\end{itemize}
O mesmo ou melhór que \textunderscore usagre\textunderscore .
(Cp. \textunderscore zagre\textunderscore )
\section{Uzífur}
\begin{itemize}
\item {Grp. gram.:m.}
\end{itemize}
Vermelhão, feito de enxôfre e mercúrio.
O mesmo que \textunderscore cinábrio\textunderscore .
\section{Uzífuro}
\begin{itemize}
\item {Grp. gram.:m.}
\end{itemize}
\end{document}