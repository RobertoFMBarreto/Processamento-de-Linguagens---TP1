\documentclass{article}
\usepackage[portuguese]{babel}
\title{R}
\begin{document}
O mesmo que \textunderscore quotiliquê\textunderscore .
\section{Rebalçar}
\begin{itemize}
\item {Grp. gram.:v. i.}
\end{itemize}
Pisar, como na balça de uvas; patinhar:«\textunderscore rebalçavam no charco alguns patos\textunderscore ». Camillo, \textunderscore Rom. de um Homem Rico\textunderscore , 47, (ed. 1890).
\section{Rebaldaria}
\begin{itemize}
\item {Grp. gram.:f.}
\end{itemize}
O mesmo que \textunderscore ribaldaria\textunderscore . Cf. Camillo, \textunderscore Sereia\textunderscore , 165, (ed. 1869).
\section{Recintar}
\begin{itemize}
\item {Grp. gram.:v. t.}
\end{itemize}
\begin{itemize}
\item {Utilização:Prov.}
\end{itemize}
\begin{itemize}
\item {Utilização:minh.}
\end{itemize}
\begin{itemize}
\item {Proveniência:(De \textunderscore re...\textunderscore  + \textunderscore cinta\textunderscore )}
\end{itemize}
\textunderscore Recintar o avental\textunderscore , levantar-lhe as pontas e segurá-las na cinta. (Colhido em Guimarães)
\section{Remanejar}
\begin{itemize}
\item {Grp. gram.:v. t.}
\end{itemize}
Manejar novamente; manejar muito.
\section{Remanejo}
\begin{itemize}
\item {Grp. gram.:m.}
\end{itemize}
Acto ou effeito de remanejar. Cf. R. Jorge, \textunderscore El Greco\textunderscore .
\section{Remanência}
\begin{itemize}
\item {Grp. gram.:f.}
\end{itemize}
Estada; conservação num lugar:«\textunderscore ...a remanência em Toledo...\textunderscore »R. Jorge, \textunderscore El Greco\textunderscore .
\section{R}
\begin{itemize}
\item {fónica:érre}
\end{itemize}
\begin{itemize}
\item {Grp. gram.:m.}
\end{itemize}
Décima oitava letra do alphabeto português.
Abrev. de \textunderscore reprovação\textunderscore .
Abrev. de \textunderscore réu\textunderscore .
Como letra numeral, valia 800 e, com um til, 80:000.
Nas receitas médicas, é abrev. de \textunderscore recipe\textunderscore .
\section{Raba}
\begin{itemize}
\item {Grp. gram.:f.}
\end{itemize}
\begin{itemize}
\item {Utilização:Des.}
\end{itemize}
Espécie de nabo.
\section{Rabaça}
\begin{itemize}
\item {Grp. gram.:f.}
\end{itemize}
\begin{itemize}
\item {Utilização:Fig.}
\end{itemize}
\begin{itemize}
\item {Proveniência:(Do lat. \textunderscore rapacia\textunderscore )}
\end{itemize}
Planta umbellífera.
Pessôa desengraçada.
\section{Rabaçal}
\begin{itemize}
\item {Grp. gram.:m.}
\end{itemize}
Terreno onde crescem rabaças.
\section{Rabaçal}
\begin{itemize}
\item {Grp. gram.:m.}
\end{itemize}
\begin{itemize}
\item {Proveniência:(De \textunderscore Rabaçal\textunderscore , n. p.)}
\end{itemize}
Variedade de queijo nacional.
\section{Rabaçaria}
\begin{itemize}
\item {Grp. gram.:f.}
\end{itemize}
\begin{itemize}
\item {Utilização:Pop.}
\end{itemize}
\begin{itemize}
\item {Utilização:Ant.}
\end{itemize}
\begin{itemize}
\item {Proveniência:(De \textunderscore rabaça\textunderscore )}
\end{itemize}
Porção de fruta de má qualidade.
Hortaliças.
\section{Rabaceiro}
\begin{itemize}
\item {Grp. gram.:adj.}
\end{itemize}
\begin{itemize}
\item {Proveniência:(De \textunderscore rabaça\textunderscore )}
\end{itemize}
Que gosta muito de fruta; que come fruta muito verde ou ordinária; que gosta de hortaliças.
\section{Rabaço}
\begin{itemize}
\item {Grp. gram.:m.}
\end{itemize}
Peixe de Portugal.
\section{Raba-coêlho}
\begin{itemize}
\item {Grp. gram.:m.}
\end{itemize}
\begin{itemize}
\item {Utilização:Prov.}
\end{itemize}
\begin{itemize}
\item {Utilização:extrem.}
\end{itemize}
Pequena ave, semelhante á cordoniz.
O mesmo que \textunderscore ràbiscoêlha\textunderscore ?
\section{Rabada}
\begin{itemize}
\item {Grp. gram.:f.}
\end{itemize}
\begin{itemize}
\item {Utilização:Ant.}
\end{itemize}
\begin{itemize}
\item {Utilização:P. us.}
\end{itemize}
\begin{itemize}
\item {Proveniência:(De \textunderscore rabo\textunderscore )}
\end{itemize}
O mesmo que \textunderscore rabadela\textunderscore .
Trança de cabello, com fita.
Rabicho.
Popa do navio:«\textunderscore no camarote da rabada...\textunderscore »(De um testamento de 1693)
O mesmo que \textunderscore nádega\textunderscore  ou \textunderscore traseiro\textunderscore . Cf. Camillo, \textunderscore Myst. de Lisb.\textunderscore , II, 27.
\section{Rabadão}
\begin{itemize}
\item {Grp. gram.:m.}
\end{itemize}
\begin{itemize}
\item {Proveniência:(Do ár. \textunderscore rabl-ad dan\textunderscore )}
\end{itemize}
Aquelle que guarda gado miúdo.
Maioral de pastores.
Pastor, subordinado ao maioral, mas mandando no zagal.
\section{Rabadão}
\begin{itemize}
\item {Grp. gram.:m.}
\end{itemize}
\begin{itemize}
\item {Utilização:Prov.}
\end{itemize}
\begin{itemize}
\item {Proveniência:(De \textunderscore rabada\textunderscore )}
\end{itemize}
Rabo da vara do lagar.
\section{Rabadela}
\begin{itemize}
\item {Grp. gram.:f.}
\end{itemize}
\begin{itemize}
\item {Utilização:Prov.}
\end{itemize}
\begin{itemize}
\item {Utilização:minh.}
\end{itemize}
\begin{itemize}
\item {Proveniência:(De \textunderscore rabada\textunderscore )}
\end{itemize}
Parte posterior do corpo das aves e mammíferos.
Cauda do peixe.
Cabo, extremo.
\section{Rabadilha}
\begin{itemize}
\item {Grp. gram.:f.}
\end{itemize}
O mesmo e mais usado que \textunderscore rabadela\textunderscore .
(Cast. \textunderscore rabadilla\textunderscore )
\section{Rabado}
\begin{itemize}
\item {Grp. gram.:adj.}
\end{itemize}
Que tem rabo.
\section{Rabadoquim}
\begin{itemize}
\item {Grp. gram.:f.}
\end{itemize}
\begin{itemize}
\item {Utilização:Des.}
\end{itemize}
Antiga e pequena peça de artilharia, espécie de colubrina.
\section{Rabalde}
\begin{itemize}
\item {Grp. gram.:m.}
\end{itemize}
(V.arrabalde)
\section{Rabalha}
\begin{itemize}
\item {Grp. gram.:adj.}
\end{itemize}
\begin{itemize}
\item {Utilização:Ant.}
\end{itemize}
Dizia-se de uma medida para líquidos, usada principalmente no Pôrto.
\section{Rab'alva}
\begin{itemize}
\item {Grp. gram.:f.}
\end{itemize}
Espécie de águia, (\textunderscore haliaetus albicilla\textunderscore ).
(Fem. de \textunderscore rab'alvo\textunderscore )
\section{Rab'alvo}
\begin{itemize}
\item {Grp. gram.:adj.}
\end{itemize}
\begin{itemize}
\item {Proveniência:(De \textunderscore rabo\textunderscore  + \textunderscore alvo\textunderscore )}
\end{itemize}
Que tem o rabo branco.
\section{Rabam}
\begin{itemize}
\item {Grp. gram.:m.}
\end{itemize}
O mesmo que \textunderscore rábano\textunderscore .
\section{Rabana}
\begin{itemize}
\item {Grp. gram.:f.}
\end{itemize}
Atabales, usados no Malabar.
Música indiana.
\section{Rabana}
\begin{itemize}
\item {Grp. gram.:f.}
\end{itemize}
\begin{itemize}
\item {Utilização:Prov.}
\end{itemize}
\begin{itemize}
\item {Utilização:alg.}
\end{itemize}
Rabona; casaco.
\section{Rabanada}
\begin{itemize}
\item {Grp. gram.:f.}
\end{itemize}
\begin{itemize}
\item {Proveniência:(De \textunderscore rabo\textunderscore )}
\end{itemize}
Pancada com o rabo.
\textunderscore Rabanada de vento\textunderscore , pé de vento, corrente de ar, rápida e forte.
\section{Rabanada}
\begin{itemize}
\item {Grp. gram.:f.}
\end{itemize}
Fatia, que, depois de frita em manteiga, se embebe em leite, ovos, etc.
(Cp. cast. \textunderscore rebanada\textunderscore )
\section{Rabanal}
\begin{itemize}
\item {Grp. gram.:m.}
\end{itemize}
Terreno, onde crescem rábanos.
\section{Rabanar}
\textunderscore v. t.\textunderscore  (e der.)
O mesmo que \textunderscore rabunar\textunderscore , etc. Cf. \textunderscore Inquér. Industr.\textunderscore , p. II, l. III, 11, 26 e 40.
\section{Rabanejo}
\begin{itemize}
\item {Grp. gram.:m.}
\end{itemize}
\begin{itemize}
\item {Utilização:Prov.}
\end{itemize}
\begin{itemize}
\item {Utilização:alent.}
\end{itemize}
O mesmo que \textunderscore rabiça\textunderscore ^1.
\section{Rabanete}
\begin{itemize}
\item {fónica:nê}
\end{itemize}
\begin{itemize}
\item {Grp. gram.:m.}
\end{itemize}
\begin{itemize}
\item {Proveniência:(De \textunderscore rábano\textunderscore )}
\end{itemize}
Espécie de rábano, de raíz curta e carnosa.
\section{Rábano}
\begin{itemize}
\item {Grp. gram.:m.}
\end{itemize}
\begin{itemize}
\item {Proveniência:(Do lat. \textunderscore raphanus\textunderscore )}
\end{itemize}
Nome de várias plantas crucíferas.
A raíz dessas plantas.
\section{Rabâno}
\begin{itemize}
\item {Grp. gram.:adj.}
\end{itemize}
\begin{itemize}
\item {Utilização:T. de Turquel}
\end{itemize}
O mesmo que \textunderscore rabão\textunderscore .
\section{Rabão}
\begin{itemize}
\item {Grp. gram.:adj.}
\end{itemize}
\begin{itemize}
\item {Utilização:Gír.}
\end{itemize}
\begin{itemize}
\item {Utilização:Prov.}
\end{itemize}
\begin{itemize}
\item {Utilização:dur.}
\end{itemize}
Que tem o rabo curto ou cortado.
O diabo.
Barco do Doiro, mais pequeno que os rabelos, achatado, pesado e com uma espadela grande e desgraciosa.
\section{Rábão}
\begin{itemize}
\item {Grp. gram.:m.}
\end{itemize}
O mesmo que \textunderscore rábano\textunderscore .
\section{Rabastel}
\begin{itemize}
\item {Grp. gram.:m.}
\end{itemize}
\begin{itemize}
\item {Utilização:T. do Fundão}
\end{itemize}
Pequeno rebanho de ovelhas.
\section{Rab'avento}
\begin{itemize}
\item {Grp. gram.:adj.}
\end{itemize}
\begin{itemize}
\item {Proveniência:(De \textunderscore rabo\textunderscore  + \textunderscore a\textunderscore  + \textunderscore vento\textunderscore )}
\end{itemize}
Que vai na direcção do vento, (falando-se do vôo das aves).
\section{Rabaz}
\begin{itemize}
\item {Grp. gram.:adj.}
\end{itemize}
\begin{itemize}
\item {Grp. gram.:M.}
\end{itemize}
\begin{itemize}
\item {Utilização:Ant.}
\end{itemize}
\begin{itemize}
\item {Proveniência:(Do lat. \textunderscore rapax\textunderscore )}
\end{itemize}
Que arrebata.
Que tira com violência.
Ladrão.
\section{Rabbi}
\begin{itemize}
\item {Grp. gram.:m.}
\end{itemize}
O mesmo que \textunderscore rabbino\textunderscore .
\section{Rabbínico}
\begin{itemize}
\item {Grp. gram.:adj.}
\end{itemize}
Relativo aos Rabbinos.
\section{Rabbinismo}
\begin{itemize}
\item {Grp. gram.:m.}
\end{itemize}
Doutrina dos Rabbinos.
\section{Rabbino}
\begin{itemize}
\item {Grp. gram.:m.}
\end{itemize}
\begin{itemize}
\item {Proveniência:(Do hebr. \textunderscore rabb\textunderscore )}
\end{itemize}
Doutor israelita, o que explica a lei entre os Hebreus.
Sacerdote judaico.
\textunderscore Grão rabbino\textunderscore , chefe supremo de uma sinagoga ou de um consistório israelita.
\section{Rabeador}
\begin{itemize}
\item {Grp. gram.:adj.}
\end{itemize}
Que rabeia.
\section{Rabeadura}
\begin{itemize}
\item {Grp. gram.:f.}
\end{itemize}
Acto de \textunderscore rabear\textunderscore .
\section{Rabear}
\begin{itemize}
\item {Grp. gram.:v. t.}
\end{itemize}
\begin{itemize}
\item {Utilização:Fig.}
\end{itemize}
\begin{itemize}
\item {Grp. gram.:V. t.}
\end{itemize}
\begin{itemize}
\item {Proveniência:(De \textunderscore rabo\textunderscore )}
\end{itemize}
Mexer a cauda.
Mexer-se.
Estar inquieto, por incômmodo ou cuidados.
Irritar-se.
Dirigir (charrua ou arado), segurando-o pela rabiça.
\section{Rabeca}
\begin{itemize}
\item {Grp. gram.:f.}
\end{itemize}
\begin{itemize}
\item {Utilização:Pop.}
\end{itemize}
\begin{itemize}
\item {Grp. gram.:M.}
\end{itemize}
Instrumento de música, com quatro cordas, que se ferem com arco.
Em Náutica, uma das velas latinas, que servem para estais.
Peixe de Portugal.
Utensílio, em que o jogador de bilhar apoia o taco, para impellir uma bola afastada.
Enxêrga de palha.
Utensílio de ferreiro, feito de um arco de aço, ligado nas extremidades por um cordel, e destinado a fazer girar a broca.
Tocador de rabeca.
(Fórma divergente de \textunderscore rebeca\textunderscore . V. \textunderscore rebeca\textunderscore )
\section{Rabecada}
\begin{itemize}
\item {Grp. gram.:f.}
\end{itemize}
\begin{itemize}
\item {Utilização:Fam.}
\end{itemize}
\begin{itemize}
\item {Utilização:Fig.}
\end{itemize}
\begin{itemize}
\item {Proveniência:(De \textunderscore rabeca\textunderscore )}
\end{itemize}
Acto de tocar rabeca.
Reprehensão, censura.
Murmuração ou diffamação em conversa íntima.
\section{Rabecão}
\begin{itemize}
\item {Grp. gram.:m.}
\end{itemize}
\begin{itemize}
\item {Utilização:Bras}
\end{itemize}
\begin{itemize}
\item {Utilização:Fam.}
\end{itemize}
\begin{itemize}
\item {Proveniência:(De \textunderscore rabeca\textunderscore )}
\end{itemize}
Instrumento músico, do feitio da rabeca, mas muito maior.
Designação do contrabaixo.
Carro fúnebre dos indigentes.
\textunderscore Tocar rabecão\textunderscore , dizer mal, diffamar.
\section{Rabeco}
\begin{itemize}
\item {Grp. gram.:m.}
\end{itemize}
\begin{itemize}
\item {Utilização:Prov.}
\end{itemize}
\begin{itemize}
\item {Utilização:dur.}
\end{itemize}
Barqueiro do Alto Doiro.
(Cp. \textunderscore rabelo\textunderscore )
\section{Rabeco}
\begin{itemize}
\item {Grp. gram.:m.}
\end{itemize}
Livro inútil, alfarrábio?:«\textunderscore ...se a minha livraria, em lugar de borrões meus e alguns rabeccos de má morte, possuira bons autores...\textunderscore »Filinto, XIII, 214.
\section{Rabeira}
\begin{itemize}
\item {Grp. gram.:f.}
\end{itemize}
\begin{itemize}
\item {Utilização:Pesc.}
\end{itemize}
\begin{itemize}
\item {Utilização:Prov.}
\end{itemize}
\begin{itemize}
\item {Utilização:alent.}
\end{itemize}
\begin{itemize}
\item {Utilização:Prov.}
\end{itemize}
\begin{itemize}
\item {Utilização:trasm.}
\end{itemize}
\begin{itemize}
\item {Utilização:Prov.}
\end{itemize}
\begin{itemize}
\item {Utilização:trasm.}
\end{itemize}
\begin{itemize}
\item {Proveniência:(De \textunderscore rabo\textunderscore )}
\end{itemize}
Rasto, peugada.
Pragana, moínha e quaesquer restos, que ficam dos cereaes, depois de joeirados.
Rêde das armações fixas de pesca, que se dirige para o lado da terra.
Cauda de vestido.
Lama ou sujidade, na parte inferior de um vestido.
Parte saliente, na traseira do carro.
Resto do grão, que fica na tremonha.
Corda, presa ao cabresto do cavallo, para o guiar ou prender.
\section{Rabeirada}
\begin{itemize}
\item {Grp. gram.:f.}
\end{itemize}
\begin{itemize}
\item {Utilização:Prov.}
\end{itemize}
\begin{itemize}
\item {Utilização:trasm.}
\end{itemize}
Pancada com o \textunderscore rabeiro\textunderscore , (rédea).
\section{Rabeiro}
\begin{itemize}
\item {Grp. gram.:m.}
\end{itemize}
\begin{itemize}
\item {Utilização:Prov.}
\end{itemize}
\begin{itemize}
\item {Utilização:trasm.}
\end{itemize}
\begin{itemize}
\item {Utilização:T. do Ribatejo}
\end{itemize}
O mesmo que \textunderscore rabeira\textunderscore .
A rédea das bêstas.
Aquelle que nas lavras segura o rabo da charrua.
\section{Rabeja}
\begin{itemize}
\item {Grp. gram.:f.}
\end{itemize}
\begin{itemize}
\item {Utilização:Prov.}
\end{itemize}
\begin{itemize}
\item {Utilização:alent.}
\end{itemize}
\begin{itemize}
\item {Proveniência:(De \textunderscore rabejar\textunderscore )}
\end{itemize}
Tosquia local da lan suja, que possa estorvar a ordenha. Cf. Ficalho, Rev. \textunderscore Tradição\textunderscore , IX, 130.
\section{Rabejador}
\begin{itemize}
\item {Grp. gram.:m.  e  adj.}
\end{itemize}
O que rabeja.
\section{Rabejar}
\begin{itemize}
\item {Grp. gram.:v. t.}
\end{itemize}
\begin{itemize}
\item {Utilização:Prov.}
\end{itemize}
\begin{itemize}
\item {Utilização:alent.}
\end{itemize}
\begin{itemize}
\item {Grp. gram.:V. i.}
\end{itemize}
Segurar pelo rabo (um toiro).
Tosquiar parcialmente a lan suja de (ovelhas), para se facilitar a ordenha.
Arrastar o vestido pelo chão, andando.
\section{Rabel}
\begin{itemize}
\item {Grp. gram.:m.}
\end{itemize}
O mesmo que \textunderscore rabil\textunderscore .
\section{Rabela}
\begin{itemize}
\item {Grp. gram.:f.}
\end{itemize}
\begin{itemize}
\item {Utilização:Prov.}
\end{itemize}
\begin{itemize}
\item {Utilização:trasm.}
\end{itemize}
\begin{itemize}
\item {Proveniência:(De \textunderscore rabo\textunderscore )}
\end{itemize}
Toda a parte posterior do arado, desde a relha á rabiça.
\section{Rabelaico}
\begin{itemize}
\item {Grp. gram.:adj.}
\end{itemize}
O mesmo que \textunderscore rabelesiano\textunderscore . Cf. Garrett, \textunderscore Fábulas\textunderscore , XXIII.
\section{Rabelaisiano}
\begin{itemize}
\item {Grp. gram.:adj.}
\end{itemize}
O mesmo que \textunderscore rabelesiano\textunderscore . Cf. Camillo, \textunderscore Vinho do Pôrto\textunderscore , 63.
\section{Rabelesiano}
\begin{itemize}
\item {Grp. gram.:adj.}
\end{itemize}
\begin{itemize}
\item {Proveniência:(De \textunderscore Rabelais\textunderscore , n. p.)}
\end{itemize}
Relativo a Rabelais; parecido ás personagens ou ao carácter literário de Rabelais. Cf. Camillo, \textunderscore Cancion. Al.\textunderscore , 505.
\section{Rabelho}
\begin{itemize}
\item {fónica:bê}
\end{itemize}
\begin{itemize}
\item {Grp. gram.:m.}
\end{itemize}
Rabiça; o mesmo que \textunderscore rabelo\textunderscore .
\section{Rabelo}
\begin{itemize}
\item {fónica:bê}
\end{itemize}
\begin{itemize}
\item {Grp. gram.:m.}
\end{itemize}
\begin{itemize}
\item {Proveniência:(De \textunderscore rabo\textunderscore )}
\end{itemize}
Rabiça.
Corda, com que o lavrador segura a rabiça.
Embarcação do Doiro, de leme comprido.
Aquelle que dirige essa embarcação.
\section{Rabelo-coêlha}
\begin{itemize}
\item {Grp. gram.:f.}
\end{itemize}
O mesmo que \textunderscore rabita\textunderscore .
\section{Rabendo}
\begin{itemize}
\item {Grp. gram.:m.}
\end{itemize}
Árvore indiana, de fibras têxteis.
\section{Rabequista}
\begin{itemize}
\item {Grp. gram.:m.  e  f.}
\end{itemize}
Pessôa, que toca rabeca.
\section{Rabeta}
\begin{itemize}
\item {fónica:bê}
\end{itemize}
\begin{itemize}
\item {Grp. gram.:f.}
\end{itemize}
\begin{itemize}
\item {Utilização:Prov.}
\end{itemize}
\begin{itemize}
\item {Utilização:Gír.}
\end{itemize}
\begin{itemize}
\item {Grp. gram.:Adj.}
\end{itemize}
\begin{itemize}
\item {Proveniência:(De \textunderscore rabo\textunderscore )}
\end{itemize}
Espécie de alvéola, (\textunderscore ruticilla tithys\textunderscore ).
Rapariga esperta.
Charuto de picar.
Diz-se de uma espécie de formiga muito pequena.
\section{Rabi}
\begin{itemize}
\item {Grp. gram.:m.}
\end{itemize}
O mesmo que \textunderscore rabino\textunderscore ^1.
\section{Rábia}
\begin{itemize}
\item {Grp. gram.:f.}
\end{itemize}
\begin{itemize}
\item {Proveniência:(Lat. \textunderscore rabies\textunderscore )}
\end{itemize}
O mesmo que \textunderscore hydrophobia\textunderscore .
\section{Rabía}
\begin{itemize}
\item {Grp. gram.:f.}
\end{itemize}
Espécie de jôgo popular.
\section{Ràbialva}
\begin{itemize}
\item {Grp. gram.:adj. f.}
\end{itemize}
\begin{itemize}
\item {Utilização:Prov.}
\end{itemize}
\begin{itemize}
\item {Utilização:trasm.}
\end{itemize}
Diz-se de uma variedade de formiga. (Colhido em Lagoaça)
\section{Ràbialvo}
\begin{itemize}
\item {Grp. gram.:adj.}
\end{itemize}
O mesmo que \textunderscore rab'alvo\textunderscore .
\section{Rabiar}
\begin{itemize}
\item {Grp. gram.:v. i.}
\end{itemize}
\begin{itemize}
\item {Utilização:Fam.}
\end{itemize}
Têr raiva; enfurecer-se.
(Cast. \textunderscore rabiar\textunderscore , de \textunderscore rabia\textunderscore , raiva)
\section{Rabiça}
\begin{itemize}
\item {Grp. gram.:f.}
\end{itemize}
\begin{itemize}
\item {Proveniência:(De \textunderscore rabo\textunderscore )}
\end{itemize}
Rabo do arado.
Eminência na parte posterior das albardas.
\section{Rabiça}
\begin{itemize}
\item {Grp. gram.:f.}
\end{itemize}
\begin{itemize}
\item {Utilização:Prov.}
\end{itemize}
\begin{itemize}
\item {Utilização:trasm.}
\end{itemize}
O mesmo que \textunderscore nabiça\textunderscore .
(Cp. gall. \textunderscore rabiça\textunderscore )
\section{Rabicão}
\begin{itemize}
\item {Grp. gram.:adj.}
\end{itemize}
\begin{itemize}
\item {Proveniência:(De \textunderscore rabo\textunderscore  + lat. \textunderscore canus\textunderscore )}
\end{itemize}
Que tem a cauda malhada de branco, (falando-se do cavallo).
\section{Rabicha}
\begin{itemize}
\item {Grp. gram.:f.}
\end{itemize}
\begin{itemize}
\item {Utilização:Prov.}
\end{itemize}
\begin{itemize}
\item {Utilização:alent.}
\end{itemize}
\begin{itemize}
\item {Utilização:Bras. de Minas}
\end{itemize}
\begin{itemize}
\item {Proveniência:(De \textunderscore rabo\textunderscore )}
\end{itemize}
A parte saliente da traseira do carro, o mesmo que \textunderscore rabeira\textunderscore .
Corrente ou tira de coiro, em que se prendem os caldeirões sôbre a trempe, nos ranchos ou povoados pobres.
\section{Rabichano}
\begin{itemize}
\item {Grp. gram.:m.}
\end{itemize}
\begin{itemize}
\item {Utilização:T. de Turquel}
\end{itemize}
O mesmo que \textunderscore rabichão\textunderscore .
\section{Rabichão}
\begin{itemize}
\item {Grp. gram.:adj.}
\end{itemize}
\begin{itemize}
\item {Proveniência:(De \textunderscore rabicho\textunderscore )}
\end{itemize}
O mesmo que \textunderscore rabão\textunderscore .
\section{Rabicheira}
\begin{itemize}
\item {Grp. gram.:f.}
\end{itemize}
\begin{itemize}
\item {Utilização:Prov.}
\end{itemize}
\begin{itemize}
\item {Proveniência:(De \textunderscore rabicho\textunderscore )}
\end{itemize}
Parte dos arreios dos muares, que passa por baixo da cauda e se prende á parte dos arreios que cinge o ventre.
\section{Rabicho}
\begin{itemize}
\item {Grp. gram.:m.}
\end{itemize}
\begin{itemize}
\item {Utilização:Bras}
\end{itemize}
\begin{itemize}
\item {Grp. gram.:Adj.}
\end{itemize}
\begin{itemize}
\item {Proveniência:(De \textunderscore rabo\textunderscore )}
\end{itemize}
Pequena trança de cabello, pendente da nuca.
Parte dos arreios da cavalgura, que passa por baixo da cauda e se prendendo[**typo de 'prende'?] á sella.
Extremidade posterior da pírtiga.
Retranca.
Cabo do almanjarra.
Amor, paixão.
Namôro.
Diz-se do toiro, que não tem pêlo na extremidade da cauda.
\section{Rábico}
\begin{itemize}
\item {Grp. gram.:adj.}
\end{itemize}
Relativo á rábia ou á hydrophobia.
\section{Rabiço}
\begin{itemize}
\item {Grp. gram.:m.}
\end{itemize}
\begin{itemize}
\item {Utilização:Prov.}
\end{itemize}
\begin{itemize}
\item {Utilização:beir.}
\end{itemize}
O mesmo que \textunderscore rabiça\textunderscore ^1 (do arado).
Variedade de pêra de Penafiel.
\section{Ràbicoêlha}
\begin{itemize}
\item {Grp. gram.:f.}
\end{itemize}
O mesmo que \textunderscore rabila\textunderscore . Cf. P. Moraes, \textunderscore Zool. Elem.\textunderscore , 390.
\section{Rabiçola}
\begin{itemize}
\item {Grp. gram.:f.}
\end{itemize}
\begin{itemize}
\item {Utilização:Prov.}
\end{itemize}
\begin{itemize}
\item {Proveniência:(De \textunderscore rabiça\textunderscore )}
\end{itemize}
Fruta ordinária, que se deixa nas árvores depois da colheita.
Planta enfèzada.
\section{Rabiçolo}
\begin{itemize}
\item {Grp. gram.:m.}
\end{itemize}
(V.rabiçola)
\section{Ràbicurto}
\begin{itemize}
\item {Grp. gram.:adj.}
\end{itemize}
\begin{itemize}
\item {Grp. gram.:M.}
\end{itemize}
\begin{itemize}
\item {Proveniência:(De \textunderscore rabo\textunderscore  + \textunderscore curto\textunderscore )}
\end{itemize}
Que tem cauda curta.
Pássaro conirostro, da fam. dos corvos.
\section{Rábido}
\begin{itemize}
\item {Grp. gram.:adj.}
\end{itemize}
\begin{itemize}
\item {Proveniência:(Lat. \textunderscore rabidus\textunderscore )}
\end{itemize}
Que tem raiva; furioso.
\section{Ràbifurcado}
\begin{itemize}
\item {Grp. gram.:adj.}
\end{itemize}
\begin{itemize}
\item {Proveniência:(De \textunderscore rabo\textunderscore  + lat. \textunderscore furca\textunderscore )}
\end{itemize}
Que tem a cauda bifurcada.
\section{Ràbigato}
\begin{itemize}
\item {Grp. gram.:m.}
\end{itemize}
Espécie de uva branca, o mesmo que \textunderscore rabo-de-gato\textunderscore .
\section{Rabigo}
\begin{itemize}
\item {Grp. gram.:adj.}
\end{itemize}
\begin{itemize}
\item {Utilização:Fig.}
\end{itemize}
\begin{itemize}
\item {Proveniência:(De \textunderscore rabo\textunderscore )}
\end{itemize}
Que move muito a cauda.
Activo; diligente.
\section{Ràbijunco}
\begin{itemize}
\item {Grp. gram.:m.}
\end{itemize}
\begin{itemize}
\item {Proveniência:(De \textunderscore rabo\textunderscore  + \textunderscore junco\textunderscore )}
\end{itemize}
Ave palmípede, o mesmo que \textunderscore arrabio\textunderscore .
\section{Rabil}
\begin{itemize}
\item {Grp. gram.:m.}
\end{itemize}
O mesmo que \textunderscore arrabil\textunderscore .
\section{Rabila}
\begin{itemize}
\item {Grp. gram.:f.}
\end{itemize}
\begin{itemize}
\item {Proveniência:(De \textunderscore rabo\textunderscore )}
\end{itemize}
Ave pernalta, (\textunderscore gallinula chloropus\textunderscore ).
\section{Rabileiro}
\begin{itemize}
\item {Grp. gram.:m.}
\end{itemize}
Tangedor de rabil.
\section{Rabilha}
\begin{itemize}
\item {Grp. gram.:f.}
\end{itemize}
O mesmo que \textunderscore rabila\textunderscore .
\section{Rabilongo}
\begin{itemize}
\item {Grp. gram.:adj.}
\end{itemize}
\begin{itemize}
\item {Grp. gram.:M.}
\end{itemize}
\begin{itemize}
\item {Utilização:Prov.}
\end{itemize}
\begin{itemize}
\item {Proveniência:(De \textunderscore rabo\textunderscore  + \textunderscore longo\textunderscore )}
\end{itemize}
Que tem cauda longa.
Espécie de pêga, também conhecida por pêga azul, (\textunderscore pica cyanea\textunderscore , Pall.).
O mesmo que \textunderscore megengra\textunderscore .
\section{Rabinho}
\begin{itemize}
\item {Grp. gram.:m.}
\end{itemize}
\begin{itemize}
\item {Utilização:Prov.}
\end{itemize}
\begin{itemize}
\item {Proveniência:(De \textunderscore rabo\textunderscore )}
\end{itemize}
Pedaço: \textunderscore já é um rabinho de um homem\textunderscore .
\section{Rabinice}
\begin{itemize}
\item {Grp. gram.:f.}
\end{itemize}
Qualidade ou acto de quem é rabino; travessura; rabugice.
\section{Rabínico}
\begin{itemize}
\item {Grp. gram.:adj.}
\end{itemize}
Relativo aos Rabinos.
\section{Rabinismo}
\begin{itemize}
\item {Grp. gram.:m.}
\end{itemize}
Doutrina dos Rabinos.
\section{Rabino}
\begin{itemize}
\item {Grp. gram.:m.}
\end{itemize}
\begin{itemize}
\item {Proveniência:(Do hebr. \textunderscore rabb\textunderscore )}
\end{itemize}
Doutor israelita, o que explica a lei entre os Hebreus.
Sacerdote judaico.
\textunderscore Grão rabino\textunderscore , chefe supremo de uma sinagoga ou de um consistório israelita.
\section{Rabino}
\begin{itemize}
\item {Grp. gram.:adj.}
\end{itemize}
\begin{itemize}
\item {Utilização:Fam.}
\end{itemize}
\begin{itemize}
\item {Proveniência:(De \textunderscore rábia\textunderscore ?)}
\end{itemize}
Rabugento; travêsso; buliçoso.
\section{Rabinostre}
\begin{itemize}
\item {Grp. gram.:m.}
\end{itemize}
\begin{itemize}
\item {Utilização:T. de Turquel}
\end{itemize}
O mesmo que \textunderscore rabioste\textunderscore .
\section{Rabiolo}
\begin{itemize}
\item {Proveniência:(Fr. \textunderscore rabiole\textunderscore )}
\end{itemize}
O mesmo que \textunderscore rabanete\textunderscore . Cf. Castilho, \textunderscore Avarento\textunderscore , III, 5.
\section{Rabiosca}
\begin{itemize}
\item {Grp. gram.:f.}
\end{itemize}
\begin{itemize}
\item {Utilização:Prov.}
\end{itemize}
\begin{itemize}
\item {Utilização:alent.}
\end{itemize}
\begin{itemize}
\item {Utilização:Prov.}
\end{itemize}
\begin{itemize}
\item {Utilização:minh.}
\end{itemize}
\begin{itemize}
\item {Grp. gram.:Pl.}
\end{itemize}
\begin{itemize}
\item {Utilização:Prov.}
\end{itemize}
\begin{itemize}
\item {Utilização:minh.}
\end{itemize}
O mesmo que \textunderscore rabiosque\textunderscore .
O mesmo que \textunderscore rodeio\textunderscore .
Armadilha.
Ratoeira.
Letras mal feitas, gatafunhos.
\section{Rabioso}
\begin{itemize}
\item {Grp. gram.:adj.}
\end{itemize}
\begin{itemize}
\item {Proveniência:(Lat. \textunderscore rabiosus\textunderscore )}
\end{itemize}
Rábido; irritado.
\section{Rabiosque}
\begin{itemize}
\item {Grp. gram.:m.}
\end{itemize}
O mesmo que \textunderscore rabioste\textunderscore .
\section{Rabioste}
\begin{itemize}
\item {Grp. gram.:m.}
\end{itemize}
\begin{itemize}
\item {Utilização:Pop.}
\end{itemize}
Rabo, nádegas.
\section{Rabiote}
\begin{itemize}
\item {Grp. gram.:m.}
\end{itemize}
\begin{itemize}
\item {Utilização:Fam.}
\end{itemize}
O mesmo que \textunderscore rabioste\textunderscore .
\section{Ràbipreto}
\begin{itemize}
\item {Grp. gram.:adj.}
\end{itemize}
\begin{itemize}
\item {Proveniência:(De \textunderscore rabo\textunderscore  + \textunderscore preto\textunderscore )}
\end{itemize}
Que tem cauda preta.
\section{Rabirruiva}
\begin{itemize}
\item {Grp. gram.:f.}
\end{itemize}
Ave, o mesmo que \textunderscore rabeta\textunderscore .
\section{Rabirruivo}
\begin{itemize}
\item {Grp. gram.:adj.}
\end{itemize}
\begin{itemize}
\item {Grp. gram.:M.}
\end{itemize}
\begin{itemize}
\item {Proveniência:(De \textunderscore rabo\textunderscore  + \textunderscore ruivo\textunderscore )}
\end{itemize}
Que tem cauda ruiva.
Ave, o mesmo que \textunderscore rabeta\textunderscore .
\section{Rabiruiva}
\begin{itemize}
\item {fónica:rui}
\end{itemize}
\begin{itemize}
\item {Grp. gram.:f.}
\end{itemize}
Ave, o mesmo que \textunderscore rabeta\textunderscore .
\section{Rabiruivo}
\begin{itemize}
\item {fónica:rui}
\end{itemize}
\begin{itemize}
\item {Grp. gram.:adj.}
\end{itemize}
\begin{itemize}
\item {Grp. gram.:M.}
\end{itemize}
\begin{itemize}
\item {Proveniência:(De \textunderscore rabo\textunderscore  + \textunderscore ruivo\textunderscore )}
\end{itemize}
Que tem cauda ruiva.
Ave, o mesmo que \textunderscore rabeta\textunderscore .
\section{Rabisaca}
\begin{itemize}
\item {fónica:sa}
\end{itemize}
\begin{itemize}
\item {Grp. gram.:f.}
\end{itemize}
\begin{itemize}
\item {Utilização:ant.}
\end{itemize}
\begin{itemize}
\item {Utilização:Pop.}
\end{itemize}
\begin{itemize}
\item {Proveniência:(De \textunderscore rabo\textunderscore  + \textunderscore sacar\textunderscore )}
\end{itemize}
Digressão feita, ás escondidas ou á pressa. Cf. Filinto, I, 287, (nota).
\section{Rabisaltão}
\begin{itemize}
\item {fónica:sal}
\end{itemize}
\begin{itemize}
\item {Grp. gram.:adj.}
\end{itemize}
\begin{itemize}
\item {Utilização:P. us.}
\end{itemize}
\begin{itemize}
\item {Proveniência:(De \textunderscore rabo\textunderscore  + \textunderscore saltar\textunderscore )}
\end{itemize}
Que se saracoteia.
\section{Rabisaltona}
\begin{itemize}
\item {fónica:sal}
\end{itemize}
\begin{itemize}
\item {Grp. gram.:adj. f.}
\end{itemize}
\begin{itemize}
\item {Proveniência:(De \textunderscore rabisaltão\textunderscore )}
\end{itemize}
Diz-se da mulhér que se saracoteia, que é sirigaita.
\section{Rabisca}
\begin{itemize}
\item {Grp. gram.:f.}
\end{itemize}
\begin{itemize}
\item {Grp. gram.:Pl.}
\end{itemize}
\begin{itemize}
\item {Utilização:Prov.}
\end{itemize}
\begin{itemize}
\item {Utilização:beir.}
\end{itemize}
Traço mal feito, garatuja.
Letras mal escritas.
Escrita de pouca importância.
Desenho insignificante.
Pequenas dividas, (por allusão aos traços, com que os taberneiros analphabetos e outros vendedores registam os seus pequenos créditos, a giz ou á penna, no tampo das vasilhas, ou em ardósias, etc.).
(Cp. \textunderscore rabisco\textunderscore ^1)
\section{Rabiscadeira}
\begin{itemize}
\item {Grp. gram.:f.}
\end{itemize}
Mulhér, que rabisca.
\section{Rabiscador}
\begin{itemize}
\item {Grp. gram.:m.  e  adj.}
\end{itemize}
\begin{itemize}
\item {Utilização:Deprec.}
\end{itemize}
O que rabisca.
Escritor reles.
\section{Rabiscar}
\begin{itemize}
\item {Grp. gram.:v. i.}
\end{itemize}
\begin{itemize}
\item {Grp. gram.:V. t.}
\end{itemize}
\begin{itemize}
\item {Proveniência:(De \textunderscore rabisca\textunderscore )}
\end{itemize}
Traçar rabiscos; fazer garatujas.
Escrever ou desenhar mal.
Cobrir de rabiscas; escrevinhar: \textunderscore rabiscar um papel\textunderscore .
\section{Rabiscar}
\begin{itemize}
\item {Grp. gram.:v. t.}
\end{itemize}
(Corr. de \textunderscore rebuscar\textunderscore ) Cf. Castilho, \textunderscore Escav. Poét.\textunderscore , 14.
\section{Rabisco}
\begin{itemize}
\item {Grp. gram.:m.}
\end{itemize}
O mesmo que \textunderscore rabisca\textunderscore .
(Talvez de \textunderscore arabisco\textunderscore , por \textunderscore arabesco\textunderscore )
\section{Rabisco}
\begin{itemize}
\item {Grp. gram.:m.}
\end{itemize}
(Corr. de \textunderscore rebusco\textunderscore )
\section{Ràbiscoêlha}
\begin{itemize}
\item {Grp. gram.:f.}
\end{itemize}
O mesmo que \textunderscore rabila\textunderscore .
\section{Rabiseco}
\begin{itemize}
\item {fónica:sê}
\end{itemize}
\begin{itemize}
\item {Grp. gram.:adj.}
\end{itemize}
\begin{itemize}
\item {Proveniência:(De \textunderscore rabo\textunderscore  + \textunderscore sêco\textunderscore )}
\end{itemize}
Que não dá fruto; entanguido; estéril.
\section{Rabissaca}
\begin{itemize}
\item {Grp. gram.:f.}
\end{itemize}
\begin{itemize}
\item {Utilização:ant.}
\end{itemize}
\begin{itemize}
\item {Utilização:Pop.}
\end{itemize}
\begin{itemize}
\item {Proveniência:(De \textunderscore rabo\textunderscore  + \textunderscore sacar\textunderscore )}
\end{itemize}
Digressão feita, ás escondidas ou á pressa. Cf. Filinto, I, 287, (nota).
\section{Rabissaltão}
\begin{itemize}
\item {Grp. gram.:adj.}
\end{itemize}
\begin{itemize}
\item {Utilização:P. us.}
\end{itemize}
\begin{itemize}
\item {Proveniência:(De \textunderscore rabo\textunderscore  + \textunderscore saltar\textunderscore )}
\end{itemize}
Que se saracoteia.
\section{Rabissaltona}
\begin{itemize}
\item {Grp. gram.:adj. f.}
\end{itemize}
\begin{itemize}
\item {Proveniência:(De \textunderscore rabissaltão\textunderscore )}
\end{itemize}
Diz-se da mulhér que se saracoteia, que é sirigaita.
\section{Rabisseco}
\begin{itemize}
\item {fónica:sê}
\end{itemize}
\begin{itemize}
\item {Grp. gram.:adj.}
\end{itemize}
\begin{itemize}
\item {Proveniência:(De \textunderscore rabo\textunderscore  + \textunderscore sêco\textunderscore )}
\end{itemize}
Que não dá fruto; entanguido; estéril.
\section{Rabisteco}
\begin{itemize}
\item {Grp. gram.:m.}
\end{itemize}
\begin{itemize}
\item {Proveniência:(De \textunderscore rabo\textunderscore )}
\end{itemize}
Nádegas de criança: \textunderscore cala-te; se não, levas dois açoites no rabistel\textunderscore .
\section{Rabistel}
\begin{itemize}
\item {Grp. gram.:m.}
\end{itemize}
\begin{itemize}
\item {Utilização:Fam.}
\end{itemize}
\begin{itemize}
\item {Proveniência:(De \textunderscore rabo\textunderscore )}
\end{itemize}
Nádegas de criança: \textunderscore cala-te; se não, levas dois açoites no rabistel\textunderscore .
\section{Rabita}
\begin{itemize}
\item {Grp. gram.:f.}
\end{itemize}
\begin{itemize}
\item {Utilização:Prov.}
\end{itemize}
\begin{itemize}
\item {Utilização:trasm.}
\end{itemize}
\begin{itemize}
\item {Proveniência:(De \textunderscore rabo\textunderscore )}
\end{itemize}
Ave, o mesmo que \textunderscore rabeta\textunderscore .
Variedade de pêra.
Colhér, com que se deita sopa nos pratos.
\section{Rabita-ferreira}
\begin{itemize}
\item {Grp. gram.:f.}
\end{itemize}
O mesmo que \textunderscore rabeta\textunderscore .
\section{Rabo}
\begin{itemize}
\item {Grp. gram.:m.}
\end{itemize}
\begin{itemize}
\item {Utilização:Pleb.}
\end{itemize}
\begin{itemize}
\item {Grp. gram.:Loc. prep.}
\end{itemize}
\begin{itemize}
\item {Utilização:Prov.}
\end{itemize}
\begin{itemize}
\item {Utilização:beir.}
\end{itemize}
\begin{itemize}
\item {Proveniência:(Do lat. \textunderscore rapum\textunderscore )}
\end{itemize}
Prolongamento extero-inferior da columna vertebral de vários animais.
Cauda.
Grupo de pennas, que nascem do uropýgio das aves.
Parte do corpo dos reptis, peixes e animaes de feitio análogo, prolongada além do ânus.
Parte saliente de certos utensílios, pela qual se seguram com a mão: \textunderscore o rabo da enxada\textunderscore .
O mesmo que \textunderscore nádega\textunderscore .
\textunderscore Têr rabo de palha\textunderscore , sêr notado ou conhecido por algum acto indigno.
\textunderscore A rabo de\textunderscore , atrás de: \textunderscore o patife, para me provocar, andou todo o dia a rabo de mim\textunderscore .
\section{Rabo-aberto}
\begin{itemize}
\item {Grp. gram.:m.}
\end{itemize}
Nome de um peixe brasileiro.
\section{Rabo-branco}
\begin{itemize}
\item {Grp. gram.:m.}
\end{itemize}
Pássaro dentirostro, (\textunderscore saxicola leucura\textunderscore , Lin.).
\section{Ràbocoêlha}
\begin{itemize}
\item {Grp. gram.:f.}
\end{itemize}
O mesmo que \textunderscore rabila\textunderscore .
\section{Rabo-de-andorinha}
\begin{itemize}
\item {Grp. gram.:f.}
\end{itemize}
Tardoz de pedra, ou ponta de madeira em fórma de leque, para engatar e ficar segura numa parede ou viga.
\section{Rabo-de-asno}
\begin{itemize}
\item {Grp. gram.:m.}
\end{itemize}
Casta de uva branca.
Designação antiga de uma planta aquática, a que se refere Pero Vaz de Caminha, descrevendo o descobrimento do Brasil.
\section{Rabo-de-bugio}
\begin{itemize}
\item {Grp. gram.:m.}
\end{itemize}
\begin{itemize}
\item {Utilização:Bras}
\end{itemize}
O mesmo que \textunderscore samambaia\textunderscore .
\section{Rabo-de-cão}
\begin{itemize}
\item {Grp. gram.:m.}
\end{itemize}
Gênero de plantas gramíneas, (\textunderscore cynosurus\textunderscore , Lin.).
\section{Rabo-de-colhér}
\begin{itemize}
\item {Grp. gram.:m.}
\end{itemize}
\begin{itemize}
\item {Utilização:T. de Penafiel}
\end{itemize}
O mesmo que \textunderscore megengra\textunderscore .
\section{Rabo-de-espada}
\begin{itemize}
\item {Grp. gram.:m.}
\end{itemize}
\begin{itemize}
\item {Utilização:Ant.}
\end{itemize}
Uma das fórmas que se davam ao graminho, em construcções náuticas. Cf. Fern. Oliveira, \textunderscore Fábr. das Naus\textunderscore .
\section{Rabo-de-foguete}
\begin{itemize}
\item {Grp. gram.:m.}
\end{itemize}
\begin{itemize}
\item {Utilização:T. da Bairrada}
\end{itemize}
Pequeno pássaro, de longa cauda.
\section{Rabo-de-gato}
\begin{itemize}
\item {Grp. gram.:m.}
\end{itemize}
Casta de uva, na região do Doiro.--Na Beira, é termo de comparação: \textunderscore azêdo, como rabo-de-gato\textunderscore .
\section{Rabo-de-junco}
\begin{itemize}
\item {Grp. gram.:m.}
\end{itemize}
Passarinho, de rabo comprido e estreito, em Angola; o mesmo que \textunderscore rabijunco\textunderscore ?
\section{Rabo-de-lebre}
\begin{itemize}
\item {Grp. gram.:m.}
\end{itemize}
Casta de uva, o mesmo que \textunderscore trincadeira\textunderscore .
Planta gramínea, (\textunderscore lugurus\textunderscore , Lin.).
\section{Rabo-de-leque}
\begin{itemize}
\item {Grp. gram.:m.}
\end{itemize}
\begin{itemize}
\item {Utilização:Constr.}
\end{itemize}
Degrau, que é mais largo de um lado que do outro, como nas escadas de caracol. Cf. L. Assumpção, \textunderscore Diccion. de Archit.\textunderscore 
\section{Rabo-de-macaco}
\begin{itemize}
\item {Grp. gram.:m.}
\end{itemize}
\begin{itemize}
\item {Utilização:Bras}
\end{itemize}
Árvore silvestre, applicável em carpintaria.
Planta forraginosa, (\textunderscore cynosurus crisiatus\textunderscore , Lin.).
\section{Rabo-de-minhoto}
\begin{itemize}
\item {Grp. gram.:m.}
\end{itemize}
\begin{itemize}
\item {Utilização:Constr.}
\end{itemize}
Peça, com secção de fórma aproximada da do leque.
\section{Rabo-de-ovelha}
\begin{itemize}
\item {Grp. gram.:m.}
\end{itemize}
Casta de uva, na região do Doiro e na Beira.
\section{Rabo-de-palha}
\begin{itemize}
\item {Grp. gram.:m.}
\end{itemize}
\begin{itemize}
\item {Utilização:Fig.}
\end{itemize}
Mancha na reputação.
Motivo de censura.
\section{Rabo-de-raposa}
\begin{itemize}
\item {Grp. gram.:f.}
\end{itemize}
\begin{itemize}
\item {Utilização:Bras}
\end{itemize}
\begin{itemize}
\item {Utilização:Náut.}
\end{itemize}
Planta medicinal do Brasil, que, esmagada, se applica em fricções nalgumas doenças cutâneas.
Planta gramínea, (\textunderscore alopecurus\textunderscore , Lin.).
Planta labiada, (\textunderscore stachis ocymastrum\textunderscore , Lin.).
Obra, que os marinheiros fazem nos chicotes de alguns cabos, com fio de vela ou de carrêta, para maior asseio.
\section{Rabo-de-tatu}
\begin{itemize}
\item {Grp. gram.:m.}
\end{itemize}
\begin{itemize}
\item {Utilização:Bras. do S}
\end{itemize}
Rebenque de coiro entrançado.
\section{Ràboleva}
\begin{itemize}
\item {Grp. gram.:m.}
\end{itemize}
\begin{itemize}
\item {Proveniência:(De \textunderscore rabo\textunderscore  + \textunderscore levar\textunderscore )}
\end{itemize}
Brincadeira de Carnaval, que consiste num pedaço de trapo ou papel, que se colloca subrepticiamente nas costas de alguém, para depois o motejar.
\section{Rabona}
\begin{itemize}
\item {Grp. gram.:f.}
\end{itemize}
\begin{itemize}
\item {Utilização:Burl.}
\end{itemize}
\begin{itemize}
\item {Utilização:Prov.}
\end{itemize}
\begin{itemize}
\item {Utilização:trasm.}
\end{itemize}
\begin{itemize}
\item {Proveniência:(De \textunderscore rabão\textunderscore )}
\end{itemize}
Jaquetão; casaco curto; casaca.
Enxada de cabo curto.
\section{Rabonar}
\begin{itemize}
\item {Grp. gram.:v. t.}
\end{itemize}
\begin{itemize}
\item {Utilização:Bras. do S}
\end{itemize}
Cortar o rabo a (um animal).
\section{Rabo-ruço}
\begin{itemize}
\item {Grp. gram.:m.}
\end{itemize}
Ave, o mesmo que \textunderscore rabeta\textunderscore .
\section{Rabo-ruivo}
\begin{itemize}
\item {Grp. gram.:m.}
\end{itemize}
Ave, o mesmo que \textunderscore pisco-ferreiro\textunderscore .
\section{Raboso}
\begin{itemize}
\item {Grp. gram.:adj.}
\end{itemize}
\begin{itemize}
\item {Proveniência:(De \textunderscore rabo\textunderscore )}
\end{itemize}
Que tem cauda grande.
\section{Rabotar}
\begin{itemize}
\item {Grp. gram.:v. t.}
\end{itemize}
Alisar com o rabote.
\section{Rabote}
\begin{itemize}
\item {Grp. gram.:m.}
\end{itemize}
\begin{itemize}
\item {Proveniência:(Do fr. \textunderscore rabot\textunderscore )}
\end{itemize}
Grande plaina de carpinteiro.
\section{Raboto}
\begin{itemize}
\item {fónica:bô}
\end{itemize}
\begin{itemize}
\item {Grp. gram.:adj.}
\end{itemize}
\begin{itemize}
\item {Utilização:Prov.}
\end{itemize}
\begin{itemize}
\item {Utilização:trasm.}
\end{itemize}
A que falta um braço ou manga: \textunderscore homem raboto\textunderscore ; \textunderscore camisa rabota\textunderscore .
\section{Rabo-torto}
\begin{itemize}
\item {Grp. gram.:m.}
\end{itemize}
\begin{itemize}
\item {Utilização:Bras. do Maranhão}
\end{itemize}
O mesmo que \textunderscore lacrau\textunderscore .
\section{Rabucho}
\begin{itemize}
\item {Grp. gram.:adj.}
\end{itemize}
\begin{itemize}
\item {Utilização:Prov.}
\end{itemize}
\begin{itemize}
\item {Utilização:trasm.}
\end{itemize}
Diz-se do cão ou do cavallo, que tem o rabo cortado.
\section{Rabudo}
\begin{itemize}
\item {Grp. gram.:adj.}
\end{itemize}
\begin{itemize}
\item {Utilização:Burl.}
\end{itemize}
\begin{itemize}
\item {Grp. gram.:M.}
\end{itemize}
\begin{itemize}
\item {Utilização:Prov.}
\end{itemize}
\begin{itemize}
\item {Utilização:trasm.}
\end{itemize}
\begin{itemize}
\item {Proveniência:(De \textunderscore rabo\textunderscore )}
\end{itemize}
Que tem rabo; raboso.
Que tem grande cauda, falando-se do vestuário.
Espécie de cogumelo azul e venenoso.
\section{Rabuge}
\begin{itemize}
\item {Grp. gram.:f.}
\end{itemize}
\begin{itemize}
\item {Utilização:Fig.}
\end{itemize}
\begin{itemize}
\item {Utilização:Bras}
\end{itemize}
\begin{itemize}
\item {Proveniência:(Do rad. do lat. \textunderscore rabies\textunderscore )}
\end{itemize}
Doença dos cães e dos porcos, espécie de sarna.
Má disposição do espirito.
Mau humor.
Madeira revessa e diffícil de lavrar.
\section{Rabugeira}
\begin{itemize}
\item {Grp. gram.:f.}
\end{itemize}
O mesmo que \textunderscore rabuge\textunderscore .
\section{Rabugem}
\begin{itemize}
\item {Grp. gram.:f.}
\end{itemize}
O mesmo que \textunderscore rabuge\textunderscore .
\section{Rabugento}
\begin{itemize}
\item {Grp. gram.:adj.}
\end{itemize}
\begin{itemize}
\item {Utilização:Fig.}
\end{itemize}
Que tem rabuge.
Mal disposto, mal humorado.
\section{Rabugice}
\begin{itemize}
\item {Grp. gram.:f.}
\end{itemize}
\begin{itemize}
\item {Proveniência:(De \textunderscore rabuge\textunderscore )}
\end{itemize}
Qualidade do que é rabugento; acto próprio de rabugento.
\section{Rabujado}
\begin{itemize}
\item {Grp. gram.:adj.}
\end{itemize}
\begin{itemize}
\item {Proveniência:(De \textunderscore rabujar\textunderscore )}
\end{itemize}
Pronunciado por entre dentes, com mau humor: \textunderscore respostas rabujadas\textunderscore .
\section{Rabujão}
\begin{itemize}
\item {Grp. gram.:m.}
\end{itemize}
Grande porção de rabuge ou sarna dos porcos.
\section{Rabujar}
\begin{itemize}
\item {Grp. gram.:v. i.}
\end{itemize}
\begin{itemize}
\item {Proveniência:(De \textunderscore rabuge\textunderscore )}
\end{itemize}
Têr rabugice.
Sêr teimoso e impertinente, (falando-se especialmente de crianças).
\section{Rabujaria}
\begin{itemize}
\item {Grp. gram.:f.}
\end{itemize}
\begin{itemize}
\item {Utilização:Des.}
\end{itemize}
O mesmo que \textunderscore rabugice\textunderscore .
\section{Rábula}
\begin{itemize}
\item {Grp. gram.:m.}
\end{itemize}
\begin{itemize}
\item {Proveniência:(Lat. \textunderscore rabula\textunderscore )}
\end{itemize}
Advogado, que fala muito, embaraçando as questões com os artifícios que a lei lhe faculta.
Homem muito falador, que não chega ás conclusões do seu arrazoado.
\section{Rabulão}
\begin{itemize}
\item {Grp. gram.:m.}
\end{itemize}
Grande rábula.
\section{Rabular}
\begin{itemize}
\item {Grp. gram.:v. i.}
\end{itemize}
Proceder como rábula; dizer ou fazer rabulices.
\section{Rabularia}
\begin{itemize}
\item {Grp. gram.:f.}
\end{itemize}
\begin{itemize}
\item {Proveniência:(De \textunderscore rábula\textunderscore )}
\end{itemize}
Rabulice; palanfrório.
Fanfarronada.
\section{Rabulice}
\begin{itemize}
\item {Grp. gram.:f.}
\end{itemize}
Acto ou dito de rábula; chicana.
\section{Rabulista}
\begin{itemize}
\item {Grp. gram.:m.  e  adj.}
\end{itemize}
Aquelle que é dado á rabulice; chicaneiro. Cf. Camillo, \textunderscore Noites de Insómn.\textunderscore , X, 37.
\section{Rabunador}
\begin{itemize}
\item {Grp. gram.:m.}
\end{itemize}
Official, encarregado de rabunar a cortiça.
\section{Rabunar}
\begin{itemize}
\item {Grp. gram.:v. t.}
\end{itemize}
Cortar em tiras (a cortiça cozida e raspada), para depois se cortarem essas tiras em quadrados, de que se fazem rolhas. Cf. \textunderscore Inquér. Industr.\textunderscore , III, 49.
(Cp. \textunderscore rabunhar\textunderscore )
\section{Rabunhar}
\begin{itemize}
\item {Grp. gram.:v. t.}
\end{itemize}
\begin{itemize}
\item {Utilização:Prov.}
\end{itemize}
\begin{itemize}
\item {Utilização:minh.}
\end{itemize}
O mesmo que \textunderscore rapar\textunderscore . (Colhido em Barcelos)
(Cp. gall. \textunderscore rabuñar\textunderscore )
\section{Rabusca}
\begin{itemize}
\item {Grp. gram.:f.}
\end{itemize}
(Corr. de \textunderscore rebusca\textunderscore ) Cf. Pato, \textunderscore Livro do Monte\textunderscore , 14.
\section{Raca}
\begin{itemize}
\item {Grp. gram.:adj.}
\end{itemize}
\begin{itemize}
\item {Grp. gram.:M.}
\end{itemize}
Palavra chaldaica, empregada no Evangelho como termo injurioso. Cf. Herculano, \textunderscore Bobo\textunderscore .
Homem sandeu, sem juizo. Cf. Duarte Nunes de Leão, \textunderscore Or. da Língua Port.\textunderscore , 193.
\section{Raça}
\begin{itemize}
\item {Grp. gram.:f.}
\end{itemize}
\begin{itemize}
\item {Proveniência:(It. \textunderscore razza\textunderscore )}
\end{itemize}
Conjunto dos indivíduos, que procedem da mesma família ou do mesmo tronco: \textunderscore a raça humana\textunderscore .
Origem; geração: \textunderscore raça nobre\textunderscore .
Conjunto de indivíduos, que conservam entre si, e através das gerações, relações de semelhança.
Cada uma das variedades da espécie humana ou de qualquer espécie de animaes: \textunderscore a raça branca\textunderscore .
Classe; espécie.
Variedade.
Estirpe; casta.
Qualidade.
\section{Raça}
\begin{itemize}
\item {Grp. gram.:f.}
\end{itemize}
\begin{itemize}
\item {Utilização:Prov.}
\end{itemize}
\textunderscore Raça de sol\textunderscore , réstia de sol ou feixe de luz, que entra pelos buracos dos telhados e das paredes.
(Cp. \textunderscore ressa\textunderscore )
\section{Raçada}
\begin{itemize}
\item {Grp. gram.:f.}
\end{itemize}
\begin{itemize}
\item {Utilização:Prov.}
\end{itemize}
\begin{itemize}
\item {Utilização:trasm.}
\end{itemize}
\begin{itemize}
\item {Proveniência:(De \textunderscore raça\textunderscore ^2)}
\end{itemize}
Golpe de sol, bastante forte, saíndo de nuvens entre-abertas.
\section{Racama}
\begin{itemize}
\item {Grp. gram.:f.}
\end{itemize}
Gênero de aves de rapina, cuja única espécie vive em Angola.
\section{Ração}
\begin{itemize}
\item {Grp. gram.:f.}
\end{itemize}
\begin{itemize}
\item {Proveniência:(Do lat. \textunderscore ratio\textunderscore )}
\end{itemize}
Porção de alimento, calculado para a refeição de um homem ou para o seu consumo diário.
Comedorias, dadas a um indivíduo por dia, semana ou mês.
Porção de palha ou de outras substâncias, applicada em cada refeição de certos animaes.
A quarta parte, que o rendeiro pagava, dos frutos de uma terra ao seu senhorio.
\section{Raças}
\begin{itemize}
\item {Grp. gram.:f. pl.}
\end{itemize}
Fenda no casco do cavallo.
(Corr. de \textunderscore rachas\textunderscore ?)
\section{Racemato}
\begin{itemize}
\item {Grp. gram.:m.}
\end{itemize}
Sal, resultante do ácido racêmico com as bases. Cf. \textunderscore Techn. Rur.\textunderscore , 20.
\section{Racêmico}
\begin{itemize}
\item {Grp. gram.:adj.}
\end{itemize}
\begin{itemize}
\item {Proveniência:(Do lat. \textunderscore racemus\textunderscore )}
\end{itemize}
Diz-se de um ácido, que se encontra em algumas espécies de uvas de Itália, Áustria e Hungria.
\section{Racemífero}
\begin{itemize}
\item {Grp. gram.:adj.}
\end{itemize}
\begin{itemize}
\item {Proveniência:(Lat. \textunderscore racemifer\textunderscore )}
\end{itemize}
Que tem cachos de uvas. Cf. Castilho, \textunderscore Fastos\textunderscore , III, 147.
\section{Racemoso}
\begin{itemize}
\item {Grp. gram.:adj.}
\end{itemize}
\begin{itemize}
\item {Utilização:Geol.}
\end{itemize}
\begin{itemize}
\item {Proveniência:(Do lat. \textunderscore racemus\textunderscore )}
\end{itemize}
Diz-se das concreções, que offerecem, grosseiramente, o aspecto de um cacho de uvas. Cf. Gonç. Guimarães, \textunderscore Geol.\textunderscore , 62.
\section{Racha}
\begin{itemize}
\item {Grp. gram.:f.}
\end{itemize}
\begin{itemize}
\item {Utilização:Prov.}
\end{itemize}
\begin{itemize}
\item {Utilização:trasm.}
\end{itemize}
\begin{itemize}
\item {Utilização:Prov.}
\end{itemize}
\begin{itemize}
\item {Utilização:trasm.}
\end{itemize}
\begin{itemize}
\item {Utilização:Prov.}
\end{itemize}
\begin{itemize}
\item {Utilização:minh.}
\end{itemize}
\begin{itemize}
\item {Grp. gram.:Loc.}
\end{itemize}
\begin{itemize}
\item {Utilização:pop.}
\end{itemize}
\begin{itemize}
\item {Proveniência:(De \textunderscore rachar\textunderscore )}
\end{itemize}
Fenda; abertura.
Estilhaço; lasca, separada por effeito de fractura ou ruptura.
Parte, quinhão.
Cavaco de lenha.
Lasca de bacalhau.
\textunderscore Saiu o pau á racha\textunderscore , o filho saiu ao pai; tal pai, tal filho. Cf. Camillo, \textunderscore Volcoens\textunderscore , 55.
\section{Rachadeira}
\begin{itemize}
\item {Grp. gram.:f.}
\end{itemize}
\begin{itemize}
\item {Proveniência:(De \textunderscore rachar\textunderscore )}
\end{itemize}
Instrumento, com que se abrem os ramos em que se faz enxertia.
\section{Rachador}
\begin{itemize}
\item {Grp. gram.:m.  e  adj.}
\end{itemize}
O que racha.
Lenhador.
\section{Rachadura}
\begin{itemize}
\item {Grp. gram.:f.}
\end{itemize}
Acto ou effeito de rachar.
\section{Rachão}
\begin{itemize}
\item {Grp. gram.:m.}
\end{itemize}
\begin{itemize}
\item {Utilização:Prov.}
\end{itemize}
\begin{itemize}
\item {Utilização:minh.}
\end{itemize}
\begin{itemize}
\item {Proveniência:(De \textunderscore racha\textunderscore )}
\end{itemize}
Segmento de tronco, rachado longitudinalmente.
Cavaca.
\section{Rachar}
\begin{itemize}
\item {Grp. gram.:v. t.}
\end{itemize}
\begin{itemize}
\item {Utilização:Prov.}
\end{itemize}
\begin{itemize}
\item {Grp. gram.:V. i.}
\end{itemize}
\begin{itemize}
\item {Proveniência:(Do lat. hyp. \textunderscore ras'culare\textunderscore )}
\end{itemize}
Separar as partes de, abrindo fenda.
Fender.
Partir com violência pelo meio; partir em estilhaços.
O mesmo que \textunderscore rachear\textunderscore .
Fender-se.
\section{Rache}
\begin{itemize}
\item {fónica:que}
\end{itemize}
\begin{itemize}
\item {Grp. gram.:f.}
\end{itemize}
O mesmo ou melhor que \textunderscore ráchis\textunderscore .
\section{Rachear}
\begin{itemize}
\item {Grp. gram.:v. t.}
\end{itemize}
\begin{itemize}
\item {Utilização:Prov.}
\end{itemize}
\begin{itemize}
\item {Proveniência:(De \textunderscore racha\textunderscore )}
\end{itemize}
Preencher com rachas ou lascas de pedra e com argamassa os vãos de (parede em construcção).
\section{Rachedo}
\begin{itemize}
\item {fónica:chê}
\end{itemize}
\begin{itemize}
\item {Grp. gram.:m.}
\end{itemize}
\begin{itemize}
\item {Utilização:Prov.}
\end{itemize}
\begin{itemize}
\item {Proveniência:(De \textunderscore racha\textunderscore )}
\end{itemize}
Porção de rachas ou lascas de pedra, com que o pedreiro preenche os vãos entre os rebos da parede.
\section{Rachel}
\begin{itemize}
\item {fónica:quel}
\end{itemize}
\begin{itemize}
\item {Grp. gram.:f.}
\end{itemize}
\begin{itemize}
\item {Proveniência:(De \textunderscore Rachel\textunderscore , n. p.)}
\end{itemize}
Planta amaryllídea, (\textunderscore amaryllis sarniensis\textunderscore ).
\section{Rachialgia}
\begin{itemize}
\item {fónica:qui}
\end{itemize}
\begin{itemize}
\item {Grp. gram.:f.}
\end{itemize}
\begin{itemize}
\item {Proveniência:(Do gr. \textunderscore rakhis\textunderscore  + \textunderscore algos\textunderscore )}
\end{itemize}
Dôr aguda em qualquer ponto da espinha dorsal.
\section{Rachicêntese}
\begin{itemize}
\item {fónica:qui}
\end{itemize}
\begin{itemize}
\item {Grp. gram.:f.}
\end{itemize}
\begin{itemize}
\item {Utilização:Cir.}
\end{itemize}
\begin{itemize}
\item {Proveniência:(Do gr. \textunderscore rakhis\textunderscore  + \textunderscore kentesis\textunderscore )}
\end{itemize}
Puncção lombar, operação que consiste na introducção de uma agulha ou trocarte fino no canal rachidiano, para introduzir algum medicamento ou retirar um pouco de líquido cephálico ou rachidiano.
\section{Rachidiano}
\begin{itemize}
\item {fónica:qui}
\end{itemize}
\begin{itemize}
\item {Grp. gram.:adj.}
\end{itemize}
\begin{itemize}
\item {Proveniência:(De \textunderscore ráchis\textunderscore )}
\end{itemize}
Relativo á espinha dorsal.
\section{Ráchis}
\begin{itemize}
\item {fónica:quis}
\end{itemize}
\begin{itemize}
\item {Grp. gram.:f.}
\end{itemize}
\begin{itemize}
\item {Utilização:Anat.}
\end{itemize}
\begin{itemize}
\item {Utilização:Bot.}
\end{itemize}
\begin{itemize}
\item {Proveniência:(Do gr. \textunderscore rakhis\textunderscore )}
\end{itemize}
Columna vertebral.
Eixo central da espiga das gramíneas.
\section{Rachisagra}
\begin{itemize}
\item {fónica:qui}
\end{itemize}
\begin{itemize}
\item {Grp. gram.:f.}
\end{itemize}
\begin{itemize}
\item {Utilização:Med.}
\end{itemize}
\begin{itemize}
\item {Utilização:Ant.}
\end{itemize}
\begin{itemize}
\item {Proveniência:(Do gr. \textunderscore rakhis\textunderscore  + \textunderscore agra\textunderscore )}
\end{itemize}
Doença de gota no espinhaço.
\section{Rachítico}
\begin{itemize}
\item {fónica:qui}
\end{itemize}
\begin{itemize}
\item {Grp. gram.:adj.}
\end{itemize}
\begin{itemize}
\item {Grp. gram.:M.}
\end{itemize}
\begin{itemize}
\item {Proveniência:(Do gr. \textunderscore rakhitis\textunderscore )}
\end{itemize}
Que tem rachitismo.
Indivíduo rachítico; indivíduo enfèzado.
\section{Rachitismo}
\begin{itemize}
\item {fónica:qui}
\end{itemize}
\begin{itemize}
\item {Grp. gram.:m.}
\end{itemize}
\begin{itemize}
\item {Utilização:Med.}
\end{itemize}
\begin{itemize}
\item {Utilização:Bot.}
\end{itemize}
\begin{itemize}
\item {Utilização:Fig.}
\end{itemize}
\begin{itemize}
\item {Proveniência:(Do gr. \textunderscore rakhitis\textunderscore )}
\end{itemize}
Perturbação mórbida da nutrição dos tecidos, tendo como resultado a imperfeição ou a suspensão do desenvolvimento do organismo, a deformação do thórax e da ráchis ou do resto do systema ossoso.
Definhamento ou deformação das plantas, tornando-se a haste curta ou nodosa.
Fraqueza das faculdades intellectuaes ou do senso moral.
\section{Racimado}
\begin{itemize}
\item {Grp. gram.:adj.}
\end{itemize}
\begin{itemize}
\item {Proveniência:(Do lat. \textunderscore racematus\textunderscore )}
\end{itemize}
Que tem cachos.
Que tem grãos, dispostos em fórma de cachos.
\section{Racímico}
\begin{itemize}
\item {Grp. gram.:adj.}
\end{itemize}
\begin{itemize}
\item {Proveniência:(De \textunderscore racimo\textunderscore )}
\end{itemize}
Diz-se de um ácido isomérico com o ácido tartárico, mas distinto dêste pelas suas propriedades ópticas e pela sua crystallização.--É outra fórma de \textunderscore racêmico\textunderscore .
\section{Racimífero}
\begin{itemize}
\item {Grp. gram.:adj.}
\end{itemize}
\begin{itemize}
\item {Utilização:Poét.}
\end{itemize}
(Outra fórma de \textunderscore racemífero\textunderscore . V. \textunderscore racemífero\textunderscore )
\section{Racimifloro}
\begin{itemize}
\item {Grp. gram.:adj.}
\end{itemize}
\begin{itemize}
\item {Proveniência:(De \textunderscore racimo\textunderscore  + \textunderscore flôr\textunderscore )}
\end{itemize}
Cujas flôres têm fórma de racimo.
\section{Racimiforme}
\begin{itemize}
\item {Grp. gram.:adj.}
\end{itemize}
\begin{itemize}
\item {Proveniência:(De \textunderscore racimo\textunderscore  + \textunderscore fórma\textunderscore )}
\end{itemize}
Que tem fórma de cacho.
Semelhante ao cacho.
\section{Racimo}
\begin{itemize}
\item {Grp. gram.:m.}
\end{itemize}
\begin{itemize}
\item {Proveniência:(Do lat. \textunderscore racemus\textunderscore .--O radical dêste latim apparece nas línguas novi-latinas, incluida a portuguesa, como se vê em \textunderscore racemoso\textunderscore , etc. Mas em português o mesmo radical é transmudado na fórma \textunderscore racim...\textunderscore , como se vê em \textunderscore racimo\textunderscore , \textunderscore racimifero\textunderscore , etc., e mantêm-se os dois radicaes. Note-se, porém, que Vieira e outros clássicos conheciam \textunderscore racimo\textunderscore , e não documentam o radical \textunderscore racem...\textunderscore )}
\end{itemize}
Cacho de uvas.
Fruto ou flôr, em fórma de cacho de uvas.
\section{Racimoso}
\begin{itemize}
\item {Grp. gram.:adj.}
\end{itemize}
\begin{itemize}
\item {Proveniência:(De \textunderscore racimo\textunderscore )}
\end{itemize}
Cheio de cachos; que tem a apparencia de cacho.
\section{Raciocinação}
\begin{itemize}
\item {Grp. gram.:f.}
\end{itemize}
\begin{itemize}
\item {Proveniência:(Lat. \textunderscore ratiocinatio\textunderscore )}
\end{itemize}
Acto ou effeito de raciocinar.
\section{Raciocinador}
\begin{itemize}
\item {Grp. gram.:m.  e  adj.}
\end{itemize}
\begin{itemize}
\item {Proveniência:(Lat. \textunderscore ratiocinator\textunderscore )}
\end{itemize}
Aquelle que raciocina.
\section{Raciocinante}
\begin{itemize}
\item {Grp. gram.:adj.}
\end{itemize}
Que raciocina.
\section{Raciocinar}
\begin{itemize}
\item {Grp. gram.:v. i.}
\end{itemize}
\begin{itemize}
\item {Proveniência:(Lat. \textunderscore ratiocinari\textunderscore )}
\end{itemize}
Fazer uso da razão; fazer raciocínios.
Ponderar; apresentar razões.
Fazer cálculos.
\section{Raciocinativo}
\begin{itemize}
\item {Grp. gram.:adj.}
\end{itemize}
\begin{itemize}
\item {Proveniência:(Lat. \textunderscore ratiocionativus\textunderscore )}
\end{itemize}
Relativo ao raciocínio; em que há raciocínio.
\section{Raciocínio}
\begin{itemize}
\item {Grp. gram.:m.}
\end{itemize}
\begin{itemize}
\item {Proveniência:(Lat. \textunderscore ratiocinium\textunderscore )}
\end{itemize}
Acto ou faculdade de raciocinar.
Encadeamento de argumentos.
Operação intellectual, com que se deduz de uma ou mais premissas um juízo.
Juízo.
Ponderação, objecção.
\section{Racionabilidade}
\begin{itemize}
\item {Grp. gram.:f.}
\end{itemize}
\begin{itemize}
\item {Proveniência:(Lat. \textunderscore rationabilitas\textunderscore )}
\end{itemize}
Qualidade do que é racional.
Faculdade de raciocinar.
\section{Racional}
\begin{itemize}
\item {Grp. gram.:adj.}
\end{itemize}
\begin{itemize}
\item {Grp. gram.:M.}
\end{itemize}
\begin{itemize}
\item {Proveniência:(Lat. \textunderscore rationalis\textunderscore )}
\end{itemize}
Que faz uso da razão; que raciocina: \textunderscore animal racional\textunderscore .
Que se concebe pela razão: \textunderscore fundamento racional\textunderscore .
Conforme á razão.
Diz-se da quantidade mathemática, que é commensurável.
O sêr que faz uso da razão, que pensa.
O homem, em geral.
Peça dos paramentos dos sacerdotes judeus.
\section{Racionalidade}
\begin{itemize}
\item {Grp. gram.:f.}
\end{itemize}
\begin{itemize}
\item {Proveniência:(Lat. \textunderscore rationalitas\textunderscore )}
\end{itemize}
O mesmo que \textunderscore racionabilidade\textunderscore .
\section{Racionalismo}
\begin{itemize}
\item {Grp. gram.:m.}
\end{itemize}
\begin{itemize}
\item {Proveniência:(De \textunderscore racional\textunderscore )}
\end{itemize}
Maneira de considerar as coisas só á luz da razão, independentemente de qualquer autoridade.
\section{Racionalista}
\begin{itemize}
\item {Grp. gram.:adj.}
\end{itemize}
\begin{itemize}
\item {Grp. gram.:M.  e  f.}
\end{itemize}
\begin{itemize}
\item {Proveniência:(De \textunderscore racional\textunderscore )}
\end{itemize}
Relativo ao racionalismo.
Pessôa, que é partidária do racionalismo.
\section{Racionalização}
\begin{itemize}
\item {Grp. gram.:f.}
\end{itemize}
Acto ou effeito de racionalizar.
\section{Racionalizar}
\begin{itemize}
\item {Grp. gram.:v. t.}
\end{itemize}
\begin{itemize}
\item {Proveniência:(Do lat. \textunderscore ratio\textunderscore )}
\end{itemize}
Tornar racional; tornar reflexivo; fazer meditar:«\textunderscore ...procurei racionalizar o meu espirito.\textunderscore »Castilho, \textunderscore Montalverne\textunderscore .
\section{Racionalmente}
\begin{itemize}
\item {Grp. gram.:adv.}
\end{itemize}
De modo racional; com reflexão; com tino, com juizo.
\section{Racionar}
\begin{itemize}
\item {Grp. gram.:v. t.}
\end{itemize}
\begin{itemize}
\item {Utilização:Bras. do S}
\end{itemize}
Dar ração a (cavallo), em hora certa.
\section{Racionável}
\begin{itemize}
\item {Grp. gram.:adj.}
\end{itemize}
\begin{itemize}
\item {Proveniência:(Do lat. \textunderscore rationabilis\textunderscore )}
\end{itemize}
O mesmo que \textunderscore razoável\textunderscore .
\section{Racionavelmente}
\begin{itemize}
\item {Grp. gram.:adv.}
\end{itemize}
De modo racionável:«\textunderscore não entendão os pays de família que peccão todas as vezes que se agastão racionavelmente com os que tem a seu cuidado.\textunderscore »\textunderscore Luz e Calor\textunderscore , 287.
\section{Racioneiro}
\begin{itemize}
\item {Grp. gram.:adj.}
\end{itemize}
\begin{itemize}
\item {Proveniência:(Do b. lat. \textunderscore rationarius\textunderscore )}
\end{itemize}
O mesmo que \textunderscore raçoeiro\textunderscore .
\section{Raçoar-se}
\begin{itemize}
\item {Grp. gram.:v. p.}
\end{itemize}
\begin{itemize}
\item {Utilização:Ant.}
\end{itemize}
Pagar cada qual a ração de frutos, que é obrigado a pagar, como pensão ou tributo.
\section{Raçoeiro}
\begin{itemize}
\item {Grp. gram.:adj.}
\end{itemize}
\begin{itemize}
\item {Grp. gram.:M.}
\end{itemize}
\begin{itemize}
\item {Utilização:Ant.}
\end{itemize}
\begin{itemize}
\item {Proveniência:(De \textunderscore ração\textunderscore )}
\end{itemize}
A quem pertence uma ração ou que a recebe.
Que dá ração. Cf. Camillo, \textunderscore Cav. em Ruínas\textunderscore , 133.
Espécie de beneficiado, em corporação religiosa.
\section{Raconto}
\begin{itemize}
\item {Grp. gram.:m.}
\end{itemize}
\begin{itemize}
\item {Utilização:Des.}
\end{itemize}
\begin{itemize}
\item {Proveniência:(It. \textunderscore racconto\textunderscore )}
\end{itemize}
O mesmo que \textunderscore narração\textunderscore .
\section{Rada}
\begin{itemize}
\item {Grp. gram.:f.}
\end{itemize}
Enseada ou pôrto, abrigado por terras mais ou menos elevadas: \textunderscore as ilhotas que limitam a rada de Macau\textunderscore .
(Cast. \textunderscore rada\textunderscore )
\section{Radar}
\textunderscore v. t.\textunderscore  (e der.) \textunderscore Ant.\textunderscore 
O mesmo que \textunderscore redrar\textunderscore , etc.
\section{Rádia}
\begin{itemize}
\item {Grp. gram.:f.}
\end{itemize}
\begin{itemize}
\item {Proveniência:(Lat. \textunderscore radia\textunderscore )}
\end{itemize}
Variedade de azeitona, conhecida dos antigos.
\section{Radiação}
\begin{itemize}
\item {Grp. gram.:f.}
\end{itemize}
\begin{itemize}
\item {Proveniência:(Lat. \textunderscore radiatio\textunderscore )}
\end{itemize}
Acto ou effeito de radiar.
\section{Radiado}
\begin{itemize}
\item {Grp. gram.:adj.}
\end{itemize}
\begin{itemize}
\item {Grp. gram.:M. pl.}
\end{itemize}
\begin{itemize}
\item {Proveniência:(De \textunderscore radiar\textunderscore )}
\end{itemize}
Disposto á maneira de raios; que tem estrias, partindo de um centro.
O mesmo que \textunderscore radiários\textunderscore .
\section{Radial}
\begin{itemize}
\item {Grp. gram.:adj.}
\end{itemize}
Relativo ao rádio.
Que emitte raios.
\section{Radiante}
\begin{itemize}
\item {Grp. gram.:adj.}
\end{itemize}
\begin{itemize}
\item {Proveniência:(Lat. \textunderscore radians\textunderscore )}
\end{itemize}
Que radia; que brilha.
Bello; esplêndido.
\section{Radiantemente}
\begin{itemize}
\item {Grp. gram.:adv.}
\end{itemize}
De modo radiante.
\section{Radiar}
\begin{itemize}
\item {Grp. gram.:v. i.}
\end{itemize}
\begin{itemize}
\item {Grp. gram.:V. t.}
\end{itemize}
\begin{itemize}
\item {Proveniência:(Lat. \textunderscore radiare\textunderscore )}
\end{itemize}
Emittir raios de luz ou calor; refulgir; resplandecer.
* Emittir radiosamente:«\textunderscore ...radiando alegria dos olhos.\textunderscore »Camillo, \textunderscore Caveira\textunderscore , 402.
Cercar de raios brilhantes; aureolar.
\section{Radiários}
\begin{itemize}
\item {Grp. gram.:m. pl.}
\end{itemize}
\begin{itemize}
\item {Utilização:Zool.}
\end{itemize}
\begin{itemize}
\item {Proveniência:(Do lat. \textunderscore radius\textunderscore )}
\end{itemize}
Classe de animaes não vertebrados, cujos órgãos estão dispostos como em volta de um eixo.
\section{Radicação}
\begin{itemize}
\item {Grp. gram.:f.}
\end{itemize}
\begin{itemize}
\item {Proveniência:(Lat. \textunderscore radicatio\textunderscore )}
\end{itemize}
Acto ou effeito de radicar.
\section{Radical}
\begin{itemize}
\item {Grp. gram.:adj.}
\end{itemize}
\begin{itemize}
\item {Grp. gram.:M.}
\end{itemize}
\begin{itemize}
\item {Utilização:Gram.}
\end{itemize}
\begin{itemize}
\item {Utilização:Chím.}
\end{itemize}
\begin{itemize}
\item {Proveniência:(Do lat. \textunderscore radix\textunderscore )}
\end{itemize}
Relativo á raiz.
Essencial, fundamental: \textunderscore defeitos radicaes\textunderscore .
Partidário do radicalismo.
Parte essencial ou invariável de uma palavra.
Sinal mathemático, que se põe atrás de uma quantidade, a que se há de extrahir a raiz.
Corpo, que, combinado com outro, entra na composição de um ácido ou de uma base.
Indivíduo, que segue o radicalismo.
\section{Radicalismo}
\begin{itemize}
\item {Grp. gram.:m.}
\end{itemize}
\begin{itemize}
\item {Proveniência:(De \textunderscore radical\textunderscore )}
\end{itemize}
Systema dos partidários da reforma profunda da sociedade política.
\section{Radicalista}
\begin{itemize}
\item {Grp. gram.:adj.}
\end{itemize}
\begin{itemize}
\item {Grp. gram.:M.}
\end{itemize}
Relativo ao radicalismo.
Partidário do radicalismo.
\section{Radicalmente}
\begin{itemize}
\item {Grp. gram.:adv.}
\end{itemize}
De modo radical; essencialmente.
\section{Radicante}
\begin{itemize}
\item {Grp. gram.:adj.}
\end{itemize}
\begin{itemize}
\item {Proveniência:(Lat. \textunderscore radicans\textunderscore )}
\end{itemize}
Que radica.
\section{Radicar}
\begin{itemize}
\item {Grp. gram.:v. t.}
\end{itemize}
\begin{itemize}
\item {Proveniência:(Lat. \textunderscore radicare\textunderscore )}
\end{itemize}
Enraizar; firmar; infundir.
\section{Radicela}
\begin{itemize}
\item {Grp. gram.:f.}
\end{itemize}
O mesmo que \textunderscore radícula\textunderscore .
\section{Radicella}
\begin{itemize}
\item {Grp. gram.:f.}
\end{itemize}
O mesmo que \textunderscore radícula\textunderscore .
\section{Radicícola}
\begin{itemize}
\item {Grp. gram.:adj.}
\end{itemize}
\begin{itemize}
\item {Proveniência:(Do lat. \textunderscore radix\textunderscore  + \textunderscore colere\textunderscore )}
\end{itemize}
Diz-se dos parasitos, que vivem na raíz das plantas.
\section{Radicifloro}
\begin{itemize}
\item {Grp. gram.:adj.}
\end{itemize}
\begin{itemize}
\item {Proveniência:(Do lat. \textunderscore radix\textunderscore  + \textunderscore flos\textunderscore , \textunderscore floris\textunderscore )}
\end{itemize}
Cujas flôres brotam de uma haste subterrânea.
\section{Radiciforme}
\begin{itemize}
\item {Grp. gram.:adj.}
\end{itemize}
\begin{itemize}
\item {Proveniência:(Do lat. \textunderscore radix\textunderscore  + \textunderscore forma\textunderscore )}
\end{itemize}
Semelhante a uma raíz.
\section{Radicívoro}
\begin{itemize}
\item {Grp. gram.:adj.}
\end{itemize}
\begin{itemize}
\item {Proveniência:(Do lat. \textunderscore radix\textunderscore  + \textunderscore vorare\textunderscore )}
\end{itemize}
Que se alimenta de raízes.
\section{Rádico}
\begin{itemize}
\item {Grp. gram.:adj.}
\end{itemize}
Relativo ao rádio.
\section{Radícola}
\begin{itemize}
\item {Grp. gram.:adj.}
\end{itemize}
\begin{itemize}
\item {Proveniência:(Do lat. \textunderscore radix\textunderscore  + \textunderscore colere\textunderscore )}
\end{itemize}
Que vive ou apparece nas raízes de um vegetal.
O mesmo que \textunderscore radicícola\textunderscore .
Diz-se da phylloxera, que se manifesta nas raízes da videira.
\section{Radicoso}
\begin{itemize}
\item {Grp. gram.:adj.}
\end{itemize}
\begin{itemize}
\item {Proveniência:(Do lat. \textunderscore radix\textunderscore )}
\end{itemize}
Que tem raízes.
\section{Radícula}
\begin{itemize}
\item {Grp. gram.:f.}
\end{itemize}
\begin{itemize}
\item {Proveniência:(Lat. \textunderscore radicula\textunderscore )}
\end{itemize}
Pequena raíz; embryão da raíz.
Objecto, semelhante a uma pequena raíz.
\section{Radiculado}
\begin{itemize}
\item {Grp. gram.:adj.}
\end{itemize}
Que tem radículas.
\section{Radicular}
\begin{itemize}
\item {Grp. gram.:adj.}
\end{itemize}
\begin{itemize}
\item {Proveniência:(Do lat. \textunderscore radicula\textunderscore )}
\end{itemize}
Relativo a raíz.
\section{Radífero}
\begin{itemize}
\item {Grp. gram.:adj.}
\end{itemize}
\begin{itemize}
\item {Proveniência:(Do lat. \textunderscore radius\textunderscore  + \textunderscore ferre\textunderscore )}
\end{itemize}
Que emitte raios luminosos.
\section{Rádio}
\begin{itemize}
\item {Grp. gram.:m.}
\end{itemize}
\begin{itemize}
\item {Utilização:Anat.}
\end{itemize}
\begin{itemize}
\item {Utilização:Ant.}
\end{itemize}
\begin{itemize}
\item {Proveniência:(Do lat. \textunderscore radium\textunderscore )}
\end{itemize}
Osso, que, com o cúbito, fórma o ante-braço.
Balestilha.
Substância, descoberta em 1899, que se contém no báryo e que emitte radiações, que atravessam com luz e calor os corpos mais opacos.
\section{Radioactividade}
\begin{itemize}
\item {Grp. gram.:f.}
\end{itemize}
\begin{itemize}
\item {Utilização:Chím.}
\end{itemize}
\begin{itemize}
\item {Proveniência:(De \textunderscore radio\textunderscore  + \textunderscore actividade\textunderscore )}
\end{itemize}
Qualidade, que o rádio tem, de communicar as suas propriedades aos corpos, que se lhe juntam no mesmo recipiente.
\section{Radioactivo}
\begin{itemize}
\item {Grp. gram.:adj.}
\end{itemize}
\begin{itemize}
\item {Proveniência:(De \textunderscore radio\textunderscore  + \textunderscore activo\textunderscore )}
\end{itemize}
Que possue radioactividade.
\section{Radioconductor}
\begin{itemize}
\item {Grp. gram.:m.}
\end{itemize}
\begin{itemize}
\item {Proveniência:(De \textunderscore rádio\textunderscore  + \textunderscore conductor\textunderscore )}
\end{itemize}
Tubo de limalha, empregado na telegraphia sem fios.
\section{Radiocondutor}
\begin{itemize}
\item {Grp. gram.:m.}
\end{itemize}
\begin{itemize}
\item {Proveniência:(De \textunderscore rádio\textunderscore  + \textunderscore condutor\textunderscore )}
\end{itemize}
Tubo de limalha, empregado na telegrafia sem fios.
\section{Radiocultura}
\begin{itemize}
\item {Grp. gram.:f.}
\end{itemize}
\begin{itemize}
\item {Proveniência:(De \textunderscore radio\textunderscore  + \textunderscore cultura\textunderscore )}
\end{itemize}
Experiências physicas, com que se applicam as diversas côres do prisma á cultura das plantas. Cf. \textunderscore Jorn. do Comm.\textunderscore , do Rio, de 18-IX-900.
\section{Radiografar}
\begin{itemize}
\item {Grp. gram.:v. t.}
\end{itemize}
Observar ou reproduzir por meio da radiografia.
\section{Radiografia}
\begin{itemize}
\item {Grp. gram.:f.}
\end{itemize}
\begin{itemize}
\item {Proveniência:(T. hybr., do lat. \textunderscore radius\textunderscore  + gr. \textunderscore graphein\textunderscore )}
\end{itemize}
Estudo dos raios luminosos.
Aplicação dos raios X ou raios de R[oe]ntgen á Medicina.
Reprodução fotográfica de uma observação radioscópica.
\section{Radiográfico}
\begin{itemize}
\item {Grp. gram.:adj.}
\end{itemize}
Relativo á radiografia.
\section{Radiograma}
\begin{itemize}
\item {Grp. gram.:m.}
\end{itemize}
\begin{itemize}
\item {Proveniência:(Do lat. \textunderscore radius\textunderscore  + gr. \textunderscore gramma\textunderscore )}
\end{itemize}
Comunicação pela telegrafia sem fios.
\section{Radiogramma}
\begin{itemize}
\item {Grp. gram.:m.}
\end{itemize}
\begin{itemize}
\item {Proveniência:(Do lat. \textunderscore radius\textunderscore  + gr. \textunderscore gramma\textunderscore )}
\end{itemize}
Communicação pela telegraphia sem fios.
\section{Radiographar}
\begin{itemize}
\item {Grp. gram.:v. t.}
\end{itemize}
Observar ou reproduzir por meio da radiographia.
\section{Radiographia}
\begin{itemize}
\item {Grp. gram.:f.}
\end{itemize}
\begin{itemize}
\item {Proveniência:(T. hybr., do lat. \textunderscore radius\textunderscore  + gr. \textunderscore graphein\textunderscore )}
\end{itemize}
Estudo dos raios luminosos.
Applicação dos raios X ou raios de R[oe]ntgen á Medicina.
Reproducção photográphica de uma observação radioscópica.
\section{Radiográphico}
\begin{itemize}
\item {Grp. gram.:adj.}
\end{itemize}
Relativo á radiographia.
\section{Radiolário}
\begin{itemize}
\item {Grp. gram.:m.}
\end{itemize}
\begin{itemize}
\item {Proveniência:(Do lat. \textunderscore radiolus\textunderscore )}
\end{itemize}
Espécie de protozoário, de formosa concha microscópica.
\section{Radiologia}
\begin{itemize}
\item {Grp. gram.:f.}
\end{itemize}
\begin{itemize}
\item {Proveniência:(Do lat. \textunderscore radius\textunderscore  + gr. \textunderscore logos\textunderscore )}
\end{itemize}
Estudo scientífico dos raios luminosos, especialmente dos raios X.
\section{Radiometria}
\begin{itemize}
\item {Grp. gram.:f.}
\end{itemize}
Emprêgo ou applicação do radiómetro.
\section{Radiómetro}
\begin{itemize}
\item {Grp. gram.:m.}
\end{itemize}
\begin{itemize}
\item {Proveniência:(Do lat. \textunderscore radius\textunderscore  + gr. \textunderscore metron\textunderscore )}
\end{itemize}
Antigo instrumento náutico, para tomar a altura dos astros; balestilha.
\section{Radiosamente}
\begin{itemize}
\item {Grp. gram.:adv.}
\end{itemize}
De modo radioso; brilhantemente; luminosamente.
\section{Radioscopia}
\begin{itemize}
\item {Grp. gram.:f.}
\end{itemize}
\begin{itemize}
\item {Proveniência:(Do lat. \textunderscore radius\textunderscore  + gr. \textunderscore skopein\textunderscore )}
\end{itemize}
Exame, por meio da imagem, formada sôbre um anteparo fluorescente, de um corpo interposto a êsse anteparo e uma fonte de raios X.
\section{Radioscópico}
\begin{itemize}
\item {Grp. gram.:adj.}
\end{itemize}
Relativo á radioscopia.
\section{Radioso}
\begin{itemize}
\item {Grp. gram.:adj.}
\end{itemize}
\begin{itemize}
\item {Utilização:Fig.}
\end{itemize}
\begin{itemize}
\item {Proveniência:(Lat. \textunderscore radiosus\textunderscore )}
\end{itemize}
Que emitte raios de luz ou calor.
Esplendoroso.
Muito alegre; jubiloso.
\section{Radiotelegrafia}
\begin{itemize}
\item {Grp. gram.:f.}
\end{itemize}
\begin{itemize}
\item {Proveniência:(De \textunderscore rádio\textunderscore  + \textunderscore telegrafia\textunderscore )}
\end{itemize}
Emprêgo da telegrafia sem fios.
\section{Radiotelegraphia}
\begin{itemize}
\item {Grp. gram.:f.}
\end{itemize}
\begin{itemize}
\item {Proveniência:(De \textunderscore rádio\textunderscore  + \textunderscore telegraphia\textunderscore )}
\end{itemize}
Emprêgo da telegraphia sem fios.
\section{Radioterapia}
\begin{itemize}
\item {Grp. gram.:f.}
\end{itemize}
\begin{itemize}
\item {Proveniência:(De \textunderscore radio\textunderscore  + \textunderscore terapia\textunderscore )}
\end{itemize}
Tratamento terapêutico pelos raios X.
\section{Radiotherapia}
\begin{itemize}
\item {Grp. gram.:f.}
\end{itemize}
\begin{itemize}
\item {Proveniência:(De \textunderscore radio\textunderscore  + \textunderscore therapia\textunderscore )}
\end{itemize}
Tratamento therapêutico pelos raios X.
\section{Radobar}
\begin{itemize}
\item {Grp. gram.:v.}
\end{itemize}
\begin{itemize}
\item {Utilização:t. Náut.}
\end{itemize}
\begin{itemize}
\item {Proveniência:(Fr. \textunderscore radouber\textunderscore )}
\end{itemize}
Fazer reparações ou consertos em (navios):«\textunderscore ...onde radobadas as náos, mandou n'huma dellas a Sancho de Thoar...\textunderscore »Filinto, \textunderscore D. Man.\textunderscore , I, 179.
\section{Radolho}
\begin{itemize}
\item {fónica:dô}
\end{itemize}
\begin{itemize}
\item {Grp. gram.:adj.}
\end{itemize}
(?):«\textunderscore Devem têr todo o cuidado na escolha da uva, tirando-lhe todo o verde que tiver, radolho e seco sem tempo...\textunderscore »\textunderscore Bibl. da G. do Campo\textunderscore , 407.
\section{Radote}
\begin{itemize}
\item {Grp. gram.:m.}
\end{itemize}
\begin{itemize}
\item {Utilização:Des.}
\end{itemize}
\begin{itemize}
\item {Proveniência:(Do lat. \textunderscore radere\textunderscore )}
\end{itemize}
Instrumento para raspar; raspadeira.
\section{Raer}
\begin{itemize}
\item {Grp. gram.:v. t.}
\end{itemize}
\begin{itemize}
\item {Proveniência:(Do lat. \textunderscore radere\textunderscore )}
\end{itemize}
Vassoirar (o forno), depois de aquecido para a cozedura.
Arrastar com o rodo (o sal), nas marinhas; rêr.
\section{Rafa}
\begin{itemize}
\item {Grp. gram.:f.}
\end{itemize}
\begin{itemize}
\item {Utilização:Gír.}
\end{itemize}
Fome; penúria.
\section{Rafa}
\begin{itemize}
\item {Grp. gram.:f.}
\end{itemize}
\begin{itemize}
\item {Utilização:Náut.}
\end{itemize}
\begin{itemize}
\item {Utilização:Ant.}
\end{itemize}
Maré forte.
(Cp. cast. \textunderscore ráfaga\textunderscore )
\section{Rafado}
\begin{itemize}
\item {Grp. gram.:adj.}
\end{itemize}
\begin{itemize}
\item {Utilização:Pop.}
\end{itemize}
Que tem rafa^1.
\section{Rafaelesco}
\begin{itemize}
\item {fónica:fa-e-lês}
\end{itemize}
\begin{itemize}
\item {Grp. gram.:adj.}
\end{itemize}
Relativo ao pintor Rafael.
\section{Rafaélico}
\begin{itemize}
\item {Grp. gram.:adj.}
\end{itemize}
O mesmo que \textunderscore rafaelesco\textunderscore . Cf. Júl. Lour. Pinto, \textunderscore Senh. Deput.\textunderscore , 48.
\section{Rafaelismo}
\begin{itemize}
\item {fónica:fa-e}
\end{itemize}
\begin{itemize}
\item {Grp. gram.:m.}
\end{itemize}
Feição característica das pinturas de Rafael.
\section{Rafaelita}
\begin{itemize}
\item {fónica:fa-e}
\end{itemize}
\begin{itemize}
\item {Grp. gram.:m.}
\end{itemize}
Pintor, da escola de Rafael.
\section{Rafar}
\begin{itemize}
\item {Grp. gram.:v. t.}
\end{itemize}
\begin{itemize}
\item {Utilização:Prov.}
\end{itemize}
\begin{itemize}
\item {Utilização:trasm.}
\end{itemize}
\begin{itemize}
\item {Proveniência:(De \textunderscore rafa\textunderscore ^1)}
\end{itemize}
Cotear, gastar com o uso: \textunderscore umas calças rafadas\textunderscore .
O mesmo que \textunderscore furtar\textunderscore .
\section{Rafeira}
\begin{itemize}
\item {Grp. gram.:adj.}
\end{itemize}
\begin{itemize}
\item {Utilização:Ant.}
\end{itemize}
\begin{itemize}
\item {Proveniência:(De \textunderscore rafar\textunderscore )}
\end{itemize}
Dizia-se de uma febre violenta.
\section{Rafeiro}
\begin{itemize}
\item {Grp. gram.:m.  e  adj.}
\end{itemize}
\begin{itemize}
\item {Utilização:deprec.}
\end{itemize}
\begin{itemize}
\item {Utilização:Fam.}
\end{itemize}
Cão, que serva para a guarda de gado.
Indivíduo impertinente, que anda sempre acompanhando outro, como o cão acompanha o dono.
\section{Ráfia}
\begin{itemize}
\item {Grp. gram.:f.}
\end{itemize}
O mesmo ou melhor que \textunderscore ráphia\textunderscore .
\section{Rafiar}
\begin{itemize}
\item {Grp. gram.:v. t.}
\end{itemize}
\begin{itemize}
\item {Utilização:Ant.}
\end{itemize}
\begin{itemize}
\item {Utilização:Fig.}
\end{itemize}
Guarnecer de fios.
Acariciar.
\section{Rafinar}
\begin{itemize}
\item {Grp. gram.:v. t.}
\end{itemize}
\begin{itemize}
\item {Utilização:Ant.}
\end{itemize}
Atirar?:«\textunderscore ...derão fogo a kũa grande mina..., a qual arrebentando rafinou para o ar o Capitão Prata\textunderscore ». \textunderscore Peregrinação\textunderscore , CXVII.
\section{Raflésia}
\begin{itemize}
\item {Grp. gram.:f.}
\end{itemize}
\begin{itemize}
\item {Proveniência:(De \textunderscore Rafles\textunderscore , n. p.)}
\end{itemize}
Flôr giganteia, a maior que se conhece, pois mede três metros de circunferência. É originária de Samatra.
\section{Rageira}
\begin{itemize}
\item {Grp. gram.:f.}
\end{itemize}
\begin{itemize}
\item {Utilização:Náut.}
\end{itemize}
Cabo, com que se amarra a embarcação á terra.
(Alter. de \textunderscore regeira\textunderscore , de \textunderscore reger\textunderscore ?)
\section{Ragel}
\begin{itemize}
\item {Grp. gram.:m.}
\end{itemize}
(?):«\textunderscore ...dezasete palmos que leva esta náo de Ragel, que peza boa conta dezasete Rumos de Quilha dezasete palmos de Ragel...\textunderscore »Fern. Oliveira, \textunderscore Fábr. das Naus\textunderscore .
\section{Ragu}
\begin{itemize}
\item {Grp. gram.:m.}
\end{itemize}
\begin{itemize}
\item {Utilização:Bras. do Rio}
\end{itemize}
Qualquer guisado ou ensopado: \textunderscore um ragu de carneiro\textunderscore .
\section{Ragueira}
\begin{itemize}
\item {Grp. gram.:f.}
\end{itemize}
\begin{itemize}
\item {Utilização:Ant.}
\end{itemize}
O mesmo que \textunderscore rageira\textunderscore . Cf. Lopo de S. Coutinho, \textunderscore Hist. do Cêrco de Dio\textunderscore , l. I, c. III.
\section{Ragura}
\begin{itemize}
\item {Grp. gram.:f.}
\end{itemize}
\begin{itemize}
\item {Utilização:Ant.}
\end{itemize}
O mesmo que \textunderscore rancora\textunderscore .
\section{Raia}
\begin{itemize}
\item {Grp. gram.:f.}
\end{itemize}
\begin{itemize}
\item {Utilização:Fam.}
\end{itemize}
\begin{itemize}
\item {Utilização:Bras. do Rio}
\end{itemize}
\begin{itemize}
\item {Utilização:Bras}
\end{itemize}
\begin{itemize}
\item {Grp. gram.:Pl.}
\end{itemize}
\begin{itemize}
\item {Utilização:Fig.}
\end{itemize}
Lista; traço; estria.
Linha da palma da mão.
Fronteira, confins.
Êrro; asneira: \textunderscore o sujeito deu raia\textunderscore .
Tacada, com que no bilhar se fere a bola do parceiro em vez da própria.
Risco, que se faz com ferro em brasa no cavallo e serve de contramarca.
O mesmo que \textunderscore pista\textunderscore , em corridas de cavallos.
O mais alto grau: \textunderscore isso é tolice que passa das raias\textunderscore .
(Cp. \textunderscore raja\textunderscore ^2)
\section{Raia}
\begin{itemize}
\item {Grp. gram.:f.}
\end{itemize}
\begin{itemize}
\item {Proveniência:(Lat. \textunderscore raja\textunderscore )}
\end{itemize}
O mesmo que \textunderscore arraia\textunderscore ^2.
\section{Raiação}
\begin{itemize}
\item {Grp. gram.:f.}
\end{itemize}
\begin{itemize}
\item {Utilização:Neol.}
\end{itemize}
Acto de raiar^1. Cf. Eça, \textunderscore Mandarim\textunderscore , 15.
\section{Raiadela}
\begin{itemize}
\item {Grp. gram.:f.}
\end{itemize}
\begin{itemize}
\item {Utilização:Prov.}
\end{itemize}
\begin{itemize}
\item {Utilização:trasm.}
\end{itemize}
\begin{itemize}
\item {Proveniência:(De \textunderscore raiar\textunderscore ^1)}
\end{itemize}
Dôr nos intestinos, momentânea mas violenta.
\section{Raial}
\begin{itemize}
\item {Grp. gram.:m.}
\end{itemize}
Antiga moéda de oiro portuguesa, correspondente a 3 libras antigas.
\section{Raiano}
\begin{itemize}
\item {Grp. gram.:m.  e  adj.}
\end{itemize}
\begin{itemize}
\item {Proveniência:(De \textunderscore raia\textunderscore ^1)}
\end{itemize}
O mesmo que \textunderscore arraiano\textunderscore .
\section{Raiar}
\begin{itemize}
\item {Grp. gram.:v. i.}
\end{itemize}
\begin{itemize}
\item {Proveniência:(Do lat. \textunderscore radiare\textunderscore )}
\end{itemize}
Radiar; despontar no horizonte; vir apparecendo; surgir: \textunderscore já o Sol raiava\textunderscore .
\section{Raiar}
\begin{itemize}
\item {Grp. gram.:v. t.}
\end{itemize}
\begin{itemize}
\item {Grp. gram.:V. i.}
\end{itemize}
\begin{itemize}
\item {Utilização:Fig.}
\end{itemize}
\begin{itemize}
\item {Proveniência:(De \textunderscore raia\textunderscore ^1)}
\end{itemize}
Traçar riscas em; estriar.
Tocar nas raias ou limites; aproximar-se; orçar: \textunderscore a pequena já raia pelos 15 annos\textunderscore .
\section{Raiberrugo}
\begin{itemize}
\item {Grp. gram.:m.}
\end{itemize}
\begin{itemize}
\item {Utilização:Prov.}
\end{itemize}
\begin{itemize}
\item {Utilização:trasm.}
\end{itemize}
O mesmo que \textunderscore rabo-ruivo\textunderscore .
\section{Raieiro}
\begin{itemize}
\item {Grp. gram.:adj.}
\end{itemize}
\begin{itemize}
\item {Utilização:Açor}
\end{itemize}
Que tem maus costumes ou mau gênio.
\section{Raigota}
\begin{itemize}
\item {fónica:ra-i}
\end{itemize}
\begin{itemize}
\item {Grp. gram.:f.}
\end{itemize}
\begin{itemize}
\item {Proveniência:(De \textunderscore raiz\textunderscore )}
\end{itemize}
Radícula.
Espigão na base das unhas.
\section{Raigotoso}
\begin{itemize}
\item {fónica:ra-i}
\end{itemize}
\begin{itemize}
\item {Grp. gram.:adj.}
\end{itemize}
Que tem raigotas.
\section{Raigrás}
\begin{itemize}
\item {Grp. gram.:m.}
\end{itemize}
\begin{itemize}
\item {Proveniência:(Do ingl. \textunderscore ray\textunderscore  + \textunderscore grass\textunderscore )}
\end{itemize}
Planta gramínea, (\textunderscore lolium perenne\textunderscore , Lin.).
\section{Raimundo-silvestre}
\begin{itemize}
\item {Grp. gram.:m.}
\end{itemize}
Planta da serra de Sintra.
(Corr. de \textunderscore ranúnculo-silvestre\textunderscore ?)
\section{Raineta}
\begin{itemize}
\item {fónica:nê}
\end{itemize}
\begin{itemize}
\item {Grp. gram.:f.  e  adj.}
\end{itemize}
\begin{itemize}
\item {Proveniência:(Fr. \textunderscore rainette\textunderscore )}
\end{itemize}
Variedade de maçan.
Espécie de ran pequena, verde-clara, que vive entre as ervas, sobe ás árvores, e solta gritos, a que outras correspondem.
\section{Raínha}
\begin{itemize}
\item {Grp. gram.:f.}
\end{itemize}
\begin{itemize}
\item {Utilização:Pesc.}
\end{itemize}
\begin{itemize}
\item {Utilização:Prov.}
\end{itemize}
\begin{itemize}
\item {Utilização:dur.}
\end{itemize}
\begin{itemize}
\item {Proveniência:(Do lat. \textunderscore regina\textunderscore )}
\end{itemize}
Mulhér de rei.
Soberana de um reino.
Coisa ou pessôa principal entre outras: \textunderscore a rainha do baile\textunderscore .
Variedade de pêra e de maçan.
Rêde triangular, que fórma a parte inferior do pano central do apparelho de galeão.
O trabalhador, que occupa a extremidade inferior de uma columna dêlles, na cava ou na redra.
\section{Raínha}
\begin{itemize}
\item {Grp. gram.:f.}
\end{itemize}
\begin{itemize}
\item {Proveniência:(De \textunderscore raia\textunderscore ^2)}
\end{itemize}
Peixe dos Açôres, também chamado \textunderscore labarda\textunderscore .
\section{Raínha-cláudia}
\begin{itemize}
\item {Grp. gram.:f.}
\end{itemize}
Espécie de ameixa, muito apreciada, e conhecida também por \textunderscore ameixa caranguejeira\textunderscore .
\section{Raínha-dos-prados}
\begin{itemize}
\item {Grp. gram.:f.}
\end{itemize}
O mesmo que \textunderscore erva-ulmeira\textunderscore .
\section{Raínha-margarida}
\begin{itemize}
\item {Grp. gram.:f.}
\end{itemize}
\begin{itemize}
\item {Utilização:Bras}
\end{itemize}
Planta dos jardins.
\section{Rainheta}
\begin{itemize}
\item {fónica:ra-i}
\end{itemize}
\begin{itemize}
\item {Grp. gram.:f.}
\end{itemize}
\begin{itemize}
\item {Utilização:Pesc.}
\end{itemize}
\begin{itemize}
\item {Proveniência:(De \textunderscore raínha\textunderscore )}
\end{itemize}
Rêde de fio grosso, que cose a copejada á raínha, no apparelho de galeão.
\section{Raínho}
\begin{itemize}
\item {Grp. gram.:adj.}
\end{itemize}
\begin{itemize}
\item {Utilização:Prov.}
\end{itemize}
\begin{itemize}
\item {Utilização:minh.}
\end{itemize}
Diz-se de uma espécie de milho que tem o grão vermelho.
\section{Rainúnculo}
\begin{itemize}
\item {Grp. gram.:m.}
\end{itemize}
\begin{itemize}
\item {Utilização:Pop.}
\end{itemize}
O mesmo que \textunderscore ranúnculo\textunderscore .
\section{Raio}
\begin{itemize}
\item {Grp. gram.:m.}
\end{itemize}
\begin{itemize}
\item {Utilização:Fig.}
\end{itemize}
\begin{itemize}
\item {Proveniência:(Do lat. \textunderscore radius\textunderscore )}
\end{itemize}
Cada um dos traços de luz, que resáem de um fóco.
Movimento do calor.
Emanação.
Cada um dos objectos, que, partindo de um centro, vão divergindo: \textunderscore os raios da roda de uma carruagem\textunderscore .
Meio diâmetro de uma circunferência.
Sinal, vislumbre.
Faísca eléctrica.
Tudo que fulmina ou destrói.
Catástrophe; fatalidade, desgraça: \textunderscore caiu-lhe o raio em casa\textunderscore .
\textunderscore Raios X\textunderscore , ou \textunderscore raios de Roentgen\textunderscore , raios luminosos, não perceptíveis á vista, que, atravessando corpos opacos e impressionando placas sensibilizadas, têm applicações therapêuticas.
\section{Raiola}
\begin{itemize}
\item {Grp. gram.:f.}
\end{itemize}
\begin{itemize}
\item {Utilização:Prov.}
\end{itemize}
\begin{itemize}
\item {Utilização:trasm.}
\end{itemize}
\begin{itemize}
\item {Utilização:beir.}
\end{itemize}
Jôgo de rapazes, em que se atira uma moéda a um conjunto de riscos, que se fazem no chão, desta fórma--, dando-se ás curvas o nome de \textunderscore queima\textunderscore .
\section{Raios-de-júpiter}
\begin{itemize}
\item {Grp. gram.:m. pl.}
\end{itemize}
Planta amaryllídea, diurética e vomitiva.
\section{Raitana}
\begin{itemize}
\item {Grp. gram.:f.}
\end{itemize}
\begin{itemize}
\item {Utilização:Prov.}
\end{itemize}
\begin{itemize}
\item {Utilização:trasm.}
\end{itemize}
O mesmo que \textunderscore leitôa\textunderscore .
\section{Raiúna}
\begin{itemize}
\item {Grp. gram.:f.}
\end{itemize}
\begin{itemize}
\item {Utilização:Pop.}
\end{itemize}
Espingarda de fuzil, curta e grossa. Cf. Capello e Ivens, I, 7.
(Cp. \textunderscore reiúna\textunderscore )
\section{Raiva}
\begin{itemize}
\item {Grp. gram.:f.}
\end{itemize}
\begin{itemize}
\item {Proveniência:(Lat. hyp. \textunderscore rabia\textunderscore , de \textunderscore rabies\textunderscore )}
\end{itemize}
Doença própria dos cães, caracterizada por accessos furiosos, desejo de morder e uma saliva própria para inocular a doença.
Hydrophobía.
Fúria; grande irritação.
Ódio.
Prurido, causado pela dentição nas gengivas das crianças.
Espécie de bolo sêco.
\section{Raiva}
\begin{itemize}
\item {Grp. gram.:f.}
\end{itemize}
\begin{itemize}
\item {Utilização:Ant.}
\end{itemize}
Labéu, infâmia; descrédito; mancha na reputação.
(Cp. \textunderscore laivo\textunderscore )
\section{Raivar}
\begin{itemize}
\item {Grp. gram.:v. i.}
\end{itemize}
Têr raiva; enfurecer-se.
Têr ânsias; agitar-se com violência.
\section{Raivecer}
\begin{itemize}
\item {Grp. gram.:v. i.}
\end{itemize}
O mesmo que \textunderscore raivar\textunderscore . Cf. Filinto, VI, 203.
\section{Raivejar}
\begin{itemize}
\item {Grp. gram.:v. i.}
\end{itemize}
Esbravejar, raivar. Cf. Camillo, \textunderscore Cav. em Ruínas\textunderscore , 190.
\section{Raivel}
\begin{itemize}
\item {Grp. gram.:m.}
\end{itemize}
\begin{itemize}
\item {Utilização:Prov.}
\end{itemize}
\begin{itemize}
\item {Utilização:trasm.}
\end{itemize}
Abundância, principalmente falando-se de frutos: \textunderscore depois do San-Lourenço, já há uvas a raivel\textunderscore .
\section{Raivença}
\begin{itemize}
\item {Grp. gram.:f.}
\end{itemize}
\begin{itemize}
\item {Utilização:Fam.}
\end{itemize}
\begin{itemize}
\item {Proveniência:(De \textunderscore raiva\textunderscore )}
\end{itemize}
Raiva ridícula; zanga de criança.
\section{Raivento}
\begin{itemize}
\item {Grp. gram.:adj.}
\end{itemize}
\begin{itemize}
\item {Proveniência:(De \textunderscore raiva\textunderscore )}
\end{itemize}
Que está enraivado; enfurecido.
Que enraivece facilmente.
\section{Raivós}
\begin{itemize}
\item {Grp. gram.:m. pl.}
\end{itemize}
\begin{itemize}
\item {Utilização:Prov.}
\end{itemize}
\begin{itemize}
\item {Utilização:trasm.}
\end{itemize}
Cogumelos comestíveis, que nascem nos lameiros, com as primeiras águas do outono.
\section{Raivosa}
\begin{itemize}
\item {Grp. gram.:f.}
\end{itemize}
Planta brasileira.
\section{Raivosamente}
\begin{itemize}
\item {Grp. gram.:adv.}
\end{itemize}
De modo raivoso; com fúria.
\section{Raivoso}
\begin{itemize}
\item {Grp. gram.:adj.}
\end{itemize}
\begin{itemize}
\item {Proveniência:(De \textunderscore raiva\textunderscore )}
\end{itemize}
Raivento; furioso.
Bravio.
\section{Raiz}
\begin{itemize}
\item {Grp. gram.:f.}
\end{itemize}
\begin{itemize}
\item {Proveniência:(Do lat. \textunderscore radix\textunderscore )}
\end{itemize}
Parte inferior das plantas, que entra geralmente pela terra e pela qual os vegetaes se fixam no solo, extrahindo dêlle o suco que os alimenta.
A parte occulta de qualquer objecto.
Parte inferior, base.
Elemento essencial de uma palavra; radical.
Fonte, princípio.
Número que, elevado a uma certa potencia, produz a quantidade a que êsse número se refere.
A essência material de uma propriedade rústica ou fazenda, em opposição aos seus rendimentos ou direitos.
\section{Raizada}
\begin{itemize}
\item {fónica:ra-i}
\end{itemize}
\begin{itemize}
\item {Grp. gram.:f.}
\end{itemize}
O mesmo que \textunderscore raizame\textunderscore .
\section{Raizado}
\begin{itemize}
\item {fónica:ra-i}
\end{itemize}
\begin{itemize}
\item {Grp. gram.:m.}
\end{itemize}
\begin{itemize}
\item {Proveniência:(De \textunderscore raiz\textunderscore )}
\end{itemize}
Vide com raízes, para plantação.
O mesmo que \textunderscore barbado\textunderscore .
\section{Raizame}
\begin{itemize}
\item {fónica:ra-i}
\end{itemize}
\begin{itemize}
\item {Grp. gram.:m.}
\end{itemize}
\begin{itemize}
\item {Proveniência:(De \textunderscore raiz\textunderscore )}
\end{itemize}
Reunião de muitas raízes.
\section{Raiz-da-bôlsa}
\begin{itemize}
\item {Grp. gram.:f.}
\end{itemize}
Gênero de plantas dioscoreáceas, (\textunderscore dioscorea daemona\textunderscore , Roxb.).
\section{Raiz-da-china}
\begin{itemize}
\item {Grp. gram.:f.}
\end{itemize}
Planta esmilácea, (\textunderscore smilax China\textunderscore ).
\section{Raiz-da-mostarda}
\begin{itemize}
\item {Grp. gram.:f.}
\end{itemize}
Gênero de plantas euphorbiáceas da Índia Portuguesa, (\textunderscore putranjiva Roxburghii\textunderscore , Wall.).
\section{Raiz-de-câmaras}
\begin{itemize}
\item {Grp. gram.:f.}
\end{itemize}
Gênero de plantas menispermáceas da Índia Portuguesa, (\textunderscore cyclea peltata\textunderscore , Hook).
\section{Raiz-de-cobra}
\begin{itemize}
\item {Grp. gram.:f.}
\end{itemize}
Gênero de plantas indianas, (\textunderscore aristolochia índica\textunderscore , Lin.).
\section{Raiz-de-empose}
\begin{itemize}
\item {Grp. gram.:m.}
\end{itemize}
Gênero de plantas liliáceas da Índia Portuguesa, (\textunderscore gloriosa superba\textunderscore , Lin.). Cf. Dalgado, \textunderscore Flora\textunderscore , 196.
\section{Raiz-de-guiné}
\begin{itemize}
\item {Grp. gram.:f.}
\end{itemize}
\begin{itemize}
\item {Utilização:Bras}
\end{itemize}
O mesmo que \textunderscore pipi\textunderscore ^3.
\section{Raiz-de-lagarto}
\begin{itemize}
\item {Grp. gram.:f.}
\end{itemize}
O mesmo que \textunderscore tiú\textunderscore  e \textunderscore jalapão\textunderscore .
\section{Raiz-de-lopes}
\begin{itemize}
\item {Grp. gram.:f.}
\end{itemize}
Gênero de plantas rutáceas da Índia Portuguesa, (\textunderscore todalia aculeata\textunderscore , Pers.).
\section{Raiz-de-solteira}
\begin{itemize}
\item {Grp. gram.:f.}
\end{itemize}
Gênero de plantas convolvuláceas da Índia Portuguesa, (\textunderscore erycibe paniculata\textunderscore , Roxb.).
\section{Raiz-divina}
\begin{itemize}
\item {Grp. gram.:f.}
\end{itemize}
Planta, da fam. das plumbagináceas, (\textunderscore armeria Welwitschii\textunderscore , Bss.) também conhecida por \textunderscore erva-divina\textunderscore .
\section{Raiz-do-brasil}
\begin{itemize}
\item {Grp. gram.:f.}
\end{itemize}
O mesmo que \textunderscore ipecacuanha\textunderscore .
\section{Raiz-doce}
\begin{itemize}
\item {Grp. gram.:f.}
\end{itemize}
O mesmo que \textunderscore alcaçuz\textunderscore .
\section{Raiz-do-espírito-santo}
\begin{itemize}
\item {Grp. gram.:f.}
\end{itemize}
O mesmo que \textunderscore angélica\textunderscore .
\section{Raiz-do-sol}
\begin{itemize}
\item {Grp. gram.:f.}
\end{itemize}
\begin{itemize}
\item {Utilização:Bras}
\end{itemize}
Planta medicinal do Pará, usada contra doenças dos olhos.
\section{Raizeira}
\begin{itemize}
\item {fónica:ra-i}
\end{itemize}
\begin{itemize}
\item {Grp. gram.:m.}
\end{itemize}
\begin{itemize}
\item {Utilização:Prov.}
\end{itemize}
\begin{itemize}
\item {Utilização:minh.}
\end{itemize}
\begin{itemize}
\item {Utilização:Prov.}
\end{itemize}
\begin{itemize}
\item {Utilização:beir.}
\end{itemize}
\begin{itemize}
\item {Proveniência:(De \textunderscore raiz\textunderscore )}
\end{itemize}
Aquillo que da árvore fica na terra, depois de cortada pelo pé.
Grupo de raízes emmaranhadas, no pé de uma árvore.
\section{Raizeiro}
\begin{itemize}
\item {fónica:ra-i}
\end{itemize}
\begin{itemize}
\item {Grp. gram.:m.}
\end{itemize}
\begin{itemize}
\item {Utilização:Prov.}
\end{itemize}
\begin{itemize}
\item {Utilização:minh.}
\end{itemize}
O mesmo que \textunderscore raizeira\textunderscore .
\section{Raiz-madre-de-deus}
\begin{itemize}
\item {Grp. gram.:f.}
\end{itemize}
Raiz medicinal de uma planta indiana, (\textunderscore gmelina asiatica\textunderscore , Lin.).
\section{Raiz-mordida}
\begin{itemize}
\item {Grp. gram.:f.}
\end{itemize}
Nome antigo de uma planta medicinal, sudorífica e antiséptica.--Ignoro a que gênero ou família ella pertença. Cf. \textunderscore Desengano para a Medicina\textunderscore , 242.
\section{Raja}
\begin{itemize}
\item {Grp. gram.:m.}
\end{itemize}
\begin{itemize}
\item {Proveniência:(Do sânscr. \textunderscore raja\textunderscore )}
\end{itemize}
Príncipe indiano, sujeito a um protectorado, e que indevidamente chamam \textunderscore rajá\textunderscore .
Antiga moéda da Índia Port.
\section{Raja}
\begin{itemize}
\item {Grp. gram.:f.}
\end{itemize}
Raia, lista, estria.
(Cp. cast. \textunderscore raja\textunderscore )
\section{Rajada}
\begin{itemize}
\item {Grp. gram.:f.}
\end{itemize}
\begin{itemize}
\item {Utilização:Fig.}
\end{itemize}
\begin{itemize}
\item {Proveniência:(Do cast. \textunderscore rajar\textunderscore )}
\end{itemize}
Ventania forte e rápida; rabanada.
Ímpeto; rasgo de eloquência.
\section{Rajada}
\begin{itemize}
\item {Grp. gram.:f.}
\end{itemize}
\begin{itemize}
\item {Utilização:Bras}
\end{itemize}
Espécie de mandioca.
\section{Rajado}
\begin{itemize}
\item {Grp. gram.:adj.}
\end{itemize}
\begin{itemize}
\item {Grp. gram.:Adj. f.}
\end{itemize}
\begin{itemize}
\item {Proveniência:(De \textunderscore rajar\textunderscore )}
\end{itemize}
Raiado, estriado.
Diz-se de uma variedade de maçan temporan, também chamada \textunderscore noivo\textunderscore  e \textunderscore de espelho\textunderscore .
\section{Rajão}
\begin{itemize}
\item {Grp. gram.:m.}
\end{itemize}
Viola de cinco cordas, usada na Madeira.
\section{Rajão}
\begin{itemize}
\item {Grp. gram.:m.}
\end{itemize}
\begin{itemize}
\item {Utilização:Prov.}
\end{itemize}
\begin{itemize}
\item {Utilização:minh.}
\end{itemize}
Torresmo, o mesmo que \textunderscore rojão\textunderscore ^3.
\section{Rajapura}
\begin{itemize}
\item {Grp. gram.:m.}
\end{itemize}
Bebida alcoólica dos Índios do Peru.
\section{Rajaputros}
\begin{itemize}
\item {Grp. gram.:m. pl.}
\end{itemize}
O mesmo que \textunderscore reisbutos\textunderscore .
\section{Rajar}
\begin{itemize}
\item {Grp. gram.:v. t.}
\end{itemize}
\begin{itemize}
\item {Utilização:Fig.}
\end{itemize}
Raiar, estriar.
Entremear.
(Cp. \textunderscore raiar\textunderscore ^2)
\section{Rajo}
\begin{itemize}
\item {Grp. gram.:m.}
\end{itemize}
\begin{itemize}
\item {Utilização:Prov.}
\end{itemize}
\begin{itemize}
\item {Utilização:minh.}
\end{itemize}
\begin{itemize}
\item {Grp. gram.:Pl.}
\end{itemize}
\begin{itemize}
\item {Utilização:T. de Caminha}
\end{itemize}
\begin{itemize}
\item {Proveniência:(De \textunderscore rajar\textunderscore )}
\end{itemize}
Parte dos pinheiros, que se corta, para a extracção da resina que êlles têm nos seus nós.
Estria. Laivo: \textunderscore tens rajos de sangue nos olhos\textunderscore .
Tentáculos de polvo.
\section{Rala}
\begin{itemize}
\item {Grp. gram.:f.}
\end{itemize}
O mesmo que \textunderscore rolão\textunderscore ^1.
(Fem. de \textunderscore ralo\textunderscore )
\section{Rala}
\begin{itemize}
\item {Grp. gram.:f.}
\end{itemize}
Planta vulgar.
\section{Rala}
\begin{itemize}
\item {Grp. gram.:f.}
\end{itemize}
\begin{itemize}
\item {Proveniência:(De \textunderscore ralo\textunderscore ^2)}
\end{itemize}
Orifício no frechal dos moínhos de vento, por onde entra a extremidade interna do mastro da vela.
\section{Rala}
\begin{itemize}
\item {Grp. gram.:f.}
\end{itemize}
\begin{itemize}
\item {Utilização:Gal}
\end{itemize}
\begin{itemize}
\item {Proveniência:(Fr. \textunderscore râle\textunderscore )}
\end{itemize}
Ruído, produzido pela passagem do ar da respiração através das mucosidades dos brônchios, tracheia e larynge, nas doenças dêsses órgãos.
Pieira.
Fervores.
Estertor. Cf. Mac. Pinto, \textunderscore Comp. de Veter.\textunderscore , I, 15 e 62.
É gallicismo inútil, mas muito usado.
\section{Ralação}
\begin{itemize}
\item {Grp. gram.:f.}
\end{itemize}
Acto ou effeito de ralar; apoquentação; moedeira.
\section{Ralador}
\begin{itemize}
\item {Grp. gram.:adj.}
\end{itemize}
\begin{itemize}
\item {Grp. gram.:M.}
\end{itemize}
\begin{itemize}
\item {Proveniência:(De \textunderscore ralar\textunderscore )}
\end{itemize}
Que rala.
Instrumento, para ralar ou reduzir a migalhas certas substâncias, friccionando-as.
\section{Raladura}
\begin{itemize}
\item {Grp. gram.:f.}
\end{itemize}
\begin{itemize}
\item {Proveniência:(De \textunderscore ralar\textunderscore )}
\end{itemize}
Conjunto dos fragmentos da substância que se passou pelo ralador.
Ralação.
\section{Ralão}
\begin{itemize}
\item {Grp. gram.:m.}
\end{itemize}
\begin{itemize}
\item {Utilização:Prov.}
\end{itemize}
\begin{itemize}
\item {Utilização:trasm.}
\end{itemize}
\begin{itemize}
\item {Proveniência:(De \textunderscore rala\textunderscore ^1)}
\end{itemize}
Pão de rala, pão de farinha grossa; rolão.
\section{Ralar}
\begin{itemize}
\item {Grp. gram.:v. t.}
\end{itemize}
\begin{itemize}
\item {Utilização:Fig.}
\end{itemize}
\begin{itemize}
\item {Proveniência:(De \textunderscore ralo\textunderscore ^1)}
\end{itemize}
Friccionar contra o ralador.
Fazer passar pelos orifícios do ralador.
Reduzir a pequenos fragmentos; triturar.
Amofinar; atormentar.
\section{Ralar}
\begin{itemize}
\item {Grp. gram.:v. i.}
\end{itemize}
\begin{itemize}
\item {Utilização:Prov.}
\end{itemize}
\begin{itemize}
\item {Utilização:trasm.}
\end{itemize}
O mesmo que \textunderscore coaxar\textunderscore .
(Por \textunderscore relar\textunderscore , de \textunderscore rela\textunderscore )
\section{Ralasso}
\begin{itemize}
\item {Grp. gram.:m.  e  adj.}
\end{itemize}
\begin{itemize}
\item {Utilização:Pop.}
\end{itemize}
Indivíduo indolente; madraço.
(Por \textunderscore relasso\textunderscore , do lat. \textunderscore relapsus\textunderscore )
\section{Ralé}
\begin{itemize}
\item {Grp. gram.:f.}
\end{itemize}
\begin{itemize}
\item {Utilização:Ant.}
\end{itemize}
\begin{itemize}
\item {Utilização:Pop.}
\end{itemize}
Camada inferior da sociedade; populacho.
Casta, qualidade, natureza.
Presa da ave de rapina.
Energia; aversão á indolência.
(Cp. \textunderscore relê\textunderscore )
\section{Raleadura}
\begin{itemize}
\item {Grp. gram.:f.}
\end{itemize}
Acto ou effeito de ralear.
\section{Raleamento}
\begin{itemize}
\item {Grp. gram.:m.}
\end{itemize}
O mesmo que \textunderscore raleadura\textunderscore .
\section{Ralear}
\begin{itemize}
\item {Grp. gram.:v. t.}
\end{itemize}
\begin{itemize}
\item {Grp. gram.:V. i.  e  p.}
\end{itemize}
Tornar ralo.
Tornar-se \textunderscore ralo\textunderscore ^2.
\section{Raleia}
\begin{itemize}
\item {Grp. gram.:f.}
\end{itemize}
\begin{itemize}
\item {Utilização:Des.}
\end{itemize}
O mesmo que \textunderscore ralé\textunderscore .
(Cast. \textunderscore ralea\textunderscore )
\section{Raleira}
\begin{itemize}
\item {Grp. gram.:f.}
\end{itemize}
\begin{itemize}
\item {Utilização:Fig.}
\end{itemize}
\begin{itemize}
\item {Proveniência:(De \textunderscore ralo\textunderscore ^2)}
\end{itemize}
Parte dos terrenos cultivados, na qual a semente não germinou, ou em que as plantas não medram.
Carência, escassez.
\section{Raleira}
\begin{itemize}
\item {Grp. gram.:f.}
\end{itemize}
O mesmo que \textunderscore ralação\textunderscore .
\section{Raleiro}
\begin{itemize}
\item {Grp. gram.:m.}
\end{itemize}
O mesmo que \textunderscore raleira\textunderscore ^1.
\section{Ralentar}
\begin{itemize}
\item {Grp. gram.:v. t.}
\end{itemize}
Tornar ralo^2; o mesmo que \textunderscore arralentar\textunderscore : \textunderscore foi para a ínsua, ralentar o milho...\textunderscore 
\section{Ralete}
\begin{itemize}
\item {fónica:lê}
\end{itemize}
\begin{itemize}
\item {Grp. gram.:m.}
\end{itemize}
\begin{itemize}
\item {Proveniência:(De \textunderscore ralo\textunderscore ^1)}
\end{itemize}
Pequena lâmina com orifícios, que se adapta a portas, confessionários, etc., para se vêr de dentro para fóra, sem se sêr visto. Cf. Camillo, \textunderscore Corja\textunderscore , 23.
\section{Ralhação}
\begin{itemize}
\item {Grp. gram.:f.}
\end{itemize}
Acto ou effeito de ralhar.
\section{Ralhador}
\begin{itemize}
\item {Grp. gram.:m.  e  adj.}
\end{itemize}
O que tem o hábito de ralhar.
\section{Ralhadura}
\begin{itemize}
\item {Grp. gram.:f.}
\end{itemize}
\begin{itemize}
\item {Utilização:Des.}
\end{itemize}
\begin{itemize}
\item {Utilização:Pop.}
\end{itemize}
O mesmo que \textunderscore ralho\textunderscore .
\section{Ralhão}
\begin{itemize}
\item {Grp. gram.:m.  e  adj.}
\end{itemize}
O mesmo que \textunderscore ralhador\textunderscore . Cf. Castilho, \textunderscore Misanthropo\textunderscore , 50.
\section{Ralhar}
\begin{itemize}
\item {Grp. gram.:v. i.}
\end{itemize}
\begin{itemize}
\item {Utilização:Prov.}
\end{itemize}
\begin{itemize}
\item {Utilização:trasm.}
\end{itemize}
\begin{itemize}
\item {Proveniência:(De um supposto \textunderscore rabular\textunderscore , de \textunderscore rábula\textunderscore ?)}
\end{itemize}
Reprehender em voz alta.
Barafustar; fazer gritaria.
Ostentar fôrça; provocar.
\section{Ralhista}
\begin{itemize}
\item {Grp. gram.:m.  e  adj.}
\end{itemize}
\begin{itemize}
\item {Utilização:Ant.}
\end{itemize}
\begin{itemize}
\item {Utilização:Pop.}
\end{itemize}
O mesmo que \textunderscore ralhador\textunderscore .
\section{Ralho}
\begin{itemize}
\item {Grp. gram.:m.}
\end{itemize}
Acto de ralhar; discussão acalorada.
\section{Ralice}
\begin{itemize}
\item {Grp. gram.:f.}
\end{itemize}
\begin{itemize}
\item {Utilização:Prov.}
\end{itemize}
O mesmo que \textunderscore ralação\textunderscore .
\section{Ralidade}
\begin{itemize}
\item {Grp. gram.:f.}
\end{itemize}
\begin{itemize}
\item {Proveniência:(De \textunderscore ralo\textunderscore ^2)}
\end{itemize}
O mesmo que \textunderscore raridade\textunderscore .
\section{Rallo}
\begin{itemize}
\item {Grp. gram.:m.}
\end{itemize}
\begin{itemize}
\item {Proveniência:(Do b. lat. \textunderscore rallus\textunderscore )}
\end{itemize}
Insecto orthóptero, espécie de grillo, nocivo ás raízes das plantas.
\section{Rallo}
\begin{itemize}
\item {Grp. gram.:m.}
\end{itemize}
Antiga embarcação indiana.
\section{Ralo}
\begin{itemize}
\item {Grp. gram.:m.}
\end{itemize}
Utensílio, o mesmo que \textunderscore ralador\textunderscore ; crivo.
O fundo da peneira.
Lâmina, crivada de orifícios, por onde se escôa água e outros líquidos para os encanamentos subterrâneos.
Lâmina, com orifícios, que se adapta ás portas, janelas, confessionários, etc., e através da qual se póde vêr de dentro o que se passa fóra sem se sêr visto.
\section{Ralo}
\begin{itemize}
\item {Grp. gram.:adj.}
\end{itemize}
Pouco espêsso; raro: \textunderscore o milho nasceu ralo\textunderscore .--Relaciona-se com \textunderscore raro\textunderscore . Neste sentido, não era desconhecido o lat. \textunderscore rallus\textunderscore . Veja-se em Plauto a \textunderscore ralla tunica\textunderscore .
\section{Ralo}
\begin{itemize}
\item {Grp. gram.:m.}
\end{itemize}
\begin{itemize}
\item {Proveniência:(Do b. lat. \textunderscore rallus\textunderscore )}
\end{itemize}
Insecto orthóptero, espécie de grillo, nocivo ás raízes das plantas.
\section{Ralo}
\begin{itemize}
\item {Grp. gram.:m.}
\end{itemize}
Antiga embarcação indiana.
\section{Raluto}
\begin{itemize}
\item {Grp. gram.:m.}
\end{itemize}
\begin{itemize}
\item {Utilização:Prov.}
\end{itemize}
\begin{itemize}
\item {Utilização:alg.}
\end{itemize}
\begin{itemize}
\item {Proveniência:(De \textunderscore ralo\textunderscore ^2)}
\end{itemize}
Terreno, em que se perdeu a plantação.
\section{Rama}
\begin{itemize}
\item {Grp. gram.:f.}
\end{itemize}
\begin{itemize}
\item {Utilização:Prov.}
\end{itemize}
\begin{itemize}
\item {Utilização:trasm.}
\end{itemize}
\begin{itemize}
\item {Grp. gram.:Loc. adv.}
\end{itemize}
\begin{itemize}
\item {Proveniência:(De \textunderscore ramo\textunderscore )}
\end{itemize}
Ramos ou fôlhas de uma planta.
Caixilho ou bastidor, em que se estiram os panos quando se fabricam.
Caixilho de ferro, com que os typógraphos apertam as fôrmas da impressão.
Cornadura dos bois.
\textunderscore Em rama\textunderscore , rudimentar, imperfeito, sem preparação, (falando-se do algodão, cêra, etc.).
\textunderscore Pela rama\textunderscore , ao de leve; superficialmente: \textunderscore eu já sabia isso pela rama\textunderscore .
\section{Rã}
\begin{itemize}
\item {Grp. gram.:f.}
\end{itemize}
\begin{itemize}
\item {Utilização:Prov.}
\end{itemize}
\begin{itemize}
\item {Utilização:alent.}
\end{itemize}
\begin{itemize}
\item {Proveniência:(Do lat. \textunderscore rama\textunderscore )}
\end{itemize}
Batracio sem cauda, que vive na água e nos lugares pantanosos.
Variedade de melão.
\section{Ramada}
\begin{itemize}
\item {Grp. gram.:f.}
\end{itemize}
\begin{itemize}
\item {Proveniência:(Do b. lat. \textunderscore ramata\textunderscore )}
\end{itemize}
Rama; ramagem.
Latada, parreira.
Abrigo para o gado vacuum, no campo.
Porção de ramos, dispostos em sebe, para abrigar ou dar sombra.
Antigo processo de pescar, lançando-se muitos ramos na água, para que nelles se acolhesse o peixe.
\section{Ramadan}
\begin{itemize}
\item {Grp. gram.:m.}
\end{itemize}
Nono mês do anno árabe, que os Muçulmanos consagram ao jejum.
(Ár. \textunderscore ramadan\textunderscore )
\section{Ramado}
\begin{itemize}
\item {Grp. gram.:adj.}
\end{itemize}
O mesmo que \textunderscore ramoso\textunderscore .
\section{Ramagem}
\begin{itemize}
\item {Grp. gram.:f.}
\end{itemize}
\begin{itemize}
\item {Proveniência:(De \textunderscore ramo\textunderscore )}
\end{itemize}
Ramos de um arvoredo.
Ramo de uma árvore.
Desenho de fôlhas e flôres sôbre um tecido.
\section{Ramal}
\begin{itemize}
\item {Grp. gram.:m.}
\end{itemize}
\begin{itemize}
\item {Proveniência:(Lat. \textunderscore ramale\textunderscore )}
\end{itemize}
Mólho de fios, para fazer cordas.
Lanço secundário de estrada ou caminho de ferro, para pôr em communicação a via principal com certa localidade ou localidades.
Ramificação.
Enfiada.
Borla de barrete.
Extremidade da bésta.
Galeria, que liga os pontos ou obras secundárias de uma mina ou fortaleza.
O mesmo que \textunderscore fiado\textunderscore ^1 ou \textunderscore fêvera\textunderscore . Cf. Herculano, \textunderscore Bobo\textunderscore , 236.
\section{Ramaldeira}
\begin{itemize}
\item {Grp. gram.:f.}
\end{itemize}
Espécie de música e dança populares.
(Provavelmente, de \textunderscore Ramalde\textunderscore , n. p. de uma povoação nos subúrbios do Pôrto)
\section{Ramalhada}
\begin{itemize}
\item {Grp. gram.:f.}
\end{itemize}
Acto ou effeito de ramalhar; sussurro.
Grande porção de ramos.
\section{Ramalhão}
\begin{itemize}
\item {Grp. gram.:m.}
\end{itemize}
\begin{itemize}
\item {Utilização:Prov.}
\end{itemize}
\begin{itemize}
\item {Utilização:trasm.}
\end{itemize}
\begin{itemize}
\item {Grp. gram.:adj.}
\end{itemize}
Ramo grande.
Comprido como um ramo.
\section{Ramalhar}
\begin{itemize}
\item {Grp. gram.:v. t.}
\end{itemize}
\begin{itemize}
\item {Grp. gram.:V. i.}
\end{itemize}
\begin{itemize}
\item {Proveniência:(De \textunderscore ramalho\textunderscore )}
\end{itemize}
Pôr em movimento os ramos de; fazer sussurrar a ramagem de.
Fazer ruído, agitando-se, (falando-se dos ramos das árvores).
Sussurrar com o vento.
\section{Ramalheira}
\begin{itemize}
\item {Grp. gram.:f.}
\end{itemize}
\begin{itemize}
\item {Utilização:Ant.}
\end{itemize}
\begin{itemize}
\item {Proveniência:(De \textunderscore ramo\textunderscore )}
\end{itemize}
O mesmo que \textunderscore ramaria\textunderscore .
Correia ou corda, que evitava que o remo saísse da forquilha.
\section{Ramalheira}
\begin{itemize}
\item {Grp. gram.:adj. f.}
\end{itemize}
\begin{itemize}
\item {Proveniência:(De \textunderscore rama\textunderscore )}
\end{itemize}
Diz-se de uma variedade de batata, avermelhada e oblonga. Cf. \textunderscore Bibl. da G. do Campo\textunderscore , 266.
\section{Ramalhete}
\begin{itemize}
\item {fónica:lhê}
\end{itemize}
\begin{itemize}
\item {Grp. gram.:m.}
\end{itemize}
\begin{itemize}
\item {Utilização:Prov.}
\end{itemize}
\begin{itemize}
\item {Utilização:minh.}
\end{itemize}
\begin{itemize}
\item {Proveniência:(De \textunderscore ramalho\textunderscore )}
\end{itemize}
Pequeno feixe de flôres, reunidas pelos pés.
Conjunto de coisas selectas e de valor especial.
\textunderscore Ramalhete-de-ginja\textunderscore , pauzinho enfeitado de flôres e ginjas, também chamado \textunderscore raposo\textunderscore .
\section{Ramalheteira}
\begin{itemize}
\item {Grp. gram.:f.}
\end{itemize}
Mulhér, que faz ou vende ramalhetes.
\section{Ramalho}
\begin{itemize}
\item {Grp. gram.:m.}
\end{itemize}
Grande ramo, geralmente cortado da árvore.
\section{Ramalhoça}
\begin{itemize}
\item {Grp. gram.:f.}
\end{itemize}
\begin{itemize}
\item {Utilização:Pop.}
\end{itemize}
\begin{itemize}
\item {Proveniência:(De \textunderscore ramalho\textunderscore )}
\end{itemize}
Grande ramalhate; ramalho.
\section{Ramalhudo}
\begin{itemize}
\item {Grp. gram.:adj.}
\end{itemize}
\begin{itemize}
\item {Utilização:Fig.}
\end{itemize}
\begin{itemize}
\item {Proveniência:(De \textunderscore ramalho\textunderscore )}
\end{itemize}
Que tem muita ramagem.
Dividido em muitos ramos.
Que tem muitas phrases e poucas ideias.
Que ramalha.
Que tem pestanas longas e bastas, (falando-se dos olhos).
\section{Ramaria}
\begin{itemize}
\item {Grp. gram.:f.}
\end{itemize}
O mesmo que \textunderscore ramagem\textunderscore .
\section{Rambana}
\begin{itemize}
\item {Grp. gram.:f.}
\end{itemize}
O mesmo que \textunderscore rabana\textunderscore ^1.
\section{Rambla}
\begin{itemize}
\item {Grp. gram.:f.}
\end{itemize}
O mesmo que \textunderscore râmbola\textunderscore .
\section{Râmbola}
\begin{itemize}
\item {Grp. gram.:f.}
\end{itemize}
\begin{itemize}
\item {Utilização:Prov.}
\end{itemize}
O mesmo que \textunderscore râmola\textunderscore .
\section{Rambotim}
\begin{itemize}
\item {Grp. gram.:m.}
\end{itemize}
Espécie de estôfo indiano.
\section{Rambutan}
\begin{itemize}
\item {Grp. gram.:m.}
\end{itemize}
Planta fructífera do Brasil, (\textunderscore nephelium lappaceum\textunderscore , Lin.).
\section{Rameira}
\begin{itemize}
\item {Grp. gram.:f.}
\end{itemize}
\begin{itemize}
\item {Utilização:T. da Bairrada}
\end{itemize}
\begin{itemize}
\item {Utilização:Prov.}
\end{itemize}
\begin{itemize}
\item {Utilização:minh.}
\end{itemize}
\begin{itemize}
\item {Proveniência:(De \textunderscore ramo\textunderscore )}
\end{itemize}
Meretriz; mulhér pública.
Urze ou queiró, de que se fazem vassoiras.
Ramo grande.
\section{Rameiro}
\begin{itemize}
\item {Grp. gram.:adj.}
\end{itemize}
\begin{itemize}
\item {Grp. gram.:M.}
\end{itemize}
\begin{itemize}
\item {Utilização:Ant.}
\end{itemize}
\begin{itemize}
\item {Proveniência:(De \textunderscore ramo\textunderscore )}
\end{itemize}
Que anda de ramo em ramo para ensaiar o vôo.
Aquelle que arremata ramos de um contrato.
Gavião, que se colhia já crescido, para a caça de altanaria.
\section{Ramela}
\textunderscore f.\textunderscore  (e der.)
O mesmo que \textunderscore remela\textunderscore :«\textunderscore mulheres encapuchadas, muito ramelosas...\textunderscore »Camillo, \textunderscore Brasileira\textunderscore , 339.
\section{Ramentos}
\begin{itemize}
\item {Grp. gram.:m. pl.}
\end{itemize}
\begin{itemize}
\item {Utilização:Ant.}
\end{itemize}
\begin{itemize}
\item {Proveniência:(Lat. \textunderscore ramentum\textunderscore )}
\end{itemize}
Partículas; fragmentos.
\section{Râmeo}
\begin{itemize}
\item {Grp. gram.:adj.}
\end{itemize}
\begin{itemize}
\item {Utilização:Bot.}
\end{itemize}
\begin{itemize}
\item {Proveniência:(Lat. \textunderscore rameus\textunderscore )}
\end{itemize}
Que nasce nos ramos das plantas, (falando-se de raízes, flôres, etc.).
\section{Ramerraneiro}
\begin{itemize}
\item {Grp. gram.:adj.}
\end{itemize}
Relativo ao ramerrão.
Opposto á theoria do progresso.
Commum, vulgar. Cf. Camillo, \textunderscore Cav. em Ruínas\textunderscore , 201.
\section{Ramerrão}
\begin{itemize}
\item {Grp. gram.:m.}
\end{itemize}
\begin{itemize}
\item {Utilização:Ext.}
\end{itemize}
Ruído successivo e monótono.
Uso constante, rotina.
(Or. ind., seg. G. Viana)
\section{Rami}
\begin{itemize}
\item {Grp. gram.:m.}
\end{itemize}
Planta urticácea, (\textunderscore urtica utilis\textunderscore ).
\section{Ramificação}
\begin{itemize}
\item {Grp. gram.:f.}
\end{itemize}
\begin{itemize}
\item {Utilização:Fig.}
\end{itemize}
Acto ou effeito de ramificar.
Cada um dos ramos, que partem do caule.
Conjunto dêsses ramos.
Propagação, diffusão.
\section{Ramificar}
\begin{itemize}
\item {Grp. gram.:v. t.}
\end{itemize}
\begin{itemize}
\item {Proveniência:(Do lat. \textunderscore ramus\textunderscore  + \textunderscore facere\textunderscore )}
\end{itemize}
Dividir em ramos; dividir; sub-dividir.
\section{Ramifloro}
\begin{itemize}
\item {Grp. gram.:adj.}
\end{itemize}
\begin{itemize}
\item {Utilização:Bot.}
\end{itemize}
\begin{itemize}
\item {Proveniência:(Do lat. \textunderscore ramus\textunderscore  + \textunderscore flos\textunderscore , \textunderscore floris\textunderscore )}
\end{itemize}
Que nasce sôbre os ramos, (falando-se de flôres).
\section{Ramiforme}
\begin{itemize}
\item {Grp. gram.:adj.}
\end{itemize}
\begin{itemize}
\item {Proveniência:(Do lat. \textunderscore ramus\textunderscore  + \textunderscore forma\textunderscore )}
\end{itemize}
Que tem fórma de ramo.
\section{Ramilhete}
\begin{itemize}
\item {fónica:lhê}
\end{itemize}
\begin{itemize}
\item {Grp. gram.:m.}
\end{itemize}
\begin{itemize}
\item {Proveniência:(De \textunderscore ramilho\textunderscore )}
\end{itemize}
O mesmo que \textunderscore ramalhete\textunderscore :«\textunderscore ...de ramilhetes o viminoso cesto\textunderscore ». Castilho, \textunderscore Fastos\textunderscore , II, 153.
\section{Ramilho}
\begin{itemize}
\item {Grp. gram.:m.}
\end{itemize}
\begin{itemize}
\item {Utilização:Prov.}
\end{itemize}
\begin{itemize}
\item {Utilização:trasm.}
\end{itemize}
Ramo pequeno.
\section{Raminho}
\begin{itemize}
\item {Grp. gram.:m.}
\end{itemize}
\begin{itemize}
\item {Utilização:Pop.}
\end{itemize}
Pequeno ramo.
Pequeno ramalhete.
Espécie de jôgo popular.
Pequeno ataque de doença, especialmente de paralysia ou estupor:«\textunderscore ...teve um raminho em criança e ficou aleijadinho\textunderscore ». Júl. Dinís, \textunderscore Pupillas\textunderscore , 112.
\section{Ramíparo}
\begin{itemize}
\item {Grp. gram.:adj.}
\end{itemize}
\begin{itemize}
\item {Proveniência:(Do lat. \textunderscore ramus\textunderscore  + \textunderscore parere\textunderscore )}
\end{itemize}
Que produz ramos.
\section{Ramisco}
\begin{itemize}
\item {Grp. gram.:m.}
\end{itemize}
\begin{itemize}
\item {Proveniência:(De \textunderscore ramo\textunderscore )}
\end{itemize}
Uva tinta, que constitue a base do vinho de Collares.
\section{Ramo}
\begin{itemize}
\item {Grp. gram.:m.}
\end{itemize}
\begin{itemize}
\item {Utilização:Bot.}
\end{itemize}
\begin{itemize}
\item {Utilização:Ext.}
\end{itemize}
\begin{itemize}
\item {Grp. gram.:Loc.}
\end{itemize}
\begin{itemize}
\item {Utilização:fam.}
\end{itemize}
\begin{itemize}
\item {Grp. gram.:Pl.}
\end{itemize}
\begin{itemize}
\item {Proveniência:(Lat. \textunderscore ramus\textunderscore )}
\end{itemize}
Divisão e subdivisão de um tronco, ou de um caule, ou braço que irrompe de um tronco ou caule.
Ramalhete.
Divisão, ramificação.
Grupo ou lote de coisas arrematadas em leilão.
Ornamento.
Cada uma das partes ou panos que constituem um lençol.
Cada um dos lanços da urdideira.
Cada uma das famílicas procedentes do mesmo tronco.
Descendente.
Descendência.
Ataque de doença: \textunderscore um ramo de estupor\textunderscore .
Magote.
\textunderscore Não pôr pé em ramo verde\textunderscore , não têr momento de descanso, não têr meios de acção.
\textunderscore Ramo de ar\textunderscore , doença repentina, produzida por corrente de ar.
\textunderscore Ramo de estupor\textunderscore , ataque de paralysia parcial, que affecta especialmente a bôca e os olhos.
Festividade religiosa, em commemoração da entrada de Christo em Jerusalém.
\section{Râmola}
\begin{itemize}
\item {Grp. gram.:f.}
\end{itemize}
Série de quadros de madeira ou de ferro, guarnecidos de escápulas, onde se estendem as peças de estôfo para secarem ao sol, nas fábricas de lanifícios.
\section{Ramonadeira}
\begin{itemize}
\item {Grp. gram.:f.}
\end{itemize}
\begin{itemize}
\item {Proveniência:(Do fr. \textunderscore ramoner\textunderscore )}
\end{itemize}
Instrumento de ferro, para desbastar pelles.
\section{Ramónia}
\begin{itemize}
\item {Grp. gram.:f.}
\end{itemize}
Espécie de videira do Brasil.
\section{Ramosidade}
\begin{itemize}
\item {Grp. gram.:f.}
\end{itemize}
Qualidade daquillo que é ramoso.
\section{Ramoso}
\begin{itemize}
\item {Grp. gram.:adj.}
\end{itemize}
\begin{itemize}
\item {Proveniência:(Lat. \textunderscore ramosus\textunderscore )}
\end{itemize}
Que tem ramos; ramalhudo.
Espêsso e longo, (falando-se de pestanas). Cf. Rebello, \textunderscore Mocidade\textunderscore , II, 120.
\section{Rampa}
\begin{itemize}
\item {Grp. gram.:f.}
\end{itemize}
\begin{itemize}
\item {Proveniência:(Fr. \textunderscore rampe\textunderscore )}
\end{itemize}
Plano inclinado; ladeira.
Palco; ribalta.
\textunderscore Pôr na rampa\textunderscore , tornar evidente ou público. Cf. Camillo, \textunderscore Hist. e Sentiment.\textunderscore , 10.
\section{Rampadoiro}
\begin{itemize}
\item {Grp. gram.:m.}
\end{itemize}
O mesmo que \textunderscore arrampadoiro\textunderscore .
\section{Rampadouro}
\begin{itemize}
\item {Grp. gram.:m.}
\end{itemize}
O mesmo que \textunderscore arrampadouro\textunderscore .
\section{Rampanar}
\begin{itemize}
\item {Grp. gram.:v. t.  e  i.}
\end{itemize}
\begin{itemize}
\item {Utilização:Prov.}
\end{itemize}
\begin{itemize}
\item {Utilização:beir.}
\end{itemize}
Namorar, galantear, fazer côrte.
(Talvez do b. lat. \textunderscore ripanare\textunderscore )
\section{Rampante}
\begin{itemize}
\item {Grp. gram.:adj.}
\end{itemize}
\begin{itemize}
\item {Utilização:Heráld.}
\end{itemize}
\begin{itemize}
\item {Proveniência:(Fr. \textunderscore rampant\textunderscore )}
\end{itemize}
Diz-se, do quadrúpede, erguido sôbre as patas traseiras, tendo a cabeça voltada para o lado direito do escudo.
\section{Rampear}
\begin{itemize}
\item {Grp. gram.:v. t.}
\end{itemize}
Cortar em rampa ou declive (um terreno): \textunderscore ...rampear bem os desaterros da ferrovia, para se não esboroarem sôbre os carris\textunderscore .
\section{Ramudo}
\begin{itemize}
\item {Grp. gram.:adj.}
\end{itemize}
\begin{itemize}
\item {Proveniência:(De \textunderscore ramo\textunderscore )}
\end{itemize}
Ramoso; denso.
\section{Ramúsculo}
\begin{itemize}
\item {Grp. gram.:m.}
\end{itemize}
\begin{itemize}
\item {Proveniência:(Lat. \textunderscore ramusculus\textunderscore )}
\end{itemize}
Pequeno ramo, raminho.
\section{Ramusculoso}
\begin{itemize}
\item {Grp. gram.:adj.}
\end{itemize}
Que tem ou apresenta ramúsculos. Cf. \textunderscore Techn. Rur.\textunderscore , I, 324.
\section{Ran}
\begin{itemize}
\item {Grp. gram.:f.}
\end{itemize}
\begin{itemize}
\item {Utilização:Prov.}
\end{itemize}
\begin{itemize}
\item {Utilização:alent.}
\end{itemize}
\begin{itemize}
\item {Proveniência:(Do lat. \textunderscore rama\textunderscore )}
\end{itemize}
Batracio sem cauda, que vive na água e nos lugares pantanosos.
Variedade de melão.
\section{Ranário}
\begin{itemize}
\item {Grp. gram.:m.}
\end{itemize}
\begin{itemize}
\item {Utilização:Neol.}
\end{itemize}
\begin{itemize}
\item {Proveniência:(Do lat. \textunderscore rana\textunderscore )}
\end{itemize}
Lugar, onde se criam rans.
Construcção, para êsse efeito.
\section{Ranatra}
\begin{itemize}
\item {Grp. gram.:f.}
\end{itemize}
\begin{itemize}
\item {Proveniência:(Do lat. \textunderscore rana\textunderscore )}
\end{itemize}
Gênero de insectos hemípteros.
\section{Ranca}
\begin{itemize}
\item {Grp. gram.:f.}
\end{itemize}
\begin{itemize}
\item {Utilização:Prov.}
\end{itemize}
Ramo ou galho, o mesmo que \textunderscore arranca\textunderscore .
\section{Rançado}
\begin{itemize}
\item {Grp. gram.:adj.}
\end{itemize}
\begin{itemize}
\item {Proveniência:(De \textunderscore rançar\textunderscore )}
\end{itemize}
O mesmo que \textunderscore rançoso\textunderscore .
Que caiu em desuso ou menosprêzo:«\textunderscore ...aquelles perfumes, hoje rançados, a que chamavam poesia\textunderscore ». Castilho, \textunderscore Escarações\textunderscore , 90.
\section{Rancalhão}
\begin{itemize}
\item {Grp. gram.:m.}
\end{itemize}
\begin{itemize}
\item {Utilização:Prov.}
\end{itemize}
Rancalho grande.
\section{Rancalho}
\begin{itemize}
\item {Grp. gram.:m.}
\end{itemize}
\begin{itemize}
\item {Utilização:Prov.}
\end{itemize}
\begin{itemize}
\item {Proveniência:(De \textunderscore ranca\textunderscore )}
\end{itemize}
Ramo de árvore; galho.
\section{Rancanca}
\begin{itemize}
\item {Grp. gram.:f.}
\end{itemize}
Ave de rapina, da América do Sul.
(Onom. do grito da ave)
\section{Rancanho}
\begin{itemize}
\item {Grp. gram.:m.}
\end{itemize}
\begin{itemize}
\item {Utilização:Prov.}
\end{itemize}
\begin{itemize}
\item {Utilização:beir.}
\end{itemize}
O mesmo que \textunderscore rancalho\textunderscore .
\section{Rançar}
\begin{itemize}
\item {Grp. gram.:v. i.}
\end{itemize}
Tomar ranço; tornar-se rançoso.
\section{Rancatrilha}
\begin{itemize}
\item {Grp. gram.:m.}
\end{itemize}
\begin{itemize}
\item {Utilização:Prov.}
\end{itemize}
\begin{itemize}
\item {Utilização:trasm.}
\end{itemize}
\begin{itemize}
\item {Proveniência:(De \textunderscore arrancar\textunderscore  + \textunderscore trilhar\textunderscore ?)}
\end{itemize}
Aquelle que coxeia, arrastando uma perna.
\section{Rancescer}
\begin{itemize}
\item {Grp. gram.:v. i.}
\end{itemize}
\begin{itemize}
\item {Proveniência:(Lat. \textunderscore rancescere\textunderscore )}
\end{itemize}
O mesmo que \textunderscore rançar\textunderscore .
\section{Ranchada}
\begin{itemize}
\item {Grp. gram.:f.}
\end{itemize}
Grande rancho; magote de gente.
\section{Rancharia}
\begin{itemize}
\item {Grp. gram.:f.}
\end{itemize}
\begin{itemize}
\item {Utilização:Bras. de Minas}
\end{itemize}
Grupo de ranchos ou casas tôscas; povoado pobre.
\section{Rancheiro}
\begin{itemize}
\item {Grp. gram.:m.}
\end{itemize}
\begin{itemize}
\item {Grp. gram.:Pl.}
\end{itemize}
\begin{itemize}
\item {Grp. gram.:Adj.}
\end{itemize}
\begin{itemize}
\item {Utilização:Bras. do S}
\end{itemize}
\begin{itemize}
\item {Proveniência:(De \textunderscore rancho\textunderscore )}
\end{itemize}
Aquelle que faz o rancho ou comida para soldados.
Os marinheiros que comem no mesmo prato.
Diz-se do cavalo que tem o habito de parar junto das casas que topa em viagem.
\section{Ranchel}
\begin{itemize}
\item {Grp. gram.:m.}
\end{itemize}
Pequeno rancho.
\section{Rancho}
\begin{itemize}
\item {Grp. gram.:m.}
\end{itemize}
\begin{itemize}
\item {Utilização:Bras}
\end{itemize}
Grupo de pessôas, andando.
Magote de gente.
Reunião de marinheiros, que comem juntos.
Comida que se fornece a soldados e marujos.
Comida, para muitos, para por escote.
Lugar, onde dormem os marinheiros, á prôa.
Choça ou telheiro, á beira dos caminhos, no interior do Brasil, para abrigo de viandantes.
\section{Rancidez}
\begin{itemize}
\item {Grp. gram.:f.}
\end{itemize}
Estado ou qualidade de râncido.
Ranço.
\section{Râncido}
\begin{itemize}
\item {Grp. gram.:adj.}
\end{itemize}
\begin{itemize}
\item {Proveniência:(Lat. \textunderscore rancidus\textunderscore )}
\end{itemize}
O mesmo que \textunderscore rançoso\textunderscore .
\section{Râncio}
\begin{itemize}
\item {Grp. gram.:adj.}
\end{itemize}
O mesmo que \textunderscore rançoso\textunderscore .
\section{Ranco}
\begin{itemize}
\item {Grp. gram.:m.}
\end{itemize}
\begin{itemize}
\item {Utilização:Prov.}
\end{itemize}
O mesmo que \textunderscore ranca\textunderscore .
\section{Ranço}
\begin{itemize}
\item {Grp. gram.:m.}
\end{itemize}
\begin{itemize}
\item {Utilização:Fig.}
\end{itemize}
\begin{itemize}
\item {Proveniência:(Do lat. \textunderscore rancere\textunderscore )}
\end{itemize}
Decomposição ou modificação de uma substância gorda em contacto com o ar; bafio.
Velharia; carácter obsoleto.
\section{Ranço}
\begin{itemize}
\item {Grp. gram.:adj.}
\end{itemize}
O mesmo que \textunderscore rançoso\textunderscore . Cf. Garrett, \textunderscore Fábulas\textunderscore , 64.
(Cp. \textunderscore râncido\textunderscore )
\section{Rancor}
\begin{itemize}
\item {Grp. gram.:m.}
\end{itemize}
\begin{itemize}
\item {Proveniência:(Lat. \textunderscore rancor\textunderscore )}
\end{itemize}
Ódio profundo; grande aversão, não manifestada.
\section{Rancora}
\begin{itemize}
\item {fónica:có}
\end{itemize}
\begin{itemize}
\item {Grp. gram.:f.}
\end{itemize}
\begin{itemize}
\item {Utilização:Ant.}
\end{itemize}
Querela; aggravo.
(Cp. \textunderscore rancor\textunderscore )
\section{Rancorar-se}
\begin{itemize}
\item {Grp. gram.:v. p.}
\end{itemize}
\begin{itemize}
\item {Utilização:Ant.}
\end{itemize}
Apresentar rancora ao juiz; queixar-se judicialmente. Cf. S. R. Viterbo, \textunderscore Elucidário\textunderscore .
\section{Rancorosamente}
\begin{itemize}
\item {Grp. gram.:adj.}
\end{itemize}
De modo rancoroso; com rancor; com ódio profundo.
\section{Rancoroso}
\begin{itemize}
\item {Grp. gram.:adj.}
\end{itemize}
\begin{itemize}
\item {Grp. gram.:M.}
\end{itemize}
\begin{itemize}
\item {Utilização:Ant.}
\end{itemize}
Que tem rancor.
Aquelle que num processo judicial é parte querelante.
\section{Rançosamente}
\begin{itemize}
\item {Grp. gram.:adv.}
\end{itemize}
De modo rançoso; com ranço; com bafio.
\section{Rançoso}
\begin{itemize}
\item {Grp. gram.:adj.}
\end{itemize}
\begin{itemize}
\item {Utilização:Fig.}
\end{itemize}
Que tem ranço; bafiento.
Prolixo.
Antiquado.
Desenxabido.
\section{Rancura}
\begin{itemize}
\item {Grp. gram.:f.}
\end{itemize}
(V.rancora)
\section{Rane}
\begin{itemize}
\item {Grp. gram.:m.}
\end{itemize}
Título honorífico, na Índia Portuguesa.
\section{Rangalheira}
\begin{itemize}
\item {Grp. gram.:f. Loc. adv.}
\end{itemize}
\begin{itemize}
\item {Utilização:Prov.}
\end{itemize}
\begin{itemize}
\item {Utilização:trasm.}
\end{itemize}
\textunderscore Á rangalheira\textunderscore , á larga, a rêgo cheio, á vontade.
(Por \textunderscore regalheira\textunderscore , de \textunderscore rêgo\textunderscore ? Por \textunderscore regaleira\textunderscore , de \textunderscore regalar\textunderscore ?)
\section{Range}
\begin{itemize}
\item {Grp. gram.:m.}
\end{itemize}
\begin{itemize}
\item {Utilização:Prov.}
\end{itemize}
\begin{itemize}
\item {Utilização:trasm.}
\end{itemize}
\begin{itemize}
\item {Proveniência:(De \textunderscore ranger\textunderscore )}
\end{itemize}
Instrumento de rapazes, formado de uma casca de noz, atravessada por um pau, que tem uma maçaneta numa extremidade e na outra uma róda, a que dá impulso uma guita, que sai do centro da noz.
\section{Rangedeira}
\begin{itemize}
\item {Grp. gram.:f.}
\end{itemize}
\begin{itemize}
\item {Proveniência:(De \textunderscore ranger\textunderscore )}
\end{itemize}
Porção de coiro ou de cortiça, que, collocada entre a palmilha e a sola do calçado, produz rangido quando se anda.
Espécie de marreco, (\textunderscore anas querquedula\textunderscore , Lin.).
\section{Rangedor}
\begin{itemize}
\item {Grp. gram.:adj.}
\end{itemize}
Que range.
\section{Rangel}
\begin{itemize}
\item {Grp. gram.:f.}
\end{itemize}
Variedade de pêra.
\section{Rangente}
\begin{itemize}
\item {Grp. gram.:adj.}
\end{itemize}
O mesmo que \textunderscore rangedor\textunderscore .
\section{Ranger}
\begin{itemize}
\item {Grp. gram.:v. i.}
\end{itemize}
\begin{itemize}
\item {Grp. gram.:V. t.}
\end{itemize}
Produzir um ruído aspero, como o de um objecto duro que roça sôbre outro; chiar.
Mover, roçando (os dentes) uns contra os outros.
(Cp. o pop. \textunderscore ringir\textunderscore )
\section{Rangido}
\begin{itemize}
\item {Grp. gram.:m.}
\end{itemize}
Acto ou effeito de ranger.
\section{Rangífer}
\begin{itemize}
\item {Grp. gram.:m.}
\end{itemize}
Mammífero ruminante, o mesmo que \textunderscore renna\textunderscore .
\section{Rangífero}
\begin{itemize}
\item {Grp. gram.:m.}
\end{itemize}
O mesmo ou melhor que \textunderscore rangífer\textunderscore .
\section{Rango}
\begin{itemize}
\item {Grp. gram.:m.}
\end{itemize}
\begin{itemize}
\item {Utilização:Gal}
\end{itemize}
\begin{itemize}
\item {Utilização:Ant.}
\end{itemize}
\begin{itemize}
\item {Proveniência:(Fr. \textunderscore rang\textunderscore )}
\end{itemize}
O mesmo que \textunderscore classe\textunderscore . Cf. \textunderscore Cancíon. da Vaticana\textunderscore .
\section{Rangomela}
\begin{itemize}
\item {Grp. gram.:f.}
\end{itemize}
\begin{itemize}
\item {Utilização:Prov.}
\end{itemize}
Ódio, aversão.
\section{Rangue-rangue}
\begin{itemize}
\item {Grp. gram.:m.}
\end{itemize}
\begin{itemize}
\item {Utilização:Ant.}
\end{itemize}
O mesmo que \textunderscore discussão\textunderscore ? Cf. \textunderscore Eufrosina\textunderscore , 110 e 212.
\section{Ranguinha}
\begin{itemize}
\item {Grp. gram.:m.  e  f.}
\end{itemize}
\begin{itemize}
\item {Utilização:Prov.}
\end{itemize}
\begin{itemize}
\item {Utilização:minh.}
\end{itemize}
Pessôa, que resmunga.
Acto de \textunderscore ranguinhar\textunderscore .
\section{Ranguinhar}
\begin{itemize}
\item {Grp. gram.:v. i.}
\end{itemize}
\begin{itemize}
\item {Utilização:Prov.}
\end{itemize}
\begin{itemize}
\item {Utilização:minh.}
\end{itemize}
O mesmo que \textunderscore resmungar\textunderscore .
Sêr respondão. (Colhido em Barcelos)
\section{Ranha}
\begin{itemize}
\item {Grp. gram.:f.}
\end{itemize}
\begin{itemize}
\item {Utilização:Prov.}
\end{itemize}
\begin{itemize}
\item {Utilização:minh.}
\end{itemize}
Declive no leito de um rio; rapido.
(Cp. \textunderscore ranhura\textunderscore )
\section{Ranhadoiro}
\begin{itemize}
\item {Grp. gram.:m.}
\end{itemize}
\begin{itemize}
\item {Utilização:Prov.}
\end{itemize}
\begin{itemize}
\item {Utilização:trasm.}
\end{itemize}
\begin{itemize}
\item {Proveniência:(De \textunderscore ranhar\textunderscore )}
\end{itemize}
Vassoiro de fôrno.
\section{Ranhadouro}
\begin{itemize}
\item {Grp. gram.:m.}
\end{itemize}
\begin{itemize}
\item {Utilização:Prov.}
\end{itemize}
\begin{itemize}
\item {Utilização:trasm.}
\end{itemize}
\begin{itemize}
\item {Proveniência:(De \textunderscore ranhar\textunderscore )}
\end{itemize}
Vassoiro de fôrno.
\section{Ranhão}
\begin{itemize}
\item {Grp. gram.:m.}
\end{itemize}
\begin{itemize}
\item {Utilização:Prov.}
\end{itemize}
\begin{itemize}
\item {Utilização:trasm.}
\end{itemize}
\begin{itemize}
\item {Proveniência:(De \textunderscore ranhar\textunderscore )}
\end{itemize}
Pau ou fragueiro, com que se varre o forno; ranhadoiro.
Ancinho metallico, com que se junta a prata, no jôgo.
\section{Ranhar}
\begin{itemize}
\item {Utilização:Prov.}
\end{itemize}
\begin{itemize}
\item {Utilização:trasm.}
\end{itemize}
\textunderscore v. t.\textunderscore  (e der.)
O mesmo que \textunderscore arranhar\textunderscore , etc.
Varrer (o forno), para se cozer o pão.
\section{Ranheta}
\begin{itemize}
\item {fónica:nhê}
\end{itemize}
\begin{itemize}
\item {Grp. gram.:m.}
\end{itemize}
\begin{itemize}
\item {Utilização:Bras}
\end{itemize}
Indivíduo impertinente, rabugento.
\section{Ranho}
\begin{itemize}
\item {Grp. gram.:m.}
\end{itemize}
\begin{itemize}
\item {Utilização:Pleb.}
\end{itemize}
Humor mucoso das fossas nasaes; muco.
\section{Ranhoada}
\begin{itemize}
\item {Grp. gram.:f.}
\end{itemize}
\begin{itemize}
\item {Utilização:Ant.}
\end{itemize}
Fressura de carneiro. Cf. Soropita, 61.
\section{Ranhoso}
\begin{itemize}
\item {Grp. gram.:adj.}
\end{itemize}
Que tem ranho.
\section{Ranhura}
\begin{itemize}
\item {Grp. gram.:f.}
\end{itemize}
\begin{itemize}
\item {Proveniência:(Fr. \textunderscore rainure\textunderscore )}
\end{itemize}
Encaixe; escavação ou entalhe na espessura de uma tábua.--É gallicísmo, talvez dispensável.
\section{Ranídeos}
\begin{itemize}
\item {Grp. gram.:m. pl.}
\end{itemize}
\begin{itemize}
\item {Proveniência:(Do lat. \textunderscore rana\textunderscore  + gr. \textunderscore eidos\textunderscore )}
\end{itemize}
Grupo de batrácios, a que pertence a ran.
\section{Rafania}
\begin{itemize}
\item {Grp. gram.:f.}
\end{itemize}
\begin{itemize}
\item {Proveniência:(Do lat. \textunderscore raphanus\textunderscore )}
\end{itemize}
Doença vulgar na Alemanha e na Suíssa, resultante do uso de certos vegetaes de má qualidade.
\section{Rafe}
\begin{itemize}
\item {Grp. gram.:f.}
\end{itemize}
\begin{itemize}
\item {Utilização:Bot.}
\end{itemize}
\begin{itemize}
\item {Proveniência:(Gr. \textunderscore raphe\textunderscore )}
\end{itemize}
Espaço, percorrido pelo trofosperma, desde o hilo do óvulo vegetal até á chalaza.
\section{Ráfidas}
\begin{itemize}
\item {Grp. gram.:f. pl.}
\end{itemize}
O mesmo que \textunderscore ráfides\textunderscore .
\section{Ráfides}
\begin{itemize}
\item {Grp. gram.:m. pl.}
\end{itemize}
\begin{itemize}
\item {Utilização:Bot.}
\end{itemize}
\begin{itemize}
\item {Proveniência:(Do gr. \textunderscore raphis\textunderscore )}
\end{itemize}
Substâncias delgadas, em fórma de agulha, nas células de alguns vegetaes.
\section{Rafídia}
\begin{itemize}
\item {Grp. gram.:f.}
\end{itemize}
\begin{itemize}
\item {Proveniência:(Do gr. \textunderscore raphis\textunderscore  + \textunderscore eidos\textunderscore )}
\end{itemize}
Gênero de insectos neurópteros.
\section{Rafigrafia}
\begin{itemize}
\item {Grp. gram.:f.}
\end{itemize}
Arte de fazer letras com ponteiro ou agulha, no ensino dos cegos.
(Cp. \textunderscore rafígrafo\textunderscore )
\section{Rafigráfico}
\begin{itemize}
\item {Grp. gram.:adj.}
\end{itemize}
Relativo á rafigrafia.
\section{Rafígrafo}
\begin{itemize}
\item {Grp. gram.:m.}
\end{itemize}
\begin{itemize}
\item {Proveniência:(Do gr. \textunderscore raphis\textunderscore , agulha, e \textunderscore graphein\textunderscore , traçar)}
\end{itemize}
Aparelho, formado de uma série de 10 teclas, terminadas em agulhas, que gravam caracteres num papel disposto sôbre uma peça horizontal de platina.
\section{Rafiócera}
\begin{itemize}
\item {Grp. gram.:f.}
\end{itemize}
\begin{itemize}
\item {Proveniência:(Do gr. \textunderscore raphis\textunderscore  + \textunderscore keras\textunderscore )}
\end{itemize}
Gênero de insectos, cuja espécie típica é originária do Brasil.
\section{Rafiolépide}
\begin{itemize}
\item {Grp. gram.:f.}
\end{itemize}
\begin{itemize}
\item {Proveniência:(Do gr. \textunderscore raphis\textunderscore  + \textunderscore lepis\textunderscore )}
\end{itemize}
Gênero de plantas pomáceas da Índia e da China.
\section{Rafionema}
\begin{itemize}
\item {Grp. gram.:f.}
\end{itemize}
Gênero de plantas asclepiadáceas.
\section{Ranilha}
\begin{itemize}
\item {Grp. gram.:f.}
\end{itemize}
\begin{itemize}
\item {Grp. gram.:Pl.}
\end{itemize}
\begin{itemize}
\item {Utilização:Prov.}
\end{itemize}
\begin{itemize}
\item {Utilização:minh.}
\end{itemize}
Saliência molle, na planta do pé de cavallo.
Traseira do carro de bêstas.
Ran verde.
(Cast. \textunderscore ranilla\textunderscore )
\section{Ranina}
\begin{itemize}
\item {Grp. gram.:f.}
\end{itemize}
Mollusco, de casca denteada e triangular, que vive nas grandes profundidades do Oceano.
\section{Ranino}
\begin{itemize}
\item {Grp. gram.:adj.}
\end{itemize}
\begin{itemize}
\item {Proveniência:(Do lat. \textunderscore rana\textunderscore )}
\end{itemize}
Diz-se das veias e artérias, situadas na parte inferior da língua.
\section{Ranóides}
\begin{itemize}
\item {Grp. gram.:f. pl.}
\end{itemize}
Família de plantas, a que, segundo o \textunderscore Thes. da Ling. Port.\textunderscore , pertence a aucuba.
\section{Ran-plan-plan}
\begin{itemize}
\item {Grp. gram.:m.}
\end{itemize}
O mesmo que \textunderscore ran-tan-plan\textunderscore .
\section{Ran-tan-plan}
\begin{itemize}
\item {Grp. gram.:m.}
\end{itemize}
\begin{itemize}
\item {Proveniência:(T. onom.)}
\end{itemize}
Voz imitativa do som do tambor.
\section{Rânula}
\begin{itemize}
\item {Grp. gram.:f.}
\end{itemize}
\begin{itemize}
\item {Proveniência:(Lat. \textunderscore ranula\textunderscore )}
\end{itemize}
Tumor, na parte inferior da língua.
\section{Ranunculáceas}
\begin{itemize}
\item {Grp. gram.:f. pl.}
\end{itemize}
Família de plantas, que tem por typo o ranúnculo.
(Fam. pl. de \textunderscore ranunculáceo\textunderscore )
\section{Ranunculáceo}
\begin{itemize}
\item {Grp. gram.:adj.}
\end{itemize}
Relativo ou semelhante ao ranúnculo.
\section{Ranúnculo}
\begin{itemize}
\item {Grp. gram.:m.}
\end{itemize}
\begin{itemize}
\item {Proveniência:(Lat. \textunderscore ranunculus\textunderscore )}
\end{itemize}
Gênero de plantas, que abrange mais de trezentas espécies, que se encontram em todo o mundo.
\section{Ranzal}
\begin{itemize}
\item {Grp. gram.:m.}
\end{itemize}
Tecido antigo. Cf. Herculano, \textunderscore Bobo\textunderscore , 151.
(Cast. \textunderscore ranzal\textunderscore )
\section{Rapa}
\begin{itemize}
\item {Grp. gram.:m.}
\end{itemize}
\begin{itemize}
\item {Utilização:Fam.}
\end{itemize}
\begin{itemize}
\item {Proveniência:(De \textunderscore rapar\textunderscore )}
\end{itemize}
Jôgo de rapazes, que consiste numa espécie de dado, com quatro faces iguaes, em uma das quaes há a palavra \textunderscore rapa\textunderscore  ou a sua inicial, e nas outras, respectivamente: T(tira), D(deixa), P(põe)
Comilão.
\section{Rapaçaio}
\begin{itemize}
\item {Grp. gram.:m.}
\end{itemize}
\begin{itemize}
\item {Utilização:Mad}
\end{itemize}
(V.rapaceiro)
\section{Rapação}
\begin{itemize}
\item {Grp. gram.:f.}
\end{itemize}
\begin{itemize}
\item {Utilização:Marn.}
\end{itemize}
\begin{itemize}
\item {Proveniência:(De \textunderscore rapar\textunderscore )}
\end{itemize}
Acto de cortar as bimbaduras com o rapão. Cf. \textunderscore Museu Techn.\textunderscore , 104.
\section{Rapace}
\begin{itemize}
\item {Grp. gram.:adj.}
\end{itemize}
\begin{itemize}
\item {Proveniência:(Lat. \textunderscore rapax\textunderscore )}
\end{itemize}
Que rouba; rapinante.
\section{Rapaceiro}
\begin{itemize}
\item {Grp. gram.:m.}
\end{itemize}
\begin{itemize}
\item {Utilização:Mad}
\end{itemize}
O mesmo que \textunderscore tinge-burro\textunderscore .
\section{Rapáceo}
\begin{itemize}
\item {Grp. gram.:adj.}
\end{itemize}
\begin{itemize}
\item {Utilização:Bot.}
\end{itemize}
Que tem fórma de rabam.
\section{Rapacidade}
\begin{itemize}
\item {Grp. gram.:f.}
\end{itemize}
\begin{itemize}
\item {Proveniência:(Lat. \textunderscore rapacitas\textunderscore )}
\end{itemize}
Qualidade do que é repace; tendência ou hábito de roubar.
\section{Rapa-colhér}
\begin{itemize}
\item {Grp. gram.:m.}
\end{itemize}
\begin{itemize}
\item {Utilização:Prov.}
\end{itemize}
\begin{itemize}
\item {Utilização:minh.}
\end{itemize}
O mesmo que \textunderscore gyrino\textunderscore . (Colhido em Famalicão)
\section{Rapadela}
\begin{itemize}
\item {Grp. gram.:f.}
\end{itemize}
Acto ou effeito de rapar.
\section{Rapado}
\begin{itemize}
\item {Grp. gram.:adj.}
\end{itemize}
Diz-se de uma espécie de trigo molle.
Que não tem barba.
Que se barbeou, sem deixar bigode nem outros cabellos da cara.
Que não tem pêlos nenhuns: \textunderscore cara rapada\textunderscore .
\section{Rapadoira}
\begin{itemize}
\item {Grp. gram.:f.}
\end{itemize}
\begin{itemize}
\item {Utilização:Prov.}
\end{itemize}
\begin{itemize}
\item {Utilização:beir.}
\end{itemize}
Pequena pá de ferro, com que a massa do pão se desprende da massadeira, quando se tende a fornada.
(Cp. \textunderscore rapadoiro\textunderscore )
\section{Rapadoiro}
\begin{itemize}
\item {Grp. gram.:m.}
\end{itemize}
\begin{itemize}
\item {Utilização:Bras}
\end{itemize}
\begin{itemize}
\item {Proveniência:(De \textunderscore rapar\textunderscore )}
\end{itemize}
Instrumento, com que se rapa.
Campo, tão limpo de vegetação, que nem serve para pasto.
\section{Rapadoura}
\begin{itemize}
\item {Grp. gram.:f.}
\end{itemize}
\begin{itemize}
\item {Utilização:Prov.}
\end{itemize}
\begin{itemize}
\item {Utilização:beir.}
\end{itemize}
Pequena pá de ferro, com que a massa do pão se desprende da massadeira, quando se tende a fornada.
(Cp. \textunderscore rapadouro\textunderscore )
\section{Rapadouro}
\begin{itemize}
\item {Grp. gram.:m.}
\end{itemize}
\begin{itemize}
\item {Utilização:Bras}
\end{itemize}
\begin{itemize}
\item {Proveniência:(De \textunderscore rapar\textunderscore )}
\end{itemize}
Instrumento, com que se rapa.
Campo, tão limpo de vegetação, que nem serve para pasto.
\section{Rapador}
\begin{itemize}
\item {Grp. gram.:m.}
\end{itemize}
\begin{itemize}
\item {Utilização:Marn.}
\end{itemize}
\begin{itemize}
\item {Proveniência:(De \textunderscore rapar\textunderscore )}
\end{itemize}
Aquelle que trabalha com o rapão. Cf. \textunderscore Museu Techn.\textunderscore , 106.
\section{Rapadura}
\begin{itemize}
\item {Grp. gram.:f.}
\end{itemize}
\begin{itemize}
\item {Utilização:Bras}
\end{itemize}
O mesmo que \textunderscore rapadela\textunderscore .
Açúcar mascavo coagulado, em fórma de pequenos tejolos quadrados, e com que se adoça o café e outras bebidas, sobretudo em viagem.
\section{Rapagão}
\begin{itemize}
\item {Grp. gram.:m.}
\end{itemize}
Rapaz corpulento ou forte.
(Cast. \textunderscore rapagon\textunderscore )
\section{Rapagem}
\begin{itemize}
\item {Grp. gram.:f.}
\end{itemize}
Acto ou effeito de rapar.
\section{Rapalhas}
\begin{itemize}
\item {Grp. gram.:f. pl.}
\end{itemize}
\begin{itemize}
\item {Utilização:Ext.}
\end{itemize}
\begin{itemize}
\item {Proveniência:(De \textunderscore rapar\textunderscore )}
\end{itemize}
Resíduos do estrume, que ficam nos curraes, quando o estrume se levanta.
Bagatela.
\section{Rapa-línguas}
\begin{itemize}
\item {Grp. gram.:f.}
\end{itemize}
\begin{itemize}
\item {Utilização:Bot.}
\end{itemize}
\begin{itemize}
\item {Proveniência:(De \textunderscore rapar\textunderscore  + \textunderscore língua\textunderscore )}
\end{itemize}
Instrumento, com que se limpa a língua.
Erva, de fôlhas ásperas, que se cria principalmente nos vallados.
\section{Rapança}
\begin{itemize}
\item {Grp. gram.:f.}
\end{itemize}
\begin{itemize}
\item {Utilização:Prov.}
\end{itemize}
\begin{itemize}
\item {Utilização:trasm.}
\end{itemize}
\begin{itemize}
\item {Proveniência:(De \textunderscore rapar\textunderscore )}
\end{itemize}
Operação supplementar da redra, nas vinhas do Doiro, a qual consiste em rapar as ervas ou plantas parasitas, que nascem nos lagedos onde se não pôde fazer a cava.
O mesmo que \textunderscore rapadoira\textunderscore .
\section{Rapante}
\begin{itemize}
\item {Grp. gram.:adj.}
\end{itemize}
Que rapa.
\section{Rapão}
\begin{itemize}
\item {Grp. gram.:m.}
\end{itemize}
\begin{itemize}
\item {Utilização:Des.}
\end{itemize}
\begin{itemize}
\item {Utilização:Pop.}
\end{itemize}
\begin{itemize}
\item {Utilização:Prov.}
\end{itemize}
\begin{itemize}
\item {Utilização:minh.}
\end{itemize}
\begin{itemize}
\item {Proveniência:(De \textunderscore rapar\textunderscore )}
\end{itemize}
Aquelle que varre ou ajunta lixo para estrumar.
Utensílio de marnoto, para cortar as bimbaduras.
O mesmo que \textunderscore raponeiro\textunderscore .
Detritos ou ervagens, que á enxada se rapam nas boiças.
\section{Ràpapé}
\begin{itemize}
\item {Grp. gram.:m.}
\end{itemize}
\begin{itemize}
\item {Utilização:Pop.}
\end{itemize}
\begin{itemize}
\item {Proveniência:(De \textunderscore rapar\textunderscore  + \textunderscore pé\textunderscore )}
\end{itemize}
Acto de arrastar o pé para trás, cumprimentando.
Acto de lisonjear; bajulação.
\section{Rapar}
\begin{itemize}
\item {Grp. gram.:v. t.}
\end{itemize}
\begin{itemize}
\item {Utilização:Prov.}
\end{itemize}
\begin{itemize}
\item {Utilização:minh.}
\end{itemize}
\begin{itemize}
\item {Utilização:Pop.}
\end{itemize}
\begin{itemize}
\item {Utilização:Fig.}
\end{itemize}
\begin{itemize}
\item {Grp. gram.:V. p.}
\end{itemize}
\begin{itemize}
\item {Proveniência:(Do germ. \textunderscore rapôn\textunderscore , arrebatar)}
\end{itemize}
Raspar; desgastar, cortando.
Cortar com enxada ervagens em; raspar ou tirar com a enxada bosta ou outros detritos em.
Furtar; roubar; extorquir ardilosamente.
Matar.
Barbear-se.
Cortar o cabello.
\section{Rapariga}
\begin{itemize}
\item {Grp. gram.:f.}
\end{itemize}
\begin{itemize}
\item {Utilização:T. do Amazonas}
\end{itemize}
\begin{itemize}
\item {Utilização:T. do Ceará}
\end{itemize}
Mulhér moça.
Moça de campo, moça rústica.
Mulhér, que está no período intermédio da infância e da adolescência ou já na adolescência.
O mesmo que \textunderscore donzella\textunderscore .
O mesmo que \textunderscore amásia\textunderscore ; meretriz.
(Fem. de \textunderscore rapaz\textunderscore ^1)
\section{Raparigaça}
\begin{itemize}
\item {Grp. gram.:f.}
\end{itemize}
Rapariga robusta e airosa. Cf. Camillo, \textunderscore Brasileira\textunderscore , 191.
\section{Raparigada}
\begin{itemize}
\item {Grp. gram.:f.}
\end{itemize}
\begin{itemize}
\item {Utilização:Fam.}
\end{itemize}
Porção de raparigas; rancho de raparigas. Cf. Júl Dinis, \textunderscore Fidalgos\textunderscore , I, 135.
\section{Raparigagem}
\begin{itemize}
\item {Grp. gram.:f.}
\end{itemize}
\begin{itemize}
\item {Utilização:Prov.}
\end{itemize}
\begin{itemize}
\item {Utilização:alent.}
\end{itemize}
O mesmo que \textunderscore raparigada\textunderscore .
\section{Raparigão}
\begin{itemize}
\item {Grp. gram.:m.}
\end{itemize}
Rapariga encorpada.
\section{Raparigo}
\begin{itemize}
\item {Grp. gram.:m.}
\end{itemize}
\begin{itemize}
\item {Utilização:T. de Melgaço e Moncorvo}
\end{itemize}
O mesmo que \textunderscore rapaz\textunderscore ^1. Cf. Castilho, \textunderscore Fastos\textunderscore , I, 573.
\section{Raparigota}
\begin{itemize}
\item {Grp. gram.:f.}
\end{itemize}
\begin{itemize}
\item {Proveniência:(De \textunderscore raparigo\textunderscore )}
\end{itemize}
Rapariga, moçoila.
\section{Raparigueiro}
\begin{itemize}
\item {Grp. gram.:adj.}
\end{itemize}
\begin{itemize}
\item {Utilização:Bras. do N}
\end{itemize}
O mesmo que \textunderscore femeeiro\textunderscore .
\section{Raparugo}
\begin{itemize}
\item {Grp. gram.:m.}
\end{itemize}
\begin{itemize}
\item {Utilização:T. de Miranda}
\end{itemize}
O mesmo que \textunderscore rapaz\textunderscore ^1.
\section{Rapa-tachos}
\begin{itemize}
\item {Grp. gram.:m.  e  f.}
\end{itemize}
\begin{itemize}
\item {Utilização:fam.}
\end{itemize}
\begin{itemize}
\item {Utilização:Pop.}
\end{itemize}
Pessôa, que come muito, aproveitando o que fica nos pratos, tachos, etc.
\section{Rapaxa}
\begin{itemize}
\item {Grp. gram.:f.}
\end{itemize}
\begin{itemize}
\item {Utilização:T. de Melgaço}
\end{itemize}
O mesmo que \textunderscore rapariga\textunderscore .
(Alter. de \textunderscore rapaza\textunderscore )
\section{Rapaz}
\begin{itemize}
\item {Grp. gram.:m.}
\end{itemize}
\begin{itemize}
\item {Utilização:Bras}
\end{itemize}
\begin{itemize}
\item {Utilização:Bras. do N}
\end{itemize}
Homem, que está no período intermédio da infância e da adolescência, ou já na adolescência.
Moço; garoto.
Preto, de pouca idade.
Criado.
\section{Rapaz}
\begin{itemize}
\item {Grp. gram.:adj.}
\end{itemize}
O mesmo que \textunderscore rapace\textunderscore .
\section{Rapaza}
\begin{itemize}
\item {Grp. gram.:f.}
\end{itemize}
\begin{itemize}
\item {Utilização:T. de Miranda e}
\end{itemize}
\begin{itemize}
\item {Utilização:ant.}
\end{itemize}
O mesmo que \textunderscore rapariga\textunderscore . Cf. \textunderscore Aulegraphia\textunderscore , 154.
(Fem. de \textunderscore rapaz\textunderscore )
\section{Rapazada}
\begin{itemize}
\item {Grp. gram.:f.}
\end{itemize}
O mesmo que \textunderscore rapaziada\textunderscore . Cf. O'Neill, \textunderscore Fab.\textunderscore , 323 e 349.
\section{Rapazão}
\begin{itemize}
\item {Grp. gram.:m.}
\end{itemize}
\begin{itemize}
\item {Utilização:Prov.}
\end{itemize}
O mesmo que \textunderscore rapagão\textunderscore .
\section{Rapazelho}
\begin{itemize}
\item {fónica:zê}
\end{itemize}
\begin{itemize}
\item {Grp. gram.:m.}
\end{itemize}
\begin{itemize}
\item {Utilização:Deprec.}
\end{itemize}
Pequeno rapaz; gaiato.
\section{Rapazete}
\begin{itemize}
\item {fónica:zê}
\end{itemize}
\begin{itemize}
\item {Grp. gram.:m.}
\end{itemize}
O mesmo que \textunderscore rapazelho\textunderscore .
\section{Rapazia}
\begin{itemize}
\item {Grp. gram.:f.}
\end{itemize}
\begin{itemize}
\item {Utilização:Ant.}
\end{itemize}
\begin{itemize}
\item {Proveniência:(De \textunderscore rapaz\textunderscore ^2. O \textunderscore Elucidário\textunderscore  deriva o termo de \textunderscore rapaz\textunderscore ^1, o que não é plausível)}
\end{itemize}
Detrimento, damno.
Subtracção violenta.
\section{Rapazia}
\begin{itemize}
\item {Grp. gram.:f.}
\end{itemize}
\begin{itemize}
\item {Utilização:P. us.}
\end{itemize}
O mesmo que \textunderscore rapazio\textunderscore .
O mesmo que \textunderscore rapaziada\textunderscore .
\section{Rapaziada}
\begin{itemize}
\item {Grp. gram.:f.}
\end{itemize}
\begin{itemize}
\item {Utilização:Ext.}
\end{itemize}
\begin{itemize}
\item {Proveniência:(De \textunderscore rapaz\textunderscore ^1)}
\end{itemize}
Reunião de rapazes.
Acto ou dito, próprio de rapazes.
Procedimento impensado.
\section{Rapazice}
\begin{itemize}
\item {Grp. gram.:f.}
\end{itemize}
O mesmo que \textunderscore rapaziada\textunderscore . Cf. Camillo, \textunderscore Cav. em Ruínas\textunderscore , 246.
\section{Rapazinho}
\begin{itemize}
\item {Grp. gram.:m.}
\end{itemize}
\begin{itemize}
\item {Proveniência:(De \textunderscore rapaz\textunderscore ^1)}
\end{itemize}
O mesmo que \textunderscore menino\textunderscore .
\section{Rapazinhos}
\begin{itemize}
\item {Grp. gram.:m. pl.}
\end{itemize}
\begin{itemize}
\item {Utilização:Bot.}
\end{itemize}
Variedade de orchídea, (\textunderscore aceras anthropophora\textunderscore , Lin.).
\section{Rapazio}
\begin{itemize}
\item {Grp. gram.:m.}
\end{itemize}
Ajuntamento de rapazes; os rapazes em geral.
\section{Rapazo}
\begin{itemize}
\item {Grp. gram.:m.}
\end{itemize}
\begin{itemize}
\item {Utilização:T. de Melgaço}
\end{itemize}
O mesmo que \textunderscore rapaz\textunderscore ^1.
\section{Rapazola}
\begin{itemize}
\item {Grp. gram.:m.}
\end{itemize}
Rapaz já crescido; homem que procede como rapaz.
\section{Rapazota}
\begin{itemize}
\item {Grp. gram.:f.}
\end{itemize}
\begin{itemize}
\item {Utilização:Prov.}
\end{itemize}
\begin{itemize}
\item {Utilização:trasm.}
\end{itemize}
Rapariga brincalhona. (Colhido em S. Marta)
\section{Rapazote}
\begin{itemize}
\item {Grp. gram.:m.}
\end{itemize}
O mesmo que \textunderscore rapazelho\textunderscore .
\section{Rapé}
\begin{itemize}
\item {Grp. gram.:m.}
\end{itemize}
\begin{itemize}
\item {Proveniência:(Fr. \textunderscore râpé\textunderscore )}
\end{itemize}
Tabaco em pó, para se cheirar.
\section{Rapeira}
\begin{itemize}
\item {Grp. gram.:f.}
\end{itemize}
Adubo de terras, constituído por pequenos mexilhões e plantas marinhas, que se apanham nas fendas dos rochedos.
(Cp. \textunderscore rapalhas\textunderscore )
\section{Rapelha}
\begin{itemize}
\item {fónica:pê}
\end{itemize}
\begin{itemize}
\item {Grp. gram.:m.}
\end{itemize}
\begin{itemize}
\item {Utilização:Prov.}
\end{itemize}
\begin{itemize}
\item {Utilização:alent.}
\end{itemize}
O mesmo que \textunderscore bicha-cadela\textunderscore .
\section{Raphania}
\begin{itemize}
\item {Grp. gram.:f.}
\end{itemize}
\begin{itemize}
\item {Proveniência:(Do lat. \textunderscore raphanus\textunderscore )}
\end{itemize}
Doença vulgar na Alemanha e na Suíssa, resultante do uso de certos vegetaes de má qualidade.
\section{Raphe}
\begin{itemize}
\item {Grp. gram.:f.}
\end{itemize}
\begin{itemize}
\item {Utilização:Bot.}
\end{itemize}
\begin{itemize}
\item {Proveniência:(Gr. \textunderscore raphe\textunderscore )}
\end{itemize}
Espaço, percorrido pelo trophosperma, desde o hilo do óvulo vegetal até á chalaza.
\section{Ráphia}
\begin{itemize}
\item {Grp. gram.:f.}
\end{itemize}
\begin{itemize}
\item {Proveniência:(Lat. \textunderscore bot.\textunderscore )}
\end{itemize}
Espécie de palmeira da África e da América, cujas fibras têxteis servem para enxertias e várias liames.
As fibras dessa palmeira: \textunderscore esta menina faz cestinhos de ráphia\textunderscore .
\section{Ráphidas}
\begin{itemize}
\item {Grp. gram.:f. pl.}
\end{itemize}
O mesmo que \textunderscore ráphides\textunderscore .
\section{Ráphides}
\begin{itemize}
\item {Grp. gram.:m. pl.}
\end{itemize}
\begin{itemize}
\item {Utilização:Bot.}
\end{itemize}
\begin{itemize}
\item {Proveniência:(Do gr. \textunderscore raphis\textunderscore )}
\end{itemize}
Substâncias delgadas, em fórma de agulha, nas céllulas de alguns vegetaes.
\section{Raphídia}
\begin{itemize}
\item {Grp. gram.:f.}
\end{itemize}
\begin{itemize}
\item {Proveniência:(Do gr. \textunderscore raphis\textunderscore  + \textunderscore eidos\textunderscore )}
\end{itemize}
Gênero de insectos neurópteros.
\section{Raphigraphia}
\begin{itemize}
\item {Grp. gram.:f.}
\end{itemize}
Arte de fazer letras com ponteiro ou agulha, no ensino dos cegos.
(Cp. \textunderscore raphígrapho\textunderscore )
\section{Raphigráphico}
\begin{itemize}
\item {Grp. gram.:adj.}
\end{itemize}
Relativo á raphigraphia.
\section{Raphígrapho}
\begin{itemize}
\item {Grp. gram.:m.}
\end{itemize}
\begin{itemize}
\item {Proveniência:(Do gr. \textunderscore raphis\textunderscore , agulha, e \textunderscore graphein\textunderscore , traçar)}
\end{itemize}
Apparelho, formado de uma série de 10 teclas, terminadas em agulhas, que gravam caracteres num papel disposto sôbre uma peça horizontal de platina.
\section{Raphiócera}
\begin{itemize}
\item {Grp. gram.:f.}
\end{itemize}
\begin{itemize}
\item {Proveniência:(Do gr. \textunderscore raphis\textunderscore  + \textunderscore keras\textunderscore )}
\end{itemize}
Gênero de insectos, cuja espécie týpica é originária do Brasil.
\section{Raphiolépide}
\begin{itemize}
\item {Grp. gram.:f.}
\end{itemize}
\begin{itemize}
\item {Proveniência:(Do gr. \textunderscore raphis\textunderscore  + \textunderscore lepis\textunderscore )}
\end{itemize}
Gênero de plantas pomáceas da Índia e da China.
\section{Raphionema}
\begin{itemize}
\item {Grp. gram.:f.}
\end{itemize}
Gênero de plantas asclepiadáceas.
\section{Rapiaça}
\begin{itemize}
\item {Grp. gram.:f.}
\end{itemize}
\begin{itemize}
\item {Utilização:Gír.}
\end{itemize}
O mesmo que \textunderscore rapioca\textunderscore .
\section{Rapichel}
\begin{itemize}
\item {Grp. gram.:m.}
\end{itemize}
\begin{itemize}
\item {Utilização:T. de Aveiro}
\end{itemize}
Utensílio, com que o pescador apanha na água a sardinha, que escapou dos sacos da rêde, que se rompeu. Cf. Rev. \textunderscore Tradição\textunderscore , IV, 18.
\section{Rapidamente}
\begin{itemize}
\item {Grp. gram.:adv.}
\end{itemize}
De modo rápido; com rapidez; ligeiramente.
\section{Rapidez}
\begin{itemize}
\item {Grp. gram.:f.}
\end{itemize}
Qualidade do que é \textunderscore rápido\textunderscore ; ligeireza; velocidade.
\section{Rápido}
\begin{itemize}
\item {Grp. gram.:adj.}
\end{itemize}
\begin{itemize}
\item {Grp. gram.:Adv.}
\end{itemize}
\begin{itemize}
\item {Grp. gram.:M.}
\end{itemize}
\begin{itemize}
\item {Proveniência:(Lat. \textunderscore rapidus\textunderscore )}
\end{itemize}
Que percorre grande espaço em pouco tempo.
Que anda depressa; veloz.
Que dura pouco: \textunderscore incômmodo rápido\textunderscore .
Que dura muito ou faz muito em pouco tempo.
Instantâneo.
Com rapidez.
Declive, no leito de um rio.
\section{Rapigar}
\begin{itemize}
\item {Grp. gram.:v. t.}
\end{itemize}
\begin{itemize}
\item {Utilização:Prov.}
\end{itemize}
\begin{itemize}
\item {Utilização:beir.}
\end{itemize}
\begin{itemize}
\item {Proveniência:(De \textunderscore rapar\textunderscore ?)}
\end{itemize}
Tirar com o rapigo a baganha a (o linho).
\section{Rapigo}
\begin{itemize}
\item {Grp. gram.:m.}
\end{itemize}
\begin{itemize}
\item {Utilização:Prov.}
\end{itemize}
\begin{itemize}
\item {Utilização:beir.}
\end{itemize}
\begin{itemize}
\item {Proveniência:(De \textunderscore rapigar\textunderscore )}
\end{itemize}
O mesmo que \textunderscore ripanço\textunderscore ^1, para tirar a baganha do linho.
\section{Rapilho}
\begin{itemize}
\item {Grp. gram.:m.}
\end{itemize}
Pedra vulcânica, fragmentada.
Plantas marítimas, aproveitadas para estrume.
(Cp. \textunderscore rapalhas\textunderscore )
\section{Rapina}
\begin{itemize}
\item {Grp. gram.:f.}
\end{itemize}
\begin{itemize}
\item {Proveniência:(Lat. \textunderscore rapina\textunderscore )}
\end{itemize}
Acto ou effeito de rapinar.
\textunderscore Ave de rapina\textunderscore , ave, caracterizada pelo bico curvo, pelas unhas fortes, etc.
\section{Rapinação}
\begin{itemize}
\item {Grp. gram.:f.}
\end{itemize}
\begin{itemize}
\item {Proveniência:(Lat. \textunderscore rapinatio\textunderscore )}
\end{itemize}
Acto de rapinar.
\section{Rapinador}
\begin{itemize}
\item {Grp. gram.:m.  e  adj.}
\end{itemize}
\begin{itemize}
\item {Proveniência:(Lat. \textunderscore rapinator\textunderscore )}
\end{itemize}
O mesmo que \textunderscore rapinante\textunderscore .
\section{Rapinagem}
\begin{itemize}
\item {Grp. gram.:f.}
\end{itemize}
Conjunto de roubos.
Qualidade do que é rapinante.
Hábito de rapinar.
\section{Rapinança}
\begin{itemize}
\item {Grp. gram.:f.}
\end{itemize}
\begin{itemize}
\item {Utilização:Pop.}
\end{itemize}
O mesmo que \textunderscore rapinagem\textunderscore .
\section{Rapinante}
\begin{itemize}
\item {Grp. gram.:m.  e  adj.}
\end{itemize}
Aquelle que rapina; o que tem o costume de rapinar.
\section{Rapinar}
\begin{itemize}
\item {Grp. gram.:v. t.}
\end{itemize}
\begin{itemize}
\item {Proveniência:(De \textunderscore rapina\textunderscore )}
\end{itemize}
Subtrahir ardilosamente; roubar; tirar com violência.
\section{Rapinhar}
\begin{itemize}
\item {Grp. gram.:v. t.}
\end{itemize}
(Corr. de \textunderscore rapinar\textunderscore )
\section{Rapinice}
\begin{itemize}
\item {Grp. gram.:f.}
\end{itemize}
O mesmo que \textunderscore rapinação\textunderscore .
\section{Rapino}
\begin{itemize}
\item {Grp. gram.:m.}
\end{itemize}
\begin{itemize}
\item {Utilização:Prov.}
\end{itemize}
\begin{itemize}
\item {Utilização:alent.}
\end{itemize}
O mesmo que \textunderscore bicha-cadela\textunderscore .
(Cp. \textunderscore rapelha\textunderscore )
\section{Rapioca}
\begin{itemize}
\item {Grp. gram.:f.}
\end{itemize}
\begin{itemize}
\item {Utilização:Chul.}
\end{itemize}
Pândega; comezana; extravagância.
\section{Rapioqueiro}
\begin{itemize}
\item {Grp. gram.:adj.}
\end{itemize}
\begin{itemize}
\item {Utilização:Chul.}
\end{itemize}
Que gosta de rapioca.
Que é pândego, ou patusco.
\section{Rapistro}
\begin{itemize}
\item {Grp. gram.:m.}
\end{itemize}
\begin{itemize}
\item {Proveniência:(Lat. \textunderscore rapistrum\textunderscore )}
\end{itemize}
Planta crucífera, espécie de rábano silvestre.
\section{Rápito}
\begin{itemize}
\item {Grp. gram.:adj.}
\end{itemize}
\begin{itemize}
\item {Utilização:Prov.}
\end{itemize}
\begin{itemize}
\item {Utilização:trasm.}
\end{itemize}
O mesmo que \textunderscore rápido\textunderscore .
\section{Rapolego}
\begin{itemize}
\item {fónica:lê}
\end{itemize}
\begin{itemize}
\item {Grp. gram.:m.}
\end{itemize}
\begin{itemize}
\item {Utilização:Ant.}
\end{itemize}
Filete retorcido e grosso, ou baínha roliça, á roda das toalhas de limpar o rosto.
\section{Rapôncio}
\begin{itemize}
\item {Grp. gram.:m.}
\end{itemize}
\begin{itemize}
\item {Proveniência:(Do lat. \textunderscore rapa\textunderscore )}
\end{itemize}
Nome de duas plantas campanuláceas.
\section{Raponço}
\begin{itemize}
\item {Grp. gram.:m.}
\end{itemize}
\begin{itemize}
\item {Proveniência:(Do lat. \textunderscore rapa\textunderscore )}
\end{itemize}
Nome de duas plantas campanuláceas.
\section{Raponeiro}
\begin{itemize}
\item {Grp. gram.:m.}
\end{itemize}
\begin{itemize}
\item {Utilização:Des.}
\end{itemize}
O mesmo que \textunderscore barbeiro\textunderscore .
\section{Raponso}
\begin{itemize}
\item {Grp. gram.:m.}
\end{itemize}
\begin{itemize}
\item {Utilização:Ant.}
\end{itemize}
O mesmo que \textunderscore responso\textunderscore . Cf. G. Vicente, I, 254.
\section{Raporte}
\begin{itemize}
\item {Grp. gram.:m.}
\end{itemize}
\begin{itemize}
\item {Utilização:Ant.}
\end{itemize}
\begin{itemize}
\item {Proveniência:(Fr. \textunderscore rapport\textunderscore )}
\end{itemize}
Relatório; informação.
\section{Raposa}
\begin{itemize}
\item {Grp. gram.:f.}
\end{itemize}
\begin{itemize}
\item {Utilização:Fam.}
\end{itemize}
\begin{itemize}
\item {Utilização:Náut.}
\end{itemize}
\begin{itemize}
\item {Utilização:Fig.}
\end{itemize}
Animal carnívoro, (\textunderscore canis vulpes\textunderscore ).
Reprovação num exame.
Cesto cylíndrico, usado na vindima.
Fôrro de madeira, debaixo das mesas do traquete.
Pessôa manhosa.
Espécie de jôgo popular.
\section{Raposada}
\begin{itemize}
\item {Grp. gram.:f.}
\end{itemize}
O mesmo que \textunderscore raposeira\textunderscore , somneca. Cf. Castilho, \textunderscore Doente de Scisma\textunderscore , 4.
\section{Raposar}
\begin{itemize}
\item {Grp. gram.:v. i.}
\end{itemize}
\begin{itemize}
\item {Utilização:Prov.}
\end{itemize}
\begin{itemize}
\item {Utilização:trasm.}
\end{itemize}
Não ir á escola, fazer parede.
\section{Raposeira}
\begin{itemize}
\item {Grp. gram.:f.}
\end{itemize}
\begin{itemize}
\item {Utilização:Pop.}
\end{itemize}
Somno tranquillo.
Bem-estar de quem se deita ao sol brando.
Bebedeira.
Cova de raposa.
\section{Raposeiro}
\begin{itemize}
\item {Grp. gram.:adj.}
\end{itemize}
\begin{itemize}
\item {Grp. gram.:M.}
\end{itemize}
\begin{itemize}
\item {Utilização:Prov.}
\end{itemize}
\begin{itemize}
\item {Proveniência:(De \textunderscore raposo\textunderscore )}
\end{itemize}
Manhoso; ardiloso; malicioso.
Aquelle que tem manha ou ronha.
Feixe de raízes, que crescem em fórma de cauda de raposa nos canos de água corrente.
\section{Raposia}
\begin{itemize}
\item {Grp. gram.:f.}
\end{itemize}
\begin{itemize}
\item {Proveniência:(De \textunderscore raposa\textunderscore )}
\end{itemize}
Manha, malícia.
\section{Raposice}
\begin{itemize}
\item {Grp. gram.:f.}
\end{itemize}
O mesmo que \textunderscore raposia\textunderscore .
\section{Raposim}
\begin{itemize}
\item {Grp. gram.:m.}
\end{itemize}
\begin{itemize}
\item {Utilização:Ant.}
\end{itemize}
Cheiro nauseabundo, o mesmo que \textunderscore raposinho\textunderscore .
\section{Raposinha}
\begin{itemize}
\item {Grp. gram.:adj. f.}
\end{itemize}
\begin{itemize}
\item {Utilização:Prov.}
\end{itemize}
\begin{itemize}
\item {Utilização:trasm.}
\end{itemize}
Diz-se da erva rasteira.
\section{Raposinho}
\begin{itemize}
\item {Grp. gram.:m.}
\end{itemize}
\begin{itemize}
\item {Proveniência:(De \textunderscore raposo\textunderscore )}
\end{itemize}
Pequeno raposo.
Cheiro nauseabundo, análogo ao da raposa; catinga.
\section{Raposinhar}
\begin{itemize}
\item {Grp. gram.:v. i.}
\end{itemize}
\begin{itemize}
\item {Proveniência:(De \textunderscore raposa\textunderscore )}
\end{itemize}
Usar de manha ou de astúcia ou de raposice.
\section{Raposino}
\begin{itemize}
\item {Grp. gram.:adj.}
\end{itemize}
Relativo a raposo, raposeiro.
\section{Raposo}
\begin{itemize}
\item {Grp. gram.:m.}
\end{itemize}
\begin{itemize}
\item {Utilização:Fig.}
\end{itemize}
\begin{itemize}
\item {Utilização:Prov.}
\end{itemize}
\begin{itemize}
\item {Utilização:Prov.}
\end{itemize}
\begin{itemize}
\item {Utilização:minh.}
\end{itemize}
Macho da raposa.
Indivíduo manhoso.
Peixe plagióstomo de grande cauda.
Conjunto de raízes, que crescem nos canos de água corrente, e que também se chama \textunderscore raposeiro\textunderscore .
Pauzinho, enfeitado de flôres e ginjas, para regalo de crianças; ramalhete-de-ginjas.
(Cp. \textunderscore raposa\textunderscore )
\section{Raptador}
\begin{itemize}
\item {Grp. gram.:m.  e  adj.}
\end{itemize}
Aquelle que rapta.
\section{Raptar}
\begin{itemize}
\item {Grp. gram.:v. t.}
\end{itemize}
\begin{itemize}
\item {Proveniência:(Lat. \textunderscore raptare\textunderscore )}
\end{itemize}
Arrebatar, roubar.
Cometer o crime de rapto contra: \textunderscore raptar uma menina\textunderscore .
\section{Rapto}
\begin{itemize}
\item {Grp. gram.:m.}
\end{itemize}
\begin{itemize}
\item {Proveniência:(Lat. \textunderscore raptus\textunderscore )}
\end{itemize}
Acto ou effeito de raptar.
Acto de roubar uma mulhér, seduzindo-a ou forçando-a.
Rapina.
Contemplação de coisas mýsticas; êxtase; arroubamento.
\section{Rapto}
\begin{itemize}
\item {Grp. gram.:adj.}
\end{itemize}
\begin{itemize}
\item {Utilização:Poét.}
\end{itemize}
\begin{itemize}
\item {Proveniência:(Lat. \textunderscore raptus\textunderscore )}
\end{itemize}
Rápido; arrebatado. Cf. \textunderscore Lusíadas\textunderscore , X, 96; F. Barreto, \textunderscore Eneida\textunderscore , II, 100; Castilho, \textunderscore Tartufo\textunderscore , 22.
\section{Raptor}
\begin{itemize}
\item {Grp. gram.:m.}
\end{itemize}
\begin{itemize}
\item {Proveniência:(Lat. \textunderscore raptor\textunderscore )}
\end{itemize}
O mesmo que \textunderscore raptador\textunderscore .
\section{Rapume}
\begin{itemize}
\item {Grp. gram.:m.}
\end{itemize}
O mesmo que \textunderscore rapilho\textunderscore .
\section{Rapúncio}
\begin{itemize}
\item {Grp. gram.:m.}
\end{itemize}
O mesmo que \textunderscore raponço\textunderscore .
\section{Raque}
\begin{itemize}
\item {Grp. gram.:m.}
\end{itemize}
\begin{itemize}
\item {Proveniência:(Do ár. \textunderscore ar-raq\textunderscore , sumo)}
\end{itemize}
Licor indiano, misturado com arroz, açúcar e noz de coco.
\section{Raque}
\begin{itemize}
\item {Grp. gram.:f.}
\end{itemize}
O mesmo ou melhor que \textunderscore ráquis\textunderscore .
\section{Raquel}
\begin{itemize}
\item {Grp. gram.:f.}
\end{itemize}
\begin{itemize}
\item {Proveniência:(De \textunderscore Rachel\textunderscore , n. p.)}
\end{itemize}
Planta amarilídea, (\textunderscore amaryllis sarniensis\textunderscore ).
\section{Raqueta}
\begin{itemize}
\item {fónica:quê}
\end{itemize}
\begin{itemize}
\item {Grp. gram.:f.}
\end{itemize}
\begin{itemize}
\item {Utilização:Bras}
\end{itemize}
\begin{itemize}
\item {Proveniência:(Fr. \textunderscore raquette\textunderscore )}
\end{itemize}
Pá, com que se joga o volante ou a péla.
Planta, o mesmo que \textunderscore cardo-palmatória\textunderscore .
\section{Raquialgia}
\begin{itemize}
\item {Grp. gram.:f.}
\end{itemize}
\begin{itemize}
\item {Proveniência:(Do gr. \textunderscore rakhis\textunderscore  + \textunderscore algos\textunderscore )}
\end{itemize}
Dôr aguda em qualquer ponto da espinha dorsal.
\section{Raquicêntese}
\begin{itemize}
\item {Grp. gram.:f.}
\end{itemize}
\begin{itemize}
\item {Utilização:Cir.}
\end{itemize}
\begin{itemize}
\item {Proveniência:(Do gr. \textunderscore rakhis\textunderscore  + \textunderscore kentesis\textunderscore )}
\end{itemize}
Punção lombar, operação que consiste na introdução de uma agulha ou trocarte fino no canal rachidiano, para introduzir algum medicamento ou retirar um pouco de líquido cefálico ou raquidiano.
\section{Raquidiano}
\begin{itemize}
\item {Grp. gram.:adj.}
\end{itemize}
\begin{itemize}
\item {Proveniência:(De \textunderscore ráquis\textunderscore )}
\end{itemize}
Relativo á espinha dorsal.
\section{Ráquis}
\begin{itemize}
\item {Grp. gram.:f.}
\end{itemize}
\begin{itemize}
\item {Utilização:Anat.}
\end{itemize}
\begin{itemize}
\item {Utilização:Bot.}
\end{itemize}
\begin{itemize}
\item {Proveniência:(Do gr. \textunderscore rakhis\textunderscore )}
\end{itemize}
Columna vertebral.
Eixo central da espiga das gramíneas.
\section{Raquisagra}
\begin{itemize}
\item {Grp. gram.:f.}
\end{itemize}
\begin{itemize}
\item {Utilização:Med.}
\end{itemize}
\begin{itemize}
\item {Utilização:Ant.}
\end{itemize}
\begin{itemize}
\item {Proveniência:(Do gr. \textunderscore rakhis\textunderscore  + \textunderscore agra\textunderscore )}
\end{itemize}
Doença de gota no espinhaço.
\section{Raquítico}
\begin{itemize}
\item {Grp. gram.:adj.}
\end{itemize}
\begin{itemize}
\item {Grp. gram.:M.}
\end{itemize}
\begin{itemize}
\item {Proveniência:(Do gr. \textunderscore rakhitis\textunderscore )}
\end{itemize}
Que tem raquitismo.
Indivíduo raquítico; indivíduo enfèzado.
\section{Raquitismo}
\begin{itemize}
\item {Grp. gram.:m.}
\end{itemize}
\begin{itemize}
\item {Utilização:Med.}
\end{itemize}
\begin{itemize}
\item {Utilização:Bot.}
\end{itemize}
\begin{itemize}
\item {Utilização:Fig.}
\end{itemize}
\begin{itemize}
\item {Proveniência:(Do gr. \textunderscore rakhitis\textunderscore )}
\end{itemize}
Perturbação mórbida da nutrição dos tecidos, tendo como resultado a imperfeição ou a suspensão do desenvolvimento do organismo, a deformação do tórax e da ráquis ou do resto do sistema ossoso.
Definhamento ou deformação das plantas, tornando-se a haste curta ou nodosa.
Fraqueza das faculdades intelectuaes ou do senso moral.
\section{Raramente}
\begin{itemize}
\item {Grp. gram.:adv.}
\end{itemize}
De modo raro; raras vezes.
\section{Rarear}
\begin{itemize}
\item {Grp. gram.:v. t.}
\end{itemize}
\begin{itemize}
\item {Grp. gram.:V. i.}
\end{itemize}
Tornar; tornar pouco denso.
Tornar-se raro; tornar-se pouco denso.
\section{Rarefacção}
\begin{itemize}
\item {Grp. gram.:f.}
\end{itemize}
Acto ou effeito de rarefazer.
\section{Rarefaciente}
\begin{itemize}
\item {Grp. gram.:adj.}
\end{itemize}
\begin{itemize}
\item {Proveniência:(Lat. \textunderscore rarefaciens\textunderscore )}
\end{itemize}
Que rarefaz.
\section{Rarefactível}
\begin{itemize}
\item {Grp. gram.:adj.}
\end{itemize}
\begin{itemize}
\item {Proveniência:(Do lat. \textunderscore rarefactus\textunderscore )}
\end{itemize}
Que se póde rarefazer.
\section{Rarefactivo}
\begin{itemize}
\item {Grp. gram.:adj.}
\end{itemize}
O mesmo que \textunderscore rarefaciente\textunderscore .
\section{Rarefacto}
\begin{itemize}
\item {Grp. gram.:adj.}
\end{itemize}
O mesmo que \textunderscore rarefeito\textunderscore .
\section{Rarefactor}
\begin{itemize}
\item {Grp. gram.:adj.}
\end{itemize}
\begin{itemize}
\item {Grp. gram.:M.}
\end{itemize}
Que rarefaz.
Instrumento ou aquillo que serve para rarefazer. Cf. \textunderscore Techn. Rur.\textunderscore , 274.
\section{Rarefazer}
\begin{itemize}
\item {Grp. gram.:v. t.}
\end{itemize}
\begin{itemize}
\item {Proveniência:(Lat. \textunderscore rarefacere\textunderscore )}
\end{itemize}
Rarear.
Tornar menos denso ou menos espêsso.
Dilatar; desgglomerar:«\textunderscore a assembleia rarefez-se.\textunderscore »Camillo, \textunderscore Volcoens\textunderscore , 61.
\section{Rarefeito}
\begin{itemize}
\item {Grp. gram.:adj.}
\end{itemize}
\begin{itemize}
\item {Proveniência:(Do lat. \textunderscore rarefactus\textunderscore )}
\end{itemize}
Que se rarefez.
Menos denso.
\section{Rareira}
\begin{itemize}
\item {Grp. gram.:f.}
\end{itemize}
\begin{itemize}
\item {Proveniência:(De \textunderscore raro\textunderscore )}
\end{itemize}
O mesmo que \textunderscore raleira\textunderscore ^1. Cf. Júl. de Castilho, \textunderscore Manuelinas\textunderscore , 209.
\section{Rarescência}
\begin{itemize}
\item {Grp. gram.:f.}
\end{itemize}
Qualidade ou estado de rarescente.
\section{Rarescente}
\begin{itemize}
\item {Grp. gram.:adj.}
\end{itemize}
\begin{itemize}
\item {Proveniência:(Lat. \textunderscore rarescens\textunderscore )}
\end{itemize}
Que se rarefaz.
\section{Rareza}
\begin{itemize}
\item {Grp. gram.:f.}
\end{itemize}
\begin{itemize}
\item {Proveniência:(Lat. \textunderscore raritas\textunderscore )}
\end{itemize}
Qualidade do que é raro.
Objecto raro.
Sucesso raro.
\section{Raridade}
\begin{itemize}
\item {Grp. gram.:f.}
\end{itemize}
\begin{itemize}
\item {Proveniência:(Lat. \textunderscore raritas\textunderscore )}
\end{itemize}
Qualidade do que é raro.
Objecto raro.
Sucesso raro.
\section{Rarifloro}
\begin{itemize}
\item {Grp. gram.:adj.}
\end{itemize}
\begin{itemize}
\item {Utilização:Bot.}
\end{itemize}
\begin{itemize}
\item {Proveniência:(Do lat. \textunderscore rarus\textunderscore  + \textunderscore flos\textunderscore )}
\end{itemize}
Que tem poucas flôres,
\section{Raripilo}
\begin{itemize}
\item {Grp. gram.:adj.}
\end{itemize}
\begin{itemize}
\item {Proveniência:(Do lat. \textunderscore rarus\textunderscore  + \textunderscore pilus\textunderscore )}
\end{itemize}
Que tem o pêlo raro.
\section{Raro}
\begin{itemize}
\item {Grp. gram.:adj.}
\end{itemize}
\begin{itemize}
\item {Grp. gram.:Adv.}
\end{itemize}
\begin{itemize}
\item {Proveniência:(Do lat. \textunderscore rarus\textunderscore )}
\end{itemize}
Que não é denso denso ou que é pouco denso.
Pouco espêsso: \textunderscore arvoredo raro\textunderscore .
Que acontece poucas vezes: \textunderscore caso raro\textunderscore .
De que há pouco; que não abunda: \textunderscore as libras hoje são raras\textunderscore .
Extraordinário.
O mesmo que \textunderscore raramente\textunderscore .
\section{Raro}
\begin{itemize}
\item {Grp. gram.:m.}
\end{itemize}
\begin{itemize}
\item {Utilização:Pop.}
\end{itemize}
Insecto, o mesmo que \textunderscore rallo\textunderscore ^1.
\section{Rás}
\begin{itemize}
\item {Grp. gram.:m.}
\end{itemize}
(V.arrás)
\section{Rasa}
\begin{itemize}
\item {Grp. gram.:f.}
\end{itemize}
\begin{itemize}
\item {Proveniência:(Lat. \textunderscore rasa\textunderscore )}
\end{itemize}
Medida antiga de sólidos, equivalente a um alqueire proximamente.
Meio alqueire.
Três alqueires (de sal).
Rasoiro.
Determinada porção de linhas manuscritas, contidas numa página, e abrangendo proximamente certo número de letras, segundo uma tabella.
O preço mais baixo: \textunderscore vender pela rasa\textunderscore .
Descrédito.
\section{Rasado}
\begin{itemize}
\item {Grp. gram.:adj.}
\end{itemize}
Nivelado com rasoira.
\section{Rasadura}
\begin{itemize}
\item {Grp. gram.:f.}
\end{itemize}
Acto ou effeito de rasar.
\section{Rasamente}
\begin{itemize}
\item {Grp. gram.:adv.}
\end{itemize}
De modo raso.
\section{Rasante}
\begin{itemize}
\item {Grp. gram.:adj.}
\end{itemize}
Que rasa.
\section{Rasão}
\begin{itemize}
\item {Grp. gram.:m.}
\end{itemize}
\begin{itemize}
\item {Utilização:Prov.}
\end{itemize}
\begin{itemize}
\item {Utilização:minh.}
\end{itemize}
\begin{itemize}
\item {Proveniência:(De \textunderscore rasar\textunderscore )}
\end{itemize}
O mesmo que \textunderscore rasoira\textunderscore .
\section{Rasar}
\begin{itemize}
\item {Grp. gram.:v. t.}
\end{itemize}
\begin{itemize}
\item {Utilização:Gal}
\end{itemize}
\begin{itemize}
\item {Proveniência:(Fr. \textunderscore raser\textunderscore )}
\end{itemize}
Medir com a rasa.
Ajustar ou acertar a medida com a rasoira.
Tornar raso, encher até á borda.
Nivelar.
Tirar o cogulo a.
Igualar.
Tocar de leve, perpassando.
\section{Rasca}
\begin{itemize}
\item {Grp. gram.:f.}
\end{itemize}
\begin{itemize}
\item {Utilização:Pop.}
\end{itemize}
\begin{itemize}
\item {Utilização:Bras}
\end{itemize}
\begin{itemize}
\item {Proveniência:(De \textunderscore rascar\textunderscore )}
\end{itemize}
Rêde de arrastar.
Pequena embarcação de dois mastros e velas latinas.
Quinhão, parte do lucro: \textunderscore levou rasca na assadura\textunderscore , teve parte em certos interesses.
O mesmo que \textunderscore bebedeira\textunderscore .
\section{Rascada}
\begin{itemize}
\item {Grp. gram.:f.}
\end{itemize}
\begin{itemize}
\item {Utilização:Náut.}
\end{itemize}
\begin{itemize}
\item {Utilização:Fam.}
\end{itemize}
\begin{itemize}
\item {Proveniência:(De \textunderscore rasca\textunderscore )}
\end{itemize}
Espécie de rêde, rasca.
O mesmo que \textunderscore enrascadura\textunderscore .
Difficuldade, entalação, apertos.
\section{Rascadeira}
\begin{itemize}
\item {Grp. gram.:f.}
\end{itemize}
\begin{itemize}
\item {Utilização:Bras. do S}
\end{itemize}
\begin{itemize}
\item {Proveniência:(De \textunderscore rascar\textunderscore ^1)}
\end{itemize}
Espécie de pente de ferro, para limpar do pó o pêlo do cavallo.
\section{Rascador}
\begin{itemize}
\item {Grp. gram.:m.}
\end{itemize}
Utensílio de ourives, para rascar.
\section{Rascadura}
\begin{itemize}
\item {Grp. gram.:f.}
\end{itemize}
\begin{itemize}
\item {Proveniência:(De \textunderscore rascar\textunderscore ^1)}
\end{itemize}
Ferimento, produzido por um corpo áspero ou cortante, actuando de lado.
\section{Rascalço}
\begin{itemize}
\item {Grp. gram.:m.}
\end{itemize}
O mesmo que \textunderscore rascasso\textunderscore .
\section{Rascância}
\begin{itemize}
\item {Grp. gram.:f.}
\end{itemize}
Qualidade do vinho que é \textunderscore rascante\textunderscore ; adstringência.
\section{Rascanhão}
\begin{itemize}
\item {Grp. gram.:m.}
\end{itemize}
\begin{itemize}
\item {Utilização:Prov.}
\end{itemize}
\begin{itemize}
\item {Utilização:trasm.}
\end{itemize}
\begin{itemize}
\item {Proveniência:(De \textunderscore rascanhar\textunderscore )}
\end{itemize}
O mesmo que \textunderscore arranhadura\textunderscore .
\section{Rascanhar}
\begin{itemize}
\item {Grp. gram.:v. t.}
\end{itemize}
\begin{itemize}
\item {Utilização:Prov.}
\end{itemize}
\begin{itemize}
\item {Utilização:trasm.}
\end{itemize}
\begin{itemize}
\item {Proveniência:(De \textunderscore rascar\textunderscore ^1)}
\end{itemize}
O mesmo que \textunderscore arranhar\textunderscore ; raspar.
\section{Rascante}
\begin{itemize}
\item {Grp. gram.:m.  e  adj.}
\end{itemize}
\begin{itemize}
\item {Proveniência:(De \textunderscore rascar\textunderscore ^1)}
\end{itemize}
O vinho que é adstringente e deixa na gargante um certo travo.
\section{Rascão}
\begin{itemize}
\item {Grp. gram.:m.}
\end{itemize}
\begin{itemize}
\item {Utilização:Ant.}
\end{itemize}
Vadio, tunante.
Uma das cordas da rêde de pescar.
Pagem.
(Cast. \textunderscore rascón\textunderscore )
\section{Rascar}
\begin{itemize}
\item {Grp. gram.:v. t.}
\end{itemize}
\begin{itemize}
\item {Proveniência:(Do lat. hyp. \textunderscore rasicare\textunderscore )}
\end{itemize}
Raspar, desbastar.
Lascar.
Arranhar.
Escoriar.
\section{Rascar}
\begin{itemize}
\item {Grp. gram.:v. i.}
\end{itemize}
\begin{itemize}
\item {Utilização:Ant.}
\end{itemize}
\begin{itemize}
\item {Utilização:Prov.}
\end{itemize}
\begin{itemize}
\item {Utilização:trasm.}
\end{itemize}
Gritar; chamar alguém em voz alta.
O mesmo que \textunderscore namorar\textunderscore .
\section{Rascasso}
\begin{itemize}
\item {Grp. gram.:m.}
\end{itemize}
Peixe triglídeo, (\textunderscore scorpoemascrapo\textunderscore ).
\section{Rasco}
\begin{itemize}
\item {Grp. gram.:m.}
\end{itemize}
\begin{itemize}
\item {Proveniência:(De \textunderscore rascar\textunderscore )}
\end{itemize}
Garfo de ferro, na extremidade de uma vara, para a apanha do mexilhão.
\section{Rascôa}
\begin{itemize}
\item {Grp. gram.:f.}
\end{itemize}
\begin{itemize}
\item {Utilização:Ant.}
\end{itemize}
Aia.
Cozinheira.
Rameira.
(Fem. de \textunderscore rascão\textunderscore )
\section{Rascoeira}
\begin{itemize}
\item {Grp. gram.:f.}
\end{itemize}
\begin{itemize}
\item {Proveniência:(De \textunderscore rascoeiro\textunderscore )}
\end{itemize}
Mulhér de má nota. Cf. \textunderscore Cancion. Al.\textunderscore , 351.
\section{Rascoeiro}
\begin{itemize}
\item {Grp. gram.:m.}
\end{itemize}
O mesmo que \textunderscore rascão\textunderscore .
\section{Rascol}
\begin{itemize}
\item {Grp. gram.:m.}
\end{itemize}
O mesmo que \textunderscore rascolnismo\textunderscore .
(Russo \textunderscore raskol\textunderscore )
\section{Rascolnismo}
\begin{itemize}
\item {Grp. gram.:m.}
\end{itemize}
\begin{itemize}
\item {Proveniência:(De \textunderscore rascol\textunderscore )}
\end{itemize}
Crença das seitas russas que, em 1659, se separaram da Igreja grega.
\section{Rascolnista}
\begin{itemize}
\item {Grp. gram.:m.}
\end{itemize}
Sectário do rascolnismo.
\section{Rascote}
\begin{itemize}
\item {Grp. gram.:m.}
\end{itemize}
O mesmo que \textunderscore rascão\textunderscore .
\section{Rascunhar}
\begin{itemize}
\item {Grp. gram.:v. t.}
\end{itemize}
Fazer o rascunho de; esboçar.
(Cast. \textunderscore rasguñar\textunderscore )
\section{Rascunho}
\begin{itemize}
\item {Grp. gram.:m.}
\end{itemize}
\begin{itemize}
\item {Proveniência:(De \textunderscore rascunhar\textunderscore )}
\end{itemize}
Delineamento de qualquer escrito.
Esbôço, minuta.
\section{Raseiro}
\begin{itemize}
\item {Grp. gram.:adj.}
\end{itemize}
\begin{itemize}
\item {Utilização:Prov.}
\end{itemize}
\begin{itemize}
\item {Utilização:beir.}
\end{itemize}
\begin{itemize}
\item {Proveniência:(De \textunderscore raso\textunderscore )}
\end{itemize}
Achatado.
Com pouco fundo:«\textunderscore ...as embarcações mais raseiras...\textunderscore »Filinto, \textunderscore D. Man.\textunderscore , II, 60.
Diz-se da medida de cereaes, cujo conteúdo foi perpassado pela rasoira, para que não houvesse cogulo mas medida justa.
\section{Rasgadamente}
\begin{itemize}
\item {Grp. gram.:adv.}
\end{itemize}
De modo rasgado; com desembaraço; claramente; francamente.
\section{Rasgadela}
\begin{itemize}
\item {Grp. gram.:f.}
\end{itemize}
O mesmo que \textunderscore rasgão\textunderscore .
\section{Rasgado}
\begin{itemize}
\item {Grp. gram.:m.  e  adj.}
\end{itemize}
\begin{itemize}
\item {Utilização:Bras}
\end{itemize}
\begin{itemize}
\item {Proveniência:(De \textunderscore rasgar\textunderscore )}
\end{itemize}
Diz-se de um toque de viola, em que se arrastam as unhas ou só o pollegar pelas cordas, sem as pontear.
\section{Rasgador}
\begin{itemize}
\item {Grp. gram.:m.  e  adj.}
\end{itemize}
O que rasga.
\section{Rasgadura}
\begin{itemize}
\item {Grp. gram.:f.}
\end{itemize}
O mesmo que \textunderscore rasgão\textunderscore .
\section{Rasgamento}
\begin{itemize}
\item {Grp. gram.:m.}
\end{itemize}
O mesmo que \textunderscore rasgão\textunderscore .
\section{Rasgão}
\begin{itemize}
\item {Grp. gram.:m.}
\end{itemize}
Acto ou effeito de rasgar; fenda, abertura.
\section{Rasgar}
\begin{itemize}
\item {Grp. gram.:v. t.}
\end{itemize}
\begin{itemize}
\item {Utilização:Carp.}
\end{itemize}
\begin{itemize}
\item {Utilização:Fig.}
\end{itemize}
Abrir fenda em (tecidos, papéis, etc.).
Golpear.
Abrir em pedaços.
Romper com violência.
Lacerar; abrir; cortar.
Abrir com cantil um friso na parte lateral e a meia grossura (de uma tábua), para a ajuntar com outra, especialmente em sobrados.
Apoquentar; torturar.
(Fórma divergente de \textunderscore resgar\textunderscore )(V.resgar)
\section{Rasgo}
\begin{itemize}
\item {Grp. gram.:m.}
\end{itemize}
\begin{itemize}
\item {Utilização:Pop.}
\end{itemize}
\begin{itemize}
\item {Proveniência:(De \textunderscore rasgar\textunderscore )}
\end{itemize}
Rasgão.
Rasgue, córte.
Rasgadura.
Acção exemplar.
Rajada de estilo.
Trecho eloquente.
Desembaraço, energia, expediente: \textunderscore êste rapaz não tem rasgo nenhum\textunderscore .
\section{Rasgue}
\begin{itemize}
\item {Grp. gram.:m.}
\end{itemize}
\begin{itemize}
\item {Utilização:Pop.}
\end{itemize}
\begin{itemize}
\item {Proveniência:(De \textunderscore rasgar\textunderscore )}
\end{itemize}
Abertura; córte; entalhe; encaixe.
\section{Rasina}
\begin{itemize}
\item {Grp. gram.:f.}
\end{itemize}
\begin{itemize}
\item {Utilização:Ant.}
\end{itemize}
\begin{itemize}
\item {Proveniência:(Do lat. \textunderscore rasis\textunderscore )}
\end{itemize}
Pez, que se reduz a pó e se empregava em pharmácia.
\section{Rasmonino}
\begin{itemize}
\item {Grp. gram.:m.}
\end{itemize}
O mesmo que \textunderscore rosmaninho\textunderscore .
\section{Rasmono}
\begin{itemize}
\item {Grp. gram.:m.}
\end{itemize}
\begin{itemize}
\item {Utilização:Prov.}
\end{itemize}
\begin{itemize}
\item {Utilização:alg.}
\end{itemize}
O mesmo que \textunderscore rosmaninho\textunderscore .
\section{Raso}
\begin{itemize}
\item {Grp. gram.:adj.}
\end{itemize}
\begin{itemize}
\item {Utilização:Fig.}
\end{itemize}
\begin{itemize}
\item {Grp. gram.:M.}
\end{itemize}
\begin{itemize}
\item {Utilização:Prov.}
\end{itemize}
\begin{itemize}
\item {Utilização:Gír.}
\end{itemize}
\begin{itemize}
\item {Proveniência:(Lat. \textunderscore rasus\textunderscore )}
\end{itemize}
Liso, plano, polido.
Que corre ao nível de.
Cérceo.
Rasteiro.
Que está cheio até ás bordas; não acogulado: \textunderscore um decalitro raso\textunderscore .
Que não tem lavores.
Que não tem nada escrito: \textunderscore papel raso\textunderscore .
Ordinário, estúpido.
Que não tem graduação, (falando-se de militares): \textunderscore soldado raso\textunderscore .
Diz-se do sapato, que tem entrada baixa ou pequeno tacão.
Campo, planície.
Res, superfície.
Tecido de seda lustrosa e fina.
Papel raso: \textunderscore assinar em público e raso\textunderscore .
Padre, frade.
\section{Rasoeira}
\begin{itemize}
\item {Grp. gram.:f.}
\end{itemize}
\begin{itemize}
\item {Utilização:Prov.}
\end{itemize}
\begin{itemize}
\item {Utilização:trasm.}
\end{itemize}
O mesmo que \textunderscore rasoira\textunderscore .
\section{Rasoila}
\begin{itemize}
\item {Grp. gram.:f.}
\end{itemize}
O mesmo que \textunderscore rasoira\textunderscore .
\section{Rasoilo}
\begin{itemize}
\item {Grp. gram.:m.}
\end{itemize}
\begin{itemize}
\item {Utilização:Prov.}
\end{itemize}
O mesmo que \textunderscore rasoira\textunderscore .
\section{Rasoira}
\begin{itemize}
\item {Grp. gram.:f.}
\end{itemize}
\begin{itemize}
\item {Proveniência:(Do lat. \textunderscore rasoria\textunderscore )}
\end{itemize}
Pau redondo e direito, que serve para tirar o cogulo nas medidas de secos.
Tudo que nivela, arrasando ou desbastando.
Tudo que iguala: \textunderscore a rasoira da morte\textunderscore .
Instrumento de entalhador ou merceneiro, para tirar as asperezas da madeira que se entalha.
Instrumento de gravador, com que se pule o granulado da chapa, no ponto a que devem corresponder os claros do desenho.
\section{Rasoirar}
\begin{itemize}
\item {Grp. gram.:v. t.}
\end{itemize}
\begin{itemize}
\item {Utilização:Fig.}
\end{itemize}
Nivelar com a rasoira.
Igualar.
\section{Rasoiro}
\begin{itemize}
\item {Grp. gram.:m.}
\end{itemize}
\begin{itemize}
\item {Utilização:Prov.}
\end{itemize}
O mesmo que \textunderscore rasoira\textunderscore .
\section{Rasolho}
\begin{itemize}
\item {fónica:sô}
\end{itemize}
\begin{itemize}
\item {Grp. gram.:m.}
\end{itemize}
O mesmo que \textunderscore rasoira\textunderscore .
\section{Raspa}
\begin{itemize}
\item {Grp. gram.:f.}
\end{itemize}
\begin{itemize}
\item {Utilização:Prov.}
\end{itemize}
\begin{itemize}
\item {Utilização:dur.}
\end{itemize}
\begin{itemize}
\item {Proveniência:(De \textunderscore raspar\textunderscore )}
\end{itemize}
Pequena lasca, apara ou qualquer dos fragmentos que se separam de um objecto, raspando-o.
Raspadeira.
Instrumento curvo de ferro, com cabo de madeira, que, nos armazens de vinho, serve para fazer as marcas da qualidade do líquido no bojo e tampo do vasilhame.
\section{Raspa}
\begin{itemize}
\item {Grp. gram.:f.}
\end{itemize}
\begin{itemize}
\item {Utilização:T. de Angola}
\end{itemize}
O mesmo que \textunderscore rebuçado\textunderscore .
\section{Raspadeira}
\begin{itemize}
\item {Grp. gram.:f.}
\end{itemize}
\begin{itemize}
\item {Proveniência:(De \textunderscore raspar\textunderscore )}
\end{itemize}
Instrumento para raspar.
Gênero de plantas urticáceas da Índia Port., (\textunderscore ficus asperrima\textunderscore , Roxb.).
\section{Raspadela}
\begin{itemize}
\item {Grp. gram.:f.}
\end{itemize}
O mesmo que \textunderscore raspagem\textunderscore .
\section{Raspador}
\begin{itemize}
\item {Grp. gram.:adj.}
\end{itemize}
\begin{itemize}
\item {Grp. gram.:M.}
\end{itemize}
Que raspa.
O mesmo que \textunderscore raspadeira\textunderscore , instrumento.
\section{Raspadura}
\begin{itemize}
\item {Grp. gram.:f.}
\end{itemize}
\begin{itemize}
\item {Proveniência:(De \textunderscore raspar\textunderscore )}
\end{itemize}
Raspagem; raspas.
\section{Raspagem}
\begin{itemize}
\item {Grp. gram.:f.}
\end{itemize}
Acto ou effeito de raspar: \textunderscore a raspagem do útero\textunderscore .
\section{Rasoura}
\begin{itemize}
\item {Grp. gram.:f.}
\end{itemize}
\begin{itemize}
\item {Proveniência:(Do lat. \textunderscore rasoria\textunderscore )}
\end{itemize}
Pau redondo e direito, que serve para tirar o cogulo nas medidas de secos.
Tudo que nivela, arrasando ou desbastando.
Tudo que iguala: \textunderscore a rasoura da morte\textunderscore .
Instrumento de entalhador ou merceneiro, para tirar as asperezas da madeira que se entalha.
Instrumento de gravador, com que se pule o granulado da chapa, no ponto a que devem corresponder os claros do desenho.
\section{Rasourar}
\begin{itemize}
\item {Grp. gram.:v. t.}
\end{itemize}
\begin{itemize}
\item {Utilização:Fig.}
\end{itemize}
Nivelar com a rasoura.
Igualar.
\section{Rasouro}
\begin{itemize}
\item {Grp. gram.:m.}
\end{itemize}
\begin{itemize}
\item {Utilização:Prov.}
\end{itemize}
O mesmo que \textunderscore rasoura\textunderscore .
\section{Raspalhista}
\begin{itemize}
\item {Grp. gram.:m.}
\end{itemize}
Sectário da medicina de Raspaial.
\section{Raspalho}
\begin{itemize}
\item {Grp. gram.:m.}
\end{itemize}
\begin{itemize}
\item {Utilização:Prov.}
\end{itemize}
\begin{itemize}
\item {Utilização:alent.}
\end{itemize}
O mesmo que \textunderscore bicha-cadela\textunderscore .
\section{Raspa-língua}
\begin{itemize}
\item {Grp. gram.:f.}
\end{itemize}
Planta rubiácea, (\textunderscore rubia peregrina\textunderscore , Lin.).
\section{Raspançadura}
\begin{itemize}
\item {Grp. gram.:f.}
\end{itemize}
Acto de raspançar; o mesmo que \textunderscore rasura\textunderscore .
\section{Raspançar}
\begin{itemize}
\item {Grp. gram.:v. t.}
\end{itemize}
\begin{itemize}
\item {Utilização:Pop.}
\end{itemize}
O mesmo que \textunderscore raspar\textunderscore .
\section{Raspanço}
\begin{itemize}
\item {Grp. gram.:m.}
\end{itemize}
\begin{itemize}
\item {Utilização:Fam.}
\end{itemize}
Acto de raspançar; raspadura; raspão.
Reprehensão.
\section{Raspão}
\begin{itemize}
\item {Grp. gram.:m.}
\end{itemize}
\begin{itemize}
\item {Proveniência:(De \textunderscore raspar\textunderscore )}
\end{itemize}
Ligeiro ferimento, feito de relance ou de través ou por attrito; arranhadura.
\section{Raspar}
\begin{itemize}
\item {Grp. gram.:v. t.}
\end{itemize}
\begin{itemize}
\item {Utilização:Fig.}
\end{itemize}
\begin{itemize}
\item {Grp. gram.:V. p.}
\end{itemize}
\begin{itemize}
\item {Utilização:Pop.}
\end{itemize}
\begin{itemize}
\item {Proveniência:(Do ant. al. \textunderscore raspon\textunderscore )}
\end{itemize}
Desbastar a superfície de, com instrumento próprio.
Alisar.
Limpar, friccionando.
Tirar, expungir.
Fazer raspão em; arranhar.
Fugir.
\section{Raspilha}
\begin{itemize}
\item {Grp. gram.:f.}
\end{itemize}
\begin{itemize}
\item {Proveniência:(De \textunderscore raspa\textunderscore ^1)}
\end{itemize}
Instrumento de tanoeiro, para raspar aduelas.
\section{Raspinhadeira}
\begin{itemize}
\item {Grp. gram.:f.}
\end{itemize}
\begin{itemize}
\item {Utilização:Marn.}
\end{itemize}
\begin{itemize}
\item {Proveniência:(De \textunderscore raspinhar\textunderscore )}
\end{itemize}
O mesmo que \textunderscore rapão\textunderscore , nas salinas.
\section{Raspinhar}
\begin{itemize}
\item {Grp. gram.:v. t.}
\end{itemize}
\begin{itemize}
\item {Proveniência:(De \textunderscore raspar\textunderscore )}
\end{itemize}
Alisar com a raspinhadeira.
\section{Rasqueta}
\begin{itemize}
\item {fónica:quê}
\end{itemize}
\begin{itemize}
\item {Grp. gram.:f.}
\end{itemize}
\begin{itemize}
\item {Utilização:Náut.}
\end{itemize}
\begin{itemize}
\item {Utilização:Anat.}
\end{itemize}
\begin{itemize}
\item {Utilização:Ant.}
\end{itemize}
\begin{itemize}
\item {Proveniência:(De \textunderscore rascar\textunderscore )}
\end{itemize}
Instrumento para raspar e limpar algumas partes do navio.
O mesmo que \textunderscore carpo\textunderscore .
\section{Rasquetear}
\begin{itemize}
\item {Grp. gram.:v. t.}
\end{itemize}
\begin{itemize}
\item {Utilização:Bras. do S}
\end{itemize}
\begin{itemize}
\item {Proveniência:(De \textunderscore rasqueta\textunderscore )}
\end{itemize}
Limpar com a rascadeira.
\section{Rasquido}
\begin{itemize}
\item {Grp. gram.:m.}
\end{itemize}
\begin{itemize}
\item {Utilização:Prov.}
\end{itemize}
Cisco.
Varredura.
Porcaria.
O mesmo que \textunderscore borralho\textunderscore ^1.
\section{Rassa}
\begin{itemize}
\item {Grp. gram.:f.}
\end{itemize}
\begin{itemize}
\item {Utilização:Prov.}
\end{itemize}
\textunderscore Rassa de sol\textunderscore , réstia de sol ou feixe de luz, que entra pelos buracos dos telhados e das paredes.
(Cp. \textunderscore ressa\textunderscore )
\section{Rastão}
\begin{itemize}
\item {Grp. gram.:m.}
\end{itemize}
\begin{itemize}
\item {Proveniência:(De \textunderscore rasto\textunderscore )}
\end{itemize}
Vara da videira, que na póda se deixa estendida pelo chão.
\section{Rastear}
\begin{itemize}
\item {Grp. gram.:v. t.  e  i.}
\end{itemize}
O mesmo que \textunderscore rastejar\textunderscore .
\section{Rasteira}
\begin{itemize}
\item {Grp. gram.:f.}
\end{itemize}
\begin{itemize}
\item {Utilização:Chul.}
\end{itemize}
\begin{itemize}
\item {Proveniência:(De \textunderscore rasteiro\textunderscore )}
\end{itemize}
O mesmo que \textunderscore cambapé\textunderscore .
\section{Rasteirinha}
\begin{itemize}
\item {Grp. gram.:f.}
\end{itemize}
Planta malvácea do Brasil.
\section{Rasteiro}
\begin{itemize}
\item {Grp. gram.:adj.}
\end{itemize}
\begin{itemize}
\item {Utilização:Bras}
\end{itemize}
\begin{itemize}
\item {Grp. gram.:M.}
\end{itemize}
\begin{itemize}
\item {Proveniência:(Do lat. \textunderscore rastrarius\textunderscore ?)}
\end{itemize}
Que se arrasta pelo chão.
Que anda de rastos.
Que anda ou vôa pouco alto ou próximo da terra.
Que se ergue pouco acima do chão: \textunderscore planta rasteira\textunderscore .
Que cresceu pouco.
Humilde.
Ordinario.
Desprezível.
Diz-se do engenho de açúcar, movido por águas que correm baixo.
Arbusto polygaláceo do Brasil.
\section{Rastejador}
\begin{itemize}
\item {Grp. gram.:m.  e  adj.}
\end{itemize}
O que rasteja.
O que vai no rasto ou na peugada de alguém ou de alguma coisa.
\section{Rastejadura}
\begin{itemize}
\item {Grp. gram.:f.}
\end{itemize}
Acto ou effeito de rastejar.
\section{Rastejante}
\begin{itemize}
\item {Grp. gram.:adj.}
\end{itemize}
\begin{itemize}
\item {Proveniência:(De \textunderscore rastejar\textunderscore )}
\end{itemize}
Rastejador; rasteiro.
\section{Rastejar}
\begin{itemize}
\item {Grp. gram.:v. t.}
\end{itemize}
\begin{itemize}
\item {Grp. gram.:V. i.}
\end{itemize}
\begin{itemize}
\item {Utilização:Fig.}
\end{itemize}
\begin{itemize}
\item {Proveniência:(De \textunderscore rasto\textunderscore )}
\end{itemize}
Seguir o rasto ou as peugadas de.
Investigar.
Arrastar-se pelo chão, andar de rastos.
Abater-se.
Sêr modesto ou baixo nos conceitos ou no estilo.
Têr sentimentos baixos.
\section{Rastejo}
\begin{itemize}
\item {Grp. gram.:m.}
\end{itemize}
Acto de rastejar.
\section{Rastellar}
\begin{itemize}
\item {Grp. gram.:v. t.}
\end{itemize}
Limpar (o linho) com rastello; assedar.
\section{Rastelar}
\begin{itemize}
\item {Grp. gram.:v. t.}
\end{itemize}
Limpar (o linho) com rastelo; assedar.
\section{Rastello}
\begin{itemize}
\item {fónica:tê}
\end{itemize}
\begin{itemize}
\item {Grp. gram.:m.}
\end{itemize}
\begin{itemize}
\item {Proveniência:(Lat. \textunderscore rastellus\textunderscore )}
\end{itemize}
Fileiras de dentes de ferro, por onde se passa o linho para o separar da estopa; sedeiro.
Grade, com dentes de pau, para aplanar a terra lavradia.
\section{Rastelo}
\begin{itemize}
\item {fónica:tê}
\end{itemize}
\begin{itemize}
\item {Grp. gram.:m.}
\end{itemize}
\begin{itemize}
\item {Proveniência:(Lat. \textunderscore rastellus\textunderscore )}
\end{itemize}
Fileiras de dentes de ferro, por onde se passa o linho para o separar da estopa; sedeiro.
Grade, com dentes de pau, para aplanar a terra lavradia.
\section{Rastilha}
\begin{itemize}
\item {Grp. gram.:f.}
\end{itemize}
\begin{itemize}
\item {Utilização:T. da Bairrada}
\end{itemize}
\begin{itemize}
\item {Proveniência:(De \textunderscore rasto\textunderscore )}
\end{itemize}
Instrumento de marceneiro e carpinteiro, para alisar curvas.
O mesmo que \textunderscore corta-chefe\textunderscore  ou \textunderscore faca inglesa\textunderscore .
Cp. \textunderscore raspilha\textunderscore .
Acto de rastilhar.
\section{Rastilhar}
\begin{itemize}
\item {Grp. gram.:v. t.}
\end{itemize}
\begin{itemize}
\item {Utilização:T. da Bairrada}
\end{itemize}
\begin{itemize}
\item {Proveniência:(De \textunderscore rastilho\textunderscore )}
\end{itemize}
Desfazer os torrões de (vinhas), deixados pela cava.
\section{Rastilho}
\begin{itemize}
\item {Grp. gram.:m.}
\end{itemize}
\begin{itemize}
\item {Proveniência:(De \textunderscore rasto\textunderscore )}
\end{itemize}
Fio coberto de pólvora ou de outra substância, para communicar fogo a alguma coisa.
Tubo ou sulco, cheio de pólvora, para o mesmo fim.
\section{Rastinhar}
\begin{itemize}
\item {Grp. gram.:v. t.}
\end{itemize}
\begin{itemize}
\item {Utilização:Prov.}
\end{itemize}
\begin{itemize}
\item {Utilização:minh.}
\end{itemize}
\begin{itemize}
\item {Proveniência:(De \textunderscore rasto\textunderscore )}
\end{itemize}
Marcar com os pés (um terreno).
\section{Rasto}
\begin{itemize}
\item {Grp. gram.:m.}
\end{itemize}
\begin{itemize}
\item {Utilização:Fig.}
\end{itemize}
\begin{itemize}
\item {Grp. gram.:Pl. loc. adv.}
\end{itemize}
\begin{itemize}
\item {Grp. gram.:Loc. adv.}
\end{itemize}
\begin{itemize}
\item {Proveniência:(Do lat. \textunderscore rastrum\textunderscore )}
\end{itemize}
Vestígio, que alguém ou um animal, andando no solo; pista, peugada: \textunderscore foi-lhe no rasto\textunderscore .
Face inferior do calçado.
Indício, sinal.
\textunderscore De rastos\textunderscore , arrastando-se pelo chão.
\textunderscore A rastos\textunderscore , o mesmo que \textunderscore de rastos\textunderscore . Cf. Camillo, \textunderscore Onde está a Fel.\textunderscore , 382.
\section{Rastolho}
\begin{itemize}
\item {fónica:tô}
\end{itemize}
\begin{itemize}
\item {Grp. gram.:m.}
\end{itemize}
Pêra da Beira-Alta, talvez o mesmo que \textunderscore baguim\textunderscore .
Barulho, balbúrdia. Cf. Camillo, \textunderscore Valcoens\textunderscore , 83.
O mesmo que \textunderscore restolho\textunderscore .
(Cp. cast. \textunderscore rastrojo\textunderscore )
\section{Rastra}
\begin{itemize}
\item {Grp. gram.:f.}
\end{itemize}
\begin{itemize}
\item {Utilização:Prov.}
\end{itemize}
\begin{itemize}
\item {Utilização:trasm.}
\end{itemize}
O mesmo que \textunderscore arrasta\textunderscore .
\section{Rastra}
\begin{itemize}
\item {Grp. gram.:f.}
\end{itemize}
\begin{itemize}
\item {Utilização:Prov.}
\end{itemize}
\begin{itemize}
\item {Utilização:trasm.}
\end{itemize}
Réstia de cebola ou de alhos.
\section{Rastrear}
\begin{itemize}
\item {Grp. gram.:v. t.  e  i.}
\end{itemize}
O mesmo que \textunderscore rastejar\textunderscore .
Calcular aproximadamente.
\section{Rastreio}
\begin{itemize}
\item {Grp. gram.:m.}
\end{itemize}
Acto de rastrear. Cf. Filinto, XV, 252.
\section{Rastreiro}
\begin{itemize}
\item {Grp. gram.:adj.}
\end{itemize}
O mesmo que \textunderscore rasteiro\textunderscore .
\section{Rastrejar}
\begin{itemize}
\item {Grp. gram.:v. t.  e  i.}
\end{itemize}
O mesmo que \textunderscore rastejar\textunderscore .
\section{Rastrilho}
\begin{itemize}
\item {Grp. gram.:m.}
\end{itemize}
\begin{itemize}
\item {Utilização:Ant.}
\end{itemize}
\begin{itemize}
\item {Proveniência:(De \textunderscore rastro\textunderscore )}
\end{itemize}
Espécie de grade, guarnecida de pontas de ferro, que se atirava do alto de uma fortaleza, para embaraçar os assaltantes.
\section{Rastro}
\begin{itemize}
\item {Grp. gram.:m.}
\end{itemize}
\begin{itemize}
\item {Utilização:Ant.}
\end{itemize}
\begin{itemize}
\item {Proveniência:(Lat. \textunderscore rastrum\textunderscore )}
\end{itemize}
Rasto.
Ancinho de ferro, para esterroar a terra lavrada.
\section{Rasulha}
\begin{itemize}
\item {Grp. gram.:m.}
\end{itemize}
\begin{itemize}
\item {Utilização:Prov.}
\end{itemize}
\begin{itemize}
\item {Utilização:trasm.}
\end{itemize}
A parte sólida do caldo de hortaliça.
\section{Rasura}
\begin{itemize}
\item {Grp. gram.:f.}
\end{itemize}
\begin{itemize}
\item {Proveniência:(Lat. \textunderscore rasura\textunderscore )}
\end{itemize}
Acto ou effeito de expungir letras ou palavras num papel ou documento.
Raspas.
Fragmentação de substâncias medicinaes, por meio do ralador ou outro instrumento.
\section{Rata}
\begin{itemize}
\item {Grp. gram.:f.}
\end{itemize}
\begin{itemize}
\item {Utilização:Prov.}
\end{itemize}
\begin{itemize}
\item {Utilização:trasm.}
\end{itemize}
\begin{itemize}
\item {Proveniência:(De \textunderscore rato\textunderscore )}
\end{itemize}
Fêmea do rato; ratazana.
O mesmo que \textunderscore toupeira\textunderscore .
\section{Ratada}
\begin{itemize}
\item {Grp. gram.:f.}
\end{itemize}
Porção de ratos, ninhada de ratos.
Ratice.
Conlúio, tranquibérnia.
\section{Ratafia}
\begin{itemize}
\item {Grp. gram.:f.}
\end{itemize}
\begin{itemize}
\item {Proveniência:(Fr. \textunderscore ratafia\textunderscore )}
\end{itemize}
Licor aromático, em que entra aguardente, açúcar, etc.
Designação genérica de licores, dôces e aromáticos.
\section{Rataínha}
\begin{itemize}
\item {Grp. gram.:f.}
\end{itemize}
\begin{itemize}
\item {Utilização:Bras}
\end{itemize}
O mesmo que \textunderscore ratânhia\textunderscore .
\section{Ratalia}
\begin{itemize}
\item {Grp. gram.:f.}
\end{itemize}
\begin{itemize}
\item {Utilização:Ant.}
\end{itemize}
Espécie de taficira.
\section{Ratanha}
\begin{itemize}
\item {Grp. gram.:f.}
\end{itemize}
O mesmo que \textunderscore ratânhia\textunderscore .
\section{Ratanhi}
\begin{itemize}
\item {Grp. gram.:m.}
\end{itemize}
\begin{itemize}
\item {Utilização:Gír.}
\end{itemize}
O mesmo que \textunderscore gazua\textunderscore ^2.
\section{Ratânhia}
\begin{itemize}
\item {Grp. gram.:f.}
\end{itemize}
Nome de duas plantas polygaláceas.
\section{Ratânia}
\begin{itemize}
\item {Grp. gram.:f.}
\end{itemize}
O mesmo que \textunderscore ratânhia\textunderscore .
\section{Ratão}
\begin{itemize}
\item {Grp. gram.:m.}
\end{itemize}
\begin{itemize}
\item {Grp. gram.:M.  e  adj.}
\end{itemize}
\begin{itemize}
\item {Utilização:Fam.}
\end{itemize}
\begin{itemize}
\item {Proveniência:(De \textunderscore rato\textunderscore )}
\end{itemize}
Grande rato.
Peixe plagióstomo.
Gracioso.
Cómico; extravagante.
\section{Ratão-falso}
\begin{itemize}
\item {Grp. gram.:m.}
\end{itemize}
Gênero de plantas da Índia Port., (\textunderscore flagellaria indica\textunderscore , Lin.).
\section{Rataplan}
\begin{itemize}
\item {Grp. gram.:m.}
\end{itemize}
\begin{itemize}
\item {Proveniência:(T. onom.)}
\end{itemize}
Voz imitativa do som do tambor; rufo.
\section{Rataplão}
\begin{itemize}
\item {Grp. gram.:m.}
\end{itemize}
\begin{itemize}
\item {Proveniência:(T. onom.)}
\end{itemize}
Voz imitativa do som do tambor; rufo.
\section{Ratar}
\begin{itemize}
\item {Grp. gram.:v. t.}
\end{itemize}
\begin{itemize}
\item {Proveniência:(De \textunderscore rato\textunderscore )}
\end{itemize}
Roêr, á maneira de rato; mordicar.
\section{Rataria}
\begin{itemize}
\item {Grp. gram.:f.}
\end{itemize}
Grande porção de ratos; os ratos. Cf. O'Neill, \textunderscore Fab.\textunderscore , 760.
\section{Ràtàtau}
\begin{itemize}
\item {Grp. gram.:m.}
\end{itemize}
\begin{itemize}
\item {Utilização:T. do Fundão}
\end{itemize}
Espécie de jôgo de asar, um tanto parecido com a roleta.
\section{Ratazana}
\begin{itemize}
\item {Grp. gram.:f.}
\end{itemize}
\begin{itemize}
\item {Grp. gram.:M.  e  f.}
\end{itemize}
\begin{itemize}
\item {Utilização:Fam.}
\end{itemize}
Rata; rato ou rata grande, (\textunderscore mus rattus\textunderscore , Cuv.).
Pessôa divertida ou ridícula.
\section{Rateação}
\begin{itemize}
\item {Grp. gram.:f.}
\end{itemize}
O mesmo que \textunderscore rateio\textunderscore .
\section{Rateadamente}
\begin{itemize}
\item {Grp. gram.:adv.}
\end{itemize}
De modo rateado; com rateio.
\section{Rateador}
\begin{itemize}
\item {Grp. gram.:m.  e  adj.}
\end{itemize}
O que rateia.
\section{Rateamento}
\begin{itemize}
\item {Grp. gram.:m.}
\end{itemize}
O mesmo que \textunderscore rateio\textunderscore .
\section{Ratear}
\begin{itemize}
\item {Grp. gram.:v. t.}
\end{itemize}
\begin{itemize}
\item {Proveniência:(Do lat. \textunderscore ratus\textunderscore )}
\end{itemize}
Dividir proporcionalmente.
\section{Rateio}
\begin{itemize}
\item {Grp. gram.:m.}
\end{itemize}
Acto ou effeito de ratear.
\section{Rateiras}
\begin{itemize}
\item {Grp. gram.:f. pl.}
\end{itemize}
\begin{itemize}
\item {Utilização:Prov.}
\end{itemize}
\begin{itemize}
\item {Utilização:trasm.}
\end{itemize}
O mesmo que \textunderscore arregateiras\textunderscore .
\section{Rateiro}
\begin{itemize}
\item {Grp. gram.:m.  e  adj.}
\end{itemize}
O que apanha ratos, (falando-se de cães ou de gatos).
\section{Ratel}
\begin{itemize}
\item {Grp. gram.:m.}
\end{itemize}
\begin{itemize}
\item {Proveniência:(De \textunderscore rato\textunderscore )}
\end{itemize}
Mammífero carnívoro do Cabo da Bôa-Esperança.
\section{Ratiabição}
\begin{itemize}
\item {Grp. gram.:f.}
\end{itemize}
\begin{itemize}
\item {Utilização:Des.}
\end{itemize}
\begin{itemize}
\item {Proveniência:(Do lat. \textunderscore ratus\textunderscore  + \textunderscore habere\textunderscore )}
\end{itemize}
O mesmo que \textunderscore ratificação\textunderscore .
Confirmação do que está feito; comprovação. Cf. \textunderscore Diccion. Exeg.\textunderscore 
\section{Ratice}
\begin{itemize}
\item {Grp. gram.:f.}
\end{itemize}
\begin{itemize}
\item {Proveniência:(De \textunderscore rato\textunderscore )}
\end{itemize}
Acto ou dito de ratão; excentricidade.
\section{Raticum}
\begin{itemize}
\item {Grp. gram.:m.}
\end{itemize}
\begin{itemize}
\item {Utilização:Bras}
\end{itemize}
Fruto silvestre e comestível.
\section{Ratificação}
\begin{itemize}
\item {Grp. gram.:f.}
\end{itemize}
Acto ou effeito de ratificar.
\section{Ratificar}
\begin{itemize}
\item {Grp. gram.:v. t.}
\end{itemize}
\begin{itemize}
\item {Proveniência:(Do lat. \textunderscore ratus\textunderscore  + \textunderscore facere\textunderscore )}
\end{itemize}
Validar, confirmar; approvar authenticamente; comprovar.
\section{Ratificável}
\begin{itemize}
\item {Grp. gram.:adj.}
\end{itemize}
Que se póde ratificar.
\section{Ratihabição}
\begin{itemize}
\item {Grp. gram.:f.}
\end{itemize}
\begin{itemize}
\item {Utilização:Des.}
\end{itemize}
\begin{itemize}
\item {Proveniência:(Do lat. \textunderscore ratus\textunderscore  + \textunderscore habere\textunderscore )}
\end{itemize}
O mesmo que \textunderscore ratificação\textunderscore .
Confirmação do que está feito; comprovação. Cf. \textunderscore Diccion. Exeg.\textunderscore 
\section{Ratim}
\begin{itemize}
\item {Grp. gram.:m.}
\end{itemize}
\begin{itemize}
\item {Utilização:Ant.}
\end{itemize}
Espécie de mandil. Cf. \textunderscore Eufrosina\textunderscore , 94.
\section{Ratina}
\begin{itemize}
\item {Grp. gram.:f.}
\end{itemize}
\begin{itemize}
\item {Proveniência:(Fr. \textunderscore ratine\textunderscore )}
\end{itemize}
Estofo de lan, com o pêlo encrespado.
\section{Ratinadora}
\begin{itemize}
\item {Grp. gram.:f.}
\end{itemize}
Máquina, com que se ratina o pano. Cf. \textunderscore Inquér. Industr.\textunderscore , 2.^a p., l. II, 107.
\section{Ratinar}
\begin{itemize}
\item {Grp. gram.:v. t.}
\end{itemize}
Encrespar como a ratina.
\section{Ratinha}
\begin{itemize}
\item {Grp. gram.:f.}
\end{itemize}
\begin{itemize}
\item {Utilização:Prov.}
\end{itemize}
\begin{itemize}
\item {Utilização:minh.}
\end{itemize}
Membro viril da criança.
\section{Ratinhador}
\begin{itemize}
\item {Grp. gram.:m.}
\end{itemize}
\begin{itemize}
\item {Utilização:Des.}
\end{itemize}
Mastim.
Homem miserável, sordidamente avarento.
\section{Ratinhar}
\begin{itemize}
\item {Grp. gram.:v. t.}
\end{itemize}
\begin{itemize}
\item {Grp. gram.:V. i.}
\end{itemize}
\begin{itemize}
\item {Proveniência:(De \textunderscore ratinho\textunderscore )}
\end{itemize}
Regatear muito, (falando-se de preços).
Fazer pecúlio, economizar exaggeradamente.
\section{Ratinheiro}
\begin{itemize}
\item {Grp. gram.:adj.}
\end{itemize}
Relativo a ratos.
Que regateia, ao ajustar o que quere comprar. Cf. Filinto, XII, 125.
\section{Ratinho}
\begin{itemize}
\item {Grp. gram.:m.}
\end{itemize}
\begin{itemize}
\item {Utilização:Pop.}
\end{itemize}
\begin{itemize}
\item {Utilização:Des.}
\end{itemize}
\begin{itemize}
\item {Utilização:Infant.}
\end{itemize}
\begin{itemize}
\item {Grp. gram.:Adj.}
\end{itemize}
\begin{itemize}
\item {Proveniência:(De \textunderscore rato\textunderscore )}
\end{itemize}
Pequeno rato.
Jornaleiro, que vai do Minho ou da Beira, não contratado, trabalhar em outras provincias, especialmente no Alentejo.
Minhoto.
Cada um dos primeiros dentes da criança.
Ratinheiro.
Que é pouco corpulento, (falando-se de certa qualidade de bois).
\section{Ratinho}
\begin{itemize}
\item {Grp. gram.:m.}
\end{itemize}
\begin{itemize}
\item {Utilização:Prov.}
\end{itemize}
\begin{itemize}
\item {Proveniência:(Do cast. \textunderscore rato\textunderscore )}
\end{itemize}
Momento; pequeno espaço de tempo.
\section{Ratívoro}
\begin{itemize}
\item {Grp. gram.:adj.}
\end{itemize}
\begin{itemize}
\item {Proveniência:(De \textunderscore rato\textunderscore  + lat. \textunderscore vorare\textunderscore )}
\end{itemize}
Que come ratos.
\section{Rato}
\begin{itemize}
\item {Grp. gram.:m.}
\end{itemize}
\begin{itemize}
\item {Grp. gram.:Adj.}
\end{itemize}
Pequeno quadrúpede roedor, de pequenas patas, cauda comprida e focinho bicudo, e de que há várias espécies, como a ratazana, o rato decúmano, o rato caseiro, (\textunderscore mus musculus\textunderscore ), o ratinho dos matos, (\textunderscore mus sylvaticus\textunderscore ), o musaranho, etc.
Espécie de peixe, (\textunderscore chimara affinis\textunderscore ).
Pedra, cujas arestas corróem a amarra do navio fundeado.
Que tem côr de rato: \textunderscore um cavallo rato\textunderscore .
Esquisito, excêntrico, ridículo:«\textunderscore que patetice tão rata!\textunderscore »Cf. Semedo, \textunderscore Composições\textunderscore .«\textunderscore Posto na forca o crê o povo rato.\textunderscore »Filinto, XII, 108.
(Ant. alt. al. \textunderscore rato\textunderscore )
\section{Rato}
\begin{itemize}
\item {Grp. gram.:adj.}
\end{itemize}
\begin{itemize}
\item {Proveniência:(Lat. \textunderscore ratus\textunderscore )}
\end{itemize}
Confirmado; reconhecido:«\textunderscore o que agora hão por rato e valioso, daqui a pouco o tornam irrito e de nenhum valor.\textunderscore »Arráiz, \textunderscore Diálogos\textunderscore , 94.
\section{Rato-almiscarado}
\begin{itemize}
\item {Grp. gram.:m.}
\end{itemize}
O mesmo que \textunderscore almiscareiro\textunderscore ? Cf. P. Moraes, \textunderscore Zool. Elem.\textunderscore , 199.
\section{Rato-da-índia}
\begin{itemize}
\item {Grp. gram.:m.}
\end{itemize}
O mesmo que \textunderscore rato-de-faraó\textunderscore .
\section{Rato-de-faraó}
\begin{itemize}
\item {Grp. gram.:m.}
\end{itemize}
\begin{itemize}
\item {Utilização:Zool.}
\end{itemize}
O mesmo que \textunderscore ichneumon\textunderscore  ou \textunderscore mangusto\textunderscore ^1.
\section{Rato-do-egipto}
\begin{itemize}
\item {Grp. gram.:m.}
\end{itemize}
O mesmo que \textunderscore rato-de-faraó\textunderscore .
\section{Ratoeira}
\begin{itemize}
\item {Grp. gram.:f.}
\end{itemize}
\begin{itemize}
\item {Utilização:Ext.}
\end{itemize}
\begin{itemize}
\item {Utilização:Gír.}
\end{itemize}
Utensílio, para apanhar ratos.
Armadilha; cilada; ardil.
Casa, onde se juntam ladrões.
\section{Ratona}
\begin{itemize}
\item {Grp. gram.:f.  e  adj.}
\end{itemize}
Mulhér ou coisa excentrica ou ridícula.
(Fem. de \textunderscore ratão\textunderscore )
\section{Ratoneiro}
\begin{itemize}
\item {Grp. gram.:m.}
\end{itemize}
\begin{itemize}
\item {Proveniência:(De \textunderscore rato\textunderscore )}
\end{itemize}
Aquelle que furta coisas de pouca monta; gatuno, larápio.
\section{Ratonice}
\begin{itemize}
\item {Grp. gram.:f.}
\end{itemize}
\begin{itemize}
\item {Proveniência:(De \textunderscore rato\textunderscore )}
\end{itemize}
Roubo insignificante; ladroíce.
\section{Ratoqueira}
\begin{itemize}
\item {Grp. gram.:f.}
\end{itemize}
\begin{itemize}
\item {Utilização:Prov.}
\end{itemize}
\begin{itemize}
\item {Utilização:trasm.}
\end{itemize}
O mesmo que \textunderscore ratoeira\textunderscore .
O mesmo que \textunderscore toupeira\textunderscore . (Colhido em Alijó)
\section{Rau}
\begin{itemize}
\item {Grp. gram.:m.}
\end{itemize}
Título honorífico na Índia.
\section{Raucísono}
\begin{itemize}
\item {fónica:so}
\end{itemize}
\begin{itemize}
\item {Grp. gram.:adj.}
\end{itemize}
\begin{itemize}
\item {Utilização:Poét.}
\end{itemize}
\begin{itemize}
\item {Proveniência:(Lat. \textunderscore raucisonus\textunderscore )}
\end{itemize}
Que tem som rouco.
\section{Raucissono}
\begin{itemize}
\item {Grp. gram.:adj.}
\end{itemize}
\begin{itemize}
\item {Utilização:Poét.}
\end{itemize}
\begin{itemize}
\item {Proveniência:(Lat. \textunderscore raucisonus\textunderscore )}
\end{itemize}
Que tem som rouco.
\section{Raucitroante}
\begin{itemize}
\item {Grp. gram.:adj.}
\end{itemize}
\begin{itemize}
\item {Proveniência:(De \textunderscore raucus\textunderscore  lat. + \textunderscore troar\textunderscore )}
\end{itemize}
Que emitte som rouco ou cavo:«\textunderscore rompem súbito estrépito atabales raucitroantes.\textunderscore »Castilho, \textunderscore Metam.\textunderscore , 194.
\section{Raudal}
\begin{itemize}
\item {Grp. gram.:m.}
\end{itemize}
\begin{itemize}
\item {Utilização:Ant.}
\end{itemize}
Torrente de água ou de outros líquidos.
Grande porção.
(Cast. \textunderscore raudal\textunderscore )
\section{Raudão}
\begin{itemize}
\item {Grp. gram.:m.}
\end{itemize}
O mesmo que \textunderscore rosilho\textunderscore .
\section{Rausa}
\begin{itemize}
\item {Grp. gram.:f.}
\end{itemize}
O mesmo que \textunderscore rauso\textunderscore .
\section{Rausador}
\begin{itemize}
\item {Grp. gram.:m.}
\end{itemize}
\begin{itemize}
\item {Utilização:Ant.}
\end{itemize}
Aquelle que rausa.
\section{Rausar}
\begin{itemize}
\item {Grp. gram.:v. t.}
\end{itemize}
\begin{itemize}
\item {Utilização:Ant.}
\end{itemize}
O mesmo que \textunderscore rouçar\textunderscore .
\section{Rauso}
\begin{itemize}
\item {Grp. gram.:m.}
\end{itemize}
\begin{itemize}
\item {Utilização:Ant.}
\end{itemize}
Acto de rausar.
\section{Raussar}
\textunderscore v. t.\textunderscore  (e der.)
O mesmo que \textunderscore rouçar\textunderscore , etc.
\section{Rausso}
\begin{itemize}
\item {Grp. gram.:m.}
\end{itemize}
\begin{itemize}
\item {Utilização:Ant.}
\end{itemize}
Acto de raussar.
Multa, que se impunha ao rouçador.
\section{Rauxar}
\textunderscore v. t.\textunderscore  (e der.)
O mesmo que \textunderscore rouçar\textunderscore , etc.
\section{Ravasca}
\begin{itemize}
\item {Grp. gram.:f.}
\end{itemize}
\begin{itemize}
\item {Utilização:Prov.}
\end{itemize}
\begin{itemize}
\item {Utilização:alent.}
\end{itemize}
O mesmo que \textunderscore zanga\textunderscore ^1.
\section{Ravasco}
\begin{itemize}
\item {Grp. gram.:m.}
\end{itemize}
\begin{itemize}
\item {Utilização:Ant.}
\end{itemize}
Homem libertino, devasso:«\textunderscore nós vimos ás prègações e os ravascos ou rascões furtam-nos quanto deixamos.\textunderscore »\textunderscore Auto de S. António\textunderscore .
(Metáth. de \textunderscore varrasco\textunderscore )
\section{Ravenala}
\begin{itemize}
\item {Grp. gram.:m.}
\end{itemize}
Grande árvore de Madagáscar, empregada em construcções.
\section{Ravensara}
\begin{itemize}
\item {Grp. gram.:m.}
\end{itemize}
Espécie de videira selvagem, em Madagáscar.
\section{Ravessa}
\begin{itemize}
\item {Grp. gram.:f.}
\end{itemize}
\begin{itemize}
\item {Utilização:Prov.}
\end{itemize}
\begin{itemize}
\item {Utilização:alent.}
\end{itemize}
\begin{itemize}
\item {Proveniência:(De \textunderscore reversa\textunderscore , fem. de \textunderscore reverso\textunderscore ?)}
\end{itemize}
Montículo, que póde abrigar contra o vento.
\section{Ravina}
\begin{itemize}
\item {Grp. gram.:f.}
\end{itemize}
\begin{itemize}
\item {Proveniência:(Fr. \textunderscore ravine\textunderscore )}
\end{itemize}
(\textunderscore Gal. inútil\textunderscore , em vez de \textunderscore barranco\textunderscore  e \textunderscore enxurro\textunderscore )
\section{Ravinhar}
\begin{itemize}
\item {Grp. gram.:v. t.}
\end{itemize}
\begin{itemize}
\item {Utilização:Prov.}
\end{itemize}
\begin{itemize}
\item {Utilização:alg.}
\end{itemize}
Contrariar, fazer opposição a.
(Por \textunderscore raivinhar\textunderscore , de \textunderscore raiva\textunderscore )
\section{Ravinhoso}
\begin{itemize}
\item {Grp. gram.:adj.}
\end{itemize}
\begin{itemize}
\item {Utilização:Ant.}
\end{itemize}
O mesmo que \textunderscore raivoso\textunderscore .
(Por \textunderscore raivinhoso\textunderscore , de \textunderscore raivinha\textunderscore )
\section{Ravinoso}
\begin{itemize}
\item {Grp. gram.:adj.}
\end{itemize}
\begin{itemize}
\item {Utilização:Gal}
\end{itemize}
Que tem ravinas.
\section{Ravióes}
\begin{itemize}
\item {Grp. gram.:m. pl.}
\end{itemize}
\begin{itemize}
\item {Proveniência:(Do it. \textunderscore ravioli\textunderscore )}
\end{itemize}
Sopa sêca de rodelas de massa, com recheio fino e saboroso. Cf. O'Neill, \textunderscore Fab.\textunderscore , 878.
\section{Ravióis}
\begin{itemize}
\item {Grp. gram.:m. pl.}
\end{itemize}
\begin{itemize}
\item {Proveniência:(Do it. \textunderscore ravioli\textunderscore )}
\end{itemize}
Sopa sêca de rodelas de massa, com recheio fino e saboroso. Cf. O'Neill, \textunderscore Fab.\textunderscore , 878.
\section{Raxa}
\begin{itemize}
\item {Grp. gram.:f.}
\end{itemize}
\begin{itemize}
\item {Utilização:Ant.}
\end{itemize}
Espécie de pano grosseiro de algodão:«\textunderscore com capotes de raxa se cobrião.\textunderscore »D. Bernárdez, \textunderscore Lima\textunderscore , 263.
\section{Raxelo}
\begin{itemize}
\item {fónica:xê}
\end{itemize}
\begin{itemize}
\item {Grp. gram.:m.}
\end{itemize}
\begin{itemize}
\item {Utilização:T. da Bairrada}
\end{itemize}
Outra fórma de \textunderscore reixelo\textunderscore .
\section{Raxeta}
\begin{itemize}
\item {fónica:xê}
\end{itemize}
\begin{itemize}
\item {Grp. gram.:m.}
\end{itemize}
\begin{itemize}
\item {Utilização:Ant.}
\end{itemize}
\begin{itemize}
\item {Proveniência:(De \textunderscore raxa\textunderscore )}
\end{itemize}
Espécie de tecido ordinário.
\section{Raz}
\begin{itemize}
\item {Grp. gram.:m.}
\end{itemize}
\begin{itemize}
\item {Utilização:Ant.}
\end{itemize}
O mesmo que \textunderscore arrás\textunderscore .
\section{Raz}
\begin{itemize}
\item {Grp. gram.:f.}
\end{itemize}
\begin{itemize}
\item {Utilização:Ant.}
\end{itemize}
Cabeça; cabeçal.
(Cp. \textunderscore rês\textunderscore )
\section{Razão}
\begin{itemize}
\item {Grp. gram.:f.}
\end{itemize}
\begin{itemize}
\item {Utilização:Prov.}
\end{itemize}
\begin{itemize}
\item {Utilização:trasm.}
\end{itemize}
\begin{itemize}
\item {Grp. gram.:M.}
\end{itemize}
\begin{itemize}
\item {Proveniência:(Do lat. \textunderscore ratio\textunderscore )}
\end{itemize}
Faculdade, com que o homem discorre, julga e se distingue dos outros animaes.
Faculdade de conhecer.
Intelligência.
Juízo prudencial; bom senso.
Justiça.
Direito.
A lei moral.
Causa, motivo: \textunderscore em razão das offensas...\textunderscore 
Prova por agumento: \textunderscore dar a razão do seu dito\textunderscore .
Conhecimento, notícias, participação.
Percentagem.
Relação entre duas quantidades mathemáticas ou entre dois números.
Conta corrente; conta.
Recado, mensagem.
Livro, em que o commerciante faz a escrituração dos seus créditos e débitos.
\section{Razo}
\begin{itemize}
\item {Grp. gram.:m.}
\end{itemize}
Estofo de seda, o mesmo que \textunderscore raso\textunderscore .
\section{Razoadamente}
\begin{itemize}
\item {Grp. gram.:adv.}
\end{itemize}
De modo razoado; razoavelmente; com tino.
\section{Razoado}
\begin{itemize}
\item {Grp. gram.:adj.}
\end{itemize}
\begin{itemize}
\item {Grp. gram.:M.}
\end{itemize}
Razoável.
O mesmo que \textunderscore arrazoado\textunderscore . Cf. Herculano, \textunderscore Lendas\textunderscore , 56.
\section{Razoador}
\begin{itemize}
\item {Grp. gram.:m.}
\end{itemize}
Aquelle que razôa. Cf. Filinto, XIII, 55.
\section{Razoamento}
\begin{itemize}
\item {Grp. gram.:m.}
\end{itemize}
Acto ou effeito de razoar.
\section{Razoar}
\begin{itemize}
\item {Grp. gram.:v. i.}
\end{itemize}
\begin{itemize}
\item {Grp. gram.:V. t.}
\end{itemize}
\begin{itemize}
\item {Utilização:Ant.}
\end{itemize}
\begin{itemize}
\item {Proveniência:(De \textunderscore razão\textunderscore )}
\end{itemize}
O mesmo que \textunderscore arrazoar\textunderscore ; raciocinar.
Defender.
\section{Razoável}
\begin{itemize}
\item {Grp. gram.:adj.}
\end{itemize}
Que é conforme á razão; moderado.
Importante; acima de medíocre.
\section{Razoavelmente}
\begin{itemize}
\item {Grp. gram.:adv.}
\end{itemize}
De modo razoável.
\section{Razonável}
\begin{itemize}
\item {Grp. gram.:adj.}
\end{itemize}
\begin{itemize}
\item {Utilização:Ant.}
\end{itemize}
\begin{itemize}
\item {Utilização:Pop.}
\end{itemize}
O mesmo que \textunderscore razoável\textunderscore .
\section{Razzia}
\begin{itemize}
\item {Grp. gram.:f.}
\end{itemize}
\begin{itemize}
\item {Proveniência:(Fr. \textunderscore razzia\textunderscore )}
\end{itemize}
(\textunderscore Gal. inútil\textunderscore , em vez de \textunderscore gaziva\textunderscore )
\section{Re...}
\begin{itemize}
\item {Grp. gram.:pref.}
\end{itemize}
\begin{itemize}
\item {Proveniência:(Lat. \textunderscore re\textunderscore )}
\end{itemize}
(Designa \textunderscore repetição\textunderscore , \textunderscore acção retrò-activa\textunderscore , \textunderscore realce\textunderscore , etc.; servindo, ás vezes, de partícula prepositiva, sem valor sensível)
\section{Ré}
\begin{itemize}
\item {Grp. gram.:f.}
\end{itemize}
\begin{itemize}
\item {Utilização:Jur.}
\end{itemize}
\begin{itemize}
\item {Proveniência:(De \textunderscore réu\textunderscore )}
\end{itemize}
Mulhér, accusada de um crime; mulhér criminosa; autora (de crime)
\section{Ré}
\begin{itemize}
\item {Grp. gram.:f.}
\end{itemize}
\begin{itemize}
\item {Grp. gram.:Loc.}
\end{itemize}
\begin{itemize}
\item {Utilização:Fig.}
\end{itemize}
\begin{itemize}
\item {Proveniência:(Do lat. \textunderscore retro\textunderscore )}
\end{itemize}
Parte do navio, entre a pôpa e o mastro grande; pôpa: \textunderscore os passageiros da ré\textunderscore .
\textunderscore Pôr de ré\textunderscore , pôr de lado, desprezar, não fazer uso de:«\textunderscore ...o nosso protegido punha os escrúpulos de ré...\textunderscore »Filinto, \textunderscore D. Man.\textunderscore , II, 199.
\section{Ré}
\begin{itemize}
\item {Grp. gram.:m.}
\end{itemize}
Segunda nota da escala musical.
Sinal representativo desta nota.
Corda de alguns instrumentos, correspondente a essa nota.
(Da 1.^a sýllaba do lat. \textunderscore resonare\textunderscore , aproveitada por G. da Arezzo, com as 1.^{as} sýllabas de algumas palavras de um hymno religioso, para a formação da antiga escala musical)
\section{Reabastecer}
\begin{itemize}
\item {Grp. gram.:v. t.}
\end{itemize}
\begin{itemize}
\item {Proveniência:(De \textunderscore re...\textunderscore  + \textunderscore abastecer\textunderscore )}
\end{itemize}
Abastecer novamente.
Abastecer muito.
\section{Reaberto}
\begin{itemize}
\item {Grp. gram.:adj.}
\end{itemize}
\begin{itemize}
\item {Proveniência:(De \textunderscore re...\textunderscore  + \textunderscore aberto\textunderscore )}
\end{itemize}
Que se reabriu; aberto novamente: \textunderscore empresário de um theatro reaberto\textunderscore .
\section{Reabertura}
\begin{itemize}
\item {Grp. gram.:f.}
\end{itemize}
\begin{itemize}
\item {Proveniência:(De \textunderscore re...\textunderscore  + \textunderscore abertura\textunderscore )}
\end{itemize}
Acto ou effeito de reabrir.
\section{Reabrir}
\begin{itemize}
\item {Grp. gram.:v. t.}
\end{itemize}
\begin{itemize}
\item {Grp. gram.:V. i.}
\end{itemize}
\begin{itemize}
\item {Proveniência:(De \textunderscore re...\textunderscore  + \textunderscore abrir\textunderscore )}
\end{itemize}
Tornar a abrir; abrir novamente.
Tornar a abrir-se.
\section{Reabsorção}
\begin{itemize}
\item {Grp. gram.:f.}
\end{itemize}
\begin{itemize}
\item {Proveniência:(De \textunderscore re...\textunderscore  + \textunderscore absorção\textunderscore )}
\end{itemize}
Acto ou efeito de reabsorver.
\section{Reabsorpção}
\begin{itemize}
\item {Grp. gram.:f.}
\end{itemize}
\begin{itemize}
\item {Proveniência:(De \textunderscore re...\textunderscore  + \textunderscore absorpção\textunderscore )}
\end{itemize}
Acto ou effeito de reabsorver.
\section{Reabsorver}
\begin{itemize}
\item {Grp. gram.:v. t.}
\end{itemize}
\begin{itemize}
\item {Proveniência:(De \textunderscore re...\textunderscore  + \textunderscore absorver\textunderscore )}
\end{itemize}
Absorver novamente.
\section{Reacção}
\begin{itemize}
\item {Grp. gram.:f.}
\end{itemize}
\begin{itemize}
\item {Utilização:Chím.}
\end{itemize}
\begin{itemize}
\item {Utilização:Phýs.}
\end{itemize}
\begin{itemize}
\item {Proveniência:(Do lat. \textunderscore reactio\textunderscore )}
\end{itemize}
Acto ou effeito de reagir.
Acção opposta a outra.
Resistência activa a qualquer esfôrço.
Manifestação dos caracteres, que distinguem um corpo, occasionada pela acção de outro corpo.
Phenómeno, resultante da acção recíproca de certos corpos.
Phenómeno physiológico e pathológico, que contrabalança a influência do agente que o determinou.
Esforços de um partido político ou social, para voltar a um estado anterior.
O partido conservador.
Ultramontanismo.
Absolutismo.
Qualquer systema contrario ao progresso social.
\section{Reaccender}
\begin{itemize}
\item {Grp. gram.:v. t.}
\end{itemize}
\begin{itemize}
\item {Utilização:Ext.}
\end{itemize}
\begin{itemize}
\item {Proveniência:(De \textunderscore re...\textunderscore  + \textunderscore accender\textunderscore )}
\end{itemize}
Accender novamente.
Activar, tornar mais ardente.
Estimular; desenvolver: \textunderscore reaccender ódios\textunderscore .
\section{Reaccional}
\begin{itemize}
\item {Grp. gram.:adj.}
\end{itemize}
Relativo a reacção.
\section{Reaccionário}
\begin{itemize}
\item {Grp. gram.:adj.}
\end{itemize}
\begin{itemize}
\item {Grp. gram.:M.}
\end{itemize}
\begin{itemize}
\item {Proveniência:(De \textunderscore reacção\textunderscore )}
\end{itemize}
Relativo ao partido da reacção ou ao seu systema.
Opposto á liberdade.
Sectário da reacção política ou social.
\section{Reaccionarismo}
\begin{itemize}
\item {Grp. gram.:m.}
\end{itemize}
Ideias ou systemas dos reaccionários.
\section{Reaccusação}
\begin{itemize}
\item {Grp. gram.:f.}
\end{itemize}
Acto ou effeito de reaccusar.
\section{Reaccusar}
\begin{itemize}
\item {Grp. gram.:v. t.}
\end{itemize}
\begin{itemize}
\item {Proveniência:(De \textunderscore re...\textunderscore  + \textunderscore accusar\textunderscore )}
\end{itemize}
Accusar outra vez; recriminar.
\section{Reacender}
\begin{itemize}
\item {Grp. gram.:v. t.}
\end{itemize}
\begin{itemize}
\item {Utilização:Ext.}
\end{itemize}
\begin{itemize}
\item {Proveniência:(De \textunderscore re...\textunderscore  + \textunderscore acender\textunderscore )}
\end{itemize}
Acender novamente.
Activar, tornar mais ardente.
Estimular; desenvolver: \textunderscore reacender ódios\textunderscore .
\section{Reacquisição}
\begin{itemize}
\item {Grp. gram.:f.}
\end{itemize}
\begin{itemize}
\item {Proveniência:(De \textunderscore re...\textunderscore  + \textunderscore acquisição\textunderscore )}
\end{itemize}
Acto ou effeito de readquirir.
\section{Reactivo}
\begin{itemize}
\item {Grp. gram.:adj.}
\end{itemize}
\begin{itemize}
\item {Grp. gram.:M.}
\end{itemize}
\begin{itemize}
\item {Proveniência:(De \textunderscore re...\textunderscore  + \textunderscore activo\textunderscore )}
\end{itemize}
Que faz reagir ou estabelece reacção.
Substância, para reacções chímicas, nos laboratórios.
\section{Reacusação}
\begin{itemize}
\item {Grp. gram.:f.}
\end{itemize}
Acto ou efeito de reacusar.
\section{Reacusar}
\begin{itemize}
\item {Grp. gram.:v. t.}
\end{itemize}
\begin{itemize}
\item {Proveniência:(De \textunderscore re...\textunderscore  + \textunderscore accusar\textunderscore )}
\end{itemize}
Acusar outra vez; recriminar.
\section{Readilho}
\begin{itemize}
\item {Grp. gram.:m.}
\end{itemize}
\begin{itemize}
\item {Utilização:Ant.}
\end{itemize}
Tecido de lan e seda.
\section{Readmissão}
\begin{itemize}
\item {Grp. gram.:f.}
\end{itemize}
\begin{itemize}
\item {Proveniência:(De \textunderscore re...\textunderscore  + \textunderscore admissão\textunderscore )}
\end{itemize}
Acto ou effeito de readmittir.
\section{Readmitir}
\begin{itemize}
\item {Grp. gram.:v. t.}
\end{itemize}
\begin{itemize}
\item {Proveniência:(De \textunderscore re...\textunderscore  + \textunderscore admitir\textunderscore )}
\end{itemize}
Admitir novamente.
\section{Readmittir}
\begin{itemize}
\item {Grp. gram.:v. t.}
\end{itemize}
\begin{itemize}
\item {Proveniência:(De \textunderscore re...\textunderscore  + \textunderscore admittir\textunderscore )}
\end{itemize}
Admittir novamente.
\section{Readquirir}
\begin{itemize}
\item {Grp. gram.:v. t.}
\end{itemize}
\begin{itemize}
\item {Proveniência:(De \textunderscore re...\textunderscore  + \textunderscore adquirir\textunderscore )}
\end{itemize}
Tornar a adquirir.
\section{Readquisição}
\begin{itemize}
\item {Grp. gram.:f.}
\end{itemize}
O mesmo que \textunderscore reacquisição\textunderscore .
\section{Reaes}
\begin{itemize}
\item {Grp. gram.:adj. f. pl.}
\end{itemize}
\begin{itemize}
\item {Utilização:Cyn.}
\end{itemize}
\begin{itemize}
\item {Proveniência:(De \textunderscore real\textunderscore ^2)}
\end{itemize}
Dizia-se das pennas, mais conhecidas por \textunderscore voadeiras\textunderscore . Cf. Fern. Pereira, \textunderscore Caça de Altan.\textunderscore 
\section{Reaffirmar}
\begin{itemize}
\item {Grp. gram.:v. t.}
\end{itemize}
\begin{itemize}
\item {Proveniência:(De \textunderscore re...\textunderscore  + \textunderscore affirmar\textunderscore )}
\end{itemize}
Affirmar novamente.
\section{Reafirmar}
\begin{itemize}
\item {Grp. gram.:v. t.}
\end{itemize}
\begin{itemize}
\item {Proveniência:(De \textunderscore re...\textunderscore  + \textunderscore afirmar\textunderscore )}
\end{itemize}
Afirmar novamente.
\section{Reagente}
\begin{itemize}
\item {Grp. gram.:adj.}
\end{itemize}
\begin{itemize}
\item {Grp. gram.:M.}
\end{itemize}
\begin{itemize}
\item {Proveniência:(De \textunderscore reagir\textunderscore )}
\end{itemize}
Que reage.
Substância que, junta com outra, faz manifestar as propriedades chímicas desta.
\section{Reaggravação}
\begin{itemize}
\item {Grp. gram.:f.}
\end{itemize}
Acto ou effeito de reaggravar.
\section{Reaggravar}
\begin{itemize}
\item {Grp. gram.:v. t.}
\end{itemize}
\begin{itemize}
\item {Proveniência:(De \textunderscore re...\textunderscore  + \textunderscore aggravar\textunderscore )}
\end{itemize}
Aggravar novamente; exacerbar.
\section{Reagir}
\begin{itemize}
\item {Grp. gram.:v. i.}
\end{itemize}
\begin{itemize}
\item {Utilização:Fig.}
\end{itemize}
\begin{itemize}
\item {Proveniência:(De \textunderscore re...\textunderscore  + \textunderscore agir\textunderscore )}
\end{itemize}
Exercer reacção.
Oppôr uma acção a outra.
Resistir, lutar.
Servir de reagente chímico.
\section{Reagravação}
\begin{itemize}
\item {Grp. gram.:f.}
\end{itemize}
Acto ou efeito de reagravar.
\section{Reagravar}
\begin{itemize}
\item {Grp. gram.:v. t.}
\end{itemize}
\begin{itemize}
\item {Proveniência:(De \textunderscore re...\textunderscore  + \textunderscore agravar\textunderscore )}
\end{itemize}
Agravar novamente; exacerbar.
\section{Reais}
\begin{itemize}
\item {Grp. gram.:adj. f. pl.}
\end{itemize}
\begin{itemize}
\item {Utilização:Cyn.}
\end{itemize}
\begin{itemize}
\item {Proveniência:(De \textunderscore real\textunderscore ^2)}
\end{itemize}
Dizia-se das pennas, mais conhecidas por \textunderscore voadeiras\textunderscore . Cf. Fern. Pereira, \textunderscore Caça de Altan.\textunderscore 
\section{Real}
\begin{itemize}
\item {Grp. gram.:adj.}
\end{itemize}
\begin{itemize}
\item {Grp. gram.:M.}
\end{itemize}
\begin{itemize}
\item {Proveniência:(Lat. \textunderscore realis\textunderscore )}
\end{itemize}
Que existe de facto; que não é imaginário.
Que não é ideal.
Effectivo.
Relativo a bens ou coisas e não a pessôas: \textunderscore uma herança com encargos reaes\textunderscore .
Aquillo que é real: \textunderscore deixou o ideal pelo real\textunderscore .
\section{Real}
\begin{itemize}
\item {Grp. gram.:adj.}
\end{itemize}
\begin{itemize}
\item {Grp. gram.:M.}
\end{itemize}
\begin{itemize}
\item {Proveniência:(Do lat. \textunderscore regalis\textunderscore )}
\end{itemize}
Relativo ao rei; digno ou próprio de rei; magnificente.
Diz-se da maior ou melhor coisa de uma série ou grupo: \textunderscore uma descompostura real\textunderscore ; \textunderscore um tiro real\textunderscore , etc.
Antiga moéda portuguesa, de valor differente, segundo as épocas.
Unidade monetária portuguesa, (pl. \textunderscore reis\textunderscore ).
\section{Real}
\textunderscore m.\textunderscore  (der.)
O mesmo que \textunderscore arraial\textunderscore . Cf. Filinto, \textunderscore D. Man.\textunderscore , II, 317, III, 264 e 303.
\section{Real-arca}
\begin{itemize}
\item {Grp. gram.:m.}
\end{itemize}
Um dos primeiros dos 33 graus hierárchicos do rito escocês da Maçonaria.
\section{Realçamento}
\begin{itemize}
\item {Grp. gram.:m.}
\end{itemize}
O mesmo que \textunderscore realce\textunderscore .
\section{Realçar}
\begin{itemize}
\item {Grp. gram.:v. t.}
\end{itemize}
\begin{itemize}
\item {Proveniência:(De \textunderscore re...\textunderscore  + \textunderscore alçar\textunderscore )}
\end{itemize}
Pôr em lugares mais alto.
Tornar distinto, saliente.
Dar mais brilho, mais fôrça ou mais valor a.
\section{Realce}
\begin{itemize}
\item {Grp. gram.:m.}
\end{itemize}
Acto ou effeito de realçar; relêvo; distinção.
\section{Realço}
\begin{itemize}
\item {Grp. gram.:m.}
\end{itemize}
(V.realce)
\section{Realdade}
\begin{itemize}
\item {Grp. gram.:f.}
\end{itemize}
O mesmo que \textunderscore realidade\textunderscore . Cf. Filinto. \textunderscore D. Man.\textunderscore , III, 241.
\section{Realegrar}
\begin{itemize}
\item {Grp. gram.:v. t.}
\end{itemize}
\begin{itemize}
\item {Proveniência:(De \textunderscore re...\textunderscore  + \textunderscore alegrar\textunderscore )}
\end{itemize}
Alegrar muito; tornar a alegrar.
\section{Realejo}
\begin{itemize}
\item {Grp. gram.:m.}
\end{itemize}
Instrumento músico, que se toca com uma manivela.
(Cast. \textunderscore realejo\textunderscore )
\section{Realengamente}
\begin{itemize}
\item {Grp. gram.:adv.}
\end{itemize}
De modo realengo; á maneira de rei.
\section{Realengo}
\begin{itemize}
\item {Grp. gram.:adj.}
\end{itemize}
\begin{itemize}
\item {Grp. gram.:M.}
\end{itemize}
\begin{itemize}
\item {Proveniência:(De \textunderscore real\textunderscore ^2)}
\end{itemize}
Real, régio.
Peixe dos Açores, também chamado \textunderscore rei\textunderscore .
\section{Realengo}
\begin{itemize}
\item {Grp. gram.:adj.}
\end{itemize}
\begin{itemize}
\item {Utilização:Prov.}
\end{itemize}
\begin{itemize}
\item {Utilização:alent.}
\end{itemize}
\begin{itemize}
\item {Grp. gram.:M.}
\end{itemize}
\begin{itemize}
\item {Utilização:T. do Fundão}
\end{itemize}
Desordenado, sem rei nem roque: \textunderscore uma casa realenga\textunderscore .
O mesmo que \textunderscore relengo\textunderscore .
\section{Realentar}
\begin{itemize}
\item {Grp. gram.:v. t.}
\end{itemize}
\begin{itemize}
\item {Proveniência:(De \textunderscore re...\textunderscore  + \textunderscore alentar\textunderscore )}
\end{itemize}
Dar novo alento a; revigorar; estimular a coragem de. Cf. Arn. Gama, \textunderscore Última Dona\textunderscore , 178.
\section{Realeza}
\begin{itemize}
\item {Grp. gram.:f.}
\end{itemize}
\begin{itemize}
\item {Utilização:Fig.}
\end{itemize}
\begin{itemize}
\item {Proveniência:(De \textunderscore real\textunderscore ^2)}
\end{itemize}
Dignidade de rei ou raínha.
Grandeza, esplendor.
\section{Realeza}
\begin{itemize}
\item {Grp. gram.:f.}
\end{itemize}
\begin{itemize}
\item {Proveniência:(De \textunderscore real\textunderscore ^1)}
\end{itemize}
O mesmo que \textunderscore realidade\textunderscore . Cf. Vieira, VII, 520.
\section{Realidade}
\begin{itemize}
\item {Grp. gram.:f.}
\end{itemize}
Qualidade do que é real; aquillo que existe de facto.
\section{Realismo}
\begin{itemize}
\item {Grp. gram.:m.}
\end{itemize}
\begin{itemize}
\item {Proveniência:(De \textunderscore real\textunderscore ^1)}
\end{itemize}
Systema dos que suppõem conhecer o mundo exterior como realidade objectiva, em opposição ao systema dos que julgam que só conhecemos as nossas impressões.
Representação artística ou literária das scenas da natureza, aproveitadas nas suas minúcias e em toda a realidade.
\section{Realismo}
\begin{itemize}
\item {Grp. gram.:m.}
\end{itemize}
\begin{itemize}
\item {Proveniência:(De \textunderscore real\textunderscore ^2)}
\end{itemize}
Systema político, em que o chefe do estado é um rei.
\section{Realista}
\begin{itemize}
\item {Grp. gram.:adj.}
\end{itemize}
\begin{itemize}
\item {Proveniência:(De \textunderscore real\textunderscore ^1)}
\end{itemize}
Partidário do realismo ou relativo ao realismo, em Philosophia, em letras, na arte.
\section{Realista}
\begin{itemize}
\item {Grp. gram.:m. ,  f.  e  adj.}
\end{itemize}
\begin{itemize}
\item {Proveniência:(De \textunderscore real\textunderscore ^2)}
\end{itemize}
Pessôa, que é partidária da realeza ou de determinado rei; legitimista.
\section{Realistar}
\begin{itemize}
\item {Grp. gram.:v. t.}
\end{itemize}
\begin{itemize}
\item {Proveniência:(De \textunderscore re...\textunderscore  + \textunderscore alistar\textunderscore )}
\end{itemize}
Tornar a alistar. Cf. Castilho, \textunderscore D. Quixote\textunderscore , XXII.
\section{Realístico}
\begin{itemize}
\item {Grp. gram.:adj.}
\end{itemize}
\begin{itemize}
\item {Proveniência:(De \textunderscore realista\textunderscore ^1)}
\end{itemize}
Relativo ao realismo^1. Cf. R. de Brito, \textunderscore Philos. do Dir.\textunderscore ,184.
\section{Realização}
\begin{itemize}
\item {Grp. gram.:f.}
\end{itemize}
Acto ou effeito de realizar.
\section{Realizar}
\begin{itemize}
\item {Grp. gram.:v. t.}
\end{itemize}
\begin{itemize}
\item {Proveniência:(De \textunderscore real\textunderscore ^1)}
\end{itemize}
Tornar real.
Praticar.
Effectuar.
Converter em dinheiro ou em valor monetário.
\section{Realizável}
\begin{itemize}
\item {Grp. gram.:adj.}
\end{itemize}
Que se póde realizar.
\section{Realmente}
\begin{itemize}
\item {Grp. gram.:adv.}
\end{itemize}
De modo real^1; com effeito.
\section{Realmente}
\begin{itemize}
\item {Grp. gram.:adv.}
\end{itemize}
\begin{itemize}
\item {Proveniência:(De \textunderscore real\textunderscore ^2)}
\end{itemize}
Á maneira de rei; majestosamente, com magnificência.
\section{Reamanhecer}
\begin{itemize}
\item {Grp. gram.:v. i.}
\end{itemize}
\begin{itemize}
\item {Utilização:Fig.}
\end{itemize}
\begin{itemize}
\item {Proveniência:(De \textunderscore re...\textunderscore  + \textunderscore amanhecer\textunderscore )}
\end{itemize}
Amanhecer novamente.
Rejuvenescer.
\section{Reame}
\begin{itemize}
\item {Grp. gram.:m.}
\end{itemize}
\begin{itemize}
\item {Utilização:Ant.}
\end{itemize}
O reino ou o govêrno do reino.
(Cp. it. \textunderscore reame\textunderscore , fr. ant. \textunderscore reaume\textunderscore )
\section{Reanhas}
\begin{itemize}
\item {Grp. gram.:m.  e  f.}
\end{itemize}
\begin{itemize}
\item {Utilização:Prov.}
\end{itemize}
\begin{itemize}
\item {Utilização:trasm.}
\end{itemize}
Pessôa, diffícil de se aturar.
\section{Reanimação}
\begin{itemize}
\item {Grp. gram.:f.}
\end{itemize}
Acto ou effeito de reanimar.
\section{Reanimador}
\begin{itemize}
\item {Grp. gram.:m.  e  adj.}
\end{itemize}
O que reanima.
\section{Reanimar}
\begin{itemize}
\item {Grp. gram.:v. t.}
\end{itemize}
\begin{itemize}
\item {Proveniência:(De \textunderscore re...\textunderscore  + \textunderscore animar\textunderscore )}
\end{itemize}
Dar mais ânimo a.
Fazer renascer o ânimo em.
Fortificar; tonificar; restituir o uso dos sentidos a.
\section{Reaparecer}
\begin{itemize}
\item {Grp. gram.:v. i.}
\end{itemize}
\begin{itemize}
\item {Proveniência:(De \textunderscore re...\textunderscore  + \textunderscore aparecer\textunderscore )}
\end{itemize}
Aparecer de novo: \textunderscore reapareceu agora um antigo jornal\textunderscore .
\section{Reaparecimento}
\begin{itemize}
\item {Grp. gram.:m.}
\end{itemize}
O mesmo que \textunderscore reaparição\textunderscore .
\section{Reaparição}
\begin{itemize}
\item {Grp. gram.:f.}
\end{itemize}
Acto ou efeito de reaparecer.
\section{Reapoderar-se}
\begin{itemize}
\item {Grp. gram.:v. p.}
\end{itemize}
Apoderar-se novamente.
\section{Reapparecer}
\begin{itemize}
\item {Grp. gram.:v. i.}
\end{itemize}
\begin{itemize}
\item {Proveniência:(De \textunderscore re...\textunderscore  + \textunderscore apparecer\textunderscore )}
\end{itemize}
Apparecer de novo: \textunderscore reappareceu agora um antigo jornal\textunderscore .
\section{Reapparecimento}
\begin{itemize}
\item {Grp. gram.:m.}
\end{itemize}
O mesmo que \textunderscore reapparição\textunderscore .
\section{Reapparição}
\begin{itemize}
\item {Grp. gram.:f.}
\end{itemize}
Acto ou effeito de reapparecer.
\section{Reapreciar}
\begin{itemize}
\item {Grp. gram.:v. t.}
\end{itemize}
\begin{itemize}
\item {Proveniência:(De \textunderscore re...\textunderscore  + \textunderscore apreciar\textunderscore )}
\end{itemize}
Apreciar novamente.
\section{Reapresentar}
\begin{itemize}
\item {Grp. gram.:v. t.}
\end{itemize}
\begin{itemize}
\item {Proveniência:(De \textunderscore re...\textunderscore  + \textunderscore apresentar\textunderscore )}
\end{itemize}
Apresentar novamente.
\section{Reaquècer}
\begin{itemize}
\item {Grp. gram.:v. t.}
\end{itemize}
\begin{itemize}
\item {Proveniência:(De \textunderscore re...\textunderscore  + \textunderscore aquècer\textunderscore )}
\end{itemize}
Aquècer novamente.
\section{Reaquisição}
\begin{itemize}
\item {Grp. gram.:f.}
\end{itemize}
\begin{itemize}
\item {Proveniência:(De \textunderscore re...\textunderscore  + \textunderscore acquisição\textunderscore )}
\end{itemize}
Acto ou effeito de reaquirir.
\section{Rearborização}
\begin{itemize}
\item {Grp. gram.:f.}
\end{itemize}
Acto ou effeito de rearborizar.
\section{Rearborizar}
\begin{itemize}
\item {Grp. gram.:v. t.}
\end{itemize}
\begin{itemize}
\item {Proveniência:(De \textunderscore re...\textunderscore  + \textunderscore arborizar\textunderscore )}
\end{itemize}
Arborizar novamente.
\section{Reascender}
\begin{itemize}
\item {Grp. gram.:v. i.}
\end{itemize}
\begin{itemize}
\item {Grp. gram.:V. t.}
\end{itemize}
\begin{itemize}
\item {Proveniência:(De \textunderscore re...\textunderscore  + \textunderscore ascender\textunderscore )}
\end{itemize}
Ascender novamente.
Fazer subir de novo.
Tornar a elevar.
\section{Reassignar}
\begin{itemize}
\item {Grp. gram.:v. t.}
\end{itemize}
\begin{itemize}
\item {Proveniência:(De \textunderscore re...\textunderscore  + \textunderscore assinar\textunderscore )}
\end{itemize}
Assinar de novo.
\section{Reassinar}
\begin{itemize}
\item {Grp. gram.:v. t.}
\end{itemize}
\begin{itemize}
\item {Proveniência:(De \textunderscore re...\textunderscore  + \textunderscore assinar\textunderscore )}
\end{itemize}
Assinar de novo.
\section{Reassumir}
\begin{itemize}
\item {Grp. gram.:v. t.}
\end{itemize}
\begin{itemize}
\item {Proveniência:(Lat. \textunderscore reassumere\textunderscore )}
\end{itemize}
Assumir de novo.
Readquirir.
Tomar novamente posse de: \textunderscore reassumir um emprêgo\textunderscore .
\section{Reassumpção}
\begin{itemize}
\item {Grp. gram.:f.}
\end{itemize}
\begin{itemize}
\item {Proveniência:(De \textunderscore re...\textunderscore  + \textunderscore assumpção\textunderscore )}
\end{itemize}
Acto ou effeito de reassumir.
\section{Reassunção}
\begin{itemize}
\item {Grp. gram.:f.}
\end{itemize}
\begin{itemize}
\item {Proveniência:(De \textunderscore re...\textunderscore  + \textunderscore assunção\textunderscore )}
\end{itemize}
Acto ou efeito de reassumir.
\section{Reata}
\begin{itemize}
\item {Grp. gram.:f.}
\end{itemize}
O mesmo que \textunderscore arreata\textunderscore . Cf. Castilho, \textunderscore D. Quixote\textunderscore , II, 263.
(Cp. \textunderscore reatar\textunderscore . Cast. \textunderscore reata\textunderscore )
\section{Reataduras}
\begin{itemize}
\item {Grp. gram.:f. pl.}
\end{itemize}
\begin{itemize}
\item {Utilização:Náut.}
\end{itemize}
\begin{itemize}
\item {Proveniência:(De \textunderscore reatar\textunderscore )}
\end{itemize}
Cordas ou chapas de ferros, com que se ligam as partes fendidas de um mastro, vêrga, etc.
\section{Reatamento}
\begin{itemize}
\item {Grp. gram.:m.}
\end{itemize}
Acto ou effeito de reatar.
\section{Reatar}
\begin{itemize}
\item {Grp. gram.:v. t.}
\end{itemize}
\begin{itemize}
\item {Proveniência:(De \textunderscore re...\textunderscore  + \textunderscore atar\textunderscore )}
\end{itemize}
Atar de novo.
Ligar com reataduras.
Proseguir em (uma coisa interrompida): \textunderscore reatar relações de amizade\textunderscore .
\section{Reatas}
\begin{itemize}
\item {Grp. gram.:f. pl.}
\end{itemize}
O mesmo que \textunderscore reataduras\textunderscore .
\section{Reato}
\begin{itemize}
\item {Grp. gram.:m.}
\end{itemize}
\begin{itemize}
\item {Utilização:Des.}
\end{itemize}
\begin{itemize}
\item {Proveniência:(Lat. \textunderscore reatus\textunderscore )}
\end{itemize}
Estado de réu.
Obrigação de cumprir penitência, imposta pelo confessor.
\section{Reaumúria}
\begin{itemize}
\item {Grp. gram.:f.}
\end{itemize}
\begin{itemize}
\item {Proveniência:(De \textunderscore Reaumur\textunderscore , n. p.)}
\end{itemize}
Gênero de plantas.
\section{Reaviar}
\begin{itemize}
\item {Grp. gram.:v. t.}
\end{itemize}
\begin{itemize}
\item {Proveniência:(De \textunderscore re...\textunderscore  + \textunderscore aviar\textunderscore )}
\end{itemize}
Fazer entrar de novo no caminho; encaminhar, orientar.
\section{Reavisar}
\begin{itemize}
\item {Grp. gram.:v. t.}
\end{itemize}
\begin{itemize}
\item {Proveniência:(De \textunderscore re...\textunderscore  + \textunderscore avisar\textunderscore )}
\end{itemize}
Avisar novamente.
Tornar prudente.
\section{Reaviso}
\begin{itemize}
\item {Grp. gram.:m.}
\end{itemize}
\begin{itemize}
\item {Proveniência:(De \textunderscore re...\textunderscore  + \textunderscore aviso\textunderscore )}
\end{itemize}
Acto ou effeito de reavisar.
\section{Reavivar}
\begin{itemize}
\item {Grp. gram.:v. t.}
\end{itemize}
\begin{itemize}
\item {Proveniência:(De \textunderscore re...\textunderscore  + \textunderscore avivar\textunderscore )}
\end{itemize}
Avivar muito.
Tornar bem lembrado.
Estimular (a memória).
\section{Rebaixa}
\begin{itemize}
\item {Grp. gram.:f.}
\end{itemize}
\begin{itemize}
\item {Proveniência:(De \textunderscore rebaixar\textunderscore )}
\end{itemize}
Acto de baixar o preço.
\section{Rebaixadeira}
\begin{itemize}
\item {Grp. gram.:f.}
\end{itemize}
\begin{itemize}
\item {Utilização:Prov.}
\end{itemize}
\begin{itemize}
\item {Utilização:beir.}
\end{itemize}
\begin{itemize}
\item {Proveniência:(De \textunderscore rebaixar\textunderscore )}
\end{itemize}
Instrumento de carpinteiro, espécie de plaina.
\section{Rebaixador}
\begin{itemize}
\item {Grp. gram.:m.}
\end{itemize}
\begin{itemize}
\item {Proveniência:(De \textunderscore rebaixar\textunderscore )}
\end{itemize}
Instrumento, com que os carpinteiros rebaixam os ângulos de uma peça de madeira.
\section{Rebaixamento}
\begin{itemize}
\item {Grp. gram.:m.}
\end{itemize}
Acto ou effeito de rebaixar.
\section{Rebaixar}
\begin{itemize}
\item {Grp. gram.:v. t.}
\end{itemize}
\begin{itemize}
\item {Utilização:Fig.}
\end{itemize}
\begin{itemize}
\item {Grp. gram.:V. i.}
\end{itemize}
\begin{itemize}
\item {Utilização:P. us.}
\end{itemize}
\begin{itemize}
\item {Grp. gram.:V. p.}
\end{itemize}
\begin{itemize}
\item {Proveniência:(De \textunderscore re...\textunderscore  + \textunderscore baixar\textunderscore )}
\end{itemize}
Fazer mais baixo.
Fazer deminuir o preço ou valor de.
Aviltar; desacreditar.
Deminuir na altura.
Aviltar-se, praticar actos indignos.
Humilhar-se.
\section{Rebaixe}
\begin{itemize}
\item {Grp. gram.:m.}
\end{itemize}
O mesmo que \textunderscore rebaixo\textunderscore .
Escavação longitudinal. Cf. \textunderscore Rev. de Guimarães\textunderscore , XV, 160.
\section{Rebaixo}
\begin{itemize}
\item {Grp. gram.:m.}
\end{itemize}
\begin{itemize}
\item {Utilização:Carp.}
\end{itemize}
\begin{itemize}
\item {Proveniência:(De \textunderscore rebaixar\textunderscore )}
\end{itemize}
Rebaixamento.
A parte que se rebaixou.
Vão de escada.
Quarto esconso, debaixo de um tecto inclinado.
Encaixe, aberto numa peça, para alli se embeber outra peça.
\section{Rebalde}
\begin{itemize}
\item {Grp. gram.:m.}
\end{itemize}
\begin{itemize}
\item {Utilização:Des.}
\end{itemize}
O mesmo que \textunderscore arrabalde\textunderscore ; vizinhança, arredores. Cf. Filinto, XV, 203.
\section{Rebaldeira}
\begin{itemize}
\item {Grp. gram.:f.}
\end{itemize}
(V.ramaldeira)
\section{Rebaldio}
\begin{itemize}
\item {Grp. gram.:adj.}
\end{itemize}
Diz-se de uma espécie de figo, fruto de figueira brava. Cf. \textunderscore Ethiópia Or.\textunderscore , I, 49.
\section{Rebalsar}
\begin{itemize}
\item {Grp. gram.:v. i.  e  p.}
\end{itemize}
Tornar-se pantanoso.
Têr a qualidade de paúl; estagnar-se.
\section{Rebamento}
\begin{itemize}
\item {Grp. gram.:m.}
\end{itemize}
\begin{itemize}
\item {Utilização:Prov.}
\end{itemize}
\begin{itemize}
\item {Utilização:beir.}
\end{itemize}
Acto de rebar.
\section{Rebanhada}
\begin{itemize}
\item {Grp. gram.:f.}
\end{itemize}
\begin{itemize}
\item {Utilização:Fig.}
\end{itemize}
Grande rebanho.
Grande ajuntamento de pessôas.
\section{Rebanhar}
\begin{itemize}
\item {Grp. gram.:v. t.}
\end{itemize}
O mesmo que \textunderscore arrebanhar\textunderscore ^1.
\section{Rebanhio}
\begin{itemize}
\item {Grp. gram.:adj.}
\end{itemize}
Que vive ou anda em rebanho.
\section{Rebanho}
\begin{itemize}
\item {Grp. gram.:m.}
\end{itemize}
\begin{itemize}
\item {Utilização:Fig.}
\end{itemize}
\begin{itemize}
\item {Proveniência:(Do lat. hyp. \textunderscore herbaneus\textunderscore , de \textunderscore herba\textunderscore )}
\end{itemize}
Porção de gado lanígero.
Porção de animaes, guardados por pastor.
Ajuntamento de animaes.
Grupo de homens, que seguem a vontade ou o arbítrio de alguém.
Conjunto dos parochianos.
Grei.
\section{Rebanho}
\begin{itemize}
\item {Grp. gram.:m.}
\end{itemize}
Ave de rapina, o mesmo que \textunderscore gavião\textunderscore  ou \textunderscore francelho\textunderscore .
(Provavelmente, corr. de \textunderscore rabanho\textunderscore , de \textunderscore rabo\textunderscore )
\section{Rebaptismo}
\begin{itemize}
\item {Grp. gram.:m.}
\end{itemize}
\begin{itemize}
\item {Proveniência:(De \textunderscore re...\textunderscore  + \textunderscore baptismo\textunderscore )}
\end{itemize}
Acto ou effeito de rebaptizar.
\section{Rebaptizar}
\begin{itemize}
\item {Grp. gram.:v. t.}
\end{itemize}
\begin{itemize}
\item {Proveniência:(De \textunderscore re...\textunderscore  + \textunderscore baptizar\textunderscore )}
\end{itemize}
Baptizar novamente.
\section{Rebar}
\begin{itemize}
\item {Grp. gram.:v. t.}
\end{itemize}
\begin{itemize}
\item {Utilização:Prov.}
\end{itemize}
\begin{itemize}
\item {Utilização:beir.}
\end{itemize}
Encher com rebos ou pequenas pedras (o vão de uma parede).
\section{Rebarba}
\begin{itemize}
\item {Grp. gram.:f.}
\end{itemize}
\begin{itemize}
\item {Proveniência:(De \textunderscore re...\textunderscore  + \textunderscore barba\textunderscore )}
\end{itemize}
Parte saliente.
Aresta.
Proeminência de obras de fundição, resultante da penetração do metal fundido na juntura das fórmas.
Intervallo, entre duas linhas regulares de composição typographica.
\section{Rebarbador}
\begin{itemize}
\item {Grp. gram.:m.}
\end{itemize}
\begin{itemize}
\item {Utilização:Náut.}
\end{itemize}
\begin{itemize}
\item {Proveniência:(De \textunderscore rebarbar\textunderscore )}
\end{itemize}
Operário, que tira as rebarbas.
\section{Rebarbar}
\begin{itemize}
\item {Grp. gram.:v. t.}
\end{itemize}
Tirar as rebarbas a.
Raspar as rebarbas de.
\section{Rebarbativo}
\begin{itemize}
\item {Grp. gram.:adj.}
\end{itemize}
\begin{itemize}
\item {Proveniência:(De \textunderscore re...\textunderscore  + \textunderscore barba\textunderscore )}
\end{itemize}
Que, por excesso de gordura, parece têr duas barbas ou apresenta refegos na pelle, por baixo da maxilla inferior.
\section{Rebatar}
\textunderscore v. t.\textunderscore  (e der.)
(V. \textunderscore arrebatar\textunderscore , etc.). Cf. Filinto, VI, 235.
\section{Rebate}
\begin{itemize}
\item {Grp. gram.:m.}
\end{itemize}
\begin{itemize}
\item {Utilização:T. da Bairrada}
\end{itemize}
\begin{itemize}
\item {Utilização:Prov.}
\end{itemize}
\begin{itemize}
\item {Utilização:alent.}
\end{itemize}
\begin{itemize}
\item {Utilização:Fig.}
\end{itemize}
\begin{itemize}
\item {Utilização:T. da Ilha-de-San-Jorge}
\end{itemize}
Acto ou effeito de rebater.
Assalto repentino.
Acto de chamar ou de dar aviso de um ataque ou acontecimento imprevisto.
Estímulo.
Degrau de escada, cuja face superior é a soleira da porta da rua.
Cavilha da charrua, que entra no forcaz.
Supposição, palpite.
Prognóstico.
Lembrança.
Inflammação no peito da mulhér.
\section{Rebatedor}
\begin{itemize}
\item {Grp. gram.:m.  e  adj.}
\end{itemize}
O que rebate.
\section{Rebater}
\begin{itemize}
\item {Grp. gram.:v. t.}
\end{itemize}
\begin{itemize}
\item {Proveniência:(De \textunderscore re...\textunderscore  + \textunderscore bater\textunderscore )}
\end{itemize}
Bater novamente.
Bater muito.
Repellir.
Debellar.
Contestar; refutar.
Conter; sustar.
Receber em desconto.
Trocar por dinheiro com desconto: \textunderscore rebater uma cautela da lotaria\textunderscore .
Adeantar com ágio (uma quantia): \textunderscore rebater o ordenado\textunderscore .
\section{Rebatimento}
\begin{itemize}
\item {Grp. gram.:m.}
\end{itemize}
Acto ou effeito de rebater.
\section{Rebatina}
\begin{itemize}
\item {Grp. gram.:f.}
\end{itemize}
(V.rebatinha)
\section{Rebatinha}
\begin{itemize}
\item {Grp. gram.:f.}
\end{itemize}
\begin{itemize}
\item {Utilização:Ant.}
\end{itemize}
Coisa muito disputada; porfia.
(Cp. cast. \textunderscore rebatiña\textunderscore )
\section{Rebato}
\begin{itemize}
\item {Grp. gram.:m.}
\end{itemize}
Soleira da porta:«\textunderscore ...debaixo da pedra, que havia de servir de rebato á porta da fortaleza...\textunderscore »Filinto, \textunderscore D. Man.\textunderscore , II, 95.
(Cp. \textunderscore rebate\textunderscore )
\section{Rebeca}
\begin{itemize}
\item {Grp. gram.:f.}
\end{itemize}
\begin{itemize}
\item {Proveniência:(Do fr. ant. \textunderscore rebec\textunderscore )}
\end{itemize}
(V.rabeca)
\section{Rebeijar}
\begin{itemize}
\item {Grp. gram.:v. t.}
\end{itemize}
\begin{itemize}
\item {Proveniência:(De \textunderscore re...\textunderscore  + \textunderscore beijar\textunderscore )}
\end{itemize}
Beijar novamente. Cf. G. Crespo, \textunderscore Nocturnos\textunderscore , 121.
\section{Rebela}
\begin{itemize}
\item {fónica:bê}
\end{itemize}
\begin{itemize}
\item {Grp. gram.:f.}
\end{itemize}
Variedade de maçan.
\section{Rebelador}
\begin{itemize}
\item {Grp. gram.:adj.}
\end{itemize}
\begin{itemize}
\item {Utilização:T. de Ceilão}
\end{itemize}
\begin{itemize}
\item {Proveniência:(De \textunderscore rebelar\textunderscore )}
\end{itemize}
O mesmo que \textunderscore rebelde\textunderscore .
\section{Rebelão}
\begin{itemize}
\item {Grp. gram.:adj.}
\end{itemize}
\begin{itemize}
\item {Utilização:Fig.}
\end{itemize}
\begin{itemize}
\item {Proveniência:(De \textunderscore rebelar\textunderscore )}
\end{itemize}
Diz-se do cavalo, que não obedece ao freio.
Teimoso; que não escuta a voz da razão.
\section{Rebelar}
\begin{itemize}
\item {Grp. gram.:v. t.}
\end{itemize}
\begin{itemize}
\item {Grp. gram.:V. i.}
\end{itemize}
\begin{itemize}
\item {Proveniência:(Lat. \textunderscore rebellare\textunderscore )}
\end{itemize}
Tornar rebelde; insurgir, revoltar.
Tornar-se rebelde. Cf. Rui Barb., \textunderscore Réplica\textunderscore , 160.
\section{Rebeldaria}
\begin{itemize}
\item {Grp. gram.:f.}
\end{itemize}
(V.rebeldia)
\section{Rebelde}
\begin{itemize}
\item {Grp. gram.:adj.}
\end{itemize}
\begin{itemize}
\item {Grp. gram.:M.}
\end{itemize}
\begin{itemize}
\item {Proveniência:(Do lat. hyp. \textunderscore rebellitare\textunderscore , freq. de \textunderscore rebellare\textunderscore )}
\end{itemize}
Que se revolta ou se insurge, especialmente contra o Govêrno ou contra as autoridades constituídas.
Insurgente.
Que faz opposição.
Teimoso.
Indomável: \textunderscore cavallo rebelde\textunderscore .
Que se debella difficilmente: \textunderscore doença rebelde\textunderscore .
Escabroso.
Desagradável: \textunderscore tempo rebelde\textunderscore .
Aquelle que é insurgente ou revoltado: \textunderscore os rebeldes venceram\textunderscore .
Desertor.
\section{Rebeldia}
\begin{itemize}
\item {Grp. gram.:f.}
\end{itemize}
\begin{itemize}
\item {Utilização:Fig.}
\end{itemize}
Acto de rebellar.
Rebellião.
Qualidade do que é rebelde.
Opposição.
Pertinácia.
\section{Rebelião}
\begin{itemize}
\item {Grp. gram.:m.}
\end{itemize}
\begin{itemize}
\item {Proveniência:(Lat. \textunderscore rebellio\textunderscore )}
\end{itemize}
Acto de se rebelar; revolta; insurreição.
\section{Rebelim}
\begin{itemize}
\item {Grp. gram.:m.}
\end{itemize}
O mesmo que \textunderscore revelim\textunderscore . Cf. \textunderscore Viriato Trág.\textunderscore , IV, 16.
\section{Rebella}
\begin{itemize}
\item {fónica:bê}
\end{itemize}
\begin{itemize}
\item {Grp. gram.:f.}
\end{itemize}
Variedade de maçan.
\section{Rebellador}
\begin{itemize}
\item {Grp. gram.:adj.}
\end{itemize}
\begin{itemize}
\item {Utilização:T. de Ceilão}
\end{itemize}
\begin{itemize}
\item {Proveniência:(De \textunderscore rebellar\textunderscore )}
\end{itemize}
O mesmo que \textunderscore rebelde\textunderscore .
\section{Rebellão}
\begin{itemize}
\item {Grp. gram.:adj.}
\end{itemize}
\begin{itemize}
\item {Utilização:Fig.}
\end{itemize}
\begin{itemize}
\item {Proveniência:(De \textunderscore rebellar\textunderscore )}
\end{itemize}
Diz-se do cavallo, que não obedece ao freio.
Teimoso; que não escuta a voz da razão.
\section{Rebellar}
\begin{itemize}
\item {Grp. gram.:v. t.}
\end{itemize}
\begin{itemize}
\item {Grp. gram.:V. i.}
\end{itemize}
\begin{itemize}
\item {Proveniência:(Lat. \textunderscore rebellare\textunderscore )}
\end{itemize}
Tornar rebelde; insurgir, revoltar.
Tornar-se rebelde. Cf. Rui Barb., \textunderscore Réplica\textunderscore , 160.
\section{Rebellião}
\begin{itemize}
\item {Grp. gram.:m.}
\end{itemize}
\begin{itemize}
\item {Proveniência:(Lat. \textunderscore rebellio\textunderscore )}
\end{itemize}
Acto de se rebellar; revolta; insurreição.
\section{Rebelloso}
\begin{itemize}
\item {Grp. gram.:adj.}
\end{itemize}
\begin{itemize}
\item {Utilização:T. de Ceilão}
\end{itemize}
\begin{itemize}
\item {Proveniência:(De \textunderscore rebellar\textunderscore )}
\end{itemize}
O mesmo que \textunderscore rebelde\textunderscore .
\section{Rebeloso}
\begin{itemize}
\item {Grp. gram.:adj.}
\end{itemize}
\begin{itemize}
\item {Utilização:T. de Ceilão}
\end{itemize}
\begin{itemize}
\item {Proveniência:(De \textunderscore rebelar\textunderscore )}
\end{itemize}
O mesmo que \textunderscore rebelde\textunderscore .
\section{Rebém}
\begin{itemize}
\item {Grp. gram.:m.}
\end{itemize}
Azorrague, com que se castigavam os condemnados.
(Cp. \textunderscore rebenque\textunderscore )
\section{Rebém}
\begin{itemize}
\item {Grp. gram.:adv.}
\end{itemize}
\begin{itemize}
\item {Utilização:Des.}
\end{itemize}
\begin{itemize}
\item {Proveniência:(De \textunderscore re...\textunderscore  + \textunderscore bem\textunderscore )}
\end{itemize}
Muito bem.
\section{Rebencaço}
\begin{itemize}
\item {Grp. gram.:m.}
\end{itemize}
\begin{itemize}
\item {Utilização:Bras. do S}
\end{itemize}
Pancada com rebenque.
\section{Rebencada}
\begin{itemize}
\item {Grp. gram.:f.}
\end{itemize}
O mesmo que \textunderscore rebencaço\textunderscore .
\section{Rebenque}
\begin{itemize}
\item {Grp. gram.:m.}
\end{itemize}
\begin{itemize}
\item {Utilização:Bras. do S}
\end{itemize}
Espécie de chicote pequeno, com que o cavalleiro incita o cavallo.
(Cast. \textunderscore rebenque\textunderscore )
\section{Rebenqueador}
\begin{itemize}
\item {Grp. gram.:m.}
\end{itemize}
\begin{itemize}
\item {Utilização:Bras}
\end{itemize}
Aquelle que rebenqueia.
O que castiga com frequência.
\section{Rebenquear}
\begin{itemize}
\item {Grp. gram.:v. t.}
\end{itemize}
Açoitar ou castigar com o rebenque.
\section{Rebenta-boi}
\begin{itemize}
\item {Grp. gram.:m.}
\end{itemize}
\begin{itemize}
\item {Utilização:Bot.}
\end{itemize}
Espécie de jarro, o mesmo que \textunderscore arrebenta-boi\textunderscore .
O mesmo que \textunderscore roseira-canina\textunderscore , ou o seu fruto.
\section{Rebentação}
\begin{itemize}
\item {Grp. gram.:f.}
\end{itemize}
Acto de rebentar. Cf. \textunderscore Rev. Agron.\textunderscore , I, 118.
O quebrar das ondas contra os rochedos ou no dorso dos navios. Cf. Camillo, \textunderscore Quéda\textunderscore , 144.
\section{Rebentão}
\begin{itemize}
\item {Grp. gram.:m.}
\end{itemize}
\begin{itemize}
\item {Utilização:Fig.}
\end{itemize}
\begin{itemize}
\item {Utilização:Pop.}
\end{itemize}
\begin{itemize}
\item {Proveniência:(De \textunderscore rebento\textunderscore )}
\end{itemize}
Haste que, brotando da raiz da planta, póde produzir novo indivíduo vegetal.
Descendente.
Leicenço, abscesso.
\section{Rebentão}
\begin{itemize}
\item {Grp. gram.:m.}
\end{itemize}
\begin{itemize}
\item {Utilização:Açor}
\end{itemize}
\begin{itemize}
\item {Proveniência:(De \textunderscore rebentar\textunderscore ?)}
\end{itemize}
Ladeira muito íngreme.
\section{Rebentar}
\begin{itemize}
\item {Grp. gram.:v. i.}
\end{itemize}
\begin{itemize}
\item {Grp. gram.:V. t.}
\end{itemize}
\begin{itemize}
\item {Proveniência:(Do lat. hyp. \textunderscore repeditare\textunderscore )}
\end{itemize}
Explodir: \textunderscore rebentou a bomba\textunderscore .
Fazer estrépito.
Estoirar.
Mostrar-se ou apparecer violentamente: \textunderscore rebentou uma tempestade\textunderscore .
Fazer grande estrondo.
Abrir-se.
Manar, nascer: \textunderscore rebentar uma fonte\textunderscore .
Desabrochar: \textunderscore a vegetação rebenta\textunderscore .
Surgir, apparecer.
Partir com estrondo.
Despedaçar com ruído.
\section{Rebentina}
\begin{itemize}
\item {Grp. gram.:f.}
\end{itemize}
\begin{itemize}
\item {Proveniência:(De \textunderscore repente\textunderscore ?)}
\end{itemize}
Raiva; accesso de fúria.
\section{Rebentinha}
\begin{itemize}
\item {Grp. gram.:f.}
\end{itemize}
O mesmo que \textunderscore rebentina\textunderscore . Cf. G. Vicente, \textunderscore Inês Pereira\textunderscore .
\section{Rebento}
\begin{itemize}
\item {Grp. gram.:m.}
\end{itemize}
\begin{itemize}
\item {Utilização:Fig.}
\end{itemize}
\begin{itemize}
\item {Proveniência:(De \textunderscore rebentar\textunderscore )}
\end{itemize}
Botão dos vegetaes; renôvo.
Producto.
\section{Rebentona}
\begin{itemize}
\item {Grp. gram.:f.}
\end{itemize}
\begin{itemize}
\item {Utilização:Bras. do S}
\end{itemize}
\begin{itemize}
\item {Proveniência:(De \textunderscore rebentar\textunderscore )}
\end{itemize}
Questão importante, que se vai decidir em breve.
\section{Rebesgado}
\begin{itemize}
\item {Grp. gram.:adj.}
\end{itemize}
\begin{itemize}
\item {Utilização:Prov.}
\end{itemize}
\begin{itemize}
\item {Utilização:trasm.}
\end{itemize}
O mesmo que \textunderscore arrevesado\textunderscore , (falando-se de nomes).
\section{Rebiasco}
\begin{itemize}
\item {Grp. gram.:m.}
\end{itemize}
\begin{itemize}
\item {Utilização:Prov.}
\end{itemize}
\begin{itemize}
\item {Utilização:trasm.}
\end{itemize}
O mesmo que \textunderscore bordado\textunderscore .
\section{Rebiassacos}
\begin{itemize}
\item {Grp. gram.:m. pl.}
\end{itemize}
\begin{itemize}
\item {Utilização:Prov.}
\end{itemize}
\begin{itemize}
\item {Utilização:trasm.}
\end{itemize}
Gatimanhos.
Momices, fosquinhas.
\section{Rebicar}
\textunderscore v. t.\textunderscore  (e der.)
O mesmo que \textunderscore arrebicar\textunderscore , etc. Cf. Filinto, XV, 295.
\section{Rebimba}
\begin{itemize}
\item {Grp. gram.:f.}
\end{itemize}
\begin{itemize}
\item {Utilização:Des.}
\end{itemize}
\begin{itemize}
\item {Grp. gram.:Loc.}
\end{itemize}
\begin{itemize}
\item {Utilização:Prov.}
\end{itemize}
Indolência, preguiça.
\textunderscore De rebimba ao malho\textunderscore , com molleza, á bambalhona.
\section{Rebique}
\begin{itemize}
\item {Grp. gram.:m.}
\end{itemize}
\begin{itemize}
\item {Proveniência:(Do ár. \textunderscore rabic\textunderscore )}
\end{itemize}
O mesmo que \textunderscore arrebique\textunderscore . Cf. Filinto, I, 104.
\section{Rebitagem}
\begin{itemize}
\item {Grp. gram.:f.}
\end{itemize}
\begin{itemize}
\item {Proveniência:(De \textunderscore rebitar\textunderscore )}
\end{itemize}
Acto de fazer o rebite a um conjunto de pregos; cravação.
\section{Rebitar}
\begin{itemize}
\item {Grp. gram.:v. t.}
\end{itemize}
O mesmo que \textunderscore arrebitar\textunderscore :«\textunderscore ...por entre o rebitar de beiços.\textunderscore »Camillo, \textunderscore Caveira\textunderscore , 356.
\section{Rebite}
\begin{itemize}
\item {Grp. gram.:m.}
\end{itemize}
\begin{itemize}
\item {Proveniência:(De \textunderscore rebitar\textunderscore )}
\end{itemize}
Dobra ou volta, na extremidade de um prego, para que se não solte da madeira.
\section{Rebo}
\begin{itemize}
\item {fónica:rê}
\end{itemize}
\begin{itemize}
\item {Grp. gram.:m.}
\end{itemize}
Pequena pedra bruta; calhau.
\section{Reboante}
\begin{itemize}
\item {Grp. gram.:adj.}
\end{itemize}
\begin{itemize}
\item {Proveniência:(Lat. \textunderscore reboans\textunderscore )}
\end{itemize}
Que rebôa.
\section{Reboar}
\begin{itemize}
\item {Grp. gram.:v. i.}
\end{itemize}
\begin{itemize}
\item {Proveniência:(Lat. \textunderscore reboare\textunderscore )}
\end{itemize}
Fazer echo; retumbar; repercutir-se.
\section{Rebocador}
\begin{itemize}
\item {Grp. gram.:adj.}
\end{itemize}
\begin{itemize}
\item {Grp. gram.:M.}
\end{itemize}
Que reboca.
Aquelle que reboca.
\section{Rebocador}
\begin{itemize}
\item {Grp. gram.:m.}
\end{itemize}
Pequeno navio a vapor, que serve para rebocar outros navios.
\section{Rebocadura}
\begin{itemize}
\item {Grp. gram.:f.}
\end{itemize}
O mesmo que \textunderscore reboque\textunderscore ^1 ou \textunderscore rebôco\textunderscore .
\section{Rebocar}
\begin{itemize}
\item {Grp. gram.:v. t.}
\end{itemize}
Revestir de rebôco.
\section{Rebocar}
\begin{itemize}
\item {Grp. gram.:v. t.}
\end{itemize}
\begin{itemize}
\item {Proveniência:(Do lat. \textunderscore remulcare\textunderscore )}
\end{itemize}
Dar reboque a.
\section{Rebôco}
\begin{itemize}
\item {Grp. gram.:m.}
\end{itemize}
\begin{itemize}
\item {Proveniência:(Do ár. \textunderscore rabug\textunderscore , arranjo, arrumação)}
\end{itemize}
Argamassa, com que se revestem paredes.
Substância, com que se reveste a superfície interior de um vaso, para o vedar, ou para outro fim.
\section{Reboço}
\begin{itemize}
\item {fónica:bô}
\end{itemize}
\begin{itemize}
\item {Grp. gram.:m.}
\end{itemize}
\begin{itemize}
\item {Utilização:Bras}
\end{itemize}
Novo emboço.
\section{Rebojo}
\begin{itemize}
\item {Grp. gram.:m.}
\end{itemize}
\begin{itemize}
\item {Utilização:Bras}
\end{itemize}
\begin{itemize}
\item {Proveniência:(De \textunderscore re...\textunderscore  + \textunderscore bójo\textunderscore )}
\end{itemize}
Curva, formada pela quéda das cachoeiras.
Desvio ou redemoínho de vento, por effeito de um corpo que encontra e lhe altera a direcção primitiva.
\section{Rebolado}
\begin{itemize}
\item {Grp. gram.:m.}
\end{itemize}
\begin{itemize}
\item {Proveniência:(De \textunderscore rebolar\textunderscore )}
\end{itemize}
Movimento de quadris, saracoteio.
\section{Rebolal}
\begin{itemize}
\item {Grp. gram.:adj.}
\end{itemize}
\begin{itemize}
\item {Utilização:Prov.}
\end{itemize}
\begin{itemize}
\item {Utilização:trasm.}
\end{itemize}
\begin{itemize}
\item {Proveniência:(De \textunderscore rebôlo\textunderscore )}
\end{itemize}
Diz-se do terreno, onde abundam seixos, de fórma irregular.
\section{Rebolão}
\begin{itemize}
\item {Grp. gram.:m.}
\end{itemize}
O mesmo que \textunderscore fanfarrão\textunderscore :«\textunderscore Jesu, como é rebolão!\textunderscore »G. Vicente, \textunderscore Auto da Índia\textunderscore .
(Cp. \textunderscore rebolaria\textunderscore ^1)
\section{Rebolar}
\begin{itemize}
\item {Grp. gram.:v. t.}
\end{itemize}
\begin{itemize}
\item {Grp. gram.:V. i.  e  p.}
\end{itemize}
\begin{itemize}
\item {Proveniência:(De \textunderscore re...\textunderscore  + \textunderscore bola\textunderscore )}
\end{itemize}
Fazer mover (um corpo mais ou menos redondo), como em volta de um eixo se faz mover uma bola.
Fazer mover como uma bola.
Mover-se em volta de um centro; rolar.
\section{Rebolaria}
\begin{itemize}
\item {Grp. gram.:f.}
\end{itemize}
Fanfarronada, bravata.
(Corr. de \textunderscore rabularía\textunderscore ?)
\section{Rebolaria}
\begin{itemize}
\item {Grp. gram.:f.}
\end{itemize}
\begin{itemize}
\item {Utilização:Ant.}
\end{itemize}
Enfeites exaggerados, como de mulhéres de má nota:«\textunderscore que são? que são...? rebolarias...\textunderscore »G. Vicente, III, 14.
\section{Rebolcar}
\begin{itemize}
\item {Grp. gram.:v. t.}
\end{itemize}
\begin{itemize}
\item {Grp. gram.:v. p.}
\end{itemize}
\begin{itemize}
\item {Proveniência:(De \textunderscore re...\textunderscore  + \textunderscore bolcar\textunderscore )}
\end{itemize}
Fazer rebolar; fazer mover como uma bola.
Revolver, virando.
Revolver-se virando-se: \textunderscore o jumento rebolcava-se na lama\textunderscore .
\section{Reboldrosa}
\begin{itemize}
\item {Grp. gram.:f.}
\end{itemize}
\begin{itemize}
\item {Utilização:Bras. do N}
\end{itemize}
Descompostura, reprimenda.
\section{Rebolear}
\begin{itemize}
\item {Grp. gram.:v. t.}
\end{itemize}
\begin{itemize}
\item {Utilização:Bras. do S}
\end{itemize}
\begin{itemize}
\item {Grp. gram.:V. p.}
\end{itemize}
\begin{itemize}
\item {Proveniência:(De \textunderscore re...\textunderscore  + \textunderscore bola\textunderscore )}
\end{itemize}
Dar movimento de rotação ao laço ou bolas contra o animal que se vai laçar.
Mover-se para um e outro lado; bambolear-se; saracotear-se.
\section{Reboleira}
\begin{itemize}
\item {Grp. gram.:f.}
\end{itemize}
A parte mais densa de uma seara, arvoredo, etc.
\section{Reboleira}
\begin{itemize}
\item {Grp. gram.:f.}
\end{itemize}
\begin{itemize}
\item {Proveniência:(De \textunderscore rebôlo\textunderscore )}
\end{itemize}
Lodo, que se accumula na caixa onde gira a pedra de amolar.
\section{Reboleiro}
\begin{itemize}
\item {Grp. gram.:m.}
\end{itemize}
\begin{itemize}
\item {Grp. gram.:Adj.}
\end{itemize}
\begin{itemize}
\item {Utilização:Prov.}
\end{itemize}
\begin{itemize}
\item {Utilização:beir.}
\end{itemize}
O mesmo que \textunderscore reboleira\textunderscore ^1.
Chocalho alongado, como a choca, mas mais pequeno.
Diz-se do castanheiro bravo, que é próprio para madeiras de construcção.
\section{Reboleta}
\begin{itemize}
\item {fónica:lê}
\end{itemize}
\begin{itemize}
\item {Grp. gram.:f.}
\end{itemize}
\begin{itemize}
\item {Utilização:Prov.}
\end{itemize}
\begin{itemize}
\item {Utilização:alent.}
\end{itemize}
Acto de rebolar: \textunderscore ...o jogador da péla atira com ella ás reboletas ao longo da série de covas...\textunderscore  Cf. Rev. \textunderscore Tradição\textunderscore , I, 4.
\section{Reboliço}
\begin{itemize}
\item {Grp. gram.:adj.}
\end{itemize}
\begin{itemize}
\item {Proveniência:(De \textunderscore rebolir\textunderscore )}
\end{itemize}
Que tem fórma de rebôlo; que rebola. Cf. Castilho, \textunderscore Geórgicas\textunderscore , 91.
\section{Reboliço}
\begin{itemize}
\item {Grp. gram.:f.}
\end{itemize}
Agitação, balbúrdia.
(Cp. \textunderscore boliço\textunderscore )
\section{Rebolir}
\begin{itemize}
\item {Grp. gram.:v. i.}
\end{itemize}
\begin{itemize}
\item {Utilização:bras}
\end{itemize}
\begin{itemize}
\item {Utilização:Pop.}
\end{itemize}
Rebolar-se.
Andar muito depressa; girar:«\textunderscore ...lá vai rebolindo a minha piasquinha.\textunderscore »(De uma canção popular) Cf. F. Barreto, \textunderscore Eneida\textunderscore , X, 179.
(Cp. \textunderscore rebolar\textunderscore )
\section{Rebôlo}
\begin{itemize}
\item {Grp. gram.:m.}
\end{itemize}
\begin{itemize}
\item {Utilização:Prov.}
\end{itemize}
\begin{itemize}
\item {Utilização:trasm.}
\end{itemize}
\begin{itemize}
\item {Utilização:Prov.}
\end{itemize}
\begin{itemize}
\item {Utilização:trasm.}
\end{itemize}
\begin{itemize}
\item {Utilização:Pop.}
\end{itemize}
\begin{itemize}
\item {Utilização:Prov.}
\end{itemize}
\begin{itemize}
\item {Utilização:alent.}
\end{itemize}
\begin{itemize}
\item {Grp. gram.:Adj.}
\end{itemize}
\begin{itemize}
\item {Utilização:Prov.}
\end{itemize}
\begin{itemize}
\item {Utilização:trasm.}
\end{itemize}
\begin{itemize}
\item {Proveniência:(De \textunderscore rebolar\textunderscore )}
\end{itemize}
Pequena mó que, girando sôbre um eixo fixado num banco ou mesa, serve para amolar instrumentos cortantes.
Massa de neve, de fórma espherica.
Qualquer seixo.
Cylindro.
Terreno, coberto de mato curto.
Diz-se do castanheiro bravio, rebordão.
\section{Reboludo}
\begin{itemize}
\item {Grp. gram.:adj.}
\end{itemize}
\begin{itemize}
\item {Proveniência:(De \textunderscore rebôlo\textunderscore )}
\end{itemize}
Grosso e arredondado.
\section{Rebombar}
\begin{itemize}
\item {Grp. gram.:v. i.}
\end{itemize}
\begin{itemize}
\item {Proveniência:(De \textunderscore re...\textunderscore  + \textunderscore bomba\textunderscore )}
\end{itemize}
Fazer grande estrondo, como o das bombas ou do trovão.
Resoar fortemente; retumbar.
\section{Rebombo}
\begin{itemize}
\item {Grp. gram.:m.}
\end{itemize}
Acto de rebombar.
Fragor ou estampido, mais ou menos prolongado.
\section{Reboníssimo}
\begin{itemize}
\item {Grp. gram.:adj.}
\end{itemize}
\begin{itemize}
\item {Utilização:Ant.}
\end{itemize}
\begin{itemize}
\item {Proveniência:(De \textunderscore re...\textunderscore  + \textunderscore boníssimo\textunderscore )}
\end{itemize}
Óptimo; extraordinariamente bom.
\section{Rebôo}
\begin{itemize}
\item {Grp. gram.:m.}
\end{itemize}
Acto de reboar.
\section{Reboque}
\begin{itemize}
\item {Grp. gram.:m.}
\end{itemize}
\begin{itemize}
\item {Utilização:Fig.}
\end{itemize}
\begin{itemize}
\item {Utilização:Prov.}
\end{itemize}
\begin{itemize}
\item {Utilização:alg.}
\end{itemize}
Acto ou effeito de rebocar^2.
Cabo, ou corda que de um navio se lança para outro, a fim de o levar na sua esteira.
Acto de levar alguém atrás de si, subordinando-a á sua direcção.
Petisqueira.
\section{Reboque}
\begin{itemize}
\item {Grp. gram.:m.}
\end{itemize}
\begin{itemize}
\item {Utilização:Des.}
\end{itemize}
O mesmo que \textunderscore rebôco\textunderscore . Cf. B. Pereira, \textunderscore Prosódia\textunderscore , vb. \textunderscore tectorium\textunderscore .
\section{Reboquear}
\begin{itemize}
\item {Grp. gram.:v. t.}
\end{itemize}
O mesmo que \textunderscore rebocar\textunderscore ^2.
\section{Reboqueiro}
\begin{itemize}
\item {Grp. gram.:m.}
\end{itemize}
\begin{itemize}
\item {Utilização:Prov.}
\end{itemize}
\begin{itemize}
\item {Utilização:alg.}
\end{itemize}
\begin{itemize}
\item {Proveniência:(De \textunderscore reboque\textunderscore )}
\end{itemize}
Amigo de petisqueiras.
\section{Reboquinho}
\begin{itemize}
\item {Grp. gram.:adj.}
\end{itemize}
\begin{itemize}
\item {Utilização:Prov.}
\end{itemize}
\begin{itemize}
\item {Utilização:trasm.}
\end{itemize}
Roliço; atarracado.
\section{Rebora}
\begin{itemize}
\item {Grp. gram.:f.}
\end{itemize}
\begin{itemize}
\item {Utilização:Ant.}
\end{itemize}
\begin{itemize}
\item {Utilização:Jur.}
\end{itemize}
\begin{itemize}
\item {Grp. gram.:Pl.}
\end{itemize}
\begin{itemize}
\item {Utilização:Ant.}
\end{itemize}
\begin{itemize}
\item {Proveniência:(De \textunderscore reborar\textunderscore )}
\end{itemize}
Alvedrio.
Decisão.
Idade exigida por lei para certos actos.
Idade da puberdade.
Confirmação de doações ou de contratos.
Donativos, luvas, por occasião de contratos.
\section{Reboração}
\begin{itemize}
\item {Grp. gram.:f.}
\end{itemize}
O mesmo que \textunderscore rebora\textunderscore .
Acto de reborar.
\section{Reborar}
\begin{itemize}
\item {Grp. gram.:v. t.}
\end{itemize}
\begin{itemize}
\item {Utilização:Ant.}
\end{itemize}
Confirmar, firmar de novo, (contrato, ajuste, etc.).
(Por \textunderscore roborar\textunderscore , do lat. \textunderscore roborare\textunderscore )
\section{Rebordagem}
\begin{itemize}
\item {Grp. gram.:f.}
\end{itemize}
\begin{itemize}
\item {Proveniência:(De \textunderscore re...\textunderscore  + \textunderscore borda\textunderscore )}
\end{itemize}
Prejuízo, soffrido pelos navios que abalroam.
Indemnização dêsse prejuízo.
Acto de rebordar (vidros).
\section{Rebordão}
\begin{itemize}
\item {Grp. gram.:adj.}
\end{itemize}
\begin{itemize}
\item {Grp. gram.:M.}
\end{itemize}
\begin{itemize}
\item {Utilização:Prov.}
\end{itemize}
\begin{itemize}
\item {Utilização:beir.}
\end{itemize}
\begin{itemize}
\item {Proveniência:(De \textunderscore re...\textunderscore  + \textunderscore borda\textunderscore )}
\end{itemize}
Bravio, silvestre, (falando-se de plantas, geralmente applicadas em bordar ou cercar terrenos com sebes vivas).
Castanheiro bravo. (Colhido na Guarda)
\section{Rebordar}
\begin{itemize}
\item {Grp. gram.:v. t.}
\end{itemize}
Tornar a bordar; bordar demoradamente.
Alisar as arestes ou cantos de (vidros polidos).
\section{Rebôrdo}
\begin{itemize}
\item {Grp. gram.:m.}
\end{itemize}
\begin{itemize}
\item {Proveniência:(De \textunderscore re...\textunderscore  + \textunderscore borda\textunderscore )}
\end{itemize}
Borda revirada.
\section{Rebordosa}
\begin{itemize}
\item {Grp. gram.:f.}
\end{itemize}
\begin{itemize}
\item {Utilização:Bras}
\end{itemize}
O mesmo que \textunderscore reboldrosa\textunderscore .
\section{Reborquiada}
\begin{itemize}
\item {Grp. gram.:f.}
\end{itemize}
\begin{itemize}
\item {Utilização:Bras. do S}
\end{itemize}
O mesmo que \textunderscore pialo\textunderscore .
\section{Rebotada}
\begin{itemize}
\item {Grp. gram.:f.}
\end{itemize}
Acto de rebotar^2.
Repellão:«\textunderscore a má bêsta descaiu o focinho com esta rebotada do dono\textunderscore ». Garrett, \textunderscore Arco\textunderscore , II, 120.
\section{Rebotado}
\begin{itemize}
\item {Grp. gram.:adj.}
\end{itemize}
\begin{itemize}
\item {Utilização:Prov.}
\end{itemize}
\begin{itemize}
\item {Utilização:minh.}
\end{itemize}
Estragado; corrompido.
(Cp. \textunderscore botado\textunderscore )
\section{Rebotalho}
\begin{itemize}
\item {Grp. gram.:m.}
\end{itemize}
\begin{itemize}
\item {Proveniência:(De \textunderscore re...\textunderscore  + \textunderscore botar\textunderscore )}
\end{itemize}
Resíduos inúteis.
Refugo.
Cigalho.
\section{Rebotar}
\begin{itemize}
\item {Grp. gram.:v. t.}
\end{itemize}
\begin{itemize}
\item {Proveniência:(De \textunderscore re...\textunderscore  + \textunderscore bôto\textunderscore )}
\end{itemize}
Embotar, tornar bôto.
\section{Rebotar}
\begin{itemize}
\item {Grp. gram.:v. t.}
\end{itemize}
\begin{itemize}
\item {Proveniência:(De \textunderscore re...\textunderscore  + \textunderscore botar\textunderscore )}
\end{itemize}
O mesmo que \textunderscore repellir\textunderscore . Cf. \textunderscore Viriato Trág.\textunderscore 
\section{Rebote}
\begin{itemize}
\item {Grp. gram.:m.}
\end{itemize}
\begin{itemize}
\item {Utilização:Bras}
\end{itemize}
O mesmo que \textunderscore rabote\textunderscore .
Segundo salto da péla ou pelota.
\section{Rebraço}
\begin{itemize}
\item {Grp. gram.:m.}
\end{itemize}
\begin{itemize}
\item {Utilização:Ant.}
\end{itemize}
\begin{itemize}
\item {Proveniência:(De \textunderscore re...\textunderscore  + \textunderscore braço\textunderscore )}
\end{itemize}
Parte da armadura, que defendia o braço, desde o cotovelo ao ombro.
\section{Rebramar}
\begin{itemize}
\item {Grp. gram.:v. i.}
\end{itemize}
\begin{itemize}
\item {Utilização:Fig.}
\end{itemize}
\begin{itemize}
\item {Proveniência:(De \textunderscore re...\textunderscore  + \textunderscore bramar\textunderscore )}
\end{itemize}
Bramar muito.
Rebombar.
Encolerizar-se.
\section{Rebramir}
\begin{itemize}
\item {Grp. gram.:v. i.}
\end{itemize}
\begin{itemize}
\item {Proveniência:(De \textunderscore re...\textunderscore  + \textunderscore bramir\textunderscore )}
\end{itemize}
Bramir intensamente; rebramar. Cf. Herculano, \textunderscore Eurico\textunderscore , 19.
\section{Rebranquear}
\begin{itemize}
\item {Grp. gram.:v. t.}
\end{itemize}
\begin{itemize}
\item {Proveniência:(De \textunderscore re...\textunderscore  + \textunderscore branquear\textunderscore )}
\end{itemize}
Branquear de novo; branquear muito.
\section{Rebria}
\begin{itemize}
\item {Grp. gram.:f. Loc. adv.}
\end{itemize}
\begin{itemize}
\item {Utilização:Prov.}
\end{itemize}
\textunderscore Á rebria\textunderscore , em abundância, a raivel.
(Talvez por \textunderscore raivelia\textunderscore , de \textunderscore raivel\textunderscore )
\section{Rebria}
\begin{itemize}
\item {Grp. gram.:f.}
\end{itemize}
\begin{itemize}
\item {Utilização:Prov.}
\end{itemize}
\begin{itemize}
\item {Utilização:beir.}
\end{itemize}
O mesmo que \textunderscore revelia\textunderscore .
Vida airada: \textunderscore o rapaz anda por aí á rebria\textunderscore .
\section{Rebrilhação}
\begin{itemize}
\item {Grp. gram.:f.}
\end{itemize}
\begin{itemize}
\item {Utilização:Neol.}
\end{itemize}
Acto de rebrilhar. Cf. Eça, \textunderscore P. Basílio\textunderscore , 106.
\section{Rebrilhante}
\begin{itemize}
\item {Grp. gram.:adj.}
\end{itemize}
\begin{itemize}
\item {Proveniência:(De \textunderscore rebrilhar\textunderscore )}
\end{itemize}
Que rebrilha; esplêndido. Cf. Alv. Mendes, \textunderscore Discursos\textunderscore , 268.
\section{Rebrilhar}
\begin{itemize}
\item {Grp. gram.:v. i.}
\end{itemize}
\begin{itemize}
\item {Proveniência:(De \textunderscore re...\textunderscore  + \textunderscore brilhar\textunderscore )}
\end{itemize}
Brilhar muito; resplandecer.
Brilhar novamente.
\section{Rebrilho}
\begin{itemize}
\item {Grp. gram.:m.}
\end{itemize}
Brilho intenso.
Acto de rebrilhar.
\section{Rebrotar}
\begin{itemize}
\item {Grp. gram.:v. i.}
\end{itemize}
\begin{itemize}
\item {Proveniência:(De \textunderscore re...\textunderscore  + \textunderscore brotar\textunderscore )}
\end{itemize}
Brotar de novo.
\section{Rebuçadamente}
\begin{itemize}
\item {Grp. gram.:adv.}
\end{itemize}
\begin{itemize}
\item {Proveniência:(De \textunderscore rebuçado\textunderscore )}
\end{itemize}
Com rebuço.
Dissimuladamente.
\section{Rebuçado}
\begin{itemize}
\item {Grp. gram.:adj.}
\end{itemize}
\begin{itemize}
\item {Grp. gram.:M.}
\end{itemize}
\begin{itemize}
\item {Utilização:Fig.}
\end{itemize}
Encoberto com rebuço.
Disfarçado.
Indivíduo rebuçado.
Pequena porção de açúcar em ponto, misturado com outras substâncias e solidificado.
Aquillo que é feito ou dito com grande apuro.
\section{Rebuçar}
\begin{itemize}
\item {Grp. gram.:v. t.}
\end{itemize}
\begin{itemize}
\item {Utilização:Fig.}
\end{itemize}
Encobrir com rebuço; esconder, velar.
Disfarçar, dissimular.
(Por \textunderscore reembuçar\textunderscore , de \textunderscore re...\textunderscore  + \textunderscore embuçar\textunderscore ?)
\section{Rebuchudo}
\begin{itemize}
\item {Grp. gram.:adj.}
\end{itemize}
\begin{itemize}
\item {Utilização:Ant.}
\end{itemize}
\begin{itemize}
\item {Proveniência:(De \textunderscore bucho\textunderscore ? Ou por \textunderscore rebojudo\textunderscore , de \textunderscore bôjo\textunderscore ?)}
\end{itemize}
Rechonchudo, roliço. Cf. G. Vicente.
\section{Rebucinho}
\begin{itemize}
\item {Grp. gram.:m.}
\end{itemize}
\begin{itemize}
\item {Utilização:Prov.}
\end{itemize}
\begin{itemize}
\item {Utilização:trasm.}
\end{itemize}
\begin{itemize}
\item {Proveniência:(De \textunderscore rebuço\textunderscore )}
\end{itemize}
Mantilha. (Colhido em Mirandela)
\section{Rebuço}
\begin{itemize}
\item {Grp. gram.:m.}
\end{itemize}
\begin{itemize}
\item {Utilização:Fig.}
\end{itemize}
\begin{itemize}
\item {Proveniência:(De \textunderscore rebuçar\textunderscore )}
\end{itemize}
Parte da capa ou capote, para encobrir o rosto.
Cabeção; golla; lapela.
Disfarce; falta de sinceridade ou franqueza: \textunderscore exprimir-se sem rebuço\textunderscore .
\section{Rebufar}
\begin{itemize}
\item {Grp. gram.:v. i.}
\end{itemize}
\begin{itemize}
\item {Utilização:Prov.}
\end{itemize}
\begin{itemize}
\item {Utilização:beir.}
\end{itemize}
\begin{itemize}
\item {Proveniência:(De \textunderscore re...\textunderscore  + \textunderscore bufar\textunderscore )}
\end{itemize}
Falar com aprumo insolente.
Têr modos altivos e descorteses.
\section{Rebufo}
\begin{itemize}
\item {Grp. gram.:m.}
\end{itemize}
\begin{itemize}
\item {Proveniência:(De \textunderscore rebufar\textunderscore )}
\end{itemize}
Acto ou expressão de quem rebufa.
\section{Rebulhana}
\begin{itemize}
\item {Grp. gram.:f.}
\end{itemize}
\begin{itemize}
\item {Utilização:Prov.}
\end{itemize}
\begin{itemize}
\item {Utilização:trasm.}
\end{itemize}
\begin{itemize}
\item {Proveniência:(De \textunderscore rebulhar\textunderscore )}
\end{itemize}
Jôgo de adivinhação, com castanhas assadas.
\section{Rebulhar}
\begin{itemize}
\item {Grp. gram.:v. t.}
\end{itemize}
\begin{itemize}
\item {Utilização:Prov.}
\end{itemize}
\begin{itemize}
\item {Utilização:trasm.}
\end{itemize}
\begin{itemize}
\item {Proveniência:(Do rad. de \textunderscore rebuscar\textunderscore )}
\end{itemize}
Remexer, rebuscar dentro de (algibeiras, uma casa, etc.).
\section{Rebulício}
\begin{itemize}
\item {Grp. gram.:m.}
\end{itemize}
\begin{itemize}
\item {Proveniência:(De \textunderscore re...\textunderscore  + \textunderscore bulicio\textunderscore )}
\end{itemize}
O mesmo ou melhor que \textunderscore rebuliço\textunderscore . Cf. Garrett, \textunderscore Fábulas\textunderscore , 50.
\section{Rebuliço}
\begin{itemize}
\item {Grp. gram.:m.}
\end{itemize}
Agrupamento e movimento de muita gente.
Agitação; balbúrdia; desordem.
(Cp. \textunderscore rebulício\textunderscore  e \textunderscore reboliço\textunderscore )
\section{Rebulir}
\begin{itemize}
\item {Grp. gram.:v. t.}
\end{itemize}
\begin{itemize}
\item {Utilização:Fig.}
\end{itemize}
\begin{itemize}
\item {Proveniência:(De \textunderscore re...\textunderscore  + \textunderscore bulir\textunderscore )}
\end{itemize}
Bulir novamente.
Retocar; corrigir.
\section{Rebusca}
\begin{itemize}
\item {Grp. gram.:f.}
\end{itemize}
Acto de rebuscar.
\section{Rebuscar}
\begin{itemize}
\item {Grp. gram.:v. t.}
\end{itemize}
\begin{itemize}
\item {Utilização:Fig.}
\end{itemize}
\begin{itemize}
\item {Proveniência:(De \textunderscore re...\textunderscore  + \textunderscore buscar\textunderscore )}
\end{itemize}
Buscar de novo.
Buscar ou pesquisar minuciosamente; respigar.
Ataviar com primor, com esmêro.
\section{Rebusco}
\begin{itemize}
\item {Grp. gram.:m.}
\end{itemize}
O mesmo que \textunderscore rebusca\textunderscore .
Diz-se especialmente a procura, que os rapazes ou gente do povo fazem nas vinhas, depois da vindima, por descobrir algum cacho que escapasse aos vindimadores.
Na Beira-Baixa, é a invasão que a gente do povo, passado o San-Martinho, faz nos soitos, para se apoderar das castanhas que por acaso lá tinham ficado.
\section{Rebusnante}
\begin{itemize}
\item {Grp. gram.:adj.}
\end{itemize}
Que rebusna.
Notório ou escandaloso:«\textunderscore rebusnantes aventuras.\textunderscore »Jazente, \textunderscore Poesias\textunderscore , II, 60.
\section{Rebusnar}
\begin{itemize}
\item {Grp. gram.:v. i.}
\end{itemize}
\begin{itemize}
\item {Utilização:Des.}
\end{itemize}
Zurrar, ornejar.
(Cast. \textunderscore rebuznar\textunderscore )
\section{Rebusno}
\begin{itemize}
\item {Grp. gram.:m.}
\end{itemize}
Estroinice, vida airada:«\textunderscore sempre serás no teu rebusno impuro.\textunderscore »Jazente, \textunderscore Poesias\textunderscore , II, 67.
\section{Rebusqueiro}
\begin{itemize}
\item {Grp. gram.:m.}
\end{itemize}
\begin{itemize}
\item {Utilização:Prov.}
\end{itemize}
\begin{itemize}
\item {Utilização:trasm.}
\end{itemize}
\begin{itemize}
\item {Utilização:beir.}
\end{itemize}
Aquelle que rebusca ou anda ao rebusco.
\section{Rebutalho}
\begin{itemize}
\item {Grp. gram.:m.}
\end{itemize}
(V.rebotalho). Cf. Filinto, IV, 215.
\section{Rebuznar}
\begin{itemize}
\item {Grp. gram.:v. i.}
\end{itemize}
\begin{itemize}
\item {Utilização:Des.}
\end{itemize}
Zurrar, ornejar.
(Cast. \textunderscore rebuznar\textunderscore )
\section{Reca}
\begin{itemize}
\item {Grp. gram.:f.}
\end{itemize}
\begin{itemize}
\item {Utilização:Prov.}
\end{itemize}
\begin{itemize}
\item {Utilização:minh.}
\end{itemize}
\begin{itemize}
\item {Utilização:Prov.}
\end{itemize}
\begin{itemize}
\item {Utilização:beir.}
\end{itemize}
Porco ou porca.
Porca.
\section{Recabdar}
\begin{itemize}
\item {Grp. gram.:v. t.}
\end{itemize}
\begin{itemize}
\item {Utilização:Ant.}
\end{itemize}
Receber por espôsa.
(Por \textunderscore recaptar\textunderscore , de \textunderscore re...\textunderscore  + \textunderscore captar\textunderscore )
\section{Recábdo}
\begin{itemize}
\item {Grp. gram.:m.}
\end{itemize}
Acto de recabdar.
Recebedoria; erário. Cf. Rebello, \textunderscore Ódio Velho\textunderscore , 89.
\section{Recabedar}
\textunderscore v. t.\textunderscore  (e der.)
O mesmo que \textunderscore recabdar\textunderscore , etc.
\section{Recábedo}
\begin{itemize}
\item {Grp. gram.:m.}
\end{itemize}
\begin{itemize}
\item {Proveniência:(De \textunderscore recabedar\textunderscore )}
\end{itemize}
O mesmo que \textunderscore recábdo\textunderscore .
\section{Recábito}
\begin{itemize}
\item {Grp. gram.:m.}
\end{itemize}
O mesmo que \textunderscore recábdo\textunderscore .
(Cp. \textunderscore recápito\textunderscore )
\section{Recacau}
\begin{itemize}
\item {Grp. gram.:m.}
\end{itemize}
\begin{itemize}
\item {Utilização:Bras}
\end{itemize}
Confusão, balbúrdia, desordem.
\section{Recachar}
\begin{itemize}
\item {Grp. gram.:v. t.}
\end{itemize}
\begin{itemize}
\item {Proveniência:(De \textunderscore recacho\textunderscore )}
\end{itemize}
Erguer com importância ou affectação (os ombros).
\section{Recachar}
\begin{itemize}
\item {Grp. gram.:v. i.}
\end{itemize}
\begin{itemize}
\item {Proveniência:(De \textunderscore re...\textunderscore  + \textunderscore cacha\textunderscore ^1)}
\end{itemize}
Corresponder com cilada a outra cilada.
\section{Recachiço}
\begin{itemize}
\item {Grp. gram.:m.}
\end{itemize}
\begin{itemize}
\item {Utilização:Prov.}
\end{itemize}
\begin{itemize}
\item {Utilização:trasm.}
\end{itemize}
\begin{itemize}
\item {Proveniência:(De \textunderscore re...\textunderscore  + \textunderscore cacha\textunderscore ^1)}
\end{itemize}
Cheiro desagradável, proveniente do suor dos sovacos, dos pés, ou de outras partes recatadas do corpo.
\section{Recacho}
\begin{itemize}
\item {Grp. gram.:m.}
\end{itemize}
\begin{itemize}
\item {Proveniência:(De \textunderscore re...\textunderscore  + \textunderscore cacho\textunderscore ^2)}
\end{itemize}
Postura affectada ou elegante.
Aprumo; elegância. Cf. \textunderscore Eufrosina\textunderscore , 21 e 218.
Desabrimento, desabafo. Cf. \textunderscore Aulegrafia\textunderscore , 100.
\section{Rècada}
\begin{itemize}
\item {Grp. gram.:f.}
\end{itemize}
\begin{itemize}
\item {Utilização:Prov.}
\end{itemize}
\begin{itemize}
\item {Utilização:trasm.}
\end{itemize}
Manada ou vara de recas.
\section{Recadar}
\begin{itemize}
\item {Grp. gram.:v. t.}
\end{itemize}
\begin{itemize}
\item {Utilização:Des.}
\end{itemize}
O mesmo que \textunderscore recatar\textunderscore ^2. Cf. Fern. Lopes.
\section{Recadeira}
\begin{itemize}
\item {Grp. gram.:f.}
\end{itemize}
\begin{itemize}
\item {Utilização:Prov.}
\end{itemize}
\begin{itemize}
\item {Utilização:Prov.}
\end{itemize}
\begin{itemize}
\item {Utilização:minh.}
\end{itemize}
\begin{itemize}
\item {Utilização:trasm.}
\end{itemize}
\begin{itemize}
\item {Proveniência:(De \textunderscore recado\textunderscore )}
\end{itemize}
Mulhér, que vai a recados ou que vai fazer compras de pequena importância, por conta alheia.
Descompostura.
\section{Recadém}
\begin{itemize}
\item {Grp. gram.:m.}
\end{itemize}
\begin{itemize}
\item {Utilização:Prov.}
\end{itemize}
Última travessa de madeira, que une as chedas, na extremidade do carro.
O mesmo que \textunderscore recavém\textunderscore . (Colhido na Guarda)
\section{Recadista}
\begin{itemize}
\item {Grp. gram.:m.  e  f.}
\end{itemize}
\begin{itemize}
\item {Proveniência:(De \textunderscore recado\textunderscore )}
\end{itemize}
Pessôa, que transmitte recados.
\section{Recado}
\begin{itemize}
\item {Grp. gram.:m.}
\end{itemize}
\begin{itemize}
\item {Utilização:Fam.}
\end{itemize}
\begin{itemize}
\item {Grp. gram.:Loc.}
\end{itemize}
\begin{itemize}
\item {Utilização:fam.}
\end{itemize}
\begin{itemize}
\item {Grp. gram.:Pl.}
\end{itemize}
Participação ou aviso, geralmente verbal.
Mensagem.
Reprehensão.
\textunderscore Tomar o recado na escada\textunderscore , dar a resposta antes de ouvir toda a pergunta; falar, antes de ouvir o que se lhe quere dizer.
Cumprimentos: \textunderscore dê-lhe os meus recados\textunderscore .
Dilingências ou pequenos serviços, incumbidos a um criado ou outro serviçal, por fóra de casa.
(Talvez do lat. \textunderscore recaptus\textunderscore )
\section{Recado}
\begin{itemize}
\item {Grp. gram.:m.}
\end{itemize}
(V.recato)
\section{Reçaga}
\begin{itemize}
\item {Grp. gram.:f.}
\end{itemize}
\begin{itemize}
\item {Utilização:Des.}
\end{itemize}
O mesmo que \textunderscore rètaguarda\textunderscore .
\section{Recaída}
\begin{itemize}
\item {Grp. gram.:f.}
\end{itemize}
Acto ou effeito de recaír.
\section{Recaidiço}
\begin{itemize}
\item {fónica:ca-i}
\end{itemize}
\begin{itemize}
\item {Grp. gram.:adj.}
\end{itemize}
\begin{itemize}
\item {Proveniência:(De \textunderscore recaír\textunderscore )}
\end{itemize}
Que recai com facilidade.
\section{Recaimento}
\begin{itemize}
\item {fónica:ca-i}
\end{itemize}
\begin{itemize}
\item {Grp. gram.:m.}
\end{itemize}
O mesmo que \textunderscore recaída\textunderscore .
\section{Recair}
\begin{itemize}
\item {Grp. gram.:v. i.}
\end{itemize}
\begin{itemize}
\item {Proveniência:(De \textunderscore re...\textunderscore  + \textunderscore caír\textunderscore )}
\end{itemize}
Cair novamente; cair muitas vezes.
Voltar a um estado anterior, que se havia deixado.
Tornar a adoecer, da mesma moléstia.
Reincidir; incidir.
Têr por objecto algum assumpto, dizer respeito: \textunderscore a conversa recaiu no tal duello\textunderscore .
\section{Recalcadamente}
\begin{itemize}
\item {Grp. gram.:adv.}
\end{itemize}
De modo recalcado.
Repetidamente.
\section{Recalcado}
\begin{itemize}
\item {Grp. gram.:adj.}
\end{itemize}
\begin{itemize}
\item {Proveniência:(De \textunderscore recalcar\textunderscore )}
\end{itemize}
Calcado muitas vezes.
Repisado; repetido.
\section{Recalcador}
\begin{itemize}
\item {Grp. gram.:m.  e  adj.}
\end{itemize}
O que recalca.
Instrumento, com que se recalca a balça.
\section{Recalcadura}
\begin{itemize}
\item {Grp. gram.:f.}
\end{itemize}
Acto ou effeito de recalcar.
\section{Recalcamento}
\begin{itemize}
\item {Grp. gram.:m.}
\end{itemize}
Acto ou effeito de recalcar.
\section{Recalcar}
\begin{itemize}
\item {Grp. gram.:v. t.}
\end{itemize}
\begin{itemize}
\item {Proveniência:(Do lat. \textunderscore recalcare\textunderscore )}
\end{itemize}
Calcar de novo; calcar muitas vezes.
Repisar.
Reprimir, comprimir, concentrar: \textunderscore recalcar pesares\textunderscore .
\section{Recalcitração}
\begin{itemize}
\item {Grp. gram.:f.}
\end{itemize}
Acto ou effeito de recalcitrar.
\section{Recalcitrância}
\begin{itemize}
\item {Grp. gram.:f.}
\end{itemize}
\begin{itemize}
\item {Utilização:bras}
\end{itemize}
\begin{itemize}
\item {Utilização:Neol.}
\end{itemize}
Qualidade de recalcitrante; recalcitração.
\section{Recalcitrante}
\begin{itemize}
\item {Grp. gram.:adj.}
\end{itemize}
\begin{itemize}
\item {Proveniência:(Lat. \textunderscore recalcitrans\textunderscore )}
\end{itemize}
Que recalcitra.
\section{Recalcitrar}
\begin{itemize}
\item {Grp. gram.:v. i.}
\end{itemize}
\begin{itemize}
\item {Proveniência:(Lat. \textunderscore recalcitrare\textunderscore )}
\end{itemize}
Dar coices.
Replicar; respingar.
Desobedecer.
Insurgir-se.
Obstinar-se.
\section{Recalcular}
\begin{itemize}
\item {Grp. gram.:v. t.}
\end{itemize}
\begin{itemize}
\item {Proveniência:(De \textunderscore re...\textunderscore  + \textunderscore calcular\textunderscore )}
\end{itemize}
Calcular novamente; calcular com cuidado. Cf. Serpa Pinto, II, 316.
\section{Recaldear}
\begin{itemize}
\item {Grp. gram.:v. t.}
\end{itemize}
\begin{itemize}
\item {Proveniência:(De \textunderscore re...\textunderscore  + \textunderscore caldear\textunderscore )}
\end{itemize}
Caldear bem; caldear de novo. Cf. Camillo, \textunderscore Cancion Al.\textunderscore , 363.
\section{Rècalha}
\begin{itemize}
\item {Grp. gram.:f.}
\end{itemize}
\begin{itemize}
\item {Utilização:Prov.}
\end{itemize}
\begin{itemize}
\item {Utilização:trasm.}
\end{itemize}
Rapariga immunda.
(Cp. \textunderscore reca\textunderscore )
\section{Recalmão}
\begin{itemize}
\item {Grp. gram.:m.}
\end{itemize}
\begin{itemize}
\item {Utilização:Marit.}
\end{itemize}
\begin{itemize}
\item {Proveniência:(De \textunderscore re...\textunderscore  + \textunderscore calmão\textunderscore )}
\end{itemize}
Intervallo sereno, nas grandes ventanias ou temporaes do mar. Cf. B. Pato, \textunderscore Livro do Monte\textunderscore , 258.
\section{Recalque}
\begin{itemize}
\item {Grp. gram.:m.}
\end{itemize}
O mesmo que \textunderscore recalcamento\textunderscore .
\section{Recamador}
\begin{itemize}
\item {Grp. gram.:m.}
\end{itemize}
Aquelle que recama; bordador.
\section{Recamadura}
\begin{itemize}
\item {Grp. gram.:f.}
\end{itemize}
O mesmo que \textunderscore recamo\textunderscore .
\section{Recamar}
\begin{itemize}
\item {Grp. gram.:v. t.}
\end{itemize}
Fazer recamo a.
Enfeitar, ornar.
Revestir.
\section{Recâmara}
\begin{itemize}
\item {Grp. gram.:f.}
\end{itemize}
\begin{itemize}
\item {Proveniência:(De \textunderscore re...\textunderscore  + \textunderscore câmara\textunderscore )}
\end{itemize}
Câmara interior.
Alfaias de serviço doméstico.
Recanto.
\section{Recambejo}
\begin{itemize}
\item {Grp. gram.:m.}
\end{itemize}
\begin{itemize}
\item {Utilização:Prov.}
\end{itemize}
\begin{itemize}
\item {Utilização:trasm.}
\end{itemize}
Caminho em ziguezague.
(Relaciona-se com \textunderscore cambiar\textunderscore ?)
\section{Recambiar}
\begin{itemize}
\item {Grp. gram.:v. t.}
\end{itemize}
\begin{itemize}
\item {Proveniência:(De \textunderscore re...\textunderscore  + \textunderscore cambiar\textunderscore )}
\end{itemize}
Devolver.
Devolver (uma letra) não paga ou não acceita.
\section{Recâmbio}
\begin{itemize}
\item {Grp. gram.:m.}
\end{itemize}
\begin{itemize}
\item {Proveniência:(De \textunderscore re...\textunderscore  + \textunderscore câmbio\textunderscore )}
\end{itemize}
Acto ou effeito de recambiar.
Despesa com o recâmbio de uma letra.
\section{Recambó}
\begin{itemize}
\item {Grp. gram.:m.}
\end{itemize}
\begin{itemize}
\item {Proveniência:(De \textunderscore recâmbio\textunderscore ?)}
\end{itemize}
Tempo que dura um jôgo de casa, ate se preencher determinado número de mãos ou partidas.
Mudança de lugares ou de parceiros, no fim dessas partidas.
Prato, ou lugar, em que se ajuntam êsses tentos.
\section{Recamo}
\begin{itemize}
\item {Grp. gram.:m.}
\end{itemize}
\begin{itemize}
\item {Utilização:Fig.}
\end{itemize}
\begin{itemize}
\item {Proveniência:(Do ár. \textunderscore rakm\textunderscore )}
\end{itemize}
Ornato a relêvo.
Ornato.
\section{Recantação}
\begin{itemize}
\item {Grp. gram.:f.}
\end{itemize}
Acto ou effeito de recantar.
\section{Recantar}
\begin{itemize}
\item {Grp. gram.:v. t.}
\end{itemize}
\begin{itemize}
\item {Proveniência:(Lat. \textunderscore recantare\textunderscore )}
\end{itemize}
Cantar novamente.
Cantar com affectação.
\section{Recanto}
\begin{itemize}
\item {Grp. gram.:m.}
\end{itemize}
\begin{itemize}
\item {Proveniência:(De \textunderscore re...\textunderscore  + \textunderscore canto\textunderscore )}
\end{itemize}
Canto esconso.
Lugar retirado ou occulto.
Esconderijo; escaninho.
\section{Recapacitar}
\begin{itemize}
\item {Grp. gram.:v. t.}
\end{itemize}
Introduzir, insinuar?:«\textunderscore ...recapacitar na memoria ao menos hum ponto do que se leo\textunderscore ». \textunderscore Luz e Calor\textunderscore , 260.
\section{Recápito}
\begin{itemize}
\item {Grp. gram.:m.}
\end{itemize}
\begin{itemize}
\item {Utilização:Ant.}
\end{itemize}
Recado que se manda por alguém.
(Parece que deve lêr-se \textunderscore recápito\textunderscore , especialmente se \textunderscore recado\textunderscore ^1 vem do lat. \textunderscore recaptus\textunderscore )
\section{Recapitulação}
\begin{itemize}
\item {Grp. gram.:f.}
\end{itemize}
\begin{itemize}
\item {Proveniência:(Lat. \textunderscore recapitulatio\textunderscore )}
\end{itemize}
Acto ou effeito de recapitular.
Resumo; sýnthese.
\section{Recapitular}
\begin{itemize}
\item {Grp. gram.:v. t.}
\end{itemize}
\begin{itemize}
\item {Proveniência:(Lat. \textunderscore recapitulare\textunderscore )}
\end{itemize}
Repetir summariamente; resumir; synthetizar.
\section{Recapturar}
\begin{itemize}
\item {Grp. gram.:v. t.}
\end{itemize}
\begin{itemize}
\item {Proveniência:(De \textunderscore re...\textunderscore  + \textunderscore capturar\textunderscore )}
\end{itemize}
Capturar novamente.
\section{Rècar}
\begin{itemize}
\item {Grp. gram.:v. t.}
\end{itemize}
\begin{itemize}
\item {Utilização:Prov.}
\end{itemize}
\begin{itemize}
\item {Utilização:minh.}
\end{itemize}
\begin{itemize}
\item {Proveniência:(De \textunderscore reco\textunderscore )}
\end{itemize}
Ralhar; resmungar.
\section{Recardar}
\begin{itemize}
\item {Grp. gram.:v. t.}
\end{itemize}
\begin{itemize}
\item {Proveniência:(De \textunderscore re...\textunderscore  + \textunderscore cardar\textunderscore )}
\end{itemize}
Cardar de novo; cardar muitas vezes. Cf. \textunderscore Inquér. Industr.\textunderscore , p. II, l. III, 67.
\section{Recarga}
\begin{itemize}
\item {Grp. gram.:f.}
\end{itemize}
\begin{itemize}
\item {Proveniência:(De \textunderscore re...\textunderscore  + \textunderscore carga\textunderscore )}
\end{itemize}
Diz-se de recarga o boi, que carrega ou investe contra o toireiro que o feriu.
\section{Recargar}
\begin{itemize}
\item {Grp. gram.:v. t.}
\end{itemize}
\begin{itemize}
\item {Proveniência:(De \textunderscore recarga\textunderscore )}
\end{itemize}
Suster com a vara o ímpeto de (um toiro).
\section{Recasar}
\begin{itemize}
\item {Grp. gram.:v. t.  e  i.}
\end{itemize}
\begin{itemize}
\item {Proveniência:(De \textunderscore re...\textunderscore  + \textunderscore casar\textunderscore )}
\end{itemize}
Tornar a casar. Cf. Castilho, \textunderscore Fastos\textunderscore , I, 284.
\section{Recata}
\begin{itemize}
\item {Grp. gram.:f.}
\end{itemize}
\begin{itemize}
\item {Proveniência:(De \textunderscore recatar\textunderscore ^1)}
\end{itemize}
(V.rebusca)
\section{Recatadamente}
\begin{itemize}
\item {Grp. gram.:adv.}
\end{itemize}
De modo recatado; com recato; honestamente: \textunderscore viver recatadamente\textunderscore .
\section{Recatado}
\begin{itemize}
\item {Grp. gram.:adj.}
\end{itemize}
Que tem modéstia, que é prudente ou sensato.
Pudico.
\section{Recatar}
\begin{itemize}
\item {Grp. gram.:v. t.}
\end{itemize}
\begin{itemize}
\item {Proveniência:(Do lat. \textunderscore re...\textunderscore  + \textunderscore captare\textunderscore )}
\end{itemize}
O mesmo que \textunderscore rebuscar\textunderscore .
\section{Recatar}
\begin{itemize}
\item {Grp. gram.:v. t.}
\end{itemize}
Pôr em recato; resguardar; acautelar.
Têr em segrêdo.
Esconder.
\section{Recativo}
\begin{itemize}
\item {Grp. gram.:m.  e  adj.}
\end{itemize}
Aquelle que está muito cativo; moralmente sujeito ou subjugado.
\section{Recato}
\begin{itemize}
\item {Grp. gram.:m.}
\end{itemize}
\begin{itemize}
\item {Proveniência:(Do lat. \textunderscore recautus\textunderscore )}
\end{itemize}
Cautela.
Resguardo.
Prudência.
Honestidade.
Lugar occulto; segredo.
\section{Recaudo}
\begin{itemize}
\item {Grp. gram.:adj.}
\end{itemize}
\begin{itemize}
\item {Utilização:Ant.}
\end{itemize}
\begin{itemize}
\item {Proveniência:(Do lat. \textunderscore recautus\textunderscore )}
\end{itemize}
Recatado; acautelado.
\section{Recautelado}
\begin{itemize}
\item {Grp. gram.:adj.}
\end{itemize}
\begin{itemize}
\item {Utilização:Des.}
\end{itemize}
Muito acautelado, muito precavido.
(Cp. lat. \textunderscore recautus\textunderscore )
\section{Recavado}
\begin{itemize}
\item {Grp. gram.:adj.}
\end{itemize}
\begin{itemize}
\item {Proveniência:(De \textunderscore recavar\textunderscore )}
\end{itemize}
Que é muito cavado ou que tem grande cavidade.
\section{Recavar}
\begin{itemize}
\item {Grp. gram.:v. t.}
\end{itemize}
\begin{itemize}
\item {Utilização:Fig.}
\end{itemize}
\begin{itemize}
\item {Proveniência:(De \textunderscore re...\textunderscore  + \textunderscore cavar\textunderscore )}
\end{itemize}
Cavar de novo.
Cavar muitas vezes.
Insistir em.
\section{Recavém}
\begin{itemize}
\item {Grp. gram.:m.}
\end{itemize}
Parte posterior do leito do carro.
\section{Recavo}
\begin{itemize}
\item {Grp. gram.:adj.}
\end{itemize}
\begin{itemize}
\item {Proveniência:(De \textunderscore re...\textunderscore  + \textunderscore cavo\textunderscore )}
\end{itemize}
O mesmo que \textunderscore recavado\textunderscore .
Que tem grande cavidade; muito cavo. Cf. Castilho. \textunderscore Geórgicas\textunderscore , 211.
\section{Receança}
\begin{itemize}
\item {Grp. gram.:f.}
\end{itemize}
\begin{itemize}
\item {Utilização:Ant.}
\end{itemize}
\begin{itemize}
\item {Proveniência:(De \textunderscore recear\textunderscore )}
\end{itemize}
O mesmo que \textunderscore receio\textunderscore .
\section{Recear}
\begin{itemize}
\item {Grp. gram.:v. t.}
\end{itemize}
\begin{itemize}
\item {Grp. gram.:V. i.}
\end{itemize}
\begin{itemize}
\item {Proveniência:(De \textunderscore re...\textunderscore  + \textunderscore zelar\textunderscore )}
\end{itemize}
Têr receio de.
Temer.
Têr receio.
\section{Recêbedo}
\begin{itemize}
\item {Grp. gram.:m.}
\end{itemize}
\begin{itemize}
\item {Utilização:Ant.}
\end{itemize}
\begin{itemize}
\item {Proveniência:(Do lat. \textunderscore receptus\textunderscore )}
\end{itemize}
O mesmo que \textunderscore recibo\textunderscore . Cf. S. R. Viterbo, \textunderscore Elucidário\textunderscore .
\section{Recebedoiro}
\begin{itemize}
\item {Grp. gram.:adj.}
\end{itemize}
\begin{itemize}
\item {Utilização:Ant.}
\end{itemize}
O mesmo que \textunderscore recebondo\textunderscore . Cf. João Ribeiro, \textunderscore Gram.\textunderscore 
\section{Recebedouro}
\begin{itemize}
\item {Grp. gram.:adj.}
\end{itemize}
\begin{itemize}
\item {Utilização:Ant.}
\end{itemize}
O mesmo que \textunderscore recebondo\textunderscore . Cf. João Ribeiro, \textunderscore Gram.\textunderscore 
\section{Recebedor}
\begin{itemize}
\item {Grp. gram.:adj.}
\end{itemize}
\begin{itemize}
\item {Grp. gram.:M.}
\end{itemize}
Que recebe.
Aquelle que recebe.
Funccionário, incumbido da arrecadação de impostos.
\section{Recebedoria}
\begin{itemize}
\item {Grp. gram.:f.}
\end{itemize}
\begin{itemize}
\item {Proveniência:(De \textunderscore recebedor\textunderscore )}
\end{itemize}
Repartição, onde se recebem impostos.
Cargo de recebedor.
\section{Receber}
\begin{itemize}
\item {Grp. gram.:v. t.}
\end{itemize}
\begin{itemize}
\item {Proveniência:(Do lat. \textunderscore recipere\textunderscore )}
\end{itemize}
Acceitar (o que se dá gratuitamente).
Acceitar em pagamento.
Adquirir: \textunderscore receber um prémio\textunderscore .
Admittir: \textunderscore receber visitas\textunderscore .
Soffrer: \textunderscore receber contrariedades\textunderscore .
Acolher.
Recolher.
Têr communicação de: \textunderscore receber notícias\textunderscore .
Casar com; tomar por espôso ou espôsa.
Aparar; apanhar: \textunderscore receber uma pancada\textunderscore .
Têr como resultado.
\section{Recebimento}
\begin{itemize}
\item {Grp. gram.:m.}
\end{itemize}
\begin{itemize}
\item {Utilização:Ant.}
\end{itemize}
\begin{itemize}
\item {Utilização:Ant.}
\end{itemize}
Acto ou effeito de receber.
Aposento, quarto, sala.
O mesmo que \textunderscore recebedoria\textunderscore :«\textunderscore ...mil reis em dinheíro, pagos no recebimento da fabrica...\textunderscore »\textunderscore Doc. do séc XVII\textunderscore , no \textunderscore Instituto\textunderscore , LIX, 171.
\section{Recebondo}
\begin{itemize}
\item {Grp. gram.:adj.}
\end{itemize}
\begin{itemize}
\item {Utilização:Ant.}
\end{itemize}
\begin{itemize}
\item {Proveniência:(Do lat. \textunderscore recipiendus\textunderscore )}
\end{itemize}
Que está no caso de sêr recebido, como paga, ou como obrigação de quem dá.
Acceitável. Cf. B. Pato, \textunderscore Ciprestes\textunderscore , 25.
\section{Receio}
\begin{itemize}
\item {Grp. gram.:m.}
\end{itemize}
\begin{itemize}
\item {Proveniência:(De \textunderscore recear\textunderscore )}
\end{itemize}
Hesitação ou incerteza, acompanhada de temor; temor.
\section{Receita}
\begin{itemize}
\item {Grp. gram.:f.}
\end{itemize}
\begin{itemize}
\item {Utilização:Fig.}
\end{itemize}
\begin{itemize}
\item {Proveniência:(Do lat. \textunderscore recepta\textunderscore )}
\end{itemize}
Aquillo que se recebe.
Rendimento: \textunderscore orçamento das receitas e despesas\textunderscore .
Quantia recebida.
Cálculo do que se há de receber.
Fórmula de medicamento: \textunderscore o médico faz receitas\textunderscore .
Fórmula, para a preparação de uma substância, em que entram vários ingredientes: \textunderscore receita para se fazer manjar-branco\textunderscore .
Conselho.
\section{Receitante}
\begin{itemize}
\item {Grp. gram.:adj.}
\end{itemize}
Que receita. Cf. A. Dinis, \textunderscore Hyssope\textunderscore , 87.
\section{Receitar}
\begin{itemize}
\item {Grp. gram.:v. t.}
\end{itemize}
\begin{itemize}
\item {Utilização:Fig.}
\end{itemize}
\begin{itemize}
\item {Grp. gram.:V. i.}
\end{itemize}
\begin{itemize}
\item {Proveniência:(Do lat. \textunderscore receptare\textunderscore )}
\end{itemize}
Ordenar, fazendo receita; prescrever: \textunderscore receitar um xarope\textunderscore .
Aconselhar.
Formular receita.
\section{Receitário}
\begin{itemize}
\item {Grp. gram.:m.}
\end{itemize}
Lugar, onde se guardam receitas.
\section{Receituário}
\begin{itemize}
\item {Grp. gram.:m.}
\end{itemize}
Conjunto de receitas.
Formulário para medicamentos.
\section{Recém...}
\begin{itemize}
\item {Grp. gram.:pref.}
\end{itemize}
\begin{itemize}
\item {Proveniência:(Lat. \textunderscore recens\textunderscore )}
\end{itemize}
(designativo de \textunderscore há pouco\textunderscore )
\section{Recém-casado}
\begin{itemize}
\item {Grp. gram.:m.  e  adj.}
\end{itemize}
O que é casado há pouco tempo.
\section{Recém-chegado}
\begin{itemize}
\item {Grp. gram.:m.  e  adj.}
\end{itemize}
O que chegou há pouco tempo.
\section{Recém-convertido}
\begin{itemize}
\item {Grp. gram.:m.  e  adj.}
\end{itemize}
O que se converteu há pouco tempo.
\section{Recém-emancipado}
\begin{itemize}
\item {Grp. gram.:adj.}
\end{itemize}
Emancipado há pouco, emancipado pouco antes. Cf. Latino, \textunderscore Elog.\textunderscore , 118.
\section{Recém-fallecido}
\begin{itemize}
\item {Grp. gram.:adj.}
\end{itemize}
Fallecido recentemente ou pouco antes.
\section{Recém-feito}
\begin{itemize}
\item {Grp. gram.:adj.}
\end{itemize}
Feito recentemente. Cf. Garrett, \textunderscore Helena\textunderscore , 50.
\section{Recém-ferir}
\begin{itemize}
\item {Grp. gram.:v. t.}
\end{itemize}
Têr ferido há pouco, recentemente. Cf. Castilho, \textunderscore Metam.\textunderscore  131.
\section{Recém-finado}
\begin{itemize}
\item {Grp. gram.:adj.}
\end{itemize}
O mesmo que \textunderscore recém-fallecido\textunderscore .
\section{Recém-findo}
\begin{itemize}
\item {Grp. gram.:adj.}
\end{itemize}
Que findou há pouco ou pouco antes. Cf. Castilho, \textunderscore Camões\textunderscore , II, 118.
\section{Recém-geado}
\begin{itemize}
\item {Grp. gram.:adj.}
\end{itemize}
Sôbre que caiu geada há pouco. Cf. Garrett, \textunderscore Flores sem Fruto\textunderscore , 64.
\section{Recém-morto}
\begin{itemize}
\item {Grp. gram.:adj.}
\end{itemize}
O mesmo que \textunderscore recém-fallecido\textunderscore . Cf. Castilho, \textunderscore Metam.\textunderscore , 196.
\section{Recém-nado}
\begin{itemize}
\item {Grp. gram.:m.  e  adj.}
\end{itemize}
\begin{itemize}
\item {Utilização:Poét.}
\end{itemize}
O que nasceu há pouco tempo.
\section{Recém-nascido}
\begin{itemize}
\item {Grp. gram.:m.  e  adj.}
\end{itemize}
O que nasceu há pouco tempo.
\section{Recém-nobre}
\begin{itemize}
\item {Grp. gram.:adj.}
\end{itemize}
Que pertence á nobreza moderna. Cf. Garrett, \textunderscore Fábulas\textunderscore , 45.
\section{Recém-plantado}
\begin{itemize}
\item {Grp. gram.:adj.}
\end{itemize}
\begin{itemize}
\item {Utilização:Fig.}
\end{itemize}
Plantado recentemente.
Transplantado ou tranferido recentemente:«\textunderscore ...a congragação recém-plantada para êste reito.\textunderscore »Castilho, \textunderscore Livrar. Cláss.\textunderscore , VII, 77.
\section{Recém-saído}
\begin{itemize}
\item {Grp. gram.:adj.}
\end{itemize}
Que saiu há pouco; recentemente saído. Cf. Castilho, \textunderscore Pal. de um Crente\textunderscore , 88.
\section{Recém-vindo}
\begin{itemize}
\item {Grp. gram.:m.  e  adj.}
\end{itemize}
O mesmo que \textunderscore recém-chegado\textunderscore .
\section{Recenar}
\begin{itemize}
\item {Grp. gram.:v. t.}
\end{itemize}
\begin{itemize}
\item {Proveniência:(It. \textunderscore recennare\textunderscore )}
\end{itemize}
Doirar ou pratear novamente.
\section{Recendência}
\begin{itemize}
\item {Grp. gram.:f.}
\end{itemize}
Qualidade de recendente.
\section{Recendente}
\begin{itemize}
\item {Grp. gram.:adj.}
\end{itemize}
Que recende.
\section{Recender}
\begin{itemize}
\item {Grp. gram.:v. t.}
\end{itemize}
\begin{itemize}
\item {Grp. gram.:V. i.}
\end{itemize}
Emittir (um aroma penetrante).
Têr cheiro agradável e intenso.
(Do \textunderscore re...\textunderscore  + ing. \textunderscore scent\textunderscore , cheiro, segundo outros diccionários. Duvido desta etymologia)
\section{Recennar}
\begin{itemize}
\item {Grp. gram.:v. t.}
\end{itemize}
\begin{itemize}
\item {Proveniência:(It. \textunderscore recennare\textunderscore )}
\end{itemize}
Doirar ou pratear novamente.
\section{Recensão}
\begin{itemize}
\item {Grp. gram.:f.}
\end{itemize}
\begin{itemize}
\item {Utilização:Des.}
\end{itemize}
\begin{itemize}
\item {Proveniência:(Lat. \textunderscore recensio\textunderscore )}
\end{itemize}
O mesmo que \textunderscore recenseamento\textunderscore . Cf. Latino, \textunderscore Or. da Cor.\textunderscore , CCV.
\section{Recenseado}
\begin{itemize}
\item {Grp. gram.:adj.}
\end{itemize}
\begin{itemize}
\item {Proveniência:(De \textunderscore recensear\textunderscore )}
\end{itemize}
Incluido numa relação ou lista.
Aquelle que foi recenseado, especialmente como eleitor ou elegível.
\section{Recenseador}
\begin{itemize}
\item {Grp. gram.:m.  e  adj.}
\end{itemize}
O mesmo que recenseia.
\section{Recenseamento}
\begin{itemize}
\item {Grp. gram.:m.}
\end{itemize}
\begin{itemize}
\item {Proveniência:(De \textunderscore recensear\textunderscore )}
\end{itemize}
Arrolamento ou inscripção de pessôas ou animaes.
Relação dos individuos, que estão em circunstâncias de desempenhar certos cargos, ou de exercer certos direitos: \textunderscore recenseamento eleitoral\textunderscore .
\section{Recensear}
\begin{itemize}
\item {Grp. gram.:v. t.}
\end{itemize}
\begin{itemize}
\item {Proveniência:(Do lat. \textunderscore recensere\textunderscore )}
\end{itemize}
Fazer o recenseamento de.
Enumerar.
Apreciar.
\section{Recenseio}
\begin{itemize}
\item {Grp. gram.:m.}
\end{itemize}
Acto ou effeito de recensear.
\section{Recental}
\begin{itemize}
\item {Grp. gram.:m.  e  adj.}
\end{itemize}
\begin{itemize}
\item {Proveniência:(De \textunderscore recente\textunderscore )}
\end{itemize}
Cordeiro de poucos mêses.
\section{Recente}
\begin{itemize}
\item {Grp. gram.:adj.}
\end{itemize}
\begin{itemize}
\item {Proveniência:(Lat. \textunderscore recens\textunderscore )}
\end{itemize}
Que succedeu há pouco tempo; que tem pouco tempo de existência.
\section{Recente-alvo}
\begin{itemize}
\item {Grp. gram.:adj.}
\end{itemize}
Lavado ou purificado há pouco tempo.
\section{Recentemente}
\begin{itemize}
\item {Grp. gram.:adv.}
\end{itemize}
\begin{itemize}
\item {Proveniência:(De \textunderscore recente\textunderscore )}
\end{itemize}
Há pouco tempo; proximamente ao tempo de agora.
\section{Recêo}
\begin{itemize}
\item {Grp. gram.:m.}
\end{itemize}
(Fórma \textunderscore ant.\textunderscore  de \textunderscore receio\textunderscore ) Cf. Usque, D. Bernárdez, etc.
\section{Receosamente}
\begin{itemize}
\item {Grp. gram.:adv.}
\end{itemize}
De modo receoso; com receio.
\section{Receoso}
\begin{itemize}
\item {Grp. gram.:adj.}
\end{itemize}
\begin{itemize}
\item {Utilização:Ant.}
\end{itemize}
Que tem receio; que revela receio; tímido; acanhado.
Que causa receio ou que é temível.
\section{Recepção}
\begin{itemize}
\item {Grp. gram.:f.}
\end{itemize}
\begin{itemize}
\item {Proveniência:(Do lat. \textunderscore receptio\textunderscore )}
\end{itemize}
Acto ou effeito de receber.
Acto de receber em certos dias visitas ou cumprimentos.
\section{Recepta}
\begin{itemize}
\item {Grp. gram.:f.}
\end{itemize}
\begin{itemize}
\item {Utilização:Ant.}
\end{itemize}
O mesmo que \textunderscore receita\textunderscore .
\section{Receptação}
\begin{itemize}
\item {Grp. gram.:f.}
\end{itemize}
\begin{itemize}
\item {Proveniência:(Lat. \textunderscore receptatio\textunderscore )}
\end{itemize}
Acto ou effeito de receptar.
\section{Receptacular}
\begin{itemize}
\item {Grp. gram.:adj.}
\end{itemize}
\begin{itemize}
\item {Utilização:Bot.}
\end{itemize}
\begin{itemize}
\item {Utilização:Bot.}
\end{itemize}
Relativo a receptáculo.
Que está sôbre um receptáculo, nas plantas.
\section{Receptáculo}
\begin{itemize}
\item {Grp. gram.:m.}
\end{itemize}
\begin{itemize}
\item {Utilização:Bot.}
\end{itemize}
\begin{itemize}
\item {Proveniência:(Lat. \textunderscore receptaculum\textunderscore )}
\end{itemize}
Lugar, onde se junta ou se guarda alguma coisa.
Abrigo.
Esconderijo.
Tanque, que recebe águas de differentes pontos.
Parte superior do pedúnculo das plantas, no qual se agrupam flôres.
\section{Receptador}
\begin{itemize}
\item {Grp. gram.:m.  e  adj.}
\end{itemize}
\begin{itemize}
\item {Proveniência:(Lat. \textunderscore receptator\textunderscore )}
\end{itemize}
O que recepta.
\section{Receptar}
\begin{itemize}
\item {Grp. gram.:v. t.}
\end{itemize}
\begin{itemize}
\item {Proveniência:(Lat. \textunderscore receptare\textunderscore )}
\end{itemize}
Dar receptáculo a.
Recolher ou esconder (coisas furtadas por outrem).
\section{Receptibilidade}
\begin{itemize}
\item {Grp. gram.:f.}
\end{itemize}
\begin{itemize}
\item {Proveniência:(Do lat. \textunderscore receptibilis\textunderscore )}
\end{itemize}
Estado do que facilmente recebe impressões ou influência de certos agentes deletérios ou therapêuticos.
\section{Receptível}
\begin{itemize}
\item {Grp. gram.:adj.}
\end{itemize}
\begin{itemize}
\item {Proveniência:(Lat. \textunderscore receptibilis\textunderscore )}
\end{itemize}
Que se póde receber; acceitável.
\section{Receptividade}
\begin{itemize}
\item {Grp. gram.:f.}
\end{itemize}
O mesmo que \textunderscore receptibilidade\textunderscore . Cf. R. de Brito, \textunderscore Philos. do Dir.\textunderscore , 10 e 122.
\section{Receptivo}
\begin{itemize}
\item {Grp. gram.:adj.}
\end{itemize}
\begin{itemize}
\item {Proveniência:(Do lat. \textunderscore receptus\textunderscore )}
\end{itemize}
Que recebe ou póde receber; impressionável.
\section{Receptor}
\begin{itemize}
\item {Grp. gram.:adj.}
\end{itemize}
\begin{itemize}
\item {Grp. gram.:M.}
\end{itemize}
\begin{itemize}
\item {Proveniência:(Lat. \textunderscore receptor\textunderscore )}
\end{itemize}
Que recebe.
O mesmo que \textunderscore recebedor\textunderscore .
Receptáculo.
Apparelho telegráphico, que recebe o boletim transmittido pelo manipulador.
O mesmo que \textunderscore receptador\textunderscore .
\section{Recesso}
\begin{itemize}
\item {Grp. gram.:m.}
\end{itemize}
\begin{itemize}
\item {Proveniência:(Lat. \textunderscore recessus\textunderscore )}
\end{itemize}
Recanto; esconso; retiro.
\section{Rechã}
\begin{itemize}
\item {Grp. gram.:f.}
\end{itemize}
\begin{itemize}
\item {Proveniência:(De \textunderscore re...\textunderscore  + \textunderscore chan\textunderscore )}
\end{itemize}
Planalto.
Chapada. Cf. Camillo, \textunderscore Esqueleto\textunderscore , 36.
\section{Rechaça}
\begin{itemize}
\item {Grp. gram.:f.}
\end{itemize}
Acto de rechaçar. Cf. Filinto, XVIII, 184.
\section{Rechaçador}
\begin{itemize}
\item {Grp. gram.:m.}
\end{itemize}
Aquelle que rechaça. Cf. Filinto, \textunderscore D. Man.\textunderscore , I, 385.
\section{Rechaçar}
\begin{itemize}
\item {Grp. gram.:v. t.}
\end{itemize}
Repellir.
Fazer retroceder, oppondo resistência.
Desbaratar.
(Cast. \textunderscore rechazar\textunderscore )
\section{Rechaço}
\begin{itemize}
\item {Grp. gram.:m.}
\end{itemize}
\begin{itemize}
\item {Utilização:Med.}
\end{itemize}
Acto ou effeito de rechaçar.
Ricochete.
Antiga dança.
Manobra, com que se sente o choque do feto, quando, impulsionado pelo dedo do parteiro, volta a occupar a sua posição, mergulhando no líquido amniático e na cavidade uterina.
\section{Rechan}
\begin{itemize}
\item {Grp. gram.:f.}
\end{itemize}
\begin{itemize}
\item {Proveniência:(De \textunderscore re...\textunderscore  + \textunderscore chan\textunderscore )}
\end{itemize}
Planalto.
Chapada. Cf. Camillo, \textunderscore Esqueleto\textunderscore , 36.
\section{Rechano}
\begin{itemize}
\item {Grp. gram.:m.}
\end{itemize}
O mesmo que \textunderscore rechão\textunderscore .
\section{Rechão}
\begin{itemize}
\item {Grp. gram.:m.}
\end{itemize}
O mesmo que \textunderscore rechan\textunderscore .
\section{Recheadamente}
\begin{itemize}
\item {Grp. gram.:adv.}
\end{itemize}
De modo recheado; com recheio.
\section{Recheado}
\begin{itemize}
\item {Grp. gram.:m.}
\end{itemize}
\begin{itemize}
\item {Proveniência:(De \textunderscore rechear\textunderscore )}
\end{itemize}
O mesmo que \textunderscore recheio\textunderscore .
\section{Recheadura}
\begin{itemize}
\item {Grp. gram.:f.}
\end{itemize}
Acto de rechear; recheio.
\section{Rechear}
\begin{itemize}
\item {Grp. gram.:v. t.}
\end{itemize}
\begin{itemize}
\item {Utilização:Fig.}
\end{itemize}
\begin{itemize}
\item {Proveniência:(De \textunderscore recheio\textunderscore )}
\end{itemize}
Encher bem.
Encher com preparado culinário ou de confeitaria.
Tornar abundante; enriquecer.
\section{Rechega}
\begin{itemize}
\item {fónica:chê}
\end{itemize}
\begin{itemize}
\item {Grp. gram.:f.}
\end{itemize}
\begin{itemize}
\item {Proveniência:(De \textunderscore re...\textunderscore  + \textunderscore chegar\textunderscore )}
\end{itemize}
Acto de fender pinheiros longitudinalmente, para se aproveitar maior porção de resina.
\section{Rechegar}
\begin{itemize}
\item {Grp. gram.:v.}
\end{itemize}
\begin{itemize}
\item {Utilização:t. Marn.}
\end{itemize}
\begin{itemize}
\item {Utilização:Prov.}
\end{itemize}
\begin{itemize}
\item {Utilização:alent.}
\end{itemize}
\begin{itemize}
\item {Proveniência:(De \textunderscore re...\textunderscore  + \textunderscore chegar\textunderscore )}
\end{itemize}
Mexer ou bulir com rodos os crystaes de chloreto de sódio (em salinas). Cf. \textunderscore Museu Etchn.\textunderscore , 108.
Dispor em monte (lenha traçada) para se meter no forno.
\section{Rechego}
\begin{itemize}
\item {fónica:chê}
\end{itemize}
\begin{itemize}
\item {Grp. gram.:m.}
\end{itemize}
\begin{itemize}
\item {Proveniência:(De \textunderscore re...\textunderscore  + \textunderscore chegar\textunderscore )}
\end{itemize}
Lugar, onde se esconde o caçador, para vigiar a caça.
\section{Recheio}
\begin{itemize}
\item {Grp. gram.:m.}
\end{itemize}
\begin{itemize}
\item {Proveniência:(De \textunderscore re...\textunderscore  + \textunderscore cheio\textunderscore )}
\end{itemize}
Aquillo que recheia; acto de rechear.
\section{Rechiar}
\begin{itemize}
\item {Grp. gram.:v. t.}
\end{itemize}
\begin{itemize}
\item {Proveniência:(De \textunderscore re...\textunderscore  + \textunderscore chiar\textunderscore )}
\end{itemize}
Chiar muito.
\section{Rechinado}
\begin{itemize}
\item {Grp. gram.:adj.}
\end{itemize}
\begin{itemize}
\item {Utilização:T. de Vouzella}
\end{itemize}
Diz-se da madeira cheia de nós e veios e, por isso, diffícil de obrar.
\section{Rechinante}
\begin{itemize}
\item {Grp. gram.:adj.}
\end{itemize}
Que rechina.
\section{Rechinar}
\begin{itemize}
\item {Grp. gram.:v. i.}
\end{itemize}
Produzir som agudo.
Ranger.
Produzir som peculiar ao do ferro em brasa sôbre a carne:«\textunderscore conservava o espêto sôbre o brasido, a rechinar, a lourejar...\textunderscore »Camillo, \textunderscore Brasileira\textunderscore , 106.
Silvar, cortando o ar, (falando-se de uma arma de arremêsso).
(Talvez t. onom.)
\section{Rechino}
\begin{itemize}
\item {Grp. gram.:m.}
\end{itemize}
Acto de rechinar.
\section{Rechonchudo}
\begin{itemize}
\item {Grp. gram.:adj.}
\end{itemize}
\begin{itemize}
\item {Utilização:Fam.}
\end{itemize}
Gordo; nédio.
\section{Rechupado}
\begin{itemize}
\item {Grp. gram.:adj.}
\end{itemize}
\begin{itemize}
\item {Proveniência:(De \textunderscore re...\textunderscore  + \textunderscore chupado\textunderscore )}
\end{itemize}
Muito chupado; muito magro.
\section{Reciário}
\begin{itemize}
\item {Grp. gram.:m.}
\end{itemize}
\begin{itemize}
\item {Proveniência:(Lat. \textunderscore retiarius\textunderscore )}
\end{itemize}
O gladiador romano, que levava uma rêde, para nella prender o adversário.
\section{Recibo}
\begin{itemize}
\item {Grp. gram.:m.}
\end{itemize}
\begin{itemize}
\item {Proveniência:(De \textunderscore receber\textunderscore )}
\end{itemize}
Declaração escrita de se têr recebido alguma coisa.
\section{Recidiva}
\begin{itemize}
\item {Grp. gram.:f.}
\end{itemize}
\begin{itemize}
\item {Utilização:Jur.}
\end{itemize}
\begin{itemize}
\item {Proveniência:(De \textunderscore recidivo\textunderscore )}
\end{itemize}
Recaída numa enfermidade.
O mesmo que \textunderscore reincidência\textunderscore .
\section{Recidivista}
\begin{itemize}
\item {Grp. gram.:m.}
\end{itemize}
\begin{itemize}
\item {Utilização:Jur.}
\end{itemize}
\begin{itemize}
\item {Proveniência:(De \textunderscore recidiva\textunderscore )}
\end{itemize}
Criminoso reincidente.
\section{Recidivo}
\begin{itemize}
\item {Grp. gram.:adj.}
\end{itemize}
\begin{itemize}
\item {Proveniência:(Lat. \textunderscore recidivus\textunderscore )}
\end{itemize}
Que reapparece: \textunderscore moléstia recidiva\textunderscore .
Que reincide; reincidente.
\section{Recife}
\begin{itemize}
\item {Grp. gram.:m.}
\end{itemize}
\begin{itemize}
\item {Proveniência:(Do ár. \textunderscore ar-recif\textunderscore )}
\end{itemize}
Um ou mais rochedos no mar, á flôr da água ou perto da costa.
\section{Recifoso}
\begin{itemize}
\item {Grp. gram.:adj.}
\end{itemize}
Em que há recifes.
\section{Recingir}
\begin{itemize}
\item {Grp. gram.:v. t.}
\end{itemize}
\begin{itemize}
\item {Proveniência:(De \textunderscore re...\textunderscore  + \textunderscore cingir\textunderscore )}
\end{itemize}
Cingir novamente.
\section{Recinto}
\begin{itemize}
\item {Grp. gram.:m.}
\end{itemize}
\begin{itemize}
\item {Proveniência:(Lat. \textunderscore recinctus\textunderscore )}
\end{itemize}
Determinado espaço.
Terreno ou espaço murado.
Santuário.
\section{Recipiendário}
\begin{itemize}
\item {Grp. gram.:m.}
\end{itemize}
\begin{itemize}
\item {Grp. gram.:M.  e  adj.}
\end{itemize}
\begin{itemize}
\item {Proveniência:(Do lat. \textunderscore recipiendus\textunderscore )}
\end{itemize}
Aquelle que é recebido solennemente numa aggremiação.
Aquelle que tem de receber qualquer coisa. Cf. \textunderscore Organização\textunderscore  da Ordem Militar da Tôrre-Espada, art. 20.
\section{Recipiente}
\begin{itemize}
\item {Grp. gram.:adj.}
\end{itemize}
\begin{itemize}
\item {Grp. gram.:M.}
\end{itemize}
\begin{itemize}
\item {Proveniência:(Lat. \textunderscore recipiens\textunderscore )}
\end{itemize}
Que recebe.
Vaso, que recebe os productos de qualquer operação chímica.
Campânula da máquina pneumática.
\section{Reciprocação}
\begin{itemize}
\item {Grp. gram.:f.}
\end{itemize}
\begin{itemize}
\item {Proveniência:(Lat. \textunderscore reciprocatio\textunderscore )}
\end{itemize}
Acto ou effeito de reciprocar; reciprocidade.
\section{Reciprocamente}
\begin{itemize}
\item {Grp. gram.:adv.}
\end{itemize}
De modo recíproco; mutuamente.
Alternadamente.
\section{Reciprocar}
\begin{itemize}
\item {Grp. gram.:v. t.}
\end{itemize}
\begin{itemize}
\item {Proveniência:(Lat. \textunderscore reciprocare\textunderscore )}
\end{itemize}
Tornar recíproco; mutuar.
Dar e receber em troca.
Compensar.
\section{Reciprocidade}
\begin{itemize}
\item {Grp. gram.:f.}
\end{itemize}
\begin{itemize}
\item {Proveniência:(Do lat. \textunderscore reciprocitas\textunderscore )}
\end{itemize}
Qualidade do que é recíproco; mutualidade.
\section{Recíproco}
\begin{itemize}
\item {Grp. gram.:adj.}
\end{itemize}
\begin{itemize}
\item {Proveniência:(Lat. \textunderscore reciprocus\textunderscore )}
\end{itemize}
Alternativo, mútuo: \textunderscore affecto recíproco\textunderscore .
Que se troca; permutado.
\section{Récita}
\begin{itemize}
\item {Grp. gram.:f.}
\end{itemize}
\begin{itemize}
\item {Utilização:Ext.}
\end{itemize}
\begin{itemize}
\item {Proveniência:(De \textunderscore recitar\textunderscore )}
\end{itemize}
Espectáculo de declamação.
Representação em theatro lýrico.
\section{Recitação}
\begin{itemize}
\item {Grp. gram.:f.}
\end{itemize}
\begin{itemize}
\item {Proveniência:(Lat. \textunderscore recitatio\textunderscore )}
\end{itemize}
Acto ou effeito de recitar.
\section{Recitado}
\begin{itemize}
\item {Grp. gram.:m.}
\end{itemize}
O mesmo que \textunderscore recitativo\textunderscore .
\section{Recitador}
\begin{itemize}
\item {Grp. gram.:m.  e  adj.}
\end{itemize}
\begin{itemize}
\item {Proveniência:(Lat. \textunderscore recitator\textunderscore )}
\end{itemize}
O que recita.
\section{Recitante}
\begin{itemize}
\item {Grp. gram.:adj.}
\end{itemize}
\begin{itemize}
\item {Grp. gram.:M.  e  f.}
\end{itemize}
\begin{itemize}
\item {Proveniência:(Lat. \textunderscore recitans\textunderscore )}
\end{itemize}
Que recita.
Diz-se da voz ou do instrumento, que executa insuladamente um trêcho musical.
Pessôa, que recíta.
\section{Recitar}
\begin{itemize}
\item {Grp. gram.:v. t.}
\end{itemize}
\begin{itemize}
\item {Proveniência:(Lat. \textunderscore recitare\textunderscore )}
\end{itemize}
Lêr em voz alta e clara.
Pronunciar, declamando.
Declamar.
Narrar.
\section{Recitativo}
\begin{itemize}
\item {Grp. gram.:m.}
\end{itemize}
\begin{itemize}
\item {Utilização:Ext.}
\end{itemize}
\begin{itemize}
\item {Grp. gram.:Adj.}
\end{itemize}
\begin{itemize}
\item {Proveniência:(De \textunderscore recitar\textunderscore )}
\end{itemize}
Parte de uma ópera, executada pelo cantor sem regularidade do rythmo, divergindo apenas da recitação em que as sýllabas são pronunciadas sôbre as notas da gamma.
Composição poética, destinada a sêr recitada com acompanhamento de música.
Que é próprio para sêr recitado.
\section{Reclamação}
\begin{itemize}
\item {Grp. gram.:f.}
\end{itemize}
\begin{itemize}
\item {Proveniência:(Lat. \textunderscore reclamatio\textunderscore )}
\end{itemize}
Acto ou effeito de reclamar.
\section{Reclamador}
\begin{itemize}
\item {Grp. gram.:m.  e  adj.}
\end{itemize}
\begin{itemize}
\item {Proveniência:(Do b. lat. \textunderscore reclamator\textunderscore )}
\end{itemize}
O que reclama.
\section{Reclamante}
\begin{itemize}
\item {Grp. gram.:m. ,  f.  e  adj.}
\end{itemize}
\begin{itemize}
\item {Proveniência:(Lat. \textunderscore reclamans\textunderscore )}
\end{itemize}
Pessôa, que reclama.
\section{Reclamar}
\begin{itemize}
\item {Grp. gram.:v. i.}
\end{itemize}
\begin{itemize}
\item {Grp. gram.:V. t.}
\end{itemize}
\begin{itemize}
\item {Proveniência:(Lat. \textunderscore reclamare\textunderscore )}
\end{itemize}
Oppor-se.
Protestar.
Fazer impugnação.
Exigir (o que foi tomado injustamente).
Exigir: \textunderscore reclamar o pagamento de uma dívida\textunderscore .
Attrahir (aves) com o reclamo de outra.
Implorar, clamar por: \textunderscore reclamar soccorro\textunderscore .
\section{Reclamável}
\begin{itemize}
\item {Grp. gram.:adj.}
\end{itemize}
\begin{itemize}
\item {Proveniência:(De \textunderscore reclamare\textunderscore )}
\end{itemize}
Que póde sêr reclamado ou exigido.
\section{Reclame}
\begin{itemize}
\item {Grp. gram.:m.}
\end{itemize}
Buraco, por onde passa uma corda, no alto dos mastros dos barcos rabelos do Doiro.
\section{Reclamo}
\begin{itemize}
\item {Grp. gram.:m.}
\end{itemize}
\begin{itemize}
\item {Proveniência:(De \textunderscore reclamar\textunderscore )}
\end{itemize}
O mesmo que \textunderscore reclamação\textunderscore .
Clamor.
Chamariz.
Instrumento, com que o caçador imita o canto das aves que quere attrahir.
Ave, ensinada para attrahir outras com o canto.
Acto de chamar a attenção.
Recommendação, feita num jornal.
Palavra ou palavras compostas no fundo de uma página, indicando a primeira palavra ou palavras da página immediata.
Final de uma phrase do papel de um actor, escripta no princípio do papel de outro, para indicar a êste quando tem de falar; deixa.
\section{Reclinação}
\begin{itemize}
\item {Grp. gram.:f.}
\end{itemize}
\begin{itemize}
\item {Proveniência:(Lat. \textunderscore reclinatio\textunderscore )}
\end{itemize}
Acto ou effeito de reclinar.
Antigo processo do operação da cataracta, que consistia em deitar a cataracta para trás.
\section{Reclinadamente}
\begin{itemize}
\item {Grp. gram.:adv.}
\end{itemize}
De modo reclinado.
Com inclinação.
\section{Reclinar}
\begin{itemize}
\item {Grp. gram.:v. t.}
\end{itemize}
\begin{itemize}
\item {Grp. gram.:V. p.}
\end{itemize}
\begin{itemize}
\item {Utilização:Ext.}
\end{itemize}
\begin{itemize}
\item {Proveniência:(Lat. \textunderscore reclinare\textunderscore )}
\end{itemize}
Recurvar; dobrar.
Encostar: \textunderscore reclinar a cabeça no travesseiro\textunderscore .
Inclinar-se; encostar-se.
Descansar, deitar-se.
\section{Reclinatório}
\begin{itemize}
\item {Grp. gram.:m.}
\end{itemize}
\begin{itemize}
\item {Proveniência:(Lat. \textunderscore reclinatorium\textunderscore )}
\end{itemize}
Objecto, próprio para alguém se reclinar ou descansar.
\section{Recluir}
\begin{itemize}
\item {Grp. gram.:v. t.}
\end{itemize}
\begin{itemize}
\item {Utilização:Des.}
\end{itemize}
\begin{itemize}
\item {Proveniência:(Lat. \textunderscore recludere\textunderscore )}
\end{itemize}
Encerrar.
\section{Reclusão}
\begin{itemize}
\item {Grp. gram.:f.}
\end{itemize}
\begin{itemize}
\item {Proveniência:(Lat. \textunderscore reclusio\textunderscore )}
\end{itemize}
Acto ou effeito de encerrar; encerramento; cárcere.
\section{Reclusar}
\begin{itemize}
\item {Grp. gram.:v. t.}
\end{itemize}
\begin{itemize}
\item {Utilização:Des.}
\end{itemize}
\begin{itemize}
\item {Proveniência:(De \textunderscore recluso\textunderscore )}
\end{itemize}
Encerrar.
\section{Recluso}
\begin{itemize}
\item {Grp. gram.:adj.}
\end{itemize}
\begin{itemize}
\item {Grp. gram.:M.}
\end{itemize}
\begin{itemize}
\item {Proveniência:(Lat. \textunderscore reclusus\textunderscore )}
\end{itemize}
Encerrado.
Que vive em convento ou em clausura.
Aquelle que vive em clausura.
Aquelle que foi condemnado a reclusão, ou a viver em casa correccional.
\section{Recluta}
\textunderscore m.\textunderscore  (e der.)
(Fórma pop. de \textunderscore recruta\textunderscore , etc.)
(Cp. ant. fr. \textunderscore recluter\textunderscore , b. lat. \textunderscore reclutare\textunderscore )
\section{Reco}
\begin{itemize}
\item {Grp. gram.:m.}
\end{itemize}
\begin{itemize}
\item {Utilização:Prov.}
\end{itemize}
\begin{itemize}
\item {Utilização:minh.}
\end{itemize}
\begin{itemize}
\item {Utilização:trasm.}
\end{itemize}
\begin{itemize}
\item {Utilização:Prov.}
\end{itemize}
\begin{itemize}
\item {Utilização:beir.}
\end{itemize}
O mesmo que \textunderscore porco\textunderscore .
O mesmo que \textunderscore marreco\textunderscore  ou \textunderscore pato\textunderscore .
\section{Recobra}
\begin{itemize}
\item {Grp. gram.:f.}
\end{itemize}
Acto de recobrar. Cf. Filinto, III, 43.
\section{Recobramento}
\begin{itemize}
\item {Grp. gram.:m.}
\end{itemize}
Acto ou effeito de recobrar.
\section{Recobrar}
\begin{itemize}
\item {Grp. gram.:v. t.}
\end{itemize}
\begin{itemize}
\item {Grp. gram.:V. p.}
\end{itemize}
\begin{itemize}
\item {Proveniência:(Do lat. \textunderscore recuperare\textunderscore )}
\end{itemize}
Adquirir novamente.
Recebêr (o que se tinha perdido); recuperar.
Restabelecer-se; readquirir ânimo, saúde, etc.
\section{Recobrável}
\begin{itemize}
\item {Grp. gram.:adj.}
\end{itemize}
Que se póde recobrar.
\section{Recobrir}
\begin{itemize}
\item {Grp. gram.:v. t.}
\end{itemize}
\begin{itemize}
\item {Proveniência:(De \textunderscore re...\textunderscore  + \textunderscore cobrir\textunderscore )}
\end{itemize}
Cobrir novamente; cobrir bem.
\section{Recôbro}
\begin{itemize}
\item {Grp. gram.:m.}
\end{itemize}
Acto ou effeito de recobrar.
Reanimação.
\section{Recochilado}
\begin{itemize}
\item {Grp. gram.:m.  e  adj.}
\end{itemize}
Homem arrependido, emendado?«\textunderscore vós tendes em mim dechado de amores, como a recochilado me podeis dar credito, que aos oragos de Delfos\textunderscore ». \textunderscore Eufrosina\textunderscore , acto I, sc. 1.
\section{Recocto}
\begin{itemize}
\item {Grp. gram.:adj.}
\end{itemize}
\begin{itemize}
\item {Utilização:Des.}
\end{itemize}
\begin{itemize}
\item {Proveniência:(Lat. \textunderscore recoctus\textunderscore )}
\end{itemize}
O mesmo que [[recozido|recozer]].
\section{Recognição}
\begin{itemize}
\item {Grp. gram.:f.}
\end{itemize}
\begin{itemize}
\item {Proveniência:(Lat. \textunderscore recognitio\textunderscore )}
\end{itemize}
Acto de reconhecer.
\section{Recognitivo}
\begin{itemize}
\item {Grp. gram.:adj.}
\end{itemize}
\begin{itemize}
\item {Proveniência:(Do lat. \textunderscore recognitum\textunderscore )}
\end{itemize}
Próprio para reconhecer ou averiguar uma coisa. Cf. Assis, \textunderscore Águas\textunderscore , 271.
\section{Recoice}
\begin{itemize}
\item {Grp. gram.:m.}
\end{itemize}
\begin{itemize}
\item {Utilização:Des.}
\end{itemize}
\begin{itemize}
\item {Proveniência:(De \textunderscore re...\textunderscore  + \textunderscore coice\textunderscore )}
\end{itemize}
Acto de recuar.
\section{Recoitar}
\begin{itemize}
\item {Grp. gram.:v. t.}
\end{itemize}
\begin{itemize}
\item {Proveniência:(De \textunderscore recoito\textunderscore )}
\end{itemize}
Recozer, (falando-se dos metaes).
\section{Recoito}
\begin{itemize}
\item {Grp. gram.:adj.}
\end{itemize}
\begin{itemize}
\item {Proveniência:(Do lat. \textunderscore recoctus\textunderscore )}
\end{itemize}
O mesmo que [[recozido|recozer]], (falando-se de metaes).
\section{Recolecta}
\begin{itemize}
\item {Grp. gram.:f.}
\end{itemize}
\begin{itemize}
\item {Utilização:Ant.}
\end{itemize}
Casa de religiosos reformados da Ordem de San-Francisco.
Freira da Ordem reformada de San-Francisco.
(Fem. de \textunderscore recollecto\textunderscore )
\section{Recolecto}
\begin{itemize}
\item {Grp. gram.:adj.}
\end{itemize}
\begin{itemize}
\item {Grp. gram.:M.}
\end{itemize}
\begin{itemize}
\item {Proveniência:(Lat. \textunderscore recollectus\textunderscore )}
\end{itemize}
Relativo á Ordem reformada de San-Francisco.
Frade dessa Ordem.
\section{Recoleição}
\begin{itemize}
\item {Grp. gram.:f.}
\end{itemize}
Casa de recolectos.
Vida de recolecto ou recolecta; austeridade religiosa. Cf. Luc. Cordeiro, \textunderscore Senh. Duq.\textunderscore , 158 e 243, e \textunderscore Supplem. ao Diccion. Port. de Algib.\textunderscore 
\section{Recoleta}
\begin{itemize}
\item {fónica:lê}
\end{itemize}
\begin{itemize}
\item {Grp. gram.:f.}
\end{itemize}
\begin{itemize}
\item {Utilização:T. de Aveiro}
\end{itemize}
\begin{itemize}
\item {Proveniência:(De \textunderscore recollecta\textunderscore ?)}
\end{itemize}
Barracão, para vivenda, com uma só vertente de telhado.
\section{Recolheito}
\begin{itemize}
\item {Grp. gram.:adj.}
\end{itemize}
\begin{itemize}
\item {Utilização:Ant.}
\end{itemize}
\begin{itemize}
\item {Proveniência:(Do lat. \textunderscore recollectus\textunderscore )}
\end{itemize}
O mesmo que \textunderscore recolhido\textunderscore .
\section{Recolha}
\begin{itemize}
\item {fónica:cô}
\end{itemize}
\begin{itemize}
\item {Grp. gram.:f.}
\end{itemize}
Acto de recolher.
\textunderscore Cocheira de recolha\textunderscore , cocheira, em que se guardam automáveis, carruagens de aluguer, etc.
Casa ou lugar, em que se recolhe gado provisoriamente, por dinheiro. Cf. \textunderscore Decreto\textunderscore  de 22-VII-905.
\section{Recolhença}
\begin{itemize}
\item {Grp. gram.:f.}
\end{itemize}
Acto de recolher os productos agrícolas; colheita.
\section{Recolher}
\begin{itemize}
\item {Grp. gram.:v. t.}
\end{itemize}
\begin{itemize}
\item {Grp. gram.:V. i.  e  p.}
\end{itemize}
\begin{itemize}
\item {Proveniência:(Do lat. \textunderscore recolligere\textunderscore )}
\end{itemize}
Colher, apanhar.
Guardar.
Dar abrigo a.
Juntar: \textunderscore recolher economias\textunderscore .
Fazer a colheita de: \textunderscore recolher o azeite\textunderscore .
Meter dentro de alguma coisa.
Receber em recompensa.
Encolher.
Inferir.
Voltar para sua casa, reentrar em casa.
Regressar á sua patria.
Abrigar-se; refugiar-se.
Concentrar-se.
Supprimir-se, reflectindo.
Retirar-se para os aposentos.
Supprimir-se (uma erupção cutânea), desenvolvendo-se interiormente.
Retrahir-se, fugindo á vida mundana.
Fazer economias.
\section{Recolhida}
\begin{itemize}
\item {Grp. gram.:f.}
\end{itemize}
Recolhimento.
Mulhér, que vive em convento, sem têr feito votos.
(Fem. de \textunderscore recolhido\textunderscore )
\section{Recolhidamente}
\begin{itemize}
\item {Grp. gram.:adv.}
\end{itemize}
De modo recolhido; com recolhimento; concentradamente.
\section{Recolhido}
\begin{itemize}
\item {Grp. gram.:adj.}
\end{itemize}
\begin{itemize}
\item {Proveniência:(De \textunderscore recolher\textunderscore )}
\end{itemize}
Que se recolheu ou recebeu: \textunderscore esmolas recolhidas\textunderscore .
Retrahido de convivência: \textunderscore o João vive muito recolhido\textunderscore .
\section{Recolhimento}
\begin{itemize}
\item {Grp. gram.:m.}
\end{itemize}
\begin{itemize}
\item {Proveniência:(De \textunderscore recolher\textunderscore )}
\end{itemize}
Acto ou effeito de recolher.
Lugar, onde se recolhe alguém ou alguma coisa.
Recato; vida concentrada; meditação.
\section{Recolho}
\begin{itemize}
\item {fónica:cô}
\end{itemize}
\begin{itemize}
\item {Grp. gram.:m.}
\end{itemize}
Acto ou effeito de recolher.
Respiração forte.
Respiração da baleia, expellindo água.
\section{Recollecta}
\begin{itemize}
\item {Grp. gram.:f.}
\end{itemize}
\begin{itemize}
\item {Utilização:Ant.}
\end{itemize}
Casa de religiosos reformados da Ordem de San-Francisco.
Freira da Ordem reformada de San-Francisco.
(Fem. de \textunderscore recollecto\textunderscore )
\section{Recollecto}
\begin{itemize}
\item {Grp. gram.:adj.}
\end{itemize}
\begin{itemize}
\item {Grp. gram.:M.}
\end{itemize}
\begin{itemize}
\item {Proveniência:(Lat. \textunderscore recollectus\textunderscore )}
\end{itemize}
Relativo á Ordem reformada de San-Francisco.
Frade dessa Ordem.
\section{Recolleição}
\begin{itemize}
\item {Grp. gram.:f.}
\end{itemize}
Casa de recollectos.
Vida de recollecto ou recollecta; austeridade religiosa. Cf. Luc. Cordeiro, \textunderscore Senh. Duq.\textunderscore , 158 e 243, e \textunderscore Supplem. ao Diccion. Port. de Algib.\textunderscore 
\section{Recollocar}
\begin{itemize}
\item {Grp. gram.:v. t.}
\end{itemize}
Collocar novamente.
\section{Recolocar}
\begin{itemize}
\item {Grp. gram.:v. t.}
\end{itemize}
Colocar novamente.
\section{Recolta}
\begin{itemize}
\item {Grp. gram.:f.}
\end{itemize}
\begin{itemize}
\item {Utilização:P. us.}
\end{itemize}
Recolhença, colheita.
(Contr. do lat. \textunderscore recollecta\textunderscore )
\section{Recoltar}
\begin{itemize}
\item {Grp. gram.:v. t.}
\end{itemize}
\begin{itemize}
\item {Utilização:Neol.}
\end{itemize}
Fazer recolta de; recolher, apanhar.
\section{Recombinação}
\begin{itemize}
\item {Grp. gram.:f.}
\end{itemize}
Acto de \textunderscore recombinar\textunderscore .
\section{Recombinar}
\begin{itemize}
\item {Grp. gram.:v.}
\end{itemize}
\begin{itemize}
\item {Utilização:t. Chím.}
\end{itemize}
\begin{itemize}
\item {Proveniência:(De \textunderscore re...\textunderscore  + \textunderscore combinar\textunderscore )}
\end{itemize}
Sujeitar a nova combinação. Cf. F. Lapa, \textunderscore Phys.\textunderscore , 107.
\section{Recomeçar}
\begin{itemize}
\item {Grp. gram.:v. t.}
\end{itemize}
\begin{itemize}
\item {Proveniência:(De \textunderscore re...\textunderscore  + \textunderscore começar\textunderscore )}
\end{itemize}
Começar novamente.
\section{Recomêço}
\begin{itemize}
\item {Grp. gram.:m.}
\end{itemize}
Acto de recomeçar.
\section{Recomendação}
\begin{itemize}
\item {Grp. gram.:f.}
\end{itemize}
\begin{itemize}
\item {Grp. gram.:Pl.}
\end{itemize}
Acto ou effeito de recomendar.
Qualidade de quem é recomendável.
Advertência.
Cumprimentos: \textunderscore mandou-me muitas recomendações\textunderscore .
\section{Recomendado}
\begin{itemize}
\item {Grp. gram.:adj.}
\end{itemize}
\begin{itemize}
\item {Grp. gram.:M.}
\end{itemize}
\begin{itemize}
\item {Proveniência:(De \textunderscore recomendar\textunderscore )}
\end{itemize}
Que é objecto de recomendação ou de empenho: \textunderscore assumpto recomendado\textunderscore .
Indivíduo recomendado ou protegido.
\section{Recomendar}
\begin{itemize}
\item {Grp. gram.:v. t.}
\end{itemize}
\begin{itemize}
\item {Grp. gram.:V. p.}
\end{itemize}
\begin{itemize}
\item {Proveniência:(Lat. \textunderscore recommendare\textunderscore )}
\end{itemize}
Confiar a alguém.
Dar a alguém a missão de.
Ordenar a alguém.
Encarecer.
Apontar como bom: \textunderscore recomendar uma obra literária\textunderscore .
Pedir cuidado para.
Solicitar.
Apresentar os cumprimentos de.
Pedir favor ou protecção para: \textunderscore recomendo-lhe o rapaz\textunderscore .
Aconselhar: \textunderscore recomendo-lhe prudência\textunderscore .
Tornar-se digno de respeito, protecção ou favor.
\section{Recomendatório}
\begin{itemize}
\item {Grp. gram.:adj.}
\end{itemize}
Que serve para recomendar; que recomenda; que serve de empenho.
\section{Recomendável}
\begin{itemize}
\item {Grp. gram.:adj.}
\end{itemize}
\begin{itemize}
\item {Proveniência:(De \textunderscore recomendar\textunderscore )}
\end{itemize}
Digno de sêr recomendado.
Digno de estima ou de respeito.
\section{Recomendavelmente}
\begin{itemize}
\item {Grp. gram.:adv.}
\end{itemize}
De modo recomendável.
\section{Recomer}
\begin{itemize}
\item {Grp. gram.:v. t.}
\end{itemize}
\begin{itemize}
\item {Proveniência:(De \textunderscore re...\textunderscore  + \textunderscore comer\textunderscore )}
\end{itemize}
Tornar a mastigar; ruminar. Cf. Vieira, VI, 300.
\section{Recommendação}
\begin{itemize}
\item {Grp. gram.:f.}
\end{itemize}
\begin{itemize}
\item {Grp. gram.:Pl.}
\end{itemize}
Acto ou effeito de recommendar.
Qualidade de quem é recommendável.
Advertência.
Cumprimentos: \textunderscore mandou-me muitas recommendações\textunderscore .
\section{Recommendado}
\begin{itemize}
\item {Grp. gram.:adj.}
\end{itemize}
\begin{itemize}
\item {Grp. gram.:M.}
\end{itemize}
\begin{itemize}
\item {Proveniência:(De \textunderscore recommendar\textunderscore )}
\end{itemize}
Que é objecto de recommendação ou de empenho: \textunderscore assumpto recommendado\textunderscore .
Indivíduo recommendado ou protegido.
\section{Recommendar}
\begin{itemize}
\item {Grp. gram.:v. t.}
\end{itemize}
\begin{itemize}
\item {Grp. gram.:V. p.}
\end{itemize}
\begin{itemize}
\item {Proveniência:(Lat. \textunderscore recommendare\textunderscore )}
\end{itemize}
Confiar a alguém.
Dar a alguém a missão de.
Ordenar a alguém.
Encarecer.
Apontar como bom: \textunderscore recommendar uma obra literária\textunderscore .
Pedir cuidado para.
Solicitar.
Apresentar os cumprimentos de.
Pedir favor ou protecção para: \textunderscore recommendo-lhe o rapaz\textunderscore .
Aconselhar: \textunderscore recommendo-lhe prudência\textunderscore .
Tornar-se digno de respeito, protecção ou favor.
\section{Recommendatório}
\begin{itemize}
\item {Grp. gram.:adj.}
\end{itemize}
Que serve para recommendar; que recommenda; que serve de empenho.
\section{Recommendável}
\begin{itemize}
\item {Grp. gram.:adj.}
\end{itemize}
\begin{itemize}
\item {Proveniência:(De \textunderscore recommendar\textunderscore )}
\end{itemize}
Digno de sêr recommendado.
Digno de estima ou de respeito.
\section{Recommendavelmente}
\begin{itemize}
\item {Grp. gram.:adv.}
\end{itemize}
De modo recommendável.
\section{Recompensa}
\begin{itemize}
\item {Grp. gram.:f.}
\end{itemize}
Actou ou effeito de recompensar; prêmio; galardão.
\section{Recompensação}
\begin{itemize}
\item {Grp. gram.:f.}
\end{itemize}
O mesmo que \textunderscore recompensa\textunderscore .
\section{Recompensador}
\begin{itemize}
\item {Grp. gram.:m.  e  adj.}
\end{itemize}
O que recompensa.
\section{Recompensamento}
\begin{itemize}
\item {Grp. gram.:m.}
\end{itemize}
\begin{itemize}
\item {Utilização:Ant.}
\end{itemize}
O mesmo que \textunderscore recompensa\textunderscore .
\section{Recompensar}
\begin{itemize}
\item {Grp. gram.:v. t.}
\end{itemize}
\begin{itemize}
\item {Proveniência:(Do lat. \textunderscore re...\textunderscore  + \textunderscore compensare\textunderscore )}
\end{itemize}
Reconhecer os serviços ou o bom procedimento de, dando-lhe alguma coisa.
Premiar; galardoar; compensar.
Pagar.
\section{Recompensável}
\begin{itemize}
\item {Grp. gram.:adj.}
\end{itemize}
\begin{itemize}
\item {Proveniência:(De \textunderscore recompensar\textunderscore )}
\end{itemize}
Digno de recompensa.
\section{Recomponente}
\begin{itemize}
\item {Grp. gram.:adj.}
\end{itemize}
Que recompõe.
\section{Recompor}
\begin{itemize}
\item {Grp. gram.:v. t.}
\end{itemize}
\begin{itemize}
\item {Proveniência:(Lat. \textunderscore recomponere\textunderscore )}
\end{itemize}
Tornar a compor.
Dispor novamente.
Dar nova fórma a.
Reorganizar: reconstruír.
Restabelecer.
Reconciliar, harmonizar.
\section{Recomposição}
\begin{itemize}
\item {Grp. gram.:f.}
\end{itemize}
Acto ou effeito de recompor.
Reconciliação.
Formação de um Ministério, conservando-se nelle algum ou alguns dos membros do Ministério antecedente.
\section{Recomposto}
\begin{itemize}
\item {Grp. gram.:adj.}
\end{itemize}
\begin{itemize}
\item {Proveniência:(Do lat. \textunderscore recompositus\textunderscore )}
\end{itemize}
Que se recompôs.
Reorganizado.
Restabelecido.
\section{Recomprar}
\begin{itemize}
\item {Grp. gram.:v. t.}
\end{itemize}
\begin{itemize}
\item {Proveniência:(De \textunderscore re...\textunderscore  + \textunderscore comprar\textunderscore )}
\end{itemize}
Tornar a comprar, comprar de novo:«\textunderscore ...lanígeros animaes recomprando...\textunderscore »Filinto, XII, 144.
\section{Recôncavo}
\begin{itemize}
\item {Grp. gram.:m.}
\end{itemize}
\begin{itemize}
\item {Proveniência:(De \textunderscore re...\textunderscore  + \textunderscore côncavo\textunderscore )}
\end{itemize}
Cavidade funda; concavidade; gruta; antro.
\section{Reconcentração}
\begin{itemize}
\item {Grp. gram.:f.}
\end{itemize}
Acto ou effeito de reconcentrar.
\section{Reconcentrar}
\begin{itemize}
\item {Grp. gram.:v. t.}
\end{itemize}
\begin{itemize}
\item {Proveniência:(De \textunderscore re...\textunderscore  + \textunderscore concentrar\textunderscore )}
\end{itemize}
Fazer convergir para um centro.
Concentrar em si.
Fixar (o cuidado, a attenção) num ponto ou num assumpto.
\section{Reconcertar}
\begin{itemize}
\item {Grp. gram.:v. t.}
\end{itemize}
\begin{itemize}
\item {Proveniência:(De \textunderscore re...\textunderscore  + \textunderscore concertar\textunderscore )}
\end{itemize}
Concertar novamente. Cf. Castilho, \textunderscore Fastos\textunderscore , II, 7.
\section{Reconciliação}
\begin{itemize}
\item {Grp. gram.:f.}
\end{itemize}
\begin{itemize}
\item {Proveniência:(Lat. \textunderscore reconciliatio\textunderscore )}
\end{itemize}
Acto ou effeito de reconciliar.
Restabelecimento de bôas relações pessoaes ou políticas.
Confissão religiosa, feita por devoção.
Admissão de um convertido no seio da Igreja.
Nova consagração de um templo que se profanou.
\section{Reconciliado}
\begin{itemize}
\item {Grp. gram.:f.}
\end{itemize}
\begin{itemize}
\item {Proveniência:(De \textunderscore reconciliar\textunderscore )}
\end{itemize}
Penitente, que se confessou e ficou absolvido.
\section{Reconciliador}
\begin{itemize}
\item {Grp. gram.:m.}
\end{itemize}
\begin{itemize}
\item {Proveniência:(Lat. \textunderscore reconciliator\textunderscore )}
\end{itemize}
Aquelle que reconcilia.
\section{Reconciliar}
\begin{itemize}
\item {Grp. gram.:v. t.}
\end{itemize}
\begin{itemize}
\item {Proveniência:(Lat. \textunderscore reconciliare\textunderscore )}
\end{itemize}
Restituir á paz ou ás bôas relações perdidas.
Estabelecer acôrdo entre; congraçar.
Restituír á graça divina.
Absolver: \textunderscore o confessor reconciliou a penitente\textunderscore .
\section{Reconciliatório}
\begin{itemize}
\item {Grp. gram.:adj.}
\end{itemize}
\begin{itemize}
\item {Proveniência:(Lat. \textunderscore reconciliatorius\textunderscore )}
\end{itemize}
Que póde reconciliar.
Que serve para reconciliar.
\section{Reconciliável}
\begin{itemize}
\item {Grp. gram.:adj.}
\end{itemize}
Que se póde reconciliar.
\section{Reconcóvio}
\begin{itemize}
\item {Grp. gram.:m.}
\end{itemize}
(Uma das extravagâncias morphológicas de Filinto, em vez de \textunderscore recôncavo\textunderscore :«\textunderscore ...nos mais encruzilhados reconcóvios...\textunderscore »Filinto, IV, 102)
\section{Recôndito}
\begin{itemize}
\item {Grp. gram.:adj.}
\end{itemize}
\begin{itemize}
\item {Grp. gram.:M.}
\end{itemize}
\begin{itemize}
\item {Proveniência:(Lat. \textunderscore reconditus\textunderscore )}
\end{itemize}
Encerrado; escondido.
Desconhecido.
Recanto, estaninho, esconso.
\section{Reconditório}
\begin{itemize}
\item {Grp. gram.:m.}
\end{itemize}
\begin{itemize}
\item {Proveniência:(Lat. \textunderscore reconditorium\textunderscore )}
\end{itemize}
Lugar occulto.
\section{Recondução}
\begin{itemize}
\item {Grp. gram.:f.}
\end{itemize}
Acto ou efeito de reconduzir.
\section{Reconducção}
\begin{itemize}
\item {Grp. gram.:f.}
\end{itemize}
Acto ou effeito de reconduzir.
\section{Reconduzir}
\begin{itemize}
\item {Grp. gram.:v. t.}
\end{itemize}
\begin{itemize}
\item {Proveniência:(Do lat. \textunderscore reconducere\textunderscore )}
\end{itemize}
Conduzir novamente; reenviar, devolver.
Adiar, prorogar.
Reeleger.
Nomear novamente para um cargo, que o nomeado exercia por tempo determinado: \textunderscore reconduzir um juiz municipal\textunderscore ; \textunderscore reconduzir um administrador de fallências\textunderscore .
\section{Reconfessar}
\begin{itemize}
\item {Grp. gram.:v. t.}
\end{itemize}
\begin{itemize}
\item {Proveniência:(De \textunderscore re...\textunderscore  + \textunderscore confessar\textunderscore )}
\end{itemize}
Confessar de novo.
\section{Reconfortador}
\begin{itemize}
\item {Grp. gram.:adj.}
\end{itemize}
O mesmo que \textunderscore reconfortante\textunderscore .
\section{Reconfortante}
\begin{itemize}
\item {Grp. gram.:adj.}
\end{itemize}
Que reconforta.
\section{Reconfortar}
\begin{itemize}
\item {Grp. gram.:v. t.}
\end{itemize}
\begin{itemize}
\item {Proveniência:(De \textunderscore re...\textunderscore  + \textunderscore confortar\textunderscore )}
\end{itemize}
Confortar muito; dar novo alento; revigorar.
\section{Reconfôrto}
\begin{itemize}
\item {Grp. gram.:m.}
\end{itemize}
Acto ou effeito de reconfortar.
\section{Recongraçar}
\begin{itemize}
\item {Grp. gram.:v. t.}
\end{itemize}
\begin{itemize}
\item {Proveniência:(De \textunderscore re...\textunderscore  + \textunderscore congraçar\textunderscore )}
\end{itemize}
O mesmo que \textunderscore reconciliar\textunderscore .
\section{Reconhecença}
\begin{itemize}
\item {Grp. gram.:f.}
\end{itemize}
\begin{itemize}
\item {Utilização:Ant.}
\end{itemize}
O mesmo que \textunderscore reconhecimento\textunderscore .
Pensão ou tributo, que se pagava a certos Bispos ou Cabidos.
\section{Reconhecer}
\begin{itemize}
\item {Grp. gram.:v. t.}
\end{itemize}
\begin{itemize}
\item {Proveniência:(Lat. \textunderscore recognoscere\textunderscore )}
\end{itemize}
Conhecer novamente (o que se tinha conhecido noutro tempo).
Entender que alguém ou alguma coisa é o mesmo que era.
Comprehender.
Verificar.
Ponderar.
Observar.
Estar certo de: \textunderscore reconhecer a fôrça de um argumento\textunderscore .
Considerar legítimo: \textunderscore reconhecer um filho natural\textunderscore .
Agradecer.
Confessar.
\section{Reconhecidamente}
\begin{itemize}
\item {Grp. gram.:adv.}
\end{itemize}
De modo reconhecido.
Com reconhecimento, com gratidão.
\section{Reconhecido}
\begin{itemize}
\item {Grp. gram.:adj.}
\end{itemize}
\begin{itemize}
\item {Proveniência:(De \textunderscore reconhecer\textunderscore )}
\end{itemize}
Agradecido, obrigado.
\section{Reconhecimento}
\begin{itemize}
\item {Grp. gram.:m.}
\end{itemize}
Acto ou effeito de reconhecer.
Gratidão.
\section{Reconhecível}
\begin{itemize}
\item {Grp. gram.:adj.}
\end{itemize}
Que se póde reconhecer:«\textunderscore quem faz versos reconhecíveis sem maiúscula...\textunderscore »Castilho, \textunderscore Outono\textunderscore , (pról.)
\section{Reconquista}
\begin{itemize}
\item {Grp. gram.:f.}
\end{itemize}
Acto ou effeito de reconquistar.
\section{Reconquistar}
\begin{itemize}
\item {Grp. gram.:v. t.}
\end{itemize}
\begin{itemize}
\item {Proveniência:(De \textunderscore re...\textunderscore  + \textunderscore conquistar\textunderscore )}
\end{itemize}
Conquistar novamente.
Conquistar (o que outro conquistara).
\section{Reconsideração}
\begin{itemize}
\item {Grp. gram.:f.}
\end{itemize}
Acto de reconsiderar.
\section{Reconsiderar}
\begin{itemize}
\item {Grp. gram.:v. t.}
\end{itemize}
\begin{itemize}
\item {Grp. gram.:V. i.}
\end{itemize}
\begin{itemize}
\item {Proveniência:(De \textunderscore re...\textunderscore  + \textunderscore considerar\textunderscore )}
\end{itemize}
Considerar ou ponderar de novo: \textunderscore reconsiderar os factos\textunderscore .
Reflectir, suspendendo resolução tomada.
Tomar nova resolução.
Desdizer-se.
Arrepender-se alguém do que tinha feito.
\section{Reconsorciar}
\begin{itemize}
\item {Grp. gram.:v. t.}
\end{itemize}
\begin{itemize}
\item {Proveniência:(De \textunderscore re...\textunderscore  + \textunderscore consorciar\textunderscore )}
\end{itemize}
Consorciar novamente.
\section{Reconstituinte}
\begin{itemize}
\item {Grp. gram.:adj.}
\end{itemize}
\begin{itemize}
\item {Grp. gram.:M.}
\end{itemize}
\begin{itemize}
\item {Proveniência:(De \textunderscore reconstituir\textunderscore )}
\end{itemize}
Que reconstitue: \textunderscore medicamento reconstituinte\textunderscore .
Medicamento, próprio para restabelecer as fôrças de um enfermo, de um convalescente ou de pessôa débil.
\section{Reconstituir}
\begin{itemize}
\item {Grp. gram.:v. t.}
\end{itemize}
\begin{itemize}
\item {Proveniência:(De \textunderscore re...\textunderscore  + \textunderscore constituir\textunderscore )}
\end{itemize}
Tornar a constituir; recompor; restabelecer.
Restaurar as fôrças de.
\section{Reconstitutivo}
\begin{itemize}
\item {Grp. gram.:adj.}
\end{itemize}
O mesmo que \textunderscore reconstituinte\textunderscore .
\section{Reconstrução}
\begin{itemize}
\item {Grp. gram.:f.}
\end{itemize}
Acto ou efeito de reconstruír.
\section{Reconstrucção}
\begin{itemize}
\item {Grp. gram.:f.}
\end{itemize}
Acto ou effeito de reconstruír.
\section{Reconstructor}
\begin{itemize}
\item {Grp. gram.:m.}
\end{itemize}
Aquelle que reconstrói.
\section{Reconstruinte}
\begin{itemize}
\item {Grp. gram.:adj.}
\end{itemize}
Que reconstrói. Cf. Camillo, \textunderscore Mar. da Fonte\textunderscore , 74.
\section{Reconstruir}
\begin{itemize}
\item {Grp. gram.:v. t.}
\end{itemize}
\begin{itemize}
\item {Proveniência:(De \textunderscore re...\textunderscore  + \textunderscore construir\textunderscore )}
\end{itemize}
Construír novamente; reorganizar; reformar.
\section{Reconstrutor}
\begin{itemize}
\item {Grp. gram.:m.}
\end{itemize}
Aquelle que reconstrói.
\section{Reconsultar}
\begin{itemize}
\item {Grp. gram.:v. t.}
\end{itemize}
\begin{itemize}
\item {Proveniência:(De \textunderscore re...\textunderscore  + \textunderscore consultar\textunderscore )}
\end{itemize}
Consultar novamente. Cf. Castilho, \textunderscore Metam.\textunderscore , 190.
\section{Recontamento}
\begin{itemize}
\item {Grp. gram.:m.}
\end{itemize}
Acto de recontar.
\section{Recontar}
\begin{itemize}
\item {Grp. gram.:v. t.}
\end{itemize}
\begin{itemize}
\item {Proveniência:(De \textunderscore re...\textunderscore  + \textunderscore contar\textunderscore )}
\end{itemize}
Tornar a contar.
Contar minuciosamente.
Calcular novamente.
Narrar.
\section{Recontente}
\begin{itemize}
\item {Grp. gram.:adj.}
\end{itemize}
\begin{itemize}
\item {Proveniência:(De \textunderscore re...\textunderscore  + \textunderscore contente\textunderscore )}
\end{itemize}
Muito contente.
\section{Reconto}
\begin{itemize}
\item {Grp. gram.:m.}
\end{itemize}
\begin{itemize}
\item {Proveniência:(De \textunderscore re\textunderscore  + \textunderscore conto\textunderscore ^2)}
\end{itemize}
Conto da lança, por detrás da haste.
\section{Reconto}
\begin{itemize}
\item {Grp. gram.:m.}
\end{itemize}
\begin{itemize}
\item {Utilização:Prov.}
\end{itemize}
\begin{itemize}
\item {Utilização:minh.}
\end{itemize}
Acto ou effeito de recontar.
Chamada de gente, para verificar a presença obrigatória de certas pessôas.
(Cp. it. \textunderscore raconto\textunderscore )
\section{Recontratar}
\begin{itemize}
\item {Grp. gram.:v. t.}
\end{itemize}
\begin{itemize}
\item {Proveniência:(De \textunderscore re...\textunderscore  + \textunderscore contratar\textunderscore )}
\end{itemize}
Contratar novamente. Cf. Vic. Pinheiro, \textunderscore S. Thomé\textunderscore , 173.
\section{Recontrato}
\begin{itemize}
\item {Grp. gram.:m.}
\end{itemize}
Acto de recontratar; renovação de contrato.
\section{Recontro}
\begin{itemize}
\item {Grp. gram.:m.}
\end{itemize}
\begin{itemize}
\item {Proveniência:(De \textunderscore re...\textunderscore . + \textunderscore encontro\textunderscore )}
\end{itemize}
Embate; encontro de forças combatentes.
Peleja de pouca duração.
Encontro furtuito.
\section{Reconvalescença}
\begin{itemize}
\item {Grp. gram.:f.}
\end{itemize}
Acto de reconvalescer.
\section{Reconvalescer}
\begin{itemize}
\item {Grp. gram.:v. i.}
\end{itemize}
\begin{itemize}
\item {Proveniência:(Do lat. \textunderscore reconvalescere\textunderscore )}
\end{itemize}
Tornar a convalescer.
\section{Reconvenção}
\begin{itemize}
\item {Grp. gram.:f.}
\end{itemize}
Acto ou effeito de reconvir.
Acção judicial, em que um réu demanda o autor por obrigação análoga ou relativa àquella por que é demandado.
\section{Reconvir}
\begin{itemize}
\item {Grp. gram.:v. t.}
\end{itemize}
\begin{itemize}
\item {Proveniência:(De \textunderscore re...\textunderscore  + \textunderscore convir\textunderscore )}
\end{itemize}
Demandar judicialmente (o autor de uma demanda), por encargos que attenuam a importância desta.
Recriminar (alguém que accusa), para deminuir o valor da accusação.
Rehaver.
Recordar. Cp. Castilho, \textunderscore Metam.\textunderscore , XXXVIII.
\section{Recopiar}
\begin{itemize}
\item {Grp. gram.:v. t.}
\end{itemize}
\begin{itemize}
\item {Proveniência:(De \textunderscore re...\textunderscore  + \textunderscore copiar\textunderscore )}
\end{itemize}
Copiar de novo. Cf. Castilho, \textunderscore Fastos\textunderscore , I, 324.
\section{Recopilação}
\begin{itemize}
\item {Grp. gram.:f.}
\end{itemize}
Acto ou effeito de recopilar.
\section{Recopiladamente}
\begin{itemize}
\item {Grp. gram.:adv.}
\end{itemize}
De modo recopilado; summariamente.
\section{Recopilador}
\begin{itemize}
\item {Grp. gram.:m.  e  adj.}
\end{itemize}
Aquelle que recopila.
\section{Recopilar}
\begin{itemize}
\item {Grp. gram.:v. t.}
\end{itemize}
O mesmo que \textunderscore compilar\textunderscore ; compendiar; juntar extractos de; colligir.
(Alter. de \textunderscore re...\textunderscore  + \textunderscore compilar\textunderscore )
\section{Recorçar}
\begin{itemize}
\item {Grp. gram.:v. t.}
\end{itemize}
\begin{itemize}
\item {Utilização:P. us.}
\end{itemize}
O mesmo que \textunderscore escorçar\textunderscore .
\section{Recordação}
\begin{itemize}
\item {Grp. gram.:f.}
\end{itemize}
\begin{itemize}
\item {Proveniência:(Lat. \textunderscore recordatio\textunderscore )}
\end{itemize}
Acto ou effeito de recordar.
Memória; lembrança.
Objecto, que relembra coisa ou pessôa: \textunderscore êste anel é uma recordação\textunderscore .
\section{Recordador}
\begin{itemize}
\item {Grp. gram.:m.  e  adj.}
\end{itemize}
O que recorda.
\section{Recordar}
\begin{itemize}
\item {Grp. gram.:v. t.}
\end{itemize}
\begin{itemize}
\item {Grp. gram.:V. i.}
\end{itemize}
\begin{itemize}
\item {Utilização:Ant.}
\end{itemize}
\begin{itemize}
\item {Proveniência:(Lat. \textunderscore recordari\textunderscore )}
\end{itemize}
Trazer á memoria; lembrar-se de.
Sêr semelhante a.
Fazer lembrar.
Acordar, despertar do somno. Cf. \textunderscore Auto de S. António\textunderscore , 10.
\section{Recordativo}
\begin{itemize}
\item {Grp. gram.:adj.}
\end{itemize}
\begin{itemize}
\item {Proveniência:(Lat. \textunderscore recordativus\textunderscore )}
\end{itemize}
Que faz recordar.
\section{Recordatório}
\begin{itemize}
\item {Grp. gram.:adj.}
\end{itemize}
O mesmo que \textunderscore recordativo\textunderscore .
\section{Recôrdo}
\begin{itemize}
\item {Grp. gram.:m.}
\end{itemize}
\begin{itemize}
\item {Utilização:P. us.}
\end{itemize}
\begin{itemize}
\item {Proveniência:(De \textunderscore recordar\textunderscore . Cp. cast. \textunderscore recuerdo\textunderscore )}
\end{itemize}
O mesmo que \textunderscore recordação\textunderscore . Cf. Camillo, \textunderscore Caveira\textunderscore , 226.
\section{Recorrecção}
\begin{itemize}
\item {Grp. gram.:f.}
\end{itemize}
Acto ou effeito de recorrigir:«\textunderscore ...é quási illegível de correcção e recorrecção.\textunderscore »Castilho, \textunderscore Livrar. Cláss.\textunderscore , VII, 121.
\section{Recorreição}
\begin{itemize}
\item {Grp. gram.:f.}
\end{itemize}
\begin{itemize}
\item {Utilização:Ant.}
\end{itemize}
Freguesia, paróchia. Cf. S. R. Viterbo, \textunderscore Elucidário\textunderscore .
\section{Recorrência}
\begin{itemize}
\item {Grp. gram.:f.}
\end{itemize}
\begin{itemize}
\item {Utilização:Ant.}
\end{itemize}
O mesmo que \textunderscore recorreição\textunderscore . Cf. S. R. Viterbo, \textunderscore Elucidário\textunderscore .
\section{Recorrente}
\begin{itemize}
\item {Grp. gram.:adj.}
\end{itemize}
\begin{itemize}
\item {Grp. gram.:M.  e  f.}
\end{itemize}
\begin{itemize}
\item {Proveniência:(Lat. \textunderscore recurrens\textunderscore )}
\end{itemize}
Que recorre.
Pessôa, que recorre de um despacho ou sentença judicial.
\section{Recorrer}
\begin{itemize}
\item {Grp. gram.:v. t.}
\end{itemize}
\begin{itemize}
\item {Grp. gram.:V. i.}
\end{itemize}
\begin{itemize}
\item {Proveniência:(Lat. \textunderscore recurrere\textunderscore )}
\end{itemize}
Correr novamente.
Tornar a passar.
Investigar.
Passar de uma linha ou medida para outra (uma composição typográphica).
Soccorrer-se.
Solicitar auxílio ou benevolencia: \textunderscore recorrer á caridade\textunderscore .
Interpor um recurso judicial; appelar.
Fazer uso, aproveitar-se.
\section{Recorrido}
\begin{itemize}
\item {Grp. gram.:m.}
\end{itemize}
Indivíduo, contra quem se interpõe recurso judicial.
\section{Recorrigir}
\begin{itemize}
\item {Grp. gram.:v. t.}
\end{itemize}
\begin{itemize}
\item {Proveniência:(De \textunderscore re...\textunderscore  + \textunderscore corrigir\textunderscore )}
\end{itemize}
Corrigir novamente.
\section{Recorrível}
\begin{itemize}
\item {Grp. gram.:adj.}
\end{itemize}
\begin{itemize}
\item {Utilização:bras}
\end{itemize}
\begin{itemize}
\item {Utilização:Neol.}
\end{itemize}
De que há recurso ou de que se póde recorrer.
\section{Recortada}
\begin{itemize}
\item {Grp. gram.:f.}
\end{itemize}
\begin{itemize}
\item {Utilização:Bras. do S}
\end{itemize}
Bailado campestre, espécie de fandango.
\section{Recortador}
\begin{itemize}
\item {Grp. gram.:m.}
\end{itemize}
\begin{itemize}
\item {Proveniência:(De \textunderscore recortar\textunderscore )}
\end{itemize}
Aquelle que, nas fábricas de rôlhas, corta a cortiça esquadrada. Cf. \textunderscore Inquér. Industr.\textunderscore , p. II, l. III, 27.
\section{Recortar}
\begin{itemize}
\item {Grp. gram.:v. t.}
\end{itemize}
\begin{itemize}
\item {Grp. gram.:V. p.}
\end{itemize}
\begin{itemize}
\item {Proveniência:(De \textunderscore re...\textunderscore  + \textunderscore cortar\textunderscore )}
\end{itemize}
Fazer recortes em.
Cortar, formando determinada figura.
Entremear.
Apresentar-se ou mostrar-se, imitando desenhos recortados: \textunderscore as montanhas recortam-se no horizonte\textunderscore .
\section{Recorte}
\begin{itemize}
\item {Grp. gram.:m.}
\end{itemize}
Acto ou effeito de recortar.
Desenho, que se obtém, recortando.
Aspecto ou estado dos objectos que parecem recortados.
\section{Recorte}
\begin{itemize}
\item {Grp. gram.:m.}
\end{itemize}
\begin{itemize}
\item {Proveniência:(De \textunderscore re...\textunderscore  + \textunderscore corte\textunderscore )}
\end{itemize}
Acto em que o toireiro se encontra no mesmo ponto com o toiro, quando êste abaixa a cabeça para marrar, e o toireiro sái por caminho differente do do toiro.
\section{Recortilha}
\begin{itemize}
\item {Grp. gram.:f.}
\end{itemize}
\begin{itemize}
\item {Proveniência:(De \textunderscore recortar\textunderscore )}
\end{itemize}
Instrumento para recortes; carretilha.
\section{Recoser}
\begin{itemize}
\item {Grp. gram.:v. t.}
\end{itemize}
\begin{itemize}
\item {Proveniência:(De \textunderscore re...\textunderscore  + \textunderscore coser\textunderscore )}
\end{itemize}
Coser de novo.
Coser muitas vezes.
\section{Recosso}
\begin{itemize}
\item {fónica:cô}
\end{itemize}
\begin{itemize}
\item {Grp. gram.:m.}
\end{itemize}
\begin{itemize}
\item {Utilização:Ant.}
\end{itemize}
Provavelmente o mesmo que \textunderscore recoice\textunderscore . Cf. Fern. Lopes, \textunderscore Chrón. de D. Fern.\textunderscore , c. 125.
\section{Recostar}
\begin{itemize}
\item {Grp. gram.:v. t.}
\end{itemize}
\begin{itemize}
\item {Proveniência:(De \textunderscore re...\textunderscore  + \textunderscore costa\textunderscore )}
\end{itemize}
Inclinar, reclinar, encostar.
\section{Recôsto}
\begin{itemize}
\item {Grp. gram.:m.}
\end{itemize}
\begin{itemize}
\item {Utilização:Ant.}
\end{itemize}
\begin{itemize}
\item {Proveniência:(De \textunderscore recostar\textunderscore )}
\end{itemize}
Reclinatório.
Lugar, próprio para alguém se recostar ou descansar.
Encosta.
\section{Récova}
\begin{itemize}
\item {Grp. gram.:f.}
\end{itemize}
\begin{itemize}
\item {Utilização:Ant.}
\end{itemize}
\begin{itemize}
\item {Proveniência:(Do ár. \textunderscore racuba\textunderscore )}
\end{itemize}
O mesmo que \textunderscore récua\textunderscore .
Comitiva de homens a cavallo. Cf. Tenreiro, \textunderscore Itiner.\textunderscore , LIII, 392.
\section{Recóva}
\begin{itemize}
\item {Grp. gram.:f.}
\end{itemize}
\begin{itemize}
\item {Proveniência:(De \textunderscore recovar\textunderscore )}
\end{itemize}
O mesmo que \textunderscore recovagem\textunderscore .
\section{Recovado}
\begin{itemize}
\item {Grp. gram.:m.}
\end{itemize}
\begin{itemize}
\item {Proveniência:(De \textunderscore recovar\textunderscore )}
\end{itemize}
Recôvo.
\section{Recovagem}
\begin{itemize}
\item {Grp. gram.:f.}
\end{itemize}
\begin{itemize}
\item {Proveniência:(De \textunderscore recovar\textunderscore )}
\end{itemize}
Carga, conduzida por almocreves ou recoveiros.
Companhia ou sociedade, que se encarrega do transporte de mercadorias, bagagens, etc.
Preço dêsse transporte.
Contrato, em que algém se obriga ao dito transporte.
Conjunto das pessôas, que acompanham e guardam as bagagens de um exército, e que não entram em combate.
\section{Recovar}
\begin{itemize}
\item {Grp. gram.:v. t.}
\end{itemize}
\begin{itemize}
\item {Proveniência:(De \textunderscore récova\textunderscore )}
\end{itemize}
Transportar de um lugar para outro (bagagens, mercadorias, etc.).
\section{Recoveira}
\begin{itemize}
\item {Grp. gram.:f.}
\end{itemize}
Pau, que os peixeiros levam ao ombro e do qual suspendem os cabazes.
Mulhér, que recova.
(Fem. de \textunderscore recoveiro\textunderscore )
\section{Recoveiro}
\begin{itemize}
\item {Grp. gram.:m.}
\end{itemize}
\begin{itemize}
\item {Proveniência:(De \textunderscore recovar\textunderscore )}
\end{itemize}
Aquelle que recova.
Almocreve.
Espécie de jôgo de cartas.
\section{Recôvo}
\begin{itemize}
\item {Grp. gram.:m.}
\end{itemize}
\begin{itemize}
\item {Utilização:Des.}
\end{itemize}
\begin{itemize}
\item {Proveniência:(Do lat. \textunderscore recubitus\textunderscore )}
\end{itemize}
Acto de estar recostado sôbre o cotovelo.
O mesmo que \textunderscore recúbito\textunderscore .
\section{Recozer}
\begin{itemize}
\item {Grp. gram.:v. t.}
\end{itemize}
\begin{itemize}
\item {Proveniência:(Do lat. \textunderscore recocere\textunderscore )}
\end{itemize}
Cozer novamente; cozer bem ou muito.
Tirar a têmpera a (metaes).
\section{Recozimento}
\begin{itemize}
\item {Grp. gram.:m.}
\end{itemize}
Acto ou effeito de recozer.
\section{Recramo}
\begin{itemize}
\item {Grp. gram.:m.}
\end{itemize}
(Fórma ant. de \textunderscore reclamo\textunderscore )
\section{Recrava}
\begin{itemize}
\item {Grp. gram.:f.}
\end{itemize}
\begin{itemize}
\item {Proveniência:(De \textunderscore recravar\textunderscore )}
\end{itemize}
Encaixe, na cantaria de um portal, para ali se embeberem as peças ou caixilhos em que se firmam as portas.
\section{Recravar}
\begin{itemize}
\item {Grp. gram.:v. t.}
\end{itemize}
\begin{itemize}
\item {Proveniência:(De \textunderscore re...\textunderscore  + \textunderscore cravar\textunderscore )}
\end{itemize}
Cravar muito; cravar novamente.
\section{Recreação}
\begin{itemize}
\item {Grp. gram.:f.}
\end{itemize}
\begin{itemize}
\item {Proveniência:(Lat. \textunderscore recreatio\textunderscore )}
\end{itemize}
O mesmo que \textunderscore recreio\textunderscore .
\section{Recreador}
\begin{itemize}
\item {Grp. gram.:adj.}
\end{itemize}
\begin{itemize}
\item {Proveniência:(Lat. \textunderscore recreator\textunderscore )}
\end{itemize}
Que recreia.
\section{Recrear}
\begin{itemize}
\item {Grp. gram.:v. t.}
\end{itemize}
\begin{itemize}
\item {Grp. gram.:V. p.}
\end{itemize}
\begin{itemize}
\item {Proveniência:(Lat. \textunderscore recreare\textunderscore )}
\end{itemize}
Proporcionar recreio a.
Alegrar; divertir.
O mesmo que \textunderscore recriar\textunderscore .
Sentir prazer ou satisfação.
Divertir-se; folgar, brincar.
\section{Recreativo}
\begin{itemize}
\item {Grp. gram.:adj.}
\end{itemize}
Que recreia; próprio para recrear.
\section{Recreável}
\begin{itemize}
\item {Grp. gram.:adj.}
\end{itemize}
\begin{itemize}
\item {Proveniência:(De \textunderscore recrear\textunderscore )}
\end{itemize}
O mesmo que \textunderscore recreativo\textunderscore ; aprazível. Cf. P. Carvalho, \textunderscore Corogr. Port.\textunderscore , III, 334.
\section{Recredencial}
\begin{itemize}
\item {Grp. gram.:f.}
\end{itemize}
\begin{itemize}
\item {Proveniência:(De \textunderscore re...\textunderscore  + \textunderscore credencial\textunderscore )}
\end{itemize}
Diploma, que os diplomatas, ao terminar as suas suas funcções, trocam com os Govêrnos, junto dos quaes têm estado acreditado.
\section{Recreio}
\begin{itemize}
\item {Grp. gram.:m.}
\end{itemize}
\begin{itemize}
\item {Proveniência:(De \textunderscore recrear\textunderscore )}
\end{itemize}
Divertimento; entretenimento; prazer.
Lugar, onde alguém se recreia.
Coisas, que recreiam.
\section{Recrementício}
\begin{itemize}
\item {Grp. gram.:adj.}
\end{itemize}
\begin{itemize}
\item {Proveniência:(De \textunderscore recremento\textunderscore )}
\end{itemize}
Diz-se das secreções que são novamente absorvidas, como a saliva.
\section{Recremento}
\begin{itemize}
\item {Grp. gram.:m.}
\end{itemize}
\begin{itemize}
\item {Proveniência:(Lat. \textunderscore recrementum\textunderscore )}
\end{itemize}
Secreção recrementícia.
\section{Recrescência}
\begin{itemize}
\item {Grp. gram.:f.}
\end{itemize}
O mesmo que \textunderscore recrescimento\textunderscore ; estado ou qualidade de recrescente.
\section{Recrescente}
\begin{itemize}
\item {Grp. gram.:adj.}
\end{itemize}
\begin{itemize}
\item {Proveniência:(Lat. \textunderscore recrescens\textunderscore )}
\end{itemize}
Que recresce.
\section{Recrescer}
\begin{itemize}
\item {Grp. gram.:v. i.}
\end{itemize}
\begin{itemize}
\item {Grp. gram.:V. p.}
\end{itemize}
\begin{itemize}
\item {Proveniência:(Lat. \textunderscore recrescere\textunderscore )}
\end{itemize}
Tornar a crescer.
Brotar novamente.
Aumentar.
Sobrevir.
Crescer.
Sobejar.
(A mesma signif.). Cf. Rui Barb., \textunderscore Réplica\textunderscore , 159.
\section{Recrescidamente}
\begin{itemize}
\item {Grp. gram.:adv.}
\end{itemize}
De modo recrescido; com aumento.
\section{Recrescido}
\begin{itemize}
\item {Grp. gram.:adj.}
\end{itemize}
Que recresceu, que aumentou.
\section{Recrescimento}
\begin{itemize}
\item {Grp. gram.:m.}
\end{itemize}
Acto ou effeito de recrescer.
\section{Recrestar}
\begin{itemize}
\item {Grp. gram.:v. t.}
\end{itemize}
\begin{itemize}
\item {Proveniência:(De \textunderscore re...\textunderscore  + \textunderscore crestar\textunderscore )}
\end{itemize}
Crestar novamente; crestar demasiadamente; requeimar.
\section{Recriação}
\begin{itemize}
\item {Grp. gram.:f.}
\end{itemize}
Acto de recriar.
\section{Recriar}
\begin{itemize}
\item {Grp. gram.:v. t.}
\end{itemize}
\begin{itemize}
\item {Proveniência:(De \textunderscore re...\textunderscore  + \textunderscore criar\textunderscore )}
\end{itemize}
Tornar a criar; criar novamente. Cf. Baganha, \textunderscore Hyg. Pec.\textunderscore , 142 e 208.
\section{Recouce}
\begin{itemize}
\item {Grp. gram.:m.}
\end{itemize}
\begin{itemize}
\item {Utilização:Des.}
\end{itemize}
\begin{itemize}
\item {Proveniência:(De \textunderscore re...\textunderscore  + \textunderscore couce\textunderscore )}
\end{itemize}
Acto de recuar.
\section{Recriminação}
\begin{itemize}
\item {Grp. gram.:f.}
\end{itemize}
Acto ou effeito de recriminar.
\section{Recriminador}
\begin{itemize}
\item {Grp. gram.:m.  e  adj.}
\end{itemize}
O que recrimina.
\section{Recriminar}
\begin{itemize}
\item {Grp. gram.:v. t.}
\end{itemize}
\begin{itemize}
\item {Proveniência:(De \textunderscore re...\textunderscore  + \textunderscore criminar\textunderscore )}
\end{itemize}
Accusar ou censurar (a quem censura ou accusa).
Reconvir.
\section{Recriminatório}
\begin{itemize}
\item {Grp. gram.:adj.}
\end{itemize}
\begin{itemize}
\item {Proveniência:(De \textunderscore recriminar\textunderscore )}
\end{itemize}
Que contém recriminação.
\section{Recru}
\begin{itemize}
\item {Grp. gram.:adj.}
\end{itemize}
\begin{itemize}
\item {Proveniência:(De \textunderscore re...\textunderscore  + \textunderscore cru\textunderscore )}
\end{itemize}
Muito cru; mal recozido.
\section{Recrudescência}
\begin{itemize}
\item {Grp. gram.:f.}
\end{itemize}
Qualidade do que é recrudescente.
\section{Recrudescente}
\begin{itemize}
\item {Grp. gram.:adj.}
\end{itemize}
\begin{itemize}
\item {Proveniência:(Lat. \textunderscore recrudescens\textunderscore )}
\end{itemize}
Que recrudesce.
\section{Recrudescer}
\begin{itemize}
\item {Grp. gram.:v. i.}
\end{itemize}
\begin{itemize}
\item {Proveniência:(Lat. \textunderscore recrudescere\textunderscore )}
\end{itemize}
Aggravar-se; exacerbar-se.
Recrescer; aumentar.
\section{Recrudescimento}
\begin{itemize}
\item {Grp. gram.:m.}
\end{itemize}
Acto ou effeito de \textunderscore recrudescer\textunderscore ; \textunderscore recrudescência\textunderscore . Cf. Júl. Lour. Pinto, \textunderscore Senh. Deput.\textunderscore , 311.
\section{Recruta}
\begin{itemize}
\item {Grp. gram.:m.}
\end{itemize}
\begin{itemize}
\item {Utilização:Fig.}
\end{itemize}
\begin{itemize}
\item {Grp. gram.:F.}
\end{itemize}
\begin{itemize}
\item {Utilização:Bras. do S}
\end{itemize}
\begin{itemize}
\item {Proveniência:(De \textunderscore recrutar\textunderscore )}
\end{itemize}
Soldado, que assentou praça há pouco tempo e ainda não está perito nos exercícios militares.
Novato, pessôa admittida recentemente num grêmio.
Conjunto de recrutas; exercício militar de recrutas: \textunderscore foi assistir á recruta\textunderscore .

Comitiva de peões, que andam, de estância a estância, arrebanhando os gados de uma fazenda.
Porção de gado arrebanhado.
\section{Recrutador}
\begin{itemize}
\item {Grp. gram.:m.}
\end{itemize}
\begin{itemize}
\item {Utilização:Bras. do S}
\end{itemize}
\begin{itemize}
\item {Proveniência:(De \textunderscore recrutar\textunderscore )}
\end{itemize}
Aquelle que recruta ou arrebanha animaes tresmalhados ou perdidos.
\section{Recrutamento}
\begin{itemize}
\item {Grp. gram.:m.}
\end{itemize}
Acto ou effeito de recrutar: \textunderscore as leis do recrutamento\textunderscore .
Conjunto de recrutas.
\section{Recrutar}
\begin{itemize}
\item {Grp. gram.:v. t.}
\end{itemize}
\begin{itemize}
\item {Utilização:Fig.}
\end{itemize}
\begin{itemize}
\item {Utilização:Bras. do S}
\end{itemize}
\begin{itemize}
\item {Proveniência:(Fr. \textunderscore recruter\textunderscore )}
\end{itemize}
Arrolar para o serviço militar.
Angariar para a vida militar.
Alliciar ou conseguir (adeptos, prosélytos, etc.).
Arrebanhar gado disperso.
\section{Recruzar}
\begin{itemize}
\item {Grp. gram.:v. t.}
\end{itemize}
\begin{itemize}
\item {Proveniência:(De \textunderscore re...\textunderscore  + \textunderscore cruzar\textunderscore )}
\end{itemize}
Tornar a cruzar; cruzar muitas vezes. Cf. Junqueiro, \textunderscore D. João\textunderscore , 400.
\section{Recruzetado}
\begin{itemize}
\item {Grp. gram.:adj.}
\end{itemize}
\begin{itemize}
\item {Utilização:Heráld.}
\end{itemize}
\begin{itemize}
\item {Proveniência:(De \textunderscore re...\textunderscore  + \textunderscore cruzeta\textunderscore )}
\end{itemize}
Que tem cruzetas.
Diz-se de uma cruz, cada uma de cujas hastes é, por sua vez, limitada por cruz.
Cada uma cruz.
\section{Recta}
\begin{itemize}
\item {Grp. gram.:f.}
\end{itemize}
\begin{itemize}
\item {Proveniência:(De \textunderscore recto\textunderscore )}
\end{itemize}
Linha recta; traço direito.
\section{Rectal}
\begin{itemize}
\item {Grp. gram.:adj.}
\end{itemize}
Relativo ao recto ou á extremidade do intestino grosso.
\section{Rectamente}
\begin{itemize}
\item {Grp. gram.:adv.}
\end{itemize}
De modo recto.
Com rectidão; honradamente.
\section{Rectangular}
\begin{itemize}
\item {Grp. gram.:adj.}
\end{itemize}
Semelhante a um rectângulo ou que tem a fórma dêlle.
\section{Rectangularidade}
\begin{itemize}
\item {Grp. gram.:f.}
\end{itemize}
Qualidade do que é rectangular.
\section{Rectângulo}
\begin{itemize}
\item {Grp. gram.:adj.}
\end{itemize}
\begin{itemize}
\item {Grp. gram.:M.}
\end{itemize}
\begin{itemize}
\item {Proveniência:(De \textunderscore recto\textunderscore  + \textunderscore ângulo\textunderscore )}
\end{itemize}
Que tem ângulos rectos.
Parallelogrammo, com ângulos rectos.
\section{Rèctèvere}
\begin{itemize}
\item {Grp. gram.:adj.}
\end{itemize}
\begin{itemize}
\item {Utilização:Prov.}
\end{itemize}
\begin{itemize}
\item {Proveniência:(Do lat. \textunderscore recte\textunderscore  + \textunderscore vere\textunderscore )}
\end{itemize}
Com rectidão, com justiça; exactamente.
\section{Recticórneo}
\begin{itemize}
\item {Grp. gram.:adj.}
\end{itemize}
\begin{itemize}
\item {Utilização:Zool.}
\end{itemize}
\begin{itemize}
\item {Proveniência:(De \textunderscore recto\textunderscore  + \textunderscore córneo\textunderscore )}
\end{itemize}
Que tem as antennas rectas.
\section{Rectidão}
\begin{itemize}
\item {Grp. gram.:m.}
\end{itemize}
\begin{itemize}
\item {Proveniência:(Do lat. \textunderscore rectitudo\textunderscore )}
\end{itemize}
Qualidade do que é recto.
Justiça; legalidade.
Integridade de carácter.
Lisura de procedimento.
\section{Rectificação}
\begin{itemize}
\item {Grp. gram.:f.}
\end{itemize}
Acto ou effeito de rectificar.
\section{Rectificador}
\begin{itemize}
\item {Grp. gram.:adj.}
\end{itemize}
\begin{itemize}
\item {Grp. gram.:M.}
\end{itemize}
\begin{itemize}
\item {Proveniência:(De \textunderscore rectificar\textunderscore )}
\end{itemize}
Que rectifica.
Apparelho, para rectificar liquidos.
\section{Rectificar}
\begin{itemize}
\item {Grp. gram.:v. t.}
\end{itemize}
\begin{itemize}
\item {Proveniência:(Lat. \textunderscore rectificare\textunderscore )}
\end{itemize}
Tornar recto.
Dispor em linha.
Corrigir; purificar, destillando de novo: \textunderscore rectificar álcool\textunderscore .
\section{Rectificativo}
\begin{itemize}
\item {Grp. gram.:adj.}
\end{itemize}
Que rectifica.
\section{Rectificável}
\begin{itemize}
\item {Grp. gram.:adj.}
\end{itemize}
Que se póde rectificar.
\section{Rectifloro}
\begin{itemize}
\item {Grp. gram.:adj.}
\end{itemize}
\begin{itemize}
\item {Utilização:Bot.}
\end{itemize}
\begin{itemize}
\item {Proveniência:(Do lat. \textunderscore rectus\textunderscore  + \textunderscore flos\textunderscore , \textunderscore floris\textunderscore )}
\end{itemize}
Que tem flôres direitas.
\section{Rectiforme}
\begin{itemize}
\item {Grp. gram.:adj.}
\end{itemize}
\begin{itemize}
\item {Proveniência:(Do lat. \textunderscore rectus\textunderscore  + \textunderscore forma\textunderscore )}
\end{itemize}
Que tem fórma direita.
\section{Rectígrado}
\begin{itemize}
\item {Grp. gram.:adj.}
\end{itemize}
\begin{itemize}
\item {Utilização:Zool.}
\end{itemize}
\begin{itemize}
\item {Proveniência:(Do lat. \textunderscore rectus\textunderscore  + \textunderscore gradus\textunderscore )}
\end{itemize}
Que se move ou anda em linha recta.
\section{Rectilíneo}
\begin{itemize}
\item {Grp. gram.:adj.}
\end{itemize}
\begin{itemize}
\item {Proveniência:(Lat. \textunderscore rectilineus\textunderscore )}
\end{itemize}
Que está em linha recta; que tem fórma de linha recta; formado de linhas rectas.
\section{Rectinérveo}
\begin{itemize}
\item {Grp. gram.:adj.}
\end{itemize}
\begin{itemize}
\item {Utilização:Bot.}
\end{itemize}
\begin{itemize}
\item {Proveniência:(Do lat. \textunderscore rectus\textunderscore  + \textunderscore nervus\textunderscore )}
\end{itemize}
Que tem nervuras direitas.
\section{Rectirostro}
\begin{itemize}
\item {fónica:rós}
\end{itemize}
\begin{itemize}
\item {Grp. gram.:adj.}
\end{itemize}
\begin{itemize}
\item {Utilização:Zool.}
\end{itemize}
\begin{itemize}
\item {Proveniência:(Do lat. \textunderscore rectus\textunderscore  + \textunderscore rostrum\textunderscore )}
\end{itemize}
Que tem o bico direito.
\section{Rectirrostro}
\begin{itemize}
\item {Grp. gram.:adj.}
\end{itemize}
\begin{itemize}
\item {Utilização:Zool.}
\end{itemize}
\begin{itemize}
\item {Proveniência:(Do lat. \textunderscore rectus\textunderscore  + \textunderscore rostrum\textunderscore )}
\end{itemize}
Que tem o bico direito.
\section{Rectite}
\begin{itemize}
\item {Grp. gram.:f.}
\end{itemize}
\begin{itemize}
\item {Proveniência:(De \textunderscore recto\textunderscore )}
\end{itemize}
Inflammação do intestino recto.
\section{Rectitude}
\begin{itemize}
\item {Grp. gram.:f.}
\end{itemize}
(V.rectidão)
\section{Recto}
\begin{itemize}
\item {Grp. gram.:adj.}
\end{itemize}
\begin{itemize}
\item {Utilização:Fig.}
\end{itemize}
\begin{itemize}
\item {Utilização:Geom.}
\end{itemize}
\begin{itemize}
\item {Grp. gram.:M.}
\end{itemize}
\begin{itemize}
\item {Proveniência:(Lat. \textunderscore rectus\textunderscore )}
\end{itemize}
Que se não inclina para parte nenhuma: \textunderscore linha recta\textunderscore .
Direito.
Vertical.
Que tem sentimentos de justiça; integro; honesto.
Verdadeiro.
Formado de linhas perpendiculares entre si, (falando se de ângulos).
Extremidade do intestino grosso.
\section{Rector}
\begin{itemize}
\item {Grp. gram.:m.}
\end{itemize}
\begin{itemize}
\item {Utilização:Ant.}
\end{itemize}
\begin{itemize}
\item {Proveniência:(Lat. \textunderscore rector\textunderscore )}
\end{itemize}
O mesmo que \textunderscore reitor\textunderscore .
Aquelle que rege ou governa:«\textunderscore ...o Clementíssimo Rector do Universo...\textunderscore »Filinto, \textunderscore D. Man.\textunderscore , I, 155.
\section{Rectro-uretral}
\begin{itemize}
\item {Grp. gram.:adj.}
\end{itemize}
Relativo ao recto e á uretra.
\section{Rectro-vaginal}
\begin{itemize}
\item {Grp. gram.:adj.}
\end{itemize}
Relativo ao recto e á vagina.
\section{Recto-vesical}
\begin{itemize}
\item {Grp. gram.:adj.}
\end{itemize}
Relativo ao recto e á bexiga.
\section{Rectriz}
\begin{itemize}
\item {Grp. gram.:f.}
\end{itemize}
\begin{itemize}
\item {Proveniência:(Lat. \textunderscore rectrix\textunderscore )}
\end{itemize}
Cada da uma das pennas da cauda das aves, próprias para dirigirem o vôo.
\section{Recúa}
\begin{itemize}
\item {Grp. gram.:f.}
\end{itemize}
O mesmo que \textunderscore recúo\textunderscore .
\section{Récua}
\begin{itemize}
\item {Grp. gram.:f.}
\end{itemize}
\begin{itemize}
\item {Utilização:Deprec.}
\end{itemize}
Ajuntamento de bêstas de carga, geralmente presas umas ás outras.
Manada de cavalgaduras.
A carga, que ellas transportam.
Súcia; caterva.
(Fórma divergente de \textunderscore récova\textunderscore )
\section{Recuada}
\begin{itemize}
\item {Grp. gram.:f.}
\end{itemize}
O mesmo que \textunderscore recúo\textunderscore .
\section{Recuadeira}
\begin{itemize}
\item {Grp. gram.:f.}
\end{itemize}
Correia que, ligada á parte anterior do varal, servia para fazer recuar as seges.
\section{Recuamento}
\begin{itemize}
\item {Grp. gram.:m.}
\end{itemize}
O mesmo que \textunderscore recúo\textunderscore .
\section{Recuanço}
\begin{itemize}
\item {Grp. gram.:m.}
\end{itemize}
\begin{itemize}
\item {Utilização:Pop.}
\end{itemize}
\begin{itemize}
\item {Proveniência:(De \textunderscore recuar\textunderscore )}
\end{itemize}
Acto de fazer recuar uma bola, no jôgo do bilhar.
O mesmo que \textunderscore recúo\textunderscore :«\textunderscore a córte saíu em recuanços.\textunderscore »Camillo, \textunderscore Brasileira\textunderscore , 151.
\section{Recuão}
\begin{itemize}
\item {Grp. gram.:m.}
\end{itemize}
Acto de recuar com fôrça ou violência. Cf. Eça, \textunderscore P. Amaro\textunderscore , 261.
\section{Recuar}
\begin{itemize}
\item {Grp. gram.:v. i.}
\end{itemize}
\begin{itemize}
\item {Utilização:Fig.}
\end{itemize}
\begin{itemize}
\item {Grp. gram.:V. t.}
\end{itemize}
\begin{itemize}
\item {Proveniência:(Do lat. hyp. \textunderscore reculare\textunderscore )}
\end{itemize}
Mover-se para trás; andar para trás.
Referir-se a um acontecimento atrasado ou já referido.
Atrasar-se.
Encurtar-se, encolher-se.
Perder terreno.
Acuar; fugir.
Têr ideias oppostas ao progresso.
Fazer andar para trás; lançar para trás.
\section{Recúbito}
\begin{itemize}
\item {Grp. gram.:m.}
\end{itemize}
\begin{itemize}
\item {Proveniência:(Lat. \textunderscore recubitus\textunderscore )}
\end{itemize}
Acto de encostar-se.
Posição de quem está encostado.
\section{Recudar}
\textunderscore v. t.\textunderscore  (e der.) \textunderscore Ant.\textunderscore 
O mesmo que \textunderscore recusar\textunderscore , etc. Cf. S. R. Viterbo, \textunderscore Elucidário\textunderscore .
\section{Recudir}
\begin{itemize}
\item {Grp. gram.:v. i.}
\end{itemize}
\begin{itemize}
\item {Utilização:Ant.}
\end{itemize}
\begin{itemize}
\item {Proveniência:(De \textunderscore re...\textunderscore  + \textunderscore acudir\textunderscore )}
\end{itemize}
Voltar para acudir; sair para serviço. Cf. \textunderscore Port. Mon. Hist.\textunderscore , \textunderscore Script.\textunderscore , 317.
Fugir. Cf. \textunderscore Port. Mon. Hist.\textunderscore , \textunderscore Script.\textunderscore , 259.
\section{Recuidar}
\begin{itemize}
\item {Grp. gram.:v. i.}
\end{itemize}
\begin{itemize}
\item {Proveniência:(De \textunderscore re...\textunderscore  + \textunderscore cuidar\textunderscore )}
\end{itemize}
Cuidar ou pensar muito.
\section{Recuitar}
\begin{itemize}
\item {Proveniência:(De \textunderscore re...\textunderscore  + lat. \textunderscore coctus\textunderscore )}
\end{itemize}
\textunderscore v. t.\textunderscore  (e der.)
O mesmo que \textunderscore recoitar\textunderscore . Cf. F. de Mendonça, \textunderscore Vocab. Techn.\textunderscore 
\section{Recúla}
\begin{itemize}
\item {Grp. gram.:adj. f.}
\end{itemize}
\begin{itemize}
\item {Utilização:Prov.}
\end{itemize}
\begin{itemize}
\item {Utilização:trasm.}
\end{itemize}
\begin{itemize}
\item {Grp. gram.:F.}
\end{itemize}
\begin{itemize}
\item {Utilização:Prov.}
\end{itemize}
\begin{itemize}
\item {Utilização:trasm.}
\end{itemize}
Diz-se da gallinha sem rabo.
O mesmo que \textunderscore récua\textunderscore , multidão, bando.
\section{Reçumar}
\begin{itemize}
\item {Grp. gram.:v. i.}
\end{itemize}
O mesmo ou melhor que \textunderscore ressumar\textunderscore .
(Cp. cast. \textunderscore rezumar\textunderscore )
\section{Recumbente}
\begin{itemize}
\item {Grp. gram.:adj.}
\end{itemize}
\begin{itemize}
\item {Proveniência:(Lat. \textunderscore recumbens\textunderscore )}
\end{itemize}
Que recumbe. Cf. Garrett, \textunderscore Helena\textunderscore , 53.
\section{Recumbir}
\begin{itemize}
\item {Grp. gram.:v. i.}
\end{itemize}
\begin{itemize}
\item {Utilização:Des.}
\end{itemize}
\begin{itemize}
\item {Proveniência:(Lat. \textunderscore recumbere\textunderscore )}
\end{itemize}
Estar encostado.
\section{Recunhar}
\begin{itemize}
\item {Grp. gram.:v. t.}
\end{itemize}
\begin{itemize}
\item {Proveniência:(De \textunderscore re...\textunderscore  + \textunderscore cunhar\textunderscore )}
\end{itemize}
Cunhar novamente.
\section{Recúo}
\begin{itemize}
\item {Grp. gram.:m.}
\end{itemize}
Acto ou effeito de recuar.
\section{Recuperação}
\begin{itemize}
\item {Grp. gram.:f.}
\end{itemize}
\begin{itemize}
\item {Proveniência:(Lat. \textunderscore recuperatio\textunderscore )}
\end{itemize}
Acto ou effeito de recuperar.
\section{Recuperador}
\begin{itemize}
\item {Grp. gram.:m.  e  adj.}
\end{itemize}
\begin{itemize}
\item {Proveniência:(Lat. \textunderscore recuperator\textunderscore )}
\end{itemize}
O que recupera.
\section{Recuperar}
\begin{itemize}
\item {Grp. gram.:v. t.}
\end{itemize}
\begin{itemize}
\item {Grp. gram.:V. p.}
\end{itemize}
\begin{itemize}
\item {Utilização:P. us.}
\end{itemize}
\begin{itemize}
\item {Proveniência:(Lat. \textunderscore recuperare\textunderscore )}
\end{itemize}
O mesmo que \textunderscore recobrar\textunderscore .
Adquirir novamente.
Sêr indemnizado ou resarcido; refazer-se. Cf. Pant. de Aveiro, \textunderscore Itiner.\textunderscore , 62 v.^o, (2.^a ed.).
\section{Recuperativo}
\begin{itemize}
\item {Grp. gram.:adj.}
\end{itemize}
\begin{itemize}
\item {Proveniência:(Lat. \textunderscore recuperativus\textunderscore )}
\end{itemize}
Que recupera.
\section{Recuperatório}
\begin{itemize}
\item {Grp. gram.:adj.}
\end{itemize}
\begin{itemize}
\item {Proveniência:(Lat. \textunderscore recuperatorius\textunderscore )}
\end{itemize}
Dizia-se dos mandados judiciaes, para que um acto voltasse ao primitivo estado.
\section{Recuperável}
\begin{itemize}
\item {Grp. gram.:adj.}
\end{itemize}
Que se póde recuperar.
\section{Recurção}
\begin{itemize}
\item {Grp. gram.:f.}
\end{itemize}
\begin{itemize}
\item {Utilização:Ant.}
\end{itemize}
Distrito; termo, freguesia. Cf. Viterbo, \textunderscore Elucid.\textunderscore 
(Cp. \textunderscore recorreição\textunderscore )
\section{Recurso}
\begin{itemize}
\item {Grp. gram.:m.}
\end{itemize}
\begin{itemize}
\item {Grp. gram.:Pl.}
\end{itemize}
\begin{itemize}
\item {Proveniência:(Lat. \textunderscore recursus\textunderscore )}
\end{itemize}
Acto ou effeito de recorrer.
Coisa, pessôa ou lugar, a que alguém recorre.
Auxílio.
Meio.
Remédio.
Acto de appellar judicialmente.
Reclamação.
Meios pecuniários; meios de vida; haveres: \textunderscore pobre de recursos\textunderscore .
\section{Recurvação}
\begin{itemize}
\item {Grp. gram.:f.}
\end{itemize}
Acto ou effeito de recurvar.
\section{Recurvadamente}
\begin{itemize}
\item {Grp. gram.:adv.}
\end{itemize}
De modo recurvado.
\section{Recurvado}
\begin{itemize}
\item {Grp. gram.:adj.}
\end{itemize}
\begin{itemize}
\item {Proveniência:(De \textunderscore recurvar\textunderscore )}
\end{itemize}
Muito curvado.
Muito inclinado para o chão.
\section{Recurvar}
\begin{itemize}
\item {Grp. gram.:v. t.}
\end{itemize}
\begin{itemize}
\item {Proveniência:(Lat. \textunderscore recurvare\textunderscore )}
\end{itemize}
Curvar novamente; curvar muito; inclinar.
\section{Recurvo}
\begin{itemize}
\item {Grp. gram.:adj.}
\end{itemize}
\begin{itemize}
\item {Proveniência:(Lat. \textunderscore recurvus\textunderscore )}
\end{itemize}
O mesmo que \textunderscore recurvado\textunderscore .
\section{Recusa}
\begin{itemize}
\item {Grp. gram.:f.}
\end{itemize}
Acto ou effeito de recusar.
\section{Recusação}
\begin{itemize}
\item {Grp. gram.:f.}
\end{itemize}
\begin{itemize}
\item {Proveniência:(Lat. \textunderscore recusatio\textunderscore )}
\end{itemize}
O mesmo que \textunderscore recusa\textunderscore .
\section{Recusador}
\begin{itemize}
\item {Grp. gram.:m.  e  adj.}
\end{itemize}
\begin{itemize}
\item {Proveniência:(De \textunderscore recusar\textunderscore )}
\end{itemize}
O que recusa.
\section{Recusante}
\begin{itemize}
\item {Grp. gram.:m. ,  f.  e  adj.}
\end{itemize}
\begin{itemize}
\item {Proveniência:(Lat. \textunderscore recusans\textunderscore )}
\end{itemize}
Pessôa, que recusa.
\section{Recusar}
\begin{itemize}
\item {Grp. gram.:v. t.}
\end{itemize}
\begin{itemize}
\item {Proveniência:(Lat. \textunderscore recusare\textunderscore )}
\end{itemize}
Não acceitar; negar.
Rejeitar.
Evitar.
\section{Recusativo}
\begin{itemize}
\item {Grp. gram.:adj.}
\end{itemize}
\begin{itemize}
\item {Proveniência:(De \textunderscore recusar\textunderscore )}
\end{itemize}
Que envolve recusa; que exprime recusa. Cf. Castilho, \textunderscore Fastos\textunderscore , III, 550.
\section{Recusável}
\begin{itemize}
\item {Grp. gram.:adj.}
\end{itemize}
\begin{itemize}
\item {Proveniência:(Lat. \textunderscore recusabilis\textunderscore )}
\end{itemize}
Que póde ou deve sêr recusado.
\section{Redacção}
\begin{itemize}
\item {Grp. gram.:f.}
\end{itemize}
\begin{itemize}
\item {Proveniência:(Lat. \textunderscore redactio\textunderscore )}
\end{itemize}
Acto ou effeito de redigir; modo de redigir.
Conjunto das pessôas, que redigem um periódico ou qualquer obra literária ou scientífica.
Casa, onde se redige um periódico.
\section{Redactor}
\begin{itemize}
\item {Grp. gram.:m.}
\end{itemize}
\begin{itemize}
\item {Proveniência:(Do lat. \textunderscore redactus\textunderscore )}
\end{itemize}
Aquelle que redige.
Aquelle que escreve habitualmente para um jornal ou jornaes.
\section{Redada}
\begin{itemize}
\item {Grp. gram.:f.}
\end{itemize}
\begin{itemize}
\item {Utilização:Prov.}
\end{itemize}
\begin{itemize}
\item {Utilização:alent.}
\end{itemize}
Acto de redar por uma vez; lanço de rêde.
Espaço, que o rebanho, nos alfirmes, occupa em cada noite.
\section{Redamar}
\begin{itemize}
\item {Grp. gram.:v. t.}
\end{itemize}
\begin{itemize}
\item {Utilização:Des.}
\end{itemize}
\begin{itemize}
\item {Proveniência:(Lat. \textunderscore redamare\textunderscore )}
\end{itemize}
Amar com reciprocidade. Cf. \textunderscore Diccion. Exeg.\textunderscore 
\section{Redanho}
\begin{itemize}
\item {Grp. gram.:m.}
\end{itemize}
\begin{itemize}
\item {Utilização:Anat.}
\end{itemize}
\begin{itemize}
\item {Utilização:Pesc.}
\end{itemize}
\begin{itemize}
\item {Utilização:Prov.}
\end{itemize}
\begin{itemize}
\item {Utilização:minh.}
\end{itemize}
\begin{itemize}
\item {Utilização:Fig.}
\end{itemize}
Dobra grande no peritoneu.
Apparelho de rêde, para a apanhar do sargaço.
Nome de outro apparelho de rêde, para a pesca do camarão.
Gordura, pegada aos intestinos dos animais.
O mesmo que \textunderscore barriga\textunderscore . Cf. Camillo, \textunderscore Cego de Landim\textunderscore , 27.
(Cp. cast. \textunderscore redaño\textunderscore )
\section{Redar}
\begin{itemize}
\item {Grp. gram.:v. t.}
\end{itemize}
Lançar (a rêde).
\section{Redar}
\begin{itemize}
\item {Grp. gram.:v. t.}
\end{itemize}
\begin{itemize}
\item {Proveniência:(De \textunderscore re...\textunderscore  + \textunderscore dar\textunderscore )}
\end{itemize}
Dar novamente.
\section{Redar}
\begin{itemize}
\item {Grp. gram.:v. t.  e  i.}
\end{itemize}
(Corr. de \textunderscore rodar\textunderscore )
\section{Redar}
\textunderscore v. t.\textunderscore  (e der.) \textunderscore Ant.\textunderscore 
O mesmo que \textunderscore redrar\textunderscore , etc.
\section{Redarguente}
\begin{itemize}
\item {fónica:gu-en}
\end{itemize}
\begin{itemize}
\item {Grp. gram.:adj.}
\end{itemize}
Que redargue.
\section{Redarguição}
\begin{itemize}
\item {fónica:gu-i}
\end{itemize}
\begin{itemize}
\item {Grp. gram.:f.}
\end{itemize}
\begin{itemize}
\item {Proveniência:(Do lat. \textunderscore redargutio\textunderscore )}
\end{itemize}
Acto ou effeito de redarguir.
\section{Redarguidor}
\begin{itemize}
\item {fónica:gu-i}
\end{itemize}
\begin{itemize}
\item {Grp. gram.:m.  e  adj.}
\end{itemize}
O que redargúe.
\section{Redarguir}
\begin{itemize}
\item {Grp. gram.:v. t.}
\end{itemize}
\begin{itemize}
\item {Proveniência:(Lat. \textunderscore redarguere\textunderscore )}
\end{itemize}
Reconvir.
Replicar a quem argúe; replicar.
Recriminar.
\section{Redarguitivamente}
\begin{itemize}
\item {fónica:gu-i}
\end{itemize}
\begin{itemize}
\item {Grp. gram.:adv.}
\end{itemize}
De modo redarguitivo.
\section{Redarguitivo}
\begin{itemize}
\item {fónica:gu-i}
\end{itemize}
\begin{itemize}
\item {Grp. gram.:adj.}
\end{itemize}
\begin{itemize}
\item {Proveniência:(De \textunderscore redarguir\textunderscore )}
\end{itemize}
Que envolve redarguição.
\section{Reddição}
\begin{itemize}
\item {Grp. gram.:f.}
\end{itemize}
\begin{itemize}
\item {Proveniência:(Lat. \textunderscore redditio\textunderscore )}
\end{itemize}
Acto de entregar; entrega; restituição. Cf. Latino, \textunderscore Hist. Pol. e Mil.\textunderscore , 14.
\section{Rêde}
\begin{itemize}
\item {Grp. gram.:f.}
\end{itemize}
\begin{itemize}
\item {Utilização:ant.}
\end{itemize}
\begin{itemize}
\item {Utilização:Gír.}
\end{itemize}
\begin{itemize}
\item {Proveniência:(Do lat. \textunderscore rete\textunderscore )}
\end{itemize}
Tecido de malha, para apanhar peixes, aves, feras.
Ligeiro tecido de malha, com que algumas mulheres seguram o cabello.
Qualquer tecido de malha.
Tecido de arame.
Conjunto de caminhos, estradas ou canos, que entroncam uns nos outros, ramificando-se.
Entrelaçamento de nervos, fibras, etc.
Cilada.
Roupa.
\section{Rédea}
\begin{itemize}
\item {Grp. gram.:f.}
\end{itemize}
\begin{itemize}
\item {Utilização:Fig.}
\end{itemize}
\begin{itemize}
\item {Proveniência:(Do b. lat. \textunderscore retena\textunderscore )}
\end{itemize}
Correia que, ligada ao freio de uma cavalgadura, serve para guiar esta.
Poder, direcção.
Govêrno, lei.
\section{Redeal}
\begin{itemize}
\item {Grp. gram.:m.}
\end{itemize}
\begin{itemize}
\item {Utilização:Prov.}
\end{itemize}
\begin{itemize}
\item {Utilização:beir.}
\end{itemize}
\begin{itemize}
\item {Proveniência:(De \textunderscore rédea\textunderscore )}
\end{itemize}
Conjunto de varas de videira, carregadas de uvas.
\section{Redeclaração}
\begin{itemize}
\item {Grp. gram.:f.}
\end{itemize}
Acto de \textunderscore redeclarar\textunderscore .
\section{Redeclarar}
\begin{itemize}
\item {Grp. gram.:v. t.}
\end{itemize}
\begin{itemize}
\item {Proveniência:(De \textunderscore re...\textunderscore  + \textunderscore declarar\textunderscore )}
\end{itemize}
Declarar novamente, declarar outra vez.
\section{Rêde-folle}
\begin{itemize}
\item {Grp. gram.:f.}
\end{itemize}
Rede em fórma de funil.
\section{Redeiro}
\begin{itemize}
\item {Grp. gram.:m.}
\end{itemize}
\begin{itemize}
\item {Utilização:T. de Buarcos}
\end{itemize}
Fabricante de rêdes.
Aquelle que conserta rêdes.
Pequena rêde de um só pano, para emmalhar, usada na pesca fluvial.
\section{Redém}
\begin{itemize}
\item {Grp. gram.:m.}
\end{itemize}
\begin{itemize}
\item {Utilização:Bras. do N}
\end{itemize}
O mesmo que \textunderscore redanho\textunderscore .
\section{Redemoinhar}
\begin{itemize}
\item {fónica:mo-i}
\end{itemize}
\begin{itemize}
\item {Grp. gram.:v. i.}
\end{itemize}
(Corr. de \textunderscore remoinhar\textunderscore )
\section{Redemoínho}
\begin{itemize}
\item {Grp. gram.:m.}
\end{itemize}
(Corr. de \textunderscore remoínho\textunderscore )
\section{Redempção}
\begin{itemize}
\item {Grp. gram.:f.}
\end{itemize}
\begin{itemize}
\item {Utilização:Ant.}
\end{itemize}
\begin{itemize}
\item {Proveniência:(Lat. \textunderscore redemptio\textunderscore )}
\end{itemize}
Acto ou effeito de reunir.
Esmolas, que se davam para remir os cativos.
\section{Redemptor}
\begin{itemize}
\item {Grp. gram.:adj.}
\end{itemize}
\begin{itemize}
\item {Grp. gram.:M.}
\end{itemize}
\begin{itemize}
\item {Utilização:Restrict.}
\end{itemize}
\begin{itemize}
\item {Utilização:Prov.}
\end{itemize}
\begin{itemize}
\item {Utilização:trasm.}
\end{itemize}
\begin{itemize}
\item {Proveniência:(Lat. \textunderscore redemptor\textunderscore )}
\end{itemize}
Que redime.
Aquelle que redime ou redimiu.
Christo.
\textunderscore Meter-se a redemptor\textunderscore , meter-se a dar sentenças naquillo que não sabe, meter-se onde não é chamado.
\section{Redemptorista}
\begin{itemize}
\item {Grp. gram.:m.}
\end{itemize}
\begin{itemize}
\item {Proveniência:(De \textunderscore redemptor\textunderscore )}
\end{itemize}
Freire de uma Ordem religiosa aleman, que entrou em Portugal em 1828 e saiu em 1833.
\section{Redenção}
\begin{itemize}
\item {Grp. gram.:f.}
\end{itemize}
\begin{itemize}
\item {Utilização:Ant.}
\end{itemize}
\begin{itemize}
\item {Proveniência:(Lat. \textunderscore redemptio\textunderscore )}
\end{itemize}
Acto ou efeito de reunir.
Esmolas, que se davam para remir os cativos.
\section{Redenho}
\begin{itemize}
\item {Grp. gram.:m.}
\end{itemize}
(V.redanho)
\section{Redente}
\begin{itemize}
\item {Grp. gram.:m.}
\end{itemize}
\begin{itemize}
\item {Proveniência:(De \textunderscore re...\textunderscore  + \textunderscore dente\textunderscore )}
\end{itemize}
Entrincheiramento, em fórma de ângulo saliente.
Resalto, na parte superior de muros construidos em terreno inclinado; e destinado a deixar niveladas essas construcções.
\section{Redentor}
\begin{itemize}
\item {Grp. gram.:adj.}
\end{itemize}
\begin{itemize}
\item {Grp. gram.:M.}
\end{itemize}
\begin{itemize}
\item {Utilização:Restrict.}
\end{itemize}
\begin{itemize}
\item {Utilização:Prov.}
\end{itemize}
\begin{itemize}
\item {Utilização:trasm.}
\end{itemize}
\begin{itemize}
\item {Proveniência:(Lat. \textunderscore redemptor\textunderscore )}
\end{itemize}
Que redime.
Aquelle que redime ou redimiu.
Christo.
\textunderscore Meter-se a redentor\textunderscore , meter-se a dar sentenças naquillo que não sabe, meter-se onde não é chamado.
\section{Redentorista}
\begin{itemize}
\item {Grp. gram.:m.}
\end{itemize}
\begin{itemize}
\item {Proveniência:(De \textunderscore redemptor\textunderscore )}
\end{itemize}
Freire de uma Ordem religiosa aleman, que entrou em Portugal em 1828 e saiu em 1833.
\section{Rêde-pé}
\begin{itemize}
\item {Grp. gram.:f.}
\end{itemize}
\begin{itemize}
\item {Utilização:Pesc.}
\end{itemize}
Rêde de arrastar para a terra e com que podem trabalhar dois homens.
\section{Redescender}
\begin{itemize}
\item {Grp. gram.:v. i.}
\end{itemize}
\begin{itemize}
\item {Proveniência:(Do lat. \textunderscore re...\textunderscore  + \textunderscore descendere\textunderscore )}
\end{itemize}
Descer novamente:«\textunderscore redescendi e recolhi-me ao tugúrio...\textunderscore »Castilho, \textunderscore Carta\textunderscore  a A. M. Pereira.
\section{Redescer}
\begin{itemize}
\item {Grp. gram.:v. i.}
\end{itemize}
\begin{itemize}
\item {Proveniência:(De \textunderscore re...\textunderscore  + \textunderscore descer\textunderscore )}
\end{itemize}
Descer de novo; redescender. Cf. Castilho, \textunderscore Geórgicas\textunderscore , 73 e 179.
\section{Redescontar}
\begin{itemize}
\item {Grp. gram.:v. t.}
\end{itemize}
\begin{itemize}
\item {Proveniência:(De \textunderscore re...\textunderscore  + \textunderscore descontar\textunderscore )}
\end{itemize}
Fazer redesconto.
\section{Redesconto}
\begin{itemize}
\item {Grp. gram.:m.}
\end{itemize}
\begin{itemize}
\item {Utilização:Jur.}
\end{itemize}
\begin{itemize}
\item {Proveniência:(De \textunderscore re...\textunderscore  + \textunderscore desconto\textunderscore )}
\end{itemize}
Acto de descontar numa praça a letra que já se descontou ao sacador ou portador.
\section{Redestilação}
\begin{itemize}
\item {Grp. gram.:f.}
\end{itemize}
Acto de redestilar.
\section{Redestilar}
\begin{itemize}
\item {Grp. gram.:v. t.}
\end{itemize}
\begin{itemize}
\item {Proveniência:(De \textunderscore re...\textunderscore  + \textunderscore destilar\textunderscore )}
\end{itemize}
Tornar a destilar, destilar de novo. Cf. \textunderscore Techn. Rur.\textunderscore , I, 491.
\section{Redestillação}
\begin{itemize}
\item {Grp. gram.:f.}
\end{itemize}
Acto de redestillar.
\section{Redestillar}
\begin{itemize}
\item {Grp. gram.:v. t.}
\end{itemize}
\begin{itemize}
\item {Proveniência:(De \textunderscore re...\textunderscore  + \textunderscore destillar\textunderscore )}
\end{itemize}
Tornar a destillar, destillar de novo. Cf. \textunderscore Techn. Rur.\textunderscore , I, 491.
\section{Redhibição}
\begin{itemize}
\item {Grp. gram.:f.}
\end{itemize}
\begin{itemize}
\item {Proveniência:(Lat. \textunderscore redhibitio\textunderscore )}
\end{itemize}
Acto ou effeito de redhibir.
\section{Redhibir}
\begin{itemize}
\item {Grp. gram.:v.}
\end{itemize}
\begin{itemize}
\item {Utilização:t. Jur.}
\end{itemize}
\begin{itemize}
\item {Proveniência:(Lat. \textunderscore redhibere\textunderscore )}
\end{itemize}
Tornar sem effeito a venda de.
Vender ao vendedor (objecto que tinha defeitos, não declarados na primeira venda).
\section{Redhibitório}
\begin{itemize}
\item {Grp. gram.:adj.}
\end{itemize}
\begin{itemize}
\item {Utilização:Jur.}
\end{itemize}
\begin{itemize}
\item {Proveniência:(Lat. \textunderscore redhibitorius\textunderscore )}
\end{itemize}
Relativo á redhibição.
Relativo a defeitos ou vícios, que autorizam a redhibição: \textunderscore vícios redhibitórios\textunderscore .
\section{Redibição}
\begin{itemize}
\item {Grp. gram.:f.}
\end{itemize}
\begin{itemize}
\item {Proveniência:(Lat. \textunderscore redhibitio\textunderscore )}
\end{itemize}
Acto ou efeito de redibir.
\section{Redibitório}
\begin{itemize}
\item {Grp. gram.:adj.}
\end{itemize}
\begin{itemize}
\item {Utilização:Jur.}
\end{itemize}
\begin{itemize}
\item {Proveniência:(Lat. \textunderscore redhibitorius\textunderscore )}
\end{itemize}
Relativo á redibição.
Relativo a defeitos ou vícios, que autorizam a redibição: \textunderscore vícios redibitórios\textunderscore .
\section{Redigir}
\begin{itemize}
\item {Grp. gram.:v. t.}
\end{itemize}
\begin{itemize}
\item {Proveniência:(Lat. \textunderscore redigere\textunderscore )}
\end{itemize}
Exprimir methodicamente por escrito: \textunderscore redigir uma carta\textunderscore .
Escrever os artigos principaes de (um periódico).
Escrever para a imprensa periódica.
\section{Redil}
\begin{itemize}
\item {Grp. gram.:m.}
\end{itemize}
\begin{itemize}
\item {Utilização:Fig.}
\end{itemize}
\begin{itemize}
\item {Proveniência:(Do lat. \textunderscore retile\textunderscore ?)}
\end{itemize}
O mesmo que \textunderscore curral\textunderscore .
Grêmio.
\section{Redimento}
\begin{itemize}
\item {Grp. gram.:m.}
\end{itemize}
\begin{itemize}
\item {Utilização:Ant.}
\end{itemize}
\begin{itemize}
\item {Proveniência:(De \textunderscore redimir\textunderscore )}
\end{itemize}
O mesmo que \textunderscore redempção\textunderscore . Cf. S. R. Viterbo, \textunderscore Elucidário\textunderscore .
\section{Redimir}
\begin{itemize}
\item {Grp. gram.:v. t.}
\end{itemize}
\begin{itemize}
\item {Proveniência:(Lat. \textunderscore redimere\textunderscore )}
\end{itemize}
O mesmo que \textunderscore remir\textunderscore .
\section{Redimível}
\begin{itemize}
\item {Grp. gram.:adj.}
\end{itemize}
\begin{itemize}
\item {Proveniência:(De \textunderscore redimir\textunderscore )}
\end{itemize}
Que póde ou deve sêr remido.
\section{Redingote}
\begin{itemize}
\item {Grp. gram.:m.}
\end{itemize}
\begin{itemize}
\item {Proveniência:(Fr. \textunderscore redingot\textunderscore )}
\end{itemize}
Casaco largo e comprido, que tem inteiriças as peças da frente; sobrecasaca.
\section{Redintegrar}
\textunderscore v. t.\textunderscore  (e der.)
O mesmo que \textunderscore reintegrar\textunderscore , etc. Cf. Castilho, \textunderscore Fastos\textunderscore , III, 171.
\section{Redissolver}
\begin{itemize}
\item {Grp. gram.:v. t.}
\end{itemize}
\begin{itemize}
\item {Proveniência:(De \textunderscore re...\textunderscore  + \textunderscore dissolver\textunderscore )}
\end{itemize}
Dissolver de novo; tornar a dissolver. Cf. \textunderscore Techn. Rur.\textunderscore  I, 382.
\section{Redistribuir}
\begin{itemize}
\item {Grp. gram.:v. t.}
\end{itemize}
Distribuir novamente. Cf. João Ribeiro, \textunderscore Gram.\textunderscore 
\section{Rédito}
\begin{itemize}
\item {Grp. gram.:m.}
\end{itemize}
\begin{itemize}
\item {Proveniência:(Lat. \textunderscore reditus\textunderscore )}
\end{itemize}
Acto de voltar; volta.
Rendimento.
Lucro; producto: \textunderscore os réditos de uma Empresa\textunderscore .
Juro.
\section{Redivinizar}
\begin{itemize}
\item {Grp. gram.:v. t.}
\end{itemize}
\begin{itemize}
\item {Proveniência:(De \textunderscore re...\textunderscore  + \textunderscore divinizar\textunderscore )}
\end{itemize}
Tornar mais divino, mais sagrado:«\textunderscore ...reproduzir divinizando o Olympo\textunderscore ». Castilho, \textunderscore Metam.\textunderscore , 290.
\section{Redivivo}
\begin{itemize}
\item {Grp. gram.:adj.}
\end{itemize}
\begin{itemize}
\item {Proveniência:(Lat. \textunderscore redivivus\textunderscore )}
\end{itemize}
Que voltou á vida; resuscitado.
Renovado; que remoçou.
\section{Redizer}
\begin{itemize}
\item {Grp. gram.:v. t.}
\end{itemize}
\begin{itemize}
\item {Proveniência:(Lat. \textunderscore redicere\textunderscore )}
\end{itemize}
Dizer novamente; dizer muitas vezes.
\section{Redízima}
\begin{itemize}
\item {Grp. gram.:f.}
\end{itemize}
Acto ou effeito de redizimar.
\section{Redizimar}
\begin{itemize}
\item {Grp. gram.:v. t.}
\end{itemize}
\begin{itemize}
\item {Proveniência:(De \textunderscore re...\textunderscore  + \textunderscore dizimar\textunderscore )}
\end{itemize}
Cobrar nova dizima de (quem já pagára dizima ou aquillo que já fôra dizimado).
\section{Redobradamente}
\begin{itemize}
\item {Grp. gram.:adv.}
\end{itemize}
De modo redobrado; em dôbro.
\section{Redobrado}
\begin{itemize}
\item {Grp. gram.:adj.}
\end{itemize}
\begin{itemize}
\item {Proveniência:(De \textunderscore redobrar\textunderscore )}
\end{itemize}
Que redobrou.
Aumentado; multiplicado: \textunderscore desenganos redobrados\textunderscore .
\section{Redobradura}
\begin{itemize}
\item {Grp. gram.:f.}
\end{itemize}
O mesmo que \textunderscore redobramento\textunderscore .
\section{Redobramento}
\begin{itemize}
\item {Grp. gram.:m.}
\end{itemize}
Acto ou effeito de \textunderscore redobrar\textunderscore .
\section{Redobrar}
\begin{itemize}
\item {Grp. gram.:v. t.}
\end{itemize}
\begin{itemize}
\item {Grp. gram.:V. i.}
\end{itemize}
\begin{itemize}
\item {Proveniência:(De \textunderscore re...\textunderscore  + \textunderscore dobrar\textunderscore )}
\end{itemize}
Dobrar de novo; aumentar muito; repetir.
Têr aumento; ampliar-se; multiplicar-se: \textunderscore o lucro redobrou\textunderscore .
\section{Redobre}
\begin{itemize}
\item {Grp. gram.:adj.}
\end{itemize}
\begin{itemize}
\item {Utilização:Fig.}
\end{itemize}
\begin{itemize}
\item {Grp. gram.:M.}
\end{itemize}
\begin{itemize}
\item {Utilização:Ant.}
\end{itemize}
\begin{itemize}
\item {Utilização:Fig.}
\end{itemize}
\begin{itemize}
\item {Proveniência:(De \textunderscore re...\textunderscore  + \textunderscore dobre\textunderscore )}
\end{itemize}
O mesmo que \textunderscore redobrado\textunderscore .
Ardiloso.
Velhaco; doble.
Repetição de arcadas na rabeca, imitando trinado.
Gorgeio.
Fôrro ou estôfo, com que se torna mais encorpada ou forte uma peça de vestuário.
Doblez, velhacaria.
\section{Redôbro}
\begin{itemize}
\item {Grp. gram.:m.}
\end{itemize}
\begin{itemize}
\item {Proveniência:(De \textunderscore re...\textunderscore  + \textunderscore dôbro\textunderscore )}
\end{itemize}
O mesmo que \textunderscore quádruplo\textunderscore .
Acto ou effeito de redobrar.
\section{Redoiça}
\textunderscore f.\textunderscore  (e der.)
O mesmo que \textunderscore retoiça\textunderscore , etc.
\section{Redolente}
\begin{itemize}
\item {Grp. gram.:adj.}
\end{itemize}
\begin{itemize}
\item {Utilização:Poét.}
\end{itemize}
\begin{itemize}
\item {Proveniência:(Lat. \textunderscore redolens\textunderscore )}
\end{itemize}
Aromático; que tem cheiro agradável.
\section{Redolho}
\begin{itemize}
\item {fónica:dô}
\end{itemize}
\begin{itemize}
\item {Grp. gram.:m.}
\end{itemize}
\begin{itemize}
\item {Utilização:Prov.}
\end{itemize}
\begin{itemize}
\item {Utilização:trasm.}
\end{itemize}
\begin{itemize}
\item {Utilização:beir.}
\end{itemize}
\begin{itemize}
\item {Grp. gram.:Adj.}
\end{itemize}
\begin{itemize}
\item {Utilização:T. de Vouzela}
\end{itemize}
Cordeiro serôdio.
Entanguido, pêco, (falando-se de frutos).
(Cp. cast. \textunderscore redrojo\textunderscore )
\section{Redoma}
\begin{itemize}
\item {Grp. gram.:f.}
\end{itemize}
\begin{itemize}
\item {Utilização:Fam.}
\end{itemize}
\begin{itemize}
\item {Proveniência:(Do b. lat. \textunderscore arrotoma\textunderscore )}
\end{itemize}
Manga de vidro, fechada de um lado e destinada a resguardar do pó objectos delicados.
Extremo, cuidado consigo próprio.
\section{Redomão}
\begin{itemize}
\item {Grp. gram.:m.}
\end{itemize}
\begin{itemize}
\item {Utilização:Bras. do S}
\end{itemize}
Cavallo novo, que já foi montado algumas vezes para se domar.
(Cast. \textunderscore redomon\textunderscore )
\section{Redomoneação}
\begin{itemize}
\item {Grp. gram.:f.}
\end{itemize}
Acto de redomonear.
\section{Redomonear}
\begin{itemize}
\item {Grp. gram.:v. t.}
\end{itemize}
\begin{itemize}
\item {Utilização:Bras. do S}
\end{itemize}
\begin{itemize}
\item {Proveniência:(De \textunderscore redomão\textunderscore )}
\end{itemize}
Sujeitar (cavallo novo), nos primeiros galopes.
\section{Redonda}
\begin{itemize}
\item {Grp. gram.:adj. f.}
\end{itemize}
\begin{itemize}
\item {Utilização:Prov.}
\end{itemize}
\begin{itemize}
\item {Utilização:trasm.}
\end{itemize}
\begin{itemize}
\item {Grp. gram.:F.}
\end{itemize}
\begin{itemize}
\item {Utilização:Prov.}
\end{itemize}
\begin{itemize}
\item {Utilização:alent.}
\end{itemize}
Diz-se da aguardente fraca, ou que se póde beber, sem fazer mal.
Variedade de malagueta.
\section{Redondal}
\begin{itemize}
\item {Grp. gram.:adj.}
\end{itemize}
\begin{itemize}
\item {Utilização:Prov.}
\end{itemize}
\begin{itemize}
\item {Utilização:trasm.}
\end{itemize}
\begin{itemize}
\item {Proveniência:(De \textunderscore redondo\textunderscore )}
\end{itemize}
Diz-se de uma variedade de azeitona.
O mesmo que \textunderscore redondil\textunderscore ?
\section{Redondamente}
\begin{itemize}
\item {Grp. gram.:adv.}
\end{itemize}
\begin{itemize}
\item {Proveniência:(De \textunderscore redondo\textunderscore )}
\end{itemize}
Em redondo, á roda.
Estendendo-se no chão: \textunderscore caiu redondamente\textunderscore .
Francamente.
Categoricamente: \textunderscore negar redondamente\textunderscore .
\section{Redondear}
\begin{itemize}
\item {Grp. gram.:v. t.}
\end{itemize}
\begin{itemize}
\item {Grp. gram.:V. i.}
\end{itemize}
Tornar redondo.
Andar á roda.
\section{Redondel}
\begin{itemize}
\item {Grp. gram.:m.}
\end{itemize}
\begin{itemize}
\item {Utilização:Neol.}
\end{itemize}
Arena, em as praças de toiros.
Arena de lutadores. Cf. Herculano, \textunderscore Lendas\textunderscore , 57.
(Cast. \textunderscore redondel\textunderscore )
\section{Redondela}
\begin{itemize}
\item {Grp. gram.:f.}
\end{itemize}
\begin{itemize}
\item {Utilização:Pop.}
\end{itemize}
O mesmo que \textunderscore rodela\textunderscore .
\section{Redondez}
\begin{itemize}
\item {Grp. gram.:f.}
\end{itemize}
O mesmo que \textunderscore redondeza\textunderscore . Cf. Filinto, \textunderscore D. Man.\textunderscore , I, 333.
\section{Redondeza}
\begin{itemize}
\item {Grp. gram.:f.}
\end{itemize}
Qualidade do que é redondo.
Esphera.
O mundo. Cf. R. Lobo, \textunderscore Côrte na Aldeia\textunderscore , I, 92.
Cercanias, arrabaldes.
Espaço de terra ou região, comprehendida em determinado circuito.
Conjunto de localidades próximas: \textunderscore asseguro-lhe que nesta redondeza não há rapariga mais guapa\textunderscore .
\section{Redondil}
\begin{itemize}
\item {Grp. gram.:adj.}
\end{itemize}
\begin{itemize}
\item {Proveniência:(De \textunderscore redondo\textunderscore )}
\end{itemize}
Redondo.
Diz-se de uma espécie de azeitonas graúdas, também conhecidas por azeitonas de Elvas.
\section{Redondilha}
\begin{itemize}
\item {Grp. gram.:f.}
\end{itemize}
\begin{itemize}
\item {Proveniência:(De \textunderscore redondo\textunderscore )}
\end{itemize}
Verso de cinco ou sete sýllabas métricas.
\section{Redondinha}
\begin{itemize}
\item {Grp. gram.:f.}
\end{itemize}
\begin{itemize}
\item {Utilização:Ant.}
\end{itemize}
O mesmo que \textunderscore redondilha\textunderscore . Cf. Castro, \textunderscore Paráfrase\textunderscore , 4.
\section{Redondo}
\begin{itemize}
\item {Grp. gram.:adj.}
\end{itemize}
\begin{itemize}
\item {Utilização:Fig.}
\end{itemize}
\begin{itemize}
\item {Grp. gram.:M.}
\end{itemize}
\begin{itemize}
\item {Utilização:Prov.}
\end{itemize}
\begin{itemize}
\item {Utilização:beir.}
\end{itemize}
Que tem fórma de círculo.
Esphérico.
Cylíndrico.
Curvo.
Completo.
Rechonchudo.
Gordo.
Casta de uva borraçal.
Instrumento de carpinteiro, espécie de plaina.
\section{Redopio}
\textunderscore m.\textunderscore  (e der.)
O mesmo que \textunderscore rodopio\textunderscore , etc. Cf. Camillo, \textunderscore Myst. de Lisb.\textunderscore , II, 232.
\section{Redór}
\begin{itemize}
\item {Grp. gram.:m.}
\end{itemize}
\begin{itemize}
\item {Proveniência:(Do lat. hyp. \textunderscore rotador\textunderscore )}
\end{itemize}
Circuito; cercanias; arrabalde.
Roda; volta.
\section{Redôr}
\begin{itemize}
\item {Grp. gram.:m.}
\end{itemize}
\begin{itemize}
\item {Utilização:Marn.}
\end{itemize}
\begin{itemize}
\item {Proveniência:(De \textunderscore rêr\textunderscore )}
\end{itemize}
Operário salineiro, que toma água para os viveiros e quebra a crosta salina. Cf. \textunderscore Museu Techn.\textunderscore , 104.
\section{Redóres}
\begin{itemize}
\item {Grp. gram.:m. pl.}
\end{itemize}
O mesmo que \textunderscore redór\textunderscore .
\section{Rèdoría}
\begin{itemize}
\item {Grp. gram.:f.}
\end{itemize}
Acto de rêr o sal; redura.
\section{Redra}
\begin{itemize}
\item {Grp. gram.:f.}
\end{itemize}
Acto de redrar.
\section{Redrar}
\begin{itemize}
\item {Grp. gram.:v. t.  e  i.}
\end{itemize}
Cavar de novo, mas ligeiramente, (as vinhas), para tirar a erva.
(Por \textunderscore rudrar\textunderscore , do lat. \textunderscore rutrum\textunderscore )
\section{Redução}
\begin{itemize}
\item {Grp. gram.:f.}
\end{itemize}
\begin{itemize}
\item {Grp. gram.:Pl.}
\end{itemize}
\begin{itemize}
\item {Utilização:Mús.}
\end{itemize}
\begin{itemize}
\item {Proveniência:(Lat. \textunderscore reductio\textunderscore )}
\end{itemize}
Acto ou effeito de reduzir.
Peças do maquinismo do órgão, que consistem num systema de varetas, rolos e pequenas alavancas, que transmittem o movimento das teclas ás válvulas dos someiros.
\section{Reducção}
\begin{itemize}
\item {Grp. gram.:f.}
\end{itemize}
\begin{itemize}
\item {Grp. gram.:Pl.}
\end{itemize}
\begin{itemize}
\item {Utilização:Mús.}
\end{itemize}
\begin{itemize}
\item {Proveniência:(Lat. \textunderscore reductio\textunderscore )}
\end{itemize}
Acto ou effeito de reduzir.
Peças do maquinismo do órgão, que consistem num systema de varetas, rolos e pequenas alavancas, que transmittem o movimento das teclas ás válvulas dos someiros.
\section{Reducente}
\begin{itemize}
\item {Grp. gram.:adj.}
\end{itemize}
\begin{itemize}
\item {Proveniência:(Lat. \textunderscore reducens\textunderscore )}
\end{itemize}
Que reduz; reductivo.
\section{Reductibilidade}
\begin{itemize}
\item {Grp. gram.:f.}
\end{itemize}
Qualidade do que é reductível.
\section{Reductivamente}
\begin{itemize}
\item {Grp. gram.:adv.}
\end{itemize}
\begin{itemize}
\item {Proveniência:(De \textunderscore reductivo\textunderscore )}
\end{itemize}
O mesmo que \textunderscore limitadamente\textunderscore .
\section{Reductível}
\begin{itemize}
\item {Grp. gram.:adj.}
\end{itemize}
\begin{itemize}
\item {Proveniência:(Do lat. \textunderscore reductus\textunderscore )}
\end{itemize}
Que se póde reduzir.
\section{Reductivo}
\begin{itemize}
\item {Grp. gram.:adj.}
\end{itemize}
\begin{itemize}
\item {Proveniência:(Do lat. \textunderscore reductus\textunderscore )}
\end{itemize}
Que póde reduzir.
\section{Reducto}
\begin{itemize}
\item {Grp. gram.:m.}
\end{itemize}
\begin{itemize}
\item {Utilização:Bras. de Mato-Grosso}
\end{itemize}
\begin{itemize}
\item {Proveniência:(Lat. \textunderscore reductus\textunderscore )}
\end{itemize}
Edificação fechada, no interior de uma fortaleza ou de outra fortificação, para aumentar a resistência desta.
Porção de terreno que, depois do trasbordamento dos rios, fica acima do nível das águas.
\section{Reductor}
\begin{itemize}
\item {Grp. gram.:m.  e  adj.}
\end{itemize}
\begin{itemize}
\item {Proveniência:(Lat. \textunderscore reductor\textunderscore )}
\end{itemize}
O que reduz.
\section{Redundância}
\begin{itemize}
\item {Grp. gram.:f.}
\end{itemize}
\begin{itemize}
\item {Proveniência:(Lat. \textunderscore redundantia\textunderscore )}
\end{itemize}
Qualidade do que é redundante; pleonasmo.
\section{Redundante}
\begin{itemize}
\item {Grp. gram.:adj.}
\end{itemize}
\begin{itemize}
\item {Proveniência:(Lat. \textunderscore redundans\textunderscore )}
\end{itemize}
Que redunda; excessivo; palavroso.
\section{Redundantemente}
\begin{itemize}
\item {Grp. gram.:adv.}
\end{itemize}
De modo redundante.
\section{Redundar}
\begin{itemize}
\item {Grp. gram.:v. i.}
\end{itemize}
\begin{itemize}
\item {Utilização:Fig.}
\end{itemize}
\begin{itemize}
\item {Proveniência:(Lat. \textunderscore redundare\textunderscore )}
\end{itemize}
Trasbordar; deitar por fóra um líquido.
Sêr muito abundante; sobejar.
Resultar.
Converter-se, reverter: \textunderscore a imprevidência redunda em desastres\textunderscore .
Sêr motivo.
\section{Reduplicação}
\begin{itemize}
\item {Grp. gram.:f.}
\end{itemize}
Acto ou effeito de reduplicar.
\section{Reduplicar}
\begin{itemize}
\item {Grp. gram.:v. t.}
\end{itemize}
\begin{itemize}
\item {Proveniência:(De \textunderscore re...\textunderscore  + \textunderscore duplicar\textunderscore )}
\end{itemize}
Dobrar novamente, duplicar outra vez; repetir.
\section{Reduplicativo}
\begin{itemize}
\item {Grp. gram.:m.  e  adj.}
\end{itemize}
\begin{itemize}
\item {Proveniência:(De \textunderscore reduplicar\textunderscore )}
\end{itemize}
O que envolve reduplicação; o que indica repetição de acto.
\section{Redura}
\begin{itemize}
\item {Grp. gram.:f.}
\end{itemize}
Acto de rêr.
Rodura.
\section{Redúvias}
\begin{itemize}
\item {Grp. gram.:f. pl.}
\end{itemize}
\begin{itemize}
\item {Proveniência:(Lat. \textunderscore reduvia\textunderscore )}
\end{itemize}
Restos de comida, que ficam nos dentes.
\section{Redúvio}
\begin{itemize}
\item {Grp. gram.:m.}
\end{itemize}
\begin{itemize}
\item {Proveniência:(Lat. \textunderscore reduvia\textunderscore )}
\end{itemize}
Gênero de insectos hemípteros.
\section{Reduvíolo}
\begin{itemize}
\item {Grp. gram.:m.}
\end{itemize}
\begin{itemize}
\item {Proveniência:(De \textunderscore redúvio\textunderscore )}
\end{itemize}
Gênero de insectos hemípteros.
\section{Reduzida}
\begin{itemize}
\item {Grp. gram.:f.}
\end{itemize}
\begin{itemize}
\item {Utilização:Mathem.}
\end{itemize}
\begin{itemize}
\item {Proveniência:(De \textunderscore reduzir\textunderscore )}
\end{itemize}
Equação, cujo grau se deminuíu.
\section{Reduzir}
\begin{itemize}
\item {Grp. gram.:v. t.}
\end{itemize}
\begin{itemize}
\item {Proveniência:(Do lat. \textunderscore reducere\textunderscore )}
\end{itemize}
Retrahir, fazer voltar ao primeiro estado.
Deminuír.
Proporcionar mau estado ou situação a: \textunderscore reduzir á miséria\textunderscore .
Subjugar, sujeitar.
Converter; transformar: \textunderscore reduzir uma vidraça a estilhaços\textunderscore .
Exprimir por certa unidade uma quantidade expressa noutra unidade differente: \textunderscore reduzir côvados a metros\textunderscore .
Simplificar.
Reproduzir em menor escala: \textunderscore reduzir um desenho\textunderscore .
Mitigar.
Restingir: \textunderscore reduzir cuidados\textunderscore .
Resumir, compendiar.
Substituir.
Desaggregar.
\section{Reduzível}
\begin{itemize}
\item {Grp. gram.:adj.}
\end{itemize}
O mesmo que \textunderscore reductível\textunderscore .
\section{Reedição}
\begin{itemize}
\item {Grp. gram.:f.}
\end{itemize}
\begin{itemize}
\item {Proveniência:(De \textunderscore re...\textunderscore  + \textunderscore edição\textunderscore )}
\end{itemize}
Nova edição; acto de reeditar.
\section{Reedificação}
\begin{itemize}
\item {Grp. gram.:f.}
\end{itemize}
Acto ou effeito de reedificar.
\section{Reedificador}
\begin{itemize}
\item {Grp. gram.:m.  e  adj.}
\end{itemize}
O que reedifica.
\section{Reedificante}
\begin{itemize}
\item {Grp. gram.:adj.}
\end{itemize}
\begin{itemize}
\item {Grp. gram.:M.}
\end{itemize}
\begin{itemize}
\item {Proveniência:(De \textunderscore reedificar\textunderscore )}
\end{itemize}
Que reedifica.
Dono de prédio, que se está reconstruíndo. Cf. Dom. Vieira, \textunderscore Thes. da Ling.\textunderscore , vb. \textunderscore edificante\textunderscore .
\section{Reedificar}
\begin{itemize}
\item {Grp. gram.:v. t.}
\end{itemize}
\begin{itemize}
\item {Proveniência:(Do b. lat. \textunderscore reedificare\textunderscore )}
\end{itemize}
Edificar novamente; reconstruír.
Reformar; restaurar.
\section{Reeditar}
\begin{itemize}
\item {Grp. gram.:v. t.}
\end{itemize}
\begin{itemize}
\item {Proveniência:(De \textunderscore re...\textunderscore  + \textunderscore editar\textunderscore )}
\end{itemize}
Editar novamente; publicar de novo. Cf. Alb. Giraldes, \textunderscore Philos.\textunderscore , 88.
\section{Reeducabilidade}
\begin{itemize}
\item {Grp. gram.:f.}
\end{itemize}
Qualidade de reeducável.
\section{Reeducação}
\begin{itemize}
\item {Grp. gram.:f.}
\end{itemize}
Acto ou effeito de reeducar.
\section{Reeducador}
\begin{itemize}
\item {Grp. gram.:adj.}
\end{itemize}
Que reeduca.
\section{Reeducar}
\begin{itemize}
\item {Grp. gram.:v. t.}
\end{itemize}
\begin{itemize}
\item {Utilização:bras}
\end{itemize}
\begin{itemize}
\item {Utilização:Neol.}
\end{itemize}
\begin{itemize}
\item {Proveniência:(De \textunderscore re...\textunderscore  + \textunderscore educar\textunderscore )}
\end{itemize}
Completar ou aperfeiçoar a educação de.
\section{Reeducável}
\begin{itemize}
\item {Grp. gram.:adj.}
\end{itemize}
Que se póde reeducar.
\section{Reeleger}
\begin{itemize}
\item {Grp. gram.:v. t.}
\end{itemize}
\begin{itemize}
\item {Proveniência:(De \textunderscore re...\textunderscore  + \textunderscore eleger\textunderscore )}
\end{itemize}
Eleger de novo.
\section{Reelegível}
\begin{itemize}
\item {Grp. gram.:adj.}
\end{itemize}
Que se póde reeleger.
\section{Reeleição}
\begin{itemize}
\item {Grp. gram.:f.}
\end{itemize}
\begin{itemize}
\item {Proveniência:(De \textunderscore re...\textunderscore  + \textunderscore eleição\textunderscore )}
\end{itemize}
Acto de reeleger.
\section{Reeleleito}
\begin{itemize}
\item {Grp. gram.:m.}
\end{itemize}
\begin{itemize}
\item {Proveniência:(De \textunderscore reeleger\textunderscore )}
\end{itemize}
Aquelle que foi eleito novamente.
\section{Reembarcar}
\begin{itemize}
\item {Grp. gram.:v. i.  e  p.}
\end{itemize}
\begin{itemize}
\item {Proveniência:(De \textunderscore re...\textunderscore  + \textunderscore embarcar\textunderscore )}
\end{itemize}
Embarcar de novo.
\section{Reembarque}
\begin{itemize}
\item {Grp. gram.:m.}
\end{itemize}
Acto de reembarcar.
\section{Reembolsar}
\begin{itemize}
\item {Grp. gram.:v. t.}
\end{itemize}
\begin{itemize}
\item {Proveniência:(De \textunderscore re...\textunderscore  + \textunderscore embolsar\textunderscore )}
\end{itemize}
Embolsar novamente.
\section{Reembolsável}
\begin{itemize}
\item {Grp. gram.:adj.}
\end{itemize}
Que se póde reembolsar.
\section{Reembôlso}
\begin{itemize}
\item {Grp. gram.:m.}
\end{itemize}
Acto ou effeito de reembolsar.
\section{Reemenda}
\begin{itemize}
\item {Grp. gram.:f.}
\end{itemize}
Acto de reemendar. Cf. Castilho, \textunderscore Fastos\textunderscore , I, 322.
\section{Reemendar}
\begin{itemize}
\item {Grp. gram.:v. t.}
\end{itemize}
\begin{itemize}
\item {Proveniência:(De \textunderscore re...\textunderscore  + \textunderscore emendar\textunderscore )}
\end{itemize}
Emendar novamente; emendar muitas vezes.
\section{Reempossar}
\begin{itemize}
\item {Grp. gram.:v. t.}
\end{itemize}
\begin{itemize}
\item {Proveniência:(De \textunderscore re...\textunderscore  + \textunderscore empossar\textunderscore )}
\end{itemize}
Confirmar na posse.
\section{Reempregar}
\begin{itemize}
\item {Grp. gram.:v. t.}
\end{itemize}
\begin{itemize}
\item {Proveniência:(De \textunderscore re...\textunderscore  + \textunderscore empregar\textunderscore )}
\end{itemize}
Empregar de novo.
\section{Reencarceração}
\begin{itemize}
\item {Grp. gram.:f.}
\end{itemize}
Acto ou effeito de reencarcerar.
\section{Reencarcerar}
\begin{itemize}
\item {Grp. gram.:v. t.}
\end{itemize}
\begin{itemize}
\item {Proveniência:(De \textunderscore re...\textunderscore  + \textunderscore encarcerar\textunderscore )}
\end{itemize}
Encarcerar de novo.
\section{Reencarnação}
\begin{itemize}
\item {Grp. gram.:f.}
\end{itemize}
\begin{itemize}
\item {Utilização:Espir.}
\end{itemize}
Acto de reencarnar.
Pluralidade das existências, relativamente a um só espírito.
\section{Reencarnar}
\begin{itemize}
\item {Grp. gram.:v. i.}
\end{itemize}
\begin{itemize}
\item {Utilização:Espir.}
\end{itemize}
\begin{itemize}
\item {Proveniência:(De \textunderscore re...\textunderscore  + \textunderscore encarnar\textunderscore )}
\end{itemize}
Reassumir a fórma humana.
Tornar a encarnar (falando-se do espírito que já teve uma ou mais existências terrenas). Cf. M. Velho, \textunderscore Man. do Espirita\textunderscore .
\section{Reencher}
\begin{itemize}
\item {Grp. gram.:v. t.}
\end{itemize}
\begin{itemize}
\item {Proveniência:(De \textunderscore re...\textunderscore  + \textunderscore encher\textunderscore )}
\end{itemize}
Encher novamente.
\section{Reenchimento}
\begin{itemize}
\item {Grp. gram.:m.}
\end{itemize}
Acto ou effeito de reencher.
\section{Reencontrar}
\begin{itemize}
\item {Grp. gram.:v. t.}
\end{itemize}
\begin{itemize}
\item {Proveniência:(De \textunderscore re...\textunderscore  + \textunderscore encontrar\textunderscore )}
\end{itemize}
Encontrar de novo.
\section{Reencontro}
\begin{itemize}
\item {Grp. gram.:m.}
\end{itemize}
Acto ou effeito de reencontrar.
O mesmo que \textunderscore recontro\textunderscore .
\section{Reencorporar}
\begin{itemize}
\item {Grp. gram.:v. t.}
\end{itemize}
\begin{itemize}
\item {Proveniência:(De \textunderscore re...\textunderscore  + \textunderscore encorpar\textunderscore )}
\end{itemize}
Encorporar novamente. Cf. Castilho, \textunderscore D. Quixote\textunderscore , II, 438.
\section{Reenlaçar}
\begin{itemize}
\item {Grp. gram.:v. t.}
\end{itemize}
\begin{itemize}
\item {Proveniência:(De \textunderscore re...\textunderscore  + \textunderscore enlaçar\textunderscore )}
\end{itemize}
Enlaçar novamente.
\section{Reenlace}
\begin{itemize}
\item {Grp. gram.:m.}
\end{itemize}
Acto de reenlaçar.
\section{Reentrância}
\begin{itemize}
\item {Grp. gram.:f.}
\end{itemize}
Qualidade de reentrante.
\section{Reentrante}
\begin{itemize}
\item {Grp. gram.:adj.}
\end{itemize}
\begin{itemize}
\item {Proveniência:(De \textunderscore reentrar\textunderscore )}
\end{itemize}
Que reentra; que fórma ângulo ou curva para dentro.
\section{Reentrar}
\begin{itemize}
\item {Grp. gram.:v. i.}
\end{itemize}
\begin{itemize}
\item {Proveniência:(De \textunderscore re...\textunderscore  + \textunderscore entrar\textunderscore )}
\end{itemize}
Tornar a entrar.
Recolher-se.
Voltar para casa.
\section{Reenviar}
\begin{itemize}
\item {Grp. gram.:v. t.}
\end{itemize}
\begin{itemize}
\item {Proveniência:(De \textunderscore re...\textunderscore  + \textunderscore enviar\textunderscore )}
\end{itemize}
Enviar novamente.
Devolver; recambiar.
\section{Reequilibrar}
\begin{itemize}
\item {Grp. gram.:v. t.}
\end{itemize}
\begin{itemize}
\item {Proveniência:(De \textunderscore re...\textunderscore  + \textunderscore equilibrar\textunderscore )}
\end{itemize}
Equilibrar de novo. Cf. Garrett, \textunderscore Port. na Balança\textunderscore , 322.
\section{Reerguer}
\begin{itemize}
\item {Grp. gram.:v. t.}
\end{itemize}
\begin{itemize}
\item {Proveniência:(De \textunderscore re...\textunderscore  + \textunderscore erguer\textunderscore )}
\end{itemize}
Erguer novamente.
\section{Reescrever}
\begin{itemize}
\item {Grp. gram.:v. t.}
\end{itemize}
Tornar a escrever, escrever de novo. Cf. Castilho, \textunderscore Monta alverne\textunderscore .
(Cp. \textunderscore rescrever\textunderscore )
\section{Reesposar-se}
\begin{itemize}
\item {Grp. gram.:v. p.}
\end{itemize}
\begin{itemize}
\item {Proveniência:(De \textunderscore re...\textunderscore  + \textunderscore esposar\textunderscore )}
\end{itemize}
Casar novamente.
\section{Reespumas}
\begin{itemize}
\item {Grp. gram.:f. pl.}
\end{itemize}
\begin{itemize}
\item {Utilização:Bras}
\end{itemize}
\begin{itemize}
\item {Proveniência:(De \textunderscore re...\textunderscore  + \textunderscore espuma\textunderscore )}
\end{itemize}
Açúcar feito da espuma da primeira espuma.
\section{Reestampar}
\begin{itemize}
\item {Grp. gram.:v. t.}
\end{itemize}
\begin{itemize}
\item {Proveniência:(De \textunderscore re...\textunderscore  + \textunderscore estampar\textunderscore )}
\end{itemize}
Estampar novamente, reeditar:«\textunderscore ...do mesmo prelo que a reestampa.\textunderscore »Castilho, \textunderscore Prólogo\textunderscore  a uma trad. do \textunderscore Judeu Errante\textunderscore .
\section{Reestudar}
\begin{itemize}
\item {Grp. gram.:v. t.}
\end{itemize}
\begin{itemize}
\item {Proveniência:(De \textunderscore re...\textunderscore  + \textunderscore estudar\textunderscore )}
\end{itemize}
Tornar a estudar; estudar muito. Cf. Castilho, \textunderscore Fausto\textunderscore , XIII.
\section{Reexistir}
\begin{itemize}
\item {Grp. gram.:v. i.}
\end{itemize}
\begin{itemize}
\item {Proveniência:(De \textunderscore re...\textunderscore  + \textunderscore existir\textunderscore )}
\end{itemize}
Tornar a existir.
Reapparecer.
Restabelecer-se (aquillo que tinha findado ou perecido).
\section{Reexpedir}
\begin{itemize}
\item {Grp. gram.:v. t.}
\end{itemize}
\begin{itemize}
\item {Proveniência:(De \textunderscore re...\textunderscore  + \textunderscore expedir\textunderscore )}
\end{itemize}
Expedir (aquillo que se recebeu).
Reexportar.
\section{Reexportação}
\begin{itemize}
\item {Grp. gram.:f.}
\end{itemize}
Acto ou effeito de reexportar.
\section{Reexportador}
\begin{itemize}
\item {Grp. gram.:m.  e  adj.}
\end{itemize}
O que reexporta.
\section{Reexportar}
\begin{itemize}
\item {Grp. gram.:v. t.}
\end{itemize}
\begin{itemize}
\item {Proveniência:(De \textunderscore re...\textunderscore  + \textunderscore exportar\textunderscore )}
\end{itemize}
Tornar a exportar; exportar (mercadoria que se importou).
\section{Refaiscar}
\begin{itemize}
\item {fónica:fa-is}
\end{itemize}
\begin{itemize}
\item {Grp. gram.:v. i.}
\end{itemize}
\begin{itemize}
\item {Proveniência:(De \textunderscore re...\textunderscore  + \textunderscore faiscar\textunderscore )}
\end{itemize}
Faiscar muitas vezes, scintillar. Cf. Alv. Mendes, \textunderscore Discursos\textunderscore , 73.
\section{Refalar}
\begin{itemize}
\item {Grp. gram.:v. i.}
\end{itemize}
\begin{itemize}
\item {Proveniência:(De \textunderscore re...\textunderscore  + \textunderscore falar\textunderscore )}
\end{itemize}
Tornar a falar.
\section{Refalsadamente}
\begin{itemize}
\item {Grp. gram.:adv.}
\end{itemize}
De modo refalsado; com deslealdade.
\section{Refalsado}
\begin{itemize}
\item {Grp. gram.:adj.}
\end{itemize}
\begin{itemize}
\item {Proveniência:(De \textunderscore re...\textunderscore  + \textunderscore falso\textunderscore )}
\end{itemize}
Em que não há sinceridade; desleal; muito falso; fingido, hypócrita.
\section{Refalsamento}
\begin{itemize}
\item {Grp. gram.:m.}
\end{itemize}
\begin{itemize}
\item {Proveniência:(De \textunderscore re...\textunderscore  + \textunderscore falso\textunderscore )}
\end{itemize}
Acto de refalsado.
Acto ou effeito de refalsear.
\section{Refalsear}
\begin{itemize}
\item {Grp. gram.:v. t.}
\end{itemize}
\begin{itemize}
\item {Proveniência:(De \textunderscore re...\textunderscore  + \textunderscore falsear\textunderscore )}
\end{itemize}
Sêr refalsado para com; atraiçoar.
\section{Refartar}
\begin{itemize}
\item {Grp. gram.:v. t.}
\end{itemize}
\begin{itemize}
\item {Proveniência:(De \textunderscore re...\textunderscore  + \textunderscore fartar\textunderscore )}
\end{itemize}
Fartar muito; saciar inteiramente. Cf. Castilho, \textunderscore Geórgicas\textunderscore , 93; B. Pato, \textunderscore Livro do Monte\textunderscore .
\section{Refasteleiro}
\begin{itemize}
\item {Grp. gram.:adj.}
\end{itemize}
\begin{itemize}
\item {Utilização:Açor. da ilha de San-Jorge}
\end{itemize}
Diz-se de um indivíduo muito activo, irrequieto ou turbulento.
(Relaciona-se com \textunderscore refestêlo\textunderscore ?)
\section{Refazedor}
\begin{itemize}
\item {Grp. gram.:m.  e  adj.}
\end{itemize}
O que refaz.
\section{Refazer}
\begin{itemize}
\item {Grp. gram.:v. t.}
\end{itemize}
\begin{itemize}
\item {Proveniência:(De \textunderscore re...\textunderscore  + \textunderscore fazer\textunderscore )}
\end{itemize}
Fazer novamente.
Restaurar.
Consertar; reformar.
Corrigir.
Reconstruir.
Restabelecer.
Dar novo alento a.
Nutrir.
Prover.
Indemnizar.
\section{Refazimento}
\begin{itemize}
\item {Grp. gram.:m.}
\end{itemize}
\begin{itemize}
\item {Utilização:Ant.}
\end{itemize}
Acto ou effeito de refazer.
Consêrto, reparo.
Compensação, que se dava a quem ficou lesado em partilhas ou num contrato.
\section{Refe}
\begin{itemize}
\item {Grp. gram.:m.}
\end{itemize}
O mesmo que \textunderscore refle\textunderscore . Cf. Camillo, \textunderscore Doze Casam.\textunderscore , 74.
\section{Refece}
\begin{itemize}
\item {Grp. gram.:adj.}
\end{itemize}
\begin{itemize}
\item {Utilização:Fig.}
\end{itemize}
\begin{itemize}
\item {Grp. gram.:Adv.}
\end{itemize}
\begin{itemize}
\item {Proveniência:(Do ár. \textunderscore rahkiç\textunderscore .)}
\end{itemize}
Que tem maus sentimentos; infame; miserável.
Ordinário.
Pobre.
Fácil.
Por baixo preço.
\section{Refècer}
\textunderscore v. i.\textunderscore  (e der.)
O mesmo que \textunderscore arrefecer\textunderscore .
\section{Refechamento}
\begin{itemize}
\item {Grp. gram.:m.}
\end{itemize}
Acto ou effeito de refechar.
\section{Refechar}
\begin{itemize}
\item {Grp. gram.:v.}
\end{itemize}
\begin{itemize}
\item {Utilização:t. Constr.}
\end{itemize}
\begin{itemize}
\item {Proveniência:(De \textunderscore re...\textunderscore  + \textunderscore fechar\textunderscore )}
\end{itemize}
Tapar as juntas de (cantaria).
\section{Refectivo}
\begin{itemize}
\item {Grp. gram.:adj.}
\end{itemize}
\begin{itemize}
\item {Proveniência:(Do lat. \textunderscore refectus\textunderscore )}
\end{itemize}
Reconstituínte, tonificante, fortificante.
\section{Refectório}
\begin{itemize}
\item {Grp. gram.:adj.}
\end{itemize}
O mesmo que \textunderscore refectivo\textunderscore .
\section{Refecundar}
\begin{itemize}
\item {Grp. gram.:v. t.}
\end{itemize}
\begin{itemize}
\item {Proveniência:(De \textunderscore re...\textunderscore  + \textunderscore fecundar\textunderscore )}
\end{itemize}
Fecundar novamente. Cf. Castilho, \textunderscore Fastos\textunderscore , II, 99.
\section{Refega}
\begin{itemize}
\item {Grp. gram.:f.}
\end{itemize}
O mesmo que \textunderscore refrega\textunderscore .
Pé de vento, redemoínho. C. Fern. Mendes, \textunderscore Peregrin.\textunderscore --Castilho diz \textunderscore réfega\textunderscore :«\textunderscore ...por isso arrosta audaz réfegas e invernias.\textunderscore »\textunderscore Geórgicas\textunderscore , 103.
\section{Refegado}
\begin{itemize}
\item {Grp. gram.:adj.}
\end{itemize}
Que tem refegos: \textunderscore e pôs as mãos na refegada pança\textunderscore .
\section{Refegar}
\begin{itemize}
\item {Grp. gram.:v. t.}
\end{itemize}
Fazer refegos em.
\section{Refêgo}
\begin{itemize}
\item {Grp. gram.:m.}
\end{itemize}
Dobra, préga no vestuário.
Festo.
Dobra na pelle das pessôas, por effeito da nutrição.
(Cp. \textunderscore rofego\textunderscore )
\section{Refeição}
\begin{itemize}
\item {Grp. gram.:f.}
\end{itemize}
\begin{itemize}
\item {Proveniência:(Do lat. \textunderscore refectio\textunderscore )}
\end{itemize}
Acto de refazer as fôrças.
Alimentos, que se tomam de cada vez em certas horas do dia.
Alimento, que se toma a qualquer hora.
O mesmo que \textunderscore substituição\textunderscore . Cf. \textunderscore Peregrinação\textunderscore , VI.
\section{Refeita}
\begin{itemize}
\item {Grp. gram.:f.}
\end{itemize}
\begin{itemize}
\item {Utilização:Gír.}
\end{itemize}
\begin{itemize}
\item {Proveniência:(De \textunderscore refeito\textunderscore )}
\end{itemize}
Ceia.
\section{Refeito}
\begin{itemize}
\item {Grp. gram.:adj.}
\end{itemize}
\begin{itemize}
\item {Utilização:Prov.}
\end{itemize}
\begin{itemize}
\item {Utilização:trasm.}
\end{itemize}
\begin{itemize}
\item {Proveniência:(De \textunderscore refazer\textunderscore )}
\end{itemize}
Tornado a fazer.
Emendado, corrigido. Cf. Garrett, \textunderscore Viagens\textunderscore , I, 125.
Que cerra os dentes, remordendo-se com raiva.
\section{Refeitoreiro}
\begin{itemize}
\item {Grp. gram.:m.}
\end{itemize}
Aquelle que trata do refeitório.
\section{Refeitório}
\begin{itemize}
\item {Grp. gram.:m.}
\end{itemize}
\begin{itemize}
\item {Proveniência:(Do lat. \textunderscore refectorium\textunderscore )}
\end{itemize}
Casa, onde se servem as refeições, nas communidades conventuaes, collegiaes, etc.
\section{Refém}
\begin{itemize}
\item {Grp. gram.:m.}
\end{itemize}
\begin{itemize}
\item {Proveniência:(Do ár. \textunderscore rehn\textunderscore )}
\end{itemize}
Pessôa que é entregue ao inimigo ou é tomada por êste para execução de um contrato, ou para obrigar a certas concessões indivíduos, ou as collectividades, de que essa pessôa faz parte.
\section{Refenas}
\begin{itemize}
\item {Grp. gram.:f. pl.}
\end{itemize}
\begin{itemize}
\item {Utilização:Ant.}
\end{itemize}
Refens. Cf. S. R. Viterbo, \textunderscore Elucidário\textunderscore .
\section{Refender}
\begin{itemize}
\item {Grp. gram.:v. t.}
\end{itemize}
\begin{itemize}
\item {Proveniência:(De \textunderscore re...\textunderscore  + \textunderscore fender\textunderscore )}
\end{itemize}
Fender novamente; fender em muitas partes; golpear.
\section{Refendimento}
\begin{itemize}
\item {Grp. gram.:m.}
\end{itemize}
Acto ou effeito de refender.
Obra de escultura em alto relêvo.
\section{Refentar}
\begin{itemize}
\item {Grp. gram.:v. t.}
\end{itemize}
O mesmo que \textunderscore arrefentar\textunderscore .
\section{Referência}
\begin{itemize}
\item {Grp. gram.:f.}
\end{itemize}
\begin{itemize}
\item {Grp. gram.:Pl.}
\end{itemize}
\begin{itemize}
\item {Proveniência:(De \textunderscore referente\textunderscore )}
\end{itemize}
Acto de referir.
Aquillo que se refere; allusão.
Informações: \textunderscore tomei criado com bôas referências\textunderscore .
\section{Referenda}
\begin{itemize}
\item {Grp. gram.:f.}
\end{itemize}
Acto ou effeito de referendar.
\section{Referendar}
\begin{itemize}
\item {Grp. gram.:v. t.}
\end{itemize}
\begin{itemize}
\item {Proveniência:(Do lat. \textunderscore referendus\textunderscore )}
\end{itemize}
Assignar, como responsável.
Assignar o Ministro (um documento legal), para que êste se publique ou execute.
\section{Referendário}
\begin{itemize}
\item {Grp. gram.:m.}
\end{itemize}
O que referenda.
\section{Referente}
\begin{itemize}
\item {Grp. gram.:adj.}
\end{itemize}
\begin{itemize}
\item {Proveniência:(Lat. \textunderscore referens\textunderscore )}
\end{itemize}
Que se refere; allusivo.
\section{Referimento}
\begin{itemize}
\item {Grp. gram.:m.}
\end{itemize}
Acto ou effeito de referir.
\section{Referir}
\begin{itemize}
\item {Grp. gram.:v. t.}
\end{itemize}
\begin{itemize}
\item {Grp. gram.:V. p.}
\end{itemize}
\begin{itemize}
\item {Proveniência:(Lat. \textunderscore referre\textunderscore )}
\end{itemize}
Narrar.
Trazer á balha.
Attribuír; applicar.
Alludir.
Dizer respeito, têr relação.
\section{Refermentação}
\begin{itemize}
\item {Grp. gram.:f.}
\end{itemize}
Acto ou effeito de refermentar. Cf. \textunderscore Techn. Rur.\textunderscore , 245.
\section{Refermentar}
\begin{itemize}
\item {Grp. gram.:v. i.}
\end{itemize}
\begin{itemize}
\item {Proveniência:(De \textunderscore re...\textunderscore  + \textunderscore fermentar\textunderscore )}
\end{itemize}
Fermentar novamente; fermentar muito.
\section{Referta}
\begin{itemize}
\item {Grp. gram.:f.}
\end{itemize}
\begin{itemize}
\item {Utilização:Ant.}
\end{itemize}
\begin{itemize}
\item {Proveniência:(Lat. \textunderscore referta\textunderscore )}
\end{itemize}
Contenda, porfia de palavras.
Refrega, escaramuça:«\textunderscore ...a gente que vinha escapada da referta de Coutinho...\textunderscore »Filinto, \textunderscore D. Man.\textunderscore , II, 142.
\section{Refertadamente}
\begin{itemize}
\item {Grp. gram.:adv.}
\end{itemize}
\begin{itemize}
\item {Utilização:Ant.}
\end{itemize}
\begin{itemize}
\item {Proveniência:(De \textunderscore refertar\textunderscore )}
\end{itemize}
De má vontade; com repugnância.
\section{Refertar}
\begin{itemize}
\item {Grp. gram.:v. t.}
\end{itemize}
\begin{itemize}
\item {Utilização:Ant.}
\end{itemize}
\begin{itemize}
\item {Proveniência:(De \textunderscore referta\textunderscore )}
\end{itemize}
Lançar em rosto (favores que se fizeram), como denunciando ingratidão.
Contender, impugnar.
\section{Referteiramente}
\begin{itemize}
\item {Grp. gram.:adv.}
\end{itemize}
\begin{itemize}
\item {Proveniência:(De \textunderscore referteiro\textunderscore )}
\end{itemize}
O mesmo que \textunderscore refertadamente\textunderscore .
\section{Referteiro}
\begin{itemize}
\item {Grp. gram.:adj.}
\end{itemize}
\begin{itemize}
\item {Utilização:Ant.}
\end{itemize}
\begin{itemize}
\item {Proveniência:(De \textunderscore refertar\textunderscore )}
\end{itemize}
Teimoso.
Que referta ou lança em rosto os benefícios que faz e de que se arrepende.
\section{Referto}
\begin{itemize}
\item {Grp. gram.:m.}
\end{itemize}
\begin{itemize}
\item {Utilização:Ant.}
\end{itemize}
\begin{itemize}
\item {Proveniência:(De \textunderscore refertar\textunderscore )}
\end{itemize}
O mesmo que \textunderscore referta\textunderscore , contenda.
\section{Referto}
\begin{itemize}
\item {Grp. gram.:adj.}
\end{itemize}
\begin{itemize}
\item {Proveniência:(Lat. \textunderscore refertus\textunderscore )}
\end{itemize}
Completamente cheio, pleno; abundante. Cf. Herculano, \textunderscore Casam. Civil\textunderscore .
\section{Refervente}
\begin{itemize}
\item {Grp. gram.:adj.}
\end{itemize}
Que referve; que ferve muito.
\section{Referver}
\begin{itemize}
\item {Grp. gram.:v. i.}
\end{itemize}
\begin{itemize}
\item {Utilização:Fig.}
\end{itemize}
\begin{itemize}
\item {Proveniência:(Do lat. \textunderscore refervere\textunderscore )}
\end{itemize}
Ferver de novo.
Ferver muito.
Fermentar.
Excitar-se.
Irritar-se; agitar-se; tumultuar.
Borbulhar.
\section{Refervimento}
\begin{itemize}
\item {Grp. gram.:m.}
\end{itemize}
Acto de referver.
\section{Refestela}
\begin{itemize}
\item {Grp. gram.:f.}
\end{itemize}
O mesmo que \textunderscore refestêlo\textunderscore . Cf. \textunderscore Eufrosina\textunderscore , 280.
\section{Refestelar-se}
\begin{itemize}
\item {Grp. gram.:v. p.}
\end{itemize}
\begin{itemize}
\item {Proveniência:(De \textunderscore refestêlo\textunderscore )}
\end{itemize}
Comprazer-se.
Foliar.
Recostar-se, repimpar-se.
\section{Refestêlo}
\begin{itemize}
\item {Grp. gram.:m.}
\end{itemize}
\begin{itemize}
\item {Utilização:Ant.}
\end{itemize}
\begin{itemize}
\item {Proveniência:(De \textunderscore re...\textunderscore  + \textunderscore festa\textunderscore )}
\end{itemize}
Folia; festa.
Estado de quem se refestela.
\section{Reféz}
\begin{itemize}
\item {Grp. gram.:adj.}
\end{itemize}
(V.refece)
\section{Refia}
\begin{itemize}
\item {Grp. gram.:f.}
\end{itemize}
\begin{itemize}
\item {Utilização:Ant.}
\end{itemize}
\begin{itemize}
\item {Proveniência:(De \textunderscore refiar\textunderscore ?)}
\end{itemize}
Ostentação.
Honraria.
\section{Refião}
\begin{itemize}
\item {Grp. gram.:m.}
\end{itemize}
O mesmo que \textunderscore rufião\textunderscore :«\textunderscore não vos fieis, porque aquillo é refião.\textunderscore »G. Vicente, \textunderscore Auto da India\textunderscore . Cf. B. Pereira, \textunderscore Prosódia\textunderscore , vb. \textunderscore lutum\textunderscore .
\section{Refiar}
\begin{itemize}
\item {Grp. gram.:v. t.}
\end{itemize}
\begin{itemize}
\item {Proveniência:(De \textunderscore re...\textunderscore  + \textunderscore fiar\textunderscore )}
\end{itemize}
Fiar de novo.
\section{Refilador}
\begin{itemize}
\item {Grp. gram.:adj.}
\end{itemize}
Que refila.
\section{Refilão}
\begin{itemize}
\item {Grp. gram.:m.  e  adj.}
\end{itemize}
O que refila.
\section{Refilar}
\begin{itemize}
\item {Grp. gram.:v. i.}
\end{itemize}
\begin{itemize}
\item {Utilização:Fig.}
\end{itemize}
\begin{itemize}
\item {Grp. gram.:V. t. Loc.}
\end{itemize}
\begin{itemize}
\item {Utilização:fam.}
\end{itemize}
\begin{itemize}
\item {Proveniência:(De \textunderscore re...\textunderscore  + \textunderscore filar\textunderscore )}
\end{itemize}
Filar novamente.
Morder em quem o morde ou em quem lhe bate, (falando-se do cão).
Reagir; recalcitrar; redarguir.
\textunderscore Refilar o dente\textunderscore , recalcitrar.
\section{Refilhar}
\begin{itemize}
\item {Grp. gram.:v. i.}
\end{itemize}
\begin{itemize}
\item {Utilização:Fig.}
\end{itemize}
Lançar refilhos.
Diffundir-se; multiplicar-se.
\section{Refilho}
\begin{itemize}
\item {Grp. gram.:m.}
\end{itemize}
\begin{itemize}
\item {Proveniência:(De \textunderscore re...\textunderscore  + \textunderscore filho\textunderscore )}
\end{itemize}
Rebento dos vegetaes.
\section{Refiltrar}
\begin{itemize}
\item {Grp. gram.:v. t.}
\end{itemize}
Filtrar novamente.
(Do \textunderscore re...\textunderscore  + \textunderscore filtrar\textunderscore )
\section{Refinação}
\begin{itemize}
\item {Grp. gram.:f.}
\end{itemize}
Acto ou effeito de refinar.
Lugar, onde se refina.
\section{Refinadamente}
\begin{itemize}
\item {Grp. gram.:adv.}
\end{itemize}
De modo refinado; requintadamente.
\section{Refinado}
\begin{itemize}
\item {Grp. gram.:adj.}
\end{itemize}
Que se refinou: \textunderscore açúcar refinado\textunderscore .
Requintado.
Completo: \textunderscore é um refinado tolo\textunderscore .
\section{Refinador}
\begin{itemize}
\item {Grp. gram.:m.  e  adj.}
\end{itemize}
O que refina.
\section{Refinadura}
\begin{itemize}
\item {Grp. gram.:f.}
\end{itemize}
O mesmo que \textunderscore refinação\textunderscore .
\section{Refinamento}
\begin{itemize}
\item {Grp. gram.:m.}
\end{itemize}
\begin{itemize}
\item {Utilização:Fig.}
\end{itemize}
\begin{itemize}
\item {Proveniência:(De \textunderscore refinar\textunderscore )}
\end{itemize}
Refinação.
Excesso, requinte.
Subtileza.
\section{Refinar}
\begin{itemize}
\item {Grp. gram.:v. t.}
\end{itemize}
\begin{itemize}
\item {Utilização:Fig.}
\end{itemize}
\begin{itemize}
\item {Grp. gram.:V. i.}
\end{itemize}
\begin{itemize}
\item {Proveniência:(De \textunderscore re...\textunderscore  + \textunderscore fino\textunderscore )}
\end{itemize}
Tornar mais fino, apurar: \textunderscore refinar aguardente\textunderscore .
Aperfeiçoar.
Tornar subtil.
Requintar.
Tornar mais saliente.
Aperfeiçoar-se.
Tornar-se mais intenso ou mais forte.
Apurar-se.
Requintar-se: \textunderscore aquella vaidade refinou\textunderscore .
\section{Refinaria}
\begin{itemize}
\item {Grp. gram.:f.}
\end{itemize}
\begin{itemize}
\item {Proveniência:(De \textunderscore refinar\textunderscore )}
\end{itemize}
Officina de refinação.
\section{Refincar}
\begin{itemize}
\item {Grp. gram.:v. i.}
\end{itemize}
\begin{itemize}
\item {Proveniência:(De \textunderscore re...\textunderscore  + \textunderscore fincar\textunderscore )}
\end{itemize}
Fincar fortemente.
\section{Refino}
\begin{itemize}
\item {Grp. gram.:m.}
\end{itemize}
(V.refinação)
\section{Refle}
\begin{itemize}
\item {Grp. gram.:m.}
\end{itemize}
\begin{itemize}
\item {Proveniência:(Do ingl. \textunderscore rifle\textunderscore )}
\end{itemize}
Espingarda curta, espécie de bacamarte.
\section{Reflectidamente}
\begin{itemize}
\item {Grp. gram.:adv.}
\end{itemize}
De modo reflectido; com reflexão; prudentemente.
\section{Reflectido}
\begin{itemize}
\item {Grp. gram.:adj.}
\end{itemize}
\begin{itemize}
\item {Proveniência:(De \textunderscore reflectir\textunderscore )}
\end{itemize}
Prudente; sensato.
\section{Reflectidor}
\begin{itemize}
\item {Grp. gram.:adv.}
\end{itemize}
\begin{itemize}
\item {Grp. gram.:M.}
\end{itemize}
Que reflecte.
Superfície que reflecte.
Quebra-luz; pantalha.
\section{Reflectir}
\begin{itemize}
\item {Grp. gram.:v. t.}
\end{itemize}
\begin{itemize}
\item {Grp. gram.:V. t.}
\end{itemize}
\begin{itemize}
\item {Proveniência:(Lat. \textunderscore reflectere\textunderscore )}
\end{itemize}
Fazer voltar para trás.
Fazer retroceder.
Reproduzir: \textunderscore o espelho reflecte a imagem\textunderscore .
Repercutir: \textunderscore a encosta reflectia os gritos\textunderscore .
Imitar.
Exprimir.
Ponderar, observar, objectar.
Traduzir.
Mudar de direcção.
Incidir.
Fazer reflexões, meditar.
Pensar maduramente.
\section{Reflectivo}
\begin{itemize}
\item {Grp. gram.:adj.}
\end{itemize}
Que pensa maduramente; que reflecte.
\section{Reflector}
\begin{itemize}
\item {Grp. gram.:m.  e  adj.}
\end{itemize}
O mesmo que \textunderscore reflectidor\textunderscore .
\section{Reflexamente}
\begin{itemize}
\item {fónica:csa}
\end{itemize}
\begin{itemize}
\item {Grp. gram.:adv.}
\end{itemize}
De modo reflexo.
\section{Reflexão}
\begin{itemize}
\item {fónica:csão}
\end{itemize}
\begin{itemize}
\item {Grp. gram.:f.}
\end{itemize}
\begin{itemize}
\item {Proveniência:(Lat. \textunderscore reflectio\textunderscore )}
\end{itemize}
Acto ou effeito de reflectir.
Reflexo.
Meditação; ponderação.
Prudência, tino.
\section{Reflexar}
\begin{itemize}
\item {fónica:csar}
\end{itemize}
\begin{itemize}
\item {Grp. gram.:v. t.  e  i.}
\end{itemize}
\begin{itemize}
\item {Utilização:Ant.}
\end{itemize}
\begin{itemize}
\item {Utilização:Pop.}
\end{itemize}
O mesmo que \textunderscore reflexionar\textunderscore .
(Cp. \textunderscore reflexão\textunderscore )
\section{Reflexibilidade}
\begin{itemize}
\item {fónica:csi}
\end{itemize}
\begin{itemize}
\item {Grp. gram.:f.}
\end{itemize}
Qualidade do que é reflexível.
\section{Reflexionar}
\begin{itemize}
\item {fónica:csi}
\end{itemize}
\begin{itemize}
\item {Grp. gram.:v. i.}
\end{itemize}
\begin{itemize}
\item {Proveniência:(Do lat. \textunderscore reflexio\textunderscore )}
\end{itemize}
Fazer reflexões, reflectir.
Objectar; ponderar.
\section{Reflexível}
\begin{itemize}
\item {fónica:csi}
\end{itemize}
\begin{itemize}
\item {Grp. gram.:adj.}
\end{itemize}
\begin{itemize}
\item {Proveniência:(De \textunderscore reflexo\textunderscore )}
\end{itemize}
Que se póde reflectir.
\section{Reflexivo}
\begin{itemize}
\item {fónica:csi}
\end{itemize}
\begin{itemize}
\item {Grp. gram.:adj.}
\end{itemize}
\begin{itemize}
\item {Utilização:Gram.}
\end{itemize}
\begin{itemize}
\item {Proveniência:(De \textunderscore reflexo\textunderscore )}
\end{itemize}
Que reflecte, que reflexiona.
Communicativo.
Que tem por complemento directo o pronome pessoal, (falando-se dos verbos).
\section{Reflexo}
\begin{itemize}
\item {fónica:cso}
\end{itemize}
\begin{itemize}
\item {Grp. gram.:adj.}
\end{itemize}
\begin{itemize}
\item {Utilização:Gram.}
\end{itemize}
\begin{itemize}
\item {Grp. gram.:M.}
\end{itemize}
\begin{itemize}
\item {Proveniência:(Lat. \textunderscore reflexus\textunderscore )}
\end{itemize}
Reflectido.
Dobrado sôbre si próprio, (falando-se de órgãos vegetaes).
Indirecto.
O mesmo que \textunderscore reflexivo\textunderscore , (falando-se de verbos).
Effeito da reflexão da luz.
Luz reflectida.
Reflexão do som ou do calor.
Imagem reflectida.
Imitação, reproducção.
Influência indirecta.
\section{Reflorecer}
\textunderscore v. i.\textunderscore  (e der.)
O mesmo que \textunderscore reflorescer\textunderscore , etc.
\section{Reflorente}
\begin{itemize}
\item {Grp. gram.:adj.}
\end{itemize}
\begin{itemize}
\item {Proveniência:(Lat. \textunderscore reflorens\textunderscore , \textunderscore reflorentis\textunderscore )}
\end{itemize}
Que refloresce. Cf. Rui Barb., \textunderscore Réplica\textunderscore , 158.
\section{Reflorescência}
\begin{itemize}
\item {Grp. gram.:f.}
\end{itemize}
Qualidade de reflorescente.
\section{Reflorescente}
\begin{itemize}
\item {Grp. gram.:adj.}
\end{itemize}
Que refloresce.
\section{Reflorescer}
\begin{itemize}
\item {Grp. gram.:v. i.}
\end{itemize}
\begin{itemize}
\item {Utilização:Fig.}
\end{itemize}
\begin{itemize}
\item {Proveniência:(Lat. \textunderscore reflorescere\textunderscore )}
\end{itemize}
Florescer novamente.
Encher-se de flôres.
Reanimar-se; restabelecer-se; rejuvenescer.
\section{Reflorescido}
\begin{itemize}
\item {Grp. gram.:adj.}
\end{itemize}
\begin{itemize}
\item {Proveniência:(De \textunderscore reflorescer\textunderscore )}
\end{itemize}
Que refloresceu.
\section{Reflorescimento}
\begin{itemize}
\item {Grp. gram.:m.}
\end{itemize}
Acto ou effeito de reflorescer.
\section{Reflorido}
\begin{itemize}
\item {Grp. gram.:adj.}
\end{itemize}
\begin{itemize}
\item {Proveniência:(De \textunderscore reflorir\textunderscore )}
\end{itemize}
Que refloriu, reflorescido.
\section{Reflorir}
\textunderscore v. i.\textunderscore  (e der.)
O mesmo que \textunderscore reflorescer\textunderscore .
\section{Refluente}
\begin{itemize}
\item {Grp. gram.:adj.}
\end{itemize}
\begin{itemize}
\item {Proveniência:(Lat. \textunderscore refluens\textunderscore )}
\end{itemize}
Que reflue.
\section{Refluir}
\begin{itemize}
\item {Grp. gram.:v. i.}
\end{itemize}
\begin{itemize}
\item {Utilização:Fig.}
\end{itemize}
\begin{itemize}
\item {Proveniência:(Lat. \textunderscore refluere\textunderscore )}
\end{itemize}
Fluir para trás.
Manar ou correr de novo para o ponto donde correu ou manou, (falando-se dos líquidos).
Voltar para trás, retroceder.
\section{Réfluo}
\begin{itemize}
\item {Grp. gram.:adj.}
\end{itemize}
\begin{itemize}
\item {Proveniência:(Lat. \textunderscore refluus\textunderscore )}
\end{itemize}
O mesmo que \textunderscore refluente\textunderscore .
\section{Refluxo}
\begin{itemize}
\item {fónica:cso}
\end{itemize}
\begin{itemize}
\item {Grp. gram.:m.}
\end{itemize}
\begin{itemize}
\item {Proveniência:(De \textunderscore re...\textunderscore  + \textunderscore fluxo\textunderscore )}
\end{itemize}
Acto ou effeito de refluir.
Movimento da maré quando vaza.
Movimento contrário e successivo a outro.
\section{Refocilamento}
\begin{itemize}
\item {Grp. gram.:m.}
\end{itemize}
Acto ou efeito de refocilar.
\section{Refocilante}
\begin{itemize}
\item {Grp. gram.:adj.}
\end{itemize}
\begin{itemize}
\item {Proveniência:(Lat. \textunderscore refocillans\textunderscore )}
\end{itemize}
Que refocila.
\section{Refocilar}
\begin{itemize}
\item {Grp. gram.:v. i.}
\end{itemize}
\begin{itemize}
\item {Grp. gram.:V. p.}
\end{itemize}
\begin{itemize}
\item {Proveniência:(Lat. \textunderscore refoccillare\textunderscore )}
\end{itemize}
Refazer.
Reforçar.
Reconstituir.
Dar folga a.
Recrear-se.
Refestelar-se.
\section{Refocillamento}
\begin{itemize}
\item {Grp. gram.:m.}
\end{itemize}
Acto ou effeito de refocillar.
\section{Refocillante}
\begin{itemize}
\item {Grp. gram.:adj.}
\end{itemize}
\begin{itemize}
\item {Proveniência:(Lat. \textunderscore refocillans\textunderscore )}
\end{itemize}
Que refocilla.
\section{Refocillar}
\begin{itemize}
\item {Grp. gram.:v. i.}
\end{itemize}
\begin{itemize}
\item {Grp. gram.:V. p.}
\end{itemize}
\begin{itemize}
\item {Proveniência:(Lat. \textunderscore refoccillare\textunderscore )}
\end{itemize}
Refazer.
Reforçar.
Reconstituir.
Dar folga a.
Recrear-se.
Refestelar-se.
\section{Refocinhado}
\begin{itemize}
\item {Grp. gram.:adj.}
\end{itemize}
\begin{itemize}
\item {Utilização:Ant.}
\end{itemize}
Dizia-se do cabello encrespado ou riçado.
\section{Refogado}
\begin{itemize}
\item {Grp. gram.:m.}
\end{itemize}
\begin{itemize}
\item {Proveniência:(De \textunderscore refogar\textunderscore )}
\end{itemize}
Môlho, em que entra cebola e outros temperos.
\section{Refogar}
\begin{itemize}
\item {Grp. gram.:v. t.}
\end{itemize}
\begin{itemize}
\item {Proveniência:(De \textunderscore re...\textunderscore  + \textunderscore fogo\textunderscore )}
\end{itemize}
Fazer ferver em azeite ou gordura, (cebola e outros temperos).
Cozinhar com refogado; guisar.
\section{Refojo}
\begin{itemize}
\item {fónica:fô}
\end{itemize}
\begin{itemize}
\item {Grp. gram.:m.}
\end{itemize}
\begin{itemize}
\item {Proveniência:(De \textunderscore fojo\textunderscore )}
\end{itemize}
Recôncavo do terreno.
Caverna ou gruta, em que se abrigam feras.
\section{Refolgar}
\begin{itemize}
\item {Grp. gram.:v. i.}
\end{itemize}
\begin{itemize}
\item {Proveniência:(De \textunderscore refôlgo\textunderscore )}
\end{itemize}
Folgar muito; descansar bem.
O mesmo que \textunderscore resfolgar\textunderscore .
\section{Refôlgo}
\begin{itemize}
\item {Grp. gram.:m.}
\end{itemize}
Descanso, alívio:«\textunderscore tomai um refôlgo aos pesos\textunderscore ». F. Manuel, \textunderscore Apólogos\textunderscore .
(Por \textunderscore refôlego\textunderscore , de \textunderscore re...\textunderscore  + \textunderscore fôlego\textunderscore )
\section{Refolhado}
\begin{itemize}
\item {Grp. gram.:adj.}
\end{itemize}
\begin{itemize}
\item {Proveniência:(De \textunderscore refolhar\textunderscore )}
\end{itemize}
Envolto em fôlhas.
Dissimulado; disfarçado. Cf. \textunderscore Eufrosina\textunderscore , 69.
\section{Refolhamento}
\begin{itemize}
\item {Grp. gram.:m.}
\end{itemize}
O mesmo que \textunderscore refôlho\textunderscore . Cf. \textunderscore Eufrosina\textunderscore , 329.
\section{Refolhar}
\begin{itemize}
\item {Grp. gram.:v. t.}
\end{itemize}
\begin{itemize}
\item {Utilização:Fig.}
\end{itemize}
\begin{itemize}
\item {Proveniência:(De \textunderscore re...\textunderscore  + \textunderscore fôlha\textunderscore )}
\end{itemize}
Envolver em fôlhas.
Disfarçar.
\section{Refôlho}
\begin{itemize}
\item {Grp. gram.:m.}
\end{itemize}
\begin{itemize}
\item {Utilização:Fig.}
\end{itemize}
\begin{itemize}
\item {Proveniência:(De \textunderscore re...\textunderscore  + \textunderscore fôlho\textunderscore )}
\end{itemize}
Prega, fôlho sobreposto a outro.
Disfarce; fingimento: \textunderscore falar sem refolhos\textunderscore .
\section{Refolhudo}
\begin{itemize}
\item {Grp. gram.:adj.}
\end{itemize}
Que tem refôlho ramudo. Cf. Castilho, \textunderscore Geórgicas\textunderscore , 225.
\section{Refomentar}
\begin{itemize}
\item {Grp. gram.:v. t.}
\end{itemize}
\begin{itemize}
\item {Proveniência:(De \textunderscore re...\textunderscore  + \textunderscore fomentar\textunderscore )}
\end{itemize}
Fomentar bem, estimular ou exercitar activamente:«\textunderscore ...não puderam refomentar os membros torpecidos...\textunderscore »Filinto, VI, 263.
\section{Reforçadamente}
\begin{itemize}
\item {Grp. gram.:adv.}
\end{itemize}
De modo reforçado; com refôrço.
\section{Reforçado}
\begin{itemize}
\item {Grp. gram.:adj.}
\end{itemize}
\begin{itemize}
\item {Proveniência:(De \textunderscore reforçar\textunderscore )}
\end{itemize}
Que readquiriu fôrças; que é robusto.
\section{Reforçar}
\begin{itemize}
\item {Grp. gram.:v. t.}
\end{itemize}
\begin{itemize}
\item {Grp. gram.:V. i.  e  p.}
\end{itemize}
\begin{itemize}
\item {Proveniência:(De \textunderscore re...\textunderscore  + \textunderscore força\textunderscore )}
\end{itemize}
Dar mais força a; dar mais solidez a; fortalecer: \textunderscore reforçar argumentos\textunderscore .
Tornar mais numeroso: \textunderscore reforçar tropas\textunderscore .
Adquirir mais força. Tornar-se mais robusto.
\section{Reforçativo}
\begin{itemize}
\item {Grp. gram.:adj.}
\end{itemize}
Quo serve para reforçar.
\section{Refôrço}
\begin{itemize}
\item {Grp. gram.:m.}
\end{itemize}
Acto ou effeito de reforçar.
Aquillo que reforça.
Tropas auxiliares.
Auxilio superveniente.
\section{Reforjar}
\begin{itemize}
\item {Grp. gram.:v. t.}
\end{itemize}
\begin{itemize}
\item {Proveniência:(De \textunderscore re...\textunderscore  + \textunderscore forjar\textunderscore )}
\end{itemize}
Forjar do novo.
\section{Reforma}
\begin{itemize}
\item {Grp. gram.:f.}
\end{itemize}
Acto ou effeito de reformar.
Fórma nova.
O mesmo que \textunderscore Protestantismo\textunderscore .
\section{Reformação}
\begin{itemize}
\item {Grp. gram.:f.}
\end{itemize}
\begin{itemize}
\item {Proveniência:(Lat. \textunderscore reformatio\textunderscore )}
\end{itemize}
O mesmo que \textunderscore reforma\textunderscore : \textunderscore a reformação dos costumes\textunderscore .
\section{Reformado}
\begin{itemize}
\item {Grp. gram.:m.}
\end{itemize}
\begin{itemize}
\item {Proveniência:(De \textunderscore reformar\textunderscore )}
\end{itemize}
Militar, que se reformou.
Aquelle que segue uma religião reformada; protestante.
\section{Reformador}
\begin{itemize}
\item {Grp. gram.:m.  e  adj.}
\end{itemize}
\begin{itemize}
\item {Proveniência:(Lat. \textunderscore reformator\textunderscore )}
\end{itemize}
O que reforma.
\section{Reformar}
\begin{itemize}
\item {Grp. gram.:v. t.}
\end{itemize}
\begin{itemize}
\item {Proveniência:(De \textunderscore re...\textunderscore  + \textunderscore forma\textunderscore )}
\end{itemize}
Dar nova fórma a.
Mudar a fórma de.
Modificar.
Melhorar.
Organizar novamente.
Corrigir: \textunderscore reformar um trabalho literário\textunderscore .
Abastecer.
Aposentar.
\section{Reformativo}
\begin{itemize}
\item {Grp. gram.:adj.}
\end{itemize}
Relativo a reforma; próprio para reformar.
\section{Reformatório}
\begin{itemize}
\item {Grp. gram.:adj.}
\end{itemize}
\begin{itemize}
\item {Grp. gram.:M.}
\end{itemize}
\begin{itemize}
\item {Proveniência:(De \textunderscore reformar\textunderscore )}
\end{itemize}
Que reforma.
Conjunto de preceitos moraes ou instructivos.
\section{Reformatriz}
\begin{itemize}
\item {Grp. gram.:adj. f.}
\end{itemize}
Que faz reformas, (falando-se de uma mulhér, de uma lei ou de outro agente feminino). Cf. Camillo, \textunderscore Pombal\textunderscore , 192.
(Flexão fem. de \textunderscore reformador\textunderscore )
\section{Reformável}
\begin{itemize}
\item {Grp. gram.:adj.}
\end{itemize}
Que se póde reformar.
\section{Reformista}
\begin{itemize}
\item {Grp. gram.:adj.}
\end{itemize}
\begin{itemize}
\item {Grp. gram.:M.}
\end{itemize}
Relativo a reforma ou aos reformistas.
Sectário de política reformadora ou da reforma nos costumes.
Designação de um partido politico, que preconizava a reforma da \textunderscore Carta Constitucional\textunderscore .
\section{Reformular}
\begin{itemize}
\item {Grp. gram.:v. t.}
\end{itemize}
\begin{itemize}
\item {Proveniência:(De \textunderscore re...\textunderscore  + \textunderscore formular\textunderscore )}
\end{itemize}
Tornar a formular. Cf. Camillo, \textunderscore Noites de Insómn.\textunderscore , VII, 92.
\section{Refornecer}
\begin{itemize}
\item {Grp. gram.:v. t.}
\end{itemize}
\begin{itemize}
\item {Proveniência:(De \textunderscore re...\textunderscore  + \textunderscore fornecer\textunderscore )}
\end{itemize}
Tornar a fornecer.
\section{Refornecimento}
\begin{itemize}
\item {Grp. gram.:m.}
\end{itemize}
Acto ou effeito de refornecer.
\section{Refortificar}
\begin{itemize}
\item {Grp. gram.:v. t.}
\end{itemize}
\begin{itemize}
\item {Proveniência:(De \textunderscore re...\textunderscore  + \textunderscore fortificar\textunderscore )}
\end{itemize}
Fortificar de novo.
\section{Refossete}
\begin{itemize}
\item {fónica:sê}
\end{itemize}
\begin{itemize}
\item {Grp. gram.:m.}
\end{itemize}
Pequeno fôsso, que se abria antigamente a meio do fôsso sêco de uma fortificação, para difficultar o assalto.
(Do \textunderscore re...\textunderscore  + \textunderscore fôsso\textunderscore )
\section{Refracção}
\begin{itemize}
\item {Grp. gram.:f.}
\end{itemize}
\begin{itemize}
\item {Proveniência:(Do lat. \textunderscore refractio\textunderscore )}
\end{itemize}
Acto ou effeito de refractar ou de refranger.
\section{Refractar}
\begin{itemize}
\item {Grp. gram.:v. t.}
\end{itemize}
\begin{itemize}
\item {Proveniência:(De \textunderscore refrocto\textunderscore )}
\end{itemize}
Quebrar; tornar reflectido.
O mesmo que \textunderscore refranger\textunderscore .
\section{Refractário}
\begin{itemize}
\item {Grp. gram.:adj.}
\end{itemize}
\begin{itemize}
\item {Grp. gram.:M.}
\end{itemize}
\begin{itemize}
\item {Proveniência:(Lat. \textunderscore refractarius\textunderscore )}
\end{itemize}
Resistente; rebelde; obstinado.
Aquelle que se não apresenta ás autoridades, subtrahindo-se ao cumprimento do serviço militar.
\section{Refractivo}
\begin{itemize}
\item {Grp. gram.:adj.}
\end{itemize}
\begin{itemize}
\item {Proveniência:(Lat. \textunderscore refractivus\textunderscore )}
\end{itemize}
Que refrange ou faz refractar.
\section{Refracto}
\begin{itemize}
\item {Grp. gram.:adj.}
\end{itemize}
\begin{itemize}
\item {Proveniência:(Lat. \textunderscore refractus\textunderscore )}
\end{itemize}
Que se refrangeu; que soffreu refracção. Cf. J. A. Macedo, \textunderscore Oriente\textunderscore , I.
\section{Refractómetro}
\begin{itemize}
\item {Grp. gram.:m.}
\end{itemize}
\begin{itemize}
\item {Proveniência:(T. hýbr., do lat. \textunderscore refractus\textunderscore  + gr. \textunderscore metron\textunderscore )}
\end{itemize}
Apparelho óptico, para se averiguar o índice de refracção dos corpos.
\section{Refraneiro}
\begin{itemize}
\item {Grp. gram.:m.}
\end{itemize}
Aquelle que faz refrãos. Cf. J. Ribeiro, \textunderscore Frases Feitas\textunderscore , I, 226.
\section{Refrangente}
\begin{itemize}
\item {Grp. gram.:adj.}
\end{itemize}
Que refrange.
\section{Refranger}
\begin{itemize}
\item {Grp. gram.:v. t.}
\end{itemize}
\begin{itemize}
\item {Proveniência:(Do lat. \textunderscore re...\textunderscore  + \textunderscore frangere\textunderscore )}
\end{itemize}
O mesmo que \textunderscore refractar\textunderscore .
\section{Refrangibilidade}
\begin{itemize}
\item {Grp. gram.:f.}
\end{itemize}
Qualidade do que é refrangível.
\section{Refrangível}
\begin{itemize}
\item {Grp. gram.:adj.}
\end{itemize}
Que se póde refranger.
\section{Refranzear}
\begin{itemize}
\item {Grp. gram.:v. i.}
\end{itemize}
\begin{itemize}
\item {Utilização:Des.}
\end{itemize}
\begin{itemize}
\item {Proveniência:(De \textunderscore refrão\textunderscore )}
\end{itemize}
Dizer gracejos.
\section{Refrão}
\begin{itemize}
\item {Grp. gram.:m.}
\end{itemize}
\begin{itemize}
\item {Proveniência:(Do fr. \textunderscore refrain\textunderscore )}
\end{itemize}
Adágio, anexim; estribilho. Cf. \textunderscore Eufrosina\textunderscore , 154.
\section{Refreadamente}
\begin{itemize}
\item {Grp. gram.:adv.}
\end{itemize}
De modo refreado; com moderação.
\section{Refreadoiro}
\begin{itemize}
\item {Grp. gram.:m.}
\end{itemize}
\begin{itemize}
\item {Utilização:Des.}
\end{itemize}
\begin{itemize}
\item {Utilização:Fig.}
\end{itemize}
\begin{itemize}
\item {Proveniência:(De \textunderscore refrear\textunderscore )}
\end{itemize}
Freio.
Aquillo que modera ou refreia os maus instintos ou maus costumes.
\section{Refreador}
\begin{itemize}
\item {Grp. gram.:m.  e  adj.}
\end{itemize}
O que refreia.
\section{Refreadouro}
\begin{itemize}
\item {Grp. gram.:m.}
\end{itemize}
\begin{itemize}
\item {Utilização:Des.}
\end{itemize}
\begin{itemize}
\item {Utilização:Fig.}
\end{itemize}
\begin{itemize}
\item {Proveniência:(De \textunderscore refrear\textunderscore )}
\end{itemize}
Freio.
Aquillo que modera ou refreia os maus instintos ou maus costumes.
\section{Refreamento}
\begin{itemize}
\item {Grp. gram.:m.}
\end{itemize}
Acto ou effeito de refrear.
\section{Refrear}
\begin{itemize}
\item {Grp. gram.:v. t.}
\end{itemize}
\begin{itemize}
\item {Utilização:Fig.}
\end{itemize}
\begin{itemize}
\item {Proveniência:(Do lat. \textunderscore refrenare\textunderscore )}
\end{itemize}
Dominar com freio; pôr freio a.
Moderar.
Sujeitar; reprimir; conter.
\section{Refreável}
\begin{itemize}
\item {Grp. gram.:adj.}
\end{itemize}
Que se póde refrear.
\section{Refrega}
\begin{itemize}
\item {Grp. gram.:f.}
\end{itemize}
\begin{itemize}
\item {Proveniência:(De \textunderscore refregar\textunderscore )}
\end{itemize}
Briga; luta entre inimigos.
Recontro.
Peleja.
Trabalho: \textunderscore as refregas da vida\textunderscore .
\section{Refregar}
\begin{itemize}
\item {Grp. gram.:v. i.}
\end{itemize}
\begin{itemize}
\item {Proveniência:(Do lat. \textunderscore refragari\textunderscore )}
\end{itemize}
Travar peleja; lutar, brigar. Cf. \textunderscore Ethiópia Or.\textunderscore , II, 332.
\section{Refreio}
\begin{itemize}
\item {Grp. gram.:m.}
\end{itemize}
Acto de refrear; aquillo com que se refreia.
\section{Refrescada}
\begin{itemize}
\item {Grp. gram.:f.}
\end{itemize}
\begin{itemize}
\item {Utilização:Des.}
\end{itemize}
Grande porção de refrescos.
Reunião de soccorros militares.
\section{Refrescamento}
\begin{itemize}
\item {Grp. gram.:m.}
\end{itemize}
Acto ou effeito de refrescar.
\section{Refrescante}
\begin{itemize}
\item {Grp. gram.:adj.}
\end{itemize}
Que refresca.
\section{Refrescar}
\begin{itemize}
\item {Grp. gram.:v. t.}
\end{itemize}
\begin{itemize}
\item {Grp. gram.:V. i.}
\end{itemize}
\begin{itemize}
\item {Proveniência:(De \textunderscore refrêsco\textunderscore )}
\end{itemize}
Tornar fresco, fazer mais fresco.
Deminuir o calor de.
Refrigerar.
Alliviar; suavizar.
Dar alento a.
Soccorrer.
Tornar-se fresco, arrefecer.
Fazer provisões de alimentos.
Tranquillizar-se.
Aumentar de intensidade (o vento). Cf. \textunderscore Rot. do Mar Verm.\textunderscore 
\section{Refrescata}
\begin{itemize}
\item {Grp. gram.:f.}
\end{itemize}
Refrescada.
Acto ou effeito de refrescar.
\section{Refrescativo}
\begin{itemize}
\item {Grp. gram.:adj.}
\end{itemize}
Que refresca; que serve para refrescar.
\section{Refrêsco}
\begin{itemize}
\item {Grp. gram.:m.}
\end{itemize}
\begin{itemize}
\item {Utilização:Des.}
\end{itemize}
\begin{itemize}
\item {Proveniência:(De \textunderscore re...\textunderscore  + \textunderscore frêsco\textunderscore )}
\end{itemize}
Aquillo que refresca.
Effeito de refrescar.
Comida ou bebida, que refrigera ou refresca: \textunderscore tomar refrescos\textunderscore .
Refrigério.
Reforços, auxílios:«\textunderscore vinham estes enemigos de refrêsco...\textunderscore »Usque, 47 v.^o.
\section{Refretar}
\begin{itemize}
\item {Grp. gram.:v. t.}
\end{itemize}
\begin{itemize}
\item {Proveniência:(De \textunderscore re...\textunderscore  + \textunderscore fretar\textunderscore )}
\end{itemize}
Fretar de novo:«\textunderscore ...e refretando a barca de Charonte...\textunderscore »Filinto, I, 164.
\section{Refricar}
\begin{itemize}
\item {Grp. gram.:v. i.}
\end{itemize}
\begin{itemize}
\item {Utilização:Bras}
\end{itemize}
\begin{itemize}
\item {Utilização:pop.}
\end{itemize}
Dar o cavaco, melindrar-se. Cf. Pacheco, \textunderscore Promptuário\textunderscore .
\section{Refrigeração}
\begin{itemize}
\item {Grp. gram.:f.}
\end{itemize}
\begin{itemize}
\item {Proveniência:(Lat. \textunderscore refrigeratio\textunderscore )}
\end{itemize}
Acto ou effeito de refrigerar.
\section{Refrigerador}
\begin{itemize}
\item {Grp. gram.:m.}
\end{itemize}
Instrumento, para refrigerar. Cf. \textunderscore Techn. Rur.\textunderscore , 255.
\section{Refrigerante}
\begin{itemize}
\item {Grp. gram.:adj.}
\end{itemize}
\begin{itemize}
\item {Grp. gram.:M.}
\end{itemize}
\begin{itemize}
\item {Proveniência:(Lat. \textunderscore refrigerans\textunderscore )}
\end{itemize}
Que refrigera.
Bebida que refresca; refrêsco.
\section{Refrigerar}
\begin{itemize}
\item {Grp. gram.:v. t.}
\end{itemize}
\begin{itemize}
\item {Proveniência:(Lat. \textunderscore refrigerare\textunderscore )}
\end{itemize}
Refrescar.
Suavizar; consolar.
\section{Refrigerativo}
\begin{itemize}
\item {Grp. gram.:m.  e  adj.}
\end{itemize}
O mesmo que \textunderscore refrigerante\textunderscore .
\section{Refrigeratório}
\begin{itemize}
\item {Grp. gram.:adj.}
\end{itemize}
\begin{itemize}
\item {Proveniência:(Lat. \textunderscore refrigeratorius\textunderscore )}
\end{itemize}
Próprio para refrigerar; refrigerante.
\section{Refrigério}
\begin{itemize}
\item {Grp. gram.:m.}
\end{itemize}
\begin{itemize}
\item {Proveniência:(Lat. \textunderscore refrigerium\textunderscore )}
\end{itemize}
Acto ou effeito de refrigerar.
Allívio, produzido pela frescura.
Consolação.
\section{Refrígero}
\begin{itemize}
\item {Grp. gram.:adj.}
\end{itemize}
\begin{itemize}
\item {Utilização:Poét.}
\end{itemize}
Que refrigéra:«\textunderscore ...donde as chuvas rifrígeras dimanão.\textunderscore »Filinto, IV, 256.
\section{Refringente}
\begin{itemize}
\item {Grp. gram.:adj.}
\end{itemize}
\begin{itemize}
\item {Proveniência:(Lat. \textunderscore refringens\textunderscore )}
\end{itemize}
O mesmo que \textunderscore refractivo\textunderscore .
\section{Refrondar}
\begin{itemize}
\item {Grp. gram.:v. t.}
\end{itemize}
\begin{itemize}
\item {Proveniência:(De \textunderscore re...\textunderscore  + \textunderscore fronde\textunderscore )}
\end{itemize}
Cobrir de folhagem, revestir de fôlhas. Cf. Castilho, \textunderscore Geórgicas\textunderscore , 105.
\section{Refrondescer}
\begin{itemize}
\item {Grp. gram.:v. i.}
\end{itemize}
\begin{itemize}
\item {Proveniência:(De \textunderscore re...\textunderscore  + \textunderscore frondescer\textunderscore )}
\end{itemize}
Frondescer novamente.
\section{Refugador}
\begin{itemize}
\item {Grp. gram.:m.  e  adj.}
\end{itemize}
O que refuga.
\section{Refugar}
\begin{itemize}
\item {Grp. gram.:v. t.}
\end{itemize}
\begin{itemize}
\item {Utilização:Bras. do S}
\end{itemize}
\begin{itemize}
\item {Proveniência:(Lat. \textunderscore refugare\textunderscore )}
\end{itemize}
Rejeitar como inútil. Desprezar; rejeitar.
Separar, apartar (gado, etc.).
\section{Refugiado}
\begin{itemize}
\item {Grp. gram.:m.}
\end{itemize}
\begin{itemize}
\item {Proveniência:(De \textunderscore refugiar-se\textunderscore )}
\end{itemize}
Aquelle que se refugiou.
\section{Refugiar-se}
\begin{itemize}
\item {Grp. gram.:v. p.}
\end{itemize}
\begin{itemize}
\item {Utilização:Fig.}
\end{itemize}
\begin{itemize}
\item {Proveniência:(Lat. \textunderscore refugere\textunderscore )}
\end{itemize}
Esconder-se ou abrigar-se.
Expatriar-se.
Procurar abrigo ou protecção.
\section{Refúgio}
\begin{itemize}
\item {Grp. gram.:m.}
\end{itemize}
\begin{itemize}
\item {Utilização:Fig.}
\end{itemize}
\begin{itemize}
\item {Proveniência:(Lat. \textunderscore refugium\textunderscore )}
\end{itemize}
Lugar, onde alguém se refugia.
Abrigo; asilo.
Amparo, auxílio; recurso.
\section{Refugir}
\begin{itemize}
\item {Grp. gram.:v. i.}
\end{itemize}
\begin{itemize}
\item {Grp. gram.:V. t.}
\end{itemize}
\begin{itemize}
\item {Proveniência:(Lat. \textunderscore refugere\textunderscore )}
\end{itemize}
Tornar a fugir; retroceder.
Procurar escapar.
Eximir-se.
Evitar; desviar-se de: \textunderscore refugir tentações\textunderscore .
\section{Refugo}
\begin{itemize}
\item {Grp. gram.:m.}
\end{itemize}
\begin{itemize}
\item {Proveniência:(De \textunderscore refugar\textunderscore )}
\end{itemize}
Aquillo que se refugou; rebotalho; resto.
\section{Refugo}
\begin{itemize}
\item {Grp. gram.:m.}
\end{itemize}
\begin{itemize}
\item {Proveniência:(De um hyp. \textunderscore refugar\textunderscore  = \textunderscore refogar\textunderscore )}
\end{itemize}
O mesmo que \textunderscore refogado\textunderscore .
\section{Refulgência}
\begin{itemize}
\item {Grp. gram.:f.}
\end{itemize}
\begin{itemize}
\item {Proveniência:(Lat. \textunderscore refulgens\textunderscore )}
\end{itemize}
Qualidade do que é refulgente.
\section{Refulgir}
\begin{itemize}
\item {Grp. gram.:v. t.}
\end{itemize}
\begin{itemize}
\item {Utilização:Fig.}
\end{itemize}
\begin{itemize}
\item {Proveniência:(Lat. \textunderscore refulgere\textunderscore )}
\end{itemize}
Brilhar muito; resplandecer.
Tornar-se distinto.
Transparecer.
\section{Refundar}
\begin{itemize}
\item {Grp. gram.:v. t.}
\end{itemize}
\begin{itemize}
\item {Proveniência:(De \textunderscore re...\textunderscore  + \textunderscore fundo\textunderscore )}
\end{itemize}
Tornar mais fundo; profundar.
\section{Refundição}
\begin{itemize}
\item {Grp. gram.:f.}
\end{itemize}
Acto ou effeito de refundir.
\section{Refundidor}
\begin{itemize}
\item {Grp. gram.:m.}
\end{itemize}
Aquelle quo refunde. Cf. Camilo, \textunderscore Corja\textunderscore , X.
\section{Refundir}
\begin{itemize}
\item {Grp. gram.:v. t.}
\end{itemize}
\begin{itemize}
\item {Proveniência:(Lat. \textunderscore refundere\textunderscore )}
\end{itemize}
Fundir novamente, derreter de novo.
Transmudar de um vaso para outro.
Reformar; corrigir: \textunderscore refundi o diccionário\textunderscore .
\section{Refunfar}
\begin{itemize}
\item {Grp. gram.:v. i.}
\end{itemize}
\begin{itemize}
\item {Utilização:Prov.}
\end{itemize}
\begin{itemize}
\item {Utilização:beir.}
\end{itemize}
O mesmo que \textunderscore resmungar\textunderscore .
(Cp. cast. \textunderscore refunfañar\textunderscore )
\section{Refunfumegar}
\begin{itemize}
\item {Grp. gram.:v. i.}
\end{itemize}
\begin{itemize}
\item {Utilização:T. da Bairrada}
\end{itemize}
\begin{itemize}
\item {Proveniência:(T. onom.)}
\end{itemize}
Suspirar.
Resfolegar.
\section{Refusão}
\begin{itemize}
\item {Grp. gram.:f.}
\end{itemize}
Acto de refusar. Cf. Castilho, \textunderscore Metam.\textunderscore , 163.
\section{Refusar}
\begin{itemize}
\item {Proveniência:(Lat. \textunderscore refusare\textunderscore )}
\end{itemize}
\textunderscore v. t.\textunderscore  (e der.)
O mesmo que \textunderscore recusar\textunderscore , etc. Cf. Filinto, \textunderscore D. Man.\textunderscore , I, 119. Camillo, \textunderscore Caveira\textunderscore , 226; Herculano, \textunderscore Abóbada\textunderscore , etc.
\section{Refustão}
\begin{itemize}
\item {Grp. gram.:m.}
\end{itemize}
\begin{itemize}
\item {Utilização:Prov.}
\end{itemize}
\begin{itemize}
\item {Utilização:trasm.}
\end{itemize}
Repellão de bêstas para um lado, espantando-se.
\section{Refustar}
\begin{itemize}
\item {Grp. gram.:v. i.}
\end{itemize}
\begin{itemize}
\item {Utilização:Prov.}
\end{itemize}
\begin{itemize}
\item {Utilização:minh.}
\end{itemize}
Fazer (o Sol) grande calor, em sítios mal arejados e por isso incômmodos.
(Cp. \textunderscore ustão\textunderscore )
\section{Refuste}
\begin{itemize}
\item {Grp. gram.:m.}
\end{itemize}
\begin{itemize}
\item {Utilização:Prov.}
\end{itemize}
\begin{itemize}
\item {Utilização:minh.}
\end{itemize}
\begin{itemize}
\item {Proveniência:(De \textunderscore refustar\textunderscore )}
\end{itemize}
Calor incômmodo do sol, em sítios mal arejados.
\section{Refutação}
\begin{itemize}
\item {Grp. gram.:f.}
\end{itemize}
\begin{itemize}
\item {Proveniência:(Lat. \textunderscore refutatio\textunderscore )}
\end{itemize}
Acto ou effeito de refutar; contestação.
\section{Refutador}
\begin{itemize}
\item {Grp. gram.:m.  e  adj.}
\end{itemize}
\begin{itemize}
\item {Proveniência:(Lat. \textunderscore refutator\textunderscore )}
\end{itemize}
O que refuta.
\section{Refutar}
\begin{itemize}
\item {Grp. gram.:v. t.}
\end{itemize}
\begin{itemize}
\item {Proveniência:(Lat. \textunderscore refutare\textunderscore )}
\end{itemize}
Dizer em contrário; redarguir.
Desmentir.
Desapprovar.
Combater com argumentos.
\section{Refutatório}
\begin{itemize}
\item {Grp. gram.:adj.}
\end{itemize}
\begin{itemize}
\item {Proveniência:(Lat. \textunderscore refutatorius\textunderscore )}
\end{itemize}
Que refuta ou que serve para refutar.
\section{Refutável}
\begin{itemize}
\item {Grp. gram.:adj.}
\end{itemize}
\begin{itemize}
\item {Proveniência:(Do lat. \textunderscore refutabilis\textunderscore )}
\end{itemize}
Que se póde refutar.
\section{Rega}
\begin{itemize}
\item {Grp. gram.:f.}
\end{itemize}
\begin{itemize}
\item {Utilização:Pop.}
\end{itemize}
Acto ou effeito de regar.
Chuva.
\section{Règabofe}
\begin{itemize}
\item {Grp. gram.:m.}
\end{itemize}
\begin{itemize}
\item {Utilização:Fam.}
\end{itemize}
\begin{itemize}
\item {Proveniência:(De \textunderscore regar\textunderscore  + \textunderscore bofe\textunderscore )}
\end{itemize}
Festa popular.
Folguedo; grande divertimento.
\section{Regaça}
\begin{itemize}
\item {Grp. gram.:f.}
\end{itemize}
\begin{itemize}
\item {Utilização:Ant.}
\end{itemize}
Retaguarda do exército.
\section{Regaçada}
\begin{itemize}
\item {Grp. gram.:f.}
\end{itemize}
O mesmo que \textunderscore arregaçada\textunderscore . Cf. Rebello, \textunderscore Contos e Lendas\textunderscore , 147.
\section{Regaçar}
\begin{itemize}
\item {Grp. gram.:v. t.  e  p.}
\end{itemize}
O mesmo que \textunderscore arregaçar\textunderscore .
\section{Regaço}
\begin{itemize}
\item {Grp. gram.:m.}
\end{itemize}
\begin{itemize}
\item {Utilização:Fig.}
\end{itemize}
\begin{itemize}
\item {Grp. gram.:Pl.}
\end{itemize}
\begin{itemize}
\item {Utilização:Ant.}
\end{itemize}
Dobra ou concavidade, formada por vestuário comprido entre a cintura e os joelhos de pessôa sentada.
Dobra, que o vestido fórma levantando-se adeante.
Espaço médio, interior.
Lugar, onde se descansa.
Tiras de seda ou de outro pano, que se cosiam atrás e adeante das alvas sacerdotaes.
(Cast. \textunderscore regazo\textunderscore )
\section{Regada}
\begin{itemize}
\item {Grp. gram.:f.}
\end{itemize}
Propriedade rústica, que é regadia.
(Fem. de \textunderscore regado\textunderscore )
\section{Regadeira}
\begin{itemize}
\item {Grp. gram.:f.}
\end{itemize}
\begin{itemize}
\item {Utilização:Des.}
\end{itemize}
\begin{itemize}
\item {Proveniência:(De \textunderscore regar\textunderscore )}
\end{itemize}
Regueira; enxurrada.
\section{Regadia}
\begin{itemize}
\item {Grp. gram.:f.}
\end{itemize}
Acto ou effeito de regar.
Terra que é regada.
(Fem. de \textunderscore regadio\textunderscore )
\section{Regadinho}
\begin{itemize}
\item {Grp. gram.:m.}
\end{itemize}
\begin{itemize}
\item {Utilização:Prov.}
\end{itemize}
\begin{itemize}
\item {Utilização:minh.}
\end{itemize}
\begin{itemize}
\item {Utilização:dur.}
\end{itemize}
\begin{itemize}
\item {Proveniência:(De \textunderscore regado\textunderscore )}
\end{itemize}
Espécie de fado.
Dança e música popular.
\section{Regadio}
\begin{itemize}
\item {Grp. gram.:adj.}
\end{itemize}
\begin{itemize}
\item {Grp. gram.:M.}
\end{itemize}
Que se rega, (falando-se de terrenos).
Acto de regar.
\section{Regado}
\begin{itemize}
\item {Grp. gram.:adj.}
\end{itemize}
\begin{itemize}
\item {Utilização:Fam.}
\end{itemize}
Que se regou: \textunderscore plantações regadas\textunderscore .
Acompanhado de um líquido: \textunderscore bifes, bem regados com vinho do Pôrto\textunderscore .
\section{Regador}
\begin{itemize}
\item {Grp. gram.:adj.}
\end{itemize}
\begin{itemize}
\item {Grp. gram.:M.}
\end{itemize}
\begin{itemize}
\item {Proveniência:(De \textunderscore regar\textunderscore )}
\end{itemize}
Que rega.
Vaso geralmente de fôlha, que serve para regar, saíndo-lhe a água por um tubo lateral, terminado geralmente em crivo.
\section{Regadura}
\begin{itemize}
\item {Grp. gram.:f.}
\end{itemize}
O mesmo que \textunderscore rega\textunderscore .
\section{Regaengo}
\begin{itemize}
\item {Grp. gram.:adj.}
\end{itemize}
\begin{itemize}
\item {Utilização:Ant.}
\end{itemize}
O mesmo que \textunderscore regalengo\textunderscore .
\section{Regaixinhas}
\begin{itemize}
\item {Grp. gram.:f. pl.}
\end{itemize}
\begin{itemize}
\item {Utilização:Prov.}
\end{itemize}
\begin{itemize}
\item {Utilização:trasm.}
\end{itemize}
Rodelas de limão ou laranja para salada.
\section{Regalada}
\begin{itemize}
\item {Grp. gram.:f. Loc. adv.}
\end{itemize}
\begin{itemize}
\item {Proveniência:(De \textunderscore regalado\textunderscore )}
\end{itemize}
\textunderscore Á regalada\textunderscore , regaladamente; á tripa fôrra.
\section{Regaladamente}
\begin{itemize}
\item {Grp. gram.:adv.}
\end{itemize}
De modo regalado; com regalo; com delícia.
\section{Regalado}
\begin{itemize}
\item {Grp. gram.:adj.}
\end{itemize}
\begin{itemize}
\item {Grp. gram.:Adv.}
\end{itemize}
\begin{itemize}
\item {Proveniência:(De \textunderscore regalar\textunderscore )}
\end{itemize}
Que se regalou; que sente regalo ou grande prazer.
Com regalo, com mimo.
\section{Regalador}
\begin{itemize}
\item {Grp. gram.:m.  e  adj.}
\end{itemize}
O que regala.
\section{Regalão}
\begin{itemize}
\item {Grp. gram.:m.  e  adj.}
\end{itemize}
\begin{itemize}
\item {Grp. gram.:M.}
\end{itemize}
O que se regala, o que tem vida regalada; folgazão.
Glotão.
Acto de se regalar; grande regalo.
\section{Regalar}
\begin{itemize}
\item {Grp. gram.:v. t.}
\end{itemize}
\begin{itemize}
\item {Utilização:Irón.}
\end{itemize}
\begin{itemize}
\item {Grp. gram.:V. i.}
\end{itemize}
\begin{itemize}
\item {Utilização:Prov.}
\end{itemize}
\begin{itemize}
\item {Utilização:minh.}
\end{itemize}
Causar regalo a; tratar com mimo; mimosear.
Brindar.
Recrear.
Tratar mal.
Passar bem; regalar-se: \textunderscore adeus e regalar\textunderscore .
(Cp. cast. \textunderscore regalar\textunderscore )
\section{Regalardoar}
\begin{itemize}
\item {Grp. gram.:v. t.}
\end{itemize}
\begin{itemize}
\item {Proveniência:(De \textunderscore re...\textunderscore  + \textunderscore galardoar\textunderscore )}
\end{itemize}
Tornar a galardoar.
\section{Regaleco}
\begin{itemize}
\item {Grp. gram.:m.}
\end{itemize}
Peixe dos mares do Norte.
\section{Regalejo}
\begin{itemize}
\item {Grp. gram.:m.}
\end{itemize}
\begin{itemize}
\item {Utilização:Ant.}
\end{itemize}
O mesmo que \textunderscore realejo\textunderscore .
\section{Regalengo}
\begin{itemize}
\item {Grp. gram.:adj.}
\end{itemize}
\begin{itemize}
\item {Proveniência:(Do b. lat. \textunderscore regalengus\textunderscore )}
\end{itemize}
O mesmo que \textunderscore reguengo\textunderscore . Cf. Herculano, \textunderscore Bobo\textunderscore , 209.
\section{Regaleza}
\begin{itemize}
\item {Grp. gram.:f.}
\end{itemize}
\begin{itemize}
\item {Utilização:Ant.}
\end{itemize}
O mesmo que \textunderscore regoliz\textunderscore .
\section{Regalheira}
\begin{itemize}
\item {Grp. gram.:f.}
\end{itemize}
\begin{itemize}
\item {Utilização:T. de Turquel}
\end{itemize}
\begin{itemize}
\item {Grp. gram.:Loc. adv.}
\end{itemize}
\begin{itemize}
\item {Utilização:Ant.}
\end{itemize}
O mesmo que \textunderscore regalice\textunderscore ^1.
\textunderscore Á regalheira\textunderscore , regaladamente; á tripa fôrra. Cf. \textunderscore Anat. Joc.\textunderscore , II, 147.
(Por \textunderscore regaleira\textunderscore , de \textunderscore regalo\textunderscore )
\section{Regalia}
\begin{itemize}
\item {Grp. gram.:f.}
\end{itemize}
\begin{itemize}
\item {Utilização:Ext.}
\end{itemize}
\begin{itemize}
\item {Proveniência:(Do lat. \textunderscore regalis\textunderscore )}
\end{itemize}
Direito próprio de rei.
Prerogativa real.
Prerogativa; privilégio.
\section{Regalice}
\begin{itemize}
\item {Grp. gram.:f.}
\end{itemize}
\begin{itemize}
\item {Utilização:Deprec.}
\end{itemize}
\begin{itemize}
\item {Proveniência:(De \textunderscore regalo\textunderscore )}
\end{itemize}
Estado do que vive regalado ou ocioso.
\section{Regalice}
\begin{itemize}
\item {Grp. gram.:f.}
\end{itemize}
(V.regoliz)
\section{Regalindo}
\begin{itemize}
\item {Grp. gram.:adj.}
\end{itemize}
\begin{itemize}
\item {Utilização:Ant.}
\end{itemize}
O mesmo que \textunderscore regalengo\textunderscore .
\section{Regalismo}
\begin{itemize}
\item {Grp. gram.:m.}
\end{itemize}
Systema dos regalistas ou dos que defendem as prerogativas do Estado, contra as pretensões da Igreja.
\section{Regalista}
\begin{itemize}
\item {Grp. gram.:m.  e  f.}
\end{itemize}
Pessôa, que defende regalias.
Pessôa, que desfruta regalias.
\section{Regalito}
\begin{itemize}
\item {Grp. gram.:m.}
\end{itemize}
\begin{itemize}
\item {Utilização:T. de Turquel}
\end{itemize}
\begin{itemize}
\item {Proveniência:(De \textunderscore recreio\textunderscore )}
\end{itemize}
Quinta ou vivenda para recreio.
\section{Regalo}
\begin{itemize}
\item {Grp. gram.:m.}
\end{itemize}
\begin{itemize}
\item {Utilização:Pesc.}
\end{itemize}
\begin{itemize}
\item {Proveniência:(De \textunderscore regalar\textunderscore )}
\end{itemize}
Prazer.
Mimo, tratamento esmerado.
Vida tranquilla e satisfeita.
Dádiva, brinde.
Utensílio, geralmente de pelles. em que se resguardam do frio as mãos.
Rêde de braços, no apparelho de arrastar.
\section{Regalona}
\begin{itemize}
\item {Grp. gram.:f.  e  adj.}
\end{itemize}
\begin{itemize}
\item {Utilização:Prov.}
\end{itemize}
\begin{itemize}
\item {Utilização:alent.}
\end{itemize}
\begin{itemize}
\item {Proveniência:(De \textunderscore regalão\textunderscore )}
\end{itemize}
Mulhér, que vive regaladamente; mulhér ociosa.
Variedade de azeitona, o mesmo que \textunderscore longal\textunderscore  ou \textunderscore sevilhana\textunderscore .
Variedade de ameixa.
\section{Regalório}
\begin{itemize}
\item {Grp. gram.:m.}
\end{itemize}
\begin{itemize}
\item {Utilização:Fam.}
\end{itemize}
\begin{itemize}
\item {Proveniência:(De \textunderscore regalar\textunderscore )}
\end{itemize}
Grande regalo ou folgança; patuscada, pândega.
\section{Regambolear}
\begin{itemize}
\item {Grp. gram.:v. t.  e  i.}
\end{itemize}
Folgar.
Regalar-se.
Dançar alegremente. Cf. \textunderscore Agostinheida\textunderscore ,16; Filinto, III, 152.
(Cp. \textunderscore regalar\textunderscore  e \textunderscore bambolear\textunderscore )
\section{Regamboleio}
\begin{itemize}
\item {Grp. gram.:m.}
\end{itemize}
Acto de regambolear. Cf. Camillo, \textunderscore Corja\textunderscore , 166.
\section{Reganhar}
\begin{itemize}
\item {Grp. gram.:v. t.}
\end{itemize}
\begin{itemize}
\item {Proveniência:(De \textunderscore re...\textunderscore  + \textunderscore ganhar\textunderscore )}
\end{itemize}
Ganhar novamente; readquirir.
\section{Reganhar}
\begin{itemize}
\item {Grp. gram.:v. i.}
\end{itemize}
\begin{itemize}
\item {Utilização:Des.}
\end{itemize}
Tremer com frio, arreganhar-se.
Morrer, arreganhando os dentes, (falando-se da ovelha). Cf. G. Vicente, \textunderscore Mofina Mendes\textunderscore .
(Cp. \textunderscore arreganhar\textunderscore )
\section{Reganho}
\begin{itemize}
\item {Grp. gram.:m.}
\end{itemize}
\begin{itemize}
\item {Utilização:Ant.}
\end{itemize}
\begin{itemize}
\item {Proveniência:(De \textunderscore reganhar\textunderscore ^2)}
\end{itemize}
Vento do Norte.
\section{Regar}
\begin{itemize}
\item {Grp. gram.:v. t.}
\end{itemize}
\begin{itemize}
\item {Utilização:Fig.}
\end{itemize}
\begin{itemize}
\item {Utilização:Fam.}
\end{itemize}
\begin{itemize}
\item {Proveniência:(Do lat. \textunderscore rigare\textunderscore )}
\end{itemize}
Banhar (a terra, as plantas, etc.).
Molhar, humedecer; borrifar.
Sustentar.
Acompanhar com bebidas (o que se come).
\section{Regardar}
\begin{itemize}
\item {Grp. gram.:v. t.}
\end{itemize}
\begin{itemize}
\item {Utilização:Ant.}
\end{itemize}
\begin{itemize}
\item {Grp. gram.:V. i.}
\end{itemize}
\begin{itemize}
\item {Proveniência:(De \textunderscore re...\textunderscore  + ant. al. \textunderscore wartên\textunderscore )}
\end{itemize}
Respeitar.
Têr deferências para com.
Olhar para trás.
\section{Regardo}
\begin{itemize}
\item {Grp. gram.:m.}
\end{itemize}
Acto ou effeito de regardar.
\section{Regata}
\begin{itemize}
\item {Grp. gram.:f.}
\end{itemize}
\begin{itemize}
\item {Proveniência:(It. \textunderscore regata\textunderscore )}
\end{itemize}
Corrida de embarcações á porfia.
\section{Regatagem}
\begin{itemize}
\item {Grp. gram.:f.}
\end{itemize}
Acto de regatar.
Compra e venda por miúdo:«\textunderscore ...e que nenhũa hortaliça pagasse por regatagem...\textunderscore »\textunderscore Carta\textunderscore  do viso-rei da Índia, D. Antão de Noronha, registada na Torre do Tombo.
\section{Regatão}
\begin{itemize}
\item {Grp. gram.:m.  e  adj.}
\end{itemize}
\begin{itemize}
\item {Grp. gram.:M.}
\end{itemize}
\begin{itemize}
\item {Utilização:T. da Bairrada}
\end{itemize}
\begin{itemize}
\item {Utilização:Prov.}
\end{itemize}
\begin{itemize}
\item {Utilização:alent.}
\end{itemize}
Aquelle que regata.
Negociante de porcos ou leitões.
Aquelle que vai comprar caça nos campos, para a vender na cidade.
\section{Regatar}
\begin{itemize}
\item {Grp. gram.:v. t.}
\end{itemize}
\begin{itemize}
\item {Proveniência:(Do lat. \textunderscore re...\textunderscore  + \textunderscore captare\textunderscore )}
\end{itemize}
Comprar e vender por miúdo.
\section{Regateador}
\begin{itemize}
\item {Grp. gram.:m.  e  adj.}
\end{itemize}
O que regateia.
\section{Regatear}
\begin{itemize}
\item {Grp. gram.:v. t.}
\end{itemize}
\begin{itemize}
\item {Utilização:Fig.}
\end{itemize}
\begin{itemize}
\item {Grp. gram.:V. i.}
\end{itemize}
\begin{itemize}
\item {Proveniência:(De \textunderscore regatar\textunderscore )}
\end{itemize}
Discutir prolixamente sôbre o preço de.
Discutir com modos rudes.
Dar ou fazer contra vontade ou indelicadamente: \textunderscore regatear favores\textunderscore .
Depreciar, deprimir.
Discutir teimosamente.
Altercar com rudeza ou com modos grosseiros e insistentes.
\section{Regateio}
\begin{itemize}
\item {Grp. gram.:m.}
\end{itemize}
Acto de regatear.
\section{Regateira}
\begin{itemize}
\item {Grp. gram.:f.}
\end{itemize}
\begin{itemize}
\item {Proveniência:(De \textunderscore regateiro\textunderscore )}
\end{itemize}
Mulhér, que regateia.
Vendedora ambulante.
Mulhér, que vende nos mercados hortaliça, peixe, etc.
Mulhér, que se serve de expressões grosseiras.
Mulhér, que discute ou ralha, servindo-se de linguagem grosseira.
\section{Regateiral}
\begin{itemize}
\item {Grp. gram.:adj.}
\end{itemize}
\begin{itemize}
\item {Utilização:Fam.}
\end{itemize}
\begin{itemize}
\item {Proveniência:(De \textunderscore regateira\textunderscore )}
\end{itemize}
Próprio de regateira. (Us. por J. A. Macedo)
\section{Regateiro}
\begin{itemize}
\item {Grp. gram.:m.}
\end{itemize}
\begin{itemize}
\item {Grp. gram.:Adj.}
\end{itemize}
\begin{itemize}
\item {Utilização:bras}
\end{itemize}
\begin{itemize}
\item {Proveniência:(De \textunderscore regatar\textunderscore )}
\end{itemize}
Homem que regateia.
Presumido, vaidoso.
\section{Regateirona}
\begin{itemize}
\item {Grp. gram.:f.}
\end{itemize}
Mulhér, que regateia muito.
\section{Regatia}
\begin{itemize}
\item {Grp. gram.:f.}
\end{itemize}
\begin{itemize}
\item {Proveniência:(De \textunderscore regatar\textunderscore )}
\end{itemize}
Vida ou hábitos de regateira.
\section{Regatinhar}
\begin{itemize}
\item {Grp. gram.:v. t.  e  i.}
\end{itemize}
\begin{itemize}
\item {Utilização:Prov.}
\end{itemize}
Regatear muito.
\section{Regato}
\begin{itemize}
\item {Grp. gram.:m.}
\end{itemize}
\begin{itemize}
\item {Utilização:T. de Alcanena}
\end{itemize}
\begin{itemize}
\item {Proveniência:(Do lat. \textunderscore rigatus\textunderscore )}
\end{itemize}
Corrente de água pouco considerável.
Pequeno ribeiro.
O mesmo que \textunderscore azenha\textunderscore .
\section{Regatôa}
\begin{itemize}
\item {Grp. gram.:f.}
\end{itemize}
\begin{itemize}
\item {Proveniência:(De \textunderscore regatão\textunderscore )}
\end{itemize}
Mulhér, que regata; regateira. Cf. \textunderscore Posturas\textunderscore  da Cam. Mun. de Tomar, art. 272.^o.
\section{Regaxa}
\begin{itemize}
\item {Grp. gram.:f.}
\end{itemize}
\begin{itemize}
\item {Utilização:Prov.}
\end{itemize}
\begin{itemize}
\item {Utilização:alent.}
\end{itemize}
O mesmo que \textunderscore narceja\textunderscore .
\section{Regedor}
\begin{itemize}
\item {Grp. gram.:adj.}
\end{itemize}
\begin{itemize}
\item {Grp. gram.:M.}
\end{itemize}
Que rege.
Indivíduo, que governa administrativamente uma paróchia.
Chefe do antigo tribunal da Relação de Lisbôa.
\section{Regedoral}
\begin{itemize}
\item {Grp. gram.:adj.}
\end{itemize}
\begin{itemize}
\item {Utilização:Fam.}
\end{itemize}
Relativo a regedor.
\section{Regedoria}
\begin{itemize}
\item {Grp. gram.:f.}
\end{itemize}
\begin{itemize}
\item {Utilização:Deprec.}
\end{itemize}
Cargo de regedor; Repartição do regedor.
Govêrno ou política mesquinha, que se preoccupa de insignificâncias ou de exclusivo interesse local ou pessoal.
\section{Regeira}
\begin{itemize}
\item {Grp. gram.:f.}
\end{itemize}
\begin{itemize}
\item {Utilização:Náut.}
\end{itemize}
\begin{itemize}
\item {Utilização:Bras. do S}
\end{itemize}
\begin{itemize}
\item {Proveniência:(De \textunderscore reger\textunderscore )}
\end{itemize}
Virador, que se prende ao anete da âncora.
Escora, que sustenta um dos madeiros do fundo do navio, quando êste é lançado á água.
Corda, com que o lavrador dirige os bois na lavoira.
\section{Regelador}
\begin{itemize}
\item {Grp. gram.:adj.}
\end{itemize}
\begin{itemize}
\item {Proveniência:(Lat. \textunderscore regelans\textunderscore )}
\end{itemize}
Que regela.
\section{Regelante}
\begin{itemize}
\item {Grp. gram.:adj.}
\end{itemize}
\begin{itemize}
\item {Proveniência:(Lat. \textunderscore regelans\textunderscore )}
\end{itemize}
Que regela.
\section{Regelar}
\begin{itemize}
\item {Grp. gram.:v. t.}
\end{itemize}
\begin{itemize}
\item {Grp. gram.:V. i.}
\end{itemize}
\begin{itemize}
\item {Proveniência:(Lat. \textunderscore regelare\textunderscore )}
\end{itemize}
Congelar, gelar.
Gelar-se.
\section{Regélido}
\begin{itemize}
\item {Grp. gram.:adj.}
\end{itemize}
\begin{itemize}
\item {Proveniência:(De \textunderscore re...\textunderscore  + \textunderscore gélido\textunderscore )}
\end{itemize}
Muito gélido; frigidíssimo. Cf. Camillo, \textunderscore Estrêl. Prop.\textunderscore , 14.
\section{Regêlo}
\begin{itemize}
\item {Grp. gram.:m.}
\end{itemize}
\begin{itemize}
\item {Utilização:Fig.}
\end{itemize}
Acto ou effeito de regelar.
Frieza de ânimo.
\section{Regência}
\begin{itemize}
\item {Grp. gram.:f.}
\end{itemize}
\begin{itemize}
\item {Utilização:Gram.}
\end{itemize}
\begin{itemize}
\item {Proveniência:(De \textunderscore regente\textunderscore )}
\end{itemize}
Acto ou effeito de reger.
Qualidade de quem é regente.
Collectividade, incumbida do govêrno provisório de um Estado.
Relação entre as palavras de uma oração ou entre as orações de um período.
\section{Regeneração}
\begin{itemize}
\item {Grp. gram.:f.}
\end{itemize}
\begin{itemize}
\item {Proveniência:(Lat. \textunderscore regeneratio\textunderscore )}
\end{itemize}
Acto ou effeito de regenerar.
Partido politico português, formado em 1851.
\section{Regenerador}
\begin{itemize}
\item {Grp. gram.:m.  e  adj.}
\end{itemize}
\begin{itemize}
\item {Proveniência:(Lat. \textunderscore regenerator\textunderscore )}
\end{itemize}
O que regenera.
Sectário do partido da Regeneração.
\section{Regenerando}
\begin{itemize}
\item {Grp. gram.:adj.}
\end{itemize}
\begin{itemize}
\item {Proveniência:(Lat. \textunderscore regenerandus\textunderscore )}
\end{itemize}
Que está para sêr regenerado.
\section{Regenerante}
\begin{itemize}
\item {Grp. gram.:adj.}
\end{itemize}
\begin{itemize}
\item {Proveniência:(Lat. \textunderscore regenerans\textunderscore )}
\end{itemize}
Que regenera.
\section{Regenerar}
\begin{itemize}
\item {Grp. gram.:v. t.}
\end{itemize}
\begin{itemize}
\item {Utilização:Fig.}
\end{itemize}
\begin{itemize}
\item {Proveniência:(Lat. \textunderscore regenerare\textunderscore )}
\end{itemize}
Tornar a gerar.
Dar vida nova a.
Restaurar.
Reorganizar.
Melhorar.
Emendar, (em sentido moral).
\section{Regenerativo}
\begin{itemize}
\item {Grp. gram.:adj.}
\end{itemize}
Que póde regenerar.
\section{Regeneratriz}
\begin{itemize}
\item {Grp. gram.:adj. f.}
\end{itemize}
\begin{itemize}
\item {Proveniência:(De \textunderscore regenerador\textunderscore )}
\end{itemize}
Que regenera; regenerativa. Cf. Camillo, \textunderscore Cav. em Ruínas\textunderscore , 20.
\section{Regenerável}
\begin{itemize}
\item {Grp. gram.:adj.}
\end{itemize}
Que se póde regenerar.
\section{Regentar}
\begin{itemize}
\item {Grp. gram.:v. t.}
\end{itemize}
\begin{itemize}
\item {Utilização:Des.}
\end{itemize}
\begin{itemize}
\item {Proveniência:(De \textunderscore regente\textunderscore )}
\end{itemize}
O mesmo que \textunderscore reger\textunderscore .
\section{Regente}
\begin{itemize}
\item {Grp. gram.:adj.}
\end{itemize}
\begin{itemize}
\item {Grp. gram.:M.}
\end{itemize}
\begin{itemize}
\item {Grp. gram.:M.  e  f.}
\end{itemize}
\begin{itemize}
\item {Proveniência:(Lat. \textunderscore regens\textunderscore )}
\end{itemize}
Que rege.
Aquelle que rege uma nação provisoriamente.
Pessôa, que rege um estado provisoriamente.
Nome de quem rege certas escolas.
Director ou directora.
\section{Reger}
\begin{itemize}
\item {Grp. gram.:v. t.}
\end{itemize}
\begin{itemize}
\item {Proveniência:(Lat. \textunderscore regere\textunderscore )}
\end{itemize}
Dirigir, governar: \textunderscore reger um barco\textunderscore .
Encaminhar.
Professar; leccionar: \textunderscore reger Mathemática\textunderscore .
Subordinar.
\section{Regerar}
\begin{itemize}
\item {Grp. gram.:v. t.}
\end{itemize}
O mesmo que \textunderscore regenerar\textunderscore .
\section{Regesto}
\begin{itemize}
\item {Grp. gram.:m.}
\end{itemize}
\begin{itemize}
\item {Proveniência:(Do lat. \textunderscore res\textunderscore  + \textunderscore gestus\textunderscore )}
\end{itemize}
Collecção manuscrita de documentos relativos a negociações:«\textunderscore dos regestos pontíficios constava que Affonso I fizera o seu reino censual á sé apostólica...\textunderscore »Herculano, \textunderscore Hist. de Port.\textunderscore , I, 499. Cf. \textunderscore idem\textunderscore , II, 458, 1.^a ed.
\section{Régia}
\begin{itemize}
\item {Grp. gram.:f.}
\end{itemize}
\begin{itemize}
\item {Utilização:Poét.}
\end{itemize}
\begin{itemize}
\item {Proveniência:(Lat. \textunderscore regia\textunderscore )}
\end{itemize}
Palácio real.
\section{Règiamente}
\begin{itemize}
\item {Grp. gram.:adv.}
\end{itemize}
\begin{itemize}
\item {Utilização:Fig.}
\end{itemize}
De modo régio.
Á maneira de reis.
Deslumbrantemente, magnificentemente.
\section{Região}
\begin{itemize}
\item {Grp. gram.:f.}
\end{itemize}
\begin{itemize}
\item {Utilização:Fig.}
\end{itemize}
\begin{itemize}
\item {Proveniência:(Lat. \textunderscore regio\textunderscore )}
\end{itemize}
Grande extensão de território.
Território ou porção de território que, por seu clima, producções ou por outros caracteres, se distingue dos territórios contíguos.
Cada uma das ramificações da administração pública, das sciências, das artes, etc.
Cada uma das divisões que se imaginam na atmosphera.
Cada uma das secções, em que, convencionalmente, se divide o corpo humano: \textunderscore a região frontal\textunderscore .
Cada uma das espheras da actividade humana.
\section{Regibó}
\begin{itemize}
\item {Grp. gram.:m.}
\end{itemize}
\begin{itemize}
\item {Utilização:Prov.}
\end{itemize}
\begin{itemize}
\item {Utilização:trasm.}
\end{itemize}
Carne de gado lanígero.
\section{Regicida}
\begin{itemize}
\item {Grp. gram.:m.  e  f.}
\end{itemize}
Pessôa que mata um Rei ou Raínha; assassino de um Soberano.
(Cp. \textunderscore regicídio\textunderscore )
\section{Regicídio}
\begin{itemize}
\item {Grp. gram.:m.}
\end{itemize}
\begin{itemize}
\item {Proveniência:(Do lat. \textunderscore rex\textunderscore  + \textunderscore caedere\textunderscore )}
\end{itemize}
Assassínio de Rei ou Raínha.
Morte violenta de um Soberano.
\section{Regidor}
\begin{itemize}
\item {Grp. gram.:m.}
\end{itemize}
(Fórma pop. de \textunderscore regedor\textunderscore )
(Cp. cast. \textunderscore regidor\textunderscore )
\section{Regifúgio}
\begin{itemize}
\item {Grp. gram.:m.}
\end{itemize}
\begin{itemize}
\item {Proveniência:(Lat. \textunderscore regifugium\textunderscore )}
\end{itemize}
Festa, que os Romanos celebravam a 24 de Fevereiro, em memória da expulsão dos Reis. Cf. Oliv. Martins, \textunderscore Hist. da Rep. Rom.\textunderscore 
\section{Regila}
\begin{itemize}
\item {Grp. gram.:f.}
\end{itemize}
\begin{itemize}
\item {Proveniência:(Lat. \textunderscore regilla\textunderscore )}
\end{itemize}
Túnica branca, bordada de púrpura, usada antigamente pelas noivas.
\section{Regilla}
\begin{itemize}
\item {Grp. gram.:f.}
\end{itemize}
\begin{itemize}
\item {Proveniência:(Lat. \textunderscore regilla\textunderscore )}
\end{itemize}
Túnica branca, bordada de púrpura, usada antigamente pelas noivas.
\section{Regime}
\begin{itemize}
\item {Grp. gram.:m.}
\end{itemize}
\begin{itemize}
\item {Utilização:Gram.}
\end{itemize}
\begin{itemize}
\item {Proveniência:(Lat. \textunderscore regimen\textunderscore )}
\end{itemize}
Acto ou modo de reger.
Systema político, por que se rege uma nação: \textunderscore o regime republicano\textunderscore .
Procedimento, modo de vida.
Dieta: \textunderscore disse-lhe o médico que mudasse de regime\textunderscore .
O mesmo que \textunderscore complemento\textunderscore .
\section{Regímen}
\begin{itemize}
\item {Grp. gram.:m.}
\end{itemize}
O mesmo que \textunderscore regime\textunderscore .
\section{Regimental}
\begin{itemize}
\item {Grp. gram.:adj.}
\end{itemize}
Relativo a regimento: \textunderscore música regimental\textunderscore .
\section{Regimentar}
\begin{itemize}
\item {Grp. gram.:adj.}
\end{itemize}
\begin{itemize}
\item {Grp. gram.:V. t.}
\end{itemize}
O mesmo que \textunderscore regimental\textunderscore .
Relativo a regulamento; regulamentar. Cf. Camillo, \textunderscore Pombal\textunderscore , III.
O mesmo que \textunderscore regulamentar\textunderscore ^1.
Dar regimento ou regulamento a.
\section{Regimento}
\begin{itemize}
\item {Grp. gram.:m.}
\end{itemize}
\begin{itemize}
\item {Utilização:Fig.}
\end{itemize}
\begin{itemize}
\item {Proveniência:(Lat. \textunderscore regimentum\textunderscore )}
\end{itemize}
Acto ou effeito de reger.
Regime.
Regulamento, estatuto: \textunderscore o regimento da Câmara dos Deputados\textunderscore .
Disciplina.
Corpo de tropas, dirigido normalmente por um coronel.
Grande quantidade de pessôas.
\section{Regina}
\begin{itemize}
\item {Grp. gram.:f.}
\end{itemize}
\begin{itemize}
\item {Proveniência:(Lat. \textunderscore regina\textunderscore )}
\end{itemize}
Espécie de serpente.
\section{Régio}
\begin{itemize}
\item {Grp. gram.:adj.}
\end{itemize}
\begin{itemize}
\item {Proveniência:(Lat. \textunderscore regius\textunderscore )}
\end{itemize}
Relativo ao Rei; próprio de Rei; real.
\section{Regional}
\begin{itemize}
\item {Grp. gram.:adj.}
\end{itemize}
\begin{itemize}
\item {Proveniência:(Lat. \textunderscore regionalis\textunderscore )}
\end{itemize}
Relalivo a uma região: \textunderscore costumes regionaes\textunderscore .
\section{Regionalismo}
\begin{itemize}
\item {Grp. gram.:m.}
\end{itemize}
\begin{itemize}
\item {Proveniência:(De \textunderscore regional\textunderscore )}
\end{itemize}
Partido ou systema dos que pugnam vigorosamente pelos interesses de uma região.
\section{Regionalista}
\begin{itemize}
\item {Grp. gram.:m.}
\end{itemize}
Defensor de interesses regionaes.
\section{Regirar}
\begin{itemize}
\item {Grp. gram.:v. t.}
\end{itemize}
\begin{itemize}
\item {Grp. gram.:V. i.}
\end{itemize}
\begin{itemize}
\item {Proveniência:(De \textunderscore re...\textunderscore  + \textunderscore girar\textunderscore )}
\end{itemize}
Fazer andar á roda.
Fazer girar novamente.
Mover-se á roda; redemoinhar.
\section{Regiro}
\begin{itemize}
\item {Grp. gram.:m.}
\end{itemize}
Acto ou effeito de regirar.
Ambages; rodeio.
\section{Registação}
\begin{itemize}
\item {Grp. gram.:f.}
\end{itemize}
Acto ou effeito de registar.
\section{Registador}
\begin{itemize}
\item {Grp. gram.:m.  e  adj.}
\end{itemize}
O que regista, que serve para registar.
\section{Registar}
\begin{itemize}
\item {Grp. gram.:v. t.}
\end{itemize}
\begin{itemize}
\item {Utilização:Ant.}
\end{itemize}
\begin{itemize}
\item {Proveniência:(De \textunderscore registo\textunderscore )}
\end{itemize}
Escrever ou lançar em livro especial.
Inscrever.
Manifestar.
Segurar no correio.
Regular; mencionar: \textunderscore registar um vocábulo\textunderscore .
O mesmo que \textunderscore revistar\textunderscore .
\section{Registável}
\begin{itemize}
\item {Grp. gram.:adj.}
\end{itemize}
Digno de registo; que se póde ou se deve registar.
\section{Registo}
\begin{itemize}
\item {Grp. gram.:m.}
\end{itemize}
\begin{itemize}
\item {Utilização:Mús.}
\end{itemize}
\begin{itemize}
\item {Utilização:Prov.}
\end{itemize}
\begin{itemize}
\item {Utilização:trasm.}
\end{itemize}
\begin{itemize}
\item {Proveniência:(Do lat. \textunderscore regestum\textunderscore )}
\end{itemize}
Acto ou effeito de registar.
Repartição, onde se registam factos ou documentos: \textunderscore registo predial\textunderscore ; \textunderscore registo civil\textunderscore .
Objecto em que se regista.
Inscripção de documentos.
Gravura religiosa ou imagem de um santo, com que se assinala uma passagem num livro de devoção.
Fortaleza ou embarcação, para serviço aduaneiro ou de fiscalização, num pôrto de mar.
Exame ou verificação, feita a bordo por pessoal aduaneiro ou da capitania do pôrto.
Escala, que mostra a fôrça alcoólica de um líquido, a tensão de um vapor, etc.
Sinal.
Parte do órgão ou de outros instrumentos, que encaminha o ar a várias partes do apparelho, produzindo sons variados.
Peça, que move os abafadores de um piano.
Chave da torneira.
Bica da fonte.
O mesmo que \textunderscore comporta\textunderscore ^1.
Peça de relógio, com a qual se adeanta ou se atrasa o andamento dos ponteiros.
Timbre da voz ou de um instrumento.
Parte do maquinismo do órgão, com que se faz calar ou resoar determinada ordem de tubos, segundo se mete ou se retira o respectivo puxador.
Peça, que prende o temão á rabiça.
\section{Registro}
\begin{itemize}
\item {Proveniência:(Do lat. hyp. \textunderscore regestulum\textunderscore )}
\end{itemize}
\textunderscore m.\textunderscore  (e der.)
(V. \textunderscore registo\textunderscore , etc.)
\section{Reglar}
\begin{itemize}
\item {Grp. gram.:v. t.}
\end{itemize}
\begin{itemize}
\item {Utilização:Ant.}
\end{itemize}
O mesmo que \textunderscore regular\textunderscore ^1.
\section{Regmatodonte}
\begin{itemize}
\item {Grp. gram.:m.}
\end{itemize}
Gênero de musgos.
\section{Regnicídio}
\begin{itemize}
\item {Grp. gram.:m.}
\end{itemize}
\begin{itemize}
\item {Utilização:Des.}
\end{itemize}
\begin{itemize}
\item {Proveniência:(Do lat. \textunderscore regnum\textunderscore  + \textunderscore caedere\textunderscore )}
\end{itemize}
Extincção de um reino ou de uma monarchia; perda da independência nacional:«\textunderscore ...arrancaram\textunderscore  (D. Sebastião) \textunderscore ao seu regnicidio africano...\textunderscore »Castilho, \textunderscore Camões\textunderscore , III, 153.
\section{Regno}
\begin{itemize}
\item {Grp. gram.:m.}
\end{itemize}
\begin{itemize}
\item {Utilização:Ant.}
\end{itemize}
\begin{itemize}
\item {Proveniência:(Lat. \textunderscore regnum\textunderscore )}
\end{itemize}
O mesmo que \textunderscore reino\textunderscore . Cf. G. Vicente.
\section{Rêgo}
\begin{itemize}
\item {Grp. gram.:m.}
\end{itemize}
\begin{itemize}
\item {Proveniência:(Do lat. \textunderscore riguus\textunderscore )}
\end{itemize}
Sulco, natural ou artificial, que conduz água ou é próprio para isso.
Sulco, feito pelo arado.
Pequena valla, em campo cultivado, para escoamento de águas.
Separação recta, feita nos cabellos da cabeça, mostrando uma linha de coiro cabelludo.
Refêgo.
\section{Regô}
\begin{itemize}
\item {Grp. gram.:m.}
\end{itemize}
\begin{itemize}
\item {Utilização:Bras}
\end{itemize}
Pano enrolado, que usam na cabeça, como ornato, as Negras africanas.
\section{Regoar}
\textunderscore v. t.\textunderscore  (e der.)
O mesmo que \textunderscore arregoar\textunderscore , etc.
\section{Regola}
\begin{itemize}
\item {Grp. gram.:f.}
\end{itemize}
\begin{itemize}
\item {Proveniência:(Do fr. \textunderscore regole\textunderscore )}
\end{itemize}
Córte, que se faz num terreno, para se marcarem os limites lateraes de uma estrada ou os limites de uma construcção qualquer.
Cp. \textunderscore rigol\textunderscore  e \textunderscore rigola\textunderscore .
\section{Regolfo}
\begin{itemize}
\item {Grp. gram.:m.}
\end{itemize}
O mesmo que \textunderscore turbina\textunderscore .
(Cast. \textunderscore regolfo\textunderscore )
\section{Regoliz}
\begin{itemize}
\item {Grp. gram.:m.}
\end{itemize}
\begin{itemize}
\item {Utilização:Ant.}
\end{itemize}
O mesmo que \textunderscore alcaçuz\textunderscore .
(Cp it. \textunderscore regolizia\textunderscore )
\section{Regorgeado}
\begin{itemize}
\item {Grp. gram.:adj.}
\end{itemize}
\begin{itemize}
\item {Proveniência:(De \textunderscore regorgear\textunderscore )}
\end{itemize}
Semelhante a regorgeio. Cf. Camillo, \textunderscore Quéda\textunderscore , 186.
\section{Regorgear}
\begin{itemize}
\item {Grp. gram.:v. i.}
\end{itemize}
\begin{itemize}
\item {Proveniência:(De \textunderscore re...\textunderscore  + \textunderscore gorgear\textunderscore )}
\end{itemize}
Gorgear muito; trinar.
\section{Regorgeio}
\begin{itemize}
\item {Grp. gram.:m.}
\end{itemize}
Acto de regorgear. Cf. Camillo, \textunderscore Estrêl. Pop.\textunderscore , 6.
\section{Regorgitação}
\begin{itemize}
\item {Grp. gram.:f.}
\end{itemize}
Acto ou effeito de regorgitar.
\section{Regorgitar}
\begin{itemize}
\item {Grp. gram.:v. t.}
\end{itemize}
\begin{itemize}
\item {Grp. gram.:V. i.}
\end{itemize}
\begin{itemize}
\item {Utilização:Fig.}
\end{itemize}
\begin{itemize}
\item {Proveniência:(De \textunderscore re...\textunderscore  + \textunderscore ingurgitar\textunderscore )}
\end{itemize}
Lançar para fóra (o que há em demasia numa cavidade).
Vomitar.
Trasbordar.
Estar muito cheio, repleto: \textunderscore o thetro regorgitava\textunderscore .
\section{Regougado}
\begin{itemize}
\item {Grp. gram.:adj.}
\end{itemize}
Que imita as raposas, voltando a cauda sôbre a anca, (falando-se dos cães).
\section{Regougar}
\begin{itemize}
\item {Grp. gram.:v. t.}
\end{itemize}
\begin{itemize}
\item {Grp. gram.:V. i.}
\end{itemize}
\begin{itemize}
\item {Utilização:Fig.}
\end{itemize}
\begin{itemize}
\item {Proveniência:(Do lat. \textunderscore re...\textunderscore  + \textunderscore cuculare\textunderscore ? Ou de \textunderscore re...\textunderscore  + \textunderscore gôgo\textunderscore ? Ou de \textunderscore regougo\textunderscore ?)}
\end{itemize}
Pronunciar ou dizer com voz áspera como a da raposa.
Gritar (a raposa).
Resmungar.
Falar com voz áspera como a da raposa.
\section{Regougo}
\begin{itemize}
\item {Grp. gram.:m.}
\end{itemize}
\begin{itemize}
\item {Proveniência:(T. onom.? Neste caso, \textunderscore regougar\textunderscore  viria de \textunderscore regougo\textunderscore )}
\end{itemize}
Voz da raposa; qualquer voz que imita a da raposa.
Acto de regougar.
\section{Regozijador}
\begin{itemize}
\item {Grp. gram.:adj.}
\end{itemize}
Que causa regozijo. Cf. Castilho, \textunderscore D. Quixote\textunderscore , II, 52.
\section{Regozijar}
\begin{itemize}
\item {Grp. gram.:v. t.}
\end{itemize}
Causar regozijo a.
Alegrar muito.
\section{Regozijo}
\begin{itemize}
\item {Grp. gram.:m.}
\end{itemize}
Grande gôso; contentamento; prazer; folia.
(Cast. \textunderscore regocijo\textunderscore . Cp. \textunderscore gôzo\textunderscore )
\section{Regra}
\begin{itemize}
\item {Grp. gram.:f.}
\end{itemize}
\begin{itemize}
\item {Grp. gram.:Loc. adv.}
\end{itemize}
\begin{itemize}
\item {Proveniência:(Do lat. \textunderscore regula\textunderscore )}
\end{itemize}
O mesmo que \textunderscore régua\textunderscore ^1.
Linha direita.
Cada uma das linhas de uma pauta ou de um papel pautado.
Linha de palavras escritas, que vai de uma á outra margem de uma página ou columna.
Preceito, norma.
Modêlo.
Aquillo que a lei ou o uso determina.
Aquillo que dirige.
Estatutos de algumas Ordens religiosas.
Moderação.
Cuidado.
Ordem.
Menstruo.
Operação arithmética, sôbre dados números.
\textunderscore Em regra\textunderscore , geralmente, quási sempre.
\section{Regraciar}
\begin{itemize}
\item {Grp. gram.:v. t.}
\end{itemize}
\begin{itemize}
\item {Utilização:Ant.}
\end{itemize}
\begin{itemize}
\item {Proveniência:(Do it. \textunderscore ringraziare\textunderscore )}
\end{itemize}
Dar graças, agradecer profundamente.
\section{Regradamente}
\begin{itemize}
\item {Grp. gram.:adv.}
\end{itemize}
De modo regrado; moderadamente; com parcimónia.
\section{Regradeira}
\begin{itemize}
\item {Grp. gram.:f.}
\end{itemize}
\begin{itemize}
\item {Proveniência:(De \textunderscore regrar\textunderscore ^1)}
\end{itemize}
Régua, com que se fazem pautas ou se traçam linhas no papel, para escrever sôbre estas.
\section{Regrado}
\begin{itemize}
\item {Grp. gram.:adj.}
\end{itemize}
\begin{itemize}
\item {Proveniência:(De \textunderscore regrar\textunderscore ^1)}
\end{itemize}
Moderado; prudente; sensato.
\section{Regrador}
\begin{itemize}
\item {Grp. gram.:m.}
\end{itemize}
O mesmo que \textunderscore regradeira\textunderscore .
\section{Regrante}
\begin{itemize}
\item {Grp. gram.:adj.}
\end{itemize}
\begin{itemize}
\item {Utilização:Ant.}
\end{itemize}
\begin{itemize}
\item {Proveniência:(Do lat. \textunderscore regulans\textunderscore )}
\end{itemize}
Que regra.
Que segue certa regra religiosa.
\section{Regrar}
\begin{itemize}
\item {Grp. gram.:v. t.}
\end{itemize}
\begin{itemize}
\item {Proveniência:(Do lat. \textunderscore regulare\textunderscore )}
\end{itemize}
Fazer regras em.
Pautar.
Accommodar á regra.
Alinhar.
Dirigir; moderar.
Uniformizar.
\section{Regrar}
\textunderscore v. t.\textunderscore  (e der.)
(Corr. trasm. de \textunderscore redrar\textunderscore , etc.)
\section{Regraxar}
\begin{itemize}
\item {Grp. gram.:v. t.}
\end{itemize}
Pintar a regraxo.
\section{Regraxo}
\begin{itemize}
\item {Grp. gram.:m.}
\end{itemize}
\begin{itemize}
\item {Proveniência:(De \textunderscore re...\textunderscore  + \textunderscore graxo\textunderscore )}
\end{itemize}
Systema de pintura, em que um objecto doirado ou prateado é coberto de tinta transparente, que deixa vêr a côr do oiro ou prata.
\section{Regressão}
\begin{itemize}
\item {Grp. gram.:f.}
\end{itemize}
\begin{itemize}
\item {Proveniência:(Lat. \textunderscore regressio\textunderscore )}
\end{itemize}
O mesmo que \textunderscore regresso\textunderscore ; volta; retrocesso.
\section{Regressar}
\begin{itemize}
\item {Grp. gram.:v. i.}
\end{itemize}
\begin{itemize}
\item {Grp. gram.:V. t.}
\end{itemize}
\begin{itemize}
\item {Proveniência:(De \textunderscore regresso\textunderscore )}
\end{itemize}
Voltar para o ponto donde se partiu; retroceder.
Fazer voltar:«\textunderscore a condessa empregou todos os recursos para as regressar ao convento.\textunderscore »Camillo, \textunderscore Caveira\textunderscore , 63.
\section{Regressivamente}
\begin{itemize}
\item {Grp. gram.:adv.}
\end{itemize}
De modo regressivo; com regressão.
\section{Regressivo}
\begin{itemize}
\item {Grp. gram.:adj.}
\end{itemize}
\begin{itemize}
\item {Utilização:Philol.}
\end{itemize}
\begin{itemize}
\item {Utilização:Fin.}
\end{itemize}
\begin{itemize}
\item {Proveniência:(De \textunderscore regresso\textunderscore )}
\end{itemize}
Que regressa; que retrograda.
Retro-activo.
Diz-se das fórmas que, por analogia, se deduzem de outras que, sendo primitivas, se suppõem derivadas como \textunderscore aceiro\textunderscore , de \textunderscore aço\textunderscore .
Diz-se do imposto, cuja percentagem vai deminuindo, segundo o rendimento em que recai.
\section{Regresso}
\begin{itemize}
\item {Grp. gram.:m.}
\end{itemize}
\begin{itemize}
\item {Utilização:T. de Turquel}
\end{itemize}
\begin{itemize}
\item {Proveniência:(Lat. \textunderscore regressus\textunderscore )}
\end{itemize}
Acto de regressar.
Recurso contra alguém.
Recreio, distracção.
\section{Regreta}
\begin{itemize}
\item {fónica:grê}
\end{itemize}
\begin{itemize}
\item {Grp. gram.:f.}
\end{itemize}
\begin{itemize}
\item {Utilização:Typ.}
\end{itemize}
\begin{itemize}
\item {Proveniência:(De \textunderscore regra\textunderscore )}
\end{itemize}
Régua pequena de typógrapho, com que se fórma o granel.
Pequena régua, com que o typógrapho mede a composição.
\section{Regrista}
\begin{itemize}
\item {Grp. gram.:m.}
\end{itemize}
\begin{itemize}
\item {Utilização:Deprec.}
\end{itemize}
\begin{itemize}
\item {Proveniência:(De \textunderscore regra\textunderscore )}
\end{itemize}
Ferrenho observador de regras ou preceitos literários. Cf. Filinto, I, 244.
\section{Régua}
\begin{itemize}
\item {Grp. gram.:f.}
\end{itemize}
\begin{itemize}
\item {Proveniência:(Do lat. \textunderscore regula\textunderscore )}
\end{itemize}
Peça direita e mais ou menos comprida, principalmente de madeira, para traçar linhas rectas.
\section{Reguante}
\begin{itemize}
\item {Grp. gram.:adj.}
\end{itemize}
\begin{itemize}
\item {Utilização:Ant.}
\end{itemize}
\begin{itemize}
\item {Proveniência:(De \textunderscore régua\textunderscore  = \textunderscore regra\textunderscore )}
\end{itemize}
O mesmo que \textunderscore regrante\textunderscore , (falando-se de certos cónegos).
\section{Reguarda}
\begin{itemize}
\item {Grp. gram.:f.}
\end{itemize}
\begin{itemize}
\item {Utilização:Ant.}
\end{itemize}
O mesmo que \textunderscore rètaguarda\textunderscore . Cf. Fernão Lopes, \textunderscore Chrón. de D. João I\textunderscore , XLV, etc.
\section{Reguardo}
\begin{itemize}
\item {Grp. gram.:m.}
\end{itemize}
\begin{itemize}
\item {Utilização:Ant.}
\end{itemize}
O mesmo que \textunderscore regardo\textunderscore .
\section{Règuatê}
\begin{itemize}
\item {Grp. gram.:f.}
\end{itemize}
\begin{itemize}
\item {Proveniência:(De \textunderscore régua\textunderscore  + \textunderscore tê\textunderscore , n. da letra T)}
\end{itemize}
Régua de carpinteiro, em forma de T.
\section{Reguçar}
\begin{itemize}
\item {Grp. gram.:v. t.}
\end{itemize}
\begin{itemize}
\item {Proveniência:(De \textunderscore re...\textunderscore  + \textunderscore aguçar\textunderscore )}
\end{itemize}
Tornar a aguçar.
\section{Regueifa}
\begin{itemize}
\item {Grp. gram.:f.}
\end{itemize}
\begin{itemize}
\item {Proveniência:(Do ár. \textunderscore raguifa\textunderscore )}
\end{itemize}
Pão ou bolo, feito da melhor farinha.
Pão em fórma de rosca.
Fogaça.
\section{Regueifeira}
\begin{itemize}
\item {Grp. gram.:f.}
\end{itemize}
Mulhér, que vende ou fabríca regueifas.
(Cp. \textunderscore regueifeiro\textunderscore )
\section{Regueifeiro}
\begin{itemize}
\item {Grp. gram.:m.}
\end{itemize}
Fabricante ou vendedor de regueifas.
\section{Regueira}
\begin{itemize}
\item {Grp. gram.:f.}
\end{itemize}
\begin{itemize}
\item {Utilização:Bras. de Minas}
\end{itemize}
O mesmo que \textunderscore regueiro\textunderscore .
\textunderscore Regueira das costas\textunderscore , depressão lombar, correspondente á espinha dorsal.
\section{Regueiro}
\begin{itemize}
\item {Grp. gram.:m.}
\end{itemize}
\begin{itemize}
\item {Proveniência:(De \textunderscore rêgo\textunderscore )}
\end{itemize}
Rêgo, por onde corre água.
Pequena corrente de água; regato.
\section{Reguenga}
\begin{itemize}
\item {Grp. gram.:f.}
\end{itemize}
\begin{itemize}
\item {Grp. gram.:Adj.}
\end{itemize}
\begin{itemize}
\item {Utilização:Prov.}
\end{itemize}
\begin{itemize}
\item {Utilização:minh.}
\end{itemize}
Variedade de maçan.
Diz-se da medida sem rasoira.
\section{Reguengo}
\begin{itemize}
\item {Grp. gram.:adj.}
\end{itemize}
\begin{itemize}
\item {Grp. gram.:M.}
\end{itemize}
\begin{itemize}
\item {Utilização:Ant.}
\end{itemize}
\begin{itemize}
\item {Proveniência:(De \textunderscore regalengo\textunderscore  &lt; \textunderscore regaengo\textunderscore  &lt; \textunderscore reguengo\textunderscore )}
\end{itemize}
Relativo a Rei; realengo.
Terra, pertencente ao património real.
Direitos que, recaíndo em certas terras, pertenciam á corôa.
\section{Reguengueiro}
\begin{itemize}
\item {Grp. gram.:adj.}
\end{itemize}
\begin{itemize}
\item {Proveniência:(Do b. lat. \textunderscore regalengarius\textunderscore )}
\end{itemize}
Relativo a reguengo; que reside em reguengo.
\section{Reguinga}
\begin{itemize}
\item {Grp. gram.:f.}
\end{itemize}
Pequeno reguingote; reguingote curto:«\textunderscore ...quis da minha reguinga... velha, mas inda inteira, tirar calças.\textunderscore »Filinto, VIII, 106.
\section{Reguingote}
\begin{itemize}
\item {Grp. gram.:m.}
\end{itemize}
\begin{itemize}
\item {Utilização:Pop.}
\end{itemize}
O mesmo que \textunderscore redingote\textunderscore .
\section{Reguingar}
\begin{itemize}
\item {Grp. gram.:v. t.}
\end{itemize}
Replicar, respingar.
(Cp. \textunderscore rezingar\textunderscore )
\section{Reguingueiro}
\begin{itemize}
\item {Grp. gram.:m.  e  adj.}
\end{itemize}
O que reguinga; rezingão.
\section{Reguinguete}
\begin{itemize}
\item {fónica:guê}
\end{itemize}
\begin{itemize}
\item {Grp. gram.:m.}
\end{itemize}
\begin{itemize}
\item {Utilização:Prov.}
\end{itemize}
\begin{itemize}
\item {Utilização:beir.}
\end{itemize}
Rapaz, que reguinga.
\section{Regulação}
\begin{itemize}
\item {Grp. gram.:f.}
\end{itemize}
Acto ou effeito de regular.
\section{Regulado}
\begin{itemize}
\item {Grp. gram.:m.}
\end{itemize}
\begin{itemize}
\item {Utilização:Gír.}
\end{itemize}
\begin{itemize}
\item {Proveniência:(De \textunderscore regular\textunderscore )}
\end{itemize}
Relógio.
\section{Regulador}
\begin{itemize}
\item {Grp. gram.:adj.}
\end{itemize}
\begin{itemize}
\item {Grp. gram.:M.}
\end{itemize}
Que regula ou serve para regular.
Peça que, na parte anterior do apo^2, serve para o elevar mais ou menos.
\section{Regulamentação}
\begin{itemize}
\item {Grp. gram.:f.}
\end{itemize}
Acto de regulamentar.
Redacção e publicação de regras ou regulamentos, concernentes a uma associação ou instituto.
\section{Regulamentar}
\begin{itemize}
\item {Grp. gram.:v. t.}
\end{itemize}
Fazer regulamento a, sujeitar a regulamento; regular.
\section{Regulamentar}
\begin{itemize}
\item {Grp. gram.:adj.}
\end{itemize}
Relativo a regulamento.
\section{Regulamentário}
\begin{itemize}
\item {Grp. gram.:adj.}
\end{itemize}
(V. \textunderscore regulamentar\textunderscore ^2)
\section{Regulamento}
\begin{itemize}
\item {Grp. gram.:m.}
\end{itemize}
Acto ou effeito de regular.
Determinação.
Regra, preceito.
Conjunto de regras.
Estatuto.
Disposição official, com que se explica e se facilita a execução de uma lei ou decreto.
\section{Regular}
\begin{itemize}
\item {Grp. gram.:v. t.}
\end{itemize}
\begin{itemize}
\item {Grp. gram.:V. i.}
\end{itemize}
\begin{itemize}
\item {Proveniência:(Lat. \textunderscore regulare\textunderscore )}
\end{itemize}
Encaminhar segundo a regra.
Sujeitar a regras.
Regulamentar.
Esclarecer ou facilitar, por meio de disposições, a execução de (lei ou decreto).
Cotejar.
Tornar conforme.
Moderar.
Reprimir.
Regularizar.
Servir de regra.
Trabalhar ou funccionar com acêrto ou regularmente: \textunderscore êste relógio não regula\textunderscore .
Equivaler aproximadamente: \textunderscore o preço do trigo tem regulado por 600 reis cada decalitro\textunderscore .
\section{Regular}
\begin{itemize}
\item {Grp. gram.:adj.}
\end{itemize}
\begin{itemize}
\item {Utilização:Gram.}
\end{itemize}
\begin{itemize}
\item {Grp. gram.:M.}
\end{itemize}
\begin{itemize}
\item {Proveniência:(Lat. \textunderscore regularis\textunderscore )}
\end{itemize}
Relativo a regra; conforme á regra.
Legal.
Natural.
Bem proporcionado: \textunderscore feições regulares\textunderscore .
Uniforme.
Que cumpre os seus deveres, pontual.
Mediano, ou que está no meio termo: \textunderscore aproveitamento regular\textunderscore .
Que vive sujeito a uma regra religiosa.
Que tem lados e ângulos iguaes entre si, (falando-se de figuras geométricas).
Conforme á regra geral da conjugação e da declinação, (falando-se de nomes ou verbos).
Aquillo que é regular, conveniente ou legítimo.
\section{Regularidade}
\begin{itemize}
\item {Grp. gram.:f.}
\end{itemize}
Qualidade do que é regular; regulamentação.
\section{Regularização}
\begin{itemize}
\item {Grp. gram.:f.}
\end{itemize}
Acto ou effeito de regularizar.
\section{Reabilitação}
\begin{itemize}
\item {Grp. gram.:f.}
\end{itemize}
Acto ou efeito de reabilitar.
Reaquisição de bôa fama ou crédito.
\section{Reabilitador}
\begin{itemize}
\item {Grp. gram.:m.  e  f.}
\end{itemize}
O que reabilita.
\section{Reabilitar}
\begin{itemize}
\item {Grp. gram.:v. t.}
\end{itemize}
\begin{itemize}
\item {Proveniência:(De \textunderscore re...\textunderscore  + \textunderscore habilitar\textunderscore )}
\end{itemize}
Fazer voltar á situação anterior.
Restituír direitos, que estavam perdidos, a.
Regenerar.
Fazer readquirir a estima ou consideração pública.
\section{Reabilitativo}
\begin{itemize}
\item {Grp. gram.:adj.}
\end{itemize}
Que serve para reabilitar.
Que envolve reabilitação. Cf. Luc. Cordeiro, \textunderscore Senh. Duq.\textunderscore , 222, 235 e 250.
\section{Reabitação}
\begin{itemize}
\item {Grp. gram.:f.}
\end{itemize}
Acto de reabitar.
\section{Reabitar}
\begin{itemize}
\item {Grp. gram.:v. t.}
\end{itemize}
\begin{itemize}
\item {Proveniência:(De \textunderscore re...\textunderscore  + \textunderscore habitar\textunderscore )}
\end{itemize}
Habitar de novo.
\section{Reabituar-se}
\begin{itemize}
\item {Grp. gram.:v. p.}
\end{itemize}
\begin{itemize}
\item {Proveniência:(De \textunderscore re...\textunderscore  + \textunderscore habituar\textunderscore )}
\end{itemize}
Tornar a habituar-se; readquirir certo costume. Cf. \textunderscore Diccion. Exeg.\textunderscore 
\section{Reaver}
\begin{itemize}
\item {Grp. gram.:v. t.}
\end{itemize}
\begin{itemize}
\item {Proveniência:(Do lat. \textunderscore rehabere\textunderscore )}
\end{itemize}
Haver novamente; recobrar.
\section{Regularizar}
\begin{itemize}
\item {Grp. gram.:v. t.}
\end{itemize}
\begin{itemize}
\item {Proveniência:(De \textunderscore regular\textunderscore ^2)}
\end{itemize}
Tornar regular; regulamentar^1.
Tornar razoável ou conveniente.
\section{Regularmente}
\begin{itemize}
\item {Grp. gram.:adv.}
\end{itemize}
De modo regular; medianamente.
Em regra; geralmente.
\section{Regulete}
\begin{itemize}
\item {fónica:lê}
\end{itemize}
\begin{itemize}
\item {Grp. gram.:m.}
\end{itemize}
\begin{itemize}
\item {Proveniência:(De \textunderscore régula\textunderscore )}
\end{itemize}
Moldura estreita e chata, que separa portas, almofadas, etc.
\section{Régulo}
\begin{itemize}
\item {Grp. gram.:m.}
\end{itemize}
\begin{itemize}
\item {Proveniência:(Lat. \textunderscore regulus\textunderscore )}
\end{itemize}
Pequeno Rei.
Soberano de um Estado bárbaro ou semibárbaro.
Estrêlla da constellação do Leão.
\section{Rehabilitação}
\begin{itemize}
\item {Grp. gram.:f.}
\end{itemize}
Acto ou effeito de rehabilitar.
Reacquisição de bôa fama ou crédito.
\section{Rehabilitador}
\begin{itemize}
\item {Grp. gram.:m.  e  f.}
\end{itemize}
O que rehabilita.
\section{Rehabilitar}
\begin{itemize}
\item {Grp. gram.:v. t.}
\end{itemize}
\begin{itemize}
\item {Proveniência:(De \textunderscore re...\textunderscore  + \textunderscore habilitar\textunderscore )}
\end{itemize}
Fazer voltar á situação anterior.
Restituír direitos, que estavam perdidos, a.
Regenerar.
Fazer readquirir a estima ou consideração pública.
\section{Rehabilitativo}
\begin{itemize}
\item {Grp. gram.:adj.}
\end{itemize}
Que serve para rehabilitar.
Que envolve rehabilitação. Cf. Luc. Cordeiro, \textunderscore Senh. Duq.\textunderscore , 222, 235 e 250.
\section{Rehabitação}
\begin{itemize}
\item {Grp. gram.:f.}
\end{itemize}
Acto de rehabitar.
\section{Rehabitar}
\begin{itemize}
\item {Grp. gram.:v. t.}
\end{itemize}
\begin{itemize}
\item {Proveniência:(De \textunderscore re...\textunderscore  + \textunderscore habitar\textunderscore )}
\end{itemize}
Habitar de novo.
\section{Rehabituar-se}
\begin{itemize}
\item {Grp. gram.:v. p.}
\end{itemize}
\begin{itemize}
\item {Proveniência:(De \textunderscore re...\textunderscore  + \textunderscore habituar\textunderscore )}
\end{itemize}
Tornar a habituar-se; readquirir certo costume. Cf. \textunderscore Diccion. Exeg.\textunderscore 
\section{Rehaver}
\begin{itemize}
\item {Grp. gram.:v. t.}
\end{itemize}
\begin{itemize}
\item {Proveniência:(Do lat. \textunderscore rehabere\textunderscore )}
\end{itemize}
Haver novamente; recobrar.
\section{Rei}
\begin{itemize}
\item {Grp. gram.:m.}
\end{itemize}
\begin{itemize}
\item {Utilização:Fig.}
\end{itemize}
\begin{itemize}
\item {Utilização:Prov.}
\end{itemize}
\begin{itemize}
\item {Utilização:dur.}
\end{itemize}
\begin{itemize}
\item {Proveniência:(Do lat. \textunderscore rex\textunderscore )}
\end{itemize}
Aquelle que rege ou governa, em especial aquelle que rege superiormente uma nação monárchica.
Soberano de certos Estados.
Monarcha.
Soberano do sexo masculino ou feminino:«\textunderscore a Senhora D. Maria II, não foi só um grande rei. Era sobretudo mãe vigilante.\textunderscore »Rebello, \textunderscore Elogio de D. Pedro V\textunderscore , 15 e 18.
Título do marido da Raínha.
Título do pai do Rei ou da Raínha.
Pessôa, que exerce poder absoluto.
O indivíduo mais notável entre os da sua classe: \textunderscore o rei do petróleo\textunderscore .
Uma das figuras nas cartas de jogar.
Peça principal do jôgo de xadrez.
O trabalhador, que occupa a extremidade de uma columna de homens que trabalham na cava ou redra.
\section{Reicongo}
\begin{itemize}
\item {Grp. gram.:m.}
\end{itemize}
\begin{itemize}
\item {Utilização:Bras. de Piauí}
\end{itemize}
O mesmo que \textunderscore japiaçu\textunderscore .
\section{Rei-coxo}
\begin{itemize}
\item {Grp. gram.:m.}
\end{itemize}
\begin{itemize}
\item {Utilização:Prov.}
\end{itemize}
\begin{itemize}
\item {Utilização:alent.}
\end{itemize}
Jôgo de rapazes.
\section{Reicua}
\begin{itemize}
\item {Grp. gram.:f.}
\end{itemize}
Instrumento de penteeiro, para aguçar os bicos dos pentes.
\section{Rei-do-mar}
\begin{itemize}
\item {Grp. gram.:m.}
\end{itemize}
\begin{itemize}
\item {Utilização:Prov.}
\end{itemize}
Ave, o mesmo que \textunderscore pica-peixe\textunderscore , ave.
\section{Reigada}
\begin{itemize}
\item {Grp. gram.:f.}
\end{itemize}
\begin{itemize}
\item {Proveniência:(De \textunderscore rêgo\textunderscore )}
\end{itemize}
Rêgo lombar.
Depressão entre as nádegas de certos animaes.
\section{Reigar}
\begin{itemize}
\item {Grp. gram.:v. t.}
\end{itemize}
\begin{itemize}
\item {Utilização:Ant.}
\end{itemize}
O mesmo que \textunderscore arraigar\textunderscore :«\textunderscore ...que ho faz reigado na terra.\textunderscore »Barros, \textunderscore Espelho de Casados\textunderscore , p. XX.
\section{Reigrás-dos-ingleses}
\begin{itemize}
\item {Grp. gram.:m.}
\end{itemize}
\begin{itemize}
\item {Utilização:Bot.}
\end{itemize}
Planta, o mesmo que \textunderscore azevém\textunderscore .
\section{Reima}
\begin{itemize}
\item {Grp. gram.:f.}
\end{itemize}
O mesmo que \textunderscore almofeira\textunderscore .
O mesmo que \textunderscore rheuma\textunderscore .
\section{Reimão}
\begin{itemize}
\item {Grp. gram.:m.}
\end{itemize}
\begin{itemize}
\item {Utilização:Des.}
\end{itemize}
Animal, que não tem habitação certa.
\section{Reimoso}
\begin{itemize}
\item {Grp. gram.:adj.}
\end{itemize}
Que tem reima.
\section{Reimpressão}
\begin{itemize}
\item {Grp. gram.:f.}
\end{itemize}
\begin{itemize}
\item {Proveniência:(De \textunderscore re...\textunderscore  + \textunderscore impressão\textunderscore )}
\end{itemize}
Acto ou effeito de reimprimir; reedição.
\section{Reimpresso}
\begin{itemize}
\item {Grp. gram.:adj.}
\end{itemize}
\begin{itemize}
\item {Proveniência:(De \textunderscore re...\textunderscore  + \textunderscore impresso\textunderscore )}
\end{itemize}
Que se reimprimiu.
\section{Reimprimir}
\begin{itemize}
\item {Grp. gram.:v. t.}
\end{itemize}
\begin{itemize}
\item {Proveniência:(De \textunderscore re...\textunderscore  + \textunderscore imprimir\textunderscore )}
\end{itemize}
Imprimir novamente; reeditar.
\section{Reimpulso}
\begin{itemize}
\item {Grp. gram.:m.}
\end{itemize}
Impulso repetido. Cf. Camillo, \textunderscore Caveira\textunderscore , 103.
\section{Reinação}
\begin{itemize}
\item {Grp. gram.:f.}
\end{itemize}
\begin{itemize}
\item {Utilização:Pop.}
\end{itemize}
\begin{itemize}
\item {Proveniência:(De \textunderscore reinar\textunderscore )}
\end{itemize}
Patuscada; pândega; troça. Cf. Camillo, \textunderscore Cancion. Al.\textunderscore , 104.
\section{Reinadio}
\begin{itemize}
\item {Grp. gram.:m.  e  adj.}
\end{itemize}
\begin{itemize}
\item {Utilização:Pop.}
\end{itemize}
\begin{itemize}
\item {Proveniência:(De \textunderscore reinar\textunderscore )}
\end{itemize}
Folgazão; pândego.
\section{Reinado}
\begin{itemize}
\item {Grp. gram.:m.}
\end{itemize}
\begin{itemize}
\item {Utilização:Fig.}
\end{itemize}
\begin{itemize}
\item {Proveniência:(De \textunderscore reinar\textunderscore )}
\end{itemize}
Tempo, que dura o govêrno de um Rei.
Tempo, que dura a preponderância de alguém.
Dominação; predomínio: \textunderscore o reinado das mulheres\textunderscore .
\section{Reinante}
\begin{itemize}
\item {Grp. gram.:adj.}
\end{itemize}
\begin{itemize}
\item {Grp. gram.:M.}
\end{itemize}
\begin{itemize}
\item {Proveniência:(Lat. \textunderscore regnans\textunderscore )}
\end{itemize}
Que reina.
Que predomina.
Que grassa: \textunderscore epidemia reinante\textunderscore .
Aquelle que reina; o rei.
\section{Reinar}
\begin{itemize}
\item {Grp. gram.:v. i.}
\end{itemize}
\begin{itemize}
\item {Utilização:Fig.}
\end{itemize}
\begin{itemize}
\item {Utilização:Pop.}
\end{itemize}
\begin{itemize}
\item {Utilização:Mad}
\end{itemize}
\begin{itemize}
\item {Proveniência:(Do lat. \textunderscore regnare\textunderscore )}
\end{itemize}
Sêr Rei.
Governar um reino.
Sêr Soberano; dominar.
Preponderar.
Estar em vigor.
Usar-se.
Grassar.
Divertir-se.
Patuscar.
Fazer troça.
Raivar, esbravejar.
\section{Reinata}
\begin{itemize}
\item {Grp. gram.:f.}
\end{itemize}
\begin{itemize}
\item {Utilização:Pop.}
\end{itemize}
\begin{itemize}
\item {Proveniência:(De \textunderscore reinar\textunderscore )}
\end{itemize}
Pândega; frescata; estroinice.
\section{Reinauguração}
\begin{itemize}
\item {fónica:re-i}
\end{itemize}
\begin{itemize}
\item {Grp. gram.:f.}
\end{itemize}
Acto de reinaugurar.
\section{Reinaugurar}
\begin{itemize}
\item {fónica:re-i}
\end{itemize}
\begin{itemize}
\item {Grp. gram.:v. t.}
\end{itemize}
\begin{itemize}
\item {Proveniência:(De \textunderscore re...\textunderscore  + \textunderscore inaugurar\textunderscore )}
\end{itemize}
Inaugurar novamente.
\section{Reincidência}
\begin{itemize}
\item {fónica:re-in}
\end{itemize}
\begin{itemize}
\item {Grp. gram.:f.}
\end{itemize}
\begin{itemize}
\item {Proveniência:(De \textunderscore reincidente\textunderscore )}
\end{itemize}
Acto ou effeito de reincidir.
Teimosia; pertinácia.
\section{Reincidente}
\begin{itemize}
\item {fónica:re-in}
\end{itemize}
\begin{itemize}
\item {Grp. gram.:adj.}
\end{itemize}
\begin{itemize}
\item {Proveniência:(De \textunderscore re...\textunderscore  + \textunderscore incidente\textunderscore )}
\end{itemize}
Que reincide.
\section{Reincidir}
\begin{itemize}
\item {fónica:re-in}
\end{itemize}
\begin{itemize}
\item {Grp. gram.:v. t.}
\end{itemize}
\begin{itemize}
\item {Proveniência:(De \textunderscore re...\textunderscore  + \textunderscore incidir\textunderscore )}
\end{itemize}
Tornar a praticar um acto.
Perpetrar mais de um crime da mesma espécie.
\section{Reincitamento}
\begin{itemize}
\item {fónica:re-in}
\end{itemize}
\begin{itemize}
\item {Grp. gram.:m.}
\end{itemize}
Acto ou effeito de reincitar.
\section{Reincitar}
\begin{itemize}
\item {fónica:re-in}
\end{itemize}
\begin{itemize}
\item {Grp. gram.:v. i.}
\end{itemize}
\begin{itemize}
\item {Proveniência:(De \textunderscore re...\textunderscore  + \textunderscore incitar\textunderscore )}
\end{itemize}
Incitar novamente.
\section{Reinel}
\begin{itemize}
\item {Grp. gram.:adj.}
\end{itemize}
O mesmo que \textunderscore reinol\textunderscore .
\section{Reineta}
\begin{itemize}
\item {fónica:nê}
\end{itemize}
\begin{itemize}
\item {Grp. gram.:f.  e  adj.}
\end{itemize}
O mesmo ou talvez melhor que \textunderscore raineta\textunderscore .
(Cp. fr. \textunderscore reinette\textunderscore )
\section{Reinfundir}
\begin{itemize}
\item {fónica:re-in}
\end{itemize}
\begin{itemize}
\item {Grp. gram.:v. i.}
\end{itemize}
\begin{itemize}
\item {Proveniência:(De \textunderscore re...\textunderscore  + \textunderscore infundir\textunderscore )}
\end{itemize}
Infundir de novo.
\section{Reinícola}
\begin{itemize}
\item {Grp. gram.:adj.}
\end{itemize}
\begin{itemize}
\item {Grp. gram.:M.}
\end{itemize}
\begin{itemize}
\item {Proveniência:(Do lat. \textunderscore regnicola\textunderscore )}
\end{itemize}
Que habita no reino.
Reinol.
Jurisconsulto, que trata especialmente de jurisprudência nacional.
\section{Reino}
\begin{itemize}
\item {Grp. gram.:m.}
\end{itemize}
\begin{itemize}
\item {Proveniência:(Do lat. \textunderscore regnum\textunderscore )}
\end{itemize}
Estado, que tem por chefe um Rei.
Monarchia.
Cada uma das três divisões, em que se agrupam todos os corpos da natureza: \textunderscore reino mineral\textunderscore .
Conjunto dos seres, que têm caracteres communs.
Dizia-se o conjunto dos negócios administrativos e políticos, que se tratam em uma das secretarias de Estado e que são dirigidos por ministro que tem a seu cargo especialmente a administração política e civíl. (A Secretaria do \textunderscore Reino\textunderscore  foi substituida pela Secretaria do \textunderscore Interior\textunderscore )
\textunderscore Reino do céu\textunderscore , a vida eterna.
\section{Reinol}
\begin{itemize}
\item {Grp. gram.:adj.}
\end{itemize}
\begin{itemize}
\item {Proveniência:(De \textunderscore reino\textunderscore )}
\end{itemize}
Próprio de reino; natural do reino.
Diz-se de uma variedade de ameixa.
Diz-se do dialecto português, falado em Ceilão e na costa occidental da Índia.
\section{Reinola}
\begin{itemize}
\item {Grp. gram.:f.}
\end{itemize}
Espécie de batata doce, que nasce pelos soitos.
\section{Reinscrever}
\begin{itemize}
\item {Grp. gram.:v. t.}
\end{itemize}
\begin{itemize}
\item {Proveniência:(De \textunderscore re...\textunderscore  + \textunderscore inscrever\textunderscore )}
\end{itemize}
Tornar a inscrever.
\section{Reinsistir}
\begin{itemize}
\item {Grp. gram.:v. i.}
\end{itemize}
Insistir repetidamente. Cf. \textunderscore Jorn.-do-Comm.\textunderscore , do Rio, de 15-VIII-905.
\section{Reinstalação}
\begin{itemize}
\item {Grp. gram.:f.}
\end{itemize}
Acto ou efeito de reinstalar.
\section{Reinstalar}
\begin{itemize}
\item {Grp. gram.:v. t.}
\end{itemize}
\begin{itemize}
\item {Proveniência:(De \textunderscore re...\textunderscore  + \textunderscore instalar\textunderscore )}
\end{itemize}
Instalar de novo. Cf. \textunderscore Inquér. Indust.\textunderscore , parte II, liv. II, 228.
\section{Reinstallação}
\begin{itemize}
\item {Grp. gram.:f.}
\end{itemize}
Acto ou effeito de reinstallar.
\section{Reinstallar}
\begin{itemize}
\item {Grp. gram.:v. t.}
\end{itemize}
\begin{itemize}
\item {Proveniência:(De \textunderscore re...\textunderscore  + \textunderscore installar\textunderscore )}
\end{itemize}
Installar de novo. Cf. \textunderscore Inquér. Indust.\textunderscore , parte II, liv. II, 228.
\section{Reinstituição}
\begin{itemize}
\item {Grp. gram.:f.}
\end{itemize}
Acto ou effeito de reinstituir.
\section{Reinstituir}
\begin{itemize}
\item {Grp. gram.:v. t.}
\end{itemize}
\begin{itemize}
\item {Proveniência:(De \textunderscore re...\textunderscore  + \textunderscore instituir\textunderscore )}
\end{itemize}
Instituir de novo.
\section{Reinsurgir-se}
\begin{itemize}
\item {Grp. gram.:v. p.}
\end{itemize}
\begin{itemize}
\item {Proveniência:(De \textunderscore re...\textunderscore  + \textunderscore insurgir\textunderscore )}
\end{itemize}
Tornar a insurgir-se.
\section{Reintegração}
\begin{itemize}
\item {Grp. gram.:f.}
\end{itemize}
\begin{itemize}
\item {Proveniência:(Do lat. \textunderscore redintegratio\textunderscore )}
\end{itemize}
Acto ou effeito de reintegrar.
\section{Reintegrar}
\begin{itemize}
\item {Grp. gram.:v. t.}
\end{itemize}
\begin{itemize}
\item {Proveniência:(Do lat. \textunderscore redintegrare\textunderscore )}
\end{itemize}
Empossar de novo; restituir algum cargo a.
\section{Reintegro}
\begin{itemize}
\item {Grp. gram.:m.}
\end{itemize}
Acto ou effeito de reintegrar.
Prémio de lotaria, correspondente á quantia que se jogou.
\section{Reinvestir}
\begin{itemize}
\item {Grp. gram.:v. t.  e  i.}
\end{itemize}
\begin{itemize}
\item {Proveniência:(De \textunderscore re...\textunderscore  + \textunderscore investir\textunderscore )}
\end{itemize}
Tornar a investir.
\section{Reinvidar}
\begin{itemize}
\item {Grp. gram.:v. t.}
\end{itemize}
\begin{itemize}
\item {Grp. gram.:V. i.}
\end{itemize}
\begin{itemize}
\item {Proveniência:(De \textunderscore re...\textunderscore  + \textunderscore invidar\textunderscore )}
\end{itemize}
Invidar novamente; invidar sôbre o invite, (ao jogo).
Replicar.
Desforrar-se; compensar um aggravo com outro maior.
\section{Reinvocação}
\begin{itemize}
\item {Grp. gram.:f.}
\end{itemize}
Acto ou effeito de reinvocar.
\section{Reinvocar}
\begin{itemize}
\item {Grp. gram.:v. t.}
\end{itemize}
\begin{itemize}
\item {Proveniência:(De \textunderscore re...\textunderscore  + \textunderscore invocar\textunderscore )}
\end{itemize}
Invocar de novo.
\section{Reio}
\begin{itemize}
\item {Grp. gram.:m. Loc. adv.}
\end{itemize}
\begin{itemize}
\item {Utilização:Ant.}
\end{itemize}
\textunderscore A reio\textunderscore , seguidamente; sem parar; a fio. Cf. \textunderscore Supplem. do Diccion. de Algib.\textunderscore 
\section{Reipersecutória}
\begin{itemize}
\item {Grp. gram.:adj. f.}
\end{itemize}
\begin{itemize}
\item {Utilização:bras}
\end{itemize}
\begin{itemize}
\item {Utilização:Jur.}
\end{itemize}
\begin{itemize}
\item {Proveniência:(Do lat. \textunderscore res\textunderscore  + \textunderscore persequi\textunderscore )}
\end{itemize}
Diz-se da acção, em que demandamos o que é nosso e anda fóra do nosso património.
\section{Rei-pescador}
\begin{itemize}
\item {Grp. gram.:m.}
\end{itemize}
\begin{itemize}
\item {Utilização:Bras}
\end{itemize}
Ave ribeirinha.
\section{Rei-queimado}
\begin{itemize}
\item {Grp. gram.:m.}
\end{itemize}
\begin{itemize}
\item {Utilização:T. da Bairrada}
\end{itemize}
Espécie de jôgo de rapazes.
\section{Reira}
\begin{itemize}
\item {Grp. gram.:f.}
\end{itemize}
\begin{itemize}
\item {Grp. gram.:Pl.}
\end{itemize}
\begin{itemize}
\item {Utilização:Pop.}
\end{itemize}
\begin{itemize}
\item {Proveniência:(Do lat. hyp. \textunderscore renaria\textunderscore , de \textunderscore renes\textunderscore )}
\end{itemize}
Dôr nos rins.
Nádegas; rins.
\section{Reis}
\textunderscore pl.\textunderscore  de \textunderscore real\textunderscore ^2.
\section{Reisada}
\begin{itemize}
\item {Grp. gram.:f.}
\end{itemize}
Espécie de representação ou folgança popular, com que se festejam os Santos-Reis.
\section{Reisbutos}
\begin{itemize}
\item {Grp. gram.:m. pl.}
\end{itemize}
\begin{itemize}
\item {Proveniência:(Do sânscr. \textunderscore rajaptura\textunderscore )}
\end{itemize}
Uma das castas nobres da Índia.
\section{Reiseiro}
\begin{itemize}
\item {Grp. gram.:m.}
\end{itemize}
\begin{itemize}
\item {Utilização:Prov.}
\end{itemize}
\begin{itemize}
\item {Utilização:minh.}
\end{itemize}
\begin{itemize}
\item {Proveniência:(De \textunderscore reis\textunderscore , pl. de \textunderscore rei\textunderscore )}
\end{itemize}
Aquelle que, com toques ou descantes, festeja o dia dos Santos-Reis, e ás vezes o Anno-Bom e o Natal.
\section{Reiteração}
\begin{itemize}
\item {fónica:re-i}
\end{itemize}
\begin{itemize}
\item {Grp. gram.:f.}
\end{itemize}
\begin{itemize}
\item {Proveniência:(Lat. \textunderscore reiterare\textunderscore )}
\end{itemize}
Acto ou effeito de reiterar.
\section{Reiteradamente}
\begin{itemize}
\item {fónica:re-i}
\end{itemize}
\begin{itemize}
\item {Grp. gram.:adv.}
\end{itemize}
De modo reiterado; repetidas vezes.
\section{Reiterar}
\begin{itemize}
\item {fónica:re-i}
\end{itemize}
\begin{itemize}
\item {Grp. gram.:v. t.}
\end{itemize}
\begin{itemize}
\item {Proveniência:(Lat. \textunderscore reiterare\textunderscore )}
\end{itemize}
Repetir; fazer novamente: \textunderscore reiterar promessas\textunderscore .
\section{Reiterativamente}
\begin{itemize}
\item {fónica:re-i}
\end{itemize}
\begin{itemize}
\item {Grp. gram.:adv.}
\end{itemize}
De modo reiterativo; reiteradamente.
\section{Reiterativo}
\begin{itemize}
\item {fónica:re-i}
\end{itemize}
\begin{itemize}
\item {Grp. gram.:adj.}
\end{itemize}
Que reitera ou serve para reiterar.
\section{Reiterável}
\begin{itemize}
\item {fónica:re-i}
\end{itemize}
\begin{itemize}
\item {Grp. gram.:adj.}
\end{itemize}
Que se póde reiterar.
\section{Reitor}
\begin{itemize}
\item {Grp. gram.:m.}
\end{itemize}
\begin{itemize}
\item {Proveniência:(Lat. \textunderscore rector\textunderscore )}
\end{itemize}
Aquelle que rege; regente.
Superior.
Chefe de certos estabelecimentos escolares.
Prior.
\section{Reitorado}
\begin{itemize}
\item {Grp. gram.:m.}
\end{itemize}
\begin{itemize}
\item {Proveniência:(De \textunderscore reitor\textunderscore )}
\end{itemize}
Tempo, que dura a reitoria.
Reitoria.
\section{Reitoral}
\begin{itemize}
\item {Grp. gram.:adj.}
\end{itemize}
Relativo a reitor.
\section{Reitoria}
\begin{itemize}
\item {Grp. gram.:f.}
\end{itemize}
Cargo ou dignidade de reitor; repartição de reitor.
\section{Reitorizar}
\begin{itemize}
\item {Grp. gram.:v. t.}
\end{itemize}
Administrar ou governar como reitor:«\textunderscore ...onde reitorizava o collégio de...\textunderscore »Camillo, \textunderscore Narcót.\textunderscore , I, 95.
\section{Reiúna}
\begin{itemize}
\item {Grp. gram.:f.  e  adj.}
\end{itemize}
Diz-se de uma espingarda curta e de fuzil, hoje desusada. Cf. Camillo, \textunderscore Brasileira\textunderscore , 382.--Na Beira-Alta, ouvi sempre \textunderscore rài-ú-na.\textunderscore (V.raiúna)(Relacionar-se-á com \textunderscore raiuno\textunderscore  ou \textunderscore raiado\textunderscore , estriado, tauxiado? O t. poderá contudo filiar-se em \textunderscore reiúno\textunderscore , designando arma usada por soldados ou por tropa do \textunderscore rei\textunderscore  ou do \textunderscore reino\textunderscore )
\section{Reiunar}
\begin{itemize}
\item {fónica:rei-u}
\end{itemize}
\begin{itemize}
\item {Grp. gram.:v. t.}
\end{itemize}
\begin{itemize}
\item {Utilização:Bras. do S}
\end{itemize}
\begin{itemize}
\item {Proveniência:(De \textunderscore reiuno\textunderscore )}
\end{itemize}
Cortar uma orelha a (um cavallo), para mostrar que é reiuno.
\section{Reiúno}
\begin{itemize}
\item {Grp. gram.:adj.}
\end{itemize}
\begin{itemize}
\item {Utilização:Bras. do S}
\end{itemize}
\begin{itemize}
\item {Proveniência:(De \textunderscore rei\textunderscore )}
\end{itemize}
Relativo ao reino, ao país ou ao Estado.
\section{Reivindicação}
\begin{itemize}
\item {Grp. gram.:f.}
\end{itemize}
Acto ou effeito de reivindicar.
\section{Reivindicador}
\begin{itemize}
\item {Grp. gram.:m.  e  adj.}
\end{itemize}
O que reivindica.
\section{Reivindicar}
\begin{itemize}
\item {Grp. gram.:v. t.}
\end{itemize}
\begin{itemize}
\item {Utilização:Fig.}
\end{itemize}
\begin{itemize}
\item {Proveniência:(Do lat. \textunderscore res\textunderscore , \textunderscore rei\textunderscore  + \textunderscore vindicare\textunderscore )}
\end{itemize}
Intentar acção judicial, para rehaver (propriedade que está na posse de outrem).
Recuperar; tentar recuperar.
\section{Reivindicativo}
\begin{itemize}
\item {Grp. gram.:adj.}
\end{itemize}
Que serve para reivindicar.
Que envolve reivindicação. Cf. Luc. Cordeiro, \textunderscore Senh. Duq.\textunderscore , 220.
\section{Reivós}
\begin{itemize}
\item {Grp. gram.:m. pl.}
\end{itemize}
O mesmo que \textunderscore raivós\textunderscore .
\section{Reixa}
\begin{itemize}
\item {Grp. gram.:f.}
\end{itemize}
Pequena tábua.
Grade de janela; gelosia.
(Cast. \textunderscore reja\textunderscore )
\section{Reixa}
\begin{itemize}
\item {Grp. gram.:f.}
\end{itemize}
\begin{itemize}
\item {Utilização:Pop.}
\end{itemize}
O mesmo que \textunderscore rixa\textunderscore .
Raiva; ódio.
\section{Reixelo}
\begin{itemize}
\item {fónica:xê}
\end{itemize}
\begin{itemize}
\item {Grp. gram.:m.}
\end{itemize}
\begin{itemize}
\item {Utilização:Prov.}
\end{itemize}
\begin{itemize}
\item {Utilização:Prov.}
\end{itemize}
\begin{itemize}
\item {Utilização:Trasm.}
\end{itemize}
\begin{itemize}
\item {Utilização:Prov.}
\end{itemize}
\begin{itemize}
\item {Utilização:minh.}
\end{itemize}
\begin{itemize}
\item {Utilização:Gír.}
\end{itemize}
Cobrito.
Leitão.
Carneiro novo.
O mesmo que \textunderscore chibarro\textunderscore .
Porco.
\section{Reizete}
\begin{itemize}
\item {fónica:zê}
\end{itemize}
\begin{itemize}
\item {Grp. gram.:m.}
\end{itemize}
\begin{itemize}
\item {Utilização:Deprec.}
\end{itemize}
Rei pouco importante.
Régulo. Cf. Latino, \textunderscore V. da Gama\textunderscore .
(Dem. de \textunderscore rei\textunderscore )
\section{Reja}
\begin{itemize}
\item {Grp. gram.:f.}
\end{itemize}
(V. \textunderscore reixa\textunderscore ^1)
\section{Rejeição}
\begin{itemize}
\item {Grp. gram.:f.}
\end{itemize}
\begin{itemize}
\item {Proveniência:(Lat. \textunderscore rejectio\textunderscore )}
\end{itemize}
Acto ou effeito de rejeitar.
\section{Rejeitar}
\begin{itemize}
\item {Grp. gram.:v. t.}
\end{itemize}
\begin{itemize}
\item {Proveniência:(Do lat. \textunderscore rejectare\textunderscore )}
\end{itemize}
Lançar fóra; atirar.
Repellir.
Recusar; negar.
Desprezar.
Desapprovar.
Expellir, vomitar.
\section{Rejeitar}
\begin{itemize}
\item {Grp. gram.:v. t.}
\end{itemize}
\begin{itemize}
\item {Utilização:Bras}
\end{itemize}
Cortar o rejeito^2 ou jarrête a (o boi).
\section{Rejeitável}
\begin{itemize}
\item {Grp. gram.:adj.}
\end{itemize}
Que se póde ou se deve rejeitar.
\section{Rejeito}
\begin{itemize}
\item {Grp. gram.:m.}
\end{itemize}
\begin{itemize}
\item {Utilização:Ant.}
\end{itemize}
\begin{itemize}
\item {Proveniência:(De \textunderscore rejeitar\textunderscore ^1)}
\end{itemize}
O mesmo que \textunderscore projéctil\textunderscore .
\section{Rejeito}
\begin{itemize}
\item {Grp. gram.:m.}
\end{itemize}
\begin{itemize}
\item {Utilização:Pop.}
\end{itemize}
(Corr. de \textunderscore jarrête\textunderscore )
\section{Rejeitoso}
\begin{itemize}
\item {Grp. gram.:adj.}
\end{itemize}
\begin{itemize}
\item {Proveniência:(De \textunderscore rejeitar\textunderscore ^1)}
\end{itemize}
Que se rejeita por sêr desagradável ou incômmodo.
Repugnante:«\textunderscore ...a vida, que, hoje, é trago de amargor rejeitoso...\textunderscore »Filinto, IX, 132.
\section{Rejeto}
\begin{itemize}
\item {fónica:jê}
\end{itemize}
\begin{itemize}
\item {Grp. gram.:m.}
\end{itemize}
\begin{itemize}
\item {Utilização:Bras. do N}
\end{itemize}
O mesmo que \textunderscore jarrête\textunderscore .
(Por \textunderscore rajete\textunderscore , metáth. de \textunderscore jarrête\textunderscore )
\section{Rejubilação}
\begin{itemize}
\item {Grp. gram.:f.}
\end{itemize}
Grande júbilo; rejúbilo.
Acto ou effeito de rejubilar. Cf. Júl. Lour. Pinto, \textunderscore Senh. Deput.\textunderscore , 5.
\section{Rejubilar}
\begin{itemize}
\item {Grp. gram.:v. t.}
\end{itemize}
\begin{itemize}
\item {Grp. gram.:V. i.  e  p.}
\end{itemize}
\begin{itemize}
\item {Proveniência:(De \textunderscore re...\textunderscore  + \textunderscore jubilar\textunderscore )}
\end{itemize}
Causar muito júbilo a.
Têr grande júbilo.
\section{Rejúbilo}
\begin{itemize}
\item {Grp. gram.:m.}
\end{itemize}
Acto ou effeito de rejubilar.
\section{Rejuncar}
\begin{itemize}
\item {Grp. gram.:v. t.}
\end{itemize}
\begin{itemize}
\item {Proveniência:(De \textunderscore re...\textunderscore  + \textunderscore juncar\textunderscore )}
\end{itemize}
Juncar novamente.
\section{Rejuntamento}
\begin{itemize}
\item {Grp. gram.:m.}
\end{itemize}
Acto ou effeito de \textunderscore rejuntar\textunderscore .
\section{Rejuntar}
\begin{itemize}
\item {Grp. gram.:v. t.}
\end{itemize}
\begin{itemize}
\item {Utilização:Bras}
\end{itemize}
\begin{itemize}
\item {Proveniência:(De \textunderscore re...\textunderscore  + \textunderscore juntar\textunderscore )}
\end{itemize}
Fechar as juntas de.
\section{Rejurar}
\begin{itemize}
\item {Grp. gram.:v. t.}
\end{itemize}
\begin{itemize}
\item {Proveniência:(De \textunderscore re...\textunderscore  + \textunderscore jurar\textunderscore )}
\end{itemize}
Jurar de novo; jurar repetidas vezes. Cf. Castilho, \textunderscore D. Quixote\textunderscore , 28 e 177.
\section{Rejuvenescência}
\begin{itemize}
\item {Grp. gram.:f.}
\end{itemize}
O mesmo que \textunderscore rejuvenescimento\textunderscore .
\section{Rejuvenescer}
\begin{itemize}
\item {Grp. gram.:v. t.}
\end{itemize}
\begin{itemize}
\item {Grp. gram.:V. i.  e  p.}
\end{itemize}
\begin{itemize}
\item {Proveniência:(Lat. \textunderscore rejuvenescere\textunderscore )}
\end{itemize}
Tornar jovem, remoçar.
Parecer jovem, não o sendo; remoçar.
\section{Rejuvenescimento}
\begin{itemize}
\item {Grp. gram.:m.}
\end{itemize}
Acto ou effeito de rejuvenescer.
\section{Rela}
\begin{itemize}
\item {Grp. gram.:f.}
\end{itemize}
\begin{itemize}
\item {Utilização:Prov.}
\end{itemize}
\begin{itemize}
\item {Utilização:beir.}
\end{itemize}
\begin{itemize}
\item {Utilização:Prov.}
\end{itemize}
\begin{itemize}
\item {Utilização:trasm.}
\end{itemize}
\begin{itemize}
\item {Utilização:Prov.}
\end{itemize}
\begin{itemize}
\item {Utilização:alent.}
\end{itemize}
\begin{itemize}
\item {Utilização:Prov.}
\end{itemize}
\begin{itemize}
\item {Utilização:Prov.}
\end{itemize}
\begin{itemize}
\item {Utilização:minh.}
\end{itemize}
Espécie de ran, que vive nas moitas, (\textunderscore rana arborea\textunderscore ).
Armadilha, para apanhar pássaros.
Homem impertinente, maçador, cegarrega.
Achaque dos cavallos, talvez o mesmo que \textunderscore aguamento\textunderscore .
Moléstia das ovelhas, espécie de tinha.
Bicho, que dá no milho.
Instrumento músico, feito de uma haste e uma roda dentada, sôbre a qual descai, girando, uma tabuínha ou outra peça com uma palheta.
Peça fixa de ferro, ou seixo côncavo, sôbre o qual gira a arvore do rodízio e que também se chama \textunderscore ran\textunderscore .
Peça côncava de metal, em que gira o pião dos portões de ferro.
Insecto ortóptero, o mesmo que \textunderscore rallo\textunderscore .
O mesmo que \textunderscore trécula\textunderscore .
(Contr. de \textunderscore raéla\textunderscore , do lat. \textunderscore ranella\textunderscore , de \textunderscore rana\textunderscore )
\section{Relação}
\begin{itemize}
\item {Grp. gram.:f.}
\end{itemize}
\begin{itemize}
\item {Grp. gram.:Pl.}
\end{itemize}
\begin{itemize}
\item {Proveniência:(Lat. \textunderscore relatio\textunderscore )}
\end{itemize}
Acto de referir; narração; descripção; notícia: \textunderscore relação de um naufrágio\textunderscore .
Lista, rol: \textunderscore uma relação dos empregados\textunderscore .
Ligação natural.
Dependência, semelhança, analogia: \textunderscore isso não tem relação com o nosso caso\textunderscore .
Comparação entre duas quantidades commensuráveis.
Espaço entre dois sons, na música.
Tribunal judicial de segunda instância.
Convivência.
Conhecimento recíproco de pessôas.
Trato.
Pessôa ou pessôas com que se convive: \textunderscore trato bem as minhas relações\textunderscore .
\section{Relacionação}
\begin{itemize}
\item {Grp. gram.:f.}
\end{itemize}
Acto ou effeito de relacionar.
\section{Relacionado}
\begin{itemize}
\item {Grp. gram.:adj.}
\end{itemize}
Que tem relação.
Que travou conhecimento com outrem.
\section{Relacionamento}
\begin{itemize}
\item {Grp. gram.:m.}
\end{itemize}
Acto ou effeito de relacionar.
\section{Relacionar}
\begin{itemize}
\item {Grp. gram.:v. t.}
\end{itemize}
\begin{itemize}
\item {Grp. gram.:V. p.}
\end{itemize}
\begin{itemize}
\item {Proveniência:(Do lat. \textunderscore relatio\textunderscore )}
\end{itemize}
Fazer relação de.
Referir.
Expor.
Arrolar.
Buscar ou estabelecer relação entre; confrontar.
Travar conhecimento com outrem.
Adquirir relações ou trato com alguém.
\section{Relacrar}
\begin{itemize}
\item {Grp. gram.:v. t.}
\end{itemize}
\begin{itemize}
\item {Proveniência:(De \textunderscore re...\textunderscore  + \textunderscore lacrar\textunderscore )}
\end{itemize}
Tornar a lacrar.
\section{Relamber}
\begin{itemize}
\item {Grp. gram.:v. t.}
\end{itemize}
\begin{itemize}
\item {Proveniência:(De \textunderscore re...\textunderscore  + \textunderscore lamber\textunderscore )}
\end{itemize}
Lamber de novo.
\section{Relambóia}
\begin{itemize}
\item {Grp. gram.:f.}
\end{itemize}
\begin{itemize}
\item {Utilização:Prov.}
\end{itemize}
\begin{itemize}
\item {Utilização:alent.}
\end{itemize}
Pêta, mentira.
\section{Relambório}
\begin{itemize}
\item {Grp. gram.:adj.}
\end{itemize}
\begin{itemize}
\item {Utilização:Chul.}
\end{itemize}
\begin{itemize}
\item {Grp. gram.:M.}
\end{itemize}
\begin{itemize}
\item {Proveniência:(De \textunderscore relamber\textunderscore )}
\end{itemize}
Insípido, reles.
Preguiçoso.
Ociosidade; pânria.
\section{Relampadejar}
\begin{itemize}
\item {Grp. gram.:v. i.}
\end{itemize}
\begin{itemize}
\item {Proveniência:(De \textunderscore relâmpado\textunderscore )}
\end{itemize}
O mesmo que \textunderscore relampaguear\textunderscore .
\section{Relâmpado}
\begin{itemize}
\item {Grp. gram.:m.}
\end{itemize}
\begin{itemize}
\item {Utilização:Ant.}
\end{itemize}
O mesmo que \textunderscore relâmpago\textunderscore . Cf. \textunderscore Lusíadas\textunderscore , V, 96; VI, 78, 84.
\section{Relâmpago}
\begin{itemize}
\item {Grp. gram.:m.}
\end{itemize}
\begin{itemize}
\item {Utilização:Fig.}
\end{itemize}
Scintillação ou luz rápida, produzida por descarga eléctrica entre duas nuvens, ou entre uma nuvem e a terra.
Luz intensa.
Aquillo que brilha rapidamente, apagando-se logo.
Aquillo que é rápido, ligeiro, transitório.
(Refl. do lat. \textunderscore lampas\textunderscore )
\section{Relampagueante}
\begin{itemize}
\item {Grp. gram.:adj.}
\end{itemize}
Que relampagueia. Cf. Júl. Lour. Pinto, \textunderscore Senh. Deput.\textunderscore , 326.
\section{Relampaguear}
\begin{itemize}
\item {Grp. gram.:v. i.}
\end{itemize}
\begin{itemize}
\item {Utilização:Fig.}
\end{itemize}
Produzirem-se relâmpagos.
Brilhar momentaneamente.
Scintillar: \textunderscore as espadas relampagueavam ao luar\textunderscore .
\section{Relampar}
\begin{itemize}
\item {Grp. gram.:v. t.}
\end{itemize}
\begin{itemize}
\item {Utilização:Prov.}
\end{itemize}
\begin{itemize}
\item {Utilização:trasm.}
\end{itemize}
\begin{itemize}
\item {Proveniência:(De \textunderscore relampo\textunderscore )}
\end{itemize}
O mesmo que \textunderscore relampejar\textunderscore .
\section{Relampeante}
\begin{itemize}
\item {Grp. gram.:adj.}
\end{itemize}
Que relampeia.
\section{Relampear}
\begin{itemize}
\item {Grp. gram.:v. i.}
\end{itemize}
O mesmo que \textunderscore relampejar\textunderscore .
\section{Relampejante}
\begin{itemize}
\item {Grp. gram.:adj.}
\end{itemize}
Que relampeja. Cf. Camillo, \textunderscore Vinho do Pôrto\textunderscore , 67.
\section{Relampejar}
\begin{itemize}
\item {Grp. gram.:v. i.}
\end{itemize}
\begin{itemize}
\item {Proveniência:(De \textunderscore relampo\textunderscore )}
\end{itemize}
O mesmo que \textunderscore relampaguear\textunderscore .
\section{Relampo}
\begin{itemize}
\item {Grp. gram.:m.}
\end{itemize}
\begin{itemize}
\item {Utilização:Pop.}
\end{itemize}
O mesmo que \textunderscore relâmpago\textunderscore . Cf. Castilho, \textunderscore Geórgicas\textunderscore , 101.
\section{Relamprar}
\begin{itemize}
\item {Grp. gram.:v. i.}
\end{itemize}
\begin{itemize}
\item {Utilização:Prov.}
\end{itemize}
\begin{itemize}
\item {Utilização:trasm.}
\end{itemize}
Brilhar.
O mesmo que \textunderscore relampar\textunderscore .
\section{Relançar}
\begin{itemize}
\item {Grp. gram.:v. t.}
\end{itemize}
\begin{itemize}
\item {Proveniência:(De \textunderscore re...\textunderscore  + \textunderscore lançar\textunderscore )}
\end{itemize}
O mesmo que \textunderscore relancear\textunderscore .
\section{Relance}
\begin{itemize}
\item {Grp. gram.:m.}
\end{itemize}
\begin{itemize}
\item {Grp. gram.:Loc. Adv.}
\end{itemize}
\begin{itemize}
\item {Proveniência:(De \textunderscore relançar\textunderscore )}
\end{itemize}
Acto ou effeito de relancear.
\textunderscore De relance\textunderscore , rapidamente, ao primeiro lance.
\section{Relance}
\begin{itemize}
\item {Grp. gram.:m.}
\end{itemize}
\begin{itemize}
\item {Proveniência:(De \textunderscore re...\textunderscore  + \textunderscore lance\textunderscore )}
\end{itemize}
Acto, em que o toireiro, depois de executar uma sorte, executa outra que os espectadores não previam.
\section{Relancear}
\begin{itemize}
\item {Grp. gram.:v. t.}
\end{itemize}
\begin{itemize}
\item {Grp. gram.:M.}
\end{itemize}
\begin{itemize}
\item {Proveniência:(De \textunderscore relance\textunderscore ^1)}
\end{itemize}
Dirigir rapidamente (a vista, os olhos).
Relance; vista de olhos.
\section{Relancina}
\begin{itemize}
\item {Grp. gram.:f.}
\end{itemize}
\begin{itemize}
\item {Utilização:Bras. do S}
\end{itemize}
O mesmo que \textunderscore relance\textunderscore ^1.
\section{Relanço}
\begin{itemize}
\item {Grp. gram.:m.}
\end{itemize}
O mesmo que \textunderscore relance\textunderscore ^1. Cf. Camillo, \textunderscore Esqueleto\textunderscore , 36.
\section{Relão}
\begin{itemize}
\item {Grp. gram.:m.}
\end{itemize}
\begin{itemize}
\item {Utilização:Ant.}
\end{itemize}
O mesmo que \textunderscore rolão\textunderscore ^1. Cf. B. Pereira, \textunderscore Prosódia\textunderscore , vb. \textunderscore simila\textunderscore .
\section{Relapsão}
\begin{itemize}
\item {Grp. gram.:f.}
\end{itemize}
\begin{itemize}
\item {Proveniência:(Do lat. \textunderscore relapsio\textunderscore )}
\end{itemize}
Acto de cair para trás.
Reincidência.
\section{Relapsia}
\begin{itemize}
\item {Grp. gram.:f.}
\end{itemize}
\begin{itemize}
\item {Proveniência:(De \textunderscore relapso\textunderscore )}
\end{itemize}
Reincidência no crime ou no êrro.
\section{Relapso}
\begin{itemize}
\item {Grp. gram.:m.  e  adj.}
\end{itemize}
\begin{itemize}
\item {Proveniência:(Lat. \textunderscore relapsus\textunderscore )}
\end{itemize}
O que reincide; contumaz; impenitente.
\section{Relar}
\begin{itemize}
\item {Proveniência:(De \textunderscore rela\textunderscore ? Ou alter. de \textunderscore ralar\textunderscore ?)}
\end{itemize}
\textunderscore v. t.\textunderscore  (e der.)
O mesmo ou melhor que \textunderscore ralar\textunderscore .
Importunar; apoquentar.
É fórma vulgar em Coimbra.
\section{Relar}
\begin{itemize}
\item {Grp. gram.:v. i.}
\end{itemize}
\begin{itemize}
\item {Proveniência:(De \textunderscore rela\textunderscore )}
\end{itemize}
O mesmo que \textunderscore coaxar\textunderscore .
\section{Relasso}
\begin{itemize}
\item {Grp. gram.:adj.}
\end{itemize}
(Cp. \textunderscore relapso\textunderscore , \textunderscore relaxo\textunderscore ^1 e \textunderscore ralasso\textunderscore )
\section{Relatar}
\begin{itemize}
\item {Grp. gram.:v. t.}
\end{itemize}
\begin{itemize}
\item {Proveniência:(De \textunderscore relato\textunderscore ^1)}
\end{itemize}
Referir, narrar; mencionar.
Fazer o relatório ou a parte preambular de (um decreto, um processo, etc.).
\section{Relativamente}
\begin{itemize}
\item {Grp. gram.:adv.}
\end{itemize}
De modo relativo, com referência.
\section{Relatividade}
\begin{itemize}
\item {Grp. gram.:f.}
\end{itemize}
Qualidade ou estado de relativo.
Condicionalidade; contingência.
\section{Relativo}
\begin{itemize}
\item {Grp. gram.:adj.}
\end{itemize}
\begin{itemize}
\item {Utilização:Gram.}
\end{itemize}
\begin{itemize}
\item {Proveniência:(Lat. \textunderscore relativus\textunderscore )}
\end{itemize}
Que se refere a uma coisa ou pessôa.
Que indica relação.
Condicional.
Accidental.
Julgado por comparação.
Dependente do que é absoluto.
Que se refere a nome ou proposição anterior.
\section{Relato}
\begin{itemize}
\item {Grp. gram.:m.}
\end{itemize}
\begin{itemize}
\item {Proveniência:(Lat. \textunderscore relatum\textunderscore )}
\end{itemize}
Acto ou effeito de relatar; relação.
\section{Relato}
\begin{itemize}
\item {Grp. gram.:m.}
\end{itemize}
\begin{itemize}
\item {Utilização:Prov.}
\end{itemize}
\begin{itemize}
\item {Utilização:trasm.}
\end{itemize}
Prisão de corda, fixa á mangedoira e que se compra já com um laço ou nó especial.
\section{Relator}
\begin{itemize}
\item {Grp. gram.:m.}
\end{itemize}
\begin{itemize}
\item {Proveniência:(Lat. \textunderscore relator\textunderscore )}
\end{itemize}
Aquelle que relata.
Aquelle que redige um relatório ou o parecer de uma commissão ou assembleia.
Aquelle que refere ou narra; narrador. Cf. Camillo, \textunderscore Volcoens\textunderscore , 131.
\section{Relatório}
\begin{itemize}
\item {Grp. gram.:m.}
\end{itemize}
\begin{itemize}
\item {Proveniência:(De \textunderscore relato\textunderscore )}
\end{itemize}
Exposição ou relação, geralmente escrita.
Exposição circunstanciada dos factos de uma administração ou de uma sociedade.
Exposição prévia dos fundamentos ou razões de um decreto, de uma lei, etc.
\section{Relaxação}
\begin{itemize}
\item {Grp. gram.:f.}
\end{itemize}
\begin{itemize}
\item {Utilização:Des.}
\end{itemize}
\begin{itemize}
\item {Proveniência:(Lat. \textunderscore relaxatio\textunderscore )}
\end{itemize}
Acto ou effeito de relaxar.
Distensão de fibras musculares.
Froixidão.
Incúria, desmazêlo.
Desregramento de costumes.
Dispensa, tolerância.
\section{Relaxadamente}
\begin{itemize}
\item {Grp. gram.:adv.}
\end{itemize}
De modo relaxado.
Com desmazêlo.
Dissolutamente.
\section{Relaxado}
\begin{itemize}
\item {Grp. gram.:adj.}
\end{itemize}
\begin{itemize}
\item {Utilização:Fig.}
\end{itemize}
Distendido, froixo, bambo: \textunderscore uma corda relaxada\textunderscore .
Que não cuida do cumprimento dos seus deveres; negligente.
Desmoralizado, dissoluto.
Deferido ou transmittido a uma autoridade, para cobrança coerciva (falando-se de impostos ou contribuições).
\section{Relaxador}
\begin{itemize}
\item {Grp. gram.:m.  e  adj.}
\end{itemize}
\begin{itemize}
\item {Proveniência:(Do lat. \textunderscore relaxator\textunderscore )}
\end{itemize}
O que relaxa.
\section{Relaxamento}
\begin{itemize}
\item {Grp. gram.:m.}
\end{itemize}
\begin{itemize}
\item {Utilização:Prov.}
\end{itemize}
\begin{itemize}
\item {Utilização:trasm.}
\end{itemize}
O mesmo que \textunderscore relaxação\textunderscore .
Distensão dos nervos do cavallo, por esfôrço violento.
\section{Relaxante}
\begin{itemize}
\item {Grp. gram.:adj.}
\end{itemize}
\begin{itemize}
\item {Proveniência:(Lat. \textunderscore relaxans\textunderscore )}
\end{itemize}
Que relaxa, que perverte, que desmoraliza. Cf. Júl. Lour. Pinto, \textunderscore Senh. Deput.\textunderscore , 44.
\section{Relaxar}
\begin{itemize}
\item {Grp. gram.:v. t.}
\end{itemize}
\begin{itemize}
\item {Utilização:Ant.}
\end{itemize}
\begin{itemize}
\item {Grp. gram.:V. p.}
\end{itemize}
\begin{itemize}
\item {Proveniência:(Lat. \textunderscore relaxare\textunderscore )}
\end{itemize}
Tornar froixo ou menos tenso.
Permittir o não cumprimento de uma lei ou de um dever a.
Moderar.
Depravar; perverter.
Fazer o relaxe de (uma contribuição).
Tornar laxo (o ventre).
Entregar á justiça secular (um réu).
Tornar-se fraco; desmoralizar-se; perverter-se.
\section{Relaxativo}
\begin{itemize}
\item {Grp. gram.:adj.}
\end{itemize}
\begin{itemize}
\item {Utilização:Des.}
\end{itemize}
\begin{itemize}
\item {Proveniência:(De \textunderscore relaxar\textunderscore )}
\end{itemize}
Laxativo, purgante. Cf. B. Pereira, \textunderscore Prosodia\textunderscore , vb. \textunderscore dosis\textunderscore .
\section{Redição}
\begin{itemize}
\item {Grp. gram.:f.}
\end{itemize}
\begin{itemize}
\item {Proveniência:(Lat. \textunderscore redditio\textunderscore )}
\end{itemize}
Acto de entregar; entrega; restituição. Cf. Latino, \textunderscore Hist. Pol. e Mil.\textunderscore , 14.
\section{Relaxe}
\begin{itemize}
\item {Grp. gram.:m.}
\end{itemize}
Acto de relaxar.
Acto de se transferir para o poder judicial a cobrança coerciva de uma contribuição, que se não pagou voluntariamente dentro dos prazos legaes.
\section{Relaxismo}
\begin{itemize}
\item {Grp. gram.:m.}
\end{itemize}
\begin{itemize}
\item {Proveniência:(De \textunderscore relaxar\textunderscore )}
\end{itemize}
Afroixamento dos preceitos moraes.
Tolerância da prevaricação ou da inobservância dos preceitos moraes.
Tendência para a dissolução dos costumes.
Hábitos de relaxista.
\section{Relaxista}
\begin{itemize}
\item {Grp. gram.:m. ,  f.  e  adj.}
\end{itemize}
\begin{itemize}
\item {Proveniência:(De \textunderscore relaxar\textunderscore )}
\end{itemize}
Pessôa que permitte ou facilita a falta de rigor no cumprimento dos deveres: \textunderscore o Chefe da Repartição é relaxista\textunderscore .
\section{Relaxo}
\begin{itemize}
\item {Grp. gram.:adj.}
\end{itemize}
\begin{itemize}
\item {Proveniência:(Lat. \textunderscore relaxus\textunderscore )}
\end{itemize}
O mesmo que \textunderscore relaxado\textunderscore .
\section{Relaxo}
\begin{itemize}
\item {Grp. gram.:m.}
\end{itemize}
\begin{itemize}
\item {Utilização:Bras. do N}
\end{itemize}
Discurso em verso.
\section{Relê}
\begin{itemize}
\item {Grp. gram.:m.}
\end{itemize}
O mesmo que \textunderscore ralé\textunderscore .
(Parece relacionar-se com \textunderscore reles\textunderscore )
\section{Relega}
\begin{itemize}
\item {Grp. gram.:f.}
\end{itemize}
\begin{itemize}
\item {Utilização:Ant.}
\end{itemize}
\begin{itemize}
\item {Proveniência:(De \textunderscore relegar\textunderscore )}
\end{itemize}
Altura das almofadas de uma porta ou janela, acima da superfície em que assentam.
\section{Relegação}
\begin{itemize}
\item {Grp. gram.:f.}
\end{itemize}
\begin{itemize}
\item {Proveniência:(Lat. \textunderscore relegatio\textunderscore )}
\end{itemize}
Acto ou effeito de relegar.
Destêrro.
\section{Relegagem}
\begin{itemize}
\item {Grp. gram.:f.}
\end{itemize}
\begin{itemize}
\item {Utilização:Ant.}
\end{itemize}
\begin{itemize}
\item {Proveniência:(De \textunderscore relêgo\textunderscore ^1)}
\end{itemize}
Quantia, que pagava quem queria vender vinho, quando os senhores das terras tinham o privilégio dessa venda.
\section{Relegar}
\begin{itemize}
\item {Grp. gram.:v. t.}
\end{itemize}
\begin{itemize}
\item {Utilização:Fig.}
\end{itemize}
\begin{itemize}
\item {Proveniência:(Lat. \textunderscore relegare\textunderscore )}
\end{itemize}
Expatriar.
Fazer sair de um lugar para outro.
Afastar; desprezar.
\section{Relêgo}
\begin{itemize}
\item {Grp. gram.:m.}
\end{itemize}
\begin{itemize}
\item {Utilização:Ant.}
\end{itemize}
\begin{itemize}
\item {Proveniência:(De \textunderscore relegar\textunderscore ?)}
\end{itemize}
Prerogativa de certos senhores de terras, aos quaes se permittia exclusivamente o vender vinho nos seus domínios.
Lagar, adega; celleiro.
\section{Relêgo}
\begin{itemize}
\item {Grp. gram.:m.}
\end{itemize}
\begin{itemize}
\item {Utilização:Prov.}
\end{itemize}
Acatamento; respeito.
Pudor, vergonha: \textunderscore a garota não tem relêgo nenhum\textunderscore .
\section{Relêgo}
\begin{itemize}
\item {Grp. gram.:m.}
\end{itemize}
Descanso?:«\textunderscore Nunca mais teria uma hora de relêgo\textunderscore ». Camillo, \textunderscore Serões\textunderscore , I, 34.
O mesmo que \textunderscore relengo\textunderscore .
\section{Relegueiro}
\begin{itemize}
\item {Grp. gram.:m.}
\end{itemize}
\begin{itemize}
\item {Proveniência:(Do b. lat. \textunderscore relegarius\textunderscore )}
\end{itemize}
Rendeiro de terras que tinham o privilégio de relêgo^1.
\section{Releiro}
\begin{itemize}
\item {Grp. gram.:m.}
\end{itemize}
\begin{itemize}
\item {Utilização:Prov.}
\end{itemize}
\begin{itemize}
\item {Utilização:alg.}
\end{itemize}
Fileira de mólhos de cereaes, o mesmo que \textunderscore relheiro\textunderscore .
\section{Releixo}
\begin{itemize}
\item {Grp. gram.:m.}
\end{itemize}
\begin{itemize}
\item {Utilização:Prov.}
\end{itemize}
\begin{itemize}
\item {Utilização:beir.}
\end{itemize}
\begin{itemize}
\item {Proveniência:(Do cast. \textunderscore relejar\textunderscore )}
\end{itemize}
Atalho á beira de um fôsso ou de um muro.
Berma.
Saliência de um muro.
Terreno inculto, á beira de um muro.
Gume de instrumento cortante.
Largo, rocio ou eido, dependência de casa de habitação.
\section{Releixo}
\begin{itemize}
\item {Grp. gram.:m.}
\end{itemize}
\begin{itemize}
\item {Utilização:Prov.}
\end{itemize}
\begin{itemize}
\item {Utilização:trasm.}
\end{itemize}
O mesmo que \textunderscore desleixo\textunderscore .
\section{Relembrança}
\begin{itemize}
\item {Grp. gram.:f.}
\end{itemize}
\begin{itemize}
\item {Utilização:Ant.}
\end{itemize}
Acto de relembrar.
\section{Relembrar}
\begin{itemize}
\item {Grp. gram.:v. t.}
\end{itemize}
\begin{itemize}
\item {Proveniência:(De \textunderscore re...\textunderscore  + \textunderscore lembrar\textunderscore )}
\end{itemize}
Tornar a lembrar; trazer de novo á memória.
\section{Relengo}
\begin{itemize}
\item {Grp. gram.:m.}
\end{itemize}
\begin{itemize}
\item {Utilização:Prov.}
\end{itemize}
\begin{itemize}
\item {Utilização:beir.}
\end{itemize}
Moderação, tento, cautela: \textunderscore homem, tenha relengo nessa língua\textunderscore !
\section{Relentar}
\begin{itemize}
\item {Grp. gram.:v. t.}
\end{itemize}
\begin{itemize}
\item {Grp. gram.:V. i.}
\end{itemize}
\begin{itemize}
\item {Proveniência:(Lat. \textunderscore relentare\textunderscore )}
\end{itemize}
Tornar lento, amollecer com humidade.
Haver relento, orvalhar.
\section{Relento}
\begin{itemize}
\item {Grp. gram.:m.}
\end{itemize}
\begin{itemize}
\item {Proveniência:(De \textunderscore re...\textunderscore  + \textunderscore lento\textunderscore )}
\end{itemize}
Humidade atmosphérica, de noite: \textunderscore dormir ao relento faz mal\textunderscore .
Froixidão orgânica, produzida pela humidade da noite.
\section{Reler}
\begin{itemize}
\item {Grp. gram.:v. t.}
\end{itemize}
\begin{itemize}
\item {Proveniência:(Do lat. \textunderscore relegere\textunderscore )}
\end{itemize}
Lêr novamente.
Lêr muitas vezes.
\section{Reles}
\begin{itemize}
\item {Grp. gram.:adj.}
\end{itemize}
\begin{itemize}
\item {Utilização:Pop.}
\end{itemize}
\begin{itemize}
\item {Proveniência:(De \textunderscore relé\textunderscore ?)}
\end{itemize}
Muito ordinário: \textunderscore um livro reles\textunderscore .
Desprezível; inútil.
\section{Rèlesmente}
\begin{itemize}
\item {Grp. gram.:adv.}
\end{itemize}
De modo reles; desprezivelmente; sem mérito. Cf. Camillo, \textunderscore Volcoens\textunderscore , 116.
\section{Relevação}
\begin{itemize}
\item {Grp. gram.:f.}
\end{itemize}
O mesmo que \textunderscore relevamento\textunderscore .
\section{Relevado}
\begin{itemize}
\item {Grp. gram.:m.}
\end{itemize}
\begin{itemize}
\item {Proveniência:(De \textunderscore relevar\textunderscore )}
\end{itemize}
Relêvo.
\section{Relevador}
\begin{itemize}
\item {Grp. gram.:m.  e  adj.}
\end{itemize}
O que releva.
\section{Relevamento}
\begin{itemize}
\item {Grp. gram.:m.}
\end{itemize}
Acto ou effeito de relevar.
\section{Relevância}
\begin{itemize}
\item {Grp. gram.:f.}
\end{itemize}
O mesmo que \textunderscore relêvo\textunderscore .
Importância, qualidade de relevante.
\section{Relevante}
\begin{itemize}
\item {Grp. gram.:adj.}
\end{itemize}
\begin{itemize}
\item {Grp. gram.:M.}
\end{itemize}
\begin{itemize}
\item {Proveniência:(Lat. \textunderscore relevans\textunderscore )}
\end{itemize}
Que releva.
Saliente, importante: \textunderscore serviços relevantes\textunderscore .
Aquillo que importa; aquillo que é preciso.
\section{Relevar}
\begin{itemize}
\item {Grp. gram.:v. t.}
\end{itemize}
\begin{itemize}
\item {Grp. gram.:V. i.}
\end{itemize}
\begin{itemize}
\item {Proveniência:(Lat. \textunderscore relevare\textunderscore )}
\end{itemize}
Tornar saliente, dar relêvo a.
Alliviar.
Absolver de.
Perdoar a culpa de; desculpar: \textunderscore relevar faltas\textunderscore .
Permittir.
Sêr preciso ou conveniente; importar: \textunderscore releva que pensemos no caso\textunderscore .
\section{Relêvo}
\begin{itemize}
\item {Grp. gram.:m.}
\end{itemize}
\begin{itemize}
\item {Utilização:Fig.}
\end{itemize}
Acto ou effeito de relevar.
Saliência, obra de esculptura, que resai da superfície natural.
Trabalho análogo, em gravura.
Evidência.
Distincção; realce: \textunderscore falar com relêvo\textunderscore .
\section{Relfa}
\begin{itemize}
\item {Grp. gram.:f.}
\end{itemize}
\begin{itemize}
\item {Utilização:Prov.}
\end{itemize}
\begin{itemize}
\item {Utilização:minh.}
\end{itemize}
Palavreado; léria.
\section{Rêlha}
\begin{itemize}
\item {Grp. gram.:f.}
\end{itemize}
\begin{itemize}
\item {Utilização:Prov.}
\end{itemize}
\begin{itemize}
\item {Utilização:minh.}
\end{itemize}
\begin{itemize}
\item {Proveniência:(Do lat. \textunderscore regula\textunderscore )}
\end{itemize}
A parte do arado ou charrua, que entra na terra.
Peça de ferro, que reforça exteriormente as rodas dos carros de bois.
Cada uma das duas travessas do madeira, que através do meão, unem interiormente, uma á outra, as cambas (da roda do carro)
Peça de madeira, que atravessa as coiçoeiras e tábuas da porta, para que não empenem.
O mesmo que \textunderscore relheira\textunderscore .
\section{Relhaço}
\begin{itemize}
\item {Grp. gram.:m.}
\end{itemize}
\begin{itemize}
\item {Utilização:Bras. do S}
\end{itemize}
Pancada com o rêlho.
\section{Relhada}
\begin{itemize}
\item {Grp. gram.:f.}
\end{itemize}
Pancada com o rêlho.
\section{Relhar}
\begin{itemize}
\item {Grp. gram.:v.}
\end{itemize}
\begin{itemize}
\item {Utilização:t. Carp.}
\end{itemize}
Atravessar com rêlha.
Pôr uma rêlha ou rêlhas em.
\section{Rêlhas}
\begin{itemize}
\item {Grp. gram.:m.  e  f.}
\end{itemize}
\begin{itemize}
\item {Utilização:Prov.}
\end{itemize}
\begin{itemize}
\item {Utilização:trasm.}
\end{itemize}
\begin{itemize}
\item {Proveniência:(De \textunderscore reles\textunderscore ?)}
\end{itemize}
Pessôa ruim.
\section{Relheira}
\begin{itemize}
\item {Grp. gram.:f.}
\end{itemize}
\begin{itemize}
\item {Proveniência:(De \textunderscore rêlha\textunderscore . Cp. gall. \textunderscore rilleira\textunderscore )}
\end{itemize}
Sulco, que as rodas do carro deixam na terra.
\section{Relheiro}
\begin{itemize}
\item {Grp. gram.:m.}
\end{itemize}
\begin{itemize}
\item {Utilização:Prov.}
\end{itemize}
\begin{itemize}
\item {Utilização:trasm.}
\end{itemize}
O mesmo que \textunderscore relheira\textunderscore .
Fileira de mólhos de trigo ou centeio, com as espigas todas para um lado, na terra que se acabou de ceifar.
\section{Rêlho}
\begin{itemize}
\item {Grp. gram.:m.}
\end{itemize}
\begin{itemize}
\item {Utilização:Ant.}
\end{itemize}
\begin{itemize}
\item {Utilização:Prov.}
\end{itemize}
\begin{itemize}
\item {Utilização:minh.}
\end{itemize}
\begin{itemize}
\item {Proveniência:(Do lat. \textunderscore rigidulus\textunderscore ?)}
\end{itemize}
Azorrague, feito de coiro torcido.
Cinturão.
Cordão metállico, que se punha á roda do chapéu.
Fivelão, com que as senhoras apertavam custosos cintos.
Pequena peça de madeira, em fórma de 8, que serve de fivela para segurar as extremidades de uma corda, com que se ata qualquer coisa.
\section{Rélho}
\begin{itemize}
\item {Grp. gram.:m.}
\end{itemize}
\begin{itemize}
\item {Utilização:Pop.}
\end{itemize}
Us. na loc. \textunderscore velho e rélho\textunderscore , com a significação talvez de \textunderscore muito velho\textunderscore .
(Contr. de \textunderscore revelho\textunderscore ?)
\section{Relhota}
\begin{itemize}
\item {Grp. gram.:m.}
\end{itemize}
Pequena rêlha.
\section{Relicário}
\begin{itemize}
\item {Grp. gram.:m.}
\end{itemize}
Caixa ou bôlsa, que contém relíquias, ou que as conteve, ou que é destinada a contê-las.
\section{Relicitação}
\begin{itemize}
\item {Grp. gram.:f.}
\end{itemize}
Acto ou effeito de relicitar.
\section{Relicitar}
\begin{itemize}
\item {Grp. gram.:v. t.}
\end{itemize}
\begin{itemize}
\item {Proveniência:(De \textunderscore re...\textunderscore  + \textunderscore licitar\textunderscore )}
\end{itemize}
Licitar novamente.
\section{Religa}
\begin{itemize}
\item {Grp. gram.:f.}
\end{itemize}
\begin{itemize}
\item {Utilização:Ant.}
\end{itemize}
O mesmo que \textunderscore relíquia\textunderscore .
\section{Religar}
\begin{itemize}
\item {Grp. gram.:v. t.}
\end{itemize}
\begin{itemize}
\item {Proveniência:(De \textunderscore re...\textunderscore  + \textunderscore ligar\textunderscore )}
\end{itemize}
Ligar de novo; atar bem.
\section{Religário}
\begin{itemize}
\item {Grp. gram.:m.}
\end{itemize}
\begin{itemize}
\item {Utilização:Ant.}
\end{itemize}
O mesmo que \textunderscore relicário\textunderscore .
\section{Religião}
\begin{itemize}
\item {Grp. gram.:f.}
\end{itemize}
\begin{itemize}
\item {Proveniência:(Lat. \textunderscore religio\textunderscore )}
\end{itemize}
Conjunto de princípios e de práticas, que constituem as relações entre o homem e a divindade.
Culto interno ou externo, prestado á divindade.
Systema religioso.
Respeito ou reverência ás coisas sagradas.
Temor de Deus.
Dever sagrado.
Ordem religiosa.
Moral independente e commum ás nações cultas.
Crença viva.
Consciência escrupulosa; escrúpulos.
\section{Religiomania}
\begin{itemize}
\item {Grp. gram.:f.}
\end{itemize}
\begin{itemize}
\item {Utilização:Neol.}
\end{itemize}
Mania religiosa.
\section{Religionário}
\begin{itemize}
\item {Grp. gram.:m.}
\end{itemize}
Sectário de uma religião. Cf. Herculano, \textunderscore Cistér\textunderscore , 72.
\section{Religiosa}
\begin{itemize}
\item {Grp. gram.:f.}
\end{itemize}
\begin{itemize}
\item {Proveniência:(De \textunderscore religioso\textunderscore )}
\end{itemize}
Mulhér, ligada por votos monásticos.
\section{Religiosamente}
\begin{itemize}
\item {Grp. gram.:adv.}
\end{itemize}
\begin{itemize}
\item {Utilização:Fig.}
\end{itemize}
De modo religioso.
Com exactidão, com pontualidade: \textunderscore cumprir religiosamente os seus deveres\textunderscore .
\section{Religiosidade}
\begin{itemize}
\item {Grp. gram.:f.}
\end{itemize}
\begin{itemize}
\item {Utilização:Neol.}
\end{itemize}
\begin{itemize}
\item {Proveniência:(Lat. \textunderscore religiositas\textunderscore )}
\end{itemize}
Qualidade do que é religioso.
Sentimento de escrúpulos religiosos.
Disposição ou tendência religiosa.
\section{Religioso}
\begin{itemize}
\item {Grp. gram.:adj.}
\end{itemize}
\begin{itemize}
\item {Grp. gram.:M.}
\end{itemize}
\begin{itemize}
\item {Proveniência:(Lat. \textunderscore religiosus\textunderscore )}
\end{itemize}
Relativo a religião.
Conforme á religião: \textunderscore sentimentos religiosos\textunderscore .
Que vive segundo as regras da religião.
Escrupuloso, pontual no cumprimento dos deveres.
Relativos a uma Ordem monástica.
Aquelle que tem religião.
Aquelle que está ligado por votos monásticos.
\section{Relimar}
\begin{itemize}
\item {Grp. gram.:v. t.}
\end{itemize}
Tornar a limar, aperfeiçoar.
(Do \textunderscore re...\textunderscore  + \textunderscore limar\textunderscore ^1)
\section{Relinchão}
\begin{itemize}
\item {Grp. gram.:adj.}
\end{itemize}
\begin{itemize}
\item {Proveniência:(De \textunderscore relinchar\textunderscore )}
\end{itemize}
O mesmo que \textunderscore rinchão\textunderscore ^1.
\section{Relinchar}
\begin{itemize}
\item {Grp. gram.:v. i.}
\end{itemize}
\begin{itemize}
\item {Proveniência:(De \textunderscore relincho\textunderscore )}
\end{itemize}
O mesmo que \textunderscore rinchar\textunderscore .
\section{Relincho}
\begin{itemize}
\item {Grp. gram.:m.}
\end{itemize}
\begin{itemize}
\item {Proveniência:(T. onom.)}
\end{itemize}
O mesmo que \textunderscore rincho\textunderscore .
\section{Relinga}
\begin{itemize}
\item {Grp. gram.:f.}
\end{itemize}
Corda, com que se atam as velas das embarcações.
(Cast. \textunderscore relíngua\textunderscore )
\section{Relingar}
\begin{itemize}
\item {Grp. gram.:v. t.  e  i.}
\end{itemize}
Pôr as relingas das velas.
Içar velas, até que as relingas lateraes fiquem tensas.
\section{Relinquição}
\begin{itemize}
\item {fónica:cu-i}
\end{itemize}
\begin{itemize}
\item {Grp. gram.:f.}
\end{itemize}
\begin{itemize}
\item {Utilização:Jur.}
\end{itemize}
\begin{itemize}
\item {Utilização:ant.}
\end{itemize}
Entrega, cedência; relinquimento:«\textunderscore saibam todos quantos virem êste público instrumento de relinquição...\textunderscore »(Cp. \textunderscore relinquimento\textunderscore )
\section{Relinquimento}
\begin{itemize}
\item {fónica:cu-i}
\end{itemize}
\begin{itemize}
\item {Grp. gram.:m.}
\end{itemize}
\begin{itemize}
\item {Utilização:Ant.}
\end{itemize}
\begin{itemize}
\item {Proveniência:(Do lat. \textunderscore relinquere\textunderscore )}
\end{itemize}
Abandono; renúncia.
\section{Relinquir}
\begin{itemize}
\item {Grp. gram.:v. t.}
\end{itemize}
\begin{itemize}
\item {Utilização:Ant.}
\end{itemize}
\begin{itemize}
\item {Proveniência:(Lat. \textunderscore relinquere\textunderscore )}
\end{itemize}
Deixar; renunciar; largar.
\section{Relíquia}
\begin{itemize}
\item {Grp. gram.:f.}
\end{itemize}
\begin{itemize}
\item {Utilização:Ext.}
\end{itemize}
\begin{itemize}
\item {Proveniência:(Lat. \textunderscore reliquiae\textunderscore )}
\end{itemize}
Parte do corpo de algum santo.
Qualquer objecto, que pertenceu a um santo ou serviu em algum acto que lhe dizia respeito.
Coisa preciosa e rara; restos respeitáveis; ruínas.
\section{Reliquiárío}
\begin{itemize}
\item {Grp. gram.:m.}
\end{itemize}
O mesmo que \textunderscore relicário\textunderscore . Cf. F. Manuel, \textunderscore Apólogos\textunderscore , I, 109.
\section{Relogiaria}
\begin{itemize}
\item {Grp. gram.:f.}
\end{itemize}
O mesmo que \textunderscore relojoaria\textunderscore .
\section{Relógio}
\begin{itemize}
\item {Grp. gram.:m.}
\end{itemize}
\begin{itemize}
\item {Utilização:Fam.}
\end{itemize}
\begin{itemize}
\item {Utilização:T. das Caldas da Raínha}
\end{itemize}
\begin{itemize}
\item {Utilização:Bras}
\end{itemize}
\begin{itemize}
\item {Utilização:Ant.}
\end{itemize}
\begin{itemize}
\item {Utilização:Pop.}
\end{itemize}
\begin{itemize}
\item {Proveniência:(Do lat. \textunderscore horologium\textunderscore )}
\end{itemize}
Instrumento, para marcar as horas.
Constellação meridional.
Achaque permanente que ficou de uma doença.
Espécie de jôgo de asar.
Planta americana, que é branca de manhan, encarnada ao meio dia e azul á noite; talvez o mesmo que \textunderscore relógio-de-vaqueiro\textunderscore . Cf. \textunderscore Jorn.-do-Comm.\textunderscore , do Rio, de 15-VI-902.
\textunderscore Relógio musical\textunderscore , o mesmo que \textunderscore acordina\textunderscore .
\textunderscore Relógio de Almada\textunderscore  o burro quando zurra.
\section{Relógio-de-vaqueiro}
\begin{itemize}
\item {Grp. gram.:m.}
\end{itemize}
Planta sertaneja do Norte do Brasil.
\section{Relógio-preto}
\begin{itemize}
\item {Grp. gram.:m.}
\end{itemize}
\begin{itemize}
\item {Utilização:Bras}
\end{itemize}
Planta malvácea, medicinal.
\section{Relojar}
\begin{itemize}
\item {Grp. gram.:v. t.}
\end{itemize}
\begin{itemize}
\item {Proveniência:(De \textunderscore relojo\textunderscore )}
\end{itemize}
Dar (horas) o relógio:«\textunderscore depois que não relojaram coisa com coisa, resolvi-me a parar.\textunderscore »F. Manuel, \textunderscore Apólogos\textunderscore .
\section{Relojeiro}
\begin{itemize}
\item {Grp. gram.:m.}
\end{itemize}
O mesmo que \textunderscore relojoeiro\textunderscore .
\section{Relojo}
\begin{itemize}
\item {Grp. gram.:m.}
\end{itemize}
O mesmo que \textunderscore relógio\textunderscore . Cf. Amador Arráiz, \textunderscore Diálogos\textunderscore .
\section{Relojoaria}
\begin{itemize}
\item {Grp. gram.:f.}
\end{itemize}
\begin{itemize}
\item {Proveniência:(De \textunderscore relojo\textunderscore )}
\end{itemize}
Arte de relojoeiro.
Maquinismo de relógios.
Casa, onde se fabricam ou se vendem relógios.
\section{Relojoeiro}
\begin{itemize}
\item {Grp. gram.:m.}
\end{itemize}
\begin{itemize}
\item {Proveniência:(De \textunderscore relojo\textunderscore )}
\end{itemize}
Fabricante ou vendedor de relógios.
\section{Reloucado}
\begin{itemize}
\item {Grp. gram.:adj.}
\end{itemize}
\begin{itemize}
\item {Proveniência:(De \textunderscore reloucar\textunderscore )}
\end{itemize}
Muito louco.
\section{Reloucar}
\begin{itemize}
\item {Grp. gram.:v. i.}
\end{itemize}
\begin{itemize}
\item {Utilização:Prov.}
\end{itemize}
\begin{itemize}
\item {Utilização:minh.}
\end{itemize}
Enlouquecer.
(Cp. \textunderscore tresloucar\textunderscore )
\section{Relouquear}
\begin{itemize}
\item {Grp. gram.:v. i.}
\end{itemize}
\begin{itemize}
\item {Utilização:Prov.}
\end{itemize}
O mesmo que \textunderscore reloucar\textunderscore . Cf. \textunderscore Viriato Trág.\textunderscore , VI, 27.
\section{Relustrar}
\begin{itemize}
\item {Grp. gram.:v. t.}
\end{itemize}
\begin{itemize}
\item {Proveniência:(De \textunderscore re...\textunderscore  + \textunderscore lustre\textunderscore )}
\end{itemize}
Dar novo lustre a.
Tornar brilhante. Cf. \textunderscore Agostinheida\textunderscore , 44.
\section{Relutação}
\begin{itemize}
\item {Grp. gram.:f.}
\end{itemize}
Acto de relutar. Cf. Camillo, \textunderscore Ôlho de Vidro\textunderscore , 181.
\section{Relutância}
\begin{itemize}
\item {Grp. gram.:f.}
\end{itemize}
Qualidade do que é relutante.
\section{Relutante}
\begin{itemize}
\item {Grp. gram.:adj.}
\end{itemize}
\begin{itemize}
\item {Proveniência:(Lat. \textunderscore reluctans\textunderscore )}
\end{itemize}
Que reluta.
\section{Relutar}
\begin{itemize}
\item {Grp. gram.:v. i.}
\end{itemize}
\begin{itemize}
\item {Proveniência:(Lat. \textunderscore reluctari\textunderscore )}
\end{itemize}
Lutar novamente.
Resistir.
Obstinar-se.
Têr repugnância ou aversão.
\section{Reluzente}
\begin{itemize}
\item {Grp. gram.:adj.}
\end{itemize}
\begin{itemize}
\item {Proveniência:(Lat. \textunderscore relucens\textunderscore )}
\end{itemize}
Que reluz.
\section{Reluzir}
\begin{itemize}
\item {Grp. gram.:v. i.}
\end{itemize}
\begin{itemize}
\item {Proveniência:(Lat. \textunderscore relucere\textunderscore )}
\end{itemize}
Luzir muito; resplandecer.
\section{Relva}
\begin{itemize}
\item {Grp. gram.:f.}
\end{itemize}
\begin{itemize}
\item {Proveniência:(Do lat. \textunderscore herba\textunderscore , seg. Körting)}
\end{itemize}
Erva rasteira e delgada.
Conjunto de ervas rasteiras e delgadas, ordinariamente gramíneas, que crescem espontâneamente nos campos e nos caminhos.
Lugar, revestido por essas ervas.
\section{Relvado}
\begin{itemize}
\item {Grp. gram.:m.}
\end{itemize}
\begin{itemize}
\item {Proveniência:(De \textunderscore relvar\textunderscore )}
\end{itemize}
Terreno, coberto de relva.
\section{Relvão}
\begin{itemize}
\item {Grp. gram.:adj.}
\end{itemize}
\begin{itemize}
\item {Grp. gram.:M.}
\end{itemize}
Que pasta na relva.
Que vive na relva.
Terreno, em que há relva crescida.
\section{Relvar}
\begin{itemize}
\item {Grp. gram.:v. t.}
\end{itemize}
\begin{itemize}
\item {Utilização:Prov.}
\end{itemize}
\begin{itemize}
\item {Utilização:trasm.}
\end{itemize}
\begin{itemize}
\item {Utilização:Prov.}
\end{itemize}
\begin{itemize}
\item {Utilização:trasm.}
\end{itemize}
\begin{itemize}
\item {Utilização:Fig.}
\end{itemize}
\begin{itemize}
\item {Grp. gram.:V. i.}
\end{itemize}
\begin{itemize}
\item {Utilização:Prov.}
\end{itemize}
\begin{itemize}
\item {Utilização:Ribatejo.}
\end{itemize}
Cobrir de relva.
Dar primeira lavra á (terra) na primavera.
Lavrar o milhão em (o restolho do centeio).
Repetir.
Relvejar.
Pastar na relva.
\section{Relvedo}
\begin{itemize}
\item {fónica:vê}
\end{itemize}
\begin{itemize}
\item {Grp. gram.:m.}
\end{itemize}
Lugar, onde cresce relva; relvado.
\section{Relvejar}
\begin{itemize}
\item {Grp. gram.:v. i.}
\end{itemize}
Cobrir-se de relva; mostrar-se coberto de relva.
\section{Relvoso}
\begin{itemize}
\item {Grp. gram.:adj.}
\end{itemize}
Em que há relva.
\section{Rem}
\begin{itemize}
\item {Grp. gram.:f.}
\end{itemize}
\begin{itemize}
\item {Utilização:Ant.}
\end{itemize}
\begin{itemize}
\item {Proveniência:(Lat. \textunderscore rem\textunderscore , accus. de \textunderscore res\textunderscore )}
\end{itemize}
Coisa; coisa nenhuma:«\textunderscore nom cata rem do que catar devia\textunderscore ». \textunderscore Cancion. de D. Dinis\textunderscore .
\section{Remada}
\begin{itemize}
\item {Grp. gram.:f.}
\end{itemize}
Pancada de remo.
Acto de impellir com o remo.
\section{Remadela}
\begin{itemize}
\item {Grp. gram.:f.}
\end{itemize}
\begin{itemize}
\item {Utilização:Fam.}
\end{itemize}
O mesmo que \textunderscore remada\textunderscore .
\section{Remador}
\begin{itemize}
\item {Grp. gram.:m.  e  adj.}
\end{itemize}
O que rema; remeiro.
\section{Remadura}
\begin{itemize}
\item {Grp. gram.:f.}
\end{itemize}
Acto ou effeito de remar.
\section{Remaescer}
\begin{itemize}
\item {fónica:ma-es}
\end{itemize}
\begin{itemize}
\item {Grp. gram.:v. i.}
\end{itemize}
\begin{itemize}
\item {Utilização:Ant.}
\end{itemize}
\begin{itemize}
\item {Proveniência:(Do lat. \textunderscore remanescere\textunderscore )}
\end{itemize}
Ficar; restar; sobejar.
\section{Remanchão}
\begin{itemize}
\item {Grp. gram.:adj.}
\end{itemize}
\begin{itemize}
\item {Utilização:Pop.}
\end{itemize}
\begin{itemize}
\item {Proveniência:(De \textunderscore remanchar\textunderscore ^2)}
\end{itemize}
Que remancha; pachorrento.
\section{Remanchar}
\begin{itemize}
\item {Grp. gram.:v. t.}
\end{itemize}
\begin{itemize}
\item {Utilização:T. de latoaria}
\end{itemize}
\begin{itemize}
\item {Proveniência:(Do cast. \textunderscore remachar\textunderscore )}
\end{itemize}
Fazer borda com o maço no fundo de (panelas, cafeteiras, etc.) sôbre a bigorna.
\section{Remanchar}
\begin{itemize}
\item {Grp. gram.:v. i.}
\end{itemize}
\begin{itemize}
\item {Grp. gram.:V. p.}
\end{itemize}
\begin{itemize}
\item {Proveniência:(De \textunderscore remancho\textunderscore )}
\end{itemize}
Tardar; demorar-se.
Sêr pachorrento.
Vir ou andar de vagar.
\section{Remancho}
\begin{itemize}
\item {Grp. gram.:m.}
\end{itemize}
\begin{itemize}
\item {Utilização:Des.}
\end{itemize}
Pachorra; indolência; descanso de mandrião.
(Do \textunderscore remanso\textunderscore ?)
\section{Remanecer}
\textunderscore v. i.\textunderscore  (e der.)
O mesmo que \textunderscore remanescer\textunderscore , etc.
\section{Remanente}
\begin{itemize}
\item {Grp. gram.:adv.}
\end{itemize}
\begin{itemize}
\item {Proveniência:(Lat. \textunderscore remanens\textunderscore )}
\end{itemize}
O mesmo que \textunderscore remanescente\textunderscore .
\section{Remanescente}
\begin{itemize}
\item {Grp. gram.:adj.}
\end{itemize}
\begin{itemize}
\item {Grp. gram.:M.}
\end{itemize}
Que remanesce.
Aquillo que sobeja ou resta: \textunderscore deixou vários legados, ficando o remanescente da herança para o sobrinho\textunderscore .
\section{Remanescer}
\begin{itemize}
\item {Grp. gram.:v. i.}
\end{itemize}
\begin{itemize}
\item {Proveniência:(Do lat. \textunderscore remanere\textunderscore )}
\end{itemize}
Restar; sobrar.
\section{Remangar}
\begin{itemize}
\item {Grp. gram.:v. i.  e  p.}
\end{itemize}
O mesmo que \textunderscore arremangar\textunderscore .
\section{Remanipular}
\begin{itemize}
\item {Grp. gram.:v. t.}
\end{itemize}
\begin{itemize}
\item {Proveniência:(De \textunderscore re...\textunderscore  + \textunderscore manipular\textunderscore )}
\end{itemize}
Manipular novamente.
\section{Remanir}
\begin{itemize}
\item {Grp. gram.:v. i.}
\end{itemize}
\begin{itemize}
\item {Utilização:Ant.}
\end{itemize}
\begin{itemize}
\item {Proveniência:(De \textunderscore re...\textunderscore  + lat. \textunderscore manere\textunderscore )}
\end{itemize}
O mesmo que \textunderscore permanecer\textunderscore .
\section{Remansado}
\begin{itemize}
\item {Grp. gram.:adj.}
\end{itemize}
\begin{itemize}
\item {Utilização:Fig.}
\end{itemize}
\begin{itemize}
\item {Proveniência:(De \textunderscore remansar\textunderscore )}
\end{itemize}
Pachorrento; vagaroso.
\section{Remansão}
\begin{itemize}
\item {Grp. gram.:m.  e  adj.}
\end{itemize}
\begin{itemize}
\item {Proveniência:(De \textunderscore remanso\textunderscore )}
\end{itemize}
O mesmo que \textunderscore mansarrão\textunderscore .
\section{Remansar-se}
\begin{itemize}
\item {Grp. gram.:v. p.}
\end{itemize}
(V.arremansar-se)
\section{Remansear}
\begin{itemize}
\item {Grp. gram.:v. i.  e  p.}
\end{itemize}
\begin{itemize}
\item {Proveniência:(De \textunderscore remanso\textunderscore ^1)}
\end{itemize}
Arremansar-se; tornar-se pachorrento; estar tranquillo.
\section{Remanso}
\begin{itemize}
\item {Grp. gram.:m.}
\end{itemize}
\begin{itemize}
\item {Proveniência:(Lat. \textunderscore remansus\textunderscore )}
\end{itemize}
Cessação de movimento.
Tranquilidade.
Quietação.
Recolhimento: \textunderscore no remanso do lar\textunderscore .
Água estagnada.
\section{Remanso}
\begin{itemize}
\item {Grp. gram.:m.}
\end{itemize}
\begin{itemize}
\item {Utilização:T. da Bairrada}
\end{itemize}
Jeito, habilidade ou ária no manejar de um instrumento de trabalho.
\section{Remansoso}
\begin{itemize}
\item {Grp. gram.:adj.}
\end{itemize}
O mesmo que \textunderscore remansado\textunderscore .
\section{Remanusear}
\begin{itemize}
\item {Grp. gram.:v. t.}
\end{itemize}
\begin{itemize}
\item {Proveniência:(De \textunderscore re...\textunderscore  + \textunderscore manusear\textunderscore )}
\end{itemize}
Manusear de novo.
Manusear muitas vezes. Cf. Filinto, VI, 226.
\section{Remar}
\begin{itemize}
\item {Grp. gram.:v. t.}
\end{itemize}
\begin{itemize}
\item {Utilização:Ant.}
\end{itemize}
\begin{itemize}
\item {Grp. gram.:V. i.}
\end{itemize}
\begin{itemize}
\item {Utilização:Fig.}
\end{itemize}
Impellir com o auxílio dos remos.
Prover de remos (um barco).
Mover os remos.
Nadar.
Adejar.
Esforçar-se, lutar.
\textunderscore Remar contra a maré\textunderscore , esforçar-se inutilmente.
\section{Remarcar}
\begin{itemize}
\item {Grp. gram.:v. t.}
\end{itemize}
\begin{itemize}
\item {Proveniência:(De \textunderscore re...\textunderscore  + \textunderscore marcar\textunderscore )}
\end{itemize}
Pôr marca nova.
Contrastar (peças de ourivezaria ou artefactos de prata). Cf. \textunderscore Inquér. Industr.\textunderscore , parte I, 62 e 63.
\section{Remaridar-se}
\begin{itemize}
\item {Grp. gram.:v. p.}
\end{itemize}
\begin{itemize}
\item {Proveniência:(De \textunderscore re...\textunderscore  + \textunderscore maridar\textunderscore )}
\end{itemize}
Casar novamente, (falando-se da mulher). Cf. \textunderscore Ordenações Filip.\textunderscore 
\section{Remartelar}
\begin{itemize}
\item {Grp. gram.:v. t.}
\end{itemize}
\begin{itemize}
\item {Proveniência:(De \textunderscore re...\textunderscore  + \textunderscore martelar\textunderscore )}
\end{itemize}
Martelar novamente; martelar muito.
\section{Remascar}
\begin{itemize}
\item {Grp. gram.:v. t.}
\end{itemize}
\begin{itemize}
\item {Proveniência:(De \textunderscore re...\textunderscore  + \textunderscore mascar\textunderscore )}
\end{itemize}
Mascar de novo; ruminar.
\section{Remasse}
\begin{itemize}
\item {Grp. gram.:m.}
\end{itemize}
Instrumento de espingardeiro.
\section{Remastigação}
\begin{itemize}
\item {Grp. gram.:f.}
\end{itemize}
Acto ou effeito de remastigar.
\section{Remastigar}
\begin{itemize}
\item {Grp. gram.:v. t.}
\end{itemize}
\begin{itemize}
\item {Proveniência:(De \textunderscore re...\textunderscore  + \textunderscore mastigar\textunderscore )}
\end{itemize}
Tornar a mastigar; mastigar bem; ruminar.
\section{Rematação}
\begin{itemize}
\item {Grp. gram.:f.}
\end{itemize}
\begin{itemize}
\item {Utilização:Ant.}
\end{itemize}
O mesmo que \textunderscore arrematação\textunderscore .
Remate; epílogo.
\section{Rematadamente}
\begin{itemize}
\item {Grp. gram.:adv.}
\end{itemize}
De modo rematado.
\section{Rematado}
\begin{itemize}
\item {Grp. gram.:adj.}
\end{itemize}
Encimado, sobreposto.
Concluído.
Completo: \textunderscore pateta rematado\textunderscore .
\section{Rematador}
\begin{itemize}
\item {Grp. gram.:m.  e  adj.}
\end{itemize}
O que remata.
\section{Rematar}
\begin{itemize}
\item {Grp. gram.:v. t.}
\end{itemize}
\begin{itemize}
\item {Grp. gram.:V. i.}
\end{itemize}
\begin{itemize}
\item {Proveniência:(De \textunderscore re...\textunderscore  + \textunderscore matar\textunderscore ?)}
\end{itemize}
Dar remate a.
Pôr fim a.
Concluír; completar; terminar.
Têr fim, concluír-se.
\section{Remate}
\begin{itemize}
\item {Grp. gram.:m.}
\end{itemize}
\begin{itemize}
\item {Utilização:Fig.}
\end{itemize}
\begin{itemize}
\item {Proveniência:(De \textunderscore rematar\textunderscore )}
\end{itemize}
Conclusão.
Acto de fechar ou concluir.
Effeito.
Fecho de uma obra de architectura.
O ponto mais elevado; o auge.
\section{Remau}
\begin{itemize}
\item {Grp. gram.:adj.}
\end{itemize}
\begin{itemize}
\item {Utilização:P. us.}
\end{itemize}
\begin{itemize}
\item {Proveniência:(De \textunderscore re...\textunderscore  + \textunderscore mau\textunderscore )}
\end{itemize}
Muito mau. Cf. Alb. Pimentel, \textunderscore Chiado\textunderscore , 4 e 42.
\section{Remear}
\begin{itemize}
\item {Grp. gram.:v. t.}
\end{itemize}
\begin{itemize}
\item {Utilização:Des.}
\end{itemize}
\begin{itemize}
\item {Proveniência:(Lat. \textunderscore remeare\textunderscore )}
\end{itemize}
Tornar para, reentrar em. Cf. \textunderscore Viriato Trág.\textunderscore , (na \textunderscore Vida do autor\textunderscore ).
\section{Remedar}
\begin{itemize}
\item {Grp. gram.:v. t.}
\end{itemize}
O mesmo que \textunderscore arremedar\textunderscore . Cf. Camillo, \textunderscore Caveira\textunderscore , 30; \textunderscore Luz e Calor\textunderscore , 597.
\section{Remedeio}
\begin{itemize}
\item {Grp. gram.:m.}
\end{itemize}
\begin{itemize}
\item {Utilização:Pop.}
\end{itemize}
\begin{itemize}
\item {Proveniência:(De \textunderscore remediar\textunderscore )}
\end{itemize}
Aquillo que attenua uma falta ou um mal.
\section{Remediado}
\begin{itemize}
\item {Grp. gram.:adj.}
\end{itemize}
Que tem alguns haveres; que vive numa decente mediania.
\section{Remediador}
\begin{itemize}
\item {Grp. gram.:m.  e  adj.}
\end{itemize}
\begin{itemize}
\item {Proveniência:(Do b. lat. \textunderscore remediator\textunderscore )}
\end{itemize}
O que remedeia.
\section{Remediar}
\begin{itemize}
\item {Grp. gram.:v. t.}
\end{itemize}
\begin{itemize}
\item {Proveniência:(Lat. \textunderscore remediare\textunderscore )}
\end{itemize}
Dar remédio a.
Obstar com previdência a.
Atalhar, obstar.
\section{Remediável}
\begin{itemize}
\item {Grp. gram.:adj.}
\end{itemize}
\begin{itemize}
\item {Proveniência:(Do lat. \textunderscore remediabilis\textunderscore )}
\end{itemize}
Que se póde remediar.
\section{Remedição}
\begin{itemize}
\item {Grp. gram.:f.}
\end{itemize}
Acto ou effeito de remedir.
\section{Remédio}
\begin{itemize}
\item {Grp. gram.:m.}
\end{itemize}
\begin{itemize}
\item {Utilização:T. de Turquel}
\end{itemize}
\begin{itemize}
\item {Utilização:Gír.}
\end{itemize}
\begin{itemize}
\item {Proveniência:(Lat. \textunderscore remedium\textunderscore )}
\end{itemize}
Aquillo que póde causar uma mudança salutar no organismo em geral ou num órgão especial.
Aquillo que cura, ou a que se attribue a faculdade de curar.
Expediente.
Auxílio.
Emenda.
Productos agricolas de primeira necessidade; abundância de alguns delles: \textunderscore há êste anno muito remédio de trigo\textunderscore .
Explicação.
\section{Remedir}
\begin{itemize}
\item {Grp. gram.:v. t.}
\end{itemize}
\begin{itemize}
\item {Proveniência:(De \textunderscore re...\textunderscore  + \textunderscore medir\textunderscore )}
\end{itemize}
Medir de novo.
\section{Remêdo}
\begin{itemize}
\item {Grp. gram.:m.}
\end{itemize}
(V.arremêdo)
\section{Remeira}
\begin{itemize}
\item {Grp. gram.:f.}
\end{itemize}
\begin{itemize}
\item {Utilização:Ant.}
\end{itemize}
\begin{itemize}
\item {Proveniência:(De \textunderscore remo\textunderscore )}
\end{itemize}
O mesmo que \textunderscore bergantim\textunderscore . Cf. Fern. de Oliveira, \textunderscore Guerra do Mar\textunderscore , 43, v.^o.
\section{Remeiro}
\begin{itemize}
\item {Grp. gram.:adj.}
\end{itemize}
\begin{itemize}
\item {Grp. gram.:M.}
\end{itemize}
\begin{itemize}
\item {Proveniência:(De \textunderscore remo\textunderscore )}
\end{itemize}
Veloz, que obedece facilmente ao impulso dos remos.
Aquelle que rema; remador.
\section{Remel}
\begin{itemize}
\item {Grp. gram.:m.}
\end{itemize}
\begin{itemize}
\item {Utilização:Fig.}
\end{itemize}
\begin{itemize}
\item {Proveniência:(De \textunderscore re...\textunderscore  + \textunderscore mel\textunderscore )}
\end{itemize}
Grande doçura:«\textunderscore o mel e remel dos deleites humanos...\textunderscore »A. Rosário, \textunderscore Frutos do Brasil\textunderscore .
\section{Remela}
\begin{itemize}
\item {Grp. gram.:f.}
\end{itemize}
\begin{itemize}
\item {Utilização:Prov.}
\end{itemize}
Substância amarelada ou esbranquiçada, que se fórma geralmente nos pontos lacrimaes.
Reima da sardinha de salmoira. (Colhido na Bairrada)
(Por \textunderscore lamella\textunderscore ?)
\section{Remelado}
\begin{itemize}
\item {Grp. gram.:adj.}
\end{itemize}
O mesmo que \textunderscore remeloso\textunderscore .
\section{Remelão}
\begin{itemize}
\item {Grp. gram.:adj.}
\end{itemize}
\begin{itemize}
\item {Proveniência:(De \textunderscore re...\textunderscore  + \textunderscore mel\textunderscore )}
\end{itemize}
Remeloso.
Diz-se do açúcar, que tem a crôsta queimada, ficando como mel, sem se granular.
\section{Remelar}
\begin{itemize}
\item {Grp. gram.:v. i.  e  p.}
\end{itemize}
Criar remela; tornar-se remelão.
\section{Remeleiro}
\begin{itemize}
\item {Grp. gram.:adj.}
\end{itemize}
O mesmo que \textunderscore remeloso\textunderscore .
\section{Remelga}
\begin{itemize}
\item {Grp. gram.:f.}
\end{itemize}
\begin{itemize}
\item {Utilização:Prov.}
\end{itemize}
\begin{itemize}
\item {Utilização:Chul.}
\end{itemize}
Batota ou jôgo de asar, entre parceiros ordinários ou pouco endinheirados.
Batota pataqueira.
\section{Remelgado}
\begin{itemize}
\item {Grp. gram.:adj.}
\end{itemize}
\begin{itemize}
\item {Utilização:Pop.}
\end{itemize}
\begin{itemize}
\item {Proveniência:(De \textunderscore remela\textunderscore )}
\end{itemize}
Que tem reviradas as bordas das pálpebras: \textunderscore olhos remelgados\textunderscore .
\section{Remelgueira}
\begin{itemize}
\item {Grp. gram.:f.}
\end{itemize}
\begin{itemize}
\item {Utilização:Prov.}
\end{itemize}
\begin{itemize}
\item {Utilização:beir.}
\end{itemize}
\begin{itemize}
\item {Utilização:fam.}
\end{itemize}
\begin{itemize}
\item {Proveniência:(De \textunderscore re...\textunderscore  + \textunderscore melgueira\textunderscore )}
\end{itemize}
Grande melgueira, grande pechincha.
\section{Remelgueiro}
\begin{itemize}
\item {Grp. gram.:m.}
\end{itemize}
\begin{itemize}
\item {Utilização:Prov.}
\end{itemize}
Indivíduo, muito dado a remelgas; batoteiro ordinário.
\section{Remelhór}
\begin{itemize}
\item {Grp. gram.:adj.}
\end{itemize}
Muito melhór. Cf. Gil Vicente, A. Prestes, Chiado, etc.
\section{Remeloso}
\begin{itemize}
\item {Grp. gram.:adj.}
\end{itemize}
Que tem ou cria remela.
\section{Remembrança}
\begin{itemize}
\item {Grp. gram.:f.}
\end{itemize}
\begin{itemize}
\item {Utilização:Ant.}
\end{itemize}
Acto ou effeito de remembrar. Cf. Fern. Lopes, \textunderscore Chrón. de D. João I\textunderscore , IV, 200.
\section{Remembrar}
\begin{itemize}
\item {Grp. gram.:v. i.}
\end{itemize}
\begin{itemize}
\item {Utilização:Ant.}
\end{itemize}
\begin{itemize}
\item {Proveniência:(Do lat. \textunderscore rememorari\textunderscore )}
\end{itemize}
O mesmo que \textunderscore relembrar\textunderscore .
\section{Rememoração}
\begin{itemize}
\item {Grp. gram.:f.}
\end{itemize}
\begin{itemize}
\item {Proveniência:(Lat. \textunderscore rememoratio\textunderscore )}
\end{itemize}
Acto ou effeito de rememorar.
\section{Rememorar}
\begin{itemize}
\item {Grp. gram.:v. t.}
\end{itemize}
\begin{itemize}
\item {Utilização:Fig.}
\end{itemize}
\begin{itemize}
\item {Proveniência:(Lat. \textunderscore rememorari\textunderscore )}
\end{itemize}
Tornar a lembrar.
Têr semelhança ou dar ideia imperfeita de.
\section{Rememorativo}
\begin{itemize}
\item {Grp. gram.:adj.}
\end{itemize}
Que rememora.
\section{Rememorável}
\begin{itemize}
\item {Grp. gram.:adj.}
\end{itemize}
\begin{itemize}
\item {Proveniência:(De \textunderscore rememorar\textunderscore )}
\end{itemize}
Digno de sêr rememorado; famoso.
\section{Remémoro}
\begin{itemize}
\item {Grp. gram.:adj.}
\end{itemize}
\begin{itemize}
\item {Utilização:Poét.}
\end{itemize}
Que rememora, que se lembra.
\section{Remendadamente}
\begin{itemize}
\item {Grp. gram.:adv.}
\end{itemize}
De modo remendado; com remendos.
\section{Remendagem}
\begin{itemize}
\item {Grp. gram.:f.}
\end{itemize}
Acto de remendar. Cf. Castilho, \textunderscore D. Quixote\textunderscore , I, 275.
\section{Remendão}
\begin{itemize}
\item {Grp. gram.:m.  e  adj.}
\end{itemize}
\begin{itemize}
\item {Proveniência:(De \textunderscore remendar\textunderscore )}
\end{itemize}
O que deita remendos.
Maltrapilho.
Indivíduo inhábil num offício; sarrafaçal.
\section{Remendar}
\begin{itemize}
\item {Grp. gram.:v. t.}
\end{itemize}
\begin{itemize}
\item {Utilização:Fig.}
\end{itemize}
\begin{itemize}
\item {Proveniência:(De \textunderscore re...\textunderscore  + \textunderscore emendar\textunderscore )}
\end{itemize}
Deitar remendos em.
Consertar com farrapos.
Fazer mistura de (coisas distintas ou oppostas).
Mesclar de estrangeirismos ou de locuções impróprias (a linguagem vernácula).
\section{Remendeira}
\begin{itemize}
\item {Grp. gram.:f.}
\end{itemize}
\begin{itemize}
\item {Proveniência:(De \textunderscore remendeiro\textunderscore )}
\end{itemize}
Mulhér, que deita remendos.
\section{Remendeiro}
\begin{itemize}
\item {Grp. gram.:m.  e  adj.}
\end{itemize}
O mesmo que \textunderscore remendão\textunderscore .
\section{Remendo}
\begin{itemize}
\item {Grp. gram.:m.}
\end{itemize}
\begin{itemize}
\item {Utilização:Pop.}
\end{itemize}
\begin{itemize}
\item {Proveniência:(De \textunderscore remendar\textunderscore )}
\end{itemize}
Pedaço de pano, com que se conserta uma parte do vestuário ou qualquer outro tecido.
Emenda.
Peça de madeira, metal, coiro, etc., com que se conserta um objecto de substância idêntica.
Qualquer consêrto.
Malha na pelle de animaes.
Remédio moral.
Disfarce de um defeito ou de qualquer inadvertência no falar.
\section{Remendona}
\begin{itemize}
\item {Grp. gram.:f.}
\end{itemize}
Mulhér, que deita remendos.
Mulhér inhábil ou desajeitada.
(Cp. \textunderscore remendão\textunderscore )
\section{Remeneio}
\begin{itemize}
\item {Grp. gram.:m.}
\end{itemize}
\begin{itemize}
\item {Proveniência:(De \textunderscore re...\textunderscore  + \textunderscore menear\textunderscore )}
\end{itemize}
Derrengue; saracoteadela.
Movimentos dengues, affectados. Cf. Filinto, IV, 241.
\section{Remenicar}
\begin{itemize}
\item {Grp. gram.:v. i.}
\end{itemize}
\begin{itemize}
\item {Utilização:Chul.}
\end{itemize}
Recalcitrar; refilar.
\section{Remense}
\begin{itemize}
\item {Grp. gram.:adj.}
\end{itemize}
Relativo á cidade de Reims. Cf. Herculano, \textunderscore Hist. de Port.\textunderscore , I, 349.
\section{Rementir}
\begin{itemize}
\item {Grp. gram.:v. i.}
\end{itemize}
\begin{itemize}
\item {Proveniência:(De \textunderscore re...\textunderscore  + \textunderscore mentir\textunderscore )}
\end{itemize}
Mentir outra vez.
Mentir muitas vezes. Cf. Castilho, \textunderscore D. Quixote\textunderscore , I, 356.
\section{Remercear}
\begin{itemize}
\item {Grp. gram.:v. t.}
\end{itemize}
\begin{itemize}
\item {Utilização:Ant.}
\end{itemize}
\begin{itemize}
\item {Proveniência:(De \textunderscore mercê\textunderscore )}
\end{itemize}
O mesmo que \textunderscore agradecer\textunderscore .
\section{Remerecedor}
\begin{itemize}
\item {Grp. gram.:adj.}
\end{itemize}
Que remerece.
\section{Remerecer}
\begin{itemize}
\item {Grp. gram.:v. t.}
\end{itemize}
\begin{itemize}
\item {Proveniência:(De \textunderscore re...\textunderscore  + \textunderscore merecer\textunderscore )}
\end{itemize}
Merecer em alto grau.
Merecer (mais do que aquillo que recebe).
\section{Remergulhar}
\begin{itemize}
\item {Grp. gram.:v. t.}
\end{itemize}
\begin{itemize}
\item {Proveniência:(De \textunderscore re...\textunderscore  + \textunderscore mergulhar\textunderscore )}
\end{itemize}
Afundar de novo.
Meter novamente debaixo de água. Cf. Castilho, \textunderscore Geórgicas\textunderscore , 199.
\section{Remessa}
\begin{itemize}
\item {Grp. gram.:f.}
\end{itemize}
\begin{itemize}
\item {Proveniência:(De \textunderscore remessar\textunderscore )}
\end{itemize}
Acto ou effeito de remeter.
Aquillo que se remeteu.
\section{Remessa}
\begin{itemize}
\item {Grp. gram.:f.}
\end{itemize}
\begin{itemize}
\item {Utilização:Prov.}
\end{itemize}
\begin{itemize}
\item {Utilização:beir.}
\end{itemize}
Vara das videiras, poupada pelos podadores, para a producção das uvas:«\textunderscore ...os poucos olhos, que o podador deixa cada vara\textunderscore  ou \textunderscore remessa, como dizem os vinhateiros da Beira...\textunderscore »Th. Ribeiro, \textunderscore Jornadas\textunderscore , I, 37.
(Relaciona-se com \textunderscore remissa\textunderscore ? Cp. \textunderscore remêsso\textunderscore )
\section{Remessão}
\begin{itemize}
\item {Grp. gram.:m.}
\end{itemize}
\begin{itemize}
\item {Utilização:Ant.}
\end{itemize}
(V.arremessão)
Arma de arremêsso. Cf. Usque, 30.
\section{Remessar}
\begin{itemize}
\item {Grp. gram.:v. t.}
\end{itemize}
\begin{itemize}
\item {Grp. gram.:V. i.}
\end{itemize}
\begin{itemize}
\item {Utilização:Ant.}
\end{itemize}
\begin{itemize}
\item {Proveniência:(De \textunderscore remêsso\textunderscore )}
\end{itemize}
O mesmo que \textunderscore arremessar\textunderscore :«\textunderscore a moéda, remessada á tôa...\textunderscore »Camillo, \textunderscore Retr. de Ricard.\textunderscore , 220.
Ir de encontro.
Bater com fôrça.
Cair sôbre alguma coisa.
\section{Remêsso}
\begin{itemize}
\item {Grp. gram.:m.}
\end{itemize}
\begin{itemize}
\item {Grp. gram.:Adj.}
\end{itemize}
\begin{itemize}
\item {Proveniência:(Do lat. \textunderscore remissus\textunderscore )}
\end{itemize}
O mesmo que \textunderscore arremêsso\textunderscore .
Arma de arremêsso.
Que se atira; que serve para se arremessar. Cf. Filinto, VI, 235.
\section{Remestre}
\begin{itemize}
\item {Grp. gram.:m.}
\end{itemize}
\begin{itemize}
\item {Utilização:Ant.}
\end{itemize}
\begin{itemize}
\item {Proveniência:(De \textunderscore re...\textunderscore  + \textunderscore mestre\textunderscore )}
\end{itemize}
Duas vezes mestre.
Homem muito sabedor.
\section{Remetedura}
\begin{itemize}
\item {Grp. gram.:f.}
\end{itemize}
\begin{itemize}
\item {Proveniência:(De \textunderscore remeter\textunderscore )}
\end{itemize}
Acto ou effeito de arremeter.
\section{Remetente}
\begin{itemize}
\item {Grp. gram.:m. ,  f.  e  adj.}
\end{itemize}
\begin{itemize}
\item {Proveniência:(Do lat. \textunderscore remittens\textunderscore )}
\end{itemize}
Pessôa, que remete.
\section{Remeter}
\begin{itemize}
\item {Grp. gram.:v. t.}
\end{itemize}
\begin{itemize}
\item {Grp. gram.:V. i.}
\end{itemize}
\begin{itemize}
\item {Grp. gram.:V. p.}
\end{itemize}
\begin{itemize}
\item {Proveniência:(Do lat. \textunderscore remittere\textunderscore )}
\end{itemize}
Mandar, enviar.
Fazer encomenda de.
Recommendar.
Sujeitar.
Fazer entrega de.
Espaçar.
Arremeter.
Arrojar-se.
Confiar-se.
Fazer referência, alludir.
\section{Remetida}
\begin{itemize}
\item {Grp. gram.:f.}
\end{itemize}
O mesmo que \textunderscore remetimento\textunderscore .
\section{Remetimento}
\begin{itemize}
\item {Grp. gram.:m.}
\end{itemize}
O mesmo que \textunderscore arremetida\textunderscore .
\section{Remexedor}
\begin{itemize}
\item {Grp. gram.:adj.}
\end{itemize}
Que remexe. Cf. Arn. Gama, \textunderscore Motim\textunderscore , 122.
\section{Remexer}
\begin{itemize}
\item {Grp. gram.:v. t.}
\end{itemize}
\begin{itemize}
\item {Grp. gram.:V. i.}
\end{itemize}
\begin{itemize}
\item {Proveniência:(De \textunderscore re...\textunderscore  + \textunderscore mexer\textunderscore )}
\end{itemize}
Mexer novamente, mexer muitas vezes.
Sacudir.
Mexer-se; agitar-se.
\section{Remexida}
\begin{itemize}
\item {Grp. gram.:f.}
\end{itemize}
\begin{itemize}
\item {Utilização:Fam.}
\end{itemize}
Acto ou effeito de remexer.
Trapalhada, confusão.
\section{Remexido}
\begin{itemize}
\item {Grp. gram.:adj.}
\end{itemize}
\begin{itemize}
\item {Utilização:Fam.}
\end{itemize}
\begin{itemize}
\item {Proveniência:(De \textunderscore remexer\textunderscore )}
\end{itemize}
Inquieto; traquinas.
\section{Remição}
\begin{itemize}
\item {Grp. gram.:f.}
\end{itemize}
Acto de remir: \textunderscore a remissão dos cativos\textunderscore .
\section{Remido}
\begin{itemize}
\item {Grp. gram.:adj.}
\end{itemize}
\begin{itemize}
\item {Proveniência:(De \textunderscore remir\textunderscore )}
\end{itemize}
Resgatado.
\section{Remidor}
\begin{itemize}
\item {Grp. gram.:m.}
\end{itemize}
\begin{itemize}
\item {Utilização:Ant.}
\end{itemize}
\begin{itemize}
\item {Proveniência:(De \textunderscore remir\textunderscore )}
\end{itemize}
O mesmo que \textunderscore redemptor\textunderscore .
\section{Rêmige}
\begin{itemize}
\item {Grp. gram.:adj.}
\end{itemize}
\begin{itemize}
\item {Proveniência:(Do lat. \textunderscore remex\textunderscore )}
\end{itemize}
Que rema.
\section{Rêmiges}
\begin{itemize}
\item {Grp. gram.:f. pl.}
\end{itemize}
\begin{itemize}
\item {Proveniência:(Do lat. \textunderscore remex\textunderscore )}
\end{itemize}
As pennas mais compridas das asas das aves; remígio.
\section{Remígias}
\begin{itemize}
\item {Grp. gram.:f. pl.}
\end{itemize}
\begin{itemize}
\item {Utilização:Neol.}
\end{itemize}
\begin{itemize}
\item {Proveniência:(Lat. \textunderscore remigia\textunderscore , pl. de \textunderscore remigium\textunderscore )}
\end{itemize}
O mesmo que \textunderscore rêmiges\textunderscore . Cf. Coêlho Neto.
\section{Remígio}
\begin{itemize}
\item {Grp. gram.:m.}
\end{itemize}
\begin{itemize}
\item {Proveniência:(Lat. \textunderscore remigium\textunderscore )}
\end{itemize}
As pennas mais compridas das asas; guias; vôo.
\section{Remigração}
\begin{itemize}
\item {Grp. gram.:f.}
\end{itemize}
Acto ou effeito de remigrar.
\section{Remigrado}
\begin{itemize}
\item {Grp. gram.:adj.}
\end{itemize}
\begin{itemize}
\item {Proveniência:(De \textunderscore remigrar\textunderscore )}
\end{itemize}
Que remigrou.
\section{Remigrar}
\begin{itemize}
\item {Grp. gram.:v. i.}
\end{itemize}
\begin{itemize}
\item {Proveniência:(Lat. \textunderscore remigrare\textunderscore )}
\end{itemize}
Voltar ao ponto ou lugar donde se tinha emigrado.
\section{Remilhão}
\begin{itemize}
\item {Grp. gram.:m.}
\end{itemize}
(Corr. de \textunderscore reminhol\textunderscore )
\section{Remilhenta}
\begin{itemize}
\item {Grp. gram.:adj.}
\end{itemize}
Designação infantil de um número superior a milhenta.
\section{Remimar}
\begin{itemize}
\item {Grp. gram.:v. t.}
\end{itemize}
\begin{itemize}
\item {Utilização:Ant.}
\end{itemize}
O mesmo que \textunderscore remir\textunderscore , resgatar.
(Cp. \textunderscore remimento\textunderscore )
\section{Remimento}
\begin{itemize}
\item {Grp. gram.:m.}
\end{itemize}
Acto de remir.
Redempção; resgate.
\section{Reminado}
\begin{itemize}
\item {Grp. gram.:adj.}
\end{itemize}
\begin{itemize}
\item {Utilização:Bras. do N}
\end{itemize}
\begin{itemize}
\item {Proveniência:(De \textunderscore reminar-se\textunderscore )}
\end{itemize}
Revoltado.
\section{Reminar-se}
\begin{itemize}
\item {Grp. gram.:v. p.}
\end{itemize}
\begin{itemize}
\item {Utilização:Bras. do N}
\end{itemize}
Insurgir-se contra alguém; revoltar-se.
(Cp. \textunderscore remenicar\textunderscore )
\section{Reminha}
\begin{itemize}
\item {Grp. gram.:pron.}
\end{itemize}
\begin{itemize}
\item {Proveniência:(De \textunderscore re...\textunderscore  + \textunderscore minha\textunderscore )}
\end{itemize}
Muito minha:«\textunderscore hoje é minha e reminha\textunderscore ». Castilho, \textunderscore Tartufo\textunderscore , 139.
\section{Reminhol}
\begin{itemize}
\item {Grp. gram.:m.}
\end{itemize}
\begin{itemize}
\item {Utilização:Bras}
\end{itemize}
Espécie de colhér de cobre, com que se mexe o açúcar nos engenhos.
\section{Reminicar}
\textunderscore v. i.\textunderscore  (e der.)
O mesmo que \textunderscore remenicar\textunderscore , etc.
\section{Reminiscência}
\begin{itemize}
\item {Grp. gram.:f.}
\end{itemize}
\begin{itemize}
\item {Proveniência:(Lat. \textunderscore reminiscentia\textunderscore )}
\end{itemize}
Faculdade de reter e reproduzir conhecimentos adquiridos.
Memória, recordação.
Conhecimcnto, que se fixou na memória.
Aquillo que se conserva na memória.
\section{Remípede}
\begin{itemize}
\item {Grp. gram.:adj.}
\end{itemize}
\begin{itemize}
\item {Utilização:Zool.}
\end{itemize}
\begin{itemize}
\item {Grp. gram.:M.}
\end{itemize}
\begin{itemize}
\item {Grp. gram.:Pl.}
\end{itemize}
\begin{itemize}
\item {Proveniência:(Do lat. \textunderscore remipes\textunderscore , \textunderscore remipedis\textunderscore )}
\end{itemize}
Que tem pés semelhantes a remos.
Gênero de crustáceos decápodes.
Família de insectos coleópteros, quo tem tarsos próprios para a natação.
\section{Remir}
\begin{itemize}
\item {Grp. gram.:v. t.}
\end{itemize}
\begin{itemize}
\item {Proveniência:(Do lat. \textunderscore redimire\textunderscore )}
\end{itemize}
Adquirir de novo.
Resgatar: \textunderscore remir cativos\textunderscore .
Tirar do poder alheio.
Salvar.
Livrar do mal.
Indemnizar.
Tornar esquecido.
Exonerar.
Desempenhar: \textunderscore remir um relógio, dado em penhor de dívida\textunderscore .
\section{Remirar}
\begin{itemize}
\item {Grp. gram.:v. t.}
\end{itemize}
\begin{itemize}
\item {Proveniência:(De \textunderscore re...\textunderscore  + \textunderscore mirar\textunderscore )}
\end{itemize}
Mirar novamente; mirar muito.
Contemplar; observar attentamente.
\section{Remírea}
\begin{itemize}
\item {Grp. gram.:f.}
\end{itemize}
\begin{itemize}
\item {Proveniência:(De \textunderscore Remirez\textunderscore , n. p.)}
\end{itemize}
Gênero de plantas cyperáceas da Guiana.
\section{Remisga}
\begin{itemize}
\item {Grp. gram.:f.}
\end{itemize}
\begin{itemize}
\item {Utilização:Prov.}
\end{itemize}
\begin{itemize}
\item {Utilização:trasm.}
\end{itemize}
Vestígios, restos de qualquer coisa.
\section{Remissa}
\begin{itemize}
\item {Grp. gram.:f.}
\end{itemize}
\begin{itemize}
\item {Utilização:Fig.}
\end{itemize}
\begin{itemize}
\item {Proveniência:(Lat. \textunderscore remissa\textunderscore )}
\end{itemize}
Quantia, reposta por um parceiro, no jôgo do voltarete.
Adiamento, reserva: \textunderscore o teu negócio ficou de remissa\textunderscore .
\section{Remissamente}
\begin{itemize}
\item {Grp. gram.:adv.}
\end{itemize}
De modo remisso; froixamente; com negligência.
\section{Remissão}
\begin{itemize}
\item {Grp. gram.:f.}
\end{itemize}
\begin{itemize}
\item {Proveniência:(Lat. \textunderscore remissio\textunderscore )}
\end{itemize}
Acto ou effeito de remittir.
Falta de rigor.
Indulgência; perdão: \textunderscore a remissão dos peccados\textunderscore .
Interrupção.
Desânimo.
Froixidão.
Remessa.
Decrescimento temporário dos symptomas de uma doença.
Remittência.
Deminuição de intensidade.
\section{Remissível}
\begin{itemize}
\item {Grp. gram.:adj.}
\end{itemize}
\begin{itemize}
\item {Proveniência:(Lat. \textunderscore remissibilis\textunderscore )}
\end{itemize}
Que pode sêr remetido ou remitido.
\section{Remissivo}
\begin{itemize}
\item {Grp. gram.:adj.}
\end{itemize}
\begin{itemize}
\item {Proveniência:(Lat. \textunderscore remissivus\textunderscore )}
\end{itemize}
Que remitte.
Que remete para outro lugar: \textunderscore índice remissívo\textunderscore .
Allusivo; que faz referências.
\section{Remisso}
\begin{itemize}
\item {Grp. gram.:adj.}
\end{itemize}
\begin{itemize}
\item {Proveniência:(Lat. \textunderscore remissus\textunderscore )}
\end{itemize}
Indolente; descuidado.
Vagaroso; demorado.
Que tem menos intensidade.
\section{Remissor}
\begin{itemize}
\item {Grp. gram.:adj.}
\end{itemize}
O mesmo que \textunderscore remissório\textunderscore .
\section{Remissório}
\begin{itemize}
\item {Grp. gram.:adj.}
\end{itemize}
\begin{itemize}
\item {Proveniência:(De \textunderscore remisso\textunderscore )}
\end{itemize}
Que remitte; que contém remissão.
\section{Remitarso}
\begin{itemize}
\item {Grp. gram.:adj.}
\end{itemize}
\begin{itemize}
\item {Utilização:Zool.}
\end{itemize}
\begin{itemize}
\item {Proveniência:(Do lat. \textunderscore remus\textunderscore  + gr. \textunderscore tarsos\textunderscore )}
\end{itemize}
Que tem os tarsos em fórma de remo.
\section{Remitência}
\begin{itemize}
\item {Grp. gram.:f.}
\end{itemize}
\begin{itemize}
\item {Proveniência:(De \textunderscore remitente\textunderscore )}
\end{itemize}
Acto ou efeito de remitir.
Interrupção ou deminuição dos sintomas de uma doença.
\section{Remittência}
\begin{itemize}
\item {Grp. gram.:f.}
\end{itemize}
\begin{itemize}
\item {Proveniência:(De \textunderscore remittente\textunderscore )}
\end{itemize}
Acto ou effeito de remittir.
Interrupção ou deminuição dos symptomas de uma doença.
\section{Remitente}
\begin{itemize}
\item {Grp. gram.:adj.}
\end{itemize}
\begin{itemize}
\item {Proveniência:(Lat. \textunderscore remittens\textunderscore )}
\end{itemize}
Que remite; que tem remitência.
\section{Remitir}
\begin{itemize}
\item {Grp. gram.:v. t.}
\end{itemize}
\begin{itemize}
\item {Grp. gram.:V. i.}
\end{itemize}
\begin{itemize}
\item {Proveniência:(Lat. \textunderscore remittere\textunderscore )}
\end{itemize}
Perdoar.
Dar-se como pago de.
Devolver.
Restituir.
Tornar brando, menos intenso.
Têr intervalos ou deminuir de intensidade, (falando-se de doenças ou dos seus sintomas).
\section{Remittente}
\begin{itemize}
\item {Grp. gram.:adj.}
\end{itemize}
\begin{itemize}
\item {Proveniência:(Lat. \textunderscore remittens\textunderscore )}
\end{itemize}
Que remitte; que tem remittência.
\section{Remittir}
\begin{itemize}
\item {Grp. gram.:v. t.}
\end{itemize}
\begin{itemize}
\item {Grp. gram.:V. i.}
\end{itemize}
\begin{itemize}
\item {Proveniência:(Lat. \textunderscore remittere\textunderscore )}
\end{itemize}
Perdoar.
Dar-se como pago de.
Devolver.
Restituir.
Tornar brando, menos intenso.
Têr intervallos ou deminuir de intensidade, (falando-se de doenças ou dos seus symptomas).
\section{Remível}
\begin{itemize}
\item {Grp. gram.:adj.}
\end{itemize}
Quo se póde remir.
\section{Remo}
\begin{itemize}
\item {Grp. gram.:m.}
\end{itemize}
\begin{itemize}
\item {Proveniência:(Lat. \textunderscore remus\textunderscore )}
\end{itemize}
Instrumento de madeira, com que se fazem navegar as pequenas embarcações, e que consta de uma haste de madeira, presa por um estropo ao bordo da embarcação, e cuja extremidade mais larga mergulha na água.
\section{Remoçador}
\begin{itemize}
\item {Grp. gram.:m.  e  adj.}
\end{itemize}
O que remoça.
\section{Remoçante}
\begin{itemize}
\item {Grp. gram.:adj.}
\end{itemize}
Que remoça.
\section{Remoção}
\begin{itemize}
\item {Grp. gram.:f.}
\end{itemize}
\begin{itemize}
\item {Proveniência:(Do lat. \textunderscore remotio\textunderscore )}
\end{itemize}
Acto ou effeito de remover.
\section{Remoçar}
\begin{itemize}
\item {Grp. gram.:v. t.}
\end{itemize}
Apreciar com remoque; censurar.
\section{Remoçar}
\begin{itemize}
\item {Grp. gram.:v. i.}
\end{itemize}
\begin{itemize}
\item {Utilização:Prov.}
\end{itemize}
\begin{itemize}
\item {Utilização:alg.}
\end{itemize}
\begin{itemize}
\item {Utilização:Prov.}
\end{itemize}
\begin{itemize}
\item {Utilização:trasm.}
\end{itemize}
\begin{itemize}
\item {Utilização:Fam.}
\end{itemize}
Diz-se do arado, quando a relha encontra pedra ou outro obstáculo que a prende.
Entalar, fazer cair na esparrela.
\section{Remoçar}
\begin{itemize}
\item {Grp. gram.:v. t.}
\end{itemize}
\begin{itemize}
\item {Grp. gram.:V. i.}
\end{itemize}
\begin{itemize}
\item {Proveniência:(De \textunderscore re...\textunderscore  + \textunderscore moço\textunderscore )}
\end{itemize}
Tornar moço.
Dar frescor juvenil a.
Dar fôrça ou vigor a.
Tornar-se moço.
Rejuvenescer.
Readquirir vigor.
\section{Remodelação}
\begin{itemize}
\item {Grp. gram.:f.}
\end{itemize}
Acto ou effeito de remodelar.
\section{Remodelar}
\begin{itemize}
\item {Grp. gram.:v. t.}
\end{itemize}
\begin{itemize}
\item {Proveniência:(De \textunderscore re...\textunderscore  + \textunderscore modelar\textunderscore )}
\end{itemize}
Modelar novamente.
\section{Remoedura}
\begin{itemize}
\item {fónica:mo-e}
\end{itemize}
\begin{itemize}
\item {Grp. gram.:f.}
\end{itemize}
Acto de remoer.
\section{Remoéla}
\begin{itemize}
\item {Grp. gram.:f.}
\end{itemize}
\begin{itemize}
\item {Proveniência:(De \textunderscore remoer\textunderscore )}
\end{itemize}
Surriada, pirraça. Cf. \textunderscore Eufrosina\textunderscore , 185.
\section{Remoer}
\begin{itemize}
\item {Grp. gram.:v. t.}
\end{itemize}
\begin{itemize}
\item {Proveniência:(De \textunderscore re...\textunderscore  + \textunderscore moer\textunderscore )}
\end{itemize}
Moer de novo.
Repisar.
Importunar.
Ruminar.
\section{Remoinhada}
\begin{itemize}
\item {fónica:mo-i}
\end{itemize}
\begin{itemize}
\item {Grp. gram.:f.}
\end{itemize}
\begin{itemize}
\item {Utilização:Neol.}
\end{itemize}
Acto de remoinhar.
\section{Remoinhar}
\begin{itemize}
\item {fónica:mo-i}
\end{itemize}
\begin{itemize}
\item {Grp. gram.:v. i.}
\end{itemize}
\begin{itemize}
\item {Proveniência:(De \textunderscore remoínho\textunderscore )}
\end{itemize}
Mover-se circularmente.
Andar á roda, formando círculos ou espiraes.
Dar voltas.
\section{Remoínho}
\begin{itemize}
\item {Grp. gram.:m.}
\end{itemize}
\begin{itemize}
\item {Proveniência:(De \textunderscore re...\textunderscore  + \textunderscore moínho\textunderscore )}
\end{itemize}
Acto ou effeito do remoinhar.
Sorvedoiro, num pego.
Encontro de ondas ou ventos oppostos.
Rajada de vento.
Tufão.
Disposição do cabello, em espiral.
\section{Remoinhoso}
\begin{itemize}
\item {fónica:mo-i}
\end{itemize}
\begin{itemize}
\item {Grp. gram.:adj.}
\end{itemize}
Que faz remoínho.
\section{Remolar}
\begin{itemize}
\item {Grp. gram.:m.}
\end{itemize}
Fabricante de remos:«\textunderscore ...o mestre remolar haverá 42$480 reis de ordenado\textunderscore ». \textunderscore Regimento para a cidade de Gôa\textunderscore .
(Cast. \textunderscore remolar\textunderscore )
\section{Remolcar}
\begin{itemize}
\item {Grp. gram.:v. t.}
\end{itemize}
\begin{itemize}
\item {Utilização:Ant.}
\end{itemize}
O mesmo que \textunderscore rebocar\textunderscore ^2. Cf. Lopo Coutinho, \textunderscore Cêrco de Dio\textunderscore , l. I, c. 3.
\section{Remolgo}
\begin{itemize}
\item {fónica:môl}
\end{itemize}
\begin{itemize}
\item {Grp. gram.:m.  e  adj.}
\end{itemize}
\begin{itemize}
\item {Utilização:Prov.}
\end{itemize}
Indivíduo madraço, preguiçoso.
\section{Remolhão}
\begin{itemize}
\item {Grp. gram.:m.}
\end{itemize}
\begin{itemize}
\item {Proveniência:(De \textunderscore re...\textunderscore  + \textunderscore mólho\textunderscore )}
\end{itemize}
Grande enfiada de minhocas, dobrada e amarrada a um fio, em fórma de borla, para a pesca das enguias. Cf. Rev. \textunderscore Caça\textunderscore , III, 3.
\section{Remolhar}
\begin{itemize}
\item {Grp. gram.:v. t.}
\end{itemize}
\begin{itemize}
\item {Proveniência:(De \textunderscore re...\textunderscore  + \textunderscore molhar\textunderscore )}
\end{itemize}
Molhar de novo; molhar bem, repassar de líquido.
\section{Remôlho}
\begin{itemize}
\item {Grp. gram.:m.}
\end{itemize}
\begin{itemize}
\item {Utilização:Fam.}
\end{itemize}
Acto de remolhar.
Doença, que obriga a estar de cama:«\textunderscore ...lançando as barbas de remôlho...\textunderscore »\textunderscore Anat. Joc.\textunderscore , II, 424.
\section{Remondagem}
\begin{itemize}
\item {Grp. gram.:f.}
\end{itemize}
Acto ou effeito de remondar.
\section{Remondar}
\begin{itemize}
\item {Grp. gram.:v. t.}
\end{itemize}
\begin{itemize}
\item {Proveniência:(Do lat. \textunderscore remundare\textunderscore )}
\end{itemize}
Mondar novamente.
\section{Remonta}
\begin{itemize}
\item {Grp. gram.:f.}
\end{itemize}
\begin{itemize}
\item {Utilização:Pop.}
\end{itemize}
\begin{itemize}
\item {Proveniência:(De \textunderscore remontar\textunderscore )}
\end{itemize}
Acquisição de gado cavallar ou muar para o exército.
Gado cavallar ou muar, para uso dos regimentos.
Pessoal, incumbido de adquirir êsse gado para serviço militar.
Consêrto; reforma.
\section{Remontar}
\begin{itemize}
\item {Grp. gram.:v. t.}
\end{itemize}
\begin{itemize}
\item {Grp. gram.:V. i.  e  p.}
\end{itemize}
\begin{itemize}
\item {Proveniência:(De \textunderscore re...\textunderscore  + \textunderscore montar\textunderscore )}
\end{itemize}
Elevar muito; fazer subir a lugar elevado.
Encimar.
Adornar as extremidades de.
Substituír por outro (o gado de um regimento).
Consertar, remendar: \textunderscore remontar botas\textunderscore .
Elevar-se muito.
Sublimar-se.
Têr procedência remota.
Referir-se a tempos passados: \textunderscore as cruzadas remontam á Idade-Média\textunderscore .
Alludir a coisas ou pessôas remotas.
\section{Remonte}
\begin{itemize}
\item {Grp. gram.:m.}
\end{itemize}
Acto de remontar.
Consêrto na parte anterior do calçado.
Cabedal, com que se faz êsse consêrto.
\section{Remontista}
\begin{itemize}
\item {Grp. gram.:m.}
\end{itemize}
\begin{itemize}
\item {Proveniência:(De \textunderscore remontar\textunderscore )}
\end{itemize}
Aquelle que faz remonta de gado cavallar.
\section{Remoque}
\begin{itemize}
\item {Grp. gram.:m.}
\end{itemize}
Motejo.
Dito reservado, que encerra conceito malicioso ou satýrico.
\section{Remoqueador}
\begin{itemize}
\item {Grp. gram.:m.  e  adj.}
\end{itemize}
O que remoqueia.
\section{Remoquear}
\begin{itemize}
\item {Grp. gram.:v. t.}
\end{itemize}
Ferir com remoques. Cf. Camillo, \textunderscore Filha do Reg.\textunderscore , 150.
\section{Rêmora}
\begin{itemize}
\item {Grp. gram.:f.}
\end{itemize}
\begin{itemize}
\item {Utilização:Ant.}
\end{itemize}
\begin{itemize}
\item {Proveniência:(Lat. \textunderscore remora\textunderscore )}
\end{itemize}
Pequeno peixe, que tem na cabeça um disco oval de bordas espêssas e contrácteis, com o qual se fixa em qualquer corpo sólido submarino.
Adiamento; dilação, obstáculo:«\textunderscore ...vontades, a cuja execução sejão rêmoras as súpplicas...\textunderscore »\textunderscore Anat. Joc.\textunderscore , II, 457.
\section{Remorado}
\begin{itemize}
\item {Grp. gram.:adj.}
\end{itemize}
\begin{itemize}
\item {Proveniência:(Do lat. \textunderscore remoratus\textunderscore )}
\end{itemize}
Retardado.
\section{Remordaz}
\begin{itemize}
\item {Grp. gram.:adj.}
\end{itemize}
\begin{itemize}
\item {Proveniência:(De \textunderscore re...\textunderscore  + \textunderscore mordaz\textunderscore )}
\end{itemize}
Que é muito mordaz.
\section{Remordedor}
\begin{itemize}
\item {Grp. gram.:m.  e  adj.}
\end{itemize}
O que remorde.
\section{Remordente}
\begin{itemize}
\item {Grp. gram.:adj.}
\end{itemize}
Que remorde. Cf. Camillo, \textunderscore Quéda\textunderscore , 157.
\section{Remorder}
\begin{itemize}
\item {Grp. gram.:v. t.}
\end{itemize}
\begin{itemize}
\item {Utilização:Fig.}
\end{itemize}
\begin{itemize}
\item {Grp. gram.:V. i.}
\end{itemize}
\begin{itemize}
\item {Grp. gram.:V. p.}
\end{itemize}
\begin{itemize}
\item {Utilização:Fig.}
\end{itemize}
\begin{itemize}
\item {Proveniência:(Lat. \textunderscore remordere\textunderscore )}
\end{itemize}
Morder de novo.
Morder muitas vezes.
Falar mal de.
Torturar, affligir.
Parafusar sôbre.
Morder muito.
Insistir.
Falar em desabono de alguém.
Morder-se muitas vezes.
Affligir-se; ralar-se.
\section{Remordimento}
\begin{itemize}
\item {Grp. gram.:m.}
\end{itemize}
Acto ou effeito de remorder.
Remorso.
\section{Remoroso}
\begin{itemize}
\item {Grp. gram.:adj.}
\end{itemize}
O mesmo que \textunderscore remorado\textunderscore .
\section{Remorso}
\begin{itemize}
\item {Grp. gram.:m.}
\end{itemize}
\begin{itemize}
\item {Proveniência:(Lat. \textunderscore remorsus\textunderscore )}
\end{itemize}
Desgôsto pungente ou sentimento doloroso, resultante do conhecimento de se haver commetido culpa ou crime.
\section{Remostar}
\begin{itemize}
\item {Grp. gram.:v. t.}
\end{itemize}
\begin{itemize}
\item {Utilização:Prov.}
\end{itemize}
\begin{itemize}
\item {Utilização:trasm.}
\end{itemize}
\begin{itemize}
\item {Proveniência:(De \textunderscore re...\textunderscore  + \textunderscore mosto\textunderscore )}
\end{itemize}
Pôr a ferver (vinho velho) com mosto do novo.
\section{Remotamente}
\begin{itemize}
\item {Grp. gram.:adv.}
\end{itemize}
Em tempo ou sítio remoto.
De maneira indirecta.
\section{Remoto}
\begin{itemize}
\item {Grp. gram.:adj.}
\end{itemize}
\begin{itemize}
\item {Proveniência:(Lat. \textunderscore remotus\textunderscore )}
\end{itemize}
Que está distante; que succedeu há muito tempo.
Muito afastado.
\section{Remover}
\begin{itemize}
\item {Proveniência:(Lat. \textunderscore removere\textunderscore )}
\end{itemize}
\textunderscore v.\textunderscore  t.
Mover novamente.
Transferir.
Tornar distante.
Afastar; livrar-se de; evitar: \textunderscore remover obstáculos\textunderscore .
Obstar a.
Demittir.
Agitar, remexer. Cf. Camillo, \textunderscore Volcoens\textunderscore , 266.
\section{Removimento}
\begin{itemize}
\item {Grp. gram.:m.}
\end{itemize}
Acto ou effeito de remover.
\section{Removível}
\begin{itemize}
\item {Grp. gram.:adj.}
\end{itemize}
Que se póde remover.
\section{Remualho}
\begin{itemize}
\item {Grp. gram.:m.}
\end{itemize}
Doença das abelhas, que se manifesta por uma espécie de farelo. Cf. \textunderscore Gaz. das Aldeias\textunderscore , de 25-11-906.
(Talvez por \textunderscore remoalho\textunderscore , de \textunderscore remoer\textunderscore )
\section{Remudar}
\begin{itemize}
\item {Grp. gram.:v. t.}
\end{itemize}
\begin{itemize}
\item {Proveniência:(De \textunderscore re...\textunderscore  + \textunderscore mudar\textunderscore )}
\end{itemize}
Mudar novamente.
\section{Remudas}
\begin{itemize}
\item {Grp. gram.:f. pl.}
\end{itemize}
\begin{itemize}
\item {Utilização:Prov.}
\end{itemize}
\begin{itemize}
\item {Utilização:trasm.}
\end{itemize}
\begin{itemize}
\item {Proveniência:(De \textunderscore re...\textunderscore  + \textunderscore muda\textunderscore )}
\end{itemize}
\textunderscore Levar\textunderscore  (quaesquer coisas) \textunderscore ás remudas\textunderscore , levá-las, poisando uma num ponto, voltando atrás para levar outra até êsse ponto, e pegando naquella até outro ponto, e assim por deante, até que se levem ambas ao seu destino.
\section{Remugar}
\begin{itemize}
\item {Grp. gram.:v. i.}
\end{itemize}
\begin{itemize}
\item {Utilização:Prov.}
\end{itemize}
\begin{itemize}
\item {Utilização:trasm.}
\end{itemize}
O mesmo que \textunderscore resmungar\textunderscore .
\section{Remugir}
\begin{itemize}
\item {Grp. gram.:v. i.}
\end{itemize}
\begin{itemize}
\item {Proveniência:(Lat. \textunderscore remugire\textunderscore )}
\end{itemize}
Tornar a mugir; mugir repetidas vezes.
Bramir; bramar.
Fazer imprecações.
\section{Remuito}
\begin{itemize}
\item {Grp. gram.:adv.}
\end{itemize}
\begin{itemize}
\item {Proveniência:(De \textunderscore re...\textunderscore  + \textunderscore muito\textunderscore )}
\end{itemize}
No mais alto grau. Cf. G. Vicente, A. Prestes, Chiado, etc.
\section{Remuncar}
\begin{itemize}
\item {Grp. gram.:v. i.}
\end{itemize}
\begin{itemize}
\item {Utilização:Prov.}
\end{itemize}
\begin{itemize}
\item {Utilização:trasm.}
\end{itemize}
O mesmo que \textunderscore resmungar\textunderscore .
\section{Remuneração}
\begin{itemize}
\item {Grp. gram.:f.}
\end{itemize}
\begin{itemize}
\item {Proveniência:(Lat. \textunderscore remuneratio\textunderscore )}
\end{itemize}
Acto ou effeito de remunerar.
Recompensa.
\section{Remunerador}
\begin{itemize}
\item {Grp. gram.:m.  e  adj.}
\end{itemize}
\begin{itemize}
\item {Proveniência:(Do lat. \textunderscore remunerator\textunderscore )}
\end{itemize}
O que remunera; o que recompensa.
\section{Remunerar}
\begin{itemize}
\item {Grp. gram.:v. t.}
\end{itemize}
\begin{itemize}
\item {Proveniência:(Lat. \textunderscore remunerari\textunderscore )}
\end{itemize}
Satisfazer; recompensar; gratificar.
\section{Remunerativo}
\begin{itemize}
\item {Grp. gram.:adj.}
\end{itemize}
Que remunera; próprio para recompensar ou remunerar.
\section{Remuneratório}
\begin{itemize}
\item {Grp. gram.:adj.}
\end{itemize}
Que remunera; próprio para recompensar ou remunerar.
\section{Remunerável}
\begin{itemize}
\item {Grp. gram.:adj.}
\end{itemize}
Que se póde ou se deve remunerar.
\section{Remuneroso}
\begin{itemize}
\item {Grp. gram.:adj.}
\end{itemize}
O mesmo que \textunderscore remuneratório\textunderscore .
\section{Remurmurar}
\begin{itemize}
\item {Grp. gram.:v. i.}
\end{itemize}
\begin{itemize}
\item {Proveniência:(De \textunderscore re...\textunderscore  + \textunderscore murmurar\textunderscore )}
\end{itemize}
Murmurar de novo, murmurar repetidas vezes.
\section{Remurmúrio}
\begin{itemize}
\item {Grp. gram.:m.}
\end{itemize}
\begin{itemize}
\item {Proveniência:(De \textunderscore re...\textunderscore  + \textunderscore murmúrio\textunderscore )}
\end{itemize}
Acto ou effeito de remurmurar.
\section{Remusátia}
\begin{itemize}
\item {Grp. gram.:f.}
\end{itemize}
\begin{itemize}
\item {Proveniência:(De \textunderscore Rémusat\textunderscore , n. p.)}
\end{itemize}
Gênero de plantas aráceas da Índia.
\section{Ren}
\begin{itemize}
\item {Grp. gram.:f.}
\end{itemize}
\begin{itemize}
\item {Utilização:Ant.}
\end{itemize}
\begin{itemize}
\item {Proveniência:(Lat. \textunderscore rem\textunderscore )}
\end{itemize}
O mesmo que \textunderscore rem\textunderscore .
Coisa nenhuma; nada:«\textunderscore mays todo esto non lhis valia ren\textunderscore ». \textunderscore Liv. das Linhagens\textunderscore  do Coll. dos Nobres.
\section{Renal}
\begin{itemize}
\item {Grp. gram.:adj.}
\end{itemize}
\begin{itemize}
\item {Proveniência:(Lat. \textunderscore renalis\textunderscore )}
\end{itemize}
Relativo aos rins.
\section{Renantera}
\begin{itemize}
\item {Grp. gram.:f.}
\end{itemize}
\begin{itemize}
\item {Proveniência:(Do gr. \textunderscore ren\textunderscore  + \textunderscore antheros\textunderscore )}
\end{itemize}
Gênero de orquídeas.
\section{Renanthera}
\begin{itemize}
\item {Grp. gram.:f.}
\end{itemize}
\begin{itemize}
\item {Proveniência:(Do gr. \textunderscore ren\textunderscore  + \textunderscore antheros\textunderscore )}
\end{itemize}
Gênero de orchídeas.
\section{Renão}
\begin{itemize}
\item {Grp. gram.:adv.}
\end{itemize}
\begin{itemize}
\item {Proveniência:(De \textunderscore re...\textunderscore  + \textunderscore não\textunderscore )}
\end{itemize}
O mesmo que \textunderscore não\textunderscore ^1, reforçadamente:«\textunderscore digo-te que renão quero\textunderscore ». G. Vicente, I, 226.
\section{Renascença}
\begin{itemize}
\item {Grp. gram.:f.}
\end{itemize}
Acto ou effeito de renascer.
Renovação.
Vida nova.
Novo impulso, dado ás artes, sciências, etc.
Renovação scientífica, literária e artística, realizada no século XV e XVI, e baseada principalmente nas letras gregas e latinas.
Época, em que se deu essa renovação: \textunderscore Camões viveu na Renascença\textunderscore .
\section{Renascente}
\begin{itemize}
\item {Grp. gram.:adj.}
\end{itemize}
\begin{itemize}
\item {Proveniência:(Lat. \textunderscore renascens\textunderscore )}
\end{itemize}
Que renasce.
\section{Renascer}
\begin{itemize}
\item {Grp. gram.:v. i.}
\end{itemize}
\begin{itemize}
\item {Proveniência:(Lat. \textunderscore renasci\textunderscore )}
\end{itemize}
Nascer de novo.
Rejuvenescer.
Renovar-se.
Resurgir, tornar a apparecer.
Lançar rebentos.
Corrigir-se, emendar-se.
\section{Renascimento}
\begin{itemize}
\item {Grp. gram.:m.}
\end{itemize}
O mesmo que \textunderscore renascença\textunderscore .
\section{Renavegar}
\begin{itemize}
\item {Grp. gram.:v. t.}
\end{itemize}
\begin{itemize}
\item {Proveniência:(De \textunderscore re...\textunderscore  + \textunderscore navegar\textunderscore )}
\end{itemize}
Tornar a navegar; navegar para o ponto donde se partiu.
\section{Rencatrilha}
\begin{itemize}
\item {Grp. gram.:m.}
\end{itemize}
O mesmo que \textunderscore rancatrilha\textunderscore .
\section{Renda}
\begin{itemize}
\item {Grp. gram.:f.}
\end{itemize}
\begin{itemize}
\item {Proveniência:(Do lat. hyp. \textunderscore retina\textunderscore , de \textunderscore rete\textunderscore )}
\end{itemize}
Tecido transparente ou obra de malha, com desenhos variados.
\section{Renda}
\begin{itemize}
\item {Grp. gram.:f.}
\end{itemize}
\begin{itemize}
\item {Proveniência:(De \textunderscore render\textunderscore )}
\end{itemize}
Quantia, que o inquilino de uma casa ou cultivador de uma fazenda paga aos senhores dos mesmos prédios.
Preço de arrendamento de um prédio.
Rendimento.
Conjunto dos rendimentos, que entram num cofre.
Receita, producto.
\section{Rendado}
\begin{itemize}
\item {Grp. gram.:m.}
\end{itemize}
\begin{itemize}
\item {Proveniência:(De \textunderscore rendar\textunderscore ^1)}
\end{itemize}
Objecto, guarnecido de renda.
Conjunto das rendas de uma peça de vestuário.
\section{Rendar}
\begin{itemize}
\item {Grp. gram.:v. t.}
\end{itemize}
Guarnecer de rendas^1.
\section{Rendar}
\begin{itemize}
\item {Grp. gram.:v. t.}
\end{itemize}
\begin{itemize}
\item {Grp. gram.:V. i.}
\end{itemize}
\begin{itemize}
\item {Proveniência:(De \textunderscore renda\textunderscore ^2)}
\end{itemize}
Dar de renda ou tomar de renda; arrendar.
Pagar renda ou pensão.
\section{Rendaria}
\begin{itemize}
\item {Grp. gram.:f.}
\end{itemize}
Arte de fazer rendas^1; indústria das rendas.
\section{Rendedoiro}
\begin{itemize}
\item {Grp. gram.:adj.}
\end{itemize}
Que rende ou produz; que promete renda ou producto: \textunderscore indústria rendedoira\textunderscore .
\section{Rendedouro}
\begin{itemize}
\item {Grp. gram.:adj.}
\end{itemize}
Que rende ou produz; que promete renda ou producto: \textunderscore indústria rendedoura\textunderscore .
\section{Rendeira}
\begin{itemize}
\item {Grp. gram.:f.}
\end{itemize}
Mulhér, que vende ou fabríca rendas.
(Cp. \textunderscore rendeiro\textunderscore ^1)
\section{Rendeira}
\begin{itemize}
\item {Grp. gram.:f.}
\end{itemize}
Mulhér, que toma de arrendamento propriedades rústicas.
Mulhér, que deu de renda alguma propriedade.
Mulhér de rendeiro^2.
(Cp. \textunderscore rendeiro\textunderscore ^2)
\section{Rendeira}
\begin{itemize}
\item {Grp. gram.:f.}
\end{itemize}
\begin{itemize}
\item {Utilização:Bras}
\end{itemize}
Passarinho branco, de cauda, cabeça e asas negras.
\section{Rendeiro}
\begin{itemize}
\item {Grp. gram.:m.}
\end{itemize}
\begin{itemize}
\item {Proveniência:(Do b. lat. \textunderscore rendarius\textunderscore )}
\end{itemize}
Fabricante ou vendedor de rendas^1.
\section{Rendeiro}
\begin{itemize}
\item {Grp. gram.:m.}
\end{itemize}
Aquelle que toma de arrendamento propriedades rústicas.
Aquelle que dá de arrendamento uma propriedade.
Arrematante ou cobrador de rendas^2.
\section{Rendengue}
\begin{itemize}
\item {Grp. gram.:m.}
\end{itemize}
\begin{itemize}
\item {Utilização:Bras. do N}
\end{itemize}
Parte do corpo humano, comprehendida entre a cintura e as virilhas.
\section{Render}
\begin{itemize}
\item {Grp. gram.:v. t.}
\end{itemize}
\begin{itemize}
\item {Grp. gram.:V. i.}
\end{itemize}
\begin{itemize}
\item {Utilização:Prov.}
\end{itemize}
\begin{itemize}
\item {Grp. gram.:V. p.}
\end{itemize}
\begin{itemize}
\item {Proveniência:(Do lat. \textunderscore reddere\textunderscore )}
\end{itemize}
Sujeitar, submeter, fazer ceder.
Dar novamente, restituír.
Satisfazer.
Prestar: \textunderscore render graças\textunderscore .
Produzir (vantagem ou lucro), têr como resultado: \textunderscore a vinha rendeu três contos\textunderscore .
Incitar.
Occupar o lugar de; substituír: \textunderscore aquelle destacamento foi render outro\textunderscore .
Alliciar.
Prostrar com fadiga, enfraquecer.
Commover.
Sêr causa de: \textunderscore render desgostos\textunderscore .
Offertar.
Fender-se.
Dar de si.
Adquirir hérnia.
Quebrar-se.
Inclinar-se.
Dar producto ou vantagem.
Sêr lucrativo.
Soltar involuntariamente ventosídades anaes, ao fazer um esfôrço. (Colhido na Bairrada)
Entregar-se, dar-se por vencido.
Prostrar-se, abater-se; sujeitar-se.
\section{Rendição}
\begin{itemize}
\item {Grp. gram.:f.}
\end{itemize}
\begin{itemize}
\item {Utilização:Ant.}
\end{itemize}
\begin{itemize}
\item {Proveniência:(Do lat. \textunderscore redditio\textunderscore )}
\end{itemize}
Acto ou effeito de render.
Acto ou effeito de remir.
\section{Rendidamente}
\begin{itemize}
\item {Grp. gram.:adv.}
\end{itemize}
De modo rendido.
Com froixidão.
Reverentemente, com todo o acatamento, á maneira de escravo ou de cativo: \textunderscore agradeço-lhe rendidamente tão alto obséquio\textunderscore .
\section{Rendido}
\begin{itemize}
\item {Grp. gram.:adj.}
\end{itemize}
\begin{itemize}
\item {Utilização:Fig.}
\end{itemize}
\begin{itemize}
\item {Proveniência:(De \textunderscore render\textunderscore )}
\end{itemize}
Fendido.
Humilhado; submisso.
Contemplativo.
Que tem hérnia ou ruptura inquinal.
\section{Rendidura}
\begin{itemize}
\item {Grp. gram.:f.}
\end{itemize}
Fenda em qualquer peça de madeira de um navio. Cf. M. de Aguiar, \textunderscore Diccion. de Marinha\textunderscore .
(Cp. \textunderscore rendido\textunderscore )
\section{Rendilha}
\begin{itemize}
\item {Grp. gram.:f.}
\end{itemize}
\begin{itemize}
\item {Proveniência:(De \textunderscore renda\textunderscore ^1)}
\end{itemize}
Renda pequena ou delicada; espiguilha.
\section{Rendilhamento}
\begin{itemize}
\item {Grp. gram.:m.}
\end{itemize}
Acto ou effeito de rendilhar.
\section{Rendilhar}
\begin{itemize}
\item {Grp. gram.:v. t.}
\end{itemize}
\begin{itemize}
\item {Proveniência:(De \textunderscore rendilha\textunderscore )}
\end{itemize}
Adornar com rendilhas.
Adornar em fórma de renda.
Recortar.
Dar ornatos caprichosos a.
\section{Rendilheira}
\begin{itemize}
\item {Grp. gram.:f.}
\end{itemize}
\begin{itemize}
\item {Proveniência:(De \textunderscore rendilha\textunderscore )}
\end{itemize}
Mulhér, que faz ou vende rendas; o mesmo que \textunderscore rendeira\textunderscore ^1. Cf. Ortigão, \textunderscore Praias\textunderscore , 16.
\section{Rendimento}
\begin{itemize}
\item {Grp. gram.:m.}
\end{itemize}
\begin{itemize}
\item {Grp. gram.:Pl.}
\end{itemize}
Acto ou effeito de render.
Acto de se render ou de se entregar.
Prestação.
Offerecimento.
Lucro, producto.
Juro.
Vantagem.
Luxação ou deslocação de osso.
Cumprimentos respeitosos.
\section{Rendoiça}
\begin{itemize}
\item {Grp. gram.:f.}
\end{itemize}
\begin{itemize}
\item {Utilização:Prov.}
\end{itemize}
\begin{itemize}
\item {Utilização:minh.}
\end{itemize}
O mesmo que \textunderscore retoiça\textunderscore .
\section{Rendor}
\begin{itemize}
\item {Grp. gram.:m.}
\end{itemize}
\begin{itemize}
\item {Utilização:Prov.}
\end{itemize}
\begin{itemize}
\item {Utilização:alg.}
\end{itemize}
O mesmo que \textunderscore rendimento\textunderscore ; lucro, proveito.
\section{Rendosamente}
\begin{itemize}
\item {Grp. gram.:adv.}
\end{itemize}
De modo rendoso; lucrativamente; com interesses.
\section{Rendoso}
\begin{itemize}
\item {Grp. gram.:adj.}
\end{itemize}
\begin{itemize}
\item {Proveniência:(De \textunderscore renda\textunderscore ^2)}
\end{itemize}
Que rende; pingue; lucrativo.
\section{Rendouça}
\begin{itemize}
\item {Grp. gram.:f.}
\end{itemize}
\begin{itemize}
\item {Utilização:Prov.}
\end{itemize}
\begin{itemize}
\item {Utilização:minh.}
\end{itemize}
O mesmo que \textunderscore retoiça\textunderscore .
\section{Renegação}
\begin{itemize}
\item {Grp. gram.:f.}
\end{itemize}
Acto de renegar. Cf. Luc. Cordeiro, \textunderscore Senh. Duq.\textunderscore , 80.
\section{Renegada}
\begin{itemize}
\item {Grp. gram.:f.}
\end{itemize}
O mesmo que \textunderscore arrenegada\textunderscore .
\section{Renegado}
\begin{itemize}
\item {Grp. gram.:m.}
\end{itemize}
\begin{itemize}
\item {Utilização:Pop.}
\end{itemize}
Aquelle que deixa a sua religião por outra.
Aquelle que deixa o ser partido, para se filiar noutro.
Aquelle que abandona as suas opiniões antigas.
Malvado.
\section{Renegador}
\begin{itemize}
\item {Grp. gram.:m.  e  adj.}
\end{itemize}
\begin{itemize}
\item {Utilização:Ant.}
\end{itemize}
O que renega.
Homem que pragueja.
\section{Renegamento}
\begin{itemize}
\item {Grp. gram.:m.}
\end{itemize}
O mesmo que \textunderscore renegação\textunderscore .
\section{Rena}
\begin{itemize}
\item {Grp. gram.:f.}
\end{itemize}
\begin{itemize}
\item {Proveniência:(Do fr. \textunderscore renne\textunderscore )}
\end{itemize}
Quadrúpede do norte, do gênero do veado.
\section{Renegar}
\begin{itemize}
\item {Grp. gram.:v. t.  e  i.}
\end{itemize}
\begin{itemize}
\item {Proveniência:(De \textunderscore re...\textunderscore  + \textunderscore negar\textunderscore )}
\end{itemize}
Abandonar a sua religião ou o seu partido por outro.
Detestar.
Rejeitar.
Desprezar; descrer.
\section{Renenhum}
\begin{itemize}
\item {Grp. gram.:adj.}
\end{itemize}
O mesmo que \textunderscore nenhum\textunderscore , reforçadamente. Cf. Castilho, \textunderscore Misanthropo\textunderscore , XII.
\section{Renete}
\begin{itemize}
\item {fónica:nê}
\end{itemize}
\begin{itemize}
\item {Grp. gram.:m.}
\end{itemize}
\begin{itemize}
\item {Proveniência:(Fr. \textunderscore rainette\textunderscore )}
\end{itemize}
Instrumento, próprio para aparar o casco das bêstas.
\section{Renga}
\begin{itemize}
\item {Grp. gram.:f.}
\end{itemize}
\begin{itemize}
\item {Utilização:Pop.}
\end{itemize}
O mesmo que \textunderscore renque\textunderscore .
\section{Rengalho}
\begin{itemize}
\item {Grp. gram.:m.}
\end{itemize}
\begin{itemize}
\item {Proveniência:(De \textunderscore rengo\textunderscore ^1)}
\end{itemize}
Tecido, em que se fazem bordados.
Rêde sem lavor.
\section{Renge}
\begin{itemize}
\item {Grp. gram.:m.}
\end{itemize}
\begin{itemize}
\item {Utilização:Ant.}
\end{itemize}
\begin{itemize}
\item {Proveniência:(Do b. lat. \textunderscore rinca\textunderscore )}
\end{itemize}
Diadema; pente.
\section{Renglão}
\begin{itemize}
\item {Grp. gram.:m.}
\end{itemize}
\begin{itemize}
\item {Utilização:P. us.}
\end{itemize}
Linha ou regra escrita.
(Cast. \textunderscore renglon\textunderscore . Cp. \textunderscore rengra\textunderscore ^1)
\section{Rengo}
\begin{itemize}
\item {Grp. gram.:m.}
\end{itemize}
\begin{itemize}
\item {Proveniência:(Do cast. \textunderscore renque\textunderscore )}
\end{itemize}
Tecido transparente, applicado principalmente em bordados.
\section{Rengo}
\begin{itemize}
\item {Grp. gram.:adj.}
\end{itemize}
\begin{itemize}
\item {Utilização:Bras. do S}
\end{itemize}
Diz-se do cavallo, que manqueja de uma perna.
(Cast. \textunderscore renco\textunderscore )
\section{Rengo}
\begin{itemize}
\item {Grp. gram.:adj.}
\end{itemize}
\begin{itemize}
\item {Utilização:Prov.}
\end{itemize}
\begin{itemize}
\item {Utilização:trasm.}
\end{itemize}
Derreado.
Côxo.
Que não tem sensibilidade em alguma das pernas.
(Cp. \textunderscore derrengar\textunderscore )
\section{Rengo}
\begin{itemize}
\item {Utilização:Prov.}
\end{itemize}
\begin{itemize}
\item {Utilização:trasm.}
\end{itemize}
O mesmo que \textunderscore escalracho\textunderscore .
\section{Rengra}
\begin{itemize}
\item {Grp. gram.:f.}
\end{itemize}
\begin{itemize}
\item {Utilização:Prov.}
\end{itemize}
\begin{itemize}
\item {Utilização:minh.}
\end{itemize}
O mesmo que \textunderscore renque\textunderscore .
(Cast. \textunderscore rengla\textunderscore )
\section{Rengra}
\begin{itemize}
\item {Grp. gram.:f.}
\end{itemize}
\begin{itemize}
\item {Utilização:Prov.}
\end{itemize}
\begin{itemize}
\item {Utilização:minh.}
\end{itemize}
O mesmo que \textunderscore rengo\textunderscore ^4. (Colhido em Mesão-Frio)
\section{Rengue}
\begin{itemize}
\item {Grp. gram.:m.}
\end{itemize}
O mesmo que \textunderscore rengo\textunderscore ^1.
\section{Rengue}
\begin{itemize}
\item {Grp. gram.:m.}
\end{itemize}
\begin{itemize}
\item {Utilização:Ant.}
\end{itemize}
O mesmo que \textunderscore renque\textunderscore .
\section{Renguear}
\begin{itemize}
\item {Grp. gram.:v. i.}
\end{itemize}
\begin{itemize}
\item {Utilização:Bras. do S}
\end{itemize}
\begin{itemize}
\item {Proveniência:(De \textunderscore rengo\textunderscore ^2)}
\end{itemize}
Sêr rengue, (falando-se do cavallo).
\section{Renhideiro}
\begin{itemize}
\item {Grp. gram.:m.}
\end{itemize}
\begin{itemize}
\item {Utilização:Bras. do S}
\end{itemize}
\begin{itemize}
\item {Proveniência:(De \textunderscore renhir\textunderscore )}
\end{itemize}
Espécie de circo, para briga de gallos.
\section{Renhido}
\begin{itemize}
\item {Grp. gram.:adj.}
\end{itemize}
\begin{itemize}
\item {Utilização:Fig.}
\end{itemize}
Sangrento.
\section{Renhimento}
\begin{itemize}
\item {Grp. gram.:m.}
\end{itemize}
Acto ou effeito de renhir.
\section{Renhir}
\begin{itemize}
\item {Grp. gram.:v. t.}
\end{itemize}
\begin{itemize}
\item {Grp. gram.:V. i.}
\end{itemize}
\begin{itemize}
\item {Proveniência:(Do lat. \textunderscore ringi\textunderscore )}
\end{itemize}
Pleitear, disputar.
Combater por.
Combater contra.
Travar.
Combater denodadamente, furiosamente.
Porfiar; altercar.
\section{Renhões}
\begin{itemize}
\item {Grp. gram.:m. pl.}
\end{itemize}
\begin{itemize}
\item {Utilização:Prov.}
\end{itemize}
\begin{itemize}
\item {Utilização:trasm.}
\end{itemize}
Testículos.
(Cp. \textunderscore rinhão\textunderscore )
\section{Renhuçar}
\textunderscore v. t.\textunderscore  e \textunderscore i.\textunderscore  (e der.)
(Fórma ant. de \textunderscore renunciar\textunderscore , etc. Cf. S. R. Viterbo, \textunderscore Elucidário\textunderscore )
\section{Reniforme}
\begin{itemize}
\item {Grp. gram.:adj.}
\end{itemize}
\begin{itemize}
\item {Proveniência:(Do lat. \textunderscore ren\textunderscore  + \textunderscore forma\textunderscore )}
\end{itemize}
Que tem fórma de rim.
\section{Renitência}
\begin{itemize}
\item {Grp. gram.:f.}
\end{itemize}
\begin{itemize}
\item {Proveniência:(Lat. \textunderscore renitentia\textunderscore )}
\end{itemize}
Qualidade do que é renitente.
Pertinácia; teimosia.
\section{Renitente}
\begin{itemize}
\item {Grp. gram.:adj.}
\end{itemize}
\begin{itemize}
\item {Proveniência:(Lat. \textunderscore renitens\textunderscore )}
\end{itemize}
Que renite.
Contumaz.
Teimoso.
\section{Renitir}
\begin{itemize}
\item {Grp. gram.:v. i.}
\end{itemize}
\begin{itemize}
\item {Proveniência:(Lat. \textunderscore reniti\textunderscore )}
\end{itemize}
Resistir; sêr contumaz; obstinar-se.
\section{Renna}
\begin{itemize}
\item {Grp. gram.:f.}
\end{itemize}
\begin{itemize}
\item {Proveniência:(Do fr. \textunderscore renne\textunderscore )}
\end{itemize}
Quadrúpede do norte, do gênero do veado.
\section{Renome}
\begin{itemize}
\item {Grp. gram.:m.}
\end{itemize}
\begin{itemize}
\item {Proveniência:(De \textunderscore re...\textunderscore  + \textunderscore nome\textunderscore )}
\end{itemize}
Nomeada; fama; crédito, bôa reputação.
\section{Renomear}
\begin{itemize}
\item {Grp. gram.:v. t.}
\end{itemize}
Dar renome a.
Celebrar.
Nomear de novo ou muitas vezes.
\section{Renova}
\begin{itemize}
\item {Grp. gram.:f.}
\end{itemize}
O mesmo que \textunderscore renôvo\textunderscore .
\section{Renovação}
\begin{itemize}
\item {Grp. gram.:f.}
\end{itemize}
\begin{itemize}
\item {Proveniência:(Lat. \textunderscore renovatio\textunderscore )}
\end{itemize}
Acto ou effeito de renovar.
\section{Renovador}
\begin{itemize}
\item {Grp. gram.:m.  e  adj.}
\end{itemize}
\begin{itemize}
\item {Proveniência:(Lat. \textunderscore renovator\textunderscore )}
\end{itemize}
O que renova.
\section{Renovamento}
\begin{itemize}
\item {Grp. gram.:m.}
\end{itemize}
O mesmo que \textunderscore renovação\textunderscore .
\section{Renovar}
\begin{itemize}
\item {Grp. gram.:v. t.}
\end{itemize}
\begin{itemize}
\item {Grp. gram.:V. i.}
\end{itemize}
\begin{itemize}
\item {Proveniência:(Lat. \textunderscore renovare\textunderscore )}
\end{itemize}
Tornar novo.
Recomeçar; repetir: \textunderscore renovar pedidos\textunderscore .
Consertar.
Melhorar.
Dar o aspecto de novo a.
Substituír por coisa melhor.
Trazer de novo á lembrança.
Excitar novamente: \textunderscore renovar soffrimentos\textunderscore .
Deitar rebentos.
Tornar a apparecer.
Succeder-se.
\section{Renovável}
\begin{itemize}
\item {Grp. gram.:adj.}
\end{itemize}
Que se póde renovar.
\section{Renôvo}
\begin{itemize}
\item {Grp. gram.:m.}
\end{itemize}
\begin{itemize}
\item {Utilização:Fig.}
\end{itemize}
\begin{itemize}
\item {Utilização:Prov.}
\end{itemize}
\begin{itemize}
\item {Utilização:trasm.}
\end{itemize}
\begin{itemize}
\item {Grp. gram.:Pl.}
\end{itemize}
\begin{itemize}
\item {Proveniência:(De \textunderscore renovar\textunderscore )}
\end{itemize}
Rebento; vergôntea.
Descendência.
O mesmo que \textunderscore horta\textunderscore .
Productos agrícolas.
\section{Renque}
\begin{itemize}
\item {Grp. gram.:m.  ou  f.}
\end{itemize}
\begin{itemize}
\item {Proveniência:(Do ant. alt. al. \textunderscore hring\textunderscore )}
\end{itemize}
Alinhamento, fileira.
Disposição de coisas ou pessôas na mesma linha: \textunderscore um renque de árvores\textunderscore .
\section{Rênquea}
\begin{itemize}
\item {Grp. gram.:f.}
\end{itemize}
\begin{itemize}
\item {Utilização:Ant.}
\end{itemize}
O mesmo que \textunderscore renque\textunderscore , fileira, linha. Cf. \textunderscore Rot. do Mar Verm.\textunderscore , 136.
\section{Rentão}
\begin{itemize}
\item {Grp. gram.:adj.}
\end{itemize}
\begin{itemize}
\item {Utilização:Prov.}
\end{itemize}
\begin{itemize}
\item {Proveniência:(De \textunderscore rente\textunderscore ^2)}
\end{itemize}
Frequentador.
Que não falta nunca em certos lugares ou a certos actos.
\section{Rentar}
\begin{itemize}
\item {Grp. gram.:v. i.}
\end{itemize}
Passar rente.
Dirigir provocações.
Alardear valentia.
Fazer namôro, dirigir galanteios:«\textunderscore observou que os estudantes rentavam á cachopa.\textunderscore »Camillo, \textunderscore Brasileira\textunderscore , 25.
\section{Rentar-se}
\begin{itemize}
\item {Grp. gram.:v. p.}
\end{itemize}
\begin{itemize}
\item {Utilização:Prov.}
\end{itemize}
\begin{itemize}
\item {Utilização:Chul.}
\end{itemize}
Peidar. (Colhido na Bairrada)
\section{Rente}
\begin{itemize}
\item {Grp. gram.:adj.}
\end{itemize}
\begin{itemize}
\item {Proveniência:(Do lat. \textunderscore haerens\textunderscore )}
\end{itemize}
Próximo, cérceo.
\textunderscore Adv.\textunderscore , que póde sêr seguido da prep. \textunderscore de\textunderscore  ou \textunderscore com\textunderscore .
Pela raiz, pelo pé: \textunderscore cortar rente uma planta\textunderscore .
Proximamente; cerce: \textunderscore passou rente da parede; passou rente com o jardim\textunderscore . Cf. Camillo, \textunderscore Volcoens\textunderscore , 88.
\section{Rente}
\begin{itemize}
\item {Grp. gram.:adj.}
\end{itemize}
\begin{itemize}
\item {Utilização:Fam.}
\end{itemize}
Prompto; assíduo; que não falta a certos actos: \textunderscore fulano é sempre rente na festa.\textunderscore 
\section{Rente}
\begin{itemize}
\item {Grp. gram.:m.}
\end{itemize}
\begin{itemize}
\item {Utilização:Pop.}
\end{itemize}
Cilada em meio do um tumulto.
Divertimento, que dá ensejo a traição:«\textunderscore hei de armar-lhe um rente, em que êlle há de cair\textunderscore ».
\section{Renteador}
\begin{itemize}
\item {Grp. gram.:m.}
\end{itemize}
Aquelle que renteia, que namora.
\section{Rentear}
\begin{itemize}
\item {Grp. gram.:v. t.}
\end{itemize}
\begin{itemize}
\item {Grp. gram.:V. i.}
\end{itemize}
\begin{itemize}
\item {Proveniência:(De \textunderscore rente\textunderscore ,^1)}
\end{itemize}
Cortar cerce.
Namorar, galantear.
\section{Rentês}
\begin{itemize}
\item {Grp. gram.:adj.}
\end{itemize}
\begin{itemize}
\item {Utilização:Prov.}
\end{itemize}
\begin{itemize}
\item {Utilização:trasm.}
\end{itemize}
O mesmo que \textunderscore rasteiro\textunderscore : \textunderscore feijão rentês\textunderscore .
\section{Rentura}
\begin{itemize}
\item {Grp. gram.:f.}
\end{itemize}
\begin{itemize}
\item {Utilização:P. us.}
\end{itemize}
\begin{itemize}
\item {Proveniência:(De \textunderscore rente\textunderscore )}
\end{itemize}
Pontaria certeira. Cf. Filinto, \textunderscore D. Man.\textunderscore , III, 310.
\section{Renuído}
\begin{itemize}
\item {Grp. gram.:m.}
\end{itemize}
\begin{itemize}
\item {Proveniência:(De \textunderscore renuír\textunderscore )}
\end{itemize}
Gesto negativo com a cabeça.
\section{Renuír}
\begin{itemize}
\item {Grp. gram.:v. t.}
\end{itemize}
\begin{itemize}
\item {Proveniência:(Lat. \textunderscore renuere\textunderscore )}
\end{itemize}
Renunciar, rejeitar.
\section{Renúncia}
\begin{itemize}
\item {Grp. gram.:f.}
\end{itemize}
Acto ou effeito de renunciar.
\section{Renunciação}
\begin{itemize}
\item {Grp. gram.:f.}
\end{itemize}
O mesmo que \textunderscore renúncia\textunderscore .
\section{Renunciador}
\begin{itemize}
\item {Grp. gram.:m.  e  adj.}
\end{itemize}
O mesmo que \textunderscore renunciante\textunderscore .
\section{Renunciamento}
\begin{itemize}
\item {Grp. gram.:m.}
\end{itemize}
O mesmo que \textunderscore renúncia\textunderscore . Cf. Eça, \textunderscore P. Amaro\textunderscore , 129.
\section{Renunciante}
\begin{itemize}
\item {Grp. gram.:m.  e  adj.}
\end{itemize}
\begin{itemize}
\item {Proveniência:(Lat. \textunderscore renuncians\textunderscore )}
\end{itemize}
O que renuncía.
\section{Renunciar}
\begin{itemize}
\item {Grp. gram.:v. t.}
\end{itemize}
\begin{itemize}
\item {Grp. gram.:V. i.}
\end{itemize}
\begin{itemize}
\item {Proveniência:(Lat. \textunderscore renunciare\textunderscore )}
\end{itemize}
Rejeitar, recusar.
Não acceitar, desistir de.
Deixar a posse de.
Abjurar.
Fazer recusa, desistência, abjuração.
Ao jôgo, deitar carta differente daquella que se devia jogar.
\section{Renunciatório}
\begin{itemize}
\item {Grp. gram.:m.}
\end{itemize}
\begin{itemize}
\item {Utilização:Jur.}
\end{itemize}
\begin{itemize}
\item {Proveniência:(De \textunderscore renunciar\textunderscore )}
\end{itemize}
Aquelle que adquire a posse renunciada por outrem.
\section{Renunciável}
\begin{itemize}
\item {Grp. gram.:adj.}
\end{itemize}
Que se póde renunciar.
\section{Renutação}
\begin{itemize}
\item {Grp. gram.:f.}
\end{itemize}
\begin{itemize}
\item {Proveniência:(Do lat. \textunderscore renutare\textunderscore )}
\end{itemize}
O mesmo que \textunderscore renuído\textunderscore .
\section{Renutrir}
\begin{itemize}
\item {Grp. gram.:v. t.}
\end{itemize}
\begin{itemize}
\item {Grp. gram.:V. i.}
\end{itemize}
\begin{itemize}
\item {Proveniência:(De \textunderscore re...\textunderscore  + \textunderscore nutrir\textunderscore )}
\end{itemize}
Nutrir novamente.
Tomar nova nutrição.
\section{Renzilha}
\begin{itemize}
\item {Grp. gram.:f.}
\end{itemize}
\begin{itemize}
\item {Utilização:Pop.}
\end{itemize}
\begin{itemize}
\item {Proveniência:(Do cast. \textunderscore rencilla\textunderscore )}
\end{itemize}
O mesmo que \textunderscore rezinga\textunderscore .
\section{Reoccupação}
\begin{itemize}
\item {Grp. gram.:m.}
\end{itemize}
Acto ou effeito de reoccupar.
\section{Reoccupar}
\begin{itemize}
\item {Grp. gram.:v. t.}
\end{itemize}
\begin{itemize}
\item {Proveniência:(De \textunderscore re...\textunderscore  + \textunderscore occupar\textunderscore )}
\end{itemize}
Occupar novamente; retomar; reconquistar.
\section{Reocupação}
\begin{itemize}
\item {Grp. gram.:m.}
\end{itemize}
Acto ou efeito de reocupar.
\section{Reocupar}
\begin{itemize}
\item {Grp. gram.:v. t.}
\end{itemize}
\begin{itemize}
\item {Proveniência:(De \textunderscore re...\textunderscore  + \textunderscore ocupar\textunderscore )}
\end{itemize}
Ocupar novamente; retomar; reconquistar.
\section{Reordenação}
\begin{itemize}
\item {Grp. gram.:f.}
\end{itemize}
Acto ou effeito de reordenar.
\section{Reordenar}
\begin{itemize}
\item {Grp. gram.:v. t.}
\end{itemize}
\begin{itemize}
\item {Proveniência:(De \textunderscore re...\textunderscore  + \textunderscore ordenar\textunderscore )}
\end{itemize}
Tornar a ordenar.
\section{Reorganização}
\begin{itemize}
\item {Grp. gram.:f.}
\end{itemize}
Acto ou effeito de reorganizar.
\section{Reorganizador}
\begin{itemize}
\item {Grp. gram.:m.  e  adj.}
\end{itemize}
O que reorganiza.
\section{Reorganizar}
\begin{itemize}
\item {Grp. gram.:v. t.}
\end{itemize}
\begin{itemize}
\item {Proveniência:(De \textunderscore re...\textunderscore  + \textunderscore organizar\textunderscore )}
\end{itemize}
Tornar a organizar; reconstituír; melhorar; reformar.
\section{Reoxidação}
\begin{itemize}
\item {fónica:csi}
\end{itemize}
\begin{itemize}
\item {Grp. gram.:f.}
\end{itemize}
Acto de reoxidar.
\section{Reoxidar}
\begin{itemize}
\item {fónica:csi}
\end{itemize}
\begin{itemize}
\item {Grp. gram.:v.}
\end{itemize}
\begin{itemize}
\item {Utilização:t. Chím.}
\end{itemize}
\begin{itemize}
\item {Proveniência:(De \textunderscore re...\textunderscore  + \textunderscore oxidar\textunderscore )}
\end{itemize}
Reduzir novamente ao estado do óxido.
\section{Reoxigenar-se}
\begin{itemize}
\item {fónica:csi}
\end{itemize}
\begin{itemize}
\item {Grp. gram.:v. p.}
\end{itemize}
\begin{itemize}
\item {Proveniência:(De \textunderscore re...\textunderscore  + \textunderscore oxigênio\textunderscore )}
\end{itemize}
Oxigenar-se novamente (o sangue que volta aos pulmões).
\section{Reoxydação}
\begin{itemize}
\item {Grp. gram.:f.}
\end{itemize}
Acto de reoxydar.
\section{Reoxydar}
\begin{itemize}
\item {Grp. gram.:v.}
\end{itemize}
\begin{itemize}
\item {Utilização:t. Chím.}
\end{itemize}
\begin{itemize}
\item {Proveniência:(De \textunderscore re...\textunderscore  + \textunderscore oxydar\textunderscore )}
\end{itemize}
Reduzir novamente ao estado do óxydo.
\section{Reoxygenar-se}
\begin{itemize}
\item {Grp. gram.:v. p.}
\end{itemize}
\begin{itemize}
\item {Proveniência:(De \textunderscore re...\textunderscore  + \textunderscore oxygênio\textunderscore )}
\end{itemize}
Oxygenar-se novamente (o sangue que volta aos pulmões).
\section{Repa}
\begin{itemize}
\item {fónica:rê}
\end{itemize}
\begin{itemize}
\item {Grp. gram.:f.}
\end{itemize}
\begin{itemize}
\item {Utilização:Pop.}
\end{itemize}
\begin{itemize}
\item {Proveniência:(Do ingl. \textunderscore rip\textunderscore )}
\end{itemize}
Cabello raro; farripas.
\section{Repagar}
\begin{itemize}
\item {Grp. gram.:v. t.}
\end{itemize}
\begin{itemize}
\item {Utilização:Pop.}
\end{itemize}
\begin{itemize}
\item {Proveniência:(De \textunderscore re...\textunderscore  + \textunderscore pagar\textunderscore )}
\end{itemize}
Pagar bem; pagar de novo.
\section{Repago}
\begin{itemize}
\item {Grp. gram.:adj.}
\end{itemize}
Que se repagou; pago novamente.
\section{Repairar}
\textunderscore v. t.\textunderscore  (e der.) \textunderscore Ant.\textunderscore 
O mesmo que \textunderscore reparar\textunderscore .
Renovar, consertar. Cf. \textunderscore Ethiópia Or.\textunderscore , l. III, c. 8.
\section{Repairo}
\begin{itemize}
\item {Grp. gram.:m.}
\end{itemize}
O mesmo que \textunderscore reparo\textunderscore . Cf. Garrett, \textunderscore Catão\textunderscore , II, 1.
Carrêta de artilharia.
\section{Repandirostro}
\begin{itemize}
\item {fónica:rós}
\end{itemize}
\begin{itemize}
\item {Grp. gram.:adj.}
\end{itemize}
\begin{itemize}
\item {Utilização:Zool.}
\end{itemize}
\begin{itemize}
\item {Proveniência:(De \textunderscore re...\textunderscore  + \textunderscore pando\textunderscore  + \textunderscore róstro\textunderscore )}
\end{itemize}
Que tem o bico muito espalmado.
\section{Repandirrostro}
\begin{itemize}
\item {Grp. gram.:adj.}
\end{itemize}
\begin{itemize}
\item {Utilização:Zool.}
\end{itemize}
\begin{itemize}
\item {Proveniência:(De \textunderscore re...\textunderscore  + \textunderscore pando\textunderscore  + \textunderscore róstro\textunderscore )}
\end{itemize}
Que tem o bico muito espalmado.
\section{Repanhar}
\begin{itemize}
\item {Grp. gram.:v. t.}
\end{itemize}
O mesmo que \textunderscore arrepanhar\textunderscore .
\section{Reparação}
\begin{itemize}
\item {Grp. gram.:f.}
\end{itemize}
\begin{itemize}
\item {Proveniência:(Lat. \textunderscore reparatio\textunderscore )}
\end{itemize}
Acto ou effeito de reparar.
Conserto, restauração.
Refórma.
Indemnização.
Satisfação, que se dá a alguém, por offensas ou injúrias.
Retractação ou acto de retirar ou invalidar com declarações palavras ou procedimento injurioso.
\section{Reparadeira}
\begin{itemize}
\item {Grp. gram.:f.  e  adj.}
\end{itemize}
Mulhér, que repara em tudo; mulhér curiosa.
\section{Reparador}
\begin{itemize}
\item {Grp. gram.:m.  e  adj.}
\end{itemize}
\begin{itemize}
\item {Proveniência:(Lat. \textunderscore reparator\textunderscore )}
\end{itemize}
O que repara.
O que melhora, ou fortifica, ou restabelece.
\section{Reparante}
\begin{itemize}
\item {Grp. gram.:adj.}
\end{itemize}
\begin{itemize}
\item {Utilização:P. us.}
\end{itemize}
Que repara.
Que faz observações ou objecções:«\textunderscore ...e me terá final, severo e reparante aristarcho.\textunderscore »Filinto, XVIII, 60.
\section{Reparar}
\begin{itemize}
\item {Grp. gram.:v. t.}
\end{itemize}
\begin{itemize}
\item {Utilização:Ant.}
\end{itemize}
\begin{itemize}
\item {Grp. gram.:V. i.}
\end{itemize}
\begin{itemize}
\item {Proveniência:(Lat. \textunderscore reparare\textunderscore )}
\end{itemize}
Preparar de novo, renovar.
Consertar.
Emendar.
Retocar.
Melhorar.
Restabelecer.
Indemnizar: \textunderscore reparar prejuízos\textunderscore .
Compensar.
Dar satisfação de.
Reforçar.
Levar soccorro a.
Observar; dar attenção a.
Acautelar-se.
\section{Reparatório}
\begin{itemize}
\item {Grp. gram.:adj.}
\end{itemize}
\begin{itemize}
\item {Proveniência:(De \textunderscore reparar\textunderscore )}
\end{itemize}
Que envolve reparação, indemnização ou retractação.
\section{Reparável}
\begin{itemize}
\item {Grp. gram.:adj.}
\end{itemize}
\begin{itemize}
\item {Proveniência:(Do lat. \textunderscore reparabilis\textunderscore )}
\end{itemize}
Que se póde reparar; remediável.
\section{Reparecer}
\textunderscore v. i.\textunderscore  (e der.)(V.reapparecer)
\section{Reparo}
\begin{itemize}
\item {Grp. gram.:m.}
\end{itemize}
Acto ou effeito de reparar.
Resguardo.
Defesa; trincheira.
Carrêta de artilharia.
\section{Repartição}
\begin{itemize}
\item {Grp. gram.:f.}
\end{itemize}
Acto ou effeito de repartir.
Divisão.
Partilha.
Secção.
Secção de Secretaría ou de uma Direcção Geral de Secretaria de Estado.
Secretaría; escritório.
\section{Repartidamente}
\begin{itemize}
\item {Grp. gram.:adv.}
\end{itemize}
\begin{itemize}
\item {Proveniência:(De \textunderscore repartido\textunderscore )}
\end{itemize}
Em partes, por partes.
\section{Repartideira}
\begin{itemize}
\item {Grp. gram.:f.}
\end{itemize}
\begin{itemize}
\item {Proveniência:(De \textunderscore repartir\textunderscore )}
\end{itemize}
Mulhér, que reparte.
Pequena vasilha de cobre, com que se reparte o mel apurado nas fórmas dos engenhos de açúcar.
\section{Repartido}
\begin{itemize}
\item {Grp. gram.:adj.}
\end{itemize}
Que se repartiu; distribuído: \textunderscore esmolas repartidas\textunderscore .
\section{Repartidor}
\begin{itemize}
\item {Grp. gram.:adj.}
\end{itemize}
\begin{itemize}
\item {Grp. gram.:M.}
\end{itemize}
Que reparte.
Aquelle que reparte.
O mesmo que \textunderscore divisor\textunderscore .
O mesmo que \textunderscore repartideira\textunderscore .
\section{Repartimento}
\begin{itemize}
\item {Grp. gram.:m.}
\end{itemize}
\begin{itemize}
\item {Proveniência:(De \textunderscore repartir\textunderscore )}
\end{itemize}
Repartição.
Lugar, separado de outros.
Compartimento, quarto.
\section{Repartir}
\begin{itemize}
\item {Grp. gram.:v. t.}
\end{itemize}
\begin{itemize}
\item {Proveniência:(De \textunderscore re...\textunderscore  + \textunderscore partir\textunderscore )}
\end{itemize}
Separar em partes.
Dividir por grupos.
Distribuír.
Dispor em vários sítios ou por differentes vezes.
Estremar.
\section{Repartitivo}
\begin{itemize}
\item {Grp. gram.:adj.}
\end{itemize}
Que serve para repartir.
\section{Repartível}
\begin{itemize}
\item {Grp. gram.:adj.}
\end{itemize}
Que se póde repartir.
\section{Repassada}
\begin{itemize}
\item {Grp. gram.:f.}
\end{itemize}
\begin{itemize}
\item {Utilização:Bras. do S}
\end{itemize}
O mesmo que \textunderscore repasse\textunderscore .
\section{Repassado}
\begin{itemize}
\item {Grp. gram.:adj.}
\end{itemize}
\begin{itemize}
\item {Proveniência:(De \textunderscore repassar\textunderscore )}
\end{itemize}
Que se repassou; embebido.
Que tem fórma de laço ou trança.
\section{Repassage}
\begin{itemize}
\item {Grp. gram.:f.}
\end{itemize}
Planta, da fam. das compostas, (\textunderscore picris echioides\textunderscore , Lin.). Cf. P. Coutinho, \textunderscore Flora\textunderscore , 668.
\section{Repassagem}
\begin{itemize}
\item {Grp. gram.:f.}
\end{itemize}
\begin{itemize}
\item {Utilização:Bras. do S}
\end{itemize}
O mesmo que \textunderscore repasse\textunderscore .
\section{Repassar}
\begin{itemize}
\item {Grp. gram.:v. t.}
\end{itemize}
\begin{itemize}
\item {Grp. gram.:V. i.}
\end{itemize}
\begin{itemize}
\item {Proveniência:(De \textunderscore re...\textunderscore  + \textunderscore passar\textunderscore )}
\end{itemize}
Passar de novo.
Embeber.
Impregnar.
Penetrar: \textunderscore o frio repassou-o\textunderscore .
Tornar a examinar.
Verter, ressumbrar.
Embeber-se, ensopar-se.
\section{Repasse}
\begin{itemize}
\item {Grp. gram.:m.}
\end{itemize}
\begin{itemize}
\item {Utilização:Bras. do S}
\end{itemize}
\begin{itemize}
\item {Proveniência:(De \textunderscore repassar\textunderscore )}
\end{itemize}
Cada uma das vezes que se monta um cavallo para o domar.
\section{Repasso}
\begin{itemize}
\item {Grp. gram.:m.}
\end{itemize}
O mesmo que \textunderscore repasse\textunderscore .
\section{Repastar}
\begin{itemize}
\item {Grp. gram.:v. t.}
\end{itemize}
\begin{itemize}
\item {Grp. gram.:V. i.}
\end{itemize}
\begin{itemize}
\item {Utilização:Fig.}
\end{itemize}
\begin{itemize}
\item {Proveniência:(De \textunderscore re...\textunderscore  + \textunderscore pastar\textunderscore )}
\end{itemize}
Apascentar de novo.
Conduzir á pastagem.
Alimentar; banquetear.
Comer abundantemente; banquetear-se.
Tomar como assumpto muito agradável; deliciar-se, comprazer-se.
\section{Repasto}
\begin{itemize}
\item {Grp. gram.:m.}
\end{itemize}
\begin{itemize}
\item {Proveniência:(De \textunderscore re...\textunderscore  + \textunderscore pasto\textunderscore )}
\end{itemize}
Grande quantidade de pasto.
Banquete; refeição.
\section{Repatanar-se}
\begin{itemize}
\item {Grp. gram.:v. p.}
\end{itemize}
O mesmo ou melhor que \textunderscore repetenar-se\textunderscore . Cf. Macedo, \textunderscore Burros\textunderscore , 370.
\section{Repatriação}
\begin{itemize}
\item {Grp. gram.:f.}
\end{itemize}
Acto de repatriar.
\section{Repatriamento}
\begin{itemize}
\item {Grp. gram.:m.}
\end{itemize}
O mesmo que \textunderscore repatriação\textunderscore .
\section{Repatriar}
\begin{itemize}
\item {Grp. gram.:v. t.}
\end{itemize}
\begin{itemize}
\item {Grp. gram.:V. p.}
\end{itemize}
Fazer tornar á pátria.
Regressar á patria.
(B. lat. \textunderscore repatriare\textunderscore )
\section{Repeçar}
\begin{itemize}
\item {Grp. gram.:v. t.}
\end{itemize}
\begin{itemize}
\item {Utilização:Ant.}
\end{itemize}
\begin{itemize}
\item {Proveniência:(De \textunderscore re...\textunderscore  + \textunderscore peça\textunderscore )}
\end{itemize}
Ajuntar, cerzir, remendar.
\section{Repechar}
\begin{itemize}
\item {Grp. gram.:v. t.}
\end{itemize}
\begin{itemize}
\item {Utilização:Bras. do S}
\end{itemize}
\begin{itemize}
\item {Proveniência:(De \textunderscore repecho\textunderscore )}
\end{itemize}
Subir a cavallo, trepar (uma ladeira). Cf. H. Murat, \textunderscore Ondas\textunderscore , III, 331.
\section{Repecho}
\begin{itemize}
\item {fónica:pê}
\end{itemize}
\begin{itemize}
\item {Grp. gram.:m.}
\end{itemize}
\begin{itemize}
\item {Utilização:Bras. do S}
\end{itemize}
\begin{itemize}
\item {Proveniência:(T. cast.)}
\end{itemize}
Ladeira.
\section{Repedir}
\begin{itemize}
\item {Grp. gram.:v. t.}
\end{itemize}
Pedir novamente, pedir com insistência. Cf. Castilho, \textunderscore Méd. á Fôrça\textunderscore , 133.
\section{Repelado}
\begin{itemize}
\item {Grp. gram.:adj.}
\end{itemize}
Que têm o pêlo arrepiado.
Sanhudo, (falando-se do gato):«\textunderscore ...outros, com que a fortuna andou ao gato repelado...\textunderscore »\textunderscore Eufrosina\textunderscore , 281.
\section{Repelão}
\begin{itemize}
\item {Grp. gram.:m.}
\end{itemize}
\begin{itemize}
\item {Proveniência:(De \textunderscore repelar\textunderscore , dizem os diccion.; creio porém que se relaciona com \textunderscore repellir\textunderscore , podendo portanto escrever-se \textunderscore repellão\textunderscore )}
\end{itemize}
Encontrão; encontro violento.
Empuxão; ataque.
\section{Repelar}
\begin{itemize}
\item {Grp. gram.:v. t.}
\end{itemize}
(V.arrepelar)
\section{Repelência}
\begin{itemize}
\item {Grp. gram.:f.}
\end{itemize}
Estado ou qualidade de repelente. Cf. Júl. Lour. Pinto, \textunderscore Senh. Deput.\textunderscore , XV; Arn. Gama, \textunderscore Segr. do Abb.\textunderscore , 53.
\section{Repelente}
\begin{itemize}
\item {Grp. gram.:adj.}
\end{itemize}
\begin{itemize}
\item {Proveniência:(Lat. \textunderscore repellens\textunderscore )}
\end{itemize}
Que repele; repgunante; nojento.
\section{Repelido}
\begin{itemize}
\item {Grp. gram.:m.}
\end{itemize}
\begin{itemize}
\item {Proveniência:(De \textunderscore repelir\textunderscore )}
\end{itemize}
Repelão; trato rude.
\section{Repelir}
\begin{itemize}
\item {Grp. gram.:v. t.}
\end{itemize}
\begin{itemize}
\item {Proveniência:(Lat. \textunderscore repellere\textunderscore )}
\end{itemize}
Impelir para trás ou para fóra.
Rebater.
Expulsar.
Não deixar entrar ou aproximar: \textunderscore a guarnição da praça repeliu os assaltantes\textunderscore .
Renunciar, recusar, evitar.
Não admitir.
Tratar, ocupar-se de:«\textunderscore o modo, com que este Reino entrou no redil da Igreja Católica, requere ser repelido de mais longe...\textunderscore »Filinto, \textunderscore D. Man.\textunderscore , I, 228.
\section{Repellência}
\begin{itemize}
\item {Grp. gram.:f.}
\end{itemize}
Estado ou qualidade de repellente. Cf. Júl. Lour. Pinto, \textunderscore Senh. Deput.\textunderscore , XV; Arn. Gama, \textunderscore Segr. do Abb.\textunderscore , 53.
\section{Repellente}
\begin{itemize}
\item {Grp. gram.:adj.}
\end{itemize}
\begin{itemize}
\item {Proveniência:(Lat. \textunderscore repellens\textunderscore )}
\end{itemize}
Que repelle; repgunante; nojento.
\section{Repellido}
\begin{itemize}
\item {Grp. gram.:m.}
\end{itemize}
\begin{itemize}
\item {Proveniência:(De \textunderscore repellir\textunderscore )}
\end{itemize}
Repelão; trato rude.
\section{Repellir}
\begin{itemize}
\item {Grp. gram.:v. t.}
\end{itemize}
\begin{itemize}
\item {Proveniência:(Lat. \textunderscore repellere\textunderscore )}
\end{itemize}
Impellir para trás ou para fóra.
Rebater.
Expulsar.
Não deixar entrar ou aproximar: \textunderscore a guarnição da praça repelliu os assaltantes\textunderscore .
Renunciar, recusar, evitar.
Não admittir.
Tratar, occupar-se de:«\textunderscore o modo, com que este Reino entrou no redil da Igreja Cathólica, requere ser repellido de mais longe...\textunderscore »Filinto, \textunderscore D. Man.\textunderscore , I, 228.
\section{Repello}
\begin{itemize}
\item {Grp. gram.:m.}
\end{itemize}
\begin{itemize}
\item {Utilização:T. de Melgaço}
\end{itemize}
\begin{itemize}
\item {Proveniência:(De \textunderscore re...\textunderscore  + \textunderscore pelle\textunderscore )}
\end{itemize}
Escalavradura ou escoriação na mão e no pé.
\section{Repelo}
\begin{itemize}
\item {Grp. gram.:m.}
\end{itemize}
\begin{itemize}
\item {Utilização:T. de Melgaço}
\end{itemize}
\begin{itemize}
\item {Proveniência:(De \textunderscore re...\textunderscore  + \textunderscore pele\textunderscore )}
\end{itemize}
Escalavradura ou escoriação na mão e no pé.
\section{Repêlo}
\begin{itemize}
\item {Grp. gram.:m.}
\end{itemize}
\begin{itemize}
\item {Proveniência:(De \textunderscore repelar\textunderscore , dizem; mas cp. a etym. de \textunderscore repelão\textunderscore )}
\end{itemize}
Repelão, violência.
\section{Rependido}
\begin{itemize}
\item {Grp. gram.:adj.}
\end{itemize}
(V.arrependido). Cf. Filinto, XVI, 169.
\section{Rependimento}
\begin{itemize}
\item {Grp. gram.:m.}
\end{itemize}
\begin{itemize}
\item {Utilização:Ant.}
\end{itemize}
Parece o mesmo que \textunderscore arrependimento\textunderscore , por aphérese. Cp. \textunderscore rependido\textunderscore . Entretanto, o \textunderscore Elucidário\textunderscore  de Viterbo tem-no como synónymo de \textunderscore recompensa\textunderscore , \textunderscore paga\textunderscore , \textunderscore satisfação\textunderscore .
\section{Repenicar}
\begin{itemize}
\item {Grp. gram.:v. t.}
\end{itemize}
Fazer dar sons agudos e repetidos, percutindo substância metállica; repicar: \textunderscore repenicar os sinos\textunderscore .
(Corr. de \textunderscore repicar\textunderscore )
\section{Repenique}
\begin{itemize}
\item {Grp. gram.:m.}
\end{itemize}
Acto de repenicar.
Som argentino ou metállico (de certas vozes); repique.
\section{Repensão}
\begin{itemize}
\item {Grp. gram.:f.}
\end{itemize}
\begin{itemize}
\item {Proveniência:(De \textunderscore re...\textunderscore  + \textunderscore pensão\textunderscore )}
\end{itemize}
Nova pensão.
Pensão, imposta num benefício que já pagava outra.
\section{Repensar}
\begin{itemize}
\item {Grp. gram.:v. t.  e  i.}
\end{itemize}
\begin{itemize}
\item {Proveniência:(De \textunderscore re...\textunderscore  + \textunderscore pensar\textunderscore )}
\end{itemize}
Pensar de novo.
Pensar maduramente; reconsiderar.
\section{Repente}
\begin{itemize}
\item {Grp. gram.:m.}
\end{itemize}
\begin{itemize}
\item {Grp. gram.:Loc. adv.}
\end{itemize}
\begin{itemize}
\item {Proveniência:(Lat. \textunderscore repente\textunderscore )}
\end{itemize}
Dito ou acto repentino, irreflectido.
Movimento espontâneo, filho da índole ou do instinto e não da reflexão.
\textunderscore De repente\textunderscore , repentinamente.
\section{Repentinamente}
\begin{itemize}
\item {Grp. gram.:adv.}
\end{itemize}
De modo repentino; subitamente; de improviso; imprevistamente.
\section{Repentino}
\begin{itemize}
\item {Grp. gram.:adj.}
\end{itemize}
\begin{itemize}
\item {Proveniência:(Lat. \textunderscore repentinus\textunderscore )}
\end{itemize}
Relativo a repente; súbito; inopinado, imprevisto: \textunderscore doença repentina\textunderscore .
\section{Repentista}
\begin{itemize}
\item {Grp. gram.:m. ,  f.  e  adj.}
\end{itemize}
\begin{itemize}
\item {Proveniência:(De \textunderscore repente\textunderscore )}
\end{itemize}
Pessôa, que faz ou diz coisas de repente.
Pessôa que improvisa.
\section{Repercussão}
\begin{itemize}
\item {Grp. gram.:f.}
\end{itemize}
\begin{itemize}
\item {Proveniência:(Lat. \textunderscore repercussio\textunderscore )}
\end{itemize}
Acto ou effeito de repercutir.
\section{Repercussivo}
\begin{itemize}
\item {Grp. gram.:m.  e  adj.}
\end{itemize}
\begin{itemize}
\item {Proveniência:(De \textunderscore repercusso\textunderscore )}
\end{itemize}
Próprio para fazer repercussão.
\section{Repercusso}
\begin{itemize}
\item {Grp. gram.:m.}
\end{itemize}
\begin{itemize}
\item {Utilização:Des.}
\end{itemize}
\begin{itemize}
\item {Proveniência:(Lat. \textunderscore repercussus\textunderscore )}
\end{itemize}
O mesmo que \textunderscore repercussão\textunderscore .
\section{Repercutente}
\begin{itemize}
\item {Grp. gram.:adj.}
\end{itemize}
Que repercute.
\section{Repercutir}
\begin{itemize}
\item {Grp. gram.:v. t.}
\end{itemize}
\begin{itemize}
\item {Grp. gram.:V. i.}
\end{itemize}
\begin{itemize}
\item {Proveniência:(Lat. \textunderscore repercutere\textunderscore )}
\end{itemize}
Reflectir; reproduzir, (falando-se dos sons).
Reflectir-se; reproduzir-se, (falando-se de sons).
\section{Reperdido}
\begin{itemize}
\item {Grp. gram.:adj.}
\end{itemize}
Muito perdido:«\textunderscore acham-se portanto, aquellas religiosas veneraveis, mais que perdidas, reperdidas.\textunderscore »Castilho, \textunderscore Livrar. Cláss.\textunderscore , VII, 84. Cf. \textunderscore Idem. Metam.\textunderscore , XXIV.
\section{Repergunta}
\begin{itemize}
\item {Grp. gram.:f.}
\end{itemize}
Acto de reperguntar.
\section{Reperguntar}
\begin{itemize}
\item {Grp. gram.:v. t.}
\end{itemize}
\begin{itemize}
\item {Proveniência:(De \textunderscore re...\textunderscore  + \textunderscore perguntar\textunderscore  ou \textunderscore preguntar\textunderscore )}
\end{itemize}
Perguntar de novo.
\section{Repertório}
\begin{itemize}
\item {Grp. gram.:m.}
\end{itemize}
\begin{itemize}
\item {Utilização:Fig.}
\end{itemize}
\begin{itemize}
\item {Proveniência:(Lat. \textunderscore repertorium\textunderscore )}
\end{itemize}
Índice.
Conjunto de notícias succintas e methodicamente dispostas.
Collecção.
Compilação.
Calendário, almanaque.
Conjunto de peças dramáticas ou de óperas.
Obras musicaes ou dramáticas, que têm sido cantadas ou representadas por um músico ou actor.
Peças musicaes, executadas num concêrto.
Conjunto das obras de um maestro ou de um actor dramático.
Pessôa, perita em certos assumptos.
\section{Repes}
\begin{itemize}
\item {Grp. gram.:m.}
\end{itemize}
\begin{itemize}
\item {Proveniência:(Do fr. \textunderscore reps\textunderscore )}
\end{itemize}
Tecido encorpado de seda, lan ou algodão, para reposteiros, estôfos de cadeiras, etc.
\section{Repesador}
\begin{itemize}
\item {Grp. gram.:m.  e  adj.}
\end{itemize}
O que repesa.
\section{Repesar}
\begin{itemize}
\item {Grp. gram.:v. t.}
\end{itemize}
\begin{itemize}
\item {Proveniência:(De \textunderscore re...\textunderscore  + \textunderscore pesar\textunderscore )}
\end{itemize}
Pesar de novo.
\section{Repêso}
\begin{itemize}
\item {Grp. gram.:m.}
\end{itemize}
Acto de repesar.
\section{Repêso}
\begin{itemize}
\item {Grp. gram.:adj.}
\end{itemize}
\begin{itemize}
\item {Proveniência:(Do lat. \textunderscore repensus\textunderscore )}
\end{itemize}
O mesmo que \textunderscore arrependido\textunderscore :«\textunderscore repêso me sinto.\textunderscore »Camillo, \textunderscore Regicida\textunderscore , 165.
\section{Repesoiro}
\begin{itemize}
\item {Grp. gram.:m.}
\end{itemize}
\begin{itemize}
\item {Utilização:Prov.}
\end{itemize}
\begin{itemize}
\item {Utilização:trasm.}
\end{itemize}
Terreno baldio, concelhio.
\section{Repesouro}
\begin{itemize}
\item {Grp. gram.:m.}
\end{itemize}
\begin{itemize}
\item {Utilização:Prov.}
\end{itemize}
\begin{itemize}
\item {Utilização:trasm.}
\end{itemize}
Terreno baldio, concelhio.
\section{Repetenadamente}
\begin{itemize}
\item {Grp. gram.:adv.}
\end{itemize}
De modo repetenado; repimpadamente.
\section{Repetenar-se}
\begin{itemize}
\item {Grp. gram.:v. p.}
\end{itemize}
\begin{itemize}
\item {Utilização:Fam.}
\end{itemize}
Refocilar-se; repoltrear-se.
\section{Repetência}
\begin{itemize}
\item {Grp. gram.:f.}
\end{itemize}
\begin{itemize}
\item {Proveniência:(Lat. \textunderscore repetentia\textunderscore )}
\end{itemize}
Repetição.
Derivação de humôres para alguma parte de um organismo.
\section{Repetente}
\begin{itemize}
\item {Grp. gram.:adj.}
\end{itemize}
\begin{itemize}
\item {Grp. gram.:M.}
\end{itemize}
\begin{itemize}
\item {Proveniência:(Lat. \textunderscore repetens\textunderscore )}
\end{itemize}
Que repete.
Estudante de uma disciplina que já cursou.
\section{Repetição}
\begin{itemize}
\item {Grp. gram.:f.}
\end{itemize}
\begin{itemize}
\item {Proveniência:(Lat. \textunderscore repetitio\textunderscore )}
\end{itemize}
Acto ou effeito de repetir.
\section{Repetidamente}
\begin{itemize}
\item {Grp. gram.:adv.}
\end{itemize}
De modo repetido; muitas vezes; frequentemente.
\section{Repetidor}
\begin{itemize}
\item {Grp. gram.:adj.}
\end{itemize}
\begin{itemize}
\item {Grp. gram.:M.}
\end{itemize}
\begin{itemize}
\item {Proveniência:(De \textunderscore repetir\textunderscore )}
\end{itemize}
Que repete.
Professor ou leccionista que repete ou explica as lições que um alumno tem de dar na aula respectiva.
\section{Repetimento}
\begin{itemize}
\item {Grp. gram.:m.}
\end{itemize}
\begin{itemize}
\item {Utilização:Des.}
\end{itemize}
(V.repetição)
\section{Repetir}
\begin{itemize}
\item {Grp. gram.:v. t.}
\end{itemize}
\begin{itemize}
\item {Grp. gram.:V. i.}
\end{itemize}
\begin{itemize}
\item {Proveniência:(Lat. \textunderscore repetere\textunderscore )}
\end{itemize}
Tornar a fazer ou a dizer.
Insistir em.
Repisar.
Repercutir.
Apparecer de novo: \textunderscore o accesso repetiu\textunderscore .
Acontecer novamente.
\section{Repetoca}
\begin{itemize}
\item {Grp. gram.:v.}
\end{itemize}
\begin{itemize}
\item {Utilização:T. do Fundão}
\end{itemize}
Reprehensão, descompostura.
\section{Repicador}
\begin{itemize}
\item {Grp. gram.:m.  e  adj.}
\end{itemize}
O que repica.
\section{Repicagem}
\begin{itemize}
\item {Grp. gram.:f.}
\end{itemize}
Acto ou effeito de repicar.
\section{Repicaponto}
\begin{itemize}
\item {Grp. gram.:m.}
\end{itemize}
\begin{itemize}
\item {Proveniência:(De \textunderscore repicar\textunderscore  + \textunderscore ponto\textunderscore )}
\end{itemize}
Perfeição; excellência.
\section{Repicar}
\begin{itemize}
\item {Grp. gram.:v. t.}
\end{itemize}
\begin{itemize}
\item {Grp. gram.:V. i.}
\end{itemize}
\begin{itemize}
\item {Proveniência:(De \textunderscore re...\textunderscore  + \textunderscore picar\textunderscore )}
\end{itemize}
Picar de novo.
Fazer tirar sons agudos e repetidos; repenicar.
Fazer repique.
\section{Repichel}
\begin{itemize}
\item {Grp. gram.:m.}
\end{itemize}
\begin{itemize}
\item {Utilização:Pesc.}
\end{itemize}
O mesmo que \textunderscore ganapão\textunderscore .
\section{Repiegas}
\begin{itemize}
\item {Grp. gram.:m.  e  f.}
\end{itemize}
\begin{itemize}
\item {Utilização:Prov.}
\end{itemize}
\begin{itemize}
\item {Utilização:trasm.}
\end{itemize}
Pessôa mentirosa. (Colhido em V. P. de Aguiar)
\section{Repieiro}
\begin{itemize}
\item {Grp. gram.:m.}
\end{itemize}
\begin{itemize}
\item {Utilização:Prov.}
\end{itemize}
\begin{itemize}
\item {Utilização:trasm.}
\end{itemize}
Tanchão, ainda com os ramos.
\section{Repilgado}
\begin{itemize}
\item {Grp. gram.:adj.}
\end{itemize}
\begin{itemize}
\item {Utilização:Prov.}
\end{itemize}
\begin{itemize}
\item {Utilização:trasm.}
\end{itemize}
Bem cheio, (saco, bôlso, etc.).
\section{Repimpadamente}
\begin{itemize}
\item {Grp. gram.:adv.}
\end{itemize}
De modo repimpado; com refestêlo.
\section{Repimpar}
\begin{itemize}
\item {Grp. gram.:v. t.}
\end{itemize}
\begin{itemize}
\item {Grp. gram.:V. p.}
\end{itemize}
\begin{itemize}
\item {Proveniência:(De \textunderscore re...\textunderscore  + \textunderscore pimpar\textunderscore )}
\end{itemize}
Encher muito a barriga de.
Refestelar-se.
Locupletar-se.
\section{Repinal}
\begin{itemize}
\item {Grp. gram.:m.  e  adj.}
\end{itemize}
Variedade de peros oblongos e doces.
\section{Repinaldo}
\begin{itemize}
\item {Grp. gram.:m.  e  adj.}
\end{itemize}
Variedade de peros oblongos e doces.
\section{Repinchado}
\begin{itemize}
\item {Grp. gram.:adj.}
\end{itemize}
Que repinchou.
\section{Repinchar}
\begin{itemize}
\item {Grp. gram.:v. i.}
\end{itemize}
\begin{itemize}
\item {Proveniência:(De \textunderscore re...\textunderscore  + \textunderscore pinchar\textunderscore )}
\end{itemize}
Resaltar depois do pisado.
\section{Repintalgado}
\begin{itemize}
\item {Grp. gram.:adj.}
\end{itemize}
Pintado de muitas côres ou de côres vivas:« \textunderscore o túrgido carão repintalgado...\textunderscore »Castilho, \textunderscore Fastos\textunderscore , II, 87.
(Do \textunderscore re...\textunderscore  + \textunderscore pintalgado\textunderscore )
\section{Repintar}
\begin{itemize}
\item {Grp. gram.:v. t.}
\end{itemize}
\begin{itemize}
\item {Grp. gram.:V. i.}
\end{itemize}
\begin{itemize}
\item {Proveniência:(De \textunderscore re...\textunderscore  + \textunderscore pintar\textunderscore )}
\end{itemize}
Tornar a pintar.
Copiar.
Reproduzir.
Tornar saliente ou mais visível; avivar.
Reproduzir-se numa página por meio de pressão, o que está escrito na página contígua.
\section{Repique}
\begin{itemize}
\item {Grp. gram.:m.}
\end{itemize}
Acto ou effeito de repicar.
Acto de repicar os sinos, geralmente em sinal de festa.
Choque de duas bolas, no bilhar, depois de carambolarem, ou logo depois de se tocarem.
Rebate de sinos, alarma. Cf. Filinto, \textunderscore D. Man.\textunderscore , III, 401.
\section{Repiquetar}
\begin{itemize}
\item {Grp. gram.:v. t.}
\end{itemize}
\begin{itemize}
\item {Proveniência:(De \textunderscore re...\textunderscore  + \textunderscore piquetar\textunderscore )}
\end{itemize}
Verificar ou corrigir a piquetagem de.
\section{Repiquete}
\begin{itemize}
\item {fónica:quê}
\end{itemize}
\begin{itemize}
\item {Grp. gram.:m.}
\end{itemize}
\begin{itemize}
\item {Utilização:Bras}
\end{itemize}
\begin{itemize}
\item {Grp. gram.:Loc.}
\end{itemize}
\begin{itemize}
\item {Utilização:Loc. de Ceará.}
\end{itemize}
O mesmo que \textunderscore ladeira\textunderscore .
Rebate de sinos, amiudado.
Enchente, que se avoluma successivamente nos rios amazónicos.
\textunderscore Repiquete de seca\textunderscore , falta do chuvas.
(Cp. cast. \textunderscore repiquete\textunderscore )
\section{Repisa}
\begin{itemize}
\item {Grp. gram.:f.}
\end{itemize}
\begin{itemize}
\item {Utilização:Constr.}
\end{itemize}
Acto ou effeito de repisar.
O mesmo que \textunderscore sapata\textunderscore .
\section{Repisamento}
\begin{itemize}
\item {Grp. gram.:m.}
\end{itemize}
Acto de repisar; repisa.
\section{Repisar}
\begin{itemize}
\item {Grp. gram.:v. t.}
\end{itemize}
\begin{itemize}
\item {Proveniência:(De \textunderscore re...\textunderscore  + \textunderscore pisar\textunderscore )}
\end{itemize}
Pisar do novo; pisar muito.
Repetir.
Mentir.
\section{Repiscar}
\begin{itemize}
\item {Grp. gram.:v. t.}
\end{itemize}
\begin{itemize}
\item {Proveniência:(De \textunderscore re...\textunderscore  + \textunderscore piscar\textunderscore )}
\end{itemize}
Piscar repetidas vezes (os olhos). Cf. Filinto, XX, 168.
\section{Repiso}
\begin{itemize}
\item {Grp. gram.:m.}
\end{itemize}
Acto de repisar; repisa:«\textunderscore ...no acanhado repiso das palavras\textunderscore ». Filinto, I, 137.
\section{Repitosca}
\begin{itemize}
\item {Grp. gram.:f.}
\end{itemize}
\begin{itemize}
\item {Utilização:Prov.}
\end{itemize}
\begin{itemize}
\item {Utilização:trasm.}
\end{itemize}
Rapariga bonita e desembaraçada.
\section{Replaina}
\begin{itemize}
\item {Grp. gram.:f.}
\end{itemize}
O mesmo que \textunderscore replaino\textunderscore .
\section{Replainar}
\begin{itemize}
\item {Grp. gram.:v. t.}
\end{itemize}
\begin{itemize}
\item {Proveniência:(De \textunderscore re...\textunderscore  + \textunderscore plainar\textunderscore )}
\end{itemize}
Moldar com o replaino; desgastar com o replaino (a madeira).
\section{Replaino}
\begin{itemize}
\item {Grp. gram.:m.}
\end{itemize}
\begin{itemize}
\item {Proveniência:(De \textunderscore replainar\textunderscore )}
\end{itemize}
Espécie de cepo de carpinteiro, para fazer molduras de portas, ou com que se rebaixam as orlas das almofadas para entrarem nos envaziados.
\section{Replantação}
\begin{itemize}
\item {Grp. gram.:f.}
\end{itemize}
Acto ou effeito de replantar.
\section{Replantar}
\begin{itemize}
\item {Grp. gram.:v. t.}
\end{itemize}
\begin{itemize}
\item {Proveniência:(De \textunderscore re...\textunderscore  + \textunderscore plantar\textunderscore )}
\end{itemize}
Tornar a plantar.
\section{Replantio}
\begin{itemize}
\item {Grp. gram.:m.}
\end{itemize}
\begin{itemize}
\item {Proveniência:(De \textunderscore re...\textunderscore  + \textunderscore plantio\textunderscore )}
\end{itemize}
O mesmo que \textunderscore replantação\textunderscore .
\section{Repleção}
\begin{itemize}
\item {Grp. gram.:f.}
\end{itemize}
\begin{itemize}
\item {Proveniência:(Lat. \textunderscore repletio\textunderscore )}
\end{itemize}
Estado do que é repleto.
\section{Replenado}
\begin{itemize}
\item {Grp. gram.:adj.}
\end{itemize}
\begin{itemize}
\item {Proveniência:(De \textunderscore repleno\textunderscore )}
\end{itemize}
O mesmo que [[terraplenado|terraplenar]].
\section{Repleno}
\begin{itemize}
\item {Grp. gram.:m.}
\end{itemize}
\begin{itemize}
\item {Proveniência:(De \textunderscore re...\textunderscore  + \textunderscore pleno\textunderscore )}
\end{itemize}
O mesmo que \textunderscore terrapleno\textunderscore .
\section{Repleto}
\begin{itemize}
\item {Grp. gram.:adj.}
\end{itemize}
\begin{itemize}
\item {Proveniência:(Lat. \textunderscore repletus\textunderscore )}
\end{itemize}
Muito cheio: \textunderscore o theatro estava repleto\textunderscore .
Abarrotado; farto.
\section{Réplica}
\begin{itemize}
\item {Grp. gram.:f.}
\end{itemize}
\begin{itemize}
\item {Utilização:Mús.}
\end{itemize}
Acto ou effeito de replicar.
Aquillo que se replíca.
Resposta a uma crítica: \textunderscore a soberba réplica de Rui Barbosa...\textunderscore 
Sinal musical, para indicar que tem de se repetir certo trecho de uma peça executada.
\section{Replicação}
\begin{itemize}
\item {Grp. gram.:f.}
\end{itemize}
\begin{itemize}
\item {Proveniência:(Lat. \textunderscore replicatio\textunderscore )}
\end{itemize}
O mesmo que \textunderscore réplica\textunderscore .
\section{Replicador}
\begin{itemize}
\item {Grp. gram.:m.  e  adj.}
\end{itemize}
O que replíca.
\section{Replicar}
\begin{itemize}
\item {Grp. gram.:v. t.}
\end{itemize}
\begin{itemize}
\item {Grp. gram.:V. i.}
\end{itemize}
\begin{itemize}
\item {Proveniência:(Lat. \textunderscore replicare\textunderscore )}
\end{itemize}
Refutar com argumentos.
Refutar, redarguir.
Responder ás objecções ou argumentos de outrem.
\section{Replicativo}
\begin{itemize}
\item {Grp. gram.:adj.}
\end{itemize}
\begin{itemize}
\item {Utilização:Bot.}
\end{itemize}
\begin{itemize}
\item {Proveniência:(Do lat. \textunderscore re...\textunderscore  + \textunderscore plicare\textunderscore )}
\end{itemize}
Diz-se das fôlhas, quando, durante a prefoliação, a parte superior é curvada e applicada sôbre a inferior, como no acónito.
\section{Repoda}
\begin{itemize}
\item {Grp. gram.:f.}
\end{itemize}
Acto de repodar.
\section{Repodar}
\begin{itemize}
\item {Grp. gram.:v. t.}
\end{itemize}
\begin{itemize}
\item {Utilização:Prov.}
\end{itemize}
\begin{itemize}
\item {Utilização:minh.}
\end{itemize}
\begin{itemize}
\item {Proveniência:(De \textunderscore re...\textunderscore  + \textunderscore podar\textunderscore )}
\end{itemize}
Podar todos os annos, (ao contrário do costume minhoto de se podar, num anno sim, e no outro não).
\section{Repoisadamente}
\begin{itemize}
\item {Grp. gram.:adv.}
\end{itemize}
De modo repoisado.
Com repoiso, com sossêgo; tranquillamente.
\section{Repoisar}
\begin{itemize}
\item {Grp. gram.:v. t.}
\end{itemize}
\begin{itemize}
\item {Grp. gram.:V. i.}
\end{itemize}
\begin{itemize}
\item {Proveniência:(Lat. \textunderscore repausare\textunderscore )}
\end{itemize}
Descansar.
Fazer pausa em.
Tranquilizar.
Estar descansado.
Estar inactivo.
Estar em poisio.
Jazer na sepultura.
\section{Repoisentado}
\begin{itemize}
\item {Grp. gram.:adj.}
\end{itemize}
\begin{itemize}
\item {Utilização:Prov.}
\end{itemize}
\begin{itemize}
\item {Utilização:trasm.}
\end{itemize}
\begin{itemize}
\item {Proveniência:(De \textunderscore repoisar\textunderscore )}
\end{itemize}
Que está chôco ou requentado, por se demorar muito na panela (o caldo, principalmente).
\section{Repoiso}
\begin{itemize}
\item {Grp. gram.:m.}
\end{itemize}
\begin{itemize}
\item {Utilização:Ant.}
\end{itemize}
Acto ou effeito de repoisar.
O mesmo que \textunderscore ancoradoiro\textunderscore . Cf. \textunderscore Supplem. ao Diccion. de Algib.\textunderscore 
\section{Repolegar}
\begin{itemize}
\item {Grp. gram.:v. t.}
\end{itemize}
Dobrar, ornar com repolêgo ou filetes.
\section{Repolêgo}
\begin{itemize}
\item {Grp. gram.:m.}
\end{itemize}
\begin{itemize}
\item {Proveniência:(De \textunderscore repolegar\textunderscore )}
\end{itemize}
Filete torcido, para ornato de certas peças.
Filete de massa, que borda uma empada.
\section{Repolga}
\begin{itemize}
\item {Grp. gram.:f.}
\end{itemize}
Casta de cogumelos, que se cria nos castanheiros.
Acto de repolgar.
\section{Repolgar}
\begin{itemize}
\item {Grp. gram.:v. t.}
\end{itemize}
(Contr. de \textunderscore repolegar\textunderscore )
\section{Repolhaço}
\begin{itemize}
\item {Grp. gram.:m.}
\end{itemize}
\begin{itemize}
\item {Utilização:Prov.}
\end{itemize}
\begin{itemize}
\item {Utilização:trasm.}
\end{itemize}
\begin{itemize}
\item {Proveniência:(De \textunderscore repôlho\textunderscore )}
\end{itemize}
Homem gordo.
\section{Repolhal}
\begin{itemize}
\item {Grp. gram.:adj.}
\end{itemize}
\begin{itemize}
\item {Grp. gram.:M.}
\end{itemize}
Relativo a repolhos.
Terreno, onde crescem repolhos.
\section{Repolhar}
\begin{itemize}
\item {Grp. gram.:v. i.}
\end{itemize}
Adquirir a fórma de repôlho: \textunderscore há couves que repolham\textunderscore .
\section{Repôlho}
\begin{itemize}
\item {Grp. gram.:m.}
\end{itemize}
\begin{itemize}
\item {Proveniência:(Do lat. hyp. \textunderscore rapunculus\textunderscore )}
\end{itemize}
Espécie de couve, cujas fôlhas se ennovelam, formando um globo.
\section{Repolhudo}
\begin{itemize}
\item {Grp. gram.:adj.}
\end{itemize}
Que tem fórma de repôlho.
Gordo, anafado.
\section{Repoltrear-se}
\begin{itemize}
\item {Grp. gram.:v. p.}
\end{itemize}
O mesmo que \textunderscore poltronear-se\textunderscore .
Refestelar-se; refocilar-se. Cf. Rebello, \textunderscore Mocidade\textunderscore , I, 88.
\section{Repôncio}
\begin{itemize}
\item {Grp. gram.:m.}
\end{itemize}
O mesmo que \textunderscore rapúncio\textunderscore .
\section{Reponta}
\begin{itemize}
\item {Grp. gram.:f.}
\end{itemize}
\begin{itemize}
\item {Proveniência:(De \textunderscore re...\textunderscore  + \textunderscore ponta\textunderscore )}
\end{itemize}
Ponta, que apparece novamente ou de tempos a tempo.
Repetição de golpe com a ponta da espada ou de lança.
\section{Repontão}
\begin{itemize}
\item {Grp. gram.:adj.}
\end{itemize}
\begin{itemize}
\item {Utilização:Fam.}
\end{itemize}
\begin{itemize}
\item {Proveniência:(De \textunderscore repontar\textunderscore ^2)}
\end{itemize}
Que responde, que recalcitra, que respinga, quando é censurado.
\section{Repontar}
\begin{itemize}
\item {Grp. gram.:v. t.}
\end{itemize}
\begin{itemize}
\item {Utilização:Bras}
\end{itemize}
\begin{itemize}
\item {Proveniência:(De \textunderscore re...\textunderscore  + \textunderscore ponto\textunderscore )}
\end{itemize}
Fazer refluír para certo ponto.
Enxotar (animaes) em certa direcção.
\section{Repontar}
\begin{itemize}
\item {Grp. gram.:v. i.}
\end{itemize}
\begin{itemize}
\item {Proveniência:(De \textunderscore re...\textunderscore  + \textunderscore ponta\textunderscore )}
\end{itemize}
Surgir novamente.
Apparecer a pouco e pouco; começar a vêr-se: \textunderscore repontava a Lua\textunderscore .
Raiar, (falando-se da aurora ou do dia).
Arremeter, voltando-se para trás.
Retorquir com aspereza, respongar, recalcitrar.
\section{Repor}
\begin{itemize}
\item {Grp. gram.:v. t.}
\end{itemize}
\begin{itemize}
\item {Proveniência:(Do lat. \textunderscore reponere\textunderscore )}
\end{itemize}
Pôr de novo.
Restituir.
Tornar a fazer.
\section{Reportação}
\begin{itemize}
\item {Grp. gram.:f.}
\end{itemize}
\begin{itemize}
\item {Proveniência:(Lat. \textunderscore reportatio\textunderscore )}
\end{itemize}
Acto ou effeito de reportar.
\section{Reportado}
\begin{itemize}
\item {Grp. gram.:adj.}
\end{itemize}
\begin{itemize}
\item {Proveniência:(De \textunderscore reportar\textunderscore )}
\end{itemize}
Moderado; discreto.
\section{Reportagem}
\begin{itemize}
\item {Grp. gram.:f.}
\end{itemize}
\begin{itemize}
\item {Utilização:Neol.}
\end{itemize}
\begin{itemize}
\item {Proveniência:(De \textunderscore repórter\textunderscore )}
\end{itemize}
Acto ou effeito de dar informações aos periódicos.
Classe dos informadores de jornaes.
\section{Reportamento}
\begin{itemize}
\item {Grp. gram.:m.}
\end{itemize}
O mesmo que \textunderscore reportação\textunderscore .
\section{Reportar}
\begin{itemize}
\item {Grp. gram.:v. t.}
\end{itemize}
\begin{itemize}
\item {Grp. gram.:V. p.}
\end{itemize}
\begin{itemize}
\item {Proveniência:(Lat. \textunderscore reportare\textunderscore )}
\end{itemize}
Virar para trás, retrahir.
Moderar.
Referir.
Moderar-se.
Alludir, referir-se: \textunderscore reporto-me ao que já disse\textunderscore .
\section{Repórter}
\begin{itemize}
\item {Grp. gram.:m.}
\end{itemize}
\begin{itemize}
\item {Utilização:Neol.}
\end{itemize}
\begin{itemize}
\item {Proveniência:(Ingl. \textunderscore reporter\textunderscore )}
\end{itemize}
Noticiarista, informador de jornaes.
Aquelle que procura notícias para a imprensa periódica.
\section{Reportório}
\begin{itemize}
\item {Grp. gram.:m.}
\end{itemize}
O mesmo que \textunderscore repertório\textunderscore .
(Alter. de \textunderscore repertório\textunderscore , se não se relaciona com \textunderscore reportar\textunderscore )
\section{Reposição}
\begin{itemize}
\item {Grp. gram.:f.}
\end{itemize}
\begin{itemize}
\item {Proveniência:(Lat. \textunderscore repositio\textunderscore )}
\end{itemize}
Acto ou effeito de repor.
\section{Repósito}
\begin{itemize}
\item {Grp. gram.:m.}
\end{itemize}
\begin{itemize}
\item {Proveniência:(Lat. \textunderscore repositus\textunderscore )}
\end{itemize}
O mesmo que \textunderscore reposte\textunderscore .
\section{Repositório}
\begin{itemize}
\item {Grp. gram.:adj.}
\end{itemize}
\begin{itemize}
\item {Utilização:Pharm.}
\end{itemize}
\begin{itemize}
\item {Grp. gram.:M.}
\end{itemize}
\begin{itemize}
\item {Proveniência:(Lat. \textunderscore repositorium\textunderscore )}
\end{itemize}
Em que se guardam medicamentos.
Sítio ou lugar, onde se conserva ou guarda alguma coisa.
Depósito.
Compilação.
Capella ou nicho á beira das ruas, em que fazem paragem as procissões ou se realizam algumas ceremónias religiosas.
\section{Reposta}
\begin{itemize}
\item {Grp. gram.:f.}
\end{itemize}
\begin{itemize}
\item {Proveniência:(Do lat. \textunderscore reposita\textunderscore . Cp. gall. \textunderscore reposta\textunderscore )}
\end{itemize}
Quantia que se repõe, no jôgo do voltarete.
O mesmo ou melhor que \textunderscore resposta\textunderscore . Cf. Moraes e Silva, \textunderscore Diccion.\textunderscore ; \textunderscore Lusíadas\textunderscore , IX, 16; \textunderscore Peregrinação\textunderscore , III; R. Pina, \textunderscore Crón. Aff. V\textunderscore , LXXIII, e LXXIX; G. Resende, \textunderscore D. João II\textunderscore , 2.^a p., c. CCXII; Usque, 7 v.^o; Pant. de Aveiro, \textunderscore Itiner.\textunderscore  89 e 101. (2.^a ed.).
\section{Repostada}
\begin{itemize}
\item {Grp. gram.:f.}
\end{itemize}
\begin{itemize}
\item {Proveniência:(De \textunderscore resposta\textunderscore )}
\end{itemize}
Réplica grosseira.
Resposta incivíl ou plebeia. Cf. Sousa, \textunderscore Vida do Arceb.\textunderscore , II, 323; Rebello, \textunderscore Contos e Lendas\textunderscore , 91.
\section{Repostaria}
\begin{itemize}
\item {Grp. gram.:f.}
\end{itemize}
\begin{itemize}
\item {Utilização:P. us.}
\end{itemize}
\begin{itemize}
\item {Proveniência:(De \textunderscore reposte\textunderscore )}
\end{itemize}
Lugar onde nas casas nobres, se fazem doces e licores.
Objectos e pessoal da copa.
\section{Reposte}
\begin{itemize}
\item {Grp. gram.:m.}
\end{itemize}
\begin{itemize}
\item {Utilização:Ant.}
\end{itemize}
\begin{itemize}
\item {Proveniência:(De \textunderscore reposto\textunderscore )}
\end{itemize}
Casa para depósito ou guadar de móveis.
O conteúdo dessa casa.
\section{Reposteiro}
\begin{itemize}
\item {Grp. gram.:m.}
\end{itemize}
\begin{itemize}
\item {Proveniência:(De \textunderscore reposto\textunderscore )}
\end{itemize}
Cortina ou peça de estôfo, com que se cobrem as portas interiores de um edifício.
\section{Reposteiro}
\begin{itemize}
\item {Grp. gram.:m.}
\end{itemize}
\begin{itemize}
\item {Proveniência:(De \textunderscore reposte\textunderscore )}
\end{itemize}
Indivíduo, que tinha a seu cargo o reposte da casa real.
Thesoureiro.
Vestiário.
\section{Reposto}
\begin{itemize}
\item {Grp. gram.:adj.}
\end{itemize}
\begin{itemize}
\item {Proveniência:(Do lat. \textunderscore repositus\textunderscore )}
\end{itemize}
Que se repôs.
Que retomou ânimo.
Que se restabeleceu.
\section{Repotrear-se}
\begin{itemize}
\item {Grp. gram.:v. p.}
\end{itemize}
(V.repoltrear-se)
\section{Repotronar-se}
\begin{itemize}
\item {Grp. gram.:v. p.}
\end{itemize}
\begin{itemize}
\item {Utilização:Prov.}
\end{itemize}
O mesmo que \textunderscore repoltrear-se\textunderscore .
\section{Repousadamente}
\begin{itemize}
\item {Grp. gram.:adv.}
\end{itemize}
De modo repousado.
Com repouso, com sossêgo; tranquilamente.
\section{Repousar}
\begin{itemize}
\item {Grp. gram.:v. t.}
\end{itemize}
\begin{itemize}
\item {Grp. gram.:V. i.}
\end{itemize}
\begin{itemize}
\item {Proveniência:(Lat. \textunderscore repausare\textunderscore )}
\end{itemize}
Descansar.
Fazer pausa em.
Tranquilizar.
Estar descansado.
Estar inactivo.
Estar em pousio.
Jazer na sepultura.
\section{Repousentado}
\begin{itemize}
\item {Grp. gram.:adj.}
\end{itemize}
\begin{itemize}
\item {Utilização:Prov.}
\end{itemize}
\begin{itemize}
\item {Utilização:trasm.}
\end{itemize}
\begin{itemize}
\item {Proveniência:(De \textunderscore repousar\textunderscore )}
\end{itemize}
Que está chôco ou requentado, por se demorar muito na panela (o caldo, principalmente).
\section{Repouso}
\begin{itemize}
\item {Grp. gram.:m.}
\end{itemize}
\begin{itemize}
\item {Utilização:Ant.}
\end{itemize}
Acto ou efeito de repousar.
O mesmo que \textunderscore ancoradouro\textunderscore . Cf. \textunderscore Supplem. ao Diccion. de Algib.\textunderscore 
\section{Repovoação}
\begin{itemize}
\item {Grp. gram.:f.}
\end{itemize}
Acto ou effeito de repovoar. Cf. Castilho, \textunderscore Metam.\textunderscore , XVI.
\section{Repovoar}
\begin{itemize}
\item {Grp. gram.:v. t.}
\end{itemize}
\begin{itemize}
\item {Proveniência:(De \textunderscore re...\textunderscore  + \textunderscore povoar\textunderscore )}
\end{itemize}
Povoar de novo.
\section{Repreendedor}
\begin{itemize}
\item {Grp. gram.:m.  e  adj.}
\end{itemize}
O que repreende.
\section{Repreender}
\begin{itemize}
\item {Grp. gram.:v. t.}
\end{itemize}
\begin{itemize}
\item {Proveniência:(Lat. \textunderscore reprehendere\textunderscore )}
\end{itemize}
Admoestar energicamente; censurar; arguir.
\section{Repreendimento}
\begin{itemize}
\item {Grp. gram.:m.}
\end{itemize}
O mesmo que \textunderscore repreensão\textunderscore .
\section{Repreensão}
\begin{itemize}
\item {Grp. gram.:f.}
\end{itemize}
\begin{itemize}
\item {Proveniência:(Lat. \textunderscore reprehensio\textunderscore )}
\end{itemize}
Acto ou efeito de repreender.
\section{Repreensível}
\begin{itemize}
\item {Grp. gram.:adj.}
\end{itemize}
\begin{itemize}
\item {Proveniência:(Do lat. \textunderscore reprehensibilis\textunderscore )}
\end{itemize}
Que merece repreensão censurável.
\section{Repreensivelmente}
\begin{itemize}
\item {Grp. gram.:adv.}
\end{itemize}
De modo repreensível: \textunderscore proceder repreensivelmente\textunderscore .
\section{Repreensivo}
\begin{itemize}
\item {Grp. gram.:adj.}
\end{itemize}
\begin{itemize}
\item {Proveniência:(Do lat. \textunderscore reprehensus\textunderscore )}
\end{itemize}
Que repreende ou envolve repreensão: \textunderscore palavras repreensivas\textunderscore .
\section{Repreensor}
\begin{itemize}
\item {Grp. gram.:m.  e  adj.}
\end{itemize}
\begin{itemize}
\item {Proveniência:(Lat. \textunderscore reprehensor\textunderscore )}
\end{itemize}
O mesmo que \textunderscore repreendedor\textunderscore .
Que contém repreensão; repreensivo.
\section{Repregar}
\begin{itemize}
\item {Grp. gram.:v. t.}
\end{itemize}
\begin{itemize}
\item {Proveniência:(De \textunderscore re...\textunderscore  + \textunderscore pregar\textunderscore )}
\end{itemize}
Tornar a pregar.
Segurar fortemente com pregos.
Enfeitar com pregaria.
\section{Reprego}
\begin{itemize}
\item {Grp. gram.:m.}
\end{itemize}
Acto ou effeito de repregar.
Parte accessória das vistas de um theatro.
\section{Repreguntar}
\begin{itemize}
\item {Grp. gram.:v. t.}
\end{itemize}
\begin{itemize}
\item {Proveniência:(De \textunderscore re...\textunderscore  + \textunderscore perguntar\textunderscore  ou \textunderscore preguntar\textunderscore )}
\end{itemize}
Perguntar de novo.
\section{Reprehendedor}
\begin{itemize}
\item {Grp. gram.:m.  e  adj.}
\end{itemize}
O que reprehende.
\section{Reprehender}
\begin{itemize}
\item {Grp. gram.:v. t.}
\end{itemize}
\begin{itemize}
\item {Proveniência:(Lat. \textunderscore reprehendere\textunderscore )}
\end{itemize}
Admoestar energicamente; censurar; arguir.
\section{Reprehendimento}
\begin{itemize}
\item {Grp. gram.:m.}
\end{itemize}
O mesmo que \textunderscore reprehensão\textunderscore .
\section{Reprehensão}
\begin{itemize}
\item {Grp. gram.:f.}
\end{itemize}
\begin{itemize}
\item {Proveniência:(Lat. \textunderscore reprehensio\textunderscore )}
\end{itemize}
Acto ou effeito de reprehender.
\section{Reprehensível}
\begin{itemize}
\item {Grp. gram.:adj.}
\end{itemize}
\begin{itemize}
\item {Proveniência:(Do lat. \textunderscore reprehensibilis\textunderscore )}
\end{itemize}
Que merece reprehensão censurável.
\section{Reprehensivelmente}
\begin{itemize}
\item {Grp. gram.:adv.}
\end{itemize}
De modo reprehensível: \textunderscore proceder reprehensivelmente\textunderscore .
\section{Reprehensivo}
\begin{itemize}
\item {Grp. gram.:adj.}
\end{itemize}
\begin{itemize}
\item {Proveniência:(Do lat. \textunderscore reprehensus\textunderscore )}
\end{itemize}
Que reprehende ou envolve reprehensão: \textunderscore palavras reprehensivas\textunderscore .
\section{Reprehensor}
\begin{itemize}
\item {Grp. gram.:m.  e  adj.}
\end{itemize}
\begin{itemize}
\item {Proveniência:(Lat. \textunderscore reprehensor\textunderscore )}
\end{itemize}
O mesmo que \textunderscore reprehendedor\textunderscore .
Que contém reprehensão; reprehensivo.
\section{Reprender}
\textunderscore v. t.\textunderscore  (e der.)
O mesmo ou melhor que \textunderscore reprehender\textunderscore :«\textunderscore andava ali a reprender o estio de nãovir\textunderscore . »Castilho, \textunderscore Geórgicas\textunderscore . Cf. \textunderscore Eufrosina\textunderscore , 5 e 6; F. Manuel, \textunderscore Apólogos\textunderscore , etc.
\section{Reprendoiro}
\begin{itemize}
\item {Grp. gram.:adj.}
\end{itemize}
\begin{itemize}
\item {Utilização:Ant.}
\end{itemize}
\begin{itemize}
\item {Proveniência:(De \textunderscore reprender\textunderscore )}
\end{itemize}
Digno de reprehensão.
\section{Reprendouro}
\begin{itemize}
\item {Grp. gram.:adj.}
\end{itemize}
\begin{itemize}
\item {Utilização:Ant.}
\end{itemize}
\begin{itemize}
\item {Proveniência:(De \textunderscore reprender\textunderscore )}
\end{itemize}
Digno de reprehensão.
\section{Reprensão}
\begin{itemize}
\item {Grp. gram.:f.}
\end{itemize}
O mesmo ou melhor que \textunderscore reprehensão\textunderscore . Cf. \textunderscore Aulegrafia\textunderscore , 4.
\section{Reprêsa}
\begin{itemize}
\item {Grp. gram.:f.}
\end{itemize}
Acto ou effeito de represar.
Repressão.
Açude.
Agua represada, para usos industriaes ou agrícolas.
Accumulação.
Mísula, peanha.
Consêrto de uma parede.
Embarcação recuperada, tendo sido apresada pelo inimigo.
\section{Represadamente}
\begin{itemize}
\item {Grp. gram.:adv.}
\end{itemize}
De modo represado.
Por meio de reprêsa.
Com repressão.
\section{Represado}
\begin{itemize}
\item {Grp. gram.:adj.}
\end{itemize}
\begin{itemize}
\item {Proveniência:(De \textunderscore represar\textunderscore )}
\end{itemize}
Que se represou: \textunderscore água represada\textunderscore .
Concentrado; reprimido: \textunderscore ódios represados\textunderscore .
\section{Represador}
\begin{itemize}
\item {Grp. gram.:m.  e  adj.}
\end{itemize}
O que représa.
\section{Represadura}
\begin{itemize}
\item {Grp. gram.:f.}
\end{itemize}
\begin{itemize}
\item {Utilização:Des.}
\end{itemize}
Acto de represar.
Acto de aprehender haveres a gente dos inimigos, em compensação do que elles haviam aprehendido.
Represália.
\section{Represália}
\begin{itemize}
\item {Grp. gram.:f.}
\end{itemize}
\begin{itemize}
\item {Proveniência:(It. \textunderscore ripresaglia\textunderscore )}
\end{itemize}
Acto, em que alguem tira o que lhe haviam tirado.
Desforra.
Vingança, tomada de alguém que offendeu o que se vinga.
Retaliação.
\section{Represar}
\begin{itemize}
\item {Grp. gram.:v. t.}
\end{itemize}
\begin{itemize}
\item {Proveniência:(De \textunderscore reprêso\textunderscore )}
\end{itemize}
Sustar o curso de; fazer parar: \textunderscore represar uma corrente\textunderscore .
Reprimir, conter: \textunderscore represar fúrias\textunderscore .
Soffrear.
Enclausurar; estorvar.
Fazer presa de, apoderar-se de.
\section{Represária}
\begin{itemize}
\item {Grp. gram.:f.}
\end{itemize}
\begin{itemize}
\item {Utilização:Ant.}
\end{itemize}
O mesmo que \textunderscore represália\textunderscore .
\section{Representação}
\begin{itemize}
\item {Grp. gram.:f.}
\end{itemize}
\begin{itemize}
\item {Proveniência:(Lat. \textunderscore repraesentatio\textunderscore )}
\end{itemize}
Acto ou effeito de representar.
Apresentação; exhibição.
Acto de exhibir em scena (drama, comédia, etc.).
Coisa que se representa.
Exposição de queixas ou pedidos, dirigida ao Govêrno ou a outras autoridades.
Ostentação inherente a um cargo.
Qualidade recommendável.
\section{Representador}
\begin{itemize}
\item {Grp. gram.:m.  e  adj.}
\end{itemize}
\begin{itemize}
\item {Proveniência:(Do lat. \textunderscore repraesentator\textunderscore )}
\end{itemize}
O que representa.
\section{Representanta}
\begin{itemize}
\item {Grp. gram.:f.  e  adj.}
\end{itemize}
\begin{itemize}
\item {Utilização:P. us.}
\end{itemize}
Aquella que representa. Cf. Garrett, \textunderscore Cancioneiro\textunderscore , I, 239.
(Flexão fem. de \textunderscore representante\textunderscore )
\section{Representante}
\begin{itemize}
\item {Grp. gram.:adj.}
\end{itemize}
\begin{itemize}
\item {Grp. gram.:M.  e  f.}
\end{itemize}
\begin{itemize}
\item {Proveniência:(Lat. \textunderscore repraesentans\textunderscore )}
\end{itemize}
Que representa.
Pessôa que representa outra ou outras.
Ministro plenipotenciário; embaixador.
\section{Representar}
\begin{itemize}
\item {Grp. gram.:v. t.}
\end{itemize}
\begin{itemize}
\item {Grp. gram.:V. i.}
\end{itemize}
\begin{itemize}
\item {Utilização:Bras. de Minas}
\end{itemize}
\begin{itemize}
\item {Proveniência:(Lat. \textunderscore repraesentare\textunderscore )}
\end{itemize}
Tornar presente.
Patentear.
Expor claramente.
Reproduzir por meio de imagem.
Figurar.
Sêr a imagem ou a reproducção de: \textunderscore aquelle quadro representa uma batalha\textunderscore .
Descrever.
Significar: \textunderscore êste livro representa longos annos de trabalho\textunderscore .
Expor por escripto ou por palavras.
Fazer sentir, objectar respeitosamente.
Têr procuração de: \textunderscore o advogado representa os clientes\textunderscore .
Sêr ministro ou embaixador de.
Estar no lugar de.
Desempenhar.
Pôr em acção no theatro: \textunderscore representar uma comédia\textunderscore .
Recitar.
Dirigir respeitosamente uma queixa.
Fazer petição.
Exercer funcções de actor.
Desempenhar qualquer papel.
Apresentar-se, apparecer: \textunderscore o José, logo de manhan, representou em minha casa\textunderscore .
\section{Representativo}
\begin{itemize}
\item {Grp. gram.:adj.}
\end{itemize}
Que representa.
Que é próprio para representar alguma coisa ou que tem por fim representá-la.
Que envolve representação.
Formado de representantes: \textunderscore assembleia representativa\textunderscore .
\section{Representável}
\begin{itemize}
\item {Grp. gram.:adj.}
\end{itemize}
Que se póde representar.
\section{Representear}
\begin{itemize}
\item {Grp. gram.:v. t.}
\end{itemize}
\begin{itemize}
\item {Proveniência:(De \textunderscore re...\textunderscore  + \textunderscore presentear\textunderscore )}
\end{itemize}
Presentear reciprocamente.
\section{Representório}
\begin{itemize}
\item {Grp. gram.:m.}
\end{itemize}
\begin{itemize}
\item {Utilização:Prov.}
\end{itemize}
\begin{itemize}
\item {Utilização:trasm.}
\end{itemize}
\begin{itemize}
\item {Proveniência:(De \textunderscore representar\textunderscore )}
\end{itemize}
Comédia popular; entremez.
\section{Reprêso}
\begin{itemize}
\item {Grp. gram.:adj.}
\end{itemize}
\begin{itemize}
\item {Proveniência:(Lat. \textunderscore reprehensus\textunderscore )}
\end{itemize}
Prêso novamente.
Represado.
\section{Repressão}
\begin{itemize}
\item {Grp. gram.:f.}
\end{itemize}
\begin{itemize}
\item {Proveniência:(Lat. \textunderscore repressio\textunderscore )}
\end{itemize}
Acto ou effeito de reprimir.
\section{Repressivo}
\begin{itemize}
\item {Grp. gram.:adj.}
\end{itemize}
\begin{itemize}
\item {Proveniência:(Do lat. \textunderscore repressus\textunderscore )}
\end{itemize}
Próprio para reprimir.
\section{Repressor}
\begin{itemize}
\item {Grp. gram.:m.  e  adj.}
\end{itemize}
\begin{itemize}
\item {Proveniência:(Lat. \textunderscore repressor\textunderscore )}
\end{itemize}
O que reprime.
\section{Repressório}
\begin{itemize}
\item {Grp. gram.:adj.}
\end{itemize}
O mesmo que \textunderscore repressor\textunderscore .
\section{Reprimenda}
\begin{itemize}
\item {Grp. gram.:f.}
\end{itemize}
\begin{itemize}
\item {Proveniência:(De \textunderscore reprimir\textunderscore )}
\end{itemize}
Admoèstação severa.
Crítica; censura:«\textunderscore as reprimendas tinham a brandura christan\textunderscore ». Camillo, \textunderscore Sereia\textunderscore , 67.
\section{Reprimidor}
\begin{itemize}
\item {Grp. gram.:m.  e  adj.}
\end{itemize}
\begin{itemize}
\item {Proveniência:(De \textunderscore reprimir\textunderscore )}
\end{itemize}
O mesmo que \textunderscore repressor\textunderscore .
\section{Reprimir}
\begin{itemize}
\item {Grp. gram.:v. t.}
\end{itemize}
\begin{itemize}
\item {Proveniência:(Lat. \textunderscore reprimere\textunderscore )}
\end{itemize}
Sustar a acção ou o movimento de.
Cohibir.
Reter.
Represar.
Disfarçar; occultar: \textunderscore reprimir queixas\textunderscore .
Prohibir.
Refrear.
Castigar.
Vexar.
Moderar.
\section{Reprimível}
\begin{itemize}
\item {Grp. gram.:adj.}
\end{itemize}
Que se póde reprimir.
\section{Reprincipiar}
\begin{itemize}
\item {Grp. gram.:v. t.}
\end{itemize}
\begin{itemize}
\item {Proveniência:(De \textunderscore re...\textunderscore  + \textunderscore principiar\textunderscore )}
\end{itemize}
Principiar de novo.
Recomeçar. Cf. Castilho, \textunderscore Geórgicas\textunderscore , 115.
\section{Réprobo}
\begin{itemize}
\item {Grp. gram.:adj.}
\end{itemize}
\begin{itemize}
\item {Grp. gram.:M.}
\end{itemize}
\begin{itemize}
\item {Proveniência:(Lat. \textunderscore reprobus\textunderscore )}
\end{itemize}
Malvado; condemnado; precito.
Indivíduo réprobo.
\section{Reprochar}
\begin{itemize}
\item {Grp. gram.:v. t.}
\end{itemize}
\begin{itemize}
\item {Proveniência:(De \textunderscore reproche\textunderscore )}
\end{itemize}
Lançar em rosto a alguém; censurar:«\textunderscore ...reprochou-lhe a ausência\textunderscore ». M. Assis, \textunderscore Quincas\textunderscore , 200.
\section{Reproche}
\begin{itemize}
\item {Grp. gram.:m.}
\end{itemize}
\begin{itemize}
\item {Utilização:Gal}
\end{itemize}
\begin{itemize}
\item {Proveniência:(Fr. \textunderscore reproche\textunderscore )}
\end{itemize}
Acto de reprochar; censura; reprimenda.
\section{Reprodução}
\begin{itemize}
\item {Grp. gram.:f.}
\end{itemize}
Acto ou efeito de reproduzir.
\section{Reproducção}
\begin{itemize}
\item {Grp. gram.:f.}
\end{itemize}
Acto ou effeito de reproduzir.
\section{Reproductibilidade}
\begin{itemize}
\item {Grp. gram.:f.}
\end{itemize}
Qualidade de reproductível.
\section{Reproductivamente}
\begin{itemize}
\item {Grp. gram.:adv.}
\end{itemize}
De modo reproductivo.
\section{Reproductível}
\begin{itemize}
\item {Grp. gram.:adj.}
\end{itemize}
O mesmo que \textunderscore reproduzível\textunderscore .
\section{Reproductivo}
\begin{itemize}
\item {Grp. gram.:adj.}
\end{itemize}
\begin{itemize}
\item {Proveniência:(De \textunderscore re...\textunderscore  + \textunderscore productivo\textunderscore )}
\end{itemize}
Que reproduz, ou que se reproduz.
\section{Reproductor}
\begin{itemize}
\item {Grp. gram.:adj.}
\end{itemize}
\begin{itemize}
\item {Grp. gram.:M.}
\end{itemize}
\begin{itemize}
\item {Proveniência:(De \textunderscore re...\textunderscore  + \textunderscore productor\textunderscore )}
\end{itemize}
Que reproduz; que serve para reproduzir: \textunderscore cavallos reproductores\textunderscore .
Aquelle que reproduz.
Animal, destinado a reproducção.
\section{Reproductriz}
\begin{itemize}
\item {Grp. gram.:adj.}
\end{itemize}
Diz-se da fêmea ou de outra entidade feminina, que reproduz ou que é destinada a agente de reproducção. Cf. Castilho, \textunderscore Fastos\textunderscore , II, 462.
(Flexão fem. de \textunderscore reproductor\textunderscore )
\section{Reprodutibilidade}
\begin{itemize}
\item {Grp. gram.:f.}
\end{itemize}
Qualidade de reprodutível.
\section{Reprodutivamente}
\begin{itemize}
\item {Grp. gram.:adv.}
\end{itemize}
De modo reprodutivo.
\section{Reprodutível}
\begin{itemize}
\item {Grp. gram.:adj.}
\end{itemize}
O mesmo que \textunderscore reproduzível\textunderscore .
\section{Reprodutivo}
\begin{itemize}
\item {Grp. gram.:adj.}
\end{itemize}
\begin{itemize}
\item {Proveniência:(De \textunderscore re...\textunderscore  + \textunderscore productivo\textunderscore )}
\end{itemize}
Que reproduz, ou que se reproduz.
\section{Reprodutor}
\begin{itemize}
\item {Grp. gram.:adj.}
\end{itemize}
\begin{itemize}
\item {Grp. gram.:M.}
\end{itemize}
\begin{itemize}
\item {Proveniência:(De \textunderscore re...\textunderscore  + \textunderscore produtor\textunderscore )}
\end{itemize}
Que reproduz; que serve para reproduzir: \textunderscore cavallos reprodutores\textunderscore .
Aquele que reproduz.
Animal, destinado a reprodução.
\section{Reprodutriz}
\begin{itemize}
\item {Grp. gram.:adj.}
\end{itemize}
Diz-se da fêmea ou de outra entidade feminina, que reproduz ou que é destinada a agente de reprodução. Cf. Castilho, \textunderscore Fastos\textunderscore , II, 462.
(Flexão fem. de \textunderscore reprodutor\textunderscore )
\section{Reproduzir}
\begin{itemize}
\item {Grp. gram.:v. t.}
\end{itemize}
\begin{itemize}
\item {Proveniência:(De \textunderscore re...\textunderscore  + \textunderscore produzir\textunderscore )}
\end{itemize}
Tornar a produzir; multiplicar (raça ou espécie).
Apresentar novamente.
Patentear outra vez.
Descrever: \textunderscore reproduzir uma batalha\textunderscore .
Imitar.
Copiar.
Commemorar.
\section{Reproduzível}
\begin{itemize}
\item {Grp. gram.:adj.}
\end{itemize}
Que se póde reproduzir.
\section{Reprofundar}
\begin{itemize}
\item {Grp. gram.:v. t.}
\end{itemize}
\begin{itemize}
\item {Grp. gram.:V. i.}
\end{itemize}
\begin{itemize}
\item {Proveniência:(De \textunderscore re...\textunderscore  + \textunderscore profundar\textunderscore )}
\end{itemize}
Profundar novamente.
Submergir-se, mergulhar.
\section{Reprometer}
\begin{itemize}
\item {Grp. gram.:v. t.}
\end{itemize}
\begin{itemize}
\item {Proveniência:(Do lat. \textunderscore repromittere\textunderscore )}
\end{itemize}
Tornar a prometer.
\section{Repromissão}
\begin{itemize}
\item {Grp. gram.:f.}
\end{itemize}
\begin{itemize}
\item {Proveniência:(Lat. \textunderscore repromissio\textunderscore )}
\end{itemize}
Acto de reprometer; promessa mútua.
\section{Reprova}
\begin{itemize}
\item {Grp. gram.:f.}
\end{itemize}
\begin{itemize}
\item {Proveniência:(Lat. \textunderscore reprovatio\textunderscore )}
\end{itemize}
Acto ou effeito de reprovar.
Censura; reproche.
Desprêzo.
\section{Reprovação}
\begin{itemize}
\item {Grp. gram.:f.}
\end{itemize}
\begin{itemize}
\item {Proveniência:(Lat. \textunderscore reprovatio\textunderscore )}
\end{itemize}
Acto ou effeito de reprovar.
Censura; reproche.
Desprêzo.
\section{Reprovadamente}
\begin{itemize}
\item {Grp. gram.:adv.}
\end{itemize}
\begin{itemize}
\item {Proveniência:(De \textunderscore reprovado\textunderscore )}
\end{itemize}
Com reprovação.
\section{Reprovado}
\begin{itemize}
\item {Grp. gram.:adj.}
\end{itemize}
\begin{itemize}
\item {Grp. gram.:M.}
\end{itemize}
Censurado; rejeitado.
Inhabilitado por exame.
Aquelle que num exame foi julgado inhabilitado.
\section{Reprovador}
\begin{itemize}
\item {Grp. gram.:m.  e  adj.}
\end{itemize}
\begin{itemize}
\item {Proveniência:(Lat. \textunderscore reprobator\textunderscore )}
\end{itemize}
O que reprova.
\section{Reprovar}
\begin{itemize}
\item {Grp. gram.:v. t.}
\end{itemize}
\begin{itemize}
\item {Proveniência:(Lat. \textunderscore reprobare\textunderscore )}
\end{itemize}
Não approvar, desapprovar.
Rejeitar, censurar com severidade.
Julgar inabilitado num exame.
Tornar réprobo.
Condemnar ás penas eternas.
\section{Reprovar}
\begin{itemize}
\item {Grp. gram.:v. t.}
\end{itemize}
\begin{itemize}
\item {Utilização:Ant.}
\end{itemize}
\begin{itemize}
\item {Proveniência:(De \textunderscore re...\textunderscore  + \textunderscore provar\textunderscore )}
\end{itemize}
Provar muitas vezes.
Provar perfeitamente. Cf. G. Vicente, I, 171.
\section{Reprovável}
\begin{itemize}
\item {Grp. gram.:adj.}
\end{itemize}
\begin{itemize}
\item {Proveniência:(Do lat. \textunderscore reprobabilis\textunderscore )}
\end{itemize}
Que se deve reprovar.
\section{Repruir}
\begin{itemize}
\item {Grp. gram.:v. t.}
\end{itemize}
\begin{itemize}
\item {Grp. gram.:V. i.}
\end{itemize}
\begin{itemize}
\item {Utilização:Fig.}
\end{itemize}
\begin{itemize}
\item {Proveniência:(De \textunderscore re...\textunderscore  + \textunderscore pruir\textunderscore )}
\end{itemize}
Produzir grande prurido em.
Excitar.
Têr cócegas.
Excitar-se.
\section{Reprurir}
\begin{itemize}
\item {Grp. gram.:v. t.  e   i.}
\end{itemize}
O mesmo que \textunderscore repruir\textunderscore .
\section{Reptação}
\begin{itemize}
\item {Grp. gram.:f.}
\end{itemize}
Acto ou effeito de reptar^1.
\section{Reptador}
\begin{itemize}
\item {Grp. gram.:m.  e  adj.}
\end{itemize}
Aquelle que repta.
\section{Reptamento}
\begin{itemize}
\item {Grp. gram.:m.}
\end{itemize}
\begin{itemize}
\item {Utilização:Des.}
\end{itemize}
O mesmo que \textunderscore reptação\textunderscore .
\section{Reptante}
\begin{itemize}
\item {Grp. gram.:m.  e  adj.}
\end{itemize}
O mesmo que \textunderscore reptador\textunderscore .
\section{Reptante}
\begin{itemize}
\item {Grp. gram.:adj.}
\end{itemize}
\begin{itemize}
\item {Utilização:Ant.}
\end{itemize}
\begin{itemize}
\item {Grp. gram.:M.}
\end{itemize}
\begin{itemize}
\item {Proveniência:(Lat. \textunderscore reptans\textunderscore )}
\end{itemize}
Que anda de rastos.
O mesmo que \textunderscore reptil\textunderscore .
\section{Reptar}
\begin{itemize}
\item {Grp. gram.:v. t.}
\end{itemize}
Accusar; provocar; desafiar.
Estar em opposição a.
(Talvez contr. do lat. \textunderscore reputare\textunderscore )
\section{Reptar}
\begin{itemize}
\item {Grp. gram.:v. i.}
\end{itemize}
\begin{itemize}
\item {Utilização:P. us.}
\end{itemize}
\begin{itemize}
\item {Proveniência:(Lat. \textunderscore reptare\textunderscore )}
\end{itemize}
Andar de rastos, arrastar-se.
Rojar-se pelo chão:«\textunderscore os que andavam, não erectos, como é próprio da família humana, senão reptando, ou arrastando-se no chão...\textunderscore »Latino, \textunderscore Vasco da Gama\textunderscore , I, 76.
\section{Repte}
\begin{itemize}
\item {Grp. gram.:m.}
\end{itemize}
\begin{itemize}
\item {Utilização:Ant.}
\end{itemize}
\begin{itemize}
\item {Proveniência:(De \textunderscore reptar\textunderscore ^1)}
\end{itemize}
O mesmo que \textunderscore repto\textunderscore . Cf. S. R. Viterbo, \textunderscore Elucidário\textunderscore .
\section{Reptil}
\begin{itemize}
\item {Grp. gram.:adj.}
\end{itemize}
\begin{itemize}
\item {Grp. gram.:M.}
\end{itemize}
\begin{itemize}
\item {Utilização:Fig.}
\end{itemize}
\begin{itemize}
\item {Proveniência:(Lat. \textunderscore reptilis\textunderscore )}
\end{itemize}
Que se arrasta, que rasteja.
Qualquer animal sem pés, que se move arrastando-se.
Qualquer animal de pés tão curtos, que parece andar de rastos.
Pessôa vil ou de instinctos baixos.
\section{Réptil}
\begin{itemize}
\item {Grp. gram.:adj.}
\end{itemize}
\begin{itemize}
\item {Utilização:Fig.}
\end{itemize}
\begin{itemize}
\item {Proveniência:(Lat. \textunderscore reptilis\textunderscore )}
\end{itemize}
Que se arrasta, que rasteja.
M.
Qualquer animal sem pés, que se move arrastando-se.
Qualquer animal de pés tão curtos, que parece andar de rastos.
Pessôa vil ou de instinctos baixos.
\section{Reptilário}
\begin{itemize}
\item {Grp. gram.:adj.}
\end{itemize}
\begin{itemize}
\item {Utilização:Neol.}
\end{itemize}
\begin{itemize}
\item {Proveniência:(De \textunderscore reptil\textunderscore )}
\end{itemize}
Relativo aos reptís, especialmente á éra geológica dos reptís.
\section{Reptília}
\begin{itemize}
\item {Grp. gram.:f.}
\end{itemize}
\begin{itemize}
\item {Utilização:Des.}
\end{itemize}
\begin{itemize}
\item {Proveniência:(Lat. \textunderscore reptilia\textunderscore , pl. de \textunderscore reptilis\textunderscore )}
\end{itemize}
O mesmo que \textunderscore reptil\textunderscore . Cf. Usque, 36 v.^o.
\section{Repto}
\begin{itemize}
\item {Grp. gram.:m.}
\end{itemize}
\begin{itemize}
\item {Proveniência:(De \textunderscore reptar\textunderscore ^1)}
\end{itemize}
O mesmo que \textunderscore reptação\textunderscore ; provocação, desafio.
\section{República}
\begin{itemize}
\item {Grp. gram.:f.}
\end{itemize}
\begin{itemize}
\item {Utilização:Restrict.}
\end{itemize}
\begin{itemize}
\item {Utilização:Fam.}
\end{itemize}
\begin{itemize}
\item {Utilização:Escol.}
\end{itemize}
\begin{itemize}
\item {Proveniência:(Lat. \textunderscore respublica\textunderscore )}
\end{itemize}
Negócios públicos.
Qualquer govêrno de um Estado.
Estado, governado por muitos indivíduos.
Govêrno, exercido por muitos indivíduos.
Fórma de govêrno, em que o supremo poder é exercido, durante tempo limitado, por um ou mais indivíduos eleitos pela Nação.
Communidade.
Agremiação sem chefe ou desordenada.
Conjunto de estudantes, que vivem em commum, na mesma casa.
\section{Republicanismo}
\begin{itemize}
\item {Grp. gram.:m.}
\end{itemize}
Qualidade do que é republicano.
Alarde de opiniões repúblicanas.
Fórma ou systema de govêrno repúblicano.
\section{Republicanizar}
\begin{itemize}
\item {Grp. gram.:v. t.}
\end{itemize}
Tornar republicano.
Converter em república.
Dar carácter republicano a.
\section{Republicano}
\begin{itemize}
\item {Grp. gram.:adj.}
\end{itemize}
\begin{itemize}
\item {Grp. gram.:M.}
\end{itemize}
Relativo á república: \textunderscore partido republicano\textunderscore .
Partidário do govêrno republicano.
Membro de uma república.
\section{Republicar}
\begin{itemize}
\item {Grp. gram.:v. t.}
\end{itemize}
\begin{itemize}
\item {Proveniência:(De \textunderscore re...\textunderscore  + \textunderscore publicar\textunderscore )}
\end{itemize}
Publicar novamente.
Reeditar.
\section{Republicida}
\begin{itemize}
\item {Grp. gram.:m.  e  f.}
\end{itemize}
\begin{itemize}
\item {Proveniência:(Do lat. \textunderscore respublica\textunderscore  + \textunderscore caedere\textunderscore )}
\end{itemize}
Pessôa, que destrói uma república.
Pessôa, que combate as instituições republicanas.
\section{Republicídio}
\begin{itemize}
\item {Grp. gram.:m.}
\end{itemize}
Acto de republicida.
\section{Repúblico}
\begin{itemize}
\item {Grp. gram.:adj.}
\end{itemize}
\begin{itemize}
\item {Grp. gram.:M.}
\end{itemize}
\begin{itemize}
\item {Proveniência:(De \textunderscore república\textunderscore )}
\end{itemize}
Relativo aos interesses da communidade dos cidadãos.
Republicano.
Aquelle que se interessa pelo bem geral.
Indivíduo republicano.
\section{Republiqueiro}
\begin{itemize}
\item {Grp. gram.:m.}
\end{itemize}
\begin{itemize}
\item {Utilização:Deprec.}
\end{itemize}
\begin{itemize}
\item {Proveniência:(De \textunderscore república\textunderscore )}
\end{itemize}
Parlapatão ou explorador, que alardeia republicanismo.
\section{Repudiação}
\begin{itemize}
\item {Grp. gram.:f.}
\end{itemize}
\begin{itemize}
\item {Proveniência:(Lat. \textunderscore repudiatio\textunderscore )}
\end{itemize}
O mesmo que \textunderscore repúdio\textunderscore .
\section{Repudiante}
\begin{itemize}
\item {Grp. gram.:m.  e  adj.}
\end{itemize}
\begin{itemize}
\item {Proveniência:(Lat. \textunderscore repudians\textunderscore )}
\end{itemize}
O que repudia.
\section{Repudiar}
\begin{itemize}
\item {Grp. gram.:v. t.}
\end{itemize}
\begin{itemize}
\item {Proveniência:(Lat. \textunderscore repudiare\textunderscore )}
\end{itemize}
Rejeitar, abandonar.
Desamparar.
Divorciar-se de (sua mulhér).
\section{Repudiável}
\begin{itemize}
\item {Grp. gram.:adj.}
\end{itemize}
Que se póde repudiar.
\section{Repúdio}
\begin{itemize}
\item {Grp. gram.:m.}
\end{itemize}
\begin{itemize}
\item {Proveniência:(Lat. \textunderscore repudium\textunderscore )}
\end{itemize}
Acto ou effeito de repudiar.
\section{Repugnador}
\begin{itemize}
\item {Grp. gram.:m.  e  adj.}
\end{itemize}
\begin{itemize}
\item {Proveniência:(Lat. \textunderscore repugnator\textunderscore )}
\end{itemize}
O que repugna.
\section{Repugnância}
\begin{itemize}
\item {Grp. gram.:f.}
\end{itemize}
\begin{itemize}
\item {Proveniência:(Lat. \textunderscore repugnantia\textunderscore )}
\end{itemize}
Qualidade do que é repugnante.
Escrúpulo em proceder de certo modo.
Aversão.
Obstáculo.
Incompatibilidade.
\section{Repugnante}
\begin{itemize}
\item {Grp. gram.:adj.}
\end{itemize}
\begin{itemize}
\item {Proveniência:(Lat. \textunderscore repugnans\textunderscore )}
\end{itemize}
Que repugna; repellente; nojento.
Que indigna.
Opposto á razão.
\section{Repugnar}
\begin{itemize}
\item {Grp. gram.:v. t.}
\end{itemize}
\begin{itemize}
\item {Grp. gram.:V. i.}
\end{itemize}
\begin{itemize}
\item {Proveniência:(Lat. \textunderscore repugnare\textunderscore )}
\end{itemize}
Reagir contra.
Recusar.
Causar aversão ou repulsão; sêr antipáthico.
Resistir, sêr contrário, oppor-se:«\textunderscore a esta jornada póde ir qualquer mulhér; e, se seu marido repugna, póde escolher outro para esse effeito.\textunderscore »M. Bernárdez.
\section{Repullulação}
\begin{itemize}
\item {Grp. gram.:f.}
\end{itemize}
Acto ou effeito de repullular.
\section{Repullular}
\begin{itemize}
\item {Grp. gram.:v. i.}
\end{itemize}
\begin{itemize}
\item {Proveniência:(Lat. \textunderscore repullulare\textunderscore )}
\end{itemize}
Pullular ou rebentar de novo.
Brotar em abundância.
Renascer.
Multiplicar-se.
\section{Repulsa}
\begin{itemize}
\item {Grp. gram.:f.}
\end{itemize}
O mesmo que \textunderscore repulsão\textunderscore .
\section{Repulsão}
\begin{itemize}
\item {Grp. gram.:f.}
\end{itemize}
\begin{itemize}
\item {Proveniência:(Lat. \textunderscore repulsio\textunderscore )}
\end{itemize}
Acto ou effeito de repellir.
Opposição; repugnância.
\section{Repulsar}
\begin{itemize}
\item {Grp. gram.:v. t.}
\end{itemize}
\begin{itemize}
\item {Proveniência:(Lat. \textunderscore repulsare\textunderscore )}
\end{itemize}
O mesmo que \textunderscore repellir\textunderscore .
Afastar.
Empurrar.
Recusar.
Oppor-se a.
Rejeitar.
\section{Repulsivo}
\begin{itemize}
\item {Grp. gram.:adj.}
\end{itemize}
\begin{itemize}
\item {Proveniência:(De \textunderscore repulso\textunderscore )}
\end{itemize}
Repellente; desagradável; que repulsa.
\section{Repulso}
\begin{itemize}
\item {Grp. gram.:adj.}
\end{itemize}
\begin{itemize}
\item {Grp. gram.:M.}
\end{itemize}
\begin{itemize}
\item {Proveniência:(Lat. \textunderscore repulsus\textunderscore )}
\end{itemize}
Repellido.
O mesmo que \textunderscore repulsão\textunderscore . Cf. Filinto, VII, 101.
\section{Repulsor}
\begin{itemize}
\item {Grp. gram.:adj.}
\end{itemize}
Que repulsa.
\section{Repululação}
\begin{itemize}
\item {Grp. gram.:f.}
\end{itemize}
Acto ou efeito de repulular.
\section{Repulular}
\begin{itemize}
\item {Grp. gram.:v. i.}
\end{itemize}
\begin{itemize}
\item {Proveniência:(Lat. \textunderscore repullulare\textunderscore )}
\end{itemize}
Pulular ou rebentar de novo.
Brotar em abundância.
Renascer.
Multiplicar-se.
\section{Repungente}
\begin{itemize}
\item {Grp. gram.:adj.}
\end{itemize}
\begin{itemize}
\item {Proveniência:(De \textunderscore re...\textunderscore  + \textunderscore pungente\textunderscore )}
\end{itemize}
Muito pungente; lancinante.
\section{Repunhante}
\begin{itemize}
\item {Grp. gram.:adj.}
\end{itemize}
\begin{itemize}
\item {Proveniência:(De \textunderscore repunhar\textunderscore )}
\end{itemize}
Que repunha, que é contrário:«\textunderscore milagres repunhantes á natureza\textunderscore ». Usque, 48.
\section{Repunhar}
\begin{itemize}
\item {Grp. gram.:v. i.}
\end{itemize}
\begin{itemize}
\item {Utilização:Ant.}
\end{itemize}
Oppor-se, sêr contrário a. Cf. Usque, \textunderscore passim\textunderscore ; Pant. de Aveiro, \textunderscore Itiner.\textunderscore , 166 v.^o, (2.^a ed.).
(Por \textunderscore repugnar\textunderscore )
\section{Repurgação}
\begin{itemize}
\item {Grp. gram.:f.}
\end{itemize}
\begin{itemize}
\item {Proveniência:(Lat. \textunderscore repurgatio\textunderscore )}
\end{itemize}
Acto de repurgar.
\section{Repurgar}
\begin{itemize}
\item {Grp. gram.:v. t.}
\end{itemize}
\begin{itemize}
\item {Proveniência:(Lat. \textunderscore repurgare\textunderscore )}
\end{itemize}
Tornar a purgar ou a limpar.
\section{Repurificação}
\begin{itemize}
\item {Grp. gram.:f.}
\end{itemize}
Acto ou effeito de repurificar.
\section{Repurificar}
\begin{itemize}
\item {Grp. gram.:v. t.}
\end{itemize}
\begin{itemize}
\item {Proveniência:(De \textunderscore re...\textunderscore  + \textunderscore purificar\textunderscore )}
\end{itemize}
Purificar de novo.
Purificar em alto grau.
Acrysolar. Cf. Eça, \textunderscore P. Amaro\textunderscore , 583.
\section{Reputação}
\begin{itemize}
\item {Grp. gram.:f.}
\end{itemize}
\begin{itemize}
\item {Proveniência:(Lat. \textunderscore reputatio\textunderscore )}
\end{itemize}
Acto ou effeito de reputar; opinião, conceito, fama.
Importância social.
\section{Reputar}
\begin{itemize}
\item {Grp. gram.:v. t.}
\end{itemize}
\begin{itemize}
\item {Proveniência:(Lat. \textunderscore reputare\textunderscore )}
\end{itemize}
Considerar, julgar.
Dar bôa fama a.
Avaliar.
\section{Repuxada}
\begin{itemize}
\item {Grp. gram.:f.}
\end{itemize}
\begin{itemize}
\item {Utilização:Prov.}
\end{itemize}
\begin{itemize}
\item {Utilização:trasm.}
\end{itemize}
\begin{itemize}
\item {Proveniência:(De \textunderscore repuxar\textunderscore )}
\end{itemize}
Reprehensão, reprimenda.
\section{Repuxador}
\begin{itemize}
\item {Grp. gram.:m.}
\end{itemize}
\begin{itemize}
\item {Proveniência:(De \textunderscore repuxar\textunderscore )}
\end{itemize}
Um dos officiaes, que servem subsidiariamente em ourivezarias.
\section{Repuxamento}
\begin{itemize}
\item {Grp. gram.:m.}
\end{itemize}
O mesmo que \textunderscore repuxão\textunderscore .
\section{Repuxão}
\begin{itemize}
\item {Grp. gram.:m.}
\end{itemize}
Acto de repuxar.
Puxão violento. Cf. Camillo, \textunderscore Estrêl. Fun.\textunderscore , 219.
\section{Repuxar}
\begin{itemize}
\item {Grp. gram.:v. t.}
\end{itemize}
\begin{itemize}
\item {Grp. gram.:V. i.}
\end{itemize}
\begin{itemize}
\item {Proveniência:(De \textunderscore re...\textunderscore  + \textunderscore puxar\textunderscore )}
\end{itemize}
Puxar com fôrça; esticar.
Retrahir, puxar para trás.
Refogar, apurar.
Reforçar com escoras.
Pôr encôsto a.
Saír em repuxo, borbotar, (falando-se de líquidos).
\section{Repuxo}
\begin{itemize}
\item {Grp. gram.:m.}
\end{itemize}
Acto ou effeito de repuxar.
Jacto.
Tubo ou construcção especial, por onde a água se eleva, saíndo em jacto contínuo.
Peça, que supporta o pé de um arco.
Botaréu.
Tira de coiro, a que se adapta um dedal, para impellir a agulha, com que se cosem as velas de navio.
Ferro, com que se embebem tarraxas na madeira.
Obra de supporte.
Acto de recuar.
\section{Requebrado}
\begin{itemize}
\item {Grp. gram.:adj.}
\end{itemize}
\begin{itemize}
\item {Utilização:Bot.}
\end{itemize}
Lânguido, amoroso: \textunderscore olhos requebrados\textunderscore .
Dobrado em fórma de cotovelo, (falando-se do pecíolo ou do folíolo).
\section{Requebrador}
\begin{itemize}
\item {Grp. gram.:m.  e  adj.}
\end{itemize}
O que requebra.
Namorador.
\section{Requebrar}
\begin{itemize}
\item {Grp. gram.:v. t.}
\end{itemize}
\begin{itemize}
\item {Utilização:Mús.}
\end{itemize}
\begin{itemize}
\item {Grp. gram.:V. p.}
\end{itemize}
\begin{itemize}
\item {Proveniência:(De \textunderscore re...\textunderscore  + \textunderscore quebrar\textunderscore )}
\end{itemize}
Mover languidamente, lascivamente.
Florear (o canto) com requebros.
Saracotear-se; derrengar-se.
\section{Requebro}
\begin{itemize}
\item {Grp. gram.:m.}
\end{itemize}
\begin{itemize}
\item {Utilização:Mús.}
\end{itemize}
Acto ou effeito de requebrar.
Inflexão lânguida da voz ou do corpo.
Movimento lascivo dos olhos.
Gesto amoroso.
Ornamento, feito de notas rápidas, para dar mais graça ao canto; o mesmo que \textunderscore trinado\textunderscore .
\section{Requeifa}
\begin{itemize}
\item {Grp. gram.:f.}
\end{itemize}
Pão fino de trigo, fabricado principalmente em Vàllongo.
O mesmo que \textunderscore regueifa\textunderscore .
\section{Requeijão}
\begin{itemize}
\item {Grp. gram.:m.}
\end{itemize}
\begin{itemize}
\item {Proveniência:(De \textunderscore re...\textunderscore  + \textunderscore queijo\textunderscore )}
\end{itemize}
Massa comestível, formada de nata, coalhada pela acção do calor.
\section{Requeijaria}
\begin{itemize}
\item {Grp. gram.:f.}
\end{itemize}
\begin{itemize}
\item {Utilização:Ant.}
\end{itemize}
\begin{itemize}
\item {Proveniência:(De \textunderscore requeijão\textunderscore )}
\end{itemize}
Fábrica de queijos e lacticínios.
\section{Requeijeiro}
\begin{itemize}
\item {Grp. gram.:m.}
\end{itemize}
\begin{itemize}
\item {Utilização:Ant.}
\end{itemize}
Fabricante de queijos e lacticínios.
(Cp. \textunderscore requeijão\textunderscore )
\section{Requeijiteira}
\begin{itemize}
\item {Grp. gram.:f.}
\end{itemize}
\begin{itemize}
\item {Utilização:Prov.}
\end{itemize}
\begin{itemize}
\item {Utilização:trasm.}
\end{itemize}
\begin{itemize}
\item {Proveniência:(De \textunderscore requeijitos\textunderscore )}
\end{itemize}
Mulhér mexeriqueira, que gosta de segredar e murmurar.
\section{Requeijitos}
\begin{itemize}
\item {Grp. gram.:m. pl.}
\end{itemize}
\begin{itemize}
\item {Utilização:Prov.}
\end{itemize}
\begin{itemize}
\item {Utilização:trasm.}
\end{itemize}
Mexericos, murmurações.
(Talvez por \textunderscore requeixitos\textunderscore , de \textunderscore queijar\textunderscore )
\section{Requeima}
\begin{itemize}
\item {Grp. gram.:f.}
\end{itemize}
\begin{itemize}
\item {Utilização:Zool.}
\end{itemize}
Peixe, o mesmo que \textunderscore requeime\textunderscore . Cf. P. Moraes., \textunderscore Zool. Elem.\textunderscore , 523 e 524.
\section{Requeimação}
\begin{itemize}
\item {Grp. gram.:f.}
\end{itemize}
Acto ou effeito de requeimar.
\section{Requeimar}
\begin{itemize}
\item {Grp. gram.:v. t.}
\end{itemize}
\begin{itemize}
\item {Grp. gram.:V. i.}
\end{itemize}
\begin{itemize}
\item {Grp. gram.:V. p.}
\end{itemize}
\begin{itemize}
\item {Utilização:Des.}
\end{itemize}
\begin{itemize}
\item {Proveniência:(De \textunderscore re...\textunderscore  + \textunderscore queimar\textunderscore )}
\end{itemize}
Queimar excessivamente.
Tornar negro pela acção do sol ou do fogo.
Crestar, tisnar.
Torrar.
Produzir ardor em.
Têr sabor acre.
Doêr-se, resentir-se.
\section{Requeime}
\begin{itemize}
\item {Grp. gram.:m.}
\end{itemize}
\begin{itemize}
\item {Proveniência:(De \textunderscore requeimar\textunderscore )}
\end{itemize}
O mesmo que \textunderscore queimo\textunderscore .
Peixe triglídio.
\section{Requeimo}
\begin{itemize}
\item {Grp. gram.:m.}
\end{itemize}
O mesmo que \textunderscore requeime\textunderscore .
Acto de requeimar.
\section{Requeixado}
\begin{itemize}
\item {Grp. gram.:adj.}
\end{itemize}
\begin{itemize}
\item {Utilização:Ant.}
\end{itemize}
Dizia-se de uma terra, localidade pequena ou despovoada.
(Talvez alter. de \textunderscore recachado\textunderscore )
\section{Requeixaria}
\begin{itemize}
\item {Grp. gram.:f.}
\end{itemize}
\begin{itemize}
\item {Utilização:Ant.}
\end{itemize}
O mesmo que \textunderscore requeijaria\textunderscore .
(Cp. \textunderscore queixo\textunderscore ^2)
\section{Requentar}
\begin{itemize}
\item {Grp. gram.:v. t.}
\end{itemize}
\begin{itemize}
\item {Grp. gram.:V. p.}
\end{itemize}
\begin{itemize}
\item {Proveniência:(De \textunderscore re...\textunderscore  + \textunderscore quente\textunderscore )}
\end{itemize}
Aquecer de novo.
Sujeitar por muito tempo á acção do calor.
Tomar fumo ou mau sabor, (falando-se de iguarias que se demoram ao lume).
\section{Requerá}
\begin{itemize}
\item {Grp. gram.:m.}
\end{itemize}
\begin{itemize}
\item {Utilização:Ant.}
\end{itemize}
Tecido, espécie de mantaz.
\section{Requeredor}
\begin{itemize}
\item {Grp. gram.:m.  e  adj.}
\end{itemize}
\begin{itemize}
\item {Proveniência:(De \textunderscore requerer\textunderscore )}
\end{itemize}
O que requere.
\section{Requerente}
\begin{itemize}
\item {Grp. gram.:m.  e  adj.}
\end{itemize}
\begin{itemize}
\item {Proveniência:(De \textunderscore requerer\textunderscore )}
\end{itemize}
O que requere.
\section{Reque-reque}
\begin{itemize}
\item {Grp. gram.:m.}
\end{itemize}
\begin{itemize}
\item {Proveniência:(T. onom.)}
\end{itemize}
Instrumento de fricção, usada por pretos.
\section{Requerer}
\begin{itemize}
\item {Grp. gram.:v. t.}
\end{itemize}
\begin{itemize}
\item {Utilização:Pop.}
\end{itemize}
\begin{itemize}
\item {Proveniência:(Do lat. \textunderscore requirere\textunderscore )}
\end{itemize}
Dirigir petição a autoridades ou a pessôas, a quem compete especialmente a faculdade de conceder o que se pede: \textunderscore requerer aposentação\textunderscore .
Pedir judicialmente: \textunderscore requerer divórcio\textunderscore .
Pedir.
Pretender.
Exigir.
Têr necessidade de.
Sêr digno de.
Requestar.
Solicitar a presença de.
Consultar (almas do outro mundo):«\textunderscore já lá têm ido padres requerer a alma...\textunderscore »Camillo.
\section{Requerimento}
\begin{itemize}
\item {Grp. gram.:m.}
\end{itemize}
Acto ou effeito de requerer.
Petição por escrito, segundo as fórmas legaes.
Petição.
\section{Requesta}
\begin{itemize}
\item {Grp. gram.:f.}
\end{itemize}
\begin{itemize}
\item {Utilização:Ant.}
\end{itemize}
\begin{itemize}
\item {Proveniência:(Lat. \textunderscore requesta\textunderscore )}
\end{itemize}
Contenda, briga; combate.
Petição.
\section{Requestar}
\begin{itemize}
\item {Grp. gram.:v. t.}
\end{itemize}
\begin{itemize}
\item {Proveniência:(De \textunderscore requesta\textunderscore )}
\end{itemize}
Solicitar.
Empenhar-se em possuír.
Supplicar.
Pretender o amôr ou bôas graças de (mulhér).
Galantear.
Sujeitar-se a.
\section{Requesto}
\begin{itemize}
\item {Grp. gram.:m.}
\end{itemize}
\begin{itemize}
\item {Utilização:Ant.}
\end{itemize}
O mesmo que \textunderscore requesta\textunderscore . Cf. S. R. Viterbo, \textunderscore Elucidário\textunderscore .
\section{Requezitos}
\begin{itemize}
\item {Grp. gram.:m. pl.}
\end{itemize}
\begin{itemize}
\item {Utilização:Prov.}
\end{itemize}
O mesmo que \textunderscore requeijitos\textunderscore .
\section{Réquie}
\begin{itemize}
\item {Grp. gram.:f.}
\end{itemize}
\begin{itemize}
\item {Utilização:Ant.}
\end{itemize}
\begin{itemize}
\item {Proveniência:(Lat. \textunderscore requies\textunderscore )}
\end{itemize}
O mesmo que \textunderscore descanso\textunderscore .
\section{Requiênia}
\begin{itemize}
\item {Grp. gram.:f.}
\end{itemize}
Gênero de plantas leguminosas.
\section{Requieto}
\begin{itemize}
\item {Grp. gram.:adj.}
\end{itemize}
\begin{itemize}
\item {Proveniência:(Lat. \textunderscore requietus\textunderscore )}
\end{itemize}
Muito quieto, sossegado.
\section{Requietório}
\begin{itemize}
\item {Grp. gram.:m.}
\end{itemize}
\begin{itemize}
\item {Proveniência:(Lat. \textunderscore requietorium\textunderscore )}
\end{itemize}
Nome, que os Romanos davam ao sepulcro, como lugar de descanso.
\section{Requietude}
\begin{itemize}
\item {Grp. gram.:f.}
\end{itemize}
Estado de requieto.
\section{Requife}
\begin{itemize}
\item {Grp. gram.:m.}
\end{itemize}
Fita estreita de passamanaria ou cordões de bicos, para enfeitar ou debruar. Cf. Camillo, \textunderscore Brasileira\textunderscore , 104.
\section{Requim}
\begin{itemize}
\item {Grp. gram.:m.}
\end{itemize}
Espécie de licor indiano.
\section{Requinho}
\begin{itemize}
\item {Grp. gram.:m.}
\end{itemize}
\begin{itemize}
\item {Utilização:Ant.}
\end{itemize}
(?):«\textunderscore ...poderás sêr um requinho ou chasquete de primeira tonsura...\textunderscore »\textunderscore Anat. Joc.\textunderscore , 37.
\section{Requinta}
\begin{itemize}
\item {Grp. gram.:f.}
\end{itemize}
\begin{itemize}
\item {Proveniência:(De \textunderscore requintar\textunderscore )}
\end{itemize}
Espécie de clarinete, de sons agudos.
Viola ou guitarra, de sons agudos, mais pequenas que as ordinárias.
\section{Requintar}
\begin{itemize}
\item {Grp. gram.:v. t.}
\end{itemize}
\begin{itemize}
\item {Utilização:Prov.}
\end{itemize}
\begin{itemize}
\item {Utilização:trasm.}
\end{itemize}
\begin{itemize}
\item {Grp. gram.:V. i.}
\end{itemize}
\begin{itemize}
\item {Grp. gram.:V. p.}
\end{itemize}
\begin{itemize}
\item {Proveniência:(De \textunderscore re...\textunderscore  + \textunderscore quinta\textunderscore ^2)}
\end{itemize}
Levar á quinta essência ou ao mais alto grau.
Apurar muito.
Tornar esquisito.
Elevar, sublimar.
Apertar muito, esticar (corda).
Replicar, recalcitrar. Cf. \textunderscore Viriato Trág.\textunderscore , XIV, 69.
Têr affectação; exaggerar-se; elevar-se muito.
\section{Requinte}
\begin{itemize}
\item {Grp. gram.:m.}
\end{itemize}
Acto ou effeito de requintar.
\section{Requirir}
\begin{itemize}
\item {Grp. gram.:m.}
\end{itemize}
\begin{itemize}
\item {Utilização:Ant.}
\end{itemize}
O mesmo que \textunderscore requerer\textunderscore . Cf. Moraes, \textunderscore Diccion.\textunderscore 
\section{Requisição}
\begin{itemize}
\item {Grp. gram.:f.}
\end{itemize}
\begin{itemize}
\item {Proveniência:(Lat. \textunderscore requisitio\textunderscore )}
\end{itemize}
Acto ou effeito de requisitar.
\section{Requisir}
\textunderscore v. t.\textunderscore  (e der.)
O mesmo que \textunderscore requisitar\textunderscore , etc.
Rogar com instância.
\section{Requisitar}
\begin{itemize}
\item {Grp. gram.:v. t.}
\end{itemize}
\begin{itemize}
\item {Proveniência:(Do lat. \textunderscore requisitus\textunderscore )}
\end{itemize}
Solicitar legalmente; reclamar; requerer; exigir.
\section{Requisito}
\begin{itemize}
\item {Grp. gram.:m.}
\end{itemize}
\begin{itemize}
\item {Proveniência:(Lat. \textunderscore requisitum\textunderscore )}
\end{itemize}
Condição, que se exige para certo fim.
Condição.
Exigência legal, necessária para certos effeitos.
\section{Requisito}
\begin{itemize}
\item {Grp. gram.:adj.}
\end{itemize}
\begin{itemize}
\item {Proveniência:(Lat. \textunderscore requisitus\textunderscore )}
\end{itemize}
Que se requisitou; que se requereu. Cf. Filinto, \textunderscore D. Man.\textunderscore , I, 21.
\section{Requisitório}
\begin{itemize}
\item {Grp. gram.:adj.}
\end{itemize}
\begin{itemize}
\item {Grp. gram.:M.}
\end{itemize}
\begin{itemize}
\item {Utilização:Jur.}
\end{itemize}
\begin{itemize}
\item {Proveniência:(De \textunderscore requisito\textunderscore ^2)}
\end{itemize}
O mesmo que \textunderscore precatório\textunderscore .
Exposição dos motivos, pelos quaes o representante do Ministério Público accusa alguém judicialmente.
Requisição escrita, feita pelo representante do Ministério Público.
\section{Rêr}
\begin{itemize}
\item {Grp. gram.:v. t.}
\end{itemize}
Rapar (o sal) na peça da salina e juntá-lo com o rôdo.
(Contr. de \textunderscore raêr\textunderscore )
\section{Rês}
\begin{itemize}
\item {Grp. gram.:f.}
\end{itemize}
\begin{itemize}
\item {Utilização:deprec.}
\end{itemize}
\begin{itemize}
\item {Utilização:Pop.}
\end{itemize}
\begin{itemize}
\item {Proveniência:(Do ár. \textunderscore ras\textunderscore )}
\end{itemize}
Qualquer quadrúpede, que serve para alimento do homem.
Pessôa de mau carácter.
\section{Rés}
\begin{itemize}
\item {Grp. gram.:adj.}
\end{itemize}
\begin{itemize}
\item {Grp. gram.:Adv.}
\end{itemize}
\begin{itemize}
\item {Proveniência:(Do lat. \textunderscore rasus\textunderscore )}
\end{itemize}
Raso, rente.
Cerce: \textunderscore passaram rés do muro\textunderscore .
\section{Res...}
\begin{itemize}
\item {Grp. gram.:pref.}
\end{itemize}
O mesmo que \textunderscore re...\textunderscore 
\section{Resaber}
\begin{itemize}
\item {fónica:sa}
\end{itemize}
\begin{itemize}
\item {Grp. gram.:v. t.}
\end{itemize}
\begin{itemize}
\item {Grp. gram.:V. i.}
\end{itemize}
\begin{itemize}
\item {Proveniência:(De \textunderscore re...\textunderscore  + \textunderscore saber\textunderscore )}
\end{itemize}
Saber bem, perfeitamente.
Têr sabor muito pronunciado.
Têr sabor que faz lembrar outro.
\section{Resabiado}
\begin{itemize}
\item {fónica:sá}
\end{itemize}
\begin{itemize}
\item {Grp. gram.:adj.}
\end{itemize}
Que resabia; espantadiço, desconfiado: \textunderscore cavallo resabiado\textunderscore .
\section{Resabiar}
\begin{itemize}
\item {fónica:sá}
\end{itemize}
\begin{itemize}
\item {Grp. gram.:v. i.}
\end{itemize}
\begin{itemize}
\item {Utilização:Fig.}
\end{itemize}
\begin{itemize}
\item {Proveniência:(De \textunderscore resábio\textunderscore )}
\end{itemize}
Tomar resaibo.
Melindrar-se, resentir-se.
\section{Resabido}
\begin{itemize}
\item {fónica:sa}
\end{itemize}
\begin{itemize}
\item {Grp. gram.:adj.}
\end{itemize}
Que sabe muito, que é erudito; experimentado.
\section{Resábio}
\begin{itemize}
\item {fónica:sá}
\end{itemize}
\begin{itemize}
\item {Grp. gram.:m.}
\end{itemize}
(Metáth. de \textunderscore resaibo\textunderscore . Cf. Filinto, \textunderscore D. Man.\textunderscore , I, 90)
\section{Resaborear}
\begin{itemize}
\item {fónica:sa}
\end{itemize}
\begin{itemize}
\item {Grp. gram.:v. t.}
\end{itemize}
\begin{itemize}
\item {Proveniência:(De \textunderscore re...\textunderscore  + \textunderscore saborear\textunderscore )}
\end{itemize}
Saborear muito; apreciar em alto grau. Cf. Eça, \textunderscore P. Amaro\textunderscore , 420.
\section{Resaca}
\begin{itemize}
\item {fónica:sá}
\end{itemize}
\begin{itemize}
\item {Grp. gram.:f.}
\end{itemize}
\begin{itemize}
\item {Utilização:Ant.}
\end{itemize}
\begin{itemize}
\item {Utilização:Fig.}
\end{itemize}
\begin{itemize}
\item {Proveniência:(De \textunderscore re...\textunderscore  + \textunderscore sacar\textunderscore )}
\end{itemize}
Movimento, feito pelas ondas, quando se desviam da praia.
Pôrto, formado pela preamar.
Fluxo e refluxo.
O mesmo que \textunderscore rètaguarda\textunderscore . Cf. \textunderscore Peregrinação\textunderscore , CL.
Volubilidade.
\section{Resacada}
\begin{itemize}
\item {fónica:sa}
\end{itemize}
\begin{itemize}
\item {Grp. gram.:f.}
\end{itemize}
\begin{itemize}
\item {Utilização:T. de bandeirantes}
\end{itemize}
Grande resaca, á beira dos rios.
\section{Resacar}
\begin{itemize}
\item {fónica:sa}
\end{itemize}
\begin{itemize}
\item {Grp. gram.:v. t.}
\end{itemize}
\begin{itemize}
\item {Proveniência:(De \textunderscore re...\textunderscore  + \textunderscore sacar\textunderscore )}
\end{itemize}
Fazer o resaque de (letra de câmbio).
\section{Resaibo}
\begin{itemize}
\item {fónica:sai}
\end{itemize}
\begin{itemize}
\item {Grp. gram.:m.}
\end{itemize}
\begin{itemize}
\item {Utilização:Fig.}
\end{itemize}
\begin{itemize}
\item {Proveniência:(De \textunderscore re...\textunderscore  + \textunderscore saibo\textunderscore )}
\end{itemize}
Mau saibo.
Ranço.
Sabor, resultante de uma substância que adheriu ao vaso por onde se bebe ou come.
Resentimento.
Indício.
\section{Resaio}
\begin{itemize}
\item {fónica:sai}
\end{itemize}
\begin{itemize}
\item {Grp. gram.:m.}
\end{itemize}
\begin{itemize}
\item {Proveniência:(De \textunderscore resair\textunderscore )}
\end{itemize}
Terreiro, á beira de uma casa; rossío. Cf. Camillo, \textunderscore Doze Casam.\textunderscore , 14; S. R. Viterbo, \textunderscore Elucidário\textunderscore .
\section{Resair}
\begin{itemize}
\item {fónica:sa}
\end{itemize}
\begin{itemize}
\item {Grp. gram.:v. i.}
\end{itemize}
\begin{itemize}
\item {Proveniência:(De \textunderscore re...\textunderscore  + \textunderscore saír\textunderscore )}
\end{itemize}
Sair de novo.
Saír acima.
Resaltar.
Sobresaír.
Mostrar-se saliente; avultar.
\section{Resalgar}
\begin{itemize}
\item {Grp. gram.:m.}
\end{itemize}
\begin{itemize}
\item {Utilização:Prov.}
\end{itemize}
\begin{itemize}
\item {Utilização:trasm.}
\end{itemize}
Pequena lagarta, que rói a caruma.
(Cp. \textunderscore rosalgar\textunderscore )
\section{Resalgário}
\begin{itemize}
\item {Grp. gram.:m.}
\end{itemize}
O mesmo que \textunderscore resalgar\textunderscore .
\section{Resaltar}
\begin{itemize}
\item {fónica:sal}
\end{itemize}
\begin{itemize}
\item {Grp. gram.:v. t.}
\end{itemize}
\begin{itemize}
\item {Grp. gram.:V. i.}
\end{itemize}
\begin{itemize}
\item {Proveniência:(De \textunderscore re...\textunderscore  + \textunderscore saltar\textunderscore )}
\end{itemize}
Tornar saliente; dar relêvo a.
Altear.
Dar muitos saltos.
Sobresair, elevar-se; resair.
\section{Resalte}
\begin{itemize}
\item {fónica:sal}
\end{itemize}
\begin{itemize}
\item {Grp. gram.:m.}
\end{itemize}
Acto ou effeito de resaltar.
Resalto; saliência. Cf. Alv. Mendes, \textunderscore Discursos\textunderscore , 207.
\section{Resaltear}
\begin{itemize}
\item {fónica:sal}
\end{itemize}
\begin{itemize}
\item {Grp. gram.:v. t.}
\end{itemize}
\begin{itemize}
\item {Proveniência:(De \textunderscore re...\textunderscore  + \textunderscore saltear\textunderscore )}
\end{itemize}
Saltear de novo.
\section{Resalto}
\begin{itemize}
\item {fónica:sal}
\end{itemize}
\begin{itemize}
\item {Grp. gram.:m.}
\end{itemize}
Acto ou effeito de resaltar.
Relêvo; saliência.
\section{Resalva}
\begin{itemize}
\item {fónica:sal}
\end{itemize}
\begin{itemize}
\item {Grp. gram.:f.}
\end{itemize}
\begin{itemize}
\item {Proveniência:(De \textunderscore resalvar\textunderscore )}
\end{itemize}
Certidão, donde consta que um indivíduo se isentou do serviço militar.
Documento, para segurança de alguém.
Excepção.
Cláusula.
Nota, para corrigir o que se escreveu; errata.
\section{Resalvar}
\begin{itemize}
\item {fónica:sal}
\end{itemize}
\begin{itemize}
\item {Grp. gram.:v. t.}
\end{itemize}
\begin{itemize}
\item {Proveniência:(Lat. \textunderscore resalvare\textunderscore )}
\end{itemize}
Dar resalva a.
Segurar com resalva.
Fazer resalva em.
Exceptuar; eximir.
Acautelar; pôr a salvo.
\section{Resanfoninar}
\begin{itemize}
\item {fónica:san}
\end{itemize}
\begin{itemize}
\item {Grp. gram.:v. i.}
\end{itemize}
\begin{itemize}
\item {Utilização:Ant.}
\end{itemize}
\begin{itemize}
\item {Proveniência:(De \textunderscore re...\textunderscore  + \textunderscore sanfoninar\textunderscore )}
\end{itemize}
Repetir impertinências, zombando:«\textunderscore ...quereis resanfoninar com a minha dôr.\textunderscore »\textunderscore Eufrosina\textunderscore , 24.
\section{Resangrar}
\begin{itemize}
\item {fónica:san}
\end{itemize}
\begin{itemize}
\item {Grp. gram.:v. t.}
\end{itemize}
\begin{itemize}
\item {Proveniência:(De \textunderscore re...\textunderscore  + \textunderscore sangrar\textunderscore )}
\end{itemize}
Sangrar de novo, tirar muito sangue a. Cf. Alv. Mendes, \textunderscore Discursos\textunderscore , 266.
\section{Resaque}
\begin{itemize}
\item {fónica:sá}
\end{itemize}
\begin{itemize}
\item {Grp. gram.:m.}
\end{itemize}
\begin{itemize}
\item {Utilização:Jur.}
\end{itemize}
\begin{itemize}
\item {Utilização:Jur.}
\end{itemize}
\begin{itemize}
\item {Utilização:Des.}
\end{itemize}
\begin{itemize}
\item {Proveniência:(De \textunderscore re...\textunderscore  + \textunderscore saque\textunderscore )}
\end{itemize}
Saque de uma nova letra de câmbio.
Segunda letra de câmbio, pela qual o portador se embolsa sôbre o sacador ou indossador de outra letra protestada, do principal desta. Cf. F. Borges, \textunderscore Diccion. Jur.\textunderscore 
\section{Resaque}
\begin{itemize}
\item {fónica:sá}
\end{itemize}
\begin{itemize}
\item {Grp. gram.:m.}
\end{itemize}
\begin{itemize}
\item {Utilização:Prov.}
\end{itemize}
O mesmo que \textunderscore resaca\textunderscore .
\section{Resarcimento}
\begin{itemize}
\item {fónica:sar}
\end{itemize}
\begin{itemize}
\item {Grp. gram.:m.}
\end{itemize}
Acto ou effeito de resarcir.
\section{Resarcir}
\begin{itemize}
\item {fónica:sar}
\end{itemize}
\begin{itemize}
\item {Grp. gram.:v. t.}
\end{itemize}
\begin{itemize}
\item {Proveniência:(Lat. \textunderscore resarcire\textunderscore )}
\end{itemize}
Compensar, indemnizar; refazer; melhorar.
\section{Resaudação}
\begin{itemize}
\item {fónica:sa}
\end{itemize}
\begin{itemize}
\item {Grp. gram.:f.}
\end{itemize}
Acto ou effeito de saudar.
\section{Resaudar}
\begin{itemize}
\item {fónica:sa}
\end{itemize}
\begin{itemize}
\item {Grp. gram.:v. t.}
\end{itemize}
\begin{itemize}
\item {Grp. gram.:V. i.}
\end{itemize}
\begin{itemize}
\item {Proveniência:(Do lat. \textunderscore resalutare\textunderscore )}
\end{itemize}
Tornar a saudar.
Saudar mutuamente.
Corresponder á saudação de alguém.
\section{Resbordo}
\begin{itemize}
\item {Grp. gram.:m.}
\end{itemize}
\begin{itemize}
\item {Utilização:Náut.}
\end{itemize}
\begin{itemize}
\item {Utilização:Bras}
\end{itemize}
\begin{itemize}
\item {Proveniência:(De \textunderscore res...\textunderscore  + \textunderscore bordo\textunderscore )}
\end{itemize}
Conjunto das pranchas, que formam o segundo sôlho do navio.
Abertura na amurada, para dar lugar á boca do canhão. Cf. M. de Aguiar, \textunderscore Diccion. da Marinha\textunderscore .
\section{Resbunar}
\begin{itemize}
\item {Grp. gram.:v. i.}
\end{itemize}
\begin{itemize}
\item {Utilização:Prov.}
\end{itemize}
O mesmo que \textunderscore ronronar\textunderscore :«\textunderscore ...uma gata malteza que lhe resbunava no regaço...\textunderscore »Camillo, \textunderscore Brasileira\textunderscore , 7.
(Cp. cast. \textunderscore resbuñar\textunderscore )
\section{Resbutos}
\begin{itemize}
\item {Grp. gram.:m. pl.}
\end{itemize}
O mesmo ou melhor que \textunderscore reisbutos\textunderscore .
\section{Rescaldamento}
\begin{itemize}
\item {Grp. gram.:m.}
\end{itemize}
Acto ou effeito de rescaldar.
\section{Rescaldar}
\begin{itemize}
\item {Grp. gram.:v. t.}
\end{itemize}
\begin{itemize}
\item {Proveniência:(De \textunderscore re...\textunderscore  + \textunderscore escaldar\textunderscore )}
\end{itemize}
Tornar a escaldar; escaldar muito.
\section{Rescaldeiro}
\begin{itemize}
\item {Grp. gram.:m.}
\end{itemize}
\begin{itemize}
\item {Proveniência:(De \textunderscore rescaldo\textunderscore )}
\end{itemize}
Prato rescaldado para conservar quentes certos môlhos ou iguarias.
Esquentador.
Braseiro.
\section{Rescaldo}
\begin{itemize}
\item {Grp. gram.:m.}
\end{itemize}
\begin{itemize}
\item {Proveniência:(De \textunderscore rescaldar\textunderscore )}
\end{itemize}
Calor, reflectido por uma fornalha ou por um incêndio.
Cinza, que contém brasas.
Acto de molhar cinzas ou brasido de um incêndio recente, para que êste se não renove.
Cinzas expulsas pelos vulcões.
Vaso, que se enche de água quente e sôbre o qual se collocam os pratos de comida, para que esta não arrefeça.
O mesmo que \textunderscore rescaldeiro\textunderscore .
\section{Rescambo}
\begin{itemize}
\item {Grp. gram.:m.}
\end{itemize}
\begin{itemize}
\item {Utilização:Ant.}
\end{itemize}
O mesmo que \textunderscore recâmbio\textunderscore . Cf. S. R. Viterbo, \textunderscore Elucidário\textunderscore .
\section{Rescender}
\textunderscore v. t.\textunderscore  e \textunderscore i.\textunderscore  (e der.)
O mesmo ou melhor que \textunderscore recender\textunderscore , etc.
\section{Rescindência}
\begin{itemize}
\item {Grp. gram.:f.}
\end{itemize}
\begin{itemize}
\item {Utilização:Des.}
\end{itemize}
O mesmo que \textunderscore rescindimento\textunderscore . Cf. F. Alex. Lobo, III, 360.
\section{Rescindimento}
\begin{itemize}
\item {Grp. gram.:m.}
\end{itemize}
Acto ou effeito de rescindir.
\section{Rescindir}
\begin{itemize}
\item {Grp. gram.:v. t.}
\end{itemize}
\begin{itemize}
\item {Proveniência:(Lat. \textunderscore rescindere\textunderscore )}
\end{itemize}
Quebrar, cortar.
Dissolver.
Invalidar; tornar nullo: \textunderscore rescindir um contrato\textunderscore .
\section{Rescisão}
\begin{itemize}
\item {Grp. gram.:f.}
\end{itemize}
\begin{itemize}
\item {Proveniência:(Lat. \textunderscore rescisio\textunderscore )}
\end{itemize}
O mesmo que \textunderscore rescindimento\textunderscore .
\section{Rescisor}
\begin{itemize}
\item {Grp. gram.:adj.}
\end{itemize}
O mesmo que \textunderscore rescisório\textunderscore .
\section{Rescisório}
\begin{itemize}
\item {Grp. gram.:adj.}
\end{itemize}
\begin{itemize}
\item {Proveniência:(Do lat. \textunderscore rescisus\textunderscore )}
\end{itemize}
Que rescinde; próprio para rescindir.
\section{Rescrever}
\begin{itemize}
\item {Grp. gram.:v. t.}
\end{itemize}
\begin{itemize}
\item {Proveniência:(Do lat. \textunderscore rescribere\textunderscore )}
\end{itemize}
Escrever de novo.
\section{Rescrição}
\begin{itemize}
\item {Grp. gram.:f.}
\end{itemize}
\begin{itemize}
\item {Proveniência:(Lat. \textunderscore rescriptio\textunderscore )}
\end{itemize}
Ordem para se pagar uma quantia.
O mesmo que \textunderscore cheque\textunderscore ^1.
\section{Rescripção}
\begin{itemize}
\item {Grp. gram.:f.}
\end{itemize}
\begin{itemize}
\item {Proveniência:(Lat. \textunderscore rescriptio\textunderscore )}
\end{itemize}
Ordem para se pagar uma quantia.
O mesmo que \textunderscore cheque\textunderscore ^1.
\section{Rescripto}
\begin{itemize}
\item {Grp. gram.:m.}
\end{itemize}
\begin{itemize}
\item {Proveniência:(Lat. \textunderscore rescriptus\textunderscore )}
\end{itemize}
Decisão pontifícia, em assumptos theológicos.
Resolução régia por escrito.
\section{Rescrito}
\begin{itemize}
\item {Grp. gram.:m.}
\end{itemize}
\begin{itemize}
\item {Proveniência:(Lat. \textunderscore rescriptus\textunderscore )}
\end{itemize}
Decisão pontifícia, em assuntos teológicos.
Resolução régia por escrito.
\section{Rés-do-chão}
\begin{itemize}
\item {Grp. gram.:m.}
\end{itemize}
Pavimento de uma casa, ao nível do solo ou da rua.
Andar térreo.
\section{Resecação}
\begin{itemize}
\item {fónica:se}
\end{itemize}
\begin{itemize}
\item {Grp. gram.:f.}
\end{itemize}
Acto ou effeito de reseccar.
\section{Resecar}
\begin{itemize}
\item {fónica:se}
\end{itemize}
\begin{itemize}
\item {Grp. gram.:v. t.}
\end{itemize}
\begin{itemize}
\item {Proveniência:(De \textunderscore re...\textunderscore  + \textunderscore secar\textunderscore )}
\end{itemize}
Tornar a secar.
Secar bem.
Sujeitar á evaporação.
\section{Resecção}
\begin{itemize}
\item {fónica:sé}
\end{itemize}
\begin{itemize}
\item {Grp. gram.:f.}
\end{itemize}
\begin{itemize}
\item {Proveniência:(Lat. \textunderscore resectio\textunderscore )}
\end{itemize}
Operação cirúrgica, que consiste em cortar uma parte, mais ou menos extensa, de um órgão.
\section{Resêco}
\begin{itemize}
\item {fónica:sê}
\end{itemize}
\begin{itemize}
\item {Grp. gram.:adj.}
\end{itemize}
\begin{itemize}
\item {Proveniência:(De \textunderscore re...\textunderscore  + \textunderscore sêco\textunderscore )}
\end{itemize}
Muito sêco.
\section{Reseda}
\begin{itemize}
\item {fónica:zê}
\end{itemize}
\begin{itemize}
\item {Grp. gram.:f.}
\end{itemize}
\begin{itemize}
\item {Proveniência:(Lat. \textunderscore reseda\textunderscore )}
\end{itemize}
Gênero de plantas aromáticas; o mesmo que \textunderscore minhonete\textunderscore .--\textunderscore Resedá\textunderscore  é pronúncia francesa, levianamente perfilhada por diccionários.
\section{Resedáceas}
\begin{itemize}
\item {Grp. gram.:f. pl.}
\end{itemize}
Família de plantas dicotyledóneas, que tem por typo a reseda.
(Fem. pl. de \textunderscore resedáceo\textunderscore )
\section{Resedáceo}
\begin{itemize}
\item {Grp. gram.:adj.}
\end{itemize}
Relativo ou semelhante á reseda.
\section{Resedal}
\begin{itemize}
\item {Grp. gram.:m.}
\end{itemize}
Lugar, onde crescem resedas.
Planta lythraríada, (\textunderscore lawsonia inermis\textunderscore ).
\section{Resegar}
\begin{itemize}
\item {fónica:se}
\end{itemize}
\begin{itemize}
\item {Grp. gram.:v. t.}
\end{itemize}
\begin{itemize}
\item {Proveniência:(Do lat. \textunderscore resecare\textunderscore )}
\end{itemize}
Segar novamente.
\section{Resegundar}
\begin{itemize}
\item {fónica:se}
\end{itemize}
\begin{itemize}
\item {Grp. gram.:v. t.}
\end{itemize}
\begin{itemize}
\item {Proveniência:(De \textunderscore re...\textunderscore  + \textunderscore segundar\textunderscore )}
\end{itemize}
Repetir muitas vezes. Cf. \textunderscore Aulegrafia\textunderscore , 31.
\section{Resegurar}
\begin{itemize}
\item {fónica:se}
\end{itemize}
\begin{itemize}
\item {Grp. gram.:v. t.}
\end{itemize}
\begin{itemize}
\item {Proveniência:(De \textunderscore re...\textunderscore  + \textunderscore segurar\textunderscore )}
\end{itemize}
Pôr novamente em seguro (um prédio, uma mercadoria, etc.).
\section{Reseguro}
\begin{itemize}
\item {fónica:se}
\end{itemize}
\begin{itemize}
\item {Grp. gram.:m.}
\end{itemize}
\begin{itemize}
\item {Grp. gram.:Adj.}
\end{itemize}
\begin{itemize}
\item {Proveniência:(De \textunderscore re...\textunderscore  + \textunderscore seguro\textunderscore )}
\end{itemize}
Renovação de um seguro de prédios, de vidas, de mercadorias, etc.
Acto de resegurar.
Novamente seguro; muito seguro; firmíssimo. Cf. Castilho, \textunderscore Geórgicas\textunderscore , 55.
\section{Reselar}
\begin{itemize}
\item {fónica:se}
\end{itemize}
\begin{itemize}
\item {Grp. gram.:v. t.}
\end{itemize}
\begin{itemize}
\item {Proveniência:(De \textunderscore re...\textunderscore  + \textunderscore sellar\textunderscore ^2)}
\end{itemize}
Pôr novo sêllo em.
\section{Resemeadura}
\begin{itemize}
\item {fónica:se}
\end{itemize}
\begin{itemize}
\item {Grp. gram.:f.}
\end{itemize}
Acto ou effeito de resemear.
\section{Resemear}
\begin{itemize}
\item {fónica:se}
\end{itemize}
\begin{itemize}
\item {Grp. gram.:v. t.}
\end{itemize}
\begin{itemize}
\item {Proveniência:(De \textunderscore re...\textunderscore  + \textunderscore semear\textunderscore )}
\end{itemize}
Semear novamente.
\section{Resenha}
\begin{itemize}
\item {Grp. gram.:f.}
\end{itemize}
Acto ou effeito de resenhar.
Relação minuciosa.
Enumeração.
Contagem.
Notícia, que comprehende certo número de nomes ou factos similares.
\section{Resenhar}
\begin{itemize}
\item {Grp. gram.:v. t.}
\end{itemize}
\begin{itemize}
\item {Proveniência:(Do lat. \textunderscore resignare\textunderscore )}
\end{itemize}
Referir minuciosamente; ennumerar por partes.
\section{Resenho}
\begin{itemize}
\item {Grp. gram.:m.}
\end{itemize}
\begin{itemize}
\item {Proveniência:(Do lat. \textunderscore re...\textunderscore  + \textunderscore signum\textunderscore )}
\end{itemize}
Anályse dos sinaes e caracteres principaes dos cavallos, para se distinguirem uns dos outros.
Marca, que se faz geralmente na perna esquerda do cavallo.
\section{Resenhor}
\begin{itemize}
\item {fónica:se}
\end{itemize}
\begin{itemize}
\item {Grp. gram.:m.}
\end{itemize}
\begin{itemize}
\item {Utilização:Des.}
\end{itemize}
\begin{itemize}
\item {Proveniência:(De \textunderscore re...\textunderscore  + \textunderscore senhor\textunderscore )}
\end{itemize}
Duas vezes senhor:«\textunderscore é o senhor sôbre senhor, resenhor, senhor dinheiro\textunderscore ». A. Prestes, \textunderscore Auto do Desembargador\textunderscore .
\section{Resentadura}
\begin{itemize}
\item {Grp. gram.:f.}
\end{itemize}
\begin{itemize}
\item {Utilização:Prov.}
\end{itemize}
\begin{itemize}
\item {Utilização:trasm.}
\end{itemize}
Pequena porção de fermento, com que se azeda uma porção de massa, que chegue para fermento de uma amassadura completa.
\section{Resentido}
\begin{itemize}
\item {fónica:sen}
\end{itemize}
\begin{itemize}
\item {Grp. gram.:adj.}
\end{itemize}
\begin{itemize}
\item {Utilização:Pop.}
\end{itemize}
Que se resentiu; melindrado.
Que começa a apodrecer, (falando-se de frutos).
\section{Resentimento}
\begin{itemize}
\item {fónica:sen}
\end{itemize}
\begin{itemize}
\item {Grp. gram.:m.}
\end{itemize}
Acto ou effeito de resentir.
\section{Resentir}
\begin{itemize}
\item {fónica:sen}
\end{itemize}
\begin{itemize}
\item {Grp. gram.:v. t.}
\end{itemize}
\begin{itemize}
\item {Grp. gram.:V. p.}
\end{itemize}
\begin{itemize}
\item {Proveniência:(De \textunderscore re...\textunderscore  + \textunderscore sentir\textunderscore )}
\end{itemize}
Sentir de novo.
Mostrar-se offendido; melindrar-se.
Dar fé.
Sentir as consequências de alguma coisa: \textunderscore a fruta resente-se do calor\textunderscore .
\section{Resequir}
\begin{itemize}
\item {fónica:se}
\end{itemize}
\begin{itemize}
\item {Grp. gram.:v. t.}
\end{itemize}
\begin{itemize}
\item {Proveniência:(De \textunderscore resêco\textunderscore )}
\end{itemize}
Secar muito; fazer perder o suco ou a humidade a.
\section{Reserenar}
\begin{itemize}
\item {fónica:se}
\end{itemize}
\begin{itemize}
\item {Grp. gram.:v. t.}
\end{itemize}
\begin{itemize}
\item {Proveniência:(De \textunderscore re...\textunderscore  + \textunderscore serenar\textunderscore )}
\end{itemize}
Tornar muito sereno; acalmar inteiramente. Cf. Castilho, \textunderscore Fastos\textunderscore , II, 171.
\section{Resereno}
\begin{itemize}
\item {fónica:se}
\end{itemize}
\begin{itemize}
\item {Grp. gram.:adj.}
\end{itemize}
\begin{itemize}
\item {Proveniência:(De \textunderscore re...\textunderscore  + \textunderscore serenar\textunderscore )}
\end{itemize}
Muito sereno; perfeitamente calmo.
Que readquiriu tranquillidade. Cf. Castilho, \textunderscore Geórgicas\textunderscore , 51.
\section{Reserva}
\begin{itemize}
\item {Grp. gram.:f.}
\end{itemize}
\begin{itemize}
\item {Utilização:Fig.}
\end{itemize}
Acto ou effeito de reservar.
Aquillo que se reserva ou que se poupa para casos imprevistos ou extraordinários.
Vasa das marinhas.
Situação dos soldados que, tendo já servido pelo tempo legal, e deixando de fazer serviço effectivo, estão todavia obrigados a voltar ao mesmo serviço quando as conveniências públicas assim o exijam.
Tropas, que só entram em combate ou em certo serviço, quando é preciso reforçar ou substituir outras que estão em combate ou noutros serviços.
Navios de guerra, promptos a soccorrer outros que estão a descoberto.
Retrahimento.
Dissimulação.
Restricção; excepção, resalva.
\section{Reservação}
\begin{itemize}
\item {Grp. gram.:f.}
\end{itemize}
\begin{itemize}
\item {Utilização:Jur.}
\end{itemize}
\begin{itemize}
\item {Proveniência:(De reservar)}
\end{itemize}
O mesmo que \textunderscore reserva\textunderscore .
Condição restrictiva de uma doação ou dos seus effeitos.
\section{Reservadamente}
\begin{itemize}
\item {Grp. gram.:adv.}
\end{itemize}
De modo reservado.
Com reserva.
\section{Reservado}
\begin{itemize}
\item {Grp. gram.:adj.}
\end{itemize}
\begin{itemize}
\item {Proveniência:(De \textunderscore reservar\textunderscore )}
\end{itemize}
Que tem reserva; em que há reserva.
Que conserva ódio a quem o offendeu.
\section{Reservador}
\begin{itemize}
\item {Grp. gram.:m.  e  adj.}
\end{itemize}
O que reserva.
\section{Reservar}
\begin{itemize}
\item {Grp. gram.:v. t.}
\end{itemize}
\begin{itemize}
\item {Proveniência:(Lat. \textunderscore reservare\textunderscore )}
\end{itemize}
Conservar, guardar, pôr de parte.
Poupar.
Demorar, adiar.
Defender; exceptuar.
Destinar para certos fins ou para certas occasiões.
Fazer segrêdo de.
Guardar para si.
\section{Reservatário}
\begin{itemize}
\item {Grp. gram.:adj.}
\end{itemize}
\begin{itemize}
\item {Utilização:Jur.}
\end{itemize}
\begin{itemize}
\item {Proveniência:(De \textunderscore reservar\textunderscore )}
\end{itemize}
Que recebe bens, incluídos na terça de uma herança, e destinados por disposição testamentária para certos effeitos, ou que ficaram fóra da partilha para effeitos especiaes.
\section{Reservativo}
\begin{itemize}
\item {Grp. gram.:adj.}
\end{itemize}
\begin{itemize}
\item {Utilização:Jur.}
\end{itemize}
\begin{itemize}
\item {Proveniência:(De \textunderscore reservar\textunderscore )}
\end{itemize}
Em que há reserva.
\textunderscore Censo reservativo\textunderscore , cessão, que alguém faz, de um prédio, reservando certa pensão annual, pelos frutos ou rendimentos dêsse prédio. Cf. \textunderscore Código Civil\textunderscore , art. 1706.
\section{Reservatório}
\begin{itemize}
\item {Grp. gram.:adj.}
\end{itemize}
\begin{itemize}
\item {Grp. gram.:M.}
\end{itemize}
\begin{itemize}
\item {Proveniência:(De \textunderscore reservar\textunderscore )}
\end{itemize}
Próprio para reservar.
Lugar, onde se reservam coisas.
Depósito de água.
Depósito; recipiente.
Lugar, onde se accumula alguma coisa.
\section{Reservir}
\begin{itemize}
\item {fónica:ser}
\end{itemize}
\begin{itemize}
\item {Grp. gram.:v. t.}
\end{itemize}
\begin{itemize}
\item {Proveniência:(De \textunderscore re...\textunderscore  + \textunderscore servir\textunderscore )}
\end{itemize}
Servir de novo.
\section{Reservista}
\begin{itemize}
\item {Grp. gram.:m.}
\end{itemize}
Soldado, que está na reserva.
\section{Resesso}
\begin{itemize}
\item {fónica:sê}
\end{itemize}
\begin{itemize}
\item {Grp. gram.:adj.}
\end{itemize}
\begin{itemize}
\item {Utilização:Pop.}
\end{itemize}
Resêco; endurecido por têr secado, (falando-se de pão ou de bolos).
(Corr. de \textunderscore resêco\textunderscore ? ou de \textunderscore re...\textunderscore  + lat. \textunderscore sessum\textunderscore , de \textunderscore sedere\textunderscore ?)
\section{Reseu}
\begin{itemize}
\item {fónica:seu}
\end{itemize}
\begin{itemize}
\item {Grp. gram.:pron.}
\end{itemize}
\begin{itemize}
\item {Proveniência:(De \textunderscore re...\textunderscore  + \textunderscore seu\textunderscore )}
\end{itemize}
Muito seu. Cf. Castilho, \textunderscore Tartufo\textunderscore , 159.
\section{Resfolegado}
\begin{itemize}
\item {Grp. gram.:adj.}
\end{itemize}
Que resfólega.
Que descansa; sossegado; sereno.
\section{Resfolegadoiro}
\begin{itemize}
\item {Grp. gram.:m.}
\end{itemize}
\begin{itemize}
\item {Proveniência:(De \textunderscore resfolegar\textunderscore )}
\end{itemize}
Respiradoiro.
Lugar, por onde entra o ar necessário para mover certos maquinismos.
\section{Resfolegadouro}
\begin{itemize}
\item {Grp. gram.:m.}
\end{itemize}
\begin{itemize}
\item {Proveniência:(De \textunderscore resfolegar\textunderscore )}
\end{itemize}
Respiradouro.
Lugar, por onde entra o ar necessário para mover certos maquinismos.
\section{Resfolegar}
\begin{itemize}
\item {Grp. gram.:v. i.}
\end{itemize}
\begin{itemize}
\item {Utilização:Fig.}
\end{itemize}
\begin{itemize}
\item {Grp. gram.:V. t.}
\end{itemize}
\begin{itemize}
\item {Proveniência:(De \textunderscore res...\textunderscore  + \textunderscore fôlego\textunderscore )}
\end{itemize}
Respirar.
Tomar fôlego.
Têr descanso.
Tranquillizar-se, repoisar.
Expellir, respirar.
Golfar:«\textunderscore os antros do coração resfolgam fogo de paixões...\textunderscore »Camillo, \textunderscore Volcoens\textunderscore , 5.--Na conjugação do verbo, observa-se a contr. \textunderscore resfolgar\textunderscore .
\section{Resfôlego}
\begin{itemize}
\item {Grp. gram.:m.}
\end{itemize}
Acto ou effeito de resfolegar.
\section{Resfolgante}
\begin{itemize}
\item {Grp. gram.:adj.}
\end{itemize}
Que resfolga. Cf. Castilho, \textunderscore Géorgicas\textunderscore , 245.
\section{Resfolgar}
\textunderscore v. t.\textunderscore  e \textunderscore i.\textunderscore  (e der.)
(Contr. de \textunderscore resfolegar\textunderscore , etc.)
\section{Resfriadeira}
\begin{itemize}
\item {Grp. gram.:f.}
\end{itemize}
\begin{itemize}
\item {Utilização:Bras}
\end{itemize}
\begin{itemize}
\item {Proveniência:(De \textunderscore resfriar\textunderscore )}
\end{itemize}
Lugar, onde resfria o açúcar, nos respectivos engenhos.
\section{Resfriado}
\begin{itemize}
\item {Grp. gram.:m.}
\end{itemize}
\begin{itemize}
\item {Proveniência:(De \textunderscore resfriar\textunderscore )}
\end{itemize}
Resfriamento.
\section{Resfriadoiro}
\begin{itemize}
\item {Grp. gram.:m.}
\end{itemize}
Lugar, onde alguma coisa se resfria; objecto, que faz resfriar.
\section{Resfriadouro}
\begin{itemize}
\item {Grp. gram.:m.}
\end{itemize}
Lugar, onde alguma coisa se resfria; objecto, que faz resfriar.
\section{Resfriador}
\begin{itemize}
\item {Grp. gram.:adj.}
\end{itemize}
\begin{itemize}
\item {Grp. gram.:M.}
\end{itemize}
Que resfria.
Vaso ou vasilha cheia de água fria, para resfriar certos objectos.
\section{Resfriamento}
\begin{itemize}
\item {Grp. gram.:m.}
\end{itemize}
Acto ou effeito de resfriar.
Doença, produzida por um grande abaixamento de temperatura.
Aguamento, (falando-se de animaes).
\section{Resfriar}
\begin{itemize}
\item {Grp. gram.:v. t.}
\end{itemize}
\begin{itemize}
\item {Utilização:Fig.}
\end{itemize}
\begin{itemize}
\item {Grp. gram.:V. i.  e  p.}
\end{itemize}
\begin{itemize}
\item {Utilização:Fig.}
\end{itemize}
\begin{itemize}
\item {Proveniência:(De \textunderscore re...\textunderscore  + \textunderscore esfriar\textunderscore )}
\end{itemize}
Esfriar novamente.
Tornar muito frio.
Sujeitar artificialmente a grande frio.
Desalentar, descoroçoar.
Tornar-se frio.
Desalentar-se, desanimar-se.
\section{Resgalha}
\begin{itemize}
\item {Grp. gram.:f.}
\end{itemize}
\begin{itemize}
\item {Utilização:Prov.}
\end{itemize}
\begin{itemize}
\item {Utilização:alent.}
\end{itemize}
\begin{itemize}
\item {Proveniência:(De \textunderscore rês\textunderscore  + \textunderscore galho\textunderscore )}
\end{itemize}
Rês, que tem galhos ou cornos: \textunderscore caçou duas resgalhas\textunderscore .
\section{Resgar}
\begin{itemize}
\item {Proveniência:(Do lat. \textunderscore resecare\textunderscore )}
\end{itemize}
\textunderscore v. t.\textunderscore  (e der.)
(Fórma divergente de \textunderscore rasgar\textunderscore , etc. Us. especialmente no Algarve)
\section{Resgatador}
\begin{itemize}
\item {Grp. gram.:m.  e  adj.}
\end{itemize}
O que resgata.
\section{Resgatar}
\begin{itemize}
\item {Grp. gram.:v. t.}
\end{itemize}
\begin{itemize}
\item {Proveniência:(Do lat. \textunderscore re...\textunderscore  + \textunderscore ex\textunderscore  + \textunderscore captare\textunderscore )}
\end{itemize}
Livrar do cativeiro, por dinheiro ou presentes.
Remir.
Cumprir: \textunderscore resgatar deveres\textunderscore .
Conseguir com difficuldade ou sacrificio.
Obter por dinheiro a restituição de (objectos penhorados e conservados em poder de outrem).
Expiar; tornar esquecida (uma culpa ou offensa).
\section{Resgatável}
\begin{itemize}
\item {Grp. gram.:adj.}
\end{itemize}
Que se póde resgatar.
\section{Resgate}
\begin{itemize}
\item {Grp. gram.:m.}
\end{itemize}
\begin{itemize}
\item {Utilização:Ant.}
\end{itemize}
Acto ou effeito de resgatar.
Preço, por que se resgata.
Quitação.
Libertação.
O mesmo que \textunderscore negócio\textunderscore .
\section{Résgo}
\begin{itemize}
\item {Grp. gram.:m.}
\end{itemize}
\begin{itemize}
\item {Utilização:Prov.}
\end{itemize}
\begin{itemize}
\item {Utilização:alg.}
\end{itemize}
\begin{itemize}
\item {Proveniência:(De \textunderscore resgar\textunderscore )}
\end{itemize}
Actividade ou desembaraço, (falando-se de uma pessôa).
Extracção, venda fácil, (falando-se de mercadorias).
\section{Resguardar}
\begin{itemize}
\item {Grp. gram.:v. t.}
\end{itemize}
\begin{itemize}
\item {Grp. gram.:V. i.}
\end{itemize}
\begin{itemize}
\item {Proveniência:(De \textunderscore re...\textunderscore  + \textunderscore esguardar\textunderscore )}
\end{itemize}
Guardar bem, com cuidado.
Pôr defesa a.
Acobertar, abrigar: \textunderscore resguardar alguém da chuva\textunderscore .
Pôr a salvo.
Reservar, poupar.
Observar; vigiar.
Defrontar com.
Estar defronte, estar voltado para determinada pessôa ou coisa.
Attender.
\section{Resguardo}
\begin{itemize}
\item {Grp. gram.:m.}
\end{itemize}
Acto ou effeito de resguardar.
Precaução.
Dieta.
Escrúpulo.
Prudência; decoro.
Defesa. Segrêdo.
Nos caminhos de ferro, o mesmo que \textunderscore desvio\textunderscore .
\section{Re-si}
\begin{itemize}
\item {Grp. gram.:adv.}
\end{itemize}
\begin{itemize}
\item {Utilização:Ant.}
\end{itemize}
O mesmo que \textunderscore sim\textunderscore , reforçadamente:«\textunderscore digo te que si, re-si.\textunderscore »G. Vicente, I, 226.
\section{Resiccação}
\begin{itemize}
\item {fónica:si}
\end{itemize}
\begin{itemize}
\item {Grp. gram.:f.}
\end{itemize}
Acto ou effeito de resiccar.
\section{Resiccar}
\begin{itemize}
\item {fónica:si}
\end{itemize}
\begin{itemize}
\item {Grp. gram.:v. t.}
\end{itemize}
\begin{itemize}
\item {Proveniência:(Lat. \textunderscore resiccare\textunderscore )}
\end{itemize}
Tornar muito sêco, resequir.
\section{Residência}
\begin{itemize}
\item {Grp. gram.:f.}
\end{itemize}
\begin{itemize}
\item {Utilização:Prov.}
\end{itemize}
\begin{itemize}
\item {Proveniência:(Lat. \textunderscore residentia\textunderscore )}
\end{itemize}
Morada habitual em um lugar.
Habitação; domicílio.
Lugar, ou casa onde se habita.
Restrictamente, habitação do párocho.
\section{Residente}
\begin{itemize}
\item {Grp. gram.:adj.}
\end{itemize}
\begin{itemize}
\item {Grp. gram.:M.}
\end{itemize}
\begin{itemize}
\item {Proveniência:(Lat. \textunderscore residens\textunderscore )}
\end{itemize}
Que reside.
Funccionário português, que tem domicílio official em terras de sobas.
\section{Resídio}
\begin{itemize}
\item {Grp. gram.:m.}
\end{itemize}
\begin{itemize}
\item {Utilização:Ant.}
\end{itemize}
\begin{itemize}
\item {Proveniência:(Lat. \textunderscore residium\textunderscore )}
\end{itemize}
Imposto de residência.
\section{Residir}
\begin{itemize}
\item {Grp. gram.:v. i.}
\end{itemize}
\begin{itemize}
\item {Proveniência:(Lat. \textunderscore residere\textunderscore )}
\end{itemize}
Estabelecer residência, morar.
Têr séde.
Estar; manifestar-se: \textunderscore o seu incômmodo reside nos intestinos\textunderscore .
\section{Residual}
\begin{itemize}
\item {Grp. gram.:adj.}
\end{itemize}
Relativo a resíduo ou próprio delle.
\section{Residuário}
\begin{itemize}
\item {Grp. gram.:adj.}
\end{itemize}
Relativo a resíduos; próprio para receber resíduos.
\section{Resíduo}
\begin{itemize}
\item {Grp. gram.:adj.}
\end{itemize}
\begin{itemize}
\item {Grp. gram.:M.}
\end{itemize}
\begin{itemize}
\item {Proveniência:(Lat. \textunderscore residuus\textunderscore )}
\end{itemize}
Que resta.
Aquillo que resta.
Substância, que resta depois de uma operação chímica e póde ainda sêr utilizada.
Fezes.
\section{Resignação}
\begin{itemize}
\item {Grp. gram.:f.}
\end{itemize}
Acto ou effeito de resignar.
Exoneração espontânea de uma graça ou de um cargo; renúncia.
Sujeição paciente ás contrariedades da vida.
Paciência; coragem, com que se soffre a desgraça.
\section{Resignadamente}
\begin{itemize}
\item {Grp. gram.:adv.}
\end{itemize}
De modo resignado.
Com resignação; pacientemente.
\section{Resignante}
\begin{itemize}
\item {Grp. gram.:m.  e  adj.}
\end{itemize}
\begin{itemize}
\item {Proveniência:(Lat. \textunderscore resignas\textunderscore )}
\end{itemize}
O que resignou um cargo, graça, ou aquillo a que tinha direito.
\section{Resignar}
\begin{itemize}
\item {Grp. gram.:v. t.}
\end{itemize}
\begin{itemize}
\item {Grp. gram.:V. p.}
\end{itemize}
\begin{itemize}
\item {Proveniência:(Lat. \textunderscore resignare\textunderscore )}
\end{itemize}
Renunciar.
Ceder espontaneamente.
Dimittir-se de.
Têr resignação, conformar-se.
\section{Resignatário}
\begin{itemize}
\item {Grp. gram.:m.  e  adj.}
\end{itemize}
Aquelle que resigna ou renuncía cargo ou dignidade.
\section{Resignável}
\begin{itemize}
\item {Grp. gram.:adj.}
\end{itemize}
Que se póde resignar.
\section{Resilir}
\begin{itemize}
\item {Proveniência:(Lat. \textunderscore resilire\textunderscore )}
\end{itemize}
\textunderscore v. t.\textunderscore  (e der.)
O mesmo que \textunderscore rescindir\textunderscore , etc.
Noutra accepção:«\textunderscore Da morte não pódem as almas resilir a região da vida.\textunderscore »\textunderscore Luz e Calor\textunderscore , 479.
\section{Resina}
\begin{itemize}
\item {Grp. gram.:f.}
\end{itemize}
\begin{itemize}
\item {Proveniência:(Lat. \textunderscore resina\textunderscore )}
\end{itemize}
Substância inflammável e untuosa, segregada por certos vegetaes, especialmente pelos pinheiros e várias coníferas.
Substância análoga, de origem animal.
\section{Resinação}
\begin{itemize}
\item {Grp. gram.:f.}
\end{itemize}
O mesmo que \textunderscore resinagem\textunderscore .
\section{Resinagem}
\begin{itemize}
\item {Grp. gram.:f.}
\end{itemize}
Acto ou effeito de resinar.
Producção de resina em as árvores.
\section{Resinar}
\begin{itemize}
\item {Grp. gram.:v. t.}
\end{itemize}
Tirar resina de.
Applicar resina a.
Misturar ou compor com resina.
\section{Resineiro}
\begin{itemize}
\item {Grp. gram.:adj.}
\end{itemize}
\begin{itemize}
\item {Grp. gram.:M.}
\end{itemize}
\begin{itemize}
\item {Utilização:Prov.}
\end{itemize}
\begin{itemize}
\item {Utilização:minh.}
\end{itemize}
Relativo a resina.
Que colhe resina ou a prepara.
Preparador ou explorador de resina de pinheiro. Cf. Th. Ribeiro, \textunderscore Jornadas\textunderscore , I, 147.
Indivíduo, que sangra os pinheiros, para lhes extrahir a resina.
Pau, ou feixe de paus resinosos, que serve de brandão para alumiar.
\section{Resinento}
\begin{itemize}
\item {Grp. gram.:adj.}
\end{itemize}
O mesmo que \textunderscore resinoso\textunderscore .
\section{Resinga}
\begin{itemize}
\item {Grp. gram.:f.}
\end{itemize}
\begin{itemize}
\item {Utilização:Ant.}
\end{itemize}
Peça de guarda-roupa, no século XVI; (ignora-se o seu uso). Cf. \textunderscore Provas da Hist. Geneal.\textunderscore 
\section{Resinífero}
\begin{itemize}
\item {Grp. gram.:adj.}
\end{itemize}
\begin{itemize}
\item {Proveniência:(Do lat. \textunderscore resina\textunderscore  + \textunderscore ferre\textunderscore )}
\end{itemize}
Que produz resina.
\section{Resinificar}
\begin{itemize}
\item {Grp. gram.:v. t.}
\end{itemize}
\begin{itemize}
\item {Proveniência:(Do lat. \textunderscore resina\textunderscore  + \textunderscore facere\textunderscore )}
\end{itemize}
Converter em resina; dar apparência de resina a.
\section{Resiniforme}
\begin{itemize}
\item {Grp. gram.:adj.}
\end{itemize}
\begin{itemize}
\item {Proveniência:(Do lat. \textunderscore resina\textunderscore  + \textunderscore forma\textunderscore )}
\end{itemize}
Que tem apparência de resina.
\section{Resínio}
\begin{itemize}
\item {Grp. gram.:adj.}
\end{itemize}
\begin{itemize}
\item {Utilização:P. us.}
\end{itemize}
Relativo a resina; que tem resina; untado com resina:«\textunderscore ...queimais resínio archote.\textunderscore »Castilho, \textunderscore Fastos\textunderscore , II, 187.
\section{Resinite}
\begin{itemize}
\item {Grp. gram.:f.}
\end{itemize}
\begin{itemize}
\item {Utilização:Miner.}
\end{itemize}
\begin{itemize}
\item {Proveniência:(De \textunderscore resina\textunderscore )}
\end{itemize}
Espécie de ópala, de aspecto resinoso.
\section{Resinito}
\begin{itemize}
\item {Grp. gram.:m.}
\end{itemize}
O mesmo ou melhor que \textunderscore resinite\textunderscore .
\section{Resinocérum}
\begin{itemize}
\item {Grp. gram.:m.}
\end{itemize}
\begin{itemize}
\item {Proveniência:(De \textunderscore resina\textunderscore  + \textunderscore cera\textunderscore )}
\end{itemize}
Medicamento, formado de cêra e resina.
\section{Resinóide}
\begin{itemize}
\item {Grp. gram.:adj.}
\end{itemize}
\begin{itemize}
\item {Proveniência:(Do lat. \textunderscore resina\textunderscore  + gr. \textunderscore eidos\textunderscore )}
\end{itemize}
O mesmo que \textunderscore resiniforme\textunderscore .
\section{Resinoso}
\begin{itemize}
\item {Grp. gram.:adj.}
\end{itemize}
\begin{itemize}
\item {Proveniência:(Lat. \textunderscore resinosus\textunderscore )}
\end{itemize}
Resinífero; resiniforme.
Coberto de resina ou de suco semelhante a ella, (falando-se de vegetaes).
\section{Resispiscência}
\begin{itemize}
\item {Grp. gram.:f.}
\end{itemize}
\begin{itemize}
\item {Proveniência:(Lat. \textunderscore resispiscentia\textunderscore )}
\end{itemize}
Arrependimento de um peccado, com o propósito de emenda; emenda moral.
\section{Resistência}
\begin{itemize}
\item {Grp. gram.:f.}
\end{itemize}
\begin{itemize}
\item {Proveniência:(Lat. \textunderscore resistentia\textunderscore )}
\end{itemize}
Acto ou effeito de resistir.
Fôrça ou qualidade de um corpo, que annulla os effeitos de outra fôrça ou de outro corpo.
Aquillo que se oppõe ao movimento de um corpo.
Obstáculo.
Reacção.
Opposição.
Luta em defesa; defesa.
\section{Resistente}
\begin{itemize}
\item {Grp. gram.:adj.}
\end{itemize}
\begin{itemize}
\item {Proveniência:(Lat. \textunderscore resistens\textunderscore )}
\end{itemize}
Que resiste.
Sólido.
Contumaz; obstinado.
\section{Resistir}
\begin{itemize}
\item {Grp. gram.:v. i.}
\end{itemize}
\begin{itemize}
\item {Grp. gram.:V. t.}
\end{itemize}
\begin{itemize}
\item {Proveniência:(Lat. \textunderscore resistere\textunderscore )}
\end{itemize}
Oppor-se, não ceder; fazer resistência.
Defender-se.
Recusar-se.
Subsistir; durar: \textunderscore esta fruta não resiste ao calor\textunderscore .
Oppor-se a:«\textunderscore ...não podendo resistir seu furioso ímpeto...\textunderscore »Usque, 47.
\section{Resistivel}
\begin{itemize}
\item {Grp. gram.:adj.}
\end{itemize}
A que se póde resistir.
\section{Resisto}
\begin{itemize}
\item {Grp. gram.:m.}
\end{itemize}
\begin{itemize}
\item {Utilização:Pop.}
\end{itemize}
O mesmo que \textunderscore registo\textunderscore ,«\textunderscore esta imagem não é resisto...\textunderscore »Filinto, XII, 242.
\section{Reslavra}
\begin{itemize}
\item {Grp. gram.:f.}
\end{itemize}
O mesmo que \textunderscore deslavra\textunderscore . Cf. \textunderscore Bibl. da G. do Campo\textunderscore , 265.
\section{Reslumbrar}
\begin{itemize}
\item {Grp. gram.:v. i.}
\end{itemize}
\begin{itemize}
\item {Proveniência:(De \textunderscore re...\textunderscore  + cast. \textunderscore lumbre\textunderscore )}
\end{itemize}
Deixar passar a luz; transluzir; transparecer.
\section{Resma}
\begin{itemize}
\item {fónica:rês}
\end{itemize}
\begin{itemize}
\item {Grp. gram.:f.}
\end{itemize}
\begin{itemize}
\item {Proveniência:(Do ár. \textunderscore rizma\textunderscore )}
\end{itemize}
Vinte mãos de papel ou quinhentas fôlhas.
\section{Resmalhar}
\begin{itemize}
\item {Grp. gram.:v. i.}
\end{itemize}
\begin{itemize}
\item {Utilização:Prov.}
\end{itemize}
\begin{itemize}
\item {Utilização:alent.}
\end{itemize}
O mesmo que \textunderscore ramalhar\textunderscore , sussurrar.
\section{Resmelengar}
\begin{itemize}
\item {Grp. gram.:v. i.}
\end{itemize}
\begin{itemize}
\item {Utilização:Bras. do N}
\end{itemize}
Sêr resmelengo, avarento.
Rabujar.
\section{Resmelengo}
\begin{itemize}
\item {Grp. gram.:adj.}
\end{itemize}
\begin{itemize}
\item {Utilização:Bras}
\end{itemize}
\begin{itemize}
\item {Utilização:Bras. do N}
\end{itemize}
Rabujento; teimoso.
Avarento.
Sovina.
(Cp. \textunderscore resmungão\textunderscore )
\section{Resmolgar}
\begin{itemize}
\item {Grp. gram.:v. i.}
\end{itemize}
\begin{itemize}
\item {Utilização:Prov.}
\end{itemize}
\begin{itemize}
\item {Utilização:alg.}
\end{itemize}
Variar, (falando-se do tempo).
\section{Resmuda}
\begin{itemize}
\item {Grp. gram.:f.}
\end{itemize}
\begin{itemize}
\item {Utilização:Pop.}
\end{itemize}
\begin{itemize}
\item {Proveniência:(De \textunderscore res...\textunderscore  + \textunderscore muda\textunderscore )}
\end{itemize}
Ordem invertida, mudança.
\section{Resmunear}
\begin{itemize}
\item {Grp. gram.:v. i.}
\end{itemize}
O mesmo que \textunderscore resmungar\textunderscore .
\section{Resmuneio}
\begin{itemize}
\item {Grp. gram.:m.}
\end{itemize}
\begin{itemize}
\item {Utilização:Neol.}
\end{itemize}
Acto de resmunear.
\section{Resmungão}
\begin{itemize}
\item {Grp. gram.:m.  e  adj.}
\end{itemize}
O que resmunga.
\section{Resmungar}
\begin{itemize}
\item {Grp. gram.:v. t.}
\end{itemize}
\begin{itemize}
\item {Grp. gram.:V. i.}
\end{itemize}
Pronunciar confusamente, por entre dentes e com mau humor.
Falar baixo, geralmente com mau humor; rezingar.
(Talvez do lat. \textunderscore remussicare\textunderscore , de \textunderscore mussare\textunderscore , falar por entre os dentes)
\section{Resmungo}
\begin{itemize}
\item {Grp. gram.:m.}
\end{itemize}
\begin{itemize}
\item {Utilização:bras}
\end{itemize}
\begin{itemize}
\item {Utilização:Neol.}
\end{itemize}
Acto de resmungar.
\section{Resmunguice}
\begin{itemize}
\item {Grp. gram.:f.}
\end{itemize}
\begin{itemize}
\item {Utilização:Fam.}
\end{itemize}
Hábito de resmungar.
\section{Resmuninhador}
\begin{itemize}
\item {Grp. gram.:m.  e  adj.}
\end{itemize}
\begin{itemize}
\item {Utilização:Pop.}
\end{itemize}
O que resmuninha.
\section{Resmuninhar}
\begin{itemize}
\item {Grp. gram.:v. i.}
\end{itemize}
\begin{itemize}
\item {Utilização:Pop.}
\end{itemize}
O mesmo que \textunderscore resmungar\textunderscore . Cf. Herculano, \textunderscore Opúsc.\textunderscore , IV, 121.
\section{Resoante}
\begin{itemize}
\item {fónica:so}
\end{itemize}
\begin{itemize}
\item {Grp. gram.:adj.}
\end{itemize}
\begin{itemize}
\item {Proveniência:(Lat. \textunderscore resonans\textunderscore )}
\end{itemize}
Que resôa.
\section{Resoar}
\begin{itemize}
\item {fónica:so}
\end{itemize}
\begin{itemize}
\item {Grp. gram.:v. t.}
\end{itemize}
\begin{itemize}
\item {Utilização:Fig.}
\end{itemize}
\begin{itemize}
\item {Grp. gram.:V. i.}
\end{itemize}
\begin{itemize}
\item {Proveniência:(Do lat. \textunderscore ressonare\textunderscore )}
\end{itemize}
Entoar; repercutir.
Cantar; tocar.
Soar de novo.
Repercutir-se.
Echoar.
Estrondear.
Tornar a dizer-se.
Fazer-se ouvir.
Espalhar-se como boato.
\section{Resobrar}
\begin{itemize}
\item {fónica:so}
\end{itemize}
\begin{itemize}
\item {Grp. gram.:v. i.}
\end{itemize}
\begin{itemize}
\item {Proveniência:(De \textunderscore re...\textunderscore  + \textunderscore sobrar\textunderscore )}
\end{itemize}
Sobrar excessivamente.
\section{Resoca}
\begin{itemize}
\item {fónica:só}
\end{itemize}
\begin{itemize}
\item {Grp. gram.:f.}
\end{itemize}
\begin{itemize}
\item {Utilização:Bras}
\end{itemize}
\begin{itemize}
\item {Proveniência:(De \textunderscore re...\textunderscore  + \textunderscore soca\textunderscore )}
\end{itemize}
Terceiro córte da cana de açúcar.
Segundos rebentos da cana.
\section{Resoldar}
\begin{itemize}
\item {fónica:sol}
\end{itemize}
\begin{itemize}
\item {Grp. gram.:v. t.}
\end{itemize}
\begin{itemize}
\item {Proveniência:(De \textunderscore re...\textunderscore  + \textunderscore soldar\textunderscore )}
\end{itemize}
Soldar novamente.
Soldar bem.
\section{Resolto}
\begin{itemize}
\item {fónica:sôl}
\end{itemize}
\begin{itemize}
\item {Grp. gram.:adj.}
\end{itemize}
\begin{itemize}
\item {Proveniência:(Do lat. \textunderscore resolutus\textunderscore )}
\end{itemize}
Dissolvido, desfeito.
\section{Resolto}
\begin{itemize}
\item {fónica:sôl}
\end{itemize}
\begin{itemize}
\item {Grp. gram.:adj.}
\end{itemize}
\begin{itemize}
\item {Proveniência:(De \textunderscore re...\textunderscore  + \textunderscore solto\textunderscore )}
\end{itemize}
Muito sôlto.
Desprendido:«\textunderscore ...esvoaça a túnica ressolta.\textunderscore »Castilho, \textunderscore Fastos\textunderscore , II, 73.
\section{Resolubilidade}
\begin{itemize}
\item {Grp. gram.:f.}
\end{itemize}
Qualidade de resolúvel.
\section{Resolução}
\begin{itemize}
\item {Grp. gram.:f.}
\end{itemize}
\begin{itemize}
\item {Utilização:Mús.}
\end{itemize}
\begin{itemize}
\item {Proveniência:(Do lat. \textunderscore resolutio\textunderscore )}
\end{itemize}
Acto ou effeito de resolver.
Soltura de ventre.
Deliberação, decisão.
Tenção, propósito.
Transformação.
Coragem, intrepidez: \textunderscore homem de resolução\textunderscore .
Mudança ou passagem de um acorde para outro, ou de uma nota para outra.
\section{Resolutamente}
\begin{itemize}
\item {Grp. gram.:adv.}
\end{itemize}
De modo resoluto.
Promptamente.
Energicamente; com audácia.
\section{Resolutivo}
\begin{itemize}
\item {Grp. gram.:adj.}
\end{itemize}
\begin{itemize}
\item {Grp. gram.:M.}
\end{itemize}
\begin{itemize}
\item {Proveniência:(De \textunderscore resoluto\textunderscore )}
\end{itemize}
Que faz cessar uma inflammação insensivelmente e sem suppuração.
Medicamento, que produz êsse effeito.
\section{Resoluto}
\begin{itemize}
\item {Grp. gram.:adj.}
\end{itemize}
\begin{itemize}
\item {Utilização:Fig.}
\end{itemize}
\begin{itemize}
\item {Proveniência:(Lat. \textunderscore resolutus\textunderscore )}
\end{itemize}
Resolto^2.
Corajoso; desembaraçado; decidido.
\section{Resolutório}
\begin{itemize}
\item {Grp. gram.:adj.}
\end{itemize}
\begin{itemize}
\item {Proveniência:(De \textunderscore resoluto\textunderscore )}
\end{itemize}
Próprio para resolver; que produz resolução.
\section{Resolúvel}
\begin{itemize}
\item {Grp. gram.:adj.}
\end{itemize}
\begin{itemize}
\item {Proveniência:(Do lat. \textunderscore resolubilis\textunderscore )}
\end{itemize}
Que se póde resolver: \textunderscore problema resolúvel\textunderscore .
\section{Resolvente}
\begin{itemize}
\item {Grp. gram.:adj.}
\end{itemize}
\begin{itemize}
\item {Grp. gram.:M.}
\end{itemize}
\begin{itemize}
\item {Proveniência:(Lat. \textunderscore resolvens\textunderscore )}
\end{itemize}
Que resolve.
Medicamento resolutivo.
\section{Resolver}
\begin{itemize}
\item {Grp. gram.:v. t.}
\end{itemize}
\begin{itemize}
\item {Grp. gram.:V. i.}
\end{itemize}
\begin{itemize}
\item {Proveniência:(Lat. \textunderscore resolvere\textunderscore )}
\end{itemize}
O mesmo que \textunderscore dissolver\textunderscore .
Separar os elementos constitutivos de.
Separar, desaggregar.
Transformar.
Desfazer sem suppuração (uma inflammação, um tumor).
Explicar, achar a solução de (um problema).
Determinar; decidir: \textunderscore resolver uma viagem\textunderscore .
Deliberar.
Despachar: \textunderscore resolver uma pretensão\textunderscore .
Desembaraçar-se.
Tomar deliberação.
Mostrar-se prompto ou disposto.
Desfazer-se insensivelmente, (falando-se de um tumor ou inflammação).
\section{Resolvido}
\begin{itemize}
\item {Grp. gram.:adj.}
\end{itemize}
\begin{itemize}
\item {Proveniência:(De \textunderscore resolver\textunderscore )}
\end{itemize}
Combinado, assente.
\section{Resolvimento}
\begin{itemize}
\item {Grp. gram.:m.}
\end{itemize}
\begin{itemize}
\item {Utilização:Des.}
\end{itemize}
\begin{itemize}
\item {Proveniência:(De \textunderscore resolver\textunderscore )}
\end{itemize}
O mesmo que \textunderscore resolução\textunderscore .
\section{Resolvível}
\begin{itemize}
\item {Grp. gram.:adj.}
\end{itemize}
Que se póde resolver; resolúvel.
\section{Resonador}
\begin{itemize}
\item {fónica:so}
\end{itemize}
\begin{itemize}
\item {Grp. gram.:m.}
\end{itemize}
Aquelle que resona. Cf. Capello e Ivens, I, 19.
\section{Resonância}
\begin{itemize}
\item {fónica:so}
\end{itemize}
\begin{itemize}
\item {Grp. gram.:f.}
\end{itemize}
\begin{itemize}
\item {Proveniência:(Lat. \textunderscore resonantia\textunderscore )}
\end{itemize}
Propriedade ou qualidade do que é resonante.
\section{Resonante}
\begin{itemize}
\item {fónica:so}
\end{itemize}
\begin{itemize}
\item {Grp. gram.:adj.}
\end{itemize}
\begin{itemize}
\item {Proveniência:(Lat. \textunderscore resonans\textunderscore )}
\end{itemize}
Que resôa.
Que resona.
\section{Resonar}
\begin{itemize}
\item {fónica:so}
\end{itemize}
\begin{itemize}
\item {Grp. gram.:v. t.}
\end{itemize}
\begin{itemize}
\item {Grp. gram.:V. i.}
\end{itemize}
\begin{itemize}
\item {Proveniência:(Lat. \textunderscore resonare\textunderscore )}
\end{itemize}
Resoar.
Fazer soar.
Respirar ruidosamente, dormindo.
Dormir.
\section{Resoprar}
\begin{itemize}
\item {fónica:so}
\end{itemize}
\begin{itemize}
\item {Grp. gram.:v. t.}
\end{itemize}
\begin{itemize}
\item {Proveniência:(De \textunderscore re...\textunderscore  + \textunderscore soprar\textunderscore )}
\end{itemize}
Tornar a soprar.
\section{Resorcina}
\begin{itemize}
\item {Grp. gram.:f.}
\end{itemize}
\begin{itemize}
\item {Utilização:Chím.}
\end{itemize}
Um dos phenóes, que derivam da benzina.
\section{Resorpção}
\begin{itemize}
\item {fónica:sor}
\end{itemize}
\begin{itemize}
\item {Grp. gram.:f.}
\end{itemize}
Acto ou effeito de \textunderscore resorver\textunderscore .
\section{Resorver}
\begin{itemize}
\item {fónica:sor}
\end{itemize}
\begin{itemize}
\item {Grp. gram.:v. t.}
\end{itemize}
\begin{itemize}
\item {Proveniência:(Do lat. \textunderscore resorbere\textunderscore )}
\end{itemize}
Sorver novamente.
\section{Respalda}
\begin{itemize}
\item {Grp. gram.:f.}
\end{itemize}
\begin{itemize}
\item {Utilização:Prov.}
\end{itemize}
\begin{itemize}
\item {Utilização:beir.}
\end{itemize}
\begin{itemize}
\item {Proveniência:(De \textunderscore respaldar\textunderscore )}
\end{itemize}
Pedra chata, com que se calça outra; calço.
\section{Respaldar}
\begin{itemize}
\item {Grp. gram.:v. t.}
\end{itemize}
\begin{itemize}
\item {Proveniência:(De \textunderscore respaldo\textunderscore )}
\end{itemize}
O mesmo que \textunderscore solfar\textunderscore ^2.
Tornar plano ou liso.
\section{Respalde}
\begin{itemize}
\item {Grp. gram.:m.}
\end{itemize}
\begin{itemize}
\item {Utilização:Gír.}
\end{itemize}
Lençol.
\section{Respaldo}
\begin{itemize}
\item {Grp. gram.:m.}
\end{itemize}
\begin{itemize}
\item {Proveniência:(De \textunderscore re...\textunderscore  + \textunderscore espalda\textunderscore )}
\end{itemize}
Acto ou effeito de respaldar.
Espaldar, espalda.
Encôsto, na traseira das carruagens.
Espécie de degrau ou banqueta, em que se colloca o crucifixo e os castiçaes do altar.
Cylindro, com que se respalda um caminho ou estrada.
Callosidade, produzida na cavalgadura pelo roçar do arção posterior da sella.
\section{Respançadura}
\begin{itemize}
\item {Grp. gram.:f.}
\end{itemize}
Acto ou effeito de respançar.
\section{Respançamento}
\begin{itemize}
\item {Grp. gram.:m.}
\end{itemize}
O mesmo que \textunderscore respançadura\textunderscore .
\section{Respançar}
\begin{itemize}
\item {Grp. gram.:v. t.}
\end{itemize}
Apagar com a raspadeira (letras); raspar.
(Por \textunderscore raspançar\textunderscore , de \textunderscore raspar\textunderscore ?)
\section{Réspe}
\begin{itemize}
\item {Grp. gram.:m.}
\end{itemize}
\begin{itemize}
\item {Utilização:Bras}
\end{itemize}
Descompostura; reprehensão.
(Corr. de \textunderscore récipe\textunderscore )
\section{Respectivamente}
\begin{itemize}
\item {Grp. gram.:adv.}
\end{itemize}
De modo respectivo ou recíproco.
Relativamente, com relação.
Com propriedade, na devida ordem.
\section{Respectivo}
\begin{itemize}
\item {Grp. gram.:adj.}
\end{itemize}
\begin{itemize}
\item {Utilização:Ant.}
\end{itemize}
\begin{itemize}
\item {Proveniência:(Do lat. \textunderscore respectus\textunderscore )}
\end{itemize}
Que diz respeito ou é relativo a cada um em particular ou a cada um em separado.
Próprio, pertencente.
Respeitador, respeitoso.
\section{Respecto}
\begin{itemize}
\item {Grp. gram.:m.}
\end{itemize}
\begin{itemize}
\item {Utilização:Ant.}
\end{itemize}
O mesmo que \textunderscore respeito\textunderscore . Cf. \textunderscore Inéd. da Hist. Port.\textunderscore , I, 109.
\section{Respeitabilidade}
\begin{itemize}
\item {Grp. gram.:f.}
\end{itemize}
Qualidade do que é respeitável.
\section{Respeitado}
\begin{itemize}
\item {Grp. gram.:adj.}
\end{itemize}
\begin{itemize}
\item {Proveniência:(De \textunderscore respeitar\textunderscore )}
\end{itemize}
Que é objecto de respeito.
Acatado.
Reverenciado.
Que, por seus méritos, se impõe ao respeito público: \textunderscore escritor respeitado\textunderscore .
\section{Respeitador}
\begin{itemize}
\item {Grp. gram.:m.  e  adj.}
\end{itemize}
O que respeita.
\section{Respeitante}
\begin{itemize}
\item {Grp. gram.:adj.}
\end{itemize}
\begin{itemize}
\item {Utilização:Neol.}
\end{itemize}
\begin{itemize}
\item {Proveniência:(De \textunderscore respeitar\textunderscore )}
\end{itemize}
Que pertence, que diz respeito.
\section{Respeitar}
\begin{itemize}
\item {Grp. gram.:v. t.}
\end{itemize}
\begin{itemize}
\item {Grp. gram.:V. i.}
\end{itemize}
\begin{itemize}
\item {Grp. gram.:V. p.}
\end{itemize}
\begin{itemize}
\item {Proveniência:(Do lat. \textunderscore respectare\textunderscore )}
\end{itemize}
Voltar-se para.
Tratar com acatamento ou reverência; honrar.
Recear.
Têr em conta.
Observar.
Attender.
Aguentar, supportar.
Dizer respeito.
Tomar certa direcção.
Pertencer.
Fazer-se respeitado, tornar-se digno de respeito.
\section{Respeitável}
\begin{itemize}
\item {Grp. gram.:adj.}
\end{itemize}
\begin{itemize}
\item {Utilização:Fig.}
\end{itemize}
\begin{itemize}
\item {Proveniência:(De \textunderscore respeitar\textunderscore )}
\end{itemize}
Digno de respeito; venerável.
Formidável, extraordinário; temeroso: \textunderscore um respeitável cataclismo\textunderscore .
\section{Respeitavelmente}
\begin{itemize}
\item {Grp. gram.:adv.}
\end{itemize}
De modo respeitável.
\section{Respeito}
\begin{itemize}
\item {Grp. gram.:m.}
\end{itemize}
\begin{itemize}
\item {Utilização:Ant.}
\end{itemize}
\begin{itemize}
\item {Grp. gram.:Loc. prep.}
\end{itemize}
\begin{itemize}
\item {Grp. gram.:Loc. prep.}
\end{itemize}
\begin{itemize}
\item {Utilização:P. us.}
\end{itemize}
\begin{itemize}
\item {Grp. gram.:Pl.}
\end{itemize}
\begin{itemize}
\item {Proveniência:(Lat. \textunderscore respectus\textunderscore )}
\end{itemize}
Acto ou effeito de respeitar.
Ponto de vista, aspecto: \textunderscore a êsse respeito, conversaremos\textunderscore .
Causa.
Relação.
Aprêço.
Reverência.
Submissão: \textunderscore têr respeito aos pais\textunderscore .
Receio; temor: \textunderscore a trovoada infunde respeito\textunderscore .
Justiça.
Importância: \textunderscore é uma riqueza de respeito\textunderscore .
Maneira de vêr ou de pensar; opinião, conceito.
\textunderscore Dizer respeito\textunderscore , têr relação, referir-se.
\textunderscore A respeito de\textunderscore , ou \textunderscore com respeito a\textunderscore , relativamente a, com relação a, pelo que toca a.
\textunderscore Respeito a\textunderscore , com relação a. Cf. Camillo, \textunderscore Quéda\textunderscore , 4. ed., 11.
Saudações, cumprimentos: \textunderscore apresento-lhe os meus respeitos\textunderscore .
\section{Respeitosamente}
\begin{itemize}
\item {Grp. gram.:adv.}
\end{itemize}
De modo respeitoso; com acatamento, com respeito.
\section{Respeitoso}
\begin{itemize}
\item {Grp. gram.:adj.}
\end{itemize}
Relativo a respeito.
Que infunde respeito.
Que guarda respeito ou deferência a.
Que mostra respeito ou acatamento: \textunderscore palavras respeitosas\textunderscore .
\section{Respe-respe}
\begin{itemize}
\item {Grp. gram.:adv.}
\end{itemize}
\begin{itemize}
\item {Utilização:Prov.}
\end{itemize}
\begin{itemize}
\item {Utilização:minh.}
\end{itemize}
O mesmo que \textunderscore rés-vés\textunderscore .
\section{Respiga}
\begin{itemize}
\item {Grp. gram.:f.}
\end{itemize}
\begin{itemize}
\item {Utilização:Carp.}
\end{itemize}
Acto ou effeito de respigar.
Encaixe, feito numa peça de madeira, para que nella entre a mecha de outra peça.
\section{Respigadeira}
\begin{itemize}
\item {Grp. gram.:f.  e  adj.}
\end{itemize}
Mulhér, que respiga.
\section{Respigadoira}
\begin{itemize}
\item {Grp. gram.:f.}
\end{itemize}
\begin{itemize}
\item {Proveniência:(De \textunderscore respigar\textunderscore )}
\end{itemize}
Utensílio das fábricas de serração. Cf. \textunderscore Inquér. Industr.\textunderscore , p. II, l. II, 102.
\section{Respigador}
\begin{itemize}
\item {Grp. gram.:m.  e  adj.}
\end{itemize}
O que respiga.
Máquina agrícola, para respigar.
\section{Respigadoura}
\begin{itemize}
\item {Grp. gram.:f.}
\end{itemize}
\begin{itemize}
\item {Proveniência:(De \textunderscore respigar\textunderscore )}
\end{itemize}
Utensílio das fábricas de serração. Cf. \textunderscore Inquér. Industr.\textunderscore , p. II, l. II, 102.
\section{Respigadura}
\begin{itemize}
\item {Grp. gram.:f.}
\end{itemize}
O mesmo que \textunderscore respiga\textunderscore .
\section{Respigão}
\begin{itemize}
\item {Grp. gram.:m.}
\end{itemize}
O mesmo que \textunderscore espigão\textunderscore  das unhas.
\section{Respigar}
\begin{itemize}
\item {Grp. gram.:v. i.}
\end{itemize}
\begin{itemize}
\item {Grp. gram.:V. t.}
\end{itemize}
\begin{itemize}
\item {Utilização:Fig.}
\end{itemize}
\begin{itemize}
\item {Proveniência:(De \textunderscore re...\textunderscore  + \textunderscore espiga\textunderscore )}
\end{itemize}
Apanhar as espigas que os ceifadores não cortaram, ou que ficaram no campo depois da ceifa.
Apanhar áquem e álem; colligir; compilar: \textunderscore respigar notícias históricas\textunderscore .
\section{Respigo}
\begin{itemize}
\item {Grp. gram.:m.}
\end{itemize}
O mesmo que \textunderscore respiga\textunderscore . Cf. Garrett, \textunderscore Camões\textunderscore , 16.
\section{Respingador}
\begin{itemize}
\item {Grp. gram.:m.  e  adj.}
\end{itemize}
\begin{itemize}
\item {Proveniência:(De \textunderscore respingar\textunderscore ^1)}
\end{itemize}
O que respinga.
\section{Respingão}
\begin{itemize}
\item {Grp. gram.:m.  e  adj.}
\end{itemize}
\begin{itemize}
\item {Proveniência:(De \textunderscore respingar\textunderscore ^1)}
\end{itemize}
O que respinga.
\section{Respingar}
\begin{itemize}
\item {Grp. gram.:v. i.}
\end{itemize}
\begin{itemize}
\item {Proveniência:(Do rad. de \textunderscore responder\textunderscore )}
\end{itemize}
Responder com maus modos.
Rezingar; recalcitrar.
Dar coices.
\section{Respingar}
\begin{itemize}
\item {Grp. gram.:v. i.}
\end{itemize}
\begin{itemize}
\item {Proveniência:(De \textunderscore res...\textunderscore  + \textunderscore pingo\textunderscore )}
\end{itemize}
Deitar borrifos ou pingos (a água).
Deitar faíscas (o lume); crepitar.
\section{Respingo}
\begin{itemize}
\item {Grp. gram.:m.}
\end{itemize}
Acto ou effeito de respingar.
\section{Respinhar}
\textunderscore v. t.\textunderscore  (e der.)
(V. \textunderscore raspinhar\textunderscore , etc.). Cf. \textunderscore Museu Techn.\textunderscore , 80.
\section{Respirabilidade}
\begin{itemize}
\item {Grp. gram.:f.}
\end{itemize}
Qualidade do que é respirável.
\section{Respiração}
\begin{itemize}
\item {Grp. gram.:f.}
\end{itemize}
\begin{itemize}
\item {Utilização:Bot.}
\end{itemize}
\begin{itemize}
\item {Proveniência:(Do lat. \textunderscore respiratio\textunderscore )}
\end{itemize}
Acto ou effeito de respirar.
Ambiente.
Bafo.
Funcção da vida das plantas, pela qual ellas absorvem ácido carbónico e exhalam oxygênio ou viceversa.
\section{Respiráculo}
\begin{itemize}
\item {Grp. gram.:m.}
\end{itemize}
\begin{itemize}
\item {Utilização:P. us.}
\end{itemize}
\begin{itemize}
\item {Proveniência:(Lat. \textunderscore respiraculum\textunderscore )}
\end{itemize}
Acto de respirar; respiração. Cf. Camillo, \textunderscore Ôlho de Vidro\textunderscore , 18.
\section{Respiradoiro}
\begin{itemize}
\item {Grp. gram.:m.}
\end{itemize}
\begin{itemize}
\item {Proveniência:(De \textunderscore respirar\textunderscore )}
\end{itemize}
Lugar, por onde entra e sai o ar.
Orifício.
Resfolegadoiro.
\section{Respirador}
\begin{itemize}
\item {Grp. gram.:adj.}
\end{itemize}
\begin{itemize}
\item {Grp. gram.:M.}
\end{itemize}
\begin{itemize}
\item {Proveniência:(De \textunderscore respirar\textunderscore )}
\end{itemize}
Que serve para a respiração.
Instrumento, que facilita a respiração.
\section{Respiradouro}
\begin{itemize}
\item {Grp. gram.:m.}
\end{itemize}
\begin{itemize}
\item {Proveniência:(De \textunderscore respirar\textunderscore )}
\end{itemize}
Lugar, por onde entra e sai o ar.
Orifício.
Resfolegadoiro.
\section{Respiramento}
\begin{itemize}
\item {Grp. gram.:m.}
\end{itemize}
\begin{itemize}
\item {Utilização:Fig.}
\end{itemize}
\begin{itemize}
\item {Proveniência:(Lat. \textunderscore respiramentum\textunderscore )}
\end{itemize}
O mesmo que \textunderscore respiração\textunderscore .
Folga, respiro.
\section{Respirar}
\begin{itemize}
\item {Grp. gram.:v. i.}
\end{itemize}
\begin{itemize}
\item {Grp. gram.:V. t.}
\end{itemize}
\begin{itemize}
\item {Proveniência:(Lat. \textunderscore respirare\textunderscore )}
\end{itemize}
Absorver o ar nos pulmões e expelli-lo em seguida.
Viver: \textunderscore prometeu amá-la em-quanto respirar\textunderscore .
Manifestar-se.
Exhalar-se.
Folgar; descansar.
Absorver e expellir (o ar).
Exhalar.
Lançar fóra, expellir.
Manifestar.
Estar impregnado de: \textunderscore o gabinete respirava aromas penetrantes\textunderscore .
Mostrar desejos de.
Alimentar-se com.
Aspirar a.
\section{Respiratório}
\begin{itemize}
\item {Grp. gram.:adj.}
\end{itemize}
\begin{itemize}
\item {Proveniência:(De \textunderscore respirar\textunderscore )}
\end{itemize}
Relativo á respiração; que auxilia a respiração.
\section{Respirável}
\begin{itemize}
\item {Grp. gram.:adj.}
\end{itemize}
Que se póde respirar; favorável á respiração: \textunderscore atmosphera respirável\textunderscore .
\section{Respiro}
\begin{itemize}
\item {Grp. gram.:m.}
\end{itemize}
\begin{itemize}
\item {Utilização:Fig.}
\end{itemize}
\begin{itemize}
\item {Utilização:Des.}
\end{itemize}
\begin{itemize}
\item {Proveniência:(De \textunderscore respirar\textunderscore )}
\end{itemize}
O mesmo que \textunderscore respiração\textunderscore .
Folga.
Espera, concedida por um crèdor, do pagamento de uma dívida.
Abertura nos fornos, que dá saída ao fumo.
Respiradoiro.
Orifício, por onde sái o ar ou um líquido.
\section{Resplandecência}
\begin{itemize}
\item {Grp. gram.:f.}
\end{itemize}
\begin{itemize}
\item {Proveniência:(De \textunderscore resplandecente\textunderscore )}
\end{itemize}
Acto ou effeito de resplandecer.
\section{Resplandecente}
\begin{itemize}
\item {Grp. gram.:adj.}
\end{itemize}
Que resplandece; muito brilhante; esplêndido.
\section{Resplandecentemente}
\begin{itemize}
\item {Grp. gram.:adv.}
\end{itemize}
De modo resplandecente.
\section{Resplandecer}
\begin{itemize}
\item {Grp. gram.:v. i.}
\end{itemize}
\begin{itemize}
\item {Utilização:Fig.}
\end{itemize}
\begin{itemize}
\item {Grp. gram.:V. t.}
\end{itemize}
\begin{itemize}
\item {Utilização:P. us.}
\end{itemize}
Brilhar muito, intensamente.
Rutilar.
Manifestar-se brilhantemente, com esplendor.
Engrandecer-se, sêr notável por virtudes ou outros merecimentos.
Fazer brilhar, fazer sobresaír.
(Alter. de \textunderscore resplendecer\textunderscore )
\section{Resplandente}
\begin{itemize}
\item {Grp. gram.:adj.}
\end{itemize}
O mesmo que \textunderscore resplendente\textunderscore . Cf. B. Pato, \textunderscore Cant. e Sát.\textunderscore , 72.
\section{Resplandimento}
\begin{itemize}
\item {Grp. gram.:m.}
\end{itemize}
\begin{itemize}
\item {Utilização:Ant.}
\end{itemize}
O mesmo que \textunderscore resplandor\textunderscore . Cf. S. R. Viterbo, \textunderscore Elucidário\textunderscore .
\section{Resplandor}
\begin{itemize}
\item {Grp. gram.:m.}
\end{itemize}
(Corr. de \textunderscore resplendor\textunderscore )
\section{Resplendecer}
\begin{itemize}
\item {Grp. gram.:v. t.  e  i.}
\end{itemize}
\begin{itemize}
\item {Proveniência:(De \textunderscore resplender\textunderscore )}
\end{itemize}
O mesmo ou melhor que \textunderscore resplandecer\textunderscore  mas menos usado.
\section{Resplendência}
\begin{itemize}
\item {Grp. gram.:f.}
\end{itemize}
\begin{itemize}
\item {Proveniência:(Lat. \textunderscore resplendentia\textunderscore )}
\end{itemize}
Qualidade de resplendente. Cf. Alv. Mendes, \textunderscore Discursos\textunderscore , 238.
\section{Resplendente}
\begin{itemize}
\item {Grp. gram.:adj.}
\end{itemize}
\begin{itemize}
\item {Proveniência:(Lat. \textunderscore resplendens\textunderscore )}
\end{itemize}
Que resplende; resplandecente; rutilante.
\section{Resplender}
\begin{itemize}
\item {Grp. gram.:v. i.}
\end{itemize}
\begin{itemize}
\item {Proveniência:(Lat. \textunderscore resplendere\textunderscore )}
\end{itemize}
O mesmo que \textunderscore resplendecer\textunderscore .
\section{Resplêndido}
\begin{itemize}
\item {Grp. gram.:adj.}
\end{itemize}
\begin{itemize}
\item {Proveniência:(De \textunderscore re...\textunderscore  + \textunderscore esplêndido\textunderscore )}
\end{itemize}
Muito esplêndido. Cf. Filinto, XIV, 134 e 173; XV, 218.
\section{Resplendor}
\begin{itemize}
\item {Grp. gram.:m.}
\end{itemize}
\begin{itemize}
\item {Utilização:Fig.}
\end{itemize}
\begin{itemize}
\item {Proveniência:(Lat. \textunderscore resplendor\textunderscore )}
\end{itemize}
Acto ou effeito de resplender.
Claridade intensa.
Auréola, corôa luminosa.
Nimbo.
Celebridade, glória.
\section{Resplendoroso}
\begin{itemize}
\item {Grp. gram.:adj.}
\end{itemize}
Que tem resplendor; resplêndido. Cf. Camillo, \textunderscore Estrêl. Fun.\textunderscore , 174.
\section{Respo}
\begin{itemize}
\item {fónica:rês}
\end{itemize}
\begin{itemize}
\item {Grp. gram.:m.}
\end{itemize}
\begin{itemize}
\item {Utilização:Gír.}
\end{itemize}
Excremento humano.
\section{Respondão}
\begin{itemize}
\item {Grp. gram.:m.  e  adj.}
\end{itemize}
\begin{itemize}
\item {Proveniência:(De \textunderscore responder\textunderscore )}
\end{itemize}
Respingão, rezingueiro.
Aquelle que falta ao respeito devido a outrem, respondendo-lhe ou replicando-lhe com maus modos.
\section{Respondedor}
\begin{itemize}
\item {Grp. gram.:m.  e  adj.}
\end{itemize}
O que responde; respondão.
\section{Respondência}
\begin{itemize}
\item {Grp. gram.:f.}
\end{itemize}
\begin{itemize}
\item {Proveniência:(De \textunderscore responder\textunderscore )}
\end{itemize}
O mesmo que \textunderscore correspondência\textunderscore .
Relações.
Lucro mercantil:«\textunderscore ...lavores que, com a conformidade e respondência que entre si tinhão...\textunderscore »Sousa, \textunderscore Vida do Arceb.\textunderscore , III, 322.
\textunderscore Fazer respondência\textunderscore , corresponder, fazer symetria. Cf. Sousa, \textunderscore Vida do Arceb.\textunderscore , V, 26.
\section{Respondente}
\begin{itemize}
\item {Grp. gram.:adj.}
\end{itemize}
\begin{itemize}
\item {Grp. gram.:M.  e  f.}
\end{itemize}
\begin{itemize}
\item {Utilização:Jur.}
\end{itemize}
\begin{itemize}
\item {Proveniência:(De \textunderscore responder\textunderscore )}
\end{itemize}
Que responde.
Pessôa, que depõe, sendo inquirida por artigos.
\section{Responder}
\begin{itemize}
\item {Grp. gram.:v. t.}
\end{itemize}
\begin{itemize}
\item {Grp. gram.:V. i.}
\end{itemize}
\begin{itemize}
\item {Proveniência:(Lat. \textunderscore respondere\textunderscore )}
\end{itemize}
Dizer ou escrever em resposta.
Replicar, retorquir.
Objectar.
Dar resposta.
Corresponder.
O mesmo quo \textunderscore responsabilizar-se\textunderscore : \textunderscore responder por uma dívida de outrem\textunderscore .
Proceder da mesma fórma.
Equivaler.
Oppor-se.
Estar defronte de.
Respingar^1.
\section{Respondível}
\begin{itemize}
\item {Grp. gram.:adj.}
\end{itemize}
A que se póde responder.
\section{Respondona}
\begin{itemize}
\item {Grp. gram.:f.}
\end{itemize}
(Fem. de \textunderscore respondão\textunderscore )
\section{Responsabilidade}
\begin{itemize}
\item {Grp. gram.:f.}
\end{itemize}
Qualidade do que é responsável.
Obrigação de responder por certos actos ou factos: \textunderscore a responsabilidade de um fiador\textunderscore .
\section{Responsabilizar}
\begin{itemize}
\item {Grp. gram.:v. t.}
\end{itemize}
\begin{itemize}
\item {Grp. gram.:V. p.}
\end{itemize}
Tornar responsável; imputar responsabilidade a.
Tornar-se responsável.
\section{Responsão}
\begin{itemize}
\item {Grp. gram.:f.}
\end{itemize}
\begin{itemize}
\item {Utilização:Ant.}
\end{itemize}
\begin{itemize}
\item {Proveniência:(Do lat. \textunderscore responsum\textunderscore )}
\end{itemize}
O mesmo que \textunderscore resposta\textunderscore . Cf. S. R. Viterbo, \textunderscore Elucidário\textunderscore .
\section{Responsar}
\begin{itemize}
\item {Grp. gram.:v. t.}
\end{itemize}
\begin{itemize}
\item {Utilização:Pop.}
\end{itemize}
\begin{itemize}
\item {Utilização:Pop.}
\end{itemize}
\begin{itemize}
\item {Utilização:Prov.}
\end{itemize}
\begin{itemize}
\item {Proveniência:(Do lat. \textunderscore responsare\textunderscore )}
\end{itemize}
Rezar responsos por.
Falar mal de (alguém).
Rezar a Santo-António, para que (alguma coisa perdida) reappareça.
Tratar mal, rogar pragas a.
\section{Responsável}
\begin{itemize}
\item {Grp. gram.:adj.}
\end{itemize}
\begin{itemize}
\item {Grp. gram.:M.}
\end{itemize}
\begin{itemize}
\item {Proveniência:(Lat. \textunderscore resposum\textunderscore )}
\end{itemize}
Que tem de cumprir obrigações suas ou alheias.
Que responde pelos seus actos ou pelos de outrem.
Que tem compromissos.
Indivíduo, que é responsável.
\section{Responsivo}
\begin{itemize}
\item {Grp. gram.:adj.}
\end{itemize}
\begin{itemize}
\item {Proveniência:(Lat. \textunderscore responsivus\textunderscore )}
\end{itemize}
Que envolve resposta.
\section{Responso}
\begin{itemize}
\item {Grp. gram.:m.}
\end{itemize}
\begin{itemize}
\item {Utilização:Fam.}
\end{itemize}
\begin{itemize}
\item {Utilização:Ant.}
\end{itemize}
\begin{itemize}
\item {Proveniência:(Lat. \textunderscore responsum\textunderscore )}
\end{itemize}
Versículos religiosos, que se rezam ou cantam depois das lições dos officios divinos.
Oração a Santo António, para que se encontrem as coisas perdidas, ou para que não succeda um mal que se receia.
Descompostura.
Murmuração.
O mesmo que \textunderscore resposta\textunderscore . Cf. Frei Fortun., \textunderscore Inéditos\textunderscore , 314.
\section{Responsório}
\begin{itemize}
\item {Grp. gram.:m.}
\end{itemize}
Collecção de responsos.
\section{Resposta}
\begin{itemize}
\item {Grp. gram.:f.}
\end{itemize}
Aquillo que se diz ou escreve áquelle, que fez uma pergunta ou formulou uma questão.
Réplica; refutação.
Solução.
Carta, que se envia a quem mandou outra e que se refere ao conteúdo desta: \textunderscore mando-lhe a resposta pelo portador\textunderscore .
Bote de florete, espada ou sabre, em seguida e em trôco ao bote do adversário.
Cada uma das bombas de um foguete: \textunderscore deitar foguetes de três respostas\textunderscore .
(Corr. de \textunderscore reposta\textunderscore , fem. de \textunderscore reposto\textunderscore , por infl. de \textunderscore responder\textunderscore )
\section{Respostada}
\begin{itemize}
\item {Grp. gram.:f.}
\end{itemize}
Resposta incivil.
\section{Resquiado}
\begin{itemize}
\item {Grp. gram.:adj.}
\end{itemize}
\begin{itemize}
\item {Utilização:Prov.}
\end{itemize}
\begin{itemize}
\item {Utilização:trasm.}
\end{itemize}
Medido escassamente, rés-vés, sem demasia nenhuma, á justa.
(Participa de \textunderscore rés\textunderscore )
\section{Resquício}
\begin{itemize}
\item {Grp. gram.:m.}
\end{itemize}
Vestígio; resíduo; fragmentos muito miúdos.
Pequena abertura.
\section{Ressa}
\begin{itemize}
\item {Grp. gram.:f.}
\end{itemize}
\begin{itemize}
\item {Utilização:Prov.}
\end{itemize}
\begin{itemize}
\item {Utilização:trasm.}
\end{itemize}
\begin{itemize}
\item {Utilização:minh.}
\end{itemize}
\begin{itemize}
\item {Utilização:Prov.}
\end{itemize}
\begin{itemize}
\item {Utilização:trasm.}
\end{itemize}
Calor do sol; sòlheira. Cf. Camillo, \textunderscore Cav. em Ruínas\textunderscore , 145.
O mesmo que \textunderscore soalheira\textunderscore .
\section{Ressaber}
\begin{itemize}
\item {Grp. gram.:v. t.}
\end{itemize}
\begin{itemize}
\item {Grp. gram.:V. i.}
\end{itemize}
\begin{itemize}
\item {Proveniência:(De \textunderscore re...\textunderscore  + \textunderscore saber\textunderscore )}
\end{itemize}
Saber bem, perfeitamente.
Têr sabor muito pronunciado.
Têr sabor que faz lembrar outro.
\section{Ressabiado}
\begin{itemize}
\item {Grp. gram.:adj.}
\end{itemize}
Que ressabia; espantadiço, desconfiado: \textunderscore cavallo ressabiado\textunderscore .
\section{Ressábiar}
\begin{itemize}
\item {Grp. gram.:v. i.}
\end{itemize}
\begin{itemize}
\item {Utilização:Fig.}
\end{itemize}
\begin{itemize}
\item {Proveniência:(De \textunderscore ressábio\textunderscore )}
\end{itemize}
Tomar ressaibo.
Melindrar-se, ressentir-se.
\section{Ressabido}
\begin{itemize}
\item {Grp. gram.:adj.}
\end{itemize}
Que sabe muito, que é erudito; experimentado.
\section{Ressábio}
\begin{itemize}
\item {Grp. gram.:m.}
\end{itemize}
(Metáth. de \textunderscore ressaibo\textunderscore . Cf. Filinto, \textunderscore D. Man.\textunderscore , I, 90)
\section{Ressaborear}
\begin{itemize}
\item {Grp. gram.:v. t.}
\end{itemize}
\begin{itemize}
\item {Proveniência:(De \textunderscore re...\textunderscore  + \textunderscore saborear\textunderscore )}
\end{itemize}
Saborear muito; apreciar em alto grau. Cf. Eça, \textunderscore P. Amaro\textunderscore , 420.
\section{Ressaca}
\begin{itemize}
\item {Grp. gram.:f.}
\end{itemize}
\begin{itemize}
\item {Utilização:Ant.}
\end{itemize}
\begin{itemize}
\item {Utilização:Fig.}
\end{itemize}
\begin{itemize}
\item {Proveniência:(De \textunderscore re...\textunderscore  + \textunderscore sacar\textunderscore )}
\end{itemize}
Movimento, feito pelas ondas, quando se desviam da praia.
Pôrto, formado pela preamar.
Fluxo e refluxo.
O mesmo que \textunderscore rètaguarda\textunderscore . Cf. \textunderscore Peregrinação\textunderscore , CL.
Volubilidade.
\section{Ressacada}
\begin{itemize}
\item {Grp. gram.:f.}
\end{itemize}
\begin{itemize}
\item {Utilização:T. de bandeirantes}
\end{itemize}
Grande ressaca, á beira dos rios.
\section{Ressacar}
\begin{itemize}
\item {Grp. gram.:v. t.}
\end{itemize}
\begin{itemize}
\item {Proveniência:(De \textunderscore re...\textunderscore  + \textunderscore sacar\textunderscore )}
\end{itemize}
Fazer o ressaque de (letra de câmbio).
\section{Ressaibo}
\begin{itemize}
\item {Grp. gram.:m.}
\end{itemize}
\begin{itemize}
\item {Utilização:Fig.}
\end{itemize}
\begin{itemize}
\item {Proveniência:(De \textunderscore re...\textunderscore  + \textunderscore saibo\textunderscore )}
\end{itemize}
Mau saibo.
Ranço.
Sabor, resultante de uma substância que adheriu ao vaso por onde se bebe ou come.
Resentimento.
Indício.
\section{Ressaio}
\begin{itemize}
\item {Grp. gram.:m.}
\end{itemize}
\begin{itemize}
\item {Proveniência:(De \textunderscore resair\textunderscore )}
\end{itemize}
Terreiro, á beira de uma casa; rossío. Cf. Camillo, \textunderscore Doze Casam.\textunderscore , 14; S. R. Viterbo, \textunderscore Elucidário\textunderscore .
\section{Ressair}
\begin{itemize}
\item {Grp. gram.:v. i.}
\end{itemize}
\begin{itemize}
\item {Proveniência:(De \textunderscore re...\textunderscore  + \textunderscore saír\textunderscore )}
\end{itemize}
Sair de novo.
Saír acima.
Resaltar.
Sobresaír.
Mostrar-se saliente; avultar.
\section{Ressaltar}
\begin{itemize}
\item {Grp. gram.:v. t.}
\end{itemize}
\begin{itemize}
\item {Grp. gram.:V. i.}
\end{itemize}
\begin{itemize}
\item {Proveniência:(De \textunderscore re...\textunderscore  + \textunderscore saltar\textunderscore )}
\end{itemize}
Tornar saliente; dar relêvo a.
Altear.
Dar muitos saltos.
Sobresair, elevar-se; resair.
\section{Ressalte}
\begin{itemize}
\item {Grp. gram.:m.}
\end{itemize}
Acto ou effeito de ressaltar.
Ressalto; saliência. Cf. Alv. Mendes, \textunderscore Discursos\textunderscore , 207.
\section{Ressaltear}
\begin{itemize}
\item {Grp. gram.:v. t.}
\end{itemize}
\begin{itemize}
\item {Proveniência:(De \textunderscore re...\textunderscore  + \textunderscore saltear\textunderscore )}
\end{itemize}
Saltear de novo.
\section{Ressalto}
\begin{itemize}
\item {Grp. gram.:m.}
\end{itemize}
Acto ou efeito de ressaltar.
Relêvo; saliência.
\section{Ressalva}
\begin{itemize}
\item {Grp. gram.:f.}
\end{itemize}
\begin{itemize}
\item {Proveniência:(De \textunderscore ressalvar\textunderscore )}
\end{itemize}
Certidão, donde consta que um indivíduo se isentou do serviço militar.
Documento, para segurança de alguém.
Excepção.
Cláusula.
Nota, para corrigir o que se escreveu; errata.
\section{Ressalvar}
\begin{itemize}
\item {Grp. gram.:v. t.}
\end{itemize}
\begin{itemize}
\item {Proveniência:(Lat. \textunderscore resalvare\textunderscore )}
\end{itemize}
Dar ressalva a.
Segurar com ressalva.
Fazer ressalva em.
Exceptuar; eximir.
Acautelar; pôr a salvo.
\section{Ressanfoninar}
\begin{itemize}
\item {Grp. gram.:v. i.}
\end{itemize}
\begin{itemize}
\item {Utilização:Ant.}
\end{itemize}
\begin{itemize}
\item {Proveniência:(De \textunderscore re...\textunderscore  + \textunderscore sanfoninar\textunderscore )}
\end{itemize}
Repetir impertinências, zombando:«\textunderscore ...quereis ressanfoninar com a minha dôr.\textunderscore »\textunderscore Eufrosina\textunderscore , 24.
\section{Ressangrar}
\begin{itemize}
\item {Grp. gram.:v. t.}
\end{itemize}
\begin{itemize}
\item {Proveniência:(De \textunderscore re...\textunderscore  + \textunderscore sangrar\textunderscore )}
\end{itemize}
Sangrar de novo, tirar muito sangue a. Cf. Alv. Mendes, \textunderscore Discursos\textunderscore , 266.
\section{Ressaque}
\begin{itemize}
\item {Grp. gram.:m.}
\end{itemize}
\begin{itemize}
\item {Utilização:Jur.}
\end{itemize}
\begin{itemize}
\item {Utilização:Jur.}
\end{itemize}
\begin{itemize}
\item {Utilização:Des.}
\end{itemize}
\begin{itemize}
\item {Proveniência:(De \textunderscore re...\textunderscore  + \textunderscore saque\textunderscore )}
\end{itemize}
Saque de uma nova letra de câmbio.
Segunda letra de câmbio, pela qual o portador se embolsa sôbre o sacador ou indossador de outra letra protestada, do principal desta. Cf. F. Borges, \textunderscore Diccion. Jur.\textunderscore 
\section{Ressaque}
\begin{itemize}
\item {Grp. gram.:m.}
\end{itemize}
\begin{itemize}
\item {Utilização:Prov.}
\end{itemize}
O mesmo que \textunderscore ressaca\textunderscore .
\section{Ressarcimento}
\begin{itemize}
\item {Grp. gram.:m.}
\end{itemize}
Acto ou efeito de ressarcir.
\section{Ressarcir}
\begin{itemize}
\item {Grp. gram.:v. t.}
\end{itemize}
\begin{itemize}
\item {Proveniência:(Lat. \textunderscore resarcire\textunderscore )}
\end{itemize}
Compensar, indemnizar; refazer; melhorar.
\section{Ressaudação}
\begin{itemize}
\item {fónica:sa-u}
\end{itemize}
\begin{itemize}
\item {Grp. gram.:f.}
\end{itemize}
Acto ou efeito de saudar.
\section{Ressaudar}
\begin{itemize}
\item {fónica:sa-u}
\end{itemize}
\begin{itemize}
\item {Grp. gram.:v. t.}
\end{itemize}
\begin{itemize}
\item {Grp. gram.:V. i.}
\end{itemize}
\begin{itemize}
\item {Proveniência:(Do lat. \textunderscore resalutare\textunderscore )}
\end{itemize}
Tornar a saudar.
Saudar mutuamente.
Corresponder á saudação de alguém.
\section{Ressecação}
\begin{itemize}
\item {Grp. gram.:f.}
\end{itemize}
Acto ou efeito de ressecar.
\section{Ressecar}
\begin{itemize}
\item {Grp. gram.:v. t.}
\end{itemize}
\begin{itemize}
\item {Proveniência:(De \textunderscore re...\textunderscore  + \textunderscore secar\textunderscore )}
\end{itemize}
Tornar a secar.
Secar bem.
Sujeitar á evaporação.
\section{Ressecção}
\begin{itemize}
\item {Grp. gram.:f.}
\end{itemize}
\begin{itemize}
\item {Proveniência:(Lat. \textunderscore resectio\textunderscore )}
\end{itemize}
Operação cirúrgica, que consiste em cortar uma parte, mais ou menos extensa, de um órgão.
\section{Ressêco}
\begin{itemize}
\item {Grp. gram.:adj.}
\end{itemize}
\begin{itemize}
\item {Proveniência:(De \textunderscore re...\textunderscore  + \textunderscore sêco\textunderscore )}
\end{itemize}
Muito sêco.
\section{Ressegar}
\begin{itemize}
\item {Grp. gram.:v. t.}
\end{itemize}
\begin{itemize}
\item {Proveniência:(Do lat. \textunderscore resecare\textunderscore )}
\end{itemize}
Segar novamente.
\section{Ressegundar}
\begin{itemize}
\item {Grp. gram.:v. t.}
\end{itemize}
\begin{itemize}
\item {Proveniência:(De \textunderscore re...\textunderscore  + \textunderscore segundar\textunderscore )}
\end{itemize}
Repetir muitas vezes. Cf. \textunderscore Aulegrafia\textunderscore , 31.
\section{Ressegurar}
\begin{itemize}
\item {Grp. gram.:v. t.}
\end{itemize}
\begin{itemize}
\item {Proveniência:(De \textunderscore re...\textunderscore  + \textunderscore segurar\textunderscore )}
\end{itemize}
Pôr novamente em seguro (um prédio, uma mercadoria, etc.).
\section{Resseguro}
\begin{itemize}
\item {Grp. gram.:m.}
\end{itemize}
\begin{itemize}
\item {Grp. gram.:Adj.}
\end{itemize}
\begin{itemize}
\item {Proveniência:(De \textunderscore re...\textunderscore  + \textunderscore seguro\textunderscore )}
\end{itemize}
Renovação de um seguro de prédios, de vidas, de mercadorias, etc.
Acto de ressegurar.
Novamente seguro; muito seguro; firmíssimo. Cf. Castilho, \textunderscore Geórgicas\textunderscore , 55.
\section{Resselar}
\begin{itemize}
\item {Grp. gram.:v. t.}
\end{itemize}
\begin{itemize}
\item {Proveniência:(De \textunderscore re...\textunderscore  + \textunderscore selar\textunderscore ^2)}
\end{itemize}
Pôr novo sêllo em.
\section{Ressemeadura}
\begin{itemize}
\item {Grp. gram.:f.}
\end{itemize}
Acto ou effeito de ressemear.
\section{Ressemear}
\begin{itemize}
\item {Grp. gram.:v. t.}
\end{itemize}
\begin{itemize}
\item {Proveniência:(De \textunderscore re...\textunderscore  + \textunderscore semear\textunderscore )}
\end{itemize}
Semear novamente.
\section{Ressenhor}
\begin{itemize}
\item {Grp. gram.:m.}
\end{itemize}
\begin{itemize}
\item {Utilização:Des.}
\end{itemize}
\begin{itemize}
\item {Proveniência:(De \textunderscore re...\textunderscore  + \textunderscore senhor\textunderscore )}
\end{itemize}
Duas vezes senhor:«\textunderscore é o senhor sôbre senhor, ressenhor, senhor dinheiro\textunderscore ». A. Prestes, \textunderscore Auto do Desembargador\textunderscore .
\section{Ressentido}
\begin{itemize}
\item {Grp. gram.:adj.}
\end{itemize}
\begin{itemize}
\item {Utilização:Pop.}
\end{itemize}
Que se ressentiu; melindrado.
Que começa a apodrecer, (falando-se de frutos).
\section{Ressentimento}
\begin{itemize}
\item {Grp. gram.:m.}
\end{itemize}
Acto ou efeito de ressentir.
\section{Ressentir}
\begin{itemize}
\item {Grp. gram.:v. t.}
\end{itemize}
\begin{itemize}
\item {Grp. gram.:V. p.}
\end{itemize}
\begin{itemize}
\item {Proveniência:(De \textunderscore re...\textunderscore  + \textunderscore sentir\textunderscore )}
\end{itemize}
Sentir de novo.
Mostrar-se ofendido; melindrar-se.
Dar fé.
Sentir as consequências de alguma coisa: \textunderscore a fruta ressente-se do calor\textunderscore .
\section{Ressequir}
\begin{itemize}
\item {Grp. gram.:v. t.}
\end{itemize}
\begin{itemize}
\item {Proveniência:(De \textunderscore ressêco\textunderscore )}
\end{itemize}
Secar muito; fazer perder o suco ou a humidade a.
\section{Resserenar}
\begin{itemize}
\item {Grp. gram.:v. t.}
\end{itemize}
\begin{itemize}
\item {Proveniência:(De \textunderscore re...\textunderscore  + \textunderscore serenar\textunderscore )}
\end{itemize}
Tornar muito sereno; acalmar inteiramente. Cf. Castilho, \textunderscore Fastos\textunderscore , II, 171.
\section{Ressereno}
\begin{itemize}
\item {Grp. gram.:adj.}
\end{itemize}
\begin{itemize}
\item {Proveniência:(De \textunderscore re...\textunderscore  + \textunderscore serenar\textunderscore )}
\end{itemize}
Muito sereno; perfeitamente calmo.
Que readquiriu tranquilidade. Cf. Castilho, \textunderscore Geórgicas\textunderscore , 51.
\section{Resservir}
\begin{itemize}
\item {Grp. gram.:v. t.}
\end{itemize}
\begin{itemize}
\item {Proveniência:(De \textunderscore re...\textunderscore  + \textunderscore servir\textunderscore )}
\end{itemize}
Servir de novo.
\section{Ressesso}
\begin{itemize}
\item {fónica:sê}
\end{itemize}
\begin{itemize}
\item {Grp. gram.:adj.}
\end{itemize}
\begin{itemize}
\item {Utilização:Pop.}
\end{itemize}
Ressêco; endurecido por têr secado, (falando-se de pão ou de bolos).
(Corr. de \textunderscore resêco\textunderscore ? ou de \textunderscore re...\textunderscore  + lat. \textunderscore sessum\textunderscore , de \textunderscore sedere\textunderscore ?)
\section{Resseu}
\begin{itemize}
\item {Grp. gram.:pron.}
\end{itemize}
\begin{itemize}
\item {Proveniência:(De \textunderscore re...\textunderscore  + \textunderscore seu\textunderscore )}
\end{itemize}
Muito seu. Cf. Castilho, \textunderscore Tartufo\textunderscore , 159.
\section{Ressicação}
\begin{itemize}
\item {Grp. gram.:f.}
\end{itemize}
Acto ou effeito de ressiccar.
\section{Ressicar}
\begin{itemize}
\item {Grp. gram.:v. t.}
\end{itemize}
\begin{itemize}
\item {Proveniência:(Lat. \textunderscore resiccare\textunderscore )}
\end{itemize}
Tornar muito sêco, ressequir.
\section{Ressio}
\begin{itemize}
\item {Grp. gram.:m.}
\end{itemize}
\begin{itemize}
\item {Utilização:Ant.}
\end{itemize}
O mesmo ou melhor que \textunderscore rossio\textunderscore :«\textunderscore ...forão aposentados nos estaos do Ressio\textunderscore ». R. Pina, \textunderscore Aff. V\textunderscore , c. CXXXI. Cf. Castilho, \textunderscore Geórg.\textunderscore , 113.
(Relaciona-se com \textunderscore ressa\textunderscore ?)
\section{Ressoante}
\begin{itemize}
\item {Grp. gram.:adj.}
\end{itemize}
\begin{itemize}
\item {Proveniência:(Lat. \textunderscore resonans\textunderscore )}
\end{itemize}
Que ressôa.
\section{Ressoar}
\begin{itemize}
\item {Grp. gram.:v. t.}
\end{itemize}
\begin{itemize}
\item {Utilização:Fig.}
\end{itemize}
\begin{itemize}
\item {Grp. gram.:V. i.}
\end{itemize}
\begin{itemize}
\item {Proveniência:(Do lat. \textunderscore ressonare\textunderscore )}
\end{itemize}
Entoar; repercutir.
Cantar; tocar.
Soar de novo.
Repercutir-se.
Ecoar.
Estrondear.
Tornar a dizer-se.
Fazer-se ouvir.
Espalhar-se como boato.
\section{Ressobrar}
\begin{itemize}
\item {Grp. gram.:v. i.}
\end{itemize}
\begin{itemize}
\item {Proveniência:(De \textunderscore re...\textunderscore  + \textunderscore sobrar\textunderscore )}
\end{itemize}
Sobrar excessivamente.
\section{Ressoca}
\begin{itemize}
\item {Grp. gram.:f.}
\end{itemize}
\begin{itemize}
\item {Utilização:Bras}
\end{itemize}
\begin{itemize}
\item {Proveniência:(De \textunderscore re...\textunderscore  + \textunderscore soca\textunderscore )}
\end{itemize}
Terceiro córte da cana de açúcar.
Segundos rebentos da cana.
\section{Ressoldar}
\begin{itemize}
\item {Grp. gram.:v. t.}
\end{itemize}
\begin{itemize}
\item {Proveniência:(De \textunderscore re...\textunderscore  + \textunderscore soldar\textunderscore )}
\end{itemize}
Soldar novamente.
Soldar bem.
\section{Ressolho}
\begin{itemize}
\item {fónica:sô}
\end{itemize}
\begin{itemize}
\item {Grp. gram.:m.}
\end{itemize}
\begin{itemize}
\item {Utilização:Prov.}
\end{itemize}
\begin{itemize}
\item {Utilização:dur.}
\end{itemize}
Pego, que redemoínha, no rio Doiro, por occasião das cheias.
\section{Ressolto}
\begin{itemize}
\item {fónica:sôl}
\end{itemize}
\begin{itemize}
\item {Grp. gram.:adj.}
\end{itemize}
\begin{itemize}
\item {Proveniência:(De \textunderscore re...\textunderscore  + \textunderscore solto\textunderscore )}
\end{itemize}
Muito sôlto.
Desprendido:«\textunderscore ...esvoaça a túnica ressolta.\textunderscore »Castilho, \textunderscore Fastos\textunderscore , II, 73.
\section{Ressonador}
\begin{itemize}
\item {Grp. gram.:m.}
\end{itemize}
Aquele que ressona. Cf. Capello e Ivens, I, 19.
\section{Ressonância}
\begin{itemize}
\item {Grp. gram.:f.}
\end{itemize}
\begin{itemize}
\item {Proveniência:(Lat. \textunderscore resonantia\textunderscore )}
\end{itemize}
Propriedade ou qualidade do que é ressonante.
\section{Ressonante}
\begin{itemize}
\item {Grp. gram.:adj.}
\end{itemize}
\begin{itemize}
\item {Proveniência:(Lat. \textunderscore resonans\textunderscore )}
\end{itemize}
Que ressôa.
Que ressona.
\section{Ressonar}
\begin{itemize}
\item {Grp. gram.:v. t.}
\end{itemize}
\begin{itemize}
\item {Grp. gram.:V. i.}
\end{itemize}
\begin{itemize}
\item {Proveniência:(Lat. \textunderscore resonare\textunderscore )}
\end{itemize}
Ressoar.
Fazer soar.
Respirar ruidosamente, dormindo.
Dormir.
\section{Ressoprar}
\begin{itemize}
\item {Grp. gram.:v. t.}
\end{itemize}
\begin{itemize}
\item {Proveniência:(De \textunderscore re...\textunderscore  + \textunderscore soprar\textunderscore )}
\end{itemize}
Tornar a soprar.
\section{Ressorpção}
\begin{itemize}
\item {Grp. gram.:f.}
\end{itemize}
Acto ou efeito de \textunderscore ressorver\textunderscore .
\section{Ressorver}
\begin{itemize}
\item {Grp. gram.:v. t.}
\end{itemize}
\begin{itemize}
\item {Proveniência:(Do lat. \textunderscore resorbere\textunderscore )}
\end{itemize}
Sorver novamente.
\section{Ressumbrar}
\textunderscore v. i.\textunderscore  (e der.)
O mesmo ou melhor que \textunderscore resumbrar\textunderscore , etc. Cf. Filinto, XX, 8.
\section{Restabelecer}
\begin{itemize}
\item {Grp. gram.:v. t.}
\end{itemize}
\begin{itemize}
\item {Grp. gram.:V. p.}
\end{itemize}
Estabelecer de novo.
Collocar no estado antigo.
Restaurar.
Recuperar.
Readquirir as fôrças ou a saúde.
Voltar ao antigo estado.
(Do \textunderscore re...\textunderscore  + \textunderscore estabelecer\textunderscore )
\section{Restabelecido}
\begin{itemize}
\item {Grp. gram.:adj.}
\end{itemize}
\begin{itemize}
\item {Proveniência:(De \textunderscore restabelecer\textunderscore )}
\end{itemize}
Que se restabeleceu.
Que recuperou as fôrças ou a saúde.
\section{Restabelecimento}
\begin{itemize}
\item {Grp. gram.:m.}
\end{itemize}
Acto ou effeito de restabelecer.
Recuperação da saúde.
\section{Resta-boi}
\begin{itemize}
\item {Grp. gram.:m.}
\end{itemize}
Planta leguminosa, (\textunderscore onomis spinosa\textunderscore ).
\section{Restagnação}
\begin{itemize}
\item {Grp. gram.:f.}
\end{itemize}
\begin{itemize}
\item {Proveniência:(Lat. \textunderscore restagnatio\textunderscore )}
\end{itemize}
O mesmo que \textunderscore estagnação\textunderscore .
\section{Restampa}
\begin{itemize}
\item {Grp. gram.:f.}
\end{itemize}
Acto ou effeito de restampar.
\section{Restampar}
\begin{itemize}
\item {Grp. gram.:v. t.}
\end{itemize}
\begin{itemize}
\item {Proveniência:(De \textunderscore re...\textunderscore  + \textunderscore estampar\textunderscore )}
\end{itemize}
Tornar a estampar; reestampar.
\section{Restante}
\begin{itemize}
\item {Grp. gram.:adj.}
\end{itemize}
\begin{itemize}
\item {Grp. gram.:M.}
\end{itemize}
\begin{itemize}
\item {Proveniência:(Lat. \textunderscore restans\textunderscore )}
\end{itemize}
Que resta.
O resto.
Aquillo que resta.
Aquelle que sobrevive.
O outro: \textunderscore dos cinco morreram dois, e os restantes salvaram-se\textunderscore .
\section{Restar}
\begin{itemize}
\item {Grp. gram.:v. i.}
\end{itemize}
\begin{itemize}
\item {Grp. gram.:V. t.}
\end{itemize}
\begin{itemize}
\item {Proveniência:(Lat. \textunderscore restare\textunderscore )}
\end{itemize}
Ficar, subsistir, depois de desapparecer a maior parte ou outras coisas ou pessôas: \textunderscore dos três filhos resta um\textunderscore .
Sobejar.
Faltar para certos fins.
Estar, por saldo, em dívida de: \textunderscore resto-lhe 30$000 reis\textunderscore .
\section{Restauração}
\begin{itemize}
\item {Grp. gram.:f.}
\end{itemize}
\begin{itemize}
\item {Proveniência:(Do lat. \textunderscore restauratio\textunderscore )}
\end{itemize}
Acto ou effeito de restaurar.
Conserto; renovação: \textunderscore restauração de um quadro\textunderscore .
Restabelecimento.
Reacquisição da independencia nacional.
Restabelecimento de uma dynastia.
\section{Ressuar}
\begin{itemize}
\item {Grp. gram.:v. i.}
\end{itemize}
\begin{itemize}
\item {Proveniência:(De \textunderscore re...\textunderscore  + \textunderscore suar\textunderscore )}
\end{itemize}
Suar muito.
\section{Ressubir}
\begin{itemize}
\item {Grp. gram.:v. t.  e  i.}
\end{itemize}
\begin{itemize}
\item {Proveniência:(De \textunderscore re...\textunderscore  + \textunderscore subir\textunderscore )}
\end{itemize}
Subir de novo, subir muitas vezes.
\section{Ressudação}
\begin{itemize}
\item {Grp. gram.:f.}
\end{itemize}
Acto ou efeito de ressudar.
\section{Ressudar}
\begin{itemize}
\item {Grp. gram.:v. t.}
\end{itemize}
\begin{itemize}
\item {Grp. gram.:V. i.}
\end{itemize}
\begin{itemize}
\item {Proveniência:(Lat. \textunderscore resudare\textunderscore )}
\end{itemize}
Destilar; expelir, suando.
Tornar a suar.
Transpirar; ressumbrar.
\section{Ressulcar}
\begin{itemize}
\item {Grp. gram.:v. t.}
\end{itemize}
\begin{itemize}
\item {Proveniência:(De \textunderscore re...\textunderscore  + \textunderscore sulcar\textunderscore )}
\end{itemize}
Sulcar de novo, sulcar muitas vezes. Cf. Alv. Mendes, \textunderscore Discursos\textunderscore , 26.
\section{Ressumação}
\begin{itemize}
\item {Grp. gram.:f.}
\end{itemize}
Acto ou efeito de ressumar.
\section{Ressumar}
\begin{itemize}
\item {Grp. gram.:v. t.  e  i.}
\end{itemize}
\begin{itemize}
\item {Proveniência:(De \textunderscore re...\textunderscore  + \textunderscore sumo\textunderscore )}
\end{itemize}
O mesmo que \textunderscore ressumbrar\textunderscore .
\section{Ressumbro}
\begin{itemize}
\item {Grp. gram.:m.}
\end{itemize}
Acto de resumbrar:«\textunderscore ...pelo rosto um ressumbro de melancolia...\textunderscore »Filinto, XX, 8.
\section{Ressumir}
\begin{itemize}
\item {Grp. gram.:v. t.  e  i}
\end{itemize}
O mesmo que \textunderscore ressumar\textunderscore .
\section{Ressunção}
\begin{itemize}
\item {Grp. gram.:f.}
\end{itemize}
\begin{itemize}
\item {Proveniência:(Lat. \textunderscore resumptio\textunderscore )}
\end{itemize}
Acto ou efeito de reassumir.
\section{Ressupinação}
\begin{itemize}
\item {Grp. gram.:f.}
\end{itemize}
\begin{itemize}
\item {Utilização:Bot.}
\end{itemize}
\begin{itemize}
\item {Proveniência:(Do lat. \textunderscore resupinatio\textunderscore )}
\end{itemize}
Estado de uma flôr, em que a pétala inferior toma o lugar da superior.
Estado de uma fôlha que, podendo estar voltada para baixo, se volta para cima.
\section{Ressupinado}
\begin{itemize}
\item {Grp. gram.:adj.}
\end{itemize}
\begin{itemize}
\item {Proveniência:(Lat. \textunderscore resupinatus\textunderscore )}
\end{itemize}
O mesmo que \textunderscore ressupino\textunderscore .
\section{Ressupinar}
\begin{itemize}
\item {Grp. gram.:v. t.}
\end{itemize}
\begin{itemize}
\item {Proveniência:(Lat. \textunderscore resupinare\textunderscore )}
\end{itemize}
Tornar ressupino.
\section{Ressupino}
\begin{itemize}
\item {Grp. gram.:adj.}
\end{itemize}
\begin{itemize}
\item {Utilização:Bot.}
\end{itemize}
\begin{itemize}
\item {Proveniência:(Lat. \textunderscore resupinus\textunderscore )}
\end{itemize}
Voltado para cima; deitado de costas.
Que tem voltadas para cima as partes que ordinariamente, em sêres da mesma espécie, estão voltadas para baixo.
\section{Ressurgente}
\begin{itemize}
\item {Grp. gram.:adj.}
\end{itemize}
Que ressurge.
\section{Ressurgido}
\begin{itemize}
\item {Grp. gram.:adj.}
\end{itemize}
Que ressurgiu; ressuscitado.
\section{Ressurgimento}
\begin{itemize}
\item {Grp. gram.:m.}
\end{itemize}
Acto de ressurgir; ressurreição. Cf. Camillo, \textunderscore Quéda\textunderscore , 187.
\section{Ressurgir}
\begin{itemize}
\item {Grp. gram.:v. i.}
\end{itemize}
\begin{itemize}
\item {Proveniência:(Lat. \textunderscore resurgere\textunderscore )}
\end{itemize}
Tornar a surgir.
Ressuscitar.
Voltar á vida.
Manifestar-se novamente: \textunderscore ressurgiu a questão agrícola\textunderscore .
\section{Ressurecto}
\begin{itemize}
\item {Grp. gram.:adj.}
\end{itemize}
\begin{itemize}
\item {Proveniência:(Lat. \textunderscore resurrectus\textunderscore )}
\end{itemize}
Que ressurgiu, que ressuscitou.
\section{Ressurreição}
\begin{itemize}
\item {Grp. gram.:f.}
\end{itemize}
\begin{itemize}
\item {Utilização:Fam.}
\end{itemize}
\begin{itemize}
\item {Utilização:Fig.}
\end{itemize}
\begin{itemize}
\item {Proveniência:(Lat. \textunderscore resurrectio\textunderscore )}
\end{itemize}
Acto de ressurgir.
Cura extraordinária, imprevista.
Renovação; vida nova.
\section{Ressurtir}
\begin{itemize}
\item {Grp. gram.:v. i.}
\end{itemize}
\begin{itemize}
\item {Proveniência:(De \textunderscore re...\textunderscore  + \textunderscore surtir\textunderscore )}
\end{itemize}
Saltar com fôrça para o ar.
Erguer-so com ímpeto.
Surgir, aparecer.
\section{Ressuscitação}
\begin{itemize}
\item {Grp. gram.:f.}
\end{itemize}
Acto ou efeito de ressuscitar; ressurreição; ressurgimento; reaparição.
\section{Ressuscitado}
\begin{itemize}
\item {Grp. gram.:m.}
\end{itemize}
Aquele que ressuscitou.
\section{Ressuscitador}
\begin{itemize}
\item {Grp. gram.:m.  e  adj.}
\end{itemize}
\begin{itemize}
\item {Proveniência:(Lat. \textunderscore resuscitator\textunderscore )}
\end{itemize}
O que ressuscita.
Aquele que restaura ou renova.
\section{Ressuscitar}
\begin{itemize}
\item {Grp. gram.:v. t.}
\end{itemize}
\begin{itemize}
\item {Utilização:Fig.}
\end{itemize}
\begin{itemize}
\item {Grp. gram.:V. i.}
\end{itemize}
\begin{itemize}
\item {Utilização:Fig.}
\end{itemize}
\begin{itemize}
\item {Proveniência:(Lat. \textunderscore resuscitare\textunderscore )}
\end{itemize}
Fazer ressurgir.
Chamar de novo á vida.
Restabelecer.
Ressurgir.
Reaparecer.
Escapar de grande perigo.
\section{Ressuscitável}
\begin{itemize}
\item {Grp. gram.:adj.}
\end{itemize}
Que se póde ressuscitar ou que póde ressuscitar.
\section{Restaurador}
\begin{itemize}
\item {Grp. gram.:m.  e  adj.}
\end{itemize}
\begin{itemize}
\item {Proveniência:(Lat. \textunderscore restaurator\textunderscore )}
\end{itemize}
O que restaura.
\section{Restaurante}
\begin{itemize}
\item {Grp. gram.:adj.}
\end{itemize}
\begin{itemize}
\item {Grp. gram.:M.}
\end{itemize}
\begin{itemize}
\item {Proveniência:(Lat. \textunderscore restaurans\textunderscore )}
\end{itemize}
Que restaura.
Aquillo que restaura.
Estabelecimento, onde se preparam e vendem comidas; casa de pasto.
\section{Restaurar}
\begin{itemize}
\item {Grp. gram.:v. t.}
\end{itemize}
\begin{itemize}
\item {Proveniência:(Lat. \textunderscore restaurare\textunderscore )}
\end{itemize}
Instaurar de novo; renovar.
Consertar.
Reintegrar.
Fazer voltar ao primitivo estado.
Pôr em vigor.
Recomeçar.
Restituir ao poder supremo.
Readquirir.
Reconquistar.
Recuperar.
\section{Restaurativo}
\begin{itemize}
\item {Grp. gram.:adj.}
\end{itemize}
Que póde restaurar; que restaura.
\section{Restaurável}
\begin{itemize}
\item {Grp. gram.:adj.}
\end{itemize}
Que se póde restaurar.
\section{Restauricar}
\begin{itemize}
\item {Grp. gram.:v. t.}
\end{itemize}
\begin{itemize}
\item {Utilização:P. us.}
\end{itemize}
Restaurar ou consertar mal.
Restaurar em parte, incompletamente. Cf. Camillo, \textunderscore Ôlho de Vidro\textunderscore , 91.
\section{Restauro}
\begin{itemize}
\item {Grp. gram.:m.}
\end{itemize}
O mesmo que \textunderscore restauração\textunderscore .
\section{Reste}
\begin{itemize}
\item {Grp. gram.:m.}
\end{itemize}
Utensílio de jôgo de bilhar, o mesmo que \textunderscore resto\textunderscore ^2.
\section{Reste}
\begin{itemize}
\item {Grp. gram.:m.}
\end{itemize}
O mesmo que \textunderscore riste\textunderscore :«\textunderscore quero também meter minha praga em reste.\textunderscore »F. Manuel, \textunderscore Apólogos\textunderscore , I, 109.
\section{Reste}
\begin{itemize}
\item {Grp. gram.:f.}
\end{itemize}
\begin{itemize}
\item {Utilização:Prov.}
\end{itemize}
\begin{itemize}
\item {Utilização:trasm.}
\end{itemize}
O mesmo que \textunderscore rodilha\textunderscore ^1.
\section{Réstea}
\begin{itemize}
\item {Grp. gram.:f.}
\end{itemize}
(V.réstia)
\section{Restelada}
\begin{itemize}
\item {Grp. gram.:f.}
\end{itemize}
\begin{itemize}
\item {Utilização:Prov.}
\end{itemize}
\begin{itemize}
\item {Utilização:Ext.}
\end{itemize}
\begin{itemize}
\item {Proveniência:(De \textunderscore restelo\textunderscore ^2)}
\end{itemize}
Porção de azeitonas, que se espalham no chão, antes do varejo.
Qualquer porção de objectos, que alastram o chão: \textunderscore uma restelada de maçans\textunderscore .
\section{Resteleira}
\begin{itemize}
\item {Grp. gram.:f.}
\end{itemize}
\begin{itemize}
\item {Utilização:Prov.}
\end{itemize}
\begin{itemize}
\item {Utilização:alent.}
\end{itemize}
O mesmo que \textunderscore restelo\textunderscore ^2.
\section{Restello}
\begin{itemize}
\item {fónica:tê}
\end{itemize}
\textunderscore m.\textunderscore  (e der.)
(V. \textunderscore rastello\textunderscore , etc.)
\section{Restelo}
\begin{itemize}
\item {fónica:tê}
\end{itemize}
\begin{itemize}
\item {Grp. gram.:m.}
\end{itemize}
Restolho?:«\textunderscore ...a cotovia aninhada no restelo da várzea...\textunderscore »Camillo, \textunderscore Ratazzi\textunderscore , 168.
\section{Restelo}
\begin{itemize}
\item {fónica:tê}
\end{itemize}
\begin{itemize}
\item {Grp. gram.:m.}
\end{itemize}
\begin{itemize}
\item {Utilização:Prov.}
\end{itemize}
\begin{itemize}
\item {Utilização:alent.}
\end{itemize}
Azeitona, que cai das oliveiras, antes do varejo.
Azeitonas, deixadas involuntariamente, debaixo das oliveiras, pelos ranchos que fazem a apanha.
\section{Restelo}
\begin{itemize}
\item {fónica:tê}
\end{itemize}
\textunderscore m.\textunderscore  (e der.)
(V. \textunderscore rastello\textunderscore , etc.)
\section{Resteva}
\begin{itemize}
\item {fónica:tê}
\end{itemize}
\begin{itemize}
\item {Grp. gram.:f.}
\end{itemize}
\begin{itemize}
\item {Proveniência:(Do lat. hyp. \textunderscore re-stipa\textunderscore )}
\end{itemize}
O mesmo que \textunderscore restolho\textunderscore .
O mesmo que \textunderscore estiva\textunderscore ^2.
\section{Réstia}
\begin{itemize}
\item {Grp. gram.:f.}
\end{itemize}
\begin{itemize}
\item {Utilização:Prov.}
\end{itemize}
\begin{itemize}
\item {Utilização:Pop.}
\end{itemize}
\begin{itemize}
\item {Proveniência:(Do lat. \textunderscore restis\textunderscore )}
\end{itemize}
Corda de junco entrançado.
Corda de caules entrelaçados: \textunderscore uma réstia de alhos\textunderscore ; \textunderscore uma réstia de cebolas\textunderscore .
O mesmo que \textunderscore rodilha\textunderscore ^1.
Feixe de luz: \textunderscore no cárcere entrava a custo uma réstia de sol\textunderscore .
Corja, súcia.
\section{Restiáceas}
\begin{itemize}
\item {Grp. gram.:f. pl.}
\end{itemize}
Família de plantas, que tem por typo o réstio.
(Fem. pl. de \textunderscore restiáceo\textunderscore )
\section{Restiáceo}
\begin{itemize}
\item {Grp. gram.:adj.}
\end{itemize}
Relativo ou semelhante ao réstio.
\section{Restiforme}
\begin{itemize}
\item {Grp. gram.:adj.}
\end{itemize}
\begin{itemize}
\item {Proveniência:(Do lat. \textunderscore restis\textunderscore  + \textunderscore fórma\textunderscore )}
\end{itemize}
Que tem fórma de réstia.
\section{Réstiga}
\begin{itemize}
\item {Grp. gram.:f.}
\end{itemize}
\begin{itemize}
\item {Utilização:Prov.}
\end{itemize}
\begin{itemize}
\item {Utilização:beir.}
\end{itemize}
O mesmo que \textunderscore réstia\textunderscore , soalheira.
\section{Restilação}
\begin{itemize}
\item {Grp. gram.:f.}
\end{itemize}
Acto ou efeito de restilar.
\section{Restilar}
\begin{itemize}
\item {Grp. gram.:v. t.}
\end{itemize}
\begin{itemize}
\item {Proveniência:(Lat. \textunderscore restillare\textunderscore )}
\end{itemize}
Destilar de novo.
\section{Restilho}
\begin{itemize}
\item {Grp. gram.:m.}
\end{itemize}
\begin{itemize}
\item {Utilização:Prov.}
\end{itemize}
\begin{itemize}
\item {Utilização:minh.}
\end{itemize}
Espécie de pente dos teares, formado de duas tábuas, ligadas por uma série de pauzinhos.
\section{Restillação}
\begin{itemize}
\item {Grp. gram.:f.}
\end{itemize}
Acto ou effeito de restillar.
\section{Restillar}
\begin{itemize}
\item {Grp. gram.:v. t.}
\end{itemize}
\begin{itemize}
\item {Proveniência:(Lat. \textunderscore restillare\textunderscore )}
\end{itemize}
Destillar de novo.
\section{Restinga}
\begin{itemize}
\item {Grp. gram.:f.}
\end{itemize}
\begin{itemize}
\item {Utilização:Bras}
\end{itemize}
\begin{itemize}
\item {Utilização:Bras. de Minhas}
\end{itemize}
Recife; escôlho; banco de areia ou de rocha no alto mar.
Pequeno matagal, á margem de um ribeiro ou em terreno fértil.
Rebotalho de terras já lavradas.
\section{Restinguir}
\begin{itemize}
\item {Grp. gram.:v. t.}
\end{itemize}
\begin{itemize}
\item {Proveniência:(Lat. \textunderscore restinguere\textunderscore )}
\end{itemize}
Extinguir novamente.
\section{Réstio}
\begin{itemize}
\item {Grp. gram.:m.}
\end{itemize}
\begin{itemize}
\item {Proveniência:(Lat. \textunderscore restio\textunderscore )}
\end{itemize}
Gênero de plantas herbáceas, cujas espécies crescem no Cabo da Bôa-Esperança, Madagáscar e Austrália.
\section{Restir}
\textunderscore v. i.\textunderscore  (e der.)
(Fórma ant. e pop. de \textunderscore resistir\textunderscore , etc. Na Beira, aínda se ouve:«\textunderscore Os batataes não restem a êste frio.\textunderscore »Cf. S. R. Viterbo, \textunderscore Elucidário\textunderscore )
(Contr. de \textunderscore resistir\textunderscore )
\section{Restirar-se}
\begin{itemize}
\item {Grp. gram.:v. p.}
\end{itemize}
\begin{itemize}
\item {Utilização:Prov.}
\end{itemize}
\begin{itemize}
\item {Utilização:minh.}
\end{itemize}
\begin{itemize}
\item {Proveniência:(De \textunderscore re...\textunderscore  + \textunderscore estirar\textunderscore )}
\end{itemize}
Estirar-se completamente no chão.
\section{Restito}
\begin{itemize}
\item {Grp. gram.:m.}
\end{itemize}
\begin{itemize}
\item {Utilização:Prov.}
\end{itemize}
\begin{itemize}
\item {Utilização:beir.}
\end{itemize}
\begin{itemize}
\item {Proveniência:(De \textunderscore resto\textunderscore )}
\end{itemize}
Pequena quantia: \textunderscore fui lá receber um restito\textunderscore .
\section{Restituição}
\begin{itemize}
\item {fónica:tu-i}
\end{itemize}
\begin{itemize}
\item {Grp. gram.:f.}
\end{itemize}
\begin{itemize}
\item {Proveniência:(Do lat. \textunderscore restitutio\textunderscore )}
\end{itemize}
Acto ou effeito de restituír.
Entrega de alguma coisa a quem ella por direito pertencia.
Regresso de um planeta á sua ábside.
Regresso ao primeiro estado.
\section{Restituidor}
\begin{itemize}
\item {fónica:tu-i}
\end{itemize}
\begin{itemize}
\item {Grp. gram.:m.  e  adj.}
\end{itemize}
\begin{itemize}
\item {Proveniência:(Do lat. \textunderscore restitutor\textunderscore )}
\end{itemize}
O que restitue.
\section{Restituir}
\begin{itemize}
\item {Grp. gram.:v. t.}
\end{itemize}
\begin{itemize}
\item {Grp. gram.:V. p.}
\end{itemize}
\begin{itemize}
\item {Proveniência:(Lat. \textunderscore restituere\textunderscore )}
\end{itemize}
Repôr; collocar de novo.
Entregar (o que se possuía indevidamente).
Fazer voltar ao primeiro estado.
Restabelecer.
Fazer readquirir: \textunderscore aquellas thermas restituíram-lhe a saúde\textunderscore .
Rehabilitar.
Indemnizar-se; provêr-se.
Voltar.
\section{Restituitório}
\begin{itemize}
\item {fónica:tu-i}
\end{itemize}
\begin{itemize}
\item {Grp. gram.:adj.}
\end{itemize}
\begin{itemize}
\item {Proveniência:(Do lat. \textunderscore restitutorius\textunderscore )}
\end{itemize}
Que envolve restituição.
Relativo a restituição.
\section{Restituível}
\begin{itemize}
\item {Grp. gram.:adj.}
\end{itemize}
Que se póde ou se deve restituir.
\section{Restiva}
\begin{itemize}
\item {Grp. gram.:f.}
\end{itemize}
\begin{itemize}
\item {Utilização:Prov.}
\end{itemize}
\begin{itemize}
\item {Utilização:Minh.}
\end{itemize}
O mesmo que \textunderscore restivada\textunderscore .
\section{Restivada}
\begin{itemize}
\item {Grp. gram.:f.}
\end{itemize}
\begin{itemize}
\item {Utilização:Prov.}
\end{itemize}
\begin{itemize}
\item {Utilização:minh.}
\end{itemize}
\begin{itemize}
\item {Proveniência:(De \textunderscore restivar\textunderscore )}
\end{itemize}
Segunda cultura de um campo, no mesmo anno; o producto dessa cultura.
\section{Restivar}
\begin{itemize}
\item {Grp. gram.:v. t.}
\end{itemize}
\begin{itemize}
\item {Utilização:Prov.}
\end{itemize}
\begin{itemize}
\item {Utilização:minh.}
\end{itemize}
Cultivar segunda vez (um terreno), como quando se semeia milho onde, no mesmo anno, se colheu centeio.
(Por \textunderscore restevar\textunderscore , de \textunderscore resteva\textunderscore )
\section{Restivo}
\begin{itemize}
\item {Grp. gram.:m.}
\end{itemize}
\begin{itemize}
\item {Proveniência:(De \textunderscore restivar\textunderscore )}
\end{itemize}
Producto da segunda cultura de um campo; restivada.
Diz-se \textunderscore milho de restivo\textunderscore  o milho que se colheu num campo, que no mesmo anno tinha produzido centeio.
\section{Resto}
\begin{itemize}
\item {Grp. gram.:m.}
\end{itemize}
\begin{itemize}
\item {Utilização:Arith.}
\end{itemize}
\begin{itemize}
\item {Grp. gram.:Pl.}
\end{itemize}
Aquillo que resta.
Aquillo que sobeja de maior porção.
As outras coisas ou pessôas, relativamente àquellas de que se está falando.
Parte de um dividendo, mais pequena que o divisor.
Differença entre dois números, um dos quaes se subtrái do outro.
Ruínas.
Despojos mortaes.
\textunderscore Restos de cozinha\textunderscore , nome que, em Paleontologia, se dá aos detritos de comestíveis e restos de objectos de cozinha, que apparecem em certas estações da idade de pedra, e que os geólogos também designam pelo termo dinamarquês \textunderscore kjoekkenmoedding\textunderscore .
\section{Resto}
\begin{itemize}
\item {Grp. gram.:m.}
\end{itemize}
\begin{itemize}
\item {Proveniência:(Do ingl. \textunderscore rest\textunderscore )}
\end{itemize}
Rabeca do bilhar, ou utensílio, em que se apoia o taco, quando as bolas estão mais distantes do jogador.
\section{Restolhada}
\begin{itemize}
\item {Grp. gram.:f.}
\end{itemize}
\begin{itemize}
\item {Utilização:Fig.}
\end{itemize}
\begin{itemize}
\item {Utilização:Ext.}
\end{itemize}
Abundância de restolho.
Ruído de quem anda por entre ou sôbre o restolho.
Ruído, barulho.
\section{Restolhal}
\begin{itemize}
\item {Grp. gram.:m.}
\end{itemize}
Terreno, em que há restolho.
\section{Restolhar}
\begin{itemize}
\item {Grp. gram.:v. i.}
\end{itemize}
\begin{itemize}
\item {Utilização:Ext.}
\end{itemize}
\begin{itemize}
\item {Proveniência:(De \textunderscore restolho\textunderscore )}
\end{itemize}
Respigar.
Fazer ruído, andando sôbre ou por entre o restolho.
Fazer ruído.
\section{Restolhiça}
\begin{itemize}
\item {Grp. gram.:f.}
\end{itemize}
\begin{itemize}
\item {Utilização:Prov.}
\end{itemize}
\begin{itemize}
\item {Utilização:alent.}
\end{itemize}
O mesmo que \textunderscore restolhada\textunderscore . Cf. Ficalho, \textunderscore Contos\textunderscore , 225.
\section{Restolho}
\begin{itemize}
\item {fónica:tô}
\end{itemize}
\begin{itemize}
\item {Grp. gram.:m.}
\end{itemize}
Parte inferior do caule das gramíneas, que ficou enraizada depois da ceifa.
Restolhal.
Barulho, o mesmo que \textunderscore restolhada\textunderscore . Cf. G. Braga, \textunderscore Mal da Delf.\textunderscore , 55.
(V. \textunderscore rastolho\textunderscore , que parece fórma mais exacta)
\section{Restralar}
\begin{itemize}
\item {Grp. gram.:v. t.}
\end{itemize}
\begin{itemize}
\item {Utilização:Prov.}
\end{itemize}
\begin{itemize}
\item {Utilização:trasm.}
\end{itemize}
O mesmo que \textunderscore arrestralar\textunderscore .
\section{Restrelo}
\begin{itemize}
\item {fónica:trê}
\end{itemize}
\begin{itemize}
\item {Grp. gram.:m.}
\end{itemize}
\begin{itemize}
\item {Utilização:Prov.}
\end{itemize}
\begin{itemize}
\item {Utilização:trasm.}
\end{itemize}
Espécie de gadanho, com que se junta palha nos restolhos.
(Cp. \textunderscore rastello\textunderscore )
\section{Restribar}
\begin{itemize}
\item {Grp. gram.:v. i.  e  p.}
\end{itemize}
\begin{itemize}
\item {Proveniência:(De \textunderscore re...\textunderscore  + \textunderscore estribar\textunderscore )}
\end{itemize}
Firmar-se bem nos estribos.
Insistir.
Oppor-se com firmeza.
\section{Restrição}
\begin{itemize}
\item {Grp. gram.:f.}
\end{itemize}
\begin{itemize}
\item {Proveniência:(Lat. \textunderscore restrictio\textunderscore )}
\end{itemize}
Acto ou efeito de restringir.
Limitação.
\section{Restricção}
\begin{itemize}
\item {Grp. gram.:f.}
\end{itemize}
\begin{itemize}
\item {Proveniência:(Lat. \textunderscore restrictio\textunderscore )}
\end{itemize}
Acto ou effeito de restringir.
Limitação.
\section{Restrictamente}
\begin{itemize}
\item {Grp. gram.:adv.}
\end{itemize}
De modo restricto.
\section{Restrictiva}
\begin{itemize}
\item {Grp. gram.:f.}
\end{itemize}
\begin{itemize}
\item {Utilização:Gram.}
\end{itemize}
\begin{itemize}
\item {Proveniência:(De \textunderscore restrictivo\textunderscore )}
\end{itemize}
Proposição incidente, que restringe o significado de outra proposição ou de uma palavra.
\section{Restrictivamente}
\begin{itemize}
\item {Grp. gram.:adv.}
\end{itemize}
De modo restrictivo; com restricção.
\section{Restrictivo}
\begin{itemize}
\item {Grp. gram.:adj.}
\end{itemize}
\begin{itemize}
\item {Proveniência:(De \textunderscore restricto\textunderscore )}
\end{itemize}
Que restringe ou envolve restricção; limitativo.
\section{Restricto}
\begin{itemize}
\item {Grp. gram.:adj.}
\end{itemize}
\begin{itemize}
\item {Proveniência:(Lat. \textunderscore restrictus\textunderscore )}
\end{itemize}
Limitado; modificado na sua extensão: \textunderscore o sentido restricto de uma palavra\textunderscore .
\section{Restrilho}
\begin{itemize}
\item {Grp. gram.:m.}
\end{itemize}
O mesmo que \textunderscore restrelo\textunderscore .
\section{Restringência}
\begin{itemize}
\item {Grp. gram.:f.}
\end{itemize}
Qualidade daquillo que é restringente.
\section{Restringente}
\begin{itemize}
\item {Grp. gram.:adj.}
\end{itemize}
\begin{itemize}
\item {Grp. gram.:M.}
\end{itemize}
\begin{itemize}
\item {Proveniência:(Lat. \textunderscore restringens\textunderscore )}
\end{itemize}
Que restringe.
Medicamento restringente.
\section{Restringimento}
\begin{itemize}
\item {Grp. gram.:m.}
\end{itemize}
Acto ou effeito de restringir.
\section{Restringir}
\begin{itemize}
\item {Grp. gram.:v. t.}
\end{itemize}
\begin{itemize}
\item {Proveniência:(Lat. \textunderscore restringere\textunderscore )}
\end{itemize}
Estreitar, apertar.
Fortificar (uma parte froixa do organismo).
Deminuir.
Encurtar.
Modificar.
Limitar.
Reduzir.
Limitar a extensão do significado de (uma palavra).
\section{Restringível}
\begin{itemize}
\item {Grp. gram.:adj.}
\end{itemize}
Que se póde restringir.
\section{Restritamente}
\begin{itemize}
\item {Grp. gram.:adv.}
\end{itemize}
De modo restrito.
\section{Restritiva}
\begin{itemize}
\item {Grp. gram.:f.}
\end{itemize}
\begin{itemize}
\item {Utilização:Gram.}
\end{itemize}
\begin{itemize}
\item {Proveniência:(De \textunderscore restritivo\textunderscore )}
\end{itemize}
Proposição incidente, que restringe o significado de outra proposição ou de uma palavra.
\section{Restritivamente}
\begin{itemize}
\item {Grp. gram.:adv.}
\end{itemize}
De modo restritivo; com restrição.
\section{Restritivo}
\begin{itemize}
\item {Grp. gram.:adj.}
\end{itemize}
\begin{itemize}
\item {Proveniência:(De \textunderscore restrito\textunderscore )}
\end{itemize}
Que restringe ou envolve restrição; limitativo.
\section{Restrito}
\begin{itemize}
\item {Grp. gram.:adj.}
\end{itemize}
\begin{itemize}
\item {Proveniência:(Lat. \textunderscore restrictus\textunderscore )}
\end{itemize}
Limitado; modificado na sua extensão: \textunderscore o sentido restrito de uma palavra\textunderscore .
\section{Restrugir}
\begin{itemize}
\item {Grp. gram.:v. i.}
\end{itemize}
\begin{itemize}
\item {Proveniência:(De \textunderscore re...\textunderscore  + \textunderscore estrugir\textunderscore )}
\end{itemize}
Tornar a estrugir; estrugir com fôrça; estrugir muitas vezes.
Retumbar.
\section{Restucar}
\begin{itemize}
\item {Grp. gram.:v. t.}
\end{itemize}
\begin{itemize}
\item {Proveniência:(De \textunderscore re...\textunderscore  + \textunderscore estucar\textunderscore )}
\end{itemize}
Tornar a estucar; estucar bem.
\section{Restumenga}
\begin{itemize}
\item {Grp. gram.:f.}
\end{itemize}
Conjunto do peixe miúdo, que o pescador vende, para comprar os adubos e preparos de caldeirada.
\section{Resuar}
\begin{itemize}
\item {fónica:su}
\end{itemize}
\begin{itemize}
\item {Grp. gram.:v. i.}
\end{itemize}
\begin{itemize}
\item {Proveniência:(De \textunderscore re...\textunderscore  + \textunderscore suar\textunderscore )}
\end{itemize}
Suar muito.
\section{Resubir}
\begin{itemize}
\item {fónica:su}
\end{itemize}
\begin{itemize}
\item {Grp. gram.:v. t.  e  i.}
\end{itemize}
\begin{itemize}
\item {Proveniência:(De \textunderscore re...\textunderscore  + \textunderscore subir\textunderscore )}
\end{itemize}
Subir de novo, subir muitas vezes.
\section{Resudação}
\begin{itemize}
\item {fónica:su}
\end{itemize}
\begin{itemize}
\item {Grp. gram.:f.}
\end{itemize}
Acto ou effeito de resudar.
\section{Resudar}
\begin{itemize}
\item {fónica:su}
\end{itemize}
\begin{itemize}
\item {Grp. gram.:v. t.}
\end{itemize}
\begin{itemize}
\item {Grp. gram.:V. i.}
\end{itemize}
\begin{itemize}
\item {Proveniência:(Lat. \textunderscore resudare\textunderscore )}
\end{itemize}
Destillar; expellir, suando.
Tornar a suar.
Transpirar; resumbrar.
\section{Resulcar}
\begin{itemize}
\item {fónica:sul}
\end{itemize}
\begin{itemize}
\item {Grp. gram.:v. t.}
\end{itemize}
\begin{itemize}
\item {Proveniência:(De \textunderscore re...\textunderscore  + \textunderscore sulcar\textunderscore )}
\end{itemize}
Sulcar de novo, sulcar muitas vezes. Cf. Alv. Mendes, \textunderscore Discursos\textunderscore , 26.
\section{Resulho}
\begin{itemize}
\item {Grp. gram.:m.}
\end{itemize}
\begin{itemize}
\item {Utilização:Prov.}
\end{itemize}
\begin{itemize}
\item {Utilização:trasm.}
\end{itemize}
A parte sólida do caldo.
\section{Resulta}
\begin{itemize}
\item {Grp. gram.:f.}
\end{itemize}
O mesmo que \textunderscore resultado\textunderscore . Cf. \textunderscore Viriato Trág.\textunderscore , 117.
\section{Resultado}
\begin{itemize}
\item {Grp. gram.:m.}
\end{itemize}
Acto ou effeito de resultar.
Resolução, deliberação.
Consequência; effeito.
Fim.
Proventos.
\section{Resultância}
\begin{itemize}
\item {Grp. gram.:f.}
\end{itemize}
(V.resultado)
\section{Resultante}
\begin{itemize}
\item {Grp. gram.:adj.}
\end{itemize}
\begin{itemize}
\item {Grp. gram.:F.}
\end{itemize}
\begin{itemize}
\item {Proveniência:(Lat. \textunderscore resultans\textunderscore )}
\end{itemize}
Que resulta.
Linha recta ou fôrça que resulta.
\section{Resultar}
\begin{itemize}
\item {Grp. gram.:v. i.}
\end{itemize}
\begin{itemize}
\item {Proveniência:(Lat. \textunderscore resultare\textunderscore )}
\end{itemize}
Sêr consequência ou effeito.
Dimanar.
Seguir-se.
Proceder.
Redundar, converter-se.
\section{Resumação}
\begin{itemize}
\item {fónica:su}
\end{itemize}
\begin{itemize}
\item {Grp. gram.:f.}
\end{itemize}
Acto ou effeito de resumar.
\section{Resumar}
\begin{itemize}
\item {fónica:su}
\end{itemize}
\begin{itemize}
\item {Grp. gram.:v. t.  e  i.}
\end{itemize}
\begin{itemize}
\item {Proveniência:(De \textunderscore re...\textunderscore  + \textunderscore sumo\textunderscore )}
\end{itemize}
O mesmo que \textunderscore resumbrar\textunderscore .
\section{Resumbrar}
\begin{itemize}
\item {fónica:sum}
\end{itemize}
\begin{itemize}
\item {Grp. gram.:v. t.}
\end{itemize}
\begin{itemize}
\item {Grp. gram.:V. t.}
\end{itemize}
\begin{itemize}
\item {Utilização:Fig.}
\end{itemize}
Gotejar; destillar.
Resudar.
Dar passagem a um liquido, coando-o.
Revelar-se, transparecer.
(Alter. de \textunderscore resumar\textunderscore ?)
\section{Resumbro}
\begin{itemize}
\item {fónica:sum}
\end{itemize}
\begin{itemize}
\item {Grp. gram.:m.}
\end{itemize}
Acto de resumbrar:«\textunderscore ...pelo rosto um resumbro de melancolia...\textunderscore »Filinto, XX, 8.
\section{Resumidamente}
\begin{itemize}
\item {Grp. gram.:adv.}
\end{itemize}
De modo resumido; em resumo; em sýnthese.
\section{Resumido}
\begin{itemize}
\item {Grp. gram.:adj.}
\end{itemize}
Que se resumiu.
Deminuido em número ou grandeza.
Abreviado, synthético: \textunderscore um tratado resumido\textunderscore .
\section{Resumidor}
\begin{itemize}
\item {Grp. gram.:m.  e  adj.}
\end{itemize}
O que resume.
\section{Resumir}
\begin{itemize}
\item {Grp. gram.:v. t.}
\end{itemize}
\begin{itemize}
\item {Proveniência:(Lat. \textunderscore resumere\textunderscore )}
\end{itemize}
Abreviar; synthetizar; recopilar.
Fazer synópse de.
Reduzir.
Condensar.
Representar em pequeno ponto.
Fazer consistir em: \textunderscore resumiu a accusação em três factos\textunderscore .
\section{Resumir}
\begin{itemize}
\item {fónica:su}
\end{itemize}
\begin{itemize}
\item {Grp. gram.:v. t.  e  i.}
\end{itemize}
O mesmo que \textunderscore resumar\textunderscore .
\section{Resumo}
\begin{itemize}
\item {Grp. gram.:m.}
\end{itemize}
Acto ou effeito de resumir.
Synópse; compilação; compêndio; recapitulação.
\section{Resumpção}
\begin{itemize}
\item {fónica:sum}
\end{itemize}
\begin{itemize}
\item {Grp. gram.:f.}
\end{itemize}
\begin{itemize}
\item {Proveniência:(Lat. \textunderscore resumptio\textunderscore )}
\end{itemize}
Acto ou effeito de reassumir.
\section{Resumpta}
\begin{itemize}
\item {Grp. gram.:f.}
\end{itemize}
\begin{itemize}
\item {Utilização:Ant.}
\end{itemize}
\begin{itemize}
\item {Proveniência:(Lat. \textunderscore resumpta\textunderscore )}
\end{itemize}
O mesmo que \textunderscore resumo\textunderscore .
\section{Resumptivo}
\begin{itemize}
\item {Grp. gram.:adj.}
\end{itemize}
\begin{itemize}
\item {Utilização:Ant.}
\end{itemize}
\begin{itemize}
\item {Proveniência:(Lat. \textunderscore resumptivus\textunderscore )}
\end{itemize}
Dizia-se do medicamento, destinado a curar e alimentar ao mesmo tempo.
\section{Resupinação}
\begin{itemize}
\item {fónica:su}
\end{itemize}
\begin{itemize}
\item {Grp. gram.:f.}
\end{itemize}
\begin{itemize}
\item {Utilização:Bot.}
\end{itemize}
\begin{itemize}
\item {Proveniência:(Do lat. \textunderscore resupinatio\textunderscore )}
\end{itemize}
Estado de uma flôr, em que a pétala inferior toma o lugar da superior.
Estado de uma fôlha que, podendo estar voltada para baixo, se volta para cima.
\section{Resupinado}
\begin{itemize}
\item {fónica:su}
\end{itemize}
\begin{itemize}
\item {Grp. gram.:adj.}
\end{itemize}
\begin{itemize}
\item {Proveniência:(Lat. \textunderscore resupinatus\textunderscore )}
\end{itemize}
O mesmo que \textunderscore resupino\textunderscore .
\section{Resupinar}
\begin{itemize}
\item {fónica:su}
\end{itemize}
\begin{itemize}
\item {Grp. gram.:v. t.}
\end{itemize}
\begin{itemize}
\item {Proveniência:(Lat. \textunderscore resupinare\textunderscore )}
\end{itemize}
Tornar resupino.
\section{Resupino}
\begin{itemize}
\item {fónica:su}
\end{itemize}
\begin{itemize}
\item {Grp. gram.:adj.}
\end{itemize}
\begin{itemize}
\item {Utilização:Bot.}
\end{itemize}
\begin{itemize}
\item {Proveniência:(Lat. \textunderscore resupinus\textunderscore )}
\end{itemize}
Voltado para cima; deitado de costas.
Que tem voltadas para cima as partes que ordinariamente, em sêres da mesma espécie, estão voltadas para baixo.
\section{Resura}
\begin{itemize}
\item {Grp. gram.:f.}
\end{itemize}
\begin{itemize}
\item {Utilização:Prov.}
\end{itemize}
\begin{itemize}
\item {Utilização:trasm.}
\end{itemize}
\begin{itemize}
\item {Utilização:Prov.}
\end{itemize}
\begin{itemize}
\item {Utilização:minh.}
\end{itemize}
Calor, que irradia da fogueira.
Dôr, em seguida ao parto.
\section{Resurgente}
\begin{itemize}
\item {fónica:sur}
\end{itemize}
\begin{itemize}
\item {Grp. gram.:adj.}
\end{itemize}
Que resurge.
\section{Resurgido}
\begin{itemize}
\item {fónica:sur}
\end{itemize}
\begin{itemize}
\item {Grp. gram.:adj.}
\end{itemize}
Que resurgiu; resuscitado.
\section{Resurgimento}
\begin{itemize}
\item {fónica:sur}
\end{itemize}
\begin{itemize}
\item {Grp. gram.:m.}
\end{itemize}
Acto de resurgir; resurreição. Cf. Camillo, \textunderscore Quéda\textunderscore , 187.
\section{Resurgir}
\begin{itemize}
\item {fónica:sur}
\end{itemize}
\begin{itemize}
\item {Grp. gram.:v. i.}
\end{itemize}
\begin{itemize}
\item {Proveniência:(Lat. \textunderscore resurgere\textunderscore )}
\end{itemize}
Tornar a surgir.
Resuscitar.
Voltar á vida.
Manifestar-se novamente: \textunderscore resurgiu a questão agrícola\textunderscore .
\section{Resurrecto}
\begin{itemize}
\item {fónica:su}
\end{itemize}
\begin{itemize}
\item {Grp. gram.:adj.}
\end{itemize}
\begin{itemize}
\item {Proveniência:(Lat. \textunderscore resurrectus\textunderscore )}
\end{itemize}
Que resurgiu, que resuscitou.
\section{Resurreição}
\begin{itemize}
\item {fónica:su}
\end{itemize}
\begin{itemize}
\item {Grp. gram.:f.}
\end{itemize}
\begin{itemize}
\item {Utilização:Fam.}
\end{itemize}
\begin{itemize}
\item {Utilização:Fig.}
\end{itemize}
\begin{itemize}
\item {Proveniência:(Lat. \textunderscore resurrectio\textunderscore )}
\end{itemize}
Acto de resurgir.
Cura extraordinária, imprevista.
Renovação; vida nova.
\section{Resurtir}
\begin{itemize}
\item {fónica:sur}
\end{itemize}
\begin{itemize}
\item {Grp. gram.:v. i.}
\end{itemize}
\begin{itemize}
\item {Proveniência:(De \textunderscore re...\textunderscore  + \textunderscore surtir\textunderscore )}
\end{itemize}
Saltar com fôrça para o ar.
Erguer-se com ímpeto.
Surgir, apparecer.
\section{Resuscitação}
\begin{itemize}
\item {fónica:sus}
\end{itemize}
\begin{itemize}
\item {Grp. gram.:f.}
\end{itemize}
Acto ou effeito de resuscitar; resurreição; resurgimento; reapparição.
\section{Resuscitado}
\begin{itemize}
\item {fónica:sus}
\end{itemize}
\begin{itemize}
\item {Grp. gram.:m.}
\end{itemize}
Aquelle que resuscitou.
\section{Resuscitador}
\begin{itemize}
\item {fónica:sus}
\end{itemize}
\begin{itemize}
\item {Grp. gram.:m.  e  adj.}
\end{itemize}
\begin{itemize}
\item {Proveniência:(Lat. \textunderscore resuscitator\textunderscore )}
\end{itemize}
O que resuscita.
Aquelle que restaura ou renova.
\section{Resuscitar}
\begin{itemize}
\item {fónica:sus}
\end{itemize}
\begin{itemize}
\item {Grp. gram.:v. t.}
\end{itemize}
\begin{itemize}
\item {Utilização:Fig.}
\end{itemize}
\begin{itemize}
\item {Grp. gram.:V. i.}
\end{itemize}
\begin{itemize}
\item {Utilização:Fig.}
\end{itemize}
\begin{itemize}
\item {Proveniência:(Lat. \textunderscore resuscitare\textunderscore )}
\end{itemize}
Fazer resurgir.
Chamar de novo á vida.
Restabelecer.
Resurgir.
Reapparecer.
Escapar de grande perigo.
\section{Resuscitável}
\begin{itemize}
\item {fónica:sus}
\end{itemize}
\begin{itemize}
\item {Grp. gram.:adj.}
\end{itemize}
Que se póde resuscitar ou que póde resuscitar.
\section{Resvaladeiro}
\begin{itemize}
\item {Grp. gram.:m.}
\end{itemize}
(V.resvaladoiro)
\section{Resvaladiço}
\begin{itemize}
\item {Grp. gram.:adj.}
\end{itemize}
\begin{itemize}
\item {Grp. gram.:M.}
\end{itemize}
\begin{itemize}
\item {Proveniência:(De \textunderscore resvalar\textunderscore )}
\end{itemize}
Por onde se resvala facilmente.
Íngreme; escorregadio; perigoso.
O mesmo que \textunderscore resvaladoiro\textunderscore .
\section{Recíolo}
\begin{itemize}
\item {Grp. gram.:m.}
\end{itemize}
\begin{itemize}
\item {Proveniência:(Lat. \textunderscore retiolum\textunderscore )}
\end{itemize}
Espécie de coifa antiga; retículo. Cf. Herculano, \textunderscore Eurico\textunderscore , (nas notas).
\section{Resvaladio}
\begin{itemize}
\item {Grp. gram.:adj.}
\end{itemize}
\begin{itemize}
\item {Grp. gram.:M.}
\end{itemize}
O mesmo que \textunderscore resvaladiço\textunderscore .
Lugar resvaladiço. Cf. Filinto, XX, 90.
\section{Resvaladoiro}
\begin{itemize}
\item {Grp. gram.:m.}
\end{itemize}
\begin{itemize}
\item {Utilização:Fig.}
\end{itemize}
\begin{itemize}
\item {Proveniência:(De \textunderscore resvalar\textunderscore )}
\end{itemize}
Lugar, por onde se resvala com facilidade; escorregadoiro; declive; despenhadeiro.
Aquillo que põe em perigo o bom nome ou a dignidade de alguém.
\section{Resvaladouro}
\begin{itemize}
\item {Grp. gram.:m.}
\end{itemize}
\begin{itemize}
\item {Utilização:Fig.}
\end{itemize}
\begin{itemize}
\item {Proveniência:(De \textunderscore resvalar\textunderscore )}
\end{itemize}
Lugar, por onde se resvala com facilidade; escorregadouro; declive; despenhadeiro.
Aquillo que põe em perigo o bom nome ou a dignidade de alguém.
\section{Resveladura}
\begin{itemize}
\item {Grp. gram.:f.}
\end{itemize}
Acto ou effeito de resvalar; vestígio de resvalo.
\section{Resvalante}
\begin{itemize}
\item {Grp. gram.:adj.}
\end{itemize}
Que resvala.
\section{Resvalar}
\begin{itemize}
\item {Grp. gram.:v. t.}
\end{itemize}
\begin{itemize}
\item {Grp. gram.:V. i.}
\end{itemize}
Atirar, lançar, fazer escorregar ou fazer cair.
Cair por um declive; escorregar.
Deslisar.
Passar de leve.
Fugir.
Cair.
Começar a errar.
(Cp. cast. \textunderscore resbalar\textunderscore )
\section{Resvalo}
\begin{itemize}
\item {Grp. gram.:m.}
\end{itemize}
Acto ou effeito de resvalar.
Lugar, por onde se resvala; declive. Cf. Camillo, \textunderscore Mar. da Fonte\textunderscore , 186.
\section{Rés-vés}
\begin{itemize}
\item {Grp. gram.:adj.}
\end{itemize}
\begin{itemize}
\item {Utilização:Pop.}
\end{itemize}
\begin{itemize}
\item {Proveniência:(De \textunderscore rés\textunderscore )}
\end{itemize}
Cerce, rente; á justa; proximamente.
\section{Retábulo}
\begin{itemize}
\item {Grp. gram.:m.}
\end{itemize}
Construcção de pedra ou madeira com lavores, que se eleva da parte posterior do altar, e que encerra em geral um quadro religioso.
Painel ou quadro, que decora um altar; painel.
(Cp. cast. \textunderscore retablo\textunderscore )
\section{Retacar}
\begin{itemize}
\item {Grp. gram.:v. t.}
\end{itemize}
\begin{itemize}
\item {Utilização:P. us.}
\end{itemize}
\begin{itemize}
\item {Proveniência:(De \textunderscore re...\textunderscore  + \textunderscore taco\textunderscore )}
\end{itemize}
Repicar ou tocar duas vezes (a bola, no jôgo do bilhar).
\section{Retaco}
\begin{itemize}
\item {Grp. gram.:adj.}
\end{itemize}
\begin{itemize}
\item {Utilização:Bras}
\end{itemize}
Baixo e grosso; atarracado.
\section{Retador}
\begin{itemize}
\item {Grp. gram.:m.}
\end{itemize}
\begin{itemize}
\item {Utilização:Pesc.}
\end{itemize}
Cabo principal do apparelho de galeão.
(Por \textunderscore arretador\textunderscore , de \textunderscore arretar\textunderscore ?)
\section{Rètaguarda}
\begin{itemize}
\item {Grp. gram.:f.}
\end{itemize}
\begin{itemize}
\item {Proveniência:(De \textunderscore retro...\textunderscore  + \textunderscore guarda\textunderscore )}
\end{itemize}
Última fila ou último esquadrão de um corpo do exército.
A parte posterior.
\section{Retalhado}
\begin{itemize}
\item {Grp. gram.:adj.}
\end{itemize}
\begin{itemize}
\item {Grp. gram.:M.}
\end{itemize}
Partido em pedaços.
Sulcado.
Recorte ornamental, em trabalhos de ollaria. Cf. P. Carvalho, \textunderscore Corogr. Port.\textunderscore , II, 118.
\section{Retalhador}
\begin{itemize}
\item {Grp. gram.:m.  e  adj.}
\end{itemize}
O que retalha.
\section{Retalhadura}
\begin{itemize}
\item {Grp. gram.:f.}
\end{itemize}
Acto ou effeito de retalhar.
\section{Retalhar}
\begin{itemize}
\item {Grp. gram.:v. t.}
\end{itemize}
\begin{itemize}
\item {Utilização:Des.}
\end{itemize}
\begin{itemize}
\item {Utilização:Bras. do S}
\end{itemize}
\begin{itemize}
\item {Proveniência:(De \textunderscore re...\textunderscore  + \textunderscore talhar\textunderscore )}
\end{itemize}
Talhar em pedaços; cortar em várias partes; despedaçar: \textunderscore matar é retalhar um porco\textunderscore .
Ferír.
Sulcar.
Recortar: \textunderscore retalhar um molde para vestidos\textunderscore .
Golpear.
Dividir.
Entrecortar.
Magoar.
Fazer mal a.
Vender por miúdo, vender a retalho.
Fazer a (um cavallo) operação que o inutiliza para a procriação, sem o castrar completamente.
\section{Retalheiro}
\begin{itemize}
\item {Grp. gram.:adj.}
\end{itemize}
\begin{itemize}
\item {Grp. gram.:M.}
\end{itemize}
Que retalha.
O que vende a retalho.
\section{Retalhista}
\begin{itemize}
\item {Grp. gram.:m.}
\end{itemize}
Negociante que vende ou compra a retalho ou por miúdo.
\section{Retalho}
\begin{itemize}
\item {Grp. gram.:m.}
\end{itemize}
\begin{itemize}
\item {Grp. gram.:Loc. adv.}
\end{itemize}
Parte de uma coisa que se retalhou.
Apara.
Porção de tecido que se cortou de uma peça.
Fracção.
\textunderscore A retalho\textunderscore , aos bocados, por miúdo.
Em pequena escala.
\section{Retaliação}
\begin{itemize}
\item {Grp. gram.:f.}
\end{itemize}
Acto ou effeito de retaliar.
\section{Retaliar}
\begin{itemize}
\item {Grp. gram.:v. t.}
\end{itemize}
\begin{itemize}
\item {Proveniência:(Lat. \textunderscore retaliare\textunderscore )}
\end{itemize}
Tratar com represálias; vingar.
Applicar a pena de talião a.
Desaggravar.
\section{Retama}
\begin{itemize}
\item {Grp. gram.:f.}
\end{itemize}
\begin{itemize}
\item {Proveniência:(Do ár. \textunderscore retama\textunderscore )}
\end{itemize}
O mesmo que \textunderscore giesta\textunderscore .
\section{Retambana}
\begin{itemize}
\item {Grp. gram.:f.}
\end{itemize}
\begin{itemize}
\item {Utilização:Pop.}
\end{itemize}
Descompostura, sarabanda.
\section{Retame}
\begin{itemize}
\item {Grp. gram.:adj.}
\end{itemize}
Diz-se do mel ou melaço, levado ao ponto de açúcar.
\section{Retâmea}
\begin{itemize}
\item {Grp. gram.:f.}
\end{itemize}
\begin{itemize}
\item {Utilização:Ant.}
\end{itemize}
Fecho superior do edifício.
(Relaciona-se com o fr. \textunderscore retamer\textunderscore ?)
\section{Retancha}
\begin{itemize}
\item {Grp. gram.:f.}
\end{itemize}
Acto de retanchar.
Vide ou barbado, para retanchar: \textunderscore plantaram-se mil retanchas\textunderscore .
\section{Retanchamento}
\begin{itemize}
\item {Grp. gram.:m.}
\end{itemize}
Acto de retanchar.
\section{Retanchar}
\begin{itemize}
\item {Grp. gram.:v. t.}
\end{itemize}
\begin{itemize}
\item {Proveniência:(De \textunderscore re\textunderscore  + \textunderscore tanchar\textunderscore )}
\end{itemize}
Substituír (o bacelo) por outro.
Cortar cerce (uma vergôntea), para crescer com mais fôrça.
\section{Retanchôa}
\begin{itemize}
\item {Grp. gram.:f.}
\end{itemize}
Acto ou effeito de retanchar.
\section{Retangueira}
\begin{itemize}
\item {Grp. gram.:f.}
\end{itemize}
\begin{itemize}
\item {Utilização:Prov.}
\end{itemize}
\begin{itemize}
\item {Utilização:beir.}
\end{itemize}
O mesmo que \textunderscore rètaguarda\textunderscore ; parte posterior; traseira.
\section{Retanto}
\begin{itemize}
\item {Grp. gram.:adv.  e  pron.}
\end{itemize}
\begin{itemize}
\item {Proveniência:(De \textunderscore re...\textunderscore  + \textunderscore tanto\textunderscore )}
\end{itemize}
O mesmo que \textunderscore tanto\textunderscore .
Reforçadamente. Cf. G. Vicente, A. Prestes, Chiado, etc.
\section{Retar}
Recusar, rejeitar.
Denunciar o crime de.
(Cp. \textunderscore arretar\textunderscore )
\section{Retardação}
\begin{itemize}
\item {Grp. gram.:f.}
\end{itemize}
\begin{itemize}
\item {Proveniência:(Lat. \textunderscore retardatio\textunderscore )}
\end{itemize}
Acto ou effeito de retardar.
\section{Retardadamente}
\begin{itemize}
\item {Grp. gram.:adv.}
\end{itemize}
De modo retardado.
Com demora; tardiamente.
\section{Retardador}
\begin{itemize}
\item {Grp. gram.:adj.}
\end{itemize}
Que retarda.
\section{Retardamento}
\begin{itemize}
\item {Grp. gram.:m.}
\end{itemize}
O mesmo que \textunderscore retardação\textunderscore .
\section{Retardança}
\begin{itemize}
\item {Grp. gram.:f.}
\end{itemize}
O mesmo que \textunderscore retardação\textunderscore .
\section{Retardão}
\begin{itemize}
\item {Grp. gram.:m.  e  adj.}
\end{itemize}
\begin{itemize}
\item {Utilização:Pop.}
\end{itemize}
\begin{itemize}
\item {Proveniência:(De \textunderscore retardar\textunderscore )}
\end{itemize}
Pachorrento.
Pouco activo.
Teimoso, (falando-se do cavallo).
\section{Retardar}
\begin{itemize}
\item {Grp. gram.:v. t.}
\end{itemize}
\begin{itemize}
\item {Grp. gram.:V. i.  e  p.}
\end{itemize}
\begin{itemize}
\item {Proveniência:(Lat. \textunderscore retardare\textunderscore )}
\end{itemize}
Tornar tardio; demorar; adiar.
Causar demora a; atrasar.
Andar de vagar, demorar-se.
\section{Retardatário}
\begin{itemize}
\item {Grp. gram.:adj.}
\end{itemize}
\begin{itemize}
\item {Proveniência:(De \textunderscore retardar\textunderscore )}
\end{itemize}
Que está ou vem atrasado.
Que chega tarde.
\section{Retardativo}
\begin{itemize}
\item {Grp. gram.:adj.}
\end{itemize}
Que retarda; retardio.
\section{Retardio}
\begin{itemize}
\item {Grp. gram.:adj.}
\end{itemize}
\begin{itemize}
\item {Proveniência:(De \textunderscore retardar\textunderscore )}
\end{itemize}
Tardio, tardo.
Serôdio.
Demorado; pachorrento.
\section{Retardo}
\begin{itemize}
\item {Grp. gram.:m.}
\end{itemize}
\begin{itemize}
\item {Utilização:Mús.}
\end{itemize}
\begin{itemize}
\item {Proveniência:(De \textunderscore retardar\textunderscore )}
\end{itemize}
Effeito da prolongação, retardando uma nota real.
\section{Retear}
\begin{itemize}
\item {Grp. gram.:v. t.}
\end{itemize}
\begin{itemize}
\item {Utilização:Ant.}
\end{itemize}
O mesmo que \textunderscore encurralar\textunderscore .
(Cp. \textunderscore redil\textunderscore )
\section{Reteimar}
\begin{itemize}
\item {Grp. gram.:v. i.}
\end{itemize}
Teimar novamente, teimar muito. Cf. Castilho, \textunderscore Mêd. á Fôrça\textunderscore , 27.
\section{Retelhação}
\begin{itemize}
\item {Grp. gram.:f.}
\end{itemize}
Acto de retelhar; retelhadura. Cf. Vic. Pinheiro, \textunderscore S. Thomé\textunderscore , 350.
\section{Retelhadura}
\begin{itemize}
\item {Grp. gram.:f.}
\end{itemize}
Acto ou effeito de retelhar.
\section{Retelhar}
\begin{itemize}
\item {Grp. gram.:v. t.}
\end{itemize}
\begin{itemize}
\item {Proveniência:(De \textunderscore re...\textunderscore  + \textunderscore telha\textunderscore )}
\end{itemize}
Telhar novamente, pôr novo telhado em.
\section{Retém}
\begin{itemize}
\item {Grp. gram.:m.}
\end{itemize}
Acto ou effeito de reter.
Coisa sobrecellente.
Resto; reserva.
Depósito.
Soldados de piquete, em certos presídios.
\section{Retemirábile}
\begin{itemize}
\item {Grp. gram.:f.}
\end{itemize}
\begin{itemize}
\item {Utilização:Anat.}
\end{itemize}
\begin{itemize}
\item {Proveniência:(Do lat. \textunderscore rete\textunderscore  + \textunderscore mirabilis\textunderscore )}
\end{itemize}
Tecido de artérias muito delgadas sobre o esphenóide.
\section{Retemperar}
\begin{itemize}
\item {Grp. gram.:v. t.}
\end{itemize}
\begin{itemize}
\item {Proveniência:(De \textunderscore re...\textunderscore  + \textunderscore temperar\textunderscore )}
\end{itemize}
Tornar a temperar; dar nova têmpera a.
Melhorar; fortificar; robustecer.
\section{Retempo}
\begin{itemize}
\item {Grp. gram.:m.}
\end{itemize}
\begin{itemize}
\item {Proveniência:(De \textunderscore re...\textunderscore  + \textunderscore tempo\textunderscore )}
\end{itemize}
Occasião muito asada; grande opportunidade.--Us. só na na loc. \textunderscore é tempo e retempo\textunderscore .
\section{Retenção}
\begin{itemize}
\item {Grp. gram.:f.}
\end{itemize}
\begin{itemize}
\item {Proveniência:(Lat. \textunderscore retentio\textunderscore )}
\end{itemize}
Acto ou effeito de reter.
Demora; detenção.
Reserva.
Retentiva.
Cárcere privado.
Accumulação de substâncias nas cavidades orgânicas donde normalmente são expellidas: \textunderscore retenção de urinas\textunderscore .
\section{Retência}
\begin{itemize}
\item {Grp. gram.:f.}
\end{itemize}
\begin{itemize}
\item {Utilização:bras}
\end{itemize}
\begin{itemize}
\item {Utilização:Neol.}
\end{itemize}
\begin{itemize}
\item {Proveniência:(De \textunderscore reter\textunderscore )}
\end{itemize}
O mesmo que \textunderscore retenção\textunderscore .
\section{Retenga-tenga}
\begin{itemize}
\item {Grp. gram.:f.}
\end{itemize}
Árvore africana, de fôlhas inteiras, verde-amareladas, e flôres amarelas, gamo-pétalas.
\section{Retenho}
\begin{itemize}
\item {Grp. gram.:m.}
\end{itemize}
\begin{itemize}
\item {Utilização:Des.}
\end{itemize}
Acto de reter.
Coisa sobrecellente.
O mesmo que \textunderscore retém\textunderscore . Cf. \textunderscore Viriato Trág.\textunderscore , XVI, 63.
\section{Retenida}
\begin{itemize}
\item {Grp. gram.:f.}
\end{itemize}
\begin{itemize}
\item {Proveniência:(De \textunderscore reter\textunderscore )}
\end{itemize}
Cabo náutico, para aguentar temporariamente uma peça.
\section{Retentiva}
\begin{itemize}
\item {Grp. gram.:f.}
\end{itemize}
Faculdade de conservar na memória as impressões recebidas.
(Fem. de \textunderscore retentivo\textunderscore )
\section{Retentivo}
\begin{itemize}
\item {Grp. gram.:adj.}
\end{itemize}
\begin{itemize}
\item {Proveniência:(Lat. \textunderscore retentivus\textunderscore )}
\end{itemize}
Que retém.
\section{Retentor}
\begin{itemize}
\item {Grp. gram.:m.  e  adj.}
\end{itemize}
\begin{itemize}
\item {Proveniência:(Lat. \textunderscore retentor\textunderscore )}
\end{itemize}
O que retém.
\section{Retentriz}
\begin{itemize}
\item {Grp. gram.:adj. f.}
\end{itemize}
Diz-se da faculdade de reter ideias ou conhecimentos. Cf. \textunderscore Diccion. Exeg.\textunderscore ; e Valdez, \textunderscore Diccion. Esp. Port.\textunderscore 
(Fem. de \textunderscore retentor\textunderscore )
\section{Reter}
\begin{itemize}
\item {Grp. gram.:v. t.}
\end{itemize}
\begin{itemize}
\item {Proveniência:(Lat. \textunderscore retinere\textunderscore )}
\end{itemize}
Segurar, para que não escape ou escorregue.
Têr firme.
Não deixar saír da mão.
Guardar, conservar em seu poder.
Lançar mão de.
Manter em cárcere privado.
Conter, refrear: \textunderscore reter os impulsos da ira\textunderscore .
Impedir.
Obrigar a estar, a permanecer.
Conservar na lembrança.
\section{Retesamento}
\begin{itemize}
\item {Grp. gram.:m.}
\end{itemize}
Acto ou effeito de retesar. Cf. Camillo, \textunderscore Brasileira\textunderscore , 281.
\section{Retesar}
\begin{itemize}
\item {Grp. gram.:v. t.}
\end{itemize}
\begin{itemize}
\item {Proveniência:(De \textunderscore reteso\textunderscore )}
\end{itemize}
Tornar tenso; esticar.
Endireitar.
Tornar rijo, teso.
\section{Retesia}
\begin{itemize}
\item {Grp. gram.:f.}
\end{itemize}
Acto de retesiar.
\section{Retesiar}
\begin{itemize}
\item {Grp. gram.:v. i.}
\end{itemize}
\begin{itemize}
\item {Utilização:Prov.}
\end{itemize}
\begin{itemize}
\item {Utilização:minh.}
\end{itemize}
\begin{itemize}
\item {Proveniência:(De \textunderscore retêso\textunderscore ?)}
\end{itemize}
Fazer desordem; têr briga.
\section{Retêso}
\begin{itemize}
\item {Grp. gram.:adj.}
\end{itemize}
\begin{itemize}
\item {Proveniência:(Do lat. \textunderscore retensus\textunderscore )}
\end{itemize}
Muito teso, muito tenso, esticado. Cf. Camillo, \textunderscore Livro Negro\textunderscore , 256.
\section{Reteúdo}
\begin{itemize}
\item {Grp. gram.:adj.}
\end{itemize}
O mesmo que \textunderscore retido\textunderscore . Cf. Filinto, \textunderscore D. Man.\textunderscore , I, 114.
\section{Retiário}
\begin{itemize}
\item {Grp. gram.:m.}
\end{itemize}
O mesmo que \textunderscore reciário\textunderscore . Cf. Filinto, XVI, 331; G. Crespo, \textunderscore Nocturnos\textunderscore , 88.
\section{Reticência}
\begin{itemize}
\item {Grp. gram.:f.}
\end{itemize}
\begin{itemize}
\item {Grp. gram.:Pl.}
\end{itemize}
\begin{itemize}
\item {Proveniência:(Lat. \textunderscore reticentia\textunderscore )}
\end{itemize}
Silêncio voluntário.
Omissão do que devia ou podia dizer-se.
Pontos successivos, que na escrita indicam aquella omissão.
\section{Retícula}
\begin{itemize}
\item {Grp. gram.:f.}
\end{itemize}
O mesmo que \textunderscore retículo\textunderscore .
\section{Reticulação}
\begin{itemize}
\item {Grp. gram.:f.}
\end{itemize}
Estado do que é reticulado.
\section{Reticulado}
\begin{itemize}
\item {Grp. gram.:adj.}
\end{itemize}
\begin{itemize}
\item {Grp. gram.:M. pl.}
\end{itemize}
\begin{itemize}
\item {Proveniência:(Lat. \textunderscore reticulatus\textunderscore )}
\end{itemize}
Que tem fórma de rede.
Que tem linhas ou nervuras cruzadas, á maneira de rêde.
Secção de polypeiros.
\section{Reticular}
\begin{itemize}
\item {Grp. gram.:adj.}
\end{itemize}
O mesmo que \textunderscore reticulado\textunderscore .
\section{Retículo}
\begin{itemize}
\item {Grp. gram.:m.}
\end{itemize}
\begin{itemize}
\item {Utilização:Bot.}
\end{itemize}
\begin{itemize}
\item {Proveniência:(Lat. \textunderscore reticulum\textunderscore )}
\end{itemize}
Pequena rêde.
Rêdezinha, em que as mulheres romanas envolviam as tranças.
Disco, contendo uma abertura central, em que se cruzam fios de platina, e destinado especialmente a medir os diâmetros dos astros.
Nervura que cérca a base das fôlhas.
\section{Retido}
\begin{itemize}
\item {Grp. gram.:adj.}
\end{itemize}
\begin{itemize}
\item {Proveniência:(De \textunderscore reter\textunderscore )}
\end{itemize}
Que se retém.
Detido.
Refreado.
\section{Retiforme}
\begin{itemize}
\item {Grp. gram.:adj.}
\end{itemize}
\begin{itemize}
\item {Proveniência:(Do lat. \textunderscore rete\textunderscore  + \textunderscore forma\textunderscore )}
\end{itemize}
Que tem fórma de rêde.
\section{Retilintar}
\begin{itemize}
\item {Grp. gram.:v. i.}
\end{itemize}
Tilintar muito ou muitas vezes. Cf. Eça, \textunderscore P. Basilio\textunderscore , 107.
\section{Retina}
\begin{itemize}
\item {Grp. gram.:f.}
\end{itemize}
\begin{itemize}
\item {Proveniência:(Lat. \textunderscore retina\textunderscore )}
\end{itemize}
A mais interior membrana do ôlho, em que se fórmam as imagens, e que transmitte a percepção ao cérebro por intermédio do nervo óptico.
\section{Retina}
\begin{itemize}
\item {Grp. gram.:f.}
\end{itemize}
(V.ratina)
\section{Retináculo}
\begin{itemize}
\item {Grp. gram.:m.}
\end{itemize}
\begin{itemize}
\item {Utilização:Bot.}
\end{itemize}
\begin{itemize}
\item {Proveniência:(Lat. \textunderscore retinaculum\textunderscore )}
\end{itemize}
Glândula, na extremidade inferior das massas pollínicas de certos vegetaes.
Ligação da semente ás paredes do fruto.
\section{Retinasfalto}
\begin{itemize}
\item {Grp. gram.:m.}
\end{itemize}
\begin{itemize}
\item {Proveniência:(Do gr. \textunderscore retine\textunderscore  + \textunderscore asfaltos\textunderscore )}
\end{itemize}
Substância mineral, de asfalto geralmente resinoso, solúvel em parte no álcool e formando um resíduo bituminoso e insolúvel.
\section{Retinasphalto}
\begin{itemize}
\item {Grp. gram.:m.}
\end{itemize}
\begin{itemize}
\item {Proveniência:(Do gr. \textunderscore retine\textunderscore  + \textunderscore asphaltos\textunderscore )}
\end{itemize}
Substância mineral, de asphalto geralmente resinoso, solúvel em parte no álcool e formando um resíduo bituminoso e insolúvel.
\section{Retinência}
\begin{itemize}
\item {Grp. gram.:f.}
\end{itemize}
\begin{itemize}
\item {Utilização:Des.}
\end{itemize}
\begin{itemize}
\item {Proveniência:(Do lat. \textunderscore retinere\textunderscore )}
\end{itemize}
O mesmo que \textunderscore retenção\textunderscore . Cf. \textunderscore Diccion. Exeg.\textunderscore 
\section{Retinérveo}
\begin{itemize}
\item {Grp. gram.:adj.}
\end{itemize}
\begin{itemize}
\item {Utilização:Bot.}
\end{itemize}
\begin{itemize}
\item {Proveniência:(De \textunderscore rete\textunderscore  lat. + \textunderscore nérveo\textunderscore )}
\end{itemize}
Que tem nervuras reticulares.
\section{Retingir}
\begin{itemize}
\item {Grp. gram.:v. t.}
\end{itemize}
\begin{itemize}
\item {Proveniência:(Lat. \textunderscore retingere\textunderscore )}
\end{itemize}
Tornar a tingir; tingir bem.
\section{Retiniano}
\begin{itemize}
\item {Grp. gram.:adj.}
\end{itemize}
Relativo á retina^1.
\section{Retínico}
\begin{itemize}
\item {Grp. gram.:adj.}
\end{itemize}
(V.retiniano)
\section{Retinente}
\begin{itemize}
\item {Grp. gram.:adj.}
\end{itemize}
Que retine.
\section{Retinir}
\begin{itemize}
\item {Grp. gram.:v. i.}
\end{itemize}
\begin{itemize}
\item {Utilização:Fig.}
\end{itemize}
\begin{itemize}
\item {Grp. gram.:V. t.}
\end{itemize}
\begin{itemize}
\item {Proveniência:(Lat. \textunderscore retinnire\textunderscore )}
\end{itemize}
Tinir muito ou por muito tempo.
Echoar.
Impressionar vivamente o ânimo.
Fazer soar ou echoar.
\section{Retinite}
\begin{itemize}
\item {Grp. gram.:f.}
\end{itemize}
\begin{itemize}
\item {Utilização:Miner.}
\end{itemize}
Inflammação da retina^1.
Espécie de meláphyro, também conhecida por \textunderscore vidro natural\textunderscore .
\section{Retinito}
\begin{itemize}
\item {Grp. gram.:m.}
\end{itemize}
O mesmo ou melhor que \textunderscore retinite\textunderscore , mineral.
\section{Retinóspora}
\begin{itemize}
\item {Grp. gram.:f.}
\end{itemize}
Planta conífera do Brasil.
\section{Retintim}
\begin{itemize}
\item {Grp. gram.:m.}
\end{itemize}
\begin{itemize}
\item {Proveniência:(T. onom.)}
\end{itemize}
Acto ou effeito de retinir.
Som de objectos metállicos que se tocam:«\textunderscore ...quando se rompeo o retintim de trombetas.\textunderscore »Filinto, \textunderscore D. Man.\textunderscore , I, 391.
\section{Retintínulo}
\begin{itemize}
\item {Grp. gram.:adj.}
\end{itemize}
\begin{itemize}
\item {Utilização:P. us.}
\end{itemize}
Que produz contínuo retintim:«\textunderscore as retintínulas campainhas...\textunderscore »Castilho, \textunderscore Fastos\textunderscore , II, 85.
(Cp. \textunderscore retintim\textunderscore )
\section{Retinto}
\begin{itemize}
\item {Grp. gram.:adj.}
\end{itemize}
\begin{itemize}
\item {Proveniência:(Do lat. \textunderscore retinctus\textunderscore )}
\end{itemize}
Que tem côr carregada.
Que tem o pêlo semelhante ao dos cavallos castanhos, (falando-se do toiro).
\section{Retinto}
\begin{itemize}
\item {Grp. gram.:m.}
\end{itemize}
\begin{itemize}
\item {Utilização:Ant.}
\end{itemize}
\begin{itemize}
\item {Proveniência:(Do lat. \textunderscore retentus\textunderscore )}
\end{itemize}
O mesmo que \textunderscore tento\textunderscore ^1, \textunderscore tino\textunderscore ^1, [[retentiva]]. Cf. G. Vicente, I, 251.
\section{Retintório}
\begin{itemize}
\item {Grp. gram.:m.}
\end{itemize}
\begin{itemize}
\item {Utilização:Prov.}
\end{itemize}
\begin{itemize}
\item {Utilização:minh.}
\end{itemize}
\begin{itemize}
\item {Proveniência:(De \textunderscore retinto\textunderscore )}
\end{itemize}
Planta, cujas raízes servem para tingir ovos, nas festas da Páschoa.
\section{Retíolo}
\begin{itemize}
\item {Grp. gram.:m.}
\end{itemize}
\begin{itemize}
\item {Proveniência:(Lat. \textunderscore retiolum\textunderscore )}
\end{itemize}
Espécie de coifa antiga; retículo. Cf. Herculano, \textunderscore Eurico\textunderscore , (nas notas).
\section{Retípede}
\begin{itemize}
\item {Grp. gram.:adj.}
\end{itemize}
\begin{itemize}
\item {Utilização:Zool.}
\end{itemize}
\begin{itemize}
\item {Proveniência:(Do lat. \textunderscore rete\textunderscore  + \textunderscore pes\textunderscore , \textunderscore pedis\textunderscore )}
\end{itemize}
Que tem os tarsos revestidos de epiderme reticulada, (falando-se de certos animaes).
\section{Retira}
\begin{itemize}
\item {Grp. gram.:f.}
\end{itemize}
\begin{itemize}
\item {Utilização:Des.}
\end{itemize}
O mesmo que \textunderscore retirada\textunderscore .
\section{Retiração}
\begin{itemize}
\item {Grp. gram.:f.}
\end{itemize}
\begin{itemize}
\item {Utilização:Typ.}
\end{itemize}
O mesmo que \textunderscore retirada\textunderscore .
Acto de imprimir o verso de uma fôlha.
\section{Retirada}
\begin{itemize}
\item {Grp. gram.:f.}
\end{itemize}
\begin{itemize}
\item {Utilização:Bras. do N}
\end{itemize}
Acto ou effeito de retirar.
Retiro.
Marcha das tropas, que se afastam ou fogem do inimigo; debandada.
Manada de gado, que, durante as grandes sécas, muda para outra fazenda da mesma região.
\section{Retiradamente}
\begin{itemize}
\item {Grp. gram.:adv.}
\end{itemize}
De modo retirado, insuladamente: \textunderscore viver retiradamente\textunderscore .
\section{Retirado}
\begin{itemize}
\item {Grp. gram.:adj.}
\end{itemize}
Solitário, ermo.
Que vive insuladamente.
\section{Retiramento}
\begin{itemize}
\item {Grp. gram.:m.}
\end{itemize}
\begin{itemize}
\item {Proveniência:(De \textunderscore retirar\textunderscore )}
\end{itemize}
Retirada; vida solitária.
\section{Retirante}
\begin{itemize}
\item {Grp. gram.:m.  e  f.}
\end{itemize}
\begin{itemize}
\item {Utilização:Bras. do N}
\end{itemize}
\begin{itemize}
\item {Proveniência:(De \textunderscore retirar\textunderscore )}
\end{itemize}
Aquelle que se retira.
Pessôa, que, durante as grandes sécas, acossada pela penúria, muda para outro Estado, ou para as serras, ou para o litoral.
Emigrante.
Faminto.
\section{Retirar}
\begin{itemize}
\item {Grp. gram.:v. t.}
\end{itemize}
\begin{itemize}
\item {Grp. gram.:V. i.  e  p.}
\end{itemize}
\begin{itemize}
\item {Proveniência:(De \textunderscore re...\textunderscore  + \textunderscore tirar\textunderscore )}
\end{itemize}
Retrahir, puxar para trás: \textunderscore retirar alguém da beira de um precipício\textunderscore .
Aproximar do si.
Tomar, levantar, recolher: \textunderscore retirar uma quantia que se jogava\textunderscore .
Tirar da presença de alguem.
Tirar.
Fazer saír.
Obter, ganhar: \textunderscore retirar lucros\textunderscore .
Desviar, livrar: \textunderscore retirar de um perigo\textunderscore .
Imprimir o verso de (uma fôlha que já tem impresso o outro lado).
Deixar de dar: \textunderscore retirar a protecção\textunderscore .
Apartar-se, ausentar-se, afastar-se; insular-se.
\section{Retiro}
\begin{itemize}
\item {Grp. gram.:m.}
\end{itemize}
\begin{itemize}
\item {Proveniência:(De \textunderscore retirar\textunderscore )}
\end{itemize}
Lugar solitário, solidão.
Lugar, onde se descansa, longe de trato social.
Remanso.
Retirada.
\section{Reto}
\begin{itemize}
\item {Grp. gram.:m.}
\end{itemize}
\begin{itemize}
\item {Utilização:Ant.}
\end{itemize}
O mesmo que \textunderscore repto\textunderscore .
\section{Retobar}
\begin{itemize}
\item {Grp. gram.:v. t.}
\end{itemize}
\begin{itemize}
\item {Utilização:Bras. do S}
\end{itemize}
O mesmo que \textunderscore retovar\textunderscore .
\section{Retocador}
\begin{itemize}
\item {Grp. gram.:m.  e  adj.}
\end{itemize}
\begin{itemize}
\item {Grp. gram.:M.}
\end{itemize}
\begin{itemize}
\item {Proveniência:(De \textunderscore retocar\textunderscore )}
\end{itemize}
O que retoca.
Instrumento, com que se tira a rebarba do oiro.
\section{Retocar}
\begin{itemize}
\item {Grp. gram.:v. t.}
\end{itemize}
\begin{itemize}
\item {Proveniência:(De \textunderscore re...\textunderscore  + \textunderscore tocar\textunderscore )}
\end{itemize}
Tocar novamente.
Emendar; corrigir; melhorar: \textunderscore retocar um quadro\textunderscore .
Tirar a rebarba, com o retocador, a.
\section{Retoiça}
\begin{itemize}
\item {Grp. gram.:f.}
\end{itemize}
\begin{itemize}
\item {Proveniência:(De \textunderscore retoiçar\textunderscore )}
\end{itemize}
Corda, suspensa pelas duas extremidades, ou assento suspenso por cordas, em que a gente se baloiça.
Baloiço; retoiço.
\section{Retoiçador}
\begin{itemize}
\item {Grp. gram.:m.  e  adj.}
\end{itemize}
O mesmo que \textunderscore retoição\textunderscore .
\section{Retoição}
\begin{itemize}
\item {Grp. gram.:m.  e  adj.}
\end{itemize}
\begin{itemize}
\item {Proveniência:(De \textunderscore retoiçar\textunderscore )}
\end{itemize}
Brincalhão; turbulento.
\section{Retoiçar}
\begin{itemize}
\item {Grp. gram.:v. i.  e  p.}
\end{itemize}
\begin{itemize}
\item {Utilização:Ext.}
\end{itemize}
Brincar na retoiça, baloiçar-se.
Traquinar; espojar-se por brincadeira.
\section{Retoiço}
\begin{itemize}
\item {Grp. gram.:m.}
\end{itemize}
Acto ou effeito de retoiçar.
\section{Retoiçôa}
\begin{itemize}
\item {Grp. gram.:f.}
\end{itemize}
Mulhér brincalhona ou foliona.
(Fem. de \textunderscore retoição\textunderscore )
\section{Retolo}
\begin{itemize}
\item {fónica:tô}
\end{itemize}
\begin{itemize}
\item {Grp. gram.:adj.}
\end{itemize}
\begin{itemize}
\item {Proveniência:(De \textunderscore re...\textunderscore  + \textunderscore tolo\textunderscore )}
\end{itemize}
Muito tolo:«\textunderscore é tolo e retolo.\textunderscore »Castilho, \textunderscore Tartufo\textunderscore .
\section{Retomada}
\begin{itemize}
\item {Grp. gram.:f.}
\end{itemize}
Acto de retomar. Cf. Villa-Maior, \textunderscore Gen. Claudino\textunderscore , 44.
\section{Retomadia}
\begin{itemize}
\item {Grp. gram.:f.}
\end{itemize}
\begin{itemize}
\item {Utilização:Jur.}
\end{itemize}
\begin{itemize}
\item {Utilização:ant.}
\end{itemize}
\begin{itemize}
\item {Proveniência:(De \textunderscore re...\textunderscore  + \textunderscore tomadia\textunderscore )}
\end{itemize}
Acto de retomar.
Nova tomadia. Cf. \textunderscore Supplem. ao Diccion. de Algib.\textunderscore 
\section{Retomar}
\begin{itemize}
\item {Grp. gram.:v. t.}
\end{itemize}
\begin{itemize}
\item {Proveniência:(De \textunderscore re...\textunderscore  + \textunderscore tomar\textunderscore )}
\end{itemize}
Tomar do novo; rehaver.
\section{Retoque}
\begin{itemize}
\item {Grp. gram.:m.}
\end{itemize}
Acto ou effeito de retocar.
\section{Retorção}
\begin{itemize}
\item {Grp. gram.:f.}
\end{itemize}
(V.retorsão)
\section{Retorce}
\begin{itemize}
\item {Grp. gram.:m.}
\end{itemize}
Acto de retorcer.
Officina, onde se retorce o fiado, nas fabricas de fiação.
\section{Retorcedeira}
\begin{itemize}
\item {Grp. gram.:f.}
\end{itemize}
\begin{itemize}
\item {Proveniência:(De \textunderscore retorcer\textunderscore )}
\end{itemize}
Máquina das fábricas de lanifícios, para unir dois ou mais fios e torcê-los juntos.
\section{Retorcedura}
\begin{itemize}
\item {Grp. gram.:f.}
\end{itemize}
Acto ou effeito de retorcer.
\section{Retorcer}
\begin{itemize}
\item {Grp. gram.:v. t.}
\end{itemize}
\begin{itemize}
\item {Grp. gram.:V. p.}
\end{itemize}
\begin{itemize}
\item {Utilização:Fig.}
\end{itemize}
\begin{itemize}
\item {Proveniência:(Do lat. \textunderscore retorquere\textunderscore )}
\end{itemize}
Torcer novamente.
Torcer muitas vezes.
Torcer-se muito; contorcer-se.
Procurar evasivas, tergiversar.
\section{Retorcida}
\begin{itemize}
\item {Grp. gram.:f.}
\end{itemize}
\begin{itemize}
\item {Utilização:Bras. do S}
\end{itemize}
\begin{itemize}
\item {Proveniência:(De \textunderscore retorcer\textunderscore )}
\end{itemize}
Bailado campestre, espécie de fandango.
\section{Retorcido}
\begin{itemize}
\item {Grp. gram.:adj.}
\end{itemize}
\begin{itemize}
\item {Proveniência:(De \textunderscore retorcer\textunderscore )}
\end{itemize}
Muito torcido.
Muito torto: \textunderscore um chavelho retorcido\textunderscore .
\section{Retornaboda}
\begin{itemize}
\item {fónica:bô}
\end{itemize}
\begin{itemize}
\item {Grp. gram.:f.}
\end{itemize}
O mesmo que \textunderscore tornaboda\textunderscore . Cf. B. Pereira, \textunderscore Prosódia\textunderscore , vb. \textunderscore nepotia\textunderscore .
\section{Retornamento}
\begin{itemize}
\item {Grp. gram.:m.}
\end{itemize}
O mesmo que \textunderscore retôrno\textunderscore .
\section{Retornança}
\begin{itemize}
\item {Grp. gram.:f.}
\end{itemize}
\begin{itemize}
\item {Utilização:Des.}
\end{itemize}
O mesmo que \textunderscore retôrno\textunderscore .
\section{Retornar}
\begin{itemize}
\item {Grp. gram.:v. i.}
\end{itemize}
\begin{itemize}
\item {Grp. gram.:V. t.}
\end{itemize}
\begin{itemize}
\item {Proveniência:(De \textunderscore re...\textunderscore  + \textunderscore tornar\textunderscore )}
\end{itemize}
Voltar ao ponto donde se partiu.
Trazer, regressando; restituir. Cf. Bernárdez, \textunderscore Luz e Calor\textunderscore , 380.
\section{Retornello}
\begin{itemize}
\item {Grp. gram.:m.}
\end{itemize}
\begin{itemize}
\item {Proveniência:(It. \textunderscore retornello\textunderscore )}
\end{itemize}
O mesmo ou melhor que \textunderscore ritornello\textunderscore . Cf. Macedo, \textunderscore Burros\textunderscore , 8.
\section{Retornelo}
\begin{itemize}
\item {Grp. gram.:m.}
\end{itemize}
\begin{itemize}
\item {Proveniência:(It. \textunderscore retornello\textunderscore )}
\end{itemize}
O mesmo ou melhor que \textunderscore ritornelo\textunderscore . Cf. Macedo, \textunderscore Burros\textunderscore , 8.
\section{Retôrno}
\begin{itemize}
\item {Grp. gram.:m.}
\end{itemize}
\begin{itemize}
\item {Utilização:Náut.}
\end{itemize}
Acto ou effeito de retornar.
Mercadoria, que se trouxe em troca da que se levou.
Aquillo que se dá em troca do que se recebeu.
Dádiva em compensação.
Grande cabo náutico, que passa por meio das papoias.
\section{Retorquir}
\begin{itemize}
\item {Grp. gram.:v. t.}
\end{itemize}
\begin{itemize}
\item {Grp. gram.:V. i.}
\end{itemize}
\begin{itemize}
\item {Proveniência:(Lat. \textunderscore retorquere\textunderscore )}
\end{itemize}
Replicar, objectar.
Contrapor.
Oppor argumento a argumento; retrucar.
\section{Retorquível}
\begin{itemize}
\item {Grp. gram.:adj.}
\end{itemize}
Que se póde retorquir.
\section{Retorsão}
\begin{itemize}
\item {Grp. gram.:f.}
\end{itemize}
\begin{itemize}
\item {Proveniência:(Do lat. \textunderscore retorsum\textunderscore )}
\end{itemize}
Acto ou effeito de retorcer.
Acto de retorquir.
Réplica.
Espécie de represálias, que consiste em estabelecer para com os estrangeiros residentes entre nós a mesma legislação que o Govêrno dêlles estabelece no seu país para com os nossos compatriotas.
\section{Retorta}
\begin{itemize}
\item {Grp. gram.:f.}
\end{itemize}
\begin{itemize}
\item {Utilização:Zool.}
\end{itemize}
\begin{itemize}
\item {Utilização:Ant.}
\end{itemize}
\begin{itemize}
\item {Proveniência:(Lat. \textunderscore retorta\textunderscore )}
\end{itemize}
Parte curva do báculo.
Vaso bojudo de vidro ou loiça, com gargalo curvo e voltado para baixo.
O mesmo que \textunderscore papa-formigas\textunderscore .
\textunderscore Retorta moirisca\textunderscore , antiga dança da Côrte, em rigoroso traje muçulmano.
\section{Retôrto}
\begin{itemize}
\item {Grp. gram.:adj.}
\end{itemize}
\begin{itemize}
\item {Proveniência:(Lat. \textunderscore retortus\textunderscore )}
\end{itemize}
Muito torto; retorcido.
\section{Retos}
\begin{itemize}
\item {Grp. gram.:m. pl.}
\end{itemize}
\begin{itemize}
\item {Utilização:Bras}
\end{itemize}
Parola, palavreado; ditos agudos.
(Talvez por \textunderscore reptos\textunderscore )
\section{Retos}
\begin{itemize}
\item {Grp. gram.:m. pl.}
\end{itemize}
\begin{itemize}
\item {Utilização:T. de Alcanena}
\end{itemize}
\begin{itemize}
\item {Proveniência:(Do lat. \textunderscore redditus\textunderscore ? do lat. \textunderscore retro\textunderscore ?)}
\end{itemize}
Juros caídos, acumulados.
\section{Retostar}
\begin{itemize}
\item {Grp. gram.:v. t.}
\end{itemize}
\begin{itemize}
\item {Proveniência:(De \textunderscore re...\textunderscore  + \textunderscore tostar\textunderscore )}
\end{itemize}
Tostar muito.
\section{Retouça}
\begin{itemize}
\item {Grp. gram.:f.}
\end{itemize}
\begin{itemize}
\item {Proveniência:(De \textunderscore retouçar\textunderscore )}
\end{itemize}
Corda, suspensa pelas duas extremidades, ou assento suspenso por cordas, em que a gente se balouça.
Balouço; retouço.
\section{Retouçador}
\begin{itemize}
\item {Grp. gram.:m.  e  adj.}
\end{itemize}
O mesmo que \textunderscore retoução\textunderscore .
\section{Retoução}
\begin{itemize}
\item {Grp. gram.:m.  e  adj.}
\end{itemize}
\begin{itemize}
\item {Proveniência:(De \textunderscore retouçar\textunderscore )}
\end{itemize}
Brincalhão; turbulento.
\section{Retoucar}
\begin{itemize}
\item {Grp. gram.:v. t.}
\end{itemize}
\begin{itemize}
\item {Proveniência:(De \textunderscore re...\textunderscore  + \textunderscore toucar\textunderscore )}
\end{itemize}
Toucar novamente.
Revestir superiormente: \textunderscore retoucar de loiros uma estátua\textunderscore .
\section{Retouçar}
\begin{itemize}
\item {Grp. gram.:v. i.  e  p.}
\end{itemize}
\begin{itemize}
\item {Utilização:Ext.}
\end{itemize}
Brincar na retouça, baloiçar-se.
Traquinar; espojar-se por brincadeira.
\section{Retouço}
\begin{itemize}
\item {Grp. gram.:m.}
\end{itemize}
Acto ou effeito de retouçar.
\section{Retouçôa}
\begin{itemize}
\item {Grp. gram.:f.}
\end{itemize}
Mulhér brincalhona ou foliona.
(Fem. de \textunderscore retoição\textunderscore )
\section{Retovar}
\begin{itemize}
\item {Grp. gram.:v. t.}
\end{itemize}
\begin{itemize}
\item {Utilização:Bras. do S}
\end{itemize}
Forrar de coiro.
\section{Retraçar}
\begin{itemize}
\item {Grp. gram.:v. t.}
\end{itemize}
\begin{itemize}
\item {Proveniência:(De \textunderscore re...\textunderscore  + \textunderscore traçar\textunderscore )}
\end{itemize}
Tornar a traçar.
Reduzir a retraço.
\section{Retraçar-se}
\begin{itemize}
\item {Grp. gram.:v. p.}
\end{itemize}
\begin{itemize}
\item {Utilização:Ant.}
\end{itemize}
\begin{itemize}
\item {Proveniência:(Do lat. \textunderscore retractare\textunderscore )}
\end{itemize}
Retrahir-se, retirar-se.
\section{Retracção}
\begin{itemize}
\item {Grp. gram.:f.}
\end{itemize}
\begin{itemize}
\item {Proveniência:(Do lat. \textunderscore retractio\textunderscore )}
\end{itemize}
Acto ou effeito de retrahir.
\section{Retraço}
\begin{itemize}
\item {Grp. gram.:m.}
\end{itemize}
\begin{itemize}
\item {Utilização:Ext.}
\end{itemize}
\begin{itemize}
\item {Proveniência:(De \textunderscore retraçar\textunderscore )}
\end{itemize}
Porção de palha retraçada.
Resíduos da palha, que se deu como ração ás bêstas.
Palha cortada miudamente.
Restos.
Bagatela.
\section{Retractação}
\begin{itemize}
\item {Grp. gram.:f.}
\end{itemize}
Acto ou effeito de retractar.
\section{Retractador}
\begin{itemize}
\item {Grp. gram.:m.  e  adj.}
\end{itemize}
O que retracta.
\section{Retractar}
\begin{itemize}
\item {Grp. gram.:v. i.}
\end{itemize}
\begin{itemize}
\item {Grp. gram.:V. p.}
\end{itemize}
\begin{itemize}
\item {Proveniência:(Lat. \textunderscore retractare\textunderscore )}
\end{itemize}
Tratar de novo.
Têr como não dito, desdizer-se de.
Desdizer-se; confessar que errou.
\section{Retractável}
\begin{itemize}
\item {Grp. gram.:adj.}
\end{itemize}
Que se póde ou se deve retractar.
\section{Retráctil}
\begin{itemize}
\item {Grp. gram.:adj.}
\end{itemize}
\begin{itemize}
\item {Proveniência:(Do lat. \textunderscore retractus\textunderscore )}
\end{itemize}
Que se retrái ou se póde retrahir.
Que produz retracção.
\section{Retractilidade}
\begin{itemize}
\item {Grp. gram.:f.}
\end{itemize}
Qualidade do que é retráctil.
\section{Retractivo}
\begin{itemize}
\item {Grp. gram.:adj.}
\end{itemize}
O mesmo que \textunderscore retráctil\textunderscore .
\section{Retracto}
\begin{itemize}
\item {Grp. gram.:m.}
\end{itemize}
Acto de retractar.
\section{Retraêr}
\textunderscore v. t.\textunderscore  (e der.)
(Fórma ant. de \textunderscore retractar\textunderscore , etc.)
\section{Retrahido}
\begin{itemize}
\item {Grp. gram.:adj.}
\end{itemize}
\begin{itemize}
\item {Utilização:Fig.}
\end{itemize}
\begin{itemize}
\item {Proveniência:(De \textunderscore retrahir\textunderscore )}
\end{itemize}
Puxado para trás.
Que procede com reserva; que não é expansivo.
Calado ou acanhado no falar.
\section{Retrahimento}
\begin{itemize}
\item {Grp. gram.:m.}
\end{itemize}
Acto ou effeito de retrahir.
Insulamento.
Retiro, lugar ermo, ou solitário.
Retirada.
Procedimento reservado.
Contracção de certas substâncias.
Deminuição de volume.
\section{Retrahir}
\begin{itemize}
\item {Grp. gram.:v. t.}
\end{itemize}
\begin{itemize}
\item {Utilização:Fig.}
\end{itemize}
\begin{itemize}
\item {Grp. gram.:V. p.}
\end{itemize}
\begin{itemize}
\item {Proveniência:(Lat. \textunderscore retrahere\textunderscore )}
\end{itemize}
Puxar para trás, retirar.
Contrahir, apertar; encolher.
Occultar.
Tornar reservado.
Impedir.
Retirar-se.
Recuar a pouco e pouco.
Insular-se, concentrar-se.
Tornar-se reservado.
Occultar o que pensa.
Retractar-se.
\section{Retraído}
\begin{itemize}
\item {Grp. gram.:adj.}
\end{itemize}
\begin{itemize}
\item {Utilização:Fig.}
\end{itemize}
\begin{itemize}
\item {Proveniência:(De \textunderscore retrair\textunderscore )}
\end{itemize}
Puxado para trás.
Que procede com reserva; que não é expansivo.
Calado ou acanhado no falar.
\section{Retraimento}
\begin{itemize}
\item {fónica:tra-i}
\end{itemize}
\begin{itemize}
\item {Grp. gram.:m.}
\end{itemize}
Acto ou efeito de retrair.
Insulamento.
Retiro, lugar ermo, ou solitário.
Retirada.
Procedimento reservado.
Contracção de certas substâncias.
Deminuição de volume.
\section{Retrair}
\begin{itemize}
\item {Grp. gram.:v. t.}
\end{itemize}
\begin{itemize}
\item {Utilização:Fig.}
\end{itemize}
\begin{itemize}
\item {Grp. gram.:V. p.}
\end{itemize}
\begin{itemize}
\item {Proveniência:(Lat. \textunderscore retrahere\textunderscore )}
\end{itemize}
Puxar para trás, retirar.
Contrahir, apertar; encolher.
Occultar.
Tornar reservado.
Impedir.
Retirar-se.
Recuar a pouco e pouco.
Insular-se, concentrar-se.
Tornar-se reservado.
Occultar o que pensa.
Retractar-se.
\section{Retrama}
\begin{itemize}
\item {Grp. gram.:f.}
\end{itemize}
\begin{itemize}
\item {Utilização:Prov.}
\end{itemize}
\begin{itemize}
\item {Utilização:trasm.}
\end{itemize}
\begin{itemize}
\item {Proveniência:(De \textunderscore retramar\textunderscore )}
\end{itemize}
Mato sêco, que se põe sôbre os tectos dos palheiros, e sôbre o qual se assenta depois o côlmo.
\section{Retramar}
\begin{itemize}
\item {Grp. gram.:v. t.}
\end{itemize}
\begin{itemize}
\item {Proveniência:(De \textunderscore re...\textunderscore  + \textunderscore tramar\textunderscore )}
\end{itemize}
Tornar a tramar.
\section{Retranca}
\begin{itemize}
\item {Grp. gram.:f.}
\end{itemize}
\begin{itemize}
\item {Utilização:Náut.}
\end{itemize}
\begin{itemize}
\item {Proveniência:(Do lat. \textunderscore retro\textunderscore  + \textunderscore anca\textunderscore )}
\end{itemize}
Correia, que passa por baixo da cauda das bêstas e cujas extremidades se ligam á parte posterior da sella.
Uma das vêrgas do mastro de mezena.
\section{Retransido}
\begin{itemize}
\item {Grp. gram.:adj.}
\end{itemize}
\begin{itemize}
\item {Proveniência:(De \textunderscore retransir\textunderscore )}
\end{itemize}
Repassado.
\section{Retransir}
\begin{itemize}
\item {fónica:zir}
\end{itemize}
\begin{itemize}
\item {Grp. gram.:v. t.}
\end{itemize}
\begin{itemize}
\item {Proveniência:(Lat. \textunderscore retransire\textunderscore )}
\end{itemize}
Traspassar; penetrar intimamente.
\section{Retratado}
\begin{itemize}
\item {Grp. gram.:adj.}
\end{itemize}
\begin{itemize}
\item {Proveniência:(De \textunderscore retratar\textunderscore )}
\end{itemize}
Reproduzido pela pintura, pela photographia, etc.
Espelhado, reflectido.
Bem descrito.
\section{Retratador}
\begin{itemize}
\item {Grp. gram.:m.  e  adj.}
\end{itemize}
\begin{itemize}
\item {Proveniência:(De \textunderscore retratar\textunderscore )}
\end{itemize}
O que retrata.
\section{Retratar}
\begin{itemize}
\item {Grp. gram.:v. i.}
\end{itemize}
\begin{itemize}
\item {Grp. gram.:V. p.}
\end{itemize}
\begin{itemize}
\item {Proveniência:(Lat. \textunderscore retractare\textunderscore )}
\end{itemize}
Tratar de novo.
Têr como não dito, desdizer-se de.
Desdizer-se; confessar que errou.
\section{Retratar}
\begin{itemize}
\item {Grp. gram.:v. t.}
\end{itemize}
\begin{itemize}
\item {Proveniência:(Lat. \textunderscore retractare\textunderscore  )}
\end{itemize}
Fazer o retrato de.
Tirar a photographia de.
Photographar, pintar ou desenhar a figura ou a imagem de.
Representar com exactidão.
Descrever perfeitamente.
Dar relêvo a.
Revelar, mostrar no rosto.
Deixar transparecer.
O mesmo que \textunderscore retractar\textunderscore .
\section{Retratista}
\begin{itemize}
\item {Grp. gram.:m.  e  f.}
\end{itemize}
Pessôa, que faz retratos.
Pessôa, que photographa.
\section{Retrato}
\begin{itemize}
\item {Grp. gram.:m.}
\end{itemize}
\begin{itemize}
\item {Proveniência:(De \textunderscore retratar\textunderscore )}
\end{itemize}
Imagem.
Desenho ou pintura, que representa a imagem de alguém.
Figura ou cópia perfeita das feições de alguém.
Pessôa, cujas feições são iguaes ás de outra.
Carácter.
Descripção.
Modêlo; cópia.
\section{Retrautar}
\textunderscore v. t.\textunderscore  (e der.)
Fórma ant. de \textunderscore retractar\textunderscore , etc.
\section{Retravar}
\begin{itemize}
\item {Grp. gram.:v. t.}
\end{itemize}
\begin{itemize}
\item {Proveniência:(De \textunderscore re...\textunderscore  + \textunderscore travar\textunderscore )}
\end{itemize}
Travar novamente.
Principiar de novo.
\section{Retrazer}
\begin{itemize}
\item {Grp. gram.:v. t.}
\end{itemize}
\begin{itemize}
\item {Proveniência:(De \textunderscore re...\textunderscore  + \textunderscore trazer\textunderscore )}
\end{itemize}
Trazer de novo. Cf. Filinto, III, 20.
\section{Retremente}
\begin{itemize}
\item {Grp. gram.:adj.}
\end{itemize}
Que retreme. Cf. Castilho, \textunderscore Geórgicas\textunderscore , 61.
\section{Retremer}
\begin{itemize}
\item {Grp. gram.:v. t.}
\end{itemize}
\begin{itemize}
\item {Proveniência:(De \textunderscore re...\textunderscore  + \textunderscore tremer\textunderscore )}
\end{itemize}
Tremer de novo.
Tremer muito ou por muito tempo:«\textunderscore ...roncos das bombardas, com que retremia a terra\textunderscore ». Filinto, \textunderscore D. Man.\textunderscore , I, 279.
\section{Retrêmulo}
\begin{itemize}
\item {Grp. gram.:adj.}
\end{itemize}
\begin{itemize}
\item {Proveniência:(De \textunderscore re...\textunderscore  + \textunderscore trémulo\textunderscore )}
\end{itemize}
Que retreme. Cf. Castilho, \textunderscore Metam.\textunderscore , 156.
\section{Retreta}
\begin{itemize}
\item {fónica:trê}
\end{itemize}
\begin{itemize}
\item {Grp. gram.:f.}
\end{itemize}
\begin{itemize}
\item {Proveniência:(Fr. \textunderscore retraite\textunderscore )}
\end{itemize}
Formatura de soldados, ao fim do dia, para se verificar se algum falta.
Criada particular da raínha ou de alguma infanta.
O mesmo que \textunderscore retrete\textunderscore .
\section{Retrete}
\begin{itemize}
\item {Grp. gram.:f.}
\end{itemize}
\begin{itemize}
\item {Grp. gram.:M.}
\end{itemize}
\begin{itemize}
\item {Utilização:Des.}
\end{itemize}
O mesmo que \textunderscore latrina\textunderscore .
A parte mais retirada de uma habitação; retiro; esconso:«\textunderscore no retrete de meu peito\textunderscore ». \textunderscore Aulegrafia\textunderscore , 100 e 144.
\section{Retribuição}
\begin{itemize}
\item {fónica:bu-i}
\end{itemize}
\begin{itemize}
\item {Grp. gram.:f.}
\end{itemize}
\begin{itemize}
\item {Proveniência:(Do lat. \textunderscore retributio\textunderscore )}
\end{itemize}
Acto ou effeito de retribuir.
Remuneração.
Compensação.
Prêmio.
Acto de reconhecer o favor que se recebe.
\section{Retribuidor}
\begin{itemize}
\item {fónica:bu-i}
\end{itemize}
\begin{itemize}
\item {Grp. gram.:m.  e  adj.}
\end{itemize}
\begin{itemize}
\item {Proveniência:(Do lat. \textunderscore retributor\textunderscore )}
\end{itemize}
O que retribue.
\section{Retribuir}
\begin{itemize}
\item {Grp. gram.:v. t.}
\end{itemize}
\begin{itemize}
\item {Proveniência:(Lat. \textunderscore retribuere\textunderscore )}
\end{itemize}
Remunerar, recompensar.
Pagar.
Premiar.
Gratificar.
Corresponder a: \textunderscore retribuir affectos\textunderscore .
Compensar.
\section{Retrilhar}
\begin{itemize}
\item {Grp. gram.:v. t.}
\end{itemize}
\begin{itemize}
\item {Proveniência:(De \textunderscore re...\textunderscore  + \textunderscore trilhar\textunderscore )}
\end{itemize}
Trilhar novamente; repisar.
\section{Retrincado}
\begin{itemize}
\item {Grp. gram.:adj.}
\end{itemize}
\begin{itemize}
\item {Utilização:Prov.}
\end{itemize}
\begin{itemize}
\item {Utilização:trasm.}
\end{itemize}
\begin{itemize}
\item {Proveniência:(De \textunderscore retrincar\textunderscore )}
\end{itemize}
Malicioso.
Dissimulado.
Que cerra os dentes com raiva; odiento.
\section{Retrincar}
\begin{itemize}
\item {Grp. gram.:v. t.}
\end{itemize}
\begin{itemize}
\item {Grp. gram.:V. i.}
\end{itemize}
\begin{itemize}
\item {Proveniência:(De \textunderscore re...\textunderscore  + \textunderscore trincar\textunderscore )}
\end{itemize}
Trincar novamente, repetidas vezes.
Dar mau sentido a.
Dar interpretação maliciosa.
Murmurar.
\section{Retro}
\begin{itemize}
\item {Grp. gram.:m.}
\end{itemize}
\begin{itemize}
\item {Proveniência:(Do lat. \textunderscore retro\textunderscore )}
\end{itemize}
Primeira página de uma fôlha.
\section{Retro...}
\begin{itemize}
\item {Grp. gram.:pref.}
\end{itemize}
\begin{itemize}
\item {Proveniência:(Do lat. \textunderscore retro\textunderscore )}
\end{itemize}
(designativo de \textunderscore atrás\textunderscore , \textunderscore para trás\textunderscore , etc.)
\section{Retroacção}
\begin{itemize}
\item {Grp. gram.:f.}
\end{itemize}
\begin{itemize}
\item {Proveniência:(De \textunderscore retro...\textunderscore  + \textunderscore acção\textunderscore )}
\end{itemize}
Effeito do que é retroactivo; acto de retroagir.
\section{Retroactivamente}
\begin{itemize}
\item {Grp. gram.:adv.}
\end{itemize}
De modo retroactivo.
\section{Retroactividade}
\begin{itemize}
\item {Grp. gram.:f.}
\end{itemize}
Qualidade do que é retroactivo.
\section{Retroactivo}
\begin{itemize}
\item {Grp. gram.:adj.}
\end{itemize}
\begin{itemize}
\item {Proveniência:(De \textunderscore retro...\textunderscore  + \textunderscore activo\textunderscore )}
\end{itemize}
Relativo a coisas passadas.
Que modifica o que está feito.
Que tem effeito sôbre o passado ou sôbre factos passados.
Que retroage.
\section{Retroactor}
\begin{itemize}
\item {Grp. gram.:m.}
\end{itemize}
\begin{itemize}
\item {Proveniência:(Do lat. \textunderscore retro...\textunderscore  + \textunderscore actor\textunderscore )}
\end{itemize}
Aquelle ou aquillo que faz retroagir. Cf. Garrett, \textunderscore Port. na Balança\textunderscore , 114 e 121.
\section{Retroagir}
\begin{itemize}
\item {Grp. gram.:v. i.}
\end{itemize}
\begin{itemize}
\item {Utilização:Neol.}
\end{itemize}
\begin{itemize}
\item {Proveniência:(Lat. \textunderscore retroagere\textunderscore )}
\end{itemize}
Têr effeito sôbre o passado.
Modificar o que está feito.
Retrahir a sua acção ao passado. Cf. Assis, \textunderscore Águas\textunderscore , 241.
\section{Retroar}
\begin{itemize}
\item {Grp. gram.:v. i.}
\end{itemize}
\begin{itemize}
\item {Proveniência:(De \textunderscore re...\textunderscore  + \textunderscore troar\textunderscore )}
\end{itemize}
Troar muito, longamente; troar de novo; retumbar.
\section{Retrocados}
\begin{itemize}
\item {Grp. gram.:m. pl.}
\end{itemize}
O mesmo que \textunderscore trocados\textunderscore .
\section{Retrocarga}
\begin{itemize}
\item {Grp. gram.:f.}
\end{itemize}
\begin{itemize}
\item {Proveniência:(De \textunderscore retro...\textunderscore  + \textunderscore carga\textunderscore )}
\end{itemize}
Acto ou effeito de carregar (uma espingarda) pela culatra.
\section{Retrocedente}
\begin{itemize}
\item {Grp. gram.:m.  e  adj.}
\end{itemize}
O que retrocede; o que faz retrocessão.
\section{Retroceder}
\begin{itemize}
\item {Grp. gram.:v. i.}
\end{itemize}
\begin{itemize}
\item {Utilização:Fig.}
\end{itemize}
\begin{itemize}
\item {Utilização:Ant.}
\end{itemize}
\begin{itemize}
\item {Grp. gram.:V. t.}
\end{itemize}
\begin{itemize}
\item {Proveniência:(Lat. \textunderscore retrocedere\textunderscore )}
\end{itemize}
Andar para trás, recuar.
Retirar-se.
Decair; desandar.
Fazer retrocessão.
O mesmo que \textunderscore apostatar\textunderscore .
Fazer retrocessão de.
\section{Retrocedimento}
\begin{itemize}
\item {Grp. gram.:m.}
\end{itemize}
O mesmo que \textunderscore retrocesso\textunderscore .
\section{Retrocessão}
\begin{itemize}
\item {Grp. gram.:f.}
\end{itemize}
\begin{itemize}
\item {Utilização:Jur.}
\end{itemize}
\begin{itemize}
\item {Proveniência:(De \textunderscore retro...\textunderscore  + \textunderscore cessão\textunderscore )}
\end{itemize}
O mesmo que \textunderscore retrocesso\textunderscore .
Cessão de um direito que se obteve também por cessão.
Mudança de lugar, feita no organismo por um princípio mórbido.
Retrahimento de cóccyx, na occasião do parto.
\section{Retrocessivo}
\begin{itemize}
\item {Grp. gram.:adj.}
\end{itemize}
\begin{itemize}
\item {Proveniência:(De \textunderscore retrocesso\textunderscore )}
\end{itemize}
Que faz retroceder; que produz retrocessão.
\section{Retrocesso}
\begin{itemize}
\item {Grp. gram.:m.}
\end{itemize}
\begin{itemize}
\item {Proveniência:(Lat. \textunderscore retrocessus\textunderscore )}
\end{itemize}
Acto ou effeito de retroceder.
Acto de voltar a um estado anterior.
Acto de retirar ou recuar.
Acto de retrogradar.
Movimento para trás.
\section{Rètroceto!}
\begin{itemize}
\item {Grp. gram.:interj.}
\end{itemize}
\begin{itemize}
\item {Utilização:Prov.}
\end{itemize}
\begin{itemize}
\item {Utilização:trasm.}
\end{itemize}
Credo!
Cruzes!
Eu te arrenego!
(Talvez da loc. biblica \textunderscore vade retro Satana\textunderscore ; mas, neste caso, deverá escrever-se \textunderscore rètrosseto\textunderscore !)
\section{Rètroflexão}
\begin{itemize}
\item {fónica:csão}
\end{itemize}
\begin{itemize}
\item {Grp. gram.:f.}
\end{itemize}
Estado do que é rètroflexo.
\section{Rètroflexo}
\begin{itemize}
\item {fónica:cso}
\end{itemize}
\begin{itemize}
\item {Grp. gram.:adj.}
\end{itemize}
\begin{itemize}
\item {Proveniência:(Lat. \textunderscore retroflexus\textunderscore )}
\end{itemize}
Que se curva ou se dobra para trás.
\section{Retrogradação}
\begin{itemize}
\item {Grp. gram.:f.}
\end{itemize}
\begin{itemize}
\item {Proveniência:(Lat. \textunderscore retrogradatio\textunderscore )}
\end{itemize}
Acto ou effeito de retrogradar.
Acto de retroceder; acto de atrasar.
\section{Retrogradamente}
\begin{itemize}
\item {Grp. gram.:adv.}
\end{itemize}
De modo retrógrado; com retrocesso.
\section{Retrogradar}
\begin{itemize}
\item {Grp. gram.:v. i.}
\end{itemize}
\begin{itemize}
\item {Proveniência:(Lat. \textunderscore retrogradi\textunderscore )}
\end{itemize}
Andar para trás, retroceder.
Recuar.
Andar para um estado anterior.
Proceder, em opposição ao progresso.
\section{Retrógrado}
\begin{itemize}
\item {Grp. gram.:adj.}
\end{itemize}
\begin{itemize}
\item {Grp. gram.:M.}
\end{itemize}
\begin{itemize}
\item {Proveniência:(Lat. \textunderscore retrogradus\textunderscore )}
\end{itemize}
Que retrograda.
Que retrocede.
Que é contrário ao progresso.
Que tem opiniões antiquadas.
Indivíduo retrógrado, reaccionário.
\section{Retrogressão}
\begin{itemize}
\item {Grp. gram.:f.}
\end{itemize}
\begin{itemize}
\item {Proveniência:(Do lat. \textunderscore retrogressus\textunderscore )}
\end{itemize}
O mesmo que \textunderscore retrogradação\textunderscore .
\section{Rètroguarda}
\begin{itemize}
\item {Grp. gram.:f.}
\end{itemize}
\begin{itemize}
\item {Utilização:Ant.}
\end{itemize}
O mesmo que \textunderscore rètaguarda\textunderscore .
\section{Retrós}
\begin{itemize}
\item {Grp. gram.:m.}
\end{itemize}
\begin{itemize}
\item {Proveniência:(Do lat. \textunderscore re...\textunderscore  + \textunderscore torsus\textunderscore )}
\end{itemize}
Fio de seda, ou conjunto de fios de seda, torcidos.
\section{Retrosaria}
\begin{itemize}
\item {Grp. gram.:f.}
\end{itemize}
Loja de retroseiro.
Porção de retrós de várias qualidades.
\section{Rètroseguir}
\begin{itemize}
\item {Grp. gram.:v. i.}
\end{itemize}
\begin{itemize}
\item {Proveniência:(De \textunderscore retro...\textunderscore  + \textunderscore seguir\textunderscore )}
\end{itemize}
Retrogradar.
\section{Retroseiro}
\begin{itemize}
\item {Grp. gram.:m.}
\end{itemize}
Vendedor de retrós ou de objectos feitos de fio de seda, de passamanes, etc.
\section{Rètrospecção}
\begin{itemize}
\item {Grp. gram.:f.}
\end{itemize}
O mesmo que \textunderscore rètrospecto\textunderscore .
\section{Rètrospectivamente}
\begin{itemize}
\item {Grp. gram.:adv.}
\end{itemize}
De modo rètrospectivo; com rètrospecção.
\section{Rètrospectividade}
\begin{itemize}
\item {Grp. gram.:f.}
\end{itemize}
Qualidade de rètrospectivo. Cf. Ortigão, \textunderscore Hollanda\textunderscore , 66.
\section{Rètrospectivo}
\begin{itemize}
\item {Grp. gram.:adj.}
\end{itemize}
\begin{itemize}
\item {Proveniência:(De \textunderscore rètrospecto\textunderscore )}
\end{itemize}
Que se volta para o passado; que olha para trás.
Relativo a coisas passadas: \textunderscore anályse retrospectiva\textunderscore .
\section{Rètrospecto}
\begin{itemize}
\item {Grp. gram.:m.}
\end{itemize}
\begin{itemize}
\item {Proveniência:(Lat. \textunderscore retrospectus\textunderscore )}
\end{itemize}
Observação de tempos ou coisas passadas.
Lance de olhos para o passado.
\section{Rètrotracção}
\begin{itemize}
\item {Grp. gram.:f.}
\end{itemize}
Acto ou effeito de rètrotrahir.
\section{Rètrotrahir}
\begin{itemize}
\item {Grp. gram.:v. t.}
\end{itemize}
\begin{itemize}
\item {Proveniência:(De \textunderscore retro...\textunderscore  + \textunderscore trahir\textunderscore )}
\end{itemize}
O mesmo que \textunderscore retrahir\textunderscore .
Fazer remontar á origem.
Fazer recuar.
\section{Rètrotrair}
\begin{itemize}
\item {Grp. gram.:v. t.}
\end{itemize}
\begin{itemize}
\item {Proveniência:(De \textunderscore retro...\textunderscore  + \textunderscore trair\textunderscore )}
\end{itemize}
O mesmo que \textunderscore retrair\textunderscore .
Fazer remontar á origem.
Fazer recuar.
\section{Rètrovenda}
\begin{itemize}
\item {Grp. gram.:f.}
\end{itemize}
O mesmo que \textunderscore rètrovendição\textunderscore .
\section{Rètrovender}
\begin{itemize}
\item {Grp. gram.:v. t.}
\end{itemize}
\begin{itemize}
\item {Proveniência:(De \textunderscore retro...\textunderscore  + \textunderscore vender\textunderscore )}
\end{itemize}
Vender, com a faculdade de desfazer o mesmo contrato.
\section{Rètrovendição}
\begin{itemize}
\item {Grp. gram.:f.}
\end{itemize}
Acto de rètrovender.
\section{Rètroversão}
\begin{itemize}
\item {Grp. gram.:f.}
\end{itemize}
\begin{itemize}
\item {Utilização:Gram.}
\end{itemize}
\begin{itemize}
\item {Proveniência:(De \textunderscore retro...\textunderscore  + \textunderscore versão\textunderscore )}
\end{itemize}
Exercício escolar, em que, traduzido um trecho de língua estranha em língua vulgar, se passa do novo para a língua donde se traduziu, alterando-se-lhe, mais ou menos, a construcção que tinha no original.
\section{Rètroverso}
\begin{itemize}
\item {Proveniência:(Lat. \textunderscore retroversus\textunderscore )}
\end{itemize}
O mesmo que \textunderscore rètrovertido\textunderscore .
\section{Rètroverter}
\begin{itemize}
\item {Grp. gram.:v. t.}
\end{itemize}
\begin{itemize}
\item {Proveniência:(Lat. \textunderscore retrovertere\textunderscore )}
\end{itemize}
Fazer voltar para trás; inclinar para trás.
Rètrotrahir.
Fazer a rètroversão de. Cf. Latino, \textunderscore Humboldt\textunderscore , 412.
\section{Rètrovertido}
\begin{itemize}
\item {Grp. gram.:adj.}
\end{itemize}
Que se rètroverteu.
Inclinado para trás.
\section{Retrucar}
\begin{itemize}
\item {Grp. gram.:v. i.}
\end{itemize}
\begin{itemize}
\item {Utilização:T. do jogador}
\end{itemize}
\begin{itemize}
\item {Grp. gram.:V. t.}
\end{itemize}
\begin{itemize}
\item {Proveniência:(De \textunderscore re...\textunderscore  + \textunderscore trucar\textunderscore )}
\end{itemize}
Reenviada a quem nos trucou.
Replicar, redarguir; objectar.
\section{Retruque}
\begin{itemize}
\item {Grp. gram.:m.}
\end{itemize}
\begin{itemize}
\item {Proveniência:(De \textunderscore re...\textunderscore  + \textunderscore truque\textunderscore )}
\end{itemize}
Acto ou effeito de retrucar.
Volta de uma bóla de bilhar sôbre a outra que a impelliu.
\section{Retruso}
\begin{itemize}
\item {Grp. gram.:adj.}
\end{itemize}
\begin{itemize}
\item {Utilização:Des.}
\end{itemize}
\begin{itemize}
\item {Proveniência:(Lat. \textunderscore retrusus\textunderscore )}
\end{itemize}
Posto em lugar esconso; acantoado.
Obscuro. Cf. Filinto, II, 177.
\section{Retumbância}
\begin{itemize}
\item {Grp. gram.:f.}
\end{itemize}
Qualidade de retumbante.
\section{Retumbante}
\begin{itemize}
\item {Grp. gram.:adj.}
\end{itemize}
Que retumba.
\section{Retumbar}
\begin{itemize}
\item {Grp. gram.:v. i.}
\end{itemize}
\begin{itemize}
\item {Grp. gram.:V. t.}
\end{itemize}
\begin{itemize}
\item {Proveniência:(T. onom.)}
\end{itemize}
Echoar; estrondear; ribombar.
Repercutir; repetir com estrondo o som de.
\section{Retumbo}
\begin{itemize}
\item {Grp. gram.:m.}
\end{itemize}
Acto de retumbar.
\section{Retundir}
\begin{itemize}
\item {Grp. gram.:v. t.}
\end{itemize}
\begin{itemize}
\item {Proveniência:(Lat. \textunderscore retundere\textunderscore )}
\end{itemize}
Repellir; moderar.
Reter.
\section{Rétzia}
\begin{itemize}
\item {Grp. gram.:f.}
\end{itemize}
\begin{itemize}
\item {Proveniência:(De \textunderscore Retzius\textunderscore , n. p.)}
\end{itemize}
Gênero de arbustos do Cabo da Bôa-Esperança.
\section{Retziáceas}
\begin{itemize}
\item {Grp. gram.:f. pl.}
\end{itemize}
Família de plantas, que tem por typo a rétzia.
(Fem. pl. de \textunderscore retziáceo\textunderscore )
\section{Retziáceo}
\begin{itemize}
\item {Grp. gram.:adj.}
\end{itemize}
Relativo ou semelhante á rétzia.
\section{Réu}
\begin{itemize}
\item {Grp. gram.:m.}
\end{itemize}
\begin{itemize}
\item {Utilização:Fig.}
\end{itemize}
\begin{itemize}
\item {Grp. gram.:Adj.}
\end{itemize}
\begin{itemize}
\item {Proveniência:(Lat. \textunderscore reus\textunderscore )}
\end{itemize}
Indivíduo, contra quem se intenta processo judicial.
O criminoso.
O accusado.
Aquelle que é responsável por alguma culpa.
Culpado, criminoso; malévolo:«\textunderscore ...aos réus designios do conde\textunderscore ». Filinto, XX, 281.
\section{Réu}
\begin{itemize}
\item {Grp. gram.:m. Loc. adv.}
\end{itemize}
\begin{itemize}
\item {Utilização:Prov.}
\end{itemize}
\begin{itemize}
\item {Utilização:alg.}
\end{itemize}
\textunderscore A réu\textunderscore , o mesmo que \textunderscore a reio\textunderscore .
(Cp. \textunderscore reio\textunderscore )
\section{Reuchliniano}
\begin{itemize}
\item {Grp. gram.:adj.}
\end{itemize}
\begin{itemize}
\item {Proveniência:(De \textunderscore Reuchlin\textunderscore , n. p.)}
\end{itemize}
Diz-se do systema de pronunciar o grego, applicando ao grego antigo a pronúncia do grego moderno.
\section{Reuchlínico}
\begin{itemize}
\item {Grp. gram.:adj.}
\end{itemize}
O mesmo ou melhor que \textunderscore reuchliniano\textunderscore .
\section{Reunião}
\begin{itemize}
\item {fónica:re-u}
\end{itemize}
\begin{itemize}
\item {Grp. gram.:f.}
\end{itemize}
Acto ou effeito de reunir.
Sarau.
Agrupamento de pessôas em casa particular ou em clube para qualquer diversão.
Conciliação.
Fusão (de partidos)
\section{Reunidor}
\begin{itemize}
\item {fónica:re-u}
\end{itemize}
\begin{itemize}
\item {Grp. gram.:m.}
\end{itemize}
\begin{itemize}
\item {Proveniência:(De \textunderscore reunir\textunderscore )}
\end{itemize}
Um dos apparelhos em fábricas de fiação. Cf. \textunderscore Inquér. Industr.\textunderscore , p. II, l. II, 123.
\section{Reunir}
\begin{itemize}
\item {fónica:re-u}
\end{itemize}
\begin{itemize}
\item {Grp. gram.:v. t.}
\end{itemize}
\begin{itemize}
\item {Grp. gram.:V. i.}
\end{itemize}
\begin{itemize}
\item {Utilização:Neol.}
\end{itemize}
\begin{itemize}
\item {Proveniência:(De \textunderscore re...\textunderscore  + \textunderscore unir\textunderscore )}
\end{itemize}
Unir novamente.
Unir.
Agrupar.
Congraçar; harmonizar.
Ligar bem.
Coser.
Convocar: \textunderscore reunir uma assembleia\textunderscore .
Possuir conjuntamente: \textunderscore homem que reúne qualidades apreciáveis\textunderscore .
Comparecer.
Agrupar-se, congregar-se.
Constituir-se em assembleia.
\section{Réu-réu}
\begin{itemize}
\item {Grp. gram.:m.}
\end{itemize}
\begin{itemize}
\item {Utilização:Prov.}
\end{itemize}
Pedaço de cordel, com uma pedrinha, ou outro objecto pesado, numa das pontas, o qual os rapazes seguram pela outra extremidade, agitando-o em círculo e soltando-o pelo ar. Quando o agitam, cantarolam:«\textunderscore réu-réu, vai ao céu, traze o meu chapéu...\textunderscore »
\section{Réu-réu!}
\begin{itemize}
\item {Grp. gram.:interj.}
\end{itemize}
\begin{itemize}
\item {Utilização:Prov.}
\end{itemize}
\begin{itemize}
\item {Utilização:beir.}
\end{itemize}
Voz, com que o rapazio imita o cantochão dos offícios de defuntos:«\textunderscore réu-réu! senhor padre Mascaréu, quem morreu já 'stá no céu...\textunderscore »
\section{Revaccinação}
\begin{itemize}
\item {Grp. gram.:f.}
\end{itemize}
Acto ou effeito de revaccinar.
\section{Revaccinar}
\begin{itemize}
\item {Grp. gram.:v. t.}
\end{itemize}
\begin{itemize}
\item {Proveniência:(De \textunderscore re...\textunderscore  + \textunderscore vaccinar\textunderscore )}
\end{itemize}
Vaccinar de novo.
\section{Revacinação}
\begin{itemize}
\item {Grp. gram.:f.}
\end{itemize}
Acto ou efeito de revacinar.
\section{Revacinar}
\begin{itemize}
\item {Grp. gram.:v. t.}
\end{itemize}
\begin{itemize}
\item {Proveniência:(De \textunderscore re...\textunderscore  + \textunderscore vacinar\textunderscore )}
\end{itemize}
Vacinar de novo.
\section{Revalenta}
\begin{itemize}
\item {Grp. gram.:f.}
\end{itemize}
Alimento medicamentoso, composto de farinha de certos legumes ou cereaes e de sal marinho. Cf. Camillo, \textunderscore Noites de Insómn.\textunderscore , I, 15.
(Relaciona-se com o lat. \textunderscore revalescere\textunderscore . Cp. fr. \textunderscore revalescière\textunderscore , que tem o significado de \textunderscore revalenta\textunderscore )
\section{Revaleste}
\begin{itemize}
\item {Grp. gram.:m.}
\end{itemize}
\begin{itemize}
\item {Utilização:Prov.}
\end{itemize}
\begin{itemize}
\item {Utilização:minh.}
\end{itemize}
Grande multidão. (Colhido em Barcelos)
\section{Revalidação}
\begin{itemize}
\item {Grp. gram.:f.}
\end{itemize}
Acto ou effeito de revalidar.
\section{Revalidar}
\begin{itemize}
\item {Grp. gram.:v. t.}
\end{itemize}
\begin{itemize}
\item {Proveniência:(De \textunderscore re...\textunderscore  + \textunderscore validar\textunderscore )}
\end{itemize}
Validar novamente; dar mais fôrça a; confirmar.
\section{Revedor}
\begin{itemize}
\item {Grp. gram.:m.  e  adj.}
\end{itemize}
\begin{itemize}
\item {Proveniência:(De \textunderscore rever\textunderscore )}
\end{itemize}
O mesmo que \textunderscore revisor\textunderscore .
Funccionário dos tribunaes de segunda instância, incumbido de rever a organização dos processos e contas respectivas.
\section{Revedor}
\begin{itemize}
\item {Grp. gram.:m.}
\end{itemize}
\begin{itemize}
\item {Utilização:Bras. do N}
\end{itemize}
\begin{itemize}
\item {Proveniência:(De \textunderscore rever\textunderscore ^2)}
\end{itemize}
Pequeno poço.
Lugar, onde a água mana aos poucos.
\section{Revel}
\begin{itemize}
\item {Grp. gram.:m. ,  f.  e  adj.}
\end{itemize}
\begin{itemize}
\item {Utilização:Jur.}
\end{itemize}
\begin{itemize}
\item {Proveniência:(Do lat. \textunderscore rebellis\textunderscore )}
\end{itemize}
Pessôa rebelde, pessôa esquiva.
Indivíduo, que não cumpre a citação que se lhe fez comparecer em juizo.
\section{Revelação}
\begin{itemize}
\item {Grp. gram.:f.}
\end{itemize}
\begin{itemize}
\item {Proveniência:(Do lat. \textunderscore revelatio\textunderscore )}
\end{itemize}
Acto ou effeito de revelar.
Inspiração sobrenatural, com que Deus faz conhecer certas coisas.
Religião revelada.
\section{Revelador}
\begin{itemize}
\item {Grp. gram.:m.  e  adj.}
\end{itemize}
\begin{itemize}
\item {Utilização:Phot.}
\end{itemize}
\begin{itemize}
\item {Proveniência:(Do lat. \textunderscore revelator\textunderscore )}
\end{itemize}
O que revela.
Diz-se do banho, que faz apparecer a imagem nas matrizes photográphicas.
\section{Revelantismo}
\begin{itemize}
\item {Grp. gram.:m.}
\end{itemize}
\begin{itemize}
\item {Proveniência:(De \textunderscore revelar\textunderscore )}
\end{itemize}
Systema philosóphico, que procura na revelação christan a solução das questões psychológicas e moraes.
\section{Revelantista}
\begin{itemize}
\item {Grp. gram.:m.}
\end{itemize}
Sectário do revelantismo.
\section{Revelar}
\begin{itemize}
\item {Grp. gram.:v. t.}
\end{itemize}
\begin{itemize}
\item {Utilização:Ant.}
\end{itemize}
\begin{itemize}
\item {Proveniência:(Lat. \textunderscore revelare\textunderscore )}
\end{itemize}
Tirar o véu a; descobrir.
Manifestar: \textunderscore revelar bôas intenções\textunderscore .
Declarar.
Divulgar: \textunderscore revelar um segrêdo\textunderscore .
Denunciar: \textunderscore revelar um crime\textunderscore .
Fazer conhecer sobrenaturalmente.
Conhecer carnalmente (uma mulher).
\section{Revelar}
\begin{itemize}
\item {Grp. gram.:v. i.}
\end{itemize}
\begin{itemize}
\item {Utilização:Ant.}
\end{itemize}
Sêr revel.
O mesmo que [[rebellar-se|rebellar]]. Cf. Usque, XXXIII, v.^o.
\section{Reveler}
\begin{itemize}
\item {Grp. gram.:v. t.}
\end{itemize}
\begin{itemize}
\item {Utilização:Ant.}
\end{itemize}
O mesmo que \textunderscore revelir\textunderscore . Cf. M. de Azevedo, \textunderscore Correc. de Abusos\textunderscore , trat. I, c. I.
\section{Revelho}
\begin{itemize}
\item {Grp. gram.:m.  e  adj.}
\end{itemize}
\begin{itemize}
\item {Proveniência:(De \textunderscore re...\textunderscore  + \textunderscore velho\textunderscore )}
\end{itemize}
Muito velho; macróbio.
\section{Revelhusco}
\begin{itemize}
\item {Grp. gram.:adj.}
\end{itemize}
\begin{itemize}
\item {Utilização:Chul.}
\end{itemize}
\begin{itemize}
\item {Proveniência:(De \textunderscore revelho\textunderscore )}
\end{itemize}
Um tanto velho. Cf. \textunderscore Aulegrafia\textunderscore , 177; \textunderscore Eufrosina\textunderscore , 82 e 331.
\section{Revelia}
\begin{itemize}
\item {Grp. gram.:f.}
\end{itemize}
\begin{itemize}
\item {Grp. gram.:Loc. adv.}
\end{itemize}
Estado do que é revel.
\textunderscore Á revelia\textunderscore , sem conhecimento ou na ausência do indivíduo revel.
Ao acaso, á tôa: \textunderscore andar passeando á revelia\textunderscore .
\section{Revelim}
\begin{itemize}
\item {Grp. gram.:m.}
\end{itemize}
\begin{itemize}
\item {Proveniência:(Do it. \textunderscore rivellino\textunderscore )}
\end{itemize}
Construcção externa e saliente, de fórma angular, para defesa de ponte, cortina, etc., nas fortificações.
\section{Revelir}
\begin{itemize}
\item {Grp. gram.:v. t.}
\end{itemize}
\begin{itemize}
\item {Proveniência:(Lat. \textunderscore revellere\textunderscore )}
\end{itemize}
Fazer derivar de uma para outra parte (humores do organismo).
Transpirar; resumar.
\section{Reveller}
\begin{itemize}
\item {Grp. gram.:v. t.}
\end{itemize}
\begin{itemize}
\item {Utilização:Ant.}
\end{itemize}
O mesmo que \textunderscore revellir\textunderscore . Cf. M. de Azevedo, \textunderscore Correc. de Abusos\textunderscore , trat. I, c. I.
\section{Revellir}
\begin{itemize}
\item {Grp. gram.:v. t.}
\end{itemize}
\begin{itemize}
\item {Proveniência:(Lat. \textunderscore revellere\textunderscore )}
\end{itemize}
Fazer derivar de uma para outra parte (humores do organismo).
Transpirar; resumar.
\section{Revélo}
\begin{itemize}
\item {Grp. gram.:m.}
\end{itemize}
\begin{itemize}
\item {Utilização:Ant.}
\end{itemize}
\begin{itemize}
\item {Proveniência:(De \textunderscore revelar\textunderscore ^2)}
\end{itemize}
O mesmo que \textunderscore rebellião\textunderscore . Cf. Usque, (\textunderscore passim\textunderscore ).
\section{Revêlo}
\begin{itemize}
\item {Grp. gram.:m.}
\end{itemize}
\begin{itemize}
\item {Utilização:Prov.}
\end{itemize}
\begin{itemize}
\item {Utilização:trasm.}
\end{itemize}
Cabrito de mais de dois meses.
\section{Revenda}
\begin{itemize}
\item {Grp. gram.:f.}
\end{itemize}
Acto ou effeito de revender.
\section{Revendão}
\begin{itemize}
\item {Grp. gram.:m.  e  adj.}
\end{itemize}
O que revende; o mesmo que \textunderscore vendilhão\textunderscore .
\section{Revendedeira}
\begin{itemize}
\item {Grp. gram.:f.}
\end{itemize}
\begin{itemize}
\item {Utilização:Prov.}
\end{itemize}
\begin{itemize}
\item {Utilização:trasm.}
\end{itemize}
Mulhér, que compra para revender (legumes, frutas, etc.).
\section{Revendedona}
\begin{itemize}
\item {Grp. gram.:f.}
\end{itemize}
\begin{itemize}
\item {Utilização:Prov.}
\end{itemize}
\begin{itemize}
\item {Utilização:trasm.}
\end{itemize}
O mesmo que \textunderscore revendedeira\textunderscore .
\section{Revendedor}
\begin{itemize}
\item {Grp. gram.:m.  e  adj.}
\end{itemize}
O que revende.
\section{Revendedora}
\begin{itemize}
\item {Grp. gram.:f.}
\end{itemize}
O mesmo que \textunderscore revendedeira\textunderscore . Cf. Macedo, \textunderscore Motim\textunderscore , II, 121.
\section{Revender}
\begin{itemize}
\item {Grp. gram.:v. t.}
\end{itemize}
\begin{itemize}
\item {Proveniência:(Lat. \textunderscore revendere\textunderscore )}
\end{itemize}
Vender de novo.
Vender (o que se tinha comprado para negócio).
\section{Revendição}
\begin{itemize}
\item {Grp. gram.:f.}
\end{itemize}
O mesmo que \textunderscore revenda\textunderscore .
\section{Revendilhão}
\begin{itemize}
\item {Grp. gram.:m.  e  adj.}
\end{itemize}
\begin{itemize}
\item {Proveniência:(De \textunderscore re...\textunderscore  + \textunderscore vendilhão\textunderscore )}
\end{itemize}
O mesmo que \textunderscore revendão\textunderscore .
\section{Revendível}
\begin{itemize}
\item {Grp. gram.:adj.}
\end{itemize}
Que se póde revender.
\section{Revenerar}
\begin{itemize}
\item {Grp. gram.:v. t.}
\end{itemize}
\begin{itemize}
\item {Proveniência:(Lat. \textunderscore revenerari\textunderscore )}
\end{itemize}
Venerar muito; reverenciar.
\section{Revens}
\begin{itemize}
\item {Grp. gram.:m. pl.}
\end{itemize}
\begin{itemize}
\item {Utilização:Ant.}
\end{itemize}
O mesmo que [[refens|refém]]:«\textunderscore o Çamori deu em revens dois fidalgos...\textunderscore »\textunderscore Ethiópia Or.\textunderscore , II, 340.
\section{Rever}
\begin{itemize}
\item {Grp. gram.:v. t.}
\end{itemize}
\begin{itemize}
\item {Grp. gram.:V. p.}
\end{itemize}
\begin{itemize}
\item {Proveniência:(Do lat. \textunderscore revidere\textunderscore )}
\end{itemize}
Vêr novamente.
Vêr com attenção.
Examinar com cuidado: \textunderscore rever provas typográphicas\textunderscore .
Vêr-se novamente.
Comprazer-se: \textunderscore revia-se no filho\textunderscore .
Mirar-se, espelhar-se: \textunderscore rever-se nas águas\textunderscore .
\section{Rever}
\begin{itemize}
\item {Grp. gram.:V. t.}
\end{itemize}
\begin{itemize}
\item {Grp. gram.:V. i.}
\end{itemize}
\begin{itemize}
\item {Utilização:Fig.}
\end{itemize}
Fazer resumar.
Resumar; transudar.
Manifestar-se, tornar-se público.
(Cp. \textunderscore revir\textunderscore ^2)
\section{Revera}
\begin{itemize}
\item {Grp. gram.:adv.}
\end{itemize}
\begin{itemize}
\item {Utilização:Prov.}
\end{itemize}
\begin{itemize}
\item {Utilização:Ant.}
\end{itemize}
\begin{itemize}
\item {Proveniência:(Lat. \textunderscore revera\textunderscore )}
\end{itemize}
Na verdade, realmente:«\textunderscore ...mas revera assim passa.\textunderscore »Arráiz. Cf. Castilho, \textunderscore Metam.\textunderscore , (pról.); \textunderscore Peregrinação\textunderscore .
\section{Reverberação}
\begin{itemize}
\item {Grp. gram.:f.}
\end{itemize}
\begin{itemize}
\item {Proveniência:(Lat. \textunderscore reverberato\textunderscore )}
\end{itemize}
Acto ou effeito de reverberar.
\section{Reverberante}
\begin{itemize}
\item {Grp. gram.:adj.}
\end{itemize}
\begin{itemize}
\item {Proveniência:(Lat. \textunderscore reverberans\textunderscore )}
\end{itemize}
Que reverbera.
\section{Reverberar}
\begin{itemize}
\item {Grp. gram.:v. t.}
\end{itemize}
\begin{itemize}
\item {Grp. gram.:V. i.}
\end{itemize}
\begin{itemize}
\item {Proveniência:(Lat. \textunderscore reverberare\textunderscore )}
\end{itemize}
Reflectir (luz ou calor).
Brilhar, reflectindo-se; resplandecer.
\section{Reverberatório}
\begin{itemize}
\item {Grp. gram.:adj.}
\end{itemize}
O mesmo que \textunderscore reverberante\textunderscore .
\section{Reverbério}
\begin{itemize}
\item {Grp. gram.:m.}
\end{itemize}
\begin{itemize}
\item {Utilização:Pop.}
\end{itemize}
\begin{itemize}
\item {Proveniência:(De \textunderscore re...\textunderscore  + \textunderscore verberar\textunderscore )}
\end{itemize}
Reprehensão severa; descompostura.
\section{Revérbero}
\begin{itemize}
\item {Grp. gram.:m.}
\end{itemize}
Acto ou effeito de reverberar.
Reflexo luminoso.
Reflexo.
Lâmina, que torna a luz mais intensa, fazendo-a concentrar em certo espaço.
Resplendor.
Parte do forno, que faz reflectir o calor, que irradia de um foco, para a substância que se aquece.
(Por \textunderscore reverbro\textunderscore , de \textunderscore reverbrar\textunderscore , por \textunderscore reverberar\textunderscore )
\section{Reverdade}
\begin{itemize}
\item {Grp. gram.:f.}
\end{itemize}
O mesmo que \textunderscore verdade\textunderscore , reforçadamente:«\textunderscore é verdade e reverdade.\textunderscore »Castilho, \textunderscore Camões\textunderscore . Cf. \textunderscore Idem\textunderscore , \textunderscore Méd. á Fôrça\textunderscore , 52.
\section{Reverdecer}
\begin{itemize}
\item {Grp. gram.:v. t.}
\end{itemize}
\begin{itemize}
\item {Utilização:Fig.}
\end{itemize}
\begin{itemize}
\item {Grp. gram.:V. i.}
\end{itemize}
\begin{itemize}
\item {Utilização:Fig.}
\end{itemize}
\begin{itemize}
\item {Proveniência:(De \textunderscore re...\textunderscore  + \textunderscore verde\textunderscore )}
\end{itemize}
Tornar verde; cobrir de verdura: \textunderscore a primavera reverdece os campos\textunderscore .
Avigorar, fortificar.
Tornar novo.
Trazer á memória.
Tornar-se verde.
Cobrir-se de verdura.
Avigorar-se; desenvolver-se; remoçar.
\section{Reverdejante}
\begin{itemize}
\item {Grp. gram.:adj.}
\end{itemize}
Que reverdeja.
\section{Reverdejar}
\begin{itemize}
\item {Grp. gram.:v. i.}
\end{itemize}
\begin{itemize}
\item {Proveniência:(De \textunderscore re...\textunderscore  + \textunderscore verdejar\textunderscore )}
\end{itemize}
Mostrar-se muito verde; verdejar muito. Cf. Camillo, \textunderscore Vinho do Pôrto\textunderscore , 8.
\section{Reverença}
\begin{itemize}
\item {Grp. gram.:f.}
\end{itemize}
\begin{itemize}
\item {Utilização:ant.}
\end{itemize}
\begin{itemize}
\item {Utilização:Pop.}
\end{itemize}
O mesmo que \textunderscore reverência\textunderscore .
\section{Reverência}
\begin{itemize}
\item {Grp. gram.:f.}
\end{itemize}
\begin{itemize}
\item {Utilização:Fig.}
\end{itemize}
\begin{itemize}
\item {Proveniência:(Lat. \textunderscore reverentia\textunderscore )}
\end{itemize}
Acatamento ás coisas sagradas ou ao que é respeitável.
Veneração.
Acatamento; respeito.
Antigo tratamento, dado aos frades das Ordens mendicantes.
\section{Reverenciador}
\begin{itemize}
\item {Grp. gram.:m.  e  adj.}
\end{itemize}
O que reverencía.
\section{Reverencial}
\begin{itemize}
\item {Grp. gram.:adj.}
\end{itemize}
Relativo a reverência.
\section{Reverenciar}
\begin{itemize}
\item {Grp. gram.:v. t.}
\end{itemize}
\begin{itemize}
\item {Proveniência:(De \textunderscore reverência\textunderscore )}
\end{itemize}
Tratar com reverência; fazer reverência a.
Honrar; venerar; adorar.
\section{Reverenciosamente}
\begin{itemize}
\item {Grp. gram.:adv.}
\end{itemize}
De modo reverencioso; com reverência; com acatamento.
\section{Reverencioso}
\begin{itemize}
\item {Grp. gram.:adj.}
\end{itemize}
\begin{itemize}
\item {Proveniência:(De \textunderscore reverência\textunderscore )}
\end{itemize}
Que reverencía; que manifesta reverência.
Ceremonioso.
\section{Reverendaço}
\begin{itemize}
\item {Grp. gram.:m.}
\end{itemize}
\begin{itemize}
\item {Utilização:Pop.}
\end{itemize}
\begin{itemize}
\item {Proveniência:(De \textunderscore reverendo\textunderscore )}
\end{itemize}
Padre gordo.
\section{Reverendas}
\begin{itemize}
\item {Grp. gram.:f. pl.}
\end{itemize}
\begin{itemize}
\item {Proveniência:(De \textunderscore reverendo\textunderscore )}
\end{itemize}
Documento, em que um Bispo permitte a um seu diocesano ordenar-se em outra diocese:«\textunderscore ali, com reverendas falsas, fez-se clérigo.\textunderscore »Camillo, \textunderscore Caveira\textunderscore , 46.
\section{Reverendíssima}
\begin{itemize}
\item {Grp. gram.:f.}
\end{itemize}
\begin{itemize}
\item {Proveniência:(De \textunderscore reverendíssimo\textunderscore )}
\end{itemize}
Tratamento, que se dá aos ecclesiásticos: \textunderscore perdôe Vossa Reverendíssima...\textunderscore 
\section{Reverendíssimo}
\begin{itemize}
\item {Grp. gram.:m.  e  adj.}
\end{itemize}
\begin{itemize}
\item {Grp. gram.:Adj.}
\end{itemize}
\begin{itemize}
\item {Utilização:Fam.}
\end{itemize}
\begin{itemize}
\item {Proveniência:(De \textunderscore reverendo\textunderscore )}
\end{itemize}
Título, que se dá aos dignitários ecclesiásticos e ainda aos padres em geral: \textunderscore o reverendíssimo Abbade... Mas oiça cá, reverendíssimo\textunderscore .
Extraordinário, pyramidal: \textunderscore uma reverendíssima asneira\textunderscore .
\section{Reverendo}
\begin{itemize}
\item {Grp. gram.:adj.}
\end{itemize}
\begin{itemize}
\item {Grp. gram.:M.  e  adj.}
\end{itemize}
\begin{itemize}
\item {Proveniência:(Lat. \textunderscore reverendus\textunderscore )}
\end{itemize}
Digno de reverência.
Título, que se dá aos dignitários ecclesiásticos e aos padres em geral: \textunderscore iam comigo dois reverendos...\textunderscore 
\section{Reverente}
\begin{itemize}
\item {Grp. gram.:adj.}
\end{itemize}
\begin{itemize}
\item {Proveniência:(Lat. \textunderscore reverens\textunderscore )}
\end{itemize}
Que reverencía; reverencioso.
\section{Reverentemente}
\begin{itemize}
\item {Grp. gram.:adv.}
\end{itemize}
De modo reverente; respeitosamente.
\section{Reveria}
\begin{itemize}
\item {Grp. gram.:f.}
\end{itemize}
\begin{itemize}
\item {Utilização:ant.}
\end{itemize}
\begin{itemize}
\item {Utilização:Pop.}
\end{itemize}
O mesmo que \textunderscore revelia\textunderscore . Cf. R. Lobo, \textunderscore Côrte na Ald.\textunderscore , I, 117.
\section{Reveria}
\begin{itemize}
\item {Grp. gram.:f.}
\end{itemize}
\begin{itemize}
\item {Utilização:T. de Turquel}
\end{itemize}
Referência, homenagem: \textunderscore a banda tocou o hymno nacional á reveria de Sua Ex.^a\textunderscore 
\section{Reverificação}
\begin{itemize}
\item {Grp. gram.:f.}
\end{itemize}
Acto ou effeito de reverificar.
\section{Reverificador}
\begin{itemize}
\item {Grp. gram.:adj.}
\end{itemize}
\begin{itemize}
\item {Grp. gram.:M.}
\end{itemize}
\begin{itemize}
\item {Proveniência:(De \textunderscore reverificar\textunderscore )}
\end{itemize}
Que reverifica.
Empregado aduaneiro, que reverifica ou contraprova o serviço dos verificadores.
\section{Reverificar}
\begin{itemize}
\item {Grp. gram.:v. t.}
\end{itemize}
\begin{itemize}
\item {Proveniência:(De \textunderscore re...\textunderscore  + \textunderscore verificar\textunderscore )}
\end{itemize}
Verificar novamente; conferir, cotejar.
\section{Revermelhar}
\begin{itemize}
\item {Grp. gram.:v. i.}
\end{itemize}
\begin{itemize}
\item {Proveniência:(De \textunderscore re...\textunderscore  + \textunderscore vermelho\textunderscore )}
\end{itemize}
Vermelhar com intensidade:«\textunderscore os tições da braseira revermelhavam...\textunderscore »Eça, na \textunderscore Rev. Occid.\textunderscore , I, 445.
\section{Revernizar}
\begin{itemize}
\item {Grp. gram.:v. t.}
\end{itemize}
\begin{itemize}
\item {Proveniência:(De \textunderscore re...\textunderscore  + \textunderscore verniz\textunderscore )}
\end{itemize}
Envernizar de novo.
Cobrir de nova camada de verniz.
\section{Reversado}
\begin{itemize}
\item {Grp. gram.:adj.}
\end{itemize}
\begin{itemize}
\item {Utilização:Náut.}
\end{itemize}
\begin{itemize}
\item {Utilização:Ant.}
\end{itemize}
Liame, que vai por cima do cabo delgado até ao cadaste da embarcação. Cf. Fern. Oliveira, \textunderscore Livro da Fábr. das Naus\textunderscore .
\section{Reversal}
\begin{itemize}
\item {Grp. gram.:adj.}
\end{itemize}
\begin{itemize}
\item {Proveniência:(De \textunderscore reverso\textunderscore )}
\end{itemize}
Que assegura ou garante promessa anterior.
\section{Reversão}
\begin{itemize}
\item {Grp. gram.:f.}
\end{itemize}
\begin{itemize}
\item {Proveniência:(Do lat. \textunderscore reversio\textunderscore )}
\end{itemize}
Acto ou effeito de reverter.
Regresso ao primeiro estado.
Devolução.
\section{Reversar}
\begin{itemize}
\item {Grp. gram.:v. t.  e  i.}
\end{itemize}
\begin{itemize}
\item {Proveniência:(Lat. \textunderscore reversare\textunderscore )}
\end{itemize}
O mesmo que \textunderscore revessar\textunderscore .
\section{Reversibilidade}
\begin{itemize}
\item {Grp. gram.:f.}
\end{itemize}
Qualidade do que é reversível.
\section{Reversível}
\begin{itemize}
\item {Grp. gram.:adj.}
\end{itemize}
Que volta ou deve voltar ao primeiro estado.
Que volta de novo.
Revirado.
\section{Reversivo}
\begin{itemize}
\item {Grp. gram.:adj.}
\end{itemize}
\begin{itemize}
\item {Proveniência:(Do lat. \textunderscore reversivus\textunderscore )}
\end{itemize}
Que volta ou deve voltar ao primeiro estado.
Que volta de novo.
Revirado.
\section{Reverso}
\begin{itemize}
\item {Grp. gram.:adj.}
\end{itemize}
\begin{itemize}
\item {Utilização:Fig.}
\end{itemize}
\begin{itemize}
\item {Utilização:Gram.}
\end{itemize}
\begin{itemize}
\item {Grp. gram.:M.}
\end{itemize}
\begin{itemize}
\item {Utilização:Ant.}
\end{itemize}
\begin{itemize}
\item {Proveniência:(Lat. \textunderscore reversus\textunderscore )}
\end{itemize}
Que tem má índole.
Convexo.
Revirado.
Diz-se das consoantes, que se proferem com o bôrdo anterior da ponta da língua na parte interna das gengivas dos incisivos superiores, como o \textunderscore r\textunderscore , o \textunderscore d\textunderscore  e o \textunderscore x\textunderscore . Cf. G. Viana, \textunderscore Pronúncia Normal\textunderscore .
Lado opposto ao principal.
Parte posterior: \textunderscore o reverso da medalha\textunderscore .
Aquillo que é contrário.
Aquelle que se presta a vícios torpes contra a natureza. Cf. S. R. Viterbo, \textunderscore Elucidário\textunderscore .
\section{Reverter}
\begin{itemize}
\item {Grp. gram.:v. i.}
\end{itemize}
\begin{itemize}
\item {Proveniência:(Lat. \textunderscore revertere\textunderscore )}
\end{itemize}
Regressar, retroceder.
Entrar de novo na posse de alguém: \textunderscore aquelle prédio reverteu para o casal, a que pertenceu\textunderscore .
Tornar-se, redundar: \textunderscore o estudo reverte em proveito\textunderscore .
\section{Revertível}
\begin{itemize}
\item {Grp. gram.:adj.}
\end{itemize}
Que se póde reverter; que póde reverter.
\section{Revés}
\begin{itemize}
\item {Grp. gram.:m.}
\end{itemize}
\begin{itemize}
\item {Utilização:Bras}
\end{itemize}
\begin{itemize}
\item {Grp. gram.:Loc. adv.}
\end{itemize}
\begin{itemize}
\item {Proveniência:(Do b. lat. \textunderscore reverse\textunderscore )}
\end{itemize}
O mesmo que \textunderscore reverso\textunderscore .
Pancada com as costas da mão.
Golpe oblíquo.
Accidente desfavorável.
Contrariedade, fatalidade, desgraça: \textunderscore soffrer reveses\textunderscore .
Acto de arremessar a péla, estando o jogador de costas voltadas para o frontão.
\textunderscore Ao revés\textunderscore , ás avessas; do lado opposto.
\section{Revesilho}
\begin{itemize}
\item {Grp. gram.:m.}
\end{itemize}
Trabalho na perna da meia, dando-se o ponto ás avessas e fazendo-se junto dêlle os mates para se estreitar a meia. Cf. Dom. Vieira, \textunderscore Thes. da Ling.\textunderscore , vb. \textunderscore revezilho\textunderscore .
(Alter. de \textunderscore reversilho\textunderscore , de \textunderscore reverso\textunderscore )
\section{Reveso}
\begin{itemize}
\item {Grp. gram.:adj.}
\end{itemize}
O mesmo que \textunderscore reverso\textunderscore .
\section{Revessa}
\begin{itemize}
\item {Grp. gram.:f.}
\end{itemize}
\begin{itemize}
\item {Proveniência:(De \textunderscore revêsso\textunderscore )}
\end{itemize}
Porção de águas correntes, em direcção contrária á do rio que corre próximo.
Corrente marítima, que se volta em direcção differente da que seguia.
Intersecção de duas vertentes do telhado, formando ângulo reintrante.
\section{Revessado}
\begin{itemize}
\item {Grp. gram.:adj.}
\end{itemize}
\begin{itemize}
\item {Proveniência:(De \textunderscore revessar\textunderscore )}
\end{itemize}
Reverso, voltado.
\section{Revessar}
\begin{itemize}
\item {Grp. gram.:v. t.  e  i.}
\end{itemize}
O mesmo que \textunderscore arrevessar\textunderscore :«\textunderscore debruçado na janela..., revessava ao caminho público golfos aziumados de vinhaça\textunderscore ». Camillo, \textunderscore Brasileira\textunderscore , 78.
\section{Revessilho}
\begin{itemize}
\item {Grp. gram.:m.}
\end{itemize}
Trabalho na perna da meia, dando-se o ponto ás avessas e fazendo-se junto dêlle os mates para se estreitar a meia. Cf. Dom. Vieira, \textunderscore Thes. da Ling.\textunderscore , vb. \textunderscore revezilho\textunderscore .
(Alter. de \textunderscore reversilho\textunderscore , de \textunderscore reverso\textunderscore )
\section{Revesso}
\begin{itemize}
\item {Grp. gram.:adj.}
\end{itemize}
O mesmo que \textunderscore reverso\textunderscore .
\section{Revêsso}
\begin{itemize}
\item {Grp. gram.:adj.}
\end{itemize}
\begin{itemize}
\item {Utilização:Fig.}
\end{itemize}
\begin{itemize}
\item {Proveniência:(Do lat. \textunderscore reversus\textunderscore )}
\end{itemize}
O mesmo que \textunderscore reverso\textunderscore .
Torcido; arrevesado: \textunderscore estilo revêsso\textunderscore .
Que dificilmente se presta para certos trabalhos: \textunderscore madeira revêssa\textunderscore .
\section{Revestidura}
\begin{itemize}
\item {Grp. gram.:f.}
\end{itemize}
Acto ou effeito de revestir.
\section{Revestimento}
\begin{itemize}
\item {Grp. gram.:m.}
\end{itemize}
Acto ou effeito de revestir.
\section{Revestir}
\begin{itemize}
\item {Grp. gram.:v. t.}
\end{itemize}
\begin{itemize}
\item {Utilização:Fig.}
\end{itemize}
\begin{itemize}
\item {Proveniência:(Lat. \textunderscore revestire\textunderscore )}
\end{itemize}
Vestir novamente.
Vestir (um hábito sôbre outro).
Vestir.
Cobrir.
Representar em si (attributos ou caracteres de outrem).
Colorir.
Fortificar, cobrindo.
Solidificar.
Enfeitar.
\section{Revezadamente}
\begin{itemize}
\item {Grp. gram.:adv.}
\end{itemize}
De modo revezado; alternadamente; a revêzes.
\section{Revezado}
\begin{itemize}
\item {Grp. gram.:adj.}
\end{itemize}
O mesmo que [[repetido|repetir]]. Cf. Filinto, VII, 59.
\section{Revezador}
\begin{itemize}
\item {Grp. gram.:m.  e  adj.}
\end{itemize}
O que reveza.
\section{Revezamento}
\begin{itemize}
\item {Grp. gram.:m.}
\end{itemize}
Acto ou effeito de revezar.
\section{Revezar}
\begin{itemize}
\item {Grp. gram.:v. t.}
\end{itemize}
\begin{itemize}
\item {Grp. gram.:V. i.}
\end{itemize}
\begin{itemize}
\item {Proveniência:(De \textunderscore re...\textunderscore  + \textunderscore vez\textunderscore )}
\end{itemize}
Substituír alternadamente.
Alternar-se; substituír-se alternadamente.
\section{Revezeiro}
\begin{itemize}
\item {Grp. gram.:m.}
\end{itemize}
\begin{itemize}
\item {Utilização:Pesc.}
\end{itemize}
\begin{itemize}
\item {Proveniência:(De \textunderscore revezar\textunderscore )}
\end{itemize}
Aquelle que, de pé, trabalha ao punho do remo.
\section{Revêzes}
\begin{itemize}
\item {Grp. gram.:f. pl.}
\end{itemize}
\begin{itemize}
\item {Proveniência:(De \textunderscore re...\textunderscore  + \textunderscore vez\textunderscore )}
\end{itemize}
Us. na loc. \textunderscore a revêzes\textunderscore , ou \textunderscore ás revêzes\textunderscore , uma vez ou outra, ás vezes, de onde em onde; alternativamente:«\textunderscore ...bebe a revêzes Brito\textunderscore ». Filinto VIII, 99 e 100.«\textunderscore ...dessas quintilhas, em que a revêzes a graça está na desinência de uma palavra, no resalto de um adjectivo\textunderscore ». Camillo, \textunderscore Curso de Liter. Port.\textunderscore , 218.
\section{Revêzo}
\begin{itemize}
\item {Grp. gram.:m.}
\end{itemize}
\begin{itemize}
\item {Proveniência:(De \textunderscore revezar\textunderscore )}
\end{itemize}
Pastagem, para onde se muda o gado, em-quanto se cria o pasto no terreno onde êsse gado andava.
\section{Revibrar}
\begin{itemize}
\item {Grp. gram.:v. t.}
\end{itemize}
\begin{itemize}
\item {Grp. gram.:V. i.}
\end{itemize}
\begin{itemize}
\item {Proveniência:(Lat. \textunderscore revibrare\textunderscore )}
\end{itemize}
Fazer vibrar muitas vezes. Cf. Garrett, \textunderscore Romanceiro\textunderscore , I, 80.
Vibrar de novo; vibrar muitas vezes. Cf. Júl. Castilho, \textunderscore Manuelinas\textunderscore , 168.
\section{Reviçamento}
\begin{itemize}
\item {Grp. gram.:m.}
\end{itemize}
Acto ou effeito de reviçar.
\section{Reviçar}
\begin{itemize}
\item {Grp. gram.:v. i.}
\end{itemize}
\begin{itemize}
\item {Proveniência:(De \textunderscore re...\textunderscore  + \textunderscore viçar\textunderscore )}
\end{itemize}
Viçar novamente; remoçar. Cf. Camillo, \textunderscore O Bem e o Mal\textunderscore , 168.
\section{Reviço}
\begin{itemize}
\item {Grp. gram.:m.}
\end{itemize}
Acto ou effeito de reviçar.
Exuberância de vida.
\section{Revida}
\begin{itemize}
\item {Grp. gram.:adj.}
\end{itemize}
\begin{itemize}
\item {Utilização:Prov.}
\end{itemize}
\begin{itemize}
\item {Utilização:minh.}
\end{itemize}
\begin{itemize}
\item {Proveniência:(De \textunderscore rever\textunderscore ^1)}
\end{itemize}
Diz-se da mulhér vaidosa, que toda se revê.
\section{Revidar}
\textunderscore v. t.\textunderscore  (e der.)
O mesmo que \textunderscore reinvidar\textunderscore , etc. Cf. \textunderscore Eufrosina\textunderscore , 144.
\section{Revide}
\begin{itemize}
\item {Grp. gram.:m.}
\end{itemize}
\begin{itemize}
\item {Utilização:Prov.}
\end{itemize}
\begin{itemize}
\item {Utilização:alent.}
\end{itemize}
Acto de revidar, no jôgo de petisca.
\section{Revido}
\begin{itemize}
\item {Grp. gram.:m.}
\end{itemize}
\begin{itemize}
\item {Utilização:Prov.}
\end{itemize}
\begin{itemize}
\item {Utilização:alent.}
\end{itemize}
O mesmo que \textunderscore revide\textunderscore .
\section{Revigorar}
\begin{itemize}
\item {Grp. gram.:v. t.}
\end{itemize}
\begin{itemize}
\item {Proveniência:(De \textunderscore re...\textunderscore  + \textunderscore vigorar\textunderscore )}
\end{itemize}
Dar novo vigor a.
\section{Revigorizar}
\textunderscore v. t.\textunderscore  (e der.)
O mesmo que \textunderscore revigorar\textunderscore , etc. Cf. P. Chagas, \textunderscore Côrte de D. João V\textunderscore , 33.
\section{Revimento}
\begin{itemize}
\item {Grp. gram.:m.}
\end{itemize}
Acto ou effeito de rever^2.
\section{Revinda}
\begin{itemize}
\item {Grp. gram.:f.}
\end{itemize}
\begin{itemize}
\item {Proveniência:(De \textunderscore re...\textunderscore  + \textunderscore vinda\textunderscore )}
\end{itemize}
Acto de revir; regresso. Cf. Filinto, \textunderscore D. Man.\textunderscore , I, 129; II, 69.
\section{Revindicar}
\textunderscore v. t.\textunderscore  (e der.)
O mesmo que \textunderscore reivindicar\textunderscore , etc.
\section{Revindicta}
\begin{itemize}
\item {Grp. gram.:f.}
\end{itemize}
\begin{itemize}
\item {Proveniência:(De \textunderscore re...\textunderscore  + \textunderscore vindicta\textunderscore )}
\end{itemize}
Vingança de uma vingança.
Desafronta; desforra.
\section{Revindita}
\begin{itemize}
\item {Grp. gram.:f.}
\end{itemize}
\begin{itemize}
\item {Proveniência:(De \textunderscore re...\textunderscore  + \textunderscore vindicta\textunderscore )}
\end{itemize}
Vingança de uma vingança.
Desafronta; desforra.
\section{Revindo}
\begin{itemize}
\item {Grp. gram.:m.}
\end{itemize}
\begin{itemize}
\item {Utilização:Ant.}
\end{itemize}
\begin{itemize}
\item {Grp. gram.:Adj.}
\end{itemize}
\begin{itemize}
\item {Proveniência:(De \textunderscore revir\textunderscore ^1)}
\end{itemize}
Berço.
Meia volta.
Dizia-se da abóbada em meio arco. Cf. S. R. Viterbo, \textunderscore Elucidário\textunderscore .
\section{Revingar}
\begin{itemize}
\item {Grp. gram.:v. t.}
\end{itemize}
\begin{itemize}
\item {Proveniência:(De \textunderscore re...\textunderscore  + \textunderscore vingar\textunderscore )}
\end{itemize}
Vingar novamente; tirar vingança de (outra vingança ou injúria)
\section{Revir}
\begin{itemize}
\item {Grp. gram.:v. i.}
\end{itemize}
\begin{itemize}
\item {Proveniência:(Do lat. \textunderscore revenire\textunderscore )}
\end{itemize}
Vir de novo; regressar.
\section{Revir}
\begin{itemize}
\item {Grp. gram.:v. i.}
\end{itemize}
\begin{itemize}
\item {Utilização:Bras}
\end{itemize}
Resumar; transpirar.
O mesmo ou melhor que \textunderscore rever\textunderscore ^2: \textunderscore esta vasilha reve muito\textunderscore .
(Contr. de \textunderscore revellir\textunderscore )
\section{Revira}
\begin{itemize}
\item {Grp. gram.:m.}
\end{itemize}
\begin{itemize}
\item {Utilização:Bras. do N}
\end{itemize}
\begin{itemize}
\item {Proveniência:(De \textunderscore revirar\textunderscore )}
\end{itemize}
Bailado de negros e de gente plebeia.
\section{Revirado}
\begin{itemize}
\item {Grp. gram.:m.}
\end{itemize}
\begin{itemize}
\item {Utilização:Bras}
\end{itemize}
Iguaria, o mesmo que \textunderscore pamonan\textunderscore .
\section{Reviramento}
\begin{itemize}
\item {Grp. gram.:m.}
\end{itemize}
Acto ou effeito de revirar.
\section{Revirão}
\begin{itemize}
\item {Grp. gram.:m.}
\end{itemize}
\begin{itemize}
\item {Utilização:Ant.}
\end{itemize}
\begin{itemize}
\item {Utilização:Pop.}
\end{itemize}
\begin{itemize}
\item {Proveniência:(De \textunderscore re...\textunderscore  + \textunderscore vira\textunderscore )}
\end{itemize}
Vira traseira do calçado.
Bofetada com as costas da mão.
\section{Revirar}
\begin{itemize}
\item {Grp. gram.:v. t.}
\end{itemize}
\begin{itemize}
\item {Grp. gram.:V. i.}
\end{itemize}
\begin{itemize}
\item {Grp. gram.:V. p.}
\end{itemize}
Tornar a virar.
Virar do avêsso.
Desviar de um rumo.
\textunderscore Revirar o caminho\textunderscore , arrepiar carreira, mudar de rumo.
\textunderscore Revirar o dente\textunderscore , refilar, recalcitrar.
\textunderscore Revirar os olhos\textunderscore , movê-los circularmente nas órbitas.
Virar-se de outro lado.
Recalcitrar, repontar.
Voltar-se.
Investir.
\section{Revira-volta}
\begin{itemize}
\item {Grp. gram.:f.}
\end{itemize}
Acto ou effeito de voltar em sentido opposto ao anterior.
Giro sôbre si mesmo; piruêta.
\section{Revirete}
\begin{itemize}
\item {fónica:virê}
\end{itemize}
\begin{itemize}
\item {Grp. gram.:m.}
\end{itemize}
\begin{itemize}
\item {Utilização:Pop.}
\end{itemize}
\begin{itemize}
\item {Proveniência:(De \textunderscore revirar\textunderscore )}
\end{itemize}
Dito argucioso, picante; repostada.
\section{Revisão}
\begin{itemize}
\item {Grp. gram.:f.}
\end{itemize}
\begin{itemize}
\item {Proveniência:(Lat. \textunderscore revisio\textunderscore )}
\end{itemize}
Acto ou effeito de rever^1.
Funcções de revisor.
Novo exame.
Nova leitura.
Anályse de uma lei, decreto ou processo, no intuito de rectificação, refórma ou annullação.
Exame e emenda de provas typográphicas.
Lugar ou compartimento privativo do revisor, em estabelecimentos typográphicos.
\section{Revisar}
\begin{itemize}
\item {Grp. gram.:v. t.}
\end{itemize}
\begin{itemize}
\item {Proveniência:(De \textunderscore re...\textunderscore  + \textunderscore visar\textunderscore )}
\end{itemize}
Visar novamente.
\section{Revisceração}
\begin{itemize}
\item {Grp. gram.:f.}
\end{itemize}
Formação de vísceras novas; nova formação de carne.
(B. lat. \textunderscore revisceratio\textunderscore )
\section{Revisionismo}
\begin{itemize}
\item {Grp. gram.:m.}
\end{itemize}
\begin{itemize}
\item {Utilização:Bras}
\end{itemize}
Partido ou systema dos revisionistas.
\section{Revisionista}
\begin{itemize}
\item {Grp. gram.:m.  e  adj.}
\end{itemize}
Partidário da revisão da Constituição política do Estado.
\section{Revisitação}
\begin{itemize}
\item {Grp. gram.:f.}
\end{itemize}
Acto ou effeito de revisitar.
\section{Revisitar}
\begin{itemize}
\item {Grp. gram.:v. t.}
\end{itemize}
\begin{itemize}
\item {Proveniência:(Lat. \textunderscore revisitare\textunderscore )}
\end{itemize}
Visitar outra vez; visitar muitas vezes.
\section{Revisível}
\begin{itemize}
\item {Grp. gram.:adj.}
\end{itemize}
Que se póde ou se deve rever.
\section{Revisor}
\begin{itemize}
\item {Grp. gram.:adj.}
\end{itemize}
\begin{itemize}
\item {Grp. gram.:M.}
\end{itemize}
\begin{itemize}
\item {Proveniência:(Do lat. \textunderscore revisus\textunderscore )}
\end{itemize}
Que revê.
Aquelle que examina provas typográphicas para as corrigir ou limpar de erros.
Indivíduo, encarregado de rever e conferir os bilhetes de passagem nos vehículos, caminhos de ferro, etc.
\section{Revisório}
\begin{itemize}
\item {Grp. gram.:adj.}
\end{itemize}
\begin{itemize}
\item {Proveniência:(Do lat. \textunderscore revisus\textunderscore )}
\end{itemize}
Relativo a revisão.
\section{Revista}
\begin{itemize}
\item {Grp. gram.:f.}
\end{itemize}
Acto ou effeito de revistar.
Peça theatral, em que se reproduzem comicamente os successos principaes do anno findo ou de uma época recente.
Exame de tropas em formatura.
\section{Revistar}
\begin{itemize}
\item {Grp. gram.:v. t.}
\end{itemize}
\begin{itemize}
\item {Proveniência:(De \textunderscore re...\textunderscore  + \textunderscore vista\textunderscore )}
\end{itemize}
Rever; examinar.
Passar busca a; dar varejo a: \textunderscore revistar uma casa\textunderscore .
\section{Revisteiro}
\begin{itemize}
\item {Grp. gram.:m.}
\end{itemize}
Aquelle que escreve revistas para os theatros.
\section{Revitalizar}
\begin{itemize}
\item {Grp. gram.:v. t.}
\end{itemize}
\begin{itemize}
\item {Utilização:Neol.}
\end{itemize}
\begin{itemize}
\item {Proveniência:(De \textunderscore re...\textunderscore  + \textunderscore vitalizar\textunderscore )}
\end{itemize}
Tornar a insuflar vida em; vitalizar novamente.
\section{Revivecer}
\begin{itemize}
\item {Grp. gram.:v. i.}
\end{itemize}
O mesmo que \textunderscore revivescer\textunderscore .
\section{Revivência}
\begin{itemize}
\item {Grp. gram.:f.}
\end{itemize}
Qualidade de revivente. Cf. Camillo, \textunderscore Brasileira\textunderscore , 382.
\section{Revivente}
\begin{itemize}
\item {Grp. gram.:adj.}
\end{itemize}
Que revive. Cf. Garrett, \textunderscore Romanceiro\textunderscore , II, XXVII.
\section{Reviver}
\begin{itemize}
\item {Grp. gram.:v. i.}
\end{itemize}
\begin{itemize}
\item {Grp. gram.:V. t.}
\end{itemize}
\begin{itemize}
\item {Utilização:Fig.}
\end{itemize}
\begin{itemize}
\item {Proveniência:(Lat. \textunderscore revivere\textunderscore )}
\end{itemize}
Voltar á vida.
Adquirir vida nova.
Revigorar-se.
Renovar-se.
Tornar a manifestar-se: \textunderscore reviveu a epidemia\textunderscore .
Trazer á lembrança.
\section{Revivescência}
\begin{itemize}
\item {Grp. gram.:f.}
\end{itemize}
O mesmo que \textunderscore revivescimento\textunderscore .
\section{Revivescer}
\begin{itemize}
\item {Grp. gram.:v. t.  e  i.}
\end{itemize}
\begin{itemize}
\item {Proveniência:(Lat. \textunderscore revivescere\textunderscore )}
\end{itemize}
O mesmo que \textunderscore reviver\textunderscore .
\section{Revivescimento}
\begin{itemize}
\item {Grp. gram.:m.}
\end{itemize}
Acto ou effeito de revivescer.
\section{Revivificação}
\begin{itemize}
\item {Grp. gram.:f.}
\end{itemize}
Acto ou effeito de revivificar.
\section{Revivificar}
\begin{itemize}
\item {Grp. gram.:v. t.}
\end{itemize}
\begin{itemize}
\item {Proveniência:(Do lat. \textunderscore re...\textunderscore  + \textunderscore vivificare\textunderscore )}
\end{itemize}
Tornar a vivificar; dar vida nova a.
\section{Reviviscência}
\begin{itemize}
\item {Grp. gram.:f.}
\end{itemize}
\begin{itemize}
\item {Proveniência:(Lat. \textunderscore reviviscentia\textunderscore )}
\end{itemize}
Acto ou effeito de reviviscer.
Revivificação.
\section{Reviviscente}
\begin{itemize}
\item {Grp. gram.:adj.}
\end{itemize}
\begin{itemize}
\item {Proveniência:(Lat. \textunderscore reviviscens\textunderscore )}
\end{itemize}
Que revivesce.
\section{Reviviscer}
\begin{itemize}
\item {Grp. gram.:v. t.  e  i.}
\end{itemize}
\begin{itemize}
\item {Proveniência:(Lat. \textunderscore reviviscere\textunderscore )}
\end{itemize}
O mesmo que \textunderscore revivescer\textunderscore .
\section{Reviviscível}
\begin{itemize}
\item {Grp. gram.:adj.}
\end{itemize}
Que póde reviviscer.
\section{Revivo}
\begin{itemize}
\item {Grp. gram.:adj.}
\end{itemize}
\begin{itemize}
\item {Proveniência:(De \textunderscore re...\textunderscore  + \textunderscore vivo\textunderscore )}
\end{itemize}
Que revive; que voltou á vida.
Que tem muita vida. Cf. Castilho, \textunderscore Fastos\textunderscore , III, 177.
\section{Revoada}
\begin{itemize}
\item {Grp. gram.:f.}
\end{itemize}
\begin{itemize}
\item {Utilização:Fig.}
\end{itemize}
Acto ou effeito de revoar.
Bando de aves, que revoam.
Ensejo, opportunidade.
\section{Revoar}
\begin{itemize}
\item {Grp. gram.:v. i.}
\end{itemize}
\begin{itemize}
\item {Utilização:Fig.}
\end{itemize}
\begin{itemize}
\item {Proveniência:(Do lat. \textunderscore revolare\textunderscore )}
\end{itemize}
Voar novamente.
Voar (a ave) para o ponto donde partira.
Voejar, esvoaçar, volitar.
Voar alto, pairar.
\section{Revocação}
\begin{itemize}
\item {Grp. gram.:f.}
\end{itemize}
\begin{itemize}
\item {Proveniência:(Do lat. \textunderscore revocatio\textunderscore )}
\end{itemize}
Acto ou effeito de revocar.
\section{Revocar}
\begin{itemize}
\item {Grp. gram.:v. t.}
\end{itemize}
\begin{itemize}
\item {Proveniência:(Lat. \textunderscore revocare\textunderscore )}
\end{itemize}
Chamar para trás.
Mandar voltar.
Chamar de novo.
Evocar.
Referir ao tempo presente (coisas passadas)
Annullar; revogar.
Restituir.
\section{Revocatória}
\begin{itemize}
\item {Grp. gram.:f.}
\end{itemize}
\begin{itemize}
\item {Proveniência:(Lat. \textunderscore revocatoria\textunderscore )}
\end{itemize}
Diploma, com que um Govêrno manda retirar o seu representante, do lugar que occupa, junto de outro Govêrno.
\section{Revocatório}
\begin{itemize}
\item {Grp. gram.:adj.}
\end{itemize}
\begin{itemize}
\item {Proveniência:(Lat. \textunderscore revocatorius\textunderscore )}
\end{itemize}
O mesmo que \textunderscore revogatório\textunderscore .
\section{Revocável}
\begin{itemize}
\item {Grp. gram.:adj.}
\end{itemize}
\begin{itemize}
\item {Proveniência:(Do lat. \textunderscore revocabilis\textunderscore )}
\end{itemize}
Que se póde revocar.
\section{Revocavelmente}
\begin{itemize}
\item {Grp. gram.:adv.}
\end{itemize}
De modo revocável.
\section{Revogabilidade}
\begin{itemize}
\item {Grp. gram.:f.}
\end{itemize}
Qualidade de revogável. Cf. R. de Brito, \textunderscore Philos. do Dir.\textunderscore , 262.
\section{Revogação}
\begin{itemize}
\item {Grp. gram.:f.}
\end{itemize}
\begin{itemize}
\item {Proveniência:(Do lat. \textunderscore revocatio\textunderscore )}
\end{itemize}
Acto ou effeito de revogar; annullação.
\section{Revogador}
\begin{itemize}
\item {Grp. gram.:m.  e  adj.}
\end{itemize}
\begin{itemize}
\item {Proveniência:(Do lat. \textunderscore revocator\textunderscore )}
\end{itemize}
O que revoga.
\section{Revogante}
\begin{itemize}
\item {Grp. gram.:adj.}
\end{itemize}
\begin{itemize}
\item {Proveniência:(Do lat. \textunderscore revocans\textunderscore )}
\end{itemize}
Que revoga.
\section{Revogar}
\begin{itemize}
\item {Grp. gram.:v. t.}
\end{itemize}
\begin{itemize}
\item {Proveniência:(Do lat. \textunderscore revocare\textunderscore )}
\end{itemize}
Tornar nullo.
Desfazer, tirar o effeito a.
Fazer que deixe de vigorar: \textunderscore revogar uma lei\textunderscore .
\section{Revogativo}
\begin{itemize}
\item {Grp. gram.:adj.}
\end{itemize}
O mesmo que \textunderscore revogatório\textunderscore : \textunderscore acção revogativa\textunderscore .
\section{Revogatória}
\begin{itemize}
\item {Grp. gram.:f.}
\end{itemize}
Documento, que envolve revogação.
(Fem. de \textunderscore revogatório\textunderscore )
\section{Revogatório}
\begin{itemize}
\item {Grp. gram.:adj.}
\end{itemize}
\begin{itemize}
\item {Proveniência:(Do lat. \textunderscore revocatorius\textunderscore )}
\end{itemize}
O mesmo que \textunderscore revogante\textunderscore .
Que contém revogação.
Relativo a revogação.
\section{Revogável}
\begin{itemize}
\item {Grp. gram.:adj.}
\end{itemize}
\begin{itemize}
\item {Proveniência:(Do lat. \textunderscore revocabilis\textunderscore )}
\end{itemize}
Que se póde revogar.
\section{Revogavelmente}
\begin{itemize}
\item {Grp. gram.:adv.}
\end{itemize}
De modo revogável.
\section{Revolcar}
\begin{itemize}
\item {Grp. gram.:v. t.}
\end{itemize}
O mesmo que \textunderscore rebolcar\textunderscore . Cf. Filinto, VI, 290.
\section{Revolitar}
\begin{itemize}
\item {Grp. gram.:v. i.}
\end{itemize}
\begin{itemize}
\item {Proveniência:(De \textunderscore re...\textunderscore  + \textunderscore volitar\textunderscore )}
\end{itemize}
Volitar muito; revolutear. Cf. Alv. Mendes, \textunderscore Discursos\textunderscore , 216.
\section{Revolta}
\begin{itemize}
\item {Grp. gram.:f.}
\end{itemize}
\begin{itemize}
\item {Utilização:Prov.}
\end{itemize}
\begin{itemize}
\item {Utilização:minh.}
\end{itemize}
\begin{itemize}
\item {Proveniência:(De \textunderscore revoltar\textunderscore )}
\end{itemize}
Acto ou effeito de revoltar.
Rebelião; sublevação.
Desordem.
Grande perturbação moral.
Volta ou curva de rio.
\section{Revoltado}
\begin{itemize}
\item {Grp. gram.:adj.}
\end{itemize}
\begin{itemize}
\item {Utilização:Fig.}
\end{itemize}
\begin{itemize}
\item {Grp. gram.:M.}
\end{itemize}
\begin{itemize}
\item {Proveniência:(De \textunderscore revoltar\textunderscore )}
\end{itemize}
Que se revoltou; que se rebellou.
Insubmisso.
Indivíduo, que se rebellou.
Aquelle que é insubmisso.
\section{Revoltador}
\begin{itemize}
\item {Grp. gram.:m.  e  adj.}
\end{itemize}
O que revolta.
\section{Revoltante}
\begin{itemize}
\item {Grp. gram.:adj.}
\end{itemize}
Que revolta.
Repugnante; repulsivo; nojento.
\section{Revoltão}
\begin{itemize}
\item {Grp. gram.:m.}
\end{itemize}
\begin{itemize}
\item {Proveniência:(De \textunderscore revolta\textunderscore )}
\end{itemize}
Movimento desordenado:«\textunderscore ...bigorna... que, a revoltões, medio os céos, e infernos\textunderscore ». Filinto, I, 233.
\section{Revoltar}
\begin{itemize}
\item {Grp. gram.:v. t.}
\end{itemize}
\begin{itemize}
\item {Utilização:Fig.}
\end{itemize}
\begin{itemize}
\item {Grp. gram.:V. i.}
\end{itemize}
\begin{itemize}
\item {Proveniência:(De \textunderscore revôlto\textunderscore )}
\end{itemize}
Tornar insubordinado.
Sublevar; agitar.
Perturbar moralmente.
Indignar; causar repugnância a.
Causar indignação: \textunderscore essas coisas revoltam\textunderscore .
\section{Revoltear}
\begin{itemize}
\item {Grp. gram.:v. t.}
\end{itemize}
\begin{itemize}
\item {Grp. gram.:V. i.}
\end{itemize}
\begin{itemize}
\item {Proveniência:(De \textunderscore re...\textunderscore  + \textunderscore volta\textunderscore )}
\end{itemize}
Dar muitas voltas a.
Fazer voltar muito.
Revolver.
Dar muitas voltas; revolver-se.
\section{Revoltilho}
\begin{itemize}
\item {Grp. gram.:m.}
\end{itemize}
\begin{itemize}
\item {Utilização:Des.}
\end{itemize}
Tumulto, motim. Cf. S. R. Viterbo, \textunderscore Elucidário\textunderscore .
\section{Revôlto}
\begin{itemize}
\item {Grp. gram.:adj.}
\end{itemize}
\begin{itemize}
\item {Proveniência:(Do lat. \textunderscore revolutus\textunderscore )}
\end{itemize}
Recurvo.
Agitado; tumultuoso; furioso: \textunderscore oceano revôlto\textunderscore .
\section{Revoltoso}
\begin{itemize}
\item {Grp. gram.:m.}
\end{itemize}
O mesmo que \textunderscore revoltado\textunderscore .
O mesmo que \textunderscore revôlto\textunderscore .
\section{Revolução}
\begin{itemize}
\item {Grp. gram.:f.}
\end{itemize}
\begin{itemize}
\item {Proveniência:(Do lat. \textunderscore revolutio\textunderscore )}
\end{itemize}
Acto ou effeito de revolver.
Volta de um astro ao ponto donde partiu.
Tempo, que um astro emprega em percorrer a sua órbita.
Movimento de rotação em volta de um eixo immóvel.
Giro.
Movimento anormal dos humores orgânicos.
Transformação violenta, e mais ou menos rápida, da situação política ou social de um Estado ou de um país.
Cada uma das transformações naturaes da superfície do globo.
Systema político, philosóphico ou religioso, em opposição ás ideias do passado.
Perturbação moral.
Indignação, nojo, repulsão; náusea.
\section{Revolucionamento}
\begin{itemize}
\item {Grp. gram.:m.}
\end{itemize}
Acto ou effeito de revolucionar.
\section{Revolucionar}
\begin{itemize}
\item {Grp. gram.:v. t.}
\end{itemize}
\begin{itemize}
\item {Proveniência:(Do lat. \textunderscore revolutio\textunderscore )}
\end{itemize}
Excitar á revolução.
Agitar por meio de ideias revolucionárias.
Revolver.
Agitar moralmente; transformar.
\section{Revolucionariamente}
\begin{itemize}
\item {Grp. gram.:adv.}
\end{itemize}
De modo revolucionário.
Á maneira de revolução.
Insubordinadamente.
\section{Revolucionário}
\begin{itemize}
\item {Grp. gram.:adj.}
\end{itemize}
\begin{itemize}
\item {Grp. gram.:M.}
\end{itemize}
\begin{itemize}
\item {Proveniência:(De \textunderscore revolucionar\textunderscore )}
\end{itemize}
Relativo a revolução.
Affeiçoado a revoluções: \textunderscore jornal revolucionário\textunderscore .
Aquelle que deseja ou provoca revolução.
Progressista.
Autor de novo systema ou de novo processo.
Aquelle que é affeiçoado a renovações políticas, moraes ou sociaes; innovador.
\section{Revoluteante}
\begin{itemize}
\item {Grp. gram.:adj.}
\end{itemize}
Que revoluteia. Cf. Júl. Lour. Pinto, \textunderscore Senh. Deput.\textunderscore , 60.
\section{Revolutear}
\begin{itemize}
\item {Grp. gram.:v. i.}
\end{itemize}
\begin{itemize}
\item {Grp. gram.:M.}
\end{itemize}
\begin{itemize}
\item {Proveniência:(De \textunderscore re...\textunderscore  + \textunderscore volutear\textunderscore )}
\end{itemize}
Revolver-se; agitar-se em vários sentidos.
Adejar, esvoaçar.
Acto de revolutear.
\section{Rabdito}
\begin{itemize}
\item {Grp. gram.:m.}
\end{itemize}
\begin{itemize}
\item {Utilização:Miner.}
\end{itemize}
\begin{itemize}
\item {Proveniência:(Do gr. \textunderscore rhabdos\textunderscore )}
\end{itemize}
Fosforeto de ferro meteórico.
\section{Rabdóide}
\begin{itemize}
\item {Grp. gram.:adj.}
\end{itemize}
O mesmo que \textunderscore rabdoídeo\textunderscore .
\section{Rabdoídeo}
\begin{itemize}
\item {Grp. gram.:adj.}
\end{itemize}
\begin{itemize}
\item {Proveniência:(Do gr. \textunderscore rhabdos\textunderscore  + \textunderscore eidos\textunderscore )}
\end{itemize}
Semelhante a uma varinha.
\section{Rabdologia}
\begin{itemize}
\item {Grp. gram.:f.}
\end{itemize}
\begin{itemize}
\item {Proveniência:(Do gr. \textunderscore rhabdos\textunderscore  + \textunderscore logos\textunderscore )}
\end{itemize}
Arte de calcular com pauzinhos, em que estão escritos os números simples.
\section{Rabdológico}
\begin{itemize}
\item {Grp. gram.:adj.}
\end{itemize}
Relativo á rabdologia.
\section{Rabdomancia}
\begin{itemize}
\item {Grp. gram.:f.}
\end{itemize}
\begin{itemize}
\item {Proveniência:(Gr. \textunderscore rhabdomanteia\textunderscore )}
\end{itemize}
Adivinhação por meio de varinha mágica.
\section{Rabdomântico}
\begin{itemize}
\item {Grp. gram.:adj.}
\end{itemize}
Relativo á rabdomancia.
\section{Rabdomioma}
\begin{itemize}
\item {Grp. gram.:m.}
\end{itemize}
\begin{itemize}
\item {Proveniência:(Do gr. \textunderscore rhabdos\textunderscore  + \textunderscore mux\textunderscore )}
\end{itemize}
Mioma, formado de músculos estriados.
\section{Rabdoscopia}
\begin{itemize}
\item {Grp. gram.:f.}
\end{itemize}
\begin{itemize}
\item {Proveniência:(Do gr. \textunderscore rhabdos\textunderscore  + \textunderscore skopein\textunderscore )}
\end{itemize}
O mesmo que \textunderscore rabdomancia\textunderscore .
\section{Rabdota}
\begin{itemize}
\item {Grp. gram.:f.}
\end{itemize}
Gênero de insectos coleópteros pentâmeros.
\section{Rabdoteca}
\begin{itemize}
\item {Grp. gram.:f.}
\end{itemize}
\begin{itemize}
\item {Proveniência:(Do gr. \textunderscore rabdos\textunderscore  + \textunderscore tekhe\textunderscore )}
\end{itemize}
Gênero de plantas, da fam. das compostas.
\section{Radinocarpo}
\begin{itemize}
\item {Grp. gram.:m.}
\end{itemize}
\begin{itemize}
\item {Proveniência:(Do gr. \textunderscore rhadinos\textunderscore  + \textunderscore karpos\textunderscore )}
\end{itemize}
Gênero de arbustos leguminosos da América do Sul.
\section{Rágada}
\begin{itemize}
\item {Grp. gram.:f.}
\end{itemize}
(V.ragádia)
\section{Ragádia}
\begin{itemize}
\item {Grp. gram.:f.}
\end{itemize}
\begin{itemize}
\item {Proveniência:(Lat. \textunderscore rhagadia\textunderscore )}
\end{itemize}
Ulceração, quási sempre em fórma de fenda, e geralmente resultante da sífilis.
\section{Ragião}
\begin{itemize}
\item {Grp. gram.:m.}
\end{itemize}
O mesmo ou melhor que \textunderscore ragio\textunderscore .
\section{Ragio}
\begin{itemize}
\item {Grp. gram.:m.}
\end{itemize}
\begin{itemize}
\item {Proveniência:(Gr. \textunderscore rhagion\textunderscore )}
\end{itemize}
Gênero de insectos coleópteros, longicórneos e lignívoros.
\section{Ragoídeo}
\begin{itemize}
\item {Grp. gram.:adj.}
\end{itemize}
\begin{itemize}
\item {Proveniência:(Do gr. \textunderscore rhax\textunderscore , \textunderscore rhagos\textunderscore  + \textunderscore eidos\textunderscore )}
\end{itemize}
Semelhante a um bago de uva.
\section{Ramnáceas}
\begin{itemize}
\item {Grp. gram.:f. pl.}
\end{itemize}
\begin{itemize}
\item {Proveniência:(De \textunderscore ramno\textunderscore )}
\end{itemize}
Família de plantas, que tem por tipo o sanguinheiro.
\section{Râmneas}
\begin{itemize}
\item {Grp. gram.:f. pl.}
\end{itemize}
(V.ramnáceas)
\section{Ramnegina}
\begin{itemize}
\item {Grp. gram.:f.}
\end{itemize}
Substância còrante, amarela, que se encontra no rhamno tinctorial, e que é isómera da ramnina.
\section{Rafanidose}
\begin{itemize}
\item {Grp. gram.:f.}
\end{itemize}
\begin{itemize}
\item {Proveniência:(Gr. \textunderscore rhaphanidosis\textunderscore )}
\end{itemize}
Suplício, que os Atenienses aplicavam aos adúlteros.
\section{Rafiolépides}
\begin{itemize}
\item {Grp. gram.:f.}
\end{itemize}
\begin{itemize}
\item {Proveniência:(Do gr. \textunderscore raphis\textunderscore  + \textunderscore lepis\textunderscore )}
\end{itemize}
Gênero de plantas rosáceas.
\section{Ramnina}
\begin{itemize}
\item {Grp. gram.:f.}
\end{itemize}
\begin{itemize}
\item {Proveniência:(De \textunderscore ramno\textunderscore )}
\end{itemize}
Substância, extraida do rhamno tinctorial e que é isómera da ramnegina.
\section{Ramno}
\begin{itemize}
\item {Grp. gram.:m.}
\end{itemize}
\begin{itemize}
\item {Proveniência:(Lat. \textunderscore rhamnus\textunderscore )}
\end{itemize}
Nome científico do \textunderscore sanguinheiro\textunderscore .
\section{Ramnoídeas}
\begin{itemize}
\item {Grp. gram.:f. pl.}
\end{itemize}
\begin{itemize}
\item {Proveniência:(Do gr. \textunderscore rhamnos\textunderscore  + \textunderscore eidos\textunderscore )}
\end{itemize}
Ordem de plantas, que contém as ramnáceas e outras.
\section{Ramnoxantina}
\begin{itemize}
\item {Grp. gram.:f.}
\end{itemize}
\begin{itemize}
\item {Proveniência:(Do gr. \textunderscore rhamnos\textunderscore  + \textunderscore xanthos\textunderscore )}
\end{itemize}
Substância, que se encontra na casca e sementes de alguns ramnos.
\section{Randa}
\begin{itemize}
\item {Grp. gram.:f.}
\end{itemize}
Pequena faca ou punhal, usado antigamente na península hispânica. Cf. C. Aires, \textunderscore Hist. do Exérc. Port.\textunderscore 
\section{Ranfoteca}
\begin{itemize}
\item {Grp. gram.:f.}
\end{itemize}
\begin{itemize}
\item {Utilização:Zool.}
\end{itemize}
\begin{itemize}
\item {Proveniência:(Do gr. \textunderscore ramphos\textunderscore  + \textunderscore theke\textunderscore )}
\end{itemize}
Tegumento córneo ou cutâneo do bico das aves.
\section{Rapôntico-da-terra}
\begin{itemize}
\item {Grp. gram.:f.}
\end{itemize}
Planta, da família das compostas, (\textunderscore centaurea tagana\textunderscore , Brot.).
\section{Rapsode}
\begin{itemize}
\item {Grp. gram.:m.}
\end{itemize}
(V.rapsodo)
\section{Rapsódia}
\begin{itemize}
\item {Grp. gram.:f.}
\end{itemize}
\begin{itemize}
\item {Utilização:Ext.}
\end{itemize}
\begin{itemize}
\item {Utilização:Mús.}
\end{itemize}
\begin{itemize}
\item {Proveniência:(Lat. \textunderscore rhapsodia\textunderscore )}
\end{itemize}
Fragmentos de cantos épicos, entre os Gregos.
Cada um dos livros de Homero.
Trêcho de uma composição poética.
Composição musical, formada de diversos cantos tradicionaes ou populares de um país.
\section{Rapsódico}
\begin{itemize}
\item {Grp. gram.:adj.}
\end{itemize}
Relativo a rapsódia.
\section{Rapsodista}
\begin{itemize}
\item {Grp. gram.:m.  e  f.}
\end{itemize}
\begin{itemize}
\item {Proveniência:(De \textunderscore rapsódia\textunderscore )}
\end{itemize}
Pessôa, que faz rapsódias, ou que faz compilação de poesias ou de quaesquer trêchos literários.
\section{Rapsodo}
\begin{itemize}
\item {fónica:sôousó}
\end{itemize}
\begin{itemize}
\item {Grp. gram.:m.}
\end{itemize}
\begin{itemize}
\item {Utilização:Fig.}
\end{itemize}
\begin{itemize}
\item {Proveniência:(Lat. \textunderscore rhapsodos\textunderscore )}
\end{itemize}
Cantor ambulante de rapsódias, na antiguidade grega.
Poéta.
\section{Rapsodomancia}
\begin{itemize}
\item {Grp. gram.:f.}
\end{itemize}
\begin{itemize}
\item {Proveniência:(Do gr. \textunderscore rhapsodos\textunderscore  + \textunderscore manteia\textunderscore )}
\end{itemize}
Espécie de adivinhação, por meio de trechos, extraidos, á sorte, das obras de um poéta.
\section{Raque}
\begin{itemize}
\item {Proveniência:(Gr. \textunderscore rhakhis\textunderscore )}
\end{itemize}
\textunderscore f.\textunderscore  (e der.)
O mesmo ou melhor que \textunderscore rache\textunderscore .
\section{Reciense}
\begin{itemize}
\item {Grp. gram.:adj.}
\end{itemize}
O mesmo que \textunderscore rético\textunderscore .
\section{Reda}
\begin{itemize}
\item {Grp. gram.:f.}
\end{itemize}
\begin{itemize}
\item {Proveniência:(Lat. \textunderscore rheda\textunderscore )}
\end{itemize}
Espécie de carroça ligeira, usada pelos antigos.
\section{Redário}
\begin{itemize}
\item {Grp. gram.:m.}
\end{itemize}
\begin{itemize}
\item {Proveniência:(Lat. \textunderscore rhedárius\textunderscore )}
\end{itemize}
Conductor de reda.
\section{Reelectómetro}
\begin{itemize}
\item {Grp. gram.:m.}
\end{itemize}
\begin{itemize}
\item {Proveniência:(De \textunderscore rhein\textunderscore  gr. + \textunderscore electómetro\textunderscore )}
\end{itemize}
Aparelho, para indicar a magnetização de uma agulha, por meio de uma corrente voltaica.
\section{Reína}
\begin{itemize}
\item {Grp. gram.:f.}
\end{itemize}
\begin{itemize}
\item {Proveniência:(Do lat. bot. \textunderscore rheum\textunderscore , rhuibarbo)}
\end{itemize}
Substância amarela, que se acha na raíz do rhuibarbo.
\section{Remático}
\begin{itemize}
\item {Grp. gram.:adj.}
\end{itemize}
Diz-se de um dos ramos da classificação morfológica das línguas, no qual o valor gramatical de um vocábulo é determinado pela sua colocação com respeito aos outros da mesma proposição.
\section{Renano}
\begin{itemize}
\item {Grp. gram.:adj.}
\end{itemize}
\begin{itemize}
\item {Proveniência:(Lat. \textunderscore rhenanus\textunderscore )}
\end{itemize}
Relativo ao rio Reno.
\section{Reobárbaro}
\begin{itemize}
\item {Grp. gram.:m.}
\end{itemize}
\begin{itemize}
\item {Utilização:Des.}
\end{itemize}
O mesmo que \textunderscore ruibarbo\textunderscore . Cf. B. Pereira, \textunderscore Prosódia\textunderscore , vb. \textunderscore rheon\textunderscore .
(Cp. \textunderscore reubárbaro\textunderscore )
\section{Reóforo}
\begin{itemize}
\item {Grp. gram.:m.}
\end{itemize}
\begin{itemize}
\item {Proveniência:(Do gr. \textunderscore rheos\textunderscore  + \textunderscore phoros\textunderscore )}
\end{itemize}
Cada um dos fios metálicos, que numa pilha conduzem as duas correntes eléctricas.
\section{Reómetro}
\begin{itemize}
\item {Grp. gram.:m.}
\end{itemize}
\begin{itemize}
\item {Proveniência:(Do gr. \textunderscore rheos\textunderscore  + \textunderscore metron\textunderscore )}
\end{itemize}
Instrumento, para medir a fôrça de uma corrente eléctrica; galvanómetro.
\section{Revoluto}
\begin{itemize}
\item {Grp. gram.:adj.}
\end{itemize}
\begin{itemize}
\item {Proveniência:(Lat. \textunderscore revolutus\textunderscore )}
\end{itemize}
Revolvido; revôlto.
\section{Revolutoso}
\begin{itemize}
\item {Grp. gram.:adj.}
\end{itemize}
\begin{itemize}
\item {Utilização:Bot.}
\end{itemize}
\begin{itemize}
\item {Proveniência:(De \textunderscore revoluto\textunderscore )}
\end{itemize}
Revirado ou enrolado para baixo ou para fóra, (falando-se de órgãos vegetaes).
\section{Revolúvel}
\begin{itemize}
\item {Grp. gram.:adj.}
\end{itemize}
Muito volúvel. Cf. Castilho, \textunderscore Metam.\textunderscore , 192.
\section{Revolve}
\begin{itemize}
\item {Grp. gram.:m.}
\end{itemize}
\begin{itemize}
\item {Utilização:Pop.}
\end{itemize}
O mesmo que \textunderscore revólver\textunderscore .
\section{Revolvedor}
\begin{itemize}
\item {Grp. gram.:m.  e  adj.}
\end{itemize}
O que revolve, o que agita.
\section{Revolver}
\begin{itemize}
\item {Grp. gram.:v. t.}
\end{itemize}
\begin{itemize}
\item {Grp. gram.:V. i.}
\end{itemize}
\begin{itemize}
\item {Utilização:Des.}
\end{itemize}
\begin{itemize}
\item {Proveniência:(Lat. \textunderscore revolvere\textunderscore )}
\end{itemize}
Volver muitas vezes.
Remexer.
Mover em giro.
Investigar; examinar minuciosamente: \textunderscore revolver archivos\textunderscore .
Agitar.
Revoltar; desordenar.
Girar.
Agitar-se.
\section{Revólver}
\begin{itemize}
\item {Grp. gram.:m.}
\end{itemize}
\begin{itemize}
\item {Proveniência:(Ingl. \textunderscore revolver\textunderscore )}
\end{itemize}
Espécie de pistola com várias culatras num cylindro giratório, a qual pode dar tantos tiros, quantas as cargas que póde conter êsse cylindro.
\section{Revolvimento}
\begin{itemize}
\item {Grp. gram.:m.}
\end{itemize}
O mesmo que \textunderscore revolução\textunderscore .
Acto de revolver.
\section{Revôo}
\begin{itemize}
\item {Grp. gram.:m.}
\end{itemize}
\begin{itemize}
\item {Proveniência:(De \textunderscore re...\textunderscore  + \textunderscore vôo\textunderscore )}
\end{itemize}
Acto de revoar.
\section{Revora}
\textunderscore f.\textunderscore  (e der.)
(V. \textunderscore rebora\textunderscore , etc.)
\section{Revosso}
\begin{itemize}
\item {Grp. gram.:pron.}
\end{itemize}
\begin{itemize}
\item {Proveniência:(De \textunderscore re...\textunderscore  + \textunderscore vosso\textunderscore )}
\end{itemize}
Muito vosso:«\textunderscore hei de ser vosso e revosso.\textunderscore »\textunderscore Filodemo\textunderscore , IV, 2.
\section{Revulsão}
\begin{itemize}
\item {Grp. gram.:f.}
\end{itemize}
\begin{itemize}
\item {Utilização:Med.}
\end{itemize}
\begin{itemize}
\item {Proveniência:(Do lat. \textunderscore revulsio\textunderscore )}
\end{itemize}
Effeito dos medicamentos revulsivos.
Derivação de humores.
\section{Revulstar}
\begin{itemize}
\item {Grp. gram.:v.}
\end{itemize}
\begin{itemize}
\item {Utilização:t. Med.}
\end{itemize}
\begin{itemize}
\item {Proveniência:(Do lat. \textunderscore revulsus\textunderscore )}
\end{itemize}
Exercer acção revulsiva em; deslocar com revulsivos.
\section{Revulsivo}
\begin{itemize}
\item {Grp. gram.:adj.}
\end{itemize}
\begin{itemize}
\item {Grp. gram.:M.}
\end{itemize}
\begin{itemize}
\item {Proveniência:(Do lat. \textunderscore revulsus\textunderscore )}
\end{itemize}
Que faz derivar uma inflammação ou humores, de um para outro ponto do organismo.
Medicamento revulsivo; derivativo.
\section{Revulsor}
\begin{itemize}
\item {Grp. gram.:m.}
\end{itemize}
\begin{itemize}
\item {Proveniência:(De \textunderscore revulsar\textunderscore )}
\end{itemize}
Instrumento, para produzir sôbre a pelle pequenas aberturas ou uma irritação artificial e desviar para aí a séde de uma affecção.
\section{Revulsório}
\begin{itemize}
\item {Grp. gram.:adj.}
\end{itemize}
O mesmo que \textunderscore revulsivo\textunderscore .
\section{Rexa}
\begin{itemize}
\item {Grp. gram.:f.}
\end{itemize}
O mesmo que \textunderscore reixa\textunderscore ^1. Cf. Camillo, \textunderscore Olho de Vidro\textunderscore , 153: Sousa, \textunderscore Vida do Arceb.\textunderscore 
\section{Rexelo}
\begin{itemize}
\item {fónica:xê}
\end{itemize}
\begin{itemize}
\item {Grp. gram.:m.}
\end{itemize}
\begin{itemize}
\item {Utilização:Prov.}
\end{itemize}
\begin{itemize}
\item {Utilização:alent.}
\end{itemize}
\begin{itemize}
\item {Utilização:trasm.}
\end{itemize}
\begin{itemize}
\item {Utilização:Prov.}
\end{itemize}
\begin{itemize}
\item {Utilização:alent.}
\end{itemize}
O mesmo que \textunderscore cordeiro\textunderscore .
Qualquer pequeno animal lanígero ou cabrum.
Designação genérica de qualquer rês ovina. Cf. Deusdado, \textunderscore Escorços Trasm.\textunderscore , 331.
O mesmo que \textunderscore reixelo\textunderscore .
\section{Rexerta}
\begin{itemize}
\item {Grp. gram.:adj. f.}
\end{itemize}
\begin{itemize}
\item {Utilização:Prov.}
\end{itemize}
\begin{itemize}
\item {Utilização:trasm.}
\end{itemize}
Diz-se da mulher deslavada, delambida e respondona, de má língua.
\section{Rexio}
\begin{itemize}
\item {Grp. gram.:m.}
\end{itemize}
\begin{itemize}
\item {Utilização:Prov.}
\end{itemize}
\begin{itemize}
\item {Utilização:trasm.}
\end{itemize}
Ar frio e cortante da noite ou da madrugada.
(Cp. \textunderscore rócio\textunderscore ^1, orvalho, cuja pronuncia exacta é \textunderscore rocío\textunderscore )
\section{Rexòxó}
\begin{itemize}
\item {Grp. gram.:m.}
\end{itemize}
\begin{itemize}
\item {Utilização:Prov.}
\end{itemize}
\begin{itemize}
\item {Utilização:minh.}
\end{itemize}
\begin{itemize}
\item {Utilização:Prov.}
\end{itemize}
\begin{itemize}
\item {Utilização:trasm.}
\end{itemize}
Lugar, exposto ao sol; soalheira.
Reverbério, descompostura áspera.
(Cp. \textunderscore rexòxó\textunderscore )
\section{Rez}
\begin{itemize}
\item {Grp. gram.:f.}
\end{itemize}
(V.rês)
\section{Reza}
\begin{itemize}
\item {Grp. gram.:f.}
\end{itemize}
Acto ou effeito de rezar.
\section{Rezada}
\begin{itemize}
\item {Grp. gram.:f.}
\end{itemize}
\begin{itemize}
\item {Utilização:Prov.}
\end{itemize}
\begin{itemize}
\item {Utilização:minh.}
\end{itemize}
\begin{itemize}
\item {Proveniência:(De \textunderscore rezar\textunderscore )}
\end{itemize}
Raza, que se faz em commum pelos mortos.
\section{Rezadeira}
\begin{itemize}
\item {Grp. gram.:f.}
\end{itemize}
\begin{itemize}
\item {Utilização:Prov.}
\end{itemize}
\begin{itemize}
\item {Proveniência:(De \textunderscore rezar\textunderscore )}
\end{itemize}
Mulhér, que faz rezas ou deita cartas, para prevêr o futuro e afugentar males.
\section{Rezadeiro}
\begin{itemize}
\item {Grp. gram.:adj.}
\end{itemize}
Que reza muito; que é muito beato. Cf. Camillo, \textunderscore Brasileira\textunderscore , 237.
\section{Rezador}
\begin{itemize}
\item {Grp. gram.:m.  e  adj.}
\end{itemize}
O que reza.
\section{Rezão}
\begin{itemize}
\item {Grp. gram.:m.}
\end{itemize}
\begin{itemize}
\item {Utilização:ant.}
\end{itemize}
\begin{itemize}
\item {Utilização:Pop.}
\end{itemize}
O mesmo que \textunderscore razão\textunderscore . Cf. Usque, 11.
\section{Rezão}
\begin{itemize}
\item {Grp. gram.:m.}
\end{itemize}
\begin{itemize}
\item {Utilização:Prov.}
\end{itemize}
\begin{itemize}
\item {Utilização:trasm.}
\end{itemize}
Aquelle que reza muito.
\section{Rezar}
\begin{itemize}
\item {Grp. gram.:v. t.}
\end{itemize}
\begin{itemize}
\item {Utilização:Pop.}
\end{itemize}
\begin{itemize}
\item {Grp. gram.:V. i.}
\end{itemize}
\begin{itemize}
\item {Utilização:Fig.}
\end{itemize}
\begin{itemize}
\item {Utilização:Pop.}
\end{itemize}
\begin{itemize}
\item {Proveniência:(Lat. \textunderscore recitare\textunderscore )}
\end{itemize}
Dizer (orações ou súpplicas religiosas).
Conter, memetállicos, que numa pilha conduzem as duas correntes eléctricas.
(Do gr. \textunderscore rheos\textunderscore  + \textunderscore phoros\textunderscore )ncionar, referir.
Resmungar; murmurar.
Orar; dirigir súpplicas á divindade ou aos santos.
Tratar; referir-se: \textunderscore a nosso respeito, assim rezava aquelle documento\textunderscore .
Resmungar.
\section{Rezaria}
\begin{itemize}
\item {Grp. gram.:f.}
\end{itemize}
Muitas rezas.
Práticas de beatério.
\section{Rezental}
\begin{itemize}
\item {Grp. gram.:m.}
\end{itemize}
\begin{itemize}
\item {Utilização:Prov.}
\end{itemize}
\begin{itemize}
\item {Utilização:trasm.}
\end{itemize}
O mesmo que \textunderscore recental\textunderscore .
\section{Rezentaleira}
\begin{itemize}
\item {Grp. gram.:f.}
\end{itemize}
\begin{itemize}
\item {Utilização:Prov.}
\end{itemize}
\begin{itemize}
\item {Utilização:trasm.}
\end{itemize}
\begin{itemize}
\item {Proveniência:(De \textunderscore rezental\textunderscore )}
\end{itemize}
Ovelha, das que andam com os rezentaes em melhor pasto, separadas do rebanho.
\section{Rezente}
\begin{itemize}
\item {Grp. gram.:adj.}
\end{itemize}
\begin{itemize}
\item {Utilização:Ant.}
\end{itemize}
O mesmo que \textunderscore recente\textunderscore . Cf. Usque, 10.
\section{Rezinga}
\begin{itemize}
\item {Grp. gram.:f.}
\end{itemize}
\begin{itemize}
\item {Utilização:Pop.}
\end{itemize}
Acto de rezingar.
\section{Rezingão}
\begin{itemize}
\item {Grp. gram.:m.  e  adj.}
\end{itemize}
O que rezinga.
\section{Rezingar}
\begin{itemize}
\item {Grp. gram.:v. i.}
\end{itemize}
\begin{itemize}
\item {Utilização:Pop.}
\end{itemize}
Altercar.
Resmungar.
Recalcitrar.
(Cp. cast. \textunderscore rezongar\textunderscore )
\section{Rezingueiro}
\begin{itemize}
\item {Grp. gram.:m.  e  adj.}
\end{itemize}
O mesmo que \textunderscore rezingão\textunderscore .
\section{Rezo}
\begin{itemize}
\item {Grp. gram.:m.}
\end{itemize}
\begin{itemize}
\item {Utilização:Fam.}
\end{itemize}
O mesmo que \textunderscore reza\textunderscore . Cf. Camillo, \textunderscore Corja\textunderscore , 23.
\section{Rezuela}
\begin{itemize}
\item {Grp. gram.:f.}
\end{itemize}
Planta da serra de Sintra.
\section{Rhabárbaro}
\begin{itemize}
\item {Grp. gram.:m.}
\end{itemize}
O mesmo que \textunderscore rhuibarbo\textunderscore . Cf. \textunderscore Pharmacopeia Port.\textunderscore 
\section{Rhabdito}
\begin{itemize}
\item {Grp. gram.:m.}
\end{itemize}
\begin{itemize}
\item {Utilização:Miner.}
\end{itemize}
\begin{itemize}
\item {Proveniência:(Do gr. \textunderscore rhabdos\textunderscore )}
\end{itemize}
Phosphoreto de ferro meteórico.
\section{Rhabdóide}
\begin{itemize}
\item {Grp. gram.:adj.}
\end{itemize}
O mesmo que \textunderscore rhabdoídeo\textunderscore .
\section{Rhabdoídeo}
\begin{itemize}
\item {Grp. gram.:adj.}
\end{itemize}
\begin{itemize}
\item {Proveniência:(Do gr. \textunderscore rhabdos\textunderscore  + \textunderscore eidos\textunderscore )}
\end{itemize}
Semelhante a uma varinha.
\section{Rhabdologia}
\begin{itemize}
\item {Grp. gram.:f.}
\end{itemize}
\begin{itemize}
\item {Proveniência:(Do gr. \textunderscore rhabdos\textunderscore  + \textunderscore logos\textunderscore )}
\end{itemize}
Arte de calcular com pauzinhos, em que estão escritos os números simples.
\section{Rhabdológico}
\begin{itemize}
\item {Grp. gram.:adj.}
\end{itemize}
Relativo á rhabdologia.
\section{Rhabdomancia}
\begin{itemize}
\item {Grp. gram.:f.}
\end{itemize}
\begin{itemize}
\item {Proveniência:(Gr. \textunderscore rhabdomanteia\textunderscore )}
\end{itemize}
Adivinhação por meio de varinha mágica.
\section{Rhabdomântico}
\begin{itemize}
\item {Grp. gram.:adj.}
\end{itemize}
Relativo á rabdomancia.
\section{Rhabdomyoma}
\begin{itemize}
\item {Grp. gram.:m.}
\end{itemize}
\begin{itemize}
\item {Proveniência:(Do gr. \textunderscore rhabdos\textunderscore  + \textunderscore mux\textunderscore )}
\end{itemize}
Myoma, formado de músculos estriados.
\section{Rhabdoscopia}
\begin{itemize}
\item {Grp. gram.:f.}
\end{itemize}
\begin{itemize}
\item {Proveniência:(Do gr. \textunderscore rhabdos\textunderscore  + \textunderscore skopein\textunderscore )}
\end{itemize}
O mesmo que \textunderscore rhabdomancia\textunderscore .
\section{Rhabdota}
\begin{itemize}
\item {Grp. gram.:f.}
\end{itemize}
Gênero de insectos coleópteros pentâmeros.
\section{Rhabdotheca}
\begin{itemize}
\item {Grp. gram.:f.}
\end{itemize}
\begin{itemize}
\item {Proveniência:(Do gr. \textunderscore rabdos\textunderscore  + \textunderscore tekhe\textunderscore )}
\end{itemize}
Gênero de plantas, da fam. das compostas.
\section{Rhache}
\begin{itemize}
\item {Proveniência:(Gr. \textunderscore rhakhis\textunderscore )}
\end{itemize}
\textunderscore f.\textunderscore  (e der.)
O mesmo ou melhor que \textunderscore rache\textunderscore .
\section{Rhadinocarpo}
\begin{itemize}
\item {Grp. gram.:m.}
\end{itemize}
\begin{itemize}
\item {Proveniência:(Do gr. \textunderscore rhadinos\textunderscore  + \textunderscore karpos\textunderscore )}
\end{itemize}
Gênero de arbustos leguminosos da América do Sul.
\section{Rhágada}
\begin{itemize}
\item {Grp. gram.:f.}
\end{itemize}
(V.rhagádia)
\section{Rhagádia}
\begin{itemize}
\item {Grp. gram.:f.}
\end{itemize}
\begin{itemize}
\item {Proveniência:(Lat. \textunderscore rhagadia\textunderscore )}
\end{itemize}
Ulceração, quási sempre em fórma de fenda, e geralmente resultante da sýphilis.
\section{Rhagião}
\begin{itemize}
\item {Grp. gram.:m.}
\end{itemize}
O mesmo ou melhor que \textunderscore rhágio\textunderscore .
\section{Rhágio}
\begin{itemize}
\item {Grp. gram.:m.}
\end{itemize}
\begin{itemize}
\item {Proveniência:(Gr. \textunderscore rhagion\textunderscore )}
\end{itemize}
Gênero de insectos coleópteros, longicórneos e lignívoros.
\section{Rhagoídeo}
\begin{itemize}
\item {Grp. gram.:adj.}
\end{itemize}
\begin{itemize}
\item {Proveniência:(Do gr. \textunderscore rhax\textunderscore , \textunderscore rhagos\textunderscore  + \textunderscore eidos\textunderscore )}
\end{itemize}
Semelhante a um bago de uva.
\section{Rhamnáceas}
\begin{itemize}
\item {Grp. gram.:f. pl.}
\end{itemize}
\begin{itemize}
\item {Proveniência:(De \textunderscore rhamno\textunderscore )}
\end{itemize}
Família de plantas, que tem por typo o sanguinheiro.
\section{Rhâmneas}
\begin{itemize}
\item {Grp. gram.:f. pl.}
\end{itemize}
(V.rhamnáceas)
\section{Rhamnegina}
\begin{itemize}
\item {Grp. gram.:f.}
\end{itemize}
Substância còrante, amarela, que se encontra no rhamno tinctorial, e que é isómera da rhamnina.
\section{Rhamnina}
\begin{itemize}
\item {Grp. gram.:f.}
\end{itemize}
\begin{itemize}
\item {Proveniência:(De \textunderscore rhamno\textunderscore )}
\end{itemize}
Substância, extrahida do rhamno tinctorial e que é isómera da rhamnegina.
\section{Rhamno}
\begin{itemize}
\item {Grp. gram.:m.}
\end{itemize}
\begin{itemize}
\item {Proveniência:(Lat. \textunderscore rhamnus\textunderscore )}
\end{itemize}
Nome scientífico do \textunderscore sanguinheiro\textunderscore .
\section{Rhamnoídeas}
\begin{itemize}
\item {Grp. gram.:f. pl.}
\end{itemize}
\begin{itemize}
\item {Proveniência:(Do gr. \textunderscore rhamnos\textunderscore  + \textunderscore eidos\textunderscore )}
\end{itemize}
Ordem de plantas, que contém as rhamnáceas e outras.
\section{Rhamnoxanthina}
\begin{itemize}
\item {Grp. gram.:f.}
\end{itemize}
\begin{itemize}
\item {Proveniência:(Do gr. \textunderscore rhamnos\textunderscore  + \textunderscore xanthos\textunderscore )}
\end{itemize}
Substância, que se encontra na casca e sementes de alguns rhamnos.
\section{Rhamphotheca}
\begin{itemize}
\item {Grp. gram.:f.}
\end{itemize}
\begin{itemize}
\item {Utilização:Zool.}
\end{itemize}
\begin{itemize}
\item {Proveniência:(Do gr. \textunderscore ramphos\textunderscore  + \textunderscore theke\textunderscore )}
\end{itemize}
Tegumento córneo ou cutâneo do bico das aves.
\section{Rhanda}
\begin{itemize}
\item {Grp. gram.:f.}
\end{itemize}
Pequena faca ou punhal, usado antigamente na península hispânica. Cf. C. Aires, \textunderscore Hist. do Exérc. Port.\textunderscore 
\section{Rhaphanidose}
\begin{itemize}
\item {Grp. gram.:f.}
\end{itemize}
\begin{itemize}
\item {Proveniência:(Gr. \textunderscore rhaphanidosis\textunderscore )}
\end{itemize}
Supplício, que os Athenienses applicavam aos adúlteros.
\section{Rhaphiolépides}
\begin{itemize}
\item {Grp. gram.:f.}
\end{itemize}
\begin{itemize}
\item {Proveniência:(Do gr. \textunderscore raphis\textunderscore  + \textunderscore lepis\textunderscore )}
\end{itemize}
Gênero de plantas rosáceas.
\section{Rhapôntico-da-terra}
\begin{itemize}
\item {Grp. gram.:f.}
\end{itemize}
Planta, da família das compostas, (\textunderscore centaurea tagana\textunderscore , Brot.).
\section{Rhapsode}
\begin{itemize}
\item {Grp. gram.:m.}
\end{itemize}
(V.rhapsodo)
\section{Rhapsódia}
\begin{itemize}
\item {Grp. gram.:f.}
\end{itemize}
\begin{itemize}
\item {Utilização:Ext.}
\end{itemize}
\begin{itemize}
\item {Utilização:Mús.}
\end{itemize}
\begin{itemize}
\item {Proveniência:(Lat. \textunderscore rhapsodia\textunderscore )}
\end{itemize}
Fragmentos de cantos épicos, entre os Gregos.
Cada um dos livros de Homero.
Trêcho de uma composição poética.
Composição musical, formada de diversos cantos tradicionaes ou populares de um país.
\section{Rhapsódico}
\begin{itemize}
\item {Grp. gram.:adj.}
\end{itemize}
Relativo a rhapsódia.
\section{Rhapsodista}
\begin{itemize}
\item {Grp. gram.:m.  e  f.}
\end{itemize}
\begin{itemize}
\item {Proveniência:(De \textunderscore rhapsódia\textunderscore )}
\end{itemize}
Pessôa, que faz rapsódias, ou que faz compilação de poesias ou de quaesquer trêchos literários.
\section{Rhapsodo}
\begin{itemize}
\item {fónica:sôousó}
\end{itemize}
\begin{itemize}
\item {Grp. gram.:m.}
\end{itemize}
\begin{itemize}
\item {Utilização:Fig.}
\end{itemize}
\begin{itemize}
\item {Proveniência:(Lat. \textunderscore rhapsodos\textunderscore )}
\end{itemize}
Cantor ambulante de rhapsódias, na antiguidade grega.
Poéta.
\section{Rhapsodomancia}
\begin{itemize}
\item {Grp. gram.:f.}
\end{itemize}
\begin{itemize}
\item {Proveniência:(Do gr. \textunderscore rhapsodos\textunderscore  + \textunderscore manteia\textunderscore )}
\end{itemize}
Espécie de adivinhação, por meio de trechos, extrahidos, á sorte, das obras de um poéta.
\section{Rheciense}
\begin{itemize}
\item {Grp. gram.:adj.}
\end{itemize}
O mesmo que \textunderscore rhético\textunderscore .
\section{Rheda}
\begin{itemize}
\item {Grp. gram.:f.}
\end{itemize}
\begin{itemize}
\item {Proveniência:(Lat. \textunderscore rheda\textunderscore )}
\end{itemize}
Espécie de carroça ligeira, usada pelos antigos.
\section{Rhedário}
\begin{itemize}
\item {Grp. gram.:m.}
\end{itemize}
\begin{itemize}
\item {Proveniência:(Lat. \textunderscore rhedárius\textunderscore )}
\end{itemize}
Conductor de rheda.
\section{Rheelectómetro}
\begin{itemize}
\item {Grp. gram.:m.}
\end{itemize}
\begin{itemize}
\item {Proveniência:(De \textunderscore rhein\textunderscore  gr. + \textunderscore electómetro\textunderscore )}
\end{itemize}
Apparelho, para indicar a magnetização de uma agulha, por meio de uma corrente voltaica.
\section{Rheína}
\begin{itemize}
\item {Grp. gram.:f.}
\end{itemize}
\begin{itemize}
\item {Proveniência:(Do lat. bot. \textunderscore rheum\textunderscore , rhuibarbo)}
\end{itemize}
Substância amarela, que se acha na raíz do rhuibarbo.
\section{Rhemático}
\begin{itemize}
\item {Grp. gram.:adj.}
\end{itemize}
Diz-se de um dos ramos da classificação morphológica das línguas, no qual o valor grammatical de um vocábulo é determinado pela sua collocação com respeito aos outros da mesma proposição.
\section{Rhenano}
\begin{itemize}
\item {Grp. gram.:adj.}
\end{itemize}
\begin{itemize}
\item {Proveniência:(Lat. \textunderscore rhenanus\textunderscore )}
\end{itemize}
Relativo ao rio Rheno.
\section{Rheobárbaro}
\begin{itemize}
\item {Grp. gram.:m.}
\end{itemize}
\begin{itemize}
\item {Utilização:Des.}
\end{itemize}
O mesmo que \textunderscore rhuibarbo\textunderscore . Cf. B. Pereira, \textunderscore Prosódia\textunderscore , vb. \textunderscore rheon\textunderscore .
(Cp. \textunderscore rheubárbaro\textunderscore )
\section{Rheómetro}
\begin{itemize}
\item {Grp. gram.:m.}
\end{itemize}
\begin{itemize}
\item {Proveniência:(Do gr. \textunderscore rheos\textunderscore  + \textunderscore metron\textunderscore )}
\end{itemize}
Instrumento, para medir a fôrça de uma corrente eléctrica; galvanómetro.
\section{Rheóphoro}
\begin{itemize}
\item {Grp. gram.:m.}
\end{itemize}
\begin{itemize}
\item {Proveniência:(Do gr. \textunderscore rheos\textunderscore  + \textunderscore phoros\textunderscore )}
\end{itemize}
Cada um dos fios metállicos, que numa pilha conduzem as duas correntes eléctricas.
\section{Réostato}
\begin{itemize}
\item {Grp. gram.:m.}
\end{itemize}
\begin{itemize}
\item {Proveniência:(Do gr. \textunderscore rheos\textunderscore  + \textunderscore statos\textunderscore )}
\end{itemize}
Apparelho, com que se torna constante a fôrça das correntes eléctricas.
\section{Reótomo}
\begin{itemize}
\item {Grp. gram.:m.}
\end{itemize}
\begin{itemize}
\item {Utilização:Phýs.}
\end{itemize}
\begin{itemize}
\item {Proveniência:(Do gr. \textunderscore rheos\textunderscore  + \textunderscore tome\textunderscore )}
\end{itemize}
Aparelho, com que se interrompe periodicamente uma corrente voltaica.
\section{Reso}
\begin{itemize}
\item {Grp. gram.:m.}
\end{itemize}
Espécie de macaco.
\section{Rético}
\begin{itemize}
\item {Grp. gram.:adj.}
\end{itemize}
\begin{itemize}
\item {Grp. gram.:M.}
\end{itemize}
Relativo á Rhécia.
Dialecto ou língua dos Grisões nos Alpes.
\section{Retor}
\begin{itemize}
\item {Grp. gram.:m.}
\end{itemize}
\begin{itemize}
\item {Proveniência:(Lat. \textunderscore rhetor\textunderscore )}
\end{itemize}
Mestre de Retórica; retórico. Cf. Latino, \textunderscore Or. da Cor.\textunderscore , CVIII.
\section{Retórica}
\begin{itemize}
\item {Grp. gram.:f.}
\end{itemize}
\begin{itemize}
\item {Proveniência:(Lat. \textunderscore rhetorica\textunderscore )}
\end{itemize}
Arte de bem falar, ou conjunto de regras relativas á eloquência.
Livro, que contém essas regras.
Tudo que se emprega num discurso, para produzir bom efeito no auditório.
Discurso empolado ou pomposo.
\section{Retoricador}
\begin{itemize}
\item {Grp. gram.:m.}
\end{itemize}
\begin{itemize}
\item {Utilização:Deprec.}
\end{itemize}
\begin{itemize}
\item {Proveniência:(De \textunderscore retoricar\textunderscore )}
\end{itemize}
Palrador; tagarela:«\textunderscore importunos retoricadores.\textunderscore »Macedo, \textunderscore Motim\textunderscore , II, 18.
\section{Retoricamente}
\begin{itemize}
\item {Grp. gram.:adv.}
\end{itemize}
De modo retórico; com linguagem enfática.
\section{Retoricão}
\begin{itemize}
\item {Grp. gram.:m.}
\end{itemize}
\begin{itemize}
\item {Utilização:Deprec.}
\end{itemize}
Homem, que presume de retórico com pouco fundamento. Cf. Macedo, \textunderscore Burros\textunderscore , 213.
\section{Retoricar}
\begin{itemize}
\item {Grp. gram.:v. i.}
\end{itemize}
Aplicar as regras da Retórica.
\section{Retoricismo}
\begin{itemize}
\item {Grp. gram.:m.}
\end{itemize}
Paixão pela Retórica.
\section{Retórico}
\begin{itemize}
\item {Grp. gram.:adj.}
\end{itemize}
\begin{itemize}
\item {Grp. gram.:M.}
\end{itemize}
\begin{itemize}
\item {Proveniência:(Lat. \textunderscore rhetoricus\textunderscore )}
\end{itemize}
Relativo á Retórica.
Falador.
Que tem estilo empolado ou pretensioso.
Tratadista de Retórica.
Orador, que declama afectadamente ou em estilo empolado e impróprio.
\section{Retorismo}
\begin{itemize}
\item {Grp. gram.:m.}
\end{itemize}
\begin{itemize}
\item {Utilização:Neol.}
\end{itemize}
Influência ou domínio da Retórica.
(Us. por Silv. Romero)
\section{Reubárbaro}
\begin{itemize}
\item {Grp. gram.:m.}
\end{itemize}
\begin{itemize}
\item {Proveniência:(Lat. \textunderscore rheubarbarum\textunderscore )}
\end{itemize}
O mesmo que \textunderscore ruibarbo\textunderscore .
\section{Reuma}
\begin{itemize}
\item {Grp. gram.:f.}
\end{itemize}
\begin{itemize}
\item {Utilização:Des.}
\end{itemize}
\begin{itemize}
\item {Proveniência:(Lat. \textunderscore rheuma\textunderscore )}
\end{itemize}
Fluxão de humores crassos.
\section{Reumametria}
\begin{itemize}
\item {Grp. gram.:f.}
\end{itemize}
Aplicação do reumâmetro.
\section{Reumamétrico}
\begin{itemize}
\item {Grp. gram.:adj.}
\end{itemize}
Relativo á reumametria.
\section{Reumâmetro}
\begin{itemize}
\item {Grp. gram.:m.}
\end{itemize}
\begin{itemize}
\item {Proveniência:(Do gr. \textunderscore rheuma\textunderscore  + \textunderscore metron\textunderscore )}
\end{itemize}
Instrumento, para medir a rapidez da corrente de um líquido.
\section{Reumatalgia}
\begin{itemize}
\item {Grp. gram.:f.}
\end{itemize}
\begin{itemize}
\item {Proveniência:(Do gr. \textunderscore rheuma\textunderscore , \textunderscore rheumatos\textunderscore  + \textunderscore algos\textunderscore )}
\end{itemize}
Dôr reumatismal.
\section{Reumatálgico}
\begin{itemize}
\item {Grp. gram.:adj.}
\end{itemize}
Relativo á reumatalgia.
\section{Reumático}
\begin{itemize}
\item {Grp. gram.:adj.}
\end{itemize}
\begin{itemize}
\item {Grp. gram.:M.}
\end{itemize}
\begin{itemize}
\item {Utilização:Pop.}
\end{itemize}
\begin{itemize}
\item {Proveniência:(Lat. \textunderscore rheumaticus\textunderscore )}
\end{itemize}
Relativo á reuma.
Reumatismal; que sofre reumatismo.
Indivíduo que sofre reumatismo.
Reumatismo: \textunderscore sofrer do reumático\textunderscore .
\section{Reumatismal}
\begin{itemize}
\item {Grp. gram.:adj.}
\end{itemize}
Relativo ao reumatismo.
\section{Reumatismo}
\begin{itemize}
\item {Grp. gram.:m.}
\end{itemize}
\begin{itemize}
\item {Proveniência:(Lat. \textunderscore rheumatismus\textunderscore )}
\end{itemize}
Dôres, que têm a sua séde principal nos músculos e articulações, sem febre nem carácter inflamatório.
\section{Reumoso}
\begin{itemize}
\item {Grp. gram.:adj.}
\end{itemize}
Que tem reuma.
\section{Réxia}
\begin{itemize}
\item {fónica:csi}
\end{itemize}
\begin{itemize}
\item {Grp. gram.:f.}
\end{itemize}
\begin{itemize}
\item {Proveniência:(Lat. \textunderscore rhexia\textunderscore )}
\end{itemize}
Gênero de plantas melastomáceas da América do Sul.
\section{Rhéostato}
\begin{itemize}
\item {Grp. gram.:m.}
\end{itemize}
\begin{itemize}
\item {Proveniência:(Do gr. \textunderscore rheos\textunderscore  + \textunderscore statos\textunderscore )}
\end{itemize}
Apparelho, com que se torna constante a fôrça das correntes eléctricas.
\section{Rheótomo}
\begin{itemize}
\item {Grp. gram.:m.}
\end{itemize}
\begin{itemize}
\item {Utilização:Phýs.}
\end{itemize}
\begin{itemize}
\item {Proveniência:(Do gr. \textunderscore rheos\textunderscore  + \textunderscore tome\textunderscore )}
\end{itemize}
Apparelho, com que se interrompe periodicamente uma corrente voltaica.
\section{Rheso}
\begin{itemize}
\item {Grp. gram.:m.}
\end{itemize}
Espécie de macaco.
\section{Rhético}
\begin{itemize}
\item {Grp. gram.:adj.}
\end{itemize}
\begin{itemize}
\item {Grp. gram.:M.}
\end{itemize}
Relativo á Rhécia.
Dialecto ou língua dos Grisões nos Alpes.
\section{Rhetor}
\begin{itemize}
\item {Grp. gram.:m.}
\end{itemize}
\begin{itemize}
\item {Proveniência:(Lat. \textunderscore rhetor\textunderscore )}
\end{itemize}
Mestre de Rhetórica; rhetórico. Cf. Latino, \textunderscore Or. da Cor.\textunderscore , CVIII.
\section{Rhetórica}
\begin{itemize}
\item {Grp. gram.:f.}
\end{itemize}
\begin{itemize}
\item {Proveniência:(Lat. \textunderscore rhetorica\textunderscore )}
\end{itemize}
Arte de bem falar, ou conjunto de regras relativas á eloquência.
Livro, que contém essas regras.
Tudo que se emprega num discurso, para produzir bom effeito no auditório.
Discurso empolado ou pomposo.
\section{Rhetoricador}
\begin{itemize}
\item {Grp. gram.:m.}
\end{itemize}
\begin{itemize}
\item {Utilização:Deprec.}
\end{itemize}
\begin{itemize}
\item {Proveniência:(De \textunderscore rhetoricar\textunderscore )}
\end{itemize}
Palrador; tagarela:«\textunderscore importunos rhetoricadores.\textunderscore »Macedo, \textunderscore Motim\textunderscore , II, 18.
\section{Rhetoricamente}
\begin{itemize}
\item {Grp. gram.:adv.}
\end{itemize}
De modo rhetórico; com linguagem emphática.
\section{Rhetoricão}
\begin{itemize}
\item {Grp. gram.:m.}
\end{itemize}
\begin{itemize}
\item {Utilização:Deprec.}
\end{itemize}
Homem, que presume de rhetórico com pouco fundamento. Cf. Macedo, \textunderscore Burros\textunderscore , 213.
\section{Rhetoricar}
\begin{itemize}
\item {Grp. gram.:v. i.}
\end{itemize}
Applicar as regras da Rhetórica.
\section{Rhetoricismo}
\begin{itemize}
\item {Grp. gram.:m.}
\end{itemize}
Paixão pela Rhetórica.
\section{Rhetórico}
\begin{itemize}
\item {Grp. gram.:adj.}
\end{itemize}
\begin{itemize}
\item {Grp. gram.:M.}
\end{itemize}
\begin{itemize}
\item {Proveniência:(Lat. \textunderscore rhetoricus\textunderscore )}
\end{itemize}
Relativo á Rhetórica.
Falador.
Que tem estilo empolado ou pretensioso.
Tratadista de Rhetórica.
Orador, que declama affectadamente ou em estilo empolado e impróprio.
\section{Rhetorismo}
\begin{itemize}
\item {Grp. gram.:m.}
\end{itemize}
\begin{itemize}
\item {Utilização:Neol.}
\end{itemize}
Influência ou domínio da Rhetórica.
(Us. por Silv. Romero)
\section{Rheto-romano}
\begin{itemize}
\item {Grp. gram.:m.}
\end{itemize}
O mesmo que \textunderscore rhético\textunderscore , língua dos Grisões.
\section{Rheubárbaro}
\begin{itemize}
\item {Grp. gram.:m.}
\end{itemize}
\begin{itemize}
\item {Proveniência:(Lat. \textunderscore rheubarbarum\textunderscore )}
\end{itemize}
O mesmo que \textunderscore rhuibarbo\textunderscore .
\section{Rheuma}
\begin{itemize}
\item {Grp. gram.:f.}
\end{itemize}
\begin{itemize}
\item {Utilização:Des.}
\end{itemize}
\begin{itemize}
\item {Proveniência:(Lat. \textunderscore rheuma\textunderscore )}
\end{itemize}
Fluxão de humores crassos.
\section{Rheumametria}
\begin{itemize}
\item {Grp. gram.:f.}
\end{itemize}
Applicação do rheumâmetro.
\section{Rheumamétrico}
\begin{itemize}
\item {Grp. gram.:adj.}
\end{itemize}
Relativo á rheumametria.
\section{Rheumâmetro}
\begin{itemize}
\item {Grp. gram.:m.}
\end{itemize}
\begin{itemize}
\item {Proveniência:(Do gr. \textunderscore rheuma\textunderscore  + \textunderscore metron\textunderscore )}
\end{itemize}
Instrumento, para medir a rapidez da corrente de um líquido.
\section{Rheumatalgia}
\begin{itemize}
\item {Grp. gram.:f.}
\end{itemize}
\begin{itemize}
\item {Proveniência:(Do gr. \textunderscore rheuma\textunderscore , \textunderscore rheumatos\textunderscore  + \textunderscore algos\textunderscore )}
\end{itemize}
Dôr rheumatismal.
\section{Rheumatálgico}
\begin{itemize}
\item {Grp. gram.:adj.}
\end{itemize}
Relativo á rheumatalgia.
\section{Rheumático}
\begin{itemize}
\item {Grp. gram.:adj.}
\end{itemize}
\begin{itemize}
\item {Grp. gram.:M.}
\end{itemize}
\begin{itemize}
\item {Utilização:Pop.}
\end{itemize}
\begin{itemize}
\item {Proveniência:(Lat. \textunderscore rheumaticus\textunderscore )}
\end{itemize}
Relativo á rheuma.
Rheumatismal; que soffre rheumatismo.
Indivíduo que soffre rheumatismo.
Rheumatismo: \textunderscore soffrer do rheumático\textunderscore .
\section{Rheumatismal}
\begin{itemize}
\item {Grp. gram.:adj.}
\end{itemize}
Relativo ao rheumatismo.
\section{Rheumatismo}
\begin{itemize}
\item {Grp. gram.:m.}
\end{itemize}
\begin{itemize}
\item {Proveniência:(Lat. \textunderscore rheumatismus\textunderscore )}
\end{itemize}
Dôres, que têm a sua séde principal nos músculos e articulações, sem febre nem carácter inflammatório.
\section{Rheumoso}
\begin{itemize}
\item {Grp. gram.:adj.}
\end{itemize}
Que tem rheuma.
\section{Rhéxia}
\begin{itemize}
\item {fónica:csi}
\end{itemize}
\begin{itemize}
\item {Grp. gram.:f.}
\end{itemize}
\begin{itemize}
\item {Proveniência:(Lat. \textunderscore rhexia\textunderscore )}
\end{itemize}
Gênero de plantas melastomáceas da América do Sul.
\section{Rhina}
\begin{itemize}
\item {Proveniência:(Do gr. \textunderscore rhis\textunderscore , \textunderscore rhinos\textunderscore )}
\end{itemize}
Gênero de insectos coleópteros tetrâmeros.
\section{Rhinalgia}
\begin{itemize}
\item {Grp. gram.:f.}
\end{itemize}
\begin{itemize}
\item {Proveniência:(Do gr. \textunderscore rhis\textunderscore , \textunderscore rhinos\textunderscore  + \textunderscore algos\textunderscore )}
\end{itemize}
Dôr no nariz.
\section{Rhinálgico}
\begin{itemize}
\item {Grp. gram.:adj.}
\end{itemize}
Relativo á rhinalgia.
\section{Rhinantháceas}
\begin{itemize}
\item {Grp. gram.:f. pl.}
\end{itemize}
Família de plantas, que tem por typo o rhinantho.
(Fem. pl. de \textunderscore rhinantháceo\textunderscore )
\section{Rhinantháceo}
\begin{itemize}
\item {Grp. gram.:adj.}
\end{itemize}
Relativo ou semelhante ao rhinantho.
\section{Rhinantho}
\begin{itemize}
\item {Grp. gram.:m.}
\end{itemize}
\begin{itemize}
\item {Proveniência:(Do gr. \textunderscore rhis\textunderscore , \textunderscore rhinos\textunderscore  + \textunderscore anthos\textunderscore )}
\end{itemize}
Planta herbácea, de flôres amarelas, e cujas fôlhas são applicadas em tinturaria.
\section{Rhinária}
\begin{itemize}
\item {Grp. gram.:f.}
\end{itemize}
Gênero de insectos coleópteros, originários da Austrália.
\section{Rhináspide}
\begin{itemize}
\item {Grp. gram.:m.}
\end{itemize}
Gênero de insectos coleópteros, originários do Brasil.
\section{Rhinasto}
\begin{itemize}
\item {Grp. gram.:m.}
\end{itemize}
Gênero de insectos coleópteros tetrâmeros.
\section{Rhinchonélia}
\begin{itemize}
\item {fónica:co}
\end{itemize}
\begin{itemize}
\item {Grp. gram.:f.}
\end{itemize}
Mollusco brachiópode do Mar-Branco.
\section{Rhinelcose}
\begin{itemize}
\item {Grp. gram.:f.}
\end{itemize}
\begin{itemize}
\item {Utilização:Med.}
\end{itemize}
\begin{itemize}
\item {Proveniência:(Do gr. \textunderscore rhis\textunderscore , \textunderscore rhinos\textunderscore  + \textunderscore elkos\textunderscore )}
\end{itemize}
Ulceração da narina.
\section{Rhinencéphalo}
\begin{itemize}
\item {Grp. gram.:m.}
\end{itemize}
\begin{itemize}
\item {Proveniência:(Do gr. \textunderscore rhis\textunderscore , \textunderscore rhinos\textunderscore  + \textunderscore enkephalon\textunderscore )}
\end{itemize}
Monstro, que tem o nariz prolongado em fórma de tromba.
\section{Rhíngia}
\begin{itemize}
\item {Grp. gram.:f.}
\end{itemize}
Insecto díptero, oval e chato.
\section{Rhingícula}
\begin{itemize}
\item {Grp. gram.:f.}
\end{itemize}
Gênero de molluscos.
\section{Rhinite}
\begin{itemize}
\item {Grp. gram.:f.}
\end{itemize}
\begin{itemize}
\item {Proveniência:(Do gr. \textunderscore rhis\textunderscore , \textunderscore rhinos\textunderscore , nariz)}
\end{itemize}
Inflammação da mucosa do nariz.
\section{Rhinobronchite}
\begin{itemize}
\item {fónica:qui}
\end{itemize}
\begin{itemize}
\item {Grp. gram.:f.}
\end{itemize}
\begin{itemize}
\item {Utilização:Med.}
\end{itemize}
\begin{itemize}
\item {Proveniência:(De \textunderscore rhis\textunderscore , \textunderscore rhinos\textunderscore  gr. + \textunderscore bronchite\textunderscore )}
\end{itemize}
Inflammação das mucosas do nariz e dos brônchios.
\section{Rhinocephalia}
\begin{itemize}
\item {Grp. gram.:f.}
\end{itemize}
Estado ou qualidade de rhinocéphalo.
\section{Rhinocéphalo}
\begin{itemize}
\item {Grp. gram.:adj.}
\end{itemize}
\begin{itemize}
\item {Proveniência:(Do gr. \textunderscore rhis\textunderscore , \textunderscore rhinos\textunderscore  + \textunderscore kephale\textunderscore )}
\end{itemize}
Que tem na abóbada craniana, para trás do bregma, uma deformação, á maneira de sella.
\section{Rhinoceronte}
\begin{itemize}
\item {Grp. gram.:m.}
\end{itemize}
\begin{itemize}
\item {Proveniência:(Gr. \textunderscore rhinokeros\textunderscore )}
\end{itemize}
Grande quadrúpede selvagem da ordem dos pachydermes.
\section{Rhinocerôntico}
\begin{itemize}
\item {Grp. gram.:adj.}
\end{itemize}
Relativo ao rhinoceronte.
\section{Rhinocerote}
\begin{itemize}
\item {Grp. gram.:m.}
\end{itemize}
(V.rhinoceronte)
\section{Rhinologia}
\begin{itemize}
\item {Grp. gram.:f.}
\end{itemize}
\begin{itemize}
\item {Proveniência:(Do gr. \textunderscore rhis\textunderscore , \textunderscore rhinos\textunderscore  + \textunderscore logos\textunderscore )}
\end{itemize}
Estudo anatómico do nariz.
\section{Rhinólopho}
\begin{itemize}
\item {Grp. gram.:m.}
\end{itemize}
\begin{itemize}
\item {Proveniência:(Do gr. \textunderscore rhis\textunderscore , \textunderscore rhinos\textunderscore  + \textunderscore lophos\textunderscore )}
\end{itemize}
Gênero de morcegos, que tem sôbre o nariz uma crista membranosa, semelhante a uma ferradura. Cf. P. Moraes, \textunderscore Zool. Elem.\textunderscore , 175.
\section{Rhinómacro}
\begin{itemize}
\item {Grp. gram.:m.}
\end{itemize}
\begin{itemize}
\item {Proveniência:(Do gr. \textunderscore rhis\textunderscore , \textunderscore rhinos\textunderscore  + \textunderscore makros\textunderscore )}
\end{itemize}
Gênero de insectos, semelhantes ao gorgulho.
\section{Rhinóphido}
\begin{itemize}
\item {Grp. gram.:adj.}
\end{itemize}
\begin{itemize}
\item {Utilização:Zool.}
\end{itemize}
\begin{itemize}
\item {Proveniência:(Do gr. \textunderscore rhis\textunderscore , \textunderscore rhinos\textunderscore  + \textunderscore ophis\textunderscore )}
\end{itemize}
Dizia-se das serpentes, cujo focinho se prolonga em fórma de tromba.
\section{Rhinophonia}
\begin{itemize}
\item {Grp. gram.:f.}
\end{itemize}
\begin{itemize}
\item {Proveniência:(Do gr. \textunderscore rhis\textunderscore , \textunderscore rhinos\textunderscore  + \textunderscore phone\textunderscore )}
\end{itemize}
Resonância da voz nas fossas nasaes.
\section{Rhinoplasta}
\begin{itemize}
\item {Grp. gram.:m.}
\end{itemize}
Aquelle que pratíca a rhinoplastia.
\section{Rhinoplastia}
\begin{itemize}
\item {Grp. gram.:f.}
\end{itemize}
\begin{itemize}
\item {Proveniência:(Do gr. \textunderscore rhis\textunderscore , \textunderscore rhinos\textunderscore  + \textunderscore plassein\textunderscore )}
\end{itemize}
Operação cirúrgica, com que se substitue artificialmente o nariz ou parte do nariz.
\section{Rhinoplástica}
\begin{itemize}
\item {Grp. gram.:f.}
\end{itemize}
(V.rhinoplastia)
\section{Rhinoplástico}
\begin{itemize}
\item {Grp. gram.:adj.}
\end{itemize}
Relativo á rhinoplastia.
\section{Rhinópomo}
\begin{itemize}
\item {Grp. gram.:m.}
\end{itemize}
\begin{itemize}
\item {Proveniência:(Do gr. \textunderscore rhis\textunderscore , \textunderscore rhinos\textunderscore  + \textunderscore poma\textunderscore )}
\end{itemize}
Gênero de mammíferos chirópteros, com as fossas nasaes providas de um lóbulo em fórma de opérculo.
\section{Rhinoptia}
\begin{itemize}
\item {Grp. gram.:f.}
\end{itemize}
\begin{itemize}
\item {Proveniência:(Do gr. \textunderscore rhis\textunderscore , \textunderscore rhinos\textunderscore  + \textunderscore optomai\textunderscore )}
\end{itemize}
Estrabismo, em que a pupilla, desviando-se do eixo visual, se aproxima do nariz.
\section{Rhinorrhagia}
\begin{itemize}
\item {Grp. gram.:f.}
\end{itemize}
\begin{itemize}
\item {Proveniência:(Do gr. \textunderscore rhis\textunderscore , \textunderscore rhinos\textunderscore  + \textunderscore rhein\textunderscore )}
\end{itemize}
Hemorrhagia nasal.
\section{Rhinorrhágico}
\begin{itemize}
\item {Grp. gram.:adj.}
\end{itemize}
Relativo á rhinorrhagia.
\section{Rhinorrhaphia}
\begin{itemize}
\item {Grp. gram.:f.}
\end{itemize}
\begin{itemize}
\item {Utilização:Med.}
\end{itemize}
\begin{itemize}
\item {Proveniência:(Do gr. \textunderscore rhis\textunderscore , \textunderscore rhinos\textunderscore  + \textunderscore graphein\textunderscore )}
\end{itemize}
Sutura dos bordos de uma chaga do nariz.
\section{Rhinorrheia}
\begin{itemize}
\item {Grp. gram.:f.}
\end{itemize}
\begin{itemize}
\item {Proveniência:(Do gr. \textunderscore rhis\textunderscore , \textunderscore rhinos\textunderscore  + \textunderscore rhein\textunderscore )}
\end{itemize}
Fluxo de mucosidades límpidas pelo nariz, sem symptomas de inflammação.
\section{Rhinoscopia}
\begin{itemize}
\item {Grp. gram.:f.}
\end{itemize}
\begin{itemize}
\item {Utilização:Med.}
\end{itemize}
Exame das fossas nasaes.
(Cp. \textunderscore rhinoscópio\textunderscore )
\section{Rhinoscópio}
\begin{itemize}
\item {Grp. gram.:m.}
\end{itemize}
\begin{itemize}
\item {Proveniência:(Do gr. \textunderscore rhis\textunderscore , \textunderscore rhinos\textunderscore  + \textunderscore skopein\textunderscore )}
\end{itemize}
Instrumento, para illuminar e deixar vêr as cavidades do nariz.
\section{Rhinostegnose}
\begin{itemize}
\item {Grp. gram.:f.}
\end{itemize}
\begin{itemize}
\item {Utilização:Med.}
\end{itemize}
\begin{itemize}
\item {Proveniência:(Do gr. \textunderscore rhis\textunderscore , \textunderscore rhinos\textunderscore  + \textunderscore stegnosis\textunderscore )}
\end{itemize}
Obstrucção das fossas nasaes.
\section{Rhinotheca}
\begin{itemize}
\item {Grp. gram.:f.}
\end{itemize}
\begin{itemize}
\item {Utilização:Zool.}
\end{itemize}
\begin{itemize}
\item {Proveniência:(Do gr. \textunderscore rhis\textunderscore , \textunderscore rhinos\textunderscore  + \textunderscore theke\textunderscore )}
\end{itemize}
Epiderme do bico das aves.
\section{Rhinotrichia}
\begin{itemize}
\item {fónica:qui}
\end{itemize}
\begin{itemize}
\item {Grp. gram.:f.}
\end{itemize}
\begin{itemize}
\item {Proveniência:(Do gr. \textunderscore rhis\textunderscore , \textunderscore rhinos\textunderscore  + \textunderscore trix\textunderscore , \textunderscore trikhos\textunderscore )}
\end{itemize}
Exuberância de pêlos no nariz.
\section{Rhipícera}
\begin{itemize}
\item {Grp. gram.:f.}
\end{itemize}
Gênero de insectos coleópteros pentâmeros.
\section{Rhipidólitho}
\begin{itemize}
\item {Grp. gram.:m.}
\end{itemize}
\begin{itemize}
\item {Utilização:Miner.}
\end{itemize}
\begin{itemize}
\item {Proveniência:(Do gr. \textunderscore rhipis\textunderscore , \textunderscore rhipidos\textunderscore  + \textunderscore lithos\textunderscore )}
\end{itemize}
Variedade de chlorito, pouco transparente.
\section{Rhipidura}
\begin{itemize}
\item {Grp. gram.:f.}
\end{itemize}
\begin{itemize}
\item {Proveniência:(Do gr. \textunderscore rhipis\textunderscore , \textunderscore rhipidos\textunderscore  + \textunderscore oura\textunderscore )}
\end{itemize}
Gênero de pássaros.
\section{Rhipiglossa}
\begin{itemize}
\item {Grp. gram.:f.}
\end{itemize}
\begin{itemize}
\item {Proveniência:(Do gr. \textunderscore rhipis\textunderscore  + \textunderscore glossa\textunderscore )}
\end{itemize}
Gênero de plantas acantháceas.
\section{Rhipíphoro}
\begin{itemize}
\item {Grp. gram.:m.}
\end{itemize}
\begin{itemize}
\item {Proveniência:(Do gr. \textunderscore rhipis\textunderscore  + \textunderscore phoros\textunderscore )}
\end{itemize}
Gênero de insectos coleópteros heterómeros.
\section{Rhipípteros}
\begin{itemize}
\item {Grp. gram.:m. pl.}
\end{itemize}
\begin{itemize}
\item {Proveniência:(Do gr. \textunderscore rhipis\textunderscore  + \textunderscore pteron\textunderscore )}
\end{itemize}
Ordem de insectos, no systema de Latreille.
\section{Rhizagra}
\begin{itemize}
\item {Grp. gram.:f.}
\end{itemize}
\begin{itemize}
\item {Proveniência:(Do gr. \textunderscore rhiza\textunderscore  + \textunderscore agra\textunderscore )}
\end{itemize}
Instrumento, próprio para extrahir as raízes dos dentes.
\section{Rhizântheas}
\begin{itemize}
\item {Grp. gram.:f. pl.}
\end{itemize}
\begin{itemize}
\item {Proveniência:(De \textunderscore rhizantho\textunderscore )}
\end{itemize}
O mesmo que \textunderscore cytíneas\textunderscore .
\section{Rhizantho}
\begin{itemize}
\item {Grp. gram.:adj.}
\end{itemize}
\begin{itemize}
\item {Proveniência:(Do gr. \textunderscore rhiza\textunderscore  + \textunderscore anthos\textunderscore )}
\end{itemize}
Diz-se das plantas, cujas flôres ou pedúnculos nascem da raíz.
\section{Rhizóbia}
\begin{itemize}
\item {Grp. gram.:f.}
\end{itemize}
\begin{itemize}
\item {Proveniência:(Do gr. \textunderscore rhiza\textunderscore  + \textunderscore bios\textunderscore )}
\end{itemize}
Gênero de insectos coleópteros pentâmeros.
\section{Rhizoblasto}
\begin{itemize}
\item {Grp. gram.:m.}
\end{itemize}
\begin{itemize}
\item {Utilização:Bot.}
\end{itemize}
\begin{itemize}
\item {Proveniência:(Do gr. \textunderscore rhiza\textunderscore  + \textunderscore blastos\textunderscore )}
\end{itemize}
Embryão, que tem uma só raíz.
\section{Rhizoboláceas}
\begin{itemize}
\item {Grp. gram.:f. pl.}
\end{itemize}
Família de plantas, que tem por typo o rhizóbolo.
(Fem. pl. de \textunderscore rhizoboláceo\textunderscore )
\section{Rhizoboláceo}
\begin{itemize}
\item {Grp. gram.:adj.}
\end{itemize}
Relativo ou semelhante ao rhizóbolo.
\section{Rhizóbolo}
\begin{itemize}
\item {Grp. gram.:m.}
\end{itemize}
\begin{itemize}
\item {Proveniência:(Do gr. \textunderscore rhiza\textunderscore  + \textunderscore bolos\textunderscore )}
\end{itemize}
Gênero de plantas da América do Sul.
\section{Rhizocárpico}
\begin{itemize}
\item {Grp. gram.:adj.}
\end{itemize}
Relativo ao vegetal rhizocárpio.
\section{Rhizocárpio}
\begin{itemize}
\item {Grp. gram.:adj.}
\end{itemize}
\begin{itemize}
\item {Proveniência:(Do gr. \textunderscore rhiza\textunderscore  + \textunderscore karpos\textunderscore )}
\end{itemize}
Diz-se dos vegetaes, de cuja raíz sáem em cada anno novas hastes fructíferas.
\section{Rhizocarpo}
\begin{itemize}
\item {Grp. gram.:adj.}
\end{itemize}
O mesmo que \textunderscore rhizocárpio\textunderscore .
\section{Rhizocéphalos}
\begin{itemize}
\item {Grp. gram.:m. pl.}
\end{itemize}
\begin{itemize}
\item {Utilização:Zool.}
\end{itemize}
\begin{itemize}
\item {Proveniência:(Do gr. \textunderscore rhiza\textunderscore  + \textunderscore kephale\textunderscore )}
\end{itemize}
Grupo de crustáceos, cuja cabeça emitte prolongamentos ocos.
\section{Rhizodo}
\begin{itemize}
\item {Grp. gram.:m.}
\end{itemize}
\begin{itemize}
\item {Proveniência:(Do gr. \textunderscore rhiza\textunderscore  + \textunderscore eidos\textunderscore )}
\end{itemize}
Gênero de insectos coleópteros pentâmeros.
\section{Rhizógono}
\begin{itemize}
\item {Grp. gram.:adj.}
\end{itemize}
\begin{itemize}
\item {Utilização:Bot.}
\end{itemize}
\begin{itemize}
\item {Proveniência:(Do gr. \textunderscore rhiza\textunderscore  + \textunderscore gonos\textunderscore )}
\end{itemize}
Diz-se da planta, cuja raíz tem órgãos reproductores.
\section{Rhizographia}
\begin{itemize}
\item {Grp. gram.:f.}
\end{itemize}
\begin{itemize}
\item {Proveniência:(Do gr. \textunderscore rhiza\textunderscore  + \textunderscore graphein\textunderscore )}
\end{itemize}
Descripção das raízes.
\section{Rhizográphico}
\begin{itemize}
\item {Grp. gram.:adj.}
\end{itemize}
Relativo á rhizographia.
\section{Rhizólitha}
\begin{itemize}
\item {Grp. gram.:f.}
\end{itemize}
\begin{itemize}
\item {Proveniência:(Do gr. \textunderscore rhiza\textunderscore  + \textunderscore lithos\textunderscore )}
\end{itemize}
Raíz fóssil.
\section{Rhizólitho}
\begin{itemize}
\item {Grp. gram.:m.}
\end{itemize}
O mesmo ou melhor que \textunderscore rhizólita\textunderscore .
\section{Rhizoma}
\begin{itemize}
\item {Grp. gram.:m.}
\end{itemize}
\begin{itemize}
\item {Utilização:Bot.}
\end{itemize}
\begin{itemize}
\item {Utilização:Pharm.}
\end{itemize}
\begin{itemize}
\item {Proveniência:(Do gr. \textunderscore rhiza\textunderscore )}
\end{itemize}
Espécie de haste subterrânea, ordinariamente horizontal.
Tintura de arnica. Cf. \textunderscore Regul. dos Preços dos Medic.\textunderscore 
\section{Rhizomatose}
\begin{itemize}
\item {Grp. gram.:f.}
\end{itemize}
\begin{itemize}
\item {Proveniência:(De \textunderscore rhizoma\textunderscore )}
\end{itemize}
Transformação de uma raíz em rhizoma.
\section{Rhizomatoso}
\begin{itemize}
\item {Grp. gram.:adj.}
\end{itemize}
Que tem rhizoma.
\section{Rhizomorpho}
\begin{itemize}
\item {Grp. gram.:adj.}
\end{itemize}
\begin{itemize}
\item {Proveniência:(Do gr. \textunderscore rhiza\textunderscore  + \textunderscore morphe\textunderscore )}
\end{itemize}
Que tem fórma de raíz.
\section{Rhizophagia}
\begin{itemize}
\item {Grp. gram.:f.}
\end{itemize}
Qualidade de rhizóphago.
\section{Rhizóphago}
\begin{itemize}
\item {Grp. gram.:adj.}
\end{itemize}
\begin{itemize}
\item {Proveniência:(Do gr. \textunderscore rhiza\textunderscore  + \textunderscore phagein\textunderscore )}
\end{itemize}
Que se alimenta de raízes.
\section{Rhizóphilo}
\begin{itemize}
\item {Grp. gram.:adj.}
\end{itemize}
\begin{itemize}
\item {Grp. gram.:M. pl.}
\end{itemize}
\begin{itemize}
\item {Proveniência:(Do gr. \textunderscore rhiza\textunderscore  + \textunderscore philos\textunderscore )}
\end{itemize}
Que vive nas raízes.
Gênero de cogumelos, que se desenvolvem nas raízes das plantas.
\section{Rhizóphora}
\begin{itemize}
\item {Grp. gram.:f.}
\end{itemize}
Gênero de plantas, o mesmo que \textunderscore rhizóphoro\textunderscore .
\section{Rhizophoráceas}
\begin{itemize}
\item {Grp. gram.:f. pl.}
\end{itemize}
Família de plantas, que tem por typo o rhizóphoro.
(Fem. pl. de \textunderscore rhizophoráceo\textunderscore )
\section{Rhizophoráceo}
\begin{itemize}
\item {Grp. gram.:adj.}
\end{itemize}
Relativo ou semelhante ao rhizóphoro.
\section{Rhizophóreas}
\begin{itemize}
\item {Grp. gram.:f. pl.}
\end{itemize}
V. \textunderscore rhizophoráceas\textunderscore .
\section{Rhizóphoro}
\begin{itemize}
\item {Grp. gram.:adj.}
\end{itemize}
\begin{itemize}
\item {Grp. gram.:M.}
\end{itemize}
\begin{itemize}
\item {Proveniência:(Do gr. \textunderscore rhiza\textunderscore  + \textunderscore phoros\textunderscore )}
\end{itemize}
Que tem raízes.
Gênero de plantas intertropicaes, cujo principal característico é terem as raízes banhadas pela água do mar.
\section{Rhizophyllo}
\begin{itemize}
\item {Grp. gram.:adj.}
\end{itemize}
\begin{itemize}
\item {Proveniência:(Do gr. \textunderscore rhiza\textunderscore  + \textunderscore phullon\textunderscore )}
\end{itemize}
Cujas fôlhas produzem raízes.
\section{Rhizóphyse}
\begin{itemize}
\item {Grp. gram.:f.}
\end{itemize}
\begin{itemize}
\item {Utilização:Bot.}
\end{itemize}
\begin{itemize}
\item {Proveniência:(Do gr. \textunderscore rhiza\textunderscore  + \textunderscore phusis\textunderscore )}
\end{itemize}
Appêndice, na extremidade de certas radículas.
\section{Rhizópode}
\begin{itemize}
\item {Grp. gram.:adj.}
\end{itemize}
\begin{itemize}
\item {Grp. gram.:M. pl.}
\end{itemize}
\begin{itemize}
\item {Proveniência:(Do gr. \textunderscore rhiza\textunderscore  + \textunderscore pous\textunderscore , \textunderscore podos\textunderscore )}
\end{itemize}
Cujos pés são semelhantes a raízes.
Animaes, cujos pés semelham raízes.
\section{Rhizopódio}
\begin{itemize}
\item {Grp. gram.:adj.}
\end{itemize}
O mesmo que \textunderscore rhizópode\textunderscore .
\section{Rhizospérmeas}
\begin{itemize}
\item {Grp. gram.:f. pl.}
\end{itemize}
\begin{itemize}
\item {Proveniência:(De \textunderscore rhizospermo\textunderscore )}
\end{itemize}
Família de plantas aquáticas.
\section{Rhizospermo}
\begin{itemize}
\item {Grp. gram.:adj.}
\end{itemize}
\begin{itemize}
\item {Proveniência:(Do gr. \textunderscore rhiza\textunderscore  + \textunderscore sperma\textunderscore )}
\end{itemize}
Diz-se dos vegetaes, cujas sementes nascem sôbre as raízes.
\section{Rhizóstoma}
\begin{itemize}
\item {Grp. gram.:f.}
\end{itemize}
\begin{itemize}
\item {Utilização:Zool.}
\end{itemize}
Gênero de acalephos medusários.
(Cp. \textunderscore rhizóstomo\textunderscore )
\section{Rhizóstomo}
\begin{itemize}
\item {Grp. gram.:adj.}
\end{itemize}
\begin{itemize}
\item {Utilização:Zool.}
\end{itemize}
\begin{itemize}
\item {Proveniência:(Do gr. \textunderscore rhiza\textunderscore  + \textunderscore stoma\textunderscore )}
\end{itemize}
Que tem muitas bocas na extremidade de filamentos semelhantes a raízes, (falando-se de certos animaes).
\section{Rhizotaxia}
\begin{itemize}
\item {fónica:csi}
\end{itemize}
\begin{itemize}
\item {Grp. gram.:f.}
\end{itemize}
\begin{itemize}
\item {Utilização:Bot.}
\end{itemize}
\begin{itemize}
\item {Proveniência:(Do gr. \textunderscore rhiza\textunderscore  + \textunderscore taxis\textunderscore )}
\end{itemize}
Disposição das radicellas sôbre a raiz da planta.
\section{Rhizotomia}
\begin{itemize}
\item {Grp. gram.:f.}
\end{itemize}
\begin{itemize}
\item {Proveniência:(Do gr. \textunderscore rhiza\textunderscore  + \textunderscore tome\textunderscore )}
\end{itemize}
Córte de raízes.
\section{Rhizótomo}
\begin{itemize}
\item {Grp. gram.:m.}
\end{itemize}
\begin{itemize}
\item {Proveniência:(Do gr. \textunderscore rhiza\textunderscore  + \textunderscore tome\textunderscore )}
\end{itemize}
Instrumento, para cortar raízes.
\section{Rhizotónico}
\begin{itemize}
\item {Grp. gram.:adj.}
\end{itemize}
\begin{itemize}
\item {Utilização:Philol.}
\end{itemize}
\begin{itemize}
\item {Proveniência:(Do gr. \textunderscore rhiza\textunderscore  + \textunderscore tonos\textunderscore )}
\end{itemize}
Diz-se das fórmas verbaes que, em português, têm, como sýllaba tónica ou dominante, a última do radical, como em \textunderscore louva\textunderscore , \textunderscore copía\textunderscore .
\section{Rhízula}
\begin{itemize}
\item {Grp. gram.:f.}
\end{itemize}
\begin{itemize}
\item {Utilização:Bot.}
\end{itemize}
\begin{itemize}
\item {Proveniência:(Do gr. \textunderscore rhiza\textunderscore )}
\end{itemize}
Cada uma das radículas dos cogumelos.
\section{Rhodalose}
\begin{itemize}
\item {Grp. gram.:f.}
\end{itemize}
Espécie de crystal.
\section{Rhodâmnia}
\begin{itemize}
\item {Grp. gram.:f.}
\end{itemize}
\begin{itemize}
\item {Proveniência:(Do gr. \textunderscore rhodon\textunderscore  + \textunderscore amnos\textunderscore )}
\end{itemize}
Arbusto myrtáceo de Samatra.
\section{Rhodanito}
\begin{itemize}
\item {Grp. gram.:m.}
\end{itemize}
\begin{itemize}
\item {Utilização:Miner.}
\end{itemize}
Silicato de manganés.
\section{Rhodato}
\begin{itemize}
\item {Grp. gram.:m.}
\end{itemize}
\begin{itemize}
\item {Utilização:Chím.}
\end{itemize}
Gênero de saes, produzidos pelo óxydo rhódico.
\section{Rhódico}
\begin{itemize}
\item {Grp. gram.:adj.}
\end{itemize}
\begin{itemize}
\item {Proveniência:(De \textunderscore rhódio\textunderscore ^1)}
\end{itemize}
Diz-se de um dos óxydos do rhódio.
\section{Rhódio}
\begin{itemize}
\item {Grp. gram.:m.}
\end{itemize}
\begin{itemize}
\item {Proveniência:(Do gr. \textunderscore rhodon\textunderscore )}
\end{itemize}
Metal pouco fusível, que se descobriu na platina do commércio.
\section{Rhódio}
\begin{itemize}
\item {Grp. gram.:adj.}
\end{itemize}
\begin{itemize}
\item {Proveniência:(Lat. \textunderscore rhodius\textunderscore )}
\end{itemize}
Relativo a Rhodes.
Diz-se do estilo moderado, que teve origem na ilha de Rhodes.
\section{Rhodiota}
\begin{itemize}
\item {Grp. gram.:m.  e  adj.}
\end{itemize}
\begin{itemize}
\item {Proveniência:(De \textunderscore Rhodes\textunderscore , n. p.)}
\end{itemize}
Habitante de Rhodes; rhódio.
\section{Rhodite}
\begin{itemize}
\item {Grp. gram.:f.}
\end{itemize}
\begin{itemize}
\item {Utilização:Miner.}
\end{itemize}
\begin{itemize}
\item {Proveniência:(Lat. \textunderscore rhoditis\textunderscore )}
\end{itemize}
Variedade de pedra, com a côr e a fórma da rosa.
\section{Rhodócera}
\begin{itemize}
\item {Grp. gram.:f.}
\end{itemize}
\begin{itemize}
\item {Proveniência:(Do gr. \textunderscore rhodon\textunderscore  + \textunderscore keras\textunderscore )}
\end{itemize}
Gênero de insectos lepidópteros diurnos.
\section{Rhodochlorito}
\begin{itemize}
\item {Grp. gram.:m.}
\end{itemize}
\begin{itemize}
\item {Utilização:Miner.}
\end{itemize}
Carbonato de manganés.
\section{Rhodochromatito}
\begin{itemize}
\item {Grp. gram.:m. f.}
\end{itemize}
\begin{itemize}
\item {Utilização:Geol.}
\end{itemize}
Uma das variedades rosadas do chlorito.
\section{Rhodochrosito}
\begin{itemize}
\item {Grp. gram.:m.}
\end{itemize}
\begin{itemize}
\item {Utilização:Miner.}
\end{itemize}
\begin{itemize}
\item {Proveniência:(Do gr. \textunderscore rhodon\textunderscore  + \textunderscore khrosis\textunderscore )}
\end{itemize}
O mesmo que \textunderscore dialogito\textunderscore .
\section{Rhododáctylo}
\begin{itemize}
\item {Grp. gram.:adj.}
\end{itemize}
\begin{itemize}
\item {Utilização:Zool.}
\end{itemize}
\begin{itemize}
\item {Proveniência:(Do gr. \textunderscore rhodon\textunderscore  + \textunderscore daktulos\textunderscore )}
\end{itemize}
Diz-se do insecto que tem asas digitaes e da côr da rosa.
\section{Rhododendráceas}
\begin{itemize}
\item {Grp. gram.:f. pl.}
\end{itemize}
Família de plantas, que tem por typo o rhododendro.
(Dem. pl. de \textunderscore rhododendráceo\textunderscore )
\section{Rhododendráceo}
\begin{itemize}
\item {Grp. gram.:adj.}
\end{itemize}
Relativo ou semelhante ao rhododendro.
\section{Rhododendro}
\begin{itemize}
\item {Grp. gram.:m.}
\end{itemize}
\begin{itemize}
\item {Proveniência:(Lat. \textunderscore rhodo\textunderscore  + \textunderscore dendron\textunderscore )}
\end{itemize}
Gênero de arbustos e árvores, entre as quaes sobresái uma de grande e formosas flôres.
Loendro; cevadilha.
\section{Rhodogastro}
\begin{itemize}
\item {Grp. gram.:adj.}
\end{itemize}
\begin{itemize}
\item {Utilização:Zool.}
\end{itemize}
\begin{itemize}
\item {Proveniência:(Do gr. \textunderscore rhodon\textunderscore  + \textunderscore gaster\textunderscore )}
\end{itemize}
Diz-se do insecto, que tem ventre vermelho.
\section{Rhodographia}
\begin{itemize}
\item {Grp. gram.:f.}
\end{itemize}
\begin{itemize}
\item {Proveniência:(Do gr. \textunderscore rhodon\textunderscore  + \textunderscore graphein\textunderscore )}
\end{itemize}
Descripção das rosas.
\section{Rhodográphico}
\begin{itemize}
\item {Grp. gram.:adj.}
\end{itemize}
Relativo á rodographia.
\section{Rhodolena}
\begin{itemize}
\item {Grp. gram.:f.}
\end{itemize}
\begin{itemize}
\item {Proveniência:(Do gr. \textunderscore rhodon\textunderscore  + \textunderscore laina\textunderscore )}
\end{itemize}
Gênero de plantas de Madagáscar.
\section{Rhodólita}
\begin{itemize}
\item {Grp. gram.:m.}
\end{itemize}
\begin{itemize}
\item {Utilização:Miner.}
\end{itemize}
\begin{itemize}
\item {Proveniência:(Do gr. \textunderscore rhodon\textunderscore  + \textunderscore lithos\textunderscore )}
\end{itemize}
Silicato de manganés, da côr da rosa.
\section{Rhodologia}
\begin{itemize}
\item {Grp. gram.:f.}
\end{itemize}
\begin{itemize}
\item {Proveniência:(Do gr. \textunderscore rhodon\textunderscore  + \textunderscore logos\textunderscore )}
\end{itemize}
Parte da Botânica, que trata das rosas.
\section{Rhodológico}
\begin{itemize}
\item {Grp. gram.:adj.}
\end{itemize}
Relativo á rhodologia.
\section{Rhodomel}
\begin{itemize}
\item {Grp. gram.:m.}
\end{itemize}
\begin{itemize}
\item {Proveniência:(Do gr. \textunderscore rhodon\textunderscore  + \textunderscore meli\textunderscore )}
\end{itemize}
Mel rosado.
\section{Rhodómela}
\begin{itemize}
\item {Grp. gram.:f.}
\end{itemize}
\begin{itemize}
\item {Proveniência:(Do gr. \textunderscore rhodon\textunderscore  + \textunderscore melas\textunderscore )}
\end{itemize}
Gênero de algas roxo-escuras.
\section{Rhodomeláceas}
\begin{itemize}
\item {Grp. gram.:f. pl.}
\end{itemize}
\begin{itemize}
\item {Utilização:Bot.}
\end{itemize}
\begin{itemize}
\item {Proveniência:(De \textunderscore rhodómela\textunderscore )}
\end{itemize}
Famílias de algas.
\section{Rhodoméleas}
\begin{itemize}
\item {Grp. gram.:f. pl.}
\end{itemize}
Tríbo de plantas phýceas, que tem por typo a rhodómela.
\section{Rhodonito}
\begin{itemize}
\item {Grp. gram.:f.}
\end{itemize}
\begin{itemize}
\item {Proveniência:(Do gr. \textunderscore rhodon\textunderscore )}
\end{itemize}
Silicato de manganés, chrystalino e côr de rosa.
\section{Rhodopeu}
\begin{itemize}
\item {Grp. gram.:adj.}
\end{itemize}
\begin{itemize}
\item {Proveniência:(Lat. \textunderscore rhodopeus\textunderscore )}
\end{itemize}
Relativo ao monte \textunderscore Rhódope\textunderscore :«\textunderscore rhodopeos alcantis...\textunderscore »Castilho, \textunderscore Geórg.\textunderscore 
\section{Rhodóptero}
\begin{itemize}
\item {Grp. gram.:adj.}
\end{itemize}
\begin{itemize}
\item {Utilização:Zool.}
\end{itemize}
\begin{itemize}
\item {Proveniência:(Do gr. \textunderscore rhodon\textunderscore  + \textunderscore pteron\textunderscore )}
\end{itemize}
Diz-se do insecto, que tem asas rosadas.
\section{Rhódora}
\begin{itemize}
\item {Grp. gram.:f.}
\end{itemize}
\begin{itemize}
\item {Proveniência:(Lat. \textunderscore rhodora\textunderscore )}
\end{itemize}
O mesmo que \textunderscore rhododendro\textunderscore .
\section{Rhodoráceas}
\begin{itemize}
\item {Grp. gram.:f. pl.}
\end{itemize}
Famílias de plantas, pouco differente das ericáceas, e á qual pertence o gênero azálea, segundo o \textunderscore Thes. da Líng. Port.\textunderscore --A azálea porém, segundo o commum dos botânicos, pertence ás ericáceas.
O mesmo que \textunderscore rhododendráceas\textunderscore .
\section{Rhodospermo}
\begin{itemize}
\item {Grp. gram.:adj.}
\end{itemize}
\begin{itemize}
\item {Utilização:Bot.}
\end{itemize}
\begin{itemize}
\item {Proveniência:(Do gr. \textunderscore rhodon\textunderscore  + \textunderscore sperma\textunderscore )}
\end{itemize}
Que tem sementes rosadas.
\section{Rhodóstomo}
\begin{itemize}
\item {Grp. gram.:adj.}
\end{itemize}
\begin{itemize}
\item {Utilização:Zool.}
\end{itemize}
\begin{itemize}
\item {Proveniência:(Do gr. \textunderscore rhodon\textunderscore  + \textunderscore stoma\textunderscore )}
\end{itemize}
Que tem a boca rosada.
\section{Rhoicisso}
\begin{itemize}
\item {Grp. gram.:f.}
\end{itemize}
\begin{itemize}
\item {Proveniência:(Do lat. \textunderscore rhois\textunderscore  + \textunderscore cissus\textunderscore )}
\end{itemize}
Gênero de videiras, da fam. das ampelídeas.
\section{Rhômbico}
\begin{itemize}
\item {Grp. gram.:adj.}
\end{itemize}
Que tem fórma de rhombo.
\section{Rhombífero}
\begin{itemize}
\item {Grp. gram.:adj.}
\end{itemize}
\begin{itemize}
\item {Utilização:Miner.}
\end{itemize}
\begin{itemize}
\item {Proveniência:(Do lat. \textunderscore rhombus\textunderscore  + \textunderscore ferre\textunderscore )}
\end{itemize}
Diz-se de um crystal, cujas facêtas são rhombas.
\section{Rhombifoliado}
\begin{itemize}
\item {Grp. gram.:adj.}
\end{itemize}
O mesmo que\textunderscore rhombifólio\textunderscore .
\section{Rhombifólio}
\begin{itemize}
\item {Grp. gram.:adj.}
\end{itemize}
\begin{itemize}
\item {Utilização:Bot.}
\end{itemize}
\begin{itemize}
\item {Proveniência:(Do lat. \textunderscore rhombus\textunderscore  + \textunderscore folium\textunderscore )}
\end{itemize}
Que tem fôlhas em fórma de rhombo.
\section{Rhombiforme}
\begin{itemize}
\item {Grp. gram.:adj.}
\end{itemize}
O mesmo que \textunderscore rhômbico\textunderscore .
\section{Rhombo}
\begin{itemize}
\item {Grp. gram.:m.}
\end{itemize}
\begin{itemize}
\item {Proveniência:(Lat. \textunderscore rhombus\textunderscore )}
\end{itemize}
Quadrilátero ou losango, de lados todos iguaes, sem que os ângulos sejam rectos.
\section{Rhombo...}
\begin{itemize}
\item {Grp. gram.:pref.}
\end{itemize}
\begin{itemize}
\item {Proveniência:(Lat. \textunderscore rhombus\textunderscore )}
\end{itemize}
(designativo de \textunderscore losango\textunderscore )
\section{Rhombocéphalo}
\begin{itemize}
\item {Grp. gram.:m.}
\end{itemize}
\begin{itemize}
\item {Utilização:Zool.}
\end{itemize}
\begin{itemize}
\item {Proveniência:(Do gr. \textunderscore rhombos\textunderscore  + \textunderscore kephale\textunderscore )}
\end{itemize}
Gênero de myriápodes.
\section{Rhombódera}
\begin{itemize}
\item {Grp. gram.:f.}
\end{itemize}
Gênero de insectos coleópteros pentâmeros.
\section{Rhomboédrico}
Que tem fórma de rhomboédro.
\section{Rhomboédro}
\begin{itemize}
\item {Grp. gram.:m.}
\end{itemize}
\begin{itemize}
\item {Proveniência:(Do gr. \textunderscore rhombos\textunderscore  + \textunderscore edra\textunderscore )}
\end{itemize}
Sólido geométrico, cujas faces são rhombiformes.
\section{Rhomboglosso}
\begin{itemize}
\item {Grp. gram.:m.}
\end{itemize}
\begin{itemize}
\item {Proveniência:(Do gr. \textunderscore rhombos\textunderscore  + \textunderscore glossa\textunderscore )}
\end{itemize}
Gênero de batrácios.
\section{Rhomboidal}
\begin{itemize}
\item {Grp. gram.:adj.}
\end{itemize}
\begin{itemize}
\item {Grp. gram.:M.  e  adj.}
\end{itemize}
Que tem a figura de rhomboide.
Diz-se de um músculo da região dorsal.
\section{Rhombóide}
\begin{itemize}
\item {Grp. gram.:m.}
\end{itemize}
\begin{itemize}
\item {Proveniência:(Lat. \textunderscore rhomboides\textunderscore )}
\end{itemize}
Figura de quatro lados, que não tem rectos os ângulos, mas iguaes os lados oppostos, e desiguaes os contíguos; parallelogrammo.
\section{Rhombopalpo}
\begin{itemize}
\item {Grp. gram.:m.}
\end{itemize}
\begin{itemize}
\item {Proveniência:(Do lat. \textunderscore rhombus\textunderscore  + \textunderscore palpus\textunderscore )}
\end{itemize}
Gênero de insectos coleópteros.
\section{Rhomborina}
\begin{itemize}
\item {Grp. gram.:f.}
\end{itemize}
Gênero de insectos coleópteros pentâmeros.
\section{Rhombósporo}
\begin{itemize}
\item {Grp. gram.:adj.}
\end{itemize}
\begin{itemize}
\item {Utilização:Bot.}
\end{itemize}
\begin{itemize}
\item {Proveniência:(Do gr. \textunderscore rhombos\textunderscore  + \textunderscore spora\textunderscore )}
\end{itemize}
Que tem sementes rhomboidaes.
\section{Rhonco}
\begin{itemize}
\item {Grp. gram.:m.}
\end{itemize}
(V. \textunderscore ronco\textunderscore ^1)
\section{Rhopalócero}
\begin{itemize}
\item {Grp. gram.:adj.}
\end{itemize}
\begin{itemize}
\item {Proveniência:(Do gr. \textunderscore rhopalon\textunderscore  + \textunderscore keras\textunderscore )}
\end{itemize}
Diz-se dos insectos, que tem as antennas terminadas em maça, ou botão.
\section{Rhopalose}
\begin{itemize}
\item {Grp. gram.:f.}
\end{itemize}
\begin{itemize}
\item {Utilização:Med.}
\end{itemize}
\begin{itemize}
\item {Proveniência:(Do gr. \textunderscore rhopalon\textunderscore )}
\end{itemize}
Moléstia, em que engrossa a extremidade dos cabellos.
\section{Rhopographia}
\begin{itemize}
\item {Grp. gram.:f.}
\end{itemize}
\begin{itemize}
\item {Proveniência:(Gr. \textunderscore rhopographia\textunderscore )}
\end{itemize}
Descripção de arvoredos ou de pequenas paisagens.
\section{Rhopográphico}
\begin{itemize}
\item {Grp. gram.:adj.}
\end{itemize}
Relativo á ropographia.
\section{Rhopógrapho}
\begin{itemize}
\item {Grp. gram.:m.}
\end{itemize}
\begin{itemize}
\item {Proveniência:(Do gr. \textunderscore rhopos\textunderscore  + \textunderscore graphein\textunderscore )}
\end{itemize}
Aquelle que descreve arvoredos ou pequenas paisagens.
\section{Rhotacismo}
\begin{itemize}
\item {Grp. gram.:m.}
\end{itemize}
\begin{itemize}
\item {Proveniência:(Do gr. \textunderscore rhotakizein\textunderscore )}
\end{itemize}
Pronúncia viciosa da letra \textunderscore r\textunderscore .
\section{Rhuibarbo}
\begin{itemize}
\item {Grp. gram.:m.}
\end{itemize}
\begin{itemize}
\item {Proveniência:(Do lat. \textunderscore Rha\textunderscore  n. p. + \textunderscore barbarus\textunderscore )}
\end{itemize}
Gênero de plantas polygóneas.
Designação collectiva das raízes medicinaes, pertencentes ás polygóneas.
O mesmo que \textunderscore rapôncio\textunderscore .
\section{Rhum}
\begin{itemize}
\item {Grp. gram.:m.}
\end{itemize}
(V.rum)
\section{Rhynchanthera}
\begin{itemize}
\item {fónica:can}
\end{itemize}
\begin{itemize}
\item {Grp. gram.:f.}
\end{itemize}
Gênero de plantas melastomáceas.
\section{Rhynchobdela}
\begin{itemize}
\item {fónica:co}
\end{itemize}
\begin{itemize}
\item {Grp. gram.:f.}
\end{itemize}
Gênero de peixes acanthopterýgios.
\section{Rhynchocarpo}
\begin{itemize}
\item {fónica:co}
\end{itemize}
\begin{itemize}
\item {Grp. gram.:f.}
\end{itemize}
\begin{itemize}
\item {Proveniência:(Do gr. \textunderscore rhunkhos\textunderscore  + \textunderscore karpos\textunderscore )}
\end{itemize}
Gênero de plantas cucurbitáceas.
\section{Rhynchocéleos}
\begin{itemize}
\item {fónica:co}
\end{itemize}
\begin{itemize}
\item {Grp. gram.:m. pl.}
\end{itemize}
\begin{itemize}
\item {Utilização:Zool.}
\end{itemize}
\begin{itemize}
\item {Proveniência:(Do gr. \textunderscore rhunkhos\textunderscore  + \textunderscore koilon\textunderscore )}
\end{itemize}
Ordem de vermes.
\section{Rhynchocéphalo}
\begin{itemize}
\item {fónica:co}
\end{itemize}
\begin{itemize}
\item {Grp. gram.:adj.}
\end{itemize}
\begin{itemize}
\item {Utilização:Zool.}
\end{itemize}
\begin{itemize}
\item {Proveniência:(Do gr. \textunderscore rhunkhos\textunderscore  + \textunderscore kephale\textunderscore )}
\end{itemize}
Que tem cabeça prolongado em fórma de bico.
\section{Rhýncodo}
\begin{itemize}
\item {Grp. gram.:m.}
\end{itemize}
\begin{itemize}
\item {Proveniência:(Do gr. \textunderscore rhunkhos\textunderscore  + \textunderscore eidos\textunderscore )}
\end{itemize}
Gênero de insectos coleópteros tetrâmeros.
\section{Rhynchoglosso}
\begin{itemize}
\item {fónica:co}
\end{itemize}
\begin{itemize}
\item {Grp. gram.:m.}
\end{itemize}
\begin{itemize}
\item {Proveniência:(Do gr. \textunderscore rhunkhos\textunderscore  + \textunderscore glossa\textunderscore )}
\end{itemize}
Gênero de plantas escrofularíneas.
\section{Rhynchóphoro}
\begin{itemize}
\item {fónica:có}
\end{itemize}
\begin{itemize}
\item {Grp. gram.:adj.}
\end{itemize}
\begin{itemize}
\item {Utilização:Zool.}
\end{itemize}
\begin{itemize}
\item {Grp. gram.:M. pl.}
\end{itemize}
\begin{itemize}
\item {Proveniência:(Do gr. \textunderscore rhunkhos\textunderscore  + \textunderscore phoros\textunderscore )}
\end{itemize}
Que tem bico, ou um bico grande.
Insectos rhynchópteros.
\section{Rhynchósia}
\begin{itemize}
\item {fónica:có}
\end{itemize}
\begin{itemize}
\item {Grp. gram.:f.}
\end{itemize}
Gênero de plantas leguminosas.
\section{Rhynchospermo}
\begin{itemize}
\item {fónica:cós}
\end{itemize}
\begin{itemize}
\item {Grp. gram.:m.}
\end{itemize}
Gênero de plantas leguminosas.
\section{Rhynchóspora}
\begin{itemize}
\item {fónica:cós}
\end{itemize}
\begin{itemize}
\item {Grp. gram.:f.}
\end{itemize}
\begin{itemize}
\item {Proveniência:(Do gr. \textunderscore rhunkhos\textunderscore  + \textunderscore spora\textunderscore )}
\end{itemize}
Gênero de plantas ciperáceas.
\section{Rhynchotheca}
\begin{itemize}
\item {Grp. gram.:f.}
\end{itemize}
\begin{itemize}
\item {Proveniência:(Do gr. \textunderscore rhunkhos\textunderscore  + \textunderscore theke\textunderscore )}
\end{itemize}
Gênero de plantas do Peru.
\section{Rhythelminto}
\begin{itemize}
\item {Grp. gram.:m.}
\end{itemize}
\begin{itemize}
\item {Proveniência:(Do gr. \textunderscore rhutis\textunderscore  + \textunderscore kelmins\textunderscore )}
\end{itemize}
Gênero de vermes intestinaes.
\section{Rhytmado}
\begin{itemize}
\item {Grp. gram.:adj.}
\end{itemize}
Que tem rhythmo.
\section{Rhýthmica}
\begin{itemize}
\item {Grp. gram.:f.}
\end{itemize}
Parte da antiga grammática, que se occupava do rhythmo dos versos gregos e latinos.
(Fem. de \textunderscore rhýtmico\textunderscore )
\section{Rhýtmico}
\begin{itemize}
\item {Grp. gram.:adj.}
\end{itemize}
\begin{itemize}
\item {Proveniência:(Lat. \textunderscore rythmicus\textunderscore )}
\end{itemize}
Relativo ao rhythmo.
\section{Rhythmo}
\begin{itemize}
\item {Grp. gram.:m.}
\end{itemize}
\begin{itemize}
\item {Utilização:Med.}
\end{itemize}
\begin{itemize}
\item {Proveniência:(Lat. \textunderscore rhytmus\textunderscore )}
\end{itemize}
Successão, com intervallos regulares, de sýllabas accentuadas ou accentos prosódicos, que impressiona agradavelmente o ouvido.
Cadência.
Proporção entre as pulsações. das artérias.
\section{Rhythmómetro}
\begin{itemize}
\item {Grp. gram.:m.}
\end{itemize}
\begin{itemize}
\item {Proveniência:(Do gr. \textunderscore rhuthmos\textunderscore  + \textunderscore metron\textunderscore )}
\end{itemize}
Antigo instrumento mecânico, com que se indicava o compasso da música e que foi substituído pelo metrónomo.
\section{Rhýtio}
\begin{itemize}
\item {Grp. gram.:m.}
\end{itemize}
\begin{itemize}
\item {Proveniência:(Lat. \textunderscore rhytion\textunderscore )}
\end{itemize}
Vaso antigo, em fórma de buzina, usado pelos Gregos.
\section{Rhythmopeia}
\begin{itemize}
\item {Grp. gram.:f.}
\end{itemize}
\begin{itemize}
\item {Proveniência:(Gr. \textunderscore rhutmopoia\textunderscore )}
\end{itemize}
Arte do rythmo.
\section{Ria}
\begin{itemize}
\item {Grp. gram.:f.}
\end{itemize}
\begin{itemize}
\item {Proveniência:(De \textunderscore rio\textunderscore )}
\end{itemize}
Esteiro ou braço de rio, que se presta geralmente á navegação.
\section{Riachão}
\begin{itemize}
\item {Grp. gram.:m.}
\end{itemize}
\begin{itemize}
\item {Utilização:Bras}
\end{itemize}
Riacho grande.
\section{Riacho}
\begin{itemize}
\item {Grp. gram.:m.}
\end{itemize}
Rio pequeno; ribeiro.
\section{Rial}
\begin{itemize}
\item {Grp. gram.:adj.}
\end{itemize}
\begin{itemize}
\item {Grp. gram.:M.}
\end{itemize}
\begin{itemize}
\item {Proveniência:(Do lat. \textunderscore regalis\textunderscore )}
\end{itemize}
Relativo ao rei; digno ou próprio de rei; magnificente.
Diz-se da maior ou melhor coisa de uma série ou grupo: \textunderscore uma descompostura rial\textunderscore ; \textunderscore um tiro rial\textunderscore , etc.
Antiga moéda portuguesa, de valor differente, segundo as épocas.
Unidade monetária portuguesa, (pl. \textunderscore reis\textunderscore ).
\section{Rialeza}
\begin{itemize}
\item {Grp. gram.:f.}
\end{itemize}
\begin{itemize}
\item {Utilização:Fig.}
\end{itemize}
\begin{itemize}
\item {Proveniência:(De \textunderscore real\textunderscore ^2)}
\end{itemize}
Dignidade de rei ou rainha.
Grandeza, esplendor.
\section{Riamba}
\begin{itemize}
\item {Grp. gram.:f.}
\end{itemize}
\begin{itemize}
\item {Utilização:Bras}
\end{itemize}
Erva, o mesmo que \textunderscore pango\textunderscore .
Designação angolense do cânhamo.
\section{Riana}
\begin{itemize}
\item {Grp. gram.:f.}
\end{itemize}
\begin{itemize}
\item {Utilização:Bras}
\end{itemize}
Espécie de forragem.
\section{Riba}
\begin{itemize}
\item {Grp. gram.:f.}
\end{itemize}
\begin{itemize}
\item {Utilização:Pop.}
\end{itemize}
\begin{itemize}
\item {Utilização:Bras. do Rio e S. Paulo}
\end{itemize}
\begin{itemize}
\item {Grp. gram.:Loc.}
\end{itemize}
\begin{itemize}
\item {Utilização:pop.}
\end{itemize}
\begin{itemize}
\item {Proveniência:(Do lat. \textunderscore ripa\textunderscore )}
\end{itemize}
Margem elevada do rio.
Ribanceira.
Collina, sobranceira a um rio.
Cóima.
Espécie de galga, com que se descasca o café, e que é posta em movimento por um animal.
\textunderscore Em riba\textunderscore , em cima.
Além disso: \textunderscore negou-me a divida, e ainda em riba me maltratou\textunderscore .
\section{Ribada}
\begin{itemize}
\item {Grp. gram.:f.}
\end{itemize}
\begin{itemize}
\item {Proveniência:(De \textunderscore riba\textunderscore )}
\end{itemize}
Riba prolongada:«\textunderscore ...encontrou uma cova na ribada do mar.\textunderscore »\textunderscore Jornada da África\textunderscore , X.
\section{Ribaldar}
\begin{itemize}
\item {Grp. gram.:v. i.}
\end{itemize}
\begin{itemize}
\item {Utilização:T. de Turquel}
\end{itemize}
Vadiar.
Mandriar.
\section{Ribaldaria}
\begin{itemize}
\item {Grp. gram.:f.}
\end{itemize}
\begin{itemize}
\item {Utilização:Pop.}
\end{itemize}
Dito ou acto próprio de ribaldo; carácter do que é ribaldo.
\section{Ribaldeira}
\begin{itemize}
\item {Grp. gram.:f.}
\end{itemize}
\begin{itemize}
\item {Utilização:T. de Turquel}
\end{itemize}
O mesmo que \textunderscore ribaldaria\textunderscore .
\section{Ribaldeiro}
\begin{itemize}
\item {Grp. gram.:adj.}
\end{itemize}
O mesmo que \textunderscore ribaldo\textunderscore .
\section{Ribaldia}
\begin{itemize}
\item {Grp. gram.:f.}
\end{itemize}
O mesmo que \textunderscore ribaldaria\textunderscore .
\section{Ribaldio}
\begin{itemize}
\item {Grp. gram.:m.  e  adj.}
\end{itemize}
\begin{itemize}
\item {Proveniência:(De \textunderscore riba\textunderscore )}
\end{itemize}
Diz-se de uma espécie de figo bravo.
\section{Ribaldo}
\begin{itemize}
\item {Grp. gram.:m.  e  adj.}
\end{itemize}
Patife; biltre; velhaco.
(Talvez por \textunderscore revaldo\textunderscore  = \textunderscore re...\textunderscore  + \textunderscore valdo\textunderscore )
\section{Ribalta}
\begin{itemize}
\item {Grp. gram.:f.}
\end{itemize}
\begin{itemize}
\item {Proveniência:(De \textunderscore riba\textunderscore  + \textunderscore alto\textunderscore )}
\end{itemize}
Série de luzes, á frente do palco, entre o pano de bôca e o lugar da orchestra. Cf. Camillo, \textunderscore Cancion. Al.\textunderscore , 339.
\section{Ribamar}
\begin{itemize}
\item {Grp. gram.:f.}
\end{itemize}
\begin{itemize}
\item {Proveniência:(De \textunderscore riba\textunderscore  + \textunderscore mar\textunderscore )}
\end{itemize}
Beira do mar; terreno á borda do mar.
\section{Ribana}
\begin{itemize}
\item {Grp. gram.:f.}
\end{itemize}
O mesmo que \textunderscore arribana\textunderscore . Cf. Camillo, \textunderscore Narcóticos\textunderscore , II, 124.
\section{Ribança}
\begin{itemize}
\item {Grp. gram.:f.}
\end{itemize}
\begin{itemize}
\item {Utilização:Des.}
\end{itemize}
\begin{itemize}
\item {Proveniência:(De \textunderscore riba\textunderscore )}
\end{itemize}
Margem do rio, talhada a pique.
\section{Ribanceira}
\begin{itemize}
\item {Grp. gram.:f.}
\end{itemize}
\begin{itemize}
\item {Proveniência:(De \textunderscore ribança\textunderscore )}
\end{itemize}
Penedia sobranceira a um rio.
Margem elevada de um rio; riba.
\section{Ribar}
\begin{itemize}
\item {Grp. gram.:v. i.}
\end{itemize}
\begin{itemize}
\item {Utilização:Obsol.}
\end{itemize}
O mesmo que \textunderscore derribar\textunderscore .
\section{Ribatejano}
\begin{itemize}
\item {Grp. gram.:m.  e  adj.}
\end{itemize}
\begin{itemize}
\item {Proveniência:(De \textunderscore riba\textunderscore  + \textunderscore Tejo\textunderscore , n. p.)}
\end{itemize}
O que vive no Riba-Tejo.
Relativo ao Riba-Tejo.
\section{Ribatejo}
\begin{itemize}
\item {Grp. gram.:f.}
\end{itemize}
Variedade de pêra, originária de Coruche.
\section{Ribeira}
\begin{itemize}
\item {Grp. gram.:f.}
\end{itemize}
\begin{itemize}
\item {Utilização:Bras. do N}
\end{itemize}
\begin{itemize}
\item {Proveniência:(Do lat. \textunderscore riparia\textunderscore )}
\end{itemize}
Porção de terreno, banhado por um rio.
Regada.
Terra marginal.
Lugar, junto ao rio; riba.
Pequeno rio; ribeiro.
Árvore de San-Thomé.
Variedade de pêra.
Districto rural, que comprehende certo número de fazendas para criar gado, as quaes se distinguem pelo nome dos rios que as banham.
\section{Ribeirada}
\begin{itemize}
\item {Grp. gram.:f.}
\end{itemize}
\begin{itemize}
\item {Utilização:Fig.}
\end{itemize}
\begin{itemize}
\item {Proveniência:(De \textunderscore ribeira\textunderscore )}
\end{itemize}
Corrente impetuosa de ribeiro.
Grande porção de líquido, escorrendo:«\textunderscore aprouve a Deus, a Santa Maria, que as ribeiradas de sangue do meu gilvaz seom ia vedadas...\textunderscore »(Da carta do arceb. de Braga, D. Lourenço, ao Abbade de Alcobaça, sôbre a batalha de Aljubarrota, 1385)
\section{Ribeirão}
\begin{itemize}
\item {Grp. gram.:m.}
\end{itemize}
\begin{itemize}
\item {Utilização:Bras}
\end{itemize}
\begin{itemize}
\item {Proveniência:(De \textunderscore ribeiro\textunderscore )}
\end{itemize}
Terreno, apropriado para nele se lavrarem minas de diamantes.
\section{Ribeirar}
\begin{itemize}
\item {Grp. gram.:v. t.}
\end{itemize}
\begin{itemize}
\item {Utilização:Bras. do N}
\end{itemize}
Marcar a ferro o lado esquerdo de (animaes pertencentes a um grupo de fazendas chamado \textunderscore ribeira\textunderscore )
\section{Ribeirinhas}
\begin{itemize}
\item {Grp. gram.:f. pl.}
\end{itemize}
O mesmo que [[aves|ave]] [[pernaltas]].
(Fem. pl. de \textunderscore ribeirinho\textunderscore )
\section{Ribeirinho}
\begin{itemize}
\item {Grp. gram.:adj.}
\end{itemize}
\begin{itemize}
\item {Grp. gram.:M.}
\end{itemize}
\begin{itemize}
\item {Proveniência:(De \textunderscore ribeiro\textunderscore )}
\end{itemize}
Que se encontra ou vive nos rios ou ribeiras.
Que vive junto ao rio ou ribeira.
Marginal; juxtafluvial.
Moço de recados.
\section{Ribeiro}
\begin{itemize}
\item {Grp. gram.:m.}
\end{itemize}
\begin{itemize}
\item {Utilização:Constr.}
\end{itemize}
\begin{itemize}
\item {Grp. gram.:Adj.}
\end{itemize}
\begin{itemize}
\item {Proveniência:(Do lat. \textunderscore riparius\textunderscore )}
\end{itemize}
Rio pequeno; regato.
Intersecção de águas de um telhado, segundo um ângulo reentrante.
Diz-se de uma espécie de trigo.
\section{Ribeiró}
\begin{itemize}
\item {Grp. gram.:m.}
\end{itemize}
\begin{itemize}
\item {Utilização:Prov.}
\end{itemize}
Ave ribeirinha.
\section{Ribésia}
\begin{itemize}
\item {Grp. gram.:f.}
\end{itemize}
Nome scientífico da groselheira.
\section{Ribesiáceas}
\begin{itemize}
\item {Grp. gram.:f. pl.}
\end{itemize}
Família de plantas, que tem por typo a ribésia.
(Fem. pl. de \textunderscore ribesiáceo\textunderscore )
\section{Ribesiáceo}
\begin{itemize}
\item {Grp. gram.:adj.}
\end{itemize}
Relativo ou semelhante á ribésia.
\section{Ribete}
\begin{itemize}
\item {fónica:bê}
\end{itemize}
\begin{itemize}
\item {Grp. gram.:m.}
\end{itemize}
\begin{itemize}
\item {Proveniência:(T. cast.)}
\end{itemize}
Cairel.
Lista para guarnecer; debrum.
\section{Ribrânquio}
\begin{itemize}
\item {Grp. gram.:adj.}
\end{itemize}
\begin{itemize}
\item {Proveniência:(De \textunderscore re...\textunderscore  + \textunderscore branco\textunderscore ?)}
\end{itemize}
Diz-se de uma casta de figos, que são vermelhos por dentro e esbranquiçados por fóra.
\section{Riça}
\begin{itemize}
\item {Grp. gram.:f.}
\end{itemize}
Pêlo, que o fabricante de chapéus tira dêstes, quando os escarduça.
(Cp. cast. \textunderscore riza\textunderscore )
\section{Riça}
\begin{itemize}
\item {Grp. gram.:adj. f.}
\end{itemize}
\begin{itemize}
\item {Proveniência:(De \textunderscore riçar\textunderscore )}
\end{itemize}
Diz-se de uma variedade de galinhas, que tem pennas encrespadas.
\section{Ricaço}
\begin{itemize}
\item {Grp. gram.:m.  e  adj.}
\end{itemize}
\begin{itemize}
\item {Utilização:Pop.}
\end{itemize}
Homem muito rico.
\section{Rica-dona}
\begin{itemize}
\item {Grp. gram.:f.}
\end{itemize}
Mulhér de rico-homem; filha ou successora de rico-homem.
\section{Ricalhoiço}
\begin{itemize}
\item {Grp. gram.:m.}
\end{itemize}
O mesmo que \textunderscore ricaço\textunderscore :«\textunderscore ...fazer jus á sopa de algum ricalhoiço...\textunderscore »Th. Ribeiro, \textunderscore Jornadas\textunderscore , II, 167.
\section{Ricamente}
\begin{itemize}
\item {Grp. gram.:adv.}
\end{itemize}
\begin{itemize}
\item {Utilização:Fig.}
\end{itemize}
De modo rico.
Ostentosamente, com luxo.
\section{Ricanho}
\begin{itemize}
\item {Grp. gram.:m.  e  adj.}
\end{itemize}
\begin{itemize}
\item {Utilização:Pop.}
\end{itemize}
Homem rico e sovina.
\section{Riçar}
\begin{itemize}
\item {Grp. gram.:v. t.}
\end{itemize}
Tornar riço ou crespo.
Encarapinhar, encaracolar (cabello).
Fazer arripiar (o cabello).
\section{Richárdia}
\begin{itemize}
\item {Grp. gram.:f.}
\end{itemize}
\begin{itemize}
\item {Proveniência:(De \textunderscore Richard.\textunderscore , n. p.)}
\end{itemize}
Gênero de plantas aráceas.
\section{Richarte}
\begin{itemize}
\item {Grp. gram.:m.  e  adj.}
\end{itemize}
\begin{itemize}
\item {Utilização:ant.}
\end{itemize}
\begin{itemize}
\item {Utilização:Chul.}
\end{itemize}
\begin{itemize}
\item {Proveniência:(Do fr. \textunderscore richard\textunderscore ?)}
\end{itemize}
Homem baixo, gordo e forte.
\section{Ric'homem}
\begin{itemize}
\item {Grp. gram.:m.}
\end{itemize}
\begin{itemize}
\item {Utilização:Ant.}
\end{itemize}
O mesmo que \textunderscore rico-homem\textunderscore .
\section{Ricinato}
\begin{itemize}
\item {Grp. gram.:m.}
\end{itemize}
\begin{itemize}
\item {Utilização:Chím.}
\end{itemize}
Sal, formado pela combinação do ácido ricínico com uma base.
\section{Ricíneas}
\begin{itemize}
\item {Grp. gram.:f. pl.}
\end{itemize}
Tríbo de plantas euphorbiáceas, que tem por typo o gênero rícino.
\section{Ricinela}
\begin{itemize}
\item {Grp. gram.:f.}
\end{itemize}
Gênero de molluscos gasterópodes.
\section{Ricínico}
\begin{itemize}
\item {Grp. gram.:adj.}
\end{itemize}
Diz-se de um ácido produzido pela saponificação do óleo de rícino.
\section{Ricinina}
\begin{itemize}
\item {Grp. gram.:f.}
\end{itemize}
Princípio purgativo do óleo de ricino.
\section{Ricínio}
\begin{itemize}
\item {Grp. gram.:m.}
\end{itemize}
\begin{itemize}
\item {Proveniência:(Lat. \textunderscore ricinium\textunderscore )}
\end{itemize}
Vestuário, que as damas romanas usavam por cima da túnica.
\section{Rícino}
\begin{itemize}
\item {Grp. gram.:m.}
\end{itemize}
\begin{itemize}
\item {Proveniência:(Lat. \textunderscore ricinus\textunderscore )}
\end{itemize}
O mesmo que \textunderscore mamona\textunderscore ^1.
Insecto parasito, mais conhecido por \textunderscore carrapato\textunderscore .
\section{Ricinocarpo}
\begin{itemize}
\item {Grp. gram.:m.}
\end{itemize}
\begin{itemize}
\item {Proveniência:(Do lat. \textunderscore ricinus\textunderscore  + gr. \textunderscore karpos\textunderscore )}
\end{itemize}
Gênero de arbustos euphorbiáceos.
\section{Ricinoleato}
\begin{itemize}
\item {Grp. gram.:m.}
\end{itemize}
\begin{itemize}
\item {Proveniência:(De \textunderscore ricino\textunderscore  + \textunderscore óleo\textunderscore )}
\end{itemize}
Sal, formado pela combinação do ácido ricinólico com uma base.
\section{Ricinólico}
\begin{itemize}
\item {Grp. gram.:adj.}
\end{itemize}
\begin{itemize}
\item {Proveniência:(De \textunderscore rícino\textunderscore  + \textunderscore óleo\textunderscore )}
\end{itemize}
Diz-se de um ácido, extrahido do óleo de rícino.
\section{Rícino-maiór}
\begin{itemize}
\item {Grp. gram.:m.}
\end{itemize}
O mesmo que \textunderscore purgueira\textunderscore .
\section{Ricinosteárico}
\begin{itemize}
\item {Grp. gram.:adj.}
\end{itemize}
Diz-se de um ácido, produzido pela saponificação do óleo de rícino.
\section{Rico}
\begin{itemize}
\item {Grp. gram.:adj.}
\end{itemize}
\begin{itemize}
\item {Utilização:Fig.}
\end{itemize}
\begin{itemize}
\item {Grp. gram.:M.}
\end{itemize}
Que tem muitos bens.
Abundante: \textunderscore colheita rica\textunderscore .
Cheio.
Opulento.
Fértil: \textunderscore terrenos ricos\textunderscore .
Magnífico; magnificente: \textunderscore rico espectáculo\textunderscore .
Bello.
Feliz.
Bom.
Contente.
Querido: \textunderscore meu rico filho\textunderscore .
Homem rico.
(Cp. b. lat. \textunderscore riccus\textunderscore )
\section{Riço}
\begin{itemize}
\item {Grp. gram.:m.}
\end{itemize}
Feixe ou pasta de cabelo ou de lan, sôbre a qual as senhoras elevavam ou adaptavam o penteado.
Tecido de lan, com o pêlo encrespado e curto.
(Cp. \textunderscore ouriço\textunderscore )
\section{Ricochetear}
\begin{itemize}
\item {Grp. gram.:v. i.}
\end{itemize}
Fazer ricochete.
\section{Ricochete}
\begin{itemize}
\item {fónica:chê}
\end{itemize}
\begin{itemize}
\item {Grp. gram.:m.}
\end{itemize}
\begin{itemize}
\item {Utilização:Fig.}
\end{itemize}
\begin{itemize}
\item {Utilização:Fam.}
\end{itemize}
\begin{itemize}
\item {Proveniência:(Fr. \textunderscore ricochet\textunderscore )}
\end{itemize}
Salto de qualquer corpo ou projéctil, depois de bater no chão ou noutro corpo.
Retrocesso.
Acontecimento, produzido por outro, á maneira de uma pedra de ricochete.
Remoque, motejo.
\section{Rico-homem}
\begin{itemize}
\item {Grp. gram.:m.}
\end{itemize}
Grande do reino, que servia o rei na guerra, e usava como insígnia pendão e caldeira, para indicar que sustentava os outros.
\section{Ricoiço}
\begin{itemize}
\item {Grp. gram.:m.  e  adj.}
\end{itemize}
O mesmo que \textunderscore ricaço\textunderscore .
\section{Rico-pobre}
\begin{itemize}
\item {Grp. gram.:m.}
\end{itemize}
Casta de uva branca algarvia.
\section{Ricótia}
\begin{itemize}
\item {Grp. gram.:f.}
\end{itemize}
Gênero de plantas crucíferas.
\section{Ricto}
\begin{itemize}
\item {Grp. gram.:m.}
\end{itemize}
O mesmo ou melhor que \textunderscore ríctus\textunderscore .
\section{Ríctus}
\begin{itemize}
\item {Grp. gram.:m.}
\end{itemize}
\begin{itemize}
\item {Proveniência:(Lat. \textunderscore rictus\textunderscore )}
\end{itemize}
A abertura da boca.
\section{Ridela}
\begin{itemize}
\item {Grp. gram.:f.}
\end{itemize}
\begin{itemize}
\item {Utilização:P. us.}
\end{itemize}
Sebe ou caniçado, á volta do carro, para não deixar caír as coisas da carga.
(Cp. lat. \textunderscore ridicula\textunderscore )
\section{Ridente}
\begin{itemize}
\item {Grp. gram.:adj.}
\end{itemize}
\begin{itemize}
\item {Utilização:poét.}
\end{itemize}
\begin{itemize}
\item {Utilização:Fig.}
\end{itemize}
\begin{itemize}
\item {Proveniência:(Lat. \textunderscore ridens\textunderscore )}
\end{itemize}
Que ri.
Satisfeito, alegre.
Que viceja.
Magnificente: \textunderscore aurora ridente\textunderscore .
\section{Ridiculamente}
\begin{itemize}
\item {Grp. gram.:adv.}
\end{itemize}
De modo ridículo.
\section{Ridicularia}
\begin{itemize}
\item {Grp. gram.:f.}
\end{itemize}
Acto ou dito ridículo.
Coisa de pequeno valor; bagatela.
\section{Ridicularizar}
\begin{itemize}
\item {Grp. gram.:v. t.}
\end{itemize}
O mesmo que \textunderscore ridiculizar\textunderscore .
\section{Ridiculez}
\begin{itemize}
\item {Grp. gram.:f.}
\end{itemize}
Qualidade de ridículo; ridicularia. Cf. M. de Assis, \textunderscore A Mão e a Luva\textunderscore , X; Júl. de Castilho, \textunderscore Lisb. Ant.\textunderscore 
\section{Ridiculeza}
\begin{itemize}
\item {Grp. gram.:f.}
\end{itemize}
\begin{itemize}
\item {Utilização:Prov.}
\end{itemize}
\begin{itemize}
\item {Utilização:beir.}
\end{itemize}
O mesmo que \textunderscore ridiculez\textunderscore .
\section{Ridiculização}
\begin{itemize}
\item {Grp. gram.:f.}
\end{itemize}
Acto de ridiculizar.
\section{Ridiculizar}
\begin{itemize}
\item {Grp. gram.:v. t.}
\end{itemize}
Tornar ridículo; escarnecer de.
\section{Ridículo}
\begin{itemize}
\item {Grp. gram.:adj.}
\end{itemize}
\begin{itemize}
\item {Utilização:Pop.}
\end{itemize}
\begin{itemize}
\item {Grp. gram.:M.}
\end{itemize}
\begin{itemize}
\item {Proveniência:(Lat. \textunderscore ridiculus\textunderscore )}
\end{itemize}
Que desperta riso ou escárneo: \textunderscore chapéu ridículo\textunderscore .
Digno de escárneo.
Irrisório; insignificante: \textunderscore pretensão ridícula\textunderscore .
Avarento, sovina: \textunderscore é tão ridículo, que não dá cinco reis a um pobre.\textunderscore 
Pessôa ou coisa ridícula.
Modo ridículo.
Acto ou effeito de ridiculizar.
\section{Ridiculoso}
\begin{itemize}
\item {Grp. gram.:adj.}
\end{itemize}
\begin{itemize}
\item {Utilização:Ant.}
\end{itemize}
O mesmo que \textunderscore ridículo\textunderscore . Cf. Camillo, \textunderscore Quéda\textunderscore , 151.
\section{Ridó}
\begin{itemize}
\item {Grp. gram.:m.}
\end{itemize}
\begin{itemize}
\item {Utilização:Ant.}
\end{itemize}
\begin{itemize}
\item {Proveniência:(Do fr. \textunderscore rideau\textunderscore )}
\end{itemize}
O mesmo que \textunderscore estore\textunderscore .
\section{Ridor}
\begin{itemize}
\item {Grp. gram.:m.  e  adj.}
\end{itemize}
\begin{itemize}
\item {Utilização:P. us.}
\end{itemize}
O que ri; homem que gosta de rir ou zombar.
\section{Riemanito}
\begin{itemize}
\item {Grp. gram.:m.}
\end{itemize}
\begin{itemize}
\item {Utilização:Miner.}
\end{itemize}
Certo silicato de alumina, infusível, solúvel nos ácidos.
\section{Riéti}
\begin{itemize}
\item {Grp. gram.:m.}
\end{itemize}
\begin{itemize}
\item {Proveniência:(T. it.)}
\end{itemize}
Variedade de trigo, procedente da Itália.
\section{Rifa}
\begin{itemize}
\item {Grp. gram.:f.}
\end{itemize}
Sorteio ou lotaria de um ou mais objectos, por meio de bilhetes numerados; rifada.
(Cast. \textunderscore rifa\textunderscore )
\section{Rifa}
\begin{itemize}
\item {Grp. gram.:f.}
\end{itemize}
\begin{itemize}
\item {Utilização:Ant.}
\end{itemize}
Caminho íngreme e acantilado.
(Corr. de \textunderscore riba\textunderscore ? ou do ár?)
\section{Rifada}
\begin{itemize}
\item {Grp. gram.:f.}
\end{itemize}
\begin{itemize}
\item {Proveniência:(De \textunderscore rifa\textunderscore )}
\end{itemize}
Porção de cartas do mesmo naipe.
\section{Rifador}
\begin{itemize}
\item {Grp. gram.:m.  e  adj.}
\end{itemize}
O que rifa.
\section{Rifador}
\begin{itemize}
\item {Grp. gram.:m.}
\end{itemize}
\begin{itemize}
\item {Utilização:Ant.}
\end{itemize}
Brigão, homem desordeiro. Cf. G. Resende, \textunderscore Cancion.\textunderscore 
\section{Rifão}
\begin{itemize}
\item {Grp. gram.:m.}
\end{itemize}
Ditado popular; proverbio; anexim.
\textunderscore Pl.\textunderscore  \textunderscore rifões\textunderscore ; entretanto, lê-se \textunderscore rifães\textunderscore  em R. Lobo, \textunderscore Côrte na Aldeia\textunderscore , II, 44.
(Corr. de \textunderscore refrão\textunderscore )
\section{Rifar}
\begin{itemize}
\item {Grp. gram.:v. t.}
\end{itemize}
\begin{itemize}
\item {Utilização:Ant.}
\end{itemize}
\begin{itemize}
\item {Utilização:Gír.}
\end{itemize}
\begin{itemize}
\item {Grp. gram.:V. i.}
\end{itemize}
\begin{itemize}
\item {Utilização:Prov.}
\end{itemize}
\begin{itemize}
\item {Utilização:minh.}
\end{itemize}
Fazer rifa de.
Sortear por bilhetes numerados: \textunderscore rifar um relógio\textunderscore .
Roubar, bifar.
Ralhar; contender.
\section{Rifar}
\begin{itemize}
\item {Grp. gram.:v. t.}
\end{itemize}
\begin{itemize}
\item {Utilização:Prov.}
\end{itemize}
\begin{itemize}
\item {Proveniência:(De \textunderscore rifa\textunderscore ^2? Cp. \textunderscore surribar\textunderscore )}
\end{itemize}
Desbastar ou escavar com alvião ou picareta (o saibro ou solão).
Rinchar brandamente:«\textunderscore ...o cavallo... rifando e escouceando...\textunderscore »Camillo, \textunderscore Filha do Reg.\textunderscore , 230.
\section{Rifaria}
\begin{itemize}
\item {Grp. gram.:f.}
\end{itemize}
\begin{itemize}
\item {Utilização:Des.}
\end{itemize}
\begin{itemize}
\item {Utilização:Pop.}
\end{itemize}
\begin{itemize}
\item {Proveniência:(De \textunderscore rifar\textunderscore ^1)}
\end{itemize}
Tumulto, desordem.
\section{Rife}
\begin{itemize}
\item {Grp. gram.:m.}
\end{itemize}
\begin{itemize}
\item {Utilização:Prov.}
\end{itemize}
Acto de rifar^2.
\section{Rifenhos}
\begin{itemize}
\item {Grp. gram.:m. pl.}
\end{itemize}
Povoadores da região do Rife, na costa setentrional de Marrocos.
\section{Rifete}
\begin{itemize}
\item {fónica:fê}
\end{itemize}
\begin{itemize}
\item {Grp. gram.:m.}
\end{itemize}
Uva, o mesmo que \textunderscore rucete\textunderscore .
\section{Rifle}
O mesmo que \textunderscore refle\textunderscore . Cf. Herculano, \textunderscore Quest. Públ.\textunderscore , II, 86.--É fórma usual no Brasil.
\section{Riga}
\begin{itemize}
\item {Grp. gram.:m.}
\end{itemize}
Linho ordinário, procedente da cidade russa do mesmo nome.
Madeira de carvalho, usada em tanoaria e procedente da mesma cidade russa.
\section{Rigaço}
\begin{itemize}
\item {Grp. gram.:m.}
\end{itemize}
\begin{itemize}
\item {Utilização:Des.}
\end{itemize}
\begin{itemize}
\item {Proveniência:(Do lat. \textunderscore rigatus\textunderscore )}
\end{itemize}
Pão, feito de trigo de regadio.
\section{Rigidamente}
\begin{itemize}
\item {Grp. gram.:adv.}
\end{itemize}
De modo rígido.
Com austeridade.
\section{Rigidez}
\begin{itemize}
\item {Grp. gram.:f.}
\end{itemize}
\begin{itemize}
\item {Utilização:Fig.}
\end{itemize}
\begin{itemize}
\item {Utilização:Med.}
\end{itemize}
Qualidade do que é rígido ou rijo.
Austeridade.
Aspereza.
Tensão do collo do útero.
\section{Rígido}
\begin{itemize}
\item {Grp. gram.:adj.}
\end{itemize}
\begin{itemize}
\item {Utilização:Fig.}
\end{itemize}
\begin{itemize}
\item {Proveniência:(Lat. \textunderscore rigidus\textunderscore )}
\end{itemize}
Teso, rijo; hirto.
Austero; severo, rigoroso.
\section{Rigodão}
\begin{itemize}
\item {Grp. gram.:m.}
\end{itemize}
\begin{itemize}
\item {Proveniência:(Fr. \textunderscore rigodon\textunderscore )}
\end{itemize}
Espécie de dança antiga, muito animada, entre dois ou mais pares.
A música, que acompanhava essa dança.
\section{Rigol}
\begin{itemize}
\item {Grp. gram.:m.}
\end{itemize}
\begin{itemize}
\item {Utilização:Prov.}
\end{itemize}
\begin{itemize}
\item {Utilização:extrem.}
\end{itemize}
Qualquer pequeno rêgo.
O mesmo que \textunderscore rigola\textunderscore .
\section{Rigola}
\begin{itemize}
\item {Grp. gram.:f.}
\end{itemize}
(V.regola)
\section{Rigolboche}
\begin{itemize}
\item {Grp. gram.:adj.}
\end{itemize}
\begin{itemize}
\item {Proveniência:(De \textunderscore Rigolboche\textunderscore , n. p.)}
\end{itemize}
Libertino, devasso; erótico. Cf. Camillo, \textunderscore Amor de Perd.\textunderscore , (pref. da 5.^a ed.)
\section{Rigor}
\begin{itemize}
\item {Grp. gram.:m.}
\end{itemize}
\begin{itemize}
\item {Utilização:Fig.}
\end{itemize}
\begin{itemize}
\item {Utilização:Bot.}
\end{itemize}
\begin{itemize}
\item {Proveniência:(Lat. \textunderscore rigor\textunderscore )}
\end{itemize}
Rigidez; dureza; fôrça.
Severidade.
Grande pontualidade.
Insensibilidade.
Concisão.
Interpretação restricta.
A maior intensidade (do frio, calor, chuva, etc.): \textunderscore no rigor no estio\textunderscore .
Planta polygónia.
\section{Rigor}
\begin{itemize}
\item {Grp. gram.:m.}
\end{itemize}
\begin{itemize}
\item {Utilização:Prov.}
\end{itemize}
\begin{itemize}
\item {Utilização:minh.}
\end{itemize}
\begin{itemize}
\item {Utilização:dur.}
\end{itemize}
\begin{itemize}
\item {Utilização:Prov.}
\end{itemize}
\begin{itemize}
\item {Utilização:trasm.}
\end{itemize}
Fita estreita de velludo, com que se debruam tamancos e chancas, e com que também se guarnecem capas das mulheres do Minho.
Faixa avermelhada no céu, ao Poente ou ao Nascente.
(Relaciona-se talvez com o lat. \textunderscore regula\textunderscore )
\section{Rigoridade}
\begin{itemize}
\item {Grp. gram.:f.}
\end{itemize}
\begin{itemize}
\item {Utilização:Ant.}
\end{itemize}
\begin{itemize}
\item {Proveniência:(De \textunderscore rigor\textunderscore )}
\end{itemize}
Tratamento rigoroso; severidade. Cf. Fernandes, \textunderscore Caça de Altan.\textunderscore , c. VII.
\section{Rigorismo}
\begin{itemize}
\item {Grp. gram.:m.}
\end{itemize}
\begin{itemize}
\item {Proveniência:(De \textunderscore rigor\textunderscore )}
\end{itemize}
Qualidade do que é rigoroso.
Demasiado rigor.
Grande severidade.
Pontualidade.
\section{Rigorista}
\begin{itemize}
\item {Grp. gram.:m. ,  f.  e  adj.}
\end{itemize}
\begin{itemize}
\item {Proveniência:(De \textunderscore rigor\textunderscore )}
\end{itemize}
Pessôa, que usa de rigorismo.
\section{Rigorosamente}
\begin{itemize}
\item {Grp. gram.:adv.}
\end{itemize}
De modo rigoroso; com rigor; severamente.
\section{Rigorosidade}
\begin{itemize}
\item {Grp. gram.:f.}
\end{itemize}
Qualidade do que é rigoroso.
\section{Rigoroso}
\begin{itemize}
\item {Grp. gram.:adj.}
\end{itemize}
\begin{itemize}
\item {Proveniência:(Lat. \textunderscore rigorosus\textunderscore )}
\end{itemize}
Que procede com rigor: \textunderscore professor rigoroso\textunderscore .
Em que há rigor: \textunderscore providências rigorosas\textunderscore .
Que revela rigor.
Deshumano.
Muito exigente; minucioso.
Conciso, restricto.
Que corresponde precisamente a uma situação, época, etc.: \textunderscore interpretação rigorosa\textunderscore .
\section{Rigueifa}
\begin{itemize}
\item {Grp. gram.:f.}
\end{itemize}
\begin{itemize}
\item {Utilização:Ant.}
\end{itemize}
O mesmo que \textunderscore regueifa\textunderscore .
(Cp. \textunderscore rigaço\textunderscore )
\section{Rijal}
\begin{itemize}
\item {Grp. gram.:adj.}
\end{itemize}
\begin{itemize}
\item {Proveniência:(De \textunderscore rijo\textunderscore )}
\end{itemize}
Diz-se de uma espécie de cerejas portuguesas.
Diz-se de de vários frutos, por opposição a \textunderscore mollar\textunderscore .
\section{Rijamente}
\begin{itemize}
\item {Grp. gram.:adv.}
\end{itemize}
\begin{itemize}
\item {Proveniência:(De \textunderscore rijo\textunderscore )}
\end{itemize}
Com rijeza, vehementemente.
A valer: \textunderscore bateu-lhe rijamente\textunderscore .
\section{Rijão}
\begin{itemize}
\item {Grp. gram.:m.}
\end{itemize}
\begin{itemize}
\item {Utilização:Prov.}
\end{itemize}
\begin{itemize}
\item {Proveniência:(De \textunderscore rijar\textunderscore )}
\end{itemize}
Pedaço de carne de porco, frito, mas não reduzido a torresmo.
\section{Rijão}
\begin{itemize}
\item {Grp. gram.:m.}
\end{itemize}
\begin{itemize}
\item {Utilização:Prov.}
\end{itemize}
O mesmo que \textunderscore torresmo\textunderscore .
(Cp.\textunderscore rojão\textunderscore ^3)
\section{Rijar}
\begin{itemize}
\item {Grp. gram.:v. t.}
\end{itemize}
\begin{itemize}
\item {Utilização:Prov.}
\end{itemize}
O mesmo que \textunderscore frigir\textunderscore  (carnes).
\section{Rijeira}
\begin{itemize}
\item {Grp. gram.:f.}
\end{itemize}
\begin{itemize}
\item {Utilização:Náut.}
\end{itemize}
\begin{itemize}
\item {Proveniência:(De \textunderscore rijo\textunderscore )}
\end{itemize}
Escorado navio no estaleiro, collocada á prôa, sob a quilha, e última das que aguentam a embarcação antes de lançada á água.
Também se diz \textunderscore regeira\textunderscore , talvez com menos propriedade.
\section{Rijeza}
\begin{itemize}
\item {Grp. gram.:f.}
\end{itemize}
Qualidade de rijo.
\section{Rijo}
\begin{itemize}
\item {Grp. gram.:adj.}
\end{itemize}
\begin{itemize}
\item {Grp. gram.:M.}
\end{itemize}
\begin{itemize}
\item {Grp. gram.:Adv.}
\end{itemize}
\begin{itemize}
\item {Proveniência:(Do lat. \textunderscore rigidus\textunderscore )}
\end{itemize}
Seguro.
Teso.
Severo.
Vigoroso; intenso; forte.
O principal, a maior parte.
Rijamente: \textunderscore bater rijo\textunderscore .
\section{Rijões}
\begin{itemize}
\item {Grp. gram.:m. pl.}
\end{itemize}
\begin{itemize}
\item {Utilização:T. do Porto}
\end{itemize}
Escumalha de ferro.
\section{Ril}
\begin{itemize}
\item {Grp. gram.:m.}
\end{itemize}
\begin{itemize}
\item {Utilização:Prov.}
\end{itemize}
\begin{itemize}
\item {Utilização:alg.}
\end{itemize}
\begin{itemize}
\item {Utilização:beir.}
\end{itemize}
\begin{itemize}
\item {Utilização:Ant.}
\end{itemize}
O mesmo que \textunderscore rim\textunderscore .
\section{Ril}
\begin{itemize}
\item {Grp. gram.:m.}
\end{itemize}
Espécie do dança. Cf. A. Macedo, \textunderscore Burros\textunderscore , 257.
(Ingl.\textunderscore reel\textunderscore )
\section{Rilada}
\begin{itemize}
\item {Grp. gram.:f.}
\end{itemize}
\begin{itemize}
\item {Utilização:Pop.}
\end{itemize}
\begin{itemize}
\item {Proveniência:(De \textunderscore ril\textunderscore ^1)}
\end{itemize}
Guisado feito de rim.
\section{Rilha-boi}
\begin{itemize}
\item {Grp. gram.:f.}
\end{itemize}
O mesmo que \textunderscore resta-boi\textunderscore .
\section{Rilhador}
\begin{itemize}
\item {Grp. gram.:m.  e  adj.}
\end{itemize}
O que rilha.
\section{Rilhadura}
\begin{itemize}
\item {Grp. gram.:f.}
\end{itemize}
Acto ou effeito de rilhar^1.
\section{Rilhar}
\begin{itemize}
\item {Grp. gram.:v. t.}
\end{itemize}
\begin{itemize}
\item {Utilização:Pop.}
\end{itemize}
Roer (objecto duro).
Trincar.
Comer, resmungando.
\section{Rilhar}
\begin{itemize}
\item {Grp. gram.:v. i.}
\end{itemize}
\begin{itemize}
\item {Utilização:T. da Bairrada}
\end{itemize}
Mirrar-se (a carne), adherindo ao osso.
Murchar ou secar a polpa de um fruto, adherindo ao caroço.
Engelhar.
(Cp.\textunderscore enrilhar\textunderscore )
\section{Rilheira}
\begin{itemize}
\item {Grp. gram.:f.}
\end{itemize}
Molde de ferro, em que os ourives vasam metal fundido, para fazerem chapas.
(Talvez corr. de \textunderscore relheira\textunderscore , de \textunderscore relha\textunderscore )
\section{Rilheiro}
\begin{itemize}
\item {Grp. gram.:m.}
\end{itemize}
\begin{itemize}
\item {Utilização:Ant.}
\end{itemize}
Corrente marítima; redemoínho de água. Cf.\textunderscore Roteiro do Mar Verm.\textunderscore , 268.
(Cp.\textunderscore relheira\textunderscore )
\section{Rilheiro}
\begin{itemize}
\item {Grp. gram.:m.}
\end{itemize}
\begin{itemize}
\item {Utilização:Prov.}
\end{itemize}
\begin{itemize}
\item {Utilização:alent.}
\end{itemize}
O mesmo que \textunderscore rolheiro\textunderscore , mólho de trigo.
\section{Rilheiro}
\begin{itemize}
\item {Grp. gram.:m.}
\end{itemize}
\begin{itemize}
\item {Utilização:Prov.}
\end{itemize}
\begin{itemize}
\item {Utilização:minh.}
\end{itemize}
\begin{itemize}
\item {Utilização:Fig.}
\end{itemize}
\begin{itemize}
\item {Proveniência:(De \textunderscore rilhar\textunderscore )}
\end{itemize}
Lugar, onde os ratos juntam e rilham o que roubam, (cereaes, nozes, castanhas etc.).
Provisão, celleiro.
Lucro.
\section{Rilhoto}
\begin{itemize}
\item {fónica:lhô}
\end{itemize}
\begin{itemize}
\item {Grp. gram.:m.}
\end{itemize}
\begin{itemize}
\item {Utilização:Prov.}
\end{itemize}
Porção de excremento, pequena e dura.
(Por \textunderscore enrilhoto\textunderscore  de \textunderscore enrilhar\textunderscore )
\section{Rim}
\begin{itemize}
\item {Grp. gram.:m.}
\end{itemize}
\begin{itemize}
\item {Grp. gram.:Pl.}
\end{itemize}
\begin{itemize}
\item {Utilização:Constr.}
\end{itemize}
\begin{itemize}
\item {Proveniência:(Lat. \textunderscore ren\textunderscore )}
\end{itemize}
Viscera, dupla, que é o órgão secretor da urina.
A parte inferior da região lombar.
Encontro das abóbadas, que descansa na emposta.
Sólido prismático, formado pelo prolongamento longitudinal do týmpano.
\section{Rima}
\begin{itemize}
\item {Grp. gram.:Pl.}
\end{itemize}
\begin{itemize}
\item {Proveniência:(Do ant. alt. al. \textunderscore rêm\textunderscore )}
\end{itemize}
\textunderscore f.\textunderscore 
Uniformidade de sons, na terminação de duas ou mais palavras.
Consoante, em que terminam os versos.
Versos.
\section{Rima}
\begin{itemize}
\item {Grp. gram.:f.}
\end{itemize}
\begin{itemize}
\item {Utilização:Prov.}
\end{itemize}
\begin{itemize}
\item {Utilização:trasm.}
\end{itemize}
\begin{itemize}
\item {Proveniência:(Lat. \textunderscore rima\textunderscore )}
\end{itemize}
Fenda, grêta.
Pequena abertura.
Pequena ferida cinzenta, na mama de fêmeas de gado.
Restos, vestígios: \textunderscore essa doença sempre deixa rima\textunderscore .
\section{Rima}
\begin{itemize}
\item {Grp. gram.:f.}
\end{itemize}
\begin{itemize}
\item {Proveniência:(Do ár. \textunderscore rizma\textunderscore ?)}
\end{itemize}
Acto ou effeito de arrimar.
Montão; ruma.
Porção de coisas, que se accumulam: \textunderscore rima de lenha\textunderscore ; \textunderscore rima de madeira...\textunderscore 
\section{Rima}
\begin{itemize}
\item {Grp. gram.:f.}
\end{itemize}
\begin{itemize}
\item {Utilização:Bras}
\end{itemize}
O mesmo que \textunderscore fruta-pão\textunderscore .
\section{Rimador}
\begin{itemize}
\item {Grp. gram.:m.  e  adj.}
\end{itemize}
\begin{itemize}
\item {Proveniência:(De \textunderscore rimar\textunderscore )}
\end{itemize}
O que faz rimas; versejador.
\section{Rimalho}
\begin{itemize}
\item {Grp. gram.:m.}
\end{itemize}
Systema ou conjunto das rimas de uma língua. Cf. João Ribeiro, \textunderscore Estética\textunderscore .
\section{Rimance}
\begin{itemize}
\item {Grp. gram.:m.}
\end{itemize}
\begin{itemize}
\item {Utilização:Ant.}
\end{itemize}
Língua vulgar.
Xácara; seguidilha.
Pequeno canto épico.
(Corr. de \textunderscore romance\textunderscore , por infl. de \textunderscore rima\textunderscore )
\section{Rimar}
\begin{itemize}
\item {Grp. gram.:v. t.}
\end{itemize}
\begin{itemize}
\item {Grp. gram.:V. i.}
\end{itemize}
\begin{itemize}
\item {Utilização:Fig.}
\end{itemize}
\begin{itemize}
\item {Proveniência:(De \textunderscore rima\textunderscore ^1)}
\end{itemize}
Pôr em versos rimados: \textunderscore rimar saudades\textunderscore .
Tornar consoantes (os versos).
Formar rima; versejar.
Sêr próprio ou decente; concordar.
\section{Rimbombar}
\begin{itemize}
\item {Grp. gram.:v. i.}
\end{itemize}
O mesmo que \textunderscore rebombar\textunderscore .
\section{Rimbombo}
\begin{itemize}
\item {Grp. gram.:m.}
\end{itemize}
O mesmo que \textunderscore rebombo\textunderscore .
\section{Rimoso}
\begin{itemize}
\item {Grp. gram.:adj.}
\end{itemize}
\begin{itemize}
\item {Proveniência:(Lat. \textunderscore rimosus\textunderscore )}
\end{itemize}
Que tem fendas, gretado.
\section{Rímula}
\begin{itemize}
\item {Grp. gram.:f.}
\end{itemize}
\begin{itemize}
\item {Utilização:Des.}
\end{itemize}
\begin{itemize}
\item {Proveniência:(Lat. \textunderscore rimula\textunderscore )}
\end{itemize}
Pequena fenda.
\section{Rina}
\begin{itemize}
\item {Proveniência:(Do gr. \textunderscore rhis\textunderscore , \textunderscore rhinos\textunderscore )}
\end{itemize}
Gênero de insectos coleópteros tetrâmeros.
\section{Rinalgia}
\begin{itemize}
\item {Grp. gram.:f.}
\end{itemize}
\begin{itemize}
\item {Proveniência:(Do gr. \textunderscore rhis\textunderscore , \textunderscore rhinos\textunderscore  + \textunderscore algos\textunderscore )}
\end{itemize}
Dôr no nariz.
\section{Rinálgico}
\begin{itemize}
\item {Grp. gram.:adj.}
\end{itemize}
Relativo á rinalgia.
\section{Rinantáceas}
\begin{itemize}
\item {Grp. gram.:f. pl.}
\end{itemize}
Família de plantas, que tem por tipo o rinanto.
(Fem. pl. de \textunderscore rhinantháceo\textunderscore )
\section{Rinantáceo}
\begin{itemize}
\item {Grp. gram.:adj.}
\end{itemize}
Relativo ou semelhante ao rinanto.
\section{Rinanto}
\begin{itemize}
\item {Grp. gram.:m.}
\end{itemize}
\begin{itemize}
\item {Proveniência:(Do gr. \textunderscore rhis\textunderscore , \textunderscore rhinos\textunderscore  + \textunderscore anthos\textunderscore )}
\end{itemize}
Planta herbácea, de flôres amarelas, e cujas fôlhas são aplicadas em tinturaria.
\section{Rinária}
\begin{itemize}
\item {Grp. gram.:f.}
\end{itemize}
Gênero de insectos coleópteros, originários da Austrália.
\section{Rináspide}
\begin{itemize}
\item {Grp. gram.:m.}
\end{itemize}
Gênero de insectos coleópteros, originários do Brasil.
\section{Rinasto}
\begin{itemize}
\item {Grp. gram.:m.}
\end{itemize}
Gênero de insectos coleópteros tetrâmeros.
\section{Rincão}
\begin{itemize}
\item {Grp. gram.:m.}
\end{itemize}
\begin{itemize}
\item {Utilização:Bras. do S}
\end{itemize}
\begin{itemize}
\item {Utilização:Carp.}
\end{itemize}
\begin{itemize}
\item {Utilização:Constr.}
\end{itemize}
Estria, que o navalhão abre na peça de artilharia, quando introduz ou retira a boca de dentro da alma.
A parte cavada nos ornatos de cantaria.
Porção de campo, em volta do qual cresce mato.
Lugar occulto, lugar afastado, recanto:«\textunderscore ...os mandou vir lá do rincão da Arábia...\textunderscore »Filinto, XVII, 180.
Cepo para fazer caneluras.
Cada uma das arestas salientes, segundo as quaes se interceptam as águas mestras e as tacaniças do telhado.
(Cast. \textunderscore rincón\textunderscore )
\section{Rinchada}
\begin{itemize}
\item {Grp. gram.:f.}
\end{itemize}
\begin{itemize}
\item {Utilização:Chul.}
\end{itemize}
\begin{itemize}
\item {Proveniência:(De \textunderscore rinchar\textunderscore )}
\end{itemize}
Gargalhada estrídula.
\section{Rinchalar}
\begin{itemize}
\item {Grp. gram.:v. i.}
\end{itemize}
\begin{itemize}
\item {Utilização:Ant.}
\end{itemize}
O mesmo que \textunderscore rinchar\textunderscore .
(Por \textunderscore rinchanar\textunderscore , de \textunderscore rinchão\textunderscore ^1)
\section{Rinchante}
\begin{itemize}
\item {Grp. gram.:adj.}
\end{itemize}
\begin{itemize}
\item {Grp. gram.:M.}
\end{itemize}
\begin{itemize}
\item {Utilização:Zool.}
\end{itemize}
\begin{itemize}
\item {Proveniência:(De \textunderscore rinchar\textunderscore )}
\end{itemize}
Que rincha muito.
Espécie de pêto, também conhecido por \textunderscore pêto-rinchão\textunderscore , \textunderscore pêto-real\textunderscore , \textunderscore pêto-verde\textunderscore  e \textunderscore cavallinho\textunderscore , (\textunderscore gecinus viridis\textunderscore , Shargi).
\section{Rinchão}
\begin{itemize}
\item {Grp. gram.:adj.}
\end{itemize}
\begin{itemize}
\item {Grp. gram.:M.}
\end{itemize}
\begin{itemize}
\item {Utilização:Zool.}
\end{itemize}
\begin{itemize}
\item {Proveniência:(De \textunderscore rinchar\textunderscore )}
\end{itemize}
Que rincha muito.
Espécie de pêto, também conhecido por \textunderscore pêto-rinchão\textunderscore , \textunderscore pêto-real\textunderscore , \textunderscore pêto-verde\textunderscore  e \textunderscore cavallinho\textunderscore , (\textunderscore gecinus viridis\textunderscore , Shargi).
\section{Rinchão}
\begin{itemize}
\item {Grp. gram.:m.}
\end{itemize}
Planta crucífera, comestível.
Variedade de pêra.
\section{Rinchar}
\begin{itemize}
\item {Grp. gram.:v. t.}
\end{itemize}
\begin{itemize}
\item {Grp. gram.:M.}
\end{itemize}
Soltar rincho.
Rincho.
\section{Rinchavelhada}
\begin{itemize}
\item {Grp. gram.:f.}
\end{itemize}
\begin{itemize}
\item {Utilização:Burl.}
\end{itemize}
\begin{itemize}
\item {Proveniência:(De \textunderscore rinchavelhar\textunderscore )}
\end{itemize}
Gargalhada destemperada.
\section{Rinchavelhar}
\begin{itemize}
\item {Grp. gram.:v. i.}
\end{itemize}
Rir ás gargalhadas, destemperadamente.
\section{Rincho}
\begin{itemize}
\item {Grp. gram.:m.}
\end{itemize}
\begin{itemize}
\item {Proveniência:(T. onom.)}
\end{itemize}
A voz do cavallo.
\section{Rincoalho}
\begin{itemize}
\item {Grp. gram.:m.}
\end{itemize}
\begin{itemize}
\item {Utilização:Prov.}
\end{itemize}
\begin{itemize}
\item {Utilização:trasm.}
\end{itemize}
Insecto, o mesmo que \textunderscore ralo\textunderscore .
Instrumento infantil.
\section{Rinconélia}
\begin{itemize}
\item {Grp. gram.:f.}
\end{itemize}
Molusco braquiópode do Mar-Branco.
\section{Rinconista}
\begin{itemize}
\item {Grp. gram.:m.}
\end{itemize}
\begin{itemize}
\item {Utilização:Bras. do S}
\end{itemize}
Guardador dos animaes que pastam num rincão.
\section{Rinelcose}
\begin{itemize}
\item {Grp. gram.:f.}
\end{itemize}
\begin{itemize}
\item {Utilização:Med.}
\end{itemize}
\begin{itemize}
\item {Proveniência:(Do gr. \textunderscore rhis\textunderscore , \textunderscore rhinos\textunderscore  + \textunderscore elkos\textunderscore )}
\end{itemize}
Ulceração da narina.
\section{Rinencéfalo}
\begin{itemize}
\item {Grp. gram.:m.}
\end{itemize}
\begin{itemize}
\item {Proveniência:(Do gr. \textunderscore rhis\textunderscore , \textunderscore rhinos\textunderscore  + \textunderscore enkephalon\textunderscore )}
\end{itemize}
Monstro, que tem o nariz prolongado em fórma de tromba.
\section{Ringer}
\begin{itemize}
\item {Grp. gram.:v. t.  e  i.}
\end{itemize}
O mesmo que \textunderscore ringir\textunderscore .
\section{Ríngia}
\begin{itemize}
\item {Grp. gram.:f.}
\end{itemize}
Insecto díptero, oval e chato.
\section{Ringícula}
\begin{itemize}
\item {Grp. gram.:f.}
\end{itemize}
Gênero de moluscos.
\section{Ringir}
\begin{itemize}
\item {Grp. gram.:v. t.}
\end{itemize}
\begin{itemize}
\item {Grp. gram.:V. i.}
\end{itemize}
\begin{itemize}
\item {Proveniência:(Do lat. \textunderscore ringi\textunderscore )}
\end{itemize}
Fazer ranger:«\textunderscore Abreu ringiu os dentes e rosnou.\textunderscore »Camillo, \textunderscore Nov. do Minh\textunderscore , I, 74.
Ranger.
\section{Ringleira}
\begin{itemize}
\item {Grp. gram.:f.}
\end{itemize}
\begin{itemize}
\item {Utilização:Prov.}
\end{itemize}
\begin{itemize}
\item {Utilização:trasm.}
\end{itemize}
Enfiada; fileira; série.
(Por \textunderscore rengreíra\textunderscore , de \textunderscore rengra\textunderscore )
\section{Rinha}
\begin{itemize}
\item {Grp. gram.:f.}
\end{itemize}
\begin{itemize}
\item {Utilização:Bras}
\end{itemize}
Briga de gallos.
(Cp.\textunderscore renhir\textunderscore )
\section{Rinhadeiro}
\begin{itemize}
\item {Grp. gram.:m.}
\end{itemize}
\begin{itemize}
\item {Utilização:Bras}
\end{itemize}
\begin{itemize}
\item {Proveniência:(De \textunderscore rinha\textunderscore )}
\end{itemize}
Lugar, onde combatem gallos.
\section{Rinhango}
\begin{itemize}
\item {Grp. gram.:m.}
\end{itemize}
O mesmo que \textunderscore dinhângoa\textunderscore . Cf. Capello e Ivens, \textunderscore De Angola ás Terras de Iaca\textunderscore .
\section{Rinhão}
\begin{itemize}
\item {Grp. gram.:m.}
\end{itemize}
\begin{itemize}
\item {Utilização:pop.}
\end{itemize}
\begin{itemize}
\item {Utilização:Ant.}
\end{itemize}
\begin{itemize}
\item {Utilização:Fig.}
\end{itemize}
\begin{itemize}
\item {Utilização:Ant.}
\end{itemize}
Rim.
Tecido adiposo, gordura.
(Cast. \textunderscore riñon\textunderscore )
\section{Rinhima-jampata}
\begin{itemize}
\item {Grp. gram.:f.}
\end{itemize}
Arbusto africano, leguminoso, de caule herbáceo, fôlhas pubescentes, e flôres amarelas raiadas de vermelho.
\section{Rinite}
\begin{itemize}
\item {Grp. gram.:f.}
\end{itemize}
\begin{itemize}
\item {Proveniência:(Do gr. \textunderscore rhis\textunderscore , \textunderscore rhinos\textunderscore , nariz)}
\end{itemize}
Inflamação da mucosa do nariz.
\section{Rinobronquite}
\begin{itemize}
\item {Grp. gram.:f.}
\end{itemize}
\begin{itemize}
\item {Utilização:Med.}
\end{itemize}
\begin{itemize}
\item {Proveniência:(De \textunderscore rhis\textunderscore , \textunderscore rhinos\textunderscore  gr. + \textunderscore bronchite\textunderscore )}
\end{itemize}
Inflamação das mucosas do nariz e dos brônquios.
\section{Rinocefalia}
\begin{itemize}
\item {Grp. gram.:f.}
\end{itemize}
Estado ou qualidade de rinocéfalo.
\section{Rinocéfalo}
\begin{itemize}
\item {Grp. gram.:adj.}
\end{itemize}
\begin{itemize}
\item {Proveniência:(Do gr. \textunderscore rhis\textunderscore , \textunderscore rhinos\textunderscore  + \textunderscore kephale\textunderscore )}
\end{itemize}
Que tem na abóbada craniana, para trás do bregma, uma deformação, á maneira de sela.
\section{Rinoceronte}
\begin{itemize}
\item {Grp. gram.:m.}
\end{itemize}
\begin{itemize}
\item {Proveniência:(Gr. \textunderscore rhinokeros\textunderscore )}
\end{itemize}
Grande quadrúpede selvagem da ordem dos paquidermes.
\section{Rinocerôntico}
\begin{itemize}
\item {Grp. gram.:adj.}
\end{itemize}
Relativo ao rinoceronte.
\section{Rinocerote}
\begin{itemize}
\item {Grp. gram.:m.}
\end{itemize}
(V.rinoceronte)
\section{Rinófido}
\begin{itemize}
\item {Grp. gram.:adj.}
\end{itemize}
\begin{itemize}
\item {Utilização:Zool.}
\end{itemize}
\begin{itemize}
\item {Proveniência:(Do gr. \textunderscore rhis\textunderscore , \textunderscore rhinos\textunderscore  + \textunderscore ophis\textunderscore )}
\end{itemize}
Dizia-se das serpentes, cujo focinho se prolonga em fórma de tromba.
\section{Rinofonia}
\begin{itemize}
\item {Grp. gram.:f.}
\end{itemize}
\begin{itemize}
\item {Proveniência:(Do gr. \textunderscore rhis\textunderscore , \textunderscore rhinos\textunderscore  + \textunderscore phone\textunderscore )}
\end{itemize}
Resonância da voz nas fossas nasaes.
\section{Rinólofo}
\begin{itemize}
\item {Grp. gram.:m.}
\end{itemize}
\begin{itemize}
\item {Proveniência:(Do gr. \textunderscore rhis\textunderscore , \textunderscore rhinos\textunderscore  + \textunderscore lophos\textunderscore )}
\end{itemize}
Gênero de morcegos, que tem sôbre o nariz uma crista membranosa, semelhante a uma ferradura. Cf. P. Moraes, \textunderscore Zool. Elem.\textunderscore , 175.
\section{Rinologia}
\begin{itemize}
\item {Grp. gram.:f.}
\end{itemize}
\begin{itemize}
\item {Proveniência:(Do gr. \textunderscore rhis\textunderscore , \textunderscore rhinos\textunderscore  + \textunderscore logos\textunderscore )}
\end{itemize}
Estudo anatómico do nariz.
\section{Rinómacro}
\begin{itemize}
\item {Grp. gram.:m.}
\end{itemize}
\begin{itemize}
\item {Proveniência:(Do gr. \textunderscore rhis\textunderscore , \textunderscore rhinos\textunderscore  + \textunderscore makros\textunderscore )}
\end{itemize}
Gênero de insectos, semelhantes ao gorgulho.
\section{Rinoplasta}
\begin{itemize}
\item {Grp. gram.:m.}
\end{itemize}
Aquele que pratíca a rinoplastia.
\section{Rinoplastia}
\begin{itemize}
\item {Grp. gram.:f.}
\end{itemize}
\begin{itemize}
\item {Proveniência:(Do gr. \textunderscore rhis\textunderscore , \textunderscore rhinos\textunderscore  + \textunderscore plassein\textunderscore )}
\end{itemize}
Operação cirúrgica, com que se substitue artificialmente o nariz ou parte do nariz.
\section{Rinoplástica}
\begin{itemize}
\item {Grp. gram.:f.}
\end{itemize}
(V.rinoplastia)
\section{Rinoplástico}
\begin{itemize}
\item {Grp. gram.:adj.}
\end{itemize}
Relativo á rinoplastia.
\section{Rinópomo}
\begin{itemize}
\item {Grp. gram.:m.}
\end{itemize}
\begin{itemize}
\item {Proveniência:(Do gr. \textunderscore rhis\textunderscore , \textunderscore rhinos\textunderscore  + \textunderscore poma\textunderscore )}
\end{itemize}
Gênero de mamíferos quirópteros, com as fossas nasaes providas de um lóbulo em fórma de opérculo.
\section{Rinoptia}
\begin{itemize}
\item {Grp. gram.:f.}
\end{itemize}
\begin{itemize}
\item {Proveniência:(Do gr. \textunderscore rhis\textunderscore , \textunderscore rhinos\textunderscore  + \textunderscore optomai\textunderscore )}
\end{itemize}
Estrabismo, em que a pupila, desviando-se do eixo visual, se aproxima do nariz.
\section{Rinorrafia}
\begin{itemize}
\item {Grp. gram.:f.}
\end{itemize}
\begin{itemize}
\item {Utilização:Med.}
\end{itemize}
\begin{itemize}
\item {Proveniência:(Do gr. \textunderscore rhis\textunderscore , \textunderscore rhinos\textunderscore  + \textunderscore graphein\textunderscore )}
\end{itemize}
Sutura dos bordos de uma chaga do nariz.
\section{Rinorragia}
\begin{itemize}
\item {Grp. gram.:f.}
\end{itemize}
\begin{itemize}
\item {Proveniência:(Do gr. \textunderscore rhis\textunderscore , \textunderscore rhinos\textunderscore  + \textunderscore rhein\textunderscore )}
\end{itemize}
Hemorragia nasal.
\section{Rinorrágico}
\begin{itemize}
\item {Grp. gram.:adj.}
\end{itemize}
Relativo á rinorragia.
\section{Rinorreia}
\begin{itemize}
\item {Grp. gram.:f.}
\end{itemize}
\begin{itemize}
\item {Proveniência:(Do gr. \textunderscore rhis\textunderscore , \textunderscore rhinos\textunderscore  + \textunderscore rhein\textunderscore )}
\end{itemize}
Fluxo de mucosidades límpidas pelo nariz, sem sintomas de inflamação.
\section{Rinoscopia}
\begin{itemize}
\item {Grp. gram.:f.}
\end{itemize}
\begin{itemize}
\item {Utilização:Med.}
\end{itemize}
Exame das fossas nasaes.
(Cp. \textunderscore rinoscópio\textunderscore )
\section{Rinoscópio}
\begin{itemize}
\item {Grp. gram.:m.}
\end{itemize}
\begin{itemize}
\item {Proveniência:(Do gr. \textunderscore rhis\textunderscore , \textunderscore rhinos\textunderscore  + \textunderscore skopein\textunderscore )}
\end{itemize}
Instrumento, para iluminar e deixar vêr as cavidades do nariz.
\section{Rinostegnose}
\begin{itemize}
\item {Grp. gram.:f.}
\end{itemize}
\begin{itemize}
\item {Utilização:Med.}
\end{itemize}
\begin{itemize}
\item {Proveniência:(Do gr. \textunderscore rhis\textunderscore , \textunderscore rhinos\textunderscore  + \textunderscore stegnosis\textunderscore )}
\end{itemize}
Obstrucção das fossas nasaes.
\section{Rinoteca}
\begin{itemize}
\item {Grp. gram.:f.}
\end{itemize}
\begin{itemize}
\item {Utilização:Zool.}
\end{itemize}
\begin{itemize}
\item {Proveniência:(Do gr. \textunderscore rhis\textunderscore , \textunderscore rhinos\textunderscore  + \textunderscore theke\textunderscore )}
\end{itemize}
Epiderme do bico das aves.
\section{Rinotriquia}
\begin{itemize}
\item {Grp. gram.:f.}
\end{itemize}
\begin{itemize}
\item {Proveniência:(Do gr. \textunderscore rhis\textunderscore , \textunderscore rhinos\textunderscore  + \textunderscore trix\textunderscore , \textunderscore trikhos\textunderscore )}
\end{itemize}
Exuberância de pêlos no nariz.
\section{Rinque}
\begin{itemize}
\item {Grp. gram.:m.}
\end{itemize}
Lugar, onde se patina.
(Do ingl.\textunderscore rink\textunderscore )
\section{Rinto}
\begin{itemize}
\item {Grp. gram.:m.}
\end{itemize}
Árvore da Índia portuguesa.
\section{Rio}
\begin{itemize}
\item {Grp. gram.:m.}
\end{itemize}
\begin{itemize}
\item {Utilização:Fig.}
\end{itemize}
\begin{itemize}
\item {Proveniência:(Do lat. \textunderscore rivus\textunderscore )}
\end{itemize}
Curso de água natural, permanente ou temporário, navegável ou não, procedente de uma fonte única ou formado pela reunião de regatos ou correntes mais pequenas, e que desagua em outro curso de água ou no mar.
Aquillo que corre como um rio.
Grande quantidade de líquido: \textunderscore chorou rios de lágrimas\textunderscore .
Grande quantidade: \textunderscore custou rios de dinheiro\textunderscore .
\section{Riodonorês}
\begin{itemize}
\item {Grp. gram.:m.}
\end{itemize}
\begin{itemize}
\item {Proveniência:(De \textunderscore Riodonor\textunderscore , n. p.)}
\end{itemize}
Dialecto trasmontano. Cf. G. Viana, \textunderscore Classificação das Línguas\textunderscore , 10.
\section{Rio-frio}
\begin{itemize}
\item {Grp. gram.:f.}
\end{itemize}
Variedade de pêra, parecida com a pêra correia.
\section{Rio-grandense-do-norte}
\begin{itemize}
\item {Grp. gram.:adj.}
\end{itemize}
\begin{itemize}
\item {Grp. gram.:M.  e  f.}
\end{itemize}
Relativo ao Estado do Rio-Grande-do-Norte.
Habitante dêsse Estado.
\section{Rio-grandense-do-sul}
\begin{itemize}
\item {Grp. gram.:adj.}
\end{itemize}
\begin{itemize}
\item {Grp. gram.:M.  e  f.}
\end{itemize}
Relativo ao Estado do Rio-Grande-do-Sul.
Habitante dêsse Estado.
\section{Ripa}
\begin{itemize}
\item {Grp. gram.:f.}
\end{itemize}
\begin{itemize}
\item {Utilização:Ant.}
\end{itemize}
O mesmo que \textunderscore riba\textunderscore .
\section{Ripa}
\begin{itemize}
\item {Grp. gram.:f.}
\end{itemize}
\begin{itemize}
\item {Utilização:Bras. do N}
\end{itemize}
Pedaço de madeira, estreito e comprido; sarrafo.
O mesmo que \textunderscore cacete\textunderscore .
(Do ingl.\textunderscore rip\textunderscore ?)
\section{Ripada}
\begin{itemize}
\item {Grp. gram.:f.}
\end{itemize}
\begin{itemize}
\item {Utilização:Extrem.}
\end{itemize}
\begin{itemize}
\item {Proveniência:(De \textunderscore ripa\textunderscore ^2)}
\end{itemize}
Pancada com ripa.
Bordoada. Cf. Camillo, \textunderscore Volcoens\textunderscore , 25.
\section{Ripadeira}
\begin{itemize}
\item {Grp. gram.:f.}
\end{itemize}
\begin{itemize}
\item {Proveniência:(De \textunderscore ripar\textunderscore ^2)}
\end{itemize}
Instrumento, para ripar ou esbagoar uvas.
Instrumento de metal, com que se ripa abóbora para doce.
\section{Ripado}
\begin{itemize}
\item {Grp. gram.:m.}
\end{itemize}
\begin{itemize}
\item {Proveniência:(De \textunderscore ripar\textunderscore ^1)}
\end{itemize}
Gradeamento ou tapume de ripas.
\section{Ripadura}
\begin{itemize}
\item {Grp. gram.:f.}
\end{itemize}
Acto de ripar^3.
\section{Ripagem}
\begin{itemize}
\item {Grp. gram.:f.}
\end{itemize}
O mesmo que \textunderscore ripadura\textunderscore .
\section{Ripal}
\begin{itemize}
\item {Grp. gram.:adj.}
\end{itemize}
Diz-se de um prego pequeno, próprio para pregar ripas^2.
\section{Ripamento}
\begin{itemize}
\item {Grp. gram.:m.}
\end{itemize}
\begin{itemize}
\item {Utilização:Bras}
\end{itemize}
\begin{itemize}
\item {Proveniência:(De \textunderscore ripar\textunderscore ^1)}
\end{itemize}
O mesmo que \textunderscore ripagem\textunderscore .
\section{Ripançar}
\begin{itemize}
\item {Grp. gram.:v. t.}
\end{itemize}
\begin{itemize}
\item {Proveniência:(De \textunderscore ripanço\textunderscore ^1)}
\end{itemize}
O mesmo que \textunderscore ripar\textunderscore ^2 (o linho).
\section{Ripanço}
\begin{itemize}
\item {Grp. gram.:m.}
\end{itemize}
\begin{itemize}
\item {Utilização:Ext.}
\end{itemize}
\begin{itemize}
\item {Utilização:Pop.}
\end{itemize}
\begin{itemize}
\item {Proveniência:(De \textunderscore ripar\textunderscore ^2)}
\end{itemize}
Instrumento, para ripar o linho ou limpá-lo da baganha.
Utensílio, com que os hortelões raspam a terra e juntam as pedras.
Espécie de sofá, em que se descansa ou se dorme a sesta.
Indolência, mandriice.
Pessôa indolente.
Livro dos Offícios da Semana Santa.
\section{Ripanço}
\begin{itemize}
\item {Grp. gram.:m.}
\end{itemize}
(V.\textunderscore raponço\textunderscore )
\section{Ripar}
\begin{itemize}
\item {Grp. gram.:v. t.}
\end{itemize}
Gradear com ripas^2; pregar ripas em.
Serrar, formando ripas.
\section{Ripar}
\begin{itemize}
\item {Grp. gram.:v. t.}
\end{itemize}
\begin{itemize}
\item {Utilização:Bras. da Baía}
\end{itemize}
\begin{itemize}
\item {Utilização:T. da Bairrada}
\end{itemize}
\begin{itemize}
\item {Grp. gram.:V. i.}
\end{itemize}
\begin{itemize}
\item {Proveniência:(Do al. \textunderscore rippen\textunderscore )}
\end{itemize}
Separar a baganha de (o linho).
Raspar ou limpar (a terra).
Cortar cerce (as crinas do cavallo).
Em caminhos de ferro, desviar (uma via férrea) do alinhamento que tem, para lhe dar outro.
Puxar ou arrancar (os cabellos).
Surripiar; bifar.
Deitar a mão, tirando com fôrça.
\section{Ripária}
\begin{itemize}
\item {Grp. gram.:f.  e  adj.}
\end{itemize}
\begin{itemize}
\item {Proveniência:(Lat. \textunderscore riparia\textunderscore )}
\end{itemize}
Diz-se de uma casta de videira, cujos cachos têm pequenos bagos.
\section{Ripário}
\begin{itemize}
\item {Grp. gram.:adj.}
\end{itemize}
\begin{itemize}
\item {Utilização:bras}
\end{itemize}
\begin{itemize}
\item {Utilização:Neol.}
\end{itemize}
\begin{itemize}
\item {Proveniência:(Lat. \textunderscore riparius\textunderscore )}
\end{itemize}
O mesmo que \textunderscore marginal\textunderscore . Cf.\textunderscore Jorn.-do-Comm.\textunderscore , do Rio, de 3-V-902.
\section{Ripeira}
\begin{itemize}
\item {Grp. gram.:f.}
\end{itemize}
\begin{itemize}
\item {Utilização:Prov.}
\end{itemize}
\begin{itemize}
\item {Utilização:Fam.}
\end{itemize}
O mesmo que \textunderscore ripa\textunderscore ^2.
Fasque; sarrafo.
Espada velha, chanfalho.
\section{Ripeiro}
\begin{itemize}
\item {Utilização:T. de Oliv. de Azeméis}
\end{itemize}
\begin{itemize}
\item {Utilização:T. da Bairrada}
\end{itemize}
\begin{itemize}
\item {Proveniência:(De \textunderscore ripa\textunderscore ^2)}
\end{itemize}
O mesmo que \textunderscore chicote\textunderscore .
Madeiro serrado em ripas, em-quanto as conserva unidas.
\section{Ripeta}
\begin{itemize}
\item {fónica:pê}
\end{itemize}
\textunderscore f. Bras.\textunderscore ?
Pequena ripa. Cf. Dom. Vieira, \textunderscore Thes. da Ling.\textunderscore , vb.\textunderscore araçá do Mato Grosso\textunderscore .
\section{Ripiado}
\begin{itemize}
\item {Grp. gram.:adj.}
\end{itemize}
Que tem rípios.
\section{Ripícera}
\begin{itemize}
\item {Grp. gram.:f.}
\end{itemize}
Gênero de insectos coleópteros pentâmeros.
\section{Ripícola}
\begin{itemize}
\item {Grp. gram.:adj.}
\end{itemize}
\begin{itemize}
\item {Proveniência:(Do lat. \textunderscore ripa\textunderscore  + \textunderscore colere\textunderscore )}
\end{itemize}
Que vive á beira dos rios ou nas ribeiras.
\section{Ripidólito}
\begin{itemize}
\item {Grp. gram.:m.}
\end{itemize}
\begin{itemize}
\item {Utilização:Miner.}
\end{itemize}
\begin{itemize}
\item {Proveniência:(Do gr. \textunderscore rhipis\textunderscore , \textunderscore rhipidos\textunderscore  + \textunderscore lithos\textunderscore )}
\end{itemize}
Variedade de clorito, pouco transparente.
\section{Ripidura}
\begin{itemize}
\item {Grp. gram.:f.}
\end{itemize}
\begin{itemize}
\item {Proveniência:(Do gr. \textunderscore rhipis\textunderscore , \textunderscore rhipidos\textunderscore  + \textunderscore oura\textunderscore )}
\end{itemize}
Gênero de pássaros.
\section{Ripíforo}
\begin{itemize}
\item {Grp. gram.:m.}
\end{itemize}
\begin{itemize}
\item {Proveniência:(Do gr. \textunderscore rhipis\textunderscore  + \textunderscore phoros\textunderscore )}
\end{itemize}
Gênero de insectos coleópteros heterómeros.
\section{Ripiglossa}
\begin{itemize}
\item {Grp. gram.:f.}
\end{itemize}
\begin{itemize}
\item {Proveniência:(Do gr. \textunderscore rhipis\textunderscore  + \textunderscore glossa\textunderscore )}
\end{itemize}
Gênero de plantas acantáceas.
\section{Rípio}
\begin{itemize}
\item {Grp. gram.:m.}
\end{itemize}
\begin{itemize}
\item {Utilização:Fig.}
\end{itemize}
Pedra miúda, com que se enchem os vãos que as grande pedras deixam na construcção das paredes.
Cascalho.
Palavra, que entra no verso, só para lhe completar a medida.
(Cp. cast. \textunderscore ripio\textunderscore )
\section{Ripípteros}
\begin{itemize}
\item {Grp. gram.:m. pl.}
\end{itemize}
\begin{itemize}
\item {Proveniência:(Do gr. \textunderscore rhipis\textunderscore  + \textunderscore pteron\textunderscore )}
\end{itemize}
Ordem de insectos, no sistema de Latreille.
\section{Ripo}
\begin{itemize}
\item {Grp. gram.:m.}
\end{itemize}
\begin{itemize}
\item {Utilização:Prov.}
\end{itemize}
\begin{itemize}
\item {Utilização:trasm.}
\end{itemize}
\begin{itemize}
\item {Proveniência:(De \textunderscore ripar\textunderscore ^2)}
\end{itemize}
Ripanço do linho, de tábuas dentadas para arrancar a baganha do linho.
\section{Riposta}
\begin{itemize}
\item {Grp. gram.:f.}
\end{itemize}
Acto de ripostar.
\section{Ripostar}
\begin{itemize}
\item {Grp. gram.:v. i.}
\end{itemize}
\begin{itemize}
\item {Proveniência:(Fr. \textunderscore riposter\textunderscore )}
\end{itemize}
Rebater a estocada, no jôgo de esgrima.
\section{Rique}
\begin{itemize}
\item {Grp. gram.:m.}
\end{itemize}
Espécie de aguardente, entre os Índios.
\section{Riqueifa}
\begin{itemize}
\item {Grp. gram.:f.}
\end{itemize}
O mesmo que \textunderscore regueifa\textunderscore . Cf. Dozy.
\section{Riqueza}
\begin{itemize}
\item {Grp. gram.:f.}
\end{itemize}
\begin{itemize}
\item {Utilização:Ext.}
\end{itemize}
Qualidade do que é rico.
Abundância; fertilidade.
Magnificência.
Abundância de recursos intellectuaes.
Aquillo que é muito productivo.
A classe dos ricos.
\section{Riquinho}
\begin{itemize}
\item {Grp. gram.:adj.}
\end{itemize}
\begin{itemize}
\item {Utilização:Açor}
\end{itemize}
Bonito, formoso.
\section{Rir}
\begin{itemize}
\item {Grp. gram.:v. i.}
\end{itemize}
\begin{itemize}
\item {Utilização:Pop.}
\end{itemize}
\begin{itemize}
\item {Utilização:Fig.}
\end{itemize}
\begin{itemize}
\item {Utilização:Gír.}
\end{itemize}
\begin{itemize}
\item {Grp. gram.:V. t.}
\end{itemize}
\begin{itemize}
\item {Utilização:P. us.}
\end{itemize}
\begin{itemize}
\item {Grp. gram.:V. p.}
\end{itemize}
\begin{itemize}
\item {Grp. gram.:M.}
\end{itemize}
\begin{itemize}
\item {Proveniência:(Do lat. \textunderscore ridere\textunderscore )}
\end{itemize}
Contrahir os músculos faciaes, geralmente em consequência de uma impressão alegre.
Mostrar fenda ou rasgão (uma peça de vestuário): \textunderscore traz as calças a rir, do lado de trás\textunderscore .
Gracejar: \textunderscore não foi por mal, foi a rir\textunderscore .
Tratar levianamente de algum assumpto.
Escarnecer: \textunderscore rir dos parvos\textunderscore .
Tinir ou tlintar.
Zombar de.
(A mesma sign. do \textunderscore v. i.\textunderscore )
Acto de rir; riso.
\section{Risa}
\begin{itemize}
\item {Grp. gram.:f.}
\end{itemize}
\begin{itemize}
\item {Utilização:Prov.}
\end{itemize}
\begin{itemize}
\item {Utilização:trasm.}
\end{itemize}
\begin{itemize}
\item {Utilização:Ant.}
\end{itemize}
Riso; risada.
(Cast. \textunderscore risa\textunderscore )
\section{Risa}
\begin{itemize}
\item {Grp. gram.:f.}
\end{itemize}
\begin{itemize}
\item {Utilização:Ant.}
\end{itemize}
Espécie de taficira.
\section{Risada}
\begin{itemize}
\item {Grp. gram.:f.}
\end{itemize}
\begin{itemize}
\item {Proveniência:(De \textunderscore riso\textunderscore )}
\end{itemize}
Riso; gargalhada.
Riso simultâneo de muita gente.
\section{Risanza}
\begin{itemize}
\item {Grp. gram.:f.}
\end{itemize}
Planta herbácea do Golungo-Alto.
\section{Risbordo}
\begin{itemize}
\item {Grp. gram.:m.}
\end{itemize}
Portinhola do navio, ao nível da água, para introduzir objectos que não entram pela escotilha.
\section{Risca}
\begin{itemize}
\item {Grp. gram.:f.}
\end{itemize}
\begin{itemize}
\item {Utilização:Gír.}
\end{itemize}
\begin{itemize}
\item {Grp. gram.:Loc. adv.}
\end{itemize}
Acto ou effeito de riscar.
Traço, sulco.
Carreiro, entre as marrafas de cabello.
Meta, no jôgo.
Desordem.
\textunderscore Á risca\textunderscore , com rigor, com pontualidade, justamente.
\section{Risca}
\begin{itemize}
\item {Grp. gram.:f.}
\end{itemize}
\begin{itemize}
\item {Utilização:Prov.}
\end{itemize}
\begin{itemize}
\item {Utilização:trasm.}
\end{itemize}
O mesmo que \textunderscore serradura\textunderscore .
\section{Riscada}
\begin{itemize}
\item {Grp. gram.:f.}
\end{itemize}
Nome que, na Maia, se dá á milheira.
\section{Riscadeira}
\begin{itemize}
\item {Grp. gram.:f.}
\end{itemize}
\begin{itemize}
\item {Proveniência:(De \textunderscore riscar\textunderscore )}
\end{itemize}
Utensílio para fazer guillochés, nas fábricas de tabacos. Cf.\textunderscore Inquér. Industr.\textunderscore , p. II, l. II, 80.
\section{Riscadinha}
\begin{itemize}
\item {Grp. gram.:f.}
\end{itemize}
\begin{itemize}
\item {Proveniência:(De \textunderscore riscado\textunderscore )}
\end{itemize}
Variedade de pêra amanteigada e doce, cuja pelle é ligeiramente estriada de amarelo.
\section{Riscadinho}
\begin{itemize}
\item {Grp. gram.:m.}
\end{itemize}
Variedade de pêro.
\section{Riscado}
\begin{itemize}
\item {Grp. gram.:m.}
\end{itemize}
\begin{itemize}
\item {Proveniência:(De \textunderscore riscar\textunderscore )}
\end{itemize}
Tecido de linho ou algodão, com listras de côr.
Casta de uva, na região do Doiro.
\section{Riscador}
\begin{itemize}
\item {Grp. gram.:m.  e  adj.}
\end{itemize}
\begin{itemize}
\item {Grp. gram.:M.}
\end{itemize}
O que risca.
Instrumento para riscar.
\section{Riscadura}
\begin{itemize}
\item {Grp. gram.:f.}
\end{itemize}
O mesmo que \textunderscore risca\textunderscore ^1.
\section{Riscamento}
\begin{itemize}
\item {Grp. gram.:m.}
\end{itemize}
O mesmo que \textunderscore risca\textunderscore ^1.
\section{Riscanhada}
\begin{itemize}
\item {Grp. gram.:f.}
\end{itemize}
\begin{itemize}
\item {Utilização:T. da Bairrada}
\end{itemize}
\begin{itemize}
\item {Proveniência:(De um hyp.\textunderscore riscanhar\textunderscore , de \textunderscore riscar\textunderscore )}
\end{itemize}
Garatujas, gatafunhos.
\section{Riscar}
\begin{itemize}
\item {Grp. gram.:v. t.}
\end{itemize}
\begin{itemize}
\item {Utilização:Fig.}
\end{itemize}
\begin{itemize}
\item {Utilização:Des.}
\end{itemize}
\begin{itemize}
\item {Grp. gram.:V. i.}
\end{itemize}
\begin{itemize}
\item {Utilização:Pop.}
\end{itemize}
\begin{itemize}
\item {Utilização:Gír.}
\end{itemize}
\begin{itemize}
\item {Utilização:T. de Turquel}
\end{itemize}
\begin{itemize}
\item {Proveniência:(Do lat. \textunderscore resecare\textunderscore ?)}
\end{itemize}
Fazer traços em.
Apagar com traços.
Delinear, traçar.
Marcar: \textunderscore riscar as estremas\textunderscore .
Excluír; expungir: \textunderscore riscar um sócio do clube\textunderscore .
O mesmo que \textunderscore arriscar\textunderscore .
Perder a amizade de alguém: \textunderscore aquelle, para mim, riscou\textunderscore .
Provocar; brigar.
Manobrar com a navalha, antes de dar facada: \textunderscore os fadistas puseram-se a riscar\textunderscore .
Delinear os primeiros passos do fandango.
Dar ordens, autoritariamente.
\section{Risco}
\begin{itemize}
\item {Grp. gram.:m.}
\end{itemize}
\begin{itemize}
\item {Utilização:Gír.}
\end{itemize}
\begin{itemize}
\item {Proveniência:(De \textunderscore riscar\textunderscore )}
\end{itemize}
O mesmo que \textunderscore risca\textunderscore ^1.
Delineamento, traçado: \textunderscore fazer o risco de um edifício\textunderscore .
Debuxo.
Facada.
\section{Risco}
\begin{itemize}
\item {Grp. gram.:m.}
\end{itemize}
Perigo; probabilidade ou possibilidade de perigo: \textunderscore estar em risco\textunderscore .
\section{Riscoso}
\begin{itemize}
\item {Grp. gram.:adj.}
\end{itemize}
Em que há risco^2 ou perigo; arriscado:«\textunderscore ...riscoso a hum ponto de perderme\textunderscore ». Usque, 49.
\section{Riscote}
\begin{itemize}
\item {Grp. gram.:m.}
\end{itemize}
Instrumento, com que o chapeleiro risca e molda as abas do chapéu.
\section{Riscunho}
\begin{itemize}
\item {Grp. gram.:m.}
\end{itemize}
\begin{itemize}
\item {Utilização:Des.}
\end{itemize}
O mesmo que \textunderscore rascunho\textunderscore . Cf. C. Guerreiro, \textunderscore Diccion. de Consoantes\textunderscore .
(Infl. de \textunderscore risco\textunderscore )
\section{Risibilidade}
\begin{itemize}
\item {Grp. gram.:f.}
\end{itemize}
\begin{itemize}
\item {Proveniência:(Lat. \textunderscore risibilitas\textunderscore )}
\end{itemize}
Qualidade do que é risível.
\section{Risível}
\begin{itemize}
\item {Grp. gram.:adj.}
\end{itemize}
\begin{itemize}
\item {Grp. gram.:M.}
\end{itemize}
\begin{itemize}
\item {Proveniência:(Do lat. \textunderscore risibilis\textunderscore )}
\end{itemize}
Digno de riso; ridículo.
Que faz rir.
Aquillo que é ridículo.
\section{Risivelmente}
\begin{itemize}
\item {Grp. gram.:adv.}
\end{itemize}
De modo risível.
\section{Riso}
\begin{itemize}
\item {Grp. gram.:m.}
\end{itemize}
\begin{itemize}
\item {Proveniência:(Lat. \textunderscore risus\textunderscore )}
\end{itemize}
Acto ou effeito de rir.
Alegria; zombaria.
Coisa ridícula.
\textunderscore Meter a riso\textunderscore , ridiculizar. Cf. Camillo, \textunderscore Mulher Fatal\textunderscore , 110.
\section{Risonhamente}
\begin{itemize}
\item {Grp. gram.:adv.}
\end{itemize}
De modo risonho; alegremente; prazenteiramente.
\section{Risonho}
\begin{itemize}
\item {Grp. gram.:adj.}
\end{itemize}
\begin{itemize}
\item {Proveniência:(Do cast. \textunderscore risueño\textunderscore )}
\end{itemize}
Que sorri.
Alegre, satisfeito.
Agradável, próspero: \textunderscore vida risonha\textunderscore .
\section{Risório}
\begin{itemize}
\item {Grp. gram.:m.  e  adj.}
\end{itemize}
\begin{itemize}
\item {Utilização:Anat.}
\end{itemize}
\begin{itemize}
\item {Proveniência:(Lat. \textunderscore risorius\textunderscore )}
\end{itemize}
Fascículo muscular, que vem da aponevrose do masséter, e se junta ao músculo triangular dos lábios.
\section{Risota}
\begin{itemize}
\item {Grp. gram.:f.}
\end{itemize}
\begin{itemize}
\item {Utilização:Pop.}
\end{itemize}
\begin{itemize}
\item {Proveniência:(De \textunderscore riso\textunderscore )}
\end{itemize}
Risada; riso de escárneo; escárneo; galhofa: \textunderscore vá de risota\textunderscore .
\section{Risote}
\begin{itemize}
\item {Grp. gram.:m.  e  adj.}
\end{itemize}
\begin{itemize}
\item {Proveniência:(De \textunderscore riso\textunderscore )}
\end{itemize}
O que zomba de tudo, até das coisas respeitáveis.
\section{Rispidamente}
\begin{itemize}
\item {Grp. gram.:adv.}
\end{itemize}
De modo ríspido; com severidade.
\section{Rispidez}
\begin{itemize}
\item {Grp. gram.:f.}
\end{itemize}
Qualidade do que é ríspido.
\section{Rispideza}
\begin{itemize}
\item {Grp. gram.:f.}
\end{itemize}
O mesmo que \textunderscore rispidez\textunderscore .
\section{Ríspido}
\begin{itemize}
\item {Grp. gram.:adj.}
\end{itemize}
\begin{itemize}
\item {Proveniência:(Do lat. \textunderscore hispidus\textunderscore )}
\end{itemize}
Áspero; severo; intratável.
Rijo.
\section{Riste}
\begin{itemize}
\item {Grp. gram.:m.}
\end{itemize}
Peça de ferro, em que o cavalleiro embebe o conto da lança, quando a leva horizontalmente, para investir.
(Cp.\textunderscore reste\textunderscore ^1)
\section{Rita}
\begin{itemize}
\item {Grp. gram.:f.}
\end{itemize}
\begin{itemize}
\item {Utilização:Prov.}
\end{itemize}
\begin{itemize}
\item {Utilização:trasm.}
\end{itemize}
A espuma do leite. (Colhido em Miranda)
(Relaciona-se com o cast. \textunderscore rita\textunderscore , grito de pastores?)
\section{Ritaforme}
\begin{itemize}
\item {Grp. gram.:m.}
\end{itemize}
Espécie de tartaranhão, (\textunderscore circus cyaneus\textunderscore ).
\section{Rítio}
\begin{itemize}
\item {Grp. gram.:m.}
\end{itemize}
\begin{itemize}
\item {Proveniência:(Lat. \textunderscore rhytion\textunderscore )}
\end{itemize}
Vaso antigo, em fórma de buzina, usado pelos Gregos.
\section{Ritmómetro}
\begin{itemize}
\item {Grp. gram.:m.}
\end{itemize}
\begin{itemize}
\item {Proveniência:(Do gr. \textunderscore rhuthmos\textunderscore  + \textunderscore metron\textunderscore )}
\end{itemize}
Antigo instrumento mecânico, com que se indicava o compasso da música e que foi substituído pelo metrónomo.
\section{Ritmopeia}
\begin{itemize}
\item {Grp. gram.:f.}
\end{itemize}
\begin{itemize}
\item {Proveniência:(Gr. \textunderscore rhutmopoia\textunderscore )}
\end{itemize}
Arte do ritmo.
\section{Rito}
\begin{itemize}
\item {Grp. gram.:m.}
\end{itemize}
\begin{itemize}
\item {Proveniência:(Lat. \textunderscore ritus\textunderscore )}
\end{itemize}
Conjunto de ceremónias, que se praticam numa religião.
Culto.
Seita.
Ceremónias de uma seita.
Quaesquer ceremónias.
Cada um dos systemas de organização maçónica: \textunderscore o rito escocês\textunderscore , \textunderscore o rito francês...\textunderscore 
\section{Ritornello}
\begin{itemize}
\item {Grp. gram.:m.}
\end{itemize}
\begin{itemize}
\item {Utilização:Fig.}
\end{itemize}
\begin{itemize}
\item {Proveniência:(It. \textunderscore ritornello\textunderscore )}
\end{itemize}
Verso ou versos, que se repetem no fim de cada estrophe de uma composição.
Estribilho.
Prelúdio musical, que se repete no decurso de uma composição.
Coisa muito repetida.
\section{Ritornelo}
\begin{itemize}
\item {Grp. gram.:m.}
\end{itemize}
\begin{itemize}
\item {Utilização:Fig.}
\end{itemize}
\begin{itemize}
\item {Proveniência:(It. \textunderscore ritornello\textunderscore )}
\end{itemize}
Verso ou versos, que se repetem no fim de cada estrofe de uma composição.
Estribilho.
Prelúdio musical, que se repete no decurso de uma composição.
Coisa muito repetida.
\section{Ritual}
\begin{itemize}
\item {Grp. gram.:adj.}
\end{itemize}
\begin{itemize}
\item {Grp. gram.:M.}
\end{itemize}
\begin{itemize}
\item {Proveniência:(Lat. \textunderscore ritualis\textunderscore )}
\end{itemize}
Relativo a ritos.
Livro, que indica os ritos ou consigna as fórmas, que se devem observar na prática das ceremónias de uma religião.
Ceremonial.
\section{Ritualismo}
\begin{itemize}
\item {Grp. gram.:m.}
\end{itemize}
\begin{itemize}
\item {Proveniência:(De \textunderscore ritual\textunderscore )}
\end{itemize}
Conjunto de ritos.
Apêgo ás ceremónias ou formalidades.
\section{Ritualista}
\begin{itemize}
\item {Grp. gram.:m.}
\end{itemize}
\begin{itemize}
\item {Grp. gram.:M. ,  f.  e  adj.}
\end{itemize}
\begin{itemize}
\item {Proveniência:(De \textunderscore ritual\textunderscore )}
\end{itemize}
O que trata ou escreve á cêrca de ritos.
Pessôa, que tem grande apêgo a ceremónias ou fórmulas.
\section{Ritualmente}
\begin{itemize}
\item {Grp. gram.:adv.}
\end{itemize}
\begin{itemize}
\item {Proveniência:(De \textunderscore ritual\textunderscore )}
\end{itemize}
Em harmonia com o rito.
\section{Ritumba}
\begin{itemize}
\item {Grp. gram.:f.}
\end{itemize}
Espécie de tambor africano, feito de um tronco, interiormente aberto, e tendo uma pelle a tapar-lhe uma das aberturas.
\section{Riúta}
\begin{itemize}
\item {Grp. gram.:f.}
\end{itemize}
Cobra venenosa de Angola.
(Do quimb.)
\section{Rival}
\begin{itemize}
\item {Grp. gram.:adj.}
\end{itemize}
\begin{itemize}
\item {Grp. gram.:M.  e  f.}
\end{itemize}
\begin{itemize}
\item {Proveniência:(Lat. \textunderscore rivalis\textunderscore )}
\end{itemize}
Que rivaliza.
Que aspira ás mesmas vantagens que outrem.
Competidor.
Pessôa rival.
\section{Rivalidade}
\begin{itemize}
\item {Grp. gram.:f.}
\end{itemize}
\begin{itemize}
\item {Proveniência:(Lat. \textunderscore rivalitas\textunderscore )}
\end{itemize}
Qualidade de quem rivaliza ou de quem é rival.
\section{Rivalizar}
\begin{itemize}
\item {Grp. gram.:v. i.}
\end{itemize}
\begin{itemize}
\item {Grp. gram.:V. t.}
\end{itemize}
\begin{itemize}
\item {Grp. gram.:V. p.}
\end{itemize}
\begin{itemize}
\item {Proveniência:(De \textunderscore rival\textunderscore )}
\end{itemize}
Disputar ou pleitear com alguém, á cêrca de títulos, qualidades, etc.
Entrar em competencia.
Disputar primazias.
Aproximar-se de outro, em méritos: \textunderscore Camões e o Tasso rivalizaram\textunderscore .
Têr ciúmes.
Igualar ou procurar igualar.
Fazer entrar em competência.
Rivalizar reciprocamente:«\textunderscore verá quantos maridos da melhor nobreza se rivalizam\textunderscore ». Camillo, \textunderscore Caveira\textunderscore , 246.
\section{Rivalizável}
\begin{itemize}
\item {Grp. gram.:adj.}
\end{itemize}
\begin{itemize}
\item {Proveniência:(De \textunderscore rivalizar\textunderscore )}
\end{itemize}
Que póde têr rival; que soffre confronto.
\section{Rivulária}
\begin{itemize}
\item {Grp. gram.:f.}
\end{itemize}
\begin{itemize}
\item {Proveniência:(Do lat. \textunderscore rivulus\textunderscore )}
\end{itemize}
Gênero de plantas phýceas, que crescem á beira da água.
\section{Rixa}
\begin{itemize}
\item {Grp. gram.:f.}
\end{itemize}
\begin{itemize}
\item {Proveniência:(Lat. \textunderscore rixa\textunderscore )}
\end{itemize}
Contenda; briga; desordem.
Discussão acalorada.
Discórdia.
\section{Rixa-á-rixa}
\begin{itemize}
\item {Grp. gram.:f.}
\end{itemize}
\begin{itemize}
\item {Utilização:T. de Albergaria-a-Velha}
\end{itemize}
Ave, o mesmo que \textunderscore pinta-cardeira\textunderscore .
\section{Rixador}
\begin{itemize}
\item {Grp. gram.:m.  e  adj.}
\end{itemize}
\begin{itemize}
\item {Proveniência:(Do lat. \textunderscore rixator\textunderscore )}
\end{itemize}
O que rixa; brigão, desordeiro.
\section{Rixar}
\begin{itemize}
\item {Grp. gram.:v. i.}
\end{itemize}
\begin{itemize}
\item {Proveniência:(Lat. \textunderscore rixari\textunderscore )}
\end{itemize}
Têr rixas com alguém.
Sêr provocante ou desordeiro.
\section{Rixar}
\begin{itemize}
\item {Grp. gram.:v. t.}
\end{itemize}
Limpar da carne adherente (o sebo)?:«\textunderscore as officinas de rixar o sebo são geralmente nos matadoiros\textunderscore  ou \textunderscore em ruas excêntricas...\textunderscore »Cf.\textunderscore Inquér. Industr.\textunderscore , p. II, l. II, 222.
\section{Rixoso}
\begin{itemize}
\item {Grp. gram.:adj.}
\end{itemize}
\begin{itemize}
\item {Proveniência:(Lat. \textunderscore rixosus\textunderscore )}
\end{itemize}
Que rixa; desordeiro; bulhento; brigão.
\section{Rizadura}
\begin{itemize}
\item {Grp. gram.:f.}
\end{itemize}
\begin{itemize}
\item {Utilização:Náut.}
\end{itemize}
Acto de rizar.
Cabo náutico de três cordões de linho.
\section{Rizagra}
\begin{itemize}
\item {Grp. gram.:f.}
\end{itemize}
\begin{itemize}
\item {Proveniência:(Do gr. \textunderscore rhiza\textunderscore  + \textunderscore agra\textunderscore )}
\end{itemize}
Instrumento, próprio para extrair as raízes dos dentes.
\section{Rizânteas}
\begin{itemize}
\item {Grp. gram.:f. pl.}
\end{itemize}
\begin{itemize}
\item {Proveniência:(De \textunderscore rizanto\textunderscore )}
\end{itemize}
O mesmo que \textunderscore citíneas\textunderscore .
\section{Rizanto}
\begin{itemize}
\item {Grp. gram.:adj.}
\end{itemize}
\begin{itemize}
\item {Proveniência:(Do gr. \textunderscore rhiza\textunderscore  + \textunderscore anthos\textunderscore )}
\end{itemize}
Diz-se das plantas, cujas flôres ou pedúnculos nascem da raíz.
\section{Rizar}
\begin{itemize}
\item {Grp. gram.:v. i.}
\end{itemize}
\begin{itemize}
\item {Utilização:Náut.}
\end{itemize}
\begin{itemize}
\item {Grp. gram.:V. t.}
\end{itemize}
Colher os rizes.
O mesmo que \textunderscore enrizar\textunderscore .
\section{Rizes}
\begin{itemize}
\item {Grp. gram.:m. pl.}
\end{itemize}
\begin{itemize}
\item {Utilização:Náut.}
\end{itemize}
\begin{itemize}
\item {Proveniência:(Do it. \textunderscore rizza\textunderscore )}
\end{itemize}
Pedaços do cabo delgado, fixos na vela do navio pelo seio, e atravessando-a em ilhós abertos nas costuras dos panos, com o fim de se deminuír a superfície da vela quando é preciso.
Ilhós, por onde passam êsses cabos.
\section{Rizóbia}
\begin{itemize}
\item {Grp. gram.:f.}
\end{itemize}
\begin{itemize}
\item {Proveniência:(Do gr. \textunderscore rhiza\textunderscore  + \textunderscore bios\textunderscore )}
\end{itemize}
Gênero de insectos coleópteros pentâmeros.
\section{Rizoblasto}
\begin{itemize}
\item {Grp. gram.:m.}
\end{itemize}
\begin{itemize}
\item {Utilização:Bot.}
\end{itemize}
\begin{itemize}
\item {Proveniência:(Do gr. \textunderscore rhiza\textunderscore  + \textunderscore blastos\textunderscore )}
\end{itemize}
Embrião, que tem uma só raíz.
\section{Rizoboláceas}
\begin{itemize}
\item {Grp. gram.:f. pl.}
\end{itemize}
Família de plantas, que tem por tipo o rizóbolo.
(Fem. pl. de \textunderscore rizoboláceo\textunderscore )
\section{Rizoboláceo}
\begin{itemize}
\item {Grp. gram.:adj.}
\end{itemize}
Relativo ou semelhante ao rizóbolo.
\section{Rizóbolo}
\begin{itemize}
\item {Grp. gram.:m.}
\end{itemize}
\begin{itemize}
\item {Proveniência:(Do gr. \textunderscore rhiza\textunderscore  + \textunderscore bolos\textunderscore )}
\end{itemize}
Gênero de plantas da América do Sul.
\section{Rizocárpico}
\begin{itemize}
\item {Grp. gram.:adj.}
\end{itemize}
Relativo ao vegetal rizocárpio.
\section{Rizocárpio}
\begin{itemize}
\item {Grp. gram.:adj.}
\end{itemize}
\begin{itemize}
\item {Proveniência:(Do gr. \textunderscore rhiza\textunderscore  + \textunderscore karpos\textunderscore )}
\end{itemize}
Diz-se dos vegetaes, de cuja raíz sáem em cada anno novas hastes fructíferas.
\section{Rizocarpo}
\begin{itemize}
\item {Grp. gram.:adj.}
\end{itemize}
O mesmo que \textunderscore rizocárpio\textunderscore .
\section{Rizocéfalos}
\begin{itemize}
\item {Grp. gram.:m. pl.}
\end{itemize}
\begin{itemize}
\item {Utilização:Zool.}
\end{itemize}
\begin{itemize}
\item {Proveniência:(Do gr. \textunderscore rhiza\textunderscore  + \textunderscore kephale\textunderscore )}
\end{itemize}
Grupo de crustáceos, cuja cabeça emitte prolongamentos ocos.
\section{Rizodo}
\begin{itemize}
\item {Grp. gram.:m.}
\end{itemize}
\begin{itemize}
\item {Proveniência:(Do gr. \textunderscore rhiza\textunderscore  + \textunderscore eidos\textunderscore )}
\end{itemize}
Gênero de insectos coleópteros pentâmeros.
\section{Rizofagia}
\begin{itemize}
\item {Grp. gram.:f.}
\end{itemize}
Qualidade de rizófago.
\section{Rizófago}
\begin{itemize}
\item {Grp. gram.:adj.}
\end{itemize}
\begin{itemize}
\item {Proveniência:(Do gr. \textunderscore rhiza\textunderscore  + \textunderscore phagein\textunderscore )}
\end{itemize}
Que se alimenta de raízes.
\section{Rizofilo}
\begin{itemize}
\item {Grp. gram.:adj.}
\end{itemize}
\begin{itemize}
\item {Proveniência:(Do gr. \textunderscore rhiza\textunderscore  + \textunderscore phullon\textunderscore )}
\end{itemize}
Cujas fôlhas produzem raízes.
\section{Rizófilo}
\begin{itemize}
\item {Grp. gram.:adj.}
\end{itemize}
\begin{itemize}
\item {Grp. gram.:M. pl.}
\end{itemize}
\begin{itemize}
\item {Proveniência:(Do gr. \textunderscore rhiza\textunderscore  + \textunderscore philos\textunderscore )}
\end{itemize}
Que vive nas raízes.
Gênero de cogumelos, que se desenvolvem nas raízes das plantas.
\section{Rizófise}
\begin{itemize}
\item {Grp. gram.:f.}
\end{itemize}
\begin{itemize}
\item {Utilização:Bot.}
\end{itemize}
\begin{itemize}
\item {Proveniência:(Do gr. \textunderscore rhiza\textunderscore  + \textunderscore phusis\textunderscore )}
\end{itemize}
Apêndice, na extremidade de certas radículas.
\section{Rizófora}
\begin{itemize}
\item {Grp. gram.:f.}
\end{itemize}
Gênero de plantas, o mesmo que \textunderscore rizóforo\textunderscore .
\section{Rizoforáceas}
\begin{itemize}
\item {Grp. gram.:f. pl.}
\end{itemize}
Família de plantas, que tem por tipo o rizóforo.
(Fem. pl. de \textunderscore rizoforáceo\textunderscore )
\section{Rizoforáceo}
\begin{itemize}
\item {Grp. gram.:adj.}
\end{itemize}
Relativo ou semelhante ao rizóforo.
\section{Rizofóreas}
\begin{itemize}
\item {Grp. gram.:f. pl.}
\end{itemize}
V. \textunderscore rizoforáceas\textunderscore .
\section{Rizóforo}
\begin{itemize}
\item {Grp. gram.:adj.}
\end{itemize}
\begin{itemize}
\item {Grp. gram.:M.}
\end{itemize}
\begin{itemize}
\item {Proveniência:(Do gr. \textunderscore rhiza\textunderscore  + \textunderscore phoros\textunderscore )}
\end{itemize}
Que tem raízes.
Gênero de plantas intertropicaes, cujo principal característico é terem as raízes banhadas pela água do mar.
\section{Rizógono}
\begin{itemize}
\item {Grp. gram.:adj.}
\end{itemize}
\begin{itemize}
\item {Utilização:Bot.}
\end{itemize}
\begin{itemize}
\item {Proveniência:(Do gr. \textunderscore rhiza\textunderscore  + \textunderscore gonos\textunderscore )}
\end{itemize}
Diz-se da planta, cuja raíz tem órgãos reproductores.
\section{Rizografia}
\begin{itemize}
\item {Grp. gram.:f.}
\end{itemize}
\begin{itemize}
\item {Proveniência:(Do gr. \textunderscore rhiza\textunderscore  + \textunderscore graphein\textunderscore )}
\end{itemize}
Descripção das raízes.
\section{Rizográfico}
\begin{itemize}
\item {Grp. gram.:adj.}
\end{itemize}
Relativo á \textunderscore rhizographia\textunderscore .
\section{Rizólita}
\begin{itemize}
\item {Grp. gram.:f.}
\end{itemize}
\begin{itemize}
\item {Proveniência:(Do gr. \textunderscore rhiza\textunderscore  + \textunderscore lithos\textunderscore )}
\end{itemize}
Raiz fóssil.
\section{Rizólito}
\begin{itemize}
\item {Grp. gram.:m.}
\end{itemize}
O mesmo ou melhor que \textunderscore rizólita\textunderscore .
\section{Rizoma}
\begin{itemize}
\item {Grp. gram.:m.}
\end{itemize}
\begin{itemize}
\item {Utilização:Bot.}
\end{itemize}
\begin{itemize}
\item {Utilização:Pharm.}
\end{itemize}
\begin{itemize}
\item {Proveniência:(Do gr. \textunderscore rhiza\textunderscore )}
\end{itemize}
Espécie de haste subterrânea, ordinariamente horizontal.
Tintura de arnica. Cf. \textunderscore Regul. dos Preços dos Medic.\textunderscore 
\section{Rizomatose}
\begin{itemize}
\item {Grp. gram.:f.}
\end{itemize}
\begin{itemize}
\item {Proveniência:(De \textunderscore rizoma\textunderscore )}
\end{itemize}
Transformação de uma raíz em rizoma.
\section{Rizomatoso}
\begin{itemize}
\item {Grp. gram.:adj.}
\end{itemize}
Que tem rizoma.
\section{Rizomorfo}
\begin{itemize}
\item {Grp. gram.:adj.}
\end{itemize}
\begin{itemize}
\item {Proveniência:(Do gr. \textunderscore rhiza\textunderscore  + \textunderscore morphe\textunderscore )}
\end{itemize}
Que tem forma de raíz.
\section{Rizópode}
\begin{itemize}
\item {Grp. gram.:adj.}
\end{itemize}
\begin{itemize}
\item {Grp. gram.:M. pl.}
\end{itemize}
\begin{itemize}
\item {Proveniência:(Do gr. \textunderscore rhiza\textunderscore  + \textunderscore pous\textunderscore , \textunderscore podos\textunderscore )}
\end{itemize}
Cujos pés são semelhantes a raízes.
Animaes, cujos pés semelham raízes.
\section{Rizopódio}
\begin{itemize}
\item {Grp. gram.:adj.}
\end{itemize}
O mesmo que \textunderscore rizópode\textunderscore .
\section{Rizos}
\begin{itemize}
\item {Grp. gram.:m. pl.}
\end{itemize}
(V.\textunderscore rizes\textunderscore )
\section{Rizospérmeas}
\begin{itemize}
\item {Grp. gram.:f. pl.}
\end{itemize}
\begin{itemize}
\item {Proveniência:(De \textunderscore rizospermo\textunderscore )}
\end{itemize}
Família de plantas aquáticas.
\section{Rizospermo}
\begin{itemize}
\item {Grp. gram.:adj.}
\end{itemize}
\begin{itemize}
\item {Proveniência:(Do gr. \textunderscore rhiza\textunderscore  + \textunderscore sperma\textunderscore )}
\end{itemize}
Diz-se dos vegetaes, cujas sementes nascem sôbre as raízes.
\section{Rizóstoma}
\begin{itemize}
\item {Grp. gram.:f.}
\end{itemize}
\begin{itemize}
\item {Utilização:Zool.}
\end{itemize}
Gênero de acalefos medusários.
(Cp. \textunderscore rizóstomo\textunderscore )
\section{Rizóstomo}
\begin{itemize}
\item {Grp. gram.:adj.}
\end{itemize}
\begin{itemize}
\item {Utilização:Zool.}
\end{itemize}
\begin{itemize}
\item {Proveniência:(Do gr. \textunderscore rhiza\textunderscore  + \textunderscore stoma\textunderscore )}
\end{itemize}
Que tem muitas bocas na extremidade de filamentos semelhantes a raízes, (falando-se de certos animaes).
\section{Rizotaxia}
\begin{itemize}
\item {fónica:csi}
\end{itemize}
\begin{itemize}
\item {Grp. gram.:f.}
\end{itemize}
\begin{itemize}
\item {Utilização:Bot.}
\end{itemize}
\begin{itemize}
\item {Proveniência:(Do gr. \textunderscore rhiza\textunderscore  + \textunderscore taxis\textunderscore )}
\end{itemize}
Disposição das radicelas sôbre a raiz da planta.
\section{Rizotomia}
\begin{itemize}
\item {Grp. gram.:f.}
\end{itemize}
\begin{itemize}
\item {Proveniência:(Do gr. \textunderscore rhiza\textunderscore  + \textunderscore tome\textunderscore )}
\end{itemize}
Córte de raízes.
\section{Rizótomo}
\begin{itemize}
\item {Grp. gram.:m.}
\end{itemize}
\begin{itemize}
\item {Proveniência:(Do gr. \textunderscore rhiza\textunderscore  + \textunderscore tome\textunderscore )}
\end{itemize}
Instrumento, para cortar raízes.
\section{Rizotónico}
\begin{itemize}
\item {Grp. gram.:adj.}
\end{itemize}
\begin{itemize}
\item {Utilização:Philol.}
\end{itemize}
\begin{itemize}
\item {Proveniência:(Do gr. \textunderscore rhiza\textunderscore  + \textunderscore tonos\textunderscore )}
\end{itemize}
Diz-se das fórmas verbaes que, em português, têm, como sílaba tónica ou dominante, a última do radical, como em \textunderscore louva\textunderscore , \textunderscore copía\textunderscore .
\section{Rízula}
\begin{itemize}
\item {Grp. gram.:f.}
\end{itemize}
\begin{itemize}
\item {Utilização:Bot.}
\end{itemize}
\begin{itemize}
\item {Proveniência:(Do gr. \textunderscore rhiza\textunderscore )}
\end{itemize}
Cada uma das radículas dos cogumelos.
\section{Roaz}
\begin{itemize}
\item {Grp. gram.:adj.}
\end{itemize}
\begin{itemize}
\item {Utilização:T. de Turquel}
\end{itemize}
\begin{itemize}
\item {Grp. gram.:M.}
\end{itemize}
Que rói.
Destruidor.
Que tem bom appetite.
O mesmo que \textunderscore toninha\textunderscore .
\section{Robaleira}
\begin{itemize}
\item {Grp. gram.:f.}
\end{itemize}
\begin{itemize}
\item {Proveniência:(De \textunderscore robalo\textunderscore )}
\end{itemize}
Rêde de três panos, usada pelos pescadores da barra do Doiro.
\section{Robalete}
\begin{itemize}
\item {fónica:lê}
\end{itemize}
\begin{itemize}
\item {Grp. gram.:m.}
\end{itemize}
\begin{itemize}
\item {Utilização:Náut.}
\end{itemize}
Robalo pequeno.
Peças de madeira, pregadas a um e outro bórdo, de popa á prôa, na parte mais bojuda da querena, para deminuír o balanço de bombordo a estibordo.
\section{Robaliço}
\begin{itemize}
\item {Grp. gram.:m.}
\end{itemize}
O mesmo que \textunderscore robalete\textunderscore .
\section{Robalinho}
\begin{itemize}
\item {Grp. gram.:m.}
\end{itemize}
Peixe cyprínida (\textunderscore leusciscus pyrenaicus\textunderscore ).
\section{Robalo}
\begin{itemize}
\item {Grp. gram.:m.}
\end{itemize}
Peixe pércida, (\textunderscore perca labrax\textunderscore ).
\section{Robe}
\begin{itemize}
\item {fónica:rô}
\end{itemize}
\begin{itemize}
\item {Grp. gram.:m.}
\end{itemize}
\begin{itemize}
\item {Utilização:Des.}
\end{itemize}
O mesmo que \textunderscore arrobe\textunderscore .
\section{Róber}
\begin{itemize}
\item {Grp. gram.:m.}
\end{itemize}
Série do duas partidas, no jôgo do \textunderscore whist\textunderscore . Cf. A. Dinis, \textunderscore Hyssope\textunderscore , 39 e 40.
(Do ingl.\textunderscore ruber\textunderscore )
\section{Robércia}
\begin{itemize}
\item {Grp. gram.:f.}
\end{itemize}
\begin{itemize}
\item {Proveniência:(De \textunderscore Robert\textunderscore , n. p.)}
\end{itemize}
Gênero de plantas, da fam. das compostas.
\section{Roberto}
\begin{itemize}
\item {Grp. gram.:adj.}
\end{itemize}
\begin{itemize}
\item {Utilização:Açor}
\end{itemize}
\begin{itemize}
\item {Proveniência:(De \textunderscore Roberto\textunderscore , n. p.)}
\end{itemize}
Travêsso, trocista.
\section{Roberval}
\begin{itemize}
\item {Grp. gram.:f.}
\end{itemize}
\begin{itemize}
\item {Proveniência:(De \textunderscore Roberval\textunderscore , n. p.)}
\end{itemize}
Espécie de balança, cujas alavancas estão collocadas inferiormente aos pratos respectivos.
\section{Robespièrrista}
\begin{itemize}
\item {Grp. gram.:f.}
\end{itemize}
Partidário de Robespièrre ou das ideias revolucionárias que dominaram em França na época do Terror.
\section{Robissão}
\begin{itemize}
\item {Grp. gram.:m.}
\end{itemize}
\begin{itemize}
\item {Utilização:Bras}
\end{itemize}
\begin{itemize}
\item {Proveniência:(Do fr. \textunderscore robe\textunderscore ?)}
\end{itemize}
O mesmo que \textunderscore sobrecasaca\textunderscore .
\section{Roble}
\begin{itemize}
\item {Grp. gram.:m.}
\end{itemize}
\begin{itemize}
\item {Utilização:Poét.}
\end{itemize}
\begin{itemize}
\item {Proveniência:(Do lat. \textunderscore robur\textunderscore )}
\end{itemize}
O mesmo que \textunderscore carvalho\textunderscore .
Árvore gigante e antiga.
\section{Robledo}
\begin{itemize}
\item {fónica:blê}
\end{itemize}
\begin{itemize}
\item {Grp. gram.:m.}
\end{itemize}
Mata de robles.
\section{Robora}
\begin{itemize}
\item {Grp. gram.:f.}
\end{itemize}
\begin{itemize}
\item {Proveniência:(De \textunderscore roborar\textunderscore )}
\end{itemize}
(Cp.\textunderscore rebora\textunderscore ). Cf. Herculano, \textunderscore Hist. de Port.\textunderscore , II, 251.
\section{Roboração}
\begin{itemize}
\item {Grp. gram.:f.}
\end{itemize}
Acto ou effeito de roborar.
\section{Roborante}
\begin{itemize}
\item {Grp. gram.:adj.}
\end{itemize}
\begin{itemize}
\item {Proveniência:(Lat. \textunderscore roborans\textunderscore )}
\end{itemize}
Que robora.
\section{Roborar}
\begin{itemize}
\item {Grp. gram.:v. t.}
\end{itemize}
\begin{itemize}
\item {Proveniência:(Lat. \textunderscore roborare\textunderscore )}
\end{itemize}
Aumentar as fôrças de.
Fortificar.
Corroborar, confirmar: \textunderscore roborar allegações\textunderscore .
\section{Roborativo}
\begin{itemize}
\item {Grp. gram.:adj.}
\end{itemize}
\begin{itemize}
\item {Proveniência:(De \textunderscore roborar\textunderscore )}
\end{itemize}
Próprio para roborar; que robora.
\section{Roboredo}
\begin{itemize}
\item {fónica:borê}
\end{itemize}
\begin{itemize}
\item {Grp. gram.:m.}
\end{itemize}
\begin{itemize}
\item {Proveniência:(Do lat. \textunderscore roboretum\textunderscore )}
\end{itemize}
O mesmo que \textunderscore robledo\textunderscore .
\section{Robóreo}
\begin{itemize}
\item {Grp. gram.:adj.}
\end{itemize}
\begin{itemize}
\item {Utilização:Poét.}
\end{itemize}
\begin{itemize}
\item {Proveniência:(Lat. \textunderscore roboreus\textunderscore )}
\end{itemize}
Feito de madeira de carvalho:«\textunderscore ...a robórea ponte...\textunderscore »Castilho, \textunderscore Fastos\textunderscore , III, 73.
\section{Roborite}
\begin{itemize}
\item {Grp. gram.:f.}
\end{itemize}
\begin{itemize}
\item {Proveniência:(Do lat. \textunderscore robur\textunderscore , \textunderscore roboris\textunderscore )}
\end{itemize}
Substância explosiva, modernamente incluída no material de guerra.
\section{Roborito}
\begin{itemize}
\item {Grp. gram.:m.}
\end{itemize}
O mesmo ou melhor que \textunderscore roborite\textunderscore .
\section{Ròborizar}
\begin{itemize}
\item {Grp. gram.:v. t.}
\end{itemize}
\begin{itemize}
\item {Utilização:Neol.}
\end{itemize}
\begin{itemize}
\item {Proveniência:(Do lat. \textunderscore roborare\textunderscore )}
\end{itemize}
Tornar forte, fortalecer, fortificar. Cf. Alv. Mendes, \textunderscore Discursos\textunderscore , 59.
\section{Robsónia}
\begin{itemize}
\item {Grp. gram.:f.}
\end{itemize}
Gênero de plantas ribesiáceas.
\section{Róbur}
\begin{itemize}
\item {Grp. gram.:m.}
\end{itemize}
\begin{itemize}
\item {Proveniência:(T. lat.)}
\end{itemize}
A parte inferior e mais temível do cárcere tulliano, em Roma.
\section{Robustamente}
\begin{itemize}
\item {Grp. gram.:adv.}
\end{itemize}
De modo robusto.
\section{Robustecedor}
\begin{itemize}
\item {Grp. gram.:adj.}
\end{itemize}
Que torna robusto. Cf. Corvo, \textunderscore Côrte de D. João V\textunderscore , 160.
(De \textunderscore robustecer\textunderscore ).
\section{Robustecer}
\begin{itemize}
\item {Grp. gram.:v. t.}
\end{itemize}
\begin{itemize}
\item {Grp. gram.:V. i.  e  p.}
\end{itemize}
Roborar.
Tornar robusto.
Engrandecer.
Exaltar; aumentar: \textunderscore robustecer a fé\textunderscore .
Tornar-se robusto; avigorar-se; exaltar-se.
Ampliar-se.
\section{Robustez}
\begin{itemize}
\item {Grp. gram.:f.}
\end{itemize}
Qualidade do que é robusto.
\section{Robusteza}
\begin{itemize}
\item {Grp. gram.:f.}
\end{itemize}
O mesmo que \textunderscore robustez\textunderscore .
\section{Robustidão}
\begin{itemize}
\item {Grp. gram.:f.}
\end{itemize}
O mesmo que \textunderscore robustez\textunderscore .
\section{Robusto}
\begin{itemize}
\item {Grp. gram.:adj.}
\end{itemize}
\begin{itemize}
\item {Proveniência:(Lat. \textunderscore robustus\textunderscore )}
\end{itemize}
Que tem fôrça, que é forte; vigoroso: \textunderscore homem robusto\textunderscore .
Duro, sólido.
Corpulento: \textunderscore árvore robusta\textunderscore .
Muito resistente.
Grosso.
Intenso.
Amplo, de grande capacidade, poderoso.
\section{Roca}
\begin{itemize}
\item {Grp. gram.:f.}
\end{itemize}
\begin{itemize}
\item {Utilização:Gír.}
\end{itemize}
Cana ou vara, que tem na extremidade um bojo, em que se enrola a estriga ou outra substância têxtil, que se há de fiar.
Cada uma das peças, com que se reforça um mastro de embarcação fendido.
Conjunto dos golpes ou tiras estreitas que, separadas entre si e postas ao comprido nas mangas dos vestídos, deixavam vêr o tecido em que se sobrepunham.
* Prov. trasm.
O mesmo que \textunderscore ladra\textunderscore , apparelho, com que se apanha fruta.
Bengala.
(Ant. alt. al. \textunderscore rocco\textunderscore )
\section{Roca}
\begin{itemize}
\item {Grp. gram.:f.}
\end{itemize}
\begin{itemize}
\item {Proveniência:(Do celt. ?)}
\end{itemize}
O mesmo que \textunderscore rocha\textunderscore .
\section{Roça}
\begin{itemize}
\item {Grp. gram.:f.}
\end{itemize}
\begin{itemize}
\item {Utilização:Bras}
\end{itemize}
\begin{itemize}
\item {Utilização:Prov.}
\end{itemize}
\begin{itemize}
\item {Utilização:trasm.}
\end{itemize}
\begin{itemize}
\item {Grp. gram.:Loc. adv.}
\end{itemize}
\begin{itemize}
\item {Utilização:Náut.}
\end{itemize}
\begin{itemize}
\item {Proveniência:(De \textunderscore roçar\textunderscore )}
\end{itemize}
O mesmo que \textunderscore roçadura\textunderscore .
Lugar, onde roçam mato.
Terreno, coberto de mato.
Mato, muito crescido.
Sementeira entre o mato ou em terreno, a que se roçou o mato.
Terreno de lavoira.
Grande porção de mato, espalhado num terreno, para se queimar.
\textunderscore Á roça\textunderscore , diz-se das âncoras, postas de prevenção, para se lançarem rapidamente á água, sendo preciso.
\section{Rocada}
\begin{itemize}
\item {Grp. gram.:f.}
\end{itemize}
\begin{itemize}
\item {Proveniência:(De \textunderscore roca\textunderscore ^1)}
\end{itemize}
Porção de linho, lan ou algodão, que se enrola ao bojo da roca.
Pancada com a roca.
\section{Rocada}
\begin{itemize}
\item {Grp. gram.:f.}
\end{itemize}
\begin{itemize}
\item {Utilização:Prov.}
\end{itemize}
\begin{itemize}
\item {Utilização:beir.}
\end{itemize}
Alter. de \textunderscore roncada\textunderscore : \textunderscore tenho somno, vou dormir uma rocada\textunderscore .
\section{Roçada}
\begin{itemize}
\item {Grp. gram.:f.}
\end{itemize}
\begin{itemize}
\item {Utilização:Bras}
\end{itemize}
\begin{itemize}
\item {Proveniência:(De \textunderscore roçar\textunderscore )}
\end{itemize}
Primeira operação, a que se procede ao derribar uma mata, e que consiste em cortar á foice as pequenas plantas, que podem embaraçar o manejo do machado.
\section{Roçadeira}
\begin{itemize}
\item {Grp. gram.:f.  e  adj.}
\end{itemize}
\begin{itemize}
\item {Grp. gram.:F.}
\end{itemize}
\begin{itemize}
\item {Utilização:Prov.}
\end{itemize}
\begin{itemize}
\item {Utilização:beir.}
\end{itemize}
\begin{itemize}
\item {Utilização:T. do Fundão}
\end{itemize}
O mesmo que \textunderscore roçadoira\textunderscore .
Mulhér, que roça mato.
Mulhér que lava ou esfrega casas.
\section{Roçadeiro}
\begin{itemize}
\item {Grp. gram.:adj.}
\end{itemize}
Que roça ou serve para roçar, (falando-se de instrumentos agricolas): \textunderscore foice roçadeira\textunderscore . Cf. Arn. Gama, \textunderscore Segr. do Abb.\textunderscore , 33, 35 e 36.
\section{Roçadela}
\begin{itemize}
\item {Grp. gram.:f.}
\end{itemize}
O mesmo que \textunderscore roçadura\textunderscore .
\section{Rocado}
\begin{itemize}
\item {Grp. gram.:adj.}
\end{itemize}
Que tem rocas^1, (falando-se das mangas dos vestidos).
\section{Rocado}
\begin{itemize}
\item {Grp. gram.:adj.}
\end{itemize}
\begin{itemize}
\item {Utilização:P. us.}
\end{itemize}
\begin{itemize}
\item {Grp. gram.:M.}
\end{itemize}
\begin{itemize}
\item {Proveniência:(De \textunderscore roca\textunderscore ^2)}
\end{itemize}
Que tem penedias.
O mesmo que \textunderscore penedia\textunderscore .
\section{Roçado}
\begin{itemize}
\item {Grp. gram.:m.}
\end{itemize}
\begin{itemize}
\item {Proveniência:(De \textunderscore roçar\textunderscore )}
\end{itemize}
Terreno, em que se roçou o mato e que se preparou para sêr cultivado.
Clareira entre o mato.
\section{Roçadoira}
\begin{itemize}
\item {Grp. gram.:f.  e  adj.}
\end{itemize}
\begin{itemize}
\item {Proveniência:(De \textunderscore roçar\textunderscore )}
\end{itemize}
Diz-se de uma foice grossa, própria para roçar mato, cortar polas, etc.
\section{Roçador}
\begin{itemize}
\item {Grp. gram.:m.  e  adj.}
\end{itemize}
O que roça.
\section{Roçadoura}
\begin{itemize}
\item {Grp. gram.:f.  e  adj.}
\end{itemize}
\begin{itemize}
\item {Proveniência:(De \textunderscore roçar\textunderscore )}
\end{itemize}
Diz-se de uma foice grossa, própria para roçar mato, cortar polas, etc.
\section{Roçadura}
\begin{itemize}
\item {Grp. gram.:f.}
\end{itemize}
Acto ou effeito de roçar.
\section{Roçagante}
\begin{itemize}
\item {Grp. gram.:adj.}
\end{itemize}
Que roçaga.
\section{Roçagar}
\begin{itemize}
\item {Grp. gram.:v. i.}
\end{itemize}
\begin{itemize}
\item {Proveniência:(De \textunderscore roçar\textunderscore )}
\end{itemize}
Roçar pelo chão, arrastar-se.
Fazer ruído, como um vestido de seda de alguém que anda.
Passar de leve.
\section{Rocal}
\begin{itemize}
\item {Grp. gram.:adj.}
\end{itemize}
\begin{itemize}
\item {Grp. gram.:M.}
\end{itemize}
\begin{itemize}
\item {Proveniência:(De \textunderscore roca\textunderscore ^2)}
\end{itemize}
Duro como pedra.
Collar de contas ou de pérolas; rocalha.
\section{Rocalha}
\begin{itemize}
\item {Grp. gram.:f.}
\end{itemize}
Porção de contas para collar ou rosário; rocal.
(Cast. \textunderscore rocalla\textunderscore )
\section{Rocamador}
\begin{itemize}
\item {Grp. gram.:m.}
\end{itemize}
Antigo instituto ou congregação hospitalar do Santo-Amador, que de França se estendeu a Portugal, acabando em tempo de Afonso V.
Também se dizia \textunderscore roca-amador\textunderscore , \textunderscore recamador\textunderscore  e \textunderscore reca-amador\textunderscore . Cf. S. R. Viterbo, \textunderscore Elucidário\textunderscore .
\section{Roçamalha}
\begin{itemize}
\item {Grp. gram.:f.}
\end{itemize}
\begin{itemize}
\item {Utilização:Ant.}
\end{itemize}
Estoraque líquido.
\section{Roça-marinha}
\begin{itemize}
\item {Grp. gram.:f.}
\end{itemize}
Planta da serra de Sintra.
\section{Rocambolesco}
\begin{itemize}
\item {Grp. gram.:m.}
\end{itemize}
\begin{itemize}
\item {Proveniência:(De \textunderscore Rocambole\textunderscore , n. p. de uma personagem de um romance de romance de Ponson)}
\end{itemize}
Relativo ou parecido ás aventuras extraordinárias ou inverosímeis de \textunderscore Rocambole\textunderscore .
Enredado, cheio de peripécias e lances imprevistos.
\section{Roçamento}
\begin{itemize}
\item {Grp. gram.:m.}
\end{itemize}
O mesmo que \textunderscore roçadura\textunderscore .
\section{Roçana}
\begin{itemize}
\item {Grp. gram.:f.}
\end{itemize}
\begin{itemize}
\item {Utilização:Prov.}
\end{itemize}
Foice roçadoira. (Colhido em Alcobaça)
\section{Rocão}
\begin{itemize}
\item {Grp. gram.:m.}
\end{itemize}
\begin{itemize}
\item {Utilização:Prov.}
\end{itemize}
\begin{itemize}
\item {Utilização:trasm.}
\end{itemize}
\begin{itemize}
\item {Proveniência:(De \textunderscore roca\textunderscore ^1)}
\end{itemize}
Pau, com um apparelho de fôlha na extremidade, para colher fruta nas árvores.
Espécie de funil de papel, com que, á míngua de correia, se aperta o linho na roca.
\section{Rocão}
\begin{itemize}
\item {Grp. gram.:m.}
\end{itemize}
\begin{itemize}
\item {Utilização:Prov.}
\end{itemize}
\begin{itemize}
\item {Utilização:minh.}
\end{itemize}
\begin{itemize}
\item {Proveniência:(De \textunderscore roca\textunderscore ^2)}
\end{itemize}
Collar de dois ou três fios de contas; rocal. Cf. Rev.\textunderscore Tradição\textunderscore , IV, 156.
\section{Rocar}
\begin{itemize}
\item {Grp. gram.:v. i.}
\end{itemize}
Fazer roque, no jôgo.
\section{Roçar}
\begin{itemize}
\item {Grp. gram.:v. t.}
\end{itemize}
\begin{itemize}
\item {Grp. gram.:V. i.}
\end{itemize}
\begin{itemize}
\item {Utilização:Gír.}
\end{itemize}
\begin{itemize}
\item {Proveniência:(Do lat. hyp.\textunderscore ruptiare\textunderscore , de \textunderscore ruptus\textunderscore )}
\end{itemize}
Cortar cerce; derribar: \textunderscore roçar mato\textunderscore .
Gastar com o attrito.
Friccionar levemente.
Passar junto de.
Esfregar.
Gastar.
Tocar de leve.
Resvalar, passar junto, passar de leve: \textunderscore elle ia passando e roçou por mim\textunderscore .
Rir-se.
\section{Roças}
\begin{itemize}
\item {Grp. gram.:m.}
\end{itemize}
\begin{itemize}
\item {Utilização:Prov.}
\end{itemize}
\begin{itemize}
\item {Utilização:trasm.}
\end{itemize}
Dentista, pouco habilitado.
\section{Rocaz}
\begin{itemize}
\item {Grp. gram.:adj.}
\end{itemize}
\begin{itemize}
\item {Grp. gram.:M.}
\end{itemize}
\begin{itemize}
\item {Proveniência:(De \textunderscore roca\textunderscore ^2)}
\end{itemize}
O mesmo que \textunderscore rochaz\textunderscore .
Peixe, o mesmo que \textunderscore rascasso\textunderscore .
\section{Rocedão}
\begin{itemize}
\item {Grp. gram.:m.}
\end{itemize}
\begin{itemize}
\item {Proveniência:(De \textunderscore roçar\textunderscore )}
\end{itemize}
Fio, com que o sapateiro liga o cabedal em volta das fôrmas.
\section{Rocega}
\begin{itemize}
\item {Grp. gram.:f.}
\end{itemize}
\begin{itemize}
\item {Utilização:Náut.}
\end{itemize}
Acto de rocegar; cabo, com que se rocega.
\section{Rocegar}
\begin{itemize}
\item {Grp. gram.:v.}
\end{itemize}
\begin{itemize}
\item {Utilização:t. Náut.}
\end{itemize}
\begin{itemize}
\item {Proveniência:(De \textunderscore roçar\textunderscore )}
\end{itemize}
Procurar com a rocega ou cabo apropriado (âncoras ou outro objecto perdido debaixo de água).
\section{Roceiro}
\begin{itemize}
\item {Grp. gram.:m.}
\end{itemize}
\begin{itemize}
\item {Utilização:Bras}
\end{itemize}
Homem que roça.
Homem, que cultiva os terrenos chamados roças.
\section{Rocena}
\begin{itemize}
\item {Grp. gram.:m.}
\end{itemize}
\begin{itemize}
\item {Utilização:Des.}
\end{itemize}
Aquelle que trabalha ou luta, servindo-se de roçadoira. Cf. Camillo, \textunderscore Mar. da Fonte\textunderscore , 317.
(Cp.\textunderscore roçana\textunderscore )
\section{Rocha}
\begin{itemize}
\item {Grp. gram.:f.}
\end{itemize}
\begin{itemize}
\item {Utilização:Ext.}
\end{itemize}
\begin{itemize}
\item {Utilização:Fig.}
\end{itemize}
\begin{itemize}
\item {Proveniência:(Fr. \textunderscore roche\textunderscore )}
\end{itemize}
Mole ou grande massa compacta de pedra muito dura.
Penedia, rochedo.
Mineral; conjunto de mineraes, que formam massa compacta.
Coisa inabalável.
\section{Rochaz}
\begin{itemize}
\item {Grp. gram.:adj.}
\end{itemize}
\begin{itemize}
\item {Proveniência:(De \textunderscore rocha\textunderscore )}
\end{itemize}
Que se cria nas rochas.
\section{Rochedo}
\begin{itemize}
\item {fónica:chê}
\end{itemize}
\begin{itemize}
\item {Grp. gram.:m.}
\end{itemize}
\begin{itemize}
\item {Utilização:Anat.}
\end{itemize}
Rocha escarpada.
Rocha alta á beira do mar.
Penhasco.
Cachopo.
Parte de cada um dos temporaes, onde se aloja o ouvido.
\section{Rochete}
\begin{itemize}
\item {fónica:chê}
\end{itemize}
\begin{itemize}
\item {Grp. gram.:m.}
\end{itemize}
\begin{itemize}
\item {Utilização:Prov.}
\end{itemize}
\begin{itemize}
\item {Utilização:trasm.}
\end{itemize}
O mesmo que \textunderscore collarinho\textunderscore . (Colhido em Sabrosa)
(Relaciona-se com \textunderscore gorgete\textunderscore ?)
\section{Rochina}
\begin{itemize}
\item {Grp. gram.:f.}
\end{itemize}
\begin{itemize}
\item {Utilização:Bras}
\end{itemize}
Espécie de mandioca.
\section{Rochinha}
\begin{itemize}
\item {Grp. gram.:adj.}
\end{itemize}
\begin{itemize}
\item {Utilização:Bras. do Rio Grande do S}
\end{itemize}
Verrinoso; insultador.
\section{Rochoso}
\begin{itemize}
\item {Grp. gram.:adj.}
\end{itemize}
Em que há rochas; formado de rochas:«\textunderscore ...em volta de umas ilhas rochosas\textunderscore ». Th. Ribeiro, \textunderscore Jornadas\textunderscore , II, 14.
\section{Rociada}
\begin{itemize}
\item {Grp. gram.:f.}
\end{itemize}
\begin{itemize}
\item {Utilização:Pop.}
\end{itemize}
Acto ou effeito de rociar.
Grande quantidade.
\section{Rociar}
\begin{itemize}
\item {Grp. gram.:v. t.}
\end{itemize}
\begin{itemize}
\item {Utilização:Fig.}
\end{itemize}
\begin{itemize}
\item {Grp. gram.:V. i.}
\end{itemize}
\begin{itemize}
\item {Proveniência:(De \textunderscore rócio\textunderscore )}
\end{itemize}
Orvalhar.
Borrifar.
Aspergir com gotas de qualquer líquido.
Humedecer.
Espalhar, á maneira de chuva ou de orvalho.
Diffundir sôbre.
Cair orvalho.
Cair em fórma de orvalho:«\textunderscore ...rociando de pelouros os contrários\textunderscore ». Filinto, \textunderscore D. Man.\textunderscore , II, 60.
\section{Rocim}
\begin{itemize}
\item {Grp. gram.:m.}
\end{itemize}
Cavallo pequeno e fraco.
(B. lat. \textunderscore runcinus\textunderscore )
\section{Rocinal}
\begin{itemize}
\item {Grp. gram.:adj.}
\end{itemize}
Próprio de rocim ou de cavallo pequeno: \textunderscore uma carga rocinal...\textunderscore 
\section{Rocinante}
\begin{itemize}
\item {Grp. gram.:m.}
\end{itemize}
\begin{itemize}
\item {Proveniência:(De \textunderscore Rocinante\textunderscore , n. p. do cavallo de D. Quixote)}
\end{itemize}
Cavallo reles, rocim.
\section{Rocinela}
\begin{itemize}
\item {Grp. gram.:f.}
\end{itemize}
Gênero de crustáceos isópodes.
\section{Ròcinha}
\begin{itemize}
\item {Grp. gram.:f.}
\end{itemize}
\begin{itemize}
\item {Utilização:Bras. do N}
\end{itemize}
\begin{itemize}
\item {Proveniência:(De \textunderscore roça\textunderscore )}
\end{itemize}
O mesmo que \textunderscore chácara\textunderscore .
\section{Rócio}
\begin{itemize}
\item {Grp. gram.:m.}
\end{itemize}
\begin{itemize}
\item {Proveniência:(Do lat. \textunderscore roscivus\textunderscore )}
\end{itemize}
O mesmo que \textunderscore orvalho\textunderscore .
\section{Rócio}
\begin{itemize}
\item {Grp. gram.:m.}
\end{itemize}
\begin{itemize}
\item {Utilização:Prov.}
\end{itemize}
\begin{itemize}
\item {Utilização:trasm.}
\end{itemize}
O mesmo que \textunderscore rôço\textunderscore ^2.
\section{Rocío}
\begin{itemize}
\item {Grp. gram.:m.}
\end{itemize}
(V.\textunderscore rossio\textunderscore )
\section{Rocío}
\begin{itemize}
\item {Grp. gram.:m.}
\end{itemize}
\begin{itemize}
\item {Proveniência:(Do lat. \textunderscore roscivus\textunderscore )}
\end{itemize}
O mesmo que \textunderscore orvalho\textunderscore .
\section{Rocioso}
\begin{itemize}
\item {Grp. gram.:adj.}
\end{itemize}
\begin{itemize}
\item {Proveniência:(De \textunderscore rócio\textunderscore ^1)}
\end{itemize}
Que tem orvalho.
\section{Rocló}
\begin{itemize}
\item {Grp. gram.:m.}
\end{itemize}
\begin{itemize}
\item {Proveniência:(Do fr. \textunderscore roquelaure\textunderscore )}
\end{itemize}
Antigo e pequeno capote de mangas, que se abotoava na frente.
\section{Rocloró}
\begin{itemize}
\item {Grp. gram.:m.}
\end{itemize}
\begin{itemize}
\item {Utilização:Ant.}
\end{itemize}
O mesmo que \textunderscore rocló\textunderscore .
\section{Roco}
\begin{itemize}
\item {Grp. gram.:m.}
\end{itemize}
\begin{itemize}
\item {Utilização:Prov.}
\end{itemize}
\begin{itemize}
\item {Utilização:trasm.}
\end{itemize}
Espécie de cogumelo comestível.
\section{Rôço}
\begin{itemize}
\item {Grp. gram.:m.}
\end{itemize}
\begin{itemize}
\item {Proveniência:(De \textunderscore roçar\textunderscore )}
\end{itemize}
Corte de pedra, acima do nível do solo.
Sulco, que os canteiros fazem nas pedras, para as dividir ou cortar.
\section{Rôço}
\begin{itemize}
\item {Grp. gram.:m.}
\end{itemize}
\begin{itemize}
\item {Utilização:trasm}
\end{itemize}
\begin{itemize}
\item {Utilização:Gír.}
\end{itemize}
Dinheiro, bagalhoça.
\section{Rococo}
\begin{itemize}
\item {fónica:cô}
\end{itemize}
\begin{itemize}
\item {Grp. gram.:m.  e  adj.}
\end{itemize}
Dizia-se de um gênero de architectura e marcenaria, vulgar no séc. XVIII, e caracterizado por linhas curvas e emmaranhadas, profusão de ornatos e embrechados.
(Cp. \textunderscore rocòcó\textunderscore , que talvez seja alter. de \textunderscore rococo\textunderscore )
\section{Rocòcó}
\begin{itemize}
\item {Grp. gram.:adj.}
\end{itemize}
\begin{itemize}
\item {Utilização:Fig.}
\end{itemize}
\begin{itemize}
\item {Grp. gram.:M.}
\end{itemize}
\begin{itemize}
\item {Proveniência:(Fr. \textunderscore recoco\textunderscore )}
\end{itemize}
Que tem muitos ornatos, mas sem graça.
Que é de mau gosto, sem sentimento artístico.
Que se resente do gosto antigo ou que está fóra da moda: \textunderscore penteado rocòcó\textunderscore .
Profusão de ornatos de mau gôsto, especialmente em architectura.
Velharia.
\section{Roçoeiro}
\begin{itemize}
\item {Grp. gram.:m.}
\end{itemize}
\begin{itemize}
\item {Utilização:Pesc.}
\end{itemize}
\begin{itemize}
\item {Proveniência:(De \textunderscore roçar\textunderscore )}
\end{itemize}
Cabo inferior da tralha das rêdes de arrastar.
\section{Roda}
\begin{itemize}
\item {Grp. gram.:f.}
\end{itemize}
\begin{itemize}
\item {Utilização:T. de Turquel}
\end{itemize}
\begin{itemize}
\item {Utilização:Pop.}
\end{itemize}
\begin{itemize}
\item {Utilização:Gír.}
\end{itemize}
\begin{itemize}
\item {Utilização:Ext.}
\end{itemize}
\begin{itemize}
\item {Proveniência:(Lat. \textunderscore rota\textunderscore )}
\end{itemize}
Peça, ou máquina simples, de fórma circular e própria para se mover em volta de um eixo.
Círculo.
Objecto, mais ou menos circular.
Pau grosso e curto, em que termina a popa e a prôa do navio.
Globo girante, em que se lançam os números de uma lotaria.
A lotaria.
Caixa cylíndrica, na portaria de um convento ou de hospício, para transmittir, girando, qualquer coisa para o interior do edifício: \textunderscore a roda dos expostos\textunderscore .
Amplidão em volta.
Volta; circunferência; giro.
Distribuição de alguma coisa por circunstantes: \textunderscore dar uma roda de vinho\textunderscore .
Cercadura de renda.
Talhada, mais ou menos circular, de certos frutos: \textunderscore uma roda de ananás\textunderscore .
Decurso de tempo.
Agrupamento de pessôas: \textunderscore vi-o numa roda de vadios\textunderscore .
Classe: \textunderscore pertence á alta roda\textunderscore .
Associação.
Pessôas, com quem se convive: \textunderscore aquelle é da minha roda\textunderscore .
Grupo de pessôas, formando circulo: \textunderscore uma dança de roda\textunderscore .
Grande número.
Accusação; invectiva \textunderscore apanhou uma roda de malcriado\textunderscore .
Cauda de alguns animaes.
Antigo instrumento de supplício.
Mancha circular no pêlo dos cavallos.
O mesmo que \textunderscore tostão\textunderscore ^1.
Hospício de enjeitados.
* \textunderscore Meter na roda\textunderscore , enjeitar (uma criança).
\section{Roda}
\begin{itemize}
\item {Grp. gram.:m.}
\end{itemize}
\begin{itemize}
\item {Proveniência:(De \textunderscore rodar\textunderscore ^1 )}
\end{itemize}
Antigo vinho da Madeira, que fez a viagem do Oriente e voltou ao Occidente.
Contrato para a divisão proporcional do tráfego da conducção do sal, entre os donos de barcos, no rio Sado.
\section{Rodada}
\begin{itemize}
\item {Grp. gram.:f.}
\end{itemize}
\begin{itemize}
\item {Utilização:Bras. do S}
\end{itemize}
\begin{itemize}
\item {Proveniência:(De \textunderscore rodar\textunderscore ^1)}
\end{itemize}
Movimento completo de uma roda.
Quéda do cavallo para a frente.
\section{Rodado}
\begin{itemize}
\item {Grp. gram.:adj.}
\end{itemize}
\begin{itemize}
\item {Utilização:Ant.}
\end{itemize}
\begin{itemize}
\item {Grp. gram.:M.}
\end{itemize}
\begin{itemize}
\item {Proveniência:(De \textunderscore rodar\textunderscore ^1)}
\end{itemize}
Que tem roda.
Decorrido:«\textunderscore rodados vinte e seis annos...\textunderscore »Camillo, \textunderscore Regicida\textunderscore , 106.
Dizia-se do cavallo, que tem malhas redondas.
Roda de um vestido.
Conjunto das rodas de um carro.
\section{Rodado}
\begin{itemize}
\item {Grp. gram.:adj.}
\end{itemize}
\begin{itemize}
\item {Proveniência:(De \textunderscore rodar\textunderscore ^2)}
\end{itemize}
Dizia-se especialmente do alqueire que se rasoirou.
\section{Rodador}
\begin{itemize}
\item {Grp. gram.:adj.}
\end{itemize}
\begin{itemize}
\item {Utilização:Bras. do S}
\end{itemize}
\begin{itemize}
\item {Proveniência:(De \textunderscore rodar\textunderscore ^1)}
\end{itemize}
Diz-se do cavallo, que cai facilmente.
\section{Roda-dos-altos-coices}
\begin{itemize}
\item {Grp. gram.:f.}
\end{itemize}
Espécie de jôgo popular.
\section{Rodagem}
\begin{itemize}
\item {Grp. gram.:f.}
\end{itemize}
\begin{itemize}
\item {Utilização:Bras}
\end{itemize}
\begin{itemize}
\item {Proveniência:(De \textunderscore rodar\textunderscore ^1)}
\end{itemize}
Conjunto de rodas num maquinismo: \textunderscore rodagem do relógio\textunderscore .
Fábrica de rodas.
Acto de rodar.
\textunderscore Caminho de rodas\textunderscore , caminho em que podem transitar carros.
\section{Rodaixinhas}
\begin{itemize}
\item {Grp. gram.:f. pl.}
\end{itemize}
\begin{itemize}
\item {Utilização:T. de Mogadoiro}
\end{itemize}
\begin{itemize}
\item {Proveniência:(De \textunderscore roda\textunderscore ^1)}
\end{itemize}
O mesmo que \textunderscore regaixinhas\textunderscore .
\section{Rodal}
\begin{itemize}
\item {Grp. gram.:m.}
\end{itemize}
\begin{itemize}
\item {Utilização:Prov.}
\end{itemize}
\begin{itemize}
\item {Proveniência:(De \textunderscore roda\textunderscore ^1)}
\end{itemize}
Parte do carro, formado de eixo e rodas.
\section{Rodalho}
\begin{itemize}
\item {Grp. gram.:m.}
\end{itemize}
Disco de barro, em que o oleiro modela as peças de barro.
\section{Rodalose}
\begin{itemize}
\item {Grp. gram.:f.}
\end{itemize}
Espécie de cristal.
\section{Rodamite}
\begin{itemize}
\item {Grp. gram.:f.}
\end{itemize}
Espécie de explosivo, recentemente descoberto, mais violento que a própria melinite, mas de diffícil e perigosa preparação.
\section{Rodâmnia}
\begin{itemize}
\item {Grp. gram.:f.}
\end{itemize}
\begin{itemize}
\item {Proveniência:(Do gr. \textunderscore rhodon\textunderscore  + \textunderscore amnos\textunderscore )}
\end{itemize}
Arbusto mirtáceo de Samatra.
\section{Rodanito}
\begin{itemize}
\item {Grp. gram.:m.}
\end{itemize}
\begin{itemize}
\item {Utilização:Miner.}
\end{itemize}
Silicato de manganés.
\section{Rodanta}
\begin{itemize}
\item {Grp. gram.:f.}
\end{itemize}
Gênero de plantas, da fam. das compostas.
\section{Rodante}
\begin{itemize}
\item {Grp. gram.:adj.}
\end{itemize}
\begin{itemize}
\item {Grp. gram.:M.}
\end{itemize}
\begin{itemize}
\item {Proveniência:(De \textunderscore rodar\textunderscore ^1)}
\end{itemize}
Que roda.
Cambão, a que se junge o boi ou se atrela outro animal, nos engenhos de tirar água de poço ou cisterna.
\section{Ròdapé}
\begin{itemize}
\item {Grp. gram.:m.}
\end{itemize}
\begin{itemize}
\item {Proveniência:(De \textunderscore roda\textunderscore ^1 + \textunderscore pé\textunderscore )}
\end{itemize}
Espécie de cortina, que pende das bordas de uma cama ou de uma marquesa até o pavimento.
Faixa de madeira, pregada ao fundo das paredes, nas salas e outros compartimentos, para que os pés das cadeiras não prejudiquem o fôrro ou estuque das paredes.
Faixa de madeira, estendida na parte infero-interior das grades de uma janela de sacada.
\section{Ròdapisa}
\begin{itemize}
\item {Grp. gram.:f.}
\end{itemize}
\begin{itemize}
\item {Proveniência:(De \textunderscore roda\textunderscore  + \textunderscore pisar\textunderscore )}
\end{itemize}
Parte inferior do vestido de mulhér.
\section{Rodar}
\begin{itemize}
\item {Grp. gram.:v. t.}
\end{itemize}
\begin{itemize}
\item {Utilização:Bras}
\end{itemize}
\begin{itemize}
\item {Grp. gram.:V. i.}
\end{itemize}
\begin{itemize}
\item {Utilização:Bras. do S}
\end{itemize}
\begin{itemize}
\item {Grp. gram.:M.}
\end{itemize}
\begin{itemize}
\item {Proveniência:(De \textunderscore roda\textunderscore )}
\end{itemize}
Fazer andar á roda; rodear.
Applicar o supplício da roda a.
Percorrer, navegando na direcção da corrente.
Andar em roda de um eixo ou centro.
Girar.
Mover-se sôbre rodas.
Caír, rolando.
Andar de carro: \textunderscore rodou para Cascaes\textunderscore .
Fazer círculo.
Decorrer: \textunderscore rodaram já muitos annos\textunderscore .
Chapar-se o cavallo com o cavalleiro, indo a galope.
Ruído de um carro ou de outro objecto, que vai rodando.
O andamento: \textunderscore o rodar da carruagem\textunderscore .
\section{Rodar}
\textunderscore v. t.\textunderscore  e \textunderscore i\textunderscore .
Trabalhar com rôdo.
\section{Rodária}
\begin{itemize}
\item {Grp. gram.:f.}
\end{itemize}
Gênero de insectos lepidópteros nocturnos.
\section{Rodato}
\begin{itemize}
\item {Grp. gram.:m.}
\end{itemize}
\begin{itemize}
\item {Utilização:Chím.}
\end{itemize}
Gênero de saes, produzidos pelo óxido ródico.
\section{Ròdavinho}
\begin{itemize}
\item {Grp. gram.:m.}
\end{itemize}
\begin{itemize}
\item {Proveniência:(De \textunderscore roda\textunderscore  + \textunderscore vinho\textunderscore )}
\end{itemize}
Parede da frente, na lagariça.
\section{Roda-viva}
\begin{itemize}
\item {Grp. gram.:f.}
\end{itemize}
Barafunda; azáfama.
Inquietação.
\section{Rodeador}
\begin{itemize}
\item {Grp. gram.:adj.}
\end{itemize}
Que rodeia. Cf. Filinto, XVI, 23.
\section{Rodeamento}
\begin{itemize}
\item {Grp. gram.:m.}
\end{itemize}
Acto ou effeito de rodear,
\section{Rodear}
\begin{itemize}
\item {Grp. gram.:v. t.}
\end{itemize}
\begin{itemize}
\item {Grp. gram.:V. i.}
\end{itemize}
\begin{itemize}
\item {Grp. gram.:V. p.}
\end{itemize}
\begin{itemize}
\item {Grp. gram.:M.}
\end{itemize}
\begin{itemize}
\item {Proveniência:(De \textunderscore roda\textunderscore )}
\end{itemize}
Andar em roda de.
Cercar; circundar.
Envolver.
Coroar.
Fazer companhia a.
Têr ou procurar a convivência de.
Ladear. Andar, desviando-se de: \textunderscore rodear um perigo\textunderscore .
Girar.
Fazer-se acompanhar: \textunderscore rodear-se de amigos\textunderscore .
Chamar a si.
Acto de rodear; rodeio; subterfúgio.
\section{Rodeio}
\begin{itemize}
\item {Grp. gram.:m.}
\end{itemize}
\begin{itemize}
\item {Utilização:Bras. do S}
\end{itemize}
Acto ou effeito de rodear.
Mudança.
Subterfúgio.
Meio indirecto para conseguir certo fim; evasiva.
Lugar, onde se reúne o gado, nos campos.
\section{Rodeira}
\begin{itemize}
\item {Grp. gram.:f.}
\end{itemize}
\begin{itemize}
\item {Utilização:Bras}
\end{itemize}
\begin{itemize}
\item {Proveniência:(De \textunderscore roda\textunderscore )}
\end{itemize}
Mulhér, encarregada do serviço da roda, nos conventos e hospícios.
Relheira, vestigio que deixam as rodas do carro.
Caminho próprio para o transito de carros.
Espécie de barco.
\section{Rodeiro}
\begin{itemize}
\item {Grp. gram.:m.}
\end{itemize}
\begin{itemize}
\item {Utilização:Prov.}
\end{itemize}
\begin{itemize}
\item {Utilização:dur.}
\end{itemize}
\begin{itemize}
\item {Grp. gram.:Adj.}
\end{itemize}
\begin{itemize}
\item {Proveniência:(De \textunderscore roda\textunderscore )}
\end{itemize}
Jôgo de duas rodas com um eixo commum.
Eixo de um carro; eixo.
Empregado, a quem incumbe a divisão ou roda do tráfego da conducção do sal, no rio Sado.
Designação genérica dos mais pequenos barcos do Doiro.
Diz-se de um maço grande, com que se encaixam e ajustam as rodas dos carros nas extremidades dos eixos.
\section{Rodela}
\begin{itemize}
\item {Grp. gram.:f.}
\end{itemize}
\begin{itemize}
\item {Proveniência:(Do lat. \textunderscore rotella\textunderscore )}
\end{itemize}
Pequena roda.
Escudo redondo.
\section{Rodeleiro}
\begin{itemize}
\item {Grp. gram.:m.  e  adj.}
\end{itemize}
O que tem rodela.
Soldado, armado de rodela.
\section{Rodelhas}
\begin{itemize}
\item {fónica:dê}
\end{itemize}
\begin{itemize}
\item {Grp. gram.:f. pl.}
\end{itemize}
\begin{itemize}
\item {Utilização:Náut.}
\end{itemize}
\begin{itemize}
\item {Proveniência:(De \textunderscore roda\textunderscore )}
\end{itemize}
Anéis de cabos náuticos, para impedir que os envergues corram.
\section{Rodelinha}
\begin{itemize}
\item {Grp. gram.:f.}
\end{itemize}
\begin{itemize}
\item {Utilização:Gír.}
\end{itemize}
\begin{itemize}
\item {Proveniência:(De \textunderscore rodela\textunderscore )}
\end{itemize}
Anel.
\section{Rodella}
\begin{itemize}
\item {Grp. gram.:f.}
\end{itemize}
\begin{itemize}
\item {Proveniência:(Do lat. \textunderscore rotella\textunderscore )}
\end{itemize}
Pequena roda.
Escudo redondo.
\section{Rodelleiro}
\begin{itemize}
\item {Grp. gram.:m.  e  adj.}
\end{itemize}
O que tem rodella.
Soldado, armado de rodella.
\section{Rodellinha}
\begin{itemize}
\item {Grp. gram.:f.}
\end{itemize}
\begin{itemize}
\item {Utilização:Gír.}
\end{itemize}
\begin{itemize}
\item {Proveniência:(De \textunderscore rodella\textunderscore )}
\end{itemize}
Anel.
\section{Rodello}
\begin{itemize}
\item {Grp. gram.:m.}
\end{itemize}
\begin{itemize}
\item {Utilização:Prov.}
\end{itemize}
Tomba, no calçado.
Pau redondo, que os pedreiros collocam debaixo das pedras, ao assentá-las, para que se não partam as arestas.
(Cp.\textunderscore rodella\textunderscore )
\section{Rodelo}
\begin{itemize}
\item {fónica:dê}
\end{itemize}
\begin{itemize}
\item {Grp. gram.:m.}
\end{itemize}
\begin{itemize}
\item {Utilização:Prov.}
\end{itemize}
Tomba, no calçado.
Pau redondo, que os pedreiros collocam debaixo das pedras, ao assentá-las, para que se não partam as arestas.
(Cp. \textunderscore rodela\textunderscore )
\section{Rodeta}
\begin{itemize}
\item {fónica:dê}
\end{itemize}
\begin{itemize}
\item {Grp. gram.:f.}
\end{itemize}
Pequena roda.
\section{Rodete}
\begin{itemize}
\item {fónica:dê}
\end{itemize}
\begin{itemize}
\item {Grp. gram.:m.}
\end{itemize}
\begin{itemize}
\item {Proveniência:(De \textunderscore roda\textunderscore ^1)}
\end{itemize}
Carrinho, em que se doba fio de seda.
Rodeta, roseta.
\section{Rodete}
\begin{itemize}
\item {Grp. gram.:m.}
\end{itemize}
\begin{itemize}
\item {Utilização:Bras}
\end{itemize}
Pequeno rôdo.
\section{Rodício}
\begin{itemize}
\item {Grp. gram.:m.}
\end{itemize}
\begin{itemize}
\item {Proveniência:(De \textunderscore roda\textunderscore ^1)}
\end{itemize}
Roseta, que termina as disciplinas para flagellação.
\section{Ródico}
\begin{itemize}
\item {Grp. gram.:adj.}
\end{itemize}
\begin{itemize}
\item {Proveniência:(De \textunderscore ródio\textunderscore ^1)}
\end{itemize}
Diz-se de um dos óxidos do ródio.
\section{Rodilha}
\begin{itemize}
\item {Grp. gram.:f.}
\end{itemize}
\begin{itemize}
\item {Utilização:Fam.}
\end{itemize}
\begin{itemize}
\item {Utilização:Gír.}
\end{itemize}
\begin{itemize}
\item {Proveniência:(De \textunderscore roda\textunderscore )}
\end{itemize}
O mesmo que \textunderscore rodoiça\textunderscore .
Esfregão ou trapo, para fazer limpeza nas cozinhas.
Pessôa desprezivel, que se obriga a todos os serviços, e que também se diz \textunderscore rodilho\textunderscore .
Gravata.
\section{Rodilha}
\begin{itemize}
\item {Grp. gram.:f.}
\end{itemize}
\begin{itemize}
\item {Utilização:Prov.}
\end{itemize}
\begin{itemize}
\item {Utilização:minh.}
\end{itemize}
O mesmo que \textunderscore joêlho\textunderscore .
(Cast. \textunderscore rodilla\textunderscore )
\section{Rodilhado}
\begin{itemize}
\item {Grp. gram.:m.}
\end{itemize}
\begin{itemize}
\item {Utilização:Des.}
\end{itemize}
\begin{itemize}
\item {Proveniência:(De \textunderscore rodilha\textunderscore ^1)}
\end{itemize}
Pano, com que as mulhéres seguravam o cabello, ao deitar-se.
\section{Rodilhão}
\begin{itemize}
\item {Grp. gram.:m.}
\end{itemize}
\begin{itemize}
\item {Utilização:Prov.}
\end{itemize}
\begin{itemize}
\item {Utilização:minh.}
\end{itemize}
\begin{itemize}
\item {Proveniência:(De \textunderscore rodilho\textunderscore )}
\end{itemize}
Rodilho grande.
Pequena roda, em zorras e carros de mão.
Peça de atafona.
Pessôa intriguista, embusteira.
\section{Rodilhão}
\begin{itemize}
\item {Grp. gram.:m.}
\end{itemize}
\begin{itemize}
\item {Utilização:Pop.}
\end{itemize}
O mesmo que \textunderscore roldão\textunderscore :«\textunderscore avançaram, numa arremetida impetuosamente esbandalhada, de rodilhão...\textunderscore »Camillo, \textunderscore Brasileira\textunderscore , 60.
\section{Rodilhão}
\begin{itemize}
\item {Grp. gram.:m.}
\end{itemize}
\begin{itemize}
\item {Utilização:Prov.}
\end{itemize}
\begin{itemize}
\item {Utilização:minh.}
\end{itemize}
Pequeno rodeiro, empregado em descer pedras do monte.
\section{Rodilhar}
\begin{itemize}
\item {Grp. gram.:v. t.}
\end{itemize}
(V.\textunderscore enrodilhar\textunderscore )
\section{Rodilheiro}
\begin{itemize}
\item {Grp. gram.:m.}
\end{itemize}
\begin{itemize}
\item {Utilização:Prov.}
\end{itemize}
\begin{itemize}
\item {Utilização:trasm.}
\end{itemize}
Encôsto, na extremidade do escano.
\section{Rodilho}
\begin{itemize}
\item {Grp. gram.:m.}
\end{itemize}
\begin{itemize}
\item {Utilização:Fam.}
\end{itemize}
\begin{itemize}
\item {Proveniência:(De \textunderscore roda\textunderscore )}
\end{itemize}
Rodilha, trapo.
Pessôa ordinária, que faz todos os serviços e que é tratada com desdém ou desprêzo.
\section{Rodilhudo}
\begin{itemize}
\item {Grp. gram.:adj.}
\end{itemize}
\begin{itemize}
\item {Utilização:Bras. do S}
\end{itemize}
Diz-se do cavallo, que apresenta nos machinhos e joêlhos inchação chrónica, arredondada, ou em fórma de rodilha.
\section{Rodim}
\begin{itemize}
\item {Grp. gram.:m.}
\end{itemize}
Peixe de Portugal.
(Dem. de \textunderscore roda\textunderscore , peixe)
\section{Rodinha}
\begin{itemize}
\item {Grp. gram.:f.}
\end{itemize}
\begin{itemize}
\item {Utilização:Gír.}
\end{itemize}
\begin{itemize}
\item {Proveniência:(De \textunderscore roda\textunderscore ^1)}
\end{itemize}
Tostão de prata.
\section{Ródio}
\begin{itemize}
\item {Grp. gram.:m.}
\end{itemize}
\begin{itemize}
\item {Proveniência:(Do gr. \textunderscore rhodon\textunderscore )}
\end{itemize}
Metal pouco fusível, que se descobriu na platina do comércio.
\section{Ródio}
\begin{itemize}
\item {Grp. gram.:adj.}
\end{itemize}
\begin{itemize}
\item {Proveniência:(Lat. \textunderscore rhodius\textunderscore )}
\end{itemize}
Relativo a Rhodes.
Diz-se do estilo moderado, que teve origem na ilha de Rhodes.
\section{Rodiosca}
\begin{itemize}
\item {Grp. gram.:f.}
\end{itemize}
\begin{itemize}
\item {Utilização:Pop.}
\end{itemize}
Rodeio ou circuito, com fins malévolos: \textunderscore anda a fazer-me rodioscas; alguma me quer elle pregar\textunderscore .
(Cp.\textunderscore rodeio\textunderscore )
\section{Rodiota}
\begin{itemize}
\item {Grp. gram.:m.  e  adj.}
\end{itemize}
\begin{itemize}
\item {Proveniência:(De \textunderscore Rhodes\textunderscore , n. p.)}
\end{itemize}
Habitante de Rhodes; ródio.
\section{Rodista}
\begin{itemize}
\item {Grp. gram.:m.}
\end{itemize}
Operário, que trabalha com a roda das ollarias. Cf.\textunderscore Inquér. Industr.\textunderscore , p. II, l. II, 200.
\section{Rodite}
\begin{itemize}
\item {Grp. gram.:f.}
\end{itemize}
\begin{itemize}
\item {Utilização:Miner.}
\end{itemize}
\begin{itemize}
\item {Proveniência:(Lat. \textunderscore rhoditis\textunderscore )}
\end{itemize}
Variedade de pedra, com a côr e a fórma da rosa.
\section{Rodízia}
\begin{itemize}
\item {Grp. gram.:f.}
\end{itemize}
Animal rotífero.
(Cp.\textunderscore rodízio\textunderscore )
\section{Rodízio}
\begin{itemize}
\item {Grp. gram.:m.}
\end{itemize}
\begin{itemize}
\item {Proveniência:(De \textunderscore roda\textunderscore )}
\end{itemize}
Peça do moínho de água, que faz andar a mó e que é movida pela água, caindo-lhe numas travessas escavadas que tem na base e que se chamam \textunderscore pennas\textunderscore .
Rodinha metálica, que se adapta aos pés de cadeiras, leitos, etc., para se poderem mover com facilidade.
Peça girante, com que se dá ás bocas de fogo a direcção conveniente.
Nome de um jôgo de rapazes.
\section{Rôdo}
\begin{itemize}
\item {Grp. gram.:m.}
\end{itemize}
\begin{itemize}
\item {Grp. gram.:Loc. adv.}
\end{itemize}
\begin{itemize}
\item {Proveniência:(Do lat. \textunderscore rutus\textunderscore )}
\end{itemize}
Utensílio de madeira, de feitio aproximado de uma enxada, e que serve para ajuntar o sal nas marinhas e os cereaes nas eiras.
Utensílio de madeira, com que se puxa a cinza do forno.
Utensílio análogo, para, em certas mesas de jôgo, aproximar e recolher o dinheiro.
O mesmo que \textunderscore tacaniça\textunderscore .
Êmbolo dos cylindros das máquinas de vapor.
Utensílio de ferro, para tirar as cinzas das caldeiras de vapor.
\textunderscore A rôdo\textunderscore , em grande quantidade; á larga.
\section{Rôdo}
\begin{itemize}
\item {Grp. gram.:m.}
\end{itemize}
\begin{itemize}
\item {Utilização:Prov.}
\end{itemize}
\begin{itemize}
\item {Utilização:minh.}
\end{itemize}
O mesmo que \textunderscore joêlho\textunderscore .
(Cp.\textunderscore rodilha\textunderscore ^2)
\section{Rôdo}
\begin{itemize}
\item {Grp. gram.:m.}
\end{itemize}
\begin{itemize}
\item {Utilização:Náut.}
\end{itemize}
\begin{itemize}
\item {Utilização:Prov.}
\end{itemize}
\begin{itemize}
\item {Utilização:trasm.}
\end{itemize}
Cabo delgado de pita, para amarração de pequenas embarcações e outros usos.
Esteira de barcéu.
(Do ingl.\textunderscore road\textunderscore )
\section{Rodó}
\begin{itemize}
\item {Grp. gram.:m.}
\end{itemize}
\begin{itemize}
\item {Utilização:Ant.}
\end{itemize}
\begin{itemize}
\item {Proveniência:(De \textunderscore roda\textunderscore ^1)}
\end{itemize}
Pequeno capote com mangas, chamado hoje \textunderscore josezinho\textunderscore .
\section{Rodócera}
\begin{itemize}
\item {Grp. gram.:f.}
\end{itemize}
\begin{itemize}
\item {Proveniência:(Do gr. \textunderscore rhodon\textunderscore  + \textunderscore keras\textunderscore )}
\end{itemize}
Gênero de insectos lepidópteros diurnos.
\section{Rodoclorito}
\begin{itemize}
\item {Grp. gram.:m.}
\end{itemize}
\begin{itemize}
\item {Utilização:Miner.}
\end{itemize}
Carbonato de manganés.
\section{Rodocromatito}
\begin{itemize}
\item {Grp. gram.:m. f.}
\end{itemize}
\begin{itemize}
\item {Utilização:Geol.}
\end{itemize}
Uma das variedades rosadas do chlorito.
\section{Rodocrosito}
\begin{itemize}
\item {Grp. gram.:m.}
\end{itemize}
\begin{itemize}
\item {Utilização:Miner.}
\end{itemize}
\begin{itemize}
\item {Proveniência:(Do gr. \textunderscore rhodon\textunderscore  + \textunderscore khrosis\textunderscore )}
\end{itemize}
O mesmo que \textunderscore dialogito\textunderscore .
\section{Rododáctilo}
\begin{itemize}
\item {Grp. gram.:adj.}
\end{itemize}
\begin{itemize}
\item {Utilização:Zool.}
\end{itemize}
\begin{itemize}
\item {Proveniência:(Do gr. \textunderscore rhodon\textunderscore  + \textunderscore daktulos\textunderscore )}
\end{itemize}
Diz-se do insecto que tem asas digitaes e da côr da rosa.
\section{Rododendráceas}
\begin{itemize}
\item {Grp. gram.:f. pl.}
\end{itemize}
Família de plantas, que tem por tipo o rododendro.
(Dem. pl. de \textunderscore rododendráceo\textunderscore )
\section{Rododendráceo}
\begin{itemize}
\item {Grp. gram.:adj.}
\end{itemize}
Relativo ou semelhante ao rododendro.
\section{Rododendro}
\begin{itemize}
\item {Grp. gram.:m.}
\end{itemize}
\begin{itemize}
\item {Proveniência:(Lat. \textunderscore rhodo\textunderscore  + \textunderscore dendron\textunderscore )}
\end{itemize}
Gênero de arbustos e árvores, entre as quaes sobresái uma de grande e formosas flôres.
Loendro; cevadilha.
\section{Rodogastro}
\begin{itemize}
\item {Grp. gram.:adj.}
\end{itemize}
\begin{itemize}
\item {Utilização:Zool.}
\end{itemize}
\begin{itemize}
\item {Proveniência:(Do gr. \textunderscore rhodon\textunderscore  + \textunderscore gaster\textunderscore )}
\end{itemize}
Diz-se do insecto, que tem ventre vermelho.
\section{Rodografia}
\begin{itemize}
\item {Grp. gram.:f.}
\end{itemize}
\begin{itemize}
\item {Proveniência:(Do gr. \textunderscore rhodon\textunderscore  + \textunderscore graphein\textunderscore )}
\end{itemize}
Descripção das rosas.
\section{Rodográfico}
\begin{itemize}
\item {Grp. gram.:adj.}
\end{itemize}
Relativo á rodografia.
\section{Rodoiça}
\begin{itemize}
\item {Grp. gram.:f.}
\end{itemize}
\begin{itemize}
\item {Proveniência:(De \textunderscore roda\textunderscore )}
\end{itemize}
Trapo, que forma uma roda, ou roda de trapos torcidos, que se põe na cabeça, para suster fardos e abrandar-lhes a pressão; rodilha.
\section{Rodolena}
\begin{itemize}
\item {Grp. gram.:f.}
\end{itemize}
\begin{itemize}
\item {Proveniência:(Do gr. \textunderscore rhodon\textunderscore  + \textunderscore laina\textunderscore )}
\end{itemize}
Gênero de plantas de Madagáscar.
\section{Rodolfinas}
\begin{itemize}
\item {Grp. gram.:adj. f. pl.}
\end{itemize}
Diz-se das tábuas ou tabelas, que indicam o movimento dos planetas e que foram oferecidas por Kepler ao imperador Rodolpho.
\section{Rodolho}
\begin{itemize}
\item {fónica:dô}
\end{itemize}
\begin{itemize}
\item {Grp. gram.:m.}
\end{itemize}
O mesmo que \textunderscore redolho\textunderscore , cordeiro serôdio, enfezado.
\section{Rodólita}
\begin{itemize}
\item {Grp. gram.:m.}
\end{itemize}
\begin{itemize}
\item {Utilização:Miner.}
\end{itemize}
\begin{itemize}
\item {Proveniência:(Do gr. \textunderscore rhodon\textunderscore  + \textunderscore lithos\textunderscore )}
\end{itemize}
Silicato de manganés, da côr da rosa.
\section{Rodologia}
\begin{itemize}
\item {Grp. gram.:f.}
\end{itemize}
\begin{itemize}
\item {Proveniência:(Do gr. \textunderscore rhodon\textunderscore  + \textunderscore logos\textunderscore )}
\end{itemize}
Parte da Botânica, que trata das rosas.
\section{Rodológico}
\begin{itemize}
\item {Grp. gram.:adj.}
\end{itemize}
Relativo á rodologia.
\section{Rodolphinas}
\begin{itemize}
\item {Grp. gram.:adj. f. pl.}
\end{itemize}
Diz-se das tábuas ou tabellas, que indicam o movimento dos planetas e que foram offerecidas por Kepler ao imperador Rodolpho.
\section{Rodomel}
\begin{itemize}
\item {Grp. gram.:m.}
\end{itemize}
\begin{itemize}
\item {Proveniência:(Do gr. \textunderscore rhodon\textunderscore  + \textunderscore meli\textunderscore )}
\end{itemize}
Mel rosado.
\section{Rodómela}
\begin{itemize}
\item {Grp. gram.:f.}
\end{itemize}
\begin{itemize}
\item {Proveniência:(Do gr. \textunderscore rhodon\textunderscore  + \textunderscore melas\textunderscore )}
\end{itemize}
Gênero de algas roxo-escuras.
\section{Rodomeláceas}
\begin{itemize}
\item {Grp. gram.:f. pl.}
\end{itemize}
\begin{itemize}
\item {Utilização:Bot.}
\end{itemize}
\begin{itemize}
\item {Proveniência:(De \textunderscore rodómela\textunderscore )}
\end{itemize}
Famílias do algas.
\section{Rodoméleas}
\begin{itemize}
\item {Grp. gram.:f. pl.}
\end{itemize}
Tríbo de plantas fíceas, que tem por tipo a rodómela.
\section{Rodonito}
\begin{itemize}
\item {Grp. gram.:f.}
\end{itemize}
\begin{itemize}
\item {Proveniência:(Do gr. \textunderscore rhodon\textunderscore )}
\end{itemize}
Silicato de manganés, cristalino e côr de rosa.
\section{Rodopelo}
\begin{itemize}
\item {fónica:pê}
\end{itemize}
\begin{itemize}
\item {Grp. gram.:m.}
\end{itemize}
\begin{itemize}
\item {Proveniência:(De \textunderscore roda\textunderscore  + \textunderscore pêlo\textunderscore )}
\end{itemize}
Remoínho de pêlo nos animaes.
\section{Rodopeu}
\begin{itemize}
\item {Grp. gram.:adj.}
\end{itemize}
\begin{itemize}
\item {Proveniência:(Lat. \textunderscore rhodopeus\textunderscore )}
\end{itemize}
Relativo ao monte \textunderscore Rhódope\textunderscore :«\textunderscore rodopeu alcantis...\textunderscore »Castilho, \textunderscore Geórg.\textunderscore 
\section{Rodopiado}
\begin{itemize}
\item {Grp. gram.:adj.}
\end{itemize}
Que rodopia; que se faz em rodopio.
\section{Rodopiar}
\begin{itemize}
\item {Grp. gram.:v. i.}
\end{itemize}
\begin{itemize}
\item {Proveniência:(De \textunderscore rodopio\textunderscore )}
\end{itemize}
Dar muitas voltas; girar muito.
\section{Rodopio}
\begin{itemize}
\item {Grp. gram.:m.}
\end{itemize}
Acto ou effeito de rodopiar.
Rodopêlo.
(Cp.\textunderscore rodopelo\textunderscore )
\section{Rodóptero}
\begin{itemize}
\item {Grp. gram.:adj.}
\end{itemize}
\begin{itemize}
\item {Utilização:Zool.}
\end{itemize}
\begin{itemize}
\item {Proveniência:(Do gr. \textunderscore rhodon\textunderscore  + \textunderscore pteron\textunderscore )}
\end{itemize}
Diz-se do insecto, que tem asas rosadas.
\section{Ródora}
\begin{itemize}
\item {Grp. gram.:f.}
\end{itemize}
\begin{itemize}
\item {Proveniência:(Lat. \textunderscore rhodora\textunderscore )}
\end{itemize}
O mesmo que \textunderscore rododendro\textunderscore .
\section{Rodoráceas}
\begin{itemize}
\item {Grp. gram.:f. pl.}
\end{itemize}
Famílias de plantas, pouco differonte das ericáceas, e á qual pertence o género azálea, segundo o \textunderscore Thes. da Líng. Port.\textunderscore --A azálea porém, segundo o commum dos botânicos, pertence ás ericáceas.
O mesmo que \textunderscore rododendráceas\textunderscore .
\section{Rodospermo}
\begin{itemize}
\item {Grp. gram.:adj.}
\end{itemize}
\begin{itemize}
\item {Utilização:Bot.}
\end{itemize}
\begin{itemize}
\item {Proveniência:(Do gr. \textunderscore rhodon\textunderscore  + \textunderscore sperma\textunderscore )}
\end{itemize}
Que tem sementes rosadas.
\section{Rodóstomo}
\begin{itemize}
\item {Grp. gram.:adj.}
\end{itemize}
\begin{itemize}
\item {Utilização:Zool.}
\end{itemize}
\begin{itemize}
\item {Proveniência:(Do gr. \textunderscore rhodon\textunderscore  + \textunderscore stoma\textunderscore )}
\end{itemize}
Que tem a boca rosada.
\section{Rodouça}
\begin{itemize}
\item {Grp. gram.:f.}
\end{itemize}
\begin{itemize}
\item {Proveniência:(De \textunderscore roda\textunderscore )}
\end{itemize}
Trapo, que forma uma roda, ou roda de trapos torcidos, que se põe na cabeça, para suster fardos e abrandar-lhes a pressão; rodilha.
\section{Rodovalho}
\begin{itemize}
\item {Grp. gram.:m.}
\end{itemize}
\begin{itemize}
\item {Utilização:Ant.}
\end{itemize}
\begin{itemize}
\item {Utilização:Pop.}
\end{itemize}
\begin{itemize}
\item {Proveniência:(Do cast. \textunderscore rodoballo\textunderscore )}
\end{itemize}
Nome de um peixe, (\textunderscore rhombus punctatus\textunderscore ).
Homem grosso e baixo.
\section{Rodriga}
\begin{itemize}
\item {Grp. gram.:f.  e  adj.}
\end{itemize}
\begin{itemize}
\item {Utilização:Prov.}
\end{itemize}
\begin{itemize}
\item {Utilização:trasm.}
\end{itemize}
\begin{itemize}
\item {Utilização:dur.}
\end{itemize}
Madeira ou estacas para vinhas e feijões.
\section{Rodrigão}
\begin{itemize}
\item {Grp. gram.:m.}
\end{itemize}
Processo de empar, que consiste em desennovelar a vide sôbre um moirão vertical. Cf.\textunderscore Techn. Rural\textunderscore , I, 552.
(Cast. \textunderscore rodrigón\textunderscore )
\section{Rodrigo-afonso}
\begin{itemize}
\item {Grp. gram.:m.}
\end{itemize}
Espécie de uva branca, talvez a mesma que \textunderscore camarate\textunderscore .
\section{Rodriguézia}
\begin{itemize}
\item {Grp. gram.:f.}
\end{itemize}
\begin{itemize}
\item {Proveniência:(De \textunderscore Rodríguez\textunderscore , n. p.)}
\end{itemize}
Gênero de orchídeas.
\section{Rodura}
\begin{itemize}
\item {Grp. gram.:f.}
\end{itemize}
\begin{itemize}
\item {Proveniência:(De \textunderscore rôdo\textunderscore ^1)}
\end{itemize}
Acto ou effeito de rodar^1.
Aquillo que se junta de uma vez com o rôdo.
\section{Rõe}
\begin{itemize}
\item {Grp. gram.:adj.}
\end{itemize}
\begin{itemize}
\item {Utilização:Pop.}
\end{itemize}
O mesmo que \textunderscore ruím\textunderscore .
\section{Roedeiro}
\begin{itemize}
\item {fónica:ro-e}
\end{itemize}
\begin{itemize}
\item {Grp. gram.:m.}
\end{itemize}
\begin{itemize}
\item {Proveniência:(De \textunderscore roer\textunderscore )}
\end{itemize}
Peça, para erguer o falcão, depois da comida.
\section{Roedor}
\begin{itemize}
\item {fónica:ro-e}
\end{itemize}
\begin{itemize}
\item {Grp. gram.:adj.}
\end{itemize}
\begin{itemize}
\item {Grp. gram.:M. pl.}
\end{itemize}
Que rói.
Ordem de mammíferos, que têm duas classes de dentes incisivos e mollares.
\section{Roedura}
\begin{itemize}
\item {fónica:ro-e}
\end{itemize}
\begin{itemize}
\item {Grp. gram.:f.}
\end{itemize}
\begin{itemize}
\item {Utilização:Gír.}
\end{itemize}
Acto ou effeito de roer.
Escoriação, causada por attrito.
Pesar, tristeza.
\section{Roel}
\begin{itemize}
\item {Grp. gram.:m.}
\end{itemize}
O mesmo que \textunderscore arruela\textunderscore .
\section{Roeméria}
\begin{itemize}
\item {fónica:ro-e}
\end{itemize}
\begin{itemize}
\item {Grp. gram.:f.}
\end{itemize}
\begin{itemize}
\item {Proveniência:(De \textunderscore Roemer\textunderscore , n. p.)}
\end{itemize}
Planta papaverácea.
\section{Roentgenterapia}
\begin{itemize}
\item {Grp. gram.:f.}
\end{itemize}
\begin{itemize}
\item {Proveniência:(De \textunderscore Roentgen\textunderscore , n. p. + \textunderscore terapia\textunderscore )}
\end{itemize}
Tratamento terapêutico pelos raios X.
\section{Roentgentherapia}
\begin{itemize}
\item {Grp. gram.:f.}
\end{itemize}
\begin{itemize}
\item {Proveniência:(De \textunderscore Roentgen\textunderscore , n. p. + \textunderscore therapia\textunderscore )}
\end{itemize}
Tratamento therapêutico pelos raios X.
\section{Roer}
\begin{itemize}
\item {Grp. gram.:v. t.}
\end{itemize}
\begin{itemize}
\item {Utilização:Fig.}
\end{itemize}
\begin{itemize}
\item {Grp. gram.:Loc.}
\end{itemize}
\begin{itemize}
\item {Utilização:fam.}
\end{itemize}
\begin{itemize}
\item {Grp. gram.:V. i.}
\end{itemize}
\begin{itemize}
\item {Proveniência:(Do lat. \textunderscore rodere\textunderscore )}
\end{itemize}
Cortar ou triturar com os dentes.
Morder.
Trilhar.
Gastar, corroer.
Destruír.
Ulcerar.
Minar.
Atormentar.
\textunderscore Roer a corda\textunderscore , faltar a uma promessa ou a um contrato.
Cortar ou gastar alguma coisa com os dentes.
\section{Rofego}
\begin{itemize}
\item {fónica:fê}
\end{itemize}
\begin{itemize}
\item {Grp. gram.:m.}
\end{itemize}
\begin{itemize}
\item {Proveniência:(De \textunderscore rofo\textunderscore )}
\end{itemize}
O mesmo ou melhor que \textunderscore refêgo\textunderscore . Cf. Camillo, \textunderscore Livro Negro\textunderscore , 106; Arn. Gama, \textunderscore Motim\textunderscore , 48.
\section{Rofo}
\begin{itemize}
\item {fónica:rô}
\end{itemize}
\begin{itemize}
\item {Grp. gram.:adj.}
\end{itemize}
\begin{itemize}
\item {Grp. gram.:M.}
\end{itemize}
\begin{itemize}
\item {Proveniência:(Do lat. \textunderscore rufus\textunderscore )}
\end{itemize}
Que tem asperezas ou rugas; que não é polido; fôsco.
Ruga; risco.
\section{Rogações}
\begin{itemize}
\item {Grp. gram.:f. pl.}
\end{itemize}
\begin{itemize}
\item {Proveniência:(Do lat. \textunderscore rogationes\textunderscore )}
\end{itemize}
Preces públicas; ladaínhas.
\section{Rogado}
\begin{itemize}
\item {Grp. gram.:adj.}
\end{itemize}
\begin{itemize}
\item {Proveniência:(De \textunderscore rogar\textunderscore )}
\end{itemize}
A quem se dirigem rogos; instado.
\textunderscore Fazer-se rogado\textunderscore , não ceder a um pedido, sem grandes instâncias para fazer valer a annuência.
\section{Rogador}
\begin{itemize}
\item {Grp. gram.:m.  e  adj.}
\end{itemize}
\begin{itemize}
\item {Proveniência:(Do lat. \textunderscore rogator\textunderscore )}
\end{itemize}
O que roga; intercessor; medianeiro.
\section{Rogal}
\begin{itemize}
\item {Grp. gram.:adj.}
\end{itemize}
\begin{itemize}
\item {Proveniência:(Lat. \textunderscore rogalis\textunderscore )}
\end{itemize}
Relativo á fogueira, em que se queimam cadáveres.
Relativo á pyra.
\section{Rogar}
\begin{itemize}
\item {Grp. gram.:v. t.}
\end{itemize}
\begin{itemize}
\item {Utilização:Prov.}
\end{itemize}
\begin{itemize}
\item {Grp. gram.:V. i.}
\end{itemize}
\begin{itemize}
\item {Proveniência:(Lat. \textunderscore rogare\textunderscore )}
\end{itemize}
Pedir com instância; pedir por favor; supplicar.
Assalariar para trabalhos agrícolas: \textunderscore rogar podadores\textunderscore .
Fazer súpplicas.
\section{Rogativa}
\begin{itemize}
\item {Grp. gram.:f.}
\end{itemize}
O mesmo que \textunderscore rôgo\textunderscore .
(Fem. de \textunderscore rogativo\textunderscore )
\section{Rogativo}
\begin{itemize}
\item {Grp. gram.:adj.}
\end{itemize}
Que roga.
Que envolve súpplica.
\section{Rogatória}
\begin{itemize}
\item {Grp. gram.:f.}
\end{itemize}
\begin{itemize}
\item {Utilização:Jur.}
\end{itemize}
\begin{itemize}
\item {Proveniência:(De rogatório)}
\end{itemize}
Rogativa.
Carta rogatória ou pedido, dirigido por fiéis de uma diocese a um metropolitano, para que certo ecclesiástico seja sagrado Bispo daquella diocese.
Pedido, que se faz a tribunaes estranjeiros, para que êstes realizem certos actos judiciaes.
\section{Rogatório}
\begin{itemize}
\item {Grp. gram.:adj.}
\end{itemize}
\begin{itemize}
\item {Proveniência:(Lat. \textunderscore rogatorius\textunderscore )}
\end{itemize}
Relativo a rôgo ou a súpplica; que envolve pedido: \textunderscore cartas rogatórias\textunderscore .
\section{Rogeira}
\begin{itemize}
\item {Grp. gram.:f.}
\end{itemize}
\begin{itemize}
\item {Utilização:Náut.}
\end{itemize}
Cabo, o mesmo que \textunderscore regeira\textunderscore .
\section{Rogéria}
\begin{itemize}
\item {Grp. gram.:f.}
\end{itemize}
\begin{itemize}
\item {Proveniência:(De \textunderscore Roger\textunderscore , n. p.)}
\end{itemize}
Gênero de plantas da África tropical.
\section{Rogiera}
\begin{itemize}
\item {Grp. gram.:f.}
\end{itemize}
\begin{itemize}
\item {Proveniência:(De \textunderscore Rogier\textunderscore , n. p.)}
\end{itemize}
Gênero de arbustos rubiáceos de Guatemala.
\section{Rôgo}
\begin{itemize}
\item {Grp. gram.:m.}
\end{itemize}
\begin{itemize}
\item {Utilização:Prov.}
\end{itemize}
Acto ou effeito de rogar.
Prece.
Antigo tributo, equivalente ao que se chamava \textunderscore geira\textunderscore .
Assalariamento de homens ou mulheres para trabalhos agrícolas.
\section{Rohan}
\begin{itemize}
\item {Grp. gram.:m.}
\end{itemize}
Espécie de tecido antigo, fabricada em Rohan.
\section{Roicisso}
\begin{itemize}
\item {Grp. gram.:f.}
\end{itemize}
\begin{itemize}
\item {Proveniência:(Do lat. \textunderscore rhois\textunderscore  + \textunderscore cissus\textunderscore )}
\end{itemize}
Gênero de videiras, da fam. das ampelideas.
\section{Roissenhor}
\begin{itemize}
\item {Grp. gram.:m.}
\end{itemize}
\begin{itemize}
\item {Proveniência:(Do cast. \textunderscore ruiseñor\textunderscore )}
\end{itemize}
Fórma antiga de \textunderscore rouxinol\textunderscore .
\section{Roixinol}
\begin{itemize}
\item {Grp. gram.:m.}
\end{itemize}
(Fórma pop. e ant. de \textunderscore rouxinol\textunderscore . Cf.\textunderscore Eufrosina\textunderscore , 169)
\section{Roixo}
\begin{itemize}
\item {Grp. gram.:adj.}
\end{itemize}
O mesmo que \textunderscore roxo\textunderscore .
\section{Rojador}
\textunderscore m.\textunderscore  e \textunderscore adj\textunderscore .
O que roja ou que se roja.
\section{Rojão}
\begin{itemize}
\item {Grp. gram.:m.}
\end{itemize}
\begin{itemize}
\item {Utilização:Pop.}
\end{itemize}
\begin{itemize}
\item {Proveniência:(De \textunderscore rojar\textunderscore )}
\end{itemize}
O mesmo que \textunderscore rôjo\textunderscore ^1.
Toque de viola, arrastado.
\section{Rojão}
\begin{itemize}
\item {Grp. gram.:m.}
\end{itemize}
Vara com choupa, para picar os toiros.
(Cast. \textunderscore rejón\textunderscore )
\section{Rojão}
\begin{itemize}
\item {Grp. gram.:m.}
\end{itemize}
O mesmo que \textunderscore torresmo\textunderscore .
(Cp.\textunderscore rijão\textunderscore ^1)
\section{Rojão}
\begin{itemize}
\item {Grp. gram.:m.}
\end{itemize}
\begin{itemize}
\item {Utilização:Bras}
\end{itemize}
Foguete.
Ruído, que o foguete produz quando sobe ao ar.
\section{Rojar}
\begin{itemize}
\item {Grp. gram.:v. t.}
\end{itemize}
\begin{itemize}
\item {Grp. gram.:V. i.}
\end{itemize}
\begin{itemize}
\item {Proveniência:(Do lat. hyp.\textunderscore rodicare\textunderscore , de \textunderscore rodere\textunderscore ?)}
\end{itemize}
Levar de rastos; arrastar.
Arrojar; arremessar.
Andar de rastos.
Arrastar-se pelo chão; roçar.
\section{Rôjo}
\begin{itemize}
\item {Grp. gram.:m.}
\end{itemize}
\begin{itemize}
\item {Utilização:Prov.}
\end{itemize}
\begin{itemize}
\item {Utilização:trasm.}
\end{itemize}
Acto ou effeito de rojar.
Som, produzido por êsse acto ou effeito.
Rodilhão (de silvas) para resguardo de paredes.
\section{Rôjo}
\begin{itemize}
\item {Grp. gram.:adj.}
\end{itemize}
\begin{itemize}
\item {Utilização:Prov.}
\end{itemize}
\begin{itemize}
\item {Utilização:trasm.}
\end{itemize}
Rubro; incandescente.
(Cast. \textunderscore rojo\textunderscore )
\section{Rojoneador}
\begin{itemize}
\item {Grp. gram.:m.}
\end{itemize}
Aquelle que rojoneia.
\section{Rojonear}
\begin{itemize}
\item {Grp. gram.:v.}
\end{itemize}
\begin{itemize}
\item {Utilização:t. Taur.}
\end{itemize}
\begin{itemize}
\item {Grp. gram.:V. i.}
\end{itemize}
Matar com o rojão^2.
Fazer a sorte do rojão.
\section{Rol}
\begin{itemize}
\item {Grp. gram.:m.}
\end{itemize}
\begin{itemize}
\item {Proveniência:(Do cast. \textunderscore rolde\textunderscore . Cf. G. Viana, \textunderscore Apostilas\textunderscore , vb.\textunderscore til\textunderscore )}
\end{itemize}
Relação, lista: \textunderscore rol da roupa suja\textunderscore .
Certo número ou porção.
\section{Rol}
\begin{itemize}
\item {Grp. gram.:m.}
\end{itemize}
\begin{itemize}
\item {Utilização:Ant.}
\end{itemize}
Peça de coiro, a que os caçadores atavam asas de aves e, fazendo-a girar, a soltavam aos falcões para os adestrar na caça. Cf. Fernandes, \textunderscore Caça de Altan.\textunderscore , (na \textunderscore Advertência\textunderscore ).
\section{Rol}
\begin{itemize}
\item {Grp. gram.:m.}
\end{itemize}
\begin{itemize}
\item {Utilização:Prov.}
\end{itemize}
\begin{itemize}
\item {Utilização:dur.}
\end{itemize}
\begin{itemize}
\item {Proveniência:(Do lat. \textunderscore ros\textunderscore , \textunderscore roris\textunderscore )}
\end{itemize}
O mesmo que \textunderscore relento\textunderscore : \textunderscore o pobrezinho passou a noite ao rol\textunderscore .
O mesmo que \textunderscore orvalho\textunderscore . Cf. Júl. Moreira, \textunderscore Estudos\textunderscore , II, 291.
\section{Róla}
\begin{itemize}
\item {Grp. gram.:f.}
\end{itemize}
\begin{itemize}
\item {Utilização:T. de Aveiro}
\end{itemize}
\begin{itemize}
\item {Proveniência:(De \textunderscore rolar\textunderscore ^1)}
\end{itemize}
\textunderscore Andar á rola\textunderscore , andar á mercê das ondas, (falando-se da bateira).
\section{Rôla}
\begin{itemize}
\item {Grp. gram.:f.}
\end{itemize}
\begin{itemize}
\item {Utilização:Gír.}
\end{itemize}
\begin{itemize}
\item {Utilização:Gír.}
\end{itemize}
\begin{itemize}
\item {Proveniência:(T. onom.)}
\end{itemize}
Ave, semelhante á pomba, (\textunderscore columba turtur\textunderscore ).
Criada, que chegou da província á cidade.
Caldo.
\section{Rôla}
\begin{itemize}
\item {Grp. gram.:f.}
\end{itemize}
\begin{itemize}
\item {Utilização:Gír.}
\end{itemize}
O mesmo que \textunderscore caldo\textunderscore .
\section{Rôla-do-mar}
\begin{itemize}
\item {Grp. gram.:f.}
\end{itemize}
\begin{itemize}
\item {Utilização:Prov.}
\end{itemize}
O mesmo que \textunderscore maçarico\textunderscore  ou \textunderscore rôla-marinha\textunderscore .
\section{Rolador}
\begin{itemize}
\item {Grp. gram.:m.}
\end{itemize}
\begin{itemize}
\item {Proveniência:(De \textunderscore rolar\textunderscore ^1)}
\end{itemize}
Peça do maquinismo da tracção eléctrica.
\section{Rolador}
\begin{itemize}
\item {Grp. gram.:adj.}
\end{itemize}
\begin{itemize}
\item {Proveniência:(De \textunderscore rolar\textunderscore ^2)}
\end{itemize}
Que rola ou arrulha.
\section{Rolagem}
\begin{itemize}
\item {Grp. gram.:f.}
\end{itemize}
\begin{itemize}
\item {Utilização:Agr.}
\end{itemize}
\begin{itemize}
\item {Proveniência:(De \textunderscore rôlo\textunderscore ^1)}
\end{itemize}
Acto de esmiuçar os torrões, que a grade não partiu, e aconchegar o terreno para lhe conservar a humidade e facilitar a germinação das sementes.
\section{Rôla-marinha}
\begin{itemize}
\item {Grp. gram.:f.}
\end{itemize}
\begin{itemize}
\item {Utilização:Prov.}
\end{itemize}
\begin{itemize}
\item {Utilização:dur.}
\end{itemize}
O mesmo que \textunderscore maçarico\textunderscore .
\section{Rolamento}
\begin{itemize}
\item {Grp. gram.:m.}
\end{itemize}
Acto de rolar^1, ou de sêr impellido pelo rôlo das águas, (falando-se do navio) Cf. Celestino Soares, \textunderscore Quadros Navaes\textunderscore .
\section{Rolândico}
\begin{itemize}
\item {Grp. gram.:adj.}
\end{itemize}
\begin{itemize}
\item {Utilização:Anat.}
\end{itemize}
\begin{itemize}
\item {Proveniência:(De \textunderscore Rolando\textunderscore , n. p.)}
\end{itemize}
Diz-se do sulco, que separa as circunvoluções parietal ascendente e parietal descendente.
\section{Rolante}
\begin{itemize}
\item {Grp. gram.:adj.}
\end{itemize}
\begin{itemize}
\item {Proveniência:(De \textunderscore rolar\textunderscore ^1)}
\end{itemize}
Que róla, que gira.
\section{Rolantear}
\begin{itemize}
\item {Grp. gram.:v. i.}
\end{itemize}
\begin{itemize}
\item {Utilização:Gír.}
\end{itemize}
\begin{itemize}
\item {Proveniência:(De \textunderscore rolante\textunderscore )}
\end{itemize}
Dar facadas.
\section{Rolão}
\begin{itemize}
\item {Grp. gram.:m.}
\end{itemize}
\begin{itemize}
\item {Utilização:Náut.}
\end{itemize}
\begin{itemize}
\item {Utilização:Prov.}
\end{itemize}
\begin{itemize}
\item {Proveniência:(De \textunderscore rôlo\textunderscore )}
\end{itemize}
A parte mais grossa do trigo moído.
Rôlo de pau, que se colloca debaixo das grandes pedras e de vários fardos, para lhes facilitar a deslocação.
Farinha de aveia, com que se faz uma decocção para lavagem do corpo.
Grande rôlo ou vagalhão:«\textunderscore vimos junto de nós o rolão e escuma dos mares\textunderscore ».\textunderscore Hist. Trág. Marit.\textunderscore , 178.
Gente ordinária, populacho, ralé.
\section{Rolão}
\begin{itemize}
\item {Grp. gram.:adj.}
\end{itemize}
(?):«\textunderscore esta espada é roloa e este broquel rolão\textunderscore ». G. Vicente, I, 229.
(Relacionar-se-á com \textunderscore rolão\textunderscore ^1 e significará \textunderscore ordinário\textunderscore , \textunderscore reles\textunderscore ?)
\section{Rolão}
\begin{itemize}
\item {Grp. gram.:m.}
\end{itemize}
\begin{itemize}
\item {Utilização:Bras. do N}
\end{itemize}
Rôla cinzenta, de carne saborosa.
\section{Rolar}
\begin{itemize}
\item {Grp. gram.:v. t.}
\end{itemize}
\begin{itemize}
\item {Utilização:Náut.}
\end{itemize}
\begin{itemize}
\item {Proveniência:(De \textunderscore rôlo\textunderscore ^1)}
\end{itemize}
Fazer girar.
Cortar em rolos ou toros (uma árvore).
\textunderscore V. i.\textunderscore  e \textunderscore p\textunderscore .
Avançar, girando sôbre si mesmo.
Rebolar-se.
Cair, girando: \textunderscore o penedo rolou pela encosta abaixo\textunderscore .
Redemoínhar.
Éncapellar-se.
Descair (a embarcação) para sotavento, deslocando obliquamente o rumo.
\section{Rolar}
\begin{itemize}
\item {Grp. gram.:v. i.}
\end{itemize}
\begin{itemize}
\item {Grp. gram.:V. t.}
\end{itemize}
\begin{itemize}
\item {Utilização:Ext.}
\end{itemize}
\begin{itemize}
\item {Proveniência:(De \textunderscore rôla\textunderscore ^1)}
\end{itemize}
Soltar a voz, (falando-se das rôlas).
Arrulhar.
Exprimir, arrulhando.
Exprimir com meiguice.
\section{Rolbélia}
\begin{itemize}
\item {Grp. gram.:f.}
\end{itemize}
Gênero de plantas gramineas.
\section{Rolda}
\begin{itemize}
\item {Grp. gram.:f.}
\end{itemize}
\begin{itemize}
\item {Utilização:Des.}
\end{itemize}
O mesmo que \textunderscore ronda\textunderscore .
Cf. D. Franc. Manuel, \textunderscore Carta de Guia\textunderscore , 135.
\section{Roldana}
\begin{itemize}
\item {Grp. gram.:f.}
\end{itemize}
\begin{itemize}
\item {Proveniência:(De \textunderscore rolutana\textunderscore , metáth. do lat. hyp.\textunderscore rotulana\textunderscore , de \textunderscore rotula\textunderscore )}
\end{itemize}
Maquinismo, com uma roda girante, por cuja circumferência cavada passa uma corda ou corrente.
\section{Roldão}
\begin{itemize}
\item {Grp. gram.:m.}
\end{itemize}
\begin{itemize}
\item {Proveniência:(Do fr. \textunderscore rondon\textunderscore )}
\end{itemize}
Confusão; sobresalto; precipitação.
\section{Roldar}
\begin{itemize}
\item {Grp. gram.:v. t.}
\end{itemize}
\begin{itemize}
\item {Utilização:Ant.}
\end{itemize}
\begin{itemize}
\item {Proveniência:(De \textunderscore rolda\textunderscore )}
\end{itemize}
O mesmo que \textunderscore rondar\textunderscore .
\section{Roldear}
\begin{itemize}
\item {Grp. gram.:v. t.}
\end{itemize}
\begin{itemize}
\item {Utilização:Prov.}
\end{itemize}
\begin{itemize}
\item {Utilização:minh.}
\end{itemize}
\begin{itemize}
\item {Proveniência:(De \textunderscore rolda\textunderscore )}
\end{itemize}
Dividir (água regadia) pelos campos ou consortes da levada ou do rêgo.
\section{Roleira}
\begin{itemize}
\item {Grp. gram.:f.}
\end{itemize}
\begin{itemize}
\item {Proveniência:(De \textunderscore rôlo\textunderscore ^1)}
\end{itemize}
Palmatória, para rôlo ou pavio de cera.
O mesmo que \textunderscore cremalheira\textunderscore .
\section{Roleiro}
\begin{itemize}
\item {Grp. gram.:adj.}
\end{itemize}
\begin{itemize}
\item {Grp. gram.:M.}
\end{itemize}
\begin{itemize}
\item {Utilização:Prov.}
\end{itemize}
\begin{itemize}
\item {Utilização:alent.}
\end{itemize}
\begin{itemize}
\item {Proveniência:(De \textunderscore rolar\textunderscore ^1)}
\end{itemize}
Que rola, que gira.
Meda de trigo, de fórma cónica.
\section{Roleiro}
\begin{itemize}
\item {Grp. gram.:adj.}
\end{itemize}
\begin{itemize}
\item {Utilização:Ant.}
\end{itemize}
\begin{itemize}
\item {Proveniência:(De \textunderscore rolar\textunderscore ^2?)}
\end{itemize}
Manso como as rôlas?:«\textunderscore ...o maior trabalho que tem com elles\textunderscore  (falcões) \textunderscore é fazellos domésticos e roleiros e mansos...\textunderscore »Fernandes, \textunderscore Caça de Altan.\textunderscore , p. III, c. I.
\section{Ròleiro}
\begin{itemize}
\item {Grp. gram.:m.}
\end{itemize}
\begin{itemize}
\item {Utilização:Prov.}
\end{itemize}
\begin{itemize}
\item {Utilização:trasm.}
\end{itemize}
Aquelle que faz o rol.
\section{Roleta}
\begin{itemize}
\item {fónica:lê}
\end{itemize}
\begin{itemize}
\item {Grp. gram.:f.}
\end{itemize}
\begin{itemize}
\item {Utilização:Fam.}
\end{itemize}
\begin{itemize}
\item {Proveniência:(Fr. \textunderscore roullete\textunderscore )}
\end{itemize}
Espécie de jôgo de asar, que consta de um cylindro, em cujo centro há uma peça com lugares numerados.
Boato falso, galga.
\section{Rolete}
\begin{itemize}
\item {fónica:lê}
\end{itemize}
\begin{itemize}
\item {Grp. gram.:m.}
\end{itemize}
\begin{itemize}
\item {Utilização:Ant.}
\end{itemize}
\begin{itemize}
\item {Utilização:Açor}
\end{itemize}
\begin{itemize}
\item {Utilização:Bras. do N}
\end{itemize}
\begin{itemize}
\item {Proveniência:(De \textunderscore rôlo\textunderscore )}
\end{itemize}
Pequeno rôlo.
Instrumento, com que os chapeleiros endireitam o fundo dos chapéus.
Trança de cabello, enrolada em espiral no alto da cabeça.
A parte mais grossa e central do rodeiro do carro.
Rodela de cana descascada, para se chupar.
\section{Rôlha}
\begin{itemize}
\item {Grp. gram.:f.}
\end{itemize}
\begin{itemize}
\item {Utilização:Chul.}
\end{itemize}
\begin{itemize}
\item {Utilização:Fig.}
\end{itemize}
\begin{itemize}
\item {Utilização:Gír.}
\end{itemize}
\begin{itemize}
\item {Proveniência:(Do lat. \textunderscore rotula\textunderscore )}
\end{itemize}
Peça geralmente cylindrica, para tapar a bôca ou o gargalo de certos vasos.
Patife.
Pessôa manhosa: \textunderscore és bôa rôlha, não há dúvida\textunderscore .
Imposição de silêncio.
Repressão da liberdade de falar ou de escrever.
Acto de fazer trinta pontos no jôgo da bisca.
Juízo, tino.
\section{Rolha-do-maluvo}
\begin{itemize}
\item {Grp. gram.:f.}
\end{itemize}
Arbusto africano, com cujas fôlhas simples, denteadas, os Indígenas tapam as cabeças do maluvo.
\section{Rolhador}
\begin{itemize}
\item {Grp. gram.:m.}
\end{itemize}
Apparelho ou utensílio, para rolhar garrafas. Cf.\textunderscore Techn. Rur.\textunderscore , I, 285.
\section{Rolhadura}
\begin{itemize}
\item {Grp. gram.:f.}
\end{itemize}
Acto de rolhar.
\section{Rolhagem}
\begin{itemize}
\item {Grp. gram.:f.}
\end{itemize}
O mesmo que \textunderscore rolhadura\textunderscore . Cf.\textunderscore Tech. Rur.\textunderscore , 284 e 288.
\section{Rolhar}
\begin{itemize}
\item {Grp. gram.:v. t.}
\end{itemize}
O mesmo que \textunderscore arrolhar\textunderscore .
\section{Rolheiro}
\begin{itemize}
\item {Grp. gram.:m.}
\end{itemize}
Aquelle que faz rôlhas.
Aquelle que trabalha em cortiça.
Mólho de trigo ou centeio, atado pelo meio.
Redemoínho de água.
O mesmo que \textunderscore roleiro\textunderscore ^1.
\section{Rolhista}
\begin{itemize}
\item {Grp. gram.:m.  e  f.}
\end{itemize}
Pessôa, que trabalha em rôlhas de cortiça. Cf.\textunderscore Inquér. Industr.\textunderscore , p. II, l. III, 230 e 231.
\section{Rôlho}
\begin{itemize}
\item {Grp. gram.:adj.}
\end{itemize}
\begin{itemize}
\item {Utilização:Pop.}
\end{itemize}
\begin{itemize}
\item {Utilização:Ant.}
\end{itemize}
Gordo; nédio.
A rótula do joêlho.
(Cp.\textunderscore rôlha\textunderscore )
\section{Roliçar}
\begin{itemize}
\item {Grp. gram.:v. t.}
\end{itemize}
\begin{itemize}
\item {Utilização:bras}
\end{itemize}
\begin{itemize}
\item {Utilização:Neol.}
\end{itemize}
Tornar roliço.
\section{Roliço}
\begin{itemize}
\item {Grp. gram.:adj.}
\end{itemize}
\begin{itemize}
\item {Proveniência:(De \textunderscore rôlo\textunderscore ^1)}
\end{itemize}
Que tem fórma de rôlo; redondo.
Gordo; nédio e de fórmas arredondadas: \textunderscore braços roliços\textunderscore .
\section{Rolieiro}
\begin{itemize}
\item {Grp. gram.:m.}
\end{itemize}
Pássaro tenuirostro, (\textunderscore coracias garrula\textunderscore , Lin.).
\section{Rolim}
\begin{itemize}
\item {Grp. gram.:m.}
\end{itemize}
O mesmo que \textunderscore peixe-roda\textunderscore , ou o mesmo que \textunderscore rodim\textunderscore ?
\section{Rolim}
\begin{itemize}
\item {Grp. gram.:m.}
\end{itemize}
Espécie de emissário ou embaixador, entre povos do extremo Oriente. Cf.\textunderscore Peregrinação\textunderscore , CLIV.
\section{Rollândia}
\begin{itemize}
\item {Grp. gram.:f.}
\end{itemize}
\begin{itemize}
\item {Proveniência:(De \textunderscore Rolland\textunderscore , n. p.)}
\end{itemize}
Gênero de arbustos lobeliáceos das ilhas de Sandwich.
\section{Rollandiano}
\begin{itemize}
\item {Grp. gram.:adj.}
\end{itemize}
\begin{itemize}
\item {Proveniência:(De \textunderscore Rolland\textunderscore , n. p.)}
\end{itemize}
Diz-se de uma notável typographia, que houve em Lisbôa; e diz-se das edições feitas nessa typographia.
\section{Rôlo}
\begin{itemize}
\item {Grp. gram.:m.}
\end{itemize}
\begin{itemize}
\item {Utilização:Fig.}
\end{itemize}
\begin{itemize}
\item {Utilização:Bras}
\end{itemize}
\begin{itemize}
\item {Utilização:Agr.}
\end{itemize}
\begin{itemize}
\item {Proveniência:(Do lat. \textunderscore robulus\textunderscore )}
\end{itemize}
Cylindro, mais ou menos comprido.
Objecto, em fórma de cylindro: \textunderscore um rôlo de papel\textunderscore .
Pavio de cera.
Embrulho, pacote.
Vagalhão.
Remoínho.
Cabello enrolado, riço.
Crivo de folha, na parte interna dos funís.
Multidão de gente.
Barulho, motim; acto de brigar corpo a corpo.
Cylindro, de superfície lisa, para partir e aconchegar os torrões que a grade não esboroou.
\section{Rôlo}
\begin{itemize}
\item {Grp. gram.:m.}
\end{itemize}
Macho da rôla.
\section{Rom}
\begin{itemize}
\item {Grp. gram.:m.}
\end{itemize}
Espécie de tinta amarela.
\section{Romã}
\begin{itemize}
\item {Grp. gram.:f.}
\end{itemize}
\begin{itemize}
\item {Utilização:Ant.}
\end{itemize}
\begin{itemize}
\item {Proveniência:(Do ár. \textunderscore romman\textunderscore )}
\end{itemize}
Fruto da romanzeira.
A parte mais grossa do mastro ou mastaréu.
Artifício pyrotéchnico, que se usava a bordo.
\section{Romagem}
\begin{itemize}
\item {Grp. gram.:f.}
\end{itemize}
O mesmo que \textunderscore romaria\textunderscore .
\section{Romaica}
\begin{itemize}
\item {Grp. gram.:f.}
\end{itemize}
\begin{itemize}
\item {Proveniência:(De \textunderscore romaico\textunderscore )}
\end{itemize}
Dança nacional dos Gregos modernos.
\section{Romaico}
\begin{itemize}
\item {Grp. gram.:adj.}
\end{itemize}
\begin{itemize}
\item {Grp. gram.:M.}
\end{itemize}
\begin{itemize}
\item {Proveniência:(Gr. \textunderscore romaikos\textunderscore )}
\end{itemize}
Relativo aos Gregos modernos ou á sua língua.
Língua moderna dos Gregos.
\section{Roman}
\begin{itemize}
\item {Grp. gram.:f.}
\end{itemize}
\begin{itemize}
\item {Utilização:Ant.}
\end{itemize}
\begin{itemize}
\item {Proveniência:(Do ár. \textunderscore romman\textunderscore )}
\end{itemize}
Fruto da romanzeira.
A parte mais grossa do mastro ou mastaréu.
Artifício pyrotéchnico, que se usava a bordo.
\section{Roman}
\begin{itemize}
\item {Grp. gram.:f.}
\end{itemize}
\begin{itemize}
\item {Utilização:Ant.}
\end{itemize}
\begin{itemize}
\item {Proveniência:(De \textunderscore romão\textunderscore )}
\end{itemize}
Mulhér, natural de Roma:«\textunderscore não ouviste já das romans e gregas com que esforço morrêrão...?\textunderscore »A. Ferreira, \textunderscore Castro\textunderscore .
\section{Romana}
\begin{itemize}
\item {Grp. gram.:f.}
\end{itemize}
\begin{itemize}
\item {Proveniência:(Do ár. \textunderscore rommana\textunderscore )}
\end{itemize}
Espécie de balança.
\section{Romanamente}
\begin{itemize}
\item {Grp. gram.:adv.}
\end{itemize}
Á maneira dos Romanos.
\section{Romança}
\begin{itemize}
\item {Grp. gram.:f.}
\end{itemize}
\begin{itemize}
\item {Utilização:Neol.}
\end{itemize}
\begin{itemize}
\item {Proveniência:(De \textunderscore romance\textunderscore )}
\end{itemize}
Canção, de assumpto histórico.
Melodia sentimental, para canto.
\section{Romançada}
\begin{itemize}
\item {Grp. gram.:f.}
\end{itemize}
\begin{itemize}
\item {Utilização:deprec.}
\end{itemize}
\begin{itemize}
\item {Utilização:Fam.}
\end{itemize}
Porção de romances; os romances.
\section{Romançaria}
\begin{itemize}
\item {Grp. gram.:f.}
\end{itemize}
\begin{itemize}
\item {Utilização:Fam.}
\end{itemize}
Porção de romances, romançada; os romances, em geral. Cf. Camillo, \textunderscore Narcót.\textunderscore , I, 118.
\section{Romance}
\begin{itemize}
\item {Grp. gram.:f.}
\end{itemize}
\begin{itemize}
\item {Utilização:Fig.}
\end{itemize}
\begin{itemize}
\item {Grp. gram.:Adj.}
\end{itemize}
\begin{itemize}
\item {Proveniência:(Do lat. \textunderscore romanice\textunderscore , adv.)}
\end{itemize}
Narração histórica, em versos simples e apropriada ao canto.
Dialecto ou conjunto de dialectos derivados do latim.
Narração, escrita em linguagem antiga.
História fabulosa, em que o autor procura despertar interesse pela pintura das paixões e costumes ou pela singularidade das aventuras.
Novella, conto.
Fantasia, objecto imaginário.
Enrêdo de falsidades.
Conjunto dos idiomas românicos; romanço.
O mesmo que \textunderscore românico\textunderscore .
\section{Romancear}
\begin{itemize}
\item {Grp. gram.:v. t.}
\end{itemize}
Contar ou descrever em romance: \textunderscore romancear paixões\textunderscore .
Dar fórma agradavel a.
Adaptar á língua vernácula (termos de outras línguas).
\section{Romanceiro}
\begin{itemize}
\item {Grp. gram.:m.}
\end{itemize}
\begin{itemize}
\item {Grp. gram.:Adj.}
\end{itemize}
Collecção de romances.
O mesmo que \textunderscore romântico\textunderscore . Cf. Castilho, \textunderscore Misanthropo\textunderscore , 185.
\section{Romanceira}
\begin{itemize}
\item {Grp. gram.:f.}
\end{itemize}
\begin{itemize}
\item {Utilização:Fam.}
\end{itemize}
Porção de romances, romançada; os romances, em geral. Cf. Camillo, \textunderscore Narcót.\textunderscore , I, 118.
\section{Romanche}
\begin{itemize}
\item {Grp. gram.:adj.}
\end{itemize}
\begin{itemize}
\item {Grp. gram.:M.}
\end{itemize}
Diz-se de alguns dialectos que são vernáculos na Suissa.
Dialecto dos Grisões; rhético.
(Outra fórma de \textunderscore romance\textunderscore )
\section{Romancice}
\begin{itemize}
\item {Grp. gram.:f.}
\end{itemize}
\begin{itemize}
\item {Utilização:Deprec.}
\end{itemize}
\begin{itemize}
\item {Proveniência:(De \textunderscore romance\textunderscore )}
\end{itemize}
Fantasia romântica; devaneio.
\section{Romancismo}
\begin{itemize}
\item {Grp. gram.:m.}
\end{itemize}
\begin{itemize}
\item {Proveniência:(De \textunderscore romance\textunderscore )}
\end{itemize}
Carácter romântico; ficções ou descripções românticas.
\section{Romancista}
\begin{itemize}
\item {Grp. gram.:m.  ou  f.}
\end{itemize}
Pessôa, que faz romances; novellista.
\section{Romanço}
\begin{itemize}
\item {Grp. gram.:adj.}
\end{itemize}
\begin{itemize}
\item {Utilização:Ant.}
\end{itemize}
\begin{itemize}
\item {Grp. gram.:M.}
\end{itemize}
Românico, novilatino.
Conjunto das línguas românicas; romance.
(Cp.\textunderscore romance\textunderscore )
\section{Romando}
\begin{itemize}
\item {Grp. gram.:adj.}
\end{itemize}
\begin{itemize}
\item {Grp. gram.:M. pl.}
\end{itemize}
Relativo aos Romandos.
Um dos povos, que constituem a moderna Suíssa.
\section{Romanescado}
\begin{itemize}
\item {Grp. gram.:adj.}
\end{itemize}
\begin{itemize}
\item {Proveniência:(De \textunderscore romanescar\textunderscore )}
\end{itemize}
Que tem modos romanescos.
\section{Romanescamente}
\begin{itemize}
\item {Grp. gram.:adv.}
\end{itemize}
De modo romanesco; romanticamente.
\section{Romanescar}
\begin{itemize}
\item {Grp. gram.:v. t.}
\end{itemize}
Tornar romanesco:«\textunderscore ...o nosso ministro, a quem sinto nesta parte muito romanescado.\textunderscore »Vieira, \textunderscore Cartas\textunderscore .
\section{Romanesco}
\begin{itemize}
\item {fónica:nês}
\end{itemize}
\begin{itemize}
\item {Grp. gram.:adj.}
\end{itemize}
\begin{itemize}
\item {Utilização:Ext.}
\end{itemize}
\begin{itemize}
\item {Grp. gram.:M.}
\end{itemize}
\begin{itemize}
\item {Proveniência:(De \textunderscore romano\textunderscore )}
\end{itemize}
Que tem o carácter de romance.
Romântico.
Maravilhoso.
Devaneador.
Apaixonado.
Fabuloso.
O carácter romântico; o gênero romanesco.
\section{Romanho}
\begin{itemize}
\item {Grp. gram.:m.}
\end{itemize}
Uma das fórmas ou linguagens da gíria dos ciganos. Cf. Ad. Coelho, \textunderscore Ciganos\textunderscore , 44.
Provavelmente, o mesmo que \textunderscore romani\textunderscore .
\section{Romani}
\begin{itemize}
\item {Grp. gram.:m.}
\end{itemize}
Língua dos ciganos do oriente da Europa, de que há treze dialectos, mais ou menos adulterados. Cf. João Ribeiro, \textunderscore Diccion. Gram.\textunderscore , 84.
\section{România}
\begin{itemize}
\item {Grp. gram.:f.}
\end{itemize}
\begin{itemize}
\item {Utilização:Ant.}
\end{itemize}
(?)«\textunderscore ...amainarão as velas de romania\textunderscore ». Gaspar Correia, \textunderscore Lendas\textunderscore , III, 666.«\textunderscore ...dois mil paus de romania, que lhe lá comprarão para o armazém de Lisboa.\textunderscore »Fern. Lopes, \textunderscore Chrón. de D. Fern.\textunderscore , c. LIV.«\textunderscore ...pelas partes que o mar arrebentava, veio de romania a carga arrombando os paioes.\textunderscore »\textunderscore Hist. Trág. Marit.\textunderscore , 52.
\section{Românico}
\begin{itemize}
\item {Grp. gram.:adj.}
\end{itemize}
\begin{itemize}
\item {Grp. gram.:M.}
\end{itemize}
\begin{itemize}
\item {Proveniência:(Lat. \textunderscore romanicus\textunderscore )}
\end{itemize}
Diz-se das línguas, que se formaram do latim vulgar.
Relativo a essas línguas.
Diz se das literaturas, em que se manifestaram os idiomas românicos.
Diz-se da architectura, especialmente da architectura religiosa, que succedeu á chamada architectura latina, predominando entre os séculos VIII e XI.
Conjunto das línguas novi-latinas.
\section{Romanim}
\begin{itemize}
\item {Grp. gram.:m.}
\end{itemize}
\begin{itemize}
\item {Proveniência:(De \textunderscore romano\textunderscore )}
\end{itemize}
Moéda, cunhada em Avinhão, quando ali se estabeleceram os Papas.
\section{Romanisco}
\begin{itemize}
\item {Grp. gram.:adj.}
\end{itemize}
\begin{itemize}
\item {Utilização:Des.}
\end{itemize}
\begin{itemize}
\item {Proveniência:(De \textunderscore romano\textunderscore )}
\end{itemize}
Perito em coisas e negócios de Roma.
\section{Romanismo}
\begin{itemize}
\item {Grp. gram.:m.}
\end{itemize}
\begin{itemize}
\item {Utilização:Jur.}
\end{itemize}
\begin{itemize}
\item {Proveniência:(De \textunderscore romano\textunderscore )}
\end{itemize}
Texto ou qualquer elemento de direito romano, introduzido no direito pátrio.
Opinião especial de um romanista.
\section{Romanista}
\begin{itemize}
\item {Grp. gram.:m.}
\end{itemize}
\begin{itemize}
\item {Proveniência:(De \textunderscore romano\textunderscore )}
\end{itemize}
Indivíduo, que se occupa de jurisprudência, de história, ou de outros assumptos concernentes aos Romanos.
Philólogo, que trata especialmente do línguas românicas e respectivas literaturas.
Partidário do Papa; papista.
\section{Romanização}
\begin{itemize}
\item {Grp. gram.:f.}
\end{itemize}
Acto ou effeito de romanizar.
\section{Romanizar}
\begin{itemize}
\item {Grp. gram.:v. t.}
\end{itemize}
\begin{itemize}
\item {Proveniência:(De \textunderscore romano\textunderscore )}
\end{itemize}
Tornar romano.
Dar alguma coisa de romanesco a.
Adaptar á índole das línguas românicas.
\section{Romanizável}
\begin{itemize}
\item {Grp. gram.:adj.}
\end{itemize}
\begin{itemize}
\item {Proveniência:(De \textunderscore romanizar\textunderscore )}
\end{itemize}
Que se póde adaptar á índole das línguas românicas.
\section{Romano}
\begin{itemize}
\item {Grp. gram.:adj.}
\end{itemize}
\begin{itemize}
\item {Grp. gram.:M.}
\end{itemize}
\begin{itemize}
\item {Proveniência:(Lat. \textunderscore romanus\textunderscore )}
\end{itemize}
Relativo a Roma ou aos romanos; românico.
Diz-se do estilo ou da architectura usada entre os séculos V e XII.
Habitante de Roma.
Dialecto de Roma.
O mesmo que \textunderscore romanho\textunderscore .
\section{Romanó}
\begin{itemize}
\item {Grp. gram.:m.}
\end{itemize}
O mesmo que \textunderscore romanho\textunderscore .
\section{Romano-celta}
\begin{itemize}
\item {Grp. gram.:adj.}
\end{itemize}
Relativo aos Romanos e Celtas.
\section{Romanólogo}
\begin{itemize}
\item {Grp. gram.:m.}
\end{itemize}
\begin{itemize}
\item {Proveniência:(De \textunderscore romano\textunderscore  + gr. \textunderscore logos\textunderscore )}
\end{itemize}
Homem, perito em Philologia românica; romanista.
\section{Romanticamente}
\begin{itemize}
\item {Grp. gram.:adv.}
\end{itemize}
De modo romântico.
\section{Romanticismo}
\begin{itemize}
\item {Grp. gram.:m.}
\end{itemize}
O mesmo que \textunderscore romantismo\textunderscore .
Qualidade do romântico ou romanesco:«\textunderscore estranhou o romantismo do caso.\textunderscore »Camillo, \textunderscore Mulher Fatal\textunderscore , 46.
\section{Romântico}
\begin{itemize}
\item {Grp. gram.:adj.}
\end{itemize}
\begin{itemize}
\item {Grp. gram.:M.}
\end{itemize}
\begin{itemize}
\item {Proveniência:(Do fr. \textunderscore romantique\textunderscore )}
\end{itemize}
Relativo a romance ou próprio dêlle.
Fantasioso.
Poético.
Devaneador: \textunderscore meninas românticas\textunderscore .
Sectário do romantismo.
Escritor, que se desvia das regras clássicas.
Aquelle ou aquillo que tem carácter romanesco.
\section{Romantismo}
\begin{itemize}
\item {Grp. gram.:m.}
\end{itemize}
\begin{itemize}
\item {Proveniência:(De \textunderscore romântico\textunderscore )}
\end{itemize}
Systema dos escritores românticos, ou daquelles que abandonaram as fórmas clássicas.
Carácter daquelle ou daquillo que é romântico, ou romanesco.
\section{Romantizar}
\begin{itemize}
\item {Grp. gram.:v. t.}
\end{itemize}
\begin{itemize}
\item {Grp. gram.:V. i.  e  p.}
\end{itemize}
Tornar romântico.
Contar em fórma de romance: \textunderscore romantizar amores\textunderscore .
Fantasiar.
Dar ares de romântico.
Idear romances:«\textunderscore ...traduzo Ovidio, romantizo, edifico os meus castellos...\textunderscore »Castilho, \textunderscore Escavações\textunderscore , 15.
\section{Romanza}
\begin{itemize}
\item {Grp. gram.:f.}
\end{itemize}
(V.\textunderscore romança\textunderscore )
\section{Romanzeira}
\begin{itemize}
\item {Grp. gram.:f.}
\end{itemize}
\begin{itemize}
\item {Proveniência:(De \textunderscore roman\textunderscore )}
\end{itemize}
Gênero de árvores, da fam. das myrtáceas.
\section{Romanzeiral}
\begin{itemize}
\item {Grp. gram.:m.}
\end{itemize}
Terreno, plantado de romanzeiras.
\section{Romão}
\begin{itemize}
\item {Grp. gram.:adj.}
\end{itemize}
\begin{itemize}
\item {Grp. gram.:M.}
\end{itemize}
\begin{itemize}
\item {Utilização:Ant.}
\end{itemize}
\begin{itemize}
\item {Proveniência:(Lat. \textunderscore romanus\textunderscore )}
\end{itemize}
O mesmo que \textunderscore românico\textunderscore .
Diz-se do estilo architectónico que vigorou desde o século V ao XII, reflexo do estilo romano, e que também se chamou byzantino, romano-byzantino e gótico, antigo.
Habitante de Roma ou súbdito do Rei de Roma.
\section{Romão}
\begin{itemize}
\item {Grp. gram.:m.}
\end{itemize}
\begin{itemize}
\item {Utilização:Prov.}
\end{itemize}
\begin{itemize}
\item {Utilização:trasm.}
\end{itemize}
Recorte no eixo do carro, onde assentam os malhetes, e que fica entre as treitoeiras.
\section{Romaria}
\begin{itemize}
\item {Grp. gram.:f.}
\end{itemize}
\begin{itemize}
\item {Utilização:Fig.}
\end{itemize}
\begin{itemize}
\item {Proveniência:(De \textunderscore romeiro\textunderscore )}
\end{itemize}
Peregrinação religiosa.
Jornada de pessôas devotas a um lugar sagrado ou de carácter religioso.
Festa de arraial.
Reunião de pessôas que, concorrendo a uma festa religiosa, formam arraial junto ao lugar dessa festa e alli se divertem.
Reunião de pessôas que jornadeiam.
Multidão.
\section{Rombamente}
\begin{itemize}
\item {Grp. gram.:adv.}
\end{itemize}
De modo rombo, estupidamente.
\section{Rômbico}
\begin{itemize}
\item {Grp. gram.:adj.}
\end{itemize}
Que tem fórma de rombo.
\section{Rombífero}
\begin{itemize}
\item {Grp. gram.:adj.}
\end{itemize}
\begin{itemize}
\item {Utilização:Miner.}
\end{itemize}
\begin{itemize}
\item {Proveniência:(Do lat. \textunderscore rhombus\textunderscore  + \textunderscore ferre\textunderscore )}
\end{itemize}
Diz-se de um cristal, cujas facêtas são rombas.
\section{Rombifoliado}
\begin{itemize}
\item {Grp. gram.:adj.}
\end{itemize}
O mesmo que\textunderscore rombifólio\textunderscore .
\section{Rombifólio}
\begin{itemize}
\item {Grp. gram.:adj.}
\end{itemize}
\begin{itemize}
\item {Utilização:Bot.}
\end{itemize}
\begin{itemize}
\item {Proveniência:(Do lat. \textunderscore rhombus\textunderscore  + \textunderscore folium\textunderscore )}
\end{itemize}
Que tem fôlhas em fórma de rombo.
\section{Rombiforme}
\begin{itemize}
\item {Grp. gram.:adj.}
\end{itemize}
O mesmo que \textunderscore rômbico\textunderscore .
\section{Rombo}
\begin{itemize}
\item {Grp. gram.:adj.}
\end{itemize}
\begin{itemize}
\item {Utilização:Fig.}
\end{itemize}
\begin{itemize}
\item {Grp. gram.:M.}
\end{itemize}
Que não é aguçado ou agudo.
Estúpido, imbecil.
Arrombamento.
Abertura, buraco.
Desfalque.
Peça, com que se tapa qualquer arrombamento no costado do navio.
(Talvez do lat. \textunderscore rhumbus\textunderscore )
\section{Rombo}
\begin{itemize}
\item {Grp. gram.:m.}
\end{itemize}
\begin{itemize}
\item {Proveniência:(Lat. \textunderscore rhombus\textunderscore )}
\end{itemize}
Quadrilátero ou losango, de lados todos iguaes, sem que os ângulos sejam rectos.
\section{Rombo...}
\begin{itemize}
\item {Grp. gram.:pref.}
\end{itemize}
\begin{itemize}
\item {Proveniência:(Lat. \textunderscore rhombus\textunderscore )}
\end{itemize}
(designativo de \textunderscore losango\textunderscore )
\section{Rombocéfalo}
\begin{itemize}
\item {Grp. gram.:m.}
\end{itemize}
\begin{itemize}
\item {Utilização:Zool.}
\end{itemize}
\begin{itemize}
\item {Proveniência:(Do gr. \textunderscore rhombos\textunderscore  + \textunderscore kephale\textunderscore )}
\end{itemize}
Gênero de miriápodes.
\section{Rombódera}
\begin{itemize}
\item {Grp. gram.:f.}
\end{itemize}
Gênero de insectos coleópteros pentâmeros.
\section{Romboédrico}
Que tem fórma de romboédro.
\section{Romboédro}
\begin{itemize}
\item {Grp. gram.:m.}
\end{itemize}
\begin{itemize}
\item {Proveniência:(Do gr. \textunderscore rhombos\textunderscore  + \textunderscore edra\textunderscore )}
\end{itemize}
Sólido geométrico, cujas faces são rombiformes.
\section{Romboglosso}
\begin{itemize}
\item {Grp. gram.:m.}
\end{itemize}
\begin{itemize}
\item {Proveniência:(Do gr. \textunderscore rhombos\textunderscore  + \textunderscore glossa\textunderscore )}
\end{itemize}
Gênero de batrácios.
\section{Romboidal}
\begin{itemize}
\item {Grp. gram.:adj.}
\end{itemize}
\begin{itemize}
\item {Grp. gram.:M.  e  adj.}
\end{itemize}
Que tem a figura de romboide.
Diz-se de um músculo da região dorsal.
\section{Rombóide}
\begin{itemize}
\item {Grp. gram.:m.}
\end{itemize}
\begin{itemize}
\item {Proveniência:(Lat. \textunderscore rhomboides\textunderscore )}
\end{itemize}
Figura de quatro lados, que não tem rectos os ângulos, mas iguaes os lados opostos, e desiguaes os contíguos; paralelogrammo.
\section{Rombopalpo}
\begin{itemize}
\item {Grp. gram.:m.}
\end{itemize}
\begin{itemize}
\item {Proveniência:(Do lat. \textunderscore rhombus\textunderscore  + \textunderscore palpus\textunderscore )}
\end{itemize}
Gênero de insectos coleópteros.
\section{Rombudo}
\begin{itemize}
\item {Grp. gram.:adj.}
\end{itemize}
\begin{itemize}
\item {Utilização:Fig.}
\end{itemize}
\begin{itemize}
\item {Proveniência:(De \textunderscore rombo\textunderscore )}
\end{itemize}
Muito rombo; mal aparado ou mal aguçado.
Estúpido, rude.
\section{Romeira}
\begin{itemize}
\item {Grp. gram.:f.}
\end{itemize}
Mulhér, que vai a uma romaria ou a lugar santo.
Espécie de cabeção, que usavam os que iam em romaria a Santiago de Compostella.
Espécie de cabeção; mantelete.
(Fem. de \textunderscore romeiro\textunderscore )
\section{Romeira}
\begin{itemize}
\item {Grp. gram.:f.}
\end{itemize}
O mesmo que \textunderscore romanzeira\textunderscore .
\section{Romeiral}
\begin{itemize}
\item {Grp. gram.:m.}
\end{itemize}
O mesmo que \textunderscore romanzeiral\textunderscore .
\section{Romeiro}
\begin{itemize}
\item {Grp. gram.:m.}
\end{itemize}
\begin{itemize}
\item {Utilização:Fig.}
\end{itemize}
\begin{itemize}
\item {Utilização:Zool.}
\end{itemize}
\begin{itemize}
\item {Proveniência:(De \textunderscore Roma\textunderscore , n. p.)}
\end{itemize}
Peregrino, homem que vai em romaria.
Propugnador de ideias novas ou grandes.
Peixe escômbrida, (\textunderscore naucrates ductor\textunderscore ).
\section{Romeíto}
\begin{itemize}
\item {Grp. gram.:m.}
\end{itemize}
\begin{itemize}
\item {Utilização:Miner.}
\end{itemize}
Antimoniato de cal, substância amarela, de fractura granulosa e insensível á acção dos ácidos.
\section{Romeliota}
\begin{itemize}
\item {Grp. gram.:adj.}
\end{itemize}
\begin{itemize}
\item {Grp. gram.:M.}
\end{itemize}
Relativo á Romélia.
Habitante da Romélia.
\section{Romenho}
\begin{itemize}
\item {Grp. gram.:m.}
\end{itemize}
(V.\textunderscore romanho\textunderscore )
\section{Romeno}
\begin{itemize}
\item {Grp. gram.:adj.}
\end{itemize}
\begin{itemize}
\item {Grp. gram.:M.}
\end{itemize}
Relativo aos estados danubianos.
Habitante da Romênia.
Lingua novilatina, falada nos estados danubianos.
\section{Romesentar}
\begin{itemize}
\item {Grp. gram.:v. i.}
\end{itemize}
\begin{itemize}
\item {Utilização:Ant.}
\end{itemize}
O mesmo que \textunderscore enfeitar\textunderscore .
(Corr. de \textunderscore ramosentar\textunderscore , de \textunderscore ramo\textunderscore ?)
\section{Rompante}
\begin{itemize}
\item {Grp. gram.:adj.}
\end{itemize}
\begin{itemize}
\item {Grp. gram.:M.}
\end{itemize}
\begin{itemize}
\item {Utilização:Pop.}
\end{itemize}
Que tem arrogância.
Que se precipita furiosamente.
Orgulhoso.
Fúria; ímpeto.
Altivez.
Primeira aduela de um arco, assentada sôbre o capitel.
(Por \textunderscore rompente\textunderscore , de \textunderscore romper\textunderscore )
\section{Rompão}
\begin{itemize}
\item {Grp. gram.:m.}
\end{itemize}
\begin{itemize}
\item {Proveniência:(De \textunderscore romper\textunderscore )}
\end{itemize}
Protuberância na face inferior da ferradura.
\section{Rompão}
\begin{itemize}
\item {Grp. gram.:m.}
\end{itemize}
\begin{itemize}
\item {Utilização:Prov.}
\end{itemize}
\begin{itemize}
\item {Utilização:trasm.}
\end{itemize}
Ímpeto; o mesmo que \textunderscore rompante\textunderscore .
\section{Rompedeira}
\begin{itemize}
\item {Grp. gram.:f.}
\end{itemize}
\begin{itemize}
\item {Proveniência:(De \textunderscore romper\textunderscore )}
\end{itemize}
Cunha encabada, com que ferreiros cortam o ferro em brasa.
Talhadeira.
Puncção, com que os serralheiros abrem furos estreitos e fundos.
\section{Rompedor}
\begin{itemize}
\item {Grp. gram.:m.  e  adj.}
\end{itemize}
O que rompe.
\section{Rompedura}
\begin{itemize}
\item {Grp. gram.:f.}
\end{itemize}
Acto ou effeito de romper.
Rasgão. Cf. Luc. Cordeiro, \textunderscore Senh. Duq.\textunderscore , 305.
\section{Rompegalas}
\begin{itemize}
\item {Grp. gram.:m.}
\end{itemize}
\begin{itemize}
\item {Utilização:Ant.}
\end{itemize}
\begin{itemize}
\item {Utilização:Pop.}
\end{itemize}
Homem esfarrapado, maltrapilho.
\section{Rompente}
\begin{itemize}
\item {Grp. gram.:adj.}
\end{itemize}
\begin{itemize}
\item {Utilização:Heráld.}
\end{itemize}
\begin{itemize}
\item {Proveniência:(Do lat. \textunderscore rumpens\textunderscore )}
\end{itemize}
Que rompe; que investe ou assalta.
Altivo, arrogante.
Diz-se a do leão e, ás vezes, do leopardo, quando, no campo do escudo, se representa de perfil, a prumo, apoiado nas patas traseiras, e a dextra deanteira erguida para o chefe do escudo, e a esquerda descaída para o contra-chefe. Cf. L. Ribeiro, \textunderscore Trat. de Armaria\textunderscore .
\section{Romper}
\begin{itemize}
\item {Grp. gram.:v. t.}
\end{itemize}
\begin{itemize}
\item {Grp. gram.:V. i.}
\end{itemize}
\begin{itemize}
\item {Grp. gram.:V. p.}
\end{itemize}
\begin{itemize}
\item {Grp. gram.:M.}
\end{itemize}
\begin{itemize}
\item {Proveniência:(Do lat. \textunderscore rumpere\textunderscore )}
\end{itemize}
Partir.
Separar em pedaços, despedaçar.
Separar.
Rasgar.
Estragar e abrir com o uso: \textunderscore romper o fato\textunderscore .
Abrir.
Arrotear: \textunderscore romper a terra\textunderscore .
Penetrar.
Traspassar; ferir.
Percorrer com difficuldade.
Abrir caminho por: \textunderscore romper a multidão\textunderscore .
Infringir.
Destruír; derrotar.
Pôr em debandada.
Dar princípio a: \textunderscore romper a marcha\textunderscore .
Interromper.
Arrojar-se contra alguém.
Abrir caminho com ímpeto.
Passar através.
Principiar.
Apparecer; despontar; nascer: \textunderscore rompia a aurora\textunderscore .
Jorrar; saír com ímpeto: \textunderscore a fonte rompe do solo\textunderscore .
Reagir: \textunderscore romper com alguém\textunderscore .
Exceder-se.
Rasgar-se; abrir-se; despedaçar-se.
O mesmo que \textunderscore rompimento\textunderscore .
\section{Rompe-saias}
\begin{itemize}
\item {Grp. gram.:f.}
\end{itemize}
Planta, da fam. das compostas.
\section{Rompe-terra}
\begin{itemize}
\item {Grp. gram.:adj.}
\end{itemize}
\begin{itemize}
\item {Utilização:Poét.}
\end{itemize}
Que penetra na terra ou que a rasga.
\section{Rompida}
\begin{itemize}
\item {Grp. gram.:f.}
\end{itemize}
\begin{itemize}
\item {Utilização:Prov.}
\end{itemize}
\begin{itemize}
\item {Utilização:Bras. do S}
\end{itemize}
Acto de romper ou desbravar terreno.
Saída do gado; acto de elle começar a correr.
\section{Rompimento}
\begin{itemize}
\item {Grp. gram.:m.}
\end{itemize}
Acto ou effeito de romper.
Rotura; quebra (de relações pessoaes ou internacionaes).
\section{Romular}
\begin{itemize}
\item {Grp. gram.:m.}
\end{itemize}
(V.\textunderscore remolar\textunderscore )
\section{Ronábea}
\begin{itemize}
\item {Grp. gram.:f.}
\end{itemize}
Gênero de plantas rubiáceas.
\section{Ronca}
\begin{itemize}
\item {Grp. gram.:f.}
\end{itemize}
\begin{itemize}
\item {Utilização:T. de Aveiro}
\end{itemize}
\begin{itemize}
\item {Utilização:Fig.}
\end{itemize}
\begin{itemize}
\item {Proveniência:(De \textunderscore roncar\textunderscore )}
\end{itemize}
O mesmo que \textunderscore roncadura\textunderscore .
Espécie de fateixa, para a pesca do peixe grosso.
Cylindro ôco, ou vasilha, cujo fundo é substituído por uma pelle de bexiga, atravessada por cordel encerado que, ao attrito da mão, tem um som rouco e áspero.
O som monótono da gaita de folles, o qual acompanha os sons agudos daquelle outro instrumento.
Máquina de vapor, cujo som forte, em occasiões de cerração, denuncia a vizinhança da barra aos navios que a demandam.
Fanfarronada.
\section{Roncada}
\begin{itemize}
\item {Grp. gram.:f.}
\end{itemize}
\begin{itemize}
\item {Utilização:Prov.}
\end{itemize}
\begin{itemize}
\item {Utilização:beir.}
\end{itemize}
\begin{itemize}
\item {Proveniência:(De \textunderscore roncar\textunderscore )}
\end{itemize}
O mesmo que \textunderscore somneca\textunderscore .
\section{Roncadeira}
\begin{itemize}
\item {Grp. gram.:f.}
\end{itemize}
\begin{itemize}
\item {Utilização:Bras. do N}
\end{itemize}
\begin{itemize}
\item {Proveniência:(De \textunderscore roncar\textunderscore )}
\end{itemize}
Instrumento, formado de um pedaço de coiro, que se ajusta á bôca de uma cabaça, usado por caçadores para imitar a voz da onça.
\section{Roncador}
\begin{itemize}
\item {Grp. gram.:m.  e  adj.}
\end{itemize}
\begin{itemize}
\item {Utilização:Fig.}
\end{itemize}
\begin{itemize}
\item {Grp. gram.:M.}
\end{itemize}
\begin{itemize}
\item {Utilização:Prov.}
\end{itemize}
\begin{itemize}
\item {Utilização:minh.}
\end{itemize}
\begin{itemize}
\item {Proveniência:(De \textunderscore roncar\textunderscore )}
\end{itemize}
O que ronca.
Farronqueiro; vaidoso. Cf. Camillo, \textunderscore Serões\textunderscore , I, 23.
Peixe de Portugal, semelhante á corvina, e que faz um ruído parecido ao grunhir do porco.
Penedo da costa ou da praia.--Nesta última accepção, talvez seja outro voc., por \textunderscore rocador\textunderscore , de \textunderscore roca\textunderscore ^2.
\section{Roncadura}
\begin{itemize}
\item {Grp. gram.:f.}
\end{itemize}
Acto ou effeito de roncar.
Bexiga cheia de vento, que rebenta com estrépito.
\section{Roncal}
\begin{itemize}
\item {Grp. gram.:m.}
\end{itemize}
\begin{itemize}
\item {Utilização:Prov.}
\end{itemize}
\begin{itemize}
\item {Utilização:trasm.}
\end{itemize}
Seara forte.
\section{Roncante}
\begin{itemize}
\item {Grp. gram.:adj.}
\end{itemize}
\begin{itemize}
\item {Proveniência:(De \textunderscore roncar\textunderscore )}
\end{itemize}
Que ronca. Cf. Eça, \textunderscore P. Amaro\textunderscore , 205.
\section{Roncão}
\begin{itemize}
\item {Grp. gram.:adj.}
\end{itemize}
O mesmo que \textunderscore roncante\textunderscore . Cf. Eça, \textunderscore P. Amaro\textunderscore , 395.
\section{Roncar}
\begin{itemize}
\item {Grp. gram.:v. i.}
\end{itemize}
\begin{itemize}
\item {Utilização:Fig.}
\end{itemize}
\begin{itemize}
\item {Grp. gram.:V. t.}
\end{itemize}
\begin{itemize}
\item {Proveniência:(Lat. \textunderscore rhonchare\textunderscore )}
\end{itemize}
Respirar ruidosamente, dormindo.
Resonar.
Emittir sons semelhantes á respiração ruidosa de quem dorme.
Dar roncos.
Fazer grande estrondo.
Soar cavernosamente.
Fazer alarde; bravatear.
Proferir com bazófia.
Dizer em tom de provocação.
\section{Roncaria}
\begin{itemize}
\item {Grp. gram.:f.}
\end{itemize}
\begin{itemize}
\item {Utilização:Fig.}
\end{itemize}
\begin{itemize}
\item {Proveniência:(De \textunderscore roncar\textunderscore )}
\end{itemize}
O mesmo que \textunderscore roncadura\textunderscore .
Fanfarronada.
Zombaria.
\section{Ronçaria}
\begin{itemize}
\item {Grp. gram.:f.}
\end{itemize}
Qualidade do que é ronceiro.
\section{Roncear}
\begin{itemize}
\item {Grp. gram.:v. i.}
\end{itemize}
\begin{itemize}
\item {Proveniência:(Do it. \textunderscore ronzare\textunderscore )}
\end{itemize}
Andar ronceiramente; mover-se de vagar.
\section{Ronceiramente}
\begin{itemize}
\item {Grp. gram.:adv.}
\end{itemize}
De modo ronceiro, lentamente; com preguiça.
\section{Ronceirice}
\begin{itemize}
\item {Grp. gram.:f.}
\end{itemize}
Qualidade ou hábito de ronceiro.
Indolência, preguiça.
Systema de quem é opposto ás ideias do progresso.
\section{Ronceirismo}
\begin{itemize}
\item {Grp. gram.:m.}
\end{itemize}
Qualidade ou hábito de ronceiro.
Indolência, preguiça.
Systema de quem é opposto ás ideias do progresso.
\section{Ronceiro}
\begin{itemize}
\item {Grp. gram.:adj.}
\end{itemize}
\begin{itemize}
\item {Proveniência:(De \textunderscore roncear\textunderscore )}
\end{itemize}
Vagaroso; lento; mandrião; indolente; pachorrento.
\section{Roncha}
\begin{itemize}
\item {Grp. gram.:f.}
\end{itemize}
\begin{itemize}
\item {Utilização:Prov.}
\end{itemize}
\begin{itemize}
\item {Utilização:trasm.}
\end{itemize}
\begin{itemize}
\item {Proveniência:(T. cast.)}
\end{itemize}
Vestígio da mordedura do piolho ou do percevejo.
Refêgo nas pernas ou braços das crianças gordas.
\section{Ronchar}
\begin{itemize}
\item {Grp. gram.:v. t.}
\end{itemize}
\begin{itemize}
\item {Utilização:Prov.}
\end{itemize}
\begin{itemize}
\item {Utilização:minh.}
\end{itemize}
Mastigar com ruído (castanhas cruas, etc.).
\section{Roncice}
\begin{itemize}
\item {Grp. gram.:f.}
\end{itemize}
Habito ou propósito de roncear.
\section{Roncinela}
\begin{itemize}
\item {Grp. gram.:f.}
\end{itemize}
Gênero de plantas rosáceas.
\section{Roncinella}
\begin{itemize}
\item {Grp. gram.:f.}
\end{itemize}
Gênero de plantas rosáceas.
\section{Ronco}
\begin{itemize}
\item {Grp. gram.:m.}
\end{itemize}
\begin{itemize}
\item {Utilização:Fig.}
\end{itemize}
\begin{itemize}
\item {Proveniência:(Lat. \textunderscore rhonchus\textunderscore )}
\end{itemize}
O mesmo que \textunderscore roncadura\textunderscore .
Respiração cava e diffícil, nos apopléticos e agonizantes.
Ronca da gaita de folles.
Acto de regougar.
O grunhir dos porcos.
Ronrom.
Som cavernoso e áspero.
Bravata; fanfarronice.
\section{Ronco}
\begin{itemize}
\item {Grp. gram.:adj.}
\end{itemize}
\begin{itemize}
\item {Utilização:Ant.}
\end{itemize}
(Corr. de \textunderscore rouco\textunderscore )
\section{Roncolho}
\begin{itemize}
\item {fónica:cô}
\end{itemize}
\begin{itemize}
\item {Grp. gram.:adj.}
\end{itemize}
Que tem um só testículo.
Mal castrado; toiruno.
\section{Ronda}
\begin{itemize}
\item {Grp. gram.:f.}
\end{itemize}
\begin{itemize}
\item {Utilização:Prov.}
\end{itemize}
\begin{itemize}
\item {Utilização:minh.}
\end{itemize}
\begin{itemize}
\item {Utilização:Prov.}
\end{itemize}
\begin{itemize}
\item {Utilização:beir.}
\end{itemize}
\begin{itemize}
\item {Utilização:T. de Barroso}
\end{itemize}
\begin{itemize}
\item {Proveniência:(Fr. \textunderscore ronde\textunderscore )}
\end{itemize}
Grupo de soldados ou de outras pessôas, que percorre as ruas, ou visita algum pôsto, velando pela manutenção da ordem.
Diligência para descobrir qualquer coisa.
Exame ou inspecção, á cêrca da bôa ordem de alguma coisa.
Espécie de jôgo de asar.
Dança de roda.
Procissão, que dá volta por determinados sítios.
Bando de transeúntes, ás vezes acompanhados de viola e outros instrumentos, o qual percorre de noite vários pontos de uma povoação, especialmente os lugares onde serôam ou descantam raparigas.
Grupo de mascarados, pelo Carnaval.
\section{Rondador}
\begin{itemize}
\item {Grp. gram.:m.  e  adj.}
\end{itemize}
O que ronda.
\section{Rondante}
\begin{itemize}
\item {Grp. gram.:adj.}
\end{itemize}
\begin{itemize}
\item {Grp. gram.:M.  e  f.}
\end{itemize}
Que ronda.
Pessôa, que ronda. Cf. P. Chagas, \textunderscore Côrte de D. João V\textunderscore , 108.
\section{Rondão}
\begin{itemize}
\item {Grp. gram.:m.}
\end{itemize}
O mesmo que \textunderscore roldão\textunderscore . Cf. Filinto, \textunderscore D. Man.\textunderscore , II, 58.
\section{Rondar}
\begin{itemize}
\item {Grp. gram.:v. i.}
\end{itemize}
\begin{itemize}
\item {Utilização:Náut.}
\end{itemize}
\begin{itemize}
\item {Grp. gram.:V. i.}
\end{itemize}
Fazer ronda a.
Vigiar.
Alar por (um cabo) até ficar tesado.
Andar á volta de, passear em tôrno de: \textunderscore rondar uma aldeia\textunderscore .
Fazer ronda; andar á volta; girar.
\section{Rondear}
\begin{itemize}
\item {Grp. gram.:v. i.}
\end{itemize}
Fazer ronda; rondar. Cf. Filinto, X, 113; XIII, 184.
\section{Rondista}
\begin{itemize}
\item {Grp. gram.:m.  e  f.}
\end{itemize}
\begin{itemize}
\item {Proveniência:(De \textunderscore ronda\textunderscore )}
\end{itemize}
Pessôa que, nos caminhos de ferro, anda rondando e observando com lanterna.
\section{Rondó}
\begin{itemize}
\item {Grp. gram.:m.}
\end{itemize}
\begin{itemize}
\item {Proveniência:(Fr. \textunderscore rondeau\textunderscore )}
\end{itemize}
Pequena} composição poética, em que o primeiro ou os primeiros versos se repetem no meio ou no fim da peça.
Composição poética, ordinariamente posta em música, e cujo primeiro ou primeiros versos se repetem no fim.
Ária, cujo thema principal se repete muitas vezes.
\section{Ronga}
\begin{itemize}
\item {Grp. gram.:m.}
\end{itemize}
Língua dos Rongas.
\section{Rongas}
\begin{itemize}
\item {Grp. gram.:m. pl.}
\end{itemize}
Indígenas de Lourenço Marques.
\section{Ronha}
\begin{itemize}
\item {Grp. gram.:f.}
\end{itemize}
\begin{itemize}
\item {Utilização:Pop.}
\end{itemize}
\begin{itemize}
\item {Proveniência:(Do lat. hyp.\textunderscore ronea\textunderscore , seg. Gröber)}
\end{itemize}
Sarna de ovelhas e cavallos.
Doença das salinas, produzida pelos ventos de Léste ou do Sul, tornando a água gordurenta e incapaz de produzir sal.
Malícia, manha, astúcia.
\section{Ronhento}
\begin{itemize}
\item {Grp. gram.:adj.}
\end{itemize}
O mesmo que \textunderscore ronhoso\textunderscore .
\section{Ronhoso}
\begin{itemize}
\item {Grp. gram.:adj.}
\end{itemize}
Que tem ronha.
\section{Ronins}
\begin{itemize}
\item {Grp. gram.:m. pl.}
\end{itemize}
Classe de vagabundos e larápios, no Japão.
\section{Ronquear}
\begin{itemize}
\item {Grp. gram.:v. t.}
\end{itemize}
Abrir, limpar e preparar em conserva (o atum).
\section{Ronqueira}
\begin{itemize}
\item {Grp. gram.:f.}
\end{itemize}
\begin{itemize}
\item {Utilização:Bras. do N}
\end{itemize}
\begin{itemize}
\item {Proveniência:(De \textunderscore ronco\textunderscore ^1)}
\end{itemize}
Ruído da respiração diffícil de quem está doente das vias respiratórias; pieira.
Doença no pulmão do gado.
Artefacto pyrotéchnico, formado de um cano de ferro, preso a um cepo e cheio de pólvora, o qual produz grande detonação, quando se lhe chega fogo á escorva.
\section{Ronquejante}
\begin{itemize}
\item {Grp. gram.:adj.}
\end{itemize}
Que ronqueja.
\section{Ronquejar}
\begin{itemize}
\item {Grp. gram.:v. t.}
\end{itemize}
\begin{itemize}
\item {Proveniência:(De \textunderscore ronco\textunderscore ^1)}
\end{itemize}
Dar roncos; roncar. Cf. Júl. Lour. Pinto, \textunderscore Senh. Deput.\textunderscore , 119 e 263.
\section{Ronquém}
\begin{itemize}
\item {Grp. gram.:adj.}
\end{itemize}
\begin{itemize}
\item {Utilização:Bras. do N}
\end{itemize}
O mesmo que \textunderscore ronquenho\textunderscore .
Que tem a voz muito grossa.
\section{Ronquenho}
\begin{itemize}
\item {Grp. gram.:adj.}
\end{itemize}
\begin{itemize}
\item {Proveniência:(De \textunderscore ronco\textunderscore ^1)}
\end{itemize}
Que tem ronqueira; que ronca.
\section{Ronquidão}
\begin{itemize}
\item {Grp. gram.:f.}
\end{itemize}
\begin{itemize}
\item {Utilização:Des.}
\end{itemize}
\begin{itemize}
\item {Proveniência:(De \textunderscore ronco\textunderscore ^1)}
\end{itemize}
O mesmo que \textunderscore ronquido\textunderscore .
Rouquidão.
\section{Ronquido}
\begin{itemize}
\item {Grp. gram.:m.}
\end{itemize}
\begin{itemize}
\item {Proveniência:(De \textunderscore ronco\textunderscore ^1)}
\end{itemize}
Ruído, produzido pelo estreitamento da tracheia do cavallo, quando caminha rapidamente.
\section{Ronrom}
\begin{itemize}
\item {Grp. gram.:m.}
\end{itemize}
\begin{itemize}
\item {Proveniência:(T. onom.)}
\end{itemize}
Rumor contínuo, produzido pela tracheia do gato, geralmente quando êste está contente ou descansando.
\section{Ronronar}
\begin{itemize}
\item {Grp. gram.:v. i.}
\end{itemize}
Fazer ronrom.
\section{Rontó}
\begin{itemize}
\item {Grp. gram.:m.}
\end{itemize}
Larva, que ataca as palmeiras, perfurando-lhes longitudinalmente o caule.
(Do conc.\textunderscore ronto\textunderscore )
\section{Rópala}
\begin{itemize}
\item {Grp. gram.:f.}
\end{itemize}
\begin{itemize}
\item {Proveniência:(Do gr. \textunderscore rhopalon\textunderscore )}
\end{itemize}
Gênero de plantas proteáceas da América tropical.
\section{Ropalómera}
\begin{itemize}
\item {Grp. gram.:f.}
\end{itemize}
\begin{itemize}
\item {Proveniência:(Do gr. \textunderscore rhopalon\textunderscore  + \textunderscore meros\textunderscore )}
\end{itemize}
Gênero de insectos dípteros.
\section{Rópia}
\begin{itemize}
\item {Grp. gram.:f.}
\end{itemize}
\begin{itemize}
\item {Utilização:Prov.}
\end{itemize}
\begin{itemize}
\item {Utilização:minh.}
\end{itemize}
Rompante; arreganho:« \textunderscore entravam pelas feiras, num arranque de rópia e pimponice...\textunderscore »Camillo, \textunderscore Brasileira\textunderscore , 92.
\section{Roque}
\begin{itemize}
\item {Grp. gram.:m.}
\end{itemize}
Cada uma das peças de xadrez, chamadas tôrres.
O mesmo que\textunderscore recambó\textunderscore .
(Do ar.\textunderscore rok\textunderscore )
\section{Roque}
\begin{itemize}
\item {Grp. gram.:m.}
\end{itemize}
Ave, o mesmo que \textunderscore roquinho\textunderscore .
\section{Roque-de-castro}
\begin{itemize}
\item {Grp. gram.:m.}
\end{itemize}
Ave, o mesmo que \textunderscore roquinho\textunderscore .
\section{Roqueira}
\begin{itemize}
\item {Grp. gram.:f.}
\end{itemize}
\begin{itemize}
\item {Utilização:Açor}
\end{itemize}
\begin{itemize}
\item {Utilização:Bras. da Baía}
\end{itemize}
\begin{itemize}
\item {Utilização:Ant.}
\end{itemize}
\begin{itemize}
\item {Proveniência:(De \textunderscore roca\textunderscore ^2)}
\end{itemize}
Antigo canhão de ferro, cujos projécteis eram pedras.
O mesmo que \textunderscore foguete\textunderscore .
Espingarda ou espécie de canhão, com que se dão tiros, nas festas do San-João.
O mesmo que \textunderscore bacamarte\textunderscore .
\section{Roqueirada}
\begin{itemize}
\item {Grp. gram.:f.}
\end{itemize}
Tiro de roqueira.
\section{Roqueiro}
\begin{itemize}
\item {Grp. gram.:adj.}
\end{itemize}
\begin{itemize}
\item {Grp. gram.:M.}
\end{itemize}
\begin{itemize}
\item {Utilização:Prov.}
\end{itemize}
\begin{itemize}
\item {Utilização:trasm.}
\end{itemize}
Relativo a roca^1.
Aquelle que faz rocas.
O mesmo que \textunderscore rocão\textunderscore ^1.
\section{Roqueiro}
\begin{itemize}
\item {Grp. gram.:adj.}
\end{itemize}
Relativo a roca^2.
Fundado sôbre rochas: \textunderscore castello roqueiro\textunderscore .
Que tem a constituição phýsica das rochas.
Dizia-se do canhão que atirava pedras.
\section{Ròquelinho}
\begin{itemize}
\item {Grp. gram.:m.}
\end{itemize}
\begin{itemize}
\item {Utilização:Prov.}
\end{itemize}
\begin{itemize}
\item {Utilização:trasm.}
\end{itemize}
Espécie de cogumelo comestível.
\section{Roqueló}
\begin{itemize}
\item {Grp. gram.:m.}
\end{itemize}
\begin{itemize}
\item {Utilização:Ant.}
\end{itemize}
O mesmo ou melhor que \textunderscore rocló\textunderscore . Cf. \textunderscore Diccion. Exeg.\textunderscore 
\section{Ròqueroque}
\begin{itemize}
\item {fónica:querró}
\end{itemize}
\begin{itemize}
\item {Grp. gram.:m.}
\end{itemize}
\begin{itemize}
\item {Utilização:Bras}
\end{itemize}
\begin{itemize}
\item {Proveniência:(T. onom.)}
\end{itemize}
Acto de roer ou trincar.
\section{Ròquerroque}
\begin{itemize}
\item {Grp. gram.:m.}
\end{itemize}
\begin{itemize}
\item {Utilização:Bras}
\end{itemize}
\begin{itemize}
\item {Proveniência:(T. onom.)}
\end{itemize}
Acto de roer ou trincar.
\section{Roqueta}
\begin{itemize}
\item {fónica:quê}
\end{itemize}
\begin{itemize}
\item {Grp. gram.:f.}
\end{itemize}
Construcção ou edículo, na parte interior das antigas praças de guerra:«\textunderscore o infante estava preso na roqueta da tôrre de Milão...\textunderscore »Camillo, \textunderscore Noites de Insómn.\textunderscore , VII, 56.
(Cast. \textunderscore roqueta\textunderscore )
\section{Roquete}
\begin{itemize}
\item {fónica:quê}
\end{itemize}
\begin{itemize}
\item {Grp. gram.:m.}
\end{itemize}
Sobrepeliz estreita, com mangas, rendas e pregas miúdas.
Triângulo do escudo heráldico.
(Dem. do b. lat. \textunderscore rocus\textunderscore )
\section{Roquete}
\begin{itemize}
\item {fónica:quê}
\end{itemize}
\begin{itemize}
\item {Grp. gram.:m.}
\end{itemize}
\begin{itemize}
\item {Proveniência:(De \textunderscore roca\textunderscore ^1)}
\end{itemize}
Apparelho, que dá movimento de rotação a uma broca.
Arco de pua. Cf.\textunderscore Inquér. Industr.\textunderscore , p. II, v. I, 216.
\section{Roquinha}
\begin{itemize}
\item {Grp. gram.:f.}
\end{itemize}
Pequena roca com cascavéis, joguete de crianças?:«\textunderscore ...só lhe lembrão, quaes roquinhas da infância, esses pesares...\textunderscore »Filinto, VII, 193.
\section{Roquinho}
\begin{itemize}
\item {Grp. gram.:m.}
\end{itemize}
\begin{itemize}
\item {Utilização:Mad}
\end{itemize}
Espécie de ave, (\textunderscore oceanodroma castro\textunderscore , Harc.).
\section{Rôr}
\begin{itemize}
\item {Grp. gram.:m.}
\end{itemize}
\begin{itemize}
\item {Utilização:Pop.}
\end{itemize}
Grande porção; multidão.
(Aphér. de \textunderscore horror\textunderscore , se não alter. de \textunderscore rol\textunderscore )
\section{Rorante}
\begin{itemize}
\item {Grp. gram.:adj.}
\end{itemize}
\begin{itemize}
\item {Utilização:Poét.}
\end{itemize}
\begin{itemize}
\item {Proveniência:(Lat. \textunderscore rorans\textunderscore )}
\end{itemize}
Que orvalha.
Que tem orvalho.
\section{Rorários}
\begin{itemize}
\item {Grp. gram.:m. pl.}
\end{itemize}
\begin{itemize}
\item {Proveniência:(Lat. \textunderscore rorarii\textunderscore )}
\end{itemize}
Antigas tropas romanas que, armadas ligeiramente, iniciavam os combates, retirando-se logo.
\section{Rorejante}
\begin{itemize}
\item {Grp. gram.:adj.}
\end{itemize}
\begin{itemize}
\item {Utilização:Poét.}
\end{itemize}
Que roreja.
\section{Rorejar}
\begin{itemize}
\item {Grp. gram.:v. t.}
\end{itemize}
\begin{itemize}
\item {Grp. gram.:V. i.}
\end{itemize}
\begin{itemize}
\item {Proveniência:(Do lat. \textunderscore ros\textunderscore , \textunderscore roris\textunderscore )}
\end{itemize}
Destillar (orvalho).
Espalhar gota a gota.
Borrifar.
Transpirar.
Brotar em gotas, borbulhar.
\section{Rorela}
\begin{itemize}
\item {Grp. gram.:f.}
\end{itemize}
\begin{itemize}
\item {Proveniência:(Lat. \textunderscore rorella\textunderscore )}
\end{itemize}
Planta, o mesmo que \textunderscore orvalhinha\textunderscore .
\section{Rorella}
\begin{itemize}
\item {Grp. gram.:f.}
\end{itemize}
\begin{itemize}
\item {Proveniência:(Lat. \textunderscore rorella\textunderscore )}
\end{itemize}
Planta, o mesmo que \textunderscore orvalhinha\textunderscore .
\section{Rórido}
\begin{itemize}
\item {Grp. gram.:adj.}
\end{itemize}
\begin{itemize}
\item {Utilização:Poét.}
\end{itemize}
\begin{itemize}
\item {Proveniência:(Lat. \textunderscore roridus\textunderscore )}
\end{itemize}
O mesmo que \textunderscore orvalhada\textunderscore .
\section{Rorífero}
\begin{itemize}
\item {Grp. gram.:adj.}
\end{itemize}
\begin{itemize}
\item {Utilização:Poét.}
\end{itemize}
\begin{itemize}
\item {Proveniência:(Lat. \textunderscore rorifer\textunderscore )}
\end{itemize}
Que tem orvalho.
\section{Rorífluo}
\begin{itemize}
\item {Grp. gram.:adj.}
\end{itemize}
\begin{itemize}
\item {Utilização:Poét.}
\end{itemize}
\begin{itemize}
\item {Proveniência:(Lat. \textunderscore rorifluus\textunderscore )}
\end{itemize}
Donde corre orvalho; rorífero.
\section{Ró-ró}
\begin{itemize}
\item {Utilização:Pop.}
\end{itemize}
T., us. no jôgo do pião, talvez proveniente do som que elle produz, esgarabulhando no chão:«\textunderscore á porta do circo, ró-ró, laranjinha, lá vai rebolindo a minha piasquinha\textunderscore ». (Canção popular)
\section{Rorqual}
\begin{itemize}
\item {Grp. gram.:m.}
\end{itemize}
Mammífero marítimo, espécie de baleia, que attinge 30 metros do comprimento, e é vulgar nos mares do Norte.
\section{Rosa}
\begin{itemize}
\item {Grp. gram.:f.}
\end{itemize}
\begin{itemize}
\item {Utilização:Mús.}
\end{itemize}
\begin{itemize}
\item {Utilização:Restrict.}
\end{itemize}
\begin{itemize}
\item {Utilização:Fig.}
\end{itemize}
\begin{itemize}
\item {Grp. gram.:Pl.}
\end{itemize}
\begin{itemize}
\item {Grp. gram.:Adj.}
\end{itemize}
\begin{itemize}
\item {Proveniência:(Lat. \textunderscore rosa\textunderscore )}
\end{itemize}
Flôr, em geral odorífera, mais ou menos vermelha, amarela ou pállida e produzida por um arbusto espinhoso.
Bôca circular e ornamentada, no tampo dos instrumentos de cordas, de dedilhar.
Flôr da roseira, de côr vermelha, tirante a claro.
Côr dessa flôr.
Parte rosada das faces.
Aquillo que tem analogia com a disposição das fôlhas da rosa.
Pessôa que, pela sua idade e belleza, lembra o viço e a côr das rosas.
Mulhér formosa.
Peça de latão, com que os encadernadores doiram os livros.
Vidraça circular, com vidros de côres variadas, nas paredes de igrejas antigas; rosaça.
\textunderscore Rosa dos ventos\textunderscore , mostrador que tem desenhados ou gravados os raios que cortam a circunferência do horizonte e correspondem á direcção dos differentes ventos.
Variedade de pêssego, de côr de rosa e polpa aromática.
Ventura; alegria: \textunderscore navegar em mar de rosas\textunderscore ; \textunderscore nem tudo na vida são rosas\textunderscore .
Que tem a côr da rosa: \textunderscore pau rosa\textunderscore ; \textunderscore diamante rosa\textunderscore .
\section{Rosa}
\begin{itemize}
\item {Grp. gram.:adj.}
\end{itemize}
\begin{itemize}
\item {Utilização:Prov.}
\end{itemize}
\begin{itemize}
\item {Utilização:trasm.}
\end{itemize}
Diz-se da enxada de gume direito, por opposição a enxada de ganchos, cujo gume, cavado no meio, termina em duas pontas.
\section{Rosa-albardeira}
\begin{itemize}
\item {Grp. gram.:f.}
\end{itemize}
Designação vulgar da peónia.
\section{Rosa-almiscarada}
\begin{itemize}
\item {Grp. gram.:f.}
\end{itemize}
Planta malvácea, (\textunderscore hibiscus abelmoschus\textunderscore , Lin.), usada em perfumaria, e cuja semente tem o cheiro do almíscar.
\section{Rosaça}
\begin{itemize}
\item {Grp. gram.:f.}
\end{itemize}
\begin{itemize}
\item {Proveniência:(Fr. \textunderscore rosace\textunderscore )}
\end{itemize}
Vidraça de côres, geralmente circular, muitas vezes rendilhada e magnífica, em igrejas.
Ornato architectónico, com aspecto de rosa, em abóbada ou superfície estucada.
Figura symétrica, terminada em circunferência, e apresentando mais ou menos analogia com uma rosa. Cf. Júl. Castilho, \textunderscore Lisb. Ant.\textunderscore 
\section{Rosácea}
\begin{itemize}
\item {Grp. gram.:f.}
\end{itemize}
O mesmo que \textunderscore rosaça\textunderscore .
\section{Rosáceas}
\begin{itemize}
\item {Grp. gram.:f. pl.}
\end{itemize}
Família de plantas, que tem por typo a rosa.
(Fem. pl. de \textunderscore rosáceo\textunderscore )
\section{Rosáceo}
\begin{itemize}
\item {Grp. gram.:adj.}
\end{itemize}
\begin{itemize}
\item {Proveniência:(Lat. \textunderscore rosaceus\textunderscore )}
\end{itemize}
Relativo ou semelhante á rosa.
\section{Rosa-chá}
\begin{itemize}
\item {Grp. gram.:f.}
\end{itemize}
Variedade de rosas pállidas, caracterizada por um aroma semelhante ao do chá.
\section{Rosácico}
\begin{itemize}
\item {Grp. gram.:adj.}
\end{itemize}
\begin{itemize}
\item {Utilização:Med.}
\end{itemize}
\begin{itemize}
\item {Proveniência:(De \textunderscore rosa\textunderscore )}
\end{itemize}
Diz-se de uma substância ácida, rosada ou roxa, depositada pela urina, depois dos accessos de febre intermittente.
\section{Rosa-cruz}
\begin{itemize}
\item {Grp. gram.:f.}
\end{itemize}
\begin{itemize}
\item {Grp. gram.:M.}
\end{itemize}
\begin{itemize}
\item {Proveniência:(De \textunderscore rosa\textunderscore  + \textunderscore cruz\textunderscore )}
\end{itemize}
Sétimo grau ou quarta ordem do rito maçónico francês.
Mação, que tem aquelle grau.
\section{Rosada}
\begin{itemize}
\item {Grp. gram.:f.}
\end{itemize}
\begin{itemize}
\item {Utilização:Prov.}
\end{itemize}
\begin{itemize}
\item {Utilização:alent.}
\end{itemize}
Variedade de roman.
\section{Rosa-da-china}
\begin{itemize}
\item {Grp. gram.:f.}
\end{itemize}
Planta malvácea, (\textunderscore rosa sinensis\textunderscore , Lin.), cujas flôres se usam para dar côr ao calçado.
Chama-se também \textunderscore flôr-de-sapato\textunderscore .
O mesmo que \textunderscore rosa-de-san-francisco\textunderscore .
\section{Rosa-damascena}
\begin{itemize}
\item {Grp. gram.:f.}
\end{itemize}
Designação scientífica das rosas pállidas.
\section{Rosa-de-cão}
\begin{itemize}
\item {Grp. gram.:f.}
\end{itemize}
O mesmo que \textunderscore roseira-canina\textunderscore .
\section{Rosa-de-gueldres}
\begin{itemize}
\item {Grp. gram.:f.}
\end{itemize}
O mesmo que [[novelos|novello]].
\section{Rosa-de-guelres}
\begin{itemize}
\item {Grp. gram.:f.}
\end{itemize}
O mesmo que [[novelos|novello]].
\section{Rosa-de-lobo}
\begin{itemize}
\item {Grp. gram.:f.}
\end{itemize}
O mesmo que \textunderscore peónia\textunderscore .
\section{Rosa-de-musgo}
\begin{itemize}
\item {Grp. gram.:f.}
\end{itemize}
Planta rosácea, (\textunderscore rosa centofolia\textunderscore , Var.; \textunderscore muscosa\textunderscore , Ser.)
\section{Rosa-de-oiro}
\begin{itemize}
\item {Grp. gram.:f.}
\end{itemize}
Planta, da fam. das compostas, também conhecida por \textunderscore cravo-de-defuntos\textunderscore , e \textunderscore cravo-fétido-da-Índia\textunderscore , (\textunderscore tagetes erecta\textunderscore , Lin.).
\section{Rosa-de-san-francisco}
\begin{itemize}
\item {Grp. gram.:f.}
\end{itemize}
Planta malvácea, (\textunderscore hisbiscus mutabilis\textunderscore , Lin.), também conhecida por \textunderscore aurora\textunderscore .
\section{Rosa-de-toucar}
\begin{itemize}
\item {Grp. gram.:f.}
\end{itemize}
Planta rosácea (\textunderscore rosa centofolia\textunderscore , Lindl.).
\section{Rosado}
\begin{itemize}
\item {Grp. gram.:adj.}
\end{itemize}
\begin{itemize}
\item {Proveniência:(Do lat. \textunderscore rosatus\textunderscore )}
\end{itemize}
Que tem a côr da rosa: \textunderscore faces rosadas\textunderscore .
Avermelhado.
Em que entra a essência das rosas: \textunderscore mel rosado\textunderscore .
\section{Rosa-do-bem-fazer}
\begin{itemize}
\item {Grp. gram.:f.}
\end{itemize}
\begin{itemize}
\item {Utilização:Açor}
\end{itemize}
Designação vulgar da flôr do sabugueiro. (Colhido em San Miguel)
\section{Rosa-do-japão}
\begin{itemize}
\item {Grp. gram.:f.}
\end{itemize}
Designação vulgar da camélia, flôr.
\section{Rosa-do-ultramar}
\begin{itemize}
\item {Grp. gram.:f.}
\end{itemize}
Planta malvácea, (\textunderscore althea rosea\textunderscore , Roxb.), que é a malva dos jardins.
\section{Rosagrana}
\begin{itemize}
\item {Grp. gram.:f.}
\end{itemize}
\begin{itemize}
\item {Utilização:Des.}
\end{itemize}
\begin{itemize}
\item {Proveniência:(De \textunderscore rosa\textunderscore  + \textunderscore gran\textunderscore ^2)}
\end{itemize}
Tecido escarlate, tirante á côr de rosa.
\section{Rosairo}
\begin{itemize}
\item {Grp. gram.:m.}
\end{itemize}
(Fórma pop. de \textunderscore rosário\textunderscore )
\section{Rosal}
\begin{itemize}
\item {Grp. gram.:m.}
\end{itemize}
O mesmo que \textunderscore roseiral\textunderscore .
\section{Rosalgar}
\begin{itemize}
\item {Grp. gram.:m.}
\end{itemize}
\begin{itemize}
\item {Proveniência:(Do ár. \textunderscore rehj-algar\textunderscore )}
\end{itemize}
Designação vulgar do óxydo de arsénio.
\section{Rosalgarino}
\begin{itemize}
\item {Grp. gram.:adj.}
\end{itemize}
Relativo ao rosalgar.
\section{Rosália}
\begin{itemize}
\item {Grp. gram.:f.}
\end{itemize}
Gênero de insectos coleópteros longicórneos.
\section{Rosálias}
\begin{itemize}
\item {Grp. gram.:f. pl.}
\end{itemize}
\begin{itemize}
\item {Proveniência:(Lat. \textunderscore rosália\textunderscore )}
\end{itemize}
Solemnidades antigas, em que se levavam rosas aos sepulcros, em homenagem aos mortos.
\section{Rosalina}
\begin{itemize}
\item {Grp. gram.:f.}
\end{itemize}
\begin{itemize}
\item {Utilização:Zool.}
\end{itemize}
Gênero de foraminíferos, que comprehende certas conchas microscópicas.
\section{Rosanilina}
\begin{itemize}
\item {Grp. gram.:f.}
\end{itemize}
\begin{itemize}
\item {Proveniência:(De \textunderscore rosa\textunderscore  + \textunderscore anil\textunderscore )}
\end{itemize}
Um dos productos da hulha.
\section{Rosão}
\begin{itemize}
\item {Grp. gram.:m.}
\end{itemize}
\begin{itemize}
\item {Proveniência:(De \textunderscore rosa\textunderscore )}
\end{itemize}
Ornato, que representa um florão na divisão das abóbadas e na juncção das nervuras.
\section{Rosário}
\begin{itemize}
\item {Grp. gram.:m.}
\end{itemize}
\begin{itemize}
\item {Proveniência:(Lat. \textunderscore rosarius\textunderscore )}
\end{itemize}
Conjunto de contas que corresponde a quinze dezenas de ave-marias e a quinze padre-nossos, que se rezam, de ordinário, em honra da Virgem ou como simples prática religiosa.
Terço ou terça parte daquellas quinze dezenas de contas, servindo para rezar.
Conjunto de sete dezenas das mesmas contas, chamado também corôa.
Enfiada; porção.
Apparelho, para extrahir água das minas.
\section{Rosário-de-jambu}
\begin{itemize}
\item {Grp. gram.:m.}
\end{itemize}
Planta myrtácea, (\textunderscore eugenia racemosa\textunderscore ).
\section{Rosaristas}
\begin{itemize}
\item {Grp. gram.:pl.}
\end{itemize}
Secção, de congregadas dominicanas, sob a invocação da Senhora do Rosário.--Tinham convento em Aveiro.
\section{Rosar-se}
\begin{itemize}
\item {Grp. gram.:v. p.}
\end{itemize}
Tomar a côr de rosa; ruborizar-se.
Envergonhar-se. Cf. Rebello, \textunderscore Mocidade\textunderscore , II, 115.
\section{Rosasólis}
\begin{itemize}
\item {fónica:só}
\end{itemize}
\begin{itemize}
\item {Grp. gram.:m.}
\end{itemize}
\begin{itemize}
\item {Proveniência:(Do lat. \textunderscore rosa\textunderscore  + \textunderscore sol\textunderscore )}
\end{itemize}
Espécie de licor, composto de aguardente rectificada, açúcar, canela e outros ingredientes aromáticos.
\section{Rosassólis}
\begin{itemize}
\item {Grp. gram.:m.}
\end{itemize}
\begin{itemize}
\item {Proveniência:(Do lat. \textunderscore rosa\textunderscore  + \textunderscore sol\textunderscore )}
\end{itemize}
Espécie de licor, composto de aguardente rectificada, açúcar, canela e outros ingredientes aromáticos.
\section{Rosato}
\begin{itemize}
\item {Grp. gram.:m.}
\end{itemize}
\begin{itemize}
\item {Utilização:Chím.}
\end{itemize}
Sal, resultante da combinação do ácido rosácico com uma base.
(Cp.\textunderscore rosácico\textunderscore )
\section{Ròsbife}
\begin{itemize}
\item {Grp. gram.:m.}
\end{itemize}
Peça de carne de vaca, pouco assada, ou assada de fórma que mantenha interiormente a côr avermelhada.
(Do ing.\textunderscore roastbeef\textunderscore )
\section{Rôsca}
\begin{itemize}
\item {Grp. gram.:f.}
\end{itemize}
\begin{itemize}
\item {Utilização:T. da Bairrada}
\end{itemize}
\begin{itemize}
\item {Utilização:Prov.}
\end{itemize}
\begin{itemize}
\item {Utilização:minh.}
\end{itemize}
\begin{itemize}
\item {Grp. gram.:M.  e  f.}
\end{itemize}
Volta em espiral, num objecto qualquer.
Tortuosidade.
Cada uma das voltas da serpente que se enrola.
Bolo ou pão torcido ou em fórma de argola.
Doença, que ataca os cães.
O mesmo que \textunderscore bebedeira\textunderscore .
Pessôa manhosa.
Espécie de jôgo popular.
Verme da terra, que ataca as raízes de certas plantas. Cf.\textunderscore Bibl. da G. do Campo\textunderscore , 304.
\section{Roscar}
\begin{itemize}
\item {Grp. gram.:v. t.}
\end{itemize}
Fazer rôscas em.
Aparafusar.
Cf.\textunderscore Inquér. Industr\textunderscore ., II p. 1. III, 11.
\section{Róscea}
\begin{itemize}
\item {Grp. gram.:f.}
\end{itemize}
Gênero de plantas gengiberáceas.
\section{Róscido}
\begin{itemize}
\item {Grp. gram.:adj.}
\end{itemize}
\begin{itemize}
\item {Utilização:Poét.}
\end{itemize}
\begin{itemize}
\item {Proveniência:(Lat. \textunderscore roscidus\textunderscore )}
\end{itemize}
O mesmo que [[orvalhado|orvalhar]].
\section{Roseira}
\begin{itemize}
\item {Grp. gram.:f.}
\end{itemize}
\begin{itemize}
\item {Proveniência:(Do lat. \textunderscore rosária\textunderscore )}
\end{itemize}
Arbusto rosáceo, que produz as rosas e é geralmente espinhoso.
\section{Roseira-canina}
\begin{itemize}
\item {Grp. gram.:f.}
\end{itemize}
Espécie de roseira, que cresce sem cultura nos matos e vallados, (\textunderscore rosa canina\textunderscore , Lin.).
\section{Roseira-do-japão}
\begin{itemize}
\item {Grp. gram.:f.}
\end{itemize}
Designação vulgar da camélia, planta.
\section{Roseira-francesa}
\begin{itemize}
\item {Grp. gram.:f.}
\end{itemize}
O mesmo que \textunderscore roseira-rubra\textunderscore .
\section{Roseiral}
\begin{itemize}
\item {Grp. gram.:m.}
\end{itemize}
Terreno, onde crescem roseiras.
\section{Roseira-rubra}
\begin{itemize}
\item {Grp. gram.:f.}
\end{itemize}
Espécie de roseira, cujos botões em infusão têm várias applicações medicinaes.
\section{Roseirista}
\begin{itemize}
\item {Grp. gram.:m.  e  f.}
\end{itemize}
Pessôa, que cultiva roseiras.
\section{Roseiro}
\begin{itemize}
\item {Grp. gram.:adj.}
\end{itemize}
Diz-se do toiro, malhado de branco, originário dos Açores.
\section{Rosela}
\begin{itemize}
\item {Grp. gram.:f.}
\end{itemize}
Planta droserácea, o mesmo que \textunderscore orvalhinha\textunderscore .
\section{Roselha}
\begin{itemize}
\item {fónica:zê}
\end{itemize}
\begin{itemize}
\item {Grp. gram.:f.}
\end{itemize}
Planta cistínea, (\textunderscore cistus albidus\textunderscore ).
\section{Roselito}
\begin{itemize}
\item {Grp. gram.:m.}
\end{itemize}
\begin{itemize}
\item {Utilização:Miner.}
\end{itemize}
Arseniato de cobalto e calcário.
\section{Rosella}
\begin{itemize}
\item {Grp. gram.:f.}
\end{itemize}
Planta droserácea, o mesmo que \textunderscore orvalhinha\textunderscore .
\section{Rosênia}
\begin{itemize}
\item {Grp. gram.:f.}
\end{itemize}
Gênero de plantas, da fam. das compostas.
\section{Rosental}
\begin{itemize}
\item {Grp. gram.:m.}
\end{itemize}
\begin{itemize}
\item {Utilização:Prov.}
\end{itemize}
O mesmo que \textunderscore recental\textunderscore . Cf. A. Serpa, \textunderscore Solaus\textunderscore , 3.
\section{Róseo}
\begin{itemize}
\item {Grp. gram.:adj.}
\end{itemize}
\begin{itemize}
\item {Proveniência:(Lat. \textunderscore roseus\textunderscore )}
\end{itemize}
Relativo a rosa.
Perfumado como a rosa.
Rosado.
Próprio da rosa.
\section{Roséola}
\begin{itemize}
\item {Grp. gram.:f.}
\end{itemize}
\begin{itemize}
\item {Utilização:Med.}
\end{itemize}
\begin{itemize}
\item {Proveniência:(De \textunderscore rosa\textunderscore )}
\end{itemize}
Moléstia cutânea, manifestada por manchas rosadas, especialmente nos braços e pernas, e determinada geralmente pela acção do sol ou pelo calor do estio.
\section{Roseta}
\begin{itemize}
\item {fónica:zê}
\end{itemize}
\begin{itemize}
\item {Grp. gram.:f.}
\end{itemize}
\begin{itemize}
\item {Utilização:Bras. do S}
\end{itemize}
\begin{itemize}
\item {Proveniência:(De \textunderscore rosa\textunderscore )}
\end{itemize}
Pequena rosa.
Rodízio.
A roda dentada da espora.
Pequena roda dentada nos compassos.
Laço ou nó de fita, que se usa na botoeira superior do casaco ou casaca, como distinctivo honorifico.
Sinal ou mancha vermelha no corpo.
Coloração especial das faces, por effeito da febre ou em certos graus da tísica.
Rodella de croché.
Pontas de capim sêco, depois de muito catado pelos animaes.
\section{Rosetão}
\begin{itemize}
\item {Grp. gram.:m.}
\end{itemize}
\begin{itemize}
\item {Proveniência:(De \textunderscore roseta\textunderscore )}
\end{itemize}
Ornato de esculptura, representando uma grande rosa. Cp.\textunderscore rosão\textunderscore .
\section{Rosete}
\begin{itemize}
\item {fónica:zê}
\end{itemize}
\begin{itemize}
\item {Grp. gram.:adj.}
\end{itemize}
\begin{itemize}
\item {Proveniência:(De \textunderscore rosa\textunderscore )}
\end{itemize}
Que tem a côr um tanto rosada.
\section{Roseteiro}
\begin{itemize}
\item {Grp. gram.:m.}
\end{itemize}
\begin{itemize}
\item {Utilização:Bras. do S}
\end{itemize}
Proprietário de chácara, cujos pastos estão reduzidos ás pontas sêcas de capim, chamadas \textunderscore roseta\textunderscore .
\section{Rosiclér}
\begin{itemize}
\item {Grp. gram.:adj.}
\end{itemize}
\begin{itemize}
\item {Grp. gram.:M.}
\end{itemize}
\begin{itemize}
\item {Proveniência:(Do fr. \textunderscore rose\textunderscore  + \textunderscore clair\textunderscore )}
\end{itemize}
Que tem a côr da rosa e da açucena.
Côr afogueada, como a da rosa.
Collar (de pérolas).
Mina de prata vermelha.
\section{Rosicré}
\begin{itemize}
\item {Grp. gram.:m.}
\end{itemize}
O mesmo que \textunderscore rosiclér\textunderscore . Cf. Sousa, \textunderscore Vida do Arceb.\textunderscore , III, 315.
\section{Rosiflor}
\begin{itemize}
\item {Grp. gram.:adj.}
\end{itemize}
\begin{itemize}
\item {Proveniência:(De \textunderscore rosa\textunderscore  + \textunderscore flôr\textunderscore )}
\end{itemize}
Que tem flôres parecidas á rosa, (falando-se do loendro):«\textunderscore ...o rosiflor loureiro...\textunderscore »Filinto, XIV, 9. Cp. fr. \textunderscore laurier-rose\textunderscore .
\section{Rosifloras}
\begin{itemize}
\item {Grp. gram.:f. pl.}
\end{itemize}
\begin{itemize}
\item {Proveniência:(De \textunderscore rosa\textunderscore  + \textunderscore flôr\textunderscore )}
\end{itemize}
Ordem de plantas, que contém as rosáceas, as pomáceas e outras.
\section{Rosigastro}
\begin{itemize}
\item {Grp. gram.:adj.}
\end{itemize}
\begin{itemize}
\item {Utilização:Zool.}
\end{itemize}
\begin{itemize}
\item {Proveniência:(Do lat. \textunderscore rosa\textunderscore  + gr. \textunderscore gaster\textunderscore )}
\end{itemize}
Que tem ventre da côr da rosa.
\section{Rosilha}
\begin{itemize}
\item {Grp. gram.:f.}
\end{itemize}
Gênero de plantas, da fam. das compostas, originárias do México.
(Cast. \textunderscore rosilla\textunderscore )
\section{Rosilho}
\begin{itemize}
\item {Grp. gram.:adj.}
\end{itemize}
Diz-se do cavallo, cujo pêlo branco e avermelhado produz o aspecto da côr rosada.
(Cast. \textunderscore rosillo\textunderscore )
\section{Rosina}
\begin{itemize}
\item {Grp. gram.:f.}
\end{itemize}
\begin{itemize}
\item {Proveniência:(De \textunderscore rosa\textunderscore )}
\end{itemize}
Antiga moéda de oiro na Toscana.
\section{Rosinha}
\begin{itemize}
\item {Grp. gram.:f.}
\end{itemize}
\begin{itemize}
\item {Utilização:Prov.}
\end{itemize}
\begin{itemize}
\item {Utilização:minh.}
\end{itemize}
Ave, o mesmo que \textunderscore cheide\textunderscore .
\section{Rosita}
\begin{itemize}
\item {Grp. gram.:f.}
\end{itemize}
\begin{itemize}
\item {Utilização:Miner.}
\end{itemize}
Silicato de alumina, com aspecto de côr da rosa.
\section{Rosito}
\begin{itemize}
\item {Grp. gram.:m.}
\end{itemize}
O mesmo ou melhor que \textunderscore rosita\textunderscore .
\section{Rosmaninhal}
\begin{itemize}
\item {Grp. gram.:m.}
\end{itemize}
Terreno, onde cresce rosmaninho.
\section{Rosmaninho}
\begin{itemize}
\item {Grp. gram.:m.}
\end{itemize}
\begin{itemize}
\item {Proveniência:(Do lat. \textunderscore rosmarinus\textunderscore )}
\end{itemize}
Planta labiada, muito aromática.
A flôr dessa planta.
\section{Rosmano}
\begin{itemize}
\item {Grp. gram.:m.}
\end{itemize}
\begin{itemize}
\item {Utilização:Prov.}
\end{itemize}
O mesmo que \textunderscore rosmaninho\textunderscore .
(Supposta fórma primitiva, de que \textunderscore rosmaninho\textunderscore  é dem. hyp.)
\section{Rosmear}
\begin{itemize}
\item {Grp. gram.:v. i.}
\end{itemize}
\begin{itemize}
\item {Utilização:Ant.}
\end{itemize}
O mesmo que \textunderscore resmungar\textunderscore . Cf. G. Vicente.
\section{Rosnadela}
\begin{itemize}
\item {Grp. gram.:f.}
\end{itemize}
Acto ou effeito de rosnar.
\section{Rosnador}
\begin{itemize}
\item {Grp. gram.:m.  e  adj.}
\end{itemize}
O que rosna.
\section{Rosnadura}
\begin{itemize}
\item {Grp. gram.:f.}
\end{itemize}
O mesmo que \textunderscore rosnadela\textunderscore .
\section{Rosnar}
\begin{itemize}
\item {Grp. gram.:v. t.}
\end{itemize}
\begin{itemize}
\item {Grp. gram.:V. i.}
\end{itemize}
\begin{itemize}
\item {Grp. gram.:V. p.}
\end{itemize}
\begin{itemize}
\item {Grp. gram.:M.}
\end{itemize}
Dizer em voz baixa, por entre os dentes; murmurar.
Resmungar.
Constar, correr como boato: \textunderscore rosna-se que o Ministério está em crise\textunderscore .
Acto de rosnar.
Voz surda do cão que, sem latir, indica ameaça e mostra os dentes.
(Alter. de \textunderscore resonar\textunderscore ?)
\section{Rosnento}
\begin{itemize}
\item {Grp. gram.:adj.}
\end{itemize}
\begin{itemize}
\item {Utilização:Bras. do N}
\end{itemize}
Que rosna muito.
\section{Rosolato}
\begin{itemize}
\item {Grp. gram.:m.}
\end{itemize}
Combinação do ácido rosólico com uma base.
\section{Rosólico}
\begin{itemize}
\item {Grp. gram.:adj.}
\end{itemize}
\begin{itemize}
\item {Proveniência:(De \textunderscore rosólio\textunderscore )}
\end{itemize}
Diz-se de um ácido, produzido pela oxydação do phenol.
\section{Rosólio}
\begin{itemize}
\item {Grp. gram.:m.}
\end{itemize}
\begin{itemize}
\item {Proveniência:(Do it. \textunderscore rosoglio\textunderscore ?)}
\end{itemize}
Espécie de ratafia, usada principalmente na Itália e na Turquia.
\section{Rosquear}
\begin{itemize}
\item {Grp. gram.:v. i.}
\end{itemize}
\begin{itemize}
\item {Utilização:Prov.}
\end{itemize}
\begin{itemize}
\item {Utilização:trasm.}
\end{itemize}
\begin{itemize}
\item {Utilização:Prov.}
\end{itemize}
\begin{itemize}
\item {Utilização:minh.}
\end{itemize}
\begin{itemize}
\item {Grp. gram.:V. t.}
\end{itemize}
\begin{itemize}
\item {Utilização:Prov.}
\end{itemize}
\begin{itemize}
\item {Utilização:minh.}
\end{itemize}
\begin{itemize}
\item {Proveniência:(De \textunderscore rôsca\textunderscore )}
\end{itemize}
Cair, formando rôscas ou rolando.
Levar castigo.
Bater, castigar.
\section{Rosquilha}
\begin{itemize}
\item {Grp. gram.:f.}
\end{itemize}
Pequena rôsca de pão.
Biscoito retorcido.
(Cast. \textunderscore rosquilla\textunderscore )
\section{Rosquilho}
\begin{itemize}
\item {Grp. gram.:m.}
\end{itemize}
O mesmo que \textunderscore rosquilha\textunderscore .
\section{Róssia}
\begin{itemize}
\item {Grp. gram.:f.}
\end{itemize}
\begin{itemize}
\item {Proveniência:(De \textunderscore Rossi\textunderscore , n. p.)}
\end{itemize}
Gênero de aves palmípedes.
\section{Rossilhonas}
\begin{itemize}
\item {Grp. gram.:f. pl.}
\end{itemize}
\begin{itemize}
\item {Utilização:Bras. do S}
\end{itemize}
Botas altas de montaria.
\section{Rossiniano}
\begin{itemize}
\item {Grp. gram.:adj.}
\end{itemize}
Relativo a Rossini, ou á sua maneira de compor música.
\section{Rossio}
\begin{itemize}
\item {Grp. gram.:m.}
\end{itemize}
Terreno, que era roçado ou fruído, em commum, pelo povo.
Logradoiro público.
Lugar espaçoso; terreiro; praça larga.
(Port. ant. \textunderscore ressio\textunderscore  V. \textunderscore ressio\textunderscore )
\section{Rostão}
\begin{itemize}
\item {Grp. gram.:m.}
\end{itemize}
\begin{itemize}
\item {Utilização:Prov.}
\end{itemize}
O mesmo que \textunderscore loiceiro\textunderscore . Cf.\textunderscore Techn. Rural\textunderscore , I, 542.
\section{Rostelária}
\begin{itemize}
\item {Grp. gram.:f.}
\end{itemize}
Gênero de plantas acantháceas.
\section{Rostideira}
\begin{itemize}
\item {Grp. gram.:f.}
\end{itemize}
\begin{itemize}
\item {Utilização:Gír.}
\end{itemize}
\begin{itemize}
\item {Proveniência:(De \textunderscore rostir\textunderscore )}
\end{itemize}
Acto de comer.
Aquillo que se come.
\section{Rostir}
\begin{itemize}
\item {Grp. gram.:v. t.}
\end{itemize}
\begin{itemize}
\item {Utilização:Gír.}
\end{itemize}
\begin{itemize}
\item {Utilização:Bras}
\end{itemize}
\begin{itemize}
\item {Proveniência:(De \textunderscore rôsto\textunderscore ? Or. ind.?)}
\end{itemize}
Maltratar.
Mastigar; comer.
Roçar, esfregar.
\section{Rosto}
\begin{itemize}
\item {fónica:rôs}
\end{itemize}
\begin{itemize}
\item {Grp. gram.:m.}
\end{itemize}
\begin{itemize}
\item {Grp. gram.:Loc. adv.}
\end{itemize}
Parte anterior da cabeça.
Cara.
Physionomia.
Presença.
Semblante.
Parte fronteira; frente.
Página do livro, em que há só ou principalmente o título da obra e o nome do autor.
O verso da medalha.
\textunderscore Fazer rosto\textunderscore , estar defronte.
Resistir.
\textunderscore De rosto\textunderscore , de frente:«\textunderscore deu com ella de rosto\textunderscore ». Camillo, \textunderscore Retr. de Ricard.\textunderscore , 117.
(Contr. de \textunderscore rostro\textunderscore )
\section{Rosto}
\begin{itemize}
\item {fónica:rôs}
\end{itemize}
\begin{itemize}
\item {Grp. gram.:m.}
\end{itemize}
\begin{itemize}
\item {Utilização:Gír.}
\end{itemize}
Sangue.
\section{Rostolho}
\begin{itemize}
\item {fónica:tô}
\end{itemize}
\begin{itemize}
\item {Grp. gram.:m.}
\end{itemize}
Peça, no rosto da fechadura.
\section{Rostrado}
\begin{itemize}
\item {Grp. gram.:adj.}
\end{itemize}
\begin{itemize}
\item {Utilização:Zool.}
\end{itemize}
Que tem focinho ou fórma de bico.
Que tem esporão ou é semelhante a êlle.
(Da lat. \textunderscore rostratus\textunderscore )
\section{Rostral}
\begin{itemize}
\item {Grp. gram.:adj.}
\end{itemize}
\begin{itemize}
\item {Utilização:Neol.}
\end{itemize}
\begin{itemize}
\item {Proveniência:(Lat. \textunderscore rostralis\textunderscore )}
\end{itemize}
Diz-se da antenna inserida, no róstro de alguns animaes.
Que representa rostros ou que é ornado de rostros.
Relativo á frente ou ao rosto dos livros ou de outras publicações literárias: \textunderscore na página rostral, bellas gravuras\textunderscore .
\section{Rostrato}
\begin{itemize}
\item {Grp. gram.:adj.}
\end{itemize}
\begin{itemize}
\item {Utilização:Bot.}
\end{itemize}
Diz-se de uma variedade de eucalypto.
\section{Ròstricórneo}
\begin{itemize}
\item {Grp. gram.:adj.}
\end{itemize}
\begin{itemize}
\item {Utilização:Zool.}
\end{itemize}
\begin{itemize}
\item {Proveniência:(De \textunderscore rostro\textunderscore  + \textunderscore córneo\textunderscore )}
\end{itemize}
Que tem a antenna debaixo de uma ponta ou espécie de bico que prolonga a cabeça.
\section{Ròstriforme}
\begin{itemize}
\item {Grp. gram.:adj.}
\end{itemize}
\begin{itemize}
\item {Proveniência:(Do lat. \textunderscore rostrum\textunderscore  + \textunderscore forma\textunderscore )}
\end{itemize}
Que tem fórma de bico.
\section{Rostrilho}
\begin{itemize}
\item {Grp. gram.:m.}
\end{itemize}
\begin{itemize}
\item {Proveniência:(De \textunderscore rostro\textunderscore )}
\end{itemize}
Radícula da semente.
\section{Rostro}
\begin{itemize}
\item {Grp. gram.:m.}
\end{itemize}
\begin{itemize}
\item {Proveniência:(Lat. \textunderscore róstrum\textunderscore )}
\end{itemize}
Bico das aves.
Remate da prôa de um navio.
Esporão dos vegetaes.
Representação da prôa de um navio antigo.
Tribuna, em que discursavam os oradores romanos, e era ornada de prôas de navios.
Sugadoiro dos insectos hemípteros.
Saliência na parte anterior de alguma coisa.
O mesmo que \textunderscore rosto\textunderscore ^1, face:«\textunderscore erguede rostro e molhade\textunderscore ».\textunderscore Cancion. da Vaticana\textunderscore .«\textunderscore Nos rostros, nas vozes, nas letras, apenas há entre milhões de homens hum que se pareça inteiramente com o outro\textunderscore ».\textunderscore Nova Floresta\textunderscore , II, 28. Cf. F. Manuel, \textunderscore Carta de Guia\textunderscore , 81; \textunderscore Jornada de África\textunderscore , X; Pant. de Aveiro, \textunderscore Itiner.\textunderscore , 140, (2.^a ed.).
\section{Rosulho}
\begin{itemize}
\item {Grp. gram.:m.}
\end{itemize}
\begin{itemize}
\item {Utilização:Ant.}
\end{itemize}
O mesmo que \textunderscore resto\textunderscore ^1.
(Talvez por \textunderscore resulho\textunderscore , do lat. \textunderscore residuus\textunderscore )
\section{Róta}
\begin{itemize}
\item {Grp. gram.:f.}
\end{itemize}
\begin{itemize}
\item {Proveniência:(Do lat. \textunderscore rupta\textunderscore )}
\end{itemize}
Peleja.
Derrota de um exército.
Direcção, caminho, rumo.
Tribunal pontifício, que resolve os pleitos sôbre benefícios.
\textunderscore De\textunderscore  ou \textunderscore em rota batida\textunderscore , em fuga; em andamento acelerado.
\section{Róta}
\begin{itemize}
\item {Grp. gram.:f.}
\end{itemize}
\begin{itemize}
\item {Utilização:T. da Índia Port}
\end{itemize}
Junco, (\textunderscore calamus rotang\textunderscore , Lin.), com que se fabricam esteiras, velas de embarcação, e assentos de cadeiras.
Bengala delgada.
(Do conc.\textunderscore rota\textunderscore )
\section{Róta}
\begin{itemize}
\item {Grp. gram.:f.}
\end{itemize}
\begin{itemize}
\item {Utilização:Mús.}
\end{itemize}
\begin{itemize}
\item {Utilização:Ant.}
\end{itemize}
Instrumento de cordas, sôbre cuja origem e descrição há divergências. Cf.\textunderscore Diccion. Mús.\textunderscore 
\section{Rôta}
\begin{itemize}
\item {Grp. gram.:f.}
\end{itemize}
\begin{itemize}
\item {Utilização:Prov.}
\end{itemize}
\begin{itemize}
\item {Utilização:trasm.}
\end{itemize}
Córte de terreno, para abertura de estrada, caminho, etc.
\section{Rotação}
\begin{itemize}
\item {Grp. gram.:f.}
\end{itemize}
\begin{itemize}
\item {Utilização:Mús.}
\end{itemize}
\begin{itemize}
\item {Proveniência:(Do lat. \textunderscore rotatio\textunderscore )}
\end{itemize}
Acto ou effeito de rodar.
Giro.
Piruêta.
Movimento giratório.
Repetição dos mesmos factos no decurso do tempo ou em épocas sucessivas.
Successão alternada de pessôas, actos ou factos: \textunderscore a rotação dos partidos no poder\textunderscore .
Systema de cylindros, nos instrumentos de metal.
\section{Rotáceo}
\begin{itemize}
\item {Grp. gram.:adj.}
\end{itemize}
\begin{itemize}
\item {Proveniência:(Do lat. \textunderscore rota\textunderscore )}
\end{itemize}
Que tem fórma de roda.
\section{Rotador}
\begin{itemize}
\item {Grp. gram.:adj.}
\end{itemize}
\begin{itemize}
\item {Grp. gram.:M.}
\end{itemize}
\begin{itemize}
\item {Proveniência:(Lat. \textunderscore rotator\textunderscore )}
\end{itemize}
Que faz rodar.
Músculo, que faz girar sôbre o seu eixo as partes a que está ligado.
Infusório, que tem o aspecto de duas rodas de engrenagem, girando em sentido inverso.
\section{Rotala}
\begin{itemize}
\item {Grp. gram.:f.}
\end{itemize}
Gênero de plantas indianas.
\section{Rotália}
\begin{itemize}
\item {Grp. gram.:f.}
\end{itemize}
Gênero do molluscos cephalópodes.
\section{Rotamente}
\begin{itemize}
\item {Grp. gram.:adv.}
\end{itemize}
\begin{itemize}
\item {Proveniência:(De \textunderscore rôto\textunderscore )}
\end{itemize}
Ás claras; francamente; sem reserva.
\section{Rotante}
\begin{itemize}
\item {Grp. gram.:adj.}
\end{itemize}
\begin{itemize}
\item {Proveniência:(Lat. \textunderscore rotans\textunderscore )}
\end{itemize}
Que roda.
\section{Rotar}
\begin{itemize}
\item {Grp. gram.:v. i.}
\end{itemize}
\begin{itemize}
\item {Utilização:Prov.}
\end{itemize}
\begin{itemize}
\item {Utilização:trasm.}
\end{itemize}
\begin{itemize}
\item {Proveniência:(Lat. \textunderscore rotare\textunderscore )}
\end{itemize}
Girar; andar á roda.
Voar (a ave ou a borboleta).
\section{Rotativismo}
\begin{itemize}
\item {Grp. gram.:m.}
\end{itemize}
\begin{itemize}
\item {Utilização:Polit.}
\end{itemize}
\begin{itemize}
\item {Proveniência:(De \textunderscore rotativo\textunderscore )}
\end{itemize}
Predomínio de dois partidos políticos, que se alternam no poder.
\section{Rotativista}
\begin{itemize}
\item {Grp. gram.:m.}
\end{itemize}
Partidário do rotativismo.
\section{Rotativo}
\begin{itemize}
\item {Grp. gram.:adj.}
\end{itemize}
\begin{itemize}
\item {Utilização:Polit.}
\end{itemize}
\begin{itemize}
\item {Proveniência:(De \textunderscore rotar\textunderscore )}
\end{itemize}
Que faz rodar.
Relativo ao rotativismo.
\section{Rotatório}
\begin{itemize}
\item {Grp. gram.:adj.}
\end{itemize}
\begin{itemize}
\item {Grp. gram.:M.}
\end{itemize}
\begin{itemize}
\item {Proveniência:(Do lat. \textunderscore rotator\textunderscore )}
\end{itemize}
Rotante; relativo a rotação.
O mesmo que \textunderscore rotador\textunderscore , infusório.
\section{Rotear}
\begin{itemize}
\item {Grp. gram.:v. t.}
\end{itemize}
\begin{itemize}
\item {Grp. gram.:V. i.}
\end{itemize}
\begin{itemize}
\item {Proveniência:(De \textunderscore rota\textunderscore ^1)}
\end{itemize}
Dirigir (uma embarcação).
Marear.
\section{Rotear}
\textunderscore v. t.\textunderscore  e \textunderscore i.\textunderscore  (e der.)
O mesmo que \textunderscore arrotear\textunderscore , etc.
\section{Rotear}
\begin{itemize}
\item {Grp. gram.:v. t.}
\end{itemize}
\begin{itemize}
\item {Utilização:T. da Índia Port}
\end{itemize}
\begin{itemize}
\item {Proveniência:(De \textunderscore rota\textunderscore ^2)}
\end{itemize}
Empalhar (cadeiras), ou tecer-lhes os assentos com rota.
\section{Rotearia}
\begin{itemize}
\item {Grp. gram.:f.}
\end{itemize}
\begin{itemize}
\item {Proveniência:(De \textunderscore rotear\textunderscore ^2)}
\end{itemize}
O mesmo que \textunderscore arroteadura\textunderscore .
\section{Roteia}
\begin{itemize}
\item {Grp. gram.:f.}
\end{itemize}
O mesmo que \textunderscore arroteia\textunderscore .
\section{Roteiro}
\begin{itemize}
\item {Grp. gram.:m.}
\end{itemize}
\begin{itemize}
\item {Utilização:Fig.}
\end{itemize}
\begin{itemize}
\item {Proveniência:(De \textunderscore rota\textunderscore ^1)}
\end{itemize}
Itinerário ou descripção escrita dos pontos que é necessário conhecer para se fazer viagem marítima.
Descripção de uma grande viagem, marítima ou terrestre: \textunderscore roteiro de Vasco da Gama\textunderscore ; \textunderscore roteiro de D. João de Castro\textunderscore .
Indicação methódica da situação e direcção dos caminhos ou ruas, praças, etc., de uma povoação: \textunderscore roteiro de Lisbôa\textunderscore .
Regulamento.
\section{Roteiro}
\begin{itemize}
\item {Grp. gram.:m.}
\end{itemize}
\begin{itemize}
\item {Utilização:T. da Índia port}
\end{itemize}
\begin{itemize}
\item {Proveniência:(De \textunderscore rota\textunderscore ^2)}
\end{itemize}
Aquelle que roteia cadeiras; palheireiro.
\section{Rotejar-se}
\begin{itemize}
\item {Grp. gram.:v. p.}
\end{itemize}
\begin{itemize}
\item {Utilização:Prov.}
\end{itemize}
\begin{itemize}
\item {Utilização:trasm.}
\end{itemize}
\begin{itemize}
\item {Proveniência:(De \textunderscore rotar\textunderscore )}
\end{itemize}
Propalar-se, dizer-se, constar.
\section{Rotela}
\begin{itemize}
\item {Grp. gram.:f.}
\end{itemize}
\begin{itemize}
\item {Utilização:Ant.}
\end{itemize}
\begin{itemize}
\item {Proveniência:(De \textunderscore roto\textunderscore )}
\end{itemize}
Rotura; rompimento.
Violência.
\section{Rotela}
\begin{itemize}
\item {Grp. gram.:f.}
\end{itemize}
Gênero de moluscos gasterópodes.
\section{Rotella}
\begin{itemize}
\item {Grp. gram.:f.}
\end{itemize}
Gênero de molluscos gasterópodes.
\section{Rótia}
\begin{itemize}
\item {Grp. gram.:f.}
\end{itemize}
Gênero de plantas, da fam. das compostas.
\section{Rotífero}
\begin{itemize}
\item {Grp. gram.:adj.}
\end{itemize}
\begin{itemize}
\item {Grp. gram.:M.}
\end{itemize}
\begin{itemize}
\item {Proveniência:(Do lat. \textunderscore rota\textunderscore  + \textunderscore ferre\textunderscore )}
\end{itemize}
Que tem roda.
O mesmo que \textunderscore rotador\textunderscore , infusório.
\section{Rotiforme}
\begin{itemize}
\item {Grp. gram.:adj.}
\end{itemize}
\begin{itemize}
\item {Proveniência:(Do lat. \textunderscore rota\textunderscore  + \textunderscore forma\textunderscore )}
\end{itemize}
Que tem fórma de roda.
\section{Rotim}
\begin{itemize}
\item {Grp. gram.:m.}
\end{itemize}
Junco, com que se entretecem assentos de cadeiras, bancos, etc.; o mesmo que \textunderscore rota\textunderscore ^2.
(Do mal.\textunderscore rótan\textunderscore )
\section{Rotina}
\begin{itemize}
\item {Grp. gram.:f.}
\end{itemize}
\begin{itemize}
\item {Utilização:Fig.}
\end{itemize}
Caminho já trilhado ou sabido.
Prática constante, uso geral.
Hábito de proceder segundo o uso, sem attender a melhoramentos ou a progressos.
Índole conservadora ou opposta ao progresso.
(Talvez do fr. \textunderscore routine\textunderscore )
\section{Rotineira}
\textunderscore f.\textunderscore 
O mesmo que \textunderscore rotina\textunderscore .
\section{Rotineiramente}
\begin{itemize}
\item {Grp. gram.:adv.}
\end{itemize}
De modo rotineiro; segundo o uso estabelecido.
\section{Rotineiro}
\begin{itemize}
\item {Grp. gram.:adj.}
\end{itemize}
\begin{itemize}
\item {Grp. gram.:M.}
\end{itemize}
\begin{itemize}
\item {Proveniência:(De \textunderscore rotina\textunderscore )}
\end{itemize}
Relativo á rotina.
Que segue a rotina.
Consuetidinário.
Indivíduo rotineiro.
\section{Rotlera}
\begin{itemize}
\item {Grp. gram.:f.}
\end{itemize}
Gênero de plantas euphorbiáceas.
\section{Rôto}
\begin{itemize}
\item {Grp. gram.:adj.}
\end{itemize}
\begin{itemize}
\item {Grp. gram.:M.}
\end{itemize}
\begin{itemize}
\item {Proveniência:(Do lat. \textunderscore ruptus\textunderscore )}
\end{itemize}
Que se rompeu.
Esburacado: \textunderscore chapéu rôto\textunderscore .
Que traz o fato esburacado ou rasgado; mal vestido.
Maltrapilho:«\textunderscore não há naquella feira um tuno, um rôto, um pilho.\textunderscore »Castilho, \textunderscore Sabichonas\textunderscore , 187.
\section{Rotoria}
\begin{itemize}
\item {Grp. gram.:f.}
\end{itemize}
\begin{itemize}
\item {Utilização:Ant.}
\end{itemize}
\begin{itemize}
\item {Proveniência:(De \textunderscore rôto\textunderscore )}
\end{itemize}
O mesmo que \textunderscore arroteia\textunderscore ; terra arroteada.
\section{Rotótipa}
\begin{itemize}
\item {Proveniência:(Do lat. \textunderscore rota\textunderscore  + \textunderscore tupos\textunderscore )}
\end{itemize}
\textunderscore f.\textunderscore 
Máquina de composição tipográfica, inventada recentemente na Áustria.
\section{Rotótypa}
\begin{itemize}
\item {Proveniência:(Do lat. \textunderscore rota\textunderscore  + \textunderscore tupos\textunderscore )}
\end{itemize}
\textunderscore f.\textunderscore 
Máquina de composição typográphica, inventada recentemente na Áustria.
\section{Rótula}
\begin{itemize}
\item {Grp. gram.:f.}
\end{itemize}
\begin{itemize}
\item {Proveniência:(Lat. \textunderscore rotula\textunderscore )}
\end{itemize}
Gelosia.
Planta borraginea.
Osso, em fórma de disco, na parte anterior da articulação da tíbia com o fêmur.
\section{Rotulado}
\begin{itemize}
\item {Grp. gram.:adj.}
\end{itemize}
\begin{itemize}
\item {Grp. gram.:M.}
\end{itemize}
\begin{itemize}
\item {Utilização:Heráld.}
\end{itemize}
Que tem rótula; semelhante a uma rótula.
Bandas e barras entresachadas no escudo, deixando espaços entre si, como losangos. Cf. L. Ribeiro, \textunderscore Trat. de Armaria\textunderscore .
\section{Rotular}
\begin{itemize}
\item {Grp. gram.:v. t.}
\end{itemize}
Pôr inscripção ou rótulo em.
\section{Rotular}
\begin{itemize}
\item {Grp. gram.:adj.}
\end{itemize}
\begin{itemize}
\item {Utilização:Anat.}
\end{itemize}
Relativo ao osso, chamado rótula.
\section{Rótulo}
\begin{itemize}
\item {Grp. gram.:m.}
\end{itemize}
\begin{itemize}
\item {Utilização:Ant.}
\end{itemize}
\begin{itemize}
\item {Utilização:Ant.}
\end{itemize}
\begin{itemize}
\item {Proveniência:(Lat. \textunderscore rotulus\textunderscore )}
\end{itemize}
Letreiro ou dístico, que indica a natureza ou o fim do objecto em que está fixo.
Ralo ou pequena grade, em portas, janelas, etc.
Rôlo de pergaminho, em que se escrevia.
Pêso de quatro arráteis. Cf. Pant. de Aveiro, \textunderscore Itiner.\textunderscore , 69 v.^o, (2.^a ed.).
\section{Rotunda}
\begin{itemize}
\item {Grp. gram.:f.}
\end{itemize}
\begin{itemize}
\item {Proveniência:(Lat. \textunderscore rotunda\textunderscore )}
\end{itemize}
Construcção circular, que, termina em cúpula arredondada.
Praça ou largo, de fórma circular ou semi-circular.
\section{Rotundicollo}
\begin{itemize}
\item {Grp. gram.:adj.}
\end{itemize}
\begin{itemize}
\item {Utilização:Zool.}
\end{itemize}
\begin{itemize}
\item {Proveniência:(De \textunderscore rotundo\textunderscore  + \textunderscore collo\textunderscore )}
\end{itemize}
Que tem pescoço redondo.
\section{Rotundicolo}
\begin{itemize}
\item {Grp. gram.:adj.}
\end{itemize}
\begin{itemize}
\item {Utilização:Zool.}
\end{itemize}
\begin{itemize}
\item {Proveniência:(De \textunderscore rotundo\textunderscore  + \textunderscore colo\textunderscore )}
\end{itemize}
Que tem pescoço redondo.
\section{Rotundidade}
\begin{itemize}
\item {Grp. gram.:f.}
\end{itemize}
\begin{itemize}
\item {Proveniência:(Lat. \textunderscore rotunditas\textunderscore )}
\end{itemize}
Qualidade do que é redondo; obesidade.
\section{Rotundifólio}
\begin{itemize}
\item {Grp. gram.:adj.}
\end{itemize}
\begin{itemize}
\item {Utilização:Bot.}
\end{itemize}
\begin{itemize}
\item {Proveniência:(Do lat. \textunderscore rotundus\textunderscore  + \textunderscore folium\textunderscore )}
\end{itemize}
Que tem fôlhas redondas.
\section{Rotundiventre}
\begin{itemize}
\item {Grp. gram.:adj.}
\end{itemize}
\begin{itemize}
\item {Utilização:Zool.}
\end{itemize}
\begin{itemize}
\item {Proveniência:(De \textunderscore rotundo\textunderscore  + \textunderscore ventre\textunderscore )}
\end{itemize}
Que tem o ventre arredondado.
\section{Rotundo}
\begin{itemize}
\item {Grp. gram.:adj.}
\end{itemize}
\begin{itemize}
\item {Utilização:Fig.}
\end{itemize}
\begin{itemize}
\item {Proveniência:(Lat. \textunderscore rotundus\textunderscore )}
\end{itemize}
O mesmo que \textunderscore redondo\textunderscore .
Gordo; obeso.
\section{Rotura}
\begin{itemize}
\item {Grp. gram.:f.}
\end{itemize}
\begin{itemize}
\item {Utilização:Ant.}
\end{itemize}
\begin{itemize}
\item {Proveniência:(De \textunderscore rôto\textunderscore )}
\end{itemize}
O mesmo ou melhor que \textunderscore ruptura\textunderscore .
O mesmo que \textunderscore rotoria\textunderscore  ou \textunderscore arroteia\textunderscore .
\section{Rouba}
\begin{itemize}
\item {Grp. gram.:f.}
\end{itemize}
\begin{itemize}
\item {Utilização:Ant.}
\end{itemize}
O mesmo que \textunderscore roubo\textunderscore .
\section{Roubadia}
\begin{itemize}
\item {Grp. gram.:f.}
\end{itemize}
\begin{itemize}
\item {Utilização:Ant.}
\end{itemize}
\begin{itemize}
\item {Proveniência:(De \textunderscore roubar\textunderscore )}
\end{itemize}
O mesmo que \textunderscore roubo\textunderscore .
\section{Roubado}
\begin{itemize}
\item {Grp. gram.:adj.}
\end{itemize}
Que foi objecto do roubo: \textunderscore um relógio roubado\textunderscore .
Que soffreu roubo: \textunderscore fiquei roubado\textunderscore .
\section{Roubador}
\begin{itemize}
\item {Grp. gram.:m.  e  f.}
\end{itemize}
Que rouba.
\section{Roubalheira}
\begin{itemize}
\item {Grp. gram.:f.}
\end{itemize}
\begin{itemize}
\item {Utilização:Fam.}
\end{itemize}
\begin{itemize}
\item {Proveniência:(De \textunderscore roubar\textunderscore )}
\end{itemize}
Roubo importante e escandaloso.
Subtracção de quantias ou valores, pertencentes ao Estado.
\section{Roubar}
\begin{itemize}
\item {Grp. gram.:v. t.}
\end{itemize}
\begin{itemize}
\item {Utilização:Fig.}
\end{itemize}
\begin{itemize}
\item {Grp. gram.:V. p.}
\end{itemize}
\begin{itemize}
\item {Utilização:Des.}
\end{itemize}
Subtrahir violentamente.
Apoderar-se injustamente de; furtar.
Arrancar.
Salvar, livrar.
Despojar.
Raptar.
Plagiar.
Arroubar, enlevar.
Esquivar-se, furtar-se.
(Cp.\textunderscore roupa\textunderscore )
\section{Roubo}
\begin{itemize}
\item {Grp. gram.:m.}
\end{itemize}
\begin{itemize}
\item {Utilização:Fig.}
\end{itemize}
Acto ou effeito de roubar.
Aquillo que se roubou.
Preço excessivo: \textunderscore um jantar por uma libra é um roubo\textunderscore .
\section{Rouca}
\begin{itemize}
\item {Grp. gram.:f.}
\end{itemize}
\begin{itemize}
\item {Utilização:Prov.}
\end{itemize}
O mesmo que \textunderscore abetoiro\textunderscore .
\section{Rouçador}
\begin{itemize}
\item {Grp. gram.:m.  e  adj.}
\end{itemize}
O que rouça.
\section{Roucamente}
\begin{itemize}
\item {Grp. gram.:adv.}
\end{itemize}
De modo rouco.
\section{Roucanho}
\begin{itemize}
\item {Grp. gram.:adj.}
\end{itemize}
O mesmo que \textunderscore rouquenho\textunderscore :«\textunderscore ...em tom roucanho e ameaçador.\textunderscore »Corvo, \textunderscore Anno na Côrte\textunderscore , III, 240.
\section{Roução}
\begin{itemize}
\item {Grp. gram.:m.  e  adj.}
\end{itemize}
\begin{itemize}
\item {Utilização:Ant.}
\end{itemize}
O mesmo que \textunderscore rouçador\textunderscore .
\section{Rouçar}
\begin{itemize}
\item {Grp. gram.:v. t.}
\end{itemize}
\begin{itemize}
\item {Utilização:Ant.}
\end{itemize}
\begin{itemize}
\item {Utilização:Prov.}
\end{itemize}
\begin{itemize}
\item {Utilização:minh.}
\end{itemize}
\begin{itemize}
\item {Proveniência:(Do lat. hyp.\textunderscore ruptiar\textunderscore , de \textunderscore ruptus\textunderscore )}
\end{itemize}
Raptar (mulher honesta ou donzella).
Forçar, violentar (uma mulher).
Roçar, esfregar.
\section{Rouco}
\begin{itemize}
\item {Grp. gram.:adj.}
\end{itemize}
\begin{itemize}
\item {Proveniência:(Do lat. \textunderscore raucus\textunderscore )}
\end{itemize}
Que enrouqueceu.
Roufenho.
Que tem som cavo.
\section{Rouço}
\begin{itemize}
\item {Grp. gram.:m.}
\end{itemize}
\begin{itemize}
\item {Utilização:Ant.}
\end{itemize}
Acto de rouçar.
\section{Roufenhar}
\begin{itemize}
\item {Grp. gram.:v. i.}
\end{itemize}
Têr voz roufenha. Cf. Camillo, \textunderscore Quéda\textunderscore , 29.
\section{Roufenho}
\begin{itemize}
\item {Grp. gram.:adj.}
\end{itemize}
\begin{itemize}
\item {Proveniência:(T. onom.)}
\end{itemize}
Que parece falar pelo nariz; fanhoso.
Que tem som cavo e áspero.
Que tem som áspero e baixo.
\section{Roupa}
\begin{itemize}
\item {Grp. gram.:f.}
\end{itemize}
\begin{itemize}
\item {Utilização:Ant.}
\end{itemize}
\begin{itemize}
\item {Utilização:Ant.}
\end{itemize}
\begin{itemize}
\item {Utilização:Fig.}
\end{itemize}
\begin{itemize}
\item {Proveniência:(Do ant. alt. al. \textunderscore roub\textunderscore )}
\end{itemize}
Designação genérica das peças de vestuário ou das peças de estôfo para a cama.
Fazenda ou tecido, próprio para essas peças.
Fato.
Espécie de capa.
\textunderscore Roupa de estrado\textunderscore , tapetes.
\textunderscore Roupa de Franceses\textunderscore , coisa, de que todos lançam mão, roubando ou queimando.
\section{Roupa-de-chambre}
\begin{itemize}
\item {Grp. gram.:f.}
\end{itemize}
Vestuário comprido e ligeiro, que os Franceses chamam \textunderscore robe-de-chambre\textunderscore . Cf.\textunderscore Anat. Joc.\textunderscore , I, 367.
\section{Roupagem}
\begin{itemize}
\item {Grp. gram.:f.}
\end{itemize}
\begin{itemize}
\item {Utilização:Fig.}
\end{itemize}
\begin{itemize}
\item {Proveniência:(De \textunderscore roupa\textunderscore )}
\end{itemize}
Roupas; rouparia.
Representação artística das roupas ou vestuário.
Exterioridade; coisa vistosa e frívola ou insignificante: \textunderscore estilo de muita roupagem e poucas ideias\textunderscore .
\section{Roupão}
\begin{itemize}
\item {Grp. gram.:m.}
\end{itemize}
\begin{itemize}
\item {Proveniência:(De \textunderscore roupa\textunderscore )}
\end{itemize}
Peça de vestuário, comprida e ampla, em geral de uso doméstico.
Chambre; bata.
\section{Roupar}
\begin{itemize}
\item {Grp. gram.:v. t.  e  p.}
\end{itemize}
O mesmo que \textunderscore enroupar\textunderscore .
\section{Rouparia}
\begin{itemize}
\item {Grp. gram.:f.}
\end{itemize}
\begin{itemize}
\item {Utilização:Prov.}
\end{itemize}
\begin{itemize}
\item {Utilização:alent.}
\end{itemize}
Porção de roupa.
Lugar, onde se vendem ou guardam roupas.
Casa, destinada especialmente ao fabrico de queijos. Cp. Rev.\textunderscore Tradição\textunderscore , IX, 130.
\section{Roupavelheiro}
\begin{itemize}
\item {Grp. gram.:m.}
\end{itemize}
\begin{itemize}
\item {Proveniência:(De \textunderscore roupa\textunderscore  + \textunderscore velho\textunderscore )}
\end{itemize}
Vendedor de fato usado; ferro-velho.
\section{Roupeira}
\begin{itemize}
\item {Grp. gram.:f.  e  adj.}
\end{itemize}
\begin{itemize}
\item {Proveniência:(De \textunderscore roupa\textunderscore )}
\end{itemize}
Espécie de uva algarvia, o mesmo que \textunderscore roupeiro\textunderscore ^2.
Mulhér, encarregada de guardar as roupas de uma família ou communidade.
\section{Roupeiro}
\begin{itemize}
\item {Grp. gram.:m.  e  adj.}
\end{itemize}
\begin{itemize}
\item {Utilização:Prov.}
\end{itemize}
\begin{itemize}
\item {Utilização:beir.}
\end{itemize}
\begin{itemize}
\item {Utilização:Prov.}
\end{itemize}
\begin{itemize}
\item {Utilização:alent.}
\end{itemize}
Indivíduo, encarregado de guardar a roupa de uma família ou communidade.
Aquelle que faz roupa.
Pastor, que faz queijos.
Qualquer indivíduo, que faz queijos.
\section{Roupeiro}
\begin{itemize}
\item {Grp. gram.:m.}
\end{itemize}
Casta de uva branca, talvez o mesmo que \textunderscore dona-branca\textunderscore .
\section{Roupeta}
\begin{itemize}
\item {fónica:pê}
\end{itemize}
\begin{itemize}
\item {Grp. gram.:f.}
\end{itemize}
\begin{itemize}
\item {Grp. gram.:M.}
\end{itemize}
\begin{itemize}
\item {Utilização:Deprec.}
\end{itemize}
\begin{itemize}
\item {Proveniência:(De \textunderscore roupa\textunderscore )}
\end{itemize}
Batina.
Padre.
\section{Roupido}
\begin{itemize}
\item {Grp. gram.:adj.}
\end{itemize}
\begin{itemize}
\item {Proveniência:(De \textunderscore roupa\textunderscore )}
\end{itemize}
Que está vestido ou provido de roupas.
\section{Roupinha}
\begin{itemize}
\item {Grp. gram.:f.}
\end{itemize}
\begin{itemize}
\item {Proveniência:(De \textunderscore roupa\textunderscore )}
\end{itemize}
Casaco curto e ajustado ao corpo, usado especialmente pelas mulheres do campo.
\section{Roupudo}
\begin{itemize}
\item {Grp. gram.:m.}
\end{itemize}
\begin{itemize}
\item {Utilização:Prov.}
\end{itemize}
\begin{itemize}
\item {Utilização:trasm.}
\end{itemize}
\begin{itemize}
\item {Proveniência:(De \textunderscore roupa\textunderscore )}
\end{itemize}
Variedade de azeitona.
\section{Rouquejar}
\begin{itemize}
\item {Grp. gram.:v. i.}
\end{itemize}
\begin{itemize}
\item {Utilização:Ext.}
\end{itemize}
\begin{itemize}
\item {Proveniência:(De \textunderscore rouco\textunderscore )}
\end{itemize}
Emittir sons roucos.
Têr rouqueira.
Troar; rugir.
\section{Rouquenho}
\begin{itemize}
\item {Grp. gram.:adj.}
\end{itemize}
Um tanto rouco; roufenho.
\section{Rouquento}
\begin{itemize}
\item {Grp. gram.:adj.}
\end{itemize}
O mesmo que \textunderscore rouquenho\textunderscore . Cf.\textunderscore Anat. Joc.\textunderscore , I, 258.
\section{Rouquice}
\begin{itemize}
\item {Grp. gram.:f.}
\end{itemize}
Estado do que é rouco.
\section{Rouquidão}
\begin{itemize}
\item {Grp. gram.:f.}
\end{itemize}
Estado do que é rouco.
\section{Rouquido}
\begin{itemize}
\item {Grp. gram.:m.}
\end{itemize}
Som, rouco produzido pela respiração de um enfermo ou moribundo.
Rouquidão.
\section{Rou-rou}
\begin{itemize}
\item {Grp. gram.:m.}
\end{itemize}
Espécie de jôgo popular.
\section{Roussinol}
\begin{itemize}
\item {Grp. gram.:m.}
\end{itemize}
\begin{itemize}
\item {Utilização:Ant.}
\end{itemize}
O mesmo que \textunderscore rouxinol\textunderscore . Cf. G. Vicente, \textunderscore Auto das Fadas\textunderscore .
\section{Rouval}
\begin{itemize}
\item {Grp. gram.:adj.}
\end{itemize}
O mesmo que \textunderscore ruival\textunderscore .
\section{Rouvinhoso}
\begin{itemize}
\item {Grp. gram.:adj.}
\end{itemize}
\begin{itemize}
\item {Utilização:Des.}
\end{itemize}
Rabugento; que está de mau humor.
Caprichoso. Cf. Moraes, \textunderscore Diccion.\textunderscore 
(Alter, de \textunderscore ruvinhoso\textunderscore )
\section{Rouxinol}
\begin{itemize}
\item {Grp. gram.:m.}
\end{itemize}
\begin{itemize}
\item {Grp. gram.:M.  e  f.}
\end{itemize}
\begin{itemize}
\item {Utilização:Fig.}
\end{itemize}
\begin{itemize}
\item {Utilização:T. de Cacilhas}
\end{itemize}
\begin{itemize}
\item {Utilização:T. do Fundão}
\end{itemize}
\begin{itemize}
\item {Proveniência:(Lat. \textunderscore lusciniola\textunderscore )}
\end{itemize}
Pequeno pássaro dentirostro, de canto suavíssimo; philomela.
Pessôa, que canta muito bem.
O mesmo que \textunderscore rouxinol-de-almada\textunderscore .
Assobio com água, usado pelos rapazes de Lisbôa, nas noites de Santo-António, San-João e San-Pedro.
Gaita, feita de vergôntea de castanheiro, a que se despega a casca, sem a rasgar.
\section{Rouxinol-de-almada}
\begin{itemize}
\item {Grp. gram.:m.}
\end{itemize}
\begin{itemize}
\item {Utilização:Pop.}
\end{itemize}
\begin{itemize}
\item {Utilização:Ant.}
\end{itemize}
O burro, quando zurra.
\section{Rouxinol-de-espadana}
\begin{itemize}
\item {Grp. gram.:m.}
\end{itemize}
\begin{itemize}
\item {Utilização:T. da Bairrada}
\end{itemize}
Ave, o mesmo que \textunderscore chinchafolles\textunderscore .
\section{Rouxinol-de-muralha}
\begin{itemize}
\item {Grp. gram.:m.}
\end{itemize}
Ave avermelhada, que faz o ninho nos muros, (\textunderscore phonicurus\textunderscore , Lin).
\section{Roxbúrgia}
\begin{itemize}
\item {Grp. gram.:f.}
\end{itemize}
Gênero de plantas, que serve de typo ás roxburgiáceas.
\section{Roxburgiáceas}
\begin{itemize}
\item {Grp. gram.:f. pl.}
\end{itemize}
Família de plantas monocotyledóneas do Japão.
\section{Roxear}
\begin{itemize}
\item {Grp. gram.:v. t.}
\end{itemize}
\begin{itemize}
\item {Grp. gram.:V. i.}
\end{itemize}
O mesmo que \textunderscore arroxear\textunderscore .
Tornar-se roxo ou purpúreo.
\section{Roxecré}
\begin{itemize}
\item {Grp. gram.:m.}
\end{itemize}
\begin{itemize}
\item {Utilização:Ant.}
\end{itemize}
O mesmo que \textunderscore rosiclér\textunderscore . Cf.\textunderscore Papéis\textunderscore  da Chancell. de D. João III, na Tôrre do Tombo.
\section{Roxeta}
\begin{itemize}
\item {fónica:xê}
\end{itemize}
\begin{itemize}
\item {Grp. gram.:f.}
\end{itemize}
Espécie de saragoça ou burel, que dantes se fabricava na Serra da Estrêlla. Cf. Côrvo, \textunderscore Anno na Côrte\textunderscore , I, 263.
\section{Roxete}
\begin{itemize}
\item {fónica:xê}
\end{itemize}
\begin{itemize}
\item {Grp. gram.:adj.}
\end{itemize}
Um tanto roxo; arroxeado. Cf. Sousa, \textunderscore Vida do Arceb.\textunderscore , I, 367.
\section{Roxinho}
\begin{itemize}
\item {Grp. gram.:m.}
\end{itemize}
Árvore, o mesmo que \textunderscore guarabu\textunderscore .
\section{Roxinol}
\begin{itemize}
\item {Grp. gram.:m.}
\end{itemize}
O mesmo que \textunderscore rouxinol\textunderscore .--G. Guimarães, \textunderscore Rev. da Univ. de Coimbra\textunderscore , I, 8, entende que \textunderscore roxinol\textunderscore  é a fórma exacta, e que \textunderscore rouxinol\textunderscore , \textunderscore roixinol\textunderscore  e \textunderscore roissinol\textunderscore  são fórmas affectadas. Entretanto, cp. o cast. \textunderscore ruiseñor\textunderscore , o catalão \textunderscore rossinyol\textunderscore , o it. \textunderscore rosignuolo\textunderscore , o fr. \textunderscore rossignol\textunderscore , etc.
\section{Roxo}
\begin{itemize}
\item {fónica:rô}
\end{itemize}
\begin{itemize}
\item {Grp. gram.:adj.}
\end{itemize}
\begin{itemize}
\item {Utilização:Ant.}
\end{itemize}
\begin{itemize}
\item {Grp. gram.:M.}
\end{itemize}
\begin{itemize}
\item {Utilização:Chul.}
\end{itemize}
\begin{itemize}
\item {Proveniência:(Do lat. \textunderscore russeus\textunderscore )}
\end{itemize}
Que tem côr tirante a rubro e violáceo.
O mesmo que \textunderscore vermelho\textunderscore : \textunderscore os Hebreus atravessaram o Mar-Roxo\textunderscore .
A côr roxa.
Vinho tinto.
\section{Roxo-terra}
\begin{itemize}
\item {Grp. gram.:f.}
\end{itemize}
Uma das variedades da argilla ferruginosa.
\section{Rozeimo}
\begin{itemize}
\item {Grp. gram.:m.}
\end{itemize}
\begin{itemize}
\item {Utilização:Prov.}
\end{itemize}
O mesmo que \textunderscore rancor\textunderscore .
\section{Rozental}
\begin{itemize}
\item {Grp. gram.:m.}
\end{itemize}
\begin{itemize}
\item {Utilização:Prov.}
\end{itemize}
Cordeiro tenro.
(Corr. do \textunderscore recental\textunderscore . Cp.\textunderscore rezental\textunderscore )
\section{Rua}
\begin{itemize}
\item {Grp. gram.:f.}
\end{itemize}
\begin{itemize}
\item {Utilização:Ext.}
\end{itemize}
\begin{itemize}
\item {Utilização:Fig.}
\end{itemize}
\begin{itemize}
\item {Proveniência:(Fr. \textunderscore rue\textunderscore , do b. lat. \textunderscore ruga\textunderscore )}
\end{itemize}
Caminho, ladeado de casas, paredes ou renques de árvores, numa povoação.
Casas, que ladeiam êsse caminho: \textunderscore os moradores da rua Augusta\textunderscore .
Habitantes dessas casas: \textunderscore a rua do Carmo alvorotou-se\textunderscore .
Conjunto dos lugares, por onde se anda numa povoação, afóra os domicílios: \textunderscore os pregões da rua\textunderscore .
A classe inferior da sociedade; a plebe: \textunderscore a rua está ao lado do Govêrno\textunderscore .
\section{Ruaça}
\begin{itemize}
\item {Grp. gram.:f.}
\end{itemize}
O mesmo que \textunderscore arruaça\textunderscore .
\section{Ruaceiro}
\begin{itemize}
\item {Grp. gram.:m.}
\end{itemize}
O mesmo que \textunderscore arruaceiro\textunderscore .
\section{Rua-dos-salgados}
\begin{itemize}
\item {Grp. gram.:f.}
\end{itemize}
Espécie de jôgo popular.
\section{Rual}
\begin{itemize}
\item {Grp. gram.:m.}
\end{itemize}
Casta de uva de Azeitão. Cf.\textunderscore Rev. Agron.\textunderscore , I, 18.
\section{Ruano}
\begin{itemize}
\item {Grp. gram.:m.  e  adj.}
\end{itemize}
O mesmo que \textunderscore ruão\textunderscore ^2.
\section{Ruante}
\begin{itemize}
\item {Grp. gram.:adj.}
\end{itemize}
Diz-se do pavão, quando ergue a cauda.
\section{Ruão}
\begin{itemize}
\item {Grp. gram.:m.}
\end{itemize}
\begin{itemize}
\item {Proveniência:(Do fr. \textunderscore Rouen\textunderscore , n. p.)}
\end{itemize}
Espécie de tecido de linho, que se fabricava em Ruão.
\section{Ruão}
\begin{itemize}
\item {Grp. gram.:m.  e  adj.}
\end{itemize}
Diz-se do cavallo, cujo pêlo é mesclado de branco e pardo, ou do cavallo de pêlo branco com malhas escuras e redondas.
(Cast. \textunderscore ruano\textunderscore )
\section{Ruão}
\begin{itemize}
\item {Grp. gram.:m.}
\end{itemize}
\begin{itemize}
\item {Utilização:T. da Bairrada}
\end{itemize}
\begin{itemize}
\item {Utilização:Ant.}
\end{itemize}
\begin{itemize}
\item {Proveniência:(De \textunderscore rua\textunderscore )}
\end{itemize}
Plebeu.
Homem do povo.
Peão.
Estrume miúdo e sêco.
Homem civilizado (de \textunderscore rua\textunderscore  ou cidade), em contraposição a homem rude, do campo.
Pão alvo, ou de trigo, próprio de gente culta.
\section{Rubago}
\begin{itemize}
\item {Grp. gram.:m.}
\end{itemize}
\begin{itemize}
\item {Utilização:Bras}
\end{itemize}
Peixe fluvial.
\section{Rubefacção}
\begin{itemize}
\item {Grp. gram.:f.}
\end{itemize}
\begin{itemize}
\item {Proveniência:(Do lat. \textunderscore rubefactus\textunderscore )}
\end{itemize}
Inflammação, acompanhada de vermelhidão na pelle.
\section{Rubefaciente}
\begin{itemize}
\item {Grp. gram.:adj.}
\end{itemize}
\begin{itemize}
\item {Grp. gram.:M.}
\end{itemize}
\begin{itemize}
\item {Proveniência:(Lat. \textunderscore rubefaciens\textunderscore )}
\end{itemize}
Que causa vermelhidão.
Preparado pharmacêutico, para produzir rubefacção.
\section{Rubejar}
\begin{itemize}
\item {Grp. gram.:v. i.}
\end{itemize}
\begin{itemize}
\item {Proveniência:(De \textunderscore rúbeo\textunderscore )}
\end{itemize}
Mostrar-se rubro ou vermelho. Cf. Júl. Ribeiro, \textunderscore Padre Belch.\textunderscore , 90.
\section{Rubelana}
\begin{itemize}
\item {Grp. gram.:f.}
\end{itemize}
Substância mineral opaca, que se encontra na Bohêmia.
\section{Rubelito}
\begin{itemize}
\item {Grp. gram.:m.}
\end{itemize}
\begin{itemize}
\item {Utilização:Miner.}
\end{itemize}
Variedade de turmalina carmesim.
\section{Rubente}
\begin{itemize}
\item {Grp. gram.:adj.}
\end{itemize}
\begin{itemize}
\item {Proveniência:(Lat. \textunderscore rubens\textunderscore )}
\end{itemize}
Que tem côr vermelha; rubro.
\section{Rúbeo}
\begin{itemize}
\item {Grp. gram.:adj.}
\end{itemize}
\begin{itemize}
\item {Proveniência:(Lat. \textunderscore rubeus\textunderscore )}
\end{itemize}
O mesmo que \textunderscore rubro\textunderscore .
\section{Rubéola}
\begin{itemize}
\item {Grp. gram.:f.}
\end{itemize}
\begin{itemize}
\item {Utilização:Med.}
\end{itemize}
Febre eruptiva, parecida com o sarampo, mas distinta delle.
(Cp. lat. \textunderscore rubellulus\textunderscore )
\section{Rubescência}
\begin{itemize}
\item {Grp. gram.:f.}
\end{itemize}
Qualidade de rubescente.
\section{Rubescente}
\begin{itemize}
\item {Grp. gram.:adj.}
\end{itemize}
Que rubesce.
\section{Rubescer}
\begin{itemize}
\item {Grp. gram.:v. i.}
\end{itemize}
O mesmo que \textunderscore enrubescer\textunderscore .
\section{Rubeta}
\begin{itemize}
\item {fónica:bê}
\end{itemize}
\begin{itemize}
\item {Grp. gram.:f.}
\end{itemize}
\begin{itemize}
\item {Proveniência:(Lat. \textunderscore rubeta\textunderscore )}
\end{itemize}
Espécie de ran, o mesmo que \textunderscore rela\textunderscore .
\section{Rubi}
\begin{itemize}
\item {Grp. gram.:m.}
\end{itemize}
\begin{itemize}
\item {Utilização:Poét.}
\end{itemize}
\begin{itemize}
\item {Proveniência:(Do lat. \textunderscore rubidus\textunderscore )}
\end{itemize}
Pedra preciosa e transparente, de côr vermelha.
Côr muito vermelha, afogueada: \textunderscore lábios de rubi\textunderscore .
\section{Rubiáceas}
\begin{itemize}
\item {Grp. gram.:f. pl.}
\end{itemize}
Família de plantas medicinaes, que tem por typo a granza.
(Fem. pl. de \textunderscore rubiáceo\textunderscore )
\section{Rubiáceo}
\begin{itemize}
\item {Grp. gram.:adj.}
\end{itemize}
\begin{itemize}
\item {Proveniência:(Do lat. \textunderscore rubia\textunderscore )}
\end{itemize}
Relativo ou semelhante á granza.
\section{Rubiânico}
\begin{itemize}
\item {Grp. gram.:adj.}
\end{itemize}
\begin{itemize}
\item {Proveniência:(De \textunderscore rubião\textunderscore )}
\end{itemize}
Diz-se de um ácido, produzido na oxydação do rubião.
\section{Rubianina}
\begin{itemize}
\item {Grp. gram.:f.}
\end{itemize}
\begin{itemize}
\item {Proveniência:(De \textunderscore rubião\textunderscore )}
\end{itemize}
Um dos productos da decomposição do rubião.
\section{Rubião}
\begin{itemize}
\item {Grp. gram.:m.}
\end{itemize}
\begin{itemize}
\item {Grp. gram.:Adj.}
\end{itemize}
\begin{itemize}
\item {Proveniência:(Do lat. \textunderscore rubia\textunderscore )}
\end{itemize}
Substância còrante, extrahida da raíz da granza.
Diz-se de uma variedade de milho molle.
Diz-se uma variedade de trigo. Cf.\textunderscore Port. ao point de vue agr.\textunderscore , 579.
\section{Rubicano}
\begin{itemize}
\item {Grp. gram.:adj.}
\end{itemize}
Diz-se do cavallo negro, baio ou alazão, com pêlos brancos, especialmente na pelle.
(Cast. \textunderscore rubicano\textunderscore )
\section{Rubicão}
\begin{itemize}
\item {Grp. gram.:m.}
\end{itemize}
\begin{itemize}
\item {Utilização:Fig.}
\end{itemize}
\begin{itemize}
\item {Proveniência:(De \textunderscore Rubicão\textunderscore , n. p.)}
\end{itemize}
Obstáculo, difficuldade: \textunderscore receava grandes perigos, mas encheu-se de coragem e passou o rubicão\textunderscore .
\section{Rubicundo}
\begin{itemize}
\item {Grp. gram.:adj.}
\end{itemize}
\begin{itemize}
\item {Proveniência:(Lat. \textunderscore rubicundus\textunderscore )}
\end{itemize}
O mesmo que \textunderscore vermelho\textunderscore .
\section{Rubidez}
\begin{itemize}
\item {Grp. gram.:f.}
\end{itemize}
\begin{itemize}
\item {Proveniência:(De \textunderscore róbido\textunderscore )}
\end{itemize}
O mesmo que \textunderscore rubor\textunderscore .
\section{Rubídio}
\begin{itemize}
\item {Grp. gram.:m.}
\end{itemize}
\begin{itemize}
\item {Proveniência:(Do lat. \textunderscore rubidus\textunderscore )}
\end{itemize}
Nome de um metal, recentemente descoberto, o qual apresenta dois veios rubros.
\section{Rúbido}
\begin{itemize}
\item {Grp. gram.:adj.}
\end{itemize}
\begin{itemize}
\item {Utilização:Poét.}
\end{itemize}
\begin{itemize}
\item {Proveniência:(Lat. \textunderscore rubidus\textunderscore )}
\end{itemize}
Vermelho; afogueado.
\section{Rubieva}
\begin{itemize}
\item {Grp. gram.:f.}
\end{itemize}
Gênero de plantas chenopodiáceas.
\section{Rubificação}
\begin{itemize}
\item {Grp. gram.:f.}
\end{itemize}
Acto ou effeito de rubificar.
\section{Rubificante}
\begin{itemize}
\item {Grp. gram.:adj.}
\end{itemize}
\begin{itemize}
\item {Proveniência:(De \textunderscore rubificar\textunderscore )}
\end{itemize}
Que rubifica.
O mesmo que \textunderscore rubefaciente\textunderscore .
\section{Rubificar}
\begin{itemize}
\item {Grp. gram.:v. t.}
\end{itemize}
\begin{itemize}
\item {Proveniência:(Do lat. \textunderscore rubeus\textunderscore  + \textunderscore facere\textunderscore )}
\end{itemize}
Tornar vermelho.
\section{Rubiforme}
\begin{itemize}
\item {Grp. gram.:adj.}
\end{itemize}
\begin{itemize}
\item {Proveniência:(Do lat. \textunderscore rubus\textunderscore  + \textunderscore forma\textunderscore )}
\end{itemize}
Semelhante á amora das silveiras.
\section{Rubigal}
\begin{itemize}
\item {Grp. gram.:adj.}
\end{itemize}
\begin{itemize}
\item {Proveniência:(Lat. \textunderscore rubigalis\textunderscore )}
\end{itemize}
Relativo ás festas, que os Romanos celebravam para evitar a ferrugem dos cereaes. Cf. Castilho, \textunderscore Fastos\textunderscore , I, 516.
\section{Rubígine}
\begin{itemize}
\item {Grp. gram.:f.}
\end{itemize}
\begin{itemize}
\item {Utilização:Poét.}
\end{itemize}
\begin{itemize}
\item {Proveniência:(Lat. \textunderscore rubigo\textunderscore )}
\end{itemize}
O mesmo que \textunderscore ferrugem\textunderscore . Cf. Castilho, \textunderscore Fastos\textunderscore , I, XXXV.--A fórma exacta seria \textunderscore rubigem\textunderscore .
\section{Rubiginoso}
\begin{itemize}
\item {Grp. gram.:adj.}
\end{itemize}
\begin{itemize}
\item {Proveniência:(Lat. \textunderscore rubiginosus\textunderscore )}
\end{itemize}
Ferrugento.
\section{Rubim}
\begin{itemize}
\item {Grp. gram.:m.}
\end{itemize}
O mesmo que \textunderscore rubi\textunderscore .
\section{Rubina}
\begin{itemize}
\item {Grp. gram.:f.}
\end{itemize}
\begin{itemize}
\item {Proveniência:(Fr. \textunderscore rubine\textunderscore )}
\end{itemize}
Preparado insecticida, com que se pulverizam as plantas, contra o pulgão e outros parasitos.
\section{Rubinéctar}
\begin{itemize}
\item {Grp. gram.:m.}
\end{itemize}
\begin{itemize}
\item {Proveniência:(De \textunderscore rúbeo\textunderscore  + \textunderscore néctar\textunderscore )}
\end{itemize}
Néctar vermelho.
Vinho:«\textunderscore ...vem a mim-rubi-néctar...\textunderscore »Filinto, IV, 102.
\section{Rubinete}
\begin{itemize}
\item {fónica:nê}
\end{itemize}
\begin{itemize}
\item {Grp. gram.:m.}
\end{itemize}
Rubim pequeno.
\section{Rubiretina}
\begin{itemize}
\item {fónica:re}
\end{itemize}
\begin{itemize}
\item {Grp. gram.:f.}
\end{itemize}
\begin{itemize}
\item {Proveniência:(Do lat. \textunderscore rubia\textunderscore  + gr. \textunderscore retine\textunderscore )}
\end{itemize}
Substância, extrahída da raíz da granza.
\section{Rubirretina}
\begin{itemize}
\item {Grp. gram.:f.}
\end{itemize}
\begin{itemize}
\item {Proveniência:(Do lat. \textunderscore rubia\textunderscore  + gr. \textunderscore retine\textunderscore )}
\end{itemize}
Substância, extraída da raíz da granza.
\section{Rubi-topázio}
\begin{itemize}
\item {Grp. gram.:m.}
\end{itemize}
Pássaro das regiões do Amazonas.
\section{Rublo}
\begin{itemize}
\item {Grp. gram.:m.}
\end{itemize}
Moéda russa, que equivale proximamente a 700 reis.
(Do russo \textunderscore rublh\textunderscore )
\section{Rubo}
\begin{itemize}
\item {Grp. gram.:m.}
\end{itemize}
\begin{itemize}
\item {Proveniência:(Lat. \textunderscore rubus\textunderscore )}
\end{itemize}
Silveira; sarça.
Amora de silva.
\section{Rubor}
\begin{itemize}
\item {Grp. gram.:m.}
\end{itemize}
\begin{itemize}
\item {Utilização:Fig.}
\end{itemize}
\begin{itemize}
\item {Proveniência:(Lat. \textunderscore rubor\textunderscore )}
\end{itemize}
Qualidade de rubro.
Vermelhidão; côr vermelha.
Pejo.
Modéstia; pudor.
\section{Ruborescer}
\begin{itemize}
\item {Grp. gram.:v. i.}
\end{itemize}
\begin{itemize}
\item {Proveniência:(De \textunderscore rubor\textunderscore )}
\end{itemize}
Tornar-se vermelho; ruborizar-se:«\textunderscore ...a avezinha canta ao ruborescer da aurora.\textunderscore »Th. Ribeiro, \textunderscore Jornadas\textunderscore , II, 205.
\section{Ruborescido}
\begin{itemize}
\item {Grp. gram.:adj.}
\end{itemize}
\begin{itemize}
\item {Utilização:Fig.}
\end{itemize}
\begin{itemize}
\item {Proveniência:(De \textunderscore ruborescer\textunderscore )}
\end{itemize}
Vermelho; còrado.
Envergonhado.
\section{Ruborização}
\begin{itemize}
\item {Grp. gram.:f.}
\end{itemize}
Acto ou effeito de ruborizar.
\section{Ruborizar}
\begin{itemize}
\item {Grp. gram.:v. t.}
\end{itemize}
\begin{itemize}
\item {Grp. gram.:V. p.}
\end{itemize}
\begin{itemize}
\item {Utilização:Fig.}
\end{itemize}
Tornar vermelho rubro.
Causar rubor a.
Còrar.
Têr pudor, envergonhar-se.
\section{Rubrica}
\begin{itemize}
\item {Grp. gram.:f.}
\end{itemize}
\begin{itemize}
\item {Utilização:Ext.}
\end{itemize}
\begin{itemize}
\item {Proveniência:(Lat. \textunderscore rubrica\textunderscore )}
\end{itemize}
(e não rúbrica, como muitos dizem)
Terra vermelha, applicada em pintura e em vários ramos da indústria.
Título dos capítulos de Direito canónico e civil, que dantes era impresso a côr vermelha.
Nota, segundo a qual se devem celebrar os Offícios Divinos, e que em geral é escrita em letra vermelha, nos breviários, missaes, etc.
Preceito, incluído nessa nota.
Sinal indicativo dos movimentos e gestos dos actores, consignado nos respectivos papéis.
Nota.
Firma, ou assinatura em breve.
\section{Rubricador}
\begin{itemize}
\item {Grp. gram.:m.  e  adj.}
\end{itemize}
O que rubrica.
\section{Rubricar}
\begin{itemize}
\item {Grp. gram.:v. t.}
\end{itemize}
\begin{itemize}
\item {Proveniência:(Lat. \textunderscore rubricare\textunderscore )}
\end{itemize}
Pôr uma rubrica ou rubricas em.
Assinalar.
Pôr marca em.
Firmar, com appelido ou com abreviatura do nome: \textunderscore o juiz rubricou o processo\textunderscore .
\section{Rubricista}
\begin{itemize}
\item {Grp. gram.:m.}
\end{itemize}
Aquelle que é perito em rubricas ecclesiásticas.
\section{Rubricollo}
\begin{itemize}
\item {Grp. gram.:adj.}
\end{itemize}
\begin{itemize}
\item {Utilização:Zool.}
\end{itemize}
\begin{itemize}
\item {Proveniência:(De \textunderscore rubro\textunderscore  + \textunderscore collo\textunderscore )}
\end{itemize}
Diz-se dos animaes, que têm pescoço vermelho.
\section{Rubricolo}
\begin{itemize}
\item {Grp. gram.:adj.}
\end{itemize}
\begin{itemize}
\item {Utilização:Zool.}
\end{itemize}
\begin{itemize}
\item {Proveniência:(De \textunderscore rubro\textunderscore  + \textunderscore colo\textunderscore )}
\end{itemize}
Diz-se dos animaes, que têm pescoço vermelho.
\section{Rubricórneo}
\begin{itemize}
\item {Grp. gram.:adj.}
\end{itemize}
\begin{itemize}
\item {Utilização:Zool.}
\end{itemize}
\begin{itemize}
\item {Proveniência:(De \textunderscore rubro\textunderscore  + \textunderscore córneo\textunderscore )}
\end{itemize}
Que tem antennas vermelhas.
\section{Rubrifloro}
\begin{itemize}
\item {Grp. gram.:adj.}
\end{itemize}
\begin{itemize}
\item {Utilização:Bot.}
\end{itemize}
\begin{itemize}
\item {Proveniência:(De \textunderscore rubro\textunderscore  + \textunderscore flôr\textunderscore )}
\end{itemize}
Que tem flôres vermelhas.
\section{Rubrigastro}
\begin{itemize}
\item {Grp. gram.:adj.}
\end{itemize}
\begin{itemize}
\item {Utilização:Zool.}
\end{itemize}
\begin{itemize}
\item {Proveniência:(Do lat. \textunderscore ruber\textunderscore  + \textunderscore gaster\textunderscore )}
\end{itemize}
Que tem o ventre vermelho.
\section{Rubriloiro}
\begin{itemize}
\item {Grp. gram.:adj.}
\end{itemize}
Que é simultaneamente rubro e loiro:«\textunderscore ...ardendo em rubriloira labareda...\textunderscore »Filinto, III, 261.
\section{Rubrilouro}
\begin{itemize}
\item {Grp. gram.:adj.}
\end{itemize}
Que é simultaneamente rubro e louro:«\textunderscore ...ardendo em rubriloura labareda...\textunderscore »Filinto, III, 261.
\section{Rubrípede}
\begin{itemize}
\item {Grp. gram.:adj.}
\end{itemize}
\begin{itemize}
\item {Utilização:Zool.}
\end{itemize}
\begin{itemize}
\item {Proveniência:(Do lat. \textunderscore ruber\textunderscore  + \textunderscore pes\textunderscore , \textunderscore pedis\textunderscore )}
\end{itemize}
Que tem pés vermelhos.
\section{Rubrirostro}
\begin{itemize}
\item {fónica:rós}
\end{itemize}
\begin{itemize}
\item {Grp. gram.:adj.}
\end{itemize}
\begin{itemize}
\item {Utilização:Zool.}
\end{itemize}
\begin{itemize}
\item {Proveniência:(De \textunderscore rubro\textunderscore  + \textunderscore rostro\textunderscore )}
\end{itemize}
Que tem bico vermelho.
\section{Rubrirrostro}
\begin{itemize}
\item {Grp. gram.:adj.}
\end{itemize}
\begin{itemize}
\item {Utilização:Zool.}
\end{itemize}
\begin{itemize}
\item {Proveniência:(De \textunderscore rubro\textunderscore  + \textunderscore rostro\textunderscore )}
\end{itemize}
Que tem bico vermelho.
\section{Rubro}
\begin{itemize}
\item {Grp. gram.:adj.}
\end{itemize}
\begin{itemize}
\item {Proveniência:(Lat. \textunderscore ruber\textunderscore )}
\end{itemize}
Muito vermelho; afogueado.
\section{Ruçar}
\begin{itemize}
\item {Grp. gram.:v. t.}
\end{itemize}
\begin{itemize}
\item {Grp. gram.:V. i.}
\end{itemize}
Tornar ruço.
Tornar-se ruço.
Envelhecer.
Começar a encanecer.
Enruçar-se.
\section{Ruçar-se}
\begin{itemize}
\item {Grp. gram.:v. p.}
\end{itemize}
\begin{itemize}
\item {Utilização:Pleb.}
\end{itemize}
Mostrar-se alegre por esperar receber ou estar para receber alguma coisa.
Sorrir de contente.--Não seria preferível a fórma \textunderscore roçar-se\textunderscore ?
\section{Rucete}
\begin{itemize}
\item {fónica:cê}
\end{itemize}
\begin{itemize}
\item {Grp. gram.:m.}
\end{itemize}
\begin{itemize}
\item {Proveniência:(De \textunderscore ruço\textunderscore )}
\end{itemize}
Casta de uva.
\section{Rucilho}
\begin{itemize}
\item {Grp. gram.:adj.}
\end{itemize}
\begin{itemize}
\item {Proveniência:(De \textunderscore ruço\textunderscore )}
\end{itemize}
Diz-se do cavallo, que tem pêlos brancos, vermelhos e pretos, misturados.
\section{Ruckéria}
\begin{itemize}
\item {Grp. gram.:f.}
\end{itemize}
\begin{itemize}
\item {Proveniência:(De \textunderscore Rucker\textunderscore , n. p.)}
\end{itemize}
Gênero de plantas, da fam. das compostas.
\section{Ruço}
\begin{itemize}
\item {Grp. gram.:adj.}
\end{itemize}
\begin{itemize}
\item {Utilização:Pop.}
\end{itemize}
\begin{itemize}
\item {Grp. gram.:M.}
\end{itemize}
\begin{itemize}
\item {Utilização:Fam.}
\end{itemize}
Pardacento: \textunderscore mula ruça\textunderscore .
Que tem cabellos brancos e pretos; grisalho.
Desbotado, que perdeu a côr.
Que tem o cabello castanho muito claro.
Cavallo, macho ou burro, de pêlo ruço.
(Cp. cast. \textunderscore rucio\textunderscore )
\section{Ruda}
\begin{itemize}
\item {Grp. gram.:f.}
\end{itemize}
O mesmo que \textunderscore arruda\textunderscore . Cf. Delgado, \textunderscore Flora\textunderscore , 28; \textunderscore Luz e Calor\textunderscore , 71.
\section{Ruda!}
\begin{itemize}
\item {Grp. gram.:interj.}
\end{itemize}
\begin{itemize}
\item {Utilização:Prov.}
\end{itemize}
Ponha-se no olho da rua! Gire! safe-se!
(Imper. irr., em vez de \textunderscore roda\textunderscore , de \textunderscore rodar\textunderscore )
\section{Rudamente}
\begin{itemize}
\item {Grp. gram.:adv.}
\end{itemize}
O mesmo que \textunderscore rudemente\textunderscore .
\section{Rude}
\begin{itemize}
\item {Grp. gram.:adj.}
\end{itemize}
\begin{itemize}
\item {Utilização:Fig.}
\end{itemize}
\begin{itemize}
\item {Proveniência:(Lat. \textunderscore rudis\textunderscore )}
\end{itemize}
Inculto, agreste.
Grosseiro.
Aspero; muito molesto: \textunderscore os rudes combates da vida\textunderscore .
Estúpido.
Desajeitado.
Severo.
Descortês: \textunderscore maneiras rudes\textunderscore .
\section{Rude!}
\begin{itemize}
\item {Grp. gram.:interj.}
\end{itemize}
O mesmo que \textunderscore ruda!\textunderscore !
\section{Rudemente}
\begin{itemize}
\item {Grp. gram.:adv.}
\end{itemize}
De modo rude.
Grosseiramente.
Asperamente, com severidade.
Cruelmente.
\section{Rudentura}
\begin{itemize}
\item {Grp. gram.:f.}
\end{itemize}
\begin{itemize}
\item {Proveniência:(Do lat. \textunderscore rudens\textunderscore , \textunderscore rudentis\textunderscore )}
\end{itemize}
Espécie de vara ou corda, que enche a base das caneluras das columnas.
\section{Rudez}
\begin{itemize}
\item {Grp. gram.:f.}
\end{itemize}
Qualidade do que é rude.
Indelicadeza, maus modos.
Severidade; deshumanidade.
\section{Rudeza}
\begin{itemize}
\item {Grp. gram.:f.}
\end{itemize}
Qualidade do que é rude.
Indelicadeza, maus modos.
Severidade; deshumanidade.
\section{Rudiário}
\begin{itemize}
\item {Grp. gram.:m.}
\end{itemize}
\begin{itemize}
\item {Proveniência:(Lat. \textunderscore rudiarius\textunderscore )}
\end{itemize}
Gladiador aposentado, a quem se entregava, como distinctivo, uma vara tôsca.
\section{Rudimental}
\begin{itemize}
\item {Grp. gram.:adj.}
\end{itemize}
O mesmo que \textunderscore rudimentar\textunderscore . Cf. Herculano, \textunderscore Hist. de Port.\textunderscore , III, 276.
\section{Rudimentar}
\begin{itemize}
\item {Grp. gram.:adj.}
\end{itemize}
Relativo a rudimento.
Que tem o carácter de rudimento: \textunderscore noções rudimentares de Álgebra\textunderscore .
\section{Rudimento}
\begin{itemize}
\item {Grp. gram.:m.}
\end{itemize}
\begin{itemize}
\item {Proveniência:(Lat. \textunderscore rudimentum\textunderscore )}
\end{itemize}
Elemento inicial; início; primeiras noções.
Conhecimento geral (de uma arte ou sciência): \textunderscore rudimentos de Agricultura\textunderscore .
Ensaio.
Miniatura ou primeiros lineamentos do um órgão vegetal ou animal.
Orgão, que ainda se não desenvolveu ou que se desenvolveu pouco.
\section{Rudista}
\begin{itemize}
\item {Grp. gram.:m.}
\end{itemize}
Mollusco fóssil, acéphalo, de concha bivalve e espêssa.
\section{Rudo}
\begin{itemize}
\item {Grp. gram.:adj.}
\end{itemize}
O mesmo que \textunderscore rude\textunderscore ^1; grosseiro:«\textunderscore ...e não de agreste avena ou frauta ruda...\textunderscore »\textunderscore Lusíadas\textunderscore , I, 5.
\section{Rueiro}
\begin{itemize}
\item {Grp. gram.:adj.}
\end{itemize}
Relativo a rua.
Que gosta de andar na rua.
Arruador.
\section{Ruela}
\begin{itemize}
\item {Grp. gram.:f.}
\end{itemize}
Pequena rua; viella.
\section{Ruela}
\begin{itemize}
\item {Grp. gram.:f.}
\end{itemize}
(V.\textunderscore arruela\textunderscore )
\section{Ruer}
\begin{itemize}
\item {Grp. gram.:v. i.}
\end{itemize}
\begin{itemize}
\item {Utilização:Des.}
\end{itemize}
O mesmo que \textunderscore ruir\textunderscore . Cf.\textunderscore Diccion. Homophonol.\textunderscore 
\section{Rufa}
\begin{itemize}
\item {Grp. gram.:f.}
\end{itemize}
\begin{itemize}
\item {Utilização:Ant.}
\end{itemize}
\begin{itemize}
\item {Proveniência:(De \textunderscore rufar\textunderscore ^1?)}
\end{itemize}
Lição, ensinamento? tosa?:«\textunderscore ...como jogador que levou rufa de um que tal\textunderscore ». R. Lobo, \textunderscore Côrte na Aldeia\textunderscore , II, 35.
\section{Rufador}
\begin{itemize}
\item {Grp. gram.:m.  e  adj.}
\end{itemize}
O que rufa.
\section{Rufar}
\begin{itemize}
\item {Grp. gram.:v. t.}
\end{itemize}
\begin{itemize}
\item {Grp. gram.:V. i.}
\end{itemize}
\begin{itemize}
\item {Proveniência:(De \textunderscore rufo\textunderscore ^2)}
\end{itemize}
Tocar, dando rufos.
Tocar rufos.
\section{Rufar}
\begin{itemize}
\item {Grp. gram.:v. t.}
\end{itemize}
\begin{itemize}
\item {Proveniência:(De \textunderscore rufo\textunderscore ^2)}
\end{itemize}
Dar fórma de rufo^2 a.
Fazer rufos^2 ou pregas em.
\section{Rufia}
\begin{itemize}
\item {Grp. gram.:m.}
\end{itemize}
\begin{itemize}
\item {Utilização:pop.}
\end{itemize}
\begin{itemize}
\item {Utilização:Ant.}
\end{itemize}
O mesmo que \textunderscore rufião\textunderscore .
\section{Rufianaço}
\begin{itemize}
\item {Grp. gram.:m.}
\end{itemize}
O mesmo que \textunderscore rufíanaz\textunderscore . Cf. Arn. Gama, \textunderscore Última Dona\textunderscore , 15 e 33.
\section{Rufíanaz}
\begin{itemize}
\item {Grp. gram.:m.  e  adj.}
\end{itemize}
\begin{itemize}
\item {Utilização:Deprec.}
\end{itemize}
\begin{itemize}
\item {Proveniência:(De \textunderscore rufião\textunderscore )}
\end{itemize}
Grande tratante.
Rufião de grande marca.
\section{Rufianesco}
\begin{itemize}
\item {Grp. gram.:adj.}
\end{itemize}
Próprio de rufião; relativo á vida de rufião.
\section{Rufião}
\begin{itemize}
\item {Grp. gram.:m.}
\end{itemize}
Aquelle que briga por causa de mulheres de má nota.
Aquelle que vive á custa de mulheres.
Alcoviteiro.
(Cp. cast. \textunderscore rufián\textunderscore )
\section{Rufiar}
\begin{itemize}
\item {Grp. gram.:v. i.}
\end{itemize}
Têr actos de rufião; têr vida de rufião.
\section{Ruficarpo}
\begin{itemize}
\item {Grp. gram.:adj.}
\end{itemize}
\begin{itemize}
\item {Utilização:Bot.}
\end{itemize}
\begin{itemize}
\item {Proveniência:(Do lat. \textunderscore rufus\textunderscore  + gr. \textunderscore karpos\textunderscore )}
\end{itemize}
Que tem frutos vermelhos.
\section{Ruficórneo}
\begin{itemize}
\item {Grp. gram.:adj.}
\end{itemize}
\begin{itemize}
\item {Utilização:Zool.}
\end{itemize}
\begin{itemize}
\item {Proveniência:(Do lat. \textunderscore rufus\textunderscore  + \textunderscore cornu\textunderscore )}
\end{itemize}
Que tem antenas vermelhas.
\section{Rufigastro}
\begin{itemize}
\item {Grp. gram.:adj.}
\end{itemize}
\begin{itemize}
\item {Utilização:Zool.}
\end{itemize}
\begin{itemize}
\item {Proveniência:(Do lat. \textunderscore rufus\textunderscore  + gr. \textunderscore gaster\textunderscore )}
\end{itemize}
Que tem ventre vermelho.
\section{Rufinérveo}
\begin{itemize}
\item {Grp. gram.:adj.}
\end{itemize}
\begin{itemize}
\item {Utilização:Zool.}
\end{itemize}
\begin{itemize}
\item {Proveniência:(Do lat. \textunderscore rufus\textunderscore  + \textunderscore nervus\textunderscore )}
\end{itemize}
Que tem nervos vermelhos.
\section{Rufi-mórico}
\begin{itemize}
\item {Grp. gram.:adj.}
\end{itemize}
Diz-se de um ácido, extrahido da amoreira.
\section{Rúfio}
(V.\textunderscore rufião\textunderscore )
\section{Rufipalpo}
\begin{itemize}
\item {Grp. gram.:adj.}
\end{itemize}
\begin{itemize}
\item {Utilização:Zool.}
\end{itemize}
\begin{itemize}
\item {Proveniência:(Do lat. \textunderscore rufus\textunderscore  + \textunderscore palpus\textunderscore )}
\end{itemize}
Que tem palpos vermelhos.
\section{Rufista}
\begin{itemize}
\item {Grp. gram.:m.}
\end{itemize}
\begin{itemize}
\item {Proveniência:(De \textunderscore rufo\textunderscore ^1)}
\end{itemize}
Aquelle que rufa.
\section{Rufitarso}
\begin{itemize}
\item {Grp. gram.:adj.}
\end{itemize}
\begin{itemize}
\item {Utilização:Zool.}
\end{itemize}
\begin{itemize}
\item {Proveniência:(De \textunderscore rufo\textunderscore ^3 + \textunderscore tarso\textunderscore )}
\end{itemize}
Que tem tarsos vermelhos.
\section{Ruflar}
\begin{itemize}
\item {Grp. gram.:v. i.}
\end{itemize}
\begin{itemize}
\item {Utilização:Angl}
\end{itemize}
\begin{itemize}
\item {Proveniência:(Do ingl. \textunderscore rufle\textunderscore )}
\end{itemize}
Agitar-se, produzindo rumor, como ave que desprende as asas.
Fazer ruge-ruge, como saias compridas de quem anda, ou como tecido engomado que se dobra ou se amachuca:«\textunderscore as saias ruflavam. Um turbilhão de musselina...\textunderscore »Camillo, \textunderscore Volcoens\textunderscore , 166.
\section{Rufo}
\begin{itemize}
\item {Grp. gram.:m.}
\end{itemize}
\begin{itemize}
\item {Utilização:Ext.}
\end{itemize}
\begin{itemize}
\item {Grp. gram.:Loc. adv.}
\end{itemize}
\begin{itemize}
\item {Proveniência:(Do ingl. \textunderscore rufle\textunderscore )}
\end{itemize}
Som trêmulo, produzido pelo tanger de duas baquetas na pelle tensa de um tambor ou de um instrumento análogo.
Som, quási semelhante ao do tambor, e produzido pelo tanger alternado dos dedos sôbre uma superfície sólida.
\textunderscore Num rufo\textunderscore , promptamente, num instante.
\section{Rufo}
\begin{itemize}
\item {Grp. gram.:m.}
\end{itemize}
\begin{itemize}
\item {Proveniência:(Do ingl. \textunderscore ruff\textunderscore )}
\end{itemize}
Enfeite ou guarnição, constante de pregas ou franzido.
Prega.
\section{Rufo}
\begin{itemize}
\item {Grp. gram.:adj.}
\end{itemize}
\begin{itemize}
\item {Utilização:Poét.}
\end{itemize}
\begin{itemize}
\item {Proveniência:(Lat. \textunderscore rufus\textunderscore )}
\end{itemize}
Ruivo.
Vermelho.
\section{Rufo}
\begin{itemize}
\item {Grp. gram.:m.}
\end{itemize}
Espécie de lima, com serrilha ou picado grosso, quási como o da grosa^2.
(Relaciona-se com \textunderscore rufo\textunderscore ^2?)
\section{Ruga}
\begin{itemize}
\item {Grp. gram.:f.}
\end{itemize}
\begin{itemize}
\item {Proveniência:(Lat. \textunderscore ruga\textunderscore )}
\end{itemize}
Prega ou gelha na pelle; carquilha.
Prega, dobra.
\section{Rugado}
\begin{itemize}
\item {Grp. gram.:adj.}
\end{itemize}
Que apresenta rugas; enrugado. Cf. Júl. Lour. Pinto, \textunderscore Senh. Deput.\textunderscore , 317; Arn. Gama, \textunderscore Segr. do Ab.\textunderscore , 108.
\section{Rugar}
\begin{itemize}
\item {Grp. gram.:v. t.}
\end{itemize}
O mesmo que \textunderscore enrugar\textunderscore .
\section{Rugedor}
\begin{itemize}
\item {Grp. gram.:adj.}
\end{itemize}
O mesmo que \textunderscore rugidor\textunderscore . Cf. Garrett, \textunderscore Camões\textunderscore , 136.
\section{Ruge-ruge}
\begin{itemize}
\item {Grp. gram.:m.}
\end{itemize}
\begin{itemize}
\item {Utilização:Pop.}
\end{itemize}
\begin{itemize}
\item {Utilização:Prov.}
\end{itemize}
\begin{itemize}
\item {Utilização:minh.}
\end{itemize}
\begin{itemize}
\item {Proveniência:(De \textunderscore rugir\textunderscore )}
\end{itemize}
Rumor de saias, que roçam o chão.
Som de um objecto, que vai rojando pelo chão.
Assuada e troça, com latas, ferros velhos, etc, que se faz á porta de quem deixou passar a Quaresma sem se desobrigar.
\section{Rugibó}
\begin{itemize}
\item {Grp. gram.:m.}
\end{itemize}
\begin{itemize}
\item {Utilização:T. de Famalicão}
\end{itemize}
Grande ruído súbito; grande estrondo.
\section{Rugido}
\begin{itemize}
\item {Grp. gram.:m.}
\end{itemize}
\begin{itemize}
\item {Utilização:Fig.}
\end{itemize}
Voz do leão.
Voz prolongada e estridente.
Bramido; som cavernoso.
\section{Rugidor}
\begin{itemize}
\item {Grp. gram.:m.  e  adj.}
\end{itemize}
O que ruge.
\section{Rugiente}
\begin{itemize}
\item {Grp. gram.:adj.}
\end{itemize}
\begin{itemize}
\item {Proveniência:(Lat. \textunderscore rugiens\textunderscore )}
\end{itemize}
Que ruge.
\section{Rugífero}
\begin{itemize}
\item {Grp. gram.:adj.}
\end{itemize}
\begin{itemize}
\item {Utilização:Poét.}
\end{itemize}
\begin{itemize}
\item {Proveniência:(Do lat. \textunderscore ruga\textunderscore  + \textunderscore ferre\textunderscore )}
\end{itemize}
O mesmo que \textunderscore rugoso\textunderscore .
\section{Rúgios}
\begin{itemize}
\item {Grp. gram.:m. pl.}
\end{itemize}
Antigo povo, alliado de Áttila. Cf. D. Ant. da Costa, \textunderscore Três Mundos\textunderscore , 2.^a ed., 242.
\section{Rugir}
\begin{itemize}
\item {Grp. gram.:v. i.}
\end{itemize}
\begin{itemize}
\item {Grp. gram.:M.}
\end{itemize}
\begin{itemize}
\item {Proveniência:(Lat. \textunderscore rugire\textunderscore )}
\end{itemize}
Emittir voz (o leão).
Urrar.
Bramir.
Estrondear.
Resoar.
Sussurrar.
Roçar pelo chão, com leve ruído; sussurrar brandamente.
Rugido.
\section{Rugitar}
\begin{itemize}
\item {Grp. gram.:v. i.}
\end{itemize}
\begin{itemize}
\item {Utilização:bras}
\end{itemize}
\begin{itemize}
\item {Utilização:Neol.}
\end{itemize}
\begin{itemize}
\item {Proveniência:(Do lat. \textunderscore rugitus\textunderscore )}
\end{itemize}
Fazer ruído; rugir. Cf. Alencar, Iracema.
\section{Rugosa}
\begin{itemize}
\item {Grp. gram.:f.}
\end{itemize}
Mollusco acéphalo.
(Fem. de \textunderscore rugoso\textunderscore )
\section{Rugosidade}
\begin{itemize}
\item {Grp. gram.:f.}
\end{itemize}
\begin{itemize}
\item {Proveniência:(Do lat. \textunderscore rugositas\textunderscore )}
\end{itemize}
Qualidade de rugoso.
\section{Rugoso}
\begin{itemize}
\item {Grp. gram.:adj.}
\end{itemize}
\begin{itemize}
\item {Grp. gram.:M.}
\end{itemize}
\begin{itemize}
\item {Proveniência:(Lat. \textunderscore rugosus\textunderscore )}
\end{itemize}
Que tem rugas; encarquilhado; engelhado: \textunderscore cara rugosa\textunderscore .
Órgão das escorvas de artilharia, destinado a provocar inflammação pelo attrito que determina.
\section{Ruh!}
\begin{itemize}
\item {Grp. gram.:interj.}
\end{itemize}
(indicativa do ruído de árvore frondosa, ao cair):«\textunderscore eis senão quando ruh!\textunderscore ». J. de Deus.
\section{Ruidar}
\begin{itemize}
\item {fónica:ru-i}
\end{itemize}
\begin{itemize}
\item {Grp. gram.:v. i.}
\end{itemize}
Produzir ruído:«\textunderscore ruidavam as Marmáridas catervas\textunderscore ». Filinto, VI, 233.
\section{Ruído}
\begin{itemize}
\item {Grp. gram.:m.}
\end{itemize}
\begin{itemize}
\item {Utilização:Fig.}
\end{itemize}
Rumor, produzido pela quéda de um corpo.
Rumor.
Qualquer estrondo.
Fragor.
Bulício.
Boato.
Renome; fama.
Ostentação, pompa: \textunderscore viver com ruído\textunderscore .
(Da mesma or. de \textunderscore rugido\textunderscore , se se explicar a quéda do \textunderscore g\textunderscore  medial)
\section{Ruidosamente}
\begin{itemize}
\item {fónica:ru-i}
\end{itemize}
\begin{itemize}
\item {Grp. gram.:adv.}
\end{itemize}
De modo ruidoso; com ruído; com estrondo.
Com pompa.
\section{Ruidoso}
\begin{itemize}
\item {fónica:ru-i}
\end{itemize}
\begin{itemize}
\item {Grp. gram.:adj.}
\end{itemize}
Que faz ruído.
Acompanhado de ruído.
Que faz sensação; pomposo.
\section{Ruím}
\begin{itemize}
\item {Grp. gram.:adj.}
\end{itemize}
\begin{itemize}
\item {Utilização:Prov.}
\end{itemize}
\begin{itemize}
\item {Utilização:alg.}
\end{itemize}
\begin{itemize}
\item {Proveniência:(De \textunderscore ruína\textunderscore , segundo uns; do hebr., segundo outros.)}
\end{itemize}
Mau; nocivo.
Inútil.
Que tem má índole.
Perverso.
Deteriorado; estragado.
Hydróphobo, damnado.
\section{Ruimmente}
\begin{itemize}
\item {fónica:ru-im}
\end{itemize}
\begin{itemize}
\item {Grp. gram.:adv.}
\end{itemize}
De modo ruím; com maldade; com perversidade.
\section{Ruína}
\begin{itemize}
\item {Grp. gram.:f.}
\end{itemize}
\begin{itemize}
\item {Utilização:Ext.}
\end{itemize}
\begin{itemize}
\item {Proveniência:(Lat. \textunderscore ruina\textunderscore )}
\end{itemize}
Acto ou effeito de ruir.
Resto de edifício desmoronado.
Perda; destruição, dissipação.
Causa de males ou de destruição.
Vestígio; reflexo.
\section{Ruinar}
\begin{itemize}
\item {fónica:ru-i}
\end{itemize}
\textunderscore v. t.\textunderscore  (e der.)
O mesmo que \textunderscore arruinar\textunderscore , etc. Cf. Filinto, \textunderscore D. Man.\textunderscore , I, 171.
\section{Ruinaria}
\begin{itemize}
\item {fónica:ru-i}
\end{itemize}
\begin{itemize}
\item {Grp. gram.:f.}
\end{itemize}
Conjunto de ruínas.
Restos de edifício, que o tempo ou alguém desmoronou. Cf. Camillo, \textunderscore Cancion. Aleg.\textunderscore , 363.
\section{Ruindade}
\begin{itemize}
\item {fónica:ru-in}
\end{itemize}
\begin{itemize}
\item {Grp. gram.:f.}
\end{itemize}
Qualidade do que é ruím.
\section{Ruinosamente}
\begin{itemize}
\item {fónica:ru-i}
\end{itemize}
\begin{itemize}
\item {Grp. gram.:adv.}
\end{itemize}
De modo ruinoso; desastradamente; desgraçadamente.
\section{Ruinoso}
\begin{itemize}
\item {fónica:ru-i}
\end{itemize}
\begin{itemize}
\item {Grp. gram.:adj.}
\end{itemize}
\begin{itemize}
\item {Proveniência:(Lat. \textunderscore ruinosus\textunderscore )}
\end{itemize}
Que está em ruína; que ameaça ruína.
Que produz ruína.
Que faz mal; nocivo.
\section{Ruiponto}
\begin{itemize}
\item {Grp. gram.:m.}
\end{itemize}
\begin{itemize}
\item {Utilização:Ant.}
\end{itemize}
O mesmo que \textunderscore raponço\textunderscore .
\section{Ruir}
\begin{itemize}
\item {Grp. gram.:v. t.}
\end{itemize}
\begin{itemize}
\item {Proveniência:(Lat. \textunderscore ruere\textunderscore )}
\end{itemize}
Cair com ímpeto e rapidamente.
Despenhar-se.
Desmoronar-se.
Correr muito.
\section{Ruiva}
\begin{itemize}
\item {Grp. gram.:f.}
\end{itemize}
\begin{itemize}
\item {Utilização:Gír.}
\end{itemize}
Nome de várias plantas rubiáceas, (\textunderscore rubia\textunderscore ).
Espécie de tordo.
Ave, o mesmo que \textunderscore seixoeira\textunderscore .
O mesmo que \textunderscore polícia\textunderscore .
\section{Ruiva}
\begin{itemize}
\item {Grp. gram.:f.}
\end{itemize}
\begin{itemize}
\item {Proveniência:(De \textunderscore ruivo\textunderscore )}
\end{itemize}
Mulhér, que tem cabello ruivo ou loiro avermelhado.
\section{Ruiva-brava}
\begin{itemize}
\item {Grp. gram.:f.}
\end{itemize}
Espécie de ruiva^1, o mesmo que \textunderscore raspa-língua\textunderscore .
\section{Ruivaca}
\begin{itemize}
\item {Grp. gram.:f.}
\end{itemize}
\begin{itemize}
\item {Proveniência:(De \textunderscore ruivo\textunderscore )}
\end{itemize}
O mesmo que \textunderscore pimpão\textunderscore , peixe.
\section{Ruivacento}
\begin{itemize}
\item {Grp. gram.:adj.}
\end{itemize}
\begin{itemize}
\item {Utilização:Neol.}
\end{itemize}
Um tanto ruivo.
\section{Ruivaco}
\begin{itemize}
\item {Grp. gram.:m.}
\end{itemize}
\begin{itemize}
\item {Utilização:T. da Bairrada}
\end{itemize}
O mesmo que \textunderscore ruivaca\textunderscore .
\section{Ruiva-da-índia}
\begin{itemize}
\item {Grp. gram.:f.}
\end{itemize}
Planta rubiácea, (\textunderscore rubia cordifolia\textunderscore , Lin.), conhecida no commércio por \textunderscore mangista\textunderscore , e cuja raíz é aproveitada em tinturaria.
\section{Ruiva-dos-tintureiros}
\begin{itemize}
\item {Grp. gram.:f.}
\end{itemize}
O mesmo que \textunderscore granza\textunderscore , planta rubiácea, (\textunderscore rubia tinctorum\textunderscore , Lin.).
\section{Ruivaes}
\begin{itemize}
\item {Grp. gram.:f.}
\end{itemize}
Variedade de pêra portuguesa, hoje desconhecida.
(Cp.\textunderscore ruival\textunderscore )
\section{Ruiva-indiana}
\begin{itemize}
\item {Grp. gram.:f.}
\end{itemize}
Planta rubiácea, (\textunderscore oldelandia umbellata\textunderscore , Lin.), cuja raíz é usada pelos tintureiros, como a de outras rubiáceas.
\section{Ruivais}
\begin{itemize}
\item {Grp. gram.:f.}
\end{itemize}
Variedade de pêra portuguesa, hoje desconhecida.
(Cp. \textunderscore ruival\textunderscore )
\section{Ruival}
\begin{itemize}
\item {Grp. gram.:adj. f.}
\end{itemize}
\begin{itemize}
\item {Utilização:Prov.}
\end{itemize}
\begin{itemize}
\item {Utilização:beir.}
\end{itemize}
Diz-se de uma variedade de pêra, muito apreciada para se secar. (Colhido em Oliv. do Hospital)
(Talvez de \textunderscore ruivo\textunderscore )
\section{Ruividão}
\begin{itemize}
\item {Grp. gram.:f.}
\end{itemize}
\begin{itemize}
\item {Utilização:Des.}
\end{itemize}
Qualidade do que é ruivo.
\section{Ruivinha}
\begin{itemize}
\item {Grp. gram.:f.}
\end{itemize}
\begin{itemize}
\item {Proveniência:(De \textunderscore ruiva\textunderscore ^1)}
\end{itemize}
Arbusto rubiáceo do Brasil.
\section{Ruivo}
\begin{itemize}
\item {Grp. gram.:adj.}
\end{itemize}
\begin{itemize}
\item {Grp. gram.:M.}
\end{itemize}
\begin{itemize}
\item {Proveniência:(Do lat. \textunderscore rubeus\textunderscore )}
\end{itemize}
Amarelo avermelhado.
Vermelho escuro.
Loiro avermelhado.
Indivíduo que tem o cabello ruivo.
Peixe acanthopterýgio.
\section{Ruivó}
\begin{itemize}
\item {Grp. gram.:m.}
\end{itemize}
\begin{itemize}
\item {Utilização:Prov.}
\end{itemize}
\begin{itemize}
\item {Utilização:beir.}
\end{itemize}
\begin{itemize}
\item {Proveniência:(Do lat. hyp.\textunderscore rubeolus\textunderscore , dem. do lat. \textunderscore rubeus\textunderscore ?)}
\end{itemize}
Espécie de tortulho, de folíolos vermelhos e cabeça branca na parte convexa.
\section{Ruízia}
\begin{itemize}
\item {Grp. gram.:f.}
\end{itemize}
\begin{itemize}
\item {Proveniência:(De \textunderscore Ruíz\textunderscore , n. p.)}
\end{itemize}
Gênero de plantas bytneriáceas.
\section{Rula-mala}
\begin{itemize}
\item {Grp. gram.:f.}
\end{itemize}
Árvore africana, monocotyledónea, talvez da família das aráceas.
\section{Rulhador}
\begin{itemize}
\item {Grp. gram.:m.  e  adj.}
\end{itemize}
\begin{itemize}
\item {Utilização:T. de Trancoso}
\end{itemize}
Indivíduo intriguista.
(Por \textunderscore enrolador\textunderscore , de \textunderscore enrolar\textunderscore ?)
\section{Rulíngia}
\begin{itemize}
\item {Grp. gram.:f.}
\end{itemize}
Gênero de plantas bytneriáceas.
\section{Rulo}
\begin{itemize}
\item {Grp. gram.:m.}
\end{itemize}
O mesmo que \textunderscore arrulho\textunderscore . Cf.\textunderscore Luz e Calor\textunderscore , 538.
\section{Rum}
\begin{itemize}
\item {Grp. gram.:m.}
\end{itemize}
\begin{itemize}
\item {Proveniência:(Do ingl. \textunderscore rum\textunderscore )}
\end{itemize}
Alcool, proveniente de destillação de melaço.
\section{Ruma}
\begin{itemize}
\item {Grp. gram.:f.}
\end{itemize}
O mesmo que \textunderscore rima\textunderscore ^2.
\section{Ruma!}
\begin{itemize}
\item {Grp. gram.:interj.}
\end{itemize}
\begin{itemize}
\item {Utilização:Bras}
\end{itemize}
Voz, que os carreiros dirigem aos bois, para os governar.
(Por \textunderscore arruma\textunderscore , de \textunderscore arrumar\textunderscore ?)
\section{Rumar}
\begin{itemize}
\item {Grp. gram.:v. t.}
\end{itemize}
Pôr em rumo (uma embarcação).
\section{Rumbo}
\begin{itemize}
\item {Grp. gram.:m.}
\end{itemize}
\begin{itemize}
\item {Utilização:Pop.}
\end{itemize}
O mesmo que \textunderscore rumo\textunderscore .
\section{Rumbodo}
\begin{itemize}
\item {Grp. gram.:m.}
\end{itemize}
Árvore urticácea da Índia, (\textunderscore ficus glomerata\textunderscore ).
O mesmo que \textunderscore rumbor\textunderscore ?
\section{Rumbor}
\begin{itemize}
\item {Grp. gram.:m.}
\end{itemize}
Árvore da India portuguesa.
\section{Rumeliota}
\begin{itemize}
\item {Grp. gram.:m.  e  adj.}
\end{itemize}
\begin{itemize}
\item {Proveniência:(De \textunderscore Rumélia\textunderscore  ou \textunderscore Romélià\textunderscore , n. p.)}
\end{itemize}
O mesmo que \textunderscore romeliota\textunderscore .
\section{Rúmen}
\begin{itemize}
\item {Grp. gram.:m.}
\end{itemize}
\begin{itemize}
\item {Proveniência:(Lat. \textunderscore rumen\textunderscore )}
\end{itemize}
Pança ou primeira cavidade do estômago dos animaes, na qual o esóphago despeja os alimentos antes de remoídos; o mesmo que \textunderscore ruminadoiro\textunderscore .
\section{Rumes}
\begin{itemize}
\item {Grp. gram.:m. pl.}
\end{itemize}
Nome que se deu na Índia aos soldados turcos ou egýpcios, filhos de christãos, mas subtrahidos, quando crianças, a seus pais, e doutrinados no Mahometismo e adestrados na arte da guerra; Mamelucos. Cf. Filinto, \textunderscore Vida de Man.\textunderscore , II, p. 41.
(Relaciona-se com romanos:«\textunderscore ...Rumes, que trazido de Roma o nome tem.\textunderscore »\textunderscore Lusíadas\textunderscore , X, 68)
\section{Rumiar}
\textunderscore v. t.\textunderscore  e \textunderscore i.\textunderscore  (e der.)
O mesmo que \textunderscore ruminar\textunderscore . Cf. Vieira, VI, 300.
\section{Rumina}
\begin{itemize}
\item {Grp. gram.:f.}
\end{itemize}
Gênero de insectos coleópteros.
\section{Ruminação}
\begin{itemize}
\item {Grp. gram.:adj.}
\end{itemize}
\begin{itemize}
\item {Proveniência:(Lat. \textunderscore ruminatio\textunderscore )}
\end{itemize}
Acto ou effeito de ruminar.
\section{Ruminadoiro}
\begin{itemize}
\item {Grp. gram.:m.}
\end{itemize}
\begin{itemize}
\item {Proveniência:(De \textunderscore ruminar\textunderscore )}
\end{itemize}
Estômago dos ruminantes, no qual se guarda a comida que hão de tornar a mastigar.
\section{Ruminador}
\begin{itemize}
\item {Grp. gram.:adj.}
\end{itemize}
Que rumina.
\section{Ruminadouro}
\begin{itemize}
\item {Grp. gram.:m.}
\end{itemize}
\begin{itemize}
\item {Proveniência:(De \textunderscore ruminar\textunderscore )}
\end{itemize}
Estômago dos ruminantes, no qual se guarda a comida que hão de tornar a mastigar.
\section{Ruminante}
\begin{itemize}
\item {Grp. gram.:adj.}
\end{itemize}
\begin{itemize}
\item {Grp. gram.:M.}
\end{itemize}
\begin{itemize}
\item {Proveniência:(Lat. \textunderscore ruminans\textunderscore )}
\end{itemize}
Que rumina.
Animal mammífero e quadrúpede, que tem a propriedade de ruminar.
\section{Ruminar}
\begin{itemize}
\item {Grp. gram.:v. t.}
\end{itemize}
\begin{itemize}
\item {Utilização:Fig.}
\end{itemize}
\begin{itemize}
\item {Grp. gram.:V. i.}
\end{itemize}
\begin{itemize}
\item {Utilização:Fig.}
\end{itemize}
\begin{itemize}
\item {Proveniência:(Lat. \textunderscore ruminare\textunderscore )}
\end{itemize}
Tornar a mastigar.
Remoer (os alimentos, que voltam do estômago á boca).
Pensar muito em.
Remascar os alimentos.
Cogitar profundamente.
\section{Rumo}
\begin{itemize}
\item {Grp. gram.:m.}
\end{itemize}
Cada um dos pontos ou linhas, que formam a rosa dos ventos.
Direcção do navio por alguma dessas linhas.
Direcção.
Caminho.
Systema, norma.
Medida antiga de marinha.
(Do holl.\textunderscore ruim\textunderscore ?)
\section{Rumor}
\begin{itemize}
\item {Grp. gram.:m.}
\end{itemize}
\begin{itemize}
\item {Grp. gram.:Loc.}
\end{itemize}
\begin{itemize}
\item {Utilização:pop.}
\end{itemize}
\begin{itemize}
\item {Proveniência:(Lat. \textunderscore rumor\textunderscore )}
\end{itemize}
Murmúrio ou ruído, produzido por coisas que se deslocam.
Murmúrio de vozes.
Sussurro.
Fama.
Boato.
\textunderscore Rumor de lua\textunderscore , mudança de lunação, com influência na saúde: \textunderscore queixa-se do reumatismo; provavelmente é rumor de lua\textunderscore .
\section{Rumorejante}
\begin{itemize}
\item {Grp. gram.:adj.}
\end{itemize}
Que rumoreja.
\section{Rumorejar}
\begin{itemize}
\item {Grp. gram.:v. i.}
\end{itemize}
\begin{itemize}
\item {Grp. gram.:V. p.}
\end{itemize}
Produzir rumor.
Sussurrar brandamente.
Correr (um boato); constar.
Dizer-se, á boca pequena ou em segrêdo.
Correr o boato de que: \textunderscore rumoreja-se que o Nobre perde a eleição\textunderscore .
\section{Rumorejo}
\begin{itemize}
\item {Grp. gram.:m.}
\end{itemize}
Acto ou effeito de rumorejar.
\section{Rumorinho}
\begin{itemize}
\item {Grp. gram.:m.}
\end{itemize}
Pequeno rumor, pequeno ruído. Cf. Camillo, \textunderscore Doze Casam.\textunderscore , 14.
\section{Rumoroso}
\begin{itemize}
\item {Grp. gram.:adj.}
\end{itemize}
Que produz rumor.
Em que há rumor.
Ruidoso:«\textunderscore ...na rumorosa cidade há casas, há quartos...\textunderscore »Th. Ribeiro, \textunderscore Jornadas\textunderscore , I, 62.
\section{Rum-rum}
\begin{itemize}
\item {Grp. gram.:m.}
\end{itemize}
(V.\textunderscore zum-zum\textunderscore )
\section{Runa}
\begin{itemize}
\item {Utilização:T. de Coimbra}
\end{itemize}
Valla; barranco. Cf. Garrett, \textunderscore Fábulas\textunderscore , 264 e 269.
\section{Runa}
\begin{itemize}
\item {Grp. gram.:f.}
\end{itemize}
Seiva de pinheiro.
\section{Runas}
\begin{itemize}
\item {Grp. gram.:f. pl.}
\end{itemize}
Caracteres, de que se serviam alguns povos, especialmente os Escandinavos, e que se gravavam em vasos de madeira e em rochedos.
(Do irlandês \textunderscore run\textunderscore )
\section{Rundo}
\begin{itemize}
\item {Grp. gram.:m.}
\end{itemize}
\begin{itemize}
\item {Utilização:T. da Áfr. Or}
\end{itemize}
Espécie de batuque.
\section{Runfão}
\begin{itemize}
\item {Grp. gram.:m.}
\end{itemize}
\begin{itemize}
\item {Utilização:Pop.}
\end{itemize}
Aquelle que amua ou tem maus modos, com qualquer pequena contrariedade.
(Por \textunderscore arrufão\textunderscore , de \textunderscore arrufar\textunderscore )
\section{Runhar}
\begin{itemize}
\item {Grp. gram.:v. t.}
\end{itemize}
\begin{itemize}
\item {Utilização:Des.}
\end{itemize}
O mesmo que \textunderscore javrar\textunderscore .
(Cp. cast. \textunderscore ruñar\textunderscore )
\section{Rúnico}
\begin{itemize}
\item {Grp. gram.:adj.}
\end{itemize}
Relativo ás runas.
Escrito em runas. Cf. Garret, \textunderscore D. Branca\textunderscore , 62.
\section{Runografia}
\begin{itemize}
\item {Grp. gram.:f.}
\end{itemize}
Tratado dos carácteres rúnicos.
\section{Runográfico}
\begin{itemize}
\item {Grp. gram.:m.}
\end{itemize}
\begin{itemize}
\item {Proveniência:(De \textunderscore runas\textunderscore  + gr. \textunderscore graphein\textunderscore )}
\end{itemize}
Aquele que escreve á cêrca de runografia.
\section{Runógrafo}
\begin{itemize}
\item {Grp. gram.:adj.}
\end{itemize}
Relativo á runografia.
\section{Runographia}
\begin{itemize}
\item {Grp. gram.:f.}
\end{itemize}
Tratado dos carácteres rúnicos.
\section{Runográphico}
\begin{itemize}
\item {Grp. gram.:m.}
\end{itemize}
\begin{itemize}
\item {Proveniência:(De \textunderscore runas\textunderscore  + gr. \textunderscore graphein\textunderscore )}
\end{itemize}
Aquelle que escreve á cêrca de runographia.
\section{Runógrapho}
\begin{itemize}
\item {Grp. gram.:adj.}
\end{itemize}
Relativo á runographia.
\section{Rupar}
\begin{itemize}
\item {Grp. gram.:v. i.}
\end{itemize}
\begin{itemize}
\item {Utilização:Prov.}
\end{itemize}
\begin{itemize}
\item {Utilização:trasm.}
\end{itemize}
Diz-se do cão que ladra, ou que investe, ladrando.
\section{Rupélia}
\begin{itemize}
\item {Grp. gram.:f.}
\end{itemize}
Gênero de insectos dípteros.
\section{Rupestre}
\begin{itemize}
\item {Grp. gram.:adj.}
\end{itemize}
\begin{itemize}
\item {Proveniência:(Do lat. \textunderscore rupes\textunderscore )}
\end{itemize}
Que cresce sôbre os rochedos, (falando-se especialmente de uma casta de videiras).
\section{Rúpia}
\begin{itemize}
\item {Grp. gram.:f.}
\end{itemize}
\begin{itemize}
\item {Utilização:Med.}
\end{itemize}
\begin{itemize}
\item {Proveniência:(Do gr. \textunderscore rupos\textunderscore )}
\end{itemize}
Inflammação da pelle, caracterizada por pequenas bolhas, que se convertem em úlceras.
\section{Rupía}
\begin{itemize}
\item {Grp. gram.:f.}
\end{itemize}
\begin{itemize}
\item {Utilização:Med.}
\end{itemize}
\begin{itemize}
\item {Proveniência:(Do gr. \textunderscore rupos\textunderscore )}
\end{itemize}
Inflammação da pelle, caracterizada por pequenas bolhas, que se convertem em úlceras.
\section{Rupía}
\begin{itemize}
\item {Grp. gram.:f.}
\end{itemize}
\begin{itemize}
\item {Proveniência:(T. indostan)}
\end{itemize}
Moéda da Índia Portuguesa, equivalente a 450 reis.
\section{Rupícola}
\begin{itemize}
\item {Grp. gram.:adj.}
\end{itemize}
\begin{itemize}
\item {Proveniência:(Do lat. \textunderscore rupes\textunderscore  + \textunderscore colere\textunderscore )}
\end{itemize}
Que vive nas rochas.
\section{Rupim}
\begin{itemize}
\item {Grp. gram.:m.  e  adj.}
\end{itemize}
\begin{itemize}
\item {Utilização:Gír.}
\end{itemize}
Rico.
(Or. ind. Cp. \textunderscore rupía\textunderscore ^2)
\section{Rúppia}
\begin{itemize}
\item {Grp. gram.:f.}
\end{itemize}
\begin{itemize}
\item {Proveniência:(De \textunderscore Ruppius\textunderscore , n. p.)}
\end{itemize}
Gênero de plantas aquáticas.
\section{Rúptil}
\begin{itemize}
\item {Grp. gram.:adj.}
\end{itemize}
\begin{itemize}
\item {Proveniência:(Do lat. \textunderscore ruptus\textunderscore )}
\end{itemize}
Que se póde romper; quebradiço.
\section{Ruptilidade}
\begin{itemize}
\item {Grp. gram.:f.}
\end{itemize}
Qualidade do que é rúptil.
\section{Ruptório}
\begin{itemize}
\item {Grp. gram.:m.}
\end{itemize}
\begin{itemize}
\item {Proveniência:(Do lat. \textunderscore ruptus\textunderscore )}
\end{itemize}
Instrumento cirúrgico, para abrir fontanelas.
\section{Ruptura}
\begin{itemize}
\item {Grp. gram.:f.}
\end{itemize}
\begin{itemize}
\item {Utilização:Med.}
\end{itemize}
\begin{itemize}
\item {Proveniência:(Lat. \textunderscore ruptura\textunderscore )}
\end{itemize}
Acto ou effeito de romper.
Interrupção.
Violação de contrato ou acôrdo.
Quebra ou córte de relações sociaes.
Hérnia; fractura.
\section{Rhópala}
\begin{itemize}
\item {Grp. gram.:f.}
\end{itemize}
\begin{itemize}
\item {Proveniência:(Do gr. \textunderscore rhopalon\textunderscore )}
\end{itemize}
Gênero de plantas proteáceas da América tropical.
\section{Rhopalómera}
\begin{itemize}
\item {Grp. gram.:f.}
\end{itemize}
\begin{itemize}
\item {Proveniência:(Do gr. \textunderscore rhopalon\textunderscore  + \textunderscore meros\textunderscore )}
\end{itemize}
Gênero de insectos dípteros.
\section{Rincantera}
\begin{itemize}
\item {Grp. gram.:f.}
\end{itemize}
Gênero de plantas melastomáceas.
\section{Rincobdela}
\begin{itemize}
\item {Grp. gram.:f.}
\end{itemize}
Gênero de peixes acantopterígios.
\section{Rincocarpo}
\begin{itemize}
\item {Grp. gram.:f.}
\end{itemize}
\begin{itemize}
\item {Proveniência:(Do gr. \textunderscore rhunkhos\textunderscore  + \textunderscore karpos\textunderscore )}
\end{itemize}
Gênero de plantas cucurbitáceas.
\section{Rincocéleos}
\begin{itemize}
\item {Grp. gram.:m. pl.}
\end{itemize}
\begin{itemize}
\item {Utilização:Zool.}
\end{itemize}
\begin{itemize}
\item {Proveniência:(Do gr. \textunderscore rhunkhos\textunderscore  + \textunderscore koilon\textunderscore )}
\end{itemize}
Ordem de vermes.
\section{Rincocéfalo}
\begin{itemize}
\item {Grp. gram.:adj.}
\end{itemize}
\begin{itemize}
\item {Utilização:Zool.}
\end{itemize}
\begin{itemize}
\item {Proveniência:(Do gr. \textunderscore rhunkhos\textunderscore  + \textunderscore kephale\textunderscore )}
\end{itemize}
Que tem cabeça prolongado em fórma de bico.
\section{Ríncodo}
\begin{itemize}
\item {Grp. gram.:m.}
\end{itemize}
\begin{itemize}
\item {Proveniência:(Do gr. \textunderscore rhunkhos\textunderscore  + \textunderscore eidos\textunderscore )}
\end{itemize}
Gênero de insectos coleópteros tetrâmeros.
\section{Rincóforo}
\begin{itemize}
\item {Grp. gram.:adj.}
\end{itemize}
\begin{itemize}
\item {Utilização:Zool.}
\end{itemize}
\begin{itemize}
\item {Grp. gram.:M. pl.}
\end{itemize}
\begin{itemize}
\item {Proveniência:(Do gr. \textunderscore rhunkhos\textunderscore  + \textunderscore phoros\textunderscore )}
\end{itemize}
Que tem bico, ou um bico grande.
Insectos rincópteros.
\section{Rincoglosso}
\begin{itemize}
\item {Grp. gram.:m.}
\end{itemize}
\begin{itemize}
\item {Proveniência:(Do gr. \textunderscore rhunkhos\textunderscore  + \textunderscore glossa\textunderscore )}
\end{itemize}
Gênero de plantas escrofularíneas.
\section{Rincósia}
\begin{itemize}
\item {Grp. gram.:f.}
\end{itemize}
Gênero de plantas leguminosas.
\section{Rincospermo}
\begin{itemize}
\item {Grp. gram.:m.}
\end{itemize}
Gênero de plantas leguminosas.
\section{Rincóspora}
\begin{itemize}
\item {Grp. gram.:f.}
\end{itemize}
\begin{itemize}
\item {Proveniência:(Do gr. \textunderscore rhunkhos\textunderscore  + \textunderscore spora\textunderscore )}
\end{itemize}
Gênero de plantas ciperáceas.
\section{Rincoteca}
\begin{itemize}
\item {Grp. gram.:f.}
\end{itemize}
\begin{itemize}
\item {Proveniência:(Do gr. \textunderscore rhunkhos\textunderscore  + \textunderscore theke\textunderscore )}
\end{itemize}
Gênero de plantas do Peru.
\section{Ritão}
\begin{itemize}
\item {Grp. gram.:m.}
\end{itemize}
\begin{itemize}
\item {Proveniência:(Gr. \textunderscore ruthon\textunderscore )}
\end{itemize}
Vaso, em que os Gregos bebiam vinho e que, primitivamente, tinha a forma de chavelho.
\section{Ritelminto}
\begin{itemize}
\item {Grp. gram.:m.}
\end{itemize}
\begin{itemize}
\item {Proveniência:(Do gr. \textunderscore rhutis\textunderscore  + \textunderscore kelmins\textunderscore )}
\end{itemize}
Gênero de vermes intestinaes.
\section{Ritidósia}
\begin{itemize}
\item {Grp. gram.:f.}
\end{itemize}
\begin{itemize}
\item {Proveniência:(Do gr. \textunderscore rutis\textunderscore , \textunderscore rutidos\textunderscore )}
\end{itemize}
Gênero de plantas australianas.
\section{Ritmado}
\begin{itemize}
\item {Grp. gram.:adj.}
\end{itemize}
Que tem ritmo.
\section{Rítmica}
\begin{itemize}
\item {Grp. gram.:f.}
\end{itemize}
Parte da antiga gramática, que se ocupava do ritmo dos versos gregos e latinos.
(Fem. de \textunderscore rítmico\textunderscore )
\section{Rítmico}
\begin{itemize}
\item {Grp. gram.:adj.}
\end{itemize}
\begin{itemize}
\item {Proveniência:(Lat. \textunderscore rythmicus\textunderscore )}
\end{itemize}
Relativo ao ritmo.
\section{Ritmo}
\begin{itemize}
\item {Grp. gram.:m.}
\end{itemize}
\begin{itemize}
\item {Utilização:Med.}
\end{itemize}
\begin{itemize}
\item {Proveniência:(Lat. \textunderscore rhytmus\textunderscore )}
\end{itemize}
Sucessão, com intervalos regulares, de sílabas accentuadas ou acentos prosódicos, que impressiona agradavelmente o ouvido.
Cadência.
Proporção entre as pulsações das artérias.
\section{Romborina}
\begin{itemize}
\item {Grp. gram.:f.}
\end{itemize}
Gênero de insectos coleópteros pentâmeros.
\section{Rombósporo}
\begin{itemize}
\item {Grp. gram.:adj.}
\end{itemize}
\begin{itemize}
\item {Utilização:Bot.}
\end{itemize}
\begin{itemize}
\item {Proveniência:(Do gr. \textunderscore rhombos\textunderscore  + \textunderscore spora\textunderscore )}
\end{itemize}
Que tem sementes romboidaes.
\section{Ropalócero}
\begin{itemize}
\item {Grp. gram.:adj.}
\end{itemize}
\begin{itemize}
\item {Proveniência:(Do gr. \textunderscore rhopalon\textunderscore  + \textunderscore keras\textunderscore )}
\end{itemize}
Diz-se dos insectos, que tem as antennas terminadas em maça, ou botão.
\section{Ropalose}
\begin{itemize}
\item {Grp. gram.:f.}
\end{itemize}
\begin{itemize}
\item {Utilização:Med.}
\end{itemize}
\begin{itemize}
\item {Proveniência:(Do gr. \textunderscore rhopalon\textunderscore )}
\end{itemize}
Moléstia, em que engrossa a extremidade dos cabelos.
\section{Ropografia}
\begin{itemize}
\item {Grp. gram.:f.}
\end{itemize}
\begin{itemize}
\item {Proveniência:(Gr. \textunderscore rhopographia\textunderscore )}
\end{itemize}
Descripção de arvoredos ou de pequenas paisagens.
\section{Ropográfico}
\begin{itemize}
\item {Grp. gram.:adj.}
\end{itemize}
Relativo á ropografia.
\section{Ropógrafo}
\begin{itemize}
\item {Grp. gram.:m.}
\end{itemize}
\begin{itemize}
\item {Proveniência:(Do gr. \textunderscore rhopos\textunderscore  + \textunderscore graphein\textunderscore )}
\end{itemize}
Aquele que descreve arvoredos ou pequenas paisagens.
\section{Rotacismo}
\begin{itemize}
\item {Grp. gram.:m.}
\end{itemize}
\begin{itemize}
\item {Proveniência:(Do gr. \textunderscore rhotakizein\textunderscore )}
\end{itemize}
Pronúncia viciosa da letra \textunderscore r\textunderscore .
\section{Ruibarbo}
\begin{itemize}
\item {Grp. gram.:m.}
\end{itemize}
Gênero de plantas poligóneas.
\section{Ruquibo}
\begin{itemize}
\item {Grp. gram.:m.}
\end{itemize}
Espécie de escudo, na Lunda.
\section{Rural}
\begin{itemize}
\item {Grp. gram.:adj.}
\end{itemize}
\begin{itemize}
\item {Proveniência:(Lat. \textunderscore ruralis\textunderscore )}
\end{itemize}
Relativo ao campo.
Que está no campo ou é próprio delle.
Agrícola; campesino.
Que não é da cidade: \textunderscore as escolas ruraes\textunderscore .
\section{Ruralismo}
\begin{itemize}
\item {Grp. gram.:m.}
\end{itemize}
\begin{itemize}
\item {Proveniência:(De \textunderscore rural\textunderscore )}
\end{itemize}
Emprêgo de scenas ruraes em obras de arte.
\section{Ruralista}
\begin{itemize}
\item {Grp. gram.:adj.}
\end{itemize}
\begin{itemize}
\item {Proveniência:(De \textunderscore rural\textunderscore )}
\end{itemize}
Diz-se do pintor e de outros artistas, que nos seus trabalhos preferem as scenas ruraes.
\section{Ruralmente}
\begin{itemize}
\item {Grp. gram.:adv.}
\end{itemize}
De modo rural.
De maneira campesina.
Á maneira de camponês.
\section{Rurícola}
\begin{itemize}
\item {Grp. gram.:adj.}
\end{itemize}
\begin{itemize}
\item {Proveniência:(Do lat. \textunderscore rus\textunderscore , \textunderscore ruris\textunderscore  + \textunderscore colere\textunderscore )}
\end{itemize}
Que vive no campo.
Que trabalha no campo; agricultor.
\section{Rurígena}
\begin{itemize}
\item {Grp. gram.:m. ,  f.  e  adj.}
\end{itemize}
\begin{itemize}
\item {Proveniência:(Do lat. \textunderscore rus\textunderscore , \textunderscore ruris\textunderscore  + ant. \textunderscore genere\textunderscore )}
\end{itemize}
Pessôa, que nasceu no campo.
\section{Rurógrafo}
\textunderscore m.\textunderscore (e des.)
O mesmo ou melhor que \textunderscore rusógrafo\textunderscore .
\section{Rurógrapho}
\textunderscore m.\textunderscore (e des.)
O mesmo ou melhor que \textunderscore rusógrapho\textunderscore .
\section{Rusélia}
\begin{itemize}
\item {Grp. gram.:f.}
\end{itemize}
Gênero de plantas escrofularíneas.
\section{Rusga}
\begin{itemize}
\item {Grp. gram.:f.}
\end{itemize}
\begin{itemize}
\item {Utilização:Pop.}
\end{itemize}
\begin{itemize}
\item {Utilização:Ant.}
\end{itemize}
\begin{itemize}
\item {Utilização:Prov.}
\end{itemize}
\begin{itemize}
\item {Utilização:trasm.}
\end{itemize}
Barulho, desordem.
Diligência policial, para prender malfeitores ou contraventores de certos regulamentos ou leis.
Correria de funccionários, destinada a prender indivíduos para soldado.
Tocata; pândega.
\section{Rusgar}
\begin{itemize}
\item {Grp. gram.:v. i.}
\end{itemize}
\begin{itemize}
\item {Utilização:Lisbôa}
\end{itemize}
\begin{itemize}
\item {Utilização:Neol.}
\end{itemize}
Fazer rusgas (a polícia). Cf. jornal \textunderscore Luta\textunderscore , de 1-II-912.
\section{Rusma}
\begin{itemize}
\item {Grp. gram.:f.}
\end{itemize}
Preparação epilatória, de que se servem os orientaes e é composta principalmente de cal viva.
(Ár. \textunderscore rusma\textunderscore )
\section{Rusníaco}
\begin{itemize}
\item {Grp. gram.:m.}
\end{itemize}
Língua, da Hungria setentrional.
\section{Rusografia}
\begin{itemize}
\item {Grp. gram.:f.}
\end{itemize}
\begin{itemize}
\item {Proveniência:(De \textunderscore rusógrafo\textunderscore )}
\end{itemize}
Tratado á cêrca dos campos ou da sua cultura.
\section{Rusográfico}
\begin{itemize}
\item {Grp. gram.:adj.}
\end{itemize}
Relativo á rusografia.
\section{Rusógrafo}
\begin{itemize}
\item {Grp. gram.:m.}
\end{itemize}
\begin{itemize}
\item {Proveniência:(Do lat. \textunderscore rus\textunderscore , \textunderscore ruris\textunderscore  + gr. \textunderscore graphein\textunderscore )}
\end{itemize}
Aquele que escreve á cêrca dos campos ou á cêrca da Agricultura.
\section{Rusographia}
\begin{itemize}
\item {Grp. gram.:f.}
\end{itemize}
\begin{itemize}
\item {Proveniência:(De \textunderscore rusógrapho\textunderscore )}
\end{itemize}
Tratado á cêrca dos campos ou da sua cultura.
\section{Rusográphico}
\begin{itemize}
\item {Grp. gram.:adj.}
\end{itemize}
Relativo á rusographia.
\section{Rusógrapho}
\begin{itemize}
\item {Grp. gram.:m.}
\end{itemize}
\begin{itemize}
\item {Proveniência:(Do lat. \textunderscore rus\textunderscore , \textunderscore ruris\textunderscore  + gr. \textunderscore graphein\textunderscore )}
\end{itemize}
Aquelle que escreve á cêrca dos campos ou á cêrca da Agricultura.
\section{Ruspone}
\begin{itemize}
\item {Grp. gram.:m.}
\end{itemize}
Antiga moéda de oiro da Toscana.
\section{Russiano}
\begin{itemize}
\item {Grp. gram.:m.  e  adj.}
\end{itemize}
O mesmo que russo^2.
\section{Russilho}
\begin{itemize}
\item {Grp. gram.:adj.}
\end{itemize}
(V.\textunderscore rucilho\textunderscore )
\section{Russínio}
\begin{itemize}
\item {Grp. gram.:m.}
\end{itemize}
\begin{itemize}
\item {Proveniência:(De \textunderscore russo\textunderscore ^2)}
\end{itemize}
Língua do ramo esclavónico.
\section{Russo}
\textunderscore m.\textunderscore  e \textunderscore adj.\textunderscore  (e der.)
(V.\textunderscore ruço\textunderscore , etc)
\section{Russo}
\begin{itemize}
\item {Grp. gram.:adj.}
\end{itemize}
\begin{itemize}
\item {Grp. gram.:M.}
\end{itemize}
Relativo á Rússia ou aos seus habitantes.
Habitante da Rússia.
Língua dos Russos.
\section{Russófilo}
\begin{itemize}
\item {Grp. gram.:adj.}
\end{itemize}
Diz-se do indivíduo afeiçoado á Rússia.
\section{Russofobia}
\begin{itemize}
\item {Grp. gram.:f.}
\end{itemize}
Qualidade de russófobo.
\section{Russófobo}
\begin{itemize}
\item {Grp. gram.:m.}
\end{itemize}
\begin{itemize}
\item {Proveniência:(De \textunderscore Rússia\textunderscore  + gr. \textunderscore phobein\textunderscore )}
\end{itemize}
Aquele que tem aversão á Rússia.
\section{Russóphilo}
\begin{itemize}
\item {Grp. gram.:adj.}
\end{itemize}
Diz-se do indivíduo affeiçoado á Rússia.
\section{Russophobia}
\begin{itemize}
\item {Grp. gram.:f.}
\end{itemize}
Qualidade de russóphobo.
\section{Russóphobo}
\begin{itemize}
\item {Grp. gram.:m.}
\end{itemize}
\begin{itemize}
\item {Proveniência:(De \textunderscore Rússia\textunderscore  + gr. \textunderscore phobein\textunderscore )}
\end{itemize}
Aquelle que tem aversão á Rússia.
\section{Rusticação}
\begin{itemize}
\item {Grp. gram.:f.}
\end{itemize}
\begin{itemize}
\item {Proveniência:(Lat. \textunderscore rusticatio\textunderscore )}
\end{itemize}
Acto ou effeito de rusticar.
Vida no campo; cultura do campo.
\section{Rusticamente}
\begin{itemize}
\item {Grp. gram.:adv.}
\end{itemize}
De modo rústico.
\section{Rusticanamente}
\begin{itemize}
\item {Grp. gram.:adv.}
\end{itemize}
De modo rusticano. Cf. Vieira, III, 321.
\section{Rusticano}
\begin{itemize}
\item {Grp. gram.:adj.}
\end{itemize}
\begin{itemize}
\item {Proveniência:(Lat. \textunderscore rusticanus\textunderscore )}
\end{itemize}
O mesmo que \textunderscore rústico\textunderscore .
\section{Rusticar}
\begin{itemize}
\item {Grp. gram.:v. t.}
\end{itemize}
\begin{itemize}
\item {Grp. gram.:V. i.}
\end{itemize}
\begin{itemize}
\item {Proveniência:(Lat. \textunderscore rusticari\textunderscore )}
\end{itemize}
Talhar, entre os ornatos em relêvo, (a pedra).
Viver no campo.
Entregar-se a trabalhos agrícolas.
\section{Rusticidade}
\begin{itemize}
\item {Grp. gram.:f.}
\end{itemize}
\begin{itemize}
\item {Proveniência:(Lat. \textunderscore rusticitas\textunderscore )}
\end{itemize}
Qualidade do que é rústico.
Indelicadeza; incivilidade.
\section{Rústico}
\begin{itemize}
\item {Grp. gram.:adj.}
\end{itemize}
\begin{itemize}
\item {Utilização:Prov.}
\end{itemize}
\begin{itemize}
\item {Utilização:trasm.}
\end{itemize}
\begin{itemize}
\item {Grp. gram.:M.}
\end{itemize}
\begin{itemize}
\item {Proveniência:(Lat. \textunderscore rusticus\textunderscore )}
\end{itemize}
Relativo ao campo ou próprio dêlle.
Rural.
Grosseiro; rude.
Em que não há arte.
O mesmo que \textunderscore robusto\textunderscore .
O mesmo que \textunderscore camponês\textunderscore .
\section{Rustificar}
\begin{itemize}
\item {Grp. gram.:v. t.}
\end{itemize}
\begin{itemize}
\item {Proveniência:(Do lat. \textunderscore rusticus\textunderscore  + \textunderscore facere\textunderscore )}
\end{itemize}
Tornar rústico; dar aspecto campesino a.
Dar modos de labrego a. Cf. Júl. Dinis, \textunderscore Morgadinha\textunderscore , 51.
\section{Rustiquez}
\begin{itemize}
\item {Grp. gram.:f.}
\end{itemize}
O mesmo que \textunderscore rusticidade\textunderscore . Cf.\textunderscore Viriato Trág.\textunderscore , 135.
\section{Rustiqueza}
\begin{itemize}
\item {Grp. gram.:f.}
\end{itemize}
O mesmo que \textunderscore rusticidade\textunderscore . Cf.\textunderscore Viriato Trág.\textunderscore , 135.
\section{Rutabaga}
\begin{itemize}
\item {Grp. gram.:f.}
\end{itemize}
Planta hýbrida, que participa das propriedades da couve e do nabo.
\section{Rutáceas}
\begin{itemize}
\item {Grp. gram.:f. pl.}
\end{itemize}
Família de plantas, que tem por typo a arruda.
(Fem. pl. de \textunderscore rutáceo\textunderscore )
\section{Rutáceo}
\begin{itemize}
\item {Grp. gram.:adj.}
\end{itemize}
\begin{itemize}
\item {Proveniência:(Do lat. \textunderscore ruta\textunderscore )}
\end{itemize}
Relativo ou semelhante á arruda.
\section{Rutáreas}
\begin{itemize}
\item {Grp. gram.:f. pl.}
\end{itemize}
\begin{itemize}
\item {Proveniência:(Do lat. \textunderscore ruta\textunderscore )}
\end{itemize}
Ordem de plantas, que abrange as rutáceas e as corianiáceas.
\section{Rutela}
\begin{itemize}
\item {Grp. gram.:f.}
\end{itemize}
\begin{itemize}
\item {Proveniência:(Lat. \textunderscore rutela\textunderscore )}
\end{itemize}
Gênero de insectos coleópteros.
\section{Rutenense}
\begin{itemize}
\item {Grp. gram.:adj.}
\end{itemize}
Relativo aos Rutenos. Cf. Herculano, \textunderscore Hist. de Port.\textunderscore , I, 492.
\section{Rutênico}
\begin{itemize}
\item {Grp. gram.:m.}
\end{itemize}
\begin{itemize}
\item {Proveniência:(De \textunderscore Rutenos\textunderscore )}
\end{itemize}
Um dos idiomas esclavónicos.
\section{Rutênio}
\begin{itemize}
\item {Grp. gram.:m.}
\end{itemize}
Metal raro e infusível, descoberto em 1846 nos mineraes de platina.
\section{Rutenos}
\begin{itemize}
\item {Grp. gram.:m. pl.}
\end{itemize}
\begin{itemize}
\item {Proveniência:(Lat. \textunderscore Rutheni\textunderscore  ou \textunderscore Ruteni\textunderscore )}
\end{itemize}
Antigo povo da Gállia. Cf.\textunderscore Lusíadas\textunderscore , III, 11.
\section{Ruthenense}
\begin{itemize}
\item {Grp. gram.:adj.}
\end{itemize}
Relativo aos Ruthenos. Cf. Herculano, \textunderscore Hist. de Port.\textunderscore , I, 492.
\section{Ruthênico}
\begin{itemize}
\item {Grp. gram.:m.}
\end{itemize}
\begin{itemize}
\item {Proveniência:(De \textunderscore Ruthenos\textunderscore )}
\end{itemize}
Um dos idiomas esclavónicos.
\section{Ruthênio}
\begin{itemize}
\item {Grp. gram.:m.}
\end{itemize}
Metal raro e infusível, descoberto em 1846 nos mineraes de platina.
\section{Ruthenos}
\begin{itemize}
\item {Grp. gram.:m. pl.}
\end{itemize}
\begin{itemize}
\item {Proveniência:(Lat. \textunderscore Rutheni\textunderscore  ou \textunderscore Ruteni\textunderscore )}
\end{itemize}
Antigo povo da Gállia. Cf.\textunderscore Lusíadas\textunderscore , III, 11.
\section{Rútico}
\begin{itemize}
\item {Grp. gram.:adj.}
\end{itemize}
\begin{itemize}
\item {Proveniência:(Do lat. \textunderscore ruta\textunderscore )}
\end{itemize}
Diz-se de diversas substâncias, extrahidas da arruda.
\section{Rutídea}
\begin{itemize}
\item {Grp. gram.:f.}
\end{itemize}
\begin{itemize}
\item {Proveniência:(Do gr. \textunderscore rutis\textunderscore )}
\end{itemize}
Gênero de plantas rubiáceas.
\section{Rútila}
\begin{itemize}
\item {Grp. gram.:f.}
\end{itemize}
\begin{itemize}
\item {Proveniência:(De \textunderscore rútilo\textunderscore )}
\end{itemize}
Oxydo titânico, de côr avermelhado.
\section{Rutilação}
\begin{itemize}
\item {Grp. gram.:f.}
\end{itemize}
Acto de rutilar.
Brilho intenso; resplendor. Cf. Júl. Lour. Pinto, \textunderscore Senh. Deput.\textunderscore , 24.
\section{Rutilância}
\begin{itemize}
\item {Grp. gram.:f.}
\end{itemize}
Qualidade do que é rutilante.
\section{Rutilante}
\begin{itemize}
\item {Grp. gram.:adj.}
\end{itemize}
\begin{itemize}
\item {Proveniência:(Lat. \textunderscore rutilans\textunderscore )}
\end{itemize}
Que rutila.
Muito brilhante; resplandecente.
\section{Rutilantemente}
\begin{itemize}
\item {Grp. gram.:adv.}
\end{itemize}
De modo rutilante.
\section{Rutilar}
\begin{itemize}
\item {Grp. gram.:v. t.}
\end{itemize}
\begin{itemize}
\item {Grp. gram.:V. i.}
\end{itemize}
\begin{itemize}
\item {Proveniência:(Lat. \textunderscore rutilare\textunderscore )}
\end{itemize}
Tornar rútilo.
Fazer brilhar muito.
Brilhar muito; resplandecer.
\section{Rutília}
\begin{itemize}
\item {Grp. gram.:f.}
\end{itemize}
Gênero de insectos dípteros.
\section{Rutilina}
\begin{itemize}
\item {Grp. gram.:f.}
\end{itemize}
\begin{itemize}
\item {Utilização:Chím.}
\end{itemize}
\begin{itemize}
\item {Proveniência:(De \textunderscore rútilo\textunderscore )}
\end{itemize}
Substância rubra, produzida pela acção do ácido sulfúrico sôbre a salicina.
\section{Rutilito}
\begin{itemize}
\item {Grp. gram.:m.}
\end{itemize}
\begin{itemize}
\item {Utilização:Miner.}
\end{itemize}
\begin{itemize}
\item {Proveniência:(De \textunderscore rútilo\textunderscore )}
\end{itemize}
Variedade de granada.
\section{Rútilo}
\begin{itemize}
\item {Grp. gram.:adj.}
\end{itemize}
\begin{itemize}
\item {Utilização:Poét.}
\end{itemize}
\begin{itemize}
\item {Proveniência:(Lat. \textunderscore rutilus\textunderscore )}
\end{itemize}
O mesmo que \textunderscore rutilante\textunderscore .
\section{Rutilo}
\begin{itemize}
\item {Grp. gram.:m.}
\end{itemize}
\begin{itemize}
\item {Utilização:bras}
\end{itemize}
\begin{itemize}
\item {Utilização:Neol.}
\end{itemize}
Acto de rutilar.
\section{Rutina}
\begin{itemize}
\item {Grp. gram.:f.}
\end{itemize}
\begin{itemize}
\item {Proveniência:(Do lat. \textunderscore ruta\textunderscore )}
\end{itemize}
Princípio antispasmódico, contido na arruda.
\section{Rutínico}
\begin{itemize}
\item {Grp. gram.:adj.}
\end{itemize}
\begin{itemize}
\item {Proveniência:(De \textunderscore rutina\textunderscore )}
\end{itemize}
Diz-se de um ácido, contido na arruda.
\section{Ruto}
\begin{itemize}
\item {Grp. gram.:m.}
\end{itemize}
\begin{itemize}
\item {Utilização:Obsol.}
\end{itemize}
\begin{itemize}
\item {Utilização:Prov.}
\end{itemize}
\begin{itemize}
\item {Utilização:trasm.}
\end{itemize}
\begin{itemize}
\item {Proveniência:(Fr. \textunderscore route\textunderscore , do lat. \textunderscore ruptus\textunderscore )}
\end{itemize}
O mesmo que \textunderscore caminho\textunderscore .
Maneira antiga de caçar aves:«\textunderscore ...que se não caçassem perdizes ao ruto.\textunderscore »\textunderscore Doc. do séc. XVI\textunderscore , no \textunderscore Instituto\textunderscore , LIX, 203.
Róta, rumo; vôo.
\section{Rútulos}
\begin{itemize}
\item {Grp. gram.:m. pl.}
\end{itemize}
\begin{itemize}
\item {Proveniência:(Lat. \textunderscore Rutuli\textunderscore )}
\end{itemize}
Povos do antigo Lácio.
\section{Ruvinhoso}
\begin{itemize}
\item {Grp. gram.:adj.}
\end{itemize}
\begin{itemize}
\item {Utilização:Fig.}
\end{itemize}
\begin{itemize}
\item {Proveniência:(Do lat. \textunderscore rubiginosus\textunderscore )}
\end{itemize}
Que tem ferrugem.
Que tem caruncho.
Carcomido.
Mal humorado.
\section{Ruxaxá!}
\begin{itemize}
\item {Grp. gram.:interj.}
\end{itemize}
\begin{itemize}
\item {Utilização:Prov.}
\end{itemize}
\begin{itemize}
\item {Utilização:minh.}
\end{itemize}
O mesmo que \textunderscore ruxoxó!\textunderscore 
\section{Ruxoxó!}
\begin{itemize}
\item {Grp. gram.:interj.}
\end{itemize}
\begin{itemize}
\item {Utilização:Ant.}
\end{itemize}
\begin{itemize}
\item {Grp. gram.:M.}
\end{itemize}
Voz, com que se enxotam as aves.
Surriada!
Assuada; troça:«\textunderscore ...eu lhe dixe, que nom hiom elles de cá enxotados de geito, que esperassem outro ruxoxó.\textunderscore »(\textunderscore Carta\textunderscore  do Arceb. de Braga, D. Lourenço ao Abbade de Alcobaça, (1385), sôbre a batalha de Aljubarrota)
\section{Rýthon}
\begin{itemize}
\item {Grp. gram.:m.}
\end{itemize}
\begin{itemize}
\item {Proveniência:(Gr. \textunderscore ruthon\textunderscore )}
\end{itemize}
Vaso, em que os Gregos bebiam vinho e que, primitivamente, tinha a forma de chavelho.
\section{Rythão}
\begin{itemize}
\item {Grp. gram.:m.}
\end{itemize}
\begin{itemize}
\item {Proveniência:(Gr. \textunderscore ruthon\textunderscore )}
\end{itemize}
Vaso, em que os Gregos bebiam vinho e que, primitivamente, tinha a forma de chavelho.
\section{Rytidósia}
\begin{itemize}
\item {Grp. gram.:f.}
\end{itemize}
\begin{itemize}
\item {Proveniência:(Do gr. \textunderscore rutis\textunderscore , \textunderscore rutidos\textunderscore )}
\end{itemize}
\end{document}