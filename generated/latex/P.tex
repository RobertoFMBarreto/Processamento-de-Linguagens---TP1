\documentclass{article}
\usepackage[portuguese]{babel}
\title{P}
\begin{document}
Substância bituminosa, que se encontra na terra, e de que se extrai a parafina.
\section{Preoccupar}
\begin{itemize}
\item {Grp. gram.:v. t.}
\end{itemize}
\begin{itemize}
\item {Proveniência:(Lat. \textunderscore praeoccupare\textunderscore )}
\end{itemize}
Prender a attenção de.
Dar cuidado a.
Impressionar.
Tornar apprehensivo, inquieto.
\section{Preopérculo}
\begin{itemize}
\item {Grp. gram.:m.}
\end{itemize}
\begin{itemize}
\item {Utilização:Zool.}
\end{itemize}
\begin{itemize}
\item {Proveniência:(De \textunderscore pre...\textunderscore  + \textunderscore opérculo\textunderscore )}
\end{itemize}
Peça óssea, por meio da qual o operculo dos peixes se articula com o crânio.
\section{Preopinante}
\begin{itemize}
\item {Grp. gram.:m. ,  f.  e  adj.}
\end{itemize}
Pessôa, que preopina, que falou antes de outrem.
\section{Preopinar}
\begin{itemize}
\item {Grp. gram.:v. i.}
\end{itemize}
\begin{itemize}
\item {Proveniência:(De \textunderscore pre...\textunderscore  + \textunderscore opinar\textunderscore )}
\end{itemize}
Opinar antes de outrem.
Expor o seu parecer, discursando, antes de outrem.
\section{Preordenação}
\begin{itemize}
\item {Grp. gram.:f.}
\end{itemize}
Acto ou effeito de \textunderscore preordenar\textunderscore .
\section{Preordenar}
\begin{itemize}
\item {Grp. gram.:v. t.}
\end{itemize}
\begin{itemize}
\item {Proveniência:(De \textunderscore pre...\textunderscore  + \textunderscore ordenar\textunderscore )}
\end{itemize}
Ordenar com antecipação.
Predestinar.
\section{Preparação}
\begin{itemize}
\item {Grp. gram.:f.}
\end{itemize}
\begin{itemize}
\item {Proveniência:(Lat. \textunderscore praeparatio\textunderscore )}
\end{itemize}
Acto, effeito ou maneira de preparar.
Producto de uma operação pharmacêutica.
\section{Preparado}
\begin{itemize}
\item {Grp. gram.:m.}
\end{itemize}
\begin{itemize}
\item {Proveniência:(De \textunderscore preparar\textunderscore )}
\end{itemize}
Producto chímico ou pharmacêutico.
\section{Preparador}
\begin{itemize}
\item {Grp. gram.:m.  e  adj.}
\end{itemize}
\begin{itemize}
\item {Proveniência:(Lat. \textunderscore praeparator\textunderscore )}
\end{itemize}
O que prepara.
\section{Preparamento}
\begin{itemize}
\item {Grp. gram.:m.}
\end{itemize}
O mesmo que \textunderscore preparação\textunderscore .
\section{Preparar}
\begin{itemize}
\item {Grp. gram.:v. t.}
\end{itemize}
\begin{itemize}
\item {Utilização:Fig.}
\end{itemize}
\begin{itemize}
\item {Utilização:Pop.}
\end{itemize}
\begin{itemize}
\item {Proveniência:(Lat. \textunderscore praeparare\textunderscore )}
\end{itemize}
Dispor com antecedência.
Apparelhar; apromptar.
Planear.
Compor.
Aperceber.
Combinar elementos para formar (um corpo, um medicamento).
Fomentar: \textunderscore preparar uma revolta\textunderscore .
Prenunciar.
O mesmo que \textunderscore capar\textunderscore : \textunderscore ó vizinha, já mandou preparar o seu gato\textunderscore ?
\section{Preparativo}
\begin{itemize}
\item {Grp. gram.:adj.}
\end{itemize}
\begin{itemize}
\item {Grp. gram.:M.}
\end{itemize}
\begin{itemize}
\item {Proveniência:(De \textunderscore preparar\textunderscore )}
\end{itemize}
O mesmo que \textunderscore preparatório\textunderscore .
O mesmo que \textunderscore preparação\textunderscore .
\section{Preparatoriamente}
\begin{itemize}
\item {Grp. gram.:adv.}
\end{itemize}
De modo preparatório; com preparação; com predisposição.
\section{Preparatoriano}
\begin{itemize}
\item {Grp. gram.:m.  e  adj.}
\end{itemize}
\begin{itemize}
\item {Utilização:bras}
\end{itemize}
\begin{itemize}
\item {Utilização:Neol.}
\end{itemize}
Diz-se do estudante de preparatórios.
\section{Preparatório}
\begin{itemize}
\item {Grp. gram.:adj.}
\end{itemize}
\begin{itemize}
\item {Grp. gram.:M. Pl.}
\end{itemize}
\begin{itemize}
\item {Proveniência:(Lat. \textunderscore praeparatorius\textunderscore )}
\end{itemize}
Que prepara.
Próprio para preparar.
Preliminar.
Disciplinas, que se professam nos institutos de instrucção secundária.
Estudos prévios, necessários para entrada numa escola superior.
\section{Preparo}
\begin{itemize}
\item {Grp. gram.:m.}
\end{itemize}
\begin{itemize}
\item {Grp. gram.:Pl.}
\end{itemize}
\begin{itemize}
\item {Proveniência:(De \textunderscore preparar\textunderscore )}
\end{itemize}
Preparação.
Apresto.
Quantia, depositada em poder do escrivão de um processo para pagamento de custas.
Enfeites, adornos.
Aviamentos ou miudezas, necessárias para o acabamento de uma peça de vestuário.
\section{Prepau}
\begin{itemize}
\item {Grp. gram.:m.}
\end{itemize}
\begin{itemize}
\item {Utilização:Náut.}
\end{itemize}
\begin{itemize}
\item {Proveniência:(De \textunderscore pre...\textunderscore  + \textunderscore pau\textunderscore )}
\end{itemize}
Peça de madeira, junto ao mastro de navio, para nella se amarrarem as escoteiras da gávea:«\textunderscore ...e encostando a cabeça no prepau do chapiteo...\textunderscore »\textunderscore Peregrin.\textunderscore , fol. 285, v.^o.
\section{Prepetana}
\begin{itemize}
\item {Grp. gram.:f.}
\end{itemize}
\begin{itemize}
\item {Utilização:Ant.}
\end{itemize}
O mesmo que \textunderscore barbatana\textunderscore :«\textunderscore ...o peixe... abaixou as prepetanas...\textunderscore »Barros, \textunderscore Déc.\textunderscore  III, l. IV, c. 7.
\section{Prepetinga}
\begin{itemize}
\item {Grp. gram.:m.}
\end{itemize}
Peixe fluvial do Brasil.
\section{Prepoém}
\begin{itemize}
\item {Grp. gram.:m.}
\end{itemize}
\begin{itemize}
\item {Utilização:Ant.}
\end{itemize}
\begin{itemize}
\item {Proveniência:(Do fr. \textunderscore pourpoint\textunderscore )}
\end{itemize}
Justilho ou corpete de mulhér.
\section{Preponderância}
\begin{itemize}
\item {Grp. gram.:f.}
\end{itemize}
\begin{itemize}
\item {Utilização:Artilh.}
\end{itemize}
Qualidade do que é preponderante; predominio; supremacia; hegemonia.
Differença entre o pêso da peça, desde a culatra aos munhões, e o pêso da mesma peça, desde os munhões á boca.
\section{Preponderante}
\begin{itemize}
\item {Grp. gram.:adj.}
\end{itemize}
\begin{itemize}
\item {Proveniência:(Lat. \textunderscore praeponderans\textunderscore )}
\end{itemize}
Que prepondera.
\section{Preponderantismo}
\begin{itemize}
\item {Grp. gram.:m.}
\end{itemize}
\begin{itemize}
\item {Proveniência:(De \textunderscore preponderante\textunderscore )}
\end{itemize}
Influência exaggerada.
Carácter oppressivo de corrilhos e mandões politicos. Cf. Camillo, \textunderscore Narcót.\textunderscore , I, 287.
\section{Preponderar}
\begin{itemize}
\item {Grp. gram.:v. i.}
\end{itemize}
\begin{itemize}
\item {Proveniência:(Lat. \textunderscore praeponderare\textunderscore )}
\end{itemize}
Pesar mais que outro ou outrem.
Predominar, prevalecer.
Têr mais influência ou importância.
\section{Preságio}
\begin{itemize}
\item {fónica:sá}
\end{itemize}
\begin{itemize}
\item {Grp. gram.:m.}
\end{itemize}
\begin{itemize}
\item {Proveniência:(Lat. \textunderscore praesagium\textunderscore )}
\end{itemize}
Previsão; prognóstico; presentimento; agoiro.
\section{Presagioso}
\begin{itemize}
\item {fónica:sa}
\end{itemize}
\begin{itemize}
\item {Grp. gram.:adj.}
\end{itemize}
Em que há preságio.
\section{Presago}
\begin{itemize}
\item {fónica:sá}
\end{itemize}
\begin{itemize}
\item {Grp. gram.:m.}
\end{itemize}
\begin{itemize}
\item {Proveniência:(Lat. \textunderscore praesagus\textunderscore )}
\end{itemize}
Que presagia; presagioso.
\section{Presar}
\textunderscore v. t.\textunderscore  (e der.)
(Fórma ant. de \textunderscore apresar\textunderscore , etc.)
\section{Presbiope}
\begin{itemize}
\item {Grp. gram.:m.}
\end{itemize}
Aquele que sofre a doença de presbiopia.
\section{Presbiopia}
\begin{itemize}
\item {Grp. gram.:f.}
\end{itemize}
\begin{itemize}
\item {Proveniência:(Do gr. \textunderscore presbus\textunderscore  + \textunderscore ops\textunderscore )}
\end{itemize}
O mesmo que \textunderscore presbitismo\textunderscore .
\section{Presbita}
\begin{itemize}
\item {Grp. gram.:m.}
\end{itemize}
O mesmo ou melhor que \textunderscore presbito\textunderscore .
\section{Presbiterado}
\begin{itemize}
\item {Grp. gram.:m.}
\end{itemize}
\begin{itemize}
\item {Proveniência:(Lat. \textunderscore presbyteratus\textunderscore )}
\end{itemize}
Dignidade de presbítero.
\section{Presbiteral}
\begin{itemize}
\item {Grp. gram.:adj.}
\end{itemize}
\begin{itemize}
\item {Proveniência:(Lat. \textunderscore presbyteralis\textunderscore )}
\end{itemize}
Relativo a presbitero.
\section{Presbiterianismo}
\begin{itemize}
\item {Grp. gram.:m.}
\end{itemize}
Seita religiosa dos Presbiterianos.
\section{Presbiteriano}
\begin{itemize}
\item {Grp. gram.:m.}
\end{itemize}
\begin{itemize}
\item {Proveniência:(De \textunderscore presbítero\textunderscore )}
\end{itemize}
Protestante, que não admite jerarquia eclesiástica, superior á dos presbíteros, e que não admite Bispos.
\section{Presbitério}
\begin{itemize}
\item {Grp. gram.:m.}
\end{itemize}
\begin{itemize}
\item {Proveniência:(Lat. \textunderscore presbyterium\textunderscore )}
\end{itemize}
Presidência do pároco.
Igreja paroquial.
Capela-mór.
Nos primeiros tempos do Cristianismo, era o conselho do Bispo, formado pelos presbíteros e diáconos residentes na séde da catedral.
Lugar, onde êsse conselho se reunia.
\section{Presbítero}
\begin{itemize}
\item {Grp. gram.:m.}
\end{itemize}
\begin{itemize}
\item {Proveniência:(Lat. \textunderscore presbyter\textunderscore )}
\end{itemize}
Sacerdote, padre.
\section{Presbitia}
\begin{itemize}
\item {Grp. gram.:f.}
\end{itemize}
O mesmo que \textunderscore presbitismo\textunderscore .
\section{Presbitismo}
\begin{itemize}
\item {Grp. gram.:m.}
\end{itemize}
\begin{itemize}
\item {Proveniência:(De \textunderscore présbito\textunderscore )}
\end{itemize}
Vista cansada.
Vista confusa, quando os objectos se encaram de perto; e nítida, quando se encaram de longe.
\section{Presbito}
\begin{itemize}
\item {Grp. gram.:m.  e  adj.}
\end{itemize}
\begin{itemize}
\item {Proveniência:(Do gr. \textunderscore presbutes\textunderscore )}
\end{itemize}
Aquele que tem presbitismo.
\section{Presbyope}
\begin{itemize}
\item {Grp. gram.:m.}
\end{itemize}
Aquelle que soffre a doença de presbyopia.
\section{Presbyopia}
\begin{itemize}
\item {Grp. gram.:f.}
\end{itemize}
\begin{itemize}
\item {Proveniência:(Do gr. \textunderscore presbus\textunderscore  + \textunderscore ops\textunderscore )}
\end{itemize}
O mesmo que \textunderscore presbytismo\textunderscore .
\section{Presbyta}
\begin{itemize}
\item {Grp. gram.:m.}
\end{itemize}
O mesmo ou melhor que \textunderscore presbyto\textunderscore .
\section{Presbyterado}
\begin{itemize}
\item {Grp. gram.:m.}
\end{itemize}
\begin{itemize}
\item {Proveniência:(Lat. \textunderscore presbyteratus\textunderscore )}
\end{itemize}
Dignidade de presbýtero.
\section{Presbyteral}
\begin{itemize}
\item {Grp. gram.:adj.}
\end{itemize}
\begin{itemize}
\item {Proveniência:(Lat. \textunderscore presbyteralis\textunderscore )}
\end{itemize}
Relativo a presbytero.
\section{Presbyteranismo}
\begin{itemize}
\item {Grp. gram.:m.}
\end{itemize}
(V.presbyterianismo)
\section{Presbyterano}
\begin{itemize}
\item {Grp. gram.:m.  e  adj.}
\end{itemize}
(V.presbyteriano)
\section{Presbyterianismo}
\begin{itemize}
\item {Grp. gram.:m.}
\end{itemize}
Seita religiosa dos Presbyterianos.
\section{Presbyteriano}
\begin{itemize}
\item {Grp. gram.:m.}
\end{itemize}
\begin{itemize}
\item {Proveniência:(De \textunderscore presbýtero\textunderscore )}
\end{itemize}
Protestante, que não admitte gerarchia ecclesiástica, superior á dos presbýteros, e que não admitte Bispos.
\section{Presbytério}
\begin{itemize}
\item {Grp. gram.:m.}
\end{itemize}
\begin{itemize}
\item {Proveniência:(Lat. \textunderscore presbyterium\textunderscore )}
\end{itemize}
Presidência do párocho.
Igreja parochial.
Capella-mór.
Nos primeiros tempos do Christianismo, era o conselho do Bispo, formado pelos presbýteros e diáconos residentes na séde da cathedral.
Lugar, onde êsse conselho se reunia.
\section{Presbýtero}
\begin{itemize}
\item {Grp. gram.:m.}
\end{itemize}
\begin{itemize}
\item {Proveniência:(Lat. \textunderscore presbyter\textunderscore )}
\end{itemize}
Sacerdote, padre.
\section{Presbytia}
\begin{itemize}
\item {Grp. gram.:f.}
\end{itemize}
O mesmo que \textunderscore presbytismo\textunderscore .
\section{Presbytismo}
\begin{itemize}
\item {Grp. gram.:m.}
\end{itemize}
\begin{itemize}
\item {Proveniência:(De \textunderscore présbyto\textunderscore )}
\end{itemize}
Vista cansada.
Vista confusa, quando os objectos se encaram de perto; e nítida, quando se encaram de longe.
\section{Presbyto}
\begin{itemize}
\item {Grp. gram.:m.  e  adj.}
\end{itemize}
\begin{itemize}
\item {Proveniência:(Do gr. \textunderscore presbutes\textunderscore )}
\end{itemize}
Aquelle que tem presbytismo.
\section{Presciência}
\begin{itemize}
\item {Grp. gram.:f.}
\end{itemize}
\begin{itemize}
\item {Proveniência:(Lat. \textunderscore praescientia\textunderscore )}
\end{itemize}
Qualidade do que é presciente; previsão.
\section{Presciente}
\begin{itemize}
\item {Grp. gram.:adj.}
\end{itemize}
\begin{itemize}
\item {Utilização:Fig.}
\end{itemize}
\begin{itemize}
\item {Proveniência:(Lat. \textunderscore praesciens\textunderscore )}
\end{itemize}
Que sabe com antecipação; que prevê o futuro; presago.
Previdente, acautelado.
\section{Prescientífico}
\begin{itemize}
\item {Grp. gram.:adj.}
\end{itemize}
\begin{itemize}
\item {Proveniência:(De \textunderscore pre...\textunderscore  + \textunderscore scientifico\textunderscore )}
\end{itemize}
Anterior ao apparecimento da sciência.
Anterior á civilização do mundo.
\section{Prescindência}
\begin{itemize}
\item {Grp. gram.:f.}
\end{itemize}
\begin{itemize}
\item {Utilização:P. us.}
\end{itemize}
Acto de prescindir. Cf. Alex. Lobo, III, 354.
\section{Prescindir}
\begin{itemize}
\item {Grp. gram.:v. i.}
\end{itemize}
\begin{itemize}
\item {Proveniência:(Lat. \textunderscore praescindere\textunderscore )}
\end{itemize}
Separar, abstrahir de.
Dispensar.
Pôr de lado.
\section{Prescindível}
\begin{itemize}
\item {Grp. gram.:adj.}
\end{itemize}
De que se póde prescindir.
\section{Prescito}
\begin{itemize}
\item {Grp. gram.:m.  e  adj.}
\end{itemize}
O mesmo ou melhor que \textunderscore precito\textunderscore .
\section{Prescrever}
\begin{itemize}
\item {Grp. gram.:v. t.}
\end{itemize}
\begin{itemize}
\item {Grp. gram.:V. i.}
\end{itemize}
\begin{itemize}
\item {Utilização:Jur.}
\end{itemize}
\begin{itemize}
\item {Proveniência:(Lat. \textunderscore praescribere\textunderscore )}
\end{itemize}
Determinar com antecipação.
Ordenar previamente e de modo explícito.
Regular; preceituar.
Ficar sem effeito (um direito), por têr decorrido certo espaço de tempo, marcado na lei.
\section{Prescribente}
\begin{itemize}
\item {Grp. gram.:adj.}
\end{itemize}
\begin{itemize}
\item {Utilização:Jur.}
\end{itemize}
\begin{itemize}
\item {Proveniência:(Lat. \textunderscore praescribens\textunderscore )}
\end{itemize}
Que prescreve.
Em que se dá a prescripção. Cf. Assis, \textunderscore Águas\textunderscore , 268.
\section{Prescrição}
\begin{itemize}
\item {Grp. gram.:f.}
\end{itemize}
\begin{itemize}
\item {Utilização:Jur.}
\end{itemize}
\begin{itemize}
\item {Proveniência:(Lat. \textunderscore praescriptio\textunderscore )}
\end{itemize}
Acto ou efeito de prescrever.
Ordem expressa.
Preceito; regra.
Extinção de um direito, que se não exerceu por determinado tempo.
Extinção de uma obrigação, cujo cumprimento não foi exigido durante determinado tempo.
\section{Prescripção}
\begin{itemize}
\item {Grp. gram.:f.}
\end{itemize}
\begin{itemize}
\item {Utilização:Jur.}
\end{itemize}
\begin{itemize}
\item {Proveniência:(Lat. \textunderscore praescriptio\textunderscore )}
\end{itemize}
Acto ou effeito de prescrever.
Ordem expressa.
Preceito; regra.
Extincção de um direito, que se não exerceu por determinado tempo.
Extincção de uma obrigação, cujo cumprimento não foi exigido durante determinado tempo.
\section{Presidência}
\begin{itemize}
\item {Grp. gram.:f.}
\end{itemize}
\begin{itemize}
\item {Utilização:Ext.}
\end{itemize}
\begin{itemize}
\item {Utilização:Pop.}
\end{itemize}
\begin{itemize}
\item {Proveniência:(De \textunderscore presidente\textunderscore )}
\end{itemize}
Acto de presidir.
Funcções ou cargo de presidente.
Tempo, durante o qual alguém exerce funcções de presidente.
Residência do presidente.
Lugar, em que se assenta o presidente.
O presidente.
Lugar de honra, á mesa de um banquete.
\section{Presidencial}
\begin{itemize}
\item {Grp. gram.:adj.}
\end{itemize}
Relativo á presidência ou ao presidente.
\section{Presidenta}
\begin{itemize}
\item {Grp. gram.:f.}
\end{itemize}
\begin{itemize}
\item {Utilização:Neol.}
\end{itemize}
Mulhér, que preside.
Mulhér de um presidente. Cf. Castilho, \textunderscore Sabichonas\textunderscore , 128.
(Fem. de \textunderscore presidente\textunderscore )
\section{Presidente}
\begin{itemize}
\item {Grp. gram.:m.  e  adj.}
\end{itemize}
\begin{itemize}
\item {Proveniência:(Lat. \textunderscore praesidens\textunderscore )}
\end{itemize}
O que preside.
Aquelle que dirige os trabalhos de uma assembleia ou de uma corporação deliberativa.
Título moderno do chefe de república.
\section{Presidiar}
\begin{itemize}
\item {Grp. gram.:v. t.}
\end{itemize}
\begin{itemize}
\item {Proveniência:(Lat. \textunderscore praesidiari\textunderscore )}
\end{itemize}
Pôr presídio a.
Pôr guardas a; reforçar.
\section{Presidiário}
\begin{itemize}
\item {Grp. gram.:adj.}
\end{itemize}
\begin{itemize}
\item {Grp. gram.:M.}
\end{itemize}
\begin{itemize}
\item {Proveniência:(Lat. \textunderscore praesidiarius\textunderscore )}
\end{itemize}
Relativo a presídio.
Preso num presídio.
Aquelle que está condemnado a trabalhar num presídio.
\section{Presídio}
\begin{itemize}
\item {Grp. gram.:m.}
\end{itemize}
\begin{itemize}
\item {Proveniência:(Lat. \textunderscore praesidium\textunderscore )}
\end{itemize}
Acto de defender uma praça.
Guarnição militar.
Praça militar.
Prisão militar.
Cadeia.
Pena de prisão, que se deve expiar numa praça de guerra: \textunderscore condemnado a presídio\textunderscore .
\section{Presidir}
\begin{itemize}
\item {Grp. gram.:v. i.}
\end{itemize}
\begin{itemize}
\item {Proveniência:(Lat. \textunderscore praesidere\textunderscore )}
\end{itemize}
Occupar o primeiro lugar ou lugar superior numa assembleia, etc.
Superintender.
Exercer as funcções, próprias de quem dirige os trabalhos ou as sessões de uma assembleia ou de uma collectividade deliberativa.
Regular a ordem.
Occupar a presidência.
Assistir, dirigindo ou regulando.
\section{Presiganga}
\begin{itemize}
\item {Grp. gram.:f.}
\end{itemize}
\begin{itemize}
\item {Proveniência:(De \textunderscore preso\textunderscore )}
\end{itemize}
Navio, que serve de prisão, ou em que se recolhem prisioneiros.
\section{Presigar}
\begin{itemize}
\item {Grp. gram.:v. t.}
\end{itemize}
\begin{itemize}
\item {Utilização:Prov.}
\end{itemize}
Acompanhar com presigo.
Conductar.
O mesmo que \textunderscore apresigar\textunderscore .
\section{Presigo}
\begin{itemize}
\item {Grp. gram.:m.}
\end{itemize}
\begin{itemize}
\item {Utilização:Prov.}
\end{itemize}
Conducto.
Carne de porco; presunto; toicinho.
Farnel.
(Outra fórma de \textunderscore apresigo\textunderscore . V. \textunderscore apresigo\textunderscore )
\section{Presilha}
\begin{itemize}
\item {Grp. gram.:f.}
\end{itemize}
\begin{itemize}
\item {Utilização:Chul.}
\end{itemize}
\begin{itemize}
\item {Proveniência:(De \textunderscore preso\textunderscore )}
\end{itemize}
Cordão ou tira de pano, etc., que tem na extremidade uma espécie de aselha ou casa, e em que se enfia um botão, para apertar, prender, etc.
Lábia, intrujice.
\section{Presilheiro}
\begin{itemize}
\item {Grp. gram.:m.}
\end{itemize}
\begin{itemize}
\item {Utilização:Chul.}
\end{itemize}
\begin{itemize}
\item {Proveniência:(De \textunderscore presilha\textunderscore )}
\end{itemize}
Finório.
Intrujão.
\section{Presimol}
\begin{itemize}
\item {Grp. gram.:m.}
\end{itemize}
Tecido antigo?:«\textunderscore véstia de presimol\textunderscore ». (De um testamento de 1693)
\section{Préslia}
\begin{itemize}
\item {Grp. gram.:f.}
\end{itemize}
\begin{itemize}
\item {Proveniência:(De \textunderscore Presl\textunderscore , n. p.)}
\end{itemize}
Gênero de plantas labiadas.
\section{Preso}
\begin{itemize}
\item {fónica:prê}
\end{itemize}
\begin{itemize}
\item {Grp. gram.:m.}
\end{itemize}
\begin{itemize}
\item {Proveniência:(Do lat. \textunderscore prehensus\textunderscore )}
\end{itemize}
Indivíduo, que está preso ou encarcerado.
Prisioneiro.
\section{Presocrático}
\begin{itemize}
\item {fónica:so}
\end{itemize}
\begin{itemize}
\item {Grp. gram.:adj.}
\end{itemize}
\begin{itemize}
\item {Proveniência:(De \textunderscore pre...\textunderscore  + \textunderscore socrático\textunderscore )}
\end{itemize}
Antetior ao tempo ou ás doutrinas de Sócrates.
\section{Presor}
\begin{itemize}
\item {Grp. gram.:m.}
\end{itemize}
Cada um dos colonos, inferiores á classe nobre e privilegiada, que no tempo dos reis de Leão reconquistavam ou recebiam em partilha terras conquistadas aos Moiros. Cf. Herculano, \textunderscore Hist. de Portugal\textunderscore ; Figanière, \textunderscore Guesto Ansures\textunderscore .
(Cp. \textunderscore presúria\textunderscore )
\section{Pressa}
\begin{itemize}
\item {Grp. gram.:f.}
\end{itemize}
\begin{itemize}
\item {Proveniência:(Do lat. \textunderscore pressus\textunderscore )}
\end{itemize}
Lígeireza; velocidade, rapidez.
Promptidão.
Urgência, caso urgente.
Azáfama.
\section{Presságio}
\begin{itemize}
\item {Grp. gram.:m.}
\end{itemize}
\begin{itemize}
\item {Proveniência:(Lat. \textunderscore praesagium\textunderscore )}
\end{itemize}
Previsão; prognóstico; pressentimento; agoiro.
\section{Pressagioso}
\begin{itemize}
\item {Grp. gram.:adj.}
\end{itemize}
Em que há presságio.
\section{Pressago}
\begin{itemize}
\item {Grp. gram.:m.}
\end{itemize}
\begin{itemize}
\item {Proveniência:(Lat. \textunderscore praesagus\textunderscore )}
\end{itemize}
Que pressagia; pressagioso.
\section{Pressão}
\begin{itemize}
\item {Grp. gram.:f.}
\end{itemize}
\begin{itemize}
\item {Utilização:Fig.}
\end{itemize}
\begin{itemize}
\item {Proveniência:(Lat. \textunderscore pressio\textunderscore )}
\end{itemize}
Acto ou effeito de premir.
Coacção; violência.
\section{Pressirostro}
\begin{itemize}
\item {fónica:ros}
\end{itemize}
\begin{itemize}
\item {Grp. gram.:adj.}
\end{itemize}
\begin{itemize}
\item {Grp. gram.:M. Pl.}
\end{itemize}
\begin{itemize}
\item {Proveniência:(Do lat. \textunderscore pressus\textunderscore  + \textunderscore rostrum\textunderscore )}
\end{itemize}
Que tem o bico comprido.
Família de aves pernaltas, que têm o bico comprido.
\section{Pressirrostro}
\begin{itemize}
\item {Grp. gram.:adj.}
\end{itemize}
\begin{itemize}
\item {Grp. gram.:M. Pl.}
\end{itemize}
\begin{itemize}
\item {Proveniência:(Do lat. \textunderscore pressus\textunderscore  + \textunderscore rostrum\textunderscore )}
\end{itemize}
Que tem o bico comprido.
Família de aves pernaltas, que têm o bico comprido.
\section{Pressocrático}
\begin{itemize}
\item {Grp. gram.:adj.}
\end{itemize}
\begin{itemize}
\item {Proveniência:(De \textunderscore pre...\textunderscore  + \textunderscore socrático\textunderscore )}
\end{itemize}
Antetior ao tempo ou ás doutrinas de Sócrates.
\section{Pressupor}
\begin{itemize}
\item {Grp. gram.:v. t.}
\end{itemize}
\begin{itemize}
\item {Proveniência:(De \textunderscore pre...\textunderscore  + \textunderscore supor\textunderscore )}
\end{itemize}
Supor antecipadamente; fazer supor.
\section{Pressura}
\begin{itemize}
\item {Grp. gram.:f.}
\end{itemize}
\begin{itemize}
\item {Utilização:Ant.}
\end{itemize}
O mesmo que \textunderscore pressa\textunderscore .
\section{Pressurosamente}
\begin{itemize}
\item {Grp. gram.:adv.}
\end{itemize}
De modo pressuroso.
Com pressa; rapidamente.
\section{Prestigiar}
\begin{itemize}
\item {Grp. gram.:v. t.}
\end{itemize}
\begin{itemize}
\item {Proveniência:(Lat. \textunderscore praestigiare\textunderscore )}
\end{itemize}
Dar prestigio a; tornar prestigioso.
\section{Prestígio}
\begin{itemize}
\item {Grp. gram.:m.}
\end{itemize}
\begin{itemize}
\item {Utilização:Fig.}
\end{itemize}
\begin{itemize}
\item {Proveniência:(Lat. \textunderscore praestigium\textunderscore )}
\end{itemize}
Illusão, attribuída a sortilégios.
Illusões, produzidas por meios naturaes.
Fascinação; attacção.
Grande influência; importância social.
\section{Prestigioso}
\begin{itemize}
\item {Grp. gram.:adj.}
\end{itemize}
\begin{itemize}
\item {Proveniência:(Lat. \textunderscore praestigiosus\textunderscore )}
\end{itemize}
Relativo a prestígio ou grande influência.
Respeitado.
\section{Prestímano}
\begin{itemize}
\item {Grp. gram.:m.}
\end{itemize}
\begin{itemize}
\item {Proveniência:(Do lat. \textunderscore praesto\textunderscore  + \textunderscore manus\textunderscore )}
\end{itemize}
O mesmo que \textunderscore prestigiador\textunderscore .
\section{Préstimo}
\begin{itemize}
\item {Grp. gram.:m.}
\end{itemize}
\begin{itemize}
\item {Proveniência:(De \textunderscore prestar\textunderscore )}
\end{itemize}
Qualidade do que presta ou do que é proveitoso; serventia.
Auxílio.
\section{Prestimoniado}
\begin{itemize}
\item {Grp. gram.:m.  e  adj.}
\end{itemize}
Aquelle que tinha prestimónio. Cf. Herculano, \textunderscore Hist. de Port.\textunderscore , IV, 303.
\section{Prestimonial}
\begin{itemize}
\item {Grp. gram.:adj.}
\end{itemize}
Relativo a prestimónio.
\section{Prestimoniário}
\begin{itemize}
\item {Grp. gram.:adj.}
\end{itemize}
\begin{itemize}
\item {Grp. gram.:M.}
\end{itemize}
\begin{itemize}
\item {Utilização:Ant.}
\end{itemize}
O mesmo que \textunderscore prestimonial\textunderscore .
Aquelle que gozava um prestimónio. Cf. P. Carvalho, \textunderscore Chorogr. Port.\textunderscore , III, 238.
\section{Prestimónio}
\begin{itemize}
\item {Grp. gram.:m.}
\end{itemize}
Pensão ou bens, destinados á sustentação de um padre, e separados das rendas de um benefício.
(B. lat. \textunderscore praestimonium\textunderscore )
\section{Prestimoso}
\begin{itemize}
\item {Grp. gram.:adj.}
\end{itemize}
Que tem préstimo; prestante: \textunderscore prestimoso escritor\textunderscore .
\section{Prestíssimo}
\begin{itemize}
\item {Grp. gram.:adv.}
\end{itemize}
\begin{itemize}
\item {Grp. gram.:M.}
\end{itemize}
\begin{itemize}
\item {Proveniência:(De \textunderscore presto\textunderscore )}
\end{itemize}
Com grande rapidez, (falando-se do andamento musical).
Trecho, em que há esse andamento.
\section{Préstite}
\begin{itemize}
\item {Grp. gram.:m.}
\end{itemize}
\begin{itemize}
\item {Proveniência:(Lat. \textunderscore praestes\textunderscore )}
\end{itemize}
Aquelle que entre os Romanos presidia a certos actos solennes.
Antiste. Cf. Castilho, \textunderscore Fastos\textunderscore , III, 19.
\section{Préstito}
\begin{itemize}
\item {Grp. gram.:m.}
\end{itemize}
\begin{itemize}
\item {Proveniência:(Lat. \textunderscore praestitus\textunderscore )}
\end{itemize}
Agrupamento de muitas pessôas em marcha; procissão; cortejo.
\section{Presto}
\begin{itemize}
\item {Grp. gram.:adv.}
\end{itemize}
\begin{itemize}
\item {Grp. gram.:M.}
\end{itemize}
\begin{itemize}
\item {Grp. gram.:Adj.}
\end{itemize}
\begin{itemize}
\item {Proveniência:(Lat. \textunderscore praesto\textunderscore )}
\end{itemize}
Com presteza, (falando-se de um andamento musical).
Trecho, em que há êsse andamento.
Ligeiro, rápido, prestes.
\section{Presumança}
\begin{itemize}
\item {Grp. gram.:f.}
\end{itemize}
\begin{itemize}
\item {Utilização:Ant.}
\end{itemize}
\begin{itemize}
\item {Utilização:Pop.}
\end{itemize}
\begin{itemize}
\item {Proveniência:(De \textunderscore presumar\textunderscore )}
\end{itemize}
O mesmo que \textunderscore presumpção\textunderscore ; vaidade.
\section{Presumar}
\begin{itemize}
\item {Grp. gram.:v. i.}
\end{itemize}
\begin{itemize}
\item {Utilização:Prov.}
\end{itemize}
\begin{itemize}
\item {Utilização:dur.}
\end{itemize}
\begin{itemize}
\item {Utilização:Ant.}
\end{itemize}
\begin{itemize}
\item {Utilização:Pop.}
\end{itemize}
O mesmo que \textunderscore presumir\textunderscore , sentir e mostrar vaidade.
(Cp. \textunderscore presumir\textunderscore )
\section{Presumida}
\begin{itemize}
\item {Grp. gram.:f.}
\end{itemize}
Mulhér, que tem presumpção.
(Fem. de \textunderscore presumido\textunderscore )
\section{Presumido}
\begin{itemize}
\item {Grp. gram.:adj.}
\end{itemize}
\begin{itemize}
\item {Grp. gram.:M.}
\end{itemize}
\begin{itemize}
\item {Proveniência:(De \textunderscore presumir\textunderscore )}
\end{itemize}
Vaidoso; affectado.
Indivíduo, que tem presumpção ou vaidade.
\section{Presumidor}
\begin{itemize}
\item {Grp. gram.:m.  e  adj.}
\end{itemize}
O que presume.
\section{Presumir}
\begin{itemize}
\item {Grp. gram.:v. t.}
\end{itemize}
\begin{itemize}
\item {Grp. gram.:V. i.}
\end{itemize}
\begin{itemize}
\item {Proveniência:(Lat. \textunderscore praesumare\textunderscore )}
\end{itemize}
Conjecturar.
Suppor.
Entender, baseando-se em certas probabilidades.
Prever
Têr presumpção ou vaidade.
\section{Presumível}
\begin{itemize}
\item {Grp. gram.:adj.}
\end{itemize}
Que se póde presumir; provável.
\section{Presumpção}
\begin{itemize}
\item {Grp. gram.:f.}
\end{itemize}
\begin{itemize}
\item {Proveniência:(Lat. \textunderscore presumptio\textunderscore )}
\end{itemize}
Acto ou effeito de presumir.
Suspeita.
Vaidade; affectação.
\section{Presumpçoso}
\begin{itemize}
\item {Grp. gram.:adj.}
\end{itemize}
\begin{itemize}
\item {Proveniência:(Lat. \textunderscore praesumptiosus\textunderscore )}
\end{itemize}
Que tem presumpção ou vaidade; presumido.
\section{Presumptivo}
\begin{itemize}
\item {Grp. gram.:adj.}
\end{itemize}
\begin{itemize}
\item {Proveniência:(Lat. \textunderscore praesumptivus\textunderscore )}
\end{itemize}
Presumível; presupposto.
\section{Presumptuoso}
\begin{itemize}
\item {Grp. gram.:adj.}
\end{itemize}
\begin{itemize}
\item {Proveniência:(Do lat. \textunderscore praesumptus\textunderscore )}
\end{itemize}
Que tem muita presumpção; mui vaidoso. Cf. Filinto, XVII, 161.
\section{Presunção}
\begin{itemize}
\item {Grp. gram.:f.}
\end{itemize}
\begin{itemize}
\item {Proveniência:(Lat. \textunderscore presumptio\textunderscore )}
\end{itemize}
Acto ou efeito de presumir.
Suspeita.
Vaidade; afectação.
\section{Presunçoso}
\begin{itemize}
\item {Grp. gram.:adj.}
\end{itemize}
\begin{itemize}
\item {Proveniência:(Lat. \textunderscore praesumptiosus\textunderscore )}
\end{itemize}
Que tem presunção ou vaidade; presumido.
\section{Presuntinho}
\begin{itemize}
\item {Grp. gram.:m.}
\end{itemize}
\begin{itemize}
\item {Utilização:Zool.}
\end{itemize}
\begin{itemize}
\item {Proveniência:(De \textunderscore presunto\textunderscore )}
\end{itemize}
Nome de alguns molluscos.
\section{Presuntivo}
\begin{itemize}
\item {Grp. gram.:adj.}
\end{itemize}
\begin{itemize}
\item {Proveniência:(Lat. \textunderscore praesumptivus\textunderscore )}
\end{itemize}
Presumível; pressuposto.
\section{Presunto}
\begin{itemize}
\item {Grp. gram.:m.}
\end{itemize}
\begin{itemize}
\item {Utilização:Gír.}
\end{itemize}
Perna e espádua de porco, salgada e curada ao fumeiro.
Presigo, naco ou porção de perna de porco.
Pessôa morta.
Variedade de pêra de Lamego.
\section{Presuntuoso}
\begin{itemize}
\item {Grp. gram.:adj.}
\end{itemize}
\begin{itemize}
\item {Proveniência:(Do lat. \textunderscore praesumptus\textunderscore )}
\end{itemize}
Que tem muita presunção; mui vaidoso. Cf. Filinto, XVII, 161.
\section{Presuppor}
\begin{itemize}
\item {fónica:su}
\end{itemize}
\begin{itemize}
\item {Grp. gram.:v. t.}
\end{itemize}
\begin{itemize}
\item {Proveniência:(De \textunderscore pre...\textunderscore  + \textunderscore suppor\textunderscore )}
\end{itemize}
Suppor antecipadamente; fazer suppor.
\section{Pretexta}
\begin{itemize}
\item {Grp. gram.:f.}
\end{itemize}
\begin{itemize}
\item {Proveniência:(Lat. \textunderscore praetexta\textunderscore )}
\end{itemize}
Toga romana, franjada de púrpura. Cf. V. de Seabra, \textunderscore Tristes\textunderscore , liv. IV., 49.
\section{Pretextado}
\begin{itemize}
\item {Grp. gram.:adj.}
\end{itemize}
\begin{itemize}
\item {Proveniência:(Lat. \textunderscore praetextatus\textunderscore )}
\end{itemize}
Vestido de pretexta. Cf. Herculano, \textunderscore Hist. de Port.\textunderscore , IV, 13.
\section{Pretextar}
\begin{itemize}
\item {Grp. gram.:v. t.}
\end{itemize}
Dar ou tomar como pretexto ou escusa.
\section{Pretexto}
\begin{itemize}
\item {Grp. gram.:m.}
\end{itemize}
\begin{itemize}
\item {Proveniência:(Lat. \textunderscore praetextus\textunderscore )}
\end{itemize}
Razão ou fundamento supposto ou imaginário; desculpa.
\section{Pretidão}
\begin{itemize}
\item {Grp. gram.:f.}
\end{itemize}
Qualidade do que é preto.
\section{Pretinha}
\begin{itemize}
\item {Grp. gram.:f.}
\end{itemize}
O mesmo que \textunderscore negrinha\textunderscore , ave.
\section{Pretinho}
\begin{itemize}
\item {Grp. gram.:m.}
\end{itemize}
Variedade de uva preta minhota.
\section{Pretipográfico}
\begin{itemize}
\item {Grp. gram.:adj.}
\end{itemize}
\begin{itemize}
\item {Utilização:Neol.}
\end{itemize}
\begin{itemize}
\item {Proveniência:(De \textunderscore pre...\textunderscore  + \textunderscore tipográfico\textunderscore )}
\end{itemize}
Anterior á invenção da imprensa.
\section{Preto}
\begin{itemize}
\item {fónica:prê}
\end{itemize}
\begin{itemize}
\item {Grp. gram.:adj.}
\end{itemize}
\begin{itemize}
\item {Grp. gram.:M.}
\end{itemize}
\begin{itemize}
\item {Grp. gram.:Loc.}
\end{itemize}
\begin{itemize}
\item {Utilização:fam.}
\end{itemize}
Diz-se dos corpos que, absorvendo os raios luminosos, apresentam a côr mais escura.
Negro.
Habitante negro de África.
Escravo preto: \textunderscore eu lá mandarei o preto com a resposta\textunderscore .
Qualidade dos corpos que, absorvendo os raios luminosos, se apresentam escuros ou da côr do ébano.
Fato negro: \textunderscore andar vestido de preto\textunderscore .
Real de cobre, moéda antiga. Cf. Arn. Gama, \textunderscore Última Dona\textunderscore , 408.
\textunderscore Pôr o preto no branco\textunderscore , escrever, tornar escrita uma declaração verbal, um contrato, etc.
\section{Préto}
\begin{itemize}
\item {Grp. gram.:adv.}
\end{itemize}
\begin{itemize}
\item {Utilização:Prov.}
\end{itemize}
\begin{itemize}
\item {Utilização:trasm.}
\end{itemize}
O mesmo que \textunderscore perto\textunderscore , (por metáth.).
\section{Preto-da-rosa}
\begin{itemize}
\item {Grp. gram.:m.}
\end{itemize}
Casta de uva do districto de Leiria.
\section{Preto-martinho}
\begin{itemize}
\item {Grp. gram.:m.}
\end{itemize}
Casta de uva do districto de Leiria.
\section{Pretónico}
\begin{itemize}
\item {Grp. gram.:adj.}
\end{itemize}
\begin{itemize}
\item {Utilização:Gram.}
\end{itemize}
\begin{itemize}
\item {Proveniência:(De \textunderscore pre...\textunderscore  + \textunderscore tónico\textunderscore )}
\end{itemize}
Dis-se da vogal ou sýllaba, que está antes da vogal ou sýllaba tónica de uma palavra.
\section{Pretor}
\begin{itemize}
\item {Grp. gram.:m.}
\end{itemize}
\begin{itemize}
\item {Utilização:Bras}
\end{itemize}
\begin{itemize}
\item {Proveniência:(Lat. \textunderscore praetor\textunderscore )}
\end{itemize}
Antigo magistrado romano.
Na Idade-média, alcaide-mór ou senhor absoluto das terras que lhe eram confiadas.
Magistrado administrativo, policial e judicial.
\section{Pretoria}
\begin{itemize}
\item {Grp. gram.:f.}
\end{itemize}
\begin{itemize}
\item {Utilização:Bras}
\end{itemize}
\begin{itemize}
\item {Proveniência:(De \textunderscore pretor\textunderscore )}
\end{itemize}
Sala, annexa aos conventos, na qual se julgavam pleitos.
Jurisdicção do pretor brasileiro.
Repartição do pretor.
\section{Pretoriana}
\begin{itemize}
\item {Grp. gram.:f.}
\end{itemize}
\begin{itemize}
\item {Proveniência:(De \textunderscore pretoriano\textunderscore )}
\end{itemize}
Guarda do pretório, entre os Romanos. Cf. Castilho, \textunderscore Fastos\textunderscore , 582.
\section{Pretoriano}
\begin{itemize}
\item {Grp. gram.:adj.}
\end{itemize}
\begin{itemize}
\item {Grp. gram.:M.}
\end{itemize}
\begin{itemize}
\item {Proveniência:(Lat. \textunderscore praetorianus\textunderscore )}
\end{itemize}
Relativo ao pretor.
Soldado da guarda pretoriana.
\section{Pretório}
\begin{itemize}
\item {Grp. gram.:m.}
\end{itemize}
\begin{itemize}
\item {Proveniência:(Lat. \textunderscore praestorium\textunderscore )}
\end{itemize}
Tenda do pretor.
Residência ou tribunal do pretor.
\section{Pretoríolo}
\begin{itemize}
\item {Grp. gram.:m.}
\end{itemize}
\begin{itemize}
\item {Proveniência:(Lat. \textunderscore praetoriolum\textunderscore )}
\end{itemize}
Termo romano, de significação incerta: segundo uns, era a casa de campo, (cf. Freund); segundo outros, era, em o navio, a câmmara do commandante, (cf. \textunderscore Diction. Abregé des Antiq.\textunderscore ); segundo outros aínda, era o mesmo que \textunderscore palacete\textunderscore , ao passo que o pretório era um palácio.
\section{Pretypográphico}
\begin{itemize}
\item {Grp. gram.:adj.}
\end{itemize}
\begin{itemize}
\item {Utilização:Neol.}
\end{itemize}
\begin{itemize}
\item {Proveniência:(De \textunderscore pre...\textunderscore  + \textunderscore typográphico\textunderscore )}
\end{itemize}
Anterior á invenção da imprensa.
\section{Prevalência}
\begin{itemize}
\item {Grp. gram.:f.}
\end{itemize}
\begin{itemize}
\item {Proveniência:(Lat. \textunderscore praevalentia\textunderscore )}
\end{itemize}
Qualidade daquelle ou daquillo que prevalece; superioridade. Cf. Júl. Castilho, \textunderscore Lisb. Ant.\textunderscore .
\section{Prevalecer}
\begin{itemize}
\item {Grp. gram.:v. i.}
\end{itemize}
\begin{itemize}
\item {Grp. gram.:V. p.}
\end{itemize}
\begin{itemize}
\item {Proveniência:(Lat. \textunderscore praevalescere\textunderscore )}
\end{itemize}
Têr mais valor.
Preponderar; sobresaír; predominar.
Vingar.
Ensoberbecer-se.
Aproveitar-se.
Tirar utilidade.
\section{Prevaricação}
\begin{itemize}
\item {Grp. gram.:f.}
\end{itemize}
\begin{itemize}
\item {Proveniência:(Lat. \textunderscore praevaricatio\textunderscore )}
\end{itemize}
Acto ou effeito de prevaricar.
\section{Prevaricador}
\begin{itemize}
\item {Grp. gram.:m.  e  adj.}
\end{itemize}
\begin{itemize}
\item {Proveniência:(Lat. \textunderscore praevaricator\textunderscore )}
\end{itemize}
O que prevarica.
\section{Prevaricar}
\begin{itemize}
\item {Grp. gram.:v. i.}
\end{itemize}
\begin{itemize}
\item {Grp. gram.:V. t.}
\end{itemize}
\begin{itemize}
\item {Proveniência:(Lat. \textunderscore praevaricari\textunderscore )}
\end{itemize}
Faltar ao cumprimento do seu dever.
Abusar do exercício das suas funcções, commetendo injustiças ou prejudicando por qualquer fórma os interesses que é obrigado a sustentar.
Perverter.
\section{Prictória}
\begin{itemize}
\item {Grp. gram.:f.}
\end{itemize}
Planta da serra de Sintra.
\section{Prima}
\begin{itemize}
\item {Grp. gram.:f.}
\end{itemize}
(Flexão fem. de \textunderscore primo\textunderscore ^1)
\section{Prima}
\begin{itemize}
\item {Grp. gram.:f.}
\end{itemize}
\begin{itemize}
\item {Grp. gram.:M.  e  f.}
\end{itemize}
\begin{itemize}
\item {Grp. gram.:M.}
\end{itemize}
\begin{itemize}
\item {Utilização:Ant.}
\end{itemize}
\begin{itemize}
\item {Proveniência:(De \textunderscore primo\textunderscore ^2)}
\end{itemize}
Primeira e mais delgada corda de alguns instrumentos músicos.
Fêmea do açor do falcão, do gavião, ou do esmerilhão: \textunderscore ...um falcão prima, um açor prima...\textunderscore 
O primeiro açor de uma ninhada. Cf. Fernandes, \textunderscore Caça de Altan.\textunderscore 
\textunderscore Horas de prima\textunderscore  ou \textunderscore quarto de prima\textunderscore , a primeira vigia de noite, das 9 ás 11 horas, nos arraiaes e navios. Cf. \textunderscore Rot. do Mar-Vermelho\textunderscore , 252, etc.
\section{Primacia}
\begin{itemize}
\item {Grp. gram.:f.}
\end{itemize}
O mesmo que \textunderscore primazia\textunderscore .
\section{Primacial}
\begin{itemize}
\item {Grp. gram.:f.}
\end{itemize}
\begin{itemize}
\item {Proveniência:(De \textunderscore primacia\textunderscore )}
\end{itemize}
Relativo ao primaz.
Em que há primazia; que é de qualidade superior.
\section{Primacialmente}
\begin{itemize}
\item {Grp. gram.:adv.}
\end{itemize}
Com primazia; de modo primacial.
\section{Primado}
\begin{itemize}
\item {Grp. gram.:m.}
\end{itemize}
\begin{itemize}
\item {Proveniência:(Lat. \textunderscore primatus\textunderscore )}
\end{itemize}
O mesmo que \textunderscore primazia\textunderscore .
Prioridade.
Superioridade.
\section{Prima-donna}
\begin{itemize}
\item {Grp. gram.:f.}
\end{itemize}
Cantora principal de uma ópera.
(Loc. it.)
\section{Primagem}
\begin{itemize}
\item {Grp. gram.:f.}
\end{itemize}
\begin{itemize}
\item {Proveniência:(Fr. \textunderscore primage\textunderscore )}
\end{itemize}
Percentagem, que se paga ao capitão de um navio.
\section{Primar}
\begin{itemize}
\item {Grp. gram.:v. i.}
\end{itemize}
\begin{itemize}
\item {Proveniência:(De \textunderscore primo\textunderscore ^2)}
\end{itemize}
Sêr o primeiro, têr a primazia ou a preferência.
Mostrar-se notável, o mais notável:«\textunderscore não primava a rainha no affecto ao ministro.\textunderscore »Latino, \textunderscore Hist. Pol. e Mil.\textunderscore , I, 233.
\section{Primariças}
\begin{itemize}
\item {Grp. gram.:f. pl.}
\end{itemize}
\begin{itemize}
\item {Utilização:Ant.}
\end{itemize}
\begin{itemize}
\item {Proveniência:(De \textunderscore primário\textunderscore )}
\end{itemize}
Foro, que consistia em dar as primeiras lampreias, que se pescavam.
\section{Primário}
\begin{itemize}
\item {Grp. gram.:adj.}
\end{itemize}
\begin{itemize}
\item {Proveniência:(Lat. \textunderscore primarius\textunderscore )}
\end{itemize}
Primeiro; que antecede outro.
Principal, fundamental.
\section{Primatas}
\begin{itemize}
\item {Grp. gram.:m. pl.}
\end{itemize}
(V.primates)
\section{Primates}
\begin{itemize}
\item {Grp. gram.:m. pl.}
\end{itemize}
\begin{itemize}
\item {Utilização:Ant.}
\end{itemize}
\begin{itemize}
\item {Utilização:Zool.}
\end{itemize}
\begin{itemize}
\item {Proveniência:(Lat. \textunderscore primates\textunderscore )}
\end{itemize}
Homens da maior nobreza.
Optimates.
Família de mammíferos, que comprehende o homem, e os animaes que mais semelhança têm com êlle.
\section{Primavera}
\begin{itemize}
\item {Grp. gram.:f.}
\end{itemize}
\begin{itemize}
\item {Utilização:Poét.}
\end{itemize}
\begin{itemize}
\item {Utilização:Ant.}
\end{itemize}
\begin{itemize}
\item {Proveniência:(Do lat. \textunderscore primus\textunderscore  + \textunderscore ver\textunderscore )}
\end{itemize}
Primeira estação do anno, a qual começa em 19 a 21 de Março.
Designação vulgar da parte do anno entre os grandes frios e os grandes calores.
Anno: \textunderscore conta apenas 20 primaveras\textunderscore .
Juventude: \textunderscore na primavera da vida\textunderscore .
Nome de algumas plantas.
Espécie de tecido.
\section{Primavera-de-flôres}
\begin{itemize}
\item {Grp. gram.:f.}
\end{itemize}
Espécie de tecido antigo de seda:«\textunderscore Deixo um vestido de hũa seda que chamão primavera-de-flôres...\textunderscore »(De um testamento do séc. XVII)
\section{Primaveral}
\begin{itemize}
\item {Grp. gram.:adj.}
\end{itemize}
Próprio da primavera; relativo á primavera.
\section{Primaverar}
\begin{itemize}
\item {Grp. gram.:v. i.}
\end{itemize}
\begin{itemize}
\item {Utilização:Neol.}
\end{itemize}
Passar a primavera.
Gozar a estação primaveral:«\textunderscore em abril iam primaverar na quinta do Flórido...\textunderscore »Camillo, \textunderscore Volcões\textunderscore , 152.
\section{Primaveril}
\begin{itemize}
\item {Grp. gram.:adj.}
\end{itemize}
(V.primaveral). Cf. Camillo, \textunderscore Nárcot.\textunderscore , I, 187.
\section{Primavero}
\begin{itemize}
\item {Grp. gram.:adj.}
\end{itemize}
O mesmo que \textunderscore primaveral\textunderscore .
\section{Primaz}
\begin{itemize}
\item {Grp. gram.:m.}
\end{itemize}
\begin{itemize}
\item {Grp. gram.:Adj.}
\end{itemize}
\begin{itemize}
\item {Proveniência:(Do lat. hyp. \textunderscore primatius\textunderscore , de \textunderscore primus\textunderscore )}
\end{itemize}
O principal, entre os Bispos e Arcebispos de uma região: \textunderscore primaz das Espanhas\textunderscore .
Que occupa o primeiro lugar. Cf. Latino, \textunderscore Humboldt\textunderscore , 489.
\section{Primazia}
\begin{itemize}
\item {Grp. gram.:f.}
\end{itemize}
\begin{itemize}
\item {Utilização:Ext.}
\end{itemize}
Dignidade de primaz.
Prioridade.
Superioridade, excellência.
Competência, rivalidade.
\section{Primeira}
\begin{itemize}
\item {Grp. gram.:f. Loc. adv.}
\end{itemize}
\begin{itemize}
\item {Proveniência:(De \textunderscore primeiro\textunderscore )}
\end{itemize}
\textunderscore Á primeira\textunderscore , ao princípio, logo á primeira vista.
\textunderscore Jôgo da primeira\textunderscore , jôgo de cartas, em que se distribuem quatro a cada parceiro; mão de quatro cartas, uma de cada naipe, naquelle jôgo.
\section{Primeiramente}
\begin{itemize}
\item {Grp. gram.:adv.}
\end{itemize}
\begin{itemize}
\item {Proveniência:(De \textunderscore primeiro\textunderscore )}
\end{itemize}
Em primeiro lugar; antes de outra coisa; antes de tudo.
\section{Primo}
\begin{itemize}
\item {Grp. gram.:adj.}
\end{itemize}
\begin{itemize}
\item {Utilização:Ant.}
\end{itemize}
\begin{itemize}
\item {Proveniência:(Lat. \textunderscore primus\textunderscore )}
\end{itemize}
Primeiro.
Excellente.
Diz-se de um número que só é divisível por si ou pela unidade.
E diz-se de uma obra que é excellente ou a primeira do seu gênero.
Perfeito.; aperfeiçoado. Cf. \textunderscore Peregrinação\textunderscore , CV.
\section{Primo}
\begin{itemize}
\item {fónica:primò}
\end{itemize}
\begin{itemize}
\item {Grp. gram.:adv.}
\end{itemize}
\begin{itemize}
\item {Proveniência:(T. lat.)}
\end{itemize}
O mesmo que \textunderscore primeiramente\textunderscore .
\section{Primogênito}
\begin{itemize}
\item {Grp. gram.:m.  e  adj.}
\end{itemize}
\begin{itemize}
\item {Proveniência:(Lat. \textunderscore primogenitus\textunderscore )}
\end{itemize}
O que foi gerado antes dos outros; filho mais velho.
\section{Primogenitor}
\begin{itemize}
\item {Grp. gram.:m.}
\end{itemize}
Pai do primogênito. Cf. Rui Barb., \textunderscore Réplica\textunderscore , II, 158.
\section{Primogenitura}
\begin{itemize}
\item {Grp. gram.:f.}
\end{itemize}
Qualidade do que é primogênito.
\section{Primo-glacial}
\begin{itemize}
\item {Grp. gram.:adj.}
\end{itemize}
\begin{itemize}
\item {Utilização:Geol.}
\end{itemize}
Diz-se de uma das cinco phases, que constituem o período plistoceno.
\section{Primor}
\begin{itemize}
\item {Grp. gram.:m.}
\end{itemize}
\begin{itemize}
\item {Proveniência:(Lat. \textunderscore primor\textunderscore )}
\end{itemize}
Qualidade superior.
Excellência, perfeição: \textunderscore cantar com primor\textunderscore .
Belleza.
Delicadeza: \textunderscore o primor do seu trato\textunderscore .
\section{Primordial}
\begin{itemize}
\item {Grp. gram.:adj.}
\end{itemize}
\begin{itemize}
\item {Proveniência:(Lat. \textunderscore primordialis\textunderscore )}
\end{itemize}
Relativo a primórdio; primitivo.
Originário; primeiro.
\section{Primordialmente}
\begin{itemize}
\item {Grp. gram.:adv.}
\end{itemize}
\begin{itemize}
\item {Proveniência:(De \textunderscore primordial\textunderscore )}
\end{itemize}
O mesmo que \textunderscore primitivamente\textunderscore .
\section{Primórdio}
\begin{itemize}
\item {Grp. gram.:m.}
\end{itemize}
\begin{itemize}
\item {Proveniência:(Lat. \textunderscore primordium\textunderscore )}
\end{itemize}
O que se ordena ou organiza primeiro.
Origem, fonte: \textunderscore os primórdios da nossa línguagem\textunderscore .
\section{Primorosamente}
\begin{itemize}
\item {Grp. gram.:adv.}
\end{itemize}
De modo primoroso; com primor, excellentemente.
\section{Primoroso}
\begin{itemize}
\item {Grp. gram.:adj.}
\end{itemize}
Em que há primor; perfeito; excellente; distinto.
\section{Prímula}
\begin{itemize}
\item {Grp. gram.:f.}
\end{itemize}
\begin{itemize}
\item {Proveniência:(Lat. \textunderscore primula\textunderscore )}
\end{itemize}
Nome scientífico de um gênero de plantas, mais conhecidas por \textunderscore primaveras\textunderscore .
\section{Primuláceas}
\begin{itemize}
\item {Grp. gram.:f. pl.}
\end{itemize}
Família de plantas, que têm por typo a prímula.
(Fem. pl. de \textunderscore primuláceo\textunderscore )
\section{Primuláceo}
\begin{itemize}
\item {Grp. gram.:adj.}
\end{itemize}
Relativo ou semelhante á prímula.
\section{Primulina}
\begin{itemize}
\item {Grp. gram.:f.}
\end{itemize}
Producto pharmacêutico, extrahido da prímula.
\section{Príncepe}
\begin{itemize}
\item {Grp. gram.:m.}
\end{itemize}
(Fórma ant. e pop. de \textunderscore príncipe\textunderscore )
\section{Princeps}
\begin{itemize}
\item {fónica:princèps}
\end{itemize}
\begin{itemize}
\item {Grp. gram.:adj.}
\end{itemize}
\begin{itemize}
\item {Proveniência:(T. lat.)}
\end{itemize}
Diz-se da primeira edição de uma obra.
\section{Princês}
\begin{itemize}
\item {Grp. gram.:m.}
\end{itemize}
Designação depreciativa ou irónica de Príncipe. Cf. G. Braga, \textunderscore Mal da Delf.\textunderscore , 217.
(Cp. \textunderscore princesa\textunderscore )
\section{Princesa}
\begin{itemize}
\item {Grp. gram.:f.}
\end{itemize}
\begin{itemize}
\item {Utilização:Ext.}
\end{itemize}
Mulhér de Príncipe.
Soberana de um principado.
Filha de Rei.
Raínha ou Soberana.
A pessôa ou coisa do sexo feminino, que é mais distinta na sua categoria: \textunderscore a princesa das nossas poetisas\textunderscore .
(Cast. \textunderscore princesa\textunderscore )
\section{Princessa}
\begin{itemize}
\item {fónica:cê}
\end{itemize}
\begin{itemize}
\item {Grp. gram.:f.}
\end{itemize}
\begin{itemize}
\item {Utilização:Ant.}
\end{itemize}
O mesmo que \textunderscore Princesa\textunderscore . Cf. \textunderscore Eufrosina\textunderscore , 43.
\section{Principado}
\begin{itemize}
\item {Grp. gram.:m.}
\end{itemize}
\begin{itemize}
\item {Grp. gram.:Pl.}
\end{itemize}
\begin{itemize}
\item {Utilização:Theol.}
\end{itemize}
\begin{itemize}
\item {Proveniência:(Lat. \textunderscore principatus\textunderscore )}
\end{itemize}
Dignidade de Príncipe.
Território ou Estado, cujo Soberano é um Príncipe ou Princesa.
Um dos nove coros dos anjos.
\section{Principal}
\begin{itemize}
\item {Grp. gram.:adj.}
\end{itemize}
\begin{itemize}
\item {Grp. gram.:M.}
\end{itemize}
\begin{itemize}
\item {Proveniência:(Lat. \textunderscore principalis\textunderscore )}
\end{itemize}
Que está em primeiro lugar.
Essencial: \textunderscore condição principal\textunderscore .
Que é o mais importante, o mais notável.
Primeiro.
Superior de uma communidade religiosa.
A pessôa mais importante; magnate: \textunderscore os principaes da cidade\textunderscore .
Aquillo que é mais importante, precípuo.
\section{Principalidade}
\begin{itemize}
\item {Grp. gram.:f.}
\end{itemize}
De modo principal.
\section{Príncipe}
\begin{itemize}
\item {Grp. gram.:m.}
\end{itemize}
\begin{itemize}
\item {Proveniência:(Lat. \textunderscore princeps\textunderscore )}
\end{itemize}
Filho ou membro de uma família reinante.
Filho mais velho do Rei.
Título dos Soberanos de alguns pequenos Estados.
Consorte da Raínha, em algumas nações.
Título de nobreza em alguns países.
O que é primeiro ou o mais notável em talento ou em outras qualidades: \textunderscore o príncipe dos poétas portugueses\textunderscore .
\section{Prisão}
\begin{itemize}
\item {Grp. gram.:f.}
\end{itemize}
\begin{itemize}
\item {Proveniência:(Do b. lat. \textunderscore presio\textunderscore )}
\end{itemize}
Acto ou effeito de prender.
Captura.
Cadeia, cárcere.
Pena de detenção, que tem de sêr expiada na cadeia.
Encerramento, clausura.
Vínculo.
Aquillo que attrái ou cativa o espírito ou o coração.
Ave que prende o falcão, o açor ou o gavião. Cf. Fernandes, \textunderscore Caça de Altan\textunderscore .
\textunderscore Prisão de ventre\textunderscore , constipação intestinal; difficuldade de evacuar.
\section{Prisão-de-deus}
\begin{itemize}
\item {Grp. gram.:f.}
\end{itemize}
\begin{itemize}
\item {Utilização:Ant.}
\end{itemize}
Qualquer doença.
\section{Prisca}
\begin{itemize}
\item {Grp. gram.:f.}
\end{itemize}
\begin{itemize}
\item {Utilização:Bras}
\end{itemize}
Ponta de cigarro, pisca.
\section{Priscilianismo}
\begin{itemize}
\item {Grp. gram.:m.}
\end{itemize}
Heresia de Priscilliano, hereje hispânico do século IV, o qual sustentava que a alma do homem vem do céu e que o princípio do mal a junta ao corpo.
\section{Priscilianista}
\begin{itemize}
\item {Grp. gram.:m.}
\end{itemize}
Sectário do priscilianismo.
\section{Prisciliano}
\begin{itemize}
\item {Grp. gram.:m.}
\end{itemize}
O mesmo que \textunderscore priscilianista\textunderscore . Cf. Ol. Martins, \textunderscore Filhos de D. João I\textunderscore , 100.
\section{Priscillianismo}
\begin{itemize}
\item {Grp. gram.:m.}
\end{itemize}
Heresia de Priscilliano, hereje hispânico do século IV, o qual sustentava que a alma do homem vem do céu e que o princípio do mal a junta ao corpo.
\section{Priscillianista}
\begin{itemize}
\item {Grp. gram.:m.}
\end{itemize}
Sectário do priscillianismo.
\section{Priscilliano}
\begin{itemize}
\item {Grp. gram.:m.}
\end{itemize}
O mesmo que \textunderscore priscillianista\textunderscore . Cf. Ol. Martins, \textunderscore Filhos de D. João I\textunderscore , 100.
\section{Prisco}
\begin{itemize}
\item {Grp. gram.:adj.}
\end{itemize}
\begin{itemize}
\item {Utilização:Poét.}
\end{itemize}
\begin{itemize}
\item {Proveniência:(Lat. \textunderscore priscus\textunderscore )}
\end{itemize}
Antigo.
Relativo a tempos passados.
\section{Prisional}
\begin{itemize}
\item {Grp. gram.:adj.}
\end{itemize}
Relativo a prisão; carcerário: \textunderscore regime prisional\textunderscore .
\section{Prisioneiro}
\begin{itemize}
\item {Grp. gram.:m.}
\end{itemize}
\begin{itemize}
\item {Proveniência:(De \textunderscore prisão\textunderscore )}
\end{itemize}
Indivíduo, privado da liberdade.
Indivíduo preso, encarcerado.
Indivíduo aprisionado em occasião de guerra.
\section{Prisma}
\begin{itemize}
\item {Grp. gram.:m.}
\end{itemize}
\begin{itemize}
\item {Utilização:Geom.}
\end{itemize}
\begin{itemize}
\item {Utilização:Geom.}
\end{itemize}
\begin{itemize}
\item {Proveniência:(Gr. \textunderscore prisma\textunderscore )}
\end{itemize}
Figura sólida, que tem duas superfícies iguaes e parallelas.
Polyedro, que tem por base dois polýgonos iguaes e parallelos, cujos lados homólogos são unidos por parallelogramos.
É triangular, quadrangular, etc., segundo as bases são dois triângulos, dois quadriláteros, etc.
Em Phýsica, é triangular e formado de crystal, vidro ou outra substância transparente.
Designação imprópria de quaesquer instrumentos, destinados a produzir ou observar a refracção dos raios luminosos.
\section{Prismado}
\begin{itemize}
\item {Grp. gram.:adj.}
\end{itemize}
\begin{itemize}
\item {Utilização:Neol.}
\end{itemize}
Que tem fórma de prisma.
Observado por prisma. Cf. Alves Mendes, \textunderscore Herculano\textunderscore , 12.
\section{Prismática}
\begin{itemize}
\item {Grp. gram.:f.}
\end{itemize}
\begin{itemize}
\item {Proveniência:(De \textunderscore prismático\textunderscore )}
\end{itemize}
Moldura polygónica da architectura ogival.
\section{Prismático}
\begin{itemize}
\item {Grp. gram.:adj.}
\end{itemize}
Que tem fórma de prisma; relativo a prisma.
\section{Prismatina}
\begin{itemize}
\item {Grp. gram.:f.}
\end{itemize}
\begin{itemize}
\item {Utilização:Miner.}
\end{itemize}
\begin{itemize}
\item {Proveniência:(De \textunderscore prisma\textunderscore )}
\end{itemize}
Silicato de alumínio e magnésio.
\section{Prismatização}
\begin{itemize}
\item {Grp. gram.:f.}
\end{itemize}
Disposição em fórma de prisma.
(Cp. \textunderscore prismatizado\textunderscore )
\section{Prismatizado}
\begin{itemize}
\item {Grp. gram.:adj.}
\end{itemize}
\begin{itemize}
\item {Proveniência:(De \textunderscore prismático\textunderscore )}
\end{itemize}
Disposto em prisma.
\section{Prismatóide}
\begin{itemize}
\item {Grp. gram.:adj.}
\end{itemize}
\begin{itemize}
\item {Proveniência:(Gr. \textunderscore prismatoides\textunderscore )}
\end{itemize}
Que deriva de um prisma.
\section{Prismóide}
\begin{itemize}
\item {Grp. gram.:adj.}
\end{itemize}
\begin{itemize}
\item {Proveniência:(Gr. \textunderscore prismoeides\textunderscore )}
\end{itemize}
Que tem fórma análoga á do prisma.
\section{Prista}
\begin{itemize}
\item {Grp. gram.:m.}
\end{itemize}
\begin{itemize}
\item {Utilização:Des.}
\end{itemize}
\begin{itemize}
\item {Proveniência:(Lat. \textunderscore prista\textunderscore )}
\end{itemize}
Aquelle que corta com serra; serrador.
\section{Prístino}
\begin{itemize}
\item {Grp. gram.:adj.}
\end{itemize}
\begin{itemize}
\item {Utilização:Poét.}
\end{itemize}
\begin{itemize}
\item {Proveniência:(Lat. \textunderscore pristinus\textunderscore )}
\end{itemize}
O mesmo que \textunderscore prisco\textunderscore .
\section{Pristipoma}
\begin{itemize}
\item {Grp. gram.:m.}
\end{itemize}
\begin{itemize}
\item {Proveniência:(Do gr. \textunderscore pristis\textunderscore  + \textunderscore poma\textunderscore )}
\end{itemize}
Gênero de peixes, notáveis pela proeminência do preopérculo.
\section{Pristipomátidas}
\begin{itemize}
\item {Grp. gram.:m. pl.}
\end{itemize}
Família de peixes acanthopterýgios, que tem por typo o pristipoma.
\section{Pristipomatídeos}
\begin{itemize}
\item {Grp. gram.:m. pl.}
\end{itemize}
Família de peixes acanthopterýgios, que tem por typo o pristipoma.
\section{Pristipomídeos}
\begin{itemize}
\item {Grp. gram.:m. pl.}
\end{itemize}
O mesmo que \textunderscore pristipomatídeos\textunderscore .
\section{Pristlena}
\begin{itemize}
\item {Grp. gram.:f.}
\end{itemize}
Gênero de plantas leguminosas.
\section{Prítica}
\begin{itemize}
\item {Grp. gram.:f.}
\end{itemize}
\begin{itemize}
\item {Utilização:Prov.}
\end{itemize}
\begin{itemize}
\item {Utilização:alg.}
\end{itemize}
\begin{itemize}
\item {Utilização:alent.}
\end{itemize}
Temão, a que se atrelam os animaes que puxam o carro. Cf. Baganha, \textunderscore Hyg. Pec.\textunderscore , 77. (Metáth. de \textunderscore pírtiga\textunderscore )
\section{Probabilista}
\begin{itemize}
\item {Grp. gram.:m.  e  f.}
\end{itemize}
\begin{itemize}
\item {Proveniência:(Do lat. \textunderscore probabilis\textunderscore )}
\end{itemize}
Pessôa, que segue o probabilismo.
\section{Probante}
\begin{itemize}
\item {Grp. gram.:adj.}
\end{itemize}
\begin{itemize}
\item {Proveniência:(Lat. \textunderscore probans\textunderscore )}
\end{itemize}
Que prova.
\section{Probático}
\begin{itemize}
\item {Grp. gram.:adj.}
\end{itemize}
\begin{itemize}
\item {Proveniência:(Gr. \textunderscore probatikos\textunderscore )}
\end{itemize}
Diz-se de uma piscina, em que se reservava água, junto do templo de Jerusalém, e em que se lavavam os animaes destinados ao sacrifício.
\section{Probatório}
\begin{itemize}
\item {Grp. gram.:adj.}
\end{itemize}
\begin{itemize}
\item {Proveniência:(Lat. \textunderscore probatorius\textunderscore )}
\end{itemize}
Relativo a prova; que contém prova; que serve de prova.
\section{Probidade}
\begin{itemize}
\item {Grp. gram.:f.}
\end{itemize}
\begin{itemize}
\item {Proveniência:(Lat. \textunderscore probitas\textunderscore )}
\end{itemize}
Qualidade do que é probo.
Integridade de carácter; honradez.
Sentimento da dignidade pessoal; pundonor.
\section{Problema}
\begin{itemize}
\item {Grp. gram.:m.}
\end{itemize}
\begin{itemize}
\item {Proveniência:(Lat. \textunderscore problema\textunderscore )}
\end{itemize}
Questão mathemática, proposta para se lhe obter a solução.
Questão; dúvida: \textunderscore problemas da linguagem\textunderscore .
Proposta duvidosa, que póde ter muitas soluções.
O que é diffícil de explicar.
\section{Problematicamente}
\begin{itemize}
\item {Grp. gram.:adv.}
\end{itemize}
De modo problemático; com incerteza.
\section{Problemático}
\begin{itemize}
\item {Grp. gram.:adj.}
\end{itemize}
\begin{itemize}
\item {Proveniência:(Lat. \textunderscore problematicus\textunderscore )}
\end{itemize}
Relativo a problema; que tem o carácter de problema.
\section{Problematizar}
\begin{itemize}
\item {Grp. gram.:v. t.}
\end{itemize}
Tornar problemático, dar fórma de problema a.
\section{Problemista}
\begin{itemize}
\item {Grp. gram.:m.}
\end{itemize}
\begin{itemize}
\item {Utilização:Bras}
\end{itemize}
\begin{itemize}
\item {Proveniência:(De \textunderscore problema\textunderscore )}
\end{itemize}
Jogador de xadrez. Cf. \textunderscore Notícia\textunderscore , do Rio, de 3-XI 900.
\section{Probo}
\begin{itemize}
\item {Grp. gram.:m.}
\end{itemize}
\begin{itemize}
\item {Proveniência:(Lat. \textunderscore probus\textunderscore )}
\end{itemize}
Que tem carácter integro; justo; honrado; recto.
\section{Probóscida}
\begin{itemize}
\item {Grp. gram.:f.}
\end{itemize}
\begin{itemize}
\item {Proveniência:(Gr. \textunderscore proboskis\textunderscore )}
\end{itemize}
Tromba do elephante.
Órgão oral dos insectos dípteros.
\section{Probóscide}
\begin{itemize}
\item {Grp. gram.:f.}
\end{itemize}
\begin{itemize}
\item {Proveniência:(Gr. \textunderscore proboskis\textunderscore )}
\end{itemize}
Tromba do elephante.
Órgão oral dos insectos dípteros.
\section{Proboscídeo}
\begin{itemize}
\item {Grp. gram.:adj.}
\end{itemize}
\begin{itemize}
\item {Grp. gram.:M. pl.}
\end{itemize}
\begin{itemize}
\item {Proveniência:(Do gr. \textunderscore proboskis\textunderscore  + \textunderscore eidos\textunderscore )}
\end{itemize}
Que tem o nariz na fórma de tromba.
Família de mammíferos proboscídeos, como o elephante.
\section{Proboscidiano}
\begin{itemize}
\item {Grp. gram.:adj.}
\end{itemize}
O mesmo que \textunderscore proboscídeo\textunderscore . Cf. Júl. Diniz, \textunderscore Morgadinha\textunderscore , 229.
\section{Proboscídio}
\begin{itemize}
\item {Grp. gram.:adj.}
\end{itemize}
O mesmo que \textunderscore proboscídeo\textunderscore . Cf. Júl. Diniz, \textunderscore Morgadinha\textunderscore , 229.
\section{Proboste}
\textunderscore m.\textunderscore  (e der.)
(Fórma ant. de \textunderscore preboste\textunderscore , etc)
\section{Procace}
\begin{itemize}
\item {Grp. gram.:adj.}
\end{itemize}
O mesmo que \textunderscore procaz\textunderscore . Cf. Filinto, II, 202.
\section{Procacidade}
\begin{itemize}
\item {Grp. gram.:f.}
\end{itemize}
\begin{itemize}
\item {Proveniência:(Lat. \textunderscore procacitas\textunderscore )}
\end{itemize}
Qualidade do que é procaz.
\section{Procarpal}
\begin{itemize}
\item {Grp. gram.:adj.}
\end{itemize}
Relativo ao procarpo.
\section{Procárpico}
\begin{itemize}
\item {Grp. gram.:adj.}
\end{itemize}
Relativo ao procarpo.
\section{Procarpo}
\begin{itemize}
\item {Grp. gram.:m.}
\end{itemize}
\begin{itemize}
\item {Utilização:Anat.}
\end{itemize}
\begin{itemize}
\item {Proveniência:(De \textunderscore pro...\textunderscore  + \textunderscore carpo\textunderscore )}
\end{itemize}
Série superior dos ossos do carpo.
\section{Procaz}
\begin{itemize}
\item {Grp. gram.:adj.}
\end{itemize}
\begin{itemize}
\item {Proveniência:(Lat. \textunderscore procax\textunderscore )}
\end{itemize}
Impudente; petulante; insolente; descarado.
\section{Procedência}
\begin{itemize}
\item {Grp. gram.:f.}
\end{itemize}
Acto ou effeito de proceder.
Qualidade do que é procedente.
Lugar, donde alguém ou alguma coisa procede; origem.
\section{Procedente}
\begin{itemize}
\item {Grp. gram.:adj.}
\end{itemize}
\begin{itemize}
\item {Proveniência:(Lat. \textunderscore procedens\textunderscore )}
\end{itemize}
Que procede; que provém; oriundo.
Consequente, lógico: \textunderscore argumentos procedentes\textunderscore .
\section{Proceder}
\begin{itemize}
\item {Grp. gram.:v. i.}
\end{itemize}
\begin{itemize}
\item {Grp. gram.:M.}
\end{itemize}
\begin{itemize}
\item {Proveniência:(Lat. \textunderscore procedere\textunderscore )}
\end{itemize}
Andar, proseguir.
Acontecer.
Deduzir-se.
Instaurar processo judicial.
Obrar.
Portar-se, comportar-se: \textunderscore tens procedido mal\textunderscore .
Têr começo ou origem.
Provir.
Modo de vida, comportamento.
Acções: \textunderscore é nobre aquelle proceder\textunderscore .
\section{Procedido}
\begin{itemize}
\item {Grp. gram.:adj.}
\end{itemize}
\begin{itemize}
\item {Proveniência:(De \textunderscore proceder\textunderscore )}
\end{itemize}
Que procede ou que se comporta (bem ou mal).
Resultante; que é consequente.
\section{Procedimento}
\begin{itemize}
\item {Grp. gram.:m.}
\end{itemize}
Acto ou effeito de proceder.
Mode de viver, comportamento.
O mesmo que \textunderscore processo\textunderscore .
\section{Proceleusmático}
\begin{itemize}
\item {Grp. gram.:m.  e  adj.}
\end{itemize}
\begin{itemize}
\item {Proveniência:(Lat. \textunderscore proceleusmaticus\textunderscore )}
\end{itemize}
Diz-se de um pé de verso grego ou latino, composto de quatro súllabas breves.
\section{Procélico}
\begin{itemize}
\item {Grp. gram.:adj.}
\end{itemize}
\begin{itemize}
\item {Utilização:Zool.}
\end{itemize}
\begin{itemize}
\item {Proveniência:(Do gr. \textunderscore pro\textunderscore  + \textunderscore koilon\textunderscore )}
\end{itemize}
Diz-se da vértebra, cujo corpo é côncavo adeante e convexo atrás.
\section{Proclama}
\begin{itemize}
\item {Grp. gram.:m.}
\end{itemize}
\begin{itemize}
\item {Proveniência:(De \textunderscore proclamar\textunderscore )}
\end{itemize}
Cada um dos pregões de casamento, lidos na egreja; proclamação.
\section{Proclamação}
\begin{itemize}
\item {Grp. gram.:f.}
\end{itemize}
\begin{itemize}
\item {Proveniência:(Lat. \textunderscore proclamatio\textunderscore )}
\end{itemize}
Acto ou effeito de proclamar.
\section{Proclamador}
\begin{itemize}
\item {Grp. gram.:m.  e  adj.}
\end{itemize}
\begin{itemize}
\item {Proveniência:(Lat. \textunderscore proclamator\textunderscore )}
\end{itemize}
O que proclama.
\section{Proclamar}
\begin{itemize}
\item {Grp. gram.:v. t.}
\end{itemize}
\begin{itemize}
\item {Proveniência:(Lat. \textunderscore proclamare\textunderscore )}
\end{itemize}
Annunciar em público e em voz alta.
Acclamar.
Publicar.
Manifestar-se publicamente a favor de.
Promulgar.
Qualificar em público.
\section{Proclamatório}
\begin{itemize}
\item {Grp. gram.:adj.}
\end{itemize}
Que envolve proclamação; proclamador. Cf. Castilho, \textunderscore Volcões\textunderscore , 130.
\section{Proclíse}
\begin{itemize}
\item {Grp. gram.:f.}
\end{itemize}
\begin{itemize}
\item {Utilização:Gram.}
\end{itemize}
\begin{itemize}
\item {Proveniência:(Do gr. \textunderscore proklinein\textunderscore )}
\end{itemize}
Emprêgo ou qualidade de palavra proclítica.
\section{Proclítica}
\begin{itemize}
\item {Grp. gram.:f.}
\end{itemize}
\begin{itemize}
\item {Utilização:Gram.}
\end{itemize}
\begin{itemize}
\item {Proveniência:(De \textunderscore proclítico\textunderscore )}
\end{itemize}
Palavra que, anteposta a outra, parece formar com esta uma só, perdendo o seu accento: \textunderscore o que João me disse; succede a cada hora; por causa de nós\textunderscore . Cf. J. Ribeiro, \textunderscore Diccion. Gram.\textunderscore .
\section{Procliticamente}
\begin{itemize}
\item {Grp. gram.:adv.}
\end{itemize}
Á maneira de proclítica; com emprêgo de proclíse.
\section{Proclítico}
\begin{itemize}
\item {Grp. gram.:m.  e  adj.}
\end{itemize}
\begin{itemize}
\item {Utilização:Gram.}
\end{itemize}
\begin{itemize}
\item {Proveniência:(T. criado pelo gram. al. Hermann e der. do gr. \textunderscore proklinen\textunderscore , inclinar para deante)}
\end{itemize}
Diz-se da palavra que, anteposta a outra, fica sujeita é accentuação desta, formando ambas como uma só palavra.
\section{Proclive}
\begin{itemize}
\item {Grp. gram.:adj.}
\end{itemize}
\begin{itemize}
\item {Proveniência:(Lat. \textunderscore proclivis\textunderscore )}
\end{itemize}
Inclinado para deante.
\section{Proclividade}
\begin{itemize}
\item {Grp. gram.:f.}
\end{itemize}
\begin{itemize}
\item {Proveniência:(Lat. \textunderscore proclivitas\textunderscore )}
\end{itemize}
Estado do que é proclive.
\section{Paranóico}
\begin{itemize}
\item {Grp. gram.:M.}
\end{itemize}
Aquelle que soffre paranóia.
\section{Paródico}
\begin{itemize}
\item {Grp. gram.:adj.}
\end{itemize}
\begin{itemize}
\item {Grp. gram.:M.}
\end{itemize}
Relativo a paródia.
Aquillo que parece paródia. Cf. R. Jorge, \textunderscore El Greco\textunderscore , 21.
\section{Paroleira}
\begin{itemize}
\item {Grp. gram.:f.}
\end{itemize}
Espécie de medida antiga? canastra?:«\textunderscore levou quatro paroleiras de azeitonas.\textunderscore »(De um testamento do séc. XVII)
\section{Patelar}
\begin{itemize}
\item {Grp. gram.:adj.}
\end{itemize}
\begin{itemize}
\item {Utilização:Anat.}
\end{itemize}
\begin{itemize}
\item {Proveniência:(Do lat. \textunderscore patella\textunderscore )}
\end{itemize}
Relativo á rótula.
\section{Patellar}
\begin{itemize}
\item {Grp. gram.:adj.}
\end{itemize}
\begin{itemize}
\item {Utilização:Anat.}
\end{itemize}
\begin{itemize}
\item {Proveniência:(Do lat. \textunderscore patella\textunderscore )}
\end{itemize}
Relativo á rótula.
\section{Peligro}
\begin{itemize}
\item {Proveniência:(T. cast.)}
\end{itemize}
\textunderscore m.\textunderscore  (e der.) \textunderscore Ant.\textunderscore 
O mesmo que \textunderscore perigo\textunderscore , etc.
\section{Pendularmente}
\begin{itemize}
\item {Grp. gram.:adv.}
\end{itemize}
De modo pendular; á maneira de pêndulo:«\textunderscore o mestre, deambulando pendularmente de um lado para outro do estrado...\textunderscore »R. Jorge, na \textunderscore Luta\textunderscore , de 6-VI-913.
\section{Pentastêmone}
\begin{itemize}
\item {Grp. gram.:f.}
\end{itemize}
\begin{itemize}
\item {Proveniência:(Do gr. \textunderscore pente\textunderscore  + \textunderscore stemon\textunderscore )}
\end{itemize}
Gênero de plantas escrofularíneas, ornamentaes, originárias da América.--É errada a fórma \textunderscore pentstemon\textunderscore , usada por vários botânicos nossos.
\section{Perineorrafia}
\begin{itemize}
\item {Grp. gram.:f.}
\end{itemize}
\begin{itemize}
\item {Utilização:Med.}
\end{itemize}
\begin{itemize}
\item {Proveniência:(Do gr. \textunderscore perinaios\textunderscore  + \textunderscore raphe\textunderscore )}
\end{itemize}
Sutura dos lábios de uma rotura do perineu.
\section{Perineorrhaphia}
\begin{itemize}
\item {Grp. gram.:f.}
\end{itemize}
\begin{itemize}
\item {Utilização:Med.}
\end{itemize}
\begin{itemize}
\item {Proveniência:(Do gr. \textunderscore perinaios\textunderscore  + \textunderscore raphe\textunderscore )}
\end{itemize}
Sutura dos lábios de uma rotura do perineu.
\section{Perineotomia}
\begin{itemize}
\item {Grp. gram.:f.}
\end{itemize}
\begin{itemize}
\item {Utilização:Med.}
\end{itemize}
\begin{itemize}
\item {Proveniência:(Do gr. \textunderscore perinaios\textunderscore  + \textunderscore tome\textunderscore )}
\end{itemize}
Incisão no perineu.
\section{Peritonismo}
\begin{itemize}
\item {Grp. gram.:m.}
\end{itemize}
\begin{itemize}
\item {Utilização:Med.}
\end{itemize}
Complexo dos symptomas da peritonite aguda, (dores, vómitos, etc.), sem que haja realmente inflammação do peritoneu.
\section{Personalmente}
\begin{itemize}
\item {Grp. gram.:adv.}
\end{itemize}
\begin{itemize}
\item {Utilização:Ant.}
\end{itemize}
O mesmo que \textunderscore pessoalmente\textunderscore .
\section{Pharyngismo}
\begin{itemize}
\item {Grp. gram.:m.}
\end{itemize}
\begin{itemize}
\item {Utilização:Med.}
\end{itemize}
Contracção espasmódica dos músculos da pharynge.
\section{Pharyngoscopia}
\begin{itemize}
\item {Grp. gram.:f.}
\end{itemize}
\begin{itemize}
\item {Utilização:Med.}
\end{itemize}
Exame da cavidade pharýngea, por meio do pharyngoscópio.
\section{Phosphatose}
\begin{itemize}
\item {Grp. gram.:f.}
\end{itemize}
Preparação alimentícia para o gado, na qual entram phosphatos.
\section{Phosphorismo}
\begin{itemize}
\item {Grp. gram.:m.}
\end{itemize}
\begin{itemize}
\item {Utilização:Med.}
\end{itemize}
Entoxicação pelo phósphoro.
\section{Pintural}
\begin{itemize}
\item {Grp. gram.:adj.}
\end{itemize}
\begin{itemize}
\item {Utilização:Neol.}
\end{itemize}
Relativo a pintura:«\textunderscore anomalia pintural\textunderscore ». R. Jorge, \textunderscore El Greco\textunderscore , 22 e 50.
\section{Pitote}
\begin{itemize}
\item {Grp. gram.:m.}
\end{itemize}
\begin{itemize}
\item {Utilização:Bras. de San-Paulo}
\end{itemize}
Rôlo de cabello, que as mulhéres formam no alto da cabeça.
\section{Placentite}
\begin{itemize}
\item {Grp. gram.:f.}
\end{itemize}
\begin{itemize}
\item {Utilização:Med.}
\end{itemize}
Inflammação da placenta.
\section{Pneumatoterapia}
\begin{itemize}
\item {Grp. gram.:f.}
\end{itemize}
\begin{itemize}
\item {Utilização:Med.}
\end{itemize}
\begin{itemize}
\item {Proveniência:(Do gr. \textunderscore pneuma\textunderscore  + \textunderscore therapeia\textunderscore )}
\end{itemize}
Tratamento terapêutico por meio do ar que se respira.
\section{Pneumatotherapia}
\begin{itemize}
\item {Grp. gram.:f.}
\end{itemize}
\begin{itemize}
\item {Utilização:Med.}
\end{itemize}
\begin{itemize}
\item {Proveniência:(Do gr. \textunderscore pneuma\textunderscore  + \textunderscore therapeia\textunderscore )}
\end{itemize}
Tratamento therapêutico por meio do ar que se respira.
\section{Pneumógrafo}
\begin{itemize}
\item {Grp. gram.:m.}
\end{itemize}
\begin{itemize}
\item {Utilização:Med.}
\end{itemize}
\begin{itemize}
\item {Proveniência:(Do gr. \textunderscore pneuma\textunderscore  + \textunderscore graphein\textunderscore )}
\end{itemize}
Aparelho, para medir a expansão circunferencial do tórax, durante os movimentos respiratórios.
\section{Pneumógrapho}
\begin{itemize}
\item {Grp. gram.:m.}
\end{itemize}
\begin{itemize}
\item {Utilização:Med.}
\end{itemize}
\begin{itemize}
\item {Proveniência:(Do gr. \textunderscore pneuma\textunderscore  + \textunderscore graphein\textunderscore )}
\end{itemize}
Apparelho, para medir a expansão circunferencial do thórax, durante os movimentos respiratórios.
\section{Podrecer}
\begin{itemize}
\item {Grp. gram.:v. i.}
\end{itemize}
\begin{itemize}
\item {Utilização:Ant.}
\end{itemize}
O mesmo que \textunderscore apodrecer\textunderscore . Cf. \textunderscore Rev. Lus.\textunderscore , XVI, 10.
\section{Polifarmácia}
\begin{itemize}
\item {Grp. gram.:f.}
\end{itemize}
\begin{itemize}
\item {Utilização:Med.}
\end{itemize}
\begin{itemize}
\item {Proveniência:(De \textunderscore poli...\textunderscore  + \textunderscore farmácia\textunderscore )}
\end{itemize}
Prescripção de muitos medicamentos.
\section{Polypharmácia}
\begin{itemize}
\item {Grp. gram.:f.}
\end{itemize}
\begin{itemize}
\item {Utilização:Med.}
\end{itemize}
\begin{itemize}
\item {Proveniência:(De \textunderscore poly...\textunderscore  + \textunderscore pharmácia\textunderscore )}
\end{itemize}
Prescripção de muitos medicamentos.
\section{Posescolar}
\begin{itemize}
\item {Grp. gram.:adj.}
\end{itemize}
\begin{itemize}
\item {Proveniência:(De \textunderscore pos...\textunderscore  + \textunderscore escolar\textunderscore )}
\end{itemize}
Que apparece depois de um período escolar: \textunderscore relatórios posescolares\textunderscore .
\section{Preferencial}
\begin{itemize}
\item {Grp. gram.:adj.}
\end{itemize}
Que tem preferência, em relação a outros: \textunderscore os direitos preferenciaes de um crèdor\textunderscore .
\section{Pregárias}
\begin{itemize}
\item {Grp. gram.:f. pl.}
\end{itemize}
\begin{itemize}
\item {Utilização:Ant.}
\end{itemize}
O mesmo que \textunderscore prègalhas\textunderscore .
\section{Prolabiado}
\begin{itemize}
\item {Grp. gram.:adj.}
\end{itemize}
\begin{itemize}
\item {Utilização:Med.}
\end{itemize}
\begin{itemize}
\item {Proveniência:(Do lat. \textunderscore prolabi\textunderscore )}
\end{itemize}
Diz-se do órgão, que se deslocou, de cima para baixo, ou que soffreu prolapso.
\section{Própio}
\textunderscore adj.\textunderscore  (e der.) \textunderscore Ant.\textunderscore  e \textunderscore pop.\textunderscore 
O mesmo que \textunderscore próprio\textunderscore , etc.
(Cp. cast. \textunderscore própio\textunderscore )
\section{Piorreia}
\begin{itemize}
\item {Grp. gram.:f.}
\end{itemize}
\begin{itemize}
\item {Utilização:Med.}
\end{itemize}
\begin{itemize}
\item {Proveniência:(Do gr. \textunderscore puon\textunderscore  + \textunderscore rhein\textunderscore )}
\end{itemize}
Derramamento de pus.
\section{Piretogênico}
\begin{itemize}
\item {Grp. gram.:adj.}
\end{itemize}
\begin{itemize}
\item {Utilização:Med.}
\end{itemize}
\begin{itemize}
\item {Proveniência:(Do gr. \textunderscore puretos\textunderscore  + \textunderscore genos\textunderscore )}
\end{itemize}
Que produz febre.
\section{Piociânico}
\begin{itemize}
\item {Grp. gram.:adj.}
\end{itemize}
\begin{itemize}
\item {Utilização:Med.}
\end{itemize}
\begin{itemize}
\item {Proveniência:(Do gr. \textunderscore puon\textunderscore  + \textunderscore kuanos\textunderscore )}
\end{itemize}
Diz-se do bacílo, que se encontra em certo pus azulado.
\section{Piócito}
\begin{itemize}
\item {Grp. gram.:m.}
\end{itemize}
\begin{itemize}
\item {Utilização:Med.}
\end{itemize}
\begin{itemize}
\item {Proveniência:(Do gr. \textunderscore puon\textunderscore  + \textunderscore kutos\textunderscore )}
\end{itemize}
Célula de pus.
\section{Prostatismo}
\begin{itemize}
\item {Grp. gram.:m.}
\end{itemize}
\begin{itemize}
\item {Utilização:Med.}
\end{itemize}
Conjunto dos accidentes, determinados pela hypertrophia da próstata.
\section{Pseudencéfalo}
\begin{itemize}
\item {Grp. gram.:m.}
\end{itemize}
\begin{itemize}
\item {Utilização:Terat.}
\end{itemize}
\begin{itemize}
\item {Proveniência:(De \textunderscore pseudo...\textunderscore  + \textunderscore encéfalo\textunderscore )}
\end{itemize}
Monstro, em que o encéfalo está substituído por um tumor vascular.
\section{Pseudencéphalo}
\begin{itemize}
\item {Grp. gram.:m.}
\end{itemize}
\begin{itemize}
\item {Utilização:Terat.}
\end{itemize}
\begin{itemize}
\item {Proveniência:(De \textunderscore pseudo...\textunderscore  + \textunderscore encéphalo\textunderscore )}
\end{itemize}
Monstro, em que o encéphalo está substituído por um tumor vascular.
\section{Psicósico}
\begin{itemize}
\item {Grp. gram.:adj.}
\end{itemize}
\begin{itemize}
\item {Grp. gram.:M.}
\end{itemize}
Relativo á psicose.
Aquele que sofre psicose:«\textunderscore a arte dos psicósicos\textunderscore ». R. Jorge, \textunderscore El Greco\textunderscore , 50.
\section{Psychósico}
\begin{itemize}
\item {fónica:có}
\end{itemize}
\begin{itemize}
\item {Grp. gram.:adj.}
\end{itemize}
\begin{itemize}
\item {Grp. gram.:M.}
\end{itemize}
Relativo á psychose.
Aquelle que soffre psychose:«\textunderscore a arte dos psicósicos\textunderscore ». R. Jorge, \textunderscore El Greco\textunderscore , 50.
\section{Pubiotomia}
\begin{itemize}
\item {Grp. gram.:f.}
\end{itemize}
\begin{itemize}
\item {Utilização:Med.}
\end{itemize}
\begin{itemize}
\item {Proveniência:(Do lat. \textunderscore pubis\textunderscore  + gr. \textunderscore tome\textunderscore )}
\end{itemize}
Operação, que consiste na secção da articulação pubiana ou do próprio púbis.
\section{Pulpite}
\begin{itemize}
\item {Grp. gram.:f.}
\end{itemize}
\begin{itemize}
\item {Utilização:Med.}
\end{itemize}
\begin{itemize}
\item {Proveniência:(Do lat. \textunderscore pulpa\textunderscore )}
\end{itemize}
Inflammação da polpa dentária.
\section{Pyorrhéa}
\begin{itemize}
\item {Grp. gram.:f.}
\end{itemize}
\begin{itemize}
\item {Utilização:Med.}
\end{itemize}
\begin{itemize}
\item {Proveniência:(Do gr. \textunderscore puon\textunderscore  + \textunderscore rhein\textunderscore )}
\end{itemize}
Derramamento de pus.
\section{Pyretogênico}
\begin{itemize}
\item {Grp. gram.:adj.}
\end{itemize}
\begin{itemize}
\item {Utilização:Med.}
\end{itemize}
\begin{itemize}
\item {Proveniência:(Do gr. \textunderscore puretos\textunderscore  + \textunderscore genos\textunderscore )}
\end{itemize}
Que produz febre.
\section{Pyocyânico}
\begin{itemize}
\item {Grp. gram.:adj.}
\end{itemize}
\begin{itemize}
\item {Utilização:Med.}
\end{itemize}
\begin{itemize}
\item {Proveniência:(Do gr. \textunderscore puon\textunderscore  + \textunderscore kuanos\textunderscore )}
\end{itemize}
Diz-se do bacíllo, que se encontra em certo pus azulado.
\section{Pyócyto}
\begin{itemize}
\item {Grp. gram.:m.}
\end{itemize}
\begin{itemize}
\item {Utilização:Med.}
\end{itemize}
\begin{itemize}
\item {Proveniência:(Do gr. \textunderscore puon\textunderscore  + \textunderscore kutos\textunderscore )}
\end{itemize}
Céllula de pus.
\section{Proco}
\begin{itemize}
\item {Grp. gram.:m.}
\end{itemize}
\begin{itemize}
\item {Proveniência:(Lat. \textunderscore procus\textunderscore )}
\end{itemize}
Aquelle que procura mulhér para casamento; pretendente:«\textunderscore se asssim praticava Penélope, era por enganar cansados rogos de importunos procos.\textunderscore »\textunderscore Hussope\textunderscore .
\section{Procônsul}
\begin{itemize}
\item {Grp. gram.:m.}
\end{itemize}
\begin{itemize}
\item {Proveniência:(Lat. \textunderscore proconsul\textunderscore )}
\end{itemize}
Antigo governador de uma província romana.
\section{Proconsulado}
\begin{itemize}
\item {Grp. gram.:m.}
\end{itemize}
\begin{itemize}
\item {Proveniência:(Lat. \textunderscore proconsulatus\textunderscore )}
\end{itemize}
Cargo de procônsul.
Tempo, durante o qual um procônsul exercia as suas funcções.
\section{Proconsular}
\begin{itemize}
\item {Grp. gram.:adj.}
\end{itemize}
\begin{itemize}
\item {Proveniência:(Lat. \textunderscore proconsularis\textunderscore )}
\end{itemize}
Relativo a procônsul ou proprio dêlle.
\section{Procoracoidal}
\begin{itemize}
\item {Grp. gram.:m.  e  adj.}
\end{itemize}
Diz-se de um osso, situado na espádua de alguns animaes.
\section{Procoracoídeo}
\begin{itemize}
\item {Grp. gram.:m.  e  adj.}
\end{itemize}
\begin{itemize}
\item {Utilização:Anat.}
\end{itemize}
\begin{itemize}
\item {Proveniência:(De \textunderscore pro...\textunderscore  + \textunderscore coracoídeo\textunderscore )}
\end{itemize}
Diz-se de um osso, situado na espádua de alguns animaes.
\section{Procrastinação}
\begin{itemize}
\item {Grp. gram.:f.}
\end{itemize}
\begin{itemize}
\item {Proveniência:(Lat. \textunderscore procrastinatio\textunderscore )}
\end{itemize}
Acto ou effeito de procrastinar.
Adiamento.
\section{Procrastinador}
\begin{itemize}
\item {Grp. gram.:m.  e  adj.}
\end{itemize}
O que procrastina.
\section{Procrastinar}
\begin{itemize}
\item {Grp. gram.:v. t.}
\end{itemize}
\begin{itemize}
\item {Grp. gram.:V. i.}
\end{itemize}
\begin{itemize}
\item {Proveniência:(Lat. \textunderscore procrastinare\textunderscore )}
\end{itemize}
Deixar para o dia immediato.
Adiar; espaçar; demorar.
Usar de delongas.
\section{Procrear}
\textunderscore v. t. e i.\textunderscore  (e der.)
O mesmo que \textunderscore procriar\textunderscore , etc.
\section{Procriação}
\begin{itemize}
\item {Grp. gram.:f.}
\end{itemize}
\begin{itemize}
\item {Proveniência:(Lat. \textunderscore procreatio\textunderscore )}
\end{itemize}
Acto ou effeito de procriar.
\section{Procriador}
\begin{itemize}
\item {Grp. gram.:m.  e  adj.}
\end{itemize}
\begin{itemize}
\item {Proveniência:(Lat. \textunderscore procreator\textunderscore )}
\end{itemize}
O que procria.
\section{Procriar}
\begin{itemize}
\item {Grp. gram.:v. t.}
\end{itemize}
\begin{itemize}
\item {Grp. gram.:V. i.}
\end{itemize}
\begin{itemize}
\item {Utilização:P. us.}
\end{itemize}
\begin{itemize}
\item {Proveniência:(Lat. \textunderscore procreare\textunderscore )}
\end{itemize}
Dar origem a.
Gerar; dar nascimento a.
Produzir.
Promover a procriação de.
Promover a cultura ou a germinação de (plantas).
Germinar.
\section{Proctite}
\begin{itemize}
\item {Grp. gram.:f.}
\end{itemize}
\begin{itemize}
\item {Proveniência:(Do gr. \textunderscore proktos\textunderscore )}
\end{itemize}
Inflammação no ânus; rectite.
\section{Proctocele}
\begin{itemize}
\item {Grp. gram.:f.}
\end{itemize}
\begin{itemize}
\item {Utilização:Med.}
\end{itemize}
\begin{itemize}
\item {Proveniência:(Do gr. \textunderscore proctos\textunderscore  + \textunderscore kele\textunderscore )}
\end{itemize}
Quéda do recto.
\section{Proctoplastia}
\begin{itemize}
\item {Grp. gram.:f.}
\end{itemize}
\begin{itemize}
\item {Utilização:Med.}
\end{itemize}
\begin{itemize}
\item {Proveniência:(Do gr. \textunderscore proktos\textunderscore  + \textunderscore plassein\textunderscore )}
\end{itemize}
Formação de um ânus perineal.
\section{Proctoptose}
\begin{itemize}
\item {Grp. gram.:f.}
\end{itemize}
\begin{itemize}
\item {Proveniência:(Do gr. \textunderscore proktos\textunderscore  + \textunderscore ptosis\textunderscore )}
\end{itemize}
O mesmo que \textunderscore proctocele\textunderscore .
\section{Prodigioso}
\begin{itemize}
\item {Grp. gram.:adj.}
\end{itemize}
\begin{itemize}
\item {Proveniência:(Lat. \textunderscore prodigiosus\textunderscore )}
\end{itemize}
Relativo a prodígio.
Que revela prodígio.
Sobrenatural.
Admirável.
\section{Pródigo}
\begin{itemize}
\item {Grp. gram.:m.  e  adj.}
\end{itemize}
\begin{itemize}
\item {Grp. gram.:M.}
\end{itemize}
\begin{itemize}
\item {Proveniência:(Lat. \textunderscore prodigus\textunderscore )}
\end{itemize}
O que despende excessivamente.
Dissipador, desperdiçador.
Generoso; o que dá profusamente, espontaneamente.
Peça de madeira, que reforça o costado do navio.
\section{Proditor}
\begin{itemize}
\item {Grp. gram.:m.}
\end{itemize}
\begin{itemize}
\item {Proveniência:(Lat. \textunderscore proditor\textunderscore )}
\end{itemize}
O mesmo que \textunderscore traidor\textunderscore . Cf. Vieira, IV, 527.
\section{Proditório}
\begin{itemize}
\item {Grp. gram.:adj.}
\end{itemize}
\begin{itemize}
\item {Proveniência:(Do lat. \textunderscore proditus\textunderscore )}
\end{itemize}
Que revela traição.
Em que há traição.
\section{Prodrómico}
\begin{itemize}
\item {Grp. gram.:adj.}
\end{itemize}
Relativo aos pródromos de uma doença.
\section{Pródromo}
\begin{itemize}
\item {Grp. gram.:m.}
\end{itemize}
\begin{itemize}
\item {Utilização:Fig.}
\end{itemize}
\begin{itemize}
\item {Proveniência:(Gr. \textunderscore prodromos\textunderscore )}
\end{itemize}
Espécie de prefácio; preliminar.
Indisposição, que precede uma doença.
Precursor.
\section{Produção}
\begin{itemize}
\item {Grp. gram.:f.}
\end{itemize}
\begin{itemize}
\item {Proveniência:(Lat. \textunderscore productio\textunderscore )}
\end{itemize}
Acto ou effeito de produzir.
\section{Producção}
\begin{itemize}
\item {Grp. gram.:f.}
\end{itemize}
\begin{itemize}
\item {Proveniência:(Lat. \textunderscore productio\textunderscore )}
\end{itemize}
Acto ou effeito de produzir.
\section{Producente}
\begin{itemize}
\item {Grp. gram.:adj.}
\end{itemize}
\begin{itemize}
\item {Proveniência:(Lat. \textunderscore producens\textunderscore )}
\end{itemize}
Que produz.
Lógico, concludente: \textunderscore razões producentes\textunderscore .
\section{Productibilidade}
\begin{itemize}
\item {Grp. gram.:f.}
\end{itemize}
Qualidade do que é productível ou productivo.
\section{Productivamente}
\begin{itemize}
\item {Grp. gram.:adv.}
\end{itemize}
De modo productivo.
\section{Productível}
\begin{itemize}
\item {Grp. gram.:adj.}
\end{itemize}
\begin{itemize}
\item {Proveniência:(Do lat. \textunderscore productus\textunderscore )}
\end{itemize}
Que se póde produzir.
\section{Productivo}
\begin{itemize}
\item {Grp. gram.:adj.}
\end{itemize}
\begin{itemize}
\item {Proveniência:(Lat. \textunderscore productivus\textunderscore )}
\end{itemize}
Que produz, fértil.
Rendoso.
\section{Producto}
\begin{itemize}
\item {Grp. gram.:m.}
\end{itemize}
\begin{itemize}
\item {Proveniência:(Lat. \textunderscore productus\textunderscore )}
\end{itemize}
Effeito de produzir; aquillo que foi produzido.
Obra.
Resultado de uma multiplicação arithmética.
Resultado de uma operação.
Benefício; rendimento.
\section{Productor}
\begin{itemize}
\item {Grp. gram.:adj.}
\end{itemize}
\begin{itemize}
\item {Grp. gram.:M.}
\end{itemize}
\begin{itemize}
\item {Proveniência:(Lat. \textunderscore productor\textunderscore )}
\end{itemize}
Que produz; productivo.
Indivíduo, que produz; autor.
O que promove producções naturaes ou industriaes.
\section{Productriz}
\begin{itemize}
\item {Grp. gram.:adj.}
\end{itemize}
(Flexão fem. de \textunderscore productor\textunderscore . Cf. Filinto, X, 3)
\section{Produtibilidade}
\begin{itemize}
\item {Grp. gram.:f.}
\end{itemize}
Qualidade do que é produtível ou produtivo.
\section{Produtivamente}
\begin{itemize}
\item {Grp. gram.:adv.}
\end{itemize}
De modo produtivo.
\section{Produtível}
\begin{itemize}
\item {Grp. gram.:adj.}
\end{itemize}
\begin{itemize}
\item {Proveniência:(Do lat. \textunderscore productus\textunderscore )}
\end{itemize}
Que se póde produzir.
\section{Produtivo}
\begin{itemize}
\item {Grp. gram.:adj.}
\end{itemize}
\begin{itemize}
\item {Proveniência:(Lat. \textunderscore productivus\textunderscore )}
\end{itemize}
Que produz, fértil.
Rendoso.
\section{Produto}
\begin{itemize}
\item {Grp. gram.:m.}
\end{itemize}
\begin{itemize}
\item {Proveniência:(Lat. \textunderscore productus\textunderscore )}
\end{itemize}
Efeito de produzir; aquilo que foi produzido.
Obra.
Resultado de uma multiplicação aritmética.
Resultado de uma operação.
Benefício; rendimento.
\section{Produzidor}
\begin{itemize}
\item {Grp. gram.:m.  e  adj.}
\end{itemize}
(V.productor)
\section{Produzir}
\begin{itemize}
\item {Grp. gram.:v. t.}
\end{itemize}
\begin{itemize}
\item {Proveniência:(Lat. \textunderscore producere\textunderscore )}
\end{itemize}
Procriar; originar.
Promover o apparecimento de: \textunderscore produzir um livro\textunderscore .
Fornecer.
Causar: \textunderscore produzir sezões\textunderscore .
Fabricar.
Exhibir, mostrar.
Allegar: \textunderscore produzir argumentos\textunderscore .
\section{Produzível}
\begin{itemize}
\item {Grp. gram.:adj.}
\end{itemize}
O mesmo que \textunderscore productível\textunderscore .
\section{Próe}
\begin{itemize}
\item {Grp. gram.:f.}
\end{itemize}
\begin{itemize}
\item {Utilização:Ant.}
\end{itemize}
O mesmo que \textunderscore prol\textunderscore ^1.
\section{Proeiro}
\begin{itemize}
\item {Grp. gram.:m.}
\end{itemize}
\begin{itemize}
\item {Proveniência:(Do b. lat. \textunderscore pronarius\textunderscore )}
\end{itemize}
Marinheiro, que vigia a prôa. Cf. Celest. Soares, \textunderscore Quadros Nav.\textunderscore , II, 83.
\section{Proejar}
\begin{itemize}
\item {Grp. gram.:v. i.}
\end{itemize}
\begin{itemize}
\item {Proveniência:(De \textunderscore prôa\textunderscore )}
\end{itemize}
Aproar; dirigir-se; navegar em determinada direcção.
\section{Proembrião}
\begin{itemize}
\item {Grp. gram.:m.}
\end{itemize}
\begin{itemize}
\item {Utilização:Bot.}
\end{itemize}
\begin{itemize}
\item {Proveniência:(Do gr. \textunderscore pro\textunderscore  + \textunderscore embruon\textunderscore )}
\end{itemize}
Órgão vegetal, de fórma variada, e resultante da germinação de certos esporos.
\section{Proembrionário}
\begin{itemize}
\item {Grp. gram.:adj.}
\end{itemize}
Relativo ao proembrião.
\section{Proembryão}
\begin{itemize}
\item {Grp. gram.:m.}
\end{itemize}
\begin{itemize}
\item {Utilização:Bot.}
\end{itemize}
\begin{itemize}
\item {Proveniência:(Do gr. \textunderscore pro\textunderscore  + \textunderscore embruon\textunderscore )}
\end{itemize}
Órgão vegetal, de fórma variada, e resultante da germinação de certos esporos.
\section{Proembryonário}
\begin{itemize}
\item {Grp. gram.:adj.}
\end{itemize}
Relativo ao proembryão.
\section{Proemial}
\begin{itemize}
\item {Grp. gram.:adj.}
\end{itemize}
Relativo a proêmio; preambular.
\section{Proemiar}
\begin{itemize}
\item {Grp. gram.:v. t.}
\end{itemize}
Fazer proêmio a; prefaciar.
\section{Proeminência}
\begin{itemize}
\item {Grp. gram.:f.}
\end{itemize}
Qualidade ou estado de proeminente.
Parte proeminente.
Elevação, oiteiro.
\section{Proeminente}
\begin{itemize}
\item {Grp. gram.:adj.}
\end{itemize}
\begin{itemize}
\item {Proveniência:(Lat. \textunderscore proeminens\textunderscore )}
\end{itemize}
Que se eleva acima daquillo que o rodeia.
Alto; preeminente.
Notável; distinto.
\section{Proeminentemente}
\begin{itemize}
\item {Grp. gram.:adv.}
\end{itemize}
De modo proeminente.
\section{Proêmio}
\begin{itemize}
\item {Grp. gram.:m.}
\end{itemize}
\begin{itemize}
\item {Proveniência:(Gr. \textunderscore prooimion\textunderscore )}
\end{itemize}
Introducção, prefácio; exórdio, princípio.
\section{Proençal}
\begin{itemize}
\item {Grp. gram.:adj.}
\end{itemize}
O mesmo que \textunderscore provençal\textunderscore . Cf. Garrett, \textunderscore Romanceiro\textunderscore , II, 132.
\section{Professorar}
\begin{itemize}
\item {Grp. gram.:v. t.  e  i.}
\end{itemize}
\begin{itemize}
\item {Utilização:Neol.}
\end{itemize}
Exercer as funcções de professor.--Dêste verbo diz Castilho, em annotações inéditas ao \textunderscore Diccion.\textunderscore  de Moraes:«filho bastardo, mal criado, impertinente, do \textunderscore v. professar\textunderscore , foi pela primeira vez, cuido eu, apresentado aqui por um mestre de meninos que dá pelo nome de Bernardo Claro e tem razão. Para que é deduzir \textunderscore professorar\textunderscore , palavra, além de tudo, de muito mau som, do s. \textunderscore professor\textunderscore , se o s. \textunderscore professor\textunderscore  é já uma derivação immediata do v. \textunderscore professar\textunderscore ? e se o \textunderscore professorar\textunderscore  não póde significar mais que o \textunderscore professar\textunderscore ?...»
\section{Profesto}
\begin{itemize}
\item {Grp. gram.:adj.}
\end{itemize}
\begin{itemize}
\item {Proveniência:(Lat. \textunderscore profestus\textunderscore )}
\end{itemize}
Dizia-se, entre os Romanos, do dia que não é feriado ou solemne, do dia em que se trabalha. Cf. Castilho, \textunderscore Fastos\textunderscore , I, 287.
\section{Profetismo}
\begin{itemize}
\item {Grp. gram.:m.}
\end{itemize}
\begin{itemize}
\item {Proveniência:(De \textunderscore profeta\textunderscore )}
\end{itemize}
Doutrina religiosa, baseada nas profecias.
Conjunto de ideias dos profetas hebreus.
\section{Profetista}
\begin{itemize}
\item {Grp. gram.:adj.}
\end{itemize}
\begin{itemize}
\item {Proveniência:(De \textunderscore profeta\textunderscore )}
\end{itemize}
Relativo ao profetismo.
Relativo a profetas.
Que tem modos de profeta ou é próprio de profeta:«\textunderscore gesto largo e profetista.\textunderscore »Camillo, \textunderscore Mar. da Fonte\textunderscore , 154.
\section{Profetizador}
\begin{itemize}
\item {Grp. gram.:m.  e  adj.}
\end{itemize}
O que profetiza.
\section{Profetizar}
\begin{itemize}
\item {Grp. gram.:v. t.}
\end{itemize}
\begin{itemize}
\item {Utilização:Fig.}
\end{itemize}
\begin{itemize}
\item {Proveniência:(Lat. \textunderscore prophetizare\textunderscore )}
\end{itemize}
Predizer como profeta; vaticinar.
Anunciar por conjecturas: \textunderscore profetizo-te desgostos\textunderscore .
\section{Proficiência}
\begin{itemize}
\item {Grp. gram.:f.}
\end{itemize}
Qualidade do que é proficiente.
Vantagem; proficuidade: \textunderscore trabalhar com proficiência\textunderscore .
\section{Proficiente}
\begin{itemize}
\item {Grp. gram.:adj.}
\end{itemize}
\begin{itemize}
\item {Proveniência:(Lat. \textunderscore proficiens\textunderscore )}
\end{itemize}
Que tem perfeito conhecimento.
Hábil; destro.
Vantajoso; profícuo.
\section{Profício}
\begin{itemize}
\item {Grp. gram.:m.}
\end{itemize}
\begin{itemize}
\item {Utilização:Prov.}
\end{itemize}
\begin{itemize}
\item {Utilização:beir.}
\end{itemize}
O mesmo que \textunderscore proveito\textunderscore .
\section{Proficuamente}
\begin{itemize}
\item {Grp. gram.:adv.}
\end{itemize}
De modo profícuo; com vantagem.
\section{Proficuidade}
\begin{itemize}
\item {fónica:cu-i}
\end{itemize}
\begin{itemize}
\item {Grp. gram.:f.}
\end{itemize}
Qualidade do que é profícuo; vantagem; utilidade.
\section{Profícuo}
\begin{itemize}
\item {Grp. gram.:adj.}
\end{itemize}
\begin{itemize}
\item {Proveniência:(Lat. \textunderscore proficuus\textunderscore )}
\end{itemize}
Útil; proveitoso; vantajoso.
\section{Profiláctico}
\begin{itemize}
\item {Grp. gram.:adj.}
\end{itemize}
\begin{itemize}
\item {Proveniência:(Gr. \textunderscore prophulaktikos\textunderscore )}
\end{itemize}
Relativo á profilaxia; que preserva.
\section{Profilaxia}
\begin{itemize}
\item {fónica:csi}
\end{itemize}
\begin{itemize}
\item {Grp. gram.:f.}
\end{itemize}
\begin{itemize}
\item {Utilização:Ext.}
\end{itemize}
\begin{itemize}
\item {Proveniência:(Gr. \textunderscore prophulaxis\textunderscore )}
\end{itemize}
Parte da Medicina, que tem por objecto as precauções contra certas enfermidades.
Emprêgo de meios, para evitar certas enfermidades.
Preservativo.
\section{Profissão}
\begin{itemize}
\item {Grp. gram.:f.}
\end{itemize}
\begin{itemize}
\item {Proveniência:(Lat. \textunderscore professio\textunderscore )}
\end{itemize}
Acto ou effeito de professar.
Declaração pública de um sentimento habitual ou de um modo de sêr habitual.
Condição social; emprêgo.
Mestér; modo de vida.
\section{Profissional}
\begin{itemize}
\item {Grp. gram.:adj.}
\end{itemize}
Relativo a profissão.
\section{Profitente}
\begin{itemize}
\item {Grp. gram.:adj.}
\end{itemize}
\begin{itemize}
\item {Proveniência:(Lat. \textunderscore profitens\textunderscore )}
\end{itemize}
Que professa.
\section{Profligação}
\begin{itemize}
\item {Grp. gram.:f.}
\end{itemize}
\begin{itemize}
\item {Proveniência:(Lat. \textunderscore profligatio\textunderscore )}
\end{itemize}
Acto ou effeito de profligar.
\section{Profligador}
\begin{itemize}
\item {Grp. gram.:m.  e  adj.}
\end{itemize}
\begin{itemize}
\item {Proveniência:(Lat. \textunderscore profligator\textunderscore )}
\end{itemize}
O que profliga.
\section{Profligar}
\begin{itemize}
\item {Grp. gram.:v. t.}
\end{itemize}
\begin{itemize}
\item {Utilização:Fig.}
\end{itemize}
\begin{itemize}
\item {Proveniência:(Lat. \textunderscore profligare\textunderscore )}
\end{itemize}
Prostrar; destruír; derrotar.
Corromper; depravar.
\section{Pròfluente}
\begin{itemize}
\item {Grp. gram.:adj.}
\end{itemize}
\begin{itemize}
\item {Proveniência:(Lat. \textunderscore profluens\textunderscore )}
\end{itemize}
Que corre em certa direcção, (falando-se de uma corrente de água). Cf. Assis, \textunderscore Águas\textunderscore , 346.
\section{Profragma}
\begin{itemize}
\item {Grp. gram.:m.}
\end{itemize}
\begin{itemize}
\item {Utilização:Zool.}
\end{itemize}
\begin{itemize}
\item {Proveniência:(Do gr. \textunderscore pro\textunderscore , adeante, \textunderscore ephragma\textunderscore , separação)}
\end{itemize}
Separação membranosa do tórax dos insectos.
\section{Prófugo}
\begin{itemize}
\item {Grp. gram.:adj.}
\end{itemize}
\begin{itemize}
\item {Proveniência:(Lat. \textunderscore profugus\textunderscore )}
\end{itemize}
Fugitivo; vagabundo; desertor.
\section{Profundador}
\begin{itemize}
\item {Grp. gram.:m.  e  adj.}
\end{itemize}
O que profunda.
\section{Profundamente}
\begin{itemize}
\item {Grp. gram.:adv.}
\end{itemize}
De modo profundo; muito para dentro; intimamente.
Em alto grau.
\section{Profundar}
\begin{itemize}
\item {Grp. gram.:v. t.}
\end{itemize}
\begin{itemize}
\item {Utilização:Fig.}
\end{itemize}
\begin{itemize}
\item {Grp. gram.:V. i.}
\end{itemize}
\begin{itemize}
\item {Proveniência:(De \textunderscore profundo\textunderscore )}
\end{itemize}
Tornar fundo; escavar.
Indagar; pesquisar.
Examinar minuciosamente; sondar; penetrar.
Entrar até muito fundo; penetrar.
\section{Profundas}
\begin{itemize}
\item {Grp. gram.:f. pl.}
\end{itemize}
\begin{itemize}
\item {Utilização:Pop.}
\end{itemize}
Profundidade; a parte mais funda.
O inferno.
(Fem. pl. de \textunderscore profundo\textunderscore )
\section{Profundável}
\begin{itemize}
\item {Grp. gram.:adj.}
\end{itemize}
Que se póde profundar.
\section{Profundez}
\begin{itemize}
\item {Grp. gram.:f.}
\end{itemize}
\begin{itemize}
\item {Proveniência:(Lat. \textunderscore profunditas\textunderscore )}
\end{itemize}
Qualidade do que é profundo.
Espessura.
Difficuldade em sêr comprehendido.
Grandeza ou intensidade extraordinária.
\section{Profundeza}
\begin{itemize}
\item {Grp. gram.:f.}
\end{itemize}
\begin{itemize}
\item {Proveniência:(Lat. \textunderscore profunditas\textunderscore )}
\end{itemize}
Qualidade do que é profundo.
Espessura.
Difficuldade em sêr comprehendido.
Grandeza ou intensidade extraordinária.
\section{Profundidade}
\begin{itemize}
\item {Grp. gram.:f.}
\end{itemize}
\begin{itemize}
\item {Proveniência:(Lat. \textunderscore profunditas\textunderscore )}
\end{itemize}
Qualidade do que é profundo.
Espessura.
Difficuldade em sêr comprehendido.
Grandeza ou intensidade extraordinária.
\section{Profundo}
\begin{itemize}
\item {Grp. gram.:adj.}
\end{itemize}
\begin{itemize}
\item {Utilização:Fig.}
\end{itemize}
\begin{itemize}
\item {Grp. gram.:M.}
\end{itemize}
\begin{itemize}
\item {Utilização:Fig.}
\end{itemize}
\begin{itemize}
\item {Grp. gram.:Adv.}
\end{itemize}
\begin{itemize}
\item {Proveniência:(Lat. \textunderscore profundus\textunderscore )}
\end{itemize}
Muito fundo.
Cujo fundo dista muito da superfície, das bordas ou da entrada: \textunderscore mar profundo\textunderscore .
Que penetra muito: \textunderscore golpe profundo\textunderscore .
Perspicaz; investigador.
Que sabe muito: \textunderscore philósopho profundo\textunderscore .
Que não é superficial.
Que procede do interior ou da parte mais funda.
Enorme; excessivo: \textunderscore desgôsto profundo\textunderscore .
Muito importante.
Que é diffícil de comprehender ou de expor.
Aquillo que é profundo.
Inferno.
O mar.
Na profundidade; profundamente.
\section{Profundura}
\begin{itemize}
\item {Grp. gram.:f.}
\end{itemize}
(V.profundidade)
\section{Profusamente}
\begin{itemize}
\item {Grp. gram.:adv.}
\end{itemize}
De modo profuso; com profusão; abundantemente.
\section{Progimnasma}
\begin{itemize}
\item {Grp. gram.:m.}
\end{itemize}
\begin{itemize}
\item {Utilização:Des.}
\end{itemize}
Preâmbulo, prefácio. Cf. Severim de Faria, \textunderscore Disc. Vários\textunderscore .
\section{Progresso}
\begin{itemize}
\item {Grp. gram.:m.}
\end{itemize}
\begin{itemize}
\item {Proveniência:(Lat. \textunderscore progressus\textunderscore )}
\end{itemize}
Marcha ou movimento para deante.
Progredimento, desenvolvimento.
Melhoramento ou aumento.
Qualquer adeantamento, em sentido favorável.
\section{Proguntar}
\textunderscore v. t.\textunderscore  (e der.)
(Uma das fórmas pop. de \textunderscore perguntar\textunderscore , etc.)
\section{Progymnasma}
\begin{itemize}
\item {Grp. gram.:m.}
\end{itemize}
\begin{itemize}
\item {Utilização:Des.}
\end{itemize}
Preâmbulo, prefácio. Cf. Severim de Faria, \textunderscore Disc. Vários\textunderscore .
\section{Prohibição}
\begin{itemize}
\item {fónica:pro-i}
\end{itemize}
\begin{itemize}
\item {Grp. gram.:f.}
\end{itemize}
\begin{itemize}
\item {Proveniência:(Lat. \textunderscore prohibitio\textunderscore )}
\end{itemize}
Acto ou effeito de prohibir.
\section{Prohibidor}
\begin{itemize}
\item {fónica:pro-i}
\end{itemize}
\begin{itemize}
\item {Grp. gram.:m.  e  adj.}
\end{itemize}
\begin{itemize}
\item {Proveniência:(Lat. \textunderscore prohibitor\textunderscore )}
\end{itemize}
O que prohibe.
\section{Prohibir}
\begin{itemize}
\item {fónica:pro-i}
\end{itemize}
\begin{itemize}
\item {Grp. gram.:v. t.}
\end{itemize}
\begin{itemize}
\item {Proveniência:(Lat. \textunderscore prohibere\textunderscore )}
\end{itemize}
Impedir que se faça, ou ordenar que não se faça.
Vedar, interdizer.
Oppor-se a; tornar defeso.
\section{Prohibitivo}
\begin{itemize}
\item {fónica:pro-i}
\end{itemize}
\begin{itemize}
\item {Grp. gram.:adj.}
\end{itemize}
\begin{itemize}
\item {Proveniência:(Do lat. \textunderscore prohibitus\textunderscore )}
\end{itemize}
Que prohibe; em que há prohibição.
\section{Prohibitório}
\begin{itemize}
\item {fónica:pro-i}
\end{itemize}
\begin{itemize}
\item {Grp. gram.:adj.}
\end{itemize}
\begin{itemize}
\item {Proveniência:(Do lat. \textunderscore prohibitorius\textunderscore )}
\end{itemize}
O mesmo que \textunderscore prohibitivo\textunderscore .
\section{Proibição}
\begin{itemize}
\item {Grp. gram.:f.}
\end{itemize}
\begin{itemize}
\item {Proveniência:(Lat. \textunderscore prohibitio\textunderscore )}
\end{itemize}
Acto ou efeito de proibir.
\section{Proibidor}
\begin{itemize}
\item {Grp. gram.:m.  e  adj.}
\end{itemize}
\begin{itemize}
\item {Proveniência:(Lat. \textunderscore prohibitor\textunderscore )}
\end{itemize}
O que proibe.
\section{Proibir}
\begin{itemize}
\item {Grp. gram.:v. t.}
\end{itemize}
\begin{itemize}
\item {Proveniência:(Lat. \textunderscore prohibere\textunderscore )}
\end{itemize}
Impedir que se faça, ou ordenar que não se faça.
Vedar, interdizer.
Opor-se a; tornar defeso.
\section{Proibitivo}
\begin{itemize}
\item {Grp. gram.:adj.}
\end{itemize}
\begin{itemize}
\item {Proveniência:(Do lat. \textunderscore prohibitus\textunderscore )}
\end{itemize}
Que proibe; em que há proibição.
\section{Proibitório}
\begin{itemize}
\item {Grp. gram.:adj.}
\end{itemize}
\begin{itemize}
\item {Proveniência:(Do lat. \textunderscore prohibitorius\textunderscore )}
\end{itemize}
O mesmo que \textunderscore proibitivo\textunderscore .
\section{Proíz}
\begin{itemize}
\item {Grp. gram.:m.  ou  f.}
\end{itemize}
\begin{itemize}
\item {Utilização:Náut.}
\end{itemize}
\begin{itemize}
\item {Proveniência:(De \textunderscore prôa\textunderscore )}
\end{itemize}
Cabo, para amarrar embarcações á terra. Cf. \textunderscore Peregrinação\textunderscore , LIV.
\section{Projecção}
\begin{itemize}
\item {Grp. gram.:f.}
\end{itemize}
\begin{itemize}
\item {Proveniência:(Lat. \textunderscore projectio\textunderscore )}
\end{itemize}
Acto ou effeito de projectar.
Figura geométrica, obtida pela incidência de prependiculares, que partem de todas as extremidades de um objecto que se quere representar.
Saliência.
Representação geográphica de um pedaço de terra ou de céu sôbre um plano.
\section{Projectação}
\begin{itemize}
\item {Grp. gram.:f.}
\end{itemize}
O mesmo que \textunderscore projecção\textunderscore .
\section{Projectar}
\begin{itemize}
\item {Grp. gram.:v. t.}
\end{itemize}
\begin{itemize}
\item {Grp. gram.:V. p.}
\end{itemize}
\begin{itemize}
\item {Proveniência:(De \textunderscore projecto\textunderscore )}
\end{itemize}
Arremessar, atirar longe.
Representar por meio de projecções.
Fazer projecto de; planear: \textunderscore projectar uma viagem\textunderscore .
Arremessar-se; caír sôbre, incidir.
Delinear-se.
Prolongar-se em sentido horizontal ou oblíquo.
\section{Projectício}
\begin{itemize}
\item {Grp. gram.:adj.}
\end{itemize}
\begin{itemize}
\item {Utilização:Des.}
\end{itemize}
\begin{itemize}
\item {Proveniência:(Lat. \textunderscore projecticius\textunderscore )}
\end{itemize}
Que se atira; que se arremessa. Cf. Filinto, XVIII, 234.
\section{Projéctil}
\begin{itemize}
\item {Grp. gram.:adj.}
\end{itemize}
\begin{itemize}
\item {Grp. gram.:M.}
\end{itemize}
\begin{itemize}
\item {Proveniência:(Lat. hyp. \textunderscore projectilis\textunderscore )}
\end{itemize}
Que póde sêr arremessado.
Que produz projecção.
Qualquer corpo sólido e pesado, que se move no espaço, abandonado a si próprio, depois de receber um impulso.
Qualquer objecto, que se arremessa para fazer mal.
Corpo, arremessado por uma boca de fogo.
\section{Projectista}
\begin{itemize}
\item {Grp. gram.:m.  e  f.}
\end{itemize}
\begin{itemize}
\item {Proveniência:(De \textunderscore projecto\textunderscore )}
\end{itemize}
Pessôa, que faz muitos planos.
\section{Projectivo}
\begin{itemize}
\item {Grp. gram.:adj.}
\end{itemize}
\begin{itemize}
\item {Utilização:Mathem.}
\end{itemize}
\begin{itemize}
\item {Proveniência:(Do lat. \textunderscore projectus\textunderscore )}
\end{itemize}
Relactivo a projecção.
\section{Projecto}
\begin{itemize}
\item {Grp. gram.:m.}
\end{itemize}
\begin{itemize}
\item {Proveniência:(Lat. \textunderscore projectus\textunderscore )}
\end{itemize}
Plano de um trabalho, de um acto.
Intenção, desígnio.
Emprehendimento.
Primeira redacção ou redacção provisória de uma lei, de uns estatutos, etc.
Plano geral de uma edificação.
\section{Projectógrafo}
\begin{itemize}
\item {Grp. gram.:m.}
\end{itemize}
\begin{itemize}
\item {Proveniência:(De \textunderscore projectar\textunderscore  + gr. \textunderscore graphein\textunderscore )}
\end{itemize}
Aparelho de sinaes nocturnos, adoptado pela marinha americana.
\section{Projectógrapho}
\begin{itemize}
\item {Grp. gram.:m.}
\end{itemize}
\begin{itemize}
\item {Proveniência:(De \textunderscore projectar\textunderscore  + gr. \textunderscore graphein\textunderscore )}
\end{itemize}
Apparelho de sinaes nocturnos, adoptado pela marinha americana.
\section{Projectoscópio}
\begin{itemize}
\item {Grp. gram.:m.}
\end{itemize}
\begin{itemize}
\item {Proveniência:(Do lat. \textunderscore projectus\textunderscore  + gr. \textunderscore skopein\textunderscore )}
\end{itemize}
Um dos nomes, mais ou menos arbitrários, que ultimamente se têm dado ao cinematógrapho.
\section{Projectura}
\begin{itemize}
\item {Grp. gram.:f.}
\end{itemize}
\begin{itemize}
\item {Utilização:Bot.}
\end{itemize}
\begin{itemize}
\item {Proveniência:(Lat. \textunderscore projectura\textunderscore )}
\end{itemize}
Saliência externa de uma parte de um edifício.
Pequena crista saliente que, partindo da origem de uma fôlha, se prolonga sôbre o tronco, de cima para baixo.
\section{Prol}
\begin{itemize}
\item {Grp. gram.:m.  ou  f.}
\end{itemize}
\begin{itemize}
\item {Proveniência:(Lat. \textunderscore pro\textunderscore ?)}
\end{itemize}
Vantagem, proveito: \textunderscore lutar em prol da pátria\textunderscore .
\section{Prol}
\begin{itemize}
\item {Grp. gram.:f.}
\end{itemize}
\begin{itemize}
\item {Utilização:Ant.}
\end{itemize}
O mesmo que \textunderscore prole\textunderscore .
\textunderscore Homem de prol\textunderscore , homem nobre.
\section{Prolação}
\begin{itemize}
\item {Grp. gram.:f.}
\end{itemize}
\begin{itemize}
\item {Proveniência:(Lat. \textunderscore prolatio\textunderscore )}
\end{itemize}
Acto ou effeito de proferir.
\section{Prolapso}
\begin{itemize}
\item {Grp. gram.:m.}
\end{itemize}
\begin{itemize}
\item {Proveniência:(Lat. \textunderscore lapsus\textunderscore )}
\end{itemize}
Saída de um órgão ou de parte de um órgão, para fóra do lugar normal: \textunderscore prolapso do útero\textunderscore .
\section{Prolongadamente}
\begin{itemize}
\item {Grp. gram.:adv.}
\end{itemize}
\begin{itemize}
\item {Proveniência:(De \textunderscore prolongar\textunderscore )}
\end{itemize}
Com prolongação.
\section{Prolongamento}
\begin{itemize}
\item {Grp. gram.:m.}
\end{itemize}
\begin{itemize}
\item {Proveniência:(De \textunderscore prolongar\textunderscore )}
\end{itemize}
Prolongação; continuação de uma acção ou da mesma direcção: \textunderscore o prolongamento de uma rua\textunderscore .
\section{Prolongar}
\begin{itemize}
\item {Grp. gram.:v. t.}
\end{itemize}
\begin{itemize}
\item {Proveniência:(Lat. \textunderscore prolongare\textunderscore )}
\end{itemize}
Tornar mais longo; aumentar a extensão ou a duração de: \textunderscore a paz da consciência prolonga a vida\textunderscore .
Estender ao longo de.
Continuar.
Protrahir.
\section{Prolongável}
\begin{itemize}
\item {Grp. gram.:adj.}
\end{itemize}
Que se póde prolongar.
\section{Prolongo}
\begin{itemize}
\item {Grp. gram.:m.}
\end{itemize}
\begin{itemize}
\item {Grp. gram.:Adj.}
\end{itemize}
\begin{itemize}
\item {Utilização:Gram.}
\end{itemize}
Parte do telhado, parallela á fronteira ou traseira da casa.
Longo ou prolongado, (falando-se de acentos phonéticos).
(De \textunderscore prolongar\textunderscore ).
\section{Proloquial}
\begin{itemize}
\item {Grp. gram.:adj.}
\end{itemize}
Relativo a prolóquio; que encerra prolóquio; que se deduz de um prolóquio. Cf. Latino, \textunderscore Camões\textunderscore , 55.
\section{Prolóquio}
\begin{itemize}
\item {Grp. gram.:m.}
\end{itemize}
\begin{itemize}
\item {Proveniência:(Lat. \textunderscore proloquium\textunderscore )}
\end{itemize}
Máxima; provérbio; anexim; ditado.
\section{Prolusão}
\begin{itemize}
\item {Grp. gram.:f.}
\end{itemize}
\begin{itemize}
\item {Proveniência:(Lat. \textunderscore prolusio\textunderscore )}
\end{itemize}
Prelúdio, preâmbulo; preparação. Cf. \textunderscore Luz e Calor\textunderscore , 434.
\section{Proluxidade}
\begin{itemize}
\item {fónica:csi}
\end{itemize}
\begin{itemize}
\item {Grp. gram.:f.}
\end{itemize}
Qualidade do que é proluxo.
\section{Proluxo}
\begin{itemize}
\item {fónica:cso}
\end{itemize}
\begin{itemize}
\item {Grp. gram.:adj.}
\end{itemize}
\begin{itemize}
\item {Utilização:Pop.}
\end{itemize}
O mesmo que \textunderscore prolixo\textunderscore .
Affectado no vestir. Cp. \textunderscore perluxo\textunderscore ^2.
(Alter. de \textunderscore prolixo\textunderscore )
\section{Promanar}
\begin{itemize}
\item {Grp. gram.:v. i.}
\end{itemize}
\begin{itemize}
\item {Proveniência:(Lat. \textunderscore promanare\textunderscore )}
\end{itemize}
Dimanar; proceder, derivar: \textunderscore da imprudência promanam muitos males\textunderscore .
\section{Promandar}
\begin{itemize}
\item {Grp. gram.:v. t.}
\end{itemize}
\begin{itemize}
\item {Proveniência:(De \textunderscore pro...\textunderscore  + \textunderscore mandar\textunderscore )}
\end{itemize}
Mandar em auxílio:«\textunderscore promanda aos Pyrenéus as nossas tropas\textunderscore ». Latino.
\section{Prombeta}
\begin{itemize}
\item {fónica:bê}
\end{itemize}
\begin{itemize}
\item {Grp. gram.:f.}
\end{itemize}
\begin{itemize}
\item {Utilização:Açor}
\end{itemize}
Peixe, o mesmo que \textunderscore prumbeta\textunderscore .
\section{Promento}
\begin{itemize}
\item {Grp. gram.:m.}
\end{itemize}
\begin{itemize}
\item {Utilização:Ant.}
\end{itemize}
\begin{itemize}
\item {Utilização:Pop.}
\end{itemize}
\begin{itemize}
\item {Proveniência:(De \textunderscore prometer\textunderscore  talvez alter. de \textunderscore prometo\textunderscore , 1.^a pess. do indic. de \textunderscore prometer\textunderscore , nasalada a segunda sýllaba por infl. do \textunderscore m\textunderscore  da mesma sýllaba, como succedeu em \textunderscore mim\textunderscore , na fórma lisboeta \textunderscore menza\textunderscore , etc.)}
\end{itemize}
O mesmo que \textunderscore prometimento\textunderscore .
\section{Promérico}
\begin{itemize}
\item {Grp. gram.:adj.}
\end{itemize}
Relativo ao prómero.
\section{Prómero}
\begin{itemize}
\item {Grp. gram.:m.}
\end{itemize}
\begin{itemize}
\item {Utilização:Anat.}
\end{itemize}
\begin{itemize}
\item {Proveniência:(Do gr. \textunderscore pros\textunderscore  + \textunderscore meros\textunderscore )}
\end{itemize}
Cada uma das partes do corpo, considerando-as separadas por um plano horizontal, que passa pelo umbigo.
\section{Promerope}
\begin{itemize}
\item {Grp. gram.:f.}
\end{itemize}
Pássaro tenuirostro da África.
\section{Promessa}
\begin{itemize}
\item {Grp. gram.:f.}
\end{itemize}
\begin{itemize}
\item {Proveniência:(Do lat. \textunderscore promissa\textunderscore )}
\end{itemize}
Acto ou effeito de prometer.
Aquillo que se promete.
Compromisso.
Offerecimento para subornar.
Voto, feito á divindade ou aos santos, e cujo cumprimento depende de se evitar um mal ou de se obter a cura de uma doença.
\section{Prometedor}
\begin{itemize}
\item {Grp. gram.:m.  e  adj.}
\end{itemize}
O que promete.
O que dá esperanças.
\section{Prometedoramente}
\begin{itemize}
\item {Grp. gram.:adv.}
\end{itemize}
De modo prometedor.
\section{Prometer}
\begin{itemize}
\item {Grp. gram.:v. t.}
\end{itemize}
\begin{itemize}
\item {Grp. gram.:V. i.}
\end{itemize}
\begin{itemize}
\item {Proveniência:(Lat. \textunderscore promittere\textunderscore )}
\end{itemize}
Obrigar-se verbalmente ou por escrito a.
Affirmar previamente.
Predizer.
Dar esperanças a.
Dar a probabilidade de.
Fazer promessas.
Dar bôas probabilidades ou esperanças: \textunderscore a pequena promete\textunderscore .
Dar indício.
\section{Prometida}
\begin{itemize}
\item {Grp. gram.:f.}
\end{itemize}
O mesmo que \textunderscore noiva\textunderscore .
(Fem. de \textunderscore prometido\textunderscore )
\section{Prometido}
\begin{itemize}
\item {Grp. gram.:adj.}
\end{itemize}
\begin{itemize}
\item {Grp. gram.:M.}
\end{itemize}
Reservado, em consequência de uma promessa.
Que tem o seu casamento contratado.
Aquillo que está prometido.
Noivo.
\section{Prometimento}
\begin{itemize}
\item {Grp. gram.:m.}
\end{itemize}
Acto ou effeito de prometer; promessa; compromisso.
\section{Prominência}
\begin{itemize}
\item {Grp. gram.:f.}
\end{itemize}
Qualidade de prominente.
\section{Prominente}
\begin{itemize}
\item {Grp. gram.:adj.}
\end{itemize}
\begin{itemize}
\item {Proveniência:(Lat. \textunderscore prominens\textunderscore )}
\end{itemize}
O mesmo que \textunderscore proeminente\textunderscore .
\section{Promiscuamente}
\begin{itemize}
\item {Grp. gram.:adv.}
\end{itemize}
De modo promíscuo.
\section{Promptuário}
\begin{itemize}
\item {Grp. gram.:m.}
\end{itemize}
\begin{itemize}
\item {Proveniência:(Lat. \textunderscore promptuarius\textunderscore )}
\end{itemize}
Lugar, em que se guardam objectos que podem sêr precisos a qualquer hora.
Livro manual de indicações úteis.
\section{Promulgação}
\begin{itemize}
\item {Grp. gram.:f.}
\end{itemize}
\begin{itemize}
\item {Proveniência:(Lat. \textunderscore promulgatio\textunderscore )}
\end{itemize}
Acto ou effeito de promulgar.
\section{Promulgador}
\begin{itemize}
\item {Grp. gram.:m.  e  adj.}
\end{itemize}
\begin{itemize}
\item {Proveniência:(Lat. \textunderscore promulgator\textunderscore )}
\end{itemize}
O que promulga.
\section{Promulgar}
\begin{itemize}
\item {Grp. gram.:v. t.}
\end{itemize}
\begin{itemize}
\item {Proveniência:(Lat. \textunderscore promulgare\textunderscore )}
\end{itemize}
Vulgarizar, transmittir ao vulgo.
Divulgar, tornal público.
Publicar (uma lei).
Publicar officialmente.
\section{Pronação}
\begin{itemize}
\item {Grp. gram.:f.}
\end{itemize}
\begin{itemize}
\item {Proveniência:(Do lat. \textunderscore pronatus\textunderscore )}
\end{itemize}
Movimento, com que se volta a mão, ficando a palma desta voltada para a terra.
Estado da mão assim voltada.
Posição de um doente, deitado sôbre o ventre.
\section{Pronador}
\begin{itemize}
\item {Grp. gram.:m.  e  adj.}
\end{itemize}
\begin{itemize}
\item {Proveniência:(Do lat. \textunderscore pronatus\textunderscore )}
\end{itemize}
Cada um dos músculos, que executam a pronação.
\section{Pronau}
\begin{itemize}
\item {Grp. gram.:m.}
\end{itemize}
\begin{itemize}
\item {Proveniência:(Lat. \textunderscore pronaus\textunderscore )}
\end{itemize}
Parte anterior dos tempos antigos.
\section{Prono}
\begin{itemize}
\item {Grp. gram.:adj.}
\end{itemize}
\begin{itemize}
\item {Utilização:Poét.}
\end{itemize}
\begin{itemize}
\item {Utilização:Fig.}
\end{itemize}
\begin{itemize}
\item {Proveniência:(Lat. \textunderscore pronus\textunderscore )}
\end{itemize}
Dobrado para a parte anterior.
Inclinado.
Tendente; disposto.
\section{Pronome}
\begin{itemize}
\item {Grp. gram.:m.}
\end{itemize}
\begin{itemize}
\item {Utilização:Gram.}
\end{itemize}
\begin{itemize}
\item {Proveniência:(Lat. \textunderscore pronomen\textunderscore )}
\end{itemize}
Palavra que se emprega em vez de um nome.
\section{Pronominal}
\begin{itemize}
\item {Grp. gram.:adj.}
\end{itemize}
\begin{itemize}
\item {Utilização:Gram.}
\end{itemize}
\begin{itemize}
\item {Proveniência:(Lat. \textunderscore pronominalis\textunderscore )}
\end{itemize}
Relativo ao pronome.
Diz-se especialmente do verbo, que se conjuga com o pronome pessoal da mesma pessôa que o sujeito:«\textunderscore eu enganei-me; nós enganámo-nos.\textunderscore »
\section{Pronominar}
\begin{itemize}
\item {Grp. gram.:v.}
\end{itemize}
\begin{itemize}
\item {Utilização:t. Gram.}
\end{itemize}
\begin{itemize}
\item {Proveniência:(Do lat. \textunderscore pronomen\textunderscore )}
\end{itemize}
Appor pronome a:«\textunderscore ...haver pronominado o verbo esvoaçar.\textunderscore »Rui Barb., \textunderscore Réplica\textunderscore , 159.
\section{Pronóstico}
\begin{itemize}
\item {Grp. gram.:Adj.}
\end{itemize}
\begin{itemize}
\item {Utilização:Pop.}
\end{itemize}
\textunderscore m.\textunderscore  (e der.)
O mesmo que \textunderscore prognóstico\textunderscore , etc.
Espevitado, sentencioso, petulante:«\textunderscore Aí vens tu com as tuas alicantinas,--retrocava, prognostica e solenne, a tia Rosa.\textunderscore »Camillo, \textunderscore Brasileira\textunderscore , 80.
\section{Prontuário}
\begin{itemize}
\item {Grp. gram.:m.}
\end{itemize}
\begin{itemize}
\item {Proveniência:(Lat. \textunderscore promptuarius\textunderscore )}
\end{itemize}
Lugar, em que se guardam objectos que podem sêr precisos a qualquer hora.
Livro manual de indicações úteis.
\section{Prónubo}
\begin{itemize}
\item {Grp. gram.:adj.}
\end{itemize}
\begin{itemize}
\item {Utilização:Poét.}
\end{itemize}
\begin{itemize}
\item {Proveniência:(Lat. \textunderscore pronubus\textunderscore )}
\end{itemize}
Relativo ao noivo ou á noiva.
\section{Pronúncia}
Acto ou modo de pronunciar.
Pronunciação.
Fala.
Despacho judicial, que declara indiciado alguém como autor ou cúmplice de um crime.
\section{Pronunciação}
\begin{itemize}
\item {Grp. gram.:f.}
\end{itemize}
\begin{itemize}
\item {Utilização:Des.}
\end{itemize}
Acto ou effeito de pronunciar.
Maneira de pronunciar.
O mesmo que \textunderscore pronúncia\textunderscore , ou despacho judicial, que indicía alguém réu ou cúmplice de um crime. Cf. F. Manuel, \textunderscore Carta de Guia\textunderscore , 11.
\section{Pronunciado}
\begin{itemize}
\item {Grp. gram.:adj.}
\end{itemize}
\begin{itemize}
\item {Proveniência:(De \textunderscore pronunciar\textunderscore )}
\end{itemize}
Indiciado como réu ou cúmplice de um crime.
\section{Pronunciamento}
\begin{itemize}
\item {Grp. gram.:m.}
\end{itemize}
\begin{itemize}
\item {Proveniência:(De \textunderscore pronunciar\textunderscore )}
\end{itemize}
Acto de pronunciar-se collectivamente contra o govêrno, contra quaesquer medidas governativas, etc.
Sublevação, revolta. Cf. Camillo, \textunderscore Mar. da Fonte\textunderscore , 180.
\section{Pronunciar}
\begin{itemize}
\item {Grp. gram.:v. t.}
\end{itemize}
\begin{itemize}
\item {Utilização:Jur.}
\end{itemize}
\begin{itemize}
\item {Grp. gram.:V. p.}
\end{itemize}
\begin{itemize}
\item {Proveniência:(Lat. \textunderscore pronuntiare\textunderscore )}
\end{itemize}
Exprimir verbalmente; proferir; articular.
Recitar.
Publicar.
Declarar com autoridade.
Tornar claro, fazer realçar.
Dar despacho de pronúncia contra.
Manifestar o que pensa ou sente.
Sublevar-se, fazer pronunciamento.
\section{Pronunciável}
\begin{itemize}
\item {Grp. gram.:adj.}
\end{itemize}
Que se póde pronunciar.
\section{Prònúncio}
\begin{itemize}
\item {Grp. gram.:m.}
\end{itemize}
\begin{itemize}
\item {Proveniência:(De \textunderscore pro...\textunderscore  + \textunderscore núncio\textunderscore )}
\end{itemize}
Ecclesiástico, investido transitóriamente nas funcções de núncio pontíficio.
\section{Propagação}
\begin{itemize}
\item {Grp. gram.:f.}
\end{itemize}
\begin{itemize}
\item {Proveniência:(Lat. \textunderscore propagatio\textunderscore )}
\end{itemize}
Acto ou effeito de propagar; diffusão; desenvolvimento: \textunderscore propagação da epidemia\textunderscore .
\section{Propagador}
\begin{itemize}
\item {Grp. gram.:m.  e  adj.}
\end{itemize}
\begin{itemize}
\item {Proveniência:(Lat. \textunderscore propagator\textunderscore )}
\end{itemize}
O que propaga; o que divulga; o que faz propaganda.
\section{Propaganda}
\begin{itemize}
\item {Grp. gram.:f.}
\end{itemize}
\begin{itemize}
\item {Proveniência:(Lat. \textunderscore propaganda\textunderscore )}
\end{itemize}
Propagação de princípios ou theorias.
Sociedade, que vulgariza certas doutrinas.
\section{Prophetismo}
\begin{itemize}
\item {Grp. gram.:m.}
\end{itemize}
\begin{itemize}
\item {Proveniência:(De \textunderscore propheta\textunderscore )}
\end{itemize}
Doutrina religiosa, baseada nas prophecias.
Conjunto de ideias dos prophetas hebreus.
\section{Prophetista}
\begin{itemize}
\item {Grp. gram.:adj.}
\end{itemize}
\begin{itemize}
\item {Proveniência:(De \textunderscore propheta\textunderscore )}
\end{itemize}
Relativo ao prophetismo.
Relativo a prophetas.
Que tem modos de propheta ou é próprio de propheta:«\textunderscore gesto largo e prophetista.\textunderscore »Camillo, \textunderscore Mar. da Fonte\textunderscore , 154.
\section{Prophetizador}
\begin{itemize}
\item {Grp. gram.:m.  e  adj.}
\end{itemize}
O que prophetiza.
\section{Prophetizar}
\begin{itemize}
\item {Grp. gram.:v. t.}
\end{itemize}
\begin{itemize}
\item {Utilização:Fig.}
\end{itemize}
\begin{itemize}
\item {Proveniência:(Lat. \textunderscore prophetizare\textunderscore )}
\end{itemize}
Predizer como propheta; vaticinar.
Annunciar por conjecturas: \textunderscore prophetizo-te desgostos\textunderscore .
\section{Prophragma}
\begin{itemize}
\item {Grp. gram.:m.}
\end{itemize}
\begin{itemize}
\item {Utilização:Zool.}
\end{itemize}
\begin{itemize}
\item {Proveniência:(Do gr. \textunderscore pro\textunderscore , adeante, \textunderscore ephragma\textunderscore , separação)}
\end{itemize}
Separação membranosa do thórax dos insectos.
\section{Prophyláctico}
\begin{itemize}
\item {Grp. gram.:adj.}
\end{itemize}
\begin{itemize}
\item {Proveniência:(Gr. \textunderscore prophulaktikos\textunderscore )}
\end{itemize}
Relativo á prophylaxia; que preserva.
\section{Prophylaxia}
\begin{itemize}
\item {fónica:csi}
\end{itemize}
\begin{itemize}
\item {Grp. gram.:f.}
\end{itemize}
\begin{itemize}
\item {Utilização:Ext.}
\end{itemize}
\begin{itemize}
\item {Proveniência:(Gr. \textunderscore prophulaxis\textunderscore )}
\end{itemize}
Parte da Medicina, que tem por objecto as precauções contra certas enfermidades.
Emprêgo de meios, para evitar certas enfermidades.
Preservativo.
\section{Propiá}
\begin{itemize}
\item {Grp. gram.:f.}
\end{itemize}
\begin{itemize}
\item {Utilização:Prov.}
\end{itemize}
\begin{itemize}
\item {Utilização:alg.}
\end{itemize}
O mesmo que \textunderscore alcâncara\textunderscore .
\section{Propiciação}
\begin{itemize}
\item {Grp. gram.:f.}
\end{itemize}
\begin{itemize}
\item {Proveniência:(Lat. \textunderscore propitatio\textunderscore )}
\end{itemize}
Acto ou effeito de propiciar.
\section{Propiciador}
\begin{itemize}
\item {Grp. gram.:m.  e  adj.}
\end{itemize}
\begin{itemize}
\item {Proveniência:(Lat. \textunderscore propitiator\textunderscore )}
\end{itemize}
O que propicía.
\section{Propiciamente}
\begin{itemize}
\item {Grp. gram.:adv.}
\end{itemize}
De modo propício.
\section{Propiciar}
\begin{itemize}
\item {Grp. gram.:v. t.}
\end{itemize}
\begin{itemize}
\item {Proveniência:(Lat. \textunderscore propitiare\textunderscore )}
\end{itemize}
Tornar propício, favorável; proporcionar.
\section{Propiciatório}
\begin{itemize}
\item {Grp. gram.:adj.}
\end{itemize}
\begin{itemize}
\item {Proveniência:(De \textunderscore propiciar\textunderscore )}
\end{itemize}
Que propicía.
\section{Propiciatório}
\begin{itemize}
\item {Grp. gram.:m.}
\end{itemize}
\begin{itemize}
\item {Proveniência:(Lat. \textunderscore propitiatorium\textunderscore )}
\end{itemize}
Lâmina de oiro, que estava em cima do tabernáculo dos Hebreus.
Vaso sagrado, em que se offerecem sacrifícios a Deus.
Aquelle ou aquillo que applaca a ira divina.
\section{Propício}
\begin{itemize}
\item {Grp. gram.:adj.}
\end{itemize}
\begin{itemize}
\item {Proveniência:(Lat. \textunderscore propitius\textunderscore )}
\end{itemize}
Que protege, favorece ou auxilia: \textunderscore ventos propícios enfunavam as velas\textunderscore .
Benévolo: \textunderscore mostrar-se propício\textunderscore .
Opportuno, apropriado: \textunderscore em occasião propícia\textunderscore .
\section{Propilamina}
\begin{itemize}
\item {Grp. gram.:f.}
\end{itemize}
Producto farmacêutico, usado no tratamento das afeções reumáticas.
\section{Propileno}
\begin{itemize}
\item {Grp. gram.:m.}
\end{itemize}
Carboneto gasoso de hidrogênio.
\section{Propileu}
\begin{itemize}
\item {Grp. gram.:m.}
\end{itemize}
\begin{itemize}
\item {Proveniência:(Lat. \textunderscore propylaeum\textunderscore )}
\end{itemize}
Entrada vasta e monumental de antigos edifícios, aberta e circundada de colunas; pronau. Cf. Latino, \textunderscore Or. da Corôa\textunderscore , p. XXXI.
\section{Propílico}
\begin{itemize}
\item {Grp. gram.:adj.}
\end{itemize}
Relativo ao propileno.
Diz-se de um dos álcooes dos vinhos, cuja fórmula química é C^{3}H^{8}O.
\section{Propina}
\begin{itemize}
\item {Grp. gram.:f.}
\end{itemize}
\begin{itemize}
\item {Proveniência:(Lat. \textunderscore propina\textunderscore )}
\end{itemize}
Gratificação.
Aquillo que se paga em certas escolas, para a realização de alguns actos.
Quantia, paga por uma só vez pelos que são admittidos em certas associações.
\section{Propinação}
\begin{itemize}
\item {Grp. gram.:f.}
\end{itemize}
\begin{itemize}
\item {Proveniência:(Lat. \textunderscore propinatio\textunderscore )}
\end{itemize}
Acto ou effeito de propinar.
\section{Propinador}
\begin{itemize}
\item {Grp. gram.:m.  e  adj.}
\end{itemize}
\begin{itemize}
\item {Proveniência:(Lat. \textunderscore propinator\textunderscore )}
\end{itemize}
O que propina.
\section{Propinar}
\begin{itemize}
\item {Grp. gram.:v. t.}
\end{itemize}
\begin{itemize}
\item {Utilização:Fig.}
\end{itemize}
\begin{itemize}
\item {Proveniência:(Lat. \textunderscore propinare\textunderscore )}
\end{itemize}
Dar a beber.
Ministrar: \textunderscore propinar veneno\textunderscore .
\section{Propinquidade}
\begin{itemize}
\item {fónica:cu-i}
\end{itemize}
\begin{itemize}
\item {Grp. gram.:f.}
\end{itemize}
\begin{itemize}
\item {Proveniência:(Lat. \textunderscore propinquitas\textunderscore )}
\end{itemize}
Qualidade do que é propínquo.
\section{Propínquo}
\begin{itemize}
\item {Grp. gram.:adj.}
\end{itemize}
\begin{itemize}
\item {Grp. gram.:M. Pl.}
\end{itemize}
\begin{itemize}
\item {Proveniência:(Lat. \textunderscore propinquus\textunderscore )}
\end{itemize}
O mesmo que \textunderscore próximo\textunderscore ; vizinho.
Parentes.
\section{Propiteco}
\begin{itemize}
\item {Grp. gram.:m.}
\end{itemize}
\begin{itemize}
\item {Proveniência:(Do gr. \textunderscore pro\textunderscore  + \textunderscore pithekos\textunderscore )}
\end{itemize}
Pequeno mamífero, o mesmo que \textunderscore indri\textunderscore .
\section{Propitheco}
\begin{itemize}
\item {Grp. gram.:m.}
\end{itemize}
\begin{itemize}
\item {Proveniência:(Do gr. \textunderscore pro\textunderscore  + \textunderscore pithekos\textunderscore )}
\end{itemize}
Pequeno mammífero, o mesmo que \textunderscore indri\textunderscore .
\section{Proplástica}
\begin{itemize}
\item {Grp. gram.:f.}
\end{itemize}
Arte de modelação em barro.
(Fem. de \textunderscore proplastico\textunderscore )
\section{Proplático}
\begin{itemize}
\item {Grp. gram.:adj.}
\end{itemize}
\begin{itemize}
\item {Grp. gram.:M.}
\end{itemize}
\begin{itemize}
\item {Proveniência:(De \textunderscore pro\textunderscore  + \textunderscore plástico\textunderscore )}
\end{itemize}
Relativo a obras de barro.
Modêlo de barro ou de cera, para trabalhos de esculptura.
\section{Propoedor}
\begin{itemize}
\item {fónica:po-e}
\end{itemize}
\begin{itemize}
\item {Grp. gram.:m.}
\end{itemize}
\begin{itemize}
\item {Utilização:Ant.}
\end{itemize}
\begin{itemize}
\item {Proveniência:(De \textunderscore propoer\textunderscore )}
\end{itemize}
Aquelle que propõe; proponente. Cf. Herculano, \textunderscore Lendas\textunderscore , I, 64.
\section{Propoer}
\begin{itemize}
\item {Grp. gram.:v. t.}
\end{itemize}
\begin{itemize}
\item {Utilização:Ant.}
\end{itemize}
O mesmo que \textunderscore propor\textunderscore .
\section{Própolis}
\begin{itemize}
\item {Grp. gram.:m.}
\end{itemize}
\begin{itemize}
\item {Proveniência:(Gr. \textunderscore propolis\textunderscore )}
\end{itemize}
Substância resinosa, avermelhada e odorífera, que as abelhas segregam e com que tapam as fendas do respectivo cortiço.
\section{Propriedade}
\begin{itemize}
\item {Grp. gram.:f.}
\end{itemize}
\begin{itemize}
\item {Proveniência:(Lat. \textunderscore proprietas\textunderscore )}
\end{itemize}
Qualidade do que é próprio.
Carácter.
Qualidade especial.
Qualidade.
Emprêgo apropriado de palavras, phrases ou estilo, em relação ao seu objecto.
Aquillo que é pertença legítima de alguém.
Aquillo que está na posse legítima de alguém ou sôbre que alguém tem direito pleno.
Prédio, prédios; bens.
Fazenda, herdade: \textunderscore comprar uma propriedade\textunderscore .
\section{Proprietária}
\begin{itemize}
\item {Grp. gram.:f.  e  adj.}
\end{itemize}
Diz-se da mulhér, que tem propriedades ou que é senhora de certos bens.
(Fem. de \textunderscore proprietário\textunderscore )
\section{Proprietariado}
\begin{itemize}
\item {Grp. gram.:m.}
\end{itemize}
\begin{itemize}
\item {Utilização:Neol.}
\end{itemize}
A classe dos proprietários de bens immóveis; preponderância dos proprietários.
\section{Proprietário}
\begin{itemize}
\item {Grp. gram.:adj.}
\end{itemize}
\begin{itemize}
\item {Grp. gram.:M.}
\end{itemize}
\begin{itemize}
\item {Proveniência:(Lat. \textunderscore proprietarius\textunderscore )}
\end{itemize}
Que tem a propriedade de alguma coisa.
Que é senhor de alguma coisa.
Que tem propriedades ou bens immóveis.
Senhor de bens immóveis; senhor de quaesquer bens.
\section{Próprio}
\begin{itemize}
\item {Grp. gram.:adj.}
\end{itemize}
\begin{itemize}
\item {Grp. gram.:Loc. adv.}
\end{itemize}
\begin{itemize}
\item {Grp. gram.:M.}
\end{itemize}
\begin{itemize}
\item {Grp. gram.:Pl.}
\end{itemize}
\begin{itemize}
\item {Proveniência:(Lat. \textunderscore proprius\textunderscore )}
\end{itemize}
Que pertence a alguém.
Peculiar; privativo: \textunderscore perdoar offensas é próprio de grandes corações\textunderscore .
Opportuno, adequado: \textunderscore em occasião própria\textunderscore .
Idêntico.
Exacto.
Textual.
Muito semelhante.
Primitivo.
Não figurado, (falando-se do sentido das palavras).
\textunderscore Á própria\textunderscore , propriamente; com exactidão.
Qualidade ou feição especial.
Portador ou mensageiro: \textunderscore carta, mandada por um próprio\textunderscore .
\textunderscore Vir ao próprio\textunderscore , vir a propósito. Cf. \textunderscore Filodemo\textunderscore , V, 2.
\textunderscore Próprios nacionaes\textunderscore , bens próprios da nação ou do Estado.
\section{Próptero}
\begin{itemize}
\item {Grp. gram.:m.}
\end{itemize}
\begin{itemize}
\item {Proveniência:(Do gr. \textunderscore pros\textunderscore  + \textunderscore pteron\textunderscore )}
\end{itemize}
Gênero de peixes fósseis.
\section{Proptoma}
\begin{itemize}
\item {Grp. gram.:m.}
\end{itemize}
\begin{itemize}
\item {Utilização:Med.}
\end{itemize}
\begin{itemize}
\item {Proveniência:(Gr. \textunderscore proptoma\textunderscore )}
\end{itemize}
Distensão excessiva de qualquer parte do corpo.
\section{Proptose}
\begin{itemize}
\item {Grp. gram.:f.}
\end{itemize}
\begin{itemize}
\item {Utilização:Med.}
\end{itemize}
\begin{itemize}
\item {Proveniência:(Gr. \textunderscore proptosis\textunderscore )}
\end{itemize}
O mesmo que \textunderscore proptoma\textunderscore .
\section{Propugnáculo}
\begin{itemize}
\item {Grp. gram.:m.}
\end{itemize}
\begin{itemize}
\item {Utilização:Fig.}
\end{itemize}
\begin{itemize}
\item {Proveniência:(Lat. \textunderscore propugnaculum\textunderscore )}
\end{itemize}
Lugar para defesa; fortificação; baluarte.
Sustentáculo, defesa.
\section{Propugnador}
\begin{itemize}
\item {Grp. gram.:m.  e  adj.}
\end{itemize}
\begin{itemize}
\item {Proveniência:(Lat. \textunderscore propugnator\textunderscore )}
\end{itemize}
O que propugna.
\section{Propugnar}
\begin{itemize}
\item {Grp. gram.:v. t.}
\end{itemize}
\begin{itemize}
\item {Grp. gram.:V. i.}
\end{itemize}
\begin{itemize}
\item {Proveniência:(Lat. \textunderscore propugnar\textunderscore )}
\end{itemize}
Defender, combatendo.
Lutar em defesa.
\section{Propulsão}
\begin{itemize}
\item {Grp. gram.:f.}
\end{itemize}
\begin{itemize}
\item {Proveniência:(Lat. \textunderscore propulsio\textunderscore )}
\end{itemize}
Acto \textunderscore ou \textunderscore  effeito de propulsar.
\section{Propulsar}
\begin{itemize}
\item {Grp. gram.:v. t.}
\end{itemize}
\begin{itemize}
\item {Proveniência:(Lat. \textunderscore propulsare\textunderscore )}
\end{itemize}
Arremessar para longe; repellir; impellir; repulsar.
\section{Propulsionador}
\begin{itemize}
\item {Grp. gram.:adj.}
\end{itemize}
Que propulsiona.
\section{Propulsionar}
\begin{itemize}
\item {Grp. gram.:v. t.}
\end{itemize}
\begin{itemize}
\item {Utilização:bras}
\end{itemize}
\begin{itemize}
\item {Utilização:Neol.}
\end{itemize}
\begin{itemize}
\item {Proveniência:(Do lat. \textunderscore propulsio\textunderscore )}
\end{itemize}
O mesmo que \textunderscore propulsar\textunderscore .
Applicar propulsão a.
\section{Propulsivo}
\begin{itemize}
\item {Grp. gram.:adj.}
\end{itemize}
\begin{itemize}
\item {Proveniência:(Do lat. \textunderscore propulsio\textunderscore )}
\end{itemize}
Que propulsa.
\section{Propulsor}
\begin{itemize}
\item {Grp. gram.:adj.}
\end{itemize}
\begin{itemize}
\item {Grp. gram.:M.}
\end{itemize}
\begin{itemize}
\item {Proveniência:(Lat. \textunderscore propulsor\textunderscore )}
\end{itemize}
Propulsivo.
O que produz propulsão.
O que transmitte movimento a certos maquinismos.
\section{Propylamina}
\begin{itemize}
\item {Grp. gram.:f.}
\end{itemize}
Producto pharmacêutico, usado no tratamento das affecções rheumáticas.
\section{Propyleno}
\begin{itemize}
\item {Grp. gram.:m.}
\end{itemize}
Carboneto gasoso de hydrogênio.
\section{Propyleu}
\begin{itemize}
\item {Grp. gram.:m.}
\end{itemize}
\begin{itemize}
\item {Proveniência:(Lat. \textunderscore propylaeum\textunderscore )}
\end{itemize}
Entrada vasta e monumental de antigos edifícios, aberta e circundada de columnas; pronau. Cf. Latino, \textunderscore Or. da Corôa\textunderscore , p. XXXI.
\section{Propýlico}
\begin{itemize}
\item {Grp. gram.:adj.}
\end{itemize}
Relativo ao propyleno.
Diz-se de um dos álcooes dos vinhos, cuja fórmula chímica é C^{3}H^{8}O.
\section{Proquestor}
\begin{itemize}
\item {Grp. gram.:m.}
\end{itemize}
\begin{itemize}
\item {Proveniência:(Lat. \textunderscore proquaestor\textunderscore )}
\end{itemize}
Substituto do questor.
Aquelle que era enviado para uma província romana, para aí exercer as funcções de questor.
\section{Proseirão}
\begin{itemize}
\item {Grp. gram.:m.}
\end{itemize}
\begin{itemize}
\item {Proveniência:(De \textunderscore prosa\textunderscore )}
\end{itemize}
Homem prosaico, que só se preoccupa de interesses materiais; burguês. Cf. Eça, \textunderscore P. Basílio\textunderscore , 8.
\section{Proselítico}
\begin{itemize}
\item {Grp. gram.:adj.}
\end{itemize}
Relativo a proselitismo. Cf. Ol. Martins, \textunderscore Filhos de D. João I\textunderscore , 252.
\section{Proselitismo}
\begin{itemize}
\item {Grp. gram.:m.}
\end{itemize}
\begin{itemize}
\item {Proveniência:(De \textunderscore prosélito\textunderscore )}
\end{itemize}
Actividade ou afan em fazer proselitos.
Conjunto de prosélitos.
\section{Prosélito}
\begin{itemize}
\item {Grp. gram.:m.}
\end{itemize}
\begin{itemize}
\item {Utilização:Ext.}
\end{itemize}
\begin{itemize}
\item {Proveniência:(Lat. \textunderscore proselytus\textunderscore )}
\end{itemize}
Pagão, que abraçou a religião judaica.
Indivíduo que abraçou uma religião, diferente da que tinha.
Indivíduo, convertido a uma doutrina, a um sistema ou a uma ideia.
Partidário; sectário.
\section{Proselýtico}
\begin{itemize}
\item {Grp. gram.:adj.}
\end{itemize}
Relativo a proselytismo. Cf. Ol. Martins, \textunderscore Filhos de D. João I\textunderscore , 252.
\section{Proselytismo}
\begin{itemize}
\item {Grp. gram.:m.}
\end{itemize}
\begin{itemize}
\item {Proveniência:(De \textunderscore prosélyto\textunderscore )}
\end{itemize}
Actividade ou afan em fazer proselytos.
Conjunto de prosélytos.
\section{Prosélyto}
\begin{itemize}
\item {Grp. gram.:m.}
\end{itemize}
\begin{itemize}
\item {Utilização:Ext.}
\end{itemize}
\begin{itemize}
\item {Proveniência:(Lat. \textunderscore proselytus\textunderscore )}
\end{itemize}
Pagão, que abraçou a religião judaica.
Indivíduo que abraçou uma religião, differente da que tinha.
Indivíduo, convertido a uma doutrina, a um systema ou a uma ideia.
Partidário; sectário.
\section{Prosena}
\begin{itemize}
\item {Grp. gram.:f.}
\end{itemize}
\begin{itemize}
\item {Proveniência:(Do gr. \textunderscore prosenes\textunderscore )}
\end{itemize}
Gênero de insectos dípteros.
\section{Prosênchyma}
\begin{itemize}
\item {fónica:qui}
\end{itemize}
\begin{itemize}
\item {Grp. gram.:m.}
\end{itemize}
\begin{itemize}
\item {Utilização:Bot.}
\end{itemize}
\begin{itemize}
\item {Proveniência:(Do gr. \textunderscore pros\textunderscore  + \textunderscore enkhuma\textunderscore )}
\end{itemize}
Tecido cellular fibroso dos vegetaes.
\section{Prosênquima}
\begin{itemize}
\item {Grp. gram.:m.}
\end{itemize}
\begin{itemize}
\item {Utilização:Bot.}
\end{itemize}
\begin{itemize}
\item {Proveniência:(Do gr. \textunderscore pros\textunderscore  + \textunderscore enkhuma\textunderscore )}
\end{itemize}
Tecido celular fibroso dos vegetaes.
\section{Proserpínia}
\begin{itemize}
\item {Grp. gram.:f.}
\end{itemize}
\begin{itemize}
\item {Proveniência:(De \textunderscore Prosérpina\textunderscore , n. p.)}
\end{itemize}
Gênero de plantas aquáticas da América do Norte.
\section{Prosificar}
\begin{itemize}
\item {Grp. gram.:v. t.}
\end{itemize}
\begin{itemize}
\item {Utilização:Neol.}
\end{itemize}
\begin{itemize}
\item {Proveniência:(Do lat. \textunderscore prosa\textunderscore  + \textunderscore facere\textunderscore )}
\end{itemize}
Tornar prosaico. Cf. A. Corvo, \textunderscore Sentimentalismo\textunderscore .
\section{Prosissima}
\begin{itemize}
\item {Grp. gram.:adj.  e  f.}
\end{itemize}
Diz-se da prosa reles, baixa:«\textunderscore ...em prosissimas prosas deslvadas.\textunderscore »Filinto, v. 249.--A formação da palavra tem analogia com as loc. pop. \textunderscore coisíssima\textunderscore , \textunderscore verdadíssima\textunderscore , etc.
\section{Prosista}
\begin{itemize}
\item {Grp. gram.:m.  e  f.}
\end{itemize}
\begin{itemize}
\item {Utilização:Bras}
\end{itemize}
Pessôa, que escreve prosa. Cf. Latino, \textunderscore Camões\textunderscore , 58; Garrett, \textunderscore Retrato de Vénus\textunderscore , 195; Camillo, \textunderscore Caveira\textunderscore , 44.
Homem palrador; contador de historietas.
Gracejador.
\section{Proslambanómeno}
\begin{itemize}
\item {Grp. gram.:m.}
\end{itemize}
\begin{itemize}
\item {Proveniência:(Gr. \textunderscore proslambanomenos\textunderscore )}
\end{itemize}
A corda mais grave, nos antigos instrumentos gregos.
\section{Prosma}
\begin{itemize}
\item {Grp. gram.:f.}
\end{itemize}
\begin{itemize}
\item {Utilização:Pop.}
\end{itemize}
Palavreado, tretas, lábia, arenga.
(Cp. \textunderscore prosa\textunderscore )
\section{Prosmeiro}
\begin{itemize}
\item {Grp. gram.:adj.}
\end{itemize}
\begin{itemize}
\item {Utilização:Prov.}
\end{itemize}
\begin{itemize}
\item {Utilização:trasm.}
\end{itemize}
Que usa prosmas.
\section{Prosmice}
\begin{itemize}
\item {Grp. gram.:f.}
\end{itemize}
\begin{itemize}
\item {Utilização:Prov.}
\end{itemize}
\begin{itemize}
\item {Utilização:trasm.}
\end{itemize}
Prosma.
Qualidade de prosmeiro.
\section{Prosódia}
\begin{itemize}
\item {Grp. gram.:f.}
\end{itemize}
\begin{itemize}
\item {Utilização:Mús.}
\end{itemize}
\begin{itemize}
\item {Proveniência:(Lat. \textunderscore prosodia\textunderscore )}
\end{itemize}
Pronúncia regular das palavras, em harmonia com a accentuação.
Pronúncia.
Parte da Grammática, que tem por objectivo a pronúncia das palavras; orthoépia.
Bôa ligação das palavras com os accentos melódicos, de fórma que as sýllabas longas e breves mantenham a accentuação própria.
\section{Prosodicamente}
\begin{itemize}
\item {Grp. gram.:adv.}
\end{itemize}
Em harmonia com prosódia; relativamente á prosódia.
\section{Prosódico}
\begin{itemize}
\item {Grp. gram.:adj.}
\end{itemize}
\begin{itemize}
\item {Proveniência:(Lat. \textunderscore prosodicus\textunderscore )}
\end{itemize}
Relativo á prosódia.
\section{Prosonomásia}
\begin{itemize}
\item {Grp. gram.:f.}
\end{itemize}
Figura de Rhetórica, fundada na semelhança das vozes.
\section{Prosopalgia}
\begin{itemize}
\item {Grp. gram.:f.}
\end{itemize}
\begin{itemize}
\item {Proveniência:(Do gr. \textunderscore prosopon\textunderscore  + \textunderscore algos\textunderscore )}
\end{itemize}
Neuralgia facial.
\section{Prosopálgico}
\begin{itemize}
\item {Grp. gram.:adj.}
\end{itemize}
Relativo a prosopalgia.
\section{Prosopito}
\begin{itemize}
\item {Grp. gram.:m.}
\end{itemize}
\begin{itemize}
\item {Utilização:Miner.}
\end{itemize}
\begin{itemize}
\item {Proveniência:(Do gr. \textunderscore prosopon\textunderscore , \textunderscore face\textunderscore )}
\end{itemize}
Fluoreto hydrotado de cálcio e alumínio.
\section{Prosopografia}
\begin{itemize}
\item {Grp. gram.:f.}
\end{itemize}
\begin{itemize}
\item {Proveniência:(Do gr. \textunderscore prosopon\textunderscore  + \textunderscore grophein\textunderscore )}
\end{itemize}
Descripção das feições do rosto.
Esbôço de uma figura.
\section{Prosopográfico}
\begin{itemize}
\item {Grp. gram.:adj.}
\end{itemize}
Relativo á prosopografia.
\section{Prosopographia}
\begin{itemize}
\item {Grp. gram.:f.}
\end{itemize}
\begin{itemize}
\item {Proveniência:(Do gr. \textunderscore prosopon\textunderscore  + \textunderscore grophein\textunderscore )}
\end{itemize}
Descripção das feições do rosto.
Esbôço de uma figura.
\section{Prosopográphico}
\begin{itemize}
\item {Grp. gram.:adj.}
\end{itemize}
Relativo á prosopographia.
\section{Prosopopaico}
\begin{itemize}
\item {Grp. gram.:adj.}
\end{itemize}
Relativo a prosopopeia.
\section{Prosopopéa}
\begin{itemize}
\item {Grp. gram.:f.}
\end{itemize}
\begin{itemize}
\item {Utilização:Fig.}
\end{itemize}
\begin{itemize}
\item {Proveniência:(Lat. \textunderscore prosopopeia\textunderscore )}
\end{itemize}
Figura de Rhetórica, que dá acção, movimento ou voz ás coisas inanimadas, e que faz falar as pessôas ausentes e até os mortos.
Discurso empolado.
\section{Prosopopeia}
\begin{itemize}
\item {Grp. gram.:f.}
\end{itemize}
\begin{itemize}
\item {Utilização:Fig.}
\end{itemize}
\begin{itemize}
\item {Proveniência:(Lat. \textunderscore prosopopeia\textunderscore )}
\end{itemize}
Figura de Rhetórica, que dá acção, movimento ou voz ás coisas inanimadas, e que faz falar as pessôas ausentes e até os mortos.
Discurso empolado.
\section{Prospectar}
\begin{itemize}
\item {Grp. gram.:v. i.}
\end{itemize}
\begin{itemize}
\item {Proveniência:(Lat. \textunderscore prospectare\textunderscore )}
\end{itemize}
Exercer funcções de prospector.
\section{Prospectiva}
\begin{itemize}
\item {Grp. gram.:f.}
\end{itemize}
\begin{itemize}
\item {Proveniência:(De \textunderscore prospectivo\textunderscore )}
\end{itemize}
O mesmo ou melhor que \textunderscore perspectiva\textunderscore .
(Us. por M. Bernárdez e L. de Sousa)
\section{Prossilogismo}
\begin{itemize}
\item {Grp. gram.:m.}
\end{itemize}
\begin{itemize}
\item {Proveniência:(Gr. \textunderscore prosullogismos\textunderscore )}
\end{itemize}
Conclusão que, numa série polissilogística, se toma como premissa de um raciocínio subsequente.
\section{Prossilogístico}
\begin{itemize}
\item {Grp. gram.:adj.}
\end{itemize}
Relativo ao prossilogismo.
\section{Prostibulário}
\begin{itemize}
\item {Grp. gram.:m.}
\end{itemize}
Homem, que frequenta prostibulos; libertino. Cf. Camillo, \textunderscore M. de Pombal\textunderscore , 245.
\section{Prostíbulo}
\begin{itemize}
\item {Grp. gram.:m.}
\end{itemize}
\begin{itemize}
\item {Proveniência:(Lat. \textunderscore prostibulum\textunderscore )}
\end{itemize}
Lugar de prostituição.
Lupanar; alcoice; bordel.
\section{Prostilo}
\begin{itemize}
\item {Grp. gram.:m.}
\end{itemize}
\begin{itemize}
\item {Proveniência:(Gr. \textunderscore prostulus\textunderscore )}
\end{itemize}
Fachada de um templo, ornada de colunas.
Templo ou edifício, com uma só ordem de colunas na parte anterior.
\section{Prostituição}
\begin{itemize}
\item {fónica:tu-i}
\end{itemize}
\begin{itemize}
\item {Grp. gram.:f.}
\end{itemize}
\begin{itemize}
\item {Proveniência:(Lat. \textunderscore prostitutio\textunderscore )}
\end{itemize}
Acto ou effeito de prostituir.
Vida desregrada.
O conjunto das prostitutas.
Vida das prostitutas.
Profanação.
\section{Prostituidor}
\begin{itemize}
\item {fónica:tu-i}
\end{itemize}
\begin{itemize}
\item {Grp. gram.:m.  e  adj.}
\end{itemize}
\begin{itemize}
\item {Proveniência:(Do lat. \textunderscore prostitutor\textunderscore )}
\end{itemize}
O que prostitue.
\section{Prostituir}
\begin{itemize}
\item {Grp. gram.:v. t.}
\end{itemize}
\begin{itemize}
\item {Utilização:Fig.}
\end{itemize}
\begin{itemize}
\item {Proveniência:(Lat. \textunderscore prostituere\textunderscore )}
\end{itemize}
Entregar á devassidão; desmoralizar.
Aviltar; deshonrar.
\section{Prostituível}
\begin{itemize}
\item {Grp. gram.:adj.}
\end{itemize}
Que se póde prostituír.
\section{Prostituta}
\begin{itemize}
\item {Grp. gram.:f.}
\end{itemize}
\begin{itemize}
\item {Proveniência:(Lat. \textunderscore prostituta\textunderscore )}
\end{itemize}
Mulhér pública; rameira; meretriz.
\section{Prostituto}
\begin{itemize}
\item {Grp. gram.:adj.}
\end{itemize}
\begin{itemize}
\item {Proveniência:(Lat. \textunderscore prostitutus\textunderscore )}
\end{itemize}
Que se prostituiu; deshonrado. Cf. Castilho, \textunderscore Metam.\textunderscore , p. XVI.
\section{Prostração}
\begin{itemize}
\item {Grp. gram.:f.}
\end{itemize}
\begin{itemize}
\item {Proveniência:(Lat. \textunderscore prostratio\textunderscore )}
\end{itemize}
Acto ou effeito de prostrar.
Enfraquecimento; grande debilidade, procedente de doença ou de cansaço.
\section{Prostramento}
\begin{itemize}
\item {Grp. gram.:m.}
\end{itemize}
O mesmo que \textunderscore prostração\textunderscore .
\section{Prostrar}
\begin{itemize}
\item {Grp. gram.:v. t.}
\end{itemize}
\begin{itemize}
\item {Proveniência:(Lat. \textunderscore prostrare\textunderscore )}
\end{itemize}
Fazer cair.
Lançar por terra.
Humilhar.
Abater.
Destruír.
Tornar fraco ou débil.
\section{Prostylo}
\begin{itemize}
\item {Grp. gram.:m.}
\end{itemize}
\begin{itemize}
\item {Proveniência:(Gr. \textunderscore prostulus\textunderscore )}
\end{itemize}
Fachada de um templo, ornada de columnas.
Templo ou edifício, com uma só ordem de columnas na parte anterior.
\section{Prosyllogismo}
\begin{itemize}
\item {fónica:si}
\end{itemize}
\begin{itemize}
\item {Grp. gram.:m.}
\end{itemize}
\begin{itemize}
\item {Proveniência:(Gr. \textunderscore prosullogismos\textunderscore )}
\end{itemize}
Conclusão que, numa série polysyllogística, se toma como premissa de um raciocínio subsequente.
\section{Prosyllogístico}
\begin{itemize}
\item {fónica:si}
\end{itemize}
\begin{itemize}
\item {Grp. gram.:adj.}
\end{itemize}
Relativo ao prosyllogismo.
\section{Protagão}
\begin{itemize}
\item {Grp. gram.:m.}
\end{itemize}
\begin{itemize}
\item {Proveniência:(Fr. \textunderscore protagon\textunderscore )}
\end{itemize}
Substância orgânica e crystallizável, que se encontra no cérebro, e entre cujos elementos há phósphoro e azoto.
\section{Protagonista}
\begin{itemize}
\item {Grp. gram.:m.  e  f.}
\end{itemize}
\begin{itemize}
\item {Utilização:Fig.}
\end{itemize}
\begin{itemize}
\item {Proveniência:(Gr. \textunderscore protagonistes\textunderscore )}
\end{itemize}
Principal personagem de uma peça dramática.
Pessôa, que desempenha ou occupa o primeiro lugar em qualquer acontecimento.
\section{Protalífero}
\begin{itemize}
\item {Grp. gram.:adj.}
\end{itemize}
Que tem protalo; em que há o protalo.
\section{Protalo}
\begin{itemize}
\item {Grp. gram.:m.}
\end{itemize}
\begin{itemize}
\item {Proveniência:(Do gr. \textunderscore pro\textunderscore  + \textunderscore thallos\textunderscore )}
\end{itemize}
O mesmo que \textunderscore proembrião\textunderscore .
\section{Protândrico}
\begin{itemize}
\item {Grp. gram.:adj.}
\end{itemize}
\begin{itemize}
\item {Utilização:Bot.}
\end{itemize}
\begin{itemize}
\item {Proveniência:(Do gr. \textunderscore protos\textunderscore , primeiro, e \textunderscore aner\textunderscore , \textunderscore andros\textunderscore  o sêr masculino)}
\end{itemize}
Diz-se da dichogamia, em que os órgãos sexuaes masculinos se desenvolvem completamente, primeiro que os femininos.
\section{Protargol}
\begin{itemize}
\item {Grp. gram.:m.}
\end{itemize}
\begin{itemize}
\item {Utilização:Chím.}
\end{itemize}
Uma das combinações da prata, com applicações análogas á do nitrato de prata.
\section{Prótase}
\begin{itemize}
\item {Grp. gram.:f.}
\end{itemize}
\begin{itemize}
\item {Proveniência:(Gr. \textunderscore protasis\textunderscore )}
\end{itemize}
Exposição do assumpto de um poema dramático.
Primeira parte de um período grammatical.
\section{Protático}
\begin{itemize}
\item {Grp. gram.:adj.}
\end{itemize}
\begin{itemize}
\item {Proveniência:(Gr. \textunderscore protatikos\textunderscore )}
\end{itemize}
Relativo á protase.
\section{Prótea}
\begin{itemize}
\item {Grp. gram.:f.}
\end{itemize}
\begin{itemize}
\item {Proveniência:(Lat. \textunderscore protea\textunderscore )}
\end{itemize}
Gênero de plantas, do Cabo da Bôa-Esperança, uma das quaes se chama \textunderscore árvore da prata\textunderscore , por têr fôlhas brancas.
\section{Proteáceas}
\begin{itemize}
\item {Grp. gram.:f. Pl.}
\end{itemize}
Família de plantas, que tem por typo a prótea.
(Fem. pl. de \textunderscore proteáceo\textunderscore )
\section{Proteáceo}
\begin{itemize}
\item {Grp. gram.:adj.}
\end{itemize}
Relativo ou semelhante á proteia.
\section{Protecção}
\begin{itemize}
\item {Grp. gram.:f.}
\end{itemize}
\begin{itemize}
\item {Utilização:Fam.}
\end{itemize}
\begin{itemize}
\item {Proveniência:(Lat. \textunderscore protectio\textunderscore )}
\end{itemize}
Acto ou effeito de proteger.
Amparo.
Dedicação pessoal àquillo ou àquelle que delle precisa.
Auxílio.
Favor ou privilégio, concedido ao exercício de certas indústrias.
Modos ou ares de protector.
Aquelle que protege.
\section{Proteccional}
\begin{itemize}
\item {Grp. gram.:adj.}
\end{itemize}
\begin{itemize}
\item {Proveniência:(Do lat. \textunderscore protectio\textunderscore )}
\end{itemize}
Relativo ao proteccionismo.
\section{Proteccionismo}
\begin{itemize}
\item {Grp. gram.:m.}
\end{itemize}
\begin{itemize}
\item {Proveniência:(Do lat. \textunderscore protectio\textunderscore )}
\end{itemize}
Systema de protecção da indústria ou do commércio nacional.
\section{Prótese}
\begin{itemize}
\item {Grp. gram.:f.}
\end{itemize}
\begin{itemize}
\item {Utilização:Gram.}
\end{itemize}
\begin{itemize}
\item {Proveniência:(Gr. \textunderscore prothesis\textunderscore )}
\end{itemize}
Aumento de uma letra ou sílaba no principio de uma palavra, sem lhe alterar o valor: \textunderscore levantar\textunderscore  = \textunderscore alevantar\textunderscore .
Substituição de um órgão ou parte de órgão do corpo por uma peça artificial: \textunderscore prótese dentária\textunderscore .
\section{Protestar}
\begin{itemize}
\item {Grp. gram.:v. t.}
\end{itemize}
\begin{itemize}
\item {Grp. gram.:V. i.}
\end{itemize}
\begin{itemize}
\item {Proveniência:(Lat. \textunderscore protestari\textunderscore )}
\end{itemize}
Prometer terminantemente, publicamente.
Fazer o protesto de (uma letra commercial).
Affirmar solennemente.
Declarar formalmente que um facto é illegal ou que se não acceita: \textunderscore protestar contra os impostos\textunderscore .
\section{Protestatório}
\begin{itemize}
\item {Grp. gram.:adj.}
\end{itemize}
Que serve para protestar.
Que envolve protesto; que significa protesto. Cf. Garrett, \textunderscore Port. na Balança\textunderscore , 329.
\section{Protesto}
\begin{itemize}
\item {Grp. gram.:m.}
\end{itemize}
\begin{itemize}
\item {Proveniência:(De \textunderscore protestar\textunderscore )}
\end{itemize}
Protestação; desígnio inabalável.
Acto jurídico, pelo qual se declara que é responsável de todas as despesas e prejuízos aquelle que devia pagar uma letra de câmbio ou documento análogo, que não foi acceito ou pago.
\section{Protético}
\begin{itemize}
\item {Grp. gram.:adj.}
\end{itemize}
Relativo á prótese; em que há prótese.
\section{Proteu}
\begin{itemize}
\item {Grp. gram.:m.}
\end{itemize}
(V.próteo)
\section{Prothallífero}
\begin{itemize}
\item {Grp. gram.:adj.}
\end{itemize}
Que tem prothallo; em que há o prothalo.
\section{Prothallo}
\begin{itemize}
\item {Grp. gram.:m.}
\end{itemize}
\begin{itemize}
\item {Proveniência:(Do gr. \textunderscore pro\textunderscore  + \textunderscore thallos\textunderscore )}
\end{itemize}
O mesmo que \textunderscore proembryão\textunderscore .
\section{Próthese}
\begin{itemize}
\item {Grp. gram.:f.}
\end{itemize}
\begin{itemize}
\item {Utilização:Gram.}
\end{itemize}
\begin{itemize}
\item {Proveniência:(Gr. \textunderscore prothesis\textunderscore )}
\end{itemize}
Aumento de uma letra ou sýllaba no principio de uma palavra, sem lhe alterar o valor: \textunderscore levantar\textunderscore  = \textunderscore alevantar\textunderscore .
Substituição de um órgão ou parte de órgão do corpo por uma peça artificial: \textunderscore próthese dentária\textunderscore .
\section{Prothético}
\begin{itemize}
\item {Grp. gram.:adj.}
\end{itemize}
Relativo á próthese; em que há próthese.
\section{Prothórax}
\begin{itemize}
\item {Grp. gram.:m.}
\end{itemize}
\begin{itemize}
\item {Proveniência:(Do gr. \textunderscore pro\textunderscore  + \textunderscore thorax\textunderscore )}
\end{itemize}
Parte anterior do thórax das abelhas, constituída por um anel.
\section{Próthyra}
\begin{itemize}
\item {Grp. gram.:m.}
\end{itemize}
\begin{itemize}
\item {Proveniência:(Lat. \textunderscore prothyrum\textunderscore )}
\end{itemize}
Entre os Gregos, espaço, adeante da parte principal de um edifício.
Vestibulo.
Entre os Romanos, alpendre sôbre duas ou quatro columnas resaíndo do alto de uma porta. Cf. Assumpção, \textunderscore Dicionn. de Archit.\textunderscore 
\section{Prothýride}
\begin{itemize}
\item {Grp. gram.:f.}
\end{itemize}
\begin{itemize}
\item {Utilização:Constr.}
\end{itemize}
\begin{itemize}
\item {Proveniência:(Do gr. \textunderscore prothuris\textunderscore )}
\end{itemize}
Ornato, no fecho de uma arcada, e coroada por cimalha dórica.
\section{Protiodeto}
\begin{itemize}
\item {fónica:dê}
\end{itemize}
\begin{itemize}
\item {Grp. gram.:m.}
\end{itemize}
O mesmo que \textunderscore proto-iodeto\textunderscore .
\section{Prótira}
\begin{itemize}
\item {Grp. gram.:m.}
\end{itemize}
\begin{itemize}
\item {Proveniência:(Lat. \textunderscore prothyrum\textunderscore )}
\end{itemize}
Entre os Gregos, espaço, adeante da parte principal de um edifício.
Vestibulo.
Entre os Romanos, alpendre sôbre duas ou quatro columnas resaíndo do alto de uma porta. Cf. Assumpção, \textunderscore Dicionn. de Archit.\textunderscore 
\section{Protíride}
\begin{itemize}
\item {Grp. gram.:f.}
\end{itemize}
\begin{itemize}
\item {Utilização:Constr.}
\end{itemize}
\begin{itemize}
\item {Proveniência:(Do gr. \textunderscore prothuris\textunderscore )}
\end{itemize}
Ornato, no fecho de uma arcada, e coroada por cimalha dórica.
\section{Protistas}
\begin{itemize}
\item {Grp. gram.:m. pl.}
\end{itemize}
\begin{itemize}
\item {Proveniência:(Do gr. \textunderscore protos\textunderscore )}
\end{itemize}
Seres pequeníssimos, ainda não bem differenciados nos seus caractéres, que parecem communs aos animaes e ás plantas.
\section{Protistologia}
\begin{itemize}
\item {Grp. gram.:f.}
\end{itemize}
Tratado dos protistas.
\section{Proto}
\begin{itemize}
\item {Grp. gram.:m.}
\end{itemize}
\begin{itemize}
\item {Utilização:Des.}
\end{itemize}
\begin{itemize}
\item {Proveniência:(Do gr. \textunderscore protos\textunderscore , primeiro)}
\end{itemize}
Chefe de officina typográphica:«\textunderscore ...erratas de impressão, lôgro de obreiros, gatunices do proto...\textunderscore »Filinto, I, 272.
\section{Proto...}
\begin{itemize}
\item {Grp. gram.:pref.}
\end{itemize}
\begin{itemize}
\item {Proveniência:(Do gr. \textunderscore protos\textunderscore )}
\end{itemize}
(designativo de \textunderscore primeiro\textunderscore )
\section{Proto-alveitar}
\begin{itemize}
\item {Grp. gram.:m.}
\end{itemize}
\begin{itemize}
\item {Utilização:Des.}
\end{itemize}
O primeiro entre os alveitares.
\section{Proto-barbeirato}
\begin{itemize}
\item {Grp. gram.:m.}
\end{itemize}
Corporação espanhola, encarregada de examinar barbeiros e sangradores e passar-lhes os respectivos títulos.
\section{Protoblástio}
\begin{itemize}
\item {Grp. gram.:m.}
\end{itemize}
\begin{itemize}
\item {Utilização:Hist. Nat.}
\end{itemize}
\begin{itemize}
\item {Proveniência:(Do gr. \textunderscore proton\textunderscore  + \textunderscore blastos\textunderscore )}
\end{itemize}
Céllula animal ou vegetal, cuja parede se não distingue da cavidade.
\section{Protobrometo}
\begin{itemize}
\item {fónica:mê}
\end{itemize}
\begin{itemize}
\item {Grp. gram.:m.}
\end{itemize}
\begin{itemize}
\item {Utilização:Chím.}
\end{itemize}
\begin{itemize}
\item {Proveniência:(De \textunderscore proto...\textunderscore  + \textunderscore brometo\textunderscore )}
\end{itemize}
Primeiro grau de combinação de um corpo simples com o bromo.
\section{Protocanónico}
\begin{itemize}
\item {Grp. gram.:adj.}
\end{itemize}
\begin{itemize}
\item {Proveniência:(De \textunderscore proto...\textunderscore  + \textunderscore canónico\textunderscore )}
\end{itemize}
Diz-se dos livros santos, que eram já reconhecidos como canónicos, antes de se formarem os cânones da Escritura.
\section{Protochloreto}
\begin{itemize}
\item {fónica:clorê}
\end{itemize}
\begin{itemize}
\item {Grp. gram.:m.}
\end{itemize}
\begin{itemize}
\item {Proveniência:(De \textunderscore proto...\textunderscore  + \textunderscore chloreto\textunderscore )}
\end{itemize}
Primeiro grau de combinação de um corpo simples como chloro.
\section{Protocloreto}
\begin{itemize}
\item {fónica:clorê}
\end{itemize}
\begin{itemize}
\item {Grp. gram.:m.}
\end{itemize}
\begin{itemize}
\item {Proveniência:(De \textunderscore proto...\textunderscore  + \textunderscore cloreto\textunderscore )}
\end{itemize}
Primeiro grau de combinação de um corpo simples como cloro.
\section{Protococco}
\begin{itemize}
\item {Grp. gram.:m.}
\end{itemize}
\begin{itemize}
\item {Proveniência:(Do gr. \textunderscore protos\textunderscore  + \textunderscore kokkos\textunderscore )}
\end{itemize}
Gênero de algas unicellulares.
\section{Protococo}
\begin{itemize}
\item {Grp. gram.:m.}
\end{itemize}
\begin{itemize}
\item {Proveniência:(Do gr. \textunderscore protos\textunderscore  + \textunderscore kokkos\textunderscore )}
\end{itemize}
Gênero de algas unicelulares.
\section{Protopirâmide}
\begin{itemize}
\item {Grp. gram.:f.}
\end{itemize}
\begin{itemize}
\item {Proveniência:(De \textunderscore proto\textunderscore  + \textunderscore piramide\textunderscore )}
\end{itemize}
Pirâmide de primeira orde, em Metalologia.
\section{Protoplasmático}
\begin{itemize}
\item {Grp. gram.:adj.}
\end{itemize}
Relativo ao protoplasma.
\section{Protoplásmico}
\begin{itemize}
\item {Grp. gram.:adj.}
\end{itemize}
O mesmo que \textunderscore protoplasmático\textunderscore .
\section{Protopódio}
\begin{itemize}
\item {Grp. gram.:m.}
\end{itemize}
\begin{itemize}
\item {Utilização:Zool.}
\end{itemize}
\begin{itemize}
\item {Proveniência:(Do gr. \textunderscore protos\textunderscore  + \textunderscore pous\textunderscore , \textunderscore podos\textunderscore )}
\end{itemize}
Parte basilar do appêndice de um crustáceo.
\section{Protoprisma}
\begin{itemize}
\item {Grp. gram.:m.}
\end{itemize}
\begin{itemize}
\item {Proveniência:(De \textunderscore proto...\textunderscore  + \textunderscore prisma\textunderscore )}
\end{itemize}
Prisma de primeira ordem, em Metallogia.
\section{Protóptero}
\begin{itemize}
\item {Grp. gram.:m.}
\end{itemize}
\begin{itemize}
\item {Proveniência:(Do gr. \textunderscore protos\textunderscore  + \textunderscore pteron\textunderscore )}
\end{itemize}
Espécie de batrácio, de membros rudimentares.
\section{Protopyrâmide}
\begin{itemize}
\item {Grp. gram.:f.}
\end{itemize}
\begin{itemize}
\item {Proveniência:(De \textunderscore proto\textunderscore  + \textunderscore pyramide\textunderscore )}
\end{itemize}
Pyrâmide de primeira orde, em Metallologia.
\section{Protórax}
\begin{itemize}
\item {Grp. gram.:m.}
\end{itemize}
\begin{itemize}
\item {Proveniência:(Do gr. \textunderscore pro\textunderscore  + \textunderscore thorax\textunderscore )}
\end{itemize}
Parte anterior do tórax das abelhas, constituída por um anel.
\section{Proto-revolucionário}
\begin{itemize}
\item {Grp. gram.:m.}
\end{itemize}
O primeiro que se revoltou. Cf. R. Jorge, \textunderscore Origem e Desenv. da ep. do Pôrto\textunderscore , 22.
\section{Protorhomboédro}
\begin{itemize}
\item {Grp. gram.:m.}
\end{itemize}
\begin{itemize}
\item {Proveniência:(De \textunderscore proto...\textunderscore  + \textunderscore rhomboédro\textunderscore )}
\end{itemize}
Rhomboédro de primeira ordem, em Metallologia.
\section{Protorromboédro}
\begin{itemize}
\item {Grp. gram.:m.}
\end{itemize}
\begin{itemize}
\item {Proveniência:(De \textunderscore proto...\textunderscore  + \textunderscore romboédro\textunderscore )}
\end{itemize}
Romboédro de primeira ordem, em Metalologia.
\section{Proto-sulfureto}
\begin{itemize}
\item {Grp. gram.:m.}
\end{itemize}
\begin{itemize}
\item {Utilização:Miner.}
\end{itemize}
\begin{itemize}
\item {Proveniência:(De \textunderscore proto...\textunderscore  + \textunderscore sulfureto\textunderscore )}
\end{itemize}
Sulfureto da primeira das três ordens, em que os sulfuretos se dividem.
\section{Prototípico}
\begin{itemize}
\item {Grp. gram.:adj.}
\end{itemize}
Relativo a protótipo; que tem o carácter de protótipo.
\section{Protótipo}
\begin{itemize}
\item {Grp. gram.:m.}
\end{itemize}
\begin{itemize}
\item {Proveniência:(De \textunderscore proto...\textunderscore  + \textunderscore tipo\textunderscore )}
\end{itemize}
Primeiro tipo, modêlo; aquilo que é exemplar.
\section{Prototipográfico}
\begin{itemize}
\item {Grp. gram.:adj.}
\end{itemize}
\begin{itemize}
\item {Proveniência:(De \textunderscore proto...\textunderscore  + \textunderscore tipográfico\textunderscore )}
\end{itemize}
Anterior á invenção da tipografia.
\section{Prototýpico}
\begin{itemize}
\item {Grp. gram.:adj.}
\end{itemize}
Relativo a protótypo; que tem o carácter de protótypo.
\section{Protótypo}
\begin{itemize}
\item {Grp. gram.:m.}
\end{itemize}
\begin{itemize}
\item {Proveniência:(De \textunderscore proto...\textunderscore  + \textunderscore typo\textunderscore )}
\end{itemize}
Primeiro typo, modêlo; aquillo que é exemplar.
\section{Prototypográphico}
\begin{itemize}
\item {Grp. gram.:adj.}
\end{itemize}
\begin{itemize}
\item {Proveniência:(De \textunderscore proto...\textunderscore  + \textunderscore typográphico\textunderscore )}
\end{itemize}
Anterior á invenção da typographia.
\section{Protovértebra}
\begin{itemize}
\item {Grp. gram.:f.}
\end{itemize}
\begin{itemize}
\item {Proveniência:(De \textunderscore proto...\textunderscore  + \textunderscore vértebra\textunderscore )}
\end{itemize}
Vértebra primitiva ou rudimentar.
\section{Protovertebral}
\begin{itemize}
\item {Grp. gram.:adj.}
\end{itemize}
Relativo á protovértebra.
\section{Protoxidado}
\begin{itemize}
\item {fónica:csi}
\end{itemize}
\begin{itemize}
\item {Grp. gram.:adj.}
\end{itemize}
\begin{itemize}
\item {Utilização:Chím.}
\end{itemize}
Convertido em estado de protóxido.
\section{Protóxido}
\begin{itemize}
\item {fónica:csi}
\end{itemize}
\begin{itemize}
\item {Grp. gram.:m.}
\end{itemize}
\begin{itemize}
\item {Utilização:Chím.}
\end{itemize}
\begin{itemize}
\item {Proveniência:(De \textunderscore proto...\textunderscore  + \textunderscore óxido\textunderscore )}
\end{itemize}
O primeiro gráu de oxidação de um corpo simples.
\section{Protoxydado}
\begin{itemize}
\item {fónica:csi}
\end{itemize}
\begin{itemize}
\item {Grp. gram.:adj.}
\end{itemize}
\begin{itemize}
\item {Utilização:Chím.}
\end{itemize}
Convertido em estado de protóxydo.
\section{Protóxydo}
\begin{itemize}
\item {fónica:csi}
\end{itemize}
\begin{itemize}
\item {Grp. gram.:m.}
\end{itemize}
\begin{itemize}
\item {Utilização:Chím.}
\end{itemize}
\begin{itemize}
\item {Proveniência:(De \textunderscore proto...\textunderscore  + \textunderscore óxydo\textunderscore )}
\end{itemize}
O primeiro gráu de oxydação de um corpo simples.
\section{Protozoário}
\begin{itemize}
\item {Grp. gram.:adj.}
\end{itemize}
\begin{itemize}
\item {Utilização:Zool.}
\end{itemize}
\begin{itemize}
\item {Grp. gram.:M.}
\end{itemize}
\begin{itemize}
\item {Proveniência:(Do gr. \textunderscore protos\textunderscore  + \textunderscore zoarion\textunderscore )}
\end{itemize}
Diz-se dos animaes, que têm a conformação mais rudimentar ou mais simples.
Cada um dêsses animaes.
\section{Protozóides}
\begin{itemize}
\item {Grp. gram.:m. Pl.}
\end{itemize}
\begin{itemize}
\item {Proveniência:(Do gr. \textunderscore protos\textunderscore  + \textunderscore zoon\textunderscore )}
\end{itemize}
O mesmo que [[espermatozoides|espermatozoide]], para os physiolistas que os consideram céllulas simples.
\section{Protrahimento}
\begin{itemize}
\item {fónica:tra-i}
\end{itemize}
\begin{itemize}
\item {Grp. gram.:m.}
\end{itemize}
Acto ou effeito de protrahir; delonga, adiamento.
\section{Protrahir}
\begin{itemize}
\item {Grp. gram.:v. t.}
\end{itemize}
\begin{itemize}
\item {Proveniência:(Lat. \textunderscore protrahere\textunderscore )}
\end{itemize}
Tirar para fóra.
Prolongar; adiar; procrastinar; espaçar.
\section{Protrahível}
\begin{itemize}
\item {Grp. gram.:adj.}
\end{itemize}
Que se póde protrahir.
\section{Protraimento}
\begin{itemize}
\item {fónica:tra-i}
\end{itemize}
\begin{itemize}
\item {Grp. gram.:m.}
\end{itemize}
Acto ou efeito de protrair; delonga, adiamento.
\section{Protrair}
\begin{itemize}
\item {Grp. gram.:v. t.}
\end{itemize}
\begin{itemize}
\item {Proveniência:(Lat. \textunderscore protrahere\textunderscore )}
\end{itemize}
Tirar para fóra.
Prolongar; adiar; procrastinar; espaçar.
\section{Protraível}
\begin{itemize}
\item {Grp. gram.:adj.}
\end{itemize}
Que se póde protrair.
\section{Protuberância}
\begin{itemize}
\item {Grp. gram.:f.}
\end{itemize}
\begin{itemize}
\item {Proveniência:(De \textunderscore protuberante\textunderscore )}
\end{itemize}
Coisa saliente; eminência.
Elevação gazoza e inflammada, em certos pontos do globo solar.
\section{Protuberante}
\begin{itemize}
\item {Grp. gram.:adj.}
\end{itemize}
\begin{itemize}
\item {Proveniência:(Lat. \textunderscore protuberans\textunderscore )}
\end{itemize}
Que tem protuberância.
\section{Protutela}
\begin{itemize}
\item {Grp. gram.:f.}
\end{itemize}
\begin{itemize}
\item {Proveniência:(Lat. \textunderscore protutela\textunderscore )}
\end{itemize}
Cargo ou funcções de protutor.
Tempo, durante o qual um protutor exerce as suas funcções.
\section{Protutor}
\begin{itemize}
\item {Grp. gram.:m.}
\end{itemize}
\begin{itemize}
\item {Proveniência:(De \textunderscore pro...\textunderscore  + \textunderscore tutor\textunderscore )}
\end{itemize}
Aquelle que, em substituição do tutor, ou juntamente com êlle e com o conselho de família, exerce a tutela.
\section{Prouveia}
\begin{itemize}
\item {Grp. gram.:f.}
\end{itemize}
\begin{itemize}
\item {Utilização:Prov.}
\end{itemize}
\begin{itemize}
\item {Utilização:trasm.}
\end{itemize}
O mesmo que \textunderscore parouveia\textunderscore .
\section{Prova}
\begin{itemize}
\item {Grp. gram.:f.}
\end{itemize}
\begin{itemize}
\item {Proveniência:(Lat. \textunderscore proba\textunderscore )}
\end{itemize}
Aquillo que mostra a verdade de uma proposição ou a realidade de um facto.
Testemunho.
Indício.
Documento justificativo.
Porfia, competência.
Experiência.
Verificação de uma operação de cálculo.
Acto de provar ou de experimentar no paladar uma substância alimentícia.
Exame das qualidades de algumas substâncias.
Transe; situação afflictiva.
Exemplar de uma composição typográphica, para nelle se fazerem correções, antes da impressão definitiva.
\section{Provençalesco}
\begin{itemize}
\item {fónica:lês}
\end{itemize}
\begin{itemize}
\item {Grp. gram.:adj.}
\end{itemize}
Relativo á poesia provençal ou aos poétas provençaes. Cf. Th. Braga, \textunderscore Camões\textunderscore , 1.
\section{Provençalismo}
\begin{itemize}
\item {Grp. gram.:m.}
\end{itemize}
Influência da literatura provençal.
Escola dos poétas provençaes. Cf. Ol. Martins, \textunderscore Camões\textunderscore , 287.
\section{Provençalista}
\begin{itemize}
\item {Grp. gram.:m.}
\end{itemize}
Aquelle que é perito na língua e literatura provençaes.
\section{Provência}
\begin{itemize}
\item {Grp. gram.:f.}
\end{itemize}
\begin{itemize}
\item {Utilização:Ant.}
\end{itemize}
O mesmo que \textunderscore província\textunderscore .
\section{Provenda}
\begin{itemize}
\item {Grp. gram.:f.}
\end{itemize}
\begin{itemize}
\item {Utilização:Ant.}
\end{itemize}
\begin{itemize}
\item {Proveniência:(De \textunderscore prover\textunderscore )}
\end{itemize}
Provisão de mantimentos.
Estaleiro, onde os navios se consertavam.
\section{Proveniência}
\begin{itemize}
\item {Grp. gram.:f.}
\end{itemize}
\begin{itemize}
\item {Proveniência:(De \textunderscore proveniente\textunderscore )}
\end{itemize}
Lugar, donde provém alguma coisa.
Procedência, origem.
\section{Proveniente}
\begin{itemize}
\item {Grp. gram.:adj.}
\end{itemize}
\begin{itemize}
\item {Proveniência:(Lat. \textunderscore proveniens\textunderscore )}
\end{itemize}
Que provém; procedente; oriundo.
\section{Provento}
\begin{itemize}
\item {Grp. gram.:m.}
\end{itemize}
\begin{itemize}
\item {Proveniência:(Lat. \textunderscore proventus\textunderscore )}
\end{itemize}
Proveito; rendimento, lucro: \textunderscore os proventos da usura\textunderscore .
\section{Prover}
\begin{itemize}
\item {Grp. gram.:v. t.}
\end{itemize}
\begin{itemize}
\item {Grp. gram.:V. i.}
\end{itemize}
\begin{itemize}
\item {Proveniência:(Do lat. \textunderscore providere\textunderscore )}
\end{itemize}
Tomar providências á cêrca de.
Regular.
Fornecer.
Ornar.
Despachar, nomear, promover: \textunderscore prover alguém num emprégo\textunderscore .
Despachar ou nomear alguém para: \textunderscore prover um lugar vago\textunderscore .
Dar providências.
Occorrer.
Acudir; dar remédio: \textunderscore prover ás necessidades do hospital\textunderscore .
\section{Proverbial}
\begin{itemize}
\item {Grp. gram.:adj.}
\end{itemize}
\begin{itemize}
\item {Proveniência:(Lat. \textunderscore proverbialis\textunderscore )}
\end{itemize}
Relativo a provérbio.
Conhecido, notório: \textunderscore a sua proverbial amabilidade\textunderscore .
\section{Proverbialmente}
\begin{itemize}
\item {Grp. gram.:adj.}
\end{itemize}
De modo proverbial.
\section{Proverbiar}
\begin{itemize}
\item {Grp. gram.:v. i.}
\end{itemize}
Usar muito de provérbios.
\section{Provérbio}
\begin{itemize}
\item {Grp. gram.:m.}
\end{itemize}
\begin{itemize}
\item {Proveniência:(Lat. \textunderscore proverbium\textunderscore )}
\end{itemize}
Sentença moral, máxima expressa em poucas palavras.
Anexim; rifão.
Desenvolvimento de uma sentença moral ou rifão, numa peça dramática: \textunderscore Musset escreveu alguns provérbios para o theatro\textunderscore .
\section{Proveta}
\begin{itemize}
\item {fónica:vê}
\end{itemize}
\begin{itemize}
\item {Grp. gram.:f.}
\end{itemize}
\begin{itemize}
\item {Proveniência:(De \textunderscore prova\textunderscore )}
\end{itemize}
Espécie de redoma pequena, para conter substâncias gasosas.
Vaso cylíndrico ou cónico, convenientemente graduado, para medição de líquidos.
\section{Provete}
\begin{itemize}
\item {fónica:vê}
\end{itemize}
\begin{itemize}
\item {Grp. gram.:m.}
\end{itemize}
\begin{itemize}
\item {Proveniência:(De \textunderscore prova\textunderscore )}
\end{itemize}
Pequeno morteiro para experiências de pólvora.
\section{Provete}
\begin{itemize}
\item {fónica:vê}
\end{itemize}
\begin{itemize}
\item {Grp. gram.:m.}
\end{itemize}
\begin{itemize}
\item {Utilização:Pop.}
\end{itemize}
\begin{itemize}
\item {Proveniência:(De \textunderscore provar\textunderscore )}
\end{itemize}
O mesmo que \textunderscore areómetro\textunderscore .
\section{Proveúdo}
\begin{itemize}
\item {Grp. gram.:adj.}
\end{itemize}
\begin{itemize}
\item {Utilização:Ant.}
\end{itemize}
O mesmo que \textunderscore próvido\textunderscore .
\section{Providamente}
\begin{itemize}
\item {Grp. gram.:adv.}
\end{itemize}
De modo próvido.
Providentemente.
\section{Providência}
\begin{itemize}
\item {Grp. gram.:f.}
\end{itemize}
\begin{itemize}
\item {Proveniência:(Lat. \textunderscore providentia\textunderscore )}
\end{itemize}
Sabedoria suprema, com que Deus tudo dirige.
Deus, considerado na sua providência.
Successo feliz.
Previdência.
Prevenção.
Disposições ou instrucções, com que se trata de regularizar certos serviços ou prover de remédio a algumas irregularidades: \textunderscore o Govêrno tomou providências sôbre o caso\textunderscore .
Variedade de pequena pêra, muito apreciada.
\section{Providencial}
\begin{itemize}
\item {Grp. gram.:adj.}
\end{itemize}
Relativo a providência.
Que produziu os melhores e necessários resultados: \textunderscore foi um auxílio providencial\textunderscore .
\section{Providencialismo}
\begin{itemize}
\item {Grp. gram.:m.}
\end{itemize}
\begin{itemize}
\item {Proveniência:(De \textunderscore providencial\textunderscore )}
\end{itemize}
Systema dos que tudo attribuem á acção da providência divina. Cf. Ol. Martins, \textunderscore Port. nos Mares\textunderscore , XIV.
\section{Providencialista}
\begin{itemize}
\item {Grp. gram.:m.}
\end{itemize}
Sectário do providencialismo. Cf. Th. Braga, \textunderscore Mod. Ideias\textunderscore , I, 95.
\section{Providencialmente}
\begin{itemize}
\item {Grp. gram.:adv.}
\end{itemize}
De modo providencial.
\section{Providenciar}
\begin{itemize}
\item {Grp. gram.:v. i.}
\end{itemize}
\begin{itemize}
\item {Grp. gram.:V. t.}
\end{itemize}
Tomar providências.
Prover, dispor providentemente.
\section{Providente}
\begin{itemize}
\item {Grp. gram.:adj.}
\end{itemize}
\begin{itemize}
\item {Proveniência:(Lat. \textunderscore providens\textunderscore )}
\end{itemize}
Que provê; providencial.
\section{Providentemente}
\begin{itemize}
\item {Grp. gram.:adv.}
\end{itemize}
De modo providente.
\section{Próvido}
\begin{itemize}
\item {Grp. gram.:adj.}
\end{itemize}
\begin{itemize}
\item {Proveniência:(Lat. \textunderscore providus\textunderscore )}
\end{itemize}
O mesmo que \textunderscore providente\textunderscore .
\section{Provisório}
\begin{itemize}
\item {Grp. gram.:adj.}
\end{itemize}
\begin{itemize}
\item {Proveniência:(Do lat. \textunderscore provisus\textunderscore )}
\end{itemize}
Feito por provisão.
Interino; passageiro, transitório; temporário.
\section{Provo}
\begin{itemize}
\item {Grp. gram.:m.}
\end{itemize}
\begin{itemize}
\item {Utilização:P. us.}
\end{itemize}
Acto de provar (vinho); prova.
\section{Provocação}
\begin{itemize}
\item {Grp. gram.:f.}
\end{itemize}
\begin{itemize}
\item {Proveniência:(Lat. \textunderscore provocatio\textunderscore )}
\end{itemize}
Acto ou effeito de provocar; desafio; tentação.
\section{Provocador}
\begin{itemize}
\item {Grp. gram.:m.  e  adj.}
\end{itemize}
\begin{itemize}
\item {Proveniência:(Lat. \textunderscore provocator\textunderscore )}
\end{itemize}
O que provoca; tentador.
\section{Provocante}
\begin{itemize}
\item {Grp. gram.:adj.}
\end{itemize}
\begin{itemize}
\item {Proveniência:(Lat. \textunderscore provocans\textunderscore )}
\end{itemize}
Que provoca; provocador.
\section{Provocar}
\begin{itemize}
\item {Grp. gram.:v. t.}
\end{itemize}
\begin{itemize}
\item {Proveniência:(Lat. \textunderscore provocare\textunderscore )}
\end{itemize}
Incitar; desafiar; estimular.
Injuriar.
Attrahir, tentar.
Originar, dar causa a.
Promover; facilitar: \textunderscore provocar tumultos\textunderscore .
\section{Provocativo}
\begin{itemize}
\item {Grp. gram.:adj.}
\end{itemize}
\begin{itemize}
\item {Proveniência:(Lat. \textunderscore provocatorius\textunderscore )}
\end{itemize}
O mesmo que \textunderscore provocante\textunderscore :«\textunderscore comédias provocativas a peccado...\textunderscore »M. Bernárdez.
\section{Proxeneta}
\begin{itemize}
\item {fónica:cse}
\end{itemize}
\begin{itemize}
\item {Grp. gram.:m.}
\end{itemize}
\begin{itemize}
\item {Proveniência:(Lat. \textunderscore proxeneta\textunderscore )}
\end{itemize}
Indivíduo intermediário; mediador.
Corretor de negócios.
\section{Proxenético}
\begin{itemize}
\item {fónica:cse}
\end{itemize}
\begin{itemize}
\item {Grp. gram.:adj.}
\end{itemize}
Relativo ao proxeneta.
\section{Proxenetismo}
\begin{itemize}
\item {fónica:cse}
\end{itemize}
\begin{itemize}
\item {Grp. gram.:m.}
\end{itemize}
Qualidade ou profissão de proxeneta.
\section{Proximal}
\begin{itemize}
\item {fónica:si}
\end{itemize}
\begin{itemize}
\item {Grp. gram.:adj.}
\end{itemize}
\begin{itemize}
\item {Utilização:Anat.}
\end{itemize}
\begin{itemize}
\item {Proveniência:(De \textunderscore próximo\textunderscore )}
\end{itemize}
Situado perto das origens dos membros do corpo.
Que fica para o lado da cabeça.
\section{Proximamente}
\begin{itemize}
\item {fónica:si}
\end{itemize}
\begin{itemize}
\item {Grp. gram.:adv.}
\end{itemize}
Em sitio próximo; perto; quási, aproximadamente.
\section{Proximidade}
\begin{itemize}
\item {fónica:si}
\end{itemize}
\begin{itemize}
\item {Grp. gram.:f.}
\end{itemize}
\begin{itemize}
\item {Grp. gram.:Pl.}
\end{itemize}
\begin{itemize}
\item {Proveniência:(Lat. \textunderscore proximitas\textunderscore )}
\end{itemize}
Condição ou estado do que é próximo.
Contiguidade; pequena distância.
Pequena demora.
Vizinhança; cercanias.
\section{Próximo}
\begin{itemize}
\item {fónica:si}
\end{itemize}
\begin{itemize}
\item {Grp. gram.:adj.}
\end{itemize}
\begin{itemize}
\item {Grp. gram.:M.}
\end{itemize}
\begin{itemize}
\item {Proveniência:(Lat. \textunderscore proximus\textunderscore )}
\end{itemize}
Que está perto; vizinho.
Que chega breve.
Immediato.
Seguinte ao actual: \textunderscore no próximo Dezembro\textunderscore .
Cada pessôa:«\textunderscore ...se algum próximo te fizer injúria...\textunderscore »M. Bernárdez.
O conjunto de todos os homens: \textunderscore amar a Deus e ao próximo\textunderscore .
\section{Prozar}
\begin{itemize}
\item {Grp. gram.:v. i.}
\end{itemize}
\begin{itemize}
\item {Utilização:Prov.}
\end{itemize}
\begin{itemize}
\item {Utilização:alg.}
\end{itemize}
Dar-se bem num terreno ou medrar (a planta).
\section{Prozoico}
\begin{itemize}
\item {Grp. gram.:adj.}
\end{itemize}
\begin{itemize}
\item {Proveniência:(Do gr. \textunderscore pro\textunderscore  + \textunderscore zoon\textunderscore )}
\end{itemize}
Anterior ao apparecimento dos seres vivos.
\section{Prudência}
\begin{itemize}
\item {Grp. gram.:f.}
\end{itemize}
\begin{itemize}
\item {Proveniência:(Lat. \textunderscore prudentia\textunderscore )}
\end{itemize}
Virtude, que leva os homens a conhecer e praticar o que lhes convém.
Seriedade, tino.
Moderação.
Cordura.
Precaução.
\section{Prudencial}
\begin{itemize}
\item {Grp. gram.:adj.}
\end{itemize}
Relativo a prudência.
\section{Prudencialmente}
\begin{itemize}
\item {Grp. gram.:adv.}
\end{itemize}
De modo prudencial.
\section{Prudente}
\begin{itemize}
\item {Grp. gram.:adv.}
\end{itemize}
\begin{itemize}
\item {Proveniência:(Lat. \textunderscore prudens\textunderscore )}
\end{itemize}
Que tem prudência, moderação ou commedimento.
Cauteloso; previdente.
Judicioso.
\section{Prudentemente}
\begin{itemize}
\item {Grp. gram.:adv.}
\end{itemize}
De modo prudente; com prudência; com discrição.
\section{Pruga}
\begin{itemize}
\item {Grp. gram.:f.}
\end{itemize}
\begin{itemize}
\item {Utilização:Pop.}
\end{itemize}
O mesmo que \textunderscore purga\textunderscore .
\section{Prugunta}
\begin{itemize}
\item {Grp. gram.:f.}
\end{itemize}
\begin{itemize}
\item {Utilização:T. de Ceilão}
\end{itemize}
O mesmo que \textunderscore pergunta\textunderscore .
\section{Pruído}
\begin{itemize}
\item {Grp. gram.:m.}
\end{itemize}
\begin{itemize}
\item {Proveniência:(De \textunderscore pruír\textunderscore )}
\end{itemize}
O mesmo que \textunderscore prurido\textunderscore .
\section{Pruír}
\begin{itemize}
\item {Grp. gram.:v. t.  e  i.}
\end{itemize}
O mesmo que \textunderscore prurir\textunderscore .
\section{Pruma}
\begin{itemize}
\item {Grp. gram.:f.}
\end{itemize}
\begin{itemize}
\item {Utilização:T. de Gaia}
\end{itemize}
\begin{itemize}
\item {Utilização:Ant.}
\end{itemize}
O mesmo que \textunderscore caruma\textunderscore .
O mesmo que \textunderscore pluma\textunderscore :«\textunderscore ...nem de cousa de pruma nem de rez.\textunderscore »D. Bernárdez, \textunderscore Lima\textunderscore , 100.
\section{Prumada}
\begin{itemize}
\item {Grp. gram.:f.}
\end{itemize}
Vertical da linha de prumo.
\section{Prumagem}
\begin{itemize}
\item {Grp. gram.:f.}
\end{itemize}
\begin{itemize}
\item {Utilização:Prov.}
\end{itemize}
\begin{itemize}
\item {Utilização:beir.}
\end{itemize}
O mesmo que \textunderscore prumada\textunderscore .
Estaca de oliveira, para reproducção por plantio.
\section{Prumagem}
\begin{itemize}
\item {Grp. gram.:f.}
\end{itemize}
\begin{itemize}
\item {Utilização:Ant.}
\end{itemize}
O mesmo que \textunderscore plumagem\textunderscore . Cf. G. Vicente, \textunderscore Templo de Apollo\textunderscore .
\section{Prumante}
\begin{itemize}
\item {Grp. gram.:f.}
\end{itemize}
\begin{itemize}
\item {Utilização:Prov.}
\end{itemize}
\begin{itemize}
\item {Utilização:minh.}
\end{itemize}
A parede que, na chaminé, fica a prumo.
\section{Psalmo}
\begin{itemize}
\item {fónica:sal}
\end{itemize}
\begin{itemize}
\item {Grp. gram.:m.}
\end{itemize}
\begin{itemize}
\item {Proveniência:(Lat. \textunderscore psalmus\textunderscore )}
\end{itemize}
Cada um dos cânticos, attribuídos a David.
Cântico de louvor a Deus.
\section{Psalmodejar}
\begin{itemize}
\item {fónica:sal}
\end{itemize}
\begin{itemize}
\item {Grp. gram.:v. t.}
\end{itemize}
O mesmo que \textunderscore psalmodiar\textunderscore . Cf. Alv. Mendes, \textunderscore Discursos\textunderscore , 39.
\section{Psalmodia}
\begin{itemize}
\item {fónica:sal}
\end{itemize}
\begin{itemize}
\item {Grp. gram.:f.}
\end{itemize}
\begin{itemize}
\item {Utilização:Fig.}
\end{itemize}
\begin{itemize}
\item {Proveniência:(Lat. \textunderscore psalmodia\textunderscore )}
\end{itemize}
Maneira de cantar ou recitar psalmos.
Monotonia em declamar, recitar, lêr ou escrever.
(A pron. exacta é \textunderscore salmódia\textunderscore , mas não se usa.)
\section{Psalmodiar}
\begin{itemize}
\item {fónica:sal}
\end{itemize}
\begin{itemize}
\item {Grp. gram.:v. t.  e  i.}
\end{itemize}
O mesmo que \textunderscore psalmear\textunderscore .
\section{Psaltério}
\begin{itemize}
\item {fónica:sal}
\end{itemize}
\begin{itemize}
\item {Grp. gram.:m.}
\end{itemize}
\begin{itemize}
\item {Utilização:Veter.}
\end{itemize}
\begin{itemize}
\item {Proveniência:(Lat. \textunderscore psalterium\textunderscore )}
\end{itemize}
Instrumento musical de cordas, que se dedilhavam ou se tocavam com o plectro.
Instrumento triangular moderno, com treze ordens de cordas, que se ferem com uma palheta.
O mesmo que \textunderscore folhoso\textunderscore , terceiro estômago dos ruminantes. Cf. Mac. Pinto, \textunderscore Comp. de Veter.\textunderscore , I, 468.
\section{Psáltria}
\begin{itemize}
\item {fónica:sal}
\end{itemize}
\begin{itemize}
\item {Grp. gram.:f.}
\end{itemize}
\begin{itemize}
\item {Proveniência:(Lat. \textunderscore psaltria\textunderscore )}
\end{itemize}
Mulhér que tocava cíthara. Cf. Camillo, \textunderscore Maria da Fonte\textunderscore , 337.
\section{Psamito}
\begin{itemize}
\item {Grp. gram.:m.}
\end{itemize}
\begin{itemize}
\item {Utilização:Geol.}
\end{itemize}
Argilla granulosa dos terrenos fossilíferos.
\section{Psammito}
\begin{itemize}
\item {Grp. gram.:m.}
\end{itemize}
\begin{itemize}
\item {Utilização:Geol.}
\end{itemize}
Argilla granulosa dos terrenos fossilíferos.
\section{Psammódio}
\begin{itemize}
\item {Grp. gram.:m.}
\end{itemize}
\begin{itemize}
\item {Proveniência:(Do gr. \textunderscore psammodes\textunderscore )}
\end{itemize}
Gênero de insectos coleópteros pentâmeros.
\section{Psamódio}
\begin{itemize}
\item {Grp. gram.:m.}
\end{itemize}
\begin{itemize}
\item {Proveniência:(Do gr. \textunderscore psammodes\textunderscore )}
\end{itemize}
Gênero de insectos coleópteros pentâmeros.
\section{Psammódromo}
\begin{itemize}
\item {Grp. gram.:m.}
\end{itemize}
\begin{itemize}
\item {Proveniência:(Do gr. \textunderscore psammos\textunderscore  + \textunderscore dromos\textunderscore )}
\end{itemize}
Gênero de reptís sáurios.
\section{Psammotherna}
\begin{itemize}
\item {Grp. gram.:f.}
\end{itemize}
Gênero de insectos hymenópteros.
\section{Psammótropha}
\begin{itemize}
\item {Grp. gram.:f.}
\end{itemize}
Gênero de plantas portuláceas.
\section{Psamódromo}
\begin{itemize}
\item {Grp. gram.:m.}
\end{itemize}
\begin{itemize}
\item {Proveniência:(Do gr. \textunderscore psammos\textunderscore  + \textunderscore dromos\textunderscore )}
\end{itemize}
Gênero de reptís sáurios.
\section{Psamoterna}
\begin{itemize}
\item {Grp. gram.:f.}
\end{itemize}
Gênero de insectos himenópteros.
\section{Psamótrofa}
\begin{itemize}
\item {Grp. gram.:f.}
\end{itemize}
Gênero de plantas portuláceas.
\section{Psaro}
\begin{itemize}
\item {Grp. gram.:m.}
\end{itemize}
Gênero de insectos dípteros.
\section{Psathyra}
\begin{itemize}
\item {Grp. gram.:f.}
\end{itemize}
\begin{itemize}
\item {Proveniência:(Do gr. \textunderscore psathuros\textunderscore )}
\end{itemize}
Gênero de plantas rubiáceas.
\section{Psathyrito}
\begin{itemize}
\item {Grp. gram.:m.}
\end{itemize}
\begin{itemize}
\item {Utilização:Miner.}
\end{itemize}
\begin{itemize}
\item {Proveniência:(Do gr. \textunderscore prathuros\textunderscore )}
\end{itemize}
Qualquer resina fóssil.
\section{Psatira}
\begin{itemize}
\item {Grp. gram.:f.}
\end{itemize}
\begin{itemize}
\item {Proveniência:(Do gr. \textunderscore psathuros\textunderscore )}
\end{itemize}
Gênero de plantas rubiáceas.
\section{Psatirito}
\begin{itemize}
\item {Grp. gram.:m.}
\end{itemize}
\begin{itemize}
\item {Utilização:Miner.}
\end{itemize}
\begin{itemize}
\item {Proveniência:(Do gr. \textunderscore prathuros\textunderscore )}
\end{itemize}
Qualquer resina fóssil.
\section{Pseca}
\begin{itemize}
\item {Grp. gram.:f.}
\end{itemize}
\begin{itemize}
\item {Proveniência:(Lat. \textunderscore psecas\textunderscore )}
\end{itemize}
Escrava romana, que penteava e aromatizava as tranças da sua ama.
\section{Psécade}
\begin{itemize}
\item {Grp. gram.:f.}
\end{itemize}
\begin{itemize}
\item {Proveniência:(Lat. \textunderscore psecas\textunderscore )}
\end{itemize}
Escrava romana, que penteava e aromatizava as tranças da sua ama.
\section{Psectrócera}
\begin{itemize}
\item {Grp. gram.:f.}
\end{itemize}
\begin{itemize}
\item {Proveniência:(Do gr. \textunderscore psektra\textunderscore  + \textunderscore keras\textunderscore )}
\end{itemize}
Gênero de insectos coleópteros.
\section{Psefisma}
\begin{itemize}
\item {Grp. gram.:m.}
\end{itemize}
\begin{itemize}
\item {Proveniência:(Lat. \textunderscore psephisma\textunderscore )}
\end{itemize}
Decreto da assembleia do povo, entre os Gregos, correspondente ao plebiscito, entre os Romanos. Cf. Latino, \textunderscore Or. da Corôa\textunderscore , XI.
\section{Psefito}
\begin{itemize}
\item {Grp. gram.:m.}
\end{itemize}
\begin{itemize}
\item {Utilização:Miner.}
\end{itemize}
\begin{itemize}
\item {Proveniência:(Do gr. \textunderscore psephos\textunderscore )}
\end{itemize}
Rocha granulosa, vermelha ou esverdeada.
\section{Psefógrafo}
\begin{itemize}
\item {Grp. gram.:m.}
\end{itemize}
\begin{itemize}
\item {Proveniência:(Do gr. \textunderscore psephos\textunderscore  + \textunderscore grapheín\textunderscore )}
\end{itemize}
Aparelho ou máquina, para registo e contagem dos votos em assembleias eleitoraes. Cf. R. Galvão, \textunderscore Vocab.\textunderscore 
\section{Psélio}
\begin{itemize}
\item {Grp. gram.:m.}
\end{itemize}
\begin{itemize}
\item {Proveniência:(Do gr. \textunderscore pselion\textunderscore )}
\end{itemize}
Gênero de plantas menispermáceas.
\section{Pselismo}
\begin{itemize}
\item {Grp. gram.:m.}
\end{itemize}
\begin{itemize}
\item {Proveniência:(Gr. \textunderscore psellismos\textunderscore )}
\end{itemize}
Nome genérico dos defeitos da fala.
\section{Psellismo}
\begin{itemize}
\item {Grp. gram.:m.}
\end{itemize}
\begin{itemize}
\item {Proveniência:(Gr. \textunderscore psellismos\textunderscore )}
\end{itemize}
Nome genérico dos defeitos da fala.
\section{Psene}
\begin{itemize}
\item {Grp. gram.:m.}
\end{itemize}
Gênero de peixes acanthopterýgios.
\section{Psephisma}
\begin{itemize}
\item {Grp. gram.:m.}
\end{itemize}
\begin{itemize}
\item {Proveniência:(Lat. \textunderscore psephisma\textunderscore )}
\end{itemize}
Decreto da assembleia do povo, entre os Gregos, correspondente ao plebiscito, entre os Romanos. Cf. Latino, \textunderscore Or. da Corôa\textunderscore , XI.
\section{Psephito}
\begin{itemize}
\item {Grp. gram.:m.}
\end{itemize}
\begin{itemize}
\item {Utilização:Miner.}
\end{itemize}
\begin{itemize}
\item {Proveniência:(Do gr. \textunderscore psephos\textunderscore )}
\end{itemize}
Rocha granulosa, vermelha ou esverdeada.
\section{Psephógrapho}
\begin{itemize}
\item {Grp. gram.:m.}
\end{itemize}
\begin{itemize}
\item {Proveniência:(Do gr. \textunderscore psephos\textunderscore  + \textunderscore grapheín\textunderscore )}
\end{itemize}
Apparelho ou máquina, para registo e contagem dos votos em assembleias eleitoraes. Cf. R. Galvão, \textunderscore Vocab.\textunderscore 
\section{Pseudantho}
\begin{itemize}
\item {Grp. gram.:m.}
\end{itemize}
\begin{itemize}
\item {Proveniência:(Do gr. \textunderscore pseudos\textunderscore  + \textunderscore anthos\textunderscore )}
\end{itemize}
Gênero de plantas euphorbiáceas.
\section{Pseudárthria}
\begin{itemize}
\item {Grp. gram.:f.}
\end{itemize}
\begin{itemize}
\item {Proveniência:(Do gr. \textunderscore pseudos\textunderscore  + \textunderscore arthron\textunderscore )}
\end{itemize}
Gênero de plantas leguminosas.
\section{Pseudarthrose}
\begin{itemize}
\item {Grp. gram.:f.}
\end{itemize}
\begin{itemize}
\item {Utilização:Med.}
\end{itemize}
\begin{itemize}
\item {Proveniência:(Do gr. \textunderscore pseudos\textunderscore  + \textunderscore arthron\textunderscore )}
\end{itemize}
Articulação accidental, entre as duas extremidades de uma fractura.
\section{Pseudesthesia}
\begin{itemize}
\item {Grp. gram.:f.}
\end{itemize}
\begin{itemize}
\item {Utilização:Med.}
\end{itemize}
\begin{itemize}
\item {Proveniência:(Do gr. \textunderscore pseudos\textunderscore  + \textunderscore aisthesis\textunderscore )}
\end{itemize}
Sensação falsa.
\section{Pseudanto}
\begin{itemize}
\item {Grp. gram.:m.}
\end{itemize}
\begin{itemize}
\item {Proveniência:(Do gr. \textunderscore pseudos\textunderscore  + \textunderscore anthos\textunderscore )}
\end{itemize}
Gênero de plantas euforbiáceas.
\section{Pseudártria}
\begin{itemize}
\item {Grp. gram.:f.}
\end{itemize}
\begin{itemize}
\item {Proveniência:(Do gr. \textunderscore pseudos\textunderscore  + \textunderscore arthron\textunderscore )}
\end{itemize}
Gênero de plantas leguminosas.
\section{Pseudartrose}
\begin{itemize}
\item {Grp. gram.:f.}
\end{itemize}
\begin{itemize}
\item {Utilização:Med.}
\end{itemize}
\begin{itemize}
\item {Proveniência:(Do gr. \textunderscore pseudos\textunderscore  + \textunderscore arthron\textunderscore )}
\end{itemize}
Articulação accidental, entre as duas extremidades de uma fractura.
\section{Pseudestesia}
\begin{itemize}
\item {Grp. gram.:f.}
\end{itemize}
\begin{itemize}
\item {Utilização:Med.}
\end{itemize}
\begin{itemize}
\item {Proveniência:(Do gr. \textunderscore pseudos\textunderscore  + \textunderscore aisthesis\textunderscore )}
\end{itemize}
Sensação falsa.
\section{Pseudiamante}
\begin{itemize}
\item {Grp. gram.:m.}
\end{itemize}
\begin{itemize}
\item {Proveniência:(De \textunderscore pseudo...\textunderscore  + \textunderscore diamante\textunderscore )}
\end{itemize}
Pedra ordinária, que imita uma pedra preciosa.
\section{Pseudiosma}
\begin{itemize}
\item {Grp. gram.:m.}
\end{itemize}
\begin{itemize}
\item {Proveniência:(Do gr. \textunderscore pseudos\textunderscore  + \textunderscore diosma\textunderscore )}
\end{itemize}
Gênero de árvores da Cochinchina.
\section{Pseudo...}
\begin{itemize}
\item {Grp. gram.:pref.}
\end{itemize}
\begin{itemize}
\item {Proveniência:(Do gr. \textunderscore pseudos\textunderscore )}
\end{itemize}
(Designativo de \textunderscore falso\textunderscore )
\section{Psicagogia}
\begin{itemize}
\item {Grp. gram.:f.}
\end{itemize}
Ceremónia mágica, com que se evocavam as almas dos mortos.
(Cp. \textunderscore psicagogo\textunderscore )
\section{Psicagogo}
\begin{itemize}
\item {Grp. gram.:m.}
\end{itemize}
\begin{itemize}
\item {Proveniência:(Do gr. \textunderscore psukhe\textunderscore  + \textunderscore agein\textunderscore )}
\end{itemize}
Aquele que praticava a psicagogia.
\section{Psiletos}
\begin{itemize}
\item {Grp. gram.:m. pl.}
\end{itemize}
Milícia adjunta á phalange macedónica, para o serviço de segurança em marcha, guarda do campo e para combate em ordem dispera.--Cada phalange dispunha de 2:000 psiletos.
\section{Psilomelana}
\begin{itemize}
\item {Grp. gram.:f.}
\end{itemize}
Um dos óxydos do manganés.
\section{Psilónia}
\begin{itemize}
\item {Grp. gram.:f.}
\end{itemize}
Gênero de cogumelos.
\section{Psique}
\begin{itemize}
\item {Grp. gram.:f.}
\end{itemize}
\begin{itemize}
\item {Utilização:Neol.}
\end{itemize}
\begin{itemize}
\item {Proveniência:(Gr. \textunderscore Psukhe\textunderscore , n. p. myth.)}
\end{itemize}
A alma.
\section{Psiqueuterpia}
\begin{itemize}
\item {Grp. gram.:f.}
\end{itemize}
\begin{itemize}
\item {Utilização:Espir.}
\end{itemize}
\begin{itemize}
\item {Proveniência:(Do gr. \textunderscore psukhe\textunderscore  + \textunderscore Euterpes\textunderscore , n. p.)}
\end{itemize}
Qualidade do médium, que toca instrumentos sob a influência dos espíritos.
\section{Psiquialgia}
\begin{itemize}
\item {Grp. gram.:f.}
\end{itemize}
\begin{itemize}
\item {Utilização:Neol.}
\end{itemize}
\begin{itemize}
\item {Proveniência:(Do gr. \textunderscore psukhe\textunderscore  + \textunderscore algas\textunderscore )}
\end{itemize}
Dôr do espírito; amargura ingênita. Cf. Sousa Martins, \textunderscore Nosographia\textunderscore .
\section{Psiquiatra}
\begin{itemize}
\item {Grp. gram.:m.}
\end{itemize}
\begin{itemize}
\item {Proveniência:(Do gr. \textunderscore psukhe\textunderscore  + \textunderscore intros\textunderscore )}
\end{itemize}
Aquele que trata de psiquiatría.
\section{Psiquiatria}
\begin{itemize}
\item {Grp. gram.:f.}
\end{itemize}
\begin{itemize}
\item {Proveniência:(De \textunderscore psiquiatra\textunderscore )}
\end{itemize}
Doutrina das doenças mentaes e do tratamento delas.
\section{Psítaca}
\begin{itemize}
\item {Grp. gram.:f.}
\end{itemize}
O mesmo que \textunderscore psítaco\textunderscore .
\section{Psitácara}
\begin{itemize}
\item {Grp. gram.:f.}
\end{itemize}
\begin{itemize}
\item {Proveniência:(De \textunderscore psítaca\textunderscore )}
\end{itemize}
Nome científico da arara.
\section{Psitáceas}
\begin{itemize}
\item {Grp. gram.:f. pl.}
\end{itemize}
\begin{itemize}
\item {Utilização:Zool.}
\end{itemize}
\begin{itemize}
\item {Proveniência:(De \textunderscore psítaco\textunderscore )}
\end{itemize}
Ordem de aves, que tem por tipo o papagaio.
\section{Psitacídios}
\begin{itemize}
\item {Grp. gram.:m. pl.}
\end{itemize}
\begin{itemize}
\item {Proveniência:(Do gr. \textunderscore psittakos\textunderscore  + \textunderscore eidos\textunderscore )}
\end{itemize}
Família de aves, que compreende os arás e os papagaios.
\section{Psitacismo}
\begin{itemize}
\item {Grp. gram.:m.}
\end{itemize}
\begin{itemize}
\item {Proveniência:(De \textunderscore psitaco\textunderscore )}
\end{itemize}
Arte de alinhar frases ôcas; verborreia.
\section{Psítaco}
\begin{itemize}
\item {Grp. gram.:m.}
\end{itemize}
\begin{itemize}
\item {Proveniência:(Lat. \textunderscore psittacus\textunderscore )}
\end{itemize}
Designação científica do papagaio.
\section{Psitacose}
\begin{itemize}
\item {Grp. gram.:f.}
\end{itemize}
\begin{itemize}
\item {Utilização:Med.}
\end{itemize}
\begin{itemize}
\item {Proveniência:(De \textunderscore psitaco\textunderscore )}
\end{itemize}
Doença, em que predominam alterações pulmonares e que é ocasionada por periquitos e papagaios.
\section{Psitácula}
\begin{itemize}
\item {Grp. gram.:f.}
\end{itemize}
\begin{itemize}
\item {Proveniência:(Do lat. \textunderscore psittacus\textunderscore )}
\end{itemize}
Designação científica do periquito.
\section{Psitáculo}
\begin{itemize}
\item {Grp. gram.:m.}
\end{itemize}
\begin{itemize}
\item {Proveniência:(Do lat. \textunderscore psittacus\textunderscore )}
\end{itemize}
Designação científica do periquito.
\section{Psíttaca}
\begin{itemize}
\item {Grp. gram.:f.}
\end{itemize}
O mesmo que \textunderscore psíttaco\textunderscore .
\section{Psittácara}
\begin{itemize}
\item {Grp. gram.:f.}
\end{itemize}
\begin{itemize}
\item {Proveniência:(De \textunderscore psíttaca\textunderscore )}
\end{itemize}
Nome scientífico da arara.
\section{Psittáceas}
\begin{itemize}
\item {Grp. gram.:f. pl.}
\end{itemize}
\begin{itemize}
\item {Utilização:Zool.}
\end{itemize}
\begin{itemize}
\item {Proveniência:(De \textunderscore psíttaco\textunderscore )}
\end{itemize}
Ordem de aves, que tem por tipo o papagaio.
\section{Psittacídios}
\begin{itemize}
\item {Grp. gram.:m. pl.}
\end{itemize}
\begin{itemize}
\item {Proveniência:(Do gr. \textunderscore psittakos\textunderscore  + \textunderscore eidos\textunderscore )}
\end{itemize}
Família de aves, que comprehende os arás e os papagaios.
\section{Psittacismo}
\begin{itemize}
\item {Grp. gram.:m.}
\end{itemize}
\begin{itemize}
\item {Proveniência:(De \textunderscore psittaco\textunderscore )}
\end{itemize}
Arte de alinhar phrases ôcas; verborreia.
\section{Psíttaco}
\begin{itemize}
\item {Grp. gram.:m.}
\end{itemize}
\begin{itemize}
\item {Proveniência:(Lat. \textunderscore psittacus\textunderscore )}
\end{itemize}
Designação scientífica do papagaio.
\section{Psittacose}
\begin{itemize}
\item {Grp. gram.:f.}
\end{itemize}
\begin{itemize}
\item {Utilização:Med.}
\end{itemize}
\begin{itemize}
\item {Proveniência:(De \textunderscore psittaco\textunderscore )}
\end{itemize}
Doença, em que predominam alterações pulmonares e que é occasionada por periquitos e papagaios.
\section{Psittácula}
\begin{itemize}
\item {Grp. gram.:f.}
\end{itemize}
\begin{itemize}
\item {Proveniência:(Do lat. \textunderscore psittacus\textunderscore )}
\end{itemize}
Designação scientífica do periquito.
\section{Psittáculo}
\begin{itemize}
\item {Grp. gram.:m.}
\end{itemize}
\begin{itemize}
\item {Proveniência:(Do lat. \textunderscore psittacus\textunderscore )}
\end{itemize}
Designação scientífica do periquito.
\section{Psôa}
\begin{itemize}
\item {Grp. gram.:f.}
\end{itemize}
Gênero de insectos coleópteros tetrâmeros.
\section{Psôas}
\begin{itemize}
\item {Grp. gram.:m.}
\end{itemize}
\begin{itemize}
\item {Utilização:Anat.}
\end{itemize}
\begin{itemize}
\item {Proveniência:(Do gr. \textunderscore psoa\textunderscore )}
\end{itemize}
Nome dos dois músculos abdominaes, que se estendem pela parte anterior das vértebras lombares.
\section{Psoíte}
\begin{itemize}
\item {Grp. gram.:f.}
\end{itemize}
Inflammação do psôas.
\section{Psora}
\begin{itemize}
\item {Grp. gram.:f.}
\end{itemize}
\begin{itemize}
\item {Proveniência:(Do gr. \textunderscore psora\textunderscore )}
\end{itemize}
Designação genérica de várias doenças de pelle, caracterizadas por vesículas ou pustulas.
\section{Psoríaco}
\begin{itemize}
\item {Grp. gram.:adj.}
\end{itemize}
\begin{itemize}
\item {Proveniência:(De \textunderscore psora\textunderscore )}
\end{itemize}
Relativo á psoríase; que padece psoríase.
\section{Psoríase}
\begin{itemize}
\item {Grp. gram.:f.}
\end{itemize}
\begin{itemize}
\item {Proveniência:(Gr. \textunderscore psoríases\textunderscore )}
\end{itemize}
Inflammação chrónica da pelle, limitada a uma parte do corpo mais ou menos extensa e apresentando ao princípio tumescências que se transformam em manchas escamosas.
\section{Psoroftalmia}
\begin{itemize}
\item {Grp. gram.:f.}
\end{itemize}
\begin{itemize}
\item {Utilização:Med.}
\end{itemize}
\begin{itemize}
\item {Proveniência:(Do gr. \textunderscore psora\textunderscore  + \textunderscore ophthalmos\textunderscore )}
\end{itemize}
Variedade de blefarite.
\section{Psorophthalmia}
\begin{itemize}
\item {Grp. gram.:f.}
\end{itemize}
\begin{itemize}
\item {Utilização:Med.}
\end{itemize}
\begin{itemize}
\item {Proveniência:(Do gr. \textunderscore psora\textunderscore  + \textunderscore ophthalmos\textunderscore )}
\end{itemize}
Variedade de blepharite.
\section{Psoropta}
\begin{itemize}
\item {Grp. gram.:m.}
\end{itemize}
Espécie de ácaro.
\section{Psorospermo}
\begin{itemize}
\item {Grp. gram.:m.}
\end{itemize}
\begin{itemize}
\item {Proveniência:(Do gr. \textunderscore psora\textunderscore  + \textunderscore sperma\textunderscore )}
\end{itemize}
Gênero de plantas hypericáceas.
\section{Psxiu}
\begin{itemize}
\item {Grp. gram.:m.}
\end{itemize}
Som sibilante, que se emprega para chamar ou mandar calar.
\section{Psychagogia}
\begin{itemize}
\item {fónica:ca}
\end{itemize}
\begin{itemize}
\item {Grp. gram.:f.}
\end{itemize}
Ceremónia mágica, com que se evocavam as almas dos mortos.
(Cp. \textunderscore psychagogo\textunderscore )
\section{Psychagogo}
\begin{itemize}
\item {fónica:ca}
\end{itemize}
\begin{itemize}
\item {Grp. gram.:m.}
\end{itemize}
\begin{itemize}
\item {Proveniência:(Do gr. \textunderscore psukhe\textunderscore  + \textunderscore agein\textunderscore )}
\end{itemize}
Aquelle que praticava a psychagogia.
\section{Psyche}
\begin{itemize}
\item {fónica:que}
\end{itemize}
\begin{itemize}
\item {Grp. gram.:f.}
\end{itemize}
\begin{itemize}
\item {Utilização:Neol.}
\end{itemize}
\begin{itemize}
\item {Proveniência:(Gr. \textunderscore Psukhe\textunderscore , n. p. myth.)}
\end{itemize}
A alma.
\section{Psycheuterpia}
\begin{itemize}
\item {fónica:queu}
\end{itemize}
\begin{itemize}
\item {Grp. gram.:f.}
\end{itemize}
\begin{itemize}
\item {Utilização:Espir.}
\end{itemize}
\begin{itemize}
\item {Proveniência:(Do gr. \textunderscore psukhe\textunderscore  + \textunderscore Euterpes\textunderscore , n. p.)}
\end{itemize}
Qualidade do médium, que toca instrumentos sob a influência dos espíritos.
\section{Psychialgia}
\begin{itemize}
\item {fónica:qui}
\end{itemize}
\begin{itemize}
\item {Grp. gram.:f.}
\end{itemize}
\begin{itemize}
\item {Utilização:Neol.}
\end{itemize}
\begin{itemize}
\item {Proveniência:(Do gr. \textunderscore psukhe\textunderscore  + \textunderscore algas\textunderscore )}
\end{itemize}
Dôr do espírito; amargura ingênita. Cf. Sousa Martins, \textunderscore Nosographia\textunderscore .
\section{Psychiatra}
\begin{itemize}
\item {fónica:qui}
\end{itemize}
\begin{itemize}
\item {Grp. gram.:m.}
\end{itemize}
\begin{itemize}
\item {Proveniência:(Do gr. \textunderscore psukhe\textunderscore  + \textunderscore intros\textunderscore )}
\end{itemize}
Aquelle que trata de psychiatría.
\section{Psychiatria}
\begin{itemize}
\item {fónica:qui}
\end{itemize}
\begin{itemize}
\item {Grp. gram.:f.}
\end{itemize}
\begin{itemize}
\item {Proveniência:(De \textunderscore psychiatra\textunderscore )}
\end{itemize}
Doutrina das doenças mentaes e do tratamento dellas.
\section{Psicofísica}
\begin{itemize}
\item {Grp. gram.:f.}
\end{itemize}
\begin{itemize}
\item {Proveniência:(De \textunderscore psicofísico\textunderscore )}
\end{itemize}
Escola fisiológica, baseada na influência recíproca do espírito e da matéria.
\section{Psicofísico}
\begin{itemize}
\item {Grp. gram.:adj.}
\end{itemize}
\begin{itemize}
\item {Proveniência:(Do gr. \textunderscore psukhe\textunderscore  + \textunderscore phusis\textunderscore )}
\end{itemize}
Relativo ao espírito e á matéria.
\section{Psicofonia}
\begin{itemize}
\item {Grp. gram.:f.}
\end{itemize}
\begin{itemize}
\item {Utilização:Espir.}
\end{itemize}
\begin{itemize}
\item {Proveniência:(Do gr. \textunderscore psukhe\textunderscore  + \textunderscore phone\textunderscore )}
\end{itemize}
Comunicação dos espíritos pela voz do médium.
\section{Psicofónico}
\begin{itemize}
\item {Grp. gram.:adj.}
\end{itemize}
Relativo á psicofonia.
\section{Psicopatia}
\begin{itemize}
\item {Grp. gram.:f.}
\end{itemize}
\begin{itemize}
\item {Utilização:Neol.}
\end{itemize}
Designação genérica de doenças mentaes.
(Cp. \textunderscore psicopata\textunderscore )
\section{Psicopático}
\begin{itemize}
\item {Grp. gram.:adj.}
\end{itemize}
Relativo á psicopatia.
\section{Psicopatologia}
\begin{itemize}
\item {Grp. gram.:f.}
\end{itemize}
\begin{itemize}
\item {Utilização:Med.}
\end{itemize}
\begin{itemize}
\item {Proveniência:(Do gr. \textunderscore psukhe\textunderscore  + \textunderscore pathos\textunderscore  + \textunderscore logos\textunderscore )}
\end{itemize}
Tratado das doenças mentaes.
\section{Psicopatológico}
\begin{itemize}
\item {Grp. gram.:adj.}
\end{itemize}
Relativo á psicopatologia.
\section{Psicoplasma}
\begin{itemize}
\item {Grp. gram.:m.}
\end{itemize}
\begin{itemize}
\item {Proveniência:(Do gr. \textunderscore psukhe\textunderscore  + \textunderscore plasma\textunderscore )}
\end{itemize}
Base material da actividade psíquica.
\section{Psicose}
\begin{itemize}
\item {Grp. gram.:f.}
\end{itemize}
\begin{itemize}
\item {Proveniência:(Do gr. \textunderscore psukê\textunderscore )}
\end{itemize}
Designação genérica das doenças meantes; psicopatia.
\section{Psicoterapia}
\begin{itemize}
\item {Grp. gram.:f.}
\end{itemize}
\begin{itemize}
\item {Utilização:Med.}
\end{itemize}
\begin{itemize}
\item {Proveniência:(Do gr. \textunderscore psukhe\textunderscore  + \textunderscore therapeia\textunderscore )}
\end{itemize}
Tratamento das doenças moraes.
\section{Psicótria}
\begin{itemize}
\item {Grp. gram.:f.}
\end{itemize}
Gênero de plantas rubiáceas.
\section{Psicrometria}
\begin{itemize}
\item {Grp. gram.:f.}
\end{itemize}
Aplicação do psicrómetro.
\section{Psicrométrico}
\begin{itemize}
\item {Grp. gram.:adj.}
\end{itemize}
Relativo á psicrometria.
\section{Psicrómetro}
\begin{itemize}
\item {Grp. gram.:m.}
\end{itemize}
\begin{itemize}
\item {Proveniência:(Do gr. \textunderscore psukhros\textunderscore  + \textunderscore metron\textunderscore )}
\end{itemize}
Instrumento, para avaliar a quantidade de vapor, contido na atmosfera.
\section{Psila}
\begin{itemize}
\item {Grp. gram.:f.}
\end{itemize}
\begin{itemize}
\item {Proveniência:(Do gr. \textunderscore psulle\textunderscore , pulga)}
\end{itemize}
Gênero de insectos hemípteros.
\section{Psilo}
\begin{itemize}
\item {Grp. gram.:m.}
\end{itemize}
Domesticador de serpentes.
(Cp. lat. \textunderscore Psylli\textunderscore , povo líbico)
\section{Psilocarpo}
\begin{itemize}
\item {Grp. gram.:m.}
\end{itemize}
\begin{itemize}
\item {Proveniência:(Do gr. \textunderscore psulle\textunderscore  + \textunderscore karpos\textunderscore )}
\end{itemize}
Gênero de plantas rubiáceas.
\section{Psilócera}
\begin{itemize}
\item {Grp. gram.:f.}
\end{itemize}
\begin{itemize}
\item {Proveniência:(Do gr. \textunderscore psulle\textunderscore  + \textunderscore keras\textunderscore )}
\end{itemize}
Gênero de insectos coleópteros.
\section{Psilógino}
\begin{itemize}
\item {Grp. gram.:m.}
\end{itemize}
\begin{itemize}
\item {Proveniência:(Do gr. \textunderscore psulle\textunderscore  + \textunderscore gune\textunderscore )}
\end{itemize}
Gênero de plantas bignoniáceas.
\section{Psilópilo}
\begin{itemize}
\item {Grp. gram.:m.}
\end{itemize}
Gênero de musgos.
\section{Psilóstoma}
\begin{itemize}
\item {Grp. gram.:f.}
\end{itemize}
Gênero de plantas rubiáceas.
\section{Psilota}
\begin{itemize}
\item {Grp. gram.:f.}
\end{itemize}
Gênero de insectos dípteros.
\section{Psiloto}
\begin{itemize}
\item {Grp. gram.:m.}
\end{itemize}
Gênero de plantas licopodiáceas.
\section{Psítia}
\begin{itemize}
\item {Grp. gram.:adj. f.}
\end{itemize}
\begin{itemize}
\item {Proveniência:(Lat. \textunderscore psythius\textunderscore )}
\end{itemize}
Dizia-se, entre os antigos, de uma variedade de uva grega. Cf. Castilho, \textunderscore Geórgicas\textunderscore , 79.
\section{Psítio}
\begin{itemize}
\item {Grp. gram.:m.}
\end{itemize}
\begin{itemize}
\item {Proveniência:(Lat. \textunderscore psythium\textunderscore )}
\end{itemize}
Vinho precioso, que os antigos fabricavam de uvas psítias, depois de sêcas.
\section{Psychopathia}
\begin{itemize}
\item {fónica:co}
\end{itemize}
\begin{itemize}
\item {Grp. gram.:f.}
\end{itemize}
\begin{itemize}
\item {Utilização:Neol.}
\end{itemize}
Designação genérica de doenças mentaes.
(Cp. \textunderscore psychopatha\textunderscore )
\section{Psychopáthico}
\begin{itemize}
\item {fónica:co}
\end{itemize}
\begin{itemize}
\item {Grp. gram.:adj.}
\end{itemize}
Relativo á psychopathia.
\section{Psychopathologia}
\begin{itemize}
\item {fónica:co}
\end{itemize}
\begin{itemize}
\item {Grp. gram.:f.}
\end{itemize}
\begin{itemize}
\item {Utilização:Med.}
\end{itemize}
\begin{itemize}
\item {Proveniência:(Do gr. \textunderscore psukhe\textunderscore  + \textunderscore pathos\textunderscore  + \textunderscore logos\textunderscore )}
\end{itemize}
Tratado das doenças mentaes.
\section{Psychopathológico}
\begin{itemize}
\item {fónica:co}
\end{itemize}
\begin{itemize}
\item {Grp. gram.:adj.}
\end{itemize}
Relativo á psychopathologia.
\section{Psychophonia}
\begin{itemize}
\item {fónica:co}
\end{itemize}
\begin{itemize}
\item {Grp. gram.:f.}
\end{itemize}
\begin{itemize}
\item {Utilização:Espir.}
\end{itemize}
\begin{itemize}
\item {Proveniência:(Do gr. \textunderscore psukhe\textunderscore  + \textunderscore phone\textunderscore )}
\end{itemize}
Communicação dos espíritos pela voz do médium.
\section{Psychophónico}
\begin{itemize}
\item {fónica:co}
\end{itemize}
\begin{itemize}
\item {Grp. gram.:adj.}
\end{itemize}
Relativo á psychophonia.
\section{Psychophýsica}
\begin{itemize}
\item {fónica:co}
\end{itemize}
\begin{itemize}
\item {Grp. gram.:f.}
\end{itemize}
\begin{itemize}
\item {Proveniência:(De \textunderscore psychophýsico\textunderscore )}
\end{itemize}
Escola physiológica, baseada na influência recíproca do espírito e da matéria.
\section{Psychophýsico}
\begin{itemize}
\item {fónica:co}
\end{itemize}
\begin{itemize}
\item {Grp. gram.:adj.}
\end{itemize}
\begin{itemize}
\item {Proveniência:(Do gr. \textunderscore psukhe\textunderscore  + \textunderscore phusis\textunderscore )}
\end{itemize}
Relativo ao espírito e á matéria.
\section{Psychoplasma}
\begin{itemize}
\item {fónica:co}
\end{itemize}
\begin{itemize}
\item {Grp. gram.:m.}
\end{itemize}
\begin{itemize}
\item {Proveniência:(Do gr. \textunderscore psukhe\textunderscore  + \textunderscore plasma\textunderscore )}
\end{itemize}
Base material da actividade psýchica.
\section{Psychose}
\begin{itemize}
\item {fónica:co}
\end{itemize}
\begin{itemize}
\item {Grp. gram.:f.}
\end{itemize}
\begin{itemize}
\item {Proveniência:(Do gr. \textunderscore psukê\textunderscore )}
\end{itemize}
Designação genérica das doenças meantes; psychopathia.
\section{Psychotherapia}
\begin{itemize}
\item {fónica:co}
\end{itemize}
\begin{itemize}
\item {Grp. gram.:f.}
\end{itemize}
\begin{itemize}
\item {Utilização:Med.}
\end{itemize}
\begin{itemize}
\item {Proveniência:(Do gr. \textunderscore psukhe\textunderscore  + \textunderscore therapeia\textunderscore )}
\end{itemize}
Tratamento das doenças moraes.
\section{Psychótria}
\begin{itemize}
\item {fónica:có}
\end{itemize}
\begin{itemize}
\item {Grp. gram.:f.}
\end{itemize}
Gênero de plantas rubiáceas.
\section{Psychrometria}
\begin{itemize}
\item {Grp. gram.:f.}
\end{itemize}
Applicação do psychrómetro.
\section{Psychrométrico}
\begin{itemize}
\item {Grp. gram.:adj.}
\end{itemize}
Relativo á psychrometria.
\section{Psychrómetro}
\begin{itemize}
\item {Grp. gram.:m.}
\end{itemize}
\begin{itemize}
\item {Proveniência:(Do gr. \textunderscore psukhros\textunderscore  + \textunderscore metron\textunderscore )}
\end{itemize}
Instrumento, para avaliar a quantidade de vapor, contido na atmosphera.
\section{Psylla}
\begin{itemize}
\item {Grp. gram.:f.}
\end{itemize}
\begin{itemize}
\item {Proveniência:(Do gr. \textunderscore psulle\textunderscore , pulga)}
\end{itemize}
Gênero de insectos hemípteros.
\section{Psyllo}
\begin{itemize}
\item {Grp. gram.:m.}
\end{itemize}
Domesticador de serpentes.
(Cp. lat. \textunderscore Psylli\textunderscore , povo lýbico)
\section{Psyllocarpo}
\begin{itemize}
\item {Grp. gram.:m.}
\end{itemize}
\begin{itemize}
\item {Proveniência:(Do gr. \textunderscore psulle\textunderscore  + \textunderscore karpos\textunderscore )}
\end{itemize}
Gênero de plantas rubiáceas.
\section{Psyllócera}
\begin{itemize}
\item {Grp. gram.:f.}
\end{itemize}
\begin{itemize}
\item {Proveniência:(Do gr. \textunderscore psulle\textunderscore  + \textunderscore keras\textunderscore )}
\end{itemize}
Gênero de insectos coleópteros.
\section{Psyllógyno}
\begin{itemize}
\item {Grp. gram.:m.}
\end{itemize}
\begin{itemize}
\item {Proveniência:(Do gr. \textunderscore psulle\textunderscore  + \textunderscore gune\textunderscore )}
\end{itemize}
Gênero de plantas bignoniáceas.
\section{Psyllónia}
\begin{itemize}
\item {Grp. gram.:f.}
\end{itemize}
Gênero de cogumelos.
\section{Psyllópilo}
\begin{itemize}
\item {Grp. gram.:m.}
\end{itemize}
Gênero de musgos.
\section{Psyllóstoma}
\begin{itemize}
\item {Grp. gram.:f.}
\end{itemize}
Gênero de plantas rubiáceas.
\section{Psyllota}
\begin{itemize}
\item {Grp. gram.:f.}
\end{itemize}
Gênero de insectos dípteros.
\section{Psylloto}
\begin{itemize}
\item {Grp. gram.:m.}
\end{itemize}
Gênero de plantas lycopodiáceas.
\section{Psýthia}
\begin{itemize}
\item {Grp. gram.:adj. f.}
\end{itemize}
\begin{itemize}
\item {Proveniência:(Lat. \textunderscore psythius\textunderscore )}
\end{itemize}
Dizia-se, entre os antigos, de uma variedade de uva grega. Cf. Castilho, \textunderscore Geórgicas\textunderscore , 79.
\section{Psýthio}
\begin{itemize}
\item {Grp. gram.:m.}
\end{itemize}
\begin{itemize}
\item {Proveniência:(Lat. \textunderscore psythium\textunderscore )}
\end{itemize}
Vinho precioso, que os antigos fabricavam de uvas psýthias, depois de sêcas.
\section{Ptármica}
\begin{itemize}
\item {Grp. gram.:f.}
\end{itemize}
\begin{itemize}
\item {Proveniência:(De \textunderscore ptármico\textunderscore )}
\end{itemize}
Gênero de plantas, da fam. das compostas.
\section{Ptármico}
\begin{itemize}
\item {Grp. gram.:adj.}
\end{itemize}
\begin{itemize}
\item {Proveniência:(Gr. \textunderscore ptarmikos\textunderscore )}
\end{itemize}
Que provoca o espirro e a secreção do muco nasal.
\section{Pteleáceas}
\begin{itemize}
\item {Grp. gram.:f. pl.}
\end{itemize}
Tríbo de plantas terebintháceas.
\section{Pterýgeno}
\begin{itemize}
\item {Grp. gram.:adj.}
\end{itemize}
\begin{itemize}
\item {Proveniência:(Do gr. \textunderscore pterux\textunderscore  + \textunderscore genos\textunderscore )}
\end{itemize}
Que nasce sôbre os fêtos.
\section{Pterígeno}
\begin{itemize}
\item {Grp. gram.:adj.}
\end{itemize}
\begin{itemize}
\item {Proveniência:(Do gr. \textunderscore pterux\textunderscore  + \textunderscore genos\textunderscore )}
\end{itemize}
Que nasce sôbre os fêtos.
\section{Pterýgio}
\begin{itemize}
\item {Grp. gram.:m.}
\end{itemize}
\begin{itemize}
\item {Proveniência:(Do gr. \textunderscore pterux\textunderscore )}
\end{itemize}
Doença, caracterizada pela formação de uma prega na conjuntiva, invadindo a córnea sob a fórma de asa de môsca.
\section{Pterígio}
\begin{itemize}
\item {Grp. gram.:m.}
\end{itemize}
\begin{itemize}
\item {Proveniência:(Do gr. \textunderscore pterux\textunderscore )}
\end{itemize}
Doença, caracterizada pela formação de uma prega na conjuntiva, invadindo a córnea sob a fórma de asa de môsca.
\section{Pterigódio}
\begin{itemize}
\item {Grp. gram.:m.}
\end{itemize}
Gênero de orquídeas.
\section{Pterigoídeo}
\begin{itemize}
\item {Grp. gram.:adj.}
\end{itemize}
\begin{itemize}
\item {Utilização:Anat.}
\end{itemize}
\begin{itemize}
\item {Proveniência:(Do gr. \textunderscore pterux\textunderscore  + \textunderscore eidos\textunderscore )}
\end{itemize}
Que tem a fórma de uma asa.
Relativo á apófise que ladeia a linha mediana do ôsso esfenóide.
\section{Pterigota}
\begin{itemize}
\item {Grp. gram.:f.}
\end{itemize}
Gênero de plantas esterculiaceas.
\section{Pterigrafia}
\begin{itemize}
\item {Grp. gram.:f.}
\end{itemize}
\begin{itemize}
\item {Proveniência:(Do gr. \textunderscore pterux\textunderscore  + \textunderscore graphein\textunderscore )}
\end{itemize}
Descripção ou tratado dos cogumelos.
\section{Pterigráfico}
\begin{itemize}
\item {Grp. gram.:adj.}
\end{itemize}
Relativo á pterigrafia.
\section{Pterígrafo}
\begin{itemize}
\item {Grp. gram.:m.}
\end{itemize}
Aquele que se ocupa da pterigrafia.
\section{Pterygódio}
\begin{itemize}
\item {Grp. gram.:m.}
\end{itemize}
Gênero de orchídeas.
\section{Pterygoídeo}
\begin{itemize}
\item {Grp. gram.:adj.}
\end{itemize}
\begin{itemize}
\item {Utilização:Anat.}
\end{itemize}
\begin{itemize}
\item {Proveniência:(Do gr. \textunderscore pterux\textunderscore  + \textunderscore eidos\textunderscore )}
\end{itemize}
Que tem a fórma de uma asa.
Relativo á apóphyse que ladeia a linha mediana do ôsso esphenóide.
\section{Pterygota}
\begin{itemize}
\item {Grp. gram.:f.}
\end{itemize}
Gênero de plantas esterculiaceas.
\section{Pterygraphia}
\begin{itemize}
\item {Grp. gram.:f.}
\end{itemize}
\begin{itemize}
\item {Proveniência:(Do gr. \textunderscore pterux\textunderscore  + \textunderscore graphein\textunderscore )}
\end{itemize}
Descripção ou tratado dos cogumelos.
\section{Pterygráphico}
\begin{itemize}
\item {Grp. gram.:adj.}
\end{itemize}
Relativo á pterygraphia.
\section{Pterýgrapho}
\begin{itemize}
\item {Grp. gram.:m.}
\end{itemize}
Aquelle que se occupa da pterygraphia.
\section{Ptialagogo}
\begin{itemize}
\item {Grp. gram.:adj.}
\end{itemize}
\begin{itemize}
\item {Utilização:Med.}
\end{itemize}
\begin{itemize}
\item {Proveniência:(Do gr. \textunderscore ptualon\textunderscore  + \textunderscore agein\textunderscore )}
\end{itemize}
Que excita a secreção da saliva.
\section{Ptialina}
\begin{itemize}
\item {Grp. gram.:f.}
\end{itemize}
\begin{itemize}
\item {Proveniência:(Do gr. \textunderscore ptualon\textunderscore )}
\end{itemize}
Substância particular, achada na saliva.
\section{Ptialismo}
\begin{itemize}
\item {Grp. gram.:m.}
\end{itemize}
\begin{itemize}
\item {Proveniência:(Gr. \textunderscore ptualismos\textunderscore )}
\end{itemize}
Secreção abundante da saliva.
\section{Ptilino}
\begin{itemize}
\item {Grp. gram.:m.}
\end{itemize}
\begin{itemize}
\item {Proveniência:(Do gr. \textunderscore ptilon\textunderscore )}
\end{itemize}
Gênero de insectos coleópteros, que furam a madeira.
\section{Ptilólitho}
\begin{itemize}
\item {Grp. gram.:m.}
\end{itemize}
\begin{itemize}
\item {Utilização:Miner.}
\end{itemize}
\begin{itemize}
\item {Proveniência:(Do gr. \textunderscore ptiilon\textunderscore  + \textunderscore lithos\textunderscore )}
\end{itemize}
Silicato hydratado de alumínio, potássio, sódio e cálcio.
\section{Ptilólito}
\begin{itemize}
\item {Grp. gram.:m.}
\end{itemize}
\begin{itemize}
\item {Utilização:Miner.}
\end{itemize}
\begin{itemize}
\item {Proveniência:(Do gr. \textunderscore ptiilon\textunderscore  + \textunderscore lithos\textunderscore )}
\end{itemize}
Silicato hidratado de alumínio, potássio, sódio e cálcio.
\section{Ptilose}
\begin{itemize}
\item {Grp. gram.:f.}
\end{itemize}
\begin{itemize}
\item {Utilização:Med.}
\end{itemize}
\begin{itemize}
\item {Proveniência:(Gr. \textunderscore ptilosis\textunderscore )}
\end{itemize}
Quéda dos cílios por inflammação chrónica do bôrdo livre das palpebras.
\section{Ptolemaico}
\begin{itemize}
\item {Grp. gram.:adj.}
\end{itemize}
Relativo ao geógrapho e astrónomo \textunderscore Ptolemeu\textunderscore , ás suas obras ou doutrinas.
\section{Ptomaína}
\begin{itemize}
\item {Grp. gram.:f.}
\end{itemize}
\begin{itemize}
\item {Utilização:Med.}
\end{itemize}
\begin{itemize}
\item {Proveniência:(Do gr. \textunderscore ptoma\textunderscore )}
\end{itemize}
Putrefacção cadavérica.
A parte putrefacta de qualquer organismo.
Infecção, que resulta dessa putrefacção.
\section{Ptose}
\begin{itemize}
\item {Grp. gram.:f.}
\end{itemize}
\begin{itemize}
\item {Utilização:Med.}
\end{itemize}
\begin{itemize}
\item {Proveniência:(Gr. \textunderscore ptosis\textunderscore )}
\end{itemize}
Quéda da pálpebra.
Deslocação de uma víscera, pelo afroixamento dos seus meios de fixidez.
\section{Ptoseonomia}
\begin{itemize}
\item {Grp. gram.:f.}
\end{itemize}
\begin{itemize}
\item {Utilização:Gram.}
\end{itemize}
\begin{itemize}
\item {Proveniência:(Do gr. \textunderscore ptosis\textunderscore  + \textunderscore nomos\textunderscore )}
\end{itemize}
Parte da grammática, que trata da flexão das palavras.
\section{Ptyalagogo}
\begin{itemize}
\item {Grp. gram.:adj.}
\end{itemize}
\begin{itemize}
\item {Utilização:Med.}
\end{itemize}
\begin{itemize}
\item {Proveniência:(Do gr. \textunderscore ptualon\textunderscore  + \textunderscore agein\textunderscore )}
\end{itemize}
Que excita a secreção da saliva.
\section{Ptyalina}
\begin{itemize}
\item {Grp. gram.:f.}
\end{itemize}
\begin{itemize}
\item {Proveniência:(Do gr. \textunderscore ptualon\textunderscore )}
\end{itemize}
Substância particular, achada na saliva.
\section{Ptyalismo}
\begin{itemize}
\item {Grp. gram.:m.}
\end{itemize}
\begin{itemize}
\item {Proveniência:(Gr. \textunderscore ptualismos\textunderscore )}
\end{itemize}
Secreção abundante da saliva.
\section{Pua}
\begin{itemize}
\item {Grp. gram.:f.}
\end{itemize}
Haste, que termina em bico.
Ponta aguçada.
Haste da espora.
Bico de verruma.
Instrumento, para furar, que se move por meio de um arco.
Intervallo, entre os dentes do pente de tear.
\section{Puado}
\begin{itemize}
\item {Grp. gram.:m.}
\end{itemize}
\begin{itemize}
\item {Proveniência:(De \textunderscore pua\textunderscore )}
\end{itemize}
Instrumento de cardador, espécie de sedeiro, onde se recarda a lan. Cf. \textunderscore Inquér. Industr.\textunderscore , p. II, l. III, 67.
\section{Puava}
\begin{itemize}
\item {Grp. gram.:adj.}
\end{itemize}
\begin{itemize}
\item {Utilização:Bras. do S}
\end{itemize}
\begin{itemize}
\item {Utilização:Fig.}
\end{itemize}
O mesmo que \textunderscore aruá\textunderscore .
Colérico, raivoso.
\section{Puba}
\begin{itemize}
\item {Grp. gram.:f.}
\end{itemize}
\begin{itemize}
\item {Utilização:Bras}
\end{itemize}
\begin{itemize}
\item {Utilização:Bras. do N}
\end{itemize}
\begin{itemize}
\item {Grp. gram.:Adj.}
\end{itemize}
\begin{itemize}
\item {Utilização:Bras}
\end{itemize}
\begin{itemize}
\item {Proveniência:(T. tupi)}
\end{itemize}
Mandioca, coberta de lama, para amollecer.
Terrenos húmidos, cobertos de capim.
Molle.
\section{Pubar}
\begin{itemize}
\item {Grp. gram.:v. t.}
\end{itemize}
\begin{itemize}
\item {Utilização:Bras}
\end{itemize}
\begin{itemize}
\item {Grp. gram.:V. i.}
\end{itemize}
\begin{itemize}
\item {Utilização:Bras. do N}
\end{itemize}
\begin{itemize}
\item {Proveniência:(De \textunderscore puba\textunderscore )}
\end{itemize}
Pôr a curtir em lama ou na água (a mandioca).
Fermentar.
Apodrecer.
\section{Pubeiro}
\begin{itemize}
\item {Grp. gram.:adj.}
\end{itemize}
Diz-se do gado, que pasta nos terrenos chamados pubas.
\section{Pubente}
\begin{itemize}
\item {Grp. gram.:adj.}
\end{itemize}
\begin{itemize}
\item {Proveniência:(Lat. \textunderscore pubens\textunderscore )}
\end{itemize}
O mesmo que \textunderscore púbere\textunderscore . Cf. Filinto, VI, 217.
\section{Puberdade}
\begin{itemize}
\item {Grp. gram.:f.}
\end{itemize}
\begin{itemize}
\item {Proveniência:(Lat. \textunderscore pubertas\textunderscore )}
\end{itemize}
Idade, em que os indivíduos se tornam aptos para a procriação.
Estado ou qualidade púbere.
\section{Púbere}
\begin{itemize}
\item {Grp. gram.:adj.}
\end{itemize}
\begin{itemize}
\item {Proveniência:(Lat. \textunderscore puber\textunderscore )}
\end{itemize}
Que chegou á puberdade.
Que começa a ter barba ou os pêlos finos que annunciam a adolescência.
\section{Pucela}
\begin{itemize}
\item {Grp. gram.:f.}
\end{itemize}
\begin{itemize}
\item {Utilização:Gal}
\end{itemize}
\begin{itemize}
\item {Utilização:Ant.}
\end{itemize}
\begin{itemize}
\item {Proveniência:(Fr. \textunderscore puccelle\textunderscore )}
\end{itemize}
Virgem; rapariga.
\section{Pucella}
\begin{itemize}
\item {Grp. gram.:f.}
\end{itemize}
\begin{itemize}
\item {Utilização:Gal}
\end{itemize}
\begin{itemize}
\item {Utilização:Ant.}
\end{itemize}
\begin{itemize}
\item {Proveniência:(Fr. \textunderscore puccelle\textunderscore )}
\end{itemize}
Virgem; rapariga.
\section{Puchanci}
\begin{itemize}
\item {Grp. gram.:m.}
\end{itemize}
Designação antiga de uma espécie de autoridades ou governadores na China. Cf. \textunderscore Peregrinação\textunderscore , CIII.
\section{Púchara}
\begin{itemize}
\item {Grp. gram.:f.}
\end{itemize}
\begin{itemize}
\item {Utilização:Prov.}
\end{itemize}
\begin{itemize}
\item {Proveniência:(Do cast. \textunderscore puchera\textunderscore )}
\end{itemize}
O mesmo que \textunderscore púcara\textunderscore .
\section{Puchissucão}
\begin{itemize}
\item {Grp. gram.:m.}
\end{itemize}
Peixe da China, verde e preto, com tres ordens de espinhos no lombo. Cf. \textunderscore Peregrinação\textunderscore , LXXII.
\section{Pucho}
\begin{itemize}
\item {Grp. gram.:m.}
\end{itemize}
\begin{itemize}
\item {Utilização:Ant.}
\end{itemize}
Planta aromática, que entrava na renda da especiaria de Gôa.
\section{Puchom}
\begin{itemize}
\item {Grp. gram.:m.  e  adj.}
\end{itemize}
Diz-se de uma variedade de chá preto, notável pelo seu aroma.
\section{Puchuri}
\begin{itemize}
\item {Grp. gram.:m.}
\end{itemize}
Nome de duas plantas lauráceas do Brasil.
\section{Puchurim}
\begin{itemize}
\item {Grp. gram.:m.}
\end{itemize}
Nome de duas plantas lauráceas do Brasil.
\section{Puctó}
\begin{itemize}
\item {Grp. gram.:m.}
\end{itemize}
Dialecto principal do Afeganistão, falado pelas classes superiores.
\section{Pucuman}
\begin{itemize}
\item {Grp. gram.:m.}
\end{itemize}
O mesmo que \textunderscore picuman\textunderscore .
\section{Pudendo}
\begin{itemize}
\item {Grp. gram.:adj.}
\end{itemize}
\begin{itemize}
\item {Proveniência:(Lat. \textunderscore pudendus\textunderscore )}
\end{itemize}
Envergonhado; vergonhoso; pudico.
Que o pudor deve recatar: \textunderscore as partes pudendas da mulhér\textunderscore .
\section{Pudente}
\begin{itemize}
\item {Grp. gram.:adj.}
\end{itemize}
\begin{itemize}
\item {Proveniência:(Lat. \textunderscore pudens\textunderscore )}
\end{itemize}
Que tem pudor; pudico; casto:«\textunderscore virgens pudentíssimas...\textunderscore »Filinto XXII, 16. Cf. Camillo, \textunderscore Noites de Insómn.\textunderscore , VIII, 78.
\section{Pudiano}
\begin{itemize}
\item {Grp. gram.:m.}
\end{itemize}
Peixe marítimo do Brasil, (\textunderscore labrus radians\textunderscore ).
\section{Pudibundo}
\begin{itemize}
\item {Grp. gram.:adj.}
\end{itemize}
\begin{itemize}
\item {Utilização:Fig.}
\end{itemize}
\begin{itemize}
\item {Proveniência:(Lat. \textunderscore pubibundus\textunderscore )}
\end{itemize}
Que tem pudor; que se envergonha.
Rubicundo, còrado.
\section{Pudicamente}
\begin{itemize}
\item {Grp. gram.:adv.}
\end{itemize}
De modo pudico.
\section{Pudicícia}
\begin{itemize}
\item {Grp. gram.:f.}
\end{itemize}
\begin{itemize}
\item {Proveniência:(Lat. \textunderscore pudicitia\textunderscore )}
\end{itemize}
Qualidade do que é pudico.
Castidade.
Honra feminina.
Pudor.
Acto ou palavras que denotam pudor.
\section{Pudico}
\begin{itemize}
\item {Grp. gram.:adj.}
\end{itemize}
\begin{itemize}
\item {Proveniência:(Lat. \textunderscore pudicus\textunderscore )}
\end{itemize}
(\textunderscore púdico\textunderscore  é pron. errónea)
Que tem pudor.
Envergonhado.
Honesto, casto.
\section{Pudim}
\begin{itemize}
\item {Grp. gram.:m.}
\end{itemize}
\begin{itemize}
\item {Utilização:Geol.}
\end{itemize}
\begin{itemize}
\item {Proveniência:(Do ingl. \textunderscore pudding\textunderscore )}
\end{itemize}
Designação de várias iguarias, constituídas por massas, e cozidas no forno, tendo por base uma ou mais substâncias que variam muito.
Conglomerado, cujos elementos são calhaus ou outros fragmentos rolados.
\section{Pudivão}
\begin{itemize}
\item {Grp. gram.:m.}
\end{itemize}
Vestuário gentílico da Índia portuguesa. Cf. Th. Ribeiro, \textunderscore Jornadas\textunderscore , II.
\section{Pudor}
\begin{itemize}
\item {Grp. gram.:m.}
\end{itemize}
\begin{itemize}
\item {Proveniência:(Lat. \textunderscore pudor\textunderscore )}
\end{itemize}
Sentimento de timidez ou vergonha, produzido pelo que póde ferir a decência, a honestidade ou a modéstia.
Vergonha, pejo.
Seriedade; pundonor.
\section{Pudoroso}
\begin{itemize}
\item {Grp. gram.:adj.}
\end{itemize}
\begin{itemize}
\item {Utilização:Neol.}
\end{itemize}
Em que há pudor; que tem pudor.
Relativo ao pudor.
\section{Pudrica-raja}
\begin{itemize}
\item {Grp. gram.:m.}
\end{itemize}
Uma das principaes autoridades de Malaca, antes da conquista portuguesa. Cf. \textunderscore Comment. de Aff. de Albuq.\textunderscore 
\section{Pudvém}
\begin{itemize}
\item {Grp. gram.:m.}
\end{itemize}
\begin{itemize}
\item {Utilização:T. da Índia Port}
\end{itemize}
\begin{itemize}
\item {Proveniência:(Do conc. \textunderscore pudvê\textunderscore )}
\end{itemize}
Pano, que os homens enrolam á volta dos quadris, como saiote.
\section{Puelche}
\begin{itemize}
\item {Grp. gram.:m.}
\end{itemize}
Lingua dos Pampas.
\section{Puera}
\begin{itemize}
\item {Grp. gram.:f.}
\end{itemize}
\begin{itemize}
\item {Utilização:Bras}
\end{itemize}
Lama, que se enxugou.
Paúl sêco pelo sol.
(Guar. \textunderscore puêra\textunderscore )
\section{Puerícia}
\begin{itemize}
\item {Grp. gram.:f.}
\end{itemize}
\begin{itemize}
\item {Proveniência:(Lat. \textunderscore puerítia\textunderscore )}
\end{itemize}
Idade pueríl.
Conjunto dos indivíduos, que estão entre a infância e a adolescência: \textunderscore a educação da puerícia\textunderscore .
\section{Puericultura}
\begin{itemize}
\item {Grp. gram.:f.}
\end{itemize}
\begin{itemize}
\item {Utilização:Neol.}
\end{itemize}
\begin{itemize}
\item {Proveniência:(Do lat. \textunderscore puer\textunderscore , \textunderscore pueri\textunderscore  + \textunderscore cultura\textunderscore )}
\end{itemize}
O mesmo que \textunderscore estirpicultura\textunderscore .
\section{Pueril}
\begin{itemize}
\item {Grp. gram.:adj.}
\end{itemize}
\begin{itemize}
\item {Proveniência:(Lat. \textunderscore purilis\textunderscore )}
\end{itemize}
Relativo ás crianças, ou aos indivíduos que passaram o período da infância e ainda não entraram na adolescência.
Relativo á idade das crianças.
Próprio de crianças ou meninos.
Ingênuo; fútil: \textunderscore divertimento pueril\textunderscore .
\section{Puerilidade}
\begin{itemize}
\item {Grp. gram.:f.}
\end{itemize}
\begin{itemize}
\item {Proveniência:(Lat. \textunderscore puerilitas\textunderscore )}
\end{itemize}
Acto pueril.
Acto ou dito próprio de crianças; futilidade.
\section{Pulário}
\begin{itemize}
\item {Grp. gram.:m.}
\end{itemize}
\begin{itemize}
\item {Proveniência:(Lat. \textunderscore pullarius\textunderscore )}
\end{itemize}
Aquele que, entre os antigos Romanos, cuidava dos galos sagrados.
\section{Pulchérrimo}
\begin{itemize}
\item {fónica:qué}
\end{itemize}
\begin{itemize}
\item {Grp. gram.:adj.}
\end{itemize}
\begin{itemize}
\item {Proveniência:(Lat. \textunderscore pulcherrimos\textunderscore )}
\end{itemize}
Muito formoso, muito pulchro.
\section{Pulchrícomo}
\begin{itemize}
\item {Grp. gram.:adj.}
\end{itemize}
\begin{itemize}
\item {Proveniência:(De \textunderscore pulchro\textunderscore  + \textunderscore coma\textunderscore )}
\end{itemize}
Que tem formosos cabêllos.
\section{Pulchritude}
\begin{itemize}
\item {Grp. gram.:f.}
\end{itemize}
\begin{itemize}
\item {Utilização:Poét.}
\end{itemize}
\begin{itemize}
\item {Proveniência:(Lat. \textunderscore pulchritudo\textunderscore )}
\end{itemize}
Qualidade do que é pulchro.
\section{Pulchro}
\begin{itemize}
\item {Grp. gram.:adj.}
\end{itemize}
\begin{itemize}
\item {Utilização:Poét.}
\end{itemize}
\begin{itemize}
\item {Proveniência:(Lat. \textunderscore pulcher\textunderscore )}
\end{itemize}
Gentil, formoso.
\section{Pulcrícomo}
\begin{itemize}
\item {Grp. gram.:adj.}
\end{itemize}
\begin{itemize}
\item {Proveniência:(De \textunderscore pulchro\textunderscore  + \textunderscore coma\textunderscore )}
\end{itemize}
Que tem formosos cabêlos.
\section{Pulcritude}
\begin{itemize}
\item {Grp. gram.:f.}
\end{itemize}
\begin{itemize}
\item {Utilização:Poét.}
\end{itemize}
\begin{itemize}
\item {Proveniência:(Lat. \textunderscore pulchritudo\textunderscore )}
\end{itemize}
Qualidade do que é pulcro.
\section{Pulcro}
\begin{itemize}
\item {Grp. gram.:adj.}
\end{itemize}
\begin{itemize}
\item {Utilização:Poét.}
\end{itemize}
\begin{itemize}
\item {Proveniência:(Lat. \textunderscore pulcher\textunderscore )}
\end{itemize}
Gentil, formoso.
\section{Pulear}
\begin{itemize}
\item {Grp. gram.:v. i.}
\end{itemize}
\begin{itemize}
\item {Utilização:Prov.}
\end{itemize}
\begin{itemize}
\item {Utilização:minh.}
\end{itemize}
\begin{itemize}
\item {Grp. gram.:v. t.}
\end{itemize}
O mesmo que \textunderscore pular\textunderscore , pinchar.
Fazer dar saltos; fazer ír pelo ar, repetidas vezes.
\section{Púlex}
\begin{itemize}
\item {Grp. gram.:m.}
\end{itemize}
Insecto áptero, que adhere aos pés do homem, podendo occasionar-lhe a morte, (\textunderscore pulex penetrans\textunderscore ).
\section{Pulga}
\begin{itemize}
\item {Grp. gram.:f.}
\end{itemize}
\begin{itemize}
\item {Utilização:Pesc.}
\end{itemize}
\begin{itemize}
\item {Proveniência:(Do lat. hyp. \textunderscore pulica\textunderscore , de \textunderscore pulex\textunderscore )}
\end{itemize}
Insecto díptero, que se alimenta do sangue do homem e do de alguns animaes.
Animalculo, que se vê saltando sôbre a areia e que, dentro da água, devora as iscas dos anzóis.
\section{Pulgaminho}
\begin{itemize}
\item {Grp. gram.:m.}
\end{itemize}
\begin{itemize}
\item {Utilização:Ant.}
\end{itemize}
O mesmo que \textunderscore pergaminho\textunderscore .
\section{Pulgão}
\begin{itemize}
\item {Grp. gram.:m.}
\end{itemize}
\begin{itemize}
\item {Proveniência:(De \textunderscore pulga\textunderscore )}
\end{itemize}
Gênero de insectos parasitos, que vivem nos vegetaes.
\section{Pulgo}
\begin{itemize}
\item {Grp. gram.:m.}
\end{itemize}
\begin{itemize}
\item {Utilização:P. us.}
\end{itemize}
Macho da pulga. Cf. Castilho, \textunderscore Fausto\textunderscore , 146.
(Cp. \textunderscore pulga\textunderscore )
\section{Pulgoso}
\begin{itemize}
\item {Grp. gram.:adj.}
\end{itemize}
O mesmo que \textunderscore pulguento\textunderscore .
\section{Pulguedo}
\begin{itemize}
\item {fónica:guê}
\end{itemize}
\begin{itemize}
\item {Grp. gram.:m.}
\end{itemize}
Grande porção de pulgas.
Sítio, onde há muitas pulgas.
\section{Pulguento}
\begin{itemize}
\item {Grp. gram.:adj.}
\end{itemize}
Que tem muitas pulgas.
\section{Pulha}
\begin{itemize}
\item {Grp. gram.:f.}
\end{itemize}
\begin{itemize}
\item {Grp. gram.:M.}
\end{itemize}
\begin{itemize}
\item {Utilização:Pop.}
\end{itemize}
\begin{itemize}
\item {Grp. gram.:Adj.}
\end{itemize}
\begin{itemize}
\item {Utilização:Pop.}
\end{itemize}
Acto de gracejar, levando alguém a fazer pergunta, cuja resposta reverte em escárneo de quem perguntou.
Embaçadela.
Locução pouco decorosa.
Peta.
Homem sem dignidade.
Biltre; trapalhão.
Desprezível; indecente; desmazelado.
(Cast. \textunderscore pulla\textunderscore )
\section{Pulha}
\begin{itemize}
\item {Grp. gram.:f.}
\end{itemize}
\begin{itemize}
\item {Utilização:Prov.}
\end{itemize}
\begin{itemize}
\item {Utilização:minh.}
\end{itemize}
O mesmo que \textunderscore pulheiro\textunderscore .
\section{Pulhador}
\begin{itemize}
\item {Grp. gram.:m.}
\end{itemize}
\begin{itemize}
\item {Utilização:Des.}
\end{itemize}
Aquelle que diz pulhas. Cf. G. Vicente, I, 260.
\section{Pulhamente}
\begin{itemize}
\item {Grp. gram.:adv.}
\end{itemize}
\begin{itemize}
\item {Proveniência:(De \textunderscore pulha\textunderscore )}
\end{itemize}
Com pulhice.
\section{Pulhastra}
\begin{itemize}
\item {Grp. gram.:m.}
\end{itemize}
O mesmo ou melhor que \textunderscore pulhastro\textunderscore .
\section{Pulhastro}
\begin{itemize}
\item {Grp. gram.:m.}
\end{itemize}
\begin{itemize}
\item {Proveniência:(De \textunderscore pulha\textunderscore )}
\end{itemize}
Indivíduo muito reles, que não tem dignidade nenhuma.
\section{Pulheiro}
\begin{itemize}
\item {Grp. gram.:m.}
\end{itemize}
\begin{itemize}
\item {Utilização:Prov.}
\end{itemize}
\begin{itemize}
\item {Utilização:minh.}
\end{itemize}
O mesmo que \textunderscore freixo\textunderscore .
\section{Pulhice}
\begin{itemize}
\item {Grp. gram.:f.}
\end{itemize}
Acto ou dito de pulha; vida miserável, pelintrice.
\section{Pulhismo}
\begin{itemize}
\item {Grp. gram.:m.}
\end{itemize}
O mesmo que \textunderscore pulhice\textunderscore .
\section{Pulicária}
\begin{itemize}
\item {Grp. gram.:f.}
\end{itemize}
\begin{itemize}
\item {Proveniência:(Do lat. \textunderscore pulex\textunderscore )}
\end{itemize}
Gênero de plantas, da fam. das compostas.
\section{Pulim}
\begin{itemize}
\item {Grp. gram.:m.}
\end{itemize}
\begin{itemize}
\item {Utilização:Pop.}
\end{itemize}
O mesmo que [[pulinho|pulo]], dem. de \textunderscore pulo\textunderscore .
\section{Pulir}
\textunderscore v. t.\textunderscore  (e der.)
O mesmo que \textunderscore polir\textunderscore .
\section{Pullário}
\begin{itemize}
\item {Grp. gram.:m.}
\end{itemize}
\begin{itemize}
\item {Proveniência:(Lat. \textunderscore pullarius\textunderscore )}
\end{itemize}
Aquelle que, entre os antigos Romanos, cuidava dos gallos sagrados.
\section{Pullulação}
\begin{itemize}
\item {Grp. gram.:f.}
\end{itemize}
Acto de pullular.
\section{Pullulante}
\begin{itemize}
\item {Grp. gram.:adj.}
\end{itemize}
\begin{itemize}
\item {Proveniência:(Lat. \textunderscore pullulans\textunderscore )}
\end{itemize}
Que pullula.
\section{Pullular}
\begin{itemize}
\item {Grp. gram.:v. t.}
\end{itemize}
\begin{itemize}
\item {Proveniência:(Lat. \textunderscore pullulare\textunderscore )}
\end{itemize}
Lançar rebento.
Germinar rapidamente.
Brotar.
Desenvolver-se.
Agitar-se.
Ferver.
Têr abundância.
\section{Pulmão}
\begin{itemize}
\item {Grp. gram.:m.}
\end{itemize}
\begin{itemize}
\item {Utilização:Anat.}
\end{itemize}
\begin{itemize}
\item {Utilização:Fig.}
\end{itemize}
\begin{itemize}
\item {Utilização:Prov.}
\end{itemize}
\begin{itemize}
\item {Utilização:minh.}
\end{itemize}
\begin{itemize}
\item {Proveniência:(Lat. \textunderscore pulmo\textunderscore )}
\end{itemize}
Órgão, contido no peito, e com o qual se effectua a respiração; bofe.
Boa voz, voz forte.
Tumor ou excrescência em qualquer parte do corpo.
\section{Pulmoeira}
\begin{itemize}
\item {Grp. gram.:f.}
\end{itemize}
\begin{itemize}
\item {Proveniência:(De \textunderscore pulmão\textunderscore )}
\end{itemize}
Doença nos pulmões dos solípedes.
\section{Pulmonar}
\begin{itemize}
\item {Grp. gram.:adj.}
\end{itemize}
\begin{itemize}
\item {Proveniência:(Lat. \textunderscore pulmonarius\textunderscore )}
\end{itemize}
Relativo ao pulmão: \textunderscore tisica pulmonar\textunderscore .
Que faz parte dos pulmões.
Que tem pulmões.
\section{Pulquérrimo}
\begin{itemize}
\item {Grp. gram.:adj.}
\end{itemize}
\begin{itemize}
\item {Proveniência:(Lat. \textunderscore pulcherrimos\textunderscore )}
\end{itemize}
Muito formoso, muito pulcro.
\section{Pululação}
\begin{itemize}
\item {Grp. gram.:f.}
\end{itemize}
Acto de pulular.
\section{Pululante}
\begin{itemize}
\item {Grp. gram.:adj.}
\end{itemize}
\begin{itemize}
\item {Proveniência:(Lat. \textunderscore pullulans\textunderscore )}
\end{itemize}
Que pulula.
\section{Pulular}
\begin{itemize}
\item {Grp. gram.:v. t.}
\end{itemize}
\begin{itemize}
\item {Proveniência:(Lat. \textunderscore pullulare\textunderscore )}
\end{itemize}
Lançar rebento.
Germinar rapidamente.
Brotar.
Desenvolver-se.
Agitar-se.
Ferver.
Têr abundância.
\section{Poraqué}
\begin{itemize}
\item {Grp. gram.:m.}
\end{itemize}
\begin{itemize}
\item {Utilização:Bras. do N}
\end{itemize}
Enguia eléctrica do Pará e do Amazonas, (\textunderscore gymnotus electricus\textunderscore , Lin.).
\section{Pulverulento}
\begin{itemize}
\item {Grp. gram.:adj.}
\end{itemize}
\begin{itemize}
\item {Proveniência:(Lat. \textunderscore pulverulentus\textunderscore )}
\end{itemize}
Cheio ou coberto de pó.
Diz-se de certas plantas, cuja epiderme parece coberta de pó.
\section{Pulveruloso}
\begin{itemize}
\item {Grp. gram.:adj.}
\end{itemize}
O mesmo que \textunderscore pulverulento\textunderscore .
\section{Pulvinar}
\begin{itemize}
\item {Grp. gram.:m.}
\end{itemize}
\begin{itemize}
\item {Proveniência:(Lat. \textunderscore pulvinar\textunderscore )}
\end{itemize}
Almofada; travesseiro; coxim.
O palanque imperial nos circos romanos.
\section{Pum!}
\begin{itemize}
\item {Grp. gram.:interj.}
\end{itemize}
O mesmo que \textunderscore tum!\textunderscore .
\section{Puma}
\begin{itemize}
\item {Grp. gram.:m.}
\end{itemize}
Leão americano (\textunderscore feliz concolor\textunderscore ), menos corpulento que os do velho continente e sem crina nem borla na cauda.
\section{Pumacaás}
\begin{itemize}
\item {Grp. gram.:m. pl.}
\end{itemize}
Indígenas do norte do Brasil.
\section{Pumba!}
\begin{itemize}
\item {Grp. gram.:interj.}
\end{itemize}
\begin{itemize}
\item {Proveniência:(T. onom.)}
\end{itemize}
O mesmo que \textunderscore bumba!\textunderscore . Cf. Camillo, \textunderscore Brasileira\textunderscore , 123.
\section{Pumbaúba}
\begin{itemize}
\item {Grp. gram.:f.}
\end{itemize}
\begin{itemize}
\item {Utilização:Bras}
\end{itemize}
Árvore silvestre, cuja casca serve para curtir coiros.
\section{Pumpro}
\begin{itemize}
\item {Grp. gram.:m.}
\end{itemize}
Grande peixe marítimo do Brasil.
\section{Pumumo}
\begin{itemize}
\item {Grp. gram.:m.}
\end{itemize}
Passaro da África occidental, (\textunderscore bucorax cafer\textunderscore ).
\section{Puna}
Savana do Peru.
\section{Puna}
\begin{itemize}
\item {Grp. gram.:f.}
\end{itemize}
Árvore gulífera, de fibras têxteis, (\textunderscore calophyllum tomentosum\textunderscore , Wicht).
\section{Puna-macha}
\begin{itemize}
\item {Grp. gram.:f.}
\end{itemize}
Árvore de Cacheu.
\section{Punanegra}
\begin{itemize}
\item {Grp. gram.:f.}
\end{itemize}
\begin{itemize}
\item {Utilização:Ant.}
\end{itemize}
(?):«\textunderscore ...dois saios... de pano fino, hum de farpado e outro em punanegra.\textunderscore »(Li êste texto algures, talvez em Fernão Lopes)
\section{Puna-Vermelha}
\begin{itemize}
\item {Grp. gram.:f.}
\end{itemize}
Árvore da Guiné portuguesa, (\textunderscore sterculia foelida\textunderscore ).
\section{Punar}
\begin{itemize}
\item {Grp. gram.:v. i.}
\end{itemize}
\begin{itemize}
\item {Utilização:Ant.}
\end{itemize}
O mesmo que \textunderscore punir\textunderscore ^2.
(Cf. \textunderscore punhar\textunderscore )
\section{Punaré}
\begin{itemize}
\item {Grp. gram.:adj.}
\end{itemize}
\begin{itemize}
\item {Utilização:Bras}
\end{itemize}
\begin{itemize}
\item {Grp. gram.:M.}
\end{itemize}
Amarelado.
Grande rato, de côr avermelhada.
\section{Punção}
\begin{itemize}
\item {Grp. gram.:f.}
\end{itemize}
\begin{itemize}
\item {Grp. gram.:M.}
\end{itemize}
\begin{itemize}
\item {Proveniência:(Lat. \textunderscore punctio\textunderscore )}
\end{itemize}
Acto ou efeito de pungir ou puncionar.
Instrumento, terminado em ponta, para furar ou gravar.
Estilete cirúrgico.
Lâmina de aço, que tem letras em relêvo, para a fundição de caracteres tipográficos, medalhas, etc.
\section{Punçar}
\begin{itemize}
\item {Grp. gram.:v.}
\end{itemize}
\begin{itemize}
\item {Utilização:t. Náut.}
\end{itemize}
O mesmo que \textunderscore punccionar\textunderscore .
\section{Puncionagem}
\begin{itemize}
\item {Grp. gram.:f.}
\end{itemize}
\begin{itemize}
\item {Proveniência:(De \textunderscore puncionar\textunderscore )}
\end{itemize}
Trabalho com o punção.
\section{Puncção}
\begin{itemize}
\item {Grp. gram.:f.}
\end{itemize}
\begin{itemize}
\item {Grp. gram.:M.}
\end{itemize}
\begin{itemize}
\item {Proveniência:(Lat. \textunderscore punctio\textunderscore )}
\end{itemize}
Acto ou effeito de pungir ou punccionar.
Instrumento, terminado em ponta, para furar ou gravar.
Estilete cirúrgico.
Lâmina de aço, que tem letras em relêvo, para a fundição de caracteres typográphicos, medalhas, etc.
\section{Puncçar}
\begin{itemize}
\item {Grp. gram.:v.}
\end{itemize}
\begin{itemize}
\item {Utilização:t. Náut.}
\end{itemize}
O mesmo que \textunderscore punccionar\textunderscore .
\section{Puncceta}
\begin{itemize}
\item {Grp. gram.:f.}
\end{itemize}
\begin{itemize}
\item {Proveniência:(De \textunderscore puncção\textunderscore )}
\end{itemize}
Instrumento, com que se cortam lâminas de ferro.
\section{Punccionagem}
\begin{itemize}
\item {Grp. gram.:f.}
\end{itemize}
\begin{itemize}
\item {Proveniência:(De \textunderscore punccionar\textunderscore )}
\end{itemize}
Trabalho com o puncção.
\section{Punccionar}
\begin{itemize}
\item {Grp. gram.:v. t.}
\end{itemize}
\begin{itemize}
\item {Proveniência:(Do lat. \textunderscore punctio\textunderscore )}
\end{itemize}
Furar com o puncção.
\section{Puncçoar}
\begin{itemize}
\item {Grp. gram.:v.}
\end{itemize}
\begin{itemize}
\item {Utilização:t. Carp.}
\end{itemize}
Furar com o puncção.
\section{Punceta}
\begin{itemize}
\item {Grp. gram.:f.}
\end{itemize}
\begin{itemize}
\item {Proveniência:(De \textunderscore punção\textunderscore )}
\end{itemize}
Instrumento, com que se cortam lâminas de ferro.
\section{Punçoar}
\begin{itemize}
\item {Grp. gram.:v.}
\end{itemize}
\begin{itemize}
\item {Utilização:t. Carp.}
\end{itemize}
Furar com o punção.
\section{Punctiforme}
\begin{itemize}
\item {Grp. gram.:adj.}
\end{itemize}
\begin{itemize}
\item {Utilização:Bot.}
\end{itemize}
\begin{itemize}
\item {Proveniência:(Do lat. \textunderscore punctus\textunderscore  + \textunderscore forma\textunderscore )}
\end{itemize}
Que tem fórma ou apparência de ponto.
\section{Punctura}
\begin{itemize}
\item {Grp. gram.:f.}
\end{itemize}
\begin{itemize}
\item {Grp. gram.:Pl.}
\end{itemize}
\begin{itemize}
\item {Proveniência:(Lat. \textunderscore punctura\textunderscore )}
\end{itemize}
Ferida ou picada, feita com puncção ou com objecto análogo.
Chapas de ferro, que tem puas nas extremidades, e em que os impressores collocam as fôlhas.
\section{Pundé}
\begin{itemize}
\item {Grp. gram.:m.}
\end{itemize}
Árvore indiana, de fibras têxteis.
\section{Pundonor}
\begin{itemize}
\item {Grp. gram.:m.}
\end{itemize}
Sentimento de dignidade; decoro; brio; cavalheirismo; denôdo.
(Cast. \textunderscore pundonor\textunderscore )
\section{Pundonorosamente}
\begin{itemize}
\item {Grp. gram.:adv.}
\end{itemize}
De modo pundonoroso; com pundonor.
\section{Pundonoroso}
\begin{itemize}
\item {Grp. gram.:adj.}
\end{itemize}
Que tem ou em que há pundonor.
Brioso; denodado.
\section{Punga}
\begin{itemize}
\item {Grp. gram.:adj.}
\end{itemize}
\begin{itemize}
\item {Utilização:Bras. do S}
\end{itemize}
\begin{itemize}
\item {Grp. gram.:M.}
\end{itemize}
\begin{itemize}
\item {Grp. gram.:F.}
\end{itemize}
\begin{itemize}
\item {Utilização:Bras. do N}
\end{itemize}
Ruím.
Sem préstimo.
Que é o último em chegar á meta, (falando-se de um cavallo, nas corridas).
Homem tolo, ínepto.
O mesmo que \textunderscore umbigada\textunderscore .
\section{Pungarecos}
\begin{itemize}
\item {Grp. gram.:m. pl.}
\end{itemize}
\begin{itemize}
\item {Utilização:Bras}
\end{itemize}
\begin{itemize}
\item {Proveniência:(De \textunderscore punga\textunderscore )}
\end{itemize}
Drogas de charlatão.
\section{Pungente}
\begin{itemize}
\item {Grp. gram.:adj.}
\end{itemize}
\begin{itemize}
\item {Proveniência:(Lat. \textunderscore pungens\textunderscore )}
\end{itemize}
Que punge; doloroso; afflictivo.
\section{Pungibarba}
\begin{itemize}
\item {Grp. gram.:m.}
\end{itemize}
\begin{itemize}
\item {Utilização:P. us.}
\end{itemize}
\begin{itemize}
\item {Proveniência:(Do lat. \textunderscore pungere\textunderscore  + \textunderscore barba\textunderscore )}
\end{itemize}
Rapaz, a quem a barba já vem apontando.
\section{Pungidor}
\begin{itemize}
\item {Grp. gram.:adj.}
\end{itemize}
Que punge.
Que afflige; que atormenta.
\section{Pungimento}
\begin{itemize}
\item {Grp. gram.:m.}
\end{itemize}
Acto ou effeito de pungir.
\section{Punnanêgra}
\begin{itemize}
\item {Grp. gram.:f.}
\end{itemize}
\begin{itemize}
\item {Utilização:Ant.}
\end{itemize}
(?):«\textunderscore ...dois saios... de pano fino, hum de farpado e outro em punnanêgra.\textunderscore »(Li êste texto algures, talvez em Fernão Lopes)
\section{Pupália}
\begin{itemize}
\item {Grp. gram.:f.}
\end{itemize}
Gênero de plantas amarantáceas.
\section{Pupila}
\begin{itemize}
\item {Grp. gram.:f.}
\end{itemize}
\begin{itemize}
\item {Proveniência:(Lat. \textunderscore pupilla\textunderscore )}
\end{itemize}
Órfan, que está sob tutela; órfan.
Educanda.
Abertura, que a membrana íris apresenta no meio, e pela qual passam os raios luminosos, até chegarem ao cristalino.
\section{Pupilagem}
\begin{itemize}
\item {Grp. gram.:f.}
\end{itemize}
\begin{itemize}
\item {Proveniência:(De \textunderscore pupilo\textunderscore  ou \textunderscore pupila\textunderscore )}
\end{itemize}
Educação de pupilo ou pupila.
Tempo que dura essa educação.
\section{Pupilar}
\begin{itemize}
\item {Grp. gram.:adj.}
\end{itemize}
\begin{itemize}
\item {Proveniência:(Lat. \textunderscore pupillaris\textunderscore )}
\end{itemize}
Relativo a pupilo ou pupila.
\section{Pupilar}
\begin{itemize}
\item {Grp. gram.:v. i.}
\end{itemize}
\begin{itemize}
\item {Proveniência:(Lat. \textunderscore pupillare\textunderscore )}
\end{itemize}
Gritar, (falando-se do pavão). Cf. Castilho, \textunderscore Fastos\textunderscore , III, 324.
\section{Pupilla}
\begin{itemize}
\item {Grp. gram.:f.}
\end{itemize}
\begin{itemize}
\item {Proveniência:(Lat. \textunderscore pupilla\textunderscore )}
\end{itemize}
Órfan, que está sob tutela; órfan.
Educanda.
Abertura, que a membrana íris apresenta no meio, e pela qual passam os raios luminosos, até chegarem ao crystallino.
\section{Pupillagem}
\begin{itemize}
\item {Grp. gram.:f.}
\end{itemize}
\begin{itemize}
\item {Proveniência:(De \textunderscore pupillo\textunderscore  ou \textunderscore pupilla\textunderscore )}
\end{itemize}
Educação de pupillo ou pupilla.
Tempo que dura essa educação.
\section{Pupillar}
\begin{itemize}
\item {Grp. gram.:adj.}
\end{itemize}
\begin{itemize}
\item {Proveniência:(Lat. \textunderscore pupillaris\textunderscore )}
\end{itemize}
Relativo a pupillo ou pupilla.
\section{Pupillar}
\begin{itemize}
\item {Grp. gram.:v. i.}
\end{itemize}
\begin{itemize}
\item {Proveniência:(Lat. \textunderscore pupillare\textunderscore )}
\end{itemize}
Gritar, (falando-se do pavão). Cf. Castilho, \textunderscore Fastos\textunderscore , III, 324.
\section{Pupillo}
\begin{itemize}
\item {Grp. gram.:m.}
\end{itemize}
\begin{itemize}
\item {Utilização:Ext.}
\end{itemize}
\begin{itemize}
\item {Proveniência:(Lat. \textunderscore pupillus\textunderscore )}
\end{itemize}
Órfão, que está sob tutela; menor.
Indivíduo protegido.
Indivíduo, que outrem, não sendo seu pai, trata e educa paternalmente.
\section{Pupilo}
\begin{itemize}
\item {Grp. gram.:m.}
\end{itemize}
\begin{itemize}
\item {Utilização:Ext.}
\end{itemize}
\begin{itemize}
\item {Proveniência:(Lat. \textunderscore pupillus\textunderscore )}
\end{itemize}
Órfão, que está sob tutela; menor.
Indivíduo protegido.
Indivíduo, que outrem, não sendo seu pai, trata e educa paternalmente.
\section{Pupíparo}
\begin{itemize}
\item {Grp. gram.:adj.}
\end{itemize}
\begin{itemize}
\item {Grp. gram.:M. pl.}
\end{itemize}
\begin{itemize}
\item {Proveniência:(Do lat. \textunderscore pupa\textunderscore  + \textunderscore parere\textunderscore )}
\end{itemize}
Diz-se dos insectos, cujos filhos nascem em estado de nympha.
Família de insectos pupíparos.
\section{Puplècia}
\begin{itemize}
\item {Grp. gram.:f.}
\end{itemize}
\begin{itemize}
\item {Utilização:Pop.}
\end{itemize}
O mesmo que \textunderscore apoplexia\textunderscore . Cf. Castilho, \textunderscore Méd. á Fôrça\textunderscore , 45.
\section{Pupu}
\begin{itemize}
\item {Grp. gram.:m.}
\end{itemize}
\begin{itemize}
\item {Utilização:Prov.}
\end{itemize}
\begin{itemize}
\item {Utilização:Infant.}
\end{itemize}
Ave da África, (\textunderscore upupa africana\textunderscore ).
Popa, tufo ou nó de cabellos, no alto da cabeça.
\section{Pupu}
\begin{itemize}
\item {Grp. gram.:m.}
\end{itemize}
\begin{itemize}
\item {Utilização:T. da Bairrada}
\end{itemize}
\begin{itemize}
\item {Utilização:infant.}
\end{itemize}
\begin{itemize}
\item {Grp. gram.:Interj.}
\end{itemize}
Gallinha.
Voz, para chamar gallinhas.
\section{Pupunha}
\begin{itemize}
\item {Grp. gram.:f.}
\end{itemize}
\begin{itemize}
\item {Utilização:Bras}
\end{itemize}
Fruta da pupunheira.
O mesmo que \textunderscore pupunheira\textunderscore .
\section{Pupunheira}
\begin{itemize}
\item {Grp. gram.:f.}
\end{itemize}
\begin{itemize}
\item {Proveniência:(De \textunderscore pupunha\textunderscore )}
\end{itemize}
Palmeira espinhosa do norte do Brasil.
\section{Pupunheiro}
\begin{itemize}
\item {Grp. gram.:m.}
\end{itemize}
O mesmo que \textunderscore pupunheira\textunderscore .
\section{Puracé}
\begin{itemize}
\item {Grp. gram.:m.}
\end{itemize}
\begin{itemize}
\item {Utilização:Bras}
\end{itemize}
\begin{itemize}
\item {Proveniência:(T. tupi)}
\end{itemize}
Bailado, que os Índios fazem, depois da festa em que celebram a admissão dos mancebos ás filas dos guerreiros.
\section{Pural}
\begin{itemize}
\item {Grp. gram.:m.}
\end{itemize}
Espécie de carvão pulverizado que, impregnado de ácido phênico, menthol e ácido benzóico, exhala, ao arder, aromas desinfectantes.
\section{Puramente}
\begin{itemize}
\item {Grp. gram.:adv.}
\end{itemize}
De modo puro.
Com pureza; com innocência.
Com simplicidade; simplesmente.
Unicamente.
\section{Puraquê}
\begin{itemize}
\item {Grp. gram.:m.}
\end{itemize}
\begin{itemize}
\item {Utilização:Bras. do N}
\end{itemize}
Enguia eléctrica do Pará e do Amazonas, (\textunderscore gymnotus electricus\textunderscore , Lin.).
\section{Purava}
\begin{itemize}
\item {Grp. gram.:f.}
\end{itemize}
Túnica de algodão branco, ornada de rosas de oiro, e usada pelos Brâhmanes.
\section{Puré}
\begin{itemize}
\item {Grp. gram.:m.}
\end{itemize}
(V.pureia)
\section{Pureia}
\begin{itemize}
\item {Grp. gram.:f.}
\end{itemize}
Espécie de sôpa, feita de legumes, carnes ou outras substâncias raladas e formando caldo grosso.
Iguaria analoga, mas mais consistente, em fórma de pão ou pudim.--É mais usual o francesismo \textunderscore puré\textunderscore , o que não invalida a nossa \textunderscore pureia\textunderscore , que eu li, pelo menos, num antigo documento conventual.
(B. lat. \textunderscore pureya\textunderscore )
\section{Purenumás}
\begin{itemize}
\item {Grp. gram.:m. pl.}
\end{itemize}
Indígenas do norte do Brasil.
\section{Pureus}
\begin{itemize}
\item {Grp. gram.:m. pl.}
\end{itemize}
Índios selvagens das margens do Japurá, no Brasil.
\section{Pureza}
\begin{itemize}
\item {Grp. gram.:f.}
\end{itemize}
Qualidade do que é puro.
Transparência.
Nitidez.
Limpidez; diaphaneidade.
Innocência.
Virgindade.
Vernaculidade (na linguagem).
\section{Purga}
\begin{itemize}
\item {Grp. gram.:f.}
\end{itemize}
\begin{itemize}
\item {Utilização:Bras}
\end{itemize}
\begin{itemize}
\item {Proveniência:(De \textunderscore purgar\textunderscore )}
\end{itemize}
Preparação pharmacêutica, ou qualquer outra substância, que faz purgar.
Nome de várias plantas medicinaes.
\section{Purgação}
\begin{itemize}
\item {Grp. gram.:f.}
\end{itemize}
\begin{itemize}
\item {Proveniência:(Lat. \textunderscore purgatio\textunderscore )}
\end{itemize}
Acto ou effeito de purgar.
Acto ou effeito de purificar.
Corrimento; gonorrheia.
\section{Puro-negro}
\begin{itemize}
\item {Grp. gram.:m.}
\end{itemize}
Casta de uva. \textunderscore G. Rev. Agron.\textunderscore , I, 18.
\section{Púrpura}
\begin{itemize}
\item {Grp. gram.:f.}
\end{itemize}
\begin{itemize}
\item {Proveniência:(Lat. \textunderscore pupura\textunderscore )}
\end{itemize}
Substância còrante, vermelho-escura, que se extrahia dantes do múrice, substituido hoje pela cochinilha.
Côr vermelha.
Antigo tecido vermelho.
Vestuário dos Reis.
Dignidade real.
Dignidade de alguns magistrados, que trajavam púrpura.
Dignidade dos Cardeaes, que usam vestes vermelhas.
Espécie de molluscos.
Doença, caracterizada por pequenas manchas purpúreas, produzidas por hemorrhagia sub-epidérmica.
\section{Purpurado}
\begin{itemize}
\item {Grp. gram.:m.}
\end{itemize}
\begin{itemize}
\item {Proveniência:(De \textunderscore purpurar\textunderscore )}
\end{itemize}
Aquelle que foi elevado a Cardeal.
\section{Purpurar}
\begin{itemize}
\item {Grp. gram.:v. t.}
\end{itemize}
\begin{itemize}
\item {Utilização:Fig.}
\end{itemize}
Tingir de vermelho.
Dar côr de púrpura a.
Vestir de púrpura.
Elevar á dignidade de Cardeal.
\section{Purpurear}
\begin{itemize}
\item {Grp. gram.:v. t.}
\end{itemize}
\begin{itemize}
\item {Grp. gram.:V. i.  e  p.}
\end{itemize}
Dar côr de púrpura a.
Tornar vermelho.
Tomar a côr da púrpura; tornar-se vermelho: \textunderscore purpurearam-se-lhe as faces\textunderscore .
\section{Purpurejar}
\begin{itemize}
\item {Grp. gram.:v. t.}
\end{itemize}
\begin{itemize}
\item {Proveniência:(De \textunderscore púrpura\textunderscore )}
\end{itemize}
Tornar purpúreo, vermelho.
\section{Purpúreo}
\begin{itemize}
\item {Grp. gram.:adj.}
\end{itemize}
\begin{itemize}
\item {Proveniência:(Lat. \textunderscore purpureus\textunderscore )}
\end{itemize}
Que tem côr de púrpura; vermelho.
\section{Purpúrico}
\begin{itemize}
\item {Grp. gram.:adj.}
\end{itemize}
\begin{itemize}
\item {Utilização:Chím.}
\end{itemize}
\begin{itemize}
\item {Proveniência:(De \textunderscore púrpura\textunderscore )}
\end{itemize}
Diz-se de um ácido orgânico, que ainda se não pôde insular, e que só é conhecido no estado de combinação com os óxydos metállicos.
\section{Purpurífero}
\begin{itemize}
\item {Grp. gram.:adj.}
\end{itemize}
\begin{itemize}
\item {Proveniência:(Do lat. \textunderscore purpura\textunderscore  + \textunderscore ferre\textunderscore )}
\end{itemize}
Que tem púrpura; que produz púrpura.
\section{Purpurina}
\begin{itemize}
\item {Grp. gram.:f.}
\end{itemize}
\begin{itemize}
\item {Proveniência:(De \textunderscore purpurino\textunderscore )}
\end{itemize}
Substância insecticida, extrahida da raiz da ruiva.
Metaes reduzidos a pó e empregados em typographia para as impressões a oiro e prata.
\section{Purpurino}
\begin{itemize}
\item {Grp. gram.:adj.}
\end{itemize}
O mesmo que \textunderscore purpúreo\textunderscore .
\section{Purpurizar}
\begin{itemize}
\item {Grp. gram.:v. t.  e  p.}
\end{itemize}
O mesmo que \textunderscore purpurear\textunderscore .
\section{Púrria}
\begin{itemize}
\item {Grp. gram.:f.}
\end{itemize}
\begin{itemize}
\item {Utilização:Gír. de Lisboa.}
\end{itemize}
Bando de garotos.
Espécie de partido, formado entre os garotos de uma freguesia ou bairro contra os garotos de outro bairro ou freguesia.
(Cp. \textunderscore púrrio\textunderscore )
\section{Púrrio}
\begin{itemize}
\item {Grp. gram.:adj.}
\end{itemize}
\begin{itemize}
\item {Utilização:Gír.}
\end{itemize}
\begin{itemize}
\item {Utilização:ant.}
\end{itemize}
\begin{itemize}
\item {Utilização:Gír.}
\end{itemize}
Muito ordinario, reles.
Bêbedo.
\section{Purulência}
\begin{itemize}
\item {Grp. gram.:f.}
\end{itemize}
\begin{itemize}
\item {Proveniência:(Lat. \textunderscore purulentia\textunderscore )}
\end{itemize}
Qualidade do que é purulento.
\section{Purulento}
\begin{itemize}
\item {Grp. gram.:adj.}
\end{itemize}
\begin{itemize}
\item {Proveniência:(Lat. \textunderscore purulentus\textunderscore )}
\end{itemize}
Cheio de pus; que segrega pus; puriforme.
\section{Puruman}
\begin{itemize}
\item {Grp. gram.:m.}
\end{itemize}
Espécie de palmeira do Brasil.
Fruto dessa árvore.
\section{Puru-puru}
\begin{itemize}
\item {Grp. gram.:m.}
\end{itemize}
\begin{itemize}
\item {Utilização:T. do Amazonas}
\end{itemize}
Moléstia parasitária do cabello, o mesmo que \textunderscore trichomycose\textunderscore .
\section{Pururuca}
\begin{itemize}
\item {Grp. gram.:f.}
\end{itemize}
\begin{itemize}
\item {Utilização:Bras}
\end{itemize}
\begin{itemize}
\item {Utilização:Bras}
\end{itemize}
\begin{itemize}
\item {Grp. gram.:Adj.}
\end{itemize}
\begin{itemize}
\item {Utilização:Bras}
\end{itemize}
Côco tenro.
Árvore applicada em construcções.
Quebradiço, friável.
(Cp. \textunderscore pororoca\textunderscore ^1)
\section{Pus}
\begin{itemize}
\item {Grp. gram.:m.}
\end{itemize}
\begin{itemize}
\item {Proveniência:(Lat. \textunderscore pus\textunderscore )}
\end{itemize}
Líquido mórbido, caracterizado por glóbulos especiaes e ordinariamente resultante de uma inflammação.
\section{Pusilânime}
\begin{itemize}
\item {Grp. gram.:adj.}
\end{itemize}
\begin{itemize}
\item {Grp. gram.:M.}
\end{itemize}
\begin{itemize}
\item {Proveniência:(Lat. \textunderscore pusillanimis\textunderscore )}
\end{itemize}
Que tem ânimo fraco.
Tímido; cobarde.
Que mostra fraqueza de ânimo.
Aquele que tem fraqueza de ânimo ou cobardia.
\section{Pusilanimemente}
\begin{itemize}
\item {Grp. gram.:adv.}
\end{itemize}
De modo pusilânime; com pusilanimidade.
\section{Pusilânimo}
\begin{itemize}
\item {Grp. gram.:adj.}
\end{itemize}
\begin{itemize}
\item {Utilização:Des.}
\end{itemize}
O mesmo que \textunderscore pusilânime\textunderscore . Cf. \textunderscore Eufrosina\textunderscore .
\section{Pusillânime}
\begin{itemize}
\item {Grp. gram.:adj.}
\end{itemize}
\begin{itemize}
\item {Grp. gram.:M.}
\end{itemize}
\begin{itemize}
\item {Proveniência:(Lat. \textunderscore pusillanimis\textunderscore )}
\end{itemize}
Que tem ânimo fraco.
Tímido; cobarde.
Que mostra fraqueza de ânimo.
Aquelle que tem fraqueza de ânimo ou cobardia.
\section{Pusillanimemente}
\begin{itemize}
\item {Grp. gram.:adv.}
\end{itemize}
De modo pusillânime; com pusillanimidade.
\section{Pusillanimidade}
\begin{itemize}
\item {Grp. gram.:f.}
\end{itemize}
\begin{itemize}
\item {Proveniência:(Lat. \textunderscore pusillanimitas\textunderscore )}
\end{itemize}
Qualidade de quem é pusillânime.
\section{Pusilanimidade}
\begin{itemize}
\item {Grp. gram.:f.}
\end{itemize}
\begin{itemize}
\item {Proveniência:(Lat. \textunderscore pusillanimitas\textunderscore )}
\end{itemize}
Qualidade de quem é pusillânime.
\section{Pusillânimo}
\begin{itemize}
\item {Grp. gram.:adj.}
\end{itemize}
\begin{itemize}
\item {Utilização:Des.}
\end{itemize}
O mesmo que \textunderscore pusillânime\textunderscore . Cf. \textunderscore Eufrosina\textunderscore .
\section{Pussá}
\begin{itemize}
\item {Grp. gram.:m.}
\end{itemize}
\begin{itemize}
\item {Utilização:Bras}
\end{itemize}
\begin{itemize}
\item {Proveniência:(T. tupi)}
\end{itemize}
Instrumento de pescar camarões, o mesmo que \textunderscore jereré\textunderscore .
\section{Pussá}
\begin{itemize}
\item {Grp. gram.:m.}
\end{itemize}
\begin{itemize}
\item {Utilização:Bras}
\end{itemize}
Fruta do pussazeiro.
\section{Puçá}
\begin{itemize}
\item {Grp. gram.:m.}
\end{itemize}
\begin{itemize}
\item {Utilização:Bras}
\end{itemize}
Fruta do pussazeiro.
\section{Putrilagem}
\begin{itemize}
\item {Grp. gram.:f.}
\end{itemize}
\begin{itemize}
\item {Proveniência:(Do lat. \textunderscore putrilago\textunderscore )}
\end{itemize}
Podridão; corrupção.
(Us. por Camillo)
\section{Putumuju}
\begin{itemize}
\item {Grp. gram.:m.}
\end{itemize}
\begin{itemize}
\item {Utilização:Bras}
\end{itemize}
Árvore silvestre, applicável a construcções navaes.
\section{Puxa}
\begin{itemize}
\item {Grp. gram.:m.}
\end{itemize}
\begin{itemize}
\item {Utilização:Bras. do N}
\end{itemize}
O mesmo que \textunderscore alfenim\textunderscore .
\section{Puxá}
\begin{itemize}
\item {Grp. gram.:m.}
\end{itemize}
\begin{itemize}
\item {Utilização:Bras}
\end{itemize}
O mesmo que \textunderscore puxado\textunderscore .
\section{Puxacares}
\begin{itemize}
\item {Grp. gram.:m. pl.}
\end{itemize}
Tribo de Índios em Mato-Grosso, no Brasil.
\section{Puxada}
\begin{itemize}
\item {Grp. gram.:f.}
\end{itemize}
Acto ou effeito de puxar.
Carta, que um parceiro joga, ao principiar a mão.
\section{Puxadeira}
\begin{itemize}
\item {Grp. gram.:f.}
\end{itemize}
\begin{itemize}
\item {Utilização:Serralh.}
\end{itemize}
\begin{itemize}
\item {Proveniência:(De \textunderscore puxar\textunderscore )}
\end{itemize}
Asa, na extremidade superior dos canos das botas, para as puxar quando se calçam.
Objecto análogo, para erguer ou puxar qualquer coisa.
Peça, para puxar rebites.
\section{Puxadinho}
\begin{itemize}
\item {Grp. gram.:m.  e  adj.}
\end{itemize}
O que se veste com esmêro; peralta.
(Dem. de \textunderscore puxado\textunderscore )
\section{Puxado}
\begin{itemize}
\item {Grp. gram.:adj.}
\end{itemize}
\begin{itemize}
\item {Utilização:Fam.}
\end{itemize}
\begin{itemize}
\item {Utilização:Ant.}
\end{itemize}
\begin{itemize}
\item {Utilização:Chul.}
\end{itemize}
\begin{itemize}
\item {Grp. gram.:M.}
\end{itemize}
\begin{itemize}
\item {Utilização:Bras}
\end{itemize}
\begin{itemize}
\item {Utilização:Bras}
\end{itemize}
\begin{itemize}
\item {Proveniência:(De \textunderscore puxar\textunderscore )}
\end{itemize}
Esmerado ou affectado no vestir.
Muito apurado, (falando-se de alguns preparados culinários).
Elevado, (falando-se do preço de alguma coisa).
O mesmo que \textunderscore bêbedo\textunderscore .
Accréscimo de uma casa para o lado do quintal, onde ordináriamente se estabelece a cozinha, dormitório de criados, etc.
Asma.
\section{Puxadoira}
\begin{itemize}
\item {Grp. gram.:f.}
\end{itemize}
\begin{itemize}
\item {Utilização:Serralh.}
\end{itemize}
Peça, para puxar rebites; o mesmo que \textunderscore puxadeira\textunderscore .
\section{Puxadoura}
\begin{itemize}
\item {Grp. gram.:f.}
\end{itemize}
\begin{itemize}
\item {Utilização:Serralh.}
\end{itemize}
Peça, para puxar rebites; o mesmo que \textunderscore puxadeira\textunderscore .
\section{Puxador}
\begin{itemize}
\item {Grp. gram.:m.}
\end{itemize}
\begin{itemize}
\item {Proveniência:(De \textunderscore puxar\textunderscore )}
\end{itemize}
Peça de madeira ou metal, por onde se puxa, para abrir gavetas, portinholas, etc.
Jogador de pau. Cf. Camillo. \textunderscore Corja\textunderscore .
\section{Puxa-frictor}
\begin{itemize}
\item {Grp. gram.:m.}
\end{itemize}
Instrumento, composto de um cordel e de um gancho, para puxar o frictor.
Uma das peças do apparelho, com que se communica fogo aos canhões e que alguns conhecem pelo nome francês \textunderscore cordon tire-feu\textunderscore .
\section{Puxamento}
\begin{itemize}
\item {Grp. gram.:m.}
\end{itemize}
\begin{itemize}
\item {Utilização:Bras. do Maranhão}
\end{itemize}
Acto de puxar.
Asma.
\section{Puxanço}
\begin{itemize}
\item {Grp. gram.:m.}
\end{itemize}
\begin{itemize}
\item {Utilização:T. do jôgo de bilhar}
\end{itemize}
\begin{itemize}
\item {Proveniência:(De \textunderscore puxar\textunderscore )}
\end{itemize}
Acto de tocar a bola, de fórma que, batendo ella noutra, vá de lado parar a certa distância.
\section{Puxante}
\begin{itemize}
\item {Grp. gram.:adj.}
\end{itemize}
\begin{itemize}
\item {Utilização:Fig.}
\end{itemize}
Que puxa.
Picante, estimulante, (falando-se de algumas comidas).
\section{Puxão}
\begin{itemize}
\item {Grp. gram.:m.}
\end{itemize}
Acto ou effeito de puxar muito; repelão, empuxão.
\section{Puxa-puxa}
\begin{itemize}
\item {Grp. gram.:f.}
\end{itemize}
\begin{itemize}
\item {Utilização:Bras}
\end{itemize}
\begin{itemize}
\item {Proveniência:(De \textunderscore puxar\textunderscore )}
\end{itemize}
Alféloa.
\section{Puxar}
\begin{itemize}
\item {Grp. gram.:v. t.}
\end{itemize}
\begin{itemize}
\item {Utilização:Bras. de Minas}
\end{itemize}
\begin{itemize}
\item {Grp. gram.:V. i.}
\end{itemize}
\begin{itemize}
\item {Utilização:Pop.}
\end{itemize}
\begin{itemize}
\item {Utilização:Fam.}
\end{itemize}
\begin{itemize}
\item {Grp. gram.:Loc.}
\end{itemize}
\begin{itemize}
\item {Utilização:pop.}
\end{itemize}
\begin{itemize}
\item {Grp. gram.:Loc.}
\end{itemize}
\begin{itemize}
\item {Utilização:pop.}
\end{itemize}
\begin{itemize}
\item {Grp. gram.:V. p.}
\end{itemize}
\begin{itemize}
\item {Utilização:Fam.}
\end{itemize}
\begin{itemize}
\item {Proveniência:(Do lat. \textunderscore pulsare\textunderscore )}
\end{itemize}
Attrahir.
Mover para si.
Arrastar.
Fazer vir atrás de si.
Arrancar.
Estirar, esticar: \textunderscore puxar uma corda\textunderscore .
Causar, provocar: \textunderscore puxar o vómito\textunderscore .
Desenvolver.
Avivar.
Aconchegar.
Chupar, absorver.
Apurar: \textunderscore puxar um guisado\textunderscore .
Servir-se de (á mesa): \textunderscore puxar os bifes\textunderscore .
Inclinar-se; tender: \textunderscore puxar para médico\textunderscore .
Exercer tracção: \textunderscore puxar por alguém\textunderscore .
Referir-se.
Fazer menção.
Fazer exigências.
Desenvolver-se.
Apurar-se, fervendo bem, (falando-se de preparados culinários).
Custar muito; sêr caro.
Esforçar-se para fazer dejecções.
\textunderscore Ir puxando\textunderscore , ir-se embora.
Morrer.
\textunderscore Puxar pela palha\textunderscore , puxar pela língua, estimular a loquela a alguém.
Esmerar-se muito no trajar.
\section{Puxativo}
\begin{itemize}
\item {Grp. gram.:adj.}
\end{itemize}
O mesmo que \textunderscore puxante\textunderscore .
\section{Puxavante}
\begin{itemize}
\item {Grp. gram.:m.}
\end{itemize}
\begin{itemize}
\item {Proveniência:(De \textunderscore puxar\textunderscore  + \textunderscore avante\textunderscore )}
\end{itemize}
Instrumento, com que os ferradores apuram o casco que hão de ferrar.
\section{Puxavão}
\begin{itemize}
\item {Grp. gram.:m.}
\end{itemize}
\begin{itemize}
\item {Utilização:Bras. do N}
\end{itemize}
Grande puxão.
\section{Puxeira}
\begin{itemize}
\item {Grp. gram.:f.}
\end{itemize}
\begin{itemize}
\item {Utilização:Bras}
\end{itemize}
\begin{itemize}
\item {Proveniência:(De \textunderscore puxar\textunderscore )}
\end{itemize}
Defluxo.
\section{Puxianas}
\begin{itemize}
\item {Grp. gram.:m. pl.}
\end{itemize}
Indígenas do norte do Brasil.
\section{Puxirão}
\begin{itemize}
\item {Grp. gram.:m.}
\end{itemize}
\begin{itemize}
\item {Utilização:Bras. do S}
\end{itemize}
O mesmo que \textunderscore muxirom\textunderscore .
\section{Pilão}
\begin{itemize}
\item {Grp. gram.:m.}
\end{itemize}
\begin{itemize}
\item {Proveniência:(Gr. \textunderscore pulon\textunderscore )}
\end{itemize}
Grande pórtico dos templos egípcios.
\section{Pilone}
\begin{itemize}
\item {Grp. gram.:m.}
\end{itemize}
O mesmo que \textunderscore pilão\textunderscore ^1. Cf. Herculano, \textunderscore Quest. Públ.\textunderscore , II, 38.
\section{Pilono}
\begin{itemize}
\item {Grp. gram.:m.}
\end{itemize}
O mesmo que \textunderscore pilão\textunderscore ^1. Cf. Herculano, \textunderscore Quest. Públ.\textunderscore , II, 38.
\section{Pilórico}
\begin{itemize}
\item {Grp. gram.:adj.}
\end{itemize}
Relativo ao piloro.
\section{Piloro}
\begin{itemize}
\item {Grp. gram.:m.}
\end{itemize}
\begin{itemize}
\item {Utilização:Anat.}
\end{itemize}
\begin{itemize}
\item {Proveniência:(Gr. \textunderscore puloros\textunderscore )}
\end{itemize}
Orifício inferior do estômago.
\section{Piogênese}
\begin{itemize}
\item {Grp. gram.:f.}
\end{itemize}
\begin{itemize}
\item {Utilização:Med.}
\end{itemize}
\begin{itemize}
\item {Proveniência:(Do gr. \textunderscore puon\textunderscore  + \textunderscore genesis\textunderscore )}
\end{itemize}
Formação de pus.
\section{Piogenia}
\begin{itemize}
\item {Grp. gram.:f.}
\end{itemize}
\begin{itemize}
\item {Proveniência:(Do gr. \textunderscore puon\textunderscore  + \textunderscore genos\textunderscore )}
\end{itemize}
O mesmo que \textunderscore piogênese\textunderscore .
\section{Piogênico}
\begin{itemize}
\item {Grp. gram.:adj.}
\end{itemize}
Relativo a piogenia.
\section{Pioemia}
\textunderscore f.\textunderscore  (e der.)
O mesmo que \textunderscore piemia\textunderscore , etc. Cf. \textunderscore Decreto\textunderscore  de 22-VII-905.
\section{Pir...}
\begin{itemize}
\item {Grp. gram.:pref.}
\end{itemize}
\begin{itemize}
\item {Proveniência:(Gr. \textunderscore pur\textunderscore )}
\end{itemize}
(designativo de \textunderscore fogo\textunderscore  ou \textunderscore inflamação\textunderscore )
\section{Pira}
\begin{itemize}
\item {Grp. gram.:f.}
\end{itemize}
\begin{itemize}
\item {Utilização:Fig.}
\end{itemize}
\begin{itemize}
\item {Proveniência:(Lat. \textunderscore pyra\textunderscore )}
\end{itemize}
Fogueira, em que os antigos reduziam os cadáveres a cinza.
Crisol.
\section{Pirafrólito}
\begin{itemize}
\item {Grp. gram.:m.}
\end{itemize}
\begin{itemize}
\item {Utilização:Miner.}
\end{itemize}
\begin{itemize}
\item {Proveniência:(Do gr. \textunderscore pur\textunderscore  + \textunderscore aphros\textunderscore  + \textunderscore lithos\textunderscore )}
\end{itemize}
Mistura de feldspato e de opala.
\section{Pírale}
\begin{itemize}
\item {Grp. gram.:f.}
\end{itemize}
\begin{itemize}
\item {Proveniência:(Lat. \textunderscore pyralis\textunderscore )}
\end{itemize}
Insecto lepidóptero, nocivo ás videiras.
\section{Pirame}
\begin{itemize}
\item {Grp. gram.:m.}
\end{itemize}
\begin{itemize}
\item {Utilização:Ant.}
\end{itemize}
O mesmo que \textunderscore pirâmide\textunderscore . Cf. Barros, \textunderscore Déc.\textunderscore  II, l. X, c. 6.
\section{Piramidal}
\begin{itemize}
\item {Grp. gram.:adj.}
\end{itemize}
\begin{itemize}
\item {Utilização:Fig.}
\end{itemize}
\begin{itemize}
\item {Grp. gram.:M.}
\end{itemize}
\begin{itemize}
\item {Utilização:Anat.}
\end{itemize}
\begin{itemize}
\item {Proveniência:(Lat. \textunderscore pyramidalis\textunderscore )}
\end{itemize}
Relativo á pirâmide.
Que tem fórma de pirâmide.
Extraordinário, notável.
Muito grande.
Terceiro osso da primeira fileira do carpo.
\section{Piramidalmente}
\begin{itemize}
\item {Grp. gram.:adv.}
\end{itemize}
\begin{itemize}
\item {Utilização:Fig.}
\end{itemize}
De modo piramidal.
Á semelhança de pirâmide.
Extraordinariamente.
\section{Pirâmide}
\begin{itemize}
\item {Grp. gram.:f.}
\end{itemize}
\begin{itemize}
\item {Proveniência:(Lat. \textunderscore pyramis\textunderscore )}
\end{itemize}
Monumento de base rectangular e de quatro faces triangulares que se reunem no cimo.
Construção de fórma piramidal.
Montanha piramidal.
Acervo ou montão de quaesquer objectos, que toma a fórma de pirâmide.
Sólido geométrico, terminado por muitos triângulos com vértice comum e um polígono por base.
\section{Pirâmido}
\begin{itemize}
\item {Grp. gram.:m.}
\end{itemize}
O mesmo ou melhor que \textunderscore piramidona\textunderscore .
\section{Piramidografia}
\begin{itemize}
\item {Grp. gram.:f.}
\end{itemize}
\begin{itemize}
\item {Proveniência:(Do gr. \textunderscore puramis\textunderscore  + \textunderscore graphein\textunderscore )}
\end{itemize}
Descripção de pirâmides.
\section{Piramidona}
\begin{itemize}
\item {Grp. gram.:f.}
\end{itemize}
\begin{itemize}
\item {Proveniência:(Do gr. \textunderscore pyramidon\textunderscore )}
\end{itemize}
Composto químico, derivado da antipirina e superior a ela, como antipirético e analgésico.
\section{Pirantina}
\begin{itemize}
\item {Grp. gram.:f.}
\end{itemize}
\begin{itemize}
\item {Proveniência:(Do gr. \textunderscore pur\textunderscore  + \textunderscore anti\textunderscore ?)}
\end{itemize}
Producto farmacêutico, usado antipirético e sedativo.
\section{Pirargilito}
\begin{itemize}
\item {Grp. gram.:m.}
\end{itemize}
\begin{itemize}
\item {Proveniência:(Do gr. \textunderscore pur\textunderscore  + \textunderscore argilla\textunderscore )}
\end{itemize}
Nome de um silicato de alumínio, magnésio e ferro.
\section{Pirausta}
\begin{itemize}
\item {Grp. gram.:f.}
\end{itemize}
\begin{itemize}
\item {Proveniência:(Lat. \textunderscore pyrausta\textunderscore )}
\end{itemize}
Gênero de insectos lepidópteros nocturnos.
\section{Pirelaína}
\begin{itemize}
\item {Grp. gram.:f.}
\end{itemize}
\begin{itemize}
\item {Proveniência:(Do gr. \textunderscore pur\textunderscore  + \textunderscore elaion\textunderscore )}
\end{itemize}
Nome que se dá, em química, aos óleos empireumáticos.
\section{Pirenaico}
\begin{itemize}
\item {Grp. gram.:adj.}
\end{itemize}
\begin{itemize}
\item {Proveniência:(Lat. \textunderscore pyrenaicus\textunderscore )}
\end{itemize}
Relativo aos Pirenéus.
\section{Pirenaína}
\begin{itemize}
\item {Grp. gram.:f.}
\end{itemize}
\begin{itemize}
\item {Utilização:Chím.}
\end{itemize}
Substância, muito variável em que sua composição e natureza, que se encontra nalgumas águas termaes dos Pirenéus.
\section{Piritoso}
\begin{itemize}
\item {Grp. gram.:adj.}
\end{itemize}
Em que há pirite.
\section{Piro...}
\begin{itemize}
\item {Grp. gram.:pref.}
\end{itemize}
O mesmo que \textunderscore pir...\textunderscore 
\section{Pirobalística}
\begin{itemize}
\item {Grp. gram.:f.}
\end{itemize}
\begin{itemize}
\item {Proveniência:(De \textunderscore piro...\textunderscore  + \textunderscore balistica\textunderscore )}
\end{itemize}
Arte de calcular o alcance das armas de fogo.
\section{Pirobolário}
\begin{itemize}
\item {Grp. gram.:m.}
\end{itemize}
Soldado, que empregava o piróbolo.
\section{Pirobolelogia}
\begin{itemize}
\item {Grp. gram.:f.}
\end{itemize}
\begin{itemize}
\item {Proveniência:(Do gr. \textunderscore purobolos\textunderscore  + \textunderscore logos\textunderscore )}
\end{itemize}
O mesmo que \textunderscore pirotecnia\textunderscore .
\section{Pirobolelógico}
\begin{itemize}
\item {Grp. gram.:adj.}
\end{itemize}
Relativo á pirobolelogia.
\section{Pirobólico}
\begin{itemize}
\item {Grp. gram.:adj.}
\end{itemize}
\begin{itemize}
\item {Proveniência:(De \textunderscore pirôbolo\textunderscore )}
\end{itemize}
Relativo á arte de disparar armas de fogo.
Relativo á pirobalística.
\section{Piróbolo}
\begin{itemize}
\item {Grp. gram.:m.}
\end{itemize}
\begin{itemize}
\item {Proveniência:(Gr. \textunderscore purobolos\textunderscore )}
\end{itemize}
Antiga máquina de guerra, que expelia projécteis inflamados.
\section{Pirobologia}
\begin{itemize}
\item {Grp. gram.:f.}
\end{itemize}
(V.pirobolelogia)
\section{Pirobológico}
\begin{itemize}
\item {Grp. gram.:adj.}
\end{itemize}
Relativo á pirobologia.
\section{Pirocatechina}
\begin{itemize}
\item {Grp. gram.:f.}
\end{itemize}
\begin{itemize}
\item {Utilização:Chím.}
\end{itemize}
Princípio cristalizável, obtido pela destilação da catechina e do ácido morintânnico.
\section{Pirócroa}
\begin{itemize}
\item {Grp. gram.:f.}
\end{itemize}
\begin{itemize}
\item {Proveniência:(Do gr. \textunderscore pur\textunderscore  + \textunderscore khroa\textunderscore )}
\end{itemize}
Insecto coleóptero heterómero.
\section{Pirodina}
\begin{itemize}
\item {Grp. gram.:f.}
\end{itemize}
Producto químico, usado em terapêutica, como antitérmico e analgésico.
\section{Piroelectricidade}
\begin{itemize}
\item {Grp. gram.:f.}
\end{itemize}
\begin{itemize}
\item {Proveniência:(De \textunderscore pyro...\textunderscore  + \textunderscore electricidade\textunderscore )}
\end{itemize}
Electricidade, desenvolvida por meio da elevação da temperatura.
\section{Piroeléctrico}
\begin{itemize}
\item {Grp. gram.:adj.}
\end{itemize}
Relativo á piroelectricidade.
\section{Piroeliómetro}
\begin{itemize}
\item {Grp. gram.:m.}
\end{itemize}
(V.pireliómetro)
\section{Pirogálhico}
\begin{itemize}
\item {Grp. gram.:adj.}
\end{itemize}
Diz-se de um ácido, resultante da destilação do ácido gálico, e que serve para a tintura dos cabelos e para outras aplicações.
\section{Pirogálico}
\begin{itemize}
\item {Grp. gram.:adj.}
\end{itemize}
(V. \textunderscore pirogálhico\textunderscore , que é a fórm. usual em port.)
\section{Pirogenado}
\begin{itemize}
\item {Grp. gram.:adj.}
\end{itemize}
\begin{itemize}
\item {Utilização:Chím.}
\end{itemize}
\begin{itemize}
\item {Proveniência:(De \textunderscore pirogêneo\textunderscore )}
\end{itemize}
Diz-se de um grupo de carbonetos.
\section{Pirogêneo}
\begin{itemize}
\item {Grp. gram.:adj.}
\end{itemize}
\begin{itemize}
\item {Utilização:Chím.}
\end{itemize}
\begin{itemize}
\item {Proveniência:(Do gr. \textunderscore pur\textunderscore  + \textunderscore genos\textunderscore )}
\end{itemize}
Diz-se dos corpos orgânicos, obtidos pela acção do calor sôbre outros compostos orgânicos.
\section{Pirogênese}
\begin{itemize}
\item {Grp. gram.:f.}
\end{itemize}
\begin{itemize}
\item {Utilização:Phýs.}
\end{itemize}
\begin{itemize}
\item {Proveniência:(Do gr. \textunderscore pur\textunderscore  + \textunderscore genea\textunderscore )}
\end{itemize}
Produção de calor.
\section{Pirogênico}
\begin{itemize}
\item {Grp. gram.:adj.}
\end{itemize}
Produzido pelo calor ou pela acção do fogo.
(Cp. \textunderscore pirogêneo\textunderscore )
\section{Pirograma}
\begin{itemize}
\item {Grp. gram.:m.}
\end{itemize}
\begin{itemize}
\item {Proveniência:(Do gr. \textunderscore pur\textunderscore  + \textunderscore gramma\textunderscore )}
\end{itemize}
Gravura em substância combustível, feita com pirolápis ou piyropincel.
\section{Pirogranito}
\begin{itemize}
\item {Grp. gram.:m.}
\end{itemize}
Moderno material de construção, formado de argila refractária, misturada com argila não refractária.
\section{Pirogravar}
\begin{itemize}
\item {Proveniência:(De \textunderscore piro...\textunderscore  + \textunderscore gravar\textunderscore )}
\end{itemize}
Desenhar ou gravar com ponta encandescente.
\section{Pirogravura}
\begin{itemize}
\item {Grp. gram.:f.}
\end{itemize}
Arte de pirogravar.
\section{Piróis}
\begin{itemize}
\item {Grp. gram.:m.}
\end{itemize}
\begin{itemize}
\item {Proveniência:(Do gr. \textunderscore puroeis\textunderscore ?)}
\end{itemize}
Acendedor de piras?
Aludindo ás fogueiras da Inquisição, diz Filinto, III, 140:«\textunderscore ...onde os piróis flammigeros assomados escumam.\textunderscore »
\section{Pirolápis}
\begin{itemize}
\item {Grp. gram.:m.}
\end{itemize}
O mesmo que \textunderscore piropincel\textunderscore .
\section{Pylão}
\begin{itemize}
\item {Grp. gram.:m.}
\end{itemize}
\begin{itemize}
\item {Proveniência:(Gr. \textunderscore pulon\textunderscore )}
\end{itemize}
Grande pórtico dos templos egýpcios.
\section{Pylone}
\begin{itemize}
\item {Grp. gram.:m.}
\end{itemize}
O mesmo que \textunderscore pylão\textunderscore . Cf. Herculano, \textunderscore Quest. Públ.\textunderscore , II, 38.
\section{Pylono}
\begin{itemize}
\item {Grp. gram.:m.}
\end{itemize}
O mesmo que \textunderscore pylão\textunderscore . Cf. Herculano, \textunderscore Quest. Públ.\textunderscore , II, 38.
\section{Pylórico}
\begin{itemize}
\item {Grp. gram.:adj.}
\end{itemize}
Relativo ao pyloro.
\section{Pyloro}
\begin{itemize}
\item {Grp. gram.:m.}
\end{itemize}
\begin{itemize}
\item {Utilização:Anat.}
\end{itemize}
\begin{itemize}
\item {Proveniência:(Gr. \textunderscore puloros\textunderscore )}
\end{itemize}
Orifício inferior do estômago.
\section{Pyogênese}
\begin{itemize}
\item {Grp. gram.:f.}
\end{itemize}
\begin{itemize}
\item {Utilização:Med.}
\end{itemize}
\begin{itemize}
\item {Proveniência:(Do gr. \textunderscore puon\textunderscore  + \textunderscore genesis\textunderscore )}
\end{itemize}
Formação de pus.
\section{Pyogenia}
\begin{itemize}
\item {Grp. gram.:f.}
\end{itemize}
\begin{itemize}
\item {Proveniência:(Do gr. \textunderscore puon\textunderscore  + \textunderscore genos\textunderscore )}
\end{itemize}
O mesmo que \textunderscore pyogênese\textunderscore .
\section{Pyogênico}
\begin{itemize}
\item {Grp. gram.:adj.}
\end{itemize}
Relativo a pyogenia.
\section{Pyohemia}
\textunderscore f.\textunderscore  (e der.)
O mesmo que \textunderscore pyemia\textunderscore , etc. Cf. \textunderscore Decreto\textunderscore  de 22-VII-905.
\section{Pyr...}
\begin{itemize}
\item {Grp. gram.:pref.}
\end{itemize}
\begin{itemize}
\item {Proveniência:(Gr. \textunderscore pur\textunderscore )}
\end{itemize}
(designativo de \textunderscore fogo\textunderscore  ou \textunderscore inflammação\textunderscore )
\section{Pyra}
\begin{itemize}
\item {Grp. gram.:f.}
\end{itemize}
\begin{itemize}
\item {Utilização:Fig.}
\end{itemize}
\begin{itemize}
\item {Proveniência:(Lat. \textunderscore pyra\textunderscore )}
\end{itemize}
Fogueira, em que os antigos reduziam os cadáveres a cinza.
Crysol.
\section{Pýrale}
\begin{itemize}
\item {Grp. gram.:f.}
\end{itemize}
\begin{itemize}
\item {Proveniência:(Lat. \textunderscore pyralis\textunderscore )}
\end{itemize}
Insecto lepidóptero, nocivo ás videiras.
\section{Pyrame}
\begin{itemize}
\item {Grp. gram.:m.}
\end{itemize}
\begin{itemize}
\item {Utilização:Ant.}
\end{itemize}
O mesmo que \textunderscore pyrâmide\textunderscore . Cf. Barros, \textunderscore Déc.\textunderscore  II, l. X, c. 6.
\section{Pyramidal}
\begin{itemize}
\item {Grp. gram.:adj.}
\end{itemize}
\begin{itemize}
\item {Utilização:Fig.}
\end{itemize}
\begin{itemize}
\item {Grp. gram.:M.}
\end{itemize}
\begin{itemize}
\item {Utilização:Anat.}
\end{itemize}
\begin{itemize}
\item {Proveniência:(Lat. \textunderscore pyramidalis\textunderscore )}
\end{itemize}
Relativo á pyrâmide.
Que tem fórma de pyrâmide.
Extraordinário, notável.
Muito grande.
Terceiro osso da primeira fileira do carpo.
\section{Pyramidalmente}
\begin{itemize}
\item {Grp. gram.:adv.}
\end{itemize}
\begin{itemize}
\item {Utilização:Fig.}
\end{itemize}
De modo pyramidal.
Á semelhança de pyrâmide.
Extraordinariamente.
\section{Pyrâmide}
\begin{itemize}
\item {Grp. gram.:f.}
\end{itemize}
\begin{itemize}
\item {Proveniência:(Lat. \textunderscore pyramis\textunderscore )}
\end{itemize}
Monumento de base rectangular e de quatro faces triangulares que se reunem no cimo.
Construcção de fórma pyramidal.
Montanha pyramidal.
Acervo ou montão de quaesquer objectos, que toma a fórma de pyrâmide.
Sólido geométrico, terminado por muitos triângulos com vértice commum e um polýgono por base.
\section{Pyrâmido}
\begin{itemize}
\item {Grp. gram.:m.}
\end{itemize}
O mesmo ou melhor que \textunderscore pyramidona\textunderscore .
\section{Pyramidographia}
\begin{itemize}
\item {Grp. gram.:f.}
\end{itemize}
\begin{itemize}
\item {Proveniência:(Do gr. \textunderscore puramis\textunderscore  + \textunderscore graphein\textunderscore )}
\end{itemize}
Descripção de pyrâmides.
\section{Pyramidona}
\begin{itemize}
\item {Grp. gram.:f.}
\end{itemize}
\begin{itemize}
\item {Proveniência:(Do gr. \textunderscore pyramidon\textunderscore )}
\end{itemize}
Composto chímico, derivado da antipyrina e superior a ella, como antipyrético e analgésico.
\section{Pyrantina}
\begin{itemize}
\item {Grp. gram.:f.}
\end{itemize}
\begin{itemize}
\item {Proveniência:(Do gr. \textunderscore pur\textunderscore  + \textunderscore anti\textunderscore ?)}
\end{itemize}
Producto pharmacêutico, usado antipyrético e sedativo.
\section{Pyraphrólitho}
\begin{itemize}
\item {Grp. gram.:m.}
\end{itemize}
\begin{itemize}
\item {Utilização:Miner.}
\end{itemize}
\begin{itemize}
\item {Proveniência:(Do gr. \textunderscore pur\textunderscore  + \textunderscore aphros\textunderscore  + \textunderscore lithos\textunderscore )}
\end{itemize}
Mistura de feldspato e de opala.
\section{Pyrargillito}
\begin{itemize}
\item {Grp. gram.:m.}
\end{itemize}
\begin{itemize}
\item {Proveniência:(Do gr. \textunderscore pur\textunderscore  + \textunderscore argilla\textunderscore )}
\end{itemize}
Nome de um silicato de alumínio, magnésio e ferro.
\section{Pyrausta}
\begin{itemize}
\item {Grp. gram.:f.}
\end{itemize}
\begin{itemize}
\item {Proveniência:(Lat. \textunderscore pyrausta\textunderscore )}
\end{itemize}
Gênero de insectos lepidópteros nocturnos.
\section{Pyrelaína}
\begin{itemize}
\item {Grp. gram.:f.}
\end{itemize}
\begin{itemize}
\item {Proveniência:(Do gr. \textunderscore pur\textunderscore  + \textunderscore elaion\textunderscore )}
\end{itemize}
Nome que se dá, em chímica, aos óleos empyreumáticos.
\section{Pyrenaico}
\begin{itemize}
\item {Grp. gram.:adj.}
\end{itemize}
\begin{itemize}
\item {Proveniência:(Lat. \textunderscore pyrenaicus\textunderscore )}
\end{itemize}
Relativo aos Pyrenéus.
\section{Pyrenaína}
\begin{itemize}
\item {Grp. gram.:f.}
\end{itemize}
\begin{itemize}
\item {Utilização:Chím.}
\end{itemize}
Substância, muito variável em que sua composição e natureza, que se encontra nalgumas águas thermaes dos Pyrenéus.
\section{Pyritoso}
\begin{itemize}
\item {Grp. gram.:adj.}
\end{itemize}
Em que há pyrite.
\section{Pyro...}
\begin{itemize}
\item {Grp. gram.:pref.}
\end{itemize}
O mesmo que \textunderscore pyr...\textunderscore 
\section{Pyrobalística}
\begin{itemize}
\item {Grp. gram.:f.}
\end{itemize}
\begin{itemize}
\item {Proveniência:(De \textunderscore pyro...\textunderscore  + \textunderscore balistica\textunderscore )}
\end{itemize}
Arte de calcular o alcance das armas de fogo.
\section{Pyrobolário}
\begin{itemize}
\item {Grp. gram.:m.}
\end{itemize}
Soldado, que empregava o pyróbolo.
\section{Pyrobolelogia}
\begin{itemize}
\item {Grp. gram.:f.}
\end{itemize}
\begin{itemize}
\item {Proveniência:(Do gr. \textunderscore purobolos\textunderscore  + \textunderscore logos\textunderscore )}
\end{itemize}
O mesmo que \textunderscore pyrotechnia\textunderscore .
\section{Pyrobolelógico}
\begin{itemize}
\item {Grp. gram.:adj.}
\end{itemize}
Relativo á pyrobolelogia.
\section{Pyrobólico}
\begin{itemize}
\item {Grp. gram.:adj.}
\end{itemize}
\begin{itemize}
\item {Proveniência:(De \textunderscore pyrôbolo\textunderscore )}
\end{itemize}
Relativo á arte de disparar armas de fogo.
Relativo á pyrobalística.
\section{Pyróbolo}
\begin{itemize}
\item {Grp. gram.:m.}
\end{itemize}
\begin{itemize}
\item {Proveniência:(Gr. \textunderscore purobolos\textunderscore )}
\end{itemize}
Antiga máquina de guerra, que expellia projécteis inflammados.
\section{Pyrobologia}
\begin{itemize}
\item {Grp. gram.:f.}
\end{itemize}
(V.pyrobolelogia)
\section{Pyrobológico}
\begin{itemize}
\item {Grp. gram.:adj.}
\end{itemize}
Relativo á pyrobologia.
\section{Pyrocatechina}
\begin{itemize}
\item {Grp. gram.:f.}
\end{itemize}
\begin{itemize}
\item {Utilização:Chím.}
\end{itemize}
Princípio crystallizável, obtido pela destillação da catechina e do ácido morintânnico.
\section{Pyróchroa}
\begin{itemize}
\item {Grp. gram.:f.}
\end{itemize}
\begin{itemize}
\item {Proveniência:(Do gr. \textunderscore pur\textunderscore  + \textunderscore khroa\textunderscore )}
\end{itemize}
Insecto coleóptero heterómero.
\section{Pyrodina}
\begin{itemize}
\item {Grp. gram.:f.}
\end{itemize}
Producto chímico, usado em therapêutica, como antithérmico e analgésico.
\section{Pyroelectricidade}
\begin{itemize}
\item {Grp. gram.:f.}
\end{itemize}
\begin{itemize}
\item {Proveniência:(De \textunderscore pyro...\textunderscore  + \textunderscore electricidade\textunderscore )}
\end{itemize}
Electricidade, desenvolvida por meio da elevação da temperatura.
\section{Pyroeléctrico}
\begin{itemize}
\item {Grp. gram.:adj.}
\end{itemize}
Relativo á pyroelectricidade.
\section{Pyroes}
\begin{itemize}
\item {Grp. gram.:m.}
\end{itemize}
\begin{itemize}
\item {Proveniência:(Do gr. \textunderscore puroeis\textunderscore ?)}
\end{itemize}
Accendedor de pyras?
Alludindo ás fogueiras da Inquisição, diz Filinto, III, 140:«\textunderscore ...onde os pyroes flammigeros assomados escumam.\textunderscore »
\section{Pyrogálhico}
\begin{itemize}
\item {Grp. gram.:adj.}
\end{itemize}
Diz-se de um ácido, resultante da destillação do ácido gállico, e que serve para a tintura dos cabellos e para outras applicações.
\section{Pyrogállico}
\begin{itemize}
\item {Grp. gram.:adj.}
\end{itemize}
(V. \textunderscore pirogálhico\textunderscore , que é a fórm. usual em port.)
\section{Pyrogenado}
\begin{itemize}
\item {Grp. gram.:adj.}
\end{itemize}
\begin{itemize}
\item {Utilização:Chím.}
\end{itemize}
\begin{itemize}
\item {Proveniência:(De \textunderscore pyrogêneo\textunderscore )}
\end{itemize}
Diz-se de um grupo de carbonetos.
\section{Pyrogêneo}
\begin{itemize}
\item {Grp. gram.:adj.}
\end{itemize}
\begin{itemize}
\item {Utilização:Chím.}
\end{itemize}
\begin{itemize}
\item {Proveniência:(Do gr. \textunderscore pur\textunderscore  + \textunderscore genos\textunderscore )}
\end{itemize}
Diz-se dos corpos orgânicos, obtidos pela acção do calor sôbre outros compostos orgânicos.
\section{Pyrogênese}
\begin{itemize}
\item {Grp. gram.:f.}
\end{itemize}
\begin{itemize}
\item {Utilização:Phýs.}
\end{itemize}
\begin{itemize}
\item {Proveniência:(Do gr. \textunderscore pur\textunderscore  + \textunderscore genea\textunderscore )}
\end{itemize}
Producção de calor.
\section{Pyrogênico}
\begin{itemize}
\item {Grp. gram.:adj.}
\end{itemize}
Produzido pelo calor ou pela acção do fogo.
(Cp. \textunderscore pyrogêneo\textunderscore )
\section{Pyrogramma}
\begin{itemize}
\item {Grp. gram.:m.}
\end{itemize}
\begin{itemize}
\item {Proveniência:(Do gr. \textunderscore pur\textunderscore  + \textunderscore gramma\textunderscore )}
\end{itemize}
Gravura em substância combustível, feita com pyrolápis ou pyropincel.
\section{Pyrogranito}
\begin{itemize}
\item {Grp. gram.:m.}
\end{itemize}
Moderno material de construcção, formado de argilla refractária, misturada com argilla não refractária.
\section{Pyrogravar}
\begin{itemize}
\item {Proveniência:(De \textunderscore pyro...\textunderscore  + \textunderscore gravar\textunderscore )}
\end{itemize}
Desenhar ou gravar com ponta encandescente.
\section{Pyrogravura}
\begin{itemize}
\item {Grp. gram.:f.}
\end{itemize}
Arte de pyrogravar.
\section{Pyroheliómetro}
\begin{itemize}
\item {Grp. gram.:m.}
\end{itemize}
(V.pyrheliómetro)
\section{Pyrolápis}
\begin{itemize}
\item {Grp. gram.:m.}
\end{itemize}
O mesmo que \textunderscore pyropincel\textunderscore .
\section{Pirósis}
\begin{itemize}
\item {Grp. gram.:f.}
\end{itemize}
O mesmo que \textunderscore pirose\textunderscore .
\section{Pirossoma}
\begin{itemize}
\item {Grp. gram.:m.}
\end{itemize}
\begin{itemize}
\item {Proveniência:(Do gr. \textunderscore pur\textunderscore  + \textunderscore soma\textunderscore )}
\end{itemize}
Um dos hematozoarios da frebre.
\section{Pirotecnia}
\begin{itemize}
\item {Grp. gram.:f.}
\end{itemize}
\begin{itemize}
\item {Proveniência:(Do gr. \textunderscore pur\textunderscore  + \textunderscore tekhne\textunderscore )}
\end{itemize}
Arte de empregar o fogo.
Conjunto dos conhecimentos necessários para a fabricação dos fogos de artifício.
\section{Pirotécnica}
\begin{itemize}
\item {Grp. gram.:f.}
\end{itemize}
O mesmo que \textunderscore pirotecnia\textunderscore .
(Fem. de \textunderscore pirotécnico\textunderscore )
\section{Pirotécnico}
\begin{itemize}
\item {Grp. gram.:adj.}
\end{itemize}
\begin{itemize}
\item {Grp. gram.:M.}
\end{itemize}
Relativo á pirotecnia.
Aquele que fabrica fogo de artifício.
\section{Pirótico}
\begin{itemize}
\item {Grp. gram.:adj.}
\end{itemize}
\begin{itemize}
\item {Grp. gram.:M.}
\end{itemize}
\begin{itemize}
\item {Proveniência:(Gr. \textunderscore purotikos\textunderscore )}
\end{itemize}
Que cauteriza.
Medicamento para cauterizar.
\section{Piroxena}
\begin{itemize}
\item {fónica:csê}
\end{itemize}
\begin{itemize}
\item {Grp. gram.:f.}
\end{itemize}
O mesmo que \textunderscore piroxênio\textunderscore .
\section{Piroxênico}
\begin{itemize}
\item {fónica:csê}
\end{itemize}
\begin{itemize}
\item {Grp. gram.:adj.}
\end{itemize}
Relativo ao piroxênio.
\section{Piroxênio}
\begin{itemize}
\item {fónica:csê}
\end{itemize}
\begin{itemize}
\item {Grp. gram.:m.}
\end{itemize}
O mesmo ou melhor que \textunderscore piroxeno\textunderscore .
\section{Piroxenite}
\begin{itemize}
\item {fónica:cse}
\end{itemize}
\begin{itemize}
\item {Grp. gram.:f.}
\end{itemize}
\begin{itemize}
\item {Utilização:Miner.}
\end{itemize}
Rocha primitiva, em que predomina o piroxênio, associado a uma plagioclase.
\section{Piroxeno}
\begin{itemize}
\item {fónica:csê}
\end{itemize}
\begin{itemize}
\item {Grp. gram.:m.}
\end{itemize}
\begin{itemize}
\item {Proveniência:(Do gr. \textunderscore pur\textunderscore , fogo, e \textunderscore xenos\textunderscore , estranho)}
\end{itemize}
Designação de diversos mineraes, que se encontram em terrenos vulcânicos, mas cuja formação se supunha procedente de via aquosa.
\section{Piróxila}
\begin{itemize}
\item {Grp. gram.:f.}
\end{itemize}
\begin{itemize}
\item {Proveniência:(Do gr. \textunderscore pur\textunderscore  + \textunderscore oxulon\textunderscore )}
\end{itemize}
Producto explosivo, o mesmo que \textunderscore algodão-pólvora\textunderscore .
\section{Piroxíleo}
\begin{itemize}
\item {fónica:csi}
\end{itemize}
\begin{itemize}
\item {Grp. gram.:m.}
\end{itemize}
O mesmo que \textunderscore colodionito\textunderscore .
\section{Piroxilina}
\begin{itemize}
\item {fónica:csi}
\end{itemize}
\begin{itemize}
\item {Grp. gram.:f.}
\end{itemize}
O mesmo ou melhor que \textunderscore piroxilino\textunderscore .
\section{Piroxilino}
\begin{itemize}
\item {fónica:csi}
\end{itemize}
\begin{itemize}
\item {Grp. gram.:f.}
\end{itemize}
O mesmo que \textunderscore piróxila\textunderscore .
\section{Piróxilo}
\begin{itemize}
\item {fónica:csi}
\end{itemize}
\begin{itemize}
\item {Grp. gram.:m.}
\end{itemize}
O mesmo que \textunderscore piróxila\textunderscore .
\section{Piroxilol}
\begin{itemize}
\item {fónica:csi}
\end{itemize}
\begin{itemize}
\item {Grp. gram.:m.}
\end{itemize}
O mesmo que \textunderscore piroxilina\textunderscore .
\section{Pirozone}
\begin{itemize}
\item {Grp. gram.:m.}
\end{itemize}
O mesmo que \textunderscore pirozónio\textunderscore .
\section{Pirozónio}
\begin{itemize}
\item {Grp. gram.:m.}
\end{itemize}
Mistura de éter e água oxigenada, que os dentistas empregam no tratamento dos dentes.
\section{Pírrica}
\begin{itemize}
\item {Grp. gram.:f.}
\end{itemize}
\begin{itemize}
\item {Proveniência:(Lat. \textunderscore pyrrhicha\textunderscore )}
\end{itemize}
Espécie de dança, que se realizava com as armas na mão.
\section{Pirríquio}
\begin{itemize}
\item {Grp. gram.:m.}
\end{itemize}
\begin{itemize}
\item {Proveniência:(Lat. \textunderscore pyrrhíchius\textunderscore )}
\end{itemize}
Pé de verso grego ou latino, composto de duas sílabas breves.
\section{Pirronicamente}
\begin{itemize}
\item {Grp. gram.:adv.}
\end{itemize}
De modo pirrónico; com teimosia.
\section{Pirrónico}
\begin{itemize}
\item {Grp. gram.:adj.}
\end{itemize}
\begin{itemize}
\item {Utilização:Ext.}
\end{itemize}
\begin{itemize}
\item {Utilização:Fam.}
\end{itemize}
\begin{itemize}
\item {Proveniência:(Do lat. \textunderscore Pyrrho\textunderscore , \textunderscore Pyrrhonis\textunderscore , n. p.)}
\end{itemize}
Que segue a doutrina do pirronismo.
Que duvida de tudo.
Teimoso.
\section{Pirronismo}
\begin{itemize}
\item {Grp. gram.:m.}
\end{itemize}
\begin{itemize}
\item {Utilização:Ext.}
\end{itemize}
\begin{itemize}
\item {Utilização:Fam.}
\end{itemize}
Sistema filosófico, que tinha por base duvidar tudo: cepticismo.
Hábito de duvidar de tudo.
Teimosia.
(Cp. \textunderscore pirrónico\textunderscore )
\section{Pirrotina}
\begin{itemize}
\item {Grp. gram.:f.}
\end{itemize}
O mesmo que \textunderscore pirrotite\textunderscore .
\section{Pirrotite}
\begin{itemize}
\item {Grp. gram.:f.}
\end{itemize}
\begin{itemize}
\item {Proveniência:(Do gr. \textunderscore purrotes\textunderscore )}
\end{itemize}
Espécie de bisulfureto, que exerce acção na agulha magnética.
\section{Pirúvico}
\begin{itemize}
\item {Grp. gram.:adj.}
\end{itemize}
\begin{itemize}
\item {Proveniência:(De \textunderscore piro...\textunderscore  + \textunderscore úvico\textunderscore )}
\end{itemize}
Diz-se de um acido, produzido pela destilação sêca do ácido tartárico.
\section{Pitagórico}
\begin{itemize}
\item {Grp. gram.:adj.}
\end{itemize}
\begin{itemize}
\item {Proveniência:(De \textunderscore Pitágoras\textunderscore , n. p.)}
\end{itemize}
Relativo a Pitágoras ou ás suas doutrinas.
\section{Pitonissa}
\begin{itemize}
\item {Grp. gram.:f.}
\end{itemize}
O mesmo que \textunderscore pitonisa\textunderscore .
\section{Pitonomorfo}
\begin{itemize}
\item {Grp. gram.:m.}
\end{itemize}
\begin{itemize}
\item {Proveniência:(Do gr. \textunderscore puthon\textunderscore  + \textunderscore morphe\textunderscore )}
\end{itemize}
Reptil fóssil, do tipo dos labyrintodontes.
\section{Piulco}
\begin{itemize}
\item {Grp. gram.:m.}
\end{itemize}
\begin{itemize}
\item {Utilização:Cir.}
\end{itemize}
\begin{itemize}
\item {Proveniência:(Gr. \textunderscore puoulkos\textunderscore )}
\end{itemize}
Instrumento, para extrair de uma cavidade do corpo matérias purulentas.
\section{Piuria}
\begin{itemize}
\item {Grp. gram.:f.}
\end{itemize}
\begin{itemize}
\item {Proveniência:(Do gr. \textunderscore puon\textunderscore  + \textunderscore ouron\textunderscore )}
\end{itemize}
Emissão de urina purulenta.
\section{Pixacanto}
\begin{itemize}
\item {fónica:csa}
\end{itemize}
\begin{itemize}
\item {Grp. gram.:m.}
\end{itemize}
\begin{itemize}
\item {Proveniência:(Lat. \textunderscore pyxacanthus\textunderscore )}
\end{itemize}
Arbusto espinhoso.
\section{Píxide}
\begin{itemize}
\item {fónica:csi}
\end{itemize}
\begin{itemize}
\item {Grp. gram.:f.}
\end{itemize}
\begin{itemize}
\item {Utilização:Bot.}
\end{itemize}
\begin{itemize}
\item {Proveniência:(Lat. \textunderscore pyxis\textunderscore , \textunderscore pyxidis\textunderscore )}
\end{itemize}
Vaso, em que se guardam as hóstias ou partículas consagradas.
Fruto, que se abre ao meio em duas valvas sobrepostas.
\section{Pixídio}
\begin{itemize}
\item {fónica:csi}
\end{itemize}
\begin{itemize}
\item {Grp. gram.:m.}
\end{itemize}
\begin{itemize}
\item {Utilização:Bot.}
\end{itemize}
\begin{itemize}
\item {Proveniência:(Do gr. \textunderscore puxidion\textunderscore )}
\end{itemize}
O mesmo ou melhor que \textunderscore píxide\textunderscore .
\section{Pixídula}
\begin{itemize}
\item {fónica:csi}
\end{itemize}
\begin{itemize}
\item {Grp. gram.:f.}
\end{itemize}
\begin{itemize}
\item {Utilização:Bot.}
\end{itemize}
Urnário dos musgos.
(Dem. de \textunderscore píxide\textunderscore )
\section{Pixínia}
\begin{itemize}
\item {fónica:csi}
\end{itemize}
\begin{itemize}
\item {Grp. gram.:f.}
\end{itemize}
\begin{itemize}
\item {Proveniência:(Do gr. \textunderscore puxis\textunderscore )}
\end{itemize}
Gênero de vermes intestinaes.
\section{Pyrósis}
\begin{itemize}
\item {Grp. gram.:f.}
\end{itemize}
O mesmo que \textunderscore pyrose\textunderscore .
\section{Pyrosoma}
\begin{itemize}
\item {fónica:só}
\end{itemize}
\begin{itemize}
\item {Grp. gram.:m.}
\end{itemize}
\begin{itemize}
\item {Proveniência:(Do gr. \textunderscore pur\textunderscore  + \textunderscore soma\textunderscore )}
\end{itemize}
Um dos hematozoarios da frebre.
\section{Pyrotechnia}
\begin{itemize}
\item {Grp. gram.:f.}
\end{itemize}
\begin{itemize}
\item {Proveniência:(Do gr. \textunderscore pur\textunderscore  + \textunderscore tekhne\textunderscore )}
\end{itemize}
Arte de empregar o fogo.
Conjunto dos conhecimentos necessários para a fabricação dos fogos de artifício.
\section{Pyrotéchnica}
\begin{itemize}
\item {Grp. gram.:f.}
\end{itemize}
O mesmo que \textunderscore pyrotechnia\textunderscore .
(Fem. de \textunderscore pyrotéchnico\textunderscore )
\section{Pyrotéchnico}
\begin{itemize}
\item {Grp. gram.:adj.}
\end{itemize}
\begin{itemize}
\item {Grp. gram.:M.}
\end{itemize}
Relativo á pyrotechnía.
Aquelle que fabrica fogo de artifício.
\section{Pyrótico}
\begin{itemize}
\item {Grp. gram.:adj.}
\end{itemize}
\begin{itemize}
\item {Grp. gram.:M.}
\end{itemize}
\begin{itemize}
\item {Proveniência:(Gr. \textunderscore purotikos\textunderscore )}
\end{itemize}
Que cauteriza.
Medicamento para cauterizar.
\section{Pyroxena}
\begin{itemize}
\item {fónica:csê}
\end{itemize}
\begin{itemize}
\item {Grp. gram.:f.}
\end{itemize}
O mesmo que \textunderscore pyroxênio\textunderscore .
\section{Pyroxênico}
\begin{itemize}
\item {fónica:csê}
\end{itemize}
\begin{itemize}
\item {Grp. gram.:adj.}
\end{itemize}
Relativo ao pyroxênio.
\section{Pyroxênio}
\begin{itemize}
\item {fónica:csê}
\end{itemize}
\begin{itemize}
\item {Grp. gram.:m.}
\end{itemize}
O mesmo ou melhor que \textunderscore pyroxeno\textunderscore .
\section{Pyroxenite}
\begin{itemize}
\item {fónica:cse}
\end{itemize}
\begin{itemize}
\item {Grp. gram.:f.}
\end{itemize}
\begin{itemize}
\item {Utilização:Miner.}
\end{itemize}
Rocha primitiva, em que predomina o pyroxênio, associado a uma plagioclase.
\section{Pyroxeno}
\begin{itemize}
\item {fónica:csê}
\end{itemize}
\begin{itemize}
\item {Grp. gram.:m.}
\end{itemize}
\begin{itemize}
\item {Proveniência:(Do gr. \textunderscore pur\textunderscore , fogo, e \textunderscore xenos\textunderscore , estranho)}
\end{itemize}
Designação de diversos mineraes, que se encontram em terrenos vulcânicos, mas cuja formação se suppunha procedente de via aquosa.
\section{Pyróxyla}
\begin{itemize}
\item {Grp. gram.:f.}
\end{itemize}
\begin{itemize}
\item {Proveniência:(Do gr. \textunderscore pur\textunderscore  + \textunderscore oxulon\textunderscore )}
\end{itemize}
Producto explosivo, o mesmo que \textunderscore algodão-pólvora\textunderscore .
\section{Pyroxýleo}
\begin{itemize}
\item {fónica:csi}
\end{itemize}
\begin{itemize}
\item {Grp. gram.:m.}
\end{itemize}
O mesmo que \textunderscore collodionito\textunderscore .
\section{Pyroxilina}
\begin{itemize}
\item {fónica:csi}
\end{itemize}
\begin{itemize}
\item {Grp. gram.:f.}
\end{itemize}
O mesmo ou melhor que \textunderscore pyroxilino\textunderscore .
\section{Pyroxilino}
\begin{itemize}
\item {fónica:csi}
\end{itemize}
\begin{itemize}
\item {Grp. gram.:f.}
\end{itemize}
O mesmo que \textunderscore pyróxyla\textunderscore .
\section{Pyróxilo}
\begin{itemize}
\item {fónica:csi}
\end{itemize}
\begin{itemize}
\item {Grp. gram.:m.}
\end{itemize}
O mesmo que \textunderscore pyróxyla\textunderscore .
\section{Pyroxilol}
\begin{itemize}
\item {fónica:csi}
\end{itemize}
\begin{itemize}
\item {Grp. gram.:m.}
\end{itemize}
O mesmo que \textunderscore pyroxilina\textunderscore .
\section{Pyrozone}
\begin{itemize}
\item {Grp. gram.:m.}
\end{itemize}
O mesmo que \textunderscore pyrozónio\textunderscore .
\section{Pyrozónio}
\begin{itemize}
\item {Grp. gram.:m.}
\end{itemize}
Mistura de éther e água oxygenada, que os dentistas empregam no tratamento dos dentes.
\section{Pýrrhicha}
\begin{itemize}
\item {fónica:ca}
\end{itemize}
\begin{itemize}
\item {Grp. gram.:f.}
\end{itemize}
\begin{itemize}
\item {Proveniência:(Lat. \textunderscore pyrrhicha\textunderscore )}
\end{itemize}
Espécie de dança, que se realizava com as armas na mão.
\section{Pyrrhíchio}
\begin{itemize}
\item {fónica:qui}
\end{itemize}
\begin{itemize}
\item {Grp. gram.:m.}
\end{itemize}
\begin{itemize}
\item {Proveniência:(Lat. \textunderscore pyrrhíchius\textunderscore )}
\end{itemize}
Pé de verso grego ou latino, composto de duas sýlabas breves.
\section{Pyrrhonicamente}
\begin{itemize}
\item {Grp. gram.:adv.}
\end{itemize}
De modo pyrrhónico; com teimosia.
\section{Pyrrhónico}
\begin{itemize}
\item {Grp. gram.:adj.}
\end{itemize}
\begin{itemize}
\item {Utilização:Ext.}
\end{itemize}
\begin{itemize}
\item {Utilização:Fam.}
\end{itemize}
\begin{itemize}
\item {Proveniência:(Do lat. \textunderscore Pyrrho\textunderscore , \textunderscore Pyrrhonis\textunderscore , n. p.)}
\end{itemize}
Que segue a doutrina do pyrrhonismo.
Que duvida de tudo.
Teimoso.
\section{Pyrrhonismo}
\begin{itemize}
\item {Grp. gram.:m.}
\end{itemize}
\begin{itemize}
\item {Utilização:Ext.}
\end{itemize}
\begin{itemize}
\item {Utilização:Fam.}
\end{itemize}
Systema philosóphico, que tinha por base duvidar tudo: scepticismo.
Hábito de duvidar de tudo.
Teimosia.
(Cp. \textunderscore pyrrhónico\textunderscore )
\section{Pyrrotina}
\begin{itemize}
\item {Grp. gram.:f.}
\end{itemize}
O mesmo que \textunderscore pyrrotite\textunderscore .
\section{Pyrrotite}
\begin{itemize}
\item {Grp. gram.:f.}
\end{itemize}
\begin{itemize}
\item {Proveniência:(Do gr. \textunderscore purrotes\textunderscore )}
\end{itemize}
Espécie de bisulfureto, que exerce acção na agulha magnética.
\section{Pyrúvico}
\begin{itemize}
\item {Grp. gram.:adj.}
\end{itemize}
\begin{itemize}
\item {Proveniência:(De \textunderscore pyro...\textunderscore  + \textunderscore úvico\textunderscore )}
\end{itemize}
Diz-se de um acido, produzido pela destillação sêca do ácido tartárico.
\section{Pythagórico}
\begin{itemize}
\item {Grp. gram.:adj.}
\end{itemize}
\begin{itemize}
\item {Proveniência:(De \textunderscore Pythágoras\textunderscore , n. p.)}
\end{itemize}
Relativo a Pythágoras ou ás suas doutrinas.
\section{Pythonissa}
\begin{itemize}
\item {Grp. gram.:f.}
\end{itemize}
O mesmo que \textunderscore pythonisa\textunderscore .
\section{Pythonomorpho}
\begin{itemize}
\item {Grp. gram.:m.}
\end{itemize}
\begin{itemize}
\item {Proveniência:(Do gr. \textunderscore puthon\textunderscore  + \textunderscore morphe\textunderscore )}
\end{itemize}
Reptil fóssil, do typo dos labyrinthodontes.
\section{Pyulco}
\begin{itemize}
\item {Grp. gram.:m.}
\end{itemize}
\begin{itemize}
\item {Utilização:Cir.}
\end{itemize}
\begin{itemize}
\item {Proveniência:(Gr. \textunderscore puoulkos\textunderscore )}
\end{itemize}
Instrumento, para extrahir de uma cavidade do corpo matérias purulentas.
\section{Pyuria}
\begin{itemize}
\item {Grp. gram.:f.}
\end{itemize}
\begin{itemize}
\item {Proveniência:(Do gr. \textunderscore puon\textunderscore  + \textunderscore ouron\textunderscore )}
\end{itemize}
Emissão de urina purulenta.
\section{Pyxacantho}
\begin{itemize}
\item {fónica:csa}
\end{itemize}
\begin{itemize}
\item {Grp. gram.:m.}
\end{itemize}
\begin{itemize}
\item {Proveniência:(Lat. \textunderscore pyxacanthus\textunderscore )}
\end{itemize}
Arbusto espinhoso.
\section{Pýxide}
\begin{itemize}
\item {fónica:csi}
\end{itemize}
\begin{itemize}
\item {Grp. gram.:f.}
\end{itemize}
\begin{itemize}
\item {Utilização:Bot.}
\end{itemize}
\begin{itemize}
\item {Proveniência:(Lat. \textunderscore pyxis\textunderscore , \textunderscore pyxidis\textunderscore )}
\end{itemize}
Vaso, em que se guardam as hóstias ou partículas consagradas.
Fruto, que se abre ao meio em duas valvas sobrepostas.
\section{Pyxídio}
\begin{itemize}
\item {fónica:csi}
\end{itemize}
\begin{itemize}
\item {Grp. gram.:m.}
\end{itemize}
\begin{itemize}
\item {Utilização:Bot.}
\end{itemize}
\begin{itemize}
\item {Proveniência:(Do gr. \textunderscore puxidion\textunderscore )}
\end{itemize}
O mesmo ou melhor que \textunderscore pýxide\textunderscore .
\section{Pyxídula}
\begin{itemize}
\item {fónica:csi}
\end{itemize}
\begin{itemize}
\item {Grp. gram.:f.}
\end{itemize}
\begin{itemize}
\item {Utilização:Bot.}
\end{itemize}
Urnário dos musgos.
(Dem. de \textunderscore pýxyde\textunderscore )
\section{Pyxínia}
\begin{itemize}
\item {fónica:csi}
\end{itemize}
\begin{itemize}
\item {Grp. gram.:f.}
\end{itemize}
\begin{itemize}
\item {Proveniência:(Do gr. \textunderscore puxis\textunderscore )}
\end{itemize}
Gênero de vermes intestinaes.
\section{P}
\begin{itemize}
\item {fónica:pê}
\end{itemize}
\begin{itemize}
\item {Grp. gram.:m.}
\end{itemize}
Décima sexta letra do alphabeto português.
Abrev. de \textunderscore padre\textunderscore , de \textunderscore pollegada\textunderscore , etc.
\textunderscore 5 p. %\textunderscore , cinco por cento.
Toma-se adjectivamente, referindo se àquillo que numa série de 16 occupa o último lugar: \textunderscore livro P\textunderscore ; \textunderscore fôlha P\textunderscore .
\section{Pá}
\begin{itemize}
\item {Grp. gram.:f.}
\end{itemize}
\begin{itemize}
\item {Proveniência:(Do lat. \textunderscore pala\textunderscore )}
\end{itemize}
Utensílio de madeira ou do ferro chato, com rebordos lateraes e um cabo.
Parte mais larga e carnuda da perna das reses.
\section{Pá!}
\begin{itemize}
\item {Grp. gram.:interj.}
\end{itemize}
Termo onomatopaico, para exprimir o som da quéda de um corpo duro ou o choque de dois corpos.
\section{Paadar}
\begin{itemize}
\item {Grp. gram.:m.}
\end{itemize}
(Uma das fórmas evolutivas de \textunderscore paladar\textunderscore . Cf. \textunderscore Eufrosina\textunderscore , 39)
\section{Pabulagem}
\begin{itemize}
\item {Grp. gram.:f.}
\end{itemize}
\begin{itemize}
\item {Utilização:Bras}
\end{itemize}
\begin{itemize}
\item {Proveniência:(De \textunderscore pábulo\textunderscore ^2)}
\end{itemize}
Mentira; embuste.
Pedantismo.
\section{Pabular}
\begin{itemize}
\item {Grp. gram.:v. i.}
\end{itemize}
\begin{itemize}
\item {Utilização:Bras. do N}
\end{itemize}
\begin{itemize}
\item {Proveniência:(De \textunderscore pábulo\textunderscore ^2)}
\end{itemize}
Gabar-se, vangloriar-se.
\section{Pábulo}
\begin{itemize}
\item {Grp. gram.:m.}
\end{itemize}
\begin{itemize}
\item {Utilização:Fig.}
\end{itemize}
\begin{itemize}
\item {Proveniência:(Lat. \textunderscore pabulum\textunderscore )}
\end{itemize}
Pasto; sustento.
Aquillo que serve de assumpto ou motivo a motejo ou maledicência.
\section{Pábulo}
\begin{itemize}
\item {Grp. gram.:adj.}
\end{itemize}
\begin{itemize}
\item {Utilização:Bras. do N}
\end{itemize}
Que é gabarola, soberbo.
(Talvez de \textunderscore Pablo\textunderscore , n. p.)
\section{Paca}
\begin{itemize}
\item {Grp. gram.:f.}
\end{itemize}
\begin{itemize}
\item {Proveniência:(T. tupi)}
\end{itemize}
Quadrúpede roedor da América do Sul, (\textunderscore cavia paca\textunderscore , Lin.).
\section{Paca}
\begin{itemize}
\item {Grp. gram.:f.}
\end{itemize}
\begin{itemize}
\item {Proveniência:(Do b. lat. \textunderscore paccus\textunderscore )}
\end{itemize}
Fardo; pacote.
\section{Paca}
\begin{itemize}
\item {Grp. gram.:f.}
\end{itemize}
Árvore da Índia portuguesa.
\section{Pacaás}
\begin{itemize}
\item {Grp. gram.:m. pl.}
\end{itemize}
\begin{itemize}
\item {Utilização:Bras}
\end{itemize}
Tríbo de Índios do Brasil, em Mato-Grosso.
\section{Pacabote}
\begin{itemize}
\item {Grp. gram.:m.}
\end{itemize}
\begin{itemize}
\item {Utilização:Ant.}
\end{itemize}
O mesmo que \textunderscore paquebote\textunderscore . Cf. \textunderscore Anat. Joc.\textunderscore , II, 425.
Espécie de pequena carruagem antiga:«\textunderscore ...apeou do seu pacabote, puxado a seis urcos\textunderscore ». Camillo, \textunderscore Cav. em Ruínas\textunderscore , 99. Cf. \textunderscore Diár.-de-Notícias\textunderscore , de Lisbôa, de 5-IX-900.
\section{Pacaça}
\begin{itemize}
\item {Grp. gram.:f.}
\end{itemize}
Árvore de Moçambique.
\section{Pacachodéus}
\begin{itemize}
\item {Grp. gram.:m. pl.}
\end{itemize}
\begin{itemize}
\item {Utilização:Bras}
\end{itemize}
Tríbo de aborígenes de Mato-Grosso.
\section{Pacahás}
\begin{itemize}
\item {Grp. gram.:m. pl.}
\end{itemize}
\begin{itemize}
\item {Utilização:Bras}
\end{itemize}
Tríbo de Índios do Brasil, em Mato-Grosso.
\section{Pacaiás}
\begin{itemize}
\item {Grp. gram.:m. pl.}
\end{itemize}
\begin{itemize}
\item {Utilização:Bras}
\end{itemize}
Tríbo de aborígenes do Pará.
\section{Pacalho}
\begin{itemize}
\item {Grp. gram.:m.}
\end{itemize}
\begin{itemize}
\item {Utilização:Bras}
\end{itemize}
O mesmo que \textunderscore perda\textunderscore .
\section{Pacamão}
\begin{itemize}
\item {Grp. gram.:m.}
\end{itemize}
Peixe do Brasil.
\section{Pàção}
\begin{itemize}
\item {Grp. gram.:adj.}
\end{itemize}
\begin{itemize}
\item {Utilização:Ant.}
\end{itemize}
\begin{itemize}
\item {Proveniência:(De \textunderscore paço\textunderscore )}
\end{itemize}
Palaciano, cortês. Cf. G. Vicente, I, 179.
\section{Pacapaca}
\begin{itemize}
\item {Grp. gram.:f.}
\end{itemize}
Ave da Guiana.
\section{Pacará}
\begin{itemize}
\item {Grp. gram.:m.}
\end{itemize}
\begin{itemize}
\item {Utilização:Bras}
\end{itemize}
Espécie de bahu ou cesto, construído de folhetas de madeira leve.
\section{Pacaratepu}
\begin{itemize}
\item {Grp. gram.:m.}
\end{itemize}
\begin{itemize}
\item {Utilização:Bras}
\end{itemize}
Planta medicinal da região do Amazonas.
\section{Pacari}
\begin{itemize}
\item {Grp. gram.:m.}
\end{itemize}
\begin{itemize}
\item {Utilização:Bras}
\end{itemize}
Cipó medicinal.
\section{Pacari}
\begin{itemize}
\item {Grp. gram.:m.}
\end{itemize}
\begin{itemize}
\item {Utilização:T. da Índia port}
\end{itemize}
Alpendre; alpendrada:«\textunderscore ...incêndio que pegou nuns pacaris de resguardo contra as chuvas\textunderscore ». Th. Ribeiro, \textunderscore Jornadas\textunderscore , II, 200.
\section{Pacarim}
\begin{itemize}
\item {Grp. gram.:m.}
\end{itemize}
\begin{itemize}
\item {Utilização:T. da Índia port}
\end{itemize}
Alpendre; alpendrada:«\textunderscore ...incêndio que pegou nuns pacarins de resguardo contra as chuvas\textunderscore ». Th. Ribeiro, \textunderscore Jornadas\textunderscore , II, 200.
\section{Pacascas}
\begin{itemize}
\item {Grp. gram.:m.}
\end{itemize}
Espécie de pão de açúcar, que nas Filippinas se fórma com o suco da palmeira.
\section{Pacaso}
\begin{itemize}
\item {Grp. gram.:m.}
\end{itemize}
Mammífero do Congo, semelhante ao búfalo.
\section{Pacata}
\begin{itemize}
\item {Grp. gram.:f.}
\end{itemize}
\begin{itemize}
\item {Utilização:Fam.}
\end{itemize}
O mesmo que \textunderscore pacatez\textunderscore .
\section{Pacatamente}
\begin{itemize}
\item {Grp. gram.:adv.}
\end{itemize}
De modo pacato; sossegadamente; pacificamente.
\section{Pacatez}
\begin{itemize}
\item {Grp. gram.:f.}
\end{itemize}
Qualidade do que é pacato; índole pacífica.
\section{Pacato}
\begin{itemize}
\item {Grp. gram.:m.  e  adj.}
\end{itemize}
\begin{itemize}
\item {Proveniência:(Lat. \textunderscore pacatus\textunderscore )}
\end{itemize}
O que é amigo da paz; pacífico.
Tranquillo; socegado.
\section{Pacau}
\begin{itemize}
\item {Grp. gram.:m.}
\end{itemize}
Antigo jôgo de cartas:«\textunderscore ...a jogar o pacau co'as confessadas\textunderscore ». Filinto, V, 122.
\section{Pacaviva}
\begin{itemize}
\item {Grp. gram.:f.}
\end{itemize}
\begin{itemize}
\item {Utilização:Bras}
\end{itemize}
Bananeira silvestre, cujo fruto não é comestível.
\section{Pàceiro}
\begin{itemize}
\item {Grp. gram.:m.  e  adj.}
\end{itemize}
\begin{itemize}
\item {Utilização:Ant.}
\end{itemize}
O que frequenta o paço real; cortesão.
Intendente ou inspector das obras e fábricas dos paços reaes.
\section{Pacejar}
\begin{itemize}
\item {Grp. gram.:v. i.}
\end{itemize}
\begin{itemize}
\item {Utilização:Ant.}
\end{itemize}
\begin{itemize}
\item {Proveniência:(De \textunderscore paço\textunderscore )}
\end{itemize}
O mesmo que \textunderscore gracejar\textunderscore .
\section{Pacés}
\begin{itemize}
\item {Grp. gram.:m. pl.}
\end{itemize}
Tríbo extinta do Alto Amazonas.
\section{Paçhá}
\begin{itemize}
\item {Grp. gram.:m.}
\end{itemize}
(V.paxá)
\section{Pachalizar}
\begin{itemize}
\item {Grp. gram.:v. i.}
\end{itemize}
\begin{itemize}
\item {Utilização:Neol.}
\end{itemize}
Proceder como um pachá:«\textunderscore ...o presbýtero, que pachalizava com as louras filhas espirituaes\textunderscore ». Camillo, \textunderscore Caveira\textunderscore , 364.
\section{Pachan}
\begin{itemize}
\item {Grp. gram.:m.}
\end{itemize}
Espécie de granito pardo da costa do Malabar.
\section{Pachão}
\begin{itemize}
\item {Grp. gram.:m.}
\end{itemize}
Pequeno peixe do mar.
O mesmo que \textunderscore pexão\textunderscore .
\section{Pacharel}
\begin{itemize}
\item {Grp. gram.:m.}
\end{itemize}
O mesmo que \textunderscore pacharil\textunderscore .
\section{Pacharil}
\begin{itemize}
\item {Grp. gram.:m.}
\end{itemize}
\begin{itemize}
\item {Proveniência:(T. as.)}
\end{itemize}
Arroz com casca.
\section{Pacharro}
\begin{itemize}
\item {Grp. gram.:m.}
\end{itemize}
Espécie de goraz das costas do Algarve.
\section{Pachel}
\begin{itemize}
\item {Grp. gram.:m.}
\end{itemize}
O mesmo que \textunderscore pacharro\textunderscore .
\section{Pachelão}
\begin{itemize}
\item {Grp. gram.:m.}
\end{itemize}
O mesmo que \textunderscore pacharro\textunderscore .
\section{Pachnólitho}
\begin{itemize}
\item {Grp. gram.:m.}
\end{itemize}
\begin{itemize}
\item {Utilização:Miner.}
\end{itemize}
\begin{itemize}
\item {Proveniência:(Do gr. \textunderscore pakhne\textunderscore , geada + \textunderscore lithos\textunderscore , pedra)}
\end{itemize}
Fluoreto de alumínio, cálcio e sódio.
\section{Pacho}
\begin{itemize}
\item {Grp. gram.:m.}
\end{itemize}
\begin{itemize}
\item {Utilização:Pop.}
\end{itemize}
O mesmo que \textunderscore parche\textunderscore :«\textunderscore queria arrancar os pachos da cabeça\textunderscore ». Camillo, \textunderscore Brasileira\textunderscore , 296.
\section{Pachochada}
\begin{itemize}
\item {Grp. gram.:f.}
\end{itemize}
O mesmo que \textunderscore pachouchada\textunderscore .
\section{Pachocho}
\begin{itemize}
\item {fónica:chô}
\end{itemize}
\begin{itemize}
\item {Grp. gram.:m.}
\end{itemize}
\begin{itemize}
\item {Utilização:Pleb.}
\end{itemize}
\begin{itemize}
\item {Utilização:T. de Barcelos}
\end{itemize}
Partes pudendas da mulhér.
Indivíduo apalermado.
\section{Pachola}
\begin{itemize}
\item {Grp. gram.:m.}
\end{itemize}
\begin{itemize}
\item {Utilização:Chul.}
\end{itemize}
Pateta.
Mandrião.
Farçola; patusco.
\section{Pacholice}
\begin{itemize}
\item {Grp. gram.:f.}
\end{itemize}
\begin{itemize}
\item {Utilização:Chul.}
\end{itemize}
Acto ou dito de pachola.
\section{Pachómetro}
\begin{itemize}
\item {fónica:có}
\end{itemize}
\begin{itemize}
\item {Grp. gram.:m.}
\end{itemize}
\begin{itemize}
\item {Proveniência:(Do gr. \textunderscore pakhos\textunderscore  + \textunderscore metron\textunderscore )}
\end{itemize}
Instrumento, para medir a espessura dos corpos.
\section{Pachorra}
\begin{itemize}
\item {fónica:chô}
\end{itemize}
\begin{itemize}
\item {Grp. gram.:f.}
\end{itemize}
Falta de diligência ou de pressa; lentidão.
Fleugma; pânria.
\section{Pachorrentamente}
\begin{itemize}
\item {Grp. gram.:adv.}
\end{itemize}
De modo pachorrento.
\section{Pachorrento}
\begin{itemize}
\item {Grp. gram.:adj.}
\end{itemize}
Que tem pachorra; feito com pachorra; que revela pachorra.
\section{Pachouchada}
\begin{itemize}
\item {Grp. gram.:f.}
\end{itemize}
\begin{itemize}
\item {Utilização:Chul.}
\end{itemize}
Tolice.
Palavrada; dito obsceno.
(Cp. cast. \textunderscore patochada\textunderscore )
\section{Pachoucheta}
\begin{itemize}
\item {fónica:chê}
\end{itemize}
\begin{itemize}
\item {Grp. gram.:f.}
\end{itemize}
\begin{itemize}
\item {Utilização:Bras}
\end{itemize}
O mesmo que \textunderscore pachouchada\textunderscore . Cf. Pacheco, \textunderscore Promptuário\textunderscore .
\section{Pachyblepharose}
\begin{itemize}
\item {fónica:qui}
\end{itemize}
\begin{itemize}
\item {Grp. gram.:f.}
\end{itemize}
\begin{itemize}
\item {Utilização:Med.}
\end{itemize}
\begin{itemize}
\item {Proveniência:(Do gr. \textunderscore pakhus\textunderscore  + \textunderscore blepharon\textunderscore )}
\end{itemize}
Espessidão do tecido das palpebras.
\section{Pachycephalia}
\begin{itemize}
\item {fónica:qui}
\end{itemize}
\begin{itemize}
\item {Grp. gram.:f.}
\end{itemize}
Qualidade ou estado de pachycéphalo.
\section{Pachycéphalo}
\begin{itemize}
\item {fónica:qui}
\end{itemize}
\begin{itemize}
\item {Grp. gram.:adj.}
\end{itemize}
\begin{itemize}
\item {Proveniência:(Do gr. \textunderscore pakhus\textunderscore  + \textunderscore kephale\textunderscore )}
\end{itemize}
Que tem as paredes do crânio espêssas.
\section{Pachycerina}
\begin{itemize}
\item {fónica:qui}
\end{itemize}
\begin{itemize}
\item {Grp. gram.:f.}
\end{itemize}
\begin{itemize}
\item {Proveniência:(Do gr. \textunderscore pakhus\textunderscore  + \textunderscore keras\textunderscore )}
\end{itemize}
Gênero de insectos dípteros.
\section{Pachychymia}
\begin{itemize}
\item {fónica:qui}
\end{itemize}
\begin{itemize}
\item {Grp. gram.:f.}
\end{itemize}
\begin{itemize}
\item {Utilização:Med.}
\end{itemize}
\begin{itemize}
\item {Proveniência:(Do gr. \textunderscore pakhus\textunderscore  + \textunderscore khumos\textunderscore )}
\end{itemize}
Espessidão mórbida dos humores.
\section{Pachydáctylo}
\begin{itemize}
\item {fónica:qui}
\end{itemize}
\begin{itemize}
\item {Grp. gram.:m.}
\end{itemize}
\begin{itemize}
\item {Proveniência:(Do gr. \textunderscore pakhus\textunderscore  + \textunderscore daktulos\textunderscore )}
\end{itemize}
Gênero de reptís sáurios.
\section{Pachydema}
\begin{itemize}
\item {fónica:qui}
\end{itemize}
\begin{itemize}
\item {Grp. gram.:m.}
\end{itemize}
Gênero de insectos coleópteros pentâmeros.
\section{Pachydendro}
\begin{itemize}
\item {fónica:qui}
\end{itemize}
\begin{itemize}
\item {Grp. gram.:m.}
\end{itemize}
\begin{itemize}
\item {Proveniência:(Do gr. \textunderscore pakhus\textunderscore  + \textunderscore dendron\textunderscore )}
\end{itemize}
Gênero de plantas liliáceas.
\section{Pachydermatocele}
\begin{itemize}
\item {fónica:qui}
\end{itemize}
\begin{itemize}
\item {Grp. gram.:m.}
\end{itemize}
\begin{itemize}
\item {Proveniência:(Do gr. \textunderscore pakhus\textunderscore  + \textunderscore derma\textunderscore  + \textunderscore kele\textunderscore )}
\end{itemize}
Tumor cutâneo, hypertrophia do tecido laminoso.
\section{Pachyderme}
\begin{itemize}
\item {fónica:qui}
\end{itemize}
\begin{itemize}
\item {Grp. gram.:adj.}
\end{itemize}
\begin{itemize}
\item {Grp. gram.:M. pl.}
\end{itemize}
\begin{itemize}
\item {Proveniência:(Do gr. \textunderscore pakhus\textunderscore  + \textunderscore derma\textunderscore )}
\end{itemize}
Que tem a pelle espessa.
Ordem de mammíferos pachydermes.
\section{Pachydérmico}
\begin{itemize}
\item {fónica:qui}
\end{itemize}
\begin{itemize}
\item {Grp. gram.:adj.}
\end{itemize}
Relativo aos pachydermes; que tem pelle semelhante á dos pachydermes.
\section{Pachydermo}
\begin{itemize}
\item {fónica:qui}
\end{itemize}
\begin{itemize}
\item {Grp. gram.:m.  e  adj.}
\end{itemize}
\begin{itemize}
\item {Proveniência:(Gr. \textunderscore pakhudermos\textunderscore )}
\end{itemize}
O mesmo ou melhor que \textunderscore pachyderme\textunderscore . Cf. R. Galvão, \textunderscore Vocab.\textunderscore 
\section{Pachygástrico}
\begin{itemize}
\item {fónica:qui}
\end{itemize}
\begin{itemize}
\item {Grp. gram.:adj.}
\end{itemize}
\begin{itemize}
\item {Utilização:Zool.}
\end{itemize}
\begin{itemize}
\item {Proveniência:(Do gr. \textunderscore pakhus\textunderscore  + \textunderscore gaster\textunderscore )}
\end{itemize}
Que tem o ventre muito grosso.
\section{Pachylêmures}
\begin{itemize}
\item {fónica:qui}
\end{itemize}
\begin{itemize}
\item {Grp. gram.:m. pl.}
\end{itemize}
\begin{itemize}
\item {Utilização:Zool.}
\end{itemize}
\begin{itemize}
\item {Proveniência:(Do gr. \textunderscore pakhus\textunderscore  + lat. \textunderscore lemures\textunderscore )}
\end{itemize}
Animaes de pelle espêssa, classificados entre os insectívoros e os carnívoros.
\section{Pachylépide}
\begin{itemize}
\item {fónica:qui}
\end{itemize}
\begin{itemize}
\item {Grp. gram.:f.}
\end{itemize}
\begin{itemize}
\item {Proveniência:(Do gr. \textunderscore pakhus\textunderscore  + \textunderscore lepis\textunderscore )}
\end{itemize}
Planta chicoriácea.
\section{Pachylócero}
\begin{itemize}
\item {fónica:qui}
\end{itemize}
\begin{itemize}
\item {Grp. gram.:m.}
\end{itemize}
Gênero de insectos coleópteros heterómeros.
\section{Pachýlopo}
\begin{itemize}
\item {fónica:qui}
\end{itemize}
\begin{itemize}
\item {Grp. gram.:m.}
\end{itemize}
Gênero de insectos coleópteros pentâmeros.
\section{Pachyma}
\begin{itemize}
\item {fónica:qui}
\end{itemize}
\begin{itemize}
\item {Grp. gram.:m.}
\end{itemize}
Gênero de cogumelos grossos e subterrâneos.
\section{Pachymeningite}
\begin{itemize}
\item {fónica:qui}
\end{itemize}
\begin{itemize}
\item {Grp. gram.:f.}
\end{itemize}
\begin{itemize}
\item {Utilização:Med.}
\end{itemize}
\begin{itemize}
\item {Proveniência:(De \textunderscore pakhus\textunderscore  + \textunderscore meningite\textunderscore )}
\end{itemize}
Inflammação da dura-máter.
\section{Pachymerina}
\begin{itemize}
\item {fónica:qui}
\end{itemize}
\begin{itemize}
\item {Grp. gram.:f.}
\end{itemize}
Gênero de insectos dípteros.
\section{Pachymorpho}
\begin{itemize}
\item {fónica:qui}
\end{itemize}
\begin{itemize}
\item {Grp. gram.:m.}
\end{itemize}
\begin{itemize}
\item {Proveniência:(Do gr. \textunderscore pakhus\textunderscore  + \textunderscore morphe\textunderscore )}
\end{itemize}
Gênero de insectos coleópteros pentâmeros.
\section{Pachyphyllo}
\begin{itemize}
\item {fónica:qui}
\end{itemize}
\begin{itemize}
\item {Grp. gram.:adj.}
\end{itemize}
\begin{itemize}
\item {Utilização:Bot.}
\end{itemize}
\begin{itemize}
\item {Proveniência:(Do gr. \textunderscore pakhus\textunderscore  + \textunderscore phullon\textunderscore )}
\end{itemize}
Que tem fôlhas espessas.
\section{Pachypleuro}
\begin{itemize}
\item {fónica:qui}
\end{itemize}
\begin{itemize}
\item {Grp. gram.:m.}
\end{itemize}
\begin{itemize}
\item {Proveniência:(Lat. bot. \textunderscore pachypleurum\textunderscore )}
\end{itemize}
Gênero de plantas umbellíferas.
\section{Pachyrina}
\begin{itemize}
\item {fónica:qui}
\end{itemize}
\begin{itemize}
\item {Grp. gram.:f.}
\end{itemize}
Gênero de insectos dípteros.
\section{Pachyrrhizo}
\begin{itemize}
\item {fónica:qui}
\end{itemize}
\begin{itemize}
\item {Proveniência:(Do gr. \textunderscore pakhus\textunderscore  + \textunderscore rhíza\textunderscore )}
\end{itemize}
Gênero de plantas leguminosas.
\section{Pachyrrhýnchidos}
\begin{itemize}
\item {fónica:qui}
\end{itemize}
\begin{itemize}
\item {Grp. gram.:m. pl.}
\end{itemize}
\begin{itemize}
\item {Utilização:Zool.}
\end{itemize}
\begin{itemize}
\item {Proveniência:(Do gr. \textunderscore pakhus\textunderscore  + \textunderscore rhunkhos\textunderscore )}
\end{itemize}
Uma das divisões, em que se classificam os insectos coleópteros tetrâmeros.
\section{Pachystêmone}
\begin{itemize}
\item {fónica:quis}
\end{itemize}
\begin{itemize}
\item {Grp. gram.:m.}
\end{itemize}
\begin{itemize}
\item {Proveniência:(Do gr. \textunderscore pakhus\textunderscore  + \textunderscore stemon\textunderscore )}
\end{itemize}
Gênero de plantas euphorbiáceas.
\section{Pachýstoma}
\begin{itemize}
\item {fónica:quis}
\end{itemize}
\begin{itemize}
\item {Grp. gram.:m.}
\end{itemize}
\begin{itemize}
\item {Proveniência:(Do gr. \textunderscore pakhus\textunderscore  + \textunderscore stoma\textunderscore )}
\end{itemize}
Gênero de orchídeas.
\section{Pachýstomo}
\begin{itemize}
\item {fónica:quis}
\end{itemize}
\begin{itemize}
\item {Grp. gram.:m.}
\end{itemize}
\begin{itemize}
\item {Proveniência:(Do gr. \textunderscore pakhus\textunderscore  + \textunderscore stoma\textunderscore )}
\end{itemize}
Gênero de insectos dípteros.
\section{Pachýtria}
\begin{itemize}
\item {fónica:qui}
\end{itemize}
\begin{itemize}
\item {Grp. gram.:f.}
\end{itemize}
Gênero de insectos coleópteros pentâmeros.
(Cp. \textunderscore pachýtrico\textunderscore )
\section{Pachýtrico}
\begin{itemize}
\item {fónica:qui}
\end{itemize}
\begin{itemize}
\item {Grp. gram.:adj.}
\end{itemize}
\begin{itemize}
\item {Utilização:Zool.}
\end{itemize}
\begin{itemize}
\item {Proveniência:(Do gr. \textunderscore pakhus\textunderscore  + \textunderscore trikhos\textunderscore )}
\end{itemize}
Que tem pêlo espesso.
\section{Pacido}
\begin{itemize}
\item {Grp. gram.:adj.}
\end{itemize}
\begin{itemize}
\item {Utilização:Ant.}
\end{itemize}
Dizia-se do campo ou terreno, a que o gado comeu toda a erva.
(Por \textunderscore pascido\textunderscore , de \textunderscore pascer\textunderscore )
\section{Paciência}
\begin{itemize}
\item {Grp. gram.:f.}
\end{itemize}
\begin{itemize}
\item {Utilização:Prov.}
\end{itemize}
\begin{itemize}
\item {Utilização:trasm.}
\end{itemize}
\begin{itemize}
\item {Grp. gram.:Interj.}
\end{itemize}
\begin{itemize}
\item {Proveniência:(Lat. \textunderscore patientia\textunderscore )}
\end{itemize}
Qualidade de quem supporta males ou incômmodos, sem se queixar.
Resignação.
Sangue frio.
Insistência tranquilla em trabalho difficil e longo.
Nome de vários jogos ou entretenimentos.
O mesmo que \textunderscore labaça\textunderscore ^1.
Rêde cónica de pesca, que tem na base um arco de arame e no vértice uma bola de chumbo.
(para exprimir \textunderscore resignação\textunderscore )
\section{Pacientar}
\begin{itemize}
\item {Grp. gram.:v. i.}
\end{itemize}
\begin{itemize}
\item {Utilização:bras}
\end{itemize}
\begin{itemize}
\item {Utilização:Neol.}
\end{itemize}
Têr paciência:«\textunderscore pacientei quanto pude\textunderscore ». M. Assis, \textunderscore B. Cubas\textunderscore .
\section{Paciente}
\begin{itemize}
\item {Grp. gram.:m. ,  f.  e  adj.}
\end{itemize}
\begin{itemize}
\item {Utilização:Gram.}
\end{itemize}
\begin{itemize}
\item {Grp. gram.:Adj.}
\end{itemize}
\begin{itemize}
\item {Proveniência:(Lat. \textunderscore patiens\textunderscore )}
\end{itemize}
Pessôa, que tem paciência.
Pessôa, que padece ou vai padecer.
Doente.
O que recebe a acção de um agente.
Complemento objectivo.
Manso, pacífico.
\section{Pacientemente}
\begin{itemize}
\item {Grp. gram.:adv.}
\end{itemize}
De modo paciente; com resignação.
\section{Pacificação}
\begin{itemize}
\item {Grp. gram.:f.}
\end{itemize}
\begin{itemize}
\item {Proveniência:(Lat. \textunderscore pacificatio\textunderscore )}
\end{itemize}
Acto ou effeito de pacificar.
\section{Pacificador}
\begin{itemize}
\item {Grp. gram.:m.  e  adj.}
\end{itemize}
\begin{itemize}
\item {Proveniência:(Lat. \textunderscore pacificator\textunderscore )}
\end{itemize}
O que pacifica.
\section{Pacificar}
\begin{itemize}
\item {Grp. gram.:v. t.}
\end{itemize}
\begin{itemize}
\item {Proveniência:(Lat. \textunderscore pacificare\textunderscore )}
\end{itemize}
Restituir a paz a; apaziguar; tranquillizar, serenar.
\section{Pacificidade}
\begin{itemize}
\item {Grp. gram.:f.}
\end{itemize}
Qualidade de pacífico.
\section{Pacífico}
\begin{itemize}
\item {Grp. gram.:adj.}
\end{itemize}
\begin{itemize}
\item {Grp. gram.:M.}
\end{itemize}
\begin{itemize}
\item {Proveniência:(Lat. \textunderscore pacificus\textunderscore )}
\end{itemize}
Amigo da paz.
Manso.
Tranquillo, sereno.
Pacato.
Indivíduo pacífico.
\section{Pacifismo}
\begin{itemize}
\item {Grp. gram.:m.}
\end{itemize}
\begin{itemize}
\item {Utilização:Neol.}
\end{itemize}
\begin{itemize}
\item {Proveniência:(T. mal derivado do fr. \textunderscore pacífer\textunderscore )}
\end{itemize}
Systema dos que pugnam pela paz universal e pelo desarmamento das nações.
\section{Pacifista}
\begin{itemize}
\item {Grp. gram.:m.  e  f.}
\end{itemize}
\begin{itemize}
\item {Utilização:Neol.}
\end{itemize}
Pessôa, partidária do pacifismo.
\section{Pacigo}
\begin{itemize}
\item {Grp. gram.:m.}
\end{itemize}
\begin{itemize}
\item {Utilização:Ant.}
\end{itemize}
O mesmo que \textunderscore pascigo\textunderscore .
\section{Pacivira}
\begin{itemize}
\item {Grp. gram.:f.}
\end{itemize}
Planta canácea, (\textunderscore canna glauca\textunderscore ).
\section{Pacnólito}
\begin{itemize}
\item {Grp. gram.:m.}
\end{itemize}
\begin{itemize}
\item {Utilização:Miner.}
\end{itemize}
\begin{itemize}
\item {Proveniência:(Do gr. \textunderscore pakhne\textunderscore , geada + \textunderscore lithos\textunderscore , pedra)}
\end{itemize}
Fluoreto de alumínio, cálcio e sódio.
\section{Paco}
\begin{itemize}
\item {Grp. gram.:m.}
\end{itemize}
Árvore de Angola.
\section{Paco}
\begin{itemize}
\item {Grp. gram.:m.}
\end{itemize}
\begin{itemize}
\item {Utilização:Ant.}
\end{itemize}
Carneiro grande e de bôa raça, originário da América.
\section{Paco}
\begin{itemize}
\item {Grp. gram.:m.}
\end{itemize}
\begin{itemize}
\item {Utilização:Ant.}
\end{itemize}
O mesmo que \textunderscore pago\textunderscore ^2.
\section{Pacó}
\begin{itemize}
\item {Grp. gram.:m.}
\end{itemize}
\begin{itemize}
\item {Proveniência:(Do conc. \textunderscore pakho\textunderscore )}
\end{itemize}
Espécie de grande morcego do Oriente.
\section{Paço}
\begin{itemize}
\item {Grp. gram.:m.}
\end{itemize}
\begin{itemize}
\item {Utilização:Ant.}
\end{itemize}
\begin{itemize}
\item {Utilização:Ant.}
\end{itemize}
\begin{itemize}
\item {Grp. gram.:Pl.}
\end{itemize}
\begin{itemize}
\item {Proveniência:(Lat. palatium &gt; it. pallazo &lt; hypoth. port. &lt; palaço &lt; paaço &lt; paço)}
\end{itemize}
Residência habitual de Reis ou Príncipes.
Residência dos Prelados ecclesiásticos e do prelado universitário.
Palácio real.
Cortesãos, que vivem com os reis.
A côrte.
Cartório de tabellião.
O mesmo que \textunderscore gracejo\textunderscore . Cf. G. Viana, \textunderscore Apostilas\textunderscore .
Residência de senhores feudaes.
Solar de família nobre.
Paços do concelho, casa, em que funcciona a câmara municipal.
\section{Pacoba}
\begin{itemize}
\item {Grp. gram.:f.}
\end{itemize}
\begin{itemize}
\item {Utilização:Bras. do N}
\end{itemize}
Fruto da pacobeira.
Nome genérico da banana.
\section{Pacobal}
\begin{itemize}
\item {Grp. gram.:m.}
\end{itemize}
\begin{itemize}
\item {Utilização:Bras. do N}
\end{itemize}
\begin{itemize}
\item {Proveniência:(De \textunderscore pacova\textunderscore )}
\end{itemize}
O mesmo que \textunderscore bananal\textunderscore .
\section{Paco-bala}
\begin{itemize}
\item {Grp. gram.:f.}
\end{itemize}
O mesmo que \textunderscore paco-balo\textunderscore .
\section{Paco-balo}
\begin{itemize}
\item {Grp. gram.:m.}
\end{itemize}
Gênero de árvores rutáceas de Angola.
\section{Pacobeira}
\begin{itemize}
\item {Grp. gram.:f.}
\end{itemize}
\begin{itemize}
\item {Proveniência:(De \textunderscore pacoba\textunderscore )}
\end{itemize}
Grande bananeira do Brasil.
\section{Paco-caatinga}
\begin{itemize}
\item {Grp. gram.:f.}
\end{itemize}
\begin{itemize}
\item {Utilização:Bras}
\end{itemize}
Planta amomácea, (\textunderscore costus Pisonis\textunderscore ).
\section{Pacóio}
\begin{itemize}
\item {Grp. gram.:m.}
\end{itemize}
\begin{itemize}
\item {Utilização:Des.}
\end{itemize}
O mesmo que \textunderscore pacóvio\textunderscore .
\section{Pacol}
\begin{itemize}
\item {Grp. gram.:m.}
\end{itemize}
Planta indiana. Cf. Th. Ribeiro, \textunderscore Jornadas\textunderscore , II, 68.
\section{Pacolé}
\begin{itemize}
\item {Grp. gram.:m.}
\end{itemize}
\begin{itemize}
\item {Utilização:Bras}
\end{itemize}
Espécie de algodoeiro.
\section{Pacómetro}
\begin{itemize}
\item {Grp. gram.:m.}
\end{itemize}
\begin{itemize}
\item {Proveniência:(Do gr. \textunderscore pakhos\textunderscore  + \textunderscore metron\textunderscore )}
\end{itemize}
Instrumento, para medir a espessura dos corpos.
\section{Pacó-pio}
\begin{itemize}
\item {Grp. gram.:m.}
\end{itemize}
\begin{itemize}
\item {Utilização:T. de Macau}
\end{itemize}
Lotaria, que se realiza duas vezes por dia em Macau.
(Do chin.)
\section{Paco-seroca}
\begin{itemize}
\item {Grp. gram.:f.}
\end{itemize}
Planta amomácea do Brasil, (\textunderscore alpinia\textunderscore , \textunderscore paco-seroca\textunderscore , Jacq.).
\section{Pacote}
\begin{itemize}
\item {Grp. gram.:m.}
\end{itemize}
\begin{itemize}
\item {Proveniência:(De \textunderscore paca\textunderscore ^2)}
\end{itemize}
Pequeno fardo; embrulho.
\section{Pacotilha}
\begin{itemize}
\item {Grp. gram.:f.}
\end{itemize}
\begin{itemize}
\item {Utilização:Gal}
\end{itemize}
\begin{itemize}
\item {Proveniência:(Fr. \textunderscore pacotille\textunderscore )}
\end{itemize}
Porção de gêneros, que um passageiro de um navio póde levar consigo, sem pagar o transporte delles.
Pacote pequeno.
Mercadorias várias e de pouca importancia, que o commandante ou passageiro de navio vai encarregado de vender em país remoto.
Mercancia ordinária.
Artefacto mal acabado ou grosseiro.
\section{Pacotilho}
\begin{itemize}
\item {Grp. gram.:m.}
\end{itemize}
Pacote pequeno:«\textunderscore um pacotilho de manuscriptos.\textunderscore »Camillo, \textunderscore Caveira\textunderscore , 6.
\section{Pacova}
\textunderscore f.\textunderscore  (e der.)
O mesmo que \textunderscore pacoba\textunderscore , etc.
\section{Pacová}
\begin{itemize}
\item {Grp. gram.:f.}
\end{itemize}
Planta amomácea do Brasil.
\section{Pacoval}
\begin{itemize}
\item {Grp. gram.:m.}
\end{itemize}
\begin{itemize}
\item {Utilização:Bras. do N}
\end{itemize}
\begin{itemize}
\item {Proveniência:(De \textunderscore pacova\textunderscore )}
\end{itemize}
O mesmo que \textunderscore bananal\textunderscore .
\section{Pacoviamente}
\begin{itemize}
\item {Grp. gram.:adv.}
\end{itemize}
De modo pacóvio; ingenuamente.
\section{Pacovice}
\begin{itemize}
\item {Grp. gram.:f.}
\end{itemize}
Qualidade ou acção de pacóvio.
\section{Pacóvio}
\begin{itemize}
\item {Grp. gram.:m.  e  adj.}
\end{itemize}
\begin{itemize}
\item {Utilização:Fam.}
\end{itemize}
Toleirão; estúpido; imbecil; idiota; simplório, parvo.
\section{Pactar}
\textunderscore v. t.\textunderscore  (e der.)
O mesmo que \textunderscore pactear\textunderscore . Cf. Th. Braga, \textunderscore Mod. Ideias\textunderscore , II, 154.
\section{Pactuário}
\begin{itemize}
\item {Grp. gram.:m.  e  adj.}
\end{itemize}
Aquelle que faz pacto; aquelle que pactua.
\section{Pactear}
\begin{itemize}
\item {Grp. gram.:v. t.  e  i.}
\end{itemize}
O mesmo que \textunderscore pactuar\textunderscore .
\section{Pacto}
\begin{itemize}
\item {Grp. gram.:m.}
\end{itemize}
\begin{itemize}
\item {Proveniência:(Lat. \textunderscore pactus\textunderscore )}
\end{itemize}
Ajuste entre duas ou mais pessôas; contrato; convenção.
Constituição.
\section{Pactolo}
\begin{itemize}
\item {Grp. gram.:m.}
\end{itemize}
Gênero de crustáceos decápodes.
\section{Pactuante}
\begin{itemize}
\item {Grp. gram.:adj.}
\end{itemize}
Que pactua; pactuário.
\section{Pactuar}
\begin{itemize}
\item {Grp. gram.:v. t.}
\end{itemize}
\begin{itemize}
\item {Grp. gram.:V. i.}
\end{itemize}
Ajustar; convencionar.
Fazer pacto com.
Fazer pacto; transigir.
\section{Pactuário}
\begin{itemize}
\item {Grp. gram.:m.}
\end{itemize}
\begin{itemize}
\item {Proveniência:(De \textunderscore patuar\textunderscore )}
\end{itemize}
Aquelle que pactua; aquelle que tem pacto ou contratos com outrem:«\textunderscore ...o sabio feiticeiro e pactuário do demonio.\textunderscore »Camillo, \textunderscore Noites de Insómn.\textunderscore , IV, 81.
\section{Pacu}
\begin{itemize}
\item {Grp. gram.:m.}
\end{itemize}
\begin{itemize}
\item {Utilização:Bras}
\end{itemize}
Nome de várias espécies de peixes de água doce.
Planta medicinal do Alto Amazonas.
(Do tupi)
\section{Pacuan}
\begin{itemize}
\item {Grp. gram.:m.}
\end{itemize}
\begin{itemize}
\item {Utilização:Bras}
\end{itemize}
Planta medicinal do Alto Amazonas.
\section{Paçuará}
\begin{itemize}
\item {Grp. gram.:m.}
\end{itemize}
\begin{itemize}
\item {Utilização:Bras}
\end{itemize}
Espécie de oiti.
\section{Pacuera}
\begin{itemize}
\item {Grp. gram.:m.}
\end{itemize}
\begin{itemize}
\item {Utilização:Bras. de Minas}
\end{itemize}
\textunderscore Bater a pacuera\textunderscore , morrer.
\section{Pacueza}
\begin{itemize}
\item {Grp. gram.:f.}
\end{itemize}
\begin{itemize}
\item {Utilização:Bras}
\end{itemize}
Fressura de boi, de carneiro ou de porco.
(Do tupi)
\section{Pacuguaçu}
\begin{itemize}
\item {Grp. gram.:m.}
\end{itemize}
\begin{itemize}
\item {Utilização:Bras}
\end{itemize}
Peixe fluvial.
\section{Pacunas}
\begin{itemize}
\item {Grp. gram.:m. pl.}
\end{itemize}
Indígenas do Norte do Brasil.
\section{Pacupeba}
\begin{itemize}
\item {Grp. gram.:f.}
\end{itemize}
\begin{itemize}
\item {Utilização:Bras}
\end{itemize}
Peixe fluvial.
\section{Pacuri}
\begin{itemize}
\item {Grp. gram.:m.}
\end{itemize}
Planta medicinal, de que os indígenas da Guiana inglesa extrahem uma substância cáustica.
\section{Pacutinga}
\begin{itemize}
\item {Grp. gram.:f.}
\end{itemize}
Espécie de pacu.
\section{Pada}
\begin{itemize}
\item {Grp. gram.:f.}
\end{itemize}
\begin{itemize}
\item {Utilização:Fig.}
\end{itemize}
Pequeno pão de farinha ordinária.
Pequena coisa, pequena porção.
(Contr. de \textunderscore panada\textunderscore , de \textunderscore panado\textunderscore )
\section{Padamarro}
\begin{itemize}
\item {Grp. gram.:m.}
\end{itemize}
\begin{itemize}
\item {Utilização:Prov.}
\end{itemize}
\begin{itemize}
\item {Utilização:trasm.}
\end{itemize}
\begin{itemize}
\item {Grp. gram.:Adj.}
\end{itemize}
\begin{itemize}
\item {Utilização:T. de Miranda}
\end{itemize}
Espécie de carvalho rasteiro.
Dorminhoco.
\section{Pàdar}
\begin{itemize}
\item {Grp. gram.:m.}
\end{itemize}
\begin{itemize}
\item {Utilização:Ant.}
\end{itemize}
O mesmo que \textunderscore paladar\textunderscore .
Véu palatino. Cf. Filinto, V, 195 e 200.
(Contr. de \textunderscore paladar\textunderscore )
\section{Padaria}
\begin{itemize}
\item {Grp. gram.:f.}
\end{itemize}
\begin{itemize}
\item {Proveniência:(De \textunderscore pada\textunderscore )}
\end{itemize}
Lugar, onde se vende ou fabríca pão.
\section{Padecedor}
\begin{itemize}
\item {Grp. gram.:m.  e  adj.}
\end{itemize}
O que padece.
\section{Padecente}
\begin{itemize}
\item {Grp. gram.:adj.}
\end{itemize}
\begin{itemize}
\item {Grp. gram.:M.  e  f.}
\end{itemize}
\begin{itemize}
\item {Grp. gram.:M.}
\end{itemize}
\begin{itemize}
\item {Utilização:Fam.}
\end{itemize}
Que padece.
Pessôa, que vai soffrer a pena de morte.
Aquelle que requesta vanmente uma mulhér.
\section{Padecer}
\begin{itemize}
\item {Grp. gram.:v. t.}
\end{itemize}
\begin{itemize}
\item {Utilização:Fig.}
\end{itemize}
\begin{itemize}
\item {Grp. gram.:V. i.}
\end{itemize}
\begin{itemize}
\item {Utilização:Des.}
\end{itemize}
\begin{itemize}
\item {Proveniência:(Do lat. hypoth. \textunderscore patescere\textunderscore )}
\end{itemize}
Supportar, soffrer; aturar.
Permittir.
Têr dôres phýsicas ou moraes.
Sêr justiçado.
\section{Padecimento}
\begin{itemize}
\item {Grp. gram.:m.}
\end{itemize}
Acto ou effeito de padecer.
Doença.
\section{Pàdeira}
\begin{itemize}
\item {Grp. gram.:f.}
\end{itemize}
Mulhér, que faz ou vende pão.
Espécie de uva tinta, também chamada \textunderscore nevoeira\textunderscore .
(Fem. de pàdeiro)
\section{Pàdeiro}
\begin{itemize}
\item {Grp. gram.:m.}
\end{itemize}
\begin{itemize}
\item {Proveniência:(De \textunderscore pada\textunderscore )}
\end{itemize}
Fabricante ou vendedor de pão.
Casta de uva, a mesma que \textunderscore nevoeira\textunderscore .
\section{Pàdejador}
\begin{itemize}
\item {Grp. gram.:m.  e  adj.}
\end{itemize}
O que padeja.
\section{Pàdejar}
\begin{itemize}
\item {Grp. gram.:v. t.}
\end{itemize}
Revolver com a pá.
(Por \textunderscore palejar\textunderscore , do lat. \textunderscore pala\textunderscore )
\section{Pàdejar}
\begin{itemize}
\item {Grp. gram.:v. i.}
\end{itemize}
\begin{itemize}
\item {Utilização:Prov.}
\end{itemize}
\begin{itemize}
\item {Proveniência:(De \textunderscore pada\textunderscore )}
\end{itemize}
Fabricar pão.
Bater massa de farinha em bacia, para a arredondar, antes de a deitar na pá que a mete no forno.
\section{Pàdejo}
\begin{itemize}
\item {Grp. gram.:m.}
\end{itemize}
Acto de pàdejar^1.
\section{Pàdejo}
\begin{itemize}
\item {Grp. gram.:m.}
\end{itemize}
\begin{itemize}
\item {Proveniência:(De \textunderscore pàdejar\textunderscore ^2)}
\end{itemize}
Mester de pàdeiro; padaria.
\section{Pàdela}
\begin{itemize}
\item {Grp. gram.:f.}
\end{itemize}
\begin{itemize}
\item {Utilização:Prov.}
\end{itemize}
Tacho largo de barro, com pequenos bordos, para usos culinários.
\section{Padeliças}
\begin{itemize}
\item {Grp. gram.:f. pl.}
\end{itemize}
\begin{itemize}
\item {Utilização:Ant.}
\end{itemize}
Pastos para animaes.
\section{Padernal}
\begin{itemize}
\item {Grp. gram.:adj.}
\end{itemize}
\begin{itemize}
\item {Utilização:Ant.}
\end{itemize}
O mesmo que \textunderscore paternal\textunderscore . Cf. Frei Fortun., \textunderscore Inéd.\textunderscore , 311.
\section{Padês}
\begin{itemize}
\item {Grp. gram.:m.}
\end{itemize}
O mesmo que \textunderscore pavês\textunderscore . Cf. \textunderscore Peregrinação\textunderscore , XXII.
\section{Padesada}
\begin{itemize}
\item {Grp. gram.:f.}
\end{itemize}
O mesmo que \textunderscore pavesada\textunderscore . Cf. Pero Lopes, \textunderscore Diár. da Naveg.\textunderscore 
\section{Padieira}
\begin{itemize}
\item {Grp. gram.:f.}
\end{itemize}
Vêrga de porta ou janela, especialmente se a verga é de madeira.
\section{Padina-pavão}
\begin{itemize}
\item {Grp. gram.:f.}
\end{itemize}
Planta submarina, cujas fôlhas são dispostas em fórma de leque.
\section{Pàdinha}
\begin{itemize}
\item {Grp. gram.:f.}
\end{itemize}
\begin{itemize}
\item {Utilização:Prov.}
\end{itemize}
\begin{itemize}
\item {Proveniência:(De \textunderscore pada\textunderscore )}
\end{itemize}
Espécie de bolo com açúcar e banha de porco; regueifa.
\section{Padinhamente}
\begin{itemize}
\item {Grp. gram.:adv.}
\end{itemize}
\begin{itemize}
\item {Utilização:Ant.}
\end{itemize}
Ás claras; publicamente.
(Cp. \textunderscore paladinamente\textunderscore )
\section{Pàdinhas}
\begin{itemize}
\item {Grp. gram.:m. pl.}
\end{itemize}
\begin{itemize}
\item {Utilização:Ant.}
\end{itemize}
\begin{itemize}
\item {Proveniência:(De \textunderscore pada\textunderscore ?)}
\end{itemize}
Uma das fórmas que se davam ao penteado.
\section{Padiola}
\begin{itemize}
\item {Grp. gram.:f.}
\end{itemize}
\begin{itemize}
\item {Utilização:Prov.}
\end{itemize}
\begin{itemize}
\item {Utilização:trasm.}
\end{itemize}
\begin{itemize}
\item {Proveniência:(Do lat. hyp. \textunderscore paleola\textunderscore )}
\end{itemize}
Tabuleiro quadrado, com um braço em cada ponta, e destinado para transportes.
Homem alto e magro.
\section{Padixá}
\begin{itemize}
\item {Grp. gram.:m.}
\end{itemize}
O imperador dos Turcos.
Cp. \textunderscore paxá\textunderscore .
(Do pers)
\section{Pado}
\begin{itemize}
\item {Grp. gram.:m.}
\end{itemize}
Espécie de azereiro, (\textunderscore prunus padus\textunderscore , Lin.), também conhecido por \textunderscore azereiro dos damnados\textunderscore . Cf. P. Coutinho, \textunderscore Flora de Port.\textunderscore , 307.
\section{Padovana}
\begin{itemize}
\item {Grp. gram.:m.}
\end{itemize}
\begin{itemize}
\item {Utilização:Ant.}
\end{itemize}
\begin{itemize}
\item {Proveniência:(It. \textunderscore padovana\textunderscore , de \textunderscore Padova\textunderscore , n. p.)}
\end{itemize}
O mesmo que \textunderscore pavana\textunderscore .
\section{Padral}
\begin{itemize}
\item {Grp. gram.:m.}
\end{itemize}
Casta de uva preta do Minho.
(Talvez de \textunderscore padre\textunderscore )
\section{Padralhada}
\begin{itemize}
\item {Grp. gram.:f.}
\end{itemize}
\begin{itemize}
\item {Utilização:Deprec.}
\end{itemize}
Grande porção de padres.
O clero.
\section{Padrão}
\begin{itemize}
\item {Grp. gram.:m.}
\end{itemize}
\begin{itemize}
\item {Utilização:Ant.}
\end{itemize}
\begin{itemize}
\item {Proveniência:(Do lat. \textunderscore patronus\textunderscore )}
\end{itemize}
Modêlo official das medidas e pesos legaes.
Modêlo.
Desenho de estamparia.
Título authêntico.
O mesmo que \textunderscore padroeiro\textunderscore .
\section{Padrão}
\begin{itemize}
\item {Grp. gram.:m.}
\end{itemize}
\begin{itemize}
\item {Utilização:Prov.}
\end{itemize}
\begin{itemize}
\item {Utilização:dur.}
\end{itemize}
Monumento de pedra, que os Portugueses erigiam em terras que iam descobrindo.
Monumento monolíthico.
Marco.
Arco, encimado de vários lavôres de madeira, cobertos de verdura e flôres, em uso nas festas e romarias.
(Por \textunderscore pedrão\textunderscore , de \textunderscore pedras\textunderscore )
\section{Padraria}
\begin{itemize}
\item {Grp. gram.:f.}
\end{itemize}
\begin{itemize}
\item {Proveniência:(De \textunderscore padre\textunderscore )}
\end{itemize}
O mesmo que \textunderscore padralhada\textunderscore . Cf. Eça, \textunderscore Padre Amaro\textunderscore , 195, 204 e 227.
\section{Padrar-se}
\begin{itemize}
\item {Grp. gram.:v. p.}
\end{itemize}
\begin{itemize}
\item {Utilização:Fam.}
\end{itemize}
Fazer-se padre.
\section{Padrasto}
\begin{itemize}
\item {Grp. gram.:m.}
\end{itemize}
\begin{itemize}
\item {Proveniência:(Do lat. \textunderscore patraster\textunderscore )}
\end{itemize}
Diz-se o indivíduo, em relação aos filhos de sua mulher, havidos de matrimónio anterior.
\section{Padrasto}
\begin{itemize}
\item {Grp. gram.:m.}
\end{itemize}
Monte sobranceiro em qualquer sítio. Cf. \textunderscore Luz e Calor\textunderscore , 17.
(Por \textunderscore pedrasto\textunderscore , de \textunderscore pedra\textunderscore )
\section{Padre}
\begin{itemize}
\item {Grp. gram.:m.}
\end{itemize}
\begin{itemize}
\item {Utilização:Ant.}
\end{itemize}
\begin{itemize}
\item {Grp. gram.:Pl.}
\end{itemize}
\begin{itemize}
\item {Proveniência:(Do lat. \textunderscore pater\textunderscore )}
\end{itemize}
Sacerdote secular; sacerdote; presbýtero.
O mesmo que \textunderscore pai\textunderscore :«\textunderscore ...herança que herdámos de padres e avós.\textunderscore »Castilho, \textunderscore Nazaré\textunderscore .
\textunderscore O santo padre\textunderscore , o Papa.
\textunderscore Padres conscriptos\textunderscore , os antigos senadores romanos.
\textunderscore Santos padres\textunderscore , os antigos e principaes escritores ecclesiásticos, que defenderam e explicaram a doutrina christan.
\section{Padreação}
\begin{itemize}
\item {Grp. gram.:f.}
\end{itemize}
Acto de padrear.
\section{Padreador}
\begin{itemize}
\item {Grp. gram.:m.  e  adj.}
\end{itemize}
\begin{itemize}
\item {Proveniência:(De \textunderscore padrear\textunderscore )}
\end{itemize}
Animal que padreia.
\section{Padrear}
\begin{itemize}
\item {Grp. gram.:v. i.}
\end{itemize}
\begin{itemize}
\item {Proveniência:(Do port. ant. \textunderscore padre\textunderscore , pai)}
\end{itemize}
Procriar, reproduzir-se, (falando-se do cavallo). Cf. Castilho, \textunderscore Geórgicas\textunderscore , 153.
\section{Padreca}
\begin{itemize}
\item {Grp. gram.:m.}
\end{itemize}
Padre que tem pouco mérito ou pequena estatura.
\section{Padreco}
\begin{itemize}
\item {Grp. gram.:m.}
\end{itemize}
\begin{itemize}
\item {Utilização:Deprec.}
\end{itemize}
Padre que tem pouco mérito ou pequena estatura.
\section{Padre-cura}
\begin{itemize}
\item {Grp. gram.:M.}
\end{itemize}
Espécie de jôgo popular.
\section{Padre-mestre}
\begin{itemize}
\item {Grp. gram.:m.}
\end{itemize}
\begin{itemize}
\item {Utilização:Fig.}
\end{itemize}
Designação ou tratamento, que se dá ao sacerdote, que é professor.
Homem sabedor; sabichão.
\section{Padre-nosso}
\begin{itemize}
\item {Grp. gram.:m.}
\end{itemize}
\begin{itemize}
\item {Proveniência:(Do lat. \textunderscore pater\textunderscore  + \textunderscore noster\textunderscore )}
\end{itemize}
Oração dominical, ensinada por Christo aos seus discípulos.
Cada uma das contas maiores de um rosário.
\section{Padre-santo}
\begin{itemize}
\item {Grp. gram.:m.}
\end{itemize}
O Papa; o Pontífice romano.
\section{Padresco}
\begin{itemize}
\item {fónica:drês}
\end{itemize}
\begin{itemize}
\item {Grp. gram.:adj.}
\end{itemize}
\begin{itemize}
\item {Utilização:Deprec.}
\end{itemize}
Relativo a padre; próprio de padre. Cf. Camillo, \textunderscore Noites de Insómn.\textunderscore , VII, 27.
\section{Padrice}
\begin{itemize}
\item {Grp. gram.:f.}
\end{itemize}
\begin{itemize}
\item {Utilização:Deprec.}
\end{itemize}
Qualidade ou acto de padre.
\section{Padrinho}
\begin{itemize}
\item {Grp. gram.:m.}
\end{itemize}
\begin{itemize}
\item {Utilização:Fig.}
\end{itemize}
\begin{itemize}
\item {Proveniência:(Do lat. \textunderscore patrinus\textunderscore )}
\end{itemize}
Testemunha de baptismo, casamento ou duello.
O que acompanha o doutorando, quando êste recebe capello.
Protector; patrono.
\section{Padroado}
\begin{itemize}
\item {Grp. gram.:m.}
\end{itemize}
\begin{itemize}
\item {Proveniência:(Lat. \textunderscore patronatus\textunderscore )}
\end{itemize}
Direito de protector, adquirido por quem fundou ou dotou uma igreja.
Direito de conferir benefícios ecclesiásticos: \textunderscore o padroado português da Índia\textunderscore .
\section{Padroeira}
\textunderscore fem.\textunderscore  de \textunderscore padroeiro\textunderscore .
\section{Padroeiro}
\begin{itemize}
\item {Grp. gram.:m.  e  adj.}
\end{itemize}
\begin{itemize}
\item {Proveniência:(Do lat. \textunderscore patronus\textunderscore )}
\end{itemize}
O que tem o direito do padroado.
Protector.
Defensor.
O que fêz doações a um mosteiro, com encargos.
\section{Padrom}
\begin{itemize}
\item {Grp. gram.:m.}
\end{itemize}
\begin{itemize}
\item {Utilização:Ant.}
\end{itemize}
O mesmo que \textunderscore padroeiro\textunderscore .
\section{Padronádiga}
\begin{itemize}
\item {Grp. gram.:f.}
\end{itemize}
\begin{itemize}
\item {Utilização:Ant.}
\end{itemize}
Herança paterna; património.
(Refl. do lat. \textunderscore patronus\textunderscore )
\section{Padroom}
\begin{itemize}
\item {Grp. gram.:m.}
\end{itemize}
\begin{itemize}
\item {Utilização:Ant.}
\end{itemize}
Grande marco de pedra; padrão.
\section{Padu}
\begin{itemize}
\item {Grp. gram.:m.}
\end{itemize}
Arbusto de Peru e do norte do Brasil, de cujas fôlhas se faz uma infusão semelhante ao chá.
\section{Paduana}
\begin{itemize}
\item {Grp. gram.:f.}
\end{itemize}
Ave gallinácea. Cf. Dom. Vieira, \textunderscore Diccion.\textunderscore , vb. \textunderscore Brahma\textunderscore .
O mesmo que \textunderscore padovana\textunderscore .
\section{Paduano}
\begin{itemize}
\item {Grp. gram.:adj.}
\end{itemize}
\begin{itemize}
\item {Grp. gram.:M.}
\end{itemize}
Relativo a Pádua.
Habitante de Pádua.
\section{Paduca}
\begin{itemize}
\item {Grp. gram.:m.}
\end{itemize}
Língua falada no México.
\section{Padumar}
\begin{itemize}
\item {Grp. gram.:v. i.}
\end{itemize}
\begin{itemize}
\item {Utilização:T. de Lanhoso}
\end{itemize}
Aguentar as consequências.
\section{Páfio}
\begin{itemize}
\item {Grp. gram.:adj.}
\end{itemize}
Relativo á cidade de Paphos:«\textunderscore ...as páfias rosas...\textunderscore ». Castilho, \textunderscore Fastos\textunderscore , II, 153.
(Cp. lat. \textunderscore paphius\textunderscore )
\section{Pafo}
\begin{itemize}
\item {Grp. gram.:m.}
\end{itemize}
\begin{itemize}
\item {Utilização:Ant.}
\end{itemize}
O mesmo que \textunderscore parágrapho\textunderscore .
\section{Pafuás}
\begin{itemize}
\item {Grp. gram.:m. pl.}
\end{itemize}
Antigo povo da Ásia. Cf. \textunderscore Peregrinação\textunderscore , XLI.
\section{Paga}
\begin{itemize}
\item {Grp. gram.:f.}
\end{itemize}
Acto ou effeito de pagar.
Aquillo com que se paga ou remunera.
\section{Pagado}
\begin{itemize}
\item {Grp. gram.:adj.}
\end{itemize}
\begin{itemize}
\item {Utilização:Ant.}
\end{itemize}
Sossegado; pacífico.
(Cp. \textunderscore pacato\textunderscore )
\section{Pagadoiro}
\begin{itemize}
\item {Grp. gram.:adj.}
\end{itemize}
\begin{itemize}
\item {Utilização:Ant.}
\end{itemize}
O mesmo que \textunderscore pagável\textunderscore .
\section{Pagador}
\begin{itemize}
\item {Grp. gram.:m.  e  adj.}
\end{itemize}
O que paga; o que faz pagamentos.
\section{Pagadoria}
\begin{itemize}
\item {Grp. gram.:f.}
\end{itemize}
\begin{itemize}
\item {Proveniência:(De \textunderscore pagador\textunderscore )}
\end{itemize}
Lugar ou Repartição pública, em que se fazem pagamentos.
\section{Pagadouro}
\begin{itemize}
\item {Grp. gram.:adj.}
\end{itemize}
\begin{itemize}
\item {Utilização:Ant.}
\end{itemize}
O mesmo que \textunderscore pagável\textunderscore .
\section{Pagaga}
\begin{itemize}
\item {Grp. gram.:f.}
\end{itemize}
Árvore da Guiana, empregada em construcções.
\section{Pagaia}
\begin{itemize}
\item {Grp. gram.:f.}
\end{itemize}
\begin{itemize}
\item {Utilização:T. de Guiné e Moçambique}
\end{itemize}
Remo de pá larga e haste curta.
\section{Pagamento}
\begin{itemize}
\item {Grp. gram.:m.}
\end{itemize}
\begin{itemize}
\item {Proveniência:(De \textunderscore pagar\textunderscore )}
\end{itemize}
O mesmo que \textunderscore paga\textunderscore .
Prestação: \textunderscore comprou a máquina de costura a pagamentos\textunderscore .
\section{Paganaes}
\begin{itemize}
\item {Grp. gram.:f. pl.}
\end{itemize}
O mesmo ou melhor que \textunderscore paganálias\textunderscore .
\section{Paganais}
\begin{itemize}
\item {Grp. gram.:f. pl.}
\end{itemize}
O mesmo ou melhor que \textunderscore paganálias\textunderscore .
\section{Paganal}
\begin{itemize}
\item {Grp. gram.:adj.}
\end{itemize}
\begin{itemize}
\item {Grp. gram.:F. pl.}
\end{itemize}
\begin{itemize}
\item {Proveniência:(Lat. \textunderscore paganalis\textunderscore )}
\end{itemize}
Relativo a Pagãos.
Relativo aos aldeões.
Festas aldeans que, entre os Romanos, se celebravam em Janeiro; paganálias.
\section{Paganálias}
\begin{itemize}
\item {Grp. gram.:f. pl.}
\end{itemize}
Antigas festas em honra de Ceres, também chamadas paganaes.
(Cp. \textunderscore paganal\textunderscore )
\section{Paganel}
\begin{itemize}
\item {Grp. gram.:m.}
\end{itemize}
\begin{itemize}
\item {Proveniência:(It. \textunderscore paganello\textunderscore )}
\end{itemize}
Peixe do Mediterrâneo.
\section{Paganismo}
\begin{itemize}
\item {Grp. gram.:m.}
\end{itemize}
\begin{itemize}
\item {Proveniência:(Do lat. \textunderscore paganus\textunderscore )}
\end{itemize}
Religião dos Pagãos.
Os Pagãos.
Polytheísmo; idolatria.
\section{Paganização}
\begin{itemize}
\item {Grp. gram.:f.}
\end{itemize}
Acto de paganizar.
\section{Paganizador}
\begin{itemize}
\item {Grp. gram.:adj.}
\end{itemize}
Que paganiza.
\section{Paganizar}
\begin{itemize}
\item {Grp. gram.:v. t.}
\end{itemize}
\begin{itemize}
\item {Grp. gram.:V. i.}
\end{itemize}
\begin{itemize}
\item {Proveniência:(Lat. \textunderscore paganizare\textunderscore )}
\end{itemize}
Tornar pagão; deschristianizar.
Pensar ou proceder como pagão.
\section{Pagano}
\begin{itemize}
\item {Grp. gram.:m.  e  adj.}
\end{itemize}
\begin{itemize}
\item {Utilização:Ant.}
\end{itemize}
O mesmo que \textunderscore pagão\textunderscore .
\section{Pagante}
\begin{itemize}
\item {Grp. gram.:m. ,  f.  e  adj.}
\end{itemize}
\begin{itemize}
\item {Proveniência:(Lat. \textunderscore pacans\textunderscore )}
\end{itemize}
Pessôa que paga.
\section{Pagão}
\begin{itemize}
\item {Grp. gram.:adj.}
\end{itemize}
\begin{itemize}
\item {Utilização:Pop.}
\end{itemize}
\begin{itemize}
\item {Grp. gram.:M.}
\end{itemize}
\begin{itemize}
\item {Proveniência:(Lat. \textunderscore paganus\textunderscore )}
\end{itemize}
Relativo ao paganismo.
Que segue o polytheísmo.
Mahometano.
Herético.
Sectário do polytheismo; idólatra.
\section{Pagapato}
\begin{itemize}
\item {Grp. gram.:m.}
\end{itemize}
Gênero de plantas myrtáceas.
\section{Pagar}
\begin{itemize}
\item {Grp. gram.:v. t.}
\end{itemize}
\begin{itemize}
\item {Grp. gram.:V. p.}
\end{itemize}
\begin{itemize}
\item {Utilização:Ant.}
\end{itemize}
\begin{itemize}
\item {Proveniência:(Do lat. \textunderscore pacare\textunderscore )}
\end{itemize}
Restituír na mesma espécie ou em espécie differente (aquillo que se deve).
Retribuír, remunerar: \textunderscore pagar serviços\textunderscore .
Satisfazer o preço de: \textunderscore pagar umas botas\textunderscore .
Recompensar de qualquer fórma, bem ou mal: \textunderscore pagar affectos com ingratidão\textunderscore .
Adquirir com sacrifício.
Expiar: \textunderscore pagou o mal que fez\textunderscore .
Estar sujeito á.
Indemnizar-se; descontar, do que há de entregar, a parte que lhe é devida.
Gostar (De \textunderscore alguma\textunderscore  coisa). Cf. \textunderscore Port. Mon. Hist.\textunderscore , 314 e 343; Pant. de Aveiro, \textunderscore Itiner.\textunderscore , 295 v.^o, (3.^a ed.).
\section{Pae}
\begin{itemize}
\item {Grp. gram.:m.}
\end{itemize}
\begin{itemize}
\item {Utilização:Fig.}
\end{itemize}
\begin{itemize}
\item {Utilização:Gír.}
\end{itemize}
\begin{itemize}
\item {Proveniência:(Do lat. \textunderscore pater\textunderscore )}
\end{itemize}
Homem, que deu o sêr a um ou mais indivíduos.
Antepassado.
Ascendente.
Animal, que deu o sêr a outro.
Primeira pessoa da Santíssima Trindade.
Tratamento affectuoso, que antecede o nome de alguns negros de certa idade: \textunderscore anda cá, ó pae João\textunderscore .
\textunderscore Pae de família\textunderscore , ou \textunderscore de famílias\textunderscore , chefe da casa, com mulhér e filhos.
Protector.
Fundador de uma instituição.
Capitão de ladrões.
\section{Pagara}
\begin{itemize}
\item {Grp. gram.:m.}
\end{itemize}
\begin{itemize}
\item {Utilização:Bras. do S}
\end{itemize}
Bailado campestre, espécie de fandango.
\section{Pagarote}
\begin{itemize}
\item {Grp. gram.:m.}
\end{itemize}
\begin{itemize}
\item {Utilização:Prov.}
\end{itemize}
\begin{itemize}
\item {Utilização:trasm.}
\end{itemize}
\begin{itemize}
\item {Proveniência:(De \textunderscore pagar\textunderscore )}
\end{itemize}
Contribuição.
Dívida.
\section{Pagastinas}
\begin{itemize}
\item {Grp. gram.:f. pl.}
\end{itemize}
\begin{itemize}
\item {Utilização:Prov.}
\end{itemize}
\begin{itemize}
\item {Utilização:trasm.}
\end{itemize}
\begin{itemize}
\item {Proveniência:(De \textunderscore pagar\textunderscore )}
\end{itemize}
Pequenas e differentes dividas, que há a pagar aquém e além.
\section{Pagável}
\begin{itemize}
\item {Grp. gram.:adj.}
\end{itemize}
Que se póde ou se deve pagar.
\section{Page}
\begin{itemize}
\item {Grp. gram.:m.}
\end{itemize}
O mesmo que \textunderscore pagem\textunderscore :«\textunderscore ...trazíam dous pages trás si...\textunderscore »Barros, \textunderscore Déc.\textunderscore  I, l. III., c. I. Cf. Filinto, \textunderscore D. Man.\textunderscore , II, 222 e III, 257.
\section{Pagé}
\begin{itemize}
\item {Grp. gram.:m.}
\end{itemize}
\begin{itemize}
\item {Utilização:Bras. do N}
\end{itemize}
Sacerdote curandeiro, entre os aborígenes; feiticeiro.
(Do tupi-guar.)
\section{Pageada}
\begin{itemize}
\item {Grp. gram.:f.}
\end{itemize}
A classe das pagens. Cf. \textunderscore Eufrosina\textunderscore , 23.
\section{Pagear}
\begin{itemize}
\item {Grp. gram.:v. t.}
\end{itemize}
O mesmo que \textunderscore apagear\textunderscore .
\section{Pagel}
\begin{itemize}
\item {Grp. gram.:m.}
\end{itemize}
O mesmo ou melhor que \textunderscore paguel\textunderscore . Cf. \textunderscore Peregrinação\textunderscore , VIII.
\section{Pagela}
\begin{itemize}
\item {Grp. gram.:f.}
\end{itemize}
\begin{itemize}
\item {Utilização:Des.}
\end{itemize}
\begin{itemize}
\item {Proveniência:(Lat. \textunderscore pagella\textunderscore )}
\end{itemize}
Parcela.
Prestação.
Pequena página.
\section{Pagelança}
\begin{itemize}
\item {Grp. gram.:f.}
\end{itemize}
\begin{itemize}
\item {Utilização:Bras. do N}
\end{itemize}
Acto de pagé.
Feitiçaria.
\section{Pagella}
\begin{itemize}
\item {Grp. gram.:f.}
\end{itemize}
\begin{itemize}
\item {Utilização:Des.}
\end{itemize}
\begin{itemize}
\item {Proveniência:(Lat. \textunderscore pagella\textunderscore )}
\end{itemize}
Parcella.
Prestação.
Pequena página.
\section{Pagem}
\begin{itemize}
\item {Grp. gram.:m.}
\end{itemize}
\begin{itemize}
\item {Proveniência:(It. \textunderscore paggio\textunderscore )}
\end{itemize}
Mancebo, que acompanhava o Rei ou pessôa nobre e lhe levava as armas quando ia para a guerra.
Cavalleiro, que nas toiradas transmitte ordens; neto.
Marinheiro, que trata da limpeza em navios de guerra.
\section{Página}
\begin{itemize}
\item {Grp. gram.:f.}
\end{itemize}
\begin{itemize}
\item {Utilização:Ext.}
\end{itemize}
\begin{itemize}
\item {Utilização:Fig.}
\end{itemize}
\begin{itemize}
\item {Utilização:Bot.}
\end{itemize}
\begin{itemize}
\item {Proveniência:(Lat. \textunderscore pagina\textunderscore )}
\end{itemize}
Cada um dos lados de uma fôlha de papel, de pergaminho, etc.
Aquillo que está escrito nesse lado.
Trecho: \textunderscore sabe de cór muitas páginas dos Lusíadas\textunderscore .
Período notável numa história ou numa biographia: \textunderscore a revolução de 1820 é uma página soberba da nossa história\textunderscore .
Cada uma das duas superfícies do limbo de uma fôlha.
\section{Paginação}
\begin{itemize}
\item {Grp. gram.:f.}
\end{itemize}
Acto de paginar.
Ordem das páginas de um volume escrito.
\section{Paginador}
\begin{itemize}
\item {Grp. gram.:m.}
\end{itemize}
\begin{itemize}
\item {Utilização:Typ.}
\end{itemize}
O encarregado de paginar um jornal ou uma fôlha.
\section{Paginar}
\begin{itemize}
\item {Grp. gram.:v. t.}
\end{itemize}
\begin{itemize}
\item {Grp. gram.:V. i.}
\end{itemize}
Numerar ordenadamente as páginas de.
Reunir a composição typográphica, para formar as páginas.
\section{Pago}
\begin{itemize}
\item {Grp. gram.:adj.}
\end{itemize}
\begin{itemize}
\item {Utilização:Fig.}
\end{itemize}
\begin{itemize}
\item {Grp. gram.:M.}
\end{itemize}
\begin{itemize}
\item {Proveniência:(De \textunderscore pagar\textunderscore )}
\end{itemize}
Entregue para pagamento: \textunderscore dinheiro pago\textunderscore .
Que se vingou.
O mesmo que \textunderscore paga\textunderscore : \textunderscore teve mau pago\textunderscore .
\section{Pago}
\begin{itemize}
\item {Grp. gram.:m.}
\end{itemize}
\begin{itemize}
\item {Utilização:P. us.}
\end{itemize}
\begin{itemize}
\item {Proveniência:(Lat. \textunderscore pagus\textunderscore )}
\end{itemize}
Pequena povoação; casal:«\textunderscore do seu pago alcandorado...\textunderscore »C. Neto, \textunderscore Saldunes\textunderscore .
\section{Pagode}
\begin{itemize}
\item {Grp. gram.:m.}
\end{itemize}
\begin{itemize}
\item {Utilização:chul.}
\end{itemize}
\begin{itemize}
\item {Utilização:Fig.}
\end{itemize}
Espécie, de templo pagão, entre alguns povos da Ásia.
Ídolo, quo se adora nesse templo.
Divertimento, pândega; bambochata.
Antiga moéda de oiro, na Índia portuguesa, que valia 9 a 10 tangas.
\section{Pagodear}
\begin{itemize}
\item {Grp. gram.:v. i.}
\end{itemize}
\begin{itemize}
\item {Proveniência:(De \textunderscore pagode\textunderscore )}
\end{itemize}
Levar vida de pândego ou estróina; pandegar.
\section{Pagodeira}
\begin{itemize}
\item {Grp. gram.:f.}
\end{itemize}
\begin{itemize}
\item {Utilização:Chul.}
\end{itemize}
Estroinice; divertimento; pagode.
\section{Pagodeiro}
\begin{itemize}
\item {Grp. gram.:adj.}
\end{itemize}
\begin{itemize}
\item {Utilização:Chul.}
\end{itemize}
\begin{itemize}
\item {Proveniência:(De \textunderscore pagode\textunderscore )}
\end{itemize}
Pândego; estróina.
\section{Pagodice}
\begin{itemize}
\item {Grp. gram.:f.}
\end{itemize}
O mesmo que \textunderscore pagodeira\textunderscore .
\section{Pagodista}
\begin{itemize}
\item {Grp. gram.:m.  e  f.}
\end{itemize}
\begin{itemize}
\item {Utilização:Chul.}
\end{itemize}
Pessôa estróina, que gosta de pagodes ou pândegas. Cf. Camillo, \textunderscore Corja\textunderscore , 315.
\section{Pagos}
\begin{itemize}
\item {Grp. gram.:m. pl.}
\end{itemize}
\begin{itemize}
\item {Utilização:Bras}
\end{itemize}
Habitação.
(V. \textunderscore pago\textunderscore ^2)
\section{Pagotém}
\begin{itemize}
\item {Grp. gram.:m.}
\end{itemize}
Espécie de turbante indiano. Cf. Th. Ribeiro, \textunderscore Jornadas\textunderscore , II, 100.
\section{Paguel}
\begin{itemize}
\item {Grp. gram.:m.}
\end{itemize}
Antiga embarcação indiana.
\section{Paguér}
\begin{itemize}
\item {Grp. gram.:m.}
\end{itemize}
O mesmo que \textunderscore paguel\textunderscore .
\section{Paguilha}
\begin{itemize}
\item {Grp. gram.:m.  e  f.}
\end{itemize}
O mesmo que \textunderscore pagante\textunderscore .
\section{Paguim}
\begin{itemize}
\item {Grp. gram.:m.}
\end{itemize}
Ave palmípede, que tem a particularidade de caminhar erecta, como o homem.
\section{Paguma}
\begin{itemize}
\item {Grp. gram.:f.}
\end{itemize}
Mammífero carnívoro de Samatra.
\section{Paguro}
\begin{itemize}
\item {Grp. gram.:m.}
\end{itemize}
\begin{itemize}
\item {Proveniência:(Lat. \textunderscore pagurus\textunderscore )}
\end{itemize}
Gênero de crustáceos decápodes.
\section{Pahô}
\begin{itemize}
\item {Grp. gram.:m.}
\end{itemize}
\begin{itemize}
\item {Utilização:Bras}
\end{itemize}
Ave, do tamanho de pomba, negra, mas com o peito vermelho.
\section{Pai}
\begin{itemize}
\item {Grp. gram.:m.}
\end{itemize}
\begin{itemize}
\item {Utilização:Fig.}
\end{itemize}
\begin{itemize}
\item {Utilização:Gír.}
\end{itemize}
\begin{itemize}
\item {Proveniência:(Do lat. \textunderscore pater\textunderscore )}
\end{itemize}
Homem, que deu o sêr a um ou mais indivíduos.
Antepassado.
Ascendente.
Animal, que deu o sêr a outro.
Primeira pessoa da Santíssima Trindade.
Tratamento affectuoso, que antecede o nome de alguns negros de certa idade: \textunderscore anda cá, ó pai João\textunderscore .
\textunderscore Pai de família\textunderscore , ou \textunderscore de famílias\textunderscore , chefe da casa, com mulhér e filhos.
Protector.
Fundador de uma instituição.
Capitão de ladrões.
\section{Paí}
\begin{itemize}
\item {Grp. gram.:m.}
\end{itemize}
O mesmo que \textunderscore cacique\textunderscore .
\section{Paiá}
\begin{itemize}
\item {Grp. gram.:m.}
\end{itemize}
Medida de capacidade em Damão, equivalente a 3 póris das ilhas de Gôa.
\section{Paiabas}
\begin{itemize}
\item {Grp. gram.:m. pl.}
\end{itemize}
\begin{itemize}
\item {Utilização:Bras}
\end{itemize}
Tríbo de aborígenes, que habitou no Pará.
\section{Paiacus}
\begin{itemize}
\item {Grp. gram.:m. pl.}
\end{itemize}
Antiga tríbo de Índios do Brasil, fundida hoje com a população de Porto-Alegre.
\section{Paiaguás}
\begin{itemize}
\item {Grp. gram.:m. pl.}
\end{itemize}
Antiga nação de Índios do Brasil, nas margens do Paraguai, em Mato-Grosso.
\section{Paianas}
\begin{itemize}
\item {Grp. gram.:m. pl.}
\end{itemize}
Indígenas do norte do Brasil.
\section{Paião}
\begin{itemize}
\item {Grp. gram.:m.}
\end{itemize}
\begin{itemize}
\item {Proveniência:(De \textunderscore paio\textunderscore ?)}
\end{itemize}
Peixe da costa de Portugal.
\section{Paiauaru}
\begin{itemize}
\item {Grp. gram.:m.}
\end{itemize}
\begin{itemize}
\item {Utilização:Bras}
\end{itemize}
Vinho de frutas, fabricado pelos Índios.
\section{Pai-avô}
\begin{itemize}
\item {Grp. gram.:m.}
\end{itemize}
\begin{itemize}
\item {Utilização:T. de Turquel}
\end{itemize}
\begin{itemize}
\item {Utilização:Bras. do N}
\end{itemize}
Um pobre homem; homem simplório.
Nome de um pássaro cinzento, cujo grito imita o seu nome.
\section{Paicogés}
\begin{itemize}
\item {Grp. gram.:m. pl.}
\end{itemize}
Tríbo de aborígenes do Pará.
\section{Pai-de-chiqueiro}
\begin{itemize}
\item {Grp. gram.:m.}
\end{itemize}
\begin{itemize}
\item {Utilização:Bras. do N}
\end{itemize}
Chibato; macho da cabra.
\section{Pai-de-égua}
\begin{itemize}
\item {Grp. gram.:m.}
\end{itemize}
\begin{itemize}
\item {Utilização:Bras. do N}
\end{itemize}
Cavallo de padreação.
\section{Pai-de-malhada}
\begin{itemize}
\item {Grp. gram.:f.}
\end{itemize}
\begin{itemize}
\item {Utilização:Bras. do N}
\end{itemize}
Marruá, que chefia uma maloca de gado vacum.
\section{Pai-de-todos}
\begin{itemize}
\item {Grp. gram.:m.}
\end{itemize}
\begin{itemize}
\item {Utilização:Pop.}
\end{itemize}
O maior dedo das mãos.
\section{Pai-de-velhacos}
\begin{itemize}
\item {Grp. gram.:m.}
\end{itemize}
Magistrado que, em Lisbôa e no Pôrto, tinha a seu cargo olhar pelos rapazes vadios, e proporcionar-lhes amos ou offícios.
\section{Pai-dos-meninos}
\begin{itemize}
\item {Grp. gram.:m.}
\end{itemize}
Official público, que na cidade do Pôrto tinha que olhar pelos enjeitados e levá-los ao juiz dos órfãos.
\section{Pai-joão}
\begin{itemize}
\item {Grp. gram.:m.}
\end{itemize}
\begin{itemize}
\item {Utilização:Bras. do N}
\end{itemize}
A parte traseira da rês.
\section{Pailão}
\begin{itemize}
\item {Grp. gram.:m.}
\end{itemize}
\begin{itemize}
\item {Utilização:Prov.}
\end{itemize}
\begin{itemize}
\item {Utilização:alg.}
\end{itemize}
Paspalhão.
\section{Pailona}
\begin{itemize}
\item {Grp. gram.:f.}
\end{itemize}
Fêmea do peixe chamado carocho.
\section{Pai-mané}
\begin{itemize}
\item {Grp. gram.:m.}
\end{itemize}
\begin{itemize}
\item {Utilização:Bras. do N}
\end{itemize}
Tolo.
Ignorante.
\section{Paina}
\begin{itemize}
\item {Grp. gram.:f.}
\end{itemize}
Espécie de algodão do Brasil.
\section{Painça}
\begin{itemize}
\item {Grp. gram.:adj.}
\end{itemize}
Diz-se da palha e da farinha de painço.
\section{Painçada}
\begin{itemize}
\item {Grp. gram.:f.}
\end{itemize}
Porção de painço. Cf. Camillo, \textunderscore Mar. da Fonte\textunderscore , 139.
\section{Painço}
\begin{itemize}
\item {Grp. gram.:m.}
\end{itemize}
\begin{itemize}
\item {Grp. gram.:Pl.}
\end{itemize}
\begin{itemize}
\item {Utilização:T. de Resende}
\end{itemize}
\begin{itemize}
\item {Proveniência:(Do lat. \textunderscore panicium\textunderscore )}
\end{itemize}
Planta gramínea, (\textunderscore panicium italicum\textunderscore ).
Grão dessa planta, também chamado milho miúdo.
O mesmo que [[carolos|carolo]].
\section{Paineira}
\begin{itemize}
\item {Grp. gram.:f.}
\end{itemize}
\begin{itemize}
\item {Utilização:Bras}
\end{itemize}
Árvore, em cujas cápsulas se contém uma espécie de lan, com que se enchem colchões.
Talvez o mesmo que \textunderscore paina\textunderscore .
\section{Painel}
\begin{itemize}
\item {Grp. gram.:m.}
\end{itemize}
\begin{itemize}
\item {Utilização:Serralh.}
\end{itemize}
\begin{itemize}
\item {Utilização:Fig.}
\end{itemize}
Quadro sobre tela ou pano.
Pintura.
Retábulo.
Almofada de portas ou janelas.
Relêvo, em fórma de moldura, sôbre um plano, em obra de architectura.
Conjunto dos panos, que formam a vela do navio.
Chapa exterior das fechaduras.
Espectáculo.
(Por \textunderscore panel\textunderscore , de \textunderscore pano\textunderscore )
\section{Paínho}
\begin{itemize}
\item {Grp. gram.:m.}
\end{itemize}
Pequeno pássaro aquático das costas de Portugal.
\section{Pai-nobre}
\begin{itemize}
\item {Grp. gram.:m.}
\end{itemize}
Ator, encarregado do papel de pai, em tragédia ou alta comédia.
\section{Paio}
\begin{itemize}
\item {Grp. gram.:m.}
\end{itemize}
Carne de porco ensacada em tripa de intestino grosso.
\section{Paiol}
\begin{itemize}
\item {Grp. gram.:m.}
\end{itemize}
\begin{itemize}
\item {Utilização:Bras. do N}
\end{itemize}
\begin{itemize}
\item {Utilização:Bras. de Minas e S. Paulo}
\end{itemize}
\begin{itemize}
\item {Utilização:Gír.}
\end{itemize}
Lugar, em que se guarda pólvora e outros petrechos de guerra.
Grande compartimento para arrecadações, num navio.
Casa para arrecadação dos gêneros da grande lavoira.
Tulha de milho.
Estômago.
(Cp. cast. \textunderscore pañol\textunderscore )
\section{Paioleiro}
\begin{itemize}
\item {Grp. gram.:m.}
\end{itemize}
Guarda do paiol.
\section{Paiorra}
\begin{itemize}
\item {fónica:ó}
\end{itemize}
\begin{itemize}
\item {Grp. gram.:f.}
\end{itemize}
\begin{itemize}
\item {Utilização:Chul.}
\end{itemize}
Mulhér baixa e gorda.
\section{Pairar}
\begin{itemize}
\item {Grp. gram.:v. t.}
\end{itemize}
\begin{itemize}
\item {Utilização:Fig.}
\end{itemize}
\begin{itemize}
\item {Grp. gram.:V. t.}
\end{itemize}
\begin{itemize}
\item {Utilização:Ant.}
\end{itemize}
\begin{itemize}
\item {Proveniência:(De \textunderscore parar\textunderscore ?)}
\end{itemize}
Estar á capa (um navio).
Bordejar.
Adejar, sem sair do mesmo ponto.
Voar lentamente.
Estar imminente: \textunderscore pairava uma trovoada\textunderscore .
Hesitar.
Suster; tolerar.
Adiar. Cf. Eufrosina, 251.
\section{Pairo}
\begin{itemize}
\item {Grp. gram.:m.}
\end{itemize}
Acto de pairar. Cf. Garrett, \textunderscore D. Branca\textunderscore , 129.
\section{País}
\begin{itemize}
\item {Grp. gram.:m.}
\end{itemize}
\begin{itemize}
\item {Proveniência:(Fr. \textunderscore pays\textunderscore , do lat. hyp. \textunderscore pagensis\textunderscore , de \textunderscore pagus\textunderscore )}
\end{itemize}
Região; nação.
Pátria.
Paisagem.
Clima.
\section{Paisagem}
\begin{itemize}
\item {fónica:pa-i}
\end{itemize}
\begin{itemize}
\item {Grp. gram.:f.}
\end{itemize}
\begin{itemize}
\item {Proveniência:(De \textunderscore país\textunderscore )}
\end{itemize}
Espaço de território, que se abrange num lance de vista.
Gênero de pintura,
\textunderscore ou\textunderscore  de literatura, representando ou descrevendo o campo ou lugares campestres.
Quadro, que representa esses lugares.
Trecho literário, que descreve perspectivas ou scenas campestres.
\section{Paisagista}
\begin{itemize}
\item {fónica:pa-i}
\end{itemize}
\begin{itemize}
\item {Grp. gram.:m.  e  f.}
\end{itemize}
Pessôa, que pinta paisagens.
Literato, que descreve o campo ou scenas da vida rústica.
\section{Paisaísta}
\begin{itemize}
\item {Grp. gram.:m.}
\end{itemize}
\begin{itemize}
\item {Proveniência:(De \textunderscore país\textunderscore )}
\end{itemize}
Aquelle que descreve uma região ou certas localidades:«\textunderscore ...paisaísta esmerado, descrevendo as avenidas, do palácio...\textunderscore »Palmeirim, \textunderscore Portugal e seus detract.\textunderscore , 67.
\section{Paisanada}
\begin{itemize}
\item {Grp. gram.:f.}
\end{itemize}
\begin{itemize}
\item {Utilização:Deprec.}
\end{itemize}
Grupo de paisanos; os paisanos:«\textunderscore ...essa paisanada que tudo lo manda.\textunderscore »Garrett, \textunderscore Arco de Sant'Anna\textunderscore , I, p. XXIII. Cf. Arn. Gama, \textunderscore Segr. do Abb.\textunderscore , 28.
\section{Paisano}
\begin{itemize}
\item {fónica:pa-i-sa-no}
\end{itemize}
\begin{itemize}
\item {Grp. gram.:adj.}
\end{itemize}
\begin{itemize}
\item {Grp. gram.:M.}
\end{itemize}
\begin{itemize}
\item {Utilização:Des.}
\end{itemize}
\begin{itemize}
\item {Grp. gram.:Loc. adv.}
\end{itemize}
\begin{itemize}
\item {Proveniência:(Fr. \textunderscore paysan\textunderscore )}
\end{itemize}
Compatrício.
Que não é militar.
Indivíduo, que não é militar.
Compatriota. Cf. Pant. de Aveiro, \textunderscore Itiner.\textunderscore , 69, (2.^a ed.).
\textunderscore Á paisana\textunderscore , á maneira de quem é militar; em traje civil.
\section{Paisista}
\begin{itemize}
\item {fónica:pa-i}
\end{itemize}
\begin{itemize}
\item {Grp. gram.:m.  e  f.}
\end{itemize}
O mesmo que \textunderscore paisagista\textunderscore .
\section{Paiurá}
\begin{itemize}
\item {Grp. gram.:m.}
\end{itemize}
\begin{itemize}
\item {Utilização:Bras}
\end{itemize}
Planta medicinal do Alto Amazonas.
\section{Paivante}
\begin{itemize}
\item {Grp. gram.:m.}
\end{itemize}
\begin{itemize}
\item {Utilização:Gír.}
\end{itemize}
O mesmo que \textunderscore cigarro\textunderscore .
\section{Pai-velho}
\begin{itemize}
\item {Grp. gram.:m.  Loc.}
\end{itemize}
\begin{itemize}
\item {Utilização:Loc. escolar.}
\end{itemize}
Traducção literal de um clássico grego ou latino, para uso dos estudantes.
\section{Paivo}
\begin{itemize}
\item {Grp. gram.:m.}
\end{itemize}
\begin{itemize}
\item {Utilização:Gír.}
\end{itemize}
Cigarro.
(Or. \textunderscore eslava\textunderscore )
\section{Paivoto}
\begin{itemize}
\item {fónica:vô}
\end{itemize}
\begin{itemize}
\item {Grp. gram.:adj.}
\end{itemize}
\begin{itemize}
\item {Utilização:Prov.}
\end{itemize}
\begin{itemize}
\item {Utilização:beir.}
\end{itemize}
\begin{itemize}
\item {Grp. gram.:M.}
\end{itemize}
\begin{itemize}
\item {Proveniência:(De \textunderscore Paiva\textunderscore , n. p.)}
\end{itemize}
Diz-se de uma das três variedades de bois da raça arouqueza. Cf. Baganha, \textunderscore Hyg. Pec.\textunderscore , 202.
Relativo a Villa-Nova-de-Paiva ou Castello-de-Paiva.
Habitante das margens do Paiva.
\section{Paixão}
\begin{itemize}
\item {Grp. gram.:f.}
\end{itemize}
\begin{itemize}
\item {Proveniência:(Lat. \textunderscore passio\textunderscore )}
\end{itemize}
Bom ou mau movimento da alma.
Amor excessivo.
Grande mágua.
Cólera.
Enthusiasmo.
Predilecção.
Desejo intenso.
Objecto da affeição.
Falta de serenidade.
Parcialidade.
Allucinação.
Soffrimento ou martýrio, (falando-se de Christo, ou dos santos que foram martyrizados).
Colorido, expressão viva, em literatura.
Parte do Evangelho, em que se narra a paixão de Christo.
\section{Paixão}
\begin{itemize}
\item {Grp. gram.:f.}
\end{itemize}
\begin{itemize}
\item {Utilização:T. de Aveiro}
\end{itemize}
Cada uma das estacas em, que se arma o botirão. Cf. Rev. \textunderscore Tradição\textunderscore , V, 34.
\section{Paixoeiro}
\begin{itemize}
\item {Grp. gram.:m.}
\end{itemize}
\begin{itemize}
\item {Utilização:Ant.}
\end{itemize}
\begin{itemize}
\item {Proveniência:(De \textunderscore paixão\textunderscore )}
\end{itemize}
Livro, que contém a narração dos evangelistas sobre a paixão de Christo.
\section{Paixoneta}
\begin{itemize}
\item {fónica:nê}
\end{itemize}
\begin{itemize}
\item {Grp. gram.:f.}
\end{itemize}
\begin{itemize}
\item {Utilização:Fam.}
\end{itemize}
Pequena paixão; amorico.
\section{Paizeiro}
\begin{itemize}
\item {Grp. gram.:adj.}
\end{itemize}
\begin{itemize}
\item {Utilização:Prov.}
\end{itemize}
\begin{itemize}
\item {Utilização:trasm.}
\end{itemize}
Que é muito amigo do seu pai.
\section{Paizinho}
\begin{itemize}
\item {Grp. gram.:m.}
\end{itemize}
\begin{itemize}
\item {Utilização:Fam.}
\end{itemize}
Preto serviçal. Cf. Garrett, \textunderscore Helena\textunderscore , 16.
(Dem. de \textunderscore pai\textunderscore )
\section{Pajamarioba}
\begin{itemize}
\item {Grp. gram.:m.}
\end{itemize}
Planta leguminosa do Brasil.
\section{Pajanélia}
\begin{itemize}
\item {Grp. gram.:f.}
\end{itemize}
Gênero de árvores indianas.
\section{Pàjão}
\begin{itemize}
\item {Grp. gram.:m.}
\end{itemize}
\begin{itemize}
\item {Proveniência:(De \textunderscore pá\textunderscore )}
\end{itemize}
Instrumento de marnoto, para alisar e comprimir a superfície dos montes de sal. Cf. \textunderscore Museu Technol.\textunderscore , 67.
\section{Pajé}
\begin{itemize}
\item {Grp. gram.:m.}
\end{itemize}
\begin{itemize}
\item {Utilização:Bras}
\end{itemize}
O mesmo ou melhor que \textunderscore pagé\textunderscore .
Planta medicinal.
\section{Pajeú}
\begin{itemize}
\item {Grp. gram.:m.}
\end{itemize}
\begin{itemize}
\item {Utilização:Bras. do N}
\end{itemize}
Faca de ponta.
Punhal.
\section{Pàjião}
\begin{itemize}
\item {Grp. gram.:m.}
\end{itemize}
O mesmo que \textunderscore pàjão\textunderscore .
\section{Pajiola}
\begin{itemize}
\item {Grp. gram.:m.}
\end{itemize}
\begin{itemize}
\item {Utilização:Ant.}
\end{itemize}
\begin{itemize}
\item {Utilização:Pop.}
\end{itemize}
\begin{itemize}
\item {Proveniência:(De \textunderscore pagem\textunderscore )}
\end{itemize}
Criado presumido, que quere dar ares de senhor ou que gosta de imitar os amos, nos modos ou no traje.
\section{Pajonistas}
\begin{itemize}
\item {Grp. gram.:m. pl.}
\end{itemize}
\begin{itemize}
\item {Proveniência:(De \textunderscore Pajon\textunderscore , n. p.)}
\end{itemize}
Sectários protestantes, que só na prègação reconhecem a graça.
\section{Pajuçara}
\begin{itemize}
\item {Grp. gram.:adj.}
\end{itemize}
\begin{itemize}
\item {Utilização:Bras. do N}
\end{itemize}
Muito grande.
Que tem estatura muito elevada.
\section{Pala}
\begin{itemize}
\item {Grp. gram.:f.}
\end{itemize}
\begin{itemize}
\item {Utilização:Heráld.}
\end{itemize}
\begin{itemize}
\item {Utilização:Bras}
\end{itemize}
\begin{itemize}
\item {Utilização:Prov.}
\end{itemize}
\begin{itemize}
\item {Utilização:trasm.}
\end{itemize}
\begin{itemize}
\item {Utilização:T. do Fundão}
\end{itemize}
\begin{itemize}
\item {Grp. gram.:M.}
\end{itemize}
\begin{itemize}
\item {Utilização:Gír. de gatunos.}
\end{itemize}
\begin{itemize}
\item {Proveniência:(Lat. \textunderscore palla\textunderscore )}
\end{itemize}
Engaste.
Peça de metal, em que se engasta uma pedra de valor.
Peça mais ou menos consistente na parte inferò-anterior de barretina, de ou objecto semelhante.
Anteparo, para livrar os olhos doentes da claridade que os molesta.
Figura que, no campo do escudo, occupa, de ordinário, o terço do campo e tem posição vertical.
Capa ou manto comprido, que usavam as matronas romanas.
Espécie de poncho de fazenda fina.
Patranha, maranhão.
Cartão, guarnecido de pano, com que o sacerdote cobre cálice.
Parte do sapato, em que assenta a fivela.
Parte da polaina, que cobre o pé.
Parte móvel de uma cartucheira, para cobrir os cartuxos.
Empenho; protecção.
Camoéca, bebedeira.
Sócio ou cumplice dos carteiristas, que recebe, immediatamente ao roubo, o objecto furtado, escondendo-o e desapparecendo ou confundindo-se com a multidão.
\section{Pala}
\begin{itemize}
\item {Grp. gram.:f.}
\end{itemize}
Espécie de embarcação asiática. Cf. Castilho, \textunderscore Fastos\textunderscore , II, 407.
\section{Palabra}
\begin{itemize}
\item {Grp. gram.:f.}
\end{itemize}
\begin{itemize}
\item {Utilização:Des.}
\end{itemize}
O mesmo que \textunderscore palavra\textunderscore . Cf. \textunderscore Filodemo\textunderscore , IV, 6.
\section{Palabre}
\begin{itemize}
\item {Grp. gram.:m.}
\end{itemize}
\begin{itemize}
\item {Utilização:T. da Guiné}
\end{itemize}
Conferência de brancos com chefes indígenas.
(Cp. \textunderscore palabra\textunderscore )
\section{Palace}
\begin{itemize}
\item {Grp. gram.:m.}
\end{itemize}
Árvore de Damão, (\textunderscore butea frondosa\textunderscore ).
\section{Palacego}
\begin{itemize}
\item {fónica:cé}
\end{itemize}
\begin{itemize}
\item {Grp. gram.:adj.}
\end{itemize}
\begin{itemize}
\item {Utilização:P. us.}
\end{itemize}
O mesmo que \textunderscore palaciano\textunderscore . Cf. Garrett, \textunderscore Port. na Balança\textunderscore , 164.
(Cast. \textunderscore palaciego\textunderscore )
\section{Palacete}
\begin{itemize}
\item {fónica:cê}
\end{itemize}
\begin{itemize}
\item {Grp. gram.:m.}
\end{itemize}
Palácio pequeno.
\section{Palacianidade}
\begin{itemize}
\item {Grp. gram.:f.}
\end{itemize}
O mesmo que \textunderscore palacianismo\textunderscore . Cf. Camillo, \textunderscore Narcóticos\textunderscore , II, 274.
\section{Palacianismo}
\begin{itemize}
\item {Grp. gram.:m.}
\end{itemize}
Qualidade ou hábitos de palaciano.
\section{Palaciano}
\begin{itemize}
\item {Grp. gram.:adj.}
\end{itemize}
\begin{itemize}
\item {Grp. gram.:M.}
\end{itemize}
Relativo a palácio.
Próprio de quem vive na côrte; aristocrático.
Cortesão; áulico.
\section{Palaciego}
\begin{itemize}
\item {fónica:ê}
\end{itemize}
\begin{itemize}
\item {Grp. gram.:adj.}
\end{itemize}
O mesmo ou melhor que \textunderscore palacego\textunderscore :«\textunderscore ...amores cortesãos e palaciegos.\textunderscore »Latino, \textunderscore Camões\textunderscore , 69.
(Cast. \textunderscore palaciego\textunderscore )
\section{Palácio}
\begin{itemize}
\item {Grp. gram.:m.}
\end{itemize}
\begin{itemize}
\item {Utilização:Ant.}
\end{itemize}
\begin{itemize}
\item {Utilização:Ant.}
\end{itemize}
\begin{itemize}
\item {Utilização:Ant.}
\end{itemize}
\begin{itemize}
\item {Utilização:Ant.}
\end{itemize}
\begin{itemize}
\item {Proveniência:(Lat. \textunderscore palatium\textunderscore )}
\end{itemize}
Casa de Reis ou de família nobre.
Casa grande e apparatosa.
Edifício majestoso.
Edifício, onde se reunia a Câmara de uma terra.
Convento, mosteiro.
Armazém ou alpendre, em que se recolhiam escaleres e outras embarcações da Corôa.
Casa, para arrecadação de armamentos navaes.
\section{Paladar}
\begin{itemize}
\item {Grp. gram.:m.}
\end{itemize}
\begin{itemize}
\item {Utilização:Fig.}
\end{itemize}
\begin{itemize}
\item {Proveniência:(Do lat. hyp. \textunderscore palatum\textunderscore )}
\end{itemize}
Parte superior da cavidade buccal.
Sentido do gôsto.
Sabor; gôsto.
\section{Paladim}
\begin{itemize}
\item {Grp. gram.:m.}
\end{itemize}
O mesmo que \textunderscore paladino\textunderscore ^1. Cf. Camillo, \textunderscore Caveira\textunderscore , 83.
\section{Paladinamente}
\begin{itemize}
\item {Grp. gram.:adv.}
\end{itemize}
\begin{itemize}
\item {Utilização:Ant.}
\end{itemize}
\begin{itemize}
\item {Proveniência:(De \textunderscore paladino\textunderscore ^2)}
\end{itemize}
Ás claras, em público.
\section{Paladínico}
\begin{itemize}
\item {Grp. gram.:adj.}
\end{itemize}
\begin{itemize}
\item {Utilização:Neol.}
\end{itemize}
\begin{itemize}
\item {Proveniência:(De \textunderscore paladino\textunderscore ^2)}
\end{itemize}
Relativo a paladino, próprio de paladino.
Esforçado; temerário. Cf. Capello e Ivens, II, 5.
\section{Paladino}
\begin{itemize}
\item {Grp. gram.:m.}
\end{itemize}
\begin{itemize}
\item {Utilização:Fig.}
\end{itemize}
\begin{itemize}
\item {Proveniência:(Do lat. \textunderscore palatinus\textunderscore )}
\end{itemize}
Cada um dos principaes cavalleiros, que acompanhavam Carlos Magno na guerra.
Cavalleiro andante.
Homem corajoso.
Defensor dedicado.
\section{Paladino}
\begin{itemize}
\item {Grp. gram.:adj.}
\end{itemize}
\begin{itemize}
\item {Utilização:Ant.}
\end{itemize}
Vulgar, sabido, commum, público.
(Cast. \textunderscore paladino\textunderscore )
\section{Paladino}
\begin{itemize}
\item {Grp. gram.:m.}
\end{itemize}
\begin{itemize}
\item {Utilização:Ant.}
\end{itemize}
O mesmo que \textunderscore palácio\textunderscore  ou pequeno palácio.
(Por \textunderscore palatino\textunderscore , do lat. \textunderscore palatium\textunderscore )
\section{Palado}
\begin{itemize}
\item {Grp. gram.:m.}
\end{itemize}
\begin{itemize}
\item {Utilização:Heráld.}
\end{itemize}
Campo coberto de palas, alternadas de metal e de côr. Cf. Leite Ribeiro, \textunderscore Trat. de Armaria\textunderscore .
\section{Palafita}
\begin{itemize}
\item {Grp. gram.:m.}
\end{itemize}
\begin{itemize}
\item {Proveniência:(It. \textunderscore palafita\textunderscore )}
\end{itemize}
Estacas, que sustentavam as habitações lacustres dos homens prehistóricos.
Cada uma das populações lacustres.
\section{Palafrém}
\begin{itemize}
\item {Grp. gram.:m.}
\end{itemize}
\begin{itemize}
\item {Utilização:Des.}
\end{itemize}
\begin{itemize}
\item {Proveniência:(Do b. lat. \textunderscore parafredus\textunderscore )}
\end{itemize}
Cavallo, que os Reis e os nobres montavam, ao entrar na cidade.
Cavallo elegante, destinado principalmente a senhoras.
\section{Palafreneiro}
\begin{itemize}
\item {Grp. gram.:m.}
\end{itemize}
Moço, que tratava do palafrém ou o acompanhava.
\section{Palagonite}
\begin{itemize}
\item {Grp. gram.:f.}
\end{itemize}
Mineral amorpho, que se encontra nos terrenos vulcânicos de \textunderscore Palagónia\textunderscore , na Sicília.
\section{Palaio}
\begin{itemize}
\item {Grp. gram.:m.}
\end{itemize}
\begin{itemize}
\item {Utilização:Prov.}
\end{itemize}
\begin{itemize}
\item {Utilização:trasm.}
\end{itemize}
\begin{itemize}
\item {Utilização:alg.}
\end{itemize}
O mesmo que \textunderscore paio\textunderscore .
\section{Palalaca}
\begin{itemize}
\item {Grp. gram.:f.}
\end{itemize}
Ave das ilhas Filippinas.
\section{Palama}
\begin{itemize}
\item {Grp. gram.:f.}
\end{itemize}
\begin{itemize}
\item {Utilização:Prov.}
\end{itemize}
Pescada moída.
\section{Palamenta}
\begin{itemize}
\item {Grp. gram.:f.}
\end{itemize}
Conjunto de mastros, vêrgas, croques, ancorotes, remos, paus de bandeira, etc., de uma embarcação pequena.
Conjunto dos objectos necessários ao serviço das bocas de fogo.
(Cast. \textunderscore palamenta\textunderscore )
\section{Pálamo}
\begin{itemize}
\item {Grp. gram.:m.}
\end{itemize}
\begin{itemize}
\item {Proveniência:(Do gr. \textunderscore palame\textunderscore )}
\end{itemize}
Membrana, entre os dedos de algumas aves, reptis e alguns mammíferos.
\section{Palanca}
\begin{itemize}
\item {Grp. gram.:f.}
\end{itemize}
\begin{itemize}
\item {Utilização:Prov.}
\end{itemize}
\begin{itemize}
\item {Utilização:alg.}
\end{itemize}
\begin{itemize}
\item {Utilização:Prov.}
\end{itemize}
\begin{itemize}
\item {Utilização:trasm.}
\end{itemize}
\begin{itemize}
\item {Utilização:Açor}
\end{itemize}
\begin{itemize}
\item {Proveniência:(Lat. hyp. \textunderscore palanca\textunderscore )}
\end{itemize}
Estacaria, coberta de terra.
Estaca.
Instrumento de caldeireiro, que serve principalmente para alisar e estanhar.
Cada um dos dois paus, forrados de preto, sôbre que se leva o caixão mortuário, quando há convidados que levam as fitas e aos quaes se não quere molestar com o pêso do caixão.
O mesmo que \textunderscore alavanca\textunderscore .
Pasta de palha moída ou de estrume, apertada com outras, em rima.
Tranca; barrote.
\section{Palanca}
\begin{itemize}
\item {Grp. gram.:f.}
\end{itemize}
Elegante animal africano, do gênero dos antílopes, (\textunderscore hyppotragus equinus\textunderscore ).
\section{Palanca}
\begin{itemize}
\item {Grp. gram.:f.}
\end{itemize}
\begin{itemize}
\item {Utilização:Ant.}
\end{itemize}
O mesmo que \textunderscore palanque\textunderscore ^1.
\section{Palancada}
\begin{itemize}
\item {Grp. gram.:f.}
\end{itemize}
Reunião de palanques.
\section{Palancar}
\begin{itemize}
\item {Grp. gram.:v. t.}
\end{itemize}
Defender com palancas^1.
\section{Palancho}
\begin{itemize}
\item {Grp. gram.:m.}
\end{itemize}
\begin{itemize}
\item {Utilização:T. do Fundão}
\end{itemize}
Homem bonacheirão.
\section{Palanco}
\begin{itemize}
\item {Grp. gram.:m.}
\end{itemize}
\begin{itemize}
\item {Utilização:Náut.}
\end{itemize}
\begin{itemize}
\item {Utilização:Prov.}
\end{itemize}
\begin{itemize}
\item {Utilização:trasm.}
\end{itemize}
Corda, que se prende á vela e que serve para a içar.
Gramínea, semelhante á aveia.
\section{Palanfrório}
\begin{itemize}
\item {Grp. gram.:m.}
\end{itemize}
(Corr. de \textunderscore palavrório\textunderscore )
\section{Palangana}
\begin{itemize}
\item {Grp. gram.:f.}
\end{itemize}
\begin{itemize}
\item {Utilização:Prov.}
\end{itemize}
\begin{itemize}
\item {Utilização:minh.}
\end{itemize}
Tabuleiro, em que vão os assados á mesa.
Grande tigela.
Infusa, cântara.
Tigelada.
(Cp. lat. \textunderscore palanga\textunderscore )
\section{Palangre}
\begin{itemize}
\item {Grp. gram.:m.}
\end{itemize}
Apparelho de pesca, variedade de espinel.
\section{Palanque}
\begin{itemize}
\item {Grp. gram.:m.}
\end{itemize}
\begin{itemize}
\item {Utilização:Bras. do S}
\end{itemize}
Estrado com degraus ao ar livre.
Palanca^1.
Moirão, que se finca no meio do curral ou em frente delle, e ao qual se prende o cavallo bravo para o arrear.
(Cp. \textunderscore palanca\textunderscore ^1)
\section{Palanque}
\begin{itemize}
\item {Grp. gram.:m.}
\end{itemize}
Passarinho canoro da África, de dorso amarelo e riscas escuras.
\section{Palanqueiro}
\begin{itemize}
\item {Grp. gram.:m.}
\end{itemize}
Constructor de palanques.
\section{Palanqueta}
\begin{itemize}
\item {fónica:quê}
\end{itemize}
\begin{itemize}
\item {Grp. gram.:f.}
\end{itemize}
\begin{itemize}
\item {Utilização:Ant.}
\end{itemize}
\begin{itemize}
\item {Proveniência:(De \textunderscore palanca\textunderscore ^1)}
\end{itemize}
Barra de ferro, terminada por duas balas fixas, e que se empregava nos combates navaes.
\section{Palanquim}
\begin{itemize}
\item {Grp. gram.:m.}
\end{itemize}
\begin{itemize}
\item {Proveniência:(Do mal. \textunderscore palangki\textunderscore ?)}
\end{itemize}
Espécie de liteira, em que as pessôas mais ricas da Índia e da China se fazem transportar, conduzidas por servos.
Machila.
Rede suspensa, em que se descansa ou dorme.
Conductor de palanquim.
\section{Palão}
\begin{itemize}
\item {Grp. gram.:m.}
\end{itemize}
\begin{itemize}
\item {Utilização:Fam.}
\end{itemize}
\begin{itemize}
\item {Proveniência:(De \textunderscore pala\textunderscore )}
\end{itemize}
Mentira; grande pêta, galga.
\section{Palária}
\begin{itemize}
\item {Grp. gram.:f.}
\end{itemize}
\begin{itemize}
\item {Proveniência:(Lat. \textunderscore palaria\textunderscore )}
\end{itemize}
Exercício militar, usado entre os Romanos, e que consiste em esgrimir com uma espada de madeira contra estacas fincadas no chão.
\section{Palasse}
\begin{itemize}
\item {Grp. gram.:m.}
\end{itemize}
Árvore de Damão, (\textunderscore butea frondosa\textunderscore ).
\section{Palatal}
\begin{itemize}
\item {Grp. gram.:adj.}
\end{itemize}
Relativo ao palato.
\section{Palatalização}
\begin{itemize}
\item {Grp. gram.:f.}
\end{itemize}
Acto de palatalizar.
\section{Palatalizar}
\begin{itemize}
\item {Grp. gram.:v. t.}
\end{itemize}
Fórma exacta, em vez de \textunderscore palatizar\textunderscore .
\section{Palatina}
\begin{itemize}
\item {Grp. gram.:f.}
\end{itemize}
\begin{itemize}
\item {Proveniência:(De uma princesa palatina, diz Littré; entretanto, cp. \textunderscore pellatina\textunderscore )}
\end{itemize}
Pelliça, que as senhoras usam no inverno ao pescoço e sôbre os ombros.
\section{Palatinado}
\begin{itemize}
\item {Grp. gram.:m.}
\end{itemize}
\begin{itemize}
\item {Proveniência:(De \textunderscore palatino\textunderscore ^2)}
\end{itemize}
Dignidade de palatino.
Região, dominada por um palatino.
Cada província da Polónia.
\section{Palatinal}
\begin{itemize}
\item {Grp. gram.:adj.}
\end{itemize}
\begin{itemize}
\item {Proveniência:(De \textunderscore palatino\textunderscore ^1)}
\end{itemize}
O mesmo que \textunderscore palatal\textunderscore .
\section{Palatino}
\begin{itemize}
\item {Grp. gram.:adj.}
\end{itemize}
\begin{itemize}
\item {Grp. gram.:M.}
\end{itemize}
\begin{itemize}
\item {Proveniência:(De \textunderscore palato\textunderscore )}
\end{itemize}
O mesmo que \textunderscore palatal\textunderscore .
Nome de dois pequenos ossos, situados na parte posterior das fossas nasaes.
\section{Palatino}
\begin{itemize}
\item {Grp. gram.:m.  e  adj.}
\end{itemize}
\begin{itemize}
\item {Proveniência:(Lat. \textunderscore palatinus\textunderscore )}
\end{itemize}
Título antigo dos que tinham emprêgo no palácio de um Príncipe.
Príncipe ou senhor, que tinha palácio e administrava justiça.
Governador de uma província polaca.
\section{Palatite}
\begin{itemize}
\item {Grp. gram.:f.}
\end{itemize}
Inflammação do palato.
\section{Palatização}
\begin{itemize}
\item {Grp. gram.:f.}
\end{itemize}
Acto de palatizar.
\section{Palatizar}
\begin{itemize}
\item {Grp. gram.:v.}
\end{itemize}
\begin{itemize}
\item {Utilização:t. Gram.}
\end{itemize}
\begin{itemize}
\item {Proveniência:(De \textunderscore palato\textunderscore )}
\end{itemize}
Tornar palatal, (falando-se de sons ou palavras).
\section{Palato}
\begin{itemize}
\item {Grp. gram.:m.}
\end{itemize}
\begin{itemize}
\item {Proveniência:(Lat. \textunderscore palatum\textunderscore )}
\end{itemize}
Paladar.
Céu da bôca.
\section{Palato-labial}
\begin{itemize}
\item {Grp. gram.:adj.}
\end{itemize}
\begin{itemize}
\item {Utilização:Anat.}
\end{itemize}
Relativo ao palato e aos lábios.
\section{Palato-lingual}
\begin{itemize}
\item {Grp. gram.:adj.}
\end{itemize}
O mesmo que \textunderscore linguo-palatal\textunderscore .
\section{Palato-pharýngeo}
\begin{itemize}
\item {Grp. gram.:m.  e  adj.}
\end{itemize}
\begin{itemize}
\item {Utilização:Anat.}
\end{itemize}
Diz-se do músculo, situado, verticalmente na parede lateral da pharynge e na abóbada palatina.
\section{Palatoplastia}
\begin{itemize}
\item {Grp. gram.:f.}
\end{itemize}
\begin{itemize}
\item {Proveniência:(Do lat. \textunderscore palatum\textunderscore  + gr. \textunderscore plassein\textunderscore )}
\end{itemize}
Restauração cirúrgica de uma parte destruída do palato.
\section{Palava}
\begin{itemize}
\item {Grp. gram.:f.}
\end{itemize}
\begin{itemize}
\item {Proveniência:(De \textunderscore Palava\textunderscore , n. p.)}
\end{itemize}
Gênero de plantas malváceas do Peru.
\section{Palavão-branco}
\begin{itemize}
\item {Grp. gram.:m.}
\end{itemize}
Variedade de eucalypto, em Timor. Cf. \textunderscore Século\textunderscore , de 30-VII-911.
\section{Palavão-preto}
\begin{itemize}
\item {Grp. gram.:m.}
\end{itemize}
Árvore de Timor.
\section{Palavra}
\begin{itemize}
\item {Grp. gram.:f.}
\end{itemize}
\begin{itemize}
\item {Grp. gram.:Loc. adv.}
\end{itemize}
\begin{itemize}
\item {Grp. gram.:Loc. adv.}
\end{itemize}
\begin{itemize}
\item {Grp. gram.:Loc. adv.}
\end{itemize}
\begin{itemize}
\item {Proveniência:(Do lat. \textunderscore parabola\textunderscore )}
\end{itemize}
Som articulado, que tem um sentido ou significação.
Vocábulo; termo.
Dicção ou phrase.
Affirmação.
Fala, faculdade de exprimir as ideias por meio da voz.
O discorrer.
Declaração.
Promessa verbal: \textunderscore não falto, dou-lhe a minha palavra\textunderscore .
Permissão de falar: \textunderscore peço a palavra\textunderscore .
\textunderscore De palavra\textunderscore , de viva voz; oralmente.
\textunderscore Pela palavra\textunderscore , absolutamente, literalmente.
\textunderscore Têr a palavra\textunderscore , têr permissão para falar numa assembleia.
\textunderscore Têr palavra\textunderscore , cumprir alguém aquillo a que se obriga.
\textunderscore Palavra de rei\textunderscore , firmeza no que se diz ou promete; qualidade de quem mantém o que diz.
Sim; com certeza.
\section{Palavrada}
\begin{itemize}
\item {Grp. gram.:f.}
\end{itemize}
Palavra grosseira, obscena.
Farronca; bravata.
\section{Palavragem}
\begin{itemize}
\item {Grp. gram.:f.}
\end{itemize}
\begin{itemize}
\item {Utilização:P. us.}
\end{itemize}
O mesmo que \textunderscore palavreado\textunderscore .
\section{Palavrão}
\begin{itemize}
\item {Grp. gram.:m.}
\end{itemize}
\begin{itemize}
\item {Utilização:Ext.}
\end{itemize}
Palavra grande e que se pronuncia difficilmente.
Termo emphático ou empolado.
Palavrada; obscenidade.
\section{Palavreado}
\begin{itemize}
\item {Grp. gram.:m.}
\end{itemize}
\begin{itemize}
\item {Proveniência:(De \textunderscore palavrear\textunderscore )}
\end{itemize}
Reunião de palavras, com pouca ligação e importância.
Loquacidade; lábia.
\section{Palavreador}
\begin{itemize}
\item {Grp. gram.:m.  e  adj.}
\end{itemize}
O que palavreia.
\section{Palavrear}
\begin{itemize}
\item {Grp. gram.:v. i.}
\end{itemize}
\begin{itemize}
\item {Proveniência:(De \textunderscore palavra\textunderscore )}
\end{itemize}
Falar muito e levianamente; tagarelar, parolar.
\section{Palavreiro}
\begin{itemize}
\item {Grp. gram.:m.  e  adj.}
\end{itemize}
O mesmo que \textunderscore palavreador\textunderscore .
\section{Palavrinha}
\begin{itemize}
\item {Grp. gram.:f.}
\end{itemize}
\begin{itemize}
\item {Grp. gram.:Interj.}
\end{itemize}
Palavra pretensiosa, alambicada.
Palavra de honra!
\section{Palavrório}
\begin{itemize}
\item {Grp. gram.:m.}
\end{itemize}
O mesmo que \textunderscore palavreado\textunderscore .
\section{Palavroso}
\begin{itemize}
\item {Grp. gram.:adj.}
\end{itemize}
\begin{itemize}
\item {Proveniência:(De \textunderscore palavra\textunderscore )}
\end{itemize}
Prolixo em palavras, que pouco exprimem; loquaz.
\section{Palco}
\begin{itemize}
\item {Grp. gram.:m.}
\end{itemize}
\begin{itemize}
\item {Utilização:Ant.}
\end{itemize}
\begin{itemize}
\item {Proveniência:(Do germ. \textunderscore palcho\textunderscore )}
\end{itemize}
Estrado.
Parte do theatro, em que os actores representam.
Leito portátil.
\section{Paleáceo}
\begin{itemize}
\item {Grp. gram.:adj.}
\end{itemize}
\begin{itemize}
\item {Utilização:Bot.}
\end{itemize}
\begin{itemize}
\item {Proveniência:(Lat. \textunderscore paleaceus\textunderscore )}
\end{itemize}
Que é da natureza da palha.
Diz-se dos órgãos vegetaes, que são providos de palha.
\section{Paleanthropologia}
\begin{itemize}
\item {Grp. gram.:f.}
\end{itemize}
\begin{itemize}
\item {Proveniência:(De \textunderscore paleo...\textunderscore  + \textunderscore anthropologia\textunderscore )}
\end{itemize}
História natural do homem primitivo. Cf. Rui Barb., \textunderscore Réplica\textunderscore , 158.
\section{Paleantropologia}
\begin{itemize}
\item {Grp. gram.:f.}
\end{itemize}
\begin{itemize}
\item {Proveniência:(De \textunderscore paleo...\textunderscore  + \textunderscore anthropologia\textunderscore )}
\end{itemize}
História natural do homem primitivo. Cf. Rui Barb., \textunderscore Réplica\textunderscore , 158.
\section{Palear}
\begin{itemize}
\item {Grp. gram.:v. t.}
\end{itemize}
\begin{itemize}
\item {Utilização:P. us.}
\end{itemize}
\begin{itemize}
\item {Proveniência:(Do lat. \textunderscore palum\textunderscore )}
\end{itemize}
Patentear; divulgar:«\textunderscore sabe assi palear suas cachas\textunderscore ». \textunderscore Aulegrafia\textunderscore , 55.
\section{Palear}
\begin{itemize}
\item {Grp. gram.:v. i.}
\end{itemize}
\begin{itemize}
\item {Utilização:Prov.}
\end{itemize}
\begin{itemize}
\item {Proveniência:(De \textunderscore paleio\textunderscore ? ou o mesmo que \textunderscore palliar\textunderscore ?)}
\end{itemize}
Palestrar sôbre coisas fúteis.
(Colhido na Bairrada)
\section{Paleárctico}
\begin{itemize}
\item {Grp. gram.:adj.}
\end{itemize}
Diz-se das regiões zoológicas, comprehendidas pela Europa, Ásia até o Himalaia, e África setentrional até o Sahará.
\section{Paleco}
\begin{itemize}
\item {Grp. gram.:m.}
\end{itemize}
\begin{itemize}
\item {Utilização:T. da Nazaré}
\end{itemize}
Sujeito adventicio, metediço.
\section{Palega}
\begin{itemize}
\item {Grp. gram.:f.}
\end{itemize}
Pequena embarcação asiática.
\section{Paleiforme}
\begin{itemize}
\item {Grp. gram.:adj.}
\end{itemize}
\begin{itemize}
\item {Proveniência:(Do lat. \textunderscore palea\textunderscore  + \textunderscore forma\textunderscore )}
\end{itemize}
Semelhante á palha.
\section{Paleio}
\begin{itemize}
\item {Grp. gram.:m.}
\end{itemize}
\begin{itemize}
\item {Utilização:Pop.}
\end{itemize}
Lábia.
Festas ou carícias interesseiras.
Palavreado.
(Us. pelo menos na Extremadura e no Minho)
\section{Palémon}
\begin{itemize}
\item {Grp. gram.:m.}
\end{itemize}
\begin{itemize}
\item {Proveniência:(Do gr. \textunderscore Palaimon\textunderscore , n. p.)}
\end{itemize}
Constellação do hemisphério boreal, hércules.
Gênero de crustáceos, a que pertence o camarão.
\section{Paleo...}
\begin{itemize}
\item {Grp. gram.:pref.}
\end{itemize}
\begin{itemize}
\item {Proveniência:(Do gr. \textunderscore palaios\textunderscore )}
\end{itemize}
(designativo de \textunderscore antigo\textunderscore )
\section{Paleoarcheologia}
\begin{itemize}
\item {fónica:que}
\end{itemize}
\begin{itemize}
\item {Grp. gram.:f.}
\end{itemize}
\begin{itemize}
\item {Proveniência:(De \textunderscore paleo...\textunderscore  + \textunderscore archeologia\textunderscore )}
\end{itemize}
Estudo archeológico dos objectos pertencentes aos homens prehistóricos.
\section{Paleoarqueologia}
\begin{itemize}
\item {Grp. gram.:f.}
\end{itemize}
\begin{itemize}
\item {Proveniência:(De \textunderscore paleo...\textunderscore  + \textunderscore archeologia\textunderscore )}
\end{itemize}
Estudo arqueológico dos objectos pertencentes aos homens prehistóricos.
\section{Paleoethnologia}
\begin{itemize}
\item {Grp. gram.:f.}
\end{itemize}
\begin{itemize}
\item {Proveniência:(De \textunderscore paleoethnólogo\textunderscore )}
\end{itemize}
Sciência das raças humanas prehistóricas.
\section{Paleoethnológico}
\begin{itemize}
\item {Grp. gram.:adj.}
\end{itemize}
Relativo á Paleoethnologia.
\section{Paleoethnologista}
\begin{itemize}
\item {Grp. gram.:m.  e  f.}
\end{itemize}
Pessôa que trata de Paleoethnologia.
\section{Paleoethnólogo}
\begin{itemize}
\item {Grp. gram.:m.}
\end{itemize}
\begin{itemize}
\item {Proveniência:(De \textunderscore paleo...\textunderscore  + \textunderscore ethnólogo\textunderscore )}
\end{itemize}
Aquelle que é versado em Paleoethnologia.
\section{Paleoetnologia}
\begin{itemize}
\item {Grp. gram.:f.}
\end{itemize}
\begin{itemize}
\item {Proveniência:(De \textunderscore paleoetnólogo\textunderscore )}
\end{itemize}
Ciência das raças humanas prehistóricas.
\section{Paleoetnológico}
\begin{itemize}
\item {Grp. gram.:adj.}
\end{itemize}
Relativo á Paleoetnologia.
\section{Paleoetnologista}
\begin{itemize}
\item {Grp. gram.:m.  e  f.}
\end{itemize}
Pessôa que trata de Paleoetnologia.
\section{Paleoetnólogo}
\begin{itemize}
\item {Grp. gram.:m.}
\end{itemize}
\begin{itemize}
\item {Proveniência:(De \textunderscore paleo...\textunderscore  + \textunderscore ethnólogo\textunderscore )}
\end{itemize}
Aquele que é versado em Paleoetnologia.
\section{Paleofitologia}
\begin{itemize}
\item {fónica:le-o}
\end{itemize}
\begin{itemize}
\item {Grp. gram.:f.}
\end{itemize}
\begin{itemize}
\item {Proveniência:(De \textunderscore paleofitólogo\textunderscore )}
\end{itemize}
Tratado de plantas fósseis.
\section{Paleofitológico}
\begin{itemize}
\item {fónica:le-o}
\end{itemize}
\begin{itemize}
\item {Grp. gram.:adj.}
\end{itemize}
Relativo á Paleofitología.
\section{Paleofitólogo}
\begin{itemize}
\item {Grp. gram.:m.}
\end{itemize}
\begin{itemize}
\item {Proveniência:(Do gr. \textunderscore palaios\textunderscore  + \textunderscore phuton\textunderscore  + \textunderscore logos\textunderscore )}
\end{itemize}
Aquele que se dedica ao estudo da Paleofitologia.
\section{Paleogêneo}
\begin{itemize}
\item {Grp. gram.:adj.}
\end{itemize}
\begin{itemize}
\item {Utilização:Geol.}
\end{itemize}
\begin{itemize}
\item {Proveniência:(Do gr. \textunderscore palaios\textunderscore  + \textunderscore genes\textunderscore )}
\end{itemize}
Diz-se de um terreno, que constitue uma secção do hessocênico na série terciária sedimentar.
\section{Paleogeografia}
\begin{itemize}
\item {Grp. gram.:f.}
\end{itemize}
\begin{itemize}
\item {Proveniência:(Do gr. \textunderscore palaios\textunderscore  + \textunderscore ge\textunderscore  + \textunderscore graphein\textunderscore )}
\end{itemize}
Geografia do globo terrestre nos tempos mais remotos.
\section{Paleogeographia}
\begin{itemize}
\item {Grp. gram.:f.}
\end{itemize}
\begin{itemize}
\item {Proveniência:(Do gr. \textunderscore palaios\textunderscore  + \textunderscore ge\textunderscore  + \textunderscore graphein\textunderscore )}
\end{itemize}
Geographia do globo terrestre nos tempos mais remotos.
\section{Paleografar}
\begin{itemize}
\item {Grp. gram.:v. i.}
\end{itemize}
Estudar ou praticar a paleografia. Cf. Camillo, \textunderscore Cancion. Al.\textunderscore , 499.
(Cp. \textunderscore paleógrafo\textunderscore )
\section{Paleografia}
\begin{itemize}
\item {Grp. gram.:f.}
\end{itemize}
\begin{itemize}
\item {Proveniência:(De \textunderscore paleógrafo\textunderscore )}
\end{itemize}
Arte de decifar escritos antigos, especialmente os diplomas manuscritos da Idade-Média.
\section{Paleógrafo}
\begin{itemize}
\item {Grp. gram.:m.}
\end{itemize}
\begin{itemize}
\item {Proveniência:(Do gr. \textunderscore palaios\textunderscore  + \textunderscore graphein\textunderscore )}
\end{itemize}
Aquele que é versado em paleografia ou se ocupa dela.
\section{Paleographar}
\begin{itemize}
\item {Grp. gram.:v. i.}
\end{itemize}
Estudar ou praticar a paleographia. Cf. Camillo, \textunderscore Cancion. Al.\textunderscore , 499.
(Cp. \textunderscore paleógrapho\textunderscore )
\section{Paleographia}
\begin{itemize}
\item {Grp. gram.:f.}
\end{itemize}
\begin{itemize}
\item {Proveniência:(De \textunderscore paleógrapho\textunderscore )}
\end{itemize}
Arte de decifar escritos antigos, especialmente os diplomas manuscritos da Idade-Média.
\section{Paleógrapho}
\begin{itemize}
\item {Grp. gram.:m.}
\end{itemize}
\begin{itemize}
\item {Proveniência:(Do gr. \textunderscore palaios\textunderscore  + \textunderscore graphein\textunderscore )}
\end{itemize}
Aquelle que é versado em paleographia ou se occupa della.
\section{Paléola}
\begin{itemize}
\item {Grp. gram.:f.}
\end{itemize}
\begin{itemize}
\item {Utilização:Bot.}
\end{itemize}
\begin{itemize}
\item {Proveniência:(Do lat. \textunderscore palea\textunderscore )}
\end{itemize}
Cada uma das pequenas escamas, que cercam o ovário de certas gramíneas.
Appêndice do clinantho.
\section{Paleolífero}
\begin{itemize}
\item {Grp. gram.:adj.}
\end{itemize}
\begin{itemize}
\item {Utilização:Bot.}
\end{itemize}
\begin{itemize}
\item {Proveniência:(De \textunderscore paléola\textunderscore  + lat. \textunderscore ferre\textunderscore )}
\end{itemize}
Que tem paléolas.
\section{Paleolíthica}
\begin{itemize}
\item {Grp. gram.:f.}
\end{itemize}
\begin{itemize}
\item {Utilização:Geol.}
\end{itemize}
Primeiro período da idade de pedra.
Idade da pedra lascada.
(Fem. de \textunderscore paleolíthico\textunderscore )
\section{Paleolíthico}
\begin{itemize}
\item {Grp. gram.:adj.}
\end{itemize}
\begin{itemize}
\item {Proveniência:(Do gr. \textunderscore palaios\textunderscore  + \textunderscore lithos\textunderscore )}
\end{itemize}
Relativo ao primeiro período da antiga idade da pedra, ou á idade da pedra lascada.
\section{Paleolítica}
\begin{itemize}
\item {Grp. gram.:f.}
\end{itemize}
\begin{itemize}
\item {Utilização:Geol.}
\end{itemize}
Primeiro período da idade de pedra.
Idade da pedra lascada.
(Fem. de \textunderscore paleolítico\textunderscore )
\section{Paleolítico}
\begin{itemize}
\item {Grp. gram.:adj.}
\end{itemize}
\begin{itemize}
\item {Proveniência:(Do gr. \textunderscore palaios\textunderscore  + \textunderscore lithos\textunderscore )}
\end{itemize}
Relativo ao primeiro período da antiga idade da pedra, ou á idade da pedra lascada.
\section{Paleologia}
\begin{itemize}
\item {Grp. gram.:f.}
\end{itemize}
Estudo das línguas antigas.
(Cp. \textunderscore paleólogo\textunderscore )
\section{Paleólogo}
\begin{itemize}
\item {Grp. gram.:m.  e  adj.}
\end{itemize}
\begin{itemize}
\item {Proveniência:(Do gr. \textunderscore palaios\textunderscore  + \textunderscore logos\textunderscore )}
\end{itemize}
O que conhece as línguas antigas.
\section{Paleontografia}
\begin{itemize}
\item {Grp. gram.:f.}
\end{itemize}
\begin{itemize}
\item {Proveniência:(Do gr. \textunderscore palaios\textunderscore  + \textunderscore ontos\textunderscore  + \textunderscore graphein\textunderscore )}
\end{itemize}
Descripção dos corpos organizados, fósseis.
\section{Paleontographia}
\begin{itemize}
\item {Grp. gram.:f.}
\end{itemize}
\begin{itemize}
\item {Proveniência:(Do gr. \textunderscore palaios\textunderscore  + \textunderscore ontos\textunderscore  + \textunderscore graphein\textunderscore )}
\end{itemize}
Descripção dos corpos organizados, fósseis.
\section{Paleontologia}
\begin{itemize}
\item {Grp. gram.:f.}
\end{itemize}
Tratado ou sciência, que se occupa dos animaes e vegetaes fósseis.
(Cp. \textunderscore paleontólogo\textunderscore )
\section{Paleontológico}
\begin{itemize}
\item {Grp. gram.:adj.}
\end{itemize}
Relativo á Paleontologia.
\section{Paleontologista}
\begin{itemize}
\item {Grp. gram.:m.  e  f.}
\end{itemize}
Aquelle que trata de Paleontologia.
\section{Paleontólogo}
\begin{itemize}
\item {Grp. gram.:m.}
\end{itemize}
\begin{itemize}
\item {Proveniência:(Do gr. \textunderscore palaios\textunderscore  + \textunderscore ontos\textunderscore  + \textunderscore logos\textunderscore )}
\end{itemize}
Aquelle que é versado em Paleontologia.
\section{Paleophytologia}
\begin{itemize}
\item {fónica:le-o}
\end{itemize}
\begin{itemize}
\item {Grp. gram.:f.}
\end{itemize}
\begin{itemize}
\item {Proveniência:(De \textunderscore paleophytólogo\textunderscore )}
\end{itemize}
Tratado de plantas fósseis.
\section{Paleophytológico}
\begin{itemize}
\item {fónica:le-o}
\end{itemize}
\begin{itemize}
\item {Grp. gram.:adj.}
\end{itemize}
Relativo á Paleophytología.
\section{Paleophytólogo}
\begin{itemize}
\item {Grp. gram.:m.}
\end{itemize}
\begin{itemize}
\item {Proveniência:(Do gr. \textunderscore palaios\textunderscore  + \textunderscore phuton\textunderscore  + \textunderscore logos\textunderscore )}
\end{itemize}
Aquelle que se dedica ao estudo da Paleophytologia.
\section{Paleotério}
\begin{itemize}
\item {Grp. gram.:m.}
\end{itemize}
\begin{itemize}
\item {Proveniência:(Do gr. \textunderscore palaios\textunderscore , antigo, e \textunderscore therion\textunderscore , animal)}
\end{itemize}
Gênero de paquidermos fósseis.
\section{Paleothério}
\begin{itemize}
\item {Grp. gram.:m.}
\end{itemize}
\begin{itemize}
\item {Proveniência:(Do gr. \textunderscore palaios\textunderscore , antigo, e \textunderscore therion\textunderscore , animal)}
\end{itemize}
Gênero de pachydermos fósseis.
\section{Paleotípico}
\begin{itemize}
\item {Grp. gram.:adj.}
\end{itemize}
Relativo a paleótipo.
\section{Paleótipo}
\begin{itemize}
\item {Grp. gram.:m.}
\end{itemize}
\begin{itemize}
\item {Proveniência:(Do gr. \textunderscore palaios\textunderscore  + \textunderscore tupos\textunderscore )}
\end{itemize}
Documento escrito, cuja graphia lhe demonstra a antiguidade.
\section{Paleotýpico}
\begin{itemize}
\item {Grp. gram.:adj.}
\end{itemize}
Relativo a paleótypo.
\section{Paleótypo}
\begin{itemize}
\item {Grp. gram.:m.}
\end{itemize}
\begin{itemize}
\item {Proveniência:(Do gr. \textunderscore palaios\textunderscore  + \textunderscore tupos\textunderscore )}
\end{itemize}
Documento escrito, cuja graphia lhe demonstra a antiguidade.
\section{Paleozóico}
\begin{itemize}
\item {Grp. gram.:adj.}
\end{itemize}
\begin{itemize}
\item {Proveniência:(Do gr. \textunderscore palaios\textunderscore  + \textunderscore zoikos\textunderscore )}
\end{itemize}
Relativo a animaes ou vegetaes, cujas espécies se extinguiram.
Diz-se do terreno, em que há vestígios fósseis dessas espécies.
\section{Paleozoítico}
\begin{itemize}
\item {Grp. gram.:adj.}
\end{itemize}
\begin{itemize}
\item {Utilização:Geol.}
\end{itemize}
Diz-se da segunda das seis phases do período philogenético, na qual os animaes superiores eram os peixes. Cf. \textunderscore Noticia\textunderscore , do Rio, de 27-VI-902.
\section{Paladamina}
\begin{itemize}
\item {Grp. gram.:f.}
\end{itemize}
\begin{itemize}
\item {Utilização:Chím.}
\end{itemize}
\begin{itemize}
\item {Proveniência:(De \textunderscore palas\textunderscore , \textunderscore palados\textunderscore  gr. + \textunderscore aminas\textunderscore )}
\end{itemize}
Substância cristalizável, escura, de aspecto resinoso, a qual faz precipitar a base dos saes de cobre e de prata.
\section{Paladiamina}
\begin{itemize}
\item {Grp. gram.:f.}
\end{itemize}
\begin{itemize}
\item {Utilização:Chím.}
\end{itemize}
Substância que dá combinações, análogas á paladamina.
(Cp. \textunderscore paladamina\textunderscore )
\section{Paladinito}
\begin{itemize}
\item {Grp. gram.:m.}
\end{itemize}
\begin{itemize}
\item {Utilização:Miner.}
\end{itemize}
Óxido de paládio.
\section{Paládio}
\begin{itemize}
\item {Grp. gram.:m.}
\end{itemize}
\begin{itemize}
\item {Utilização:Fig.}
\end{itemize}
\begin{itemize}
\item {Proveniência:(Do gr. \textunderscore palladion\textunderscore )}
\end{itemize}
Estátua de Pallas, venerada pelos Troianos, como penhor da sua conservação.
Salvaguarda; protecção.
Metal simples, da côr do chumbo e pouco fusível.
\section{Palanestesia}
\begin{itemize}
\item {Grp. gram.:f.}
\end{itemize}
\begin{itemize}
\item {Utilização:Med.}
\end{itemize}
\begin{itemize}
\item {Proveniência:(De \textunderscore pallo\textunderscore  gr. + \textunderscore anesthesia\textunderscore )}
\end{itemize}
Abolição da sensibilidade vibratória.
\section{Palejar}
\begin{itemize}
\item {Grp. gram.:v. i.}
\end{itemize}
\begin{itemize}
\item {Utilização:Bras}
\end{itemize}
\begin{itemize}
\item {Utilização:Neol.}
\end{itemize}
Vocábulo, proposto indevidamente por Alencar, em vez de \textunderscore palidejar\textunderscore .
\section{Palente}
\begin{itemize}
\item {Grp. gram.:adj.}
\end{itemize}
\begin{itemize}
\item {Utilização:Poét.}
\end{itemize}
\begin{itemize}
\item {Proveniência:(Lat. \textunderscore pallens\textunderscore )}
\end{itemize}
Que se mostra pálido; que palideja.
O mesmo que \textunderscore pálido\textunderscore . Cf. Castilho, \textunderscore Geórgicas\textunderscore , 217.
\section{Paleozoologia}
\begin{itemize}
\item {Grp. gram.:f.}
\end{itemize}
\begin{itemize}
\item {Proveniência:(Do gr. \textunderscore palaios\textunderscore  + \textunderscore logos\textunderscore )}
\end{itemize}
Tratado á cêrca dos animais fósseis.
\section{Paleozoológico}
\begin{itemize}
\item {Grp. gram.:adj.}
\end{itemize}
Relativo á Paleozoologia.
\section{Paleozoologista}
\begin{itemize}
\item {Grp. gram.:m.}
\end{itemize}
Tratadista de Paleozoologia.
\section{Palerma}
\begin{itemize}
\item {Grp. gram.:m. ,  f.  e  adj.}
\end{itemize}
Pessôa tola, néscia, imbecil, idiota.
Pacóvio, parvajola.
\section{Palermice}
\begin{itemize}
\item {Grp. gram.:f.}
\end{itemize}
Qualidade, acto ou dito de palerma.
\section{Palestesia}
\begin{itemize}
\item {Grp. gram.:f.}
\end{itemize}
\begin{itemize}
\item {Proveniência:(De \textunderscore pallo\textunderscore  gr. + \textunderscore esthesia\textunderscore )}
\end{itemize}
Sensibilidade vibratória.
\section{Palestina}
\begin{itemize}
\item {Grp. gram.:f.}
\end{itemize}
\begin{itemize}
\item {Utilização:Typ.}
\end{itemize}
Carácter typográphico de 22 pontos.
\section{Palestinos}
\begin{itemize}
\item {Grp. gram.:m. pl.}
\end{itemize}
Habitantes da Palestina.
Uma das colónias muçulmanas que, na Idade-Média, se estabeleceram na Península Hispânica. Cf. Herculano, \textunderscore Hist. de Port.\textunderscore , 1.^a ed., III, 201.
\section{Palestra}
\begin{itemize}
\item {Grp. gram.:f.}
\end{itemize}
\begin{itemize}
\item {Utilização:Ant.}
\end{itemize}
\begin{itemize}
\item {Proveniência:(Lat. \textunderscore palaestra\textunderscore )}
\end{itemize}
Conversa; discussão ligeira; cavaco.
Lugar para exercícios gymnásticos, na Grécia e em Roma.
\section{Palestrante}
\begin{itemize}
\item {Grp. gram.:m.}
\end{itemize}
Aquelle que palestra.
\section{Palestrar}
\begin{itemize}
\item {Grp. gram.:v. i.}
\end{itemize}
\begin{itemize}
\item {Grp. gram.:V. t.}
\end{itemize}
Conversar; discutir ligeiramente; cavaquear.
Acompanhar com palestra (uma refeição, um passeio, etc.).
\section{Palestrear}
\begin{itemize}
\item {Grp. gram.:v. i.}
\end{itemize}
O mesmo que \textunderscore palestrar\textunderscore . Cf. Camillo, \textunderscore Noites de Insómn.\textunderscore , III, 28.
\section{Palestreiro}
\begin{itemize}
\item {Grp. gram.:adj.}
\end{itemize}
O mesmo que \textunderscore palrador\textunderscore . Cf. Castilho, \textunderscore Tosquia\textunderscore .
\section{Paléstrica}
\begin{itemize}
\item {Grp. gram.:f.}
\end{itemize}
\begin{itemize}
\item {Utilização:Des.}
\end{itemize}
\begin{itemize}
\item {Proveniência:(De \textunderscore paléstrico\textunderscore )}
\end{itemize}
O mesmo que \textunderscore gymnástica\textunderscore , ou parte da Gymnástica, que se compõe da luta, da carreira, do salto, etc.
\section{Paléstrico}
\begin{itemize}
\item {Grp. gram.:adj.}
\end{itemize}
\begin{itemize}
\item {Utilização:Des.}
\end{itemize}
\begin{itemize}
\item {Proveniência:(Lat. \textunderscore palaestricus\textunderscore )}
\end{itemize}
O mesmo que \textunderscore gymnástico\textunderscore .
\section{Palestriniano}
\begin{itemize}
\item {Grp. gram.:adj.}
\end{itemize}
\begin{itemize}
\item {Utilização:Mús.}
\end{itemize}
Relativo ao compositor Palestrina ou ao seu estilo.
\section{Palestrita}
\begin{itemize}
\item {Grp. gram.:m.  e  f.}
\end{itemize}
\begin{itemize}
\item {Proveniência:(Gr. \textunderscore palaistrites\textunderscore )}
\end{itemize}
Pessôa que frequentava os gymnásios da Grécia e Roma.
\section{Paleta}
\begin{itemize}
\item {fónica:lê}
\end{itemize}
\begin{itemize}
\item {Grp. gram.:f.}
\end{itemize}
\begin{itemize}
\item {Utilização:Bras}
\end{itemize}
\begin{itemize}
\item {Grp. gram.:Pl.}
\end{itemize}
\begin{itemize}
\item {Proveniência:(De \textunderscore pala\textunderscore )}
\end{itemize}
Tábua delgada, em que os pintores dispõem e combinam as tintas, e que elles sustentam na mão esquerda com o dedo pollegar, enfiado num orifício da mesma paleta.
A parte mais alta e grossa das pernas deanteiras do cavallo e do boi.
Instrumentos, para modelar em barro ou cera.
\section{Paletear}
\begin{itemize}
\item {Grp. gram.:v. t.}
\end{itemize}
\begin{itemize}
\item {Utilização:Bras. do S}
\end{itemize}
Esporear na paleta.
\section{Paletó}
\begin{itemize}
\item {Grp. gram.:m.}
\end{itemize}
\begin{itemize}
\item {Proveniência:(Fr. \textunderscore paletot\textunderscore )}
\end{itemize}
Casaco largo, que se veste por cima de outro, ou por cima da casaca; sobretudo.
\section{Paleu}
\begin{itemize}
\item {Grp. gram.:m.}
\end{itemize}
\begin{itemize}
\item {Utilização:Ant.}
\end{itemize}
Discurso?:«\textunderscore ...notifica terrestres jogos, paleus, palestras...\textunderscore »\textunderscore Viriato Trág.\textunderscore , XI, 11.
\section{Palfreneiro}
\begin{itemize}
\item {Grp. gram.:m.}
\end{itemize}
(V.palafreneiro). Cf. Filinto, VII, 111.
\section{Palha}
\begin{itemize}
\item {Grp. gram.:f.}
\end{itemize}
\begin{itemize}
\item {Utilização:Fig.}
\end{itemize}
\begin{itemize}
\item {Utilização:Fig.}
\end{itemize}
\begin{itemize}
\item {Grp. gram.:Loc. adv.}
\end{itemize}
\begin{itemize}
\item {Proveniência:(Lat. \textunderscore palea\textunderscore )}
\end{itemize}
Haste, especialmente a haste sêca, das plantas gramíneas, despojadas dos grãos.
Accumulação de hastes desta natureza.
Junco sêco, com que se entretece o assento de cadeiras.
Insignificância, bagatela.
Qualquer coisa vistosa ou apparatosa, mas de pouco valor ou importância, ou de effeito passageiro.
\textunderscore Por dá cá aquella palha\textunderscore , sem motivo sério; por uma insignificância.
\textunderscore Travar palha\textunderscore , motejar. Cf. \textunderscore Eufrosina\textunderscore , 6.
\section{Palhabote}
\begin{itemize}
\item {Grp. gram.:m.}
\end{itemize}
\begin{itemize}
\item {Proveniência:(Do ingl. \textunderscore pilot-boat\textunderscore )}
\end{itemize}
Barco de dois mastros muito juntos e vela triangular ou latina.
\section{Palhaboteiro}
\begin{itemize}
\item {Grp. gram.:m.}
\end{itemize}
Tripulante de palhabote.
\section{Palhaça}
\begin{itemize}
\item {Grp. gram.:adj. f.}
\end{itemize}
\begin{itemize}
\item {Utilização:Ant.}
\end{itemize}
Dizia-se das habitações feitas de palha. Cf. Castanheda, l. I, c. III.
\section{Palhaçada}
\begin{itemize}
\item {Grp. gram.:f.}
\end{itemize}
Acto ou dito próprio de palhaço.
Scena burlesca.
Agrupamento de palhaços.
\section{Palhacarga}
\begin{itemize}
\item {Grp. gram.:f.}
\end{itemize}
Variedade de junça.
\section{Palhaço}
\begin{itemize}
\item {Grp. gram.:adj.}
\end{itemize}
\begin{itemize}
\item {Grp. gram.:M.}
\end{itemize}
Vestido ou feito de palha.
Bobo; saltimbanco; arlequim.
\section{Palhada}
\begin{itemize}
\item {Grp. gram.:f.}
\end{itemize}
\begin{itemize}
\item {Utilização:Prov.}
\end{itemize}
\begin{itemize}
\item {Utilização:minh.}
\end{itemize}
\begin{itemize}
\item {Utilização:Fig.}
\end{itemize}
\begin{itemize}
\item {Proveniência:(De \textunderscore palha\textunderscore )}
\end{itemize}
Mistura de palha e farelo, para alimento de animaes.
Mistura de palha e erva, para alimento de bois.
Palavrório; estopada.
Vianda ordinária.
\section{Palha-de-água}
\begin{itemize}
\item {Grp. gram.:f.}
\end{itemize}
Planta gramínea de Cabo Verde, (\textunderscore eleusine indica\textunderscore , Gaert.).
\section{Palha-de-arco}
\begin{itemize}
\item {Grp. gram.:f.}
\end{itemize}
Árvore medicinal da Guiné.
\section{Palhagem}
\begin{itemize}
\item {Grp. gram.:f.}
\end{itemize}
Montão de palha.
\section{Palhal}
\begin{itemize}
\item {Grp. gram.:m.}
\end{itemize}
\begin{itemize}
\item {Proveniência:(Do lat. \textunderscore palealis\textunderscore )}
\end{itemize}
Casa coberta de palha; palhoça.
\section{Palhana}
\begin{itemize}
\item {Grp. gram.:f.}
\end{itemize}
\begin{itemize}
\item {Utilização:T. de Alcanena}
\end{itemize}
\begin{itemize}
\item {Proveniência:(De \textunderscore palha\textunderscore )}
\end{itemize}
Cabana.
Pequena casa de habitação.
\section{Palha-preta}
\begin{itemize}
\item {Grp. gram.:f.}
\end{itemize}
Planta trepadeira da Guiné.
\section{Palhar}
\begin{itemize}
\item {Grp. gram.:m.}
\end{itemize}
O mesmo que \textunderscore palhal\textunderscore .
\section{Palha-tomás}
\begin{itemize}
\item {Grp. gram.:f.}
\end{itemize}
Planta gramínea de Cabo-Verde, (\textunderscore chloris radiata\textunderscore , Sw.).
\section{Palhatório}
\begin{itemize}
\item {Grp. gram.:m.}
\end{itemize}
\begin{itemize}
\item {Utilização:Ant.}
\end{itemize}
Locutório nos conventos.
O mesmo que \textunderscore parlatório\textunderscore .
\section{Palhegal}
\begin{itemize}
\item {Grp. gram.:m.}
\end{itemize}
Terra, em que há muita palha.
\section{Palheira}
\begin{itemize}
\item {Grp. gram.:f.}
\end{itemize}
\begin{itemize}
\item {Utilização:Prov.}
\end{itemize}
\begin{itemize}
\item {Utilização:beir.}
\end{itemize}
\begin{itemize}
\item {Utilização:Prov.}
\end{itemize}
\begin{itemize}
\item {Utilização:minh.}
\end{itemize}
\begin{itemize}
\item {Utilização:Prov.}
\end{itemize}
\begin{itemize}
\item {Utilização:trasm.}
\end{itemize}
Casa, onde se guarda palha.
Uma haste de trigo ou de plantas congêneres, sem grãos.
Linha ou pedaço de piaçaba, que se emprega nos armelos.
Paveia de colmo, que os rapazes introduzem nas covas, volteando-a, para caçar grilos.
\section{Palheirão}
\begin{itemize}
\item {Grp. gram.:m.}
\end{itemize}
\begin{itemize}
\item {Grp. gram.:M.  e  adj.}
\end{itemize}
\begin{itemize}
\item {Utilização:Fig.}
\end{itemize}
Palheiro grande.
Pessôa, que discorre prolixamente e sem clareza.
Livro extenso e pouco claro.
\section{Palheireira}
\textunderscore fem.\textunderscore  de palheireiro.
\section{Palheireiro}
\begin{itemize}
\item {Grp. gram.:m.  e  adj.}
\end{itemize}
\begin{itemize}
\item {Proveniência:(De \textunderscore palheiro\textunderscore )}
\end{itemize}
Aquelle que vende palha.
Aquelle que faz assentos de palha para cadeiras, bancos, etc.
\section{Palheiro}
\begin{itemize}
\item {Grp. gram.:m.}
\end{itemize}
\begin{itemize}
\item {Utilização:Prov.}
\end{itemize}
\begin{itemize}
\item {Utilização:Marn.}
\end{itemize}
\begin{itemize}
\item {Grp. gram.:Pl.}
\end{itemize}
\begin{itemize}
\item {Utilização:Prov.}
\end{itemize}
\begin{itemize}
\item {Utilização:dur.}
\end{itemize}
\begin{itemize}
\item {Proveniência:(De \textunderscore palha\textunderscore )}
\end{itemize}
Lugar ou casa, em que se guarda palha.
Meda de palha.
Armazem de madeira, em que alguns proprietários salineiros recolhem o producto das marinhas. Cf. \textunderscore Museu Technol.\textunderscore , 67.
Povoação de pescadores, á beira-mar, em terreno de jurisdicção marítima: \textunderscore os palheiros do Furadoiro\textunderscore .
\section{Palheta}
\begin{itemize}
\item {fónica:lhê}
\end{itemize}
\begin{itemize}
\item {Grp. gram.:f.}
\end{itemize}
\begin{itemize}
\item {Utilização:Prov.}
\end{itemize}
\begin{itemize}
\item {Utilização:minh.}
\end{itemize}
\begin{itemize}
\item {Utilização:Pop.}
\end{itemize}
\begin{itemize}
\item {Grp. gram.:Loc.}
\end{itemize}
\begin{itemize}
\item {Utilização:pop.}
\end{itemize}
\begin{itemize}
\item {Proveniência:(De \textunderscore palha\textunderscore )}
\end{itemize}
Lâmína de metal ou madeira, que, em certos instrumentos de sopro, dá as vibrações do som.
Paleta.
Peça, em que tocam os dentes da roda mais pequena dos relógios.
Espécie de cunha, que aperta o encedoiro contra o pírtigo.
Propulsor das rodas hydráulicas.
Pau, com que se joga a péla.
Lâmina de madeira, com que os entalhadores e esculptores modelam obras de gêsso ou de outra substância malleável.
Peça da urdideira, com orifícios por onde passam os fios do ramo e que também se chama espadilha.
Calçado, bota.
\textunderscore Passar as palhetas\textunderscore , esgueirar-se, escapulir-se. Cf. Castilho, \textunderscore D. Quixote\textunderscore , II, 87.
\section{Palhetada}
\begin{itemize}
\item {Grp. gram.:f.}
\end{itemize}
\begin{itemize}
\item {Utilização:Fig.}
\end{itemize}
Som, produzido pela palheta.
Movimento da palheta.
Obra de um momento: \textunderscore resolve uma questão em duas palhetadas\textunderscore .
\section{Palhetão}
\begin{itemize}
\item {Grp. gram.:m.}
\end{itemize}
\begin{itemize}
\item {Proveniência:(De \textunderscore palheta\textunderscore )}
\end{itemize}
Parte da chave, que impelle a lingueta da fechadura.
Palheta grande, ou coisa parecida:«\textunderscore ...chamalote guarnecido de palhetões de prata\textunderscore ». Fr. Alex. Paixão.
\section{Palhetar}
\begin{itemize}
\item {Grp. gram.:v. i.}
\end{itemize}
\begin{itemize}
\item {Utilização:Des.}
\end{itemize}
\begin{itemize}
\item {Proveniência:(De \textunderscore palheta\textunderscore , que se relaciona com \textunderscore palito\textunderscore , donde veio a loc. \textunderscore servir de palito\textunderscore , dar-se ao desfrute. Cp. \textunderscore palhetear\textunderscore )}
\end{itemize}
Entreter-se? fazer perguntas capciosas?«\textunderscore ...vinha da Inquisição buscar-me um sbirro, porque os clérigos tristes, a seu gôsto, comigo palhetassem\textunderscore ». Filinto, III, 25.
\section{Palhetaria}
\begin{itemize}
\item {Grp. gram.:f.}
\end{itemize}
\begin{itemize}
\item {Utilização:Mús.}
\end{itemize}
Collecção dos registos de órgão, cujos tubos produzem o som por meio de palhetas.
\section{Palhete}
\begin{itemize}
\item {fónica:lhê}
\end{itemize}
\begin{itemize}
\item {Grp. gram.:adj.}
\end{itemize}
\begin{itemize}
\item {Proveniência:(De \textunderscore palha\textunderscore )}
\end{itemize}
Que tem côr de palha; que é pouco carregado na côr, (falando-se do vinho).
\section{Palhete}
\begin{itemize}
\item {fónica:lhê}
\end{itemize}
\begin{itemize}
\item {Grp. gram.:adj.}
\end{itemize}
\begin{itemize}
\item {Utilização:Prov.}
\end{itemize}
\begin{itemize}
\item {Utilização:beir.}
\end{itemize}
Espécie de formão estreito.
\section{Palhetear}
\begin{itemize}
\item {Grp. gram.:v. i.}
\end{itemize}
\begin{itemize}
\item {Proveniência:(De \textunderscore palheta\textunderscore )}
\end{itemize}
Conversar, mofando.
Desfrutar a pessôa com quem se está falando.
\section{Palhiçar}
\begin{itemize}
\item {Grp. gram.:v. t.}
\end{itemize}
\begin{itemize}
\item {Utilização:Bras}
\end{itemize}
Construir com palhiço; cobrir de palha ou de colmo.
\section{Palhiço}
\begin{itemize}
\item {Grp. gram.:m.}
\end{itemize}
\begin{itemize}
\item {Grp. gram.:Adj.}
\end{itemize}
Palha traçada ou moída; palha miúda; colmo.
Feito de palha: \textunderscore o capote palhiço dos pastores do norte\textunderscore .
\section{Palhinha}
\begin{itemize}
\item {Grp. gram.:f.}
\end{itemize}
\begin{itemize}
\item {Grp. gram.:Loc.}
\end{itemize}
\begin{itemize}
\item {Utilização:Pop.}
\end{itemize}
Pedaço de palha.
Palha de cadeiras, bancos, etc.
\textunderscore Tirar palhinha\textunderscore , desfrutar, fazer mofa.
\section{Palhito}
\begin{itemize}
\item {Grp. gram.:m.}
\end{itemize}
\begin{itemize}
\item {Utilização:Prov.}
\end{itemize}
\begin{itemize}
\item {Utilização:trasm.}
\end{itemize}
Palito, fósforo.
\section{Palhoça}
\begin{itemize}
\item {Grp. gram.:f.}
\end{itemize}
Casa coberta de palha, palhal; palhota.
Capa de palha.
\section{Palhoceiro}
\begin{itemize}
\item {Grp. gram.:m.}
\end{itemize}
\begin{itemize}
\item {Proveniência:(De \textunderscore palhoça\textunderscore )}
\end{itemize}
Aquelle que faz capas de palha.
\section{Palhoso}
\begin{itemize}
\item {Grp. gram.:adj.}
\end{itemize}
Relativo ou semelhante á palha. Cf. F. Lapa, \textunderscore Chím. Agr.\textunderscore , 361.
\section{Palhota}
\begin{itemize}
\item {Grp. gram.:f.}
\end{itemize}
\begin{itemize}
\item {Proveniência:(De \textunderscore palha\textunderscore )}
\end{itemize}
Capa de palha, com que se cobrem homens do campo, para se resguardarem da chuva.
Habitação de negros, na África oriental; palhoça.
\section{Palhote}
\begin{itemize}
\item {Grp. gram.:m.}
\end{itemize}
O mesmo que \textunderscore palhoça\textunderscore .
\section{Palhouco}
\begin{itemize}
\item {Grp. gram.:adj.}
\end{itemize}
\begin{itemize}
\item {Utilização:Açor}
\end{itemize}
Pateta, idiota.
\section{Palhuço}
\begin{itemize}
\item {Grp. gram.:m.}
\end{itemize}
\begin{itemize}
\item {Utilização:Prov.}
\end{itemize}
\begin{itemize}
\item {Utilização:minh.}
\end{itemize}
Palha miúda e moída, palhiço.
\section{Páli}
\begin{itemize}
\item {Grp. gram.:m.}
\end{itemize}
\begin{itemize}
\item {Grp. gram.:adj.}
\end{itemize}
Língua sagrada do Ceilão.
Relativo a essa língua.
\section{Paliação}
\begin{itemize}
\item {Grp. gram.:f.}
\end{itemize}
Acto de paliar.
\section{Paliador}
\begin{itemize}
\item {Grp. gram.:m.  e  adj.}
\end{itemize}
O que palia.
\section{Paliar}
\begin{itemize}
\item {Grp. gram.:v. t.}
\end{itemize}
\begin{itemize}
\item {Grp. gram.:V. i.}
\end{itemize}
\begin{itemize}
\item {Proveniência:(Lat. \textunderscore palliare\textunderscore )}
\end{itemize}
Encobrir com falsa aparência, disfarçar.
Atenuar.
Remediar provisoriamente; entreter; adiar.
Empregar paliativos.
\section{Paliativo}
\begin{itemize}
\item {Grp. gram.:adj.}
\end{itemize}
\begin{itemize}
\item {Grp. gram.:m.}
\end{itemize}
Que serve para paliar.
Medicamento que, sem curar radicalmente um mal, lhe atenua as manifestações, adiando um desenlace funesto.
Delonga, que mantém uma espectativa.
Aquilo que entretém indefinidamente uma esperança ou um desejo.
\section{Paliçada}
\begin{itemize}
\item {Grp. gram.:f.}
\end{itemize}
\begin{itemize}
\item {Proveniência:(Do b. lat. \textunderscore palicia\textunderscore )}
\end{itemize}
Estacada para defesa.
Arena, para lutas, torneios ou justas.
\section{Palicário}
\begin{itemize}
\item {Grp. gram.:m.}
\end{itemize}
\begin{itemize}
\item {Proveniência:(Do gr. mod. \textunderscore pallikarion\textunderscore )}
\end{itemize}
Miliciano grego, na guerra da independência.
\section{Palificação}
\begin{itemize}
\item {Grp. gram.:f.}
\end{itemize}
Acto de palificar.
\section{Palificar}
\begin{itemize}
\item {Grp. gram.:v. t.}
\end{itemize}
\begin{itemize}
\item {Utilização:Neol.}
\end{itemize}
\begin{itemize}
\item {Proveniência:(Do lat. \textunderscore palus\textunderscore  + \textunderscore facere\textunderscore )}
\end{itemize}
Segurar com estacas.
\section{Palilho}
\begin{itemize}
\item {Grp. gram.:m.}
\end{itemize}
Rôlo, em que os tintureiros enfiam as meadas, para as enxugarem.
(Cp. cast. \textunderscore pallillo\textunderscore )
\section{Palília}
\begin{itemize}
\item {Grp. gram.:f.}
\end{itemize}
\begin{itemize}
\item {Proveniência:(Lat. \textunderscore palilia\textunderscore )}
\end{itemize}
Festa dos pastores, que se celebrava em Roma a 21 de Abril, anniversário da fundação da cidade. Cf. Castilho, \textunderscore Fastos\textunderscore , II, 203.
\section{Palilo}
\begin{itemize}
\item {Grp. gram.:m.}
\end{itemize}
Árvore fructífera do Brasil.
\section{Palilogia}
\begin{itemize}
\item {Grp. gram.:f.}
\end{itemize}
\begin{itemize}
\item {Proveniência:(Lat. \textunderscore palilogia\textunderscore )}
\end{itemize}
Repetição de uma ideia ou palavra.
\section{Palimpséstico}
\begin{itemize}
\item {Grp. gram.:adj.}
\end{itemize}
Relativo a palimpsesto.
\section{Palimpsesto}
\begin{itemize}
\item {Grp. gram.:m.}
\end{itemize}
\begin{itemize}
\item {Proveniência:(Lat. \textunderscore palimpsestus\textunderscore )}
\end{itemize}
Manuscrito em pergaminho, raspado por copistas da Idade-Média, para dar lugar a nova escrita, debaixo da qual se tem conseguido modernamente avivar os primeiros caracteres. Cf. Castilho, \textunderscore Fastos\textunderscore , I, 512.
\section{Palindromia}
\begin{itemize}
\item {Grp. gram.:f.}
\end{itemize}
\begin{itemize}
\item {Utilização:Med.}
\end{itemize}
Recrudescimento ou recaída de certas doenças, nas quaes, segundo alguns autores, os líquidos se accumulam nos órgãos interiores.
(Cp. \textunderscore palíndromo\textunderscore )
\section{Palíndromo}
\begin{itemize}
\item {Grp. gram.:adj.}
\end{itemize}
\begin{itemize}
\item {Proveniência:(Gr. \textunderscore palindromos\textunderscore )}
\end{itemize}
Diz-se do verso ou phrase, que tem o mesmo sentido, quer se leia de esquerda para a direita, quer da direita para a esquerda.
\section{Palingenesia}
\begin{itemize}
\item {Grp. gram.:f.}
\end{itemize}
\begin{itemize}
\item {Proveniência:(Do gr. \textunderscore palin\textunderscore  + \textunderscore genesis\textunderscore )}
\end{itemize}
Renascimento.
Regeneração.
Systema philosóphico, segundo o qual as revoluções se reproduzem em determinada ordem.
Artifício óptico, que faz apparecer um objecto num lugar onde realmente não há objecto algum.
\section{Palinódia}
\begin{itemize}
\item {Grp. gram.:f.}
\end{itemize}
\begin{itemize}
\item {Proveniência:(Do gr. \textunderscore palin\textunderscore  + \textunderscore ode\textunderscore )}
\end{itemize}
Retractação, feita num poema, daquillo que se disse no outro.
Retractação.
\section{Palinódico}
\begin{itemize}
\item {Grp. gram.:adj.}
\end{itemize}
Relativo á palinódia.
\section{Palinodista}
\begin{itemize}
\item {Grp. gram.:m.}
\end{itemize}
\begin{itemize}
\item {Proveniência:(De \textunderscore palinódia\textunderscore )}
\end{itemize}
Aquelle que faz palinódias.
Aquelle que se desdiz.
\section{Palinuro}
\begin{itemize}
\item {Grp. gram.:m.}
\end{itemize}
\begin{itemize}
\item {Utilização:Poét.}
\end{itemize}
\begin{itemize}
\item {Utilização:Bras}
\end{itemize}
\begin{itemize}
\item {Proveniência:(De \textunderscore Palinuro\textunderscore , n. p.)}
\end{itemize}
Piloto.
Instrumento náutico. Cf. \textunderscore Tarifa das Alfândegas\textunderscore , do Brasil, 105.
\section{Palissandra}
\begin{itemize}
\item {Grp. gram.:f.}
\end{itemize}
O mesmo e mais usado que \textunderscore palissandro\textunderscore .
\section{Palissandro}
\begin{itemize}
\item {Grp. gram.:m.}
\end{itemize}
Árvore da zona tórrida.
Madeira dessa árvore, empregada especialmente em trabalhos de ebanistas.
\section{Palista}
\begin{itemize}
\item {Grp. gram.:m.}
\end{itemize}
Aquelle que é perito em páli. Cf. Vasc. Abreu, \textunderscore Contos da Índia\textunderscore .
\section{Palitar}
\begin{itemize}
\item {Grp. gram.:v. t.}
\end{itemize}
Limpar com um palito; esgaravatar.
\section{Paliteira}
\begin{itemize}
\item {Grp. gram.:f.}
\end{itemize}
Planta umbellífera, também conhecida por \textunderscore bisnaga das searas\textunderscore , (\textunderscore ammi visnaga\textunderscore , Lam.).
\section{Paliteiro}
\begin{itemize}
\item {Grp. gram.:m.}
\end{itemize}
Fabricante ou vendedor de palitos.
Objecto, onde se collocam os palitos para serviço de mesa.
\section{Palito}
\begin{itemize}
\item {Grp. gram.:m.}
\end{itemize}
\begin{itemize}
\item {Utilização:Pop.}
\end{itemize}
\begin{itemize}
\item {Utilização:Fig.}
\end{itemize}
\begin{itemize}
\item {Utilização:Gír.}
\end{itemize}
\begin{itemize}
\item {Utilização:Gír.}
\end{itemize}
\begin{itemize}
\item {Grp. gram.:Pl.}
\end{itemize}
\begin{itemize}
\item {Grp. gram.:Loc.}
\end{itemize}
\begin{itemize}
\item {Utilização:fam.}
\end{itemize}
\begin{itemize}
\item {Proveniência:(T. cast. de \textunderscore palo\textunderscore , pau)}
\end{itemize}
Pauzinho aguçado, com que se limpam ou esgaravatam os dentes.
Pequeno bolo, chato, e comprido.
Fósforo, especialmente o de enxôfre.
Pessôa muito magra.
Punhal.
Cigarro.
Pontas de boi.
\textunderscore Dança dos palitos\textunderscore , (v. \textunderscore palotes\textunderscore ).
\textunderscore Servir de palito\textunderscore , sêr desfrutado, servir de recreio a outrem, sêr objecto de mofa.
\section{Paliúro}
\begin{itemize}
\item {Grp. gram.:m.}
\end{itemize}
\begin{itemize}
\item {Proveniência:(Lat. \textunderscore paliurus\textunderscore )}
\end{itemize}
Planta rhamnácea, o mesmo que \textunderscore espinheiro\textunderscore  (\textunderscore paliurnus aculeatus\textunderscore ).
\section{Palla}
\begin{itemize}
\item {Grp. gram.:f.}
\end{itemize}
(V. \textunderscore pala\textunderscore ^1)
\section{Palla}
\begin{itemize}
\item {Grp. gram.:f.}
\end{itemize}
Espécie de embarcação asiática.
\section{Palladamina}
\begin{itemize}
\item {Grp. gram.:f.}
\end{itemize}
\begin{itemize}
\item {Utilização:Chím.}
\end{itemize}
\begin{itemize}
\item {Proveniência:(De \textunderscore pallas\textunderscore , \textunderscore pallados\textunderscore  gr. \textunderscore aminas\textunderscore )}
\end{itemize}
Substância crystallizável, escura, de aspecto resinoso, a qual faz precipitar a base dos saes de cobre e de prata.
\section{Palladiamina}
\begin{itemize}
\item {Grp. gram.:f.}
\end{itemize}
\begin{itemize}
\item {Utilização:Chím.}
\end{itemize}
Substância que dá combinações, análogas á palladamina.
(Cp. \textunderscore palladamina\textunderscore )
\section{Palladinito}
\begin{itemize}
\item {Grp. gram.:m.}
\end{itemize}
\begin{itemize}
\item {Utilização:Miner.}
\end{itemize}
Óxydo de palládio.
\section{Palládio}
\begin{itemize}
\item {Grp. gram.:m.}
\end{itemize}
\begin{itemize}
\item {Utilização:Fig.}
\end{itemize}
\begin{itemize}
\item {Proveniência:(Do gr. \textunderscore palladion\textunderscore )}
\end{itemize}
Estátua de Pallas, venerada pelos Troianos, como penhor da sua conservação.
Salvaguarda; protecção.
Metal simples, da côr do chumbo e pouco fusível.
\section{Pallanesthesia}
\begin{itemize}
\item {Grp. gram.:f.}
\end{itemize}
\begin{itemize}
\item {Utilização:Med.}
\end{itemize}
\begin{itemize}
\item {Proveniência:(De \textunderscore pallo\textunderscore  gr. + \textunderscore anesthesia\textunderscore )}
\end{itemize}
Abolição da sensibilidade vibratória.
\section{Pallejar}
\begin{itemize}
\item {Grp. gram.:v. i.}
\end{itemize}
\begin{itemize}
\item {Utilização:Bras}
\end{itemize}
\begin{itemize}
\item {Utilização:Neol.}
\end{itemize}
Vocábulo, proposto indevidamente por Alencar, em vez de \textunderscore pallidejar\textunderscore .
\section{Pallente}
\begin{itemize}
\item {Grp. gram.:adj.}
\end{itemize}
\begin{itemize}
\item {Utilização:Poét.}
\end{itemize}
\begin{itemize}
\item {Proveniência:(Lat. \textunderscore pallens\textunderscore )}
\end{itemize}
Que se mostra pállido; que pallideja.
O mesmo que \textunderscore pállido\textunderscore . Cf. Castilho, \textunderscore Geórgicas\textunderscore , 217.
\section{Pallesthesia}
\begin{itemize}
\item {Grp. gram.:f.}
\end{itemize}
\begin{itemize}
\item {Proveniência:(De \textunderscore pallo\textunderscore  gr. + \textunderscore esthesia\textunderscore )}
\end{itemize}
Sensibilidade vibratória.
\section{Palliação}
\begin{itemize}
\item {Grp. gram.:f.}
\end{itemize}
Acto de palliar.
\section{Palliador}
\begin{itemize}
\item {Grp. gram.:m.  e  adj.}
\end{itemize}
O que pallia.
\section{Palliar}
\begin{itemize}
\item {Grp. gram.:v. t.}
\end{itemize}
\begin{itemize}
\item {Grp. gram.:V. i.}
\end{itemize}
\begin{itemize}
\item {Proveniência:(Lat. \textunderscore palliare\textunderscore )}
\end{itemize}
Encobrir com falsa apparência, disfarçar.
Attenuar.
Remediar provisoriamente; entreter; adiar.
Empregar palliativos.
\section{Palliativo}
\begin{itemize}
\item {Grp. gram.:adj.}
\end{itemize}
\begin{itemize}
\item {Grp. gram.:m.}
\end{itemize}
Que serve para palliar.
Medicamento que, sem curar radicalmente um mal, lhe attenua as manifestações, adiando um desenlace funesto.
Delonga, que mantém uma espectativa.
Aquillo que entretém indefinidamente uma esperança ou um desejo.
\section{Palidejar}
\begin{itemize}
\item {Grp. gram.:v. i.}
\end{itemize}
Mostrar-se pálido.
Têr côr pálida: \textunderscore a lua palideja\textunderscore .
\section{Palidez}
\begin{itemize}
\item {Grp. gram.:f.}
\end{itemize}
Estado de pálido ou descòrado.
\section{Pálido}
\begin{itemize}
\item {Grp. gram.:adj.}
\end{itemize}
\begin{itemize}
\item {Proveniência:(Lat. \textunderscore pallidus\textunderscore )}
\end{itemize}
Que perdeu a sua côr viva e animada, (falando-se do rosto ou da pele).
Descòrado.
Pouco intenso, froixo: \textunderscore descripção pálida\textunderscore .
Desmaiado; que não tem colorido.
\section{Pálio}
\begin{itemize}
\item {Grp. gram.:m.}
\end{itemize}
\begin{itemize}
\item {Utilização:Ant.}
\end{itemize}
\begin{itemize}
\item {Utilização:Fig.}
\end{itemize}
\begin{itemize}
\item {Proveniência:(Lat. \textunderscore pallium\textunderscore )}
\end{itemize}
Sôbrecéu portátil, sustentado por varas, que se leva nos cortejos ou procissões, para cobrir a pessôa que se festeja ou o sacerdote que leva a hóstia consagrada.
Ornamento de lan branca, com cruzes pretas, concedido pelo Papa a certos Prelados.
Capa.
Pompa; triunfo.
\section{Paliobrânquio}
\begin{itemize}
\item {Grp. gram.:adj.}
\end{itemize}
\begin{itemize}
\item {Utilização:Zool.}
\end{itemize}
Que tem as brânquias cobertas de uma membrana carnuda.
\section{Pallidejar}
\begin{itemize}
\item {Grp. gram.:v. i.}
\end{itemize}
Mostrar-se pállido.
Têr côr pállida: \textunderscore a lua pallideja\textunderscore .
\section{Pallidez}
\begin{itemize}
\item {Grp. gram.:f.}
\end{itemize}
Estado de pállido ou descòrado.
\section{Pállido}
\begin{itemize}
\item {Grp. gram.:adj.}
\end{itemize}
\begin{itemize}
\item {Proveniência:(Lat. \textunderscore pallidus\textunderscore )}
\end{itemize}
Que perdeu a sua côr viva e animada, (falando-se do rosto ou da pelle).
Descòrado.
Pouco intenso, froixo: \textunderscore descripção pállida\textunderscore .
Desmaiado; que não tem colorido.
\section{Pállio}
\begin{itemize}
\item {Grp. gram.:m.}
\end{itemize}
\begin{itemize}
\item {Utilização:Ant.}
\end{itemize}
\begin{itemize}
\item {Utilização:Fig.}
\end{itemize}
\begin{itemize}
\item {Proveniência:(Lat. \textunderscore pallium\textunderscore )}
\end{itemize}
Sôbrecéu portátil, sustentado por varas, que se leva nos cortejos ou procissões, para cobrir a pessôa que se festeja ou o sacerdote que leva a hóstia consagrada.
Ornamento de lan branca, com cruzes pretas, concedido pelo Papa a certos Prelados.
Capa.
Pompa; triumpho.
\section{Palliobrânchio}
\begin{itemize}
\item {fónica:qui}
\end{itemize}
\begin{itemize}
\item {Grp. gram.:adj.}
\end{itemize}
\begin{itemize}
\item {Utilização:Zool.}
\end{itemize}
Que tem as brânchias cobertas de uma membrana carnuda.
\section{Pallor}
\begin{itemize}
\item {Grp. gram.:m.}
\end{itemize}
\begin{itemize}
\item {Utilização:Poét.}
\end{itemize}
\begin{itemize}
\item {Proveniência:(Lat. \textunderscore pallor\textunderscore )}
\end{itemize}
O mesmo que \textunderscore pallidez\textunderscore .
\section{Pallorejar}
\begin{itemize}
\item {Grp. gram.:v. i.}
\end{itemize}
\begin{itemize}
\item {Utilização:Neol.}
\end{itemize}
Têr pallor, pallidejar.
\section{Palma}
\begin{itemize}
\item {Grp. gram.:f.}
\end{itemize}
\begin{itemize}
\item {Utilização:Prov.}
\end{itemize}
\begin{itemize}
\item {Utilização:trasm.}
\end{itemize}
\begin{itemize}
\item {Utilização:Fig.}
\end{itemize}
\begin{itemize}
\item {Grp. gram.:Pl.}
\end{itemize}
\begin{itemize}
\item {Proveniência:(Lat. \textunderscore palma\textunderscore )}
\end{itemize}
Ramo de palmeira.
Palmeira.
\textunderscore Palma da mão\textunderscore , parte côncava entre o pulso e os dedos.
O mesmo que \textunderscore palmatoada\textunderscore , e o mesmo que \textunderscore palmatória\textunderscore .
Victória.
\textunderscore Levar a palma\textunderscore , sobrelevar, levar as lampas, avantajar-se.
Triumpho, applausos.
\textunderscore Dar palmas\textunderscore , applaudir.
\textunderscore Trazer nas palmas\textunderscore , tratar com muito cuidado, desvelo ou amor.
\section{Palmáceas}
\begin{itemize}
\item {Grp. gram.:f. pl.}
\end{itemize}
Família de plantas, que tem por typo a tamareira.
(Fem. pl. de \textunderscore palmáceo\textunderscore )
\section{Palmáceo}
\begin{itemize}
\item {Grp. gram.:adj.}
\end{itemize}
\begin{itemize}
\item {Proveniência:(De \textunderscore palma\textunderscore )}
\end{itemize}
Relativo ou semelhante á tamareira.
\section{Palma-Chrísti}
\begin{itemize}
\item {Grp. gram.:m.}
\end{itemize}
\begin{itemize}
\item {Proveniência:(Do lat. \textunderscore palma\textunderscore  + \textunderscore Christus\textunderscore )}
\end{itemize}
O mesmo que \textunderscore rícino\textunderscore .
\section{Palmada}
\begin{itemize}
\item {Grp. gram.:f.}
\end{itemize}
Pancada com a palma da mão.
\section{Palmanço}
\begin{itemize}
\item {Grp. gram.:m.}
\end{itemize}
\begin{itemize}
\item {Utilização:Gír.}
\end{itemize}
\begin{itemize}
\item {Proveniência:(De \textunderscore palmar\textunderscore )}
\end{itemize}
Furto.
\section{Palmar}
\begin{itemize}
\item {Grp. gram.:v. t.}
\end{itemize}
\begin{itemize}
\item {Utilização:Chul.}
\end{itemize}
\begin{itemize}
\item {Grp. gram.:m.}
\end{itemize}
\begin{itemize}
\item {Grp. gram.:Adj.}
\end{itemize}
\begin{itemize}
\item {Utilização:Fig.}
\end{itemize}
\begin{itemize}
\item {Grp. gram.:M.}
\end{itemize}
\begin{itemize}
\item {Proveniência:(De \textunderscore palma\textunderscore )}
\end{itemize}
Furtar, bifar.
Terreno, em que crescera palmeiras.
Povoação entre palmeiras.
Relativo á palma da mão.
Que é do comprimento de um palmo.
Evidente; grande, desconforme: \textunderscore erros palmares\textunderscore .
O mesmo que \textunderscore palmeiral\textunderscore .
\section{Palmati...}
\begin{itemize}
\item {Grp. gram.:pref.}
\end{itemize}
\begin{itemize}
\item {Proveniência:(Do lat. \textunderscore palmatus\textunderscore )}
\end{itemize}
(que significa \textunderscore dividido como os dedos da mão\textunderscore )
\section{Palmatífido}
\begin{itemize}
\item {Grp. gram.:adj.}
\end{itemize}
\begin{itemize}
\item {Utilização:Bot.}
\end{itemize}
\begin{itemize}
\item {Proveniência:(Do lat. \textunderscore palmatus\textunderscore  + \textunderscore findere\textunderscore )}
\end{itemize}
Diz-se da fôlha e, excepcionalmente, de outro órgão vegetal, quando as suas divisões se prolongam até meio do limbo, dando-lhe aspecto de palma.
\section{Palmatifloro}
\begin{itemize}
\item {Grp. gram.:adj.}
\end{itemize}
\begin{itemize}
\item {Utilização:Bot.}
\end{itemize}
\begin{itemize}
\item {Proveniência:(Do lat. \textunderscore palmatus\textunderscore  + \textunderscore flos\textunderscore )}
\end{itemize}
Que tem corolla em fórma de palma.
\section{Palmatifoliado}
\begin{itemize}
\item {Grp. gram.:adj.}
\end{itemize}
\begin{itemize}
\item {Utilização:Bot.}
\end{itemize}
\begin{itemize}
\item {Proveniência:(Do lat. \textunderscore palmatus\textunderscore  + \textunderscore folium\textunderscore )}
\end{itemize}
Que tem fôlhas em fórma de palma.
\section{Palmatiforme}
\begin{itemize}
\item {Grp. gram.:adj.}
\end{itemize}
\begin{itemize}
\item {Utilização:Bot.}
\end{itemize}
\begin{itemize}
\item {Proveniência:(Do lat. \textunderscore palmatus\textunderscore  + \textunderscore forma\textunderscore )}
\end{itemize}
Diz-se da corolla, cujas fôlhas parecem têr, mas não têm, fórma de palma.
\section{Palmatilobado}
\begin{itemize}
\item {Grp. gram.:adj.}
\end{itemize}
\begin{itemize}
\item {Utilização:Bot.}
\end{itemize}
\begin{itemize}
\item {Proveniência:(Do lat. \textunderscore palmatus\textunderscore  + gr. \textunderscore lobos\textunderscore )}
\end{itemize}
Que tem lóbulos na superfície.
\section{Palmatinérveo}
\begin{itemize}
\item {Grp. gram.:adj.}
\end{itemize}
\begin{itemize}
\item {Utilização:Bot.}
\end{itemize}
\begin{itemize}
\item {Proveniência:(Do lat. \textunderscore palmatus\textunderscore  + \textunderscore nervus\textunderscore )}
\end{itemize}
Que tem nervuras em fórma de palma.
\section{Palmatipartido}
\begin{itemize}
\item {Grp. gram.:adj.}
\end{itemize}
\begin{itemize}
\item {Utilização:Bot.}
\end{itemize}
\begin{itemize}
\item {Proveniência:(De \textunderscore palmati...\textunderscore  + \textunderscore partido\textunderscore )}
\end{itemize}
Diz-se do órgão vegetal, cujas divisões chegam até á base.
\section{Palmato}
\begin{itemize}
\item {Grp. gram.:m.}
\end{itemize}
\begin{itemize}
\item {Utilização:Chím.}
\end{itemize}
\begin{itemize}
\item {Proveniência:(De \textunderscore palma-Chrísti\textunderscore )}
\end{itemize}
Combinação do ácido pálmico com os álcalis ou com os radicaes alcoólicos.
\section{Palmatoada}
\begin{itemize}
\item {Grp. gram.:f.}
\end{itemize}
\begin{itemize}
\item {Proveniência:(Do rad. de \textunderscore palmatória\textunderscore )}
\end{itemize}
Pancada de palmatória.
\section{Palmatória}
\begin{itemize}
\item {Grp. gram.:f.}
\end{itemize}
\begin{itemize}
\item {Utilização:Bras}
\end{itemize}
\begin{itemize}
\item {Proveniência:(Lat. \textunderscore palmatoria\textunderscore )}
\end{itemize}
Instrumento de madeira, com que se bate nas palmas das mãos, por castigo.
Utensílio, composto de uma espécie de prato, com um bocal, em que se sustenta e accende uma vela.
Peça de ferro, para arredondar o fundo dos copos de vidro.
Espécie de cacto.
\section{Palmatória-do-inferno}
\begin{itemize}
\item {Grp. gram.:m.}
\end{itemize}
Árvore, o mesmo que \textunderscore figueira-da-índia\textunderscore .
\section{Palmatorres}
\begin{itemize}
\item {fónica:tô}
\end{itemize}
\begin{itemize}
\item {Grp. gram.:m.}
\end{itemize}
\begin{itemize}
\item {Utilização:Pesc.}
\end{itemize}
Parte da rêde rabeira, nas armações de atum.
\section{Palmear}
\begin{itemize}
\item {Grp. gram.:v. t.}
\end{itemize}
\begin{itemize}
\item {Grp. gram.:v. i.}
\end{itemize}
Applaudir, batendo as palmas das mãos.
Impellir com a mão (um pequeno barco).
Bater as palmas, applaudindo.
\section{Palmeira}
\begin{itemize}
\item {Grp. gram.:f.}
\end{itemize}
\begin{itemize}
\item {Proveniência:(De \textunderscore palma\textunderscore )}
\end{itemize}
Nome commum a todas as palmáceas.
Tamareira, (\textunderscore phoenix dactylifera\textunderscore , Lin.).
\section{Palmeira-anan}
\begin{itemize}
\item {Grp. gram.:f.}
\end{itemize}
O mesmo que \textunderscore palmeira-das-vassoiras\textunderscore .
\section{Palmeira-das-vassoiras}
\begin{itemize}
\item {Grp. gram.:f.}
\end{itemize}
Variedade de palmeira, (\textunderscore chamaerops humilis\textunderscore , Lin.), frequente no Algarve. Cf. P. Coutinho, \textunderscore Flora de Port.\textunderscore , 115.
\section{Palmeiral}
\begin{itemize}
\item {Grp. gram.:m.}
\end{itemize}
Bosque de palmeiras.
\section{Palmeireiro}
\begin{itemize}
\item {Grp. gram.:m.}
\end{itemize}
Aquelle que, na Índia portuguesa, se occupa da cultura das palmeiras.
\section{Palmeirim}
\begin{itemize}
\item {Grp. gram.:m.}
\end{itemize}
\begin{itemize}
\item {Utilização:Ant.}
\end{itemize}
\begin{itemize}
\item {Proveniência:(De \textunderscore palma\textunderscore )}
\end{itemize}
Estranjeiro, peregrino.
\section{Palmeiro}
\begin{itemize}
\item {Grp. gram.:m.}
\end{itemize}
O mesmo que \textunderscore palmeirim\textunderscore .
\section{Palmeirô}
\begin{itemize}
\item {Grp. gram.:adj.}
\end{itemize}
Que tem proximamente um palmo do comprido: \textunderscore robalo palmeiro\textunderscore .
\section{Palmejar}
\begin{itemize}
\item {Grp. gram.:v. i.}
\end{itemize}
\begin{itemize}
\item {Grp. gram.:M.}
\end{itemize}
\begin{itemize}
\item {Proveniência:(De \textunderscore palma\textunderscore )}
\end{itemize}
O mesmo que \textunderscore palmear\textunderscore .
Prancha, que reveste interiormente o arcaboiço do navio.
\section{Palmela}
\begin{itemize}
\item {Grp. gram.:f.}
\end{itemize}
\begin{itemize}
\item {Utilização:Bras}
\end{itemize}
\begin{itemize}
\item {Grp. gram.:M.}
\end{itemize}
\begin{itemize}
\item {Utilização:T. de Turquel}
\end{itemize}
Vegetações dos pântanos.
Homem espalmado, magrizella.
(Do \textunderscore palma\textunderscore )
\section{Palmelão}
\begin{itemize}
\item {Grp. gram.:m.  e  adj.}
\end{itemize}
\begin{itemize}
\item {Proveniência:(De \textunderscore Palmella\textunderscore , n. p.)}
\end{itemize}
Vento rijo que sopra de Palmella para Lisbôa.
\section{Palmeta}
\begin{itemize}
\item {fónica:mê}
\end{itemize}
\begin{itemize}
\item {Grp. gram.:f.}
\end{itemize}
\begin{itemize}
\item {Utilização:Fam.}
\end{itemize}
\begin{itemize}
\item {Utilização:Carp.}
\end{itemize}
\begin{itemize}
\item {Proveniência:(De \textunderscore palma\textunderscore )}
\end{itemize}
Cunha, com que se faz levantar ou abaixar a culatra da peça de artilharia.
Peça delgada, com que se aperfeiçôa o furo, feito pelo puncção dos serralheiros.
Pancada na palma da mão, por castigo.
Espécie de postigo no tomadoiro das marinhas, para abrir ou fechar a communicação entre o rio e o viveiro.
Lâmina de ferro, para auxiliar a firmeza da cunha, com que se quere abrir uma pedra.
Pequena cunha de madeira.
Peixe das costas do Algarve, de carne avermelhada e saborosa.
\section{Palmetear}
\begin{itemize}
\item {Grp. gram.:v.}
\end{itemize}
\begin{itemize}
\item {Utilização:t. Carp.}
\end{itemize}
\begin{itemize}
\item {Proveniência:(De \textunderscore palmeta\textunderscore )}
\end{itemize}
Meter palmetas nos furos das peças de (caixilho ou porta), quando essas peças se grudam.
\section{Palmicheio}
\begin{itemize}
\item {Grp. gram.:adj.}
\end{itemize}
\begin{itemize}
\item {Proveniência:(De \textunderscore palma\textunderscore  + \textunderscore cheio\textunderscore )}
\end{itemize}
Diz-se do casco do pé dos solípedes, quando a sua face plantar é convexa, excedendo o nível do bôrdo circular inferior do mesmo casco.
\section{Pálmico}
\begin{itemize}
\item {Grp. gram.:adj.}
\end{itemize}
\begin{itemize}
\item {Proveniência:(De \textunderscore palma-Chrísti\textunderscore )}
\end{itemize}
Diz-se de um ácido, proveniente da modificação molecular que experimenta o ácido ricinólico, quando sujeito á acção dos vapores nitrosos.
\section{Palmífero}
\begin{itemize}
\item {Grp. gram.:adj.}
\end{itemize}
\begin{itemize}
\item {Proveniência:(Lat. \textunderscore palmifer\textunderscore )}
\end{itemize}
Que produz palmeiras; abundante de palmeiras.
\section{Palmiforme}
\begin{itemize}
\item {Grp. gram.:adj.}
\end{itemize}
\begin{itemize}
\item {Proveniência:(Do lat. \textunderscore palma\textunderscore  + \textunderscore forma\textunderscore )}
\end{itemize}
Semelhante á palma.
\section{Palmilha}
\begin{itemize}
\item {Grp. gram.:f.}
\end{itemize}
Revestimento interior da sola do calçado.
Parte da meia, que cobre a planta do pé.
Tecido antigo.
(Cast. \textunderscore palmilla\textunderscore )
\section{Palmilhadeira}
\begin{itemize}
\item {Grp. gram.:f.}
\end{itemize}
Mulher que palmilha meias.
\section{Palmilhar}
\begin{itemize}
\item {Grp. gram.:v. t.}
\end{itemize}
\begin{itemize}
\item {Grp. gram.:V. i.}
\end{itemize}
Pôr palmilhas em.
Percorrer a pé.
Calcar com os pés.
Andar a pé. Cf. Castilho, \textunderscore Fausto\textunderscore , 329.
\section{Palmina}
\begin{itemize}
\item {Grp. gram.:f.}
\end{itemize}
\begin{itemize}
\item {Utilização:Chím.}
\end{itemize}
Substância, resultante da acção do ácido azótico sôbre o óleo de rícino.
(Cp. \textunderscore pálmico\textunderscore )
\section{Palminervado}
\begin{itemize}
\item {Grp. gram.:adj.}
\end{itemize}
\begin{itemize}
\item {Utilização:Bot.}
\end{itemize}
\begin{itemize}
\item {Proveniência:(De \textunderscore palma\textunderscore  + \textunderscore nervo\textunderscore )}
\end{itemize}
Em que, com a nervura principal, partem do pecíolo outras nervuras, divergentes como os dedos de uma ave.
\section{Palminhas}
\begin{itemize}
\item {Grp. gram.:f. pl.}
\end{itemize}
\begin{itemize}
\item {Utilização:Infant.}
\end{itemize}
\begin{itemize}
\item {Proveniência:(De \textunderscore palma\textunderscore )}
\end{itemize}
Us. na loc. \textunderscore dar palminhas\textunderscore , dar palmas, applaudir.
\textunderscore Trazer nas palminhas\textunderscore , tratar com muito carinho, apaparicar.
\section{Palmípede}
\begin{itemize}
\item {Grp. gram.:adj.}
\end{itemize}
\begin{itemize}
\item {Grp. gram.:M. pl.}
\end{itemize}
\begin{itemize}
\item {Proveniência:(Lat. \textunderscore palmipes\textunderscore )}
\end{itemize}
Que tem os dedos dos pés unidos por uma membrana.
Ordem de aves palmípedes.
Família de quadrúpedes roedores, a que pertence o castor.
\section{Palmiste}
\begin{itemize}
\item {Grp. gram.:m.}
\end{itemize}
Espécie de palmeira; fruto dessa árvore.
Óleo de palma.
\section{Palmital}
\begin{itemize}
\item {Grp. gram.:m.}
\end{itemize}
Terreno, onde crescem palmitos; palmar.
\section{Palmitato}
\begin{itemize}
\item {Grp. gram.:m.}
\end{itemize}
\begin{itemize}
\item {Proveniência:(De \textunderscore palmítico\textunderscore )}
\end{itemize}
Combinação do ácido palmítico com uma base.
\section{Palmiteira}
\begin{itemize}
\item {Grp. gram.:f.}
\end{itemize}
O mesmo que \textunderscore palmitiqueira\textunderscore .
\section{Palmiteso}
\begin{itemize}
\item {fónica:tê}
\end{itemize}
\begin{itemize}
\item {Grp. gram.:adj.}
\end{itemize}
\begin{itemize}
\item {Proveniência:(De \textunderscore palma\textunderscore  + \textunderscore teso\textunderscore )}
\end{itemize}
Diz-se do casco do pé dos solípedes, quando a superfície plantar é plana e ao nível do bôrdo circular inferior do mesmo casco.
\section{Palmítico}
\begin{itemize}
\item {Grp. gram.:adj.}
\end{itemize}
Diz-se de um ácido, que constitue o principal elemento das velas de estearina.
\section{Palmitina}
\begin{itemize}
\item {Grp. gram.:f.}
\end{itemize}
\begin{itemize}
\item {Proveniência:(De \textunderscore palmito\textunderscore )}
\end{itemize}
Substância particular, que se acha no óleo de palma.
\section{Palmitiqueira}
\begin{itemize}
\item {Grp. gram.:f.}
\end{itemize}
O mesmo que \textunderscore palmito\textunderscore ^1, árvore.
\section{Palmito}
\begin{itemize}
\item {Grp. gram.:m.}
\end{itemize}
\begin{itemize}
\item {Proveniência:(De \textunderscore palma\textunderscore )}
\end{itemize}
Espécie de palmeira.
Palma.
Ramo ou fôlha de palmeira.
Ramo de palmeira, que se adorna e benze no Domingo de Ramos.
Miolo doce de palmeira.
Ramo de flôres, que as crianças ou as donzelas levam nas mãos, depois de mortas, como sýmbolo de innocência.
Grande ramo de flôres e frutos, usado nalgumas festas.
\section{Palmito}
\begin{itemize}
\item {Grp. gram.:m.}
\end{itemize}
\begin{itemize}
\item {Utilização:T. da Áfr. Occid. Port}
\end{itemize}
Nome de um lagarto das árvores.
\section{Palmitos}
\begin{itemize}
\item {Grp. gram.:m.}
\end{itemize}
Gênero de plantas liliáceas da Índia portuguêsa.
\section{Palmo}
\begin{itemize}
\item {Grp. gram.:m.}
\end{itemize}
\begin{itemize}
\item {Proveniência:(De \textunderscore palma\textunderscore )}
\end{itemize}
Medida, igual á distância entre as extremidades dos dedos pollegar e mínimo de uma mão aberta.
Extensão de 8 pollegadas ou 22 centímetros.
\section{Palmoira}
\begin{itemize}
\item {Grp. gram.:f.}
\end{itemize}
\begin{itemize}
\item {Utilização:Bras}
\end{itemize}
\begin{itemize}
\item {Proveniência:(De \textunderscore palma\textunderscore )}
\end{itemize}
Pé das aves palmípedes.
\section{Palmoneta}
\begin{itemize}
\item {fónica:nê}
\end{itemize}
\begin{itemize}
\item {Grp. gram.:f.}
\end{itemize}
O mesmo que \textunderscore palmonete\textunderscore .
\section{Palmonete}
\begin{itemize}
\item {fónica:nê}
\end{itemize}
\begin{itemize}
\item {Grp. gram.:m.}
\end{itemize}
Peixe marítimo da costa de Portugal.
\section{Palmumás}
\begin{itemize}
\item {Grp. gram.:m. pl.}
\end{itemize}
Indígenas do norte do Brasil.
\section{Paló}
\begin{itemize}
\item {Grp. gram.:m.}
\end{itemize}
\begin{itemize}
\item {Utilização:T. da Índia port}
\end{itemize}
\begin{itemize}
\item {Grp. gram.:Adj.}
\end{itemize}
A colheita dos cocos.
Diz-se do pano ordinário.
\section{Paloguindão}
\begin{itemize}
\item {Grp. gram.:m.}
\end{itemize}
Instrumento guerreiro, entre os Chineses. Cf. \textunderscore Peregrinação\textunderscore , CXVIII.
\section{Paloio}
\begin{itemize}
\item {Grp. gram.:adj.}
\end{itemize}
\begin{itemize}
\item {Utilização:Prov.}
\end{itemize}
\begin{itemize}
\item {Utilização:alg.}
\end{itemize}
Volumoso e informe; grosseiro.
(Cp. \textunderscore maloio\textunderscore )
\section{Paloma}
\begin{itemize}
\item {Grp. gram.:f.}
\end{itemize}
\begin{itemize}
\item {Utilização:Des.}
\end{itemize}
Espécie de cabo náutico.
Fio grosso, o mesmo que \textunderscore palomar\textunderscore .
O mesmo que \textunderscore pomba\textunderscore :«\textunderscore ...a melopêa das palomas de Iduméa.\textunderscore »J. Castilho, \textunderscore Manuelinas\textunderscore , 29.
(Cast. \textunderscore paloma\textunderscore )
\section{Palomadura}
\begin{itemize}
\item {Grp. gram.:f.}
\end{itemize}
Acto de palomar^1.
\section{Palomar}
\begin{itemize}
\item {Grp. gram.:v.}
\end{itemize}
\begin{itemize}
\item {Utilização:t. Náut.}
\end{itemize}
\begin{itemize}
\item {Grp. gram.:M.}
\end{itemize}
Coser os panos de (uma vela).
Fio grosso, com que se cosem as velas.
(Cast. \textunderscore palomar\textunderscore )
\section{Palomar}
\begin{itemize}
\item {Grp. gram.:m.}
\end{itemize}
\begin{itemize}
\item {Utilização:Ant.}
\end{itemize}
\begin{itemize}
\item {Proveniência:(Do cast. \textunderscore paloma\textunderscore )}
\end{itemize}
O mesmo que \textunderscore pombal\textunderscore .
\section{Palomba}
\begin{itemize}
\item {Grp. gram.:f.}
\end{itemize}
\begin{itemize}
\item {Utilização:Náut.}
\end{itemize}
\begin{itemize}
\item {Proveniência:(Do lat. \textunderscore palumba\textunderscore )}
\end{itemize}
Corda da vela do estai.
Fio grosso, com que se cosem velas.
Paloma, palomar.
\section{Palombadura}
\begin{itemize}
\item {Grp. gram.:f.}
\end{itemize}
Acto de palombar.
\section{Palombar}
\begin{itemize}
\item {Grp. gram.:v.}
\end{itemize}
\begin{itemize}
\item {Utilização:t. Náut.}
\end{itemize}
Coser (velas) com palomba.
\section{Palombeta}
\begin{itemize}
\item {fónica:bê}
\end{itemize}
\begin{itemize}
\item {Grp. gram.:f.}
\end{itemize}
\begin{itemize}
\item {Utilização:Bras}
\end{itemize}
Saboroso peixe marítimo.
\section{Palombino}
\begin{itemize}
\item {Grp. gram.:m.}
\end{itemize}
\begin{itemize}
\item {Proveniência:(De \textunderscore palomba\textunderscore )}
\end{itemize}
Mármore branco e fino, que se acha em monumentos antigos.
\section{Palomeira}
\begin{itemize}
\item {Grp. gram.:f.}
\end{itemize}
\begin{itemize}
\item {Utilização:Ant.}
\end{itemize}
O mesmo que \textunderscore paloma\textunderscore , cabo náutico. Cf. Azurara, \textunderscore Chrón. do Conde D. Pedro\textunderscore , c. LVIII, 359.
\section{Palonço}
\begin{itemize}
\item {Grp. gram.:m.  e  adj.}
\end{itemize}
\begin{itemize}
\item {Utilização:Chul.}
\end{itemize}
Pacóvio; imbecil; tolo.
\section{Palonzano}
\begin{itemize}
\item {Grp. gram.:adj.}
\end{itemize}
\begin{itemize}
\item {Utilização:Prov.}
\end{itemize}
\begin{itemize}
\item {Utilização:trasm.}
\end{itemize}
Bruto, estúpido.
(Cp. \textunderscore palonço\textunderscore )
\section{Palor}
\begin{itemize}
\item {Grp. gram.:m.}
\end{itemize}
\begin{itemize}
\item {Utilização:Poét.}
\end{itemize}
\begin{itemize}
\item {Proveniência:(Lat. \textunderscore pallor\textunderscore )}
\end{itemize}
O mesmo que \textunderscore palidez\textunderscore .
\section{Palotes}
\begin{itemize}
\item {Grp. gram.:m. pl.}
\end{itemize}
\begin{itemize}
\item {Proveniência:(Do lat. \textunderscore palum\textunderscore )}
\end{itemize}
\textunderscore Dança dos palotes\textunderscore , ou dos \textunderscore paulitos\textunderscore , ou dos \textunderscore palitos\textunderscore , divertimento us. em terras de Miranda, dançando oito ou déz moços que, armados de bastões, batem com elles nos dos companheiros, ao mesmo tempo que saltam e se cruzam e se voltam.
\section{Palpabilizar}
\begin{itemize}
\item {Grp. gram.:v. t.}
\end{itemize}
\begin{itemize}
\item {Utilização:Neol.}
\end{itemize}
\begin{itemize}
\item {Proveniência:(Do lat. \textunderscore palpabilis\textunderscore )}
\end{itemize}
Tornar palpável; evidenciar. Cr. Alv. Mendes, \textunderscore Discursos\textunderscore , 262.
\section{Palpação}
\begin{itemize}
\item {Grp. gram.:f.}
\end{itemize}
O mesmo que \textunderscore apalpadela\textunderscore .
\section{Palpadela}
\begin{itemize}
\item {Grp. gram.:f.}
\end{itemize}
O mesmo que \textunderscore apalpadela\textunderscore .
\section{Palpar}
\begin{itemize}
\item {Grp. gram.:v. t.}
\end{itemize}
\begin{itemize}
\item {Proveniência:(Lat. \textunderscore palpare\textunderscore )}
\end{itemize}
O mesmo que \textunderscore apalpar\textunderscore .
\section{Palpável}
\begin{itemize}
\item {Grp. gram.:adj.}
\end{itemize}
\begin{itemize}
\item {Utilização:Fig.}
\end{itemize}
Que se póde palpar.
Evidente; que não deixa dúvidas.
\section{Palpavelmente}
\begin{itemize}
\item {Grp. gram.:adv.}
\end{itemize}
De modo palpável; claramente; evidentemente; sem contestação.
\section{Pálpebra}
\begin{itemize}
\item {Grp. gram.:f.}
\end{itemize}
\begin{itemize}
\item {Proveniência:(Lat. \textunderscore palpebra\textunderscore )}
\end{itemize}
Membrana móvel, que cobre externamente o ôlho.
\section{Palpebrado}
\begin{itemize}
\item {Grp. gram.:adj.}
\end{itemize}
Que tem pálpebras.
\section{Palpebral}
\begin{itemize}
\item {Grp. gram.:adj.}
\end{itemize}
Relativo as pálpebras.
\section{Palpebrite}
\begin{itemize}
\item {Grp. gram.:f.}
\end{itemize}
Inflammação da pálpebra.
\section{Palpitação}
\begin{itemize}
\item {Grp. gram.:f.}
\end{itemize}
\begin{itemize}
\item {Proveniência:(Lat. \textunderscore palpitatio\textunderscore )}
\end{itemize}
Acto de palpitar.
\section{Palpitante}
\begin{itemize}
\item {Grp. gram.:adj.}
\end{itemize}
\begin{itemize}
\item {Utilização:Fig.}
\end{itemize}
\begin{itemize}
\item {Proveniência:(Lat. \textunderscore palpitans\textunderscore )}
\end{itemize}
Que palpita; que tem apparência de vida.
Recente e notável: \textunderscore assumptos palpitantes\textunderscore .
\section{Palpitar}
\begin{itemize}
\item {Grp. gram.:v. i.}
\end{itemize}
\begin{itemize}
\item {Grp. gram.:v. i.}
\end{itemize}
\begin{itemize}
\item {Utilização:Fig.}
\end{itemize}
\begin{itemize}
\item {Proveniência:(Lat. \textunderscore palpitare\textunderscore )}
\end{itemize}
Pulsar.
Têr agitação convulsiva.
Commover-se.
Têr sobresalto.
Presentir; suppor: \textunderscore palpita-me que chove hoje\textunderscore .
\section{Palpite}
\begin{itemize}
\item {Grp. gram.:m.}
\end{itemize}
\begin{itemize}
\item {Utilização:fam.}
\end{itemize}
\begin{itemize}
\item {Utilização:Fig.}
\end{itemize}
\begin{itemize}
\item {Proveniência:(De \textunderscore palpitar\textunderscore )}
\end{itemize}
O mesmo que \textunderscore palpitação\textunderscore .
Presentimento.
\section{Palpo}
\begin{itemize}
\item {Grp. gram.:m.}
\end{itemize}
\begin{itemize}
\item {Utilização:Zool.}
\end{itemize}
\begin{itemize}
\item {Proveniência:(Lat. \textunderscore palpus\textunderscore )}
\end{itemize}
Cada um dos dois appêndices articulados da bôca dos insectos.
\section{Palra}
\begin{itemize}
\item {Grp. gram.:f.}
\end{itemize}
\begin{itemize}
\item {Utilização:Pop.}
\end{itemize}
\begin{itemize}
\item {Proveniência:(De \textunderscore palrar\textunderscore )}
\end{itemize}
Conversação; loquacidade; tagarelice.
Palavra.
\section{Palração}
\begin{itemize}
\item {Grp. gram.:f.}
\end{itemize}
\begin{itemize}
\item {Utilização:Des.}
\end{itemize}
\begin{itemize}
\item {Proveniência:(De \textunderscore palrar\textunderscore )}
\end{itemize}
Vozes de muita gente; falario.
\section{Palradeiro}
\begin{itemize}
\item {Grp. gram.:adj.}
\end{itemize}
\begin{itemize}
\item {Proveniência:(De \textunderscore palrar\textunderscore )}
\end{itemize}
O mesmo que \textunderscore palreiro\textunderscore . Cf. Garrett, \textunderscore Flôres sem Fr.\textunderscore , 48.
\section{Palrador}
\begin{itemize}
\item {Grp. gram.:m.  e  adj.}
\end{itemize}
Aquelle que palra; parolador; tagarela.
\section{Palrança}
\begin{itemize}
\item {Grp. gram.:f.}
\end{itemize}
\begin{itemize}
\item {Utilização:Ant.}
\end{itemize}
\begin{itemize}
\item {Utilização:Pop.}
\end{itemize}
O mesmo que \textunderscore palração\textunderscore .
\section{Palrante}
\begin{itemize}
\item {Grp. gram.:m.}
\end{itemize}
\begin{itemize}
\item {Utilização:Gír.}
\end{itemize}
O mesmo que \textunderscore relógio\textunderscore .
\section{Palrar}
\begin{itemize}
\item {Grp. gram.:v. i.}
\end{itemize}
Articular sons, vazios de sentido; chalrear.
Conversar; tagarelar; palestrar.
(Metáth. de \textunderscore parlar\textunderscore )
\section{Palraria}
\begin{itemize}
\item {Grp. gram.:f.}
\end{itemize}
Tagarelice; palra.
\section{Palratório}
\begin{itemize}
\item {Grp. gram.:m.}
\end{itemize}
O mesmo que \textunderscore parlatório\textunderscore .
\section{Pálrea}
\begin{itemize}
\item {Grp. gram.:f.}
\end{itemize}
O mesmo que \textunderscore palra\textunderscore .
\section{Palrear}
\begin{itemize}
\item {Grp. gram.:v. i.}
\end{itemize}
O mesmo que \textunderscore palrar\textunderscore . Cf. Garrett, \textunderscore Romanceiro\textunderscore , I, 241.
\section{Palreiro}
\begin{itemize}
\item {Grp. gram.:adj.}
\end{itemize}
\begin{itemize}
\item {Proveniência:(De \textunderscore palra\textunderscore )}
\end{itemize}
Que palra; palrador; tagarela; chocalheiro.
\section{Palrelha}
\begin{itemize}
\item {Grp. gram.:adj.}
\end{itemize}
\begin{itemize}
\item {Utilização:Ant.}
\end{itemize}
O mesmo que \textunderscore palreiro\textunderscore .
\section{Palrice}
\begin{itemize}
\item {Grp. gram.:f.}
\end{itemize}
Acto ou effeito de palrar.
\section{Palrónio}
\begin{itemize}
\item {Grp. gram.:m.  e  adj.}
\end{itemize}
O mesmo que \textunderscore palreiro\textunderscore .
\section{Paludamento}
\begin{itemize}
\item {Grp. gram.:m.}
\end{itemize}
\begin{itemize}
\item {Proveniência:(Lat. \textunderscore paludamentum\textunderscore )}
\end{itemize}
Manto branco ou de púrpura, usado pelos generaes romanos e depois pelos Imperadores.
\section{Palude}
\begin{itemize}
\item {Grp. gram.:m.}
\end{itemize}
\begin{itemize}
\item {Utilização:Ant.}
\end{itemize}
\begin{itemize}
\item {Proveniência:(Lat. \textunderscore palus\textunderscore , \textunderscore paludis\textunderscore )}
\end{itemize}
Paúl, lagôa. Cf. B. Pato, \textunderscore Cant. e Sát.\textunderscore , 222.
\section{Paludeína}
\begin{itemize}
\item {Grp. gram.:f.}
\end{itemize}
Muco da paludina, usado em pharmácia.
(Cp. \textunderscore paludina\textunderscore )
\section{Paludial}
\begin{itemize}
\item {Grp. gram.:adj.}
\end{itemize}
\begin{itemize}
\item {Proveniência:(De \textunderscore palude\textunderscore )}
\end{itemize}
Relativo a paúes ou lagôas.
Paludoso; pantanoso. Cf. Castilho, \textunderscore Fastos\textunderscore , I, 410.
\section{Paludícola}
\begin{itemize}
\item {Grp. gram.:f.}
\end{itemize}
\begin{itemize}
\item {Proveniência:(Do lat. \textunderscore palus\textunderscore  + \textunderscore colere\textunderscore )}
\end{itemize}
Gênero de reptis amphíbios, da fam. dos sapos.
\section{Paludina}
\begin{itemize}
\item {Grp. gram.:f.}
\end{itemize}
Mollusco gasterópode de água doce.
\section{Paludismo}
\begin{itemize}
\item {Grp. gram.:m.}
\end{itemize}
O mesmo que \textunderscore impaludismo\textunderscore .
\section{Paludoso}
\begin{itemize}
\item {Grp. gram.:adj.}
\end{itemize}
\begin{itemize}
\item {Proveniência:(Lat. \textunderscore paludosus\textunderscore )}
\end{itemize}
Em que há paúes; pantanoso.
Causado por emanações dos paúes: \textunderscore febres paludosas\textunderscore .
Miasmático.
\section{Palurdice}
\begin{itemize}
\item {Grp. gram.:f.}
\end{itemize}
Qualidade, acto ou dito de palúrdio.
Patetice; qualidade de lorpa.
\section{Palúrdio}
\begin{itemize}
\item {Grp. gram.:m.  e  adj.}
\end{itemize}
\begin{itemize}
\item {Utilização:Chul.}
\end{itemize}
\begin{itemize}
\item {Grp. gram.:M.}
\end{itemize}
\begin{itemize}
\item {Utilização:Gír.}
\end{itemize}
Pacóvio; estúpido; idiota.
Pai.
(Cp. cast. \textunderscore palurdo\textunderscore )
\section{Palustre}
\begin{itemize}
\item {Grp. gram.:adj.}
\end{itemize}
\begin{itemize}
\item {Proveniência:(Lat. \textunderscore palustris\textunderscore )}
\end{itemize}
Relativo a paúes; paludoso.
Que vive em paúes ou lagos.
\section{Paluta}
\begin{itemize}
\item {Grp. gram.:f.}
\end{itemize}
\begin{itemize}
\item {Utilização:Prov.}
\end{itemize}
\begin{itemize}
\item {Utilização:alent.}
\end{itemize}
\begin{itemize}
\item {Proveniência:(Do lat. \textunderscore palus\textunderscore , \textunderscore pali\textunderscore )}
\end{itemize}
Pau curto, com que num jôgo, chamado \textunderscore aro\textunderscore , se toma, se desvia ou se impelle uma bóla.
\section{Pamá}
\begin{itemize}
\item {Grp. gram.:m.}
\end{itemize}
Fruto silvestre do Brasil.
\section{Pamas}
\begin{itemize}
\item {Grp. gram.:m. pl.}
\end{itemize}
Índios bravos do Brasil, dos quaes ainda resta uma aldeia nas margens do Madeira.
\section{Pambo}
\begin{itemize}
\item {Grp. gram.:m.}
\end{itemize}
Espécie de peixe chato do mar das Índias. Cf. Capello e Ivens, II, 148.
\section{Pambotano}
\begin{itemize}
\item {Grp. gram.:m.}
\end{itemize}
Planta leguminosa e medicinal, originária do México, e considerada como um succedâneo da quinina.
\section{Pamonan}
\begin{itemize}
\item {Grp. gram.:m.}
\end{itemize}
\begin{itemize}
\item {Utilização:Bras}
\end{itemize}
Iguaria, feita de farinha de mandioca ou de milho, com feijão, carne ou peixe.
(Do tupi)
\section{Pamonha}
\begin{itemize}
\item {Grp. gram.:f.}
\end{itemize}
\begin{itemize}
\item {Utilização:Bras}
\end{itemize}
\begin{itemize}
\item {Grp. gram.:M.  e  f.}
\end{itemize}
\begin{itemize}
\item {Utilização:Fig.}
\end{itemize}
Espécie de bolo, feito de farinha, com açúcar, leite, etc.
Pessôa estúpida, parva, desajeitada.
\section{Pampa}
\begin{itemize}
\item {Grp. gram.:adj.}
\end{itemize}
\begin{itemize}
\item {Utilização:Bras}
\end{itemize}
\begin{itemize}
\item {Grp. gram.:M.}
\end{itemize}
Diz-se do cavallo, que tem duas côres.
Espécie de gato do Paraguai.
\section{Pampango}
\begin{itemize}
\item {Grp. gram.:m.}
\end{itemize}
Um dos dialectos das Filippinas.
\section{Pâmpano}
\begin{itemize}
\item {Grp. gram.:m.}
\end{itemize}
Pequeno peixe fluvial, o mesmo que \textunderscore pampo\textunderscore ^1.
\section{Pâmpano}
\begin{itemize}
\item {Grp. gram.:m.}
\end{itemize}
\begin{itemize}
\item {Proveniência:(Do lat. \textunderscore pampinus\textunderscore )}
\end{itemize}
Sarmento, ramo tenro de videira; parra.
\section{Pampanoso}
\begin{itemize}
\item {Grp. gram.:adj.}
\end{itemize}
Que tem pâmpanos^2; cheio de pâmpanos^2.
\section{Pampas}
\begin{itemize}
\item {Grp. gram.:m. pl.}
\end{itemize}
Grandes planícies da América meridional, entrecortadas por bosques de palmeiras.
\section{Pampeiro}
\begin{itemize}
\item {Grp. gram.:m.}
\end{itemize}
Vento forte, que sopra do sudoéste ou dos pampas da República Argentina.
\section{Pamphletário}
\begin{itemize}
\item {Grp. gram.:adj.}
\end{itemize}
\begin{itemize}
\item {Grp. gram.:M.}
\end{itemize}
Relativo a pamphleto.
Aquelle que faz pamphletos.
\section{Pamphleteiro}
\begin{itemize}
\item {Grp. gram.:m.  e  adj.}
\end{itemize}
\begin{itemize}
\item {Utilização:Deprec.}
\end{itemize}
O mesmo que \textunderscore pamphletário\textunderscore .
\section{Pamphletista}
\begin{itemize}
\item {Grp. gram.:m.  e  f.}
\end{itemize}
Pessôa, que escreve pamphletos.
\section{Pamphleto}
\begin{itemize}
\item {fónica:flê}
\end{itemize}
\begin{itemize}
\item {Grp. gram.:m.}
\end{itemize}
\begin{itemize}
\item {Proveniência:(Ingl. \textunderscore pamphlet\textunderscore )}
\end{itemize}
Pequeno livro, folheto, especialmente destinado a assumptos políticos, e escrito em estilo violento.
\section{Pampílio}
\begin{itemize}
\item {Grp. gram.:m.}
\end{itemize}
\begin{itemize}
\item {Utilização:Mad}
\end{itemize}
O mesmo que \textunderscore pampilho\textunderscore , planta.
\section{Pampilho}
\begin{itemize}
\item {Grp. gram.:m.}
\end{itemize}
Garrocha, vara comprida, que termina em aguilhão.
Nome de várias plantas, entre as quaes o \textunderscore pampilho aquático\textunderscore ; \textunderscore pampilho espinhoso\textunderscore ; \textunderscore pampilho marítimo\textunderscore , etc.
\section{Pampillo}
\begin{itemize}
\item {Grp. gram.:m.}
\end{itemize}
\begin{itemize}
\item {Utilização:Ant.}
\end{itemize}
O mesmo que \textunderscore pampilho\textunderscore , planta:«\textunderscore e farei calar as rans e florescer os pampillos\textunderscore »G. Vicente, \textunderscore Auto da Lusit.\textunderscore 
\section{Pampíneo}
\begin{itemize}
\item {Grp. gram.:adj.}
\end{itemize}
\begin{itemize}
\item {Proveniência:(Lat. \textunderscore pampineus\textunderscore )}
\end{itemize}
Relativo ao pâmpano; pampanoso.
\section{Pampinoso}
\begin{itemize}
\item {Grp. gram.:adj.}
\end{itemize}
\begin{itemize}
\item {Proveniência:(Lat. \textunderscore pampinosus\textunderscore )}
\end{itemize}
O mesmo que \textunderscore pampanoso\textunderscore .
\section{Pamplonês}
\begin{itemize}
\item {Grp. gram.:m.}
\end{itemize}
Habitante de Pamplona.
\section{Pampo}
\begin{itemize}
\item {Grp. gram.:m.}
\end{itemize}
Nome de duas espécies de peixes escômbridas.
\section{Pampo}
\begin{itemize}
\item {Grp. gram.:m.}
\end{itemize}
\begin{itemize}
\item {Utilização:Prov.}
\end{itemize}
\begin{itemize}
\item {Utilização:trasm.}
\end{itemize}
O mesmo que \textunderscore pâmpano\textunderscore ^1.
\section{Pampolinha}
\begin{itemize}
\item {Grp. gram.:f.}
\end{itemize}
Espécie de jôgo popular, também conhecido por \textunderscore argolinha\textunderscore .
\section{Pamposto}
\begin{itemize}
\item {Grp. gram.:m.}
\end{itemize}
(V.pão-pôsto)
\section{Pan}
\begin{itemize}
\item {Grp. gram.:m.}
\end{itemize}
\begin{itemize}
\item {Proveniência:(Lat. \textunderscore Pan\textunderscore )}
\end{itemize}
Personagem mythológica, adorada pelos pastores e sýmbolo da natureza.
\section{Pan}
\begin{itemize}
\item {Grp. gram.:m.}
\end{itemize}
Outra designação de bétele.
\section{Pan...}
\begin{itemize}
\item {Grp. gram.:pref.}
\end{itemize}
\begin{itemize}
\item {Proveniência:(Do gr. \textunderscore Pan\textunderscore , n. p.)}
\end{itemize}
(designativo de \textunderscore todo\textunderscore  ou \textunderscore tudo\textunderscore )
\section{Panabásio}
\begin{itemize}
\item {Grp. gram.:m.}
\end{itemize}
\begin{itemize}
\item {Utilização:Miner.}
\end{itemize}
\begin{itemize}
\item {Proveniência:(Do gr. \textunderscore pan\textunderscore  + \textunderscore basis\textunderscore )}
\end{itemize}
Cobre cinzento antimonial, antimónio-sulfureto de cobre, prata, ferro e zinco. Cf. R. Galvão, \textunderscore Vocab.\textunderscore 
\section{Panaça}
\begin{itemize}
\item {Grp. gram.:m.}
\end{itemize}
\begin{itemize}
\item {Utilização:Prov.}
\end{itemize}
\begin{itemize}
\item {Utilização:beir.}
\end{itemize}
Marido, que teme a mulher.
Panal de palha. (Colhido no Fundão)
(Cp. \textunderscore panal\textunderscore ^2)
\section{Panacarica}
\begin{itemize}
\item {Grp. gram.:f.}
\end{itemize}
\begin{itemize}
\item {Utilização:Bras. do N}
\end{itemize}
\begin{itemize}
\item {Proveniência:(T. tupi)}
\end{itemize}
Tôldo de palha nas igaratés.
\section{Panacéa}
\begin{itemize}
\item {Grp. gram.:f.}
\end{itemize}
\begin{itemize}
\item {Utilização:Ext.}
\end{itemize}
\begin{itemize}
\item {Utilização:Bras}
\end{itemize}
\begin{itemize}
\item {Utilização:Pop.}
\end{itemize}
\begin{itemize}
\item {Proveniência:(Lat. \textunderscore panacea\textunderscore )}
\end{itemize}
Planta imaginária, a que se attribuía a virtude de curar todas as doenças.
Remédio para todos os males.
Preparado pharmacêutico, que contém certas propriedades geraes.
Qualquer vegetal, especialmente a bôlsa-de-pastor, que se applica a todas as moléstias.
\section{Panaceia}
\begin{itemize}
\item {Grp. gram.:f.}
\end{itemize}
\begin{itemize}
\item {Utilização:Ext.}
\end{itemize}
\begin{itemize}
\item {Utilização:Bras}
\end{itemize}
\begin{itemize}
\item {Utilização:Pop.}
\end{itemize}
\begin{itemize}
\item {Proveniência:(Lat. \textunderscore panacea\textunderscore )}
\end{itemize}
Planta imaginária, a que se attribuía a virtude de curar todas as doenças.
Remédio para todos os males.
Preparado pharmacêutico, que contém certas propriedades geraes.
Qualquer vegetal, especialmente a bôlsa-de-pastor, que se applica a todas as moléstias.
\section{Panaceio}
\begin{itemize}
\item {Grp. gram.:m.}
\end{itemize}
\begin{itemize}
\item {Utilização:Ant.}
\end{itemize}
O mesmo que \textunderscore panaceia\textunderscore .
\section{Panacha}
\begin{itemize}
\item {Grp. gram.:f.}
\end{itemize}
\begin{itemize}
\item {Utilização:T. de Bragança}
\end{itemize}
Prostituta de baixa extracção; rameira reles.
\section{Panacu}
\begin{itemize}
\item {Grp. gram.:m.}
\end{itemize}
Espécie de cesto do Brasil.
\section{Panacum}
\begin{itemize}
\item {Grp. gram.:m.}
\end{itemize}
O mesmo que \textunderscore panacu\textunderscore .
\section{Panada}
\begin{itemize}
\item {Grp. gram.:f.}
\end{itemize}
\begin{itemize}
\item {Utilização:Açor}
\end{itemize}
\begin{itemize}
\item {Utilização:Prov.}
\end{itemize}
\begin{itemize}
\item {Utilização:minh.}
\end{itemize}
Pano, que serve de envoltório ou que fórma troixa.
Paveia.
\section{Panadeira}
\begin{itemize}
\item {Grp. gram.:f.}
\end{itemize}
\begin{itemize}
\item {Utilização:Ant.}
\end{itemize}
O mesmo que \textunderscore pàdeira\textunderscore . Cf. G. Vicente, II, 161.
(Cast. \textunderscore panadera\textunderscore )
\section{Panadilha}
\begin{itemize}
\item {Grp. gram.:f.}
\end{itemize}
\begin{itemize}
\item {Utilização:Prov.}
\end{itemize}
\begin{itemize}
\item {Utilização:minh.}
\end{itemize}
Pão com recheio, cozido no forno.
(Cp. \textunderscore empanadilha\textunderscore )
\section{Panado}
\begin{itemize}
\item {Grp. gram.:adj.}
\end{itemize}
\begin{itemize}
\item {Proveniência:(Do lat. hyp. \textunderscore panatus\textunderscore )}
\end{itemize}
Que tem pão ralado; coberto de pão ralado.
Que se coou através de pão ralado: \textunderscore água panada\textunderscore .
\section{Panadura}
\begin{itemize}
\item {Grp. gram.:f.}
\end{itemize}
Eixo da moenda de açúcar.
\section{Panal}
\begin{itemize}
\item {Grp. gram.:m.}
\end{itemize}
Pau que, collocado sob a quilha de um barco, o arroja para terra.
\section{Panal}
\begin{itemize}
\item {Grp. gram.:m.}
\end{itemize}
\begin{itemize}
\item {Utilização:Prov.}
\end{itemize}
\begin{itemize}
\item {Utilização:trasm.}
\end{itemize}
\begin{itemize}
\item {Utilização:T. de Buarcos}
\end{itemize}
\begin{itemize}
\item {Utilização:Fig.}
\end{itemize}
\begin{itemize}
\item {Proveniência:(De \textunderscore pano\textunderscore )}
\end{itemize}
Pano, em que se estende ou envolve alguma coisa.
\textunderscore Panal de palha\textunderscore , pano cheio de palha.
Vela de moínho. Cf. \textunderscore Caveira\textunderscore , 321.
Pequeno tapume de tábuas, que resguarda, a alguma distância, a mó do moínho de cereaes, para impedir que a farinha se espalhe por longe e se misture com ciscos.
O mesmo que \textunderscore cueiro\textunderscore .
Cada um dos rolos de madeira, sôbre os quaes os barcos chatos se dirigem da praia para o mar.
Pacóvio, parvo.
\section{Panamá}
\begin{itemize}
\item {Grp. gram.:m.}
\end{itemize}
\begin{itemize}
\item {Utilização:Fig.}
\end{itemize}
\begin{itemize}
\item {Proveniência:(De \textunderscore Panamá\textunderscore , n. p. Cp. \textunderscore panamista\textunderscore )}
\end{itemize}
Chapéu americano, fabricado com tiras das fôlhas da palmeira chamada bóbonax.
Erva tinctória do Alto Amazonas.
Administração ruinosa do uma Companhia mercantil ou industrial, cujos administradores procuram locuptar-se á custa dos accionistas.
\section{Panamista}
\begin{itemize}
\item {Grp. gram.:m.}
\end{itemize}
\begin{itemize}
\item {Utilização:Neol.}
\end{itemize}
\begin{itemize}
\item {Utilização:Fig.}
\end{itemize}
Indivíduo do Panamá.
Indivíduo, que se envolveu nas questões financeira da Companhia que tratava da abertura do isthmo de Panamá.
Homem, envolvido em syndicatos e outras negociações de carácter obscuro ou suspeito.
\section{Panão}
\begin{itemize}
\item {Grp. gram.:m.}
\end{itemize}
\begin{itemize}
\item {Utilização:Prov.}
\end{itemize}
\begin{itemize}
\item {Utilização:alent.}
\end{itemize}
\begin{itemize}
\item {Proveniência:(De \textunderscore pano\textunderscore )}
\end{itemize}
Pano grande, que se colloca debaixo das oliveiras, para caír sôbre êlle a azeitona que se vareja.
\section{Pana-panazi}
\begin{itemize}
\item {Grp. gram.:m.}
\end{itemize}
Planta clusiácea do Brasil.
\section{Panar}
\begin{itemize}
\item {Grp. gram.:v. t.}
\end{itemize}
\begin{itemize}
\item {Proveniência:(Do lat. \textunderscore panis\textunderscore )}
\end{itemize}
Deitar pão torrado em (água), coando-o depois, para uso de doentes.
Cobrir com pão raiado: \textunderscore costelletas panadas\textunderscore .
Relativo a pão:«\textunderscore ...fermentação panar.\textunderscore »Ferreira Lapa.
\section{Panarei}
\begin{itemize}
\item {Grp. gram.:adj.}
\end{itemize}
\begin{itemize}
\item {Utilização:Des.}
\end{itemize}
\begin{itemize}
\item {Proveniência:(De \textunderscore pano\textunderscore  + \textunderscore rei\textunderscore ?)}
\end{itemize}
Dizia-se de uma espécie de calhamaço ou de pano grosso de linho.--Não sei se é t. port., mas vejo-o no \textunderscore Diccion.\textunderscore  de Dom. Vieira, vb. \textunderscore calhamaço\textunderscore .
\section{Panaria}
\begin{itemize}
\item {Grp. gram.:f.}
\end{itemize}
\begin{itemize}
\item {Utilização:Ant.}
\end{itemize}
\begin{itemize}
\item {Proveniência:(Do lat. \textunderscore panis\textunderscore )}
\end{itemize}
Tulha, celleiro.
\section{Panarício}
\begin{itemize}
\item {Grp. gram.:m.}
\end{itemize}
\begin{itemize}
\item {Utilização:Med.}
\end{itemize}
\begin{itemize}
\item {Proveniência:(Lat. \textunderscore panaricium\textunderscore )}
\end{itemize}
Tumor, na extremidade dos dedos ou na raíz das unhas.
\textunderscore Panarício analgésico\textunderscore , doença de Morvan, fórma attenuada da lepra.
\section{Panariz}
\begin{itemize}
\item {Grp. gram.:m.}
\end{itemize}
\begin{itemize}
\item {Utilização:Prov.}
\end{itemize}
O mesmo que \textunderscore panarício\textunderscore .
\section{Panascal}
\begin{itemize}
\item {Grp. gram.:m.}
\end{itemize}
\begin{itemize}
\item {Proveniência:(Do b. lat. \textunderscore pannascalis\textunderscore )}
\end{itemize}
O mesmo que \textunderscore panasqueira\textunderscore .
\section{Panasco}
\begin{itemize}
\item {Grp. gram.:m.}
\end{itemize}
\begin{itemize}
\item {Utilização:Prov.}
\end{itemize}
\begin{itemize}
\item {Utilização:minh.}
\end{itemize}
\begin{itemize}
\item {Utilização:T. de Viana}
\end{itemize}
Erva umbellífera, que serve para pastos.
Terreno, geralmente alagado e em que cresce erva.
Indivíduo boçal, lorpa.
(Cp. \textunderscore panascal\textunderscore )
\section{Panásio}
\begin{itemize}
\item {Grp. gram.:m.}
\end{itemize}
\begin{itemize}
\item {Utilização:Pop.}
\end{itemize}
\begin{itemize}
\item {Utilização:Bras}
\end{itemize}
Pontapé.
Bofetada.
Pranchada, pancada dada de prancha com a espada.
\section{Panaso}
\begin{itemize}
\item {Grp. gram.:m.}
\end{itemize}
\begin{itemize}
\item {Utilização:Pop.}
\end{itemize}
O mesmo que \textunderscore panásio\textunderscore . Cf. Arn Gama, \textunderscore Últ. Dona\textunderscore , 17.
\section{Panasqueira}
\begin{itemize}
\item {Grp. gram.:f.}
\end{itemize}
\begin{itemize}
\item {Utilização:Pop.}
\end{itemize}
\begin{itemize}
\item {Proveniência:(De \textunderscore panasco\textunderscore )}
\end{itemize}
Terreno, onde cresce panasco.
Terra sertaneja, povoação pouco importante.
\section{Panasqueiro}
\begin{itemize}
\item {Grp. gram.:m.}
\end{itemize}
\begin{itemize}
\item {Grp. gram.:M.  e  adj.}
\end{itemize}
\begin{itemize}
\item {Utilização:Pop.}
\end{itemize}
\begin{itemize}
\item {Proveniência:(De \textunderscore panasco\textunderscore )}
\end{itemize}
O mesmo que \textunderscore panasqueira\textunderscore .
Panasco.
Indivíduo de modos ou vestuário grosseiros.
\section{Panateias}
\begin{itemize}
\item {Grp. gram.:f. pl.}
\end{itemize}
Antigas festas gregas, que recordavam a aliança dos povos da Attica.
Talvez o mesmo que \textunderscore panateneias\textunderscore .
\section{Panateneias}
\begin{itemize}
\item {Grp. gram.:f. pl.}
\end{itemize}
\begin{itemize}
\item {Proveniência:(Do gr. \textunderscore pan\textunderscore  + \textunderscore Athene\textunderscore )}
\end{itemize}
Grande festa dos Atenienses, em honra de Minerva. Cf. Filinto, XV, 269.
\section{Panateneio}
\begin{itemize}
\item {Grp. gram.:adj}
\end{itemize}
Relativo ás panateneias.
\section{Panatheias}
\begin{itemize}
\item {Grp. gram.:f. pl.}
\end{itemize}
Antigas festas gregas, que recordavam a alliança dos povos da Attica.
Talvez o mesmo que \textunderscore panatheneias\textunderscore .
\section{Panatheneias}
\begin{itemize}
\item {Grp. gram.:f. pl.}
\end{itemize}
\begin{itemize}
\item {Proveniência:(Do gr. \textunderscore pan\textunderscore  + \textunderscore Athene\textunderscore )}
\end{itemize}
Grande festa dos Athenienses, em honra de Minerva. Cf. Filinto, XV, 269.
\section{Panatheneio}
\begin{itemize}
\item {Grp. gram.:adj}
\end{itemize}
Relativo ás panatheneias.
\section{Panatis}
\begin{itemize}
\item {Grp. gram.:m. pl.}
\end{itemize}
Tríbo de Índios do Brasil, que habitaram na serra de Panati, donde tiraram o nome.
\section{Pança}
\begin{itemize}
\item {Grp. gram.:f.}
\end{itemize}
\begin{itemize}
\item {Utilização:Prov.}
\end{itemize}
\begin{itemize}
\item {Utilização:minh.}
\end{itemize}
\begin{itemize}
\item {Grp. gram.:Pl.}
\end{itemize}
\begin{itemize}
\item {Utilização:Bras}
\end{itemize}
Alavanca de madeira.
O mesmo que \textunderscore calcadeira\textunderscore .
\textunderscore Andar en panca\textunderscore , andar muito atarefado, sem saber bem o que deve fazer.
\textunderscore Dar pancas\textunderscore , sobresair em qualquer acto ou coisa.
(Contr. de \textunderscore palanca\textunderscore ^1)
\section{Pancá}
\begin{itemize}
\item {Grp. gram.:m.}
\end{itemize}
\begin{itemize}
\item {Proveniência:(Do conc. \textunderscore pankha\textunderscore )}
\end{itemize}
Grande ventarola, suspensa do tecto, e agitada por servos, para arejar e refrescar a casa, entre os Índios.
\section{Pança}
\begin{itemize}
\item {Grp. gram.:f.}
\end{itemize}
\begin{itemize}
\item {Utilização:Chul.}
\end{itemize}
\begin{itemize}
\item {Proveniência:(Do lat. \textunderscore pantex\textunderscore )}
\end{itemize}
O maior estômago dos ruminantes.
Grande barriga.
\section{Pancada}
\begin{itemize}
\item {Grp. gram.:f.}
\end{itemize}
\begin{itemize}
\item {Utilização:Fig.}
\end{itemize}
\begin{itemize}
\item {Utilização:Pop.}
\end{itemize}
\begin{itemize}
\item {Utilização:Pop.}
\end{itemize}
\begin{itemize}
\item {Grp. gram.:Adj.}
\end{itemize}
\begin{itemize}
\item {Utilização:Bras. do N}
\end{itemize}
\begin{itemize}
\item {Proveniência:(De \textunderscore panca\textunderscore )}
\end{itemize}
Choque de um corpo contra outro.
Baque.
Acto de bater.
Som, produzido pelo pêndulo do relógio.
Bordoada.
Pulsação.
Presentimento.
Mania, telha.
Tendência, vocação, sestro. Cf. Garrett, \textunderscore Filippa\textunderscore , 216.
\textunderscore Instrumento de pancada\textunderscore , instrumento de percussão, como o bombo, os pratos, etc.
Brutal; grosseiro.
\section{Pançada}
\begin{itemize}
\item {Grp. gram.:f.}
\end{itemize}
Pancada na pança.
Enchimento do estômago; fartote.
\section{Pancadão}
\begin{itemize}
\item {Grp. gram.:m.}
\end{itemize}
\begin{itemize}
\item {Utilização:Bras. do N}
\end{itemize}
O mesmo que \textunderscore peixão\textunderscore .
\section{Pancadaria}
\begin{itemize}
\item {Grp. gram.:f.}
\end{itemize}
Muitas pancadas; bordoada; sova.
Desordem, em que se dão muitas pancadas.
Conjunto de instrumentos musicaes de pancada, numa orchesta, banda ou philarmónica.
\section{Pancadista}
\begin{itemize}
\item {Grp. gram.:m.  e  adj.}
\end{itemize}
\begin{itemize}
\item {Utilização:T. da Bairrada}
\end{itemize}
\begin{itemize}
\item {Proveniência:(De \textunderscore pancada\textunderscore )}
\end{itemize}
O que tem telha ou mania.
\section{Pancadola}
\begin{itemize}
\item {Grp. gram.:m.  e  adj.}
\end{itemize}
\begin{itemize}
\item {Utilização:T. de Turquel}
\end{itemize}
O mesmo que \textunderscore pancadista\textunderscore .
\section{Pancaio}
\begin{itemize}
\item {Grp. gram.:adj.}
\end{itemize}
\begin{itemize}
\item {Proveniência:(Lat. \textunderscore panchaius\textunderscore )}
\end{itemize}
Relativo á Panchaia, região da Arábia Feliz, onde abundava o incenso ou o bálsamo.
Diz-se do perfume do incenso ou do bálsamo:«\textunderscore ...aromas pancaios...\textunderscore »\textunderscore Anat. Joc.\textunderscore , p. 113.
\section{Pancão}
\begin{itemize}
\item {Grp. gram.:m.}
\end{itemize}
\begin{itemize}
\item {Utilização:Prov.}
\end{itemize}
\begin{itemize}
\item {Utilização:trasm.}
\end{itemize}
\begin{itemize}
\item {Utilização:pop.}
\end{itemize}
Homem maníaco, telhudo.
(Cp. \textunderscore pancada\textunderscore )
\section{Pancárpia}
\begin{itemize}
\item {Grp. gram.:f.}
\end{itemize}
\begin{itemize}
\item {Utilização:Ant.}
\end{itemize}
O mesmo que \textunderscore pancarta\textunderscore ?«\textunderscore ...todas as minhas pancarpias, que em papel ou encadernadas se acharem, quero que...\textunderscore »(De um testamento do séc. XVII)
\section{Pancarta}
\begin{itemize}
\item {Grp. gram.:f.}
\end{itemize}
Antigo diploma, com que os reis confirmavam a acquisição de bens ou direitos, feita por uma igreja ou convento. Cf. Ducange; cf. Herculano, \textunderscore Hist. de Port.\textunderscore , III, 408.
(B. lat. \textunderscore pancharta\textunderscore )
\section{Pancas}
\begin{itemize}
\item {Grp. gram.:f. pl.}
\end{itemize}
\begin{itemize}
\item {Utilização:Fig.}
\end{itemize}
\begin{itemize}
\item {Utilização:Bras}
\end{itemize}
Apêrto, difficuldade.
\textunderscore Dar pancas\textunderscore , ostentar importância, brilhar.
(Pl. de \textunderscore panca\textunderscore )
\section{Panceira}
\begin{itemize}
\item {Grp. gram.:f.}
\end{itemize}
\begin{itemize}
\item {Utilização:Ant.}
\end{itemize}
\begin{itemize}
\item {Proveniência:(De \textunderscore pança\textunderscore )}
\end{itemize}
Parte da armadura, com que os guerreiros resguardavam o ventre. Cf. Fern. Lopes. \textunderscore D. João I\textunderscore , p. II, c. 38.
\section{Panchaio}
\begin{itemize}
\item {fónica:cai}
\end{itemize}
\begin{itemize}
\item {Grp. gram.:adj.}
\end{itemize}
\begin{itemize}
\item {Proveniência:(Lat. \textunderscore panchaius\textunderscore )}
\end{itemize}
Relativo á Panchaia, região da Arábia Feliz, onde abundava o incenso ou o bálsamo.
Diz-se do perfume do incenso ou do bálsamo:«\textunderscore ...aromas panchaios...\textunderscore »\textunderscore Anat. Joc.\textunderscore , p. 113.
\section{Pancham}
\begin{itemize}
\item {Grp. gram.:m.}
\end{itemize}
Granito pardo da Índia.
\section{Panchão}
\begin{itemize}
\item {Grp. gram.:m.}
\end{itemize}
Espécie de foguete que se fabríca em Macau e que dalli se exporta em grande quantidade.
\section{Panchresto}
\begin{itemize}
\item {fónica:crés}
\end{itemize}
\begin{itemize}
\item {Grp. gram.:m.}
\end{itemize}
\begin{itemize}
\item {Proveniência:(Gr. \textunderscore pankhrestos\textunderscore )}
\end{itemize}
O mesmo que \textunderscore panaceia\textunderscore .
\section{Panchymagogo}
\begin{itemize}
\item {fónica:qui}
\end{itemize}
\begin{itemize}
\item {Grp. gram.:m.  e  adj.}
\end{itemize}
\begin{itemize}
\item {Proveniência:(Gr. \textunderscore pankumagogos\textunderscore )}
\end{itemize}
Diz-se das substâncias purgativas, a que se attribuía a faculdade de expulsar todos os humores.
\section{Panclastita}
\begin{itemize}
\item {Grp. gram.:f.}
\end{itemize}
\begin{itemize}
\item {Utilização:Chím.}
\end{itemize}
\begin{itemize}
\item {Proveniência:(Do gr. \textunderscore pan\textunderscore  + \textunderscore klao\textunderscore )}
\end{itemize}
Composto explosivo, resultante da acção do peróxydo de azoto sôbre vários corpos carbonetados.
\section{Panco}
\begin{itemize}
\item {Grp. gram.:m.}
\end{itemize}
\begin{itemize}
\item {Utilização:Prov.}
\end{itemize}
Panca, alavanca tôsca de pau.
\section{Pancrácio}
\begin{itemize}
\item {Grp. gram.:m.}
\end{itemize}
\begin{itemize}
\item {Utilização:Pop.}
\end{itemize}
Pateta; idiota; simplório; pascácio. Cf. B. Pato, \textunderscore Cant. e Sát.\textunderscore , 206.
\section{Pancrácio}
\begin{itemize}
\item {Grp. gram.:m.}
\end{itemize}
\begin{itemize}
\item {Utilização:Bras}
\end{itemize}
Planta de jardins, (\textunderscore lilium maritimum\textunderscore ).
\section{Pancreadene}
\begin{itemize}
\item {Grp. gram.:m.}
\end{itemize}
Substância pharmaceutica, preparada com pâncreas fresco.
\section{Pâncreas}
\begin{itemize}
\item {Grp. gram.:m.}
\end{itemize}
\begin{itemize}
\item {Proveniência:(Gr. \textunderscore pankreas\textunderscore )}
\end{itemize}
Glândula, situada no abdome, e cuja funcção é realizar, por meio do líquido que segrega, a digestão das substâncias gordas.
Líquido segregado por essa glândula.
\section{Pancreatalgia}
\begin{itemize}
\item {Grp. gram.:f.}
\end{itemize}
\begin{itemize}
\item {Proveniência:(Do gr. \textunderscore pankreas\textunderscore  + \textunderscore algos\textunderscore )}
\end{itemize}
Dôr no pâncreas.
\section{Pancreático}
\begin{itemize}
\item {Grp. gram.:adj.}
\end{itemize}
Relativo ao pâncreas, ou que é produzido por elle.
\section{Pancreatina}
\begin{itemize}
\item {Grp. gram.:f.}
\end{itemize}
\begin{itemize}
\item {Proveniência:(De \textunderscore pancreático\textunderscore )}
\end{itemize}
Substância, que se encontra no suco pancreático, e é empregada no tratamento de certas dyspepsias.
\section{Pancreatite}
\begin{itemize}
\item {Grp. gram.:f.}
\end{itemize}
Inflammação do pâncreas.
\section{Pancresto}
\begin{itemize}
\item {Grp. gram.:m.}
\end{itemize}
\begin{itemize}
\item {Proveniência:(Gr. \textunderscore pankhrestos\textunderscore )}
\end{itemize}
O mesmo que \textunderscore panaceia\textunderscore .
\section{Pançudo}
\begin{itemize}
\item {Grp. gram.:adj.}
\end{itemize}
Que tem pança volumosa; barrigudo.
\section{Panda}
\begin{itemize}
\item {Grp. gram.:f.}
\end{itemize}
Árvore africana, da fam. das leguminosas, (Welw, \textunderscore berlinia angolensis\textunderscore ?).
\section{Panda}
\begin{itemize}
\item {Grp. gram.:f.}
\end{itemize}
\begin{itemize}
\item {Utilização:Pesc.}
\end{itemize}
Bóia de cortiça, na tralha superior dos apparelhos de arrastar.
(Cp. \textunderscore pando\textunderscore ^2)
\section{Panda}
\begin{itemize}
\item {Grp. gram.:f.}
\end{itemize}
Ave pernalta, asiática.
\section{Pandâneas}
\begin{itemize}
\item {Grp. gram.:f. pl.}
\end{itemize}
\begin{itemize}
\item {Proveniência:(De \textunderscore pandano\textunderscore )}
\end{itemize}
Ordem de plantas perennes, intertropicaes, alimentares.
\section{Pandanga}
\begin{itemize}
\item {Grp. gram.:f.}
\end{itemize}
\begin{itemize}
\item {Utilização:Prov.}
\end{itemize}
\begin{itemize}
\item {Utilização:minh.}
\end{itemize}
Episódio engraçado; coisa cómica. Cf. Camillo, \textunderscore Volcões\textunderscore , 194, (ed. 1886).
(Alter. burlesca de \textunderscore pândega\textunderscore ?)
\section{Pandano}
\begin{itemize}
\item {Grp. gram.:m.}
\end{itemize}
\begin{itemize}
\item {Proveniência:(Do mal. \textunderscore pandang\textunderscore )}
\end{itemize}
Gênero de plantas, que são o typo das pandâneas.
\section{Pandarana}
\begin{itemize}
\item {Grp. gram.:m.}
\end{itemize}
\begin{itemize}
\item {Utilização:Des.}
\end{itemize}
\begin{itemize}
\item {Proveniência:(De \textunderscore Pandarane\textunderscore , n. p.)}
\end{itemize}
O mesmo que \textunderscore pantana\textunderscore .
\section{Pândaro}
\begin{itemize}
\item {Grp. gram.:m.}
\end{itemize}
Gênero de crustáceos.
\section{Pandaxocoxôco}
\begin{itemize}
\item {Grp. gram.:m.}
\end{itemize}
Nome de dois pássaros da África occidental.
\section{Pandear}
\begin{itemize}
\item {Grp. gram.:v. t.}
\end{itemize}
Tornar pando.
Enconcar.
\section{Pandecta}
\begin{itemize}
\item {Grp. gram.:f.}
\end{itemize}
\begin{itemize}
\item {Grp. gram.:Pl.}
\end{itemize}
\begin{itemize}
\item {Proveniência:(Lat. \textunderscore pandectae\textunderscore )}
\end{itemize}
Espécie de caractéres typográphicos.
Compilação das decisões dos antigos jurisconsultos, convertidas em lei por Justiniano.
\section{Pândega}
\begin{itemize}
\item {Grp. gram.:f.}
\end{itemize}
\begin{itemize}
\item {Utilização:Pop.}
\end{itemize}
Comezaina; patuscada; extravagância, estroinice.
Folguedo ruidoso.
Vadiagem alegre.
\section{Pandegar}
\begin{itemize}
\item {Grp. gram.:v. i.}
\end{itemize}
\begin{itemize}
\item {Utilização:Pop.}
\end{itemize}
Andar em pândegas; estroinar.
\section{Pândego}
\begin{itemize}
\item {Grp. gram.:m.  e  adj.}
\end{itemize}
\begin{itemize}
\item {Utilização:Pop.}
\end{itemize}
Aquelle que é amigo de pândegas, de vida airada; estroina.
(Cp. \textunderscore pândega\textunderscore )
\section{Pandeireiro}
\begin{itemize}
\item {Grp. gram.:m.}
\end{itemize}
Fabricante de pandeiros; tocador de pandeiro.
\section{Pandeireta}
\begin{itemize}
\item {fónica:deirê}
\end{itemize}
\begin{itemize}
\item {Grp. gram.:f.}
\end{itemize}
Pequeno pandeiro.
\section{Pandeirinha}
\begin{itemize}
\item {Grp. gram.:f.}
\end{itemize}
Planta gramínea, (\textunderscore briza minor\textunderscore , Lin.).
\section{Pandeiro}
\begin{itemize}
\item {Grp. gram.:m.}
\end{itemize}
\begin{itemize}
\item {Utilização:Náut.}
\end{itemize}
\begin{itemize}
\item {Proveniência:(Do lat. \textunderscore pandurium\textunderscore )}
\end{itemize}
Instrumento músico de percussão, formado de um arco com guisos ou lâminas metállicas que se chocam, e sôbre o qual se estica uma pelle, que se tange, batendo-a com a mão, com os cotovelos, etc.
\textunderscore Pandeiro de cabos\textunderscore , cordas enroladas em voltas circulares.
\section{Pandemia}
\begin{itemize}
\item {Grp. gram.:f.}
\end{itemize}
\begin{itemize}
\item {Proveniência:(Gr. \textunderscore pandemia\textunderscore )}
\end{itemize}
Doença que, ao mesmo tempo, ataca muitos indivíduos na mesma localidade.
\section{Pandêmico}
\begin{itemize}
\item {Grp. gram.:adj.}
\end{itemize}
\begin{itemize}
\item {Proveniência:(Gr. \textunderscore pandemikos\textunderscore )}
\end{itemize}
Que tem o carácter da pandemia.
\section{Pandemónico}
\begin{itemize}
\item {Grp. gram.:adj.}
\end{itemize}
Relativo a pandemónio:«\textunderscore distúrbio pandemónico\textunderscore ». R. Jorge, \textunderscore Epid. de Lisbôa\textunderscore , 2.
\section{Pandemónio}
\begin{itemize}
\item {Grp. gram.:m.}
\end{itemize}
\begin{itemize}
\item {Proveniência:(Do gr. \textunderscore pan\textunderscore  + \textunderscore daimon\textunderscore )}
\end{itemize}
Conluio de indivíduos, para fazer mal ou armar desordens.
Tumulto; ajuntamento tumultuoso.
Grande confusão, babel.
\section{Pandereta}
\begin{itemize}
\item {fónica:derê}
\end{itemize}
\begin{itemize}
\item {Grp. gram.:f.}
\end{itemize}
\begin{itemize}
\item {Utilização:Ant.}
\end{itemize}
O mesmo que \textunderscore pandeireta\textunderscore . Cf. Júl. Castilho, \textunderscore Manuelinas\textunderscore , 132.
(Cast. \textunderscore pandereta\textunderscore )
\section{Pandiculação}
\begin{itemize}
\item {Grp. gram.:f.}
\end{itemize}
\begin{itemize}
\item {Proveniência:(Lat. \textunderscore pandiculatio\textunderscore )}
\end{itemize}
Acto de se espreguiçar alguém.
\section{Pandilha}
\begin{itemize}
\item {Grp. gram.:m.}
\end{itemize}
\begin{itemize}
\item {Grp. gram.:F.}
\end{itemize}
\begin{itemize}
\item {Utilização:Ant.}
\end{itemize}
Indivíduo, que entra em conluios, para enganar outrem.
Biltre; pulha; farroupilha; bigorrilha.
Conluio entre vários indivíduos, para enganar alguém.
(Cast. \textunderscore pandilla\textunderscore )
\section{Pandilha}
\begin{itemize}
\item {Grp. gram.:adj. m.}
\end{itemize}
\begin{itemize}
\item {Utilização:Prov.}
\end{itemize}
\begin{itemize}
\item {Utilização:alg.}
\end{itemize}
Diz-se de uma espécie de milho menos graúdo e de cana mais baixa que a do milhão.
\section{Pandilhar}
\begin{itemize}
\item {Grp. gram.:v. t.}
\end{itemize}
Ter vida de pandilha.
Vadiar.
\section{Pandilheiro}
\begin{itemize}
\item {Grp. gram.:m.}
\end{itemize}
O mesmo que \textunderscore pandilha\textunderscore ^1.
\section{Panarmónico}
\begin{itemize}
\item {Grp. gram.:m.}
\end{itemize}
\begin{itemize}
\item {Proveniência:(De \textunderscore pan...\textunderscore  + \textunderscore harmonia\textunderscore )}
\end{itemize}
Espécie de órgão, que toca árias, imitando diferentes instrumentos de sopro.
\section{Panarmónio}
\begin{itemize}
\item {Grp. gram.:m.}
\end{itemize}
O mesmo ou melhor que \textunderscore panarmónico\textunderscore .
\section{Pandinamismo}
\begin{itemize}
\item {Grp. gram.:m.}
\end{itemize}
\begin{itemize}
\item {Proveniência:(Do gr. \textunderscore pan\textunderscore  + \textunderscore dunamis\textunderscore )}
\end{itemize}
Sistema filosófico dos que defendem a actividade essencial de tudo, e portanto a modificação constante do universo por simples impulso de fôrças.
\section{Pandinamista}
\begin{itemize}
\item {Grp. gram.:m.}
\end{itemize}
Sectário ou defensor do pandinamismo.
\section{Pandinamómetro}
\begin{itemize}
\item {Grp. gram.:m.}
\end{itemize}
\begin{itemize}
\item {Proveniência:(De \textunderscore pan\textunderscore  gr. + \textunderscore dinamómetro\textunderscore )}
\end{itemize}
Aparelho, para medir ou determinar o trabalho mecânico, produzido por um motor ou despendido por uma máquina.
\section{Pândita}
\begin{itemize}
\item {Grp. gram.:m.}
\end{itemize}
Doutor ou sabedor da côrte dos antigos reis indios. Cf. Vasc. Abreu, \textunderscore Contos da Índia\textunderscore .
\section{Pando}
\begin{itemize}
\item {Grp. gram.:m}
\end{itemize}
O mesmo que \textunderscore pampo\textunderscore ^1.
\section{Pando}
\begin{itemize}
\item {Grp. gram.:adj.}
\end{itemize}
\begin{itemize}
\item {Proveniência:(Lat. \textunderscore pandus\textunderscore )}
\end{itemize}
Cheio; inflado.
Enfunado; inchado.
Largo.
Aberto e encurvado:«\textunderscore Co'os pandos braços Huol accorre...\textunderscore »Filinto, VII, 95.
\section{Pandora}
\begin{itemize}
\item {Grp. gram.:f.}
\end{itemize}
\begin{itemize}
\item {Proveniência:(De \textunderscore Pandora\textunderscore , n. p.)}
\end{itemize}
Gênero de molluscos acéphalos.
\section{Pandora}
\begin{itemize}
\item {Grp. gram.:f.}
\end{itemize}
\begin{itemize}
\item {Proveniência:(Do lat. \textunderscore pandura\textunderscore )}
\end{itemize}
Instrumento que é o baixo da mandolina, com dezanove cordas metállicas.
\section{Pandorca}
\begin{itemize}
\item {Grp. gram.:f.}
\end{itemize}
O mesmo que \textunderscore pandorga\textunderscore .
\section{Pandorga}
\begin{itemize}
\item {Grp. gram.:f.}
\end{itemize}
\begin{itemize}
\item {Utilização:Pop.}
\end{itemize}
\begin{itemize}
\item {Utilização:Pleb.}
\end{itemize}
\begin{itemize}
\item {Utilização:Bras. do S}
\end{itemize}
\begin{itemize}
\item {Grp. gram.:M.}
\end{itemize}
\begin{itemize}
\item {Utilização:Pop.}
\end{itemize}
\begin{itemize}
\item {Utilização:Ant.}
\end{itemize}
\begin{itemize}
\item {Utilização:Pop.}
\end{itemize}
Música, desafinada ou sem compasso.
Mulher obesa.
Papagaio de papel, com que se divertem as crianças.
Homem obeso, desajeitado.
Homem, que gosta da ociosidade; pândego; valdevinos.
\section{Pandorina}
\begin{itemize}
\item {Grp. gram.:f.}
\end{itemize}
\begin{itemize}
\item {Proveniência:(De \textunderscore pandora\textunderscore )}
\end{itemize}
Gênero de infusórios.
\section{Pandoro}
\begin{itemize}
\item {fónica:dô}
\end{itemize}
\begin{itemize}
\item {Grp. gram.:m.}
\end{itemize}
\begin{itemize}
\item {Utilização:T. da Áfr. Or. Port}
\end{itemize}
O mesmo que \textunderscore feiticeiro\textunderscore .
\section{Pandulhar}
\begin{itemize}
\item {Grp. gram.:v. i.}
\end{itemize}
\begin{itemize}
\item {Utilização:Pesc.}
\end{itemize}
\begin{itemize}
\item {Proveniência:(De \textunderscore pandulho\textunderscore )}
\end{itemize}
Levantar a tralha dos pandulhos, para tirar o peixe emmalhado.
\section{Pandulho}
\begin{itemize}
\item {Grp. gram.:m.}
\end{itemize}
\begin{itemize}
\item {Utilização:Pesc.}
\end{itemize}
\begin{itemize}
\item {Proveniência:(De \textunderscore pando\textunderscore ? Por \textunderscore bandulho\textunderscore ?)}
\end{itemize}
Lastro da tralha inferior das rêdes.
Pedra grande, presa a uma corda, e que serve de âncora a barcos pequenos.
\section{Panduriforme}
\begin{itemize}
\item {Grp. gram.:adj.}
\end{itemize}
\begin{itemize}
\item {Utilização:Bot.}
\end{itemize}
\begin{itemize}
\item {Proveniência:(Do lat. \textunderscore pandura\textunderscore  + \textunderscore forma\textunderscore )}
\end{itemize}
Diz-se das fôlhas, quando oblongas, arredondadas nas extremidades e apertadas no meio, de ambos os lados, dando o aspecto da viola ou bandurra.
\section{Pandynamismo}
\begin{itemize}
\item {Grp. gram.:m.}
\end{itemize}
\begin{itemize}
\item {Proveniência:(Do gr. \textunderscore pan\textunderscore  + \textunderscore dunamis\textunderscore )}
\end{itemize}
Systema philosóphico dos que defendem a actividade essencial de tudo, e portanto a modificação constante do universo por simples impulso de fôrças.
\section{Pandynamista}
\begin{itemize}
\item {Grp. gram.:m.}
\end{itemize}
Sectário ou defensor do pandynamismo.
\section{Pandynamómetro}
\begin{itemize}
\item {Grp. gram.:m.}
\end{itemize}
\begin{itemize}
\item {Proveniência:(De \textunderscore pan\textunderscore  gr. + \textunderscore dynamómetro\textunderscore )}
\end{itemize}
Apparelho, para medir ou determinar o trabalho mecânico, produzido por um motor ou despendido por uma máquina.
\section{Panegiricado}
\begin{itemize}
\item {Grp. gram.:adj.}
\end{itemize}
Que contém panegírico:«\textunderscore ...versos panegiricados.\textunderscore »Filinto, VIII, 243.
\section{Panegirical}
\begin{itemize}
\item {Grp. gram.:adj.}
\end{itemize}
\begin{itemize}
\item {Proveniência:(De \textunderscore panegírico\textunderscore )}
\end{itemize}
Relativo a panegírico; que envolve louvor ou panegírico. Cf. Filinto, II, 203.
\section{Panegiricar}
\begin{itemize}
\item {Grp. gram.:v. t.}
\end{itemize}
Fazer o panegírico de. Cf. Filinto, IX, 115.
\section{Panegírico}
\begin{itemize}
\item {Grp. gram.:m.}
\end{itemize}
\begin{itemize}
\item {Utilização:Ext.}
\end{itemize}
\begin{itemize}
\item {Grp. gram.:Adj.}
\end{itemize}
\begin{itemize}
\item {Proveniência:(Gr. \textunderscore panegurikos\textunderscore )}
\end{itemize}
Discurso em louvor do alguém.
Elogio.
Próprio para louvar.
\section{Panegiriqueiro}
\begin{itemize}
\item {Grp. gram.:m.}
\end{itemize}
\begin{itemize}
\item {Utilização:Deprec.}
\end{itemize}
O mesmo que \textunderscore panegirista\textunderscore . Cf. Filinto, IV, 163.
\section{Panegirista}
\begin{itemize}
\item {Grp. gram.:m.  e  f.}
\end{itemize}
\begin{itemize}
\item {Proveniência:(Gr. \textunderscore panegurystes\textunderscore )}
\end{itemize}
Pessôa, que faz um panegírico; pessôa, que elogia.
\section{Panegyricado}
\begin{itemize}
\item {Grp. gram.:adj.}
\end{itemize}
Que contém panegýrico:«\textunderscore ...versos panegyricados.\textunderscore »Filinto, VIII, 243.
\section{Panegyrical}
\begin{itemize}
\item {Grp. gram.:adj.}
\end{itemize}
\begin{itemize}
\item {Proveniência:(De \textunderscore panegýrico\textunderscore )}
\end{itemize}
Relativo a panegýrico; que envolve louvor ou panegýrico. Cf. Filinto, II, 203.
\section{Panegyricar}
\begin{itemize}
\item {Grp. gram.:v. t.}
\end{itemize}
Fazer o panegýrico de. Cf. Filinto, IX, 115.
\section{Panegýrico}
\begin{itemize}
\item {Grp. gram.:m.}
\end{itemize}
\begin{itemize}
\item {Utilização:Ext.}
\end{itemize}
\begin{itemize}
\item {Grp. gram.:Adj.}
\end{itemize}
\begin{itemize}
\item {Proveniência:(Gr. \textunderscore panegurikos\textunderscore )}
\end{itemize}
Discurso em louvor do alguém.
Elogio.
Próprio para louvar.
\section{Panegyriqueiro}
\begin{itemize}
\item {Grp. gram.:m.}
\end{itemize}
\begin{itemize}
\item {Utilização:Deprec.}
\end{itemize}
O mesmo que \textunderscore panegyrista\textunderscore . Cf. Filinto, IV, 163.
\section{Panegyrista}
\begin{itemize}
\item {Grp. gram.:m.  e  f.}
\end{itemize}
\begin{itemize}
\item {Proveniência:(Gr. \textunderscore panegurystes\textunderscore )}
\end{itemize}
Pessôa, que faz um panegýrico; pessôa, que elogia.
\section{Paneiro}
\begin{itemize}
\item {Grp. gram.:m.}
\end{itemize}
\begin{itemize}
\item {Proveniência:(Do fr. \textunderscore panier\textunderscore )}
\end{itemize}
Espécie de cesto.
Bancada na ré dos pequenos barcos, destinada aos passageiros.
Solho móvel dessa parte da ré.
Espécie de carruagem de vêrga. Cf. Ortigão, \textunderscore Hollanda\textunderscore , 175.
\section{Paneiro}
\begin{itemize}
\item {Grp. gram.:m.}
\end{itemize}
\begin{itemize}
\item {Utilização:Prov.}
\end{itemize}
\begin{itemize}
\item {Utilização:alent.}
\end{itemize}
Vendedor ambulante de panos ou fazendas de algodão.
\section{Panejamento}
\begin{itemize}
\item {Grp. gram.:m.}
\end{itemize}
Acto ou effeito de panejar.
\section{Panejar}
\begin{itemize}
\item {Grp. gram.:v. t.}
\end{itemize}
\begin{itemize}
\item {Grp. gram.:V. i.}
\end{itemize}
\begin{itemize}
\item {Proveniência:(De \textunderscore pano\textunderscore )}
\end{itemize}
Pintar as vestes de; representar vestido.
Abanar, agitar-se, (falando-se do pano de um navio).
\section{Panela}
\begin{itemize}
\item {Grp. gram.:f.}
\end{itemize}
\begin{itemize}
\item {Utilização:Bras}
\end{itemize}
\begin{itemize}
\item {Utilização:T. do Fundão}
\end{itemize}
\begin{itemize}
\item {Utilização:Prov.}
\end{itemize}
\begin{itemize}
\item {Utilização:Gír.}
\end{itemize}
\begin{itemize}
\item {Utilização:Gír.}
\end{itemize}
Vaso, mais ou menos fundo, de barro ou metal, para serviço culinário.
Cada um dos compartimentos subterrâneos, de que se compõe um formigueiro de saúba.
Um quarto de alqueire, em medida de azeite ou mel.
Respiração ruidosa, nos doentes ou agonizantes; pieira, farfalheira; estertor.
Carruagem.
Nádegas.
(Dem. do lat. vulgar \textunderscore pana\textunderscore , do lat. \textunderscore pátina\textunderscore )
\section{Panelada}
\begin{itemize}
\item {Grp. gram.:f.}
\end{itemize}
\begin{itemize}
\item {Utilização:Fam.}
\end{itemize}
\begin{itemize}
\item {Proveniência:(De \textunderscore panela\textunderscore )}
\end{itemize}
Aquillo que uma panela póde conter.
Grande porção de panelas.
Accumulação de mucosidades na larynge e nos bronchios.
Ruído, que o ar produz, passando por essas mucosidades.
\section{Paneleira}
\begin{itemize}
\item {Grp. gram.:f.}
\end{itemize}
\begin{itemize}
\item {Utilização:Prov.}
\end{itemize}
\begin{itemize}
\item {Utilização:beir.}
\end{itemize}
Mulhér, que faz ou vende panelas de barro preto.
(Cp. \textunderscore paneleiro\textunderscore )
\section{Paneleiro}
\begin{itemize}
\item {Grp. gram.:m.}
\end{itemize}
\begin{itemize}
\item {Utilização:Prov.}
\end{itemize}
\begin{itemize}
\item {Utilização:beir.}
\end{itemize}
\begin{itemize}
\item {Grp. gram.:Adj.}
\end{itemize}
\begin{itemize}
\item {Utilização:Prov.}
\end{itemize}
Fabricante ou vendedor de panelas de barro preto.
Diz-se de uma espécie de cabresto. Cf. Camillo, \textunderscore Esboços\textunderscore , 116, (3.^a ed.).
\section{Panelênico}
\begin{itemize}
\item {Grp. gram.:adj.}
\end{itemize}
Relativo ao panelenismo.
\section{Panelenismo}
\begin{itemize}
\item {Grp. gram.:m.}
\end{itemize}
\begin{itemize}
\item {Proveniência:(Do gr. \textunderscore pan\textunderscore  + \textunderscore hellen\textunderscore )}
\end{itemize}
Tendência dos Gregos para constituírem uma só nacionalidade.
\section{Panelicídio}
\begin{itemize}
\item {Grp. gram.:m.}
\end{itemize}
\begin{itemize}
\item {Utilização:Burl.}
\end{itemize}
\begin{itemize}
\item {Proveniência:(De \textunderscore panela\textunderscore  + lat. \textunderscore caedere\textunderscore )}
\end{itemize}
Acto de partir panelas. Cf. Herculano, \textunderscore Cister\textunderscore , II, 26.
\section{Panelinha}
\begin{itemize}
\item {Grp. gram.:f.}
\end{itemize}
Pequena panela.
 \textunderscore Fig.\textunderscore  e \textunderscore pop.\textunderscore 
Conluio para fins pouco decorosos.
Intriga.
Súcia.
(Dem. de \textunderscore panela\textunderscore )
\section{Panelo}
\begin{itemize}
\item {fónica:nê}
\end{itemize}
\begin{itemize}
\item {Grp. gram.:m.}
\end{itemize}
\begin{itemize}
\item {Utilização:Prov.}
\end{itemize}
Pequena panela de barro.
Respiração ruidosa, pieira, panela.
\section{Panema}
\begin{itemize}
\item {Grp. gram.:m. ,  f.  e  adj.}
\end{itemize}
\begin{itemize}
\item {Utilização:Bras. do N}
\end{itemize}
\begin{itemize}
\item {Utilização:Bras. do N}
\end{itemize}
Pessôa infeliz.
Pessôa que, indo á caça ou á pesca, nada colheu.
Mollangueirão; imbecil; mau.
Pessôa enguiçada, pessôa a quem fizeram feitiços.
(Do tupi \textunderscore panêua\textunderscore )
\section{Panete}
\begin{itemize}
\item {fónica:nê}
\end{itemize}
\begin{itemize}
\item {Grp. gram.:f.}
\end{itemize}
\begin{itemize}
\item {Utilização:Des.}
\end{itemize}
\begin{itemize}
\item {Proveniência:(Do lat. \textunderscore panis\textunderscore )}
\end{itemize}
Pequeno pão.
\section{Panetela}
\begin{itemize}
\item {Grp. gram.:f.}
\end{itemize}
Variedade de charuto havanês, comprido e delgado.
\section{Pânfilo}
\begin{itemize}
\item {Grp. gram.:m.}
\end{itemize}
(?):«\textunderscore ...bello como outros panfilos que ahi ha.\textunderscore »\textunderscore Aulegrafia\textunderscore , 176.
(Relaciona-se com o lat. \textunderscore Pamphilus\textunderscore , n. p.?)
\section{Panfletário}
\begin{itemize}
\item {Grp. gram.:adj.}
\end{itemize}
\begin{itemize}
\item {Grp. gram.:M.}
\end{itemize}
Relativo a panfleto.
Aquele que faz panfletos.
\section{Panfleteiro}
\begin{itemize}
\item {Grp. gram.:m.  e  adj.}
\end{itemize}
\begin{itemize}
\item {Utilização:Deprec.}
\end{itemize}
O mesmo que \textunderscore panfletário\textunderscore .
\section{Panfletista}
\begin{itemize}
\item {Grp. gram.:m.  e  f.}
\end{itemize}
Pessôa, que escreve panfletos.
\section{Panfleto}
\begin{itemize}
\item {fónica:flê}
\end{itemize}
\begin{itemize}
\item {Grp. gram.:m.}
\end{itemize}
\begin{itemize}
\item {Proveniência:(Ingl. \textunderscore pamphlet\textunderscore )}
\end{itemize}
Pequeno livro, folheto, especialmente destinado a assuntos políticos, e escrito em estilo violento.
\section{Panga}
\begin{itemize}
\item {Grp. gram.:m.}
\end{itemize}
\begin{itemize}
\item {Utilização:Prov.}
\end{itemize}
\begin{itemize}
\item {Utilização:alg.}
\end{itemize}
\begin{itemize}
\item {Grp. gram.:F.}
\end{itemize}
Homem mulherengo.
Mulhér mollangueirona.
\section{Panga}
\begin{itemize}
\item {Grp. gram.:m.}
\end{itemize}
Um dos dialectos das Filippinas.
\section{Pangaia}
\begin{itemize}
\item {Grp. gram.:f.}
\end{itemize}
Espécie de remo africano. Cf. Capello e Ivens, II, 212.
\section{Pangaiada}
\begin{itemize}
\item {Grp. gram.:f.}
\end{itemize}
Porção de pangaios^1. Cf. \textunderscore Agostinheida\textunderscore , 136.
\section{Pangaiar}
\begin{itemize}
\item {Grp. gram.:v. i.}
\end{itemize}
Guiar um pangaio^1; remar.
\section{Pangaio}
\begin{itemize}
\item {Grp. gram.:m.}
\end{itemize}
\begin{itemize}
\item {Utilização:Prov.}
\end{itemize}
\begin{itemize}
\item {Utilização:minh.}
\end{itemize}
\begin{itemize}
\item {Utilização:Prov.}
\end{itemize}
\begin{itemize}
\item {Utilização:trasm.}
\end{itemize}
Pequena embarcação asiática.
Mandrião; rapaz que trabalha pouco.
O mesmo que \textunderscore peralvilho\textunderscore .
\section{Pangaio}
\begin{itemize}
\item {Grp. gram.:m.}
\end{itemize}
\begin{itemize}
\item {Utilização:Prov.}
\end{itemize}
\begin{itemize}
\item {Utilização:alg.}
\end{itemize}
Plataforma coberta, nas estações do caminho de ferro, também conhecida por \textunderscore marquesa\textunderscore .
\section{Pangajôa}
\begin{itemize}
\item {Grp. gram.:f.}
\end{itemize}
Embarcação asiática.
(Cp. \textunderscore pangaio\textunderscore ^1)
\section{Pangaré}
\begin{itemize}
\item {Grp. gram.:m.}
\end{itemize}
\begin{itemize}
\item {Utilização:Bras}
\end{itemize}
\begin{itemize}
\item {Grp. gram.:M. ,  f.  e  adj.}
\end{itemize}
\begin{itemize}
\item {Utilização:Bras. do S}
\end{itemize}
Cavallo estragado, reles.
Diz-se do cavallo mais claro que o doiradilho.
\section{Pangeiro}
\begin{itemize}
\item {Grp. gram.:m.}
\end{itemize}
Árvore do Damão, (\textunderscore erythreina indica\textunderscore ).
\section{Pangelíngua}
\begin{itemize}
\item {Grp. gram.:f. Loc.}
\end{itemize}
\begin{itemize}
\item {Utilização:trasm}
\end{itemize}
\begin{itemize}
\item {Utilização:pop.}
\end{itemize}
\textunderscore Lêr a alguem a pangelíngua\textunderscore , cantar-lhe um parôlo, dizer-lhe tudo sem papas na língua.
(Da loc. lat. \textunderscore pange\textunderscore , \textunderscore lingua\textunderscore  de um hymno de San-Thomás, que se canta em algumas solennidades ecclesiásticas)
\section{Pangelungos}
\begin{itemize}
\item {Grp. gram.:m. pl.}
\end{itemize}
Antigo povo do reino do Congo. Cf. Barros, \textunderscore Déc.\textunderscore  I, l. VIII, c. 4.
\section{Pangermanismo}
\begin{itemize}
\item {Grp. gram.:m.}
\end{itemize}
\begin{itemize}
\item {Proveniência:(De \textunderscore pan...\textunderscore  + \textunderscore germanismo\textunderscore )}
\end{itemize}
Systema político, tendente a reunir ao império alemão todos os povos germânicos. Cf. A. Candido, \textunderscore Philos. Polit.\textunderscore , 61.
\section{Pangermanista}
\begin{itemize}
\item {Grp. gram.:m.}
\end{itemize}
Partidário do pangermanismo.
\section{Pangeu}
\begin{itemize}
\item {Grp. gram.:adj.}
\end{itemize}
\begin{itemize}
\item {Proveniência:(Lat. \textunderscore pangaeus\textunderscore )}
\end{itemize}
Relativo ao monte Pangeia:«\textunderscore ...pangeas assomadas...\textunderscore »Castilho, \textunderscore Geórg.\textunderscore 
\section{Pangiáceas}
\begin{itemize}
\item {Grp. gram.:f. pl.}
\end{itemize}
\begin{itemize}
\item {Proveniência:(Do lat. bot. \textunderscore pangium\textunderscore )}
\end{itemize}
Família de plantas, sepáradas das bixáceas, e que comprehendem as que tem um número definido de estames, igual ao das pétalas.
\section{Pango}
\begin{itemize}
\item {Grp. gram.:m.}
\end{itemize}
Erva myrtácea (\textunderscore cannabis indica\textunderscore ), também conhecida por \textunderscore leamba\textunderscore , e que é fumada por alguns indígenas da África.
\section{Pangolim}
\begin{itemize}
\item {Grp. gram.:m.}
\end{itemize}
Mammífero africano.
\section{Pangueira}
\begin{itemize}
\item {Grp. gram.:f.}
\end{itemize}
Árvore de Moçambique, (\textunderscore terminalia chebula\textunderscore ).
\section{Panguejava}
\begin{itemize}
\item {Grp. gram.:f.}
\end{itemize}
\begin{itemize}
\item {Utilização:Ant.}
\end{itemize}
Embarcação asiática, comprida, chata e muito veloz. Cf. Castanheda, \textunderscore Descobr.\textunderscore , III, 175.
\section{Panhames}
\begin{itemize}
\item {Grp. gram.:m. pl.}
\end{itemize}
\begin{itemize}
\item {Utilização:Bras}
\end{itemize}
Tríbo de aborígenes de Mato-Grosso.
\section{Panhão}
\begin{itemize}
\item {Grp. gram.:m.}
\end{itemize}
\begin{itemize}
\item {Utilização:Prov.}
\end{itemize}
\begin{itemize}
\item {Utilização:minh.}
\end{itemize}
Rapaz ou rapariga acanhada.
\section{Panharmónico}
\begin{itemize}
\item {fónica:nar}
\end{itemize}
\begin{itemize}
\item {Grp. gram.:m.}
\end{itemize}
\begin{itemize}
\item {Proveniência:(De \textunderscore pan...\textunderscore  + \textunderscore harmonia\textunderscore )}
\end{itemize}
Espécie de órgão, que toca árias, imitando differentes instrumentos de sopro.
\section{Panharmónio}
\begin{itemize}
\item {fónica:nar}
\end{itemize}
\begin{itemize}
\item {Grp. gram.:m.}
\end{itemize}
O mesmo ou melhor que \textunderscore panharmónico\textunderscore .
\section{Panheira}
\begin{itemize}
\item {Grp. gram.:f.}
\end{itemize}
Planta aromática da Índia portuguesa, que parece sêr o mesmo que sumaúma. Cf. Delgado, \textunderscore Flora\textunderscore , 21.
\section{Panhellênico}
\begin{itemize}
\item {fónica:ne}
\end{itemize}
\begin{itemize}
\item {Grp. gram.:adj.}
\end{itemize}
Relativo ao panhellenismo.
\section{Panhellenismo}
\begin{itemize}
\item {fónica:ne}
\end{itemize}
\begin{itemize}
\item {Grp. gram.:m.}
\end{itemize}
\begin{itemize}
\item {Proveniência:(Do gr. \textunderscore pan\textunderscore  + \textunderscore hellen\textunderscore )}
\end{itemize}
Tendência dos Gregos para constituírem uma só nacionalidade.
\section{Panho}
\begin{itemize}
\item {Grp. gram.:m.}
\end{itemize}
\begin{itemize}
\item {Utilização:Ant.}
\end{itemize}
O mesmo que \textunderscore pano\textunderscore .
\section{Panhota}
\begin{itemize}
\item {Grp. gram.:f.}
\end{itemize}
\begin{itemize}
\item {Utilização:Prov.}
\end{itemize}
\begin{itemize}
\item {Utilização:alent.}
\end{itemize}
\begin{itemize}
\item {Proveniência:(Do rad. do lat. \textunderscore panis\textunderscore )}
\end{itemize}
Pão pequeno.
\section{Panhote}
\begin{itemize}
\item {Grp. gram.:m.}
\end{itemize}
\begin{itemize}
\item {Utilização:Bras}
\end{itemize}
Monte, que tem o aspecto de um pão.
(Cp. \textunderscore panhota\textunderscore )
\section{Panical}
\begin{itemize}
\item {Grp. gram.:m.}
\end{itemize}
\begin{itemize}
\item {Utilização:Ant.}
\end{itemize}
Mestre de armas, na Índia portuguesa.
\section{Paníceas}
\begin{itemize}
\item {Grp. gram.:f. pl.}
\end{itemize}
\begin{itemize}
\item {Proveniência:(De \textunderscore paníceo\textunderscore )}
\end{itemize}
Gênero de plantas, que tem por typo o pânico^2.
\section{Paníceo}
\begin{itemize}
\item {Grp. gram.:adj.}
\end{itemize}
Relativo ou semelhante ao \textunderscore pânico\textunderscore ^2.
\section{Pânico}
\begin{itemize}
\item {Grp. gram.:adj.}
\end{itemize}
\begin{itemize}
\item {Grp. gram.:M.}
\end{itemize}
\begin{itemize}
\item {Proveniência:(Lat. \textunderscore panicus\textunderscore )}
\end{itemize}
Que assusta sem motivos.
Terror infundado.
\section{Pânico}
\begin{itemize}
\item {Grp. gram.:m.}
\end{itemize}
\begin{itemize}
\item {Proveniência:(Lat. \textunderscore panicum\textunderscore )}
\end{itemize}
Gênero de plantas, a que pertence o paínço.
\section{Paniconografia}
\begin{itemize}
\item {Grp. gram.:f.}
\end{itemize}
\begin{itemize}
\item {Proveniência:(Do gr. \textunderscore pan\textunderscore  + \textunderscore eicon\textunderscore  + \textunderscore graphein\textunderscore )}
\end{itemize}
Processo de gravura química. Cf. R. Galvão, \textunderscore Vocab.\textunderscore 
\section{Paniconographia}
\begin{itemize}
\item {Grp. gram.:f.}
\end{itemize}
\begin{itemize}
\item {Proveniência:(Do gr. \textunderscore pan\textunderscore  + \textunderscore eicon\textunderscore  + \textunderscore graphein\textunderscore )}
\end{itemize}
Processo de gravura chímica. Cf. R. Galvão, \textunderscore Vocab.\textunderscore 
\section{Panícula}
\begin{itemize}
\item {Grp. gram.:f.}
\end{itemize}
\begin{itemize}
\item {Utilização:Bot.}
\end{itemize}
\begin{itemize}
\item {Utilização:Veter.}
\end{itemize}
\begin{itemize}
\item {Proveniência:(Lat. \textunderscore panícula\textunderscore )}
\end{itemize}
Inflorescência, caracterizada pela reunião de espigas, formando cachos.
Pterýgio, cuja prega chega a cobrir toda a córnea do ôlho. Cf. Macedo Pinto, \textunderscore Comp. de Veter.\textunderscore , I, 261.
\section{Paniculado}
\begin{itemize}
\item {Grp. gram.:adj.}
\end{itemize}
Que tem panícula; panicular.
\section{Panicular}
\begin{itemize}
\item {Grp. gram.:adj.}
\end{itemize}
Que tem fórma de panícula.
\section{Paniculários}
\begin{itemize}
\item {Grp. gram.:m. pl.}
\end{itemize}
\begin{itemize}
\item {Proveniência:(Lat. \textunderscore pannicularia\textunderscore )}
\end{itemize}
Despojos ou vestes, que pertenceram aos justiçados, na antiguidade romana.
\section{Panículo}
\begin{itemize}
\item {Grp. gram.:m.}
\end{itemize}
\begin{itemize}
\item {Utilização:Prov.}
\end{itemize}
\begin{itemize}
\item {Utilização:alent.}
\end{itemize}
Membrana serosa, que reveste as vísceras.
(Cp. \textunderscore panícula\textunderscore )
\section{Panicum}
\begin{itemize}
\item {Grp. gram.:m.}
\end{itemize}
\begin{itemize}
\item {Utilização:Bras. do N}
\end{itemize}
Espécie de cesto.
\section{Paniégo}
\begin{itemize}
\item {Grp. gram.:adj.}
\end{itemize}
\begin{itemize}
\item {Utilização:Prov.}
\end{itemize}
\begin{itemize}
\item {Utilização:trasm.}
\end{itemize}
\begin{itemize}
\item {Proveniência:(Do rad. do lat. \textunderscore panis\textunderscore )}
\end{itemize}
Que gosta muito de pão; que come muito pão.
\section{Panífero}
\begin{itemize}
\item {Grp. gram.:adj.}
\end{itemize}
\begin{itemize}
\item {Utilização:Poét.}
\end{itemize}
\begin{itemize}
\item {Proveniência:(Do lat. \textunderscore panis\textunderscore  + \textunderscore ferre\textunderscore )}
\end{itemize}
Que produz cereaes. Cf. Castilho, \textunderscore Fastos\textunderscore , III, 483.
\section{Panificação}
\begin{itemize}
\item {Grp. gram.:f.}
\end{itemize}
Acto ou effeito de panificar.
\section{Panificador}
\begin{itemize}
\item {Grp. gram.:m.}
\end{itemize}
\begin{itemize}
\item {Proveniência:(De \textunderscore panificar\textunderscore )}
\end{itemize}
Fabricante de pão.
\section{Panificar}
\begin{itemize}
\item {Grp. gram.:v. t.}
\end{itemize}
\begin{itemize}
\item {Proveniência:(Lat. \textunderscore panificare\textunderscore )}
\end{itemize}
Converter em pão.
\section{Panificável}
\begin{itemize}
\item {Grp. gram.:adj.}
\end{itemize}
Que se póde panificar.
\section{Paniguado}
\begin{itemize}
\item {Grp. gram.:adj.}
\end{itemize}
O mesmo que \textunderscore apaniguado\textunderscore .
\section{Paninho}
\begin{itemize}
\item {Grp. gram.:m.}
\end{itemize}
Pano fino de algodão.
(Dem. de \textunderscore pano\textunderscore )
\section{Panino}
\begin{itemize}
\item {Grp. gram.:m.}
\end{itemize}
\begin{itemize}
\item {Utilização:Prov.}
\end{itemize}
\begin{itemize}
\item {Utilização:alg.}
\end{itemize}
O mesmo que \textunderscore paninho\textunderscore .
\section{Panja}
\begin{itemize}
\item {Grp. gram.:m.}
\end{itemize}
Antiga medida de Moçambique, correspondente a pouco mais de 5 litros.
\section{Panjabi}
\begin{itemize}
\item {Grp. gram.:m.}
\end{itemize}
Dialecto de sânscrito.
\section{Panjorca}
\begin{itemize}
\item {Grp. gram.:f.}
\end{itemize}
\begin{itemize}
\item {Utilização:Prov.}
\end{itemize}
\begin{itemize}
\item {Utilização:minh.}
\end{itemize}
Mulhér corpulenta e desajeitada.
(Cp. \textunderscore pandorga\textunderscore )
\section{Panmastite}
\begin{itemize}
\item {Grp. gram.:f.}
\end{itemize}
\begin{itemize}
\item {Utilização:Med.}
\end{itemize}
\begin{itemize}
\item {Proveniência:(Do gr. \textunderscore pan\textunderscore  + \textunderscore mastos\textunderscore )}
\end{itemize}
Inflammação total da mama ou fleimão diffuso do seio.
\section{Panniculários}
\begin{itemize}
\item {Grp. gram.:m. pl.}
\end{itemize}
\begin{itemize}
\item {Proveniência:(Lat. \textunderscore pannicularia\textunderscore )}
\end{itemize}
Despojos ou vestes, que pertenceram aos justiçados, na antiguidade romana.
\section{Pannónios}
\begin{itemize}
\item {Grp. gram.:m. pl.}
\end{itemize}
Habitantes da Pannónia. Cf. \textunderscore Lusíadas\textunderscore , III, 11.
\section{Pano}
\begin{itemize}
\item {Grp. gram.:m.}
\end{itemize}
\begin{itemize}
\item {Utilização:Prov.}
\end{itemize}
\begin{itemize}
\item {Utilização:alent.}
\end{itemize}
\begin{itemize}
\item {Utilização:P. us.}
\end{itemize}
\begin{itemize}
\item {Grp. gram.:Pl.   Loc.}
\end{itemize}
\begin{itemize}
\item {Utilização:pop.}
\end{itemize}
\begin{itemize}
\item {Proveniência:(Lat. \textunderscore panus\textunderscore )}
\end{itemize}
Tecido, feito de fio de linho, algodão, lan, etc.
Velas de um navio.
Nódoas, que apparecem na pelle, produzidas por gravidez ou soffrimento hepáthico.
Lado de uma construcção que tem mais de uma face.
Parte interior da chaminé, fronteira e superior ao lar.
Infiltração pathológica vasculizada da superfície córnea.
O mesmo que \textunderscore guardanapo\textunderscore .
O mesmo que \textunderscore tanga\textunderscore ^1.
\textunderscore Panos de armar\textunderscore , imposturas, simulações.
\section{Pano-cru}
\begin{itemize}
\item {Grp. gram.:m.}
\end{itemize}
Variedade de tecido de algodão, que não foi còrado depois da tecedura.
\section{Panoftalmia}
\begin{itemize}
\item {Grp. gram.:f.}
\end{itemize}
O mesmo que \textunderscore panoftalmite\textunderscore .
\section{Panoftalmite}
\begin{itemize}
\item {Grp. gram.:f.}
\end{itemize}
\begin{itemize}
\item {Proveniência:(Do gr. \textunderscore pan\textunderscore  + \textunderscore ophthalmos\textunderscore )}
\end{itemize}
Inflamação generalizada do ôlho, incluindo o tecido orbitário.
\section{Panónios}
\begin{itemize}
\item {Grp. gram.:m. pl.}
\end{itemize}
Habitantes da Panónia. Cf. \textunderscore Lusíadas\textunderscore , III, 11.
\section{Panophtalmia}
\begin{itemize}
\item {Grp. gram.:f.}
\end{itemize}
O mesmo que \textunderscore panophthalmite\textunderscore .
\section{Panophthalmite}
\begin{itemize}
\item {Grp. gram.:f.}
\end{itemize}
\begin{itemize}
\item {Proveniência:(Do gr. \textunderscore pan\textunderscore  + \textunderscore ophthalmos\textunderscore )}
\end{itemize}
Inflammação generalizada do ôlho, incluindo o tecido orbitário.
\section{Panópiro}
\begin{itemize}
\item {Grp. gram.:adj.}
\end{itemize}
\begin{itemize}
\item {Utilização:Zool.}
\end{itemize}
\begin{itemize}
\item {Proveniência:(Do gr. \textunderscore pan\textunderscore  + \textunderscore pur\textunderscore )}
\end{itemize}
Que é fosforescente em toda a sua extensão, (falando-se de uma espécie de alforrecas ou pelágias).
\section{Panóplia}
\begin{itemize}
\item {Grp. gram.:f.}
\end{itemize}
\begin{itemize}
\item {Proveniência:(Do gr. \textunderscore pan\textunderscore  + \textunderscore opla\textunderscore )}
\end{itemize}
Armadura de um cavalleiro da Idade-Média.
Escudo, em que se collocam differentes armas, e com que se adornam paredes.
Casa de armas.
Trophéu.
\section{Panópyro}
\begin{itemize}
\item {Grp. gram.:adj.}
\end{itemize}
\begin{itemize}
\item {Utilização:Zool.}
\end{itemize}
\begin{itemize}
\item {Proveniência:(Do gr. \textunderscore pan\textunderscore  + \textunderscore pur\textunderscore )}
\end{itemize}
Que é fosforescente em toda a sua extensão, (falando-se de uma espécie de alforrecas ou pelágias)
\section{Panorama}
\begin{itemize}
\item {Grp. gram.:m.}
\end{itemize}
\begin{itemize}
\item {Utilização:Ext.}
\end{itemize}
\begin{itemize}
\item {Utilização:Fig.}
\end{itemize}
\begin{itemize}
\item {Proveniência:(Do gr. \textunderscore pan\textunderscore  + \textunderscore orama\textunderscore )}
\end{itemize}
Grande quadro cylíndrico, collocado do maneira que o espectador, estando no centro, vô os objectos como se estivesse observando no alto de uma montanha todo o horizonte que o rodeasse.
Paisagem.
Quadro, que representa vista extensa.
Grande exposição.
\section{Panoramágrafo}
\begin{itemize}
\item {Grp. gram.:m.}
\end{itemize}
\begin{itemize}
\item {Proveniência:(De \textunderscore panorama\textunderscore  + gr. \textunderscore graphein\textunderscore )}
\end{itemize}
Um dos nomes, mais ou menos caprichosos, que ultimamente se têm dado ao cinematógrafo.
\section{Panoramágrapho}
\begin{itemize}
\item {Grp. gram.:m.}
\end{itemize}
\begin{itemize}
\item {Proveniência:(De \textunderscore panorama\textunderscore  + gr. \textunderscore graphein\textunderscore )}
\end{itemize}
Um dos nomes, mais ou menos caprichosos, que ultimamente se têm dado ao cinematógrapho.
\section{Panorâmico}
\begin{itemize}
\item {Grp. gram.:adj.}
\end{itemize}
Relativo a panorama, relativo a paisagens. Cf. Capello e Ivens, II, 115.
\section{Panórgão}
\begin{itemize}
\item {Grp. gram.:m.}
\end{itemize}
Instrumento duplo, composto de um piano e um pequeno harmónio.
\section{Panorógrafo}
\begin{itemize}
\item {Grp. gram.:m.}
\end{itemize}
\begin{itemize}
\item {Proveniência:(Do gr. \textunderscore pan\textunderscore  + \textunderscore orao\textunderscore  + \textunderscore graphein\textunderscore )}
\end{itemize}
Instrumento, com que se obtém rapidamente numa superficie plana o desenvolvimento de uma perspectiva circular. Cf. R. Galvão, \textunderscore Vocab\textunderscore .
\section{Panorógrapho}
\begin{itemize}
\item {Grp. gram.:m.}
\end{itemize}
\begin{itemize}
\item {Proveniência:(Do gr. \textunderscore pan\textunderscore  + \textunderscore orao\textunderscore  + \textunderscore graphein\textunderscore )}
\end{itemize}
Instrumento, com que se obtém rapidamente numa superficie plana o desenvolvimento de uma perspectiva circular. Cf. R. Galvão, \textunderscore Vocab\textunderscore .
\section{Panorpa}
\begin{itemize}
\item {Grp. gram.:f.}
\end{itemize}
\begin{itemize}
\item {Proveniência:(Do gr. \textunderscore pan\textunderscore  + \textunderscore orpe\textunderscore )}
\end{itemize}
Gênero de insectos neurópteros.
\section{Panos}
\begin{itemize}
\item {Grp. gram.:m. pl.}
\end{itemize}
Indígenas do norte do Brasil.
\section{Panosteíte}
\begin{itemize}
\item {Grp. gram.:f.}
\end{itemize}
\begin{itemize}
\item {Utilização:Med.}
\end{itemize}
\begin{itemize}
\item {Proveniência:(Do gr. \textunderscore pan\textunderscore  + \textunderscore osteon\textunderscore )}
\end{itemize}
Osteomyelite aguda.
\section{Panotilha}
\begin{itemize}
\item {Grp. gram.:m.}
\end{itemize}
\begin{itemize}
\item {Utilização:Prov.}
\end{itemize}
\begin{itemize}
\item {Utilização:minh.}
\end{itemize}
O mesmo que \textunderscore pandilha\textunderscore ^1; bisbórria; pulha.
\section{Panoura}
\begin{itemize}
\item {Grp. gram.:f.}
\end{itemize}
Embarcação asiática:«\textunderscore sessenta alabardeiros com panouras e alabardas...\textunderscore »\textunderscore Peregrinação\textunderscore , LXVIII.
\section{Pan-psychismo}
\begin{itemize}
\item {Grp. gram.:m.}
\end{itemize}
O mesmo que \textunderscore psycho-dynamismo\textunderscore .
\section{Panqueca}
\begin{itemize}
\item {Grp. gram.:f.}
\end{itemize}
\begin{itemize}
\item {Utilização:Bras}
\end{itemize}
Doce, feito de farinha de trigo, ovos, leite, manteiga, etc.
\textunderscore Estar na panqueca\textunderscore , estar em descanso, estar ocioso.
\section{Panquimagogo}
\begin{itemize}
\item {Grp. gram.:m.  e  adj.}
\end{itemize}
\begin{itemize}
\item {Proveniência:(Gr. \textunderscore pankumagogos\textunderscore )}
\end{itemize}
Diz-se das substâncias purgativas, a que se atribuía a faculdade de expulsar todos os humores.
\section{Pânria}
\begin{itemize}
\item {Grp. gram.:f.}
\end{itemize}
\begin{itemize}
\item {Utilização:Pop.}
\end{itemize}
\begin{itemize}
\item {Grp. gram.:M.  e  f.}
\end{itemize}
Mandriice, indolência.
Pessôa indolente, preguiçosa.
\section{Panriar}
\begin{itemize}
\item {Grp. gram.:v. i.}
\end{itemize}
\begin{itemize}
\item {Proveniência:(De \textunderscore pânria\textunderscore )}
\end{itemize}
Mandriar, viver na indolência ou ociosidade.
\section{Panró}
\begin{itemize}
\item {Grp. gram.:m.}
\end{itemize}
Nome, que na Índia portuguesa se dá á cobra-capello.
\section{Panslavismo}
\begin{itemize}
\item {Grp. gram.:m.}
\end{itemize}
\begin{itemize}
\item {Proveniência:(De \textunderscore pan...\textunderscore  + \textunderscore eslavo\textunderscore )}
\end{itemize}
Systema político da Rússia, tendente a unir todos os povos eslavos ao império moscovita.
\section{Panslavista}
\begin{itemize}
\item {Grp. gram.:adj.}
\end{itemize}
\begin{itemize}
\item {Grp. gram.:M.}
\end{itemize}
Relativo ao panslavismo.
Partidário do panslavismo.
\section{Pansofia}
\begin{itemize}
\item {Grp. gram.:f.}
\end{itemize}
\begin{itemize}
\item {Proveniência:(Do gr. \textunderscore pan\textunderscore  + \textunderscore sophia\textunderscore )}
\end{itemize}
Todo o saber.
A ciência universal.
\section{Pansófico}
\begin{itemize}
\item {Grp. gram.:adj.}
\end{itemize}
Relativo á pansofia. Cf. Th. Braga, \textunderscore Mod. Ideias\textunderscore , II, 218.
\section{Pansophia}
\begin{itemize}
\item {Grp. gram.:f.}
\end{itemize}
\begin{itemize}
\item {Proveniência:(Do gr. \textunderscore pan\textunderscore  + \textunderscore sophia\textunderscore )}
\end{itemize}
Todo o saber.
A sciência universal.
\section{Pansóphico}
\begin{itemize}
\item {Grp. gram.:adj.}
\end{itemize}
Relativo á pansophia. Cf. Th. Braga, \textunderscore Mod. Ideias\textunderscore , II, 218.
\section{Panspermia}
\begin{itemize}
\item {Grp. gram.:f.}
\end{itemize}
\begin{itemize}
\item {Proveniência:(Do gr. \textunderscore pan\textunderscore  + \textunderscore sperma\textunderscore )}
\end{itemize}
Systema dos que entendem que os germes dos seres organizados estão espalhados por toda a parte, aguardando apenas que lhe promovam o desenvolvimento circunstâncias favoráveis.
Ovulação espontânea.
\section{Panspérmico}
\begin{itemize}
\item {Grp. gram.:adj.}
\end{itemize}
Relativo á panspermia.
\section{Panspermista}
\begin{itemize}
\item {Grp. gram.:m.}
\end{itemize}
Sectário da panspermia.
\section{Pantafaçudo}
\begin{itemize}
\item {Grp. gram.:adj.}
\end{itemize}
\begin{itemize}
\item {Utilização:Fig.}
\end{itemize}
Bochechudo.
Ridiculamente extraordinário, exótico.
Monstruoso.
\section{Pantagruélico}
\begin{itemize}
\item {Grp. gram.:adj.}
\end{itemize}
Digno de Pantagruel.
Conforme aos hábitos do Pantagruel.
Relativo a comezanas. Cf. Rebello, \textunderscore Contos e Lendas\textunderscore , 92.
(Cp. \textunderscore pantagruelismo\textunderscore )
\section{Pantagruelismo}
\begin{itemize}
\item {Grp. gram.:m.}
\end{itemize}
\begin{itemize}
\item {Proveniência:(De \textunderscore Pantagruel\textunderscore , n. p.)}
\end{itemize}
Espécie de philosophia epicurista.
Systema dos que se preoccupam exclusivamente dos gozos materiaes da vida.
\section{Pantagruelista}
\begin{itemize}
\item {Grp. gram.:m.}
\end{itemize}
Partidário do pantagruelismo. Cf. Th. Braga, \textunderscore Mod. Ideias\textunderscore , I, 316.
\section{Pantalão}
\begin{itemize}
\item {Grp. gram.:m.}
\end{itemize}
Homem que usa pantalonas.
Peralvilho:«\textunderscore e a cada pantalão que descobris na rua, com o chapéu um pouco arreado para a nuca...\textunderscore »Camillo, \textunderscore Homem de Brios\textunderscore , 54.
\section{Pantaleão}
\begin{itemize}
\item {Grp. gram.:m.}
\end{itemize}
\begin{itemize}
\item {Utilização:Prov.}
\end{itemize}
\begin{itemize}
\item {Utilização:Prov.}
\end{itemize}
\begin{itemize}
\item {Utilização:trasm.}
\end{itemize}
\begin{itemize}
\item {Utilização:Chul.}
\end{itemize}
Homem desmazelado, mal vestido.
O mesmo que \textunderscore pênis\textunderscore .
\section{Pantalha}
\begin{itemize}
\item {Grp. gram.:f.}
\end{itemize}
Bandeira de candeeiro.
Peça de metal, pano ou cartão, com que se resguarda uma luz, para lhe attenuar a intensidade, ou para a fazer reflectir em certo ponto.--Termo usado em parte do Alentejo e muito preferível ao francesismo \textunderscore abaju\textunderscore .
(Cast. \textunderscore pantalla\textunderscore )
\section{Pantalonas}
\begin{itemize}
\item {Grp. gram.:f. pl.}
\end{itemize}
\begin{itemize}
\item {Utilização:Des.}
\end{itemize}
\begin{itemize}
\item {Proveniência:(Do fr. \textunderscore pantalon\textunderscore )}
\end{itemize}
Calças.
\section{Pantana}
\begin{itemize}
\item {Grp. gram.:f.}
\end{itemize}
\begin{itemize}
\item {Utilização:Fam.}
\end{itemize}
\begin{itemize}
\item {Utilização:Gymn.}
\end{itemize}
Ruína, dissipação de haveres: \textunderscore aquelle deu em pantana\textunderscore .
Salto lateral, em que as mãos tocam no chão, em-quanto o corpo executa meia volta.
(Da mesma or. que \textunderscore pântano\textunderscore , cuja pronúncia exacta seria \textunderscore pantâno\textunderscore )
\section{Pantanal}
\begin{itemize}
\item {Grp. gram.:m.}
\end{itemize}
Grande pântano; grande lodaçal.
\section{Pantaneiro}
\begin{itemize}
\item {Grp. gram.:m.}
\end{itemize}
\begin{itemize}
\item {Utilização:Bras}
\end{itemize}
Boi de certa raça de Mato-Grosso.
\section{Pantanizar}
\begin{itemize}
\item {Grp. gram.:v. i.}
\end{itemize}
\begin{itemize}
\item {Utilização:Neol.}
\end{itemize}
Converter em pântano; tornar paludoso. Cf. V. Pinheiro, \textunderscore As Ilhas de S. Thomé,\textunderscore  369.
\section{Pântano}
\begin{itemize}
\item {Grp. gram.:m.}
\end{itemize}
(\textunderscore pantâno\textunderscore  é a pron. exacta, mas des.)
Paúl, atoleiro; lodaçal.
(Cast. \textunderscore pantano\textunderscore )
\section{Pantanoso}
\begin{itemize}
\item {Grp. gram.:adj.}
\end{itemize}
Que tem pântanos; alagadiço.
\section{Panteão}
\begin{itemize}
\item {Grp. gram.:m.}
\end{itemize}
\begin{itemize}
\item {Proveniência:(Lat. \textunderscore pantheon\textunderscore )}
\end{itemize}
Templo da antiga Roma, dedicado a todos os deuses e de fórma redonda.
Igreja, em fórma do pantheão.
Edifício, em que se depositam os restos mortaes daquelles que illustraram a sua pátria ou fizeram grandes serviços á humanidade.
\section{Pantear}
\begin{itemize}
\item {Grp. gram.:v. t.}
\end{itemize}
\begin{itemize}
\item {Grp. gram.:V. i.}
\end{itemize}
Motejar de.
Dizer futilidades.
\section{Panteiro}
\begin{itemize}
\item {Grp. gram.:m.}
\end{itemize}
\begin{itemize}
\item {Utilização:Prov.}
\end{itemize}
\begin{itemize}
\item {Utilização:trasm.}
\end{itemize}
Casarão alto e velho.
\section{Panteísmo}
\begin{itemize}
\item {Grp. gram.:m.}
\end{itemize}
\begin{itemize}
\item {Proveniência:(Do gr. \textunderscore pan\textunderscore  + \textunderscore theos\textunderscore )}
\end{itemize}
Sistema filosófico, em que Deus é a universalidade dos sêres ou o conjunto de tudo quanto existe.
\section{Panteísta}
\begin{itemize}
\item {Grp. gram.:adj.}
\end{itemize}
\begin{itemize}
\item {Grp. gram.:M.}
\end{itemize}
Relativo ao panteísmo.
Sectario do panteísmo.
\section{Pantera}
\begin{itemize}
\item {Grp. gram.:f.}
\end{itemize}
\begin{itemize}
\item {Utilização:Fig.}
\end{itemize}
\begin{itemize}
\item {Proveniência:(Lat. \textunderscore panthera\textunderscore )}
\end{itemize}
Quadrúpede felino, de pele mosqueada, (\textunderscore felis pardus\textunderscore ).
Pessoa furiosa ou cruel.
\section{Pantesma}
\begin{itemize}
\item {Grp. gram.:f.}
\end{itemize}
O mesmo que \textunderscore avantesma\textunderscore . Cf. Castilho, \textunderscore Avarento\textunderscore , 161.
\section{Pantheão}
\begin{itemize}
\item {Grp. gram.:m.}
\end{itemize}
\begin{itemize}
\item {Proveniência:(Lat. \textunderscore pantheon\textunderscore )}
\end{itemize}
Templo da antiga Roma, dedicado a todos os deuses e de fórma redonda.
Igreja, em fórma do pantheão.
Edifício, em que se depositam os restos mortaes daquelles que illustraram a sua pátria ou fizeram grandes serviços á humanidade.
\section{Pantheísmo}
\begin{itemize}
\item {Grp. gram.:m.}
\end{itemize}
\begin{itemize}
\item {Proveniência:(Do gr. \textunderscore pan\textunderscore  + \textunderscore theos\textunderscore )}
\end{itemize}
Systema philosóphico, em que Deus é a universalidade dos sêres ou o conjunto de tudo quanto existe.
\section{Pantheísta}
\begin{itemize}
\item {Grp. gram.:adj.}
\end{itemize}
\begin{itemize}
\item {Grp. gram.:M.}
\end{itemize}
Relativo ao pantheísmo.
Sectario do pantheísmo.
\section{Panthera}
\begin{itemize}
\item {Grp. gram.:f.}
\end{itemize}
\begin{itemize}
\item {Utilização:Fig.}
\end{itemize}
\begin{itemize}
\item {Proveniência:(Lat. \textunderscore panthera\textunderscore )}
\end{itemize}
Quadrúpede felino, de pelle mosqueada, (\textunderscore felis pardus\textunderscore ).
Pessoa furiosa ou cruel.
\section{Pantim}
\begin{itemize}
\item {Grp. gram.:m.}
\end{itemize}
\begin{itemize}
\item {Utilização:Bras. do N}
\end{itemize}
Boato; notícia que póde assustar.
\section{Pantofagia}
\begin{itemize}
\item {Grp. gram.:f.}
\end{itemize}
Hábito ou qualidade de pantófago.
\section{Pantófago}
\begin{itemize}
\item {Grp. gram.:adj.}
\end{itemize}
\begin{itemize}
\item {Proveniência:(Do gr. \textunderscore pantos\textunderscore  + \textunderscore phagein\textunderscore )}
\end{itemize}
Que come muito, que come de tudo indistintamente.
\section{Pantofobia}
\begin{itemize}
\item {Grp. gram.:f.}
\end{itemize}
\begin{itemize}
\item {Proveniência:(Do gr. \textunderscore pantos\textunderscore , tudo, e \textunderscore phobeín\textunderscore , temer)}
\end{itemize}
Fobia complexa, medo de tudo.
\section{Pantofobo}
\begin{itemize}
\item {Grp. gram.:m.}
\end{itemize}
Aquele que tem pantofobia.
\section{Pantogamia}
\begin{itemize}
\item {Grp. gram.:f.}
\end{itemize}
\begin{itemize}
\item {Proveniência:(Do gr. \textunderscore pantos\textunderscore  + \textunderscore gamos\textunderscore )}
\end{itemize}
Modo de procriação, em que os machos e as fêmeas cohabitam com quaesquer animaes de sexo opposto ao seu, em-quanto sentem a necessidade da reproducção.
\section{Pantogâmico}
\begin{itemize}
\item {Grp. gram.:adj.}
\end{itemize}
Relativo á pantogamia.
\section{Pantografia}
\begin{itemize}
\item {Grp. gram.:f.}
\end{itemize}
Aplicação do pantógrafo.
\section{Pantográfico}
\begin{itemize}
\item {Grp. gram.:adj}
\end{itemize}
Relativo á pantografia.
\section{Pantógrafo}
\begin{itemize}
\item {Grp. gram.:m.}
\end{itemize}
\begin{itemize}
\item {Proveniência:(Do gr. \textunderscore pantos\textunderscore  + \textunderscore graphein\textunderscore )}
\end{itemize}
Instrumento, com que se copiam mecanicamente desenhos e gravuras.
\section{Pantographia}
\begin{itemize}
\item {Grp. gram.:f.}
\end{itemize}
Applicação do pantógrapho.
\section{Pantográphico}
\begin{itemize}
\item {Grp. gram.:adj}
\end{itemize}
Relativo á pantographia.
\section{Pantógrapho}
\begin{itemize}
\item {Grp. gram.:m.}
\end{itemize}
\begin{itemize}
\item {Proveniência:(Do gr. \textunderscore pantos\textunderscore  + \textunderscore graphein\textunderscore )}
\end{itemize}
Instrumento, com que se copiam mecanicamente desenhos e gravuras.
\section{Pantólogo}
\begin{itemize}
\item {Grp. gram.:m.}
\end{itemize}
\begin{itemize}
\item {Proveniência:(Gr. \textunderscore pantologos\textunderscore )}
\end{itemize}
Aquelle que sabe de tudo.
Encyclopedista. Cf. Latino, \textunderscore Humboldt\textunderscore , 458.
\section{Pantómetro}
\begin{itemize}
\item {Grp. gram.:m.}
\end{itemize}
\begin{itemize}
\item {Proveniência:(Do gr. \textunderscore pantos\textunderscore  + \textunderscore metron\textunderscore )}
\end{itemize}
Instrumento desusado, composto de três réguas móveis, que servem para determinar os três ângulos de um triângulo.
\section{Pantomima}
\begin{itemize}
\item {Grp. gram.:f.}
\end{itemize}
\begin{itemize}
\item {Utilização:pop.}
\end{itemize}
\begin{itemize}
\item {Utilização:Fig.}
\end{itemize}
Arte ou acto de exprimir as ideias ou sentimentos, por meio de gestos.
Embuste, lôgro.
(Fem. de \textunderscore pantomímo\textunderscore )
\section{Pantomimar}
\begin{itemize}
\item {Grp. gram.:v. i.}
\end{itemize}
Fazer pantomimas.
Lograr alguém. Cf. Castilho, \textunderscore Metam.\textunderscore , p. XV.
\section{Pantomimeiro}
\begin{itemize}
\item {Grp. gram.:m.}
\end{itemize}
Aquelle que pantomima.
\section{Pantomimice}
\begin{itemize}
\item {Grp. gram.:f.}
\end{itemize}
\begin{itemize}
\item {Proveniência:(De \textunderscore pantomima\textunderscore )}
\end{itemize}
Dito ou acto de pantomimeiro.
\section{Pantomímico}
\begin{itemize}
\item {Grp. gram.:adj.}
\end{itemize}
Relativo a pantomima.
\section{Pantomimo}
\begin{itemize}
\item {Grp. gram.:m.}
\end{itemize}
\begin{itemize}
\item {Utilização:Des.}
\end{itemize}
\begin{itemize}
\item {Proveniência:(Gr. \textunderscore pantomimos\textunderscore )}
\end{itemize}
Actor, que representa pantomimas.
\section{Pantomina}
\begin{itemize}
\item {Grp. gram.:f.}
\end{itemize}
\begin{itemize}
\item {Utilização:Pop.}
\end{itemize}
Intrujice, embuste.
(Corr. de \textunderscore pantomima\textunderscore )
\section{Pantominar}
\begin{itemize}
\item {Grp. gram.:v. i.}
\end{itemize}
\begin{itemize}
\item {Utilização:Pop.}
\end{itemize}
O mesmo que \textunderscore pantomimar\textunderscore .
\section{Pantomineiro}
\begin{itemize}
\item {Grp. gram.:m.  e  adj.}
\end{itemize}
\begin{itemize}
\item {Utilização:Pop.}
\end{itemize}
\begin{itemize}
\item {Proveniência:(De \textunderscore pantomina\textunderscore )}
\end{itemize}
Intrujão, trapaceiro.
\section{Pantominice}
\begin{itemize}
\item {Grp. gram.:f.}
\end{itemize}
\begin{itemize}
\item {Utilização:Pop.}
\end{itemize}
O mesmo que \textunderscore pantomina\textunderscore . Cf. Camillo, \textunderscore Brasileira\textunderscore , 168.
\section{Pantopelagiano}
\begin{itemize}
\item {Grp. gram.:adj.}
\end{itemize}
\begin{itemize}
\item {Proveniência:(Do gr. \textunderscore pantos\textunderscore  + \textunderscore pelagos\textunderscore )}
\end{itemize}
Diz-se das aves que cruzam o alto mar.
\section{Pantopelágico}
\begin{itemize}
\item {Grp. gram.:adj.}
\end{itemize}
\begin{itemize}
\item {Utilização:Zool.}
\end{itemize}
\begin{itemize}
\item {Proveniência:(Do gr. \textunderscore pantos\textunderscore  + \textunderscore pelagos\textunderscore )}
\end{itemize}
Diz-se das aves que cruzam o alto mar.
\section{Pantophagia}
\begin{itemize}
\item {Grp. gram.:f.}
\end{itemize}
Hábito ou qualidade de pantóphago.
\section{Pantóphago}
\begin{itemize}
\item {Grp. gram.:adj.}
\end{itemize}
\begin{itemize}
\item {Proveniência:(Do gr. \textunderscore pantos\textunderscore  + \textunderscore phagein\textunderscore )}
\end{itemize}
Que come muito, que come de tudo indistintamente.
\section{Pantophobia}
\begin{itemize}
\item {Grp. gram.:f.}
\end{itemize}
\begin{itemize}
\item {Proveniência:(Do gr. \textunderscore pantos\textunderscore , tudo, e \textunderscore phobeín\textunderscore , temer)}
\end{itemize}
Phobia complexa, medo de tudo.
\section{Pantóphobo}
\begin{itemize}
\item {Grp. gram.:m.}
\end{itemize}
Aquelle que tem pantophobia.
\section{Pantópodes}
\begin{itemize}
\item {Grp. gram.:m. pl.}
\end{itemize}
\begin{itemize}
\item {Utilização:Zool.}
\end{itemize}
\begin{itemize}
\item {Proveniência:(Do gr. \textunderscore pan\textunderscore  + \textunderscore pous\textunderscore , \textunderscore podos\textunderscore )}
\end{itemize}
Grupo de arthrópodes marínhos, de pequeno corpo e enormes patas. Cf. B. Galvão, \textunderscore Vocab.\textunderscore 
\section{Pantopolista}
\begin{itemize}
\item {Grp. gram.:adj.}
\end{itemize}
\begin{itemize}
\item {Utilização:Neol.}
\end{itemize}
\begin{itemize}
\item {Proveniência:(Do gr. \textunderscore pantos\textunderscore  + \textunderscore polis\textunderscore )}
\end{itemize}
Relativo a todas as cidades ou a todas as terras; cosmopolita. Cf. Júl. Ribeiro, \textunderscore Carne\textunderscore .
\section{Pantóptero}
\begin{itemize}
\item {Grp. gram.:adj.}
\end{itemize}
\begin{itemize}
\item {Utilização:Zool.}
\end{itemize}
\begin{itemize}
\item {Proveniência:(Do gr. \textunderscore pantos\textunderscore  + \textunderscore pteron\textunderscore )}
\end{itemize}
Diz-se dos peixes, que têm todas as barbatanas, á excepção das ventraes.
\section{Pantouco}
\begin{itemize}
\item {Grp. gram.:adj.}
\end{itemize}
\begin{itemize}
\item {Utilização:Açor}
\end{itemize}
Pateta, idiota.
\section{Pantoscópio}
\begin{itemize}
\item {Grp. gram.:m.}
\end{itemize}
\begin{itemize}
\item {Proveniência:(Do gr. \textunderscore pan\textunderscore , \textunderscore pantos\textunderscore  + \textunderscore skopein\textunderscore )}
\end{itemize}
Objectiva photográphica, especial.
\section{Pantrigueira}
\begin{itemize}
\item {Grp. gram.:f.}
\end{itemize}
\begin{itemize}
\item {Utilização:Prov.}
\end{itemize}
\begin{itemize}
\item {Utilização:beir.}
\end{itemize}
\begin{itemize}
\item {Proveniência:(De \textunderscore pão\textunderscore  + \textunderscore trigo\textunderscore )}
\end{itemize}
Vendedora ambulante de pão de trigo.
\section{Pantufa}
\begin{itemize}
\item {Grp. gram.:f.}
\end{itemize}
\begin{itemize}
\item {Utilização:Burl.}
\end{itemize}
\begin{itemize}
\item {Utilização:Ant.}
\end{itemize}
\begin{itemize}
\item {Utilização:Pop.}
\end{itemize}
O mesmo que \textunderscore pantufo\textunderscore .
Pandorga.
Mulher mal vestida ou com fatos muito largos.
Mulhér grosseira, mas muito enfeitada.
\section{Pantufo}
\begin{itemize}
\item {Grp. gram.:m.}
\end{itemize}
\begin{itemize}
\item {Utilização:Gír.}
\end{itemize}
\begin{itemize}
\item {Proveniência:(Do fr. \textunderscore pantoufle\textunderscore )}
\end{itemize}
Sapato, feito de estôfo encorpado, para agasalho.
Homem gordo ou barrigudo.
\section{Panturra}
\begin{itemize}
\item {Grp. gram.:f.}
\end{itemize}
\begin{itemize}
\item {Utilização:Chul.}
\end{itemize}
Grande barriga.
Prosápia, vaidade.
(Cp. lat. \textunderscore pantex\textunderscore )
\section{Panturrilha}
\begin{itemize}
\item {Grp. gram.:f.}
\end{itemize}
\begin{itemize}
\item {Utilização:Pop.}
\end{itemize}
\begin{itemize}
\item {Utilização:Fig.}
\end{itemize}
Barriga da perna.
Chumaço, posto por baixo das meias, para dar ás pernas a apparência de gordas.
(Cast. \textunderscore pantorrilla\textunderscore , Cp. \textunderscore panturra\textunderscore )
\section{Panzeiro}
\begin{itemize}
\item {Grp. gram.:adj.}
\end{itemize}
\begin{itemize}
\item {Utilização:Prov.}
\end{itemize}
Que gosta muito de pão.
\section{Panzoótico}
\begin{itemize}
\item {Grp. gram.:adj.}
\end{itemize}
\begin{itemize}
\item {Proveniência:(Do gr. \textunderscore pan\textunderscore  + \textunderscore zoon\textunderscore )}
\end{itemize}
Diz-se da doença que, numa região, ataca os animaes da mesma espécie. Cf. Macedo Pinto, \textunderscore Comp. de Veter.\textunderscore , I, 36.
\section{Pão}
\begin{itemize}
\item {Grp. gram.:m.}
\end{itemize}
\begin{itemize}
\item {Utilização:Ext.}
\end{itemize}
\begin{itemize}
\item {Grp. gram.:Loc. adv.}
\end{itemize}
\begin{itemize}
\item {Grp. gram.:Loc.}
\end{itemize}
\begin{itemize}
\item {Utilização:fig.}
\end{itemize}
\begin{itemize}
\item {Grp. gram.:Loc.}
\end{itemize}
\begin{itemize}
\item {Utilização:fig.}
\end{itemize}
\begin{itemize}
\item {Proveniência:(Do lat. \textunderscore panis\textunderscore )}
\end{itemize}
Substância alimentícia, feita de farinha amassada e cozida.
A planta do trigo:«\textunderscore o pão já sazonado.\textunderscore »M. Bernárdez, \textunderscore N. Floresta\textunderscore , I, 70.
Grão de cereaes: \textunderscore moêr o pão\textunderscore .
Sustento: \textunderscore ganhar o pão para os filhos\textunderscore .
Sustento de cada dia.
Meios de vida.
Hóstia consagrada.
Variedade de pêra ordinária.
\textunderscore Pão, pão, queijo, queijo\textunderscore , com clareza, com franqueza, sem rodeios.
\textunderscore A pão e laranja\textunderscore , com poucos ou nenhuns recursos; na última extremidade.
\textunderscore Pão do espirito\textunderscore , a instrucção.
\section{Paô}
\begin{itemize}
\item {Grp. gram.:m.}
\end{itemize}
\begin{itemize}
\item {Utilização:Bras}
\end{itemize}
Ave, do tamanho de pomba, negra, mas com o peito vermelho.
\section{Pão-de-bugio}
\begin{itemize}
\item {Grp. gram.:m.}
\end{itemize}
Antiga designação portuguesa da adansónia.
\section{Pão-de-leite}
\begin{itemize}
\item {Grp. gram.:m.}
\end{itemize}
Planta primulácea, (\textunderscore primula acaulis\textunderscore , Lin.).
\section{Pão-de-ló}
\begin{itemize}
\item {Grp. gram.:m.}
\end{itemize}
Variedade de bolo leve.
\section{Pão-de-lózinho}
\begin{itemize}
\item {Grp. gram.:m.}
\end{itemize}
Pequeno pão de ló:«\textunderscore ...treze pão-de-lózinhos em lembrança.\textunderscore »Filinto.
\section{Pão-de-san-joão}
\begin{itemize}
\item {Grp. gram.:m.}
\end{itemize}
\begin{itemize}
\item {Utilização:Bras}
\end{itemize}
O mesmo que \textunderscore alfarrobeira\textunderscore .
\section{Pão-do-chile}
\begin{itemize}
\item {Grp. gram.:m.}
\end{itemize}
\begin{itemize}
\item {Utilização:Bras}
\end{itemize}
Espécie de mandioca.
\section{Pão-e-queijo}
\begin{itemize}
\item {Grp. gram.:m.}
\end{itemize}
Designação pop. da planta \textunderscore prímula\textunderscore  ou \textunderscore primavera\textunderscore .
\section{Pão-porcino}
\begin{itemize}
\item {Grp. gram.:m.}
\end{itemize}
\begin{itemize}
\item {Utilização:Bot.}
\end{itemize}
O mesmo que \textunderscore cyclame\textunderscore .
\section{Pão-por-Deus}
\begin{itemize}
\item {Grp. gram.:m.}
\end{itemize}
\begin{itemize}
\item {Utilização:Açor}
\end{itemize}
O mesmo que \textunderscore santórum\textunderscore .
\section{Pão-pôsto}
\begin{itemize}
\item {Grp. gram.:m.}
\end{itemize}
Gênero de plantas compostas, (\textunderscore anacyclus valentinus\textunderscore , Lin.).
\section{Pãozinho}
\begin{itemize}
\item {Grp. gram.:m.  e  adj.}
\end{itemize}
\begin{itemize}
\item {Utilização:Burl.}
\end{itemize}
\begin{itemize}
\item {Proveniência:(De \textunderscore pão\textunderscore )}
\end{itemize}
Indivíduo presumido e piegas, que se presta a sêr desfrutado.
Ridículo.
Passarinho canoro, de plumagem variegada.
\section{Papa}
\begin{itemize}
\item {Grp. gram.:m.}
\end{itemize}
\begin{itemize}
\item {Proveniência:(Lat. \textunderscore papa\textunderscore , pai)}
\end{itemize}
O chefe da Igreja cathólica; Padre-santo.
\section{Papa}
\begin{itemize}
\item {Grp. gram.:f.}
\end{itemize}
\begin{itemize}
\item {Utilização:Infant.}
\end{itemize}
\begin{itemize}
\item {Grp. gram.:Pl.}
\end{itemize}
\begin{itemize}
\item {Proveniência:(Lat. \textunderscore papa\textunderscore )}
\end{itemize}
Qualquer alimento.
Farinha, cozida em água ou leite.
Substância cozida, de apparência pouco consistente.
Qualquer substância molle, com aspecto de papas: \textunderscore papas de linhaça\textunderscore .
\section{Papa}
\begin{itemize}
\item {Grp. gram.:f.}
\end{itemize}
Árvore de Moçambique.
\section{Papa}
\begin{itemize}
\item {Grp. gram.:f.}
\end{itemize}
\textunderscore Cobertor de papa\textunderscore , cobertor pesado e felpudo, de lan.
\section{Papá}
\begin{itemize}
\item {Grp. gram.:m.}
\end{itemize}
\begin{itemize}
\item {Utilização:Infant.}
\end{itemize}
\begin{itemize}
\item {Proveniência:(Lat. \textunderscore papa\textunderscore )}
\end{itemize}
Pai.
\section{Papa-abelhas}
\begin{itemize}
\item {Grp. gram.:m.}
\end{itemize}
O mesmo que \textunderscore abelheiro\textunderscore .
O mesmo que \textunderscore megengra\textunderscore .
\section{Papa-açorda}
\begin{itemize}
\item {Grp. gram.:m.  e  f.}
\end{itemize}
\begin{itemize}
\item {Utilização:Fam.}
\end{itemize}
Pessôa indolente, mollangueira.
\section{Papa-amoras}
\begin{itemize}
\item {Grp. gram.:m.}
\end{itemize}
\begin{itemize}
\item {Utilização:Prov.}
\end{itemize}
\begin{itemize}
\item {Utilização:dur.}
\end{itemize}
Espécie de tutinegra, (\textunderscore curruca cinerea\textunderscore , Lin.).
\section{Papa-arroz}
\begin{itemize}
\item {Grp. gram.:m.}
\end{itemize}
\begin{itemize}
\item {Utilização:Bras}
\end{itemize}
Pássaro pequeno e negro.
\section{Papa-capim}
\begin{itemize}
\item {Grp. gram.:m.}
\end{itemize}
\begin{itemize}
\item {Utilização:Bras. da Baía}
\end{itemize}
Passarinho, o mesmo que \textunderscore colleira\textunderscore .
\section{Papa-ceia}
\begin{itemize}
\item {Grp. gram.:m.}
\end{itemize}
\begin{itemize}
\item {Utilização:Bras. do N}
\end{itemize}
\begin{itemize}
\item {Utilização:Pop.}
\end{itemize}
A estrêlla da tarde.
\section{Papada}
\begin{itemize}
\item {Grp. gram.:f.}
\end{itemize}
O mesmo que \textunderscore papeira\textunderscore .
\section{Papado}
\begin{itemize}
\item {Grp. gram.:m.}
\end{itemize}
Dignidade do Papa.
Tempo, durante o qual se exerce a dignidade de um Papa.
\section{Papaeira}
\begin{itemize}
\item {Grp. gram.:f.}
\end{itemize}
O mesmo que \textunderscore papaia\textunderscore . Cf. Dalgado, \textunderscore Flora\textunderscore , 80.
\section{Pàpafigo}
\begin{itemize}
\item {Grp. gram.:m.}
\end{itemize}
\begin{itemize}
\item {Utilização:Náut.}
\end{itemize}
Formoso pássaro, (\textunderscore oriolus galbula\textunderscore , Lin.).
Cada uma das velas mais baixas de um navio.
\textunderscore Ir papafigo\textunderscore , navegar, com os papafigos meio desenrolados.
\section{Pàpafina}
\begin{itemize}
\item {Grp. gram.:m.}
\end{itemize}
\begin{itemize}
\item {Grp. gram.:Adj.}
\end{itemize}
\begin{itemize}
\item {Utilização:Fam.}
\end{itemize}
\begin{itemize}
\item {Utilização:Fig.}
\end{itemize}
\begin{itemize}
\item {Proveniência:(De \textunderscore papa\textunderscore ^2 + \textunderscore fino\textunderscore )}
\end{itemize}
Individuo ridiculo, pãozinho.
Saboroso.
Magnifico.
\section{Pàpaformigas}
\begin{itemize}
\item {Grp. gram.:m.}
\end{itemize}
\begin{itemize}
\item {Proveniência:(De \textunderscore papar\textunderscore  + \textunderscore formiga\textunderscore )}
\end{itemize}
Família de mammíferos, que se nutrem principalmente de formigas.
Gênero de pássaros, que comem principalmente formigas.
Espécie de tordo americano, o mesmo que \textunderscore piadeira\textunderscore , o qual imita o gorgeio das outras aves.
\section{Papagaia}
\begin{itemize}
\item {Grp. gram.:f.}
\end{itemize}
Fêmea do papagaio.
\section{Papagaial}
\begin{itemize}
\item {Grp. gram.:adj.}
\end{itemize}
\begin{itemize}
\item {Utilização:Fig.}
\end{itemize}
Relativo a papagaio.
Próprio de papagaio.
Inconsciente, incoherente. Cf. Camillo \textunderscore Cancion. Al.\textunderscore , 436.
\section{Papagaio}
\begin{itemize}
\item {Grp. gram.:m.}
\end{itemize}
\begin{itemize}
\item {Utilização:Fig.}
\end{itemize}
\begin{itemize}
\item {Utilização:Bras}
\end{itemize}
\begin{itemize}
\item {Proveniência:(Do ár. \textunderscore babagá\textunderscore ?)}
\end{itemize}
Ave trepadora, de bico grosso e recurvo, a qual imita facilmente a voz humana.
Pessôa, que repete inconscientemente o que lhe dizem ou o que lê.
Pedaço de papel, mais ou menos triangular ou oval, que se estende sôbre um aro ou sôbre uma cruzeta leve, e que, preso por um fio que se não larga, se deixa voejar á feição do vento.
Peça de ferro, com que se conserva horizontalmente a cana do leme.
Tabique ou grade, que divide a varanda do mesmo andar, indicando que êste tem mais de um inquilino.
Separação análoga entre sacadas do mesmo andar.
Espécie de cabide ou cantoneira, semelhante a gaiola de papagaio.
Espécie de cueiro triangular.
Planta balsamínea, também conhecida por \textunderscore melindre\textunderscore .
Peixe marítimo.
\section{Papagaio-do-mar}
\begin{itemize}
\item {Grp. gram.:m.}
\end{itemize}
Ave aquática, (\textunderscore fratercula arctica\textunderscore ).
\section{Papagaíto}
\begin{itemize}
\item {Grp. gram.:m.}
\end{itemize}
\begin{itemize}
\item {Utilização:Prov.}
\end{itemize}
\begin{itemize}
\item {Utilização:minh.}
\end{itemize}
Flôr e planta, vulgarmente conhecidas por \textunderscore esporas\textunderscore , (\textunderscore delphinium Aiacis\textunderscore ).
\section{Papagalho}
\begin{itemize}
\item {Grp. gram.:m.}
\end{itemize}
Vento forte, que sopra nas costas do México.
\section{Páfia}
\begin{itemize}
\item {Grp. gram.:f.}
\end{itemize}
Molusco acéfalo.
\section{Papagarro}
\begin{itemize}
\item {Grp. gram.:m.}
\end{itemize}
\begin{itemize}
\item {Utilização:Mad}
\end{itemize}
Ave, também conhecida por \textunderscore patagarro\textunderscore .
\section{Papa-gente}
\begin{itemize}
\item {Grp. gram.:m.  e  f.}
\end{itemize}
\begin{itemize}
\item {Utilização:Pop.}
\end{itemize}
Pessôa anthropóphaga.
Fantasma, papão.
Pessôa irritada.
\section{Papaguçar}
\begin{itemize}
\item {Grp. gram.:v. t.  e  i.}
\end{itemize}
O mesmo que \textunderscore paparicar\textunderscore .
\section{Papagueador}
\begin{itemize}
\item {Grp. gram.:m.  e  adj.}
\end{itemize}
Aquelle que papagueia.
\section{Papagueamento}
\begin{itemize}
\item {Grp. gram.:m.}
\end{itemize}
Acto ou effeito de papaguear.
\section{Papaguear}
\begin{itemize}
\item {Grp. gram.:v. t.}
\end{itemize}
\begin{itemize}
\item {Grp. gram.:V. i.}
\end{itemize}
\begin{itemize}
\item {Proveniência:(De \textunderscore papagaio\textunderscore )}
\end{itemize}
Repetir inconscientemente, como um papagaio.
Dizer sem nexo.
Falar inconscientemente; tagarelar.
\section{Papaguela}
\begin{itemize}
\item {Grp. gram.:f.}
\end{itemize}
Arbusto myrtáceo do Brasil.
\section{Papa-hóstias}
\begin{itemize}
\item {Grp. gram.:m.  e  f.}
\end{itemize}
\begin{itemize}
\item {Utilização:Pop.}
\end{itemize}
\begin{itemize}
\item {Utilização:Burl.}
\end{itemize}
Pessôa muito beata, que ouve muitas Missas, que communga muitas vezes; papa-santos.
O mesmo que \textunderscore padre\textunderscore .
\section{Papai}
\begin{itemize}
\item {Grp. gram.:m.}
\end{itemize}
\begin{itemize}
\item {Utilização:Bras}
\end{itemize}
O mesmo que \textunderscore papá\textunderscore .
\section{Papaia}
\begin{itemize}
\item {Grp. gram.:f.}
\end{itemize}
O mesmo que \textunderscore mamoeiro\textunderscore .
\section{Papaíáceas}
\begin{itemize}
\item {Grp. gram.:f. pl.}
\end{itemize}
\begin{itemize}
\item {Proveniência:(De \textunderscore papaia\textunderscore )}
\end{itemize}
Família de plantas, que tem por typo o mamoeiro.
\section{Papaieira}
\begin{itemize}
\item {Grp. gram.:f.}
\end{itemize}
\begin{itemize}
\item {Utilização:T. de Cabo-Verde}
\end{itemize}
O mesmo que \textunderscore papaia\textunderscore .
\section{Papaína}
\begin{itemize}
\item {Grp. gram.:f.}
\end{itemize}
\begin{itemize}
\item {Proveniência:(De \textunderscore papaia\textunderscore )}
\end{itemize}
Producto pharmacêutico, extrahido da \textunderscore carica-papaya\textunderscore , e, geralmente, com as mesmas applicações que a pancreatina.
\section{Papa-jantares}
\begin{itemize}
\item {Grp. gram.:m.  e  f.}
\end{itemize}
Pessôa, que tem por hábito comer em casas alheias ou viver á custa de outrem.
Parasito.
\section{Papal}
\begin{itemize}
\item {Grp. gram.:adj.}
\end{itemize}
Relativo ao Papa: \textunderscore autoridade papal\textunderscore .
\section{Papa-léguas}
\begin{itemize}
\item {Grp. gram.:m.  e  f.}
\end{itemize}
Pessôa, que anda muito.
\section{Papalhaz}
\begin{itemize}
\item {Grp. gram.:m.}
\end{itemize}
\begin{itemize}
\item {Utilização:Prov.}
\end{itemize}
\begin{itemize}
\item {Utilização:dur.}
\end{itemize}
O mesmo que \textunderscore codorniz\textunderscore .
\section{Papalino}
\begin{itemize}
\item {Grp. gram.:adj.}
\end{itemize}
O mesmo que \textunderscore papal\textunderscore ; próprio do Papa.
\section{Papalva}
\begin{itemize}
\item {Grp. gram.:f.}
\end{itemize}
Mulher simplória.
O mesmo que \textunderscore toirão\textunderscore .
(Fem. de \textunderscore papalvo\textunderscore )
\section{Papalvice}
\begin{itemize}
\item {Grp. gram.:f.}
\end{itemize}
Qualidade, acto ou dito de papalvo; os papalvos. Cf. Filinto, XIII, 103.
\section{Papalvo}
\begin{itemize}
\item {Grp. gram.:m.}
\end{itemize}
\begin{itemize}
\item {Utilização:Pop.}
\end{itemize}
Indivíduo simplório, lorpa; pateta.
\section{Papalvro}
\begin{itemize}
\item {Grp. gram.:m.}
\end{itemize}
\begin{itemize}
\item {Utilização:Prov.}
\end{itemize}
\begin{itemize}
\item {Utilização:trasm.}
\end{itemize}
Fuínha pequena, de pelle fina.
\section{Papa-mel}
\begin{itemize}
\item {Grp. gram.:m.}
\end{itemize}
O mesmo que \textunderscore irara\textunderscore .
\section{Papa-missas}
\begin{itemize}
\item {Grp. gram.:m.  e  f.}
\end{itemize}
\begin{itemize}
\item {Utilização:Pop.}
\end{itemize}
Pessôa muito igrejeira, que ouve muitas Missas; papa-santos.
\section{Pàpamôscas}
\begin{itemize}
\item {Grp. gram.:m.}
\end{itemize}
\begin{itemize}
\item {Grp. gram.:M.  e  f.}
\end{itemize}
\begin{itemize}
\item {Utilização:Fig.}
\end{itemize}
\begin{itemize}
\item {Proveniência:(De \textunderscore papar\textunderscore  + \textunderscore môsca\textunderscore )}
\end{itemize}
Pássaro dentirostro, taralhão.
Pequeno reptil, que se sustenta de môscas.
Aranha, que se nutre de môscas.
Lorpa, pessôa simplória; papa-açorda.
\section{Papanazes}
\begin{itemize}
\item {Grp. gram.:m. pl.}
\end{itemize}
Tribo de aborígenes, que habitou entre Pôrto-Seguro e Espírito-Santo, no Brasil.
\section{Papança}
\begin{itemize}
\item {Grp. gram.:f.}
\end{itemize}
\begin{itemize}
\item {Utilização:Fam.}
\end{itemize}
\begin{itemize}
\item {Proveniência:(De \textunderscore papar\textunderscore )}
\end{itemize}
Aquillo que se come; os comestíveis.
Comezana.
\section{Papa-novenas}
\begin{itemize}
\item {Grp. gram.:f.}
\end{itemize}
Beata fingida. Cf. Júl. Dinís, \textunderscore Pupillas\textunderscore , 249.
\section{Papão}
\begin{itemize}
\item {Grp. gram.:m.}
\end{itemize}
\begin{itemize}
\item {Proveniência:(De \textunderscore papar\textunderscore )}
\end{itemize}
Monstro imaginário, com que se mete medo ás crianças.
Fantasma.
\section{Papa-ovo}
\begin{itemize}
\item {Grp. gram.:f.}
\end{itemize}
Cobra venenosa do Brasil.
\section{Papa-pintos}
\begin{itemize}
\item {Grp. gram.:m.}
\end{itemize}
\begin{itemize}
\item {Utilização:Prov.}
\end{itemize}
\begin{itemize}
\item {Utilização:minh.}
\end{itemize}
\begin{itemize}
\item {Utilização:Bras}
\end{itemize}
\begin{itemize}
\item {Proveniência:(De \textunderscore papar\textunderscore  + \textunderscore pinto\textunderscore )}
\end{itemize}
O mesmo que \textunderscore milhafre\textunderscore .
Espécie de cobra grande e parda.
\section{Papar}
\begin{itemize}
\item {Grp. gram.:v. t.  e  i.}
\end{itemize}
\begin{itemize}
\item {Utilização:Infant.}
\end{itemize}
\begin{itemize}
\item {Utilização:Pop.}
\end{itemize}
\begin{itemize}
\item {Proveniência:(De \textunderscore papa\textunderscore ^2)}
\end{itemize}
Comer.
Ganhar.
Extorquir.
\section{Papa-ratos}
\begin{itemize}
\item {Grp. gram.:m.}
\end{itemize}
\begin{itemize}
\item {Utilização:Pop.}
\end{itemize}
Ave pernalta, (\textunderscore buphus comatus\textunderscore ).
Gatafunhos.
\section{Papari}
\begin{itemize}
\item {Grp. gram.:m.}
\end{itemize}
Planta leguminosa de Dio.
\section{Paparicar}
\begin{itemize}
\item {Grp. gram.:v. t.}
\end{itemize}
\begin{itemize}
\item {Grp. gram.:V. i.}
\end{itemize}
\begin{itemize}
\item {Proveniência:(De \textunderscore paparico\textunderscore )}
\end{itemize}
Comer aos poucos, debicar; apaparicar.
Comer pouco ou aos poucos.
\section{Paparicho}
\begin{itemize}
\item {Grp. gram.:m.}
\end{itemize}
O mesmo que \textunderscore paparico\textunderscore .
\section{Paparico}
\begin{itemize}
\item {Grp. gram.:m.}
\end{itemize}
\begin{itemize}
\item {Proveniência:(De \textunderscore papar\textunderscore )}
\end{itemize}
Mimos ou afagos, com que se tratam doentes ou pessôas queridas.
Pequenas e delicadas iguarias; gulodices.
\section{Paparicos}
\begin{itemize}
\item {Grp. gram.:m. pl.}
\end{itemize}
\begin{itemize}
\item {Proveniência:(De \textunderscore papar\textunderscore )}
\end{itemize}
Mimos ou afagos, com que se tratam doentes ou pessôas queridas.
Pequenas e delicadas iguarias; gulodices.
\section{Paparoca}
\begin{itemize}
\item {Grp. gram.:f.}
\end{itemize}
\begin{itemize}
\item {Utilização:Pop.}
\end{itemize}
\begin{itemize}
\item {Proveniência:(De \textunderscore papar\textunderscore )}
\end{itemize}
Comida, refeição.
\section{Paparota}
\begin{itemize}
\item {Grp. gram.:f.  e  adj.}
\end{itemize}
\begin{itemize}
\item {Utilização:Prov.}
\end{itemize}
\begin{itemize}
\item {Utilização:trasm.}
\end{itemize}
Mulhér apalermada, mansarrona, uma bôca-aberta.
\section{Paparote}
\begin{itemize}
\item {Grp. gram.:m.}
\end{itemize}
\begin{itemize}
\item {Utilização:Prov.}
\end{itemize}
\begin{itemize}
\item {Proveniência:(De \textunderscore papar\textunderscore )}
\end{itemize}
Caldo de castanhas piladas. Cf. \textunderscore Eufrosina\textunderscore , 288.
\section{Paparraça}
\begin{itemize}
\item {Grp. gram.:f.}
\end{itemize}
\begin{itemize}
\item {Utilização:Açor}
\end{itemize}
O mesmo que \textunderscore galinhola\textunderscore .
\section{Paparraz}
\begin{itemize}
\item {Grp. gram.:m.}
\end{itemize}
\begin{itemize}
\item {Proveniência:(Do ár. \textunderscore habberrás\textunderscore ?)}
\end{itemize}
Planta ranunculácea, (\textunderscore delphinum staphisagria\textunderscore ).
\section{Paparreta}
\begin{itemize}
\item {fónica:rê}
\end{itemize}
\begin{itemize}
\item {Grp. gram.:m.}
\end{itemize}
\begin{itemize}
\item {Grp. gram.:Adj.}
\end{itemize}
Homem pretensioso, mas sem mérito. Cf. Camillo, \textunderscore Brasileira\textunderscore , 169.
Ridiculamente pretensioso:«\textunderscore vaidade paparreta\textunderscore ». Camillo, \textunderscore Brasileira\textunderscore , 139.
\section{Pàparriba}
\begin{itemize}
\item {Grp. gram.:adv.}
\end{itemize}
\begin{itemize}
\item {Utilização:Ant.}
\end{itemize}
\begin{itemize}
\item {Proveniência:(De \textunderscore papo\textunderscore  + \textunderscore arriba\textunderscore )}
\end{itemize}
Na ociosidade; barriga para o ar.
\section{Paparróia}
\begin{itemize}
\item {Grp. gram.:f.}
\end{itemize}
O mesmo que \textunderscore paparraça\textunderscore .
\section{Paparrotada}
\begin{itemize}
\item {Grp. gram.:f.}
\end{itemize}
\begin{itemize}
\item {Proveniência:(De \textunderscore paparrotão\textunderscore )}
\end{itemize}
Acto ou dito de paparrotão.
Comida de porcos, lavagem.
Comida mal feita.
\section{Paparrotagem}
\begin{itemize}
\item {Grp. gram.:f.}
\end{itemize}
O mesmo que \textunderscore paparrotada\textunderscore .
\section{Paparrotão}
\begin{itemize}
\item {Grp. gram.:m.}
\end{itemize}
\begin{itemize}
\item {Utilização:Fam.}
\end{itemize}
\begin{itemize}
\item {Grp. gram.:Adj.}
\end{itemize}
\begin{itemize}
\item {Proveniência:(De \textunderscore papa\textunderscore ^2 + \textunderscore arrotar\textunderscore )}
\end{itemize}
Impostor, parlapatão.
Jactancioso.
Vaidoso, mas sem mérito.
\section{Paparrotear}
\begin{itemize}
\item {Grp. gram.:v. t.  e  i.}
\end{itemize}
\begin{itemize}
\item {Utilização:Neol.}
\end{itemize}
\begin{itemize}
\item {Proveniência:(De \textunderscore paparrotão\textunderscore )}
\end{itemize}
Alardear, impostura; dizer com paparrotice.
\section{Paparrotice}
\begin{itemize}
\item {Grp. gram.:f.}
\end{itemize}
\begin{itemize}
\item {Utilização:Fam.}
\end{itemize}
\begin{itemize}
\item {Proveniência:(De \textunderscore paparrotão\textunderscore )}
\end{itemize}
Dito ou acto de paparrotão.
Bazófia, impostura.
\section{Paparuca}
\begin{itemize}
\item {Grp. gram.:f.}
\end{itemize}
O mesmo que \textunderscore paparoca\textunderscore .
\section{Papa-santos}
\begin{itemize}
\item {Grp. gram.:m.  e  f.}
\end{itemize}
Pessôa muito devota, muito igrejeira. Cf. Filinto, V, 33.
\section{Papa-tabaco}
\begin{itemize}
\item {Grp. gram.:m.}
\end{itemize}
\begin{itemize}
\item {Utilização:Fam.}
\end{itemize}
Peixe, da fam. dos pércidas.
Aquelle que cheira muito rapé.
\section{Papa-terra}
\begin{itemize}
\item {Grp. gram.:m.}
\end{itemize}
\begin{itemize}
\item {Utilização:Bras}
\end{itemize}
Peixe marítimo.
Nome de uma ave brasileira.
\section{Papa-vento}
\begin{itemize}
\item {Grp. gram.:m.}
\end{itemize}
O mesmo que \textunderscore camaleão\textunderscore ^1.
\section{Papaveráceas}
\begin{itemize}
\item {Grp. gram.:f. pl.}
\end{itemize}
\begin{itemize}
\item {Proveniência:(De \textunderscore papaveráceo\textunderscore )}
\end{itemize}
Família de plantas, que tem por typo a papoila.
\section{Papaveráceo}
\begin{itemize}
\item {Grp. gram.:adj.}
\end{itemize}
\begin{itemize}
\item {Proveniência:(Do lat. \textunderscore papaver\textunderscore )}
\end{itemize}
Relativo ou semelhante á papoila.
\section{Papaverina}
\begin{itemize}
\item {Grp. gram.:f.}
\end{itemize}
\begin{itemize}
\item {Proveniência:(Do lat. \textunderscore papaver\textunderscore )}
\end{itemize}
Um dos alcalóides do ópio, usado em Medicina como narcótico.
\section{Papazana}
\begin{itemize}
\item {Grp. gram.:f.}
\end{itemize}
\begin{itemize}
\item {Proveniência:(De \textunderscore papar\textunderscore )}
\end{itemize}
O mesmo que \textunderscore comezaina\textunderscore .
\section{Papazes}
\begin{itemize}
\item {Grp. gram.:m. pl.}
\end{itemize}
\begin{itemize}
\item {Utilização:Ant.}
\end{itemize}
Sacerdotes gregos. Cf. Pant. de Aveiro, \textunderscore Itiner.\textunderscore , 29, (2.^a ed.).
\section{Papear}
\begin{itemize}
\item {Grp. gram.:v. i.}
\end{itemize}
Falar muito; chilrear.
(Outra fórma de \textunderscore pipiar\textunderscore )
\section{Papear}
\begin{itemize}
\item {Grp. gram.:v. i.}
\end{itemize}
\begin{itemize}
\item {Utilização:Prov.}
\end{itemize}
\begin{itemize}
\item {Utilização:alg.}
\end{itemize}
\begin{itemize}
\item {Utilização:Ant.}
\end{itemize}
\begin{itemize}
\item {Proveniência:(T. onom.)}
\end{itemize}
Mover os beiços, como quem reza ou fala só para si.
Falar baixo, cochichar. Cf. G. Vicente, III, 93.
\section{Papeira}
\begin{itemize}
\item {Grp. gram.:f.}
\end{itemize}
\begin{itemize}
\item {Proveniência:(De \textunderscore papo\textunderscore )}
\end{itemize}
Bronchocele.
Inflammação da parótida.
Trasorelho.
Papo.
Arbusto borragíneo do Brasil.
\section{Papeiro}
\begin{itemize}
\item {Grp. gram.:m.}
\end{itemize}
\begin{itemize}
\item {Proveniência:(De \textunderscore papa\textunderscore ^2)}
\end{itemize}
Vaso, para cozer papas ou migas, ou para guisar batatas desfeitas.
\section{Papéis}
\begin{itemize}
\item {Grp. gram.:m. pl.}
\end{itemize}
(V.papeles)
\section{Papejar}
\begin{itemize}
\item {Grp. gram.:v. i.}
\end{itemize}
\begin{itemize}
\item {Proveniência:(De \textunderscore papo\textunderscore ? ou T. onom.?)}
\end{itemize}
O mesmo que \textunderscore latejar\textunderscore :«\textunderscore as artérias... papejavam muito grossas...\textunderscore »Camillo, \textunderscore Volcoens\textunderscore , 163.
\section{Papel}
\begin{itemize}
\item {Grp. gram.:m.}
\end{itemize}
\begin{itemize}
\item {Grp. gram.:Pl.}
\end{itemize}
\begin{itemize}
\item {Utilização:Fam.}
\end{itemize}
\begin{itemize}
\item {Proveniência:(Do b. lat. \textunderscore papillus\textunderscore )}
\end{itemize}
Tecido, que se fabricava com papyro, e em que os antigos escreviam.
Fôlha, fabricada de trapos e outras substâncias, para nella se escrever, imprimir, desenhar, etc.
Cada uma das partes de uma peça theatral, representada por um actor.
Acção, funcções: \textunderscore cumpre, o seu dever e está no seu papel\textunderscore .
Documento escrito.
Designação genérica de documentos, que certificam contratos, habilitações, profissão, etc., de alguém.
Periódicos.
\section{Papel}
\begin{itemize}
\item {Grp. gram.:f.}
\end{itemize}
Idioma africano, falado na Guiné portuguesa.
(Cp. \textunderscore papeles\textunderscore )
\section{Papelada}
\begin{itemize}
\item {Grp. gram.:f.}
\end{itemize}
\begin{itemize}
\item {Proveniência:(De \textunderscore papel\textunderscore )}
\end{itemize}
Grande porção de papéis; papéis em desordem.
Conjunto de documentos.
\section{Papelagem}
\begin{itemize}
\item {Grp. gram.:f.}
\end{itemize}
O mesmo que \textunderscore papelada\textunderscore .
\section{Papelão}
\begin{itemize}
\item {Grp. gram.:m.}
\end{itemize}
\begin{itemize}
\item {Utilização:Fig.}
\end{itemize}
\begin{itemize}
\item {Proveniência:(De \textunderscore papel\textunderscore )}
\end{itemize}
Papel encorpado e forte.
Parlapatão, paspalhão.
\section{Papelaria}
\begin{itemize}
\item {Grp. gram.:f.}
\end{itemize}
Casa ou loja, em que se vende papel e objectos de escriptório.
\section{Papeleira}
\begin{itemize}
\item {Grp. gram.:f.}
\end{itemize}
\begin{itemize}
\item {Proveniência:(De \textunderscore papel\textunderscore )}
\end{itemize}
Espécie de mesa com tampa inclinada, e com gavetas em que se guardam papéis.
\section{Papeleiro}
\begin{itemize}
\item {Grp. gram.:m.}
\end{itemize}
Aquelle que trabalha no fabríco do papel.
\section{Papelejo}
\begin{itemize}
\item {Grp. gram.:m.}
\end{itemize}
Papel sem importância; papelucho.
\section{Papeles}
\begin{itemize}
\item {Grp. gram.:m. pl.}
\end{itemize}
Tríbo da Guiné, também conhecida por \textunderscore papéis\textunderscore .
\section{Papeleta}
\begin{itemize}
\item {fónica:lê}
\end{itemize}
\begin{itemize}
\item {Grp. gram.:f.}
\end{itemize}
\begin{itemize}
\item {Utilização:T. de Lisbôa}
\end{itemize}
\begin{itemize}
\item {Utilização:Deprec.}
\end{itemize}
\begin{itemize}
\item {Proveniência:(De \textunderscore papel\textunderscore )}
\end{itemize}
Papel, que se fixa num lugar, para que uma classe de indivíduos ou todos o leiam.
Cartaz.
Documento comprovativo da identidade dos criados e criadas.
Jornal periódico.
\section{Papelete}
\begin{itemize}
\item {fónica:lê}
\end{itemize}
\begin{itemize}
\item {Grp. gram.:m.}
\end{itemize}
\begin{itemize}
\item {Utilização:Fam.}
\end{itemize}
O mesmo que \textunderscore papelinho\textunderscore .
\section{Papeliço}
\begin{itemize}
\item {Grp. gram.:m.}
\end{itemize}
Pequeno embrulho de papel.
\section{Papelico}
\begin{itemize}
\item {Grp. gram.:m.}
\end{itemize}
Papel de pouca importância Escrito de pouco valor:«\textunderscore outros papelicos meus, que por ahi andão impressos...\textunderscore »Filinto, VI, 111.
\section{Papelinho}
\begin{itemize}
\item {Grp. gram.:m.}
\end{itemize}
\begin{itemize}
\item {Grp. gram.:Pl.}
\end{itemize}
Papel pequeno.
Fragmentos de papel, que se atiram brincando, no Carnaval.
\section{Papelista}
\begin{itemize}
\item {Grp. gram.:m. ,  f.  e  adj.}
\end{itemize}
\begin{itemize}
\item {Proveniência:(De \textunderscore papel\textunderscore )}
\end{itemize}
Pessôa, que trata de papéis ou investiga documentos antigos.
\section{Papelístico}
\begin{itemize}
\item {Grp. gram.:adj.}
\end{itemize}
\begin{itemize}
\item {Utilização:Fam.}
\end{itemize}
Relativo a papelada confusa e pouco útil.
Próprio de papelista.
\section{Papelório}
\begin{itemize}
\item {Grp. gram.:m.}
\end{itemize}
Papel sem importância:«\textunderscore um papelório, que sempre na algibeira traz.\textunderscore »Filinto, XIX, 243.
\section{Papelota}
\begin{itemize}
\item {Grp. gram.:f.}
\end{itemize}
\begin{itemize}
\item {Utilização:Escol.}
\end{itemize}
Pedaço de papel dobrado ou amassado com saliva, formando uma espécie de pequena bola, que os estudantes menos attentos á prelecção costumam atirar uns aos outros, na aula.
(Cp. \textunderscore papelotes\textunderscore )
\section{Papelotes}
\begin{itemize}
\item {Grp. gram.:m. pl.}
\end{itemize}
\begin{itemize}
\item {Proveniência:(De \textunderscore papel\textunderscore )}
\end{itemize}
Pedaços de papel enrolado.
Fragmentos de papel, em que se enrola o cabello, para o encrespar.
\section{Papelucho}
\begin{itemize}
\item {Grp. gram.:m.}
\end{itemize}
\begin{itemize}
\item {Utilização:Pop.}
\end{itemize}
\begin{itemize}
\item {Utilização:Deprec.}
\end{itemize}
\begin{itemize}
\item {Proveniência:(De \textunderscore papel\textunderscore )}
\end{itemize}
Papel de pouca importância.
Papel de embrulhos.
Pedaço de papel.
Periódico.
\section{Papesa}
\begin{itemize}
\item {Grp. gram.:f.}
\end{itemize}
(V.papisa)
\section{Páphia}
\begin{itemize}
\item {Grp. gram.:f.}
\end{itemize}
Mollusco acéphalo.
\section{Papião}
\begin{itemize}
\item {Grp. gram.:m.}
\end{itemize}
Macaco da África, (\textunderscore simia sphinx\textunderscore ).
\section{Papícola}
\begin{itemize}
\item {Grp. gram.:m.}
\end{itemize}
\begin{itemize}
\item {Proveniência:(Do lat. \textunderscore papa\textunderscore  + \textunderscore colere\textunderscore )}
\end{itemize}
O mesmo que \textunderscore papista\textunderscore ^1.
\section{Papila}
\begin{itemize}
\item {Grp. gram.:f.}
\end{itemize}
\begin{itemize}
\item {Proveniência:(Lat. \textunderscore papilla\textunderscore )}
\end{itemize}
Pequena saliência cónica, na superficie da pele ou das mucosas, formada de ramificações nervosas ou vasculares.
Protuberância cónica, em diversos órgãos vegetaes.
Bico da mama.
Terminação inter-ocular do nervo óptico.
\section{Papilar}
\begin{itemize}
\item {Grp. gram.:adj.}
\end{itemize}
\begin{itemize}
\item {Utilização:Bot.}
\end{itemize}
\begin{itemize}
\item {Proveniência:(De \textunderscore papila\textunderscore )}
\end{itemize}
Diz-se das glândulas, que têm alguma semelhança com as papilas da língua e que, de ordinário, cobrem a superficie inferior das fôlhas das labiadas.
\section{Papilho}
\begin{itemize}
\item {Grp. gram.:m.}
\end{itemize}
\begin{itemize}
\item {Utilização:Bot.}
\end{itemize}
\begin{itemize}
\item {Proveniência:(Do lat. \textunderscore papilla\textunderscore )}
\end{itemize}
Appêndice do fruto e semente de várias plantas.
\section{Papilhoso}
\begin{itemize}
\item {Grp. gram.:adj.}
\end{itemize}
Que tem papilhos.
\section{Papílio}
\begin{itemize}
\item {Grp. gram.:m.}
\end{itemize}
\begin{itemize}
\item {Proveniência:(Lat. \textunderscore papilio\textunderscore )}
\end{itemize}
Lepidóptero com seis patas, próprias para a locomoção, em ambos os sexos, e tendo côncavo o bôrdo das asas inferiores.
\section{Papilionáceas}
\begin{itemize}
\item {Grp. gram.:f. pl.}
\end{itemize}
\begin{itemize}
\item {Proveniência:(De \textunderscore papilionáceo\textunderscore )}
\end{itemize}
Família de plantas leguminosas.
\section{Papilionáceo}
\begin{itemize}
\item {Grp. gram.:adj.}
\end{itemize}
\begin{itemize}
\item {Utilização:Bot.}
\end{itemize}
\begin{itemize}
\item {Proveniência:(Do lat. \textunderscore papilio\textunderscore )}
\end{itemize}
Relativo ou semelhante á borboleta.
Diz-se das corollas irregulares, compostas de cinco pétalas desiguaes, que pela sua disposição têm alguma semelhança com uma borboleta de asas abertas.
\section{Papilionídios}
\begin{itemize}
\item {Grp. gram.:m. pl.}
\end{itemize}
\begin{itemize}
\item {Utilização:Zool.}
\end{itemize}
\begin{itemize}
\item {Proveniência:(Do lat. \textunderscore papilio\textunderscore  + gr. \textunderscore eidos\textunderscore )}
\end{itemize}
Família de lepidópteros hexápodes.
\section{Papilla}
\begin{itemize}
\item {Grp. gram.:f.}
\end{itemize}
\begin{itemize}
\item {Proveniência:(Lat. \textunderscore papilla\textunderscore )}
\end{itemize}
Pequena saliência cónica, na superficie da pelle ou das mucosas, formada de ramificações nervosas ou vasculares.
Protuberância cónica, em diversos órgãos vegetaes.
Bico da mama.
Terminação inter-ocular do nervo óptico.
\section{Papillar}
\begin{itemize}
\item {Grp. gram.:adj.}
\end{itemize}
\begin{itemize}
\item {Utilização:Bot.}
\end{itemize}
\begin{itemize}
\item {Proveniência:(De \textunderscore papilla\textunderscore )}
\end{itemize}
Diz-se das glândulas, que têm alguma semelhança com as papillas da língua e que, de ordinário, cobrem a superficie inferior das fôlhas das labiadas.
\section{Papilloma}
\begin{itemize}
\item {Grp. gram.:m.}
\end{itemize}
\begin{itemize}
\item {Proveniência:(De \textunderscore papilla\textunderscore )}
\end{itemize}
Variedade de epithelioma, caracterizada pelo aumento do volume nas papillas de pelle ou da mucosa.
\section{Papilo}
\begin{itemize}
\item {Grp. gram.:m.}
\end{itemize}
\begin{itemize}
\item {Utilização:Ant.}
\end{itemize}
Papel de linho ou de trapos.
(B. lat. \textunderscore papilus\textunderscore )
\section{Papiloma}
\begin{itemize}
\item {Grp. gram.:m.}
\end{itemize}
\begin{itemize}
\item {Proveniência:(De \textunderscore papilla\textunderscore )}
\end{itemize}
Variedade de epithelioma, caracterizada pelo aumento do volume nas papillas de pelle ou da mucosa.
\section{Papinho-amarelo}
\begin{itemize}
\item {Grp. gram.:m.}
\end{itemize}
\begin{itemize}
\item {Utilização:Mad}
\end{itemize}
O mesmo que \textunderscore canário\textunderscore ^1.
\section{Papinho-encarnado}
\begin{itemize}
\item {Grp. gram.:m.}
\end{itemize}
Pássaro da Madeira, (\textunderscore fringilla cannabina\textunderscore , Lin.).
\section{Papiri}
\begin{itemize}
\item {Grp. gram.:m.}
\end{itemize}
\begin{itemize}
\item {Utilização:Bras. do N}
\end{itemize}
Abrigo, feito de fôlhas, para evitar a chuva.
\section{Papisa}
\begin{itemize}
\item {Grp. gram.:f.}
\end{itemize}
\begin{itemize}
\item {Proveniência:(De \textunderscore Papa\textunderscore ^1)}
\end{itemize}
Mulhér, que exerce as funcções de Papa, (alludindo-se á lenda, segundo a qual, uma mulhér chamada Joanna occupara o throno pontificio sob o nome de João VIII). Cf. Herculano, \textunderscore Opúsc.\textunderscore , III, 108.
\section{Papismo}
\begin{itemize}
\item {Grp. gram.:m.}
\end{itemize}
\begin{itemize}
\item {Proveniência:(De \textunderscore Papa\textunderscore ^1)}
\end{itemize}
Influência dos Papas.
Igreja cathólica, segundo a expressão dos Protestantes.
\section{Papista}
\begin{itemize}
\item {Grp. gram.:m. ,  f.  e  adj.}
\end{itemize}
\begin{itemize}
\item {Grp. gram.:Loc.}
\end{itemize}
\begin{itemize}
\item {Utilização:fig.}
\end{itemize}
\begin{itemize}
\item {Proveniência:(De \textunderscore Papa\textunderscore ^1)}
\end{itemize}
Pessôa, partidária da supremacia do Papa.
Nome, dado aos Cathólicos pelos Protestantes.
\textunderscore Sêr mais papista que o Papa\textunderscore , apaixonar-se por um assumpto, mais que o principal interessado nelle.
\section{Papista}
\begin{itemize}
\item {Grp. gram.:m.  e  f.}
\end{itemize}
\begin{itemize}
\item {Utilização:Bras. do N}
\end{itemize}
\begin{itemize}
\item {Grp. gram.:M.}
\end{itemize}
Pessôa, que tem o vício de comer terra.
Nome de um peixe marítimo.
\section{Papo}
\begin{itemize}
\item {Grp. gram.:m.}
\end{itemize}
\begin{itemize}
\item {Utilização:Pop.}
\end{itemize}
\begin{itemize}
\item {Utilização:Fam.}
\end{itemize}
\begin{itemize}
\item {Utilização:Fig.}
\end{itemize}
\begin{itemize}
\item {Utilização:Prov.}
\end{itemize}
\begin{itemize}
\item {Utilização:alent.}
\end{itemize}
\begin{itemize}
\item {Utilização:Ant.}
\end{itemize}
\begin{itemize}
\item {Grp. gram.:Loc.}
\end{itemize}
\begin{itemize}
\item {Utilização:fam.}
\end{itemize}
\begin{itemize}
\item {Proveniência:(De \textunderscore papar\textunderscore ?)}
\end{itemize}
Bôlsa membranosa, em que as aves juntam os alimentos depois de engulidos e antes de passarem á moéla.
Distensão dos músculos faciaes, formando bôlsa membranosa, em algumas espécies de macacos e roedores.
Papeira.
Parte de vestuário, que cobre o peito, formando geralmente tufos ou grandes pregas.
Espécie de bôlso ou tufo, formado por uma peça de vestuário mal talhada ou mal cosida.
Estômago.
Arrogância, soberba.
\textunderscore Fazer papo\textunderscore , fazer rôsto, resistir.
Causar vaidade. Cf. \textunderscore Eufrosina\textunderscore , 131.
\textunderscore Falar de papo\textunderscore , falar com ares de importância ou superioridade.
\section{Papocar}
\begin{itemize}
\item {Grp. gram.:v. t.  e  i.}
\end{itemize}
\begin{itemize}
\item {Utilização:Bras. do N}
\end{itemize}
O mesmo que \textunderscore pipocar\textunderscore .
\section{Papoco}
\begin{itemize}
\item {Grp. gram.:m.}
\end{itemize}
\begin{itemize}
\item {Utilização:Bras. do N}
\end{itemize}
O mesmo que \textunderscore pipoco\textunderscore .
\section{Papo-de-anjo}
\begin{itemize}
\item {Grp. gram.:m.}
\end{itemize}
Espécie de doce.
\section{Papo-de-peru}
\begin{itemize}
\item {Grp. gram.:m.}
\end{itemize}
\begin{itemize}
\item {Utilização:Bras}
\end{itemize}
Planta medicinal.
\section{Papoias}
\begin{itemize}
\item {Grp. gram.:f. pl.}
\end{itemize}
Peças de navio, em que se fixam as roldanas das adriças.
\section{Papoiço}
\begin{itemize}
\item {Grp. gram.:m.}
\end{itemize}
\begin{itemize}
\item {Utilização:T. da Bairrada}
\end{itemize}
\begin{itemize}
\item {Proveniência:(De \textunderscore papo\textunderscore )}
\end{itemize}
Inchação, tumor.
\section{Papoila}
\begin{itemize}
\item {Grp. gram.:f.}
\end{itemize}
\begin{itemize}
\item {Utilização:Prov.}
\end{itemize}
\begin{itemize}
\item {Utilização:alent.}
\end{itemize}
Gênero de plantas, que serve de typo ás papaveráceas, e de que se extrai o ópio.--Entre as suas mais conhecidas espécies, nota-se a \textunderscore papoila das praias\textunderscore , a \textunderscore papoila longa\textunderscore , a \textunderscore papoila ordinária\textunderscore , a \textunderscore papoila pelada\textunderscore , a \textunderscore papoila pontuda\textunderscore , etc. Cf. P. Coutinho, \textunderscore Flora de Port.\textunderscore , 243.A flôr dessa planta.
A flôr da esteva.
(Cast. \textunderscore ababol\textunderscore , \textunderscore amapola\textunderscore )
\section{Papilho}
\begin{itemize}
\item {Grp. gram.:m.}
\end{itemize}
Pequeno papo.
\section{Papiráceo}
\begin{itemize}
\item {Grp. gram.:adj.}
\end{itemize}
\begin{itemize}
\item {Proveniência:(De \textunderscore papiro\textunderscore )}
\end{itemize}
Semelhante ao papel.
\section{Papíreo}
\begin{itemize}
\item {Grp. gram.:adj.}
\end{itemize}
Relativo ao papiro.
\section{Papirífero}
\begin{itemize}
\item {Grp. gram.:adj.}
\end{itemize}
\begin{itemize}
\item {Proveniência:(Do lat. \textunderscore papyrus\textunderscore  + \textunderscore ferre\textunderscore )}
\end{itemize}
Diz-se das plantas, cuja casca serve ou póde servir para a fabricação do papel.
\section{Papiro}
\begin{itemize}
\item {Grp. gram.:m.}
\end{itemize}
\begin{itemize}
\item {Proveniência:(Lat. \textunderscore papyrus\textunderscore )}
\end{itemize}
Variedade de cana, cuja haste, formada de folhetos sobrepostos, servia antigamente, depois de certa preparação, para nela se escrever.
Fôlha para escrever, feita com papiro.
Manuscrito antigo, feito em papiro.
\section{Papirólitha}
\begin{itemize}
\item {Grp. gram.:f.}
\end{itemize}
\begin{itemize}
\item {Proveniência:(Do gr. \textunderscore papuros\textunderscore  + \textunderscore lithos\textunderscore )}
\end{itemize}
Massa de papel endurecida, que começa a empregar-se em construcções, na América.
\section{Papo}
\begin{itemize}
\item {Grp. gram.:m.}
\end{itemize}
\begin{itemize}
\item {Utilização:Bot.}
\end{itemize}
\begin{itemize}
\item {Proveniência:(Do gr. \textunderscore pappos\textunderscore )}
\end{itemize}
Excrescência em fórma de penacho, sobreposta a certas sementes, depois de passada a florescência.
\section{Papóforo}
\begin{itemize}
\item {Grp. gram.:m.}
\end{itemize}
\begin{itemize}
\item {Proveniência:(Do gr. \textunderscore pappos\textunderscore  + \textunderscore phoros\textunderscore )}
\end{itemize}
Gênero de plantas gramíneas.
\section{Papoilas}
\begin{itemize}
\item {Grp. gram.:f. pl.}
\end{itemize}
(Corr. de \textunderscore papoias\textunderscore )
\section{Papola}
\begin{itemize}
\item {Grp. gram.:m.}
\end{itemize}
\begin{itemize}
\item {Utilização:Prov.}
\end{itemize}
\begin{itemize}
\item {Utilização:trasm.}
\end{itemize}
Homem bruto ou imbecil.
\section{Papolino}
\begin{itemize}
\item {Grp. gram.:adj.}
\end{itemize}
\begin{itemize}
\item {Utilização:Prov.}
\end{itemize}
\begin{itemize}
\item {Utilização:trasm.}
\end{itemize}
\begin{itemize}
\item {Proveniência:(De \textunderscore papo\textunderscore ?)}
\end{itemize}
Profundo, figadal, (falando-se de ódio).
\section{Papo-roxo}
\begin{itemize}
\item {Grp. gram.:m.}
\end{itemize}
Pássaro da Madeira, (\textunderscore erithacus rubecula\textunderscore , Lin.).
\section{Papote}
\begin{itemize}
\item {Grp. gram.:m.}
\end{itemize}
\begin{itemize}
\item {Utilização:Prov.}
\end{itemize}
\begin{itemize}
\item {Utilização:trasm.}
\end{itemize}
Variedade de picanço, o mesmo que \textunderscore pica-porco\textunderscore .
\section{Papoula}
\begin{itemize}
\item {Grp. gram.:f.}
\end{itemize}
\begin{itemize}
\item {Utilização:Prov.}
\end{itemize}
\begin{itemize}
\item {Utilização:alent.}
\end{itemize}
Gênero de plantas, que serve de typo ás papaveráceas, e de que se extrai o ópio.--Entre as suas mais conhecidas espécies, nota-se a \textunderscore papoula das praias\textunderscore , a \textunderscore papoula longa\textunderscore , a \textunderscore papoula ordinária\textunderscore , a \textunderscore papoula pelada\textunderscore , a \textunderscore papoula pontuda\textunderscore , etc. Cf. P. Coutinho, \textunderscore Flora de Port.\textunderscore , 243.
A flôr dessa planta.
A flôr da esteva.
(Cast. \textunderscore ababol\textunderscore , \textunderscore amapola\textunderscore )
\section{Pappilho}
\begin{itemize}
\item {Grp. gram.:m.}
\end{itemize}
Pequeno pappo.
\section{Pappo}
\begin{itemize}
\item {Grp. gram.:m.}
\end{itemize}
\begin{itemize}
\item {Utilização:Bot.}
\end{itemize}
\begin{itemize}
\item {Proveniência:(Do gr. \textunderscore pappos\textunderscore )}
\end{itemize}
Excrescência em fórma de pennacho, sobreposta a certas sementes, depois de passada a florescência.
\section{Pappóphoro}
\begin{itemize}
\item {Grp. gram.:m.}
\end{itemize}
\begin{itemize}
\item {Proveniência:(Do gr. \textunderscore pappos\textunderscore  + \textunderscore phoros\textunderscore )}
\end{itemize}
Gênero de plantas gramíneas.
\section{Papuas}
\begin{itemize}
\item {Grp. gram.:m. pl.}
\end{itemize}
Povos selvagens da Oceânia. Cf. Couto, \textunderscore Décad.\textunderscore , LV, l. VI, c. 5.
\section{Papuco}
\begin{itemize}
\item {Grp. gram.:m.}
\end{itemize}
\begin{itemize}
\item {Utilização:Bras}
\end{itemize}
O mesmo que \textunderscore batuera\textunderscore .
\section{Papudo}
\begin{itemize}
\item {Grp. gram.:adj.}
\end{itemize}
\begin{itemize}
\item {Utilização:Fig.}
\end{itemize}
Que tem grande papo.
Proeminente; arqueado.
\section{Papujar}
\begin{itemize}
\item {Grp. gram.:v. i.}
\end{itemize}
\begin{itemize}
\item {Utilização:Prov.}
\end{itemize}
\begin{itemize}
\item {Utilização:minh.}
\end{itemize}
\begin{itemize}
\item {Proveniência:(T. onom.)}
\end{itemize}
Produzir certo movimento e som intermittente, em consequência de ar ou gases, e formar bolhas successivas, como os ovos, quando se estrellam: \textunderscore os ovos já papujam...\textunderscore 
\section{Pápula}
\begin{itemize}
\item {Grp. gram.:f.}
\end{itemize}
\begin{itemize}
\item {Proveniência:(Lat. \textunderscore papula\textunderscore )}
\end{itemize}
Borbulha na pelle, sem conter pus nem serosidade.
Pequena protuberância, cheia de líquido aquoso, na epiderme de algumas plantas.
\section{Papuloso}
\begin{itemize}
\item {Grp. gram.:adj.}
\end{itemize}
Que tem pápulas.
\section{Papuzes}
\begin{itemize}
\item {Grp. gram.:m. pl.}
\end{itemize}
\begin{itemize}
\item {Utilização:Ant.}
\end{itemize}
\begin{itemize}
\item {Proveniência:(Do pers. \textunderscore pa\textunderscore , pé, e \textunderscore pux\textunderscore , cobertura)}
\end{itemize}
Chinelos, usados pelos Orientaes.
\section{Papyráceo}
\begin{itemize}
\item {Grp. gram.:adj.}
\end{itemize}
\begin{itemize}
\item {Proveniência:(De \textunderscore papyro\textunderscore )}
\end{itemize}
Semelhante ao papel.
\section{Papýreo}
\begin{itemize}
\item {Grp. gram.:adj.}
\end{itemize}
Relativo ao papyro.
\section{Papyrífero}
\begin{itemize}
\item {Grp. gram.:adj.}
\end{itemize}
\begin{itemize}
\item {Proveniência:(Do lat. \textunderscore papyrus\textunderscore  + \textunderscore ferre\textunderscore )}
\end{itemize}
Diz-se das plantas, cuja casca serve ou póde servir para a fabricação do papel.
\section{Papyro}
\begin{itemize}
\item {Grp. gram.:m.}
\end{itemize}
\begin{itemize}
\item {Proveniência:(Lat. \textunderscore papyrus\textunderscore )}
\end{itemize}
Variedade de cana, cuja haste, formada de folhetos sobrepostos, servia antigamente, depois de certa preparação, para nella se escrever.
Folha para escrever, feita com papyro.
Manuscrito antigo, feito em papyro.
\section{Papyrólitha}
\begin{itemize}
\item {Grp. gram.:f.}
\end{itemize}
\begin{itemize}
\item {Proveniência:(Do gr. \textunderscore papuros\textunderscore  + \textunderscore lithos\textunderscore )}
\end{itemize}
Massa de papel endurecida, que começa a empregar-se em construcções, na América.
\section{Paquan}
\begin{itemize}
\item {Grp. gram.:m.}
\end{itemize}
Planta gramínea do Brasil.
\section{Paquarima}
\begin{itemize}
\item {Grp. gram.:m.}
\end{itemize}
Peixe da Guiana inglesa, (\textunderscore phractocephalus bicolor\textunderscore ).
\section{Paquebote}
\begin{itemize}
\item {Grp. gram.:m.}
\end{itemize}
\begin{itemize}
\item {Utilização:Ant.}
\end{itemize}
\begin{itemize}
\item {Proveniência:(Do ingl. \textunderscore packet-boat\textunderscore )}
\end{itemize}
Barco, que transportava correspondência; paquete.
Espécie do pequena carruagem antiga. Cf. \textunderscore Diar. de Notícias\textunderscore , de Lisbôa, de 5-IX-900.
\section{Paqueboteiro}
\begin{itemize}
\item {Grp. gram.:m.}
\end{itemize}
Tripulante do paquebote.
\section{Paqueiro}
\begin{itemize}
\item {Grp. gram.:m.  e  adj.}
\end{itemize}
\begin{itemize}
\item {Utilização:Bras}
\end{itemize}
Diz-se do cão adestrado na caçada da paca.
\section{Paqueta}
\begin{itemize}
\item {Grp. gram.:f.}
\end{itemize}
\begin{itemize}
\item {Utilização:Prov.}
\end{itemize}
Rapariga que faz recados e pequenos serviços fóra de casa.
(Cp. \textunderscore paquête\textunderscore )
\section{Paquête}
\begin{itemize}
\item {Grp. gram.:m.}
\end{itemize}
\begin{itemize}
\item {Utilização:Ant.}
\end{itemize}
\begin{itemize}
\item {Utilização:Fig.}
\end{itemize}
\begin{itemize}
\item {Utilização:Des.}
\end{itemize}
\begin{itemize}
\item {Proveniência:(Do ingl. \textunderscore packet\textunderscore )}
\end{itemize}
Grande navio, ordinariamente a vapor, para transporte de passageiros, mercadorias e correspondência.
Barco pequeno e ligeiro, para transmissão de avisos.
Moço de recados.
Criado, ainda moço.
Alcoviteiro.
\textunderscore Papel paquete\textunderscore , papel fino, usado principalmente em correspondências que se deseja tenham o menor pêso possível.
\section{Paquéte}
\begin{itemize}
\item {Grp. gram.:f.}
\end{itemize}
\begin{itemize}
\item {Utilização:Chapel.}
\end{itemize}
Conjunto das diversas qualidades de pêlo, que podem entrar na fabricação dos chapéus.
\section{Paquiblefarose}
\begin{itemize}
\item {Grp. gram.:f.}
\end{itemize}
\begin{itemize}
\item {Utilização:Med.}
\end{itemize}
\begin{itemize}
\item {Proveniência:(Do gr. \textunderscore pakhus\textunderscore  + \textunderscore blepharon\textunderscore )}
\end{itemize}
Espessidão do tecido das palpebras.
\section{Paquicefalia}
\begin{itemize}
\item {Grp. gram.:f.}
\end{itemize}
Qualidade ou estado de paquicéfalo.
\section{Paquicéfalo}
\begin{itemize}
\item {Grp. gram.:adj.}
\end{itemize}
\begin{itemize}
\item {Proveniência:(Do gr. \textunderscore pakhus\textunderscore  + \textunderscore kephale\textunderscore )}
\end{itemize}
Que tem as paredes do crânio espêssas.
\section{Paquicerina}
\begin{itemize}
\item {Grp. gram.:f.}
\end{itemize}
\begin{itemize}
\item {Proveniência:(Do gr. \textunderscore pakhus\textunderscore  + \textunderscore keras\textunderscore )}
\end{itemize}
Gênero de insectos dípteros.
\section{Paquife}
\begin{itemize}
\item {Grp. gram.:m.}
\end{itemize}
\begin{itemize}
\item {Utilização:Heráld.}
\end{itemize}
Folhagem ornamental, que se estende pelo escudo, saindo do elmo.
Ornato arquitectónico de folhagens.
\section{Paquifilo}
\begin{itemize}
\item {Grp. gram.:adj.}
\end{itemize}
\begin{itemize}
\item {Utilização:Bot.}
\end{itemize}
\begin{itemize}
\item {Proveniência:(Do gr. \textunderscore pakhus\textunderscore  + \textunderscore phullon\textunderscore )}
\end{itemize}
Que tem fôlhas espessas.
\section{Paquidáctilo}
\begin{itemize}
\item {Grp. gram.:m.}
\end{itemize}
\begin{itemize}
\item {Proveniência:(Do gr. \textunderscore pakhus\textunderscore  + \textunderscore daktulos\textunderscore )}
\end{itemize}
Gênero de reptís sáurios.
\section{Paquidema}
\begin{itemize}
\item {Grp. gram.:m.}
\end{itemize}
Gênero de insectos coleópteros pentâmeros.
\section{Paquidendro}
\begin{itemize}
\item {Grp. gram.:m.}
\end{itemize}
\begin{itemize}
\item {Proveniência:(Do gr. \textunderscore pakhus\textunderscore  + \textunderscore dendron\textunderscore )}
\end{itemize}
Gênero de plantas liliáceas.
\section{Paquidermatocele}
\begin{itemize}
\item {Grp. gram.:m.}
\end{itemize}
\begin{itemize}
\item {Proveniência:(Do gr. \textunderscore pakhus\textunderscore  + \textunderscore derma\textunderscore  + \textunderscore kele\textunderscore )}
\end{itemize}
Tumor cutâneo, hipertrofia do tecido laminoso.
\section{Paquiderme}
\begin{itemize}
\item {Grp. gram.:adj.}
\end{itemize}
\begin{itemize}
\item {Grp. gram.:M. pl.}
\end{itemize}
\begin{itemize}
\item {Proveniência:(Do gr. \textunderscore pakhus\textunderscore  + \textunderscore derma\textunderscore )}
\end{itemize}
Que tem a pele espessa.
Ordem de mamíferos paquidermes.
\section{Paquidérmico}
\begin{itemize}
\item {Grp. gram.:adj.}
\end{itemize}
Relativo aos paquidermes; que tem pele semelhante á dos paquidermes.
\section{Paquidermo}
\begin{itemize}
\item {Grp. gram.:m.  e  adj.}
\end{itemize}
\begin{itemize}
\item {Proveniência:(Gr. \textunderscore pakhudermos\textunderscore )}
\end{itemize}
O mesmo ou melhor que \textunderscore paquiderme\textunderscore . Cf. R. Galvão, \textunderscore Vocab.\textunderscore 
\section{Paquigástrico}
\begin{itemize}
\item {Grp. gram.:adj.}
\end{itemize}
\begin{itemize}
\item {Utilização:Zool.}
\end{itemize}
\begin{itemize}
\item {Proveniência:(Do gr. \textunderscore pakhus\textunderscore  + \textunderscore gaster\textunderscore )}
\end{itemize}
Que tem o ventre muito grosso.
\section{Paquilêmures}
\begin{itemize}
\item {Grp. gram.:m. pl.}
\end{itemize}
\begin{itemize}
\item {Utilização:Zool.}
\end{itemize}
\begin{itemize}
\item {Proveniência:(Do gr. \textunderscore pakhus\textunderscore  + lat. \textunderscore lemures\textunderscore )}
\end{itemize}
Animaes de pele espêssa, classificados entre os insectívoros e os carnívoros.
\section{Paquilépide}
\begin{itemize}
\item {Grp. gram.:f.}
\end{itemize}
\begin{itemize}
\item {Proveniência:(Do gr. \textunderscore pakhus\textunderscore  + \textunderscore lepis\textunderscore )}
\end{itemize}
Planta quicoriácea.
\section{Paquilócero}
\begin{itemize}
\item {Grp. gram.:m.}
\end{itemize}
Gênero de insectos coleópteros heterómeros.
\section{Paquílopo}
\begin{itemize}
\item {Grp. gram.:m.}
\end{itemize}
Gênero de insectos coleópteros pentâmeros.
\section{Paquima}
\begin{itemize}
\item {Grp. gram.:m.}
\end{itemize}
Gênero de cogumelos grossos e subterrâneos.
\section{Paquimeningite}
\begin{itemize}
\item {Grp. gram.:f.}
\end{itemize}
\begin{itemize}
\item {Utilização:Med.}
\end{itemize}
\begin{itemize}
\item {Proveniência:(De \textunderscore pakhus\textunderscore  + \textunderscore meningite\textunderscore )}
\end{itemize}
Inflamação da dura-máter.
\section{Paquimerina}
\begin{itemize}
\item {Grp. gram.:f.}
\end{itemize}
Gênero de insectos dípteros.
\section{Paquimorfo}
\begin{itemize}
\item {Grp. gram.:m.}
\end{itemize}
\begin{itemize}
\item {Proveniência:(Do gr. \textunderscore pakhus\textunderscore  + \textunderscore morphe\textunderscore )}
\end{itemize}
Gênero de insectos coleópteros pentâmeros.
\section{Paquipleuro}
\begin{itemize}
\item {Grp. gram.:m.}
\end{itemize}
\begin{itemize}
\item {Proveniência:(Lat. bot. \textunderscore pachypleurum\textunderscore )}
\end{itemize}
Gênero de plantas umbeliferas.
\section{Paquiquimia}
\begin{itemize}
\item {Grp. gram.:f.}
\end{itemize}
\begin{itemize}
\item {Utilização:Med.}
\end{itemize}
\begin{itemize}
\item {Proveniência:(Do gr. \textunderscore pakhus\textunderscore  + \textunderscore khumos\textunderscore )}
\end{itemize}
Espessidão mórbida dos humores.
\section{Paquirina}
\begin{itemize}
\item {Grp. gram.:f.}
\end{itemize}
Gênero de insectos dípteros.
\section{Paquirrizo}
\begin{itemize}
\item {Proveniência:(Do gr. \textunderscore pakhus\textunderscore  + \textunderscore rhíza\textunderscore )}
\end{itemize}
Gênero de plantas leguminosas.
\section{Paquirrínquidos}
\begin{itemize}
\item {Grp. gram.:m. pl.}
\end{itemize}
\begin{itemize}
\item {Utilização:Zool.}
\end{itemize}
\begin{itemize}
\item {Proveniência:(Do gr. \textunderscore pakhus\textunderscore  + \textunderscore rhunkhos\textunderscore )}
\end{itemize}
Uma das divisões, em que se classificam os insectos coleópteros tetrâmeros.
\section{Paquistêmone}
\begin{itemize}
\item {Grp. gram.:m.}
\end{itemize}
\begin{itemize}
\item {Proveniência:(Do gr. \textunderscore pakhus\textunderscore  + \textunderscore stemon\textunderscore )}
\end{itemize}
Gênero de plantas euforbiáceas.
\section{Paquístoma}
\begin{itemize}
\item {Grp. gram.:m.}
\end{itemize}
\begin{itemize}
\item {Proveniência:(Do gr. \textunderscore pakhus\textunderscore  + \textunderscore stoma\textunderscore )}
\end{itemize}
Gênero de orquídeas.
\section{Paquístomo}
\begin{itemize}
\item {Grp. gram.:m.}
\end{itemize}
\begin{itemize}
\item {Proveniência:(Do gr. \textunderscore pakhus\textunderscore  + \textunderscore stoma\textunderscore )}
\end{itemize}
Gênero de insectos dípteros.
\section{Paquítria}
\begin{itemize}
\item {Grp. gram.:f.}
\end{itemize}
Gênero de insectos coleópteros pentâmeros.
(Cp. \textunderscore paquítrico\textunderscore )
\section{Paquítrico}
\begin{itemize}
\item {Grp. gram.:adj.}
\end{itemize}
\begin{itemize}
\item {Utilização:Zool.}
\end{itemize}
\begin{itemize}
\item {Proveniência:(Do gr. \textunderscore pakhus\textunderscore  + \textunderscore trikhos\textunderscore )}
\end{itemize}
Que tem pêlo espesso.
\section{Par}
\begin{itemize}
\item {Grp. gram.:adj.}
\end{itemize}
\begin{itemize}
\item {Grp. gram.:M.}
\end{itemize}
\begin{itemize}
\item {Utilização:Fam.}
\end{itemize}
\begin{itemize}
\item {Grp. gram.:Pl.}
\end{itemize}
\begin{itemize}
\item {Utilização:Ant.}
\end{itemize}
\begin{itemize}
\item {Grp. gram.:Loc. adv.}
\end{itemize}
\begin{itemize}
\item {Grp. gram.:Loc. adv.}
\end{itemize}
\begin{itemize}
\item {Proveniência:(Lat. \textunderscore par\textunderscore )}
\end{itemize}
Igual; semelhante.
Que é divisível por dois, (falando-se de um número).
Disposto symetricamente dos dois lados de um eixo, nos vegetaes.
O macho e a fêmea de certas aves.
Duas pessôas do mesmo ou differente sexo, especialmente marido e mulher.
Duas pessôas que dançam juntas: \textunderscore o par marcante\textunderscore .
Cada uma dessas duas pessoas: \textunderscore fui seu par na contradança\textunderscore .
Utensílio ou peça do vestuário, formado de duas partes iguaes, como ceroilas, calças, etc.
Conjunto de duas peças semelhantes, das quaes uma não se usa sem a outra: \textunderscore um par de meias\textunderscore .
Parelha, junta.
Cada um dos membros da Câmara alta, que houve em Portugal e que há em Inglaterra.
Grande número indeterminado: \textunderscore andou um par da léguas\textunderscore !
Indivíduos de certa classe: \textunderscore dá-se bem com os seus pares\textunderscore .
Principaes vassallos de um senhor feudal.
Grandes vassallos do rei.
Os paladinos de Carlos Magno.
\textunderscore A par\textunderscore , ao lado, junto, ao mesmo tempo.
\textunderscore Em par\textunderscore , o mesmo que \textunderscore a par\textunderscore . Cf. Castilho, \textunderscore Colloq. Ald.\textunderscore , 180, 230.
\section{Para}
\begin{itemize}
\item {fónica:pâ}
\end{itemize}
\begin{itemize}
\item {Grp. gram.:prep.}
\end{itemize}
Na direcção de: \textunderscore navegou para o Norte\textunderscore .
A fim de: \textunderscore procurou-me para conversarmos\textunderscore .
Com destino a.
Em proporção de.
Apropriado a: \textunderscore pano bom para camisas\textunderscore .
Relativamente a; etc.
(Port. ant. \textunderscore pera\textunderscore )
\section{Pará}
\begin{itemize}
\item {Grp. gram.:m.}
\end{itemize}
Antiga medida de Cochim e Damão, correspondente a 15 litros e meio proximamente.
\section{Para...}
\begin{itemize}
\item {Grp. gram.:pref.}
\end{itemize}
\begin{itemize}
\item {Proveniência:(Do gr. \textunderscore para\textunderscore )}
\end{itemize}
(designativo de proximidade, comparação, opposição)
\section{Pára-água}
\begin{itemize}
\item {Grp. gram.:m.}
\end{itemize}
(V.guarda-chuva)
\section{Parabânico}
\begin{itemize}
\item {Grp. gram.:adj.}
\end{itemize}
Diz-se de um ácido, produzido pela decomposição do ácido úrico, sob a acção do ácido nítrico.
\section{Parábase}
\begin{itemize}
\item {Grp. gram.:f.}
\end{itemize}
\begin{itemize}
\item {Proveniência:(Gr. \textunderscore parabasis\textunderscore )}
\end{itemize}
Parte do côro, na comédia grega, em que se faziam referências críticas.
Parte da comédia grega, em que o actor falava aos espectadores, em assumpto estranho á comédia.
\section{Parabem}
\begin{itemize}
\item {Grp. gram.:m.}
\end{itemize}
\begin{itemize}
\item {Proveniência:(De \textunderscore para\textunderscore  + \textunderscore bem\textunderscore )}
\end{itemize}
Felicitação, congratulação.
\section{Parabens}
\begin{itemize}
\item {Grp. gram.:m. pl.}
\end{itemize}
\begin{itemize}
\item {Proveniência:(De \textunderscore para\textunderscore  + \textunderscore bem\textunderscore )}
\end{itemize}
Felicitação, congratulação.
\section{Parablasto}
\begin{itemize}
\item {Grp. gram.:adj.}
\end{itemize}
\begin{itemize}
\item {Utilização:Med.}
\end{itemize}
\begin{itemize}
\item {Proveniência:(Do gr. \textunderscore para\textunderscore  + \textunderscore blastos\textunderscore )}
\end{itemize}
Diz-se das doenças, que são acompanhadas de alterações anatómicas nos tecidos, como os exanthemas, etc.
\section{Parábola}
\begin{itemize}
\item {Grp. gram.:f.}
\end{itemize}
\begin{itemize}
\item {Utilização:Geom.}
\end{itemize}
\begin{itemize}
\item {Utilização:Prov.}
\end{itemize}
\begin{itemize}
\item {Utilização:trasm.}
\end{itemize}
\begin{itemize}
\item {Proveniência:(Lat. \textunderscore parabola\textunderscore )}
\end{itemize}
Espécie do allegoria, ou comparação de objectos remotamente relacionados, contendo geralmente preceito ou doutrina moral: \textunderscore a parábola do filho pródigo\textunderscore .
Curva plana, cujos pontos distam igualmente de um ponto fixo e de uma recta fixa.
Dobadoira dobrada, isto é, com duas ordens de travessas, cruzadas.
\section{Parabolano}
\begin{itemize}
\item {Grp. gram.:m.}
\end{itemize}
\begin{itemize}
\item {Proveniência:(Lat. \textunderscore parabolanus\textunderscore )}
\end{itemize}
Serventuário ecclesiástico, que tinha a seu cargo o tratamento dos enfermos.
\section{Parabolão}
\begin{itemize}
\item {Grp. gram.:m.}
\end{itemize}
Grande parábola. Cf. Camillo, \textunderscore Ôlho de Vidro\textunderscore , 99.
\section{Parabolicamente}
\begin{itemize}
\item {Grp. gram.:adv.}
\end{itemize}
Do modo parabólico; por meio do parábola: \textunderscore falar parabolicamente\textunderscore .
\section{Parabólico}
\begin{itemize}
\item {Grp. gram.:adj.}
\end{itemize}
\begin{itemize}
\item {Proveniência:(Lat. \textunderscore parabolicus\textunderscore )}
\end{itemize}
Relativo ou semelhante a parábola.
\section{Parabolismo}
\begin{itemize}
\item {Grp. gram.:m.}
\end{itemize}
\begin{itemize}
\item {Proveniência:(De \textunderscore parábola\textunderscore )}
\end{itemize}
Carácter parabólico.
\section{Parabolista}
\begin{itemize}
\item {Grp. gram.:m.}
\end{itemize}
Aquelle que expõe parábolas.
\section{Parabolizar}
\begin{itemize}
\item {Grp. gram.:v. i.}
\end{itemize}
Expor por parábolas. Cf. André Mascarenhas, \textunderscore Espanha Destruída\textunderscore , pról.
\section{Parabolóide}
\begin{itemize}
\item {Grp. gram.:m.}
\end{itemize}
\begin{itemize}
\item {Utilização:Geom.}
\end{itemize}
\begin{itemize}
\item {Grp. gram.:Adj.}
\end{itemize}
\begin{itemize}
\item {Proveniência:(Do gr. \textunderscore parabole\textunderscore  + \textunderscore eidos\textunderscore )}
\end{itemize}
Superfície, gerada por uma parábola, que se move sôbre outra, contida em diversos planos.
Que tem fórma de parábola geométrica.
\section{Paracanaxi}
\begin{itemize}
\item {Grp. gram.:m.}
\end{itemize}
\begin{itemize}
\item {Utilização:Bras}
\end{itemize}
Árvore tinctória do Alto Amazonas.
\section{Paracarpo}
\begin{itemize}
\item {Grp. gram.:m.}
\end{itemize}
\begin{itemize}
\item {Utilização:Bot.}
\end{itemize}
\begin{itemize}
\item {Proveniência:(Do gr. \textunderscore para\textunderscore  + \textunderscore karpos\textunderscore )}
\end{itemize}
Ovário, que abortou por qualquer causa natural.
\section{Paracaúba}
\begin{itemize}
\item {Grp. gram.:f.}
\end{itemize}
Planta leguminosa do Brasil.
\section{Paracéfalo}
\begin{itemize}
\item {Grp. gram.:m.}
\end{itemize}
\begin{itemize}
\item {Proveniência:(Do gr. \textunderscore para\textunderscore  + \textunderscore kephale\textunderscore )}
\end{itemize}
Monstro, de cabeça volumosa, disforme, com apenas um rudimento de bôca e de órgãos sensórios.
\section{Paracelsismo}
\begin{itemize}
\item {Grp. gram.:m.}
\end{itemize}
Systema dos Paracelsistas.
\section{Paracelsistas}
\begin{itemize}
\item {Grp. gram.:m. pl.}
\end{itemize}
Sectários de Paracelso, que atacou o systema médico de Galeno e deu aos medicamentos mineraes uma importância que antes não tinham.
\section{Paracentese}
\begin{itemize}
\item {Grp. gram.:f.}
\end{itemize}
\begin{itemize}
\item {Proveniência:(Lat. \textunderscore paracentesis\textunderscore )}
\end{itemize}
Qualquer operação, com que se faz evacuar um líquido, accumulado numa parte do corpo.
\section{Paracéphalo}
\begin{itemize}
\item {Grp. gram.:m.}
\end{itemize}
\begin{itemize}
\item {Proveniência:(Do gr. \textunderscore para\textunderscore  + \textunderscore kephale\textunderscore )}
\end{itemize}
Monstro, de cabeça volumosa, disforme, com apenas um rudimento de bôca e de órgãos sensórios.
\section{Parachim}
\begin{itemize}
\item {Grp. gram.:m.}
\end{itemize}
\begin{itemize}
\item {Proveniência:(T. onom. Cp. \textunderscore patachim\textunderscore )}
\end{itemize}
O mesmo que \textunderscore megengra\textunderscore .
\section{Parachlorophenol}
\begin{itemize}
\item {Grp. gram.:m.}
\end{itemize}
Produto pharmacêutico, derivado do chloro e do ácido phênico.
\section{Pára-choque}
\begin{itemize}
\item {Grp. gram.:m.}
\end{itemize}
Cada uma das bombas ou peças circulares que, nas extremidades de uma carruagem de caminho de ferro, recebem, á paragem do comboio, o choque das peças analogas de outra carruagem, attenuando-o por meio de molas a que se ligam.
\section{Parachronismo}
\begin{itemize}
\item {Grp. gram.:m.}
\end{itemize}
\begin{itemize}
\item {Proveniência:(Do gr. \textunderscore para\textunderscore  + \textunderscore khronos\textunderscore )}
\end{itemize}
O mesmo que \textunderscore metachronismo\textunderscore .
\section{Paraciânico}
\begin{itemize}
\item {Grp. gram.:adj.}
\end{itemize}
\begin{itemize}
\item {Utilização:Chím.}
\end{itemize}
\begin{itemize}
\item {Proveniência:(De \textunderscore paracíano\textunderscore )}
\end{itemize}
O mesmo que \textunderscore fulmínico\textunderscore .
\section{Paracíano}
\begin{itemize}
\item {Grp. gram.:m.}
\end{itemize}
\begin{itemize}
\item {Utilização:Chím.}
\end{itemize}
\begin{itemize}
\item {Proveniência:(De \textunderscore para...\textunderscore  + \textunderscore cíano\textunderscore )}
\end{itemize}
Um dos productos da decomposição do ciano pela água, o álcool e o amoníaco.
\section{Paracianogênio}
\begin{itemize}
\item {Grp. gram.:m.}
\end{itemize}
\begin{itemize}
\item {Utilização:Chím.}
\end{itemize}
\begin{itemize}
\item {Proveniência:(De \textunderscore para...\textunderscore  + \textunderscore cianogênio\textunderscore )}
\end{itemize}
Substância negra, isómera com o cíanogênio, e que se fórma nos vasos em que se aquece o cianeto para preparar o gás cianogênico.
\section{Paraciesia}
\begin{itemize}
\item {Grp. gram.:f.}
\end{itemize}
\begin{itemize}
\item {Utilização:Med.}
\end{itemize}
\begin{itemize}
\item {Proveniência:(Do gr. \textunderscore para\textunderscore  + \textunderscore kuesis\textunderscore )}
\end{itemize}
Gravidez extra-uterina.
\section{Para-cima}
\begin{itemize}
\item {Grp. gram.:loc. adv.}
\end{itemize}
Para a parte superior; para o alto.
\section{Paracinancia}
\begin{itemize}
\item {Grp. gram.:f.}
\end{itemize}
\begin{itemize}
\item {Utilização:Med.}
\end{itemize}
\begin{itemize}
\item {Proveniência:(De \textunderscore para\textunderscore  + lat. \textunderscore cynanche\textunderscore )}
\end{itemize}
Angina ligeira.
\section{Paracismeiro}
\begin{itemize}
\item {Grp. gram.:m.}
\end{itemize}
\begin{itemize}
\item {Utilização:Prov.}
\end{itemize}
\begin{itemize}
\item {Utilização:trasm.}
\end{itemize}
\begin{itemize}
\item {Proveniência:(De \textunderscore paracismo\textunderscore )}
\end{itemize}
O mesmo que \textunderscore moteneteiro\textunderscore ; pantomimeiro.
\section{Paracismice}
\begin{itemize}
\item {Grp. gram.:f.}
\end{itemize}
\begin{itemize}
\item {Utilização:Prov.}
\end{itemize}
\begin{itemize}
\item {Utilização:trasm.}
\end{itemize}
Dito ou acto de parocismeiro; pantomimice.
\section{Paracismo}
\begin{itemize}
\item {Grp. gram.:m.}
\end{itemize}
\begin{itemize}
\item {Utilização:Des.}
\end{itemize}
O mesmo que \textunderscore paroxismo\textunderscore . Cf. Vieira, IX, 479.
\section{Paráclase}
\begin{itemize}
\item {Grp. gram.:f.}
\end{itemize}
\begin{itemize}
\item {Proveniência:(Do gr. \textunderscore para\textunderscore  + \textunderscore klao\textunderscore )}
\end{itemize}
Folha ou fractura de rochas por escorregamento, conservando-se, ou não, horizontaes os estratos.
\section{Paracleteadura}
\begin{itemize}
\item {Grp. gram.:f}
\end{itemize}
Acto de paracletear; suggestão. Cf. B. Pereira, \textunderscore Prosódia\textunderscore , vb. \textunderscore suggestio\textunderscore .
\section{Paracletear}
\begin{itemize}
\item {Grp. gram.:v. t.}
\end{itemize}
\begin{itemize}
\item {Utilização:Des.}
\end{itemize}
\begin{itemize}
\item {Proveniência:(De \textunderscore Paracleto\textunderscore )}
\end{itemize}
Apontar, suggerir a (alguém) o que deve responder.
\section{Paracleto}
\begin{itemize}
\item {Grp. gram.:m.}
\end{itemize}
\begin{itemize}
\item {Utilização:Fig.}
\end{itemize}
\begin{itemize}
\item {Proveniência:(Lat. \textunderscore paracletus\textunderscore )}
\end{itemize}
Espírito Santo.
Mentor; intercessor.
\section{Paraclorofenol}
\begin{itemize}
\item {Grp. gram.:m.}
\end{itemize}
Produto farmacêutico, derivado do cloro e do ácido fênico.
\section{Paracmástico}
\begin{itemize}
\item {Grp. gram.:adj.}
\end{itemize}
\begin{itemize}
\item {Utilização:Med.}
\end{itemize}
\begin{itemize}
\item {Proveniência:(Do gr. \textunderscore para\textunderscore  + \textunderscore akme\textunderscore )}
\end{itemize}
Que começa a deminuír, (falando-se de uma doença).
\section{Paracomênico}
\begin{itemize}
\item {Grp. gram.:adj.}
\end{itemize}
\begin{itemize}
\item {Utilização:Chím.}
\end{itemize}
Diz-se de um producto crystallizável da decomposição do ácido mecónico.
\section{Paracorola}
\begin{itemize}
\item {Grp. gram.:f.}
\end{itemize}
\begin{itemize}
\item {Utilização:Bot.}
\end{itemize}
\begin{itemize}
\item {Proveniência:(De \textunderscore para...\textunderscore  + \textunderscore corola\textunderscore )}
\end{itemize}
Espécie de pequena corola, situada no meio da corola própriamente dita, como sucede nos narcisos.
\section{Paracorolla}
\begin{itemize}
\item {Grp. gram.:f.}
\end{itemize}
\begin{itemize}
\item {Utilização:Bot.}
\end{itemize}
\begin{itemize}
\item {Proveniência:(De \textunderscore para...\textunderscore  + \textunderscore corolla\textunderscore )}
\end{itemize}
Espécie de pequena corolla, situada no meio da corolla própriamente dita, como succede nos narcisos.
\section{Paracoronal}
\begin{itemize}
\item {Grp. gram.:m.  e  adj.}
\end{itemize}
\begin{itemize}
\item {Utilização:Neol.}
\end{itemize}
\begin{itemize}
\item {Proveniência:(De \textunderscore para...\textunderscore  + \textunderscore coronal\textunderscore )}
\end{itemize}
Diz-se de cada um dos planos parallelos ao plano vertical transversal, em Anatomia.
\section{Paracronismo}
\begin{itemize}
\item {Grp. gram.:m.}
\end{itemize}
\begin{itemize}
\item {Proveniência:(Do gr. \textunderscore para\textunderscore  + \textunderscore khronos\textunderscore )}
\end{itemize}
O mesmo que \textunderscore metacronismo\textunderscore .
\section{Paracuri}
\begin{itemize}
\item {Grp. gram.:m.}
\end{itemize}
O mesmo que \textunderscore meladinha\textunderscore .
\section{Paracusia}
\begin{itemize}
\item {Grp. gram.:f.}
\end{itemize}
\begin{itemize}
\item {Utilização:Med.}
\end{itemize}
\begin{itemize}
\item {Proveniência:(Do gr. \textunderscore parakouein\textunderscore )}
\end{itemize}
Zumbido nos ouvidos.
Estado de quem ouve ruídos imaginários ou que só existem no interior do ouvido.
\section{Paracutaca}
\begin{itemize}
\item {Grp. gram.:f.}
\end{itemize}
\begin{itemize}
\item {Utilização:Bras}
\end{itemize}
Árvore do valle do Amazonas, da qual se alimentam tartarugas.
\section{Paracuuba}
\begin{itemize}
\item {Grp. gram.:f.}
\end{itemize}
Árvore do Brasil, cuja madeira se emprega em marcenaria.
O mesmo que \textunderscore paracaúba\textunderscore ?
\section{Paracyânico}
\begin{itemize}
\item {Grp. gram.:adj.}
\end{itemize}
\begin{itemize}
\item {Utilização:Chím.}
\end{itemize}
\begin{itemize}
\item {Proveniência:(De \textunderscore paracýano\textunderscore )}
\end{itemize}
O mesmo que \textunderscore fulmínico\textunderscore .
\section{Paracýano}
\begin{itemize}
\item {Grp. gram.:m.}
\end{itemize}
\begin{itemize}
\item {Utilização:Chím.}
\end{itemize}
\begin{itemize}
\item {Proveniência:(De \textunderscore para...\textunderscore  + \textunderscore cýano\textunderscore )}
\end{itemize}
Um dos productos da decomposição do cyano pela água, o álcool e o ammoníaco.
\section{Paracyanogênio}
\begin{itemize}
\item {Grp. gram.:m.}
\end{itemize}
\begin{itemize}
\item {Utilização:Chím.}
\end{itemize}
\begin{itemize}
\item {Proveniência:(De \textunderscore para...\textunderscore  + \textunderscore cyanogênio\textunderscore )}
\end{itemize}
Substância negra, isómera com o cýanogênio, e que se fórma nos vasos em que se aquece o cyaneto para preparar o gás cyanogênico.
\section{Paracyesia}
\begin{itemize}
\item {Grp. gram.:f.}
\end{itemize}
\begin{itemize}
\item {Utilização:Med.}
\end{itemize}
\begin{itemize}
\item {Proveniência:(Do gr. \textunderscore para\textunderscore  + \textunderscore kuesis\textunderscore )}
\end{itemize}
Gravidez extra-uterina.
\section{Paracynancia}
\begin{itemize}
\item {Grp. gram.:f.}
\end{itemize}
\begin{itemize}
\item {Utilização:Med.}
\end{itemize}
\begin{itemize}
\item {Proveniência:(De \textunderscore para\textunderscore  + lat. \textunderscore cynanche\textunderscore )}
\end{itemize}
Angina ligeira.
\section{Parada}
\begin{itemize}
\item {Grp. gram.:f.}
\end{itemize}
\begin{itemize}
\item {Utilização:Ant.}
\end{itemize}
Acto ou efeito de parar.
Lugar onde se pára.
Demora.
Estação.
Pausa.
Quantia, que se aposta ou arrisca no jôgo, por cada vez.
Reunião de tropas ou a sua passagem, para exercicio ou revista.
Acto de se defender de um golpe, no jôgo da esgrima.
O mesmo que \textunderscore jantar\textunderscore .
\section{Paradáctilo}
\begin{itemize}
\item {Grp. gram.:m.}
\end{itemize}
\begin{itemize}
\item {Utilização:Zool.}
\end{itemize}
\begin{itemize}
\item {Proveniência:(Do gr. \textunderscore para\textunderscore  + \textunderscore daktulos\textunderscore )}
\end{itemize}
Parte lateral dos dedos das aves.
\section{Paradáctylo}
\begin{itemize}
\item {Grp. gram.:m.}
\end{itemize}
\begin{itemize}
\item {Utilização:Zool.}
\end{itemize}
\begin{itemize}
\item {Proveniência:(Do gr. \textunderscore para\textunderscore  + \textunderscore daktulos\textunderscore )}
\end{itemize}
Parte lateral dos dedos das aves.
\section{Paradeiro}
\begin{itemize}
\item {Grp. gram.:m.}
\end{itemize}
\begin{itemize}
\item {Proveniência:(De \textunderscore parar\textunderscore )}
\end{itemize}
Lugar, onde alguma coisa está, pára ou finda: \textunderscore ignora-se o paradeiro do desertor\textunderscore .
\section{Paradela}
\begin{itemize}
\item {Grp. gram.:f.}
\end{itemize}
\begin{itemize}
\item {Utilização:Prov.}
\end{itemize}
\begin{itemize}
\item {Utilização:alg.}
\end{itemize}
Lugar exposto ou desabrigado.
\section{Paradementes}
\begin{itemize}
\item {Grp. gram.:interj.}
\end{itemize}
\begin{itemize}
\item {Utilização:Ant.}
\end{itemize}
O mesmo que \textunderscore paramentes\textunderscore .
\section{Paradigma}
\begin{itemize}
\item {Grp. gram.:m.}
\end{itemize}
\begin{itemize}
\item {Utilização:Gram.}
\end{itemize}
\begin{itemize}
\item {Proveniência:(Lat. \textunderscore paradigma\textunderscore )}
\end{itemize}
Modêlo.
Exemplo ou typo de conjugação ou declinação grammatical.
\section{Paradigmal}
\begin{itemize}
\item {Grp. gram.:adj.}
\end{itemize}
Relativo a paradigma.
\section{Paradisíaco}
\begin{itemize}
\item {Grp. gram.:adj.}
\end{itemize}
\begin{itemize}
\item {Proveniência:(Lat. \textunderscore paradisiacus\textunderscore )}
\end{itemize}
Relativo ao paraíso; próprio do paraíso; celeste.
Magnífico.
\section{Paradísico}
\begin{itemize}
\item {Grp. gram.:adj.}
\end{itemize}
\begin{itemize}
\item {Proveniência:(Do lat. \textunderscore paradisus\textunderscore )}
\end{itemize}
O mesmo que \textunderscore paradisíaco\textunderscore . Cf. Alv. Mendes, \textunderscore Plágios\textunderscore , 61.
\section{Parado}
\begin{itemize}
\item {Grp. gram.:adj.}
\end{itemize}
Que deixou de andar ou de se mover.
\textunderscore Bem parado\textunderscore , que tem probabilidades de bom êxito, que está bem encaminhado.
\section{Paradoiro}
\begin{itemize}
\item {Grp. gram.:m.}
\end{itemize}
O mesmo que \textunderscore paradeiro\textunderscore .
\section{Parador}
\begin{itemize}
\item {Grp. gram.:m.}
\end{itemize}
\begin{itemize}
\item {Utilização:Bras. do S}
\end{itemize}
\begin{itemize}
\item {Proveniência:(De \textunderscore parar\textunderscore )}
\end{itemize}
Cavalleiro, que tem facilidade em ficar do pé, quando o cavallo cai.
\section{Paradouro}
\begin{itemize}
\item {Grp. gram.:m.}
\end{itemize}
O mesmo que \textunderscore paradeiro\textunderscore .
\section{Paradoxal}
\begin{itemize}
\item {fónica:csal}
\end{itemize}
\begin{itemize}
\item {Grp. gram.:adj.}
\end{itemize}
Que envolve paradoxo, ou que é da natureza delle: \textunderscore these paradoxal\textunderscore .
\section{Paradoxo}
\begin{itemize}
\item {fónica:cso}
\end{itemize}
\begin{itemize}
\item {Grp. gram.:m.}
\end{itemize}
\begin{itemize}
\item {Grp. gram.:Adj.}
\end{itemize}
\begin{itemize}
\item {Proveniência:(Lat. \textunderscore paradoxon\textunderscore )}
\end{itemize}
Opinião contrária á opinião commum; desconchavo; asneira.
Paradoxal.
\section{Paraense}
\begin{itemize}
\item {Grp. gram.:adj.}
\end{itemize}
\begin{itemize}
\item {Grp. gram.:M.}
\end{itemize}
Relativo ao Pará.
Habitante do Pará.
\section{Pára-fogo}
\begin{itemize}
\item {Grp. gram.:m.}
\end{itemize}
O mesmo que \textunderscore guarda-fogo\textunderscore .
\section{Paraffina}
\begin{itemize}
\item {Grp. gram.:f.}
\end{itemize}
\begin{itemize}
\item {Proveniência:(Do lat. \textunderscore parum affinis\textunderscore , porque, quando foi descoberta, não se lhe achava affinidade com outros corpos)}
\end{itemize}
Substância sólida e branca, extrahida dos xistos bituminosos, e formada de carbóne e hydrogênio, nas proporções necessárias para dar luz brilhante.
Resíduo da destillação do petróleo.
\section{Paraffinagem}
\begin{itemize}
\item {Grp. gram.:f.}
\end{itemize}
Acto de paraffinar.
\section{Paraffinar}
\begin{itemize}
\item {Grp. gram.:v. t.}
\end{itemize}
Converter em paraffina; dar a natureza da paraffina a; misturar com paraffina.
\section{Paraffinaria}
\begin{itemize}
\item {Grp. gram.:f.}
\end{itemize}
Fábrica de velas de paraffina.
\section{Paraffinização}
\begin{itemize}
\item {Grp. gram.:f.}
\end{itemize}
Acto ou effeito de paraffinizar.
\section{Paraffinizar}
\begin{itemize}
\item {Grp. gram.:v. t.}
\end{itemize}
O mesmo que \textunderscore paraffinar\textunderscore .
\section{Parafina}
\begin{itemize}
\item {Grp. gram.:f.}
\end{itemize}
\begin{itemize}
\item {Proveniência:(Do lat. \textunderscore parum affinis\textunderscore , porque, quando foi descoberta, não se lhe achava afinidade com outros corpos)}
\end{itemize}
Substância sólida e branca, extrahida dos xistos bituminosos, e formada de carbóne e hidrogênio, nas proporções necessárias para dar luz brilhante.
Resíduo da destilação do petróleo.
\section{Parafinagem}
\begin{itemize}
\item {Grp. gram.:f.}
\end{itemize}
Acto de parafinar.
\section{Parafinar}
\begin{itemize}
\item {Grp. gram.:v. t.}
\end{itemize}
Converter em parafina; dar a natureza da parafina a; misturar com parafina.
\section{Parafinaria}
\begin{itemize}
\item {Grp. gram.:f.}
\end{itemize}
Fábrica de velas de parafina.
\section{Parafinização}
\begin{itemize}
\item {Grp. gram.:f.}
\end{itemize}
Acto ou efeito de parafinizar.
\section{Parafinizar}
\begin{itemize}
\item {Grp. gram.:v. t.}
\end{itemize}
O mesmo que \textunderscore parafinar\textunderscore .
\section{Parafitas}
\begin{itemize}
\item {Grp. gram.:m.}
\end{itemize}
\begin{itemize}
\item {Utilização:Prov.}
\end{itemize}
\begin{itemize}
\item {Utilização:trasm.}
\end{itemize}
Janota, peralta, pisa-flôres.
\section{Paraformaldehydo}
\begin{itemize}
\item {Grp. gram.:m.}
\end{itemize}
\begin{itemize}
\item {Utilização:Chím.}
\end{itemize}
Formalina sólida.
\section{Paraformaldeído}
\begin{itemize}
\item {Grp. gram.:m.}
\end{itemize}
\begin{itemize}
\item {Utilização:Chím.}
\end{itemize}
Formalina sólida.
\section{Parafusa}
\begin{itemize}
\item {Grp. gram.:f.}
\end{itemize}
\begin{itemize}
\item {Utilização:Prov.}
\end{itemize}
\begin{itemize}
\item {Utilização:trasm.}
\end{itemize}
O mesmo que \textunderscore fusa\textunderscore ^2.
\section{Parafusador}
\begin{itemize}
\item {Grp. gram.:m.  e  adj.}
\end{itemize}
Aquelle ou aquillo que parafusa.
\section{Parafusão}
\begin{itemize}
\item {Grp. gram.:adj.}
\end{itemize}
\begin{itemize}
\item {Proveniência:(De \textunderscore parafuso\textunderscore )}
\end{itemize}
Diz-se de uma espécie de turbilho.
\section{Parafusar}
\begin{itemize}
\item {Grp. gram.:v. t.}
\end{itemize}
\begin{itemize}
\item {Grp. gram.:V. i.}
\end{itemize}
Apertar por meio de parafuso ou rosca.
Têr a ideia fixa em alguma coisa; scismar, matutar.
Fazer indagações; especular.
\section{Parafusaria}
\begin{itemize}
\item {Grp. gram.:v. t.}
\end{itemize}
Fábrica de parafusos.
\section{Parafuso}
\begin{itemize}
\item {Grp. gram.:m.}
\end{itemize}
\begin{itemize}
\item {Proveniência:(De \textunderscore fuso\textunderscore )}
\end{itemize}
Cylindro, sulcado em espiral, e destinado a entrar numa peça chamada porca, cujo respectivo orificio é também sulcado em espiral, mas em que os sulcos correspondem ás saliências do cylindro.
Rosca, tarraxa.
\textunderscore Parafuso de chamada\textunderscore , ou \textunderscore de reclamo\textunderscore , o parafuso que serve para ajustar o retículo do óculo astronómico ou topográphico com o objecto de mira.
\section{Paraganas}
\begin{itemize}
\item {Grp. gram.:f. pl.}
\end{itemize}
\begin{itemize}
\item {Utilização:Ant.}
\end{itemize}
Bens feudaes, com o encargo de certos serviços.
\section{Paragão}
\begin{itemize}
\item {Grp. gram.:m.}
\end{itemize}
\begin{itemize}
\item {Utilização:Des.}
\end{itemize}
Semelhança; comparação.
(Cast. \textunderscore paragon\textunderscore )
\section{Paragem}
\begin{itemize}
\item {Grp. gram.:f.}
\end{itemize}
Acto de parar.
Lugar onde se pára; parada.
Parte do mar, accessível á navegação.
\section{Paragenesia}
\begin{itemize}
\item {Grp. gram.:f.}
\end{itemize}
\begin{itemize}
\item {Utilização:Anthrop.}
\end{itemize}
\begin{itemize}
\item {Proveniência:(Do gr. \textunderscore para\textunderscore  + \textunderscore genesis\textunderscore )}
\end{itemize}
Homogenesía, em que os mestiços de primeiro sangue são estéreis até segunda geração, e em que os do segundo sangue são indefinidamente férteis. Cf. Ed. Burnay, \textunderscore Craníologia\textunderscore , 64.
\section{Paragenésico}
\begin{itemize}
\item {Grp. gram.:adj.}
\end{itemize}
\begin{itemize}
\item {Proveniência:(De \textunderscore paragenesia\textunderscore )}
\end{itemize}
Diz-se da homogenesia com posteridade.
\section{Parageusia}
\begin{itemize}
\item {Grp. gram.:f.}
\end{itemize}
\begin{itemize}
\item {Proveniência:(Do gr. \textunderscore para\textunderscore  + \textunderscore geusis\textunderscore )}
\end{itemize}
O mesmo ou melhor que \textunderscore pageustia\textunderscore . Cf. B. Galvão, \textunderscore Vocab.\textunderscore 
\section{Parageustia}
\begin{itemize}
\item {Grp. gram.:f.}
\end{itemize}
\begin{itemize}
\item {Utilização:Med.}
\end{itemize}
\begin{itemize}
\item {Proveniência:(Do gr. \textunderscore para\textunderscore  + \textunderscore geustos\textunderscore )}
\end{itemize}
Perversão do sentido do gôsto.
\section{Paraglobulina}
\begin{itemize}
\item {Grp. gram.:f.}
\end{itemize}
\begin{itemize}
\item {Utilização:Chím.}
\end{itemize}
Corpo composto, que se extrái dos glóbulos do sangue, do soro, etc.
\section{Paragoge}
\begin{itemize}
\item {Grp. gram.:f.}
\end{itemize}
\begin{itemize}
\item {Utilização:Gram.}
\end{itemize}
\begin{itemize}
\item {Proveniência:(Lat. \textunderscore paragoge\textunderscore )}
\end{itemize}
Addição de uma letra ou sýllaba no fim de uma palavra.
\section{Paragógico}
\begin{itemize}
\item {Grp. gram.:adj.}
\end{itemize}
Em que há paragoge; que se junta no fim de uma palavra.
\section{Paragonar}
\begin{itemize}
\item {Grp. gram.:v. t.}
\end{itemize}
\begin{itemize}
\item {Proveniência:(De \textunderscore paragão\textunderscore . Cp. it. \textunderscore paragonare\textunderscore )}
\end{itemize}
Comparar; assemelhar. Cf. Latino, \textunderscore Vasco da Gama\textunderscore , I, 61.
\section{Paragrafar}
\begin{itemize}
\item {Grp. gram.:v. t.}
\end{itemize}
Dividir em parágrafos.
\section{Parágrafo}
\begin{itemize}
\item {Grp. gram.:m.}
\end{itemize}
\begin{itemize}
\item {Proveniência:(Lat. \textunderscore paragraphus\textunderscore )}
\end{itemize}
Pequena secção de um discurso ou capítulo.
Sinal, que separa as secções de um discurso ou capítulo.
\section{Paragrama}
\begin{itemize}
\item {Grp. gram.:m.}
\end{itemize}
\begin{itemize}
\item {Utilização:Neol.}
\end{itemize}
\begin{itemize}
\item {Proveniência:(Lat. \textunderscore paragramma\textunderscore )}
\end{itemize}
Êrro de ortografia.
\section{Paragramatismo}
\begin{itemize}
\item {Grp. gram.:m.}
\end{itemize}
\begin{itemize}
\item {Proveniência:(De \textunderscore paragrama\textunderscore )}
\end{itemize}
O mesmo que \textunderscore aliteração\textunderscore .
\section{Paragramma}
\begin{itemize}
\item {Grp. gram.:m.}
\end{itemize}
\begin{itemize}
\item {Utilização:Neol.}
\end{itemize}
\begin{itemize}
\item {Proveniência:(Lat. \textunderscore paragramma\textunderscore )}
\end{itemize}
Êrro de orthographia.
\section{Paragrammatismo}
\begin{itemize}
\item {Grp. gram.:m.}
\end{itemize}
\begin{itemize}
\item {Proveniência:(De \textunderscore paragramma\textunderscore )}
\end{itemize}
O mesmo que \textunderscore aliteração\textunderscore .
\section{Paragraphar}
\begin{itemize}
\item {Grp. gram.:v. t.}
\end{itemize}
Dividir em parágraphos.
\section{Parágrapho}
\begin{itemize}
\item {Grp. gram.:m.}
\end{itemize}
\begin{itemize}
\item {Proveniência:(Lat. \textunderscore paragraphus\textunderscore )}
\end{itemize}
Pequena secção de um discurso ou capítulo.
Sinal, que separa as secções de um discurso ou capítulo.
\section{Paraguaiano}
\begin{itemize}
\item {Grp. gram.:adj.}
\end{itemize}
\begin{itemize}
\item {Grp. gram.:M.}
\end{itemize}
Relativo ao Paraguai.
Habitante do Paraguai.
\section{Paraguaio}
\begin{itemize}
\item {Grp. gram.:m.  e  adj.}
\end{itemize}
O mesmo que \textunderscore paraguaiano\textunderscore .
\section{Paraguás}
\begin{itemize}
\item {Grp. gram.:m. pl.}
\end{itemize}
Tríbo de índios do Brasil, nas margens do Paraguaçu.
\section{Paraguatan}
\begin{itemize}
\item {Grp. gram.:m.}
\end{itemize}
Árvore, cuja casca produz uma substância vermelha, tinctória, e cujas fôlhas dão um suco análogo á laca, (\textunderscore macrocnemum tínctorium\textunderscore , Wilden).
\section{Paraíba}
\begin{itemize}
\item {Grp. gram.:f.}
\end{itemize}
Árvore rutácea do Brasil.
\section{Paraibano}
\begin{itemize}
\item {Grp. gram.:adj.}
\end{itemize}
\begin{itemize}
\item {Grp. gram.:M.}
\end{itemize}
Relativo a Paraíba.
Habitante de Paraíba.
\section{Paraíso}
\begin{itemize}
\item {Grp. gram.:m.}
\end{itemize}
\begin{itemize}
\item {Utilização:Fam.}
\end{itemize}
\begin{itemize}
\item {Utilização:Bras}
\end{itemize}
\begin{itemize}
\item {Utilização:T. de Lisbôa. Gal}
\end{itemize}
\begin{itemize}
\item {Proveniência:(Do lat. \textunderscore paradisus\textunderscore )}
\end{itemize}
Grande jardim, entre os antigos Persas.
Éden, lugar de delícias, em que Deus collocou Adão e Eva.
Céu, bem-aventurança.
Lugar aprazível.
Gênero de árvores copadas.
\textunderscore Ave do paraíso\textunderscore , pássaro conirostro, notável pela belleza das pennas.
Conjuncto de logares que há, para espectadores, por cima dos camarotes e junto ao tecto, nalguns theatros.
\section{Paral}
\begin{itemize}
\item {Grp. gram.:m.}
\end{itemize}
\begin{itemize}
\item {Utilização:Mad}
\end{itemize}
Cada uma das travessas de madeira, pregadas na parte inferior do carrinho-do-monte, e a cujos topos deanteiros se atam as cordas ou correias, com que os conductores arrastam aquelle vehículo.
\section{Paraláctico}
\begin{itemize}
\item {Grp. gram.:adj.}
\end{itemize}
Relativo á paralaxe.
\section{Paralalia}
\begin{itemize}
\item {Grp. gram.:f.}
\end{itemize}
\begin{itemize}
\item {Utilização:Med.}
\end{itemize}
\begin{itemize}
\item {Proveniência:(Do gr. \textunderscore para\textunderscore  + \textunderscore lalein\textunderscore )}
\end{itemize}
Defeito da voz.
Desapparecimento, temporário ou permanente, da faculdade de falar.
\section{Paralampsia}
\begin{itemize}
\item {Grp. gram.:f.}
\end{itemize}
\begin{itemize}
\item {Utilização:Med.}
\end{itemize}
\begin{itemize}
\item {Proveniência:(Gr. \textunderscore paralampsis\textunderscore )}
\end{itemize}
Mancha branca na córnea.
\section{Paralaxe}
\begin{itemize}
\item {fónica:cse}
\end{itemize}
\begin{itemize}
\item {Grp. gram.:f.}
\end{itemize}
\begin{itemize}
\item {Proveniência:(Gr. \textunderscore parallaxis\textunderscore )}
\end{itemize}
Ângulo, formado de duas rectas, uma das quaes se dirige ao centro da terra e a outra ao ponto em que se acha um observador.
\section{Paralbumina}
\begin{itemize}
\item {Grp. gram.:f.}
\end{itemize}
Matéria azotada, que se encontra no líquido dos cystos do ovário, ao mesmo tempo que a albumina propriamente dita.
(Cast. e it. \textunderscore paralbumina\textunderscore )
\section{Paraldehydo}
\begin{itemize}
\item {Grp. gram.:m.}
\end{itemize}
Producto pharmacêutico, com propriedades hypnóticas.
\section{Paraldeído}
\begin{itemize}
\item {Grp. gram.:m.}
\end{itemize}
Producto farmacêutico, com propriedades hipnóticas.
\section{Paralela}
\begin{itemize}
\item {Grp. gram.:f.}
\end{itemize}
\begin{itemize}
\item {Grp. gram.:Pl.}
\end{itemize}
\begin{itemize}
\item {Utilização:Gymn.}
\end{itemize}
Linha ou superfície, equidistante de outra em toda a sua extensão.
Aparelho, formado de duas barras, que se mantêm firmes a pequena e igual distância.
(Fem. de \textunderscore parallelo\textunderscore )
\section{Paralelamente}
\begin{itemize}
\item {Grp. gram.:adv.}
\end{itemize}
De modo paralelo.
\section{Paralelepipedal}
\begin{itemize}
\item {Grp. gram.:adj.}
\end{itemize}
Relativo a paralelepípedo; que tem fórma de paralelepípedo. Cf. \textunderscore Museu Techn.\textunderscore , 52.
\section{Paralelepipédico}
\begin{itemize}
\item {Grp. gram.:adj.}
\end{itemize}
Relativo ou semelhante ao paralelepípedo; paralelepipedal.
\section{Paralelepípedo}
\begin{itemize}
\item {Grp. gram.:m.}
\end{itemize}
\begin{itemize}
\item {Proveniência:(Lat. \textunderscore parallelepipedum\textunderscore )}
\end{itemize}
Sólido geométrico, terminado por seis paralelogramos, dos quaes os opostos são iguaes e paralelos.
\section{Paralélico}
\begin{itemize}
\item {Grp. gram.:adj.}
\end{itemize}
\begin{itemize}
\item {Utilização:Bot.}
\end{itemize}
\begin{itemize}
\item {Proveniência:(De \textunderscore paralelo\textunderscore )}
\end{itemize}
Diz-se do repartimento placentário, quando se alarga paralelamente ao plano das válvulas, de um pericarpo bivalvular.
\section{Paralelígeros}
\begin{itemize}
\item {Grp. gram.:m. pl.}
\end{itemize}
\begin{itemize}
\item {Proveniência:(Do lat. \textunderscore parallelus\textunderscore  + \textunderscore gerere\textunderscore )}
\end{itemize}
Casta de aranhas, que têm os olhos sôbre duas linhas paralelas.
\section{Paralelinérveo}
\begin{itemize}
\item {Grp. gram.:adj.}
\end{itemize}
\begin{itemize}
\item {Utilização:Bot.}
\end{itemize}
\begin{itemize}
\item {Proveniência:(De \textunderscore paralelo\textunderscore  + \textunderscore nervo\textunderscore )}
\end{itemize}
Diz-se das fôlhas, que têm as nervuras paralelas entre si.
\section{Paralelismo}
\begin{itemize}
\item {Grp. gram.:m.}
\end{itemize}
\begin{itemize}
\item {Utilização:Fig.}
\end{itemize}
\begin{itemize}
\item {Proveniência:(De \textunderscore paralelo\textunderscore )}
\end{itemize}
Estado de duas linhas ou superfícies paralelas.
Correspondência entre objectos, comparada ao paralelismo das linhas.
\section{Paralelístico}
\begin{itemize}
\item {Grp. gram.:adj.}
\end{itemize}
Diz-se das composições poéticas de G. Vicente, da Vaticana, etc., em que há paralelismo de estrofes, e em que a ideia se repete com mais ou menos variedade.
(Cp. \textunderscore paralelismo\textunderscore )
\section{Paralelivenoso}
\begin{itemize}
\item {Grp. gram.:adj.}
\end{itemize}
\begin{itemize}
\item {Utilização:Bot.}
\end{itemize}
Diz-se das fôlhas, quando os seus veios são aparentes e paralelos entre si.
\section{Paralelizar}
\begin{itemize}
\item {Grp. gram.:v. t.}
\end{itemize}
\begin{itemize}
\item {Utilização:Neol.}
\end{itemize}
Tornar paralelo.
\section{Paralelizável}
\begin{itemize}
\item {Grp. gram.:adj.}
\end{itemize}
\begin{itemize}
\item {Utilização:Neol.}
\end{itemize}
Que se póde paralelizar. Cf. R. Jorge, \textunderscore Sousa Martins\textunderscore , 13.
\section{Paralelo}
\begin{itemize}
\item {Grp. gram.:adj.}
\end{itemize}
\begin{itemize}
\item {Utilização:Fig.}
\end{itemize}
\begin{itemize}
\item {Grp. gram.:M.}
\end{itemize}
\begin{itemize}
\item {Utilização:Astron.}
\end{itemize}
\begin{itemize}
\item {Utilização:Fig.}
\end{itemize}
\begin{itemize}
\item {Proveniência:(Lat. \textunderscore parallelus\textunderscore )}
\end{itemize}
Diz-se de duas linhas ou superfícies equidistantes uma da outra, em toda a sua extensão.
Que marcha a par de outro ou progride na mesma proporção.
Análogo.
Cada um dos círculos menores, perpendiculares ao meridiano.
Confronto.
\section{Paralelogrâmico}
\begin{itemize}
\item {Grp. gram.:adj.}
\end{itemize}
Que tem fórma de paralelogramo.
\section{Paralelogramo}
\begin{itemize}
\item {Grp. gram.:m.}
\end{itemize}
\begin{itemize}
\item {Proveniência:(Lat. \textunderscore parallelogrammus\textunderscore )}
\end{itemize}
Quadrilátero, cujos lados opostos são iguaes e paralelos.
\section{Paralexia}
\begin{itemize}
\item {fónica:csi}
\end{itemize}
\begin{itemize}
\item {Grp. gram.:f.}
\end{itemize}
\begin{itemize}
\item {Utilização:Med.}
\end{itemize}
\begin{itemize}
\item {Proveniência:(Do gr. \textunderscore para\textunderscore  + \textunderscore lexis\textunderscore )}
\end{itemize}
Confusão, feita pelo doente em palavras escritas, trocando umas por outras, trocando letras, etc.
\section{Paralheiro}
\begin{itemize}
\item {Grp. gram.:m.}
\end{itemize}
Vasilha, em que se baldeia o melaço, na fabricação do açúcar.
\section{Parálio}
\begin{itemize}
\item {Grp. gram.:adj.}
\end{itemize}
\begin{itemize}
\item {Proveniência:(Lat. paralius)}
\end{itemize}
Próximo do mar; marítimo.
\section{Paralipómenos}
\begin{itemize}
\item {Grp. gram.:m. pl.}
\end{itemize}
\begin{itemize}
\item {Utilização:Fig.}
\end{itemize}
\begin{itemize}
\item {Proveniência:(Lat. \textunderscore paralipomena\textunderscore )}
\end{itemize}
Parte da \textunderscore Biblia\textunderscore , em supplemento ao livro dos Reis.
Supplemento a qualquer obra literária.
\section{Paralipse}
\begin{itemize}
\item {Grp. gram.:f.}
\end{itemize}
\begin{itemize}
\item {Proveniência:(Gr. \textunderscore paralipsis\textunderscore )}
\end{itemize}
Figura de Rhetórica, pela qual se fixa a attenção num objecto, fingindo desviá-la.
\section{Paralisar}
\begin{itemize}
\item {Grp. gram.:v. t.}
\end{itemize}
\begin{itemize}
\item {Utilização:Fig.}
\end{itemize}
\begin{itemize}
\item {Grp. gram.:V. i.}
\end{itemize}
\begin{itemize}
\item {Proveniência:(De \textunderscore paralisia\textunderscore )}
\end{itemize}
Tornar paralítico.
Tornar inerte, neutralizar.
Tornar estacionário.
Sofrer paralisia.
Não progredir; tornar-se estacionário: \textunderscore o commércio paralisa\textunderscore .
\section{Parálise}
\begin{itemize}
\item {Grp. gram.:f.}
\end{itemize}
\begin{itemize}
\item {Proveniência:(Lat. \textunderscore paralysis\textunderscore )}
\end{itemize}
O mesmo que \textunderscore paralisia\textunderscore .
\section{Paralisia}
\begin{itemize}
\item {Grp. gram.:f.}
\end{itemize}
\begin{itemize}
\item {Utilização:Fig.}
\end{itemize}
\begin{itemize}
\item {Proveniência:(Do gr. \textunderscore paralusis\textunderscore )}
\end{itemize}
Privação ou deminuição de sensibilidade e movimento voluntário, ou só de uma destas coisas.
Entorpecimento, marasmo.
\section{Paraliticar}
\begin{itemize}
\item {Grp. gram.:v. t.}
\end{itemize}
\begin{itemize}
\item {Utilização:Des.}
\end{itemize}
\begin{itemize}
\item {Grp. gram.:V. i.}
\end{itemize}
\begin{itemize}
\item {Utilização:Des.}
\end{itemize}
Tornar paralítico.
Tornar-se paralítico.
\section{Paralítico}
\begin{itemize}
\item {Grp. gram.:m.  e  adj.}
\end{itemize}
\begin{itemize}
\item {Proveniência:(Lat. \textunderscore paralyticus\textunderscore )}
\end{itemize}
O que teve ou tem paralisia.
\section{Paralláctico}
\begin{itemize}
\item {Grp. gram.:adj.}
\end{itemize}
Relativo á parallaxe.
\section{Parallaxe}
\begin{itemize}
\item {fónica:cse}
\end{itemize}
\begin{itemize}
\item {Grp. gram.:f.}
\end{itemize}
\begin{itemize}
\item {Proveniência:(Gr. \textunderscore parallaxis\textunderscore )}
\end{itemize}
Ângulo, formado de duas rectas, uma das quaes se dirige ao centro da terra e a outra ao ponto em que se acha um observador.
\section{Parallela}
\begin{itemize}
\item {Grp. gram.:f.}
\end{itemize}
\begin{itemize}
\item {Grp. gram.:Pl.}
\end{itemize}
\begin{itemize}
\item {Utilização:Gymn.}
\end{itemize}
Linha ou superfície, equidistante de outra em toda a sua extensão.
Apparelho, formado de duas barras, que se mantêm firmes a pequena e igual distância.
(Fem. de \textunderscore parallelo\textunderscore )
\section{Parallelamente}
\begin{itemize}
\item {Grp. gram.:adv.}
\end{itemize}
De modo parallelo.
\section{Parallelepipedal}
\begin{itemize}
\item {Grp. gram.:adj.}
\end{itemize}
Relativo a parallelepípedo; que tem fórma de parallelepípedo. Cf. \textunderscore Museu Techn.\textunderscore , 52.
\section{Parallelepipédico}
\begin{itemize}
\item {Grp. gram.:adj.}
\end{itemize}
Relativo ou semelhante ao parallelepípedo; parallelepipedal.
\section{Parallelepípedo}
\begin{itemize}
\item {Grp. gram.:m.}
\end{itemize}
\begin{itemize}
\item {Proveniência:(Lat. \textunderscore parallelepipedum\textunderscore )}
\end{itemize}
Sólido geométrico, terminado por seis parallelogrammos, dos quaes os oppostos são iguaes e parallelos.
\section{Parallélico}
\begin{itemize}
\item {Grp. gram.:adj.}
\end{itemize}
\begin{itemize}
\item {Utilização:Bot.}
\end{itemize}
\begin{itemize}
\item {Proveniência:(De \textunderscore parallelo\textunderscore )}
\end{itemize}
Diz-se do repartimento placentário, quando se alarga parallelamente ao plano das válvulas, de um pericarpo bivalvular.
\section{Parallelígeros}
\begin{itemize}
\item {Grp. gram.:m. pl.}
\end{itemize}
\begin{itemize}
\item {Proveniência:(Do lat. \textunderscore parallelus\textunderscore  + \textunderscore gerere\textunderscore )}
\end{itemize}
Casta de aranhas, que têm os olhos sôbre duas linhas parallelas.
\section{Parallelinérveo}
\begin{itemize}
\item {Grp. gram.:adj.}
\end{itemize}
\begin{itemize}
\item {Utilização:Bot.}
\end{itemize}
\begin{itemize}
\item {Proveniência:(De \textunderscore parallelo\textunderscore  + \textunderscore nervo\textunderscore )}
\end{itemize}
Diz-se das fôlhas, que têm as nervuras parallelas entre si.
\section{Parallelismo}
\begin{itemize}
\item {Grp. gram.:m.}
\end{itemize}
\begin{itemize}
\item {Utilização:Fig.}
\end{itemize}
\begin{itemize}
\item {Proveniência:(De \textunderscore parallelo\textunderscore )}
\end{itemize}
Estado de duas linhas ou superfícies parallelas.
Correspondência entre objectos, comparada ao parallelismo das linhas.
\section{Parallelístico}
\begin{itemize}
\item {Grp. gram.:adj.}
\end{itemize}
Diz-se das composições poéticas de G. Vicente, da Vaticana, etc., em que há parallelismo de estrophes, e em que a ideia se repete com mais ou menos variedade.
(Cp. \textunderscore parallelismo\textunderscore )
\section{Parallelivenoso}
\begin{itemize}
\item {Grp. gram.:adj.}
\end{itemize}
\begin{itemize}
\item {Utilização:Bot.}
\end{itemize}
Diz-se das fôlhas, quando os seus veios são apparentes e parallelos entre si.
\section{Parallelizar}
\begin{itemize}
\item {Grp. gram.:v. t.}
\end{itemize}
\begin{itemize}
\item {Utilização:Neol.}
\end{itemize}
Tornar parallelo.
\section{Parallelizável}
\begin{itemize}
\item {Grp. gram.:adj.}
\end{itemize}
\begin{itemize}
\item {Utilização:Neol.}
\end{itemize}
Que se póde parallelizar. Cf. R. Jorge, \textunderscore Sousa Martins\textunderscore , 13.
\section{Parallelo}
\begin{itemize}
\item {Grp. gram.:adj.}
\end{itemize}
\begin{itemize}
\item {Utilização:Fig.}
\end{itemize}
\begin{itemize}
\item {Grp. gram.:M.}
\end{itemize}
\begin{itemize}
\item {Utilização:Astron.}
\end{itemize}
\begin{itemize}
\item {Utilização:Fig.}
\end{itemize}
\begin{itemize}
\item {Proveniência:(Lat. \textunderscore parallelus\textunderscore )}
\end{itemize}
Diz-se de duas linhas ou superfícies equidistantes uma da outra, em toda a sua extensão.
Que marcha a par de outro ou progride na mesma proporção.
Análogo.
Cada um dos círculos menores, perpendiculares ao meridiano.
Confronto.
\section{Parallelogrâmmico}
\begin{itemize}
\item {Grp. gram.:adj.}
\end{itemize}
Que tem fórma de parallelogrammo.
\section{Parallelogrammo}
\begin{itemize}
\item {Grp. gram.:m.}
\end{itemize}
\begin{itemize}
\item {Proveniência:(Lat. \textunderscore parallelogrammus\textunderscore )}
\end{itemize}
Quadrilátero, cujos lados oppostos são iguaes e parallelos.
\section{Páralo}
\begin{itemize}
\item {Grp. gram.:m.}
\end{itemize}
\begin{itemize}
\item {Proveniência:(Lat. \textunderscore paralos\textunderscore )}
\end{itemize}
Navio sagrado dos Athenienses, que só se empregava em serviço da religião ou do Estado.
\section{Paralogia}
\begin{itemize}
\item {Grp. gram.:f.}
\end{itemize}
\begin{itemize}
\item {Utilização:Med.}
\end{itemize}
\begin{itemize}
\item {Proveniência:(Do gr. \textunderscore para\textunderscore  + \textunderscore logos\textunderscore )}
\end{itemize}
Confusão na palavra, por demora do pensamento.
\section{Paralogismo}
\begin{itemize}
\item {Grp. gram.:m.}
\end{itemize}
\begin{itemize}
\item {Proveniência:(Gr. \textunderscore paralogismos\textunderscore )}
\end{itemize}
Falso raciocínio.
\section{Paralta}
\begin{itemize}
\item {Grp. gram.:m.  e  f.}
\end{itemize}
O mesmo que \textunderscore peralta\textunderscore .
\section{Paraltice}
\begin{itemize}
\item {Grp. gram.:f.}
\end{itemize}
O mesmo que \textunderscore peraltice\textunderscore .
\section{Pára-luz}
\begin{itemize}
\item {Grp. gram.:m.}
\end{itemize}
Quebra-luz, pantalha.
Qualquer peça, com que se intercepta ou modifica a acção da luz de um candeeiro, de uma vela, etc.
\section{Paralvilho}
\begin{itemize}
\item {Grp. gram.:m.}
\end{itemize}
O mesmo que \textunderscore peralvilho\textunderscore .
\section{Paralysar}
\begin{itemize}
\item {Grp. gram.:v. t.}
\end{itemize}
\begin{itemize}
\item {Utilização:Fig.}
\end{itemize}
\begin{itemize}
\item {Grp. gram.:V. i.}
\end{itemize}
\begin{itemize}
\item {Proveniência:(De \textunderscore paralysia\textunderscore )}
\end{itemize}
Tornar paralýtico.
Tornar inerte, neutralizar.
Tornar estacionário.
Soffrer paralysia.
Não progredir; tornar-se estacionário: \textunderscore o commércio paralysa\textunderscore .
\section{Parályse}
\begin{itemize}
\item {Grp. gram.:f.}
\end{itemize}
\begin{itemize}
\item {Proveniência:(Lat. \textunderscore paralysis\textunderscore )}
\end{itemize}
O mesmo que \textunderscore paralysia\textunderscore .
\section{Paralysia}
\begin{itemize}
\item {Grp. gram.:f.}
\end{itemize}
\begin{itemize}
\item {Utilização:Fig.}
\end{itemize}
\begin{itemize}
\item {Proveniência:(Do gr. \textunderscore paralusis\textunderscore )}
\end{itemize}
Privação ou deminuição de sensibilidade e movimento voluntário, ou só de uma destas coisas.
Entorpecimento, marasmo.
\section{Paralyticar}
\begin{itemize}
\item {Grp. gram.:v. t.}
\end{itemize}
\begin{itemize}
\item {Utilização:Des.}
\end{itemize}
\begin{itemize}
\item {Grp. gram.:V. i.}
\end{itemize}
\begin{itemize}
\item {Utilização:Des.}
\end{itemize}
Tornar paralýtico.
Tornar-se paralýtico.
\section{Paralýtico}
\begin{itemize}
\item {Grp. gram.:m.  e  adj.}
\end{itemize}
\begin{itemize}
\item {Proveniência:(Lat. \textunderscore paralyticus\textunderscore )}
\end{itemize}
O que teve ou tem paralysia.
\section{Paramagnético}
\begin{itemize}
\item {Grp. gram.:adj.}
\end{itemize}
Que tem a direcção dada pelo paramagnetismo.
\section{Paramagnetismo}
\begin{itemize}
\item {Grp. gram.:m.}
\end{itemize}
\begin{itemize}
\item {Utilização:Phýs.}
\end{itemize}
\begin{itemize}
\item {Proveniência:(De \textunderscore parallelo\textunderscore  + \textunderscore magnetismo\textunderscore )}
\end{itemize}
Propriedade, que o magnetismo tem, de dar aos corpos direcção parallela á linha dos polos, quando êsses corpos estão collocados entre os dois polos de um potente electroíman, recurvado em fórma de ferradura.
\section{Paramaleato}
\begin{itemize}
\item {Grp. gram.:m.}
\end{itemize}
Designação genérica dos saes do ácido paramaleico.
\section{Paramaleico}
\begin{itemize}
\item {Grp. gram.:adj.}
\end{itemize}
\begin{itemize}
\item {Proveniência:(De \textunderscore para...\textunderscore  + \textunderscore málico\textunderscore )}
\end{itemize}
Diz-se do um ácido, produzido pela destillação sêca do ácido málico.
\section{Parambelo}
\begin{itemize}
\item {Grp. gram.:m.}
\end{itemize}
\begin{itemize}
\item {Utilização:Prov.}
\end{itemize}
Casa velha e arruinada.
\section{Paramécia}
\begin{itemize}
\item {Grp. gram.:f.}
\end{itemize}
O mesmo que \textunderscore paramécio\textunderscore .
\section{Paramécio}
\begin{itemize}
\item {Grp. gram.:m.}
\end{itemize}
\begin{itemize}
\item {Proveniência:(Do gr. \textunderscore paramekes\textunderscore )}
\end{itemize}
Infusório chato e oblongo.
\section{Paramecónico}
\begin{itemize}
\item {Grp. gram.:adj.}
\end{itemize}
\begin{itemize}
\item {Proveniência:(De \textunderscore para...\textunderscore  + \textunderscore mecónico\textunderscore )}
\end{itemize}
Diz-se um ácido, que se obtém, fazendo ferver em água o ácido mecónico.
\section{Paramenispermina}
\begin{itemize}
\item {Grp. gram.:f.}
\end{itemize}
\begin{itemize}
\item {Utilização:Chím.}
\end{itemize}
\begin{itemize}
\item {Proveniência:(De \textunderscore para...\textunderscore  + \textunderscore menispermina\textunderscore )}
\end{itemize}
Substância orgânica, que tem a mesma composição que a menispermina e que se encontra com esta na coca do Levante.
\section{Paramentar}
\begin{itemize}
\item {Grp. gram.:v. t.}
\end{itemize}
\begin{itemize}
\item {Utilização:Constr.}
\end{itemize}
Cobrir com paramentos.
Adornar; enfeitar.
Apparelhar, para encorporar em paramento.
\section{Paramenteira}
\begin{itemize}
\item {Grp. gram.:f.}
\end{itemize}
Mulhér, que trabalha em paramentos ecclesiásticos.
\section{Paramenteiro}
\begin{itemize}
\item {Grp. gram.:m.}
\end{itemize}
Alfaiate de paramentos ecclesiásticos.
\section{Pàramentes!}
\begin{itemize}
\item {Grp. gram.:interj.}
\end{itemize}
\begin{itemize}
\item {Utilização:Ant.}
\end{itemize}
\begin{itemize}
\item {Proveniência:(De \textunderscore parar\textunderscore  + \textunderscore mente\textunderscore )}
\end{itemize}
Repara bem! cuidado!
\section{Paramento}
\begin{itemize}
\item {Grp. gram.:m.}
\end{itemize}
\begin{itemize}
\item {Utilização:Constr.}
\end{itemize}
\begin{itemize}
\item {Utilização:Ant.}
\end{itemize}
\begin{itemize}
\item {Proveniência:(Lat. \textunderscore paramentum\textunderscore )}
\end{itemize}
Peça de ornato.
Peça de vestuário, empregada em ceremónias de igreja.
Alfaia ornamental, em igreja ou capella.
Superfície polida de uma pedra ou peça de madeira, próprias para construcção.
Face anterior ou posterior de uma parede.
Modo de vida, comportamento; acção.
\section{Paramérico}
\begin{itemize}
\item {Grp. gram.:adj.}
\end{itemize}
Relativo ao parâmero.
\section{Paramério}
\begin{itemize}
\item {Grp. gram.:m.}
\end{itemize}
O mesmo ou melhor que \textunderscore parâmero\textunderscore .
\section{Parâmero}
\begin{itemize}
\item {Grp. gram.:m.}
\end{itemize}
\begin{itemize}
\item {Utilização:Anat.}
\end{itemize}
\begin{itemize}
\item {Proveniência:(Do gr. \textunderscore para\textunderscore  + \textunderscore meros\textunderscore )}
\end{itemize}
Cada uma das partes do organismo, separadas pelos paracoronaes.
\section{Parâmese}
\begin{itemize}
\item {Grp. gram.:m.}
\end{itemize}
\begin{itemize}
\item {Proveniência:(Lat. \textunderscore paramese\textunderscore )}
\end{itemize}
A quinta corda da lyra.
\section{Paramétrico}
\begin{itemize}
\item {Grp. gram.:adj.}
\end{itemize}
Relativo ao parâmetro.
\section{Parametrite}
\begin{itemize}
\item {Grp. gram.:f.}
\end{itemize}
\begin{itemize}
\item {Proveniência:(Do gr. \textunderscore para\textunderscore  + \textunderscore metra\textunderscore )}
\end{itemize}
Inflammação do tecido, que envolve o útero.
\section{Parâmetro}
\begin{itemize}
\item {Grp. gram.:m.}
\end{itemize}
\begin{itemize}
\item {Utilização:Mathem.}
\end{itemize}
\begin{itemize}
\item {Proveniência:(Do gr. \textunderscore para\textunderscore  + \textunderscore metron\textunderscore )}
\end{itemize}
Linha constante e invariável, que entra na equação ou construcção de uma curva.
\section{Paramimia}
\begin{itemize}
\item {Grp. gram.:f.}
\end{itemize}
\begin{itemize}
\item {Utilização:Med.}
\end{itemize}
\begin{itemize}
\item {Proveniência:(Do gr. \textunderscore para\textunderscore  + \textunderscore mimoumai\textunderscore )}
\end{itemize}
Confusão mórbida na execução dos gestos.
\section{Paramnesía}
\begin{itemize}
\item {Grp. gram.:f.}
\end{itemize}
\begin{itemize}
\item {Utilização:Med.}
\end{itemize}
\begin{itemize}
\item {Proveniência:(Do gr. \textunderscore para\textunderscore  + \textunderscore mnesis\textunderscore )}
\end{itemize}
Perturbação na faculdade de falar, caracterizada pelo esquecimento da significação das palavras que se ouvem, com suggestão instintiva dos sons ainda conhecidos mas mal applicados.
Perturbação da memória.
\section{Páramo}
\begin{itemize}
\item {Grp. gram.:m.}
\end{itemize}
\begin{itemize}
\item {Utilização:Ext.}
\end{itemize}
Planície deserta.
Firmamento.
(Cast. \textunderscore páramo\textunderscore )
\section{Parâmo}
\begin{itemize}
\item {Grp. gram.:m.}
\end{itemize}
\begin{itemize}
\item {Utilização:Ant.}
\end{itemize}
O mesmo que \textunderscore paranho\textunderscore .
\section{Paramorfina}
\begin{itemize}
\item {Grp. gram.:f.}
\end{itemize}
\begin{itemize}
\item {Proveniência:(De \textunderscore para...\textunderscore  + \textunderscore morfina\textunderscore )}
\end{itemize}
Substância vizinha da morfina e que se encontra no ópio.
\section{Paramorfismo}
\begin{itemize}
\item {Grp. gram.:m.}
\end{itemize}
\begin{itemize}
\item {Utilização:Chím.}
\end{itemize}
\begin{itemize}
\item {Proveniência:(Do gr. \textunderscore para\textunderscore  + \textunderscore morphe\textunderscore )}
\end{itemize}
Relação entre substâncias chimicamente análogas, e cujas fórmas cristalinas são muito próximas umas das outras.
\section{Paramorphina}
\begin{itemize}
\item {Grp. gram.:f.}
\end{itemize}
\begin{itemize}
\item {Proveniência:(De \textunderscore para...\textunderscore  + \textunderscore morphina\textunderscore )}
\end{itemize}
Substância vizinha da morphina e que se encontra no ópio.
\section{Paramorphismo}
\begin{itemize}
\item {Grp. gram.:m.}
\end{itemize}
\begin{itemize}
\item {Utilização:Chím.}
\end{itemize}
\begin{itemize}
\item {Proveniência:(Do gr. \textunderscore para\textunderscore  + \textunderscore morphe\textunderscore )}
\end{itemize}
Relação entre substâncias chimicamente análogas, e cujas fórmas crystallinas são muito próximas umas das outras.
\section{Paramúcico}
\begin{itemize}
\item {Grp. gram.:adj.}
\end{itemize}
\begin{itemize}
\item {Proveniência:(De \textunderscore para...\textunderscore  + \textunderscore mucico\textunderscore )}
\end{itemize}
Diz-se de um ácido, que se obtém pela ebullição prolongada do ácido múcico.
\section{Pára-muro}
\begin{itemize}
\item {Grp. gram.:m.}
\end{itemize}
Antiga peça de artilharia. Cf. \textunderscore Livro das Monções\textunderscore , 13.
\section{Paramusia}
\begin{itemize}
\item {Grp. gram.:f.}
\end{itemize}
\begin{itemize}
\item {Utilização:Med.}
\end{itemize}
\begin{itemize}
\item {Proveniência:(Do gr. \textunderscore para\textunderscore  + \textunderscore mousa\textunderscore )}
\end{itemize}
Perturbação mórbida da faculdade musical.
\section{Parafasia}
\begin{itemize}
\item {Grp. gram.:f.}
\end{itemize}
\begin{itemize}
\item {Utilização:Med.}
\end{itemize}
\begin{itemize}
\item {Proveniência:(Do gr. \textunderscore para\textunderscore  + \textunderscore phasis\textunderscore )}
\end{itemize}
Perturbação da linguagem, em que as palavras não correspondem ás ideias.
\section{Parafernais}
\begin{itemize}
\item {Grp. gram.:f. pl.  e  adj.}
\end{itemize}
\begin{itemize}
\item {Proveniência:(Do gr. \textunderscore para\textunderscore  + \textunderscore pherne\textunderscore )}
\end{itemize}
Bens, que não são dotaes, mas que a mulher póde gozar ou administrar independentemente.
\section{Parafimose}
\begin{itemize}
\item {Grp. gram.:f.}
\end{itemize}
\begin{itemize}
\item {Utilização:Med.}
\end{itemize}
\begin{itemize}
\item {Proveniência:(Do gr. \textunderscore para\textunderscore  + \textunderscore phimos\textunderscore )}
\end{itemize}
Inflamação e desviramento do prepúcio, por fórma que não póde recobrir a glande.
\section{Paráfise}
\begin{itemize}
\item {Grp. gram.:f.}
\end{itemize}
\begin{itemize}
\item {Utilização:Bot.}
\end{itemize}
\begin{itemize}
\item {Proveniência:(Gr. \textunderscore paraphusis\textunderscore )}
\end{itemize}
Cada um dos filetes, que acompanham os órgãos da fructificação dos musgos.
\section{Parafonia}
\begin{itemize}
\item {Grp. gram.:f.}
\end{itemize}
\begin{itemize}
\item {Proveniência:(Do gr. \textunderscore para\textunderscore  + \textunderscore phone\textunderscore )}
\end{itemize}
Defeito da voz, consistindo num timbre desagradável.
\section{Parafonista}
\begin{itemize}
\item {Grp. gram.:f.}
\end{itemize}
Nome, que se dava a cada um dos dois primeiros cantores, num côro de cantochão.
(Cp. \textunderscore parafonia\textunderscore )
\section{Parafosfato}
\begin{itemize}
\item {Grp. gram.:m.}
\end{itemize}
Gênero de saes, produzidos pelo ácido parafosfórico.
\section{Parafosfórico}
\begin{itemize}
\item {Grp. gram.:adj.}
\end{itemize}
\begin{itemize}
\item {Proveniência:(De \textunderscore para...\textunderscore  + \textunderscore fosfórico\textunderscore )}
\end{itemize}
Diz-se do ácido fosfórico, quando, sujeito a grande calor, adquire propriedades que não tinha.
\section{Paráfrase}
\begin{itemize}
\item {Grp. gram.:f.}
\end{itemize}
\begin{itemize}
\item {Proveniência:(Lat. \textunderscore paraphrasis\textunderscore )}
\end{itemize}
Desenvolvimento do texto de um livro ou documento.
Traducção livre ou desenvolvida.
\section{Parafrasear}
\begin{itemize}
\item {Grp. gram.:v. t.}
\end{itemize}
\begin{itemize}
\item {Proveniência:(De \textunderscore paráfrase\textunderscore )}
\end{itemize}
Explicar, desenvolvendo.
Traduzir livremente.
Desenvolver por palavra ou por escrito.
\section{Parafrasia}
\begin{itemize}
\item {Grp. gram.:f.}
\end{itemize}
\begin{itemize}
\item {Utilização:Med.}
\end{itemize}
\begin{itemize}
\item {Proveniência:(Do gr. \textunderscore para\textunderscore  + \textunderscore phrasis\textunderscore )}
\end{itemize}
Perturbação da palavra.
\section{Parafrasta}
\begin{itemize}
\item {Grp. gram.:m.}
\end{itemize}
\begin{itemize}
\item {Proveniência:(Lat. \textunderscore paraphrastes\textunderscore )}
\end{itemize}
Autor de paráfrases.
\section{Parafraste}
\begin{itemize}
\item {Grp. gram.:m.}
\end{itemize}
(V.parafrasta)
\section{Parafrásticamente}
\begin{itemize}
\item {Grp. gram.:adv.}
\end{itemize}
De modo parafrástico.
\section{Parafrástico}
\begin{itemize}
\item {Grp. gram.:adj.}
\end{itemize}
Relativo a paráfrase: \textunderscore tradução parafrástica\textunderscore .
\section{Parafrósine}
\begin{itemize}
\item {Grp. gram.:f.}
\end{itemize}
\begin{itemize}
\item {Utilização:Med.}
\end{itemize}
\begin{itemize}
\item {Proveniência:(Gr. \textunderscore paraphrosune\textunderscore )}
\end{itemize}
Delírio febril.
\section{Paranaçu}
\begin{itemize}
\item {Grp. gram.:m.}
\end{itemize}
Espécie de macaco do Brasil.
\section{Paranaense}
\begin{itemize}
\item {Grp. gram.:adj.}
\end{itemize}
\begin{itemize}
\item {Grp. gram.:M.  e  f.}
\end{itemize}
Relativo á antiga província brasileira, hoje Estado, do Paraná.
Habitante dessa província.
\section{Paranaftalina}
\begin{itemize}
\item {Grp. gram.:f.}
\end{itemize}
Substância vizinha da naftalina.
\section{Paranàmirim}
\begin{itemize}
\item {Grp. gram.:m.}
\end{itemize}
\begin{itemize}
\item {Utilização:Bras. do N}
\end{itemize}
\begin{itemize}
\item {Proveniência:(Do guar. \textunderscore paraná-miri\textunderscore )}
\end{itemize}
Pequeno rio; braço de rio; canal.
\section{Paranaphthalina}
\begin{itemize}
\item {Grp. gram.:f.}
\end{itemize}
Substância vizinha da naphtalina.
\section{Paranatello}
\begin{itemize}
\item {Grp. gram.:m.}
\end{itemize}
\begin{itemize}
\item {Proveniência:(Lat. \textunderscore paranatellon\textunderscore )}
\end{itemize}
Conjunto dos astros que se mostram no horizonte, quando o Sol entra num signo do zodíaco, segundo a Astronomia dos Egýpcios.
\section{Paranatellôntico}
\begin{itemize}
\item {Grp. gram.:adj.}
\end{itemize}
Relativo ao paranatello; dependente do paranatello.
\section{Paranatelo}
\begin{itemize}
\item {Grp. gram.:m.}
\end{itemize}
\begin{itemize}
\item {Proveniência:(Lat. \textunderscore paranatellon\textunderscore )}
\end{itemize}
Conjunto dos astros que se mostram no horizonte, quando o Sol entra num signo do zodíaco, segundo a Astronomia dos Egípcios.
\section{Paranatelôntico}
\begin{itemize}
\item {Grp. gram.:adj.}
\end{itemize}
Relativo ao paranatelo; dependente do paranatelo.
\section{Paranassu}
\begin{itemize}
\item {Grp. gram.:m.}
\end{itemize}
Espécie de macaco do Brasil.
\section{Parança}
\begin{itemize}
\item {Grp. gram.:f.}
\end{itemize}
Acto de parar.
Parada; demora; descanso. Cf. Castilho, \textunderscore Fausto\textunderscore , 152.
\section{Parangona}
\begin{itemize}
\item {Grp. gram.:f.}
\end{itemize}
Variedade de caracteres typográphicos, de corpo alto.
(Cast. \textunderscore parangón\textunderscore )
\section{Paranheira}
\begin{itemize}
\item {Grp. gram.:f.}
\end{itemize}
\begin{itemize}
\item {Utilização:Prov.}
\end{itemize}
\begin{itemize}
\item {Utilização:minh.}
\end{itemize}
Padieira ou vêrga da porta do forno.
(Cf. \textunderscore pranheira\textunderscore )
\section{Paranho}
\begin{itemize}
\item {Grp. gram.:m.}
\end{itemize}
\begin{itemize}
\item {Utilização:Ant.}
\end{itemize}
Terra privilegiada.
Coito, honra.
\section{Paraninfar}
\begin{itemize}
\item {Grp. gram.:v. t.}
\end{itemize}
\begin{itemize}
\item {Proveniência:(De \textunderscore paraninfo\textunderscore )}
\end{itemize}
Sêr padrinho ou testemunha no casamento ou baptismo de.
\section{Paraninfo}
\begin{itemize}
\item {Grp. gram.:m.}
\end{itemize}
\begin{itemize}
\item {Utilização:Ext.}
\end{itemize}
\begin{itemize}
\item {Utilização:Fig.}
\end{itemize}
\begin{itemize}
\item {Proveniência:(Lat. \textunderscore paranymphus\textunderscore )}
\end{itemize}
Antigamente, amigo do noivo, que ia com êste buscar a noiva.
Hoje, padrinho de casamento.
Padrinho ou testemunha de baptismo.
Protector.
\section{Paranóia}
\begin{itemize}
\item {Grp. gram.:f.}
\end{itemize}
\begin{itemize}
\item {Utilização:Med.}
\end{itemize}
Loucura systematizada, isto é, que não tem por base a mania ou a melancolia e que, na sua evolução contínua, não offerece tendências para a demência propriamento dita. Cf. Júl. de Matos, \textunderscore Paranóia\textunderscore , 66.
\section{Paranôico}
\begin{itemize}
\item {Grp. gram.:adj.}
\end{itemize}
Relativo á paranóia.
\section{Paranomásia}
\begin{itemize}
\item {Grp. gram.:f.}
\end{itemize}
\begin{itemize}
\item {Proveniência:(Do gr. \textunderscore para\textunderscore  + \textunderscore onoma\textunderscore )}
\end{itemize}
Semelhança entre palavras de línguas differentes, indicando origem commum.
\section{Paranomia}
\begin{itemize}
\item {Grp. gram.:f.}
\end{itemize}
\begin{itemize}
\item {Utilização:Jur.}
\end{itemize}
\begin{itemize}
\item {Utilização:Ant.}
\end{itemize}
\begin{itemize}
\item {Proveniência:(Do gr. \textunderscore para\textunderscore  + \textunderscore nomos\textunderscore )}
\end{itemize}
Delicto, que, entre os Gregos, consistia em se fazerem propostas, em opposição ás leis.
\section{Paranone}
\begin{itemize}
\item {Grp. gram.:m.}
\end{itemize}
Embarcação asiática, para transporte de mercadorias.
\section{Paranthina}
\begin{itemize}
\item {Grp. gram.:f.}
\end{itemize}
\begin{itemize}
\item {Utilização:Miner.}
\end{itemize}
\begin{itemize}
\item {Proveniência:(Do gr. \textunderscore paranthein\textunderscore )}
\end{itemize}
Espécie de granada.
\section{Parantina}
\begin{itemize}
\item {Grp. gram.:f.}
\end{itemize}
\begin{itemize}
\item {Utilização:Miner.}
\end{itemize}
\begin{itemize}
\item {Proveniência:(Do gr. \textunderscore paranthein\textunderscore )}
\end{itemize}
Espécie de granada.
\section{Paranymphar}
\begin{itemize}
\item {Grp. gram.:v. t.}
\end{itemize}
\begin{itemize}
\item {Proveniência:(De \textunderscore paranympho\textunderscore )}
\end{itemize}
Sêr padrinho ou testemunha no casamento ou baptismo de.
\section{Paranympho}
\begin{itemize}
\item {Grp. gram.:m.}
\end{itemize}
\begin{itemize}
\item {Utilização:Ext.}
\end{itemize}
\begin{itemize}
\item {Utilização:Fig.}
\end{itemize}
\begin{itemize}
\item {Proveniência:(Lat. \textunderscore paranymphus\textunderscore )}
\end{itemize}
Antigamente, amigo do noivo, que ia com êste buscar a noiva.
Hoje, padrinho de casamento.
Padrinho ou testemunha de baptismo.
Protector.
\section{Parão}
\begin{itemize}
\item {Grp. gram.:m.}
\end{itemize}
Arma, usada pelos Timores, espécie de foice roçadoira, com a ponta levemente curva.
\section{Parapanda}
\begin{itemize}
\item {Grp. gram.:f.}
\end{itemize}
\begin{itemize}
\item {Utilização:Ant.}
\end{itemize}
Trombeta de Cafres.
\section{Parapectina}
\begin{itemize}
\item {Grp. gram.:f.}
\end{itemize}
\begin{itemize}
\item {Utilização:Chím.}
\end{itemize}
\begin{itemize}
\item {Proveniência:(De \textunderscore para...\textunderscore  + \textunderscore pectina\textunderscore )}
\end{itemize}
Corpo isómero á pectina, que se obtém, fazendo ferver esta por muito tempo.
\section{Parapegma}
\begin{itemize}
\item {Grp. gram.:m.}
\end{itemize}
\begin{itemize}
\item {Grp. gram.:Pl.}
\end{itemize}
\begin{itemize}
\item {Proveniência:(Lat. \textunderscore parapegma\textunderscore )}
\end{itemize}
Prancha de cobre, em que se gravavam ordenanças e proclamações, e que se affixava num poste, para sêr lida pelo público, entre os Antigos.
Tábuas astronómicas, em que os Sýrios e os Phenícios indicavam o nascer e o desapparecer dos astros.
\section{Parapeitar}
\begin{itemize}
\item {Grp. gram.:v. t.}
\end{itemize}
Formar o parapeito de.
\section{Parapeito}
\begin{itemize}
\item {Grp. gram.:m.}
\end{itemize}
\begin{itemize}
\item {Proveniência:(De \textunderscore parar\textunderscore  + \textunderscore peito\textunderscore )}
\end{itemize}
Parede ou resguardo, que se eleva á altura do peito ou pouco menos.
Cobertura de madeira, na parte inferior da janela, em que se apoiam os braços de quem chega á mesma janela.
Parte superior de uma trincheira de fortificação, para resguardar os defensores, podendo êstes fazer fogo por cima della.
\section{Parapétalo}
\begin{itemize}
\item {Grp. gram.:adj.}
\end{itemize}
\begin{itemize}
\item {Utilização:Bot.}
\end{itemize}
Diz-se das partes de uma corolla, quando são, mais ou menos, semelhantes ás pétalas, mas situadas mais interiormente, como no helléboro.
\section{Paraphasia}
\begin{itemize}
\item {Grp. gram.:f.}
\end{itemize}
\begin{itemize}
\item {Utilização:Med.}
\end{itemize}
\begin{itemize}
\item {Proveniência:(Do gr. \textunderscore para\textunderscore  + \textunderscore phasis\textunderscore )}
\end{itemize}
Perturbação da linguagem, em que as palavras não correspondem ás ideias.
\section{Paraphernaes}
\begin{itemize}
\item {Grp. gram.:f. pl.  e  adj.}
\end{itemize}
\begin{itemize}
\item {Proveniência:(Do gr. \textunderscore para\textunderscore  + \textunderscore pherne\textunderscore )}
\end{itemize}
Bens, que não são dotaes, mas que a mulher póde gozar ou administrar independentemente.
\section{Paraphimose}
\begin{itemize}
\item {Grp. gram.:f.}
\end{itemize}
\begin{itemize}
\item {Utilização:Med.}
\end{itemize}
\begin{itemize}
\item {Proveniência:(Do gr. \textunderscore para\textunderscore  + \textunderscore phimos\textunderscore )}
\end{itemize}
Inflamação e desviramento do prepúcio, por fórma que não póde recobrir a glande.
\section{Paraphonia}
\begin{itemize}
\item {Grp. gram.:f.}
\end{itemize}
\begin{itemize}
\item {Proveniência:(Do gr. \textunderscore para\textunderscore  + \textunderscore phone\textunderscore )}
\end{itemize}
Defeito da voz, consistindo num timbre desagradável.
\section{Paraphonista}
\begin{itemize}
\item {Grp. gram.:f.}
\end{itemize}
Nome, que se dava a cada um dos dois primeiros cantores, num côro de cantochão.
(Cp. \textunderscore paraphonia\textunderscore )
\section{Paraphosphato}
\begin{itemize}
\item {Grp. gram.:m.}
\end{itemize}
Gênero de saes, produzidos pelo ácido paraphosphórico.
\section{Paraphosphórico}
\begin{itemize}
\item {Grp. gram.:adj.}
\end{itemize}
\begin{itemize}
\item {Proveniência:(De \textunderscore para...\textunderscore  + \textunderscore phosphórico\textunderscore )}
\end{itemize}
Diz-se do ácido phosphórico, quando, sujeito a grande calor, adquire propriedades que não tinha.
\section{Paráphrase}
\begin{itemize}
\item {Grp. gram.:f.}
\end{itemize}
\begin{itemize}
\item {Proveniência:(Lat. \textunderscore paraphrasis\textunderscore )}
\end{itemize}
Desenvolvimento do texto de um livro ou documento.
Traducção livre ou desenvolvida.
\section{Paraphrasear}
\begin{itemize}
\item {Grp. gram.:v. t.}
\end{itemize}
\begin{itemize}
\item {Proveniência:(De \textunderscore paráphrase\textunderscore )}
\end{itemize}
Explicar, desenvolvendo.
Traduzir livremente.
Desenvolver por palavra ou por escrito.
\section{Paraphrasia}
\begin{itemize}
\item {Grp. gram.:f.}
\end{itemize}
\begin{itemize}
\item {Utilização:Med.}
\end{itemize}
\begin{itemize}
\item {Proveniência:(Do gr. \textunderscore para\textunderscore  + \textunderscore phrasis\textunderscore )}
\end{itemize}
Perturbação da palavra.
\section{Paraphrasta}
\begin{itemize}
\item {Grp. gram.:m.}
\end{itemize}
\begin{itemize}
\item {Proveniência:(Lat. \textunderscore paraphrastes\textunderscore )}
\end{itemize}
Autor de paráphrases.
\section{Paraphraste}
\begin{itemize}
\item {Grp. gram.:m.}
\end{itemize}
(V.paraphrasta)
\section{Paraphrásticamente}
\begin{itemize}
\item {Grp. gram.:adv.}
\end{itemize}
De modo paraphrástico.
\section{Paraphrástico}
\begin{itemize}
\item {Grp. gram.:adj.}
\end{itemize}
Relativo a paráphrase: \textunderscore traducção paraphrástica\textunderscore .
\section{Paraphrósyne}
\begin{itemize}
\item {Grp. gram.:f.}
\end{itemize}
\begin{itemize}
\item {Utilização:Med.}
\end{itemize}
\begin{itemize}
\item {Proveniência:(Gr. \textunderscore paraphrosune\textunderscore )}
\end{itemize}
Delírio febril.
\section{Paráphyse}
\begin{itemize}
\item {Grp. gram.:f.}
\end{itemize}
\begin{itemize}
\item {Utilização:Bot.}
\end{itemize}
\begin{itemize}
\item {Proveniência:(Gr. \textunderscore paraphusis\textunderscore )}
\end{itemize}
Cada um dos filetes, que acompanham os órgãos da fructificação dos musgos.
\section{Paraplegia}
\begin{itemize}
\item {Grp. gram.:f.}
\end{itemize}
\begin{itemize}
\item {Proveniência:(Gr. \textunderscore paraplegia\textunderscore )}
\end{itemize}
Paralysia dos membros inferiores.
\section{Parapleura}
\begin{itemize}
\item {Grp. gram.:f.}
\end{itemize}
\begin{itemize}
\item {Utilização:Hist. Nat.}
\end{itemize}
\begin{itemize}
\item {Proveniência:(Do gr. \textunderscore para\textunderscore  + \textunderscore pleuron\textunderscore )}
\end{itemize}
Cada uma das peças, que constituem o thórax dos insectos.
\section{Parapleurisia}
\begin{itemize}
\item {Grp. gram.:f.}
\end{itemize}
\begin{itemize}
\item {Utilização:Med.}
\end{itemize}
\begin{itemize}
\item {Proveniência:(De \textunderscore para...\textunderscore  + \textunderscore pleurisia\textunderscore )}
\end{itemize}
Falsa pleurisia.
\section{Paraplexia}
\begin{itemize}
\item {fónica:csi}
\end{itemize}
\begin{itemize}
\item {Grp. gram.:f.}
\end{itemize}
\begin{itemize}
\item {Proveniência:(Lat. \textunderscore paraplexia\textunderscore )}
\end{itemize}
Nome, que se dá algumas vezes á paraplegia e á paralysia.
\section{Parapodário}
\begin{itemize}
\item {Grp. gram.:adj.}
\end{itemize}
Que tem parápodes.
\section{Parápode}
\begin{itemize}
\item {Grp. gram.:m.}
\end{itemize}
\begin{itemize}
\item {Utilização:Zool.}
\end{itemize}
\begin{itemize}
\item {Proveniência:(Do gr. \textunderscore para\textunderscore  + \textunderscore pous\textunderscore , \textunderscore podos\textunderscore )}
\end{itemize}
Carúncula carnuda do cutículo dos vermes, que serve geralmente para a locomoção.
\section{Parapoplexia}
\begin{itemize}
\item {fónica:csi}
\end{itemize}
\begin{itemize}
\item {Grp. gram.:f.}
\end{itemize}
(V.paraplexia)
\section{Parapurgativo}
\begin{itemize}
\item {Grp. gram.:adj.}
\end{itemize}
\begin{itemize}
\item {Proveniência:(De \textunderscore para...\textunderscore  + \textunderscore purgativo\textunderscore )}
\end{itemize}
Diz-se do medicamento, que suspende a acção purgativa.
\section{Paraquê}
\begin{itemize}
\item {Grp. gram.:m.}
\end{itemize}
\begin{itemize}
\item {Proveniência:(De \textunderscore para\textunderscore  + \textunderscore que\textunderscore )}
\end{itemize}
Alvo ou mira de uma acção. Cf. Ol. Martins, \textunderscore Filhos de D. João I\textunderscore , 32.
\section{Pára-quédas}
\begin{itemize}
\item {Grp. gram.:m.}
\end{itemize}
\begin{itemize}
\item {Proveniência:(De \textunderscore parar\textunderscore  + \textunderscore quéda\textunderscore )}
\end{itemize}
Apparelho, para deminuir a velocidade da quéda dos corpos.
\section{Parar}
\begin{itemize}
\item {Grp. gram.:v. i.}
\end{itemize}
\begin{itemize}
\item {Grp. gram.:V. t.}
\end{itemize}
\begin{itemize}
\item {Utilização:Ant.}
\end{itemize}
\begin{itemize}
\item {Proveniência:(Lat. \textunderscore parare\textunderscore )}
\end{itemize}
Deixar de andar: \textunderscore o cavallo parou\textunderscore .
Cessar de mover-se ou de operar: \textunderscore o relógio parou\textunderscore .
Findar.
Estacionar, conservar-se.
Impedir o movimento de: \textunderscore o cocheiro parou o trem\textunderscore .
Fixar.
Aparar: \textunderscore parar um golpe de sabre\textunderscore .
Deminuír a velocidade ou a intensidade de.
Apontar ao jôgo (uma quantia).
\textunderscore Parar mentes\textunderscore , reflectir, têr cuidado.
\section{Pára-raios}
\begin{itemize}
\item {Grp. gram.:m.}
\end{itemize}
\begin{itemize}
\item {Proveniência:(De \textunderscore parar\textunderscore  + \textunderscore raio\textunderscore )}
\end{itemize}
Apparelho, formado principalmente de uma haste metállica, e destinado a attrahir as descargas eléctricas da atmosphera, livrando dellas os lugares ou edifícios próximos.
\section{Parari}
\begin{itemize}
\item {Grp. gram.:m.}
\end{itemize}
\begin{itemize}
\item {Utilização:Bras}
\end{itemize}
Espécie de pomba.
Erva tinctória do Alto Amazonas.
\section{Pararthrema}
\begin{itemize}
\item {Grp. gram.:m.}
\end{itemize}
\begin{itemize}
\item {Utilização:Cir.}
\end{itemize}
\begin{itemize}
\item {Proveniência:(Gr. \textunderscore pararthrema\textunderscore )}
\end{itemize}
Luxação incompleta.
\section{Parartrema}
\begin{itemize}
\item {Grp. gram.:m.}
\end{itemize}
\begin{itemize}
\item {Utilização:Cir.}
\end{itemize}
\begin{itemize}
\item {Proveniência:(Gr. \textunderscore pararthrema\textunderscore )}
\end{itemize}
Luxação incompleta.
\section{Pararu}
\begin{itemize}
\item {Grp. gram.:m.}
\end{itemize}
Planta medicinal da Guiana inglesa.
\section{Parasanga}
\begin{itemize}
\item {Grp. gram.:f.}
\end{itemize}
\begin{itemize}
\item {Proveniência:(Lat. \textunderscore parasanga\textunderscore )}
\end{itemize}
Medida itinerária da Pérsia.
\section{Parásceve}
\begin{itemize}
\item {Grp. gram.:f.}
\end{itemize}
\begin{itemize}
\item {Proveniência:(Lat. \textunderscore parasceve\textunderscore )}
\end{itemize}
Nome, que os Judeus davam á sexta-feira, porque começavam a preparar-se então para a festa do dia seguinte.
\section{Paraselene}
\begin{itemize}
\item {fónica:sé}
\end{itemize}
\begin{itemize}
\item {Grp. gram.:f.}
\end{itemize}
\begin{itemize}
\item {Proveniência:(Do gr. \textunderscore para\textunderscore  + \textunderscore selene\textunderscore )}
\end{itemize}
Círculo luminoso, que se observa ás vezes em roda da Lua.
\section{Paraselênio}
\begin{itemize}
\item {fónica:se}
\end{itemize}
\begin{itemize}
\item {Grp. gram.:m.}
\end{itemize}
O mesmo ou melhor que \textunderscore paraselene\textunderscore .
\section{Parasematographia}
\begin{itemize}
\item {fónica:se}
\end{itemize}
\begin{itemize}
\item {Grp. gram.:f.}
\end{itemize}
O mesmo que \textunderscore heráldica\textunderscore .
\section{Parasematográphico}
\begin{itemize}
\item {fónica:se}
\end{itemize}
\begin{itemize}
\item {Grp. gram.:m.}
\end{itemize}
Relativo á parasematographia.
\section{Parasematógrapho}
\begin{itemize}
\item {fónica:se}
\end{itemize}
\begin{itemize}
\item {Grp. gram.:m.}
\end{itemize}
Aquelle que é perito em parasematographia.
\section{Parasigmatismo}
\begin{itemize}
\item {fónica:si}
\end{itemize}
\begin{itemize}
\item {Grp. gram.:m.}
\end{itemize}
\begin{itemize}
\item {Utilização:Med.}
\end{itemize}
\begin{itemize}
\item {Proveniência:(Do gr. \textunderscore para\textunderscore  + \textunderscore sigma\textunderscore )}
\end{itemize}
Vício de pronúncia, em que se troca o \textunderscore s\textunderscore  por outra letra.
\section{Parasita}
\begin{itemize}
\item {Grp. gram.:m.  e  f.}
\end{itemize}
\begin{itemize}
\item {Grp. gram.:Adj.}
\end{itemize}
\begin{itemize}
\item {Proveniência:(Lat. \textunderscore parasitus\textunderscore )}
\end{itemize}
Animal, que se nutre do sangue de outro.
Vegetal, que se nutre da seiva de outro.
Indivíduo habituado a comer em casa alheia; papa-jantares.
Que nasce ou cresce em outros corpos organizados, mortos ou vivos.
Que vive á custa alheia.
\section{Parasitar}
\begin{itemize}
\item {Grp. gram.:v. i.}
\end{itemize}
O mesmo que \textunderscore parasitear\textunderscore .
\section{Parasitário}
\begin{itemize}
\item {Grp. gram.:adj.}
\end{itemize}
Relativo a parasita; que tem as propriedades de animal parasito. Cf. Baldaque, \textunderscore Pesca\textunderscore .
\section{Parasitear}
\begin{itemize}
\item {Grp. gram.:v. i.}
\end{itemize}
Viver como parasita.
\section{Parasiticida}
\begin{itemize}
\item {Grp. gram.:m.  e  adj.}
\end{itemize}
O que destrói os parasitos.
(Do \textunderscore parasito\textunderscore  lat. + \textunderscore us. caedere\textunderscore )
\section{Parasítico}
\begin{itemize}
\item {Grp. gram.:adj.}
\end{itemize}
\begin{itemize}
\item {Proveniência:(Lat. \textunderscore parasiticus\textunderscore )}
\end{itemize}
Relativo ao parasito.
\section{Parasitífero}
\begin{itemize}
\item {Grp. gram.:adj.}
\end{itemize}
Que tem ou alimenta parasitos.
(Do \textunderscore parasito\textunderscore  lat. + \textunderscore us. ferre\textunderscore )
\section{Parasitismo}
\begin{itemize}
\item {Grp. gram.:m.}
\end{itemize}
Qualidade ou estado do que é parasito; habitos de parasito.
\section{Parasito}
\begin{itemize}
\item {Grp. gram.:m.  e  f.}
\end{itemize}
\begin{itemize}
\item {Grp. gram.:Adj.}
\end{itemize}
\begin{itemize}
\item {Proveniência:(Lat. \textunderscore parasitus\textunderscore )}
\end{itemize}
Animal, que se nutre do sangue de outro.
Vegetal, que se nutre da seiva de outro.
Indivíduo habituado a comer em casa alheia; papa-jantares.
Que nasce ou cresce em outros corpos organizados, mortos ou vivos.
Que vive á custa alheia.
\section{Parasito}
\begin{itemize}
\item {Grp. gram.:m.}
\end{itemize}
\begin{itemize}
\item {Proveniência:(Lat. \textunderscore parasitus\textunderscore )}
\end{itemize}
O mesmo ou melhor que \textunderscore parasita\textunderscore .--No Brasil, aquella fórma é adoptada por Caminhoá, \textunderscore Bot. Ger.\textunderscore , e por Pacheco da Silva, \textunderscore Promptuário\textunderscore , 10.
\section{Parasitogenia}
\begin{itemize}
\item {Grp. gram.:f.}
\end{itemize}
\begin{itemize}
\item {Proveniência:(Do gr. \textunderscore parasitos\textunderscore  + \textunderscore genes\textunderscore )}
\end{itemize}
Conjunto dos phenómenos physiológico-pathológicos, pelos quaes os organismos vivos, cachéticos e fracos, se tornam aptos para o nascimento e reproducção dos helminthos e dos ácaros.
\section{Parasitologia}
\begin{itemize}
\item {Grp. gram.:f.}
\end{itemize}
\begin{itemize}
\item {Utilização:Med.}
\end{itemize}
\begin{itemize}
\item {Proveniência:(Do gr. \textunderscore parasitos\textunderscore  + \textunderscore logos\textunderscore )}
\end{itemize}
Estudo scientífico do parasito.
\section{Parasselene}
\begin{itemize}
\item {Grp. gram.:f.}
\end{itemize}
\begin{itemize}
\item {Proveniência:(Do gr. \textunderscore para\textunderscore  + \textunderscore selene\textunderscore )}
\end{itemize}
Círculo luminoso, que se observa ás vezes em roda da Lua.
\section{Parasselênio}
\begin{itemize}
\item {Grp. gram.:m.}
\end{itemize}
O mesmo ou melhor que \textunderscore parasselene\textunderscore .
\section{Parassematografia}
\begin{itemize}
\item {Grp. gram.:f.}
\end{itemize}
O mesmo que \textunderscore heráldica\textunderscore .
\section{Parassematográfico}
\begin{itemize}
\item {Grp. gram.:m.}
\end{itemize}
Relativo á parasematografia.
\section{Parassematógrafo}
\begin{itemize}
\item {Grp. gram.:m.}
\end{itemize}
Aquele que é perito em parasematografia.
\section{Parassigmatismo}
\begin{itemize}
\item {Grp. gram.:m.}
\end{itemize}
\begin{itemize}
\item {Utilização:Med.}
\end{itemize}
\begin{itemize}
\item {Proveniência:(Do gr. \textunderscore para\textunderscore  + \textunderscore sigma\textunderscore )}
\end{itemize}
Vício de pronúncia, em que se troca o \textunderscore s\textunderscore  por outra letra.
\section{Parasitofobia}
\begin{itemize}
\item {Grp. gram.:f.}
\end{itemize}
\begin{itemize}
\item {Utilização:Med.}
\end{itemize}
\begin{itemize}
\item {Proveniência:(Do gr. \textunderscore parasitos\textunderscore  + \textunderscore phobein\textunderscore )}
\end{itemize}
Mêdo mórbido de contrair moléstia cutâneas parasitárias.
\section{Parasitóforo}
\begin{itemize}
\item {Grp. gram.:adj.}
\end{itemize}
\begin{itemize}
\item {Proveniência:(Do gr. \textunderscore parasitos\textunderscore  + \textunderscore phoros\textunderscore )}
\end{itemize}
O mesmo que \textunderscore parasitífero\textunderscore .
\section{Parasitológico}
\begin{itemize}
\item {Grp. gram.:adj.}
\end{itemize}
Relativo á parasitologia. Cf. \textunderscore Jorn.-do-Comm.\textunderscore , do Rio, de 11-V-901.
\section{Parasitophobia}
\begin{itemize}
\item {Grp. gram.:f.}
\end{itemize}
\begin{itemize}
\item {Utilização:Med.}
\end{itemize}
\begin{itemize}
\item {Proveniência:(Do gr. \textunderscore parasitos\textunderscore  + \textunderscore phobein\textunderscore )}
\end{itemize}
Mêdo mórbido de contrahir moléstia cutâneas parasitárias.
\section{Parasitóphoro}
\begin{itemize}
\item {Grp. gram.:adj.}
\end{itemize}
\begin{itemize}
\item {Proveniência:(Do gr. \textunderscore parasitos\textunderscore  + \textunderscore phoros\textunderscore )}
\end{itemize}
O mesmo que \textunderscore parasitífero\textunderscore .
\section{Pára-sol}
\begin{itemize}
\item {Grp. gram.:m.}
\end{itemize}
(V.guarda-sol)Cf. Capello e Ivens, II, 32.
\section{Parassífilis}
\begin{itemize}
\item {Grp. gram.:f.}
\end{itemize}
\begin{itemize}
\item {Utilização:Med.}
\end{itemize}
\begin{itemize}
\item {Proveniência:(De \textunderscore para...\textunderscore  + \textunderscore sífilis\textunderscore )}
\end{itemize}
Sífilis hereditária.
Casos mórbidos, desenvolvidos sob a influência da sífilis, sem participação da natureza desta.
\section{Parastado}
\begin{itemize}
\item {Grp. gram.:adj.}
\end{itemize}
\begin{itemize}
\item {Utilização:Bot.}
\end{itemize}
Diz-se dos filamentos estéreis, compostos do muitas séries de céllulas, e situados entre as pétalas e os estames.
\section{Parastaminia}
\begin{itemize}
\item {Grp. gram.:f.}
\end{itemize}
\begin{itemize}
\item {Utilização:Bot.}
\end{itemize}
\begin{itemize}
\item {Proveniência:(De \textunderscore para...\textunderscore  + \textunderscore estame\textunderscore )}
\end{itemize}
Estado dos estames abortados, ou dos órgãos que, parecendo estames, não exercem as funcções dêstes.
\section{Parastática}
\begin{itemize}
\item {Grp. gram.:f.}
\end{itemize}
\begin{itemize}
\item {Proveniência:(Lat. \textunderscore parastatica\textunderscore )}
\end{itemize}
Pilastra, que decorava as extremidades angulares dos edifícios antigos, ficando parte della embebida na parede.
\section{Parastilo}
\begin{itemize}
\item {Grp. gram.:m.}
\end{itemize}
\begin{itemize}
\item {Utilização:Bot.}
\end{itemize}
\begin{itemize}
\item {Proveniência:(Do gr. \textunderscore para\textunderscore  + \textunderscore stulos\textunderscore )}
\end{itemize}
Pistilo abortado, ou órgão que parece pistilo e não exerce as funções dêste.
\section{Parastylo}
\begin{itemize}
\item {Grp. gram.:m.}
\end{itemize}
\begin{itemize}
\item {Utilização:Bot.}
\end{itemize}
\begin{itemize}
\item {Proveniência:(Do gr. \textunderscore para\textunderscore  + \textunderscore stulos\textunderscore )}
\end{itemize}
Pistillo abortado, ou órgão que parece pistillo e não exerce as funcções dêste.
\section{Parasýphilis}
\begin{itemize}
\item {fónica:si}
\end{itemize}
\begin{itemize}
\item {Grp. gram.:f.}
\end{itemize}
\begin{itemize}
\item {Utilização:Med.}
\end{itemize}
\begin{itemize}
\item {Proveniência:(De \textunderscore para...\textunderscore  + \textunderscore sýphilis\textunderscore )}
\end{itemize}
Sýphilis hereditária.
Casos mórbidos, desenvolvidos sob a influência da sýphilis, sem participação da natureza desta.
\section{Paratarso}
\begin{itemize}
\item {Grp. gram.:m.}
\end{itemize}
\begin{itemize}
\item {Utilização:Zool.}
\end{itemize}
\begin{itemize}
\item {Proveniência:(De \textunderscore para...\textunderscore  + \textunderscore tarso\textunderscore )}
\end{itemize}
Parte lateral do tarso das aves.
\section{Paratartarato}
\begin{itemize}
\item {Grp. gram.:m.}
\end{itemize}
\begin{itemize}
\item {Utilização:Chím.}
\end{itemize}
Gênero de saes, produzidos pelo ácido paratárarico.
\section{Paratárarico}
\begin{itemize}
\item {Grp. gram.:adj.}
\end{itemize}
Diz-se de um ácido, o mesmo que o \textunderscore racêmico\textunderscore .
\section{Paratenar}
\begin{itemize}
\item {Grp. gram.:m.}
\end{itemize}
\begin{itemize}
\item {Utilização:Anat.}
\end{itemize}
\begin{itemize}
\item {Proveniência:(De \textunderscore para...\textunderscore  + \textunderscore thenar\textunderscore )}
\end{itemize}
Nome, com que se designava uma parte do músculo abductor do dedo mínimo do pé, (grande parathenar), e o pequeno flector do mesmo dedo, (pequeno parathenar).
\section{Parathenar}
\begin{itemize}
\item {Grp. gram.:m.}
\end{itemize}
\begin{itemize}
\item {Utilização:Anat.}
\end{itemize}
\begin{itemize}
\item {Proveniência:(De \textunderscore para...\textunderscore  + \textunderscore thenar\textunderscore )}
\end{itemize}
Nome, com que se designava uma parte do músculo abductor do dedo mínimo do pé, (grande parathenar), e o pequeno flector do mesmo dedo, (pequeno parathenar)
\section{Parati}
\begin{itemize}
\item {Grp. gram.:m.}
\end{itemize}
\begin{itemize}
\item {Utilização:Bras}
\end{itemize}
Peixe, semelhante á taínha, mas menor.
Aguardente de cana, fabricada no município de Parati.
\section{Paratitlário}
\begin{itemize}
\item {Grp. gram.:m.}
\end{itemize}
Autor de paratitlos.
\section{Paratitlos}
\begin{itemize}
\item {Grp. gram.:m. pl.}
\end{itemize}
Curta explicação ou glosa dos títulos do Digesto e de outras compilações de leis, para se lhes conhecer a matéria e a ligação.
(B. lat.\textunderscore  paratitla\textunderscore )
\section{Paratoma}
\begin{itemize}
\item {Grp. gram.:m.}
\end{itemize}
\begin{itemize}
\item {Utilização:Zool.}
\end{itemize}
\begin{itemize}
\item {Proveniência:(Do gr. \textunderscore para\textunderscore  + \textunderscore tome\textunderscore )}
\end{itemize}
Parte lateral do bico das aves.
\section{Paratopia}
\begin{itemize}
\item {Grp. gram.:f.}
\end{itemize}
\begin{itemize}
\item {Utilização:Med.}
\end{itemize}
\begin{itemize}
\item {Proveniência:(Do gr. \textunderscore para\textunderscore  + \textunderscore topos\textunderscore )}
\end{itemize}
Qualquer deslocação, como hérnia, luxação, etc.
\section{Paratrima}
\begin{itemize}
\item {Grp. gram.:f.}
\end{itemize}
\begin{itemize}
\item {Utilização:Med.}
\end{itemize}
\begin{itemize}
\item {Proveniência:(Gr. \textunderscore paratrimma\textunderscore )}
\end{itemize}
Espécie de eritema, produzido por pressão forte e constante numa parte da superfície cutânea.
\section{Paratrimma}
\begin{itemize}
\item {Grp. gram.:f.}
\end{itemize}
\begin{itemize}
\item {Utilização:Med.}
\end{itemize}
\begin{itemize}
\item {Proveniência:(Gr. \textunderscore paratrimma\textunderscore )}
\end{itemize}
Espécie de erythema, produzido por pressão forte e constante numa parte da superfície cutânea.
\section{Paratucu}
\begin{itemize}
\item {Grp. gram.:m.}
\end{itemize}
Jasmim silvestre do Pará.
\section{Paratudo}
\begin{itemize}
\item {Grp. gram.:m.}
\end{itemize}
\begin{itemize}
\item {Utilização:Bras}
\end{itemize}
\begin{itemize}
\item {Proveniência:(De \textunderscore para\textunderscore  + \textunderscore tudo\textunderscore )}
\end{itemize}
Nome de várias plantas do Brasil; raíz dessas plantas.
\section{Paraturá}
\begin{itemize}
\item {Grp. gram.:m.}
\end{itemize}
Planta cyperácea do Brasil.
\section{Parau}
\begin{itemize}
\item {Grp. gram.:m.}
\end{itemize}
\begin{itemize}
\item {Proveniência:(Do mal. \textunderscore prahu\textunderscore )}
\end{itemize}
Navio de guerra indiano.
Pequeno barco oriental.
\section{Parauacu}
\begin{itemize}
\item {Grp. gram.:m.}
\end{itemize}
\begin{itemize}
\item {Utilização:Bras}
\end{itemize}
Espécie de macaco.
\section{Parauacu-bóia}
\begin{itemize}
\item {Grp. gram.:f.}
\end{itemize}
Serpente do Brasil.
\section{Parauamás}
\begin{itemize}
\item {Grp. gram.:m. pl.}
\end{itemize}
Indígenas do Norte do Brasil.
O mesmo que \textunderscore parauanas\textunderscore ?--Uma das duas fórmas é talvez errada. Cf. Araújo e Amazonas, \textunderscore Diccion. Topogr.\textunderscore 
\section{Parauanas}
\begin{itemize}
\item {Grp. gram.:m. pl.}
\end{itemize}
\begin{itemize}
\item {Utilização:Bras}
\end{itemize}
Tríbo de aborigenes que habitou no Pará.
\section{Parauchene}
\begin{itemize}
\item {fónica:quê}
\end{itemize}
\begin{itemize}
\item {Grp. gram.:m.}
\end{itemize}
\begin{itemize}
\item {Utilização:Zool.}
\end{itemize}
\begin{itemize}
\item {Proveniência:(Do gr. \textunderscore para\textunderscore  + \textunderscore aukhene\textunderscore )}
\end{itemize}
Parte lateral do pescoço dos mammíferos e das aves.
\section{Parauquene}
\begin{itemize}
\item {Grp. gram.:m.}
\end{itemize}
\begin{itemize}
\item {Utilização:Zool.}
\end{itemize}
\begin{itemize}
\item {Proveniência:(Do gr. \textunderscore para\textunderscore  + \textunderscore aukhene\textunderscore )}
\end{itemize}
Parte lateral do pescoço dos mammíferos e das aves.
\section{Paraús}
\begin{itemize}
\item {Grp. gram.:m. pl.}
\end{itemize}
Povos do Norte do Brasil.
\section{Paravante}
\begin{itemize}
\item {Grp. gram.:m.}
\end{itemize}
\begin{itemize}
\item {Grp. gram.:Adv.}
\end{itemize}
\begin{itemize}
\item {Utilização:Náut.}
\end{itemize}
\begin{itemize}
\item {Proveniência:(De \textunderscore para\textunderscore  + \textunderscore avante\textunderscore )}
\end{itemize}
Parte de um navio, comprehendida entre a prôa e o mastro grande.
Por ante-avante.
\section{Paravás}
\begin{itemize}
\item {Grp. gram.:m. pl.}
\end{itemize}
Povos da Índia.
\section{Pára-vento}
\begin{itemize}
\item {Grp. gram.:m.}
\end{itemize}
O mesmo que \textunderscore guarda-vento\textunderscore .
\section{Paravianas}
\begin{itemize}
\item {Grp. gram.:m. pl.}
\end{itemize}
Indígenas do Norte do Brasil.
(Cp. \textunderscore parauanas\textunderscore )
\section{Paraviso}
\begin{itemize}
\item {Grp. gram.:m.}
\end{itemize}
\begin{itemize}
\item {Utilização:Ant.}
\end{itemize}
O mesmo que \textunderscore paraíso\textunderscore .
\section{Parávoa}
\begin{itemize}
\item {Grp. gram.:f.}
\end{itemize}
\begin{itemize}
\item {Utilização:Ant.}
\end{itemize}
\begin{itemize}
\item {Proveniência:(Do lat. \textunderscore parabola\textunderscore )}
\end{itemize}
O mesmo que \textunderscore palavra\textunderscore . Cf. Viterbo, \textunderscore Elucid.\textunderscore 
\section{Parávora}
\begin{itemize}
\item {Grp. gram.:f.}
\end{itemize}
\begin{itemize}
\item {Utilização:Ant.}
\end{itemize}
O mesmo que \textunderscore palavra\textunderscore . Cf. Frei Fortun., \textunderscore Inéd.\textunderscore , 311.
\section{Parazónio}
\begin{itemize}
\item {Grp. gram.:m.}
\end{itemize}
\begin{itemize}
\item {Proveniência:(Lat. \textunderscore parazonium\textunderscore )}
\end{itemize}
Espada curta, com talabarte, accessório ordinário das estátuas de Marte e dos heróis.
\section{Parca}
\begin{itemize}
\item {Grp. gram.:f.}
\end{itemize}
\begin{itemize}
\item {Utilização:Fig.}
\end{itemize}
\begin{itemize}
\item {Proveniência:(Lat. \textunderscore parca\textunderscore )}
\end{itemize}
Cada uma das três deusas, que fiavam e cortavam o fio da vida, segundo a Mythologia latina.
A morte.
\section{Parcagem}
\begin{itemize}
\item {Grp. gram.:f.}
\end{itemize}
\begin{itemize}
\item {Utilização:Gal}
\end{itemize}
\begin{itemize}
\item {Proveniência:(Fr. \textunderscore parcage\textunderscore )}
\end{itemize}
Permanência do gado em malhada. Cf. Macedo Pinto, \textunderscore Comp. de Veter.\textunderscore , II, 611.
\section{Parcamente}
\begin{itemize}
\item {Grp. gram.:adv.}
\end{itemize}
De modo parco; economicamente; com frugalidade.
\section{Parçaria}
\begin{itemize}
\item {Grp. gram.:f.}
\end{itemize}
\begin{itemize}
\item {Proveniência:(De \textunderscore parceiro\textunderscore )}
\end{itemize}
Reunião de indivíduos para um fim de interesse commum.
Sociedade, companhia. Cf. Camillo, \textunderscore Caveira\textunderscore , 229; \textunderscore Eufrosina\textunderscore , 104; Cortesão, \textunderscore Subsídios\textunderscore , etc.
\section{Parceiro}
\begin{itemize}
\item {Grp. gram.:adj.}
\end{itemize}
\begin{itemize}
\item {Grp. gram.:M.}
\end{itemize}
\begin{itemize}
\item {Utilização:Pop.}
\end{itemize}
\begin{itemize}
\item {Utilização:T. da Bairrada}
\end{itemize}
\begin{itemize}
\item {Proveniência:(Do lat. \textunderscore partiarius\textunderscore )}
\end{itemize}
Parelho; par; semelhante.
Sócio.
Comparte; companheiro.
Pessôa, com quem se joga.
Espertalhão.
Tratamento recíproco dos que foram mordomos de uma festividade, no mesmo anno; tratamento recíproco dos pais dos cônjuges.
\section{Parcel}
\begin{itemize}
\item {Grp. gram.:m.}
\end{itemize}
Escôlho.
Baixio; recife.
\section{Parcela}
\begin{itemize}
\item {Grp. gram.:f.}
\end{itemize}
\begin{itemize}
\item {Utilização:Arith.}
\end{itemize}
\begin{itemize}
\item {Proveniência:(Do lat. hyp. \textunderscore particella\textunderscore )}
\end{itemize}
Pequena parte, fragmento.
Verba.
Cada um dos números que se somam.
\section{Parcelado}
\begin{itemize}
\item {Grp. gram.:adj.}
\end{itemize}
\begin{itemize}
\item {Utilização:Bras}
\end{itemize}
\begin{itemize}
\item {Proveniência:(De \textunderscore parcelar\textunderscore )}
\end{itemize}
Diz-se dos exames preparatórios para as escolas superiores, feitos separadamente, por disciplina.
\section{Parcelar}
\begin{itemize}
\item {Grp. gram.:v. t.}
\end{itemize}
\begin{itemize}
\item {Grp. gram.:Adj.}
\end{itemize}
Dividir em parcelas.
Dividido ou feito em parcelas.
\section{Parcelário}
\begin{itemize}
\item {Grp. gram.:adj.}
\end{itemize}
Dividido em parcelas; parcelar.
Dizia-se especialmente do sistema, por que eram cultivadas as propriedades dos antigos possessores, sistema em que as parcelas constituíam colónias. Cf. Herculano, \textunderscore Hist. de Port.\textunderscore , III.
\section{Parcella}
\begin{itemize}
\item {Grp. gram.:f.}
\end{itemize}
\begin{itemize}
\item {Utilização:Arith.}
\end{itemize}
\begin{itemize}
\item {Proveniência:(Do lat. hyp. \textunderscore particella\textunderscore )}
\end{itemize}
Pequena parte, fragmento.
Verba.
Cada um dos números que se sommam.
\section{Parcellado}
\begin{itemize}
\item {Grp. gram.:adj.}
\end{itemize}
\begin{itemize}
\item {Utilização:Bras}
\end{itemize}
\begin{itemize}
\item {Proveniência:(De \textunderscore parcellar\textunderscore )}
\end{itemize}
Diz-se dos exames preparatórios para as escolas superiores, feitos separadamente, por disciplina.
\section{Parcellar}
\begin{itemize}
\item {Grp. gram.:v. t.}
\end{itemize}
\begin{itemize}
\item {Grp. gram.:Adj.}
\end{itemize}
Dividir em parcellas.
Dividido ou feito em parcellas.
\section{Parcellário}
\begin{itemize}
\item {Grp. gram.:adj.}
\end{itemize}
Dividido em parcellas; parcellar.
Dizia-se especialmente do systema, por que eram cultivadas as propriedades dos antigos possessores, systema em que as parcellas constituíam colónias. Cf. Herculano, \textunderscore Hist. de Port.\textunderscore , III.
\section{Parceria}
\begin{itemize}
\item {Grp. gram.:f.}
\end{itemize}
(V.parçaria)
\section{Parcha}
\begin{itemize}
\item {Grp. gram.:f.}
\end{itemize}
Casulo, em que o bicho da seda morreu de doença.
\section{Parche}
\begin{itemize}
\item {Grp. gram.:m.}
\end{itemize}
\begin{itemize}
\item {Proveniência:(Do lat. \textunderscore parthicum\textunderscore  (vellum))}
\end{itemize}
Pano ou panos, embebidos num líquido e applicados sôbre uma parte doente do corpo, para combater uma dôr ou inflammação.
\section{Parchear}
\begin{itemize}
\item {Grp. gram.:v. t.}
\end{itemize}
\begin{itemize}
\item {Proveniência:(De \textunderscore parche\textunderscore )}
\end{itemize}
Pôr parches em.
Pôr parches no cachaço ou na cabeça do (toiro).
\section{Parcial}
\begin{itemize}
\item {Grp. gram.:adj.}
\end{itemize}
\begin{itemize}
\item {Grp. gram.:M.  e  f.}
\end{itemize}
\begin{itemize}
\item {Proveniência:(Lat. \textunderscore partialis\textunderscore )}
\end{itemize}
Que é parte de um todo.
Que toma parte numa questão, acto ou emprehendimento.
Que se realiza por partes: \textunderscore pagamento parcial de uma divida\textunderscore .
Favorável a uma das partes, num litígio.
Partidário; sectário.
Pessôa, que é partidária de alguém ou que segue algum partido ou systema.
\section{Parcialidade}
\begin{itemize}
\item {Grp. gram.:f.}
\end{itemize}
Qualidade do que é parcial.
Partido, facção; paixão partidária.
\section{Parcialismo}
\begin{itemize}
\item {Grp. gram.:m.}
\end{itemize}
O mesmo que \textunderscore parcialidade\textunderscore .
\section{Parcializar}
\begin{itemize}
\item {Grp. gram.:v. t.}
\end{itemize}
Tornar parcial.
Associar a um bando ou facção.
\section{Parcialmente}
\begin{itemize}
\item {Grp. gram.:adv.}
\end{itemize}
De modo parcial; sem independencia.
Pouco a pouco; por partes.
\section{Parciário}
\begin{itemize}
\item {Grp. gram.:m.  e  adj.}
\end{itemize}
\begin{itemize}
\item {Proveniência:(Lat. \textunderscore partiarius\textunderscore )}
\end{itemize}
Aquelle que compartilha ou que tem parte em alguma coisa; quinhoeiro; participante.
Dizia-se especialmente do antigo systema de colonização, em que os frutos das terra eram repartidos entre os senhores e os agricultores.
Dizia-se do agricultor, que repartia os frutos das terras com o senhorio das mesmas. Cf. Herculano, \textunderscore Hist. de Port.\textunderscore , III, 299.
\section{Parcimónia}
\begin{itemize}
\item {Grp. gram.:f.}
\end{itemize}
\begin{itemize}
\item {Proveniência:(Lat. \textunderscore parcimonia\textunderscore )}
\end{itemize}
Qualidade do que é parco; acto de poupar; economia.
\section{Parcimoniosamente}
\begin{itemize}
\item {Grp. gram.:adv.}
\end{itemize}
De modo parcimonioso.
Com economia; moderadamente.
\section{Parcimonioso}
\begin{itemize}
\item {Grp. gram.:adj.}
\end{itemize}
\begin{itemize}
\item {Proveniência:(De \textunderscore parcimónia\textunderscore )}
\end{itemize}
Parco. Que economiza.
Frugal.
\section{Parcioneiro}
\begin{itemize}
\item {Grp. gram.:m.}
\end{itemize}
\begin{itemize}
\item {Utilização:Ant.}
\end{itemize}
\begin{itemize}
\item {Proveniência:(Do lat. \textunderscore partio\textunderscore )}
\end{itemize}
Cúmplice; parcial; membro de um bando ou de um partido.
\section{Parcíssimo}
\begin{itemize}
\item {Grp. gram.:adj.}
\end{itemize}
Muito parco.
\section{Parco}
\begin{itemize}
\item {Grp. gram.:adj.}
\end{itemize}
\begin{itemize}
\item {Proveniência:(Lat. \textunderscore parcus\textunderscore )}
\end{itemize}
Que economiza ou que poupa; poupador.
Frugal: \textunderscore um jantar parco\textunderscore .
\section{Parco}
\begin{itemize}
\item {Grp. gram.:m.}
\end{itemize}
\begin{itemize}
\item {Utilização:Des.}
\end{itemize}
O mesmo que \textunderscore parque\textunderscore . Cf. \textunderscore Viriato Trág.\textunderscore , XVIII, 19.
\section{Parda}
\begin{itemize}
\item {Grp. gram.:f.}
\end{itemize}
\begin{itemize}
\item {Utilização:Prov.}
\end{itemize}
\begin{itemize}
\item {Utilização:alg.}
\end{itemize}
\begin{itemize}
\item {Utilização:Prov.}
\end{itemize}
\begin{itemize}
\item {Utilização:trasm.}
\end{itemize}
\begin{itemize}
\item {Utilização:Prov.}
\end{itemize}
\begin{itemize}
\item {Utilização:dur.}
\end{itemize}
\begin{itemize}
\item {Proveniência:(De \textunderscore pardo\textunderscore )}
\end{itemize}
Planta papilionácea, (\textunderscore ervum monanthus\textunderscore ).
Aguardente de medronho.
O mesmo que \textunderscore lentilha\textunderscore .
O mesmo que \textunderscore maçarico\textunderscore , ave.
\section{Pardacento}
\begin{itemize}
\item {Grp. gram.:adj.}
\end{itemize}
\begin{itemize}
\item {Proveniência:(De \textunderscore pardaço\textunderscore )}
\end{itemize}
Tirante a pardo; pardo.
\section{Pardaço}
\begin{itemize}
\item {Grp. gram.:adj.}
\end{itemize}
\begin{itemize}
\item {Proveniência:(De \textunderscore pardo\textunderscore )}
\end{itemize}
O mesmo que \textunderscore pardacento\textunderscore .
\section{Pardaínha}
\begin{itemize}
\item {Grp. gram.:f.  e  adj.}
\end{itemize}
\begin{itemize}
\item {Proveniência:(De \textunderscore pardo\textunderscore )}
\end{itemize}
Variedade de uva.
\section{Pardaínho}
\begin{itemize}
\item {Grp. gram.:m.}
\end{itemize}
O mesmo que \textunderscore pardaínha\textunderscore .
Espécie de pardal, (\textunderscore fringilla doméstica\textunderscore ).
\section{Pardal}
\begin{itemize}
\item {Grp. gram.:m.}
\end{itemize}
\begin{itemize}
\item {Utilização:T. de Alcanena}
\end{itemize}
\begin{itemize}
\item {Utilização:Gír.}
\end{itemize}
\begin{itemize}
\item {Grp. gram.:Adj.}
\end{itemize}
\begin{itemize}
\item {Proveniência:(Do lat. \textunderscore pardalus\textunderscore )}
\end{itemize}
Pequeno pássaro conirostro, (\textunderscore passer domésticus\textunderscore ).
As partes pudendas da mulhér.
Espião policial.
Diz-se de uma variedade de uva tinta e ordinária do Minho.
\section{Pardalada}
\begin{itemize}
\item {Grp. gram.:f.}
\end{itemize}
Porção de pardaes.
\section{Pardal-da-Índia}
\begin{itemize}
\item {Grp. gram.:m.}
\end{itemize}
O mesmo que \textunderscore pardaloco\textunderscore .
\section{Pardal-do-monte}
\begin{itemize}
\item {Grp. gram.:m.}
\end{itemize}
O mesmo que \textunderscore piriz\textunderscore .
\section{Pardaleira}
\begin{itemize}
\item {Grp. gram.:f.}
\end{itemize}
\begin{itemize}
\item {Utilização:Prov.}
\end{itemize}
Terra sáfara, estéril ou com pouca vegetação.
\section{Pardaleja}
\begin{itemize}
\item {Grp. gram.:f.}
\end{itemize}
\begin{itemize}
\item {Utilização:Prov.}
\end{itemize}
O mesmo que \textunderscore pardaloca\textunderscore .
\section{Pardal-francês}
\begin{itemize}
\item {Grp. gram.:m.}
\end{itemize}
Espécie de pardal, (\textunderscore fringilla petronia\textunderscore ); piriz.
\section{Pardal-gírio}
\begin{itemize}
\item {Grp. gram.:m.}
\end{itemize}
O mesmo que \textunderscore piriz\textunderscore .
\section{Pardal-moirisco}
\begin{itemize}
\item {Grp. gram.:m.}
\end{itemize}
\begin{itemize}
\item {Utilização:T. da Bairrada}
\end{itemize}
Espécie de pardal, maior que os vulgares.
\section{Pardal-montês}
\begin{itemize}
\item {Grp. gram.:m.}
\end{itemize}
Espécie de pardal, (\textunderscore fringilla montana\textunderscore ).
\section{Pardaloca}
\begin{itemize}
\item {Grp. gram.:f.}
\end{itemize}
Fêmea do pardal:«\textunderscore ando doente do peito, és a causa do meu mal; diz o mesmo a pardaloca, a respeito do pardal\textunderscore ». (De uma canção popular)
\section{Pardaloco}
\begin{itemize}
\item {fónica:lô}
\end{itemize}
\begin{itemize}
\item {Grp. gram.:m.}
\end{itemize}
\begin{itemize}
\item {Utilização:T. da Bairrada}
\end{itemize}
O mesmo que \textunderscore dom-fafe\textunderscore .
\section{Pardal-real}
\begin{itemize}
\item {Grp. gram.:m.}
\end{itemize}
O mesmo que \textunderscore pica-porco\textunderscore .
\section{Pardantho}
\begin{itemize}
\item {Grp. gram.:m.}
\end{itemize}
\begin{itemize}
\item {Proveniência:(Do gr. \textunderscore pardos\textunderscore  + \textunderscore anthos\textunderscore )}
\end{itemize}
Gênero de plantas irídeas.
\section{Pardanto}
\begin{itemize}
\item {Grp. gram.:m.}
\end{itemize}
\begin{itemize}
\item {Proveniência:(Do gr. \textunderscore pardos\textunderscore  + \textunderscore anthos\textunderscore )}
\end{itemize}
Gênero de plantas irídeas.
\section{Pardau}
\textunderscore m.\textunderscore 
Antiga moeda da Índia portuguesa, de valor variável entre 220 e 1$250 reis.
\section{Pardau}
\begin{itemize}
\item {Grp. gram.:m.}
\end{itemize}
\begin{itemize}
\item {Utilização:Mad}
\end{itemize}
O mesmo que \textunderscore pardal-francês\textunderscore .
\section{Pardavasco}
\begin{itemize}
\item {Grp. gram.:m.}
\end{itemize}
\begin{itemize}
\item {Utilização:Bras}
\end{itemize}
Mestiço de negro e índio; índio meio amulatado, pardo escuro.
\section{Pardeeiro}
\begin{itemize}
\item {Grp. gram.:m.}
\end{itemize}
\begin{itemize}
\item {Proveniência:(Do lat. hyp. \textunderscore parietenarius\textunderscore ?)}
\end{itemize}
Casa em ruínas.
Edifício velho.
\section{Pardeja}
\begin{itemize}
\item {Grp. gram.:f.}
\end{itemize}
Fêmea do pardejo.
\section{Pardejar}
\begin{itemize}
\item {Grp. gram.:v. i.}
\end{itemize}
Fazer-se pardo o dia, ao findar; entardecer. Us. por Camillo.
\section{Pardejo}
\begin{itemize}
\item {Grp. gram.:m.}
\end{itemize}
\begin{itemize}
\item {Utilização:Prov.}
\end{itemize}
\begin{itemize}
\item {Utilização:minh.}
\end{itemize}
O mesmo que \textunderscore pardal\textunderscore .
\section{Pardela}
\begin{itemize}
\item {Grp. gram.:f.}
\end{itemize}
\begin{itemize}
\item {Proveniência:(De \textunderscore pardo\textunderscore )}
\end{itemize}
Gênero de aves aquáticas, cujas espécies são a pardela de bico preto, a pardela de bico branco e a pardela preta.
O mesmo que \textunderscore cagarra\textunderscore , (\textunderscore puffinus major\textunderscore , Temm.).
\section{Pardelha}
\begin{itemize}
\item {fónica:dê}
\end{itemize}
\begin{itemize}
\item {Grp. gram.:f.}
\end{itemize}
\begin{itemize}
\item {Utilização:Prov.}
\end{itemize}
\begin{itemize}
\item {Utilização:minh.}
\end{itemize}
Nome de duas espécies de peixes.
Fem. de \textunderscore pardelho\textunderscore .
\section{Pardelhas!}
\begin{itemize}
\item {fónica:dê}
\end{itemize}
\begin{itemize}
\item {Grp. gram.:interj.  e  adv.}
\end{itemize}
\begin{itemize}
\item {Utilização:Ant.}
\end{itemize}
Por Deus! em verdade! realmente!
(Cp. \textunderscore pardês!\textunderscore )
\section{Pardelho}
\begin{itemize}
\item {fónica:dê}
\end{itemize}
\begin{itemize}
\item {Grp. gram.:m.}
\end{itemize}
\begin{itemize}
\item {Utilização:Prov.}
\end{itemize}
\begin{itemize}
\item {Utilização:minh.}
\end{itemize}
O mesmo que \textunderscore pardal\textunderscore .
\section{Pardelhos}
\begin{itemize}
\item {fónica:dê}
\end{itemize}
\begin{itemize}
\item {Grp. gram.:m. pl.}
\end{itemize}
\begin{itemize}
\item {Utilização:Prov.}
\end{itemize}
\begin{itemize}
\item {Utilização:trasm.}
\end{itemize}
Rede de pesca.
\section{Pardelo}
\begin{itemize}
\item {fónica:dê}
\end{itemize}
\begin{itemize}
\item {Grp. gram.:m.}
\end{itemize}
\begin{itemize}
\item {Utilização:Ant.}
\end{itemize}
O mesmo que \textunderscore pardal\textunderscore .
\section{Pardenomancia}
\begin{itemize}
\item {Grp. gram.:f.}
\end{itemize}
Antiga e supposta arte de adivinhar se uma mulher está ou não virgem, empregando-se uma bebida, que ella não deve vomitar, ou cingindo-se-lhe ao pescoço uma fita, que não deve tirar-se facilmente por cima da cabeça, se a mulhér está pura. Cf. Castilho, \textunderscore Fastos\textunderscore , III, 323.
\section{Pardento}
\begin{itemize}
\item {Grp. gram.:adj.}
\end{itemize}
O mesmo que \textunderscore pardacento\textunderscore .
\section{Pardêos!}
\begin{itemize}
\item {Grp. gram.:interj.}
\end{itemize}
\begin{itemize}
\item {Utilização:Ant.}
\end{itemize}
O mesmo que \textunderscore pardês!\textunderscore . Cf. G. Vicente, III, 9.
\section{Pardês!}
\begin{itemize}
\item {Grp. gram.:interj.}
\end{itemize}
\begin{itemize}
\item {Utilização:Ant.}
\end{itemize}
O mesmo que \textunderscore pardelhas!\textunderscore .
(Provavelmente alter. de \textunderscore perdês\textunderscore , forma euphêmica do loc. \textunderscore per-Deus\textunderscore  = \textunderscore por Deus\textunderscore , como \textunderscore diacho\textunderscore  e \textunderscore dialho\textunderscore  são euphemismos de \textunderscore diabo\textunderscore )
\section{Pardexo}
\begin{itemize}
\item {Grp. gram.:m.}
\end{itemize}
\begin{itemize}
\item {Utilização:Des.}
\end{itemize}
O mesmo que \textunderscore pardieiro\textunderscore . Cf. Filinto, XIII, 153; XVI, p. 56.
(Contr. de \textunderscore paredeiro\textunderscore , de \textunderscore parede\textunderscore )
\section{Pardicas!}
\begin{itemize}
\item {Grp. gram.:interj.}
\end{itemize}
\begin{itemize}
\item {Utilização:Ant.}
\end{itemize}
O mesmo que \textunderscore pardelhas!\textunderscore ? Cf. G. Vicente, I, 262.
\section{Pardieiro}
\begin{itemize}
\item {Grp. gram.:m.}
\end{itemize}
\begin{itemize}
\item {Proveniência:(Do lat. hyp. \textunderscore parietenarius\textunderscore ?)}
\end{itemize}
Casa em ruínas.
Edifício velho.
\section{Pardilheira}
\begin{itemize}
\item {Grp. gram.:f.}
\end{itemize}
\begin{itemize}
\item {Proveniência:(De \textunderscore pardilho\textunderscore )}
\end{itemize}
Ave palmípede, (\textunderscore anas angustirostri\textunderscore ).
\section{Pardilho}
\begin{itemize}
\item {Grp. gram.:adj.}
\end{itemize}
\begin{itemize}
\item {Utilização:Des.}
\end{itemize}
\begin{itemize}
\item {Grp. gram.:M.}
\end{itemize}
\begin{itemize}
\item {Utilização:Ant.}
\end{itemize}
O mesmo que \textunderscore pardacento\textunderscore .
Espécie de pano pardo. Cf. Pant. de Aveiro, \textunderscore Itiner.\textunderscore  8, (2.^a ed.).
\section{Pardinho}
\begin{itemize}
\item {Grp. gram.:m.}
\end{itemize}
\begin{itemize}
\item {Utilização:Prov.}
\end{itemize}
\begin{itemize}
\item {Utilização:trasm.}
\end{itemize}
\begin{itemize}
\item {Utilização:Bras. do Rio}
\end{itemize}
O mesmo que \textunderscore cartaxo\textunderscore .
Papa-capim pardo.
\section{Pardizela}
\begin{itemize}
\item {Grp. gram.:f.}
\end{itemize}
\begin{itemize}
\item {Utilização:Prov.}
\end{itemize}
\begin{itemize}
\item {Utilização:minh.}
\end{itemize}
Agulheta de ferro, o mesmo que \textunderscore tendilha\textunderscore .
\section{Pardo}
\begin{itemize}
\item {Grp. gram.:adj.}
\end{itemize}
\begin{itemize}
\item {Grp. gram.:Loc.}
\end{itemize}
\begin{itemize}
\item {Utilização:pop.}
\end{itemize}
\begin{itemize}
\item {Grp. gram.:M.}
\end{itemize}
\begin{itemize}
\item {Utilização:Prov.}
\end{itemize}
\begin{itemize}
\item {Utilização:minh.}
\end{itemize}
\begin{itemize}
\item {Proveniência:(Do lat. \textunderscore pallidus\textunderscore ?)}
\end{itemize}
Que tem côr intermédia a preto e branco, quási escuro.
\textunderscore Vêr-se em calças pardas\textunderscore , vêr-se muito embaraçado, vêr-se em talas, em apuros.
O mesmo que \textunderscore mulato\textunderscore ^1.
Pássaro das cercanias do Pôrto, (\textunderscore larus cinereus\textunderscore , Briss.).
O mesmo que \textunderscore burel\textunderscore .
\section{Pardo}
\begin{itemize}
\item {Grp. gram.:m.}
\end{itemize}
\begin{itemize}
\item {Utilização:Ant.}
\end{itemize}
Parque; coitada.
(Metáth. de \textunderscore prado\textunderscore )
\section{Pardo}
\begin{itemize}
\item {Grp. gram.:m.}
\end{itemize}
O mesmo que \textunderscore leopardo\textunderscore . Cf. B. Pereira, \textunderscore Prosodia\textunderscore , vb. \textunderscore tigris\textunderscore .
\section{Pardoca}
\begin{itemize}
\item {Grp. gram.:f.}
\end{itemize}
O mesmo que \textunderscore pardaloca\textunderscore .
\section{Pardo-africano}
\begin{itemize}
\item {Grp. gram.:m.}
\end{itemize}
Variedade de maçan.
\section{Pardo-bravio}
\begin{itemize}
\item {Grp. gram.:m.}
\end{itemize}
Variedade de maçan.
\section{Pardo-doce}
\begin{itemize}
\item {Grp. gram.:m.}
\end{itemize}
Variedade de maçan.
\section{Pardo-lindo}
\begin{itemize}
\item {Grp. gram.:m.}
\end{itemize}
O mesmo que \textunderscore lindo-pardo\textunderscore .
\section{Pardo-mato}
\begin{itemize}
\item {Grp. gram.:m.}
\end{itemize}
Variedade de maçan.
\section{Pardo-violeta}
\begin{itemize}
\item {Grp. gram.:adj.}
\end{itemize}
Que participa do pardo e da côr da violeta.
\section{Pardusco}
\begin{itemize}
\item {Grp. gram.:adj.}
\end{itemize}
O mesmo que \textunderscore pardacento\textunderscore .
\section{Párea}
\begin{itemize}
\item {Grp. gram.:f.}
\end{itemize}
Régua, com que se mede a altura das pipas e tonéis.
(Cp. \textunderscore pareia\textunderscore  e \textunderscore pareio\textunderscore )
\section{Pareador}
\begin{itemize}
\item {Grp. gram.:m.}
\end{itemize}
Aquelle que pareia.
\section{Parear}
\begin{itemize}
\item {Grp. gram.:v. t.}
\end{itemize}
\begin{itemize}
\item {Utilização:Prov.}
\end{itemize}
\begin{itemize}
\item {Utilização:Ant.}
\end{itemize}
Medir com a pareia.
Emparelhar, formar par.
(Cp. cast. \textunderscore parêar\textunderscore , de par)
\section{Páreas}
\begin{itemize}
\item {Grp. gram.:f. pl.}
\end{itemize}
\begin{itemize}
\item {Utilização:T. de Turquel}
\end{itemize}
Tributo, pago antigamente por um Soberano ou Estado a outro, em reconhecimento de vassalagem.
Satisfação, desaggravo: \textunderscore tomar páreas a alguém\textunderscore .
(Relaciona-se com o lat. \textunderscore parere\textunderscore , obedecer?)
\section{Páreas}
\begin{itemize}
\item {Grp. gram.:f. pl.}
\end{itemize}
\begin{itemize}
\item {Grp. gram.:Loc.}
\end{itemize}
\begin{itemize}
\item {Utilização:Loc. da Bairrada.}
\end{itemize}
\begin{itemize}
\item {Proveniência:(Do rad. do lat. \textunderscore parere\textunderscore )}
\end{itemize}
Membrana, que envolve o féto, ou parte do cordão umbilical, que fica na madre depois da expulsão do féto; secundinas.
\textunderscore Tirar as páreas\textunderscore , pedir explicações, provocar razões.
\section{Parecença}
\begin{itemize}
\item {Grp. gram.:f.}
\end{itemize}
\begin{itemize}
\item {Proveniência:(De \textunderscore parere\textunderscore , parir)}
\end{itemize}
Qualidade de quem se parece com outro; semelhança.
\section{Parecente}
\begin{itemize}
\item {Grp. gram.:adj.}
\end{itemize}
\begin{itemize}
\item {Utilização:Des.}
\end{itemize}
\begin{itemize}
\item {Proveniência:(De \textunderscore parecer\textunderscore )}
\end{itemize}
Semelhante, parecido.
\section{Parecer}
\begin{itemize}
\item {Grp. gram.:v. i.}
\end{itemize}
\begin{itemize}
\item {Grp. gram.:V. p.}
\end{itemize}
\begin{itemize}
\item {Grp. gram.:M.}
\end{itemize}
\begin{itemize}
\item {Proveniência:(Do lat. hypoth. \textunderscore parescere\textunderscore )}
\end{itemize}
Apresentar-se, mostrar-se:«\textunderscore ...quando a frota pareceu diante dos muros...\textunderscore »Azurara, \textunderscore D. João I\textunderscore , LXIX.
Têr semelhança.
Dar ares de.
Sêr apparentemente: \textunderscore parece verdade e é mentira\textunderscore .
Afigurar-se.
Representar-se ao entendimento.
Sêr semelhante ou quási semelhante.
Têr a apparência de alguém ou de alguma coisa: \textunderscore o filho parece-se com o pai\textunderscore .
Dar ares.
Apparência.
Aspecto physionómico: \textunderscore tens bom parecer\textunderscore .
Opinião; conceito: \textunderscore formar parecer\textunderscore .
\section{Parechema}
\begin{itemize}
\item {fónica:quê}
\end{itemize}
\begin{itemize}
\item {Grp. gram.:m.}
\end{itemize}
\begin{itemize}
\item {Utilização:Gram.}
\end{itemize}
\begin{itemize}
\item {Proveniência:(Gr. \textunderscore parekhema\textunderscore )}
\end{itemize}
Defeito de linguagem, que consiste em collocar ao lado de uma sýllaba outra sýllaba com o mesmo som, como em \textunderscore tenra rama\textunderscore , \textunderscore tropa parada\textunderscore , \textunderscore tronco coberto\textunderscore , etc.
\section{Parecido}
\begin{itemize}
\item {Grp. gram.:adj.}
\end{itemize}
\begin{itemize}
\item {Proveniência:(De \textunderscore parecer\textunderscore )}
\end{itemize}
Que se parece a ou com; semelhante.
\section{Parecis}
\begin{itemize}
\item {Grp. gram.:m. pl.}
\end{itemize}
Antiga nação de Índios do Brasil, ao norte de Mato-Grosso, o mesmo que \textunderscore paricis\textunderscore .
\section{Parectase}
\begin{itemize}
\item {Grp. gram.:f.}
\end{itemize}
\begin{itemize}
\item {Utilização:Gram.}
\end{itemize}
Adjuncção de elementos phónicos intermédios, para tornar euphónica uma palavra. Cf. João Ribeiro, \textunderscore Diccion. Gram.\textunderscore 
\section{Paredão}
\begin{itemize}
\item {Grp. gram.:m.}
\end{itemize}
Grande parede; muralha.
\section{Parede}
\begin{itemize}
\item {fónica:parê}
\end{itemize}
\begin{itemize}
\item {Grp. gram.:f.}
\end{itemize}
\begin{itemize}
\item {Utilização:Ext.}
\end{itemize}
\begin{itemize}
\item {Utilização:Escol.}
\end{itemize}
\begin{itemize}
\item {Grp. gram.:Loc. adv.}
\end{itemize}
\begin{itemize}
\item {Proveniência:(Do lat. \textunderscore paries\textunderscore , \textunderscore parietis\textunderscore )}
\end{itemize}
Muro que, geralmente, sustenta o madeiramento de um edifício.
Muro; tapume.
Vedação de qualquer espaço.
Substância córnea, que envolve as partes vivas do pé do cavallo. Cf. Leon, \textunderscore Arte de Ferrar\textunderscore , 13.
O mesmo ou melhor que \textunderscore greve\textunderscore .
\textunderscore Fazer parede\textunderscore , fazer greve; associar-se com outrem para certos fins.
Faltar á aula, cabular.
\textunderscore Paredes meias\textunderscore , com a simples separação, que uma parede ou paredes determinam entre vizinhos: \textunderscore mora paredes meias comigo\textunderscore .
\section{Paredeiro}
\begin{itemize}
\item {Grp. gram.:m.}
\end{itemize}
\begin{itemize}
\item {Utilização:Ant.}
\end{itemize}
\begin{itemize}
\item {Proveniência:(De \textunderscore parede\textunderscore )}
\end{itemize}
O mesmo que \textunderscore pardieiro\textunderscore .
\section{Paredista}
\begin{itemize}
\item {Grp. gram.:adj.}
\end{itemize}
\begin{itemize}
\item {Grp. gram.:M.}
\end{itemize}
Relativo a parede ou greve: \textunderscore o movimento paredista dos cocheiros de praça\textunderscore .
Aquelle que faz parede ou greve.
\section{Paregoria}
\begin{itemize}
\item {Grp. gram.:f.}
\end{itemize}
\begin{itemize}
\item {Proveniência:(Lat. \textunderscore paregoria\textunderscore )}
\end{itemize}
Qualidade ou acção do que é paregórico.
\section{Paregórico}
\begin{itemize}
\item {Grp. gram.:adj.}
\end{itemize}
\begin{itemize}
\item {Utilização:Med.}
\end{itemize}
\begin{itemize}
\item {Proveniência:(Lat. \textunderscore paregoricus\textunderscore )}
\end{itemize}
Que suaviza ou acalma dores.
Calmante; anódyno.
\section{Pareia}
\begin{itemize}
\item {Grp. gram.:f.}
\end{itemize}
O mesmo ou melhor que \textunderscore párea\textunderscore .--Vejo registadas as duas fórmas; inclino-me porém a que só uma é authêntica e que será \textunderscore pareia\textunderscore , de \textunderscore parear\textunderscore , de \textunderscore par\textunderscore .
Régua, com que se mede a altura das pipas e tonéis.
(Cp. \textunderscore párea\textunderscore  e \textunderscore pareio\textunderscore )
\section{Pareio}
\begin{itemize}
\item {Grp. gram.:m.}
\end{itemize}
Acto de parear.--É forma, que provavelmente deve substituir o \textunderscore páreo\textunderscore , equivocamente registado pelos nossos diccionaristas. Unicamente Viterbo, \textunderscore Elucid.\textunderscore , regista \textunderscore pareio\textunderscore  ao lado de \textunderscore páreo\textunderscore .
Cp. \textunderscore páreo\textunderscore .
\section{Parelha}
\begin{itemize}
\item {fónica:parê}
\end{itemize}
\begin{itemize}
\item {Grp. gram.:f.}
\end{itemize}
\begin{itemize}
\item {Utilização:Fam.}
\end{itemize}
\begin{itemize}
\item {Utilização:Bras. do N}
\end{itemize}
\begin{itemize}
\item {Proveniência:(Do lat. \textunderscore parilia\textunderscore )}
\end{itemize}
Par de alguns animaes, especialmente muares ou cavallares.
Um par: \textunderscore uma parelha de coices\textunderscore .
Pessôa ou coisa semelhante a outra.
Cepo de carpinteiro, com um ou dois erros, para abrir o filete, com que uma tábua há de emparelhar com outra.
\textunderscore Pegar parelha\textunderscore , correr o pareio, correrem dois cavalleiros a par, para vêr qual das cavalgaduras anda mais.
\section{Parelhamente}
\begin{itemize}
\item {Grp. gram.:adv.}
\end{itemize}
\begin{itemize}
\item {Proveniência:(De \textunderscore parelho\textunderscore )}
\end{itemize}
Semelhantemente; a par; também.
\section{Parelhão}
\begin{itemize}
\item {Grp. gram.:m.}
\end{itemize}
\begin{itemize}
\item {Utilização:Prov.}
\end{itemize}
\begin{itemize}
\item {Utilização:trasm.}
\end{itemize}
Insecto, que come a fôlha da videira.
\section{Parelhar}
\begin{itemize}
\item {Grp. gram.:v. t.}
\end{itemize}
\begin{itemize}
\item {Utilização:Gír.}
\end{itemize}
\begin{itemize}
\item {Proveniência:(De \textunderscore parelha\textunderscore )}
\end{itemize}
Divertir.
\section{Parelheiro}
\begin{itemize}
\item {Grp. gram.:adj.}
\end{itemize}
\begin{itemize}
\item {Utilização:Bras}
\end{itemize}
\begin{itemize}
\item {Proveniência:(De \textunderscore parelha\textunderscore )}
\end{itemize}
Diz-se do cavallo, que na marcha ou corrida emparelha com outro.
\section{Parelho}
\begin{itemize}
\item {fónica:parê}
\end{itemize}
\begin{itemize}
\item {Grp. gram.:adj.}
\end{itemize}
\begin{itemize}
\item {Grp. gram.:M.}
\end{itemize}
\begin{itemize}
\item {Utilização:Prov.}
\end{itemize}
\begin{itemize}
\item {Utilização:beir.}
\end{itemize}
\begin{itemize}
\item {Proveniência:(Do lat. hyp. \textunderscore pariculus\textunderscore , de \textunderscore par\textunderscore )}
\end{itemize}
Que se fórma de partes iguaes.
Semelhante aos da sua espécie.
Que vai ou anda a par: \textunderscore estes bois são pouco parelhos\textunderscore .
Cada uma das duas fiadas de pedra parallelas, sôbre que assentam as capas dos alvanéis ou aqueductos de minas, e entre as quaes está o rêgo, por onde corre a água.
\section{Parélio}
\begin{itemize}
\item {Grp. gram.:m.}
\end{itemize}
\begin{itemize}
\item {Proveniência:(Lat. \textunderscore parhelion\textunderscore )}
\end{itemize}
Imagem do Sol, reflectida numa nuvem.
\section{Parêmbole}
\begin{itemize}
\item {Grp. gram.:f.}
\end{itemize}
\begin{itemize}
\item {Utilização:Gram.}
\end{itemize}
\begin{itemize}
\item {Proveniência:(Lat. \textunderscore parembole\textunderscore )}
\end{itemize}
Espécie de parênthese, em que o sentido da phrase incidente tem relação directa com o assumpto da phrase principal.
\section{Parêmia}
\begin{itemize}
\item {Grp. gram.:f.}
\end{itemize}
\begin{itemize}
\item {Proveniência:(Lat. \textunderscore paroemia\textunderscore )}
\end{itemize}
Curta allegoria ou provérbio.
\section{Paremíaco}
\begin{itemize}
\item {Grp. gram.:adj.}
\end{itemize}
\begin{itemize}
\item {Proveniência:(Gr. \textunderscore paroimiakos\textunderscore )}
\end{itemize}
Diz-se de um verso grego ou latino, que póde considerar-se como os três últimos pés do hexâmetro, precedidos de uma sýllaba longa ou duas breves.
\section{Paremiógrafo}
\begin{itemize}
\item {Grp. gram.:m.}
\end{itemize}
\begin{itemize}
\item {Proveniência:(Do gr. \textunderscore paroimos\textunderscore  + \textunderscore graphein\textunderscore )}
\end{itemize}
Autor ou coleccionador de provérbios.
\section{Paremiógrapho}
\begin{itemize}
\item {Grp. gram.:m.}
\end{itemize}
\begin{itemize}
\item {Proveniência:(Do gr. \textunderscore paroimos\textunderscore  + \textunderscore graphein\textunderscore )}
\end{itemize}
Autor ou colleccionador de provérbios.
\section{Paremiologia}
\begin{itemize}
\item {Grp. gram.:f.}
\end{itemize}
\begin{itemize}
\item {Proveniência:(Do gr. \textunderscore paroimia\textunderscore  + \textunderscore logos\textunderscore )}
\end{itemize}
Collecção de provérbios.
Tratado á cêrca de provérbios.
\section{Paremptose}
\begin{itemize}
\item {Grp. gram.:f.}
\end{itemize}
\begin{itemize}
\item {Utilização:Gram.}
\end{itemize}
\begin{itemize}
\item {Proveniência:(Gr. \textunderscore paremptosis\textunderscore )}
\end{itemize}
Espécie de epênthese, que consiste em introduzir-se numa palavra uma consoante que não fórma sýllaba, como em \textunderscore succo\textunderscore  ou \textunderscore lettra\textunderscore , por \textunderscore suco\textunderscore  ou \textunderscore letra\textunderscore .
\section{Parencéfalo}
\begin{itemize}
\item {Grp. gram.:m.}
\end{itemize}
\begin{itemize}
\item {Utilização:Anat.}
\end{itemize}
\begin{itemize}
\item {Proveniência:(Do gr. \textunderscore para\textunderscore  + \textunderscore enkephalos\textunderscore )}
\end{itemize}
Um dos nomes do cerebelo.
\section{Parencefalocele}
\begin{itemize}
\item {Grp. gram.:m.}
\end{itemize}
\begin{itemize}
\item {Utilização:Med.}
\end{itemize}
\begin{itemize}
\item {Proveniência:(Do gr. \textunderscore parenkephalos\textunderscore  + \textunderscore kele\textunderscore )}
\end{itemize}
Tumor, que sái por uma abertura do ocipício.
\section{Parencéphalo}
\begin{itemize}
\item {Grp. gram.:m.}
\end{itemize}
\begin{itemize}
\item {Utilização:Anat.}
\end{itemize}
\begin{itemize}
\item {Proveniência:(Do gr. \textunderscore para\textunderscore  + \textunderscore enkephalos\textunderscore )}
\end{itemize}
Um dos nomes do cerebello.
\section{Parencephalocele}
\begin{itemize}
\item {Grp. gram.:m.}
\end{itemize}
\begin{itemize}
\item {Utilização:Med.}
\end{itemize}
\begin{itemize}
\item {Proveniência:(Do gr. \textunderscore parenkephalos\textunderscore  + \textunderscore kele\textunderscore )}
\end{itemize}
Tumor, que sái por uma abertura do occipício.
\section{Parênchyma}
\begin{itemize}
\item {fónica:qui}
\end{itemize}
\begin{itemize}
\item {Grp. gram.:m.}
\end{itemize}
\begin{itemize}
\item {Utilização:Anat.}
\end{itemize}
\begin{itemize}
\item {Utilização:Bot.}
\end{itemize}
\begin{itemize}
\item {Proveniência:(Gr. \textunderscore parenkhuma\textunderscore )}
\end{itemize}
Tecido próprio das vísceras e das glândulas.
Tecido utricular dos vegetaes, também chamado polpa.
\section{Parenchymatoso}
\begin{itemize}
\item {fónica:qui}
\end{itemize}
\begin{itemize}
\item {Grp. gram.:adj.}
\end{itemize}
Relativo ao parênchyma.
\section{Parênchymo}
\begin{itemize}
\item {fónica:qui}
\end{itemize}
\begin{itemize}
\item {Grp. gram.:m.}
\end{itemize}
O mesmo que \textunderscore parênchyma\textunderscore .
\section{Parénese}
\begin{itemize}
\item {Grp. gram.:f.}
\end{itemize}
\begin{itemize}
\item {Proveniência:(Gr. \textunderscore parainesis\textunderscore )}
\end{itemize}
Discurso moral.
\section{Parenética}
\begin{itemize}
\item {Grp. gram.:f.}
\end{itemize}
Eloquência sagrada.
(Fem. de \textunderscore parenético\textunderscore )
\section{Parenético}
\begin{itemize}
\item {Grp. gram.:adj.}
\end{itemize}
Relativo á parénese.
\section{Parênquima}
\begin{itemize}
\item {Grp. gram.:m.}
\end{itemize}
\begin{itemize}
\item {Utilização:Anat.}
\end{itemize}
\begin{itemize}
\item {Utilização:Bot.}
\end{itemize}
\begin{itemize}
\item {Proveniência:(Gr. \textunderscore parenkhuma\textunderscore )}
\end{itemize}
Tecido próprio das vísceras e das glândulas.
Tecido utricular dos vegetaes, também chamado polpa.
\section{Parenquimatoso}
\begin{itemize}
\item {Grp. gram.:adj.}
\end{itemize}
Relativo ao parênquima.
\section{Parênquimo}
\begin{itemize}
\item {Grp. gram.:m.}
\end{itemize}
O mesmo que \textunderscore parênquima\textunderscore .
\section{Parenta}
(flexão fem. de \textunderscore parente\textunderscore )
\section{Parental}
\begin{itemize}
\item {Grp. gram.:adj.}
\end{itemize}
\begin{itemize}
\item {Grp. gram.:Pl.}
\end{itemize}
\begin{itemize}
\item {Proveniência:(Lat. \textunderscore parentalis\textunderscore )}
\end{itemize}
Relativo a pai e mãe:«\textunderscore ...porque a não pungem nem abrangem... os parentaes exemplos?\textunderscore »Filinto, II, 299.
Festas fúnebres, em honra dos pais, entre os Romanos. Cf. Castilho, \textunderscore Fastos\textunderscore , I, 135.
\section{Parentar}
\begin{itemize}
\item {Grp. gram.:v. t.}
\end{itemize}
(V. \textunderscore aparentar\textunderscore ^1)
\section{Parente}
\begin{itemize}
\item {Grp. gram.:m.}
\end{itemize}
\begin{itemize}
\item {Grp. gram.:Adj.}
\end{itemize}
\begin{itemize}
\item {Utilização:Fig.}
\end{itemize}
\begin{itemize}
\item {Proveniência:(Lat. \textunderscore parens\textunderscore )}
\end{itemize}
Indivíduo que, em relação a outro ou outros, pertence á mesma família.
Que está ligado a outrem por vínculos de sangue.
Que pertence á mesma família.
Semelhante, parecido.
\section{Parentear}
\begin{itemize}
\item {Grp. gram.:v. i.}
\end{itemize}
\begin{itemize}
\item {Utilização:Ant.}
\end{itemize}
\begin{itemize}
\item {Utilização:Fam.}
\end{itemize}
Sêr parente, têr parentesco.
\section{Parenteira}
\begin{itemize}
\item {Grp. gram.:f.}
\end{itemize}
\begin{itemize}
\item {Utilização:Ant.}
\end{itemize}
O mesmo que \textunderscore parentela\textunderscore . Cf. Sim. Mach., f. 29, v.^o.
\section{Parenteiro}
\begin{itemize}
\item {Grp. gram.:m.  e  adj.}
\end{itemize}
\begin{itemize}
\item {Utilização:Des.}
\end{itemize}
\begin{itemize}
\item {Proveniência:(De \textunderscore parente\textunderscore )}
\end{itemize}
O que é dedicado aos seus parentes ou os proteje:«\textunderscore ...eleitores..., o prior escolherá sempre aquelles que forem menos parenteiros.\textunderscore »(De um doc. do séc. XVI, cit. por S. de Frias, \textunderscore Pombeiro\textunderscore , 199)
\section{Parentela}
\begin{itemize}
\item {Grp. gram.:f.}
\end{itemize}
\begin{itemize}
\item {Proveniência:(De \textunderscore parente\textunderscore )}
\end{itemize}
Os parentes, considerados collectivamente.
\section{Parentesco}
\begin{itemize}
\item {fónica:tês}
\end{itemize}
\begin{itemize}
\item {Grp. gram.:m.}
\end{itemize}
\begin{itemize}
\item {Utilização:Fig.}
\end{itemize}
Qualidade de parente.
Semelhança; connexão.
\section{Parêntese}
\begin{itemize}
\item {Grp. gram.:m.}
\end{itemize}
\begin{itemize}
\item {Proveniência:(Gr. \textunderscore parenthesis\textunderscore )}
\end{itemize}
Phrase, que fórma sentido distinto e separado do sentido do período, em que ella está interposta.
Sinaes, que encerram essa phrase.
\section{Parêntesis}
\begin{itemize}
\item {Grp. gram.:m.}
\end{itemize}
(V.parênthese)
\section{Parentético}
\begin{itemize}
\item {Grp. gram.:adj.}
\end{itemize}
Relativo a parênthese.
\section{Parênthese}
\begin{itemize}
\item {Grp. gram.:m.}
\end{itemize}
\begin{itemize}
\item {Proveniência:(Gr. \textunderscore parenthesis\textunderscore )}
\end{itemize}
Phrase, que fórma sentido distinto e separado do sentido do período, em que ella está interposta.
Sinaes, que encerram essa phrase.
\section{Parênthesis}
\begin{itemize}
\item {Grp. gram.:m.}
\end{itemize}
(V.parênthese)
\section{Parenthético}
\begin{itemize}
\item {Grp. gram.:adj.}
\end{itemize}
Relativo a parênthese.
\section{Páreo}
\begin{itemize}
\item {Grp. gram.:m.}
\end{itemize}
\begin{itemize}
\item {Utilização:Ant.}
\end{itemize}
\begin{itemize}
\item {Utilização:Fig.}
\end{itemize}
Corrida a cavallo ou a pé, em que dois indivíduos partiam a par, para ganhar um prêmio o que primeiro chegasse á meta.
O prêmio dessas corridas.
Emulação, disputa. Cf Rui de Pina, \textunderscore Chrónica de D. João II\textunderscore .--Todos os diccion. port. do meu conhecimento, desde Moraes, fazem proparoxýtono o voc., accentuando-o na 1.^a sýllaba, e suppondo-o derivado de \textunderscore par\textunderscore . É provavelmente êrro, resultante, sobretudo, de que nas velhas chrónicas escasseavam accentos gráphicos, dando lugar ao arbítrio. Em português, como no castelhano, temos \textunderscore parear\textunderscore , (collocar a par); e, assim como no cast. o substantivo verbal é \textunderscore parêo\textunderscore , substantivo verbal português deverá sêr \textunderscore pareio\textunderscore , (forma ant. \textunderscore parêo\textunderscore ). Cp. \textunderscore recear\textunderscore , \textunderscore passear\textunderscore , etc. e os ant. subst. verbaes \textunderscore recêo\textunderscore , \textunderscore passêo\textunderscore , etc., hoje \textunderscore receio\textunderscore , \textunderscore passeio\textunderscore , etc. Portanto, \textunderscore pareio\textunderscore , não \textunderscore páreo\textunderscore .
\section{Parequema}
\begin{itemize}
\item {Grp. gram.:m.}
\end{itemize}
\begin{itemize}
\item {Utilização:Gram.}
\end{itemize}
\begin{itemize}
\item {Proveniência:(Gr. \textunderscore parekhema\textunderscore )}
\end{itemize}
Defeito de linguagem, que consiste em collocar ao lado de uma sílaba outra sílaba com o mesmo som, como em \textunderscore tenra rama\textunderscore , \textunderscore tropa parada\textunderscore , \textunderscore tronco coberto\textunderscore , etc.
\section{Parere}
\begin{itemize}
\item {Grp. gram.:m.}
\end{itemize}
\begin{itemize}
\item {Utilização:Jur.}
\end{itemize}
\begin{itemize}
\item {Utilização:des.}
\end{itemize}
\begin{itemize}
\item {Proveniência:(T. it.)}
\end{itemize}
Voto de negociante sôbre questões de commércio.
Parecer commercial. Cf. F. Borges, \textunderscore Diccion. Jur.\textunderscore 
\section{Parergo}
\begin{itemize}
\item {Grp. gram.:m.}
\end{itemize}
\begin{itemize}
\item {Utilização:Ant.}
\end{itemize}
\begin{itemize}
\item {Proveniência:(Lat. \textunderscore parergon\textunderscore )}
\end{itemize}
Accrescentamento.
Ornato.
\section{Pares-e-nónes}
\begin{itemize}
\item {Grp. gram.:m.}
\end{itemize}
\begin{itemize}
\item {Utilização:Des.}
\end{itemize}
Jôgo popular, o mesmo que \textunderscore par-ou-pernão\textunderscore . Cf. B. Pereira, \textunderscore Prosódia\textunderscore , vb. \textunderscore artiasinus\textunderscore .
\section{Paresia}
\begin{itemize}
\item {Grp. gram.:f.}
\end{itemize}
\begin{itemize}
\item {Proveniência:(Gr. \textunderscore paresis\textunderscore )}
\end{itemize}
Paralysia incompleta, isto é, paralysia de nervo ou músculo que não perdeu inteiramente a sensibilidade e o movimento.
\section{Paresiação}
\begin{itemize}
\item {Grp. gram.:f.}
\end{itemize}
Acto ou effeito de paresiar.
\section{Paresiar}
\begin{itemize}
\item {Grp. gram.:v. t.}
\end{itemize}
\begin{itemize}
\item {Utilização:Neol.}
\end{itemize}
Causar paresia a.
\section{Parestesia}
\begin{itemize}
\item {Grp. gram.:f.}
\end{itemize}
\begin{itemize}
\item {Utilização:Med.}
\end{itemize}
\begin{itemize}
\item {Proveniência:(Do gr. \textunderscore para\textunderscore  + \textunderscore aisthesis\textunderscore )}
\end{itemize}
Desordem na sensibilidade, caracterizada por sensações anormaes e alucinações sensoriaes.
\section{Parestésico}
\begin{itemize}
\item {Grp. gram.:adj.}
\end{itemize}
Relativo á parestesia.
Diz-se do medicamento contra a parestesia.
\section{Paresthesia}
\begin{itemize}
\item {Grp. gram.:f.}
\end{itemize}
\begin{itemize}
\item {Utilização:Med.}
\end{itemize}
\begin{itemize}
\item {Proveniência:(Do gr. \textunderscore para\textunderscore  + \textunderscore aisthesis\textunderscore )}
\end{itemize}
Desordem na sensibilidade, caracterizada por sensações anormaes e allucinações sensoriaes.
\section{Paresthésico}
\begin{itemize}
\item {Grp. gram.:adj.}
\end{itemize}
Relativo á paresthesia.
Diz-se do medicamento contra a paresthesia.
\section{Parga}
\begin{itemize}
\item {Grp. gram.:f.}
\end{itemize}
\begin{itemize}
\item {Utilização:T. de Turquel}
\end{itemize}
Pilha, ruma.
\section{Pargasita}
\begin{itemize}
\item {Grp. gram.:f.}
\end{itemize}
\begin{itemize}
\item {Utilização:Miner.}
\end{itemize}
\begin{itemize}
\item {Proveniência:(De \textunderscore Pargas\textunderscore , n. p. de uma ilha na costa da Finlândia)}
\end{itemize}
Silicato de magnésia, cal e alumina.
\section{Pargasito}
\begin{itemize}
\item {Grp. gram.:m.}
\end{itemize}
O mesmo ou melhor que \textunderscore pargasita\textunderscore .
\section{Pargo}
\begin{itemize}
\item {Grp. gram.:m.}
\end{itemize}
\begin{itemize}
\item {Proveniência:(Do lat. \textunderscore sparus\textunderscore )}
\end{itemize}
Peixe muito apreciado, da fam. dos esparóides.
\section{Pargo-de-mitra}
\begin{itemize}
\item {Grp. gram.:m.}
\end{itemize}
\begin{itemize}
\item {Utilização:Prov.}
\end{itemize}
\begin{itemize}
\item {Utilização:alg.}
\end{itemize}
O mesmo que \textunderscore capatão\textunderscore .
\section{Pargo-de-morro}
\begin{itemize}
\item {Grp. gram.:m.}
\end{itemize}
O mesmo que \textunderscore capatão\textunderscore .
\section{Pargueiro}
\begin{itemize}
\item {Grp. gram.:m.}
\end{itemize}
\begin{itemize}
\item {Utilização:Açor}
\end{itemize}
\begin{itemize}
\item {Proveniência:(De \textunderscore pargo\textunderscore )}
\end{itemize}
Espécie de anzol, revestido de arame enrolado para que os peixes o não rompam com os dentes.
\section{Parguete}
\begin{itemize}
\item {fónica:guê}
\end{itemize}
\begin{itemize}
\item {Grp. gram.:m.}
\end{itemize}
Pequeno pargo.
\section{Parhálio}
\begin{itemize}
\item {Grp. gram.:adj.}
\end{itemize}
(V.parálio)
\section{Parhélio}
\begin{itemize}
\item {Grp. gram.:m.}
\end{itemize}
(V.parélio)
\section{Pari}
\begin{itemize}
\item {Grp. gram.:m.}
\end{itemize}
\begin{itemize}
\item {Utilização:Bras. do N}
\end{itemize}
\begin{itemize}
\item {Proveniência:(T. tupi)}
\end{itemize}
Tapume ou tecido de varas, erguido no rio para pesca.
\section{Pária}
\begin{itemize}
\item {Grp. gram.:m.}
\end{itemize}
\begin{itemize}
\item {Utilização:Ext.}
\end{itemize}
Designação de cada indivíduo de uma raça definida, mas desprezada entre os Índios.
Homem desprezado ou repellido pelos outros.
(Do tamul \textunderscore parai\textunderscore )
\section{Pariá}
\begin{itemize}
\item {Grp. gram.:m.}
\end{itemize}
\begin{itemize}
\item {Utilização:Ext.}
\end{itemize}
Designação de cada indivíduo de uma raça definida, mas desprezada entre os Índios.
Homem desprezado ou repellido pelos outros.
(Do tamul \textunderscore parai\textunderscore )
\section{Pariambo}
\begin{itemize}
\item {Grp. gram.:m.}
\end{itemize}
\begin{itemize}
\item {Proveniência:(Lat. \textunderscore pariambus\textunderscore )}
\end{itemize}
Pé de verso latino, que também se chama \textunderscore pýrrhico\textunderscore .
\section{Pariana}
\begin{itemize}
\item {Grp. gram.:f.}
\end{itemize}
Gênero de plantas gramíneas.
\section{Parianas}
\begin{itemize}
\item {Grp. gram.:m. pl.}
\end{itemize}
Indígenas do norte do Brasil.
\section{Párias}
\begin{itemize}
\item {Grp. gram.:f. pl.}
\end{itemize}
\begin{itemize}
\item {Utilização:T. de Turquel}
\end{itemize}
Tributo, pago antigamente por um Soberano ou Estado a outro, em reconhecimento de vassalagem.
Satisfação, desaggravo: \textunderscore tomar párias a alguém\textunderscore .
(Relaciona-se com o lat. \textunderscore parere\textunderscore , obedecer?)
\section{Pariato}
\begin{itemize}
\item {Grp. gram.:m.}
\end{itemize}
\begin{itemize}
\item {Proveniência:(De \textunderscore par\textunderscore )}
\end{itemize}
Dignidade de par do reino.
\section{Paricá}
\begin{itemize}
\item {Grp. gram.:m.}
\end{itemize}
\begin{itemize}
\item {Utilização:Bras. do N}
\end{itemize}
Árvore leguminosa, de cuja fruta torrada e triturada usam os selvagens á maneira de rapé.
\section{Parição}
\begin{itemize}
\item {Grp. gram.:f.}
\end{itemize}
\begin{itemize}
\item {Utilização:Bras. do N}
\end{itemize}
Acto de parir, (tratando-se de animaes). Cf. Rev. \textunderscore Tradição\textunderscore , V, 146.
Multiplicação annual do gado: \textunderscore na parição de 1912, apanhei oitenta bezerros\textunderscore .
\section{Pariceiro}
\begin{itemize}
\item {Grp. gram.:m.  e  adj.}
\end{itemize}
\begin{itemize}
\item {Utilização:Bras}
\end{itemize}
\begin{itemize}
\item {Utilização:pop.}
\end{itemize}
O mesmo que \textunderscore parceiro\textunderscore .
\section{Paricis}
\begin{itemize}
\item {Grp. gram.:m. pl.}
\end{itemize}
Tríbo de aborígenes de Mato-Grosso.
\section{Parida}
\begin{itemize}
\item {Grp. gram.:adj.}
\end{itemize}
Mulher ou qualquer fêmea, que pariu.
(Fem. de \textunderscore parido\textunderscore )
\section{Paridade}
\begin{itemize}
\item {Grp. gram.:f.}
\end{itemize}
\begin{itemize}
\item {Proveniência:(Lat. \textunderscore paritas\textunderscore )}
\end{itemize}
Qualidade de par ou igual.
Parecença, semelhança, analogia.
\section{Paridade}
\begin{itemize}
\item {Grp. gram.:f.}
\end{itemize}
\begin{itemize}
\item {Utilização:Prov.}
\end{itemize}
\begin{itemize}
\item {Utilização:alent.}
\end{itemize}
\begin{itemize}
\item {Proveniência:(De \textunderscore parir\textunderscore )}
\end{itemize}
Rebanho de ovelhas paridas. Cf. Rev. \textunderscore Tradição\textunderscore , 98.
\section{Parídeas}
\begin{itemize}
\item {Grp. gram.:f. pl.}
\end{itemize}
\begin{itemize}
\item {Proveniência:(Do gr. \textunderscore Paris\textunderscore , n. p. + \textunderscore eidos\textunderscore )}
\end{itemize}
Secção de plantas asparagíneas, que comprehende os gêneros que têm três ou quatro estigmas distintos.
\section{Parideira}
\begin{itemize}
\item {Grp. gram.:adj.}
\end{itemize}
\begin{itemize}
\item {Utilização:Bras. do N}
\end{itemize}
Diz-se da fêmea, que está em idade de parir.
Diz-se da fêmea que pare annualmente.
\section{Paridela}
\begin{itemize}
\item {Grp. gram.:f.}
\end{itemize}
\begin{itemize}
\item {Utilização:Pop.}
\end{itemize}
O mesmo que \textunderscore parto\textunderscore .
\section{Paridura}
\begin{itemize}
\item {Grp. gram.:f.}
\end{itemize}
(V.parto)
\section{Parietal}
\begin{itemize}
\item {Grp. gram.:adj.}
\end{itemize}
\begin{itemize}
\item {Grp. gram.:M.}
\end{itemize}
\begin{itemize}
\item {Utilização:Anat.}
\end{itemize}
\begin{itemize}
\item {Proveniência:(Lat. \textunderscore parietalis\textunderscore )}
\end{itemize}
Relativo a parede.
Que se póde pendurar na parede: \textunderscore quadros parietaes\textunderscore .
Cada um dos ossos, que formam os lados do crânio.
\section{Parietária}
\begin{itemize}
\item {Grp. gram.:f.}
\end{itemize}
Planta, que cresce nas paredes.
Alfavaca de cobra, (\textunderscore parietaria officinalis\textunderscore , Lin.).
(Fem. de \textunderscore parietário\textunderscore )
\section{Parietário}
\begin{itemize}
\item {Grp. gram.:adj.}
\end{itemize}
\begin{itemize}
\item {Proveniência:(Lat. \textunderscore parietarius\textunderscore )}
\end{itemize}
O mesmo que \textunderscore parietal\textunderscore .
Que cresce nas paredes.
\section{Parietina}
\begin{itemize}
\item {Grp. gram.:f.}
\end{itemize}
Substância extrahida de um líchen, (\textunderscore parmelia parietina\textunderscore ).
\section{Pariforme}
\begin{itemize}
\item {Grp. gram.:f.}
\end{itemize}
\begin{itemize}
\item {Proveniência:(Do lat. \textunderscore par\textunderscore  + \textunderscore forma\textunderscore )}
\end{itemize}
Que tem fórma igual.
\section{Pariglina}
\begin{itemize}
\item {Grp. gram.:f.}
\end{itemize}
Substância, alcalina, extrahida da salsaparilha.
\section{Parilidade}
\begin{itemize}
\item {Grp. gram.:f.}
\end{itemize}
\begin{itemize}
\item {Proveniência:(Lat. \textunderscore parilitas\textunderscore )}
\end{itemize}
O mesmo que \textunderscore paridade\textunderscore ^1.
\section{Parílias}
\begin{itemize}
\item {Grp. gram.:f. pl.}
\end{itemize}
\begin{itemize}
\item {Proveniência:(Lat. \textunderscore parilia\textunderscore )}
\end{itemize}
Festa, que as damas grávidas, na antiga Roma celebravam, para têr parto feliz.
\section{Parimento}
\begin{itemize}
\item {Grp. gram.:m.}
\end{itemize}
\begin{itemize}
\item {Utilização:Ant.}
\end{itemize}
\begin{itemize}
\item {Proveniência:(De \textunderscore parir\textunderscore )}
\end{itemize}
O mesmo que \textunderscore parto\textunderscore .
\section{Parinari}
\begin{itemize}
\item {Grp. gram.:m.}
\end{itemize}
\begin{itemize}
\item {Utilização:Bras. do N}
\end{itemize}
Nome de uma árvore, cuja casca é aproveitada como isca pelos Índios do Amazonas.
\section{Parinervado}
\begin{itemize}
\item {Grp. gram.:adj.}
\end{itemize}
\begin{itemize}
\item {Utilização:Bot.}
\end{itemize}
\begin{itemize}
\item {Proveniência:(Do lat. \textunderscore par\textunderscore  + \textunderscore nervus\textunderscore )}
\end{itemize}
Que tem organização invariável.
\section{Parintins}
\begin{itemize}
\item {Grp. gram.:m. pl.}
\end{itemize}
O mesmo que \textunderscore Parintintins\textunderscore .
\section{Parintintins}
\begin{itemize}
\item {Grp. gram.:m. pl.}
\end{itemize}
Tríbo selvagem e antropóphaga, na margem direita do rio Madeira, (Brasil).
\section{Pário}
\begin{itemize}
\item {Grp. gram.:m.}
\end{itemize}
\begin{itemize}
\item {Utilização:Ant.}
\end{itemize}
O mesmo que \textunderscore parelho\textunderscore .
(Cp. \textunderscore páreo\textunderscore )
\section{Pariocela}
\begin{itemize}
\item {Grp. gram.:f.}
\end{itemize}
Gênero de reptís sáurios.
\section{Pariparoba}
\begin{itemize}
\item {Grp. gram.:f.}
\end{itemize}
\begin{itemize}
\item {Utilização:Bras}
\end{itemize}
O mesmo que \textunderscore malvaísco\textunderscore .
\section{Paripinnulado}
\begin{itemize}
\item {Grp. gram.:adj.}
\end{itemize}
\begin{itemize}
\item {Utilização:Bot.}
\end{itemize}
Diz-se das fôlhas compostas, cujos folíolos são ligados aos pares e que, no ápice commum, não têm folíolo solitário nem gavinha.
\section{Paripinulado}
\begin{itemize}
\item {Grp. gram.:adj.}
\end{itemize}
\begin{itemize}
\item {Utilização:Bot.}
\end{itemize}
Diz-se das fôlhas compostas, cujos folíolos são ligados aos pares e que, no ápice commum, não têm folíolo solitário nem gavinha.
\section{Pariquis}
\begin{itemize}
\item {Grp. gram.:m. pl.}
\end{itemize}
Tríbo de aborígenes do Pará.
\section{Parir}
\begin{itemize}
\item {Grp. gram.:v. t.}
\end{itemize}
\begin{itemize}
\item {Utilização:Fam.}
\end{itemize}
\begin{itemize}
\item {Proveniência:(Lat. \textunderscore p[-a]rere\textunderscore )}
\end{itemize}
Dar á luz ou expellir do útero, tratando-se das mulheres e de todas as fêmeas vivíparas, em relação aos seres que geram.
Produzir.
\section{Pariseta}
\begin{itemize}
\item {fónica:zê}
\end{itemize}
\begin{itemize}
\item {Grp. gram.:f.}
\end{itemize}
Gênero de plantas liliáceas medicinaes.
(Dem. do lat. \textunderscore paris\textunderscore )
\section{Parisiano}
\begin{itemize}
\item {Grp. gram.:m.  e  adj.}
\end{itemize}
O mesmo que \textunderscore parisiense\textunderscore . Cf. Cand. Lusit., \textunderscore Arte Poética\textunderscore .
\section{Parisiense}
\begin{itemize}
\item {Grp. gram.:adj.}
\end{itemize}
\begin{itemize}
\item {Grp. gram.:M.}
\end{itemize}
\begin{itemize}
\item {Proveniência:(Do lat. \textunderscore Parisium\textunderscore , n. p.)}
\end{itemize}
Relativo a Paris.
Habitante de París.
\section{Parisino}
\begin{itemize}
\item {Grp. gram.:adj.}
\end{itemize}
\begin{itemize}
\item {Utilização:P. us.}
\end{itemize}
Relativo a París; parisiense. Cf. Filinto, XV, 36; \textunderscore Hyssope\textunderscore , 128.
\section{Parísios}
\begin{itemize}
\item {Grp. gram.:m. pl.}
\end{itemize}
\begin{itemize}
\item {Proveniência:(Lat. \textunderscore parisii\textunderscore )}
\end{itemize}
Antigo povo da Gállia, nas vizinhanças de Lutécia.
\section{Parisita}
\begin{itemize}
\item {Grp. gram.:f.}
\end{itemize}
\begin{itemize}
\item {Proveniência:(De \textunderscore Paris\textunderscore , n. p.)}
\end{itemize}
Mineral, que se encontra nas minas de esmeraldas da Nova-Granada.
\section{Parisito}
\begin{itemize}
\item {Grp. gram.:m.}
\end{itemize}
O mesmo ou melhor que \textunderscore parisita\textunderscore .
\section{Parisonante}
\begin{itemize}
\item {fónica:so}
\end{itemize}
\begin{itemize}
\item {Grp. gram.:adj.}
\end{itemize}
\begin{itemize}
\item {Utilização:Gram.}
\end{itemize}
\begin{itemize}
\item {Proveniência:(Do lat. \textunderscore par\textunderscore  + \textunderscore sonans\textunderscore , \textunderscore sonantis\textunderscore )}
\end{itemize}
Diz-se das conjugações regulares nalgumas línguas, como o alemão.
\section{Parissílabo}
\begin{itemize}
\item {Grp. gram.:adj.}
\end{itemize}
\begin{itemize}
\item {Utilização:Gram.}
\end{itemize}
\begin{itemize}
\item {Proveniência:(Do lat. \textunderscore par\textunderscore  + \textunderscore syllaba\textunderscore )}
\end{itemize}
Diz-se dos substantivos e adjectivos latinos, que têm tantas sílabas em o nominativo como no genitivo.
\section{Parissonante}
\begin{itemize}
\item {Grp. gram.:adj.}
\end{itemize}
\begin{itemize}
\item {Utilização:Gram.}
\end{itemize}
\begin{itemize}
\item {Proveniência:(Do lat. \textunderscore par\textunderscore  + \textunderscore sonans\textunderscore , \textunderscore sonantis\textunderscore )}
\end{itemize}
Diz-se das conjugações regulares nalgumas línguas, como o alemão.
\section{Parisýllabo}
\begin{itemize}
\item {fónica:si}
\end{itemize}
\begin{itemize}
\item {Grp. gram.:adj.}
\end{itemize}
\begin{itemize}
\item {Utilização:Gram.}
\end{itemize}
\begin{itemize}
\item {Proveniência:(Do lat. \textunderscore par\textunderscore  + \textunderscore syllaba\textunderscore )}
\end{itemize}
Diz-se dos substantivos e adjectivos latinos, que têm tantas sýllabas em o nominativo como no genitivo.
\section{Parita}
\begin{itemize}
\item {Grp. gram.:m.}
\end{itemize}
\begin{itemize}
\item {Utilização:Bras. do N}
\end{itemize}
\begin{itemize}
\item {Proveniência:(T. tupi)}
\end{itemize}
Cada uma das varas, a que se ligam as extremidades do pari.
\section{Parizataco}
\begin{itemize}
\item {Grp. gram.:m.}
\end{itemize}
Árvore oleácea da Índia portuguesa, (\textunderscore nyctanthes arbortristis\textunderscore , Lin.).
\section{Parkinsónia}
\begin{itemize}
\item {Grp. gram.:f.}
\end{itemize}
\begin{itemize}
\item {Proveniência:(De \textunderscore Parkinson\textunderscore , n. p.)}
\end{itemize}
Gênero de plantas leguminosas.
\section{Parla}
\begin{itemize}
\item {Grp. gram.:f.}
\end{itemize}
\begin{itemize}
\item {Utilização:Pop.}
\end{itemize}
\begin{itemize}
\item {Proveniência:(De \textunderscore parlar\textunderscore )}
\end{itemize}
Conversa; falatório.
\section{Parlamentação}
\begin{itemize}
\item {Grp. gram.:f.}
\end{itemize}
Acto ou effeito de parlamentar^2. Cf. Arn. Gama, \textunderscore Motim\textunderscore , 205.
\section{Parlamentar}
\begin{itemize}
\item {Grp. gram.:adj.}
\end{itemize}
\begin{itemize}
\item {Grp. gram.:M.}
\end{itemize}
Relativo ao parlamento: \textunderscore discussão parlamentar\textunderscore .
Membro de um parlamento.
\section{Parlamentar}
\begin{itemize}
\item {Grp. gram.:v. i.}
\end{itemize}
O mesmo que \textunderscore parlamentear\textunderscore .
\section{Parlamentário}
\begin{itemize}
\item {Grp. gram.:adj.}
\end{itemize}
\begin{itemize}
\item {Grp. gram.:M.}
\end{itemize}
\begin{itemize}
\item {Proveniência:(De \textunderscore parlamento\textunderscore )}
\end{itemize}
Que parlamenteia, ou que é próprio para parlamentar.
Aquelle que parlamenteia.
Navio, que conduz o indivíduo que vai parlamentear em navio inimigo.
\section{Parlamentarismo}
\begin{itemize}
\item {Grp. gram.:m.}
\end{itemize}
\begin{itemize}
\item {Proveniência:(De \textunderscore parlamentar\textunderscore )}
\end{itemize}
Systema parlamentar.
Influência do parlamento nos negócios públicos.
\section{Parlamentarista}
\begin{itemize}
\item {Grp. gram.:m.}
\end{itemize}
Sectário do parlamentarismo.
\section{Parlamentear}
\begin{itemize}
\item {Grp. gram.:v. i.}
\end{itemize}
\begin{itemize}
\item {Utilização:Fig.}
\end{itemize}
\begin{itemize}
\item {Proveniência:(De \textunderscore parlamento\textunderscore )}
\end{itemize}
Fazer ou acceitar propostas, sôbre negócios de guerra, entre arraiaes ou fôrças contrárias.
Entrar em negociações.
\section{Parlamento}
\begin{itemize}
\item {Grp. gram.:m.}
\end{itemize}
\begin{itemize}
\item {Utilização:Des.}
\end{itemize}
\begin{itemize}
\item {Proveniência:(De \textunderscore parlar\textunderscore )}
\end{itemize}
Câmaras legislativas.
Câmara dos Deputados.
Antiga assembleia dos Grandes, em França.
Acto de falar.
\section{Parlante}
\begin{itemize}
\item {Grp. gram.:m.}
\end{itemize}
\begin{itemize}
\item {Utilização:Gír.}
\end{itemize}
O mesmo que \textunderscore advogado\textunderscore .
\section{Parlapatão}
\begin{itemize}
\item {Grp. gram.:m.  e  adj.}
\end{itemize}
\begin{itemize}
\item {Proveniência:(De \textunderscore parlar\textunderscore )}
\end{itemize}
Homem vaidoso, impostor, embusteiro.
\section{Parlapatear}
\begin{itemize}
\item {Grp. gram.:v. t.  e  i.}
\end{itemize}
\begin{itemize}
\item {Utilização:Neol.}
\end{itemize}
\begin{itemize}
\item {Proveniência:(De \textunderscore parlapatão\textunderscore )}
\end{itemize}
Proceder como parlapatão.
Alardear com impostura; paparrotear.
\section{Parlapatice}
\begin{itemize}
\item {Grp. gram.:f.}
\end{itemize}
Modos, acção ou dito de parlapatão.
\section{Parlar}
\begin{itemize}
\item {Grp. gram.:v. i.}
\end{itemize}
(Contr. de \textunderscore parolar\textunderscore )
\section{Parlatório}
\begin{itemize}
\item {Grp. gram.:m.}
\end{itemize}
\begin{itemize}
\item {Proveniência:(De \textunderscore parlar\textunderscore )}
\end{itemize}
Locutório.
Conversa; falatório, falario.
\section{Parlenda}
\begin{itemize}
\item {Grp. gram.:f.}
\end{itemize}
\begin{itemize}
\item {Proveniência:(De \textunderscore parlar\textunderscore )}
\end{itemize}
O mesmo ou melhor que \textunderscore parlenga\textunderscore :«\textunderscore ...encurtar a séca desta estiradissima parlenda\textunderscore ». Filinto, I, 149 e XIX, 252.
\section{Parlenga}
\begin{itemize}
\item {Grp. gram.:f.}
\end{itemize}
\begin{itemize}
\item {Proveniência:(De \textunderscore parlar\textunderscore )}
\end{itemize}
Bacharelice; palavriado.
Discussão; rixa.
\section{Parma}
\begin{itemize}
\item {Grp. gram.:f.}
\end{itemize}
\begin{itemize}
\item {Proveniência:(Lat. \textunderscore parma\textunderscore )}
\end{itemize}
Escudo circular, usado antigamente entre os Romanos.
\section{Parmélia}
\begin{itemize}
\item {Grp. gram.:f.}
\end{itemize}
\begin{itemize}
\item {Proveniência:(Do lat. \textunderscore parma\textunderscore )}
\end{itemize}
Gênero de líchens.
\section{Parmeliáceas}
\begin{itemize}
\item {Grp. gram.:f. pl.}
\end{itemize}
\begin{itemize}
\item {Utilização:Bot.}
\end{itemize}
\begin{itemize}
\item {Proveniência:(De \textunderscore parmélia\textunderscore )}
\end{itemize}
Família de líchens, no méthodo de Fee.
\section{Parmesão}
\begin{itemize}
\item {Grp. gram.:m.  e  adj.}
\end{itemize}
\begin{itemize}
\item {Proveniência:(De \textunderscore Parma\textunderscore , n. p.)}
\end{itemize}
Variedade de queijo italiano.
\section{Parnaíba}
\begin{itemize}
\item {Grp. gram.:f.}
\end{itemize}
\begin{itemize}
\item {Utilização:Bras}
\end{itemize}
Facalhão, com que se retalha a carne nos açougues da Baía.
(Do tupi)
\section{Parnão}
\begin{itemize}
\item {Grp. gram.:adj.}
\end{itemize}
\begin{itemize}
\item {Utilização:Pop.}
\end{itemize}
\begin{itemize}
\item {Proveniência:(De \textunderscore par\textunderscore  + \textunderscore não\textunderscore )}
\end{itemize}
Que não é par; ímpar.
\section{Parnásia}
\begin{itemize}
\item {Grp. gram.:f.}
\end{itemize}
\begin{itemize}
\item {Proveniência:(De \textunderscore Parnaso\textunderscore , n. p.)}
\end{itemize}
Gênero de plantas de regiões pantanosas.
\section{Parnasianismo}
\begin{itemize}
\item {Grp. gram.:m.}
\end{itemize}
Escola ou doutrina dos Parnasianos.
\section{Parnasiano}
\begin{itemize}
\item {Grp. gram.:adj.}
\end{itemize}
\begin{itemize}
\item {Grp. gram.:M.}
\end{itemize}
\begin{itemize}
\item {Proveniência:(De \textunderscore Parnaso\textunderscore , n. p.)}
\end{itemize}
Diz-se dos sectários de certa escola poética, que procura antes de tudo a delicadeza e a perfeição da fórma.
Poéta dessa escola.
\section{Parnasismo}
\begin{itemize}
\item {Grp. gram.:m.}
\end{itemize}
O mesmo que \textunderscore parnasianismo\textunderscore . Cf. Th. Braga, \textunderscore Mod. Ideias\textunderscore , I, VI.
\section{Parnaso}
\begin{itemize}
\item {Grp. gram.:m.}
\end{itemize}
\begin{itemize}
\item {Utilização:Fig.}
\end{itemize}
\begin{itemize}
\item {Proveniência:(De \textunderscore Parnaso\textunderscore , n. p.)}
\end{itemize}
Montanha da Phócida, consagrada a Apollo e ás Musas.
A poesia.
Os poétas, considerados collectivamente.
Collecção do poesias de autores diversos; anthologia.
Florilégio poético.
\section{Parnássia}
\begin{itemize}
\item {Grp. gram.:f.}
\end{itemize}
\begin{itemize}
\item {Proveniência:(De \textunderscore Parnasso\textunderscore , n. p.)}
\end{itemize}
Gênero de plantas droseráceas, que cresce em lugares húmidos.
\section{Parnassino}
\begin{itemize}
\item {Grp. gram.:m.}
\end{itemize}
Director de synagoga, entre os Israelitas.
\section{Parne}
\begin{itemize}
\item {Grp. gram.:m.}
\end{itemize}
\begin{itemize}
\item {Utilização:Náut.}
\end{itemize}
(?):«\textunderscore aquella alma de mestre, mal que se apanhou piçó, endinhou-me a mão no golpe, picou-me com o parne só\textunderscore ». Canção marítima.
\section{Parné}
\begin{itemize}
\item {Grp. gram.:m.}
\end{itemize}
\begin{itemize}
\item {Utilização:Gír.}
\end{itemize}
Dinheiro.
(Caló \textunderscore parné\textunderscore )
\section{Paro}
\begin{itemize}
\item {Grp. gram.:m.}
\end{itemize}
\begin{itemize}
\item {Utilização:Pop.}
\end{itemize}
Acto de parar.
Sossêgo; quietação: \textunderscore o meu pequeno é um traquinas, nunca tem paro\textunderscore .
\section{Paró}
\begin{itemize}
\item {Grp. gram.:m.}
\end{itemize}
O mesmo que \textunderscore parau\textunderscore .
\section{Paroara}
\begin{itemize}
\item {Grp. gram.:m.}
\end{itemize}
\begin{itemize}
\item {Utilização:Bras. do Ceará}
\end{itemize}
Bella ave americana, (\textunderscore loxiía dominicana\textunderscore , Lin.), de cabeça vermelha e corpo preto e branco.
Forasteiro, residente no Pará ou Amazonas.
\section{Paroba}
\begin{itemize}
\item {Grp. gram.:f.}
\end{itemize}
\begin{itemize}
\item {Utilização:Bras}
\end{itemize}
O mesmo que \textunderscore peroba\textunderscore .
\section{Parochia}
\begin{itemize}
\item {fónica:qui}
\end{itemize}
\begin{itemize}
\item {Grp. gram.:f.}
\end{itemize}
\begin{itemize}
\item {Proveniência:(Lat. \textunderscore parochia\textunderscore )}
\end{itemize}
População subordinada ecclesiásticamente a um párocho.
Igreja matriz; freguesia.
\section{Parochial}
\begin{itemize}
\item {fónica:qui}
\end{itemize}
\begin{itemize}
\item {Grp. gram.:adj.}
\end{itemize}
\begin{itemize}
\item {Proveniência:(Lat. \textunderscore parochialis\textunderscore )}
\end{itemize}
Relativo ou pertencente á paróchia.
\section{Parochiano}
\begin{itemize}
\item {fónica:qui}
\end{itemize}
\begin{itemize}
\item {Grp. gram.:m.  e  adj.}
\end{itemize}
\begin{itemize}
\item {Proveniência:(Do b. lat. \textunderscore parochianus\textunderscore )}
\end{itemize}
Habitante de uma paróchia.
\section{Parochiar}
\begin{itemize}
\item {fónica:qui}
\end{itemize}
\begin{itemize}
\item {Grp. gram.:v. t.}
\end{itemize}
\begin{itemize}
\item {Grp. gram.:V. i.}
\end{itemize}
\begin{itemize}
\item {Proveniência:(De \textunderscore paróchia\textunderscore )}
\end{itemize}
Administrar como párocho.
Exercer funcções de párocho.
\section{Párocho}
\begin{itemize}
\item {fónica:co}
\end{itemize}
\begin{itemize}
\item {Grp. gram.:m.}
\end{itemize}
\begin{itemize}
\item {Proveniência:(Lat. \textunderscore parochus\textunderscore )}
\end{itemize}
Sacerdote, que tem a seu cargo uma paróchia.
Prior; abbade; vigário; reitor; cura.
Pastor espiritual.
\section{Pároco}
\begin{itemize}
\item {Grp. gram.:m.}
\end{itemize}
\begin{itemize}
\item {Proveniência:(Lat. \textunderscore parochus\textunderscore )}
\end{itemize}
Sacerdote, que tem a seu cargo uma paróquia.
Prior; abade; vigário; reitor; cura.
Pastor espiritual.
\section{Paródia}
\begin{itemize}
\item {Grp. gram.:f.}
\end{itemize}
\begin{itemize}
\item {Utilização:Ext.}
\end{itemize}
\begin{itemize}
\item {Proveniência:(Lat. \textunderscore parodia\textunderscore )}
\end{itemize}
Imitação burlesca de uma obra literária.
Imitação burlesca.
Dança de Carnaval, nas ruas. Cf. \textunderscore Diário-do-Gov.\textunderscore , de 5-II-901.
\section{Parodiar}
\begin{itemize}
\item {Grp. gram.:v. t.}
\end{itemize}
\begin{itemize}
\item {Proveniência:(De \textunderscore paródia\textunderscore )}
\end{itemize}
Imitar burlescamente.
\section{Parodista}
\begin{itemize}
\item {Grp. gram.:m.  e  f.}
\end{itemize}
Pessôa, que faz paródias.
\section{Parodonte}
\begin{itemize}
\item {Grp. gram.:m.}
\end{itemize}
\begin{itemize}
\item {Utilização:Med.}
\end{itemize}
\begin{itemize}
\item {Proveniência:(Do gr. \textunderscore para\textunderscore  + \textunderscore odous\textunderscore )}
\end{itemize}
Tubérculo doloroso nas gengivas.
\section{Paroftalmia}
\begin{itemize}
\item {Grp. gram.:f.}
\end{itemize}
\begin{itemize}
\item {Utilização:Med.}
\end{itemize}
\begin{itemize}
\item {Proveniência:(Do gr. \textunderscore para\textunderscore  + \textunderscore ophthalmos\textunderscore )}
\end{itemize}
Oftalmia palpebral.
\section{Parol}
\begin{itemize}
\item {Grp. gram.:m.}
\end{itemize}
\begin{itemize}
\item {Proveniência:(Do cast. \textunderscore perol\textunderscore , tacho de arame)}
\end{itemize}
Vasilha, em que se apara o sumo da cana, nos engenhos de açúcar.
\section{Parola}
\begin{itemize}
\item {Grp. gram.:f.}
\end{itemize}
\begin{itemize}
\item {Utilização:Prov.}
\end{itemize}
\begin{itemize}
\item {Utilização:trasm.}
\end{itemize}
\begin{itemize}
\item {Proveniência:(De \textunderscore parolar\textunderscore )}
\end{itemize}
Palanfrório.
Palavras ocas.
Loquacidade; trela.
Mentira, pêta.
\section{Parolador}
\begin{itemize}
\item {Grp. gram.:m.  e  adj.}
\end{itemize}
\begin{itemize}
\item {Proveniência:(De \textunderscore parolar\textunderscore )}
\end{itemize}
O mesmo que \textunderscore paroleiro\textunderscore .
\section{Parolagem}
\begin{itemize}
\item {Grp. gram.:f.}
\end{itemize}
Acto de parolar.
\section{Parolamento}
\begin{itemize}
\item {Grp. gram.:m.}
\end{itemize}
O mesmo que \textunderscore parolagem\textunderscore . Cf. Ortigão, \textunderscore Holanda\textunderscore , 29.
\section{Parolar}
\begin{itemize}
\item {Grp. gram.:v. i.}
\end{itemize}
Tagarelar, dizer parolas.
(Cp. \textunderscore parolear\textunderscore , no dialecto picardo)
\section{Parolear}
\begin{itemize}
\item {Grp. gram.:v. i.}
\end{itemize}
O mesmo que \textunderscore parolar\textunderscore .
\section{Paroleira}
\begin{itemize}
\item {Grp. gram.:f.}
\end{itemize}
O mesmo que \textunderscore parola\textunderscore . Cf. Camillo, \textunderscore Corja\textunderscore .
\section{Paroleiro}
\begin{itemize}
\item {Grp. gram.:m.}
\end{itemize}
Aquelle que diz parolas; embusteiro; palrador.
\section{Parolento}
\begin{itemize}
\item {Grp. gram.:adj.}
\end{itemize}
\begin{itemize}
\item {Utilização:Des.}
\end{itemize}
O mesmo que \textunderscore paroleiro\textunderscore .
\section{Parolice}
\begin{itemize}
\item {Grp. gram.:f.}
\end{itemize}
\begin{itemize}
\item {Proveniência:(De \textunderscore parola\textunderscore )}
\end{itemize}
Qualidade de paroleiro; acto de parolar. Cf. Júl. Lour. Pinto, \textunderscore Senhor Deput.\textunderscore , 278.
\section{Parolim}
\begin{itemize}
\item {Grp. gram.:m.}
\end{itemize}
Acto de tresdobrar a parada, no jôgo.
(Cast. \textunderscore paroli\textunderscore )
\section{Parolo}
\begin{itemize}
\item {fónica:parô}
\end{itemize}
\begin{itemize}
\item {Grp. gram.:m. Loc.}
\end{itemize}
\begin{itemize}
\item {Utilização:pop.}
\end{itemize}
\textunderscore Cantar um parolo a alguém\textunderscore , corrigi-lo, admoestá-lo severamente, dizer-lhe o bom e o bonito.
(Cp. \textunderscore parola\textunderscore )
\section{Parónica}
\begin{itemize}
\item {Grp. gram.:f.}
\end{itemize}
Planta, a que os antigos atribuiam a virtude de curar o panarício.
(Cp. \textunderscore paroníquia\textunderscore )
\section{Paronímia}
\begin{itemize}
\item {Grp. gram.:f.}
\end{itemize}
Qualidade do que é parónimo.
\section{Paronímico}
\begin{itemize}
\item {Grp. gram.:adj.}
\end{itemize}
\begin{itemize}
\item {Grp. gram.:M.}
\end{itemize}
\begin{itemize}
\item {Proveniência:(Gr. \textunderscore paronumos\textunderscore )}
\end{itemize}
Diz-se das palavras, que tem som semelhante ao de outras, podendo confundir-se com estas, quando ouvidas por pessôas pouco instruídas.
Vocábulo parónimo.
\section{Parónimo}
\begin{itemize}
\item {Grp. gram.:adj.}
\end{itemize}
\begin{itemize}
\item {Grp. gram.:M.}
\end{itemize}
\begin{itemize}
\item {Proveniência:(Gr. \textunderscore paronumos\textunderscore )}
\end{itemize}
Diz-se das palavras, que tem som semelhante ao de outras, podendo confundir-se com estas, quando ouvidas por pessôas pouco instruídas.
Vocábulo parónimo.
\section{Paroníquia}
\begin{itemize}
\item {Grp. gram.:f.}
\end{itemize}
\begin{itemize}
\item {Proveniência:(Lat. \textunderscore paronychia\textunderscore )}
\end{itemize}
O mesmo que \textunderscore panarício\textunderscore .
\section{Paroníquias}
\begin{itemize}
\item {Grp. gram.:f. pl.}
\end{itemize}
Família de plantas, vizinhas das cariofiláceas, e formada de gêneros análogos ao da parónica.
(Cp. \textunderscore parónica\textunderscore  e \textunderscore paroníquia\textunderscore )
\section{Paronomásia}
\begin{itemize}
\item {Grp. gram.:f.}
\end{itemize}
\begin{itemize}
\item {Proveniência:(Lat. \textunderscore paronomasia\textunderscore )}
\end{itemize}
Uso de palavras, semelhantes no som, mas differentes no sentido.
Semelhança entre palavras de differentes línguas, indicativa de uma origem commum.
\section{Parónyca}
\begin{itemize}
\item {Grp. gram.:f.}
\end{itemize}
Planta, a que os antigos attribuiam a virtude de curar o panarício.
(Cp. \textunderscore paronýchia\textunderscore )
\section{Paronýchia}
\begin{itemize}
\item {fónica:qui}
\end{itemize}
\begin{itemize}
\item {Grp. gram.:f.}
\end{itemize}
\begin{itemize}
\item {Proveniência:(Lat. \textunderscore paronychia\textunderscore )}
\end{itemize}
O mesmo que \textunderscore panarício\textunderscore .
\section{Paronýchias}
\begin{itemize}
\item {fónica:qui}
\end{itemize}
\begin{itemize}
\item {Grp. gram.:f. pl.}
\end{itemize}
Família de plantas, vizinhas das caryophylláceas, e formada de gêneros análogos ao da parónyca.
(Cp. \textunderscore parónica\textunderscore  e \textunderscore paronýchia\textunderscore )
\section{Paronýmia}
\begin{itemize}
\item {Grp. gram.:f.}
\end{itemize}
Qualidade do que é parónymo.
\section{Paronýmico}
\begin{itemize}
\item {Grp. gram.:adj.}
\end{itemize}
\begin{itemize}
\item {Grp. gram.:M.}
\end{itemize}
\begin{itemize}
\item {Proveniência:(Gr. \textunderscore paronumos\textunderscore )}
\end{itemize}
Diz-se das palavras, que tem som semelhante ao de outras, podendo confundir-se com estas, quando ouvidas por pessôas pouco instruídas.
Vocábulo parónymo.
\section{Parónymo}
\begin{itemize}
\item {Grp. gram.:adj.}
\end{itemize}
\begin{itemize}
\item {Grp. gram.:M.}
\end{itemize}
\begin{itemize}
\item {Proveniência:(Gr. \textunderscore paronumos\textunderscore )}
\end{itemize}
Diz-se das palavras, que tem som semelhante ao de outras, podendo confundir-se com estas, quando ouvidas por pessôas pouco instruídas.
Vocábulo parónymo.
\section{Parophtalmia}
\begin{itemize}
\item {Grp. gram.:f.}
\end{itemize}
\begin{itemize}
\item {Utilização:Med.}
\end{itemize}
\begin{itemize}
\item {Proveniência:(Do gr. \textunderscore para\textunderscore  + \textunderscore ophthalmos\textunderscore )}
\end{itemize}
Ophtalmia palpebral.
\section{Parópia}
\begin{itemize}
\item {Grp. gram.:f.}
\end{itemize}
\begin{itemize}
\item {Utilização:Anat.}
\end{itemize}
\begin{itemize}
\item {Proveniência:(Do gr. \textunderscore para\textunderscore  + \textunderscore ops\textunderscore )}
\end{itemize}
Designação antiga do ângulo externo do ôlho.
\section{Paropía}
\begin{itemize}
\item {Grp. gram.:f.}
\end{itemize}
\begin{itemize}
\item {Utilização:Anat.}
\end{itemize}
\begin{itemize}
\item {Proveniência:(Do gr. \textunderscore para\textunderscore  + \textunderscore ops\textunderscore )}
\end{itemize}
Designação antiga do ângulo externo do ôlho.
\section{Paropsia}
\begin{itemize}
\item {Grp. gram.:f.}
\end{itemize}
\begin{itemize}
\item {Proveniência:(Do gr. \textunderscore para\textunderscore  + \textunderscore opsis\textunderscore )}
\end{itemize}
Designação genêrica dos defeitos da vista.
\section{Paroptese}
\begin{itemize}
\item {Grp. gram.:f.}
\end{itemize}
\begin{itemize}
\item {Utilização:Med.}
\end{itemize}
Suor, produzido pela collocação do doente numa estufa.
\section{Paróptico}
\begin{itemize}
\item {Grp. gram.:adj.}
\end{itemize}
\begin{itemize}
\item {Utilização:Phýs.}
\end{itemize}
\begin{itemize}
\item {Proveniência:(Do gr. \textunderscore para\textunderscore  + \textunderscore optikos\textunderscore )}
\end{itemize}
Diz-se do calor, produzido por uma luz que soffreu diffracção.
\section{Paróquia}
\begin{itemize}
\item {Grp. gram.:f.}
\end{itemize}
\begin{itemize}
\item {Proveniência:(Lat. \textunderscore parochia\textunderscore )}
\end{itemize}
População subordinada ecclesiásticamente a um pároco.
Igreja matriz; freguesia.
\section{Paroquial}
\begin{itemize}
\item {Grp. gram.:adj.}
\end{itemize}
\begin{itemize}
\item {Proveniência:(Lat. \textunderscore parochialis\textunderscore )}
\end{itemize}
Relativo ou pertencente á paróquia.
\section{Paroquiano}
\begin{itemize}
\item {Grp. gram.:m.  e  adj.}
\end{itemize}
\begin{itemize}
\item {Proveniência:(Do b. lat. \textunderscore parochianus\textunderscore )}
\end{itemize}
Habitante de uma paróquia.
\section{Paroquiar}
\begin{itemize}
\item {Grp. gram.:v. t.}
\end{itemize}
\begin{itemize}
\item {Grp. gram.:V. i.}
\end{itemize}
\begin{itemize}
\item {Proveniência:(De \textunderscore paróquia\textunderscore )}
\end{itemize}
Administrar como pároco.
Exercer funções de pároco.
\section{Parorase}
\begin{itemize}
\item {Grp. gram.:f.}
\end{itemize}
\begin{itemize}
\item {Utilização:Med.}
\end{itemize}
Perversão da vista, caracterizada pela difficuldade de distinguir bem a côr dos objectos.
\section{Parosmia}
\begin{itemize}
\item {Grp. gram.:f.}
\end{itemize}
\begin{itemize}
\item {Utilização:Med.}
\end{itemize}
\begin{itemize}
\item {Proveniência:(Do gr. \textunderscore para\textunderscore  + \textunderscore osme\textunderscore )}
\end{itemize}
Perversão do olfacto.
\section{Parótia}
\begin{itemize}
\item {Grp. gram.:f.}
\end{itemize}
Gênero de plantas hamamelídeas.
\section{Parótico}
\begin{itemize}
\item {Grp. gram.:adj.}
\end{itemize}
\begin{itemize}
\item {Utilização:Anat.}
\end{itemize}
\begin{itemize}
\item {Proveniência:(Do gr. \textunderscore para\textunderscore  + \textunderscore ous\textunderscore , \textunderscore otos\textunderscore )}
\end{itemize}
Que está perto da orelha.
\section{Parótida}
\begin{itemize}
\item {Grp. gram.:f.}
\end{itemize}
\begin{itemize}
\item {Utilização:Anat.}
\end{itemize}
\begin{itemize}
\item {Proveniência:(Gr. \textunderscore parotis\textunderscore )}
\end{itemize}
Cada uma das glândulas salivares, situadas atrás das orelhas.
\section{Parotideano}
\begin{itemize}
\item {Grp. gram.:adj.}
\end{itemize}
Relativo á parótida.
\section{Parotidite}
\begin{itemize}
\item {Grp. gram.:f.}
\end{itemize}
Inflammação de parótida.
\section{Par-ou-pernão}
\begin{itemize}
\item {Grp. gram.:m.}
\end{itemize}
Espécie de jôgo popular.
\section{Parouveia}
\begin{itemize}
\item {Grp. gram.:f.}
\end{itemize}
\begin{itemize}
\item {Utilização:Prov.}
\end{itemize}
\begin{itemize}
\item {Utilização:trasm.}
\end{itemize}
Lugar alto, exposto ao vento.
\section{Parouvela}
\begin{itemize}
\item {Grp. gram.:f.}
\end{itemize}
\begin{itemize}
\item {Utilização:Prov.}
\end{itemize}
\begin{itemize}
\item {Utilização:alg.}
\end{itemize}
\begin{itemize}
\item {Utilização:Prov.}
\end{itemize}
\begin{itemize}
\item {Utilização:trasm.}
\end{itemize}
\begin{itemize}
\item {Proveniência:(De \textunderscore parouvelar\textunderscore )}
\end{itemize}
Palavreado; parola; arrazoado:«\textunderscore sor Christina, que ouviu tal parouvela...\textunderscore »Filinto, X, 135.
Lugar onde bate o vento, parouveia.
Ventania.
\section{Parouvelar}
\begin{itemize}
\item {Grp. gram.:v. i.}
\end{itemize}
\begin{itemize}
\item {Utilização:Ant.}
\end{itemize}
Parolar, palestrear. Cf. Vicente, I, 252.
(Por \textunderscore parabolar\textunderscore , de \textunderscore parábola\textunderscore )
\section{Paroxísmico}
\begin{itemize}
\item {fónica:csis}
\end{itemize}
\begin{itemize}
\item {Grp. gram.:adj.}
\end{itemize}
Relativo a paroxismo.
\section{Paroxismo}
\begin{itemize}
\item {fónica:csis}
\end{itemize}
\begin{itemize}
\item {Grp. gram.:m.}
\end{itemize}
\begin{itemize}
\item {Grp. gram.:Pl.}
\end{itemize}
\begin{itemize}
\item {Proveniência:(Gr. \textunderscore paroxusmos\textunderscore )}
\end{itemize}
A maior intensidade de um acesso, de uma dôr, etc.
Estertor do moribundo, agonia.
\section{Paroxístico}
\begin{itemize}
\item {fónica:csis}
\end{itemize}
\begin{itemize}
\item {Grp. gram.:adj.}
\end{itemize}
Relativo a paroxismo.
\section{Paroxitónico}
\begin{itemize}
\item {fónica:csi}
\end{itemize}
\begin{itemize}
\item {Grp. gram.:adj.}
\end{itemize}
O mesmo que \textunderscore paroxítono\textunderscore .
\section{Paroxítono}
\begin{itemize}
\item {fónica:csi}
\end{itemize}
\begin{itemize}
\item {Grp. gram.:adj.}
\end{itemize}
\begin{itemize}
\item {Utilização:Gram.}
\end{itemize}
\begin{itemize}
\item {Grp. gram.:M.}
\end{itemize}
\begin{itemize}
\item {Proveniência:(Gr. \textunderscore paroxutonos\textunderscore )}
\end{itemize}
Que tem a acentuação tónica na penúltima sílaba.
Palavra paroxítona.
\section{Paroxýsmico}
\begin{itemize}
\item {fónica:csis}
\end{itemize}
\begin{itemize}
\item {Grp. gram.:adj.}
\end{itemize}
Relativo a paroxysmo.
\section{Paroxysmo}
\begin{itemize}
\item {fónica:csis}
\end{itemize}
\begin{itemize}
\item {Grp. gram.:m.}
\end{itemize}
\begin{itemize}
\item {Grp. gram.:Pl.}
\end{itemize}
\begin{itemize}
\item {Proveniência:(Gr. \textunderscore paroxusmos\textunderscore )}
\end{itemize}
A maior intensidade de um accesso, de uma dôr, etc.
Estertor do moribundo, agonia.
\section{Paroxýstico}
\begin{itemize}
\item {fónica:csis}
\end{itemize}
\begin{itemize}
\item {Grp. gram.:adj.}
\end{itemize}
Relativo a paroxysmo.
\section{Paroxytónico}
\begin{itemize}
\item {fónica:csi}
\end{itemize}
\begin{itemize}
\item {Grp. gram.:adj.}
\end{itemize}
O mesmo que \textunderscore paroxýtono\textunderscore .
\section{Paroxýtono}
\begin{itemize}
\item {fónica:csi}
\end{itemize}
\begin{itemize}
\item {Grp. gram.:adj.}
\end{itemize}
\begin{itemize}
\item {Utilização:Gram.}
\end{itemize}
\begin{itemize}
\item {Grp. gram.:M.}
\end{itemize}
\begin{itemize}
\item {Proveniência:(Gr. \textunderscore paroxutonos\textunderscore )}
\end{itemize}
Que tem a accentuação tónica na penúltima sýllaba.
Palavra paroxýtona.
\section{Párpado}
\begin{itemize}
\item {Grp. gram.:m.}
\end{itemize}
\begin{itemize}
\item {Utilização:P. us.}
\end{itemize}
O mesmo que \textunderscore pálpebra\textunderscore .
\section{Parpalhaça}
\begin{itemize}
\item {Grp. gram.:f.}
\end{itemize}
O mesmo que \textunderscore parpalhó\textunderscore .
\section{Parpalhaz}
\begin{itemize}
\item {Grp. gram.:f.}
\end{itemize}
\begin{itemize}
\item {Utilização:Prov.}
\end{itemize}
\begin{itemize}
\item {Utilização:trasm.}
\end{itemize}
O mesmo que \textunderscore parpalhó\textunderscore .
\section{Parpalhó}
\begin{itemize}
\item {Grp. gram.:f.}
\end{itemize}
\begin{itemize}
\item {Utilização:Prov.}
\end{itemize}
\begin{itemize}
\item {Utilização:beir.}
\end{itemize}
\begin{itemize}
\item {Proveniência:(T. onom.)}
\end{itemize}
O mesmo que \textunderscore codorniz\textunderscore .
\section{Parpalhós}
\begin{itemize}
\item {Grp. gram.:f.}
\end{itemize}
\begin{itemize}
\item {Utilização:Prov.}
\end{itemize}
\begin{itemize}
\item {Utilização:beir.}
\end{itemize}
O mesmo que \textunderscore parpalhó\textunderscore .
\section{Parque}
\begin{itemize}
\item {Grp. gram.:m.}
\end{itemize}
\begin{itemize}
\item {Proveniência:(Do b. lat. \textunderscore parcus\textunderscore )}
\end{itemize}
Terreno murado ou vedado, onde há caça.
(Neste sentido é gallicismo, a que corresponde o port. \textunderscore tapada\textunderscore  ou \textunderscore coutada\textunderscore )
Jardim extenso e murado.
Lugar, em que se guardam munições de guerra ou petrechos de artilharia.
\section{Parquete}
\begin{itemize}
\item {fónica:quê}
\end{itemize}
\begin{itemize}
\item {Grp. gram.:m.}
\end{itemize}
\begin{itemize}
\item {Utilização:Neol.}
\end{itemize}
\begin{itemize}
\item {Proveniência:(Fr. \textunderscore parquet\textunderscore )}
\end{itemize}
Pavimento, formado de differentes pedaços de madeira, dispostos de fórma que constituem desenhos e figuras.
\section{Parra}
\begin{itemize}
\item {Grp. gram.:f.}
\end{itemize}
\begin{itemize}
\item {Utilização:Fig.}
\end{itemize}
Fôlhas de videira; pâmpano.
Parlapatice; palavreado oco: \textunderscore muita parra e pouca uva\textunderscore .
\section{Parra}
\begin{itemize}
\item {Grp. gram.:f.}
\end{itemize}
\begin{itemize}
\item {Utilização:T. de Serpa e do Algarve}
\end{itemize}
Vasilha de barro, usada especialmente para guardar banha de porco e para aparar o álcool, ao sair do alambique.
\section{Parra}
\begin{itemize}
\item {Grp. gram.:f.}
\end{itemize}
\begin{itemize}
\item {Utilização:Prov.}
\end{itemize}
\begin{itemize}
\item {Utilização:trasm.}
\end{itemize}
O mesmo que \textunderscore pato\textunderscore ^1.
(Cp. \textunderscore parro\textunderscore )
\section{Parracho}
\begin{itemize}
\item {Grp. gram.:adj.}
\end{itemize}
\begin{itemize}
\item {Utilização:Prov.}
\end{itemize}
\begin{itemize}
\item {Grp. gram.:M.}
\end{itemize}
\begin{itemize}
\item {Proveniência:(De \textunderscore parro\textunderscore )}
\end{itemize}
Que tem pouca altura.
Rasteiro.
Atarracado.
Homem atarracado, baixo.
\section{Parrado}
\begin{itemize}
\item {Grp. gram.:adj.}
\end{itemize}
\begin{itemize}
\item {Utilização:Prov.}
\end{itemize}
\begin{itemize}
\item {Utilização:trasm.}
\end{itemize}
\begin{itemize}
\item {Proveniência:(De \textunderscore parrar-se\textunderscore )}
\end{itemize}
Que tem as orelhas caídas, (falando-se do boi).
Orelhudo.
Apatetado.
\section{Párrafo}
\begin{itemize}
\item {Grp. gram.:m.}
\end{itemize}
\begin{itemize}
\item {Utilização:Ant.}
\end{itemize}
O mesmo que \textunderscore parágrapho\textunderscore . Cf. \textunderscore Eufrosina\textunderscore , 152.
\section{Parrameiro}
\begin{itemize}
\item {Grp. gram.:m.}
\end{itemize}
\begin{itemize}
\item {Utilização:Gír.}
\end{itemize}
\begin{itemize}
\item {Proveniência:(Do fr. ant. \textunderscore paramer\textunderscore )}
\end{itemize}
Partes pudendas da mulhér.
\section{Parrana}
\begin{itemize}
\item {Grp. gram.:m.  e  adj.}
\end{itemize}
\begin{itemize}
\item {Utilização:Gír.}
\end{itemize}
\begin{itemize}
\item {Grp. gram.:F.}
\end{itemize}
\begin{itemize}
\item {Utilização:Prov.}
\end{itemize}
\begin{itemize}
\item {Utilização:trasm.}
\end{itemize}
Aquelle que anda mal vestido.
Gebo.
Homem reles.
Mollangueiro, retardatário, indifferente ao progresso.
O mesmo que \textunderscore mandrião\textunderscore .
O mesmo que \textunderscore preguiça\textunderscore .
\section{Parranamente}
\begin{itemize}
\item {Grp. gram.:adv.}
\end{itemize}
De modo parrana:«\textunderscore ...deputado parranamente beldroegas.\textunderscore »Camillo, \textunderscore Novel. do Minho\textunderscore , II, 12.
\section{Parrançar}
\begin{itemize}
\item {Grp. gram.:v. i.}
\end{itemize}
\begin{itemize}
\item {Utilização:Gír.}
\end{itemize}
Proceder como parrana, mandriar.
(Cp. \textunderscore parrana\textunderscore )
\section{Parranice}
\begin{itemize}
\item {Grp. gram.:f.}
\end{itemize}
\begin{itemize}
\item {Utilização:Fam.}
\end{itemize}
Modos ou qualidade de parrana.
Mandriice.
\section{Parrar-se}
\begin{itemize}
\item {Grp. gram.:v. p.}
\end{itemize}
Cobrir-se de parras.
Alastrar-se.
\section{Parrascano}
\begin{itemize}
\item {Grp. gram.:m.}
\end{itemize}
\begin{itemize}
\item {Utilização:Prov.}
\end{itemize}
O mesmo que \textunderscore serrano\textunderscore .
\section{Parreco}
\begin{itemize}
\item {Grp. gram.:m.}
\end{itemize}
\begin{itemize}
\item {Utilização:Prov.}
\end{itemize}
O mesmo que \textunderscore marreco\textunderscore :«\textunderscore ...deixou em Alemquer a consorte, cuidando dos trigaes e dos parrecos.\textunderscore »Camillo, \textunderscore Noites de Insómn.\textunderscore , V, 80.
\section{Parreira}
\begin{itemize}
\item {Grp. gram.:f.}
\end{itemize}
\begin{itemize}
\item {Utilização:Prov.}
\end{itemize}
\begin{itemize}
\item {Utilização:alent.}
\end{itemize}
\begin{itemize}
\item {Proveniência:(De \textunderscore parra\textunderscore )}
\end{itemize}
Cepa, cujos ramos se estendem em latada.
Rêde tresmalho, fixa, para emmalhar peixes fóra da água, e usada na ria de Aveiro.
\textunderscore Parreira de carne\textunderscore , enfiadas de chouriços, pendentes do tecto.
\section{Parreira-brava}
\begin{itemize}
\item {Grp. gram.:f.}
\end{itemize}
\begin{itemize}
\item {Utilização:Bras}
\end{itemize}
O mesmo que \textunderscore bútua\textunderscore .
\section{Parreira-da-velha}
\begin{itemize}
\item {Grp. gram.:f.}
\end{itemize}
Casta de uva de Sintra. Cf. \textunderscore Rev. Agron.\textunderscore , I, 18.
\section{Parreiral}
\begin{itemize}
\item {Grp. gram.:m.}
\end{itemize}
Série de parreiras.
\section{Parreira-matias}
\begin{itemize}
\item {Grp. gram.:f.}
\end{itemize}
Casta de uva preta de Collares.
\section{Parreirol}
\begin{itemize}
\item {Grp. gram.:f.}
\end{itemize}
\begin{itemize}
\item {Utilização:T. da Bairrada}
\end{itemize}
Diz-se do vinho de uvas da parreira.
\section{Parrhésia}
\begin{itemize}
\item {Grp. gram.:f.}
\end{itemize}
\begin{itemize}
\item {Proveniência:(Gr. \textunderscore parrhesia\textunderscore )}
\end{itemize}
Affirmação arrojada; atrevimento oratório.
\section{Parricida}
\begin{itemize}
\item {Grp. gram.:m. ,  f.  e  adj.}
\end{itemize}
\begin{itemize}
\item {Proveniência:(Lat. \textunderscore parricida\textunderscore )}
\end{itemize}
Pessôa, que mata pai, mãe, avô ou avó.
\section{Parricídio}
\begin{itemize}
\item {Grp. gram.:m.}
\end{itemize}
\begin{itemize}
\item {Proveniência:(Lat. \textunderscore parricidium\textunderscore )}
\end{itemize}
Crime de parrícida.
\section{Parricrinito}
\begin{itemize}
\item {Grp. gram.:m.}
\end{itemize}
\begin{itemize}
\item {Utilização:Des.}
\end{itemize}
\begin{itemize}
\item {Proveniência:(De \textunderscore parra\textunderscore  + \textunderscore crinito\textunderscore )}
\end{itemize}
Encimado de parras, coroado de parras:«\textunderscore e um gordo almude parricrinito...\textunderscore »Filinto, IV, 102.
\section{Parrilha}
\begin{itemize}
\item {Grp. gram.:f.}
\end{itemize}
Saragoça ordinária.
\section{Parrinheira}
\begin{itemize}
\item {Grp. gram.:f.}
\end{itemize}
\begin{itemize}
\item {Utilização:Prov.}
\end{itemize}
\begin{itemize}
\item {Utilização:minh.}
\end{itemize}
O mesmo que \textunderscore pranheira\textunderscore .
\section{Parro}
\begin{itemize}
\item {Grp. gram.:m.}
\end{itemize}
\begin{itemize}
\item {Utilização:Prov.}
\end{itemize}
\begin{itemize}
\item {Utilização:trasm.}
\end{itemize}
Pato grande.
\section{Párroco}
\textunderscore m.\textunderscore  (e der.)
Fórma pop. de \textunderscore párocho\textunderscore , etc. Cf. Filinto, V, 11.
(Cp. cast. e it. \textunderscore párroco\textunderscore )
\section{Parrolo}
\begin{itemize}
\item {fónica:rô}
\end{itemize}
\begin{itemize}
\item {Grp. gram.:m.}
\end{itemize}
\begin{itemize}
\item {Utilização:T. de Monção}
\end{itemize}
O mesmo que \textunderscore dinheiro\textunderscore .
\section{Parronar}
\begin{itemize}
\item {Grp. gram.:v. i.}
\end{itemize}
\begin{itemize}
\item {Utilização:T. de Águeda}
\end{itemize}
\begin{itemize}
\item {Utilização:Ext.}
\end{itemize}
\begin{itemize}
\item {Proveniência:(De \textunderscore Parronas\textunderscore , n. p.)}
\end{itemize}
Parolar, cavaquear, (no botequim das Parronas).
Colher ou espalhar boatos; mexericar.
\section{Parronice}
\begin{itemize}
\item {Grp. gram.:f.}
\end{itemize}
Qualidade ou hábito de parrono.
\section{Parrono}
\begin{itemize}
\item {Grp. gram.:m.}
\end{itemize}
\begin{itemize}
\item {Utilização:T. de Águeda}
\end{itemize}
Aquelle que parrona.
\section{Parruá}
\begin{itemize}
\item {Grp. gram.:m.}
\end{itemize}
\begin{itemize}
\item {Utilização:T. de curtidor}
\end{itemize}
\begin{itemize}
\item {Proveniência:(Do fr. \textunderscore paroi\textunderscore ?)}
\end{itemize}
Grande bastidor, onde se collocam as pelles para se alisar o carnaz.
\section{Parrudo}
\begin{itemize}
\item {Grp. gram.:m.}
\end{itemize}
\begin{itemize}
\item {Utilização:Pop.}
\end{itemize}
Homem baixo e grosso.
\section{Parruma}
\begin{itemize}
\item {Grp. gram.:f.}
\end{itemize}
\begin{itemize}
\item {Utilização:Prov.}
\end{itemize}
(V.perruma)
\section{Párrya}
\begin{itemize}
\item {Grp. gram.:f.}
\end{itemize}
\begin{itemize}
\item {Proveniência:(De \textunderscore Parry\textunderscore , n. p.)}
\end{itemize}
Gênero de plantas crucíferas do norte da Ásia e da América.
\section{Parsano}
\begin{itemize}
\item {Grp. gram.:adj.}
\end{itemize}
Relativo aos Parses.
\section{Parse}
\begin{itemize}
\item {Grp. gram.:m.}
\end{itemize}
\begin{itemize}
\item {Proveniência:(De \textunderscore Pars\textunderscore , n. p.)}
\end{itemize}
Sectário de Zoroastro.
Guebro.
Língua dos Guebros ou Parses.
\section{Párseo}
\begin{itemize}
\item {Grp. gram.:m.}
\end{itemize}
\begin{itemize}
\item {Utilização:Ant.}
\end{itemize}
O mesmo que \textunderscore parse\textunderscore .
\section{Pársi}
\begin{itemize}
\item {Grp. gram.:m.}
\end{itemize}
O mesmo que \textunderscore parse\textunderscore .
\section{Parsina}
\begin{itemize}
\item {Grp. gram.:m.}
\end{itemize}
Mulhér, de casta dos Parses.
\section{Parsismo}
\begin{itemize}
\item {Grp. gram.:m.}
\end{itemize}
\begin{itemize}
\item {Proveniência:(De \textunderscore parse\textunderscore )}
\end{itemize}
Religião de Zoroastro.
\section{Parsônsia}
\begin{itemize}
\item {Grp. gram.:f.}
\end{itemize}
\begin{itemize}
\item {Proveniência:(De \textunderscore Parsons\textunderscore , n. p.)}
\end{itemize}
Gênero de plantas apocýneas.
\section{Partasana}
\begin{itemize}
\item {Grp. gram.:f.}
\end{itemize}
\begin{itemize}
\item {Utilização:Ant.}
\end{itemize}
\begin{itemize}
\item {Grp. gram.:M.}
\end{itemize}
\begin{itemize}
\item {Utilização:Prov.}
\end{itemize}
\begin{itemize}
\item {Utilização:beir.}
\end{itemize}
Alabarda aguda e larga de infantaria.
Labrego, homem rústico ou boçal.
(Cp. fr. \textunderscore pertuisane\textunderscore )
\section{Parte}
\begin{itemize}
\item {Grp. gram.:f.}
\end{itemize}
\begin{itemize}
\item {Utilização:Jur.}
\end{itemize}
\begin{itemize}
\item {Grp. gram.:Loc.}
\end{itemize}
\begin{itemize}
\item {Utilização:pop.}
\end{itemize}
\begin{itemize}
\item {Grp. gram.:Pl.}
\end{itemize}
\begin{itemize}
\item {Utilização:T. do Minho e Bairrada}
\end{itemize}
\begin{itemize}
\item {Utilização:pop.}
\end{itemize}
\begin{itemize}
\item {Utilização:Pop.}
\end{itemize}
\begin{itemize}
\item {Proveniência:(Lat. \textunderscore pars\textunderscore , \textunderscore partis\textunderscore )}
\end{itemize}
Porção de um todo.
Lote.
Divisão: \textunderscore parte de uma casa\textunderscore .
Ponto, lugar: \textunderscore não pára em parte nenhuma\textunderscore .
O que numa peça de música compete a cada voz ou a cada instrumento.
O que numa representação theatral compete a cada actor.
Litigante: \textunderscore partes legítimas num processo\textunderscore .
Partido, facção.
Lado: \textunderscore na parte de trás\textunderscore .
Communicação verbal ou escrita: \textunderscore o polícia deu parte do crime\textunderscore .
\textunderscore Sêr parte\textunderscore , sêr parcial. Cf. \textunderscore Eufrosina\textunderscore , 117.
Litigar em processo.
\textunderscore De fóra parte\textunderscore , excepto. Cf. Camillo, \textunderscore Curso de Liter.\textunderscore , 31.
Qualidades, prendas: \textunderscore perito em ronha e mais partes\textunderscore .
Partido, facção.
Endróminas, momices; habilidades de saltimbanco.
\textunderscore Partes fracas\textunderscore , ou simplesmente \textunderscore partes\textunderscore , apparelho sexual masculino.
\section{Parteira}
\begin{itemize}
\item {Grp. gram.:f.}
\end{itemize}
Mulhér, que assiste a partos, auxiliando ou soccorrendo as parturientes.
(Fem. de \textunderscore parteiro\textunderscore )
\section{Parteiro}
\begin{itemize}
\item {Grp. gram.:m.}
\end{itemize}
\begin{itemize}
\item {Grp. gram.:Adj.}
\end{itemize}
\begin{itemize}
\item {Proveniência:(De \textunderscore parto\textunderscore )}
\end{itemize}
Médico ou cirurgião, que assiste a partos, ou é especialista em Obstetrícia.
Perito em Obstetrícia.
\section{Partejamento}
\begin{itemize}
\item {Grp. gram.:m.}
\end{itemize}
Acto de partejar.
\section{Partejar}
\begin{itemize}
\item {Grp. gram.:v. t.}
\end{itemize}
\begin{itemize}
\item {Grp. gram.:V. i.}
\end{itemize}
\begin{itemize}
\item {Proveniência:(De \textunderscore parto\textunderscore )}
\end{itemize}
Servir de parteiro ou parteira a.
O mesmo que \textunderscore parir\textunderscore .
\section{Partejo}
\begin{itemize}
\item {Grp. gram.:m.}
\end{itemize}
Acto de partejar.
Offício de parteira. Cf. Filinto, VI, 36.
\section{Partenão}
\begin{itemize}
\item {Grp. gram.:m.}
\end{itemize}
\begin{itemize}
\item {Proveniência:(Do gr. \textunderscore parthenon\textunderscore )}
\end{itemize}
Aposento de donzelas, no lugar mais afastado das habitações, entre os antigos Gregos.
\section{Partênia}
\begin{itemize}
\item {Grp. gram.:f.}
\end{itemize}
\begin{itemize}
\item {Proveniência:(Do gr. \textunderscore parthenos\textunderscore )}
\end{itemize}
Gênero de plantas da América equatorial.
\section{Partenianos}
\begin{itemize}
\item {Grp. gram.:f. pl.}
\end{itemize}
O mesmo que \textunderscore partênios\textunderscore .
\section{Partenina}
\begin{itemize}
\item {Grp. gram.:f.}
\end{itemize}
\begin{itemize}
\item {Proveniência:(De \textunderscore partênia\textunderscore )}
\end{itemize}
Substância febrífuga.
\section{Partênios}
\begin{itemize}
\item {Grp. gram.:m. pl.}
\end{itemize}
\begin{itemize}
\item {Proveniência:(Do gr. \textunderscore parthenos\textunderscore , mulhér não casada)}
\end{itemize}
Filhos ilegítimos, nascidos em Esparta durante a guerra de Messênia.
\section{Partenocisso}
\begin{itemize}
\item {Grp. gram.:f. pl.}
\end{itemize}
\begin{itemize}
\item {Proveniência:(Do lat. \textunderscore parthenium\textunderscore  + \textunderscore cissus\textunderscore )}
\end{itemize}
Videiras virgens, que constituem um gênero da fam. das ampelídeas.
\section{Partenogênese}
\begin{itemize}
\item {Grp. gram.:f.}
\end{itemize}
\begin{itemize}
\item {Utilização:Bot.}
\end{itemize}
\begin{itemize}
\item {Proveniência:(Do gr. \textunderscore parthenos\textunderscore , virgem, e \textunderscore génesis\textunderscore , geração)}
\end{itemize}
Geração, proveniente de fêmeas virgens.
Suposta reproducção sem fecundação. Cf. G. Guimarães, \textunderscore Vocab. Etym.\textunderscore ; Caminhoá, \textunderscore Bot. Ger.\textunderscore  e \textunderscore Méd.\textunderscore 
\section{Partenogenésico}
\begin{itemize}
\item {Grp. gram.:adj.}
\end{itemize}
O mesmo que \textunderscore partenogenético\textunderscore .
\section{Partenogênesis}
\begin{itemize}
\item {Grp. gram.:m.  e  f.}
\end{itemize}
O mesmo que \textunderscore partenogênese\textunderscore .
\section{Partenogenético}
\begin{itemize}
\item {Grp. gram.:adj.}
\end{itemize}
Relativo á partenogênese.
\section{Partenologia}
\begin{itemize}
\item {Grp. gram.:f.}
\end{itemize}
\begin{itemize}
\item {Proveniência:(Do gr. \textunderscore parthenos\textunderscore  + \textunderscore logos\textunderscore )}
\end{itemize}
Tratado médico á cêrca das virgens.
\section{Partênope}
\begin{itemize}
\item {Grp. gram.:f.}
\end{itemize}
\begin{itemize}
\item {Proveniência:(De \textunderscore Parthenope\textunderscore , n. p. ant. de Nápoles)}
\end{itemize}
Pequeno planeta, descoberto em Nápoles em 1850.
Molusco, que habita nas grandes profundidades do Oceano e cujo aspecto é repugnante.
\section{Partenopeu}
\begin{itemize}
\item {Grp. gram.:adj.}
\end{itemize}
\begin{itemize}
\item {Utilização:P. us.}
\end{itemize}
\begin{itemize}
\item {Proveniência:(Lat. \textunderscore parthenopaeus\textunderscore )}
\end{itemize}
Relativo a Nápoles.
\section{Parthenão}
\begin{itemize}
\item {Grp. gram.:m.}
\end{itemize}
\begin{itemize}
\item {Proveniência:(Do gr. \textunderscore parthenon\textunderscore )}
\end{itemize}
Aposento de donzellas, no lugar mais afastado das habitações, entre os antigos Gregos.
\section{Parthênia}
\begin{itemize}
\item {Grp. gram.:f.}
\end{itemize}
\begin{itemize}
\item {Proveniência:(Do gr. \textunderscore parthenos\textunderscore )}
\end{itemize}
Gênero de plantas da América equatorial.
\section{Parthenianos}
\begin{itemize}
\item {Grp. gram.:f. pl.}
\end{itemize}
O mesmo que \textunderscore parthênios\textunderscore .
\section{Parthenina}
\begin{itemize}
\item {Grp. gram.:f.}
\end{itemize}
\begin{itemize}
\item {Proveniência:(De \textunderscore parthênia\textunderscore )}
\end{itemize}
Substância febrífuga.
\section{Parthênios}
\begin{itemize}
\item {Grp. gram.:m. pl.}
\end{itemize}
\begin{itemize}
\item {Proveniência:(Do gr. \textunderscore parthenos\textunderscore , mulhér não casada)}
\end{itemize}
Filhos illegítimos, nascidos em Esparta durante a guerra de Messênia.
\section{Parthenocisso}
\begin{itemize}
\item {Grp. gram.:f. pl.}
\end{itemize}
\begin{itemize}
\item {Proveniência:(Do lat. \textunderscore parthenium\textunderscore  + \textunderscore cissus\textunderscore )}
\end{itemize}
Videiras virgens, que constituem um gênero da fam. das ampelídeas.
\section{Parthenogênese}
\begin{itemize}
\item {Grp. gram.:f.}
\end{itemize}
\begin{itemize}
\item {Utilização:Bot.}
\end{itemize}
\begin{itemize}
\item {Proveniência:(Do gr. \textunderscore parthenos\textunderscore , virgem, e \textunderscore génesis\textunderscore , geração)}
\end{itemize}
Geração, proveniente de fêmeas virgens.
Supposta reproducção sem fecundação. Cf. G. Guimarães, \textunderscore Vocab. Etym.\textunderscore ; Caminhoá, \textunderscore Bot. Ger.\textunderscore  e \textunderscore Méd.\textunderscore 
\section{Parthenogenésico}
\begin{itemize}
\item {Grp. gram.:adj.}
\end{itemize}
O mesmo que \textunderscore parthenogenético\textunderscore .
\section{Parthenogênesis}
\begin{itemize}
\item {Grp. gram.:m.  e  f.}
\end{itemize}
O mesmo que \textunderscore parthenogênese\textunderscore .
\section{Parthenogenético}
\begin{itemize}
\item {Grp. gram.:adj.}
\end{itemize}
Relativo á parthenogênese.
\section{Parthenologia}
\begin{itemize}
\item {Grp. gram.:f.}
\end{itemize}
\begin{itemize}
\item {Proveniência:(Do gr. \textunderscore parthenos\textunderscore  + \textunderscore logos\textunderscore )}
\end{itemize}
Tratado médico á cêrca das virgens.
\section{Parthênope}
\begin{itemize}
\item {Grp. gram.:f.}
\end{itemize}
\begin{itemize}
\item {Proveniência:(De \textunderscore Parthenope\textunderscore , n. p. ant. de Nápoles)}
\end{itemize}
Pequeno planeta, descoberto em Nápoles em 1850.
Mollusco, que habita nas grandes profundidades do Oceano e cujo aspecto é repugnante.
\section{Parthenopeu}
\begin{itemize}
\item {Grp. gram.:adj.}
\end{itemize}
\begin{itemize}
\item {Utilização:P. us.}
\end{itemize}
\begin{itemize}
\item {Proveniência:(Lat. \textunderscore parthenopaeus\textunderscore )}
\end{itemize}
Relativo a Nápoles.
\section{Párthico}
\begin{itemize}
\item {Grp. gram.:adj.}
\end{itemize}
Relativo aos Parthos.
Dizia-se, entre os antigos, de uma qualidade do pão leve e poroso, que embebia facilmente os liquidos. Cf. Castilho, \textunderscore Fastos\textunderscore , III, 479.
\section{Parthos}
\begin{itemize}
\item {Grp. gram.:m. pl.}
\end{itemize}
\begin{itemize}
\item {Proveniência:(Lat. \textunderscore parthi\textunderscore )}
\end{itemize}
Antigo povo da Ásia.
\section{Partição}
\begin{itemize}
\item {Grp. gram.:f.}
\end{itemize}
Acto de partir.
\section{Particimeiro}
\begin{itemize}
\item {Grp. gram.:m.  e  adj.}
\end{itemize}
\begin{itemize}
\item {Utilização:Ant.}
\end{itemize}
\begin{itemize}
\item {Proveniência:(De \textunderscore partição\textunderscore )}
\end{itemize}
Participante; sócio.
\section{Participação}
\begin{itemize}
\item {Grp. gram.:f.}
\end{itemize}
\begin{itemize}
\item {Proveniência:(Lat. \textunderscore participatio\textunderscore )}
\end{itemize}
Acto ou effeito de participar.
\section{Participador}
\begin{itemize}
\item {Grp. gram.:m.  e  adj.}
\end{itemize}
O que participa.
\section{Participante}
\begin{itemize}
\item {Grp. gram.:adj.}
\end{itemize}
\begin{itemize}
\item {Proveniência:(Lat. \textunderscore participans\textunderscore )}
\end{itemize}
Que participa.
\section{Participar}
\begin{itemize}
\item {Grp. gram.:v. i.}
\end{itemize}
\begin{itemize}
\item {Grp. gram.:V. t.}
\end{itemize}
\begin{itemize}
\item {Proveniência:(Lat. \textunderscore participare\textunderscore )}
\end{itemize}
Tomar parte; associar-se.
Têr qualidades ou natureza communs a outro indivíduo ou coisa.
Communicar; tornar alguém sciente de: \textunderscore participar um casamento\textunderscore .
\section{Participável}
\begin{itemize}
\item {Grp. gram.:adj.}
\end{itemize}
Que se póde participar.
\section{Partícipe}
\begin{itemize}
\item {Grp. gram.:m.  e  adj.}
\end{itemize}
\begin{itemize}
\item {Proveniência:(Lat. \textunderscore particeps\textunderscore )}
\end{itemize}
O mesmo que \textunderscore participador\textunderscore .
\section{Participial}
\begin{itemize}
\item {Grp. gram.:adj.}
\end{itemize}
\begin{itemize}
\item {Utilização:Gram.}
\end{itemize}
\begin{itemize}
\item {Proveniência:(Lat. \textunderscore participialis\textunderscore )}
\end{itemize}
Relativo ao particípio.
\section{Particípio}
\begin{itemize}
\item {Grp. gram.:m.}
\end{itemize}
\begin{itemize}
\item {Utilização:Gram.}
\end{itemize}
\begin{itemize}
\item {Proveniência:(Lat. \textunderscore participium\textunderscore )}
\end{itemize}
Palavra, que participa da natureza do verbo e do adjectivo.
\section{Pártico}
\begin{itemize}
\item {Grp. gram.:adj.}
\end{itemize}
Relativo aos Partos.
Dizia-se, entre os antigos, de uma qualidade do pão leve e poroso, que embebia facilmente os liquidos. Cf. Castilho, \textunderscore Fastos\textunderscore , III, 479.
\section{Partícula}
\begin{itemize}
\item {Grp. gram.:f.}
\end{itemize}
\begin{itemize}
\item {Utilização:Gram.}
\end{itemize}
\begin{itemize}
\item {Proveniência:(Lat. \textunderscore particula\textunderscore )}
\end{itemize}
Pequena parte.
Pequena hóstia.
Qualquer palavra invariável, especialmente as muito curtas ou monosyllábicas.
\section{Particular}
\begin{itemize}
\item {Grp. gram.:adj.}
\end{itemize}
\begin{itemize}
\item {Grp. gram.:M.}
\end{itemize}
\begin{itemize}
\item {Grp. gram.:Pl.}
\end{itemize}
\begin{itemize}
\item {Proveniência:(Lat. \textunderscore particularis\textunderscore )}
\end{itemize}
Relativo exclusivamente a certas pessôas ou coisas.
Peculiar, especial.
Minucioso.
Que não é de uso geral do público: \textunderscore gabinete particular\textunderscore .
Fóra do commum, extraordinário.
Secreto, íntimo: \textunderscore negócios particulares\textunderscore .
Aquillo que é particular.
Um indivíduo qualquer.
Minuciosidades, pormenores.
\section{Particulariar}
\begin{itemize}
\item {Grp. gram.:v. t.}
\end{itemize}
\begin{itemize}
\item {Utilização:Ant.}
\end{itemize}
O mesmo que \textunderscore particularizar\textunderscore . Cf. Pant. de Aveiro, 145 v.^o, (2.^a ed.).
\section{Particularidade}
\begin{itemize}
\item {Grp. gram.:f.}
\end{itemize}
\begin{itemize}
\item {Proveniência:(Lat. \textunderscore particularitas\textunderscore )}
\end{itemize}
Qualidade do que é particular; especialidade.
\section{Particularismo}
\begin{itemize}
\item {Grp. gram.:m.}
\end{itemize}
\begin{itemize}
\item {Utilização:Neol.}
\end{itemize}
\begin{itemize}
\item {Utilização:Anthrop.}
\end{itemize}
Qualidade de particular.
Especialidade.
Qualidade privativa.
Qualidade dos povos, em cuja organização prepondera o sentimento individual. Cf. Ortigão, \textunderscore Holanda\textunderscore , 340.
\section{Particularista}
\begin{itemize}
\item {Grp. gram.:adj.}
\end{itemize}
\begin{itemize}
\item {Grp. gram.:M.}
\end{itemize}
\begin{itemize}
\item {Proveniência:(De \textunderscore particular\textunderscore )}
\end{itemize}
Relativo a particularismo.
Aquelle que particulariza ou individualiza. Cf. Ol. Martins, \textunderscore Camões\textunderscore , 311 e 318.
\section{Particularização}
\begin{itemize}
\item {Grp. gram.:f.}
\end{itemize}
Acto ou effeito de particularizar.
\section{Particularizador}
\begin{itemize}
\item {Grp. gram.:adj.}
\end{itemize}
Que particulariza.
\section{Particularizar}
\begin{itemize}
\item {Grp. gram.:v. t.}
\end{itemize}
\begin{itemize}
\item {Proveniência:(De \textunderscore particular\textunderscore )}
\end{itemize}
Narrar minuciosamente.
Mencionar especialmente; individualizar.
\section{Particularmente}
\begin{itemize}
\item {Grp. gram.:adv.}
\end{itemize}
De modo particular.
Em segrêdo.
\section{Partida}
\begin{itemize}
\item {Grp. gram.:f.}
\end{itemize}
\begin{itemize}
\item {Utilização:Fam.}
\end{itemize}
Acto de partir.
Número de jogos, necessário para que um parceiro ganhe.
Reunião de pessôas, com o fim de se distrahirem.
Serão.
Porção de mercadorias, expedidas ou recebidas para commércio.
Remessa.
Trôço de gente armada.
Pirraça; acinte: \textunderscore pregou-lhe uma partida\textunderscore .
\section{Partidão}
\begin{itemize}
\item {Grp. gram.:m.}
\end{itemize}
\begin{itemize}
\item {Utilização:Fam.}
\end{itemize}
\begin{itemize}
\item {Proveniência:(De \textunderscore partido\textunderscore ^2)}
\end{itemize}
Bom arranjo; bôa collocação.
Bom casamento, casamento rico. Cf. Eça, \textunderscore Padre Amaro\textunderscore , 96.
\section{Partidário}
\begin{itemize}
\item {Grp. gram.:m.  e  adj.}
\end{itemize}
O que é membro de um partido; sectário; correligionário.
\section{Partidarismo}
\begin{itemize}
\item {Grp. gram.:m.}
\end{itemize}
\begin{itemize}
\item {Proveniência:(De \textunderscore partidário\textunderscore )}
\end{itemize}
Paixão partidária; proselytismo.
\section{Partidista}
\begin{itemize}
\item {Grp. gram.:m. ,  f.  e  adj.}
\end{itemize}
Pessôa, que segue um partido.
Pessôa, apaixonada por um partido.
\section{Partido}
\begin{itemize}
\item {Grp. gram.:adj.}
\end{itemize}
Que se partiu: \textunderscore loiça partida\textunderscore .
Feito em pedaços.
Quebrado: \textunderscore bengala partida\textunderscore .
\section{Partido}
\begin{itemize}
\item {Grp. gram.:m.}
\end{itemize}
\begin{itemize}
\item {Utilização:Bras}
\end{itemize}
\begin{itemize}
\item {Utilização:Bras}
\end{itemize}
\begin{itemize}
\item {Proveniência:(De \textunderscore parte\textunderscore )}
\end{itemize}
Conjunto de indivíduos, que seguem o mesmo systema ou ideias, especialmente em política.
Parcialidade, facção, bando.
Partes.
Vantagem, proveito.
Expediente, meio: \textunderscore tomou o partido de se calar\textunderscore .
Grande extensão de terreno, plantado de cana de açúcar.
Reunião de dois ou mais pelotários, combinados no jôgo contra outros, também combinados.
\textunderscore Dar partido\textunderscore , dar um jogador vantagens a outro: \textunderscore no bilhar, deu-me, de partido, déz carambolas\textunderscore .
\section{Partidoiras}
\begin{itemize}
\item {Grp. gram.:f. pl.}
\end{itemize}
\begin{itemize}
\item {Proveniência:(De \textunderscore partir\textunderscore )}
\end{itemize}
Pennas, na parte inferior das asas de algumas aves.
\section{Partidor}
\begin{itemize}
\item {Grp. gram.:m.  e  adj.}
\end{itemize}
\begin{itemize}
\item {Proveniência:(De \textunderscore partir\textunderscore )}
\end{itemize}
O que faz partilhas; repartidor.
\section{Partija}
\begin{itemize}
\item {Grp. gram.:f.}
\end{itemize}
\begin{itemize}
\item {Utilização:Ant.}
\end{itemize}
Porção de coisas; multidão.
(Cp. \textunderscore partida\textunderscore )
\section{Partilha}
\begin{itemize}
\item {Grp. gram.:f.}
\end{itemize}
\begin{itemize}
\item {Proveniência:(Do lat. \textunderscore particula\textunderscore )}
\end{itemize}
Repartição dos bens de uma herança.
Divisão de lucros.
Divisão, repartição.
Quinhão.
Attributo.
Pequena peça oblonga de madeira, com que os carpinteiros alargam ou estreitam o rebaixamento feito pelo cantil.
\section{Partilhar}
\begin{itemize}
\item {Grp. gram.:v. t.}
\end{itemize}
Fazer partilha de.
Repartir.
Dividir em partes.
Tomar parte em.--Esta última accepção é rejeitada pelos mestres, que a substituem por \textunderscore participar de\textunderscore .
\section{Partimento}
\begin{itemize}
\item {Grp. gram.:m.}
\end{itemize}
\begin{itemize}
\item {Utilização:P. us.}
\end{itemize}
\begin{itemize}
\item {Utilização:Des.}
\end{itemize}
\begin{itemize}
\item {Utilização:Bot.}
\end{itemize}
O mesmo que \textunderscore compartimento\textunderscore .
Divisão.
Acto de partir, partida, saída.
Lâmina membranosa e vertical, que divide a cavidade do pericarpo em céllulas, e é formada pelo endocarpo.
\section{Partir}
\begin{itemize}
\item {Grp. gram.:v. t.}
\end{itemize}
\begin{itemize}
\item {Grp. gram.:V. i.}
\end{itemize}
\begin{itemize}
\item {Grp. gram.:V. p.}
\end{itemize}
\begin{itemize}
\item {Proveniência:(Lat. \textunderscore partire\textunderscore )}
\end{itemize}
Dividir em partes: \textunderscore partir um pão\textunderscore .
Abrir; quebrar: \textunderscore partir um vidro\textunderscore .
Separar; distribuír.
Mover-se para outro lugar.
Pôr-se a caminho; seguir viagem: \textunderscore partiu para o Pôrto\textunderscore .
Saír.
Principiar, têr origem.
Proseguir.
Confinar, sêr limítrophe:«\textunderscore a quinta da Ermida partia com a delle.\textunderscore »Camillo, \textunderscore Brasileira\textunderscore , 365.
Ausentar-se; seguir viagem.
\section{Partista}
\begin{itemize}
\item {Grp. gram.:adj.}
\end{itemize}
\begin{itemize}
\item {Utilização:Bras}
\end{itemize}
Arisco, assustadiço, (falando-se do cavallo).
\section{Partitivo}
\begin{itemize}
\item {Grp. gram.:adj.}
\end{itemize}
\begin{itemize}
\item {Utilização:Gram.}
\end{itemize}
\begin{itemize}
\item {Proveniência:(De \textunderscore partir\textunderscore )}
\end{itemize}
Que reparte.
Que limita a significação de uma palavra.
\section{Partitura}
\begin{itemize}
\item {Grp. gram.:f.}
\end{itemize}
\begin{itemize}
\item {Proveniência:(It. \textunderscore partitura\textunderscore )}
\end{itemize}
Conjunto das partes, que constituem uma obra musical.
\section{Partível}
\begin{itemize}
\item {Grp. gram.:adj.}
\end{itemize}
\begin{itemize}
\item {Proveniência:(Lat. \textunderscore partibilis\textunderscore )}
\end{itemize}
Que se póde partir.
\section{Parto}
\begin{itemize}
\item {Grp. gram.:m.}
\end{itemize}
\begin{itemize}
\item {Utilização:Fig.}
\end{itemize}
\begin{itemize}
\item {Proveniência:(Lat. \textunderscore partus\textunderscore )}
\end{itemize}
Acto de parir.
Producto, invenção.
\section{Partos}
\begin{itemize}
\item {Grp. gram.:m. pl.}
\end{itemize}
\begin{itemize}
\item {Proveniência:(Lat. \textunderscore parthi\textunderscore )}
\end{itemize}
Antigo povo da Ásia.
\section{Partuno}
\begin{itemize}
\item {Grp. gram.:adj.}
\end{itemize}
\begin{itemize}
\item {Utilização:Ant.}
\end{itemize}
Corr. de \textunderscore importuno\textunderscore ?«Como é \textunderscore partuno\textunderscore , Jesu!»G. Vicente, I, 140.
\section{Parturejar}
\begin{itemize}
\item {Grp. gram.:v. t.}
\end{itemize}
\begin{itemize}
\item {Utilização:Fig.}
\end{itemize}
Produzir ou dar á luz (muitas coisas). Cf. Alv. Mendes, \textunderscore Discursos\textunderscore , 34 e 70. Us. também por Camillo.
(Cp. \textunderscore parturir\textunderscore )
\section{Parturição}
\begin{itemize}
\item {Grp. gram.:f.}
\end{itemize}
\begin{itemize}
\item {Proveniência:(Lat. \textunderscore parturitio\textunderscore )}
\end{itemize}
Parto natural.
\section{Parturiente}
\begin{itemize}
\item {Grp. gram.:f.  e  adj.}
\end{itemize}
\begin{itemize}
\item {Proveniência:(Lat. \textunderscore parturiens\textunderscore )}
\end{itemize}
A mulher ou qualquer fêmea, que está para parir ou que pariu há pouco.
\section{Parturir}
\begin{itemize}
\item {Grp. gram.:v. i.}
\end{itemize}
\begin{itemize}
\item {Proveniência:(Lat. \textunderscore parturir\textunderscore )}
\end{itemize}
O mesmo que \textunderscore parir\textunderscore .
\section{Paru}
\begin{itemize}
\item {Grp. gram.:m.}
\end{itemize}
Peixe acanthopterýgio.
\section{Parúlia}
\begin{itemize}
\item {Grp. gram.:f.}
\end{itemize}
(V.parúlide)
\section{Parúlida}
\begin{itemize}
\item {Grp. gram.:f.}
\end{itemize}
(V.parúlide)
\section{Parúlide}
\begin{itemize}
\item {Grp. gram.:f.}
\end{itemize}
\begin{itemize}
\item {Utilização:Des.}
\end{itemize}
\begin{itemize}
\item {Proveniência:(Do gr. \textunderscore paroulis\textunderscore , \textunderscore paroulidos\textunderscore )}
\end{itemize}
Tumor nas gengivas.
\section{Parva}
\begin{itemize}
\item {Grp. gram.:f.}
\end{itemize}
Ligeira refeição, antes de almôço, ou em vez de almôço; dejejuadoiro.
Quantia pequena.
(Fem. de \textunderscore parvo\textunderscore )
\section{Parvajola}
\begin{itemize}
\item {Grp. gram.:m.  e  f.}
\end{itemize}
\begin{itemize}
\item {Utilização:Burl.}
\end{itemize}
\begin{itemize}
\item {Proveniência:(De \textunderscore parvo\textunderscore )}
\end{itemize}
Pateta; idiota; lapónio.
\section{Parvalhão}
\begin{itemize}
\item {Grp. gram.:m.}
\end{itemize}
O mesmo que \textunderscore parvajola\textunderscore .
Homem da província.
\section{Parvalheira}
\begin{itemize}
\item {Grp. gram.:f.}
\end{itemize}
\begin{itemize}
\item {Utilização:Pop.}
\end{itemize}
\begin{itemize}
\item {Proveniência:(De \textunderscore parvo\textunderscore )}
\end{itemize}
A província.
A vida de aldeia.
\section{Parvalhice}
\begin{itemize}
\item {Grp. gram.:f.}
\end{itemize}
Acção ou dito de parvo.
\section{Parvamente}
\begin{itemize}
\item {Grp. gram.:adv.}
\end{itemize}
De modo parvo; tolamente; á maneira de idiota.
\section{Parvi}
\begin{itemize}
\item {Grp. gram.:m.}
\end{itemize}
Árvore indiana, de fibras têxteis.
\section{Parvidade}
\begin{itemize}
\item {Grp. gram.:f.}
\end{itemize}
\begin{itemize}
\item {Proveniência:(Lat. \textunderscore parvitas\textunderscore )}
\end{itemize}
O mesmo que \textunderscore pequenez\textunderscore .
Qualidade do que é parvo.
\section{Parvo}
\begin{itemize}
\item {Grp. gram.:adj.}
\end{itemize}
\begin{itemize}
\item {Grp. gram.:M.}
\end{itemize}
\begin{itemize}
\item {Proveniência:(Lat. \textunderscore parvus\textunderscore )}
\end{itemize}
Pequeno.
Tolo.
Idiota.
Indivíduo parvo ou atoleimado.
\section{Párvoa}
\begin{itemize}
\item {Grp. gram.:f.  e  adj.}
\end{itemize}
\begin{itemize}
\item {Proveniência:(Do lat. \textunderscore parvula\textunderscore )}
\end{itemize}
Mulhér, que é parva:«\textunderscore ...poemas que essa párvoa não leu...\textunderscore »Filinto, IV, 216.
\section{Parvoalho}
\begin{itemize}
\item {Grp. gram.:m.}
\end{itemize}
\begin{itemize}
\item {Utilização:Bras}
\end{itemize}
Grande parvo; parvalhão. Cf. Pacheco, \textunderscore Promptuário\textunderscore .
\section{Parvoamente}
\begin{itemize}
\item {Grp. gram.:adv.}
\end{itemize}
O mesmo que \textunderscore parvamente\textunderscore .
\section{Parvoeira}
\begin{itemize}
\item {Grp. gram.:f.}
\end{itemize}
O mesmo que \textunderscore parvoíce\textunderscore . Cf. Castilho, \textunderscore Sabichonas\textunderscore , 85.
\section{Parvoeirão}
\begin{itemize}
\item {Grp. gram.:m.  e  adj.}
\end{itemize}
\begin{itemize}
\item {Proveniência:(De \textunderscore parvo\textunderscore )}
\end{itemize}
Homem muito parvo; toleirão.
\section{Parvoeirar}
\begin{itemize}
\item {Grp. gram.:v. i.}
\end{itemize}
Falar ou proceder como parvo.
\section{Parvoejar}
\begin{itemize}
\item {Grp. gram.:v. i.}
\end{itemize}
Falar ou proceder como parvo.
\section{Parvoiçada}
\begin{itemize}
\item {fónica:vo-i}
\end{itemize}
\begin{itemize}
\item {Grp. gram.:f.}
\end{itemize}
Acto ou dito de parvo.
Qualidade ou estado de parvo; demência.
\section{Parvoíce}
\begin{itemize}
\item {Grp. gram.:f.}
\end{itemize}
Acto ou dito de parvo.
Qualidade ou estado de parvo; demência.
\section{Parvoidade}
\begin{itemize}
\item {fónica:vo-i}
\end{itemize}
\begin{itemize}
\item {Grp. gram.:f.}
\end{itemize}
Qualidade de parvo; parvoíce. Cf. Arn. Gama, \textunderscore Segr. do Abbade\textunderscore , 373.
\section{Parvoínho}
\begin{itemize}
\item {Grp. gram.:m.}
\end{itemize}
(dem. de \textunderscore parvo\textunderscore )
\section{Parvolina}
\begin{itemize}
\item {Grp. gram.:f.}
\end{itemize}
\begin{itemize}
\item {Utilização:Chím.}
\end{itemize}
Alcali, que se encontra no alcatrão.
\section{Parvonês}
\begin{itemize}
\item {Grp. gram.:m.}
\end{itemize}
\begin{itemize}
\item {Utilização:Fam.}
\end{itemize}
Indivíduo da parvónia, ocioso ou parvo.
\section{Parvónia}
\begin{itemize}
\item {Grp. gram.:f.}
\end{itemize}
\begin{itemize}
\item {Utilização:Fam.}
\end{itemize}
A capital.
Vida de ociosos e murmuradores. Cf. O'Neill, \textunderscore Fabul.\textunderscore , 655.
A vida da província; a província: \textunderscore um deputado chegou hoje da parvónia\textunderscore .
\section{Parvu}
\begin{itemize}
\item {Grp. gram.:m.}
\end{itemize}
\begin{itemize}
\item {Utilização:Ant.}
\end{itemize}
Escrivão índio.
\section{Parvulez}
\begin{itemize}
\item {Grp. gram.:f.}
\end{itemize}
\begin{itemize}
\item {Proveniência:(De \textunderscore párvulo\textunderscore )}
\end{itemize}
Puerilidade; idade infantil.
Parvoíce.
\section{Parvuleza}
\begin{itemize}
\item {Grp. gram.:f.}
\end{itemize}
O mesmo que \textunderscore parvulez\textunderscore . Cf. Camillo, \textunderscore Ratazzi\textunderscore , 22.
\section{Párvulo}
\begin{itemize}
\item {Grp. gram.:m.}
\end{itemize}
\begin{itemize}
\item {Grp. gram.:Adj.}
\end{itemize}
\begin{itemize}
\item {Utilização:Ant.}
\end{itemize}
\begin{itemize}
\item {Proveniência:(Lat. \textunderscore parvulus\textunderscore )}
\end{itemize}
Criança.
Pequenino.
Parvo, idiota. Cf. Bernárdez, \textunderscore Luz e Calor\textunderscore , 74.
\section{Pasca!}
\begin{itemize}
\item {Grp. gram.:interj.}
\end{itemize}
\begin{itemize}
\item {Utilização:Açor}
\end{itemize}
Toma!
\section{Pascacice}
\begin{itemize}
\item {Grp. gram.:f.}
\end{itemize}
Acto, dito ou qualidade de pascácio.
\section{Pascácio}
\begin{itemize}
\item {Grp. gram.:m.}
\end{itemize}
\begin{itemize}
\item {Utilização:Pop.}
\end{itemize}
Idiota.
Lorpa; indivíduo muito simplório.
\section{Pascal}
\begin{itemize}
\item {Grp. gram.:adj.}
\end{itemize}
\begin{itemize}
\item {Proveniência:(Lat. \textunderscore paschalis\textunderscore )}
\end{itemize}
Relativo á Páscoa.
\section{Pascália}
\begin{itemize}
\item {Grp. gram.:f.}
\end{itemize}
\begin{itemize}
\item {Proveniência:(De \textunderscore Pascal\textunderscore , n. p.)}
\end{itemize}
Gênero de plantas, da fam. das compostas.
\section{Pascalina}
\begin{itemize}
\item {Grp. gram.:f.}
\end{itemize}
\begin{itemize}
\item {Proveniência:(De \textunderscore Pascal\textunderscore , n. p.)}
\end{itemize}
Máquina, para fazer exactas operações arithméticas sem necessidade de raciocinar.
\section{Pascaró}
\begin{itemize}
\item {Grp. gram.:adj.}
\end{itemize}
\begin{itemize}
\item {Utilização:Prov.}
\end{itemize}
\begin{itemize}
\item {Utilização:trasm.}
\end{itemize}
Parvo, tolo.
(Cp. \textunderscore pascácio\textunderscore )
\section{Pascentar}
\textunderscore v. t.\textunderscore  e \textunderscore p.\textunderscore  (e der.)
O mesmo que \textunderscore apascentar\textunderscore , etc.
\section{Pascer}
\begin{itemize}
\item {Grp. gram.:v. t.}
\end{itemize}
\begin{itemize}
\item {Utilização:Fig.}
\end{itemize}
\begin{itemize}
\item {Grp. gram.:V. i.}
\end{itemize}
\begin{itemize}
\item {Utilização:Fig.}
\end{itemize}
\begin{itemize}
\item {Proveniência:(Lat. \textunderscore pascere\textunderscore )}
\end{itemize}
O mesmo que \textunderscore pastar\textunderscore .
Agradar a.
Apascentar-se.
Comprazer-se, recrear-se.
\section{Paschal}
\begin{itemize}
\item {fónica:cal}
\end{itemize}
\begin{itemize}
\item {Grp. gram.:adj.}
\end{itemize}
\begin{itemize}
\item {Proveniência:(Lat. \textunderscore paschalis\textunderscore )}
\end{itemize}
Relativo á Páscoa.
\section{Páschoa}
\begin{itemize}
\item {fónica:co}
\end{itemize}
\begin{itemize}
\item {Grp. gram.:f.}
\end{itemize}
\begin{itemize}
\item {Utilização:Fam.}
\end{itemize}
\begin{itemize}
\item {Utilização:Prov.}
\end{itemize}
\begin{itemize}
\item {Utilização:alent.}
\end{itemize}
\begin{itemize}
\item {Proveniência:(Do lat. \textunderscore pascha\textunderscore )}
\end{itemize}
Festa annual, que os Judeus celebravam, em memória da sua saída do Egypto.
Festa annual, que os Christãos celebram, em memória da resurreição de Christo.
Pessôa garrida e alegre.
Variedade de couve.
\section{Paschoal}
\begin{itemize}
\item {fónica:co}
\end{itemize}
\begin{itemize}
\item {Grp. gram.:adj.}
\end{itemize}
O mesmo que \textunderscore paschal\textunderscore .
\section{Paschoar}
\begin{itemize}
\item {fónica:co}
\end{itemize}
\begin{itemize}
\item {Grp. gram.:v. i.}
\end{itemize}
Festejar a Páschoa.
\section{Paschoéla}
\begin{itemize}
\item {fónica:co}
\end{itemize}
\begin{itemize}
\item {Grp. gram.:f.}
\end{itemize}
\begin{itemize}
\item {Proveniência:(De \textunderscore Páschoa\textunderscore )}
\end{itemize}
Domingo immediato ao da Páschoa.
Semana immediata á Semana Santa.
\section{Pascigo}
\begin{itemize}
\item {Grp. gram.:m.}
\end{itemize}
\begin{itemize}
\item {Utilização:Des.}
\end{itemize}
O mesmo que \textunderscore pastagem\textunderscore .
\section{Pascigoso}
\begin{itemize}
\item {Grp. gram.:adj.}
\end{itemize}
Abundante em pascigos.
\section{Páscoa}
\begin{itemize}
\item {Grp. gram.:f.}
\end{itemize}
\begin{itemize}
\item {Utilização:Fam.}
\end{itemize}
\begin{itemize}
\item {Utilização:Prov.}
\end{itemize}
\begin{itemize}
\item {Utilização:alent.}
\end{itemize}
\begin{itemize}
\item {Proveniência:(Do lat. \textunderscore pascha\textunderscore )}
\end{itemize}
Festa anual, que os Judeus celebravam, em memória da sua saída do Egipto.
Festa anual, que os cristãos celebram, em memória da resurreição de Cristo.
Pessôa garrida e alegre.
Variedade de couve.
\section{Pascoal-bailão}
\begin{itemize}
\item {Grp. gram.:m.}
\end{itemize}
\begin{itemize}
\item {Utilização:Prov.}
\end{itemize}
\begin{itemize}
\item {Utilização:beir.}
\end{itemize}
Pateta, idiota.
\section{Pascoal-gomes}
\begin{itemize}
\item {Grp. gram.:m.}
\end{itemize}
Casta de uva de Azeitão.
\section{Pascoal}
\begin{itemize}
\item {Grp. gram.:adj.}
\end{itemize}
O mesmo que \textunderscore pascal\textunderscore .
\section{Pascoar}
\begin{itemize}
\item {Grp. gram.:v. i.}
\end{itemize}
Festejar a Páscoa.
\section{Pascoeira}
\begin{itemize}
\item {Grp. gram.:f.}
\end{itemize}
Espécie de grama parasita.
\section{Pascoéla}
\begin{itemize}
\item {Grp. gram.:f.}
\end{itemize}
\begin{itemize}
\item {Proveniência:(De \textunderscore Páscoa\textunderscore )}
\end{itemize}
Domingo imediato ao da Páscoa.
Semana imediata á Semana Santa.
\section{Pascoínha}
\begin{itemize}
\item {Grp. gram.:f.}
\end{itemize}
Planta, de flôres amarelas.--Floresce pela Páschoa, e daí o seu nome.
\section{Pasguate}
\begin{itemize}
\item {Grp. gram.:m.}
\end{itemize}
\begin{itemize}
\item {Utilização:T. do Fundão}
\end{itemize}
Idiota; pacóvio.
\section{Pasigrafia}
\begin{itemize}
\item {Grp. gram.:f.}
\end{itemize}
\begin{itemize}
\item {Proveniência:(Do gr. \textunderscore pas\textunderscore  + \textunderscore graphein\textunderscore )}
\end{itemize}
Escrita universal.
Qualquer sistema de escrever, imaginado no intuito de sêr compreendida, sem tradução, por todos os povos.
Um dos sistemas de abreviaturas taquigráficas.
\section{Pasigráfico}
\begin{itemize}
\item {Grp. gram.:adj.}
\end{itemize}
Relativo á pasigrafia.
\section{Pasigraphia}
\begin{itemize}
\item {Grp. gram.:f.}
\end{itemize}
\begin{itemize}
\item {Proveniência:(Do gr. \textunderscore pas\textunderscore  + \textunderscore graphein\textunderscore )}
\end{itemize}
Escrita universal.
Qualquer systema de escrever, imaginado no intuito de sêr comprehendida, sem traducção, por todos os povos.
Um dos systemas de abreviaturas tachygráphicas.
\section{Pasigráphico}
\begin{itemize}
\item {Grp. gram.:adj.}
\end{itemize}
Relativo á pasigraphia.
\section{Pasma}
\begin{itemize}
\item {Grp. gram.:f.}
\end{itemize}
\begin{itemize}
\item {Utilização:Gír.}
\end{itemize}
Sentinela.
\section{Pasmacear}
\begin{itemize}
\item {Grp. gram.:v. i.}
\end{itemize}
\begin{itemize}
\item {Utilização:Fam.}
\end{itemize}
Têr pasmaceira.
Passar vida airada, olhando para tudo, por imbecilidade ou a fim de matar o tempo.
\section{Pasmaceira}
\begin{itemize}
\item {Grp. gram.:f.}
\end{itemize}
\begin{itemize}
\item {Utilização:Pop.}
\end{itemize}
Pasmo; admiração imbecil.
\section{Pasmadamente}
\begin{itemize}
\item {Grp. gram.:adv.}
\end{itemize}
De modo pasmado.
Com pasmo; com admiração imbecil.
\section{Pasmado}
\begin{itemize}
\item {Grp. gram.:adj.}
\end{itemize}
\begin{itemize}
\item {Grp. gram.:M.}
\end{itemize}
\begin{itemize}
\item {Utilização:Bras. de Minas}
\end{itemize}
Espantado; surprehendido.
Inexpressivo; apalermado; que não tem vivacidade: \textunderscore um olhar pasmado\textunderscore .
Pau, fincado no meio do campo: \textunderscore o senhor vai andando e, quando vir um pasmado, volte á direita\textunderscore .
\section{Pasmão}
\begin{itemize}
\item {Grp. gram.:m.}
\end{itemize}
\begin{itemize}
\item {Utilização:Prov.}
\end{itemize}
\begin{itemize}
\item {Utilização:minh.}
\end{itemize}
\begin{itemize}
\item {Proveniência:(De \textunderscore pasmo\textunderscore )}
\end{itemize}
Homem, que pasma de tudo; pacóvio.
\section{Pasmar}
\begin{itemize}
\item {Grp. gram.:v. t.}
\end{itemize}
\begin{itemize}
\item {Grp. gram.:V. i.}
\end{itemize}
Causar pasmo em; espantar; deslumbrar.
Admirar-se.
Desmaiar.
\section{Pasmarota}
\begin{itemize}
\item {Grp. gram.:f.}
\end{itemize}
\begin{itemize}
\item {Utilização:Prov.}
\end{itemize}
\begin{itemize}
\item {Utilização:trasm.}
\end{itemize}
\begin{itemize}
\item {Proveniência:(De \textunderscore pasmar\textunderscore )}
\end{itemize}
O mesmo que \textunderscore paparota\textunderscore .
\section{Pasmatório}
\begin{itemize}
\item {Grp. gram.:m.}
\end{itemize}
\begin{itemize}
\item {Proveniência:(De \textunderscore pasmar\textunderscore )}
\end{itemize}
Grande pasmo.
Praça ou lugar, em que estacionam ou vagueiam pessôas ociosas.
\section{Pasmo}
\begin{itemize}
\item {Grp. gram.:m.}
\end{itemize}
\begin{itemize}
\item {Grp. gram.:Adj.}
\end{itemize}
\begin{itemize}
\item {Utilização:Bras}
\end{itemize}
\begin{itemize}
\item {Proveniência:(Do lat. \textunderscore spasmus\textunderscore )}
\end{itemize}
Assombro, espanto.
Desfallecimento, desmaio.
Pasmado, assombrado: \textunderscore fiquei pasmo\textunderscore !
\section{Pasmosamente}
\begin{itemize}
\item {Grp. gram.:adv.}
\end{itemize}
De modo pasmoso; espantosamente; extraordinariamente.
\section{Pasmoso}
\begin{itemize}
\item {Grp. gram.:adj.}
\end{itemize}
Que produz pasmo; assombroso, admirável.
\section{Paspalhão}
\begin{itemize}
\item {Grp. gram.:m.  e  adj.}
\end{itemize}
\begin{itemize}
\item {Grp. gram.:M.}
\end{itemize}
\begin{itemize}
\item {Utilização:Prov.}
\end{itemize}
\begin{itemize}
\item {Utilização:Prov.}
\end{itemize}
\begin{itemize}
\item {Utilização:alent.}
\end{itemize}
\begin{itemize}
\item {Proveniência:(De \textunderscore paspalho\textunderscore )}
\end{itemize}
Parvo; lorpa.
O mesmo que \textunderscore codorniz\textunderscore .
Nome de uma dança de roda.
\section{Paspalhice}
\begin{itemize}
\item {Grp. gram.:f.}
\end{itemize}
\begin{itemize}
\item {Proveniência:(De \textunderscore paspalho\textunderscore )}
\end{itemize}
Acto ou dito próprio de paspalhão.
\section{Paspalho}
\begin{itemize}
\item {Grp. gram.:m.}
\end{itemize}
O mesmo que \textunderscore paspalhão\textunderscore .
Espantalho.
Pessôa inútil.
\section{Páspalo}
\begin{itemize}
\item {Grp. gram.:m.}
\end{itemize}
Gênero de plantas gramíneas.
\section{Pasquim}
\begin{itemize}
\item {Grp. gram.:m.}
\end{itemize}
\begin{itemize}
\item {Proveniência:(It. \textunderscore pasquino\textunderscore )}
\end{itemize}
Sátira, affixada em lugar público.
Jornal ou folheto, que diffama.
\section{Pasquinada}
\begin{itemize}
\item {Grp. gram.:f.}
\end{itemize}
O mesmo que \textunderscore pasquim\textunderscore .
\section{Pasquinagem}
\begin{itemize}
\item {Grp. gram.:f.}
\end{itemize}
Acto de pasquinar.
Hábitos de pasquineiro.
Diffamação por escrito, em papéis avulsos, como pasquins, jornaes, manifestos, etc.
\section{Pasquinar}
\begin{itemize}
\item {Grp. gram.:v. t.}
\end{itemize}
\begin{itemize}
\item {Grp. gram.:V. i.}
\end{itemize}
Satirizar por meio de pasquins.
Fazer pasquins.
\section{Pasquineiro}
\begin{itemize}
\item {Grp. gram.:m.}
\end{itemize}
Autor de pasquim ou de pasquins; diffamador.
\section{Pasquinhas}
\begin{itemize}
\item {Grp. gram.:f.}
\end{itemize}
Planta leguminosa, (\textunderscore coronilla glauca\textunderscore , Lin.). Cf. P. Coutinho, \textunderscore Flora de Port.\textunderscore , 357.-- Suponho que é confusão e escrita errónea, em vez de \textunderscore pascoínhas\textunderscore .
Cp. \textunderscore pascoínha\textunderscore .
\section{Passa}
\begin{itemize}
\item {Grp. gram.:f.}
\end{itemize}
\begin{itemize}
\item {Proveniência:(Do lat. \textunderscore passus\textunderscore )}
\end{itemize}
Fruta sêca, principalmente uva.
\section{Passa-culpas}
\begin{itemize}
\item {Grp. gram.:m.  e  f.}
\end{itemize}
Pessôa muito indulgente, que tudo desculpa; confessor indulgente.
\section{Passada}
\begin{itemize}
\item {Grp. gram.:f.}
\end{itemize}
\begin{itemize}
\item {Utilização:Ant.}
\end{itemize}
\begin{itemize}
\item {Grp. gram.:Loc. adv.}
\end{itemize}
\begin{itemize}
\item {Proveniência:(De \textunderscore passar\textunderscore )}
\end{itemize}
Movimento dos pés para andar; passo.
Espaço, comprehendido entre os pontos em que poisam os pés, andando.
Antiga medida de quatro palmos.
Permissão tácita.
Disfarce.
\textunderscore De passada\textunderscore , de passagem; conjuntamente, ao mesmo tempo. Cf. \textunderscore Eufrosina\textunderscore , 190.
\section{Passadeira}
\begin{itemize}
\item {Grp. gram.:f.}
\end{itemize}
\begin{itemize}
\item {Utilização:T. do Fundão}
\end{itemize}
\begin{itemize}
\item {Proveniência:(De \textunderscore passar\textunderscore )}
\end{itemize}
Alpondras.
Cada um dos degraus, feitos de alvenaria, tejolo ou telha, em cima dos telhados.
Apparelho, para avaliar o calibre das balas de artilharia.
Vaso, com que se passa de uns tachos para outros o mellado nos engenhos de açúcar.
Braçadeira.
Larga tira ou espécie de teia, que se estende nos pavimentos e escadas, para sôbre ella se passar.
Cabo náutico de três cordões de linho.
Correia de coiro, que passa no olhal da prítica, no carro de bois, e se prende na canga para aguentar o carro nas descidas.
Lugar, onde se põe fruta a secar.
\section{Passa-de-Viseu}
\begin{itemize}
\item {Grp. gram.:f.}
\end{itemize}
Variedade de pêra, bôa para secar.
\section{Passa-dez}
\begin{itemize}
\item {Grp. gram.:m.}
\end{itemize}
Jôgo de dados, em que se perde, quando se tira um número superior a dez.
\section{Passadiço}
\begin{itemize}
\item {Grp. gram.:m.}
\end{itemize}
\begin{itemize}
\item {Grp. gram.:adj.}
\end{itemize}
\begin{itemize}
\item {Proveniência:(De \textunderscore passar\textunderscore )}
\end{itemize}
Passagem.
Corredor ou galeria que communica dois edifícios.
Passeio lateral das ruas. Cf. Camillo, \textunderscore Enxertado\textunderscore , 192.
Passageiro, transitório; breve.
\section{Passadio}
\begin{itemize}
\item {Grp. gram.:m.}
\end{itemize}
\begin{itemize}
\item {Proveniência:(De \textunderscore passar\textunderscore )}
\end{itemize}
Alimentação diária.
\section{Passado}
\begin{itemize}
\item {Grp. gram.:adj.}
\end{itemize}
\begin{itemize}
\item {Grp. gram.:M.}
\end{itemize}
\begin{itemize}
\item {Grp. gram.:Pl.}
\end{itemize}
Atordoado, espantado: \textunderscore fiquei passado\textunderscore !
Sêco: \textunderscore uvas passadas\textunderscore .
Decorrido, findo: \textunderscore no anno passado\textunderscore .
Realizado anteriormente: \textunderscore na sessão passada\textunderscore .
Que teve certa preparação.
O tempo que passou.
O que se fez ou disse anteriormente.
Antepassados. Cf. Usque, 52.
\section{Passadoiro}
\begin{itemize}
\item {Grp. gram.:m.}
\end{itemize}
\begin{itemize}
\item {Utilização:Marn.}
\end{itemize}
\begin{itemize}
\item {Proveniência:(De \textunderscore passar\textunderscore )}
\end{itemize}
Lugar, por onde se passa.
Ponto de passagem; communicação:«\textunderscore ...os passadouros para a ilha nem poucos nem diffíceis\textunderscore ». Filinto, \textunderscore Vida de D. Man.\textunderscore , II, 174.
Pequeno muro, por onde os marnotos conduzem o sal, do taboleiro para as eiras.
\section{Passador}
\begin{itemize}
\item {Grp. gram.:adj.}
\end{itemize}
\begin{itemize}
\item {Grp. gram.:M.}
\end{itemize}
\begin{itemize}
\item {Grp. gram.:Pron.}
\end{itemize}
\begin{itemize}
\item {Utilização:alent}
\end{itemize}
\begin{itemize}
\item {Utilização:Ant.}
\end{itemize}
Que passa.
Aquelle que faz passar, ou que transporta.
Desencaminhador.
Aquelle que ardilosamente vende ou dá com interesse objectos falsos ou de pouco valor real ou superior.
Utensílio culinário com orifícios, por onde se espreme ou côa qualquer massa; coador.
Homem do norte, que trabalha temporariamente na limpeza de herdades ou no fabríco de carvão.
Passa-culpas. Cf. \textunderscore Luz e Calor\textunderscore , 511.
\section{Passadouro}
\begin{itemize}
\item {Grp. gram.:m.}
\end{itemize}
\begin{itemize}
\item {Utilização:Marn.}
\end{itemize}
\begin{itemize}
\item {Proveniência:(De \textunderscore passar\textunderscore )}
\end{itemize}
Lugar, por onde se passa.
Ponto de passagem; communicação:«\textunderscore ...os passadouros para a ilha nem poucos nem diffíceis\textunderscore ». Filinto, \textunderscore Vida de D. Man.\textunderscore , II, 174.
Pequeno muro, por onde os marnotos conduzem o sal, do taboleiro para as eiras.
\section{Passa-fomes}
\begin{itemize}
\item {Grp. gram.:m.}
\end{itemize}
Nome que, nas Caldas-da-Rainha, se dá ao \textunderscore papa-formigas\textunderscore .
\section{Passa-fóra!}
\begin{itemize}
\item {Grp. gram.:interj.}
\end{itemize}
(para enxotar cães, ou indicativa de repulsão e desprêzo). Cf. Castilho, \textunderscore Fausto\textunderscore , 276.
\section{Passageiramente}
\begin{itemize}
\item {Grp. gram.:adv.}
\end{itemize}
De modo passageiro; transitoriamente.
\section{Passageiro}
\begin{itemize}
\item {Grp. gram.:adj.}
\end{itemize}
\begin{itemize}
\item {Grp. gram.:M.}
\end{itemize}
\begin{itemize}
\item {Utilização:Bras}
\end{itemize}
\begin{itemize}
\item {Proveniência:(De \textunderscore passagem\textunderscore )}
\end{itemize}
Em que há passagem ou por onde passa muita gente.
Transitório, ephêmero: \textunderscore gozos passageiros\textunderscore .
Que tem pequena importância: \textunderscore culpas passageiras\textunderscore .
Deminuto.
Aquelle que vai passando, em viagem; transeunte, viandante.
Navegante, com exclusão dos tripulantes.
Barqueiro, que transporta gente, de uma para outra margem do rio.
\section{Passagem}
\begin{itemize}
\item {Grp. gram.:f.}
\end{itemize}
\begin{itemize}
\item {Utilização:Prov.}
\end{itemize}
\begin{itemize}
\item {Utilização:Ant.}
\end{itemize}
\begin{itemize}
\item {Proveniência:(Do b. lat. \textunderscore passago\textunderscore )}
\end{itemize}
Acto ou effeito de passar.
Lugar, onde se passa.
Quantia, com que o passageiro paga a passagem em qualquer vehículo.
Ponteado, com que se tapa um buraco ou rasgão num tecido, passando e repassando o fio.
Episódio ou trecho de uma obra.
Conjuntura.
Acontecimento.
Caso engraçado; episódio cómico.
Transição.
Traspasse.
Passadiço.
Transgressão.
Tributo, que pagavam os que passassem por certas terras.--Como \textunderscore episódio\textunderscore  ou \textunderscore trecho de obra\textunderscore , é considerado gallicismo por alguns.
\section{Passajar}
\begin{itemize}
\item {Grp. gram.:v. t.}
\end{itemize}
Dar passagens ou pontos em roupa, para a consertar.
\section{Passal}
\begin{itemize}
\item {Grp. gram.:m.}
\end{itemize}
\begin{itemize}
\item {Proveniência:(De \textunderscore passo\textunderscore )}
\end{itemize}
Terreno cultivado, annexo e pertencente á residência de um párocho ou prelado.
Antiga medida agrária.
\section{Pássalo}
\begin{itemize}
\item {Grp. gram.:m.}
\end{itemize}
\begin{itemize}
\item {Proveniência:(Do gr. \textunderscore passalos\textunderscore )}
\end{itemize}
Gênero de insectos coleópteros, da fam. dos lamellicórneos.
\section{Passamanar}
\begin{itemize}
\item {Grp. gram.:v. t.}
\end{itemize}
O mesmo que \textunderscore apassamanar\textunderscore .
\section{Passamanaria}
\begin{itemize}
\item {Grp. gram.:f.}
\end{itemize}
Obra de passamanes; offício ou estabelecimento de passamanes.
\section{Passamaneiro}
\begin{itemize}
\item {Grp. gram.:m.}
\end{itemize}
Fabricante ou vendedor de passamanes.
\section{Passamanes}
\begin{itemize}
\item {Grp. gram.:m. pl.}
\end{itemize}
\begin{itemize}
\item {Proveniência:(It. \textunderscore passamano\textunderscore )}
\end{itemize}
Fitas ou galões, entretecidos a fios de prata, oiro ou seda.
\section{Passamaque}
\begin{itemize}
\item {Grp. gram.:m.}
\end{itemize}
\begin{itemize}
\item {Utilização:Des.}
\end{itemize}
Espécie de calçado. Cf. Castilho, \textunderscore D. Quixote\textunderscore , I, 306.
\section{Passamente}
\begin{itemize}
\item {Grp. gram.:adv.}
\end{itemize}
\begin{itemize}
\item {Utilização:Ant.}
\end{itemize}
\begin{itemize}
\item {Proveniência:(De \textunderscore passo\textunderscore )}
\end{itemize}
Devagar.
Em segrêdo; em voz baixa.
\section{Passamento}
\begin{itemize}
\item {Grp. gram.:m.}
\end{itemize}
\begin{itemize}
\item {Proveniência:(De \textunderscore passar\textunderscore )}
\end{itemize}
Morte, fallecimento.
Estertor de moribundo.
\section{Pàssamuros}
\begin{itemize}
\item {Grp. gram.:m.}
\end{itemize}
\begin{itemize}
\item {Proveniência:(De \textunderscore passar\textunderscore  + \textunderscore muro\textunderscore )}
\end{itemize}
Espécie de canhão antigo de ferro.
\section{Passandito}
(dem. de \textunderscore passando\textunderscore , gerúndio de \textunderscore passar\textunderscore ):«\textunderscore andamos... passandito a par...\textunderscore »Castilho, \textunderscore Sonho de Uma Noite\textunderscore .
Cp. \textunderscore dormindinho\textunderscore .
\section{Passanito}
\begin{itemize}
\item {Grp. gram.:m.}
\end{itemize}
\begin{itemize}
\item {Utilização:Bras}
\end{itemize}
\begin{itemize}
\item {Utilização:Fam.}
\end{itemize}
Homem de pouca importância; qualquer sujeito.
(Cp. \textunderscore passante\textunderscore )
\section{Passa-novas}
\begin{itemize}
\item {Grp. gram.:m.  e  f.}
\end{itemize}
\begin{itemize}
\item {Utilização:Prov.}
\end{itemize}
Pessôa chocalheira, mexeriqueira. (Colhido em Turquel)
\section{Passante}
\begin{itemize}
\item {Grp. gram.:adj.}
\end{itemize}
\begin{itemize}
\item {Grp. gram.:M.}
\end{itemize}
\begin{itemize}
\item {Grp. gram.:Loc. prep.}
\end{itemize}
Que passa.
Excellente.
Indivíduo, que vai passando; transeunte. Cf. Filinto, XVII, 135.
\textunderscore Passante de\textunderscore , mais de. Cf. \textunderscore Peregrinação\textunderscore , XXXVI.
\section{Passa-passa}
\begin{itemize}
\item {Grp. gram.:m.}
\end{itemize}
O mesmo que \textunderscore passe-passe\textunderscore . Cf. B. Pereira, \textunderscore Prosodia\textunderscore , vb. \textunderscore prestigiator\textunderscore .
\section{Passa-pé}
\begin{itemize}
\item {Grp. gram.:m.}
\end{itemize}
\begin{itemize}
\item {Proveniência:(Do fr. \textunderscore passe-pied\textunderscore )}
\end{itemize}
Espécie de dança antiga a três tempos e de movimento rápido.
\section{Pàssapelo}
\begin{itemize}
\item {fónica:pê}
\end{itemize}
\begin{itemize}
\item {Grp. gram.:m.}
\end{itemize}
\begin{itemize}
\item {Utilização:Ant.}
\end{itemize}
\begin{itemize}
\item {Proveniência:(De \textunderscore passar\textunderscore  + \textunderscore pêlo\textunderscore )}
\end{itemize}
Guarnição de pelles, que resalta da orla das roupas.
Vivos da farda militar. Cf. Garção, II, 10.
\section{Passa-pié}
\begin{itemize}
\item {Grp. gram.:m.}
\end{itemize}
\begin{itemize}
\item {Proveniência:(Fr. \textunderscore passe-pied\textunderscore )}
\end{itemize}
O mesmo que \textunderscore passa-pé\textunderscore . Cf. Filinto, IX, 143.
\section{Passa-piôlho}
\begin{itemize}
\item {Grp. gram.:m.}
\end{itemize}
\begin{itemize}
\item {Utilização:Pop.}
\end{itemize}
Talhe de barba, de uma á outra orelha, por baixo do queixo.
\section{Pàssaporte}
\begin{itemize}
\item {Grp. gram.:m.}
\end{itemize}
\begin{itemize}
\item {Utilização:Fam.}
\end{itemize}
\begin{itemize}
\item {Proveniência:(Fr. \textunderscore passeport\textunderscore )}
\end{itemize}
Licença escrita, para que alguém possa saír do país.
Salvo-conducto.
Faculdade ampla, licença.
\section{Passar}
\begin{itemize}
\item {Grp. gram.:v. t.}
\end{itemize}
\begin{itemize}
\item {Utilização:Des.}
\end{itemize}
\begin{itemize}
\item {Grp. gram.:V. i.}
\end{itemize}
\begin{itemize}
\item {Grp. gram.:V. p.}
\end{itemize}
Transpor; atravessar, passando de um lado para outro: \textunderscore passar o rio\textunderscore .
Transportar: \textunderscore passar uma criança ao collo\textunderscore .
Pôr em circulação; impingir: \textunderscore passar dinheiro falso\textunderscore .
Entregar: \textunderscore passou o furto ao cúmplice\textunderscore .
Furar de lado a lado: \textunderscore passou-o com o punhal\textunderscore .
Coar-se através de; coar; peneirar.
Exceder.
Enfiar: \textunderscore passar uma linha pela agulha\textunderscore .
Consumir.
Fazer deslizar sôbre alguma coisa.
Secar: \textunderscore passar uvas\textunderscore .
Padecer: \textunderscore passar tormentos\textunderscore .
Desfrutar: \textunderscore passar noites deliciosas\textunderscore .
Desculpar.
Mudar de lugar: \textunderscore passou do primeiro para o segundo andar\textunderscore .
Mudar de situação ou occupação: \textunderscore passou de padre para negociante\textunderscore .
Transitar.
Deslizar.
Prolongar-se.
Introduzir-se: \textunderscore as bruxas passam pelas fechaduras\textunderscore .
Fazer transição.
Extinguir-se, acabar: \textunderscore êsse bello tempo passou\textunderscore .
Tornar-se esquecido.
Morrer.
Declarar que se não faz jôgo, ao voltarete.
Sêr tolerável, acceitável.
Transmittir-se: \textunderscore passou de pais a filhos\textunderscore .
Decorrer: \textunderscore o tempo passa\textunderscore .
Viver.
Alimentar-se: \textunderscore aquelles criados passam muito bem\textunderscore .
Achar-se: \textunderscore vou passando bem\textunderscore .
Remediar-se.
Acontecer: \textunderscore contaram-me o que ontem passou\textunderscore .
Mudar de partido.
Succeder: \textunderscore coisas que se passam\textunderscore .
(B. lat. \textunderscore passare\textunderscore )
\section{Pássara}
\begin{itemize}
\item {Grp. gram.:f.}
\end{itemize}
\begin{itemize}
\item {Utilização:Gír.}
\end{itemize}
\begin{itemize}
\item {Utilização:Ant.}
\end{itemize}
Partes pudendas da mulhér, o mesmo que \textunderscore passarinha\textunderscore .
O mesmo que \textunderscore perdiz\textunderscore .
\section{Passarada}
\begin{itemize}
\item {Grp. gram.:f.}
\end{itemize}
Porção de pássaros; os pássaros.
\section{Passarão}
\begin{itemize}
\item {Grp. gram.:m.}
\end{itemize}
\begin{itemize}
\item {Proveniência:(De \textunderscore pássaro\textunderscore )}
\end{itemize}
Ave grande.
\section{Passaredo}
\begin{itemize}
\item {fónica:sarê}
\end{itemize}
\begin{itemize}
\item {Grp. gram.:m.}
\end{itemize}
Grande número de pássaros, considerados em geral.
\section{Passareira}
\begin{itemize}
\item {Grp. gram.:f.}
\end{itemize}
Gaiola grande, para criação de pássaros; aviário.
Casta de azeitona, também conhecida por \textunderscore zambulha\textunderscore  e \textunderscore enxeta\textunderscore .
\section{Passareiro}
\begin{itemize}
\item {Grp. gram.:m.}
\end{itemize}
\begin{itemize}
\item {Grp. gram.:Adj.}
\end{itemize}
\begin{itemize}
\item {Utilização:Prov.}
\end{itemize}
\begin{itemize}
\item {Proveniência:(De \textunderscore pássaro\textunderscore )}
\end{itemize}
O mesmo que \textunderscore passarinheiro\textunderscore .
Diz-se do cão de caça, que no campo se distrai com as pequenas aves que passam.
\section{Passareiro}
\begin{itemize}
\item {Grp. gram.:m.}
\end{itemize}
\begin{itemize}
\item {Utilização:Ant.}
\end{itemize}
\begin{itemize}
\item {Proveniência:(De \textunderscore pássara\textunderscore )}
\end{itemize}
Caçador de perdizes.
\section{Passarela}
\begin{itemize}
\item {Grp. gram.:f.}
\end{itemize}
\begin{itemize}
\item {Utilização:Prov.}
\end{itemize}
\begin{itemize}
\item {Utilização:beir.}
\end{itemize}
\begin{itemize}
\item {Proveniência:(De \textunderscore pássaro\textunderscore )}
\end{itemize}
Qualquer passarinho.
\section{Passarinha}
\begin{itemize}
\item {Grp. gram.:f.}
\end{itemize}
\begin{itemize}
\item {Utilização:Gír.}
\end{itemize}
\begin{itemize}
\item {Proveniência:(De \textunderscore pássaro\textunderscore )}
\end{itemize}
Baço de porco com gordura.
Partes pudendas da mulhér.
\section{Passarinhada}
\begin{itemize}
\item {Grp. gram.:f.}
\end{itemize}
O mesmo que \textunderscore passaredo\textunderscore .
\section{Passarinhagem}
\begin{itemize}
\item {Grp. gram.:f.}
\end{itemize}
\begin{itemize}
\item {Proveniência:(De \textunderscore passarinho\textunderscore )}
\end{itemize}
Caça de pássaros. Cf. Filinto, \textunderscore D. Man.\textunderscore , I, 187.
\section{Passarinhar}
\begin{itemize}
\item {Grp. gram.:v. i.}
\end{itemize}
\begin{itemize}
\item {Utilização:Bras}
\end{itemize}
Andar á caça de pássaros.
Vadiar.
Espantar-se o cavallo.
\section{Passarinheiro}
\begin{itemize}
\item {Grp. gram.:m.}
\end{itemize}
\begin{itemize}
\item {Utilização:Bras}
\end{itemize}
\begin{itemize}
\item {Proveniência:(De \textunderscore passarinhar\textunderscore )}
\end{itemize}
Caçador, criador ou vendedor de pássaros.
Cavallo espantadiço.
\section{Passarinho}
\begin{itemize}
\item {Grp. gram.:m.}
\end{itemize}
\begin{itemize}
\item {Utilização:Bras}
\end{itemize}
\begin{itemize}
\item {Utilização:Bras}
\end{itemize}
Pequeno pássaro.
Casta de uva tinta do Minho.
Árvore silvestre, de flôres vermelhas ou amarelas.
Espécie de planta parasita.
\section{Passarinho-a-olhar}
\begin{itemize}
\item {Grp. gram.:m.}
\end{itemize}
Espécie de jôgo popular.
\section{Passarinho-á-orelha}
\begin{itemize}
\item {Grp. gram.:m.}
\end{itemize}
O mesmo que \textunderscore passarinho-a-olhar\textunderscore .
\section{Passarinho-branco}
\begin{itemize}
\item {Grp. gram.:m.}
\end{itemize}
Uma das espécies da uva chamada \textunderscore passarinho\textunderscore .
\section{Passarinho-bravo}
\begin{itemize}
\item {Grp. gram.:m.}
\end{itemize}
Uma das espécies da uva chamada \textunderscore passarinho\textunderscore .
\section{Passarinho-trigueiro}
\begin{itemize}
\item {Grp. gram.:m.}
\end{itemize}
O mesmo que \textunderscore trigueirão\textunderscore .
\section{Passa-rios}
\begin{itemize}
\item {Grp. gram.:m.}
\end{itemize}
O mesmo que \textunderscore ostraceiro\textunderscore .
E o mesmo que \textunderscore pica-peixe\textunderscore , ave.
\section{Pássaro}
\begin{itemize}
\item {Grp. gram.:m.}
\end{itemize}
\begin{itemize}
\item {Utilização:Pop.}
\end{itemize}
\begin{itemize}
\item {Proveniência:(Lat. \textunderscore passer\textunderscore )}
\end{itemize}
Qualquer ave da ordem dos pásseres.
Pequena ave.
Homem astuto.
\section{Pássaro-do-mel}
\begin{itemize}
\item {Grp. gram.:m.}
\end{itemize}
Pequena ave africana, (\textunderscore cucus indicator\textunderscore ).
\section{Passarola}
\begin{itemize}
\item {Grp. gram.:f.}
\end{itemize}
Grande ave.
Nome, que se deu ao aeróstato, inventado por Bartholomeu de Gusmão.
\section{Passarolo}
\begin{itemize}
\item {fónica:sarô}
\end{itemize}
\begin{itemize}
\item {Grp. gram.:m.}
\end{itemize}
Pássaro grande. Cf. Camillo, \textunderscore Filha do Arced.\textunderscore , c. XVI.
\section{Pássaro-vôa}
\begin{itemize}
\item {Grp. gram.:m.}
\end{itemize}
Espécie de jôgo popular.
\section{Passatempear}
\begin{itemize}
\item {Grp. gram.:v. t.}
\end{itemize}
\begin{itemize}
\item {Utilização:P. us.}
\end{itemize}
\begin{itemize}
\item {Proveniência:(De \textunderscore pàssatempo\textunderscore )}
\end{itemize}
Facultar passatempos a; servir de passatempo a; recrear;«\textunderscore passatempeai os reis, lisonjeai-os...\textunderscore »Filinto, XIII, 64.
\section{Pàssatempo}
\begin{itemize}
\item {Grp. gram.:m.}
\end{itemize}
\begin{itemize}
\item {Proveniência:(De \textunderscore passar\textunderscore  + \textunderscore tempo\textunderscore )}
\end{itemize}
Divertimento; diversão.
\section{Passavante}
\begin{itemize}
\item {Grp. gram.:m.}
\end{itemize}
\begin{itemize}
\item {Proveniência:(De \textunderscore passar\textunderscore  + \textunderscore avante\textunderscore )}
\end{itemize}
Official de casa real, que antigamente tinha a seu cargo annunciar paz ou guerra.
\section{Pàssavolante}
\begin{itemize}
\item {Grp. gram.:m.}
\end{itemize}
\begin{itemize}
\item {Utilização:Ant.}
\end{itemize}
\begin{itemize}
\item {Proveniência:(De \textunderscore passar\textunderscore  + \textunderscore volante\textunderscore )}
\end{itemize}
Espécie de canhão de madeira, bronzeado, só para fazer número em baterias.
\section{Passe}
\begin{itemize}
\item {Grp. gram.:m.}
\end{itemize}
\begin{itemize}
\item {Grp. gram.:Pl.}
\end{itemize}
\begin{itemize}
\item {Proveniência:(De \textunderscore passar\textunderscore )}
\end{itemize}
Licença; consentimento.
Permissão, para ir de um lugar para outro.
Bilhete de trânsito gratuito.
Bilhete de trânsito.
Acto de passar um toiro á capa.
Acto de passar as mãos, repetidas vezes, por deante dos olhos de quem se quere magnetizar.
\section{Passeado}
\begin{itemize}
\item {Grp. gram.:adj.}
\end{itemize}
Diz-se do vinho que se prepara, calcando as uvas com os pés calçados em tamancos ou sapatos ferrados.
\section{Passeadoiro}
\begin{itemize}
\item {Grp. gram.:m.}
\end{itemize}
Lugar, em que se passeia.
\section{Passeador}
\begin{itemize}
\item {Grp. gram.:adj.}
\end{itemize}
\begin{itemize}
\item {Grp. gram.:M.}
\end{itemize}
\begin{itemize}
\item {Utilização:T. da Nazaré}
\end{itemize}
\begin{itemize}
\item {Proveniência:(De \textunderscore passear\textunderscore )}
\end{itemize}
Que passeia.
Aquelle que passeia.
Um dos pescadores que trabalham no levantamento das redes, e que executa nesse serviço um movimento semelhante a passeio, indo e voltando.
\section{Passeadouro}
\begin{itemize}
\item {Grp. gram.:m.}
\end{itemize}
Lugar, em que se passeia.
\section{Passeandito}
(fórma demin. do gerúndio de \textunderscore passear\textunderscore :«\textunderscore eu e ella andámos muito manas passeandito a par.\textunderscore »Castilho, \textunderscore Sonho\textunderscore )
\section{Passeante}
\begin{itemize}
\item {Grp. gram.:m. ,  f.  e  adj.}
\end{itemize}
\begin{itemize}
\item {Proveniência:(De \textunderscore passear\textunderscore )}
\end{itemize}
Pessôa, que passeia.
Pessôa, que vadia.
\section{Passear}
\begin{itemize}
\item {Grp. gram.:v. t.}
\end{itemize}
\begin{itemize}
\item {Utilização:Fig.}
\end{itemize}
\begin{itemize}
\item {Grp. gram.:V. i.}
\end{itemize}
\begin{itemize}
\item {Utilização:Fig.}
\end{itemize}
\begin{itemize}
\item {Proveniência:(De \textunderscore passo\textunderscore )}
\end{itemize}
Conduzir em passeio: \textunderscore passear crianças\textunderscore .
Percorrer em passeio:«\textunderscore ...passeando gentilmente o gracioso campo.\textunderscore »Usque.
Espraiar, espalhar: \textunderscore passear a vista pelas praias\textunderscore .
Andar a passo; dar passos.
Fazer exercício, caminhando.
Andar por distracção.
Jornadear para recreio.
Mover-se devagar.
\section{Passeata}
\begin{itemize}
\item {Grp. gram.:f.}
\end{itemize}
\begin{itemize}
\item {Utilização:Fam.}
\end{itemize}
Pequeno passeio.
\section{Passeio}
\begin{itemize}
\item {Grp. gram.:m.}
\end{itemize}
Acto ou effeito de passear.
Lugar, em que se passeia.
Jardim, em que se póde passear.
Parte lateral e um pouco elevada de algumas ruas, destinada especialmente ao trânsito de pessôas a pé.
\section{Passeira}
\begin{itemize}
\item {Grp. gram.:f.}
\end{itemize}
\begin{itemize}
\item {Proveniência:(Do lat. \textunderscore passaria\textunderscore )}
\end{itemize}
Lugar, onde se secam frutas.
Sítio, em que se guardam passas.
\section{Passeiro}
\begin{itemize}
\item {Grp. gram.:adj.}
\end{itemize}
\begin{itemize}
\item {Utilização:Fig.}
\end{itemize}
Que anda devagar, a passo.
Vagaroso; negligente.
\section{Passeivão}
\begin{itemize}
\item {Grp. gram.:m.}
\end{itemize}
Residência dos antigos reis dos Vatas? Cf. \textunderscore Peregrinação\textunderscore , XVIII.
\section{Passento}
\begin{itemize}
\item {Grp. gram.:adj.}
\end{itemize}
\begin{itemize}
\item {Proveniência:(De \textunderscore passar\textunderscore )}
\end{itemize}
Diz-se das substâncias, que um líquido repassa facilmente: \textunderscore há papel muito passento\textunderscore .
\section{Passe-passe}
\begin{itemize}
\item {Grp. gram.:m.}
\end{itemize}
\begin{itemize}
\item {Utilização:Pop.}
\end{itemize}
\begin{itemize}
\item {Proveniência:(De \textunderscore passar\textunderscore )}
\end{itemize}
O mesmo que \textunderscore prestidigitação\textunderscore . Cf. \textunderscore Eufrosina\textunderscore , 261.
\section{Pásseres}
\begin{itemize}
\item {Grp. gram.:m.}
\end{itemize}
\begin{itemize}
\item {Proveniência:(Lat. \textunderscore passer\textunderscore )}
\end{itemize}
Ordem de aves pequenas, que comprehende os dentirostros, conirostros, fissirostros, tenuirostros e syndáctylos.
\section{Passés}
\begin{itemize}
\item {Grp. gram.:m. pl.}
\end{itemize}
Índios selvagens das margens do Japurá, no Brasil.
\section{Passibilidade}
\begin{itemize}
\item {Grp. gram.:f.}
\end{itemize}
\begin{itemize}
\item {Proveniência:(Lat. \textunderscore passibilitas\textunderscore )}
\end{itemize}
Qualidade do que é passível ou passivo.
\section{Passiflora}
\begin{itemize}
\item {Grp. gram.:f.}
\end{itemize}
\begin{itemize}
\item {Proveniência:(Do lat. \textunderscore passio\textunderscore  + \textunderscore flos\textunderscore , \textunderscore floris\textunderscore )}
\end{itemize}
Gênero de plantas, que crescem especialmente na América tropical, e a que pertence o martýrio.
\section{Passifloráceas}
\begin{itemize}
\item {Grp. gram.:f. pl.}
\end{itemize}
O mesmo ou melhor que \textunderscore passiflóreas\textunderscore .
\section{Passiflóreas}
\begin{itemize}
\item {Grp. gram.:f. pl.}
\end{itemize}
\begin{itemize}
\item {Proveniência:(De \textunderscore passiflóreo\textunderscore )}
\end{itemize}
Família de plantas, que tem por typo a passiflora.
\section{Passiflóreo}
\begin{itemize}
\item {Grp. gram.:adj.}
\end{itemize}
Relativo ou semelhante á passiflora.
\section{Passiflorina}
\begin{itemize}
\item {Grp. gram.:f.}
\end{itemize}
Alcali, extrahido da raíz de uma passiflora.
\section{Passigrafia}
\textunderscore f.\textunderscore  (e der.)
O mesmo ou melhor que \textunderscore pasigrafia\textunderscore , etc.
\section{Passigraphia}
\textunderscore f.\textunderscore  (e der.)
O mesmo ou melhor que \textunderscore pasigrafia\textunderscore , etc.
\section{Passilargo}
\begin{itemize}
\item {Grp. gram.:adj.}
\end{itemize}
\begin{itemize}
\item {Proveniência:(De \textunderscore passo\textunderscore ^1 + \textunderscore largo\textunderscore )}
\end{itemize}
Que dá passos largos.
\section{Passinhas}
\begin{itemize}
\item {Grp. gram.:m.}
\end{itemize}
\begin{itemize}
\item {Utilização:Ant.}
\end{itemize}
\begin{itemize}
\item {Utilização:Fam.}
\end{itemize}
\begin{itemize}
\item {Proveniência:(De \textunderscore passo\textunderscore )}
\end{itemize}
Indivíduo, que tem passo miúdo; pisa-flôres; salta-pocinhas.
\section{Passional}
\begin{itemize}
\item {Grp. gram.:adj.}
\end{itemize}
\begin{itemize}
\item {Utilização:Neol.}
\end{itemize}
\begin{itemize}
\item {Grp. gram.:M.}
\end{itemize}
\begin{itemize}
\item {Proveniência:(Lat. \textunderscore passionalis\textunderscore )}
\end{itemize}
Relativo a paixão.
Susceptível de paixão.
O mesmo que \textunderscore passionário\textunderscore .
\section{Passionário}
\begin{itemize}
\item {Grp. gram.:m.}
\end{itemize}
\begin{itemize}
\item {Proveniência:(Do lat. \textunderscore passio\textunderscore )}
\end{itemize}
O mesmo que \textunderscore paixoeiro\textunderscore .
\section{Passioneiro}
\begin{itemize}
\item {Grp. gram.:m.}
\end{itemize}
O mesmo que \textunderscore paixoeiro\textunderscore .
Qualquer livro, que contenha os cantos litúrgicos, próprios da Semana Santa.
\section{Passiva}
\begin{itemize}
\item {Grp. gram.:f.}
\end{itemize}
\begin{itemize}
\item {Utilização:Gram.}
\end{itemize}
\begin{itemize}
\item {Proveniência:(De \textunderscore passivo\textunderscore )}
\end{itemize}
Voz passiva dos verbos.
\section{Passivamente}
\begin{itemize}
\item {Grp. gram.:adv.}
\end{itemize}
De modo passivo.
\section{Passivar}
\begin{itemize}
\item {Grp. gram.:v.}
\end{itemize}
\begin{itemize}
\item {Utilização:t. Gram.}
\end{itemize}
\begin{itemize}
\item {Utilização:Fig.}
\end{itemize}
\begin{itemize}
\item {Proveniência:(De \textunderscore passivo\textunderscore )}
\end{itemize}
Dar significação ou fórma passiva a (um verbo).
Tornar indifferente.
\section{Passível}
\begin{itemize}
\item {Grp. gram.:adj.}
\end{itemize}
\begin{itemize}
\item {Proveniência:(Do lat. \textunderscore passibilis\textunderscore )}
\end{itemize}
Susceptível de sensações, de soffrimento, de alegria, etc. Cf. Vieira, VI, 20.
\section{Passividade}
\begin{itemize}
\item {Grp. gram.:f.}
\end{itemize}
\begin{itemize}
\item {Proveniência:(Do lat. \textunderscore passivitas\textunderscore )}
\end{itemize}
Qualidade do que é passivo.
\section{Passivo}
\begin{itemize}
\item {Grp. gram.:adj.}
\end{itemize}
\begin{itemize}
\item {Utilização:Gram.}
\end{itemize}
\begin{itemize}
\item {Proveniência:(Lat. \textunderscore passivus\textunderscore )}
\end{itemize}
Que soffre uma acção ou impressão.
Que não exerce acção.
Inerte; indifferente.
Diz-se dos verbos, cuja acção é recebida pelo respectivo sujeito.
\section{Passo}
\begin{itemize}
\item {Grp. gram.:m.}
\end{itemize}
\begin{itemize}
\item {Utilização:Fam.}
\end{itemize}
\begin{itemize}
\item {Utilização:Ant.}
\end{itemize}
\begin{itemize}
\item {Grp. gram.:Adv.}
\end{itemize}
\begin{itemize}
\item {Grp. gram.:Loc. adv.}
\end{itemize}
\begin{itemize}
\item {Grp. gram.:Loc. conj.}
\end{itemize}
\begin{itemize}
\item {Grp. gram.:Loc. adv.}
\end{itemize}
\begin{itemize}
\item {Proveniência:(Lat. \textunderscore passus\textunderscore )}
\end{itemize}
Acto de collocar um pé adeante do outro, para andar.
Marcha.
Caminho.
Modo de andar.
Vestígio do pé, pègada.
Passagem, geralmente estreita.
Espaço, comprehendido entre os dois pés, quando um delles se desloca na marcha.
Conjuntura, situação.
Negócio.
Acção.
Caso.
Antiga medida itinerária.
Cada um dos episódios da paixão de Christo.
Representação dêsses episódios.
Episódio de uma obra literária.
Caso divertido.
Vau em um rio.
Lentamente; sem ruido:«\textunderscore ...afastêmo-nos mui passo.\textunderscore »Júl. Castilho, \textunderscore Manuelinas\textunderscore , 261.
\textunderscore Filhar passos de segurança\textunderscore , tomar hábito religioso, fazer-se frade. Cf. \textunderscore Port. Mon. Hist.\textunderscore , \textunderscore Script.\textunderscore , 309.
\textunderscore A cada passo\textunderscore , vulgarmente, frequentemente.
\textunderscore Ao passo que\textunderscore , ao mesmo tempo que; em-quanto.
\textunderscore Ao mesmo passo\textunderscore , ou \textunderscore do mesmo passo\textunderscore , ao mesmo tempo, simultaneamente.
\section{Passo}
\begin{itemize}
\item {Grp. gram.:adj.}
\end{itemize}
\begin{itemize}
\item {Utilização:Ant.}
\end{itemize}
\begin{itemize}
\item {Proveniência:(Lat. \textunderscore passus\textunderscore , de \textunderscore pati\textunderscore ?)}
\end{itemize}
Soffredor? paciente?:«\textunderscore O Deos lá de cima tão longo e tão passo\textunderscore ». G. Vicente, 307.
\section{Passo}
\begin{itemize}
\item {Grp. gram.:adj.}
\end{itemize}
\begin{itemize}
\item {Proveniência:(De \textunderscore passar\textunderscore )}
\end{itemize}
O mesmo que \textunderscore passado\textunderscore  ou sêco, (falando-se de fruta): \textunderscore figos passos\textunderscore . Cf. Castilho, \textunderscore D. Quixote\textunderscore , I, 320.
\section{Passoca}
\begin{itemize}
\item {Grp. gram.:f.}
\end{itemize}
\begin{itemize}
\item {Utilização:Bras}
\end{itemize}
\begin{itemize}
\item {Utilização:Bras. do N}
\end{itemize}
Iguaria, feita de carne assada ou frita com manteiga e pisada com farinha.
Castanha do Maranhão, torrada e misturada com farinha de mandioca e açúcar.
(Do tupi)
\section{Passoélo}
\begin{itemize}
\item {Grp. gram.:m.}
\end{itemize}
\begin{itemize}
\item {Utilização:Bras}
\end{itemize}
Alforje de coiro cru.
\section{Passota}
\begin{itemize}
\item {Grp. gram.:f.}
\end{itemize}
\begin{itemize}
\item {Utilização:Prov.}
\end{itemize}
\begin{itemize}
\item {Utilização:trasm.}
\end{itemize}
Cereja sêca, engelhada.
(Cp. \textunderscore passa\textunderscore )
\section{Passozinho}
\begin{itemize}
\item {Grp. gram.:adv.}
\end{itemize}
\begin{itemize}
\item {Utilização:Ant.}
\end{itemize}
\begin{itemize}
\item {Proveniência:(De \textunderscore passo\textunderscore ^1)}
\end{itemize}
Devagar; mansamente.
\section{Pasta}
\begin{itemize}
\item {Grp. gram.:f.}
\end{itemize}
\begin{itemize}
\item {Utilização:Fig.}
\end{itemize}
\begin{itemize}
\item {Utilização:Pop.}
\end{itemize}
\begin{itemize}
\item {Proveniência:(Lat. \textunderscore pasta\textunderscore )}
\end{itemize}
Porção de massa achatada.
Porção de metal fundido, e ainda não trabalhado.
Espécie de carteira de papelão, de coiro ou de madeira, para acondicionar desenhos, etc.
Cargo de Ministro de Estado: \textunderscore a pasta da justiça\textunderscore .
Pessôa indolente e bronca.
\section{Pastadeira}
\begin{itemize}
\item {Grp. gram.:adj. f.}
\end{itemize}
\begin{itemize}
\item {Proveniência:(De \textunderscore pastar\textunderscore )}
\end{itemize}
Diz-se da raiz, que se estende á superfície da terra. (Colhido em Arganil)
\section{Pastagem}
\begin{itemize}
\item {Grp. gram.:f.}
\end{itemize}
O mesmo que \textunderscore pasto\textunderscore .
Lugar, onde pasta ou onde póde pastar gado.
Vegetação, própria para nella pastar o gado.
\section{Pastar}
\begin{itemize}
\item {Grp. gram.:v. t.}
\end{itemize}
\begin{itemize}
\item {Grp. gram.:V. i.}
\end{itemize}
\begin{itemize}
\item {Utilização:Fig.}
\end{itemize}
\begin{itemize}
\item {Proveniência:(Do b. lat. \textunderscore pastare\textunderscore )}
\end{itemize}
Comer a erva de.
Comer a erva que ainda está na terra, (falando-se de gados).
Pascer.
Comprazer-se; nutrir-se.
\section{Pastejar}
\begin{itemize}
\item {Grp. gram.:v. i.}
\end{itemize}
O mesmo que \textunderscore pastar\textunderscore . Cf. Filinto, XIII, 191.
\section{Pastel}
\begin{itemize}
\item {Grp. gram.:m.}
\end{itemize}
\begin{itemize}
\item {Utilização:Fam.}
\end{itemize}
\begin{itemize}
\item {Proveniência:(Do lat. hyp. \textunderscore pastellum\textunderscore )}
\end{itemize}
Massa de farinha cozida no forno, e contendo carne, ou peixe, ou doce, etc.
Porção de caracteres typográphicos, misturados e confundidos.
Pessôa indolente.
Processo de desenhar ou pintar com lápis de côres.
Pintura ou desenho, feito por êsse processo.
\section{Pastelão}
\begin{itemize}
\item {Grp. gram.:m.}
\end{itemize}
Grande pastel ou grande empada.
\section{Pastelaria}
\begin{itemize}
\item {Grp. gram.:f.}
\end{itemize}
\begin{itemize}
\item {Proveniência:(De \textunderscore pastel\textunderscore )}
\end{itemize}
Estabelecimento ou arte de pasteleiro.
\section{Pastel-dos-tintureiros}
\begin{itemize}
\item {Grp. gram.:m.}
\end{itemize}
Planta crucífera, (\textunderscore isatis tinctoria\textunderscore , Brotero).
\section{Pasteleira}
\textunderscore fem.\textunderscore  de pasteleiro.
\section{Pasteleiro}
\begin{itemize}
\item {Grp. gram.:m.}
\end{itemize}
\begin{itemize}
\item {Proveniência:(De \textunderscore pastel\textunderscore )}
\end{itemize}
Fabricante ou vendedor de pastéis.
\section{Pastelista}
\begin{itemize}
\item {Grp. gram.:m.  e  f.}
\end{itemize}
Pessôa, que pinta ou desenha, segundo o processo chamado \textunderscore pastel\textunderscore .
\section{Pasteurela}
\begin{itemize}
\item {Grp. gram.:f.}
\end{itemize}
O vírus da pasteurelose.
\section{Pasteurella}
\begin{itemize}
\item {Grp. gram.:f.}
\end{itemize}
O vírus da pasteurellose.
\section{Pasteurellose}
\begin{itemize}
\item {Grp. gram.:f.}
\end{itemize}
Designação genérica de várias doenças de animaes, como a febre typhóide dos cavallos, as septicemias hemorrhágicas do boi, carneiro, etc. Cf. \textunderscore Rev. Med. Veterinaria\textunderscore , I, 5.
\section{Pasteurelose}
\begin{itemize}
\item {Grp. gram.:f.}
\end{itemize}
Designação genérica de várias doenças de animaes, como a febre tifóide dos cavalos, as septicemias hemorrágicas do boi, carneiro, etc. Cf. \textunderscore Rev. Med. Veterinaria\textunderscore , I, 5.
\section{Pasteuriano}
\begin{itemize}
\item {Grp. gram.:adj.}
\end{itemize}
Relativo ao sábio Pasteur.
\section{Pasteurização}
\begin{itemize}
\item {Grp. gram.:f.}
\end{itemize}
Acto ou effeito de pasteurizar.
\section{Pasteurizador}
\begin{itemize}
\item {Grp. gram.:m.}
\end{itemize}
Apparelho para pasteurizar. Cf. \textunderscore Século\textunderscore , de 28-VII-900.
\section{Pasteurizar}
\begin{itemize}
\item {Grp. gram.:v. t.}
\end{itemize}
\begin{itemize}
\item {Proveniência:(De \textunderscore Pasteur\textunderscore , n. p.)}
\end{itemize}
Esterilizar (o leite), aquecendo-o a 50 graus, e fazendo-o esfriar de repente.--Sôbre a pronúncia dêste voc. e seus der., bem como sôbre a vantagem da sua substituição por \textunderscore pastorizar\textunderscore , etc., v. G. Viana, \textunderscore Apostilas\textunderscore .
\section{Pasticho}
\begin{itemize}
\item {Grp. gram.:m.}
\end{itemize}
\begin{itemize}
\item {Utilização:Gal}
\end{itemize}
\begin{itemize}
\item {Proveniência:(Fr. \textunderscore pastiche\textunderscore )}
\end{itemize}
Obra ordinária de pintura, cujo autor imitou mal um mestre, no desenho, no colorido, etc.
Imitação ruím de uma bôa obra literária. Cf. Th. Braga, \textunderscore Mod. Ideias\textunderscore , I, 207.
\section{Pastilha}
\begin{itemize}
\item {Grp. gram.:f.}
\end{itemize}
\begin{itemize}
\item {Proveniência:(De \textunderscore pasta\textunderscore )}
\end{itemize}
Pasta de açúcar, contendo uma essência ou um medicamento.
Medicamento, contido em duas rodelas de hóstia, que se adaptam na orla, tomando o aspecto de pastilha propriamente dita.
\section{Pastilheiro}
\begin{itemize}
\item {Grp. gram.:m.}
\end{itemize}
\begin{itemize}
\item {Utilização:Bras}
\end{itemize}
Vaso de madeira ou metal, para preparo de pastilhas. Cf. \textunderscore Tarifa das Alfând.\textunderscore , do Brasil.
\section{Pastinaga}
\begin{itemize}
\item {Grp. gram.:f.}
\end{itemize}
\begin{itemize}
\item {Proveniência:(Lat. \textunderscore pastinaca\textunderscore )}
\end{itemize}
Nome scientífico da bisnaga, planta.
\section{Pastinhar}
\begin{itemize}
\item {Grp. gram.:v. i.}
\end{itemize}
\begin{itemize}
\item {Utilização:Pop.}
\end{itemize}
Comer pouco ou sem appetite.
Provar iguarias, em vez de as comer.
Debicar.
(Cp. \textunderscore pastinheiro\textunderscore )
\section{Pastinheiro}
\begin{itemize}
\item {Grp. gram.:adj.}
\end{itemize}
\begin{itemize}
\item {Utilização:Fam.}
\end{itemize}
\begin{itemize}
\item {Proveniência:(De \textunderscore pasto\textunderscore ?)}
\end{itemize}
O mesmo que \textunderscore debiqueiro\textunderscore .
\section{Pastio}
\begin{itemize}
\item {Grp. gram.:m.}
\end{itemize}
Terreno, em que há pastagem; pasto.
\section{Pasto}
\begin{itemize}
\item {Grp. gram.:m.}
\end{itemize}
\begin{itemize}
\item {Utilização:Fig.}
\end{itemize}
\begin{itemize}
\item {Proveniência:(Lat. \textunderscore pastus\textunderscore )}
\end{itemize}
Alimento do gado; pascigo.
Comida.
Alimento espiritual.
Regosijo.
Assumpto.
\textunderscore Casa de pasto\textunderscore , espécie de estalagem, em que se serve comida a fregueses de occasião, que geralmente não pernoitam alli.
\textunderscore Vinho de pasto\textunderscore , vinho commum, usado de preferência e em geral nas refeições.
\section{Pastó}
\begin{itemize}
\item {Grp. gram.:m.}
\end{itemize}
O mesmo que \textunderscore afegane\textunderscore .
\section{Pastor}
\begin{itemize}
\item {Grp. gram.:m.}
\end{itemize}
\begin{itemize}
\item {Utilização:Fig.}
\end{itemize}
\begin{itemize}
\item {Grp. gram.:M.  e  f.}
\end{itemize}
\begin{itemize}
\item {Utilização:Ant.}
\end{itemize}
\begin{itemize}
\item {Grp. gram.:Adj.}
\end{itemize}
\begin{itemize}
\item {Utilização:Bras}
\end{itemize}
\begin{itemize}
\item {Proveniência:(Lat. \textunderscore pastor\textunderscore )}
\end{itemize}
Aquelle que guarda gado; zagal, pegureiro.
Párocho.
Rapaz ou rapariga.
Campestre.
Diz-se do cavallo para padreação.
\section{Pastora}
\begin{itemize}
\item {fónica:tô}
\end{itemize}
\begin{itemize}
\item {Grp. gram.:f.}
\end{itemize}
Rapariga ou mulhér, que guarda gado.
(Fem. de \textunderscore pastor\textunderscore )
\section{Pastoral}
\begin{itemize}
\item {Grp. gram.:adj.}
\end{itemize}
\begin{itemize}
\item {Grp. gram.:F.}
\end{itemize}
\begin{itemize}
\item {Proveniência:(Lat. \textunderscore pastoralis\textunderscore )}
\end{itemize}
Relativo a pastor.
Offício ou carta circular, dirigida por um Prelado ao clero ou aos fiéis da sua diocese.
Composição poética do gênero pastoril; égloga.
\section{Pastorar}
\begin{itemize}
\item {Grp. gram.:v. t.}
\end{itemize}
O mesmo que \textunderscore pastorear\textunderscore . Cf. Alv. Azevedo, \textunderscore Cancion. da Madeira\textunderscore , 53.
\section{Pastoreação}
\begin{itemize}
\item {Grp. gram.:f.}
\end{itemize}
Acto de pastorear. Cf. Baganha, \textunderscore Hyg. Pec.\textunderscore , 228.
\section{Pastorear}
\begin{itemize}
\item {Grp. gram.:v. t.}
\end{itemize}
\begin{itemize}
\item {Utilização:Fig.}
\end{itemize}
\begin{itemize}
\item {Proveniência:(De \textunderscore pastor\textunderscore )}
\end{itemize}
Guiar no pasto.
Guardar (o gado que anda pastando).
Dirigir, governar: \textunderscore pastorear uma diocese\textunderscore .
\section{Pastoreio}
\begin{itemize}
\item {Grp. gram.:m.}
\end{itemize}
Indústria de pastorear.
\section{Pastorela}
\begin{itemize}
\item {Grp. gram.:f.}
\end{itemize}
\begin{itemize}
\item {Proveniência:(De \textunderscore pastor\textunderscore )}
\end{itemize}
Antigo diálogo pastoril, figurado entre uma pastora e um cavalleiro.
Canto pastoril; égloga.
Pastorinha. Cf. Garrett, \textunderscore Fabulas\textunderscore , 56.
\section{Pastorícia}
\begin{itemize}
\item {Grp. gram.:f.}
\end{itemize}
\begin{itemize}
\item {Proveniência:(De \textunderscore pastorício\textunderscore )}
\end{itemize}
Profissão de pastor.
\section{Pastoricidas}
\begin{itemize}
\item {Grp. gram.:f. pl.}
\end{itemize}
\begin{itemize}
\item {Proveniência:(Do lat. \textunderscore pastor\textunderscore  + \textunderscore caedere\textunderscore )}
\end{itemize}
Nome, que se deu aos Anabaptistas ingleses, os quaes, no século XVI, voltavam o seu furor especialmente contra os padres, assassinando-os onde os encontrassem.
\section{Pastorício}
\begin{itemize}
\item {Grp. gram.:adj.}
\end{itemize}
\begin{itemize}
\item {Proveniência:(Lat. \textunderscore pastoricius\textunderscore )}
\end{itemize}
Relativo a pastores ou á indústria dos gados.
\section{Pastoril}
\begin{itemize}
\item {Grp. gram.:adj.}
\end{itemize}
\begin{itemize}
\item {Utilização:Fig.}
\end{itemize}
\begin{itemize}
\item {Proveniência:(De \textunderscore pastor\textunderscore )}
\end{itemize}
O mesmo que pastoral.
Relativo á vida de pastor.
Próprio de pastor: \textunderscore frauta pastoril\textunderscore .
Campezino rústico; bucólico: \textunderscore a vida pastoril\textunderscore .
\section{Pastorização}
\begin{itemize}
\item {Grp. gram.:f.}
\end{itemize}
Acto de pastorizar.
\section{Pastorizar}
\begin{itemize}
\item {Grp. gram.:v. t.}
\end{itemize}
O mesmo que \textunderscore pastorear\textunderscore .
\section{Pastoso}
\begin{itemize}
\item {Grp. gram.:adj.}
\end{itemize}
\begin{itemize}
\item {Utilização:Fig.}
\end{itemize}
\begin{itemize}
\item {Proveniência:(De \textunderscore pasta\textunderscore )}
\end{itemize}
Que está em pasta.
Viscoso.
Xaroposo.
Diz-se da voz arrastada e pouco clara.
\section{Pastovinador}
\begin{itemize}
\item {Grp. gram.:m.}
\end{itemize}
Enothérmico, para o aquècimento dos vinhos pelo processo de Pasteur.
\section{Pastranha}
\begin{itemize}
\item {Grp. gram.:m.  e  f.}
\end{itemize}
\begin{itemize}
\item {Utilização:Bras. do S}
\end{itemize}
Pessôa mollangueira, pouco atilada.
\section{Pastrano}
\begin{itemize}
\item {Grp. gram.:m.  e  adj.}
\end{itemize}
Indivíduo grosseiro, rústico:«\textunderscore é de pastrano andar ás ordens da consorte.\textunderscore »Castilho, \textunderscore Sabichonas\textunderscore , 84.
(Por \textunderscore pastorano\textunderscore , de \textunderscore pastor\textunderscore )
\section{Pastura}
\begin{itemize}
\item {Grp. gram.:f.}
\end{itemize}
\begin{itemize}
\item {Utilização:Ant.}
\end{itemize}
O mesmo que \textunderscore pasto\textunderscore :«\textunderscore ó vales de tam alta pastura.\textunderscore »Usque, 11.
\section{Pastural}
\begin{itemize}
\item {Grp. gram.:adj.}
\end{itemize}
\begin{itemize}
\item {Utilização:Ant.}
\end{itemize}
Relativo a pastura.
\section{Pata}
\begin{itemize}
\item {Grp. gram.:f.}
\end{itemize}
A fêmea do pato.
(Fem. de \textunderscore pato\textunderscore )
\section{Pata}
\begin{itemize}
\item {Grp. gram.:f.}
\end{itemize}
\begin{itemize}
\item {Utilização:Chul.}
\end{itemize}
\begin{itemize}
\item {Utilização:Chul.}
\end{itemize}
Pé ou mão de animal.
Extremidade da âncora.
Pé grande.
Pé.
\textunderscore Andar á pata\textunderscore , andar a pé.
(Or. germ.)
\section{Pataca}
\begin{itemize}
\item {Grp. gram.:f.}
\end{itemize}
\begin{itemize}
\item {Utilização:Bras}
\end{itemize}
\begin{itemize}
\item {Proveniência:(Do ár. \textunderscore bátaca\textunderscore ?)}
\end{itemize}
Moéda brasileira de prata.
Quantia de dinheiro igual a 320 reis.
\section{Patacão}
\begin{itemize}
\item {Grp. gram.:m.}
\end{itemize}
\begin{itemize}
\item {Proveniência:(De \textunderscore pataca\textunderscore )}
\end{itemize}
Moéda de cobre do tempo de D. João III.
Antiga moéda brasileira.
Moéda uruguaiana, do valor de 972 reis proximamente.
\section{Patacaria}
\begin{itemize}
\item {Grp. gram.:f.}
\end{itemize}
\begin{itemize}
\item {Utilização:Pop.}
\end{itemize}
Dinheiro.
Porção de patacos.
\section{Patachim}
\begin{itemize}
\item {Grp. gram.:m.}
\end{itemize}
\begin{itemize}
\item {Proveniência:(T. onom. Cp. \textunderscore chapim\textunderscore ^2)}
\end{itemize}
O mesmo que \textunderscore megengra\textunderscore .
\section{Patachina}
\begin{itemize}
\item {Grp. gram.:f.}
\end{itemize}
A fêmea do patachim.
\section{Patacho}
\begin{itemize}
\item {Grp. gram.:m.}
\end{itemize}
Embarcação de dois mastros.
(Cast. \textunderscore patacho\textunderscore )
\section{Patacho}
\begin{itemize}
\item {Grp. gram.:m.}
\end{itemize}
\begin{itemize}
\item {Utilização:Bras. do N}
\end{itemize}
Facão de lâmina curta e larga.
\section{Pàtachoca}
\begin{itemize}
\item {Grp. gram.:m.}
\end{itemize}
\begin{itemize}
\item {Utilização:Chul.}
\end{itemize}
\begin{itemize}
\item {Grp. gram.:F.}
\end{itemize}
\begin{itemize}
\item {Proveniência:(De \textunderscore pata\textunderscore  + \textunderscore chôco\textunderscore )}
\end{itemize}
Servente de sacristia.
Mulhér gorda e indolente.
\section{Patachós}
\begin{itemize}
\item {Grp. gram.:m. pl.}
\end{itemize}
Antiga nação de Índios, que dominaram nos territórios da Baía, no Brasil.
\section{Pataco}
\begin{itemize}
\item {Grp. gram.:m.}
\end{itemize}
\begin{itemize}
\item {Utilização:Fig.}
\end{itemize}
\begin{itemize}
\item {Grp. gram.:Pl.}
\end{itemize}
\begin{itemize}
\item {Utilização:Pop.}
\end{itemize}
\begin{itemize}
\item {Grp. gram.:Loc.}
\end{itemize}
\begin{itemize}
\item {Utilização:Fam.}
\end{itemize}
Antiga moéda de bronze, do valor de 40 reis.
Homem estúpido.
Dinheiro.
\textunderscore Passar a patacos\textunderscore , vender.
(Refl. de \textunderscore pataca\textunderscore ?)
\section{Patacoada}
\begin{itemize}
\item {Grp. gram.:f.}
\end{itemize}
\begin{itemize}
\item {Proveniência:(De \textunderscore pataco\textunderscore )}
\end{itemize}
Impostura ridícula, jactância.
Disparate.
\section{Patada}
\begin{itemize}
\item {Grp. gram.:f.}
\end{itemize}
\begin{itemize}
\item {Utilização:Fig.}
\end{itemize}
Pancada com a pata.
Acção indecorosa; tolice.
\section{Pata-galharda}
\begin{itemize}
\item {Grp. gram.:f.}
\end{itemize}
Jôgo de rapazes, no qual o jogador deve bater com o pau noutro que está erguido no chão e, levantando-o com a pancada, bater-lhe no ar e em certas direcções. Cf. Ficalho, \textunderscore Contos\textunderscore , 186.
\section{Patagão}
\begin{itemize}
\item {Grp. gram.:adj.}
\end{itemize}
\begin{itemize}
\item {Grp. gram.:M.}
\end{itemize}
Relativo á Patagónia.
Habitante da Patagónia.
\section{Patagarro}
\begin{itemize}
\item {Grp. gram.:m.}
\end{itemize}
\begin{itemize}
\item {Utilização:Mad}
\end{itemize}
Espécie de ave, (\textunderscore puffinus anglorum\textunderscore , Temm.).
\section{Patágio}
\begin{itemize}
\item {Grp. gram.:m.}
\end{itemize}
\begin{itemize}
\item {Utilização:Ant.}
\end{itemize}
\begin{itemize}
\item {Proveniência:(Lat. \textunderscore patagium\textunderscore )}
\end{itemize}
Membrana, que serve de asas ao morcego.
Franja larga, com que as damas romanas guarneciam os vestidos.
\section{Patagónio}
\begin{itemize}
\item {Grp. gram.:m.  e  adj.}
\end{itemize}
O mesmo que \textunderscore patagão\textunderscore .
\section{Pataia}
\begin{itemize}
\item {Grp. gram.:f.}
\end{itemize}
\begin{itemize}
\item {Utilização:Ant.}
\end{itemize}
Celleiro ou tulha, na Índia portuguesa.
\section{Pataica}
\begin{itemize}
\item {Grp. gram.:f.}
\end{itemize}
\begin{itemize}
\item {Utilização:Ant.}
\end{itemize}
Moéda de Cambaia.
\section{Pataloco}
\begin{itemize}
\item {fónica:lô}
\end{itemize}
\begin{itemize}
\item {Grp. gram.:m.  e  adj.}
\end{itemize}
\begin{itemize}
\item {Utilização:Prov.}
\end{itemize}
O mesmo que \textunderscore pataloto\textunderscore .
\section{Pataloto}
\begin{itemize}
\item {fónica:lô}
\end{itemize}
\begin{itemize}
\item {Grp. gram.:m.  e  adj.}
\end{itemize}
\begin{itemize}
\item {Utilização:Prov.}
\end{itemize}
Porcalhão.
Idiota.
(Cp. \textunderscore patau\textunderscore )
\section{Patalou}
\begin{itemize}
\item {Grp. gram.:m.}
\end{itemize}
O mesmo que \textunderscore ranúnculo\textunderscore .
(Cp. \textunderscore pataluco\textunderscore )
\section{Patalou}
\begin{itemize}
\item {Grp. gram.:m.}
\end{itemize}
\begin{itemize}
\item {Utilização:Ant.}
\end{itemize}
O mesmo que \textunderscore pataloto\textunderscore .
\section{Patalou-dos-valles}
\begin{itemize}
\item {Grp. gram.:m.}
\end{itemize}
O mesmo que \textunderscore patalou\textunderscore ^1. Cf. B. Pereira, \textunderscore Prosódia\textunderscore , vb. \textunderscore sardonia\textunderscore .
\section{Pataluco}
\begin{itemize}
\item {Grp. gram.:m.}
\end{itemize}
Planta ranunculácea.
\section{Patamal}
\begin{itemize}
\item {Grp. gram.:m.}
\end{itemize}
\begin{itemize}
\item {Utilização:Prov.}
\end{itemize}
O mesmo que \textunderscore patamar\textunderscore .
\section{Patamar}
\begin{itemize}
\item {Grp. gram.:m.}
\end{itemize}
\begin{itemize}
\item {Utilização:Ant.}
\end{itemize}
\begin{itemize}
\item {Utilização:Ant.}
\end{itemize}
Espaço, mais ou menos largo no topo de uma escada ou de cada lanço de escadas.
Embarcação costeira da Índia. Cf. \textunderscore Corresp. de D. João de Castro\textunderscore , nos ms. da casa San-Lourenço, Tôrre do Tombo.
Andarilho, postilhão. Cf. Barros, \textunderscore Déc.\textunderscore  II, l. I, c. 5.
\section{Patamarim}
\begin{itemize}
\item {Grp. gram.:m.}
\end{itemize}
Espécie de embarcação indiana, costeira, de dois mastros.
(Cp. \textunderscore patamar\textunderscore )
\section{Patamaz}
\begin{itemize}
\item {Grp. gram.:m.  e  adj.}
\end{itemize}
\begin{itemize}
\item {Utilização:Chul.}
\end{itemize}
Santarrão.
Parvoeirão.
\section{Patameco}
\begin{itemize}
\item {Grp. gram.:m. Pl.}
\end{itemize}
Partes pudendas da mulhér. (Colhido na Bairrada)
\section{Patana}
\begin{itemize}
\item {Grp. gram.:f.}
\end{itemize}
Nome, que na Índia se dava a cada um dos grupos das ilhas de Maldiva, subordinado a uma ilha principal. Cf. Barros, \textunderscore Déc.\textunderscore  III, l. III, c. 7.
\section{Patanes}
\begin{itemize}
\item {Grp. gram.:m. pl.}
\end{itemize}
Antigos povos, vizinhos dos Mogores. Cf. Barros, \textunderscore Déc.\textunderscore  IV, l. IV, c. I.
\section{Pataqueira}
\begin{itemize}
\item {Grp. gram.:f.}
\end{itemize}
\begin{itemize}
\item {Utilização:Fam.}
\end{itemize}
\begin{itemize}
\item {Utilização:Bras}
\end{itemize}
\begin{itemize}
\item {Proveniência:(De \textunderscore pataco\textunderscore )}
\end{itemize}
Jôgo muito barato.
Jôgo de asar, em que os pontos são gente ordinária.
Planta aromática, medicinal, contra o \textunderscore beriberi\textunderscore .
\section{Pataqueiro}
\begin{itemize}
\item {Grp. gram.:adj.}
\end{itemize}
\begin{itemize}
\item {Utilização:Fam.}
\end{itemize}
\begin{itemize}
\item {Utilização:Fig.}
\end{itemize}
\begin{itemize}
\item {Proveniência:(De \textunderscore pataco\textunderscore )}
\end{itemize}
Que se vende a pataco.
Diz-se do jôgo, em que os jogadores arriscam só cobre ou pouco dinheiro.
Muito barato; ordinário.
\section{Pataranha}
\begin{itemize}
\item {Grp. gram.:f.}
\end{itemize}
(Corr. de \textunderscore patranha\textunderscore )
\section{Pataranha}
\begin{itemize}
\item {Grp. gram.:m.  e  f.}
\end{itemize}
\begin{itemize}
\item {Utilização:Prov.}
\end{itemize}
\begin{itemize}
\item {Utilização:trasm.}
\end{itemize}
Pessôa, que vê pouco.
\section{Patarata}
\begin{itemize}
\item {Grp. gram.:f.}
\end{itemize}
\begin{itemize}
\item {Grp. gram.:M. ,  f.  e  adj.}
\end{itemize}
Ostentação ridícula.
Mentirola.
Pessôa, que diz pataratas ou mentiras.
Embófia.
Pessôa, que alardeia qualidades que não tem.
(Cast. \textunderscore patarata\textunderscore )
\section{Pataratar}
\begin{itemize}
\item {Grp. gram.:v. i.}
\end{itemize}
O mesmo que \textunderscore pataratear\textunderscore .
\section{Pataratear}
\begin{itemize}
\item {Grp. gram.:v. i.}
\end{itemize}
Dizer pataratas ou patranhas.
Sêr vaidoso ou patarata.
\section{Patarateiro}
\begin{itemize}
\item {Grp. gram.:m.  e  adj.}
\end{itemize}
Aquelle que diz pataratas ou patranhas.
\section{Pataratice}
\begin{itemize}
\item {Grp. gram.:f.}
\end{itemize}
\begin{itemize}
\item {Proveniência:(De \textunderscore patarata\textunderscore )}
\end{itemize}
Acto ou dito de patarateiro.
\section{Pataratismo}
\begin{itemize}
\item {Grp. gram.:m.}
\end{itemize}
O mesmo que \textunderscore pataratice\textunderscore .
Hábitos de patarata.
Os pataratas. Cf. Camillo, \textunderscore Narcót.\textunderscore , II, 326.
\section{Patareco}
\begin{itemize}
\item {Grp. gram.:m.}
\end{itemize}
\begin{itemize}
\item {Utilização:T. de Alcobaça}
\end{itemize}
\begin{itemize}
\item {Utilização:Prov.}
\end{itemize}
\begin{itemize}
\item {Utilização:minh.}
\end{itemize}
Feijão verde.
Pão pequeno ou bôla, que se faz para os rapazes, quando se coze o pão.
\section{Pataréo}
\begin{itemize}
\item {Grp. gram.:m.}
\end{itemize}
\begin{itemize}
\item {Utilização:P. us.}
\end{itemize}
O mesmo que \textunderscore patamar\textunderscore .
\section{Pataréu}
\begin{itemize}
\item {Grp. gram.:m.}
\end{itemize}
\begin{itemize}
\item {Utilização:P. us.}
\end{itemize}
O mesmo que \textunderscore patamar\textunderscore .
\section{Pataroco}
\begin{itemize}
\item {fónica:tarô}
\end{itemize}
\begin{itemize}
\item {Grp. gram.:adj.}
\end{itemize}
\begin{itemize}
\item {Utilização:Prov.}
\end{itemize}
\begin{itemize}
\item {Utilização:alg.}
\end{itemize}
Idiota; parvo.
(Cp. \textunderscore patalou\textunderscore ^1)
\section{Patarral}
\begin{itemize}
\item {Grp. gram.:m.}
\end{itemize}
O mesmo que \textunderscore patarrás\textunderscore .
\section{Patarrás}
\begin{itemize}
\item {Grp. gram.:m.}
\end{itemize}
\begin{itemize}
\item {Utilização:Náut.}
\end{itemize}
\begin{itemize}
\item {Proveniência:(It. \textunderscore patarasso\textunderscore )}
\end{itemize}
Calabre, que amarra os mastros ao costado da embarcação.
\section{Patarreca}
\begin{itemize}
\item {Grp. gram.:m.  e  f.}
\end{itemize}
\begin{itemize}
\item {Utilização:Prov.}
\end{itemize}
\begin{itemize}
\item {Utilização:beir.}
\end{itemize}
Pessôa muito baixa, atarracada.
\section{Patarrego}
\begin{itemize}
\item {fónica:rê}
\end{itemize}
\begin{itemize}
\item {Grp. gram.:m.}
\end{itemize}
\begin{itemize}
\item {Utilização:Prov.}
\end{itemize}
Pequena propriedade rústica.
Quinchoso.
\section{Pàtarroxa}
\begin{itemize}
\item {fónica:rô}
\end{itemize}
\begin{itemize}
\item {Grp. gram.:f.}
\end{itemize}
Peixe plagióstomo, de côr cinzenta avermelhada na parte superior, e de dentes ponteagudos.
\section{Patassol}
\begin{itemize}
\item {Grp. gram.:m.}
\end{itemize}
\begin{itemize}
\item {Utilização:Prov.}
\end{itemize}
\begin{itemize}
\item {Utilização:trasm.}
\end{itemize}
Pequeno insecto vermelho, com pintas pretas e redondas.
\section{Patassola}
\begin{itemize}
\item {Grp. gram.:f.}
\end{itemize}
\begin{itemize}
\item {Utilização:Prov.}
\end{itemize}
\begin{itemize}
\item {Utilização:trasm.}
\end{itemize}
Fêmea do \textunderscore patassol\textunderscore .
\section{Patata}
\begin{itemize}
\item {Grp. gram.:f.}
\end{itemize}
\begin{itemize}
\item {Utilização:T. de Moncorvo}
\end{itemize}
O mesmo que \textunderscore batata\textunderscore .
\section{Patativa}
\begin{itemize}
\item {Grp. gram.:f.}
\end{itemize}
\begin{itemize}
\item {Utilização:Bras}
\end{itemize}
Ave cinzenta e canora.
\section{Pata-toada}
\begin{itemize}
\item {Grp. gram.:f.}
\end{itemize}
\begin{itemize}
\item {Utilização:Burl.}
\end{itemize}
O mesmo que \textunderscore palmatoada\textunderscore , quando applicada a um indivíduo asnático:«\textunderscore ...obriga-o a declinar os pronomes, a compasso de pata-toadas.\textunderscore »Camillo, \textunderscore Cancion. Alegre\textunderscore .
\section{Patatrás!}
\begin{itemize}
\item {Grp. gram.:interj.}
\end{itemize}
\begin{itemize}
\item {Proveniência:(T. onom.)}
\end{itemize}
O mesmo que \textunderscore zás!\textunderscore  Cf. Eça, \textunderscore P. Basilio\textunderscore , 347.
\section{Patau}
\begin{itemize}
\item {Grp. gram.:m.}
\end{itemize}
\begin{itemize}
\item {Utilização:Pop.}
\end{itemize}
\begin{itemize}
\item {Proveniência:(De \textunderscore pato\textunderscore )}
\end{itemize}
Homem parvo, simplório.
\section{Patauá}
\begin{itemize}
\item {fónica:ta-u}
\end{itemize}
\begin{itemize}
\item {Grp. gram.:m.}
\end{itemize}
Planta oleagínea do Brasil.
\section{Patavar}
\begin{itemize}
\item {Grp. gram.:adj.}
\end{itemize}
\begin{itemize}
\item {Utilização:Des.}
\end{itemize}
Relativo á Holanda, ou procedente da Holanda?:«\textunderscore finos, como lenços patavares.\textunderscore »Filinto, V, 182.
(Talvez por \textunderscore batavar\textunderscore , de \textunderscore Batávia\textunderscore  = Holanda)
\section{Patavina}
\begin{itemize}
\item {Grp. gram.:f.}
\end{itemize}
\begin{itemize}
\item {Utilização:Pop.}
\end{itemize}
\begin{itemize}
\item {Grp. gram.:M.}
\end{itemize}
\begin{itemize}
\item {Utilização:Prov.}
\end{itemize}
\begin{itemize}
\item {Utilização:beir.}
\end{itemize}
Coisa nenhuma; nada: \textunderscore é idiota; não percebe patavina\textunderscore .
Pateta, asno, idiota.
(Relaciona-se com \textunderscore patavino\textunderscore ?)
\section{Patavinice}
\begin{itemize}
\item {Grp. gram.:f.}
\end{itemize}
Qualidade ou acto de patavina ou de pateta. Cf. Camillo, \textunderscore Sc. da Foz\textunderscore , 165.
\section{Patavinidade}
\begin{itemize}
\item {Grp. gram.:f.}
\end{itemize}
\begin{itemize}
\item {Proveniência:(Do lat. \textunderscore patavinitas\textunderscore )}
\end{itemize}
Latinidade provinciana, que era propria dos habitantes de Pádua, e de que há vestígios nas obras de Tito Lívio, natural daquella cidade. Cf. Castilho, \textunderscore Fastos\textunderscore , I, 203; Júl. Castilho, \textunderscore Lisb. Ant.\textunderscore 
\section{Patavinismo}
\begin{itemize}
\item {Grp. gram.:m.}
\end{itemize}
O mesmo que \textunderscore patavinidade\textunderscore .
\section{Patavino}
\begin{itemize}
\item {Grp. gram.:adj.}
\end{itemize}
\begin{itemize}
\item {Grp. gram.:M.}
\end{itemize}
\begin{itemize}
\item {Proveniência:(Lat. \textunderscore patavinus\textunderscore )}
\end{itemize}
Relativo a Pádua.
Habitante de Pádua.
\section{Pataz}
\begin{itemize}
\item {Grp. gram.:m.}
\end{itemize}
Espécie de macaco africano (\textunderscore simia patas\textunderscore ).--Melhor orthogr. será \textunderscore patás\textunderscore .
\section{Pate}
\begin{itemize}
\item {Grp. gram.:m.}
\end{itemize}
Chefe de povoação, na Índia.
\section{Pateada}
\begin{itemize}
\item {Grp. gram.:f.}
\end{itemize}
Acto de patear^1.
\section{Pateadura}
\begin{itemize}
\item {Grp. gram.:f.}
\end{itemize}
(V.pateada)
\section{Pateante}
\begin{itemize}
\item {Grp. gram.:adj.}
\end{itemize}
Que pateia.
\section{Patear}
\begin{itemize}
\item {Grp. gram.:v. t.}
\end{itemize}
\begin{itemize}
\item {Utilização:Prov.}
\end{itemize}
\begin{itemize}
\item {Utilização:alg.}
\end{itemize}
\begin{itemize}
\item {Grp. gram.:V. i.}
\end{itemize}
\begin{itemize}
\item {Proveniência:(De \textunderscore pata\textunderscore ^2)}
\end{itemize}
Censurar ou reprovar, batendo com os pés no chão: \textunderscore patear um drama\textunderscore .
Pisar, calcar.
Bater com as patas.
Bater com os pés no chão, em sinal de desagrado ou reprovação.
Bater com os pés, dançando ou saltando. Cf. Filinto, \textunderscore D. Man.\textunderscore , I, 183.
\section{Patear}
\begin{itemize}
\item {Grp. gram.:v. i.}
\end{itemize}
Succumbir, morrer.
Dar-se por vencido.
Sêr mal succedido.
(Por \textunderscore pactear\textunderscore ? Ou de \textunderscore pata\textunderscore , por analogia com a loc. \textunderscore espichar a canela\textunderscore ?)
\section{Pateca}
\begin{itemize}
\item {Grp. gram.:f.}
\end{itemize}
\begin{itemize}
\item {Utilização:Des.}
\end{itemize}
\begin{itemize}
\item {Proveniência:(Do ár. \textunderscore batique\textunderscore )}
\end{itemize}
O mesmo que \textunderscore melancia\textunderscore  ou \textunderscore melancieira\textunderscore . Cf. Garcia Horta, \textunderscore Colloq.\textunderscore  XXVI e LXIII.
\section{Pategar}
\begin{itemize}
\item {Grp. gram.:v. i.}
\end{itemize}
\begin{itemize}
\item {Utilização:Prov.}
\end{itemize}
Sêr patego; proceder como patego.
\section{Patego}
\begin{itemize}
\item {fónica:tê}
\end{itemize}
\begin{itemize}
\item {Grp. gram.:m.  e  adj.}
\end{itemize}
\begin{itemize}
\item {Utilização:Pop.}
\end{itemize}
\begin{itemize}
\item {Proveniência:(De \textunderscore pato\textunderscore )}
\end{itemize}
Lorpa, pateta, simplório.
\section{Pateguice}
\begin{itemize}
\item {Grp. gram.:f.}
\end{itemize}
\begin{itemize}
\item {Utilização:Pop.}
\end{itemize}
Qualidade de quem é patego; acto ou dito de patego.
\section{Pàteira}
\begin{itemize}
\item {Grp. gram.:f.}
\end{itemize}
\begin{itemize}
\item {Utilização:Prov.}
\end{itemize}
\begin{itemize}
\item {Utilização:Prov.}
\end{itemize}
\begin{itemize}
\item {Utilização:alent.}
\end{itemize}
\begin{itemize}
\item {Utilização:T. de Aveiro}
\end{itemize}
\begin{itemize}
\item {Utilização:Ant.}
\end{itemize}
\begin{itemize}
\item {Proveniência:(De \textunderscore pato\textunderscore )}
\end{itemize}
Espingarda para a caça dos patos.
Mulhér, encarregada da cozinha de malteses.
Nome, que se dá a vários pontos da bacia hydrográphica do Vouga, permanentemente alagados de água, que fórma pequenas lagôas.
Charco; pântano.
O mesmo que \textunderscore bilharda\textunderscore .
\section{Pàteiro}
\begin{itemize}
\item {Grp. gram.:m.}
\end{itemize}
\begin{itemize}
\item {Utilização:Ant.}
\end{itemize}
\begin{itemize}
\item {Utilização:Ant.}
\end{itemize}
\begin{itemize}
\item {Utilização:Prov.}
\end{itemize}
\begin{itemize}
\item {Utilização:alent.}
\end{itemize}
\begin{itemize}
\item {Proveniência:(De \textunderscore pato\textunderscore )}
\end{itemize}
Guardador ou criador de patos.
Frade leigo, encarregado da copa de um convento.
Taberneiro, que vendia comestíveis.
O encarregado da cozinha de malteses.
\section{Pàteiro}
\begin{itemize}
\item {Grp. gram.:m.}
\end{itemize}
\begin{itemize}
\item {Utilização:Prov.}
\end{itemize}
\begin{itemize}
\item {Utilização:trasm.}
\end{itemize}
\begin{itemize}
\item {Proveniência:(De \textunderscore pata\textunderscore ^2)}
\end{itemize}
Vagaroso.
\section{Patejar}
\begin{itemize}
\item {Grp. gram.:v. i.}
\end{itemize}
\begin{itemize}
\item {Utilização:Ant.}
\end{itemize}
O mesmo que \textunderscore patinhar\textunderscore .
Escabujar ou debater-se num laço, sem poder livrar-se.
\section{Patela}
\begin{itemize}
\item {Grp. gram.:f.}
\end{itemize}
\begin{itemize}
\item {Proveniência:(Lat. \textunderscore patella\textunderscore )}
\end{itemize}
Gênero de moluscos gasterópodes.
Rótula do joelho.
Disco de ferro, marcado com pontos, e usado num jôgo popular para se atirar ao meco.
Nome dêsse jogo.
\section{Patelha}
\begin{itemize}
\item {fónica:tê}
\end{itemize}
\begin{itemize}
\item {Grp. gram.:f.}
\end{itemize}
Parte inferior do leme, e a parte da quilha sôbre que ella se move.
\section{Patella}
\begin{itemize}
\item {Grp. gram.:f.}
\end{itemize}
\begin{itemize}
\item {Proveniência:(Lat. \textunderscore patella\textunderscore )}
\end{itemize}
Gênero de molluscos gasterópodes.
Rótula do joelho.
Disco de ferro, marcado com pontos, e usado num jôgo popular para se atirar ao meco.
Nome dêsse jogo.
\section{Patelo}
\begin{itemize}
\item {fónica:tê}
\end{itemize}
\begin{itemize}
\item {Grp. gram.:m.}
\end{itemize}
\begin{itemize}
\item {Utilização:Prov.}
\end{itemize}
Caranguejo, que se emprega no adubo das terras.
Pequena arraia.
\section{Pátem}
\begin{itemize}
\item {Grp. gram.:m.}
\end{itemize}
\begin{itemize}
\item {Utilização:T. da Índia Port}
\end{itemize}
\begin{itemize}
\item {Proveniência:(Do conc. \textunderscore patẽ\textunderscore )}
\end{itemize}
O mesmo que \textunderscore piteira\textunderscore ^1.
\section{Patemar}
\begin{itemize}
\item {Grp. gram.:m.}
\end{itemize}
Espécie de navio indiano.
(Cp. \textunderscore patamar\textunderscore )
\section{Patena}
\begin{itemize}
\item {Grp. gram.:m.}
\end{itemize}
\begin{itemize}
\item {Utilização:Prov.}
\end{itemize}
\begin{itemize}
\item {Utilização:trasm.}
\end{itemize}
\begin{itemize}
\item {Proveniência:(Lat. \textunderscore patena\textunderscore )}
\end{itemize}
Lâmina de metal um pouco convexa, sôbre que se colloca a hóstia á Missa.
Cada uma das pennas do rodízio da azenha.
\section{Pátena}
\begin{itemize}
\item {Grp. gram.:m.}
\end{itemize}
\begin{itemize}
\item {Utilização:Prov.}
\end{itemize}
\begin{itemize}
\item {Utilização:trasm.}
\end{itemize}
\begin{itemize}
\item {Proveniência:(Lat. \textunderscore patena\textunderscore )}
\end{itemize}
Lâmina de metal um pouco convexa, sôbre que se colloca a hóstia á Missa.
Cada uma das pennas do rodízio da azenha.
\section{Patença}
\begin{itemize}
\item {Grp. gram.:f.}
\end{itemize}
Peixe, espécie de solho.--Melhor orthogr. seria \textunderscore patensa\textunderscore , se a origem fôsse o lat. \textunderscore platessa\textunderscore .
\section{Patente}
\begin{itemize}
\item {Grp. gram.:adj.}
\end{itemize}
\begin{itemize}
\item {Grp. gram.:F.}
\end{itemize}
\begin{itemize}
\item {Utilização:Carp.}
\end{itemize}
\begin{itemize}
\item {Proveniência:(Lat. \textunderscore patens\textunderscore )}
\end{itemize}
Aberto, accessível: \textunderscore entrada patente, numa casa\textunderscore .
Evidente; claro: \textunderscore verdade patente\textunderscore .
\textunderscore Pano patente\textunderscore , espécie de tecido de algodão.
Título official de uma concessão ou privilégio.
Diploma de um membro de confraria.
Espécie de contribuição, que os que entram numa sociedade pagam em benefício dos sócios mais antigos.
Mola inglesa, que se emprega nos guardaventos.
\section{Patentear}
\begin{itemize}
\item {Grp. gram.:v. t.}
\end{itemize}
\begin{itemize}
\item {Utilização:Neol.}
\end{itemize}
Tornar patente; franquear.
Evidenciar.
Conceder patente de invenção.
\section{Patentemente}
\begin{itemize}
\item {Grp. gram.:adv.}
\end{itemize}
De modo patente; claramente; evidentemente; á vista de todos.
\section{Patentizar}
\begin{itemize}
\item {Grp. gram.:v. t.}
\end{itemize}
\begin{itemize}
\item {Utilização:inútil}
\end{itemize}
\begin{itemize}
\item {Utilização:Neol.}
\end{itemize}
O mesmo que \textunderscore patentear\textunderscore . Cf. Alv. Mendes, \textunderscore Discursos\textunderscore , 60.
\section{Páteo}
\begin{itemize}
\item {Grp. gram.:m.}
\end{itemize}
\begin{itemize}
\item {Utilização:Ant.}
\end{itemize}
Terreno murado, annexo a um edifício.
Recinto descoberto, no interior de um edifício ou rodeado por edifícios.
Vestíbulo.
Átrio.
Grande saguão.
Edifício ou aulas, em que se professavam humanidades.
(Cp. cast. \textunderscore pátio\textunderscore )
\section{Pátera}
\begin{itemize}
\item {Grp. gram.:f.}
\end{itemize}
\begin{itemize}
\item {Proveniência:(Lat. \textunderscore patera\textunderscore )}
\end{itemize}
Espécie de taça, usada nos sacrifícios antigos.
\section{Patéra}
\begin{itemize}
\item {Grp. gram.:f.}
\end{itemize}
\begin{itemize}
\item {Proveniência:(Do lat. \textunderscore patera\textunderscore )}
\end{itemize}
Espécie de escápula, mais ou menos ornamental, donde pendem as braçadeiras das cortinas.--A pronúncia exacta seria \textunderscore pátera\textunderscore , mas não se usa.
\section{Paternal}
\begin{itemize}
\item {Grp. gram.:adj.}
\end{itemize}
O mesmo que \textunderscore paterno\textunderscore .
\section{Paternalmente}
\begin{itemize}
\item {Grp. gram.:adv.}
\end{itemize}
De modo paternal.
\section{Paternianos}
\begin{itemize}
\item {Grp. gram.:m. pl.}
\end{itemize}
Herejes devassos, que sustentavam que a carne humana é obra do demónio.
\section{Paternidade}
\begin{itemize}
\item {Grp. gram.:f.}
\end{itemize}
\begin{itemize}
\item {Proveniência:(Lat. \textunderscore paternitas\textunderscore )}
\end{itemize}
Qualidade de quem é pai.
Título, que se dava aos religiosos.
\section{Paterno}
\begin{itemize}
\item {Grp. gram.:adj.}
\end{itemize}
\begin{itemize}
\item {Proveniência:(Lat. \textunderscore paternus\textunderscore )}
\end{itemize}
Relativo a pai.
Próprio de pai.
Procedente do pai.
Relativo á casa, em que nascemos.
Relativo á pátria.
\section{Pater-noster}
\begin{itemize}
\item {fónica:páter-nóstèr}
\end{itemize}
\begin{itemize}
\item {Grp. gram.:m.}
\end{itemize}
\begin{itemize}
\item {Utilização:Bras}
\end{itemize}
Oração dominical, o mesmo que \textunderscore Padre-nosso\textunderscore .
Série de anzoes, collocados em diversas alturas com iscas.
(Loc. lat.)
\section{Patersónia}
\begin{itemize}
\item {Grp. gram.:f.}
\end{itemize}
\begin{itemize}
\item {Proveniência:(De \textunderscore Paterson\textunderscore , n. p.)}
\end{itemize}
Gênero de arbustos australianos.
\section{Patesca}
\begin{itemize}
\item {fónica:tês}
\end{itemize}
\begin{itemize}
\item {Grp. gram.:f.}
\end{itemize}
\begin{itemize}
\item {Utilização:Náut.}
\end{itemize}
\begin{itemize}
\item {Grp. gram.:Adj.}
\end{itemize}
Peça de poleame, com uma só roldana, differente dos moitões, cadernaes, etc., e em que o cabo, que se pretende alar por elle, não é enfiado, mas encapellado por uma abertura lateral da respectiva caixa.
Diz-se da roda que é inteiriça.
\section{Pàteta}
\begin{itemize}
\item {Grp. gram.:m.  e  f.}
\end{itemize}
\begin{itemize}
\item {Proveniência:(De \textunderscore pato\textunderscore ? ou do fr. \textunderscore pas-de-tête\textunderscore ?)}
\end{itemize}
Pessôa maluca, tola, idiota.
\section{Pàtetar}
\begin{itemize}
\item {Grp. gram.:v. i.}
\end{itemize}
\begin{itemize}
\item {Proveniência:(De \textunderscore pateta\textunderscore )}
\end{itemize}
Fazer ou dizer patetices.
\section{Pàtetear}
\begin{itemize}
\item {Grp. gram.:v. i.}
\end{itemize}
\begin{itemize}
\item {Utilização:Bras}
\end{itemize}
O mesmo que \textunderscore pàtetar\textunderscore .
O mesmo que \textunderscore titubear\textunderscore .
\section{Pateticamente}
\begin{itemize}
\item {Grp. gram.:adj.}
\end{itemize}
De modo patético.
Com entusiasmo, com enternecimento.
\section{Pàtetice}
\begin{itemize}
\item {Grp. gram.:f.}
\end{itemize}
Acto ou dito de pateta.
\section{Patético}
\begin{itemize}
\item {Grp. gram.:adj.}
\end{itemize}
\begin{itemize}
\item {Grp. gram.:M.}
\end{itemize}
\begin{itemize}
\item {Proveniência:(Lat. \textunderscore patheticus\textunderscore )}
\end{itemize}
Que comove a alma; que enternece; tocante: \textunderscore scenas patéticas\textunderscore .
\textunderscore Músculo patético\textunderscore , o músculo oblíquo do ôlho.
Aquilo que comove; comoção.
Arte de despertar nos outros os sentimentos ou afectos de que estamos possuidos.
\section{Patetismo}
\begin{itemize}
\item {Grp. gram.:m.}
\end{itemize}
\begin{itemize}
\item {Utilização:P. us.}
\end{itemize}
Arte de mover os afectos.
(Cp. \textunderscore patético\textunderscore )
\section{Patheticamente}
\begin{itemize}
\item {Grp. gram.:adj.}
\end{itemize}
De modo pathético.
Com enthusiasmo, com enternecimento.
\section{Pathético}
\begin{itemize}
\item {Grp. gram.:adj.}
\end{itemize}
\begin{itemize}
\item {Grp. gram.:M.}
\end{itemize}
\begin{itemize}
\item {Proveniência:(Lat. \textunderscore patheticus\textunderscore )}
\end{itemize}
Que commove a alma; que enternece; tocante: \textunderscore scenas pathéticas\textunderscore .
\textunderscore Músculo pathético\textunderscore , o músculo oblíquo do ôlho.
Aquillo que commove; commoção.
Arte de despertar nos outros os sentimentos ou affectos de que estamos possuidos.
\section{Pathetismo}
\begin{itemize}
\item {Grp. gram.:m.}
\end{itemize}
\begin{itemize}
\item {Utilização:P. us.}
\end{itemize}
Arte de mover os affectos.
(Cp. \textunderscore pathético\textunderscore )
\section{Páthico}
\begin{itemize}
\item {Grp. gram.:adj.}
\end{itemize}
\begin{itemize}
\item {Utilização:Poét.}
\end{itemize}
\begin{itemize}
\item {Proveniência:(Lat. \textunderscore pathicus\textunderscore )}
\end{itemize}
Que se presta á devassidão.
Libidinoso; libertino. Cf. Costa Lobo, \textunderscore Sat. de Juv.\textunderscore , I, 127.
\section{Pathoderme}
\begin{itemize}
\item {Grp. gram.:m.}
\end{itemize}
\begin{itemize}
\item {Proveniência:(Do gr. \textunderscore pathos\textunderscore  + \textunderscore derma\textunderscore )}
\end{itemize}
Gênero de insectos coleópteros tetrâmeros.
\section{Pathogênese}
\begin{itemize}
\item {Grp. gram.:f.}
\end{itemize}
O mesmo que \textunderscore pathogenia\textunderscore .
\section{Pathogenesia}
\begin{itemize}
\item {Grp. gram.:f.}
\end{itemize}
O mesmo que \textunderscore pathogenia\textunderscore .
\section{Pathogenético}
\begin{itemize}
\item {Grp. gram.:adj.}
\end{itemize}
Relativo á pathogenesia.
\section{Pathogenia}
\begin{itemize}
\item {Grp. gram.:f.}
\end{itemize}
\begin{itemize}
\item {Proveniência:(Do gr. \textunderscore pathos\textunderscore  + \textunderscore genes\textunderscore )}
\end{itemize}
Parte da Pathologia, que trata da maneira como as doenças principiam ou se desenvolvem.
\section{Pathogênico}
\begin{itemize}
\item {Grp. gram.:adj.}
\end{itemize}
Relativo á pathogenia.
\section{Pathognomoníaco}
\begin{itemize}
\item {Grp. gram.:adj.}
\end{itemize}
O mesmo que \textunderscore pathognomónico\textunderscore . Cf. Pacheco, \textunderscore Promptuário\textunderscore .
\section{Pathognomónica}
\begin{itemize}
\item {Grp. gram.:f.}
\end{itemize}
\begin{itemize}
\item {Proveniência:(Do gr. \textunderscore pathos\textunderscore  + \textunderscore gnomon\textunderscore )}
\end{itemize}
Parte da Medicina, que trata dos symptomas das doenças.
Sciência dos indícios das paixões.
\section{Pathognomónico}
\begin{itemize}
\item {Grp. gram.:adj.}
\end{itemize}
Relativo aos sinaes próprios e constantes de cada doença.
(Cp. \textunderscore pathognomónica\textunderscore )
\section{Pathologia}
\begin{itemize}
\item {Grp. gram.:f.}
\end{itemize}
\begin{itemize}
\item {Proveniência:(Do gr. \textunderscore pathos\textunderscore  + \textunderscore logos\textunderscore )}
\end{itemize}
Sciência, que trata da origem, symptomas e natureza das doenças.
\section{Pathologicamente}
\begin{itemize}
\item {Grp. gram.:adv.}
\end{itemize}
De modo pathológico; segundo a Pathologia.
\section{Pathológico}
\begin{itemize}
\item {Grp. gram.:adj.}
\end{itemize}
Relativo á Pathologia.
\section{Pathologista}
\begin{itemize}
\item {Grp. gram.:m.  e  f.}
\end{itemize}
Pessôa, que se occupa da Pathologia.
\section{Pathophobia}
\begin{itemize}
\item {Grp. gram.:f.}
\end{itemize}
\begin{itemize}
\item {Proveniência:(Do gr. \textunderscore pathos\textunderscore  + \textunderscore phobein\textunderscore )}
\end{itemize}
Mêdo ou receio angustioso de qualquer doença.
\section{Páti}
\begin{itemize}
\item {Grp. gram.:m.}
\end{itemize}
\begin{itemize}
\item {Proveniência:(T. tupi)}
\end{itemize}
Planta oleagínea do Brasil, espécie de palmeira.
\section{Patibular}
\begin{itemize}
\item {Grp. gram.:adj.}
\end{itemize}
Relativo a patíbulo.
Que tem aspecto de criminoso.
Que traz á ideia o crime ou o remorso: \textunderscore rosto patibular\textunderscore .
\section{Patíbulo}
\begin{itemize}
\item {Grp. gram.:m.}
\end{itemize}
\begin{itemize}
\item {Proveniência:(Lat. \textunderscore patibulum\textunderscore )}
\end{itemize}
Estrado ou lugar, sôbre o qual se applica a pena de morte aos condemnados.
Fôrca.
Cadafalso; guilhotina.
\section{Pático}
\begin{itemize}
\item {Grp. gram.:adj.}
\end{itemize}
\begin{itemize}
\item {Utilização:Poét.}
\end{itemize}
\begin{itemize}
\item {Proveniência:(Lat. \textunderscore pathicus\textunderscore )}
\end{itemize}
Que se presta á devassidão.
Libidinoso; libertino. Cf. Costa Lobo, \textunderscore Sat. de Juv.\textunderscore , I, 127.
\section{Patifa}
\begin{itemize}
\item {Grp. gram.:f.  e  adj.}
\end{itemize}
Flexão fem. de \textunderscore patife\textunderscore ^1. Cf. Castilho, \textunderscore D. Quixote\textunderscore , I, 499.
\section{Patifão}
\begin{itemize}
\item {Grp. gram.:m.}
\end{itemize}
Grande patife.
\section{Patifaria}
\begin{itemize}
\item {Grp. gram.:f.}
\end{itemize}
Acto de patife^1; maroteira.
\section{Patifaria}
\begin{itemize}
\item {Grp. gram.:f.}
\end{itemize}
\begin{itemize}
\item {Utilização:Bras. de Minas}
\end{itemize}
\begin{itemize}
\item {Proveniência:(De \textunderscore patife\textunderscore ^2)}
\end{itemize}
Falta de coragem.
Cobardia.
\section{Patife}
\begin{itemize}
\item {Grp. gram.:m.  e  adj.}
\end{itemize}
Desavergonhado; maroto; biltre.
\section{Patife}
\begin{itemize}
\item {Grp. gram.:m. ,  f.  e  adj.}
\end{itemize}
\begin{itemize}
\item {Utilização:Bras. de San-Paulo}
\end{itemize}
Pessôa débil, fraca, tímida: \textunderscore quando chove, sou tão patife, que não saio de casa\textunderscore .
(Relaciona-se com \textunderscore patível\textunderscore ?)
\section{Patife}
\begin{itemize}
\item {Grp. gram.:f.}
\end{itemize}
\begin{itemize}
\item {Utilização:T. de Lamego}
\end{itemize}
Pequena caixa para simonte ou rapé, a qual tem na extremidade um bico com orifício, por onde sai o conteúdo.
\section{Patifório}
\begin{itemize}
\item {Grp. gram.:m.}
\end{itemize}
\begin{itemize}
\item {Utilização:Fam.}
\end{itemize}
Patife hábil e sonso; patifão.
\section{Patigabiraba}
\begin{itemize}
\item {Grp. gram.:f.}
\end{itemize}
Espécie de côco. Cf. Dom. Vieira, vb. \textunderscore côco\textunderscore .
\section{Patiguá}
\begin{itemize}
\item {Grp. gram.:m.}
\end{itemize}
\begin{itemize}
\item {Utilização:Bras}
\end{itemize}
Cesto, em que os gentios guardam as redes.
\section{Patilado}
\begin{itemize}
\item {Grp. gram.:m.}
\end{itemize}
O mesmo que \textunderscore patilau\textunderscore .
\section{Patilau}
\begin{itemize}
\item {Grp. gram.:m.}
\end{itemize}
\begin{itemize}
\item {Utilização:T. de Viana}
\end{itemize}
Amontoado de caranguejos, para adubo das terras.
\section{Patilha}
\begin{itemize}
\item {Grp. gram.:f.}
\end{itemize}
Fio achatado de prata ou oiro.
Parte posterior e um pouco elevada do sellim.
A parte inferior de um carril de via férrea.
Peça que, no velocípede, assenta sobre a roda e a impede de mover-se.
(Cp. fr. \textunderscore patte\textunderscore )
\section{Patilhão}
\begin{itemize}
\item {Grp. gram.:m.}
\end{itemize}
\begin{itemize}
\item {Utilização:Náut.}
\end{itemize}
\begin{itemize}
\item {Proveniência:(De \textunderscore patilha\textunderscore )}
\end{itemize}
Fórma muito saliente da roda de prôa.
\section{Patim}
\begin{itemize}
\item {Grp. gram.:m.}
\end{itemize}
Pequeno patamar.
(Dem. de \textunderscore pátio\textunderscore ?)
\section{Patim}
\begin{itemize}
\item {Grp. gram.:m.}
\end{itemize}
\begin{itemize}
\item {Proveniência:(Fr. \textunderscore patin\textunderscore )}
\end{itemize}
Calçado, para andar sôbre o gêlo, ou para patinar.
\section{Pátina}
\begin{itemize}
\item {Grp. gram.:f.}
\end{itemize}
\begin{itemize}
\item {Proveniência:(Lat. \textunderscore patina\textunderscore )}
\end{itemize}
Oxydação das tintas pela acção do tempo e sua gradual transformação pela acção da luz.
Carbonato, que se fórma na superfície das medalhas e estátuas de bronze, alterando-a.
Concreção terrosa, na superfície dos mármores antigos.--Os diccion. dizem \textunderscore patína\textunderscore , por infl. do fr. \textunderscore patine\textunderscore .
\section{Patinador}
\begin{itemize}
\item {Grp. gram.:m.  e  adj.}
\end{itemize}
Aquelle que patina.
\section{Patinagem}
\begin{itemize}
\item {Grp. gram.:f.}
\end{itemize}
Acto de patinar.
\section{Patinar}
\begin{itemize}
\item {Grp. gram.:v. i.}
\end{itemize}
Andar com patins, especialmente sôbre o gêlo.
\section{Patinha}
\begin{itemize}
\item {Grp. gram.:f.}
\end{itemize}
\begin{itemize}
\item {Utilização:Açor}
\end{itemize}
Erva, com que se pescam salemas, depois de preparada com mel.
\section{Patinhar}
\begin{itemize}
\item {Grp. gram.:v. i.}
\end{itemize}
\begin{itemize}
\item {Proveniência:(De \textunderscore pato\textunderscore )}
\end{itemize}
Agitar a água, como fazem os patos num tanque ou numa corrente.
Bater na água com as mãos ou com os pés.
Diz-se da máquina de combóio, quando as rodas giram, sem que a máquina ande.
\section{Patinhas}
\begin{itemize}
\item {Grp. gram.:f. pl.}
\end{itemize}
Espécie de jôgo popular.
\section{Patinheiro}
\begin{itemize}
\item {Grp. gram.:m.}
\end{itemize}
\begin{itemize}
\item {Utilização:Prov.}
\end{itemize}
\begin{itemize}
\item {Utilização:beir.}
\end{itemize}
\begin{itemize}
\item {Proveniência:(De \textunderscore patinhar\textunderscore )}
\end{itemize}
Caminho ou lugar lamacento, por onde se passa patinhando.
\section{Patinho}
\begin{itemize}
\item {Grp. gram.:m.}
\end{itemize}
\begin{itemize}
\item {Utilização:T. da Bairrada}
\end{itemize}
O mesmo que \textunderscore patau\textunderscore .
Jôgo popular.
\section{Pátio}
\begin{itemize}
\item {Grp. gram.:m.}
\end{itemize}
\begin{itemize}
\item {Utilização:Ant.}
\end{itemize}
O mesmo ou melhor que \textunderscore páteo\textunderscore .
Terreno murado, annexo a um edifício.
Recinto descoberto, no interior de um edifício ou rodeado por edifícios.
Vestíbulo.
Átrio.
Grande saguão.
Edifício ou aulas, em que se professavam humanidades.
(Cp. cast. \textunderscore pátio\textunderscore )
\section{Patira}
\begin{itemize}
\item {Grp. gram.:m.}
\end{itemize}
Espécie de porco de algumas florestas da América.
\section{Patível}
\begin{itemize}
\item {Grp. gram.:adj.}
\end{itemize}
\begin{itemize}
\item {Proveniência:(Do lat. \textunderscore patibilis\textunderscore )}
\end{itemize}
Que se póde soffrer; tolerável.
\section{Pato}
\begin{itemize}
\item {Grp. gram.:m.}
\end{itemize}
\begin{itemize}
\item {Utilização:Chul.}
\end{itemize}
\begin{itemize}
\item {Utilização:Bras. do N}
\end{itemize}
\begin{itemize}
\item {Proveniência:(Do ár. \textunderscore bat\textunderscore ?)}
\end{itemize}
Ave palmípede, da fam. dos lamellirostos.
Idiota; parvo.
Mau jogador.
\section{Pato}
\begin{itemize}
\item {Grp. gram.:adj.}
\end{itemize}
\begin{itemize}
\item {Utilização:Prov.}
\end{itemize}
O mesmo que [[empatado|empatar]], (falando-se do jôgo).
\section{Pató}
\begin{itemize}
\item {Grp. gram.:f.}
\end{itemize}
\begin{itemize}
\item {Utilização:Des.}
\end{itemize}
\begin{itemize}
\item {Proveniência:(T. as.)}
\end{itemize}
O mesmo que \textunderscore ponte\textunderscore .
\section{Patoá}
\begin{itemize}
\item {Grp. gram.:m.}
\end{itemize}
\begin{itemize}
\item {Proveniência:(Fr. \textunderscore patois\textunderscore )}
\end{itemize}
Cada um dos vários dialectos franceses, como o normando, o picardo, etc.
\section{Pato-arminho}
\begin{itemize}
\item {Grp. gram.:m.}
\end{itemize}
\begin{itemize}
\item {Utilização:Bras}
\end{itemize}
Gênero de marrecos.
\section{Patocho}
\begin{itemize}
\item {fónica:tô}
\end{itemize}
\begin{itemize}
\item {Grp. gram.:m.}
\end{itemize}
\begin{itemize}
\item {Utilização:Prov.}
\end{itemize}
\begin{itemize}
\item {Utilização:alg.}
\end{itemize}
O mesmo que \textunderscore patego\textunderscore .
\section{Pato-colhereiro}
\begin{itemize}
\item {Grp. gram.:m.}
\end{itemize}
Ave, o mesmo que \textunderscore colhereira\textunderscore . Cf. P. Moraes, \textunderscore Zool. Elem.\textunderscore , 402.
\section{Patoderme}
\begin{itemize}
\item {Grp. gram.:m.}
\end{itemize}
\begin{itemize}
\item {Proveniência:(Do gr. \textunderscore pathos\textunderscore  + \textunderscore derma\textunderscore )}
\end{itemize}
Gênero de insectos coleópteros tetrâmeros.
\section{Patofobia}
\begin{itemize}
\item {Grp. gram.:f.}
\end{itemize}
\begin{itemize}
\item {Proveniência:(Do gr. \textunderscore pathos\textunderscore  + \textunderscore phobein\textunderscore )}
\end{itemize}
Mêdo ou receio angustioso de qualquer doença.
\section{Patogênese}
\begin{itemize}
\item {Grp. gram.:f.}
\end{itemize}
O mesmo que \textunderscore patogenia\textunderscore .
\section{Patogenesia}
\begin{itemize}
\item {Grp. gram.:f.}
\end{itemize}
O mesmo que \textunderscore patogenia\textunderscore .
\section{Patogenético}
\begin{itemize}
\item {Grp. gram.:adj.}
\end{itemize}
Relativo á patogenesia.
\section{Patogenia}
\begin{itemize}
\item {Grp. gram.:f.}
\end{itemize}
\begin{itemize}
\item {Proveniência:(Do gr. \textunderscore pathos\textunderscore  + \textunderscore genes\textunderscore )}
\end{itemize}
Parte da Patologia, que trata da maneira como as doenças principiam ou se desenvolvem.
\section{Patogênico}
\begin{itemize}
\item {Grp. gram.:adj.}
\end{itemize}
Relativo á patogenia.
\section{Patognomoníaco}
\begin{itemize}
\item {Grp. gram.:adj.}
\end{itemize}
O mesmo que \textunderscore patognomónico\textunderscore . Cf. Pacheco, \textunderscore Promptuário\textunderscore .
\section{Patognomónica}
\begin{itemize}
\item {Grp. gram.:f.}
\end{itemize}
\begin{itemize}
\item {Proveniência:(Do gr. \textunderscore pathos\textunderscore  + \textunderscore gnomon\textunderscore )}
\end{itemize}
Parte da Medicina, que trata dos simptomas das doenças.
Ciência dos indícios das paixões.
\section{Patognomónico}
\begin{itemize}
\item {Grp. gram.:adj.}
\end{itemize}
Relativo aos sinaes próprios e constantes de cada doença.
(Cp. \textunderscore patognomónica\textunderscore )
\section{Patola}
\begin{itemize}
\item {Grp. gram.:f.}
\end{itemize}
\begin{itemize}
\item {Grp. gram.:M.  e  adj.}
\end{itemize}
\begin{itemize}
\item {Proveniência:(De \textunderscore pato\textunderscore )}
\end{itemize}
Diz-se \textunderscore ganso patola\textunderscore  uma ave palmípede, da fam. dos pelicanídeos.
Parvo, estúpido.
\section{Pátola}
\begin{itemize}
\item {Grp. gram.:f.}
\end{itemize}
Espécie de tecido.
Planta cucurbitácea da Índia portuguesa, (\textunderscore luffa acutangula\textunderscore , Roxb.).
(Do canará \textunderscore pattuda\textunderscore ?)
\section{Patologia}
\begin{itemize}
\item {Grp. gram.:f.}
\end{itemize}
\begin{itemize}
\item {Proveniência:(Do gr. \textunderscore pathos\textunderscore  + \textunderscore logos\textunderscore )}
\end{itemize}
Ciência, que trata da origem, simptomas e natureza das doenças.
\section{Patologicamente}
\begin{itemize}
\item {Grp. gram.:adv.}
\end{itemize}
De modo patológico; segundo a Patologia.
\section{Patológico}
\begin{itemize}
\item {Grp. gram.:adj.}
\end{itemize}
Relativo á Patologia.
\section{Patologista}
\begin{itemize}
\item {Grp. gram.:m.  e  f.}
\end{itemize}
Pessôa, que se ocupa da Patologia.
\section{Pato-marinho}
\begin{itemize}
\item {Grp. gram.:m.}
\end{itemize}
\begin{itemize}
\item {Utilização:Bras}
\end{itemize}
Ave aquática, de bico de peru, asas pequenas e sem pennas que nada quasi sempre debaixo da água. Cf. B. C. Rubim, \textunderscore Vocab. Bras.\textunderscore 
\section{Pato-mudo}
\begin{itemize}
\item {Grp. gram.:m.  e  adj.}
\end{itemize}
\begin{itemize}
\item {Utilização:Fam.}
\end{itemize}
Indivíduo que, nas assembleias deliberativas, especialmente no parlamento, não faz uso da palavra.
\section{Patonha}
\begin{itemize}
\item {Grp. gram.:f.}
\end{itemize}
\begin{itemize}
\item {Utilização:T. da Bairrada}
\end{itemize}
Grande pata.
Pé desmesurado.
\section{Patorá-das-praias}
\begin{itemize}
\item {Grp. gram.:m.}
\end{itemize}
\begin{itemize}
\item {Utilização:Bras}
\end{itemize}
Espécie de forragem gramínea.
\section{Pato-real}
\begin{itemize}
\item {Grp. gram.:m.}
\end{itemize}
Ave, o mesmo que \textunderscore adem\textunderscore  ou \textunderscore lavanco\textunderscore . Cf. P. Moraes, \textunderscore Zool. Elem.\textunderscore , 400.
\section{Patornear}
\begin{itemize}
\item {Grp. gram.:v. i.}
\end{itemize}
\begin{itemize}
\item {Utilização:Ant.}
\end{itemize}
Parolar; dar á língua. Cf. G. Vicente, I, 167.
\section{Patorra}
\begin{itemize}
\item {fónica:tô}
\end{itemize}
\begin{itemize}
\item {Grp. gram.:f.  e  adj.}
\end{itemize}
\begin{itemize}
\item {Grp. gram.:F.}
\end{itemize}
\begin{itemize}
\item {Utilização:Fam.}
\end{itemize}
Variedade de uva tinta.
Pata grande; pé enorme.
\section{Patos}
\begin{itemize}
\item {Grp. gram.:m. pl.}
\end{itemize}
Tríbo de índios carijós, que habitavam nas margens da lagôa dos Patos, no Brasil.
\section{Patota}
\begin{itemize}
\item {Grp. gram.:f.}
\end{itemize}
\begin{itemize}
\item {Utilização:Bras}
\end{itemize}
O mesmo que \textunderscore batota\textunderscore ^1.
\section{Patoteiro}
\begin{itemize}
\item {Grp. gram.:m.}
\end{itemize}
\begin{itemize}
\item {Utilização:Bras}
\end{itemize}
\begin{itemize}
\item {Proveniência:(De \textunderscore patota\textunderscore )}
\end{itemize}
O mesmo que \textunderscore batoteiro\textunderscore .
\section{Pato-trombeteiro}
\begin{itemize}
\item {Grp. gram.:m.}
\end{itemize}
Ave, o mesmo que \textunderscore colhereira\textunderscore . Cf. P. Moraes, \textunderscore Zool. Elem.\textunderscore , 402.
\section{Patrajona}
\begin{itemize}
\item {Grp. gram.:f.}
\end{itemize}
\begin{itemize}
\item {Utilização:Gír.}
\end{itemize}
Meretriz de soldados, que os acompanha de terra em terra.
(Relaciona-se com \textunderscore patrazana\textunderscore ?)
\section{Patranha}
\begin{itemize}
\item {Grp. gram.:f.}
\end{itemize}
Grande pêta.
História mentirosa; maranhão.
(Cp. cast. \textunderscore patraña\textunderscore )
\section{Patranhada}
\begin{itemize}
\item {Grp. gram.:f.}
\end{itemize}
\begin{itemize}
\item {Proveniência:(De \textunderscore patranha\textunderscore )}
\end{itemize}
Série de patranhas; narração mentirosa. Cf. Castilho, \textunderscore D. Quixote\textunderscore , I, 243.
\section{Patranheiro}
\begin{itemize}
\item {Grp. gram.:m.  e  adj.}
\end{itemize}
Aquelle que diz patranhas.
\section{Patranhento}
\begin{itemize}
\item {Grp. gram.:adj.}
\end{itemize}
O mesmo que \textunderscore patranheiro\textunderscore .
\section{Patranhoso}
\begin{itemize}
\item {Grp. gram.:adj.}
\end{itemize}
Relativo a patranha; em que há patranha: \textunderscore história patranhosa\textunderscore . Cf. Camillo, \textunderscore Corja\textunderscore , 97.
\section{Patrão}
\begin{itemize}
\item {Grp. gram.:m.}
\end{itemize}
\begin{itemize}
\item {Utilização:Ant.}
\end{itemize}
\begin{itemize}
\item {Proveniência:(Do lat. \textunderscore patronus\textunderscore )}
\end{itemize}
Chefe ou proprietário de um estabelecimento ou fábrica.
Aquelle que dirige uma pequena embarcação.
Dono da casa, em relação aos que o servem.
Qualquer indivíduo, em relação aos serviçaes que remunera.
Patrono.
O mesmo que \textunderscore orago\textunderscore .
\section{Patrão}
\begin{itemize}
\item {Grp. gram.:m.}
\end{itemize}
\begin{itemize}
\item {Utilização:Ant.}
\end{itemize}
O mesmo de \textunderscore padrão\textunderscore .
\section{Patrão-mór}
\begin{itemize}
\item {Grp. gram.:m.}
\end{itemize}
\begin{itemize}
\item {Utilização:Bras}
\end{itemize}
Funccionário, que dirige certos serviços ou officinas do Estado.
\section{Patrazana}
\begin{itemize}
\item {Grp. gram.:m.}
\end{itemize}
\begin{itemize}
\item {Utilização:Pop.}
\end{itemize}
\begin{itemize}
\item {Proveniência:(It. \textunderscore partigiano\textunderscore . Cp. \textunderscore partasana\textunderscore )}
\end{itemize}
Soldado da antiga guarda nacional.
Qualquer sujeito.
Homem gordo e bonacheirão.
\section{Pátria}
\begin{itemize}
\item {Grp. gram.:f.}
\end{itemize}
\begin{itemize}
\item {Proveniência:(Lat. \textunderscore patria\textunderscore )}
\end{itemize}
País em que nascemos.
Qualquer terra ou localidade, em que nascemos:«\textunderscore pátria minha Alanquer...\textunderscore », Camões.
Nacionalidade.
Berço.
\section{Patrial}
\begin{itemize}
\item {Grp. gram.:adj.}
\end{itemize}
\begin{itemize}
\item {Utilização:P. us.}
\end{itemize}
O mesmo que \textunderscore patriótico\textunderscore  ou \textunderscore pátrio\textunderscore : \textunderscore será nossa divisa patrial união\textunderscore .
Refrão de uma cantiga patriótica das cercanias do Pôrto.
\section{Patriarca}
\begin{itemize}
\item {Grp. gram.:m.}
\end{itemize}
\begin{itemize}
\item {Proveniência:(Gr. \textunderscore patriarkhes\textunderscore )}
\end{itemize}
Chefe de família, entre os antigos.
Prelado de algumas grandes dioceses.
Chefe da Igreja grega.
Nome dos primeiros fundadores de algumas Ordens religiosas; homem velho e respeitável.
\section{Patriarcádo}
\begin{itemize}
\item {Grp. gram.:m.}
\end{itemize}
Dignidade ou jurisdição de patriarca.
Diocese dirigida por um patriarca.
\section{Patriarcal}
\begin{itemize}
\item {Grp. gram.:adj.}
\end{itemize}
\begin{itemize}
\item {Grp. gram.:f.}
\end{itemize}
Relativo a patriarca ou a patriarcado; (ext.) respeitável, venerado; pacífico; bondôso.
Igreja, que tem cadeira patriarcal; sé patriarcal.
\section{Patriarcalmente}
\begin{itemize}
\item {Grp. gram.:adv.}
\end{itemize}
De modo patriarcal; á maneira dos patriarcas.
\section{Patriarcha}
\begin{itemize}
\item {fónica:ca}
\end{itemize}
\begin{itemize}
\item {Grp. gram.:m.}
\end{itemize}
\begin{itemize}
\item {Proveniência:(Gr. \textunderscore patriarkhes\textunderscore )}
\end{itemize}
Chefe de família, entre os antigos.
Prelado de algumas grandes dioceses.
Chefe da Igreja grega.
Nome dos primeiros fundadores de algumas Ordens religiosas; homem velho e respeitável.
\section{Patriarchado}
\begin{itemize}
\item {fónica:cá}
\end{itemize}
\begin{itemize}
\item {Grp. gram.:m.}
\end{itemize}
Dignidade ou jurisdição de patriarcha.
Diocese dirigida por um patriarcha.
\section{Patriarchal}
\begin{itemize}
\item {fónica:cal}
\end{itemize}
\begin{itemize}
\item {Grp. gram.:adj.}
\end{itemize}
\begin{itemize}
\item {Grp. gram.:f.}
\end{itemize}
Relativo a patriarcha ou a patriarchado; (ext.) respeitável, venerado; pacífico; bondôso.
Igreja, que tem cadeira patriarchal; sé patriarchal.
\section{Patriarchalmente}
\begin{itemize}
\item {fónica:cal}
\end{itemize}
\begin{itemize}
\item {Grp. gram.:adv.}
\end{itemize}
De modo patriarchal; á maneira dos patriarchas.
\section{Patriarchia}
\begin{itemize}
\item {fónica:qui}
\end{itemize}
\begin{itemize}
\item {Grp. gram.:f.}
\end{itemize}
O mesmo que \textunderscore patriarchado\textunderscore .
\section{Patriarquia}
\begin{itemize}
\item {Grp. gram.:f.}
\end{itemize}
O mesmo que \textunderscore patriarcádo\textunderscore .
\section{Patriciado}
\begin{itemize}
\item {Grp. gram.:m.}
\end{itemize}
\begin{itemize}
\item {Proveniência:(De \textunderscore patrício\textunderscore )}
\end{itemize}
Estado de patrício, entre os Romanos; classe dos nobres.
\section{Patriciano}
\begin{itemize}
\item {Grp. gram.:adj.}
\end{itemize}
\begin{itemize}
\item {Utilização:Gal}
\end{itemize}
\begin{itemize}
\item {Proveniência:(Fr. \textunderscore patricien\textunderscore )}
\end{itemize}
O mesmo que \textunderscore patrício\textunderscore . Cf. Garrett, \textunderscore Romanceiro\textunderscore , I, XXIV.
\section{Patriciato}
\begin{itemize}
\item {Grp. gram.:m.}
\end{itemize}
O mesmo que \textunderscore patriciado\textunderscore .
\section{Patrício}
\begin{itemize}
\item {Grp. gram.:adj.}
\end{itemize}
\begin{itemize}
\item {Grp. gram.:M.}
\end{itemize}
\begin{itemize}
\item {Proveniência:(Lat. \textunderscore patricius\textunderscore )}
\end{itemize}
Relativo á classe dos nobres, entre os Romanos.
Aristocrático.
Distinto; elegante.
Indivíduo da classe dos nobres, entre os Romanos.
Aristocrata, homem nobre.
Aquelle que, em relação a outros, nasceu na mesma localidade ou país.
\section{Patrimoniado}
\begin{itemize}
\item {Grp. gram.:adj.}
\end{itemize}
Que tem património.
Que recebeu património. Cf. F. Alex. Lobo, III, 415.
\section{Patrimonial}
\begin{itemize}
\item {Grp. gram.:adj.}
\end{itemize}
\begin{itemize}
\item {Proveniência:(Lat. \textunderscore patrimonialis\textunderscore )}
\end{itemize}
Relativo a património.
\section{Património}
\begin{itemize}
\item {Grp. gram.:m.}
\end{itemize}
\begin{itemize}
\item {Proveniência:(Lat. \textunderscore patrimonium\textunderscore )}
\end{itemize}
Herança paterna.
Bens de família.
Bens necessários para a ordenação de um ecclesiástico.
\section{Patrínia}
\begin{itemize}
\item {Grp. gram.:f.}
\end{itemize}
\begin{itemize}
\item {Proveniência:(De \textunderscore Patrin\textunderscore , n. p.)}
\end{itemize}
Gênero de plantas valerianáceas.
\section{Pátrio}
\begin{itemize}
\item {Grp. gram.:adj.}
\end{itemize}
\begin{itemize}
\item {Proveniência:(Lat. \textunderscore patrius\textunderscore )}
\end{itemize}
Relativo á pátria.
Relativo aos pais: \textunderscore o pátrio poder\textunderscore .
\section{Patriota}
\begin{itemize}
\item {Grp. gram.:m.  e  f.}
\end{itemize}
\begin{itemize}
\item {Proveniência:(Lat. \textunderscore patriota\textunderscore )}
\end{itemize}
Pessôa patrícia.
Pessôa, que ama a sua pátria e deseja servi-la.
\section{Patrioteiro}
\begin{itemize}
\item {Grp. gram.:m.  e  adj.}
\end{itemize}
\begin{itemize}
\item {Utilização:Deprec.}
\end{itemize}
Aquelle que alardeia patriotismo, ou que está sempre a inculcar-se patriota.
\section{Patrioticamente}
\begin{itemize}
\item {Grp. gram.:adv.}
\end{itemize}
De modo patriótico; com patriotismo; á maneira de patriota.
\section{Patriotice}
\begin{itemize}
\item {Grp. gram.:f.}
\end{itemize}
\begin{itemize}
\item {Utilização:Deprec.}
\end{itemize}
Qualidade de patriota.
\section{Patriótico}
\begin{itemize}
\item {Grp. gram.:adj.}
\end{itemize}
\begin{itemize}
\item {Proveniência:(Lat. \textunderscore patrioticus\textunderscore )}
\end{itemize}
Relativo a patriota; que revela amor á pátria: \textunderscore hymnos patrióticos\textunderscore .
\section{Patriotismo}
\begin{itemize}
\item {Grp. gram.:m.}
\end{itemize}
\begin{itemize}
\item {Utilização:Chul.}
\end{itemize}
Qualidade de quem é patriota.
Amor á pátria.
Mamas de mulhér. Cf. Camillo, \textunderscore Brasileira\textunderscore , 78.
\section{Patrística}
\begin{itemize}
\item {Grp. gram.:f.}
\end{itemize}
\begin{itemize}
\item {Proveniência:(Do lat. \textunderscore pater\textunderscore , \textunderscore patris\textunderscore )}
\end{itemize}
Sciência, que se occupa da doutrina dos Santos-Padres.
\section{Patrizar}
\begin{itemize}
\item {Grp. gram.:v. i.}
\end{itemize}
\begin{itemize}
\item {Utilização:Ant.}
\end{itemize}
Servir a pátria; sêr bom patriota. Cf. Barros, \textunderscore Asia\textunderscore , pról.
(Cp. \textunderscore patrício\textunderscore )
\section{Patrôa}
\begin{itemize}
\item {Grp. gram.:f.}
\end{itemize}
\begin{itemize}
\item {Utilização:Pop.}
\end{itemize}
\begin{itemize}
\item {Proveniência:(De \textunderscore patrão\textunderscore )}
\end{itemize}
Mulher de patrão.
Dona de casa.
Ama, em relação a criados.
Mulhér, que dirige certos serviços ou estabelecimento.
Espôsa: \textunderscore levei ontem a minha patrôa ao theatro\textunderscore .
\section{Patrocinador}
\begin{itemize}
\item {Grp. gram.:m.  e  adj.}
\end{itemize}
Aquelle que patrocina.
\section{Patrocinar}
\begin{itemize}
\item {Grp. gram.:v. t.}
\end{itemize}
\begin{itemize}
\item {Proveniência:(Lat. \textunderscore patrocinare\textunderscore )}
\end{itemize}
Dar patrocínio a.
Proteger.
\section{Patrocinato}
\begin{itemize}
\item {Grp. gram.:m.}
\end{itemize}
Acto de patrocinar.
Patrocínio; patronato. Cf. Eça, \textunderscore P. Basilio\textunderscore , 461.
\section{Patrocínio}
\begin{itemize}
\item {Grp. gram.:m.}
\end{itemize}
\begin{itemize}
\item {Proveniência:(Lat. \textunderscore patrocinium\textunderscore )}
\end{itemize}
Amparo, auxílio, protecção.
\section{Patrologia}
\begin{itemize}
\item {Grp. gram.:f.}
\end{itemize}
\begin{itemize}
\item {Proveniência:(Do lat. \textunderscore pater\textunderscore , \textunderscore patris\textunderscore  + gr. \textunderscore logos\textunderscore )}
\end{itemize}
Estudo da vida e obras dos Padres da Igreja.
\section{Patromoria}
\begin{itemize}
\item {Grp. gram.:f.}
\end{itemize}
\begin{itemize}
\item {Utilização:Bras}
\end{itemize}
\begin{itemize}
\item {Utilização:Neol.}
\end{itemize}
Cargo ou dignidade de patrão-mór.
\section{Patrona}
\begin{itemize}
\item {Grp. gram.:f.}
\end{itemize}
\begin{itemize}
\item {Utilização:Prov.}
\end{itemize}
\begin{itemize}
\item {Utilização:Bras. do N}
\end{itemize}
\begin{itemize}
\item {Utilização:Gír.}
\end{itemize}
\begin{itemize}
\item {Proveniência:(Lat. \textunderscore patrona\textunderscore )}
\end{itemize}
Protectora; padroeira.
Pequena mala, para cartuxos dos soldados de infantaria.
Grande algibeira solta, usada por mulheres.
O mesmo que \textunderscore patuá\textunderscore , saco de coiro.
Parrameiro.
\section{Patronado}
\begin{itemize}
\item {Grp. gram.:m.}
\end{itemize}
O mesmo que \textunderscore patronato\textunderscore .
\section{Patronagem}
\begin{itemize}
\item {Grp. gram.:f.}
\end{itemize}
\begin{itemize}
\item {Proveniência:(De \textunderscore patrono\textunderscore )}
\end{itemize}
O mesmo que \textunderscore patrocínio\textunderscore .
\section{Patronal}
\begin{itemize}
\item {Grp. gram.:adj.}
\end{itemize}
\begin{itemize}
\item {Utilização:Neol.}
\end{itemize}
\begin{itemize}
\item {Proveniência:(Lat. \textunderscore patronalis\textunderscore )}
\end{itemize}
Relativo a patrão; próprio de patrão. Cf. \textunderscore Diár.-de-Notícias\textunderscore , de 23-IX-98.
\section{Patronato}
\begin{itemize}
\item {Grp. gram.:m.}
\end{itemize}
\begin{itemize}
\item {Proveniência:(Lat. \textunderscore patronatus\textunderscore )}
\end{itemize}
Patrocínio; padroado.
\section{Patronear}
\begin{itemize}
\item {Grp. gram.:v. t.}
\end{itemize}
\begin{itemize}
\item {Grp. gram.:V. i.}
\end{itemize}
Servir de patrono a.
Patrocinar.
Dirigir como patrão.
Tomar ares de patrão.
Falar muito sôbre bagatelas.
\section{Patronímico}
\begin{itemize}
\item {Grp. gram.:adj.}
\end{itemize}
\begin{itemize}
\item {Proveniência:(Lat. \textunderscore patronymicus\textunderscore )}
\end{itemize}
Relativo a pai, especialmente a respeito de nomes de família.
Relativo ao nome paterno; que designa o nome de pai: \textunderscore Rodrigues é nome patronímico, isto é, mostrava que Rodrigues era filho de Rodrigo\textunderscore .
\section{Patronizar}
\begin{itemize}
\item {Grp. gram.:v. t.}
\end{itemize}
O mesmo que \textunderscore patronear\textunderscore .
\section{Patrono}
\begin{itemize}
\item {Grp. gram.:m.}
\end{itemize}
\begin{itemize}
\item {Proveniência:(Lat. patronus)}
\end{itemize}
Patrocinador.
Defensor.
Advogado, em relação aos seus constituintes.
Designação do senhor, em relação aos seus libertos, entre os Romanos.
\section{Patronýmico}
\begin{itemize}
\item {Grp. gram.:adj.}
\end{itemize}
\begin{itemize}
\item {Proveniência:(Lat. \textunderscore patronymicus\textunderscore )}
\end{itemize}
Relativo a pai, especialmente a respeito de nomes de família.
Relativo ao nome paterno; que designa o nome de pai: \textunderscore Rodrigues é nome patronýmico, isto é, mostrava que Rodrigues era filho de Rodrigo\textunderscore .
\section{Patruça}
\begin{itemize}
\item {Grp. gram.:f.}
\end{itemize}
Peixe, espécie de solho.--Alguns escrevem patrussa.
\section{Patrúcia}
\begin{itemize}
\item {Grp. gram.:f.}
\end{itemize}
\begin{itemize}
\item {Utilização:Pesc.}
\end{itemize}
O mesmo que \textunderscore patruça\textunderscore .
\section{Patruicida}
\begin{itemize}
\item {fónica:tru-i}
\end{itemize}
\begin{itemize}
\item {Grp. gram.:m.}
\end{itemize}
Aquelle que commete patruicídio.
\section{Patruicídio}
\begin{itemize}
\item {fónica:tru-i}
\end{itemize}
\begin{itemize}
\item {Grp. gram.:m.}
\end{itemize}
\begin{itemize}
\item {Proveniência:(Do lat. \textunderscore patruus\textunderscore  + \textunderscore caedere\textunderscore )}
\end{itemize}
Assassínio de tio paterno. Cf. Rev. \textunderscore Movimento Méd.\textunderscore , VI, 148.
\section{Patrulha}
\begin{itemize}
\item {Grp. gram.:f.}
\end{itemize}
\begin{itemize}
\item {Utilização:Fig.}
\end{itemize}
\begin{itemize}
\item {Utilização:Deprec.}
\end{itemize}
Ronda de soldados.
Grupo de pessôas, que andam a passo, como as rondas.
Pequeno agrupamento político.
Bando de vádios, súcia.
(Cast. \textunderscore patrulla\textunderscore )
\section{Patrulhar}
\begin{itemize}
\item {Grp. gram.:v. t.}
\end{itemize}
\begin{itemize}
\item {Grp. gram.:V. i.}
\end{itemize}
Guarnecer ou vigiar com patrulhas.
Fazer ronda em patrulha.
\section{Patuá}
\begin{itemize}
\item {Grp. gram.:m.}
\end{itemize}
\begin{itemize}
\item {Utilização:Bras}
\end{itemize}
O mesmo que \textunderscore patiguá\textunderscore .
Saco de coiro ou de pano, que se leva a tiracolo.
Bôlsa de caça.
Nome commum a diversos receptáculos móveis, em que se transportam quaesquer objectos.
Espécie de cesto, com compartimentos para comida, loiças, etc., usado nas viagens fluviaes.
Espécie de amuleto, que consiste em um saquinho de coiro, contendo cabeças de cobra e outras coisas, a que se attribuem qualidades milagrosas, e que os crédulos trazem ao pescoço para os livrar de malefícios.
(Do tupi \textunderscore patauá\textunderscore )
\section{Patudo}
\begin{itemize}
\item {Grp. gram.:adj.}
\end{itemize}
Que tem patas grandes.
\section{Patuguá}
\begin{itemize}
\item {Grp. gram.:m.}
\end{itemize}
O mesmo que \textunderscore patiguá\textunderscore .
\section{Patuléa}
\begin{itemize}
\item {Grp. gram.:f.}
\end{itemize}
\begin{itemize}
\item {Grp. gram.:M.}
\end{itemize}
Partido popular, que se organizou em Portugal, em 1846.
Membro dêsse partido.
(Cast. \textunderscore patulea\textunderscore , do caló \textunderscore patulé\textunderscore , rústico)
\section{Patulear}
\begin{itemize}
\item {Grp. gram.:v. t.}
\end{itemize}
Tornar partidário da patuleia. Cf. Camillo, \textunderscore Mar. da Fonte\textunderscore , 7.
\section{Patuleia}
\begin{itemize}
\item {Grp. gram.:f.}
\end{itemize}
\begin{itemize}
\item {Grp. gram.:M.}
\end{itemize}
Partido popular, que se organizou em Portugal, em 1846.
Membro dêsse partido.
(Cast. \textunderscore patulea\textunderscore , do caló \textunderscore patulé\textunderscore , rústico)
\section{Pátulo}
\begin{itemize}
\item {Grp. gram.:adj.}
\end{itemize}
\begin{itemize}
\item {Utilização:Poét.}
\end{itemize}
\begin{itemize}
\item {Proveniência:(Lat. \textunderscore patulus\textunderscore )}
\end{itemize}
Franqueado, patente.
\section{Patuno}
\begin{itemize}
\item {Grp. gram.:m.}
\end{itemize}
\begin{itemize}
\item {Utilização:Gír.}
\end{itemize}
Partes pudendas da mulhér.
\section{Paturé}
\begin{itemize}
\item {Grp. gram.:m.}
\end{itemize}
\begin{itemize}
\item {Utilização:Bras}
\end{itemize}
Espécie de marreco pequeno.
\section{Patureba}
\begin{itemize}
\item {Grp. gram.:f.}
\end{itemize}
\begin{itemize}
\item {Utilização:Bras. do Rio}
\end{itemize}
Bagre salgado.
\section{Patureba}
\begin{itemize}
\item {Grp. gram.:m. ,  f.  e  adj.}
\end{itemize}
\begin{itemize}
\item {Utilização:Bras}
\end{itemize}
Pessôa sem préstimo, atoleimada.
\section{Paturi}
\begin{itemize}
\item {Grp. gram.:m.}
\end{itemize}
\begin{itemize}
\item {Utilização:Bras. do N}
\end{itemize}
O mesmo que \textunderscore paturé\textunderscore .
\section{Patuscada}
\begin{itemize}
\item {Grp. gram.:f.}
\end{itemize}
\begin{itemize}
\item {Utilização:Pop.}
\end{itemize}
\begin{itemize}
\item {Proveniência:(De \textunderscore patusco\textunderscore ^1)}
\end{itemize}
Ajuntamento festivo de pessôas, que se reuniram para comer e beber.
Pândega; folgança.
\section{Patuscar}
\begin{itemize}
\item {Grp. gram.:v. i.}
\end{itemize}
Andar em patuscadas; fazer patuscadas.
\section{Patusco}
\begin{itemize}
\item {Grp. gram.:m.  e  adj.}
\end{itemize}
Aquelle que gosta de patuscadas.
Brincalhão; ridículo; extravagante.
\section{Patusco}
\begin{itemize}
\item {Grp. gram.:m.}
\end{itemize}
\begin{itemize}
\item {Utilização:Prov.}
\end{itemize}
\begin{itemize}
\item {Utilização:Prov.}
\end{itemize}
\begin{itemize}
\item {Utilização:minh.}
\end{itemize}
Pequeno pão ou bolo de trigo, ás vezes doce.
Pequeno pão de centeio ou milho, o mesmo que \textunderscore patareco\textunderscore .
(Por \textunderscore padusco\textunderscore , de \textunderscore pada\textunderscore )
\section{Patusqueiro}
\begin{itemize}
\item {Grp. gram.:adj.}
\end{itemize}
\begin{itemize}
\item {Utilização:Prov.}
\end{itemize}
\begin{itemize}
\item {Utilização:trasm.}
\end{itemize}
\begin{itemize}
\item {Proveniência:(De \textunderscore patuscar\textunderscore )}
\end{itemize}
Alegre, divertido.
\section{Pau}
\begin{itemize}
\item {Grp. gram.:m.}
\end{itemize}
\begin{itemize}
\item {Utilização:Ext.}
\end{itemize}
\begin{itemize}
\item {Grp. gram.:Pl.}
\end{itemize}
\begin{itemize}
\item {Grp. gram.:Loc.}
\end{itemize}
\begin{itemize}
\item {Utilização:fam.}
\end{itemize}
\begin{itemize}
\item {Proveniência:(Do lat. \textunderscore palus\textunderscore , \textunderscore pali\textunderscore )}
\end{itemize}
Pedaço de madeira.
Madeira.
Cajado.
Viga.
Ripa.
Vara.
Castigo corporal: \textunderscore isto, só a pau\textunderscore !
Chifre.
Nome de várias plantas.
Um dos naipes pretos das cartas de jogar.
\textunderscore A dar com um pau\textunderscore , em grande quantidade: \textunderscore a vinha produziu uvas a dar com um pau\textunderscore .
\textunderscore Dar por paus e por pedras\textunderscore , irritar-se muito.
\textunderscore Pau para toda a obra\textunderscore , pessôa que serve para tudo, ou que se applica a muitas e differentes coisas.
\textunderscore Jogar com pau de dois bicos\textunderscore , defender ao mesmo tempo ideias oppostas.
\textunderscore Bandeira a meio pau\textunderscore , bandeira a meia haste.
\section{Pau-água}
\begin{itemize}
\item {Grp. gram.:m.}
\end{itemize}
Árvore medicinal da ilha de San-Thomé.
\section{Paual}
\begin{itemize}
\item {Grp. gram.:m.}
\end{itemize}
Antigo e pequeno peso de Malaca.
\section{Pau-alho}
\begin{itemize}
\item {Grp. gram.:m.}
\end{itemize}
Grande árvore medicinal da ilha de San-Thomé.
\section{Pau-ama}
\begin{itemize}
\item {Grp. gram.:m.}
\end{itemize}
Árvore da ilha do San-Thomé, própria para construcções.
\section{Pau-amarelo}
\begin{itemize}
\item {Grp. gram.:m.}
\end{itemize}
\begin{itemize}
\item {Utilização:Bras}
\end{itemize}
Árvore silvestre, de bôa madeira para construcções.
\section{Pau-a-pique}
\begin{itemize}
\item {Grp. gram.:m.}
\end{itemize}
\begin{itemize}
\item {Utilização:Bras. do S}
\end{itemize}
Parede de ripas ou varas, umas verticaes e outras horizontaes.
Taipa.
\section{Pau-azami}
\begin{itemize}
\item {Grp. gram.:m.}
\end{itemize}
Planta condimentosa da ilha de San-Thomé.
\section{Pau-azeite}
\begin{itemize}
\item {Grp. gram.:m.}
\end{itemize}
Árvore de Angola, (\textunderscore sideroxylon densiflorum\textunderscore ).
\section{Pau-bala}
\begin{itemize}
\item {Grp. gram.:m.}
\end{itemize}
Planta meliácea, (\textunderscore trichilia guarea\textunderscore ).
\section{Pau-bálsamo}
\begin{itemize}
\item {Grp. gram.:m.}
\end{itemize}
Planta leguminosa, de que se extrai o chamado \textunderscore bálsamo do Peru\textunderscore .
\section{Pau-barro}
\begin{itemize}
\item {Grp. gram.:m.}
\end{itemize}
Árvore de Timor.
\section{Pau-branco}
\begin{itemize}
\item {Grp. gram.:m.}
\end{itemize}
Grande árvore santhomense, cujas raízes tem propriedades purgativas.
Arvore açoreana, (\textunderscore picconia excelsa\textunderscore ).
\section{Pau-brasil}
\begin{itemize}
\item {Grp. gram.:m.}
\end{itemize}
O mesmo que \textunderscore pau-preto\textunderscore .
\section{Pau-breu}
\begin{itemize}
\item {Grp. gram.:m.}
\end{itemize}
Madeira brasileira.
\section{Pau-caixão}
\begin{itemize}
\item {Grp. gram.:m.}
\end{itemize}
Árvore santhomense, de casca e raíz medicinaes.
\section{Pau-campeche}
\begin{itemize}
\item {Grp. gram.:m.}
\end{itemize}
(V.campeche)
\section{Pau-canela}
\begin{itemize}
\item {Grp. gram.:m.}
\end{itemize}
O mesmo que \textunderscore caneleira\textunderscore .
\section{Pau-capitão}
\begin{itemize}
\item {Grp. gram.:m.}
\end{itemize}
Grande árvore santhomense, que attinge 50 metros de altura.
\section{Pau-cardoso}
\begin{itemize}
\item {Grp. gram.:m.}
\end{itemize}
Espécie de fêto.
\section{Pau-carga}
\begin{itemize}
\item {Grp. gram.:m.}
\end{itemize}
Planta brasileira, (\textunderscore cascaria usucaris\textunderscore ).
\section{Pau-catinga}
\begin{itemize}
\item {Grp. gram.:m.}
\end{itemize}
\begin{itemize}
\item {Utilização:Bras}
\end{itemize}
Planta medicinal.
\section{Pau-cavallo}
\begin{itemize}
\item {Grp. gram.:m.}
\end{itemize}
Planta verbenácea, (\textunderscore vitex nigrum\textunderscore ).
\section{Pau-caxique}
\begin{itemize}
\item {Grp. gram.:m.}
\end{itemize}
Árvore meliácea africana.
\section{Pau-cera}
\begin{itemize}
\item {Grp. gram.:m.}
\end{itemize}
Árvore brasileira, bôa para construcções.
\section{Paucifloro}
\begin{itemize}
\item {Grp. gram.:adj.}
\end{itemize}
\begin{itemize}
\item {Utilização:Bot.}
\end{itemize}
\begin{itemize}
\item {Proveniência:(Do lat. \textunderscore paucus\textunderscore  + \textunderscore flos\textunderscore )}
\end{itemize}
Que apresenta pequeno número de flôres.
\section{Pauciradiado}
\begin{itemize}
\item {fónica:ra}
\end{itemize}
\begin{itemize}
\item {Grp. gram.:adj.}
\end{itemize}
\begin{itemize}
\item {Utilização:Bot.}
\end{itemize}
\begin{itemize}
\item {Proveniência:(Do lat. \textunderscore paucus\textunderscore  + \textunderscore radius\textunderscore )}
\end{itemize}
Diz-se das flôres compostas, quando têm pequenos números de raios.
\section{Paucirradiado}
\begin{itemize}
\item {Grp. gram.:adj.}
\end{itemize}
\begin{itemize}
\item {Utilização:Bot.}
\end{itemize}
\begin{itemize}
\item {Proveniência:(Do lat. \textunderscore paucus\textunderscore  + \textunderscore radius\textunderscore )}
\end{itemize}
Diz-se das flôres compostas, quando têm pequenos números de raios.
\section{Pauciseriado}
\begin{itemize}
\item {fónica:se}
\end{itemize}
\begin{itemize}
\item {Grp. gram.:adj.}
\end{itemize}
\begin{itemize}
\item {Utilização:Bot.}
\end{itemize}
\begin{itemize}
\item {Proveniência:(Do lat. \textunderscore paucus\textunderscore  + \textunderscore series\textunderscore )}
\end{itemize}
Dividido em poucas séries.
\section{Paucisseriado}
\begin{itemize}
\item {Grp. gram.:adj.}
\end{itemize}
\begin{itemize}
\item {Utilização:Bot.}
\end{itemize}
\begin{itemize}
\item {Proveniência:(Do lat. \textunderscore paucus\textunderscore  + \textunderscore series\textunderscore )}
\end{itemize}
Dividido em poucas séries.
\section{Pau-cobra}
\begin{itemize}
\item {Grp. gram.:m.}
\end{itemize}
Árvore rutácea, (\textunderscore quassia ophyoryza\textunderscore ).
\section{Pau-costus}
\begin{itemize}
\item {Grp. gram.:m.}
\end{itemize}
Árvore rutácea da Índia portuguesa, (\textunderscore aeronychia laurifolia\textunderscore , Blume).
\section{Pau-cravo}
\begin{itemize}
\item {Grp. gram.:m.}
\end{itemize}
Planta laurácea, (\textunderscore dicypellium caryophyllatum\textunderscore ).
\section{Pauda}
\begin{itemize}
\item {Grp. gram.:f.}
\end{itemize}
Formosa ave africana.
\section{Pau-dado}
\begin{itemize}
\item {Grp. gram.:m.}
\end{itemize}
Planta medicinal da ilha de San-Thomé.
\section{Pau-de-água}
\begin{itemize}
\item {Grp. gram.:m.}
\end{itemize}
\begin{itemize}
\item {Utilização:Bras}
\end{itemize}
Árvore silvestre, cujas raízes segregam um líquido que mata a sêde aos viajantes.
\section{Pau-de-arco}
\begin{itemize}
\item {Grp. gram.:m.}
\end{itemize}
Planta bigoniácea, (\textunderscore bignonia chrysantha\textunderscore ).
\section{Pau-de-bóla}
\begin{itemize}
\item {Grp. gram.:m.}
\end{itemize}
Árvore da Guiné, cujas fôlhas têm propriedades cáusticas.
\section{Pau-de-breu}
\begin{itemize}
\item {Grp. gram.:m.}
\end{itemize}
\begin{itemize}
\item {Utilização:Bras}
\end{itemize}
O mesmo que \textunderscore pau-breu\textunderscore .
\section{Pau-de-caca}
\begin{itemize}
\item {Grp. gram.:m.}
\end{itemize}
Planta capparídea, (\textunderscore capparis amygdalina\textunderscore ).
\section{Pau-de-cachimbo}
\begin{itemize}
\item {Grp. gram.:m.}
\end{itemize}
Planta borragínea, (\textunderscore heliotropium punctatus\textunderscore ).
\section{Pau-de-carne}
\begin{itemize}
\item {Grp. gram.:m.}
\end{itemize}
O mesmo que \textunderscore pau-carga\textunderscore .
\section{Pau-de-chanca}
\begin{itemize}
\item {Grp. gram.:f.}
\end{itemize}
Planta esterculiácea da Índia portuguesa, (\textunderscore helicteres isora\textunderscore , Lin.).
\section{Pau-de-cobra}
\begin{itemize}
\item {Grp. gram.:m.}
\end{itemize}
Gênero de plantas apocýneas da Índia portuguesa, (\textunderscore rawolfia serpentina\textunderscore , Benth.).
Gênero de plantas loganiáceas da Índia portuguesa, (\textunderscore strychnos colubrina\textunderscore , Lin.).
Gênero de plantas aristolochiáceas da Índia portuguesa, (\textunderscore aristolochia indica\textunderscore , Lin.). Cf. Dalgado, \textunderscore Flora\textunderscore , 117, 124 e 158.
\section{Pau-de-colhér}
\begin{itemize}
\item {Grp. gram.:m.}
\end{itemize}
Planta apocýnea, (\textunderscore tabernaemontana echinata\textunderscore ).
\section{Pau-de-combra}
\begin{itemize}
\item {Grp. gram.:m.}
\end{itemize}
Árvore de Moçambique, espécie de acácia brava.
\section{Pau-de-conta}
\begin{itemize}
\item {Grp. gram.:m.}
\end{itemize}
Árvore da Guiné portuguesa, (\textunderscore swietenia mahegoni\textunderscore ).
\section{Pau-de-elephante}
\begin{itemize}
\item {Grp. gram.:m.}
\end{itemize}
Árvore da Áfr. Port.
\section{Pau-de-embira}
\begin{itemize}
\item {Grp. gram.:m.}
\end{itemize}
Planta anonácea do Brasil.
\section{Pau-de-escrever}
\begin{itemize}
\item {Grp. gram.:m.}
\end{itemize}
\begin{itemize}
\item {Utilização:T. da Áfr. Or. Port}
\end{itemize}
Poste telegráphico.
\section{Pau-de-faia}
\begin{itemize}
\item {Grp. gram.:m.}
\end{itemize}
Grande árvore medicinal da Guiné.
\section{Pau-de-farinha}
\begin{itemize}
\item {Grp. gram.:m.}
\end{itemize}
Nome, que na Índia portuguesa se dá á mandioca e á tapioca.
\section{Pau-de-lacre}
\begin{itemize}
\item {Grp. gram.:m.}
\end{itemize}
Árvore hypericácea, de cujo tronco, e por meio de incisões, se extrai a goma lacre.
\section{Pau-de-leite}
\begin{itemize}
\item {Grp. gram.:m.}
\end{itemize}
Corpulenta árvore da Guiné, que produz borracha, e cujo suco é purgativo.
\section{Pau-de-macaco}
\begin{itemize}
\item {Grp. gram.:m.}
\end{itemize}
\begin{itemize}
\item {Utilização:Bras}
\end{itemize}
Árvore silvestre.
\section{Pau-de-majerioba}
\begin{itemize}
\item {Grp. gram.:m.}
\end{itemize}
\begin{itemize}
\item {Utilização:Bras}
\end{itemize}
Planta medicinal.
\section{Pau-de-merda}
\begin{itemize}
\item {Grp. gram.:m.}
\end{itemize}
O mesmo que \textunderscore pau-sujo\textunderscore .
\section{Pau-de-pente}
\begin{itemize}
\item {Grp. gram.:m.}
\end{itemize}
O mesmo que \textunderscore pau-forquilha\textunderscore .
\section{Pau-de-porco}
\begin{itemize}
\item {Grp. gram.:m.}
\end{itemize}
Planta terebinthácea, (\textunderscore bursera gummifera\textunderscore ).
\section{Pau-de-quiabo}
\begin{itemize}
\item {Grp. gram.:m.}
\end{itemize}
Planta laurínea, (\textunderscore laurus speciosa\textunderscore ).
\section{Pau-de-rato}
\begin{itemize}
\item {Grp. gram.:m.}
\end{itemize}
\begin{itemize}
\item {Utilização:Bras. do N}
\end{itemize}
Árvore dos sertões de Piauí.
\section{Pau-de-rosa}
\begin{itemize}
\item {Grp. gram.:m.}
\end{itemize}
Planta malvácea da Índia portuguesa, (\textunderscore thespesia populnea\textunderscore , Correia).
\section{Pau-de-sabão}
\begin{itemize}
\item {Grp. gram.:m.}
\end{itemize}
Árvore da Guiné, que attinge 5 metros de altura.
\section{Pau-de-salanca}
\begin{itemize}
\item {Grp. gram.:m.}
\end{itemize}
Árvore da ilha de San-Thomé.
\section{Pau-de-sangue}
\begin{itemize}
\item {Grp. gram.:m.}
\end{itemize}
O mesmo que \textunderscore pau-sangue\textunderscore .
\section{Pau-de-san-josé}
\begin{itemize}
\item {Grp. gram.:m.}
\end{itemize}
Planta, espécie de lírio, de flôres rôxas.
\section{Pau-de-santa-luzia}
\begin{itemize}
\item {Grp. gram.:m.}
\end{itemize}
Planta leguminosa, (\textunderscore monadelphia decandria\textunderscore ).
\section{Pau-de-sapan}
\begin{itemize}
\item {Grp. gram.:m.}
\end{itemize}
Planta cesalpínea da Índia portuguesa, (\textunderscore caesalpinea paniculata\textunderscore , Roxb.).
\section{Pau-de-sassafrás}
\begin{itemize}
\item {Grp. gram.:m.}
\end{itemize}
Planta laurácea, (\textunderscore laurus sassafras\textunderscore ).
\section{Pau-de-semana}
\begin{itemize}
\item {Grp. gram.:m.}
\end{itemize}
\begin{itemize}
\item {Utilização:Bras}
\end{itemize}
Espécie de palmeira silvestre.
\section{Pau-de-tagara}
\begin{itemize}
\item {Grp. gram.:m.}
\end{itemize}
Grande árvore da Guiné, que attinge 25 metros de altura.
\section{Pau-de-tingui}
\begin{itemize}
\item {Grp. gram.:m.}
\end{itemize}
Planta sapindácea, (\textunderscore mogonia pubescens\textunderscore ).
\section{Pau-de-ar}
\begin{itemize}
\item {Grp. gram.:m.}
\end{itemize}
Euphemismo, com que se designa o \textunderscore corno\textunderscore . Cf. O'Neill, \textunderscore Fabul.\textunderscore , 957.
\section{Pau-doce}
\begin{itemize}
\item {Grp. gram.:m.}
\end{itemize}
Pequena árvore da Guiné, de fôlhas medicinaes.
\section{Pau-do-novato}
\begin{itemize}
\item {Grp. gram.:m.}
\end{itemize}
Planta polygoniácea, (\textunderscore tripterix americana\textunderscore ).
\section{Pau-do-serrote}
\begin{itemize}
\item {Grp. gram.:m.}
\end{itemize}
Planta leguminosa, (\textunderscore hoffmanuseggia petra\textunderscore ).
\section{Pau-dos-feiticeiros}
\begin{itemize}
\item {Grp. gram.:m.}
\end{itemize}
Árvore africana, cuja casca venenosa, é objecto de commércio.
\section{Pau-dos-olhos}
\begin{itemize}
\item {Grp. gram.:m.}
\end{itemize}
\begin{itemize}
\item {Utilização:Bras}
\end{itemize}
Árvore silvestre, empregada em construcções, e cujo fumo é mui nocivo á vista.
\section{Pau-espinha}
\begin{itemize}
\item {Grp. gram.:m.}
\end{itemize}
Árvore medicinal da ilha de San-Thomé.
\section{Pau-esteira}
\begin{itemize}
\item {Grp. gram.:f.}
\end{itemize}
Árvore medicinal da ilha de San-Thomé.
\section{Pau-ferro}
\begin{itemize}
\item {Grp. gram.:m.}
\end{itemize}
Gênero de árvores intertropicaes, cuja madeira é universalmente estimada, pela sua resistência e duração, (\textunderscore acacia catchu\textunderscore , Wild.).
Casta de uva de Azeitão.
\section{Pau-ferro-da-índia}
\begin{itemize}
\item {Grp. gram.:m.}
\end{itemize}
Gênero de plantas gutíferas da Índia portuguesa, (\textunderscore mesua ferrea\textunderscore , Lin.).
\section{Pau-formiga}
\begin{itemize}
\item {Grp. gram.:m.}
\end{itemize}
Árvore brasileira, quási sempre coberta de formigas.
\section{Pau-forquilha}
\begin{itemize}
\item {Grp. gram.:m.}
\end{itemize}
Grande árvore apocýnea do Brasil, (\textunderscore geissospermum velosii\textunderscore ), o mesmo que \textunderscore pau-pereiro\textunderscore .
\section{Pau-geremu}
\begin{itemize}
\item {Grp. gram.:m.}
\end{itemize}
Planta chenopodiácea, (\textunderscore spinacea gerimu\textunderscore ).
\section{Pau-goma-arábica}
\begin{itemize}
\item {Grp. gram.:f.}
\end{itemize}
Árvore angolense, espécie de acácia.
\section{Pau-joão-lopes}
\begin{itemize}
\item {Grp. gram.:m.}
\end{itemize}
Árvore de Timor.
\section{Paúl}
\begin{itemize}
\item {Grp. gram.:m.}
\end{itemize}
\begin{itemize}
\item {Proveniência:(De \textunderscore padule\textunderscore , metáth. do lat. \textunderscore palude\textunderscore , de \textunderscore palus\textunderscore , \textunderscore paludis\textunderscore )}
\end{itemize}
Porção de água estagnada; terra alagadiça; pântano.
\section{Paula}
\begin{itemize}
\item {Grp. gram.:f.}
\end{itemize}
\begin{itemize}
\item {Utilização:T. de Monção}
\end{itemize}
Moéda de prata, de 500 reis.
\section{Pau-lacre}
\begin{itemize}
\item {Grp. gram.:m.}
\end{itemize}
O mesmo que \textunderscore pau-de-lacre\textunderscore .
\section{Paulada}
\begin{itemize}
\item {Grp. gram.:f.}
\end{itemize}
\begin{itemize}
\item {Utilização:Prov.}
\end{itemize}
\begin{itemize}
\item {Utilização:trasm.}
\end{itemize}
\begin{itemize}
\item {Proveniência:(De \textunderscore pau\textunderscore )}
\end{itemize}
Pancada com pau; cacetada.
Choque de um pião contra outro.
\section{Pau-lágrima}
\begin{itemize}
\item {Grp. gram.:m.}
\end{itemize}
Árvore das regiões do Amazonas.
\section{Paulatinamente}
\begin{itemize}
\item {Grp. gram.:adv.}
\end{itemize}
De modo paulatino; a pouco e pouco; lentamente.
\section{Paulatino}
\begin{itemize}
\item {Grp. gram.:adj.}
\end{itemize}
\begin{itemize}
\item {Proveniência:(Do lat. \textunderscore paulatim\textunderscore )}
\end{itemize}
Feito a pouco e pouco; vagaroso.
\section{Paulianistas}
\begin{itemize}
\item {Grp. gram.:m. pl.}
\end{itemize}
Herejes, sectários de Paulo de Samosata, que no século III sustentou que Christo era um homem, animado de espirito divino.
\section{Paulina}
\begin{itemize}
\item {Grp. gram.:f.}
\end{itemize}
\begin{itemize}
\item {Utilização:Fam.}
\end{itemize}
\begin{itemize}
\item {Proveniência:(De \textunderscore Paulo\textunderscore , n. p.)}
\end{itemize}
Breve de excommunhão comminatória.
Praga.
\section{Paulínea}
\begin{itemize}
\item {Grp. gram.:f.}
\end{itemize}
\begin{itemize}
\item {Proveniência:(De \textunderscore Paulli\textunderscore , n. p.)}
\end{itemize}
Gênero de plantas trepadeiras da África e da América.
\section{Paulínia}
\begin{itemize}
\item {Grp. gram.:f.}
\end{itemize}
Gênero de plantas do Amazonas.
\section{Paulista}
\begin{itemize}
\item {Grp. gram.:m.}
\end{itemize}
\begin{itemize}
\item {Utilização:Fig.}
\end{itemize}
\begin{itemize}
\item {Utilização:Bras}
\end{itemize}
\begin{itemize}
\item {Grp. gram.:Adj.}
\end{itemize}
Frade da Ordem de San-Paulo.
Homem teimoso.
Habitante do Estado de San-Paulo.
Relativo a êsse Estado: \textunderscore a imprensa paulista\textunderscore .
\section{Paulistano}
\begin{itemize}
\item {Grp. gram.:adj.}
\end{itemize}
\begin{itemize}
\item {Utilização:P. us.}
\end{itemize}
\begin{itemize}
\item {Grp. gram.:M.}
\end{itemize}
Relativo á cidade de San Paulo, no Brasil, ou antes aos Paulistas.
Habitante dessa cidade.
\section{Pauliteiro}
\begin{itemize}
\item {Grp. gram.:m.}
\end{itemize}
Aquelle que entra na dança dos paulitos.
\section{Paulito}
\begin{itemize}
\item {Grp. gram.:m.}
\end{itemize}
\begin{itemize}
\item {Utilização:Pop.}
\end{itemize}
\begin{itemize}
\item {Grp. gram.:Pl.}
\end{itemize}
Pequeno pau, que serve de fito no jôgo de bilhar, da malha e outros.
Phósphoro de pau.
\textunderscore Dança dos paulitos\textunderscore , dança mirandesa, também conhecida por \textunderscore dança dos palotes\textunderscore . V. \textunderscore palotes\textunderscore .
(Por \textunderscore palito\textunderscore , do lat. \textunderscore palus\textunderscore )
\section{Paullínea}
\begin{itemize}
\item {Grp. gram.:f.}
\end{itemize}
\begin{itemize}
\item {Proveniência:(De \textunderscore Paulli\textunderscore , n. p.)}
\end{itemize}
Gênero de plantas trepadeiras da África e da América.
\section{Paúlo}
\begin{itemize}
\item {Grp. gram.:m.}
\end{itemize}
\begin{itemize}
\item {Utilização:Prov.}
\end{itemize}
\begin{itemize}
\item {Utilização:minh.}
\end{itemize}
O mesmo que \textunderscore paúl\textunderscore .
Cerrado para o gado.
\section{Paulóvnia}
\begin{itemize}
\item {Grp. gram.:f.}
\end{itemize}
\begin{itemize}
\item {Proveniência:(De \textunderscore Paulovna\textunderscore , n. p.)}
\end{itemize}
Árvore do Japão, de grandes fôlhas e de flôres muito aromáticas.
\section{Pau-maçan}
\begin{itemize}
\item {Grp. gram.:m.}
\end{itemize}
Árvore de Timor.
\section{Pau-mamão}
\begin{itemize}
\item {Grp. gram.:m.}
\end{itemize}
\begin{itemize}
\item {Utilização:Bras}
\end{itemize}
Árvore silvestre.
\section{Pau-marfim}
\begin{itemize}
\item {Grp. gram.:m.}
\end{itemize}
\begin{itemize}
\item {Utilização:Bras}
\end{itemize}
Árvore silvestre.
\section{Pau-milho}
\begin{itemize}
\item {Grp. gram.:m.}
\end{itemize}
Grande árvore medicinal da ilha de San-Thomé.
\section{Pau-moéda}
\begin{itemize}
\item {Grp. gram.:m.}
\end{itemize}
O mesmo que \textunderscore seringueira\textunderscore .
\section{Pau-mucumbi}
\begin{itemize}
\item {Grp. gram.:m.}
\end{itemize}
Árvore anacardiácea e medicinal de Angola, (\textunderscore odina acida\textunderscore , Walp.).
\section{Pau-mulato}
\begin{itemize}
\item {Grp. gram.:m.}
\end{itemize}
\begin{itemize}
\item {Utilização:Bras}
\end{itemize}
Árvore silvestre, cuja madeira é applicada em marcenaria.
\section{Pau-musence}
\begin{itemize}
\item {Grp. gram.:m.}
\end{itemize}
(V. \textunderscore muzungo\textunderscore ^2)
\section{Pau-óleo}
\begin{itemize}
\item {Grp. gram.:m.}
\end{itemize}
Árvore medicinal da ilha de San-Thomé e de Angola.
\section{Pau-osso}
\begin{itemize}
\item {Grp. gram.:m.}
\end{itemize}
\begin{itemize}
\item {Utilização:Bras}
\end{itemize}
Forragem arbórea.
\section{Pau-papel}
\begin{itemize}
\item {Grp. gram.:m.}
\end{itemize}
Planta melastomácea.
\section{Pau-paraíba}
\begin{itemize}
\item {Grp. gram.:m.}
\end{itemize}
\begin{itemize}
\item {Utilização:Bras}
\end{itemize}
Árvore silvestre, de madeira alvissima.
\section{Paupeira}
\begin{itemize}
\item {Grp. gram.:f.}
\end{itemize}
Planta apocýnea do Brasil.
\section{Pau-pente}
\begin{itemize}
\item {Grp. gram.:m.}
\end{itemize}
O mesmo que \textunderscore pau-de-pente\textunderscore ; e o mesmo que \textunderscore pau-pereira\textunderscore .
\section{Pau-pereira}
\begin{itemize}
\item {Grp. gram.:f.}
\end{itemize}
(Fórma incorrecta, em vez de \textunderscore pau-pereiro\textunderscore )
\section{Pau-pereiro}
\begin{itemize}
\item {Grp. gram.:m.}
\end{itemize}
Grande árvore apocýnea do Brasil, (\textunderscore geissos permum laeve\textunderscore ), ou, antes, a casca dessa árvore.
\section{Paupérie}
\begin{itemize}
\item {Grp. gram.:f.}
\end{itemize}
\begin{itemize}
\item {Proveniência:(Lat. \textunderscore pauperies\textunderscore )}
\end{itemize}
Pauperismo; miséria.
\section{Pauperismo}
\begin{itemize}
\item {Grp. gram.:m.}
\end{itemize}
\begin{itemize}
\item {Proveniência:(Do lat. \textunderscore pauper\textunderscore )}
\end{itemize}
Classe dos pobres; a miséria.
\section{Paupérrimo}
\begin{itemize}
\item {Grp. gram.:adj.}
\end{itemize}
\begin{itemize}
\item {Proveniência:(Lat. \textunderscore pauperrimus\textunderscore )}
\end{itemize}
Muito pobre.
\section{Pau-pêssego}
\begin{itemize}
\item {Grp. gram.:m.}
\end{itemize}
Árvore da ilha de San-Thomé, (\textunderscore chitranthus manu\textunderscore ).
\section{Pau-piaçaba}
\begin{itemize}
\item {Grp. gram.:m.}
\end{itemize}
O mesmo que \textunderscore piaçaba\textunderscore .
\section{Pau-pombo}
\begin{itemize}
\item {Grp. gram.:m.}
\end{itemize}
Espécie de anacardo, (\textunderscore odina francoana\textunderscore ).
\section{Pau-porco}
\begin{itemize}
\item {Grp. gram.:m.}
\end{itemize}
Árvore terebinthácea, o mesmo que \textunderscore pau-de-porco\textunderscore .
\section{Pau-preto}
\begin{itemize}
\item {Grp. gram.:m.}
\end{itemize}
\begin{itemize}
\item {Utilização:T. das Caldas-da-Raínha}
\end{itemize}
Árvore leguminosa, intertropical, de madeira escura e resistente, (\textunderscore dalbergia latifolia\textunderscore , Roxb.).
Casta de maçan escura e um tanto comprida.
\section{Pau-quicongo}
\begin{itemize}
\item {Grp. gram.:m.}
\end{itemize}
Árvore medicinal da África.
\section{Pau-quime}
\begin{itemize}
\item {Grp. gram.:m.}
\end{itemize}
Árvore santhomense.
\section{Pau-quizemba}
\begin{itemize}
\item {Grp. gram.:m.}
\end{itemize}
Robusta planta trepadeira de Angola, de tronco lenhoso, espinhoso e achatado.
\section{Pau-raínha}
\begin{itemize}
\item {Grp. gram.:m.}
\end{itemize}
Árvore leguminosa do Pará.
\section{Pau-real}
\begin{itemize}
\item {Grp. gram.:m.}
\end{itemize}
Nos estaleiros de construcção naval, dá-se êste nome a um madeiro grosso, comprido, são e sem nós, próprio para um bom mastro.
Mastro real.
\section{Pau-rosa}
\begin{itemize}
\item {Grp. gram.:m.}
\end{itemize}
Árvore africana, (\textunderscore physocalymna florida\textunderscore ).
\section{Pau-rosado}
\begin{itemize}
\item {Grp. gram.:m.}
\end{itemize}
Planta leguminosa, (\textunderscore caesalpinia brasiliensis\textunderscore  ou \textunderscore echinata\textunderscore ).
\section{Pau-roxo}
\begin{itemize}
\item {Grp. gram.:m.}
\end{itemize}
\begin{itemize}
\item {Utilização:Bras}
\end{itemize}
Árvore silvestre.
\section{Pausa}
\begin{itemize}
\item {Grp. gram.:f.}
\end{itemize}
\begin{itemize}
\item {Proveniência:(Lat. \textunderscore pausa\textunderscore )}
\end{itemize}
Interrupção de um acto por algum tempo.
Vagar.
Sinal, com que na música se indicam as interrupções.
Intervallo das vigas de um madeiramento.
\section{Pausadamente}
\begin{itemize}
\item {Grp. gram.:adv.}
\end{itemize}
De modo pausado.
Com vagar; com pausa; lentamente.
\section{Pausado}
\begin{itemize}
\item {Grp. gram.:adj.}
\end{itemize}
\begin{itemize}
\item {Proveniência:(De \textunderscore pausar\textunderscore )}
\end{itemize}
Vagaroso; lento; feito com pausa.
Pronunciado vagarosamente.
\section{Pausagem}
\begin{itemize}
\item {Grp. gram.:f.}
\end{itemize}
\begin{itemize}
\item {Proveniência:(De \textunderscore pausa\textunderscore )}
\end{itemize}
Madeiramento, cujas vigas deixam entre si intervallos ou pausas.
\section{Pau-salgado-machô}
\begin{itemize}
\item {Grp. gram.:m.}
\end{itemize}
Árvore da Índia portuguesa, (\textunderscore bruguiera gymnorhiza\textunderscore , Lamk).
\section{Pau-sangue}
\begin{itemize}
\item {Grp. gram.:m.}
\end{itemize}
Árvore medicinal da ilha de San-Thomé.
O mesmo que \textunderscore urilha\textunderscore .
\section{Pau-santo}
\begin{itemize}
\item {Grp. gram.:m.}
\end{itemize}
Árvore rutácea intertropical, (\textunderscore guaiacum officinalis\textunderscore ).
\section{Pausar}
\begin{itemize}
\item {Grp. gram.:v. t.}
\end{itemize}
\begin{itemize}
\item {Grp. gram.:V. i.}
\end{itemize}
Demorar.
Descansar.
Tornar lento, vagaroso: \textunderscore pausar a voz\textunderscore .
Fazer pausa.
\section{Pau-seringa}
\begin{itemize}
\item {Grp. gram.:m.}
\end{itemize}
O mesmo que \textunderscore seringueira\textunderscore .
\section{Pau-setim}
\begin{itemize}
\item {Grp. gram.:m.}
\end{itemize}
Planta apocýnea, (\textunderscore aspidoperma speciosa\textunderscore ).
\section{Páusia}
\begin{itemize}
\item {Grp. gram.:f.}
\end{itemize}
\begin{itemize}
\item {Proveniência:(Lat. \textunderscore pausia\textunderscore )}
\end{itemize}
Variedade de azeitona, conhecida dos antigos, talvez a mesma que chamamos \textunderscore verdeal\textunderscore .
\section{Pausimenia}
\begin{itemize}
\item {Grp. gram.:f.}
\end{itemize}
\begin{itemize}
\item {Proveniência:(Do gr. \textunderscore pausis\textunderscore  + \textunderscore men\textunderscore )}
\end{itemize}
Interrupção do mênstruo; menopausa.
\section{Paus-mandados}
\begin{itemize}
\item {Grp. gram.:m. pl.}
\end{itemize}
Espécie de jôgo popular, também conhecido por \textunderscore arquinhos\textunderscore .
\section{Pau-sujo}
\begin{itemize}
\item {Grp. gram.:m.}
\end{itemize}
Planta urticácea da Índia portuguesa, mais conhecida por \textunderscore pau-de-merda\textunderscore , (\textunderscore bosia trinervia\textunderscore , Roxb.).
\section{Pauta}
\begin{itemize}
\item {Grp. gram.:f.}
\end{itemize}
Papel com traços parallelos e equidistantes, sôbre o qual se assenta um papel translúcido, para nêste se escrever horizontalmente.
Conjunto das cinco linhas parallelas, em que se escreve a música.
Lista; relação.
Tarifa: \textunderscore a pauta das alfândegas\textunderscore .
\section{Pautação}
\begin{itemize}
\item {Grp. gram.:f.}
\end{itemize}
Acto de pautar.
\section{Pautado}
\begin{itemize}
\item {Grp. gram.:adj.}
\end{itemize}
\begin{itemize}
\item {Utilização:Fig.}
\end{itemize}
\begin{itemize}
\item {Proveniência:(De \textunderscore pautar\textunderscore )}
\end{itemize}
Traçado com riscos parallelos; relacionado; posto em rol.
Moderado.
Methódico.
\section{Pautador}
\begin{itemize}
\item {Grp. gram.:m.}
\end{itemize}
Aquelle que pauta.
\section{Pautal}
\begin{itemize}
\item {Grp. gram.:adj.}
\end{itemize}
Relativo á pauta.
Consignado na pauta, especialmente na pauta alfandegária: \textunderscore direitos pautaes\textunderscore .
\section{Pautar}
\begin{itemize}
\item {Grp. gram.:v. t.}
\end{itemize}
\begin{itemize}
\item {Utilização:Fig.}
\end{itemize}
Riscar á maneira de pauta.
Dirigir.
Relacionar.
Adaptar.
Ajustar.
Pôr em pauta.
Tornar moderado, methódico.
\section{Pautas}
\begin{itemize}
\item {Grp. gram.:f. pl.}
\end{itemize}
\begin{itemize}
\item {Utilização:Bras}
\end{itemize}
\begin{itemize}
\item {Utilização:pop.}
\end{itemize}
O mesmo que \textunderscore pacto\textunderscore :«\textunderscore tenho pautas com o demónio\textunderscore ». Juv. Galeno, \textunderscore Lendas\textunderscore .
\section{Pauteação}
\begin{itemize}
\item {Grp. gram.:f.}
\end{itemize}
Acto de pautear.
\section{Pautear}
\begin{itemize}
\item {Grp. gram.:v. i.}
\end{itemize}
\begin{itemize}
\item {Utilização:Bras}
\end{itemize}
Entreter-se, conversando; conversar futilmente.
(Relaciona-se com \textunderscore tautear\textunderscore ?)
\section{Pau-terra}
\begin{itemize}
\item {Grp. gram.:m.}
\end{itemize}
Fruto brasileiro, de suco preto, que se applica contra as frieiras.
\section{Pau-terras-grandes}
\begin{itemize}
\item {Grp. gram.:m.}
\end{itemize}
Planta, (\textunderscore gualea grandiflora\textunderscore ).
\section{Pau-terras-pequenas}
\begin{itemize}
\item {Grp. gram.:m.}
\end{itemize}
Variedade de pau-terras-grandes.
\section{Pauto}
\begin{itemize}
\item {Grp. gram.:m.}
\end{itemize}
\begin{itemize}
\item {Utilização:Ant.}
\end{itemize}
O mesmo que \textunderscore pacto\textunderscore . Cf. Usque, 23 v.^o.
\section{Pau-triste}
\begin{itemize}
\item {Grp. gram.:m.}
\end{itemize}
Árvore de Timor.
\section{Pauxianas}
\begin{itemize}
\item {Grp. gram.:m. pl.}
\end{itemize}
\begin{itemize}
\item {Utilização:Bras}
\end{itemize}
Tríbo de aborígenes do Pará.
\section{Pauzeiro}
\begin{itemize}
\item {Grp. gram.:m.}
\end{itemize}
\begin{itemize}
\item {Utilização:Prov.}
\end{itemize}
\begin{itemize}
\item {Utilização:trasm.}
\end{itemize}
\begin{itemize}
\item {Proveniência:(De \textunderscore pau\textunderscore )}
\end{itemize}
Carpinteiro, que prepara madeira para tamancos. Cf. \textunderscore Inquér. Industr.\textunderscore , p. II, l. III, 68.
\section{Pauzinho}
\begin{itemize}
\item {Grp. gram.:m.}
\end{itemize}
\begin{itemize}
\item {Grp. gram.:Pl.}
\end{itemize}
\begin{itemize}
\item {Utilização:Fam.}
\end{itemize}
Pequeno pau.
Mexerico, intriga.
\textunderscore Mexer\textunderscore  ou \textunderscore tocar os pauzinhos\textunderscore , intrigar, enredar.
Empregar os meios necessários para se resolver um negócio.
Sêr, ás occultas, o agente ou causa de certos resultados ou da solução de certas pretensões.
\section{Pava}
\begin{itemize}
\item {Grp. gram.:f.}
\end{itemize}
\begin{itemize}
\item {Utilização:Ant.}
\end{itemize}
Cêsto, que servia para a venda do arroz em Bengala.
\section{Pavana}
\begin{itemize}
\item {Grp. gram.:f.}
\end{itemize}
\begin{itemize}
\item {Utilização:Chul.}
\end{itemize}
\begin{itemize}
\item {Utilização:Bras}
\end{itemize}
Dança de sala, muito usada, nos séculos XVI e XVII, de compasso binário e andamento vagaroso.
Música, que acompanhava essa dança.
Descompostura.
Sova, tunda: \textunderscore tocou-lhe a pavana\textunderscore .
O mesmo que \textunderscore palmatória\textunderscore .
(Contr. de \textunderscore padovana\textunderscore )
\section{Pavão}
\begin{itemize}
\item {Grp. gram.:m.}
\end{itemize}
\begin{itemize}
\item {Proveniência:(Lat. \textunderscore pavo\textunderscore )}
\end{itemize}
Grande ave, da fam. das gallináceas.
\section{Pavão-bode}
\begin{itemize}
\item {Grp. gram.:m.}
\end{itemize}
Pavão de Bengala, (\textunderscore tragopan satyrus\textunderscore ), que tem por cima dos olhos uma espécie de pequenos chavelhos.
\section{Paveia}
\begin{itemize}
\item {Grp. gram.:f.}
\end{itemize}
Mólho de palha ou de feno.
Cada um dos montículos de mato roçado, que os roçadores vão formando, para depois se encarrarem facilmente, quando não são para alli mesmo se reduzirem a cinza, que servirá de adubo ao terreno, para a sementeira de trigo ou centeio.
\section{Pavês}
\begin{itemize}
\item {Grp. gram.:m.}
\end{itemize}
\begin{itemize}
\item {Proveniência:(Do b. lat. \textunderscore pavensis\textunderscore )}
\end{itemize}
Escudo grande.
Armação de madeira, para resguardo da tripulação de um navio.
\section{Pavesada}
\begin{itemize}
\item {Grp. gram.:f.}
\end{itemize}
Resguardo feito de paveses.
(Fem. de \textunderscore pavesado\textunderscore )
\section{Pavesado}
\begin{itemize}
\item {Grp. gram.:adj.}
\end{itemize}
Guarnecido de paveses.
\section{Pavesadura}
\begin{itemize}
\item {Grp. gram.:f.}
\end{itemize}
\begin{itemize}
\item {Proveniência:(De \textunderscore pavesar\textunderscore )}
\end{itemize}
Guarnição de paveses.
Pavês. Cf. Filinto, \textunderscore D. Man.\textunderscore , III, 403.
\section{Pavesar}
\begin{itemize}
\item {Grp. gram.:v. t.}
\end{itemize}
O mesmo que \textunderscore empavesar\textunderscore .
\section{Pavia}
\begin{itemize}
\item {Grp. gram.:m.}
\end{itemize}
\begin{itemize}
\item {Utilização:Prov.}
\end{itemize}
\begin{itemize}
\item {Utilização:trasm.}
\end{itemize}
Casta de pêssegos. (Colhido em Boticas)
\section{Pávido}
\begin{itemize}
\item {Grp. gram.:adj.}
\end{itemize}
\begin{itemize}
\item {Proveniência:(Lat. \textunderscore pavidus\textunderscore )}
\end{itemize}
Que tem pavôr; assustado; assombrado.
\section{Pavieira}
\begin{itemize}
\item {Grp. gram.:f.}
\end{itemize}
O mesmo que \textunderscore padieira\textunderscore .
\section{Pavieira}
\begin{itemize}
\item {Grp. gram.:f.}
\end{itemize}
Árvore, que dá pavias.
\section{Pavilhão}
\begin{itemize}
\item {Grp. gram.:m.}
\end{itemize}
\begin{itemize}
\item {Utilização:Anat.}
\end{itemize}
\begin{itemize}
\item {Proveniência:(Lat. \textunderscore papilio\textunderscore )}
\end{itemize}
Ligeira construcção de madeira, geralmente portátil.
Barraca; tenda.
Espécie de alpendre, annexo á parte principal de um edifício.
Parte exterior do canal auditivo.
Extremidade larga de alguns instrumentos de sôpro.
Bandeira.
Symbolo marítimo de uma nacionalidade.
Poder marítimo de uma nação.
\section{Pavimentação}
\begin{itemize}
\item {Grp. gram.:f.}
\end{itemize}
Acto ou effeito de pavimentar.
\section{Pavimentar}
\begin{itemize}
\item {Grp. gram.:v. t.}
\end{itemize}
Fazer pavimento em.
Construír com pavimentos. Cf. Arn. Gama, \textunderscore Motim\textunderscore , 57.
\section{Pavimento}
\begin{itemize}
\item {Grp. gram.:m.}
\end{itemize}
\begin{itemize}
\item {Proveniência:(Lat. \textunderscore pavimentum\textunderscore )}
\end{itemize}
Chão; sobrado.
Cada um dos andares de um edifício.
\section{Pavio}
\begin{itemize}
\item {Grp. gram.:m.}
\end{itemize}
Torcida.
Rôlo de cera, que envolve uma torcida.
(Cp. cast. \textunderscore pábilo\textunderscore )
\section{Pávio}
\begin{itemize}
\item {Grp. gram.:m.}
\end{itemize}
\begin{itemize}
\item {Proveniência:(De \textunderscore Pávia\textunderscore , n. p.)}
\end{itemize}
Casta de pêssego, de polpa adherente ao caroço.
\section{Paviola}
\begin{itemize}
\item {Grp. gram.:f.}
\end{itemize}
\begin{itemize}
\item {Utilização:Prov.}
\end{itemize}
O mesmo que \textunderscore padiola\textunderscore .
\section{Pavôa}
\begin{itemize}
\item {Grp. gram.:f.}
\end{itemize}
Fêmea do pavão.
\section{Pavô}
\begin{itemize}
\item {Grp. gram.:m.}
\end{itemize}
Ave brasileira, o mesmo que \textunderscore paô\textunderscore .
\section{Pavonaço}
\begin{itemize}
\item {Grp. gram.:adj.}
\end{itemize}
\begin{itemize}
\item {Utilização:P. us.}
\end{itemize}
Que tem côr de violeta. Cf. Vieira, I, 114; Nunes do Lião, \textunderscore Ortograf.\textunderscore , 87.
(Cp. \textunderscore pavonado\textunderscore )
\section{Pavonada}
\begin{itemize}
\item {Grp. gram.:f.}
\end{itemize}
\begin{itemize}
\item {Utilização:Fig.}
\end{itemize}
\begin{itemize}
\item {Proveniência:(Do lat. \textunderscore pavo\textunderscore , \textunderscore pavonis\textunderscore )}
\end{itemize}
Acto de o pavão formar leque com a cauda.
Jactância, vaidade.
\section{Pavonado}
\begin{itemize}
\item {Grp. gram.:adj.}
\end{itemize}
\begin{itemize}
\item {Utilização:Des.}
\end{itemize}
\begin{itemize}
\item {Proveniência:(Do lat. \textunderscore pavo\textunderscore )}
\end{itemize}
Que tem côres de pavão:«\textunderscore ...horizontes pavonados...\textunderscore »Filinto, XIV, 220.
\section{Pavoncinho}
\begin{itemize}
\item {Grp. gram.:m.}
\end{itemize}
O mesmo que \textunderscore pavoncino\textunderscore .
\section{Pavoncino}
\begin{itemize}
\item {Grp. gram.:m.}
\end{itemize}
\begin{itemize}
\item {Proveniência:(De \textunderscore pavão\textunderscore )}
\end{itemize}
Ave pernalta, o mesmo que \textunderscore abibe\textunderscore .
\section{Pavonear}
\begin{itemize}
\item {Grp. gram.:v. t.}
\end{itemize}
\begin{itemize}
\item {Grp. gram.:V. p.}
\end{itemize}
\begin{itemize}
\item {Proveniência:(Do lat. \textunderscore pavo\textunderscore , \textunderscore pavonis\textunderscore )}
\end{itemize}
Ornar garridamente.
Ostentar; exhibir com vaidade.
Ensoberbecer-se; ufanar-se.
\section{Pavónia}
\begin{itemize}
\item {Grp. gram.:f.}
\end{itemize}
\begin{itemize}
\item {Proveniência:(De \textunderscore Pavon\textunderscore , n. p.)}
\end{itemize}
Gênero de plantas malváceas.
\section{Pavor}
\begin{itemize}
\item {Grp. gram.:m.}
\end{itemize}
\begin{itemize}
\item {Proveniência:(Lat. \textunderscore pavor\textunderscore )}
\end{itemize}
Grande susto; terror.
\section{Pavorear}
\begin{itemize}
\item {Grp. gram.:v. t.}
\end{itemize}
\begin{itemize}
\item {Utilização:Des.}
\end{itemize}
O mesmo que \textunderscore apavorar\textunderscore . Cf. \textunderscore Viriato Trág.\textunderscore , III, 104.
\section{Pavorosa}
\begin{itemize}
\item {Grp. gram.:f.}
\end{itemize}
Notícia que apavora; boato de revolta.
(Fem. de \textunderscore pavoroso\textunderscore )
\section{Pavorosamente}
\begin{itemize}
\item {Grp. gram.:adv.}
\end{itemize}
De modo pavoroso; com pavor; com terror; causando pavor.
\section{Pavoroso}
\begin{itemize}
\item {Grp. gram.:adj.}
\end{itemize}
Que infunde pavor; medonho; horroroso.
\section{Paxá}
\begin{itemize}
\item {Grp. gram.:m.}
\end{itemize}
Título dos governadores de províncias turcas.
(Contr. de padixá. V. \textunderscore padixá\textunderscore )
\section{Paxaxo}
\begin{itemize}
\item {Utilização:Bras}
\end{itemize}
\begin{itemize}
\item {Utilização:Gír. de ciganos.}
\end{itemize}
Pé largo.
\section{Paxiúba}
\begin{itemize}
\item {Grp. gram.:f.}
\end{itemize}
\begin{itemize}
\item {Utilização:Bras. do N}
\end{itemize}
Espécie de palmeira.
(Do tupi)
\section{Paxtónia}
\begin{itemize}
\item {Grp. gram.:f.}
\end{itemize}
\begin{itemize}
\item {Proveniência:(De \textunderscore Paxton\textunderscore , n. p.)}
\end{itemize}
Gênero de orchídeas.
\section{Paz}
\begin{itemize}
\item {Grp. gram.:f.}
\end{itemize}
\begin{itemize}
\item {Grp. gram.:M.  e  f.}
\end{itemize}
\begin{itemize}
\item {Utilização:Fam.}
\end{itemize}
\begin{itemize}
\item {Proveniência:(Lat. \textunderscore pax\textunderscore )}
\end{itemize}
Relações tranquillas de um Estado ou de uma nação.
Tranquillidade pública.
Tranquillidade, sossêgo, descanso.
Silêncio.
Paz de alma, pessôa indolente, inerte ou pacífica.
\section{Pàzada}
\begin{itemize}
\item {Grp. gram.:f.}
\end{itemize}
O que se póde conter numa pá.
Pancada com pá; pancada.
\section{Pazão}
\begin{itemize}
\item {Grp. gram.:m.}
\end{itemize}
Espécie de antílope indiano, (\textunderscore orix\textunderscore ).
\section{Pàzear}
\begin{itemize}
\item {Grp. gram.:V. i.}
\end{itemize}
Estabelecer harmonia ou paz.
Jogar á paz, para desempate.
\section{Pazenda}
\begin{itemize}
\item {Grp. gram.:m.}
\end{itemize}
Grupo de línguas do ramo irânico.
\section{Pé}
\begin{itemize}
\item {Grp. gram.:m.}
\end{itemize}
\begin{itemize}
\item {Utilização:Prov.}
\end{itemize}
\begin{itemize}
\item {Utilização:minh.}
\end{itemize}
\begin{itemize}
\item {Utilização:Carp.}
\end{itemize}
\begin{itemize}
\item {Utilização:Ant.}
\end{itemize}
\begin{itemize}
\item {Grp. gram.:Loc. adv.}
\end{itemize}
\begin{itemize}
\item {Grp. gram.:Loc. adv.}
\end{itemize}
\begin{itemize}
\item {Grp. gram.:Loc. adv.}
\end{itemize}
\begin{itemize}
\item {Utilização:Ant.}
\end{itemize}
\begin{itemize}
\item {Grp. gram.:Loc.}
\end{itemize}
\begin{itemize}
\item {Utilização:Loc. da Bairrada.}
\end{itemize}
\begin{itemize}
\item {Grp. gram.:Loc.}
\end{itemize}
\begin{itemize}
\item {Utilização:port}
\end{itemize}
\begin{itemize}
\item {Utilização:Loc. da África}
\end{itemize}
\begin{itemize}
\item {Utilização:Bras. de Minas}
\end{itemize}
\begin{itemize}
\item {Utilização:Constr.}
\end{itemize}
\begin{itemize}
\item {Grp. gram.:Loc.}
\end{itemize}
\begin{itemize}
\item {Utilização:fig.}
\end{itemize}
\begin{itemize}
\item {Proveniência:(Do lat. \textunderscore pes\textunderscore , \textunderscore pedis\textunderscore )}
\end{itemize}
Parte inferior da perna, que assenta no solo e sustenta o corpo do homem e dos animaes.
Pata.
Medida de extensão, equivalente a 33 centimetros.
Chispe.
Parte inferior de vários objectos, ou pela qual se segura alguma coisa.
Pedestal.
Objecto, em que se firma outro.
Bôrras, lia.
Pretexto, motivo.
Parceiro que, nos jogos de vasa, deita a carta depois dos outros todos.
Situação de um negócio ou de uma empresa.
Base.
A mó inferior das azenhas.
O que fica das uvas, depois de se lhes espremer o primeiro suco.
Parte de um verso grego ou latino, que consta de duas até quatro sýllabas.
Peça da frente de um degrau.
O mesmo que \textunderscore glosa\textunderscore . Cf. \textunderscore Eufrosina\textunderscore , 245.
\textunderscore Pôr-se em pé\textunderscore , levantar-se, erguer-se.
\textunderscore Pé fresco\textunderscore , partidário da Patuleia.
\textunderscore De pé atrás\textunderscore , com reserva, com desconfiança.
\textunderscore Meter os pés na algibeira de\textunderscore , desfrutar.
\textunderscore Estar de pé\textunderscore , estar levantado, erguido.
\textunderscore A morte em pé\textunderscore , pessôa que anda muito doente ou muito magra.
\textunderscore Em pé de guerra\textunderscore , em estado de poder combater.
\textunderscore Ao pé da letra\textunderscore , literalmente, segundo palavras ou disposições expressas.
\textunderscore Do pé para a mão\textunderscore , logo, immediatamente.
\textunderscore Estar a pé\textunderscore , têr-se já erguido da cama.
* \textunderscore Pés de gallinha\textunderscore , rugas, junto ás commissuras das pálpebras.
\textunderscore Fazer pé de alferes\textunderscore , namorar ou galantear uma dama.
\textunderscore Pé ante pé\textunderscore , com passo miúdo e lento, como de quem espia ou não deseja sêr notado.
\textunderscore Em\textunderscore  ou \textunderscore com pés de lan\textunderscore , sorrateiramente.
\textunderscore Estar com os pés para a cova\textunderscore , estar moribundo ou com pouca vida.
\textunderscore A sete pés\textunderscore , ou \textunderscore a pés de cavallo\textunderscore , correndo muito: \textunderscore fugiu a sete pés\textunderscore .
\textunderscore Tomar pé\textunderscore , têr certeza.
\textunderscore Tomar pé\textunderscore , tocar o fundo da água com os pés.
\textunderscore Pegar pé\textunderscore , prestar vassallagem, obediência, (falando-se de indígenas).
\textunderscore Fazer pé atrás\textunderscore , recuar, para se firmar.
\textunderscore Passar o pé a alguém\textunderscore , abandoná-lo.
\textunderscore Agua de pé\textunderscore , a que rega um terreno, sem sêr preciso elevá-la artificialmente.
\textunderscore Rega de pé\textunderscore , a rega que se faz com agua de pé. Cf. \textunderscore Eufrosina\textunderscore , 265.
\textunderscore Abrir o pé\textunderscore , fugir.
\textunderscore Pé direito\textunderscore , altura, do pavimento ao tecto. Cf. \textunderscore Decreto\textunderscore  de 22-VII-905.
\textunderscore Apertar o pé\textunderscore , apressar o passo:«\textunderscore fê-la apertar o pé deante de si\textunderscore ». Camillo, \textunderscore Vinte Hor. de Lit.\textunderscore , 230.
\section{Pẽa}
\textunderscore f.\textunderscore  (e der.)
(Fórma ant. de \textunderscore pena\textunderscore , etc.)
\section{Peaça}
\begin{itemize}
\item {Grp. gram.:f.}
\end{itemize}
\begin{itemize}
\item {Proveniência:(De \textunderscore peia\textunderscore )}
\end{itemize}
Correia, com que se prende o boi á canga pelos paus; corneira.
\section{Peadoiro}
\begin{itemize}
\item {Grp. gram.:m.}
\end{itemize}
\begin{itemize}
\item {Utilização:Bras. do N}
\end{itemize}
Lugar onde se peiam as cavalgaduras.
\section{Pẽadoiro}
\begin{itemize}
\item {Grp. gram.:m.}
\end{itemize}
\begin{itemize}
\item {Utilização:Ant.}
\end{itemize}
\begin{itemize}
\item {Proveniência:(De \textunderscore pẽar\textunderscore )}
\end{itemize}
Digno de pena, de castigo.
\section{Peador}
\begin{itemize}
\item {Grp. gram.:m.}
\end{itemize}
\begin{itemize}
\item {Utilização:Bras do N}
\end{itemize}
\begin{itemize}
\item {Proveniência:(De \textunderscore pear\textunderscore )}
\end{itemize}
Lugar, onde se peiam as cavalgaduras, deixando-as a pastar.
\section{Peadouro}
\begin{itemize}
\item {Grp. gram.:m.}
\end{itemize}
\begin{itemize}
\item {Utilização:Bras. do N}
\end{itemize}
Lugar onde se peiam as cavalgaduras.
\section{Pẽadouro}
\begin{itemize}
\item {Grp. gram.:m.}
\end{itemize}
\begin{itemize}
\item {Utilização:Ant.}
\end{itemize}
\begin{itemize}
\item {Proveniência:(De \textunderscore pẽar\textunderscore )}
\end{itemize}
Digno de pena, de castigo.
\section{Peães}
\begin{itemize}
\item {Grp. gram.:m. pl.}
\end{itemize}
\begin{itemize}
\item {Utilização:Ant.}
\end{itemize}
\begin{itemize}
\item {Proveniência:(De \textunderscore pé\textunderscore )}
\end{itemize}
Os que andam a pé; peões. Cf. \textunderscore Tombo do Estado da Índia\textunderscore , 19 e 44; Filinto, \textunderscore D. Man.\textunderscore , II, 117.
\section{Peageiro}
\begin{itemize}
\item {Grp. gram.:m.}
\end{itemize}
\begin{itemize}
\item {Utilização:Ant.}
\end{itemize}
Cobrador de peagens.
\section{Peagem}
\begin{itemize}
\item {Grp. gram.:f.}
\end{itemize}
\begin{itemize}
\item {Utilização:Ant.}
\end{itemize}
\begin{itemize}
\item {Proveniência:(Do fr. \textunderscore péage\textunderscore )}
\end{itemize}
Imposto, que se pagava pela passagem por uma ponte, etc.
Portagem.
A quarta parte dos preços das tarifas de caminhos de ferro, correspondente á remuneração do capital.
\section{Pé-agudo}
\begin{itemize}
\item {Grp. gram.:m.}
\end{itemize}
Casta de uva preta, na região do Doiro, o mesmo que \textunderscore pardaínha\textunderscore .
\section{Peal}
\begin{itemize}
\item {Grp. gram.:m.}
\end{itemize}
\begin{itemize}
\item {Utilização:Prov.}
\end{itemize}
\begin{itemize}
\item {Utilização:trasm.}
\end{itemize}
Presilha, que liga á palma do pé a meia, que só cobre a perna, desde o joêlho ao tornozelo: \textunderscore meias de peal\textunderscore . Cf. Deusdado, \textunderscore Escorços\textunderscore , 151.
\section{Pealador}
\begin{itemize}
\item {Grp. gram.:m.}
\end{itemize}
Aquelle que peala.
\section{Pealar}
\begin{itemize}
\item {Grp. gram.:v. t.}
\end{itemize}
\begin{itemize}
\item {Utilização:Bras}
\end{itemize}
\begin{itemize}
\item {Utilização:Fig.}
\end{itemize}
Segurar com pealo.
Illudir.
\section{Pealhas}
\begin{itemize}
\item {Grp. gram.:f. pl.}
\end{itemize}
\begin{itemize}
\item {Utilização:Prov.}
\end{itemize}
\begin{itemize}
\item {Proveniência:(De \textunderscore pé\textunderscore )}
\end{itemize}
O mesmo que [[peúgas|peúga]].
\section{Pealhos}
\begin{itemize}
\item {Grp. gram.:m. pl.}
\end{itemize}
O mesmo que \textunderscore pealhas\textunderscore .
\section{Pealo}
\begin{itemize}
\item {Grp. gram.:m.}
\end{itemize}
\begin{itemize}
\item {Utilização:Bras}
\end{itemize}
\begin{itemize}
\item {Proveniência:(De \textunderscore pear\textunderscore )}
\end{itemize}
Laço, com que se prendem os cavallos pelas mãos, quando elles vão correndo.
Acto de arremessar o laço.
\section{Peança}
\begin{itemize}
\item {Grp. gram.:m.}
\end{itemize}
\begin{itemize}
\item {Utilização:T. de Turquel}
\end{itemize}
Homem acanhado ou inhábil.
\section{Peanha}
\begin{itemize}
\item {Grp. gram.:f.}
\end{itemize}
\begin{itemize}
\item {Proveniência:(Do lat. \textunderscore pedanea\textunderscore )}
\end{itemize}
Pequeno pedestal, que sustenta uma imagem, uma cruz, etc.
Peça do tear, em que o tecelão assenta os pés, para fazer subir ou descer os liços que cruzam os fios.
\section{Peanho}
\begin{itemize}
\item {Grp. gram.:m.}
\end{itemize}
\begin{itemize}
\item {Proveniência:(Do lat. \textunderscore pedaneus\textunderscore )}
\end{itemize}
Parte inferior de um navio.
\section{Peão}
\begin{itemize}
\item {Grp. gram.:m.}
\end{itemize}
\begin{itemize}
\item {Utilização:Ext.}
\end{itemize}
\begin{itemize}
\item {Utilização:Bras}
\end{itemize}
\begin{itemize}
\item {Utilização:Náut.}
\end{itemize}
\begin{itemize}
\item {Utilização:T. da Bairrada}
\end{itemize}
\begin{itemize}
\item {Utilização:Prov.}
\end{itemize}
\begin{itemize}
\item {Utilização:trasm.}
\end{itemize}
\begin{itemize}
\item {Proveniência:(Do lat. \textunderscore pedaneus\textunderscore )}
\end{itemize}
Indivíduo, que anda a pé.
Soldado de infantaria.
Cada uma das pequenas peças de xadrez, que se collocam na frente, e são as primeiras a avançar.
Plebeu.
Homem assalariado para trabalhos do campo.
Peça de ferro num mastro, para encaixe da vêrga do traquete ou da vela grande.
O mesmo que \textunderscore moirão\textunderscore .
O mesmo que \textunderscore canamão\textunderscore .
\section{Peão}
\begin{itemize}
\item {Grp. gram.:m.}
\end{itemize}
\begin{itemize}
\item {Utilização:Bras}
\end{itemize}
\begin{itemize}
\item {Proveniência:(De \textunderscore pear\textunderscore )}
\end{itemize}
Homem que, montado a cavallo, agarra bois a laço.
\section{Pear}
\begin{itemize}
\item {Grp. gram.:v. t.}
\end{itemize}
\begin{itemize}
\item {Utilização:Fig.}
\end{itemize}
Lançar peias a; prender com peia.
Embaraçar; impedir.
\section{Pẽar}
\begin{itemize}
\item {Grp. gram.:v. t.}
\end{itemize}
\begin{itemize}
\item {Utilização:Ant.}
\end{itemize}
\begin{itemize}
\item {Proveniência:(De \textunderscore pẽa\textunderscore )}
\end{itemize}
Applicar pena a; castigar, punir.
\section{Pearça}
\begin{itemize}
\item {Grp. gram.:f.}
\end{itemize}
\begin{itemize}
\item {Utilização:Prov.}
\end{itemize}
\begin{itemize}
\item {Utilização:alent.}
\end{itemize}
O mesmo que \textunderscore peaça\textunderscore .
\section{Peba}
\begin{itemize}
\item {Grp. gram.:f.}
\end{itemize}
\begin{itemize}
\item {Utilização:Bras}
\end{itemize}
Espécie de tatu, de cabeça achatada.
\section{Pebrina}
\begin{itemize}
\item {Grp. gram.:f.}
\end{itemize}
Espécie de doença epidemica dos bichos da seda.
\section{Peça}
\begin{itemize}
\item {Grp. gram.:f.}
\end{itemize}
\begin{itemize}
\item {Utilização:Fig.}
\end{itemize}
\begin{itemize}
\item {Grp. gram.:Loc. adv.}
\end{itemize}
\begin{itemize}
\item {Grp. gram.:Pl.}
\end{itemize}
\begin{itemize}
\item {Utilização:Heráld.}
\end{itemize}
\begin{itemize}
\item {Proveniência:(Do b. lat. \textunderscore petia\textunderscore )}
\end{itemize}
Pedaço, parte.
Cada um dos elementos, que constituem um todo.
Cada uma das pedras ou figuras, em jogos de tabuleiro.
Teia de pano.
Compartimento de uma casa.
Antiga moéda portuguesa de oiro.
Objecto de metal precioso.
Documento, que faz parte de um processo.
Composição dramática.
Qualquer artefacto.
Accessório de fôlha, com que os fulistas tiram o pêlo ás pelles.
Canhão.
Cada um dos compartimentos da marinha, nos quaes se produz o sal.
Ludíbrio, lôgro.
Pessôa maliciosa ou de má índole.
\textunderscore Pregar peça\textunderscore , fazer partida; causar contrariedade ou desgôsto.
Objectos, que se collocam no campo do escudo e que também são conhecidos por \textunderscore móveis\textunderscore  ou \textunderscore figuras\textunderscore .
\section{Pecadaço}
\begin{itemize}
\item {Grp. gram.:m.}
\end{itemize}
\begin{itemize}
\item {Utilização:Chul.}
\end{itemize}
O mesmo que \textunderscore pecadão\textunderscore .
\section{Pecadão}
\begin{itemize}
\item {Grp. gram.:m.}
\end{itemize}
Grande pecado.
\section{Pecadilho}
\begin{itemize}
\item {Grp. gram.:m.}
\end{itemize}
Pecado ou culpa leve.
Pequeno defeito.
\section{Pecado}
\begin{itemize}
\item {Grp. gram.:m.}
\end{itemize}
\begin{itemize}
\item {Utilização:Ext.}
\end{itemize}
\begin{itemize}
\item {Utilização:Prov.}
\end{itemize}
\begin{itemize}
\item {Proveniência:(Lat. \textunderscore peccatum\textunderscore )}
\end{itemize}
Transgressão de um preceito religioso.
Transgressão de qualquer preceito ou regra; culpa.
Maldade; vício.
O mesmo que \textunderscore demónio\textunderscore : \textunderscore são tentações do pecado\textunderscore .
\section{Pecador}
\begin{itemize}
\item {Grp. gram.:adj.}
\end{itemize}
\begin{itemize}
\item {Grp. gram.:M.}
\end{itemize}
\begin{itemize}
\item {Proveniência:(Lat. \textunderscore peccator\textunderscore )}
\end{itemize}
Que peca.
Propenso a pecar.
Aquele que peca.
Penitente.
Aquele que tem certos defeitos ou vícios.
\section{Pecadora}
\begin{itemize}
\item {Grp. gram.:f.}
\end{itemize}
Mulhér, que pecou.
Mulhér, que transgrediu os deveres da castidade.
(Fem. de \textunderscore peccador\textunderscore )
\section{Pecamente}
\begin{itemize}
\item {Grp. gram.:adv.}
\end{itemize}
De modo pêco.
\section{Pecaminosamente}
\begin{itemize}
\item {Grp. gram.:adv.}
\end{itemize}
De modo peccaminoso.
Com peccado; peccando.
\section{Pecaminoso}
\begin{itemize}
\item {Grp. gram.:adj.}
\end{itemize}
\begin{itemize}
\item {Proveniência:(Do lat. \textunderscore peccamen\textunderscore )}
\end{itemize}
Que é da natureza do pecado: \textunderscore actos pecaminosos\textunderscore .
Em que há muitos pecados: \textunderscore vida pecaminosa\textunderscore .
\section{Pecante}
\begin{itemize}
\item {Grp. gram.:m. ,  f.  e  adj.}
\end{itemize}
\begin{itemize}
\item {Grp. gram.:Loc. interj.}
\end{itemize}
\begin{itemize}
\item {Utilização:Prov.}
\end{itemize}
\begin{itemize}
\item {Proveniência:(Lat. \textunderscore peccans\textunderscore )}
\end{itemize}
Pessôa que peca por habito, que tem baldas.
Pecador.
\textunderscore Pecante de ti\textunderscore , coitado de ti. (Colhido na Bairrada)
\section{Pecar}
\begin{itemize}
\item {Grp. gram.:v. i.}
\end{itemize}
\begin{itemize}
\item {Utilização:Ext.}
\end{itemize}
\begin{itemize}
\item {Proveniência:(Lat. \textunderscore peccare\textunderscore )}
\end{itemize}
Transgredir preceito religioso.
Faltar a qualquer dever.
Errar.
Incorrer em censura; sêr digno de censura.
\section{Pecar}
\begin{itemize}
\item {Grp. gram.:v. t.}
\end{itemize}
Tornar-se pêco.
\section{Pecari}
\begin{itemize}
\item {Grp. gram.:m.}
\end{itemize}
Mammífero americano, muito semelhante a um pequeno porco.
\section{Pecatriz}
\begin{itemize}
\item {Grp. gram.:f.  e  adj.}
\end{itemize}
\begin{itemize}
\item {Utilização:Des.}
\end{itemize}
\begin{itemize}
\item {Proveniência:(Do lat. \textunderscore peccatrix\textunderscore )}
\end{itemize}
Mulhér pecadora.
\section{Pecável}
\begin{itemize}
\item {Grp. gram.:adj.}
\end{itemize}
Susceptível de pecar.
\section{Peccadaço}
\begin{itemize}
\item {Grp. gram.:m.}
\end{itemize}
\begin{itemize}
\item {Utilização:Chul.}
\end{itemize}
O mesmo que \textunderscore peccadão\textunderscore .
\section{Peccadão}
\begin{itemize}
\item {Grp. gram.:m.}
\end{itemize}
Grande peccado.
\section{Peccadilho}
\begin{itemize}
\item {Grp. gram.:m.}
\end{itemize}
Peccado ou culpa leve.
Pequeno defeito.
\section{Peccado}
\begin{itemize}
\item {Grp. gram.:m.}
\end{itemize}
\begin{itemize}
\item {Utilização:Ext.}
\end{itemize}
\begin{itemize}
\item {Utilização:Prov.}
\end{itemize}
\begin{itemize}
\item {Proveniência:(Lat. \textunderscore peccatum\textunderscore )}
\end{itemize}
Transgressão de um preceito religioso.
Transgressão de qualquer preceito ou regra; culpa.
Maldade; vício.
O mesmo que \textunderscore demónio\textunderscore : \textunderscore são tentações do peccado\textunderscore .
\section{Peccador}
\begin{itemize}
\item {Grp. gram.:adj.}
\end{itemize}
\begin{itemize}
\item {Grp. gram.:M.}
\end{itemize}
\begin{itemize}
\item {Proveniência:(Lat. \textunderscore peccator\textunderscore )}
\end{itemize}
Que pecca.
Propenso a peccar.
Aquelle que pecca.
Penitente.
Aquelle que tem certos defeitos ou vícios.
\section{Peccadora}
\begin{itemize}
\item {Grp. gram.:f.}
\end{itemize}
Mulhér, que peccou.
Mulhér, que transgrediu os deveres da castidade.
(Fem. de \textunderscore peccador\textunderscore )
\section{Peccaminosamente}
\begin{itemize}
\item {Grp. gram.:adv.}
\end{itemize}
De modo peccaminoso.
Com peccado; peccando.
\section{Peccaminoso}
\begin{itemize}
\item {Grp. gram.:adj.}
\end{itemize}
\begin{itemize}
\item {Proveniência:(Do lat. \textunderscore peccamen\textunderscore )}
\end{itemize}
Que é da natureza do peccado: \textunderscore actos peccaminosos\textunderscore .
Em que há muitos peccados: \textunderscore vida peccaminosa\textunderscore .
\section{Peccante}
\begin{itemize}
\item {Grp. gram.:m. ,  f.  e  adj.}
\end{itemize}
\begin{itemize}
\item {Grp. gram.:Loc. interj.}
\end{itemize}
\begin{itemize}
\item {Utilização:Prov.}
\end{itemize}
\begin{itemize}
\item {Proveniência:(Lat. \textunderscore peccans\textunderscore )}
\end{itemize}
Pessôa que pecca por habito, que tem baldas.
Peccador.
\textunderscore Peccante de ti\textunderscore , coitado de ti. (Colhido na Bairrada)
\section{Peccar}
\begin{itemize}
\item {Grp. gram.:v. i.}
\end{itemize}
\begin{itemize}
\item {Utilização:Ext.}
\end{itemize}
\begin{itemize}
\item {Proveniência:(Lat. \textunderscore peccare\textunderscore )}
\end{itemize}
Transgredir preceito religioso.
Faltar a qualquer dever.
Errar.
Incorrer em censura; sêr digno de censura.
\section{Peccatriz}
\begin{itemize}
\item {Grp. gram.:f.  e  adj.}
\end{itemize}
\begin{itemize}
\item {Utilização:Des.}
\end{itemize}
\begin{itemize}
\item {Proveniência:(Do lat. \textunderscore peccatrix\textunderscore )}
\end{itemize}
Mulhér peccadora.
\section{Peccável}
\begin{itemize}
\item {Grp. gram.:adj.}
\end{itemize}
Susceptível de peccar.
\section{Pecenho}
\begin{itemize}
\item {Grp. gram.:adj.}
\end{itemize}
O mesmo que \textunderscore pezenho\textunderscore .
\section{Pecha}
\begin{itemize}
\item {Grp. gram.:f.}
\end{itemize}
Defeito; mau costume; balda.
\section{Pechada}
\begin{itemize}
\item {Grp. gram.:f.}
\end{itemize}
\begin{itemize}
\item {Utilização:Bras}
\end{itemize}
Embate de dois cavalleiros, correndo em sentido opposto.
(Cast. \textunderscore pechada\textunderscore )
\section{Pechador}
\begin{itemize}
\item {Grp. gram.:m.}
\end{itemize}
\begin{itemize}
\item {Utilização:Bras. do S}
\end{itemize}
\begin{itemize}
\item {Utilização:Fig.}
\end{itemize}
\begin{itemize}
\item {Proveniência:(De \textunderscore pechar-se\textunderscore )}
\end{itemize}
Diz-se do cavallo que dá pechadas.
Homem, que tem o hábito de pedir dinheiro emprestado; pedinchão.
\section{Pechar-se}
\begin{itemize}
\item {Grp. gram.:v. p.}
\end{itemize}
\begin{itemize}
\item {Utilização:Bras. do S}
\end{itemize}
\begin{itemize}
\item {Proveniência:(De \textunderscore pechada\textunderscore )}
\end{itemize}
Chocar-se ou esbarrar um cavalleiro com outro.
\section{Pechelingues}
\begin{itemize}
\item {Grp. gram.:m. pl.}
\end{itemize}
Corsários, piratas:«\textunderscore ...quando em Zelanda não houver Pechelingues; quando em Argel não houver Turcos...\textunderscore »Vieira, XII, 120. Cf. \textunderscore Anat. Joc.\textunderscore , 319.
(Cp. \textunderscore pichelingue\textunderscore )
\section{Pechém}
\begin{itemize}
\item {Grp. gram.:m.}
\end{itemize}
O mesmo que \textunderscore pexão\textunderscore . Cf. Villarinho, \textunderscore Viticultura\textunderscore .
\section{Pechilingues}
\begin{itemize}
\item {Grp. gram.:m. pl.}
\end{itemize}
Corsários, piratas:«\textunderscore ...quando em Zelanda não houver Pechilingues; quando em Argel não houver Turcos...\textunderscore »Vieira, XII, 120. Cf. \textunderscore Anat. Joc.\textunderscore , 319.
(Cp. \textunderscore pichelingue\textunderscore )
\section{Pechincha}
\begin{itemize}
\item {Grp. gram.:f.}
\end{itemize}
\begin{itemize}
\item {Utilização:Pop.}
\end{itemize}
Grande conveniência.
Lucro inesperado ou immerecido.
Êxito vantajoso; vantagem.
(Talvez por \textunderscore pichincha\textunderscore  de \textunderscore pichincho\textunderscore )
\section{Pechinchar}
\begin{itemize}
\item {Grp. gram.:v. t.}
\end{itemize}
\begin{itemize}
\item {Utilização:Pop.}
\end{itemize}
\begin{itemize}
\item {Grp. gram.:V. i.}
\end{itemize}
\begin{itemize}
\item {Proveniência:(De \textunderscore pechincha\textunderscore )}
\end{itemize}
Ganhar inesperadamente ou immerecidamente.
Lucrar; alcançar.
Receber vantagens ou lucros inesperados ou immerecidos.
\section{Pechincheiro}
\begin{itemize}
\item {Grp. gram.:m.  e  adj.}
\end{itemize}
Aquelle que pechincha, ou que procura pechinchas.
\section{Pechinchinho}
\begin{itemize}
\item {Grp. gram.:adj.}
\end{itemize}
\begin{itemize}
\item {Utilização:Açor}
\end{itemize}
Muito pequenino.
Pequerruchinho.
\section{Pechincho}
\begin{itemize}
\item {Grp. gram.:m.  e  adj.}
\end{itemize}
\begin{itemize}
\item {Utilização:Açor}
\end{itemize}
O mesmo que \textunderscore pequerrucho\textunderscore .
\section{Pechingado}
\begin{itemize}
\item {Grp. gram.:adj.}
\end{itemize}
\begin{itemize}
\item {Utilização:T. de Turquel}
\end{itemize}
Um pouco perturbado por bebida alcoólica.
\section{Pechiringar}
\begin{itemize}
\item {Grp. gram.:v. i.}
\end{itemize}
\begin{itemize}
\item {Utilização:Bras. do N}
\end{itemize}
Dar qualquer coisa com mesquinhez.
Arriscar pouco dinheiro no jôgo.
\section{Pechisbeque}
\begin{itemize}
\item {Grp. gram.:m.}
\end{itemize}
\begin{itemize}
\item {Proveniência:(Do ingl. \textunderscore pinchbeck\textunderscore )}
\end{itemize}
Liga de cobre e zinco, imitando oiro.
\section{Pechorim}
\begin{itemize}
\item {Grp. gram.:m.}
\end{itemize}
\begin{itemize}
\item {Utilização:Bras}
\end{itemize}
Árvore silvestre.
Substância medicinal, extrahida dessa árvore.
\section{Pechoso}
\begin{itemize}
\item {Grp. gram.:adj.}
\end{itemize}
Que tem pecha; defeituoso.
Que acha pecha em tudo; caturra.
\section{Pechotada}
\begin{itemize}
\item {Grp. gram.:f.}
\end{itemize}
Acto de pechote.
Tolice ou disparate commetido ao jôgo.
\section{Pechote}
\begin{itemize}
\item {Grp. gram.:m.}
\end{itemize}
\begin{itemize}
\item {Utilização:Pop.}
\end{itemize}
\begin{itemize}
\item {Proveniência:(De \textunderscore pecha\textunderscore ?)}
\end{itemize}
Aquelle que joga mal.
Novato; ignorante.
\section{Pechurane}
\begin{itemize}
\item {Grp. gram.:m.}
\end{itemize}
O mesmo que \textunderscore pechurano\textunderscore .
\section{Pechurano}
\begin{itemize}
\item {Grp. gram.:m.}
\end{itemize}
\begin{itemize}
\item {Utilização:Miner.}
\end{itemize}
\begin{itemize}
\item {Proveniência:(De \textunderscore pech\textunderscore  al. + \textunderscore urano\textunderscore )}
\end{itemize}
Urano preto, que se encontra quási sempre associado a minérios de chumbo e prata, e do qual se extrai o polónio e depois o báryo, que contém o rádio.
\section{Pechyagra}
\begin{itemize}
\item {fónica:qui}
\end{itemize}
\begin{itemize}
\item {Grp. gram.:f.}
\end{itemize}
\begin{itemize}
\item {Utilização:Med.}
\end{itemize}
\begin{itemize}
\item {Proveniência:(Do gr. \textunderscore pekhus\textunderscore  + \textunderscore agra\textunderscore )}
\end{itemize}
Dôr de gota, que se fixou no cotovelo.
\section{Pecilochromático}
\begin{itemize}
\item {Grp. gram.:adj.}
\end{itemize}
\begin{itemize}
\item {Proveniência:(Do gr. \textunderscore poikilos\textunderscore , sarapintado, e \textunderscore khromatikos\textunderscore )}
\end{itemize}
Pintado de várias côres.
Variegado.
\section{Pecilocromático}
\begin{itemize}
\item {Grp. gram.:adj.}
\end{itemize}
\begin{itemize}
\item {Proveniência:(Do gr. \textunderscore poikilos\textunderscore , sarapintado, e \textunderscore khromatikos\textunderscore )}
\end{itemize}
Pintado de várias côres.
Variegado.
\section{Peciolação}
\begin{itemize}
\item {Grp. gram.:f.}
\end{itemize}
\begin{itemize}
\item {Utilização:Bot.}
\end{itemize}
Estado das flôres que têm pecíolos.
\section{Pecioláceo}
\begin{itemize}
\item {Grp. gram.:adj.}
\end{itemize}
\begin{itemize}
\item {Utilização:Bot.}
\end{itemize}
Diz-se dos botões, cujas escamas são formadas de pecíolos abortados, como succede em a nogueira.
\section{Peciolado}
\begin{itemize}
\item {Grp. gram.:adj.}
\end{itemize}
Que tem pecíolos.
\section{Peciolar}
\begin{itemize}
\item {Grp. gram.:adj.}
\end{itemize}
Relativo ao pecíolo; peciolado.
\section{Pecioleano}
\begin{itemize}
\item {Grp. gram.:adj.}
\end{itemize}
\begin{itemize}
\item {Utilização:Bot.}
\end{itemize}
Diz-se dos órgãos, provenientes da degeneração do pecíolo.
\section{Pecíolo}
\begin{itemize}
\item {Grp. gram.:m.}
\end{itemize}
\begin{itemize}
\item {Proveniência:(Lat. \textunderscore petiolus\textunderscore )}
\end{itemize}
Parte da fôlha, que prende o limbo ao tronco.
\section{Peciolular}
\begin{itemize}
\item {Grp. gram.:adj.}
\end{itemize}
\begin{itemize}
\item {Utilização:Bot.}
\end{itemize}
Diz-se das estípulas que nas flôres compostas se inserem sôbre os peciólulos.
\section{Peciólulo}
\begin{itemize}
\item {Grp. gram.:m.}
\end{itemize}
\begin{itemize}
\item {Utilização:Bot.}
\end{itemize}
Última ramificação de um pecíolo commum, nas flôres compostas.
Pequeno pecíolo, em que se sustenta um folíolo.
\section{Pêco}
\begin{itemize}
\item {Grp. gram.:m.}
\end{itemize}
\begin{itemize}
\item {Grp. gram.:Adj.}
\end{itemize}
\begin{itemize}
\item {Utilização:Fig.}
\end{itemize}
Doença dos vegetaes, que os faz definhar.
Definhamento.
Que definhou; que não chegou a medrar: \textunderscore frutos pecos\textunderscore .
Bronco, estúpido.
\section{Pêco}
\begin{itemize}
\item {Grp. gram.:adj.}
\end{itemize}
\begin{itemize}
\item {Utilização:Prov.}
\end{itemize}
\begin{itemize}
\item {Utilização:minh.}
\end{itemize}
Maçador.
Meticuloso.
Rabugento.
(Cp. \textunderscore pêto\textunderscore ^3)
\section{Pecoapá}
\begin{itemize}
\item {Grp. gram.:m.}
\end{itemize}
\begin{itemize}
\item {Utilização:Bras. do N}
\end{itemize}
Ave gallinácea.
\section{Peconha}
\begin{itemize}
\item {Grp. gram.:f.}
\end{itemize}
\begin{itemize}
\item {Utilização:Bras. do N}
\end{itemize}
\begin{itemize}
\item {Proveniência:(Do guar. \textunderscore pycōi\textunderscore )}
\end{itemize}
Ligas de embira, em que se metem os pés, para subir ás árvores sem galhos.
\section{Peçonha}
\begin{itemize}
\item {Grp. gram.:f.}
\end{itemize}
\begin{itemize}
\item {Utilização:Fig.}
\end{itemize}
\begin{itemize}
\item {Utilização:Pop.}
\end{itemize}
Veneno, segregado por alguns animaes.
Veneno.
Malícia, maldade.
Faíscas eléctricas.
(Relaciona-se com o lat. \textunderscore potio\textunderscore , \textunderscore potionis\textunderscore )
\section{Peçonhento}
\begin{itemize}
\item {Grp. gram.:adj.}
\end{itemize}
Que tem peçonha; envenenado.
\section{Pécora}
\begin{itemize}
\item {Grp. gram.:f.}
\end{itemize}
\begin{itemize}
\item {Utilização:Burl.}
\end{itemize}
\begin{itemize}
\item {Utilização:Prov.}
\end{itemize}
\begin{itemize}
\item {Utilização:trasm.}
\end{itemize}
\begin{itemize}
\item {Utilização:Des.}
\end{itemize}
Mulhér desprezível.
Rameira.
Rapariga leviana, que dá attenção a todos os galanteios.
Qualquer mulhér, (em sentido depreciativo). Cf. Filinto, VIII, 47.
(B. lat. \textunderscore pecora\textunderscore )
\section{Pecorear}
\begin{itemize}
\item {Grp. gram.:v. i.}
\end{itemize}
\begin{itemize}
\item {Utilização:Ant.}
\end{itemize}
\begin{itemize}
\item {Proveniência:(Do lat. \textunderscore pecus\textunderscore , \textunderscore pecoris\textunderscore )}
\end{itemize}
Passar a noite no campo, como o gado na malhada. Cf. \textunderscore Viriato Trág.\textunderscore , 696.
\section{Peçós}
\begin{itemize}
\item {Grp. gram.:m. pl.}
\end{itemize}
\begin{itemize}
\item {Utilização:Prov.}
\end{itemize}
\begin{itemize}
\item {Utilização:beir.}
\end{itemize}
\begin{itemize}
\item {Proveniência:(De \textunderscore peça\textunderscore ?)}
\end{itemize}
Fios da urdidura, que ficam sem trama no fim da teia, e de que se fazem torcidas.
\section{Pé-coxinho}
\begin{itemize}
\item {Grp. gram.:m.}
\end{itemize}
Espécie de jôgo de crianças.
Acto de caminhar com um só pé, suspendendo o outro.
\section{Pectar}
\begin{itemize}
\item {Grp. gram.:v. t.}
\end{itemize}
\begin{itemize}
\item {Utilização:Ant.}
\end{itemize}
\begin{itemize}
\item {Grp. gram.:V. i.}
\end{itemize}
O mesmo que \textunderscore peitar\textunderscore .
Pagar peita, tributo ou multa.
\section{Pectato}
\begin{itemize}
\item {Grp. gram.:m.}
\end{itemize}
\begin{itemize}
\item {Utilização:Chím.}
\end{itemize}
Designação genérica dos saes, formados pelo ácido péctico com as bases.
\section{Pécten}
\begin{itemize}
\item {Grp. gram.:m.}
\end{itemize}
\begin{itemize}
\item {Utilização:Anat.}
\end{itemize}
\begin{itemize}
\item {Proveniência:(Lat. \textunderscore pecten\textunderscore )}
\end{itemize}
O osso do púbis.
\section{Péctico}
\begin{itemize}
\item {Grp. gram.:adj.}
\end{itemize}
\begin{itemize}
\item {Proveniência:(De \textunderscore pectina\textunderscore )}
\end{itemize}
Diz-se de um ácido, produzido pela acção da potassa sôbre a pectina.
\section{Pectina}
\begin{itemize}
\item {Grp. gram.:f.}
\end{itemize}
\begin{itemize}
\item {Proveniência:(Do gr. \textunderscore pektos\textunderscore )}
\end{itemize}
Princípio especial, que se encontra num grande número de frutos.
\section{Pectíneo}
\begin{itemize}
\item {Grp. gram.:adj.}
\end{itemize}
Que tem fórma de pente.
Relativo ao púbis.
\section{Pectinibrânchio}
\begin{itemize}
\item {fónica:qui}
\end{itemize}
\begin{itemize}
\item {Grp. gram.:adj.}
\end{itemize}
\begin{itemize}
\item {Proveniência:(De \textunderscore pecten\textunderscore  lat. + \textunderscore brânchia\textunderscore )}
\end{itemize}
Que tem brânchias em fórma de pente, (falando-se do peixe).
\section{Pectinibrânquio}
\begin{itemize}
\item {Grp. gram.:adj.}
\end{itemize}
\begin{itemize}
\item {Proveniência:(De \textunderscore pecten\textunderscore  lat. + \textunderscore brânchia\textunderscore )}
\end{itemize}
Que tem brânquias em fórma de pente, (falando-se do peixe).
\section{Pectinicórneo}
\begin{itemize}
\item {Grp. gram.:adj.}
\end{itemize}
\begin{itemize}
\item {Utilização:Zool.}
\end{itemize}
\begin{itemize}
\item {Proveniência:(Do lat. \textunderscore pecten\textunderscore  + \textunderscore cornu\textunderscore )}
\end{itemize}
Que tem os cornos ou as antennas em fórma de pente.
\section{Pectinoso}
\begin{itemize}
\item {Grp. gram.:adj.}
\end{itemize}
\begin{itemize}
\item {Proveniência:(Do lat. \textunderscore pecten\textunderscore )}
\end{itemize}
Que tem fórma ou aspecto de pente. Cf. \textunderscore Tech. Rur.\textunderscore , 311 e 323.
\section{Pectófito}
\begin{itemize}
\item {Grp. gram.:m.}
\end{itemize}
Gênero de plantas umbelíferas.
\section{Pectóphito}
\begin{itemize}
\item {Grp. gram.:m.}
\end{itemize}
Gênero de plantas umbellíferas.
\section{Pectoraes}
\begin{itemize}
\item {Grp. gram.:m. Pl.}
\end{itemize}
\begin{itemize}
\item {Proveniência:(Lat. \textunderscore pectoralis\textunderscore )}
\end{itemize}
Divisão ou classe de peixes, também chamados thorácicos.
\section{Pectorais}
\begin{itemize}
\item {Grp. gram.:m. Pl.}
\end{itemize}
\begin{itemize}
\item {Proveniência:(Lat. \textunderscore pectoralis\textunderscore )}
\end{itemize}
Divisão ou classe de peixes, também chamados torácicos.
\section{Pectoral}
\begin{itemize}
\item {Grp. gram.:adj.}
\end{itemize}
O mesmo que \textunderscore peitoral\textunderscore .
\section{Pectoriloquía}
\begin{itemize}
\item {Grp. gram.:f.}
\end{itemize}
\begin{itemize}
\item {Proveniência:(Do lat. \textunderscore pectus\textunderscore , \textunderscore pectoris\textunderscore  + \textunderscore loquí\textunderscore )}
\end{itemize}
Estado de alguns doentes, cuja voz parece partir directamente do peito.
\section{Pectoríloquo}
\begin{itemize}
\item {Grp. gram.:adj.}
\end{itemize}
Que apresenta o phenómeno da pectoriloquia.
\section{Pectose}
\begin{itemize}
\item {Grp. gram.:f.}
\end{itemize}
\begin{itemize}
\item {Utilização:Chím.}
\end{itemize}
Princípio, que se extrái dos frutos verdes, da cenoira e do nabo, e cuja composição é ainda desconhecida.
(Cp. \textunderscore pectina\textunderscore )
\section{Pectósico}
\begin{itemize}
\item {Grp. gram.:adj.}
\end{itemize}
\begin{itemize}
\item {Proveniência:(De \textunderscore pectose\textunderscore )}
\end{itemize}
Diz-se de um ácido, que se fórma ao introduzir-se pectose numa dissolução de pectina, e que se precipita em estado gelatinoso.
\section{Pecuária}
\begin{itemize}
\item {Grp. gram.:f.}
\end{itemize}
Arte de criar e tratar gado.
(Fem. de \textunderscore pecuário\textunderscore )
\section{Pecuário}
\begin{itemize}
\item {Grp. gram.:adj.}
\end{itemize}
\begin{itemize}
\item {Grp. gram.:M.}
\end{itemize}
\begin{itemize}
\item {Proveniência:(Lat. \textunderscore pecuarius\textunderscore )}
\end{itemize}
Relativo a gados.
Criador ou tratador de gados. Cf. Castilho, \textunderscore Geórgicas\textunderscore , 169.
\section{Pecuínha}
\begin{itemize}
\item {Grp. gram.:f.}
\end{itemize}
O mesmo que \textunderscore picuínha\textunderscore :«\textunderscore ...pecuinhas com que os animos d'huma parte e doutra se assanhárão.\textunderscore »Filinto, \textunderscore D. Man.\textunderscore , II, 334.
\section{Peculador}
\begin{itemize}
\item {Grp. gram.:m.}
\end{itemize}
\begin{itemize}
\item {Proveniência:(Lat. \textunderscore peculator\textunderscore )}
\end{itemize}
Aquelle que commete peculato.
Aquelle que administra sem probidade os rendimentos do Estado ou que os desencaminha em proveito próprio. Cf. Garrett, \textunderscore Catão\textunderscore , 162.
\section{Peculatário}
\begin{itemize}
\item {Grp. gram.:adj.}
\end{itemize}
\begin{itemize}
\item {Proveniência:(Do lat. \textunderscore peculatus\textunderscore )}
\end{itemize}
Aquelle que pratica peculato; peculador.
\section{Peculato}
\begin{itemize}
\item {Grp. gram.:m.}
\end{itemize}
\begin{itemize}
\item {Proveniência:(Lat. \textunderscore peculatus\textunderscore )}
\end{itemize}
Furto de dinheiro ou rendimentos públicos por pessôa que os administra ou guarda.
\section{Peculiar}
\begin{itemize}
\item {Grp. gram.:adj.}
\end{itemize}
\begin{itemize}
\item {Proveniência:(Lat. \textunderscore peculiaris\textunderscore )}
\end{itemize}
Relativo a pecúlio.
Especial, próprio, particular.
\section{Peculiaridade}
\begin{itemize}
\item {Grp. gram.:f.}
\end{itemize}
Qualidade do que é peculiar.
\section{Peculiarmente}
\begin{itemize}
\item {Grp. gram.:adv.}
\end{itemize}
De modo peculiar; privativamente; especialmente.
\section{Pecúlio}
\begin{itemize}
\item {Grp. gram.:m.}
\end{itemize}
\begin{itemize}
\item {Proveniência:(Lat. \textunderscore peculium\textunderscore )}
\end{itemize}
Dinheiro, accumulado por trabalho ou economia.
Reserva de dinheiro.
Bens, património.
Conjunto de coisas, notícias ou apontamentos, relativos a certo assumpto ou especialidade.
Collecção valiosa.
Complexo de conhecimentos sôbre certos assumptos.
\section{Pecunia}
\begin{itemize}
\item {Grp. gram.:f.}
\end{itemize}
\begin{itemize}
\item {Utilização:Fam.}
\end{itemize}
\begin{itemize}
\item {Proveniência:(Lat. \textunderscore pecunia\textunderscore )}
\end{itemize}
O mesmo que \textunderscore dinheiro\textunderscore .
\section{Pecuniária}
\begin{itemize}
\item {Grp. gram.:f.}
\end{itemize}
\begin{itemize}
\item {Utilização:Ant.}
\end{itemize}
O mesmo que \textunderscore dinheiro\textunderscore .
(Fem. de \textunderscore pecuniário\textunderscore )
\section{Pecuniário}
\begin{itemize}
\item {Grp. gram.:adj.}
\end{itemize}
\begin{itemize}
\item {Proveniência:(Lat. \textunderscore pecuniarius\textunderscore )}
\end{itemize}
Relativo a dinheiro; representado por dinheiro.
\section{Pecunioso}
\begin{itemize}
\item {Grp. gram.:adj.}
\end{itemize}
\begin{itemize}
\item {Proveniência:(Lat. \textunderscore pecuniosus\textunderscore )}
\end{itemize}
Que tem muito dinheiro; opulento.
\section{Pedaça}
\begin{itemize}
\item {Grp. gram.:f.}
\end{itemize}
\begin{itemize}
\item {Utilização:Burl.}
\end{itemize}
Flexão feminina de \textunderscore pedaço\textunderscore , usada na expressão:«\textunderscore pedaça de asno.\textunderscore »Castilho, \textunderscore Méd. á Fôrça\textunderscore , 89.
\section{Pedaço}
\begin{itemize}
\item {Grp. gram.:m.}
\end{itemize}
\begin{itemize}
\item {Proveniência:(Do b. lat. \textunderscore pitatium\textunderscore )}
\end{itemize}
Parte de um todo, considerada separadamente.
Fragmento; porção.
Bocado, naco.
Pequeno espaço de tempo: \textunderscore saiu daqui há pedaço\textunderscore .
Trecho.
\section{Pedágio}
\begin{itemize}
\item {Grp. gram.:m.}
\end{itemize}
\begin{itemize}
\item {Utilização:bras}
\end{itemize}
\begin{itemize}
\item {Utilização:Ant.}
\end{itemize}
\begin{itemize}
\item {Proveniência:(Do lat. hyp. \textunderscore peduticum\textunderscore )}
\end{itemize}
Tributo de passagem por uma ponte; portagem. Cf. Pant. de Aveiro, \textunderscore Itiner.\textunderscore , 54 v.^o, (2.^a ed.).
\section{Pedagogia}
\begin{itemize}
\item {Grp. gram.:f.}
\end{itemize}
\begin{itemize}
\item {Proveniência:(Do lat. \textunderscore pedagogia\textunderscore )}
\end{itemize}
Arte da educação e do ensino; educação moral das crianças.
Modos de pedagogo.
\section{Pedagogicamente}
\begin{itemize}
\item {Grp. gram.:adv.}
\end{itemize}
De modo pedagógico.
Segundo os processos da pedagogia; á maneira de pedagogo.
\section{Pedagogice}
\begin{itemize}
\item {Grp. gram.:f.}
\end{itemize}
Mania ou presumpção de pedagogo.
\section{Pedagógico}
\begin{itemize}
\item {Grp. gram.:adj.}
\end{itemize}
Relativo á pedagogia.
\section{Pedagogismo}
\begin{itemize}
\item {Grp. gram.:m.}
\end{itemize}
Systema ou processos dos pedagogos.
\section{Pedagogista}
\begin{itemize}
\item {Grp. gram.:m.  e  f.}
\end{itemize}
Pessôa, que trata de pedagogia.
\section{Pedagogo}
\begin{itemize}
\item {fónica:góougô}
\end{itemize}
\begin{itemize}
\item {Grp. gram.:m.}
\end{itemize}
\begin{itemize}
\item {Utilização:Ext.}
\end{itemize}
\begin{itemize}
\item {Proveniência:(Lat. \textunderscore pedagogus\textunderscore )}
\end{itemize}
Escravo que, na antiguidade, acompanhava as crianças ás escolas.
Mestre de crianças.
Aquelle que exerce a pedagogia ou que se occupa dos méthodos de educar e ensinar.
Aquelle que se arroga o direito de censurar os outros.
Aquelle que alardeia erudição.
Pedante.
\section{Pedal}
\begin{itemize}
\item {Grp. gram.:m.}
\end{itemize}
\begin{itemize}
\item {Proveniência:(Lat. \textunderscore pedalis\textunderscore )}
\end{itemize}
Cada uma das duas teclas, ou ambas, que, na parte inferior dos pianos e órgãos, se movem com o pé.
Peça do velocípede, na qual se assenta o pé.
Peça da máquina de costura ou de outros apparelhos, na qual se assenta o pé, para imprimir movimento a êsses apparelhos.
\section{Pedalada}
\begin{itemize}
\item {Grp. gram.:f.}
\end{itemize}
Cada impulso, dado ao pedal.
\section{Pedalagem}
\begin{itemize}
\item {Grp. gram.:f.}
\end{itemize}
Acto de pedalar.
\section{Pedalar}
\begin{itemize}
\item {Grp. gram.:v. i.}
\end{itemize}
Movêr os pedaes.
\section{Pedaleiro}
\begin{itemize}
\item {Grp. gram.:m.}
\end{itemize}
\begin{itemize}
\item {Proveniência:(De \textunderscore pedal\textunderscore )}
\end{itemize}
Eixo grande das bicycletas.
\section{Pedália}
\begin{itemize}
\item {Grp. gram.:f.}
\end{itemize}
\begin{itemize}
\item {Proveniência:(De \textunderscore pedal\textunderscore )}
\end{itemize}
Gênero de plantas, cuja espécie typo cresce no Malabar.
\section{Pedaliforme}
\begin{itemize}
\item {Grp. gram.:adj.}
\end{itemize}
\begin{itemize}
\item {Utilização:Bot.}
\end{itemize}
\begin{itemize}
\item {Proveniência:(Do lat. \textunderscore pedalis\textunderscore  + \textunderscore forma\textunderscore )}
\end{itemize}
Diz-se das fôlhas, cujas nervuras não têm vasos.
\section{Pedalíneas}
\begin{itemize}
\item {Grp. gram.:f. pl.}
\end{itemize}
Família de plantas, criada por Brown, para collocar alguns gêneros, como a pedália.
\section{Pedalíneo}
\begin{itemize}
\item {Grp. gram.:adj.}
\end{itemize}
Relativo ou semelhante á pedália.
\section{Pedalinérveo}
\begin{itemize}
\item {Grp. gram.:adj.}
\end{itemize}
\begin{itemize}
\item {Utilização:Bot.}
\end{itemize}
Diz-se das fôlhas, em que a base do limbo lança duas nervuras principaes, muito divergentes, tendo cada uma, sôbre o lado interior, nervuras secundárias, parallelas entre si e perpendiculares ás principaes.
\section{Pedâneo}
\begin{itemize}
\item {Grp. gram.:adj.}
\end{itemize}
\begin{itemize}
\item {Proveniência:(Lat. \textunderscore pedaneus\textunderscore )}
\end{itemize}
Diz-se dos juízes que, nas localidades menos importantes, julgavam de pé.
\section{Pedantaria}
\begin{itemize}
\item {Grp. gram.:f.}
\end{itemize}
Qualidade daquelle ou daquillo que é pedante.
Pedantismo.
Ostentação de erudição.
\section{Pedante}
\begin{itemize}
\item {Grp. gram.:m.}
\end{itemize}
\begin{itemize}
\item {Grp. gram.:Adj.}
\end{itemize}
\begin{itemize}
\item {Proveniência:(It. \textunderscore pedante\textunderscore )}
\end{itemize}
Aquelle que faz ostentação de conhecimentos superiores aos que possue.
Parlapatão; impostor; charlatão.
Vaidoso no falar ou na maneira com que se apresenta; pretensioso:«\textunderscore ...casquilha dama de falar pedante...\textunderscore »Roussado, \textunderscore Roberto\textunderscore .
\section{Pedantear}
\begin{itemize}
\item {Grp. gram.:v. i.}
\end{itemize}
Alardear sciência quem a não possue; fazer-se pedante.
\section{Pedantescamente}
\begin{itemize}
\item {Grp. gram.:adv.}
\end{itemize}
De modo pedantesco.
Com pedantaria; á maneira de pedante.
\section{Pedantesco}
\begin{itemize}
\item {fónica:tês}
\end{itemize}
\begin{itemize}
\item {Grp. gram.:adj.}
\end{itemize}
Em que há pedantismo; affectado; próprio de pedante.
\section{Pedantice}
\begin{itemize}
\item {Grp. gram.:f.}
\end{itemize}
Acto ou dito de pedante; pedantismo. Cf. Garrett, \textunderscore Fábulas\textunderscore , 46.
\section{Pedantismo}
\begin{itemize}
\item {Grp. gram.:m.}
\end{itemize}
O mesmo que \textunderscore pedantaria\textunderscore .
Acto ou modos de pedante.
\section{Pedantocracia}
\begin{itemize}
\item {Grp. gram.:f.}
\end{itemize}
\begin{itemize}
\item {Utilização:Neol.}
\end{itemize}
\begin{itemize}
\item {Proveniência:(De \textunderscore pedante\textunderscore  + gr. \textunderscore kratein\textunderscore )}
\end{itemize}
Govêrno ou predomínio dos pedantes.
Influência do pedantismo ou das mediocridades ambiciosas.
\section{Pedantocrático}
\begin{itemize}
\item {Grp. gram.:adj.}
\end{itemize}
Relativo á pedantocracia. Cf. Th. Braga, \textunderscore Mod. Ideias\textunderscore , II, 13.
\section{Pedarchia}
\begin{itemize}
\item {fónica:qui}
\end{itemize}
\begin{itemize}
\item {Grp. gram.:f.}
\end{itemize}
\begin{itemize}
\item {Utilização:Irón.}
\end{itemize}
\begin{itemize}
\item {Proveniência:(Do gr. \textunderscore pais\textunderscore , \textunderscore paidos\textunderscore  + \textunderscore arkein\textunderscore )}
\end{itemize}
Govêrno de crianças.
\section{Pedárchico}
\begin{itemize}
\item {fónica:qui}
\end{itemize}
\begin{itemize}
\item {Grp. gram.:adj.}
\end{itemize}
Relativo á pedarchia.
\section{Pedária}
\begin{itemize}
\item {Grp. gram.:f.}
\end{itemize}
Gênero de insectos coleópteros pentâmeros.
\section{Pedarquia}
\begin{itemize}
\item {Grp. gram.:f.}
\end{itemize}
\begin{itemize}
\item {Utilização:Irón.}
\end{itemize}
\begin{itemize}
\item {Proveniência:(Do gr. \textunderscore pais\textunderscore , \textunderscore paidos\textunderscore  + \textunderscore arkein\textunderscore )}
\end{itemize}
Govêrno de crianças.
\section{Pedárquico}
\begin{itemize}
\item {Grp. gram.:adj.}
\end{itemize}
Relativo á pedarquia.
\section{Pedarthrocacia}
\begin{itemize}
\item {Grp. gram.:f.}
\end{itemize}
\begin{itemize}
\item {Utilização:Med.}
\end{itemize}
\begin{itemize}
\item {Proveniência:(Do gr. \textunderscore pais\textunderscore , \textunderscore paidos\textunderscore  + \textunderscore arthron\textunderscore  + \textunderscore kake\textunderscore )}
\end{itemize}
Doença articular das crianças.
\section{Pedartrocacia}
\begin{itemize}
\item {Grp. gram.:f.}
\end{itemize}
\begin{itemize}
\item {Utilização:Med.}
\end{itemize}
\begin{itemize}
\item {Proveniência:(Do gr. \textunderscore pais\textunderscore , \textunderscore paidos\textunderscore  + \textunderscore arthron\textunderscore  + \textunderscore kake\textunderscore )}
\end{itemize}
Doença articular das crianças.
\section{Pedatilobado}
\begin{itemize}
\item {Grp. gram.:adj.}
\end{itemize}
\begin{itemize}
\item {Utilização:Bot.}
\end{itemize}
Diz-se dos lóbulos, quando são apedados.
\section{Pedatipartido}
\begin{itemize}
\item {Grp. gram.:adj.}
\end{itemize}
\begin{itemize}
\item {Utilização:Bot.}
\end{itemize}
Dividido profundamente e cujas divisões são apedadas.
\section{Pedauca}
\begin{itemize}
\item {Grp. gram.:f.}
\end{itemize}
\begin{itemize}
\item {Proveniência:(Do lat. \textunderscore pes\textunderscore , \textunderscore pedis\textunderscore  + b. lat. \textunderscore auca\textunderscore )}
\end{itemize}
Imagem ou estátua de mulhér, com pés de pata, que se encontra nalguns monumentos da Idade-Média, e que se dizia representar a raínha Bertha, mãe de Carlos-Magno.
\section{Pé-de-alferes}
\begin{itemize}
\item {Grp. gram.:m.}
\end{itemize}
\begin{itemize}
\item {Utilização:Fam.}
\end{itemize}
O mesmo que \textunderscore namôro\textunderscore . Cf. Castilho, \textunderscore Fausto\textunderscore , 243.
\section{Pé-de-altar}
\begin{itemize}
\item {Grp. gram.:m.}
\end{itemize}
Rendimento, que os párochos auferem dos casamentos, enterros e baptizados.
\section{Pé-de-amigo}
\begin{itemize}
\item {Grp. gram.:m.}
\end{itemize}
\begin{itemize}
\item {Utilização:Prov.}
\end{itemize}
\begin{itemize}
\item {Utilização:trasm.}
\end{itemize}
Columna prismática de cantaria, a meio de uma quadra ou pavimento térreo, para sustentar os topes de duas traves, quando só uma não chega de parede a parede.
\section{Pé-de-boi}
\begin{itemize}
\item {Grp. gram.:m.}
\end{itemize}
\begin{itemize}
\item {Utilização:Bot.}
\end{itemize}
Homem, aferrado aos costumes antigos e desprezador de modas.
Variedade de maçan.
\section{Pé-de-burro}
\begin{itemize}
\item {Grp. gram.:m.}
\end{itemize}
Planta bulbosa, espécie de inhame silvestre. Cf. \textunderscore Diár.-do-Congresso\textunderscore , do Brasil, de 11-X-900.
Planta iridácea, também conhecida por \textunderscore açafrão bravo\textunderscore .
Nome de outra planta da mesma fam., (\textunderscore iris sysirinchium\textunderscore , Lin.).
\section{Pé-de-cabra}
\begin{itemize}
\item {Grp. gram.:m.}
\end{itemize}
Alavanca de ferro, com a extremidade fendida, á semelhança de um pé de cabra: \textunderscore o gatuno arrombou a porta com um pé-de-cabra\textunderscore .
\section{Pé-de-cantiga}
\begin{itemize}
\item {Grp. gram.:m.}
\end{itemize}
\begin{itemize}
\item {Utilização:Fam.}
\end{itemize}
Pretexto.
Estribilho, adágio.
\section{Pé-de-cavallo}
\begin{itemize}
\item {Grp. gram.:f.}
\end{itemize}
O mesmo que \textunderscore codagem\textunderscore .
\section{Pé-de-chumbo}
\begin{itemize}
\item {Grp. gram.:m.}
\end{itemize}
\begin{itemize}
\item {Utilização:Fig.}
\end{itemize}
Pessôa, que anda lentamente.
Pessôa séria, grave.
\section{Pé-de-gallinha}
\begin{itemize}
\item {Grp. gram.:m.}
\end{itemize}
Planta gramínea de Cabo-Verde.
\section{Pé-de-gallo}
\begin{itemize}
\item {Grp. gram.:m.}
\end{itemize}
O mesmo que \textunderscore lúpulo\textunderscore .
\section{Pé-de-ganso}
\begin{itemize}
\item {Grp. gram.:m.}
\end{itemize}
O mesmo que \textunderscore anserina\textunderscore .
\section{Pé-de-gato}
\begin{itemize}
\item {Grp. gram.:m.}
\end{itemize}
Planta, da fam. das compostas, (\textunderscore goaphalium dinicum\textunderscore ).
\section{Pedeireiro}
\begin{itemize}
\item {Grp. gram.:m.}
\end{itemize}
Espécie de milhafre. Cf. \textunderscore Século\textunderscore , de 10-X-97.
\section{Pé-de-leão}
\begin{itemize}
\item {Grp. gram.:m.}
\end{itemize}
O mesmo que \textunderscore alchemila\textunderscore .
\section{Pé-de-lebre}
\begin{itemize}
\item {Grp. gram.:m.}
\end{itemize}
Planta, o mesmo que \textunderscore lagopo\textunderscore .
\section{Pé-de-moleque}
\begin{itemize}
\item {Grp. gram.:m.}
\end{itemize}
\begin{itemize}
\item {Utilização:Bras}
\end{itemize}
Espécie de doce de mendobi.
\section{Pé-de-moleque}
\begin{itemize}
\item {Grp. gram.:m.}
\end{itemize}
\begin{itemize}
\item {Utilização:Bras}
\end{itemize}
O mesmo que \textunderscore manauê\textunderscore .
\section{Pé-de-perdiz}
\begin{itemize}
\item {Grp. gram.:m.}
\end{itemize}
Casta de uva branca, na região do Doiro, na Beira e no Minho.
Variedade de pêra.
\section{Pé-de-pombo}
\begin{itemize}
\item {Grp. gram.:m.}
\end{itemize}
\begin{itemize}
\item {Utilização:Prov.}
\end{itemize}
\begin{itemize}
\item {Utilização:alent.}
\end{itemize}
Variedade de pêra.
\section{Pé-de-prata}
\begin{itemize}
\item {Grp. gram.:m.}
\end{itemize}
Planta, da serra de Sintra.
\section{Pederasta}
\begin{itemize}
\item {Grp. gram.:m.}
\end{itemize}
\begin{itemize}
\item {Proveniência:(Do lat. \textunderscore paederastes\textunderscore )}
\end{itemize}
Indivíduo, que tem o vício da pederastia.
\section{Pederastia}
\begin{itemize}
\item {Grp. gram.:f.}
\end{itemize}
\begin{itemize}
\item {Proveniência:(Lat. \textunderscore paederastia\textunderscore )}
\end{itemize}
Vício contra a natureza, ou amor repugnante de um homem a um rapaz ou a outro homem.
\section{Pedéria}
\begin{itemize}
\item {Grp. gram.:f.}
\end{itemize}
Gênero de plantas rubiáceas.
(Cp. \textunderscore pédero\textunderscore )
\section{Pedernal}
\begin{itemize}
\item {Grp. gram.:m.}
\end{itemize}
\begin{itemize}
\item {Grp. gram.:Adj.}
\end{itemize}
O mesmo que \textunderscore pederneira\textunderscore .
Veio de pederneira.
Rocha viva.
Relativo a pedra; pétreo:«\textunderscore ...as entranhas pedernaes dos rochedos\textunderscore ». Filinto, I, 237.
\section{Pedernão}
\begin{itemize}
\item {Grp. gram.:m.}
\end{itemize}
Casta de uva minhota.
(Cp. \textunderscore pederneira\textunderscore )
\section{Pederneira}
\begin{itemize}
\item {Grp. gram.:f.}
\end{itemize}
\begin{itemize}
\item {Utilização:Miner.}
\end{itemize}
Pedra duríssima que, ferida pelo fuzil, produz fogo.
Designação vulgar do silex pyrómacho.
(Por \textunderscore pedreneira\textunderscore , de \textunderscore pedra\textunderscore )
\section{Pédero}
\begin{itemize}
\item {Grp. gram.:m.}
\end{itemize}
\begin{itemize}
\item {Proveniência:(Do gr. \textunderscore paideros\textunderscore )}
\end{itemize}
Gênero de insectos coleópteros pentâmeros.
\section{Pé-de-rola}
\begin{itemize}
\item {Grp. gram.:m.}
\end{itemize}
Variedade de pêra, talvez a mesma que \textunderscore pé-de-perdiz\textunderscore .
\section{Pedestal}
\begin{itemize}
\item {Grp. gram.:m.}
\end{itemize}
\begin{itemize}
\item {Proveniência:(It. \textunderscore pedestallo\textunderscore )}
\end{itemize}
Peça de metal, pedra ou madeira, que sustenta uma estátua, uma columna, etc.
Peanha.
Suppedâneo.
Plintho; base.
\section{Pedestre}
\begin{itemize}
\item {Grp. gram.:adj.}
\end{itemize}
\begin{itemize}
\item {Utilização:Fig.}
\end{itemize}
\begin{itemize}
\item {Proveniência:(Lat. \textunderscore pedester\textunderscore )}
\end{itemize}
Que anda ou está a pé.
Humilde.
\textunderscore Latim pedestre\textunderscore , o latim dos arraiaes, o latim vulgar, na decadência dos Romanos.
\section{Pedestremente}
\begin{itemize}
\item {Grp. gram.:adv.}
\end{itemize}
De modo pedestre; a pé.
\section{Pedestrianismo}
\begin{itemize}
\item {Grp. gram.:m.}
\end{itemize}
\begin{itemize}
\item {Proveniência:(De \textunderscore pedestriano\textunderscore )}
\end{itemize}
Systema ou costume de fazer grandes marchas a pé.
Gênero de desafio ou luta entre corredores ou andarilhos.
\section{Pedestriano}
\begin{itemize}
\item {Grp. gram.:m.}
\end{itemize}
\begin{itemize}
\item {Proveniência:(Ingl. \textunderscore pedestrian\textunderscore )}
\end{itemize}
O que marcha ou corre em luta ou concurso com outrem.
\section{Pé-de-vento}
\begin{itemize}
\item {Grp. gram.:m.}
\end{itemize}
Tufão; furacão.
\section{Pediário}
\begin{itemize}
\item {Grp. gram.:adj.}
\end{itemize}
\begin{itemize}
\item {Utilização:Bot.}
\end{itemize}
\begin{itemize}
\item {Proveniência:(Do lat. \textunderscore pes\textunderscore , \textunderscore pedis\textunderscore )}
\end{itemize}
O mesmo que \textunderscore apedado\textunderscore .
\section{Pediatra}
\begin{itemize}
\item {Grp. gram.:m.}
\end{itemize}
Médico de crianças.
(Cp. \textunderscore pediatria\textunderscore )
\section{Pediatria}
\begin{itemize}
\item {Grp. gram.:f.}
\end{itemize}
\begin{itemize}
\item {Proveniência:(Do gr. \textunderscore pais\textunderscore , \textunderscore paidos\textunderscore  + \textunderscore iatreia\textunderscore )}
\end{itemize}
Medicina das crianças.
\section{Pediátrico}
\begin{itemize}
\item {Grp. gram.:adj.}
\end{itemize}
Relativo a pediatria.
\section{Pedição}
\begin{itemize}
\item {Grp. gram.:f.}
\end{itemize}
\begin{itemize}
\item {Utilização:Ant.}
\end{itemize}
O mesmo que \textunderscore petição\textunderscore .
\section{Pedicel}
\begin{itemize}
\item {Grp. gram.:m.}
\end{itemize}
O mesmo que \textunderscore pedicello\textunderscore :«\textunderscore ...iam levar ao pedicel da bonina...\textunderscore »Camillo, \textunderscore Mulhér Fatal\textunderscore , 32.
\section{Pedicelado}
\begin{itemize}
\item {Grp. gram.:adj.}
\end{itemize}
Que tem pedicello.
\section{Pedicellado}
\begin{itemize}
\item {Grp. gram.:adj.}
\end{itemize}
Que tem pedicello.
\section{Pedicello}
\begin{itemize}
\item {Grp. gram.:m.}
\end{itemize}
\begin{itemize}
\item {Utilização:Bot.}
\end{itemize}
\begin{itemize}
\item {Proveniência:(Lat. \textunderscore pedicellus\textunderscore )}
\end{itemize}
Divisão extrema de um pedúnculo ramificado.
\section{Pedicéllulo}
\begin{itemize}
\item {Grp. gram.:adj.}
\end{itemize}
\begin{itemize}
\item {Utilização:Bot.}
\end{itemize}
\begin{itemize}
\item {Proveniência:(De \textunderscore pedicello\textunderscore )}
\end{itemize}
Supporte filiforme em uma cavidade do clinantho.
\section{Pedicelo}
\begin{itemize}
\item {Grp. gram.:m.}
\end{itemize}
\begin{itemize}
\item {Utilização:Bot.}
\end{itemize}
\begin{itemize}
\item {Proveniência:(Lat. \textunderscore pedicellus\textunderscore )}
\end{itemize}
Divisão extrema de um pedúnculo ramificado.
\section{Pedicélulo}
\begin{itemize}
\item {Grp. gram.:adj.}
\end{itemize}
\begin{itemize}
\item {Utilização:Bot.}
\end{itemize}
\begin{itemize}
\item {Proveniência:(De \textunderscore pedicelo\textunderscore )}
\end{itemize}
Suporte filiforme em uma cavidade do clinanto.
\section{Pediculado}
\begin{itemize}
\item {Grp. gram.:adj.}
\end{itemize}
Ligado a pedículo; ligado por pedículo.
\section{Pedicular}
\begin{itemize}
\item {Grp. gram.:m.}
\end{itemize}
\begin{itemize}
\item {Proveniência:(Lat. \textunderscore pedicularis\textunderscore )}
\end{itemize}
Planta escrofularínea.
\section{Pediculária}
\begin{itemize}
\item {Grp. gram.:f.}
\end{itemize}
Gênero de plantas escrofularineas, o mesmo que \textunderscore pedicular\textunderscore .
\section{Pedículo}
\begin{itemize}
\item {Grp. gram.:m.}
\end{itemize}
\begin{itemize}
\item {Proveniência:(Lat. \textunderscore pediculus\textunderscore )}
\end{itemize}
Supporte de qualquer órgão vegetal.
Qualquer parte adelgaçada que supporta um órgão ou parte de um órgão animal.
Pé dos cogumelos.
\section{Pedicuro}
\begin{itemize}
\item {Grp. gram.:m.}
\end{itemize}
\begin{itemize}
\item {Proveniência:(Do lat. \textunderscore pes\textunderscore , \textunderscore pedis\textunderscore  + \textunderscore cura\textunderscore )}
\end{itemize}
Aquelle que se dedica á extirpação dos callos.
\section{Pedida}
\begin{itemize}
\item {Grp. gram.:f.}
\end{itemize}
\begin{itemize}
\item {Utilização:Ant.}
\end{itemize}
\begin{itemize}
\item {Utilização:Ant.}
\end{itemize}
Carta que, no jôgo do \textunderscore trinta e um\textunderscore , o jogador pede, para perfazer o número que deseja.
Licença, que os emphyteutas pediam ao senhorio, para ceifar o terreno que cultivavam.
Espécie de tributo, que os senhores das terras arrecadavam de seus vassallos, a título de coisa pedida. Cf. Viterbo, \textunderscore Elucidário\textunderscore .
O mesmo que \textunderscore pedido\textunderscore :«\textunderscore saluo a pedida do Papa ou a pedida noua dElRey\textunderscore ». \textunderscore Documento\textunderscore  de 1351. Cf. \textunderscore Revista de Guimarães\textunderscore , XXXIII, 17.
\section{Pedido}
\begin{itemize}
\item {Grp. gram.:m.}
\end{itemize}
\begin{itemize}
\item {Grp. gram.:Adj.}
\end{itemize}
Acto de pedir.
Aquillo que se pediu.
Affluência de compradores solicitando certa mercadoria.
\textunderscore Missa pedida\textunderscore , a que há de sêr rezada, mediante esmolas, solicitadas na rua e de porta em porta.
\section{Pedidor}
\begin{itemize}
\item {Grp. gram.:m.  e  adj.}
\end{itemize}
\begin{itemize}
\item {Utilização:P. us.}
\end{itemize}
Aquelle que pede; peticionário.
O mesmo que \textunderscore pedinte\textunderscore :«\textunderscore ...vosso pedidor e servo...\textunderscore »Filinto, \textunderscore D. Man.\textunderscore , I, 14.
\section{Pediforme}
\begin{itemize}
\item {Grp. gram.:adj.}
\end{itemize}
\begin{itemize}
\item {Proveniência:(Do lat. \textunderscore pes\textunderscore , \textunderscore pedis\textunderscore  + \textunderscore forma\textunderscore )}
\end{itemize}
Que tem fórma de pé.
\section{Pedigolho}
\begin{itemize}
\item {fónica:gô}
\end{itemize}
\begin{itemize}
\item {Grp. gram.:m.}
\end{itemize}
O mesmo que \textunderscore pedigonho\textunderscore .
\section{Pedigonho}
\begin{itemize}
\item {Grp. gram.:m.}
\end{itemize}
O mesmo que \textunderscore pedinchão\textunderscore .
(Cast. \textunderscore pedigueño\textunderscore )
\section{Pedilantho}
\begin{itemize}
\item {Grp. gram.:m.}
\end{itemize}
\begin{itemize}
\item {Proveniência:(Do gr. \textunderscore pedilon\textunderscore , sapato, e \textunderscore anthos\textunderscore , flôr)}
\end{itemize}
Gênero de plantas euphorbiáceas.
\section{Pedilanto}
\begin{itemize}
\item {Grp. gram.:m.}
\end{itemize}
\begin{itemize}
\item {Proveniência:(Do gr. \textunderscore pedilon\textunderscore , sapato, e \textunderscore anthos\textunderscore , flôr)}
\end{itemize}
Gênero de plantas euforbiáceas.
\section{Pedilúvio}
\begin{itemize}
\item {Grp. gram.:m.}
\end{itemize}
\begin{itemize}
\item {Proveniência:(Do lat. \textunderscore pes\textunderscore , \textunderscore pedis\textunderscore  + \textunderscore luere\textunderscore )}
\end{itemize}
Banho aos pés.
\section{Pedímano}
\begin{itemize}
\item {Grp. gram.:adj.}
\end{itemize}
\begin{itemize}
\item {Grp. gram.:M. pl.}
\end{itemize}
\begin{itemize}
\item {Proveniência:(Do lat. \textunderscore pes\textunderscore , \textunderscore pedis\textunderscore  + \textunderscore manus\textunderscore )}
\end{itemize}
Diz-se dos mammíferos, que se servem dos membros posteriores como de mãos.
Tríbo de mammíferos, em cujos pés posteriores o pollegar é opposto aos outros dedos.
\section{Pedimento}
\begin{itemize}
\item {Grp. gram.:m.}
\end{itemize}
Acto de pedir.
Petição; pedido; rôgo:«\textunderscore ...feito a pedimento dos cónegos de S. Vicente...\textunderscore »\textunderscore Auto de Santo-António\textunderscore . Cf. A. Pimentel, \textunderscore Chiado\textunderscore , XIV.
\section{Pedincha}
\begin{itemize}
\item {Grp. gram.:f.}
\end{itemize}
Acto de pedinchar.
\section{Pedinchão}
\begin{itemize}
\item {Grp. gram.:m.  e  adj.}
\end{itemize}
Aquelle que pedincha.
\section{Pedinchar}
\begin{itemize}
\item {Grp. gram.:v. t.  e  i.}
\end{itemize}
Pedir impertinentemente; pedir muito; pedir com lamúria.
\section{Pedinornito}
\begin{itemize}
\item {Grp. gram.:adj.}
\end{itemize}
\begin{itemize}
\item {Utilização:Zool.}
\end{itemize}
\begin{itemize}
\item {Proveniência:(Do gr. \textunderscore pedion\textunderscore  + \textunderscore ornis\textunderscore )}
\end{itemize}
Diz-se das aves, que vivem nas planícies.--Deveria dizer-se \textunderscore pediornito\textunderscore .
\section{Pedintão}
\begin{itemize}
\item {Grp. gram.:m.}
\end{itemize}
O mesmo que \textunderscore pedinchão\textunderscore .
\section{Pedintaria}
\begin{itemize}
\item {Grp. gram.:f.}
\end{itemize}
\begin{itemize}
\item {Proveniência:(De \textunderscore pedinte\textunderscore )}
\end{itemize}
Classe dos mendigos; mendicidade.
\section{Pedinte}
\begin{itemize}
\item {Grp. gram.:adj.}
\end{itemize}
\begin{itemize}
\item {Grp. gram.:M.  e  f.}
\end{itemize}
Que pede.
Que mendiga.
Pessôa, que pede.
Pessôa, que mendiga.
\section{Pedionalgia}
\begin{itemize}
\item {Grp. gram.:f.}
\end{itemize}
\begin{itemize}
\item {Utilização:Med.}
\end{itemize}
\begin{itemize}
\item {Proveniência:(Do gr. \textunderscore pedion\textunderscore  + \textunderscore algos\textunderscore )}
\end{itemize}
Doença, que tem apparecido nalgumas épocas e nalguns países e que se manifesta por dôr intensa na planta do pé.
\section{Pedionite}
\begin{itemize}
\item {Grp. gram.:f.}
\end{itemize}
\begin{itemize}
\item {Utilização:Miner.}
\end{itemize}
\begin{itemize}
\item {Proveniência:(Do gr. \textunderscore pedion\textunderscore )}
\end{itemize}
Variedade de feldspatho.
\section{Pediónomo}
\begin{itemize}
\item {Grp. gram.:adj.}
\end{itemize}
\begin{itemize}
\item {Utilização:Zool.}
\end{itemize}
\begin{itemize}
\item {Proveniência:(Do gr. \textunderscore pedion\textunderscore  + \textunderscore nemein\textunderscore )}
\end{itemize}
Que vive nos campos.
\section{Pedipalpos}
\begin{itemize}
\item {Grp. gram.:m. pl.}
\end{itemize}
\begin{itemize}
\item {Utilização:Zool.}
\end{itemize}
\begin{itemize}
\item {Proveniência:(Do lat. \textunderscore pes\textunderscore , \textunderscore pedis\textunderscore  + \textunderscore palpus\textunderscore )}
\end{itemize}
Família de arachnídeos, cujos palpos têm fórma de braços.
\section{Pedir}
\begin{itemize}
\item {Grp. gram.:v. t.}
\end{itemize}
\begin{itemize}
\item {Grp. gram.:V. i.}
\end{itemize}
\begin{itemize}
\item {Proveniência:(Lat. \textunderscore petere\textunderscore )}
\end{itemize}
Solicitar, rogar: \textunderscore pedir protecção\textunderscore .
Implorar: \textunderscore pedir esmola\textunderscore .
Reclamar.
Pretender ou exigir como preço do que se vende: \textunderscore a peixeira pede cinco tostões por cada linguado\textunderscore .
Requerer; demandar: \textunderscore pedir justiça\textunderscore .
Orar: \textunderscore pede por mim a Deus\textunderscore .
\section{Pédite}
\begin{itemize}
\item {Grp. gram.:adj.}
\end{itemize}
\begin{itemize}
\item {Utilização:Poét.}
\end{itemize}
\begin{itemize}
\item {Proveniência:(Do lat. \textunderscore pedes\textunderscore , \textunderscore peditis\textunderscore )}
\end{itemize}
Relativo a infantaria:«\textunderscore ...a pédite hoste cobrem\textunderscore ». Filinto, XIV, 229.
\section{Peditório}
\begin{itemize}
\item {Grp. gram.:m.}
\end{itemize}
Acto de pedir a várias pessôas ou ao público, para fins de caridade ou religião.
Supplica repetida e instante.
\section{Pedive}
\begin{itemize}
\item {Grp. gram.:f.}
\end{itemize}
(\textunderscore Metáth. pop.\textunderscore  de \textunderscore pevide\textunderscore )
\section{Pedivoso}
\begin{itemize}
\item {Grp. gram.:adj.}
\end{itemize}
\begin{itemize}
\item {Utilização:T. da Bairrada}
\end{itemize}
\begin{itemize}
\item {Proveniência:(De \textunderscore pedive\textunderscore )}
\end{itemize}
Que tem difficuldade em pronunciar o \textunderscore r\textunderscore .
Que não pronuncía bem qualquer palavra.
A que não occorrem palavras para se exprimir bem.
\section{Pedo}
\begin{itemize}
\item {fónica:pê}
\end{itemize}
\begin{itemize}
\item {Grp. gram.:m.}
\end{itemize}
Árvore indiana, de fibras têxteis.
\section{Pedo}
\begin{itemize}
\item {Grp. gram.:m.}
\end{itemize}
\begin{itemize}
\item {Utilização:Ant.}
\end{itemize}
O mesmo que \textunderscore ovo\textunderscore . Cf. \textunderscore Diccion. de Nomes, Vozes e Coisas...\textunderscore 
\section{Pedocaédro}
\begin{itemize}
\item {Grp. gram.:m.}
\end{itemize}
Designação primitiva do monocyclo.
\section{Pedófilo}
\begin{itemize}
\item {Grp. gram.:adj.}
\end{itemize}
\begin{itemize}
\item {Utilização:P. us.}
\end{itemize}
\begin{itemize}
\item {Proveniência:(Do gr. \textunderscore pais\textunderscore , \textunderscore paidos\textunderscore  + \textunderscore philos\textunderscore )}
\end{itemize}
Que gosta de crianças.
\section{Pedoflebotomia}
\begin{itemize}
\item {Grp. gram.:f.}
\end{itemize}
\begin{itemize}
\item {Utilização:Med.}
\end{itemize}
Sangría das crianças.
\section{Pedoiro}
\begin{itemize}
\item {Grp. gram.:m.}
\end{itemize}
\begin{itemize}
\item {Utilização:Prov.}
\end{itemize}
\begin{itemize}
\item {Utilização:beir.}
\end{itemize}
\begin{itemize}
\item {Utilização:trasm.}
\end{itemize}
O mesmo que \textunderscore poidoiro\textunderscore .
\section{Pedoiro}
\begin{itemize}
\item {Grp. gram.:m.}
\end{itemize}
\begin{itemize}
\item {Utilização:Prov.}
\end{itemize}
\begin{itemize}
\item {Utilização:trasm.}
\end{itemize}
Último resto do mealheiro.
(Relaciona-se com \textunderscore pedoiro\textunderscore ^1? Ou é contr. de \textunderscore pé-de-oiro\textunderscore , por allusão ao \textunderscore pé de meia\textunderscore , em que os antigos juntavam as suas economias?)
\section{Pedologia}
\begin{itemize}
\item {Grp. gram.:f.}
\end{itemize}
\begin{itemize}
\item {Proveniência:(Do gr. \textunderscore pais\textunderscore , \textunderscore paidos\textunderscore  + \textunderscore logos\textunderscore )}
\end{itemize}
Tratado de educação infantil.
Sciência do primeiro ensino.
\section{Pedómetro}
\begin{itemize}
\item {Grp. gram.:m.}
\end{itemize}
\begin{itemize}
\item {Proveniência:(Do lat. \textunderscore pes\textunderscore , \textunderscore pedis\textunderscore  + gr. \textunderscore metron\textunderscore )}
\end{itemize}
Instrumento, com que se contam os passos de quem marcha.
\section{Pé-do-morto}
\begin{itemize}
\item {Grp. gram.:m.}
\end{itemize}
Planta capparidácea da Índia portuguesa, (\textunderscore capparis trifoliata\textunderscore , Roxb.).
\section{Pedonomia}
\begin{itemize}
\item {Grp. gram.:f.}
\end{itemize}
\begin{itemize}
\item {Proveniência:(Do gr. \textunderscore pais\textunderscore , \textunderscore paidos\textunderscore  + \textunderscore nomos\textunderscore )}
\end{itemize}
Conjunto dos preceitos sôbre a instrucção primária.
\section{Pedóphilo}
\begin{itemize}
\item {Grp. gram.:adj.}
\end{itemize}
\begin{itemize}
\item {Utilização:P. us.}
\end{itemize}
\begin{itemize}
\item {Proveniência:(Do gr. \textunderscore pais\textunderscore , \textunderscore paidos\textunderscore  + \textunderscore philos\textunderscore )}
\end{itemize}
Que gosta de crianças.
\section{Pedophlebotomia}
\begin{itemize}
\item {Grp. gram.:f.}
\end{itemize}
\begin{itemize}
\item {Utilização:Med.}
\end{itemize}
Sangría das crianças.
\section{Pedotechnia}
\begin{itemize}
\item {Grp. gram.:f.}
\end{itemize}
\begin{itemize}
\item {Proveniência:(Do gr. \textunderscore pais\textunderscore , \textunderscore paidos\textunderscore  + \textunderscore tekne\textunderscore )}
\end{itemize}
Applicação prática da pedologia.
\section{Pedotecnia}
\begin{itemize}
\item {Grp. gram.:f.}
\end{itemize}
\begin{itemize}
\item {Proveniência:(Do gr. \textunderscore pais\textunderscore , \textunderscore paidos\textunderscore  + \textunderscore tekne\textunderscore )}
\end{itemize}
Aplicação prática da pedologia.
\section{Pedótriba}
\begin{itemize}
\item {Grp. gram.:m.}
\end{itemize}
\begin{itemize}
\item {Proveniência:(Do gr. \textunderscore pais\textunderscore , \textunderscore paidos\textunderscore  + \textunderscore tribein\textunderscore )}
\end{itemize}
Professor de gymnástica; gymnasta.
\section{Pedotríbica}
\begin{itemize}
\item {Grp. gram.:f.}
\end{itemize}
\begin{itemize}
\item {Utilização:Ant.}
\end{itemize}
Gymnástica.
(Fem. de \textunderscore pedotríbico\textunderscore )
\section{Pedotríbico}
\begin{itemize}
\item {Grp. gram.:adj.}
\end{itemize}
\begin{itemize}
\item {Proveniência:(De \textunderscore pedótríba\textunderscore )}
\end{itemize}
Relativo a exercícios gymnásticos.
\section{Pedotrofia}
\begin{itemize}
\item {Grp. gram.:f.}
\end{itemize}
Modo de alimentar crianças, para que se lhes desenvolva compleição vigorosa.
(Cp. \textunderscore pedótrofo\textunderscore )
\section{Pedótrofo}
\begin{itemize}
\item {Grp. gram.:m.}
\end{itemize}
\begin{itemize}
\item {Proveniência:(Do gr. \textunderscore pais\textunderscore , \textunderscore paidos\textunderscore  + \textunderscore trophe\textunderscore )}
\end{itemize}
Aquele que ensina ou pratíca a pedotrofia.
\section{Pedotrophia}
\begin{itemize}
\item {Grp. gram.:f.}
\end{itemize}
Modo de alimentar crianças, para que se lhes desenvolva compleição vigorosa.
(Cp. \textunderscore pedótropho\textunderscore )
\section{Pedótropho}
\begin{itemize}
\item {Grp. gram.:m.}
\end{itemize}
\begin{itemize}
\item {Proveniência:(Do gr. \textunderscore pais\textunderscore , \textunderscore paidos\textunderscore  + \textunderscore trophe\textunderscore )}
\end{itemize}
Aquelle que ensina ou pratíca a pedotrophia.
\section{Pedra}
\begin{itemize}
\item {Grp. gram.:f.}
\end{itemize}
\begin{itemize}
\item {Utilização:Fig.}
\end{itemize}
\begin{itemize}
\item {Utilização:Prov.}
\end{itemize}
\begin{itemize}
\item {Grp. gram.:Pl.}
\end{itemize}
\begin{itemize}
\item {Utilização:Prov.}
\end{itemize}
\begin{itemize}
\item {Proveniência:(Lat. \textunderscore petra\textunderscore )}
\end{itemize}
Corpo duro e sólido, da natureza das rochas, empregado especialmente na construcção de edifícios.
Nome de várias pedras, empregadas em diversos usos.
Concrecação calcária, formada na bexiga, nos rins, etc.
Pedaço de ardósia encaixilhado, para nelle se fazerem cálculos e contas.
Quadro preto de madeira, destinado ao mesmo fim.
Porção de uma substância sólida e dura: \textunderscore uma pedra de sal\textunderscore .
Fragmento.
Lápide, lage, que cobre uma sepultura.
Saraiva.
Mineral, de origem ígnea, precioso pela sua raridade e mais conhecido por \textunderscore pedra preciosa\textunderscore .
\textunderscore Estar de pedra e cal\textunderscore , estar seguro, muito unido.
\textunderscore Coração de pedra\textunderscore , coração frio, empedernido.
\textunderscore Sêr de pedra\textunderscore , sêr insensível, inexorável.
\textunderscore Pedra angular\textunderscore , base, fundamento.
\textunderscore Trazer pedra no sapato\textunderscore , andar desconfiado.
\textunderscore Carvão de pedra\textunderscore , hulha, carvão mineral ou fóssil, que serve de combustível.
\textunderscore Pedra de toque\textunderscore , o mesmo que \textunderscore jaspe-negro\textunderscore .
\textunderscore Pedra de linho\textunderscore , pêso de oito arráteis de linho espadelado.
\textunderscore Pedra broeira\textunderscore , grés amarelado e um pouco brando.
Os ovos dos ninhos dos pássaros.
\section{Pedraço}
\begin{itemize}
\item {Grp. gram.:m.}
\end{itemize}
\begin{itemize}
\item {Utilização:Prov.}
\end{itemize}
\begin{itemize}
\item {Utilização:minh.}
\end{itemize}
\begin{itemize}
\item {Proveniência:(De \textunderscore pedra\textunderscore )}
\end{itemize}
Granizo, saraiva.
\section{Pedrada}
\begin{itemize}
\item {Grp. gram.:f.}
\end{itemize}
\begin{itemize}
\item {Utilização:Fig.}
\end{itemize}
\begin{itemize}
\item {Proveniência:(Do b. lat. \textunderscore petrata\textunderscore )}
\end{itemize}
Arremêsso de pedra.
Pancada com pedra que se arremessou.
Offensa, insulto.
\section{Pedra-de-fogo}
\begin{itemize}
\item {Grp. gram.:f.}
\end{itemize}
O mesmo que \textunderscore pederneira\textunderscore .
\section{Pedra-de-raio}
\begin{itemize}
\item {Grp. gram.:f.}
\end{itemize}
Designação vulgar do aerólitho e de algumas pedras polidas da idade neolíthica.
\section{Pedrado}
\begin{itemize}
\item {Grp. gram.:adj.}
\end{itemize}
\begin{itemize}
\item {Proveniência:(De \textunderscore pedra\textunderscore )}
\end{itemize}
Empedrado.
Salpicado de preto e branco:«\textunderscore a talha... pedrada.\textunderscore »R. Lobo, \textunderscore Eglogas\textunderscore , 110.
\section{Pedra-escrófula}
\begin{itemize}
\item {Grp. gram.:f.}
\end{itemize}
Nome, que nalguns pontos da África se dá á doença do somno. Cf. Capello e Ivens, I, 135.
\section{Pedragoso}
\begin{itemize}
\item {Grp. gram.:adj.}
\end{itemize}
(V.pedregoso)
\section{Pedra-íman}
\begin{itemize}
\item {Grp. gram.:f.}
\end{itemize}
Íman natural.
\section{Pedra-infernal}
\begin{itemize}
\item {Grp. gram.:f.}
\end{itemize}
Nitrato de prata crystallizado.
\section{Pedral}
\begin{itemize}
\item {Grp. gram.:adj.}
\end{itemize}
\begin{itemize}
\item {Grp. gram.:M.}
\end{itemize}
Relativo a pedra.
Cheia de pedras ou em que há muitas pedras.
Casta de figo, negro por fóra e encarnado por dentro.
\section{Pedranceira}
\begin{itemize}
\item {Grp. gram.:f.}
\end{itemize}
Monte de pedras.
\section{Pedrão}
\begin{itemize}
\item {Grp. gram.:m.}
\end{itemize}
\begin{itemize}
\item {Utilização:Des.}
\end{itemize}
\begin{itemize}
\item {Proveniência:(De \textunderscore pedra\textunderscore )}
\end{itemize}
O mesmo ou melhor que \textunderscore padrão\textunderscore .
\section{Pedra-pomes}
\begin{itemize}
\item {Grp. gram.:f.}
\end{itemize}
(V.pomes)
\section{Pedraria}
\begin{itemize}
\item {Grp. gram.:f.}
\end{itemize}
Porção de pedras para cantaria.
Quantidade de pedras preciosas; jóias.
\section{Pedra-ume}
\begin{itemize}
\item {Grp. gram.:f.}
\end{itemize}
Sulfato de alumina e potassa.
\section{Pedregal}
\begin{itemize}
\item {Grp. gram.:m.}
\end{itemize}
Lugar, onde há muitas pedras.
\section{Pedregão}
\begin{itemize}
\item {Grp. gram.:m.}
\end{itemize}
\begin{itemize}
\item {Utilização:Prov.}
\end{itemize}
Nó da madeira.
\section{Pedregoso}
\begin{itemize}
\item {Grp. gram.:adj.}
\end{itemize}
\begin{itemize}
\item {Proveniência:(Do lat. \textunderscore petricosus\textunderscore )}
\end{itemize}
Em que que há muitas pedras: \textunderscore caminho pedregoso\textunderscore .
\section{Pedreguento}
\begin{itemize}
\item {Grp. gram.:adj.}
\end{itemize}
\begin{itemize}
\item {Utilização:Bras. do N}
\end{itemize}
O mesmo que \textunderscore pedregoso\textunderscore .
\section{Pedregulhento}
\begin{itemize}
\item {Grp. gram.:adj.}
\end{itemize}
Que tem muitos pedregulhos.
\section{Pedregulho}
\begin{itemize}
\item {Grp. gram.:m.}
\end{itemize}
\begin{itemize}
\item {Utilização:Prov.}
\end{itemize}
\begin{itemize}
\item {Utilização:beir.}
\end{itemize}
\begin{itemize}
\item {Utilização:Bras. do N}
\end{itemize}
\begin{itemize}
\item {Proveniência:(Do lat. hyp. \textunderscore petriculeus\textunderscore ?)}
\end{itemize}
Grande pedra; calhau, penedo.
Montão de pedras.
Lugar, onde há muitas pedras miúdas.
\section{Pedreira}
\begin{itemize}
\item {Grp. gram.:f.}
\end{itemize}
\begin{itemize}
\item {Utilização:T. de Turquel}
\end{itemize}
Lugar ou rocha, donde se extrái pedra.
Protecção, empenhos.
\section{Pedreirada}
\begin{itemize}
\item {Grp. gram.:f.}
\end{itemize}
\begin{itemize}
\item {Utilização:Deprec.}
\end{itemize}
Os pedreiros-livres. Cf. Camillo, \textunderscore Mar. da Fonte\textunderscore , 231.
\section{Pedreiral}
\begin{itemize}
\item {Grp. gram.:adj.}
\end{itemize}
Relativo a pedreiros-livres. Cf. Macedo, \textunderscore Burros\textunderscore , 250 e 320.
\section{Pedreirinho}
\begin{itemize}
\item {Grp. gram.:m.}
\end{itemize}
Espécie de andorinha, (\textunderscore cotyle riparía\textunderscore , Lin.).
\section{Pedreiro}
\begin{itemize}
\item {Grp. gram.:m.}
\end{itemize}
\begin{itemize}
\item {Utilização:Bras. de Goiás}
\end{itemize}
\begin{itemize}
\item {Utilização:Pop.}
\end{itemize}
\begin{itemize}
\item {Grp. gram.:Adj.}
\end{itemize}
\begin{itemize}
\item {Proveniência:(Do b. lat. \textunderscore petrarius\textunderscore )}
\end{itemize}
Aquelle que trabalha em obras de pedra.
Espécie de canhão antigo, que expellia projécteis de pedra.
Boi, o mesmo que \textunderscore junqueira\textunderscore .
\textunderscore Pedreiro livre\textunderscore  o mesmo que \textunderscore mação\textunderscore ^2.
Homem ímpio, incrédulo.
Dizia-se da peça de artilharia, que atirava pedras.
\section{Pedreiro}
\begin{itemize}
\item {Grp. gram.:m.}
\end{itemize}
O mesmo que \textunderscore gaivão\textunderscore ^1.
(Contr. de \textunderscore pedeireiro\textunderscore ?)
\section{Pedreiro-das-barreiras}
\begin{itemize}
\item {Grp. gram.:m.}
\end{itemize}
Ave, o mesmo que \textunderscore pedreirinho\textunderscore .
\section{Pedreneira}
\begin{itemize}
\item {Grp. gram.:f.}
\end{itemize}
O mesmo ou melhor que \textunderscore pederneira\textunderscore .
\section{Pedrento}
\begin{itemize}
\item {Grp. gram.:adj.}
\end{itemize}
\begin{itemize}
\item {Utilização:Neol.}
\end{itemize}
\begin{itemize}
\item {Proveniência:(De \textunderscore pedra\textunderscore )}
\end{itemize}
O mesmo que \textunderscore pedrês\textunderscore ^1.
Diz-se especialmente da côr, que os cúmulos dão ao céu. Cf. Rev. \textunderscore Tradição\textunderscore , V, 11.
\section{Pedrês}
\begin{itemize}
\item {Grp. gram.:adj.}
\end{itemize}
\begin{itemize}
\item {Grp. gram.:Adj.}
\end{itemize}
\begin{itemize}
\item {Utilização:Açor}
\end{itemize}
\begin{itemize}
\item {Proveniência:(Do lat. \textunderscore petrensis\textunderscore )}
\end{itemize}
Salpicado de preto e branco: \textunderscore gallinha pedrês\textunderscore .
Feito de pedras brancas e pretas.
Que estala como a pedra.
\section{Pedrês}
\begin{itemize}
\item {Grp. gram.:m.}
\end{itemize}
\begin{itemize}
\item {Utilização:T. do Fundão}
\end{itemize}
O mesmo que aldrava ou fecho da porta.
\section{Pedrinha-na-boca}
\begin{itemize}
\item {Grp. gram.:f.}
\end{itemize}
Espécie de jôgo popular.
\section{Pedrinhas}
\begin{itemize}
\item {Grp. gram.:f. pl.}
\end{itemize}
\begin{itemize}
\item {Utilização:Gír. de rapazes.}
\end{itemize}
\begin{itemize}
\item {Proveniência:(De \textunderscore pedra\textunderscore )}
\end{itemize}
Espécie de jôgo popular.
Ovos em ninho de pássaros, trocando-se assim o nome, para que lá não vão as formigas.
\section{Pedrinho}
\begin{itemize}
\item {Grp. gram.:adj.}
\end{itemize}
Feito de pedra: \textunderscore pontes pedrinhas\textunderscore , \textunderscore lagares pedrinhos\textunderscore .
\section{Pedrisco}
\begin{itemize}
\item {Grp. gram.:m.}
\end{itemize}
\begin{itemize}
\item {Proveniência:(De \textunderscore pedra\textunderscore )}
\end{itemize}
Saraiva miúda.
\section{Pedriscoso}
\begin{itemize}
\item {Grp. gram.:adj.}
\end{itemize}
Relativo a pedrisco.
Que lança pedrisco:«\textunderscore ...borrascas pedriscosas.\textunderscore »Filinto, VI, 251.
\section{Pedrista}
\begin{itemize}
\item {Grp. gram.:m.  e  adj.}
\end{itemize}
Partidário de D. Pedro IV, em Portugal.
\section{Pedroiço}
\begin{itemize}
\item {Grp. gram.:m.}
\end{itemize}
Montão de pedras.
\section{Pedro-quinto}
\begin{itemize}
\item {Grp. gram.:m.}
\end{itemize}
\begin{itemize}
\item {Proveniência:(De \textunderscore Pedro Quinto\textunderscore , n. p.)}
\end{itemize}
Espécie de capa curta, com mangas falsas.
\section{Pedroso}
\begin{itemize}
\item {Grp. gram.:adj.}
\end{itemize}
\begin{itemize}
\item {Proveniência:(Do lat. \textunderscore petrosus\textunderscore )}
\end{itemize}
Pedregoso; que é de natureza ou consistência da pedra.
\section{Pedrouço}
\begin{itemize}
\item {Grp. gram.:m.}
\end{itemize}
Montão de pedras.
\section{Pedro-ximenes}
\begin{itemize}
\item {Grp. gram.:m.}
\end{itemize}
Casta de videira do Brasil.
\section{Pedrulho}
\begin{itemize}
\item {Grp. gram.:m.}
\end{itemize}
\begin{itemize}
\item {Utilização:Pesc.}
\end{itemize}
Pedras miúdas, que constituem o pandulho.
\section{Pedunculado}
\begin{itemize}
\item {Grp. gram.:adj.}
\end{itemize}
Que tem pedúnculo.
\section{Peduncular}
\begin{itemize}
\item {Grp. gram.:adj.}
\end{itemize}
Relativo a pedúnculo.
\section{Pedunculeaneo}
\begin{itemize}
\item {Grp. gram.:adj.}
\end{itemize}
\begin{itemize}
\item {Utilização:Bot.}
\end{itemize}
\begin{itemize}
\item {Proveniência:(De \textunderscore pedúnculo\textunderscore )}
\end{itemize}
Diz-se das partes vegetaes, resultantes da degeneração do pedúnculo, como as gavinhas da videira.
\section{Pedúnculo}
\begin{itemize}
\item {Grp. gram.:m.}
\end{itemize}
\begin{itemize}
\item {Proveniência:(Lat. \textunderscore pedunculus\textunderscore )}
\end{itemize}
Pé da flôr ou do fruto.
Supporte de qualquer órgão animal.
\section{Pedunculoso}
\begin{itemize}
\item {Grp. gram.:adj.}
\end{itemize}
O mesmo que \textunderscore pedunculado\textunderscore .
\section{Peeira}
\begin{itemize}
\item {Grp. gram.:f.}
\end{itemize}
\begin{itemize}
\item {Proveniência:(De \textunderscore pear\textunderscore )}
\end{itemize}
Ulceração da pelle, entre as unhas, no gado bovino.
Doença, que produz febre e faz coxear as reses.
\section{Peeiro}
\begin{itemize}
\item {Grp. gram.:m.}
\end{itemize}
\begin{itemize}
\item {Utilização:Bras. do N}
\end{itemize}
O mesmo que \textunderscore peadoiro\textunderscore .
\section{Peendença}
\begin{itemize}
\item {Grp. gram.:f.}
\end{itemize}
\begin{itemize}
\item {Utilização:Obsol.}
\end{itemize}
O mesmo que \textunderscore penitência\textunderscore .
\section{P. E. F.}
Abrev. da loc. \textunderscore por especial favor\textunderscore .
Us. no sobrescrito das cartas, de que alguém é obsequiosamente portador.
\section{Pé-fresco}
\begin{itemize}
\item {Grp. gram.:m.}
\end{itemize}
\begin{itemize}
\item {Utilização:Chul.}
\end{itemize}
Gaiato, garoto.
Plebeu.
Democrata.
Homem que anda descalço.
Partidário da patuleia.
\section{Péga}
\begin{itemize}
\item {Grp. gram.:f.}
\end{itemize}
\begin{itemize}
\item {Utilização:Gír.}
\end{itemize}
Acto de pegar.
Acto de agarrar o toiro com as mãos.
Cabo, asa, ou parte de um objecto, por onde se pega.
Aso ou ensejo.
Discussão acalorada, desavença: \textunderscore os dois sócios tiveram péga\textunderscore .
A verdade.
\section{Péga}
\begin{itemize}
\item {Grp. gram.:f.}
\end{itemize}
\begin{itemize}
\item {Utilização:Bras}
\end{itemize}
\begin{itemize}
\item {Proveniência:(Do lat. \textunderscore pedica\textunderscore )}
\end{itemize}
Laço ou braga de ferro, com que se prendiam os pés dos escravos fugitivos.
Recrutamento forçado.
\section{Pêga}
\begin{itemize}
\item {Grp. gram.:f.}
\end{itemize}
\begin{itemize}
\item {Utilização:Náut.}
\end{itemize}
\begin{itemize}
\item {Utilização:Burl.}
\end{itemize}
\begin{itemize}
\item {Grp. gram.:Adj.}
\end{itemize}
\begin{itemize}
\item {Proveniência:(Do lat. \textunderscore pica\textunderscore )}
\end{itemize}
Pássaro corvídeo.
Peça de madeira, que cobre a cabeça do mastro.
Mulhér feia.
Qualquer mulhér.
Diz-se do cavallo que é malhado.
\section{Pêga}
\begin{itemize}
\item {Grp. gram.:f.}
\end{itemize}
O mesmo que \textunderscore péga\textunderscore ^2.
\section{Pêga-cuca}
\begin{itemize}
\item {Grp. gram.:f.}
\end{itemize}
Nome que, em Estarreja, se dá ao cuco rabilongo.
\section{Pègada}
\begin{itemize}
\item {Grp. gram.:f.}
\end{itemize}
\begin{itemize}
\item {Utilização:Fig.}
\end{itemize}
\begin{itemize}
\item {Proveniência:(Do lat. \textunderscore pedicata\textunderscore )}
\end{itemize}
Vestígio, que o pé deixa no solo; peugada.
Vestígio.
\section{Pègadas}
\begin{itemize}
\item {Grp. gram.:f. pl.}
\end{itemize}
O mesmo que \textunderscore apegadas\textunderscore . Cf. \textunderscore Port. Ant. e Mod.\textunderscore , vb. \textunderscore rabello\textunderscore .
\section{Pegadeira}
\begin{itemize}
\item {Grp. gram.:f.}
\end{itemize}
\begin{itemize}
\item {Proveniência:(De \textunderscore pegar\textunderscore )}
\end{itemize}
Parte, por onde se pegam ou se fazem mover alguns apparelhos; péga.
\section{Pegadiço}
\begin{itemize}
\item {Grp. gram.:adj.}
\end{itemize}
\begin{itemize}
\item {Utilização:Fig.}
\end{itemize}
\begin{itemize}
\item {Proveniência:(De \textunderscore pegado\textunderscore )}
\end{itemize}
Que se pega com facilidade; viscoso.
Contagioso.
Maçador.
Que tem o costume de conversar largamente, importunando.
\section{Pegadilha}
\begin{itemize}
\item {Grp. gram.:f.}
\end{itemize}
\begin{itemize}
\item {Proveniência:(De \textunderscore pegar\textunderscore )}
\end{itemize}
Discussão acalorada.
Desavença.
Pretexto para contender, para brigar.
\section{Pegado}
\begin{itemize}
\item {Grp. gram.:adj.}
\end{itemize}
\begin{itemize}
\item {Utilização:Ant.}
\end{itemize}
\begin{itemize}
\item {Proveniência:(De \textunderscore pegar\textunderscore )}
\end{itemize}
Contiguo, vizinho: \textunderscore na casa pegada á minha\textunderscore .
Aproximado; amigo.
Breado ou untado com pez. Cf. B. Pereira, \textunderscore Prosódia\textunderscore , vb. \textunderscore picatus\textunderscore .
\section{Pegadoiro}
\begin{itemize}
\item {Grp. gram.:m.}
\end{itemize}
\begin{itemize}
\item {Proveniência:(De \textunderscore pegar\textunderscore )}
\end{itemize}
Parte de um objecto ou utensílio por onde se lhe pega; cabo: \textunderscore o pegadoiro da sertan\textunderscore .
\section{Pêga-do-mar}
\begin{itemize}
\item {Grp. gram.:f.}
\end{itemize}
Ave marítima, também conhecida por \textunderscore negrinha\textunderscore  e \textunderscore pato-negro\textunderscore .
\section{Pegador}
\begin{itemize}
\item {Grp. gram.:m.  e  adj.}
\end{itemize}
Aquelle que pega.
Moço de forcado.
Peixe, o mesmo que \textunderscore rêmora\textunderscore .
\section{Pegadouro}
\begin{itemize}
\item {Grp. gram.:m.}
\end{itemize}
\begin{itemize}
\item {Proveniência:(De \textunderscore pegar\textunderscore )}
\end{itemize}
Parte de um objecto ou utensílio por onde se lhe pega; cabo: \textunderscore o pegadouro da sertan\textunderscore .
\section{Pegadura}
\begin{itemize}
\item {Grp. gram.:f.}
\end{itemize}
O mesmo que \textunderscore péga\textunderscore ^1.
\section{Péga-flôr}
\begin{itemize}
\item {Grp. gram.:m.}
\end{itemize}
O mesmo que \textunderscore picaflor\textunderscore .
\section{Péga-fogo}
\begin{itemize}
\item {Grp. gram.:m.}
\end{itemize}
\begin{itemize}
\item {Utilização:Bras. do S}
\end{itemize}
Bailado campestre, espécie de fandango.
\section{Pegajoso}
\begin{itemize}
\item {Grp. gram.:adj.}
\end{itemize}
O mesmo que \textunderscore pegadiço\textunderscore .
\section{Pegalhoso}
\begin{itemize}
\item {Grp. gram.:adj.}
\end{itemize}
\begin{itemize}
\item {Utilização:Prov.}
\end{itemize}
\begin{itemize}
\item {Utilização:trasm.}
\end{itemize}
Que acceita sempre quanto lhe offereçam de comer.
O mesmo que \textunderscore pegajoso\textunderscore .
\section{Pegamaça}
\begin{itemize}
\item {Grp. gram.:f.}
\end{itemize}
O mesmo que \textunderscore bardana\textunderscore .
\section{Pegamaço}
\begin{itemize}
\item {Grp. gram.:m.}
\end{itemize}
\begin{itemize}
\item {Utilização:Fam.}
\end{itemize}
\begin{itemize}
\item {Utilização:Bot.}
\end{itemize}
\begin{itemize}
\item {Utilização:Bras}
\end{itemize}
\begin{itemize}
\item {Proveniência:(De \textunderscore pegar\textunderscore )}
\end{itemize}
Massa para grudar.
Homem maçador.
Aresta ou pragana curta, que termina em ponta aguda e curva, á maneira de gancho.
Camada de lama, que se fórma nas estradas, em seguida a uma pequena chuvada.
\section{Péga-mão}
\begin{itemize}
\item {Grp. gram.:m.}
\end{itemize}
\begin{itemize}
\item {Utilização:T. de Alcanena}
\end{itemize}
Cabo, ou parte de um objecto, instrumento ou utensílio, por onde elle se maneja ou se segura.
\section{Pegamento}
\begin{itemize}
\item {Grp. gram.:m.}
\end{itemize}
O mesmo que \textunderscore pegadura\textunderscore .
\section{Péga-môsca}
\begin{itemize}
\item {Grp. gram.:f.}
\end{itemize}
Planta ornamental do Brasil, (\textunderscore apogynum\textunderscore ).
\section{Peganhento}
\begin{itemize}
\item {Grp. gram.:adj.}
\end{itemize}
O mesmo que \textunderscore pegadiço\textunderscore .
\section{Peganho}
\begin{itemize}
\item {Grp. gram.:m.}
\end{itemize}
\begin{itemize}
\item {Utilização:Ant.}
\end{itemize}
\begin{itemize}
\item {Proveniência:(De \textunderscore pegar\textunderscore )}
\end{itemize}
Redemoínho, pé de vento. Cf. Castanheda, \textunderscore Descobr.\textunderscore , I, 98.
\section{Pègão}
\begin{itemize}
\item {Grp. gram.:m.}
\end{itemize}
Grande pégo.
\section{Pègão}
\begin{itemize}
\item {Grp. gram.:m.}
\end{itemize}
\begin{itemize}
\item {Proveniência:(De \textunderscore pé\textunderscore )}
\end{itemize}
Grande pilar ou supporte.
Grande pé de vento, peganho.
\section{Pègão}
\begin{itemize}
\item {Grp. gram.:m.}
\end{itemize}
\begin{itemize}
\item {Utilização:Prov.}
\end{itemize}
\begin{itemize}
\item {Utilização:minh.}
\end{itemize}
Pequeno rasgão na roupa.
\section{Péga-péga}
\begin{itemize}
\item {Grp. gram.:m.}
\end{itemize}
\begin{itemize}
\item {Utilização:Bras. do N}
\end{itemize}
O mesmo que \textunderscore péga\textunderscore ^2.
\section{Péga-péga-môsca}
\begin{itemize}
\item {Grp. gram.:f.}
\end{itemize}
Planta medicinal da ilha de San-Thomé.
O mesmo que \textunderscore péga-môsca\textunderscore ?
\section{Péga-pinto}
\begin{itemize}
\item {Grp. gram.:f.}
\end{itemize}
\begin{itemize}
\item {Utilização:Bras. do N}
\end{itemize}
Planta medicinal.
\section{Pegar}
\begin{itemize}
\item {Grp. gram.:v. t.}
\end{itemize}
\begin{itemize}
\item {Utilização:Ant.}
\end{itemize}
\begin{itemize}
\item {Grp. gram.:V. i.}
\end{itemize}
\begin{itemize}
\item {Grp. gram.:V. p.}
\end{itemize}
\begin{itemize}
\item {Proveniência:(Lat. \textunderscore picare\textunderscore )}
\end{itemize}
Unir; collar: \textunderscore pegar estampílhas\textunderscore .
Segurar, agarrar.
Communicar: \textunderscore pegar moléstias\textunderscore .
Empesgar, brear, untar com pez. Cf. B. Pereira, \textunderscore Prosódia\textunderscore , vb. \textunderscore oppico\textunderscore .
Collar-se; ficar adherente.
Lançar a mão.
Criar raízes: \textunderscore a tanchoeira pegou\textunderscore .
Diffundir-se, generalizar-se: \textunderscore em pegando a moda...\textunderscore 
Pôr embaraço.
Começar, dar princípio.
Confinar, estar contíguo: \textunderscore o quintal péga com a casa\textunderscore .
Tomar posse.
Produzir effeito: \textunderscore a accusação pegou\textunderscore .
Communicar-se.
Tornar-se adherente.
Esturrar-se, (falando-se de iguarias).
Não querer andar, (falando-se dos animaes).
Sêr importuno, pegadiço.
Tornar-se contínuo: \textunderscore o frio pegou-se\textunderscore .
\section{Pêga-real}
\begin{itemize}
\item {Grp. gram.:f.}
\end{itemize}
Peça do navio, para manter verticalmente o mastaréu da gávea.
\section{Pêgas}
\begin{itemize}
\item {Grp. gram.:m.}
\end{itemize}
\begin{itemize}
\item {Utilização:Fam.}
\end{itemize}
\begin{itemize}
\item {Proveniência:(De \textunderscore Pêgas\textunderscore , n. p. de um antigo jurisconsulto)}
\end{itemize}
Advogado rábula, chicaneiro.
\section{Péga-saia}
\begin{itemize}
\item {Grp. gram.:f.}
\end{itemize}
Nome que, em Cabo-Verde, se dá á planta rubiácea, que no continente é conhecida por \textunderscore amor-de-hortelo\textunderscore . Cf. o jornal \textunderscore O Mundo\textunderscore , de 23-XI-912.
\section{Pegaseu}
\begin{itemize}
\item {Grp. gram.:adj.}
\end{itemize}
\begin{itemize}
\item {Utilização:Poét.}
\end{itemize}
\begin{itemize}
\item {Proveniência:(Do lat. \textunderscore pegaseus\textunderscore )}
\end{itemize}
Relativo a Pégaso, cavallo mythológico. Cf. \textunderscore Eufrosina\textunderscore , 287.
\section{Pegasiano}
\begin{itemize}
\item {Grp. gram.:adj.}
\end{itemize}
\begin{itemize}
\item {Proveniência:(Lat. \textunderscore pegasianus\textunderscore )}
\end{itemize}
Dizia-se, em Roma, de certos decretos senatoriaes, baseados na doutrina de um jurisconsulto célebre, chamado Pégaso. Cf. Costa Lobo, \textunderscore Sát. de Juv.\textunderscore , I, 237.
\section{Pégaso}
\begin{itemize}
\item {Grp. gram.:m.}
\end{itemize}
\begin{itemize}
\item {Proveniência:(Lat. \textunderscore Pegasus\textunderscore )}
\end{itemize}
Cavallo mythológico, que fez nascer a fonte Hypocrene.
Constellação do hemisphério boreal.
\section{Pegma}
\begin{itemize}
\item {Grp. gram.:m.}
\end{itemize}
\begin{itemize}
\item {Proveniência:(Lat. \textunderscore pegma\textunderscore )}
\end{itemize}
Apparelho theatral, usado pelos Romanos, que automaticamente se erguia, se abaixava e se abria, fazendo apparecer e desapparecer os actores.
\section{Pegmatite}
\begin{itemize}
\item {Grp. gram.:f.}
\end{itemize}
\begin{itemize}
\item {Utilização:Miner.}
\end{itemize}
\begin{itemize}
\item {Proveniência:(Do gr. \textunderscore pegma\textunderscore , \textunderscore pegmatos\textunderscore , concreção)}
\end{itemize}
Espécie de granulite, de grossos elementos.
\section{Pegmatítico}
\begin{itemize}
\item {Grp. gram.:adj.}
\end{itemize}
Relativo á pegmatite.
\section{Pegmatóide}
\begin{itemize}
\item {Grp. gram.:adj.}
\end{itemize}
\begin{itemize}
\item {Proveniência:(Do gr. \textunderscore pegmu\textunderscore , \textunderscore pegmatos\textunderscore  + \textunderscore eidos\textunderscore )}
\end{itemize}
Que tem o carácter da pegmatite.
\section{Pegmatólitho}
\begin{itemize}
\item {Grp. gram.:m.}
\end{itemize}
\begin{itemize}
\item {Utilização:Miner.}
\end{itemize}
\begin{itemize}
\item {Proveniência:(Do gr. \textunderscore pegma\textunderscore , \textunderscore pegmatos\textunderscore  + \textunderscore lithos\textunderscore )}
\end{itemize}
Variedade de orthose, ás vezes rósea ou amarelada.
\section{Pegmatólito}
\begin{itemize}
\item {Grp. gram.:m.}
\end{itemize}
\begin{itemize}
\item {Utilização:Miner.}
\end{itemize}
\begin{itemize}
\item {Proveniência:(Do gr. \textunderscore pegma\textunderscore , \textunderscore pegmatos\textunderscore  + \textunderscore lithos\textunderscore )}
\end{itemize}
Variedade de ortose, ás vezes rósea ou amarelada.
\section{Pegmina}
\begin{itemize}
\item {Grp. gram.:f.}
\end{itemize}
\begin{itemize}
\item {Utilização:Med.}
\end{itemize}
\begin{itemize}
\item {Proveniência:(Do gr. \textunderscore pegnumi\textunderscore , coágulo)}
\end{itemize}
Côdea inflammatória no coágulo da sangria.
\section{Pégo}
\begin{itemize}
\item {Grp. gram.:m.}
\end{itemize}
\begin{itemize}
\item {Utilização:Fig.}
\end{itemize}
O ponto mais fundo de um rio, lago, etc.
Voragem; abysmo.
(Contr. de \textunderscore pélago\textunderscore )
\section{Pêgo}
\begin{itemize}
\item {Grp. gram.:m.}
\end{itemize}
\begin{itemize}
\item {Utilização:Prov.}
\end{itemize}
Macho da pêga.
\section{Pêgo}
\begin{itemize}
\item {Utilização:Prov.}
\end{itemize}
\begin{itemize}
\item {Utilização:minh.}
\end{itemize}
Pequena refeição dos trabalhadores, entre o almôço e o jantar.
Pequena porção de comída; petisqueira:«\textunderscore Vinham jogadores professos armar a forquilha ao pedreiro com cartas marcadas e pêgo\textunderscore ». Camillo, \textunderscore Brasileira\textunderscore , 304.
(Cp. \textunderscore peguilho\textunderscore )
\section{Pêgo}
\begin{itemize}
\item {Grp. gram.:adj.}
\end{itemize}
Diz-se de uma variedade de milho, também conhecido por \textunderscore milho rôxo\textunderscore . Cf. \textunderscore Bibl. da G. do Campo\textunderscore , 303.
\section{Pegomancia}
\begin{itemize}
\item {Grp. gram.:f.}
\end{itemize}
\begin{itemize}
\item {Proveniência:(Do gr. \textunderscore pege\textunderscore  + \textunderscore manteia\textunderscore )}
\end{itemize}
Espécie de adivinhação, que se praticava observando o movimento das águas das fontes.
\section{Pegomântico}
\begin{itemize}
\item {Grp. gram.:adj.}
\end{itemize}
Relativo á pegomancia.
\section{Pegómia}
\begin{itemize}
\item {Grp. gram.:f.}
\end{itemize}
Gênero de insectos dípteros.
\section{Pegómya}
\begin{itemize}
\item {Grp. gram.:f.}
\end{itemize}
Gênero de insectos dípteros.
\section{Pégos-verdes}
\begin{itemize}
\item {Grp. gram.:m. pl.}
\end{itemize}
Monges, não professos nem sacerdotes, que tinham convento junto a Portimão e estavam sujeitos á jurisdicção do Bispo do Algarve.
\section{Pegu}
\begin{itemize}
\item {Grp. gram.:m.}
\end{itemize}
\begin{itemize}
\item {Utilização:Ant.}
\end{itemize}
O mesmo que \textunderscore peguano\textunderscore . Cf. \textunderscore Barros, Déc.\textunderscore  II, l. 6, c. 6.
\section{Peguano}
\begin{itemize}
\item {Grp. gram.:m.}
\end{itemize}
Indivíduo, natural do Pegu.
\section{Pègudo}
\begin{itemize}
\item {Grp. gram.:m.}
\end{itemize}
O mesmo que \textunderscore pé-agudo\textunderscore .
\section{Pegueiro}
\begin{itemize}
\item {Grp. gram.:m.}
\end{itemize}
\begin{itemize}
\item {Proveniência:(Do lat. \textunderscore picarius\textunderscore )}
\end{itemize}
Fabricante de pez.
\section{Peguenhento}
\begin{itemize}
\item {Grp. gram.:adj.}
\end{itemize}
O mesmo que \textunderscore peganhento\textunderscore . Cf. Camillo, \textunderscore Volcoens\textunderscore , 61.
\section{Peguenho}
\begin{itemize}
\item {Grp. gram.:adj.}
\end{itemize}
O mesmo que \textunderscore peguenhento\textunderscore .
\section{Peguia}
\begin{itemize}
\item {Grp. gram.:f.}
\end{itemize}
\begin{itemize}
\item {Utilização:Prov.}
\end{itemize}
\begin{itemize}
\item {Utilização:alent.}
\end{itemize}
\begin{itemize}
\item {Utilização:T. de Alcanena}
\end{itemize}
O mesmo que \textunderscore pégo\textunderscore . Cf. Ficalho, \textunderscore Contos\textunderscore , 34.
Sítio, onde há muitos pégos, no rio.
\section{Peguial}
\begin{itemize}
\item {Grp. gram.:m.}
\end{itemize}
\begin{itemize}
\item {Utilização:Ant.}
\end{itemize}
Guardador de ovelhas.
(Cp. \textunderscore pegulhal\textunderscore )
\section{Peguilha}
\begin{itemize}
\item {Grp. gram.:f.}
\end{itemize}
\begin{itemize}
\item {Utilização:Pop.}
\end{itemize}
\begin{itemize}
\item {Proveniência:(De \textunderscore pega\textunderscore ^1)}
\end{itemize}
Começo de altercação.
Provocação.
Ditos picantes e provocadores.
Pegadilha.
\section{Peguilhar}
\begin{itemize}
\item {Grp. gram.:v. i.}
\end{itemize}
\begin{itemize}
\item {Utilização:Fam.}
\end{itemize}
\begin{itemize}
\item {Proveniência:(De \textunderscore peguilha\textunderscore )}
\end{itemize}
Levantar dúvidas ou questões sôbre futilidades.
Têr ditos ou phrases, tendentes a provocar disputa.
Discutir bagatelas.
\section{Peguilhento}
\begin{itemize}
\item {Grp. gram.:adj.}
\end{itemize}
Que gosta de peguilhar; que costuma peguilhar.
Provocador por palavras.
\section{Peguilho}
\begin{itemize}
\item {Grp. gram.:m.}
\end{itemize}
\begin{itemize}
\item {Utilização:Pesc.}
\end{itemize}
\begin{itemize}
\item {Utilização:Prov.}
\end{itemize}
Aquillo que péga ou colla.
Embaraço; causa de demora.
Pretexto para contender; pegadilha.
Ferro ou âncora, que, debaixo de água, fica com uma parte acima do nivel do solo e que por isso rompe as redes.
Conducto de queijo, azeitonas, etc., que se acompanha com pão.
\section{Peguinhar}
\begin{itemize}
\item {Grp. gram.:v. t.}
\end{itemize}
\begin{itemize}
\item {Utilização:Ant.}
\end{itemize}
\begin{itemize}
\item {Utilização:Fig.}
\end{itemize}
\begin{itemize}
\item {Grp. gram.:V. i.}
\end{itemize}
Calcar aos pés.
Contrariar; provocar.
Fazer provocação.
(Cp. \textunderscore peguilhar\textunderscore )
\section{Pegulhal}
\begin{itemize}
\item {Grp. gram.:m.}
\end{itemize}
\begin{itemize}
\item {Utilização:Des.}
\end{itemize}
\begin{itemize}
\item {Utilização:Prov.}
\end{itemize}
\begin{itemize}
\item {Utilização:alent.}
\end{itemize}
\begin{itemize}
\item {Utilização:beir.}
\end{itemize}
\begin{itemize}
\item {Utilização:Ant.}
\end{itemize}
\begin{itemize}
\item {Proveniência:(Do lat. \textunderscore pecus\textunderscore )}
\end{itemize}
O mesmo que \textunderscore rebanho\textunderscore ^1.
Porção de ovelhas, pertencentes a um pastor, e que êste apascenta juntamente com o rebanho de seu patrão.
Pegureiro.
\section{Pegulho}
\begin{itemize}
\item {Grp. gram.:m.}
\end{itemize}
O mesmo que \textunderscore pecúlio\textunderscore .
\section{Pegulho}
\begin{itemize}
\item {Grp. gram.:m.}
\end{itemize}
\begin{itemize}
\item {Utilização:Des.}
\end{itemize}
\begin{itemize}
\item {Utilização:Prov.}
\end{itemize}
\begin{itemize}
\item {Utilização:trasm.}
\end{itemize}
O mesmo que \textunderscore peguilha\textunderscore .
Rapazinho, que discorre e fala como pessôa crescida.
\section{Pegungo}
\begin{itemize}
\item {Grp. gram.:m.}
\end{itemize}
\begin{itemize}
\item {Utilização:Ant.}
\end{itemize}
\begin{itemize}
\item {Utilização:Pop.}
\end{itemize}
Pé grande, pesunho.
\section{Pegureira}
\begin{itemize}
\item {Grp. gram.:f.}
\end{itemize}
Rapariga ou mulhér, que guarda gado.
(Fem. de \textunderscore pegureiro\textunderscore )
\section{Pegureiro}
\begin{itemize}
\item {Grp. gram.:m.}
\end{itemize}
\begin{itemize}
\item {Utilização:Bras}
\end{itemize}
\begin{itemize}
\item {Grp. gram.:Adj.}
\end{itemize}
\begin{itemize}
\item {Proveniência:(Do lat. hyp. \textunderscore pecorarius\textunderscore , de \textunderscore pecus\textunderscore )}
\end{itemize}
Guardador de gado; pastor; zagal.
Cão de caça, cão de gado:«\textunderscore ...lambe o pegureiro o caçador.\textunderscore »Araújo Porto-Alegre.
O mesmo que \textunderscore pastoril\textunderscore . Cf. Castilho, \textunderscore Primavera\textunderscore , 99.
\section{Peia}
\begin{itemize}
\item {Grp. gram.:f.}
\end{itemize}
\begin{itemize}
\item {Utilização:Fig.}
\end{itemize}
\begin{itemize}
\item {Utilização:Bras}
\end{itemize}
\begin{itemize}
\item {Proveniência:(Do lat. \textunderscore pedica\textunderscore ?)}
\end{itemize}
Corda ou laço, com que se prendem os pés das bêstas.
Péga^2.
Embaraço, impedimento: \textunderscore falar sem peias\textunderscore .
Chicote; correia.
\section{Peidão}
\begin{itemize}
\item {Grp. gram.:adj.}
\end{itemize}
\begin{itemize}
\item {Utilização:Bras. do N}
\end{itemize}
\begin{itemize}
\item {Utilização:Pleb.}
\end{itemize}
Que peida muito.
\section{Peidar}
\begin{itemize}
\item {Grp. gram.:v. i.}
\end{itemize}
\begin{itemize}
\item {Utilização:Pleb.}
\end{itemize}
\begin{itemize}
\item {Proveniência:(De \textunderscore peido\textunderscore )}
\end{itemize}
Emittir ventosidades pelo ânus, com estrépito.
\section{Peido}
\begin{itemize}
\item {Grp. gram.:m.}
\end{itemize}
\begin{itemize}
\item {Utilização:Pleb.}
\end{itemize}
\begin{itemize}
\item {Proveniência:(Do lat. \textunderscore peditum\textunderscore )}
\end{itemize}
Acto de peidar.
\section{Peidorrada}
\begin{itemize}
\item {Grp. gram.:f.}
\end{itemize}
\begin{itemize}
\item {Utilização:Pleb.}
\end{itemize}
\begin{itemize}
\item {Proveniência:(De \textunderscore peido\textunderscore )}
\end{itemize}
Acto de peidar muito.
\section{Peidorrear}
\begin{itemize}
\item {Grp. gram.:v. i.}
\end{itemize}
O mesmo que \textunderscore peidar\textunderscore .
\section{Peidorreira}
\begin{itemize}
\item {Grp. gram.:f.}
\end{itemize}
\begin{itemize}
\item {Utilização:Bras. do Rio}
\end{itemize}
Fruto silvestre, do tamanho do pêssego, e cujo fruto é comestível.
\section{Peidorreiro}
\begin{itemize}
\item {Grp. gram.:m.  e  adj.}
\end{itemize}
\begin{itemize}
\item {Utilização:Fig.}
\end{itemize}
\begin{itemize}
\item {Proveniência:(De \textunderscore peido\textunderscore )}
\end{itemize}
Aquelle que peida.
Nojento.
\section{Peidorreta}
\begin{itemize}
\item {fónica:rê}
\end{itemize}
\begin{itemize}
\item {Grp. gram.:m.}
\end{itemize}
\begin{itemize}
\item {Utilização:Pleb.}
\end{itemize}
\begin{itemize}
\item {Proveniência:(De \textunderscore peidorrear\textunderscore )}
\end{itemize}
Acto de simular com a bôca o estrépito das ventosidades anaes.
\section{Peina}
\begin{itemize}
\item {Grp. gram.:f.}
\end{itemize}
\begin{itemize}
\item {Utilização:Prov.}
\end{itemize}
\begin{itemize}
\item {Utilização:trasm.}
\end{itemize}
\begin{itemize}
\item {Proveniência:(Do fr. \textunderscore peigne\textunderscore )}
\end{itemize}
Pequeno pente de pau.
\section{Peinaços}
\begin{itemize}
\item {Grp. gram.:m. pl.}
\end{itemize}
\begin{itemize}
\item {Utilização:Prov.}
\end{itemize}
\begin{itemize}
\item {Utilização:trasm.}
\end{itemize}
\begin{itemize}
\item {Proveniência:(De \textunderscore peina\textunderscore )}
\end{itemize}
Dentes perpendiculares ao plano das rodas de certos engenhos.
\section{Peinar-se}
\begin{itemize}
\item {Grp. gram.:v. t.}
\end{itemize}
\begin{itemize}
\item {Utilização:Prov.}
\end{itemize}
\begin{itemize}
\item {Utilização:trasm.}
\end{itemize}
Pentear-se, principalmente com peina.
\section{Peiouga}
\begin{itemize}
\item {Grp. gram.:f.}
\end{itemize}
\begin{itemize}
\item {Utilização:Ant.}
\end{itemize}
Pé de porco, chispe.
\section{Peirão}
\begin{itemize}
\item {Grp. gram.:m.}
\end{itemize}
\begin{itemize}
\item {Utilização:Prov.}
\end{itemize}
\begin{itemize}
\item {Utilização:beir.}
\end{itemize}
Pedra tosca e longa, que serve de pilar; pilar.
(Talvez de \textunderscore pedrão\textunderscore  = \textunderscore padrão\textunderscore )
\section{Peirau}
\begin{itemize}
\item {Grp. gram.:m.}
\end{itemize}
O mesmo que \textunderscore perau\textunderscore .
\section{Peita}
\begin{itemize}
\item {Grp. gram.:f.}
\end{itemize}
\begin{itemize}
\item {Proveniência:(Do lat. \textunderscore pactum\textunderscore )}
\end{itemize}
Tributo, que era pago pelos que não eram fidalgos.
Dádiva, com o fim de subornar.
Subôrno.
Crime, commettido por quem recebe dádivas, para praticar certos actos dependentes das suas funcções públicas.
\section{Peitaça}
\begin{itemize}
\item {Grp. gram.:f.}
\end{itemize}
\begin{itemize}
\item {Utilização:Ant.}
\end{itemize}
Câmara, em certos barcos ou juncos asiáticos.
\section{Peitalha}
\begin{itemize}
\item {Grp. gram.:f.}
\end{itemize}
\begin{itemize}
\item {Utilização:Prov.}
\end{itemize}
\begin{itemize}
\item {Utilização:minh.}
\end{itemize}
Cada uma das balizas, com que se delimita aquella parte de um campo lavrado, a que se vai lançar a semente, lançada a qual, se deslocam para delimitar outra parte, que em seguida vai também receber semente.
\section{Peitalhar}
\begin{itemize}
\item {Grp. gram.:v. t.}
\end{itemize}
\begin{itemize}
\item {Utilização:Prov.}
\end{itemize}
\begin{itemize}
\item {Utilização:minh.}
\end{itemize}
Pôr peitalhas em.
\section{Peitar}
\begin{itemize}
\item {Grp. gram.:v. t.}
\end{itemize}
\begin{itemize}
\item {Utilização:Ant.}
\end{itemize}
\begin{itemize}
\item {Utilização:T. da Bairrada}
\end{itemize}
\begin{itemize}
\item {Proveniência:(De \textunderscore peita\textunderscore )}
\end{itemize}
Procurar corromper com dádivas; subornar.
Pagar, satisfazer. Cf. Figanière, \textunderscore G. Ansures\textunderscore .
Magoar moralmente, melindrar.
\section{Peiteiro}
\begin{itemize}
\item {Grp. gram.:m.  e  adj.}
\end{itemize}
\begin{itemize}
\item {Utilização:Ant.}
\end{itemize}
Aquelle que peita.
Aquelle que pagava o tributo da peita.
\section{Peitica}
\begin{itemize}
\item {Grp. gram.:f.}
\end{itemize}
\begin{itemize}
\item {Utilização:Bras. do N}
\end{itemize}
Espécie de ave.
Pessôa impertinente.
Duende.
\section{Peitilho}
\begin{itemize}
\item {Grp. gram.:m.}
\end{itemize}
Aquillo que reveste o peito.
Peça de vestuário, que se colloca sôbre o peito, semelhando o peitilho da camisa.
\section{Peito}
\begin{itemize}
\item {Grp. gram.:m.}
\end{itemize}
\begin{itemize}
\item {Grp. gram.:Loc. adv.}
\end{itemize}
\begin{itemize}
\item {Grp. gram.:Loc. adv.}
\end{itemize}
\begin{itemize}
\item {Grp. gram.:Loc. adv.}
\end{itemize}
\begin{itemize}
\item {Proveniência:(Do lat. \textunderscore pectus\textunderscore )}
\end{itemize}
Parte do corpo, que contém os pulmões e o coração.
Parte anterior e externa da caixa thorácica: \textunderscore descobrir o peito\textunderscore .
Peitilho.
Órgãos da respiração.
Coragem.
Magnanimidade.
O mesmo que \textunderscore peitoril\textunderscore . Cf. Rebello, \textunderscore Contos e Lendas\textunderscore , 29.
O mesmo que \textunderscore úbere\textunderscore  e \textunderscore mama\textunderscore : \textunderscore criança de peito\textunderscore ; \textunderscore os peitos encaroçaram-lhe\textunderscore .
\textunderscore Janela de peito\textunderscore , janela de parapeito ou aberta um pouco acima do pavimento.
\textunderscore Peito de prova\textunderscore , o mesmo que \textunderscore coiraça\textunderscore . Cf. \textunderscore Conquista do Pegu\textunderscore , IX.
\textunderscore A peito\textunderscore , com empenho, com bôa vontade:«\textunderscore ...amavam-no a peito\textunderscore ». Camillo, \textunderscore Noites de Lam.\textunderscore , 234.
\textunderscore Do peito\textunderscore , do fundo do coração; em verdade; devéras.
\textunderscore Pôr peito\textunderscore , fazer diligência, empregar esforços. Cf. Filinto, \textunderscore D. Man.\textunderscore , I, 32.
\textunderscore De peito feito\textunderscore , propositadamente.
\section{Peito-celeste}
\begin{itemize}
\item {Grp. gram.:m.}
\end{itemize}
Pequenina ave africana, graciosa e canora, (\textunderscore estrelda angolensis\textunderscore ).
\section{Peito-de-canga}
\begin{itemize}
\item {Grp. gram.:f.}
\end{itemize}
Parte curva, entre os buracos dos canzis, nos carros alentejanos de parelha.
\section{Peito-de-moça}
\begin{itemize}
\item {Grp. gram.:m.}
\end{itemize}
\begin{itemize}
\item {Utilização:Bras. do N}
\end{itemize}
Fruto agreste.
A planta, que dá êsse fruto, e que no Maranhão chamam \textunderscore beringela\textunderscore .
\section{Peito-de-morte}
\begin{itemize}
\item {Grp. gram.:m.}
\end{itemize}
Nó de cabo náutico, que aperta duas peças.
\section{Peitogueira}
\begin{itemize}
\item {Grp. gram.:f.}
\end{itemize}
\begin{itemize}
\item {Utilização:Ant.}
\end{itemize}
\begin{itemize}
\item {Proveniência:(De \textunderscore peito\textunderscore )}
\end{itemize}
Rouquidão; tosse. Cf. G. Vicente, \textunderscore Inês Pereira\textunderscore .
\section{Peitolargo}
\begin{itemize}
\item {Grp. gram.:m.}
\end{itemize}
\begin{itemize}
\item {Utilização:Bras. da Baía}
\end{itemize}
\begin{itemize}
\item {Proveniência:(De \textunderscore peito\textunderscore  + \textunderscore largo\textunderscore )}
\end{itemize}
Valentão.
\section{Peitoral}
\begin{itemize}
\item {Grp. gram.:adj.}
\end{itemize}
\begin{itemize}
\item {Grp. gram.:M.}
\end{itemize}
\begin{itemize}
\item {Utilização:Bras. do N}
\end{itemize}
\begin{itemize}
\item {Proveniência:(Do lat. \textunderscore pectoralis\textunderscore )}
\end{itemize}
Relativo ao peito.
Que dá fôrça ao peito; fortificante: \textunderscore um caldo peitoral\textunderscore .
Medicamente contra padecimentos do peito: \textunderscore tomar um peitoral\textunderscore .
Correia, que cinge o peito do cavallo.
Parte externa e anterior do peito da cavalgadura.
Pedaço de pelle, com que o vaqueiro resguarda o peito, e que se prende por correias ao pescoço e á cintura.
\section{Peitoril}
\begin{itemize}
\item {Grp. gram.:m.}
\end{itemize}
\begin{itemize}
\item {Proveniência:(Do lat. hyp. \textunderscore pectorilis\textunderscore , de \textunderscore pectus\textunderscore , \textunderscore pectoris\textunderscore )}
\end{itemize}
O mesmo que \textunderscore parapeito\textunderscore .
Pedra, que fórma o limiar da bôca dos fornos em que se coze o pão.
\section{Peitu}
\begin{itemize}
\item {Grp. gram.:m.}
\end{itemize}
\begin{itemize}
\item {Utilização:Ant.}
\end{itemize}
Foro ou pensão.
(Cp. \textunderscore peita\textunderscore )
\section{Peituda}
\begin{itemize}
\item {Grp. gram.:adj. f.}
\end{itemize}
\begin{itemize}
\item {Utilização:Prov.}
\end{itemize}
\begin{itemize}
\item {Utilização:bras}
\end{itemize}
\begin{itemize}
\item {Proveniência:(De \textunderscore peitudo\textunderscore )}
\end{itemize}
Mulhér, que tem grandes peitos ou mamas.
\section{Peitudo}
\begin{itemize}
\item {Grp. gram.:adj.}
\end{itemize}
\begin{itemize}
\item {Utilização:Bras. de Minas}
\end{itemize}
Que tem peito grande ou forte: \textunderscore animal peitudo\textunderscore .
Valente.
Que é bom cantador.
\section{Peixão}
\begin{itemize}
\item {Grp. gram.:m.}
\end{itemize}
\begin{itemize}
\item {Utilização:Pop.}
\end{itemize}
Grande peixe.
\textunderscore T. de Aveiro\textunderscore  e \textunderscore Cascaes\textunderscore .
Besugo pequeno.
Mulhér corpulenta e perfeita.
\section{Peixe}
\begin{itemize}
\item {Grp. gram.:m.}
\end{itemize}
\begin{itemize}
\item {Utilização:T. de Turquel}
\end{itemize}
\begin{itemize}
\item {Utilização:Prolóq.}
\end{itemize}
\begin{itemize}
\item {Proveniência:(Do lat. \textunderscore piscis\textunderscore )}
\end{itemize}
Animal vertebrado, que nasce e vive na água, respirando por guelras.
Moça vistosa e garrida.
\textunderscore Filho de peixe sabe nadar\textunderscore , diz-se dos filhos que herdaram bôas qualidades dos pais.
\section{Peixe-agulha}
\begin{itemize}
\item {Grp. gram.:m.}
\end{itemize}
Nome de vários peixes escômbridas.
\section{Peixe-anjo}
\begin{itemize}
\item {Grp. gram.:m.}
\end{itemize}
Espécie de peixe, (\textunderscore squatina vulgaris\textunderscore ).
\section{Peixe-aranha}
\begin{itemize}
\item {Grp. gram.:m.}
\end{itemize}
Nome, de duas espécies de peixes trachinídeos, (\textunderscore trachinus vipera\textunderscore , Cuv.).
\section{Peixe-boi}
\begin{itemize}
\item {Grp. gram.:m.}
\end{itemize}
(V.manatim)
\section{Peixe-branco}
\begin{itemize}
\item {Grp. gram.:m.}
\end{itemize}
Nome que em algumas partes do norte do país se dá ao robalo pequeno.
\section{Peixe-cabra}
\begin{itemize}
\item {Grp. gram.:m.}
\end{itemize}
Espécie de peixe triglídeo.
\section{Peixe-cachorro}
\begin{itemize}
\item {Grp. gram.:m.}
\end{itemize}
Peixe do norte do Brasil.
\section{Peixe-cão}
\begin{itemize}
\item {Grp. gram.:m.}
\end{itemize}
Esqualo da costa de Portugal.
\section{Peixe-carago}
\begin{itemize}
\item {Grp. gram.:m.}
\end{itemize}
Espécie de peixe, (\textunderscore pseudo triaris microdon\textunderscore ).
\section{Peixe-cobra}
\begin{itemize}
\item {Grp. gram.:m.}
\end{itemize}
Espécie de peixe anguilliforme.
\section{Peixe-coêlho}
\begin{itemize}
\item {Grp. gram.:m.}
\end{itemize}
Espécie de peixe, (\textunderscore chinoera monstruosa\textunderscore ).
\section{Peixe-curvo}
\begin{itemize}
\item {Grp. gram.:m.}
\end{itemize}
Espécie de peixe escômbrida.
\section{Peixe-de-caixa}
\begin{itemize}
\item {Grp. gram.:m.}
\end{itemize}
\begin{itemize}
\item {Utilização:Pesc.}
\end{itemize}
Qualquer peixe, que tenha cartilagem, como as lulas.
\section{Peixe-eléctrico}
\begin{itemize}
\item {Grp. gram.:adj.}
\end{itemize}
\begin{itemize}
\item {Utilização:T. do Maranhão}
\end{itemize}
O mesmo que \textunderscore puraquê\textunderscore .
\section{Peixe-escolar}
\begin{itemize}
\item {Grp. gram.:m.}
\end{itemize}
Espécie de pescada.
\section{Peixe-espada}
\begin{itemize}
\item {Grp. gram.:m.}
\end{itemize}
\begin{itemize}
\item {Utilização:Gír.}
\end{itemize}
Nome de vários peixes escômbridas.
Espadeirada.
\section{Peixe-espada-preto}
\begin{itemize}
\item {Grp. gram.:m.}
\end{itemize}
Peixe marítimo, (\textunderscore aphanopus carbo\textunderscore ). Cf. P. Moraes, \textunderscore Zool. Elem.\textunderscore , 524.
\section{Peixe-frito}
\begin{itemize}
\item {Grp. gram.:m.}
\end{itemize}
\begin{itemize}
\item {Utilização:Bras. de Minas}
\end{itemize}
\begin{itemize}
\item {Proveniência:(T. onom.)}
\end{itemize}
Avezinha canora, cujo canto parece dizer o seu nome.
\section{Peixe-gallo}
\begin{itemize}
\item {Grp. gram.:m.}
\end{itemize}
Espécie de peixe, (\textunderscore chinoera antarctica\textunderscore ).
\section{Peixeira}
\begin{itemize}
\item {Grp. gram.:f.}
\end{itemize}
Vendedora de peixe.
(Fem. de \textunderscore peixeiro\textunderscore )
\section{Peixeiro}
\begin{itemize}
\item {Grp. gram.:m.}
\end{itemize}
Vendedor de peixe.
\section{Peixelim}
\begin{itemize}
\item {Grp. gram.:m.}
\end{itemize}
Peixe miúdo do mar.
\section{Peixe-lua}
\begin{itemize}
\item {Grp. gram.:m.}
\end{itemize}
Variedade de peixe, (\textunderscore orthogoricus mola\textunderscore , Bloch.).
\section{Peixe-macaco}
\begin{itemize}
\item {Grp. gram.:m.}
\end{itemize}
Peixe da ria de Aveiro, (\textunderscore blennius gattorugine\textunderscore , Brunn.).
\section{Peixe-martelo}
\begin{itemize}
\item {Grp. gram.:m.}
\end{itemize}
Peixe plagióstomo, pardo, de cabeça larga.
\section{Peixe-mulhér}
\begin{itemize}
\item {Grp. gram.:m.}
\end{itemize}
A fêmea do peixe-boi, (\textunderscore manatus senegalensis\textunderscore ):«\textunderscore ... em cuja ponta me atou, com duas conchas de peixe-mulher.\textunderscore »F. Manuel, \textunderscore Apólogos\textunderscore , I, 111.
\section{Peixe-pau}
\begin{itemize}
\item {Grp. gram.:m.}
\end{itemize}
Espécie de peixe, (\textunderscore callianymus lyra\textunderscore ).
\section{Peixe-pedra}
\begin{itemize}
\item {Grp. gram.:m.}
\end{itemize}
Peixe do Amazonas, de cabeça venenosa. Cf. \textunderscore Jorn.-do-Comm.\textunderscore , do Rio, de 12-XI-901.
\section{Peixe-penna}
\begin{itemize}
\item {Grp. gram.:m.}
\end{itemize}
Variedade de ruivo.
\section{Peixe-pica}
\begin{itemize}
\item {Grp. gram.:m.}
\end{itemize}
Peixe da ria de Aveiro, (\textunderscore motella mustela\textunderscore , Gunth.).
\section{Peixe-pimenta}
\begin{itemize}
\item {Grp. gram.:m.}
\end{itemize}
(V.peixe-pau)
\section{Peixe-piolho}
\begin{itemize}
\item {Grp. gram.:m.}
\end{itemize}
O mesmo que \textunderscore agarrador\textunderscore , peixe.
\section{Peixe-pombo}
\begin{itemize}
\item {Grp. gram.:m.}
\end{itemize}
(V. \textunderscore pampo\textunderscore ^1)
\section{Peixe-porco}
\begin{itemize}
\item {Grp. gram.:m.}
\end{itemize}
Peixe plagióstomo, pardo-escuro, de focinho curto e grosso.
\section{Peixe-prego}
\begin{itemize}
\item {Grp. gram.:m.}
\end{itemize}
Peixe plagióstomo, de focinho longo e chato, e pelle guarnecida de tubérculos ponteagudos.
\section{Peixe-rato}
\begin{itemize}
\item {Grp. gram.:m.}
\end{itemize}
Pequeno peixe do mar alto.
\section{Peixe-rei}
\begin{itemize}
\item {Grp. gram.:m.}
\end{itemize}
Espécie de peixe mugiloide, (\textunderscore accípenser sturio\textunderscore , Lin.).
\section{Peixe-roda}
\begin{itemize}
\item {Grp. gram.:m.}
\end{itemize}
Espécie de peixe plectognatho.
\section{Peixe-sapo}
\begin{itemize}
\item {Grp. gram.:m.}
\end{itemize}
\begin{itemize}
\item {Utilização:minh}
\end{itemize}
\begin{itemize}
\item {Utilização:Pesc.}
\end{itemize}
\begin{itemize}
\item {Utilização:T. da Bairrada}
\end{itemize}
O mesmo que \textunderscore tamboril\textunderscore , peixe.
O mesmo que \textunderscore gyrino\textunderscore .
\section{Peixe-serra}
\begin{itemize}
\item {Grp. gram.:m.}
\end{itemize}
O mesmo que \textunderscore espadarte\textunderscore .
\section{Peixe-vermelho}
\begin{itemize}
\item {Grp. gram.:m.}
\end{itemize}
Peixe, o mesmo que \textunderscore pinhão\textunderscore , (\textunderscore carassius auratus\textunderscore ).
\section{Peixe-voador}
\begin{itemize}
\item {Grp. gram.:m.}
\end{itemize}
Espécie de peixe esoce.
\section{Peixezinho}
\begin{itemize}
\item {Grp. gram.:m.}
\end{itemize}
Peixe pequeno. Cf. \textunderscore Luz e Calor\textunderscore , 557.
\section{Peixe-zôrra}
\begin{itemize}
\item {Grp. gram.:m.}
\end{itemize}
Peixe marítimo da costa de Portugal.
\section{Peixinheiro}
\begin{itemize}
\item {Grp. gram.:m.}
\end{itemize}
\begin{itemize}
\item {Utilização:T. de Nazaré}
\end{itemize}
\begin{itemize}
\item {Proveniência:(De \textunderscore peixinho\textunderscore , dem. de \textunderscore peixe\textunderscore )}
\end{itemize}
Almocreve, que vende peixe.
\section{Peixol}
\begin{itemize}
\item {Grp. gram.:m.}
\end{itemize}
\begin{itemize}
\item {Utilização:Ant.}
\end{itemize}
Pau, em que se fechavam os tamboretes do mastro das embarcações. Cf. Fern. Oliveira, \textunderscore Livro da Fabr. das Naus\textunderscore .
\section{Peixota}
\begin{itemize}
\item {Grp. gram.:f.}
\end{itemize}
\begin{itemize}
\item {Utilização:Prov.}
\end{itemize}
\begin{itemize}
\item {Utilização:beir.}
\end{itemize}
Pescada.
Qualquer peixe.
Posta de peixe: \textunderscore uma peixota de bacalhau\textunderscore .
\section{Peixôtoa}
\begin{itemize}
\item {Grp. gram.:f.}
\end{itemize}
\begin{itemize}
\item {Proveniência:(De \textunderscore Peixôto\textunderscore , n. p.)}
\end{itemize}
Gênero de arbustos da América austral.
\section{Pejada}
\begin{itemize}
\item {Grp. gram.:adj.}
\end{itemize}
Diz-se da mulhér e da fêmea dos animaes, em estado de prenhez.
(Fem. de \textunderscore pejado\textunderscore )
\section{Pejado}
\begin{itemize}
\item {Grp. gram.:adj.}
\end{itemize}
Repleto; carregado.
Que sente pejo; acanhado; envergonhado; pejoso.
\section{Pejadoiro}
\begin{itemize}
\item {Grp. gram.:m.}
\end{itemize}
\begin{itemize}
\item {Utilização:Prov.}
\end{itemize}
Apparelho, com que se faz parar o moínho, cortando a água.
\section{Pejadouro}
\begin{itemize}
\item {Grp. gram.:m.}
\end{itemize}
\begin{itemize}
\item {Utilização:Prov.}
\end{itemize}
Apparelho, com que se faz parar o moínho, cortando a água.
\section{Pejamento}
\begin{itemize}
\item {Grp. gram.:m.}
\end{itemize}
Acto ou effeito de pejar.
Aquillo que peja ou embaraça; estôrvo.
\section{Pejar}
\begin{itemize}
\item {Grp. gram.:v. t.}
\end{itemize}
\begin{itemize}
\item {Utilização:Prov.}
\end{itemize}
\begin{itemize}
\item {Grp. gram.:V. i.}
\end{itemize}
\begin{itemize}
\item {Utilização:Bras}
\end{itemize}
\begin{itemize}
\item {Grp. gram.:V. p.}
\end{itemize}
\begin{itemize}
\item {Proveniência:(Do lat. \textunderscore pedicare\textunderscore )}
\end{itemize}
Embaraçar, impedir.
Encher; sobrecarregar.
Fazer parar (o moínho, cortando ou represando a água).
Tornar-se prenhe.
Deixar de moer (o engenho).
Embaraçar-se; envergonhar-se; têr pejo.
Têr receio; hesitar.
\section{Pejeiro}
\begin{itemize}
\item {Grp. gram.:m.}
\end{itemize}
\begin{itemize}
\item {Utilização:Prov.}
\end{itemize}
\begin{itemize}
\item {Utilização:minh.}
\end{itemize}
\begin{itemize}
\item {Proveniência:(De \textunderscore pejar\textunderscore )}
\end{itemize}
Porção de terra, ou torrão, com que se atalha a água de um rêgo ou de uma cale, fazendo que ella se desvie para outra direcção ou para outro rêgo.
\section{Pejerecum}
\begin{itemize}
\item {Grp. gram.:m.}
\end{itemize}
O mesmo que \textunderscore pijerecum\textunderscore .
\section{Pejo}
\begin{itemize}
\item {Grp. gram.:m.}
\end{itemize}
\begin{itemize}
\item {Utilização:Marn.}
\end{itemize}
\begin{itemize}
\item {Utilização:Ant.}
\end{itemize}
\begin{itemize}
\item {Proveniência:(De \textunderscore pejar\textunderscore )}
\end{itemize}
Pudor; acanhamento; vergonha.
O maior reservatório das marinhas de sal.
Impedimento, estôrvo.
\section{Pejorar}
\begin{itemize}
\item {Grp. gram.:v. t.}
\end{itemize}
\begin{itemize}
\item {Proveniência:(Lat. \textunderscore peiorare\textunderscore )}
\end{itemize}
Depreciar; aviltar; rebaixar.
Tornar piór. Cf. Garrett, \textunderscore Port. na Balança\textunderscore , 51.
\section{Pejorativo}
\begin{itemize}
\item {Grp. gram.:adj.}
\end{itemize}
\begin{itemize}
\item {Proveniência:(Do lat. \textunderscore peior\textunderscore )}
\end{itemize}
Diz-se do vocábulo, que adquiriu ou tende a adquirir, ou a que se dá, um sentido torpe, obsceno ou simplesmente desagradável.
\section{Pejoso}
\begin{itemize}
\item {Grp. gram.:adj.}
\end{itemize}
\begin{itemize}
\item {Utilização:Des.}
\end{itemize}
Que tem pejo; envergonhado; acanhado, tímido.
\section{Péla}
\begin{itemize}
\item {Grp. gram.:f.}
\end{itemize}
\begin{itemize}
\item {Utilização:Fig.}
\end{itemize}
\begin{itemize}
\item {Proveniência:(Do lat. \textunderscore píla\textunderscore )}
\end{itemize}
Bóla, para brinquedo de crianças, feita especialmente de borracha.
Bóla, que no jôgo se impelle com a raqueta.
Joguete, ludíbrio.
\section{Péla}
\begin{itemize}
\item {Grp. gram.:f.}
\end{itemize}
\begin{itemize}
\item {Utilização:Prov.}
\end{itemize}
\begin{itemize}
\item {Utilização:beir.}
\end{itemize}
O mesmo que \textunderscore sertan\textunderscore  ou \textunderscore frigideira\textunderscore .
(Accepção extensiva de \textunderscore pella\textunderscore ?)
\section{Péla}
\begin{itemize}
\item {Grp. gram.:f.}
\end{itemize}
\begin{itemize}
\item {Utilização:Prov.}
\end{itemize}
\begin{itemize}
\item {Utilização:minh.}
\end{itemize}
A parte mais grossa da haste mais grossa, nos moínhos movidos a água.
\section{Pela}
\begin{itemize}
\item {Grp. gram.:f.}
\end{itemize}
Cada camada de cortiça nos sobreiros.
Acto de pelar.
\section{Pela}
\begin{itemize}
\item {fónica:pe-la}
\end{itemize}
Expressão contrahida, equivalente a \textunderscore por a\textunderscore , e formada de \textunderscore per\textunderscore  + \textunderscore a\textunderscore , ou de \textunderscore per\textunderscore  + \textunderscore la\textunderscore .
Alguns escrevem \textunderscore pe'la\textunderscore .
\section{Pelacil}
\begin{itemize}
\item {Grp. gram.:m.}
\end{itemize}
O mesmo que \textunderscore vindima\textunderscore .
Colheita do azeite.
\section{Pelacir}
\begin{itemize}
\item {Grp. gram.:m.}
\end{itemize}
\begin{itemize}
\item {Utilização:Ant.}
\end{itemize}
O mesmo que \textunderscore vindima\textunderscore .
Colheita do azeite.
\section{Pelada}
\begin{itemize}
\item {Grp. gram.:f.}
\end{itemize}
\begin{itemize}
\item {Utilização:Prov.}
\end{itemize}
\begin{itemize}
\item {Utilização:alent.}
\end{itemize}
\begin{itemize}
\item {Utilização:Med.}
\end{itemize}
\begin{itemize}
\item {Proveniência:(De \textunderscore pelado\textunderscore )}
\end{itemize}
Clareira no mato.
Dermatose, que ataca as regiões pilosas do corpo e é caracterizada por alopecia rápida.
\section{Pelado}
\begin{itemize}
\item {Grp. gram.:adj.}
\end{itemize}
\begin{itemize}
\item {Utilização:Fam.}
\end{itemize}
\begin{itemize}
\item {Grp. gram.:M.}
\end{itemize}
Que não tem pêlo.
Calvo.
Finório.
Aquelle que é calvo.
\section{Pelador}
\begin{itemize}
\item {Grp. gram.:m.  e  adj.}
\end{itemize}
Aquelle que pela.
\section{Peladura}
\begin{itemize}
\item {Grp. gram.:f.}
\end{itemize}
Acto de \textunderscore pelar\textunderscore ^1.
\section{Pelage}
\begin{itemize}
\item {Grp. gram.:f.}
\end{itemize}
O mesmo que \textunderscore pelagem\textunderscore .
Todo o pêlo que cobre qualquer animal. Cf. Baganha, \textunderscore Hyg. Pec.\textunderscore , 105.
\section{Pelagem}
\begin{itemize}
\item {Grp. gram.:f.}
\end{itemize}
O pêlo dos animaes.
\section{Pelágia}
\begin{itemize}
\item {Grp. gram.:f.}
\end{itemize}
\begin{itemize}
\item {Proveniência:(Do lat. \textunderscore pelagus\textunderscore )}
\end{itemize}
Espécie de alforreca.
Gênero de polypeiros.
\section{Pelagiano}
\begin{itemize}
\item {Grp. gram.:m.}
\end{itemize}
\begin{itemize}
\item {Proveniência:(De \textunderscore pélago\textunderscore )}
\end{itemize}
O mesmo que \textunderscore albatroz\textunderscore .
\section{Pelágico}
\begin{itemize}
\item {Grp. gram.:adj.}
\end{itemize}
Relativo ao pélago; marítimo, oceânico.
\section{Pelágio}
\begin{itemize}
\item {Grp. gram.:adj.}
\end{itemize}
\begin{itemize}
\item {Utilização:P. us.}
\end{itemize}
\begin{itemize}
\item {Proveniência:(Gr. \textunderscore pelagios\textunderscore )}
\end{itemize}
O mesmo que \textunderscore pelágico\textunderscore .
\section{Pelágio-noctiluco}
\begin{itemize}
\item {Grp. gram.:m.}
\end{itemize}
Celenterado phosphorescente, com a fórma de medusa.
\section{Pélago}
\begin{itemize}
\item {Grp. gram.:m.}
\end{itemize}
\begin{itemize}
\item {Utilização:Fig.}
\end{itemize}
\begin{itemize}
\item {Utilização:Ant.}
\end{itemize}
\begin{itemize}
\item {Proveniência:(Lat. \textunderscore pelagus\textunderscore )}
\end{itemize}
Profundidade do mar.
Mar alto; Oceano.
Abysmo; profundidade; immensidade.
Qualquer corrente de água, ribeiro, rio, etc.
Tanque.
\section{Pelagosáurio}
\begin{itemize}
\item {fónica:sau}
\end{itemize}
\begin{itemize}
\item {Grp. gram.:m.}
\end{itemize}
\begin{itemize}
\item {Proveniência:(Do gr. \textunderscore pelagos\textunderscore  + \textunderscore sauros\textunderscore )}
\end{itemize}
Gênero de reptis, parecidos ao crocodilo.
\section{Pelagoscopia}
\begin{itemize}
\item {Grp. gram.:f.}
\end{itemize}
Arte de examinar o fundo das águas.
(Cp. \textunderscore pelagoscópio\textunderscore )
\section{Pelagoscópico}
\begin{itemize}
\item {Grp. gram.:adj.}
\end{itemize}
Relativo á pelagoscopia.
\section{Pelagoscópio}
\begin{itemize}
\item {Grp. gram.:m.}
\end{itemize}
\begin{itemize}
\item {Proveniência:(Do gr. \textunderscore pelagos\textunderscore  + \textunderscore skopein\textunderscore )}
\end{itemize}
Instrumento, para observar os objectos collocados no fundo da água.
\section{Pelagosito}
\begin{itemize}
\item {Grp. gram.:m.}
\end{itemize}
\begin{itemize}
\item {Utilização:Miner.}
\end{itemize}
\begin{itemize}
\item {Proveniência:(De \textunderscore pélago\textunderscore )}
\end{itemize}
Encrustação calcária, de origem marinha.
\section{Pelagossáurio}
\begin{itemize}
\item {Grp. gram.:m.}
\end{itemize}
\begin{itemize}
\item {Proveniência:(Do gr. \textunderscore pelagos\textunderscore  + \textunderscore sauros\textunderscore )}
\end{itemize}
Gênero de reptis, parecidos ao crocodilo.
\section{Pelagra}
\begin{itemize}
\item {Grp. gram.:f.}
\end{itemize}
\begin{itemize}
\item {Utilização:Med.}
\end{itemize}
\begin{itemize}
\item {Proveniência:(T. hybr., do lat. \textunderscore pellis\textunderscore  + gr. \textunderscore agra\textunderscore )}
\end{itemize}
Doença, que principia por certos simptomas na pele, seguidos de graves alterações na membrana mucosa do canal digestivo, a que sucedem perturbações do sistema nervoso e a morte.
\section{Pelagroso}
\begin{itemize}
\item {Grp. gram.:adj.}
\end{itemize}
\begin{itemize}
\item {Grp. gram.:M.}
\end{itemize}
Relativo á pelagra.
O doente de pelagra.
\section{Pelame}
\begin{itemize}
\item {Grp. gram.:m.}
\end{itemize}
O mesmo que \textunderscore pelagem\textunderscore .
Operação de mergulhar as pelles em água e cal, para se lhes tirar o pêlo.
Tanque, para curtir coiros ou pelles. Cf. \textunderscore Port. Ant. e Mod.\textunderscore  vb. \textunderscore Nicolau\textunderscore , p. 61 e 70.
\section{Pelame}
\begin{itemize}
\item {Grp. gram.:m.}
\end{itemize}
\begin{itemize}
\item {Proveniência:(Do lat. hyp. \textunderscore pellamen\textunderscore )}
\end{itemize}
Porção de peles; pele de animaes; coirama.
\section{Pelanca}
\begin{itemize}
\item {Grp. gram.:f.}
\end{itemize}
\begin{itemize}
\item {Proveniência:(De \textunderscore pelle\textunderscore )}
\end{itemize}
Pele mole e pendente.
Carne magra ou engelhada.
\section{Pelanga}
\begin{itemize}
\item {Grp. gram.:f.}
\end{itemize}
\begin{itemize}
\item {Proveniência:(De \textunderscore pele\textunderscore )}
\end{itemize}
Pele mole e pendente.
Carne magra ou engelhada.
\section{Pelangana}
\begin{itemize}
\item {Grp. gram.:f.}
\end{itemize}
\begin{itemize}
\item {Proveniência:(De \textunderscore pele\textunderscore )}
\end{itemize}
Pelanga.
\section{Pelão}
\begin{itemize}
\item {Grp. gram.:m.}
\end{itemize}
\begin{itemize}
\item {Utilização:Des.}
\end{itemize}
Ricaço, com pouca intelligência.
Magnate de segunda ordem.
Janota safado. Cf. \textunderscore Hyssope\textunderscore , 16 e 94; Filinto, VIII, 30 e 299; \textunderscore D. Man.\textunderscore , III, 76.
(Talvez t. ind.)
\section{Pelão}
\begin{itemize}
\item {Grp. gram.:m.}
\end{itemize}
\begin{itemize}
\item {Utilização:Prov.}
\end{itemize}
\begin{itemize}
\item {Utilização:alent.}
\end{itemize}
\begin{itemize}
\item {Proveniência:(De \textunderscore pêlo\textunderscore ^1)}
\end{itemize}
Us. na loc. \textunderscore em pelão\textunderscore , em pêlo, nu.
\section{Pelar}
\begin{itemize}
\item {Grp. gram.:v. t.}
\end{itemize}
Tirar o pêlo a.
\section{Pelar}
\begin{itemize}
\item {Grp. gram.:v. t.}
\end{itemize}
\begin{itemize}
\item {Grp. gram.:V. p.}
\end{itemize}
\begin{itemize}
\item {Utilização:Fig.}
\end{itemize}
Tirar a pele de; descascar.
Ficar sem pele, por esta lhe caír.
Gostar muito: \textunderscore o Rui pela-se por nozes\textunderscore .
\section{Pelar}
\begin{itemize}
\item {Grp. gram.:v.}
\end{itemize}
\begin{itemize}
\item {Utilização:t. Marn.}
\end{itemize}
\begin{itemize}
\item {Proveniência:(De \textunderscore péla\textunderscore ^1)}
\end{itemize}
Passar de mão em mão ou formigar (o torrão para as motas).
\section{Pelargonato}
\begin{itemize}
\item {Grp. gram.:m.}
\end{itemize}
Combinação do ácido pelargónico com uma base.
\section{Pelargónico}
\begin{itemize}
\item {Grp. gram.:adj.}
\end{itemize}
Diz-se de um ácido, que existe no óleo de pelargónio rosado.
\section{Pelargónio}
\begin{itemize}
\item {Grp. gram.:m.}
\end{itemize}
\begin{itemize}
\item {Proveniência:(Do gr. \textunderscore pelargos\textunderscore )}
\end{itemize}
Gênero de plantas ornamentaes, da fam. das geraniáceas.
Espécie de borragem.
\section{Pelaria}
\begin{itemize}
\item {Grp. gram.:f.}
\end{itemize}
Pelame; estabelecimento, onde se vendem peles.
\section{Pelásgico}
\begin{itemize}
\item {Grp. gram.:adj.}
\end{itemize}
Relativo aos Pelasgos ou ao seu tempo.
\section{Pelasgos}
\begin{itemize}
\item {Grp. gram.:m. pl.}
\end{itemize}
\begin{itemize}
\item {Proveniência:(Lat. \textunderscore Pelasgi\textunderscore )}
\end{itemize}
Habitantes primitivos da Grécia e da Itália.
\section{Pelataria}
\begin{itemize}
\item {Grp. gram.:f.}
\end{itemize}
Comércio de peles; pelaria. Cf. Latino, \textunderscore Hist. Pol. e Mil.\textunderscore , II, 253.
(Cp. \textunderscore pelatina\textunderscore )
\section{Pelatina}
\begin{itemize}
\item {Grp. gram.:f.}
\end{itemize}
\begin{itemize}
\item {Utilização:Ant.}
\end{itemize}
Enfeite de peles, usado ao pescoço por mulheres, estendendo-se até os joelhos.
O mesmo que \textunderscore boma\textunderscore ^1.
(Cp. \textunderscore pelitina\textunderscore )
\section{Peldraca}
\begin{itemize}
\item {Grp. gram.:f.}
\end{itemize}
\begin{itemize}
\item {Utilização:Prov.}
\end{itemize}
\begin{itemize}
\item {Utilização:trasm.}
\end{itemize}
O mesmo que \textunderscore pelanca\textunderscore .
\section{Pele}
\begin{itemize}
\item {Grp. gram.:f.}
\end{itemize}
\begin{itemize}
\item {Utilização:Fam.}
\end{itemize}
\begin{itemize}
\item {Grp. gram.:Loc.}
\end{itemize}
\begin{itemize}
\item {Utilização:Fam.}
\end{itemize}
\begin{itemize}
\item {Proveniência:(Lat. \textunderscore pellis\textunderscore )}
\end{itemize}
Membrana espessa, que envolve e cobre exteriormente todas as partes do corpo humano, bem como do corpo dos animaes vertebrados e de muitos animaes sem vértebras.
Epiderme.
Casca dos frutos e legumes.
Parte da pele, flácida e pendente.
Parte coriácea da carne comestível.
Coiro dos animaes, separado do corpo.
O mesmo que \textunderscore corpo\textunderscore : \textunderscore não lhe queria estar na pele\textunderscore .
\textunderscore Jurar pela pele de\textunderscore , protestar que fará mal a.
\section{Pelebreu}
\begin{itemize}
\item {Grp. gram.:m.}
\end{itemize}
\begin{itemize}
\item {Utilização:Bras. do N}
\end{itemize}
Pintainho pelado.
\section{Peleca}
\begin{itemize}
\item {Grp. gram.:f.}
\end{itemize}
\begin{itemize}
\item {Utilização:Prov.}
\end{itemize}
\begin{itemize}
\item {Utilização:trasm.}
\end{itemize}
Pequena pele.
\section{Pelechar}
\begin{itemize}
\item {Grp. gram.:v. i.}
\end{itemize}
\begin{itemize}
\item {Utilização:Bras. do S}
\end{itemize}
\begin{itemize}
\item {Proveniência:(Do cast. \textunderscore pelo\textunderscore  + \textunderscore echar\textunderscore )}
\end{itemize}
Mudar de pêlo o animal.
\section{Pelega}
\begin{itemize}
\item {fónica:lê}
\end{itemize}
\begin{itemize}
\item {Grp. gram.:f.}
\end{itemize}
\begin{itemize}
\item {Utilização:Bras}
\end{itemize}
\begin{itemize}
\item {Proveniência:(De \textunderscore pelego\textunderscore )}
\end{itemize}
Grande nota de banco, nota de valor.
\section{Pelegada}
\begin{itemize}
\item {Grp. gram.:m.}
\end{itemize}
Bando de pelegos ou homens rústicos.
Porção de labregos.
\section{Pelego}
\begin{itemize}
\item {fónica:lê}
\end{itemize}
\begin{itemize}
\item {Grp. gram.:m.}
\end{itemize}
\begin{itemize}
\item {Utilização:Bras}
\end{itemize}
\begin{itemize}
\item {Utilização:Pop.}
\end{itemize}
\begin{itemize}
\item {Utilização:Fig.}
\end{itemize}
\begin{itemize}
\item {Proveniência:(Do cast. \textunderscore pellejo\textunderscore )}
\end{itemize}
Pele de carneiro, servindo de xairel.
Labrego, homem rústico.
Dificuldade.
\section{Pelego}
\begin{itemize}
\item {fónica:lê}
\end{itemize}
\begin{itemize}
\item {Grp. gram.:m.}
\end{itemize}
\begin{itemize}
\item {Utilização:Prov.}
\end{itemize}
\begin{itemize}
\item {Proveniência:(De \textunderscore pêlo\textunderscore ?)}
\end{itemize}
Homem bronco, labrosta; patego.
\section{Pelegrinar}
\begin{itemize}
\item {Grp. gram.:v. i.}
\end{itemize}
\begin{itemize}
\item {Utilização:ant.}
\end{itemize}
\begin{itemize}
\item {Utilização:Pop.}
\end{itemize}
\begin{itemize}
\item {Proveniência:(De \textunderscore pelegrino\textunderscore )}
\end{itemize}
O mesmo que \textunderscore peregrinar\textunderscore . Cf. Usque, 8.
\section{Pelegrino}
\begin{itemize}
\item {Grp. gram.:m.}
\end{itemize}
\begin{itemize}
\item {Utilização:pop.}
\end{itemize}
\begin{itemize}
\item {Utilização:Ant.}
\end{itemize}
O mesmo que \textunderscore peregrino\textunderscore .
\section{Peleia}
\begin{itemize}
\item {Grp. gram.:f.}
\end{itemize}
\begin{itemize}
\item {Utilização:Constr.}
\end{itemize}
Angulo, formado por duas peças de madeira, sôbre as quaes se firmam as tábuas de um andaime: \textunderscore alugam-se bailéus e peleias\textunderscore .
\section{Peleira}
\begin{itemize}
\item {Grp. gram.:f.}
\end{itemize}
\begin{itemize}
\item {Utilização:T. de Pare -de-Coira}
\end{itemize}
\begin{itemize}
\item {Utilização:des.}
\end{itemize}
Mulhér, incumbida de arrancar o pêlo das pelles para a fula.
O mesmo que \textunderscore doença\textunderscore .
Tareia, sova.
(Cp. \textunderscore pêlo\textunderscore )
\section{Peleira}
\begin{itemize}
\item {Grp. gram.:f.}
\end{itemize}
\begin{itemize}
\item {Utilização:Prov.}
\end{itemize}
\begin{itemize}
\item {Utilização:trasm.}
\end{itemize}
Grande bebedeira.
(Talvez por \textunderscore pieleira\textunderscore , de \textunderscore piela\textunderscore )
\section{Peleira}
\begin{itemize}
\item {Grp. gram.:f.}
\end{itemize}
\begin{itemize}
\item {Utilização:Prov.}
\end{itemize}
\begin{itemize}
\item {Utilização:trasm.}
\end{itemize}
Fraqueza; doença.
Embaraço, pelego.
\section{Peleiro}
\begin{itemize}
\item {Grp. gram.:m.}
\end{itemize}
Vendedor de peles.
Aquele que as prepara para o comércio.
\section{Peleja}
\begin{itemize}
\item {Grp. gram.:f.}
\end{itemize}
Acto de pelejar.
Combate.
Briga; contenda; ralhos.
\section{Pelejador}
\begin{itemize}
\item {Grp. gram.:m.}
\end{itemize}
Aquelle que peleja.
Bulhento, desordeiro. Cf. Herculano, \textunderscore Hist. de Port.\textunderscore , II, 91 e 96.
\section{Pelejante}
\begin{itemize}
\item {Grp. gram.:adj.}
\end{itemize}
Que peleja. Cf. Castilho, \textunderscore D. Quixote\textunderscore , II, 93.
\section{Pelejar}
\begin{itemize}
\item {Grp. gram.:v. i.}
\end{itemize}
\begin{itemize}
\item {Utilização:Ext.}
\end{itemize}
\begin{itemize}
\item {Grp. gram.:V. t.}
\end{itemize}
Travar luta ou combate.
Brigar; bater-se.
Pugnar.
Desavir-se; discutir com vehemência.
Travar (combate, luta, etc.)
(Cast. \textunderscore peleare\textunderscore )
\section{Pelejo}
\begin{itemize}
\item {Grp. gram.:m.}
\end{itemize}
\begin{itemize}
\item {Utilização:Ant.}
\end{itemize}
Saio.
Pele.
Prostituta.
\section{Pelém}
\begin{itemize}
\item {Grp. gram.:m.  e  adj.}
\end{itemize}
\begin{itemize}
\item {Utilização:Prov.}
\end{itemize}
\begin{itemize}
\item {Utilização:beir.}
\end{itemize}
\begin{itemize}
\item {Utilização:trasm.}
\end{itemize}
Homem entanguido.
Magrizela.
Chòchinha; inhenho. Cf. Camillo, \textunderscore Brasileira\textunderscore , 27.
\section{Peleôa}
\begin{itemize}
\item {Grp. gram.:f.}
\end{itemize}
\begin{itemize}
\item {Utilização:Ant.}
\end{itemize}
Mulhér rixosa, desordeira.
(Por \textunderscore pelejôa\textunderscore , de um hypoth. \textunderscore pelejão\textunderscore , de \textunderscore pelejar\textunderscore )
\section{Peleona}
\begin{itemize}
\item {Grp. gram.:f.}
\end{itemize}
\begin{itemize}
\item {Utilização:Des.}
\end{itemize}
Rixa, contenda.
(Cast. \textunderscore peleona\textunderscore )
\section{Pé-leve}
\begin{itemize}
\item {Grp. gram.:m.}
\end{itemize}
\begin{itemize}
\item {Utilização:Bras. da Baía}
\end{itemize}
Vadio.
Sujeito reles.
\section{Pélevi}
\begin{itemize}
\item {Grp. gram.:m.}
\end{itemize}
Antigo idioma árico, falado na Média.
\section{Peléxia}
\begin{itemize}
\item {fónica:csi}
\end{itemize}
\begin{itemize}
\item {Grp. gram.:f.}
\end{itemize}
Gênero de orchídeas.
\section{Pelhancas}
\begin{itemize}
\item {Grp. gram.:f. pl.}
\end{itemize}
\begin{itemize}
\item {Utilização:Pop.}
\end{itemize}
O mesmo que \textunderscore pelharancas\textunderscore .
\section{Pelharancas}
\begin{itemize}
\item {Grp. gram.:f. pl.}
\end{itemize}
\begin{itemize}
\item {Utilização:Pop.}
\end{itemize}
O mesmo que \textunderscore pellanga\textunderscore . Cf. Filinto, VII, 74.
\section{Pelica}
\begin{itemize}
\item {Grp. gram.:f.}
\end{itemize}
Pele fina, curtida e preparada para luvas, etc.
\section{Peliça}
\begin{itemize}
\item {Grp. gram.:f.}
\end{itemize}
Peça do vestuário, feita ou forrada de peles finas e macias.
(B. lat. \textunderscore pellicia\textunderscore )
\section{Pelicanas}
\begin{itemize}
\item {Grp. gram.:f. pl.}
\end{itemize}
\begin{itemize}
\item {Utilização:Prov.}
\end{itemize}
\begin{itemize}
\item {Utilização:trasm.}
\end{itemize}
Arrecadas, brincos.
\section{Pelicanídeo}
\begin{itemize}
\item {Grp. gram.:adj.}
\end{itemize}
\begin{itemize}
\item {Grp. gram.:M. pl.}
\end{itemize}
Relativo ou semelhante ao pelicano.
Família de aves, que tem por typo o pelicano.
\section{Pelicano}
\begin{itemize}
\item {Grp. gram.:m.}
\end{itemize}
\begin{itemize}
\item {Proveniência:(Lat. \textunderscore pelicanus\textunderscore )}
\end{itemize}
Ave aquática palmípede.
Instrumento, para arrancar dentes.
Antiga peça de artilharia.
\section{Pelicanos}
\begin{itemize}
\item {Grp. gram.:m. pl.}
\end{itemize}
\begin{itemize}
\item {Utilização:Prov.}
\end{itemize}
\begin{itemize}
\item {Utilização:trasm.}
\end{itemize}
\begin{itemize}
\item {Proveniência:(De \textunderscore pêlo\textunderscore )}
\end{itemize}
O mesmo que \textunderscore pendorelhos\textunderscore .
\section{Pelicária}
\begin{itemize}
\item {Grp. gram.:f.}
\end{itemize}
Nome de duas plantas da serra de Sintra, \textunderscore pelicária maior\textunderscore  e \textunderscore pelicária menor\textunderscore .
\section{Pelicaria}
\begin{itemize}
\item {Grp. gram.:f.}
\end{itemize}
\begin{itemize}
\item {Utilização:Ant.}
\end{itemize}
\begin{itemize}
\item {Proveniência:(De \textunderscore pelica\textunderscore )}
\end{itemize}
Grande porção de peles.
Indústria de peleiro.
Fábrica, em que se preparam peles. Cf. \textunderscore Foral de Pombeiro da Beira\textunderscore .
\section{Peliceiro}
\begin{itemize}
\item {Grp. gram.:m.}
\end{itemize}
\begin{itemize}
\item {Utilização:Ant.}
\end{itemize}
\begin{itemize}
\item {Proveniência:(Do b. lat. \textunderscore pelliciarius\textunderscore )}
\end{itemize}
Curtidor de peles.
\section{Pelico}
\begin{itemize}
\item {Grp. gram.:m.}
\end{itemize}
Fato de péles de carneiro.
Secundinas ou envoltório do féto no ventre materno. Cf. Bernárdez, \textunderscore Luz e Calor\textunderscore , 6.
\section{Pelicreiro}
\begin{itemize}
\item {Grp. gram.:m.}
\end{itemize}
\begin{itemize}
\item {Utilização:Prov.}
\end{itemize}
\begin{itemize}
\item {Utilização:trasm.}
\end{itemize}
\begin{itemize}
\item {Proveniência:(Do lat. hyp. \textunderscore pellicularius\textunderscore )}
\end{itemize}
Negociante de peles.
\section{Película}
\begin{itemize}
\item {Grp. gram.:f.}
\end{itemize}
\begin{itemize}
\item {Utilização:Bot.}
\end{itemize}
\begin{itemize}
\item {Proveniência:(Lat. \textunderscore pellicula\textunderscore )}
\end{itemize}
Pele delgada e fina.
Epiderme.
Membrana muito delgada, que envolve certos órgãos.
\section{Pelintra}
\begin{itemize}
\item {Grp. gram.:m.  e  f.}
\end{itemize}
\begin{itemize}
\item {Grp. gram.:Adj.}
\end{itemize}
\begin{itemize}
\item {Utilização:Bras}
\end{itemize}
Pessôa pobre ou mal vestida, mas com pretensões de figurar ou dar na vista.
Pessôa, que não tem dinheiro nenhum.
Miserável ou mal trajado, mas pretensioso.
Próprio de quem não tem nada e pretende mostrar que tem alguma coisa.
Esfarrapado; maltrapilho.
Bem trajado; peralta; adamado.
\section{Pelintrante}
\begin{itemize}
\item {Grp. gram.:m.}
\end{itemize}
\begin{itemize}
\item {Utilização:Prov.}
\end{itemize}
\begin{itemize}
\item {Utilização:minh.}
\end{itemize}
O mesmo que \textunderscore pelitrate\textunderscore ; pelintrão.
\section{Pelintrão}
\begin{itemize}
\item {Grp. gram.:m.}
\end{itemize}
\begin{itemize}
\item {Utilização:Pop.}
\end{itemize}
\begin{itemize}
\item {Proveniência:(De \textunderscore pelintra\textunderscore )}
\end{itemize}
Sujeito maltrapilho, esfarrapado.
\section{Pelintrar}
\begin{itemize}
\item {Grp. gram.:v. t.}
\end{itemize}
Reduzir ao estado de pelintra. Cf. Camillo, \textunderscore Corja\textunderscore , 169.
\section{Pelintraria}
\begin{itemize}
\item {Grp. gram.:f.}
\end{itemize}
Quantidade de pelintras; os pelintras.
Pelintrice. Cf. Camillo, \textunderscore Noites de Insómn.\textunderscore , VIII, 83.
\section{Pelintrice}
\begin{itemize}
\item {Grp. gram.:f.}
\end{itemize}
Acto ou estado de pelintra; sovinice.
\section{Pelintrismo}
\begin{itemize}
\item {Grp. gram.:m.}
\end{itemize}
O mesmo que \textunderscore pelintrice\textunderscore .
\section{Pelioma}
\begin{itemize}
\item {Grp. gram.:m.}
\end{itemize}
\begin{itemize}
\item {Utilização:Med.}
\end{itemize}
\begin{itemize}
\item {Proveniência:(Gr. \textunderscore pelioma\textunderscore )}
\end{itemize}
Mancha lívida na pelle.
\section{Pelítico}
\begin{itemize}
\item {Grp. gram.:adj.}
\end{itemize}
\begin{itemize}
\item {Utilização:Geol.}
\end{itemize}
\begin{itemize}
\item {Proveniência:(Do gr. \textunderscore pelos\textunderscore , lodo)}
\end{itemize}
Diz-se das rochas, cujos grãos são indistintos a ôlho nu, porque resultam do endurecimento de massas lodosas.
\section{Pelitrapo}
\begin{itemize}
\item {Grp. gram.:m.}
\end{itemize}
\begin{itemize}
\item {Utilização:Des.}
\end{itemize}
Indivíduo roto ou mal vestido; maltrapilho. Cf. \textunderscore Diccion. de Nomes, Vozes e Coisas...\textunderscore 
(Por \textunderscore pelintrapo\textunderscore , de \textunderscore pelintra\textunderscore ? Ou de \textunderscore pelle\textunderscore  + \textunderscore trapo\textunderscore ?)
\section{Pelitrate}
\begin{itemize}
\item {Grp. gram.:m.}
\end{itemize}
O mesmo que \textunderscore pelitrapo\textunderscore . Cf. Camillo, \textunderscore Cancion. Al.\textunderscore , 190.
\section{Pelitre}
\begin{itemize}
\item {Grp. gram.:m.}
\end{itemize}
(Corr. de \textunderscore pýrethro\textunderscore )
\section{Pella}
\begin{itemize}
\item {Grp. gram.:f.}
\end{itemize}
Cada camada de cortiça nos sobreiros.
Acto de pellar.
\section{Pellador}
\begin{itemize}
\item {Grp. gram.:m.  e  adj.}
\end{itemize}
Aquelle que pella.
\section{Pelladura}
\begin{itemize}
\item {Grp. gram.:f.}
\end{itemize}
Acto ou effeito de pellar.
\section{Pellagra}
\begin{itemize}
\item {Grp. gram.:f.}
\end{itemize}
\begin{itemize}
\item {Utilização:Med.}
\end{itemize}
\begin{itemize}
\item {Proveniência:(T. hybr., do lat. \textunderscore pellis\textunderscore  + gr. \textunderscore agra\textunderscore )}
\end{itemize}
Doença, que principia por certos symptomas na pelle, seguidos de graves alterações na membrana mucosa do canal digestivo, a que succedem perturbações do systema nervoso e a morte.
\section{Pellagroso}
\begin{itemize}
\item {Grp. gram.:adj.}
\end{itemize}
\begin{itemize}
\item {Grp. gram.:M.}
\end{itemize}
Relativo á pellagra.
O doente de pellagra.
\section{Pellame}
\begin{itemize}
\item {Grp. gram.:m.}
\end{itemize}
\begin{itemize}
\item {Proveniência:(Do lat. hyp. \textunderscore pellamen\textunderscore )}
\end{itemize}
Porção de pelles; pelle de animaes; coirama.
\section{Pellanca}
\begin{itemize}
\item {Grp. gram.:f.}
\end{itemize}
\begin{itemize}
\item {Proveniência:(De \textunderscore pelle\textunderscore )}
\end{itemize}
Pelle molle e pendente.
Carne magra ou engelhada.
\section{Pellanga}
\begin{itemize}
\item {Grp. gram.:f.}
\end{itemize}
\begin{itemize}
\item {Proveniência:(De \textunderscore pelle\textunderscore )}
\end{itemize}
Pelle molle e pendente.
Carne magra ou engelhada.
\section{Pellangana}
\begin{itemize}
\item {Grp. gram.:f.}
\end{itemize}
\begin{itemize}
\item {Proveniência:(De \textunderscore pelle\textunderscore )}
\end{itemize}
Pellanga.
\section{Pellar}
\begin{itemize}
\item {Grp. gram.:v. t.}
\end{itemize}
\begin{itemize}
\item {Grp. gram.:V. p.}
\end{itemize}
\begin{itemize}
\item {Utilização:Fig.}
\end{itemize}
Tirar a pelle de; descascar.
Ficar sem pelle, por esta lhe caír.
Gostar muito: \textunderscore o Rui pella-se por nozes\textunderscore .
\section{Pellaria}
\begin{itemize}
\item {Grp. gram.:f.}
\end{itemize}
Pellame; estabelecimento, onde se vendem pelles.
\section{Pellataria}
\begin{itemize}
\item {Grp. gram.:f.}
\end{itemize}
Commércio de pelles; pellaria. Cf. Latino, \textunderscore Hist. Pol. e Mil.\textunderscore , II, 253.
(Cp. \textunderscore pellatina\textunderscore )
\section{Pellatina}
\begin{itemize}
\item {Grp. gram.:f.}
\end{itemize}
\begin{itemize}
\item {Utilização:Ant.}
\end{itemize}
Enfeite de pelles, usado ao pescoço por mulheres, estendendo-se até os joelhos.
O mesmo que \textunderscore boma\textunderscore ^1.
(Cp. \textunderscore pellitina\textunderscore )
\section{Pelle}
\begin{itemize}
\item {Grp. gram.:f.}
\end{itemize}
\begin{itemize}
\item {Utilização:Fam.}
\end{itemize}
\begin{itemize}
\item {Grp. gram.:Loc.}
\end{itemize}
\begin{itemize}
\item {Utilização:Fam.}
\end{itemize}
\begin{itemize}
\item {Proveniência:(Lat. \textunderscore pellis\textunderscore )}
\end{itemize}
Membrana espessa, que envolve e cobre exteriormente todas as partes do corpo humano, bem como do corpo dos animaes vertebrados e de muitos animaes sem vértebras.
Epiderme.
Casca dos frutos e legumes.
Parte da pelle, fláccida e pendente.
Parte coriácea da carne comestível.
Coiro dos animaes, separado do corpo.
O mesmo que \textunderscore corpo\textunderscore : \textunderscore não lhe queria estar na pelle\textunderscore .
\textunderscore Jurar pela pelle de\textunderscore , protestar que fará mal a.
\section{Pelleca}
\begin{itemize}
\item {Grp. gram.:f.}
\end{itemize}
\begin{itemize}
\item {Utilização:Prov.}
\end{itemize}
\begin{itemize}
\item {Utilização:trasm.}
\end{itemize}
Pequena pelle.
\section{Pelle-de-sapo}
\begin{itemize}
\item {Grp. gram.:f.}
\end{itemize}
O mesmo que \textunderscore mançanica\textunderscore .
\section{Pelle-de-vinho}
\begin{itemize}
\item {Grp. gram.:f.}
\end{itemize}
Antiga medida portuguesa, correspondente a 3 almudes.
\section{Pelle-do-diabo}
\begin{itemize}
\item {Grp. gram.:f.}
\end{itemize}
Espécie de tecido, muito consistente e duradoiro, usado em fato de homem e também conhecido por \textunderscore dura-sempre\textunderscore .
(Colhido na Bairrada)
\section{Pellega}
\begin{itemize}
\item {fónica:lê}
\end{itemize}
\begin{itemize}
\item {Grp. gram.:f.}
\end{itemize}
\begin{itemize}
\item {Utilização:Bras}
\end{itemize}
\begin{itemize}
\item {Proveniência:(De \textunderscore pellego\textunderscore )}
\end{itemize}
Grande nota de banco, nota de valor.
\section{Pellegada}
\begin{itemize}
\item {Grp. gram.:m.}
\end{itemize}
Bando de pellegos ou homens rústicos.
Porção de labregos.
\section{Pellego}
\begin{itemize}
\item {fónica:lê}
\end{itemize}
\begin{itemize}
\item {Grp. gram.:m.}
\end{itemize}
\begin{itemize}
\item {Utilização:Bras}
\end{itemize}
\begin{itemize}
\item {Utilização:Pop.}
\end{itemize}
\begin{itemize}
\item {Utilização:Fig.}
\end{itemize}
\begin{itemize}
\item {Proveniência:(Do cast. \textunderscore pellejo\textunderscore )}
\end{itemize}
Pelle de carneiro, servindo de xairel.
Labrego, homem rústico.
Difficuldade.
\section{Pelleira}
\begin{itemize}
\item {Grp. gram.:f.}
\end{itemize}
\begin{itemize}
\item {Utilização:Prov.}
\end{itemize}
\begin{itemize}
\item {Utilização:trasm.}
\end{itemize}
Fraqueza; doença.
Embaraço, pellego.
\section{Pelleiro}
\begin{itemize}
\item {Grp. gram.:m.}
\end{itemize}
Vendedor de pelles.
Aquelle que as prepara para o commércio.
\section{Pellejo}
\begin{itemize}
\item {Grp. gram.:m.}
\end{itemize}
\begin{itemize}
\item {Utilização:Ant.}
\end{itemize}
Saio.
Pelle.
Prostituta.
\section{Pelles-vermelhas}
\begin{itemize}
\item {Grp. gram.:m. pl.}
\end{itemize}
Designação genérica das tríbos aborígenas da América do Norte.
\section{Pellica}
\begin{itemize}
\item {Grp. gram.:f.}
\end{itemize}
Pelle fina, curtida e preparada para luvas, etc.
\section{Pelliça}
\begin{itemize}
\item {Grp. gram.:f.}
\end{itemize}
Peça do vestuário, feita ou forrada de pelles finas e macias.
(B. lat. \textunderscore pellicia\textunderscore )
\section{Pellicaria}
\begin{itemize}
\item {Grp. gram.:f.}
\end{itemize}
\begin{itemize}
\item {Utilização:Ant.}
\end{itemize}
\begin{itemize}
\item {Proveniência:(De \textunderscore pellica\textunderscore )}
\end{itemize}
Grande porção de pelles.
Indústria de pelleiro.
Fábrica, em que se preparam pelles. Cf. \textunderscore Foral de Pombeiro da Beira\textunderscore .
\section{Pelliceiro}
\begin{itemize}
\item {Grp. gram.:m.}
\end{itemize}
\begin{itemize}
\item {Utilização:Ant.}
\end{itemize}
\begin{itemize}
\item {Proveniência:(Do b. lat. \textunderscore pelliciarius\textunderscore )}
\end{itemize}
Curtidor de pelles.
\section{Pellico}
\begin{itemize}
\item {Grp. gram.:m.}
\end{itemize}
Fato de pélles de carneiro.
Secundinas ou envoltório do féto no ventre materno. Cf. Bernárdez, \textunderscore Luz e Calor\textunderscore , 6.
\section{Pellicreiro}
\begin{itemize}
\item {Grp. gram.:m.}
\end{itemize}
\begin{itemize}
\item {Utilização:Prov.}
\end{itemize}
\begin{itemize}
\item {Utilização:trasm.}
\end{itemize}
\begin{itemize}
\item {Proveniência:(Do lat. hyp. \textunderscore pellicularius\textunderscore )}
\end{itemize}
Negociante de pelles.
\section{Pellícula}
\begin{itemize}
\item {Grp. gram.:f.}
\end{itemize}
\begin{itemize}
\item {Utilização:Bot.}
\end{itemize}
\begin{itemize}
\item {Proveniência:(Lat. \textunderscore pellicula\textunderscore )}
\end{itemize}
Pelle delgada e fina.
Epiderme.
Membrana muito delgada, que envolve certos órgãos.
\section{Pelicular}
\begin{itemize}
\item {Grp. gram.:adj.}
\end{itemize}
\begin{itemize}
\item {Utilização:Bot.}
\end{itemize}
\begin{itemize}
\item {Proveniência:(De \textunderscore pelicula\textunderscore )}
\end{itemize}
Diz-se do perisperma, formado de uma lâmina delgada, como o das labiadas.
\section{Pelidar}
\begin{itemize}
\item {Grp. gram.:v. i.}
\end{itemize}
\begin{itemize}
\item {Utilização:Pop.}
\end{itemize}
Invocar auxílio; gritar por soccorro; recorrer, soccorrer-se. Cf. Camillo, \textunderscore Suicída\textunderscore , 135.
(Aphérs. de \textunderscore appelidar\textunderscore )
\section{Peliqueiro}
\begin{itemize}
\item {Grp. gram.:m.}
\end{itemize}
Aquele que trabalha em pelica.
Vendedor de pelicas.
\section{Pelitaria}
\begin{itemize}
\item {Grp. gram.:f.}
\end{itemize}
Porção de peles; pele de animal; peliça:«\textunderscore Forrão suas roupas de pelitarias de vários animaes.\textunderscore »Filinto, \textunderscore D. Man.\textunderscore , III, 212.
(Cp. \textunderscore peliteiro\textunderscore )
\section{Peliteiro}
\begin{itemize}
\item {Grp. gram.:m.}
\end{itemize}
\begin{itemize}
\item {Utilização:Ant.}
\end{itemize}
O mesmo que \textunderscore peliceiro\textunderscore . Cf. \textunderscore Eufrosina\textunderscore , 143.
(B. lat. \textunderscore pelletarius\textunderscore )
\section{Pelitina}
\begin{itemize}
\item {Grp. gram.:f.}
\end{itemize}
\begin{itemize}
\item {Utilização:Ant.}
\end{itemize}
Peliça para agasalho do pescoço; pelatina.
(Cp. \textunderscore peliteiro\textunderscore )
\section{Pellicular}
\begin{itemize}
\item {Grp. gram.:adj.}
\end{itemize}
\begin{itemize}
\item {Utilização:Bot.}
\end{itemize}
\begin{itemize}
\item {Proveniência:(De \textunderscore pellicula\textunderscore )}
\end{itemize}
Diz-se do perisperma, formado de uma lâmina delgada, como o das labiadas.
\section{Pellidar}
\begin{itemize}
\item {Grp. gram.:v. i.}
\end{itemize}
\begin{itemize}
\item {Utilização:Pop.}
\end{itemize}
Invocar auxílio; gritar por soccorro; recorrer, soccorrer-se. Cf. Camillo, \textunderscore Suicída\textunderscore , 135.
(Aphérs. de \textunderscore appelidar\textunderscore )
\section{Pelliqueiro}
\begin{itemize}
\item {Grp. gram.:m.}
\end{itemize}
Aquelle que trabalha em pellica.
Vendedor de pellicas.
\section{Pellitaria}
\begin{itemize}
\item {Grp. gram.:f.}
\end{itemize}
Porção de pelles; pelle de animal; pelliça:«\textunderscore Forrão suas roupas de pellitarias de vários animaes.\textunderscore »Filinto, \textunderscore D. Man.\textunderscore , III, 212.
(Cp. \textunderscore pelliteiro\textunderscore )
\section{Pelliteiro}
\begin{itemize}
\item {Grp. gram.:m.}
\end{itemize}
\begin{itemize}
\item {Utilização:Ant.}
\end{itemize}
O mesmo que \textunderscore pelliceiro\textunderscore . Cf. \textunderscore Eufrosina\textunderscore , 143.
(B. lat. \textunderscore pelletarius\textunderscore )
\section{Pellitina}
\begin{itemize}
\item {Grp. gram.:f.}
\end{itemize}
\begin{itemize}
\item {Utilização:Ant.}
\end{itemize}
Pelliça para agasalho do pescoço; pellatina.
(Cp. \textunderscore pelliteiro\textunderscore )
\section{Pellota}
\begin{itemize}
\item {Grp. gram.:f.}
\end{itemize}
\begin{itemize}
\item {Utilização:Bras}
\end{itemize}
\begin{itemize}
\item {Proveniência:(De \textunderscore pelle\textunderscore )}
\end{itemize}
Jangada de coiro.
\section{Pellote}
\begin{itemize}
\item {Grp. gram.:m.}
\end{itemize}
O mesmo que \textunderscore pelliça\textunderscore .
\section{Pellotina}
\begin{itemize}
\item {Grp. gram.:f.}
\end{itemize}
Alcalóide soporífico, extrahido de uma variedade de cacto mexicano.
\section{Pêlo}
\begin{itemize}
\item {Grp. gram.:m.}
\end{itemize}
\begin{itemize}
\item {Utilização:Prov.}
\end{itemize}
\begin{itemize}
\item {Utilização:minh.}
\end{itemize}
\begin{itemize}
\item {Grp. gram.:Loc. adv.}
\end{itemize}
\begin{itemize}
\item {Utilização:Bras. do S}
\end{itemize}
\begin{itemize}
\item {Grp. gram.:Loc.}
\end{itemize}
\begin{itemize}
\item {Utilização:pop.}
\end{itemize}
\begin{itemize}
\item {Proveniência:(Do lat. \textunderscore pilus\textunderscore )}
\end{itemize}
Fios delgados, que crescem na pelle dos animaes.
Cada um dêsses pêlos.
Cabellos.
Pennugem.
Campo de erva.
\textunderscore A pêlo\textunderscore , a proposito.
\textunderscore Viajar de pêlo a pêlo\textunderscore , fazer viagem sem mudar de cavalgadura.
\textunderscore Ir ao pêlo de alguém\textunderscore , bater-lhe.
\section{Pelo}
\begin{itemize}
\item {fónica:pe-loouplo}
\end{itemize}
Contr. da preposição \textunderscore per\textunderscore  + \textunderscore o\textunderscore  ou \textunderscore lo\textunderscore .--Os letrados e os diccion. costumam dizêr \textunderscore pêlo\textunderscore ; mas o carácter proclítico da palavra e pronúncia popular, (\textunderscore p'lo motivo\textunderscore , \textunderscore p'los modos\textunderscore ) indicam que é surda a modulação das duas vogaes da palavra: \textunderscore pe-lu...\textunderscore 
(Cp. \textunderscore polo\textunderscore )
\section{Peloemia}
\begin{itemize}
\item {fónica:lo-e}
\end{itemize}
\begin{itemize}
\item {Grp. gram.:f.}
\end{itemize}
\begin{itemize}
\item {Utilização:Veter.}
\end{itemize}
\begin{itemize}
\item {Proveniência:(Do gr. \textunderscore pelos\textunderscore , lodo, e \textunderscore haina\textunderscore , sangue)}
\end{itemize}
Estado do sangue, espêsso e denegrido em algumas moléstias de animaes.
\section{Pelohemia}
\begin{itemize}
\item {Grp. gram.:f.}
\end{itemize}
\begin{itemize}
\item {Utilização:Veter.}
\end{itemize}
\begin{itemize}
\item {Proveniência:(Do gr. \textunderscore pelos\textunderscore , lodo, e \textunderscore haina\textunderscore , sangue)}
\end{itemize}
Estado do sangue, espêsso e denegrido em algumas moléstias de animaes.
\section{Peloirada}
\begin{itemize}
\item {Grp. gram.:f.}
\end{itemize}
Tiro de peloiro.
\section{Peloirinho}
\begin{itemize}
\item {Grp. gram.:m.}
\end{itemize}
Columna de pedra, em praça ou sitio público, junto da qual se expunham e castigavam os criminosos.
(Talvez de \textunderscore peloiro\textunderscore , por allusão á esphera que encima ordinariamente essa columna)
\section{Peloiro}
\begin{itemize}
\item {Grp. gram.:m.}
\end{itemize}
\begin{itemize}
\item {Utilização:Ant.}
\end{itemize}
\begin{itemize}
\item {Grp. gram.:Pl.}
\end{itemize}
\begin{itemize}
\item {Proveniência:(Do b. lat. \textunderscore pilorium\textunderscore )}
\end{itemize}
Bala de metal, que se empregava em algumas peças de artilharia.
Cada um dos serviços, em que se costuma dividir a administração collectiva de um município.
Bóla de cera, em que se incluía o voto de cada eleitor.
Espécie de jôgo de rapazes.
\section{Pelome}
\begin{itemize}
\item {Grp. gram.:m.}
\end{itemize}
\begin{itemize}
\item {Utilização:T. de Alcanena}
\end{itemize}
\begin{itemize}
\item {Utilização:Ant.}
\end{itemize}
O mesmo que \textunderscore pelame\textunderscore ^1, tanque para curtir pelles. Cf. B. Pereira, vb. \textunderscore nautea\textunderscore .
\section{Pelopeias}
\begin{itemize}
\item {Grp. gram.:f. pl.}
\end{itemize}
Antigas festas gregas em honra de Pélops.
\section{Pelópico}
\begin{itemize}
\item {Grp. gram.:adj.}
\end{itemize}
Diz-se do ácido de pelópio.
\section{Pelópio}
\begin{itemize}
\item {Grp. gram.:m.}
\end{itemize}
Corpo mineral, mal conhecido ainda.
\section{Pelória}
\begin{itemize}
\item {Grp. gram.:f.}
\end{itemize}
\begin{itemize}
\item {Proveniência:(Do gr. \textunderscore pelor\textunderscore , prodígio)}
\end{itemize}
Monstruosidade vegetal, por alteração de fórma.
\section{Peloso}
\begin{itemize}
\item {Grp. gram.:adj.}
\end{itemize}
O mesmo que \textunderscore peludo\textunderscore .
\section{Pelota}
\begin{itemize}
\item {Grp. gram.:f.}
\end{itemize}
\begin{itemize}
\item {Utilização:Prov.}
\end{itemize}
\begin{itemize}
\item {Utilização:Prov.}
\end{itemize}
\begin{itemize}
\item {Utilização:alent.}
\end{itemize}
\begin{itemize}
\item {Utilização:Bras}
\end{itemize}
Pequena péla.
Bóla de metal.
Instrumento cirúrgico, para fazer compressões.
Almofada na funda herniária.
Espécie de almofada, com que o chapeleiro alisa os chapéus depois de engomados.
Cada um dos pedaços da massa do pão, que se tende separadamente.
Pequena bóla de neve.
Jôgo de rapazes.
(V.pellota)
\section{Pelota}
\begin{itemize}
\item {Grp. gram.:f.}
\end{itemize}
\begin{itemize}
\item {Utilização:Bras}
\end{itemize}
\begin{itemize}
\item {Proveniência:(De \textunderscore pele\textunderscore )}
\end{itemize}
Jangada de coiro.
\section{Pelotina}
\begin{itemize}
\item {Grp. gram.:f.}
\end{itemize}
Alcalóide soporífico, extraido de uma variedade de cacto mexicano.
\section{Pelotada}
\begin{itemize}
\item {Grp. gram.:f.}
\end{itemize}
\begin{itemize}
\item {Proveniência:(De \textunderscore pelota\textunderscore )}
\end{itemize}
Brinquedo com bólas de neve.
\section{Pelotanto}
\begin{itemize}
\item {Grp. gram.:Loc. conj.}
\end{itemize}
\begin{itemize}
\item {Proveniência:(De \textunderscore pelo\textunderscore  + \textunderscore tanto\textunderscore )}
\end{itemize}
O mesmo que \textunderscore portanto\textunderscore . Cf. Camillo, \textunderscore Caveira\textunderscore , 111.
\section{Pelotão}
\begin{itemize}
\item {Grp. gram.:m.}
\end{itemize}
Pelota grande.
Multidão.
Cada uma das três partes em que se divide uma companhia de soldados.
\section{Pelotar}
\begin{itemize}
\item {Grp. gram.:m.}
\end{itemize}
\begin{itemize}
\item {Utilização:Bras}
\end{itemize}
O mesmo que \textunderscore pelotário\textunderscore .
\section{Pelotário}
\begin{itemize}
\item {Grp. gram.:m.}
\end{itemize}
Jogador de pelota:«\textunderscore era conhecido no Rio como jogador e pelotário\textunderscore ». \textunderscore Século\textunderscore , de 12-II-900.
\section{Pelote}
\begin{itemize}
\item {Grp. gram.:m.}
\end{itemize}
O mesmo que \textunderscore peliça\textunderscore .
\section{Pelote}
\begin{itemize}
\item {Grp. gram.:m.}
\end{itemize}
\begin{itemize}
\item {Utilização:Fam.}
\end{itemize}
\begin{itemize}
\item {Proveniência:(De \textunderscore pêlo\textunderscore )}
\end{itemize}
Antigo vestuário de grandes abas.
Nudez.
\section{Pelotear}
\begin{itemize}
\item {Grp. gram.:v. i.}
\end{itemize}
\begin{itemize}
\item {Proveniência:(De \textunderscore pelota\textunderscore )}
\end{itemize}
Açoitar, maltratar. Cf. Filinto, XXII, 111.
\section{Peloteiro}
\begin{itemize}
\item {Grp. gram.:m.}
\end{itemize}
Aquelle que faz ou vende pelotas.
\section{Pelotense}
\begin{itemize}
\item {Grp. gram.:adj.}
\end{itemize}
Relativo a Pelotas, cidade brasileira.
\section{Pelotica}
\begin{itemize}
\item {Grp. gram.:f.}
\end{itemize}
\begin{itemize}
\item {Proveniência:(De \textunderscore pelota\textunderscore )}
\end{itemize}
Prestidigitação; sorte de habilidades com as mãos.
\section{Pelotiqueiro}
\begin{itemize}
\item {Grp. gram.:m.}
\end{itemize}
Aquelle que faz peloticas; saltimbanco.
\section{Pelourada}
\begin{itemize}
\item {Grp. gram.:f.}
\end{itemize}
Tiro de pelouro.
\section{Pelourinho}
\begin{itemize}
\item {Grp. gram.:m.}
\end{itemize}
Coluna de pedra, em praça ou sitio público, junto da qual se expunham e castigavam os criminosos.
(Talvez de \textunderscore pelouro\textunderscore , por alusão á esfera que encima ordinariamente essa coluna)
\section{Pelouro}
\begin{itemize}
\item {Grp. gram.:m.}
\end{itemize}
\begin{itemize}
\item {Utilização:Ant.}
\end{itemize}
\begin{itemize}
\item {Grp. gram.:Pl.}
\end{itemize}
\begin{itemize}
\item {Proveniência:(Do b. lat. \textunderscore pilorium\textunderscore )}
\end{itemize}
Bala de metal, que se empregava em algumas peças de artilharia.
Cada um dos serviços, em que se costuma dividir a administração colectiva de um município.
Bóla de cera, em que se incluía o voto de cada eleitor.
Espécie de jôgo de rapazes.
\section{Pelta}
\begin{itemize}
\item {Grp. gram.:f.}
\end{itemize}
\begin{itemize}
\item {Proveniência:(Lat. \textunderscore pelta\textunderscore )}
\end{itemize}
Pequeno escudo, em fórma de crescente, usado noutros tempos pelos Thrácios e outros povos antigos.
\section{Peltária}
\begin{itemize}
\item {Grp. gram.:f.}
\end{itemize}
\begin{itemize}
\item {Proveniência:(Do lat. \textunderscore pelta\textunderscore )}
\end{itemize}
Planta crucífera.
\section{Peltastos}
\begin{itemize}
\item {Grp. gram.:m. pl.}
\end{itemize}
\begin{itemize}
\item {Proveniência:(Do gr. \textunderscore pelte\textunderscore )}
\end{itemize}
Uma das espécies da infantaria regular, na phalange macedónica, onde não tinha lugar fixo, collocando-se ordinariamente atrás dos hoplitas.
\section{Peltífido}
\begin{itemize}
\item {Grp. gram.:adj.}
\end{itemize}
\begin{itemize}
\item {Utilização:Bot.}
\end{itemize}
\begin{itemize}
\item {Proveniência:(Do lat. \textunderscore pelta\textunderscore  + \textunderscore findere\textunderscore )}
\end{itemize}
Diz-se das fôlhas peltinérveas fendidas.
\section{Peltiforme}
\begin{itemize}
\item {Grp. gram.:adj.}
\end{itemize}
\begin{itemize}
\item {Utilização:Bot.}
\end{itemize}
\begin{itemize}
\item {Proveniência:(Do lat. \textunderscore pelta\textunderscore  + \textunderscore forma\textunderscore )}
\end{itemize}
Que tem fórma de pelta ou de pequeno escudo.
\section{Peltígera}
\begin{itemize}
\item {Grp. gram.:f.}
\end{itemize}
\begin{itemize}
\item {Proveniência:(Do lat. \textunderscore pelta\textunderscore  + \textunderscore gerere\textunderscore )}
\end{itemize}
Gênero de líchens.
\section{Peltinérveo}
\begin{itemize}
\item {Grp. gram.:adj.}
\end{itemize}
\begin{itemize}
\item {Utilização:Bot.}
\end{itemize}
\begin{itemize}
\item {Proveniência:(Do lat. \textunderscore pelta\textunderscore  + \textunderscore nervus\textunderscore )}
\end{itemize}
Diz-se das fôlhas, cujas nervuras são arredondadas e partem do ápice de pecíolo, afastando-se divergentemente sôbre o mesmo plano.
\section{Peltipartido}
\begin{itemize}
\item {Grp. gram.:adj.}
\end{itemize}
\begin{itemize}
\item {Utilização:Bot.}
\end{itemize}
\begin{itemize}
\item {Proveniência:(Do lat. \textunderscore pelta\textunderscore  + \textunderscore partitus\textunderscore )}
\end{itemize}
Diz-se das fôlhas peltinérveas, cujas divisões se acham soldadas entre si na sua base.
\section{Peltospermo}
\begin{itemize}
\item {Grp. gram.:m.}
\end{itemize}
\begin{itemize}
\item {Proveniência:(Do gr. \textunderscore pelte\textunderscore  + \textunderscore sperma\textunderscore )}
\end{itemize}
Gênero de plantas bignoniáceas.
\section{Peltre}
\begin{itemize}
\item {Grp. gram.:m.}
\end{itemize}
\begin{itemize}
\item {Proveniência:(Do al. \textunderscore spelter\textunderscore , zinco)}
\end{itemize}
Metal, talvez liga de cobre e estanho, com que se cunhou moéda em Portugal.
\section{Pelúcia}
\begin{itemize}
\item {Grp. gram.:f.}
\end{itemize}
\begin{itemize}
\item {Proveniência:(De \textunderscore pêlo\textunderscore ^1)}
\end{itemize}
Tecido de lan, seda, etc., felpudo de um lado.
\section{Peluda}
\begin{itemize}
\item {Grp. gram.:f.}
\end{itemize}
\begin{itemize}
\item {Utilização:mil.}
\end{itemize}
\begin{itemize}
\item {Utilização:Gír.}
\end{itemize}
\begin{itemize}
\item {Proveniência:(De \textunderscore peludo\textunderscore )}
\end{itemize}
Vida de paisano: \textunderscore o soldado Miguel passou á peluda\textunderscore .
\section{Peludo}
\begin{itemize}
\item {Grp. gram.:adj.}
\end{itemize}
\begin{itemize}
\item {Utilização:Fig.}
\end{itemize}
\begin{itemize}
\item {Grp. gram.:M.}
\end{itemize}
\begin{itemize}
\item {Proveniência:(De \textunderscore pêlo\textunderscore )}
\end{itemize}
Que tem muito pêlo.
Coberto de pêlo.
Bisonho; desconfiado; tímido.
Indivíduo que é peludo.
\section{Pelugem}
\begin{itemize}
\item {Grp. gram.:f.}
\end{itemize}
Conjunto de pêlos.
\section{Peluginoso}
\begin{itemize}
\item {Grp. gram.:adj.}
\end{itemize}
\begin{itemize}
\item {Proveniência:(De \textunderscore pelugem\textunderscore )}
\end{itemize}
Que tem pelugem ou pêlos.
\section{Pelváptero}
\begin{itemize}
\item {Grp. gram.:adj.}
\end{itemize}
\begin{itemize}
\item {Utilização:Zool.}
\end{itemize}
\begin{itemize}
\item {Proveniência:(De \textunderscore pelve\textunderscore  + \textunderscore áptero\textunderscore )}
\end{itemize}
Que não tem barbatanas ventraes.
\section{Pelve}
\begin{itemize}
\item {Grp. gram.:f.}
\end{itemize}
\begin{itemize}
\item {Utilização:Anat.}
\end{itemize}
\begin{itemize}
\item {Proveniência:(Lat. \textunderscore pelvis\textunderscore )}
\end{itemize}
Bacia, que termina inferiormente o tronco humano.
\section{Pélvico}
\begin{itemize}
\item {Grp. gram.:adj.}
\end{itemize}
Relativo á pelve.
\section{Pelvi-crural}
\begin{itemize}
\item {Grp. gram.:adj.}
\end{itemize}
\begin{itemize}
\item {Utilização:Anat.}
\end{itemize}
\begin{itemize}
\item {Proveniência:(Do lat. \textunderscore pelvis\textunderscore  + \textunderscore crus\textunderscore )}
\end{itemize}
Relativo á coxa e á bacia.
\section{Pelviforme}
\begin{itemize}
\item {Grp. gram.:adj.}
\end{itemize}
\begin{itemize}
\item {Proveniência:(Do lat. \textunderscore pelvis\textunderscore  + \textunderscore forma\textunderscore )}
\end{itemize}
Que tem fórma de bacia ou taça.
\section{Pelvimetria}
\begin{itemize}
\item {Grp. gram.:f.}
\end{itemize}
\begin{itemize}
\item {Proveniência:(De \textunderscore pelvímetro\textunderscore )}
\end{itemize}
Operação obstétrica, que tem por fim determinar a distensão das differentes partes da pelve.
\section{Pelvímetro}
\begin{itemize}
\item {Grp. gram.:m.}
\end{itemize}
\begin{itemize}
\item {Proveniência:(Do lat. \textunderscore pelvis\textunderscore  + gr. \textunderscore metron\textunderscore )}
\end{itemize}
Instrumento, com que se mede o diâmetro da pelve da mulher.
\section{Pélvis}
\begin{itemize}
\item {Grp. gram.:f.}
\end{itemize}
O mesmo que \textunderscore pelve\textunderscore .
\section{Pêmphigo}
\begin{itemize}
\item {Grp. gram.:m.}
\end{itemize}
\begin{itemize}
\item {Proveniência:(Do gr. \textunderscore pemphix\textunderscore )}
\end{itemize}
Bolhas na pelle dos animaes, contendo líquido seroso.
\section{Pemphigóide}
\begin{itemize}
\item {Grp. gram.:adj.}
\end{itemize}
\begin{itemize}
\item {Utilização:Med.}
\end{itemize}
\begin{itemize}
\item {Proveniência:(Do gr. \textunderscore pemphigodes\textunderscore )}
\end{itemize}
Dizia-se da febre, que acompanha o pêmphigo.
\section{Pena}
\begin{itemize}
\item {Grp. gram.:f.}
\end{itemize}
\begin{itemize}
\item {Proveniência:(Lat. \textunderscore poena\textunderscore )}
\end{itemize}
Aquillo que se faz soffrer a alguém por um delicto commettido; punição.
Soffrimento; desgôsto.
Desgraça.
Piedade, compaixão.
Cuidado.
\section{Pena}
\begin{itemize}
\item {Grp. gram.:f.}
\end{itemize}
\begin{itemize}
\item {Utilização:Ant.}
\end{itemize}
O mesmo que \textunderscore penha\textunderscore . (Ainda us. em Turquel)
\section{Penação}
\begin{itemize}
\item {Grp. gram.:f.}
\end{itemize}
Acto de penar. Cf. Garrett, \textunderscore Romanceiro\textunderscore , II, 166.
\section{Penadamente}
\begin{itemize}
\item {Grp. gram.:adv.}
\end{itemize}
\begin{itemize}
\item {Proveniência:(De \textunderscore penado\textunderscore )}
\end{itemize}
Com pena, com afflicção.
\section{Penadeira}
\begin{itemize}
\item {Grp. gram.:f.}
\end{itemize}
Espécie de peixe da Póvoa de Varzim, (\textunderscore lophius pescatorius\textunderscore , Lin.).
\section{Penado}
\begin{itemize}
\item {Grp. gram.:adj.}
\end{itemize}
\begin{itemize}
\item {Proveniência:(De \textunderscore penar\textunderscore )}
\end{itemize}
Que está penando: \textunderscore alma penada\textunderscore .
\section{Penafidelense}
\begin{itemize}
\item {Grp. gram.:adj.}
\end{itemize}
\begin{itemize}
\item {Grp. gram.:M.}
\end{itemize}
\begin{itemize}
\item {Proveniência:(Do lat. hyp. \textunderscore Penafidelis\textunderscore , n. p.)}
\end{itemize}
Relativo a Penafiel.
Habitante de Penafiel.
\section{Penafiel}
\begin{itemize}
\item {Grp. gram.:m.}
\end{itemize}
\begin{itemize}
\item {Utilização:Burl.}
\end{itemize}
Dança e ária popular do norte do país.
\textunderscore Casaca de Penafiel\textunderscore , o mesmo que \textunderscore albarda\textunderscore . Cf. Alb. Pimentel, \textunderscore As Alegres Canções do Norte\textunderscore , 73 e 95.
\section{Penaguiota}
\begin{itemize}
\item {Grp. gram.:adj.}
\end{itemize}
Relativo a Penaguião.
Cp. \textunderscore malvasia-penaguiota\textunderscore .
\section{Penal}
\begin{itemize}
\item {Grp. gram.:adj.}
\end{itemize}
\begin{itemize}
\item {Proveniência:(Lat. \textunderscore poenalis\textunderscore )}
\end{itemize}
Relativo a penas judiciaes: \textunderscore reforma penal\textunderscore .
Relativo ao código ou ás leis, que se occupam dos delictos e crimes, designando a pena que lhes corresponde.
Que inflige penas.
\section{Penalidade}
\begin{itemize}
\item {Grp. gram.:f.}
\end{itemize}
\begin{itemize}
\item {Proveniência:(De \textunderscore penal\textunderscore )}
\end{itemize}
Conjunto ou systema de penas, que a lei impõe: \textunderscore a penalidade na Índia\textunderscore .
Natureza da pena.
Pena, castigo.
\section{Penalista}
\begin{itemize}
\item {Grp. gram.:m.}
\end{itemize}
Aquelle que é versado em direito penal.
\section{Penalizar}
\begin{itemize}
\item {Grp. gram.:v. t.}
\end{itemize}
Causar pena ou dó a; pungir.
\section{Penalogia}
\begin{itemize}
\item {Grp. gram.:f.}
\end{itemize}
\begin{itemize}
\item {Utilização:Jur.}
\end{itemize}
\begin{itemize}
\item {Proveniência:(Do lat. \textunderscore poena\textunderscore  + gr. \textunderscore logos\textunderscore )}
\end{itemize}
Tratado das penas.
\section{Penalogista}
\begin{itemize}
\item {Grp. gram.:m.}
\end{itemize}
O mesmo que \textunderscore penalista\textunderscore .
\section{Penamacor}
\begin{itemize}
\item {Grp. gram.:m.}
\end{itemize}
\begin{itemize}
\item {Proveniência:(De \textunderscore Penamacôr\textunderscore , n. p.)}
\end{itemize}
Casta de uva beirôa.
\section{Penamar}
\begin{itemize}
\item {Grp. gram.:adj.}
\end{itemize}
Diz-se da pérola, que tem pouco lustre.
\section{Penamilha}
\begin{itemize}
\item {Grp. gram.:f.}
\end{itemize}
\begin{itemize}
\item {Utilização:Ant.}
\end{itemize}
Pesar, arrependimento?:«\textunderscore ...nam tem os pecadores, nem penamilha, nem penamilha-mar por hum correr...\textunderscore »\textunderscore Eufrosina\textunderscore , acto I, sc. II.
\section{Penante}
\begin{itemize}
\item {Grp. gram.:m.}
\end{itemize}
\begin{itemize}
\item {Utilização:Gír.}
\end{itemize}
Chapéu alto.
\section{Penão}
\begin{itemize}
\item {Grp. gram.:m.}
\end{itemize}
\begin{itemize}
\item {Utilização:Des.}
\end{itemize}
\begin{itemize}
\item {Proveniência:(Do fr. \textunderscore penon\textunderscore )}
\end{itemize}
Bandeira de navio.
Galhardete.
\section{Penar}
\begin{itemize}
\item {Grp. gram.:v. i.}
\end{itemize}
\begin{itemize}
\item {Grp. gram.:V. t.}
\end{itemize}
\begin{itemize}
\item {Utilização:Des.}
\end{itemize}
\begin{itemize}
\item {Utilização:Ant.}
\end{itemize}
Soffrer pena; padecer.
Têr pesares.
Causar pena a; desgostar.
Castigar.
\section{Penaroso}
\begin{itemize}
\item {Grp. gram.:adj.}
\end{itemize}
\begin{itemize}
\item {Utilização:T. de Villa Viçosa}
\end{itemize}
O mesmo que \textunderscore pesaroso\textunderscore .
\section{Penates}
\begin{itemize}
\item {Grp. gram.:m. pl.}
\end{itemize}
\begin{itemize}
\item {Utilização:Fig.}
\end{itemize}
\begin{itemize}
\item {Proveniência:(Lat. \textunderscore penates\textunderscore )}
\end{itemize}
Deuses domesticos, na antiga Roma.
Lares; família.
Casa paterna.
\section{Penável}
\begin{itemize}
\item {Grp. gram.:adj.}
\end{itemize}
\begin{itemize}
\item {Utilização:Ant.}
\end{itemize}
\begin{itemize}
\item {Proveniência:(De \textunderscore penar\textunderscore )}
\end{itemize}
Digno de castigo.
O mesmo que \textunderscore penal\textunderscore .
\section{Penca}
\begin{itemize}
\item {Grp. gram.:f.}
\end{itemize}
\begin{itemize}
\item {Utilização:Fig.}
\end{itemize}
\begin{itemize}
\item {Utilização:Bras}
\end{itemize}
\begin{itemize}
\item {Utilização:Prov.}
\end{itemize}
\begin{itemize}
\item {Utilização:dur.}
\end{itemize}
\begin{itemize}
\item {Utilização:beir.}
\end{itemize}
\begin{itemize}
\item {Utilização:Bras. da Baía}
\end{itemize}
\begin{itemize}
\item {Utilização:T. do Fundão}
\end{itemize}
Fôlha grossa e carnuda de alguns vegetaes.
Nariz grande.
Cada um dos grupos fructíferos dos cachos de bananas.
Variedade de couve, também conhecida por \textunderscore couve de cortar\textunderscore , porque se lhe corta o conjunto das fôlhas, deixando-se-lhe o tronco, para que deite espigos ou grelos.
Porção: \textunderscore penca de filhos\textunderscore , muitos filhos; \textunderscore penca de chaves\textunderscore , argola com muitas chaves; \textunderscore penca de laranjas\textunderscore , cacho de laranjas.
Manjar branco. Cf. G. Braga, \textunderscore Mal da Delf.\textunderscore , 186.
\section{Pencão}
\begin{itemize}
\item {Grp. gram.:m.}
\end{itemize}
\begin{itemize}
\item {Utilização:Prov.}
\end{itemize}
\begin{itemize}
\item {Utilização:trasm.}
\end{itemize}
\begin{itemize}
\item {Proveniência:(De \textunderscore penca\textunderscore )}
\end{itemize}
Pedúnculo dos frutos.
\section{Pencha}
\begin{itemize}
\item {Grp. gram.:f.}
\end{itemize}
\begin{itemize}
\item {Utilização:Gír.}
\end{itemize}
As partes pudendas da mulhér.
\section{Pencudo}
\begin{itemize}
\item {Grp. gram.:adj.}
\end{itemize}
\begin{itemize}
\item {Utilização:Fam.}
\end{itemize}
Que tem grande penca ou nariz; narigudo.
\section{Pendacosta}
\begin{itemize}
\item {Grp. gram.:m.}
\end{itemize}
\begin{itemize}
\item {Utilização:Bras}
\end{itemize}
\begin{itemize}
\item {Proveniência:(De \textunderscore pano\textunderscore  + \textunderscore da\textunderscore  + \textunderscore costa\textunderscore )}
\end{itemize}
Chale grosseiro, próprio das africanas.
\section{Pendal}
\begin{itemize}
\item {Grp. gram.:m.}
\end{itemize}
\begin{itemize}
\item {Utilização:T. da Índia port}
\end{itemize}
Espécie de barraca alpendrada.
\section{Pendanga}
\begin{itemize}
\item {Grp. gram.:f.}
\end{itemize}
\begin{itemize}
\item {Utilização:Ant.}
\end{itemize}
\begin{itemize}
\item {Proveniência:(De \textunderscore pender\textunderscore ?)}
\end{itemize}
Accessório? appenso? ou pendenga?:«\textunderscore ...no maior calor da escripta viérão pendangas mais urgentes.\textunderscore »Filinto XI, 146.
\section{Pendão}
\begin{itemize}
\item {Grp. gram.:m.}
\end{itemize}
\begin{itemize}
\item {Proveniência:(Do b. lat. \textunderscore pendo\textunderscore )}
\end{itemize}
Bandeira; balsão; estandarte.
Insígnia.
Bandeira do milho.
Espécie de grande bandeira, armada em vêrga e que é levada em algumas procissões.
\section{Pendedela}
\begin{itemize}
\item {Grp. gram.:f.}
\end{itemize}
\begin{itemize}
\item {Utilização:Prov.}
\end{itemize}
\begin{itemize}
\item {Utilização:alg.}
\end{itemize}
\begin{itemize}
\item {Proveniência:(De \textunderscore pender\textunderscore )}
\end{itemize}
Acto de cabecear com somno.
\section{Pendença}
\begin{itemize}
\item {Grp. gram.:f.}
\end{itemize}
\begin{itemize}
\item {Utilização:Ant.}
\end{itemize}
\begin{itemize}
\item {Utilização:Pop.}
\end{itemize}
O mesmo que \textunderscore penitência\textunderscore .
\section{Pendença}
\begin{itemize}
\item {Grp. gram.:f.}
\end{itemize}
\begin{itemize}
\item {Utilização:Ant.}
\end{itemize}
Pendência, conflicto. Cf. \textunderscore Eufrosina\textunderscore , 34.
\section{Pendençal}
\begin{itemize}
\item {Grp. gram.:adj.}
\end{itemize}
\begin{itemize}
\item {Utilização:Ant.}
\end{itemize}
O mesmo que \textunderscore pendenceiro\textunderscore .
\section{Pendenceiro}
\begin{itemize}
\item {Grp. gram.:adj.}
\end{itemize}
\begin{itemize}
\item {Utilização:Ant.}
\end{itemize}
\begin{itemize}
\item {Proveniência:(De \textunderscore pendença\textunderscore ^1)}
\end{itemize}
O mesmo que \textunderscore penitenciário\textunderscore . Cf. Arn. Gama, \textunderscore Bailio\textunderscore , 88.
\section{Pendência}
\begin{itemize}
\item {Grp. gram.:f.}
\end{itemize}
\begin{itemize}
\item {Proveniência:(De \textunderscore pender\textunderscore )}
\end{itemize}
Qualidade daquillo que pende.
Desavença, conflicto.
Tempo, em que uma questão judicial está correndo ou pendente de recurso ou sentença: \textunderscore na pendência da causa\textunderscore .
\section{Pendenciar}
\begin{itemize}
\item {Grp. gram.:v. i.}
\end{itemize}
\begin{itemize}
\item {Utilização:P. us.}
\end{itemize}
Têr pendência ou conflicto.
\section{Pendenga}
\begin{itemize}
\item {Grp. gram.:f.}
\end{itemize}
\begin{itemize}
\item {Utilização:Bras}
\end{itemize}
Pendência.
Briga; conflicto.
(Alter. de \textunderscore pendência\textunderscore )
\section{Pendengues}
\begin{itemize}
\item {Grp. gram.:m. pl.}
\end{itemize}
\begin{itemize}
\item {Utilização:Prov.}
\end{itemize}
\begin{itemize}
\item {Proveniência:(De \textunderscore pender\textunderscore . Cp. \textunderscore pendente\textunderscore )}
\end{itemize}
Espécie de arrecadas.
\section{Pendente}
\begin{itemize}
\item {Grp. gram.:adj.}
\end{itemize}
\begin{itemize}
\item {Utilização:Fig.}
\end{itemize}
\begin{itemize}
\item {Grp. gram.:M.}
\end{itemize}
\begin{itemize}
\item {Utilização:Constr.}
\end{itemize}
\begin{itemize}
\item {Proveniência:(Lat. \textunderscore pendens\textunderscore )}
\end{itemize}
Que pende; suspenso; pendurado.
Que ainda não foi colhido: \textunderscore frutos pendentes\textunderscore .
Imminente.
Inclinado.
Fixo, attento: \textunderscore ...todo o auditório estava pendente dos lábios do orador\textunderscore .
Pingente:«\textunderscore ...e poreis estes pendentes, em cada orelha seu.\textunderscore »G. Vicente, \textunderscore Auto da Alma\textunderscore .
Cada uma das superfícies contínuas e curvas, que estabelecem ligação entre a superfície de uma abóbada esphérica ou ellipsoidal e as paredes, columnas ou pilastras, que limitam um edifício de base quadrada, a que essa abóbada serve de cobertura.
Parte, que pende, da orla de um escudo ou bandeira.
\section{Pender}
\begin{itemize}
\item {Grp. gram.:v. i.}
\end{itemize}
\begin{itemize}
\item {Utilização:Fig.}
\end{itemize}
\begin{itemize}
\item {Grp. gram.:V. t.}
\end{itemize}
\begin{itemize}
\item {Utilização:Fig.}
\end{itemize}
\begin{itemize}
\item {Proveniência:(Lat. \textunderscore pendere\textunderscore )}
\end{itemize}
Estar suspenso ou pendurado.
Inclinar-se, descair: \textunderscore o Sol já vai pendendo\textunderscore .
Depender.
Estar imminente.
Estar para caír.
Dar preferência; têr propensão: \textunderscore o rapaz pende para a música\textunderscore .
Dependurar:«\textunderscore ...único enfeite que a religiosa pendia do collo.\textunderscore »Camillo, \textunderscore Caveira\textunderscore , 163.
Fazer caír.
Fazer murcho: \textunderscore o calor pendeu as flôres\textunderscore .
\section{Penderica}
\begin{itemize}
\item {Grp. gram.:f.}
\end{itemize}
\begin{itemize}
\item {Proveniência:(De \textunderscore pender\textunderscore )}
\end{itemize}
Qualquer pequena coisa pendente; berloque; penduricalho.
\section{Penderico}
\begin{itemize}
\item {Grp. gram.:m.}
\end{itemize}
\begin{itemize}
\item {Utilização:Pop.}
\end{itemize}
\begin{itemize}
\item {Proveniência:(De \textunderscore pender\textunderscore )}
\end{itemize}
Qualquer pequena coisa pendente; berloque; penduricalho.
\section{Penderucalho}
\begin{itemize}
\item {Grp. gram.:m.}
\end{itemize}
O mesmo que \textunderscore penduricalho\textunderscore .
\section{Pendoar}
\begin{itemize}
\item {Grp. gram.:v. i.}
\end{itemize}
\begin{itemize}
\item {Utilização:Bras}
\end{itemize}
O mesmo que \textunderscore apendoar\textunderscore .
\section{Pendoença}
\begin{itemize}
\item {Grp. gram.:f.}
\end{itemize}
\begin{itemize}
\item {Utilização:Ant.}
\end{itemize}
Sinal de verdadeiro arrependimento.
O mesmo que \textunderscore pendença\textunderscore ^1.
\section{Pena}
\begin{itemize}
\item {Grp. gram.:f.}
\end{itemize}
\begin{itemize}
\item {Utilização:Náut.}
\end{itemize}
\begin{itemize}
\item {Proveniência:(Lat. \textunderscore penna\textunderscore )}
\end{itemize}
Cada uma das peças, que revestem o corpo das aves.
Pluma.
Tubo de pena, preparado para com êle se escrever.
Cálamo.
Pequena peça de metal ou de chifre, com que se escreve.
Aparo.
O aparo, com a respectiva caneta.
Trabalho de escrita.
Classe dos escritores.
Maneira de escrever.
Escritor.
Parte da vela latina, que vai fixar-se no penol da carangueja.
Parte espalmada da bigorna.
Cada uma das asas do rodízio do moínho, nas quais bate a água que o move.
\textunderscore Pena de água\textunderscore , medida usada em partilhas de água, da grossura aproximada do uma pena de pato.
\section{Penáceo}
\begin{itemize}
\item {Grp. gram.:adj.}
\end{itemize}
\begin{itemize}
\item {Utilização:Zool.}
\end{itemize}
\begin{itemize}
\item {Utilização:Bot.}
\end{itemize}
Semelhante a uma pena.
\section{Penacheira}
\begin{itemize}
\item {Grp. gram.:f.}
\end{itemize}
\begin{itemize}
\item {Utilização:Bras}
\end{itemize}
O mesmo que \textunderscore penacheiro\textunderscore .
\section{Penacheiro}
\begin{itemize}
\item {Grp. gram.:m.}
\end{itemize}
\begin{itemize}
\item {Proveniência:(De \textunderscore penacho\textunderscore )}
\end{itemize}
Planta mirtácea, (\textunderscore callístemon lophantum\textunderscore ).
\section{Penacho}
\begin{itemize}
\item {Grp. gram.:m.}
\end{itemize}
\begin{itemize}
\item {Utilização:Ant.}
\end{itemize}
\begin{itemize}
\item {Utilização:Fig.}
\end{itemize}
\begin{itemize}
\item {Proveniência:(Do b. lat. \textunderscore pennasculum\textunderscore , de \textunderscore pena\textunderscore )}
\end{itemize}
Conjunto de penas, com que se adornam chapéus, capacetes, etc.
Crista.
Utensílio de lan, para limpar instrumentos de sôpro.
Parte triangular de uma abóbada, que sustenta a volta de uma cúpula.
Ostentação, gala.
Direcção, governo; comando.
\section{Penada}
\begin{itemize}
\item {Grp. gram.:f.}
\end{itemize}
Traço de pena.
Tinta, que a pena toma, de cada vez que se molha no tinteiro.
Palavras, escritas com uma penada.
\section{Penado}
\begin{itemize}
\item {Grp. gram.:adj.}
\end{itemize}
Que tem penas.
\section{Penagris}
\begin{itemize}
\item {Grp. gram.:m.}
\end{itemize}
\begin{itemize}
\item {Proveniência:(De \textunderscore pena\textunderscore  + \textunderscore gris\textunderscore )}
\end{itemize}
Penugem parda.
\section{Penão}
\begin{itemize}
\item {Grp. gram.:m.}
\end{itemize}
\begin{itemize}
\item {Utilização:Ant.}
\end{itemize}
\begin{itemize}
\item {Proveniência:(De \textunderscore pena\textunderscore )}
\end{itemize}
Pena grande; estilete, com que se escrevia.
\section{Penatífido}
\begin{itemize}
\item {Grp. gram.:adj.}
\end{itemize}
\begin{itemize}
\item {Utilização:Bot.}
\end{itemize}
\begin{itemize}
\item {Proveniência:(Do lat. \textunderscore pennatus\textunderscore  + \textunderscore findere\textunderscore )}
\end{itemize}
Diz-se das fôlhas, que têm recortes pouco fundos e dispostos á maneira dos folíolos das folhas pinuladas.
\section{Penatilobado}
\begin{itemize}
\item {Grp. gram.:adj.}
\end{itemize}
\begin{itemize}
\item {Utilização:Bot.}
\end{itemize}
Diz-se das fôlhas que, tendo nervuras pinuladas, são divididas em muitos lóbulos, cuja profundidade é variável.
\section{Pêndola}
\begin{itemize}
\item {Grp. gram.:f.}
\end{itemize}
\begin{itemize}
\item {Utilização:Ant.}
\end{itemize}
Penna de escrever.
(Corr. de \textunderscore pênnula\textunderscore , dem. de \textunderscore penna\textunderscore ?)
\section{Pendor}
\begin{itemize}
\item {Grp. gram.:m.}
\end{itemize}
\begin{itemize}
\item {Utilização:Ant.}
\end{itemize}
\begin{itemize}
\item {Proveniência:(De \textunderscore pender\textunderscore )}
\end{itemize}
Declive.
Obliquidade.
Índole, propensão.
O mesmo que \textunderscore pêso\textunderscore :«\textunderscore ...quebram as escadas com o pendor da gente...\textunderscore »Filinto, \textunderscore D. Man.\textunderscore , III, 7.
Acto de ancorar.
\section{Pendorada}
\begin{itemize}
\item {Grp. gram.:f.}
\end{itemize}
\begin{itemize}
\item {Utilização:Geogr.}
\end{itemize}
Série de pendores ou encostas.
\section{Pendorar}
\begin{itemize}
\item {Grp. gram.:v. i.}
\end{itemize}
O mesmo que \textunderscore pendoar\textunderscore .
\section{Pendorelhos}
\begin{itemize}
\item {fónica:dorê}
\end{itemize}
\begin{itemize}
\item {Grp. gram.:m. pl.}
\end{itemize}
\begin{itemize}
\item {Utilização:Prov.}
\end{itemize}
\begin{itemize}
\item {Utilização:trasm.}
\end{itemize}
\begin{itemize}
\item {Proveniência:(De \textunderscore pendor\textunderscore )}
\end{itemize}
Conjunto dos pelos, que forram a parte interna da orelha do boi.
\section{Pêndula}
\begin{itemize}
\item {Grp. gram.:f.}
\end{itemize}
Relógio de pêndulo; pêndulo.
(Fem. de \textunderscore pêndulo\textunderscore )
\section{Pendular}
\begin{itemize}
\item {Grp. gram.:adj.}
\end{itemize}
Relativo a pêndulo.
\section{Pendulifloro}
\begin{itemize}
\item {Grp. gram.:adj.}
\end{itemize}
\begin{itemize}
\item {Utilização:Bot.}
\end{itemize}
\begin{itemize}
\item {Proveniência:(Do lat. \textunderscore pendulus\textunderscore  + \textunderscore flos\textunderscore )}
\end{itemize}
Diz-se das plantas, que têm as flôres pendentes pela curvatura dos pedúnculos.
\section{Pendulifoliado}
\begin{itemize}
\item {Grp. gram.:adj.}
\end{itemize}
\begin{itemize}
\item {Utilização:Bot.}
\end{itemize}
\begin{itemize}
\item {Proveniência:(Do lat. \textunderscore pendulus\textunderscore  + \textunderscore folium\textunderscore )}
\end{itemize}
Diz-se das plantas, que têm as fôlhas pendentes.
\section{Pendulista}
\begin{itemize}
\item {Grp. gram.:m.}
\end{itemize}
\begin{itemize}
\item {Utilização:Des.}
\end{itemize}
\begin{itemize}
\item {Proveniência:(De \textunderscore pêndulo\textunderscore )}
\end{itemize}
O mesmo que \textunderscore relojoeiro\textunderscore ? Cf. Camillo, \textunderscore Noites de Insómn.\textunderscore , VII, 75.
\section{Pêndulo}
\begin{itemize}
\item {Grp. gram.:m.}
\end{itemize}
\begin{itemize}
\item {Utilização:Fig.}
\end{itemize}
\begin{itemize}
\item {Grp. gram.:Adj.}
\end{itemize}
\begin{itemize}
\item {Proveniência:(Lat. \textunderscore pendulus\textunderscore )}
\end{itemize}
Corpo pesado, suspenso na extremidade inferior de um fio ou de uma vara metállica, e que serve para aprumar ou realizar o movimento de vaivém.
Aquillo que se move ou trabalha com intervallos regulares.
O que se faz com intervallos regulares.
Que está pendente.
Que tem ramos pendentes ou inclinados para baixo:«\textunderscore ...a acácia pêndula...\textunderscore »Castilho, \textunderscore Geórgicas\textunderscore , 295.
\section{Pendura}
\begin{itemize}
\item {Grp. gram.:f.}
\end{itemize}
\begin{itemize}
\item {Utilização:Gír.}
\end{itemize}
Acto de pendurar.
Coisa pendurada.
Lâmpada.
\section{Pendura-amarela}
\begin{itemize}
\item {Grp. gram.:f.}
\end{itemize}
Casta de uva de Azeitão. Cf. \textunderscore Revista Agron.\textunderscore , I, 18.
\section{Pendural}
\begin{itemize}
\item {Grp. gram.:m.}
\end{itemize}
\begin{itemize}
\item {Utilização:Carp.}
\end{itemize}
\begin{itemize}
\item {Proveniência:(De \textunderscore pendura\textunderscore )}
\end{itemize}
Cada uma das peças verticaes, que ligam o travessanho ao frechal.
Pequena viga, que desce do vértice da asna.
Peça central de um tecto, donde pende gancho ou cordão, que sustenta uma lâmpada. Cf. Herculano, \textunderscore Cister\textunderscore , II, 10.
\section{Pendurar}
\begin{itemize}
\item {Grp. gram.:v. t.}
\end{itemize}
\begin{itemize}
\item {Utilização:Fig.}
\end{itemize}
\begin{itemize}
\item {Proveniência:(Do b. lat. \textunderscore pendorare\textunderscore ?)}
\end{itemize}
Suspender.
Collocar alto (um objecto), sem que êste toque no chão.
Fixar, prender:«\textunderscore ...pendurados os olhos e os ouvidos do que nos haveis de dizer.\textunderscore »Filinto, \textunderscore D. Man.\textunderscore , I, 82.
\section{Pendureza}
\begin{itemize}
\item {Grp. gram.:f.}
\end{itemize}
\begin{itemize}
\item {Utilização:Chul.}
\end{itemize}
\begin{itemize}
\item {Proveniência:(De \textunderscore pendurar\textunderscore )}
\end{itemize}
Coisa pendurada.
Peça de fita ou renda, que, pregada no fato, é mais comprida do que deve sêr.
\section{Penduricalho}
\begin{itemize}
\item {Grp. gram.:m.}
\end{itemize}
\begin{itemize}
\item {Utilização:Burl.}
\end{itemize}
\begin{itemize}
\item {Proveniência:(De \textunderscore pendura\textunderscore )}
\end{itemize}
Coisa pendente, para enfeite ou adôrno.
Pingente.
Berloque.
Condecoração.
\section{Peneáceas}
\begin{itemize}
\item {Grp. gram.:f. pl.}
\end{itemize}
Família de plantas, que tem por typo a \textunderscore peneia\textunderscore .
\section{Penedia}
\begin{itemize}
\item {Grp. gram.:f.}
\end{itemize}
Reunião de penedos; rocha.
\section{Penedio}
\begin{itemize}
\item {Grp. gram.:m.}
\end{itemize}
O mesmo que \textunderscore penedia\textunderscore . Cf. Camillo, \textunderscore M. de Pombal\textunderscore , 64.
\section{Penedo}
\begin{itemize}
\item {fónica:nê}
\end{itemize}
\begin{itemize}
\item {Grp. gram.:m.}
\end{itemize}
\begin{itemize}
\item {Proveniência:(Do b. lat. \textunderscore pennetum\textunderscore )}
\end{itemize}
Grande pedra; calhau; penha.
\section{Peneficar}
\textunderscore v. t.\textunderscore  (e der.)
O mesmo que \textunderscore penificar\textunderscore , etc.
\section{Peneia}
\begin{itemize}
\item {Grp. gram.:f.}
\end{itemize}
\begin{itemize}
\item {Proveniência:(De \textunderscore Peneia\textunderscore , n. p. myth.)}
\end{itemize}
Gênero de plantas dicotyledóneas do Cabo da Bôa-Esperança.
\section{Peneira}
\begin{itemize}
\item {Grp. gram.:f.}
\end{itemize}
\begin{itemize}
\item {Utilização:Prov.}
\end{itemize}
\begin{itemize}
\item {Utilização:Pesc.}
\end{itemize}
\begin{itemize}
\item {Utilização:Gír.}
\end{itemize}
\begin{itemize}
\item {Utilização:trasm}
\end{itemize}
\begin{itemize}
\item {Utilização:mil.}
\end{itemize}
\begin{itemize}
\item {Utilização:Gír.}
\end{itemize}
\begin{itemize}
\item {Proveniência:(Do lat. \textunderscore panaria\textunderscore )}
\end{itemize}
Utensílio circular de madeira, cujo fundo é formado de fios entrançados de seda ou crina e que serve para separar substâncias pulverizadas das partes mais grossas.
Crivo; joeira.
Chuva miúda.
Borboleta.
Apparelho para pescar camarão.
Fome ou sede.
Rancho aguado.
\section{Peneiração}
\begin{itemize}
\item {Grp. gram.:f.}
\end{itemize}
Acto ou trabalho de peneirar.
\section{Peneirada}
\begin{itemize}
\item {Grp. gram.:f.}
\end{itemize}
\begin{itemize}
\item {Proveniência:(De \textunderscore peneirar\textunderscore )}
\end{itemize}
O mesmo que \textunderscore peneiração\textunderscore .
Aquillo que se peneira de uma vez.
\section{Peneirado}
\begin{itemize}
\item {Grp. gram.:adj.}
\end{itemize}
Que se saracoteia:«\textunderscore ...e retirou-se, muito peneirada.\textunderscore »Camillo, \textunderscore Vulcões\textunderscore , 169.
\section{Peneirador}
\begin{itemize}
\item {Grp. gram.:adj.}
\end{itemize}
\begin{itemize}
\item {Grp. gram.:M.}
\end{itemize}
\begin{itemize}
\item {Utilização:Prov.}
\end{itemize}
\begin{itemize}
\item {Utilização:alg.}
\end{itemize}
\begin{itemize}
\item {Proveniência:(De \textunderscore peneirar\textunderscore )}
\end{itemize}
Que peneira.
Aquelle que peneira.
Espécie de alcofa, com que se peneira farinha; o mesmo que \textunderscore caparão\textunderscore .
\section{Peneirar}
\begin{itemize}
\item {Grp. gram.:v. t.}
\end{itemize}
\begin{itemize}
\item {Grp. gram.:V. p.}
\end{itemize}
\begin{itemize}
\item {Utilização:Fig.}
\end{itemize}
\begin{itemize}
\item {Grp. gram.:V. i.}
\end{itemize}
\begin{itemize}
\item {Utilização:Bras}
\end{itemize}
Fazer passar pela peneira.
Saracotear-se, andando.
Chuviscar.
\section{Peneireiro}
\begin{itemize}
\item {Grp. gram.:m.}
\end{itemize}
\begin{itemize}
\item {Utilização:Prov.}
\end{itemize}
\begin{itemize}
\item {Utilização:trasm}
\end{itemize}
\begin{itemize}
\item {Utilização:mil.}
\end{itemize}
\begin{itemize}
\item {Utilização:Gír.}
\end{itemize}
Fabricante ou vendedor de peneiras.
Aquelle que trabalha com peneira.
Milhafre.
Diabo.
O official, que fornece rancho aguado ás tropas.
\section{Peneiro}
\begin{itemize}
\item {Grp. gram.:m.}
\end{itemize}
\begin{itemize}
\item {Proveniência:(Lat. \textunderscore panarium\textunderscore )}
\end{itemize}
Peneira grande, empregada em algumas padarias, para separar da farinha o farelo.
\section{Peneiro}
\begin{itemize}
\item {Grp. gram.:m.}
\end{itemize}
\begin{itemize}
\item {Utilização:Ant.}
\end{itemize}
\begin{itemize}
\item {Grp. gram.:F.}
\end{itemize}
\begin{itemize}
\item {Utilização:Ext.}
\end{itemize}
Espécie de entretela nas abas da casaca.
Cada uma das duas abas da casaca. Cf. Filinto, V, 3.
(Por \textunderscore paneiro\textunderscore , de \textunderscore pano\textunderscore ? Ou por \textunderscore penneiro\textunderscore , de \textunderscore penna\textunderscore ?)
\section{Penela}
\begin{itemize}
\item {Grp. gram.:f.}
\end{itemize}
\begin{itemize}
\item {Proveniência:(De \textunderscore pena\textunderscore ^2)}
\end{itemize}
Oiteiro; pequena penha.
\section{Peneplano}
\begin{itemize}
\item {Grp. gram.:m.}
\end{itemize}
\begin{itemize}
\item {Proveniência:(Do lat. \textunderscore pene\textunderscore  + \textunderscore planus\textunderscore )}
\end{itemize}
Região quási plana.
\section{Peneplano}
\begin{itemize}
\item {Grp. gram.:m.}
\end{itemize}
\begin{itemize}
\item {Utilização:Geol.}
\end{itemize}
\begin{itemize}
\item {Proveniência:(Fr. \textunderscore peneplain\textunderscore )}
\end{itemize}
Solo quási plano.
\section{Penetra}
\begin{itemize}
\item {Grp. gram.:m. ,  f.  e  adj.}
\end{itemize}
\begin{itemize}
\item {Utilização:Pop.}
\end{itemize}
\begin{itemize}
\item {Proveniência:(De \textunderscore penetrar\textunderscore )}
\end{itemize}
Pessôa petulante, perliquitete.
Peralta, catita.
\section{Penetrabilidade}
\begin{itemize}
\item {Grp. gram.:f.}
\end{itemize}
\begin{itemize}
\item {Proveniência:(Do lat. \textunderscore penetrabilis\textunderscore )}
\end{itemize}
Qualidade do que é penetrável.
\section{Penetração}
\begin{itemize}
\item {Grp. gram.:f.}
\end{itemize}
\begin{itemize}
\item {Utilização:Ext.}
\end{itemize}
\begin{itemize}
\item {Proveniência:(Lat. \textunderscore penetratio\textunderscore )}
\end{itemize}
Acto ou effeito de penetrar.
Facilidade de comprehensão; perspicácia.
\section{Penetrador}
\begin{itemize}
\item {Grp. gram.:adj.}
\end{itemize}
\begin{itemize}
\item {Proveniência:(Lat. \textunderscore penetrator\textunderscore )}
\end{itemize}
O mesmo que \textunderscore penetrante\textunderscore .
\section{Penetraes}
\begin{itemize}
\item {Grp. gram.:m. pl.}
\end{itemize}
\begin{itemize}
\item {Proveniência:(Lat. \textunderscore penetralia\textunderscore )}
\end{itemize}
O interior; a parte mais íntima: \textunderscore os penetraes da consciência\textunderscore .
\section{Penetrais}
\begin{itemize}
\item {Grp. gram.:m. pl.}
\end{itemize}
\begin{itemize}
\item {Proveniência:(Lat. \textunderscore penetralia\textunderscore )}
\end{itemize}
O interior; a parte mais íntima: \textunderscore os penetrais da consciência\textunderscore .
\section{Penetrante}
\begin{itemize}
\item {Grp. gram.:adj.}
\end{itemize}
\begin{itemize}
\item {Utilização:Fig.}
\end{itemize}
\begin{itemize}
\item {Proveniência:(Lat. \textunderscore penetrans\textunderscore )}
\end{itemize}
Que penetra.
Que punge; intenso: \textunderscore dôres penetrantes\textunderscore .
Agudo.
Talentoso.
Perspicaz; sagaz: \textunderscore espírito penetrante\textunderscore .
\section{Penetrar}
\begin{itemize}
\item {Grp. gram.:v. t.}
\end{itemize}
\begin{itemize}
\item {Utilização:Ext.}
\end{itemize}
\begin{itemize}
\item {Grp. gram.:V. i.}
\end{itemize}
\begin{itemize}
\item {Utilização:Ext.}
\end{itemize}
\begin{itemize}
\item {Grp. gram.:V. p.}
\end{itemize}
\begin{itemize}
\item {Proveniência:(Lat. \textunderscore penetrare\textunderscore )}
\end{itemize}
Invadir; entrar dentro de.
Atravessar: \textunderscore a faca penetrou-lhe o coração\textunderscore .
Repassar: \textunderscore êste frio penetra os ossos\textunderscore .
Transpor.
Comprehender.
Descortinar: \textunderscore penetrar um segrêdo\textunderscore .
Introduzir-se: \textunderscore o bandido penetrou na loja\textunderscore .
Entrar no ânimo de alguém, insinuar-se.
Tomar conhecimento.
Convencer-se profundamente: \textunderscore penetrou-se dos seus deveres\textunderscore .
\section{Penetrativo}
\begin{itemize}
\item {Grp. gram.:adj.}
\end{itemize}
O mesmo que \textunderscore penetrante\textunderscore .
\section{Penetrável}
\begin{itemize}
\item {Grp. gram.:adj.}
\end{itemize}
\begin{itemize}
\item {Proveniência:(Lat. \textunderscore penetrabilis\textunderscore )}
\end{itemize}
Que póde sêr penetrado.
\section{Pênfigo}
\begin{itemize}
\item {Grp. gram.:m.}
\end{itemize}
\begin{itemize}
\item {Proveniência:(Do gr. \textunderscore pemphix\textunderscore )}
\end{itemize}
Bolhas na pele dos animaes, contendo líquido seroso.
\section{Penfigóide}
\begin{itemize}
\item {Grp. gram.:adj.}
\end{itemize}
\begin{itemize}
\item {Utilização:Med.}
\end{itemize}
\begin{itemize}
\item {Proveniência:(Do gr. \textunderscore pemphigodes\textunderscore )}
\end{itemize}
Dizia-se da febre, que acompanha o pênfigo.
\section{Pengó}
\begin{itemize}
\item {Grp. gram.:m.}
\end{itemize}
\begin{itemize}
\item {Utilização:Bras}
\end{itemize}
Capenga apalermado.
\section{Penguim}
\begin{itemize}
\item {Grp. gram.:m.}
\end{itemize}
Gênero de aves palmípedes da Europa boreal.
\section{Penha}
\begin{itemize}
\item {Grp. gram.:f.}
\end{itemize}
Rocha, penhasco.
(Cast. \textunderscore peña\textunderscore )
\section{Penhascal}
\begin{itemize}
\item {Grp. gram.:m.}
\end{itemize}
\begin{itemize}
\item {Proveniência:(De \textunderscore penhasco\textunderscore )}
\end{itemize}
O mesmo que \textunderscore penhasqueira\textunderscore . Cf. Camillo, \textunderscore Brasileira\textunderscore , 242.
\section{Penhasco}
\begin{itemize}
\item {Grp. gram.:m.}
\end{itemize}
\begin{itemize}
\item {Proveniência:(De \textunderscore penha\textunderscore )}
\end{itemize}
Penha elevada; rocha extensa.
\section{Penhascoso}
\begin{itemize}
\item {Grp. gram.:adj.}
\end{itemize}
Em que há penhascos.
\section{Penhasqueira}
\begin{itemize}
\item {Grp. gram.:f.}
\end{itemize}
Série de penhascos.
\section{Penhor}
\begin{itemize}
\item {Grp. gram.:m.}
\end{itemize}
\begin{itemize}
\item {Utilização:Fig.}
\end{itemize}
\begin{itemize}
\item {Proveniência:(Do lat. \textunderscore pignus\textunderscore )}
\end{itemize}
Objecto, que alguém recebe, para segurança de uma dívida ou empréstimo.
Garantia do pagamento de uma dívida.
Prova, garantia, sinal.
Espécie de jôgo popular.
\section{Penhóra}
\begin{itemize}
\item {Grp. gram.:f.}
\end{itemize}
\begin{itemize}
\item {Proveniência:(Do b. lat. \textunderscore pignora\textunderscore )}
\end{itemize}
Apprehensão de bens de um devedor, para pagamento judicial e respectivas custas.
Execução judicial, para o pagamento de quantia determinada.
\section{Penhorado}
\begin{itemize}
\item {Grp. gram.:adj.}
\end{itemize}
\begin{itemize}
\item {Utilização:Fig.}
\end{itemize}
Em que recaiu penhóra: \textunderscore bens penhorados\textunderscore .
Muito agradecido, muito grato.
\section{Penhorante}
\begin{itemize}
\item {Grp. gram.:adj.}
\end{itemize}
\begin{itemize}
\item {Utilização:Fig.}
\end{itemize}
Que penhóra, que torna grato.
Que obriga a reconhecimento: \textunderscore favores penhorantes\textunderscore .
\section{Penhorar}
\begin{itemize}
\item {Grp. gram.:v. t.}
\end{itemize}
\begin{itemize}
\item {Utilização:Fig.}
\end{itemize}
\begin{itemize}
\item {Proveniência:(De \textunderscore penhor\textunderscore )}
\end{itemize}
Fazer penhora em.
Apprehender em virtude de processo executivo.
Dar em garantia; afiançar.
Tornar agradecido, dar causa á gratidão de.
Impor gratidão, cativar.
\section{Penhorável}
\begin{itemize}
\item {Grp. gram.:adj.}
\end{itemize}
Que se póde penhorar.
\section{Penhorista}
\begin{itemize}
\item {Grp. gram.:adj.}
\end{itemize}
\begin{itemize}
\item {Grp. gram.:M.}
\end{itemize}
\begin{itemize}
\item {Proveniência:(De \textunderscore penhor\textunderscore )}
\end{itemize}
Relativo a penhores ou aos agiotas.
Aquelle que tem casa de penhores; agiota.
\section{Pênia}
\begin{itemize}
\item {Grp. gram.:f.}
\end{itemize}
Gênero de insectos coleópteros pentâmeros.
\section{Peniano}
\begin{itemize}
\item {Grp. gram.:adj.}
\end{itemize}
\begin{itemize}
\item {Utilização:Anat.}
\end{itemize}
Relativo ao pênis.
\section{Penicada}
\begin{itemize}
\item {Grp. gram.:f.}
\end{itemize}
Porção de urina ou de excrementos, contida num penico.
\section{Penicar}
\begin{itemize}
\item {Grp. gram.:v. t.}
\end{itemize}
\begin{itemize}
\item {Utilização:Prov.}
\end{itemize}
\begin{itemize}
\item {Utilização:minh.}
\end{itemize}
O mesmo que \textunderscore depenicar\textunderscore .
\section{Penicilária}
\begin{itemize}
\item {Grp. gram.:f.}
\end{itemize}
\begin{itemize}
\item {Proveniência:(De \textunderscore penicilo\textunderscore )}
\end{itemize}
Gênero de plantas gramíneas.
\section{Penicillária}
\begin{itemize}
\item {Grp. gram.:f.}
\end{itemize}
\begin{itemize}
\item {Proveniência:(De \textunderscore penicillo\textunderscore )}
\end{itemize}
Gênero de plantas gramíneas.
\section{Penicillo}
\begin{itemize}
\item {Grp. gram.:m.}
\end{itemize}
\begin{itemize}
\item {Proveniência:(Lat. \textunderscore penicillus\textunderscore )}
\end{itemize}
Concha univalve.
\section{Penicilo}
\begin{itemize}
\item {Grp. gram.:m.}
\end{itemize}
\begin{itemize}
\item {Proveniência:(Lat. \textunderscore penicillus\textunderscore )}
\end{itemize}
Concha univalve.
\section{Penico}
\begin{itemize}
\item {Grp. gram.:m.}
\end{itemize}
\begin{itemize}
\item {Utilização:Pleb.}
\end{itemize}
\begin{itemize}
\item {Proveniência:(Do cast. \textunderscore perico\textunderscore , dem. de \textunderscore Pero\textunderscore , n. p.)}
\end{itemize}
Vaso de loiça ou de metal, para urinas e ainda para dejecções.
Bacio; bispote.
\section{Penicreiro}
\begin{itemize}
\item {Grp. gram.:m.}
\end{itemize}
Arbusto, o mesmo que \textunderscore catapereiro\textunderscore .
\section{Penificar}
\begin{itemize}
\item {Grp. gram.:v. t.}
\end{itemize}
\begin{itemize}
\item {Utilização:Ant.}
\end{itemize}
\begin{itemize}
\item {Proveniência:(Do lat. \textunderscore poena\textunderscore  + \textunderscore facere\textunderscore )}
\end{itemize}
Impor pena a; castigar.
\section{Penilho}
\begin{itemize}
\item {Grp. gram.:m.}
\end{itemize}
Espécie de planta, mencionada por Brotero.
\section{Península}
\begin{itemize}
\item {Grp. gram.:f.}
\end{itemize}
\begin{itemize}
\item {Proveniência:(Lat. \textunderscore peninsula\textunderscore )}
\end{itemize}
Região, cercada de água, por todos os lados, excepto por um, que se liga a outra região, geralmente mais vasta.
\section{Peninsular}
\begin{itemize}
\item {Grp. gram.:adj.}
\end{itemize}
\begin{itemize}
\item {Grp. gram.:M.  e  f.}
\end{itemize}
\begin{itemize}
\item {Utilização:Restrict.}
\end{itemize}
Relativo a península.
Pessôa, que habita numa península ou della é natural.
Pessôa, que é da Península Hispânica.
\section{Peniposte}
\begin{itemize}
\item {Grp. gram.:m.}
\end{itemize}
\begin{itemize}
\item {Utilização:Prov.}
\end{itemize}
\begin{itemize}
\item {Utilização:trasm.}
\end{itemize}
Homem metediço e parasito.
\section{Peniqueira}
\begin{itemize}
\item {Grp. gram.:f.}
\end{itemize}
\begin{itemize}
\item {Utilização:Pleb.}
\end{itemize}
\begin{itemize}
\item {Utilização:Chul.}
\end{itemize}
Móvel de quarto de cama, em que se guarda o penico.
Mesa de cabeceira.
Criada de quarto.
\section{Pênis}
\begin{itemize}
\item {Grp. gram.:m.}
\end{itemize}
\begin{itemize}
\item {Utilização:Anat.}
\end{itemize}
\begin{itemize}
\item {Proveniência:(Lat. \textunderscore penis\textunderscore )}
\end{itemize}
Órgão viril da geração.
\section{Peniscar}
\begin{itemize}
\item {Grp. gram.:v. i.}
\end{itemize}
\begin{itemize}
\item {Utilização:Prov.}
\end{itemize}
Comer pouco, sem appetite; debicar.
\section{Penisco}
\begin{itemize}
\item {Grp. gram.:m.}
\end{itemize}
\begin{itemize}
\item {Proveniência:(Do lat. \textunderscore pinus\textunderscore )}
\end{itemize}
Porção de pinhão miúdo.
Semente de pinheiro bravo.
\section{Penisqueiro}
\begin{itemize}
\item {Grp. gram.:adj.}
\end{itemize}
\begin{itemize}
\item {Utilização:Prov.}
\end{itemize}
\begin{itemize}
\item {Proveniência:(De \textunderscore peniscar\textunderscore )}
\end{itemize}
Que penisca; que come pouco.
\section{Penite}
\begin{itemize}
\item {Grp. gram.:f.}
\end{itemize}
\begin{itemize}
\item {Utilização:Med.}
\end{itemize}
Inflammação do pênis.
\section{Penitela}
\begin{itemize}
\item {Grp. gram.:f.}
\end{itemize}
\begin{itemize}
\item {Utilização:Ant.}
\end{itemize}
Espécie de iguaria.
(Por \textunderscore panitela\textunderscore , do lat. \textunderscore panis\textunderscore ?)
\section{Penitência}
\begin{itemize}
\item {Grp. gram.:f.}
\end{itemize}
\begin{itemize}
\item {Utilização:Irón.}
\end{itemize}
\begin{itemize}
\item {Proveniência:(Lat. \textunderscore poenitentia\textunderscore )}
\end{itemize}
Arrependimento de uma culpa ou peccado.
Pena, que o confessor impõe, para remissão de peccados.
Sacrifícios, que se fazem para expiação de peccados.
Incômmodo, tormento.
Arrependimento.
Comezana, gáudio, pândega. Cf. Castilho, \textunderscore D. Quixote\textunderscore , II, 30.
\section{Penintencial}
\begin{itemize}
\item {Grp. gram.:adj.}
\end{itemize}
\begin{itemize}
\item {Grp. gram.:M.}
\end{itemize}
\begin{itemize}
\item {Proveniência:(Lat. \textunderscore poenitentialis\textunderscore )}
\end{itemize}
Relativo a penitência.
Ritual das penitências.
\section{Penitenciar}
\begin{itemize}
\item {Grp. gram.:v. t.}
\end{itemize}
\begin{itemize}
\item {Grp. gram.:V. p.}
\end{itemize}
Impor como penitência.
Arrepender-se.
Fazer sacrifícios para expiação e peccados.
\section{Penitenciaría}
\begin{itemize}
\item {Grp. gram.:f.}
\end{itemize}
\begin{itemize}
\item {Proveniência:(De \textunderscore penitência\textunderscore )}
\end{itemize}
Tribunal pontifício, em que se resolvem os negócios da privada competência do Papa.
\section{Penitenciária}
\begin{itemize}
\item {Grp. gram.:f.}
\end{itemize}
Edifício público, em que, por sentença judicial, se prendem certos criminosos, separados uns dos outros em céllulas.
(Fem. de \textunderscore penitenciário\textunderscore )
\section{Penitenciário}
\begin{itemize}
\item {Grp. gram.:adj.}
\end{itemize}
\begin{itemize}
\item {Grp. gram.:M.}
\end{itemize}
\begin{itemize}
\item {Proveniência:(De \textunderscore penitência\textunderscore )}
\end{itemize}
Penitencial.
Relativo ao systema de prisões, em que os criminosos vivem insulados ou separados em céllulas.
Presidente da penitenciaría pontifícia.
Indivíduo, preso em penitenciária.
Aquelle que impõe penitência ou castiga o penitente.
\section{Penitencieiro}
\begin{itemize}
\item {Grp. gram.:m.}
\end{itemize}
\begin{itemize}
\item {Proveniência:(De \textunderscore penitenciaría\textunderscore )}
\end{itemize}
Cardeal, membro da Penitenciaría.
Padre ou frade, que era confessor em certas igrejas ou capellas. Cf. J. B. Castro, \textunderscore Mappa de Porto\textunderscore , III.
\section{Penitente}
\begin{itemize}
\item {Grp. gram.:m. ,  f.  e  adj.}
\end{itemize}
\begin{itemize}
\item {Grp. gram.:M. pl.}
\end{itemize}
\begin{itemize}
\item {Proveniência:(Lat. \textunderscore poenitens\textunderscore )}
\end{itemize}
Pessôa, que se arrepende.
Pessôa, que faz penitência.
Pessôa, que faz confissão de seus peccados ao sacerdote.
Frades franciscanos.
\section{Penível}
\begin{itemize}
\item {Grp. gram.:adj.}
\end{itemize}
\begin{itemize}
\item {Utilização:inútil}
\end{itemize}
\begin{itemize}
\item {Utilização:Gal}
\end{itemize}
\begin{itemize}
\item {Proveniência:(Fr. \textunderscore penible\textunderscore )}
\end{itemize}
Que causa pena ou mágoa.
Fatigante; penoso. Cf. Th. Braga, \textunderscore Mod. Ideias\textunderscore , I, 414; T. de Queirós, \textunderscore Com. do Campo\textunderscore , II, 159.
\section{Penna}
\begin{itemize}
\item {Grp. gram.:f.}
\end{itemize}
\begin{itemize}
\item {Utilização:Náut.}
\end{itemize}
\begin{itemize}
\item {Proveniência:(Lat. \textunderscore penna\textunderscore )}
\end{itemize}
Cada uma das peças, que revestem o corpo das aves.
Pluma.
Tubo de penna, preparado para com êlle se escrever.
Cálamo.
Pequena peça de metal ou de chifre, com que se escreve.
Aparo.
O aparo, com a respectiva caneta.
Trabalho de escrita.
Classe dos escritores.
Maneira de escrever.
Escritor.
Parte da vela latina, que vai fixar-se no penol da carangueja.
Parte espalmada da bigorna.
Cada uma das asas do rodízio do moínho, nas quaes bate a água que o move.
\textunderscore Penna de água\textunderscore , medida usada em partilhas de água, da grossura aproximada do uma penna de pato.
\section{Pennáceo}
\begin{itemize}
\item {Grp. gram.:adj.}
\end{itemize}
\begin{itemize}
\item {Utilização:Zool.}
\end{itemize}
\begin{itemize}
\item {Utilização:Bot.}
\end{itemize}
Semelhante a uma penna.
\section{Pennacheira}
\begin{itemize}
\item {Grp. gram.:f.}
\end{itemize}
\begin{itemize}
\item {Utilização:Bras}
\end{itemize}
O mesmo que \textunderscore pennacheiro\textunderscore .
\section{Pennacheiro}
\begin{itemize}
\item {Grp. gram.:m.}
\end{itemize}
\begin{itemize}
\item {Proveniência:(De \textunderscore pennacho\textunderscore )}
\end{itemize}
Planta myrtácea, (\textunderscore callístemon lophantum\textunderscore ).
\section{Pennacho}
\begin{itemize}
\item {Grp. gram.:m.}
\end{itemize}
\begin{itemize}
\item {Utilização:Ant.}
\end{itemize}
\begin{itemize}
\item {Utilização:Fig.}
\end{itemize}
\begin{itemize}
\item {Proveniência:(Do b. lat. \textunderscore pennasculum\textunderscore , de \textunderscore penna\textunderscore )}
\end{itemize}
Conjunto de pennas, com que se adornam chapéus, capacetes, etc.
Crista.
Utensílio de lan, para limpar instrumentos de sôpro.
Parte triangular de uma abóbada, que sustenta a volta de uma cúpula.
Ostentação, gala.
Direcção, governo; commando.
\section{Pennada}
\begin{itemize}
\item {Grp. gram.:f.}
\end{itemize}
Traço de penna.
Tinta, que a penna toma, de cada vez que se molha no tinteiro.
Palavras, escritas com uma pennada.
\section{Pennado}
\begin{itemize}
\item {Grp. gram.:adj.}
\end{itemize}
Que tem pennas.
\section{Pennagris}
\begin{itemize}
\item {Grp. gram.:m.}
\end{itemize}
\begin{itemize}
\item {Proveniência:(De \textunderscore penna\textunderscore  + \textunderscore gris\textunderscore )}
\end{itemize}
Pennugem parda.
\section{Pennântia}
\begin{itemize}
\item {Grp. gram.:f.}
\end{itemize}
\begin{itemize}
\item {Proveniência:(De \textunderscore Pennant\textunderscore , n. p.)}
\end{itemize}
Gênero de plantas dicotyledóneas.
\section{Pennão}
\begin{itemize}
\item {Grp. gram.:m.}
\end{itemize}
\begin{itemize}
\item {Utilização:Ant.}
\end{itemize}
\begin{itemize}
\item {Proveniência:(De \textunderscore penna\textunderscore )}
\end{itemize}
Penna grande; estilete, com que se escrevia.
\section{Pennatífido}
\begin{itemize}
\item {Grp. gram.:adj.}
\end{itemize}
\begin{itemize}
\item {Utilização:Bot.}
\end{itemize}
\begin{itemize}
\item {Proveniência:(Do lat. \textunderscore pennatus\textunderscore  + \textunderscore findere\textunderscore )}
\end{itemize}
Diz-se das fôlhas, que têm recortes pouco fundos e dispostos á maneira dos folíolos das folhas pinnuladas.
\section{Pennatilobado}
\begin{itemize}
\item {Grp. gram.:adj.}
\end{itemize}
\begin{itemize}
\item {Utilização:Bot.}
\end{itemize}
Diz-se das fôlhas que, tendo nervuras pinnuladas, são divididas em muitos lóbulos, cuja profundidade é variável.
\section{Penátula}
\begin{itemize}
\item {Grp. gram.:f.}
\end{itemize}
Gênero de pólipos nadadores.
\section{Penatulários}
\begin{itemize}
\item {Grp. gram.:m. pl.}
\end{itemize}
\begin{itemize}
\item {Proveniência:(De \textunderscore penátula\textunderscore )}
\end{itemize}
Família de polipeiros, da classe dos zoófitos.
\section{Penego}
\begin{itemize}
\item {fónica:nê}
\end{itemize}
\begin{itemize}
\item {Grp. gram.:m.}
\end{itemize}
\begin{itemize}
\item {Utilização:Ant.}
\end{itemize}
Travesseiro ou almofada cheia de penas.
\section{Penejar}
\begin{itemize}
\item {Grp. gram.:v. t.}
\end{itemize}
Escrever; desenhar á pena.
\section{Penífero}
\begin{itemize}
\item {Grp. gram.:adj.}
\end{itemize}
\begin{itemize}
\item {Proveniência:(Do lat. \textunderscore pennifer\textunderscore )}
\end{itemize}
O mesmo que \textunderscore penígero\textunderscore . Cf. F. Barreto, \textunderscore Eneida\textunderscore , I, 120.
\section{Peniforme}
\begin{itemize}
\item {Grp. gram.:adj.}
\end{itemize}
\begin{itemize}
\item {Proveniência:(Do lat. \textunderscore penna\textunderscore  + \textunderscore forma\textunderscore )}
\end{itemize}
Que tem fórma de pena.
\section{Penígero}
\begin{itemize}
\item {Grp. gram.:adj.}
\end{itemize}
\begin{itemize}
\item {Proveniência:(Do lat. \textunderscore penna\textunderscore  + \textunderscore gerere\textunderscore )}
\end{itemize}
Que tem penas.
\section{Penina}
\begin{itemize}
\item {Grp. gram.:f.}
\end{itemize}
\begin{itemize}
\item {Utilização:Geol.}
\end{itemize}
\begin{itemize}
\item {Proveniência:(De \textunderscore pena\textunderscore )}
\end{itemize}
Uma das três espécies de clorites.
\section{Peninervado}
\begin{itemize}
\item {Grp. gram.:adj.}
\end{itemize}
\begin{itemize}
\item {Utilização:Bot.}
\end{itemize}
\begin{itemize}
\item {Proveniência:(De \textunderscore penna\textunderscore  + \textunderscore nervo\textunderscore )}
\end{itemize}
Cuja nervura principal se ramifica em nervuras secundárias, dispostas como as barbas de uma pena.
\section{Peninérveo}
\begin{itemize}
\item {Grp. gram.:adj.}
\end{itemize}
O mesmo que \textunderscore peninervado\textunderscore .
\section{Penipotente}
\begin{itemize}
\item {Grp. gram.:adj.}
\end{itemize}
\begin{itemize}
\item {Utilização:Poét.}
\end{itemize}
\begin{itemize}
\item {Proveniência:(Lat. \textunderscore pennipotens\textunderscore )}
\end{itemize}
Que vôa muito; que tem grande vigor nas asas.
\section{Pennatulários}
\begin{itemize}
\item {Grp. gram.:m. pl.}
\end{itemize}
\begin{itemize}
\item {Proveniência:(De \textunderscore pennátula\textunderscore )}
\end{itemize}
Família de polypeiros, da classe dos zoóphytos.
\section{Pennátula}
\begin{itemize}
\item {Grp. gram.:f.}
\end{itemize}
Gênero de pólypos nadadores.
\section{Pennego}
\begin{itemize}
\item {fónica:nê}
\end{itemize}
\begin{itemize}
\item {Grp. gram.:m.}
\end{itemize}
\begin{itemize}
\item {Utilização:Ant.}
\end{itemize}
Travesseiro ou almofada cheia de pennas.
\section{Pennejar}
\begin{itemize}
\item {Grp. gram.:v. t.}
\end{itemize}
Escrever; desenhar á penna.
\section{Pennífero}
\begin{itemize}
\item {Grp. gram.:adj.}
\end{itemize}
\begin{itemize}
\item {Proveniência:(Do lat. \textunderscore pennifer\textunderscore )}
\end{itemize}
O mesmo que \textunderscore penígero\textunderscore . Cf. F. Barreto, \textunderscore Eneida\textunderscore , I, 120.
\section{Penniforme}
\begin{itemize}
\item {Grp. gram.:adj.}
\end{itemize}
\begin{itemize}
\item {Proveniência:(Do lat. \textunderscore penna\textunderscore  + \textunderscore forma\textunderscore )}
\end{itemize}
Que tem fórma de penna.
\section{Pennígero}
\begin{itemize}
\item {Grp. gram.:adj.}
\end{itemize}
\begin{itemize}
\item {Proveniência:(Do lat. \textunderscore penna\textunderscore  + \textunderscore gerere\textunderscore )}
\end{itemize}
Que tem pennas.
\section{Pennina}
\begin{itemize}
\item {Grp. gram.:f.}
\end{itemize}
\begin{itemize}
\item {Utilização:Geol.}
\end{itemize}
\begin{itemize}
\item {Proveniência:(De \textunderscore penna\textunderscore )}
\end{itemize}
Uma das três espécies de chlorites.
\section{Penninervado}
\begin{itemize}
\item {Grp. gram.:adj.}
\end{itemize}
\begin{itemize}
\item {Utilização:Bot.}
\end{itemize}
\begin{itemize}
\item {Proveniência:(De \textunderscore penna\textunderscore  + \textunderscore nervo\textunderscore )}
\end{itemize}
Cuja nervura principal se ramifica em nervuras secundárias, dispostas como as barbas de uma penna.
\section{Penninérveo}
\begin{itemize}
\item {Grp. gram.:adj.}
\end{itemize}
O mesmo que \textunderscore penninervado\textunderscore .
\section{Pennipotente}
\begin{itemize}
\item {Grp. gram.:adj.}
\end{itemize}
\begin{itemize}
\item {Utilização:Poét.}
\end{itemize}
\begin{itemize}
\item {Proveniência:(Lat. \textunderscore pennipotens\textunderscore )}
\end{itemize}
Que vôa muito; que tem grande vigor nas asas.
\section{Pennosa}
\begin{itemize}
\item {Grp. gram.:f.}
\end{itemize}
\begin{itemize}
\item {Utilização:Gír.}
\end{itemize}
\begin{itemize}
\item {Proveniência:(De \textunderscore penna\textunderscore )}
\end{itemize}
O mesmo que \textunderscore gallinha\textunderscore , especialmente gallínha magra. Cf. Camillo, \textunderscore Mar. da Font.\textunderscore , 252.
\section{Pennudo}
\begin{itemize}
\item {Grp. gram.:adj.}
\end{itemize}
O mesmo que \textunderscore pennígero\textunderscore .
\section{Pennugem}
\begin{itemize}
\item {Grp. gram.:f.}
\end{itemize}
\begin{itemize}
\item {Proveniência:(De \textunderscore penna\textunderscore )}
\end{itemize}
As pennas, que primeiro nascem nas aves.
Os pelos e cabellos, que primeiro nascem.
Pêlo macio e curto.
Buço.
Froixel.
Espécie de pelos, nas cascas de frutos ou plantas.
\section{Pennujar}
\begin{itemize}
\item {Grp. gram.:v. i.}
\end{itemize}
Mostrar-se coberto de pennugem.
\section{Pennujento}
\begin{itemize}
\item {Grp. gram.:adj.}
\end{itemize}
Cheio ou coberto de pennugem.
\section{Pennujoso}
\begin{itemize}
\item {Grp. gram.:adj.}
\end{itemize}
O mesmo que \textunderscore pennujento\textunderscore .
\section{Pênnula}
\begin{itemize}
\item {Grp. gram.:f.}
\end{itemize}
\begin{itemize}
\item {Proveniência:(Lat. \textunderscore pennula\textunderscore )}
\end{itemize}
Penna, com que os jograes do século XIII tocavam a cíthara, e com que, na antiguidade, se faziam vibrar outros instrumentos. Cf. \textunderscore Cancion. da Vaticana\textunderscore .
\section{Penoca}
\begin{itemize}
\item {fónica:nô}
\end{itemize}
\begin{itemize}
\item {Grp. gram.:f.}
\end{itemize}
\begin{itemize}
\item {Proveniência:(De \textunderscore pena\textunderscore ^2)}
\end{itemize}
O mesmo ou talvez melhor que \textunderscore pinoco\textunderscore ; penhasco, rochedo elevado.
\section{Penoco}
\begin{itemize}
\item {fónica:nô}
\end{itemize}
\begin{itemize}
\item {Grp. gram.:m.}
\end{itemize}
\begin{itemize}
\item {Proveniência:(De \textunderscore pena\textunderscore ^2)}
\end{itemize}
O mesmo ou talvez melhor que \textunderscore pinoco\textunderscore ; penhasco, rochedo elevado.
\section{Penol}
\begin{itemize}
\item {Grp. gram.:m.}
\end{itemize}
\begin{itemize}
\item {Utilização:Náut.}
\end{itemize}
Ponta da vêrga, nos navios.
Lais superior da vêrga.
\section{Penos}
\begin{itemize}
\item {Grp. gram.:m. pl.}
\end{itemize}
\begin{itemize}
\item {Proveniência:(Lat. \textunderscore Poeni\textunderscore )}
\end{itemize}
O mesmo que [[carthagineses|carthaginês]].
\section{Penosa}
\begin{itemize}
\item {Grp. gram.:f.}
\end{itemize}
\begin{itemize}
\item {Utilização:Gír.}
\end{itemize}
\begin{itemize}
\item {Proveniência:(De \textunderscore pena\textunderscore )}
\end{itemize}
O mesmo que \textunderscore galinha\textunderscore , especialmente galínha magra. Cf. Camillo, \textunderscore Mar. da Font.\textunderscore , 252.
\section{Penosamente}
\begin{itemize}
\item {Grp. gram.:adv.}
\end{itemize}
De modo penoso; com sacrifício; com sofrimento.
\section{Penoso}
\begin{itemize}
\item {Grp. gram.:adj.}
\end{itemize}
Que causa pena; que incommoda; doloroso.
\section{Penque}
\begin{itemize}
\item {Grp. gram.:m.}
\end{itemize}
O mesmo que \textunderscore pinque\textunderscore .
\section{Pensador}
\begin{itemize}
\item {Grp. gram.:m.  e  adj.}
\end{itemize}
\begin{itemize}
\item {Proveniência:(De \textunderscore pensar\textunderscore )}
\end{itemize}
Aquelle que pensa.
O que faz profundas observações; philósopho.
\section{Pensamentear}
\begin{itemize}
\item {Grp. gram.:v. t.}
\end{itemize}
\begin{itemize}
\item {Utilização:Des.}
\end{itemize}
\begin{itemize}
\item {Proveniência:(De \textunderscore pensamento\textunderscore )}
\end{itemize}
O mesmo que \textunderscore pensar\textunderscore . Cf. Filinto, XVIII, 208; \textunderscore D. Man.\textunderscore , I, 312, e II, 98.
\section{Pensamento}
\begin{itemize}
\item {Grp. gram.:m.}
\end{itemize}
Acto ou effeito de pensar.
Qualquer acto de intelligência.
Fantasia.
Ideia; Espírito.
\section{Pensamentos}
\begin{itemize}
\item {Grp. gram.:m. pl.}
\end{itemize}
\begin{itemize}
\item {Utilização:Ant.}
\end{itemize}
Arrecadas, com filigrana de oiro.
\section{Pensante}
\begin{itemize}
\item {Grp. gram.:adj.}
\end{itemize}
\begin{itemize}
\item {Proveniência:(Lat. \textunderscore pensans\textunderscore )}
\end{itemize}
Que pensa; que faz uso da razão.
\section{Pensão}
\begin{itemize}
\item {Grp. gram.:f.}
\end{itemize}
\begin{itemize}
\item {Utilização:Gír.}
\end{itemize}
\begin{itemize}
\item {Proveniência:(Lat. \textunderscore pensio\textunderscore )}
\end{itemize}
Renda annual ou mensal, que se paga vitaliciamente ou por determinado tempo.
Foro.
Retribuição da educação e sustento de um alumno de collégio.
Encargo, ónus.
Trabalho.
\section{Pensar}
\begin{itemize}
\item {Grp. gram.:v. i.}
\end{itemize}
\begin{itemize}
\item {Grp. gram.:Loc.}
\end{itemize}
\begin{itemize}
\item {Utilização:pop.}
\end{itemize}
\begin{itemize}
\item {Grp. gram.:V. t.}
\end{itemize}
\begin{itemize}
\item {Grp. gram.:M.}
\end{itemize}
\begin{itemize}
\item {Proveniência:(Lat. \textunderscore pensare\textunderscore )}
\end{itemize}
Formar ou combinar ideias.
Fazer reflexões.
Sêr de tal ou tal parecer ou opinião.
Raciocinar.
Tencionar.
Têr cuidado.
\textunderscore Pensar na morte da bezerra\textunderscore , meditar tristemente.
Imaginar.
Combinar, reflectindo.
Têr no espírito: \textunderscore penso coisas tristes\textunderscore .
Fazer ideia de.
Acreditar, julgar: \textunderscore penso que andas mal\textunderscore .
Meditar.
Prever.
Têr cuidado de.
Dar ração a (gado).
Tratar convenientemente.
Fazer curativo ao ferimento de.
Pensamento.
Opinião.
Tino, prudência: \textunderscore a pequena já tem muito pensar\textunderscore .
\section{Pensativo}
\begin{itemize}
\item {Grp. gram.:adj.}
\end{itemize}
Que pensa.
Meditativo; absorto num pensamento.
\section{Penseroso}
\begin{itemize}
\item {Grp. gram.:adj.}
\end{itemize}
\begin{itemize}
\item {Utilização:inútil}
\end{itemize}
\begin{itemize}
\item {Utilização:Neol.}
\end{itemize}
\begin{itemize}
\item {Proveniência:(It. \textunderscore pensieroso\textunderscore )}
\end{itemize}
O mesmo que \textunderscore meditativo\textunderscore .
\section{Pênsil}
\begin{itemize}
\item {Grp. gram.:adj.}
\end{itemize}
\begin{itemize}
\item {Proveniência:(Lat. \textunderscore pensilis\textunderscore )}
\end{itemize}
Suspenso; construído sôbre columnas ou abóbadas.
\section{Pênsile}
\begin{itemize}
\item {Grp. gram.:adj.}
\end{itemize}
\begin{itemize}
\item {Grp. gram.:Pl.}
\end{itemize}
O mesmo que \textunderscore pênsil\textunderscore : \textunderscore ponte pênsile\textunderscore .
\textunderscore pênseis\textunderscore .--Filinto, XIV, 90, talvez por exigência do verso, ou porque attribuísse a pênsil o plural \textunderscore pênsiles\textunderscore , escreveu:«\textunderscore ...alli pensiles... as galerias...\textunderscore »
\section{Pensionar}
\begin{itemize}
\item {Grp. gram.:v. t.}
\end{itemize}
\begin{itemize}
\item {Proveniência:(Do lat. \textunderscore pensio\textunderscore )}
\end{itemize}
Impor pensão a.
\section{Pensionário}
\begin{itemize}
\item {Grp. gram.:m.  e  adj.}
\end{itemize}
\begin{itemize}
\item {Grp. gram.:Adj.}
\end{itemize}
\begin{itemize}
\item {Proveniência:(Do lat. \textunderscore pensio\textunderscore )}
\end{itemize}
Pensionista.
Relativo a pensão.
\section{Pensioneiro}
\begin{itemize}
\item {Grp. gram.:adj.}
\end{itemize}
\begin{itemize}
\item {Proveniência:(Do lat. \textunderscore pensio\textunderscore )}
\end{itemize}
Que paga pensão.
\section{Pensionista}
\begin{itemize}
\item {Grp. gram.:m. ,  f.  e  adj.}
\end{itemize}
\begin{itemize}
\item {Grp. gram.:F.}
\end{itemize}
\begin{itemize}
\item {Proveniência:(Do lat. \textunderscore pensio\textunderscore )}
\end{itemize}
Pessôa, que recebe uma pensão, especialmente do Estado.
Alumno, a quem o Estado paga pensão: \textunderscore pensionista do Collégio Militar\textunderscore .
Recolhida ou noviça, que paga pensão no convento.
\section{Penso}
\begin{itemize}
\item {Grp. gram.:m.}
\end{itemize}
\begin{itemize}
\item {Utilização:Pesc.}
\end{itemize}
\begin{itemize}
\item {Utilização:ant.}
\end{itemize}
\begin{itemize}
\item {Utilização:Pop.}
\end{itemize}
\begin{itemize}
\item {Proveniência:(De \textunderscore pensar\textunderscore )}
\end{itemize}
Tratamento de crianças ou animaes, relativo a sustento, limpeza, curativo, etc.
Curativo.
Ração para o gado.
Linha, que se prende á rêde e que o pescador sustenta na mão, para dar fé da entrada do peixe, pelo estremecimento que êste transmitte á linha.
O mesmo que \textunderscore pensamento\textunderscore  ou \textunderscore ideia\textunderscore . Cf. \textunderscore Eufrosina\textunderscore , 153.
\textunderscore Sêr de bom penso\textunderscore , sêr de bôa mantença, engordar facilmente.
\section{Penso}
\begin{itemize}
\item {Grp. gram.:adj.}
\end{itemize}
\begin{itemize}
\item {Utilização:Bras. do N}
\end{itemize}
\begin{itemize}
\item {Proveniência:(Lat. \textunderscore pensus\textunderscore , de \textunderscore pendo\textunderscore ?)}
\end{itemize}
Pendido, inclinado: \textunderscore ramagem pensa\textunderscore ; \textunderscore ombros pensos\textunderscore .
\section{Pensoso}
\begin{itemize}
\item {Grp. gram.:adj.}
\end{itemize}
\begin{itemize}
\item {Utilização:Ant.}
\end{itemize}
\begin{itemize}
\item {Proveniência:(De \textunderscore penso\textunderscore ^1)}
\end{itemize}
O mesmo que \textunderscore pensativo\textunderscore . Cf. Garrett, \textunderscore D. Branca\textunderscore , 76.
\section{Penta...}
\begin{itemize}
\item {Grp. gram.:pref.}
\end{itemize}
\begin{itemize}
\item {Proveniência:(Do gr. \textunderscore pente\textunderscore )}
\end{itemize}
(designativo de \textunderscore cinco\textunderscore )
\section{Pentacâmara}
\begin{itemize}
\item {Grp. gram.:adj.}
\end{itemize}
\begin{itemize}
\item {Utilização:Bot.}
\end{itemize}
\begin{itemize}
\item {Proveniência:(De \textunderscore penta...\textunderscore  + \textunderscore câmara\textunderscore )}
\end{itemize}
Fruto de cinco câmaras, como na \textunderscore clematis erecta\textunderscore .
\section{Pentacarpo}
\begin{itemize}
\item {Grp. gram.:adj.}
\end{itemize}
\begin{itemize}
\item {Utilização:Bot.}
\end{itemize}
\begin{itemize}
\item {Proveniência:(Do gr. \textunderscore pente\textunderscore  + \textunderscore karpos\textunderscore )}
\end{itemize}
Diz-se do fruto, que tem cinco carpellas.
\section{Pentacontaédro}
\begin{itemize}
\item {Grp. gram.:adj.}
\end{itemize}
\begin{itemize}
\item {Utilização:Miner.}
\end{itemize}
Diz-se de um crystal, cuja superfície apresenta cincoenta faces.
\section{Pentacontarco}
\begin{itemize}
\item {Grp. gram.:m.}
\end{itemize}
\begin{itemize}
\item {Proveniência:(Gr. \textunderscore pentakontarkhos\textunderscore )}
\end{itemize}
Commandante de quinhentos homens, na antiga Crécia.
\section{Pentacórdio}
\begin{itemize}
\item {Grp. gram.:m.}
\end{itemize}
\begin{itemize}
\item {Proveniência:(Lat. \textunderscore pentachordus\textunderscore )}
\end{itemize}
Instrumento de cinco cordas. Cf. Castilho, \textunderscore Fastos\textunderscore , III, 205.
\section{Pentacordo}
\begin{itemize}
\item {Grp. gram.:m.}
\end{itemize}
\begin{itemize}
\item {Proveniência:(Lat. \textunderscore pentachordus\textunderscore )}
\end{itemize}
Instrumento de cinco cordas. Cf. Castilho, \textunderscore Fastos\textunderscore , III, 205.
\section{Pentacosiarchia}
\begin{itemize}
\item {fónica:qui}
\end{itemize}
\begin{itemize}
\item {Grp. gram.:f.}
\end{itemize}
Formatura de 512 homens ou duas syntagmas, na phalange macedónica.
\section{Pentacosiarquia}
\begin{itemize}
\item {Grp. gram.:f.}
\end{itemize}
Formatura de 512 homens ou duas sintagmas, na falange macedónica.
\section{Pentacósmia}
\begin{itemize}
\item {Grp. gram.:f.}
\end{itemize}
Gênero de insectos coleópteros longicórneos.
\section{Pentacótomo}
\begin{itemize}
\item {Grp. gram.:adj.}
\end{itemize}
\begin{itemize}
\item {Utilização:Bot.}
\end{itemize}
Que se divide em cinco partes.
\section{Pentacripto}
\begin{itemize}
\item {Grp. gram.:m.}
\end{itemize}
\begin{itemize}
\item {Proveniência:(Do gr. \textunderscore pente\textunderscore  + \textunderscore kruptos\textunderscore )}
\end{itemize}
Gênero de plantas umbelíferas.
\section{Pentacróstico}
\begin{itemize}
\item {Grp. gram.:m.}
\end{itemize}
\begin{itemize}
\item {Proveniência:(De \textunderscore penta...\textunderscore  + \textunderscore acróstico\textunderscore )}
\end{itemize}
Grupo de versos, composto de maneira que se lê cinco vezes o nome que é objecto de um acróstico, dividindo-se toda a composição em cinco partes, de alto a baixo.
\section{Pentacrypto}
\begin{itemize}
\item {Grp. gram.:m.}
\end{itemize}
\begin{itemize}
\item {Proveniência:(Do gr. \textunderscore pente\textunderscore  + \textunderscore kruptos\textunderscore )}
\end{itemize}
Gênero de plantas umbellíferas.
\section{Pentadáctilo}
\begin{itemize}
\item {Grp. gram.:adj.}
\end{itemize}
\begin{itemize}
\item {Utilização:Bot.}
\end{itemize}
\begin{itemize}
\item {Proveniência:(Do gr. \textunderscore pente\textunderscore  + \textunderscore daktulos\textunderscore )}
\end{itemize}
Que tem cinco dedos.
Que tem cinco divisões, (falando-se das fôlhas).
\section{Pentadáctylo}
\begin{itemize}
\item {Grp. gram.:adj.}
\end{itemize}
\begin{itemize}
\item {Utilização:Bot.}
\end{itemize}
\begin{itemize}
\item {Proveniência:(Do gr. \textunderscore pente\textunderscore  + \textunderscore daktulos\textunderscore )}
\end{itemize}
Que tem cinco dedos.
Que tem cinco divisões, (falando-se das fôlhas).
\section{Pentadecágono}
\begin{itemize}
\item {Grp. gram.:adj.}
\end{itemize}
\begin{itemize}
\item {Utilização:Mathem.}
\end{itemize}
Que tem quinze lados e quinze ângulos.
\section{Pentadecylparatolylcetona}
\begin{itemize}
\item {Grp. gram.:f.}
\end{itemize}
Substância chímica, que se torna luminescente sob a acção dos raios da luz ordinária ou das radiações Roëntgen.
\section{Pentadelfo}
\begin{itemize}
\item {Grp. gram.:adj.}
\end{itemize}
\begin{itemize}
\item {Utilização:Bot.}
\end{itemize}
\begin{itemize}
\item {Proveniência:(Do gr. \textunderscore pente\textunderscore  + \textunderscore adelphos\textunderscore )}
\end{itemize}
Diz-se dos estames, reunidos em cinco fascículos e com muitas anteras.
\section{Pentadelpho}
\begin{itemize}
\item {Grp. gram.:adj.}
\end{itemize}
\begin{itemize}
\item {Utilização:Bot.}
\end{itemize}
\begin{itemize}
\item {Proveniência:(Do gr. \textunderscore pente\textunderscore  + \textunderscore adelphos\textunderscore )}
\end{itemize}
Diz-se dos estames, reunidos em cinco fascículos e com muitas antheras.
\section{Pentadesma}
\begin{itemize}
\item {Grp. gram.:f.}
\end{itemize}
\begin{itemize}
\item {Proveniência:(Do gr. \textunderscore pente\textunderscore  + \textunderscore desmos\textunderscore )}
\end{itemize}
Gênero de plantas elusiáceas.
\section{Pentaédro}
\begin{itemize}
\item {Grp. gram.:m.}
\end{itemize}
\begin{itemize}
\item {Utilização:Mathem.}
\end{itemize}
\begin{itemize}
\item {Proveniência:(Do gr. \textunderscore pente\textunderscore  + \textunderscore edra\textunderscore )}
\end{itemize}
Figura sólida, terminada por cinco faces.
\section{Pentafilo}
\begin{itemize}
\item {Grp. gram.:adj.}
\end{itemize}
\begin{itemize}
\item {Utilização:Bot.}
\end{itemize}
\begin{itemize}
\item {Proveniência:(Do gr. \textunderscore pente\textunderscore  + \textunderscore phullon\textunderscore )}
\end{itemize}
O mesmo que \textunderscore pentasépalo\textunderscore .
Que tem cinco fôlhas ou cinco folíolos.
\section{Pentagínia}
\begin{itemize}
\item {Grp. gram.:f.}
\end{itemize}
\begin{itemize}
\item {Proveniência:(Do gr. \textunderscore pente\textunderscore  + \textunderscore gune\textunderscore )}
\end{itemize}
Ordem de vegetaes que, no sistema de Linneu, compreende as plantas pentáginas.
\section{Pentaginía}
\begin{itemize}
\item {Grp. gram.:f.}
\end{itemize}
Qualidade de pentágino.
(Cp. \textunderscore pentagínia\textunderscore )
\section{Pentagínico}
\begin{itemize}
\item {Grp. gram.:adj.}
\end{itemize}
O mesmo que \textunderscore pentágino\textunderscore .
\section{Pentágino}
\begin{itemize}
\item {Grp. gram.:adj.}
\end{itemize}
\begin{itemize}
\item {Utilização:Bot.}
\end{itemize}
\begin{itemize}
\item {Proveniência:(Do gr. \textunderscore pente\textunderscore  + \textunderscore gune\textunderscore )}
\end{itemize}
Diz-se das flôres que têm cinco pistilos, ou das plantas, cujas flôres têm cinco pistilos.
\section{Pentagonal}
\begin{itemize}
\item {Grp. gram.:adj.}
\end{itemize}
Relativo a pentágono.
Que tem cinco lados.
\section{Pentágono}
\begin{itemize}
\item {Grp. gram.:m.}
\end{itemize}
\begin{itemize}
\item {Proveniência:(Gr. \textunderscore pentagonos\textunderscore )}
\end{itemize}
Polýgono de cinco lados.
\section{Pentagrafia}
\begin{itemize}
\item {Grp. gram.:f.}
\end{itemize}
Arte de aplicar o pentágrafo.
\section{Pentágrafo}
\begin{itemize}
\item {Grp. gram.:m.}
\end{itemize}
\begin{itemize}
\item {Proveniência:(Do gr. \textunderscore pente\textunderscore  + \textunderscore graphein\textunderscore )}
\end{itemize}
Instrumento, com que póde copiar desenhos quem não sabe desenhar.
\section{Pentagrama}
\begin{itemize}
\item {Grp. gram.:f.}
\end{itemize}
\begin{itemize}
\item {Proveniência:(De \textunderscore penta...\textunderscore  + \textunderscore gramma\textunderscore )}
\end{itemize}
Pauta da música.
Figura simbólica ou mágica de cinco letras.
\section{Pentagramma}
\begin{itemize}
\item {Grp. gram.:f.}
\end{itemize}
\begin{itemize}
\item {Proveniência:(De \textunderscore penta...\textunderscore  + \textunderscore gramma\textunderscore )}
\end{itemize}
Pauta da música.
Figura symbólica ou mágica de cinco letras.
\section{Pentagraphia}
\begin{itemize}
\item {Grp. gram.:f.}
\end{itemize}
Arte de applicar o pentágrapho.
\section{Pentágrapho}
\begin{itemize}
\item {Grp. gram.:m.}
\end{itemize}
\begin{itemize}
\item {Proveniência:(Do gr. \textunderscore pente\textunderscore  + \textunderscore graphein\textunderscore )}
\end{itemize}
Instrumento, com que póde copiar desenhos quem não sabe desenhar.
\section{Pentagýnia}
\begin{itemize}
\item {Grp. gram.:f.}
\end{itemize}
\begin{itemize}
\item {Proveniência:(Do gr. \textunderscore pente\textunderscore  + \textunderscore gune\textunderscore )}
\end{itemize}
Ordem de vegetaes que, no systema de Linneu, comprehende as plantas pentágynas.
\section{Pentagynía}
\begin{itemize}
\item {Grp. gram.:f.}
\end{itemize}
Qualidade de pentágyno.
(Cp. \textunderscore pentagýnia\textunderscore )
\section{Pentagýnico}
\begin{itemize}
\item {Grp. gram.:adj.}
\end{itemize}
O mesmo que \textunderscore pentágyno\textunderscore .
\section{Pentágyno}
\begin{itemize}
\item {Grp. gram.:adj.}
\end{itemize}
\begin{itemize}
\item {Utilização:Bot.}
\end{itemize}
\begin{itemize}
\item {Proveniência:(Do gr. \textunderscore pente\textunderscore  + \textunderscore gune\textunderscore )}
\end{itemize}
Diz-se das flôres que têm cinco pistillos, ou das plantas, cujas flôres têm cinco pistillos.
\section{Pentahídrico}
\begin{itemize}
\item {Grp. gram.:adj.}
\end{itemize}
\begin{itemize}
\item {Utilização:Chím.}
\end{itemize}
\begin{itemize}
\item {Proveniência:(Do gr. \textunderscore pente\textunderscore  + \textunderscore hudor\textunderscore )}
\end{itemize}
Que contém cinco vezes tanto hidrogênio como outro corpo.
\section{Pentahýdrico}
\begin{itemize}
\item {Grp. gram.:adj.}
\end{itemize}
\begin{itemize}
\item {Utilização:Chím.}
\end{itemize}
\begin{itemize}
\item {Proveniência:(Do gr. \textunderscore pente\textunderscore  + \textunderscore hudor\textunderscore )}
\end{itemize}
Que contém cinco vezes tanto hydrogênio como outro corpo.
\section{Pental}
\begin{itemize}
\item {Grp. gram.:m.}
\end{itemize}
\begin{itemize}
\item {Utilização:Chím.}
\end{itemize}
O mesmo que \textunderscore amylênio\textunderscore .
\section{Pentalépido}
\begin{itemize}
\item {Grp. gram.:adj.}
\end{itemize}
\begin{itemize}
\item {Utilização:Bot.}
\end{itemize}
\begin{itemize}
\item {Proveniência:(Do gr. \textunderscore pente\textunderscore  + \textunderscore lepis\textunderscore )}
\end{itemize}
Diz-se das partes vegetaes, que têm cinco estames.
\section{Pentáloba}
\begin{itemize}
\item {Grp. gram.:f.}
\end{itemize}
\begin{itemize}
\item {Proveniência:(Do gr. \textunderscore pente\textunderscore  + \textunderscore lobos\textunderscore )}
\end{itemize}
Gênero de plantas violaríneas.
\section{Pentâmero}
\begin{itemize}
\item {Grp. gram.:adj.}
\end{itemize}
\begin{itemize}
\item {Utilização:Anat.}
\end{itemize}
\begin{itemize}
\item {Grp. gram.:M. pl.}
\end{itemize}
\begin{itemize}
\item {Proveniência:(Do gr. \textunderscore pente\textunderscore  + \textunderscore meros\textunderscore )}
\end{itemize}
Que tem cinco divisões ou artículos.
Insectos pentâmeros.
\section{Pentâmetro}
\begin{itemize}
\item {Grp. gram.:m.  e  adj.}
\end{itemize}
\begin{itemize}
\item {Proveniência:(Gr. \textunderscore pentametros\textunderscore )}
\end{itemize}
Verso de cinco pés, grego ou latino.
\section{Pentaminas}
\begin{itemize}
\item {Grp. gram.:f. pl.}
\end{itemize}
\begin{itemize}
\item {Utilização:Chím.}
\end{itemize}
Aminas, formadas por cinco moléculas de ammoníaco.
\section{Pentândría}
\begin{itemize}
\item {Grp. gram.:f.}
\end{itemize}
\begin{itemize}
\item {Proveniência:(Do gr. \textunderscore pente\textunderscore  + \textunderscore aner\textunderscore , \textunderscore andros\textunderscore )}
\end{itemize}
Quinta classe dos vegetaes, no systema de Linneu.
\section{Pentandro}
\begin{itemize}
\item {Grp. gram.:adj.}
\end{itemize}
\begin{itemize}
\item {Utilização:Bot.}
\end{itemize}
\begin{itemize}
\item {Proveniência:(Do gr. \textunderscore pente\textunderscore  + \textunderscore aner\textunderscore , \textunderscore andros\textunderscore )}
\end{itemize}
Que tem cinco estames, livres entre si.
\section{Pentangular}
\begin{itemize}
\item {Grp. gram.:adj.}
\end{itemize}
\begin{itemize}
\item {Proveniência:(T. hybr., de \textunderscore penta...\textunderscore  + \textunderscore ângulo\textunderscore )}
\end{itemize}
Que tem cinco ângulos.
\section{Pentano}
\begin{itemize}
\item {Grp. gram.:m.}
\end{itemize}
\begin{itemize}
\item {Utilização:Chím.}
\end{itemize}
Um dos carbonetos do grupo formênico.
\section{Pentantéreo}
\begin{itemize}
\item {Grp. gram.:adj.}
\end{itemize}
\begin{itemize}
\item {Proveniência:(De \textunderscore penta...\textunderscore  + \textunderscore anthera\textunderscore )}
\end{itemize}
Que tem cinco anteras.
\section{Pentanthéreo}
\begin{itemize}
\item {Grp. gram.:adj.}
\end{itemize}
\begin{itemize}
\item {Proveniência:(De \textunderscore penta...\textunderscore  + \textunderscore anthera\textunderscore )}
\end{itemize}
Que tem cinco antheras.
\section{Pentantho}
\begin{itemize}
\item {Grp. gram.:m.}
\end{itemize}
\begin{itemize}
\item {Proveniência:(Do gr. \textunderscore pente\textunderscore  + \textunderscore anthos\textunderscore )}
\end{itemize}
Gênero de plantas, da fam. das compostas.
\section{Pentanto}
\begin{itemize}
\item {Grp. gram.:m.}
\end{itemize}
\begin{itemize}
\item {Proveniência:(Do gr. \textunderscore pente\textunderscore  + \textunderscore anthos\textunderscore )}
\end{itemize}
Gênero de plantas, da fam. das compostas.
\section{Pentapétalo}
\begin{itemize}
\item {Grp. gram.:adj.}
\end{itemize}
\begin{itemize}
\item {Utilização:Bot.}
\end{itemize}
\begin{itemize}
\item {Proveniência:(Do gr. \textunderscore pente\textunderscore  + \textunderscore petalon\textunderscore )}
\end{itemize}
Diz-se da corolla, quando formada de cinco pétalas distintas.
\section{Pentaphyllo}
\begin{itemize}
\item {Grp. gram.:adj.}
\end{itemize}
\begin{itemize}
\item {Utilização:Bot.}
\end{itemize}
\begin{itemize}
\item {Proveniência:(Do gr. \textunderscore pente\textunderscore  + \textunderscore phullon\textunderscore )}
\end{itemize}
O mesmo que \textunderscore pentasépalo\textunderscore .
Que tem cinco fôlhas ou cinco folíolos.
\section{Pentaplostêmone}
\begin{itemize}
\item {Grp. gram.:adj.}
\end{itemize}
\begin{itemize}
\item {Utilização:Bot.}
\end{itemize}
\begin{itemize}
\item {Proveniência:(Do gr. \textunderscore pentaploos\textunderscore  + \textunderscore stemon\textunderscore )}
\end{itemize}
Diz-se da flôr, em que o número dos estames é cinco vezes maior que o das divisões da corolla.
\section{Pentápode}
\begin{itemize}
\item {Grp. gram.:m.}
\end{itemize}
\begin{itemize}
\item {Proveniência:(Do gr. \textunderscore pente\textunderscore  + \textunderscore pous\textunderscore , \textunderscore podos\textunderscore )}
\end{itemize}
Gênero de peixes do mar das Índias.
\section{Pentápole}
\begin{itemize}
\item {Grp. gram.:f.}
\end{itemize}
\begin{itemize}
\item {Utilização:Geogr.}
\end{itemize}
\begin{itemize}
\item {Proveniência:(Do gr. \textunderscore pente\textunderscore  + \textunderscore polis\textunderscore )}
\end{itemize}
Território, que abrangia cinco cidades.
\section{Pentáptera}
\begin{itemize}
\item {Grp. gram.:f.}
\end{itemize}
Gênero de plantas, da fam. das combretáceas.
(Cp. \textunderscore pentáptero\textunderscore )
\section{Pentáptero}
\begin{itemize}
\item {Grp. gram.:adj.}
\end{itemize}
\begin{itemize}
\item {Utilização:Bot.}
\end{itemize}
\begin{itemize}
\item {Proveniência:(Do gr. \textunderscore pente\textunderscore  + \textunderscore pteron\textunderscore )}
\end{itemize}
Diz-se da carcérula, que tem cinco asas.
\section{Pentaptoto}
\begin{itemize}
\item {Grp. gram.:adj.}
\end{itemize}
\begin{itemize}
\item {Utilização:Gram.}
\end{itemize}
\begin{itemize}
\item {Proveniência:(Do gr. \textunderscore pente\textunderscore  + \textunderscore ptotos\textunderscore )}
\end{itemize}
Diz-se dos nomes latinos que, no singular, têm cinco terminações differentes, como \textunderscore servus\textunderscore , \textunderscore dux\textunderscore , etc.
\section{Pentarca}
\begin{itemize}
\item {Grp. gram.:m.}
\end{itemize}
\begin{itemize}
\item {Proveniência:(Do gr. \textunderscore pente\textunderscore  + \textunderscore arkhein\textunderscore )}
\end{itemize}
Membro de uma pentarchia.
\section{Pentarcado}
\begin{itemize}
\item {Grp. gram.:m.}
\end{itemize}
Dignidade ou funções de pentarca.
\section{Pentarcha}
\begin{itemize}
\item {fónica:ca}
\end{itemize}
\begin{itemize}
\item {Grp. gram.:m.}
\end{itemize}
\begin{itemize}
\item {Proveniência:(Do gr. \textunderscore pente\textunderscore  + \textunderscore arkhein\textunderscore )}
\end{itemize}
Membro de uma pentarchia.
\section{Pentarchado}
\begin{itemize}
\item {fónica:ca}
\end{itemize}
\begin{itemize}
\item {Grp. gram.:m.}
\end{itemize}
Dignidade ou funcções de pentarcha.
\section{Pentarchia}
\begin{itemize}
\item {fónica:qui}
\end{itemize}
\begin{itemize}
\item {Grp. gram.:f.}
\end{itemize}
Govêrno, exercido por cinco chefes.
(Cp. \textunderscore pentarcha\textunderscore )
\section{Pentarino}
\begin{itemize}
\item {Grp. gram.:adj.}
\end{itemize}
\begin{itemize}
\item {Utilização:Bot.}
\end{itemize}
(V.pentandro)
\section{Pentarquia}
\begin{itemize}
\item {Grp. gram.:f.}
\end{itemize}
Govêrno, exercido por cinco chefes.
(Cp. \textunderscore pentarca\textunderscore )
\section{Pentasépalo}
\begin{itemize}
\item {fónica:se}
\end{itemize}
\begin{itemize}
\item {Grp. gram.:adj.}
\end{itemize}
\begin{itemize}
\item {Utilização:Bot.}
\end{itemize}
\begin{itemize}
\item {Proveniência:(Do gr. \textunderscore pente\textunderscore  + \textunderscore sepale\textunderscore )}
\end{itemize}
Diz-se do cálice, que tem cinco sépalas.
\section{Pentaspermo}
\begin{itemize}
\item {Grp. gram.:adj.}
\end{itemize}
\begin{itemize}
\item {Utilização:Bot.}
\end{itemize}
\begin{itemize}
\item {Proveniência:(Do gr. \textunderscore pente\textunderscore  + \textunderscore sperma\textunderscore )}
\end{itemize}
Diz-se do fruto ou da céllula do pericarpo, que contém cinco grãos.
\section{Pentassépalo}
\begin{itemize}
\item {Grp. gram.:adj.}
\end{itemize}
\begin{itemize}
\item {Utilização:Bot.}
\end{itemize}
\begin{itemize}
\item {Proveniência:(Do gr. \textunderscore pente\textunderscore  + \textunderscore sepale\textunderscore )}
\end{itemize}
Diz-se do cálice, que tem cinco sépalas.
\section{Pentassílabo}
\begin{itemize}
\item {Grp. gram.:m.  e  adj.}
\end{itemize}
\begin{itemize}
\item {Proveniência:(Do gr. \textunderscore pente\textunderscore  + \textunderscore sullabe\textunderscore )}
\end{itemize}
Que tem cinco sílabas.
\section{Pentastilo}
\begin{itemize}
\item {Grp. gram.:m.}
\end{itemize}
\begin{itemize}
\item {Grp. gram.:Adj.}
\end{itemize}
\begin{itemize}
\item {Utilização:Bot.}
\end{itemize}
\begin{itemize}
\item {Proveniência:(Do gr. \textunderscore pente\textunderscore  + \textunderscore stulos\textunderscore )}
\end{itemize}
Pórtico, ou edifício, com cinco columnas no frontispício.
Diz-se do ovário, que tem cinco estiletes.
\section{Pentastylo}
\begin{itemize}
\item {Grp. gram.:m.}
\end{itemize}
\begin{itemize}
\item {Grp. gram.:Adj.}
\end{itemize}
\begin{itemize}
\item {Utilização:Bot.}
\end{itemize}
\begin{itemize}
\item {Proveniência:(Do gr. \textunderscore pente\textunderscore  + \textunderscore stulos\textunderscore )}
\end{itemize}
Pórtico, ou edifício, com cinco columnas no frontispício.
Diz-se do ovário, que tem cinco estiletes.
\section{Pentasýllabo}
\begin{itemize}
\item {Grp. gram.:m.  e  adj.}
\end{itemize}
\begin{itemize}
\item {Proveniência:(Do gr. \textunderscore pente\textunderscore  + \textunderscore sullabe\textunderscore )}
\end{itemize}
Que tem cinco sýllabas.
\section{Pentateucho}
\begin{itemize}
\item {fónica:co}
\end{itemize}
\begin{itemize}
\item {Grp. gram.:m.}
\end{itemize}
\begin{itemize}
\item {Proveniência:(Gr. \textunderscore pentateukhos\textunderscore )}
\end{itemize}
Os cinco primeiros livros da \textunderscore Bíblia\textunderscore .
\section{Pentateuco}
\begin{itemize}
\item {Grp. gram.:m.}
\end{itemize}
\begin{itemize}
\item {Proveniência:(Gr. \textunderscore pentateukhos\textunderscore )}
\end{itemize}
Os cinco primeiros livros da \textunderscore Bíblia\textunderscore .
\section{Pentathlo}
\begin{itemize}
\item {Grp. gram.:m.}
\end{itemize}
\begin{itemize}
\item {Proveniência:(Gr. \textunderscore pentathlon\textunderscore )}
\end{itemize}
Entre os antigos Gregos, o conjunto dos cinco exercícios,--da luta, da carreira, da barra, do salto e da frecha.
\section{Pentatlo}
\begin{itemize}
\item {Grp. gram.:m.}
\end{itemize}
\begin{itemize}
\item {Proveniência:(Gr. \textunderscore pentathlon\textunderscore )}
\end{itemize}
Entre os antigos Gregos, o conjunto dos cinco exercícios,--da luta, da carreira, da barra, do salto e da frecha.
\section{Pentatómico}
\begin{itemize}
\item {Grp. gram.:adj.}
\end{itemize}
\begin{itemize}
\item {Proveniência:(Do gr. \textunderscore pente\textunderscore  + \textunderscore tome\textunderscore )}
\end{itemize}
Diz-se de um metallóide do quinto grupo, como o arsênio, o bismutho, etc.; e dos metaes da quinta classe.
\section{Pentatónico}
\begin{itemize}
\item {Grp. gram.:adj.}
\end{itemize}
Relativo ao pentátono.
\section{Pentátono}
\begin{itemize}
\item {Grp. gram.:m.}
\end{itemize}
\begin{itemize}
\item {Proveniência:(Do gr. \textunderscore pente\textunderscore  + \textunderscore tonos\textunderscore )}
\end{itemize}
Intervallo de cinco sons na antiga música grega.
\section{Pente}
\begin{itemize}
\item {Grp. gram.:m.}
\end{itemize}
\begin{itemize}
\item {Utilização:Gír.}
\end{itemize}
\begin{itemize}
\item {Utilização:Gír. de Lisbôa.}
\end{itemize}
\begin{itemize}
\item {Utilização:Prov.}
\end{itemize}
\begin{itemize}
\item {Proveniência:(Do lat. \textunderscore pecten\textunderscore )}
\end{itemize}
Instrumento, com que se alisa o cabello ou se limpa a cabeça.
Utensílio análogo, com que as senhoras seguram ou enfeitam o cabello.
Caixilho, com aberturas perpendiculares, por onde passam os fios de uma teia.
Instrumento de ferro, com que os cardadores preparam a lan.
Utensílio de bordadeira, para limpar bordados de ponto alto.
Peça de madeira, com muitos furos, pelos quaes passam os fios, com que os esteireiros apertam ou comprimem as esteiras, á proporção que as vão fazendo.
Nome, que alguns anatómicos dão ao púbis.
No jôgo da bisca, é o facto de um parceiro ou de uma parçaria ganhar a partida, fazendo quatro jogos seguidos, sem que o jogador ou jogadores contrários tenham feito algum.
O mesmo que \textunderscore amásia\textunderscore .
Mocetona airosa.
Extremidade das aduelas, desde o javre em deante e que fórma a testeira da pipa ou tonel.
\section{Penteaço}
\begin{itemize}
\item {Grp. gram.:m.}
\end{itemize}
\begin{itemize}
\item {Utilização:Carp.}
\end{itemize}
\begin{itemize}
\item {Proveniência:(De \textunderscore pente\textunderscore )}
\end{itemize}
Tabuão ou tronco, que se serra em vários fios, não chegando a serração até ao fim do tabuão, mas ficando ainda ligadas por um pequeno espaço todas as tábuas ou ripas que se serraram.
\section{Penteadeira}
\begin{itemize}
\item {Grp. gram.:f.}
\end{itemize}
\begin{itemize}
\item {Utilização:Prov.}
\end{itemize}
\begin{itemize}
\item {Utilização:alent.}
\end{itemize}
\begin{itemize}
\item {Proveniência:(De \textunderscore pentear\textunderscore )}
\end{itemize}
O mesmo que \textunderscore cabelleireira\textunderscore .
\section{Penteadela}
\begin{itemize}
\item {Grp. gram.:f.}
\end{itemize}
\begin{itemize}
\item {Utilização:Prov.}
\end{itemize}
\begin{itemize}
\item {Utilização:minh.}
\end{itemize}
Acto ou effeito do pentear ligeiramente ou á pressa.
Sova, tunda.
\section{Penteado}
\begin{itemize}
\item {Grp. gram.:m.}
\end{itemize}
\begin{itemize}
\item {Proveniência:(De \textunderscore pentear\textunderscore )}
\end{itemize}
Compostura do cabello.
Toucado.
\section{Penteador}
\begin{itemize}
\item {Grp. gram.:adj.}
\end{itemize}
\begin{itemize}
\item {Grp. gram.:M.}
\end{itemize}
\begin{itemize}
\item {Proveniência:(De \textunderscore pentear\textunderscore )}
\end{itemize}
Que penteia.
Indivíduo, que penteia.
Roupão ou espécie de toalha, que se colloca nos hombros de quem se penteia ou corta o cabello.
\section{Penteadura}
\begin{itemize}
\item {Grp. gram.:f.}
\end{itemize}
Acto ou effeito de pentear.
\section{Pentear}
\begin{itemize}
\item {Grp. gram.:v. t.}
\end{itemize}
\begin{itemize}
\item {Grp. gram.:Loc.}
\end{itemize}
\begin{itemize}
\item {Utilização:Prov.}
\end{itemize}
\begin{itemize}
\item {Grp. gram.:V. p.}
\end{itemize}
\begin{itemize}
\item {Utilização:Fig.}
\end{itemize}
Compor, alisar ou limpar com o pente (os cabellos).
\textunderscore Vá pentear macacos\textunderscore , vá bujiar, vá tratar de outra vida.
Aspirar; preparar-se: \textunderscore penteia-se para deputado\textunderscore .
\section{Pentearia}
\begin{itemize}
\item {Grp. gram.:f.}
\end{itemize}
\begin{itemize}
\item {Proveniência:(De \textunderscore pentear\textunderscore )}
\end{itemize}
Officina ou estabelecimento de pentieiro.
\section{Pentecoste}
\begin{itemize}
\item {Grp. gram.:m.}
\end{itemize}
\begin{itemize}
\item {Proveniência:(Gr. \textunderscore pentekoste\textunderscore )}
\end{itemize}
Festa, com que os Christãos celebram o sétimo Domingo depois da Páschoa, em memória da descida do Espirito-Santo sôbre os Apóstolos.
Festa, que os Judeus celebravam no segundo dia depois da Páschoa.
\section{Pentedecágono}
\begin{itemize}
\item {Grp. gram.:m.}
\end{itemize}
\begin{itemize}
\item {Proveniência:(Do gr. \textunderscore pente\textunderscore  + \textunderscore deka\textunderscore  + \textunderscore gonos\textunderscore )}
\end{itemize}
Polýgono de 15 lados.
\section{Pentélico}
\begin{itemize}
\item {Grp. gram.:adj.}
\end{itemize}
\begin{itemize}
\item {Proveniência:(Gr. \textunderscore pentelikos\textunderscore )}
\end{itemize}
Diz-se do mármore, muito apreciado, que abundava no monte Pentélico, perto de Athenas.
\section{Pêntem}
\begin{itemize}
\item {Grp. gram.:m.}
\end{itemize}
\begin{itemize}
\item {Utilização:Ant.}
\end{itemize}
O mesmo que \textunderscore pente\textunderscore .
Instrumento músico. Cf. Camões, \textunderscore Filodemo\textunderscore , act. V, sc. 2; \textunderscore Lusiadas\textunderscore , VI, 17.
\section{Penteola}
\begin{itemize}
\item {Grp. gram.:f.}
\end{itemize}
\begin{itemize}
\item {Proveniência:(De \textunderscore pente\textunderscore )}
\end{itemize}
Mollusco acéphalo.
Concha de romeiro.
\section{Pentétria}
\begin{itemize}
\item {Grp. gram.:f.}
\end{itemize}
Gênero de insectos dípteros.
\section{Penthétria}
\begin{itemize}
\item {Grp. gram.:f.}
\end{itemize}
Gênero de insectos dípteros.
\section{Penthóphera}
\begin{itemize}
\item {Grp. gram.:f.}
\end{itemize}
Gênero de insectos lepidópteros nocturnos.
\section{Penthoro}
\begin{itemize}
\item {Grp. gram.:m.}
\end{itemize}
Gênero de plantas crassuláceas.
\section{Pentieiro}
\begin{itemize}
\item {Grp. gram.:m.}
\end{itemize}
\begin{itemize}
\item {Proveniência:(Do lat. \textunderscore pectinarius\textunderscore )}
\end{itemize}
Fabricante ou vendedor de pentes.
\section{Pentlândia}
\begin{itemize}
\item {Grp. gram.:f.}
\end{itemize}
Gênero do plantas amaryllídeas.
\section{Pentófera}
\begin{itemize}
\item {Grp. gram.:f.}
\end{itemize}
Gênero de insectos lepidópteros nocturnos.
\section{Pentoro}
\begin{itemize}
\item {Grp. gram.:m.}
\end{itemize}
Gênero de plantas crassuláceas.
\section{Penudo}
\begin{itemize}
\item {Grp. gram.:adj.}
\end{itemize}
O mesmo que \textunderscore pennígero\textunderscore .
\section{Penugem}
\begin{itemize}
\item {Grp. gram.:f.}
\end{itemize}
\begin{itemize}
\item {Proveniência:(De \textunderscore pena\textunderscore )}
\end{itemize}
As penas, que primeiro nascem nas aves.
Os pelos e cabelos, que primeiro nascem.
Pêlo macio e curto.
Buço.
Froixel.
Espécie de pelos, nas cascas de frutos ou plantas.
\section{Penujar}
\begin{itemize}
\item {Grp. gram.:v. i.}
\end{itemize}
Mostrar-se coberto de penugem.
\section{Penujento}
\begin{itemize}
\item {Grp. gram.:adj.}
\end{itemize}
Cheio ou coberto de penugem.
\section{Penujoso}
\begin{itemize}
\item {Grp. gram.:adj.}
\end{itemize}
O mesmo que \textunderscore penujento\textunderscore .
\section{Pênula}
\begin{itemize}
\item {Grp. gram.:f.}
\end{itemize}
\begin{itemize}
\item {Proveniência:(Lat. \textunderscore pennula\textunderscore )}
\end{itemize}
Pena, com que os jograes do século XIII tocavam a cítara, e com que, na antiguidade, se faziam vibrar outros instrumentos. Cf. \textunderscore Cancion. da Vaticana\textunderscore .
\section{Pênula}
\begin{itemize}
\item {Grp. gram.:f.}
\end{itemize}
\begin{itemize}
\item {Utilização:Ant.}
\end{itemize}
\begin{itemize}
\item {Proveniência:(Lat. \textunderscore paenula\textunderscore )}
\end{itemize}
Manto, capa.
\section{Penúltimo}
\begin{itemize}
\item {Grp. gram.:adj.}
\end{itemize}
\begin{itemize}
\item {Proveniência:(Lat. \textunderscore penultimus\textunderscore )}
\end{itemize}
Que precede immediatamente o último.
\section{Penumbra}
\begin{itemize}
\item {Grp. gram.:f.}
\end{itemize}
\begin{itemize}
\item {Utilização:Ext.}
\end{itemize}
\begin{itemize}
\item {Utilização:Fig.}
\end{itemize}
\begin{itemize}
\item {Proveniência:(Do lat. \textunderscore pene\textunderscore  + \textunderscore umbra\textunderscore )}
\end{itemize}
Sombra incompleta, produzida por um corpo, que não intercepta inteiramente os raios luminosos.
Meia luz.
Gradação de luz para a sombra.
Retrahimento, insulamento: \textunderscore compraz-se em viver na penumbra\textunderscore .
\section{Penumbrar}
\begin{itemize}
\item {Grp. gram.:v. t.}
\end{itemize}
\begin{itemize}
\item {Utilização:bras}
\end{itemize}
\begin{itemize}
\item {Utilização:Neol.}
\end{itemize}
Causar penumbra em. Cf. Af. Celso, \textunderscore Minha Filha\textunderscore .
\section{Penumbroso}
\begin{itemize}
\item {Grp. gram.:adj.}
\end{itemize}
Em que há penumbra.
\section{Penúria}
\begin{itemize}
\item {Grp. gram.:f.}
\end{itemize}
\begin{itemize}
\item {Proveniência:(Lat. \textunderscore penuria\textunderscore )}
\end{itemize}
Miséria extrema; privação do que é necessário; pobreza.
\section{Peó}
\begin{itemize}
\item {Grp. gram.:m.}
\end{itemize}
\begin{itemize}
\item {Proveniência:(Do lat. hyp. \textunderscore pediola\textunderscore )}
\end{itemize}
Correia, que os caçadores de altanaria punham nos sancos do falcão ou do açor. Cf. Fernandes, \textunderscore Caça de Altan.\textunderscore , onde se lê \textunderscore pió\textunderscore .
\section{Peoada}
\begin{itemize}
\item {Grp. gram.:f.}
\end{itemize}
\begin{itemize}
\item {Proveniência:(De \textunderscore peão\textunderscore )}
\end{itemize}
O mesmo que \textunderscore peonagem\textunderscore . Cf. Herculano, \textunderscore Bobo\textunderscore , 220.
\section{Peoagem}
\begin{itemize}
\item {Grp. gram.:f.}
\end{itemize}
\begin{itemize}
\item {Utilização:Ant.}
\end{itemize}
O mesmo que \textunderscore peonagem\textunderscore .
\section{Péon}
\begin{itemize}
\item {Grp. gram.:m.}
\end{itemize}
\begin{itemize}
\item {Proveniência:(Lat. \textunderscore paeon\textunderscore )}
\end{itemize}
Pé de verso grego e latino, composto de três sýllabas breves e uma longa.
\section{Peonagem}
\begin{itemize}
\item {Grp. gram.:f.}
\end{itemize}
\begin{itemize}
\item {Proveniência:(De \textunderscore peão\textunderscore )}
\end{itemize}
Gente, que anda a pé; peões.
Soldados de infantaria:«\textunderscore a peonagem, que restava ao infante, abandonou-o...\textunderscore »Oliv. Martins, \textunderscore Filhos de D. João I\textunderscore , 311.
\section{Peónia}
\begin{itemize}
\item {Grp. gram.:f.}
\end{itemize}
\begin{itemize}
\item {Proveniência:(Lat. \textunderscore paeonia\textunderscore )}
\end{itemize}
Planta ranunculácea.
\section{Peoniáceas}
\begin{itemize}
\item {Grp. gram.:f.}
\end{itemize}
\begin{itemize}
\item {Proveniência:(De \textunderscore peónia\textunderscore )}
\end{itemize}
Tríbo de plantas ranunculáceas, estabelecidas por De-Candolle.
\section{Peopaias}
\begin{itemize}
\item {Grp. gram.:m.}
\end{itemize}
Índios do Brasil, nas vizinhanças do Xingu.
\section{Peór}
\begin{itemize}
\item {Grp. gram.:adj.}
\end{itemize}
\begin{itemize}
\item {Grp. gram.:Adv.}
\end{itemize}
\begin{itemize}
\item {Proveniência:(Lat. \textunderscore peior\textunderscore )}
\end{itemize}
Mais mau.
Que se aggrava, em sentido desfavorável.
Que excede outro em maldade ou má qualidade.
De modo mais mau; em situação ou circumstâncias mais desfavoráveis.
\section{Peóra}
\begin{itemize}
\item {Grp. gram.:f.}
\end{itemize}
Acto ou effeito de peorar.
\section{Peoramento}
\begin{itemize}
\item {Grp. gram.:m.}
\end{itemize}
O mesmo que \textunderscore peóra\textunderscore .
\section{Peorar}
\begin{itemize}
\item {Grp. gram.:v. t.}
\end{itemize}
\begin{itemize}
\item {Grp. gram.:V. i.}
\end{itemize}
\begin{itemize}
\item {Proveniência:(Do lat. \textunderscore pejorare\textunderscore )}
\end{itemize}
Tornar peór, mudar para peór estado.
Tornar-se peór.
Aggravar-se, passar a peór estado: \textunderscore o doente peorou\textunderscore .
\section{Peorativo}
\begin{itemize}
\item {Grp. gram.:adj.}
\end{itemize}
(V.pejorativo)
\section{Peoria}
\begin{itemize}
\item {Grp. gram.:f.}
\end{itemize}
Qualidade do que é peór.
Peóra.
\section{Peouga}
\begin{itemize}
\item {Grp. gram.:f.}
\end{itemize}
\begin{itemize}
\item {Utilização:Ant.}
\end{itemize}
Pé de porco.
(Cp. \textunderscore peúga\textunderscore )
\section{Pepa}
\begin{itemize}
\item {Grp. gram.:f.}
\end{itemize}
Espécie de guitarra, usada pelas damas chinesas. Cf. \textunderscore Diccion. Mus.\textunderscore 
\section{Pepasmo}
\begin{itemize}
\item {Grp. gram.:m.}
\end{itemize}
\begin{itemize}
\item {Utilização:Med.}
\end{itemize}
\begin{itemize}
\item {Proveniência:(Gr. \textunderscore pepasmos\textunderscore )}
\end{itemize}
Estado, em que uma doença perdeu já o seu carácter agudo.
\section{Pepástico}
\begin{itemize}
\item {Grp. gram.:adj.}
\end{itemize}
\begin{itemize}
\item {Utilização:Med.}
\end{itemize}
\begin{itemize}
\item {Proveniência:(Gr. \textunderscore pepastikos\textunderscore )}
\end{itemize}
Diz-se do medicamento, que facilita a digestão dos alimentos.
\section{Pepe}
\begin{itemize}
\item {Grp. gram.:m.}
\end{itemize}
Nome de duas árvores africanas, o pepe claro e o pepe escuro.
\section{Pèpé}
\begin{itemize}
\item {Grp. gram.:adj.}
\end{itemize}
\begin{itemize}
\item {Utilização:Bras}
\end{itemize}
\begin{itemize}
\item {Utilização:fam.}
\end{itemize}
\begin{itemize}
\item {Proveniência:(De \textunderscore pé\textunderscore )}
\end{itemize}
Coxo; que manqueja de um pé.
\section{Pé-pecelho}
\begin{itemize}
\item {fónica:cê}
\end{itemize}
\begin{itemize}
\item {Grp. gram.:m.}
\end{itemize}
\begin{itemize}
\item {Utilização:T. de Alcanena}
\end{itemize}
Jôgo de rapazes, o mesmo que \textunderscore jôgo do homem\textunderscore .
\section{Pepéis}
\begin{itemize}
\item {Grp. gram.:m. pl.}
\end{itemize}
(V.papeles)
\section{Pepel}
\begin{itemize}
\item {Grp. gram.:m.}
\end{itemize}
O mesmo ou melhor que \textunderscore papel\textunderscore ^2.
\section{Peperómia}
\begin{itemize}
\item {Grp. gram.:f.}
\end{itemize}
\begin{itemize}
\item {Proveniência:(Do lat. \textunderscore piper\textunderscore  + gr. \textunderscore omos\textunderscore )}
\end{itemize}
Gênero de plantas piperáceas.
\section{Pépia}
\begin{itemize}
\item {Grp. gram.:f.}
\end{itemize}
\begin{itemize}
\item {Utilização:Prov.}
\end{itemize}
\begin{itemize}
\item {Utilização:trasm.}
\end{itemize}
O mesmo que \textunderscore concubina\textunderscore .
\section{Pepinal}
\begin{itemize}
\item {Grp. gram.:m.}
\end{itemize}
Terreno, onde crescem pepinos.
\section{Pepinar}
\begin{itemize}
\item {Grp. gram.:v. t.  e  i.}
\end{itemize}
\begin{itemize}
\item {Utilização:Bras}
\end{itemize}
Comer aos poucos, devagar; peniscar.
\section{Pepineira}
\begin{itemize}
\item {Grp. gram.:f.}
\end{itemize}
\begin{itemize}
\item {Utilização:Pop.}
\end{itemize}
\begin{itemize}
\item {Utilização:Fig.}
\end{itemize}
\begin{itemize}
\item {Proveniência:(De \textunderscore pepino\textunderscore )}
\end{itemize}
Pepinal.
Viveiro; mina.
Patuscada, pândega.
Pechincha.
\section{Pepineiro}
\begin{itemize}
\item {Grp. gram.:m.}
\end{itemize}
\begin{itemize}
\item {Proveniência:(De \textunderscore pepino\textunderscore )}
\end{itemize}
Planta cucurbitácea, (\textunderscore cucumus sativus\textunderscore ).
\section{Pepinela}
\begin{itemize}
\item {Grp. gram.:f.}
\end{itemize}
\begin{itemize}
\item {Utilização:Pop.}
\end{itemize}
O mesmo que \textunderscore pimpinela\textunderscore . Cf. \textunderscore Bibl. da G. do Campo\textunderscore , 310.
\section{Pepino}
\begin{itemize}
\item {Grp. gram.:m.}
\end{itemize}
\begin{itemize}
\item {Proveniência:(Do lat. \textunderscore pepo\textunderscore )}
\end{itemize}
Fruto do pepineiro; pepineiro.
\section{Pepino-chôco}
\begin{itemize}
\item {Grp. gram.:m.}
\end{itemize}
\begin{itemize}
\item {Utilização:Prov.}
\end{itemize}
\begin{itemize}
\item {Utilização:alent.}
\end{itemize}
Homem fraco ou doente.
\section{Pepira}
\begin{itemize}
\item {Grp. gram.:f.}
\end{itemize}
\begin{itemize}
\item {Utilização:Bras}
\end{itemize}
Pássaro, nocivo aos frutos.
\section{Pepita}
\begin{itemize}
\item {Grp. gram.:f.}
\end{itemize}
\begin{itemize}
\item {Utilização:Miner.}
\end{itemize}
Grão ou paleta de oiro, que se acha entre as areias de alguns rios. Cf. Camillo, \textunderscore Amor de Perd.\textunderscore , XXV.
\section{Pepitória}
\begin{itemize}
\item {Grp. gram.:f.}
\end{itemize}
\begin{itemize}
\item {Utilização:Ant.}
\end{itemize}
\begin{itemize}
\item {Proveniência:(Do b. lat. \textunderscore piperitoria\textunderscore , de \textunderscore piper\textunderscore )}
\end{itemize}
Guisado, feito de miudezas de aves.
\section{Peplo}
\begin{itemize}
\item {Grp. gram.:m.}
\end{itemize}
\begin{itemize}
\item {Proveniência:(Lat. \textunderscore peplum\textunderscore )}
\end{itemize}
Capa comprida das matronas romanas, de rico tecido e côres vivas, ornada de pregos de oiro e figuras de deuses e heróis. Cf. Franc. Barreto, \textunderscore Eneida\textunderscore , I, 112.
\section{Pepoaza}
\begin{itemize}
\item {Grp. gram.:f.}
\end{itemize}
Ave das margens do rio La-Plata, (\textunderscore taenioptera pepoaza\textunderscore ).
Nome, que os Guaranis davam a toda a ave, cujas asas são atravessadas por uma lista ou faixa de côr differente da do resto das pennas.
\section{Pèpolim}
\begin{itemize}
\item {Grp. gram.:adj.}
\end{itemize}
\begin{itemize}
\item {Utilização:Ant.}
\end{itemize}
\begin{itemize}
\item {Proveniência:(De \textunderscore pé\textunderscore  + \textunderscore pulinho\textunderscore , dem. de \textunderscore pulo\textunderscore )}
\end{itemize}
Coxo.
\section{Peponídeo}
\begin{itemize}
\item {Grp. gram.:adj.}
\end{itemize}
\begin{itemize}
\item {Proveniência:(Do lat. \textunderscore pepo\textunderscore  + gr. \textunderscore eidos\textunderscore )}
\end{itemize}
Diz-se do fruto que tem mesocarpo volumoso e carnudo, e grande cavidade cheia de placentas com muita semente.
\section{Pepsia}
\begin{itemize}
\item {Grp. gram.:f.}
\end{itemize}
\begin{itemize}
\item {Proveniência:(Gr. \textunderscore pepsis\textunderscore )}
\end{itemize}
Cocção dos alimentos no estômago.
\section{Pepsina}
\begin{itemize}
\item {Grp. gram.:f.}
\end{itemize}
\begin{itemize}
\item {Proveniência:(Do gr. \textunderscore pepsis\textunderscore )}
\end{itemize}
Substância amarelada, espécie de fermento, que, com o suco gástrico, serve para dissolver os alimentos no estômago.
\section{Peptagogo}
\begin{itemize}
\item {Grp. gram.:adj.}
\end{itemize}
\begin{itemize}
\item {Utilização:Med.}
\end{itemize}
Diz-se dos agentes, que augmentam a secrecção do suco gástrico.
\section{Péptico}
\begin{itemize}
\item {Grp. gram.:adj.}
\end{itemize}
\begin{itemize}
\item {Proveniência:(Gr. \textunderscore peptikos\textunderscore )}
\end{itemize}
Que auxilía a digestão dos alimentos.
\section{Peptocola}
\begin{itemize}
\item {Grp. gram.:f.}
\end{itemize}
Producto pharmacêutico, empregado no tratamento da neurasthenia.
\section{Peptógeno}
\begin{itemize}
\item {Grp. gram.:adj.}
\end{itemize}
\begin{itemize}
\item {Utilização:Chím.}
\end{itemize}
\begin{itemize}
\item {Proveniência:(Do gr. \textunderscore peptos\textunderscore  + \textunderscore genos\textunderscore )}
\end{itemize}
Diz-se das substâncias, que, ingeridas, têm a propriedade de determinar a producção de pepsina no suco gástrico.
\section{Peptona}
\begin{itemize}
\item {Grp. gram.:f.}
\end{itemize}
Producto da digestão gástrica das substâncias azotadas ou albuminóides, o qual póde sêr obtido artificialmente nos laboratórios chímicos.
(Cp. \textunderscore pepsina\textunderscore )
\section{Peptonato}
\begin{itemize}
\item {Grp. gram.:m.}
\end{itemize}
\begin{itemize}
\item {Proveniência:(De \textunderscore peptona\textunderscore )}
\end{itemize}
Producto chímico, obtido pela acção das peptonas sôbre certos saes metállicos.
\section{Peptonização}
\begin{itemize}
\item {Grp. gram.:f.}
\end{itemize}
Acto de peptonizar.
\section{Peptonizar}
\begin{itemize}
\item {Grp. gram.:v. t.}
\end{itemize}
Converter em peptona no estômago (os alimentos), com a acção da pepsina.
\section{Peptonuria}
\begin{itemize}
\item {Grp. gram.:f.}
\end{itemize}
\begin{itemize}
\item {Proveniência:(De \textunderscore peptona\textunderscore  + gr. \textunderscore ouron\textunderscore )}
\end{itemize}
Presença da peptona na urina.
\section{Peptoxina}
\begin{itemize}
\item {fónica:csi}
\end{itemize}
\begin{itemize}
\item {Grp. gram.:f.}
\end{itemize}
Ptomaína, que se encontra no suco gástrico.
\section{Pepuíra}
\begin{itemize}
\item {Grp. gram.:f.}
\end{itemize}
\begin{itemize}
\item {Utilização:Bras}
\end{itemize}
Gallinha pequena.
\section{Pequear}
\begin{itemize}
\item {Grp. gram.:v. i.}
\end{itemize}
\begin{itemize}
\item {Utilização:Prov.}
\end{itemize}
\begin{itemize}
\item {Utilização:minh.}
\end{itemize}
\begin{itemize}
\item {Proveniência:(De \textunderscore pêco\textunderscore )}
\end{itemize}
Trabalhar com molleza; calacear.
\section{Pequenada}
\begin{itemize}
\item {Grp. gram.:f.}
\end{itemize}
Conjunto de filhos pequenos; filharada.
\section{Pequenez}
\begin{itemize}
\item {Grp. gram.:f.}
\end{itemize}
\begin{itemize}
\item {Utilização:Fig.}
\end{itemize}
Qualidade do que é pequeno.
Meninice.
Mesquinhez. Humildade.
\section{Pequeneza}
\begin{itemize}
\item {Grp. gram.:f.}
\end{itemize}
O mesmo que \textunderscore pequenez\textunderscore .
\section{Pequenhez}
\begin{itemize}
\item {Grp. gram.:f.}
\end{itemize}
\begin{itemize}
\item {Utilização:Ant.}
\end{itemize}
O mesmo que \textunderscore pequenez\textunderscore .
\section{Pequenice}
\begin{itemize}
\item {Grp. gram.:f.}
\end{itemize}
\begin{itemize}
\item {Proveniência:(De \textunderscore pequeno\textunderscore )}
\end{itemize}
Pequice, ninharia.
\section{Pequeninez}
\begin{itemize}
\item {Grp. gram.:f.}
\end{itemize}
\begin{itemize}
\item {Utilização:Des.}
\end{itemize}
Qualidade de pequenino.
Pusillanimidade; cobardia.
\section{Pequeninho}
\begin{itemize}
\item {Grp. gram.:adj.}
\end{itemize}
\begin{itemize}
\item {Utilização:Prov.}
\end{itemize}
\begin{itemize}
\item {Utilização:Ant.}
\end{itemize}
O mesmo que \textunderscore pequenino\textunderscore :«\textunderscore ...pedra muito pequeninha.\textunderscore »G. Vicente, \textunderscore Auto da Índia\textunderscore .
\section{Pequenino}
\begin{itemize}
\item {Grp. gram.:adj.}
\end{itemize}
\begin{itemize}
\item {Grp. gram.:M.}
\end{itemize}
Muito pequeno.
Menino.
\section{Pequenitátis}
\begin{itemize}
\item {Grp. gram.:m.  e  f.}
\end{itemize}
\begin{itemize}
\item {Utilização:Fam.}
\end{itemize}
O mesmo que \textunderscore criança\textunderscore .
\section{Pequeno}
\begin{itemize}
\item {Grp. gram.:adj.}
\end{itemize}
\begin{itemize}
\item {Grp. gram.:M.}
\end{itemize}
\begin{itemize}
\item {Utilização:Ant.}
\end{itemize}
\begin{itemize}
\item {Grp. gram.:Pl.}
\end{itemize}
Que tem pouca extensão ou volume.
Que é criança.
Que é de pequena estatura.
Limitado.
Que tem pouco valor ou importância.
Mesquinho, humilde.
Menino; rapaz.
Pedaço, bocado:«\textunderscore um pequeno de mau caminho.\textunderscore »Pant. de Aveiro, \textunderscore Itiner.\textunderscore , 196, (3.^a ed.).
Plebe; a classe inferior da sociedade.
(Ref. de \textunderscore pêco\textunderscore ?)
\section{Pequenote}
\begin{itemize}
\item {Grp. gram.:adj.}
\end{itemize}
\begin{itemize}
\item {Grp. gram.:M.}
\end{itemize}
Um tanto pequeno.
Rapaz.
\section{Pequerralho}
\begin{itemize}
\item {Grp. gram.:m.  e  adj.}
\end{itemize}
\begin{itemize}
\item {Utilização:Prov.}
\end{itemize}
\begin{itemize}
\item {Utilização:alg.}
\end{itemize}
Pequeno, pequerrucho.
(Cp. \textunderscore pequerrucho\textunderscore )
\section{Pequerrelho}
\begin{itemize}
\item {fónica:rê}
\end{itemize}
\begin{itemize}
\item {Grp. gram.:m.}
\end{itemize}
\begin{itemize}
\item {Utilização:Prov.}
\end{itemize}
\begin{itemize}
\item {Utilização:minh.}
\end{itemize}
O mesmo que \textunderscore pequerralho\textunderscore .
\section{Pequerruchada}
\begin{itemize}
\item {Grp. gram.:f.}
\end{itemize}
Porção de pequerruchos.
\section{Pequerrucho}
\begin{itemize}
\item {Grp. gram.:m.  e  adj.}
\end{itemize}
\begin{itemize}
\item {Proveniência:(Do rad. de \textunderscore pequeno\textunderscore )}
\end{itemize}
Menino, criança. Cf. Castilho, \textunderscore Fausto\textunderscore , 241.
\section{Pequi}
\begin{itemize}
\item {Grp. gram.:m.}
\end{itemize}
Planta silvestre do Brasil; fructo dessa planta.
\section{Pequiá}
\begin{itemize}
\item {Grp. gram.:m.}
\end{itemize}
O mesmo que \textunderscore pequi\textunderscore ^1.
\section{Pequiagra}
\begin{itemize}
\item {Grp. gram.:f.}
\end{itemize}
\begin{itemize}
\item {Utilização:Med.}
\end{itemize}
\begin{itemize}
\item {Proveniência:(Do gr. \textunderscore pekhus\textunderscore  + \textunderscore agra\textunderscore )}
\end{itemize}
Dôr de gota, que se fixou no cotovelo.
\section{Pequice}
\begin{itemize}
\item {Grp. gram.:f.}
\end{itemize}
Acto ou dito de quem é pêco.
Frioleira.
Caturrice.
Sandice.
\section{Pequim}
\begin{itemize}
\item {Grp. gram.:m.}
\end{itemize}
\begin{itemize}
\item {Utilização:Bras}
\end{itemize}
O mesmo que \textunderscore pequi\textunderscore ^1.
\section{Pequito}
\begin{itemize}
\item {Grp. gram.:m.}
\end{itemize}
\begin{itemize}
\item {Utilização:Prov.}
\end{itemize}
\begin{itemize}
\item {Utilização:trasm.}
\end{itemize}
O mesmo que \textunderscore periquito\textunderscore ^1.
\section{Per}
\begin{itemize}
\item {Grp. gram.:prep.}
\end{itemize}
\begin{itemize}
\item {Grp. gram.:Loc. adv.}
\end{itemize}
\begin{itemize}
\item {Proveniência:(Lat. \textunderscore per\textunderscore )}
\end{itemize}
O mesmo que \textunderscore por\textunderscore .
\textunderscore De per si\textunderscore , cada um por sua vez; insuladamente.
\section{Per...}
\begin{itemize}
\item {Grp. gram.:pref.}
\end{itemize}
\begin{itemize}
\item {Proveniência:(Lat. \textunderscore per\textunderscore )}
\end{itemize}
Designativo de \textunderscore intensidade\textunderscore  ou \textunderscore aumento\textunderscore , e da \textunderscore maior qualidade\textunderscore  do elemento electrò-negativo, que póde entrar em uma combinação chímica.
\section{Pêra}
\begin{itemize}
\item {Grp. gram.:f.}
\end{itemize}
\begin{itemize}
\item {Utilização:Prov.}
\end{itemize}
\begin{itemize}
\item {Utilização:alent.}
\end{itemize}
\begin{itemize}
\item {Grp. gram.:Pl.}
\end{itemize}
\begin{itemize}
\item {Proveniência:(Do lat. \textunderscore pira\textunderscore , pl. de \textunderscore pirum\textunderscore )}
\end{itemize}
Fruto da pereira.
Variedade de melão.
Porção da barba, que se deixa crescer no queixo.
\textunderscore Têr para peras\textunderscore , têr imminente grande trabalho ou soffrimento.
\section{Pera}
\begin{itemize}
\item {fónica:pe-ra}
\end{itemize}
\begin{itemize}
\item {Grp. gram.:prep.}
\end{itemize}
\begin{itemize}
\item {Utilização:Ant.}
\end{itemize}
\begin{itemize}
\item {Proveniência:(Do lat. \textunderscore per\textunderscore  + \textunderscore ad\textunderscore )}
\end{itemize}
O mesmo que \textunderscore para\textunderscore .
\section{Pera...}
\begin{itemize}
\item {Grp. gram.:pref.}
\end{itemize}
\begin{itemize}
\item {Utilização:Ant.}
\end{itemize}
O mesmo que \textunderscore para...\textunderscore 
\section{Pêra-água}
\begin{itemize}
\item {Grp. gram.:f.}
\end{itemize}
O mesmo que \textunderscore pêra-de-água\textunderscore .
\section{Peracumbé}
\begin{itemize}
\item {Grp. gram.:m.}
\end{itemize}
\begin{itemize}
\item {Utilização:Des.}
\end{itemize}
Allivio? satisfação?:«\textunderscore ...quisera o peracumbé das suas noticias.\textunderscore »\textunderscore Anat. Joc.\textunderscore , 470.
\section{Perada}
\begin{itemize}
\item {Grp. gram.:f.}
\end{itemize}
\begin{itemize}
\item {Utilização:T. da Índia port}
\end{itemize}
Dôce de pêra.
Vinho de pêras, com que algumas vezes se fabríca em França o conhaque.
Dôce de tejolo.
\section{Pêra-da-aguieira}
\begin{itemize}
\item {Grp. gram.:f.}
\end{itemize}
\begin{itemize}
\item {Proveniência:(De \textunderscore Aguieira\textunderscore , n. p. de uma povoação do concelho de Nellas)}
\end{itemize}
Variedade de pêra muito sumarenta e aromática.
\section{Pêra-de-água}
\begin{itemize}
\item {Grp. gram.:f.}
\end{itemize}
Variedade de pêra, talvez a mesma que \textunderscore dona-joaquina\textunderscore .
\section{Pêra-de-almeida}
\begin{itemize}
\item {Grp. gram.:f.}
\end{itemize}
Variedade de pêra originária de França, onde se chama \textunderscore duchesse d'Angoulême\textunderscore .
\section{Pêra-de-arrátel}
\begin{itemize}
\item {Grp. gram.:f.}
\end{itemize}
Variedade de pêra, também conhecida por \textunderscore gigante\textunderscore  e \textunderscore três-em-prato\textunderscore .
\section{Pêra-de-bom-cristão}
\begin{itemize}
\item {Grp. gram.:f.}
\end{itemize}
Variedade de pêra portuguesa.
\section{Pêra-de-Cristo}
\begin{itemize}
\item {Grp. gram.:f.}
\end{itemize}
Variedade de pêra, talvez a mesma que \textunderscore marquesa\textunderscore .
\section{Pêra-de-engonxo}
\begin{itemize}
\item {Grp. gram.:f.}
\end{itemize}
Variedade de pêra. Cf. P. Moraes, \textunderscore Man. de Agric.\textunderscore 
\section{Pêra-de-ferro}
\begin{itemize}
\item {Grp. gram.:f.}
\end{itemize}
Variedade de pêra beirôa.
\section{Pêra-de-jesus}
\begin{itemize}
\item {Grp. gram.:f.}
\end{itemize}
Variedade de pêra portuguesa.
\section{Pêra-de-pé-curto}
\begin{itemize}
\item {Grp. gram.:f.}
\end{itemize}
Variedade de pêra.
\section{Pêra-de-refêgo}
\begin{itemize}
\item {Grp. gram.:f.}
\end{itemize}
O mesmo que \textunderscore pêra-de-sete-cotovelos\textunderscore .
\section{Pêra-de-rei}
\begin{itemize}
\item {Grp. gram.:f.}
\end{itemize}
Variedade de pêra.
\section{Pêra-de-rosa}
\begin{itemize}
\item {Grp. gram.:f.}
\end{itemize}
Variedade de pêra pardacenta e redonda.
\section{Pêra-de-sete-cotovelos}
\begin{itemize}
\item {Grp. gram.:f.}
\end{itemize}
\begin{itemize}
\item {Utilização:Prov.}
\end{itemize}
\begin{itemize}
\item {Utilização:trasm.}
\end{itemize}
Variedade de pêra inverniça.
\section{Pêra-de-unto}
\begin{itemize}
\item {Grp. gram.:f.}
\end{itemize}
Variedade de pêra pequena e doce.
\section{Perado}
\begin{itemize}
\item {Grp. gram.:m.}
\end{itemize}
Arvoreta madeirense, (\textunderscore ilex perado\textunderscore , Ait.), cuja madeira é empregada em embutidos e em outras pequenas obras de marcenaria. Cf. \textunderscore Bol. da Socied. de Geogr.\textunderscore , XXX, 611.
\section{Pêra-do-conde}
\begin{itemize}
\item {Grp. gram.:f.}
\end{itemize}
Variedade de pêra, muito conhecida no districto de Lisbôa.
\section{Pêra-do-norte}
\begin{itemize}
\item {Grp. gram.:f.}
\end{itemize}
Variedade de pêra.
\section{Pêra-dos-anjos}
\begin{itemize}
\item {Grp. gram.:f.}
\end{itemize}
Variedade de pêra amarelada e um pouco comprida.
\section{Pêra-dos-bairraes}
\begin{itemize}
\item {Grp. gram.:f.}
\end{itemize}
Variedade de pêra sumarenta, que amadurece de Janeiro a Abril. Cf. \textunderscore Diccion. das Peras\textunderscore , 11.
\section{Pêra-dos-freires}
\begin{itemize}
\item {Grp. gram.:f.}
\end{itemize}
Antiga variedade de pêra portuguesa, hoje desconhecida.
\section{Pêra-figo}
\begin{itemize}
\item {Grp. gram.:f.}
\end{itemize}
Variedade de pêra estimada.
\section{Perafita}
\begin{itemize}
\item {Grp. gram.:f.}
\end{itemize}
\begin{itemize}
\item {Utilização:Ant.}
\end{itemize}
Provavelmente, grande pedra ou antigo monumento megalíthico.
(Por \textunderscore pedra-ficta\textunderscore  = pedra fixa)
\section{Pêra-gigante}
\begin{itemize}
\item {Grp. gram.:f.}
\end{itemize}
(V.pêra-de-arrátel)
\section{Peragração}
\begin{itemize}
\item {Grp. gram.:f.}
\end{itemize}
\begin{itemize}
\item {Utilização:Astron.}
\end{itemize}
\begin{itemize}
\item {Proveniência:(Lat. \textunderscore peragratio\textunderscore )}
\end{itemize}
Revolução de um astro em volta de um ponto zodiacal.
\section{Peragratório}
\begin{itemize}
\item {Grp. gram.:adj.}
\end{itemize}
\begin{itemize}
\item {Proveniência:(Do lat. \textunderscore peragratus\textunderscore )}
\end{itemize}
Que serve para percorrer.
\section{Peraíba}
\begin{itemize}
\item {Grp. gram.:f.}
\end{itemize}
\begin{itemize}
\item {Utilização:Bras}
\end{itemize}
Peixe fluvial.
\section{Peral}
\begin{itemize}
\item {Grp. gram.:adj.}
\end{itemize}
\begin{itemize}
\item {Grp. gram.:M.}
\end{itemize}
Relativo ou semelhante a pêra.
Pomar de pereiras.
\section{Pêra-lemos}
\begin{itemize}
\item {Grp. gram.:f.}
\end{itemize}
Pêra muito apreciada, originária do Valle-de-Besteiros.
\section{Peralta}
\begin{itemize}
\item {Grp. gram.:m.}
\end{itemize}
Indivíduo affectado nos modos ou trajes.
Janota; peralvilho.
\section{Peraltear}
\begin{itemize}
\item {Grp. gram.:v. i.}
\end{itemize}
Têr vida de peralta.
\section{Peraltice}
\begin{itemize}
\item {Grp. gram.:f.}
\end{itemize}
Qualidade de peralta.
\section{Peraltismo}
\begin{itemize}
\item {Grp. gram.:m.}
\end{itemize}
O mesmo que \textunderscore peraltice\textunderscore .
Os peraltas.
Modos de peralta. Cf. Arn. Gama, \textunderscore Motim\textunderscore , 134.
\section{Peralto}
\begin{itemize}
\item {Grp. gram.:m.}
\end{itemize}
Árvore santhomense.
\section{Peralvilhada}
\begin{itemize}
\item {Grp. gram.:f.}
\end{itemize}
O mesmo que \textunderscore peralvilhice\textunderscore .
Bando de peralvilhos.
Os peralvilhos. Cf. C. Guerreiro, \textunderscore Diccion. de Consoantes\textunderscore .
\section{Peralvilhar}
\begin{itemize}
\item {Grp. gram.:v. i.}
\end{itemize}
Sêr peralvilho.
\section{Peralvilhice}
\begin{itemize}
\item {Grp. gram.:f.}
\end{itemize}
Acto ou qualidade de peralvilho; peraltice. Cf. Camillo, \textunderscore Cancion. Al.\textunderscore , 254.
\section{Peralvilho}
\begin{itemize}
\item {Grp. gram.:m.}
\end{itemize}
O mesmo que \textunderscore peralta\textunderscore .
\section{Pêra-marmelo}
\begin{itemize}
\item {Grp. gram.:m.}
\end{itemize}
Variedade de pêra ordinária.
\section{Perambeira}
\begin{itemize}
\item {Grp. gram.:f.}
\end{itemize}
\begin{itemize}
\item {Utilização:Bras. de Minas}
\end{itemize}
Precipício, abysmo.
\section{Perambular}
\begin{itemize}
\item {Grp. gram.:v. i.}
\end{itemize}
\begin{itemize}
\item {Utilização:bras}
\end{itemize}
\begin{itemize}
\item {Utilização:Neol.}
\end{itemize}
\begin{itemize}
\item {Proveniência:(Lat. \textunderscore perambulare\textunderscore )}
\end{itemize}
Passear; vaguear.
\section{Peramele-narigudo}
\begin{itemize}
\item {Grp. gram.:m.}
\end{itemize}
Espécie de sarigueira, semelhante ao rato.
\section{Perangária}
\begin{itemize}
\item {Grp. gram.:f.}
\end{itemize}
\begin{itemize}
\item {Utilização:Ant.}
\end{itemize}
Serviço obrigatório de antigos vassalos.
Cp. \textunderscore angária\textunderscore  e \textunderscore angueira\textunderscore .
(B. lat. \textunderscore perangaria\textunderscore . Cf. Du-Cange)
\section{Pêra-noiva}
\begin{itemize}
\item {Grp. gram.:f.}
\end{itemize}
\begin{itemize}
\item {Utilização:T. de Leiria}
\end{itemize}
O mesmo que \textunderscore pêra-pérola\textunderscore .
\section{Perante}
\begin{itemize}
\item {Grp. gram.:prep.}
\end{itemize}
\begin{itemize}
\item {Proveniência:(De \textunderscore per\textunderscore  + \textunderscore ante\textunderscore )}
\end{itemize}
Na presença de; deante de; ante.
\section{Pé-rapado}
\begin{itemize}
\item {Grp. gram.:m.}
\end{itemize}
\begin{itemize}
\item {Utilização:Bras}
\end{itemize}
\begin{itemize}
\item {Utilização:Fam.}
\end{itemize}
Homem de baixa condição.
\section{Pêra-pão}
\begin{itemize}
\item {Grp. gram.:f.}
\end{itemize}
\begin{itemize}
\item {Utilização:Des.}
\end{itemize}
Espécie de pêra desenxabida:«\textunderscore ...mais sem sabor que huma pera-pão.\textunderscore »Camões, \textunderscore Seleuco\textunderscore , pról.
\section{Pêra-parda}
\begin{itemize}
\item {Grp. gram.:f.}
\end{itemize}
Variedade de pêra muito apreciada, e também conhecida por \textunderscore fidalga\textunderscore  e \textunderscore atequipera\textunderscore .
\section{Pêra-parda-de-Lisbôa}
\begin{itemize}
\item {Grp. gram.:f.}
\end{itemize}
Espécie de pêra-parda, muito bôa para cozer.
\section{Pêra-pérola}
\begin{itemize}
\item {Grp. gram.:f.}
\end{itemize}
Variedade de pêra estimada.
\section{Perapétalo}
\begin{itemize}
\item {Grp. gram.:adj.}
\end{itemize}
\begin{itemize}
\item {Utilização:Bot.}
\end{itemize}
Diz-se de quaesquer appêndices, situados sôbre o cálice.
\section{Pêra-prata}
\begin{itemize}
\item {Grp. gram.:f.}
\end{itemize}
O mesmo que \textunderscore pêra-água\textunderscore .
\section{Pêra-rosa}
\begin{itemize}
\item {Grp. gram.:f.}
\end{itemize}
\begin{itemize}
\item {Utilização:T. das Caldas da Raínha}
\end{itemize}
Espécie de pêra encarnada, muito saborosa.
\section{Perau}
\begin{itemize}
\item {Grp. gram.:m.}
\end{itemize}
Linha inferior da margem, onde começa o leito do rio, e que a maré cobre e descobre.
Pégo.
Poço fundo.
Cova na areia, formada debaixo de água pela arrebentação das ondas.
(Relaciona-se com \textunderscore prão\textunderscore ?)
\section{Perauta}
\begin{itemize}
\item {Grp. gram.:m.}
\end{itemize}
\begin{itemize}
\item {Utilização:Prov.}
\end{itemize}
\begin{itemize}
\item {Utilização:trasm.}
\end{itemize}
O mesmo que \textunderscore peralta\textunderscore .
\section{Perava}
\begin{itemize}
\item {Grp. gram.:f.}
\end{itemize}
\begin{itemize}
\item {Utilização:Ant.}
\end{itemize}
O mesmo que \textunderscore parávoa\textunderscore .
\section{Perca}
\begin{itemize}
\item {Grp. gram.:f.}
\end{itemize}
\begin{itemize}
\item {Proveniência:(Lat. \textunderscore perca\textunderscore )}
\end{itemize}
Peixe, da fam. dos pércidas.
\section{Perca}
\begin{itemize}
\item {Grp. gram.:f.}
\end{itemize}
\begin{itemize}
\item {Utilização:Pop.}
\end{itemize}
O mesmo que \textunderscore perda\textunderscore .
Prejuizo, damno.
(Da 1.^a pessoa do pres. do indic. de \textunderscore perder\textunderscore )
\section{Percalçar}
\begin{itemize}
\item {Grp. gram.:v. t.}
\end{itemize}
\begin{itemize}
\item {Utilização:Des.}
\end{itemize}
\begin{itemize}
\item {Proveniência:(De \textunderscore percalço\textunderscore )}
\end{itemize}
Alcançar, lucrar.
\section{Percalço}
\begin{itemize}
\item {Grp. gram.:m.}
\end{itemize}
\begin{itemize}
\item {Utilização:Fam.}
\end{itemize}
\begin{itemize}
\item {Proveniência:(De \textunderscore per\textunderscore  + \textunderscore calço\textunderscore )}
\end{itemize}
Lucro.
Vantagem casual.
Proventos.
Transtôrno, incômmodo, inherente a uma profissão, estado, etc.
\section{Percale}
\begin{itemize}
\item {Grp. gram.:m.}
\end{itemize}
\begin{itemize}
\item {Proveniência:(Fr. \textunderscore percale\textunderscore )}
\end{itemize}
Tecido fino de algodão, sem o pêlo que há nos tecidos desta natureza.
\section{Percalina}
\begin{itemize}
\item {Grp. gram.:f.}
\end{itemize}
Tecido forte de algodão, sem pêlo, usado principalmente em encadernações.
(Cp. cast. \textunderscore perealina\textunderscore )
\section{Percâmbrico}
\begin{itemize}
\item {Grp. gram.:adj.}
\end{itemize}
\begin{itemize}
\item {Utilização:Geol.}
\end{itemize}
Diz-se de um dos terrenos paleozoicos.
\section{Percar}
\begin{itemize}
\item {Grp. gram.:v. t.}
\end{itemize}
\begin{itemize}
\item {Utilização:Des.}
\end{itemize}
Causar perca a.
Damnificar:«\textunderscore ...com flexas encravar os nossos e percallos com amiudada multidão de peloiros\textunderscore ». Filinto, \textunderscore D. Man.\textunderscore , I, 278.
\section{Percário}
\begin{itemize}
\item {Grp. gram.:adj.}
\end{itemize}
\begin{itemize}
\item {Proveniência:(De \textunderscore perca\textunderscore )}
\end{itemize}
Incerto; arriscado.
Que póde causar damno; precário.--T., criado por Th. Braga, \textunderscore Camões\textunderscore , 22, por o julgar mais expressivo que \textunderscore precário\textunderscore .
\section{Perceba}
\begin{itemize}
\item {Grp. gram.:f.}
\end{itemize}
O mesmo que \textunderscore percebe\textunderscore .
\section{Percebe}
\begin{itemize}
\item {Grp. gram.:m.}
\end{itemize}
\begin{itemize}
\item {Utilização:Pesc.}
\end{itemize}
Marisco, o mesmo que \textunderscore perceve\textunderscore .
\section{Perceber}
\begin{itemize}
\item {Grp. gram.:v. t.}
\end{itemize}
\begin{itemize}
\item {Proveniência:(Do lat. \textunderscore percipere\textunderscore )}
\end{itemize}
Adquirir, por meio dos sentidos, conhecimento de.
Conhecer; formar ideia de; comprehender.
Ver bem.
Ver ao longe; enxergar.
Receber.
\section{Percebimento}
\begin{itemize}
\item {Grp. gram.:m.}
\end{itemize}
Acto de perceber.
\section{Percebível}
\begin{itemize}
\item {Grp. gram.:adj.}
\end{itemize}
\begin{itemize}
\item {Utilização:P. us.}
\end{itemize}
\begin{itemize}
\item {Proveniência:(De \textunderscore perceber\textunderscore )}
\end{itemize}
Que se póde perceber.
Perceptível.
\section{Percentagem}
\begin{itemize}
\item {Grp. gram.:f.}
\end{itemize}
\begin{itemize}
\item {Proveniência:(De \textunderscore per\textunderscore  + \textunderscore cento\textunderscore )}
\end{itemize}
Quantia paga, ou recebida na proporção de um tanto por cento.
Proporção.
Prestação, proporcionada a certa quantia ou a certos lucros.
\section{Percepção}
\begin{itemize}
\item {Grp. gram.:f.}
\end{itemize}
\begin{itemize}
\item {Proveniência:(Lat. \textunderscore perceptio\textunderscore )}
\end{itemize}
Acto ou effeito de perceber.
\section{Perceptibilidade}
\begin{itemize}
\item {Grp. gram.:f.}
\end{itemize}
\begin{itemize}
\item {Proveniência:(Do lat. \textunderscore perceptibilis\textunderscore )}
\end{itemize}
Faculdade de perceber.
Qualidade do que é perceptível.
\section{Perceptível}
\begin{itemize}
\item {Grp. gram.:adj.}
\end{itemize}
\begin{itemize}
\item {Proveniência:(Lat. \textunderscore perceptibilis\textunderscore )}
\end{itemize}
Que póde sêr percebido.
\section{Perceptivelmente}
\begin{itemize}
\item {Grp. gram.:adv.}
\end{itemize}
De modo perceptível.
\section{Perceptivo}
\begin{itemize}
\item {Grp. gram.:adj.}
\end{itemize}
\begin{itemize}
\item {Proveniência:(Do lat. \textunderscore perceptus\textunderscore )}
\end{itemize}
Que tem a faculdade de perceber.
\section{Perceve}
\begin{itemize}
\item {Grp. gram.:m.}
\end{itemize}
Marisco de água salgada, que Filinto, VIII, 127, descreve assim:«\textunderscore Marisco aferrado ás rochas do mar, longo como um dedo mínimo, corpo verde, cabeça vermelha, muito saboroso\textunderscore ».
\section{Percevejada}
\begin{itemize}
\item {Grp. gram.:f.}
\end{itemize}
\begin{itemize}
\item {Proveniência:(De \textunderscore percevejo\textunderscore ^1)}
\end{itemize}
Porção de percevejos.
\section{Percevejar}
\begin{itemize}
\item {Grp. gram.:v. t.  e  i.}
\end{itemize}
\begin{itemize}
\item {Utilização:Burl.}
\end{itemize}
O mesmo que \textunderscore perceber\textunderscore :«\textunderscore agora percevejou\textunderscore ».
Castilho, \textunderscore Med. á Fôrça\textunderscore , 66.«\textunderscore E é que não percevejo\textunderscore ». \textunderscore Ib.\textunderscore , \textunderscore id.\textunderscore , 155.
\section{Percevejo}
\begin{itemize}
\item {Grp. gram.:m.}
\end{itemize}
\begin{itemize}
\item {Utilização:Prov.}
\end{itemize}
\begin{itemize}
\item {Proveniência:(De \textunderscore perceve\textunderscore , provavelmente)}
\end{itemize}
Insecto hemíptero parasito, (\textunderscore cimex\textunderscore ).
Nome de alguns insectos parasitos, como o chamado \textunderscore percevejo das couves\textunderscore .
\section{Percevejo}
\begin{itemize}
\item {Grp. gram.:m.}
\end{itemize}
(Traducção arbitrária do fr. \textunderscore punaise\textunderscore , para designar uma \textunderscore tacha\textunderscore , pequeno prego de cabeça chata e uma só perna, com que se fixa sôbre uma superfície lisa um papel em que se quere desenhar)
\section{Percevelho}
\begin{itemize}
\item {fónica:vê}
\end{itemize}
\begin{itemize}
\item {Grp. gram.:m.}
\end{itemize}
\begin{itemize}
\item {Utilização:Prov.}
\end{itemize}
O mesmo que \textunderscore percevejo\textunderscore ^1.
\section{Percha}
\begin{itemize}
\item {Grp. gram.:f.}
\end{itemize}
\begin{itemize}
\item {Utilização:Ant.}
\end{itemize}
\begin{itemize}
\item {Proveniência:(Lat. \textunderscore pertica\textunderscore )}
\end{itemize}
Vara comprida de madeira.
Pau.
Cada uma das molduras que ornam a prôa do navio.
Pilar delgado e redondo, curvado no alto para formar arco e nervura de ogiva.
Máquina, composta de um ou mais tambores guarnecidos de corda, para puxar e tornar parallelo o pêlo dos estofos, depois de apisoados.
\section{Perchamento}
\begin{itemize}
\item {Grp. gram.:m.}
\end{itemize}
Acto de perchar.
\section{Perchar}
\begin{itemize}
\item {Grp. gram.:v. t.}
\end{itemize}
Sujeitar (um pano) á percha (máquina). Cf. \textunderscore Inquér. Industr.\textunderscore , II p., l. III, 105.
\section{Percherão}
\begin{itemize}
\item {Grp. gram.:m.}
\end{itemize}
\begin{itemize}
\item {Proveniência:(Fr. \textunderscore percheron\textunderscore )}
\end{itemize}
Espécie de cavallo francês, apreciado em parelhas de tiro ligeiro. Cf. Ortigão, \textunderscore Holanda\textunderscore , 67.
\section{Perchlorado}
\begin{itemize}
\item {Grp. gram.:adj.}
\end{itemize}
\begin{itemize}
\item {Utilização:Chím.}
\end{itemize}
\begin{itemize}
\item {Proveniência:(De \textunderscore per...\textunderscore  + \textunderscore chloro\textunderscore )}
\end{itemize}
Que contém a maior quantidade possível de chloro.
\section{Perchlorato}
\begin{itemize}
\item {Grp. gram.:m.}
\end{itemize}
\begin{itemize}
\item {Utilização:Chím.}
\end{itemize}
Designação genérica dos saes de ácido perchlórico.
\section{Perchloreto}
\begin{itemize}
\item {fónica:clorê}
\end{itemize}
\begin{itemize}
\item {Grp. gram.:m.}
\end{itemize}
\begin{itemize}
\item {Utilização:Chím.}
\end{itemize}
\begin{itemize}
\item {Proveniência:(De \textunderscore per...\textunderscore  + \textunderscore chloreto\textunderscore )}
\end{itemize}
Designação genérica dos chloretos que offerecem o maior número de equivalentes de chloro por cada equivalente do corpo simples.
\section{Perchlórico}
\begin{itemize}
\item {Grp. gram.:adj.}
\end{itemize}
\begin{itemize}
\item {Utilização:Chím.}
\end{itemize}
Diz-se do ácido, que contém a maior proporção de oxygênio.
\section{Pércidas}
\begin{itemize}
\item {Grp. gram.:m. pl.}
\end{itemize}
\begin{itemize}
\item {Utilização:Zool.}
\end{itemize}
\begin{itemize}
\item {Proveniência:(De \textunderscore perca\textunderscore ^1 + gr. \textunderscore eidos\textunderscore )}
\end{itemize}
Família de peixes, que tem por typo a perca.
\section{Percinta}
\begin{itemize}
\item {Grp. gram.:f.}
\end{itemize}
\begin{itemize}
\item {Utilização:Náut.}
\end{itemize}
\begin{itemize}
\item {Proveniência:(Lat. \textunderscore percincta\textunderscore )}
\end{itemize}
Tira de lona ou de brim alcatroado, com que se forram cabos engaiados.
\section{Percintar}
\begin{itemize}
\item {Grp. gram.:v.}
\end{itemize}
\begin{itemize}
\item {Utilização:t. Náut.}
\end{itemize}
\begin{itemize}
\item {Utilização:Ext.}
\end{itemize}
\begin{itemize}
\item {Proveniência:(De \textunderscore percinta\textunderscore )}
\end{itemize}
Enrolar em espiral tiras de lona ou brim alcatroado em (um cabo engaiado)
Cingir por todos os lados.
\section{Perclorado}
\begin{itemize}
\item {Grp. gram.:adj.}
\end{itemize}
\begin{itemize}
\item {Utilização:Chím.}
\end{itemize}
\begin{itemize}
\item {Proveniência:(De \textunderscore per...\textunderscore  + \textunderscore cloro\textunderscore )}
\end{itemize}
Que contém a maior quantidade possível de cloro.
\section{Perclorato}
\begin{itemize}
\item {Grp. gram.:m.}
\end{itemize}
\begin{itemize}
\item {Utilização:Chím.}
\end{itemize}
Designação genérica dos saes de ácido perclórico.
\section{Percloreto}
\begin{itemize}
\item {fónica:clorê}
\end{itemize}
\begin{itemize}
\item {Grp. gram.:m.}
\end{itemize}
\begin{itemize}
\item {Utilização:Chím.}
\end{itemize}
\begin{itemize}
\item {Proveniência:(De \textunderscore per...\textunderscore  + \textunderscore cloreto\textunderscore )}
\end{itemize}
Designação genérica dos cloretos que oferecem o maior número de equivalentes de cloro por cada equivalente do corpo simples.
\section{Perclórico}
\begin{itemize}
\item {Grp. gram.:adj.}
\end{itemize}
\begin{itemize}
\item {Utilização:Chím.}
\end{itemize}
Diz-se do ácido, que contém a maior proporção de oxigênio.
\section{Percluso}
\begin{itemize}
\item {Grp. gram.:adj.}
\end{itemize}
\begin{itemize}
\item {Utilização:Med.}
\end{itemize}
\begin{itemize}
\item {Proveniência:(Lat. \textunderscore perclusus\textunderscore )}
\end{itemize}
Impossibilitado de exercer as funcções de locomoção.
\section{Percóides}
\begin{itemize}
\item {Grp. gram.:m. pl.}
\end{itemize}
O mesmo que \textunderscore pércidas\textunderscore .
\section{Perçolana}
\begin{itemize}
\item {Grp. gram.:f.}
\end{itemize}
Fórma ant. de \textunderscore porcelana\textunderscore . Cf. \textunderscore Peregrinação\textunderscore , LV.
\section{Percorrer}
\begin{itemize}
\item {Grp. gram.:v. t.}
\end{itemize}
\begin{itemize}
\item {Proveniência:(Do lat. \textunderscore percurrere\textunderscore )}
\end{itemize}
Correr por.
Explorar, esquadrinhar.
Observar ligeira e successivamente.
Passar através de.
\section{Percósia}
\begin{itemize}
\item {Grp. gram.:f.}
\end{itemize}
Gênero de insectos coleópteros pentâmeros.
\section{Percuciente}
\begin{itemize}
\item {Grp. gram.:adj.}
\end{itemize}
\begin{itemize}
\item {Proveniência:(Lat. \textunderscore percutiens\textunderscore )}
\end{itemize}
Que percute.
\section{Percudir}
\begin{itemize}
\item {Grp. gram.:v. t.}
\end{itemize}
\begin{itemize}
\item {Utilização:Ant.}
\end{itemize}
O mesmo que \textunderscore percutir\textunderscore .
\section{Perculso}
\begin{itemize}
\item {Grp. gram.:adj.}
\end{itemize}
\begin{itemize}
\item {Utilização:Poét.}
\end{itemize}
\begin{itemize}
\item {Proveniência:(Lat. \textunderscore perculsus\textunderscore )}
\end{itemize}
Abalado violentamente. Cf. Castilho, \textunderscore Fastos\textunderscore , III, 177.
\section{Percurso}
\begin{itemize}
\item {Grp. gram.:m.}
\end{itemize}
\begin{itemize}
\item {Proveniência:(Lat. \textunderscore percursus\textunderscore )}
\end{itemize}
Acto ou effeito de percorrer.
Espaço percorrido.
Movimento.
\section{Percussão}
\begin{itemize}
\item {Grp. gram.:f.}
\end{itemize}
\begin{itemize}
\item {Proveniência:(Lat. \textunderscore percussio\textunderscore )}
\end{itemize}
Acto ou effeito de percutir; embate ou choque de dois corpos.
\section{Percussor}
\begin{itemize}
\item {Grp. gram.:adj.}
\end{itemize}
\begin{itemize}
\item {Grp. gram.:M.}
\end{itemize}
\begin{itemize}
\item {Proveniência:(Lat. \textunderscore percussor\textunderscore )}
\end{itemize}
Que percute.
Aquillo que percute.
Peça metállica, em fórma de agulha, que percute uma cápsula fulminante, para transmittir fôgo á polvora.
\section{Percutâneo}
\begin{itemize}
\item {Grp. gram.:adj.}
\end{itemize}
\begin{itemize}
\item {Utilização:Med.}
\end{itemize}
Diz-se do medicamento, que se applica sôbre a pelle, sem friccionar e sem levantar a epiderme.
\section{Percutidor}
\begin{itemize}
\item {Grp. gram.:m.  e  adj.}
\end{itemize}
\begin{itemize}
\item {Proveniência:(De \textunderscore percutir\textunderscore )}
\end{itemize}
O mesmo que \textunderscore percussor\textunderscore .
\section{Percutir}
\begin{itemize}
\item {Grp. gram.:v. t.}
\end{itemize}
\begin{itemize}
\item {Proveniência:(Lat. \textunderscore percutere\textunderscore )}
\end{itemize}
Bater.
Esbarrar contra.
\section{Percutor}
\begin{itemize}
\item {Grp. gram.:m.}
\end{itemize}
\begin{itemize}
\item {Proveniência:(De \textunderscore percutir\textunderscore )}
\end{itemize}
Peça que, em certos maquinismos, serve para percutir; o mesmo que \textunderscore percussor\textunderscore , fórma preferível.
\section{Perda}
\begin{itemize}
\item {fónica:pêr}
\end{itemize}
\begin{itemize}
\item {Grp. gram.:f.}
\end{itemize}
\begin{itemize}
\item {Utilização:Ext.}
\end{itemize}
Acto de perder.
Desapparecimento.
Extravio.
Desgraça.
Destruição.
(Contr. de \textunderscore pérdida\textunderscore ? Cp. cast. \textunderscore pérdida\textunderscore , lat. \textunderscore perdita\textunderscore )
\section{Perdante}
\begin{itemize}
\item {Grp. gram.:prep.}
\end{itemize}
\begin{itemize}
\item {Utilização:Ant.}
\end{itemize}
Ante; perante; diante de.
(Contr. de \textunderscore per\textunderscore  + \textunderscore de\textunderscore  + \textunderscore ante\textunderscore )
\section{Perdão}
\begin{itemize}
\item {Grp. gram.:m.}
\end{itemize}
\begin{itemize}
\item {Proveniência:(De \textunderscore perdoar\textunderscore )}
\end{itemize}
Remissão de pena.
Indulto; desculpa.
\section{Perdedor}
\begin{itemize}
\item {Grp. gram.:m.  e  adj.}
\end{itemize}
Aquelle que perde. Cf. \textunderscore Viriato Trág.\textunderscore , XIV, 30.
\section{Perdente}
\begin{itemize}
\item {Grp. gram.:adj.}
\end{itemize}
\begin{itemize}
\item {Utilização:P. us.}
\end{itemize}
Que perde. Cf. Alb. Pimentel, \textunderscore Chiado\textunderscore , 32.
\section{Perder}
\begin{itemize}
\item {Grp. gram.:v. t.}
\end{itemize}
\begin{itemize}
\item {Grp. gram.:V. i.}
\end{itemize}
\begin{itemize}
\item {Grp. gram.:V. p.}
\end{itemize}
\begin{itemize}
\item {Utilização:Fam.}
\end{itemize}
\begin{itemize}
\item {Proveniência:(Lat. \textunderscore perdere\textunderscore )}
\end{itemize}
Sêr privado de (alguma coisa que se possuía).
Não obter (uma vantagem), que se teria podido conseguir.
Sêr privado de; deixar de têr: \textunderscore perder a saúde\textunderscore .
Deixar de occupar.
Deixar fugir.
Deixar tomar: \textunderscore perder uma fortaleza\textunderscore .
Destruír.
Causar a ruína de.
Corromper: \textunderscore perdem-no as más companhias\textunderscore .
Causar a condemnação de.
Não sentir.
Esquecer-se de.
Não tomar sentido em.
Soffrer quebra em.
Têr prejuízo pecuniário.
Não conseguir um lucro esperado ou desejado.
Passar a peor estado ou condição.
Cessar de fruír certas vantagens ou interesses.
Deminuír de valor ou de conceito.
Ficar vencido ao jôgo.
Desapparecer.
Extraviar-se.
Sumir-se no mar, naufragar.
Arruinar-se.
Ficar desbaratado ou vencido.
Inutilizar-se, frustrar-se.
Sêr expungido ou obliterado.
Confundir-se, desordenar-se.
Desmoralizar-se, perverter-se.
Desgraçar-se.
Tornar-se absorto, concentrado.
Gostar muito de alguém ou de alguma coisa.
\section{Perdição}
\begin{itemize}
\item {Grp. gram.:f.}
\end{itemize}
\begin{itemize}
\item {Proveniência:(Lat. \textunderscore perditio\textunderscore )}
\end{itemize}
Acto ou effeito de perder.
Desgraça.
Deshonra.
Immoralidade.
Irreligiosidade.
\section{Perdíceo}
\begin{itemize}
\item {Grp. gram.:adj.}
\end{itemize}
\begin{itemize}
\item {Proveniência:(Do lat. \textunderscore perdix\textunderscore , \textunderscore perdicis\textunderscore )}
\end{itemize}
Relativo ou semelhante á perdiz.
\section{Perdícias}
\begin{itemize}
\item {Grp. gram.:f. pl.}
\end{itemize}
Família de aves, admitida por alguns naturalistas, e que tem por tipo a perdiz.
(Fem. pl. de \textunderscore perdício\textunderscore )
\section{Perdida}
\begin{itemize}
\item {Grp. gram.:f.}
\end{itemize}
(V.perda)
Mulhér perdida ou prostituída.
\section{Perdidamente}
\begin{itemize}
\item {Grp. gram.:adv.}
\end{itemize}
\begin{itemize}
\item {Proveniência:(De \textunderscore perdido\textunderscore )}
\end{itemize}
Exaggeradamente: \textunderscore rir perdidamente\textunderscore .
Com perda.
Loucamente.
Desvantajosamente.
\section{Perdidas}
\begin{itemize}
\item {fónica:ás}
\end{itemize}
\begin{itemize}
\item {Grp. gram.:loc. adv.}
\end{itemize}
Perdidamente, loucamente:«\textunderscore ...pois que ás perdidas me não amaste.\textunderscore »Filinto, XIX, 220.
\section{Perdidiço}
\begin{itemize}
\item {Grp. gram.:adj.}
\end{itemize}
Que facilmente se perde.
\section{Perdido}
\begin{itemize}
\item {Grp. gram.:adj.}
\end{itemize}
\begin{itemize}
\item {Grp. gram.:M.}
\end{itemize}
\begin{itemize}
\item {Utilização:Bras}
\end{itemize}
Disperso.
Naufragado.
Extraviado.
Que gosta muito: \textunderscore é perdido por morángos\textunderscore .
Muito apaixonado.
Immoral.
Devasso.
Que está em perigo de morte: \textunderscore o doente está perdido\textunderscore .
\textunderscore Perdido de riso\textunderscore , que não póde conter o riso; que ri ás gargalhadas.
Qualquer coisa perdida.
Espécie de mandioca.
\section{Perdidor}
\begin{itemize}
\item {Grp. gram.:m.  e  adj.}
\end{itemize}
O mesmo que \textunderscore perdedor\textunderscore :«\textunderscore ...do perdidor de sua filha.\textunderscore »Camillo, \textunderscore Caveira\textunderscore , 50.
\section{Perdidoso}
\begin{itemize}
\item {Grp. gram.:adj.}
\end{itemize}
\begin{itemize}
\item {Utilização:Ant.}
\end{itemize}
\begin{itemize}
\item {Proveniência:(De \textunderscore perdido\textunderscore )}
\end{itemize}
Que soffreu perda.
Prejudicial.
\section{Perdigão}
\begin{itemize}
\item {Grp. gram.:m.}
\end{itemize}
Macho da perdiz.
(Cp. cast. \textunderscore perdigón\textunderscore )
\section{Perdigotar}
\begin{itemize}
\item {Grp. gram.:v. i.}
\end{itemize}
\begin{itemize}
\item {Utilização:Prov.}
\end{itemize}
\begin{itemize}
\item {Utilização:beir.}
\end{itemize}
Deitar salpicos de saliva ou perdigotos, quando fala.
\section{Perdigoto}
\begin{itemize}
\item {fónica:gô}
\end{itemize}
\begin{itemize}
\item {Grp. gram.:m.}
\end{itemize}
\begin{itemize}
\item {Utilização:Pop.}
\end{itemize}
Pequena perdiz.
Salpico de saliva.
\section{Perdigueira}
\begin{itemize}
\item {Grp. gram.:f.}
\end{itemize}
\begin{itemize}
\item {Proveniência:(De \textunderscore perdigueiro\textunderscore )}
\end{itemize}
Cadella, que é bôa ou própria para a caça das perdizes.
\section{Perdigueiro}
\begin{itemize}
\item {Grp. gram.:m.}
\end{itemize}
\begin{itemize}
\item {Grp. gram.:Adj.}
\end{itemize}
\begin{itemize}
\item {Proveniência:(Do lat. hyp. \textunderscore perdicarius\textunderscore , de \textunderscore perdix\textunderscore )}
\end{itemize}
Cão, próprio para caça de perdizes.
Que caça perdizes.
\section{Perdimento}
\begin{itemize}
\item {Grp. gram.:m.}
\end{itemize}
O mesmo que \textunderscore perdição\textunderscore .
\section{Perdita}
\begin{itemize}
\item {Grp. gram.:adj.}
\end{itemize}
\begin{itemize}
\item {Utilização:Ant.}
\end{itemize}
\begin{itemize}
\item {Proveniência:(De \textunderscore per\textunderscore  + \textunderscore dita\textunderscore )}
\end{itemize}
O mesmo que \textunderscore porventura\textunderscore .
\section{Perdível}
\begin{itemize}
\item {Grp. gram.:adj.}
\end{itemize}
Que se póde perder.
Que é de lucro incerto.
\section{Perdiz}
\begin{itemize}
\item {Grp. gram.:f.}
\end{itemize}
\begin{itemize}
\item {Utilização:Chul.}
\end{itemize}
\begin{itemize}
\item {Proveniência:(Lat. \textunderscore perdix\textunderscore )}
\end{itemize}
Gênero de aves gallináceas, cuja espécie mais conhecida entre nós tem os tarsos, o bico e os olhos avermelhados.
Perda, prejuízo.
\section{Perdizíneas}
\begin{itemize}
\item {Grp. gram.:f. pl.}
\end{itemize}
\begin{itemize}
\item {Proveniência:(De \textunderscore perdiz\textunderscore )}
\end{itemize}
O mesmo que \textunderscore perdícias\textunderscore .
\section{Perdizita}
\begin{itemize}
\item {Grp. gram.:f.}
\end{itemize}
\begin{itemize}
\item {Utilização:Miner.}
\end{itemize}
\begin{itemize}
\item {Proveniência:(De \textunderscore perdiz\textunderscore )}
\end{itemize}
Variedade de pedra, que tem a côr da penna de perdiz.
\section{Perdizite}
\begin{itemize}
\item {Grp. gram.:f.}
\end{itemize}
\begin{itemize}
\item {Utilização:Miner.}
\end{itemize}
\begin{itemize}
\item {Proveniência:(De \textunderscore perdiz\textunderscore )}
\end{itemize}
Variedade de pedra, que tem a côr da penna de perdiz.
\section{Perdoador}
\begin{itemize}
\item {Grp. gram.:m.  e  adj.}
\end{itemize}
Aquelle que perdôa facilmente.
\section{Perdoamento}
\begin{itemize}
\item {Grp. gram.:m.}
\end{itemize}
\begin{itemize}
\item {Utilização:Ant.}
\end{itemize}
Acto de perdoar; perdão. Cf. \textunderscore Port. Mon. Script.\textunderscore , 286.
\section{Perdoança}
\begin{itemize}
\item {Grp. gram.:f.}
\end{itemize}
\begin{itemize}
\item {Utilização:Prov.}
\end{itemize}
\begin{itemize}
\item {Utilização:minh.}
\end{itemize}
\begin{itemize}
\item {Utilização:Ant.}
\end{itemize}
\begin{itemize}
\item {Proveniência:(De \textunderscore perdoar\textunderscore )}
\end{itemize}
Perdão; dispensa.--No Minho, o lavrador pede ao senhorio \textunderscore perdoança\textunderscore  de alguns alqueires no pagamento da renda, quando a colheita foi reduzida por alguma calamidade.
\section{Perdoar}
\begin{itemize}
\item {Grp. gram.:v. t.}
\end{itemize}
\begin{itemize}
\item {Proveniência:(Do b. lat. \textunderscore perdonare\textunderscore )}
\end{itemize}
Conceder perdão a.
Absolver de (culpa, dívida, etc.).
Desculpar.
Poupar.
\section{Perdoável}
\begin{itemize}
\item {Grp. gram.:adj.}
\end{itemize}
Que se póde perdoar.
Digno de perdão.
\section{Perdudo}
\begin{itemize}
\item {Grp. gram.:adj.}
\end{itemize}
\begin{itemize}
\item {Utilização:Ant.}
\end{itemize}
O mesmo que \textunderscore perdido\textunderscore ; dissipado, gasto.
\section{Perdulariamente}
\begin{itemize}
\item {Grp. gram.:adv.}
\end{itemize}
De modo perdulário.
\section{Perdulário}
\begin{itemize}
\item {Grp. gram.:adj.}
\end{itemize}
\begin{itemize}
\item {Proveniência:(Do rad. de \textunderscore perder\textunderscore )}
\end{itemize}
Que gasta excessivamente; dissipador; extravagante.
\section{Perduração}
\begin{itemize}
\item {Grp. gram.:f.}
\end{itemize}
\begin{itemize}
\item {Proveniência:(Lat. \textunderscore perduratio\textunderscore )}
\end{itemize}
Grande duração; acto de perdurar.
\section{Perdurar}
\begin{itemize}
\item {Grp. gram.:v. i.}
\end{itemize}
\begin{itemize}
\item {Proveniência:(Lat. \textunderscore perdurare\textunderscore )}
\end{itemize}
Durar muito. Cf. Camillo, \textunderscore Noites de Insómn.\textunderscore , IV, 81.
\section{Perdurável}
\begin{itemize}
\item {Grp. gram.:adj.}
\end{itemize}
\begin{itemize}
\item {Proveniência:(Do lat. \textunderscore perdurabilis\textunderscore )}
\end{itemize}
Que póde durar muito; duradoiro; eterno.
\section{Perduravelmente}
\begin{itemize}
\item {Grp. gram.:adv.}
\end{itemize}
De modo perdurável.
\section{Pereba}
\begin{itemize}
\item {Grp. gram.:f.}
\end{itemize}
\begin{itemize}
\item {Utilização:Bras. do N}
\end{itemize}
\begin{itemize}
\item {Proveniência:(Do guar. \textunderscore peréb\textunderscore )}
\end{itemize}
Sarna.
Erupção herpética.
Pequena ferida.
\section{Perebento}
\begin{itemize}
\item {Grp. gram.:adj.}
\end{itemize}
\begin{itemize}
\item {Utilização:Bras. do N}
\end{itemize}
Que tem pereba.
\section{Perecedoiro}
\begin{itemize}
\item {Grp. gram.:adj.}
\end{itemize}
O mesmo que \textunderscore perecedor\textunderscore .
\section{Perecedor}
\begin{itemize}
\item {Grp. gram.:adj.}
\end{itemize}
Que há de perecer; que há de findar; mortal.
\section{Perecedouro}
\begin{itemize}
\item {Grp. gram.:adj.}
\end{itemize}
O mesmo que \textunderscore perecedor\textunderscore .
\section{Perecer}
\begin{itemize}
\item {Grp. gram.:v. i.}
\end{itemize}
\begin{itemize}
\item {Proveniência:(Do lat. hypoth. \textunderscore perescere\textunderscore )}
\end{itemize}
Acabar; findar; deixar de existir, morrer.
Sêr assolado.
\section{Perecimento}
\begin{itemize}
\item {Grp. gram.:m.}
\end{itemize}
Acto de perecer.
Extincção.
Esgotamento; definhamento.
\section{Perecível}
\begin{itemize}
\item {Grp. gram.:adj.}
\end{itemize}
O mesmo que \textunderscore perecedoiro\textunderscore .
Que é susceptível de perecer; que póde extinguir-se.
\section{Peregrim}
\begin{itemize}
\item {Grp. gram.:m.}
\end{itemize}
\begin{itemize}
\item {Utilização:Ant.}
\end{itemize}
O mesmo que \textunderscore peregrino\textunderscore .
\section{Peregrinação}
\begin{itemize}
\item {Grp. gram.:f.}
\end{itemize}
\begin{itemize}
\item {Proveniência:(Lat. \textunderscore peregrinatio\textunderscore )}
\end{itemize}
Acto de peregrinar.
\section{Peregrinador}
\begin{itemize}
\item {Grp. gram.:m.  e  adj.}
\end{itemize}
\begin{itemize}
\item {Proveniência:(Lat. \textunderscore peregrinator\textunderscore )}
\end{itemize}
O mesmo que \textunderscore peregrino\textunderscore .
\section{Peregrinamente}
\begin{itemize}
\item {Grp. gram.:adv.}
\end{itemize}
De modo peregrino.
Extraordinariamente; admiravelmente.
\section{Peregrinante}
\begin{itemize}
\item {Grp. gram.:m. ,  f.  e  adj.}
\end{itemize}
\begin{itemize}
\item {Proveniência:(Lat. \textunderscore peregrinans\textunderscore )}
\end{itemize}
Pessôa que peregrina.
\section{Peregrinar}
\begin{itemize}
\item {Grp. gram.:v. t.}
\end{itemize}
\begin{itemize}
\item {Grp. gram.:V. i.}
\end{itemize}
\begin{itemize}
\item {Proveniência:(Lat. \textunderscore peregrinari\textunderscore )}
\end{itemize}
Andar em viagem por (terras distantes).
Percorrer, viajando:«\textunderscore ...têr peregrinado terras de herejes\textunderscore ». Camillo, \textunderscore Caveira\textunderscore , 15.
Viajar por terras distantes.
Fazer viagem ou romaria a lugares santos.
Divagar.
Viajar.
\section{Peregrinismo}
\begin{itemize}
\item {Grp. gram.:m.}
\end{itemize}
\begin{itemize}
\item {Utilização:Bras}
\end{itemize}
\begin{itemize}
\item {Proveniência:(De \textunderscore peregrino\textunderscore )}
\end{itemize}
Emprêgo de palavra ou phrase estranha ao idioma vernáculo; estrangeirismo. Cf. Franc. Alex. Lobo, II, 26.
Raridade; excellência.
\section{Peregrino}
\begin{itemize}
\item {Grp. gram.:m.  e  adj.}
\end{itemize}
\begin{itemize}
\item {Proveniência:(Lat. \textunderscore peregrinus\textunderscore )}
\end{itemize}
Aquelle que peregrina.
Estranho, estrangeiro.
Excellente; que tem bondade ou belleza rara.
\section{Pereiorá}
\begin{itemize}
\item {Grp. gram.:m.}
\end{itemize}
Árvore laurácea do Brasil.
\section{Pereira}
\begin{itemize}
\item {Grp. gram.:f.}
\end{itemize}
\begin{itemize}
\item {Proveniência:(Do b. lat. \textunderscore peraria\textunderscore )}
\end{itemize}
Nome de variadas espécies de árvores pomáceas, (\textunderscore pirus communis\textunderscore ).
\section{Pereira-dos-diabos}
\begin{itemize}
\item {Grp. gram.:f.}
\end{itemize}
Árvore santalácea da Índia portuguesa, (\textunderscore scleropyrum wallichianum\textunderscore , Arn.).
\section{Pereiral}
\begin{itemize}
\item {Grp. gram.:m.}
\end{itemize}
\begin{itemize}
\item {Proveniência:(De \textunderscore pereira\textunderscore )}
\end{itemize}
O mesmo que \textunderscore peral\textunderscore .
\section{Pereirara}
\begin{itemize}
\item {Grp. gram.:f.}
\end{itemize}
\begin{itemize}
\item {Utilização:Bras}
\end{itemize}
Gênero de árvores silvestres, bôas para obras de carpintaria.
O mesmo que \textunderscore pereiorá\textunderscore ?
\section{Pereirina}
\begin{itemize}
\item {Grp. gram.:f.}
\end{itemize}
Medicamento antithérmico.
\section{Pereiro}
\begin{itemize}
\item {Grp. gram.:m.}
\end{itemize}
\begin{itemize}
\item {Grp. gram.:Adj.}
\end{itemize}
\begin{itemize}
\item {Utilização:Prov.}
\end{itemize}
\begin{itemize}
\item {Utilização:minh.}
\end{itemize}
\begin{itemize}
\item {Proveniência:(Do b. lat. \textunderscore perarius\textunderscore )}
\end{itemize}
Árvore pomácea, variedade de macieira.
Diz-se do boi, que tem as pontas muito levantadas.
\section{Pereli}
\begin{itemize}
\item {Grp. gram.:adv.}
\end{itemize}
\begin{itemize}
\item {Utilização:Ant.}
\end{itemize}
Por ali.
Cp. \textunderscore perequi\textunderscore .
\section{Perelita}
\begin{itemize}
\item {Grp. gram.:f.}
\end{itemize}
\begin{itemize}
\item {Utilização:Miner.}
\end{itemize}
Ágata da Sibéria.
\section{Perelite}
\begin{itemize}
\item {Grp. gram.:f.}
\end{itemize}
\begin{itemize}
\item {Utilização:Miner.}
\end{itemize}
Ágata da Sibéria.
\section{Perém}
\begin{itemize}
\item {Grp. gram.:conj.}
\end{itemize}
\begin{itemize}
\item {Utilização:Ant.}
\end{itemize}
O mesmo que \textunderscore porém\textunderscore . Cf. G. Vicente.
\section{Perempção}
\begin{itemize}
\item {Grp. gram.:f.}
\end{itemize}
\begin{itemize}
\item {Utilização:Jur.}
\end{itemize}
\begin{itemize}
\item {Proveniência:(Lat. \textunderscore peremptio\textunderscore )}
\end{itemize}
Espécie de prescripção em processos.
\section{Perempto}
\begin{itemize}
\item {Grp. gram.:adj.}
\end{itemize}
\begin{itemize}
\item {Utilização:Jur.}
\end{itemize}
\begin{itemize}
\item {Proveniência:(Lat. \textunderscore peremptus\textunderscore )}
\end{itemize}
Extincto por prescripção.
\section{Peremptoriamente}
\begin{itemize}
\item {Grp. gram.:adv.}
\end{itemize}
De modo peremptório; terminantemente.
\section{Peremptório}
\begin{itemize}
\item {Grp. gram.:adj.}
\end{itemize}
\begin{itemize}
\item {Proveniência:(Lat. \textunderscore peremptorius\textunderscore )}
\end{itemize}
Que perime; decisivo; terminante.
\section{Perenal}
\begin{itemize}
\item {Grp. gram.:adj.}
\end{itemize}
O mesmo que \textunderscore perene\textunderscore .
\section{Perenalmente}
\begin{itemize}
\item {Grp. gram.:adv.}
\end{itemize}
De modo perenal.
\section{Perendengues}
\begin{itemize}
\item {Grp. gram.:m. pl.}
\end{itemize}
\begin{itemize}
\item {Utilização:bras}
\end{itemize}
\begin{itemize}
\item {Utilização:Ant.}
\end{itemize}
\begin{itemize}
\item {Proveniência:(T. cast.)}
\end{itemize}
Penduricalhos, para ornato; berloques:«\textunderscore outra moda, que annunciava a vinda do peralta, pelos guizos e perendengues do relogio...\textunderscore »Filinto, IV, 5.
\section{Perene}
\begin{itemize}
\item {Grp. gram.:adj.}
\end{itemize}
\begin{itemize}
\item {Proveniência:(Lat. \textunderscore perennis\textunderscore )}
\end{itemize}
Que dura por muitos anos; que não tem fim; eterno.
Incessante: \textunderscore fonte de água perene\textunderscore .
\section{Perenemente}
\begin{itemize}
\item {Grp. gram.:adv.}
\end{itemize}
De modo perene.
Continuamente; sem cessar.
\section{Perenidade}
\begin{itemize}
\item {Grp. gram.:f.}
\end{itemize}
\begin{itemize}
\item {Proveniência:(Lat. \textunderscore perennitas\textunderscore )}
\end{itemize}
Qualidade do que é perene.
\section{Perennal}
\begin{itemize}
\item {Grp. gram.:adj.}
\end{itemize}
O mesmo que \textunderscore perenne\textunderscore .
\section{Perennalmente}
\begin{itemize}
\item {Grp. gram.:adv.}
\end{itemize}
De modo perennal.
\section{Perenne}
\begin{itemize}
\item {Grp. gram.:adj.}
\end{itemize}
\begin{itemize}
\item {Proveniência:(Lat. \textunderscore perennis\textunderscore )}
\end{itemize}
Que dura por muitos annos; que não tem fim; eterno.
Incessante: \textunderscore fonte de água perenne\textunderscore .
\section{Perennemente}
\begin{itemize}
\item {Grp. gram.:adv.}
\end{itemize}
De modo perenne.
Continuamente; sem cessar.
\section{Perennidade}
\begin{itemize}
\item {Grp. gram.:f.}
\end{itemize}
\begin{itemize}
\item {Proveniência:(Lat. \textunderscore perennitas\textunderscore )}
\end{itemize}
Qualidade do que é perenne.
\section{Perequação}
\begin{itemize}
\item {Grp. gram.:f.}
\end{itemize}
\begin{itemize}
\item {Proveniência:(Lat. \textunderscore peraequatio\textunderscore )}
\end{itemize}
Acto de tornar igual uma coisa entre muitas pessoas: \textunderscore perequação do imposto; perequação das promoções militares...\textunderscore 
\section{Perequi}
\begin{itemize}
\item {Grp. gram.:adv.}
\end{itemize}
Por aqui:«\textunderscore perequi, entrou, pereli saíu.\textunderscore »G. Vicente, \textunderscore Auto das Fadas\textunderscore .
\section{Perereca}
\begin{itemize}
\item {Grp. gram.:f.}
\end{itemize}
\begin{itemize}
\item {Utilização:Bras}
\end{itemize}
\begin{itemize}
\item {Grp. gram.:M.  e  f.}
\end{itemize}
\begin{itemize}
\item {Utilização:Fig.}
\end{itemize}
Pequeno batrácio verde.
Pessôa ou animal de pequena estatura.
\section{Pererecar}
\begin{itemize}
\item {Grp. gram.:v. i.}
\end{itemize}
\begin{itemize}
\item {Utilização:Bras}
\end{itemize}
\begin{itemize}
\item {Proveniência:(De \textunderscore perereca\textunderscore )}
\end{itemize}
Mover-se vertiginosamente de um lado para outro; ficar desnorteado.
\section{Peretanda}
\begin{itemize}
\item {Grp. gram.:m.}
\end{itemize}
Cavalleiro, que, na antiga China, tomava parte em guardas de honra. Cf. \textunderscore Peregrinação\textunderscore , CVI.
Espécie de carregador, entre os antigos Chineses. Cf. \textunderscore Peregrinação\textunderscore , CXXXIX.
\section{Pereva}
\begin{itemize}
\item {Grp. gram.:f.}
\end{itemize}
\begin{itemize}
\item {Utilização:Bras. do S}
\end{itemize}
O mesmo que \textunderscore pereba\textunderscore .
\section{Perfazer}
\begin{itemize}
\item {Grp. gram.:v. t.}
\end{itemize}
\begin{itemize}
\item {Proveniência:(De \textunderscore per...\textunderscore  + \textunderscore fazer\textunderscore )}
\end{itemize}
Concluír.
Preencher o número de: \textunderscore perfaço hoje 39 annos\textunderscore .
Acabar de fazer; executar.
\section{Perfazimento}
\begin{itemize}
\item {Grp. gram.:m.}
\end{itemize}
Acto ou effeito de perfazer.
\section{Perfectibilidade}
\begin{itemize}
\item {Grp. gram.:f.}
\end{itemize}
Qualidade de perfectível.
\section{Perfectível}
\begin{itemize}
\item {Grp. gram.:adj.}
\end{itemize}
\begin{itemize}
\item {Proveniência:(Do lat. \textunderscore perfectus\textunderscore )}
\end{itemize}
Que póde aperfeiçoar-se; susceptível de perfeição.
\section{Perfectivo}
\begin{itemize}
\item {Grp. gram.:adj.}
\end{itemize}
\begin{itemize}
\item {Proveniência:(Lat. \textunderscore perfectivus\textunderscore )}
\end{itemize}
Que perfaz.
Que mostra perfeição.
\section{Perfeição}
\begin{itemize}
\item {Grp. gram.:f.}
\end{itemize}
\begin{itemize}
\item {Proveniência:(Lat. \textunderscore perfectio\textunderscore )}
\end{itemize}
Execução completa.
Acabamento; preenchimento.
Bondade ou excellência, em elevado grau.
Correcção.
Pureza.
Formosura; primor; requinte; mestria.
\section{Perfeiçoar}
\begin{itemize}
\item {Grp. gram.:v. t.}
\end{itemize}
O mesmo que \textunderscore aperfeiçoar\textunderscore .
\section{Perfeitação}
\begin{itemize}
\item {Grp. gram.:f.}
\end{itemize}
\begin{itemize}
\item {Utilização:Ant.}
\end{itemize}
Melhoria; utilidade; proveito.
(Talvez corr. de \textunderscore profeitação\textunderscore , do lat. \textunderscore profectus\textunderscore , se \textunderscore perfeitação\textunderscore  não vem de \textunderscore perfeito\textunderscore )
\section{Perfeitaço}
\begin{itemize}
\item {Grp. gram.:f.}
\end{itemize}
\begin{itemize}
\item {Utilização:Fam.}
\end{itemize}
Bem feito de corpo; galhardo. Cf. Castilho, \textunderscore Méd. á Fôrça\textunderscore , 102.
\section{Perfeitamente}
\begin{itemize}
\item {Grp. gram.:adv.}
\end{itemize}
De modo perfeito.
Completamente.
Satisfatoriamente.
Com perfeição; sem defeito.
\section{Perfeito}
\begin{itemize}
\item {Grp. gram.:adj.}
\end{itemize}
\begin{itemize}
\item {Utilização:Gram.}
\end{itemize}
\begin{itemize}
\item {Proveniência:(Do lat. \textunderscore perfectus\textunderscore )}
\end{itemize}
Que reúne todas as qualidades bôas; que não tem defeito: \textunderscore obra perfeita\textunderscore .
Diz-se dos tempos dos verbos, que exprimem uma acção ou estado já passado, em relação a certa época.
Magistral; notável; destro: \textunderscore perfeito cavalleiro\textunderscore .
\section{Perfidamente}
\begin{itemize}
\item {Grp. gram.:adv.}
\end{itemize}
De modo pérfido; com perfídia; traiçoeiramente.
\section{Perfídia}
\begin{itemize}
\item {Grp. gram.:f.}
\end{itemize}
\begin{itemize}
\item {Proveniência:(Lat. \textunderscore perfidia\textunderscore )}
\end{itemize}
Acto ou qualidade de pérfido.
\section{Pérfido}
\begin{itemize}
\item {Grp. gram.:adj.}
\end{itemize}
\begin{itemize}
\item {Proveniência:(Lat. \textunderscore perfidus\textunderscore )}
\end{itemize}
Que falta á sua fé; que não tem palavra ou lealdade; traidor; infiel.
Falso; que revela traição: \textunderscore palavras pérfidas\textunderscore .
\section{Perfil}
\begin{itemize}
\item {Grp. gram.:m.}
\end{itemize}
\begin{itemize}
\item {Utilização:Des.}
\end{itemize}
\begin{itemize}
\item {Proveniência:(Do b. lat. \textunderscore perfilum\textunderscore )}
\end{itemize}
Delineamento do rosto de alguém, visto de lado.
Aspecto.
Representação de um objecto, visto só de um lado.
Plano de um edifício ou delineação delle, como se fosse cortado perpendicularmente.
Lado.
Acto de alinhar, (falando-se de tropas).
O mesmo que \textunderscore debrum\textunderscore  ou ornato subtil da borda de uma peça de vestuário. Cf. L. Cordeiro, \textunderscore Senh. Duquesa\textunderscore , 5 e 300.
\section{Perfilar}
\begin{itemize}
\item {Grp. gram.:v. t.}
\end{itemize}
Traçar o perfil de.
Pôr em linha; endireitar, aprumar.
Debruar, orlar.
\section{Perfilhação}
\begin{itemize}
\item {Grp. gram.:f.}
\end{itemize}
Acto ou effeito de perfilhar.
\section{Perfilhador}
\begin{itemize}
\item {Grp. gram.:m.  e  adj.}
\end{itemize}
Aquelle que perfilha.
\section{Perfilhamento}
\begin{itemize}
\item {Grp. gram.:m.}
\end{itemize}
O mesmo que \textunderscore perfilhação\textunderscore .
\section{Perfilhar}
\begin{itemize}
\item {Grp. gram.:v. t.}
\end{itemize}
\begin{itemize}
\item {Utilização:P. us.}
\end{itemize}
\begin{itemize}
\item {Proveniência:(Do b. lat. hyp. \textunderscore perfiliare\textunderscore )}
\end{itemize}
Receber legalmente como filho; adoptar.
Attribuír:«\textunderscore Na mesma tela dos discursos que me perfilham, reconheço eu alguns remendos de minha pobreza.\textunderscore »Vieira, I, pról.
\section{Perfinca}
\begin{itemize}
\item {Grp. gram.:f.}
\end{itemize}
Traços, feitos no terreno pelo \textunderscore autor\textunderscore  ou maioral dos valladores, para indicar os limites dos córtes.
(Cp. \textunderscore fincar\textunderscore )
\section{Perfloração}
\begin{itemize}
\item {Grp. gram.:f.}
\end{itemize}
O mesmo que \textunderscore florescência\textunderscore .
\section{Perfolhada}
\begin{itemize}
\item {Grp. gram.:f.}
\end{itemize}
\begin{itemize}
\item {Proveniência:(De \textunderscore per\textunderscore  + \textunderscore fôlha\textunderscore )}
\end{itemize}
Planta umbellífera, (\textunderscore bupleurum protractum\textunderscore ).
\section{Perfolhado}
\begin{itemize}
\item {Grp. gram.:f.}
\end{itemize}
\begin{itemize}
\item {Utilização:Bot.}
\end{itemize}
\begin{itemize}
\item {Proveniência:(De \textunderscore per\textunderscore  + \textunderscore fôlha\textunderscore )}
\end{itemize}
Que está soldado naturalmente na base, (falando-se das fôlhas vegetaes).
\section{Perfolheação}
\begin{itemize}
\item {Grp. gram.:f.}
\end{itemize}
\begin{itemize}
\item {Proveniência:(De \textunderscore per\textunderscore  + \textunderscore fôlha\textunderscore )}
\end{itemize}
Acto ou effeito de se tornar perfolhado.
\section{Perfoliação}
\begin{itemize}
\item {Grp. gram.:f.}
\end{itemize}
\begin{itemize}
\item {Proveniência:(Do lat. \textunderscore per\textunderscore  + \textunderscore folium\textunderscore )}
\end{itemize}
O mesmo que \textunderscore perfolheação\textunderscore .
\section{Perfoliata-montana}
\begin{itemize}
\item {Grp. gram.:f.}
\end{itemize}
Planta da serra de Sintra.
\section{Perfulgência}
\begin{itemize}
\item {Grp. gram.:f.}
\end{itemize}
Qualidade de perfulgente.
\section{Perfulgente}
\begin{itemize}
\item {Grp. gram.:adj.}
\end{itemize}
\begin{itemize}
\item {Proveniência:(Lat. \textunderscore perfulgens\textunderscore )}
\end{itemize}
Que brilha muito, que resplandece.
\section{Perfumador}
\begin{itemize}
\item {Grp. gram.:adj.}
\end{itemize}
\begin{itemize}
\item {Grp. gram.:M.}
\end{itemize}
Que perfuma.
Vaso, em que se queimam perfumes.
\section{Perfumadura}
\begin{itemize}
\item {Grp. gram.:f.}
\end{itemize}
\begin{itemize}
\item {Utilização:P. us.}
\end{itemize}
Acto ou effeito de perfumar.
O mesmo que \textunderscore perfume\textunderscore . Cf. G. Vicente.
\section{Perfumar}
\begin{itemize}
\item {Grp. gram.:v. t.}
\end{itemize}
\begin{itemize}
\item {Proveniência:(De \textunderscore per\textunderscore  + \textunderscore fumo\textunderscore )}
\end{itemize}
Espalhar perfume em ou sôbre: \textunderscore perfumar um lenço\textunderscore .
Encher de perfume; tornar aromático, odorífero.
\section{Perfumaria}
\begin{itemize}
\item {Grp. gram.:f.}
\end{itemize}
Fábrica de perfumes.
Estabelecimento, onde se vendem perfumes.
Conjunto de perfumes.
Perfume.
\section{Perfume}
\begin{itemize}
\item {Grp. gram.:m.}
\end{itemize}
\begin{itemize}
\item {Utilização:Fig.}
\end{itemize}
\begin{itemize}
\item {Proveniência:(De \textunderscore perfumar\textunderscore )}
\end{itemize}
Cheiro agradável, que se exhala, como fumo ou como vapor, de uma substância odorífera.
Substância aromática, especialmente a que se vende nas perfumarias.
Suavidade; effeito agradável.
\section{Perfumismo}
\begin{itemize}
\item {Grp. gram.:m.}
\end{itemize}
Vicio da embriaguez por meio de perfumes.
\section{Perfumista}
\begin{itemize}
\item {Grp. gram.:m.  e  f.}
\end{itemize}
Pessôa, que vende ou fabríca perfumes.
\section{Perfumoso}
\begin{itemize}
\item {Grp. gram.:adj.}
\end{itemize}
Que exhala perfume; odorífero.
\section{Perfunctoriamente}
\begin{itemize}
\item {Grp. gram.:adv.}
\end{itemize}
De modo perfunctório; de relance; succintamente.
\section{Perfunctório}
\begin{itemize}
\item {Grp. gram.:adj.}
\end{itemize}
\begin{itemize}
\item {Proveniência:(Lat. \textunderscore perfunctorius\textunderscore )}
\end{itemize}
Que se pratíca em cumprimento de uma obrigação, ou sem fim útil.
Pouco útil; superficial; ligeiro.
\section{Perfuração}
\begin{itemize}
\item {Grp. gram.:f.}
\end{itemize}
Acto ou effeito de perfurar.
\section{Perfurador}
\begin{itemize}
\item {Grp. gram.:m.  e  adj.}
\end{itemize}
Aquillo que perfura ou é próprio para perfurar.
\section{Perfurante}
\begin{itemize}
\item {Grp. gram.:adj.}
\end{itemize}
\begin{itemize}
\item {Proveniência:(Lat. \textunderscore perforans\textunderscore )}
\end{itemize}
Que perfura.
\section{Perfurar}
\begin{itemize}
\item {Grp. gram.:v. t.}
\end{itemize}
\begin{itemize}
\item {Proveniência:(Lat. \textunderscore perforare\textunderscore )}
\end{itemize}
Fazer furo ou furos em; penetrar.
\section{Perfuso}
\begin{itemize}
\item {Grp. gram.:adj.}
\end{itemize}
\begin{itemize}
\item {Utilização:Bot.}
\end{itemize}
\begin{itemize}
\item {Proveniência:(Do lat. \textunderscore per\textunderscore  + \textunderscore fusus\textunderscore )}
\end{itemize}
Diz-se dos grãos, que se acham espalhados no interior dos frutos.
\section{Pergamilheiro}
\begin{itemize}
\item {Grp. gram.:m.}
\end{itemize}
\begin{itemize}
\item {Utilização:Ant.}
\end{itemize}
O mesmo que \textunderscore pergaminheiro\textunderscore .
\section{Pergamináceo}
\begin{itemize}
\item {Grp. gram.:adj.}
\end{itemize}
O mesmo ou melhor que \textunderscore pergaminháceo\textunderscore . Cf. Arn. Gama, \textunderscore Últ. Dona\textunderscore , 55.
\section{Pergamingo}
\begin{itemize}
\item {Grp. gram.:m.}
\end{itemize}
\begin{itemize}
\item {Utilização:Ant.}
\end{itemize}
O mesmo que \textunderscore pergaminho\textunderscore . Cf. \textunderscore Luz e Calor\textunderscore , 401.
\section{Pergaminháceo}
\begin{itemize}
\item {Grp. gram.:adj.}
\end{itemize}
Que tem o aspecto de pergaminho:«\textunderscore ...os musculos... pergaminháceos...\textunderscore »Camillo, \textunderscore Volcões\textunderscore , 154.
\section{Pergaminharia}
\begin{itemize}
\item {Grp. gram.:f.}
\end{itemize}
\begin{itemize}
\item {Proveniência:(De \textunderscore pergaminho\textunderscore )}
\end{itemize}
Commércio ou indústria de pergaminheiro.
\section{Pergaminheiro}
\begin{itemize}
\item {Grp. gram.:m.}
\end{itemize}
Aquelle que prepara ou vende pergaminho.
\section{Pergaminho}
\begin{itemize}
\item {Grp. gram.:m.}
\end{itemize}
\begin{itemize}
\item {Grp. gram.:Pl.}
\end{itemize}
\begin{itemize}
\item {Utilização:Fig.}
\end{itemize}
\begin{itemize}
\item {Proveniência:(Do lat. \textunderscore pergamina\textunderscore )}
\end{itemize}
Pelle de carneiro, de ovelha ou cordeiro, preparada com alúmen, e que serve especialmente para nella se escreverem coisas, que se querem conservadas por muito tempo.
Documento escrito em pergaminho.
Títulos de nobreza.
\section{Pergulária}
\begin{itemize}
\item {Grp. gram.:f.}
\end{itemize}
Gênero de plantas asclepiadáceas.
\section{Pergunta}
\begin{itemize}
\item {Grp. gram.:f.}
\end{itemize}
\begin{itemize}
\item {Proveniência:(De \textunderscore perguntar\textunderscore )}
\end{itemize}
Phrase ou phrases, com que se interroga; interrogação.
Inquirição; quesito.
\section{Perguntador}
\begin{itemize}
\item {Grp. gram.:m.  e  adj.}
\end{itemize}
O que pergunta; o que gosta de perguntar; aquelle que é curioso.
\section{Perguntante}
\begin{itemize}
\item {Grp. gram.:m.}
\end{itemize}
\begin{itemize}
\item {Proveniência:(De \textunderscore perguntar\textunderscore )}
\end{itemize}
Aquelle que faz perguntas; aquelle que interroga. Cf. Castilho, \textunderscore D. Quixote\textunderscore , II, 340.
\section{Perguntão}
\begin{itemize}
\item {Grp. gram.:m.}
\end{itemize}
Aquelle que pergunta muito, que quer saber tudo por curiosidade. Cf. V. de Seabra, \textunderscore Sát. e Ep.\textunderscore  II, 66.
\section{Perguntar}
\begin{itemize}
\item {Grp. gram.:v. t.}
\end{itemize}
\begin{itemize}
\item {Grp. gram.:V. i.}
\end{itemize}
Fazer perguntas a.
Inquirir; investigar.
Fazer perguntas.
Pedir ou buscar esclarecimentos.
(Indica-se, como etym., o lat. \textunderscore percontari\textunderscore ; os philólogos, porém, mostram que o lat. \textunderscore percontari\textunderscore  não podia produzir o port. \textunderscore perguntar\textunderscore , visto que o \textunderscore c\textunderscore  de \textunderscore percontari\textunderscore  não está entre sonoras. Portanto, ou o lat. \textunderscore percontari\textunderscore  passou a \textunderscore precontari\textunderscore , produzindo em port. e cast. \textunderscore preguntar\textunderscore , ou é outra a etym., de que é inseparável o pref. \textunderscore pre\textunderscore . A lei philológica do \textunderscore c\textunderscore  inter-vocálico e a uniformidade do cast. com o port. pop. em \textunderscore preguntar\textunderscore , não deixam aos philólogos modernos a menor dúvida de que \textunderscore preguntar\textunderscore  é a verdadeira orthogr. e de que \textunderscore perguntar\textunderscore  não tem justificação scientífica. Cf. V. Abreu, \textunderscore Liter. e Ling. Sanscr.\textunderscore , vocabulário, p. 160, onde se reproduzem as razões, anteriormente adduzidas por Gonç. Viana)
\section{Perhydrol}
\begin{itemize}
\item {Grp. gram.:m.}
\end{itemize}
Solução de água oxygenada, que se emprega na conservação do leite e noutros casos.
\section{Peri}
\begin{itemize}
\item {Grp. gram.:m.}
\end{itemize}
O mesmo que \textunderscore piri\textunderscore .
\section{Peri...}
\begin{itemize}
\item {Grp. gram.:pref.}
\end{itemize}
\begin{itemize}
\item {Proveniência:(Do gr. \textunderscore peri\textunderscore )}
\end{itemize}
(designativo de \textunderscore á roda\textunderscore )
\section{Periaca}
\begin{itemize}
\item {Grp. gram.:f.}
\end{itemize}
\begin{itemize}
\item {Utilização:Bras}
\end{itemize}
Árvore silvestre.
\section{Periacto}
\begin{itemize}
\item {Grp. gram.:m.}
\end{itemize}
Máquina de guerra, usada antigamente pelos Gregos.
Apparelho, que os Gregos usavam nos theatros para a mutação do scenário.
\section{Periambo}
\begin{itemize}
\item {Grp. gram.:m.}
\end{itemize}
\begin{itemize}
\item {Proveniência:(Lat. \textunderscore periambus\textunderscore )}
\end{itemize}
O mesmo que \textunderscore pyrrhíchio\textunderscore .
\section{Periândrico}
\begin{itemize}
\item {Grp. gram.:adj.}
\end{itemize}
\begin{itemize}
\item {Utilização:Bot.}
\end{itemize}
\begin{itemize}
\item {Proveniência:(Do gr. \textunderscore peri\textunderscore  + \textunderscore aner\textunderscore , \textunderscore andros\textunderscore )}
\end{itemize}
Que cérca os estames das flôres.
\section{Periandro}
\begin{itemize}
\item {Grp. gram.:m.}
\end{itemize}
\begin{itemize}
\item {Proveniência:(Do gr. \textunderscore peri\textunderscore  + \textunderscore aner\textunderscore , \textunderscore andros\textunderscore )}
\end{itemize}
Gênero de plantas leguminosas da América do Sul.
\section{Periantado}
\begin{itemize}
\item {Grp. gram.:adj.}
\end{itemize}
\begin{itemize}
\item {Utilização:Bot.}
\end{itemize}
Que tem perianto.
\section{Periantan}
\begin{itemize}
\item {Grp. gram.:m.}
\end{itemize}
\begin{itemize}
\item {Utilização:Bras. do N}
\end{itemize}
\begin{itemize}
\item {Proveniência:(Do guar. \textunderscore pery\textunderscore  + \textunderscore ãtã\textunderscore )}
\end{itemize}
Montão de canarana, que se junta nas margens dos rios, ou que vai fluctuando, arrastada pela corrente.
\section{Periânteo}
\begin{itemize}
\item {Grp. gram.:adj.}
\end{itemize}
\begin{itemize}
\item {Utilização:Bot.}
\end{itemize}
Que tem perianto simples.
\section{Perianthado}
\begin{itemize}
\item {Grp. gram.:adj.}
\end{itemize}
\begin{itemize}
\item {Utilização:Bot.}
\end{itemize}
Que tem periantho.
\section{Periântheo}
\begin{itemize}
\item {Grp. gram.:adj.}
\end{itemize}
\begin{itemize}
\item {Utilização:Bot.}
\end{itemize}
Que tem periantho simples.
\section{Periânthio}
\begin{itemize}
\item {Grp. gram.:m.}
\end{itemize}
O mesmo que \textunderscore periantho\textunderscore .
\section{Periantho}
\begin{itemize}
\item {Grp. gram.:m.}
\end{itemize}
\begin{itemize}
\item {Utilização:Bot.}
\end{itemize}
\begin{itemize}
\item {Proveniência:(Do gr. \textunderscore peri\textunderscore  + \textunderscore anthos\textunderscore )}
\end{itemize}
Invólucro exterior da flôr.
\section{Periântio}
\begin{itemize}
\item {Grp. gram.:m.}
\end{itemize}
O mesmo que \textunderscore perianto\textunderscore .
\section{Perianto}
\begin{itemize}
\item {Grp. gram.:m.}
\end{itemize}
\begin{itemize}
\item {Utilização:Bot.}
\end{itemize}
\begin{itemize}
\item {Proveniência:(Do gr. \textunderscore peri\textunderscore  + \textunderscore anthos\textunderscore )}
\end{itemize}
Invólucro exterior da flôr.
\section{Periapendicite}
\begin{itemize}
\item {Grp. gram.:f.}
\end{itemize}
\begin{itemize}
\item {Utilização:Med.}
\end{itemize}
\begin{itemize}
\item {Proveniência:(De \textunderscore peri...\textunderscore  + \textunderscore apendicite\textunderscore )}
\end{itemize}
Inflamação do tecido celular, que envolve o apêndice do ceco.
\section{Periappendicite}
\begin{itemize}
\item {Grp. gram.:f.}
\end{itemize}
\begin{itemize}
\item {Utilização:Med.}
\end{itemize}
\begin{itemize}
\item {Proveniência:(De \textunderscore peri...\textunderscore  + \textunderscore appendicite\textunderscore )}
\end{itemize}
Inflammação do tecido cellular, que envolve o apêndice do ceco.
\section{Periatis}
\begin{itemize}
\item {Grp. gram.:m. pl.}
\end{itemize}
\begin{itemize}
\item {Utilização:Bras}
\end{itemize}
Tríbo de indígenas do Pará.
\section{Periatris}
\begin{itemize}
\item {Grp. gram.:m. pl.}
\end{itemize}
O mesmo que \textunderscore periatis\textunderscore . Cf. Araújo e Amazonas, \textunderscore Diccion. Topogr.\textunderscore 
\section{Peribare}
\begin{itemize}
\item {Grp. gram.:m.}
\end{itemize}
\begin{itemize}
\item {Proveniência:(Gr. \textunderscore peribaris\textunderscore )}
\end{itemize}
Calçado, próprio de escravos, entre os antigos:«\textunderscore não calço o servil peribare, mas o livre cothurno pérsico\textunderscore ». Sous. Monteiro, \textunderscore Amores de Júlia\textunderscore , 28.
Calçado de mulhér, na antiguidade.
\section{Periblema}
\begin{itemize}
\item {Grp. gram.:m.}
\end{itemize}
\begin{itemize}
\item {Utilização:Bot.}
\end{itemize}
\begin{itemize}
\item {Proveniência:(Gr. \textunderscore periblema\textunderscore )}
\end{itemize}
Uma das partes, em que primeiro se differenceia o meristema primitivo, e que reveste immediatamente o que há de sêr o cylindro central, sendo, por seu lado, envolvido pelo dermatogênio. Cf. Gonç. Guimarães, \textunderscore Vocab. Etym.\textunderscore , II.
\section{Períbolo}
\begin{itemize}
\item {Grp. gram.:m.}
\end{itemize}
\begin{itemize}
\item {Utilização:Ant.}
\end{itemize}
\begin{itemize}
\item {Proveniência:(Lat. \textunderscore peribolus\textunderscore )}
\end{itemize}
Espaço entre um edifício e o muro que o cérca.
Pátio.
Arvoredo em volta dos templos, ordinariamente fechado por um muro; adro.
\section{Peribrose}
\begin{itemize}
\item {Grp. gram.:f.}
\end{itemize}
\begin{itemize}
\item {Proveniência:(Do gr. \textunderscore peri\textunderscore  + \textunderscore broskein\textunderscore )}
\end{itemize}
Ulceração das pálpebras.
\section{Pericalícia}
\begin{itemize}
\item {Grp. gram.:f.}
\end{itemize}
\begin{itemize}
\item {Utilização:Bot.}
\end{itemize}
\begin{itemize}
\item {Proveniência:(De \textunderscore peri...\textunderscore  + \textunderscore cálice\textunderscore )}
\end{itemize}
Nome, que Desvaux deu á sexta classe do méthodo de Jussieu.
\section{Pericardiário}
\begin{itemize}
\item {Grp. gram.:adj.}
\end{itemize}
Que se fórma no pericárdio.
\section{Pericárdico}
\begin{itemize}
\item {Grp. gram.:adj.}
\end{itemize}
O mesmo que \textunderscore pericardino\textunderscore .
\section{Pericardino}
\begin{itemize}
\item {Grp. gram.:adj.}
\end{itemize}
Relativo ao pericárdio.
\section{Pericárdio}
\begin{itemize}
\item {Grp. gram.:m.}
\end{itemize}
\begin{itemize}
\item {Proveniência:(Gr. \textunderscore perikardion\textunderscore )}
\end{itemize}
Saco membranoso, que envolve o coração.
\section{Pericardite}
\begin{itemize}
\item {Grp. gram.:f.}
\end{itemize}
Inflammação do pericárdio.
\section{Pericarpial}
\begin{itemize}
\item {Grp. gram.:adj.}
\end{itemize}
Que se desenvolve no pericarpo.
\section{Pericárpico}
\begin{itemize}
\item {Grp. gram.:adj.}
\end{itemize}
Relativo ao pericarpo.
\section{Pericarpo}
\begin{itemize}
\item {Grp. gram.:m.}
\end{itemize}
\begin{itemize}
\item {Proveniência:(Lat. \textunderscore pericarpum\textunderscore )}
\end{itemize}
Conjunto dos invólucros de uma semente ou fruto.
\section{Pericentral}
\begin{itemize}
\item {Grp. gram.:adj.}
\end{itemize}
\begin{itemize}
\item {Utilização:Neol.}
\end{itemize}
\begin{itemize}
\item {Proveniência:(De \textunderscore peri...\textunderscore  + \textunderscore central\textunderscore )}
\end{itemize}
Collocado interiormente em volta de alguma coisa.
Diz-se especialmente do systema ósseo, que abrange os ossos malares, lacrimaes, nasaes, palatinos, turbinados, mandibular e hyóide.
\section{Pericêntrico}
\begin{itemize}
\item {Grp. gram.:adj.}
\end{itemize}
\begin{itemize}
\item {Utilização:Bot.}
\end{itemize}
\begin{itemize}
\item {Proveniência:(De \textunderscore peri...\textunderscore  + \textunderscore centro\textunderscore )}
\end{itemize}
Diz-se da inserção dos estames, quando estes adherem ao cálice plano um tanto côncavo.
\section{Pericerático}
\begin{itemize}
\item {Grp. gram.:adj.}
\end{itemize}
\begin{itemize}
\item {Utilização:Med.}
\end{itemize}
\begin{itemize}
\item {Proveniência:(Do gr. \textunderscore peri\textunderscore  + \textunderscore keros\textunderscore )}
\end{itemize}
Diz-se do círculo vascular, que se vê á roda da córnea, nas ceratites.
\section{Perichécio}
\begin{itemize}
\item {fónica:qué}
\end{itemize}
\begin{itemize}
\item {Grp. gram.:m.}
\end{itemize}
\begin{itemize}
\item {Utilização:Bot.}
\end{itemize}
\begin{itemize}
\item {Proveniência:(Do gr. \textunderscore peri\textunderscore  + \textunderscore khaita\textunderscore )}
\end{itemize}
Reunião de pequenas fôlhas ou bractéolas, na base do pedicello que sustenta o urnário dos musgos.
\section{Perichôndrio}
\begin{itemize}
\item {fónica:con}
\end{itemize}
\begin{itemize}
\item {Grp. gram.:m.}
\end{itemize}
O mesmo ou melhor que \textunderscore perichondro\textunderscore .
\section{Perichondrite}
\begin{itemize}
\item {fónica:con}
\end{itemize}
\begin{itemize}
\item {Grp. gram.:f.}
\end{itemize}
\begin{itemize}
\item {Proveniência:(De \textunderscore perichondro\textunderscore )}
\end{itemize}
Inflammação do perichondro.
\section{Perichondro}
\begin{itemize}
\item {fónica:con}
\end{itemize}
\begin{itemize}
\item {Grp. gram.:m.}
\end{itemize}
\begin{itemize}
\item {Proveniência:(Do gr. \textunderscore peri\textunderscore  + \textunderscore khondros\textunderscore )}
\end{itemize}
Membrana fibrosa, que reveste as cartilagens.
\section{Perícia}
\begin{itemize}
\item {Grp. gram.:f.}
\end{itemize}
\begin{itemize}
\item {Proveniência:(Lat. \textunderscore peritia\textunderscore )}
\end{itemize}
Qualidade do que é perito; destreza; habilidade.
\section{Pericial}
\begin{itemize}
\item {Grp. gram.:adj.}
\end{itemize}
\begin{itemize}
\item {Utilização:Jur.}
\end{itemize}
\begin{itemize}
\item {Proveniência:(De \textunderscore perícia\textunderscore )}
\end{itemize}
Relativo a peritos: \textunderscore o laudo pericial\textunderscore .
\section{Periclina}
\begin{itemize}
\item {Grp. gram.:f.}
\end{itemize}
(V. \textunderscore periclínio\textunderscore ^2)
\section{Pericliniforme}
\begin{itemize}
\item {Grp. gram.:adj.}
\end{itemize}
\begin{itemize}
\item {Utilização:Bot.}
\end{itemize}
Que tem fórma de periclino.
\section{Periclínio}
\begin{itemize}
\item {Grp. gram.:m.}
\end{itemize}
\begin{itemize}
\item {Utilização:Bot.}
\end{itemize}
\begin{itemize}
\item {Proveniência:(Do gr. \textunderscore peri\textunderscore  + \textunderscore kline\textunderscore )}
\end{itemize}
Reunião de brácteas imbricadas e dispostas em tôrno de uma porção de flôres inseridas num receptáculo commum, como succede nas dáhlias.
\section{Periclínio}
\begin{itemize}
\item {Grp. gram.:m.}
\end{itemize}
\begin{itemize}
\item {Utilização:Miner.}
\end{itemize}
\begin{itemize}
\item {Proveniência:(Do gr. \textunderscore peri\textunderscore  + \textunderscore kline\textunderscore )}
\end{itemize}
Variedade de feldspatho.
\section{Periclino}
\begin{itemize}
\item {Grp. gram.:m.}
\end{itemize}
O mesmo que \textunderscore periclínio\textunderscore ^1.
\section{Periclitante}
\begin{itemize}
\item {Grp. gram.:adj.}
\end{itemize}
\begin{itemize}
\item {Proveniência:(Lat. \textunderscore periclitans\textunderscore )}
\end{itemize}
Que periclita, que corre perigo.
\section{Periclitar}
\begin{itemize}
\item {Grp. gram.:v. i.}
\end{itemize}
\begin{itemize}
\item {Proveniência:(Lat. \textunderscore periclitari\textunderscore )}
\end{itemize}
Estar em perigo.
\section{Pericolpite}
\begin{itemize}
\item {Grp. gram.:f.}
\end{itemize}
\begin{itemize}
\item {Utilização:Med.}
\end{itemize}
\begin{itemize}
\item {Proveniência:(Do gr. \textunderscore peri\textunderscore  + \textunderscore kolpos\textunderscore )}
\end{itemize}
Inflammação em roda da vagina.
\section{Pericôndrio}
\begin{itemize}
\item {Grp. gram.:m.}
\end{itemize}
\begin{itemize}
\item {Proveniência:(Do gr. \textunderscore peri\textunderscore  + \textunderscore khondros\textunderscore )}
\end{itemize}
Membrana fibrosa, que reveste as cartilagens.
\section{Pericôndrio}
\begin{itemize}
\item {Grp. gram.:m.}
\end{itemize}
O mesmo ou melhor que \textunderscore pericondro\textunderscore .
\section{Pericondrite}
\begin{itemize}
\item {Grp. gram.:f.}
\end{itemize}
\begin{itemize}
\item {Proveniência:(De \textunderscore pericondro\textunderscore )}
\end{itemize}
Inflamação do pericondro.
\section{Pericondro}
\begin{itemize}
\item {Grp. gram.:m.}
\end{itemize}
\begin{itemize}
\item {Proveniência:(Do gr. \textunderscore peri\textunderscore  + \textunderscore khondros\textunderscore )}
\end{itemize}
Membrana fibrosa, que reveste as cartilagens.
\section{Perícope}
\begin{itemize}
\item {Grp. gram.:f.}
\end{itemize}
\begin{itemize}
\item {Proveniência:(Lat. \textunderscore pericope\textunderscore )}
\end{itemize}
Secção ou parágrapho, (falando se de livros sagrados).
\section{Pericorólia}
\begin{itemize}
\item {Grp. gram.:f.}
\end{itemize}
\begin{itemize}
\item {Proveniência:(De \textunderscore peri...\textunderscore  + \textunderscore corola\textunderscore )}
\end{itemize}
Classe de plantas que, no sistema de Jussieu, abrange as dicotiledóneas monopétalas de estames políginos.
\section{Pericoróllia}
\begin{itemize}
\item {Grp. gram.:f.}
\end{itemize}
\begin{itemize}
\item {Proveniência:(De \textunderscore peri...\textunderscore  + \textunderscore corolla\textunderscore )}
\end{itemize}
Classe de plantas que, no systema de Jussieu, abrange as dicotyledóneas monopétalas de estames polýginos.
\section{Periciclo}
\begin{itemize}
\item {Grp. gram.:m.}
\end{itemize}
\begin{itemize}
\item {Utilização:Bot.}
\end{itemize}
\begin{itemize}
\item {Proveniência:(Gr. \textunderscore perikuklos\textunderscore )}
\end{itemize}
Capa celular, que constitue a parte mais externa do cilindro central da raíz, caule e ramos das plantas vasculares.
\section{Pericistite}
\begin{itemize}
\item {Grp. gram.:f.}
\end{itemize}
\begin{itemize}
\item {Utilização:Med.}
\end{itemize}
\begin{itemize}
\item {Proveniência:(Do gr. \textunderscore peri\textunderscore  + \textunderscore kustis\textunderscore )}
\end{itemize}
Inflamação dos tecidos que rodeiam a bexiga.
\section{Pericrânio}
\begin{itemize}
\item {Grp. gram.:m.}
\end{itemize}
\begin{itemize}
\item {Utilização:Anat.}
\end{itemize}
\begin{itemize}
\item {Proveniência:(Gr. \textunderscore perikranion\textunderscore )}
\end{itemize}
Periósteo, que reveste a superfície externa do crânio.
\section{Pericyclo}
\begin{itemize}
\item {Grp. gram.:m.}
\end{itemize}
\begin{itemize}
\item {Utilização:Bot.}
\end{itemize}
\begin{itemize}
\item {Proveniência:(Gr. \textunderscore perikuklos\textunderscore )}
\end{itemize}
Capa cellular, que constitue a parte mais externa do cylindro central da raíz, caule e ramos das plantas vasculares.
\section{Pericystite}
\begin{itemize}
\item {Grp. gram.:f.}
\end{itemize}
\begin{itemize}
\item {Utilização:Med.}
\end{itemize}
\begin{itemize}
\item {Proveniência:(Do gr. \textunderscore peri\textunderscore  + \textunderscore kustis\textunderscore )}
\end{itemize}
Inflammação dos tecidos que rodeiam a bexiga.
\section{Peridás}
Indígenas do norte do Brasil.
\section{Perídea}
\begin{itemize}
\item {Grp. gram.:f.}
\end{itemize}
Gênero de insectos lepidópteros nocturnos.
\section{Perididimite}
\begin{itemize}
\item {Grp. gram.:f.}
\end{itemize}
\begin{itemize}
\item {Utilização:Med.}
\end{itemize}
Inflamação do peridídimo.
\section{Peridídimo}
\begin{itemize}
\item {Grp. gram.:m.}
\end{itemize}
\begin{itemize}
\item {Utilização:Anat.}
\end{itemize}
\begin{itemize}
\item {Proveniência:(Do gr. \textunderscore peri\textunderscore  + \textunderscore didumos\textunderscore )}
\end{itemize}
Túnica albuginosa, que reveste os testículos.
\section{Perididymite}
\begin{itemize}
\item {Grp. gram.:f.}
\end{itemize}
\begin{itemize}
\item {Utilização:Med.}
\end{itemize}
Inflammação do peridídymo.
\section{Peridídymo}
\begin{itemize}
\item {Grp. gram.:m.}
\end{itemize}
\begin{itemize}
\item {Utilização:Anat.}
\end{itemize}
\begin{itemize}
\item {Proveniência:(Do gr. \textunderscore peri\textunderscore  + \textunderscore didumos\textunderscore )}
\end{itemize}
Túnica albuginosa, que reveste os testículos.
\section{Peridíneos}
\begin{itemize}
\item {Grp. gram.:m. pl.}
\end{itemize}
Grupo, ainda não classificado, de animaes ou plantas.
\section{Perídio}
\begin{itemize}
\item {Grp. gram.:m.}
\end{itemize}
\begin{itemize}
\item {Utilização:Bot.}
\end{itemize}
Conceptáculo que, nos cogumelos, cérca as partes da fructificação.
(Provavelmente, do gr. \textunderscore periduein\textunderscore , envolver)
\section{Peridíola}
\begin{itemize}
\item {Grp. gram.:f.}
\end{itemize}
\begin{itemize}
\item {Utilização:Bot.}
\end{itemize}
O perídio interior, quando nos cogumelos se observam dois.
\section{Peridíolo}
\begin{itemize}
\item {Grp. gram.:m.}
\end{itemize}
O mesmo que \textunderscore peridíola\textunderscore .
\section{Peridiscal}
\begin{itemize}
\item {Grp. gram.:adj.}
\end{itemize}
\begin{itemize}
\item {Utilização:Bot.}
\end{itemize}
\begin{itemize}
\item {Proveniência:(De \textunderscore peri...\textunderscore  + \textunderscore disco\textunderscore )}
\end{itemize}
Diz-se da inserção dos estames, quando estes se inserem á volta da base de um disco.
\section{Peridótico}
\begin{itemize}
\item {Grp. gram.:adj.}
\end{itemize}
\begin{itemize}
\item {Utilização:Miner.}
\end{itemize}
\begin{itemize}
\item {Proveniência:(De \textunderscore peridoto\textunderscore )}
\end{itemize}
Diz-se das rochas, em que predomina o peridoto.
\section{Peridotite}
\begin{itemize}
\item {Grp. gram.:f.}
\end{itemize}
\begin{itemize}
\item {Utilização:Miner.}
\end{itemize}
Rocha primitiva, com posta de perídoto, só, ou acompanhado de outro mineral.
\section{Peridotito}
\begin{itemize}
\item {Grp. gram.:f.}
\end{itemize}
\begin{itemize}
\item {Utilização:Miner.}
\end{itemize}
Rocha primitiva, com posta de perídoto, só, ou acompanhado de outro mineral.
\section{Peridoto}
\begin{itemize}
\item {Grp. gram.:m.}
\end{itemize}
\begin{itemize}
\item {Proveniência:(Fr. \textunderscore péridot\textunderscore )}
\end{itemize}
Pedra preciosa, que é um silicato de bases variáveis.
\section{Peridotoso}
\begin{itemize}
\item {Grp. gram.:adj.}
\end{itemize}
Que contém grãos de perídoto.
\section{Peridrol}
\begin{itemize}
\item {Grp. gram.:m.}
\end{itemize}
Solução de água oxigenada, que se emprega na conservação do leite e noutros casos.
\section{Perídromo}
\begin{itemize}
\item {Grp. gram.:m.}
\end{itemize}
\begin{itemize}
\item {Proveniência:(Gr. \textunderscore peridromos\textunderscore )}
\end{itemize}
Espaço coberto, em volta de um edifício.
\section{Perídyo}
\begin{itemize}
\item {Grp. gram.:m.}
\end{itemize}
\begin{itemize}
\item {Utilização:Bot.}
\end{itemize}
Conceptáculo que, nos cogumelos, cérca as partes da fructificação.
(Provavelmente, do gr. \textunderscore periduein\textunderscore , envolver)
\section{Periecos}
\begin{itemize}
\item {Grp. gram.:m. pl.}
\end{itemize}
\begin{itemize}
\item {Utilização:Geogr.}
\end{itemize}
\begin{itemize}
\item {Proveniência:(Gr. \textunderscore perioikos\textunderscore )}
\end{itemize}
Habitantes da Terra que, tendo a mesma latitude, têm longitude opposta ou 190° de differença.
\section{Periélio}
\begin{itemize}
\item {Grp. gram.:m.}
\end{itemize}
\begin{itemize}
\item {Utilização:Astron.}
\end{itemize}
\begin{itemize}
\item {Proveniência:(Do gr. \textunderscore peri\textunderscore  + \textunderscore helios\textunderscore )}
\end{itemize}
Extremidade do grande eixo da órbita de um planeta, que fica mais perto do Sol.
\section{Periérese}
\begin{itemize}
\item {Grp. gram.:f.}
\end{itemize}
\begin{itemize}
\item {Utilização:Med.}
\end{itemize}
\begin{itemize}
\item {Proveniência:(Gr. \textunderscore periairesis\textunderscore )}
\end{itemize}
Incisão circular, com que se circumscrevia a base dos grandes abcessos.
\section{Periergia}
\begin{itemize}
\item {Grp. gram.:f.}
\end{itemize}
\begin{itemize}
\item {Proveniência:(Gr. \textunderscore periergia\textunderscore )}
\end{itemize}
Excessivo apuro de linguagem.
\section{Periforme}
\begin{itemize}
\item {Grp. gram.:adj.}
\end{itemize}
\begin{itemize}
\item {Proveniência:(De \textunderscore pêra\textunderscore  + \textunderscore fórma\textunderscore )}
\end{itemize}
Que tem fórma de pêra. Cf. Capello e Ivens, II, 75.
\section{Perigador}
\begin{itemize}
\item {Grp. gram.:adj.}
\end{itemize}
\begin{itemize}
\item {Utilização:P. us.}
\end{itemize}
\begin{itemize}
\item {Proveniência:(De \textunderscore perigar\textunderscore )}
\end{itemize}
Que ameaça perigos; em que há perigo:«\textunderscore ...da perigadora peleja...\textunderscore »Filinto, \textunderscore D. Man.\textunderscore , 110.
\section{Perigalho}
\begin{itemize}
\item {Grp. gram.:m.}
\end{itemize}
\begin{itemize}
\item {Utilização:Náut.}
\end{itemize}
Pelle da barba ou do pescoço, descaída por velhice ou magreza.
Cabo, para levantar o centro de um tôldo ou sustenatar superiormente o mastro de mezena.
(Cast. \textunderscore perigallo\textunderscore )
\section{Perigar}
\begin{itemize}
\item {Grp. gram.:v. i.}
\end{itemize}
\begin{itemize}
\item {Utilização:Prov.}
\end{itemize}
Correr perigo; estar em perigo; periclitar.
Abortar involuntariamente.
\section{Perigeu}
\begin{itemize}
\item {Grp. gram.:m.}
\end{itemize}
\begin{itemize}
\item {Utilização:Astron.}
\end{itemize}
\begin{itemize}
\item {Proveniência:(Gr. \textunderscore perigeion\textunderscore )}
\end{itemize}
Ponto, em que a órbita de um planeta está mais próxima da Terra.
\section{Periginia}
\begin{itemize}
\item {Grp. gram.:f.}
\end{itemize}
\begin{itemize}
\item {Utilização:Bot.}
\end{itemize}
Estado ou disposição de perígino.
\section{Periginândrio}
\begin{itemize}
\item {Grp. gram.:m.}
\end{itemize}
\begin{itemize}
\item {Utilização:Bot.}
\end{itemize}
\begin{itemize}
\item {Proveniência:(Do gr. \textunderscore peri\textunderscore  + \textunderscore gune\textunderscore  + \textunderscore aner\textunderscore )}
\end{itemize}
O mesmo que \textunderscore perianto\textunderscore .
\section{Periginandro}
\begin{itemize}
\item {Grp. gram.:m.}
\end{itemize}
O mesmo que \textunderscore periginândrio\textunderscore .
\section{Perígino}
\begin{itemize}
\item {Grp. gram.:adj.}
\end{itemize}
\begin{itemize}
\item {Utilização:Bot.}
\end{itemize}
\begin{itemize}
\item {Proveniência:(Do gr. \textunderscore peri\textunderscore  + \textunderscore gune\textunderscore )}
\end{itemize}
Diz-se dos estames, que se inserem á volta do órgão sexual feminino da flôr.
\section{Perigo}
\begin{itemize}
\item {Grp. gram.:m.}
\end{itemize}
\begin{itemize}
\item {Utilização:Prov.}
\end{itemize}
\begin{itemize}
\item {Proveniência:(Do lat. \textunderscore periculum\textunderscore )}
\end{itemize}
Estado, em que alguma coisa se receia; risco.
Gravidade.
Abôrto involuntário.
\section{Perigolati}
\begin{itemize}
\item {Grp. gram.:m.}
\end{itemize}
Planta da serra de Sintra.
\section{Perigónio}
\begin{itemize}
\item {Grp. gram.:m.}
\end{itemize}
O mesmo ou melhor que \textunderscore perígono\textunderscore .
\section{Perígono}
\begin{itemize}
\item {Grp. gram.:m.}
\end{itemize}
\begin{itemize}
\item {Utilização:Bot.}
\end{itemize}
\begin{itemize}
\item {Utilização:Miner.}
\end{itemize}
\begin{itemize}
\item {Proveniência:(Do gr. \textunderscore peri\textunderscore  + \textunderscore gonos\textunderscore )}
\end{itemize}
Invólucro immediato dos órgãos sexuaes das flôres, cujo cálice e corolla estão soldados em toda a extensão.
Variedade de ágata.
O mesmo que \textunderscore periantho\textunderscore .
\section{Perigosamente}
\begin{itemize}
\item {Grp. gram.:adv.}
\end{itemize}
De modo perigoso; com perigo ou risco.
\section{Perigoso}
\begin{itemize}
\item {Grp. gram.:adj.}
\end{itemize}
\begin{itemize}
\item {Proveniência:(Do lat. \textunderscore periculosus\textunderscore )}
\end{itemize}
Em que há perigo; que causa perigo; que ameaça perigo.
\section{Perígrafo}
\begin{itemize}
\item {Grp. gram.:m.}
\end{itemize}
\begin{itemize}
\item {Utilização:Anat.}
\end{itemize}
\begin{itemize}
\item {Proveniência:(Do gr. \textunderscore peri\textunderscore  + \textunderscore graphein\textunderscore )}
\end{itemize}
Inserção tendinosa dos músculos rectos do abdome.
\section{Perígrapho}
\begin{itemize}
\item {Grp. gram.:m.}
\end{itemize}
\begin{itemize}
\item {Utilização:Anat.}
\end{itemize}
\begin{itemize}
\item {Proveniência:(Do gr. \textunderscore peri\textunderscore  + \textunderscore graphein\textunderscore )}
\end{itemize}
Inserção tendinosa dos músculos rectos do abdome.
\section{Periguado}
\begin{itemize}
\item {Grp. gram.:adj.}
\end{itemize}
\begin{itemize}
\item {Utilização:Ant.}
\end{itemize}
Que corre perigo; que tem imminente qualquer damno.
\section{Perigynândrio}
\begin{itemize}
\item {Grp. gram.:m.}
\end{itemize}
\begin{itemize}
\item {Utilização:Bot.}
\end{itemize}
\begin{itemize}
\item {Proveniência:(Do gr. \textunderscore peri\textunderscore  + \textunderscore gune\textunderscore  + \textunderscore aner\textunderscore )}
\end{itemize}
O mesmo que \textunderscore periantho\textunderscore .
\section{Perigynandro}
\begin{itemize}
\item {Grp. gram.:m.}
\end{itemize}
O mesmo que \textunderscore perigynândrio\textunderscore .
\section{Perigynia}
\begin{itemize}
\item {Grp. gram.:f.}
\end{itemize}
\begin{itemize}
\item {Utilização:Bot.}
\end{itemize}
Estado ou disposição de perígino.
\section{Perígyno}
\begin{itemize}
\item {Grp. gram.:adj.}
\end{itemize}
\begin{itemize}
\item {Utilização:Bot.}
\end{itemize}
\begin{itemize}
\item {Proveniência:(Do gr. \textunderscore peri\textunderscore  + \textunderscore gune\textunderscore )}
\end{itemize}
Diz-se dos estames, que se inserem á volta do órgão sexual feminino da flôr.
\section{Perila}
\begin{itemize}
\item {Grp. gram.:f.}
\end{itemize}
Planta annual, de flôr encarnada.
\section{Perilha}
\begin{itemize}
\item {Grp. gram.:f.}
\end{itemize}
Ornato, semelhante a uma pêra.
\section{Perilo}
\begin{itemize}
\item {Grp. gram.:m.}
\end{itemize}
Remate pyramidal, muito agudo.
\section{Perilômia}
\begin{itemize}
\item {Grp. gram.:f.}
\end{itemize}
\begin{itemize}
\item {Proveniência:(Do gr. \textunderscore peri\textunderscore  + \textunderscore loma\textunderscore )}
\end{itemize}
Gênero de plantas labíadas do Peru.
\section{Perimetria}
\begin{itemize}
\item {Grp. gram.:f.}
\end{itemize}
Medida do perímetro.
\section{Perimétrico}
\begin{itemize}
\item {Grp. gram.:adj.}
\end{itemize}
Relativo ao perímetro.
\section{Peremitrite}
\begin{itemize}
\item {Grp. gram.:f.}
\end{itemize}
\begin{itemize}
\item {Utilização:Med.}
\end{itemize}
\begin{itemize}
\item {Proveniência:(Do gr. \textunderscore peri\textunderscore  + \textunderscore metra\textunderscore )}
\end{itemize}
Inflamação do tecido laminoso, que cérca o útero.
\section{Perímetro}
\begin{itemize}
\item {Grp. gram.:m.}
\end{itemize}
\begin{itemize}
\item {Proveniência:(Lat. \textunderscore perimetros\textunderscore )}
\end{itemize}
Linha, que contorna uma figura.
Circunferência.
Campímetro.
\section{Perimir}
\begin{itemize}
\item {Grp. gram.:v. t.}
\end{itemize}
\begin{itemize}
\item {Proveniência:(Lat. \textunderscore perimere\textunderscore )}
\end{itemize}
Pôr termo a (acção ou instância judicial).
\section{Perimísio}
\begin{itemize}
\item {Grp. gram.:m.}
\end{itemize}
\begin{itemize}
\item {Utilização:Anat.}
\end{itemize}
\begin{itemize}
\item {Proveniência:(Do gr. \textunderscore peri\textunderscore  + \textunderscore mus\textunderscore )}
\end{itemize}
Tecido laminoso, em volta dos fascículos secundários de muitos fascículos primitivos dos músculos.
\section{Perimorfose}
\begin{itemize}
\item {Grp. gram.:f.}
\end{itemize}
\begin{itemize}
\item {Proveniência:(Do gr. \textunderscore peri\textunderscore  + \textunderscore morphe\textunderscore )}
\end{itemize}
Transformação de uma larva em crisálida.
\section{Perimorphose}
\begin{itemize}
\item {Grp. gram.:f.}
\end{itemize}
\begin{itemize}
\item {Proveniência:(Do gr. \textunderscore peri\textunderscore  + \textunderscore morphe\textunderscore )}
\end{itemize}
Transformação de uma larva em crysállida.
\section{Perimýsio}
\begin{itemize}
\item {Grp. gram.:m.}
\end{itemize}
\begin{itemize}
\item {Utilização:Anat.}
\end{itemize}
\begin{itemize}
\item {Proveniência:(Do gr. \textunderscore peri\textunderscore  + \textunderscore mus\textunderscore )}
\end{itemize}
Tecido laminoso, em volta dos fascículos secundários de muitos fascículos primitivos dos músculos.
\section{Periná}
\begin{itemize}
\item {Grp. gram.:m.}
\end{itemize}
O mesmo que \textunderscore cana-de-macaco\textunderscore .
\section{Perineal}
\begin{itemize}
\item {Grp. gram.:adj.}
\end{itemize}
Relativo ao perinéu.
\section{Perinefrite}
\begin{itemize}
\item {Grp. gram.:f.}
\end{itemize}
\begin{itemize}
\item {Utilização:Med.}
\end{itemize}
\begin{itemize}
\item {Proveniência:(De \textunderscore peri...\textunderscore  + \textunderscore nephrite\textunderscore )}
\end{itemize}
Inflamação do tecido que envolve o rim, por oposição a \textunderscore nefrite\textunderscore , que é a inflamação do próprio rim.
\section{Perineocele}
\begin{itemize}
\item {Grp. gram.:m.}
\end{itemize}
\begin{itemize}
\item {Utilização:Med.}
\end{itemize}
\begin{itemize}
\item {Proveniência:(Do gr. \textunderscore perinaios\textunderscore  + \textunderscore kele\textunderscore )}
\end{itemize}
Hérnia do perinéu.
\section{Perineoplastia}
\begin{itemize}
\item {Grp. gram.:f.}
\end{itemize}
\begin{itemize}
\item {Utilização:Cir.}
\end{itemize}
Autoplastia da região perineal.
\section{Perinephrite}
\begin{itemize}
\item {Grp. gram.:f.}
\end{itemize}
\begin{itemize}
\item {Utilização:Med.}
\end{itemize}
\begin{itemize}
\item {Proveniência:(De \textunderscore peri...\textunderscore  + \textunderscore nephrite\textunderscore )}
\end{itemize}
Inflammação do tecido que envolve o rim, por opposição a \textunderscore nephrite\textunderscore , que é a inflammação do próprio rim.
\section{Perinéu}
\begin{itemize}
\item {Grp. gram.:m.}
\end{itemize}
\begin{itemize}
\item {Utilização:Anat.}
\end{itemize}
\begin{itemize}
\item {Proveniência:(Gr. \textunderscore perinaios\textunderscore )}
\end{itemize}
Espaço entre o ânus e os órgãos sexuaes.
\section{Perineuro}
\begin{itemize}
\item {Grp. gram.:m.}
\end{itemize}
\begin{itemize}
\item {Utilização:Anat.}
\end{itemize}
\begin{itemize}
\item {Proveniência:(Do gr. \textunderscore peri\textunderscore  + \textunderscore neuron\textunderscore )}
\end{itemize}
Tecido conjunctivo, que liga as fibras de um nervo.
\section{Peringueiro}
\begin{itemize}
\item {Grp. gram.:m.}
\end{itemize}
\begin{itemize}
\item {Utilização:T. do Fundão}
\end{itemize}
Mendigo, pedinte; pobretão.
\section{Periodeuta}
\begin{itemize}
\item {Grp. gram.:m.}
\end{itemize}
\begin{itemize}
\item {Proveniência:(Lat. \textunderscore periodeuta\textunderscore )}
\end{itemize}
Espécie de inspector ecclesiástico, no rito grego.
\section{Periodical}
\begin{itemize}
\item {Grp. gram.:adj.}
\end{itemize}
\begin{itemize}
\item {Utilização:Deprec.}
\end{itemize}
Relativo a periódicos ou a jornalistas. Cf. Camillo, \textunderscore Cancion. Al.\textunderscore , 253.
\section{Periodicamente}
\begin{itemize}
\item {Grp. gram.:adv.}
\end{itemize}
De modo periódico.
\section{Periodicidade}
\begin{itemize}
\item {Grp. gram.:f.}
\end{itemize}
Qualidade daquillo que é periódico.
\section{Periodicista}
\begin{itemize}
\item {Grp. gram.:m.}
\end{itemize}
\begin{itemize}
\item {Utilização:P. us.}
\end{itemize}
Aquelle que escreve ou redige periódicos ou um periódico. Cf. Sousa Monteiro, \textunderscore Elog. de Lat.\textunderscore 
\section{Periódico}
\begin{itemize}
\item {Grp. gram.:adj.}
\end{itemize}
\begin{itemize}
\item {Utilização:Arith.}
\end{itemize}
\begin{itemize}
\item {Grp. gram.:M.}
\end{itemize}
\begin{itemize}
\item {Proveniência:(Lat. \textunderscore periodicus\textunderscore )}
\end{itemize}
Relativo a período.
Que se repete em determinados tempos.
Que em certas horas ou em certos dias manifesta certos phenómenos ou symptomas.
Intermittente: \textunderscore febres periodicas\textunderscore .
Diz-se das fracções decimaes, cujos algarismos, ou somente alguns, se reproduzem na mesma ordem infinitamente.
Diz-se das obras ou publicações, que apparecem em tempos determinados.
Gazeta, que tem dias fixos para a sua publicação.
\section{Periodiqueiro}
\begin{itemize}
\item {Grp. gram.:m.  e  adj.}
\end{itemize}
\begin{itemize}
\item {Utilização:Deprec.}
\end{itemize}
Redactor de periódicos.
O que escreve em jornaes.
\section{Periodiquista}
\begin{itemize}
\item {Grp. gram.:m.}
\end{itemize}
(V.periodicista)
\section{Periodista}
\begin{itemize}
\item {Grp. gram.:m.}
\end{itemize}
\begin{itemize}
\item {Proveniência:(Fr. \textunderscore periodiste\textunderscore )}
\end{itemize}
Aquelle que escreve periódicos ou em periódicos; periodicista.
\section{Periodístico}
\begin{itemize}
\item {Grp. gram.:adj.}
\end{itemize}
Relativo a periodista.
\section{Periodismo}
\begin{itemize}
\item {Grp. gram.:m.}
\end{itemize}
O mesmo quo \textunderscore jornalismo\textunderscore .
(Por \textunderscore periodicista\textunderscore , de \textunderscore periódico\textunderscore )
\section{Periodização}
\begin{itemize}
\item {Grp. gram.:f.}
\end{itemize}
Acto ou effeito de periodizar.
\section{Periferia}
\begin{itemize}
\item {Grp. gram.:f.}
\end{itemize}
\begin{itemize}
\item {Proveniência:(Lat. \textunderscore peripheria\textunderscore )}
\end{itemize}
Contôrno de uma figura curvilínea.
Superfície de um sólido.
Circunferência.
\section{Periférico}
\begin{itemize}
\item {Grp. gram.:adj.}
\end{itemize}
\begin{itemize}
\item {Utilização:Bot.}
\end{itemize}
Relativo a periferia.
Diz-se do perisperma, quando envolve e occulta o embrião.
E diz-se do embrião, quando cérca o perisperma, no todo ou em parte.
\section{Periforanto}
\begin{itemize}
\item {Grp. gram.:m.}
\end{itemize}
\begin{itemize}
\item {Utilização:Bot.}
\end{itemize}
\begin{itemize}
\item {Proveniência:(De \textunderscore peri...\textunderscore  + \textunderscore foranto\textunderscore )}
\end{itemize}
Conjunto das brácteas que, nas sinantéreas, cercam a reunião das flôres.
\section{Períforo}
\begin{itemize}
\item {Grp. gram.:m.}
\end{itemize}
\begin{itemize}
\item {Utilização:Bot.}
\end{itemize}
\begin{itemize}
\item {Proveniência:(Do gr. \textunderscore peri\textunderscore  + \textunderscore phoros\textunderscore )}
\end{itemize}
Órgão, que sustenta o ovário de certas plantas e dá ligamento ás pétalas e aos estames.
\section{Perífrase}
\begin{itemize}
\item {Grp. gram.:f.}
\end{itemize}
\begin{itemize}
\item {Proveniência:(Lat. \textunderscore periphrasis\textunderscore )}
\end{itemize}
Conjunto de palavras ou frases, empregadas em vêz de uma só palavra ou termo próprio.
Circunlóquio.
\section{Perifrástico}
\begin{itemize}
\item {Grp. gram.:adj.}
\end{itemize}
\begin{itemize}
\item {Proveniência:(Gr. \textunderscore periphrastikos\textunderscore )}
\end{itemize}
Relativo á perífrase.
\section{Periodizar}
\begin{itemize}
\item {Grp. gram.:v. t.}
\end{itemize}
\begin{itemize}
\item {Utilização:Neol.}
\end{itemize}
Dividir em períodos; expôr por períodos.
\section{Período}
\begin{itemize}
\item {Grp. gram.:m.}
\end{itemize}
\begin{itemize}
\item {Utilização:Arith.}
\end{itemize}
\begin{itemize}
\item {Utilização:Mús.}
\end{itemize}
\begin{itemize}
\item {Proveniência:(Lat. \textunderscore periodüs\textunderscore )}
\end{itemize}
Tempo, que decorre entre dois factos.
Tempo, durante o qual um astro faz a sua revolução.
Espaço de tempo, entre dois accessos successivos de febre intermittente.
Phrase, que é composta de mais de um membro e fórma sentido completo e independente.
Qualquer espaço de tempo.
Parte de uma fracção periódica, que se reproduz infinitamente.
Phrase musical, composta de diversos desenhos, agrupados em dois ou mais rythmos, formando sentido completo.
\section{Periodologia}
\begin{itemize}
\item {Grp. gram.:f.}
\end{itemize}
\begin{itemize}
\item {Utilização:Mús.}
\end{itemize}
\begin{itemize}
\item {Proveniência:(Do gr. \textunderscore periodos\textunderscore  + \textunderscore logos\textunderscore )}
\end{itemize}
Parte da sciência da composição musical, em que se trata da formação e ligação dos períodos.
\section{Periodontite}
\begin{itemize}
\item {Grp. gram.:f.}
\end{itemize}
\begin{itemize}
\item {Utilização:Med.}
\end{itemize}
\begin{itemize}
\item {Proveniência:(Do gr. \textunderscore peri\textunderscore  + \textunderscore odous\textunderscore , \textunderscore odontos\textunderscore )}
\end{itemize}
Inflammação da membrana que cérca o dente.
\section{Perioftalmia}
\begin{itemize}
\item {Grp. gram.:f.}
\end{itemize}
\begin{itemize}
\item {Proveniência:(De \textunderscore peri...\textunderscore  + \textunderscore ophthalmia\textunderscore )}
\end{itemize}
Inflamação no bordo das pálpebras.
\section{Periophthalmia}
\begin{itemize}
\item {Grp. gram.:f.}
\end{itemize}
\begin{itemize}
\item {Proveniência:(De \textunderscore peri...\textunderscore  + \textunderscore ophthalmia\textunderscore )}
\end{itemize}
Inflammação no bordo das pálpebras.
\section{Períoplo}
\begin{itemize}
\item {Grp. gram.:m.}
\end{itemize}
\begin{itemize}
\item {Utilização:Veter.}
\end{itemize}
\begin{itemize}
\item {Proveniência:(Do gr. \textunderscore peri\textunderscore  + \textunderscore hople\textunderscore )}
\end{itemize}
Lâmina córnea, que reveste o bôrdo superior do casco dos solípedes.
\section{Periórbita}
\begin{itemize}
\item {Grp. gram.:f.}
\end{itemize}
\begin{itemize}
\item {Utilização:Anat.}
\end{itemize}
\begin{itemize}
\item {Proveniência:(De \textunderscore peri...\textunderscore  + \textunderscore órbita\textunderscore )}
\end{itemize}
Periósteo, que reveste a cavidade orbitária.
\section{Periorthógono}
\begin{itemize}
\item {Grp. gram.:adj.}
\end{itemize}
\begin{itemize}
\item {Utilização:Miner.}
\end{itemize}
\begin{itemize}
\item {Proveniência:(De \textunderscore peri...\textunderscore  + \textunderscore orthógono\textunderscore )}
\end{itemize}
Diz-se de um prisma rhomboidal, que se converteu em prisma rectangular.
\section{Periortógono}
\begin{itemize}
\item {Grp. gram.:adj.}
\end{itemize}
\begin{itemize}
\item {Utilização:Miner.}
\end{itemize}
\begin{itemize}
\item {Proveniência:(De \textunderscore peri...\textunderscore  + \textunderscore ortógono\textunderscore )}
\end{itemize}
Diz-se de um prisma romboidal, que se converteu em prisma rectangular.
\section{Periósseo}
\begin{itemize}
\item {Grp. gram.:m.}
\end{itemize}
\begin{itemize}
\item {Utilização:Zool.}
\end{itemize}
O mesmo que \textunderscore periósteo\textunderscore . Cf. Max. Lemos, \textunderscore Zool.\textunderscore , 18.
\section{Periostal}
\begin{itemize}
\item {Grp. gram.:adj.}
\end{itemize}
Relativo ao periósteo.
\section{Periósteo}
\begin{itemize}
\item {Grp. gram.:m.}
\end{itemize}
\begin{itemize}
\item {Utilização:Anat.}
\end{itemize}
\begin{itemize}
\item {Proveniência:(Gr. \textunderscore periosteon\textunderscore )}
\end{itemize}
Membrana fibrosa, que reveste os ossos.
\section{Periosteófito}
\begin{itemize}
\item {Grp. gram.:m.}
\end{itemize}
\begin{itemize}
\item {Utilização:Med.}
\end{itemize}
\begin{itemize}
\item {Proveniência:(Do gr. \textunderscore peri\textunderscore  + \textunderscore osteon\textunderscore  + \textunderscore phuton\textunderscore )}
\end{itemize}
Formação óssea, partindo do periósteo.
\section{Periosteóphyto}
\begin{itemize}
\item {Grp. gram.:m.}
\end{itemize}
\begin{itemize}
\item {Utilização:Med.}
\end{itemize}
\begin{itemize}
\item {Proveniência:(Do gr. \textunderscore peri\textunderscore  + \textunderscore osteon\textunderscore  + \textunderscore phuton\textunderscore )}
\end{itemize}
Formação óssea, partindo do periósteo.
\section{Periosteotomia}
\begin{itemize}
\item {Grp. gram.:f.}
\end{itemize}
\begin{itemize}
\item {Utilização:Cir.}
\end{itemize}
\begin{itemize}
\item {Proveniência:(Do gr. \textunderscore peri\textunderscore  + \textunderscore osteon\textunderscore  + \textunderscore tome\textunderscore )}
\end{itemize}
Operação, que consiste em cortar parte de um periósteo, e separá-lo do tumor que elle cobre.
\section{Periostite}
\begin{itemize}
\item {Grp. gram.:f.}
\end{itemize}
Inflammação do periósteo.
\section{Periostógono}
\begin{itemize}
\item {Grp. gram.:adj.}
\end{itemize}
\begin{itemize}
\item {Utilização:Miner.}
\end{itemize}
Diz-se do crystal que, de prisma rhomboidal, passou a prisma rectangular.
\section{Periostose}
\begin{itemize}
\item {Grp. gram.:f.}
\end{itemize}
Inchação do periósteo.
\section{Perióstraco}
\begin{itemize}
\item {Grp. gram.:m.}
\end{itemize}
\begin{itemize}
\item {Utilização:Zool.}
\end{itemize}
\begin{itemize}
\item {Proveniência:(Do gr. \textunderscore peri\textunderscore  + \textunderscore ostrakon\textunderscore )}
\end{itemize}
Epiderme das conchas.
\section{Periovular}
\begin{itemize}
\item {Grp. gram.:adj.}
\end{itemize}
\begin{itemize}
\item {Utilização:Anat.}
\end{itemize}
\begin{itemize}
\item {Proveniência:(De \textunderscore peri...\textunderscore  + \textunderscore óvulo\textunderscore )}
\end{itemize}
Que cérca o óvulo.
\section{Periparoba}
\begin{itemize}
\item {Grp. gram.:m.}
\end{itemize}
Planta piperácea do Brasil.
\section{Peripateticamente}
\begin{itemize}
\item {Grp. gram.:adv.}
\end{itemize}
Á maneira dos peripatéticos; de modo peripatético.
\section{Peripatético}
\begin{itemize}
\item {Grp. gram.:adj.}
\end{itemize}
\begin{itemize}
\item {Grp. gram.:M.}
\end{itemize}
\begin{itemize}
\item {Proveniência:(Lat. \textunderscore peripatetici\textunderscore )}
\end{itemize}
Relativo á philosophia de Aristóteles.
Que segue a doutrina de Aristóteles.
Que se ensina, passeando.
Sectário de Aristóteles.
\section{Peripatetismo}
\begin{itemize}
\item {Grp. gram.:m.}
\end{itemize}
\begin{itemize}
\item {Proveniência:(De \textunderscore peripatético\textunderscore )}
\end{itemize}
Philosophia de Aristóteles.
\section{Perípato}
\begin{itemize}
\item {Grp. gram.:m.}
\end{itemize}
\begin{itemize}
\item {Proveniência:(Do gr. \textunderscore peripatein\textunderscore )}
\end{itemize}
O systema de Aristóteles; o peripatetismo.
\section{Peripécia}
\begin{itemize}
\item {Grp. gram.:f.}
\end{itemize}
\begin{itemize}
\item {Utilização:Fam.}
\end{itemize}
\begin{itemize}
\item {Proveniência:(Gr. \textunderscore peripeteia\textunderscore )}
\end{itemize}
Successo num poema, numa peça theatral, etc., que muda a face das coisas.
Successo imprevisto; incidente.--A pronúncia exacta sería \textunderscore peripecía\textunderscore , mas não se usa.
\section{Peripetalía}
\begin{itemize}
\item {Grp. gram.:f.}
\end{itemize}
\begin{itemize}
\item {Utilização:Bot.}
\end{itemize}
Estado ou disposição de peripétalo.
\section{Peripétalo}
\begin{itemize}
\item {Grp. gram.:adj.}
\end{itemize}
\begin{itemize}
\item {Utilização:Bot.}
\end{itemize}
\begin{itemize}
\item {Proveniência:(De \textunderscore peri...\textunderscore  + \textunderscore pétala\textunderscore )}
\end{itemize}
Que rodeia as pétalas ou as corollas das flôres.
\section{Peripheria}
\begin{itemize}
\item {Grp. gram.:f.}
\end{itemize}
\begin{itemize}
\item {Proveniência:(Lat. \textunderscore peripheria\textunderscore )}
\end{itemize}
Contôrno de uma figura curvilínea.
Superfície de um sólido.
Circunferência.
\section{Periphérico}
\begin{itemize}
\item {Grp. gram.:adj.}
\end{itemize}
\begin{itemize}
\item {Utilização:Bot.}
\end{itemize}
Relativo a peripheria.
Diz-se do perisperma, quando envolve e occulta o embryão.
E diz-se do embryão, quando cérca o perisperma, no todo ou em parte.
\section{Periphorantho}
\begin{itemize}
\item {Grp. gram.:m.}
\end{itemize}
\begin{itemize}
\item {Utilização:Bot.}
\end{itemize}
\begin{itemize}
\item {Proveniência:(De \textunderscore peri...\textunderscore  + \textunderscore phorantho\textunderscore )}
\end{itemize}
Conjunto das brácteas que, nas synanthéreas, cercam a reunião das flôres.
\section{Períphoro}
\begin{itemize}
\item {Grp. gram.:m.}
\end{itemize}
\begin{itemize}
\item {Utilização:Bot.}
\end{itemize}
\begin{itemize}
\item {Proveniência:(Do gr. \textunderscore peri\textunderscore  + \textunderscore phoros\textunderscore )}
\end{itemize}
Órgão, que sustenta o ovário de certas plantas e dá ligamento ás pétalas e aos estames.
\section{Períphrase}
\begin{itemize}
\item {Grp. gram.:f.}
\end{itemize}
\begin{itemize}
\item {Proveniência:(Lat. \textunderscore periphrasis\textunderscore )}
\end{itemize}
Conjunto de palavras ou phrases, empregadas em vêz de uma só palavra ou termo próprio.
Circunlóquio.
\section{Periphrástico}
\begin{itemize}
\item {Grp. gram.:adj.}
\end{itemize}
\begin{itemize}
\item {Proveniência:(Gr. \textunderscore periphrastikos\textunderscore )}
\end{itemize}
Relativo á períphrase.
\section{Peripiema}
\begin{itemize}
\item {Grp. gram.:m.}
\end{itemize}
\begin{itemize}
\item {Utilização:Med.}
\end{itemize}
\begin{itemize}
\item {Proveniência:(Gr. \textunderscore peripuema\textunderscore )}
\end{itemize}
Supuração em volta de um órgão.
\section{Peripitinga}
\begin{itemize}
\item {Grp. gram.:f.}
\end{itemize}
\begin{itemize}
\item {Utilização:Bras}
\end{itemize}
Peixe fluvial.
\section{Peripleroma}
\begin{itemize}
\item {Grp. gram.:m.}
\end{itemize}
\begin{itemize}
\item {Utilização:Rhet.}
\end{itemize}
Addição de uma palavra, que é inútil ao sentido da phrase.
\section{Périplo}
\begin{itemize}
\item {Grp. gram.:m.}
\end{itemize}
\begin{itemize}
\item {Proveniência:(Lat. \textunderscore periplus\textunderscore )}
\end{itemize}
Navegação á volta de um mar ou pelas costas de um país.
Relação de uma viagem dêsse gênero.
\section{Periploca}
\begin{itemize}
\item {Grp. gram.:f.}
\end{itemize}
\begin{itemize}
\item {Proveniência:(Do gr. \textunderscore periploke\textunderscore )}
\end{itemize}
Gênero de plantas asclepiadáceas.
\section{Peripneumonia}
\begin{itemize}
\item {Grp. gram.:f.}
\end{itemize}
\begin{itemize}
\item {Proveniência:(Lat. \textunderscore peripneumonia\textunderscore )}
\end{itemize}
Inflamação do pulmão.
\section{Peripneumónico}
\begin{itemize}
\item {Grp. gram.:adj.}
\end{itemize}
\begin{itemize}
\item {Grp. gram.:M.  e  adj.}
\end{itemize}
Relativo á peripneumonia.
Aquelle que soffre peripneumonia.
\section{Perípode}
\begin{itemize}
\item {Grp. gram.:m.}
\end{itemize}
\begin{itemize}
\item {Utilização:Bot.}
\end{itemize}
\begin{itemize}
\item {Proveniência:(Do gr. \textunderscore peri\textunderscore  + \textunderscore pous\textunderscore , \textunderscore podos\textunderscore )}
\end{itemize}
Invólucro dos musgos.
\section{Periproctite}
\begin{itemize}
\item {Grp. gram.:m.}
\end{itemize}
\begin{itemize}
\item {Utilização:Med.}
\end{itemize}
\begin{itemize}
\item {Proveniência:(Do gr. \textunderscore peri\textunderscore  + \textunderscore proktos\textunderscore )}
\end{itemize}
Inflammação do intestino recto.
\section{Peripterado}
\begin{itemize}
\item {Grp. gram.:adj.}
\end{itemize}
\begin{itemize}
\item {Utilização:Bot.}
\end{itemize}
Diz-se do fruto ou do grão, quando cercado por uma expansão membranosa á maneira de asa.
(Cp. \textunderscore períptero\textunderscore )
\section{Períptero}
\begin{itemize}
\item {Grp. gram.:m.}
\end{itemize}
\begin{itemize}
\item {Proveniência:(Gr. \textunderscore peripterun\textunderscore )}
\end{itemize}
Edifício, que em toda a volta tem columnas insoladas.
\section{Peripyema}
\begin{itemize}
\item {Grp. gram.:m.}
\end{itemize}
\begin{itemize}
\item {Utilização:Med.}
\end{itemize}
\begin{itemize}
\item {Proveniência:(Gr. \textunderscore peripuema\textunderscore )}
\end{itemize}
Suppuração em volta de um órgão.
\section{Periquita}
\begin{itemize}
\item {Grp. gram.:f.}
\end{itemize}
Espécie de videira portuguesa; o fruto della.
\section{Periquiteira}
\begin{itemize}
\item {Grp. gram.:f.}
\end{itemize}
O mesmo que \textunderscore gurindiba\textunderscore .
\section{Periquito}
\begin{itemize}
\item {Grp. gram.:m.}
\end{itemize}
\begin{itemize}
\item {Proveniência:(Do it. \textunderscore perrochetto\textunderscore )}
\end{itemize}
Pequena ave, semelhante ao papagaio.
\section{Periquito}
\begin{itemize}
\item {Grp. gram.:m.}
\end{itemize}
\begin{itemize}
\item {Utilização:Prov.}
\end{itemize}
\begin{itemize}
\item {Utilização:trasm.}
\end{itemize}
O mesmo que \textunderscore paio\textunderscore .
\section{Periscélide}
\begin{itemize}
\item {Grp. gram.:f.}
\end{itemize}
\begin{itemize}
\item {Proveniência:(Lat. \textunderscore periscelis\textunderscore )}
\end{itemize}
Argola de oiro ou prata, com que as mulheres romanas cingiam e adornavam as pernas por cima do artelho.
\section{Periscélio}
\begin{itemize}
\item {Grp. gram.:m.}
\end{itemize}
\begin{itemize}
\item {Proveniência:(Lat. \textunderscore periscelium\textunderscore )}
\end{itemize}
O mesmo que \textunderscore periscélide\textunderscore .
\section{Períscios}
\begin{itemize}
\item {Grp. gram.:m. pl.}
\end{itemize}
\begin{itemize}
\item {Utilização:Geogr.}
\end{itemize}
\begin{itemize}
\item {Proveniência:(Gr. \textunderscore periskíos\textunderscore )}
\end{itemize}
Habitantes das zonas glaciaes, a sombra dos quaes, em um só dia, se projecta successivamente para todos os lados do horizonte.
\section{Periscópico}
\begin{itemize}
\item {Grp. gram.:adj.}
\end{itemize}
\begin{itemize}
\item {Utilização:Med.}
\end{itemize}
Relativo ao periscópio.
Diz-se de um vidro, em fórma de menisco, empregado em evitar a desigualdade e as confusões da vista.
\section{Periscópio}
\begin{itemize}
\item {Grp. gram.:m.}
\end{itemize}
\begin{itemize}
\item {Proveniência:(Do gr. \textunderscore peri\textunderscore  + \textunderscore skopein\textunderscore )}
\end{itemize}
O mesmo que \textunderscore caleidoscópio\textunderscore  ou, antes, \textunderscore calidoscópio\textunderscore .
\section{Perisperma}
\begin{itemize}
\item {Grp. gram.:m.}
\end{itemize}
\begin{itemize}
\item {Utilização:Bot.}
\end{itemize}
\begin{itemize}
\item {Proveniência:(Do gr. \textunderscore peri\textunderscore  + \textunderscore sperma\textunderscore )}
\end{itemize}
Invólucro da semente das plantas.
\section{Perispermado}
\begin{itemize}
\item {Grp. gram.:adj.}
\end{itemize}
\begin{itemize}
\item {Utilização:Bot.}
\end{itemize}
Diz-se dos grãos ou amêndoas, que têm um perisperma.
\section{Perispérmico}
\begin{itemize}
\item {Grp. gram.:adj.}
\end{itemize}
\begin{itemize}
\item {Utilização:Bot.}
\end{itemize}
Que tem albúmen ou perisperma.
\section{Perispiritual}
\begin{itemize}
\item {Grp. gram.:adj.}
\end{itemize}
O mesmo que \textunderscore perispirítico\textunderscore .
\section{Perispirítico}
\begin{itemize}
\item {Grp. gram.:adj.}
\end{itemize}
\begin{itemize}
\item {Utilização:Espir.}
\end{itemize}
Relativo a perispírito.
\section{Perispírito}
\begin{itemize}
\item {Grp. gram.:m.}
\end{itemize}
\begin{itemize}
\item {Utilização:Espir.}
\end{itemize}
\begin{itemize}
\item {Proveniência:(De \textunderscore peri...\textunderscore  + \textunderscore espírito\textunderscore )}
\end{itemize}
Organismo homogêneo que, segundo os espiritistas, desempenha todas as funcções da vida psýchica ou da vida separada do corpo, funcções que, na vida terrena, correspondem a outros tantos sentidos. Cf. M. Velho, \textunderscore Man. do Espirita\textunderscore .
\section{Perisplenite}
\begin{itemize}
\item {Grp. gram.:f.}
\end{itemize}
\begin{itemize}
\item {Utilização:Med.}
\end{itemize}
\begin{itemize}
\item {Proveniência:(Do gr. \textunderscore peri\textunderscore  + \textunderscore splen\textunderscore )}
\end{itemize}
Inflammação do peritonéu que cérca o baço.
\section{Perispómeno}
\begin{itemize}
\item {Grp. gram.:adj.}
\end{itemize}
\begin{itemize}
\item {Utilização:Gram.}
\end{itemize}
\begin{itemize}
\item {Proveniência:(Gr. \textunderscore perispomenos\textunderscore )}
\end{itemize}
Diz-se das palavras monosyllábicas, cuja vogal tem modulação fechada.
\section{Perisporango}
\begin{itemize}
\item {Grp. gram.:m.}
\end{itemize}
\begin{itemize}
\item {Utilização:Bot.}
\end{itemize}
\begin{itemize}
\item {Proveniência:(De \textunderscore peri...\textunderscore  + \textunderscore esporango\textunderscore )}
\end{itemize}
Membrana, que envolve os corpúsculos, reproductores dos fêtos.
\section{Perissístole}
\begin{itemize}
\item {Grp. gram.:f.}
\end{itemize}
\begin{itemize}
\item {Proveniência:(De \textunderscore peri...\textunderscore  + \textunderscore sístole\textunderscore )}
\end{itemize}
Intervalo entre a sístole e a diástole.
\section{Perissodáctylos}
\begin{itemize}
\item {Grp. gram.:m. pl.}
\end{itemize}
\begin{itemize}
\item {Utilização:Zool.}
\end{itemize}
\begin{itemize}
\item {Proveniência:(Do gr. \textunderscore perissos\textunderscore  + \textunderscore daktulos\textunderscore )}
\end{itemize}
Ordem da classe dos mammíferos, que comprehende os animaes, cujos dedos são em número ímpar, como os tridigitados e os solípedes.
\section{Perissologia}
\begin{itemize}
\item {Grp. gram.:f.}
\end{itemize}
\begin{itemize}
\item {Utilização:Rhet.}
\end{itemize}
\begin{itemize}
\item {Proveniência:(Lat. \textunderscore perissologia\textunderscore )}
\end{itemize}
Repetição, por diversos termos, de um pensamento já expresso.
Espécie de pleonasmo.
\section{Perissológico}
\begin{itemize}
\item {Grp. gram.:adj.}
\end{itemize}
Relativo á perissologia.
\section{Perissólogo}
\begin{itemize}
\item {Grp. gram.:m.}
\end{itemize}
\begin{itemize}
\item {Proveniência:(Gr. \textunderscore perissos\textunderscore  + \textunderscore logos\textunderscore )}
\end{itemize}
Aquelle que fala muito ou que se exprime por circunlóquios.
\section{Peristáchio}
\begin{itemize}
\item {fónica:qui}
\end{itemize}
\begin{itemize}
\item {Grp. gram.:m.}
\end{itemize}
\begin{itemize}
\item {Utilização:Bot.}
\end{itemize}
\begin{itemize}
\item {Proveniência:(Do gr. \textunderscore peri\textunderscore  + \textunderscore stakhus\textunderscore )}
\end{itemize}
Invólucro exterior das flôres das gramíneas.
\section{Peristáltico}
\begin{itemize}
\item {Grp. gram.:adj.}
\end{itemize}
\begin{itemize}
\item {Utilização:Med.}
\end{itemize}
\begin{itemize}
\item {Proveniência:(Gr. \textunderscore peristaltikos\textunderscore )}
\end{itemize}
Diz-se da contracção successiva das fibras circulares da túnica musculosa do estômago e do intestino, quando essa contracção se realiza de cima para baixo.
\section{Peristamínia}
\begin{itemize}
\item {Grp. gram.:f.}
\end{itemize}
\begin{itemize}
\item {Utilização:Bot.}
\end{itemize}
\begin{itemize}
\item {Proveniência:(De \textunderscore peri...\textunderscore  + \textunderscore estame\textunderscore )}
\end{itemize}
Classe de plantas, que, no systema de Jussieu, abrange as dicotyledóneas apétalas de estames perýginos.
\section{Peristáquio}
\begin{itemize}
\item {fónica:qui}
\end{itemize}
\begin{itemize}
\item {Grp. gram.:m.}
\end{itemize}
\begin{itemize}
\item {Utilização:Bot.}
\end{itemize}
\begin{itemize}
\item {Proveniência:(Do gr. \textunderscore peri\textunderscore  + \textunderscore stakhus\textunderscore )}
\end{itemize}
Invólucro exterior das flôres das gramíneas.
\section{Perístase}
\begin{itemize}
\item {Grp. gram.:m.}
\end{itemize}
\begin{itemize}
\item {Proveniência:(Lat. \textunderscore peristasis\textunderscore )}
\end{itemize}
Assumpto completo de um discurso, com todas as suas minuciosidades.
\section{Perísteros}
\begin{itemize}
\item {Grp. gram.:m. pl.}
\end{itemize}
\begin{itemize}
\item {Proveniência:(Do gr. \textunderscore peristera\textunderscore , pombo)}
\end{itemize}
Tríbo de gallináceas, que comprehende pombas e rôlas.
\section{Peristéthio}
\begin{itemize}
\item {Grp. gram.:m.}
\end{itemize}
\begin{itemize}
\item {Utilização:Zool.}
\end{itemize}
\begin{itemize}
\item {Proveniência:(Do gr. \textunderscore peri\textunderscore  + \textunderscore stethos\textunderscore , esterno)}
\end{itemize}
A parte do peito dos insectos, que está situada entre os braços e as patas médias.
\section{Peristétio}
\begin{itemize}
\item {Grp. gram.:m.}
\end{itemize}
\begin{itemize}
\item {Utilização:Zool.}
\end{itemize}
\begin{itemize}
\item {Proveniência:(Do gr. \textunderscore peri\textunderscore  + \textunderscore stethos\textunderscore , esterno)}
\end{itemize}
A parte do peito dos insectos, que está situada entre os braços e as patas médias.
\section{Peristílico}
\begin{itemize}
\item {Grp. gram.:adj.}
\end{itemize}
\begin{itemize}
\item {Utilização:Bot.}
\end{itemize}
\begin{itemize}
\item {Proveniência:(De \textunderscore perístilo\textunderscore )}
\end{itemize}
Diz-se da inserção, quando o ovário é completamente inferior e os estames são inseridos entre êle e o cálice.
\section{Peristilo}
\begin{itemize}
\item {Grp. gram.:m.}
\end{itemize}
\begin{itemize}
\item {Proveniência:(Lat. \textunderscore peristylum\textunderscore )}
\end{itemize}
Lugar interiormente cercado de columnas, como os claustros dos conventos.
Conjunto de columnas insuladas, na fronteira de um edifício.
Aquillo que serve de introducção; o que antecede.
\section{Perístole}
\begin{itemize}
\item {Grp. gram.:f.}
\end{itemize}
\begin{itemize}
\item {Proveniência:(Gr. \textunderscore peristole\textunderscore )}
\end{itemize}
Acção peristáltica do canal intestinal.
\section{Perístoma}
\begin{itemize}
\item {Grp. gram.:m.}
\end{itemize}
\begin{itemize}
\item {Utilização:Bot.}
\end{itemize}
\begin{itemize}
\item {Utilização:Zool.}
\end{itemize}
\begin{itemize}
\item {Proveniência:(Do gr. \textunderscore peri\textunderscore  + \textunderscore stoma\textunderscore )}
\end{itemize}
Guarnição filamentosa, á volta do orifício da urna dos musgos.
Cavidade da cabeça da mosca, onde se recolhe a tromba.
Espessura de uma concha univalve, na direcção da sua abertura.
\section{Peristomado}
\begin{itemize}
\item {Grp. gram.:adj.}
\end{itemize}
\begin{itemize}
\item {Utilização:Bot.}
\end{itemize}
Que tem perístoma.
\section{Peristómico}
\begin{itemize}
\item {Grp. gram.:adj.}
\end{itemize}
\begin{itemize}
\item {Utilização:Bot.}
\end{itemize}
\begin{itemize}
\item {Proveniência:(De \textunderscore perístoma\textunderscore )}
\end{itemize}
Que está em relação com o orifício do calix.
\section{Peristómio}
\begin{itemize}
\item {Grp. gram.:m.}
\end{itemize}
(V.perístoma). Cf. R. Galvão, \textunderscore Vocab.\textunderscore 
\section{Peristýlico}
\begin{itemize}
\item {Grp. gram.:adj.}
\end{itemize}
\begin{itemize}
\item {Utilização:Bot.}
\end{itemize}
\begin{itemize}
\item {Proveniência:(De \textunderscore perístylo\textunderscore )}
\end{itemize}
Diz-se da inserção, quando o ovário é completamente inferior e os estames são inseridos entre êlle e o cálice.
\section{Peristylo}
\begin{itemize}
\item {Grp. gram.:m.}
\end{itemize}
\begin{itemize}
\item {Proveniência:(Lat. \textunderscore peristylum\textunderscore )}
\end{itemize}
Lugar interiormente cercado de columnas, como os claustros dos conventos.
Conjunto de columnas insuladas, na fronteira de um edifício.
Aquillo que serve de introducção; o que antecede.
\section{Perisýstole}
\begin{itemize}
\item {fónica:sis}
\end{itemize}
\begin{itemize}
\item {Grp. gram.:f.}
\end{itemize}
\begin{itemize}
\item {Proveniência:(De \textunderscore peri...\textunderscore  + \textunderscore sýstole\textunderscore )}
\end{itemize}
Intervallo entre a sýstole e a diástole.
\section{Peritécio}
\begin{itemize}
\item {Grp. gram.:m.}
\end{itemize}
\begin{itemize}
\item {Utilização:Bot.}
\end{itemize}
\begin{itemize}
\item {Proveniência:(Do gr. \textunderscore peri\textunderscore  + \textunderscore theke\textunderscore )}
\end{itemize}
O mesmo que \textunderscore pericarpo\textunderscore .
\section{Perithécio}
\begin{itemize}
\item {Grp. gram.:m.}
\end{itemize}
\begin{itemize}
\item {Utilização:Bot.}
\end{itemize}
\begin{itemize}
\item {Proveniência:(Do gr. \textunderscore peri\textunderscore  + \textunderscore theke\textunderscore )}
\end{itemize}
O mesmo que \textunderscore pericarpo\textunderscore .
\section{Peritiflite}
\begin{itemize}
\item {Grp. gram.:f.}
\end{itemize}
\begin{itemize}
\item {Utilização:Med.}
\end{itemize}
\begin{itemize}
\item {Proveniência:(Do gr. \textunderscore peri\textunderscore  + \textunderscore tuphlos\textunderscore )}
\end{itemize}
Inflamação do tecido celular que envolve o ceco.
\section{Perito}
\begin{itemize}
\item {Grp. gram.:adj.}
\end{itemize}
\begin{itemize}
\item {Grp. gram.:M.}
\end{itemize}
\begin{itemize}
\item {Proveniência:(Lat. \textunderscore peritus\textunderscore )}
\end{itemize}
Experimentado, sabedor.
Hábil; douto; prático.
Aquelle que é sabedor ou prático, em determinados assumptos.
Avaliador, nomeado judicialmente.
Aquelle que é judicialmente nomeado para um exame ou vistoria; louvado.
\section{Peritoneal}
\begin{itemize}
\item {Grp. gram.:adj.}
\end{itemize}
Relativo ao peritoneu.
\section{Peritoneu}
\begin{itemize}
\item {Grp. gram.:m.}
\end{itemize}
\begin{itemize}
\item {Utilização:Anat.}
\end{itemize}
\begin{itemize}
\item {Proveniência:(Gr. \textunderscore peritonaion\textunderscore )}
\end{itemize}
Membrana serosa, que reveste interiormente o ventre.
\section{Peritónio}
\begin{itemize}
\item {Grp. gram.:m.}
\end{itemize}
\begin{itemize}
\item {Proveniência:(Gr. \textunderscore peritonion\textunderscore )}
\end{itemize}
(V.peritoneu)
\section{Peritonite}
\begin{itemize}
\item {Grp. gram.:f.}
\end{itemize}
\begin{itemize}
\item {Proveniência:(De \textunderscore peritónio\textunderscore )}
\end{itemize}
Inflammação do peritoneu.
\section{Perítropo}
\begin{itemize}
\item {Grp. gram.:adj.}
\end{itemize}
\begin{itemize}
\item {Utilização:Bot.}
\end{itemize}
\begin{itemize}
\item {Proveniência:(Do gr. \textunderscore peri\textunderscore  + \textunderscore trepein\textunderscore )}
\end{itemize}
Que se dirige do eixo do fruto para os lados do pericarpo.
\section{Perituro}
\begin{itemize}
\item {Grp. gram.:adj.}
\end{itemize}
\begin{itemize}
\item {Proveniência:(Lat. \textunderscore periturus\textunderscore )}
\end{itemize}
Que há de perecer ou acabar.
\section{Perityphlite}
\begin{itemize}
\item {Grp. gram.:f.}
\end{itemize}
\begin{itemize}
\item {Utilização:Med.}
\end{itemize}
\begin{itemize}
\item {Proveniência:(Do gr. \textunderscore peri\textunderscore  + \textunderscore tuphlos\textunderscore )}
\end{itemize}
Inflammação do tecido cellular que envolve o ceco.
\section{Peri-uterino}
\begin{itemize}
\item {Grp. gram.:adj.}
\end{itemize}
\begin{itemize}
\item {Utilização:Anat.}
\end{itemize}
Situado em roda do útero.
\section{Perjuramente}
\begin{itemize}
\item {Grp. gram.:adv.}
\end{itemize}
\begin{itemize}
\item {Proveniência:(De \textunderscore perjuro\textunderscore )}
\end{itemize}
Com perjúrio.
\section{Perjurar}
\begin{itemize}
\item {Grp. gram.:v. t.}
\end{itemize}
\begin{itemize}
\item {Grp. gram.:V. i.}
\end{itemize}
\begin{itemize}
\item {Proveniência:(Lat. \textunderscore perjurare\textunderscore )}
\end{itemize}
Abjurar: \textunderscore perjurar uma crença\textunderscore .
Jurar falso; commeter perjúrio.
Faltar a promessas juradas.
\section{Perjúrio}
\begin{itemize}
\item {Grp. gram.:m.}
\end{itemize}
\begin{itemize}
\item {Proveniência:(Lat. \textunderscore perjurium\textunderscore )}
\end{itemize}
Acto de perjurar; juramento falso.
\section{Perjuro}
\begin{itemize}
\item {Grp. gram.:m.  e  adj.}
\end{itemize}
\begin{itemize}
\item {Proveniência:(Lat. \textunderscore perjurus\textunderscore )}
\end{itemize}
Aquelle que jura falso, ou que falta á fé jurada.
\section{Perkinismo}
\begin{itemize}
\item {Grp. gram.:m.}
\end{itemize}
\begin{itemize}
\item {Proveniência:(De \textunderscore Perkins\textunderscore , n. p.)}
\end{itemize}
Systema curativo do doutor Perkins, systema que consistia em passar na parte doente a ponta de duas agulhas de metal differente.
\section{Perla}
\begin{itemize}
\item {Grp. gram.:f.}
\end{itemize}
(Fórma divergente e antiga de \textunderscore pérola\textunderscore . Cp. cast. \textunderscore perla\textunderscore )
\section{Perlar}
\begin{itemize}
\item {Grp. gram.:v. t.}
\end{itemize}
\begin{itemize}
\item {Proveniência:(De \textunderscore perla\textunderscore )}
\end{itemize}
Dar fórma de pérola a; perolizar.
Tornar como que revestido de pérolas:«\textunderscore ...sacchareto de alga perlada\textunderscore ». \textunderscore Diár.-do-Govêrno\textunderscore , 128, de 1882.
\section{Perlasso}
\begin{itemize}
\item {Grp. gram.:m.}
\end{itemize}
\begin{itemize}
\item {Proveniência:(Do al. \textunderscore perlasche\textunderscore )}
\end{itemize}
Nome que, no commércio, se dá ás potassas mais puras e mais brancas.
\section{Perlavar}
\begin{itemize}
\item {Grp. gram.:v. t.}
\end{itemize}
\begin{itemize}
\item {Proveniência:(Lat. \textunderscore perlavare\textunderscore )}
\end{itemize}
Lavar inteiramente; purificar. Cf. Castilho, \textunderscore Fastos\textunderscore , III, 51.
\section{Perlenga}
\begin{itemize}
\item {Grp. gram.:f.}
\end{itemize}
\begin{itemize}
\item {Utilização:Pop.}
\end{itemize}
O mesmo que \textunderscore parlenga\textunderscore .
\section{Perlengada}
\begin{itemize}
\item {Grp. gram.:f.}
\end{itemize}
Grande perlenga.
\section{Perleúdo}
\begin{itemize}
\item {Grp. gram.:adj.}
\end{itemize}
\begin{itemize}
\item {Utilização:Deprec.}
\end{itemize}
\begin{itemize}
\item {Proveniência:(De \textunderscore per...\textunderscore  + \textunderscore leúdo\textunderscore , part. ant. de \textunderscore ler\textunderscore )}
\end{itemize}
Muito lido, muito sabedor. Cf. Pacheco da Silva, \textunderscore Promptuário\textunderscore , 7.
\section{Perlífero}
\begin{itemize}
\item {Grp. gram.:adj.}
\end{itemize}
(V.perolífero)
\section{Perlincafuzes}
\begin{itemize}
\item {Grp. gram.:m. pl.}
\end{itemize}
Maneira de falar, figurada ou enigmática:«\textunderscore lá com perlincafuzes não me sei entender\textunderscore ». Castilho, \textunderscore Sabichonas\textunderscore , 203.
\section{Perliquiteta}
\begin{itemize}
\item {Grp. gram.:adj.}
\end{itemize}
O mesmo que \textunderscore perliquiteto\textunderscore . Cf. Camillo, \textunderscore Cancion. Al.\textunderscore , 131.
\section{Perliquitete}
\begin{itemize}
\item {Grp. gram.:adj.}
\end{itemize}
Espevitado, presumido, pronóstico:«\textunderscore ...tarelos perliquitetes...\textunderscore »Filinto, XI, 147.
\section{Perliquiteto}
\begin{itemize}
\item {Grp. gram.:adj.}
\end{itemize}
\begin{itemize}
\item {Utilização:Pop.}
\end{itemize}
Espevitado, presumido, pronóstico:«\textunderscore ...tarelos perliquitetos...\textunderscore »Filinto, XI, 147.
\section{Perlítico}
\begin{itemize}
\item {Grp. gram.:adj.}
\end{itemize}
Relativo ao perlito.
\section{Perlito}
\begin{itemize}
\item {Grp. gram.:m.}
\end{itemize}
\begin{itemize}
\item {Proveniência:(De \textunderscore perla\textunderscore )}
\end{itemize}
Substância mineral, que se apresenta sob uma fórma semelhante á da pérola.
\section{Periquécio}
\begin{itemize}
\item {Grp. gram.:m.}
\end{itemize}
\begin{itemize}
\item {Utilização:Bot.}
\end{itemize}
\begin{itemize}
\item {Proveniência:(Do gr. \textunderscore peri\textunderscore  + \textunderscore khaita\textunderscore )}
\end{itemize}
Reunião de pequenas fôlhas ou bractéolas, na base do pedicelo que sustenta o urnário dos musgos.
\section{Perlonga}
\begin{itemize}
\item {Grp. gram.:f.}
\end{itemize}
\begin{itemize}
\item {Utilização:Des.}
\end{itemize}
Delonga; acto de perlongar.
Demora capciosa ou fraudulenta. Cf. \textunderscore Eufrosina\textunderscore , act. I, sc. 1.^a.
\section{Perlongar}
\begin{itemize}
\item {Grp. gram.:v. t.}
\end{itemize}
\begin{itemize}
\item {Utilização:Ant.}
\end{itemize}
\begin{itemize}
\item {Proveniência:(De \textunderscore per...\textunderscore  + \textunderscore longo\textunderscore )}
\end{itemize}
Ir ao longo de; costear.
Demorar, adiar, procrastinar.
\section{Perlongo}
\begin{itemize}
\item {Grp. gram.:m.}
\end{itemize}
\begin{itemize}
\item {Utilização:Bras. do Rio}
\end{itemize}
\begin{itemize}
\item {Proveniência:(De \textunderscore per\textunderscore  + \textunderscore longo\textunderscore )}
\end{itemize}
Telhado, de um e outro lado da cumeeira.
\section{Perluí}
\begin{itemize}
\item {Grp. gram.:m.}
\end{itemize}
\begin{itemize}
\item {Utilização:Prov.}
\end{itemize}
O mesmo que \textunderscore alcaravão\textunderscore .
\section{Perluiz}
\begin{itemize}
\item {Grp. gram.:m.}
\end{itemize}
\begin{itemize}
\item {Utilização:Prov.}
\end{itemize}
\begin{itemize}
\item {Utilização:trasm.}
\end{itemize}
O mesmo que \textunderscore alcaravão\textunderscore .
\section{Perlustrar}
\begin{itemize}
\item {Grp. gram.:v. t.}
\end{itemize}
\begin{itemize}
\item {Proveniência:(Lat. \textunderscore perlustrare\textunderscore )}
\end{itemize}
Percorrer com a vista; observar com diligência.
\section{Perluxidade}
\begin{itemize}
\item {fónica:csi}
\end{itemize}
\begin{itemize}
\item {Grp. gram.:f.}
\end{itemize}
Qualidade de perluxo^1. Cf. Filinto, III, 205.
\section{Perluxo}
\begin{itemize}
\item {Grp. gram.:adj.}
\end{itemize}
\begin{itemize}
\item {Utilização:Pop.}
\end{itemize}
(V.prolixo)
\section{Perluxo}
\begin{itemize}
\item {Grp. gram.:adj.}
\end{itemize}
\begin{itemize}
\item {Utilização:Prov.}
\end{itemize}
Muito luxuoso e presumido; que no trajar apparenta mais do que é.
\section{Permanecente}
\begin{itemize}
\item {Grp. gram.:adj.}
\end{itemize}
Que permanece; estável; duradoiro. Cf. Camillo, \textunderscore Noites de Insómn.\textunderscore , X, 23.
\section{Permanecer}
\begin{itemize}
\item {Grp. gram.:v. i.}
\end{itemize}
\begin{itemize}
\item {Proveniência:(Lat. hypoth. \textunderscore permanescere\textunderscore )}
\end{itemize}
Perseverar; insistir; ficar.
Continuar a existir.
Demorar-se em alguma parte.
\section{Permanência}
\begin{itemize}
\item {Grp. gram.:f.}
\end{itemize}
Acto do permanecer.
Estado do que é permanente.
Constância; perseverança.
\section{Permanente}
\begin{itemize}
\item {Grp. gram.:adj.}
\end{itemize}
\begin{itemize}
\item {Proveniência:(Lat. \textunderscore permanens\textunderscore )}
\end{itemize}
Que permanece; duradoiro; contínuo; ininterrupto.
\section{Permanentemente}
\begin{itemize}
\item {Grp. gram.:adv.}
\end{itemize}
De modo permanente; constantemente; com perseverança.
\section{Permeabilidade}
\begin{itemize}
\item {Grp. gram.:f.}
\end{itemize}
Qualidade do que é permeável.
\section{Permear}
\begin{itemize}
\item {Grp. gram.:v. t.}
\end{itemize}
\begin{itemize}
\item {Grp. gram.:V. i.}
\end{itemize}
Fazer passar pelo meio; entremear.
Atravessar.
Estar de permeio.
Sobrevir.
\section{Permeável}
\begin{itemize}
\item {Grp. gram.:adj.}
\end{itemize}
\begin{itemize}
\item {Proveniência:(De \textunderscore permear\textunderscore )}
\end{itemize}
Susceptível de sêr repassado ou traspassado.
\section{Permedida}
\begin{itemize}
\item {Grp. gram.:f.}
\end{itemize}
O mesmo que \textunderscore permediva\textunderscore .
\section{Permediva}
\begin{itemize}
\item {Grp. gram.:f.}
\end{itemize}
\begin{itemize}
\item {Utilização:Ant.}
\end{itemize}
O primeiro sável ou lampreia, que em cada anno o pescador tirava da água.
(Provavelmente, alter. de \textunderscore primitiva\textunderscore )
\section{Permeio}
\begin{itemize}
\item {Grp. gram.:adv.}
\end{itemize}
\begin{itemize}
\item {Grp. gram.:Loc. adv.}
\end{itemize}
\begin{itemize}
\item {Proveniência:(De \textunderscore per\textunderscore  + \textunderscore meio\textunderscore )}
\end{itemize}
No meio:«\textunderscore deu-me uma cadeira e, com o balcão permeio, falou-me...\textunderscore »M. Assis, \textunderscore B. Cubas\textunderscore .
\textunderscore De permeio\textunderscore , através, dentro, misturadaraente.
\section{Permiano}
\begin{itemize}
\item {Grp. gram.:m.  e  adj.}
\end{itemize}
\begin{itemize}
\item {Utilização:Geol.}
\end{itemize}
\begin{itemize}
\item {Proveniência:(De \textunderscore Perm\textunderscore , n. p.)}
\end{itemize}
Diz-se do um dos terrenos da transição da época primitiva para a secundária, e que é o calcário magnésico dos geólogos ingleses.
Cp. \textunderscore permo-carbónico\textunderscore .
Língua uralo-altaica do grupo ugro-finlandês.
\section{Pérmico}
\begin{itemize}
\item {Grp. gram.:adj.}
\end{itemize}
O mesmo que \textunderscore permiano\textunderscore .
\section{Permissão}
\begin{itemize}
\item {Grp. gram.:f.}
\end{itemize}
\begin{itemize}
\item {Proveniência:(Lat. \textunderscore permissio\textunderscore )}
\end{itemize}
Acto de permittir; licença, consentimento.
Figura oratória, em que se deixa aos ouvintes ou adversários a decisão de alguma coisa.
\section{Permissivamente}
\begin{itemize}
\item {Grp. gram.:adv.}
\end{itemize}
De modo permissivo.
\section{Permissível}
\begin{itemize}
\item {Grp. gram.:adj.}
\end{itemize}
\begin{itemize}
\item {Proveniência:(De \textunderscore permissivo\textunderscore )}
\end{itemize}
Que se póde permittir; lícito; admissível.
\section{Permissivo}
\begin{itemize}
\item {Grp. gram.:adj.}
\end{itemize}
\begin{itemize}
\item {Proveniência:(Do lat. \textunderscore permissus\textunderscore )}
\end{itemize}
Que dá ou envolve permissão.
\section{Permisso}
\begin{itemize}
\item {Grp. gram.:m.}
\end{itemize}
\begin{itemize}
\item {Utilização:P. us.}
\end{itemize}
\begin{itemize}
\item {Proveniência:(Lat. \textunderscore permissum\textunderscore )}
\end{itemize}
O mesmo que \textunderscore permissão\textunderscore .
\section{Permissor}
\begin{itemize}
\item {Grp. gram.:adj.}
\end{itemize}
O mesmo que \textunderscore permissário\textunderscore .
\section{Permissário}
\begin{itemize}
\item {Grp. gram.:adj.}
\end{itemize}
\begin{itemize}
\item {Proveniência:(Do lat. \textunderscore permissor\textunderscore )}
\end{itemize}
Que permitte; que envolve permissão.
\section{Permisto}
\begin{itemize}
\item {Grp. gram.:adj.}
\end{itemize}
\begin{itemize}
\item {Proveniência:(Lat. \textunderscore permixtus\textunderscore  e \textunderscore permistus\textunderscore )}
\end{itemize}
Muito misturado; confundido, amalgamado. Cf. Castilho, \textunderscore Fastos\textunderscore , III, 47.
\section{Permitir}
\begin{itemize}
\item {Grp. gram.:v. t.}
\end{itemize}
\begin{itemize}
\item {Proveniência:(Lat. \textunderscore permittere\textunderscore )}
\end{itemize}
Dar licença a; consentir; autorizar; dar ocasião a; tolerar.
\section{Permittir}
\begin{itemize}
\item {Grp. gram.:v. t.}
\end{itemize}
\begin{itemize}
\item {Proveniência:(Lat. \textunderscore permittere\textunderscore )}
\end{itemize}
Dar licença a; consentir; autorizar; dar occasião a; tolerar.
\section{Permixto}
\begin{itemize}
\item {Grp. gram.:adj.}
\end{itemize}
\begin{itemize}
\item {Proveniência:(Lat. \textunderscore permixtus\textunderscore  e \textunderscore permistus\textunderscore )}
\end{itemize}
Muito misturado; confundido, amalgamado. Cf. Castilho, \textunderscore Fastos\textunderscore , III, 47.
\section{Permo-carbónico}
\begin{itemize}
\item {Grp. gram.:adj.}
\end{itemize}
\begin{itemize}
\item {Utilização:Geol.}
\end{itemize}
Diz-se de um terreno, que constitue um dos systemas da série primária ou paleozóica, e que é formado pelas camadas que ligam a parte superior do devónico á parte inferior do triádico, as quaes foram dantes classificadas em dois systemas distintos, carbonífero e permiano.
\section{Permudar}
\textunderscore v. t.\textunderscore  (e der.)
(Fórma des. de \textunderscore permutar\textunderscore , etc.)
\section{Permuta}
\begin{itemize}
\item {Grp. gram.:f.}
\end{itemize}
O mesmo que \textunderscore permutação\textunderscore .
\section{Permutabilidade}
\begin{itemize}
\item {Grp. gram.:f.}
\end{itemize}
Qualidade de permutável.
\section{Permutação}
\begin{itemize}
\item {Grp. gram.:f.}
\end{itemize}
\begin{itemize}
\item {Utilização:Gram.}
\end{itemize}
\begin{itemize}
\item {Utilização:Mathem.}
\end{itemize}
\begin{itemize}
\item {Proveniência:(Lat. \textunderscore permutatio\textunderscore )}
\end{itemize}
Acto ou effeito de permutar.
Câmbio.
Substituição de uma letra por outra.
Transposição ou arranjo, que se póde operar com a totalidade de objectos differentes.
\section{Permutador}
\begin{itemize}
\item {Grp. gram.:m.  e  adj.}
\end{itemize}
O que permuta.
\section{Permutar}
\begin{itemize}
\item {Grp. gram.:v. t.}
\end{itemize}
\begin{itemize}
\item {Utilização:Fig.}
\end{itemize}
\begin{itemize}
\item {Proveniência:(Lat. \textunderscore permutare\textunderscore )}
\end{itemize}
Dar reciprocamente.
Trocar.
Fazer participar reciprocamente; partilhar.
\section{Permutável}
\begin{itemize}
\item {Grp. gram.:adj.}
\end{itemize}
\begin{itemize}
\item {Proveniência:(Lat. \textunderscore permutabilis\textunderscore )}
\end{itemize}
Que se póde permutar.
\section{Perna}
\begin{itemize}
\item {Grp. gram.:f.}
\end{itemize}
\begin{itemize}
\item {Utilização:Ext.}
\end{itemize}
\begin{itemize}
\item {Utilização:Prov.}
\end{itemize}
\begin{itemize}
\item {Utilização:alent.}
\end{itemize}
\begin{itemize}
\item {Utilização:Fam.}
\end{itemize}
\begin{itemize}
\item {Grp. gram.:Loc.}
\end{itemize}
\begin{itemize}
\item {Utilização:fam.}
\end{itemize}
\begin{itemize}
\item {Utilização:Prov.}
\end{itemize}
\begin{itemize}
\item {Utilização:minh.}
\end{itemize}
\begin{itemize}
\item {Utilização:T. da Bairrada}
\end{itemize}
\begin{itemize}
\item {Proveniência:(Lat. \textunderscore perna\textunderscore )}
\end{itemize}
Cada um dos membros locomotores do corpo humano.
Cada um dos membros posteriores de alguns irracionaes.
Cada um dos membros locomotores de mammíferos, aves, insectos, etc.
Ramo, ramificação.
Cada um dos dois lados da asna.
Nome de várias peças de supporte.
Qualquer dos prolongamentos de um objecto, que se bifurca ou se ramifica.
Medida para a compra da rêde, que se emprega nos alfirmes.
\textunderscore Dar á perna\textunderscore , andar bem, andar depressa.
\textunderscore Fazer uma perna\textunderscore , tomar o lugar do parceiro ao jôgo.
Conluiar-se ou entrar numa negociação.
\textunderscore Têr alguém á perna\textunderscore , sêr perseguido, maçado, ameaçado por alguém. \textunderscore Loc. adv.\textunderscore 
\textunderscore Com uma perna ás costas\textunderscore , com grande facilidade.
\textunderscore Passar a perna a alguém\textunderscore , levar-lhe vantagem, excedê-lo.
\textunderscore Perna gorda\textunderscore , a coxa da perna.
\section{Pernaça}
\begin{itemize}
\item {Grp. gram.:f.}
\end{itemize}
\begin{itemize}
\item {Utilização:Pop.}
\end{itemize}
Perna gorda.
\section{Pernada}
\begin{itemize}
\item {Grp. gram.:f.}
\end{itemize}
\begin{itemize}
\item {Utilização:Náut.}
\end{itemize}
\begin{itemize}
\item {Utilização:Prov.}
\end{itemize}
\begin{itemize}
\item {Proveniência:(De \textunderscore perna\textunderscore )}
\end{itemize}
Passo largo.
Ramo de árvore, pôla.
Peça saliente de madeira.
Coice, pontapé.
\textunderscore Pernada real\textunderscore , cada um dos ramos principaes de uma árvore.
\section{Perna-de-moça}
\begin{itemize}
\item {Grp. gram.:f.}
\end{itemize}
Peixe de Portugal, chamado também \textunderscore dentudo\textunderscore .
\section{Perna-de-serra}
\begin{itemize}
\item {Grp. gram.:f.}
\end{itemize}
\begin{itemize}
\item {Utilização:Bras}
\end{itemize}
Peça de madeira, preparada de certa fórma para construcções. Cf. B. C. Rubim, \textunderscore Vocab. Bras.\textunderscore 
\section{Perna-fina}
\begin{itemize}
\item {Grp. gram.:m.}
\end{itemize}
\begin{itemize}
\item {Utilização:Fam.}
\end{itemize}
Janota, peralta:«\textunderscore trocam pai e mãe pelo primeiro perna-fina, que lhes empisca o ôlho ao dote.\textunderscore »Camillo, \textunderscore C. Ângela\textunderscore , 23.
\section{Perna-longa}
\begin{itemize}
\item {Grp. gram.:f.}
\end{itemize}
O mesmo que \textunderscore pernilongo\textunderscore , ave.
\section{Pernaltas}
\begin{itemize}
\item {Grp. gram.:f. pl.}
\end{itemize}
Ordem de aves, caracterizada por tarsos muito compridos.
(Fem. pl. de \textunderscore pernalto\textunderscore )
\section{Pernalteira}
\begin{itemize}
\item {Grp. gram.:f.}
\end{itemize}
\begin{itemize}
\item {Utilização:T. das Caldas da Raínha}
\end{itemize}
\begin{itemize}
\item {Proveniência:(De \textunderscore pernalteiro\textunderscore )}
\end{itemize}
Mulhér que, ao andar, ergue muito as saias, deixando vêr as pernas.
\section{Pernalteiro}
\begin{itemize}
\item {Grp. gram.:adj.}
\end{itemize}
\begin{itemize}
\item {Proveniência:(De \textunderscore perna\textunderscore  + \textunderscore alto\textunderscore )}
\end{itemize}
Que tem pernas altas; pernalto. Cf. Baganha, \textunderscore Hyg. Pec.\textunderscore , 238.
\section{Pernalto}
\begin{itemize}
\item {Grp. gram.:adj.}
\end{itemize}
\begin{itemize}
\item {Proveniência:(De \textunderscore perna\textunderscore  + \textunderscore alto\textunderscore )}
\end{itemize}
Que tem pernas altas.
\section{Pernaltudo}
\begin{itemize}
\item {Grp. gram.:adj.}
\end{itemize}
\begin{itemize}
\item {Proveniência:(De \textunderscore pernalto\textunderscore )}
\end{itemize}
Que tem pernas altas. Cf. P. Moraes, \textunderscore Econ. Rur.\textunderscore , 274.
\section{Perna-manca}
\begin{itemize}
\item {Grp. gram.:f.}
\end{itemize}
\begin{itemize}
\item {Utilização:Bras}
\end{itemize}
Travéssa de madeira.
\section{Pernambucano}
\begin{itemize}
\item {Grp. gram.:adj.}
\end{itemize}
\begin{itemize}
\item {Grp. gram.:M.}
\end{itemize}
Relativo a Pernambuco.
Habitante de Pernambuco.
\section{Pernão}
\begin{itemize}
\item {Grp. gram.:adj.}
\end{itemize}
(Corr. pop. de \textunderscore parnão\textunderscore )
\section{Pernão}
\begin{itemize}
\item {Grp. gram.:m.}
\end{itemize}
A parte mais alta e mais grossa da perna dos quadrúpedes. Cf. Baganha, \textunderscore Hyg. Pec.\textunderscore , 199.
Perna grossa.
\section{Perna-vermelha}
\begin{itemize}
\item {Grp. gram.:f.}
\end{itemize}
\begin{itemize}
\item {Utilização:Prov.}
\end{itemize}
\begin{itemize}
\item {Utilização:alg.}
\end{itemize}
O mesmo que \textunderscore maçarico\textunderscore .
\section{Pernavilheiro}
\begin{itemize}
\item {Grp. gram.:m.}
\end{itemize}
\begin{itemize}
\item {Utilização:Ant.}
\end{itemize}
Certa madeira, de fácil polimento, apresentando ao meio a côr do ébano, e sendo amarela nas bordas.
\section{Perne}
\begin{itemize}
\item {Grp. gram.:m.}
\end{itemize}
\begin{itemize}
\item {Utilização:Prov.}
\end{itemize}
Espigão de metal, com que se ligam duas peças de mármore. (Colhido em Arganil)
\section{Pérnea}
\begin{itemize}
\item {Grp. gram.:f.}
\end{itemize}
\begin{itemize}
\item {Proveniência:(De \textunderscore perna\textunderscore )}
\end{itemize}
Gênero de molluscos acéphalos, de concha bivalve.
\section{Pernear}
\begin{itemize}
\item {Grp. gram.:v. i.}
\end{itemize}
Agitar violentamente as pernas; espernear; dar pulos.
\section{Perneira}
\begin{itemize}
\item {Grp. gram.:f.}
\end{itemize}
\begin{itemize}
\item {Utilização:Prov.}
\end{itemize}
\begin{itemize}
\item {Utilização:Prov.}
\end{itemize}
\begin{itemize}
\item {Utilização:trasm.}
\end{itemize}
\begin{itemize}
\item {Utilização:Prov.}
\end{itemize}
\begin{itemize}
\item {Utilização:trasm.}
\end{itemize}
\begin{itemize}
\item {Grp. gram.:Pl.}
\end{itemize}
Doença, que ataca as pernas do gado bovino.
Cada uma das peças das calças, por onde se enfiam as pernas.
Pé de certos vegetaes: \textunderscore perneira de salsa, de craveiro\textunderscore , etc.
Antrachnose da videira.
Pequena porção.
Polainas de coiro ou de pano grosso.
\section{Perneta}
\begin{itemize}
\item {fónica:nê}
\end{itemize}
\begin{itemize}
\item {Grp. gram.:f.}
\end{itemize}
\begin{itemize}
\item {Utilização:Prov.}
\end{itemize}
\begin{itemize}
\item {Utilização:beir.}
\end{itemize}
\begin{itemize}
\item {Grp. gram.:M.  e  f.}
\end{itemize}
\begin{itemize}
\item {Utilização:Bras. do N}
\end{itemize}
Perna pequena.
Teima, birra; capricho tolo.
Pessôa, a quem falta uma perna, ou que tem lesa uma das pernas.
\section{Pernície}
\begin{itemize}
\item {Grp. gram.:f.}
\end{itemize}
\begin{itemize}
\item {Proveniência:(Lat. \textunderscore pernicies\textunderscore )}
\end{itemize}
Destruição, ruína; prejuízo.
\section{Perniciosa}
\begin{itemize}
\item {Grp. gram.:f.}
\end{itemize}
\begin{itemize}
\item {Proveniência:(De \textunderscore pernicioso\textunderscore )}
\end{itemize}
Febre intermittente, de carácter muito grave, acompanhada de delírio e quási sempre mortal.
\section{Perniciosamente}
\begin{itemize}
\item {Grp. gram.:adv.}
\end{itemize}
De modo pernicioso.
\section{Perniciosidade}
\begin{itemize}
\item {Grp. gram.:f.}
\end{itemize}
\begin{itemize}
\item {Utilização:Med.}
\end{itemize}
\begin{itemize}
\item {Proveniência:(Lat. \textunderscore perniciositas\textunderscore )}
\end{itemize}
Qualidade de pernicioso, em certas febres.
\section{Pernicioso}
\begin{itemize}
\item {Grp. gram.:adj.}
\end{itemize}
\begin{itemize}
\item {Proveniência:(Lat. \textunderscore perniciosus\textunderscore )}
\end{itemize}
Prejudicial; perigoso; ruinoso.
Diz-se de uma febre, que também se designa substantivamente por \textunderscore perniciosa\textunderscore .
V. \textunderscore perniciosa\textunderscore .
\section{Pernicurto}
\begin{itemize}
\item {Grp. gram.:adj.}
\end{itemize}
\begin{itemize}
\item {Proveniência:(De \textunderscore perna\textunderscore  + \textunderscore curto\textunderscore )}
\end{itemize}
Que tem pernas curtas.
\section{Pernigrande}
\begin{itemize}
\item {Grp. gram.:adj.}
\end{itemize}
Que tem pernas grandes. Cf. Macedo, \textunderscore Burros\textunderscore , 231.
\section{Pernil}
\begin{itemize}
\item {Grp. gram.:m.}
\end{itemize}
\begin{itemize}
\item {Utilização:Pop.}
\end{itemize}
Parte mais delgada da perna do porco e de outros animaes.
Perna magra.
\textunderscore Esticar o pernil\textunderscore , morrer.
\section{Pernilongo}
\begin{itemize}
\item {Grp. gram.:adj.}
\end{itemize}
\begin{itemize}
\item {Grp. gram.:M.}
\end{itemize}
\begin{itemize}
\item {Utilização:Bras}
\end{itemize}
\begin{itemize}
\item {Proveniência:(De \textunderscore perna\textunderscore  + \textunderscore longo\textunderscore )}
\end{itemize}
Que tem pernas compridas.
Ave ribeirinha, (\textunderscore hímantopus candidus\textunderscore , Bonnat.).
Variedade de mosquito.
\section{Pernim}
\begin{itemize}
\item {Grp. gram.:m.}
\end{itemize}
\begin{itemize}
\item {Utilização:T. da Índia port}
\end{itemize}
Homem que, mascando certas fôlhas e atirando-as a uma cobra-capello, a entorpece, podendo agarrá-la, tirar-lhe os dentes venenosos e torná-la inoffensiva; encantador de cobras. Cf. \textunderscore Século\textunderscore , de 25-VI-903.
\section{Pernitroso}
\begin{itemize}
\item {Grp. gram.:adj.}
\end{itemize}
\begin{itemize}
\item {Utilização:Chím.}
\end{itemize}
\begin{itemize}
\item {Proveniência:(De \textunderscore per...\textunderscore  + \textunderscore nítroso\textunderscore )}
\end{itemize}
Diz-se de um ácido, o mesmo que \textunderscore hypoazótico\textunderscore .
\section{Perno}
\begin{itemize}
\item {Grp. gram.:m.}
\end{itemize}
Pequeno eixo ou cavilha cylíndrica de vários mecanismos.
Espécie de agulha grande, que as mulheres usavam na cabeça como ornato.
(Cp. cast. \textunderscore perno\textunderscore )
\section{Pernoita}
\begin{itemize}
\item {Grp. gram.:f.}
\end{itemize}
O mesmo que \textunderscore pernoitamento\textunderscore .
\section{Pernoitamento}
\begin{itemize}
\item {Grp. gram.:m.}
\end{itemize}
Acto ou effeito de pernoitar. Cf. \textunderscore Decreto\textunderscore  de 25-II-911.
\section{Pernoitar}
\begin{itemize}
\item {Grp. gram.:v. i.}
\end{itemize}
\begin{itemize}
\item {Proveniência:(Do lat. \textunderscore pernoctare\textunderscore )}
\end{itemize}
Ficar de noite; tomar poisada; passar a noite.
\section{Pernóstico}
\begin{itemize}
\item {Grp. gram.:adj.}
\end{itemize}
\begin{itemize}
\item {Utilização:Pop.}
\end{itemize}
O mesmo que \textunderscore pronóstico\textunderscore .
Repontão.
\section{Pernudo}
\begin{itemize}
\item {Grp. gram.:adj.}
\end{itemize}
Que tem pernas grandes. Cf. Castilho, \textunderscore Fausto\textunderscore , 333.
\section{Pero}
\begin{itemize}
\item {fónica:pê}
\end{itemize}
\begin{itemize}
\item {Grp. gram.:m.}
\end{itemize}
\begin{itemize}
\item {Utilização:T. de Alcanena}
\end{itemize}
\begin{itemize}
\item {Proveniência:(Do lat. \textunderscore pirum\textunderscore )}
\end{itemize}
Maçan doce e oblonga, fruto do pereiro.
Malápio.
Variedade de pêra pequena e temporan.
\section{Pero}
\begin{itemize}
\item {Grp. gram.:conj.}
\end{itemize}
\begin{itemize}
\item {Utilização:Ant.}
\end{itemize}
\begin{itemize}
\item {Proveniência:(Do lat. \textunderscore per\textunderscore  + \textunderscore hoc\textunderscore )}
\end{itemize}
Porém; mas; ainda que.
\section{Peró}
\begin{itemize}
\item {Grp. gram.:conj.}
\end{itemize}
\begin{itemize}
\item {Utilização:Ant.}
\end{itemize}
\begin{itemize}
\item {Proveniência:(Do lat. \textunderscore per\textunderscore  + \textunderscore hoc\textunderscore )}
\end{itemize}
Porém; mas; ainda que.
\section{Peroá}
\begin{itemize}
\item {Grp. gram.:m.}
\end{itemize}
\begin{itemize}
\item {Utilização:Bras}
\end{itemize}
Peixe fluvial.
\section{Peroba}
\begin{itemize}
\item {Grp. gram.:f.}
\end{itemize}
\begin{itemize}
\item {Utilização:Bras}
\end{itemize}
Nome commum de várias árvores de construcção.--Os diccion. dizem perobá. Creio que é êrro.
(Talvez do tupi \textunderscore ipe\textunderscore  + \textunderscore roba\textunderscore )
\section{Perobinha-do-campo}
\begin{itemize}
\item {Grp. gram.:f.}
\end{itemize}
\begin{itemize}
\item {Utilização:Bras}
\end{itemize}
Planta leguminosa, medicinal, (\textunderscore leptolobium elegans\textunderscore ).
\section{Perobinho}
\begin{itemize}
\item {Grp. gram.:m.}
\end{itemize}
\begin{itemize}
\item {Proveniência:(De \textunderscore peroba\textunderscore )}
\end{itemize}
Planta brasileira, da fam. das bignoniáceas.
\section{Perogi}
\begin{itemize}
\item {Grp. gram.:m.}
\end{itemize}
\begin{itemize}
\item {Utilização:Ant.}
\end{itemize}
Moéda portuguesa de Dio.
(Cp. \textunderscore perozil\textunderscore )
\section{Perol}
\begin{itemize}
\item {Grp. gram.:m.}
\end{itemize}
\begin{itemize}
\item {Utilização:Ant.}
\end{itemize}
Porém, todavia. Cf. G. Vicente.
(Cp. \textunderscore pero\textunderscore ^2)
\section{Pérola}
\begin{itemize}
\item {Grp. gram.:f.}
\end{itemize}
\begin{itemize}
\item {Utilização:Fig.}
\end{itemize}
\begin{itemize}
\item {Proveniência:(Do b. lat. \textunderscore perulus\textunderscore , talvez do lat. \textunderscore pirum\textunderscore )}
\end{itemize}
Glóbulo branco, que se fórma em certas conchas.
Aljofre.
Variedade de peras muito apreciadas.
Espécie de uva branca.
Casta de uva algarvia. Cf. \textunderscore Rev. Agron.\textunderscore , I, 18.
Diz-se de uma variedade de chá.
Pessôa muito bondosa ou de distintas qualidades.
\section{Peroleira}
\begin{itemize}
\item {Grp. gram.:f.}
\end{itemize}
\begin{itemize}
\item {Proveniência:(De \textunderscore pérola\textunderscore )}
\end{itemize}
Vasilha afunilada, para guardar azeitonas.
Mollusco acéphalo, em cuja concha se produz a pérola.
\section{Perolífero}
\begin{itemize}
\item {Grp. gram.:adj.}
\end{itemize}
\begin{itemize}
\item {Proveniência:(De \textunderscore pérola\textunderscore  + lat. \textunderscore ferre\textunderscore )}
\end{itemize}
Diz-se das conchas, em que se formam as pérolas.
\section{Perolizar}
\begin{itemize}
\item {Grp. gram.:v. t.}
\end{itemize}
Dar côr ou apparência de pérola a. Cf. Alv. Mendes, \textunderscore Discursos\textunderscore , 171.
\section{Perómelos}
\begin{itemize}
\item {Grp. gram.:m. pl.}
\end{itemize}
\begin{itemize}
\item {Proveniência:(Do gr. \textunderscore peros\textunderscore  + \textunderscore melos\textunderscore )}
\end{itemize}
Um dos três grupos em que se divide a ordem dos batrácios.
\section{Peroneal}
\begin{itemize}
\item {Grp. gram.:adj.}
\end{itemize}
Relativo ao peroneu.
\section{Peronema}
\begin{itemize}
\item {Grp. gram.:f.}
\end{itemize}
Gênero de plantas verbenáceas.
\section{Peróneo}
\begin{itemize}
\item {Grp. gram.:adj.}
\end{itemize}
\begin{itemize}
\item {Grp. gram.:M.}
\end{itemize}
O mesmo que \textunderscore peroneal\textunderscore .
O mesmo que \textunderscore peroneu\textunderscore .
\section{Peroneu}
\begin{itemize}
\item {Grp. gram.:adj.}
\end{itemize}
\begin{itemize}
\item {Grp. gram.:M.}
\end{itemize}
\begin{itemize}
\item {Proveniência:(Lat. anatómico \textunderscore peronaeum\textunderscore , do gr. \textunderscore perone\textunderscore )}
\end{itemize}
Peroneal.
Osso de perna, que fica ao lado da tíbia.
\section{Perónia}
\begin{itemize}
\item {Grp. gram.:f.}
\end{itemize}
\begin{itemize}
\item {Proveniência:(De \textunderscore Péron\textunderscore , n. p.)}
\end{itemize}
Gênero de molluscos gasterópodes dos mares austraes.
\section{Peronina}
\begin{itemize}
\item {Grp. gram.:f.}
\end{itemize}
Substância pharmacêutica, que é o chlorydrato do éther benzýlico da morphina.
\section{Peronóspora-vitícola}
\begin{itemize}
\item {Grp. gram.:f.}
\end{itemize}
\begin{itemize}
\item {Proveniência:(Do gr. \textunderscore perone\textunderscore  + \textunderscore spora\textunderscore )}
\end{itemize}
Cogumelo microscópico, que origína o míldio.
\section{Peronospóreas}
\begin{itemize}
\item {Grp. gram.:f.}
\end{itemize}
\begin{itemize}
\item {Proveniência:(De \textunderscore peronóspora\textunderscore )}
\end{itemize}
Família de cogumelos.
\section{Peroração}
\begin{itemize}
\item {Grp. gram.:f.}
\end{itemize}
\begin{itemize}
\item {Proveniência:(Lat. \textunderscore peroratio\textunderscore )}
\end{itemize}
Última parte de um discurso.
Pequeno discurso.
Última parte de uma symphonia.
\section{Perorador}
\begin{itemize}
\item {Grp. gram.:m.  e  adj.}
\end{itemize}
Aquelle que perora; orador.
\section{Perorar}
\begin{itemize}
\item {Grp. gram.:v. i.}
\end{itemize}
\begin{itemize}
\item {Proveniência:(Lat. \textunderscore perorare\textunderscore )}
\end{itemize}
Terminar um discurso.
Discursar affectadamente.
Falar a favor de alguém.
\section{Peros}
\begin{itemize}
\item {fónica:pê}
\end{itemize}
\begin{itemize}
\item {Grp. gram.:m. pl.}
\end{itemize}
\begin{itemize}
\item {Utilização:Bras. de Minas}
\end{itemize}
\begin{itemize}
\item {Utilização:ant.}
\end{itemize}
Os Portugueses.
\section{Perova}
\begin{itemize}
\item {Grp. gram.:f.}
\end{itemize}
O mesmo que \textunderscore peroba\textunderscore . Cf. Júl. Ribeiro, \textunderscore Carne\textunderscore .
\section{Peroxidar}
\begin{itemize}
\item {fónica:csi}
\end{itemize}
\begin{itemize}
\item {Grp. gram.:v. t.}
\end{itemize}
\begin{itemize}
\item {Proveniência:(De \textunderscore per...\textunderscore  + \textunderscore oxidar\textunderscore )}
\end{itemize}
Oxidar no mais alto grau.
\section{Peróxido}
\begin{itemize}
\item {fónica:csi}
\end{itemize}
\begin{itemize}
\item {Grp. gram.:m.}
\end{itemize}
\begin{itemize}
\item {Proveniência:(De \textunderscore per...\textunderscore  + \textunderscore óxido\textunderscore )}
\end{itemize}
Combinação de um corpo simples com a maior porção de oxigênio, que ele póde absorver.
\section{Peroxydar}
\begin{itemize}
\item {fónica:csi}
\end{itemize}
\begin{itemize}
\item {Grp. gram.:v. t.}
\end{itemize}
\begin{itemize}
\item {Proveniência:(De \textunderscore per...\textunderscore  + \textunderscore oxydar\textunderscore )}
\end{itemize}
Oxydar no mais alto grau.
\section{Peróxydo}
\begin{itemize}
\item {fónica:csi}
\end{itemize}
\begin{itemize}
\item {Grp. gram.:m.}
\end{itemize}
\begin{itemize}
\item {Proveniência:(De \textunderscore per...\textunderscore  + \textunderscore óxydo\textunderscore )}
\end{itemize}
Combinação de um corpo simples com a maior porção de oxygênio, que elle póde absorver.
\section{Perozil}
\begin{itemize}
\item {Grp. gram.:m.}
\end{itemize}
\begin{itemize}
\item {Utilização:Ant.}
\end{itemize}
Moéda de liga de cobre e prata, em Cambaia.
\section{Perpassar}
\begin{itemize}
\item {Grp. gram.:v. i.}
\end{itemize}
\begin{itemize}
\item {Grp. gram.:V. t.}
\end{itemize}
\begin{itemize}
\item {Proveniência:(De \textunderscore per...\textunderscore  + \textunderscore passar\textunderscore )}
\end{itemize}
Passar junto; seguir certa direcção.
Mover-se; decorrer: \textunderscore perpassaram annos\textunderscore .
Preterir, postergar:«\textunderscore era impossivel a El-Rei perpassar o peditorio de sua irmã\textunderscore ...»Filinto, \textunderscore D. Man.\textunderscore , I, 31.
\section{Perpassável}
\begin{itemize}
\item {Grp. gram.:adj.}
\end{itemize}
Que se póde passar; tolerável.
\section{Perpendicular}
\begin{itemize}
\item {Grp. gram.:adj.}
\end{itemize}
\begin{itemize}
\item {Grp. gram.:F.}
\end{itemize}
\begin{itemize}
\item {Proveniência:(Lat. \textunderscore perpendicularis\textunderscore )}
\end{itemize}
Que cái sôbre uma linha ou superfície, formando com ella ângulo recto.
Linha perpendicular.
\section{Perpendicularidade}
\begin{itemize}
\item {Grp. gram.:f.}
\end{itemize}
Qualidade do que é perpendicular.
\section{Perpendicularmente}
\begin{itemize}
\item {Grp. gram.:adv.}
\end{itemize}
De modo perpendicular.
\section{Perpendículo}
\begin{itemize}
\item {Grp. gram.:m.}
\end{itemize}
\begin{itemize}
\item {Proveniência:(Lat. \textunderscore perpendiculum\textunderscore )}
\end{itemize}
Fio de prumo.
\section{Perpetração}
\begin{itemize}
\item {Grp. gram.:f.}
\end{itemize}
\begin{itemize}
\item {Proveniência:(Lat. \textunderscore perpetratio\textunderscore )}
\end{itemize}
Acto ou effeito de perpetrar.
\section{Perpetrador}
\begin{itemize}
\item {Grp. gram.:m.  e  adj.}
\end{itemize}
\begin{itemize}
\item {Proveniência:(Lat. \textunderscore perpetrator\textunderscore )}
\end{itemize}
Aquelle que perpetrou.
\section{Perpetrar}
\begin{itemize}
\item {Grp. gram.:v. t.}
\end{itemize}
\begin{itemize}
\item {Proveniência:(Lat. \textunderscore perpetrare\textunderscore )}
\end{itemize}
Perfazer.
Realizar.
Commeter (acto condemnável): \textunderscore perpetrar um crime\textunderscore ; \textunderscore perpetrar erros\textunderscore .
\section{Perpétua}
\begin{itemize}
\item {Grp. gram.:f.}
\end{itemize}
Nome de várias plantas, da fam. das compostas.
Flôr dessas plantas.
(Fem. de \textunderscore perpétuo\textunderscore )
\section{Perpetuação}
\begin{itemize}
\item {Grp. gram.:f.}
\end{itemize}
Acto ou effeito de perpetuar.
\section{Perpetuador}
\begin{itemize}
\item {Grp. gram.:m.  e  adj.}
\end{itemize}
O que perpetúa.
\section{Perpetuamente}
\begin{itemize}
\item {Grp. gram.:adv.}
\end{itemize}
De modo perpétuo.
\section{Perpetuamento}
\begin{itemize}
\item {Grp. gram.:m.}
\end{itemize}
O mesmo que \textunderscore perpetuação\textunderscore .
\section{Perpetuana}
\begin{itemize}
\item {Grp. gram.:f.}
\end{itemize}
\begin{itemize}
\item {Proveniência:(De \textunderscore perpétuo\textunderscore )}
\end{itemize}
Antigo tecido de lan.
\section{Perpetuar}
\begin{itemize}
\item {Grp. gram.:v. t.}
\end{itemize}
\begin{itemize}
\item {Proveniência:(Lat. \textunderscore perpetuare\textunderscore )}
\end{itemize}
Fazer perpétuo.
Tornar muito duradoiro.
Immortalizar.
Propagar por muito tempo.
\section{Perpetuidade}
\begin{itemize}
\item {Grp. gram.:f.}
\end{itemize}
\begin{itemize}
\item {Proveniência:(Lat. \textunderscore perpetuitas\textunderscore )}
\end{itemize}
Qualidade do que é perpétuo.
\section{Perpetuísta}
\begin{itemize}
\item {Grp. gram.:adj.}
\end{itemize}
\begin{itemize}
\item {Proveniência:(De \textunderscore perpétuo\textunderscore )}
\end{itemize}
Relativo ao systema dos que, em Anthropologia, consideram perpétua uma espécie.
\section{Perpétuo}
\begin{itemize}
\item {Grp. gram.:adj.}
\end{itemize}
\begin{itemize}
\item {Proveniência:(Lat. \textunderscore perpetuus\textunderscore )}
\end{itemize}
Contínuo, continuado.
Constante.
Ininterrupto, eterno.
Inalterável.
Vitalício.
\section{Perpianho}
\begin{itemize}
\item {Grp. gram.:m.}
\end{itemize}
Pedra, que tem toda a largura de uma parede, e apparelhadas as quatro faces.
(Cast. \textunderscore perpiaño\textunderscore )
\section{Perplexamente}
\begin{itemize}
\item {Grp. gram.:adv.}
\end{itemize}
De modo perplexo.
\section{Perplexão}
\begin{itemize}
\item {Grp. gram.:f.}
\end{itemize}
Qualidade do que é perplexo.
\section{Perplexidade}
\begin{itemize}
\item {Grp. gram.:f.}
\end{itemize}
\begin{itemize}
\item {Proveniência:(Lat. \textunderscore perplexitas\textunderscore )}
\end{itemize}
Qualidade do que é perplexo.
\section{Perplexo}
\begin{itemize}
\item {Grp. gram.:adj.}
\end{itemize}
\begin{itemize}
\item {Proveniência:(Lat. \textunderscore perplexus\textunderscore )}
\end{itemize}
Indeciso; irresoluto; duvidoso.
\section{Perpoém}
\begin{itemize}
\item {Grp. gram.:m.}
\end{itemize}
\begin{itemize}
\item {Utilização:Ant.}
\end{itemize}
\begin{itemize}
\item {Proveniência:(Fr. \textunderscore pourpoint\textunderscore )}
\end{itemize}
Gibão de abas largas.
\section{Perpol!}
\begin{itemize}
\item {Grp. gram.:interj.}
\end{itemize}
\begin{itemize}
\item {Proveniência:(Do lat. \textunderscore per\textunderscore  + \textunderscore Pollux\textunderscore , n. p.)}
\end{itemize}
Fórma de juramento, entre os Pagãos; por Póllux.
\section{Perponte}
\begin{itemize}
\item {Grp. gram.:m.}
\end{itemize}
\begin{itemize}
\item {Proveniência:(Do cast. \textunderscore perpunte\textunderscore )}
\end{itemize}
Antigo gibão acolchoado, usado por guerreiros.
\section{Perponto}
\begin{itemize}
\item {Grp. gram.:m.}
\end{itemize}
O mesmo que \textunderscore perponte\textunderscore . C. Herculano, \textunderscore Bobo\textunderscore , 328.
\section{Perpunto}
\begin{itemize}
\item {Grp. gram.:m.}
\end{itemize}
(V.perponte)
\section{Perquirir}
\begin{itemize}
\item {Grp. gram.:v. t.}
\end{itemize}
\begin{itemize}
\item {Proveniência:(Lat. \textunderscore perquirere\textunderscore )}
\end{itemize}
Investigar escrupulosamente.
Indagar, inquirir minuciosamente. Cf. Castilho, \textunderscore Fausto\textunderscore , XIV.
\section{Perquisição}
\begin{itemize}
\item {Grp. gram.:f.}
\end{itemize}
\begin{itemize}
\item {Proveniência:(Lat. \textunderscore perquisitio\textunderscore )}
\end{itemize}
Investigação; inquirição minuciosa.
\section{Perra}
\begin{itemize}
\item {fónica:pê}
\end{itemize}
\begin{itemize}
\item {Grp. gram.:f.}
\end{itemize}
\begin{itemize}
\item {Utilização:Des.}
\end{itemize}
\begin{itemize}
\item {Utilização:Prov.}
\end{itemize}
\begin{itemize}
\item {Proveniência:(De \textunderscore perro\textunderscore )}
\end{itemize}
Cadella.
O mesmo que \textunderscore tarrantana\textunderscore .
\section{Perrada}
\begin{itemize}
\item {Grp. gram.:f.}
\end{itemize}
\begin{itemize}
\item {Utilização:Prov.}
\end{itemize}
\begin{itemize}
\item {Utilização:alent.}
\end{itemize}
\begin{itemize}
\item {Proveniência:(De \textunderscore perro\textunderscore )}
\end{itemize}
Desfeita, pirraça.
\section{Perraria}
\begin{itemize}
\item {Grp. gram.:f.}
\end{itemize}
\begin{itemize}
\item {Proveniência:(De \textunderscore perro\textunderscore )}
\end{itemize}
Pirraça, desfeita, perrice.
\section{Perreiro}
\begin{itemize}
\item {Grp. gram.:m.}
\end{itemize}
\begin{itemize}
\item {Utilização:Prov.}
\end{itemize}
\begin{itemize}
\item {Utilização:Bras}
\end{itemize}
\begin{itemize}
\item {Utilização:alent.}
\end{itemize}
\begin{itemize}
\item {Proveniência:(De \textunderscore perro\textunderscore )}
\end{itemize}
Guarda de matilha.
\section{Perrengue}
\begin{itemize}
\item {Proveniência:(T. cast.)}
\end{itemize}
\textunderscore m.\textunderscore , \textunderscore f.\textunderscore  e \textunderscore adj. Bras. do Rio\textunderscore .
Encanzinado; birrento.
\section{Perrexil}
\begin{itemize}
\item {Grp. gram.:m.}
\end{itemize}
\begin{itemize}
\item {Utilização:Fig.}
\end{itemize}
\begin{itemize}
\item {Proveniência:(Do cast. \textunderscore perejil\textunderscore )}
\end{itemize}
Planta umbellífera.
Casta de uva branca.
Aquillo que estimula o appetite:«\textunderscore ...o paladar cansado carece de perrexis estimulantes\textunderscore ». Camillo, \textunderscore Caveira\textunderscore , 39.
\section{Perrice}
\begin{itemize}
\item {Grp. gram.:f.}
\end{itemize}
\begin{itemize}
\item {Utilização:Pop.}
\end{itemize}
\begin{itemize}
\item {Proveniência:(De \textunderscore perro\textunderscore )}
\end{itemize}
Pertinácia; teimosia; caturrice.
Maldade; pirraça.
\section{Perrilha}
\begin{itemize}
\item {Grp. gram.:f.}
\end{itemize}
\begin{itemize}
\item {Utilização:T. de Monção}
\end{itemize}
Vento frio do Nordeste. Cf. O. Pratt, \textunderscore Ling. Minh.\textunderscore 
\section{Perrisco}
\begin{itemize}
\item {Grp. gram.:m.}
\end{itemize}
\begin{itemize}
\item {Utilização:Prov.}
\end{itemize}
\begin{itemize}
\item {Utilização:beir.}
\end{itemize}
Pequena malhada, improvisada, de tábuas ou cancellas, para conter as ovelhas, que se vão ordenhar.
(Cp. \textunderscore aprisco\textunderscore ?)
\section{Perro}
\begin{itemize}
\item {fónica:pê}
\end{itemize}
\begin{itemize}
\item {Grp. gram.:m.}
\end{itemize}
\begin{itemize}
\item {Utilização:Deprec.}
\end{itemize}
\begin{itemize}
\item {Grp. gram.:Adj.}
\end{itemize}
\begin{itemize}
\item {Grp. gram.:Pl. Loc.}
\end{itemize}
\begin{itemize}
\item {Utilização:fam.}
\end{itemize}
O mesmo que \textunderscore cão\textunderscore ^1.
Homem vil; canalha.
Pertinaz; teimoso.
Mau.
Resistente.
Que se não abre nem se fecha facilmente; emperrado: \textunderscore porta perra\textunderscore .
\textunderscore Dar-se a perros\textunderscore , dar tratos á imaginação para descobrir ou resolver qualquer coisa.
\section{Perruca}
\begin{itemize}
\item {Grp. gram.:f.}
\end{itemize}
O mesmo que \textunderscore peruca\textunderscore . Cf. Castilho, \textunderscore Avarento\textunderscore , 83.
\section{Perrum}
\begin{itemize}
\item {Grp. gram.:m.}
\end{itemize}
\begin{itemize}
\item {Proveniência:(De \textunderscore perro\textunderscore )}
\end{itemize}
Casta de uva branca do Alentejo e do Algarve.
\section{Perruma}
\begin{itemize}
\item {Grp. gram.:f.}
\end{itemize}
\begin{itemize}
\item {Proveniência:(De \textunderscore perro\textunderscore )}
\end{itemize}
Pão ordinário, feito de farelo, para os cães.
\section{Persa}
\begin{itemize}
\item {Grp. gram.:adj.}
\end{itemize}
\begin{itemize}
\item {Grp. gram.:M.}
\end{itemize}
\begin{itemize}
\item {Proveniência:(Lat. \textunderscore persa\textunderscore )}
\end{itemize}
Relativo á Pérsia.
Habitante da Pérsia; língua dos persas.
\section{Perscrutação}
\begin{itemize}
\item {Grp. gram.:f.}
\end{itemize}
\begin{itemize}
\item {Proveniência:(Lat. \textunderscore perscrutatio\textunderscore )}
\end{itemize}
Acto ou effeito de perscrutar.
\section{Perscrutador}
\begin{itemize}
\item {Grp. gram.:m.  e  adj.}
\end{itemize}
\begin{itemize}
\item {Proveniência:(Lat. \textunderscore perscrutator\textunderscore )}
\end{itemize}
O que perscruta.
\section{Perscrutar}
\begin{itemize}
\item {Grp. gram.:v. t.}
\end{itemize}
\begin{itemize}
\item {Proveniência:(Lat. \textunderscore perscrutari\textunderscore )}
\end{itemize}
Esquadrinhar; investigar minuciosamente, penetrar.
\section{Perscrutável}
\begin{itemize}
\item {Grp. gram.:adj.}
\end{itemize}
\begin{itemize}
\item {Proveniência:(Lat. \textunderscore perscrutabilis\textunderscore )}
\end{itemize}
Que se póde perscrutar.
\section{Persecução}
\begin{itemize}
\item {Grp. gram.:f.}
\end{itemize}
\begin{itemize}
\item {Proveniência:(Lat. \textunderscore persecutio\textunderscore )}
\end{itemize}
O mesmo que \textunderscore perseguição\textunderscore .
\section{Pêrsego}
\begin{itemize}
\item {Grp. gram.:m.}
\end{itemize}
\begin{itemize}
\item {Utilização:Prov.}
\end{itemize}
\begin{itemize}
\item {Utilização:trasm.}
\end{itemize}
\begin{itemize}
\item {Proveniência:(Do lat. \textunderscore persicus\textunderscore )}
\end{itemize}
O mesmo que \textunderscore pêssego\textunderscore .
\section{Persegueiro}
\begin{itemize}
\item {Grp. gram.:m.}
\end{itemize}
\begin{itemize}
\item {Utilização:Prov.}
\end{itemize}
\begin{itemize}
\item {Utilização:trasm.}
\end{itemize}
\begin{itemize}
\item {Proveniência:(De \textunderscore pêrsego\textunderscore )}
\end{itemize}
O mesmo que \textunderscore pessegueiro\textunderscore .
\section{Perseguição}
\begin{itemize}
\item {Grp. gram.:f.}
\end{itemize}
Acto ou effeito de perseguir.
\section{Perseguidor}
\begin{itemize}
\item {Grp. gram.:m.  e  adj.}
\end{itemize}
O que persegue.
\section{Perseguimento}
\begin{itemize}
\item {Grp. gram.:m.}
\end{itemize}
O mesmo que \textunderscore perseguição\textunderscore .
\section{Perseguir}
\begin{itemize}
\item {Grp. gram.:v. t.}
\end{itemize}
\begin{itemize}
\item {Proveniência:(Do lat. \textunderscore persequi\textunderscore )}
\end{itemize}
Seguir de perto.
Acossar.
Importunar.
Punir; fazer punir.
Ir no encalço de.
Vexar, fazendo violência.
\section{Perseia}
\begin{itemize}
\item {Grp. gram.:f.}
\end{itemize}
\begin{itemize}
\item {Utilização:Ant.}
\end{itemize}
Alfaia ou jóia de grande valor. Cf. \textunderscore Viriato Trág.\textunderscore , III, 37.
\section{Perseidade}
\begin{itemize}
\item {fónica:se-i}
\end{itemize}
\begin{itemize}
\item {Grp. gram.:f.}
\end{itemize}
\begin{itemize}
\item {Proveniência:(Lat. barb. \textunderscore perseitas\textunderscore )}
\end{itemize}
Expressão, usada na philosophia escolástica, para designar a qualidade daquillo que existe de per si.
\section{Persemelhante}
\begin{itemize}
\item {Grp. gram.:adj.}
\end{itemize}
\begin{itemize}
\item {Utilização:Ant.}
\end{itemize}
\begin{itemize}
\item {Proveniência:(De \textunderscore per\textunderscore  + \textunderscore semelhante\textunderscore )}
\end{itemize}
Mui semelhantemente.
\section{Pérseo}
\begin{itemize}
\item {Grp. gram.:adj.}
\end{itemize}
Relativo á Pérsia.
Natural da Pérsia:«\textunderscore ...mercadores índios, pérseos e arábios...\textunderscore »Filinto, \textunderscore D. Man.\textunderscore , II, 75.
\section{Persevão}
\begin{itemize}
\item {Grp. gram.:f.}
\end{itemize}
Parte interior do cóche, onde o passageiro apoia os pés.
\section{Perseverança}
\begin{itemize}
\item {Grp. gram.:f.}
\end{itemize}
\begin{itemize}
\item {Proveniência:(Lat. \textunderscore perseverantia\textunderscore )}
\end{itemize}
Qualidade ou acto do que é perseverante.
\section{Perseverância}
\begin{itemize}
\item {Grp. gram.:f.}
\end{itemize}
\begin{itemize}
\item {Utilização:Ant.}
\end{itemize}
O mesmo que \textunderscore perseverança\textunderscore . Cf. Usque 33, v.^o.
\section{Perseverante}
\begin{itemize}
\item {Grp. gram.:adj.}
\end{itemize}
\begin{itemize}
\item {Proveniência:(Lat. \textunderscore perseverans\textunderscore )}
\end{itemize}
Que persevera.
\section{Perseverantemente}
\begin{itemize}
\item {Grp. gram.:adv.}
\end{itemize}
De modo perseverante.
Com insistência.
\section{Perseverar}
\begin{itemize}
\item {Grp. gram.:v. i.}
\end{itemize}
\begin{itemize}
\item {Proveniência:(Lat. \textunderscore perseverare\textunderscore )}
\end{itemize}
Sêr constante.
Têr firmeza.
Persistir; proseguir: \textunderscore perseverar no crime\textunderscore .
\section{Persiana}
\begin{itemize}
\item {Grp. gram.:f.}
\end{itemize}
\begin{itemize}
\item {Proveniência:(Do fr. \textunderscore persienne\textunderscore )}
\end{itemize}
Caixilho de tabuínhas móveis, que se colloca por fóra das janelas ou sacadas, para que não entre o Sol, nem se devasse o interior da casa.
\section{Persiano}
\begin{itemize}
\item {Grp. gram.:adj.}
\end{itemize}
\begin{itemize}
\item {Grp. gram.:M.}
\end{itemize}
Relativo á Pérsia.
A língua persa de hoje.
Indivíduo, natural da Pérsia. Cf. Pant. de Aveiro, \textunderscore Itiner.\textunderscore , 11, (2.^a ed.).
\section{Persicária}
\begin{itemize}
\item {Grp. gram.:f.}
\end{itemize}
\begin{itemize}
\item {Utilização:Bot.}
\end{itemize}
\begin{itemize}
\item {Proveniência:(Do b. lat. \textunderscore persicarius\textunderscore , pessegueiro)}
\end{itemize}
Nome de muitas espécies de plantas polygoniáceas.
\section{Persicita}
\begin{itemize}
\item {Grp. gram.:f.}
\end{itemize}
\begin{itemize}
\item {Utilização:Miner.}
\end{itemize}
\begin{itemize}
\item {Proveniência:(Do lat. \textunderscore persicus\textunderscore )}
\end{itemize}
Pedra argillosa, que tem a semelhança de um pêssego.
\section{Persicito}
\begin{itemize}
\item {Grp. gram.:m.}
\end{itemize}
O mesmo ou melhor que \textunderscore persicita\textunderscore .
\section{Pérsico}
\begin{itemize}
\item {Grp. gram.:adj.}
\end{itemize}
\begin{itemize}
\item {Grp. gram.:M.}
\end{itemize}
Relativo á Pérsia.
Língua pérsica. Cf. Filinto, \textunderscore D. Man.\textunderscore , II, 89.
\section{Persigal}
\begin{itemize}
\item {Grp. gram.:m.}
\end{itemize}
\begin{itemize}
\item {Utilização:Ant.}
\end{itemize}
\begin{itemize}
\item {Proveniência:(De \textunderscore persigo\textunderscore )}
\end{itemize}
Manada de porcos.
Cortelho, curral.
\section{Persignação}
\begin{itemize}
\item {Grp. gram.:f.}
\end{itemize}
\begin{itemize}
\item {Proveniência:(De \textunderscore persignar-se\textunderscore )}
\end{itemize}
Acto de quem se persigna. Cf. Arn. Gama, \textunderscore Últ. Dona\textunderscore , 381.
\section{Persignar-se}
\begin{itemize}
\item {Grp. gram.:v. p.}
\end{itemize}
\begin{itemize}
\item {Proveniência:(Lat. \textunderscore persignare\textunderscore )}
\end{itemize}
Benzer-se, fazendo com o dedo pollegar da mão direita uma cruz na testa, outra na bôca e outra no peito.
\section{Persigo}
\begin{itemize}
\item {Grp. gram.:m.}
\end{itemize}
O mesmo que \textunderscore presigo\textunderscore .
\section{Persina}
\begin{itemize}
\item {Grp. gram.:f.}
\end{itemize}
\begin{itemize}
\item {Utilização:Prov.}
\end{itemize}
\begin{itemize}
\item {Utilização:trasm.}
\end{itemize}
Importunação, como a de uma chusma de mendigos que nos perseguem.
\section{Pérsio}
\begin{itemize}
\item {Grp. gram.:adj.}
\end{itemize}
\begin{itemize}
\item {Grp. gram.:M.}
\end{itemize}
O mesmo que \textunderscore persiano\textunderscore  e \textunderscore pérsico\textunderscore .
Indivíduo natural da Pérsia:«\textunderscore sabem discernir um do outro os Pérsios, os Arábios...\textunderscore ». Garcia Orta, \textunderscore Collóq.\textunderscore  II.
\section{Persistência}
\begin{itemize}
\item {Grp. gram.:f.}
\end{itemize}
Qualidade ou acto do que é persistente; constância; perseverança.
\section{Persistente}
\begin{itemize}
\item {Grp. gram.:adj.}
\end{itemize}
\begin{itemize}
\item {Proveniência:(Lat. \textunderscore persistens\textunderscore )}
\end{itemize}
Que persiste.
Constante.
Teimoso; contumaz.
\section{Persistir}
\begin{itemize}
\item {Grp. gram.:v. i.}
\end{itemize}
\begin{itemize}
\item {Proveniência:(Lat. \textunderscore persistere\textunderscore )}
\end{itemize}
Continuar a existir.
Durar; permanecer.
Perseverar.
\section{Persôa}
\begin{itemize}
\item {Grp. gram.:f.}
\end{itemize}
(Fórma ant. de \textunderscore pessôa\textunderscore . Cf. Frei Fortun., \textunderscore Inéd.\textunderscore , 311)
\section{Persoitar}
\begin{itemize}
\item {Grp. gram.:v. i.}
\end{itemize}
\begin{itemize}
\item {Utilização:Prov.}
\end{itemize}
\begin{itemize}
\item {Utilização:dur.}
\end{itemize}
\begin{itemize}
\item {Proveniência:(De \textunderscore per\textunderscore  + \textunderscore soito\textunderscore )}
\end{itemize}
Andar nos soitos, ao rebusco das castanhas.
\section{Persolver}
\begin{itemize}
\item {Grp. gram.:v. t.}
\end{itemize}
\begin{itemize}
\item {Proveniência:(Lat. \textunderscore persolvere\textunderscore )}
\end{itemize}
Pagar inteiramente, desobrigar-se de.
\section{Personadas}
\begin{itemize}
\item {Grp. gram.:m. pl.}
\end{itemize}
\begin{itemize}
\item {Proveniência:(Lat. \textunderscore personata\textunderscore )}
\end{itemize}
Família de plantas dicotyledóneas, o mesmo que \textunderscore escrofularíneas\textunderscore .
\section{Personado}
\begin{itemize}
\item {Grp. gram.:adj.}
\end{itemize}
\begin{itemize}
\item {Utilização:Bot.}
\end{itemize}
\begin{itemize}
\item {Proveniência:(Lat. \textunderscore personatus\textunderscore )}
\end{itemize}
O mesmo que \textunderscore mascarino\textunderscore .
\section{Personagem}
\begin{itemize}
\item {Grp. gram.:f.}
\end{itemize}
\begin{itemize}
\item {Proveniência:(Do lat. \textunderscore persona\textunderscore )}
\end{itemize}
Pessôa importante, notável.
Figura dramática.
Cada uma das pessôas, que figuram numa narração, num poêma ou num acontecimento.--Como t. masculino, ouve-se amiúde e lê-se algumas vezes, mas é gallicismo.
\section{Personalidade}
\begin{itemize}
\item {Grp. gram.:f.}
\end{itemize}
\begin{itemize}
\item {Proveniência:(Do lat. \textunderscore personalis\textunderscore )}
\end{itemize}
Qualidade do que é pessoal.
Carácter exclusivo e essencial de uma pessôa.
Aquillo que a distingue de outra.
\section{Personalismo}
\begin{itemize}
\item {Grp. gram.:m.}
\end{itemize}
\begin{itemize}
\item {Utilização:Neol.}
\end{itemize}
Qualidade de pessoal, de subjectivo.
(Cp. \textunderscore personalidade\textunderscore )
\section{Personalíssimo}
\begin{itemize}
\item {Grp. gram.:adj.}
\end{itemize}
\begin{itemize}
\item {Proveniência:(Do lat. \textunderscore personalis\textunderscore )}
\end{itemize}
Muito pessoal; muito subjectivo. Cf. Herculano, \textunderscore Opúsc.\textunderscore , III, 270.
\section{Personalização}
\begin{itemize}
\item {Grp. gram.:f.}
\end{itemize}
Acto ou effeito de personalizar.
\section{Personalizar}
\begin{itemize}
\item {Grp. gram.:v. t.}
\end{itemize}
\begin{itemize}
\item {Grp. gram.:V. i.}
\end{itemize}
\begin{itemize}
\item {Proveniência:(Do lat. \textunderscore personalis\textunderscore )}
\end{itemize}
Personificar; nomear ou indicar a pessôa de.
Fazer allusões injuriosas.
\section{Personária}
\begin{itemize}
\item {Grp. gram.:f.}
\end{itemize}
\begin{itemize}
\item {Utilização:Ant.}
\end{itemize}
\begin{itemize}
\item {Proveniência:(Do lat. \textunderscore persona\textunderscore )}
\end{itemize}
Procuração bastante; representação pessoal.
\section{Persoutar}
\begin{itemize}
\item {Grp. gram.:v. i.}
\end{itemize}
\begin{itemize}
\item {Utilização:Prov.}
\end{itemize}
\begin{itemize}
\item {Utilização:dur.}
\end{itemize}
\begin{itemize}
\item {Proveniência:(De \textunderscore per\textunderscore  + \textunderscore soito\textunderscore )}
\end{itemize}
Andar nos soitos, ao rebusco das castanhas.
\section{Personato}
\begin{itemize}
\item {Grp. gram.:adj.}
\end{itemize}
\begin{itemize}
\item {Proveniência:(Lat. \textunderscore personatus\textunderscore )}
\end{itemize}
Dizia-se das comédias, cujos actores se apresentavam mascarados em scena.
\section{Personificação}
\begin{itemize}
\item {Grp. gram.:f.}
\end{itemize}
Acto ou effeito de personificar.
Pessôa, que representa um princípio, uma ideia, etc.
Prosopopeia.
\section{Personificar}
\begin{itemize}
\item {Grp. gram.:v. t.}
\end{itemize}
\begin{itemize}
\item {Proveniência:(Do lat. \textunderscore persona\textunderscore  + \textunderscore facere\textunderscore )}
\end{itemize}
Considerar como pessôa; attribuir qualidades pessoaes a.
Representar por meio de uma pessôa.
Sêr o modêlo de: \textunderscore personificar a virtude\textunderscore .
\section{Persoterrar}
\begin{itemize}
\item {Grp. gram.:v. t.}
\end{itemize}
\begin{itemize}
\item {Utilização:Ant.}
\end{itemize}
\begin{itemize}
\item {Proveniência:(De \textunderscore per...\textunderscore  + \textunderscore soterrar\textunderscore )}
\end{itemize}
Enterrar completamente.
Terminar os offícios fúnebres por alma de.
\section{Perspéctico}
\begin{itemize}
\item {Grp. gram.:adj.}
\end{itemize}
\begin{itemize}
\item {Proveniência:(Do lat. \textunderscore perspectus\textunderscore )}
\end{itemize}
Relativo á perspectiva.
\section{Perspectiva}
\begin{itemize}
\item {Grp. gram.:f.}
\end{itemize}
\begin{itemize}
\item {Utilização:Mathem.}
\end{itemize}
\begin{itemize}
\item {Proveniência:(Lat. \textunderscore perspectiva\textunderscore )}
\end{itemize}
Sciência, que ensina a representar num plano os objectos, de maneira que elles se mostrem, guardadas as distâncias e situações.
Pintura, que representa edifícios ou paisagens a distância, e que, collocada na extremidade de uma galeria ou alameda, illude agradavelmente a vista; panorama.
Aspecto dos objectos, observados de longe.
Apparência.
Probabilidade.
Concha univalve.
\textunderscore Eixo de perspectiva\textunderscore , linha recta, que representa o lugar geométrico dos pontos de intersecção dos raios homólogos de dois feixes perspectivos de rectas.
\section{Perspectivação}
\begin{itemize}
\item {Grp. gram.:f.}
\end{itemize}
Acto ou effeito de perspectivar.
\section{Perspectivar}
\begin{itemize}
\item {Grp. gram.:v. t.}
\end{itemize}
\begin{itemize}
\item {Utilização:Neol.}
\end{itemize}
Pôr em perspectiva.
\section{Perspectivo}
\begin{itemize}
\item {Grp. gram.:adj.}
\end{itemize}
\begin{itemize}
\item {Utilização:Des.}
\end{itemize}
O mesmo que \textunderscore perspéctico\textunderscore .
\section{Perspectógrafo}
\begin{itemize}
\item {Grp. gram.:m.}
\end{itemize}
\begin{itemize}
\item {Utilização:Mathem.}
\end{itemize}
\begin{itemize}
\item {Proveniência:(T. hybr., de \textunderscore perspéctiva\textunderscore  + gr. \textunderscore graphein\textunderscore )}
\end{itemize}
Aparelho auxiliar, para fazer o desenho de sólidos em perspectiva.
\section{Perspectógrapho}
\begin{itemize}
\item {Grp. gram.:m.}
\end{itemize}
\begin{itemize}
\item {Utilização:Mathem.}
\end{itemize}
\begin{itemize}
\item {Proveniência:(T. hybr., de \textunderscore perspéctiva\textunderscore  + gr. \textunderscore graphein\textunderscore )}
\end{itemize}
Apparelho auxiliar, para fazer o desenho de sólidos em perspectiva.
\section{Perspicácia}
\begin{itemize}
\item {Grp. gram.:f.}
\end{itemize}
\begin{itemize}
\item {Proveniência:(Lat. \textunderscore perspicacia\textunderscore )}
\end{itemize}
Qualidade de quem é perspicaz.
\section{Perspicaz}
\begin{itemize}
\item {Grp. gram.:adj.}
\end{itemize}
\begin{itemize}
\item {Proveniência:(Lat. \textunderscore perspicax\textunderscore )}
\end{itemize}
Que vê bem.
Sagaz.
Penetrante: \textunderscore vista perspicaz\textunderscore .
Que tem agudeza de espírito; talentoso.
\section{Perspicazmente}
\begin{itemize}
\item {Grp. gram.:adv.}
\end{itemize}
De modo perspicaz.
\section{Perspicuamente}
\begin{itemize}
\item {Grp. gram.:adv.}
\end{itemize}
De modo perspícuo.
\section{Perspicuidade}
\begin{itemize}
\item {fónica:cu-i}
\end{itemize}
\begin{itemize}
\item {Grp. gram.:f.}
\end{itemize}
\begin{itemize}
\item {Proveniência:(Lat. \textunderscore perspicuitas\textunderscore )}
\end{itemize}
Qualidade do que é perspícuo.
\section{Perspícuo}
\begin{itemize}
\item {Grp. gram.:adj.}
\end{itemize}
\begin{itemize}
\item {Proveniência:(Lat. \textunderscore perspicuus\textunderscore )}
\end{itemize}
Que se póde vêr nitidamente; claro; evidente.
O mesmo que \textunderscore perspicaz\textunderscore :«\textunderscore ...ôlho perspícuo...\textunderscore »Camillo, \textunderscore Volcões\textunderscore , 21.
\section{Perspiração}
\begin{itemize}
\item {Grp. gram.:f.}
\end{itemize}
\begin{itemize}
\item {Utilização:Med.}
\end{itemize}
\begin{itemize}
\item {Proveniência:(Do lat. \textunderscore perspirare\textunderscore )}
\end{itemize}
Exhalação insensível, á superfície da pelle ou de uma membrana serosa.
\section{Perspirar}
\begin{itemize}
\item {Grp. gram.:v. i.}
\end{itemize}
\begin{itemize}
\item {Utilização:Med.}
\end{itemize}
\begin{itemize}
\item {Proveniência:(Lat. \textunderscore perspirare\textunderscore )}
\end{itemize}
Transpirar insensivelmente em toda a superfície.
\section{Perspiratório}
\begin{itemize}
\item {Grp. gram.:adj.}
\end{itemize}
\begin{itemize}
\item {Utilização:Med.}
\end{itemize}
Resultante da perspiração.
(Cp. \textunderscore perspiração\textunderscore )
\section{Perstricção}
\begin{itemize}
\item {Grp. gram.:f.}
\end{itemize}
\begin{itemize}
\item {Utilização:Med.}
\end{itemize}
\begin{itemize}
\item {Proveniência:(Lat. \textunderscore perstrictio\textunderscore )}
\end{itemize}
Applicação de ligaduras muito apertadas.
\section{Persuadição}
\begin{itemize}
\item {Grp. gram.:f.}
\end{itemize}
\begin{itemize}
\item {Utilização:P. us.}
\end{itemize}
\begin{itemize}
\item {Proveniência:(De \textunderscore persuadir\textunderscore )}
\end{itemize}
O mesmo que \textunderscore persuasão\textunderscore . Cf. F. Manuel, \textunderscore Carta de Guia\textunderscore , 92.
\section{Persuadimento}
\begin{itemize}
\item {Grp. gram.:m.}
\end{itemize}
O mesmo que \textunderscore persuasão\textunderscore .
\section{Persuadir}
\begin{itemize}
\item {Grp. gram.:v. t.}
\end{itemize}
\begin{itemize}
\item {Grp. gram.:V. i.}
\end{itemize}
\begin{itemize}
\item {Grp. gram.:V. p.}
\end{itemize}
\begin{itemize}
\item {Proveniência:(Lat. \textunderscore persuadere\textunderscore )}
\end{itemize}
Determinar a vontade de.
Levar a crer.
Induzir; aconselhar.
Produzir convicção.
Têr ou adquirir a convicção.
Estar sciente.
Formar juizo.--O \textunderscore v. p.\textunderscore  pede geralmente depois de si \textunderscore de\textunderscore  ou \textunderscore que\textunderscore :«\textunderscore persuado-me disto;\textunderscore »«\textunderscore persuado-me de que é verdade;\textunderscore »«\textunderscore persuado-me que é verdade.\textunderscore »Raramente é seguido da prep. \textunderscore a\textunderscore , mas o seu emprêgo é clássico:«\textunderscore ...pelo que me não posso persuadir a outra coisa senão que este rescripto foi negociado...\textunderscore ». Sousa, \textunderscore Vida do Arceb.\textunderscore , II, 137.
\section{Persuadível}
\begin{itemize}
\item {Grp. gram.:adj.}
\end{itemize}
Que se persuade facilmente.
\section{Persuasão}
\begin{itemize}
\item {Grp. gram.:f.}
\end{itemize}
\begin{itemize}
\item {Proveniência:(Lat. \textunderscore persuasio\textunderscore )}
\end{itemize}
Acto ou effeito de persuadir.
Convicção.
\section{Persuasiva}
\begin{itemize}
\item {Grp. gram.:f.}
\end{itemize}
Habilidade de persuadir.
(Fem. de \textunderscore persuasivo\textunderscore )
\section{Persuasível}
\begin{itemize}
\item {Grp. gram.:adj.}
\end{itemize}
Que persuade.
\section{Persuasivo}
\begin{itemize}
\item {Grp. gram.:adj.}
\end{itemize}
\begin{itemize}
\item {Proveniência:(Do lat. \textunderscore persuasus\textunderscore )}
\end{itemize}
Que persuade.
\section{Persuasor}
\begin{itemize}
\item {Grp. gram.:m.  e  adj.}
\end{itemize}
\begin{itemize}
\item {Proveniência:(Lat. \textunderscore persuasor\textunderscore )}
\end{itemize}
O que persuade.
\section{Persuasória}
\begin{itemize}
\item {Grp. gram.:f.}
\end{itemize}
Motivo, que persuade.
(Fem. de \textunderscore persuasório\textunderscore )
\section{Persuasório}
\begin{itemize}
\item {Grp. gram.:adj.}
\end{itemize}
\begin{itemize}
\item {Proveniência:(Lat. \textunderscore persuasorius\textunderscore )}
\end{itemize}
O mesmo que \textunderscore persuasivo\textunderscore .
\section{Persulfocianídrico}
\begin{itemize}
\item {Grp. gram.:adj.}
\end{itemize}
\begin{itemize}
\item {Utilização:Chím.}
\end{itemize}
Diz-se do ácido, produzido pela transformação do ácido sulfocianídrico, sob a influência dos ácidos mineraes.
\section{Persulfocianogênio}
\begin{itemize}
\item {Grp. gram.:m.}
\end{itemize}
\begin{itemize}
\item {Utilização:Chím.}
\end{itemize}
Ácido persulfocianídrico, em que um equivalente de cianogênio foi substituído por um equivalente de hidrogênio.
\section{Persulfocyanhýdrico}
\begin{itemize}
\item {fónica:ni}
\end{itemize}
\begin{itemize}
\item {Grp. gram.:adj.}
\end{itemize}
\begin{itemize}
\item {Utilização:Chím.}
\end{itemize}
Diz-se do ácido, produzido pela transformação do ácido sulfocyanhýdrico, sob a influência dos ácidos mineraes.
\section{Persulfocyanogênio}
\begin{itemize}
\item {Grp. gram.:m.}
\end{itemize}
\begin{itemize}
\item {Utilização:Chím.}
\end{itemize}
Ácido persulfocyanhýdrico, em que um equivalente de cyanogênio foi substituído por um equivalente de hydrogênio.
\section{Persulfureto}
\begin{itemize}
\item {fónica:furê}
\end{itemize}
\begin{itemize}
\item {Grp. gram.:m.}
\end{itemize}
Sulfureto que contém a maior porção possível de enxôfre.
\section{Pertença}
\begin{itemize}
\item {Grp. gram.:f.}
\end{itemize}
\begin{itemize}
\item {Proveniência:(De \textunderscore pertencer\textunderscore )}
\end{itemize}
Aquillo que faz parte de alguma coisa.
Attribuição.
Domínio.
Accessório: \textunderscore as pertenças de um prédio\textunderscore .
\section{Pertence}
\begin{itemize}
\item {Grp. gram.:m.}
\end{itemize}
\begin{itemize}
\item {Proveniência:(De \textunderscore pertencer\textunderscore )}
\end{itemize}
O mesmo que \textunderscore pertença\textunderscore .
Declaração, que se faz em certos títulos, designando a pessôa a quem se transmitte a propriedade dos mesmos.
\section{Pertencente}
\begin{itemize}
\item {Grp. gram.:adj.}
\end{itemize}
Que pertence.
Relativo, concernente.
\section{Pertencer}
\begin{itemize}
\item {Grp. gram.:v. i.}
\end{itemize}
\begin{itemize}
\item {Proveniência:(Do lat. hyp. \textunderscore pertinescere\textunderscore )}
\end{itemize}
Sêr parte.
Sêr propriedade: \textunderscore êste livro pertence-me\textunderscore .
Caber.
Sêr da jurisdição ou obrigação de alguém: \textunderscore pertence aos pais o curar dos filhos\textunderscore .
Sêr relativo, dizer respeito.
\section{Pertencimento}
\begin{itemize}
\item {Grp. gram.:m.}
\end{itemize}
\begin{itemize}
\item {Utilização:Ant.}
\end{itemize}
O mesmo que \textunderscore pertença\textunderscore .
\section{Pertentar}
\begin{itemize}
\item {Grp. gram.:v. t.}
\end{itemize}
\begin{itemize}
\item {Proveniência:(De \textunderscore per...\textunderscore  + \textunderscore tentar\textunderscore )}
\end{itemize}
Tornar a tentar; tentar muitas vezes:«\textunderscore tenta a morte vencer; pertenta e balda...\textunderscore »Castilho, \textunderscore Metam.\textunderscore , 98.
\section{Pértica}
\begin{itemize}
\item {Grp. gram.:f.}
\end{itemize}
Medida gótica de dois palmos. Cf. Herculano, \textunderscore Eurico\textunderscore , 327.
\section{Pértiga}
\begin{itemize}
\item {Grp. gram.:f.}
\end{itemize}
O mesmo que \textunderscore pírtiga\textunderscore .
\section{Pertigueiro}
\begin{itemize}
\item {Grp. gram.:m.}
\end{itemize}
\begin{itemize}
\item {Utilização:Ant.}
\end{itemize}
\begin{itemize}
\item {Proveniência:(De \textunderscore pértiga\textunderscore )}
\end{itemize}
O mesmo que \textunderscore alferes\textunderscore  ou \textunderscore porta-bandeira\textunderscore .
\section{Pertinácia}
\begin{itemize}
\item {Grp. gram.:f.}
\end{itemize}
\begin{itemize}
\item {Proveniência:(Lat. \textunderscore pertinacia\textunderscore )}
\end{itemize}
Qualidade do que é pertinaz.
\section{Pertinaz}
\begin{itemize}
\item {Grp. gram.:adj.}
\end{itemize}
\begin{itemize}
\item {Proveniência:(Lat. \textunderscore pertinax\textunderscore )}
\end{itemize}
Muito tenaz; obstinado, teimoso.
Persistente: \textunderscore doença pertinaz\textunderscore .
\section{Pertinazmente}
\begin{itemize}
\item {Grp. gram.:adv.}
\end{itemize}
De modo pertinaz.
\section{Pertinência}
\begin{itemize}
\item {Grp. gram.:f.}
\end{itemize}
\begin{itemize}
\item {Utilização:Des.}
\end{itemize}
\begin{itemize}
\item {Proveniência:(De \textunderscore pertinente\textunderscore )}
\end{itemize}
O mesmo que \textunderscore pertença\textunderscore .
\section{Pertinente}
\begin{itemize}
\item {Grp. gram.:adj.}
\end{itemize}
\begin{itemize}
\item {Proveniência:(Lat. \textunderscore pertinens\textunderscore )}
\end{itemize}
Pertencente.
Concernente.
Próprio; apropositado.
\section{Perto}
\begin{itemize}
\item {Grp. gram.:adv.}
\end{itemize}
\begin{itemize}
\item {Grp. gram.:Loc. prep.}
\end{itemize}
\begin{itemize}
\item {Grp. gram.:M. pl.}
\end{itemize}
\begin{itemize}
\item {Grp. gram.:Adj.}
\end{itemize}
A pequena distância: \textunderscore o Carlos mora perto\textunderscore .
\textunderscore Perto de\textunderscore , junto de, próximo de: \textunderscore gastou perto de quatro contos\textunderscore .
Objectos próximos.
Qualidades, que se distinguem, estando-se junto das pessôas ou objectos que as possuem:«\textunderscore ...mulhér com uns pertos de taberna...\textunderscore »\textunderscore Anat. Joc.\textunderscore , I, 258.
Que está a pequena distância; próximo:«\textunderscore ...dos mais pertos oiteiros...\textunderscore »Filinto, VI, 105.
(Talvez contr. do lat. \textunderscore perrectus\textunderscore )
\section{Pertucha}
\textunderscore f.\textunderscore  e (der.)
O mesmo que \textunderscore portucha\textunderscore , etc.
\section{Perturbabilidade}
\begin{itemize}
\item {Grp. gram.:f.}
\end{itemize}
Qualidade de perturbável.
\section{Perturbação}
\begin{itemize}
\item {Grp. gram.:f.}
\end{itemize}
\begin{itemize}
\item {Proveniência:(Lat. \textunderscore perturbatio\textunderscore )}
\end{itemize}
Acto ou effeito de perturbar.
Tontura.
Confusão, desordem.
\section{Perturbadamente}
\begin{itemize}
\item {Grp. gram.:adv.}
\end{itemize}
De modo perturbado; com perturbação.
\section{Perturbador}
\begin{itemize}
\item {Grp. gram.:m.  e  adj.}
\end{itemize}
\begin{itemize}
\item {Proveniência:(Lat. \textunderscore perturbator\textunderscore )}
\end{itemize}
O que perturba.
\section{Perturbar}
\begin{itemize}
\item {Grp. gram.:v. t.}
\end{itemize}
\begin{itemize}
\item {Proveniência:(Lat. \textunderscore perturbare\textunderscore )}
\end{itemize}
Alterar; desnortear.
Estontear.
Desarranjar.
Atrapalhar.
Commover.
Intimidar.
Confundir.
Agitar; desassossegar.
\section{Perturbável}
\begin{itemize}
\item {Grp. gram.:adj.}
\end{itemize}
Que se póde perturbar.
\section{Perturbativo}
\begin{itemize}
\item {Grp. gram.:adj.}
\end{itemize}
\begin{itemize}
\item {Proveniência:(Do lat. \textunderscore perturbatus\textunderscore )}
\end{itemize}
Que perturba.
\section{Perturbatório}
\begin{itemize}
\item {Grp. gram.:adj.}
\end{itemize}
\begin{itemize}
\item {Proveniência:(Do lat. \textunderscore perturbatus\textunderscore )}
\end{itemize}
Perturbativo.
Oscillatório.
\section{Pertuso}
\begin{itemize}
\item {Grp. gram.:adj.}
\end{itemize}
\begin{itemize}
\item {Utilização:Bot.}
\end{itemize}
\begin{itemize}
\item {Proveniência:(Lat. \textunderscore pertusus\textunderscore , perfurado)}
\end{itemize}
Diz-se das fôlhas vegetaes, em que há interrupção do limbo. Cf. Caminhoá, \textunderscore Bot. Ger. e Med.\textunderscore 
\section{Pertussina}
\begin{itemize}
\item {Grp. gram.:f.}
\end{itemize}
\begin{itemize}
\item {Utilização:Pharm.}
\end{itemize}
Extracto de tomilho, applicado contra bronchites, etc.
(Por \textunderscore pertusina\textunderscore , de \textunderscore pertuso\textunderscore ?)
\section{Peru}
\begin{itemize}
\item {Grp. gram.:m.}
\end{itemize}
\begin{itemize}
\item {Utilização:Bras}
\end{itemize}
\begin{itemize}
\item {Utilização:Bras}
\end{itemize}
\begin{itemize}
\item {Utilização:Bras. do Rio}
\end{itemize}
Grande ave gallinácea, (\textunderscore meleagris\textunderscore )
Cortejador, namorado ridículo.
Aquelle que está vendo jogar cartas; mirão.
Grande embarcação, em fórma de canôa.
\section{Peru}
\begin{itemize}
\item {Grp. gram.:m.}
\end{itemize}
Língua dos indígenas do Peru; quichua.
\section{Perua}
\begin{itemize}
\item {Grp. gram.:f.}
\end{itemize}
\begin{itemize}
\item {Utilização:Pop.}
\end{itemize}
A fêmea do peru.
Embriaguez.
(Fem. de \textunderscore peru\textunderscore )
\section{Peruanas}
\begin{itemize}
\item {Grp. gram.:m. pl.}
\end{itemize}
Gentios, que habitavam nas margens do Rio-Branco, no Brasil.
\section{Peruanismo}
\begin{itemize}
\item {Grp. gram.:m.}
\end{itemize}
\begin{itemize}
\item {Proveniência:(De \textunderscore peruano\textunderscore )}
\end{itemize}
Expressão privativa do Peru.
\section{Peruano}
\begin{itemize}
\item {Grp. gram.:adj.}
\end{itemize}
\begin{itemize}
\item {Grp. gram.:M.}
\end{itemize}
Relativo á república do Peru ou aos seus habitantes.
Habitante do Peru.
\section{Peruar}
\begin{itemize}
\item {Grp. gram.:v. t.}
\end{itemize}
\begin{itemize}
\item {Utilização:Bras}
\end{itemize}
\begin{itemize}
\item {Proveniência:(De \textunderscore peru\textunderscore ^1)}
\end{itemize}
Cortejar, requestar.
Assistir a (jôgo de cartas).
\section{Peruca}
\begin{itemize}
\item {Grp. gram.:f.}
\end{itemize}
\begin{itemize}
\item {Proveniência:(Do fr. \textunderscore perruque\textunderscore )}
\end{itemize}
Cabelleira postiça.
\section{Peru-do-mato}
\begin{itemize}
\item {Grp. gram.:m.}
\end{itemize}
Espécie de tucano da África, (\textunderscore bucorax caffer\textunderscore , Bocage).
\section{Perueiro}
\begin{itemize}
\item {Grp. gram.:adj.}
\end{itemize}
Relativo a peru:«\textunderscore ...a ninhada perueira\textunderscore ». Filinto, XIII, 310.
\section{Pérula}
\begin{itemize}
\item {Grp. gram.:f.}
\end{itemize}
\begin{itemize}
\item {Utilização:Bot.}
\end{itemize}
\begin{itemize}
\item {Proveniência:(Lat. \textunderscore perula\textunderscore )}
\end{itemize}
Invólucro dos botões ou gomos, no systema de Mysbel.
\section{Perulado}
\begin{itemize}
\item {Grp. gram.:adj.}
\end{itemize}
\begin{itemize}
\item {Utilização:Bot.}
\end{itemize}
Que tem pérula.
\section{Peruleira}
\begin{itemize}
\item {Grp. gram.:f.}
\end{itemize}
\begin{itemize}
\item {Utilização:Ant.}
\end{itemize}
Vasilha de almude.
(Cp. cast. \textunderscore perulero\textunderscore )
\section{Perum}
\begin{itemize}
\item {Grp. gram.:m.}
\end{itemize}
\begin{itemize}
\item {Utilização:Pop.}
\end{itemize}
O mesmo que \textunderscore peru\textunderscore ^1. Cf. Filinto, VIII, 30.
\section{Perunca}
\begin{itemize}
\item {Grp. gram.:f.}
\end{itemize}
\begin{itemize}
\item {Utilização:Chul.}
\end{itemize}
O mesmo que \textunderscore bebedeira\textunderscore .
(Cp. \textunderscore perua\textunderscore  e \textunderscore peruca\textunderscore )
\section{Perunzete}
\begin{itemize}
\item {fónica:zê}
\end{itemize}
\begin{itemize}
\item {Grp. gram.:m.}
\end{itemize}
Pequeno perum; perum muito novo. Cf. Filinto, XIII, 309.
\section{Peruruca}
\begin{itemize}
\item {Grp. gram.:f.}
\end{itemize}
\begin{itemize}
\item {Utilização:Bras}
\end{itemize}
Espécie de milho.
\section{Perusino}
\begin{itemize}
\item {Grp. gram.:m.  e  adj.}
\end{itemize}
\begin{itemize}
\item {Proveniência:(Lat. \textunderscore perusinus\textunderscore )}
\end{itemize}
Aquelle que é natural de Perusa.
\section{Peruviano}
\begin{itemize}
\item {Grp. gram.:m.  e  adj.}
\end{itemize}
O mesmo que \textunderscore peruano\textunderscore .
\section{Pervagar}
\begin{itemize}
\item {Grp. gram.:v. t.}
\end{itemize}
\begin{itemize}
\item {Proveniência:(Lat. \textunderscore pervagari\textunderscore )}
\end{itemize}
Percorrer em várias direcções.
Cruzar; atravessar. Cf. Alv. Mendes, \textunderscore Discursos\textunderscore , 206.
\section{Pervage}
\begin{itemize}
\item {Grp. gram.:f.}
\end{itemize}
\begin{itemize}
\item {Utilização:Prov.}
\end{itemize}
\begin{itemize}
\item {Utilização:minh.}
\end{itemize}
Mergulhão de videira.
\section{Pervencer}
\begin{itemize}
\item {Grp. gram.:v. t.}
\end{itemize}
\begin{itemize}
\item {Utilização:Ant.}
\end{itemize}
\begin{itemize}
\item {Proveniência:(Lat. \textunderscore pervincere\textunderscore )}
\end{itemize}
Subverter.
Quebrar.
Destruír.
Vencer completamente.
\section{Perversamente}
\begin{itemize}
\item {Grp. gram.:adv.}
\end{itemize}
De modo perverso; com perversidade.
\section{Perversão}
\begin{itemize}
\item {Grp. gram.:f.}
\end{itemize}
\begin{itemize}
\item {Proveniência:(Lat. \textunderscore perversio\textunderscore )}
\end{itemize}
Acto ou effeito de perverter.
Corrupção.
Passagem para peór estado: \textunderscore perversão de sensibilidade\textunderscore .
\section{Perversidade}
\begin{itemize}
\item {Grp. gram.:f.}
\end{itemize}
\begin{itemize}
\item {Proveniência:(Lat. \textunderscore perversitas\textunderscore )}
\end{itemize}
Qualidade do que é perverso.
Índole ruím ou ferina; fereza.
\section{Perverso}
\begin{itemize}
\item {Grp. gram.:adj.}
\end{itemize}
\begin{itemize}
\item {Proveniência:(Lat. \textunderscore perversus\textunderscore )}
\end{itemize}
Que tem muito má índole; mau, traiçoeiro; ferino.
\section{Perversor}
\begin{itemize}
\item {Grp. gram.:m.  e  adj.}
\end{itemize}
\begin{itemize}
\item {Proveniência:(De \textunderscore perverso\textunderscore )}
\end{itemize}
O mesmo que \textunderscore pervertedor\textunderscore . Cf. Latino, \textunderscore Or. da Corôa\textunderscore , CCXXVI.
\section{Pervertedor}
\begin{itemize}
\item {Grp. gram.:m.  e  adj.}
\end{itemize}
O que perverte.
\section{Perverter}
\begin{itemize}
\item {Grp. gram.:v. t.}
\end{itemize}
\begin{itemize}
\item {Proveniência:(Lat. \textunderscore pervertere\textunderscore )}
\end{itemize}
Tornar perverso; tornar peór.
Corromper, depravar: \textunderscore há livros que pervertem incautos\textunderscore .
Alterar, transtornar.
Desvirtuar; interpretar mal: \textunderscore perverter o sentido de uma phrase\textunderscore .
\section{Pervicácia}
\begin{itemize}
\item {Grp. gram.:f.}
\end{itemize}
\begin{itemize}
\item {Proveniência:(Lat. \textunderscore pervicatia\textunderscore )}
\end{itemize}
Qualidade do que é pervicaz. Cf. \textunderscore Ulysséa\textunderscore , X, 9.
\section{Pervicaz}
\begin{itemize}
\item {Grp. gram.:adj.}
\end{itemize}
\begin{itemize}
\item {Utilização:Des.}
\end{itemize}
\begin{itemize}
\item {Proveniência:(Lat. \textunderscore pervicax\textunderscore )}
\end{itemize}
O mesmo que \textunderscore pertinaz\textunderscore .
\section{Pervicazmente}
\begin{itemize}
\item {Grp. gram.:adv.}
\end{itemize}
De modo pervicaz.
\section{Pervígil}
\begin{itemize}
\item {Grp. gram.:m.}
\end{itemize}
\begin{itemize}
\item {Proveniência:(Lat. \textunderscore pervígil\textunderscore )}
\end{itemize}
Aquelle que não dorme:«\textunderscore Pereira, o pervígil diurno e nocturno.\textunderscore »Camillo, \textunderscore Noites de Insómn.\textunderscore 
\section{Pervigília}
\begin{itemize}
\item {Grp. gram.:f.}
\end{itemize}
\begin{itemize}
\item {Proveniência:(Lat. \textunderscore pervigilia\textunderscore )}
\end{itemize}
Grande vigília.
(Us. por Castilho.)
\section{Pervinca}
\begin{itemize}
\item {Grp. gram.:f.}
\end{itemize}
\begin{itemize}
\item {Proveniência:(Lat. \textunderscore pervinca\textunderscore )}
\end{itemize}
Planta apocýnea, (\textunderscore vinca major\textunderscore ).
\section{Pervinca}
\begin{itemize}
\item {Grp. gram.:f.}
\end{itemize}
\begin{itemize}
\item {Utilização:Marn.}
\end{itemize}
Faixa de terreno, sôbre a maracha, para o lado interior da peça da salina.
\section{Pervinco}
\begin{itemize}
\item {Grp. gram.:m.}
\end{itemize}
\begin{itemize}
\item {Utilização:Ant.}
\end{itemize}
Parente muito próximo. Cf. Figanière, \textunderscore G. Ansures\textunderscore .
\section{Pérvio}
\begin{itemize}
\item {Grp. gram.:adj.}
\end{itemize}
\begin{itemize}
\item {Proveniência:(Lat. \textunderscore pervius\textunderscore )}
\end{itemize}
Que dá passagem; em que se póde transitar.
Franco, patente.
\section{Perxina}
\begin{itemize}
\item {Grp. gram.:f.}
\end{itemize}
Triângulo curvilíneo, que faz parte de uma abóbada, reforçando-a.--Vejo o t. nos outros diccion., mas não o vejo abonado nem conheço a sua razão morphológica. Os mais ponderados citam-lhe, como fonte, o cast. \textunderscore pechina\textunderscore , mas não conheço êste voc. no cast.; e, se existisse e fôsse a fonte do nosso, teríamos \textunderscore perchina\textunderscore  e não \textunderscore perxina\textunderscore .
\section{Pêsa}
\begin{itemize}
\item {Grp. gram.:f.}
\end{itemize}
\begin{itemize}
\item {Utilização:Prov.}
\end{itemize}
\begin{itemize}
\item {Utilização:trasm.}
\end{itemize}
Manhuça (de \textunderscore linho\textunderscore ), feixe de doze estrigas espadeladas.
(Cast. \textunderscore pesa\textunderscore )
\section{Pesa-aguardente}
\begin{itemize}
\item {Grp. gram.:m.}
\end{itemize}
O mesmo que \textunderscore alcoómetro\textunderscore .
\section{Pesa-cartas}
\begin{itemize}
\item {Grp. gram.:m.}
\end{itemize}
Pequeno apparelho, para indicar o pêso das cartas.
\section{Pesada}
\begin{itemize}
\item {Grp. gram.:f.}
\end{itemize}
\begin{itemize}
\item {Utilização:Chul.}
\end{itemize}
Aquillo que se pesa de uma vez em balança.
A mão.
(Fem. de \textunderscore pesado\textunderscore )
\section{Pesadamente}
\begin{itemize}
\item {Grp. gram.:adv.}
\end{itemize}
De modo pesado.
Vagarosamente.
Monotonamente; sem graça, sem vivacidade.
\section{Pesadão}
\begin{itemize}
\item {Grp. gram.:adj.}
\end{itemize}
Muito pesado.
Que anda com difficuldade, por causa da gordura.
Mollangueiro. Cf. Baganha, \textunderscore Hyg. Pec.\textunderscore , 81.
\section{Pesadelo}
\begin{itemize}
\item {fónica:dê}
\end{itemize}
\begin{itemize}
\item {Grp. gram.:m.}
\end{itemize}
\begin{itemize}
\item {Utilização:Fig.}
\end{itemize}
\begin{itemize}
\item {Proveniência:(De \textunderscore pesado\textunderscore )}
\end{itemize}
Agitação ou oppressão, que se dá durante o somno, em resultado de sonhos afflictivos.
Mau sonho.
Marasmo.
Pessôa importuna.
Coisa, que enfada ou molesta.
\section{Pesadinho}
\begin{itemize}
\item {Grp. gram.:adj.  Loc.}
\end{itemize}
\begin{itemize}
\item {Utilização:Loc. de Lamego.}
\end{itemize}
\begin{itemize}
\item {Proveniência:(De \textunderscore pesado\textunderscore )}
\end{itemize}
\textunderscore Andar pesadinho a cera\textunderscore , andar arriscado, andar em perigo, andar muito doente.--É allusão á promessa, que as pessôas devotas fazem aos santos, de lhes offerecer, em cera, o pêso de uma pessôa querida, se esta escapar de doença grave.
\section{Pesado}
\begin{itemize}
\item {Grp. gram.:adj.}
\end{itemize}
\begin{itemize}
\item {Utilização:Pop.}
\end{itemize}
Que tem muito pêso.
Trabalhoso.
Aborrecido, molesto: \textunderscore tornar-se pesado a alguém\textunderscore .
Lento, vagaroso.
Que não tem elegância ou vivacidade.
Grosseiro; injurioso.
Diffícil de digerir: \textunderscore comida pesada\textunderscore .
\section{Pesador}
\begin{itemize}
\item {Grp. gram.:m.  e  adj.}
\end{itemize}
O que pesa, o que serve para pesar ou para se calcular o pêso de alguma coisa.
\section{Pesadora}
\begin{itemize}
\item {Grp. gram.:f.}
\end{itemize}
\begin{itemize}
\item {Proveniência:(De \textunderscore pesador\textunderscore )}
\end{itemize}
Máquina para pesar. Cf. \textunderscore Inquér. Industr.\textunderscore , III parte, 147.
\section{Pesadume}
\begin{itemize}
\item {Grp. gram.:m.}
\end{itemize}
Carga, pêso.
Azedume.
Má vontade.
Tristeza, pesar.
(Cp. cast. \textunderscore pesadumbre\textunderscore )
\section{Pesagem}
\begin{itemize}
\item {Grp. gram.:m.}
\end{itemize}
Acto de pesar.
Lugar, em que se pesam os cavalleiros, em corridas de cavallos.
\section{Pesa-leite}
\begin{itemize}
\item {Grp. gram.:m.}
\end{itemize}
Designação vulgar do galactómetro.
\section{Pesa-licor}
\begin{itemize}
\item {Grp. gram.:m.}
\end{itemize}
Designação vulgar do areómetro.
\section{Pêsame}
\begin{itemize}
\item {Grp. gram.:m.}
\end{itemize}
Expressão de sentimentos ou pesar, pelo fallecimento de alguém ou por qualquer infortúnio.
(Relaciona-se com \textunderscore pesar\textunderscore )
\section{Pêsames}
\begin{itemize}
\item {Grp. gram.:m. pl.}
\end{itemize}
Expressão de sentimentos ou pesar, pelo fallecimento de alguém ou por qualquer infortúnio.
(Relaciona-se com \textunderscore pesar\textunderscore )
\section{Pesa-mosto}
\begin{itemize}
\item {Grp. gram.:m.}
\end{itemize}
Espécie de areómetro, de applicações análogas ás do gleucómetro.
\section{Pesante}
\begin{itemize}
\item {Grp. gram.:m.}
\end{itemize}
\begin{itemize}
\item {Grp. gram.:Adj.}
\end{itemize}
\begin{itemize}
\item {Utilização:Ant.}
\end{itemize}
\begin{itemize}
\item {Proveniência:(De \textunderscore pesar\textunderscore )}
\end{itemize}
Antiga moéda portuguesa, de valor desconhecido.
Pesaroso, descontente.
\section{Pesar}
\begin{itemize}
\item {Grp. gram.:v. t.}
\end{itemize}
\begin{itemize}
\item {Utilização:Fig.}
\end{itemize}
\begin{itemize}
\item {Grp. gram.:V. i.}
\end{itemize}
\begin{itemize}
\item {Utilização:Fig.}
\end{itemize}
\begin{itemize}
\item {Grp. gram.:M.}
\end{itemize}
\begin{itemize}
\item {Grp. gram.:Interj.}
\end{itemize}
\begin{itemize}
\item {Utilização:ant.}
\end{itemize}
\begin{itemize}
\item {Proveniência:(Do lat. \textunderscore pensare\textunderscore )}
\end{itemize}
Determinar o pêso de.
Pôr na balança, para conhecer o pêso.
Sopesar.
Têr o pêso de.
Apreciar.
Ponderar, julgar.
Calcular.
Exercer pressão.
Fazer pêso: \textunderscore aquillo pesa muito\textunderscore .
Affligir.
Causar tristeza.
Sêr objecto de sentimento.
Sêr importuno, molesto.
Tristeza.
Desgôsto, sentimento.
Arrependimento.
\textunderscore Pesar do meu quinto avô\textunderscore ! ou \textunderscore pesar de Fés\textunderscore ! ou \textunderscore pesar dos moiros todos\textunderscore ! ainda bem! Cf. \textunderscore Eufrosina\textunderscore , 277, etc.
\section{Pesarosamente}
\begin{itemize}
\item {Grp. gram.:adv.}
\end{itemize}
De modo pesaroso; com pesar.
\section{Pesaroso}
\begin{itemize}
\item {Grp. gram.:adj.}
\end{itemize}
Que tem pesar; em que há pesar.
\section{Pesca}
\begin{itemize}
\item {Grp. gram.:f.}
\end{itemize}
\begin{itemize}
\item {Utilização:Ext.}
\end{itemize}
Acto ou arte de pescar: \textunderscore foi para a pesca\textunderscore .
Aquillo que se pescou: \textunderscore a pesca foi abundante\textunderscore .
Acto de tirar alguma coisa da água.
Investigação, procura.
\section{Pescada}
\begin{itemize}
\item {Grp. gram.:f.}
\end{itemize}
\begin{itemize}
\item {Utilização:Pop.}
\end{itemize}
\begin{itemize}
\item {Proveniência:(De \textunderscore pescar\textunderscore )}
\end{itemize}
Peixe malacopterýgio, (\textunderscore gadus meluccius\textunderscore ).
Mulhér anthipática.
\section{Pescada-carvoeira}
\begin{itemize}
\item {Grp. gram.:f.}
\end{itemize}
Peixe gádido, (\textunderscore gadus carbonarius\textunderscore ).
\section{Pescada-marlonga}
\begin{itemize}
\item {Grp. gram.:f.}
\end{itemize}
Peixe gádido, (\textunderscore gadus-merlangus\textunderscore ).
\section{Pescada-pollacha}
\begin{itemize}
\item {Grp. gram.:f.}
\end{itemize}
Peixe gádido, (\textunderscore gadus pollachius\textunderscore ).
\section{Pescada-preta}
\begin{itemize}
\item {Grp. gram.:f.}
\end{itemize}
Peixe escômbrida.
\section{Pescadeira}
\begin{itemize}
\item {Grp. gram.:f.}
\end{itemize}
\begin{itemize}
\item {Utilização:Ant.}
\end{itemize}
\begin{itemize}
\item {Proveniência:(De \textunderscore pescadeiro\textunderscore )}
\end{itemize}
Mulhér, que vende pescada; peixeira. Cf. \textunderscore Livro dos Acórdãos\textunderscore , da Câmara de Guimarães, (1692)
\section{Pescadeira}
\begin{itemize}
\item {Grp. gram.:f.}
\end{itemize}
\begin{itemize}
\item {Utilização:Des.}
\end{itemize}
\begin{itemize}
\item {Proveniência:(De \textunderscore pescar\textunderscore )}
\end{itemize}
Espécie de bomba, para tirar dos tonéis o vinho, que já não chega á altura da torneira.
\section{Pescadeiro}
\begin{itemize}
\item {Grp. gram.:m.}
\end{itemize}
\begin{itemize}
\item {Utilização:Ant.}
\end{itemize}
Vendedor de pescado; peixeiro.
\section{Pescadinha}
\begin{itemize}
\item {Grp. gram.:f.}
\end{itemize}
Peça de metal, cylíndrica e delgada, com que os chapeleiros fazem o rebordo aos chapéus.
\textunderscore Pescadinha de rabo na bôca\textunderscore , pescada pequena ou pescadinha-marmota, frita e preparada em fórma quási de anel, com o rabo preso na bôca.
\section{Pescadinha-marmota}
\begin{itemize}
\item {Grp. gram.:f.}
\end{itemize}
Pescada pequena, a que também se dá o simples nome de \textunderscore marmota\textunderscore .
\section{Pescado}
\begin{itemize}
\item {Grp. gram.:m.}
\end{itemize}
\begin{itemize}
\item {Proveniência:(De \textunderscore pescar\textunderscore )}
\end{itemize}
Aquillo que se pesca: \textunderscore o imposto sôbre o pescado\textunderscore .
Serviço municipal, relativo a pesca e pescadores.
\section{Pescador}
\begin{itemize}
\item {Grp. gram.:m.}
\end{itemize}
\begin{itemize}
\item {Grp. gram.:Adj.}
\end{itemize}
\begin{itemize}
\item {Proveniência:(Do lat. \textunderscore piscator\textunderscore )}
\end{itemize}
Aquelle que pesca.
O que vive de pescar.
O mesmo que \textunderscore martim-pescador\textunderscore .
Que pesca; próprio para pescar.
Relativo á pesca.
\section{Pescado-real}
\begin{itemize}
\item {Grp. gram.:m.}
\end{itemize}
\begin{itemize}
\item {Utilização:Ant.}
\end{itemize}
Peixe, o mesmo que \textunderscore linguado\textunderscore .
\section{Pescal}
\begin{itemize}
\item {Grp. gram.:m.}
\end{itemize}
\begin{itemize}
\item {Utilização:Prov.}
\end{itemize}
Pequena peça ou cavilha de ferro, que, entalada entre o cabo e o ôlho da enxada, assegura a posição correcta do mesmo cabo.
O mesmo que \textunderscore pescaz\textunderscore .
\section{Pescanço}
\begin{itemize}
\item {Grp. gram.:m.}
\end{itemize}
\begin{itemize}
\item {Utilização:Fam.}
\end{itemize}
\begin{itemize}
\item {Proveniência:(De \textunderscore pescar\textunderscore )}
\end{itemize}
Acto de espreitar o jôgo de um parceiro.
\section{Pescante}
\begin{itemize}
\item {Grp. gram.:m.}
\end{itemize}
\begin{itemize}
\item {Utilização:Des.}
\end{itemize}
Tábua, atrás das seges e coches, onde os criados vão em pé? Almofada, onde se senta o cocheiro? Cf. Garrett, \textunderscore Helena\textunderscore , 62.
\section{Pescar}
\begin{itemize}
\item {Grp. gram.:v. t.}
\end{itemize}
\begin{itemize}
\item {Utilização:Ext.}
\end{itemize}
\begin{itemize}
\item {Utilização:Pop.}
\end{itemize}
\begin{itemize}
\item {Grp. gram.:V. i.}
\end{itemize}
\begin{itemize}
\item {Utilização:Pop.}
\end{itemize}
\begin{itemize}
\item {Proveniência:(Do lat. \textunderscore piscari\textunderscore )}
\end{itemize}
Colhêr na água ou apanhar (peixe).
Apanhar, como se apanha o peixe.
Investigar, descobrir.
Conseguir com manha: \textunderscore pescar uma herança\textunderscore .
Perceber, comprehender.
Occupar-se da pesca.
Têr ideias, conhecimentos: \textunderscore não pescas nada disto\textunderscore .
\section{Pescarejo}
\begin{itemize}
\item {Grp. gram.:adj.}
\end{itemize}
\begin{itemize}
\item {Utilização:Des.}
\end{itemize}
\begin{itemize}
\item {Proveniência:(De \textunderscore pescar\textunderscore )}
\end{itemize}
Relativo a pesca; próprio para pesca.
\section{Pescarez}
\begin{itemize}
\item {Grp. gram.:adj.}
\end{itemize}
(V.pescarejo)
\section{Pescaria}
\begin{itemize}
\item {Grp. gram.:f.}
\end{itemize}
Arte de pescar.
O mesmo que \textunderscore pesca\textunderscore .
Grande quantidade de peixe.
\section{Pescaz}
\begin{itemize}
\item {Grp. gram.:m.}
\end{itemize}
Cunha, com que se une o arado á rabiça.
\section{Pescoçada}
\begin{itemize}
\item {Grp. gram.:f.}
\end{itemize}
Pancada no pescoço.
\section{Pescoção}
\begin{itemize}
\item {Grp. gram.:m.}
\end{itemize}
\begin{itemize}
\item {Utilização:Pop.}
\end{itemize}
O mesmo que \textunderscore pescoçada\textunderscore .
\section{Pescoceira}
\begin{itemize}
\item {Grp. gram.:f.}
\end{itemize}
\begin{itemize}
\item {Utilização:Chul.}
\end{itemize}
Pescoço; cachaço.
\section{Pescoceiro}
\begin{itemize}
\item {Grp. gram.:adj.}
\end{itemize}
\begin{itemize}
\item {Utilização:Bras. do S}
\end{itemize}
\begin{itemize}
\item {Utilização:Fig.}
\end{itemize}
Diz-se do cavallo que, laçado pelo pescoço, não obedece aos tirões do laçador.
O mesmo que \textunderscore caloteiro\textunderscore .
\section{Pescócia}
\begin{itemize}
\item {Grp. gram.:f.}
\end{itemize}
\begin{itemize}
\item {Utilização:Prov.}
\end{itemize}
Armadilha, que apanha os pássaros pelo pescoço.
\section{Pescocinho}
\begin{itemize}
\item {Grp. gram.:m.}
\end{itemize}
\begin{itemize}
\item {Proveniência:(De \textunderscore pescoço\textunderscore )}
\end{itemize}
Debrum branco, de tirar e pôr, nas lobas e batinas.
\section{Pescoço}
\begin{itemize}
\item {fónica:cô}
\end{itemize}
\begin{itemize}
\item {Grp. gram.:m.}
\end{itemize}
\begin{itemize}
\item {Utilização:Ext.}
\end{itemize}
\begin{itemize}
\item {Utilização:Prov.}
\end{itemize}
Parte do corpo, entre a cabeça e o tronco.
Garganta, collo.
Cachaço.
Gargalo: \textunderscore o pescoço de uma garrafa\textunderscore .
Altivez, soberba, arrogância: \textunderscore falou-me com um tal pescoço\textunderscore !...
(Cast. \textunderscore pescuezo\textunderscore )
\section{Pescoçudo}
\begin{itemize}
\item {Grp. gram.:adj.}
\end{itemize}
Que tem o pescoço grosso.
\section{Pescorência}
\begin{itemize}
\item {Grp. gram.:f.}
\end{itemize}
\begin{itemize}
\item {Utilização:Pop.}
\end{itemize}
Rapariga leviana.
\section{Pescorenço}
\begin{itemize}
\item {Grp. gram.:m.}
\end{itemize}
\begin{itemize}
\item {Utilização:Prov.}
\end{itemize}
\begin{itemize}
\item {Utilização:alent.}
\end{itemize}
O mesmo que \textunderscore namôro\textunderscore .
\section{Pescota}
\begin{itemize}
\item {Grp. gram.:f.}
\end{itemize}
\begin{itemize}
\item {Utilização:Ant.}
\end{itemize}
O mesmo que \textunderscore pescada\textunderscore .
\section{Pés-de-gallinha}
\begin{itemize}
\item {Grp. gram.:m. pl.}
\end{itemize}
(Cp. \textunderscore pé\textunderscore )
\section{Pés-de-lebre}
\begin{itemize}
\item {Grp. gram.:m. pl.}
\end{itemize}
Carris recurvados que, formando ângulo entre si, acompanham o coração do cruzamento das linhas férreas e fazem mudar a direcção do combóio.
\section{Pés-de-moleque}
\begin{itemize}
\item {Grp. gram.:m. pl.}
\end{itemize}
\begin{itemize}
\item {Utilização:Bras}
\end{itemize}
Espécie de doce, o mesmo que \textunderscore alcamonia\textunderscore .
\section{Pesebre}
\begin{itemize}
\item {Grp. gram.:m.}
\end{itemize}
Lugar, designado na manjadoira para cada cavalgadura.
(Cast. \textunderscore pesebre\textunderscore )
\section{Pesenho}
\begin{itemize}
\item {Grp. gram.:adj.}
\end{itemize}
(Talvez fórma incorrecta, em vez de \textunderscore pezanho\textunderscore ). Cf. \textunderscore Viriato Trág.\textunderscore , XI, 107.
\section{Peseta}
\begin{itemize}
\item {fónica:zê}
\end{itemize}
\begin{itemize}
\item {Grp. gram.:f.}
\end{itemize}
\begin{itemize}
\item {Utilização:Fig.}
\end{itemize}
Moéda espanhola de prata, correspondente á 5.^a parte do pêso (moéda), e do valor de 180 reis portugueses.
Pessôa, de qualidades pouco recommendáveis: \textunderscore saíste-me uma bôa peseta\textunderscore !
(Dem. de \textunderscore pêso\textunderscore )
\section{Pêsga}
\begin{itemize}
\item {Grp. gram.:f.}
\end{itemize}
Acto de barrar interiormente com pez as vasilhas de barro, em que há de fermentar a uva.--A rigorosa orthogr. sería \textunderscore pêzga\textunderscore .
\section{Pesgar}
\begin{itemize}
\item {Grp. gram.:v. t.}
\end{itemize}
Barrar com pesga. Cf. \textunderscore Techn. Rur.\textunderscore , 173.
\section{Pèsipello, a}
\begin{itemize}
\item {Grp. gram.:loc. adv.}
\end{itemize}
A pé e descalço.
\section{Pèsipelo, a}
\begin{itemize}
\item {Grp. gram.:loc. adv.}
\end{itemize}
A pé e descalço.
\section{Pés-negros}
\begin{itemize}
\item {Grp. gram.:m. pl.}
\end{itemize}
Uma das tríbos do território indiano dos Estados-Unidos.
\section{Pêso}
\begin{itemize}
\item {Grp. gram.:m.}
\end{itemize}
\begin{itemize}
\item {Utilização:Ext.}
\end{itemize}
\begin{itemize}
\item {Utilização:Fig.}
\end{itemize}
\begin{itemize}
\item {Grp. gram.:Loc. adv.}
\end{itemize}
\begin{itemize}
\item {Utilização:Pop.}
\end{itemize}
\begin{itemize}
\item {Proveniência:(Do lat. \textunderscore pensus\textunderscore )}
\end{itemize}
Resultado da acção, que a gravidade exerce num corpo.
Gravidade, inherente aos corpos, ou á pressão que elles exercem no obstáculo que se opõe directamente á sua quéda.
Grande cone cylindrico de pedra, que é o acumulador de energia nas antigas prensas dos lagares.
Unidade, com que se avalia essa pressão.
Peça de metal, que representa essa unidade.
Moéda espanhola e de algumas nações latino-americanas, equivalente a 900 reis aproximadamente.
Aquillo que exerce pressão.
Quantidade.
Aquillo que incommoda ou afadiga.
Incómmodo, que se sente á maneira de pressão.
Depressão ou oppressão.
Encargo, ónus.
Importância, autoridade: \textunderscore escritores de pêso\textunderscore .
Fôrça.
\textunderscore Em pêso\textunderscore , completamente; na totalidade.
Dois quilos: \textunderscore as favas estão a 50 reis o pêso\textunderscore .
\section{Pespegar}
\begin{itemize}
\item {Grp. gram.:v. t.}
\end{itemize}
\begin{itemize}
\item {Utilização:Fam.}
\end{itemize}
\begin{itemize}
\item {Grp. gram.:V. p.}
\end{itemize}
\begin{itemize}
\item {Utilização:Fam.}
\end{itemize}
\begin{itemize}
\item {Proveniência:(De \textunderscore post\textunderscore , lat. + \textunderscore pegar\textunderscore )}
\end{itemize}
Impingir.
Dar com violência, com mau modo.
Assentar; applicar: \textunderscore pespegar uma bofetada\textunderscore .
Permanecer num lugar, maçando ou enfadando a pessôa a quem fala ou os donos da casa onde está.
Parolar inutilmente, tomando a outros o tempo de que precisam.
\section{Pespêgo}
\begin{itemize}
\item {Grp. gram.:m.}
\end{itemize}
\begin{itemize}
\item {Utilização:Fam.}
\end{itemize}
\begin{itemize}
\item {Proveniência:(De \textunderscore pespegar\textunderscore )}
\end{itemize}
Empecilho, estôrvo.
Pessôa que embaraça ou molesta.
\section{Pespeneiro}
\begin{itemize}
\item {Grp. gram.:m.}
\end{itemize}
\begin{itemize}
\item {Utilização:Prov.}
\end{itemize}
\begin{itemize}
\item {Utilização:trasm.}
\end{itemize}
Peça de ferro, que atravessa a rabiça do arado e segura dos lados as orelheiras.
\section{Pespilhar}
\begin{itemize}
\item {Grp. gram.:m.}
\end{itemize}
\begin{itemize}
\item {Utilização:Prov.}
\end{itemize}
\begin{itemize}
\item {Utilização:trasm.}
\end{itemize}
Peça, ordinariamente de ferro, que liga os barbiões do carro ao tabuleiro.
\section{Pespita}
\begin{itemize}
\item {Grp. gram.:f.}
\end{itemize}
O mesmo que \textunderscore arvéloa\textunderscore .
\section{Pespontar}
\begin{itemize}
\item {Grp. gram.:v. t.}
\end{itemize}
\begin{itemize}
\item {Grp. gram.:V. i.}
\end{itemize}
\begin{itemize}
\item {Utilização:Fig.}
\end{itemize}
Dar pesponto em; coser a pesponto.
Presumir; timbrar:«\textunderscore ...os que entre elles pespontavão de sabedores...\textunderscore »Filinto, \textunderscore D. Man.\textunderscore , II, 116.
\section{Pesponteado}
\begin{itemize}
\item {Grp. gram.:adj.}
\end{itemize}
\begin{itemize}
\item {Utilização:Fig.}
\end{itemize}
Feito com etiqueta, com todo o apuro:«\textunderscore Muito ceremoniática e muito pesponteada foi a ceia...\textunderscore »Filinto, XX, 189.
\section{Pespontear}
\begin{itemize}
\item {Grp. gram.:v. t.}
\end{itemize}
O mesmo que \textunderscore pespontar\textunderscore .
\section{Pesponto}
\begin{itemize}
\item {Grp. gram.:m.}
\end{itemize}
Ponto de costura, em que a agulha entra um pouco atrás do lugar por onde saiu.
(Cp. cast. \textunderscore pespunte\textunderscore )
\section{Pesporrência}
\begin{itemize}
\item {Grp. gram.:f.}
\end{itemize}
\begin{itemize}
\item {Utilização:Chul.}
\end{itemize}
Prosápia balofa; arrogância.
\section{Pesqueira}
\begin{itemize}
\item {Grp. gram.:f.}
\end{itemize}
\begin{itemize}
\item {Grp. gram.:Adj.}
\end{itemize}
\begin{itemize}
\item {Proveniência:(Do b. lat. \textunderscore piscaria\textunderscore )}
\end{itemize}
Lugar, em que há armações de pesca.
Armação de pesca.
Diz-se de uma espécie de águia, (\textunderscore pandion haliaetus\textunderscore , Cuv.).
\section{Pesqueiro}
\begin{itemize}
\item {Grp. gram.:m.}
\end{itemize}
\begin{itemize}
\item {Utilização:Pesc.}
\end{itemize}
\begin{itemize}
\item {Proveniência:(De \textunderscore pesca\textunderscore )}
\end{itemize}
Fio, com aselha numa extremidade e anzol na outra.
Local, que serve de comedoiro, viveiro ou abrigo para peixes.
\section{Pesquisa}
\begin{itemize}
\item {Grp. gram.:f.}
\end{itemize}
Acto ou effeito de pesquisar; indagação; investigação.
\section{Pesquisador}
\begin{itemize}
\item {Grp. gram.:m.  e  adj.}
\end{itemize}
O que pesquisa.
\section{Pesquisar}
\begin{itemize}
\item {Grp. gram.:v. t.}
\end{itemize}
\begin{itemize}
\item {Proveniência:(Do lat. \textunderscore per\textunderscore  + \textunderscore quaeso\textunderscore )}
\end{itemize}
Buscar com diligência; indagar; inquirir.
Informar-se á cêrca de.
\section{Pesquisição}
\begin{itemize}
\item {Grp. gram.:f.}
\end{itemize}
Pesquisa, investigação. Cf. Latino, \textunderscore Or. da Corôa\textunderscore , CIX.
(Por \textunderscore perquirição\textunderscore , do lat. \textunderscore perquiritio\textunderscore , sob infl. de \textunderscore pesquisar\textunderscore )
\section{Pessá}
\begin{itemize}
\item {Grp. gram.:m.}
\end{itemize}
\begin{itemize}
\item {Utilização:Bras. do N}
\end{itemize}
O mesmo que \textunderscore pussá\textunderscore ^1.
\section{Pessário}
\begin{itemize}
\item {Grp. gram.:m.}
\end{itemize}
\begin{itemize}
\item {Utilização:Med.}
\end{itemize}
\begin{itemize}
\item {Proveniência:(Lat. \textunderscore pessarium\textunderscore )}
\end{itemize}
Apparelho circular de cauchu, para remediar a quéda ou descida do útero.
\section{Pessegada}
\begin{itemize}
\item {Grp. gram.:f.}
\end{itemize}
Dôce de pêssegos. Cf. C. Guerreiro, \textunderscore Diccion. de Consoantes\textunderscore .
\section{Pessegal}
\begin{itemize}
\item {Grp. gram.:m.}
\end{itemize}
\begin{itemize}
\item {Grp. gram.:Adj.}
\end{itemize}
\begin{itemize}
\item {Utilização:Prov.}
\end{itemize}
\begin{itemize}
\item {Utilização:alent.}
\end{itemize}
\begin{itemize}
\item {Proveniência:(De \textunderscore pêssego\textunderscore )}
\end{itemize}
Pomar de pessegueiros.
Diz-se de uma variedade de ameixa.
\section{Pêssego}
\begin{itemize}
\item {Grp. gram.:m.}
\end{itemize}
\begin{itemize}
\item {Proveniência:(Do lat. \textunderscore persicus\textunderscore )}
\end{itemize}
Fruto do pessegueiro.
\section{Pessegudo}
\begin{itemize}
\item {Grp. gram.:adj.}
\end{itemize}
\begin{itemize}
\item {Utilização:T. do Fundão}
\end{itemize}
O mesmo que \textunderscore rancoroso\textunderscore .
\section{Pessegueiro}
\begin{itemize}
\item {Grp. gram.:m.}
\end{itemize}
\begin{itemize}
\item {Proveniência:(De \textunderscore pêssego\textunderscore )}
\end{itemize}
Árvore amýgdalácea, (\textunderscore amygdalus persicus\textunderscore ).
\section{Pessimamente}
\begin{itemize}
\item {Grp. gram.:adv.}
\end{itemize}
De modo péssimo; muito mal.
\section{Pessimismo}
\begin{itemize}
\item {Grp. gram.:m.}
\end{itemize}
Opinião ou systema dos que acham tudo péssimo.
Philosophia ou systema dos que não têm fé no progresso moral e material, na melhoria das actuaes condições sociaes, na evolução para o bem e para o óptimo.
\section{Pessimista}
\begin{itemize}
\item {Grp. gram.:adj.}
\end{itemize}
\begin{itemize}
\item {Grp. gram.:M.}
\end{itemize}
\begin{itemize}
\item {Proveniência:(De \textunderscore péssimo\textunderscore )}
\end{itemize}
Relativo ao pessimismo ou aos pessimistas.
Partidário do pessimismo.
\section{Péssimo}
\begin{itemize}
\item {Grp. gram.:adj.}
\end{itemize}
\begin{itemize}
\item {Proveniência:(Lat. \textunderscore pessimus\textunderscore )}
\end{itemize}
Muito mau.
\section{Pessôa}
\begin{itemize}
\item {Grp. gram.:f.}
\end{itemize}
\begin{itemize}
\item {Utilização:Gram.}
\end{itemize}
\begin{itemize}
\item {Utilização:Restrict.}
\end{itemize}
\begin{itemize}
\item {Utilização:Ant.}
\end{itemize}
\begin{itemize}
\item {Utilização:Ant.}
\end{itemize}
\begin{itemize}
\item {Grp. gram.:Loc. adv.}
\end{itemize}
\begin{itemize}
\item {Proveniência:(Lat. \textunderscore persona\textunderscore )}
\end{itemize}
Um homem ou uma mulhér.
Sêr moral ou jurídico, (em opposição a coisas).
Personagem; individualidade.
Cada uma das relações do sujeito de uma oração com a fórma, pela qual o verbo exprime essas relações.
Aquelle que reina, o monarcha. Cf. Rebello, \textunderscore Elog. Hist. de D. Pedro V\textunderscore , 34.
Aquelle que tinha qualquer dignidade ou prebenda numa cathedral.
Párocho, que recebia a parte principal dos dízimos, sem obrigação de cura de almas.
\textunderscore Em pessôa\textunderscore , pessoalmente.
\section{Pessoádego}
\begin{itemize}
\item {Grp. gram.:m.}
\end{itemize}
\begin{itemize}
\item {Utilização:Jur.}
\end{itemize}
\begin{itemize}
\item {Utilização:ant.}
\end{itemize}
\begin{itemize}
\item {Proveniência:(Do lat. hyp. \textunderscore personaticum\textunderscore )}
\end{itemize}
Direito de pessoeiro ou cabecel.
\section{Pessoádigo}
\begin{itemize}
\item {Grp. gram.:m.}
\end{itemize}
\begin{itemize}
\item {Utilização:Ant.}
\end{itemize}
O mesmo que \textunderscore pessoádego\textunderscore .
Posse de qualquer coisa.
\section{Pessoal}
\begin{itemize}
\item {Grp. gram.:adj.}
\end{itemize}
\begin{itemize}
\item {Grp. gram.:M.}
\end{itemize}
\begin{itemize}
\item {Proveniência:(Do lat. \textunderscore personalis\textunderscore )}
\end{itemize}
Relativo a pessôa.
Individual.
Próprio de certa pessôa: \textunderscore interesse pessoal\textunderscore .
Conjunto dos indivíduos, incumbidos de certos serviços: \textunderscore o pessoal dos carros eléctricos\textunderscore .
\section{Pessoalizar}
\begin{itemize}
\item {Grp. gram.:v. t.}
\end{itemize}
\begin{itemize}
\item {Proveniência:(De \textunderscore pessoal\textunderscore )}
\end{itemize}
O mesmo que \textunderscore personificar\textunderscore . Cf. \textunderscore Hyssope\textunderscore , 136.
\section{Pessoalmente}
\begin{itemize}
\item {Grp. gram.:adv.}
\end{itemize}
De modo pessoal; por si próprio.
\section{Pessoaría}
\begin{itemize}
\item {Grp. gram.:f.}
\end{itemize}
Cargo e attribuições do pessoeiro.
(Cp. \textunderscore pessoeiro\textunderscore )
\section{Pessoária}
\begin{itemize}
\item {Grp. gram.:f.}
\end{itemize}
\begin{itemize}
\item {Utilização:Jur.}
\end{itemize}
\begin{itemize}
\item {Utilização:ant.}
\end{itemize}
\begin{itemize}
\item {Proveniência:(De \textunderscore pessôa\textunderscore )}
\end{itemize}
Qualquer acção, intentada pelo cabecel, em virtude do domínio útil que tinha nos respectivos bens.
\section{Pessoavelmente}
\begin{itemize}
\item {Grp. gram.:adv.}
\end{itemize}
\begin{itemize}
\item {Utilização:Ant.}
\end{itemize}
O mesmo que \textunderscore pessoalmente\textunderscore .
\section{Pessoeiro}
\begin{itemize}
\item {Grp. gram.:m.}
\end{itemize}
\begin{itemize}
\item {Utilização:Ant.}
\end{itemize}
\begin{itemize}
\item {Proveniência:(De \textunderscore pessôa\textunderscore )}
\end{itemize}
Cabeça de casal; cabecel.
\section{Pestalózzia}
\begin{itemize}
\item {Grp. gram.:f.}
\end{itemize}
\begin{itemize}
\item {Proveniência:(De \textunderscore Pestalozzi\textunderscore , n. p.)}
\end{itemize}
Gênero de cogumelos, que formam manchas negras nas hastes e nas fôlhas dos vegetaes vivos.
\section{Pestana}
\begin{itemize}
\item {Grp. gram.:f.}
\end{itemize}
\begin{itemize}
\item {Utilização:Mús.}
\end{itemize}
\begin{itemize}
\item {Utilização:Mús.}
\end{itemize}
\begin{itemize}
\item {Utilização:Pop.}
\end{itemize}
\begin{itemize}
\item {Utilização:Bras}
\end{itemize}
\begin{itemize}
\item {Utilização:pop.}
\end{itemize}
\begin{itemize}
\item {Grp. gram.:Loc.}
\end{itemize}
\begin{itemize}
\item {Utilização:fig.}
\end{itemize}
Cada um dos pêlos que bordam as pálpebras.
Cílio, celha.
Tira da uma peça de vestuário, em que se abriram botoeiras.
Filete de refôrço, que há nos instrumentos de cordas, junto das chavelhas.
Applicação horizontal do dedo indicador esquerdo, ao comprimir mais de uma corda do violão, violoncello, etc.
O mesmo que \textunderscore barbatana\textunderscore .
Somno ligeiro: \textunderscore fazer uma pestana\textunderscore , ir descansar um pouco.
\textunderscore Queimar as pestanas\textunderscore , estudar muito.
\section{Pestanear}
\begin{itemize}
\item {Grp. gram.:v. i.}
\end{itemize}
O mesmo que \textunderscore pestanejar\textunderscore .
\section{Pestanejante}
\begin{itemize}
\item {Grp. gram.:adj.}
\end{itemize}
Que pestaneja.
\section{Pestanejar}
\begin{itemize}
\item {Grp. gram.:v. i.}
\end{itemize}
Mover as pestanas, abrindo e fechando os olhos.
Tremeluzir, (falando-se de estrêllas). Cf. \textunderscore Laura de Anfriso\textunderscore , 137.
\section{Pestanejo}
\begin{itemize}
\item {Grp. gram.:m.}
\end{itemize}
Acto de pestanejar.
\section{Pestanudo}
\begin{itemize}
\item {Grp. gram.:adj.}
\end{itemize}
Que tem grandes pestanas.
\section{Peste}
\begin{itemize}
\item {Grp. gram.:f.}
\end{itemize}
\begin{itemize}
\item {Utilização:Fig.}
\end{itemize}
\begin{itemize}
\item {Utilização:Pop.}
\end{itemize}
\begin{itemize}
\item {Proveniência:(Lat. \textunderscore pestis\textunderscore )}
\end{itemize}
Grave doença contagiosa.
Epidemia.
Coisa perniciosa, funesta.
Aquillo que corrompe ou desmoraliza.
Mau cheiro.
Pessôa de má índole ou rebugenta: \textunderscore êste rapaz é uma peste\textunderscore .
Faíscas eléctricas.
\section{Pestear}
\begin{itemize}
\item {Grp. gram.:v. t.}
\end{itemize}
\begin{itemize}
\item {Utilização:Neol.}
\end{itemize}
O mesmo que \textunderscore empestar\textunderscore .
\section{Pestelença}
\begin{itemize}
\item {Grp. gram.:f.}
\end{itemize}
\begin{itemize}
\item {Utilização:Ant.}
\end{itemize}
O mesmo que \textunderscore pestilência\textunderscore .
\section{Pestença}
\begin{itemize}
\item {Grp. gram.:f.}
\end{itemize}
\begin{itemize}
\item {Utilização:Ant.}
\end{itemize}
O mesmo que \textunderscore pestilência\textunderscore .
\section{Pestenença}
\begin{itemize}
\item {Grp. gram.:f.}
\end{itemize}
\begin{itemize}
\item {Utilização:Ant.}
\end{itemize}
O mesmo que \textunderscore pestilência\textunderscore .
\section{Peste-russa}
\begin{itemize}
\item {Grp. gram.:f.}
\end{itemize}
Doença, o mesmo que \textunderscore influência\textunderscore .
\section{Pestiferamente}
\begin{itemize}
\item {Grp. gram.:adv.}
\end{itemize}
De modo pestífero.
\section{Pestiferar}
\begin{itemize}
\item {Grp. gram.:v. t.}
\end{itemize}
\begin{itemize}
\item {Utilização:Neol.}
\end{itemize}
\begin{itemize}
\item {Proveniência:(Do lat. \textunderscore pestis\textunderscore  + \textunderscore ferre\textunderscore )}
\end{itemize}
O mesmo que \textunderscore empestar\textunderscore .
Tornar nocivo á saúde.
\section{Pestífero}
\begin{itemize}
\item {Grp. gram.:adj.}
\end{itemize}
\begin{itemize}
\item {Utilização:Neol.}
\end{itemize}
\begin{itemize}
\item {Grp. gram.:M.}
\end{itemize}
\begin{itemize}
\item {Proveniência:(Do lat. \textunderscore pestis\textunderscore  + \textunderscore ferre\textunderscore )}
\end{itemize}
Que produz peste.
Pernicioso, que corrompe.
Que está atacado de peste.
Doente de peste.
\section{Pestilença}
\begin{itemize}
\item {Grp. gram.:f.}
\end{itemize}
\begin{itemize}
\item {Utilização:Ant.}
\end{itemize}
O mesmo que \textunderscore pestilência\textunderscore .
\section{Pestilência}
\begin{itemize}
\item {Grp. gram.:f.}
\end{itemize}
\begin{itemize}
\item {Proveniência:(Lat. \textunderscore pestilentia\textunderscore )}
\end{itemize}
Peste, contágio.
\section{Pestilencial}
\begin{itemize}
\item {Grp. gram.:adj.}
\end{itemize}
\begin{itemize}
\item {Utilização:Fig.}
\end{itemize}
\begin{itemize}
\item {Proveniência:(De \textunderscore pestilência\textunderscore )}
\end{itemize}
Relativo a peste.
Pestífero; mephítico.
Que corrompe ou desmoraliza.
\section{Pestilencialmente}
\begin{itemize}
\item {Grp. gram.:adv.}
\end{itemize}
De modo pestilencial.
\section{Pestilenciar}
\begin{itemize}
\item {Grp. gram.:v. t.}
\end{itemize}
\begin{itemize}
\item {Utilização:P. us.}
\end{itemize}
\begin{itemize}
\item {Proveniência:(De \textunderscore pestilência\textunderscore )}
\end{itemize}
Tornar pestilento; empestar.
\section{Pestilencioso}
\begin{itemize}
\item {Grp. gram.:adj.}
\end{itemize}
\begin{itemize}
\item {Proveniência:(Lat. \textunderscore pestilentiosus\textunderscore )}
\end{itemize}
O mesmo que \textunderscore pestilencial\textunderscore .
\section{Pestilente}
\begin{itemize}
\item {Grp. gram.:adj.}
\end{itemize}
\begin{itemize}
\item {Proveniência:(Lat. \textunderscore pestilens\textunderscore )}
\end{itemize}
O mesmo e melhor que \textunderscore pestilento\textunderscore .
\section{Pestilento}
\begin{itemize}
\item {Grp. gram.:adj.}
\end{itemize}
\begin{itemize}
\item {Proveniência:(Lat. \textunderscore pestilentus\textunderscore )}
\end{itemize}
Pestilencial; pestífero.
\section{Pestilo}
\begin{itemize}
\item {Grp. gram.:m.}
\end{itemize}
Fecho ou aldraba; tranqueta.
(Cast. \textunderscore pestillo\textunderscore )
\section{Pestinhar}
\textunderscore v. i.\textunderscore  (e der.)
(V. \textunderscore pastinhar\textunderscore , etc.)
\section{Pestoso}
\begin{itemize}
\item {Grp. gram.:m.  e  adj.}
\end{itemize}
\begin{itemize}
\item {Utilização:Neol.}
\end{itemize}
Doente de peste, especialmente de peste bubónica.
\section{Pesume}
\begin{itemize}
\item {Grp. gram.:m.}
\end{itemize}
\begin{itemize}
\item {Utilização:Des.}
\end{itemize}
O mesmo que \textunderscore pêso\textunderscore .
\section{Pesunho}
\begin{itemize}
\item {Grp. gram.:m.}
\end{itemize}
\begin{itemize}
\item {Utilização:Burl.}
\end{itemize}
Pé de porco; chispe.
Pé grande e mal feito.
\section{Pêta}
\begin{itemize}
\item {Grp. gram.:f.}
\end{itemize}
Mentira.
Machadinha.
Lula.
Mancha no ôlho do cavallo.
Machadinha nas costas do podão.
Orelha do sacho.
\section{Pêta}
\begin{itemize}
\item {Grp. gram.:f.}
\end{itemize}
\begin{itemize}
\item {Utilização:T. da Bairrada}
\end{itemize}
\begin{itemize}
\item {Utilização:Bras. do N}
\end{itemize}
Fígado assado, de porco.
Espécie de bolo leve de tapioca.
\section{Pétala}
\begin{itemize}
\item {Grp. gram.:f.}
\end{itemize}
\begin{itemize}
\item {Utilização:Bot.}
\end{itemize}
\begin{itemize}
\item {Proveniência:(Do gr. \textunderscore petallon\textunderscore )}
\end{itemize}
Cada uma das peças, que constituem a corolla.
\section{Petalado}
\begin{itemize}
\item {Grp. gram.:adj.}
\end{itemize}
\begin{itemize}
\item {Utilização:Bot.}
\end{itemize}
Que tem pétala ou pétalas.
\section{Petalânteas}
\begin{itemize}
\item {Grp. gram.:f. pl.}
\end{itemize}
\begin{itemize}
\item {Proveniência:(Do gr. \textunderscore petalon\textunderscore  + \textunderscore anthos\textunderscore )}
\end{itemize}
Classe de plantas monopétalas, admitida por alguns naturalistas.
\section{Petalantera}
\begin{itemize}
\item {Grp. gram.:f.}
\end{itemize}
\begin{itemize}
\item {Proveniência:(Do gr. \textunderscore petalon\textunderscore  + \textunderscore anthera\textunderscore )}
\end{itemize}
Gênero de plantas lauráceas.
\section{Petalântheas}
\begin{itemize}
\item {Grp. gram.:f. pl.}
\end{itemize}
\begin{itemize}
\item {Proveniência:(Do gr. \textunderscore petalon\textunderscore  + \textunderscore anthos\textunderscore )}
\end{itemize}
Classe de plantas monopétalas, admittida por alguns naturalistas.
\section{Petalanthera}
\begin{itemize}
\item {Grp. gram.:f.}
\end{itemize}
\begin{itemize}
\item {Proveniência:(Do gr. \textunderscore petalon\textunderscore  + \textunderscore anthera\textunderscore )}
\end{itemize}
Gênero de plantas lauráceas.
\section{Petaleação}
\begin{itemize}
\item {Grp. gram.:f.}
\end{itemize}
\begin{itemize}
\item {Utilização:Bot.}
\end{itemize}
\begin{itemize}
\item {Proveniência:(De \textunderscore pétala\textunderscore )}
\end{itemize}
Disposição dos tegumentos floraes, antes da sua abertura completa.
\section{Petaliforme}
\begin{itemize}
\item {Grp. gram.:adj.}
\end{itemize}
\begin{itemize}
\item {Proveniência:(De \textunderscore pétala\textunderscore  + \textunderscore fórma\textunderscore )}
\end{itemize}
Que tem fórma de pétala.
\section{Petalino}
\begin{itemize}
\item {Grp. gram.:adj.}
\end{itemize}
Relativo a pétala; que tem fórma de pétala.
\section{Petálio}
\begin{itemize}
\item {Grp. gram.:m.}
\end{itemize}
\begin{itemize}
\item {Utilização:Pharm.}
\end{itemize}
\begin{itemize}
\item {Proveniência:(De \textunderscore pétala\textunderscore )}
\end{itemize}
Unguento de fôlhas de nardo.
\section{Petalismo}
\begin{itemize}
\item {Grp. gram.:m.}
\end{itemize}
\begin{itemize}
\item {Proveniência:(Gr. \textunderscore petalismos\textunderscore )}
\end{itemize}
Modo de julgar, estabelecido em Syracusa, correspondente ao ostracismo atheniense, e assim chamado porque os votos eram inscritos em fôlhas de árvore.
\section{Petalita}
\begin{itemize}
\item {Grp. gram.:f.}
\end{itemize}
\begin{itemize}
\item {Utilização:Miner.}
\end{itemize}
Variedade de feldspatho.
\section{Petalito}
\begin{itemize}
\item {Grp. gram.:m.}
\end{itemize}
O mesmo ou melhor que \textunderscore petalita\textunderscore .
\section{Pétalo}
\begin{itemize}
\item {Grp. gram.:m.}
\end{itemize}
\begin{itemize}
\item {Proveniência:(Lat. \textunderscore petalum\textunderscore )}
\end{itemize}
(Fórma preferível a \textunderscore pétala\textunderscore , mas desusada)
\section{Petalóide}
\begin{itemize}
\item {Grp. gram.:adj.}
\end{itemize}
\begin{itemize}
\item {Proveniência:(Do gr. \textunderscore petalon\textunderscore  + \textunderscore eidos\textunderscore )}
\end{itemize}
Semelhante a uma pétala.
\section{Petalomania}
\begin{itemize}
\item {Grp. gram.:f.}
\end{itemize}
\begin{itemize}
\item {Utilização:Bot.}
\end{itemize}
\begin{itemize}
\item {Proveniência:(Do gr. \textunderscore petalon\textunderscore  + \textunderscore mania\textunderscore )}
\end{itemize}
Tendência, que certas partes da flôr têm, para tomar o aspecto e a consistência de uma corolla.
\section{Petalópode}
\begin{itemize}
\item {Grp. gram.:m.}
\end{itemize}
\begin{itemize}
\item {Utilização:Zool.}
\end{itemize}
\begin{itemize}
\item {Grp. gram.:M. pl.}
\end{itemize}
\begin{itemize}
\item {Proveniência:(Do gr. \textunderscore petalon\textunderscore  + \textunderscore pous\textunderscore , \textunderscore podos\textunderscore )}
\end{itemize}
Que tem pés membranosos.
Família de zoóphytos.
\section{Petalosomos}
\begin{itemize}
\item {fónica:sô}
\end{itemize}
\begin{itemize}
\item {Grp. gram.:m. pl.}
\end{itemize}
\begin{itemize}
\item {Utilização:Zool.}
\end{itemize}
\begin{itemize}
\item {Proveniência:(Do gr. \textunderscore petalon\textunderscore  + \textunderscore soma\textunderscore )}
\end{itemize}
Família de peixes ósseos, cujo côrpo tem fórma de lâmina.
\section{Petalossomos}
\begin{itemize}
\item {Grp. gram.:m. pl.}
\end{itemize}
\begin{itemize}
\item {Utilização:Zool.}
\end{itemize}
\begin{itemize}
\item {Proveniência:(Do gr. \textunderscore petalon\textunderscore  + \textunderscore soma\textunderscore )}
\end{itemize}
Família de peixes ósseos, cujo côrpo tem fórma de lâmina.
\section{Petalostêmone}
\begin{itemize}
\item {Grp. gram.:m.}
\end{itemize}
\begin{itemize}
\item {Proveniência:(Do gr. \textunderscore petalon\textunderscore  + \textunderscore stemon\textunderscore )}
\end{itemize}
Gênero de plantas leguminosas.
\section{Petalótoma}
\begin{itemize}
\item {Grp. gram.:f.}
\end{itemize}
\begin{itemize}
\item {Proveniência:(Do gr. \textunderscore petalon\textunderscore  + \textunderscore tome\textunderscore )}
\end{itemize}
Gênero de plantas myrtáceas.
\section{Petalura}
\begin{itemize}
\item {Grp. gram.:f.}
\end{itemize}
\begin{itemize}
\item {Proveniência:(Do gr. \textunderscore petalon\textunderscore  + \textunderscore oura\textunderscore )}
\end{itemize}
Gênero de insectos neurópteros.
\section{Petaniscar}
\begin{itemize}
\item {Grp. gram.:v. t.}
\end{itemize}
\begin{itemize}
\item {Utilização:T. da Bairrada}
\end{itemize}
Ferir lume com petanisco.
\section{Petanisco}
\begin{itemize}
\item {Grp. gram.:m.}
\end{itemize}
\begin{itemize}
\item {Utilização:T. da Bairrada}
\end{itemize}
Petisco ou fuzil, com que se fere lume.
(Cp. \textunderscore petisco\textunderscore )
\section{Petão}
\begin{itemize}
\item {Grp. gram.:m.}
\end{itemize}
\begin{itemize}
\item {Utilização:Prov.}
\end{itemize}
\begin{itemize}
\item {Utilização:minh.}
\end{itemize}
Pedra redonda e insulada, submarina.
\section{Petar}
\begin{itemize}
\item {Grp. gram.:v. i.}
\end{itemize}
Dizer pêtas.
\section{Petar}
\begin{itemize}
\item {Grp. gram.:v. i.}
\end{itemize}
\begin{itemize}
\item {Utilização:Pop.}
\end{itemize}
Serrazinar.
Moer.
Sêr importuno, maçador.
\section{Petar}
\begin{itemize}
\item {Grp. gram.:v. t.}
\end{itemize}
\begin{itemize}
\item {Utilização:Prov.}
\end{itemize}
\begin{itemize}
\item {Utilização:minh.}
\end{itemize}
Partir em bocadinhos. Cf. G. Viana, \textunderscore Apostilas\textunderscore , vb. \textunderscore pitança\textunderscore .
\section{Petardar}
\begin{itemize}
\item {Grp. gram.:v. t.}
\end{itemize}
O mesmo que \textunderscore petardear\textunderscore .
\section{Petardear}
\begin{itemize}
\item {Grp. gram.:v. t.}
\end{itemize}
Fazer saltar com petardos.
\section{Petardeiro}
\begin{itemize}
\item {Grp. gram.:m.}
\end{itemize}
Aquelle que fazia ou applicava petardos.
\section{Petardo}
\begin{itemize}
\item {Grp. gram.:m.}
\end{itemize}
\begin{itemize}
\item {Proveniência:(Fr. \textunderscore pétard\textunderscore )}
\end{itemize}
Espécie de caixa, cheia de pólvora, e que se applicava para fazer saltar as portas das cidades, barreiras, etc.
Bomba.
\section{Petarola}
\begin{itemize}
\item {Grp. gram.:f.}
\end{itemize}
\begin{itemize}
\item {Utilização:Pop.}
\end{itemize}
\begin{itemize}
\item {Grp. gram.:M.}
\end{itemize}
Grande pêta, mentira evidente.
Indivíduo trapaceiro, homem que diz muitas pêtas.
\section{Petaurista}
\begin{itemize}
\item {Grp. gram.:m.}
\end{itemize}
\begin{itemize}
\item {Proveniência:(Lat. \textunderscore petaurista\textunderscore )}
\end{itemize}
Equilibrista ou funâmbulo, entre os antigos Romanos.
\section{Petauristário}
\begin{itemize}
\item {Grp. gram.:m.}
\end{itemize}
O mesmo que \textunderscore petaurista\textunderscore .
\section{Petauro}
\begin{itemize}
\item {Grp. gram.:m.}
\end{itemize}
\begin{itemize}
\item {Proveniência:(Lat. \textunderscore petaurum\textunderscore )}
\end{itemize}
Tablado de acrobatas e funâmbulos, na antiga Roma.
\section{Pete}
\begin{itemize}
\item {Grp. gram.:m.}
\end{itemize}
\begin{itemize}
\item {Utilização:T. da África. Or. Port}
\end{itemize}
Manilha, quási sempre de arame amarelo.
\section{Petear}
\begin{itemize}
\item {Grp. gram.:v. i.}
\end{itemize}
O mesmo que \textunderscore petar\textunderscore ^1.
\section{Peteca}
\begin{itemize}
\item {Grp. gram.:f.}
\end{itemize}
\begin{itemize}
\item {Utilização:Bras}
\end{itemize}
\begin{itemize}
\item {Utilização:Fig.}
\end{itemize}
Bóla de coiro, achatada, com que brincam crianças.
Pedaço de cortiça, para o mesmo fim.
Joguete.
Coisa ou pessôa, de que se faz pouco caso.
(Do \textunderscore tupi\textunderscore )
\section{Petechial}
\begin{itemize}
\item {fónica:qui}
\end{itemize}
\begin{itemize}
\item {Grp. gram.:adj.}
\end{itemize}
Relativo a petéchias; que tem petéchias.
\section{Petéchias}
\begin{itemize}
\item {fónica:qui}
\end{itemize}
\begin{itemize}
\item {Grp. gram.:f. pl.}
\end{itemize}
\begin{itemize}
\item {Proveniência:(It. \textunderscore petecchie\textunderscore )}
\end{itemize}
Manchas vermelhas, semelhantes a mordeduras de pulgas, e que se manifestam na pelle, no curso de algumas doenças agudas.
\section{Petefe}
\begin{itemize}
\item {Grp. gram.:m.}
\end{itemize}
\begin{itemize}
\item {Utilização:T. do Fundão}
\end{itemize}
Qualquer defeito.
Mácula.
(Cp. \textunderscore bitafe\textunderscore )
\section{Petegar}
\begin{itemize}
\item {Grp. gram.:v. t.}
\end{itemize}
\begin{itemize}
\item {Utilização:Ant.}
\end{itemize}
Cortar com a pêta do (podão).
Cortar com machado.
\section{Peteiro}
\begin{itemize}
\item {Grp. gram.:m.  e  adj.}
\end{itemize}
O que diz petas; patranheiro.
\section{Peteiro}
\begin{itemize}
\item {Grp. gram.:m.}
\end{itemize}
\begin{itemize}
\item {Utilização:Prov.}
\end{itemize}
\begin{itemize}
\item {Utilização:minh.}
\end{itemize}
O mesmo que \textunderscore mealheiro\textunderscore .
(Cp. \textunderscore petar\textunderscore ^3)
\section{Peteleca}
\begin{itemize}
\item {Grp. gram.:f.}
\end{itemize}
\begin{itemize}
\item {Utilização:Bras}
\end{itemize}
Bofetada.
\section{Peteleco}
\begin{itemize}
\item {Grp. gram.:m.}
\end{itemize}
\begin{itemize}
\item {Utilização:Bras}
\end{itemize}
Pancada com a mão ou com o pé, por brincadeira; peteleca.
\section{Petém}
\begin{itemize}
\item {Grp. gram.:m.}
\end{itemize}
\begin{itemize}
\item {Utilização:Prov.}
\end{itemize}
\begin{itemize}
\item {Utilização:alent.}
\end{itemize}
\begin{itemize}
\item {Proveniência:(De \textunderscore pé\textunderscore )}
\end{itemize}
Tronco de árvore, preso ainda á terra pelas raízes.
\section{Petema}
\begin{itemize}
\item {Grp. gram.:f.}
\end{itemize}
O mesmo que \textunderscore petume\textunderscore .
\section{Petenera}
\begin{itemize}
\item {Grp. gram.:f.}
\end{itemize}
Espécie de dança e música espanhola.
\section{Petequear}
\begin{itemize}
\item {Grp. gram.:v. i.}
\end{itemize}
\begin{itemize}
\item {Utilização:Bras}
\end{itemize}
Jogar a peteca.
\section{Petequial}
\begin{itemize}
\item {Grp. gram.:adj.}
\end{itemize}
Relativo a petéquias; que tem petéquias.
\section{Petéquias}
\begin{itemize}
\item {Grp. gram.:f. pl.}
\end{itemize}
\begin{itemize}
\item {Proveniência:(It. \textunderscore petecchie\textunderscore )}
\end{itemize}
Manchas vermelhas, semelhantes a mordeduras de pulgas, e que se manifestam na pele, no curso de algumas doenças agudas.
\section{Pé-terra}
\begin{itemize}
\item {Grp. gram.:m.}
\end{itemize}
\begin{itemize}
\item {Grp. gram.:Loc. adv.}
\end{itemize}
\begin{itemize}
\item {Utilização:Ant.}
\end{itemize}
Pequena moéda de oiro, no tempo de el-rei D. Fernando.
\textunderscore De pé\textunderscore , a pé.
\section{Petiá}
\begin{itemize}
\item {Grp. gram.:f.}
\end{itemize}
Espécie de madeira fina do Brasil.
\section{Petição}
\begin{itemize}
\item {Grp. gram.:f.}
\end{itemize}
\begin{itemize}
\item {Proveniência:(Lat. \textunderscore petitio\textunderscore )}
\end{itemize}
Acto de pedir.
Súpplica; pretensão; requerimento.
\section{Peticar}
\begin{itemize}
\item {Grp. gram.:v. i.}
\end{itemize}
\begin{itemize}
\item {Utilização:T. do Fundão}
\end{itemize}
Comer alguma coisa, entre o almôço e o jantar.
(Cp. \textunderscore petiscar\textunderscore )
\section{Peticego}
\begin{itemize}
\item {Grp. gram.:m.  e  adj.}
\end{itemize}
\begin{itemize}
\item {Utilização:Pop.}
\end{itemize}
\begin{itemize}
\item {Proveniência:(De \textunderscore pêto\textunderscore ^1 + \textunderscore cego?\textunderscore )}
\end{itemize}
O que tem a vista curta.
Aquelle que tem olhos muito pequenos, ou pouco abertos, e remelosos ou inflammados.
\section{Peticionar}
\begin{itemize}
\item {Grp. gram.:v. i.}
\end{itemize}
Fazer petição.
\section{Peticionário}
\begin{itemize}
\item {Grp. gram.:m.}
\end{itemize}
\begin{itemize}
\item {Utilização:Jur.}
\end{itemize}
Aquelle que faz petição.
Aquelle que intenta demanda em juízo.
\section{Petiço}
\begin{itemize}
\item {Grp. gram.:m.}
\end{itemize}
\begin{itemize}
\item {Utilização:Bras}
\end{itemize}
\begin{itemize}
\item {Proveniência:(De \textunderscore pé\textunderscore )}
\end{itemize}
Cavallo de pernas curtas.
\section{Petiçote}
\begin{itemize}
\item {Grp. gram.:m.}
\end{itemize}
\begin{itemize}
\item {Utilização:Bras. do S}
\end{itemize}
Petiço pequeno.
\section{Petigris}
\begin{itemize}
\item {Grp. gram.:m.}
\end{itemize}
\begin{itemize}
\item {Proveniência:(Do fr. \textunderscore petit-gris\textunderscore )}
\end{itemize}
O mesmo que \textunderscore esquilo\textunderscore .
\section{Petim}
\begin{itemize}
\item {Grp. gram.:m.}
\end{itemize}
\begin{itemize}
\item {Utilização:Prov.}
\end{itemize}
\begin{itemize}
\item {Utilização:minh.}
\end{itemize}
Pão de trigo, sôbre o comprido; rosca de pão.
\section{Petima}
\begin{itemize}
\item {Grp. gram.:f.}
\end{itemize}
O mesmo que \textunderscore petume\textunderscore .
\section{Petimbuaba}
\begin{itemize}
\item {Grp. gram.:f.}
\end{itemize}
Peixe esquamodermo, (\textunderscore fistularia\textunderscore ).
\section{Petimetre}
\begin{itemize}
\item {Grp. gram.:m.  e  adj.}
\end{itemize}
\begin{itemize}
\item {Proveniência:(Do fr. \textunderscore petit-maitre\textunderscore )}
\end{itemize}
Peralta.
Janota ridículo; pãozinho.
\section{Petinga}
\begin{itemize}
\item {Grp. gram.:f.}
\end{itemize}
Sardinha miúda, (\textunderscore clupea sprattus\textunderscore , Lin.).
Peixe miúdo, que serve para isca.
\section{Petinga}
\begin{itemize}
\item {Grp. gram.:f.}
\end{itemize}
\begin{itemize}
\item {Utilização:Bras}
\end{itemize}
Árvore de Mato-Grosso.
\section{Petinha}
\begin{itemize}
\item {Grp. gram.:f.}
\end{itemize}
\begin{itemize}
\item {Utilização:Zool.}
\end{itemize}
Espécie de sombria, (\textunderscore anthus pratensis\textunderscore , Lin.).
\section{Petinho}
\begin{itemize}
\item {Grp. gram.:m.}
\end{itemize}
Pássaro dentirostro, (\textunderscore turdus iliacus\textunderscore ).
\section{Petintal}
\begin{itemize}
\item {Grp. gram.:m.}
\end{itemize}
Dispenseiro, que servia a bordo das galés.
Calafate.
Fabricante de embarcações.
\section{Petintuíba}
\begin{itemize}
\item {Grp. gram.:f.}
\end{itemize}
Árvore corpulenta do Brasil.
\section{Petipé}
\begin{itemize}
\item {Grp. gram.:m.}
\end{itemize}
Escala ou régua com divisões, usada por architectos.
Escala de reducções em mappas e cartas.
\section{Petique}
\begin{itemize}
\item {Grp. gram.:m.}
\end{itemize}
\begin{itemize}
\item {Utilização:T. do Fundão}
\end{itemize}
\begin{itemize}
\item {Proveniência:(De \textunderscore peticar\textunderscore )}
\end{itemize}
Refeição ligeira entre o almôço e o jantar, especialmente a que se dá aos malhadores; piqueta.
\section{Petisca}
\begin{itemize}
\item {Grp. gram.:f.}
\end{itemize}
\begin{itemize}
\item {Utilização:Prov.}
\end{itemize}
\begin{itemize}
\item {Utilização:alent.}
\end{itemize}
Jôgo de rapazes, que atiram pedras a uma moéda collocada no chão, ganhando-a aquelle que lhe acertar.
Jôgo, em que se atira uma pequena chapa de ferro ou uma moéda a uma navalha fixada no chão.
\section{Petiscar}
\begin{itemize}
\item {Grp. gram.:v. t.}
\end{itemize}
\begin{itemize}
\item {Utilização:Fam.}
\end{itemize}
\begin{itemize}
\item {Utilização:Prov.}
\end{itemize}
\begin{itemize}
\item {Utilização:trasm.}
\end{itemize}
\begin{itemize}
\item {Grp. gram.:V. i.}
\end{itemize}
Comer com pouco appetite; provar.
Saborear.
Conhecer superficialmente.
Tocar, tanger (animaes): \textunderscore petisca lá êsse burro\textunderscore .
Comer petiscos.
Ferir a pederneira com o fuzil, para fazer lume.
Bater com a aldrava na porta, para que a abram.
\section{Petisco}
\begin{itemize}
\item {Grp. gram.:m.}
\end{itemize}
\begin{itemize}
\item {Utilização:Fam.}
\end{itemize}
Comida saborosa.
Gulodice.
Fuzil, para ferir lume na pederneira.
Indivíduo ridículo, pãozinho, pintalegrete. Cf. G. Braga, \textunderscore Mal da Delf.\textunderscore , 183.
\section{Petiseco}
\begin{itemize}
\item {fónica:sê}
\end{itemize}
\begin{itemize}
\item {Grp. gram.:adj.}
\end{itemize}
\begin{itemize}
\item {Utilização:Pop.}
\end{itemize}
Murcho; entanguido; pêco.
\section{Petisqueira}
\begin{itemize}
\item {Grp. gram.:f.}
\end{itemize}
\begin{itemize}
\item {Utilização:Pop.}
\end{itemize}
\begin{itemize}
\item {Utilização:Pesc.}
\end{itemize}
\begin{itemize}
\item {Proveniência:(De \textunderscore petiscar\textunderscore )}
\end{itemize}
Pitéu; o que se petisca.
Espécie de rede tresmalho, usada pelos pescadores de Buarcos.
\section{Petisqueiro}
\begin{itemize}
\item {Grp. gram.:m.}
\end{itemize}
\begin{itemize}
\item {Utilização:Bras. do N}
\end{itemize}
Armário, onde se guardam petiscos ou viandas.
\section{Petisquice}
\begin{itemize}
\item {Grp. gram.:f.}
\end{itemize}
\begin{itemize}
\item {Utilização:Fam.}
\end{itemize}
Acto ou qualidade de petisco ou de indivíduo ridiculamente pretensioso.
\section{Petisseco}
\begin{itemize}
\item {Grp. gram.:adj.}
\end{itemize}
\begin{itemize}
\item {Utilização:Pop.}
\end{itemize}
Murcho; entanguido; pêco.
\section{Petites}
\begin{itemize}
\item {Grp. gram.:adj. pl.}
\end{itemize}
\begin{itemize}
\item {Proveniência:(Fr. \textunderscore petit\textunderscore )}
\end{itemize}
Dizia-se de uma espécie de torneses, que o rei D. Fernando mandou cunhar.
\section{Petitinga}
\begin{itemize}
\item {Grp. gram.:f.}
\end{itemize}
\begin{itemize}
\item {Utilização:Bras}
\end{itemize}
Pequeno peixe fluvial.
(Cp. \textunderscore petinga\textunderscore ^1)
\section{Petitor}
\begin{itemize}
\item {Grp. gram.:m.}
\end{itemize}
\begin{itemize}
\item {Utilização:Des.}
\end{itemize}
\begin{itemize}
\item {Proveniência:(Lat. \textunderscore petitor\textunderscore )}
\end{itemize}
Aquelle que pede; intercessor.
Advogado.
\section{Petitório}
\begin{itemize}
\item {Grp. gram.:adj.}
\end{itemize}
\begin{itemize}
\item {Utilização:Jur.}
\end{itemize}
\begin{itemize}
\item {Grp. gram.:M.}
\end{itemize}
\begin{itemize}
\item {Proveniência:(Lat. \textunderscore petitorius\textunderscore )}
\end{itemize}
Relativo a pedido ou petição.
Com que se pede.
Diz-se do juízo, em que se requerem as acções ordinárias.
O mesmo que \textunderscore petição\textunderscore  ou \textunderscore pedido\textunderscore . Cf. Assis, \textunderscore Águas\textunderscore , 209.
Convento medieval, destinado especialmente a curar os enfermos da erysipela epidêmica, conhecida por \textunderscore fogo-de-Santo-Antão\textunderscore . Cf. Deusdado, \textunderscore Escorços\textunderscore , 17.
\section{Petiz}
\begin{itemize}
\item {Grp. gram.:adj.}
\end{itemize}
\begin{itemize}
\item {Utilização:Fam.}
\end{itemize}
\begin{itemize}
\item {Grp. gram.:M.}
\end{itemize}
\begin{itemize}
\item {Proveniência:(Do fr. \textunderscore petit\textunderscore )}
\end{itemize}
Pequeno.
Menino.
\section{Petizada}
\begin{itemize}
\item {Grp. gram.:f.}
\end{itemize}
\begin{itemize}
\item {Utilização:Fam.}
\end{itemize}
\begin{itemize}
\item {Proveniência:(De \textunderscore petiz\textunderscore )}
\end{itemize}
Os petizes; reunião de petizes.
\section{Péto}
\begin{itemize}
\item {Grp. gram.:m. Loc. adv.}
\end{itemize}
\begin{itemize}
\item {Utilização:Prov.}
\end{itemize}
\begin{itemize}
\item {Utilização:trasm.}
\end{itemize}
\textunderscore De péto\textunderscore , de propósito, expressamente.
(Provavelmente, alter. de \textunderscore peita\textunderscore )
\section{Pêto}
\begin{itemize}
\item {Grp. gram.:adj.}
\end{itemize}
\begin{itemize}
\item {Grp. gram.:M.}
\end{itemize}
\begin{itemize}
\item {Utilização:Prov.}
\end{itemize}
\begin{itemize}
\item {Utilização:trasm.}
\end{itemize}
Que é vesgo ou estrabão.
Pássaro trepador, pica-pau, (\textunderscore picus major\textunderscore ).
Machadinha nas costas do podão.
Pêta do sacho.
\section{Pêto}
\begin{itemize}
\item {Grp. gram.:m.}
\end{itemize}
\begin{itemize}
\item {Utilização:Prov.}
\end{itemize}
O mesmo que \textunderscore mealheiro\textunderscore .
\section{Pêto}
\begin{itemize}
\item {Grp. gram.:adj.}
\end{itemize}
\begin{itemize}
\item {Proveniência:(De \textunderscore petar\textunderscore ^2)}
\end{itemize}
Maçador.
Impertinente.
\section{Pêto-gallego}
\begin{itemize}
\item {Grp. gram.:m.}
\end{itemize}
Ave de rapina, (\textunderscore picus minor\textunderscore , Lin.).
\section{Petorra}
\begin{itemize}
\item {Grp. gram.:f.}
\end{itemize}
\begin{itemize}
\item {Utilização:P. us.}
\end{itemize}
O mesmo que \textunderscore piôrra\textunderscore .
\section{Peto-real}
\begin{itemize}
\item {Grp. gram.:m.}
\end{itemize}
Ave, o mesmo que \textunderscore rinchão\textunderscore .
\section{Peto-rinchão}
\begin{itemize}
\item {Grp. gram.:m.}
\end{itemize}
Ave, o mesmo que \textunderscore rinchão\textunderscore .
\section{Petórrito}
\begin{itemize}
\item {Grp. gram.:m.}
\end{itemize}
\begin{itemize}
\item {Proveniência:(Lat. \textunderscore petorritum\textunderscore )}
\end{itemize}
Carroça de quatro rodas, usada pelos antigos Romanos, á imitação dos Gállios.
\section{Pêto-verde}
\begin{itemize}
\item {Grp. gram.:m.}
\end{itemize}
Ave, o mesmo que \textunderscore rinchão\textunderscore . Cf. P. Moraes, \textunderscore Zool. Elem.\textunderscore , 302.
\section{Pêto-verdeal}
\begin{itemize}
\item {Grp. gram.:m.}
\end{itemize}
Ave, o mesmo que \textunderscore rinchão\textunderscore . Cf. P. Moraes, \textunderscore Zool. Elem.\textunderscore , 302.
\section{Petra}
\begin{itemize}
\item {Grp. gram.:f.}
\end{itemize}
\begin{itemize}
\item {Utilização:Prov.}
\end{itemize}
\begin{itemize}
\item {Utilização:beir.}
\end{itemize}
Pedaço de sangue coagulado.
\section{Petrarquesco}
\begin{itemize}
\item {Grp. gram.:adj.}
\end{itemize}
Relativo a Petrarca.
Semelhante ao carácter das poesias de Petrarca. Cf. Latino, \textunderscore Camões\textunderscore , 61.
\section{Petrarquismo}
\begin{itemize}
\item {Grp. gram.:m.}
\end{itemize}
Imitação da maneira poética de Petrarca. Cf. Ol. Martins, \textunderscore Camões\textunderscore , 50.
\section{Petrarquista}
\begin{itemize}
\item {Grp. gram.:adj.}
\end{itemize}
\begin{itemize}
\item {Grp. gram.:M.}
\end{itemize}
Relativo a Petrarca, ou á sua maneira de poetar.
Sectário da maneira poética de Petrarca. Cf. Ol. Martins, \textunderscore Camões\textunderscore , 50 e 54.
\section{Petrechar}
\begin{itemize}
\item {Grp. gram.:v. t.}
\end{itemize}
\begin{itemize}
\item {Utilização:Fig.}
\end{itemize}
Prover de petrechos.
Aperceber, dar os meios necessários a.
\section{Petrechos}
\begin{itemize}
\item {fónica:trê}
\end{itemize}
\begin{itemize}
\item {Grp. gram.:m. pl.}
\end{itemize}
Munições e instrumentos de guerra.
Quaesquer meios ou objectos necessários, para a execução de qualquer coisa: \textunderscore petrechos de caça\textunderscore .
(Cast. \textunderscore pertrechos\textunderscore )
\section{Petrel}
\begin{itemize}
\item {Grp. gram.:m.}
\end{itemize}
Ave palmípede aquática.
\section{Pétreo}
\begin{itemize}
\item {Grp. gram.:adj.}
\end{itemize}
\begin{itemize}
\item {Utilização:Fig.}
\end{itemize}
\begin{itemize}
\item {Proveniência:(Lat. \textunderscore petreus\textunderscore )}
\end{itemize}
Que é de pedra.
Resistente como a pedra.
Que tem a apparência de pedra.
Insensível, deshumano: \textunderscore coração pétreo\textunderscore .
\section{Petrícola}
\begin{itemize}
\item {Grp. gram.:f.}
\end{itemize}
\begin{itemize}
\item {Proveniência:(Do lat. \textunderscore petra\textunderscore  + \textunderscore colere\textunderscore )}
\end{itemize}
Gênero de molluscos.
\section{Petrificação}
\begin{itemize}
\item {Grp. gram.:f.}
\end{itemize}
Acto ou effeito de petrificar.
\section{Petrificador}
\begin{itemize}
\item {Grp. gram.:adj.}
\end{itemize}
Que petrifica.
\section{Petrificar}
\begin{itemize}
\item {Grp. gram.:v. t.}
\end{itemize}
\begin{itemize}
\item {Utilização:Fig.}
\end{itemize}
\begin{itemize}
\item {Proveniência:(Do lat. \textunderscore petra\textunderscore  + \textunderscore facere\textunderscore )}
\end{itemize}
Transformar em pedra.
Immobilizar como a pedra.
Empedernir.
Causar espanto a.
Encrostar.
\section{Petrífico}
\begin{itemize}
\item {Grp. gram.:adj.}
\end{itemize}
Que petrifica.
\section{Petrina}
\begin{itemize}
\item {Grp. gram.:f.}
\end{itemize}
\begin{itemize}
\item {Utilização:Ant.}
\end{itemize}
Peito; cintura.
Cinto. Cf. \textunderscore Eufrosina\textunderscore , 20 e 93; \textunderscore Lusíadas\textunderscore , II, 36.
\section{Petróbio}
\begin{itemize}
\item {Grp. gram.:m.}
\end{itemize}
\begin{itemize}
\item {Proveniência:(Do gr. \textunderscore petra\textunderscore  + \textunderscore bios\textunderscore )}
\end{itemize}
Gênero de plantas, da fam. das compostas.
\section{Petrobrúsios}
\begin{itemize}
\item {Grp. gram.:m. pl.}
\end{itemize}
Herejes do século XII, que rejeitavam o baptismo das crianças, sacrifício da Missa, etc.
\section{Petrófila}
\begin{itemize}
\item {Grp. gram.:f.}
\end{itemize}
\begin{itemize}
\item {Proveniência:(Do gr. \textunderscore petra\textunderscore  + \textunderscore philos\textunderscore )}
\end{itemize}
Gênero de plantas proteáceas.
\section{Petrófilo}
\begin{itemize}
\item {Grp. gram.:m.}
\end{itemize}
\begin{itemize}
\item {Proveniência:(Do gr. \textunderscore petra\textunderscore  + \textunderscore philos\textunderscore )}
\end{itemize}
Gênero de insectos coleópteros pentâmeros.
\section{Petrognosia}
\begin{itemize}
\item {Grp. gram.:f.}
\end{itemize}
\begin{itemize}
\item {Proveniência:(Do gr. \textunderscore petra\textunderscore  + \textunderscore gnosis\textunderscore )}
\end{itemize}
Parte da História Natural, que se occupa dos mineraes.
\section{Petrografia}
\begin{itemize}
\item {Grp. gram.:f.}
\end{itemize}
\begin{itemize}
\item {Proveniência:(Do gr. \textunderscore petra\textunderscore  + \textunderscore graphein\textunderscore )}
\end{itemize}
Descripção das pedras.
\section{Petrográfico}
\begin{itemize}
\item {Grp. gram.:adj.}
\end{itemize}
Relativo á Petrografia.
\section{Petrógrafo}
\begin{itemize}
\item {Grp. gram.:m.}
\end{itemize}
Aquele que é perito em Petrografia.
\section{Petrographia}
\begin{itemize}
\item {Grp. gram.:f.}
\end{itemize}
\begin{itemize}
\item {Proveniência:(Do gr. \textunderscore petra\textunderscore  + \textunderscore graphein\textunderscore )}
\end{itemize}
Descripção das pedras.
\section{Petrográphico}
\begin{itemize}
\item {Grp. gram.:adj.}
\end{itemize}
Relativo á Petrographia.
\section{Petrógrapho}
\begin{itemize}
\item {Grp. gram.:m.}
\end{itemize}
Aquelle que é perito em Petrographia.
\section{Petrolaria}
\begin{itemize}
\item {Grp. gram.:f.}
\end{itemize}
Fábrica de petróleo.
\section{Petroleiro}
\begin{itemize}
\item {Grp. gram.:m.}
\end{itemize}
\begin{itemize}
\item {Utilização:Fig.}
\end{itemize}
Aquelle que emprega petróleo, como meio de destruição.
Revolucionário; communista.
\section{Petroleno}
\begin{itemize}
\item {Grp. gram.:m.}
\end{itemize}
\begin{itemize}
\item {Utilização:Chím.}
\end{itemize}
Carbureto de hydrogênio, que se acha combinado com o asphalto em muitas espécies de resinas fósseis.
\section{Petróleo}
\begin{itemize}
\item {Grp. gram.:m.}
\end{itemize}
\begin{itemize}
\item {Proveniência:(Do lat. \textunderscore petra\textunderscore  + \textunderscore oleum\textunderscore )}
\end{itemize}
Óleo mineral, empregado principalmente como substância illuminante.
\textunderscore Petróleo bruto\textunderscore , o petróleo, tal qual sai da mina.
\section{Petrolífero}
\begin{itemize}
\item {Grp. gram.:adj.}
\end{itemize}
\begin{itemize}
\item {Proveniência:(Do lat. \textunderscore petra\textunderscore  + \textunderscore oleum\textunderscore  + \textunderscore ferre\textunderscore )}
\end{itemize}
Que contém ou produz petróleo.
Diz-se especialmente dos terrenos, em que se póde explorar o petróleo: \textunderscore na região petrolífera de Angola\textunderscore .
\section{Petrolina}
\begin{itemize}
\item {Grp. gram.:f.}
\end{itemize}
Substância gôrda do petróleo; petróleo.
\section{Petroline}
\begin{itemize}
\item {Grp. gram.:m.}
\end{itemize}
(V.petrolina)
\section{Petrolista}
\begin{itemize}
\item {Grp. gram.:m.}
\end{itemize}
O mesmo que \textunderscore petroleiro\textunderscore . Cf. Camillo, \textunderscore Ratazzi\textunderscore , 8.
\section{Petrologia}
\begin{itemize}
\item {Grp. gram.:f.}
\end{itemize}
\begin{itemize}
\item {Proveniência:(Do gr. \textunderscore petra\textunderscore  + \textunderscore logos\textunderscore )}
\end{itemize}
Parte da Geologia, que se occupa das rochas.
\section{Petrológico}
\begin{itemize}
\item {Grp. gram.:adj.}
\end{itemize}
Relativo á Petrologia.
\section{Petrologista}
\begin{itemize}
\item {Grp. gram.:m.}
\end{itemize}
Aquelle que escreve sôbre Petrologia ou que della se occupa scientificamente.
\section{Petro-occipital}
\begin{itemize}
\item {Grp. gram.:adj.}
\end{itemize}
\begin{itemize}
\item {Utilização:Anat.}
\end{itemize}
Relativo á apóphyse temporal e ao occipício.
\section{Petróphila}
\begin{itemize}
\item {Grp. gram.:f.}
\end{itemize}
\begin{itemize}
\item {Proveniência:(Do gr. \textunderscore petra\textunderscore  + \textunderscore philos\textunderscore )}
\end{itemize}
Gênero de plantas proteáceas.
\section{Petróphilo}
\begin{itemize}
\item {Grp. gram.:m.}
\end{itemize}
\begin{itemize}
\item {Proveniência:(Do gr. \textunderscore petra\textunderscore  + \textunderscore philos\textunderscore )}
\end{itemize}
Gênero de insectos coleópteros pentâmeros.
\section{Petrosílex}
\begin{itemize}
\item {fónica:si}
\end{itemize}
\begin{itemize}
\item {Grp. gram.:m.}
\end{itemize}
O mesmo que \textunderscore felsito\textunderscore .
\section{Petrosilicoso}
\begin{itemize}
\item {fónica:si}
\end{itemize}
\begin{itemize}
\item {Grp. gram.:adj.}
\end{itemize}
Que tem o carácter de petrosilex.
\section{Petroso}
\begin{itemize}
\item {Grp. gram.:adj.}
\end{itemize}
\begin{itemize}
\item {Proveniência:(Lat. \textunderscore petrosus\textunderscore )}
\end{itemize}
O mesmo que \textunderscore pétreo\textunderscore .
\section{Petrossílex}
\begin{itemize}
\item {Grp. gram.:m.}
\end{itemize}
O mesmo que \textunderscore felsito\textunderscore .
\section{Petrossilicoso}
\begin{itemize}
\item {Grp. gram.:adj.}
\end{itemize}
Que tem o carácter de petrossilex.
\section{Petulância}
\begin{itemize}
\item {Grp. gram.:f.}
\end{itemize}
\begin{itemize}
\item {Proveniência:(Lat. \textunderscore petulantia\textunderscore )}
\end{itemize}
Qualidade do que é petulante; immodéstia; ousadia.
\section{Petulante}
\begin{itemize}
\item {Grp. gram.:adj.}
\end{itemize}
\begin{itemize}
\item {Proveniência:(Lat. \textunderscore petulans\textunderscore )}
\end{itemize}
Immodesto, atrevido.
Insolente; desavergonhado.
\section{Petulantemente}
\begin{itemize}
\item {Grp. gram.:adv.}
\end{itemize}
De modo petulante.
\section{Petum}
\begin{itemize}
\item {Grp. gram.:m.}
\end{itemize}
\begin{itemize}
\item {Utilização:Bras}
\end{itemize}
O mesmo que \textunderscore petume\textunderscore .
\section{Petume}
\begin{itemize}
\item {Grp. gram.:m.}
\end{itemize}
\begin{itemize}
\item {Utilização:Bras}
\end{itemize}
Nome tupi do tabaco.
\section{Petúnculo}
\begin{itemize}
\item {Grp. gram.:m.}
\end{itemize}
Gênero de molluscos.
\section{Petunga}
\begin{itemize}
\item {Grp. gram.:f.}
\end{itemize}
Gênero de plantas rubiáceas.
\section{Petúnia}
\begin{itemize}
\item {Grp. gram.:f.}
\end{itemize}
\begin{itemize}
\item {Proveniência:(De \textunderscore petum\textunderscore )}
\end{itemize}
Gênero de plantas herbáceas e viscosas, originárias da América do Sul, e cujas flôres formosas são vulgares nos principaes jardins.
\section{Péu!}
\begin{itemize}
\item {Grp. gram.:interj.}
\end{itemize}
\begin{itemize}
\item {Proveniência:(De \textunderscore chapéu\textunderscore )}
\end{itemize}
(para que se descubra quem tem o chapéu na cabeça, depois de levantado o pano, numa representação theatral)
\section{Peucedanina}
\begin{itemize}
\item {Grp. gram.:f.}
\end{itemize}
Substância amarga e aromática, que se extrái da raíz do peucédano.
\section{Peucedanite}
\begin{itemize}
\item {Grp. gram.:f.}
\end{itemize}
\begin{itemize}
\item {Utilização:Chím.}
\end{itemize}
Substância crystallizável, extrahida da raíz do peucédano.
\section{Peucédano}
\begin{itemize}
\item {Grp. gram.:m.}
\end{itemize}
\begin{itemize}
\item {Proveniência:(Lat. \textunderscore peucedanum\textunderscore )}
\end{itemize}
Planta umbellífera.
\section{Peúco}
\begin{itemize}
\item {Grp. gram.:m.}
\end{itemize}
\begin{itemize}
\item {Utilização:Prov.}
\end{itemize}
\begin{itemize}
\item {Utilização:trasm.}
\end{itemize}
O mesmo que \textunderscore peúga\textunderscore .
\section{Peúga}
\begin{itemize}
\item {Grp. gram.:f.}
\end{itemize}
\begin{itemize}
\item {Utilização:Ant.}
\end{itemize}
\begin{itemize}
\item {Proveniência:(Do lat. hyp. \textunderscore peduca\textunderscore , de \textunderscore pes\textunderscore , \textunderscore pedis\textunderscore )}
\end{itemize}
Meia curta; cothurno.
Sapato.
\section{Peugada}
\begin{itemize}
\item {fónica:pe-u}
\end{itemize}
\begin{itemize}
\item {Grp. gram.:f.}
\end{itemize}
O mesmo que \textunderscore pègada\textunderscore .
Vestígio, rasto, encalço: \textunderscore foi-lhe na peugada\textunderscore .
(Cp. \textunderscore peúga\textunderscore )
\section{Peúva}
\begin{itemize}
\item {Grp. gram.:f.}
\end{itemize}
Planta bignoniácea do Brasil.
\section{Pevida}
\begin{itemize}
\item {Grp. gram.:f.}
\end{itemize}
\begin{itemize}
\item {Utilização:Ant.}
\end{itemize}
\begin{itemize}
\item {Proveniência:(Do b. lat. \textunderscore pipita\textunderscore )}
\end{itemize}
O mesmo que \textunderscore pevide\textunderscore .
\section{Pevide}
\begin{itemize}
\item {Grp. gram.:f.}
\end{itemize}
\begin{itemize}
\item {Utilização:Gír.}
\end{itemize}
Semente de vários frutos carnosos.
Pellícula mórbida, na língua de algumas aves.
Parte carbonizada de torcida ou pavio.
Espécie de massa de farinha, com a fórma de pevide de melão.
Defeito, que não deixa pronunciar o \textunderscore r\textunderscore .
Ânus.
Variedade de pera, a mesma que \textunderscore sorvete\textunderscore .
(Cp. \textunderscore pevida\textunderscore )
\section{Pevidoso}
\begin{itemize}
\item {Grp. gram.:adj.}
\end{itemize}
Que tem pevide.
\section{Pevitada}
\begin{itemize}
\item {Grp. gram.:f.}
\end{itemize}
Poção, feita com pevides guisadas e diluidas em água.
\section{Pexã}
\begin{itemize}
\item {Grp. gram.:f.  e  adj.}
\end{itemize}
Variedade de uva serôdia; o mesmo que \textunderscore pexão\textunderscore .
\section{Pexan}
\begin{itemize}
\item {Grp. gram.:f.  e  adj.}
\end{itemize}
Variedade de uva serôdia; o mesmo que \textunderscore pexão\textunderscore .
\section{Pexão}
\begin{itemize}
\item {Grp. gram.:m.}
\end{itemize}
Casta de uva do districto de Leiria.
\section{Pexe}
\begin{itemize}
\item {Grp. gram.:m.}
\end{itemize}
\begin{itemize}
\item {Utilização:ant.}
\end{itemize}
\begin{itemize}
\item {Utilização:Pop.}
\end{itemize}
O mesmo que \textunderscore peixe\textunderscore . Cf. \textunderscore Lusíadas\textunderscore , V, 27 e VI, 24.
\section{Pexego}
\begin{itemize}
\item {fónica:xê}
\end{itemize}
\begin{itemize}
\item {Grp. gram.:m.}
\end{itemize}
\begin{itemize}
\item {Utilização:Prov.}
\end{itemize}
\begin{itemize}
\item {Utilização:trasm.}
\end{itemize}
O mesmo que \textunderscore pêssego\textunderscore .
\section{Pexerica}
\begin{itemize}
\item {Grp. gram.:f.}
\end{itemize}
\begin{itemize}
\item {Utilização:Bras}
\end{itemize}
Planta, de flôres escarlates, e cujas fôlhas são verdes de um lado e roxas do outro.
\section{Pexote}
\begin{itemize}
\item {Grp. gram.:m.}
\end{itemize}
O mesmo ou melhor que \textunderscore pechote\textunderscore , se a etym. é a loc. chin. \textunderscore pe xot\textunderscore , (não sei).
\section{Pez}
\begin{itemize}
\item {Grp. gram.:m.}
\end{itemize}
\begin{itemize}
\item {Proveniência:(Do lat. \textunderscore pix\textunderscore )}
\end{itemize}
Secreção resinosa de várias árvores coníferas, especialmente do pinheiro.
Alcatrão; breu.
\section{Pezanho}
\begin{itemize}
\item {Grp. gram.:adj.}
\end{itemize}
Que tem côr de pez.
\section{Pezenho}
\begin{itemize}
\item {Grp. gram.:adj.}
\end{itemize}
O mesmo ou melhor que \textunderscore pezanho\textunderscore .
\section{Pezgada}
\begin{itemize}
\item {Grp. gram.:adj.}
\end{itemize}
\begin{itemize}
\item {Utilização:Prov.}
\end{itemize}
\begin{itemize}
\item {Proveniência:(De \textunderscore pez\textunderscore )}
\end{itemize}
Diz-se de certas talhas, em que se coze o vinho.
\section{Pezgar}
\begin{itemize}
\item {Grp. gram.:v. t.}
\end{itemize}
O mesmo ou melhor que \textunderscore pesgar\textunderscore . Cf. B. Pereira, \textunderscore Prosodia\textunderscore , vb. \textunderscore pissaliphes\textunderscore .
\section{Pèzinhos}
\begin{itemize}
\item {Grp. gram.:m. pl.}
\end{itemize}
\begin{itemize}
\item {Utilização:Prov.}
\end{itemize}
\begin{itemize}
\item {Utilização:alent.}
\end{itemize}
\begin{itemize}
\item {Proveniência:(De \textunderscore pé\textunderscore )}
\end{itemize}
Peúgas.
\section{Peziza}
\begin{itemize}
\item {Grp. gram.:f.}
\end{itemize}
\begin{itemize}
\item {Proveniência:(Do gr. \textunderscore pezikos\textunderscore )}
\end{itemize}
Gênero de cogumelos, que crescem na terra sôbre substâncias vegetaes.
\section{Pezizóide}
\begin{itemize}
\item {Grp. gram.:adj.}
\end{itemize}
\begin{itemize}
\item {Grp. gram.:Pl.}
\end{itemize}
\begin{itemize}
\item {Proveniência:(Do gr. \textunderscore pezikos\textunderscore  + \textunderscore eidos\textunderscore )}
\end{itemize}
Relativo ou semelhante á peziza.
Família de cogumelos, que têm por typo a peziza.
\section{Pezóporo}
\begin{itemize}
\item {Grp. gram.:m.}
\end{itemize}
\begin{itemize}
\item {Proveniência:(Do gr. \textunderscore pezos\textunderscore  + \textunderscore poreuein\textunderscore )}
\end{itemize}
Gênero de aves aprehensoras, um pouco semelhantes ao papagaio.
\section{Pezunho}
\begin{itemize}
\item {Grp. gram.:m.}
\end{itemize}
O mesmo ou melhor que \textunderscore pesunho\textunderscore . Cf. G. Viana, \textunderscore Apostilas\textunderscore .
\section{Pg.}
Abrev. de \textunderscore pagou\textunderscore , em livros de contas do commércio a retalho.
Abrev. de \textunderscore português\textunderscore , em algumas obras de Philologia.
\section{Ph...}
Grupo de letras, que entra na composição de muitas palavras de origem grega, mas dispensável, se não inútil, no português, e já dispensado no cast. e no it. Tem o valor de \textunderscore f\textunderscore , que o póde substituir legitimamente. Portanto, \textunderscore philósopho\textunderscore  = \textunderscore filósofo\textunderscore , \textunderscore phenol\textunderscore  = \textunderscore fenol\textunderscore , etc.
\section{Phaca}
\begin{itemize}
\item {Grp. gram.:f.}
\end{itemize}
\begin{itemize}
\item {Proveniência:(Do gr. \textunderscore phake\textunderscore )}
\end{itemize}
Gênero de plantas leguminosas.
\section{Phacélia}
\begin{itemize}
\item {Grp. gram.:f.}
\end{itemize}
\begin{itemize}
\item {Proveniência:(Do gr. \textunderscore phakellos\textunderscore )}
\end{itemize}
Gênero de plantas da América do Norte.
\section{Phacocele}
\begin{itemize}
\item {Grp. gram.:f.}
\end{itemize}
\begin{itemize}
\item {Proveniência:(Do gr. \textunderscore phakos\textunderscore  + \textunderscore kele\textunderscore )}
\end{itemize}
Hérnia do crystallino do ôlho.
\section{Phacohydropisia}
\begin{itemize}
\item {Grp. gram.:f.}
\end{itemize}
\begin{itemize}
\item {Utilização:Med.}
\end{itemize}
Hydropisia da cápsula do crystallino.
\section{Phacóide}
\begin{itemize}
\item {Grp. gram.:adj.}
\end{itemize}
\begin{itemize}
\item {Proveniência:(Lat. \textunderscore phacoides\textunderscore )}
\end{itemize}
Que tem fórma de lentilha.
\section{Phacólitho}
\begin{itemize}
\item {Grp. gram.:m.}
\end{itemize}
\begin{itemize}
\item {Utilização:Miner.}
\end{itemize}
\begin{itemize}
\item {Proveniência:(Do gr. \textunderscore phakos\textunderscore  + \textunderscore lithos\textunderscore )}
\end{itemize}
Zeólitho, de base de potassa, soda e cal.
\section{Phacomalacia}
\begin{itemize}
\item {Grp. gram.:f.}
\end{itemize}
\begin{itemize}
\item {Utilização:Med.}
\end{itemize}
\begin{itemize}
\item {Proveniência:(Do gr. \textunderscore phakos\textunderscore  + \textunderscore malakia\textunderscore )}
\end{itemize}
Amollecimento do crystallino.
\section{Phacómetro}
\begin{itemize}
\item {Grp. gram.:m.}
\end{itemize}
\begin{itemize}
\item {Proveniência:(Do gr. \textunderscore phakos\textunderscore  + \textunderscore metron\textunderscore )}
\end{itemize}
Instrumento, para medir as lentes ou determinar-lhes o foco.
\section{Phaconina}
\begin{itemize}
\item {Grp. gram.:f.}
\end{itemize}
\begin{itemize}
\item {Proveniência:(Do gr. \textunderscore phakos\textunderscore )}
\end{itemize}
Substância particular, que se acha no crystallino do ôlho.
\section{Phacosclerose}
\begin{itemize}
\item {Grp. gram.:f.}
\end{itemize}
\begin{itemize}
\item {Utilização:Med.}
\end{itemize}
\begin{itemize}
\item {Proveniência:(Do gr. \textunderscore phakos\textunderscore  + \textunderscore skleros\textunderscore )}
\end{itemize}
Endurecimento do crystallino.
\section{Phacoscopia}
\begin{itemize}
\item {Grp. gram.:f.}
\end{itemize}
\begin{itemize}
\item {Utilização:Med.}
\end{itemize}
\begin{itemize}
\item {Proveniência:(Do gr. \textunderscore phakos\textunderscore  + \textunderscore skopein\textunderscore )}
\end{itemize}
Exploração subjectiva ou pessoal dos meios do globo ocular.
\section{Phaetonte}
\begin{itemize}
\item {Grp. gram.:m.}
\end{itemize}
\begin{itemize}
\item {Proveniência:(De \textunderscore Phaetonte\textunderscore , n. p. myth. de um filho e cocheiro de Júpiter)}
\end{itemize}
Pequena carruagem de quatro rodas, ligeira e descoberta:«\textunderscore ...parar-lhe á porta um phaetonte aéreo.\textunderscore »Filinto, XI, 173.
\section{Phagedênico}
\begin{itemize}
\item {Grp. gram.:adj.}
\end{itemize}
\begin{itemize}
\item {Utilização:Med.}
\end{itemize}
\begin{itemize}
\item {Proveniência:(Gr. \textunderscore phagedainikos\textunderscore )}
\end{itemize}
Diz-se da substância, que corrói a carne morta.
Diz-se da água, que é uma solução de deutochloreto de mercúrio em água de cal.
Diz-se das úlceras corrosivas.
\section{Phagedenismo}
\begin{itemize}
\item {Grp. gram.:m.}
\end{itemize}
Estado ou qualidade de phagedênico.
\section{Phagocytos}
\begin{itemize}
\item {Grp. gram.:m. pl.}
\end{itemize}
\begin{itemize}
\item {Proveniência:(Do gr. \textunderscore phagein\textunderscore  + \textunderscore kutos\textunderscore )}
\end{itemize}
Micróbios benéficos, que destroem ou absorvem os prejudiciaes.
\section{Phagocytose}
\begin{itemize}
\item {Grp. gram.:f.}
\end{itemize}
\begin{itemize}
\item {Proveniência:(Do gr. \textunderscore phagein\textunderscore  + \textunderscore kutos\textunderscore )}
\end{itemize}
Destruição dos micróbios por meio de certas céllulas vivas do organismo, que absorvem as bactérias e as digerem.
\section{Phalacrose}
\begin{itemize}
\item {Grp. gram.:f.}
\end{itemize}
\begin{itemize}
\item {Utilização:Med.}
\end{itemize}
\begin{itemize}
\item {Proveniência:(Gr. \textunderscore phalakrosis\textunderscore )}
\end{itemize}
Quéda dos cabellos; calvície.
\section{Phalangarchía}
\begin{itemize}
\item {fónica:qui}
\end{itemize}
\begin{itemize}
\item {Grp. gram.:f.}
\end{itemize}
Phalange elementar, que, entre os gregos antigos, devia theoricamente sêr composta de 256 homens de frente, por 16 de fundo.
\section{Phalange}
\begin{itemize}
\item {Grp. gram.:f.}
\end{itemize}
\begin{itemize}
\item {Utilização:Anat.}
\end{itemize}
\begin{itemize}
\item {Utilização:Fig.}
\end{itemize}
\begin{itemize}
\item {Proveniência:(Lat. \textunderscore phalanx\textunderscore , \textunderscore phalangis\textunderscore )}
\end{itemize}
Nome, que os Gregos davam á sua ínfantaria.
Corpo de tropas.
Communa societária, no systema de Fourier.
Cada um dos ossos dos dedos, especialmente o ôsso que se articula com o metacarpo.
Multidão.
\section{Phalangeal}
\begin{itemize}
\item {Grp. gram.:adj.}
\end{itemize}
Relativo ás phalanges dos dedos.
\section{Phalangeano}
\begin{itemize}
\item {Grp. gram.:adj.}
\end{itemize}
O mesmo que \textunderscore phalangeal\textunderscore .
\section{Phalangeta}
\begin{itemize}
\item {fónica:gê}
\end{itemize}
\begin{itemize}
\item {Grp. gram.:f.}
\end{itemize}
\begin{itemize}
\item {Utilização:Anat.}
\end{itemize}
Cada uma das últimas phalanges dos dedos ou cada uma das phalanges que têm as unhas.
\section{Phalanginha}
\begin{itemize}
\item {Grp. gram.:f.}
\end{itemize}
\begin{itemize}
\item {Utilização:Anat.}
\end{itemize}
Cada uma das phalanges médias dos dedos em que há três.
\section{Phalângio}
\begin{itemize}
\item {Grp. gram.:m.}
\end{itemize}
Planta ornamental.
O mesmo que \textunderscore phalangita\textunderscore .
\section{Phalangita}
\begin{itemize}
\item {Grp. gram.:m.}
\end{itemize}
\begin{itemize}
\item {Proveniência:(Lat. \textunderscore phalangita\textunderscore )}
\end{itemize}
Soldado de uma phalange, nos exércitos gregos.
\section{Phalansterianismo}
\begin{itemize}
\item {Grp. gram.:m.}
\end{itemize}
O mesmo que \textunderscore phalansterismo\textunderscore .
\section{Phalansteriano}
\begin{itemize}
\item {Grp. gram.:m.  e  adj.}
\end{itemize}
\begin{itemize}
\item {Proveniência:(De \textunderscore phalanstério\textunderscore )}
\end{itemize}
O que habita num phalanstério; sectário de Fourier.
\section{Phalanstério}
\begin{itemize}
\item {Grp. gram.:m.}
\end{itemize}
\begin{itemize}
\item {Proveniência:(De \textunderscore phalange\textunderscore )}
\end{itemize}
Povoação societária, regida pelo systema de Fourier.
\section{Phalansterismo}
\begin{itemize}
\item {Grp. gram.:m.}
\end{itemize}
Conjunto das doutrinas que devem praticar-se no phalanstério.
\section{Phalarídeas}
\begin{itemize}
\item {Grp. gram.:f. pl.}
\end{itemize}
\begin{itemize}
\item {Utilização:Bot.}
\end{itemize}
\begin{itemize}
\item {Proveniência:(Do gr. \textunderscore phalaris\textunderscore , painço)}
\end{itemize}
Tríbo de gramíneas.
\section{Phalécio}
\begin{itemize}
\item {Grp. gram.:m.}
\end{itemize}
\begin{itemize}
\item {Proveniência:(Lat. \textunderscore phalaecium\textunderscore )}
\end{itemize}
O mesmo ou melhor que \textunderscore phalêucio\textunderscore .
\section{Phalena}
\begin{itemize}
\item {Grp. gram.:f.}
\end{itemize}
\begin{itemize}
\item {Proveniência:(Do gr. \textunderscore phalaina\textunderscore )}
\end{itemize}
Espécie de borboleta nocturna.
\section{Phálera}
\begin{itemize}
\item {Grp. gram.:f.}
\end{itemize}
\begin{itemize}
\item {Proveniência:(Lat. \textunderscore phalerae\textunderscore )}
\end{itemize}
Collar de oiro e prata, usado por patrícios e guerreiros, entre os antigos Romanos.
\section{Phaleríneas}
\begin{itemize}
\item {Grp. gram.:f. pl.}
\end{itemize}
\begin{itemize}
\item {Utilização:Bot.}
\end{itemize}
\begin{itemize}
\item {Proveniência:(Do gr. \textunderscore phaleros\textunderscore )}
\end{itemize}
Tríbo de thymeliáceas.
\section{Phalêucio}
\begin{itemize}
\item {Grp. gram.:m.  e  adj.}
\end{itemize}
\begin{itemize}
\item {Proveniência:(Fr. \textunderscore phaleuce\textunderscore )}
\end{itemize}
Verso de cinco pés, entre os Gregos e Romanos.
\section{Phalisco}
\begin{itemize}
\item {Grp. gram.:m.}
\end{itemize}
\begin{itemize}
\item {Proveniência:(Lat. \textunderscore phaliscus\textunderscore )}
\end{itemize}
Verso latino de quatro pés, sendo dáctylos os três primeiros e espondeu o último.
\section{Phallagogia}
\begin{itemize}
\item {Grp. gram.:f.}
\end{itemize}
\begin{itemize}
\item {Proveniência:(Do gr. \textunderscore phallos\textunderscore  + \textunderscore agein\textunderscore )}
\end{itemize}
Festa grega, em que o phallo era conduzido em procissão.
\section{Phallagónias}
\begin{itemize}
\item {Grp. gram.:f. pl.}
\end{itemize}
Antigas festas gregas em honra de Priapo.
(Cp. \textunderscore phallophórias\textunderscore )
\section{Phállicas}
\begin{itemize}
\item {Grp. gram.:f. pl.}
\end{itemize}
\begin{itemize}
\item {Proveniência:(De \textunderscore phállico\textunderscore )}
\end{itemize}
O mesmo que \textunderscore phallophórias\textunderscore .
\section{Phallicismo}
\begin{itemize}
\item {Grp. gram.:m.}
\end{itemize}
Culto phállico.
(Cp. \textunderscore phállico\textunderscore )
\section{Phállico}
\begin{itemize}
\item {Grp. gram.:adj.}
\end{itemize}
Relativo ao phallo ou ao seu culto.
\section{Phallite}
\begin{itemize}
\item {Grp. gram.:f.}
\end{itemize}
\begin{itemize}
\item {Utilização:Med.}
\end{itemize}
\begin{itemize}
\item {Proveniência:(Do gr. \textunderscore phallos\textunderscore )}
\end{itemize}
Inflammação do pênis.
\section{Phallo}
\begin{itemize}
\item {Grp. gram.:m.}
\end{itemize}
\begin{itemize}
\item {Proveniência:(Do gr. \textunderscore phallos\textunderscore )}
\end{itemize}
Representação do membro viril, adorada entre os antigos, como sýmbolo da fecundidade da natureza, e correspondente ao linga dos Índios.
\section{Phallodynia}
\begin{itemize}
\item {Grp. gram.:f.}
\end{itemize}
\begin{itemize}
\item {Utilização:Med.}
\end{itemize}
\begin{itemize}
\item {Proveniência:(Do gr. \textunderscore phallos\textunderscore  + \textunderscore odune\textunderscore )}
\end{itemize}
Dôr no pênis.
\section{Phallóides}
\begin{itemize}
\item {Grp. gram.:m. pl.}
\end{itemize}
\begin{itemize}
\item {Utilização:Bot.}
\end{itemize}
\begin{itemize}
\item {Proveniência:(Do gr. \textunderscore phallos\textunderscore  + \textunderscore eidos\textunderscore )}
\end{itemize}
Cogumelos, que formam uma secção no systema de Broguiart.
\section{Phallophórias}
\begin{itemize}
\item {Grp. gram.:f. pl.}
\end{itemize}
\begin{itemize}
\item {Proveniência:(De \textunderscore phallóphoro\textunderscore )}
\end{itemize}
Festas pagans em honra do phallo.
\section{Phallóphoro}
\begin{itemize}
\item {Grp. gram.:m.}
\end{itemize}
\begin{itemize}
\item {Proveniência:(Do gr. \textunderscore phallos\textunderscore  + \textunderscore phorein\textunderscore )}
\end{itemize}
Sacerdote grego, que em procissões ou dias de festas transportava o phallo.
\section{Phallorrhagia}
\begin{itemize}
\item {Grp. gram.:f.}
\end{itemize}
\begin{itemize}
\item {Utilização:Med.}
\end{itemize}
\begin{itemize}
\item {Proveniência:(Do gr. \textunderscore phallos\textunderscore  + \textunderscore rhagein\textunderscore )}
\end{itemize}
Hemorragía á superficie do pênis.
\section{Phanerantho}
\begin{itemize}
\item {Grp. gram.:adj.}
\end{itemize}
\begin{itemize}
\item {Utilização:Bot.}
\end{itemize}
\begin{itemize}
\item {Proveniência:(Do gr. \textunderscore phaneros\textunderscore  + \textunderscore anthos\textunderscore )}
\end{itemize}
Que tem flôres apparentes.
\section{Phânero}
\begin{itemize}
\item {Grp. gram.:m.}
\end{itemize}
\begin{itemize}
\item {Utilização:Anat.}
\end{itemize}
\begin{itemize}
\item {Proveniência:(Gr. \textunderscore phaneros\textunderscore )}
\end{itemize}
Qualquer producção visível e persistente á superfície da pelle, como os pêlos, cornos, etc.
\section{Phanero...}
\begin{itemize}
\item {Grp. gram.:pref.}
\end{itemize}
\begin{itemize}
\item {Proveniência:(Gr. \textunderscore phaneros\textunderscore )}
\end{itemize}
(designativo de \textunderscore manifesto\textunderscore , \textunderscore apparente\textunderscore )
\section{Phanerocarpo}
\begin{itemize}
\item {Grp. gram.:adj.}
\end{itemize}
\begin{itemize}
\item {Utilização:Bot.}
\end{itemize}
\begin{itemize}
\item {Proveniência:(Do gr. \textunderscore phaneros\textunderscore  + \textunderscore karpos\textunderscore )}
\end{itemize}
Que tem apparentes os frutos ou os corpúsculos reproductores.
\section{Phanerocotyledóneas}
\begin{itemize}
\item {Grp. gram.:f. pl.}
\end{itemize}
Plantas, cujos cotylédones são apparentes ou fáceis de distinguir.
O mesmo que \textunderscore dicotyledóneas\textunderscore .
\section{Phanerogamia}
\begin{itemize}
\item {Grp. gram.:f.}
\end{itemize}
\begin{itemize}
\item {Utilização:Bot.}
\end{itemize}
\begin{itemize}
\item {Proveniência:(Do gr. \textunderscore phaneros\textunderscore  + \textunderscore gamos\textunderscore )}
\end{itemize}
Estado de uma planta ou animal, que tem os órgãos sexuaes apparentes.
\section{Phanerogâmicas}
\begin{itemize}
\item {Grp. gram.:f. pl.}
\end{itemize}
\begin{itemize}
\item {Utilização:Bot.}
\end{itemize}
Grande divisão do reino vegetal, que abrange todas as espécies que têm órgãos sexuaes apparentes.
(Fem. pl. de \textunderscore phanerogâmico\textunderscore )
\section{Phanerogâmico}
\begin{itemize}
\item {Grp. gram.:adj.}
\end{itemize}
\begin{itemize}
\item {Proveniência:(De \textunderscore phanerogamia\textunderscore )}
\end{itemize}
Que tem órgãos sexuaes apparentes, (falando-se de plantas).
\section{Phanerógamo}
\begin{itemize}
\item {Grp. gram.:adj.}
\end{itemize}
O mesmo que \textunderscore phanerogâmico\textunderscore .
\section{Phaneróphoro}
\begin{itemize}
\item {Grp. gram.:adj.}
\end{itemize}
\begin{itemize}
\item {Utilização:Anat.}
\end{itemize}
\begin{itemize}
\item {Proveniência:(Do gr. \textunderscore phaneros\textunderscore  + \textunderscore phoros\textunderscore )}
\end{itemize}
Que tem phâneros.
\section{Phantasia}
\textunderscore f.\textunderscore  (e der.)
(V. \textunderscore fantasia\textunderscore , etc.)
\section{Pharaó}
\begin{itemize}
\item {Grp. gram.:m.}
\end{itemize}
\begin{itemize}
\item {Proveniência:(Lat. \textunderscore pharao\textunderscore , do hebr.)}
\end{itemize}
Título commum a soberanos do antigo Egypto.
\section{Pharaónico}
\begin{itemize}
\item {Grp. gram.:adj.}
\end{itemize}
Relativo aos pharaós ou ao seu tempo.
\section{Pharetrado}
\begin{itemize}
\item {Grp. gram.:adj.}
\end{itemize}
\begin{itemize}
\item {Utilização:Poét.}
\end{itemize}
\begin{itemize}
\item {Proveniência:(Do lat. \textunderscore pharetratus\textunderscore )}
\end{itemize}
Que usa ou leva aljava. Cf. Castilho, \textunderscore Geórg.\textunderscore , 259.
\section{Pharisaico}
\begin{itemize}
\item {Grp. gram.:adj.}
\end{itemize}
\begin{itemize}
\item {Utilização:Fig.}
\end{itemize}
\begin{itemize}
\item {Proveniência:(Lat. \textunderscore pharisaicus\textunderscore )}
\end{itemize}
Relativo a phariseu; próprio de phariseu.
Hypócrita.
\section{Pharisaísmo}
\begin{itemize}
\item {Grp. gram.:m.}
\end{itemize}
Carácter dos phariseus; hypocrisia.
\section{Phariseu}
\begin{itemize}
\item {Grp. gram.:m.}
\end{itemize}
\begin{itemize}
\item {Utilização:Fig.}
\end{itemize}
\begin{itemize}
\item {Utilização:Pop.}
\end{itemize}
\begin{itemize}
\item {Proveniência:(Lat. \textunderscore pharisaeus\textunderscore )}
\end{itemize}
Membro de uma seita judaica, caracterizada por ostentar grande santidade exterior.
Aquelle que apparenta santidade, não a tendo.
Indivíduo, cujo aspecto repellente denota má índole.
\section{Phariseu}
\begin{itemize}
\item {Grp. gram.:m.}
\end{itemize}
\begin{itemize}
\item {Utilização:Ant.}
\end{itemize}
\begin{itemize}
\item {Utilização:Pop.}
\end{itemize}
Enxergão de palha.
\section{Pharmaceuta}
\begin{itemize}
\item {Grp. gram.:m.}
\end{itemize}
O mesmo que \textunderscore pharmacêutico\textunderscore .
\section{Pharmacêutico}
\begin{itemize}
\item {Grp. gram.:adj.}
\end{itemize}
\begin{itemize}
\item {Grp. gram.:M.}
\end{itemize}
\begin{itemize}
\item {Proveniência:(Lat. \textunderscore pharmaceuticus\textunderscore )}
\end{itemize}
Relativo a pharmácia.
Aquelle que exerce a pharmácia; boticário.
\section{Pharmácia}
\begin{itemize}
\item {Grp. gram.:f.}
\end{itemize}
\begin{itemize}
\item {Proveniência:(Gr. \textunderscore pharmakeia\textunderscore )}
\end{itemize}
Arte de preparar medicamentos e de conhecer e conservar as drogas simples.
Estabelecimento, em que se preparam ou vendem medicamentos.
Botica.
Profissão de pharmacêutico.
Collecção de medicamentos.--A pronúncia exacta seria \textunderscore farmacía\textunderscore , que se não usa.
\section{Pharmacoco}
\begin{itemize}
\item {fónica:cô}
\end{itemize}
\begin{itemize}
\item {Grp. gram.:m.}
\end{itemize}
\begin{itemize}
\item {Utilização:T. de Coímbra}
\end{itemize}
Estudante de pharmácia; pharmacopola.
(Cp. \textunderscore pharmácia\textunderscore )
\section{Pharmacodynamia}
\begin{itemize}
\item {Grp. gram.:f.}
\end{itemize}
\begin{itemize}
\item {Proveniência:(Do gr. \textunderscore pharmakon\textunderscore  + \textunderscore dunamis\textunderscore )}
\end{itemize}
Estudo dos effeitos physiológicos dos medicamentos, ou da acção destes no organismo em estado de saúde. Cf. \textunderscore Pharmacopeia Port.\textunderscore 
\section{Pharmacognosia}
\begin{itemize}
\item {Grp. gram.:f.}
\end{itemize}
Estudo das drogas, antes de preparadas na pharmácia. Cf. \textunderscore Pharmacopeia Port.\textunderscore 
\section{Pharmacographia}
\begin{itemize}
\item {Grp. gram.:f.}
\end{itemize}
\begin{itemize}
\item {Proveniência:(Do gr. \textunderscore pharmakon\textunderscore  + \textunderscore graphein\textunderscore )}
\end{itemize}
Tratado das substâncias medicinaes.
\section{Pharmacográphico}
\begin{itemize}
\item {Grp. gram.:adj.}
\end{itemize}
Relativo á pharmacographia.
\section{Pharmacólitho}
\begin{itemize}
\item {Grp. gram.:m.}
\end{itemize}
\begin{itemize}
\item {Utilização:Chím.}
\end{itemize}
\begin{itemize}
\item {Proveniência:(Do gr. \textunderscore pharmakon\textunderscore  + \textunderscore lithos\textunderscore )}
\end{itemize}
Cal arseniatada da Alemanha.
\section{Pharmacologia}
\begin{itemize}
\item {Grp. gram.:f.}
\end{itemize}
\begin{itemize}
\item {Proveniência:(Do gr. \textunderscore pharmakon\textunderscore  + \textunderscore logos\textunderscore )}
\end{itemize}
Parte da matéria médica, que tem por objecto fazer conhecer os medicamentos e ensinar a applicação delles.
\section{Pharmacológico}
\begin{itemize}
\item {Grp. gram.:adj.}
\end{itemize}
Relativo á pharmacologia.
\section{Pharmacopeia}
\begin{itemize}
\item {Grp. gram.:f.}
\end{itemize}
\begin{itemize}
\item {Proveniência:(Gr. \textunderscore pharmacopeia\textunderscore )}
\end{itemize}
Livro, que ensina a prepararar e compôr medicamentos; tratado á cêrca de medicamentos.
\section{Pharmacopola}
\begin{itemize}
\item {Grp. gram.:m.}
\end{itemize}
\begin{itemize}
\item {Utilização:Burl.}
\end{itemize}
\begin{itemize}
\item {Proveniência:(Lat. \textunderscore pharmacopola\textunderscore )}
\end{itemize}
Boticário.
Charlatão.
\section{Pharmacopolia}
\begin{itemize}
\item {Grp. gram.:f.}
\end{itemize}
\begin{itemize}
\item {Proveniência:(De \textunderscore pharmacopola\textunderscore )}
\end{itemize}
Pharmácia.
Collécção de medicamentos.
\section{Pharmacopólio}
\begin{itemize}
\item {Grp. gram.:adj.}
\end{itemize}
Relativo a pharmacopola.
\section{Pharmacoposia}
\begin{itemize}
\item {Grp. gram.:f.}
\end{itemize}
\begin{itemize}
\item {Proveniência:(Gr. \textunderscore pharmacoposia\textunderscore )}
\end{itemize}
Acto de tomar um medicamento, especialmente um purgante.
\section{Pharmacosiderite}
\begin{itemize}
\item {fónica:si}
\end{itemize}
\begin{itemize}
\item {Grp. gram.:f.}
\end{itemize}
Ferro arseniatado, associado a filões de estanho.
\section{Pharmacosiderito}
\begin{itemize}
\item {fónica:si}
\end{itemize}
\begin{itemize}
\item {Grp. gram.:m.}
\end{itemize}
O mesmo ou melhor que \textunderscore pharmacosiderite\textunderscore .
\section{Pharmacotechnia}
\begin{itemize}
\item {Grp. gram.:f.}
\end{itemize}
\begin{itemize}
\item {Proveniência:(Do gr. \textunderscore pharmakon\textunderscore  + \textunderscore tekhne\textunderscore )}
\end{itemize}
Tratado das preparações pharmacêuticas.
\section{Pharmacotéchnico}
\begin{itemize}
\item {Grp. gram.:adj.}
\end{itemize}
Relativo á pharmacotechnia.
\section{Pharmacotherapia}
\begin{itemize}
\item {Grp. gram.:f.}
\end{itemize}
Estudo dos effeitos dos medicamentos no organismo enfermo.
Conhecimento das fórmulas pharmacêuticas.
Modo de dosar e ministrar medicamentos. Cf. \textunderscore Pharmacopeia Port.\textunderscore 
\section{Pharo}
\begin{itemize}
\item {Grp. gram.:m.}
\end{itemize}
Terra ou lugar, em que há pharol, para guia de navegantes.
O mesmo que \textunderscore pharol\textunderscore .
\section{Pharol}
\begin{itemize}
\item {Grp. gram.:m.}
\end{itemize}
\begin{itemize}
\item {Utilização:Náut.}
\end{itemize}
\begin{itemize}
\item {Utilização:Fig.}
\end{itemize}
\begin{itemize}
\item {Utilização:Taur.}
\end{itemize}
\begin{itemize}
\item {Proveniência:(Do gr. \textunderscore Pharos\textunderscore , n. p.)}
\end{itemize}
Candeeiro volante, usado a bordo e nos postos semaphóricos, para communicação de sinaes ou para cumprimento do regulamento dos portos e das regras internacionaes sôbre a maneira de evitar abalroamentos.
\textunderscore Pharol de eclipses\textunderscore , farol para illuminação de costas e baixios, disposto de modo que os differentes sectores do horizonte são alumiados e ficam em trevas, alternada e successivamente.
Lampião na popa da embarcação ou na gávea do mastaréu da gata.
Coisa que alumia.
Direcção.
Director.
Guia; norte.
Nome de uma sorte de bandarilheiro.
\section{Pharoleiro}
\begin{itemize}
\item {Grp. gram.:m.}
\end{itemize}
Indivíduo, encarregado de guardar ou tratar de um pharol.
\section{Pharolete}
\begin{itemize}
\item {fónica:lê}
\end{itemize}
\begin{itemize}
\item {Grp. gram.:m.}
\end{itemize}
\begin{itemize}
\item {Utilização:Bras}
\end{itemize}
Pequeno pharol.
\section{Pharolim}
\begin{itemize}
\item {Grp. gram.:m.}
\end{itemize}
Pequeno pharol.
\section{Pharolização}
\begin{itemize}
\item {Grp. gram.:f.}
\end{itemize}
\begin{itemize}
\item {Utilização:Neol.}
\end{itemize}
Acto de pharolizar ou estabelecer pharoes.
\section{Pharolizar}
\begin{itemize}
\item {Grp. gram.:v. t.}
\end{itemize}
\begin{itemize}
\item {Utilização:Neol.}
\end{itemize}
\begin{itemize}
\item {Utilização:Fig.}
\end{itemize}
\begin{itemize}
\item {Proveniência:(De \textunderscore pharol\textunderscore )}
\end{itemize}
Estabelecer pharoes em: \textunderscore pharolizar um pôrto de mar\textunderscore .
Espalhar luz sôbre; illuminar; esclarecêr.
\section{Pharsálico}
\begin{itemize}
\item {Grp. gram.:adj.}
\end{itemize}
Relativo á Pharsália.
\section{Pharynge}
\begin{itemize}
\item {Grp. gram.:f.}
\end{itemize}
\begin{itemize}
\item {Proveniência:(Do gr. \textunderscore pharunge\textunderscore )}
\end{itemize}
Cavidade músculo-membranosa, entre a bôca e a parte superior do esóphago.
\section{Pharýngeo}
\begin{itemize}
\item {Grp. gram.:adj.}
\end{itemize}
Relativo á pharynge.
\section{Pharyngite}
\begin{itemize}
\item {Grp. gram.:f.}
\end{itemize}
Inflammação da pharynge.
\section{Pharyngocele}
\begin{itemize}
\item {Grp. gram.:m.}
\end{itemize}
\begin{itemize}
\item {Proveniência:(Do gr. \textunderscore pharunge\textunderscore  + \textunderscore kele\textunderscore )}
\end{itemize}
Espécie de tumor, resultante de uma dilatação anormal da pharynge.
\section{Pharyngographia}
\begin{itemize}
\item {Grp. gram.:f.}
\end{itemize}
\begin{itemize}
\item {Proveniência:(Do gr. \textunderscore pharunge\textunderscore  + \textunderscore graphein\textunderscore )}
\end{itemize}
Descripção da pharynge.
\section{Pharyngográphico}
\begin{itemize}
\item {Grp. gram.:adj.}
\end{itemize}
Relativo á pharyngographia.
\section{Pharyngo-laryngite}
\begin{itemize}
\item {Grp. gram.:f.}
\end{itemize}
Pharyngite, complicada com laryngite.
\section{Pharyngologia}
\begin{itemize}
\item {Grp. gram.:f.}
\end{itemize}
\begin{itemize}
\item {Proveniência:(Do gr. \textunderscore pharunge\textunderscore  + \textunderscore logos\textunderscore )}
\end{itemize}
Tratado á cêrca da pharynge.
\section{Pharyngológico}
\begin{itemize}
\item {Grp. gram.:adj.}
\end{itemize}
Relativo á pharyngologia.
\section{Pharyngoplegia}
\begin{itemize}
\item {Grp. gram.:f.}
\end{itemize}
\begin{itemize}
\item {Proveniência:(Do gr. \textunderscore pharunge\textunderscore  + \textunderscore plessein\textunderscore )}
\end{itemize}
Paralysia da pharynge.
\section{Pharyngoplégico}
\begin{itemize}
\item {Grp. gram.:adj.}
\end{itemize}
Relativo á pharyngoplegia.
\section{Pharyngoscópio}
\begin{itemize}
\item {Grp. gram.:m.}
\end{itemize}
\begin{itemize}
\item {Utilização:Med.}
\end{itemize}
\begin{itemize}
\item {Proveniência:(Do gr. \textunderscore pharunge\textunderscore  + \textunderscore skopein\textunderscore )}
\end{itemize}
Instrumento, para observar a pharynge.
\section{Pharyngóstomo}
\begin{itemize}
\item {Grp. gram.:adj.}
\end{itemize}
\begin{itemize}
\item {Proveniência:(Do gr. \textunderscore pharungè\textunderscore  + \textunderscore stoma\textunderscore )}
\end{itemize}
Diz-se dos animaes, cuja bôca é constituida pelos bordos do esóphago.
\section{Pharyngotomia}
\begin{itemize}
\item {Grp. gram.:f.}
\end{itemize}
\begin{itemize}
\item {Proveniência:(De \textunderscore pharyngótomo\textunderscore )}
\end{itemize}
Incisão na pharynge.
\section{Pharyngótomo}
\begin{itemize}
\item {Grp. gram.:m.}
\end{itemize}
\begin{itemize}
\item {Proveniência:(Do gr. \textunderscore pharunge\textunderscore  + \textunderscore tome\textunderscore )}
\end{itemize}
Instrumento, com que se pratica a pharyngotomia.
\section{Phascáceas}
\begin{itemize}
\item {Grp. gram.:pl.}
\end{itemize}
\begin{itemize}
\item {Utilização:Bot.}
\end{itemize}
\begin{itemize}
\item {Proveniência:(Do gr. \textunderscore phaskos\textunderscore )}
\end{itemize}
Ordem de musgos.
\section{Phascólomo}
\begin{itemize}
\item {Grp. gram.:m.}
\end{itemize}
\begin{itemize}
\item {Proveniência:(Do gr. \textunderscore phaskolon\textunderscore  + \textunderscore mus\textunderscore )}
\end{itemize}
Gênero de mammíferos australianos.
\section{Phase}
\begin{itemize}
\item {Grp. gram.:f.}
\end{itemize}
\begin{itemize}
\item {Proveniência:(Gr. \textunderscore phasis\textunderscore )}
\end{itemize}
Cada um dos aspectos diversos da Lua e de outros planetas, segundo a maneira, com que recebem a luz do Sol.
Cada uma das modificações successivas que se notam em certas coisas: \textunderscore as phases de uma questão\textunderscore .
Cada um dos differentes aspectos que uma coisa apresenta successivamente: \textunderscore as phases da vida\textunderscore .
\section{Phaseolar}
\begin{itemize}
\item {Grp. gram.:adj.}
\end{itemize}
\begin{itemize}
\item {Proveniência:(Do lat. \textunderscore phaseolum\textunderscore )}
\end{itemize}
Que tem forma de feijão, (o rim, por exemplo).
\section{Phaseóleas}
\begin{itemize}
\item {Grp. gram.:f. pl.}
\end{itemize}
\begin{itemize}
\item {Proveniência:(De \textunderscore phaséolo\textunderscore )}
\end{itemize}
Tríbo de plantas leguminosas, no systema de De-Condolle.
\section{Phaseólico}
\begin{itemize}
\item {Grp. gram.:adj.}
\end{itemize}
\begin{itemize}
\item {Proveniência:(De \textunderscore phaséolo\textunderscore )}
\end{itemize}
Diz-se de um ácido, que existe em certas qualidades de feijão.
\section{Phaseolina}
\begin{itemize}
\item {Grp. gram.:f.}
\end{itemize}
\begin{itemize}
\item {Proveniência:(De \textunderscore phaséolo\textunderscore )}
\end{itemize}
Substância crystallina, extrahida de uma espécie de feijão.
\section{Phaséolo}
\begin{itemize}
\item {Grp. gram.:m.}
\end{itemize}
\begin{itemize}
\item {Proveniência:(Lat. \textunderscore phaseolus\textunderscore )}
\end{itemize}
Nome scientífico do feijão.
\section{Phateusim}
\begin{itemize}
\item {Grp. gram.:m.  e  adj.}
\end{itemize}
(V.emphyteuta)
\section{Phatniorrhagia}
\begin{itemize}
\item {Grp. gram.:f.}
\end{itemize}
\begin{itemize}
\item {Proveniência:(Do gr. \textunderscore phatnia\textunderscore  + \textunderscore rhagein\textunderscore )}
\end{itemize}
Hemorrhagía pelo alvéolo de um dente.
\section{Phebeu}
\begin{itemize}
\item {Grp. gram.:adj.}
\end{itemize}
\begin{itemize}
\item {Proveniência:(Lat. \textunderscore phoebeus\textunderscore )}
\end{itemize}
Relativo ao Sol.
\section{Phebo}
\begin{itemize}
\item {Grp. gram.:m.}
\end{itemize}
\begin{itemize}
\item {Utilização:Poét.}
\end{itemize}
\begin{itemize}
\item {Proveniência:(Do gr. \textunderscore Phoibos\textunderscore , n. p.)}
\end{itemize}
O Sol.
\section{Pheléu}
\begin{itemize}
\item {Grp. gram.:m.}
\end{itemize}
Gênero de plantas gramíneas, (\textunderscore pheleum pratense\textunderscore , Lin).
O mesmo que \textunderscore fléolo\textunderscore ? Cf. \textunderscore Gazeta das Ald.\textunderscore , n.^o 108.
\section{Phellândrio}
\begin{itemize}
\item {Grp. gram.:m.}
\end{itemize}
\begin{itemize}
\item {Proveniência:(Lat. \textunderscore phellandrion\textunderscore )}
\end{itemize}
Planta umbellífera e venenosa dos terrenos pantanosos.
\section{Phellocarpo}
\begin{itemize}
\item {Grp. gram.:m.}
\end{itemize}
\begin{itemize}
\item {Proveniência:(Do gr. \textunderscore phellos\textunderscore  + \textunderscore karpos\textunderscore )}
\end{itemize}
Gênero de plantas leguminosas da América tropical.
\section{Phelloderme}
\begin{itemize}
\item {Grp. gram.:f.}
\end{itemize}
\begin{itemize}
\item {Utilização:Bot.}
\end{itemize}
\begin{itemize}
\item {Proveniência:(Do gr. \textunderscore phellos\textunderscore  + \textunderscore derma\textunderscore )}
\end{itemize}
Zona, que reveste a endoderme ou parte interna do systema tegumentar, por baixo da capa suberosa.
\section{Phellogênio}
\begin{itemize}
\item {Grp. gram.:m.}
\end{itemize}
\begin{itemize}
\item {Utilização:Bot.}
\end{itemize}
\begin{itemize}
\item {Proveniência:(Do gr. \textunderscore phellos\textunderscore  + \textunderscore genea\textunderscore )}
\end{itemize}
Parte do systema tegumentar que dá origem á phelloderme.
\section{Phelloplástica}
\begin{itemize}
\item {Grp. gram.:f.}
\end{itemize}
\begin{itemize}
\item {Utilização:Restrict.}
\end{itemize}
\begin{itemize}
\item {Proveniência:(Do gr. \textunderscore phellos\textunderscore  + \textunderscore plassein\textunderscore )}
\end{itemize}
Arte de esculpir em cortiça.
Arte de representar em cortiça monumentos de architectura.
\section{Phellose}
\begin{itemize}
\item {Grp. gram.:f.}
\end{itemize}
\begin{itemize}
\item {Proveniência:(Do gr. \textunderscore phellos\textunderscore )}
\end{itemize}
Producção accidental de uma espécie de cortiça em alguns vegetaes.
\section{Phena}
\begin{itemize}
\item {Grp. gram.:f.}
\end{itemize}
Espécie de abutre.
\section{Phenacetina}
\begin{itemize}
\item {Grp. gram.:f.}
\end{itemize}
Composto chímico, empregado como febrífugo ou antiséptico, e analgésico ou antineurálgico.
\section{Phenato}
\begin{itemize}
\item {Grp. gram.:m.}
\end{itemize}
\begin{itemize}
\item {Utilização:Chím.}
\end{itemize}
Gênero de sáes, formados pelo ácido phênico.
\section{Phene}
\begin{itemize}
\item {Grp. gram.:m.}
\end{itemize}
O mesmo que \textunderscore benzina\textunderscore .
\section{Phenedina}
\begin{itemize}
\item {Grp. gram.:f.}
\end{itemize}
O mesmo que \textunderscore phenacetina\textunderscore .
\section{Phenetol}
\begin{itemize}
\item {Grp. gram.:m.}
\end{itemize}
\begin{itemize}
\item {Utilização:Chím.}
\end{itemize}
Phenato de ethylo.
\section{Phênices}
\begin{itemize}
\item {Grp. gram.:m. pl.}
\end{itemize}
\begin{itemize}
\item {Proveniência:(Lat. \textunderscore phoeníces\textunderscore )}
\end{itemize}
O mesmo que [[phenícios|phenício]].
\section{Phenício}
\begin{itemize}
\item {Grp. gram.:adj.}
\end{itemize}
\begin{itemize}
\item {Grp. gram.:M.}
\end{itemize}
Relativo á Phenícia ou aos seus habitantes.
Habitante da Phenícia.
Língua semítica, falada pelos Phenícios.
\section{Phenicito}
\begin{itemize}
\item {Grp. gram.:m.}
\end{itemize}
\begin{itemize}
\item {Utilização:Miner.}
\end{itemize}
\begin{itemize}
\item {Proveniência:(Do gr. \textunderscore phoinix\textunderscore , vermelho)}
\end{itemize}
Variedade de chumbo chromatado.
\section{Phênico}
\begin{itemize}
\item {Grp. gram.:adj.}
\end{itemize}
\begin{itemize}
\item {Proveniência:(Do gr. \textunderscore phainos\textunderscore )}
\end{itemize}
Diz-se de um ácido, extrahido do alcatrão da hulha.
Relativo ao phenol.
\section{Phenicóptero}
\begin{itemize}
\item {Grp. gram.:m.}
\end{itemize}
\begin{itemize}
\item {Proveniência:(Gr. \textunderscore phoinikopteros\textunderscore )}
\end{itemize}
Ave pernalta.
\section{Phenigma}
\begin{itemize}
\item {Grp. gram.:m.}
\end{itemize}
\begin{itemize}
\item {Proveniência:(Do gr. \textunderscore phoinix\textunderscore )}
\end{itemize}
Rubefacção da pelle, produzida por sinapismos.
\section{Phênix}
\begin{itemize}
\item {fónica:nis}
\end{itemize}
\begin{itemize}
\item {Grp. gram.:f.}
\end{itemize}
\begin{itemize}
\item {Proveniência:(Lat. \textunderscore phoenix\textunderscore )}
\end{itemize}
Ave fabulosa que, segundo a Mythologia, vivia muitos séculos, e que, queimada, renascia da sua cinza.
Constellação austral.
Pessôa ou coisa única no seu gênero.
\section{Phenocarpo}
\begin{itemize}
\item {Grp. gram.:m.}
\end{itemize}
\begin{itemize}
\item {Utilização:Bot.}
\end{itemize}
Dizia-se o fruto que, não adherindo ás partes vizinhas, é por isso muito apparente.
\section{Phenocolla}
\begin{itemize}
\item {Grp. gram.:f.}
\end{itemize}
Producto pharmacêutico, com propriedades analgésicas e antithérmicas.
\section{Phenodina}
\begin{itemize}
\item {Grp. gram.:f.}
\end{itemize}
O mesmo que \textunderscore hematosina\textunderscore .
\section{Phenogamia}
\begin{itemize}
\item {Grp. gram.:f.}
\end{itemize}
\begin{itemize}
\item {Proveniência:(Do gr. \textunderscore phoinos\textunderscore  + \textunderscore gamos\textunderscore )}
\end{itemize}
Estado do que é phenogâmico.
\section{Phenogâmico}
\begin{itemize}
\item {Grp. gram.:adj.}
\end{itemize}
\begin{itemize}
\item {Proveniência:(De \textunderscore phenogamia\textunderscore )}
\end{itemize}
Diz-se do vegetal ou do animal, que tem apparentes os órgãos sexuaes.
\section{Phenol}
\begin{itemize}
\item {Grp. gram.:m.}
\end{itemize}
\begin{itemize}
\item {Grp. gram.:Pl.}
\end{itemize}
\begin{itemize}
\item {Proveniência:(Do gr. \textunderscore phainein\textunderscore )}
\end{itemize}
Substância, extrahida dos óleos, que fornecem o alcatrão do gás.
O mesmo que ácido phênico.
Corpos ternários, compostos de carbone, hydrogênio e oxygênio, e provenientes de um carbureto pela substituição de um átomo de hydrogênio por um oxhydrilo.
\section{Phenolite}
\begin{itemize}
\item {Grp. gram.:f.}
\end{itemize}
\begin{itemize}
\item {Utilização:Geol.}
\end{itemize}
Espécie de rocha eruptiva da época posterciária.
\section{Phenolito}
\begin{itemize}
\item {Grp. gram.:m.}
\end{itemize}
O mesmo ou melhor que \textunderscore phenolite\textunderscore .
\section{Phenolphtateína}
\begin{itemize}
\item {Grp. gram.:f.}
\end{itemize}
Composição pharmacêutica de phenol e nephthalina.
\section{Phenomenal}
\begin{itemize}
\item {Grp. gram.:adj.}
\end{itemize}
Que é da natureza do phenómeno.
Admirável; espantoso; singular.
\section{Phenomenalidade}
\begin{itemize}
\item {Grp. gram.:f.}
\end{itemize}
Qualidade do que é phenomenal.
\section{Phenomenização}
\begin{itemize}
\item {Grp. gram.:f.}
\end{itemize}
\begin{itemize}
\item {Utilização:Espir.}
\end{itemize}
Acto ou effeito de phenomenizar-se.
Producção de phenómenos.
\section{Phenomenizar-se}
\begin{itemize}
\item {Grp. gram.:v. p.}
\end{itemize}
\begin{itemize}
\item {Utilização:Neol.}
\end{itemize}
\begin{itemize}
\item {Proveniência:(De \textunderscore phenómeno\textunderscore )}
\end{itemize}
Realizar-se, manifestar-se:«\textunderscore ...talento que se phenomeniza cá em baixo por actos de pequeno alcance.\textunderscore »Tobias Barreto.
\section{Phenómeno}
\begin{itemize}
\item {Grp. gram.:m.}
\end{itemize}
\begin{itemize}
\item {Proveniência:(Gr. \textunderscore phainomenon\textunderscore )}
\end{itemize}
Tudo aquillo em que se exerce a acção dos sentidos ou que póde impressionar a nossa sensibilidade, phýsica ou moralmente.
Facto.
Tudo que se observa de extraordinário no ar ou no céu.
Maravilha.
O que é raro e surprehendente.
\section{Phenomenologia}
\begin{itemize}
\item {Grp. gram.:f.}
\end{itemize}
\begin{itemize}
\item {Proveniência:(Do gr. \textunderscore phainomenon\textunderscore  + \textunderscore logos\textunderscore )}
\end{itemize}
Tratado á cêrca dos phenómenos.
\section{Phenomenoso}
\begin{itemize}
\item {Grp. gram.:adj.}
\end{itemize}
\begin{itemize}
\item {Utilização:P. us.}
\end{itemize}
\begin{itemize}
\item {Proveniência:(De \textunderscore phenómeno\textunderscore )}
\end{itemize}
Extraordinário, admirável.
\section{Phenosalyl}
\begin{itemize}
\item {fónica:sa}
\end{itemize}
\begin{itemize}
\item {Grp. gram.:m.}
\end{itemize}
Producto pharmacêutico, com propriedades antisépticas.
\section{Phenylacetyleno}
\begin{itemize}
\item {Grp. gram.:m.}
\end{itemize}
\begin{itemize}
\item {Utilização:Chím.}
\end{itemize}
Um dos carbonetos do grupo pyrogenado.
\section{Phenylhydroquinazolina}
\begin{itemize}
\item {fónica:li}
\end{itemize}
\begin{itemize}
\item {Grp. gram.:f.}
\end{itemize}
Producto pharmacêutico, amargo e excitante do estômago.
\section{Phenylmethânio}
\begin{itemize}
\item {Grp. gram.:m.}
\end{itemize}
Producto pharmacêutico, analgésico e antithérmico.
\section{Phenylo}
\begin{itemize}
\item {Grp. gram.:m.}
\end{itemize}
\begin{itemize}
\item {Utilização:Chím.}
\end{itemize}
Radical hypothético do grupo phênico.
\section{Pheophýcias}
\begin{itemize}
\item {fónica:fe-o}
\end{itemize}
\begin{itemize}
\item {Grp. gram.:f. pl.}
\end{itemize}
\begin{itemize}
\item {Utilização:Geol.}
\end{itemize}
\begin{itemize}
\item {Proveniência:(Do gr. \textunderscore phaios\textunderscore , pardo, e \textunderscore phukos\textunderscore , alga)}
\end{itemize}
Ordem de algas fósseis.
\section{Phíala}
\begin{itemize}
\item {Grp. gram.:f.}
\end{itemize}
\begin{itemize}
\item {Proveniência:(Lat. \textunderscore phiala\textunderscore )}
\end{itemize}
Espécie de taça, usada pelos antigos, e que muitas vezes se offerecia como brinde.
\section{Philadelpho}
\begin{itemize}
\item {Grp. gram.:m.}
\end{itemize}
\begin{itemize}
\item {Proveniência:(Gr. \textunderscore philadelphos\textunderscore )}
\end{itemize}
Membro de uma seita religiosa da Inglaterra, no século XVII; membro de uma sociedade secreta da França, no tempo do primeiro império.
\section{Philadelphos}
\begin{itemize}
\item {Grp. gram.:m. pl.}
\end{itemize}
\begin{itemize}
\item {Proveniência:(Do gr. \textunderscore philos\textunderscore  + \textunderscore adelphos\textunderscore )}
\end{itemize}
Família de pólypos.
\section{Philagónia}
\begin{itemize}
\item {Grp. gram.:f.}
\end{itemize}
Árvore das florestas de Java.
\section{Philandra}
\begin{itemize}
\item {Grp. gram.:f.}
\end{itemize}
\begin{itemize}
\item {Proveniência:(Do gr. \textunderscore philos\textunderscore  + \textunderscore aner\textunderscore , \textunderscore andros\textunderscore )}
\end{itemize}
Nome de duas espécies de sarigueias e de um canguru da Índia.
\section{Philantho}
\begin{itemize}
\item {Grp. gram.:m.}
\end{itemize}
\begin{itemize}
\item {Proveniência:(Do gr. \textunderscore philos\textunderscore  + \textunderscore anthos\textunderscore )}
\end{itemize}
Pássaro de Bengala.
\section{Philanthropia}
\begin{itemize}
\item {Grp. gram.:f.}
\end{itemize}
\begin{itemize}
\item {Proveniência:(Lat. \textunderscore philantropia\textunderscore )}
\end{itemize}
Amor á humanidade; caridade.
\section{Philanthropicamente}
\begin{itemize}
\item {Grp. gram.:adv.}
\end{itemize}
De modo philantrópico.
\section{Philanthrópico}
\begin{itemize}
\item {Grp. gram.:adj.}
\end{itemize}
Relativo á philanthropia.
Inspirado pela philantropia.
\section{Philanthropismo}
\begin{itemize}
\item {Grp. gram.:m.}
\end{itemize}
Affectação de philantropia.
\section{Philanthropo}
\begin{itemize}
\item {Grp. gram.:m.  e  adj.}
\end{itemize}
\begin{itemize}
\item {Proveniência:(Lat. \textunderscore philanthropos\textunderscore )}
\end{itemize}
O que é dotado de philanthropia.
O que ama os seus semelhantes; humanitário.
\section{Philanthropomania}
\begin{itemize}
\item {Grp. gram.:f.}
\end{itemize}
\begin{itemize}
\item {Proveniência:(De \textunderscore philanthropia\textunderscore  + \textunderscore mania\textunderscore )}
\end{itemize}
Philanthropia ridícula ou pouco sincera.
\section{Philargýria}
\begin{itemize}
\item {Grp. gram.:f.}
\end{itemize}
\begin{itemize}
\item {Proveniência:(Lat. \textunderscore philargyria\textunderscore )}
\end{itemize}
O mesmo que \textunderscore avareza\textunderscore .
\section{Philarmónica}
\begin{itemize}
\item {Grp. gram.:f.}
\end{itemize}
\begin{itemize}
\item {Utilização:Gír.}
\end{itemize}
Sociedade musical.
Banda de música.
A policia, apitando.
(Fem. de \textunderscore philarmónico\textunderscore )
\section{Philarmónico}
\begin{itemize}
\item {Grp. gram.:adj.}
\end{itemize}
\begin{itemize}
\item {Proveniência:(De \textunderscore philo...\textunderscore  + \textunderscore harmónico\textunderscore )}
\end{itemize}
Que é amigo da harmonia ou da música.
Diz-se especialmente de certas sociedades musicaes.
\section{Philatelia}
\begin{itemize}
\item {Grp. gram.:f.}
\end{itemize}
\begin{itemize}
\item {Proveniência:(Do gr. \textunderscore philos\textunderscore  + \textunderscore ateleia\textunderscore )}
\end{itemize}
Estudo dos sellos do correio, usados nas diversas nações, e methodicamente colleccionados.--R. Galvão, \textunderscore Vocab.\textunderscore , entende que a forma exacta é \textunderscore philotelia\textunderscore , por a suppor derivada do gr. \textunderscore philos\textunderscore  + \textunderscore telos\textunderscore .
\section{Philatélico}
\begin{itemize}
\item {Grp. gram.:adj.}
\end{itemize}
Relativo á philatelia.
\section{Philatelismo}
\begin{itemize}
\item {Grp. gram.:m.}
\end{itemize}
Gôsto pela philatelia; prática dêsse gôsto
\section{Philatelista}
\begin{itemize}
\item {Grp. gram.:m.}
\end{itemize}
\begin{itemize}
\item {Proveniência:(De \textunderscore philatelia\textunderscore )}
\end{itemize}
Colleccionador de sellos do correio.
\section{Philáucia}
\begin{itemize}
\item {Grp. gram.:f.}
\end{itemize}
\begin{itemize}
\item {Proveniência:(Gr. \textunderscore philautia\textunderscore )}
\end{itemize}
Amor próprio; egoismo; presumpção, vaidade.
\section{Philaucioso}
\begin{itemize}
\item {Grp. gram.:adj.}
\end{itemize}
Que tem philáucia.
\section{Philédono}
\begin{itemize}
\item {Grp. gram.:m.}
\end{itemize}
Gênero de pássaros, estabelecido por Cuvier na fam. dos dentirostros.
\section{Philíppica}
\begin{itemize}
\item {Grp. gram.:f.}
\end{itemize}
\begin{itemize}
\item {Utilização:Ext.}
\end{itemize}
\begin{itemize}
\item {Proveniência:(De \textunderscore Philippe\textunderscore , n. p.)}
\end{itemize}
Oração de Demósthenes contra Phillippe de Macedónia.
Cada uma das orações de Cícero contra Marco-António.
Discurso violento e injurioso; sátira acerba.
\section{Philippista}
\begin{itemize}
\item {Grp. gram.:m.}
\end{itemize}
Sectário de Luís Philippe, em França.
\section{Philipsita}
\begin{itemize}
\item {Grp. gram.:f.}
\end{itemize}
\begin{itemize}
\item {Proveniência:(De \textunderscore Philips\textunderscore , n. p.)}
\end{itemize}
Sulfureto de cobre e de ferro.
\section{Philipsito}
\begin{itemize}
\item {Grp. gram.:m.}
\end{itemize}
O mesmo ou melhor que \textunderscore philipsita\textunderscore .
\section{Philisteu}
\begin{itemize}
\item {Grp. gram.:m.}
\end{itemize}
\begin{itemize}
\item {Utilização:Pop.}
\end{itemize}
\begin{itemize}
\item {Proveniência:(De \textunderscore philisteus\textunderscore )}
\end{itemize}
Homem corpulento e desajeitado; brutamontes.
\section{Philisteus}
\begin{itemize}
\item {Grp. gram.:m. pl.}
\end{itemize}
\begin{itemize}
\item {Proveniência:(Lat. \textunderscore philistaei\textunderscore )}
\end{itemize}
Um dos povos que habitavam a Palestina, antes da conquista dêste país pelos Hebreus.
\section{Philistinos}
\begin{itemize}
\item {Grp. gram.:m. pl.}
\end{itemize}
O mesmo que \textunderscore Philisteus\textunderscore .
Nome que, entre os estudantes alemães, se dá aos indivíduos estranhos ás universidades, mormente os negociantes.
\section{Phillyreia}
\begin{itemize}
\item {Grp. gram.:f.}
\end{itemize}
\begin{itemize}
\item {Proveniência:(Do gr. \textunderscore phillurea\textunderscore )}
\end{itemize}
Árvore oleagínea do sul da Europa.
\section{Phillyrena}
\begin{itemize}
\item {Grp. gram.:f.}
\end{itemize}
\begin{itemize}
\item {Utilização:Chím.}
\end{itemize}
Princípio, que se extrai da casca da phillyreia.
\section{Philo...}
\begin{itemize}
\item {Grp. gram.:pref.}
\end{itemize}
\begin{itemize}
\item {Proveniência:(Do gr. \textunderscore philos\textunderscore )}
\end{itemize}
(designativo de \textunderscore amizade\textunderscore , \textunderscore amor\textunderscore , \textunderscore tendência\textunderscore )
\section{Philocýnico}
\begin{itemize}
\item {Grp. gram.:adj.}
\end{itemize}
\begin{itemize}
\item {Utilização:Neol.}
\end{itemize}
\begin{itemize}
\item {Proveniência:(Do gr. \textunderscore philos\textunderscore  + \textunderscore kuon\textunderscore )}
\end{itemize}
Que gosta de cães; amigo dos cães.
\section{Philodendro}
\begin{itemize}
\item {Grp. gram.:m.}
\end{itemize}
\begin{itemize}
\item {Proveniência:(Do gr. \textunderscore philos\textunderscore  + \textunderscore dendron\textunderscore )}
\end{itemize}
Planta ornamental, de grandes e formosas fôlhas, originária da América do Sul.
\section{Philodendro-imbê}
\begin{itemize}
\item {Grp. gram.:m.}
\end{itemize}
O mesmo que \textunderscore cipó-do-imbê\textunderscore .
\section{Philodérmico}
\begin{itemize}
\item {Grp. gram.:adj.}
\end{itemize}
\begin{itemize}
\item {Proveniência:(Do gr. \textunderscore philos\textunderscore  + \textunderscore derma\textunderscore )}
\end{itemize}
Diz-se dos preparados, que conservam a maciez e frescura da pelle.
\section{Philodynasta}
\begin{itemize}
\item {Grp. gram.:adj.}
\end{itemize}
Affeiçoado a uma dynastia. Cf. Camillo, \textunderscore Noites de Insómn.\textunderscore , III, 80.
\section{Philogenitura}
\begin{itemize}
\item {Grp. gram.:f.}
\end{itemize}
\begin{itemize}
\item {Proveniência:(T. hybr., do gr. \textunderscore philos\textunderscore  + lat. \textunderscore genitura\textunderscore )}
\end{itemize}
Amor, que conduz á procriação de filhos.
\section{Philo-grego}
\begin{itemize}
\item {Grp. gram.:m.}
\end{itemize}
O mesmo que \textunderscore philo-helleno\textunderscore .
\section{Philogynia}
\begin{itemize}
\item {Grp. gram.:f.}
\end{itemize}
\begin{itemize}
\item {Utilização:Bras}
\end{itemize}
Amor ás mulheres.
Theoria de igualdade intellectual do homem e da mulher.
(Cp. \textunderscore philógyno\textunderscore )
\section{Philogýnio}
\begin{itemize}
\item {Grp. gram.:adj.}
\end{itemize}
O mesmo que \textunderscore philógyno\textunderscore . Cf. Camillo, \textunderscore Brasileira\textunderscore , 326.
\section{Philógyno}
\begin{itemize}
\item {Grp. gram.:adj.}
\end{itemize}
\begin{itemize}
\item {Proveniência:(Do gr. \textunderscore philos\textunderscore  + \textunderscore gune\textunderscore )}
\end{itemize}
Que tem inclinação para as mulheres; apaixonado por mulheres; femeeiro.
\section{Philo-hellenismo}
\begin{itemize}
\item {Grp. gram.:m.}
\end{itemize}
Affeição á nação grega e aos seus interesses, especialmente por opposição á dominação turca.
\section{Philo-helleno}
\begin{itemize}
\item {Grp. gram.:m.}
\end{itemize}
Amigo da Grécia, das suas artes, da sua civilização.
Indivíduo, favorável á independência grega.
Voluntário, em serviço da moderna Grécia.
\section{Philologia}
\begin{itemize}
\item {Grp. gram.:f.}
\end{itemize}
\begin{itemize}
\item {Proveniência:(De \textunderscore philólogo\textunderscore )}
\end{itemize}
Estudo e conhecimento de uma língua, como instrumento e meio de uma litteratura.
Conhecimento geral das bellas-letras, línguas, crítica, etc.
\section{Philológico}
\begin{itemize}
\item {Grp. gram.:adj.}
\end{itemize}
Relativo á Philologia.
\section{Philologista}
\begin{itemize}
\item {Grp. gram.:m. ,  f.  e  adj.}
\end{itemize}
Pessôa, que se dedica á Philologia.
\section{Philólogo}
\begin{itemize}
\item {Grp. gram.:m.}
\end{itemize}
\begin{itemize}
\item {Proveniência:(Gr. \textunderscore philologos\textunderscore )}
\end{itemize}
Aquelle que é versado ou perito em Philologia.
\section{Philomáthico}
\begin{itemize}
\item {Grp. gram.:adj.}
\end{itemize}
\begin{itemize}
\item {Proveniência:(Do gr. \textunderscore philos\textunderscore  + \textunderscore mathein\textunderscore )}
\end{itemize}
Que ama as sciências.
\section{Philomela}
\begin{itemize}
\item {Grp. gram.:f.}
\end{itemize}
\begin{itemize}
\item {Utilização:Poét.}
\end{itemize}
\begin{itemize}
\item {Proveniência:(Do gr. \textunderscore Philomela\textunderscore , n. p.)}
\end{itemize}
O mesmo que \textunderscore rouxinol\textunderscore .
\section{Philonegro}
\begin{itemize}
\item {fónica:nê}
\end{itemize}
\begin{itemize}
\item {Grp. gram.:m.  e  adj.}
\end{itemize}
\begin{itemize}
\item {Proveniência:(De \textunderscore philo...\textunderscore  + \textunderscore negro\textunderscore )}
\end{itemize}
Indivíduo, que gosta dos Negros; protector ou defensor dos Negros. Cf. Garrett, \textunderscore Helena\textunderscore , 94.
\section{Philónio}
\begin{itemize}
\item {Grp. gram.:m.}
\end{itemize}
Electuário, de composição muito complexa.
\section{Philosophal}
\begin{itemize}
\item {Grp. gram.:adj.}
\end{itemize}
\begin{itemize}
\item {Utilização:Fig.}
\end{itemize}
O mesmo que \textunderscore philosóphico\textunderscore .
\textunderscore Pedra philosophal\textunderscore , segrêdo imaginário, que os alchimistas procuravam penetrar, para converter metaes ordinários em metaes preciosos.
Coisa diffícil de se descobrir ou de se realizar.
\section{Philosophante}
\begin{itemize}
\item {Grp. gram.:m.  e  adj.}
\end{itemize}
\begin{itemize}
\item {Utilização:Deprec.}
\end{itemize}
\begin{itemize}
\item {Proveniência:(Lat. \textunderscore philosophans\textunderscore )}
\end{itemize}
Philósopho.
O que discorre disparatadamente, com pretensões a erudito.
\section{Philosophar}
\begin{itemize}
\item {Grp. gram.:v. i.}
\end{itemize}
\begin{itemize}
\item {Proveniência:(Lat. \textunderscore philosophari\textunderscore )}
\end{itemize}
Raciocinar sôbre assumptos philosóphicos.
Discutir ou discorrer sôbre qualquer matéria scientífica; raciocinar.
\section{Philosophastro}
\begin{itemize}
\item {Grp. gram.:m.}
\end{itemize}
\begin{itemize}
\item {Proveniência:(Lat. \textunderscore philosophaster\textunderscore )}
\end{itemize}
Indivíduo, que se suppõe philósopho, e que discorre sem acêrto.
\section{Philosophear}
\begin{itemize}
\item {Grp. gram.:v. t.}
\end{itemize}
O mesmo que \textunderscore philosophar\textunderscore . Cf. Camillo, \textunderscore Noites de Insómn.\textunderscore , VI, 74.
\section{Philosophia}
\begin{itemize}
\item {Grp. gram.:f.}
\end{itemize}
\begin{itemize}
\item {Proveniência:(Lat. \textunderscore philosophia\textunderscore )}
\end{itemize}
Sciência geral dos princípios e causas, ou systema de noções geraes sobre o conjunto das coisas.
Cada um dos systemas particulares de Philosophia.
Firmeza ou elevação de espírito, com que o homem se colloca acima dos successos e preconceitos: \textunderscore proceder com philosophia\textunderscore .
Sabedoria.
\section{Philosophicamente}
\begin{itemize}
\item {Grp. gram.:adv.}
\end{itemize}
De modo philosóphico; segundo a Philosophia; á maneira de philósopho.
\section{Philosophice}
\begin{itemize}
\item {Grp. gram.:f.}
\end{itemize}
\begin{itemize}
\item {Utilização:Deprec.}
\end{itemize}
Qualidade de quem philosopha ridiculamente.
\section{Philosóphico}
\begin{itemize}
\item {Grp. gram.:adj.}
\end{itemize}
\begin{itemize}
\item {Proveniência:(Lat. \textunderscore philosophicus\textunderscore )}
\end{itemize}
Relativo á Philosophia ou aos philósophos; peculiar aos philósophos.
\section{Philosophismo}
\begin{itemize}
\item {Grp. gram.:m.}
\end{itemize}
\begin{itemize}
\item {Proveniência:(De \textunderscore philosophia\textunderscore )}
\end{itemize}
Mania philosóphica; falsa Philosophia.
\section{Philósopho}
\begin{itemize}
\item {Grp. gram.:m.  e  adj.}
\end{itemize}
\begin{itemize}
\item {Proveniência:(Lat. \textunderscore philosophus\textunderscore )}
\end{itemize}
Amigo da sabedoria ou o que se applica ao estudo dos princípios e causas.
Sábio.
Livre pensador.
O que tem um viver sereno e tranquillo, indifferente ás coisas e preconceitos ou convenções do mundo.
\section{Philotéchnico}
\begin{itemize}
\item {Grp. gram.:adj.}
\end{itemize}
Que ama as artes.
(Cp. lat. \textunderscore philotechnus\textunderscore )
\section{Philotimia}
\begin{itemize}
\item {Grp. gram.:f.}
\end{itemize}
\begin{itemize}
\item {Proveniência:(Gr. \textunderscore philotimia\textunderscore )}
\end{itemize}
Amor da honra ou das honras.
\section{Phílyca}
\begin{itemize}
\item {Grp. gram.:f.}
\end{itemize}
\begin{itemize}
\item {Proveniência:(Do gr. \textunderscore philuke\textunderscore )}
\end{itemize}
Gênero de plantas rhamnáceas do Cabo da Bôa-Esperança.
\section{Phimose}
\begin{itemize}
\item {Grp. gram.:f.}
\end{itemize}
\begin{itemize}
\item {Utilização:Med.}
\end{itemize}
\begin{itemize}
\item {Proveniência:(Gr. \textunderscore phimosis\textunderscore )}
\end{itemize}
Apêrto natural ou accidental, que não deixa que o prepúcio se retire para trás.
\section{Phlebectasía}
\begin{itemize}
\item {Grp. gram.:f.}
\end{itemize}
\begin{itemize}
\item {Utilização:Med.}
\end{itemize}
\begin{itemize}
\item {Proveniência:(Do gr. \textunderscore phleps\textunderscore  + \textunderscore ektasis\textunderscore )}
\end{itemize}
Dilatação de uma veia.
\section{Phlebenterismo}
\begin{itemize}
\item {Grp. gram.:m.}
\end{itemize}
\begin{itemize}
\item {Proveniência:(Do gr. \textunderscore phleps\textunderscore  + \textunderscore enteron\textunderscore )}
\end{itemize}
Theoria anatómica, segundo a qual se suppõe que em certos seres o systema circulatório desapparece e é substituido pelo tubo digestivo.
\section{Phlébico}
\begin{itemize}
\item {Grp. gram.:adj.}
\end{itemize}
\begin{itemize}
\item {Utilização:Med.}
\end{itemize}
\begin{itemize}
\item {Proveniência:(Do gr. \textunderscore phleps\textunderscore )}
\end{itemize}
Relativo ás veias.
\section{Phlebite}
\begin{itemize}
\item {Grp. gram.:f.}
\end{itemize}
\begin{itemize}
\item {Utilização:Med.}
\end{itemize}
\begin{itemize}
\item {Proveniência:(Do gr. \textunderscore phleps\textunderscore )}
\end{itemize}
Inflammação da membrana interna das veias.
\section{Phlebographia}
\begin{itemize}
\item {Grp. gram.:f.}
\end{itemize}
\begin{itemize}
\item {Proveniência:(De \textunderscore phlebógrapho\textunderscore )}
\end{itemize}
Descripção das veias.
\section{Phlebográphico}
\begin{itemize}
\item {Grp. gram.:adj.}
\end{itemize}
Relativo á phlebographia.
\section{Phlebógrapho}
\begin{itemize}
\item {Grp. gram.:m.}
\end{itemize}
\begin{itemize}
\item {Proveniência:(Do gr. \textunderscore phleps\textunderscore , \textunderscore phlebos\textunderscore  + \textunderscore graphein\textunderscore )}
\end{itemize}
Anatomista, que descreve as veias.
\section{Phlebólitho}
\begin{itemize}
\item {Grp. gram.:f.}
\end{itemize}
\begin{itemize}
\item {Proveniência:(Do gr. \textunderscore phleps\textunderscore , \textunderscore phlebos\textunderscore  + \textunderscore lithos\textunderscore )}
\end{itemize}
Concreção calcária, que se fórma numa veia varicosa.
\section{Phlebomalacia}
\begin{itemize}
\item {Grp. gram.:f.}
\end{itemize}
\begin{itemize}
\item {Utilização:Med.}
\end{itemize}
\begin{itemize}
\item {Proveniência:(Do gr. \textunderscore phleps\textunderscore  + \textunderscore malakia\textunderscore )}
\end{itemize}
Amollecimento mòrbido das veias.
\section{Phlebopalia}
\begin{itemize}
\item {Grp. gram.:f.}
\end{itemize}
\begin{itemize}
\item {Utilização:Med.}
\end{itemize}
\begin{itemize}
\item {Proveniência:(Gr. \textunderscore phlebopalia\textunderscore )}
\end{itemize}
Pulsação das veias.
\section{Phlebóptero}
\begin{itemize}
\item {Grp. gram.:adj.}
\end{itemize}
\begin{itemize}
\item {Utilização:Zool.}
\end{itemize}
\begin{itemize}
\item {Proveniência:(Do gr. \textunderscore phleps\textunderscore  + \textunderscore pteron\textunderscore )}
\end{itemize}
Diz-se dos insectos, que têm asas venosas.
\section{Phleborrhagia}
\begin{itemize}
\item {Grp. gram.:f.}
\end{itemize}
\begin{itemize}
\item {Proveniência:(Do gr. \textunderscore phleps\textunderscore , \textunderscore phlebos\textunderscore  + \textunderscore rhagein\textunderscore )}
\end{itemize}
Ruptura de uma veia; hemorrhagia das veias.
\section{Phlebotomia}
\begin{itemize}
\item {Grp. gram.:f.}
\end{itemize}
\begin{itemize}
\item {Proveniência:(Lat. \textunderscore phlebotomia\textunderscore )}
\end{itemize}
Sangria; arte de sangrar.
\section{Phlebotómico}
\begin{itemize}
\item {Grp. gram.:adj.}
\end{itemize}
Relativo á phlebotomia.
\section{Phlebótomo}
\begin{itemize}
\item {Grp. gram.:m.}
\end{itemize}
\begin{itemize}
\item {Proveniência:(Lat. \textunderscore phlebotomus\textunderscore )}
\end{itemize}
Instrumento, usado principalmente na Alemanha, para fazer sangrias.
\section{Phlegethonte}
\begin{itemize}
\item {Grp. gram.:m.}
\end{itemize}
\begin{itemize}
\item {Proveniência:(Lat. \textunderscore phlegethon\textunderscore , \textunderscore phlegethontis\textunderscore )}
\end{itemize}
Rio infernal; rio escuro:«\textunderscore desagôam tenebrosos phlegethontes.\textunderscore »\textunderscore Viriato Trág.\textunderscore , X, 34.
\section{Phlegmão}
\begin{itemize}
\item {Grp. gram.:m.}
\end{itemize}
(V.fleimão)
\section{Phlegmasia}
\begin{itemize}
\item {Grp. gram.:f.}
\end{itemize}
\begin{itemize}
\item {Utilização:Med.}
\end{itemize}
\begin{itemize}
\item {Proveniência:(Gr. \textunderscore phlegmasia\textunderscore )}
\end{itemize}
O mesmo que \textunderscore inflammação\textunderscore .
\section{Phlegmásico}
\begin{itemize}
\item {Grp. gram.:adj.}
\end{itemize}
Relativo á phlegmasia.
\section{Phlegmatorrhagia}
\begin{itemize}
\item {Grp. gram.:f.}
\end{itemize}
\begin{itemize}
\item {Utilização:Med.}
\end{itemize}
\begin{itemize}
\item {Proveniência:(Do gr. \textunderscore phlegma\textunderscore  + \textunderscore rhagein\textunderscore )}
\end{itemize}
Excreção abundante, pelas narinas, de uma mucosidade límpida, não acompanhada de inflammação.
\section{Phlegmonosa}
\begin{itemize}
\item {Grp. gram.:adj. f.}
\end{itemize}
\begin{itemize}
\item {Utilização:Med.}
\end{itemize}
\begin{itemize}
\item {Proveniência:(Do lat. \textunderscore phlegmone\textunderscore )}
\end{itemize}
Diz-se de uma variedade de angina, determinada por tumor ou fleimão.
Diz-se da gastrite do cão, e da pharyngite do cão e do cavallo. Cf. A. Torgo, \textunderscore Carteira de um Veterinário\textunderscore .
\section{Phléolo}
\begin{itemize}
\item {Grp. gram.:m.}
\end{itemize}
\begin{itemize}
\item {Proveniência:(Do gr. \textunderscore phleos\textunderscore )}
\end{itemize}
Gênero de plantas gramíneas, que se dá bem nos terrenos áridos e collinas saibrosas.
\section{Phlogístico}
\begin{itemize}
\item {Grp. gram.:adj.}
\end{itemize}
\begin{itemize}
\item {Grp. gram.:M.}
\end{itemize}
\begin{itemize}
\item {Proveniência:(Gr. \textunderscore phlogistikos\textunderscore )}
\end{itemize}
Que desenvolve calor interno.
Produzido por inflammação.
Fluido particular, que se suppunha inherente aos corpos, para explicar a combustão.
\section{Phlogisto}
\begin{itemize}
\item {Grp. gram.:m.}
\end{itemize}
\begin{itemize}
\item {Proveniência:(Gr. \textunderscore phlogistos\textunderscore )}
\end{itemize}
Fluido, de grande movimento vibratório, manifestado pela sensação do calor e luz, e que, junto a uma substância, explicava, para os antigos chímicos, o phenómeno da combustão.
\section{Phlogistologia}
\begin{itemize}
\item {Grp. gram.:f.}
\end{itemize}
\begin{itemize}
\item {Proveniência:(Do gr. \textunderscore phlogistos\textunderscore  + \textunderscore logos\textunderscore )}
\end{itemize}
Tratado á cêrca das substâncias combustíveis.
\section{Phlogistológico}
\begin{itemize}
\item {Grp. gram.:adj.}
\end{itemize}
Relativo á phlogistologia.
\section{Phlogogênico}
\begin{itemize}
\item {Grp. gram.:adj.}
\end{itemize}
\begin{itemize}
\item {Proveniência:(Do gr. \textunderscore phlogos\textunderscore  + \textunderscore genes\textunderscore )}
\end{itemize}
Que produz inflammação.
\section{Phlogógeno}
\begin{itemize}
\item {Grp. gram.:adj.}
\end{itemize}
O mesmo que \textunderscore phlogogênico\textunderscore .
\section{Phlogose}
\begin{itemize}
\item {Grp. gram.:f.}
\end{itemize}
\begin{itemize}
\item {Utilização:Med.}
\end{itemize}
\begin{itemize}
\item {Proveniência:(Gr. \textunderscore phlogosis\textunderscore )}
\end{itemize}
O mesmo que \textunderscore phlegmasia\textunderscore .
Inflammação ligeira ou superfícial.
\section{Phlómide}
\begin{itemize}
\item {Grp. gram.:f.}
\end{itemize}
\begin{itemize}
\item {Proveniência:(Lat. \textunderscore phlomis\textunderscore )}
\end{itemize}
Gênero de plantas labiadas.
\section{Phlooplastía}
\begin{itemize}
\item {Grp. gram.:f.}
\end{itemize}
\begin{itemize}
\item {Utilização:Bot.}
\end{itemize}
\begin{itemize}
\item {Proveniência:(Do gr. \textunderscore phloos\textunderscore  + \textunderscore plassein\textunderscore )}
\end{itemize}
Reparação ou renovação da casca das árvores.
\section{Phloorrhizina}
\begin{itemize}
\item {Grp. gram.:f.}
\end{itemize}
\begin{itemize}
\item {Utilização:Chím.}
\end{itemize}
\begin{itemize}
\item {Proveniência:(Do gr. \textunderscore phloos\textunderscore  + \textunderscore rhiza\textunderscore )}
\end{itemize}
Substância crystallizável, que se extrai das raízes de algumas árvores pomíferas.
\section{Phloretato}
\begin{itemize}
\item {Grp. gram.:m.}
\end{itemize}
\begin{itemize}
\item {Proveniência:(De \textunderscore phlorético\textunderscore )}
\end{itemize}
Combinação do ácido phlorético com uma base.
\section{Phlorético}
\begin{itemize}
\item {Grp. gram.:adj.}
\end{itemize}
\begin{itemize}
\item {Proveniência:(Do gr. \textunderscore phloios\textunderscore )}
\end{itemize}
Diz-se de um ácido, resultante da acção da potassa cáustica sôbre a phloretina.
\section{Phloretina}
\begin{itemize}
\item {Grp. gram.:f.}
\end{itemize}
\begin{itemize}
\item {Proveniência:(Do gr. \textunderscore phloios\textunderscore  + \textunderscore retine\textunderscore )}
\end{itemize}
Matéria orgânica neutra, formada sob a influência dos ácidos mineraes.
\section{Phloroglusina}
\begin{itemize}
\item {Grp. gram.:f.}
\end{itemize}
\begin{itemize}
\item {Utilização:Chím.}
\end{itemize}
Phenol triatómico isómero do ácido pyrogálico.
\section{Phlox}
\begin{itemize}
\item {Grp. gram.:m.}
\end{itemize}
\begin{itemize}
\item {Proveniência:(Lat. \textunderscore flox\textunderscore , \textunderscore flogis\textunderscore )}
\end{itemize}
Gênero de plantas polemoniáceas, cultivadas em jardins e notáveis pelo aroma e belleza das suas flôres.--A fórma exacta seria \textunderscore phloge\textunderscore  ou \textunderscore floge\textunderscore .
\section{Phlyctena}
\begin{itemize}
\item {Grp. gram.:f.}
\end{itemize}
\begin{itemize}
\item {Utilização:Med.}
\end{itemize}
\begin{itemize}
\item {Proveniência:(Gr. \textunderscore phluktaina\textunderscore )}
\end{itemize}
Pequena empôla vesicular e transparente.
Pústula, de natureza lymphática, na conjuntiva ou na córnea.
\section{Phlyctenóide}
\begin{itemize}
\item {Grp. gram.:adj.}
\end{itemize}
\begin{itemize}
\item {Utilização:Med.}
\end{itemize}
\begin{itemize}
\item {Proveniência:(Do gr. \textunderscore phluktaina\textunderscore  + \textunderscore eidos\textunderscore )}
\end{itemize}
Semelhante á phlyctena.
\section{Phlyctenular}
\begin{itemize}
\item {Grp. gram.:adj.}
\end{itemize}
\begin{itemize}
\item {Utilização:Med.}
\end{itemize}
\begin{itemize}
\item {Proveniência:(Do hypoth. \textunderscore phlytênula\textunderscore , dem. de \textunderscore phlyctena\textunderscore )}
\end{itemize}
Que apresenta pequenas phlyctenas.
\section{Phobia}
\begin{itemize}
\item {Grp. gram.:f.}
\end{itemize}
\begin{itemize}
\item {Proveniência:(Do gr. \textunderscore phobein\textunderscore , temer)}
\end{itemize}
Designação genérica das differentes espécies de medo mórbido, como agoraphobía, a thalassophobía, etc.
\section{Phobophobía}
\begin{itemize}
\item {Grp. gram.:f.}
\end{itemize}
\begin{itemize}
\item {Proveniência:(Do gr. \textunderscore phobein\textunderscore  + \textunderscore phobein\textunderscore )}
\end{itemize}
O mesmo que \textunderscore nosophobia\textunderscore .
\section{Phobóphobo}
\begin{itemize}
\item {Grp. gram.:m.}
\end{itemize}
O mesmo que \textunderscore nosóphobo\textunderscore .
\section{Phoca}
\begin{itemize}
\item {Grp. gram.:f.}
\end{itemize}
\begin{itemize}
\item {Grp. gram.:M.  e  f.}
\end{itemize}
\begin{itemize}
\item {Utilização:Pop.}
\end{itemize}
\begin{itemize}
\item {Proveniência:(Lat. \textunderscore phoca\textunderscore )}
\end{itemize}
Gênero de mammíferos amphíbios.--Em Filinto e Camões, é do gênero masculino. Cf. \textunderscore D. Man.\textunderscore , I, 67:«\textunderscore ...matárão hum grande foca\textunderscore ».
Pessôa avarenta, sovina.
\section{Phocáceos}
\begin{itemize}
\item {Grp. gram.:m. pl.}
\end{itemize}
Família de mammíferos, que tem por typo a phoca.
\section{Phoceia}
\begin{itemize}
\item {Grp. gram.:f.}
\end{itemize}
\begin{itemize}
\item {Proveniência:(Do lat. \textunderscore Phocaea\textunderscore , n. p.)}
\end{itemize}
Pequeno planeta, descoberto em 1853.
\section{Phocena}
\begin{itemize}
\item {Grp. gram.:f.}
\end{itemize}
\begin{itemize}
\item {Proveniência:(Gr. \textunderscore phokaina\textunderscore )}
\end{itemize}
Gênero de cetáceos, a que pertence o porco marinho.
\section{Phocenato}
\begin{itemize}
\item {Grp. gram.:m.}
\end{itemize}
Sal, resultante da combinação do ácido phocênico com uma base.
\section{Phocênico}
\begin{itemize}
\item {Grp. gram.:adj.}
\end{itemize}
\begin{itemize}
\item {Proveniência:(De \textunderscore phocenina\textunderscore )}
\end{itemize}
Diz-se de um ácido, que se descobriu nos óleos dos mammíferos marinhos.
\section{Phocenina}
\begin{itemize}
\item {Grp. gram.:f.}
\end{itemize}
\begin{itemize}
\item {Proveniência:(Do gr. \textunderscore phokaína\textunderscore )}
\end{itemize}
Princípio gordo dos óleos dos mammíferos marinhos.
\section{Phócio}
\begin{itemize}
\item {Grp. gram.:adj.}
\end{itemize}
\begin{itemize}
\item {Proveniência:(Lat. \textunderscore phocius\textunderscore )}
\end{itemize}
Relativo á Phócida, região da Grécia, onde está o monte Parnaso e a fonte de Castália. Cf. Latino, \textunderscore Or. da Corôa\textunderscore , p. XC.
\section{Phocomelia}
\begin{itemize}
\item {Grp. gram.:f.}
\end{itemize}
Estado ou qualidade de quem é phocómelo.
\section{Phocómelo}
\begin{itemize}
\item {Grp. gram.:m.}
\end{itemize}
\begin{itemize}
\item {Proveniência:(Do gr. \textunderscore phoke\textunderscore  + \textunderscore melos\textunderscore )}
\end{itemize}
Monstro que, sem braços nem pernas, parece têr as mãos e os pés insertos immediatamente no tronco, como succede com as phocas.
\section{Phólada}
\begin{itemize}
\item {Grp. gram.:f.}
\end{itemize}
\begin{itemize}
\item {Proveniência:(Gr. \textunderscore pholas\textunderscore )}
\end{itemize}
Mollusco acéphalo.
\section{Pholadite}
\begin{itemize}
\item {Grp. gram.:f.}
\end{itemize}
\begin{itemize}
\item {Utilização:Miner.}
\end{itemize}
Phólada fóssil.
\section{Pholadito}
\begin{itemize}
\item {Grp. gram.:m.}
\end{itemize}
O mesmo ou melhor que \textunderscore pholadite\textunderscore .
\section{Pholeosântheas}
\begin{itemize}
\item {fónica:le-o}
\end{itemize}
\begin{itemize}
\item {Grp. gram.:f. pl.}
\end{itemize}
Secção de plantas urticáceas, no systema de Blume.
\section{Pholerite}
\begin{itemize}
\item {Grp. gram.:f.}
\end{itemize}
\begin{itemize}
\item {Utilização:Geol.}
\end{itemize}
O mesmo que \textunderscore pholidite\textunderscore .
\section{Pholidite}
\begin{itemize}
\item {Grp. gram.:f.}
\end{itemize}
\begin{itemize}
\item {Utilização:Geol.}
\end{itemize}
\begin{itemize}
\item {Proveniência:(Do gr. \textunderscore pholis\textunderscore , escama)}
\end{itemize}
Uma das mais importantes variedades de caulim, a qual apparece em lâminas ou escamas.
\section{Pholidito}
\begin{itemize}
\item {Grp. gram.:m.}
\end{itemize}
O mesmo ou melhor que \textunderscore pholidite\textunderscore .
\section{Pholídoto}
\begin{itemize}
\item {Grp. gram.:adj.}
\end{itemize}
\begin{itemize}
\item {Utilização:Hist. Nat.}
\end{itemize}
\begin{itemize}
\item {Proveniência:(Gr. \textunderscore pholidotos\textunderscore )}
\end{itemize}
Coberto de escamas.
\section{Phoma}
\begin{itemize}
\item {Grp. gram.:m.}
\end{itemize}
Gênero de cogumelos.
\section{Phonação}
\begin{itemize}
\item {Grp. gram.:f.}
\end{itemize}
\begin{itemize}
\item {Proveniência:(Do gr. \textunderscore phone\textunderscore )}
\end{itemize}
Producção physiológica da voz.
\section{Phonador}
\begin{itemize}
\item {Grp. gram.:adj.}
\end{itemize}
\begin{itemize}
\item {Proveniência:(Do gr. \textunderscore phone\textunderscore )}
\end{itemize}
Que produz voz.
Diz-se especialmente do apparelho, formado pelos órgãos da voz.
\section{Phonalidade}
\begin{itemize}
\item {Grp. gram.:f.}
\end{itemize}
\begin{itemize}
\item {Proveniência:(Do gr. \textunderscore phone\textunderscore )}
\end{itemize}
Carácter dos sons de uma língua.
\section{Phonascia}
\begin{itemize}
\item {Grp. gram.:f.}
\end{itemize}
\begin{itemize}
\item {Proveniência:(Gr. \textunderscore phonaskia\textunderscore )}
\end{itemize}
Arte de exercitar a voz.
\section{Phonasco}
\begin{itemize}
\item {Grp. gram.:m.}
\end{itemize}
\begin{itemize}
\item {Proveniência:(Lat. \textunderscore phonascus\textunderscore )}
\end{itemize}
Professor de declamação, entre os antigos.
\section{Phonautógrapho}
\begin{itemize}
\item {Grp. gram.:m.}
\end{itemize}
\begin{itemize}
\item {Proveniência:(Do gr. \textunderscore phone\textunderscore  + \textunderscore autos\textunderscore  + \textunderscore graphein\textunderscore )}
\end{itemize}
Apparelho de acústica, para reproduzir graphicamente os sons articulados ou as vibrações sonoras.
\section{Phonema}
\begin{itemize}
\item {Grp. gram.:m.}
\end{itemize}
\begin{itemize}
\item {Proveniência:(Gr. \textunderscore phonema\textunderscore )}
\end{itemize}
Qualquer som articulado.
\section{Phonendoscopia}
\begin{itemize}
\item {Grp. gram.:f.}
\end{itemize}
Applicação do phonendoscópio.
\section{Phonendoscópico}
\begin{itemize}
\item {Grp. gram.:adj.}
\end{itemize}
Relativo á phonendoscopía.
\section{Phonendoscópio}
\begin{itemize}
\item {Grp. gram.:m.}
\end{itemize}
\begin{itemize}
\item {Proveniência:(Do gr. \textunderscore phone\textunderscore  + \textunderscore endos\textunderscore  + \textunderscore skopein\textunderscore )}
\end{itemize}
Apparelho, inventado recentemente, (1898), pelo professor Bianchi, e que, posto em communicação com os ouvidos de um médico, póde determinar a situação, fórma e volume das vísceras.
\section{Phonética}
\begin{itemize}
\item {Grp. gram.:f.}
\end{itemize}
\begin{itemize}
\item {Utilização:Philol.}
\end{itemize}
\begin{itemize}
\item {Proveniência:(De \textunderscore phonético\textunderscore )}
\end{itemize}
Estudo dos sons articulados, considerados como elementos dos vocábulos.
\section{Phoneticamente}
\begin{itemize}
\item {Grp. gram.:adv.}
\end{itemize}
De modo phonético; segundo a phonética.
\section{Phoneticismo}
\begin{itemize}
\item {Grp. gram.:m.}
\end{itemize}
O mesmo que \textunderscore phonetismo\textunderscore .
\section{Phoneticista}
\begin{itemize}
\item {Grp. gram.:m.}
\end{itemize}
Philólogo, que trata especialmente de phonética.
\section{Phonético}
\begin{itemize}
\item {Grp. gram.:adj.}
\end{itemize}
\begin{itemize}
\item {Proveniência:(Gr. \textunderscore phonetikos\textunderscore )}
\end{itemize}
Relativo á voz ou ao som das palavras.
\section{Phonetismo}
\begin{itemize}
\item {Grp. gram.:m.}
\end{itemize}
\begin{itemize}
\item {Proveniência:(De \textunderscore phonético\textunderscore )}
\end{itemize}
Maneira de representar as ideias, representando os sons.
\section{Phonetista}
\begin{itemize}
\item {Grp. gram.:m.}
\end{itemize}
O mesmo ou melhor que \textunderscore phoneticista\textunderscore .
\section{Phónica}
\begin{itemize}
\item {Grp. gram.:f.}
\end{itemize}
\begin{itemize}
\item {Utilização:Phýs.}
\end{itemize}
Arte de combinar os sons, segundo as leis da acústica.
(Fem. de \textunderscore phónico\textunderscore )
\section{Phónico}
\begin{itemize}
\item {Grp. gram.:adj.}
\end{itemize}
\begin{itemize}
\item {Proveniência:(Do gr. \textunderscore phone\textunderscore )}
\end{itemize}
Relativo á voz ou ao som.
\section{Phono...}
\begin{itemize}
\item {Proveniência:(Do gr. \textunderscore phone\textunderscore )}
\end{itemize}
Elemento, que entra na formação de várias palavras, significando \textunderscore som\textunderscore  ou \textunderscore voz\textunderscore .
\section{Phonocâmptico}
\begin{itemize}
\item {Grp. gram.:adj.}
\end{itemize}
\begin{itemize}
\item {Utilização:Phýs.}
\end{itemize}
\begin{itemize}
\item {Proveniência:(Do gr. \textunderscore phone\textunderscore  + \textunderscore kamptein\textunderscore )}
\end{itemize}
Relativo á reflexão do som.
\section{Phonogramma}
\begin{itemize}
\item {Grp. gram.:m.}
\end{itemize}
\begin{itemize}
\item {Proveniência:(Do gr. \textunderscore phone\textunderscore  + \textunderscore gramma\textunderscore )}
\end{itemize}
Figura, obtida pelos processos da phonographia.
\section{Phonographia}
\begin{itemize}
\item {Grp. gram.:f.}
\end{itemize}
\begin{itemize}
\item {Proveniência:(De \textunderscore phonógrapho\textunderscore )}
\end{itemize}
Modo de representar os sons por meio de palavras.
Representação gráphica das vibrações dos corpos sonoros.
\section{Phonográphico}
\begin{itemize}
\item {Grp. gram.:adj.}
\end{itemize}
Relativo á phonographia.
\section{Phonógrapho}
\begin{itemize}
\item {Grp. gram.:m.}
\end{itemize}
\begin{itemize}
\item {Proveniência:(Do gr. \textunderscore phone\textunderscore  + \textunderscore graphein\textunderscore )}
\end{itemize}
Instrumento, que conserva e reproduz os sons ou vibrações sonoras.
\section{Phonólita}
\begin{itemize}
\item {Grp. gram.:f.}
\end{itemize}
O mesmo que \textunderscore phonólitho\textunderscore .
\section{Phonolíthico}
\begin{itemize}
\item {Grp. gram.:adj.}
\end{itemize}
Relativo ao phonólitho.
\section{Phonólitho}
\begin{itemize}
\item {Grp. gram.:f.}
\end{itemize}
\begin{itemize}
\item {Proveniência:(Do gr. \textunderscore phone\textunderscore  + \textunderscore lithos\textunderscore )}
\end{itemize}
Gênero de rochas vulcânicas, que emittem um som especial, quando percutidas por um corpo duro.
\section{Phonologia}
\begin{itemize}
\item {Grp. gram.:f.}
\end{itemize}
\begin{itemize}
\item {Utilização:Philol.}
\end{itemize}
\begin{itemize}
\item {Proveniência:(Do gr. \textunderscore phone\textunderscore  + \textunderscore logos\textunderscore )}
\end{itemize}
Tratado dos sons elementares e fundamentaes das línguas, das modificações dêsses sons representados por vocábulos, e da correcta pronúncia dêstes.
\section{Phonológico}
\begin{itemize}
\item {Grp. gram.:adj.}
\end{itemize}
Relativo á phonologia.
\section{Phonometria}
\begin{itemize}
\item {Grp. gram.:f.}
\end{itemize}
Applicação do phonómetro.
\section{Phonómetro}
\begin{itemize}
\item {Grp. gram.:m.}
\end{itemize}
\begin{itemize}
\item {Proveniência:(Do gr. \textunderscore phone\textunderscore  + \textunderscore metron\textunderscore )}
\end{itemize}
Instrumento, com que se mede a intensidade do som ou da voz.
\section{Phonophobia}
\begin{itemize}
\item {Grp. gram.:f.}
\end{itemize}
\begin{itemize}
\item {Utilização:Med.}
\end{itemize}
\begin{itemize}
\item {Proveniência:(Do gr. \textunderscore phone\textunderscore  + \textunderscore phobein\textunderscore )}
\end{itemize}
Mêdo de falar em voz alta. Cf. Sousa Martins, \textunderscore Nosograph.\textunderscore 
\section{Phonóphobo}
\begin{itemize}
\item {Grp. gram.:adj.}
\end{itemize}
\begin{itemize}
\item {Utilização:Med.}
\end{itemize}
Que padece phonophobia.
\section{Phonospasmo}
\begin{itemize}
\item {Grp. gram.:m.}
\end{itemize}
\begin{itemize}
\item {Utilização:Med.}
\end{itemize}
\begin{itemize}
\item {Proveniência:(De \textunderscore phono...\textunderscore  + \textunderscore espasmo\textunderscore )}
\end{itemize}
Espasmo ou convulsão, que acompanha a emissão da voz.
\section{Phorantho}
\begin{itemize}
\item {Grp. gram.:m.}
\end{itemize}
\begin{itemize}
\item {Proveniência:(Do gr. \textunderscore phoros\textunderscore  + \textunderscore anthos\textunderscore )}
\end{itemize}
Nome, dado por alguns botânicos ao receptáculo das flôres compostas.
\section{Phórmio}
\begin{itemize}
\item {Grp. gram.:m.}
\end{itemize}
\begin{itemize}
\item {Proveniência:(Do gr. \textunderscore phormion\textunderscore , fio)}
\end{itemize}
Gênero de plantas liliáceas, (\textunderscore phormium temax\textunderscore ), também conhecido por \textunderscore linho da Nova-Zelândia\textunderscore .
\section{Phoronomia}
\begin{itemize}
\item {Grp. gram.:f.}
\end{itemize}
\begin{itemize}
\item {Proveniência:(Do gr. \textunderscore phora\textunderscore  + \textunderscore nomos\textunderscore )}
\end{itemize}
Sciência das leis do equilíbrio e do movimento dos corpos.
Mecânica.
\section{Phosgênio}
\begin{itemize}
\item {Grp. gram.:m.}
\end{itemize}
\begin{itemize}
\item {Utilização:Chím.}
\end{itemize}
\begin{itemize}
\item {Proveniência:(Do gr. \textunderscore phos\textunderscore  + \textunderscore genos\textunderscore )}
\end{itemize}
Gás, resultante da acção dos raios solares numa mistura, em partes iguaes, de gás chloro e de gás óxydo de carbóne.
\section{Phosphatado}
\begin{itemize}
\item {Grp. gram.:adj.}
\end{itemize}
Que se acha em estado de phosphato; que tem posphato.
\section{Phosphático}
\begin{itemize}
\item {Grp. gram.:adj.}
\end{itemize}
Formado de phosphato; relativo a phosphato.
\section{Phosphatina}
\begin{itemize}
\item {Grp. gram.:f.}
\end{itemize}
\begin{itemize}
\item {Proveniência:(De \textunderscore phosphato\textunderscore )}
\end{itemize}
Preparação, em que entra farinha de arroz e de tapioca, fécula de batata, araruta, cacau e phosphato de cal, para alimentação de crianças. Cf. \textunderscore Jorn.-do-Comm.\textunderscore , do Rio, de 6-VII-902.
\section{Phosphato}
\begin{itemize}
\item {Grp. gram.:m.}
\end{itemize}
\begin{itemize}
\item {Proveniência:(De \textunderscore phósphoro\textunderscore )}
\end{itemize}
Sal, que resulta da combinação do ácido phosphórico com uma base.
\section{Phosphaturia}
\begin{itemize}
\item {Grp. gram.:f.}
\end{itemize}
\begin{itemize}
\item {Utilização:Med.}
\end{itemize}
\begin{itemize}
\item {Proveniência:(De \textunderscore phosphato\textunderscore  + gr. \textunderscore ourein\textunderscore )}
\end{itemize}
Perda de phosphato pela urina.
\section{Phosphena}
\begin{itemize}
\item {Grp. gram.:f.}
\end{itemize}
O mesmo que \textunderscore phospheno\textunderscore .
\section{Phospheno}
\begin{itemize}
\item {Grp. gram.:m.}
\end{itemize}
\begin{itemize}
\item {Proveniência:(Do gr. \textunderscore phos\textunderscore  + \textunderscore phainos\textunderscore )}
\end{itemize}
Impressão luminosa, que resulta da compressão do ôlho, estando as pálpebras fechadas.
\section{Phosphito}
\begin{itemize}
\item {Grp. gram.:m.}
\end{itemize}
Gênero de saes, produzidos pela combinação do ácido phosphoroso com as bases.
\section{Phosphoglycerato}
\begin{itemize}
\item {Grp. gram.:m.}
\end{itemize}
Sal, resultante do ácido phosphoglycérico com uma base.
\section{Phosphoglycérico}
\begin{itemize}
\item {Grp. gram.:adj.}
\end{itemize}
Diz-se de um ácido, resultante do desdobramento do protagão, sob a acção da água de baryta concentrada, ou que se fórma pela mistura de glycerina com ácido phosphórico.
\section{Phosphorar}
\begin{itemize}
\item {Grp. gram.:v. t.}
\end{itemize}
Combinar ou misturar com phósphoro.
\section{Phosphorear}
\begin{itemize}
\item {Grp. gram.:v. i.}
\end{itemize}
Brilhar como o phósphoro.
\section{Phosphoreira}
\begin{itemize}
\item {Grp. gram.:f.}
\end{itemize}
Caixinha ou utensílio, para guardar phósphoros.
\section{Phosphoreiro}
\begin{itemize}
\item {Grp. gram.:m.}
\end{itemize}
Aquelle que trabalha no fabríco de phósphoros.
\section{Phosphorejante}
\begin{itemize}
\item {Grp. gram.:adj.}
\end{itemize}
Que phosphoreja.
\section{Phosphorejar}
\begin{itemize}
\item {Grp. gram.:v. i.}
\end{itemize}
\begin{itemize}
\item {Utilização:Neol.}
\end{itemize}
Brilhar como phósphoro inflammado; chammejar.
\section{Phosphóreo}
\begin{itemize}
\item {Grp. gram.:adj.}
\end{itemize}
\begin{itemize}
\item {Proveniência:(Lat. \textunderscore phosphoreus\textunderscore )}
\end{itemize}
O mesmo que \textunderscore phosphórico\textunderscore ; que tem phósphoro.
\section{Phosphorescência}
\begin{itemize}
\item {Grp. gram.:f.}
\end{itemize}
Propriedade dos corpos phosphorescentes.
\section{Phosphorescente}
\begin{itemize}
\item {Grp. gram.:adj.}
\end{itemize}
\begin{itemize}
\item {Proveniência:(De \textunderscore phósphoro\textunderscore )}
\end{itemize}
Que brilha na obscuridade, sem calor nem combustão.
Que, sendo friccionado, se torna luminoso ou se sujeita a uma descarga electrica.
\section{Phosphorescer}
\begin{itemize}
\item {Grp. gram.:v. i.}
\end{itemize}
\begin{itemize}
\item {Proveniência:(De \textunderscore phósphoro\textunderscore )}
\end{itemize}
Lançar brilho phosphorescente. Cf. Latino, \textunderscore Camões\textunderscore , 119.
\section{Phosphoreto}
\begin{itemize}
\item {fónica:forê}
\end{itemize}
\begin{itemize}
\item {Grp. gram.:m.}
\end{itemize}
\begin{itemize}
\item {Proveniência:(De \textunderscore phósphoro\textunderscore )}
\end{itemize}
Combinação mineral ou orgânica, que contém phósphoro como elemento electro-negativo.
\section{Phosphórico}
\begin{itemize}
\item {Grp. gram.:adj.}
\end{itemize}
\begin{itemize}
\item {Utilização:Pop.}
\end{itemize}
Relativo a phósphoro.
Que brilha como phósphoro.
Diz-se de um ácido, formado pela combustão do phósphoro.
Embaraçado; diffícil.
Irascível.
\section{Phosphorinos}
\begin{itemize}
\item {Grp. gram.:m. pl.}
\end{itemize}
Uma das quatro ordens dos oxysaes, a qual comprehende a turquesa, o nitro, etc.
(Cp. \textunderscore phósphoro\textunderscore )
\section{Phosphoríphoro}
\begin{itemize}
\item {Grp. gram.:adj.}
\end{itemize}
\begin{itemize}
\item {Proveniência:(Do gr. \textunderscore phosphoros\textunderscore  + \textunderscore phoros\textunderscore )}
\end{itemize}
Diz-se dos animaes, em que uma parte do corpo é phosphorescente.
\section{Phosphorista}
\begin{itemize}
\item {Grp. gram.:m.}
\end{itemize}
Manipulador de phósphoros.
\section{Phosphorita}
\begin{itemize}
\item {Grp. gram.:f.}
\end{itemize}
O mesmo que \textunderscore phosphorito\textunderscore .
\section{Phosphorite}
\begin{itemize}
\item {Grp. gram.:f.}
\end{itemize}
O mesmo que \textunderscore phosphorito\textunderscore .
\section{Phosphorito}
\begin{itemize}
\item {Grp. gram.:m.}
\end{itemize}
\begin{itemize}
\item {Proveniência:(De \textunderscore phósphoro\textunderscore )}
\end{itemize}
Phosphato de ferro natural.
\section{Phosphorização}
\begin{itemize}
\item {Grp. gram.:f.}
\end{itemize}
Acto ou effeito de phosphorizar.
Influência ou formação de phosphato calcário na economia animal.
\section{Phosphorizar}
\begin{itemize}
\item {Grp. gram.:v. t.}
\end{itemize}
\begin{itemize}
\item {Proveniência:(De \textunderscore phósphoro\textunderscore )}
\end{itemize}
Tornar phosphórico.
Converter em phosphato.
\section{Phósphoro}
\begin{itemize}
\item {Grp. gram.:m.}
\end{itemize}
\begin{itemize}
\item {Grp. gram.:Pl.}
\end{itemize}
\begin{itemize}
\item {Utilização:Chul.}
\end{itemize}
\begin{itemize}
\item {Proveniência:(Gr. \textunderscore phósphoros\textunderscore )}
\end{itemize}
Corpo simples, combustível, luminoso na obscuridade e ardendo ao contacto do ar.
Palito ou pavio, em cuja extremidade há uma substância, que se inflamma com a fricção.
\textunderscore Phósphoros de espera gallego\textunderscore , phósphoros de enxôfre.
\section{Phosphoroscópio}
\begin{itemize}
\item {Grp. gram.:m.}
\end{itemize}
\begin{itemize}
\item {Proveniência:(Do gr. \textunderscore phosphoros\textunderscore  + \textunderscore skopein\textunderscore )}
\end{itemize}
Instrumento, para observar a phosphorescência dos corpos.
\section{Phosphoroso}
\begin{itemize}
\item {Grp. gram.:adj.}
\end{itemize}
\begin{itemize}
\item {Proveniência:(De \textunderscore phósphoro\textunderscore )}
\end{itemize}
Phosphóreo.
Diz-se do ácido, também chamado phosphórico.
\section{Phosphosiderito}
\begin{itemize}
\item {fónica:si}
\end{itemize}
\begin{itemize}
\item {Grp. gram.:m.}
\end{itemize}
\begin{itemize}
\item {Utilização:Miner.}
\end{itemize}
\begin{itemize}
\item {Proveniência:(De \textunderscore phósphoro\textunderscore  + gr. \textunderscore sideros\textunderscore )}
\end{itemize}
Phosphato hydratado de ferro.
\section{Phosphovinato}
\begin{itemize}
\item {Grp. gram.:m.}
\end{itemize}
Combinação do ácido phosphovínico com uma base.
\section{Phosphovínico}
\begin{itemize}
\item {Grp. gram.:adj.}
\end{itemize}
\begin{itemize}
\item {Proveniência:(De \textunderscore phosphórico\textunderscore  + \textunderscore vínico\textunderscore )}
\end{itemize}
Diz-se de um ácido, composto do ácido phosphórico e elementos de álcool.
\section{Photínia}
\begin{itemize}
\item {Grp. gram.:f.}
\end{itemize}
\begin{itemize}
\item {Proveniência:(Do gr. \textunderscore photeinos\textunderscore )}
\end{itemize}
Gênero de árvores rosáceas da Califórnia e da Ásia tropical.
\section{Photinianos}
\begin{itemize}
\item {Grp. gram.:m. pl.}
\end{itemize}
Herejes do século IV, que negavam ao Espírito-Santo a personalidade divina e sustentavam que Jesus era filho de José.
\section{Photismo}
\begin{itemize}
\item {Grp. gram.:m.}
\end{itemize}
\begin{itemize}
\item {Proveniência:(Do gr. \textunderscore phos\textunderscore , \textunderscore photos\textunderscore )}
\end{itemize}
Sensação visual secundária. Cf. R. Galvão, \textunderscore Vocab.\textunderscore 
\section{Photo...}
\begin{itemize}
\item {Grp. gram.:pref.}
\end{itemize}
\begin{itemize}
\item {Proveniência:(Do gr. \textunderscore phos\textunderscore , \textunderscore photos\textunderscore )}
\end{itemize}
(designativo de \textunderscore luz\textunderscore )
\section{Photocalco}
\begin{itemize}
\item {Grp. gram.:m.}
\end{itemize}
\begin{itemize}
\item {Proveniência:(Do gr. \textunderscore phos\textunderscore , \textunderscore photos\textunderscore  + \textunderscore khalkos\textunderscore )}
\end{itemize}
Pequeno apparelho, espécie de câmara escura simplificada, para facilitar o desenho de uma paisagem, de um monumento, etc.
\section{Photocartographia}
\begin{itemize}
\item {Grp. gram.:f.}
\end{itemize}
\begin{itemize}
\item {Proveniência:(De \textunderscore photo...\textunderscore  + \textunderscore cartographia\textunderscore )}
\end{itemize}
Applicação da photographia a reproducções cartográphicas.
\section{Photocerâmica}
\begin{itemize}
\item {Grp. gram.:f.}
\end{itemize}
\begin{itemize}
\item {Proveniência:(De \textunderscore photo...\textunderscore  + \textunderscore cerâmica\textunderscore )}
\end{itemize}
Applicação da photographia á reproducção de desenhos em loiça.
\section{Photochímica}
\begin{itemize}
\item {fónica:qui}
\end{itemize}
\begin{itemize}
\item {Grp. gram.:f.}
\end{itemize}
\begin{itemize}
\item {Proveniência:(De \textunderscore photo...\textunderscore  + \textunderscore chímica\textunderscore )}
\end{itemize}
Theoria das acções chímicas da luz.
\section{Photochromaticamente}
\begin{itemize}
\item {fónica:cro}
\end{itemize}
\begin{itemize}
\item {Grp. gram.:adv.}
\end{itemize}
De modo photocromático.
\section{Photochromático}
\begin{itemize}
\item {fónica:cro}
\end{itemize}
\begin{itemize}
\item {Grp. gram.:adj.}
\end{itemize}
\begin{itemize}
\item {Proveniência:(De \textunderscore photo...\textunderscore  + \textunderscore chromático\textunderscore )}
\end{itemize}
Relativo á reproducção das côres pela photographia.
\section{Photochromographia}
\begin{itemize}
\item {Grp. gram.:f.}
\end{itemize}
\begin{itemize}
\item {Proveniência:(De \textunderscore photo...\textunderscore  + \textunderscore chromographia\textunderscore )}
\end{itemize}
Processo photográphico, com que se obtém a imagem colorida.
\section{Photocollographia}
\begin{itemize}
\item {Grp. gram.:f.}
\end{itemize}
Reproducção photográphica, que tem por base a gelatina.
(Do gr.\textunderscore  phos\textunderscore , \textunderscore photos\textunderscore  + \textunderscore kolla\textunderscore  + \textunderscore graphein\textunderscore )
\section{Photocópia}
\begin{itemize}
\item {Grp. gram.:f.}
\end{itemize}
\begin{itemize}
\item {Proveniência:(De \textunderscore photo...\textunderscore  + \textunderscore cópia\textunderscore )}
\end{itemize}
Reproducção de uma imagem em papel chímico, impressionável pela luz e mediante uma matriz transparente.
\section{Photodoscópio}
\begin{itemize}
\item {Grp. gram.:m.}
\end{itemize}
\begin{itemize}
\item {Utilização:Phýs.}
\end{itemize}
\begin{itemize}
\item {Proveniência:(Do gr. \textunderscore photodes\textunderscore  + \textunderscore skopein\textunderscore )}
\end{itemize}
Apparelho para observar a luz.
\section{Photo-eléctrico}
\begin{itemize}
\item {Grp. gram.:adj.}
\end{itemize}
\begin{itemize}
\item {Proveniência:(De \textunderscore photo...\textunderscore  + \textunderscore eléctrico\textunderscore )}
\end{itemize}
Que fornece ou transmitte luz eléctrica.
\section{Photofulgural}
\begin{itemize}
\item {Grp. gram.:adj.}
\end{itemize}
\begin{itemize}
\item {Utilização:Neol.}
\end{itemize}
\begin{itemize}
\item {Proveniência:(De \textunderscore photo...\textunderscore  + \textunderscore fulgural\textunderscore )}
\end{itemize}
Relativo aos processos photográphicos, que se exercem através dos corpos opacos.
\section{Photogênico}
\begin{itemize}
\item {Grp. gram.:adj.}
\end{itemize}
Que produz imagens por meio da luz.
Que se representa bem pela photographia.
(Cp. \textunderscore photogênio\textunderscore )
\section{Photogênio}
\begin{itemize}
\item {Grp. gram.:m.}
\end{itemize}
\begin{itemize}
\item {Proveniência:(Do gr. \textunderscore photos\textunderscore  + \textunderscore genea\textunderscore )}
\end{itemize}
Designação scientífica do petróleo do commércio.
\section{Photogramma}
\begin{itemize}
\item {Grp. gram.:m.}
\end{itemize}
\begin{itemize}
\item {Proveniência:(De \textunderscore photo...\textunderscore  + \textunderscore gramma\textunderscore )}
\end{itemize}
Qualquer reproducção photográphica.
\section{Photographar}
\begin{itemize}
\item {Grp. gram.:v. t.}
\end{itemize}
\begin{itemize}
\item {Utilização:Fig.}
\end{itemize}
\begin{itemize}
\item {Proveniência:(De \textunderscore photógrapho\textunderscore )}
\end{itemize}
Reproduzir pela photographia a imagem de; retratar.
Descrever exactamente.
\section{Photographia}
\begin{itemize}
\item {Grp. gram.:f.}
\end{itemize}
\begin{itemize}
\item {Utilização:Fig.}
\end{itemize}
\begin{itemize}
\item {Proveniência:(De \textunderscore photógrapho\textunderscore )}
\end{itemize}
Processo ou arte de fixar, numa chapa sensível, pelo auxilio da luz, a imagem dos objectos, que estão deante de uma câmara escura.
Cópia fiel, reproducção exacta.
\section{Photographicamente}
\begin{itemize}
\item {Grp. gram.:adv.}
\end{itemize}
\begin{itemize}
\item {Proveniência:(De \textunderscore photográphico\textunderscore )}
\end{itemize}
Por meio de photographia.
\section{Photográphico}
\begin{itemize}
\item {Grp. gram.:adj.}
\end{itemize}
Relativo á photographia.
\section{Photógrapho}
\begin{itemize}
\item {Grp. gram.:m.}
\end{itemize}
\begin{itemize}
\item {Proveniência:(Do gr. \textunderscore phos\textunderscore , \textunderscore photos\textunderscore  + \textunderscore graphein\textunderscore )}
\end{itemize}
Aquelle que se occupa de photographia, ou que exerce a photographia.
\section{Photogravura}
\begin{itemize}
\item {Grp. gram.:f.}
\end{itemize}
\begin{itemize}
\item {Proveniência:(De \textunderscore photo...\textunderscore  + \textunderscore gravura\textunderscore )}
\end{itemize}
Conjuncto dos processos photográphicos, por meio dos quaes se produzem pranchas gravadas.
\section{Photolithographia}
\begin{itemize}
\item {Grp. gram.:f.}
\end{itemize}
\begin{itemize}
\item {Proveniência:(De \textunderscore photo...\textunderscore  + \textunderscore lithographia\textunderscore )}
\end{itemize}
Processo, com que se transporta para a pedra lithográphica uma prova photográphica.
\section{Photologia}
\begin{itemize}
\item {Grp. gram.:f.}
\end{itemize}
\begin{itemize}
\item {Proveniência:(Do gr. \textunderscore phos\textunderscore , \textunderscore photos\textunderscore  + \textunderscore logos\textunderscore )}
\end{itemize}
Tratado á cêrca da luz.
\section{Photológico}
\begin{itemize}
\item {Grp. gram.:adj.}
\end{itemize}
Relativo á photologia.
\section{Photomagnético}
\begin{itemize}
\item {Grp. gram.:adj.}
\end{itemize}
\begin{itemize}
\item {Proveniência:(De \textunderscore photo...\textunderscore  + \textunderscore magnético\textunderscore )}
\end{itemize}
Relativo aos phenómenos magnéticos, devidos á acção da luz.
\section{Photometria}
\begin{itemize}
\item {Grp. gram.:f.}
\end{itemize}
Applicação do photómetro.
\section{Photométrico}
\begin{itemize}
\item {Grp. gram.:adj.}
\end{itemize}
Relativo á photometria.
\section{Photómetro}
\begin{itemize}
\item {Grp. gram.:m.}
\end{itemize}
\begin{itemize}
\item {Proveniência:(Do gr. \textunderscore phos\textunderscore , \textunderscore photos\textunderscore  + \textunderscore metron\textunderscore )}
\end{itemize}
Instrumento, com que se avalia a intensidade da luz.
\section{Photomicrographia}
\begin{itemize}
\item {Grp. gram.:f.}
\end{itemize}
\begin{itemize}
\item {Proveniência:(Do gr. \textunderscore phos\textunderscore , \textunderscore photos\textunderscore  + \textunderscore mikos\textunderscore  + \textunderscore graphein\textunderscore )}
\end{itemize}
Reproducção photográphica de objectos muito pequenos ou microscópicos.
\section{Photomicrográphico}
\begin{itemize}
\item {Grp. gram.:adj.}
\end{itemize}
Relativo á photomicrographia.
\section{Photominiatura}
\begin{itemize}
\item {Grp. gram.:f.}
\end{itemize}
Processo para reduzir a pequenas dimensões, com o auxílio da photographia, quadros, paisagens, desenhos, etc.
\section{Photominiaturista}
\begin{itemize}
\item {Grp. gram.:m.  e  f.}
\end{itemize}
Pessôa, que exerce a photominiatura.
\section{Photophobia}
\begin{itemize}
\item {Grp. gram.:f.}
\end{itemize}
\begin{itemize}
\item {Proveniência:(De \textunderscore photo...\textunderscore  + \textunderscore phobia\textunderscore )}
\end{itemize}
Aversão á luz.
\section{Photóphoro}
\begin{itemize}
\item {Grp. gram.:adj.}
\end{itemize}
\begin{itemize}
\item {Proveniência:(Do gr. \textunderscore phos\textunderscore , \textunderscore photos\textunderscore  + \textunderscore phoros\textunderscore )}
\end{itemize}
Diz-se de todo apparelho, que permitte obter um feixe luminoso, dirigido sôbre um ponto dado.
\section{Photopsia}
\begin{itemize}
\item {Grp. gram.:f.}
\end{itemize}
\begin{itemize}
\item {Proveniência:(Do gr. \textunderscore phos\textunderscore , \textunderscore photos\textunderscore  + \textunderscore ops\textunderscore )}
\end{itemize}
Visão de traços luminosos que não existem.
\section{Photosculptura}
\begin{itemize}
\item {Grp. gram.:f.}
\end{itemize}
\begin{itemize}
\item {Proveniência:(De \textunderscore photo...\textunderscore  + \textunderscore esculptura\textunderscore )}
\end{itemize}
Processo photográphico, com que, reunindo os perfis de uma pessôa, se obtem uma espécie de estatueta.
\section{Photosphera}
\begin{itemize}
\item {Grp. gram.:f.}
\end{itemize}
\begin{itemize}
\item {Proveniência:(De \textunderscore photo...\textunderscore  + \textunderscore esphera\textunderscore )}
\end{itemize}
Atmosphera luminosa do Sol.
\section{Phototaxia}
\begin{itemize}
\item {fónica:csi}
\end{itemize}
\begin{itemize}
\item {Grp. gram.:f.}
\end{itemize}
\begin{itemize}
\item {Proveniência:(Do gr. \textunderscore phos\textunderscore , \textunderscore photos\textunderscore  + \textunderscore taxis\textunderscore )}
\end{itemize}
Propriedade, que o protoplasma tem, de reagir á luz.
\section{Phototelegraphia}
\begin{itemize}
\item {Grp. gram.:f.}
\end{itemize}
\begin{itemize}
\item {Proveniência:(De \textunderscore photo\textunderscore  + \textunderscore telegraphia\textunderscore )}
\end{itemize}
Reproducção de uma imagem a distância, por meio do fio eléctrico.
\section{Phototherapia}
\begin{itemize}
\item {Grp. gram.:f.}
\end{itemize}
\begin{itemize}
\item {Proveniência:(De \textunderscore photo...\textunderscore  + \textunderscore therapia\textunderscore )}
\end{itemize}
Tratamento médico pela acção da luz.
\section{Phototherápico}
\begin{itemize}
\item {Grp. gram.:adj.}
\end{itemize}
Relativo á phototherapia.
\section{Phototopographia}
\begin{itemize}
\item {Grp. gram.:f.}
\end{itemize}
\begin{itemize}
\item {Proveniência:(De \textunderscore photo...\textunderscore  + \textunderscore topographia\textunderscore )}
\end{itemize}
Arte de photographar lugares ou paisagens.
\section{Phototypar}
\begin{itemize}
\item {Grp. gram.:v. t.}
\end{itemize}
O mesmo que \textunderscore phototypiar\textunderscore .
\section{Phototypia}
\begin{itemize}
\item {Grp. gram.:f.}
\end{itemize}
O mesmo que \textunderscore phototypographia\textunderscore .
\section{Phototypiar}
\begin{itemize}
\item {Grp. gram.:v. t.}
\end{itemize}
\begin{itemize}
\item {Utilização:Neol.}
\end{itemize}
\begin{itemize}
\item {Proveniência:(De \textunderscore phototypia\textunderscore )}
\end{itemize}
Reproduzir (desenho, figura ou paisagem), pelo processo phototypográphico. Cf. Alv. Mendes, \textunderscore Herculano\textunderscore , 44.
\section{Phototypographia}
\begin{itemize}
\item {Grp. gram.:f.}
\end{itemize}
\begin{itemize}
\item {Proveniência:(De \textunderscore photo...\textunderscore  + \textunderscore typographia\textunderscore )}
\end{itemize}
Processo, com que se reproduzem pela photographia trabalhos typográphicos.
\section{Phototypográphico}
\begin{itemize}
\item {Grp. gram.:adj.}
\end{itemize}
Relativo á phototypographia.
\section{Phototypogravura}
\begin{itemize}
\item {Grp. gram.:f.}
\end{itemize}
Processo photográphico, próprio para tiragens typográphicas, dando-se meias tintas.
\section{Photozincographia}
\begin{itemize}
\item {Grp. gram.:f.}
\end{itemize}
\begin{itemize}
\item {Proveniência:(De \textunderscore photo...\textunderscore  + \textunderscore zincographia\textunderscore )}
\end{itemize}
Processo de heliogravura, sôbre zinco.
\section{Photozincográphico}
\begin{itemize}
\item {Grp. gram.:adj.}
\end{itemize}
Relativo á photozincographia.
\section{Phragma}
\begin{itemize}
\item {Grp. gram.:m.}
\end{itemize}
\begin{itemize}
\item {Utilização:Bot.}
\end{itemize}
\begin{itemize}
\item {Proveniência:(Gr. \textunderscore phragma\textunderscore )}
\end{itemize}
Nome, dado por Linck aos septos transversaes dos frutos.
\section{Phrase}
\begin{itemize}
\item {Grp. gram.:f.}
\end{itemize}
\begin{itemize}
\item {Proveniência:(Lat. \textunderscore phrasis\textunderscore )}
\end{itemize}
Reunião de palavras, que formam sentido completo.
Locução; expressão.
Conjunto de sons musicaes, com uma pausa depois do último.
\section{Phraseado}
\begin{itemize}
\item {Grp. gram.:adj.}
\end{itemize}
\begin{itemize}
\item {Grp. gram.:M.}
\end{itemize}
\begin{itemize}
\item {Proveniência:(De \textunderscore phrasear\textunderscore )}
\end{itemize}
Que está disposto em phrases.
Modo de dizer ou de escever.
Conjunto de palavras.
\section{Phraseador}
\begin{itemize}
\item {Grp. gram.:m.  e  adj.}
\end{itemize}
O que phraseia.
\section{Phrasear}
\begin{itemize}
\item {Grp. gram.:v. i.}
\end{itemize}
Fazer phrases.
\section{Phraseologia}
\begin{itemize}
\item {Grp. gram.:f}
\end{itemize}
\begin{itemize}
\item {Proveniência:(Do gr. \textunderscore phrasis\textunderscore  + \textunderscore logos\textunderscore )}
\end{itemize}
Parte da Grammática, em que se estuda a construcção da phrase.
Construcção de phrase.
\section{Phraseologicamente}
\begin{itemize}
\item {Grp. gram.:adv.}
\end{itemize}
\begin{itemize}
\item {Proveniência:(De \textunderscore phraseológico\textunderscore )}
\end{itemize}
Segundo as regras da phraseologia.
\section{Phraseológico}
\begin{itemize}
\item {Grp. gram.:adj.}
\end{itemize}
Relativo á phraseologia.
\section{Phrásis}
\begin{itemize}
\item {Grp. gram.:m.}
\end{itemize}
\begin{itemize}
\item {Utilização:Ant.}
\end{itemize}
O mesmo que \textunderscore phrase\textunderscore ; discurso. Cf. \textunderscore Eufrosina\textunderscore , 189.
\section{Phratria}
\begin{itemize}
\item {Grp. gram.:f.}
\end{itemize}
\begin{itemize}
\item {Proveniência:(Gr. \textunderscore phratria\textunderscore )}
\end{itemize}
Cada uma das três divisões de cada tríbo atheniense.
\section{Phrenalgia}
\begin{itemize}
\item {Grp. gram.:f.}
\end{itemize}
\begin{itemize}
\item {Utilização:Med.}
\end{itemize}
\begin{itemize}
\item {Proveniência:(Do gr. \textunderscore phren\textunderscore  + \textunderscore algos\textunderscore )}
\end{itemize}
Dôr rheumática na cabeça.
\section{Phrênico}
\begin{itemize}
\item {Grp. gram.:adj.}
\end{itemize}
\begin{itemize}
\item {Proveniência:(Do gr. \textunderscore phren\textunderscore )}
\end{itemize}
Relativo ao diaphragma.
\section{Phrenite}
\begin{itemize}
\item {Grp. gram.:f.}
\end{itemize}
\begin{itemize}
\item {Proveniência:(Do gr. \textunderscore phren\textunderscore )}
\end{itemize}
Inflammação do diaphragma.
\section{Phrenogástrico}
\begin{itemize}
\item {Grp. gram.:adj.}
\end{itemize}
\begin{itemize}
\item {Utilização:Anat.}
\end{itemize}
\begin{itemize}
\item {Proveniência:(Do gr. \textunderscore phren\textunderscore  + \textunderscore gaster\textunderscore )}
\end{itemize}
Relativo ao estômago e ao diaphragma.
\section{Phrenoglottismo}
\begin{itemize}
\item {Grp. gram.:m.}
\end{itemize}
\begin{itemize}
\item {Utilização:Med.}
\end{itemize}
\begin{itemize}
\item {Proveniência:(Do gr. \textunderscore phren\textunderscore  + \textunderscore glotta\textunderscore )}
\end{itemize}
Espasmo da glotte e do diaphragma.
\section{Phrenologia}
\begin{itemize}
\item {Grp. gram.:f.}
\end{itemize}
Systema physiológico, que considera a conformação e as protuberâncias do cérebro como indicativas das diversas faculdades ou disposições innatas do indivíduo.
(Cp. \textunderscore phrenólogo\textunderscore )
\section{Phrenologicamente}
\begin{itemize}
\item {Grp. gram.:adv.}
\end{itemize}
De modo phrenológico.
Segundo a phrenologia.
\section{Phrenológico}
\begin{itemize}
\item {Grp. gram.:adj.}
\end{itemize}
Relativo á phrenologia.
\section{Phrenologismo}
\begin{itemize}
\item {Grp. gram.:m.}
\end{itemize}
\begin{itemize}
\item {Proveniência:(De \textunderscore phrenologia\textunderscore )}
\end{itemize}
Theoria dos phrenólogos; phrenologia. Cf. Ed. Burnay, \textunderscore Craniologia\textunderscore , 106.
\section{Phrenologista}
\begin{itemize}
\item {Grp. gram.:m.  e  f.}
\end{itemize}
Pessôa, que trata de phrenologia; pessôa, partidária da phrenologia.
\section{Phrenólogo}
\begin{itemize}
\item {Grp. gram.:m.}
\end{itemize}
\begin{itemize}
\item {Proveniência:(Do gr. \textunderscore phren\textunderscore , intelligência + \textunderscore logos\textunderscore , tratado)}
\end{itemize}
Aquelle que é versado em phrenologia.
\section{Phrenopatha}
\begin{itemize}
\item {Grp. gram.:m.}
\end{itemize}
\begin{itemize}
\item {Proveniência:(Do gr. \textunderscore phren\textunderscore , \textunderscore phenos\textunderscore  + \textunderscore pathos\textunderscore )}
\end{itemize}
O que padece phrenopathia.
\section{Phrenopathia}
\begin{itemize}
\item {Grp. gram.:f.}
\end{itemize}
Doença mental.
(Cp. \textunderscore phrenopatha\textunderscore )
\section{Phrenopáthico}
\begin{itemize}
\item {Grp. gram.:adj.}
\end{itemize}
Relativo á phrenopathia.
\section{Phronema}
\begin{itemize}
\item {Grp. gram.:m.}
\end{itemize}
\begin{itemize}
\item {Utilização:Philos.}
\end{itemize}
\begin{itemize}
\item {Proveniência:(Gr. \textunderscore phronema\textunderscore )}
\end{itemize}
Foco do pensamento, no cérebro, onde se executa o trabalho da razão pura e que é distinto dos focos sensórios.
\section{Phronetas}
\begin{itemize}
\item {Grp. gram.:m. pl.}
\end{itemize}
\begin{itemize}
\item {Utilização:Philos.}
\end{itemize}
Centros de associação, para a formação do pensamento.
\section{Phrygânio}
\begin{itemize}
\item {Grp. gram.:f.}
\end{itemize}
\begin{itemize}
\item {Proveniência:(Lat. \textunderscore phryganius\textunderscore )}
\end{itemize}
Gênero de insectos hemípteros, cujas espécies são quási todas europeias.
\section{Phrýgio}
\begin{itemize}
\item {Grp. gram.:adj.}
\end{itemize}
\begin{itemize}
\item {Grp. gram.:M.}
\end{itemize}
\begin{itemize}
\item {Proveniência:(Lat. \textunderscore phrygius\textunderscore )}
\end{itemize}
Relativo á Phrygia ou aos seus habitantes.
Diz-se de um barrete encarnado, adoptado em França, no tempo da primeira República e semelhante ao que usavam os Phrýgios.
Habitante da Phrýgia.
Um dos idiomas mais antigos do Oriente.
\section{Phthalâmico}
\begin{itemize}
\item {Grp. gram.:adj.}
\end{itemize}
Diz-se de um ácido, resultante da dissolução de ácido phtálico em o ammoníaco.
\section{Phthalato}
\begin{itemize}
\item {Grp. gram.:m.}
\end{itemize}
\begin{itemize}
\item {Utilização:Chím.}
\end{itemize}
Sal, formado pelo ácido phtálico com uma base.
\section{Phthálico}
\begin{itemize}
\item {Grp. gram.:adj.}
\end{itemize}
Diz-se de um ácido, produzido pela acção do ácido azótico sôbre o bichloreto de naphthalina.
\section{Phthanite}
\begin{itemize}
\item {Grp. gram.:f.}
\end{itemize}
O mesmo que \textunderscore phthanito\textunderscore .
\section{Phthanito}
\begin{itemize}
\item {Grp. gram.:m.}
\end{itemize}
\begin{itemize}
\item {Utilização:Miner.}
\end{itemize}
\begin{itemize}
\item {Proveniência:(Do gr. \textunderscore phthanein\textunderscore )}
\end{itemize}
Silex preto.
\section{Phthiríase}
\begin{itemize}
\item {Grp. gram.:f.}
\end{itemize}
\begin{itemize}
\item {Utilização:Med.}
\end{itemize}
\begin{itemize}
\item {Utilização:Bot.}
\end{itemize}
\begin{itemize}
\item {Proveniência:(Lat. \textunderscore phthiriasis\textunderscore )}
\end{itemize}
Doença, que consiste em uma excessiva multiplicação de piolhos.
Doença de vegetaes, em que êlles se cobrem de pequeníssimos parasitos.
\section{Phthisiógeno}
\begin{itemize}
\item {Grp. gram.:adj.}
\end{itemize}
\begin{itemize}
\item {Proveniência:(Do gr. \textunderscore phthisis\textunderscore  + \textunderscore genes\textunderscore )}
\end{itemize}
Que produz tísica.
\section{Phthisiologia}
\begin{itemize}
\item {Grp. gram.:f.}
\end{itemize}
\begin{itemize}
\item {Proveniência:(Do gr. \textunderscore phthisis\textunderscore  + \textunderscore logos\textunderscore )}
\end{itemize}
Tratado médico á cêrca da tísica.
\section{Phthisiologista}
\begin{itemize}
\item {Grp. gram.:m.}
\end{itemize}
Aquelle que é perito em phthisiologia.
\section{Phthisuria}
\begin{itemize}
\item {Grp. gram.:f.}
\end{itemize}
\begin{itemize}
\item {Utilização:Med.}
\end{itemize}
\begin{itemize}
\item {Proveniência:(Do gr. \textunderscore phthisis\textunderscore  + \textunderscore ouron\textunderscore )}
\end{itemize}
Consumpção phýsica, produzida por uma excessiva secreção de urina, especialmente de urina açucarada.
\section{Phthórico}
\begin{itemize}
\item {Grp. gram.:adj.}
\end{itemize}
Relativo ao phthório.
\section{Phthório}
\begin{itemize}
\item {Grp. gram.:m.}
\end{itemize}
\begin{itemize}
\item {Utilização:Chím.}
\end{itemize}
\begin{itemize}
\item {Proveniência:(Lat. \textunderscore phthorius\textunderscore )}
\end{itemize}
Nome, dado por Ampére ao fluor, porque êste corrói os vasos em que se contém.
\section{Phýceas}
\begin{itemize}
\item {Grp. gram.:f. pl.}
\end{itemize}
\begin{itemize}
\item {Proveniência:(Do gr. \textunderscore phukos\textunderscore , alga)}
\end{itemize}
Plantas aquáticas, de organização simples e fórma variada.
\section{Phycite}
\begin{itemize}
\item {Grp. gram.:f.}
\end{itemize}
\begin{itemize}
\item {Utilização:Chím.}
\end{itemize}
\begin{itemize}
\item {Proveniência:(Do gr. \textunderscore phukos\textunderscore )}
\end{itemize}
Substância crystallina, que se acha numa alga, e é o \textunderscore protococcus vulgaris\textunderscore .
\section{Phycocyano}
\begin{itemize}
\item {Grp. gram.:m.}
\end{itemize}
\begin{itemize}
\item {Proveniência:(Do gr. \textunderscore phukos\textunderscore  + \textunderscore kuanon\textunderscore )}
\end{itemize}
Substância còrante, azulada, extrahida de certas algas.
\section{Phycóide}
\begin{itemize}
\item {Grp. gram.:adj.}
\end{itemize}
\begin{itemize}
\item {Proveniência:(Do gr. \textunderscore phukos\textunderscore  + \textunderscore eidos\textunderscore )}
\end{itemize}
Semelhante ás algas.
\section{Phycolíchens}
\begin{itemize}
\item {Grp. gram.:m. pl.}
\end{itemize}
\begin{itemize}
\item {Proveniência:(Do gr. \textunderscore phukon\textunderscore  + \textunderscore likhen\textunderscore )}
\end{itemize}
Líchens que, pela sua configuração, se aproximam das algas.
\section{Phycologia}
\begin{itemize}
\item {Grp. gram.:f.}
\end{itemize}
\begin{itemize}
\item {Proveniência:(Do gr. \textunderscore phukos\textunderscore  + \textunderscore logos\textunderscore )}
\end{itemize}
Parte da Botânica, que trata das algas.
\section{Phycológico}
\begin{itemize}
\item {Grp. gram.:adj.}
\end{itemize}
Relativo á phycologia.
\section{Phycologista}
\begin{itemize}
\item {Grp. gram.:m.}
\end{itemize}
Naturalista, que trata de phycologia.
\section{Phylactera}
\begin{itemize}
\item {Grp. gram.:f.}
\end{itemize}
Espécie de banda ou bandeirola que, por cima dos escudos ou insuladamente, exhibe uma divisa ou legenda.
(Cp. \textunderscore phylactério\textunderscore )
\section{Phylactério}
\begin{itemize}
\item {Grp. gram.:m.}
\end{itemize}
\begin{itemize}
\item {Proveniência:(Lat. \textunderscore phylacterium\textunderscore )}
\end{itemize}
Nome, que os antigos davam a amuletos.
Pedaço de pelle ou pergaminho, em que estavam escritos os mandamentos de Deus, e que os Judeus traziam consigo.
\section{Phylarcho}
\begin{itemize}
\item {Grp. gram.:m.}
\end{itemize}
\begin{itemize}
\item {Proveniência:(Lat. \textunderscore phylarchus\textunderscore )}
\end{itemize}
Chefe de tríbo, nos primeiros tempos da república atheniense.
\section{Phyllântheas}
\begin{itemize}
\item {Grp. gram.:f.}
\end{itemize}
\begin{itemize}
\item {Utilização:Bot.}
\end{itemize}
\begin{itemize}
\item {Proveniência:(De \textunderscore phyllantho\textunderscore )}
\end{itemize}
Tríbo de euphorbiáceas.
\section{Phyllantho}
\begin{itemize}
\item {Grp. gram.:adj.}
\end{itemize}
\begin{itemize}
\item {Utilização:Bot.}
\end{itemize}
\begin{itemize}
\item {Grp. gram.:M.}
\end{itemize}
\begin{itemize}
\item {Proveniência:(Lat. \textunderscore phyllantes\textunderscore )}
\end{itemize}
Cujas flôres repoisam sôbre as fôlhas.
Gênero de plantas euphorbiáceas.
\section{Phyllídea}
\begin{itemize}
\item {Grp. gram.:f.}
\end{itemize}
\begin{itemize}
\item {Proveniência:(Do gr. \textunderscore phullon\textunderscore  + \textunderscore eidos\textunderscore )}
\end{itemize}
Gênero de molluscos gasterópodes.
\section{Phyllite}
\begin{itemize}
\item {Grp. gram.:f.}
\end{itemize}
O mesmo que \textunderscore phyllito\textunderscore .
\section{Phyllito}
\begin{itemize}
\item {Grp. gram.:m.}
\end{itemize}
\begin{itemize}
\item {Utilização:Miner.}
\end{itemize}
\begin{itemize}
\item {Proveniência:(Do gr. \textunderscore phullon\textunderscore )}
\end{itemize}
Variedade de chloritóide.
\section{Phyllo}
\begin{itemize}
\item {Grp. gram.:m.}
\end{itemize}
\begin{itemize}
\item {Utilização:Bot.}
\end{itemize}
\begin{itemize}
\item {Proveniência:(Lat. \textunderscore phyllon\textunderscore )}
\end{itemize}
O mesmo que \textunderscore sépala\textunderscore , segundo Linck.
\section{Phyllode}
\begin{itemize}
\item {Grp. gram.:f.}
\end{itemize}
\begin{itemize}
\item {Utilização:Bot.}
\end{itemize}
\begin{itemize}
\item {Proveniência:(Do gr. \textunderscore phullodes\textunderscore )}
\end{itemize}
Pecíolo muito largo, que tomou a apparência de uma fôlha, mas não chegou a formá-la.
\section{Phyllódio}
\begin{itemize}
\item {Grp. gram.:m.}
\end{itemize}
O mesmo que \textunderscore phyllode\textunderscore .
\section{Phyllóide}
\begin{itemize}
\item {Grp. gram.:adj.}
\end{itemize}
\begin{itemize}
\item {Grp. gram.:M.}
\end{itemize}
\begin{itemize}
\item {Proveniência:(Do gr. \textunderscore phullon\textunderscore  + \textunderscore eidos\textunderscore )}
\end{itemize}
Que tem a fórma de uma fôlha.
O mesmo que \textunderscore phyllode\textunderscore .
\section{Phylloidinação}
\begin{itemize}
\item {Grp. gram.:f.}
\end{itemize}
\begin{itemize}
\item {Utilização:Bot.}
\end{itemize}
Transformação lenta das fôlhas em phyllóides.
\section{Phyllolóbeas}
\begin{itemize}
\item {Grp. gram.:f. pl.}
\end{itemize}
\begin{itemize}
\item {Utilização:Bot.}
\end{itemize}
Um dos dois grupos, em que De-Candolle dividiu a fam. das leguminosas.
\section{Phylloma}
\begin{itemize}
\item {Grp. gram.:m.}
\end{itemize}
\begin{itemize}
\item {Utilização:Bot.}
\end{itemize}
\begin{itemize}
\item {Proveniência:(Do gr. \textunderscore phullon\textunderscore )}
\end{itemize}
Conjunto de germes, destinados á producção das fôlhas.
\section{Phyllóphago}
\begin{itemize}
\item {Grp. gram.:adj.}
\end{itemize}
\begin{itemize}
\item {Grp. gram.:M. pl.}
\end{itemize}
\begin{itemize}
\item {Proveniência:(Do gr. \textunderscore phullon\textunderscore  + \textunderscore phagein\textunderscore )}
\end{itemize}
Que se alimenta de fôlhas.
Insectos phyllóphagos.
\section{Phyllópodes}
\begin{itemize}
\item {Grp. gram.:m. pl.}
\end{itemize}
\begin{itemize}
\item {Utilização:Zool.}
\end{itemize}
\begin{itemize}
\item {Proveniência:(Do gr. \textunderscore phullon\textunderscore  + \textunderscore pous\textunderscore , \textunderscore podos\textunderscore )}
\end{itemize}
Ordem de crustáceos, de patas dilatadas, em fórma de lâminas delgadas.
\section{Phylloptosia}
\begin{itemize}
\item {Grp. gram.:f.}
\end{itemize}
\begin{itemize}
\item {Proveniência:(Do gr. \textunderscore phullon\textunderscore  + \textunderscore ptosis\textunderscore )}
\end{itemize}
Moléstia vegetal, caracterizada pela quéda das fôlhas, fóra do tempo próprio.
\section{Phyllosomo}
\begin{itemize}
\item {fónica:sô}
\end{itemize}
\begin{itemize}
\item {Grp. gram.:m.}
\end{itemize}
\begin{itemize}
\item {Proveniência:(Do gr. \textunderscore phullon\textunderscore  + \textunderscore soma\textunderscore )}
\end{itemize}
Larva da lagosta.
\section{Phyllotaxia}
\begin{itemize}
\item {fónica:csi}
\end{itemize}
\begin{itemize}
\item {Grp. gram.:f.}
\end{itemize}
\begin{itemize}
\item {Utilização:Bot.}
\end{itemize}
\begin{itemize}
\item {Proveniência:(Do gr. \textunderscore phullon\textunderscore  + \textunderscore taxis\textunderscore )}
\end{itemize}
Estudo das leis, que presidem á disposição das fôlhas na haste.
\section{Phylloxanthina}
\begin{itemize}
\item {Grp. gram.:f.}
\end{itemize}
\begin{itemize}
\item {Utilização:Chím.}
\end{itemize}
\begin{itemize}
\item {Proveniência:(Do gr. \textunderscore phullon\textunderscore  + \textunderscore xanthos\textunderscore )}
\end{itemize}
Princípio còrante amarelo, que existe na chlorophylla.
\section{Phylloxera}
\begin{itemize}
\item {fónica:cse}
\end{itemize}
\begin{itemize}
\item {Grp. gram.:f.}
\end{itemize}
\begin{itemize}
\item {Proveniência:(Do gr. \textunderscore phullon\textunderscore  + \textunderscore xeros\textunderscore )}
\end{itemize}
Gênero de insectos hemípteros.
Doença das videiras, causada por um insecto dêsse gênero, \textunderscore phylloxera vastatrix\textunderscore .
\section{Phylloxerado}
\begin{itemize}
\item {fónica:cse}
\end{itemize}
\begin{itemize}
\item {Grp. gram.:adj.}
\end{itemize}
Atacado de phylloxera.
\section{Phylloxericida}
\begin{itemize}
\item {Grp. gram.:adj.}
\end{itemize}
\begin{itemize}
\item {Proveniência:(De \textunderscore phylloxera\textunderscore  + lat. \textunderscore caedere\textunderscore )}
\end{itemize}
Que destrói a phylloxera; que se applica contra a phylloxera.
\section{Phylloxérico}
\begin{itemize}
\item {fónica:csé}
\end{itemize}
\begin{itemize}
\item {Grp. gram.:adj.}
\end{itemize}
Relativo á phylloxera.
\section{Phýllula}
\begin{itemize}
\item {Grp. gram.:f.}
\end{itemize}
\begin{itemize}
\item {Utilização:Bot.}
\end{itemize}
\begin{itemize}
\item {Proveniência:(Do gr. \textunderscore phullon\textunderscore )}
\end{itemize}
Cicatriz, que a quéda da fôlha deixa no ramo.
\section{Phyllogenia}
\begin{itemize}
\item {Grp. gram.:f.}
\end{itemize}
\begin{itemize}
\item {Utilização:Biol.}
\end{itemize}
\begin{itemize}
\item {Proveniência:(Do gr. \textunderscore phule\textunderscore , espécie + \textunderscore genos\textunderscore , geração)}
\end{itemize}
Successão genética das espécies orgânicas.
\section{Phyllogênico}
\begin{itemize}
\item {Grp. gram.:adj.}
\end{itemize}
Relativo á phylogenia.
\section{Phyma}
\begin{itemize}
\item {Grp. gram.:m.}
\end{itemize}
\begin{itemize}
\item {Proveniência:(Lat. \textunderscore phyma\textunderscore )}
\end{itemize}
Tumor inflammatório, que se eleva sôbre a pelle.
\section{Phymatina}
\begin{itemize}
\item {Grp. gram.:f.}
\end{itemize}
\begin{itemize}
\item {Utilização:Chím.}
\end{itemize}
\begin{itemize}
\item {Proveniência:(Do gr. \textunderscore phuma\textunderscore )}
\end{itemize}
Substância orgânica, própria dos tubérculos.
\section{Phymatóide}
\begin{itemize}
\item {Grp. gram.:adj.}
\end{itemize}
\begin{itemize}
\item {Utilização:Anat.}
\end{itemize}
\begin{itemize}
\item {Proveniência:(Do gr. \textunderscore phuma\textunderscore  + \textunderscore eidos\textunderscore )}
\end{itemize}
Diz-se do tecido mórbido de côr amarelada.
\section{Phymatose}
\begin{itemize}
\item {Grp. gram.:f.}
\end{itemize}
\begin{itemize}
\item {Utilização:Med.}
\end{itemize}
\begin{itemize}
\item {Proveniência:(Do gr. \textunderscore phuma\textunderscore )}
\end{itemize}
Doença tuberculosa.
\section{Phymatoso}
\begin{itemize}
\item {Grp. gram.:m.  e  adj.}
\end{itemize}
\begin{itemize}
\item {Proveniência:(Do gr. \textunderscore phuma\textunderscore )}
\end{itemize}
O mesmo que \textunderscore tuberculoso\textunderscore .
\section{Phýmico}
\begin{itemize}
\item {Grp. gram.:adj.}
\end{itemize}
\begin{itemize}
\item {Proveniência:(Do gr. \textunderscore phuma\textunderscore )}
\end{itemize}
Relativo a tubérculos ou á tuberculose: \textunderscore o mal phýmico\textunderscore . Cf. \textunderscore Jorn.-do-Comm.\textunderscore , do Rio, de 4-XI-902.
\section{Physália}
\begin{itemize}
\item {Grp. gram.:f.}
\end{itemize}
\begin{itemize}
\item {Proveniência:(Do gr. \textunderscore phusalis\textunderscore )}
\end{itemize}
Animal marinho, que tem o aspecto de uma simples vesícula membranosa, donde pendem longos tentáculos, providos de uma espécie de ventosas, com que o animal fixa os peixes para os devorar.
\section{Physálide}
\begin{itemize}
\item {Grp. gram.:f.}
\end{itemize}
\begin{itemize}
\item {Proveniência:(Do gr. \textunderscore phusalis\textunderscore )}
\end{itemize}
Gênero de plantas solâneas.
\section{Physalina}
\begin{itemize}
\item {Grp. gram.:f.}
\end{itemize}
\begin{itemize}
\item {Proveniência:(De \textunderscore physálide\textunderscore )}
\end{itemize}
Substância amarga, extrahida de uma espécie de physálide e considerada como succedâneo da quina.
\section{Physális}
\begin{itemize}
\item {Grp. gram.:f.}
\end{itemize}
O mesmo que \textunderscore physálide\textunderscore .
\section{Physalite}
\begin{itemize}
\item {Grp. gram.:f.}
\end{itemize}
\begin{itemize}
\item {Proveniência:(Do gr. \textunderscore phusalis\textunderscore )}
\end{itemize}
Variedade de topázio.
\section{Physalito}
\begin{itemize}
\item {Grp. gram.:m.}
\end{itemize}
O mesmo ou melhor que \textunderscore physalite\textunderscore .
\section{Physarmónica}
\begin{itemize}
\item {Grp. gram.:f.}
\end{itemize}
Antigo instrumento, em que se utilizavam lâminas metállicas, postas em vibração por um folle.
\section{Physconia}
\begin{itemize}
\item {Grp. gram.:f.}
\end{itemize}
\begin{itemize}
\item {Utilização:Med.}
\end{itemize}
\begin{itemize}
\item {Proveniência:(Do gr. \textunderscore phuskon\textunderscore )}
\end{itemize}
Tumefacção de uma parte do abdome, sem tympanite nem fluctuação.
\section{Physema}
\begin{itemize}
\item {Grp. gram.:m.}
\end{itemize}
\begin{itemize}
\item {Utilização:Bot.}
\end{itemize}
\begin{itemize}
\item {Proveniência:(Lat. \textunderscore physema\textunderscore )}
\end{itemize}
Parte das algas, também chamada fôlha.
\section{Physetérios}
\begin{itemize}
\item {Grp. gram.:m. pl.}
\end{itemize}
Gênero de baleotes.
\section{Physiantho}
\begin{itemize}
\item {Grp. gram.:m.}
\end{itemize}
\begin{itemize}
\item {Proveniência:(Do gr. \textunderscore phusis\textunderscore  + \textunderscore anthos\textunderscore )}
\end{itemize}
Espécie de trepadeira vivaz.
\section{Phýsica}
\begin{itemize}
\item {Grp. gram.:f.}
\end{itemize}
\begin{itemize}
\item {Utilização:Ant.}
\end{itemize}
\begin{itemize}
\item {Utilização:Des.}
\end{itemize}
\begin{itemize}
\item {Proveniência:(Lat. \textunderscore Physica\textunderscore )}
\end{itemize}
Sciência do movimento e das acções recíprocas dos corpos, considerando êstes sob o ponto de vista da sua composição e decomposição, como na Chímica.
Conhecimento de toda a natureza material.
Medicina.
\section{Physicamente}
\begin{itemize}
\item {Grp. gram.:adv.}
\end{itemize}
De modo phýsico.
Segundo a Phýsica.
De facto, realmente; materialmente: \textunderscore isso é physicamente impossível\textunderscore .
\section{Physicismo}
\begin{itemize}
\item {Grp. gram.:m.}
\end{itemize}
\begin{itemize}
\item {Proveniência:(De \textunderscore phýsico\textunderscore )}
\end{itemize}
Systema dos que explicam o universo pela relação das fôrças phýsicas.
\section{Physicista}
\begin{itemize}
\item {Grp. gram.:m.}
\end{itemize}
\begin{itemize}
\item {Utilização:Neol.}
\end{itemize}
Aquelle que se dedica aos problemas da Phýsica.
Partidário do physicismo.
\section{Phýsico}
\begin{itemize}
\item {Grp. gram.:adj.}
\end{itemize}
\begin{itemize}
\item {Grp. gram.:M.}
\end{itemize}
\begin{itemize}
\item {Utilização:Des.}
\end{itemize}
\begin{itemize}
\item {Proveniência:(Lat. \textunderscore physicus\textunderscore )}
\end{itemize}
Relativo ás condições e leis da natureza.
Material.
Corpóreo.
Natural.
Conjunto das qualidades exteriores do homem.
Aspecto; configuração.
Conjunto das funcções physiológicas.
Aquelle que estuda Phýsica ou é perito nella.
Médico.
\section{Phýsico-chímico}
\begin{itemize}
\item {Grp. gram.:adj.}
\end{itemize}
Relativo á Phýsica e á Chímica, simultaneamente.
\section{Phýsico-mathemático}
\begin{itemize}
\item {Grp. gram.:adj.}
\end{itemize}
Relativo á Phýsica e á Mathemática simultaneamente.
\section{Phýsico-mechânico}
\begin{itemize}
\item {Grp. gram.:adj.}
\end{itemize}
Relativo á Phýsica e á Mecânica simultaneamente.
\section{Phýsico-therapia}
\begin{itemize}
\item {Grp. gram.:f.}
\end{itemize}
O mesmo ou melhor que \textunderscore physiopathia\textunderscore .
\section{Physiocracia}
\begin{itemize}
\item {Grp. gram.:f.}
\end{itemize}
\begin{itemize}
\item {Proveniência:(De \textunderscore physiócrata\textunderscore )}
\end{itemize}
Doutrina dos physiocratas.
\section{Physiocrata}
\begin{itemize}
\item {Grp. gram.:m.}
\end{itemize}
\begin{itemize}
\item {Proveniência:(Do gr. \textunderscore phusis\textunderscore  + \textunderscore kratos\textunderscore )}
\end{itemize}
Economista, que considera as fôrças da natureza, e especialmente as da terra, como fonte principal da riqueza pública.
\section{Physiocrático}
\begin{itemize}
\item {Grp. gram.:adj.}
\end{itemize}
Relativo á physiocracia.
\section{Physiogenia}
\begin{itemize}
\item {Grp. gram.:f.}
\end{itemize}
\begin{itemize}
\item {Proveniência:(Do gr. \textunderscore phusis\textunderscore  + \textunderscore genos\textunderscore )}
\end{itemize}
Desenvolvimento natural do organismo.
\section{Physiognomonia}
\begin{itemize}
\item {Grp. gram.:f.}
\end{itemize}
\begin{itemize}
\item {Proveniência:(Do gr. \textunderscore phusis\textunderscore  + \textunderscore gnomon\textunderscore )}
\end{itemize}
Supposta sciência, que determina as qualidades e inclinações do homem pelas feições do rosto.
\section{Physiognomónico}
\begin{itemize}
\item {Grp. gram.:adj.}
\end{itemize}
Relativo á physiognomonia.
\section{Physiognomonista}
\begin{itemize}
\item {Grp. gram.:m.}
\end{itemize}
Aquelle que se dedica a estudos physiognomónicos, ou que escreve á cêrca da physiognomonia.
\section{Physiographia}
\begin{itemize}
\item {Grp. gram.:f.}
\end{itemize}
\begin{itemize}
\item {Proveniência:(Do gr. \textunderscore phusis\textunderscore  + \textunderscore graphein\textunderscore )}
\end{itemize}
Descripção da natureza ou dos seus productos.
\section{Physiográphico}
\begin{itemize}
\item {Grp. gram.:adj.}
\end{itemize}
Relativo á physiographia.
\section{Physiologia}
\begin{itemize}
\item {Grp. gram.:f.}
\end{itemize}
\begin{itemize}
\item {Proveniência:(Lat. \textunderscore physiologia\textunderscore )}
\end{itemize}
Sciência, que trata das funcções dos órgãos nos seres vivos, vegetaes e animaes.
Tratado de Physiologia.
\section{Physiologicamente}
\begin{itemize}
\item {Grp. gram.:adv.}
\end{itemize}
De modo physiológico; segundo a Physiologia.
\section{Physiológico}
\begin{itemize}
\item {Grp. gram.:adj.}
\end{itemize}
\begin{itemize}
\item {Proveniência:(Lat. \textunderscore physiologicus\textunderscore )}
\end{itemize}
Relativo á Physiologia.
\section{Physiologista}
\begin{itemize}
\item {Grp. gram.:m.  e  f.}
\end{itemize}
Pessôa, que trata de Physiologia.
\section{Physiólogo}
\begin{itemize}
\item {Grp. gram.:m.}
\end{itemize}
\begin{itemize}
\item {Proveniência:(Lat. \textunderscore physiologus\textunderscore )}
\end{itemize}
Aquelle que é versado em Physiologia.
\section{Physionomia}
\begin{itemize}
\item {Grp. gram.:f.}
\end{itemize}
Conjunto das feições do rosto.
Aspecto, ar.
Cara, rosto.
Conjunto de caracteres especiaes.
(Contr. de \textunderscore physiognomonia\textunderscore )
\section{Physionómico}
\begin{itemize}
\item {Grp. gram.:adj.}
\end{itemize}
Relativo á physionomia.
\section{Physionomismo}
\begin{itemize}
\item {Grp. gram.:m.}
\end{itemize}
\begin{itemize}
\item {Proveniência:(De \textunderscore physionomia\textunderscore )}
\end{itemize}
Theoria ou systema dos physionomistas.
\section{Physionomista}
\begin{itemize}
\item {Grp. gram.:m.  e  f.}
\end{itemize}
Pessôa, que conhece a índole de outra pela observação da sua physionomia.
\section{Physiopatha}
\begin{itemize}
\item {Grp. gram.:m.}
\end{itemize}
Aquelle que exerce a physiopathia.
\section{Physiopathia}
\begin{itemize}
\item {Grp. gram.:f.}
\end{itemize}
\begin{itemize}
\item {Proveniência:(Do gr. \textunderscore phusis\textunderscore  + \textunderscore pathos\textunderscore )}
\end{itemize}
Systema therapêutico, que emprega exclusivamente os recursos da natureza.
\section{Physiopáthico}
\begin{itemize}
\item {Grp. gram.:adj.}
\end{itemize}
Relativo á physiopathia.
\section{Physiopsychologista}
\begin{itemize}
\item {Grp. gram.:m.}
\end{itemize}
Aquelle que se occupa de Physiologia e Psychologia. Cf. Th. Braga, \textunderscore Mod. Ideias\textunderscore , I, 284.
\section{Physiotherapia}
\begin{itemize}
\item {Grp. gram.:f.}
\end{itemize}
\begin{itemize}
\item {Proveniência:(Do gr. \textunderscore phusis\textunderscore  + \textunderscore therapeia\textunderscore )}
\end{itemize}
Emprêgo dos agentes naturaes, (água, ar, etc.), como meios therapêuticos.
\section{Physocarpo}
\begin{itemize}
\item {Grp. gram.:adj.}
\end{itemize}
\begin{itemize}
\item {Utilização:Bot.}
\end{itemize}
\begin{itemize}
\item {Proveniência:(Do gr. \textunderscore phusa\textunderscore  + \textunderscore karpos\textunderscore )}
\end{itemize}
Que tem frutos vesiculosos.
\section{Physocele}
\begin{itemize}
\item {Grp. gram.:f.}
\end{itemize}
\begin{itemize}
\item {Proveniência:(Do gr. \textunderscore phusa\textunderscore  + \textunderscore kele\textunderscore )}
\end{itemize}
Hérnia intestinal, distendida pelos gases até o escroto.
\section{Physóide}
\begin{itemize}
\item {Grp. gram.:adj.}
\end{itemize}
\begin{itemize}
\item {Proveniência:(Do gr. \textunderscore phusa\textunderscore  + \textunderscore eidos\textunderscore )}
\end{itemize}
Que tem fórma de bexiga.
\section{Physometria}
\begin{itemize}
\item {Grp. gram.:f.}
\end{itemize}
\begin{itemize}
\item {Utilização:Med.}
\end{itemize}
\begin{itemize}
\item {Proveniência:(Do gr. \textunderscore phusa\textunderscore  + \textunderscore metra\textunderscore )}
\end{itemize}
Distensão do útero, causada por gazes.
\section{Physóphoros}
\begin{itemize}
\item {Grp. gram.:m. pl.}
\end{itemize}
\begin{itemize}
\item {Utilização:Zool.}
\end{itemize}
\begin{itemize}
\item {Proveniência:(Do gr. \textunderscore phusa\textunderscore  + \textunderscore phoros\textunderscore )}
\end{itemize}
Celenterados, com bôlsas cheias de ar, que lhes permittem fluctuar na água.
\section{Physospermo}
\begin{itemize}
\item {Grp. gram.:m.}
\end{itemize}
\begin{itemize}
\item {Proveniência:(Do gr. \textunderscore phusa\textunderscore  + \textunderscore sperma\textunderscore )}
\end{itemize}
Gênero de plantas umbellíferas.
\section{Physostigma}
\begin{itemize}
\item {Grp. gram.:f.}
\end{itemize}
Espécie de fava medicinal.
\section{Physostigmina}
\begin{itemize}
\item {Grp. gram.:f.}
\end{itemize}
Alcalóide da physostigma, usado em therapêutica ocular e mais conhecido por \textunderscore iserina\textunderscore .
\section{Physothórax}
\begin{itemize}
\item {Grp. gram.:m.}
\end{itemize}
\begin{itemize}
\item {Utilização:Med.}
\end{itemize}
\begin{itemize}
\item {Proveniência:(Do gr. \textunderscore phusa\textunderscore  + \textunderscore thorax\textunderscore )}
\end{itemize}
Accumulações de gases na cavidade da pleura.
\section{Phytina}
\begin{itemize}
\item {Grp. gram.:f.}
\end{itemize}
\begin{itemize}
\item {Proveniência:(Do gr. \textunderscore phuton\textunderscore )}
\end{itemize}
Combinação phosphórica, extrahida de sementes e que se mistura com o alimento das crianças.
\section{Phyto...}
\begin{itemize}
\item {Grp. gram.:pref.}
\end{itemize}
\begin{itemize}
\item {Proveniência:(Do gr. \textunderscore phuton\textunderscore )}
\end{itemize}
(designativo de \textunderscore vegetal\textunderscore )
\section{Phytochímica}
\begin{itemize}
\item {fónica:qui}
\end{itemize}
\begin{itemize}
\item {Grp. gram.:f.}
\end{itemize}
\begin{itemize}
\item {Utilização:P. us.}
\end{itemize}
\begin{itemize}
\item {Proveniência:(De \textunderscore phyto...\textunderscore  + \textunderscore Chímica\textunderscore )}
\end{itemize}
Chímica vegetal.
\section{Phytochímico}
\begin{itemize}
\item {fónica:qui}
\end{itemize}
\begin{itemize}
\item {Grp. gram.:adj.}
\end{itemize}
Relativo á Phytochímica.
\section{Phytogêneo}
\begin{itemize}
\item {Grp. gram.:adj.}
\end{itemize}
\begin{itemize}
\item {Proveniência:(Do gr. \textunderscore phuton\textunderscore  + \textunderscore genes\textunderscore )}
\end{itemize}
Produzido por vegetaes.
\section{Phytogênese}
\begin{itemize}
\item {Grp. gram.:f.}
\end{itemize}
O mesmo que \textunderscore phytogenia\textunderscore .
\section{Phytogenia}
\begin{itemize}
\item {Grp. gram.:f.}
\end{itemize}
Designação scientífica da vegetação ou da producção vegetal.
(Cp. \textunderscore phytogêneo\textunderscore )
\section{Phytogênico}
\begin{itemize}
\item {Grp. gram.:adj.}
\end{itemize}
Relativo á phytogenia.
\section{Phytogeographia}
\begin{itemize}
\item {Grp. gram.:f.}
\end{itemize}
\begin{itemize}
\item {Proveniência:(De \textunderscore phyto...\textunderscore  + \textunderscore geographia\textunderscore )}
\end{itemize}
Descripção da distribuição de plantas no globo.
\section{Phytogeográphico}
\begin{itemize}
\item {Grp. gram.:adj.}
\end{itemize}
Relativo á phytogeographia.
\section{Phytognomia}
\begin{itemize}
\item {Grp. gram.:f.}
\end{itemize}
\begin{itemize}
\item {Utilização:Bot.}
\end{itemize}
\begin{itemize}
\item {Proveniência:(Do gr. \textunderscore phuton\textunderscore  + \textunderscore gnomon\textunderscore )}
\end{itemize}
Conhecimento das partes, que constituem os vegetaes.
\section{Phytognomónica}
\begin{itemize}
\item {Grp. gram.:f.}
\end{itemize}
\begin{itemize}
\item {Proveniência:(Do gr. \textunderscore phuton\textunderscore  + \textunderscore gnomon\textunderscore )}
\end{itemize}
Nome, dado por Porta ao systema de determinar a applicação medicinal das plantas pela sua conformação ou coloração.
\section{Phytographia}
\begin{itemize}
\item {Grp. gram.:f.}
\end{itemize}
\begin{itemize}
\item {Utilização:Bot.}
\end{itemize}
Descripção methódica e natural dos differentes typos vegetaes, sob o ponto de vista da sua classificação.
(Cp. \textunderscore phytógrapho\textunderscore )
\section{Phytográphico}
\begin{itemize}
\item {Grp. gram.:adj.}
\end{itemize}
Relativo á phytographia.
\section{Phytógrapho}
\begin{itemize}
\item {Grp. gram.:m.}
\end{itemize}
\begin{itemize}
\item {Proveniência:(Do gr. \textunderscore phuton\textunderscore  + \textunderscore graphein\textunderscore )}
\end{itemize}
Aquelle que se dedica á phytographia.
\section{Phytóide}
\begin{itemize}
\item {Grp. gram.:adj.}
\end{itemize}
\begin{itemize}
\item {Proveniência:(Do gr. \textunderscore phuton\textunderscore  + \textunderscore eidos\textunderscore )}
\end{itemize}
Relativo ou semelhante a planta.
\section{Phytolaca}
\begin{itemize}
\item {Grp. gram.:f.}
\end{itemize}
\begin{itemize}
\item {Proveniência:(De \textunderscore phuton\textunderscore  gr. + \textunderscore laca\textunderscore )}
\end{itemize}
Gênero de plantas tinctórias das regiões quentes.
\section{Phytolaceáceas}
\begin{itemize}
\item {Grp. gram.:f. pl.}
\end{itemize}
O mesmo que \textunderscore phytoláceas\textunderscore .
\section{Phytoláceas}
\begin{itemize}
\item {Grp. gram.:f. pl.}
\end{itemize}
Tríbo de plantas, que tem por typo a phytolaca.
(Fem. pl. de \textunderscore phytolaceo\textunderscore )
\section{Phytoláceo}
\begin{itemize}
\item {Grp. gram.:adj.}
\end{itemize}
Relativo ou semelhante á phytolaca.
\section{Phytólitho}
\begin{itemize}
\item {Grp. gram.:m.}
\end{itemize}
\begin{itemize}
\item {Proveniência:(Do gr. \textunderscore phuton\textunderscore  + \textunderscore lithos\textunderscore )}
\end{itemize}
Vegetal fóssil.
Pedra, que apresenta o vestígio de uma planta.
Concreção pedregosa, que se encontra em algumas plantas, como nos bambus.
\section{Phytologia}
\begin{itemize}
\item {Grp. gram.:f.}
\end{itemize}
\begin{itemize}
\item {Proveniência:(Do gr. \textunderscore phuton\textunderscore  + \textunderscore logos\textunderscore )}
\end{itemize}
Tratado ou classificação das plantas; Botânica.
\section{Phytológico}
\begin{itemize}
\item {Grp. gram.:adj.}
\end{itemize}
Relativo á phytologia.
\section{Phytonícia}
\begin{itemize}
\item {Grp. gram.:f.}
\end{itemize}
Uma das suppostas artes de adivinhar, usadas pelos antigos.
\section{Phytonomia}
\begin{itemize}
\item {Grp. gram.:f.}
\end{itemize}
\begin{itemize}
\item {Proveniência:(Do gr. \textunderscore phuton\textunderscore  + \textunderscore nomos\textunderscore )}
\end{itemize}
Parte da Botânica, que trata das leis da vegetação.
\section{Phytonómico}
\begin{itemize}
\item {Grp. gram.:adj.}
\end{itemize}
Relativo a phytonomia.
\section{Phytonose}
\begin{itemize}
\item {Grp. gram.:f.}
\end{itemize}
\begin{itemize}
\item {Proveniência:(Do gr. \textunderscore phuton\textunderscore  + \textunderscore nosos\textunderscore )}
\end{itemize}
Nome genérico das doenças dos vegetaes.
\section{Phytonymia}
\begin{itemize}
\item {Grp. gram.:f.}
\end{itemize}
Qualidade de phytónymo.
Nomenclatura vegetal.
\section{Phytónymo}
\begin{itemize}
\item {Grp. gram.:adj.}
\end{itemize}
\begin{itemize}
\item {Utilização:Neol.}
\end{itemize}
\begin{itemize}
\item {Proveniência:(Do gr. \textunderscore phuton\textunderscore  + \textunderscore onuma\textunderscore )}
\end{itemize}
Diz-se do indivíduo, cujo nome ou appellido é tirado de uma planta: \textunderscore Carvalho\textunderscore , \textunderscore Figueira\textunderscore , \textunderscore Oliveira\textunderscore , \textunderscore Pinheiro\textunderscore , etc.
\section{Phytophagia}
\begin{itemize}
\item {Grp. gram.:f.}
\end{itemize}
Qualidade de phytóphago.
\section{Phytóphago}
\begin{itemize}
\item {Grp. gram.:adj.}
\end{itemize}
\begin{itemize}
\item {Proveniência:(Do gr. \textunderscore phuton\textunderscore  + \textunderscore phagein\textunderscore )}
\end{itemize}
Que se alimenta de vegetaes.
\section{Phytotechnia}
\begin{itemize}
\item {Grp. gram.:f.}
\end{itemize}
\begin{itemize}
\item {Proveniência:(Do gr. \textunderscore phuton\textunderscore  + \textunderscore tekne\textunderscore )}
\end{itemize}
Parte da Botânica, que tem por objecto a classificação e nomenclatura das plantas, bem como a utilidade que dellas se póde auferir.
\section{Phytotéchnico}
\begin{itemize}
\item {Grp. gram.:adj.}
\end{itemize}
Relativo á phytotechnia.
\section{Phytoterosia}
\begin{itemize}
\item {Grp. gram.:f.}
\end{itemize}
Parte da Botânica, que trata das alterações mórbidas dos vegetaes.
Pathologia vegetal.
\section{Phytotomia}
\begin{itemize}
\item {Grp. gram.:f.}
\end{itemize}
\begin{itemize}
\item {Proveniência:(Do gr. \textunderscore phuton\textunderscore  + \textunderscore tome\textunderscore )}
\end{itemize}
Anatomia vegetal.
\section{Phytotómico}
\begin{itemize}
\item {Grp. gram.:adj.}
\end{itemize}
Relativo á phytotomia.
\section{Phytotypólitho}
\begin{itemize}
\item {Grp. gram.:m.}
\end{itemize}
\begin{itemize}
\item {Proveniência:(Do gr. \textunderscore phuton\textunderscore  + \textunderscore tupos\textunderscore  + \textunderscore lithos\textunderscore )}
\end{itemize}
Substância mineral, que contém o vestígio de um vegetal.
\section{Phytózoário}
\begin{itemize}
\item {Grp. gram.:adj.}
\end{itemize}
\begin{itemize}
\item {Grp. gram.:M. pl.}
\end{itemize}
\begin{itemize}
\item {Utilização:Zool.}
\end{itemize}
\begin{itemize}
\item {Proveniência:(Do gr. \textunderscore phuton\textunderscore  + \textunderscore zoon\textunderscore )}
\end{itemize}
Diz-se dos seres, que se suppõem intermediários ás plantas e aos animaes.
Seres phytozoários.
Animaes, que tem configuração radiada e que, em geral, formam colónias arborescentes.
\section{Pia}
\begin{itemize}
\item {Grp. gram.:f.}
\end{itemize}
\begin{itemize}
\item {Utilização:T. de Lisbôa}
\end{itemize}
\begin{itemize}
\item {Utilização:T. de Alcanena}
\end{itemize}
\begin{itemize}
\item {Utilização:Prov.}
\end{itemize}
\begin{itemize}
\item {Proveniência:(Do lat. \textunderscore pila\textunderscore )}
\end{itemize}
Vaso de pedra, para líquidos.
Carlinga.
Abertura circular, com a configuração de um vaso fixo, em communicação com a canalização geral, e pela qual, nos domicílios, se faz o despejo de immundícies.
Cisterna ou cavidade, natural ou artificial, em que se guarda água no verão, para abastecimento dos povoados.
Sepultura, cavada em rocha.
\section{Piá}
\begin{itemize}
\item {Grp. gram.:m.}
\end{itemize}
\begin{itemize}
\item {Utilização:Bras}
\end{itemize}
Menino; filho de caboclo; rapaz.
\section{Piaba}
\begin{itemize}
\item {Grp. gram.:f.}
\end{itemize}
\begin{itemize}
\item {Utilização:Bras}
\end{itemize}
Espécie de peixe de água doce.
\section{Piabanha}
\begin{itemize}
\item {Grp. gram.:f.}
\end{itemize}
\begin{itemize}
\item {Utilização:Bras}
\end{itemize}
Peixe fluvial.
\section{Piabuco}
\begin{itemize}
\item {Grp. gram.:m.}
\end{itemize}
\begin{itemize}
\item {Utilização:Bras}
\end{itemize}
Espécie de salmão da América do Sul.
\section{Piaca}
\begin{itemize}
\item {Grp. gram.:f.}
\end{itemize}
Árvore leguminosa do Brasil.
\section{Piaçá}
\begin{itemize}
\item {Grp. gram.:m.}
\end{itemize}
Espécie de palmeira do Brasil.
Variedade de junco consistente, de que se fazem vassoiras.
Vassoira dêsse junco.
(Do tupi)
\section{Piaçaba}
\begin{itemize}
\item {Grp. gram.:m.}
\end{itemize}
Espécie de palmeira do Brasil.
Variedade de junco consistente, de que se fazem vassoiras.
Vassoira dêsse junco.
(Do tupi)
\section{Piaçoca}
\begin{itemize}
\item {Grp. gram.:f.}
\end{itemize}
\begin{itemize}
\item {Utilização:Bras}
\end{itemize}
Ave ribeirinha da região do Purus.
\section{Piacular}
\begin{itemize}
\item {Grp. gram.:adj.}
\end{itemize}
\begin{itemize}
\item {Utilização:Ant.}
\end{itemize}
\begin{itemize}
\item {Proveniência:(Lat. \textunderscore piacularis\textunderscore )}
\end{itemize}
Expiatório.
\section{Piáculo}
\begin{itemize}
\item {Grp. gram.:m.}
\end{itemize}
\begin{itemize}
\item {Utilização:Ant.}
\end{itemize}
\begin{itemize}
\item {Proveniência:(Lat. \textunderscore piaculum\textunderscore )}
\end{itemize}
Sacrifício expiatório.
Delito.
\section{Piada}
\begin{itemize}
\item {Grp. gram.:f.}
\end{itemize}
\begin{itemize}
\item {Utilização:Pop.}
\end{itemize}
O mesmo que \textunderscore piado\textunderscore .
Chalaça picante; picuínha.
(Fem. de \textunderscore piado\textunderscore )
\section{Piada}
\begin{itemize}
\item {Grp. gram.:f.}
\end{itemize}
\begin{itemize}
\item {Utilização:Prov.}
\end{itemize}
\begin{itemize}
\item {Utilização:trasm.}
\end{itemize}
\begin{itemize}
\item {Proveniência:(De \textunderscore pia\textunderscore )}
\end{itemize}
Porção (de azeitona) que entra por cada vez na vasa.
\section{Piadeira}
\begin{itemize}
\item {Grp. gram.:f.}
\end{itemize}
\begin{itemize}
\item {Utilização:Pop.}
\end{itemize}
\begin{itemize}
\item {Proveniência:(De \textunderscore piar\textunderscore )}
\end{itemize}
Pàpaformigas.
Assobiadeira.
Ave ribeirinha, (\textunderscore mareca penelope\textunderscore , Lin.).
O mesmo que \textunderscore pieira\textunderscore .
\section{Piadeiro}
\begin{itemize}
\item {Grp. gram.:m.}
\end{itemize}
Ave, também conhecida por \textunderscore pàpaformigas\textunderscore , \textunderscore piadeira\textunderscore , \textunderscore pêto-da-chuva\textunderscore  e \textunderscore torcicollo\textunderscore , (\textunderscore yunx torquilla\textunderscore , Lin.).
\section{Piadeiro}
\begin{itemize}
\item {Grp. gram.:m.}
\end{itemize}
\begin{itemize}
\item {Utilização:Prov.}
\end{itemize}
\begin{itemize}
\item {Utilização:trasm.}
\end{itemize}
\begin{itemize}
\item {Proveniência:(De \textunderscore piar\textunderscore ^3)}
\end{itemize}
Apparelho, para descascar milho.
\section{Piadinha}
\begin{itemize}
\item {Grp. gram.:f.}
\end{itemize}
\begin{itemize}
\item {Utilização:Pop.}
\end{itemize}
Pequena piada, picuínha.
\section{Piadista}
\begin{itemize}
\item {Grp. gram.:m. ,  f.  e  adj.}
\end{itemize}
\begin{itemize}
\item {Utilização:Pop.}
\end{itemize}
Pessôa, que diz piadas.
\section{Piado}
\begin{itemize}
\item {Grp. gram.:m.}
\end{itemize}
\begin{itemize}
\item {Proveniência:(De \textunderscore piar\textunderscore ^1)}
\end{itemize}
Pio; pieira.
\section{Piadoiro}
\begin{itemize}
\item {Grp. gram.:m.}
\end{itemize}
\begin{itemize}
\item {Utilização:Gír.}
\end{itemize}
\begin{itemize}
\item {Proveniência:(De \textunderscore piar\textunderscore ^2)}
\end{itemize}
Cálix da igreja.
\section{Piadouro}
\begin{itemize}
\item {Grp. gram.:m.}
\end{itemize}
\begin{itemize}
\item {Utilização:Gír.}
\end{itemize}
\begin{itemize}
\item {Proveniência:(De \textunderscore piar\textunderscore ^2)}
\end{itemize}
Cálix da igreja.
\section{Piafé}
\begin{itemize}
\item {Grp. gram.:m.}
\end{itemize}
\begin{itemize}
\item {Proveniência:(Do fr. \textunderscore piaffer\textunderscore )}
\end{itemize}
Movimento de cavallo, que bate no chão com os pés e com as mãos, sem andar.
\section{Piaga}
\begin{itemize}
\item {Grp. gram.:m.}
\end{itemize}
(T., erradamente admittido por escritores brasileiros, como Gonçalves Dias, illudidos por um êrro typográphico, com que se compôs a palavra \textunderscore pagé\textunderscore . Cf. Gonç. Dias, \textunderscore Poésias\textunderscore , 271 e 293.)
(Cp. \textunderscore pagé\textunderscore )
\section{Pialador}
\begin{itemize}
\item {Grp. gram.:m.}
\end{itemize}
\begin{itemize}
\item {Utilização:Bras. do S}
\end{itemize}
Aquelle que piala.
\section{Pialar}
\begin{itemize}
\item {Grp. gram.:v. t.}
\end{itemize}
\begin{itemize}
\item {Utilização:Bras. do S}
\end{itemize}
Pear pelas mãos e fazer cair (um animal que foge).
(Relaciona-se com \textunderscore pear\textunderscore )
\section{Pialo}
\begin{itemize}
\item {Grp. gram.:m.}
\end{itemize}
\begin{itemize}
\item {Utilização:Bras. do S}
\end{itemize}
Acto de pialar.
\section{Pia-máter}
\begin{itemize}
\item {Grp. gram.:f.}
\end{itemize}
\begin{itemize}
\item {Utilização:Anat.}
\end{itemize}
\begin{itemize}
\item {Proveniência:(Do lat. \textunderscore pius\textunderscore  + \textunderscore mater\textunderscore )}
\end{itemize}
A membrana mais interna das que envolvem o apparelho cérebro-espinal.
\section{Piambre}
\begin{itemize}
\item {Grp. gram.:m.}
\end{itemize}
\begin{itemize}
\item {Utilização:Ant.}
\end{itemize}
\begin{itemize}
\item {Proveniência:(T. as.)}
\end{itemize}
O mesmo que \textunderscore tribuna\textunderscore .
Espécie de andas ou liteira, usada dantes na China. Cf. \textunderscore Peregrinação\textunderscore , CXXI.
\section{Piamente}
\begin{itemize}
\item {Grp. gram.:adv.}
\end{itemize}
De modo pio; com piedade; devotamente.
\section{Pia-milhos}
\begin{itemize}
\item {Grp. gram.:m.}
\end{itemize}
\begin{itemize}
\item {Utilização:Prov.}
\end{itemize}
\begin{itemize}
\item {Utilização:trasm.}
\end{itemize}
Aquelle que trabalha com o piadeiro^2.
\section{Piampara}
\begin{itemize}
\item {Grp. gram.:f.}
\end{itemize}
\begin{itemize}
\item {Utilização:Bras}
\end{itemize}
Peixe fluvial.
\section{Pian}
\begin{itemize}
\item {Grp. gram.:m.}
\end{itemize}
\begin{itemize}
\item {Utilização:Bras}
\end{itemize}
Tumor, o mesmo que \textunderscore bubão\textunderscore .
\section{Piançar}
\begin{itemize}
\item {Grp. gram.:v. t.}
\end{itemize}
\begin{itemize}
\item {Utilização:bras}
\end{itemize}
\begin{itemize}
\item {Utilização:Neol.}
\end{itemize}
Desejar ardentemente. Cf. Alencar, \textunderscore Minas de Prata\textunderscore , II, 425.
\section{Pianinho}
\begin{itemize}
\item {Grp. gram.:m.}
\end{itemize}
\begin{itemize}
\item {Utilização:Gír.}
\end{itemize}
\begin{itemize}
\item {Proveniência:(De \textunderscore piano\textunderscore )}
\end{itemize}
Guitarra.
\section{Pianíssimo}
\begin{itemize}
\item {Grp. gram.:adv.}
\end{itemize}
\begin{itemize}
\item {Proveniência:(T. it.)}
\end{itemize}
Suavemente, brandamente, (tratando-se de música).
\section{Pianista}
\begin{itemize}
\item {Grp. gram.:m.  e  f.}
\end{itemize}
Pessôa, que sabe tocar piano ou que toca piano.
\section{Pianístico}
\begin{itemize}
\item {Grp. gram.:adj.}
\end{itemize}
Relativo a piano ou a pianista.
\section{Piano}
\begin{itemize}
\item {Grp. gram.:m.}
\end{itemize}
\begin{itemize}
\item {Proveniência:(It. \textunderscore piano\textunderscore )}
\end{itemize}
Instrumento musical, em que as notas são dadas por percussão num teclado, que faz vibrar um systema de cordas dentro de uma caixa sonora.
\section{Piano}
\begin{itemize}
\item {Grp. gram.:adv.}
\end{itemize}
\begin{itemize}
\item {Proveniência:(T. it.)}
\end{itemize}
Com pouca fôrça, pausadamente, (tratando-se de música).
\section{Pianoforte}
\begin{itemize}
\item {Grp. gram.:m.}
\end{itemize}
Designação primitiva do piano, ainda hoje conservada no italiano.
\section{Pianógrafo}
\begin{itemize}
\item {Grp. gram.:m.}
\end{itemize}
O mesmo que \textunderscore melógrapho\textunderscore .
\section{Pianógrapho}
\begin{itemize}
\item {Grp. gram.:m.}
\end{itemize}
O mesmo que \textunderscore melógrapho\textunderscore .
\section{Pianola}
\begin{itemize}
\item {Grp. gram.:f.}
\end{itemize}
Máquina, que se applica aos pianos, para os fazer tocar automaticamente.
\section{Pião}
\begin{itemize}
\item {Grp. gram.:m.}
\end{itemize}
\begin{itemize}
\item {Utilização:Prov.}
\end{itemize}
\begin{itemize}
\item {Utilização:Prov.}
\end{itemize}
\begin{itemize}
\item {Utilização:alg.}
\end{itemize}
Peça de metal ou madeira, em fórma proximamente cónica, com um ferrão na ponta, e os rapazes jogam, enrolando-lhe e desenrolando-lhe uma guita.
Frade de pedra.
Eixo do moínho de vento, sôbre o qual giram as velas.
Flanco, em que giram tropas.
Pinhão.
O mesmo que \textunderscore peão\textunderscore ^1.
(Cp. \textunderscore peão\textunderscore ^1)
\section{Pião}
\begin{itemize}
\item {Grp. gram.:m.}
\end{itemize}
O mesmo que \textunderscore peão\textunderscore ^2.
\section{Pião}
\begin{itemize}
\item {Grp. gram.:m.}
\end{itemize}
\begin{itemize}
\item {Utilização:T. de Turquel}
\end{itemize}
\begin{itemize}
\item {Proveniência:(De \textunderscore pia\textunderscore )}
\end{itemize}
Grande cavidade natural ou artificial, aberta em rocha.
\section{Pia-pia}
\begin{itemize}
\item {Grp. gram.:f.}
\end{itemize}
\begin{itemize}
\item {Proveniência:(De \textunderscore piar\textunderscore )}
\end{itemize}
Nome de alguns pássaros fissirostros de Angola.
\section{Piar}
\begin{itemize}
\item {Grp. gram.:v. i.}
\end{itemize}
\begin{itemize}
\item {Proveniência:(Lat. \textunderscore pipiare\textunderscore )}
\end{itemize}
Dar pios.
\section{Piar}
\begin{itemize}
\item {Grp. gram.:v. i.}
\end{itemize}
\begin{itemize}
\item {Utilização:Gír.}
\end{itemize}
\begin{itemize}
\item {Grp. gram.:V. t.}
\end{itemize}
\begin{itemize}
\item {Utilização:Gír.}
\end{itemize}
Beber vinho.
Beber (qualquer líquido alcoólico).
(Do caló, \textunderscore piyar\textunderscore , beber)
\section{Piar}
\begin{itemize}
\item {Grp. gram.:v. t.}
\end{itemize}
\begin{itemize}
\item {Utilização:Prov.}
\end{itemize}
\begin{itemize}
\item {Utilização:trasm.}
\end{itemize}
\begin{itemize}
\item {Proveniência:(Do lat. \textunderscore pilare\textunderscore )}
\end{itemize}
Descascar (milho).
\section{Piar}
\begin{itemize}
\item {Grp. gram.:m.}
\end{itemize}
\begin{itemize}
\item {Utilização:Ant.}
\end{itemize}
O mesmo que \textunderscore pilar\textunderscore ^2.
\section{Piara}
\begin{itemize}
\item {Grp. gram.:f.}
\end{itemize}
\begin{itemize}
\item {Utilização:Prov.}
\end{itemize}
\begin{itemize}
\item {Utilização:trasm.}
\end{itemize}
Bando de animaes; multidão de gente.
Agrupamento de animaes, da mesma idade e tamanho.
(Cast. \textunderscore piara\textunderscore )
\section{Piarda}
\begin{itemize}
\item {Grp. gram.:f.}
\end{itemize}
Pequeno peixe dos rios do norte do país.
\section{Piaremia}
\begin{itemize}
\item {fónica:re}
\end{itemize}
\begin{itemize}
\item {Grp. gram.:f.}
\end{itemize}
\begin{itemize}
\item {Utilização:Med.}
\end{itemize}
\begin{itemize}
\item {Proveniência:(Do gr. \textunderscore piar\textunderscore , gordura + \textunderscore haima\textunderscore , sangue)}
\end{itemize}
Estado do sangue, em que êste apresenta côr opalina, lactescente, devida á gordura em emulsão.
\section{Piarremia}
\begin{itemize}
\item {Grp. gram.:f.}
\end{itemize}
\begin{itemize}
\item {Utilização:Med.}
\end{itemize}
\begin{itemize}
\item {Proveniência:(Do gr. \textunderscore piar\textunderscore , gordura + \textunderscore haima\textunderscore , sangue)}
\end{itemize}
Estado do sangue, em que êste apresenta côr opalina, lactescente, devida á gordura em emulsão.
\section{Piasca}
\begin{itemize}
\item {Grp. gram.:f.}
\end{itemize}
\begin{itemize}
\item {Utilização:Prov.}
\end{itemize}
Pequeno pião; piorra.
\section{Piassá}
\begin{itemize}
\item {Grp. gram.:m.}
\end{itemize}
(V.piaçá)
\section{Piassaba}
\begin{itemize}
\item {Grp. gram.:m.}
\end{itemize}
(V.piaçá)
\section{Piassava}
\begin{itemize}
\item {Grp. gram.:m.}
\end{itemize}
(V.piaçaba)
\section{Piastra}
\begin{itemize}
\item {Grp. gram.:f.}
\end{itemize}
\begin{itemize}
\item {Proveniência:(It. \textunderscore piastra\textunderscore )}
\end{itemize}
Moéda de prata, adoptada em vários países, com valores diversos.
\section{Piastrão}
\begin{itemize}
\item {Grp. gram.:m.}
\end{itemize}
\begin{itemize}
\item {Proveniência:(It. \textunderscore piastrone\textunderscore )}
\end{itemize}
Parte deanteira da coiraça.
\section{Piau}
\begin{itemize}
\item {Grp. gram.:m.}
\end{itemize}
\begin{itemize}
\item {Utilização:Bras}
\end{itemize}
Árvore de Moçambique.
Peixe fluvial.
\section{Piauiense}
\begin{itemize}
\item {fónica:au-i}
\end{itemize}
\begin{itemize}
\item {Grp. gram.:adj.}
\end{itemize}
\begin{itemize}
\item {Utilização:Bras}
\end{itemize}
\begin{itemize}
\item {Grp. gram.:M.  e  f.}
\end{itemize}
Relativo á antiga província de Piaui.
Habitante dessa província.
\section{Pica}
\begin{itemize}
\item {Grp. gram.:f.}
\end{itemize}
\begin{itemize}
\item {Utilização:Ant.}
\end{itemize}
Pique.
Cada uma das peças delgadas, que entram na construcção da prôa e da pôpa do navio.
(Cp. \textunderscore pico\textunderscore ^1)
\section{Pica}
\begin{itemize}
\item {Grp. gram.:f.}
\end{itemize}
\begin{itemize}
\item {Utilização:Med.}
\end{itemize}
\begin{itemize}
\item {Proveniência:(Fr. \textunderscore pica\textunderscore )}
\end{itemize}
Uma das doenças bradytróphicas.
\section{Pica}
\begin{itemize}
\item {Grp. gram.:f.}
\end{itemize}
\begin{itemize}
\item {Utilização:Prov.}
\end{itemize}
\begin{itemize}
\item {Utilização:trasm.}
\end{itemize}
Acto de picar ou sachar o milho.
Sacho, que se emprega nesse serviço.
\section{Pica}
\begin{itemize}
\item {Grp. gram.:f.}
\end{itemize}
Peixe de Portugal.
O mesmo que \textunderscore picha\textunderscore ^2.
\section{Picabeca}
\begin{itemize}
\item {Grp. gram.:f.}
\end{itemize}
\begin{itemize}
\item {Utilização:Prov.}
\end{itemize}
\begin{itemize}
\item {Utilização:dur.}
\end{itemize}
O mesmo que \textunderscore alvião\textunderscore .
\section{Picaburro}
\begin{itemize}
\item {Grp. gram.:m.}
\end{itemize}
\begin{itemize}
\item {Utilização:Mad}
\end{itemize}
O mesmo que \textunderscore tinge-burro\textunderscore .
\section{Picaço}
\begin{itemize}
\item {Grp. gram.:adj.}
\end{itemize}
\begin{itemize}
\item {Utilização:Bras}
\end{itemize}
Diz-se do cavallo escuro com pés brancos.
(Corr. de \textunderscore pigarço\textunderscore )
\section{Picacuroba}
\begin{itemize}
\item {Grp. gram.:f.}
\end{itemize}
Espécie de rôla do Brasil.
\section{Picada}
\begin{itemize}
\item {Grp. gram.:f.}
\end{itemize}
\begin{itemize}
\item {Utilização:Pop.}
\end{itemize}
\begin{itemize}
\item {Utilização:Fig.}
\end{itemize}
Acto ou effeito de picar.
Ferida, feita com objecto aguçado.
Mordedura de insecto.
Bicada.
Facada.
Contrariedade, desgôsto.
Caminho estreito ou atalho, em linha recta, através do mato.
Pequena porção de carne, que os tratadores de falcões para caça de altanaria davam a essas aves, para que ellas se familiarizassem com êlles. Cf. Fernandes, \textunderscore Caça de Altan.\textunderscore 
\section{Picada}
\begin{itemize}
\item {Grp. gram.:f.}
\end{itemize}
\begin{itemize}
\item {Proveniência:(De \textunderscore pico\textunderscore ^1)}
\end{itemize}
Cume do monte, pico. Cf. Ol. Martins, \textunderscore Camões\textunderscore , 312.
\section{Pica-de-el-rei}
\begin{itemize}
\item {Grp. gram.:f.}
\end{itemize}
Pequeno peixe vermelho.
\section{Picadeira}
\begin{itemize}
\item {Grp. gram.:f.}
\end{itemize}
\begin{itemize}
\item {Utilização:Prov.}
\end{itemize}
\begin{itemize}
\item {Utilização:alent.}
\end{itemize}
\begin{itemize}
\item {Proveniência:(De \textunderscore picar\textunderscore )}
\end{itemize}
Ferro, com que se picam as mós.
Picareta.
Pequeno martelo de pedreiro, com gume.
Espécie de pequeno chocalho.
\section{Picadeiro}
\begin{itemize}
\item {Grp. gram.:m.}
\end{itemize}
\begin{itemize}
\item {Utilização:Ant.}
\end{itemize}
\begin{itemize}
\item {Utilização:T. de San-Miguel}
\end{itemize}
\begin{itemize}
\item {Proveniência:(De \textunderscore picar\textunderscore )}
\end{itemize}
Lugar, onde se adestram cavallos ou se fazem exercícios de equitação.
Cada uma das peças, em que assenta a quilha do navio que se está construíndo.
Peça na extremidade dos bancos de carpinteiro, na qual se entala a tábua em que se trabalha.
Cepo, sobre que os tanoeiros encurvam as aduelas.
Picador, cavaleiro.
Estância de madeira, (porque a primeira se abriu onde havía um picadeiro).
\section{Picadela}
\begin{itemize}
\item {Grp. gram.:f.}
\end{itemize}
O mesmo que \textunderscore picada\textunderscore ^1.
\section{Picadete}
\begin{itemize}
\item {fónica:dê}
\end{itemize}
\begin{itemize}
\item {Grp. gram.:adj.}
\end{itemize}
Dem. de \textunderscore picado\textunderscore ^2. Cf. Castilho, \textunderscore D. Quixote\textunderscore , I, 95.
\section{Picadilho}
\begin{itemize}
\item {Grp. gram.:m.}
\end{itemize}
\begin{itemize}
\item {Proveniência:(De \textunderscore picado\textunderscore ^2)}
\end{itemize}
Espécie de tabaco.
\section{Picadinho}
\begin{itemize}
\item {Grp. gram.:adj.}
\end{itemize}
\begin{itemize}
\item {Utilização:Des.}
\end{itemize}
\begin{itemize}
\item {Utilização:Pop.}
\end{itemize}
\begin{itemize}
\item {Proveniência:(De \textunderscore picado\textunderscore ^2)}
\end{itemize}
Que se melindra com facilidade; que se offende por dá-cá-aquella-palha.
\section{Picado}
\begin{itemize}
\item {Grp. gram.:m.}
\end{itemize}
\begin{itemize}
\item {Utilização:Bras. do Rio}
\end{itemize}
O mesmo que \textunderscore cacundê\textunderscore .
\section{Picado}
\begin{itemize}
\item {Grp. gram.:adj.}
\end{itemize}
\begin{itemize}
\item {Grp. gram.:M.}
\end{itemize}
\begin{itemize}
\item {Utilização:Mús.}
\end{itemize}
\begin{itemize}
\item {Proveniência:(De \textunderscore picar\textunderscore )}
\end{itemize}
Marcado com pintas ou sinaes.
Diz-se do mar, quando agitado ou encapellado. Cf. Pant. de Aveiro, \textunderscore Itiner.\textunderscore 
Aspereza de uma superfície picada.
Iguaria, com carne ou peixe muito cortado á faca.
Recorte na extremidade de certas peças de vestuário.
Trecho, que se executa, separando os sons ligeiramente.
\section{Picador}
\begin{itemize}
\item {Grp. gram.:adj.}
\end{itemize}
\begin{itemize}
\item {Grp. gram.:M.}
\end{itemize}
\begin{itemize}
\item {Utilização:Bras}
\end{itemize}
\begin{itemize}
\item {Utilização:Bras}
\end{itemize}
\begin{itemize}
\item {Proveniência:(De \textunderscore picar\textunderscore )}
\end{itemize}
Que pica.
Aquelle que pica.
Mestre de equitação.
Aquelle que abre os atalhos chamados \textunderscore picadas\textunderscore .
Instrumento, com que se cortam ou se furam os bilhetes dos passageiros, em caminhos de ferro.
\section{Picadura}
\begin{itemize}
\item {Grp. gram.:f.}
\end{itemize}
O mesmo que \textunderscore picada\textunderscore ^1.
Palha e feno picado, muito em uso ao Norte da França.
\section{Picafigo}
\begin{itemize}
\item {Grp. gram.:m.}
\end{itemize}
\begin{itemize}
\item {Utilização:T. da Bairrada}
\end{itemize}
Ave, o mesmo que \textunderscore pàpafigo\textunderscore .
\section{Picaflor}
\begin{itemize}
\item {Grp. gram.:m.}
\end{itemize}
\begin{itemize}
\item {Proveniência:(De \textunderscore picar\textunderscore  + \textunderscore flôr\textunderscore )}
\end{itemize}
O mesmo que \textunderscore beijaflor\textunderscore .
\section{Picafolha}
\begin{itemize}
\item {fónica:fô}
\end{itemize}
\begin{itemize}
\item {Grp. gram.:f.}
\end{itemize}
Arbusto, o mesmo que \textunderscore azevinho\textunderscore .
\section{Picafumo}
\begin{itemize}
\item {Grp. gram.:m.}
\end{itemize}
\begin{itemize}
\item {Utilização:Bras. do N}
\end{itemize}
Sovina, avarento.
\section{Picaim}
\begin{itemize}
\item {Grp. gram.:m.}
\end{itemize}
(V.poaia)
\section{Pical}
\begin{itemize}
\item {Grp. gram.:m.  e  adj.}
\end{itemize}
Variedade de uva preta do Minho.
\section{Picamá}
\begin{itemize}
\item {Grp. gram.:m.}
\end{itemize}
\begin{itemize}
\item {Utilização:Bras}
\end{itemize}
Utensílio culinário.
\section{Picamilho}
\begin{itemize}
\item {Grp. gram.:m.}
\end{itemize}
\begin{itemize}
\item {Proveniência:(De \textunderscore picar\textunderscore  + \textunderscore milho\textunderscore )}
\end{itemize}
Aquelle que come muita brôa; broeiro.
Pessôa ordinária.
\section{Pica-nariz}
\begin{itemize}
\item {Grp. gram.:m.}
\end{itemize}
\begin{itemize}
\item {Utilização:Prov.}
\end{itemize}
\begin{itemize}
\item {Utilização:beir.}
\end{itemize}
Flôr de uma espécie de goivo, cujos estames duros e afiados picam o nariz de quem desprevenidamente a vai cheirar.
\section{Picancilha}
\textunderscore fem.\textunderscore  de picancilho.
\section{Picancilho}
\begin{itemize}
\item {Grp. gram.:m.}
\end{itemize}
\begin{itemize}
\item {Proveniência:(De \textunderscore picanço\textunderscore )}
\end{itemize}
Ave trepadora.
O mesmo que \textunderscore trepadeira\textunderscore .
\section{Picanço}
\begin{itemize}
\item {Grp. gram.:m.}
\end{itemize}
\begin{itemize}
\item {Utilização:Prov.}
\end{itemize}
\begin{itemize}
\item {Utilização:trasm.}
\end{itemize}
\begin{itemize}
\item {Utilização:Prov.}
\end{itemize}
\begin{itemize}
\item {Proveniência:(Do lat. \textunderscore picus\textunderscore )}
\end{itemize}
Nome de algumas aves trepadoras, como o \textunderscore pica-pau verde\textunderscore  ou \textunderscore pêto-real\textunderscore , (\textunderscore gecinus viridis\textunderscore ), e o \textunderscore pica-porco\textunderscore , (\textunderscore lanius collurio\textunderscore , Lin.).
O mesmo que \textunderscore pespilhar\textunderscore .
Cegonha, para tirar água dos poços.
\section{Picanço-bacoreiro}
\begin{itemize}
\item {Grp. gram.:m.}
\end{itemize}
O mesmo que \textunderscore picanço-real\textunderscore .
\section{Picanço-real}
\begin{itemize}
\item {Grp. gram.:m.}
\end{itemize}
Espécie de picanço indígena, frequente no sul do país.
\section{Picanha}
\begin{itemize}
\item {Grp. gram.:f.}
\end{itemize}
\begin{itemize}
\item {Utilização:Bras. do S}
\end{itemize}
Parte posterior da região lombar do boi.
(Cast. ant. \textunderscore picaña\textunderscore )
\section{Picante}
\begin{itemize}
\item {Grp. gram.:adj.}
\end{itemize}
\begin{itemize}
\item {Utilização:Pop.}
\end{itemize}
\begin{itemize}
\item {Utilização:Fig.}
\end{itemize}
\begin{itemize}
\item {Grp. gram.:M.}
\end{itemize}
Que pica.
Que excita o paladar.
Salgado.
Malicioso, mordaz: \textunderscore ditos picantes\textunderscore .
Aquillo que estimula o appetite ou que provoca.
Qualidade provocante. Cf. Eça, \textunderscore P. Basílio\textunderscore , 159.
\section{Picão}
\begin{itemize}
\item {Grp. gram.:m.}
\end{itemize}
\begin{itemize}
\item {Utilização:Prov.}
\end{itemize}
\begin{itemize}
\item {Utilização:trasm.}
\end{itemize}
\begin{itemize}
\item {Utilização:Prov.}
\end{itemize}
\begin{itemize}
\item {Proveniência:(De \textunderscore picar\textunderscore )}
\end{itemize}
Instrumento de canteiro, para picar pedra.
Instrumento de cavouqueiro, o mesmo que \textunderscore picareta\textunderscore .
Instrumento de lavoira, para picar ou sachar milho.
Planta do Brasil, (\textunderscore bidens pilosa\textunderscore ).
Pique do vinho verde.
\section{Picão}
\begin{itemize}
\item {Grp. gram.:m.}
\end{itemize}
\begin{itemize}
\item {Utilização:Ant.}
\end{itemize}
\begin{itemize}
\item {Proveniência:(De \textunderscore picar\textunderscore ?)}
\end{itemize}
Valentão; brigão.
\section{Picão}
\begin{itemize}
\item {Grp. gram.:m.}
\end{itemize}
\begin{itemize}
\item {Utilização:Prov.}
\end{itemize}
\begin{itemize}
\item {Utilização:trasm.}
\end{itemize}
\begin{itemize}
\item {Proveniência:(De \textunderscore pico\textunderscore ^1)}
\end{itemize}
O ponto mais alto de um fraguedo.
\section{Picão}
\begin{itemize}
\item {Grp. gram.:m.}
\end{itemize}
\begin{itemize}
\item {Utilização:Prov.}
\end{itemize}
\begin{itemize}
\item {Utilização:alent.}
\end{itemize}
Carvão miúdo, feito de chapotas.
\section{Picão-da-praia}
\begin{itemize}
\item {Grp. gram.:m.}
\end{itemize}
\begin{itemize}
\item {Utilização:Bras}
\end{itemize}
Planta synanthérea, medicinal, (\textunderscore avanthospermum xanthioides\textunderscore ).
\section{Pica-ôlho}
\begin{itemize}
\item {Grp. gram.:m.}
\end{itemize}
Casta de uva minhota.
\section{Pica-osso}
\begin{itemize}
\item {Grp. gram.:m.}
\end{itemize}
Ave de rapina, (\textunderscore vultur monachus\textunderscore ).
\section{Pica-pau}
\begin{itemize}
\item {Grp. gram.:m.}
\end{itemize}
\begin{itemize}
\item {Utilização:Bras}
\end{itemize}
Nome de várias aves trepadoras.
Pêto; picanço.
Espingarda antiga, de carregar pela boca.
\section{Pica-peixe}
\begin{itemize}
\item {Grp. gram.:m.}
\end{itemize}
\begin{itemize}
\item {Utilização:Náut.}
\end{itemize}
Ave syndáctyla, que se alimenta de peixes.
Pontalete de madeira, que desce da pêga do gurupés.
\section{Pica-pôlho}
\begin{itemize}
\item {Grp. gram.:m.}
\end{itemize}
Variedade de uva preta do Minho.
\section{Pica-ponto}
\begin{itemize}
\item {Grp. gram.:m.}
\end{itemize}
\begin{itemize}
\item {Utilização:Prov.}
\end{itemize}
\begin{itemize}
\item {Utilização:trasm.}
\end{itemize}
Espécie de sovela.
Instrumento de sapateiro, para imprimir uma espécie de recortes ao lado do pesponto, na borda da sola do calçado.
\section{Pica-porco}
\begin{itemize}
\item {Grp. gram.:m.}
\end{itemize}
Ave, variedade de picanço.
\section{Pica-porta}
\begin{itemize}
\item {Grp. gram.:m.}
\end{itemize}
Nome, que nos Açores se dá á aldrava.
(Cp. cast. \textunderscore picaporte\textunderscore , trinco)
\section{Picar}
\begin{itemize}
\item {Grp. gram.:v. t.}
\end{itemize}
\begin{itemize}
\item {Utilização:Fig.}
\end{itemize}
\begin{itemize}
\item {Utilização:Pop.}
\end{itemize}
\begin{itemize}
\item {Utilização:Chul.}
\end{itemize}
\begin{itemize}
\item {Utilização:Mús.}
\end{itemize}
\begin{itemize}
\item {Utilização:Prov.}
\end{itemize}
\begin{itemize}
\item {Utilização:trasm.}
\end{itemize}
\begin{itemize}
\item {Grp. gram.:V. i.}
\end{itemize}
\begin{itemize}
\item {Utilização:Prov.}
\end{itemize}
\begin{itemize}
\item {Utilização:minh.}
\end{itemize}
\begin{itemize}
\item {Grp. gram.:V. p.}
\end{itemize}
\begin{itemize}
\item {Proveniência:(De \textunderscore pico\textunderscore ^1)}
\end{itemize}
Ferir com objecto aguçado ou perfurante: \textunderscore picar um dedo com a agulha\textunderscore .
Bicar.
Abrir buracos em.
Arpoar.
Golpear.
Partir em fragmentos.
Lascar; espicaçar.
Estimular.
Causar impressão dolorosa a.
Aguilhoar.
Irritar.
Apressar.
Causar prurido ou comichão a.
Furtar.
Articular (sons) ligeiramente, sem acentuação, mas separando-os um pouco.
Dar a primeira sacha em: \textunderscore picar o milho\textunderscore .
Subir de preço: \textunderscore o milho, na feira, já começou a picar\textunderscore .
Produzir comichão, prurido ou ardor: \textunderscore as urtigas picam\textunderscore .
Sêr ferido por objecto perfurante: \textunderscore picou-se num prego\textunderscore .
Melindrar-se ou magoar-se, por offensa ou provocação recebida: \textunderscore picou-se com aquella allusão\textunderscore .
\section{Picaramente}
\begin{itemize}
\item {Grp. gram.:adv.}
\end{itemize}
De modo pícaro; com astúcia; com patifaria.
\section{Picarato}
\begin{itemize}
\item {Grp. gram.:m.}
\end{itemize}
\begin{itemize}
\item {Utilização:Açor}
\end{itemize}
Aquelle que é natural da Ilha do Pico.
\section{Picarço}
\begin{itemize}
\item {Grp. gram.:adj.}
\end{itemize}
\begin{itemize}
\item {Utilização:Prov.}
\end{itemize}
O mesmo que \textunderscore pigarço\textunderscore . (Colhido na Bairrada)
\section{Picardia}
\begin{itemize}
\item {Grp. gram.:f.}
\end{itemize}
Acção de pícaro; velhacaria; pirraça.
(Por \textunderscore picaria\textunderscore , de \textunderscore pícaro\textunderscore )
\section{Picardo}
\begin{itemize}
\item {Grp. gram.:m.}
\end{itemize}
Dialecto da Picardia.
\section{Picarescamente}
\begin{itemize}
\item {Grp. gram.:adv.}
\end{itemize}
De modo picaresco; ridiculamente.
\section{Picaresco}
\begin{itemize}
\item {Grp. gram.:adj.}
\end{itemize}
Próprio de pícaro; burlesco; ridículo.
\section{Picareta}
\begin{itemize}
\item {fónica:carê}
\end{itemize}
\begin{itemize}
\item {Grp. gram.:f.}
\end{itemize}
\begin{itemize}
\item {Proveniência:(De \textunderscore picar\textunderscore )}
\end{itemize}
Instrumento de ferro, para escavar terra, arrancar pedras, etc.; alvião.
\section{Picaria}
\begin{itemize}
\item {Grp. gram.:f.}
\end{itemize}
\begin{itemize}
\item {Proveniência:(De \textunderscore picar\textunderscore )}
\end{itemize}
Arte de equitação.
Lugar, onde se exerce a arte de picadeiro.
\section{Picarnel}
\begin{itemize}
\item {Grp. gram.:m.}
\end{itemize}
\begin{itemize}
\item {Utilização:Prov.}
\end{itemize}
\begin{itemize}
\item {Utilização:trasm.}
\end{itemize}
\begin{itemize}
\item {Utilização:Ant.}
\end{itemize}
Azenha provisória, que se assenta sôbre as pedras da ribeira, para aproveitar no verão algum veio mais grosso de água. Cf. S. R. Viterbo, \textunderscore Elucid.\textunderscore , vb. \textunderscore camba\textunderscore .
\section{Pícaro}
\begin{itemize}
\item {Grp. gram.:adj.}
\end{itemize}
Ardiloso, astuto; que é patife.
Ridículo.
(Cast. \textunderscore picaro\textunderscore )
\section{Picaró}
\begin{itemize}
\item {Grp. gram.:m.}
\end{itemize}
\begin{itemize}
\item {Utilização:Ant.}
\end{itemize}
O mesmo que \textunderscore picó\textunderscore ?:«\textunderscore ...atados com dois bocados de picó azul, que já se não chamão fitas de picaró\textunderscore ». \textunderscore Anat. Joc.\textunderscore , 75.
\section{Picaroto}
\begin{itemize}
\item {fónica:carô}
\end{itemize}
\begin{itemize}
\item {Grp. gram.:m.}
\end{itemize}
\begin{itemize}
\item {Utilização:Ant.}
\end{itemize}
\begin{itemize}
\item {Proveniência:(De \textunderscore pico\textunderscore ^1? Ou por \textunderscore pincarôto\textunderscore , de \textunderscore píncaro\textunderscore ?)}
\end{itemize}
O ponto mais alto; cume; vértice. Cf. Soropita, \textunderscore Poes. e Pros.\textunderscore , 40.
\section{Piçarra}
\begin{itemize}
\item {Grp. gram.:f.}
\end{itemize}
Terra, misturada com areia e pedra; cascalho.
Cascalheira.
Xisto, cujas fôlhas são mecanicamente separáveis.
Pedreira; penedia.
(Cast. \textunderscore pizarra\textunderscore )
\section{Piçarral}
\begin{itemize}
\item {Grp. gram.:m.}
\end{itemize}
Lugar onde há piçarra.
\section{Piçarreira}
\begin{itemize}
\item {Grp. gram.:f.}
\end{itemize}
\begin{itemize}
\item {Utilização:Prov.}
\end{itemize}
\begin{itemize}
\item {Utilização:trasm.}
\end{itemize}
\begin{itemize}
\item {Proveniência:(De \textunderscore piçarra\textunderscore )}
\end{itemize}
Rocha de xisto.
\section{Piçarro}
\begin{itemize}
\item {Grp. gram.:m.}
\end{itemize}
O mesmo que \textunderscore piçarra\textunderscore .
\section{Piçarroso}
\begin{itemize}
\item {Grp. gram.:adj.}
\end{itemize}
Abundante de piçarra; que tem a natureza de piçarra.
\section{Picarucu}
\begin{itemize}
\item {Grp. gram.:m.}
\end{itemize}
\begin{itemize}
\item {Utilização:Bras}
\end{itemize}
Peixe fluvial.
\section{Piçasfalto}
\begin{itemize}
\item {Grp. gram.:m.}
\end{itemize}
\begin{itemize}
\item {Proveniência:(De \textunderscore pix\textunderscore  lat. + \textunderscore asphalto\textunderscore )}
\end{itemize}
Mistura de pez e betume.
\section{Piçasphaltho}
\begin{itemize}
\item {Grp. gram.:m.}
\end{itemize}
\begin{itemize}
\item {Proveniência:(De \textunderscore pix\textunderscore  lat. + \textunderscore asphalto\textunderscore )}
\end{itemize}
Mistura de pez e betume.
\section{Picatoste}
\begin{itemize}
\item {Grp. gram.:m.}
\end{itemize}
Iguaria de carneiro, ovos e pão ralado.
(Cast. \textunderscore picatoste\textunderscore )
\section{Picaveco}
\begin{itemize}
\item {Grp. gram.:m.}
\end{itemize}
\begin{itemize}
\item {Utilização:Des.}
\end{itemize}
Janota; peralta:«\textunderscore ...uns certos picavecos apetrechados que todo o seu cabedal empregam em contemplação de amor.\textunderscore »Soropita, \textunderscore Poes. e Pros.\textunderscore , 41.
\section{Picentino}
\begin{itemize}
\item {Grp. gram.:adj.}
\end{itemize}
\begin{itemize}
\item {Proveniência:(Lat. \textunderscore picentinus\textunderscore )}
\end{itemize}
Relativo ao Picentino ou Piceno, hoje Marca de Ancona, região da Itália occidental, notável por seus pomares e olivedos.
Dizia-se especialmente, entre os Romanos, de uma qualidade de pão de luxo, o mais estimado. Cf. Castilho, \textunderscore Fastos\textunderscore , III, 479.
\section{Píceo}
\begin{itemize}
\item {Grp. gram.:adj.}
\end{itemize}
\begin{itemize}
\item {Proveniência:(Lat. \textunderscore piceus\textunderscore )}
\end{itemize}
Que é da natureza do pez.
Semelhante a pez; que produz pez.
Feito de pez. Cf. F. Barreto, \textunderscore Eneida\textunderscore , III, 129.
\section{Picha}
\begin{itemize}
\item {Grp. gram.:f.}
\end{itemize}
\begin{itemize}
\item {Utilização:Prov.}
\end{itemize}
Galheta.
(Cp. \textunderscore pichel\textunderscore )
\section{Picha}
\begin{itemize}
\item {Grp. gram.:f.}
\end{itemize}
\begin{itemize}
\item {Utilização:Prov.}
\end{itemize}
Nome que, na Figueira da Foz, se dá ao camarão pequeno.
Camarão pequeno.
\section{Picha}
\begin{itemize}
\item {Grp. gram.:f.}
\end{itemize}
\begin{itemize}
\item {Utilização:T. de Alcanena}
\end{itemize}
O mesmo que \textunderscore pênis\textunderscore .
\section{Pichaí}
\begin{itemize}
\item {Grp. gram.:m.}
\end{itemize}
O mesmo que \textunderscore pichaim\textunderscore .
\section{Pichaim}
\begin{itemize}
\item {Grp. gram.:m.}
\end{itemize}
\begin{itemize}
\item {Utilização:Bras. do N}
\end{itemize}
\begin{itemize}
\item {Grp. gram.:Adj.}
\end{itemize}
O mesmo que \textunderscore carapinha\textunderscore .
Diz-se do cabello encarapinhado.
\section{Pichão}
\begin{itemize}
\item {Grp. gram.:m.}
\end{itemize}
\begin{itemize}
\item {Utilização:Prov.}
\end{itemize}
\begin{itemize}
\item {Utilização:minh.}
\end{itemize}
Pombo pequeno, borracho.
(Cp. al. \textunderscore pichon\textunderscore )
\section{Piche}
\begin{itemize}
\item {Grp. gram.:m.}
\end{itemize}
\begin{itemize}
\item {Proveniência:(Do ingl. \textunderscore pitch\textunderscore )}
\end{itemize}
Espécie de alcatrão, que se obtém na depuração phýsica do gás.
\section{Picheiro}
\begin{itemize}
\item {Grp. gram.:m.}
\end{itemize}
\begin{itemize}
\item {Utilização:Prov.}
\end{itemize}
\begin{itemize}
\item {Utilização:T. de Pare -de-Coira}
\end{itemize}
\begin{itemize}
\item {Utilização:des.}
\end{itemize}
\begin{itemize}
\item {Proveniência:(De \textunderscore picha\textunderscore )}
\end{itemize}
Vaso para conter leite.
Porrão de barro, que se põe ao lume.
\section{Pichel}
\begin{itemize}
\item {Grp. gram.:m.}
\end{itemize}
\begin{itemize}
\item {Utilização:Prov.}
\end{itemize}
\begin{itemize}
\item {Utilização:minh.}
\end{itemize}
Vasilha antiga, para onde se tira vinho das pipas ou tonéis.
Cântaro pequeno.
Pequeno vaso antigo, geralmente de estanho, para beber vinho.
(Cp. it. \textunderscore bicchiere\textunderscore )
\section{Pichela}
\begin{itemize}
\item {Grp. gram.:f.}
\end{itemize}
\begin{itemize}
\item {Utilização:Prov.}
\end{itemize}
\begin{itemize}
\item {Utilização:minh.}
\end{itemize}
\begin{itemize}
\item {Proveniência:(De \textunderscore picho\textunderscore )}
\end{itemize}
Caçarola de barro.
\section{Pichelaria}
\begin{itemize}
\item {Grp. gram.:f.}
\end{itemize}
\begin{itemize}
\item {Proveniência:(De \textunderscore pichel\textunderscore )}
\end{itemize}
Officina de picheleiro.
Offício ou obra de picheleiro.
\section{Picheleiro}
\begin{itemize}
\item {Grp. gram.:m.}
\end{itemize}
\begin{itemize}
\item {Utilização:Prov.}
\end{itemize}
\begin{itemize}
\item {Utilização:minh.}
\end{itemize}
\begin{itemize}
\item {Proveniência:(De \textunderscore pichel\textunderscore )}
\end{itemize}
Fabricante de pichéis.
Fabricante ou vendedor de obras de estanho.
Pichel ou cântaro de barro negro.
\section{Pichelim}
\begin{itemize}
\item {Grp. gram.:m.}
\end{itemize}
\begin{itemize}
\item {Utilização:Prov.}
\end{itemize}
\begin{itemize}
\item {Utilização:minh.}
\end{itemize}
\begin{itemize}
\item {Proveniência:(De \textunderscore pichel\textunderscore )}
\end{itemize}
Infusa pequena.
\section{Pichelim}
\begin{itemize}
\item {Grp. gram.:m.}
\end{itemize}
\begin{itemize}
\item {Utilização:Prov.}
\end{itemize}
Carne de peixe, chamado carocho, immergido em salmoira, lavado depois e sêco ao sol, o qual, assim preparado, é exportado da costa do Minho para o Alentejo e Sul da Espanha.
\section{Pichelingue}
\begin{itemize}
\item {Grp. gram.:m.}
\end{itemize}
\begin{itemize}
\item {Utilização:Des.}
\end{itemize}
\begin{itemize}
\item {Utilização:Pop.}
\end{itemize}
Ladrão, larápio.
(Cast. \textunderscore pichilingue\textunderscore , t. car.)
\section{Pichelo}
\begin{itemize}
\item {fónica:chê}
\end{itemize}
\begin{itemize}
\item {Grp. gram.:m.}
\end{itemize}
O mesmo que \textunderscore pichela\textunderscore .
\section{Pichi}
\begin{itemize}
\item {Grp. gram.:m.}
\end{itemize}
Planta medicinal, de propriedades diuréticas, (\textunderscore fabiana imbricata\textunderscore ).
\section{Pichincho}
\begin{itemize}
\item {Grp. gram.:adj.}
\end{itemize}
\begin{itemize}
\item {Utilização:Açor}
\end{itemize}
Muito pequeno; pequenino.
\section{Picho}
\begin{itemize}
\item {Grp. gram.:m.}
\end{itemize}
\begin{itemize}
\item {Utilização:Prov.}
\end{itemize}
\begin{itemize}
\item {Utilização:minh.}
\end{itemize}
\begin{itemize}
\item {Utilização:Prov.}
\end{itemize}
\begin{itemize}
\item {Utilização:T. de Pare -de-Coira}
\end{itemize}
\begin{itemize}
\item {Utilização:des.}
\end{itemize}
O mesmo que \textunderscore pichel\textunderscore .
Pequeno pote de barro.
Carrapito de cabello, no alto da cabeça.
O mesmo que \textunderscore chocolateira\textunderscore .
\section{Picholeta}
\begin{itemize}
\item {fónica:lê}
\end{itemize}
\begin{itemize}
\item {Grp. gram.:f.}
\end{itemize}
\begin{itemize}
\item {Utilização:Fam.}
\end{itemize}
\textunderscore Dar a picholeta por\textunderscore , gostar muito de.
\section{Pichori}
\begin{itemize}
\item {Grp. gram.:m.}
\end{itemize}
Vestuário festivo para homem, entre os Parses.
\section{Pichorra}
\begin{itemize}
\item {fónica:chô}
\end{itemize}
\begin{itemize}
\item {Grp. gram.:f.}
\end{itemize}
\begin{itemize}
\item {Utilização:Prov.}
\end{itemize}
\begin{itemize}
\item {Utilização:trasm.}
\end{itemize}
Pichel com bico.
Pequena cântara de barro branco com bico.
\section{Pichorro}
\begin{itemize}
\item {fónica:chô}
\end{itemize}
\begin{itemize}
\item {Grp. gram.:adj.}
\end{itemize}
\begin{itemize}
\item {Utilização:Prov.}
\end{itemize}
\begin{itemize}
\item {Utilização:alg.}
\end{itemize}
Diz-se de uma espécie de milho, de cana alta.
\section{Pichorro}
\begin{itemize}
\item {fónica:chô}
\end{itemize}
\begin{itemize}
\item {Grp. gram.:m.}
\end{itemize}
O mesmo que \textunderscore pichorra\textunderscore .
\section{Pichuá}
\begin{itemize}
\item {Grp. gram.:m.}
\end{itemize}
Planta euphorbiácea do Brasil.
\section{Pichurim}
\begin{itemize}
\item {Grp. gram.:m.}
\end{itemize}
Planta laurínea do Brasil, o mesmo que \textunderscore puchuri\textunderscore .
\section{Picica}
\begin{itemize}
\item {Grp. gram.:f.}
\end{itemize}
\begin{itemize}
\item {Utilização:Bras. do N}
\end{itemize}
Pessôa de pequena estatura.
Coisa insignificante.
\section{Pico}
\begin{itemize}
\item {Grp. gram.:m.}
\end{itemize}
\begin{itemize}
\item {Utilização:Prov.}
\end{itemize}
\begin{itemize}
\item {Utilização:minh.}
\end{itemize}
\begin{itemize}
\item {Utilização:Bras. do N}
\end{itemize}
\begin{itemize}
\item {Utilização:Fig.}
\end{itemize}
\begin{itemize}
\item {Proveniência:(Do celt. \textunderscore pic\textunderscore )}
\end{itemize}
Ponta aguda; bico.
Espinho.
Picão.
Maço de carvalho, com que os oleiros trituram o barro numa pia de madeira, depois de doseadas as côres.
Ponto elevado, cume.
Pica-pau.
Espécie de pêlo de alguns vegetaes, que produz comichão em quem o toca.
Acidez; pique, sabor picante.
Chiste, graça.
\section{Pico}
\begin{itemize}
\item {Grp. gram.:adv.}
\end{itemize}
\begin{itemize}
\item {Utilização:Pop.}
\end{itemize}
Pouco mais.--Precede-o sempre a conj. \textunderscore e\textunderscore :«duas libras \textunderscore e pico\textunderscore »;«quatro horas \textunderscore e pico\textunderscore ».
\section{Pico}
\begin{itemize}
\item {Grp. gram.:m.}
\end{itemize}
Antigo pêso da China, correspondente a 100 cates ou a pouco mais de 61 kilogrammas, e adoptado em Timor.
\section{Picó}
\begin{itemize}
\item {Grp. gram.:m.}
\end{itemize}
\begin{itemize}
\item {Utilização:Des.}
\end{itemize}
Espécie de pano, o mesmo que \textunderscore picote\textunderscore ^1. Cf. \textunderscore Anat. Joc.\textunderscore , 75.
\section{Piçó}
\begin{itemize}
\item {Grp. gram.:adj.}
\end{itemize}
\begin{itemize}
\item {Utilização:Gír.}
\end{itemize}
O mesmo que \textunderscore bêbedo\textunderscore :«\textunderscore aquella alma de mestre, mal que me apanhou piçó\textunderscore ...». (Canção marítima)
(Cp. conc. \textunderscore piçô\textunderscore , doido)
\section{Picoá}
\begin{itemize}
\item {Grp. gram.:m.}
\end{itemize}
\begin{itemize}
\item {Utilização:Bras}
\end{itemize}
\begin{itemize}
\item {Utilização:Bras. de Minas}
\end{itemize}
\begin{itemize}
\item {Proveniência:(Do guar. \textunderscore apiquá\textunderscore )}
\end{itemize}
Mala de pano, em que se levam roupas ou comidas, em viagem.
Peça ôca de chifre ou de outra substância, em que os Mineiros guardam diamantes.
\section{Picoar}
\begin{itemize}
\item {Grp. gram.:v. t.}
\end{itemize}
\begin{itemize}
\item {Utilização:Bras. de Minas}
\end{itemize}
\begin{itemize}
\item {Proveniência:(De \textunderscore picoá\textunderscore )}
\end{itemize}
Arrecadar.
Enthesoirar; guardar em saco: \textunderscore picoar dinheiro\textunderscore .
\section{Pico-de-el-rei}
\begin{itemize}
\item {Grp. gram.:m.}
\end{itemize}
Peixe de Portugal, o mesmo que \textunderscore pica-de-el-rei\textunderscore .
\section{Picola}
\begin{itemize}
\item {Grp. gram.:f.}
\end{itemize}
\begin{itemize}
\item {Proveniência:(De \textunderscore pico\textunderscore ^1)}
\end{itemize}
Instrumento de canteiro, para alisar pedra.
\section{Picoleia, á}
\begin{itemize}
\item {Grp. gram.:loc. adv.}
\end{itemize}
\begin{itemize}
\item {Utilização:Prov.}
\end{itemize}
\begin{itemize}
\item {Utilização:minh.}
\end{itemize}
Á escuta.
\section{Picolinhas}
\begin{itemize}
\item {Grp. gram.:f. pl.}
\end{itemize}
\begin{itemize}
\item {Utilização:Prov.}
\end{itemize}
\begin{itemize}
\item {Utilização:trasm.}
\end{itemize}
Intrigas, picuínhas.
\section{Piconeiro}
\begin{itemize}
\item {Grp. gram.:m.}
\end{itemize}
\begin{itemize}
\item {Utilização:Prov.}
\end{itemize}
\begin{itemize}
\item {Utilização:alent.}
\end{itemize}
Vendedor de picão, carvão miúdo.
\section{Piçorelho}
\begin{itemize}
\item {Grp. gram.:m.}
\end{itemize}
\begin{itemize}
\item {Utilização:Prov.}
\end{itemize}
O mesmo que \textunderscore pica-peixe\textunderscore , ave.
\section{Piçós}
\begin{itemize}
\item {Grp. gram.:m. pl.}
\end{itemize}
\begin{itemize}
\item {Utilização:Prov.}
\end{itemize}
\begin{itemize}
\item {Utilização:trasm.}
\end{itemize}
O mesmo que \textunderscore peçós\textunderscore .
\section{Picoso}
\begin{itemize}
\item {Grp. gram.:adj.}
\end{itemize}
Que tem picos.
Que termina em pico ou cume.
Alto.
\section{Picota}
\begin{itemize}
\item {Grp. gram.:f.}
\end{itemize}
\begin{itemize}
\item {Utilização:T. do Ribatejo}
\end{itemize}
\begin{itemize}
\item {Proveniência:(De \textunderscore pico\textunderscore ^1)}
\end{itemize}
Poste ou madeiro, erguido em praça pública, e em cuja extremidade superior se expunham as cabeças dos justiçados.
Poste, guarnecido de argolas e correntes, onde se executavam penas ignominiosas, açoitando-se os delinquentes ou sendo expostos á irrisão pública.
Haste do êmbolo de uma bomba.
Engenho ou cegonha, para elevar água.
\section{Picota}
\begin{itemize}
\item {Grp. gram.:f.}
\end{itemize}
\begin{itemize}
\item {Utilização:Bras}
\end{itemize}
Ave, o mesmo que \textunderscore gallinha-da-índia\textunderscore .
\section{Picotá}
\begin{itemize}
\item {Grp. gram.:m.}
\end{itemize}
Vantagem no pêso, fruída pelos mercadores, nas praças commerciaes do Oriente. Cf. Felner, \textunderscore Subs. para a Hist. da Índia Port.\textunderscore , VII.
\section{Picotagem}
\begin{itemize}
\item {Grp. gram.:f.}
\end{itemize}
\begin{itemize}
\item {Utilização:Bras}
\end{itemize}
Acto de picotar.
\section{Picotar}
\begin{itemize}
\item {Grp. gram.:v. t.}
\end{itemize}
\begin{itemize}
\item {Utilização:Bras}
\end{itemize}
Cortar com o picador, com que os revisores do caminho de ferro cortam os bilhetes dos passageiros. Cf. \textunderscore Notícia\textunderscore , do Rio, de 28-XI-901.
(Cp. \textunderscore picar\textunderscore )
\section{Picote}
\begin{itemize}
\item {Grp. gram.:m.}
\end{itemize}
Pano grosseiro, mais conhecido por \textunderscore picoto\textunderscore ^2.
(Cast. \textunderscore picote\textunderscore )
\section{Picote}
\begin{itemize}
\item {Grp. gram.:m.}
\end{itemize}
\begin{itemize}
\item {Proveniência:(Fr. \textunderscore picot\textunderscore , de \textunderscore pic\textunderscore )}
\end{itemize}
Ponto de rendaria, que consiste numa pequenina argola de linha, e usada em rendas leves e finas.
\section{Picotilho}
\begin{itemize}
\item {Grp. gram.:m.}
\end{itemize}
Pano picoto, menos grosso que o picote^1.
\section{Picoto}
\begin{itemize}
\item {fónica:cô}
\end{itemize}
\begin{itemize}
\item {Grp. gram.:m.}
\end{itemize}
\begin{itemize}
\item {Proveniência:(De \textunderscore pico\textunderscore ^1)}
\end{itemize}
Cimo agudo de um monte.
Marco ou columna de pedra, collocada no cimo de um monte.
Pyrâmide de triangulação.
\section{Picoto}
\begin{itemize}
\item {fónica:cô}
\end{itemize}
\begin{itemize}
\item {Grp. gram.:m.  e  adj.}
\end{itemize}
Diz-se de um pano grosseiro de lan, também chamado \textunderscore picote\textunderscore ^1.
\section{Pícria}
\begin{itemize}
\item {Grp. gram.:f.}
\end{itemize}
Gênero de plantas gesneriáceas.
\section{Pícrico}
\begin{itemize}
\item {Grp. gram.:adj.}
\end{itemize}
\begin{itemize}
\item {Proveniência:(Do gr. \textunderscore pikros\textunderscore )}
\end{itemize}
Diz-se de um ácido desinfectante, produzido pela acção do ácido azótico sôbre o phênico, muito usado em Medicina no tratamento de queimaduras superficiaes, e, na Indústria, como substância còrante.
\section{Picrina}
\begin{itemize}
\item {Grp. gram.:f.}
\end{itemize}
\begin{itemize}
\item {Proveniência:(De \textunderscore pícrico\textunderscore )}
\end{itemize}
Substância amarga da dedaleira.
\section{Picrito}
\begin{itemize}
\item {Grp. gram.:m.}
\end{itemize}
\begin{itemize}
\item {Proveniência:(Do gr. \textunderscore pikros\textunderscore )}
\end{itemize}
Variedade de carbonato de cal e de magnésia.
\section{Picroaconitina}
\begin{itemize}
\item {Grp. gram.:f.}
\end{itemize}
\begin{itemize}
\item {Utilização:Chím.}
\end{itemize}
\begin{itemize}
\item {Proveniência:(De \textunderscore pikros\textunderscore , gr. + \textunderscore aconitina\textunderscore )}
\end{itemize}
Base orgânica amorpha.
\section{Picrocholo}
\begin{itemize}
\item {fónica:co}
\end{itemize}
\begin{itemize}
\item {Grp. gram.:adj.}
\end{itemize}
\begin{itemize}
\item {Utilização:Med.}
\end{itemize}
\begin{itemize}
\item {Proveniência:(Do gr. \textunderscore pikros\textunderscore  + \textunderscore khole\textunderscore )}
\end{itemize}
Em que abunda a bílis.
\section{Picrocolo}
\begin{itemize}
\item {Grp. gram.:adj.}
\end{itemize}
\begin{itemize}
\item {Utilização:Med.}
\end{itemize}
\begin{itemize}
\item {Proveniência:(Do gr. \textunderscore pikros\textunderscore  + \textunderscore khole\textunderscore )}
\end{itemize}
Em que abunda a bílis.
\section{Picrofílio}
\begin{itemize}
\item {Grp. gram.:m.}
\end{itemize}
\begin{itemize}
\item {Utilização:Miner.}
\end{itemize}
\begin{itemize}
\item {Proveniência:(Do gr. \textunderscore pikros\textunderscore  + \textunderscore phullon\textunderscore )}
\end{itemize}
Variedade de talco.
\section{Picroglício}
\begin{itemize}
\item {Grp. gram.:m.}
\end{itemize}
\begin{itemize}
\item {Utilização:Chím.}
\end{itemize}
\begin{itemize}
\item {Proveniência:(Do gr. \textunderscore pikros\textunderscore  + \textunderscore glukus\textunderscore )}
\end{itemize}
Substância amarga, contida na dulcamara.
\section{Picroglýcio}
\begin{itemize}
\item {Grp. gram.:m.}
\end{itemize}
\begin{itemize}
\item {Utilização:Chím.}
\end{itemize}
\begin{itemize}
\item {Proveniência:(Do gr. \textunderscore pikros\textunderscore  + \textunderscore glukus\textunderscore )}
\end{itemize}
Substância amarga, contida na dulcamara.
\section{Picrólitho}
\begin{itemize}
\item {Grp. gram.:m.}
\end{itemize}
\begin{itemize}
\item {Utilização:Miner.}
\end{itemize}
Variedade de serpentina.
\section{Picrólito}
\begin{itemize}
\item {Grp. gram.:m.}
\end{itemize}
\begin{itemize}
\item {Utilização:Miner.}
\end{itemize}
Variedade de serpentina.
\section{Picromel}
\begin{itemize}
\item {Grp. gram.:m.}
\end{itemize}
\begin{itemize}
\item {Utilização:Chím.}
\end{itemize}
\begin{itemize}
\item {Proveniência:(De \textunderscore pikros\textunderscore  gr. + \textunderscore mel\textunderscore )}
\end{itemize}
Substância incolor, de aspecto e consistência iguaes á terebinthina.
\section{Picromerito}
\begin{itemize}
\item {Grp. gram.:m.}
\end{itemize}
\begin{itemize}
\item {Utilização:Miner.}
\end{itemize}
\begin{itemize}
\item {Proveniência:(Do gr. \textunderscore pikros\textunderscore  + \textunderscore meris\textunderscore )}
\end{itemize}
Sulfato hydratado de magnésio e potássio.
\section{Picrophýllio}
\begin{itemize}
\item {Grp. gram.:m.}
\end{itemize}
\begin{itemize}
\item {Utilização:Miner.}
\end{itemize}
\begin{itemize}
\item {Proveniência:(Do gr. \textunderscore pikros\textunderscore  + \textunderscore phullon\textunderscore )}
\end{itemize}
Variedade de talco.
\section{Picrotoxina}
\begin{itemize}
\item {fónica:csi}
\end{itemize}
\begin{itemize}
\item {Grp. gram.:f.}
\end{itemize}
Substância, extrahida da coca do Levante e usada em therapêutica contra a epilepsia e outras moléstias nervosas.
\section{Pictoresco}
\begin{itemize}
\item {Grp. gram.:adj.}
\end{itemize}
\begin{itemize}
\item {Proveniência:(Do lat. \textunderscore pictor\textunderscore )}
\end{itemize}
O mesmo que \textunderscore pinturesco\textunderscore .
\section{Pictoricamente}
\begin{itemize}
\item {Grp. gram.:adv.}
\end{itemize}
De modo pictórico.
\section{Pictórico}
\begin{itemize}
\item {Grp. gram.:adj.}
\end{itemize}
\begin{itemize}
\item {Proveniência:(Do lat. \textunderscore pictor\textunderscore )}
\end{itemize}
Relativo á pintura.
\section{Pictural}
\begin{itemize}
\item {Grp. gram.:adj.}
\end{itemize}
\begin{itemize}
\item {Proveniência:(Do lat. \textunderscore pictura\textunderscore )}
\end{itemize}
O mesmo que \textunderscore pictoresco\textunderscore .
\section{Picuá}
\begin{itemize}
\item {Grp. gram.:m.}
\end{itemize}
O mesmo que \textunderscore picoá\textunderscore .
\section{Picúculo}
\begin{itemize}
\item {Grp. gram.:m.}
\end{itemize}
Gênero de pássaros tenuirostros.
\section{Picuínha}
\begin{itemize}
\item {Grp. gram.:f.}
\end{itemize}
\begin{itemize}
\item {Proveniência:(De \textunderscore pico\textunderscore ^1)}
\end{itemize}
Os primeiros pios da ave.
Chiste; remoque; allusão picante.
\section{Picuipinima}
\begin{itemize}
\item {Grp. gram.:f.}
\end{itemize}
Rôla do Brasil.
\section{Picuipita}
\begin{itemize}
\item {Grp. gram.:m.}
\end{itemize}
Pombo do Paraguai.
\section{Picuman}
\begin{itemize}
\item {Grp. gram.:m.}
\end{itemize}
\begin{itemize}
\item {Utilização:Bras}
\end{itemize}
Fuligem.
(Do tupi \textunderscore apepocuman\textunderscore )
\section{Picumno}
\begin{itemize}
\item {Grp. gram.:m.}
\end{itemize}
Gênero de aves trepadoras da Índia e da América.
\section{Pida}
\begin{itemize}
\item {Grp. gram.:f.}
\end{itemize}
\begin{itemize}
\item {Utilização:Prov.}
\end{itemize}
\begin{itemize}
\item {Utilização:alent.}
\end{itemize}
\begin{itemize}
\item {Utilização:Fam.}
\end{itemize}
\begin{itemize}
\item {Proveniência:(De \textunderscore pedir\textunderscore . \textunderscore Pido\textunderscore  é fórma ant. do indic. pres. de \textunderscore pedir\textunderscore )}
\end{itemize}
Acto de pedir esmola.
\section{Pidir}
\begin{itemize}
\item {Grp. gram.:v. t.}
\end{itemize}
\begin{itemize}
\item {Utilização:Ant.}
\end{itemize}
O mesmo que \textunderscore pedir\textunderscore . Cf. Usque, 47.
\section{Piedade}
\begin{itemize}
\item {Grp. gram.:f.}
\end{itemize}
\begin{itemize}
\item {Proveniência:(Lat. \textunderscore pietas\textunderscore )}
\end{itemize}
Amor ás coisas da Religião.
Religiosidade, devoção.
Compaixão; sentimento, despertado pelos soffrimentos alheios e que nos leva a mitigá-los ou a desejar remediá-los.
Pena, dó.
\section{Piedosamente}
\begin{itemize}
\item {Grp. gram.:adv.}
\end{itemize}
De modo piedoso; com piedade.
\section{Piedoso}
\begin{itemize}
\item {Grp. gram.:adj.}
\end{itemize}
\begin{itemize}
\item {Proveniência:(Lat. \textunderscore pietosus\textunderscore )}
\end{itemize}
Que tem ou revela piedade.
\section{Piegas}
\begin{itemize}
\item {Grp. gram.:m. ,  f.  e  adj.}
\end{itemize}
\begin{itemize}
\item {Utilização:Pop.}
\end{itemize}
\begin{itemize}
\item {Grp. gram.:M.}
\end{itemize}
\begin{itemize}
\item {Utilização:Gír.}
\end{itemize}
Pessôa, que se embaraça com pequenas coisas, niquenta, ridícula.
Pênis.
\section{Piegasmente}
\begin{itemize}
\item {Grp. gram.:adv.}
\end{itemize}
De modo piegas.
\section{Piego}
\begin{itemize}
\item {Grp. gram.:adj.}
\end{itemize}
\begin{itemize}
\item {Utilização:Prov.}
\end{itemize}
\begin{itemize}
\item {Utilização:trasm.}
\end{itemize}
Que anda perto; que se aproxima.
\section{Piegueiro}
\begin{itemize}
\item {Grp. gram.:adj.}
\end{itemize}
\begin{itemize}
\item {Utilização:Bras}
\end{itemize}
\begin{itemize}
\item {Proveniência:(De \textunderscore piegas\textunderscore )}
\end{itemize}
Meigo, muito dócil.
Que faz carícias como uma criança.
\section{Pieguice}
\begin{itemize}
\item {Grp. gram.:f.}
\end{itemize}
Qualidade de quem é piegas.
Sentimentalidade excessiva ou affectada.
\section{Pieira}
\begin{itemize}
\item {Grp. gram.:f.}
\end{itemize}
\begin{itemize}
\item {Proveniência:(De \textunderscore piar\textunderscore ^1)}
\end{itemize}
Som, produzido pela respiração diffícil de um doente.
\section{Pieiro}
\begin{itemize}
\item {Grp. gram.:m.}
\end{itemize}
\begin{itemize}
\item {Utilização:Prov.}
\end{itemize}
\begin{itemize}
\item {Utilização:trasm.}
\end{itemize}
\begin{itemize}
\item {Proveniência:(De \textunderscore piar\textunderscore ^3)}
\end{itemize}
Pia de pedra, em que se descasca o milho branco.
\section{Piela}
\begin{itemize}
\item {Grp. gram.:f.}
\end{itemize}
\begin{itemize}
\item {Utilização:Gír.}
\end{itemize}
\begin{itemize}
\item {Proveniência:(De \textunderscore piar\textunderscore ^2)}
\end{itemize}
Bebedeira.
\section{Piemontês}
\begin{itemize}
\item {Grp. gram.:adj.}
\end{itemize}
\begin{itemize}
\item {Grp. gram.:M.}
\end{itemize}
Relativo ao Piemonte.
Habitante do Piemonte.
Dialecto italiano, falado pelos Piemonteses.
\section{Piemontite}
\begin{itemize}
\item {Grp. gram.:f.}
\end{itemize}
\begin{itemize}
\item {Utilização:Miner.}
\end{itemize}
Variedade do epidoto.
\section{Pientíssimo}
\begin{itemize}
\item {Grp. gram.:adj.}
\end{itemize}
\begin{itemize}
\item {Proveniência:(Do lat. \textunderscore piens\textunderscore )}
\end{itemize}
Muito piedoso, muito pio; piíssimo. Cf. Filinto, III, 101.
\section{Piérico}
\begin{itemize}
\item {Grp. gram.:adj.}
\end{itemize}
Diz-se de um ácido, proposto para a fabricação de uma nova espécie de cerveja económica.
\section{Piéride}
\begin{itemize}
\item {Grp. gram.:f.}
\end{itemize}
\begin{itemize}
\item {Proveniência:(De \textunderscore Piéride\textunderscore , n. p.)}
\end{itemize}
Gênero de insectos, a que pertence a lagarta das couves.
\section{Piério}
\begin{itemize}
\item {Grp. gram.:adj.}
\end{itemize}
\begin{itemize}
\item {Utilização:Poét.}
\end{itemize}
\begin{itemize}
\item {Proveniência:(Lat. \textunderscore pierius\textunderscore )}
\end{itemize}
Relativo ás musas ou á poesia.
\section{Pietismo}
\begin{itemize}
\item {Grp. gram.:m.}
\end{itemize}
\begin{itemize}
\item {Utilização:Fam.}
\end{itemize}
\begin{itemize}
\item {Proveniência:(Fr. \textunderscore pietisme\textunderscore )}
\end{itemize}
Seita christan, que se desenvolveu na Prússia, e cuja doutrina era a plena subordinação á letra do Evangelho.
O mesmo que \textunderscore fanatismo\textunderscore .
\section{Pietista}
\begin{itemize}
\item {Grp. gram.:m.}
\end{itemize}
Sectário do pietismo.
\section{Piezómetro}
\begin{itemize}
\item {Grp. gram.:m.}
\end{itemize}
\begin{itemize}
\item {Utilização:Phýs.}
\end{itemize}
\begin{itemize}
\item {Proveniência:(Do gr. \textunderscore piezein\textunderscore  + \textunderscore metron\textunderscore )}
\end{itemize}
Apparelho, para avaliar a compressibilidade ou a tensão dos líquidos.
\section{Pífano}
\begin{itemize}
\item {Grp. gram.:m.}
\end{itemize}
O mesmo que \textunderscore pifaro\textunderscore .
\section{Pifão}
\begin{itemize}
\item {Grp. gram.:m.}
\end{itemize}
\begin{itemize}
\item {Utilização:Chul.}
\end{itemize}
O mesmo que \textunderscore bebedeira\textunderscore :«\textunderscore ...com quantos litros hei de apanhar um pifão.\textunderscore »(Cantiga pop.)
\section{Pifaro}
\begin{itemize}
\item {Grp. gram.:m.}
\end{itemize}
Instrumento popular e pastoril, do feitio de uma frauta, mas sem chaves.
(Cast. \textunderscore pifaro\textunderscore )
\section{Pifiamente}
\begin{itemize}
\item {Grp. gram.:adv.}
\end{itemize}
De modo pífio; grosseiramente.
\section{Pífio}
\begin{itemize}
\item {Grp. gram.:adj.}
\end{itemize}
\begin{itemize}
\item {Utilização:Pleb.}
\end{itemize}
\begin{itemize}
\item {Proveniência:(Do cast. \textunderscore pifiar\textunderscore )}
\end{itemize}
Reles; vil; grosseiro.
\section{Pifre}
\begin{itemize}
\item {Grp. gram.:m.}
\end{itemize}
Fórma p. us. de \textunderscore pífaro\textunderscore :«\textunderscore ...os frades vem marchando ao som dos pifres.\textunderscore »Filinto, V, 310.
\section{Pigaça}
\begin{itemize}
\item {Grp. gram.:f.  e  adj.}
\end{itemize}
Diz-se de uma variedade de pêra, de que há várias espécies, como a \textunderscore pigaça-gigante\textunderscore , a \textunderscore pigaça-de-inverno\textunderscore , a \textunderscore pigaça-do-minho\textunderscore , a \textunderscore pigaça-do-verão\textunderscore .
(Cp. \textunderscore pigarça\textunderscore )
\section{Pigar}
\begin{itemize}
\item {Grp. gram.:v. t.  e  i.}
\end{itemize}
\begin{itemize}
\item {Utilização:Gír.}
\end{itemize}
O mesmo que \textunderscore roubar\textunderscore .
\section{Pigarça}
\begin{itemize}
\item {Grp. gram.:f.  e  adj.}
\end{itemize}
\begin{itemize}
\item {Proveniência:(De \textunderscore pigarço\textunderscore )}
\end{itemize}
O mesmo que \textunderscore pigaça\textunderscore .
\section{Pigarço}
\begin{itemize}
\item {Grp. gram.:adj.}
\end{itemize}
Diz-se do cavallo, malhado de preto e branco ou de côr grisalha.
\section{Pigarrar}
\begin{itemize}
\item {Grp. gram.:v. i.}
\end{itemize}
O mesmo que \textunderscore pigarrear\textunderscore . Cf. Júl. Lour. Pinto, \textunderscore Senhor Deput.\textunderscore , 410.
\section{Pigarrear}
\begin{itemize}
\item {Grp. gram.:v. i.}
\end{itemize}
Têr pigarro, tossir com pigarro. Cf. Eça, \textunderscore P. Basilio\textunderscore , 187.
\section{Pigarrento}
\begin{itemize}
\item {Grp. gram.:adj.}
\end{itemize}
Que tem pigarro.
Que póde causar pigarro: \textunderscore tabaco pigarrento\textunderscore .
\section{Pigarro}
\begin{itemize}
\item {Grp. gram.:m.}
\end{itemize}
A dor ou embaraço na garganta, produzido pela adherência de mucosidades, pelo fumo ou por outra causa.
(Por \textunderscore picarro\textunderscore , de \textunderscore pico\textunderscore ^1)
\section{Pigarro}
\begin{itemize}
\item {Grp. gram.:m.}
\end{itemize}
\begin{itemize}
\item {Utilização:Prov.}
\end{itemize}
\begin{itemize}
\item {Utilização:minh.}
\end{itemize}
Pau, que sustenta o cabeçalho do carro, para que êste, tirada a canga, não poise no chão; estronca.
(Por \textunderscore pegarro\textunderscore , de \textunderscore pegar\textunderscore ?)
\section{Pigarroso}
\begin{itemize}
\item {Grp. gram.:adj.}
\end{itemize}
Que tem pigarro; que revela pigarro; produzido por pigarro. Cf. Camillo, \textunderscore Corja\textunderscore , 215.
\section{Pigmentação}
\begin{itemize}
\item {Grp. gram.:f.}
\end{itemize}
\begin{itemize}
\item {Utilização:Pathol.}
\end{itemize}
\begin{itemize}
\item {Proveniência:(De \textunderscore pigmentar\textunderscore )}
\end{itemize}
Producção de uma substância còrante na economia animal.
\section{Pigmentado}
\begin{itemize}
\item {Grp. gram.:adj.}
\end{itemize}
\begin{itemize}
\item {Proveniência:(Lat. \textunderscore pigmentatus\textunderscore )}
\end{itemize}
Que tem pigmento.
\section{Pigmentar}
\begin{itemize}
\item {Grp. gram.:v. t.}
\end{itemize}
\begin{itemize}
\item {Proveniência:(De \textunderscore pigmento\textunderscore )}
\end{itemize}
Dar a côr da pelle a.
Dar côr a. Cf. Júl. Lour. Pinto, \textunderscore Senhor Deput.\textunderscore , 115.
\section{Pigmentário}
\begin{itemize}
\item {Grp. gram.:m.}
\end{itemize}
\begin{itemize}
\item {Proveniência:(Lat. \textunderscore pigmentarius\textunderscore )}
\end{itemize}
Vendedor ou fabricante de cosméticos, perfumes, pomadas, etc., na antiga Roma.
\section{Pigmento}
\begin{itemize}
\item {Grp. gram.:m.}
\end{itemize}
\begin{itemize}
\item {Proveniência:(Lat. \textunderscore pigmentum\textunderscore )}
\end{itemize}
Substância escura ou roxa, que determina a côr da pelle, nos homens e nos animaes.
\section{Pignoratício}
\begin{itemize}
\item {Grp. gram.:adj.}
\end{itemize}
\begin{itemize}
\item {Utilização:Jur.}
\end{itemize}
\begin{itemize}
\item {Proveniência:(Lat. \textunderscore pignotaticius\textunderscore )}
\end{itemize}
Dizia-se do título de venda, passado pelo devedor ao credor, sob estipulação de que o vendedor poderia retirar os bens vendidos durante certo tempo, e que delles gozaria a título de aluguel, mediante certa somma, que ordinariamente era igual aos juros da somma emprestada e pela qual se fez a venda.
Relativo a penhor. Cf. Ferreira Borges, \textunderscore Diccion. Jur.\textunderscore 
\section{Pigotita}
\begin{itemize}
\item {Grp. gram.:f.}
\end{itemize}
Rocha aluminosa, que se encontra em Inglaterra.
\section{Pigotito}
\begin{itemize}
\item {Grp. gram.:m.}
\end{itemize}
O mesmo ou melhor que \textunderscore pigotita\textunderscore .
\section{Pigro}
\begin{itemize}
\item {Grp. gram.:adj.}
\end{itemize}
\begin{itemize}
\item {Utilização:Poét.}
\end{itemize}
\begin{itemize}
\item {Proveniência:(Do lat. \textunderscore piger\textunderscore )}
\end{itemize}
Froixo, indolente.
\section{Piguancha}
\begin{itemize}
\item {Grp. gram.:f.}
\end{itemize}
\begin{itemize}
\item {Utilização:Bras. do S}
\end{itemize}
O mesmo que \textunderscore chininha\textunderscore .
\section{Pihum}
\begin{itemize}
\item {Grp. gram.:m.}
\end{itemize}
\begin{itemize}
\item {Utilização:Bras. do N}
\end{itemize}
Insecto, muito incômmodo, da região do Purus.
\section{Pijama}
\begin{itemize}
\item {Grp. gram.:m.}
\end{itemize}
Vestuário caseiro, geralmente amplo e ligeiro, composto de calças e de um casaco curto, que se abotôa até o pescoço.
\section{Pijeiro}
\begin{itemize}
\item {Grp. gram.:m.}
\end{itemize}
\begin{itemize}
\item {Utilização:Prov.}
\end{itemize}
\begin{itemize}
\item {Utilização:minh.}
\end{itemize}
Talhadoiro, ponto em que se reparte a água, que, vindo de um açude ou tanque, é destinada a regar, alternadamente, terras contíguas.
\section{Pijerecum}
\begin{itemize}
\item {Grp. gram.:m.}
\end{itemize}
\begin{itemize}
\item {Utilização:Bras}
\end{itemize}
Planta anonácea, originária da África.
\section{Piladeira}
\begin{itemize}
\item {Grp. gram.:f.}
\end{itemize}
\begin{itemize}
\item {Utilização:T. da Índia port}
\end{itemize}
\begin{itemize}
\item {Proveniência:(De \textunderscore pilar\textunderscore ^1)}
\end{itemize}
Mulhér, que descasca arroz no mussó.
\section{Pilado}
\begin{itemize}
\item {Grp. gram.:m.}
\end{itemize}
\begin{itemize}
\item {Grp. gram.:Adj.}
\end{itemize}
\begin{itemize}
\item {Proveniência:(De \textunderscore pilar\textunderscore ^1)}
\end{itemize}
Espécie de crustáceo.
Caranguejo vulgar das águas costeiras, que se emprega como adubo das terras.
Diz-se das castanhas descascadas e sêcas.
\section{Pilador}
\begin{itemize}
\item {Grp. gram.:m.  e  adj.}
\end{itemize}
O que pila.
\section{Pilano}
\begin{itemize}
\item {Grp. gram.:m.}
\end{itemize}
\begin{itemize}
\item {Proveniência:(Lat. \textunderscore pilanus\textunderscore )}
\end{itemize}
Soldado romano, armado de dardo.
\section{Pilão}
\begin{itemize}
\item {Grp. gram.:m.}
\end{itemize}
\begin{itemize}
\item {Utilização:Bras}
\end{itemize}
\begin{itemize}
\item {Utilização:Prov.}
\end{itemize}
\begin{itemize}
\item {Utilização:trasm.}
\end{itemize}
\begin{itemize}
\item {Proveniência:(De \textunderscore pilar\textunderscore ^1)}
\end{itemize}
Peça, com que se móem substâncias no gral.
Pêso, empregado na balança romana.
Maço dos moínhos, em que se pisa o papel, a casca de carvalho, etc.
Pão de açúcar, em fórma cónica.
Gral de pau rijo, onde se descasca e tritura arroz, café, milho, etc.
Órgão operador do bate-estacas.
O mesmo que \textunderscore piadeiro\textunderscore ^2.
\section{Pilão}
\begin{itemize}
\item {Grp. gram.:m.}
\end{itemize}
\begin{itemize}
\item {Utilização:Prov.}
\end{itemize}
\begin{itemize}
\item {Utilização:beir.}
\end{itemize}
Terreiro ou espécie de picadeiro circular, cujo centro é occupado pelo picador, que com uma corda tensa adestra, sem mudar de lugar, o cavallo na marcha, obrigando-o a girar, a trote, a galope ou a passo, em volta do terreiro.
\section{Pilão}
\begin{itemize}
\item {Grp. gram.:m.}
\end{itemize}
\begin{itemize}
\item {Utilização:Prov.}
\end{itemize}
\begin{itemize}
\item {Utilização:beir.}
\end{itemize}
Pelintra.
Miserável. (Colhido em Lamego)
\section{Pila-pila!}
\begin{itemize}
\item {Grp. gram.:interj.}
\end{itemize}
\begin{itemize}
\item {Utilização:Prov.}
\end{itemize}
(Serve para chamar gallinhas)
\section{Pilar}
\begin{itemize}
\item {Grp. gram.:v. t.}
\end{itemize}
\begin{itemize}
\item {Proveniência:(Lat. \textunderscore pilare\textunderscore )}
\end{itemize}
Pisar no pilão.
Descascar; descascar para secar, (falando-se de castanhas).
\section{Pilar}
\begin{itemize}
\item {Grp. gram.:m.}
\end{itemize}
\begin{itemize}
\item {Proveniência:(Lat. \textunderscore pilaris\textunderscore )}
\end{itemize}
Simples columna, que sustenta uma construcção qualquer.
\section{Pilarete}
\begin{itemize}
\item {fónica:larê}
\end{itemize}
\begin{itemize}
\item {Grp. gram.:m.}
\end{itemize}
Pequeno pilar. Cf. Sousa, \textunderscore Vida do Arceb.\textunderscore , III, 218.
\section{Pilarte}
\begin{itemize}
\item {Grp. gram.:m.}
\end{itemize}
Antiga moéda de prata, portuguesa, de 13 reis e 2 ceitís, primitivamente, e, depois, de 7 ceitís.
\section{Pilastra}
\begin{itemize}
\item {Grp. gram.:f.}
\end{itemize}
\begin{itemize}
\item {Proveniência:(It. \textunderscore pilastro\textunderscore )}
\end{itemize}
Pilar de quatro faces, em geral adherente por uma dellas a uma edificação ou parede.
\section{Pilastrão}
\begin{itemize}
\item {Grp. gram.:m.}
\end{itemize}
Plastra grande. Cf. Sousa, \textunderscore Vida do Arceb\textunderscore .
\section{Pilatas}
\begin{itemize}
\item {Grp. gram.:m.}
\end{itemize}
\begin{itemize}
\item {Utilização:Prov.}
\end{itemize}
\begin{itemize}
\item {Utilização:trasm.}
\end{itemize}
\begin{itemize}
\item {Utilização:Fam.}
\end{itemize}
O mesmo que \textunderscore garoto\textunderscore .
\section{Pilatos}
\begin{itemize}
\item {Grp. gram.:m.}
\end{itemize}
Nome, que se dava a uma pequena bandeira, usada na procissão de dois de Novembro.
\section{Pilau}
\begin{itemize}
\item {Grp. gram.:m.}
\end{itemize}
\begin{itemize}
\item {Proveniência:(Do pers. \textunderscore pilar\textunderscore )}
\end{itemize}
O mesmo ou melhor que \textunderscore polau\textunderscore .
\section{Pildar}
\begin{itemize}
\item {Grp. gram.:v. i.}
\end{itemize}
\begin{itemize}
\item {Utilização:Chul.}
\end{itemize}
Fugir; esgueirar-se.
\section{Pildra}
\begin{itemize}
\item {Grp. gram.:f.}
\end{itemize}
\begin{itemize}
\item {Utilização:Prov.}
\end{itemize}
\begin{itemize}
\item {Utilização:dur.}
\end{itemize}
\begin{itemize}
\item {Utilização:Gír.}
\end{itemize}
O mesmo que \textunderscore tarambola\textunderscore .
Catre; cama.
(Cp. \textunderscore pilra\textunderscore )
\section{Pilé}
\begin{itemize}
\item {Grp. gram.:m.  e  adj.}
\end{itemize}
\begin{itemize}
\item {Proveniência:(Fr. \textunderscore pilé\textunderscore )}
\end{itemize}
Diz-se do açúcar crystallizado, em fragmentos ou lascas.
\section{Pileatos}
\begin{itemize}
\item {Grp. gram.:m. pl.}
\end{itemize}
\begin{itemize}
\item {Proveniência:(Lat. \textunderscore pileatus\textunderscore )}
\end{itemize}
Romanos que, nas festas saturnaes, usavam píleo.
\section{Pileca}
\begin{itemize}
\item {Grp. gram.:f.}
\end{itemize}
\begin{itemize}
\item {Utilização:Pop.}
\end{itemize}
Cavalgadura pequena e ordinária.
\section{Pilecra}
\begin{itemize}
\item {Grp. gram.:f.}
\end{itemize}
\begin{itemize}
\item {Utilização:Prov.}
\end{itemize}
\begin{itemize}
\item {Utilização:trasm.}
\end{itemize}
O mesmo que \textunderscore pileca\textunderscore .
Homem fraco.
(Cp. \textunderscore boneca\textunderscore  e \textunderscore bonecra\textunderscore )
\section{Pilenta}
\begin{itemize}
\item {Grp. gram.:f.}
\end{itemize}
\begin{itemize}
\item {Proveniência:(Lat. \textunderscore pilentum\textunderscore )}
\end{itemize}
Coche de molas, como que suspenso sôbre duas rodas, usado pelas matronas romanas.
\section{Pilento}
\begin{itemize}
\item {Grp. gram.:m.}
\end{itemize}
\begin{itemize}
\item {Proveniência:(Lat. \textunderscore pilentum\textunderscore )}
\end{itemize}
Coche de molas, como que suspenso sôbre duas rodas, usado pelas matronas romanas.
\section{Píleo}
\begin{itemize}
\item {Grp. gram.:m.}
\end{itemize}
\begin{itemize}
\item {Proveniência:(Lat. \textunderscore pileus\textunderscore )}
\end{itemize}
Barrete próprio de Bispos.
Barrete de feltro, do feitio de metade de um ovo, o qual se adaptava justamente á cabeça, e que os Romanos usavam nos saturnaes e noutras solennidades.--Dado a um escravo, era sinal de alforria.
\section{Pileorrhiza}
\begin{itemize}
\item {Grp. gram.:m.}
\end{itemize}
\begin{itemize}
\item {Utilização:Bot.}
\end{itemize}
\begin{itemize}
\item {Proveniência:(Do gr. \textunderscore pileos\textunderscore  + \textunderscore rhiza\textunderscore )}
\end{itemize}
Espécie de coifa, que reveste a extremidade da raíz.
\section{Pileorriza}
\begin{itemize}
\item {Grp. gram.:m.}
\end{itemize}
\begin{itemize}
\item {Utilização:Bot.}
\end{itemize}
\begin{itemize}
\item {Proveniência:(Do gr. \textunderscore pileos\textunderscore  + \textunderscore rhiza\textunderscore )}
\end{itemize}
Espécie de coifa, que reveste a extremidade da raíz.
\section{Pileque}
\begin{itemize}
\item {Grp. gram.:m.}
\end{itemize}
\begin{itemize}
\item {Utilização:Bras}
\end{itemize}
Argola de borracha.
\section{Pileque}
\begin{itemize}
\item {Grp. gram.:m.}
\end{itemize}
\begin{itemize}
\item {Utilização:Bras}
\end{itemize}
Ligeira embriaguez.
\section{Pileu}
\begin{itemize}
\item {Grp. gram.:m.}
\end{itemize}
\begin{itemize}
\item {Utilização:Prov.}
\end{itemize}
\begin{itemize}
\item {Utilização:trasm.}
\end{itemize}
Bigorrilhas, bisbórria, safardana.
(Cp. \textunderscore pilho\textunderscore )
\section{Pilha}
\begin{itemize}
\item {Grp. gram.:f.}
\end{itemize}
\begin{itemize}
\item {Proveniência:(Do lat. hyp. \textunderscore pilea\textunderscore )}
\end{itemize}
Montão de coisas; ajuntamento.
Apparelho, em que se desenvolvem correntes eléctricas.
\section{Pilha}
\begin{itemize}
\item {Grp. gram.:m.}
\end{itemize}
\begin{itemize}
\item {Proveniência:(De \textunderscore pilhar\textunderscore )}
\end{itemize}
Nome de um jôgo de cartas.
Jôgo de rapazes, com um bocado de pau quadrado, que tem numa das faces a palavra \textunderscore pilha\textunderscore .
Acto de pilhar, gatunice, roubo:«\textunderscore ...andar á pilha da mais gorda gallinha\textunderscore ». Camillo, \textunderscore Quéda\textunderscore , 243. (2.^a ed., 1887).
\section{Pilha-gallinhas}
\begin{itemize}
\item {Grp. gram.:m.}
\end{itemize}
\begin{itemize}
\item {Utilização:Chul.}
\end{itemize}
Gabão ordinário e grande.
\section{Pilhagem}
\begin{itemize}
\item {Grp. gram.:f.}
\end{itemize}
Acto ou effeito de pilhar; aquillo que se pilhou.
\section{Pilhante}
\begin{itemize}
\item {Grp. gram.:m. ,  f.  e  adj.}
\end{itemize}
Pessôa que pilha; larápio.
\section{Pilhar}
\begin{itemize}
\item {Grp. gram.:v. t.}
\end{itemize}
\begin{itemize}
\item {Proveniência:(Do lat. hyp. \textunderscore piliare\textunderscore )}
\end{itemize}
Apanhar.
Agarrar.
Furtar.
Fazer saque em.
Encontrar, surprehendendo.
\section{Pilha-ratos}
\begin{itemize}
\item {Grp. gram.:m.}
\end{itemize}
Ave de rapina, (\textunderscore circus cyaneus\textunderscore , Lin.).
\section{Pilharengo}
\begin{itemize}
\item {Grp. gram.:adj.}
\end{itemize}
\begin{itemize}
\item {Proveniência:(De \textunderscore pilhar\textunderscore )}
\end{itemize}
Relativo a pilhagem ou a larápio.
Que tem o hábito de ratoneiro. Cf. Camillo, \textunderscore Brasileira\textunderscore , 247.
\section{Pilharete}
\begin{itemize}
\item {fónica:lharê}
\end{itemize}
\begin{itemize}
\item {Grp. gram.:m.}
\end{itemize}
\begin{itemize}
\item {Utilização:Pop.}
\end{itemize}
\begin{itemize}
\item {Proveniência:(De \textunderscore pilhar\textunderscore )}
\end{itemize}
Ratoneiro, larápio.
\section{Pilharia}
\begin{itemize}
\item {Grp. gram.:f.}
\end{itemize}
\begin{itemize}
\item {Utilização:Ant.}
\end{itemize}
\begin{itemize}
\item {Utilização:Fam.}
\end{itemize}
\begin{itemize}
\item {Proveniência:(De \textunderscore pilhar\textunderscore )}
\end{itemize}
O mesmo que \textunderscore pilhagem\textunderscore .
\section{Pilha-três}
\begin{itemize}
\item {Grp. gram.:m.}
\end{itemize}
Jôgo popular, em que diversas pessôas, formando grupos de duas, fogem em diversas direcções, procurando evitar que um terceiro apanhe alguma.
\section{Pilheira}
\begin{itemize}
\item {Grp. gram.:f.}
\end{itemize}
\begin{itemize}
\item {Utilização:Prov.}
\end{itemize}
\begin{itemize}
\item {Utilização:trasm.}
\end{itemize}
\begin{itemize}
\item {Proveniência:(De \textunderscore pilha\textunderscore ^1)}
\end{itemize}
Lugar, onde há coisas empilhadas.
Lugar, annexo á lareira, no qual se juntam as cinzas.
Cantareira, aberta na parede.
\section{Pilheiro}
\begin{itemize}
\item {Grp. gram.:m.}
\end{itemize}
\begin{itemize}
\item {Utilização:Prov.}
\end{itemize}
\begin{itemize}
\item {Utilização:dur.}
\end{itemize}
\begin{itemize}
\item {Proveniência:(De \textunderscore pilha\textunderscore ^1)}
\end{itemize}
Depósito de águas para qualquer serviço.
Orifícios nas paredes dos socalcos, por onde sáem e crescem as vides.
\section{Pilhéria}
\begin{itemize}
\item {Grp. gram.:f.}
\end{itemize}
\begin{itemize}
\item {Utilização:Pop.}
\end{itemize}
\begin{itemize}
\item {Proveniência:(De \textunderscore pilha\textunderscore ^1)}
\end{itemize}
Chiste; facécia.
Qualidade de quem é espirituoso.
\section{Pilheriar}
\begin{itemize}
\item {Grp. gram.:v. i.}
\end{itemize}
\begin{itemize}
\item {Utilização:Neol.}
\end{itemize}
Dizer pilhérias.
\section{Pilheta}
\begin{itemize}
\item {fónica:lhê}
\end{itemize}
\begin{itemize}
\item {Grp. gram.:f.}
\end{itemize}
\begin{itemize}
\item {Proveniência:(De \textunderscore pilha\textunderscore ^1)}
\end{itemize}
Vaso ou selha, mais estreita no fundo do que nas bordas; gamela.
\section{Pilho}
\begin{itemize}
\item {Grp. gram.:m.}
\end{itemize}
\begin{itemize}
\item {Utilização:Pop.}
\end{itemize}
\begin{itemize}
\item {Proveniência:(De \textunderscore pilhar\textunderscore )}
\end{itemize}
Malandro; patife; gatuno:«\textunderscore ...honrados pilhos, fóra!\textunderscore »Sousa Monteiro, \textunderscore Falstaff\textunderscore , 24.«\textunderscore Não há naquella feira um tuno, um roto, um pilho\textunderscore ». Castilho, \textunderscore Sabichonas\textunderscore , 182.
\section{Pília}
\begin{itemize}
\item {Grp. gram.:f.}
\end{itemize}
Planta ornamental.
\section{Pilídio}
\begin{itemize}
\item {Grp. gram.:m.}
\end{itemize}
\begin{itemize}
\item {Utilização:Zool.}
\end{itemize}
Larva dos turbelariados.
\section{Pilífero}
\begin{itemize}
\item {Grp. gram.:adj.}
\end{itemize}
\begin{itemize}
\item {Utilização:Bot.}
\end{itemize}
\begin{itemize}
\item {Proveniência:(Do lat. \textunderscore pilus\textunderscore  + \textunderscore ferre\textunderscore )}
\end{itemize}
Que tem pêlos.
\section{Piliforme}
\begin{itemize}
\item {Grp. gram.:adj.}
\end{itemize}
\begin{itemize}
\item {Proveniência:(Do lat. \textunderscore pilus\textunderscore  + \textunderscore forma\textunderscore )}
\end{itemize}
Que tem fórma de pêlo.
\section{Pilinha-pilinha!}
\begin{itemize}
\item {Grp. gram.:interj.}
\end{itemize}
(para chamar gallinhas)
(Cp. \textunderscore pila-pila!\textunderscore )
\section{Pilípede}
\begin{itemize}
\item {Grp. gram.:adj.}
\end{itemize}
\begin{itemize}
\item {Proveniência:(Do lat. \textunderscore pilus\textunderscore  + \textunderscore pes\textunderscore , \textunderscore pedis\textunderscore )}
\end{itemize}
Que tem pêlos nos pés.
\section{Pilo}
\begin{itemize}
\item {Grp. gram.:m.}
\end{itemize}
\begin{itemize}
\item {Proveniência:(Lat. \textunderscore pilum\textunderscore )}
\end{itemize}
Espécie de dardo, entre os Romanos.
\section{Piloada}
\begin{itemize}
\item {Grp. gram.:f.}
\end{itemize}
Pancada com o pilão.
\section{Pilobóleas}
\begin{itemize}
\item {Grp. gram.:f.}
\end{itemize}
\begin{itemize}
\item {Utilização:Bot.}
\end{itemize}
\begin{itemize}
\item {Proveniência:(De \textunderscore pilóbolo\textunderscore )}
\end{itemize}
Tríbo de cogumelos.
\section{Pilóbolo}
\begin{itemize}
\item {Grp. gram.:m.}
\end{itemize}
\begin{itemize}
\item {Proveniência:(Do gr. \textunderscore pilos\textunderscore  + \textunderscore bolos\textunderscore )}
\end{itemize}
Gênero de cogumelos.
\section{Pilocarpadas}
\begin{itemize}
\item {Grp. gram.:f. pl.}
\end{itemize}
\begin{itemize}
\item {Proveniência:(De \textunderscore pilocarpo\textunderscore )}
\end{itemize}
Tríbo de plantas pilocarpíneas.
\section{Pilocarpina}
\begin{itemize}
\item {Grp. gram.:f.}
\end{itemize}
\begin{itemize}
\item {Proveniência:(De \textunderscore policarpo\textunderscore )}
\end{itemize}
Alcalóide medicinal, extrahido de plantas policarpíneas.
\section{Pilocarpíneas}
\begin{itemize}
\item {Grp. gram.:f. pl.}
\end{itemize}
Família de plantas, a que pertence o policarpo, a estercúlia e outras.
\section{Pilocas!}
\begin{itemize}
\item {Grp. gram.:interj.}
\end{itemize}
\begin{itemize}
\item {Utilização:Prov.}
\end{itemize}
\begin{itemize}
\item {Utilização:trasm.}
\end{itemize}
O mesmo que \textunderscore pila-pila!\textunderscore 
\section{Pilóia}
\begin{itemize}
\item {Grp. gram.:f.}
\end{itemize}
\begin{itemize}
\item {Utilização:Bras}
\end{itemize}
O mesmo que \textunderscore mandureba\textunderscore ; aguardente.
\section{Piloira}
\begin{itemize}
\item {Grp. gram.:f.}
\end{itemize}
\begin{itemize}
\item {Utilização:Bras. do N}
\end{itemize}
Accesso de loucura.
Vertigem.
\section{Piloirada}
\begin{itemize}
\item {Grp. gram.:f.}
\end{itemize}
\begin{itemize}
\item {Utilização:Bras. do N}
\end{itemize}
\begin{itemize}
\item {Proveniência:(De \textunderscore piloira\textunderscore )}
\end{itemize}
Acção própria de quem não tem juízo.
\section{Pilongo}
\begin{itemize}
\item {Grp. gram.:m.}
\end{itemize}
Casta de uva.
\section{Pilosela}
\begin{itemize}
\item {Grp. gram.:f.}
\end{itemize}
\begin{itemize}
\item {Proveniência:(Lat. \textunderscore pilosella\textunderscore )}
\end{itemize}
Planta chicoriácea, de caule e fôlhas pilosas.
\section{Pilosella}
\begin{itemize}
\item {Grp. gram.:f.}
\end{itemize}
\begin{itemize}
\item {Proveniência:(Lat. \textunderscore pilosella\textunderscore )}
\end{itemize}
Planta chicoriácea, de caule e fôlhas pilosas.
\section{Pilosella-das-boticas}
\begin{itemize}
\item {Grp. gram.:f.}
\end{itemize}
Planta, da fam. das compostas, (\textunderscore hieracium pilosella\textunderscore , Lin.). Cf. P. Coutinho, \textunderscore Flora\textunderscore , 678.
\section{Pilosidade}
\begin{itemize}
\item {Grp. gram.:f.}
\end{itemize}
Qualidade do que é piloso.
\section{Pilosismo}
\begin{itemize}
\item {Grp. gram.:m.}
\end{itemize}
\begin{itemize}
\item {Proveniência:(Do lat. \textunderscore pilosus\textunderscore )}
\end{itemize}
Desenvolvimento anormal de pêlos, num ponto em que geralmente pouco ou nada crescem.
\section{Piloso}
\begin{itemize}
\item {Grp. gram.:adj.}
\end{itemize}
\begin{itemize}
\item {Utilização:Bot.}
\end{itemize}
\begin{itemize}
\item {Proveniência:(Lat. \textunderscore pilosus\textunderscore )}
\end{itemize}
O mesmo que \textunderscore peludo\textunderscore .
O mesmo que \textunderscore pubescente\textunderscore .
\section{Pilota}
\begin{itemize}
\item {Grp. gram.:f.}
\end{itemize}
\begin{itemize}
\item {Utilização:Pop.}
\end{itemize}
\begin{itemize}
\item {Proveniência:(De \textunderscore pilar\textunderscore ^1)}
\end{itemize}
Estafa.
Prejuizo.
Derrota.
Critica severa.
Arguição de defeitos.
\section{Pilotagem}
\begin{itemize}
\item {Grp. gram.:f.}
\end{itemize}
Offício ou arte de pilôto.
Serviços prestados pelo pilôto.
\section{Pilotar}
\begin{itemize}
\item {Grp. gram.:v. t.}
\end{itemize}
\begin{itemize}
\item {Grp. gram.:V. i.}
\end{itemize}
Governar, como pilôto.
Exercer as funcções de pilôto.
\section{Pilotear}
\begin{itemize}
\item {Grp. gram.:v. t.  e  i.}
\end{itemize}
O mesmo que \textunderscore pilotar\textunderscore .
\section{Pilotear}
\begin{itemize}
\item {Grp. gram.:v. t.}
\end{itemize}
\begin{itemize}
\item {Utilização:Fam.}
\end{itemize}
Dar pilota em.
Vencer moralmente, acompanhando a victória com motejos.
Arguir ironicamente.
Corrigir, criticando.
\section{Pilôto}
\begin{itemize}
\item {Grp. gram.:m.}
\end{itemize}
\begin{itemize}
\item {Utilização:Fig.}
\end{itemize}
\begin{itemize}
\item {Utilização:Bras}
\end{itemize}
\begin{itemize}
\item {Utilização:Bot.}
\end{itemize}
\begin{itemize}
\item {Proveniência:(It. \textunderscore piloto\textunderscore )}
\end{itemize}
Aquelle que dirige uma embarcação mercante, subordinado ao commandante.
O que dirige um navio, só á entrada de um pôrto.
Guia, director.
Pequeno peixe, que costuma andar adeante do tubarão.
Planta hypericácea, (\textunderscore helianthemum lasianthum\textunderscore , Lam.).
\section{Pilôto}
\begin{itemize}
\item {Grp. gram.:adj.}
\end{itemize}
Diz-se de uma variedade de pano, espécie de briche.
(Talvez por \textunderscore peloto\textunderscore , de \textunderscore pêlo\textunderscore ^1)
\section{Pilôto}
\begin{itemize}
\item {Grp. gram.:m.}
\end{itemize}
\begin{itemize}
\item {Utilização:Bras do N}
\end{itemize}
Indivíduo ou animal muito gordo.
\section{Pilra}
\begin{itemize}
\item {Grp. gram.:f.}
\end{itemize}
\begin{itemize}
\item {Utilização:Ant.}
\end{itemize}
\begin{itemize}
\item {Utilização:Gír.}
\end{itemize}
Cama.
(Cp. \textunderscore pílula\textunderscore )
\section{Pilrão}
\begin{itemize}
\item {Grp. gram.:m.}
\end{itemize}
Peixe da costa marítima de Portugal.
\section{Pilrete}
\begin{itemize}
\item {fónica:rê}
\end{itemize}
\begin{itemize}
\item {Grp. gram.:m.}
\end{itemize}
\begin{itemize}
\item {Utilização:Chul.}
\end{itemize}
\begin{itemize}
\item {Utilização:Prov.}
\end{itemize}
\begin{itemize}
\item {Utilização:trasm.}
\end{itemize}
\begin{itemize}
\item {Proveniência:(De \textunderscore pilrito\textunderscore ?)}
\end{itemize}
Homem de pequena estatura; homúnculo.
Rapariga buliçosa e tagarela.
\section{Pilriteiro}
\begin{itemize}
\item {Grp. gram.:m.}
\end{itemize}
\begin{itemize}
\item {Proveniência:(De \textunderscore pilrito\textunderscore )}
\end{itemize}
Planta da fam. das pomáceas, (\textunderscore crataegus oxyacantha\textunderscore ).
\section{Pilrito}
\begin{itemize}
\item {Grp. gram.:m.}
\end{itemize}
Fruto do pilriteiro.
(Por \textunderscore pirlito\textunderscore , do lat. \textunderscore pirula\textunderscore , de \textunderscore pirum\textunderscore )
\section{Pílula}
\begin{itemize}
\item {Grp. gram.:f.}
\end{itemize}
\begin{itemize}
\item {Utilização:Fig.}
\end{itemize}
\begin{itemize}
\item {Utilização:Gír.}
\end{itemize}
\begin{itemize}
\item {Proveniência:(Lat. \textunderscore pilula\textunderscore )}
\end{itemize}
Espécie de confeito pharmacêutico, destinado a engulir-se inteiro.
Aquillo que se supporta difficilmente; coisa desagradável.
Cama.
\section{Pilulador}
\begin{itemize}
\item {Grp. gram.:m.}
\end{itemize}
\begin{itemize}
\item {Proveniência:(De \textunderscore pilular\textunderscore )}
\end{itemize}
Instrumento, para dividir a massa, de que se fazem pilulas.
\section{Pilular}
\begin{itemize}
\item {Grp. gram.:adj.}
\end{itemize}
Que tem fórma ou natureza de pílula.
Que se póde dividir em pílulas.
\section{Pilulária}
\begin{itemize}
\item {Grp. gram.:f.}
\end{itemize}
\begin{itemize}
\item {Proveniência:(De \textunderscore pílula\textunderscore )}
\end{itemize}
Gênero de plantas, das regiões pantanosas da Europa.
\section{Piluleiro}
\begin{itemize}
\item {Grp. gram.:m.}
\end{itemize}
\begin{itemize}
\item {Utilização:Bras}
\end{itemize}
Vaso ou utensílio de madeira ou metal, com que se preparam pílulas.
\section{Pilungo}
\begin{itemize}
\item {Grp. gram.:m.}
\end{itemize}
\begin{itemize}
\item {Utilização:Bras. do S}
\end{itemize}
Cavallo ruivo.
\section{Pimbes}
\begin{itemize}
\item {Grp. gram.:m. pl.}
\end{itemize}
O mesmo que \textunderscore maraves\textunderscore .
\section{Pimélia}
\begin{itemize}
\item {Grp. gram.:f.}
\end{itemize}
\begin{itemize}
\item {Proveniência:(Do gr. \textunderscore pimele\textunderscore )}
\end{itemize}
Gênero de insectos coleópteros heterómeros.
\section{Pimélico}
\begin{itemize}
\item {Grp. gram.:adj.}
\end{itemize}
\begin{itemize}
\item {Proveniência:(Do gr. \textunderscore pimele\textunderscore )}
\end{itemize}
Diz-se de um ácido, resultante da acção do ácido azótico no ácido oleico.
\section{Pimelite}
\begin{itemize}
\item {Grp. gram.:f.}
\end{itemize}
\begin{itemize}
\item {Utilização:Med.}
\end{itemize}
\begin{itemize}
\item {Proveniência:(Do gr. \textunderscore pimele\textunderscore , gordura)}
\end{itemize}
Inflammação do tecido adiposo.
\section{Pimelito}
\begin{itemize}
\item {Grp. gram.:m.}
\end{itemize}
\begin{itemize}
\item {Utilização:Miner.}
\end{itemize}
\begin{itemize}
\item {Proveniência:(Do gr. \textunderscore pimele\textunderscore )}
\end{itemize}
Minério oxydado de níquel.
\section{Pimelose}
\begin{itemize}
\item {Grp. gram.:f.}
\end{itemize}
\begin{itemize}
\item {Utilização:Med.}
\end{itemize}
\begin{itemize}
\item {Proveniência:(Do gr. \textunderscore pimele\textunderscore , gordura)}
\end{itemize}
O mesmo que \textunderscore obesidade\textunderscore .
\section{Pimenta}
\begin{itemize}
\item {Grp. gram.:f.}
\end{itemize}
Nome de várias plantas piperáceas, a que pertence o \textunderscore bétel\textunderscore , a \textunderscore pimenta-longa\textunderscore , (\textunderscore piper longum\textunderscore , Lin.) e a \textunderscore pimenta vulgar\textunderscore , (\textunderscore piper nigrum\textunderscore , Lin.) também conhecida por \textunderscore pimenta-negra\textunderscore , \textunderscore pimenta da Índia\textunderscore  ou \textunderscore pimenta-canarim\textunderscore .
É também o nome de algumas plantas solâneas.
(Cp. \textunderscore pimento\textunderscore )
\section{Pimenta-albarran}
\begin{itemize}
\item {Grp. gram.:f.}
\end{itemize}
Planta rutácea da Índia portuguesa, (\textunderscore zanthoxylum rhetsa\textunderscore , De-Cand.).
\section{Pimenta-da-costa}
\begin{itemize}
\item {Grp. gram.:f.}
\end{itemize}
\begin{itemize}
\item {Utilização:Bras}
\end{itemize}
Fruta africana, cujas sementes são empregadas como condimento e tem o ardor da pimenta.
\section{Pimenta-de-água}
\begin{itemize}
\item {Grp. gram.:f.}
\end{itemize}
\begin{itemize}
\item {Utilização:Bras}
\end{itemize}
O mesmo que \textunderscore erva-do-bicho\textunderscore .
\section{Pimental}
\begin{itemize}
\item {Grp. gram.:m.}
\end{itemize}
\begin{itemize}
\item {Proveniência:(De \textunderscore pimenta\textunderscore )}
\end{itemize}
Lugar, onde crescem pimenteiras.
\section{Pimentão}
\begin{itemize}
\item {Grp. gram.:m.}
\end{itemize}
Fruto vermelho e picante de uma planta solânea do Brasil.
Planta solânea, de que são variedades o \textunderscore pimentão doce\textunderscore  ou \textunderscore pimento\textunderscore , o \textunderscore pimentão cornicabra\textunderscore , o \textunderscore pimentão de cheiro\textunderscore , o \textunderscore pimentão maçan\textunderscore , o \textunderscore pimentão catalão\textunderscore , o \textunderscore pimentão de Caiena\textunderscore . Cf. P. Coutinho, \textunderscore Flora\textunderscore , 436 e 337.
\section{Pimenteira}
\begin{itemize}
\item {Grp. gram.:f.}
\end{itemize}
\begin{itemize}
\item {Proveniência:(De \textunderscore pimenta\textunderscore )}
\end{itemize}
Pimenta, árvore.
Pequeno vaso, em que se leva á mesa o fruto da pimenta, reduzido a pó, e que também se chama \textunderscore pimenteiro\textunderscore .
\section{Pimenteiras}
\begin{itemize}
\item {Grp. gram.:m. pl.}
\end{itemize}
\begin{itemize}
\item {Utilização:Bras}
\end{itemize}
Indígenas de Piauí.
O mesmo que \textunderscore pimenteiros\textunderscore .
\section{Pimenteiro}
\begin{itemize}
\item {Grp. gram.:m.}
\end{itemize}
Planta solânea, que dá o pimento.
\section{Pimenteiro}
\begin{itemize}
\item {Grp. gram.:m.}
\end{itemize}
\begin{itemize}
\item {Utilização:Des.}
\end{itemize}
\begin{itemize}
\item {Proveniência:(De \textunderscore pimenta\textunderscore )}
\end{itemize}
Vaso, em que se serve a pimenta, á mesa, e que também se chama \textunderscore pimenteira\textunderscore .
\section{Pimenteiros}
\begin{itemize}
\item {Grp. gram.:m. pl.}
\end{itemize}
Nome, que os Europeus deram a umas cabildas de Índios do Brasil, que habitavam nas margens da lagôa dos Pimenteiros.
\section{Pimentinha}
\begin{itemize}
\item {Grp. gram.:f.}
\end{itemize}
O mesmo que \textunderscore combarim\textunderscore .
\section{Pimento}
\begin{itemize}
\item {Grp. gram.:m.}
\end{itemize}
\begin{itemize}
\item {Proveniência:(Do lat. \textunderscore pigmentum\textunderscore )}
\end{itemize}
Planta solânea, (\textunderscore solanum pseudo capsicum\textunderscore ), também conhecida por \textunderscore pimentão doce\textunderscore .
Fruto desta planta, também chamado \textunderscore pimentão doce\textunderscore .
\section{Pimpalhão}
\begin{itemize}
\item {Grp. gram.:m.}
\end{itemize}
O mesmo que \textunderscore tentilhão\textunderscore , em Santo-Thyrso, Fafe, etc.
(Cp. \textunderscore pimpim\textunderscore )
\section{Pimpante}
\begin{itemize}
\item {Grp. gram.:m.  e  adj.}
\end{itemize}
\begin{itemize}
\item {Proveniência:(De \textunderscore pimpar\textunderscore )}
\end{itemize}
O mesmo que \textunderscore pimpão\textunderscore .
\section{Pimpão}
\begin{itemize}
\item {Grp. gram.:m.  e  adj.}
\end{itemize}
\begin{itemize}
\item {Grp. gram.:M.}
\end{itemize}
\begin{itemize}
\item {Utilização:Gír.}
\end{itemize}
\begin{itemize}
\item {Proveniência:(De \textunderscore pimpar\textunderscore )}
\end{itemize}
Valentão; jactancioso; janota.
Peixe, o mesmo que \textunderscore ruivaca\textunderscore .
Pimento.
\section{Pimpar}
\begin{itemize}
\item {Grp. gram.:v. i.}
\end{itemize}
Pompear; figurar; fazer ostentação.
Divertir-se.
(Talvez do lat. \textunderscore pompare\textunderscore )
\section{Pimpar}
\begin{itemize}
\item {Grp. gram.:v. t.}
\end{itemize}
\begin{itemize}
\item {Utilização:Prov.}
\end{itemize}
\begin{itemize}
\item {Utilização:trasm.}
\end{itemize}
\begin{itemize}
\item {Proveniência:(T. onom.?)}
\end{itemize}
Bater, sovar.
\section{Pimpim}
\begin{itemize}
\item {Grp. gram.:m.}
\end{itemize}
O mesmo que \textunderscore tentilhão\textunderscore , na Foz-do-Doiro, Esmoriz e Candal.
Peixe de Portugal.
\section{Pimpinela}
\begin{itemize}
\item {Grp. gram.:f.}
\end{itemize}
\begin{itemize}
\item {Proveniência:(Lat. \textunderscore pimpinelle\textunderscore )}
\end{itemize}
Erva rosácea, hortense e medicinal.
\section{Pimpinela-da-itália}
\begin{itemize}
\item {Grp. gram.:f.}
\end{itemize}
O mesmo que \textunderscore sanguisorba\textunderscore .
\section{Pimpinéleas}
\begin{itemize}
\item {Grp. gram.:f. pl.}
\end{itemize}
\begin{itemize}
\item {Proveniência:(De \textunderscore pimpinela\textunderscore )}
\end{itemize}
Tríbo de plantas umbellíferas, que comprehende as de frutos ovóides, alongados e ordinariamente estriados. Cf. Richard, \textunderscore Elem. de Bot.\textunderscore 
\section{Pimplar}
\begin{itemize}
\item {Grp. gram.:v. i.}
\end{itemize}
Florear com graça o pimpleu.
\section{Pimpleu}
\begin{itemize}
\item {Grp. gram.:m.}
\end{itemize}
\begin{itemize}
\item {Utilização:Des.}
\end{itemize}
Pequena garrocha enfeitada, no toireio.
\section{Pimpôl}
\begin{itemize}
\item {Grp. gram.:m.}
\end{itemize}
\begin{itemize}
\item {Proveniência:(Do conc. \textunderscore pimpala\textunderscore )}
\end{itemize}
Árvore indiana, (\textunderscore ficus indica\textunderscore , ou \textunderscore ficus religiosa\textunderscore ).
\section{Pimpolho}
\begin{itemize}
\item {fónica:pô}
\end{itemize}
\begin{itemize}
\item {Grp. gram.:m.}
\end{itemize}
\begin{itemize}
\item {Utilização:Fig.}
\end{itemize}
Rebento da videira.
Sarmento.
Rebento, vergôntea.
Rapazote.
(Cp. cast. \textunderscore pimpollo\textunderscore )
\section{Pimpolo}
\begin{itemize}
\item {fónica:pô}
\end{itemize}
\begin{itemize}
\item {Grp. gram.:m.}
\end{itemize}
O mesmo que \textunderscore pimpôl\textunderscore .
\section{Pimponamente}
\begin{itemize}
\item {Grp. gram.:adv.}
\end{itemize}
\begin{itemize}
\item {Proveniência:(De \textunderscore pimpão\textunderscore )}
\end{itemize}
Com modos de pimpão; arrogantemente:«\textunderscore o janota redarguiu pimponamente...\textunderscore »Camillo, \textunderscore Volcões\textunderscore , 36.
\section{Pimponar}
\begin{itemize}
\item {Grp. gram.:v. i.}
\end{itemize}
\begin{itemize}
\item {Grp. gram.:V. p.}
\end{itemize}
Mostrar-se pimpão; pimpar.
(A mesma sign.)
\section{Pimponear}
\begin{itemize}
\item {Grp. gram.:v. i.}
\end{itemize}
O mesmo que \textunderscore pimponar\textunderscore .
\section{Pimponete}
\begin{itemize}
\item {fónica:nê}
\end{itemize}
\begin{itemize}
\item {Grp. gram.:m.}
\end{itemize}
\begin{itemize}
\item {Utilização:Fam.}
\end{itemize}
\begin{itemize}
\item {Proveniência:(De \textunderscore pimpão\textunderscore )}
\end{itemize}
Janota ridículo; pintalegrete; petimetre; embófia.
\section{Pimponice}
\begin{itemize}
\item {Grp. gram.:f.}
\end{itemize}
Acto ou modos de pimpão:«\textunderscore entravam pelas feiras num arranque de rópia e pimponice.\textunderscore »Camillo, \textunderscore Brasileira\textunderscore , 92.
\section{Pimpulhão}
\begin{itemize}
\item {Grp. gram.:m.}
\end{itemize}
\begin{itemize}
\item {Utilização:Prov.}
\end{itemize}
\begin{itemize}
\item {Utilização:minh.}
\end{itemize}
O mesmo que \textunderscore tentilhão\textunderscore .
(Cp. \textunderscore pimpalhão\textunderscore )
\section{Pina}
\begin{itemize}
\item {Grp. gram.:f.}
\end{itemize}
\begin{itemize}
\item {Proveniência:(Lat. \textunderscore pina\textunderscore )}
\end{itemize}
Cada uma das peças, que formam a circunferência da roda de um vehículo.
\section{Pinaça}
\begin{itemize}
\item {Grp. gram.:f.}
\end{itemize}
\begin{itemize}
\item {Utilização:Des.}
\end{itemize}
Barco pequeno e estreito.
Corda, com que se levanta o cepo dos engenhos que se chamam macacos.
\section{Pinacóide}
\begin{itemize}
\item {Grp. gram.:adj.}
\end{itemize}
\begin{itemize}
\item {Utilização:Miner.}
\end{itemize}
\begin{itemize}
\item {Proveniência:(Do gr. \textunderscore pinax\textunderscore  + \textunderscore eidos\textunderscore )}
\end{itemize}
Diz-se da fórma, limitada por dois planos parallelos entre si e a dois eixos crystallográphicos.
\section{Pinacoteca}
\begin{itemize}
\item {Grp. gram.:f.}
\end{itemize}
\begin{itemize}
\item {Proveniência:(Do gr. \textunderscore pinakotheke\textunderscore )}
\end{itemize}
Designação antiquada de uma colecção de quadros; museu; hoje, designação especial da galeria do rei da Baviera.
\section{Pinacotheca}
\begin{itemize}
\item {Grp. gram.:f.}
\end{itemize}
\begin{itemize}
\item {Proveniência:(Do gr. \textunderscore pinakotheke\textunderscore )}
\end{itemize}
Designação antiquada de uma collecção de quadros; museu; hoje, designação especial da galeria do rei da Baviera.
\section{Pináculo}
\begin{itemize}
\item {Grp. gram.:m.}
\end{itemize}
\begin{itemize}
\item {Utilização:Fig.}
\end{itemize}
\begin{itemize}
\item {Proveniência:(Lat. \textunderscore pinnaculum\textunderscore , de \textunderscore pinna\textunderscore )}
\end{itemize}
O ponto mais elevado de um edifício, monte, etc.
O mais alto grau.
\section{Pinadela}
\begin{itemize}
\item {Grp. gram.:f.}
\end{itemize}
Acto de pinar.
\section{Pinador}
\begin{itemize}
\item {Grp. gram.:m.}
\end{itemize}
\begin{itemize}
\item {Proveniência:(De \textunderscore pino\textunderscore )}
\end{itemize}
Instrumento, espécie de cravador ou furador, com que o sapateiro faz os furos para cravar os pinos no calçado.
\section{Pinalho}
\begin{itemize}
\item {Grp. gram.:m.}
\end{itemize}
\begin{itemize}
\item {Utilização:Prov.}
\end{itemize}
\begin{itemize}
\item {Utilização:trasm.}
\end{itemize}
O mesmo que \textunderscore cabeçalha\textunderscore . (Colhido em Boticas)
\section{Pinante}
\begin{itemize}
\item {Grp. gram.:m.}
\end{itemize}
\begin{itemize}
\item {Proveniência:(De \textunderscore pinar\textunderscore ^1)}
\end{itemize}
Aquelle que pina muito.
Garanhão.
\section{Pinar}
\begin{itemize}
\item {Grp. gram.:v. i.}
\end{itemize}
\begin{itemize}
\item {Utilização:Chul.}
\end{itemize}
Têr cópula carnal.
\section{Pinar}
\begin{itemize}
\item {Grp. gram.:v. i.}
\end{itemize}
\begin{itemize}
\item {Utilização:T. de Turquel}
\end{itemize}
Estar visivelmente alegre; folgar.
\section{Pinárico}
\begin{itemize}
\item {Grp. gram.:adj.}
\end{itemize}
\begin{itemize}
\item {Utilização:Chím.}
\end{itemize}
\begin{itemize}
\item {Proveniência:(Do lat. \textunderscore pinus\textunderscore )}
\end{itemize}
Diz-se de um dos ácidos contidos na resina do pinheiro.
\section{Pinás}
\begin{itemize}
\item {Grp. gram.:m.}
\end{itemize}
\begin{itemize}
\item {Utilização:T. da Bairrada}
\end{itemize}
O mesmo que \textunderscore pinásio\textunderscore .
\section{Pinasco}
\begin{itemize}
\item {Grp. gram.:m.}
\end{itemize}
\begin{itemize}
\item {Utilização:Prov.}
\end{itemize}
\begin{itemize}
\item {Utilização:minh.}
\end{itemize}
O mesmo que \textunderscore pináculo\textunderscore .
\section{Pinásio}
\begin{itemize}
\item {Grp. gram.:m.}
\end{itemize}
\begin{itemize}
\item {Utilização:Prov.}
\end{itemize}
\begin{itemize}
\item {Utilização:beir.}
\end{itemize}
\begin{itemize}
\item {Utilização:Prov.}
\end{itemize}
\begin{itemize}
\item {Utilização:beir.}
\end{itemize}
\begin{itemize}
\item {Proveniência:(De \textunderscore pina\textunderscore ?)}
\end{itemize}
Cada uma das peças que, nas portas e janelas envidraçadas, separam e sustentam os vidros.
Cada uma das peças de cantaria, que ladeiam chaminés, em cozinhas.
Cada uma das tábuas verticaes que sustentam a tábua horizontal, em que se assentam os pés, no degrau de uma escada.
Tábua que, no fôrro dos tectos, fica em plano inferior a duas lateraes que lhe cobrem os lados.
\section{Pinaúma}
\begin{itemize}
\item {Grp. gram.:f.}
\end{itemize}
\begin{itemize}
\item {Utilização:Bras}
\end{itemize}
Designação vulgar dos echínides.
\section{Pinça}
\begin{itemize}
\item {Grp. gram.:f.}
\end{itemize}
Pequena tenaz.
Barra de ferro, em fórma de S, destinada a serviço de bomba, a bordo.
Parte infero-interior do casco do cavallo; parte da ferradura, que corresponde a essa parte do casco.
(Cast. \textunderscore pinza\textunderscore )
\section{Pinção}
\begin{itemize}
\item {Grp. gram.:m.}
\end{itemize}
O mesmo que \textunderscore pinçote\textunderscore .
\section{Pinção}
\begin{itemize}
\item {Grp. gram.:m.}
\end{itemize}
\begin{itemize}
\item {Utilização:Prov.}
\end{itemize}
\begin{itemize}
\item {Utilização:trasm.}
\end{itemize}
Pedúnculo dos frutos.
\section{Pinção}
\begin{itemize}
\item {Grp. gram.:m.}
\end{itemize}
\begin{itemize}
\item {Utilização:Prov.}
\end{itemize}
\begin{itemize}
\item {Utilização:trasm.}
\end{itemize}
O mesmo que \textunderscore canamão\textunderscore .
\section{Píncaro}
\begin{itemize}
\item {Grp. gram.:m.}
\end{itemize}
\begin{itemize}
\item {Utilização:Prov.}
\end{itemize}
\begin{itemize}
\item {Utilização:beir.}
\end{itemize}
O mesmo que \textunderscore pináculo\textunderscore .
Pedúnculo dos frutos, especialmente da cereja e da ginja.
\section{Pincel}
\begin{itemize}
\item {Grp. gram.:m.}
\end{itemize}
\begin{itemize}
\item {Utilização:Fig.}
\end{itemize}
\begin{itemize}
\item {Proveniência:(Lat. \textunderscore penicillus\textunderscore )}
\end{itemize}
Instrumento para tomar e estender tintas, colla, etc., sôbre uma superficie.
Brocha.
Gênero de algas corallinas.
Espécie de toupeira americana.
Maneira de pintar; o pintor.
\section{Pincelada}
\begin{itemize}
\item {Grp. gram.:f.}
\end{itemize}
Traço feito com o pincel, toque de pincel.
\section{Pincelagem}
\begin{itemize}
\item {Grp. gram.:f.}
\end{itemize}
O mesmo que \textunderscore pincelada\textunderscore .
\section{Pincelar}
\begin{itemize}
\item {Grp. gram.:v. t.}
\end{itemize}
Pintar com pincel.
\section{Pinceleiro}
\begin{itemize}
\item {Grp. gram.:m.}
\end{itemize}
Fabricante ou vendedor de pincéis.
\section{Pinceta}
\begin{itemize}
\item {fónica:cê}
\end{itemize}
\begin{itemize}
\item {Grp. gram.:f.}
\end{itemize}
\begin{itemize}
\item {Proveniência:(De \textunderscore pinça\textunderscore )}
\end{itemize}
Pinça com que se formam as asas nos vasos de vidro.
\section{Pincéus}
\begin{itemize}
\item {Grp. gram.:m. pl.}
\end{itemize}
\begin{itemize}
\item {Utilização:Ant.}
\end{itemize}
Usou-se na loc. \textunderscore falar por pincéus\textunderscore , falar por figuras ou com subtileza. Cf. G. Vicente, \textunderscore Auto da Feira\textunderscore .
(Talvez corr. de \textunderscore pincéis\textunderscore , por allusão ás figuras do estylo)
\section{Pincha}
\begin{itemize}
\item {Grp. gram.:f.}
\end{itemize}
(V. \textunderscore picha\textunderscore ^1)
\section{Pincha}
\begin{itemize}
\item {Grp. gram.:f.}
\end{itemize}
\begin{itemize}
\item {Utilização:T. de Gaia}
\end{itemize}
O jôgo do botão.
\section{Pinchar}
\begin{itemize}
\item {Grp. gram.:v. t.}
\end{itemize}
\begin{itemize}
\item {Grp. gram.:V. i.}
\end{itemize}
\begin{itemize}
\item {Proveniência:(Do lat. \textunderscore pinsare\textunderscore ?)}
\end{itemize}
Fazer saltar; empurrar.
Dar pinchos; trepar.
\section{Pinchar}
\begin{itemize}
\item {Grp. gram.:v. t.}
\end{itemize}
\begin{itemize}
\item {Utilização:Prov.}
\end{itemize}
\begin{itemize}
\item {Utilização:beir.}
\end{itemize}
\begin{itemize}
\item {Proveniência:(De \textunderscore pincho\textunderscore ^1)}
\end{itemize}
Fechar com o pincho (uma porta).
\section{Pinche}
\begin{itemize}
\item {Grp. gram.:m.}
\end{itemize}
\begin{itemize}
\item {Utilização:Prov.}
\end{itemize}
\begin{itemize}
\item {Utilização:alent.}
\end{itemize}
Jôgo de rapazes.
\section{Pincho}
\begin{itemize}
\item {Grp. gram.:m.}
\end{itemize}
\begin{itemize}
\item {Utilização:Prov.}
\end{itemize}
\begin{itemize}
\item {Utilização:beir.}
\end{itemize}
Lingueta de ferro, que levanta a tranqueta da aldrava.
\section{Pincho}
\begin{itemize}
\item {Grp. gram.:m.}
\end{itemize}
\begin{itemize}
\item {Proveniência:(De \textunderscore pinchar\textunderscore ^1)}
\end{itemize}
Pulo, cabriola.
\section{Pincho}
\begin{itemize}
\item {Grp. gram.:m.}
\end{itemize}
O mesmo que \textunderscore palangre\textunderscore .
\section{Pinçote}
\begin{itemize}
\item {Grp. gram.:m.}
\end{itemize}
\begin{itemize}
\item {Utilização:Náut.}
\end{itemize}
Pau, na extremidade da cana do leme.
(Cast. \textunderscore pinzote\textunderscore )
\section{Pincre}
\begin{itemize}
\item {Grp. gram.:adj.}
\end{itemize}
\begin{itemize}
\item {Utilização:Prov.}
\end{itemize}
\begin{itemize}
\item {Utilização:alg.}
\end{itemize}
Diz-se do figo meio passado ou quási sêco.
\section{Pínculas}
\begin{itemize}
\item {Grp. gram.:f. pl. Loc. adv.}
\end{itemize}
\begin{itemize}
\item {Utilização:Prov.}
\end{itemize}
\begin{itemize}
\item {Utilização:beir.}
\end{itemize}
\textunderscore Por pínculas\textunderscore , por um triz; por um quási nada.
\section{Pindá}
\begin{itemize}
\item {Grp. gram.:m.}
\end{itemize}
O mesmo que \textunderscore pindoba\textunderscore .
\section{Pindá}
\begin{itemize}
\item {Grp. gram.:m.}
\end{itemize}
\begin{itemize}
\item {Utilização:Bras}
\end{itemize}
Nome, que dão ao anzol os indígenas do norte do Brasil.
\section{Pindabuna}
\begin{itemize}
\item {Grp. gram.:f.}
\end{itemize}
\begin{itemize}
\item {Utilização:Bras}
\end{itemize}
Árvore silvestre, de cerne preto, e bôa para carpintaria.
\section{Pindaíba}
\begin{itemize}
\item {Grp. gram.:f.}
\end{itemize}
\begin{itemize}
\item {Utilização:Bras}
\end{itemize}
\begin{itemize}
\item {Utilização:gír. de estudantes.}
\end{itemize}
Corda, feita de palha de coqueiro; ibira.
Falta de dinheiro.
\section{Pindaíva}
\begin{itemize}
\item {Grp. gram.:f.}
\end{itemize}
O mesmo que \textunderscore pindaíba\textunderscore .
\section{Pindaricamente}
\begin{itemize}
\item {Grp. gram.:adv.}
\end{itemize}
\begin{itemize}
\item {Utilização:Fam.}
\end{itemize}
De modo pindárico.
Á maneira de Píndaro.
Excellentemente.
\section{Pindárico}
\begin{itemize}
\item {Grp. gram.:adj.}
\end{itemize}
\begin{itemize}
\item {Utilização:Fam.}
\end{itemize}
\begin{itemize}
\item {Proveniência:(Lat. \textunderscore pindaricus\textunderscore )}
\end{itemize}
Relativo a Píndaro; semelhante á maneira poética de Píndaro.
Óptimo: \textunderscore fez um discurso pindárico\textunderscore .
\section{Pindarismo}
\begin{itemize}
\item {Grp. gram.:m.}
\end{itemize}
Imitação do gênero poético de Píndaro.
\section{Pindarista}
\begin{itemize}
\item {Grp. gram.:adj.}
\end{itemize}
\begin{itemize}
\item {Proveniência:(De \textunderscore Píndaro\textunderscore , n. p.)}
\end{itemize}
Relativo ao pindarismo.
Que tem qualquer semelhança ou relação com a índole das poesias pindáricas. Cf. Camillo, \textunderscore Narcót.\textunderscore , II, 201.
Aquelle que fala em estilo bombástico. Cf. Castilho, \textunderscore Montalverne\textunderscore .
\section{Pindarizar}
\begin{itemize}
\item {Grp. gram.:v. t.}
\end{itemize}
\begin{itemize}
\item {Grp. gram.:V. i.}
\end{itemize}
\begin{itemize}
\item {Proveniência:(De \textunderscore Píndaro\textunderscore , n. p.)}
\end{itemize}
Louvar exaggeradamente. Cf. Camillo, \textunderscore M. de Pombal\textunderscore , 95.
Poetar como Píndaro. Cf. Filinto, VIII, 45.
\section{Pindaúva}
\begin{itemize}
\item {Grp. gram.:f.}
\end{itemize}
O mesmo que \textunderscore pindaíba\textunderscore .
\section{Pindérico}
\begin{itemize}
\item {Grp. gram.:adj.}
\end{itemize}
\begin{itemize}
\item {Utilização:Chul.}
\end{itemize}
Magnífico; excelente: \textunderscore aquilo foi um pagode pindérico\textunderscore .
(Cp. \textunderscore pindárico\textunderscore )
\section{Pindi}
\begin{itemize}
\item {Grp. gram.:m.}
\end{itemize}
\begin{itemize}
\item {Utilização:T. da Afr. Or. Port}
\end{itemize}
Espécie de esteira.
\section{Pindoba}
\begin{itemize}
\item {Grp. gram.:f.}
\end{itemize}
\begin{itemize}
\item {Utilização:Bras}
\end{itemize}
\begin{itemize}
\item {Proveniência:(Do guar. \textunderscore pindo\textunderscore  + \textunderscore ob\textunderscore )}
\end{itemize}
Nome de algumas espécies de palmeiras silvestres.
\section{Pindova}
\begin{itemize}
\item {Grp. gram.:f.}
\end{itemize}
O mesmo que \textunderscore pindoba\textunderscore .
\section{Pindra}
\begin{itemize}
\item {Grp. gram.:f.}
\end{itemize}
\begin{itemize}
\item {Utilização:Ant.}
\end{itemize}
Penhor.
\section{Pindrar}
\begin{itemize}
\item {Grp. gram.:v. t.}
\end{itemize}
\begin{itemize}
\item {Utilização:Ant.}
\end{itemize}
Penhorar.
\section{Píneo}
\begin{itemize}
\item {Grp. gram.:adj.}
\end{itemize}
\begin{itemize}
\item {Utilização:Poét.}
\end{itemize}
\begin{itemize}
\item {Proveniência:(Lat. \textunderscore pineus\textunderscore )}
\end{itemize}
Relativo a pinheiro; feito de pinho.
\section{Pinéu}
\begin{itemize}
\item {Grp. gram.:m.}
\end{itemize}
Antigo jôgo de meninos, os quaes atiravam uma pedra ao ar, dizendo:«\textunderscore pinéu, pinéu, que vais para o céu, torna a caír, e guarda a cabeça de quem ella ferir\textunderscore ». Cf. \textunderscore Diccion. de Nomes, Vozes e Coisas\textunderscore , (ms.).
\section{Pinéu}
\begin{itemize}
\item {Grp. gram.:m.}
\end{itemize}
\begin{itemize}
\item {Utilização:Bras}
\end{itemize}
Nome de um passarinho.
\section{Pinga}
\begin{itemize}
\item {Grp. gram.:f.}
\end{itemize}
\begin{itemize}
\item {Utilização:Pop.}
\end{itemize}
\begin{itemize}
\item {Grp. gram.:M.}
\end{itemize}
\begin{itemize}
\item {Utilização:Pop.}
\end{itemize}
\begin{itemize}
\item {Proveniência:(De \textunderscore pingar\textunderscore )}
\end{itemize}
Pequeníssima quantidade de líquido; gota.
Vinho; um copo de qualquer bebida: \textunderscore foram beber uma pinga\textunderscore .
Homem sem dinheiro, pelintra.
\section{Pinga}
\begin{itemize}
\item {Grp. gram.:f.}
\end{itemize}
\begin{itemize}
\item {Utilização:T. de Macau}
\end{itemize}
Vara de cana da Índia, que se traz ao ombro, e em cujas extremidades se penduram cabazes ou outros objectos.
\section{Pingaço}
\begin{itemize}
\item {Grp. gram.:m.}
\end{itemize}
\begin{itemize}
\item {Utilização:Bras. do S}
\end{itemize}
\begin{itemize}
\item {Proveniência:(De \textunderscore pingo\textunderscore )}
\end{itemize}
Cavallo muito bom e bonito.
\section{Pingada}
\begin{itemize}
\item {Grp. gram.:f.}
\end{itemize}
\begin{itemize}
\item {Utilização:Bras. do S}
\end{itemize}
Reunião de cavallos pingos, de bons cavallos.
\section{Pingadeira}
\begin{itemize}
\item {Grp. gram.:f.}
\end{itemize}
\begin{itemize}
\item {Utilização:Pop.}
\end{itemize}
\begin{itemize}
\item {Utilização:Chul.}
\end{itemize}
Vaso, em que se recolhem os pingos da carne que se assa.
Coisa que pinga.
Acto de pingar.
Série de pingos.
Negócio, que vai rendendo sempre.
Despesa contínua.
O mesmo que \textunderscore blennorrheia\textunderscore .
\section{Pingado}
\begin{itemize}
\item {Grp. gram.:adj.}
\end{itemize}
\begin{itemize}
\item {Utilização:Ant.}
\end{itemize}
\begin{itemize}
\item {Proveniência:(De \textunderscore pingar\textunderscore )}
\end{itemize}
Castigado com pingos de gordura fervente ou azeite de candeia acesa, como se fazia a escravos negros e moiros.
\textunderscore Gato-pingado\textunderscore , aquelle que leva tocha nos enterros, ao lado dos carros funerários.
\section{Pingadoiro}
\begin{itemize}
\item {Grp. gram.:m.}
\end{itemize}
(V.pingadeira)
\section{Pingalete}
\begin{itemize}
\item {fónica:lê}
\end{itemize}
\begin{itemize}
\item {Grp. gram.:m.}
\end{itemize}
\begin{itemize}
\item {Utilização:Prov.}
\end{itemize}
\begin{itemize}
\item {Utilização:beir.}
\end{itemize}
O mesmo que \textunderscore pinguelete\textunderscore .
Espécie de prego, usado a bordo.
\section{Pingalhado}
\begin{itemize}
\item {Grp. gram.:adj.}
\end{itemize}
\begin{itemize}
\item {Utilização:T. de Turquel}
\end{itemize}
\begin{itemize}
\item {Proveniência:(De \textunderscore pingalho\textunderscore )}
\end{itemize}
Um pouco perturbado por bebida alcoólica; pechingado.
\section{Pingalhareta}
\begin{itemize}
\item {fónica:lharê}
\end{itemize}
\begin{itemize}
\item {Grp. gram.:f.}
\end{itemize}
\begin{itemize}
\item {Utilização:Pop.}
\end{itemize}
\begin{itemize}
\item {Proveniência:(De \textunderscore pinga\textunderscore )}
\end{itemize}
Mulhér ordinária e esfarrapada.
Mulhér, que anda bebericando pelas tascas.
\section{Pingalhete}
\begin{itemize}
\item {fónica:lhê}
\end{itemize}
\begin{itemize}
\item {Grp. gram.:m.}
\end{itemize}
O mesmo que \textunderscore pinguelete\textunderscore .
\section{Pingalho}
\begin{itemize}
\item {Grp. gram.:m.}
\end{itemize}
\begin{itemize}
\item {Utilização:Fam.}
\end{itemize}
\begin{itemize}
\item {Utilização:Fam.}
\end{itemize}
\begin{itemize}
\item {Proveniência:(De \textunderscore pinga\textunderscore )}
\end{itemize}
Pinga, porção de bebida.
Pessôa desmazelada no vestir; farroupilha.
\section{Pingalim}
\begin{itemize}
\item {Grp. gram.:m.}
\end{itemize}
Chicote delgado e comprido, usado pelos cocheiros.
(Talvez por \textunderscore bengalim\textunderscore , dem. de \textunderscore bengala\textunderscore )
\section{Pinganéis}
\begin{itemize}
\item {Grp. gram.:m. pl.}
\end{itemize}
\begin{itemize}
\item {Utilização:Prov.}
\end{itemize}
\begin{itemize}
\item {Utilização:trasm.}
\end{itemize}
\begin{itemize}
\item {Proveniência:(De \textunderscore pingar\textunderscore )}
\end{itemize}
Pingentes de gêlo nos beiraes dos telhados.
\section{Pinganelos}
\begin{itemize}
\item {Grp. gram.:m. pl.}
\end{itemize}
\begin{itemize}
\item {Utilização:Prov.}
\end{itemize}
\begin{itemize}
\item {Utilização:trasm.}
\end{itemize}
O mesmo que \textunderscore pinganéis\textunderscore .
\section{Pingante}
\begin{itemize}
\item {Grp. gram.:adj.}
\end{itemize}
\begin{itemize}
\item {Grp. gram.:M.}
\end{itemize}
\begin{itemize}
\item {Utilização:Chul.}
\end{itemize}
\begin{itemize}
\item {Proveniência:(De \textunderscore pingar\textunderscore )}
\end{itemize}
Que pinga.
Indivíduo muito pobre, miserável.
\section{Pingão}
\begin{itemize}
\item {Grp. gram.:m.}
\end{itemize}
\begin{itemize}
\item {Utilização:Prov.}
\end{itemize}
\begin{itemize}
\item {Utilização:beir.}
\end{itemize}
\begin{itemize}
\item {Grp. gram.:M.  e  adj.}
\end{itemize}
\begin{itemize}
\item {Utilização:Prov.}
\end{itemize}
\begin{itemize}
\item {Utilização:minh.}
\end{itemize}
Pessôa de grande estatura. (Colhido no Fundão)
Palerma, pacóvio.
\section{Pingar}
\begin{itemize}
\item {Grp. gram.:v. t.}
\end{itemize}
\begin{itemize}
\item {Grp. gram.:V. i.}
\end{itemize}
\begin{itemize}
\item {Proveniência:(De \textunderscore pingo\textunderscore )}
\end{itemize}
Deitar pingos em; verter aos pingos: \textunderscore a candeia pinga azeite\textunderscore .
Deixar caír sôbre (o corpo do padecente) pingos de resina ou azeite a ferver. Cf. \textunderscore Peregrinação\textunderscore , CCV.
Caír em pingos ou ás gotas.
Deitar água, aos pingos: \textunderscore trazer o capote a pingar\textunderscore .
Chover pouco: \textunderscore começou agora a pingar\textunderscore .
Principiar a chover: \textunderscore já pinga\textunderscore .
Render ou dar proveito successivamente, aos poucos: \textunderscore a mendicidade é offício que vai pingando\textunderscore .
\section{Pingar}
\begin{itemize}
\item {Grp. gram.:v. i.}
\end{itemize}
\begin{itemize}
\item {Utilização:Prov.}
\end{itemize}
\begin{itemize}
\item {Utilização:trasm.}
\end{itemize}
\begin{itemize}
\item {Proveniência:(Do lat. hyp. \textunderscore pendicare\textunderscore )}
\end{itemize}
Cabecear com somno.
\section{Pinatífido}
\begin{itemize}
\item {Grp. gram.:adj.}
\end{itemize}
\begin{itemize}
\item {Utilização:Bot.}
\end{itemize}
\begin{itemize}
\item {Proveniência:(Do lat. \textunderscore pinna\textunderscore  + \textunderscore findere\textunderscore )}
\end{itemize}
Diz-se das fôlhas, que são fendidas como as pennas.
\section{Pingarelho}
\begin{itemize}
\item {fónica:garê}
\end{itemize}
\begin{itemize}
\item {Grp. gram.:m.}
\end{itemize}
\begin{itemize}
\item {Utilização:Prov.}
\end{itemize}
\begin{itemize}
\item {Utilização:minh.}
\end{itemize}
Pelintra; homem esfarrapado. Cf. Camillo, \textunderscore Corja\textunderscore , 158.
Qualquer coisa mal segura, prestes a caír.
(Cp. \textunderscore pingante\textunderscore  e \textunderscore pingar\textunderscore ^2)
\section{Pingemoiro}
\begin{itemize}
\item {Grp. gram.:m.}
\end{itemize}
\begin{itemize}
\item {Utilização:Prov.}
\end{itemize}
\begin{itemize}
\item {Utilização:trasm.}
\end{itemize}
Planta, o mesmo que \textunderscore cangemoiro\textunderscore .
\section{Pingente}
\begin{itemize}
\item {Grp. gram.:m.}
\end{itemize}
Pequeno objecto pendente; brinco de orelha; berloque.
(Cp. cast. \textunderscore pinjante\textunderscore )
\section{Pingo}
\begin{itemize}
\item {Grp. gram.:m.}
\end{itemize}
\begin{itemize}
\item {Utilização:Bras}
\end{itemize}
\begin{itemize}
\item {Proveniência:(Lat. \textunderscore pinguis\textunderscore )}
\end{itemize}
Banha de porco derretida.
Gota de gordura.
Gordura.
Gota.
Mucosidade nasal.
Pequena porção de solda, com que os latoeiros tapam orifícios em vasilhas de cozinha, usadas.
Cavallo de sella; bom cavallo.
\section{Pingoé}
\begin{itemize}
\item {Grp. gram.:m.}
\end{itemize}
\begin{itemize}
\item {Utilização:T. da Áfr. Or. Port}
\end{itemize}
Mólho de lenha.
\section{Pingola}
\begin{itemize}
\item {Grp. gram.:f.}
\end{itemize}
\begin{itemize}
\item {Utilização:Pop.}
\end{itemize}
\begin{itemize}
\item {Proveniência:(De \textunderscore pinga\textunderscore )}
\end{itemize}
O mesmo que \textunderscore pingoleta\textunderscore .
\section{Pingoleta}
\begin{itemize}
\item {fónica:lê}
\end{itemize}
\begin{itemize}
\item {Grp. gram.:f.}
\end{itemize}
\begin{itemize}
\item {Utilização:Pop.}
\end{itemize}
\begin{itemize}
\item {Proveniência:(De \textunderscore pingola\textunderscore )}
\end{itemize}
Pequena porção de vinho para beber.
Copo de qualquer bebida.
Pinga. Cf. Camillo, \textunderscore Am. de Perdição\textunderscore , 67, (ed. monum.).
\section{Pingolindo}
\begin{itemize}
\item {Grp. gram.:m.}
\end{itemize}
\begin{itemize}
\item {Utilização:Bras}
\end{itemize}
\begin{itemize}
\item {Proveniência:(De \textunderscore pingo\textunderscore  + \textunderscore lindo\textunderscore )}
\end{itemize}
Cavallo bonito.
\section{Pingorça}
\begin{itemize}
\item {Grp. gram.:f.}
\end{itemize}
\begin{itemize}
\item {Utilização:Prov.}
\end{itemize}
\begin{itemize}
\item {Utilização:beir.}
\end{itemize}
Mulhér alta e deselegante.
(Cp. \textunderscore pingão\textunderscore )
\section{Pingos}
\begin{itemize}
\item {Grp. gram.:m. pl.}
\end{itemize}
\begin{itemize}
\item {Utilização:Ant.}
\end{itemize}
\begin{itemize}
\item {Proveniência:(De \textunderscore pingo\textunderscore )}
\end{itemize}
Paredes de pedra miúda.
\section{Pingoso}
\begin{itemize}
\item {Grp. gram.:adj.}
\end{itemize}
Que pinga, que deita pingos. Cf. Filinto, I, 142.
\section{Pinguaciba}
\begin{itemize}
\item {Grp. gram.:f.}
\end{itemize}
O mesmo que \textunderscore pau-pereira\textunderscore .
\section{Pingue}
\begin{itemize}
\item {Grp. gram.:adj.}
\end{itemize}
\begin{itemize}
\item {Grp. gram.:M.}
\end{itemize}
\begin{itemize}
\item {Proveniência:(Lat. \textunderscore pinguis\textunderscore )}
\end{itemize}
Gordo.
Productivo, fértil; abundante.
Que dá muito lucro; rendoso.
O mesmo que \textunderscore pingo\textunderscore :«\textunderscore ...e em grosso pingue de toicinho gordo...\textunderscore »Garrett, \textunderscore D. Branca\textunderscore , X, II.
\section{Pinguécula}
\begin{itemize}
\item {Grp. gram.:f.}
\end{itemize}
\begin{itemize}
\item {Proveniência:(Do lat. \textunderscore pinguis\textunderscore )}
\end{itemize}
Pequena saliência pathológica no branco do ôlho, do lado interno da córnea.
\section{Pingueiro}
\begin{itemize}
\item {Grp. gram.:adj.}
\end{itemize}
\begin{itemize}
\item {Utilização:Burl.}
\end{itemize}
\begin{itemize}
\item {Proveniência:(De \textunderscore pinga\textunderscore )}
\end{itemize}
Bêbedo, embriagado.
\section{Pingueiro}
\begin{itemize}
\item {Grp. gram.:m.}
\end{itemize}
\begin{itemize}
\item {Utilização:Prov.}
\end{itemize}
\begin{itemize}
\item {Utilização:minh.}
\end{itemize}
Tacho para o pingo.
\section{Pinguela}
\begin{itemize}
\item {Grp. gram.:f.}
\end{itemize}
\begin{itemize}
\item {Utilização:Prov.}
\end{itemize}
\begin{itemize}
\item {Utilização:minh.}
\end{itemize}
Pauzinho, com que se arma o laço para apanhar aves.
Gancho, com que se armam ratoeiras.
Viga ou prancha, atravessada sôbre um rio, servindo de ponte.
Pedra única que, em meio de um regato estreito, facilita a passagem.
\section{Pinguelete}
\begin{itemize}
\item {fónica:lê}
\end{itemize}
\begin{itemize}
\item {Grp. gram.:m.}
\end{itemize}
\begin{itemize}
\item {Utilização:Prov.}
\end{itemize}
\begin{itemize}
\item {Utilização:beir.}
\end{itemize}
\begin{itemize}
\item {Proveniência:(De \textunderscore pinguela\textunderscore )}
\end{itemize}
Prego ou pauzinho, com que se ampara a grileira, nas pescócias.
\section{Pinguelo}
\begin{itemize}
\item {fónica:guê}
\end{itemize}
\begin{itemize}
\item {Grp. gram.:m.}
\end{itemize}
\begin{itemize}
\item {Utilização:Prov.}
\end{itemize}
\begin{itemize}
\item {Utilização:trasm.}
\end{itemize}
O mesmo que \textunderscore pinguela\textunderscore .
Rapaz ou rapariga palerma. (Colhido em Alijó)
\section{Pinguiça}
\begin{itemize}
\item {fónica:gu-i}
\end{itemize}
\begin{itemize}
\item {Grp. gram.:f.}
\end{itemize}
Espécie de uva mexicana.
\section{Pinguicho}
\begin{itemize}
\item {Grp. gram.:m.}
\end{itemize}
\begin{itemize}
\item {Utilização:Fam.}
\end{itemize}
\begin{itemize}
\item {Proveniência:(De \textunderscore pingo\textunderscore )}
\end{itemize}
Pequeníssima porção de qualquer líquido.
\section{Pinguim}
\begin{itemize}
\item {Grp. gram.:m.}
\end{itemize}
Planta bromeliácea do Brasil.
\section{Pingúrria}
\begin{itemize}
\item {Grp. gram.:f.}
\end{itemize}
\begin{itemize}
\item {Utilização:Prov.}
\end{itemize}
\begin{itemize}
\item {Utilização:trasm.}
\end{itemize}
Mulhér desairosa ou desmazelada.
(Cp. \textunderscore pingorça\textunderscore )
\section{Pingúrrio}
\begin{itemize}
\item {Grp. gram.:m.}
\end{itemize}
\begin{itemize}
\item {Utilização:Fam.}
\end{itemize}
Que é pobretão; pelintra.
\section{Pinha}
\begin{itemize}
\item {Grp. gram.:f.}
\end{itemize}
\begin{itemize}
\item {Utilização:Gír.}
\end{itemize}
\begin{itemize}
\item {Utilização:Bras}
\end{itemize}
\begin{itemize}
\item {Utilização:Prov.}
\end{itemize}
\begin{itemize}
\item {Utilização:trasm.}
\end{itemize}
\begin{itemize}
\item {Utilização:Bras}
\end{itemize}
\begin{itemize}
\item {Proveniência:(Do lat. \textunderscore pinea\textunderscore )}
\end{itemize}
Fruto do pinheiro.
Fruto ou objecto, semelhante na fórma ovóide á pinha.
Ajuntamento de pessôas ou coisas.
Planta do Brasil, o mesmo que \textunderscore queimadeira\textunderscore .
A cabeça.
O mesmo que \textunderscore ata\textunderscore .
Presente de núpcias, bedalha.
O mesmo que \textunderscore fruta-de-conde\textunderscore .
\section{Pinha-alta}
\begin{itemize}
\item {Grp. gram.:f.}
\end{itemize}
Fruto da pinheira.
\section{Pinhal}
\begin{itemize}
\item {Grp. gram.:m.}
\end{itemize}
\begin{itemize}
\item {Grp. gram.:Loc.}
\end{itemize}
\begin{itemize}
\item {Utilização:fig.}
\end{itemize}
\begin{itemize}
\item {Proveniência:(De \textunderscore pinho\textunderscore )}
\end{itemize}
Mata de pinheiros.
Casta de uva.
\textunderscore Pinhal da Azambuja\textunderscore , sítio, casa ou Repartição pública, onde se cometem muitos roubos.
\section{Pinhão}
\begin{itemize}
\item {Grp. gram.:m.}
\end{itemize}
\begin{itemize}
\item {Proveniência:(De \textunderscore pinha\textunderscore )}
\end{itemize}
Semente do pinheiro.
Planta anonácea de Cabo-Verde.
\section{Pinhão-bravo}
\begin{itemize}
\item {Grp. gram.:m.}
\end{itemize}
Arbusto euphorbiáceo, (\textunderscore jatropa herbacea\textunderscore ).
\section{Pinhão-de-purga}
\begin{itemize}
\item {Grp. gram.:m.}
\end{itemize}
O mesmo que \textunderscore purgueira\textunderscore .
\section{Pinheira}
\begin{itemize}
\item {Grp. gram.:f.}
\end{itemize}
\begin{itemize}
\item {Utilização:Prov.}
\end{itemize}
\begin{itemize}
\item {Utilização:alent.}
\end{itemize}
\begin{itemize}
\item {Proveniência:(De \textunderscore pinha\textunderscore )}
\end{itemize}
Árvore anonácea do Brasil (\textunderscore anona squamosa\textunderscore ).
Casta de uva, na região do Doiro.
Pinheiro manso.
\section{Pinheiral}
\begin{itemize}
\item {Grp. gram.:m.}
\end{itemize}
\begin{itemize}
\item {Proveniência:(De \textunderscore pinheiro\textunderscore )}
\end{itemize}
O mesmo que \textunderscore pinhal\textunderscore .
\section{Pinheirame}
\begin{itemize}
\item {Grp. gram.:m.}
\end{itemize}
\begin{itemize}
\item {Utilização:Fam.}
\end{itemize}
Conjunto de muitos pinheiros. C. Th. Ribeiro, \textunderscore Jornadas\textunderscore , I, 144.
\section{Pinheirinho}
\begin{itemize}
\item {Grp. gram.:m.  e  adj.}
\end{itemize}
\begin{itemize}
\item {Proveniência:(De \textunderscore pinheiro\textunderscore )}
\end{itemize}
Diz-se de uma variedade de feijão, também conhecida por \textunderscore laranjeiro\textunderscore .
\section{Pinheiro}
\begin{itemize}
\item {Grp. gram.:m.}
\end{itemize}
\begin{itemize}
\item {Grp. gram.:Adj.}
\end{itemize}
\begin{itemize}
\item {Utilização:Bras. do N}
\end{itemize}
\begin{itemize}
\item {Proveniência:(Do b. lat. \textunderscore piniarius\textunderscore )}
\end{itemize}
Gênero de árvores coníferas, (\textunderscore pinus\textunderscore ), de que há várias espécies, como: o pinheiro manso, (\textunderscore pinus pinea\textunderscore ); o pinheiro bravo, (\textunderscore pinus pinaster\textunderscore ); o pinheiro de Riga, (\textunderscore pinus sylvestris\textunderscore ); o pinheiro chorão, ou pinheiro do México; o pinheiro insigne; o pinheiro silvestre, (\textunderscore pinus sylvestris\textunderscore ); o pinheiro francês ou pinheiro de Alepo, (\textunderscore pinus alepensis\textunderscore ); o pinheiro mollar, etc.
Diz-se da rês que tem os chifres direitos.
\section{Pinheiro-baboso}
\begin{itemize}
\item {Grp. gram.:m.}
\end{itemize}
O mesmo que \textunderscore orvalho-do-sol\textunderscore .
\section{Pinheiro-branco}
\begin{itemize}
\item {Grp. gram.:m.}
\end{itemize}
O mesmo que \textunderscore pinheiro-de-alepo\textunderscore .
\section{Pinheiro-casquinha}
\begin{itemize}
\item {Grp. gram.:f.}
\end{itemize}
\begin{itemize}
\item {Utilização:T. de Alcanena}
\end{itemize}
O mesmo que \textunderscore pinheiro-de-alepo\textunderscore .
\section{Pinheiro-de-alepo}
\begin{itemize}
\item {Grp. gram.:m.}
\end{itemize}
Variedade de pinheiro, (\textunderscore pinus alepensis\textunderscore , Aiton).
\section{Pinheiro-de-jerusalém}
\begin{itemize}
\item {Grp. gram.:m.}
\end{itemize}
O mesmo que \textunderscore pinheiro-de-alepo\textunderscore .
\section{Pinheiro-de-purga}
\begin{itemize}
\item {Grp. gram.:m.}
\end{itemize}
O mesmo que \textunderscore purgueira\textunderscore .
\section{Pinheiro-larício}
\begin{itemize}
\item {Grp. gram.:m.}
\end{itemize}
O mesmo que \textunderscore lárice\textunderscore , (\textunderscore pinus larix\textunderscore , Lin.).
\section{Pinhel}
\begin{itemize}
\item {Grp. gram.:m.}
\end{itemize}
\begin{itemize}
\item {Utilização:Prov.}
\end{itemize}
\begin{itemize}
\item {Utilização:dur.}
\end{itemize}
\begin{itemize}
\item {Proveniência:(De \textunderscore pinho\textunderscore )}
\end{itemize}
Caruma sêca.
\section{Pinhiforme}
\begin{itemize}
\item {Grp. gram.:adj.}
\end{itemize}
\begin{itemize}
\item {Utilização:Neol.}
\end{itemize}
\begin{itemize}
\item {Proveniência:(De \textunderscore pinha\textunderscore  + \textunderscore fórma\textunderscore )}
\end{itemize}
Que tem fórma de pinha. Cf. Júl. Ribeiro, \textunderscore Carne\textunderscore .
\section{Pinho}
\begin{itemize}
\item {Grp. gram.:m.}
\end{itemize}
\begin{itemize}
\item {Utilização:Prov.}
\end{itemize}
\begin{itemize}
\item {Utilização:dur.}
\end{itemize}
\begin{itemize}
\item {Utilização:Bras. de Minas}
\end{itemize}
Madeira de pinheiro.
Caruma sêca.
O mesmo que \textunderscore viola\textunderscore ^1.
\textunderscore Chorar o pinho\textunderscore , tocar viola.
\textunderscore Pinho de Flandres\textunderscore , madeira de uma espécie de pinheiro bravo, (\textunderscore pinus sylvestris\textunderscore ).
\section{Pinhoada}
\begin{itemize}
\item {Grp. gram.:f.}
\end{itemize}
\begin{itemize}
\item {Utilização:T. da Bairrada}
\end{itemize}
Confeito de pinhões e mel.
Pasta comestível, feita de mel e pinhões.
Os dentes; a dentadura: \textunderscore e pôs-se a arreganhar a pinhoada\textunderscore .
\section{Pinhoan}
\begin{itemize}
\item {Grp. gram.:m.}
\end{itemize}
\begin{itemize}
\item {Utilização:Bras}
\end{itemize}
Árvore silvestre, de bôa madeira, para construcções navaes.
O mesmo que \textunderscore tapinhoan\textunderscore ?
\section{Pinhoca}
\begin{itemize}
\item {Grp. gram.:f.}
\end{itemize}
\begin{itemize}
\item {Utilização:Prov.}
\end{itemize}
\begin{itemize}
\item {Utilização:alg.}
\end{itemize}
\begin{itemize}
\item {Proveniência:(De \textunderscore pinha\textunderscore )}
\end{itemize}
Agrupamento; porção.
Acto de apinhar.
Cacho; pinhota.
\section{Pinhoca}
\begin{itemize}
\item {Grp. gram.:f.}
\end{itemize}
\begin{itemize}
\item {Utilização:Prov.}
\end{itemize}
\begin{itemize}
\item {Utilização:beir.}
\end{itemize}
\begin{itemize}
\item {Utilização:Ant.}
\end{itemize}
Cada um dos canziz, que seguram a canga no pescoço dos bois; pinhola.
\section{Pinhoéla}
\begin{itemize}
\item {Grp. gram.:f.}
\end{itemize}
Tecido de seda, com círculos avelludados:«\textunderscore deixo um vestido de pinhoéla...\textunderscore ». (De um testamento de 1693)
\section{Pinhões-de-ratos}
\begin{itemize}
\item {Grp. gram.:m. pl.}
\end{itemize}
Planta crassulácea, (\textunderscore sedum album\textunderscore , Lin.), também conhecida por \textunderscore arroz-dos-telhados\textunderscore .
\section{Pinhola}
\begin{itemize}
\item {Grp. gram.:f.}
\end{itemize}
\begin{itemize}
\item {Utilização:Prov.}
\end{itemize}
\begin{itemize}
\item {Utilização:beir.}
\end{itemize}
Cada um dos canziz, que seguram a canga ao pescoço do boi.
Mollusco gasterópode.
\section{Pinhota}
\begin{itemize}
\item {Grp. gram.:f.}
\end{itemize}
\begin{itemize}
\item {Proveniência:(De \textunderscore pinha\textunderscore )}
\end{itemize}
Cacho de flôres; corymbo.
\section{Pinhum}
\begin{itemize}
\item {Grp. gram.:m.}
\end{itemize}
\begin{itemize}
\item {Utilização:Bras}
\end{itemize}
O mesmo que \textunderscore pium\textunderscore .
\section{Pínico}
\begin{itemize}
\item {Grp. gram.:adj.}
\end{itemize}
Diz-se de um dos três ácidos contidos na resina do pinheiro.
\section{Pinífero}
\begin{itemize}
\item {Grp. gram.:adj.}
\end{itemize}
\begin{itemize}
\item {Utilização:Poét.}
\end{itemize}
O mesmo que \textunderscore pinígero\textunderscore .
\section{Piniforme}
\begin{itemize}
\item {Grp. gram.:adj.}
\end{itemize}
O mesmo ou melhor que \textunderscore pinhiforme\textunderscore .
\section{Pinígero}
\begin{itemize}
\item {Grp. gram.:adj.}
\end{itemize}
\begin{itemize}
\item {Utilização:Poét.}
\end{itemize}
\begin{itemize}
\item {Proveniência:(Lat. \textunderscore piniger\textunderscore )}
\end{itemize}
Que tem pinheiros; plantado de pinheiros. Cf. Castilho, \textunderscore Fastos\textunderscore , II, 13.
\section{Pinilho}
\begin{itemize}
\item {Grp. gram.:m.}
\end{itemize}
Espécie de planta, mencionada por Brotero.
\section{Pinima}
\begin{itemize}
\item {Grp. gram.:f.}
\end{itemize}
\begin{itemize}
\item {Utilização:Bras}
\end{itemize}
Ave, espécie de mutum.
\section{Pinipicrina}
\begin{itemize}
\item {Grp. gram.:f.}
\end{itemize}
\begin{itemize}
\item {Proveniência:(Do lat. \textunderscore pinus\textunderscore )}
\end{itemize}
Substância amarga, achada nos pinheiros da Escócia.
\section{Pinita}
\begin{itemize}
\item {Grp. gram.:f.}
\end{itemize}
\begin{itemize}
\item {Utilização:Miner.}
\end{itemize}
\begin{itemize}
\item {Proveniência:(De \textunderscore Pini\textunderscore , n. p.)}
\end{itemize}
Silicato de alumina e ferro.
\section{Pinito}
\begin{itemize}
\item {Grp. gram.:m.}
\end{itemize}
O mesmo ou melhor que \textunderscore pinita\textunderscore .
\section{Pinnatífido}
\begin{itemize}
\item {Grp. gram.:adj.}
\end{itemize}
\begin{itemize}
\item {Utilização:Bot.}
\end{itemize}
\begin{itemize}
\item {Proveniência:(Do lat. \textunderscore pinna\textunderscore  + \textunderscore findere\textunderscore )}
\end{itemize}
Diz-se das fôlhas, que são fendidas como as pennas.
\section{Pínnula}
\begin{itemize}
\item {Grp. gram.:f.}
\end{itemize}
\begin{itemize}
\item {Utilização:Zool.}
\end{itemize}
\begin{itemize}
\item {Utilização:Bot.}
\end{itemize}
\begin{itemize}
\item {Proveniência:(Lat. \textunderscore pínnula\textunderscore )}
\end{itemize}
Cada uma das peças laminares que, situadas nos extremos da alidade, têm um orifício, por onde passam os raios visuaes, para se fazerem alinhamentos.
Gênero de molluscos.
Cada um dos folíolos ou divisões das fôlhas compostas.
\section{Pinnulado}
\begin{itemize}
\item {Grp. gram.:adj.}
\end{itemize}
Que tem pínnulas.
\section{Pino}
\begin{itemize}
\item {Grp. gram.:m.}
\end{itemize}
\begin{itemize}
\item {Utilização:Prov.}
\end{itemize}
\begin{itemize}
\item {Proveniência:(Do ingl. \textunderscore pin\textunderscore ?)}
\end{itemize}
Espécie de prego de pinho ou cana, usado pelos sapateiros.
O ponto mais alto.
Zenithe.
Auge: \textunderscore no pino do calor\textunderscore .
Queimadeira, (planta).
Espécie de jôgo popular.
Cada um dos paus, a que se atira a malha ou a bola, jogando.
\section{Pino}
\begin{itemize}
\item {Grp. gram.:m.}
\end{itemize}
\begin{itemize}
\item {Utilização:Chul.}
\end{itemize}
\begin{itemize}
\item {Proveniência:(De \textunderscore pinar\textunderscore ^1)}
\end{itemize}
Casa de meretrizes.
\section{Pinoca}
\begin{itemize}
\item {Grp. gram.:m.  e  adj.}
\end{itemize}
\begin{itemize}
\item {Utilização:Chul.}
\end{itemize}
Indivíduo casquilho, janota.
O mesmo que \textunderscore pelintra\textunderscore .
\section{Pinoco}
\begin{itemize}
\item {Grp. gram.:m.}
\end{itemize}
\begin{itemize}
\item {Utilização:Prov.}
\end{itemize}
\begin{itemize}
\item {Utilização:trasm.}
\end{itemize}
\begin{itemize}
\item {Proveniência:(De \textunderscore pino\textunderscore ^1)}
\end{itemize}
O ponto mais alto de um monte ou de uma serra.
Marco geodésico.
Figura de neve, feita pelos rapazes, brincando.
\section{Pinoguaçu}
\begin{itemize}
\item {Grp. gram.:m.}
\end{itemize}
O mesmo que \textunderscore mamoeiro\textunderscore .
\section{Pinóia}
\begin{itemize}
\item {Grp. gram.:f.}
\end{itemize}
\begin{itemize}
\item {Utilização:Chul.}
\end{itemize}
\begin{itemize}
\item {Utilização:Gír. de Lisbôa.}
\end{itemize}
Mulhér taful e de costumes fáceis.
Pechincha, bom negócio.
\section{Pinóio}
\begin{itemize}
\item {Grp. gram.:m.}
\end{itemize}
\begin{itemize}
\item {Utilização:Prov.}
\end{itemize}
\begin{itemize}
\item {Utilização:trasm.}
\end{itemize}
\begin{itemize}
\item {Utilização:Bras}
\end{itemize}
Tunante, vádio, gandaieiro.
Coisa ordinária.
\section{Pinote}
\begin{itemize}
\item {Grp. gram.:m.}
\end{itemize}
\begin{itemize}
\item {Proveniência:(De \textunderscore pino\textunderscore ^1)}
\end{itemize}
Salto de cavalgadura.
Pulo; piruêta.
\section{Pinotear}
\begin{itemize}
\item {Grp. gram.:v. i.}
\end{itemize}
(V.espinotear)
\section{Pinque}
\begin{itemize}
\item {Grp. gram.:m.}
\end{itemize}
Embarcação do Mediterrâneo.
\section{Pinta}
\begin{itemize}
\item {Grp. gram.:f.}
\end{itemize}
\begin{itemize}
\item {Utilização:Pop.}
\end{itemize}
\begin{itemize}
\item {Utilização:Açor}
\end{itemize}
\begin{itemize}
\item {Utilização:Chul.}
\end{itemize}
\begin{itemize}
\item {Proveniência:(Do lat. hyp. \textunderscore pincta\textunderscore )}
\end{itemize}
Pequena mancha, malha.
Apparência, physionomia: \textunderscore pessôa de má pinta\textunderscore .
As partes pudendas da mulhér.
\section{Pinta}
\begin{itemize}
\item {Grp. gram.:f.}
\end{itemize}
O mesmo que \textunderscore pintaínha\textunderscore .
(Fem. de \textunderscore pinto\textunderscore )
\section{Pinta}
\begin{itemize}
\item {Grp. gram.:f.}
\end{itemize}
Antiga medida portuguesa, de 3 quartilhos, para líquidos, ou da quarta parte de um alqueire, para sólidos.
\section{Pinta-caldeira}
\begin{itemize}
\item {Grp. gram.:m.}
\end{itemize}
\begin{itemize}
\item {Utilização:Prov.}
\end{itemize}
Ave, o mesmo que \textunderscore ferreirinho\textunderscore , também conhecida por \textunderscore papa-abelhas\textunderscore , \textunderscore cedovém\textunderscore , \textunderscore fradisco\textunderscore , etc.
\section{Pinta-cardeira}
\begin{itemize}
\item {Grp. gram.:f.}
\end{itemize}
Nome que, nos arredores de Coímbra, se dá ao \textunderscore pintasilgo\textunderscore .
O mesmo que \textunderscore ferreirinho\textunderscore , ou \textunderscore pinta-caldeira\textunderscore .
\section{Pinta-cega}
\begin{itemize}
\item {Grp. gram.:f.}
\end{itemize}
O mesmo que \textunderscore noitibó\textunderscore .
\section{Pintada}
\begin{itemize}
\item {Grp. gram.:f.}
\end{itemize}
Gallinha da Índia.
Boga do rio Minho.
(Fem. de \textunderscore pintado\textunderscore )
\section{Pinta-da-erva}
\begin{itemize}
\item {Grp. gram.:f.}
\end{itemize}
Ave pernalta aquática, (\textunderscore porzanna marmorata\textunderscore ). Cf. P. Moraes, \textunderscore Zool. Elem.\textunderscore , 389.
\section{Pinta-de-água}
\begin{itemize}
\item {Grp. gram.:f.}
\end{itemize}
Ave, o mesmo que \textunderscore fura-mato\textunderscore . Cf. P. Moraes, \textunderscore Zool. Elem.\textunderscore , 388.
\section{Pintadela}
\begin{itemize}
\item {Grp. gram.:f.}
\end{itemize}
Acto de pintar ligeiramente; uma demão de tinta.
\section{Pintadina}
\begin{itemize}
\item {Grp. gram.:f.}
\end{itemize}
Mollusco, que destila a pérola.
Ostra perlífera, (\textunderscore meleagrina margaritifera\textunderscore ).
\section{Pintado}
\begin{itemize}
\item {Grp. gram.:adj.}
\end{itemize}
\begin{itemize}
\item {Utilização:Fig.}
\end{itemize}
\begin{itemize}
\item {Proveniência:(De \textunderscore pintar\textunderscore )}
\end{itemize}
Completo; perfeito: \textunderscore nem o mais pintado faria aquillo\textunderscore .
\section{Pintador}
\begin{itemize}
\item {Grp. gram.:m.}
\end{itemize}
\begin{itemize}
\item {Utilização:Des.}
\end{itemize}
\begin{itemize}
\item {Proveniência:(De \textunderscore pintar\textunderscore )}
\end{itemize}
O mesmo que \textunderscore pintor\textunderscore .--Em Coímbra, havia uma \textunderscore rua dos Pintadores\textunderscore . Cf. O periódico \textunderscore Conimbricense\textunderscore  de 1-IV-99.
\section{Pintaínha}
\textunderscore fem.\textunderscore  de \textunderscore pintaínho\textunderscore .
\section{Pintaínho}
\begin{itemize}
\item {Grp. gram.:m.}
\end{itemize}
Pequeno pinto, ainda implume ou quási implume.
Espécie de jôgo popular.
O mesmo que \textunderscore pintinho\textunderscore .
\section{Pintalegreiro}
\begin{itemize}
\item {Grp. gram.:adj.}
\end{itemize}
\begin{itemize}
\item {Utilização:P. us.}
\end{itemize}
Relativo a pintalegrete.
Casquilho e presumido. Cf. Arn. Gama, \textunderscore Segr. do Abb.\textunderscore , 85.
(Cp. \textunderscore pintalegrete\textunderscore )
\section{Pintalegrete}
\begin{itemize}
\item {fónica:grê}
\end{itemize}
\begin{itemize}
\item {Grp. gram.:m.  e  adj.}
\end{itemize}
\begin{itemize}
\item {Grp. gram.:Adj.}
\end{itemize}
\begin{itemize}
\item {Utilização:Prov.}
\end{itemize}
\begin{itemize}
\item {Utilização:trasm.}
\end{itemize}
\begin{itemize}
\item {Proveniência:(De \textunderscore pintar\textunderscore  + \textunderscore alegrete\textunderscore )}
\end{itemize}
Homem vaidoso, affectadamente vestido; peralta.
O mesmo que \textunderscore alegre\textunderscore . (Colhido em V. P. de Aguiar)
\section{Pintalegrismo}
\begin{itemize}
\item {Grp. gram.:m.}
\end{itemize}
Qualidade de pintalegrete ou de pintalegreiro. Cf. Arn. Gama, \textunderscore Últ. Dona\textunderscore , 26.
\section{Pintalgar}
\begin{itemize}
\item {Grp. gram.:v. t.}
\end{itemize}
\begin{itemize}
\item {Proveniência:(De \textunderscore pintar\textunderscore )}
\end{itemize}
Pintar de côres variegadas; encher de côres diversas.
Sarapintar.
\section{Pintalha}
\begin{itemize}
\item {Grp. gram.:f.}
\end{itemize}
\begin{itemize}
\item {Utilização:Prov.}
\end{itemize}
\begin{itemize}
\item {Utilização:T. de Aveiro}
\end{itemize}
Enxadada, com que, de espaço a espaço, se indicam os limites de um terreno.
Caniço ou estaca, com que se limita o viveiro de piscicultura ou o espaço em que se colhe o molliço.
\section{Pintalhão}
\begin{itemize}
\item {Grp. gram.:m.}
\end{itemize}
\begin{itemize}
\item {Utilização:Prov.}
\end{itemize}
O mesmo que \textunderscore colhereira\textunderscore .
Nome que, nalgumas terras do Minho, dão ao tentilhão.
\section{Pintalhar}
\begin{itemize}
\item {Grp. gram.:v. t.}
\end{itemize}
\begin{itemize}
\item {Utilização:T. de Aveiro}
\end{itemize}
Pôr pintalhas em.
\section{Pintalrar}
\begin{itemize}
\item {Grp. gram.:v. t.}
\end{itemize}
\begin{itemize}
\item {Utilização:Prov.}
\end{itemize}
(V.pintalgar)
\section{Pinta-monos}
\begin{itemize}
\item {Grp. gram.:m.}
\end{itemize}
\begin{itemize}
\item {Utilização:Fam.}
\end{itemize}
Pintor ordinário.
\section{Pintão}
\begin{itemize}
\item {Grp. gram.:m.}
\end{itemize}
\begin{itemize}
\item {Utilização:Prov.}
\end{itemize}
\begin{itemize}
\item {Proveniência:(De \textunderscore pinto\textunderscore )}
\end{itemize}
O filho da gallinhola.
Janota, pintalegrete. Cf. \textunderscore Eufrosina\textunderscore , 42.
\section{Pintar}
\begin{itemize}
\item {Grp. gram.:v. t.}
\end{itemize}
\begin{itemize}
\item {Utilização:Pop.}
\end{itemize}
\begin{itemize}
\item {Grp. gram.:Loc.}
\end{itemize}
\begin{itemize}
\item {Utilização:pop.}
\end{itemize}
\begin{itemize}
\item {Utilização:Prov.}
\end{itemize}
\begin{itemize}
\item {Utilização:minh.}
\end{itemize}
\begin{itemize}
\item {Grp. gram.:V. i.}
\end{itemize}
\begin{itemize}
\item {Proveniência:(Do lat. hyp. \textunderscore pinctare\textunderscore )}
\end{itemize}
Cobrir de tinta; representar por traços ou côres: \textunderscore pintar uma cruz\textunderscore .
Colorir.
Descrever minuciosamente, fielmente: \textunderscore pintar uma tempestade\textunderscore .
Burlar; illudir.
\textunderscore Pintar a manta\textunderscore , fazer coisas extraordinárias, fazer diabruras.
Dar nas vistas, sêr vistoso ou taful:«(a Genoveva) \textunderscore pinta a manta nas romarias\textunderscore ». Camillo, \textunderscore Brasileira\textunderscore , 242.
Começar a colorir-se: \textunderscore as uvas já pintam\textunderscore .
\section{Pintarroxo}
\begin{itemize}
\item {fónica:rô}
\end{itemize}
\begin{itemize}
\item {Grp. gram.:m.}
\end{itemize}
\begin{itemize}
\item {Proveniência:(De \textunderscore pintar\textunderscore  + \textunderscore roxo\textunderscore )}
\end{itemize}
Pássaro conirostro, (\textunderscore linota cannabina\textunderscore ).
\section{Pintasilga}
\begin{itemize}
\item {fónica:sil}
\end{itemize}
\begin{itemize}
\item {Grp. gram.:f.}
\end{itemize}
\begin{itemize}
\item {Utilização:Mad}
\end{itemize}
O mesmo que \textunderscore pintasilgo\textunderscore .
\section{Pintasilgo}
\begin{itemize}
\item {fónica:sil}
\end{itemize}
\begin{itemize}
\item {Grp. gram.:m.}
\end{itemize}
Pássaro conirostro, (\textunderscore carduelis elegans\textunderscore , Steph.).
(Cp. \textunderscore pintasirgo\textunderscore )
\section{Pintasilgo-da-terra}
\begin{itemize}
\item {Grp. gram.:m.}
\end{itemize}
\begin{itemize}
\item {Utilização:Bras}
\end{itemize}
Pintasilgo escuro do Brasil.
\section{Pintasilgo-derrabado}
\begin{itemize}
\item {Grp. gram.:m.}
\end{itemize}
\begin{itemize}
\item {Utilização:Mad}
\end{itemize}
O mesmo que \textunderscore abibe\textunderscore .
\section{Pintasilgo-do-reino}
\begin{itemize}
\item {Grp. gram.:m.}
\end{itemize}
Nome que, no Brasil, se dá ao pintasilgo de Portugal.
\section{Pintasilgo-verde}
\begin{itemize}
\item {Grp. gram.:m.}
\end{itemize}
Ave, também chamada por lugre. Cf. P. Moraes, \textunderscore Zool. Elem.\textunderscore , 313.
\section{Pintasilvo}
\begin{itemize}
\item {fónica:sil}
\end{itemize}
\begin{itemize}
\item {Grp. gram.:m.}
\end{itemize}
\begin{itemize}
\item {Utilização:Mad}
\end{itemize}
O mesmo que \textunderscore pintasilgo\textunderscore .
\section{Pintasirgo}
\begin{itemize}
\item {fónica:sir}
\end{itemize}
\begin{itemize}
\item {Grp. gram.:m.}
\end{itemize}
\begin{itemize}
\item {Proveniência:(De \textunderscore pintar\textunderscore  + \textunderscore sirgo\textunderscore ?)}
\end{itemize}
O mesmo ou melhor que \textunderscore pintasilgo\textunderscore ?
\section{Pintassilga}
\begin{itemize}
\item {Grp. gram.:f.}
\end{itemize}
\begin{itemize}
\item {Utilização:Mad}
\end{itemize}
O mesmo que \textunderscore pintasilgo\textunderscore .
\section{Pintassilgo}
\begin{itemize}
\item {Grp. gram.:m.}
\end{itemize}
Pássaro conirostro, (\textunderscore carduelis elegans\textunderscore , Steph.).
(Cp. \textunderscore pintassirgo\textunderscore )
\section{Pintassilvo}
\begin{itemize}
\item {Grp. gram.:m.}
\end{itemize}
\begin{itemize}
\item {Utilização:Mad}
\end{itemize}
O mesmo que \textunderscore pintassilgo\textunderscore .
\section{Pintassirgo}
\begin{itemize}
\item {Grp. gram.:m.}
\end{itemize}
\begin{itemize}
\item {Proveniência:(De \textunderscore pintar\textunderscore  + \textunderscore sirgo\textunderscore ?)}
\end{itemize}
O mesmo ou melhor que \textunderscore pintassilgo\textunderscore ?
\section{Pintaxilgo}
\begin{itemize}
\item {Grp. gram.:m.}
\end{itemize}
(V.pintexilgo)
\section{Pintelho}
\begin{itemize}
\item {fónica:tê}
\end{itemize}
\begin{itemize}
\item {Grp. gram.:m.}
\end{itemize}
\begin{itemize}
\item {Utilização:Mad}
\end{itemize}
Espécie de pintasilgo, (\textunderscore puffinus obscurus\textunderscore , Temm.).
\section{Pintexilgo}
\begin{itemize}
\item {Grp. gram.:m.}
\end{itemize}
O mesmo que \textunderscore pintasilgo\textunderscore .--G. Guimarães, \textunderscore Rev. da Univ. de Coímbra\textunderscore , I, 8, entende que \textunderscore pintexilgo\textunderscore  é fórma onomatopaica e exacta, devendo considerar-se corrupção desta as demais, com que se nomeia a respectiva ave.
\section{Pintinho}
\begin{itemize}
\item {Grp. gram.:m.}
\end{itemize}
\begin{itemize}
\item {Utilização:Mad}
\end{itemize}
O mesmo que \textunderscore pintelho\textunderscore .
\section{Pintiparado}
\begin{itemize}
\item {Grp. gram.:adj.}
\end{itemize}
\begin{itemize}
\item {Utilização:Des.}
\end{itemize}
\begin{itemize}
\item {Proveniência:(Do lat. \textunderscore pictus\textunderscore  + \textunderscore paratus\textunderscore )}
\end{itemize}
Pintado ao vivo; bem figurado, bem representado. Cf. Filinto, XIX, 252.
\section{Pinto}
\begin{itemize}
\item {Grp. gram.:m.}
\end{itemize}
\begin{itemize}
\item {Utilização:Gír.}
\end{itemize}
\begin{itemize}
\item {Utilização:Bras. do Maranhão}
\end{itemize}
\begin{itemize}
\item {Proveniência:(Lat. hyp. \textunderscore pinctus\textunderscore  = \textunderscore pictus\textunderscore )}
\end{itemize}
Frangaínho.
Moéda portuguesa de prata, do valor de 480 reis; cruzado novo.
Criança.
\textunderscore Fazer pinto\textunderscore , diz-se do criado, que furta pequenas quantias nas compras diárias.
\textunderscore Estar num pinto\textunderscore , ou \textunderscore ficar num pinto\textunderscore , estar molhado ou ficar molhado, como um pinto ao sair da casca.
\section{Pinto-bravo}
\begin{itemize}
\item {Grp. gram.:m.}
\end{itemize}
O mesmo que \textunderscore codornizão\textunderscore .
\section{Pinto-cardeiro}
\begin{itemize}
\item {Grp. gram.:m.}
\end{itemize}
\begin{itemize}
\item {Utilização:T. da Bairrada}
\end{itemize}
O mesmo que \textunderscore pinta-cardeira\textunderscore .
\section{Pintor}
\begin{itemize}
\item {Grp. gram.:m.}
\end{itemize}
\begin{itemize}
\item {Utilização:T. de Ílhavo}
\end{itemize}
\begin{itemize}
\item {Grp. gram.:Pl.}
\end{itemize}
\begin{itemize}
\item {Utilização:Prov.}
\end{itemize}
\begin{itemize}
\item {Utilização:minh.}
\end{itemize}
\begin{itemize}
\item {Proveniência:(Lat. hyp. \textunderscore pinctor\textunderscore  = \textunderscore pictor\textunderscore )}
\end{itemize}
Aquelle que pinta.
O que sabe ou exerce a arte de pintar.
Aquelle que mente por chalaça.
Primeiros bagos coloridos nos cachos de uvas.
\section{Pintora}
\begin{itemize}
\item {Grp. gram.:f.}
\end{itemize}
\begin{itemize}
\item {Proveniência:(De \textunderscore pintor\textunderscore )}
\end{itemize}
Mulhér, que pinta ou que sabe pintar.
\section{Pintorizado}
\begin{itemize}
\item {Grp. gram.:adj.}
\end{itemize}
\begin{itemize}
\item {Utilização:Bras}
\end{itemize}
O mesmo que \textunderscore pinturesco\textunderscore .
\section{Pintorrocha}
\begin{itemize}
\item {fónica:rô}
\end{itemize}
\begin{itemize}
\item {Grp. gram.:f.}
\end{itemize}
\begin{itemize}
\item {Utilização:Mad}
\end{itemize}
O mesmo que \textunderscore pintorroxo\textunderscore .
\section{Pintorroxo}
\begin{itemize}
\item {fónica:rô}
\end{itemize}
\begin{itemize}
\item {Grp. gram.:m.}
\end{itemize}
\begin{itemize}
\item {Utilização:T. da Bairrada e da Mad}
\end{itemize}
O mesmo que \textunderscore pintarroxo\textunderscore .
\section{Pintura}
\begin{itemize}
\item {Grp. gram.:f.}
\end{itemize}
\begin{itemize}
\item {Utilização:Fig.}
\end{itemize}
\begin{itemize}
\item {Proveniência:(Lat. hyp. \textunderscore pinctura\textunderscore  = \textunderscore pictura\textunderscore )}
\end{itemize}
Arte de pintar.
Obra feita por pintor.
Quadro; côr.
Descripção minuciosa.
Representação escrita ou verbal.
Pessôa formosa; coisa perfeita.
\section{Pinturesco}
\begin{itemize}
\item {Grp. gram.:adj.}
\end{itemize}
\begin{itemize}
\item {Utilização:Fig.}
\end{itemize}
\begin{itemize}
\item {Grp. gram.:M.}
\end{itemize}
\begin{itemize}
\item {Proveniência:(De \textunderscore pintura\textunderscore )}
\end{itemize}
Pictórico.
Próprio para sêr pintado.
Que merece sêr pintado.
Recreativo.
Imaginoso.
Scintillante.
Aquillo que é pinturesco. Cf. Herculano, \textunderscore Eurico\textunderscore , c. XIII.
\section{Pinturilar}
\begin{itemize}
\item {Grp. gram.:v. t.}
\end{itemize}
\begin{itemize}
\item {Utilização:Neol.}
\end{itemize}
\begin{itemize}
\item {Proveniência:(De \textunderscore pintura\textunderscore )}
\end{itemize}
Pintar sem arte, como ao acaso, a capricho, sem escolha de tintas.
\section{Pínula}
\begin{itemize}
\item {Grp. gram.:f.}
\end{itemize}
\begin{itemize}
\item {Utilização:Zool.}
\end{itemize}
\begin{itemize}
\item {Utilização:Bot.}
\end{itemize}
\begin{itemize}
\item {Proveniência:(Lat. \textunderscore pínnula\textunderscore )}
\end{itemize}
Cada uma das peças laminares que, situadas nos extremos da alidade, têm um orifício, por onde passam os raios visuaes, para se fazerem alinhamentos.
Gênero de moluscos.
Cada um dos folíolos ou divisões das fôlhas compostas.
\section{Pinulado}
\begin{itemize}
\item {Grp. gram.:adj.}
\end{itemize}
Que tem pínulas.
\section{Pinzel}
\begin{itemize}
\item {Grp. gram.:m.}
\end{itemize}
\begin{itemize}
\item {Utilização:Ant.}
\end{itemize}
O mesmo que \textunderscore pincel\textunderscore . Cf. Pant. de Aveiro, \textunderscore Itiner.\textunderscore , 2 v.^o, (2.^a ed.).
\section{Pio}
\begin{itemize}
\item {Grp. gram.:m.}
\end{itemize}
\begin{itemize}
\item {Grp. gram.:Loc. interj.}
\end{itemize}
\begin{itemize}
\item {Proveniência:(De \textunderscore piar\textunderscore )}
\end{itemize}
Acto de piar.
Voz de algumas aves, especialmente do mocho.
Som imitativo do grito de algumas aves.
\textunderscore Nem pio\textunderscore !, silêncio!
\section{Pio}
\begin{itemize}
\item {Grp. gram.:m.}
\end{itemize}
\begin{itemize}
\item {Utilização:Prov.}
\end{itemize}
\begin{itemize}
\item {Utilização:trasm.}
\end{itemize}
\begin{itemize}
\item {Utilização:Prov.}
\end{itemize}
\begin{itemize}
\item {Utilização:alent.}
\end{itemize}
Pia grande ou pequeno lagar, em que se pisam uvas.
O mesmo que \textunderscore pieiro\textunderscore .
O mesmo que \textunderscore tanque\textunderscore .
(Cp. \textunderscore pia\textunderscore )
\section{Pio}
\begin{itemize}
\item {Grp. gram.:adj.}
\end{itemize}
\begin{itemize}
\item {Proveniência:(Lat. \textunderscore pius\textunderscore )}
\end{itemize}
Piedoso.
Que revela piedade ou caridade.
Caridoso; compassivo.
\section{Pio}
\begin{itemize}
\item {Grp. gram.:adj.}
\end{itemize}
\begin{itemize}
\item {Utilização:Gír.}
\end{itemize}
\begin{itemize}
\item {Grp. gram.:M.}
\end{itemize}
\begin{itemize}
\item {Utilização:Gír.}
\end{itemize}
Bêbedo.
Vinho.
(Cp. \textunderscore piar\textunderscore ^2)
\section{Pió}
\begin{itemize}
\item {Grp. gram.:m.}
\end{itemize}
\begin{itemize}
\item {Proveniência:(Do lat. hyp. \textunderscore pediola\textunderscore )}
\end{itemize}
Correia, que os caçadores de altanaria punham nos sancos do falcão ou do açor. Cf. Fernandes, \textunderscore Caça de Altan.\textunderscore , onde se lê \textunderscore pió\textunderscore .
\section{Pió!}
\begin{itemize}
\item {Utilização:Ant.}
\end{itemize}
Designação onom. da voz de gallinhas novas.
\section{Piôa}
\begin{itemize}
\item {Grp. gram.:f.}
\end{itemize}
\begin{itemize}
\item {Utilização:Prov.}
\end{itemize}
O mesmo que \textunderscore pitorra\textunderscore .
\section{Piocamecrans}
\begin{itemize}
\item {Grp. gram.:m. pl.}
\end{itemize}
Tríbo de Índios do Brasil, ao oriente do Tocantins.
\section{Pioguinha}
\begin{itemize}
\item {Grp. gram.:f.}
\end{itemize}
\begin{itemize}
\item {Utilização:Prov.}
\end{itemize}
\begin{itemize}
\item {Utilização:alent.}
\end{itemize}
Pão pequeno.
(Por \textunderscore piorrinha\textunderscore , de \textunderscore piorra\textunderscore !)
\section{Pioio}
\begin{itemize}
\item {Grp. gram.:m.}
\end{itemize}
\begin{itemize}
\item {Utilização:Bras}
\end{itemize}
Planta, também chamada \textunderscore piôlho\textunderscore .
\section{Piolhada}
\begin{itemize}
\item {Grp. gram.:f.}
\end{itemize}
Porção de piolhos.
O mesmo que \textunderscore piolheira\textunderscore .
\section{Piolhar}
\begin{itemize}
\item {Grp. gram.:v. i.}
\end{itemize}
\begin{itemize}
\item {Utilização:Açor}
\end{itemize}
\begin{itemize}
\item {Proveniência:(De \textunderscore piolho\textunderscore )}
\end{itemize}
Formar-se neblina.
\section{Piolharia}
\begin{itemize}
\item {Grp. gram.:f.}
\end{itemize}
\begin{itemize}
\item {Utilização:Fig.}
\end{itemize}
\begin{itemize}
\item {Utilização:Des.}
\end{itemize}
\begin{itemize}
\item {Utilização:Fam.}
\end{itemize}
Grande porção de piolhos.
Pobreza extrema.
Villania; indignidade.
\section{Piolheira}
\begin{itemize}
\item {Grp. gram.:f.}
\end{itemize}
\begin{itemize}
\item {Utilização:Fig.}
\end{itemize}
\begin{itemize}
\item {Utilização:Pop.}
\end{itemize}
\begin{itemize}
\item {Proveniência:(De \textunderscore piolho\textunderscore )}
\end{itemize}
Nome de uma erva.
Piolharia.
Habitação immunda.
Conjunto de coisas immundas ou miseráveis.
Negócio de pouca monta.
\section{Piolheiro}
\begin{itemize}
\item {Grp. gram.:adj.}
\end{itemize}
\begin{itemize}
\item {Grp. gram.:M.}
\end{itemize}
Que cria piolhos; coberto de piolhos.
Indivíduo piolhento.
\section{Piolhento}
\begin{itemize}
\item {Grp. gram.:adj.}
\end{itemize}
\begin{itemize}
\item {Grp. gram.:M.}
\end{itemize}
Que cria piolhos; coberto de piolhos.
Indivíduo piolhento.
\section{Piolhice}
\begin{itemize}
\item {Grp. gram.:f.}
\end{itemize}
\begin{itemize}
\item {Utilização:Des.}
\end{itemize}
\begin{itemize}
\item {Utilização:Pop.}
\end{itemize}
\begin{itemize}
\item {Proveniência:(De \textunderscore piolho\textunderscore )}
\end{itemize}
Mesquinharia; coisa sem importância; questiúncula.
\section{Piolho}
\begin{itemize}
\item {fónica:ô}
\end{itemize}
\begin{itemize}
\item {Grp. gram.:m.}
\end{itemize}
\begin{itemize}
\item {Utilização:Açor}
\end{itemize}
\begin{itemize}
\item {Proveniência:(Do lat. hyp. \textunderscore peduculus\textunderscore )}
\end{itemize}
Insecto parasita, de que há várias espécies, sendo a principal o piolho da cabeça, (\textunderscore pediculus capitis\textunderscore ).
Arvore esmilácea do Brasil.
\textunderscore Chuva de piolho\textunderscore , o mesmo que \textunderscore neblina\textunderscore .
\section{Piolho-da-língua}
\begin{itemize}
\item {Grp. gram.:m.}
\end{itemize}
Espécie de borbilhões, que nascem sob a língua dos bovídeos e que têm a fórma de vermes. Cf. Macedo Pinto, \textunderscore Comp. de Veter.\textunderscore , I, 452.
\section{Piolho-de-cobra}
\begin{itemize}
\item {Grp. gram.:m.}
\end{itemize}
\begin{itemize}
\item {Utilização:Bras}
\end{itemize}
\begin{itemize}
\item {Utilização:Fig.}
\end{itemize}
O mesmo que \textunderscore centopeia\textunderscore .
Longo artigo de gazeta.
\section{Piolhoso}
\begin{itemize}
\item {Grp. gram.:m.  e  adj.}
\end{itemize}
O mesmo que \textunderscore piolhento\textunderscore .
\section{Piona}
\begin{itemize}
\item {Grp. gram.:f.}
\end{itemize}
\begin{itemize}
\item {Utilização:Prov.}
\end{itemize}
\begin{itemize}
\item {Utilização:trasm.}
\end{itemize}
Pião ordinário.
\section{Pioneiro}
\begin{itemize}
\item {Grp. gram.:m.}
\end{itemize}
\begin{itemize}
\item {Utilização:Neol.}
\end{itemize}
\begin{itemize}
\item {Utilização:Pesc.}
\end{itemize}
\begin{itemize}
\item {Proveniência:(Do fr. \textunderscore pionnier\textunderscore )}
\end{itemize}
Explorador de sertões.
O primeiro que abre ou descobre caminho através de uma região mal conhecida.
Cabo, com uma pedra amarrada numa extremidade e que da popa um barco se lança á água, para evitar as guinadas produzidas por correntes fortes.--É gal. dispensável. Melhor seria \textunderscore deanteiro\textunderscore .
\section{Pio-nono}
\begin{itemize}
\item {Grp. gram.:m.}
\end{itemize}
Espécie de capa curta sem cabeção, para homem.
\section{Piór}
\begin{itemize}
\item {Grp. gram.:adj.  e  adv.}
\end{itemize}
\begin{itemize}
\item {Grp. gram.:adj.}
\end{itemize}
\begin{itemize}
\item {Grp. gram.:Adv.}
\end{itemize}
\begin{itemize}
\item {Proveniência:(Lat. \textunderscore peior\textunderscore )}
\end{itemize}
O mesmo ou melhor que \textunderscore peór\textunderscore . Cf. Castilho, \textunderscore Tartufo\textunderscore , 15; \textunderscore Eufrosina\textunderscore , 44; G. Viana, \textunderscore Apostilas\textunderscore ; etc.
Mais mau.
Que se aggrava, em sentido desfavorável.
Que excede outro em maldade ou má qualidade.
De modo mais mau; em situação ou circumstâncias mais desfavoráveis.
\section{Piornal}
\begin{itemize}
\item {Grp. gram.:m.}
\end{itemize}
Campo, onde crescem piornos.
\section{Piorneira}
\begin{itemize}
\item {Grp. gram.:f.}
\end{itemize}
\begin{itemize}
\item {Utilização:Prov.}
\end{itemize}
\begin{itemize}
\item {Utilização:alent.}
\end{itemize}
O mesmo que \textunderscore piorno\textunderscore .
\section{Piorno}
\begin{itemize}
\item {fónica:ôr}
\end{itemize}
\begin{itemize}
\item {Grp. gram.:m.}
\end{itemize}
Planta leguminosa, (\textunderscore retama\textunderscore ).
Giesta brava, de suco amargo.
\section{Piôrra}
\begin{itemize}
\item {Grp. gram.:f.}
\end{itemize}
\begin{itemize}
\item {Proveniência:(De \textunderscore pião\textunderscore )}
\end{itemize}
Pião pequeno, pitôrra. Cf. Filinto, XIII, 193.
\section{Piôrro}
\begin{itemize}
\item {Grp. gram.:m.}
\end{itemize}
\begin{itemize}
\item {Utilização:Prov.}
\end{itemize}
\begin{itemize}
\item {Utilização:beir.}
\end{itemize}
\begin{itemize}
\item {Utilização:Fig.}
\end{itemize}
O mesmo que \textunderscore piôrra\textunderscore .
Indivíduo de pequena estatura.
\section{Piovês}
\begin{itemize}
\item {Grp. gram.:m.}
\end{itemize}
\begin{itemize}
\item {Utilização:Gír.}
\end{itemize}
Vinho.
(Cp. fr. \textunderscore pivois\textunderscore )
\section{Pipa}
\begin{itemize}
\item {Grp. gram.:f.}
\end{itemize}
\begin{itemize}
\item {Utilização:Bras. do N}
\end{itemize}
\begin{itemize}
\item {Utilização:Bras}
\end{itemize}
\begin{itemize}
\item {Utilização:Pop.}
\end{itemize}
\begin{itemize}
\item {Grp. gram.:Loc. adv.}
\end{itemize}
\begin{itemize}
\item {Utilização:fam.}
\end{itemize}
\begin{itemize}
\item {Proveniência:(Do lat. \textunderscore pipare\textunderscore ?)}
\end{itemize}
Vasilha bojuda de madeira, para vinhos e outros líquidos.
Grande saco de coiro cru, para guardar cereaes.
Espécie de cachimbo.
Pessôa gorda e baixa.
\textunderscore De três em pipa\textunderscore , valentemente, optimamente.
\section{Pipá}
\begin{itemize}
\item {Grp. gram.:m.}
\end{itemize}
Gênero de batrácios da América do Sul.
(Cp. \textunderscore pipal\textunderscore ^2)
\section{Pipal}
\begin{itemize}
\item {Grp. gram.:m.}
\end{itemize}
Árvore moreácea, (\textunderscore ficus índica\textunderscore ).
\section{Pipal}
\begin{itemize}
\item {Grp. gram.:m.}
\end{itemize}
Nome antigo do \textunderscore pipá\textunderscore .
\section{Piparota}
\begin{itemize}
\item {Grp. gram.:f.}
\end{itemize}
\begin{itemize}
\item {Utilização:T. do Fundão}
\end{itemize}
Gaita, feita de um pedaço de caule de trigo verde, centeio ou cevada.
\section{Piparote}
\begin{itemize}
\item {Grp. gram.:m.}
\end{itemize}
Pancada com a cabeça do dedo médio ou do índex, apoiado sôbre o pollegar, e soltando-se com fôrça.
(Cp. cast. \textunderscore papirote\textunderscore )
\section{Piperáceas}
\begin{itemize}
\item {Grp. gram.:f. pl.}
\end{itemize}
Família de plantas, que tem por typo a pimenteira.
(Fem. pl. de \textunderscore piperáceo\textunderscore )
\section{Piperáceo}
\begin{itemize}
\item {Grp. gram.:adj.}
\end{itemize}
\begin{itemize}
\item {Proveniência:(Do lat. \textunderscore piper\textunderscore )}
\end{itemize}
Relativo ou semelhante á pimenteira.
\section{Piperadizina}
\begin{itemize}
\item {Grp. gram.:f.}
\end{itemize}
O mesmo que \textunderscore piperidina\textunderscore .
\section{Piperazina}
\begin{itemize}
\item {Grp. gram.:f.}
\end{itemize}
O mesmo que \textunderscore piperidina\textunderscore .
\section{Piperidina}
\begin{itemize}
\item {Grp. gram.:f.}
\end{itemize}
\begin{itemize}
\item {Utilização:Chím.}
\end{itemize}
\begin{itemize}
\item {Proveniência:(Do lat. \textunderscore piper\textunderscore )}
\end{itemize}
Base volátil, resultante do desdobramento da piperina peles álcalis.
\section{Piperina}
\begin{itemize}
\item {Grp. gram.:f.}
\end{itemize}
\begin{itemize}
\item {Proveniência:(Do lat. \textunderscore piper\textunderscore )}
\end{itemize}
Producto pharmacêutico, que possue propriedades diuréticas, esternutatórias e diaphoréticas.
Alcalóide, descoberto na pimenta preta.
\section{Piperino}
\begin{itemize}
\item {Grp. gram.:m.}
\end{itemize}
\begin{itemize}
\item {Proveniência:(Lat. \textunderscore piperinus\textunderscore )}
\end{itemize}
Rocha porosa e vulcânica, vulgar cercanias de Roma e usada em construcções.
\section{Piperioca}
\begin{itemize}
\item {Grp. gram.:f.}
\end{itemize}
Planta cyperácea do Brasil.
\section{Piperíteas}
\begin{itemize}
\item {Grp. gram.:f. pl.}
\end{itemize}
O mesmo que \textunderscore piperáceas\textunderscore .
\section{Pipeta}
\begin{itemize}
\item {fónica:pê}
\end{itemize}
\begin{itemize}
\item {Grp. gram.:f.}
\end{itemize}
\begin{itemize}
\item {Proveniência:(De \textunderscore pipa\textunderscore )}
\end{itemize}
Bomba das adegas.
Tubo, que se introduz no batoque dos tonéis com o vinho e se retira tapando-se-lhe com o dedo o orifício superior.
\section{Pipi}
\begin{itemize}
\item {Grp. gram.:m.}
\end{itemize}
Árvore medicinal das regiões do Amazonas.
\section{Pipi}
\begin{itemize}
\item {Grp. gram.:f.}
\end{itemize}
\begin{itemize}
\item {Proveniência:(De \textunderscore Pipi\textunderscore , design. inf. de \textunderscore piedade\textunderscore . Cf. \textunderscore Diccion. das Peras\textunderscore , vb. \textunderscore pipi\textunderscore )}
\end{itemize}
Variedade de pêra doce e aromática, originária dos subúrbios de Viseu, onde se colheu pela primeira vez em 1877.
\section{Pipi}
\begin{itemize}
\item {Grp. gram.:m.}
\end{itemize}
\begin{itemize}
\item {Proveniência:(De \textunderscore pio\textunderscore ^1)}
\end{itemize}
Qualquer ave, especialmente gallinácea.
Voz, com que se chamam as gallináceas, para lhes dar de comer.
Nome de uma ave de Dio.
\section{Pipi}
\begin{itemize}
\item {Grp. gram.:m.}
\end{itemize}
\begin{itemize}
\item {Utilização:Infant.}
\end{itemize}
Membro viril das crianças.
(Cp. lat. \textunderscore pipinnus\textunderscore )
\section{Pipia}
\begin{itemize}
\item {Grp. gram.:f.}
\end{itemize}
\begin{itemize}
\item {Proveniência:(De \textunderscore pipiar\textunderscore )}
\end{itemize}
Pequeno tubo, feito geralmente da cana do trigo ou da cevada, e com que se produz um som estridente, abrindo-se-lhe uma fenda na extremidade, e soprando-se por ella.
Nome de uma ave brasileira.
\section{Pipião}
\begin{itemize}
\item {Grp. gram.:m.}
\end{itemize}
Antiga e pequena moéda portuguesa, do valor de duas mealhas.
\section{Pipiar}
\begin{itemize}
\item {Grp. gram.:v. i.}
\end{itemize}
\begin{itemize}
\item {Grp. gram.:M.}
\end{itemize}
\begin{itemize}
\item {Proveniência:(Lat. \textunderscore pipiare\textunderscore )}
\end{itemize}
O mesmo que \textunderscore pipilar\textunderscore .
O piar das aves.
\section{Pipilante}
\begin{itemize}
\item {Grp. gram.:adj.}
\end{itemize}
Que pipila. Cf. Filinto, II, 8.
\section{Pipilar}
\begin{itemize}
\item {Grp. gram.:v. i.}
\end{itemize}
\begin{itemize}
\item {Grp. gram.:M.}
\end{itemize}
\begin{itemize}
\item {Proveniência:(Lat. \textunderscore pipilare\textunderscore )}
\end{itemize}
O mesmo que \textunderscore piar\textunderscore ^1, (tratando-se de aves).
Produzir som semelhante á voz das aves.
O piar das aves.
\section{Pipilo}
\begin{itemize}
\item {Grp. gram.:m.}
\end{itemize}
Acto de pipilar.
\section{Pipilro}
\begin{itemize}
\item {Grp. gram.:m.}
\end{itemize}
Designação vulgar de uma planta parietária, de fôlhas circulares.
\section{Pipio}
\begin{itemize}
\item {Grp. gram.:m.}
\end{itemize}
\begin{itemize}
\item {Utilização:Bras}
\end{itemize}
Acto de pipiar.
Pintaínho.
\section{Pipira}
\begin{itemize}
\item {Grp. gram.:f.}
\end{itemize}
\begin{itemize}
\item {Utilização:Bras. do N}
\end{itemize}
\begin{itemize}
\item {Utilização:T. de Piauí}
\end{itemize}
Pássaro azul-cinzento ou pardo-escuro.
Rapariga, empregada em fábrica de tecidos.
\section{Pipiral}
\begin{itemize}
\item {Grp. gram.:m.}
\end{itemize}
\begin{itemize}
\item {Utilização:T. de Piauí}
\end{itemize}
Bairro de pipiras (raparigas de fábrica).
Reunião ou baile de pipiras.
\section{Pipirete}
\begin{itemize}
\item {fónica:pirê}
\end{itemize}
\begin{itemize}
\item {Grp. gram.:m.}
\end{itemize}
\begin{itemize}
\item {Utilização:Prov.}
\end{itemize}
\begin{itemize}
\item {Utilização:trasm.}
\end{itemize}
Acepipe; petisqueira.
Caspacho.
\section{Pipiri}
\begin{itemize}
\item {Grp. gram.:m.}
\end{itemize}
Planta cyperácea do Brasil.
Ave americana, cujo nome é a expressão onomatopaica do seu nome.
\section{Pipitar}
\begin{itemize}
\item {Grp. gram.:v. i.}
\end{itemize}
O mesmo que \textunderscore pipilar\textunderscore .
\section{Pipito}
\begin{itemize}
\item {Grp. gram.:m.}
\end{itemize}
Acto de pipitar:«\textunderscore retine já pelos ares o pipito dos sahycos\textunderscore ». Araújo Porto-Alegre.
\section{Pipo}
\begin{itemize}
\item {Grp. gram.:m.}
\end{itemize}
\begin{itemize}
\item {Utilização:T. da Bairrada}
\end{itemize}
\begin{itemize}
\item {Proveniência:(De \textunderscore pipa\textunderscore )}
\end{itemize}
Pipa pequena.
Barril.
Tubo, por onde se espipa o líquido contido em certas vasilhas.
Variedade de pêra.
Ouvido da espingarda que se carrega pela bôca.
\section{Pipoca}
\begin{itemize}
\item {Grp. gram.:f.}
\end{itemize}
\begin{itemize}
\item {Utilização:Bras}
\end{itemize}
Grão de milho, arrebentado ao calor do fogo, para se comer como biscoitos.
O mesmo que \textunderscore varíola-mansa\textunderscore .
(Do tupi \textunderscore apoc\textunderscore )
\section{Pipocamento}
\begin{itemize}
\item {Grp. gram.:m.}
\end{itemize}
Acto ou effeito de
\section{Pipocar}
\begin{itemize}
\item {Grp. gram.:v. t.  e  i.}
\end{itemize}
\begin{itemize}
\item {Utilização:Bras}
\end{itemize}
\begin{itemize}
\item {Grp. gram.:V. i.}
\end{itemize}
\begin{itemize}
\item {Proveniência:(De \textunderscore pipoca\textunderscore )}
\end{itemize}
Arrebentar; estalar.
Ferver em borbotões.
\section{Pipoco}
\begin{itemize}
\item {fónica:pô}
\end{itemize}
\begin{itemize}
\item {Grp. gram.:m.}
\end{itemize}
\begin{itemize}
\item {Utilização:Bras. do N}
\end{itemize}
\begin{itemize}
\item {Proveniência:(De \textunderscore pipocar\textunderscore )}
\end{itemize}
Estalada.
Contenda acalorada; desordem.
\section{Pipote}
\begin{itemize}
\item {Grp. gram.:m.}
\end{itemize}
O mesmo que \textunderscore pipo\textunderscore .
\section{Pipra}
\begin{itemize}
\item {Grp. gram.:f.}
\end{itemize}
Pássaro dentirostro da América do Sul.
\section{Píqua}
\begin{itemize}
\item {Grp. gram.:f.}
\end{itemize}
Peixe de Portugal.
(Cp. \textunderscore picha\textunderscore ^2, \textunderscore pica\textunderscore ^4 e \textunderscore pica-de-el-rei\textunderscore )
\section{Pique}
\begin{itemize}
\item {Grp. gram.:m.}
\end{itemize}
\begin{itemize}
\item {Utilização:Prov.}
\end{itemize}
\begin{itemize}
\item {Utilização:trasm.}
\end{itemize}
\begin{itemize}
\item {Grp. gram.:Loc. adv.}
\end{itemize}
\begin{itemize}
\item {Grp. gram.:Pl.}
\end{itemize}
\begin{itemize}
\item {Utilização:Náut.}
\end{itemize}
Espécie de lança antiga.
Sabor picante, pico.
Teima, questão, rixa.
\textunderscore A pique\textunderscore , em perigo; ao fundo; a ponto; verticalmente, a prumo: \textunderscore o navio foi a pique\textunderscore .
Lais das caranguejas do navio.
(Cp. \textunderscore pico\textunderscore ^1)
\section{Pique}
\begin{itemize}
\item {Grp. gram.:m.}
\end{itemize}
\begin{itemize}
\item {Utilização:Bras}
\end{itemize}
\begin{itemize}
\item {Utilização:Des.}
\end{itemize}
\begin{itemize}
\item {Proveniência:(De \textunderscore picar\textunderscore )}
\end{itemize}
Cartão de côr, com um desenho picado a alfinetes, e em que trabalham as rendeiras de bilros.
Acto de picar o tabaco, em as fabricas de cigarros e charutos.
Acto de picar o mato, para designar a direcção dos atalhos chamados \textunderscore picadas\textunderscore .
Acinte, propósito:«\textunderscore um autor..., por pique ou por desdem tem comparado...\textunderscore »Filinto, V, 151.
\section{Pique}
\begin{itemize}
\item {Grp. gram.:m.}
\end{itemize}
\begin{itemize}
\item {Utilização:Prov.}
\end{itemize}
\begin{itemize}
\item {Utilização:alg.}
\end{itemize}
Nome de um peixe, de cujos fígados se extrái um óleo medicinal, a que chamam \textunderscore azeite de pique\textunderscore . Será a \textunderscore píqua\textunderscore ?
\section{Piqué}
\begin{itemize}
\item {Grp. gram.:m.}
\end{itemize}
\begin{itemize}
\item {Proveniência:(Fr. \textunderscore piqué\textunderscore )}
\end{itemize}
Tecido, atravessado por séries de pontos muito apertados para lhe deminuírem a espessura.
\section{Piqueiro}
\begin{itemize}
\item {Grp. gram.:m.}
\end{itemize}
\begin{itemize}
\item {Utilização:Taur.}
\end{itemize}
\begin{itemize}
\item {Proveniência:(De \textunderscore picar\textunderscore )}
\end{itemize}
Aquelle que pica toiros com vara curta.
\section{Piqueiro}
\begin{itemize}
\item {Grp. gram.:m.}
\end{itemize}
\begin{itemize}
\item {Utilização:Ant.}
\end{itemize}
\begin{itemize}
\item {Proveniência:(De \textunderscore pique\textunderscore )}
\end{itemize}
Homem, armado de pique ou lança. Cf. R. Lobo, \textunderscore Côrte na Ald.\textunderscore , II, 88.
\section{Piquenique}
\begin{itemize}
\item {Grp. gram.:m.}
\end{itemize}
\begin{itemize}
\item {Proveniência:(Do ingl. \textunderscore picknick\textunderscore )}
\end{itemize}
Refeição festiva no campo, geralmente entre indivíduos de diversas famílias.
\section{Piqueno}
\begin{itemize}
\item {Grp. gram.:m.  e  adj.}
\end{itemize}
O mesmo que \textunderscore pequeno\textunderscore .
\section{Piqueta}
\begin{itemize}
\item {fónica:quê}
\end{itemize}
\begin{itemize}
\item {Grp. gram.:f.}
\end{itemize}
\begin{itemize}
\item {Utilização:Prov.}
\end{itemize}
\begin{itemize}
\item {Utilização:beir.}
\end{itemize}
Pequena refeição entre o almôço e o jantar; lanche.
(Por \textunderscore pequeta\textunderscore , relacionando-se com \textunderscore pequeno\textunderscore ?)
\section{Piqueta}
\begin{itemize}
\item {fónica:quê}
\end{itemize}
\begin{itemize}
\item {Grp. gram.:f.}
\end{itemize}
\begin{itemize}
\item {Proveniência:(Do fr. \textunderscore piquet\textunderscore )}
\end{itemize}
Cada uma das estacas, que se cravam no chão, para demarcar terreno.
\section{Piquetagem}
\begin{itemize}
\item {Grp. gram.:f.}
\end{itemize}
Acto de piquetar.
\section{Piquetar}
\begin{itemize}
\item {Grp. gram.:v. t.}
\end{itemize}
\begin{itemize}
\item {Proveniência:(Do fr. \textunderscore piquet\textunderscore )}
\end{itemize}
Cravar estacas em (um terreno), ao estudar-se um traçado de estrada.
\section{Piquete}
\begin{itemize}
\item {fónica:quê}
\end{itemize}
\begin{itemize}
\item {Grp. gram.:m.}
\end{itemize}
\begin{itemize}
\item {Proveniência:(Fr. \textunderscore piquet\textunderscore )}
\end{itemize}
Porção de soldados, que formavam guarda avançada.
Trôço de soldados a cavallo, encarregados de uma guarda de honra ou de outro serviço extraordinário.
Porção de empregados, a quem compete certo serviço por turno.
\section{Piquete}
\begin{itemize}
\item {fónica:quê}
\end{itemize}
\begin{itemize}
\item {Grp. gram.:m.}
\end{itemize}
\begin{itemize}
\item {Utilização:Bras}
\end{itemize}
O mesmo que \textunderscore potreiro\textunderscore .
\section{Piqui}
\begin{itemize}
\item {Grp. gram.:m.}
\end{itemize}
Planta sapindácea do Brasil, o mesmo que \textunderscore piquizeiro\textunderscore .
Fruto de piquizeiro.
\section{Piquiá}
\begin{itemize}
\item {Grp. gram.:m.}
\end{itemize}
Planta agreste do Brasil.
Espécie de nassa ou cesto para apanhar peixes.
(Talvez outra fórma de \textunderscore picoá\textunderscore )
\section{Piquira}
\begin{itemize}
\item {Grp. gram.:m.}
\end{itemize}
\begin{itemize}
\item {Utilização:Bras. do Rio}
\end{itemize}
Cavallo de raça pequena.
Peixe pequeno do Paraguai.
\section{Piquizeiro}
\begin{itemize}
\item {Grp. gram.:m.}
\end{itemize}
\begin{itemize}
\item {Utilização:Bras}
\end{itemize}
Grande árvore sapindácea, de excellente madeira para construcção naval.
\section{Pira}
\begin{itemize}
\item {Grp. gram.:f.}
\end{itemize}
\begin{itemize}
\item {Utilização:Bras}
\end{itemize}
\begin{itemize}
\item {Proveniência:(T. tupi)}
\end{itemize}
Doença de pelle nos animaes.
Sarna; gafeira.
\section{Pirá}
\begin{itemize}
\item {Grp. gram.:m.}
\end{itemize}
\begin{itemize}
\item {Utilização:Bras}
\end{itemize}
\begin{itemize}
\item {Proveniência:(T. tupi)}
\end{itemize}
Designação genérica de peixe.
\section{Pirá-andirá}
\begin{itemize}
\item {Grp. gram.:m.}
\end{itemize}
Espécie de peixe do Amazonas.
\section{Pirabebe}
\begin{itemize}
\item {Grp. gram.:m.}
\end{itemize}
Peixe voador do Brasil.
\section{Piracá}
\begin{itemize}
\item {Grp. gram.:m.}
\end{itemize}
Árvore das regiões do Amazonas.
\section{Piracatinga}
\begin{itemize}
\item {Grp. gram.:f.}
\end{itemize}
Peixe do Amazonas.
\section{Piracaúba}
\begin{itemize}
\item {Grp. gram.:f.}
\end{itemize}
O mesmo que \textunderscore cujumari\textunderscore .
\section{Piracema}
\begin{itemize}
\item {Grp. gram.:f.}
\end{itemize}
\begin{itemize}
\item {Utilização:Bras}
\end{itemize}
Estação, em que se manifesta a arribação do peixe fluvial em grandes cardumes.
(Do tupi \textunderscore pirá\textunderscore  + \textunderscore acen\textunderscore )
\section{Piracuí}
\begin{itemize}
\item {Grp. gram.:m.}
\end{itemize}
\begin{itemize}
\item {Utilização:Bras. do N}
\end{itemize}
\begin{itemize}
\item {Proveniência:(T. tupi)}
\end{itemize}
Iguaria, feita de peixe sêco, reduzido a pó.
\section{Piracuim}
\begin{itemize}
\item {Grp. gram.:m.}
\end{itemize}
O mesmo que \textunderscore piracuí\textunderscore .
\section{Pirafedes}
\begin{itemize}
\item {Grp. gram.:m.  e  f.}
\end{itemize}
\begin{itemize}
\item {Utilização:Prov.}
\end{itemize}
\begin{itemize}
\item {Utilização:trasm.}
\end{itemize}
Pessôa engraçada. (Colhido em V. P. de Aguiar)
\section{Piragaia}
\begin{itemize}
\item {Grp. gram.:f.}
\end{itemize}
O mesmo que \textunderscore cipó-sumá\textunderscore .
\section{Piraí}
\begin{itemize}
\item {Grp. gram.:m.}
\end{itemize}
\begin{itemize}
\item {Utilização:Bras}
\end{itemize}
\begin{itemize}
\item {Proveniência:(De \textunderscore pira\textunderscore )}
\end{itemize}
Azorrague de coiro cru.
\section{Piraiba}
\begin{itemize}
\item {Grp. gram.:m.}
\end{itemize}
\begin{itemize}
\item {Utilização:Bras}
\end{itemize}
Peixe do Amazonas.
\section{Pirá-inambu}
\begin{itemize}
\item {Grp. gram.:m.}
\end{itemize}
O mesmo que \textunderscore piranambu\textunderscore .
\section{Pirajá}
\begin{itemize}
\item {Grp. gram.:m.}
\end{itemize}
\begin{itemize}
\item {Utilização:Bras}
\end{itemize}
Aguaceiro, acompanhado de vento.
\section{Piramutá}
\begin{itemize}
\item {Grp. gram.:m.}
\end{itemize}
\begin{itemize}
\item {Utilização:Bras}
\end{itemize}
Peixe do Amazonas e de alguns dos seus affluentes.
\section{Piranambu}
\begin{itemize}
\item {Grp. gram.:m.}
\end{itemize}
\begin{itemize}
\item {Utilização:Bras}
\end{itemize}
Peixe do rio Purus.
\section{Piranduba}
\begin{itemize}
\item {Grp. gram.:f.}
\end{itemize}
\begin{itemize}
\item {Utilização:Bras}
\end{itemize}
Árvore silvestre, empregada em carpintaria.
\section{Piranga}
\begin{itemize}
\item {Grp. gram.:f.}
\end{itemize}
\begin{itemize}
\item {Utilização:Pop.}
\end{itemize}
\begin{itemize}
\item {Grp. gram.:Adj.}
\end{itemize}
\begin{itemize}
\item {Utilização:Pop.}
\end{itemize}
Barro vermelho do Brasil.
Peixe fluvial do Brasil.
Falta de dinheiro, penúria.
Reles, pelintra.
\section{Pirangar}
\begin{itemize}
\item {Grp. gram.:v. i.}
\end{itemize}
\begin{itemize}
\item {Utilização:Pop.}
\end{itemize}
\begin{itemize}
\item {Proveniência:(De \textunderscore piranga\textunderscore )}
\end{itemize}
O mesmo que \textunderscore mendigar\textunderscore . Cf. Camillo, \textunderscore Corja\textunderscore ; \textunderscore Eusébio\textunderscore , 199.
\section{Pirangaria}
\begin{itemize}
\item {Grp. gram.:f.}
\end{itemize}
Falta de dinheiro; pelintrice. Cf. Castilho, \textunderscore Avarento\textunderscore , 303.
\section{Pirange}
\begin{itemize}
\item {Grp. gram.:m.}
\end{itemize}
Carro de seis rodas, na Índia.
\section{Pirangueiro}
\begin{itemize}
\item {Grp. gram.:adj.}
\end{itemize}
\begin{itemize}
\item {Utilização:Des.}
\end{itemize}
\begin{itemize}
\item {Proveniência:(De \textunderscore piranga\textunderscore )}
\end{itemize}
Reles, ridículo, desprezível.--Vejo no \textunderscore Anatómico Joc.\textunderscore , 319, uma«\textunderscore piranqueira derrota de pechelingues devotos...\textunderscore »mas, se neste passo não há êrro de imprensa, creio que o autor usou \textunderscore piranqueiro\textunderscore  por \textunderscore pirangueiro\textunderscore .
\section{Piranha}
\begin{itemize}
\item {Grp. gram.:f.}
\end{itemize}
Peixe do Tocantins, de dentes anavalhados e mordedura perigosa.
\section{Piranha}
\begin{itemize}
\item {Grp. gram.:f.}
\end{itemize}
\begin{itemize}
\item {Utilização:Bras}
\end{itemize}
Ave preta do Amazonas, de cauda bipartida.
\section{Piranhaúba}
\begin{itemize}
\item {Grp. gram.:f.}
\end{itemize}
Árvore do Brasil, cuja madeira se emprega em marcenaria.
\section{Piranheira}
\begin{itemize}
\item {Grp. gram.:f.}
\end{itemize}
\begin{itemize}
\item {Utilização:Bras}
\end{itemize}
Árvore amazónica, de que se fazem canôas.
\section{Pirão}
\begin{itemize}
\item {Grp. gram.:m.}
\end{itemize}
\begin{itemize}
\item {Utilização:Bras}
\end{itemize}
Papas de mandioca.
\section{Pirapema}
\begin{itemize}
\item {Grp. gram.:f.}
\end{itemize}
\begin{itemize}
\item {Utilização:Bras}
\end{itemize}
Peixe do Amazonas.
\section{Pirapitinga}
\begin{itemize}
\item {Grp. gram.:f.}
\end{itemize}
Peixe do Amazonas.
\section{Pirapucu}
\begin{itemize}
\item {Grp. gram.:m.}
\end{itemize}
Peixe do Norte do Brasil.
\section{Piraquara}
\begin{itemize}
\item {Grp. gram.:m.  e  f.}
\end{itemize}
\begin{itemize}
\item {Utilização:Bras}
\end{itemize}
\begin{itemize}
\item {Proveniência:(T. guar.)}
\end{itemize}
Pessôa, que habita nas margens do Paraíba do Sul e que se dedica á pesca.
\section{Piraqué}
\begin{itemize}
\item {Grp. gram.:m.}
\end{itemize}
\begin{itemize}
\item {Utilização:Bras}
\end{itemize}
Variedade de peixe eléctrico, o mesmo que \textunderscore puraqué\textunderscore .
\section{Piraquém}
\begin{itemize}
\item {Grp. gram.:m.}
\end{itemize}
\begin{itemize}
\item {Utilização:Bras}
\end{itemize}
Espécie de côco.
\section{Piraquera}
\begin{itemize}
\item {Grp. gram.:f.}
\end{itemize}
\begin{itemize}
\item {Utilização:Bras. do N}
\end{itemize}
Pesca nocturna, em que se arpôa o peixe, á luz de fachos.
(Do tupi \textunderscore pirá\textunderscore  + \textunderscore her\textunderscore )
\section{Pirarara}
\begin{itemize}
\item {Grp. gram.:m.}
\end{itemize}
\begin{itemize}
\item {Utilização:Bras}
\end{itemize}
Grande peixe do rio Purus.
\section{Pirar-se}
\begin{itemize}
\item {Grp. gram.:v. p.}
\end{itemize}
\begin{itemize}
\item {Utilização:Chul.}
\end{itemize}
Safar-se; fugir.
(Do caló \textunderscore pirar\textunderscore , andar)
\section{Pirarucu}
\begin{itemize}
\item {Grp. gram.:m.}
\end{itemize}
Peixe do norte do Brasil, muito apreciado e de grandes dimensões, semelhante ao bacalhau.
\section{Pirata}
\begin{itemize}
\item {Grp. gram.:m.}
\end{itemize}
\begin{itemize}
\item {Utilização:Ext.}
\end{itemize}
\begin{itemize}
\item {Utilização:Gír.}
\end{itemize}
\begin{itemize}
\item {Utilização:Prov.}
\end{itemize}
\begin{itemize}
\item {Utilização:trasm.}
\end{itemize}
\begin{itemize}
\item {Proveniência:(Lat. \textunderscore pirata\textunderscore )}
\end{itemize}
Salteador, que cruza os mares, só para roubar.
Ladrão marítimo.
Navio de pirata.
Ladrão.
Cabo de polícia.
Malandro.
\section{Piratagem}
\begin{itemize}
\item {Grp. gram.:f.}
\end{itemize}
Roubo, feito por piratas; acto de piratear.
\section{Piratapioca}
\begin{itemize}
\item {Grp. gram.:f.}
\end{itemize}
Peixe do Brasil.
\section{Pirataria}
\begin{itemize}
\item {Grp. gram.:f.}
\end{itemize}
\begin{itemize}
\item {Utilização:Ext.}
\end{itemize}
\begin{itemize}
\item {Utilização:Fig.}
\end{itemize}
Vida de pirata; acto de pirata.
Roubo, extorsão.
Patifaria.
\section{Piratatu}
\begin{itemize}
\item {Grp. gram.:m.}
\end{itemize}
\begin{itemize}
\item {Utilização:Bras. do N}
\end{itemize}
Peixe, cuja cabeça se parece com a do tatu.
\section{Piratear}
\begin{itemize}
\item {Grp. gram.:v. i.}
\end{itemize}
Têr vida de pirata; extorquír como os piratas.
\section{Piratear}
\begin{itemize}
\item {Grp. gram.:v. t.}
\end{itemize}
Roubar como pirata. Cf. Camillo, \textunderscore Paço de Ninães\textunderscore , 138.
\section{Pirático}
\begin{itemize}
\item {Grp. gram.:adj.}
\end{itemize}
Relativo a pirata.
\section{Piratinga}
\begin{itemize}
\item {Grp. gram.:f.}
\end{itemize}
\begin{itemize}
\item {Utilização:Bras}
\end{itemize}
Grande peixe do Tocantins.
\section{Piratiningano}
\begin{itemize}
\item {Grp. gram.:m.}
\end{itemize}
\begin{itemize}
\item {Utilização:Bras}
\end{itemize}
\begin{itemize}
\item {Utilização:Ant.}
\end{itemize}
\begin{itemize}
\item {Proveniência:(De \textunderscore Piratininga\textunderscore , n. p.)}
\end{itemize}
Habitante da cidade de San-Paulo.
\section{Piraupeua}
\begin{itemize}
\item {Grp. gram.:f.}
\end{itemize}
\begin{itemize}
\item {Utilização:Bras}
\end{itemize}
O mesmo que \textunderscore pirapema\textunderscore .
\section{Pire}
\begin{itemize}
\item {Grp. gram.:m.}
\end{itemize}
\begin{itemize}
\item {Utilização:Gír.}
\end{itemize}
Prato.
(Cp. \textunderscore pires\textunderscore )
\section{Pirenga}
\begin{itemize}
\item {Grp. gram.:f.}
\end{itemize}
O mesmo que \textunderscore carajuru\textunderscore .
\section{Pirento}
\begin{itemize}
\item {Grp. gram.:adj.}
\end{itemize}
\begin{itemize}
\item {Utilização:Bras}
\end{itemize}
Que soffre pira.
\section{Pires}
\begin{itemize}
\item {Grp. gram.:m.}
\end{itemize}
\begin{itemize}
\item {Proveniência:(De or. indiana)}
\end{itemize}
Pequeno prato, em que se colloca uma chávena ou outro pequeno vaso, de natureza igual á delle.
Pequeno prato.
\section{Pireza}
\begin{itemize}
\item {Grp. gram.:f.}
\end{itemize}
\begin{itemize}
\item {Utilização:Chul.}
\end{itemize}
Acto de pirar-se. Cf. Camillo, \textunderscore Myst. de Lisb.\textunderscore , II, 10.
\section{Piri}
\begin{itemize}
\item {Grp. gram.:m.}
\end{itemize}
\begin{itemize}
\item {Utilização:Bras. do N}
\end{itemize}
\begin{itemize}
\item {Proveniência:(T. tupi)}
\end{itemize}
Espécie de junco, que cresce nos terrenos pantanosos.
Brejo, em que se desenvolve essa planta.
\section{Piria}
\begin{itemize}
\item {Grp. gram.:f.}
\end{itemize}
Pássaro dentirostro africano.
\section{Piriá}
\begin{itemize}
\item {Grp. gram.:m.}
\end{itemize}
\begin{itemize}
\item {Utilização:Bras. do Rio}
\end{itemize}
Espécie de rata de água.
\section{Piriantan}
\begin{itemize}
\item {Grp. gram.:m.}
\end{itemize}
O mesmo que \textunderscore periantan\textunderscore .
\section{Piricão}
\begin{itemize}
\item {Grp. gram.:m.}
\end{itemize}
Planta da serra de Sintra.
O mesmo que \textunderscore hypericão\textunderscore ? Cf. Juromenha, \textunderscore Cintra Pint.\textunderscore , 191.
\section{Piriche}
\begin{itemize}
\item {Grp. gram.:m.}
\end{itemize}
Pequena embarcação de guerra, na Índia.
\section{Piriforme}
\begin{itemize}
\item {Grp. gram.:adj.}
\end{itemize}
\begin{itemize}
\item {Proveniência:(Do lat. \textunderscore pirum\textunderscore  + \textunderscore forma\textunderscore )}
\end{itemize}
Que tem fórma de pêra. Cf. \textunderscore Techn. Rur.\textunderscore , 50.
\section{Pirinambu}
\begin{itemize}
\item {Grp. gram.:m.}
\end{itemize}
Peixe brasileiro, (\textunderscore pimelodes pirinambu\textunderscore ).
\section{Pirinola}
\begin{itemize}
\item {Grp. gram.:f.}
\end{itemize}
O mesmo que \textunderscore rapa\textunderscore .
\section{Piripiri}
\begin{itemize}
\item {Grp. gram.:m.}
\end{itemize}
\begin{itemize}
\item {Utilização:Bras}
\end{itemize}
O mesmo que \textunderscore piri\textunderscore .
\section{Piriquitete}
\begin{itemize}
\item {fónica:tê}
\end{itemize}
\begin{itemize}
\item {Grp. gram.:adj.}
\end{itemize}
\begin{itemize}
\item {Utilização:Bras. do N}
\end{itemize}
Que traja sem ostentação, mas com cuidado e decência.
(Cp. \textunderscore perliquitete\textunderscore )
\section{Piriquiti}
\begin{itemize}
\item {Grp. gram.:m.}
\end{itemize}
Planta canácea do Brasil, (\textunderscore canna glauca\textunderscore ).
\section{Piriri}
\begin{itemize}
\item {Grp. gram.:m.}
\end{itemize}
\begin{itemize}
\item {Utilização:Bras. do Rio}
\end{itemize}
\begin{itemize}
\item {Proveniência:(T. tupi)}
\end{itemize}
Arbusto euphorbiáceo do Brasil.
O mesmo que \textunderscore diarreia\textunderscore .
\section{Piririca}
\begin{itemize}
\item {Grp. gram.:adj.}
\end{itemize}
\begin{itemize}
\item {Utilização:Bras}
\end{itemize}
Áspero como a lixa.
(Do tupi \textunderscore piriri\textunderscore )
\section{Piriricar}
\begin{itemize}
\item {Grp. gram.:v. i.}
\end{itemize}
\begin{itemize}
\item {Utilização:Bras}
\end{itemize}
Produzir um ligeiro estremecimento na água.
(Cp. \textunderscore pererecar\textunderscore )
\section{Piriz}
\begin{itemize}
\item {Grp. gram.:m.}
\end{itemize}
\begin{itemize}
\item {Utilização:Prov.}
\end{itemize}
\begin{itemize}
\item {Utilização:trasm.}
\end{itemize}
Espécie de pardal, o pardal dos rochedos ou pardal do monte, também chamado pardal francês, (\textunderscore passer petronia\textunderscore , Lin.)
\section{Pirliteiro}
\begin{itemize}
\item {Grp. gram.:m.}
\end{itemize}
O mesmo que \textunderscore pilrito\textunderscore  e \textunderscore pilriteiro\textunderscore . Cf. P. Coutinho, \textunderscore Flora\textunderscore , 289.
\section{Pirlito}
\begin{itemize}
\item {Grp. gram.:m.}
\end{itemize}
O mesmo que \textunderscore pilrito\textunderscore  e \textunderscore pilriteiro\textunderscore . Cf. P. Coutinho, \textunderscore Flora\textunderscore , 289.
\section{Piro}
\begin{itemize}
\item {Grp. gram.:m.}
\end{itemize}
\begin{itemize}
\item {Utilização:Gír.}
\end{itemize}
Acto de pirar-se.
\section{Piroca}
\begin{itemize}
\item {Grp. gram.:f.}
\end{itemize}
\begin{itemize}
\item {Utilização:Chul.}
\end{itemize}
\begin{itemize}
\item {Grp. gram.:Adj.}
\end{itemize}
\begin{itemize}
\item {Utilização:Bras}
\end{itemize}
\begin{itemize}
\item {Utilização:Bras. do N}
\end{itemize}
\begin{itemize}
\item {Proveniência:(T. tupi)}
\end{itemize}
O mesmo que \textunderscore pênis\textunderscore .
Calvo; pelado.
Sovina, avarento.
\section{Pirocar}
\begin{itemize}
\item {Grp. gram.:v. t.}
\end{itemize}
\begin{itemize}
\item {Utilização:Bras}
\end{itemize}
\begin{itemize}
\item {Proveniência:(De \textunderscore piroca\textunderscore )}
\end{itemize}
Esfolar; descascar.
\section{Piroga}
\begin{itemize}
\item {Grp. gram.:f.}
\end{itemize}
Barco estreito, comprido e veloz, usado por indígenas da América e da África.
\section{Pírola}
\begin{itemize}
\item {Grp. gram.:f.}
\end{itemize}
\begin{itemize}
\item {Utilização:Fig.}
\end{itemize}
Fórma pop. de \textunderscore pílula\textunderscore .«\textunderscore ...calar-se, engulir a pírola, estudar os clássicos...\textunderscore »Filinto, V, 289.
Pessôa manhosa ou de má índole: \textunderscore aquelle sujeito sempre me saíu uma pírola!...\textunderscore 
\section{Pirole}
\begin{itemize}
\item {Grp. gram.:m.}
\end{itemize}
Planta da serra de Sintra.
\section{Pirolé}
\begin{itemize}
\item {Grp. gram.:m.}
\end{itemize}
\begin{itemize}
\item {Utilização:Prov.}
\end{itemize}
O mesmo que \textunderscore alcaravão\textunderscore .
\section{Pirolito}
\begin{itemize}
\item {Grp. gram.:m.}
\end{itemize}
\begin{itemize}
\item {Utilização:Gír. de Lisbôa.}
\end{itemize}
Nome de um estribilho popular.
Espécie de bebida ordinária, gasosa, servida nos quiosques das praças e passeios públicos.
(Corr. de \textunderscore pirlito\textunderscore )
\section{Pirosca}
\begin{itemize}
\item {Grp. gram.:f.}
\end{itemize}
\begin{itemize}
\item {Utilização:Bras. de Minas}
\end{itemize}
Espécie de jôgo infantil.
\section{Pirraça}
\begin{itemize}
\item {Grp. gram.:f.}
\end{itemize}
Desfeita; partida; acinte.
(Por \textunderscore perraça\textunderscore , de \textunderscore perro\textunderscore )
\section{Pirracento}
\begin{itemize}
\item {Grp. gram.:adj.}
\end{itemize}
Que gosta de fazer pirraças.
\section{Pirralho}
\begin{itemize}
\item {Grp. gram.:m.}
\end{itemize}
\begin{itemize}
\item {Utilização:Prov.}
\end{itemize}
\begin{itemize}
\item {Utilização:Bras}
\end{itemize}
\begin{itemize}
\item {Utilização:beir.}
\end{itemize}
Criança; criançola.
Homem de pequena estatura.
\section{Pirraria}
\begin{itemize}
\item {Grp. gram.:f.}
\end{itemize}
\begin{itemize}
\item {Utilização:T. da Bairrada}
\end{itemize}
O mesmo que \textunderscore pirraça\textunderscore ; arrelia.
\section{Pirré}
\begin{itemize}
\item {Grp. gram.:m.}
\end{itemize}
\begin{itemize}
\item {Utilização:Prov.}
\end{itemize}
O mesmo que \textunderscore pirralho\textunderscore .
\section{Pirronice}
\begin{itemize}
\item {Grp. gram.:f.}
\end{itemize}
\begin{itemize}
\item {Utilização:Fam.}
\end{itemize}
Qualidade de pirrónico.
Desconfiança systemática; obstinação por pirraça; perrice.
\section{Pírtiga}
\begin{itemize}
\item {Grp. gram.:f.}
\end{itemize}
\begin{itemize}
\item {Proveniência:(Do lat. \textunderscore pertica\textunderscore )}
\end{itemize}
Vara.
Peça central do leito do carro, que se estende desde o recavém até á canga; cabeçalho.
\section{Pírtigo}
\begin{itemize}
\item {Grp. gram.:m.}
\end{itemize}
\begin{itemize}
\item {Proveniência:(De \textunderscore pírtiga\textunderscore )}
\end{itemize}
Vara do mangoal, a parte mais comprida do mangoal, á qual alguns também chamam \textunderscore mango\textunderscore . Cf. Viana, \textunderscore Apostilas\textunderscore , vb. \textunderscore mango\textunderscore .
\section{Piruêta}
\begin{itemize}
\item {Grp. gram.:f.}
\end{itemize}
Volta, dada pelo cavallo sôbre uma das mãos.
Giro sôbre um dos pés.
Pulo.
(Cast. \textunderscore pirueta\textunderscore )
\section{Piruetar}
\begin{itemize}
\item {Grp. gram.:v. i.}
\end{itemize}
Fazer piruêtas.
Girar sôbre um dos pés; cabriolar. Cf. Castilho, \textunderscore Fausto\textunderscore , 374.
\section{Pírula}
\begin{itemize}
\item {Grp. gram.:f.}
\end{itemize}
\begin{itemize}
\item {Utilização:Fig.}
\end{itemize}
O mesmo que \textunderscore pírola\textunderscore .
Fórma pop. de \textunderscore pílula\textunderscore .«\textunderscore ...calar-se, engulir a pírula, estudar os clássicos...\textunderscore »Filinto, V, 289.
Pessôa manhosa ou de má índole: \textunderscore aquelle sujeito sempre me saíu uma pírula!...\textunderscore 
\section{Piruleiro}
\begin{itemize}
\item {Grp. gram.:m.}
\end{itemize}
\begin{itemize}
\item {Utilização:Ant.}
\end{itemize}
\begin{itemize}
\item {Proveniência:(De \textunderscore pírula\textunderscore )}
\end{itemize}
Fabricante ou vendedor de pílulas.
\section{Pirum}
\begin{itemize}
\item {Grp. gram.:m.}
\end{itemize}
\begin{itemize}
\item {Utilização:Pop.}
\end{itemize}
O mesmo ou melhor que \textunderscore peru\textunderscore ^1.
\section{Pirunga}
\begin{itemize}
\item {Grp. gram.:f.}
\end{itemize}
(V.mapirunga)
\section{Piruruca}
\begin{itemize}
\item {Grp. gram.:f.}
\end{itemize}
\begin{itemize}
\item {Utilização:Bras}
\end{itemize}
Espécie de saibro grosso, misturado com pedra miúda.
\section{Pisa}
\begin{itemize}
\item {Grp. gram.:f.}
\end{itemize}
Acto de pisar.
Maceração das uvas com os pés, no lagar.
Sova, tunda.
\section{Pisada}
\begin{itemize}
\item {Grp. gram.:f.}
\end{itemize}
\begin{itemize}
\item {Utilização:Prov.}
\end{itemize}
\begin{itemize}
\item {Utilização:trasm.}
\end{itemize}
\begin{itemize}
\item {Proveniência:(De \textunderscore pisar\textunderscore )}
\end{itemize}
Pègada.
Pisadela.
Pisa de uvas.
\section{Pisadela}
\begin{itemize}
\item {Grp. gram.:f.}
\end{itemize}
Acto ou effeito de pisar.
\section{Pisador}
\begin{itemize}
\item {Grp. gram.:m.  e  adj.}
\end{itemize}
\begin{itemize}
\item {Grp. gram.:M.}
\end{itemize}
O que pisa.
Apparelho moderno, para pisar uvas.
\section{Pisadura}
\begin{itemize}
\item {Grp. gram.:f.}
\end{itemize}
\begin{itemize}
\item {Proveniência:(De \textunderscore pisar\textunderscore )}
\end{itemize}
Vestígio de pisada; contusão.
Atropelamento.
\section{Pisa-flôres}
\begin{itemize}
\item {Grp. gram.:m.}
\end{itemize}
\begin{itemize}
\item {Utilização:Pop.}
\end{itemize}
Homem affectado no andar, presumido, adamado.
Salta-pocinhas; pisa-verdes.
\section{Pisa-mansinho}
\begin{itemize}
\item {Grp. gram.:m.  e  adj.}
\end{itemize}
O que é sonso, que tem ronha.
\section{Pisamento}
\begin{itemize}
\item {Grp. gram.:m.}
\end{itemize}
O mesmo que \textunderscore pisadela\textunderscore .
\section{Pisano}
\begin{itemize}
\item {Grp. gram.:adj.}
\end{itemize}
Relativo á cidade de Pisa. Cf. Herculano, \textunderscore Hist. de Port.\textunderscore , I, 379.
\section{Pisão}
\begin{itemize}
\item {Grp. gram.:m.}
\end{itemize}
\begin{itemize}
\item {Proveniência:(Do b. lat. \textunderscore piso\textunderscore , \textunderscore pisonis\textunderscore )}
\end{itemize}
Máquina, com que se aperta o pano, para o tornar mais consistente e tapado.
\section{Pisão}
\begin{itemize}
\item {Grp. gram.:m.}
\end{itemize}
O mesmo que \textunderscore pinçote\textunderscore . Cf. \textunderscore Hist. Trág. Marit.\textunderscore , 324.
(Corr. de \textunderscore pinção\textunderscore )
\section{Pisar}
\begin{itemize}
\item {Grp. gram.:v. t.}
\end{itemize}
\begin{itemize}
\item {Proveniência:(Do lat. \textunderscore pinsare\textunderscore )}
\end{itemize}
Pôr o pé sôbre: \textunderscore pisar uma casca de laranja\textunderscore .
Bater com o pé ou com os pés.
Calcar.
Magoar; contundir.
Esmagar.
Atropelar.
Trilhar; moer com o pilão.
Amassar.
Macerar.
Subjugar.
Achatar.
Esterroar.
Offender.
Percorrer.
Andar sôbre, andar por.
Atravessar.
\section{Pisa-verdes}
\begin{itemize}
\item {Grp. gram.:m.}
\end{itemize}
\begin{itemize}
\item {Utilização:Prov.}
\end{itemize}
\begin{itemize}
\item {Proveniência:(De \textunderscore pisar\textunderscore  + \textunderscore verde\textunderscore )}
\end{itemize}
Janota presumido, de passinhos miúdos e affectados. Cf. Camillo, \textunderscore Quéda\textunderscore , 57.
\section{Pisca}
\begin{itemize}
\item {Grp. gram.:f.}
\end{itemize}
\begin{itemize}
\item {Utilização:Prov.}
\end{itemize}
\begin{itemize}
\item {Utilização:trasm.}
\end{itemize}
Coisa muitíssimo pequena; pequeno grão.
Pó.
Fagulha.
Ponta de cigarro.
(Cast. \textunderscore pizca\textunderscore )
\section{Piscação}
\begin{itemize}
\item {Grp. gram.:f.}
\end{itemize}
\begin{itemize}
\item {Utilização:Des.}
\end{itemize}
O mesmo que \textunderscore piscadela\textunderscore . Cf. \textunderscore Anat. Joc.\textunderscore , 57.
\section{Piscadela}
\begin{itemize}
\item {Grp. gram.:f.}
\end{itemize}
Acto de piscar.
Sinal, que se faz piscando.
\section{Piscar}
\begin{itemize}
\item {Grp. gram.:v. t.}
\end{itemize}
\begin{itemize}
\item {Proveniência:(It. \textunderscore pizzicare\textunderscore )}
\end{itemize}
Abrir um pouco (os olhos).
Fechar e abrir rapidamente (os olhos).
Dar sinal a, piscando os olhos.
\section{Piscativo}
\begin{itemize}
\item {Grp. gram.:adj.}
\end{itemize}
\begin{itemize}
\item {Utilização:Des.}
\end{itemize}
O mesmo que \textunderscore piscatório\textunderscore .
Que se emprega a pescar:«\textunderscore aviso á piscativa gente.\textunderscore »Filinto, XIII, 176.
\section{Piscatória}
\begin{itemize}
\item {Grp. gram.:f.}
\end{itemize}
Idýlio, ou composição poética, em que falam, ou entram como personagens, pescadores ou homens do mar.
(Fem. de \textunderscore piscatório\textunderscore )
\section{Piscatório}
\begin{itemize}
\item {Grp. gram.:adj.}
\end{itemize}
\begin{itemize}
\item {Proveniência:(Lat. \textunderscore piscatorius\textunderscore )}
\end{itemize}
Relativo á pesca ou aos pescadores.
\section{Písceo}
\begin{itemize}
\item {Grp. gram.:adj.}
\end{itemize}
Relativo a peixe. Cf. Filinto, VIII, 165.
\section{Pisces}
\begin{itemize}
\item {Grp. gram.:m. pl.}
\end{itemize}
\begin{itemize}
\item {Proveniência:(Lat. \textunderscore pisces\textunderscore )}
\end{itemize}
Signo do Zodíaco, representado por dois peixes.
\section{Piscículo}
\begin{itemize}
\item {Grp. gram.:m.}
\end{itemize}
\begin{itemize}
\item {Utilização:P. us.}
\end{itemize}
\begin{itemize}
\item {Proveniência:(Lat. \textunderscore pisciculus\textunderscore )}
\end{itemize}
Pequeno peixe.
\section{Piscicultor}
\begin{itemize}
\item {Grp. gram.:m.}
\end{itemize}
Aquelle que se occupa de piscicultura.
\section{Piscicultura}
\begin{itemize}
\item {Grp. gram.:f.}
\end{itemize}
\begin{itemize}
\item {Proveniência:(Do lat. \textunderscore piscis\textunderscore  + \textunderscore cultura\textunderscore )}
\end{itemize}
Arte de criar e multiplicar os peixes.
\section{Piscídia-erythrina}
\begin{itemize}
\item {Grp. gram.:f.}
\end{itemize}
Substância pharmacêutica, empregada no tratamento do \textunderscore delirium tremens\textunderscore , da dysmenorrheia, etc.
\section{Pisciforme}
\begin{itemize}
\item {Grp. gram.:adj.}
\end{itemize}
\begin{itemize}
\item {Proveniência:(Do lat. \textunderscore piscis\textunderscore  + \textunderscore forma\textunderscore )}
\end{itemize}
Que tem fórma de peixe.
\section{Piscina}
\begin{itemize}
\item {Grp. gram.:f.}
\end{itemize}
\begin{itemize}
\item {Utilização:Ext.}
\end{itemize}
\begin{itemize}
\item {Utilização:Fig.}
\end{itemize}
\begin{itemize}
\item {Proveniência:(Lat. \textunderscore piscina\textunderscore )}
\end{itemize}
Reservatório de água, em que se costumava criar peixes.
Tanque, em que bebe o gado, ou tanque para lavagem de roupa.
Tanque ou tina fixa, em que se toma banho.
Lavatório, onde os frades iam lavar as mãos depois da comida.
Reservatório de água, em que, junto ao templo de Jerusalém, se lavavam os animaes destinados ao sacrifício.
Pia baptismal.
Tudo que purifica.
Sacramento da penitência.
\section{Piscinal}
\begin{itemize}
\item {Grp. gram.:adj.}
\end{itemize}
Que vive em piscina.
\section{Piscívoro}
\begin{itemize}
\item {Grp. gram.:adj.}
\end{itemize}
\begin{itemize}
\item {Proveniência:(Lat. \textunderscore píscivorus\textunderscore )}
\end{itemize}
Que se nutre de peixes.
\section{Pisco}
\begin{itemize}
\item {Grp. gram.:adj.}
\end{itemize}
\begin{itemize}
\item {Grp. gram.:M.}
\end{itemize}
\begin{itemize}
\item {Proveniência:(De \textunderscore piscar\textunderscore )}
\end{itemize}
Que pisca os olhos.
Entreaberto, (falando-se dos olhos).
Pássaro dentirostro, de peito ruivo, (\textunderscore rubecula familiaris\textunderscore , Blyth.).
\section{Pisco-chilreiro}
\begin{itemize}
\item {Grp. gram.:m.}
\end{itemize}
O mesmo que \textunderscore dom-fafe\textunderscore .
\section{Pisco-ferreiro}
\begin{itemize}
\item {Grp. gram.:m.}
\end{itemize}
Pássaro da fam. dos tordos, (\textunderscore ruticilla titys\textunderscore , Scop.).
\section{Píscola}
\begin{itemize}
\item {Grp. gram.:f.}
\end{itemize}
\begin{itemize}
\item {Proveniência:(Do lat. \textunderscore bis\textunderscore  + \textunderscore colere\textunderscore ?)}
\end{itemize}
Dois ou mais arados, que lavram juntos.
\section{Piscorência}
\begin{itemize}
\item {Grp. gram.:f.}
\end{itemize}
\begin{itemize}
\item {Utilização:Chul.}
\end{itemize}
O mesmo que \textunderscore pescorência\textunderscore .
\section{Pisco-ribeiro}
\begin{itemize}
\item {Grp. gram.:m.}
\end{itemize}
\begin{itemize}
\item {Utilização:Prov.}
\end{itemize}
O mesmo que \textunderscore pica-peixe\textunderscore , ave.
\section{Piscoso}
\begin{itemize}
\item {Grp. gram.:adj.}
\end{itemize}
\begin{itemize}
\item {Proveniência:(Lat. \textunderscore piscosus\textunderscore )}
\end{itemize}
Em que há muito peixe.
\section{Píseo}
\begin{itemize}
\item {Grp. gram.:m.}
\end{itemize}
\begin{itemize}
\item {Proveniência:(Lat. \textunderscore pisum\textunderscore )}
\end{itemize}
Ervilha grossa.
\section{Pisgar-se}
\begin{itemize}
\item {Grp. gram.:v. p.}
\end{itemize}
\begin{itemize}
\item {Utilização:Prov.}
\end{itemize}
\begin{itemize}
\item {Utilização:trasm.}
\end{itemize}
O mesmo que \textunderscore pirar-se\textunderscore :«\textunderscore ...que o fidalgo fizera bem em se pisgar com o doutor...\textunderscore »Camillo, \textunderscore Brasileira\textunderscore , 78.
\section{Pisiforme}
\begin{itemize}
\item {Grp. gram.:adj.}
\end{itemize}
\begin{itemize}
\item {Proveniência:(Do lat. \textunderscore pisum\textunderscore  + \textunderscore forma\textunderscore )}
\end{itemize}
Que tem o volume e a fórma de ervilha.
\section{Piso}
\begin{itemize}
\item {Grp. gram.:m.}
\end{itemize}
\begin{itemize}
\item {Proveniência:(De \textunderscore pisar\textunderscore )}
\end{itemize}
Modo de andar.
Terreno ou lugar, em que se anda.
Pavimento; chão.
Propína, que as freiras pagavam, ao entrar no convento.
A face superior dos degraus.
\section{Pisoada}
\begin{itemize}
\item {Grp. gram.:f.}
\end{itemize}
\begin{itemize}
\item {Proveniência:(De \textunderscore pisoar\textunderscore )}
\end{itemize}
Porção de lan ou de outra substância, que se pisou de uma vez.
\section{Pisoador}
\begin{itemize}
\item {Grp. gram.:m.}
\end{itemize}
Aquelle ou aquillo que pisôa.
\section{Pisoagem}
\begin{itemize}
\item {Grp. gram.:f.}
\end{itemize}
O mesmo que \textunderscore pisoamento\textunderscore .
\section{Pisoamento}
\begin{itemize}
\item {Grp. gram.:m.}
\end{itemize}
Acto de pisoar.
\section{Pisoar}
\begin{itemize}
\item {Grp. gram.:v. t.}
\end{itemize}
Bater com o pisão (o pano).
\section{Pisoeiro}
\begin{itemize}
\item {Grp. gram.:m.}
\end{itemize}
O mesmo que \textunderscore pisador\textunderscore .
\section{Pisólita}
\begin{itemize}
\item {Grp. gram.:f.}
\end{itemize}
O mesmo que \textunderscore pisólito\textunderscore .
\section{Pisólitha}
\begin{itemize}
\item {Grp. gram.:f.}
\end{itemize}
O mesmo que \textunderscore pisólitho\textunderscore .
\section{Pisolíthico}
\begin{itemize}
\item {Grp. gram.:adj.}
\end{itemize}
\begin{itemize}
\item {Utilização:Miner.}
\end{itemize}
Diz-se da textura das rochas, quando a massa rochosa se compõe de pisólithos.
\section{Pisólitho}
\begin{itemize}
\item {Grp. gram.:m.}
\end{itemize}
\begin{itemize}
\item {Utilização:Geol.}
\end{itemize}
\begin{itemize}
\item {Proveniência:(Do gr. \textunderscore pison\textunderscore , ervilha, e \textunderscore lithos\textunderscore , pedra)}
\end{itemize}
Concreção calcárea, da dimensão de uma ervilha.
\section{Pisolítico}
\begin{itemize}
\item {Grp. gram.:adj.}
\end{itemize}
\begin{itemize}
\item {Utilização:Miner.}
\end{itemize}
Diz-se da textura das rochas, quando a massa rochosa se compõe de pisólitos.
\section{Pisólito}
\begin{itemize}
\item {Grp. gram.:m.}
\end{itemize}
\begin{itemize}
\item {Utilização:Geol.}
\end{itemize}
\begin{itemize}
\item {Proveniência:(Do gr. \textunderscore pison\textunderscore , ervilha, e \textunderscore lithos\textunderscore , pedra)}
\end{itemize}
Concreção calcárea, da dimensão de uma ervilha.
\section{Pispirreta}
\begin{itemize}
\item {fónica:rê}
\end{itemize}
\begin{itemize}
\item {Grp. gram.:f.}
\end{itemize}
\begin{itemize}
\item {Utilização:Prov.}
\end{itemize}
\begin{itemize}
\item {Utilização:trasm.}
\end{itemize}
O mesmo que \textunderscore pilrete\textunderscore .
\section{Pisqueiro}
\begin{itemize}
\item {Grp. gram.:adj.}
\end{itemize}
Que pisca os olhos.
\section{Pisquila}
\begin{itemize}
\item {Grp. gram.:m.  e  f.}
\end{itemize}
\begin{itemize}
\item {Utilização:Bras}
\end{itemize}
Pessôa franzina e de pequena estatura.
\section{Pissandó}
\begin{itemize}
\item {Grp. gram.:m.}
\end{itemize}
\begin{itemize}
\item {Utilização:Bras}
\end{itemize}
Espécie de palmeira.
\section{Pissarra}
\textunderscore f.\textunderscore  (e der.)
(V. \textunderscore piçarra\textunderscore , etc., que é a bôa orthogr.)
\section{Pissasfáltico}
\begin{itemize}
\item {Grp. gram.:adj.}
\end{itemize}
Relativo ou semelhante ao pissasfalto.
\section{Pissasfalto}
\begin{itemize}
\item {Grp. gram.:m.}
\end{itemize}
\begin{itemize}
\item {Proveniência:(Lat. \textunderscore pissasphaltus\textunderscore )}
\end{itemize}
Espécie de betume, que, misturado com suco de cedro, servia para se embalsamarem os cadáveres, em Roma e noutros povos.
\section{Pissaspháltico}
\begin{itemize}
\item {Grp. gram.:adj.}
\end{itemize}
Relativo ou semelhante ao pissasphalto.
\section{Pissasphalto}
\begin{itemize}
\item {Grp. gram.:m.}
\end{itemize}
\begin{itemize}
\item {Proveniência:(Lat. \textunderscore pissasphaltus\textunderscore )}
\end{itemize}
Espécie de betume, que, misturado com suco de cedro, servia para se embalsamarem os cadáveres, em Roma e noutros povos.
\section{Pisseleu}
\begin{itemize}
\item {Grp. gram.:m.}
\end{itemize}
\begin{itemize}
\item {Proveniência:(Gr. \textunderscore pisselaion\textunderscore )}
\end{itemize}
Substância líquida oleosa, que se separa do pez.
\section{Pissitar}
\begin{itemize}
\item {Grp. gram.:v. i.}
\end{itemize}
\begin{itemize}
\item {Proveniência:(Lat. \textunderscore pisitare\textunderscore )}
\end{itemize}
Gritar, (falando-se do estorninho). Cf. Castilho, \textunderscore Fastos\textunderscore , III, 324.
\section{Pissondora}
\begin{itemize}
\item {Grp. gram.:f.}
\end{itemize}
Árvore indiana, de fibras têxteis.
O mesmo que \textunderscore pissandó\textunderscore ?
\section{Pissota}
\begin{itemize}
\item {Grp. gram.:f.}
\end{itemize}
\begin{itemize}
\item {Utilização:Ant.}
\end{itemize}
O mesmo que \textunderscore pescada\textunderscore .
(Alter. de \textunderscore peixota\textunderscore ?)
\section{Pista}
\begin{itemize}
\item {Grp. gram.:f.}
\end{itemize}
\begin{itemize}
\item {Utilização:Ext.}
\end{itemize}
\begin{itemize}
\item {Proveniência:(Do lat. \textunderscore pistus\textunderscore )}
\end{itemize}
Rasto dos animaes, no terreno em que passaram.
Pègada.
Muro circular, dentro e junto do qual corre o cavallo.
Parte do hypódromo, em que correm os cavallos.
Encalço, procura: \textunderscore foi-lhe na pista\textunderscore .
\section{Pistacha}
\begin{itemize}
\item {Grp. gram.:f.}
\end{itemize}
\begin{itemize}
\item {Utilização:Gal}
\end{itemize}
\begin{itemize}
\item {Proveniência:(Do fr. \textunderscore pistache\textunderscore )}
\end{itemize}
O mesmo que \textunderscore pistácia\textunderscore .
\section{Pistacheiro}
\begin{itemize}
\item {Grp. gram.:m.}
\end{itemize}
\begin{itemize}
\item {Utilização:Gal}
\end{itemize}
\begin{itemize}
\item {Proveniência:(Fr. \textunderscore pistachier\textunderscore )}
\end{itemize}
O mesmo que \textunderscore pistácia\textunderscore . Cf. Dom. Vieira. \textunderscore Thes. da Ling. Port.\textunderscore , vb. \textunderscore lentisco\textunderscore .
\section{Pistácia}
\begin{itemize}
\item {Grp. gram.:f.}
\end{itemize}
\begin{itemize}
\item {Proveniência:(Lat. \textunderscore pistácia\textunderscore )}
\end{itemize}
Árvore terebinthácea, (\textunderscore pistácia vera\textunderscore ).
\section{Pistácio}
\begin{itemize}
\item {Grp. gram.:m.}
\end{itemize}
Semente, fruto, da pistácia.
\section{Pistacito}
\begin{itemize}
\item {Grp. gram.:m.}
\end{itemize}
\begin{itemize}
\item {Utilização:Miner.}
\end{itemize}
\begin{itemize}
\item {Proveniência:(Do gr. \textunderscore pistake\textunderscore )}
\end{itemize}
Variedade de epidoto.
\section{Pistão}
\begin{itemize}
\item {Grp. gram.:m.}
\end{itemize}
\begin{itemize}
\item {Utilização:Gal}
\end{itemize}
\begin{itemize}
\item {Proveniência:(Fr. \textunderscore piston\textunderscore )}
\end{itemize}
O mesmo que \textunderscore êmbolo\textunderscore .
\section{Pistilar}
\begin{itemize}
\item {Grp. gram.:adj.}
\end{itemize}
\begin{itemize}
\item {Utilização:Bot.}
\end{itemize}
Relativo ao pistilo.
Diz-se das flôres, em que a mudança é devida á degeneração petalóide das partes do pistilo.
\section{Pistillar}
\begin{itemize}
\item {Grp. gram.:adj.}
\end{itemize}
\begin{itemize}
\item {Utilização:Bot.}
\end{itemize}
Relativo ao pistillo.
Diz-se das flôres, em que a mudança é devida á degeneração petalóide das partes do pistillo.
\section{Pistillo}
\begin{itemize}
\item {Grp. gram.:m.}
\end{itemize}
\begin{itemize}
\item {Proveniência:(Lat. \textunderscore pistillus\textunderscore )}
\end{itemize}
Órgão sexual feminino dos vegetaes phanerogâmicos.
\section{Pistilloso}
\begin{itemize}
\item {Grp. gram.:adj.}
\end{itemize}
Que tem pistillo.
\section{Pistilo}
\begin{itemize}
\item {Grp. gram.:m.}
\end{itemize}
\begin{itemize}
\item {Proveniência:(Lat. \textunderscore pistillus\textunderscore )}
\end{itemize}
Órgão sexual feminino dos vegetaes fanerogâmicos.
\section{Pistiloso}
\begin{itemize}
\item {Grp. gram.:adj.}
\end{itemize}
Que tem pistilo.
\section{Pistola}
\begin{itemize}
\item {Grp. gram.:f.}
\end{itemize}
Pequena arma de fogo, do feitio da espingarda, mas que se segura e se dispara com uma só mão.
Peça de fogo de artifício, que, em fórma de canudo, dispara glóbulos luminosos.
Moéda de oiro espanhola.
Antiga moéda francesa:«\textunderscore ...as pistolas francesas...\textunderscore »F. Manuel, \textunderscore Apólogos\textunderscore .
(Talvez de \textunderscore Pistoia\textunderscore , n. p.)
\section{Pistolada}
\begin{itemize}
\item {Grp. gram.:f.}
\end{itemize}
Tiro de pistola.
\section{Pistoleiro}
\begin{itemize}
\item {Grp. gram.:m.}
\end{itemize}
Soldado francês, que, no século XVI, usava pistola na cavallaria.
\section{Pistoleta}
\begin{itemize}
\item {fónica:lê}
\end{itemize}
\begin{itemize}
\item {Grp. gram.:f.}
\end{itemize}
Espécie de jôgo de bisca, entre dois parceiros, cada um dos quaes joga com nove cartas.
\section{Pistoletas}
\begin{itemize}
\item {fónica:lê}
\end{itemize}
\begin{itemize}
\item {Grp. gram.:f. pl.}
\end{itemize}
Espécie de jôgo de bisca, entre dois parceiros, cada um dos quaes joga com nove cartas.
\section{Pistolete}
\begin{itemize}
\item {fónica:lê}
\end{itemize}
\begin{itemize}
\item {Grp. gram.:m.}
\end{itemize}
Pequena pistola.
Utensílio de mineiro, espécie de broca, que um trabalhador segura, em quanto outro a bate com a maçacopla.
\section{Pistolo}
\begin{itemize}
\item {fónica:tô}
\end{itemize}
\begin{itemize}
\item {Grp. gram.:m.}
\end{itemize}
\begin{itemize}
\item {Utilização:Prov.}
\end{itemize}
Cunha de ferro, para rachar pedra. (Colhido no Fundão)
\section{Pistolóchia}
\begin{itemize}
\item {fónica:qui}
\end{itemize}
\begin{itemize}
\item {Grp. gram.:f.}
\end{itemize}
Planta, da fam. das aristolochiáceas, (\textunderscore aristolochia pastolochia\textunderscore , Lin.).
\section{Pistolóquia}
\begin{itemize}
\item {Grp. gram.:f.}
\end{itemize}
Planta, da fam. das aristolochiáceas, (\textunderscore aristolochia pastolochia\textunderscore , Lin.).
\section{Pistor}
\begin{itemize}
\item {Grp. gram.:m.}
\end{itemize}
\begin{itemize}
\item {Utilização:Poét.}
\end{itemize}
\begin{itemize}
\item {Proveniência:(Lat. \textunderscore pistor\textunderscore )}
\end{itemize}
O mesmo que \textunderscore pàdeiro\textunderscore . Cf. Castilho, \textunderscore Fastos\textunderscore , III, 470.
\section{Pistrina}
\begin{itemize}
\item {Grp. gram.:f.}
\end{itemize}
\begin{itemize}
\item {Utilização:Poét.}
\end{itemize}
\begin{itemize}
\item {Proveniência:(Lat. \textunderscore pistrina\textunderscore )}
\end{itemize}
Lugar, onde se fabríca o pão; pàdaria. Cf. Castilho, \textunderscore Fastos\textunderscore , III, 469.
\section{Pita}
\begin{itemize}
\item {Grp. gram.:f.}
\end{itemize}
Fio ou fios, que se tiram da fôlha da piteira.
Trança, feita com fios dêsses.
Piteira.
\section{Pita}
\begin{itemize}
\item {Grp. gram.:f.}
\end{itemize}
\begin{itemize}
\item {Utilização:Prov.}
\end{itemize}
\begin{itemize}
\item {Utilização:beir.}
\end{itemize}
\begin{itemize}
\item {Grp. gram.:M.}
\end{itemize}
\begin{itemize}
\item {Utilização:Prov.}
\end{itemize}
\begin{itemize}
\item {Utilização:trasm.}
\end{itemize}
Fórma pop. de \textunderscore pinta\textunderscore ^2.
O mesmo que \textunderscore gallinha\textunderscore .
Homem maricas, effeminado.
\section{Pítaca}
\begin{itemize}
\item {Grp. gram.:f.}
\end{itemize}
Collecção de játacas. Cf. V. Abreu, \textunderscore Contos da Índia\textunderscore .
\section{Pita-cega}
\begin{itemize}
\item {Grp. gram.:f.}
\end{itemize}
(V.pinta-cega)
\section{Pitada}
\begin{itemize}
\item {Grp. gram.:f.}
\end{itemize}
\begin{itemize}
\item {Utilização:Restrict.}
\end{itemize}
\begin{itemize}
\item {Utilização:Ext.}
\end{itemize}
\begin{itemize}
\item {Utilização:Pop.}
\end{itemize}
\begin{itemize}
\item {Utilização:Gír.}
\end{itemize}
\begin{itemize}
\item {Utilização:Chul.}
\end{itemize}
\begin{itemize}
\item {Utilização:Prov.}
\end{itemize}
\begin{itemize}
\item {Utilização:trasm.}
\end{itemize}
Pequena porção de pó de qualquer substância, que se toma entre o dedo pollegar e o indicador.
Pequena porção de rapé, que se toma entre o dedo pollegar e o indicador e que se absorve pelo nariz.
Pequena porção de qualquer coisa.
Exhalação de mau cheiro.
Prostituta.
Acto de têr cópula carnal.
Bebedeira.
(Diz-se que vem de um rad. céltico, \textunderscore peto\textunderscore  = \textunderscore pito\textunderscore , designativo de \textunderscore pouco\textunderscore  ou \textunderscore pequeno\textunderscore . Entretanto, não poderá relacionar-se com o bras. \textunderscore pitar\textunderscore ?)
\section{Pitadear}
\begin{itemize}
\item {Grp. gram.:v. i.}
\end{itemize}
\begin{itemize}
\item {Utilização:Neol.}
\end{itemize}
\begin{itemize}
\item {Grp. gram.:V. t.}
\end{itemize}
Tomar pitadas de rapé.
Absorver pelo nariz.
Acompanhar com pitada: \textunderscore pitadear conselhos\textunderscore . Cf. Camillo, \textunderscore Corja\textunderscore , 246.
\section{Pitador}
\begin{itemize}
\item {Grp. gram.:m.}
\end{itemize}
\begin{itemize}
\item {Utilização:Bras. de Minas}
\end{itemize}
\begin{itemize}
\item {Proveniência:(De \textunderscore pitar\textunderscore )}
\end{itemize}
Fumador.
\section{Pitafe}
\begin{itemize}
\item {Grp. gram.:m.}
\end{itemize}
\begin{itemize}
\item {Utilização:Açor}
\end{itemize}
O mesmo que \textunderscore bitafe\textunderscore .
\section{Pitambor}
\begin{itemize}
\item {Grp. gram.:m.}
\end{itemize}
Vestuário precioso de bailadeira indiana. Cf. Th. Ribeiro, \textunderscore Jornadas\textunderscore , II, 105.
\section{Pitanan}
\begin{itemize}
\item {Grp. gram.:m.}
\end{itemize}
\begin{itemize}
\item {Utilização:Bras}
\end{itemize}
Nome de uma ave do Alto Amazonas.
\section{Pitança}
\begin{itemize}
\item {Grp. gram.:f.}
\end{itemize}
\begin{itemize}
\item {Utilização:Ext.}
\end{itemize}
Ração diária de alimentos.
Iguaria, que só se apresenta em dias de festa.
Pensão.
Esmola.
O comer.
(Cp. cast. \textunderscore pitanza\textunderscore )
\section{Pitanceiro}
\begin{itemize}
\item {Grp. gram.:m.}
\end{itemize}
\begin{itemize}
\item {Utilização:Ant.}
\end{itemize}
\begin{itemize}
\item {Proveniência:(De \textunderscore pitança\textunderscore )}
\end{itemize}
Distribuidor dos rendimentos de um convento.
\section{Pitanga}
\begin{itemize}
\item {Grp. gram.:f.}
\end{itemize}
Palavra expletiva, que entra em algumas locuções brasileiras, com a significação de \textunderscore menino\textunderscore .
(Do tupi)
\section{Pitanga}
\begin{itemize}
\item {Grp. gram.:f.}
\end{itemize}
Fruto da pitangueira; pitangueira.
(Do tupi)
\section{Pitangatuba}
\begin{itemize}
\item {Grp. gram.:f.}
\end{itemize}
Árvore fructífera do Brasil.
\section{Pitangueira}
\begin{itemize}
\item {Grp. gram.:f.}
\end{itemize}
Planta myrtácea, (\textunderscore eugenia uniflora\textunderscore ).
Nome, dado no Brasil, a outras plantas myrtáceas.
\section{Pitar}
\begin{itemize}
\item {Grp. gram.:v. i.}
\end{itemize}
\begin{itemize}
\item {Utilização:Bras}
\end{itemize}
Fumar em cachimbo.
Fumar cigarros ou charutos.
(Do tupi \textunderscore piter\textunderscore , chupar, sorver)
\section{Pitar}
\begin{itemize}
\item {Grp. gram.:v. t.}
\end{itemize}
\begin{itemize}
\item {Utilização:Prov.}
\end{itemize}
\begin{itemize}
\item {Utilização:trasm.}
\end{itemize}
\begin{itemize}
\item {Proveniência:(De \textunderscore pito\textunderscore ^3?)}
\end{itemize}
Traçar, fender.
Furar. (Colhido em Valpaços)
\section{Pitch-pine}
\begin{itemize}
\item {fónica:pitxpaine}
\end{itemize}
\begin{itemize}
\item {Grp. gram.:m.}
\end{itemize}
Estrangeirismo corrente, com que se designa uma espécie de pinheiro da América, (\textunderscore pinus rígida\textunderscore , Miller), cuja madeira rija é muito empregada em marcenaria e carpintaria.
\section{Pitecantropóide}
\begin{itemize}
\item {Grp. gram.:m.}
\end{itemize}
\begin{itemize}
\item {Proveniência:(Do gr. \textunderscore pitekos\textunderscore  + \textunderscore anthropos\textunderscore )}
\end{itemize}
Designação de um sêr, intermediário ao homem e ao macaco, e de que se acharam em Java alguns vestígios fósseis.
\section{Piteciano}
\begin{itemize}
\item {Grp. gram.:adj.}
\end{itemize}
\begin{itemize}
\item {Proveniência:(De \textunderscore pitheco\textunderscore )}
\end{itemize}
Relativo a macaco.
\section{Pitéco}
\begin{itemize}
\item {Grp. gram.:m.}
\end{itemize}
\begin{itemize}
\item {Proveniência:(Gr. \textunderscore pithekos\textunderscore )}
\end{itemize}
Espécie de macaco sem cauda.
\section{Pitecóide}
\begin{itemize}
\item {Grp. gram.:adj.}
\end{itemize}
\begin{itemize}
\item {Proveniência:(Do gr. \textunderscore pithekos\textunderscore  + \textunderscore eidos\textunderscore )}
\end{itemize}
Relativo ou semelhante ao piteco.
\section{Piteira}
\begin{itemize}
\item {Grp. gram.:f.}
\end{itemize}
\begin{itemize}
\item {Utilização:Gír.}
\end{itemize}
\begin{itemize}
\item {Utilização:Ext.}
\end{itemize}
\begin{itemize}
\item {Utilização:Prov.}
\end{itemize}
\begin{itemize}
\item {Utilização:trasm.}
\end{itemize}
Planta amaryllídea.
Aguardente de figos.
Bebedeira.
O mesmo que \textunderscore dívida\textunderscore  ou \textunderscore calote\textunderscore .
\section{Piteira}
\begin{itemize}
\item {Grp. gram.:f.}
\end{itemize}
\begin{itemize}
\item {Utilização:Bras}
\end{itemize}
\begin{itemize}
\item {Proveniência:(De \textunderscore pito\textunderscore ^2)}
\end{itemize}
O mesmo que \textunderscore boquilha\textunderscore .
\section{Piteireiro}
\begin{itemize}
\item {Grp. gram.:m.  e  adj.}
\end{itemize}
\begin{itemize}
\item {Utilização:Pop.}
\end{itemize}
\begin{itemize}
\item {Proveniência:(De \textunderscore piteira\textunderscore )}
\end{itemize}
O que costuma embriagar-se; bêbedo.
\section{Pitém}
\begin{itemize}
\item {Grp. gram.:m.}
\end{itemize}
\begin{itemize}
\item {Utilização:Carp.}
\end{itemize}
Pequena escavação ou entalhe numa viga, a fim de que o prego, que a há de segurar, entre mais fundo e a fixe melhor.
\section{Pitéo}
\begin{itemize}
\item {Grp. gram.:m.}
\end{itemize}
\begin{itemize}
\item {Utilização:Fam.}
\end{itemize}
Petisco; iguaria saborosa; gulodice.
(Cp. \textunderscore pitada\textunderscore )
\section{Pitéu}
\begin{itemize}
\item {Grp. gram.:m.}
\end{itemize}
\begin{itemize}
\item {Utilização:Fam.}
\end{itemize}
Petisco; iguaria saborosa; gulodice.
(Cp. \textunderscore pitada\textunderscore )
\section{Pithecanthropóide}
\begin{itemize}
\item {Grp. gram.:m.}
\end{itemize}
\begin{itemize}
\item {Proveniência:(Do gr. \textunderscore pitekos\textunderscore  + \textunderscore anthropos\textunderscore )}
\end{itemize}
Designação de um sêr, intermediário ao homem e ao macaco, e de que se acharam em Java alguns vestígios fósseis.
\section{Pitheciano}
\begin{itemize}
\item {Grp. gram.:adj.}
\end{itemize}
\begin{itemize}
\item {Proveniência:(De \textunderscore pitheco\textunderscore )}
\end{itemize}
Relativo a macaco.
\section{Pitheco}
\begin{itemize}
\item {Grp. gram.:m.}
\end{itemize}
\begin{itemize}
\item {Proveniência:(Gr. \textunderscore pithekos\textunderscore )}
\end{itemize}
Espécie de macaco sem cauda.
\section{Pithecóide}
\begin{itemize}
\item {Grp. gram.:adj.}
\end{itemize}
\begin{itemize}
\item {Proveniência:(Do gr. \textunderscore pithekos\textunderscore  + \textunderscore eidos\textunderscore )}
\end{itemize}
Relativo ou semelhante ao pitheco.
\section{Pithiatismo}
\begin{itemize}
\item {Grp. gram.:f.}
\end{itemize}
\begin{itemize}
\item {Utilização:Med.}
\end{itemize}
\begin{itemize}
\item {Proveniência:(Do gr. \textunderscore peitho\textunderscore  + \textunderscore iatos\textunderscore )}
\end{itemize}
Perturbação mental, curável pela persuasão.
\section{Pitiatismo}
\begin{itemize}
\item {Grp. gram.:f.}
\end{itemize}
\begin{itemize}
\item {Utilização:Med.}
\end{itemize}
\begin{itemize}
\item {Proveniência:(Do gr. \textunderscore peitho\textunderscore  + \textunderscore iatos\textunderscore )}
\end{itemize}
Perturbação mental, curável pela persuasão.
\section{Pítiga}
\begin{itemize}
\item {Grp. gram.:f.}
\end{itemize}
\begin{itemize}
\item {Utilização:Prov.}
\end{itemize}
\begin{itemize}
\item {Utilização:trasm.}
\end{itemize}
Vergôntea nova, rebento de árvore.
(Cp. \textunderscore pírtiga\textunderscore )
\section{Pitigares}
\begin{itemize}
\item {Grp. gram.:m. pl.}
\end{itemize}
(V.potigares)
\section{Pítigo}
\begin{itemize}
\item {Grp. gram.:m.}
\end{itemize}
\begin{itemize}
\item {Utilização:Prov.}
\end{itemize}
\begin{itemize}
\item {Utilização:trasm.}
\end{itemize}
O mesmo que \textunderscore pírtigo\textunderscore .
\section{Pitimbóia}
\begin{itemize}
\item {Grp. gram.:f.}
\end{itemize}
\begin{itemize}
\item {Utilização:Bras}
\end{itemize}
Apparelho, para auxiliar a pesca dos camarões.
(Do tupi \textunderscore pitiboana\textunderscore )
\section{Pitinga}
\begin{itemize}
\item {Grp. gram.:adj.}
\end{itemize}
\begin{itemize}
\item {Utilização:Bras. do N}
\end{itemize}
Branco, claro.
(Do guarani)
\section{Pitinho}
\begin{itemize}
\item {Grp. gram.:m.}
\end{itemize}
\begin{itemize}
\item {Utilização:Prov.}
\end{itemize}
\begin{itemize}
\item {Utilização:trasm.}
\end{itemize}
\begin{itemize}
\item {Proveniência:(De \textunderscore pito\textunderscore ^1)}
\end{itemize}
O mesmo que \textunderscore mulatinho\textunderscore , passarinho implume.
\section{Pitiríase}
\begin{itemize}
\item {Grp. gram.:f.}
\end{itemize}
\begin{itemize}
\item {Proveniência:(Gr. \textunderscore pituriasis\textunderscore )}
\end{itemize}
Doença de pele, caracterizada por pequenas manchas escamosas e rosadas.
\section{Pitiriásico}
\begin{itemize}
\item {Grp. gram.:adj.}
\end{itemize}
Relativo á pitiríase.
\section{Pititinga}
\begin{itemize}
\item {Grp. gram.:f.}
\end{itemize}
\begin{itemize}
\item {Utilização:Bras}
\end{itemize}
Peixe miudinho.
(Cp. \textunderscore petinga\textunderscore ^1)
\section{Pitiú}
\begin{itemize}
\item {Grp. gram.:m.}
\end{itemize}
\begin{itemize}
\item {Utilização:Bras. do N}
\end{itemize}
Cheiro peculiar ao peixe; mau cheiro.
(Tupi \textunderscore pitiú\textunderscore )
\section{Pitium}
\begin{itemize}
\item {Grp. gram.:m.}
\end{itemize}
O mesmo que \textunderscore pitiú\textunderscore .
\section{Pitlo}
\begin{itemize}
\item {Grp. gram.:m.}
\end{itemize}
\begin{itemize}
\item {Utilização:T. de Turquel}
\end{itemize}
O mesmo que \textunderscore pírtigo\textunderscore .
\section{Pito}
\begin{itemize}
\item {Grp. gram.:m.}
\end{itemize}
\begin{itemize}
\item {Utilização:Prov.}
\end{itemize}
\begin{itemize}
\item {Utilização:minh.}
\end{itemize}
\begin{itemize}
\item {Utilização:Chul.}
\end{itemize}
\begin{itemize}
\item {Utilização:Pop.}
\end{itemize}
(Corr. pop. de \textunderscore pinto\textunderscore )
O mesmo que \textunderscore clitóris\textunderscore .
As partes pudendas da mulhér.
\section{Pito}
\begin{itemize}
\item {Grp. gram.:m.}
\end{itemize}
\begin{itemize}
\item {Utilização:Bras}
\end{itemize}
Cachimbo.
Cigarro.
\section{Pito}
\begin{itemize}
\item {Grp. gram.:m.}
\end{itemize}
\begin{itemize}
\item {Utilização:Prov.}
\end{itemize}
\begin{itemize}
\item {Utilização:trasm.}
\end{itemize}
\begin{itemize}
\item {Utilização:minh.}
\end{itemize}
O interior podre da fruta.
\section{Pito}
\begin{itemize}
\item {Grp. gram.:m.}
\end{itemize}
\begin{itemize}
\item {Utilização:Bras. do N}
\end{itemize}
Pequena reprehensão; censura.
\section{Pito}
\begin{itemize}
\item {Grp. gram.:m.}
\end{itemize}
\begin{itemize}
\item {Utilização:Ant.}
\end{itemize}
O mesmo que \textunderscore apito\textunderscore . Cf. Pant. de Aveiro, \textunderscore Itiner.\textunderscore , 30, (2.^a ed.).
\section{Pitoco}
\begin{itemize}
\item {fónica:tô}
\end{itemize}
\begin{itemize}
\item {Grp. gram.:m.}
\end{itemize}
\begin{itemize}
\item {Utilização:Bras. do N}
\end{itemize}
\begin{itemize}
\item {Proveniência:(De \textunderscore pito\textunderscore ^2)}
\end{itemize}
Pedaço de cachimbo; cachimbo quebrado.
\section{Pito-de-água}
\begin{itemize}
\item {Grp. gram.:m.}
\end{itemize}
\begin{itemize}
\item {Utilização:T. da Bairrada}
\end{itemize}
Espécie de ave aquática.
\section{Pitoma}
\begin{itemize}
\item {Grp. gram.:f.}
\end{itemize}
Fruta brasileira no território de Campos.
O mesmo que \textunderscore pitomba\textunderscore ?
\section{Pitomba}
\begin{itemize}
\item {Grp. gram.:f.}
\end{itemize}
Fruto da pitombeira; pitombeira.
\section{Pitombeira}
\begin{itemize}
\item {Grp. gram.:f.}
\end{itemize}
Árvore sapindácea e fructífera do Brasil.
\section{Pitombeiro}
\begin{itemize}
\item {Grp. gram.:m.}
\end{itemize}
\begin{itemize}
\item {Utilização:Bras}
\end{itemize}
O mesmo que \textunderscore pitombeira\textunderscore .
\section{Pitombo}
\begin{itemize}
\item {Grp. gram.:m.}
\end{itemize}
\begin{itemize}
\item {Utilização:Bras}
\end{itemize}
O mesmo que \textunderscore pitomba\textunderscore .
\section{Pitora}
\begin{itemize}
\item {Grp. gram.:f.}
\end{itemize}
Talhadas de lombo, guisadas com toicinho.
\section{Pitoresco}
\begin{itemize}
\item {fónica:torês}
\end{itemize}
\begin{itemize}
\item {Proveniência:(It. \textunderscore pittoresco\textunderscore )}
\end{itemize}
\textunderscore m.\textunderscore  e \textunderscore adj.\textunderscore  (e der.)
O mesmo que \textunderscore pinturesco\textunderscore , etc.
\section{Pitorra}
\begin{itemize}
\item {fónica:tô}
\end{itemize}
\begin{itemize}
\item {Grp. gram.:f.}
\end{itemize}
\begin{itemize}
\item {Grp. gram.:M.  e  f.}
\end{itemize}
\begin{itemize}
\item {Grp. gram.:M.}
\end{itemize}
Piorra.
Pessôa atarracada.
Gênero de molluscos.
(Cp. \textunderscore pitada\textunderscore )
\section{Pitosco}
\begin{itemize}
\item {Grp. gram.:adj.}
\end{itemize}
\begin{itemize}
\item {Utilização:Prov.}
\end{itemize}
\begin{itemize}
\item {Utilização:trasm.}
\end{itemize}
O mesmo que \textunderscore pitosga\textunderscore .
\section{Pitosga}
\begin{itemize}
\item {Grp. gram.:adj.}
\end{itemize}
\begin{itemize}
\item {Utilização:Pop.}
\end{itemize}
\begin{itemize}
\item {Grp. gram.:M.  e  f.}
\end{itemize}
Mýope; que vê pouco.
Que pisca os olhos.
Pessôa, que pisca habitualmente os olhos.
Pessôa remelosa.
\section{Pitosporáceas}
\begin{itemize}
\item {Grp. gram.:f. pl.}
\end{itemize}
Família de plantas dicotiledóneas, que tem por tipo o pitósporo.
\section{Pitospóreas}
\begin{itemize}
\item {Grp. gram.:f. pl.}
\end{itemize}
O mesmo que \textunderscore pitosporáceas\textunderscore .
\section{Pitósporo}
\begin{itemize}
\item {Grp. gram.:m.}
\end{itemize}
\begin{itemize}
\item {Proveniência:(Do gr. \textunderscore pitta\textunderscore , pez, e sporos, grão)}
\end{itemize}
Arbusto das regiões quentes do antigo continente, de fôlhas alternas e flôres solitárias ou diversamente agrupadas.
\section{Pitribi}
\begin{itemize}
\item {Grp. gram.:m.}
\end{itemize}
\begin{itemize}
\item {Utilização:Bras}
\end{itemize}
Árvore silvestre.
\section{Pittoresco}
\begin{itemize}
\item {fónica:torês}
\end{itemize}
\begin{itemize}
\item {Proveniência:(It. \textunderscore pittoresco\textunderscore )}
\end{itemize}
\textunderscore m.\textunderscore  e \textunderscore adj.\textunderscore  (e der.)
O mesmo que \textunderscore pinturesco\textunderscore , etc.
\section{Pittosporáceas}
\begin{itemize}
\item {Grp. gram.:f. pl.}
\end{itemize}
Família de plantas dicotyledóneas, que tem por typo o pittósporo.
\section{Pittospóreas}
\begin{itemize}
\item {Grp. gram.:f. pl.}
\end{itemize}
O mesmo que \textunderscore pittosporáceas\textunderscore .
\section{Pittósporo}
\begin{itemize}
\item {Grp. gram.:m.}
\end{itemize}
\begin{itemize}
\item {Proveniência:(Do gr. \textunderscore pitta\textunderscore , pez, e \textunderscore sporos\textunderscore , grão)}
\end{itemize}
Arbusto das regiões quentes do antigo continente, de fôlhas alternas e flôres solitárias ou diversamente agrupadas.
\section{Pitu}
\begin{itemize}
\item {Grp. gram.:m.}
\end{itemize}
\begin{itemize}
\item {Utilização:Bras}
\end{itemize}
Espécie de pífaro, usado entre os indígenas de San-Thomé.
Peixe fluvial.
\section{Pituá}
\begin{itemize}
\item {Grp. gram.:m.}
\end{itemize}
Pequeno pincel, feito de sedas finas e usado por doiradores.
\section{Pituba}
\begin{itemize}
\item {Grp. gram.:adj.}
\end{itemize}
\begin{itemize}
\item {Utilização:Bras}
\end{itemize}
\begin{itemize}
\item {Proveniência:(T. tupi)}
\end{itemize}
Diz-se da pessôa preguiçosa, fraca ou cobarde.
\section{Pituim}
\begin{itemize}
\item {Grp. gram.:m.}
\end{itemize}
\begin{itemize}
\item {Utilização:Bras. do N}
\end{itemize}
Bodum; suor mal cheiroso dos pretos.
\section{Pituíta}
\begin{itemize}
\item {Grp. gram.:f.}
\end{itemize}
\begin{itemize}
\item {Proveniência:(Lat. \textunderscore pituita\textunderscore )}
\end{itemize}
Humor branco e viscoso, segregado por vários órgãos do corpo, especialmente o que vem do nariz e dos brônchios.
Líquido aquoso, procedente de expectoração ou de vómito.
\section{Pituitária}
\begin{itemize}
\item {fónica:tu-i}
\end{itemize}
\begin{itemize}
\item {Grp. gram.:f.}
\end{itemize}
Membrana mucosa, que reveste as cavidades nasaes.
(Fem. de \textunderscore pituitário\textunderscore )
\section{Pituitário}
\begin{itemize}
\item {fónica:tu-i}
\end{itemize}
\begin{itemize}
\item {Grp. gram.:adj.}
\end{itemize}
\begin{itemize}
\item {Proveniência:(Lat. \textunderscore pituitarius\textunderscore )}
\end{itemize}
Relativo á pituíta ou que tem o carácter della.
Diz-se da glândula que existe na fossa do esphenoide.
\section{Pituitoso}
\begin{itemize}
\item {fónica:tu-i}
\end{itemize}
\begin{itemize}
\item {Grp. gram.:adj.}
\end{itemize}
\begin{itemize}
\item {Proveniência:(Lat. \textunderscore pituitosus\textunderscore )}
\end{itemize}
Cheio ou abundante de pituíta.
\section{Pitumarama}
\begin{itemize}
\item {Grp. gram.:f.}
\end{itemize}
Planta genciânea do Brasil.
\section{Pitura}
\begin{itemize}
\item {Grp. gram.:f.}
\end{itemize}
\begin{itemize}
\item {Utilização:Bras}
\end{itemize}
\begin{itemize}
\item {Proveniência:(De \textunderscore pito\textunderscore ^2)}
\end{itemize}
Tabaco.
\section{Pityríase}
\begin{itemize}
\item {Grp. gram.:f.}
\end{itemize}
\begin{itemize}
\item {Proveniência:(Gr. \textunderscore pituriasis\textunderscore )}
\end{itemize}
Doença de pelle, caracterizada por pequenas manchas escamosas e rosadas.
\section{Pityriásico}
\begin{itemize}
\item {Grp. gram.:adj.}
\end{itemize}
Relativo á pityríase.
\section{Piúca}
\begin{itemize}
\item {Grp. gram.:m.}
\end{itemize}
\begin{itemize}
\item {Utilização:Bras}
\end{itemize}
Pau sêco, que se esfarela, tornando-se muito combustível.
\section{Piúgo}
\begin{itemize}
\item {Grp. gram.:m.}
\end{itemize}
\begin{itemize}
\item {Utilização:Ant.}
\end{itemize}
Parede, feita de pedra miúda e solta.
\section{Pium}
\begin{itemize}
\item {Grp. gram.:m.}
\end{itemize}
\begin{itemize}
\item {Utilização:Bras. do N}
\end{itemize}
\begin{itemize}
\item {Proveniência:(T. tupi)}
\end{itemize}
Espécie de mosquito.
Insecto, muito incômmodo, da região do Purus.
\section{Piu! piu!}
\begin{itemize}
\item {Grp. gram.:interj.}
\end{itemize}
\begin{itemize}
\item {Utilização:Prov.}
\end{itemize}
\begin{itemize}
\item {Utilização:trasm.}
\end{itemize}
O mesmo que \textunderscore pila-pila!\textunderscore .
\section{Piúva}
\begin{itemize}
\item {Grp. gram.:f.}
\end{itemize}
\begin{itemize}
\item {Utilização:Bras}
\end{itemize}
O mesmo que \textunderscore ipé\textunderscore .
\section{Piverada}
\begin{itemize}
\item {Grp. gram.:f.}
\end{itemize}
\begin{itemize}
\item {Proveniência:(Do lat. \textunderscore piperatus\textunderscore )}
\end{itemize}
Guisado, em que entra pimenta, sal, azeite, vinagre e alhos:«\textunderscore são piveradas que a necessidade achou para seus fastios.\textunderscore »Aulegrafia, 39.
\section{Pivete}
\begin{itemize}
\item {fónica:vê}
\end{itemize}
\begin{itemize}
\item {Grp. gram.:m.}
\end{itemize}
\begin{itemize}
\item {Utilização:Deprec.}
\end{itemize}
\begin{itemize}
\item {Utilização:Prov.}
\end{itemize}
\begin{itemize}
\item {Utilização:trasm.}
\end{itemize}
\begin{itemize}
\item {Proveniência:(Do cast. \textunderscore pebete\textunderscore )}
\end{itemize}
Qualquer substância aromática, que se queira para perfumar.
Mau cheiro.
O mesmo que \textunderscore pegulho\textunderscore ^1.
\section{Piveteiro}
\begin{itemize}
\item {Grp. gram.:m.}
\end{itemize}
Vasilha, em que se colloca ou se queima o pivete.
\section{Pixaím}
\begin{itemize}
\item {Grp. gram.:adj.}
\end{itemize}
\begin{itemize}
\item {Utilização:Bras}
\end{itemize}
Que tem carapinha como os Negros.
(Do tupi \textunderscore iapixaim\textunderscore )
\section{Pixé}
\begin{itemize}
\item {Grp. gram.:adj.}
\end{itemize}
\begin{itemize}
\item {Utilização:Bras}
\end{itemize}
\begin{itemize}
\item {Grp. gram.:M.}
\end{itemize}
\begin{itemize}
\item {Utilização:Bras. do N}
\end{itemize}
\begin{itemize}
\item {Proveniência:(T. tupi.)}
\end{itemize}
Diz-se da comida que tem fumo.
Mau cheiro.
\section{Pixeu}
\begin{itemize}
\item {Grp. gram.:m.}
\end{itemize}
\begin{itemize}
\item {Utilização:Bras. da Baía}
\end{itemize}
Partes pudendas da mulhér.
\section{Pixídio}
\begin{itemize}
\item {Grp. gram.:m.}
\end{itemize}
Gênero de frutos sêcos e uniloculares, como na sapucaia. Cf. Caminhoá, \textunderscore Bot. Ger. e Med.\textunderscore 
(Cp. \textunderscore pýxide\textunderscore )
\section{Pixilinga}
\begin{itemize}
\item {Grp. gram.:f.}
\end{itemize}
\begin{itemize}
\item {Utilização:Bras. do N}
\end{itemize}
Piolho de gallinha.
\section{Pixirica}
\begin{itemize}
\item {Grp. gram.:f.}
\end{itemize}
Planta melastomácea do Brasil.
\section{Pixiricuçu}
\begin{itemize}
\item {Grp. gram.:m.}
\end{itemize}
Planta melastomácea do Brasil.
\section{Pixiricussu}
\begin{itemize}
\item {Grp. gram.:m.}
\end{itemize}
Planta melastomácea do Brasil.
\section{Pixispixis}
\begin{itemize}
\item {Grp. gram.:m.}
\end{itemize}
\begin{itemize}
\item {Utilização:Bras}
\end{itemize}
Tríbo de indígenas do Pará.
\section{Pisorga}
\begin{itemize}
\item {Grp. gram.:f.}
\end{itemize}
\begin{itemize}
\item {Utilização:Des.}
\end{itemize}
\begin{itemize}
\item {Utilização:Gír.}
\end{itemize}
\begin{itemize}
\item {Utilização:Prov.}
\end{itemize}
\begin{itemize}
\item {Utilização:beir.}
\end{itemize}
\begin{itemize}
\item {Utilização:Prov.}
\end{itemize}
\begin{itemize}
\item {Utilização:trasm.}
\end{itemize}
\begin{itemize}
\item {Grp. gram.:Adj.}
\end{itemize}
Bebedeira.
Beberagem imunda ou adulterada.
Vinho ruim, ou aguado.
O mesmo que \textunderscore bêbedo\textunderscore .
\section{Pixôxo}
\begin{itemize}
\item {Grp. gram.:m.}
\end{itemize}
\begin{itemize}
\item {Utilização:Bras}
\end{itemize}
Pássaro damninho aos arrozaes.
\section{Pixuna}
\begin{itemize}
\item {Grp. gram.:adj.}
\end{itemize}
O mesmo que \textunderscore una\textunderscore .
\section{Pizorga}
\begin{itemize}
\item {Grp. gram.:f.}
\end{itemize}
\begin{itemize}
\item {Utilização:Des.}
\end{itemize}
\begin{itemize}
\item {Utilização:Gír.}
\end{itemize}
\begin{itemize}
\item {Utilização:Prov.}
\end{itemize}
\begin{itemize}
\item {Utilização:beir.}
\end{itemize}
\begin{itemize}
\item {Utilização:Prov.}
\end{itemize}
\begin{itemize}
\item {Utilização:trasm.}
\end{itemize}
\begin{itemize}
\item {Grp. gram.:Adj.}
\end{itemize}
Bebedeira.
Beberagem immunda ou adulterada.
Vinho ruim, ou aguado.
O mesmo que \textunderscore bêbedo\textunderscore .
\section{Placa}
\begin{itemize}
\item {Grp. gram.:f.}
\end{itemize}
\begin{itemize}
\item {Utilização:Gír.}
\end{itemize}
Chapa, lâmina de metal.
Espécie de escápula, fixa na parede, num piano, etc., e destinada a sustentar uma vela ou um candeeiro.
Palmatória de metal, em que se firma uma vela. Cf. Camillo, \textunderscore Brasileira\textunderscore , 199.
Insígnia de dignitário, condecoração.
Moéda de prata.--No sentido de \textunderscore chapa\textunderscore  ou \textunderscore lâmina\textunderscore , é t. afrancesado, no conceito de alguns críticos.
(Cast. \textunderscore placa\textunderscore )
\section{Placabilidade}
\begin{itemize}
\item {Grp. gram.:f.}
\end{itemize}
\begin{itemize}
\item {Proveniência:(Lat. \textunderscore placabilitas\textunderscore )}
\end{itemize}
Qualidade do que é placável.
\section{Placar}
\begin{itemize}
\item {Proveniência:(Lat. \textunderscore placare\textunderscore )}
\end{itemize}
\textunderscore v. t.\textunderscore  (e der.)
O mesmo que \textunderscore aplacar\textunderscore , etc. Cf. Castilho, \textunderscore Geórgicas\textunderscore , 293.
\section{Placar}
\begin{itemize}
\item {Grp. gram.:m.}
\end{itemize}
\begin{itemize}
\item {Utilização:Fam.}
\end{itemize}
\begin{itemize}
\item {Proveniência:(Fr. \textunderscore placard\textunderscore )}
\end{itemize}
Venera, condecoração.
\section{Placável}
\begin{itemize}
\item {Grp. gram.:adj.}
\end{itemize}
\begin{itemize}
\item {Proveniência:(Lat. \textunderscore placabilis\textunderscore )}
\end{itemize}
Que se póde applacar.
\section{Placença}
\begin{itemize}
\item {Grp. gram.:f.}
\end{itemize}
\begin{itemize}
\item {Utilização:Ant.}
\end{itemize}
\begin{itemize}
\item {Proveniência:(Do lat. \textunderscore placentia\textunderscore )}
\end{itemize}
O mesmo que \textunderscore beneplácito\textunderscore ; approvação.
Arbítrio.
\section{Placenta}
\begin{itemize}
\item {Grp. gram.:m.  ou  f.}
\end{itemize}
\begin{itemize}
\item {Proveniência:(Lat. \textunderscore placenta\textunderscore )}
\end{itemize}
Massa esponjosa que, durante a gestação, estabelece communicação entre a mãe e o filho.
Órgão vegetal, a que estão ligados os óvulos.
\section{Placentação}
\begin{itemize}
\item {Grp. gram.:f.}
\end{itemize}
Disposição das placentas no ovário vegetal.
\section{Placentário}
\begin{itemize}
\item {Grp. gram.:adj.}
\end{itemize}
\begin{itemize}
\item {Grp. gram.:M.}
\end{itemize}
Relativo á placenta.
Parte do fruto, formada pela reunião de muitas placentas.
\section{Placentim}
\begin{itemize}
\item {Grp. gram.:m.}
\end{itemize}
\begin{itemize}
\item {Utilização:Ant.}
\end{itemize}
\begin{itemize}
\item {Proveniência:(Lat. \textunderscore placentinus\textunderscore )}
\end{itemize}
Homem natural de Plaçença, em Itália.
Mercador de Lisbôa, natural de Plaçença.
\section{Placentino}
\begin{itemize}
\item {Grp. gram.:m.}
\end{itemize}
\begin{itemize}
\item {Utilização:Ant.}
\end{itemize}
\begin{itemize}
\item {Proveniência:(Lat. \textunderscore placentinus\textunderscore )}
\end{itemize}
Homem natural de Plaçença, em Itália.
Mercador de Lisbôa, natural de Plaçença.
\section{Placidamente}
\begin{itemize}
\item {Grp. gram.:adv.}
\end{itemize}
De modo plácido.
\section{Placidez}
\begin{itemize}
\item {Grp. gram.:f.}
\end{itemize}
Qualidade ou estado do que é plácido.
\section{Plácido}
\begin{itemize}
\item {Grp. gram.:adj.}
\end{itemize}
\begin{itemize}
\item {Proveniência:(Lat. \textunderscore placidus\textunderscore )}
\end{itemize}
Tranquillo, sereno: \textunderscore lagos plácidos\textunderscore .
Pacífico.
Que mostra serenidade de espírito.
Brando.
Em que há paz e sossêgo: \textunderscore horas plácidas\textunderscore .
\section{Placimento}
\begin{itemize}
\item {Grp. gram.:m.}
\end{itemize}
\begin{itemize}
\item {Utilização:Ant.}
\end{itemize}
O mesmo que \textunderscore placença\textunderscore ^1.
\section{Plácito}
\begin{itemize}
\item {Grp. gram.:m.}
\end{itemize}
\begin{itemize}
\item {Grp. gram.:M. pl.}
\end{itemize}
\begin{itemize}
\item {Utilização:Des.}
\end{itemize}
\begin{itemize}
\item {Proveniência:(Lat. \textunderscore placitum\textunderscore )}
\end{itemize}
Beneplácito; approvação.
Pacto.
Promessa.
Sentenças de moralistas ou philósophos.
\section{Placóides}
\begin{itemize}
\item {Grp. gram.:m. pl.}
\end{itemize}
\begin{itemize}
\item {Proveniência:(Do gr. \textunderscore plax\textunderscore  + \textunderscore eidos\textunderscore )}
\end{itemize}
Designação dos peixes sem escamas, segundo Agassiz.
\section{Plaga}
\begin{itemize}
\item {Grp. gram.:f.}
\end{itemize}
\begin{itemize}
\item {Proveniência:(Lat. \textunderscore plaga\textunderscore )}
\end{itemize}
Região; tracto de terreno.
\section{Plaga}
\begin{itemize}
\item {Grp. gram.:f.}
\end{itemize}
Designação antiga de certo tom musical.
(Cf. Du-Cange, vb. \textunderscore plaga\textunderscore )
\section{Plagal}
\begin{itemize}
\item {Grp. gram.:adj.}
\end{itemize}
\begin{itemize}
\item {Proveniência:(Lat. \textunderscore plagalis\textunderscore )}
\end{itemize}
Relativo a plaga^2: \textunderscore tom plagal\textunderscore .--Dos oito tons do cantochão, quatro dizem-se \textunderscore authênticos\textunderscore , e quatro dizem-se \textunderscore plagaes\textunderscore .
\section{Plagiador}
\begin{itemize}
\item {Grp. gram.:m.}
\end{itemize}
\begin{itemize}
\item {Proveniência:(Do lat. \textunderscore plagiator\textunderscore )}
\end{itemize}
O mesmo que \textunderscore plagiário\textunderscore .
\section{Plagiante}
\begin{itemize}
\item {Grp. gram.:adj.}
\end{itemize}
Que plagia. Cf. Alves Mendes, \textunderscore Plágios\textunderscore , 51.
\section{Plagiar}
\begin{itemize}
\item {Grp. gram.:v. t.}
\end{itemize}
\begin{itemize}
\item {Proveniência:(Lat. \textunderscore plagiare\textunderscore )}
\end{itemize}
Subscrever ou apresentar como seu (um trabalho alheio).
Imitar servilmente (trabalho de outrem).
Respigar, forragear.
\section{Plagiário}
\begin{itemize}
\item {Grp. gram.:m.}
\end{itemize}
\begin{itemize}
\item {Proveniência:(Lat. \textunderscore plagiarius\textunderscore )}
\end{itemize}
Aquelle que apresenta como seu um trabalho literário ou scientífico de outrem.
Entre os antigos, aquelle que roubava os escravos para os vender.
\section{Plagiato}
\begin{itemize}
\item {Grp. gram.:m.}
\end{itemize}
\begin{itemize}
\item {Proveniência:(Lat. \textunderscore plagiatus\textunderscore )}
\end{itemize}
Acto ou effeito de plagiar.
\section{Plagiédrico}
\begin{itemize}
\item {Grp. gram.:adj.}
\end{itemize}
O mesmo ou melhor que \textunderscore plagiedro\textunderscore .
\section{Plagiedro}
\begin{itemize}
\item {Grp. gram.:adj.}
\end{itemize}
\begin{itemize}
\item {Utilização:Miner.}
\end{itemize}
\begin{itemize}
\item {Proveniência:(Do gr. \textunderscore plagios\textunderscore  + \textunderscore edra\textunderscore )}
\end{itemize}
Que tem facetas oblíquas.
\section{Plágio}
\begin{itemize}
\item {Grp. gram.:m.}
\end{itemize}
\begin{itemize}
\item {Proveniência:(Lat. \textunderscore plagium\textunderscore )}
\end{itemize}
O mesmo que \textunderscore plagiato\textunderscore .
\section{Plagiobásico}
\begin{itemize}
\item {Grp. gram.:adj.}
\end{itemize}
\begin{itemize}
\item {Utilização:Miner.}
\end{itemize}
\begin{itemize}
\item {Proveniência:(Do gr. \textunderscore plagios\textunderscore  + \textunderscore basis\textunderscore )}
\end{itemize}
Diz-se dos systemas que têm coordenadas oblíquas.
\section{Plagiocefalia}
\begin{itemize}
\item {Grp. gram.:f.}
\end{itemize}
Qualidade de plagiocéfalo.
\section{Plagiocéfalo}
\begin{itemize}
\item {Grp. gram.:adj.}
\end{itemize}
\begin{itemize}
\item {Proveniência:(Do gr. \textunderscore plagios\textunderscore  + \textunderscore kephale\textunderscore )}
\end{itemize}
Que tem a cabeça obliquamente deformada, isto é, que tem a fronte deprimida, e prolongada para trás a parte posterior.
\section{Plagiocephalia}
\begin{itemize}
\item {Grp. gram.:f.}
\end{itemize}
Qualidade de plagiocéphalo.
\section{Plagiocéphalo}
\begin{itemize}
\item {Grp. gram.:adj.}
\end{itemize}
\begin{itemize}
\item {Proveniência:(Do gr. \textunderscore plagios\textunderscore  + \textunderscore kephale\textunderscore )}
\end{itemize}
Que tem a cabeça obliquamente deformada, isto é, que tem a fronte deprimida, e prolongada para trás a parte posterior.
\section{Plagioclase}
\begin{itemize}
\item {Grp. gram.:f.}
\end{itemize}
\begin{itemize}
\item {Utilização:Miner.}
\end{itemize}
\begin{itemize}
\item {Proveniência:(Do gr. \textunderscore plagios\textunderscore  + \textunderscore klasis\textunderscore )}
\end{itemize}
Estado dos feldspathos, cujas duas clivagens são ligeiramente inclinadas.
\section{Plagioclásico}
\begin{itemize}
\item {Grp. gram.:adj.}
\end{itemize}
Relativo á plagioclase.
\section{Plagionita}
\begin{itemize}
\item {Grp. gram.:f.}
\end{itemize}
\begin{itemize}
\item {Proveniência:(Do gr. \textunderscore plagios\textunderscore , oblíquo)}
\end{itemize}
Sulfureto duplo de chumbo e antimónio, cuja crystallização deriva de um prisma rhomboidal oblíquo.
\section{Plagióstomo}
\begin{itemize}
\item {Grp. gram.:adj.}
\end{itemize}
\begin{itemize}
\item {Utilização:Zool.}
\end{itemize}
\begin{itemize}
\item {Grp. gram.:M. pl.}
\end{itemize}
\begin{itemize}
\item {Proveniência:(Do gr. \textunderscore plagios\textunderscore  + \textunderscore stoma\textunderscore )}
\end{itemize}
Que tem a boca oblíqua ou transversal.
Família de peixes cartilaginosos.
Gênero de molluscos acéphalos.
\section{Plaina}
\begin{itemize}
\item {Grp. gram.:f.}
\end{itemize}
\begin{itemize}
\item {Utilização:Carp.}
\end{itemize}
Instrumento, para alisar madeira, especialmente tábuas.
(Fem. de \textunderscore plaino\textunderscore )
\section{Plainete}
\begin{itemize}
\item {fónica:nê}
\end{itemize}
\begin{itemize}
\item {Grp. gram.:m.}
\end{itemize}
Instrumento, para cinzelar metaes. Cf. Castilho, \textunderscore Fastos\textunderscore , II, 379.
(Cp. \textunderscore plaina\textunderscore )
\section{Plainista}
\begin{itemize}
\item {Grp. gram.:m.}
\end{itemize}
\begin{itemize}
\item {Utilização:T. de Lisbôa}
\end{itemize}
Operário, que trabalha com plaina, em officina de carpinteiro.
\section{Plaino}
\begin{itemize}
\item {Grp. gram.:m.}
\end{itemize}
\begin{itemize}
\item {Grp. gram.:Adj.}
\end{itemize}
O mesmo que \textunderscore planície\textunderscore .
Campina.
Campo.
Chapada.
O mesmo que \textunderscore plano\textunderscore . Cf. Castilho, \textunderscore Geórgicas\textunderscore , 49.
(Cp. \textunderscore plano\textunderscore )
\section{Plameira}
\begin{itemize}
\item {Grp. gram.:f.}
\end{itemize}
\begin{itemize}
\item {Utilização:Prov.}
\end{itemize}
\begin{itemize}
\item {Utilização:trasm.}
\end{itemize}
Pedra que, na cozinha, fica acima da lareira e um pouco saliente, servindo de base a uma pilheira ou esconso, onde se guardam pregos velhos, pevides, etc.
\section{Plana}
\begin{itemize}
\item {Grp. gram.:f.}
\end{itemize}
\begin{itemize}
\item {Proveniência:(De \textunderscore plano\textunderscore )}
\end{itemize}
Categoria, classe: \textunderscore é um patife de primeira plana\textunderscore .
\section{Planado}
\begin{itemize}
\item {Grp. gram.:adj.}
\end{itemize}
\begin{itemize}
\item {Utilização:Neol.}
\end{itemize}
Diz-se do vôo, que o aeroplano faz, parecendo conservar-se no mesmo ponto.--Melhor seria \textunderscore pairo\textunderscore .
\section{Planalto}
\begin{itemize}
\item {Grp. gram.:m.}
\end{itemize}
\begin{itemize}
\item {Proveniência:(De \textunderscore plano\textunderscore  + \textunderscore alto\textunderscore )}
\end{itemize}
Terreno elevado e plano; planície sôbre montes; achada^2.
\section{Planamente}
\begin{itemize}
\item {Grp. gram.:adv.}
\end{itemize}
De modo plano.
\section{Planarado}
\begin{itemize}
\item {Grp. gram.:m.}
\end{itemize}
\begin{itemize}
\item {Utilização:Des.}
\end{itemize}
\begin{itemize}
\item {Proveniência:(De \textunderscore plano\textunderscore  + \textunderscore arado\textunderscore )}
\end{itemize}
Arado com rodas.
\section{Plancha}
\begin{itemize}
\item {Grp. gram.:f.}
\end{itemize}
O mesmo que \textunderscore prancha\textunderscore . Cf. Filinto, \textunderscore D. Man.\textunderscore , I. 106 e 226.
(Cast. \textunderscore plancha\textunderscore )
\section{Planchear}
\begin{itemize}
\item {Grp. gram.:v. i.}
\end{itemize}
\begin{itemize}
\item {Utilização:Bras}
\end{itemize}
\begin{itemize}
\item {Proveniência:(De \textunderscore plancha\textunderscore )}
\end{itemize}
Caír de lado (o cavallo com o cavalleiro).
\section{Plancheta}
\begin{itemize}
\item {fónica:chê}
\end{itemize}
\begin{itemize}
\item {Grp. gram.:f.}
\end{itemize}
(V.prancheta)(Cast. \textunderscore plancheta\textunderscore )
\section{Planear}
\begin{itemize}
\item {Grp. gram.:v. t.}
\end{itemize}
Fazer o plano de; projectar; tencionar.
\section{Planejar}
\begin{itemize}
\item {Grp. gram.:v. t.}
\end{itemize}
O mesmo que \textunderscore planear\textunderscore . Cf. Castilho, \textunderscore Fastos\textunderscore , II, 181.
\section{Planêta}
\begin{itemize}
\item {Grp. gram.:m.}
\end{itemize}
\begin{itemize}
\item {Proveniência:(Lat. \textunderscore planeta\textunderscore )}
\end{itemize}
Corpo celeste, que gira em volta do Sol, de que recebe a luz, e que se chama \textunderscore planêta primário\textunderscore .
Corpo celeste, que gira em volta de um planeta primário e que, por isso, se chama \textunderscore planêta secundário\textunderscore : \textunderscore a Terra é um planêta primário, e a Lua um planêta secundário\textunderscore .
\section{Planéta}
\begin{itemize}
\item {Grp. gram.:f.}
\end{itemize}
Casula sacerdotal.
(B. lat. \textunderscore planeta\textunderscore )
\section{Planetário}
\begin{itemize}
\item {Grp. gram.:adj.}
\end{itemize}
\begin{itemize}
\item {Grp. gram.:M.}
\end{itemize}
\begin{itemize}
\item {Proveniência:(Lat. \textunderscore planetarius\textunderscore )}
\end{itemize}
Relativo a planêtas.
Maquinismo, com que se representa o movimento dos planêtas.
\section{Planete}
\begin{itemize}
\item {fónica:nê}
\end{itemize}
\begin{itemize}
\item {Grp. gram.:m.}
\end{itemize}
\begin{itemize}
\item {Proveniência:(De \textunderscore plano\textunderscore )}
\end{itemize}
Parte lisa, um pouco mais alta que a gravura, na moéda. Cf. F. de Mendonça, \textunderscore Vocab. Techn.\textunderscore 
\section{Planeza}
\begin{itemize}
\item {Grp. gram.:f.}
\end{itemize}
Estado do que é plano.
O mesmo que \textunderscore planície\textunderscore .
\section{Plangana}
\begin{itemize}
\item {Grp. gram.:f.}
\end{itemize}
\begin{itemize}
\item {Utilização:Prov.}
\end{itemize}
\begin{itemize}
\item {Utilização:beir.}
\end{itemize}
O mesmo que \textunderscore palangana\textunderscore .
\section{Plangência}
\begin{itemize}
\item {Grp. gram.:f.}
\end{itemize}
Estado ou qualidade de plangente. Cf. Camillo, \textunderscore Mar. da Fonte\textunderscore , 167; \textunderscore Narcót.\textunderscore , I, 283.
\section{Plangente}
\begin{itemize}
\item {Grp. gram.:adj.}
\end{itemize}
\begin{itemize}
\item {Proveniência:(Lat. \textunderscore plangens\textunderscore )}
\end{itemize}
Que chora; lastimoso; que se pranteia; triste: \textunderscore vozes plangentes\textunderscore .
\section{Plangentemente}
\begin{itemize}
\item {Grp. gram.:adv.}
\end{itemize}
De modo plangente.
\section{Planger}
\begin{itemize}
\item {Grp. gram.:v. i.}
\end{itemize}
\begin{itemize}
\item {Utilização:Des.}
\end{itemize}
\begin{itemize}
\item {Proveniência:(Do lat. \textunderscore plangere\textunderscore )}
\end{itemize}
Chorar; lastimar-se.
Soar tristemente: \textunderscore o sino plange\textunderscore .
\section{Plangitivo}
\begin{itemize}
\item {Grp. gram.:adj.}
\end{itemize}
\begin{itemize}
\item {Utilização:Neol.}
\end{itemize}
Em que há pranto ou grande tristeza; prantivo; lacrimoso. Cf. Th. Braga, \textunderscore Mod. Ideias\textunderscore , I, 340; Camillo, \textunderscore Cancion. Al.\textunderscore , 234.
(Cp. \textunderscore plangente\textunderscore )
\section{Planice}
\begin{itemize}
\item {Grp. gram.:f.}
\end{itemize}
O mesmo que \textunderscore planície\textunderscore .
\section{Planície}
\begin{itemize}
\item {Grp. gram.:f.}
\end{itemize}
\begin{itemize}
\item {Proveniência:(Lat. \textunderscore planities\textunderscore )}
\end{itemize}
Grande porção de terreno plano.
Campo; campina.
\section{Planicórneo}
\begin{itemize}
\item {Grp. gram.:adj.}
\end{itemize}
\begin{itemize}
\item {Utilização:Zool.}
\end{itemize}
\begin{itemize}
\item {Proveniência:(Do lat. \textunderscore planus\textunderscore  + \textunderscore cornu\textunderscore )}
\end{itemize}
Que tem cornos achatados.
\section{Planificação}
\begin{itemize}
\item {Grp. gram.:f.}
\end{itemize}
Acto ou effeito de planificar.
\section{Planificar}
\begin{itemize}
\item {Grp. gram.:v. t.}
\end{itemize}
\begin{itemize}
\item {Proveniência:(Do lat. \textunderscore planus\textunderscore  + \textunderscore facere\textunderscore )}
\end{itemize}
Desenhar ou traçar num plano (os vários accidentes de uma perspectiva).
\section{Planifólio}
\begin{itemize}
\item {Grp. gram.:adj.}
\end{itemize}
\begin{itemize}
\item {Utilização:Bot.}
\end{itemize}
\begin{itemize}
\item {Proveniência:(Do lat. \textunderscore planus\textunderscore  + \textunderscore folium\textunderscore )}
\end{itemize}
Que tem fôlhas planas. Cf. Vic. Machado, \textunderscore As Ilhas de San-Thomé\textunderscore , 127.
\section{Planiforme}
\begin{itemize}
\item {Grp. gram.:adj.}
\end{itemize}
\begin{itemize}
\item {Grp. gram.:M. pl.}
\end{itemize}
\begin{itemize}
\item {Proveniência:(Do lat. \textunderscore planus\textunderscore  + \textunderscore forma\textunderscore )}
\end{itemize}
Que tem fórma achatada; chato.
Família ou ordem de insectos planiformes.
\section{Planiglobo}
\begin{itemize}
\item {Grp. gram.:m.}
\end{itemize}
\begin{itemize}
\item {Proveniência:(De \textunderscore plano\textunderscore  + \textunderscore globo\textunderscore )}
\end{itemize}
O mesmo que \textunderscore planisphério\textunderscore .
\section{Planimetria}
\begin{itemize}
\item {Grp. gram.:f.}
\end{itemize}
\begin{itemize}
\item {Proveniência:(De \textunderscore plano\textunderscore  + gr. \textunderscore metron\textunderscore )}
\end{itemize}
Parte da Geometria, que ensina a medir os planos e as superfícies.
\section{Planimétrico}
\begin{itemize}
\item {Grp. gram.:adj.}
\end{itemize}
Relativo á planimetria.
\section{Planímetro}
\begin{itemize}
\item {Grp. gram.:m.}
\end{itemize}
\begin{itemize}
\item {Proveniência:(Do lat. \textunderscore planus\textunderscore  + gr. \textunderscore metron\textunderscore )}
\end{itemize}
Instrumento, para medir as superfícies planas.
\section{Planipédias}
\begin{itemize}
\item {Grp. gram.:f. pl.}
\end{itemize}
\begin{itemize}
\item {Proveniência:(Do lat. \textunderscore planipedius\textunderscore )}
\end{itemize}
Antigas comédias romanas, representadas nas ruas e praças por saltimbancos ou pantomimeiros descalços.
\section{Planipene}
\begin{itemize}
\item {Grp. gram.:adj.}
\end{itemize}
\begin{itemize}
\item {Grp. gram.:M. pl.}
\end{itemize}
\begin{itemize}
\item {Proveniência:(Do lat. \textunderscore planus\textunderscore  + \textunderscore penna\textunderscore )}
\end{itemize}
Que tem penas planas.
Família de insectos neurópteros.
\section{Planipenne}
\begin{itemize}
\item {Grp. gram.:adj.}
\end{itemize}
\begin{itemize}
\item {Grp. gram.:M. pl.}
\end{itemize}
\begin{itemize}
\item {Proveniência:(Do lat. \textunderscore planus\textunderscore  + \textunderscore penna\textunderscore )}
\end{itemize}
Que tem pennas planas.
Família de insectos neurópteros.
\section{Planisférico}
\begin{itemize}
\item {Grp. gram.:adj.}
\end{itemize}
Relativo ao planisfério.
\section{Planisfério}
\begin{itemize}
\item {Grp. gram.:m.}
\end{itemize}
\begin{itemize}
\item {Utilização:Restrict.}
\end{itemize}
\begin{itemize}
\item {Proveniência:(Do lat. \textunderscore planus\textunderscore  + \textunderscore sphera\textunderscore )}
\end{itemize}
Representação de um globo ou esfera, sôbre um plano.
Mapa, que representa os dois hemisférios terrestres em superfícies planas.
\section{Planisphérico}
\begin{itemize}
\item {Grp. gram.:adj.}
\end{itemize}
Relativo ao planisphério.
\section{Planisphério}
\begin{itemize}
\item {Grp. gram.:m.}
\end{itemize}
\begin{itemize}
\item {Utilização:Restrict.}
\end{itemize}
\begin{itemize}
\item {Proveniência:(Do lat. \textunderscore planus\textunderscore  + \textunderscore sphera\textunderscore )}
\end{itemize}
Representação de um globo ou esphera, sôbre um plano.
Mappa, que representa os dois hemisphérios terrestres em superfícies planas.
\section{Planizar}
\textunderscore v. t.\textunderscore  (e der.)
O mesmo que \textunderscore planear\textunderscore , etc.:«\textunderscore ...á medida que planizava a sua ida ao Pôrto.\textunderscore »Camillo, \textunderscore Am. de Perdição\textunderscore , (ed. monum.) 162.
\section{Plano}
\begin{itemize}
\item {Grp. gram.:adj.}
\end{itemize}
\begin{itemize}
\item {Utilização:Fig.}
\end{itemize}
\begin{itemize}
\item {Grp. gram.:M.}
\end{itemize}
\begin{itemize}
\item {Utilização:Neol.}
\end{itemize}
\begin{itemize}
\item {Grp. gram.:Loc. adv.}
\end{itemize}
\begin{itemize}
\item {Proveniência:(Lat. \textunderscore planus\textunderscore )}
\end{itemize}
Diz-se de qualquer superfície, que não tem desigualdades, nem curvas, nem ondulações.
Liso.
Em que se póde assentar uma linha recta em todas as direcções.
Claro, corrente, fácil.
Superfície plana.
Campo.
Desenho, que representa a projecção horizontal de um edifício, cidade, etc.; planta.
Projecto, programma; intenção.
Cada uma das lâminas horizontaes, que servem como de asas ao aeroplano.
\textunderscore De plano\textunderscore , promptamente; summariamente.
\section{Planorba}
\begin{itemize}
\item {Grp. gram.:f.}
\end{itemize}
\begin{itemize}
\item {Proveniência:(Do lat. \textunderscore planus\textunderscore  + \textunderscore orbis\textunderscore )}
\end{itemize}
Gênero de molluscos gasterópodes, (\textunderscore planorbis discus\textunderscore ).
\section{Planqueta}
\begin{itemize}
\item {fónica:quê}
\end{itemize}
\begin{itemize}
\item {Grp. gram.:f.}
\end{itemize}
\begin{itemize}
\item {Proveniência:(Fr. \textunderscore planchette\textunderscore )}
\end{itemize}
Peça de ferro, composta de duas malhas, ligadas por uma haste, e empregada em combates navaes, para desmastrear navios.
\section{Planta}
\begin{itemize}
\item {Grp. gram.:f.}
\end{itemize}
\begin{itemize}
\item {Utilização:Bot.}
\end{itemize}
\begin{itemize}
\item {Utilização:Restrict.}
\end{itemize}
\begin{itemize}
\item {Utilização:Anat.}
\end{itemize}
\begin{itemize}
\item {Utilização:Ext.}
\end{itemize}
\begin{itemize}
\item {Proveniência:(Lat. \textunderscore planta\textunderscore )}
\end{itemize}
Nome genérico, que comprehende todos os vegetaes.
Vegetal, de que se não extrái madeira ou que não é árvore.
Parte do pé, que assenta sôbre o chão.
O mesmo que \textunderscore pé\textunderscore : \textunderscore foi ajoelhar-lhe ás plantas\textunderscore .
Desenho, que representa, em projecção horizontal, um edifício, uma cidade, etc.
\section{Plantação}
\begin{itemize}
\item {Grp. gram.:f.}
\end{itemize}
\begin{itemize}
\item {Proveniência:(Do lat. \textunderscore plantatio\textunderscore )}
\end{itemize}
Acto ou effeito de plantar.
Terreno, em que crescem plantas.
\section{Plantada}
\begin{itemize}
\item {Grp. gram.:f.}
\end{itemize}
\begin{itemize}
\item {Utilização:Ant.}
\end{itemize}
Chão plantado de vinha ou árvores.
(Fem. de \textunderscore plantado\textunderscore )
\section{Plantador}
\begin{itemize}
\item {Grp. gram.:m.  e  adj.}
\end{itemize}
\begin{itemize}
\item {Grp. gram.:M.}
\end{itemize}
\begin{itemize}
\item {Proveniência:(De \textunderscore plantar\textunderscore )}
\end{itemize}
O que planta.
Instrumento, para plantio de bacêllo.
\section{Plantagem}
\begin{itemize}
\item {Grp. gram.:f.}
\end{itemize}
O mesmo que \textunderscore tanchagem\textunderscore .
\section{Plantagíneas}
\begin{itemize}
\item {Grp. gram.:f. pl.}
\end{itemize}
Família de plantas herbáceas, que tem por typo a tanchagem.
(Fem. pl. de \textunderscore plantagíneo\textunderscore )
\section{Plantagíneo}
\begin{itemize}
\item {Grp. gram.:adj.}
\end{itemize}
\begin{itemize}
\item {Proveniência:(Do lat. \textunderscore plantago\textunderscore )}
\end{itemize}
Relativo ou semelhante á tanchagem.
\section{Plantal}
\begin{itemize}
\item {Grp. gram.:adj.}
\end{itemize}
\begin{itemize}
\item {Utilização:Anat.}
\end{itemize}
Relativo á planta do pé; o mesmo que \textunderscore plantar\textunderscore , \textunderscore adj.\textunderscore 
\section{Planta-nova}
\begin{itemize}
\item {Grp. gram.:f.}
\end{itemize}
Variedade de videira da Bairrada.
Uva dessa videira.
\section{Plantão}
\begin{itemize}
\item {Grp. gram.:m.}
\end{itemize}
\begin{itemize}
\item {Proveniência:(Fr. \textunderscore planton\textunderscore )}
\end{itemize}
Serviço policial, distribuido diariamente a um soldado, dentro da própria caserna, companhia ou bataria.
\section{Plantar}
\begin{itemize}
\item {Grp. gram.:v. t.}
\end{itemize}
\begin{itemize}
\item {Grp. gram.:Adj.}
\end{itemize}
\begin{itemize}
\item {Proveniência:(Lat. \textunderscore plantare\textunderscore )}
\end{itemize}
Meter na terra para aí criar raízes (um vegetal): \textunderscore plantar couves\textunderscore .
Semear.
Cultivar.
Collocar: \textunderscore plantar um livro na estante\textunderscore .
Cravar na terra verticalmente: \textunderscore plantar uma bandeirola\textunderscore .
Assentar.
Estabelecer.
Dispor vegetaes em: \textunderscore plantar um canteiro\textunderscore .
Relativo á planta do pé.
\section{Plantear}
\begin{itemize}
\item {Grp. gram.:v. t.}
\end{itemize}
\begin{itemize}
\item {Utilização:P. us.}
\end{itemize}
\begin{itemize}
\item {Proveniência:(De \textunderscore planta\textunderscore )}
\end{itemize}
Desenhar ou fazer a planta de (uma construcção).
\section{Plantear}
\begin{itemize}
\item {Grp. gram.:v. t.}
\end{itemize}
\begin{itemize}
\item {Utilização:Ant.}
\end{itemize}
\begin{itemize}
\item {Proveniência:(Do lat. \textunderscore planctus\textunderscore )}
\end{itemize}
O mesmo que \textunderscore prantear\textunderscore . Cf. Pant. de Aveiro, \textunderscore Itiner.\textunderscore , 176, (2.^a ed.).
\section{Plantígrado}
\begin{itemize}
\item {Grp. gram.:adj.}
\end{itemize}
\begin{itemize}
\item {Grp. gram.:M. pl.}
\end{itemize}
\begin{itemize}
\item {Proveniência:(Do lat. \textunderscore plantus\textunderscore  + \textunderscore gradi\textunderscore )}
\end{itemize}
Que anda sôbre as plantas dos pés.
Tríbo de mammíferos plantígrados, a que pertence o urso.
\section{Plantio}
\begin{itemize}
\item {Grp. gram.:m.}
\end{itemize}
O mesmo que \textunderscore plantação\textunderscore .
\section{Planto}
\begin{itemize}
\item {Grp. gram.:m.}
\end{itemize}
\begin{itemize}
\item {Utilização:Ant.}
\end{itemize}
O mesmo que \textunderscore pranto\textunderscore . Cf. Usque, XIII, 35, etc.
\section{Plantomania}
\begin{itemize}
\item {Grp. gram.:f.}
\end{itemize}
\begin{itemize}
\item {Proveniência:(De \textunderscore planta\textunderscore  + \textunderscore mania\textunderscore )}
\end{itemize}
Mania das plantações.
\section{Plântula}
\begin{itemize}
\item {Grp. gram.:f.}
\end{itemize}
Embryão vegetal, que começa a desenvolver-se pelo acto da germinação.
(Dem. de \textunderscore planta\textunderscore )
\section{Plantulação}
\begin{itemize}
\item {Grp. gram.:f.}
\end{itemize}
\begin{itemize}
\item {Utilização:Des.}
\end{itemize}
\begin{itemize}
\item {Proveniência:(De \textunderscore plântula\textunderscore )}
\end{itemize}
O mesmo que \textunderscore germinação\textunderscore .
\section{Planturoso}
\begin{itemize}
\item {Grp. gram.:adj.}
\end{itemize}
\begin{itemize}
\item {Utilização:bras}
\end{itemize}
\begin{itemize}
\item {Utilização:Gal}
\end{itemize}
\begin{itemize}
\item {Utilização:Fig.}
\end{itemize}
\begin{itemize}
\item {Proveniência:(Fr. \textunderscore plantureux\textunderscore )}
\end{itemize}
Crescido, volumoso:«\textunderscore ...repolhos planturosos\textunderscore ». C. Neto.
Abundante, copioso: \textunderscore jantar planturoso\textunderscore .
Prolixo; abundante de ideias: \textunderscore estilo planturoso\textunderscore .
\section{Planura}
\begin{itemize}
\item {Grp. gram.:f.}
\end{itemize}
\begin{itemize}
\item {Proveniência:(De \textunderscore plano\textunderscore )}
\end{itemize}
Planície; planalto.
\section{Plaqué}
\begin{itemize}
\item {Grp. gram.:m.}
\end{itemize}
\begin{itemize}
\item {Utilização:Gal}
\end{itemize}
\begin{itemize}
\item {Proveniência:(Fr. \textunderscore plaqué\textunderscore )}
\end{itemize}
Fôlha de metal, o mesmo que \textunderscore casquinha\textunderscore .
\section{Plasma}
\begin{itemize}
\item {Grp. gram.:m.}
\end{itemize}
\begin{itemize}
\item {Utilização:Miner.}
\end{itemize}
\begin{itemize}
\item {Proveniência:(Gr. \textunderscore plasma\textunderscore )}
\end{itemize}
A parte líquida do sangue, em que nadam os glóbulos microscópicos.
Espécie de quartzo muito translúcido.
\section{Plasmado}
\begin{itemize}
\item {Grp. gram.:adj.}
\end{itemize}
Feito; organizado; constituido.
Repleto:«\textunderscore ...homem de muita astúcia, plasmado de mentiras e de fraudes\textunderscore ». Filinto, \textunderscore D. Man.\textunderscore , III, 12.
\section{Plasmão}
\begin{itemize}
\item {Grp. gram.:m.}
\end{itemize}
Matéria albuminóide, extrahida do leite.
\section{Plasmar}
\begin{itemize}
\item {Grp. gram.:v. t.}
\end{itemize}
\begin{itemize}
\item {Proveniência:(Lat. \textunderscore plasmare\textunderscore )}
\end{itemize}
Modelar em gêsso, barro, etc.
\section{Plasmático}
\begin{itemize}
\item {Grp. gram.:adj.}
\end{itemize}
Relativo ao plasma.
\section{Plasmódio}
\begin{itemize}
\item {Grp. gram.:m.}
\end{itemize}
\begin{itemize}
\item {Proveniência:(Do gr. \textunderscore plasma\textunderscore )}
\end{itemize}
Massa protoplásmica, resultante da fusão de céllulas primitivamente independentes.
O micróbio do impaludismo.
\section{Plasmogenia}
\begin{itemize}
\item {Grp. gram.:f.}
\end{itemize}
\begin{itemize}
\item {Proveniência:(Do gr. \textunderscore plasma\textunderscore  + \textunderscore genos\textunderscore )}
\end{itemize}
Theoria dos que pretendem que a primordial matéria viva se formou espontaneamente em líquido, que já continha a essência dessa matéria.
\section{Plasmópara}
\begin{itemize}
\item {Grp. gram.:m.}
\end{itemize}
\begin{itemize}
\item {Proveniência:(De \textunderscore plasma\textunderscore )}
\end{itemize}
Espécie de cogumelo.
\section{Plástica}
\begin{itemize}
\item {Grp. gram.:f.}
\end{itemize}
Arte de plasmar.
Arte de reconstituir artificialmente uma parte arruinada do corpo humano.
Conformação geral do corpo humano.
(Fem. de \textunderscore plástico\textunderscore )
\section{Plasticamente}
\begin{itemize}
\item {Grp. gram.:adv.}
\end{itemize}
\begin{itemize}
\item {Proveniência:(De \textunderscore plástico\textunderscore )}
\end{itemize}
Relativamente á plástica ou conformação do corpo humano.
Quanto ás fórmas ou feições do indivíduo.
\section{Plasticidade}
\begin{itemize}
\item {Grp. gram.:f.}
\end{itemize}
Qualidade de plástico.
\section{Plástico}
\begin{itemize}
\item {Grp. gram.:adj.}
\end{itemize}
\begin{itemize}
\item {Proveniência:(Lat. \textunderscore plasticus\textunderscore )}
\end{itemize}
Relativo á plástica.
Que tem o poder de formar.
Que serve para formar.
Que é susceptível de receber differentes fórmas ou de sêr modelado com os dedos: \textunderscore substâncias plásticas\textunderscore .
\section{Plastidário}
\begin{itemize}
\item {Grp. gram.:adj.}
\end{itemize}
Relativo a plastídio.
\section{Plastídio}
\begin{itemize}
\item {Grp. gram.:m.}
\end{itemize}
\begin{itemize}
\item {Proveniência:(Do gr. \textunderscore plastes\textunderscore )}
\end{itemize}
Pequena massa de protoplasma, que fórma um organismo elementar, livre ou associado, em organismo pluricellular.
\section{Plastídulo}
\begin{itemize}
\item {Grp. gram.:adj.}
\end{itemize}
\begin{itemize}
\item {Proveniência:(De \textunderscore plastídio\textunderscore )}
\end{itemize}
Granulação protoplásmica, á cêrca da qual tem havido várias concepcções.
\section{Plastilina}
\begin{itemize}
\item {Grp. gram.:f.}
\end{itemize}
Espécie de argilla, que serve para modelar; o mesmo que \textunderscore cerasina\textunderscore ^2.
\section{Plastodinamia}
\begin{itemize}
\item {Grp. gram.:f.}
\end{itemize}
\begin{itemize}
\item {Proveniência:(Do gr. \textunderscore plassein\textunderscore  + \textunderscore dunamos\textunderscore )}
\end{itemize}
Fôrça, que constitue e desenvolve os órgãos.
\section{Plastodynamia}
\begin{itemize}
\item {Grp. gram.:f.}
\end{itemize}
\begin{itemize}
\item {Proveniência:(Do gr. \textunderscore plassein\textunderscore  + \textunderscore dunamos\textunderscore )}
\end{itemize}
Fôrça, que constitue e desenvolve os órgãos.
\section{Plastrão}
\begin{itemize}
\item {Grp. gram.:m.}
\end{itemize}
\begin{itemize}
\item {Proveniência:(It. \textunderscore piastrone\textunderscore )}
\end{itemize}
Coiraça.
Almofada de esgrimista.
Gravata, que cobre o peito; peitilho.
\section{Platabanda}
\begin{itemize}
\item {Grp. gram.:f.}
\end{itemize}
(V.platibanda)
\section{Plataforma}
\begin{itemize}
\item {Grp. gram.:f.}
\end{itemize}
\begin{itemize}
\item {Utilização:Pop.}
\end{itemize}
\begin{itemize}
\item {Proveniência:(Fr. \textunderscore plate\textunderscore  + \textunderscore forme\textunderscore )}
\end{itemize}
Espécie de terraço; eirado.
Construcção de terra ou madeira, para assentar a artilharia.
Estrado, na parte posterior ou anterior de alguns vehículos.
Vagão raso ou sem bordos.
Tabuleiro circular, que se move em tôrno de um eixo, para deslocar carruagens de caminhos de ferro.
Simulacro; apparência: \textunderscore diz que é honrado, mas aquillo é plataforma\textunderscore .
\section{Platagónio}
\begin{itemize}
\item {Grp. gram.:m.}
\end{itemize}
\begin{itemize}
\item {Proveniência:(Gr. \textunderscore platagonion\textunderscore )}
\end{itemize}
Antigo instrumento grego de percussão.
\section{Platâneas}
\begin{itemize}
\item {Grp. gram.:f. pl.}
\end{itemize}
Família de plantas, que tem por typo o plátano.
(Fem. pl. de \textunderscore platâneo\textunderscore )
\section{Platâneo}
\begin{itemize}
\item {Grp. gram.:adj.}
\end{itemize}
Relativo ou semelhante ao plátano.
\section{Platanista}
\begin{itemize}
\item {Grp. gram.:f.}
\end{itemize}
\begin{itemize}
\item {Proveniência:(Gr. \textunderscore platanistes\textunderscore )}
\end{itemize}
Lugar, sombreado de plátanos, onde a mocidade de Esparta fazia exercícios gymnásticos. Cf. Filinto, XV, 238.
\section{Plátano}
\begin{itemize}
\item {Grp. gram.:m.}
\end{itemize}
\begin{itemize}
\item {Proveniência:(Lat. \textunderscore platanus\textunderscore )}
\end{itemize}
Gênero de árvores, de fôlhas muito largas e longos ramos.
\section{Plátano-bastardo}
\begin{itemize}
\item {Grp. gram.:m.}
\end{itemize}
\begin{itemize}
\item {Utilização:Bot.}
\end{itemize}
Espécie de \textunderscore bordo\textunderscore ^2, (\textunderscore acer pseudo-platanus\textunderscore , Lin.).
\section{Platantera}
\begin{itemize}
\item {Grp. gram.:f.}
\end{itemize}
\begin{itemize}
\item {Proveniência:(Do gr. \textunderscore platus\textunderscore  + \textunderscore anthera\textunderscore )}
\end{itemize}
Gênero de orcquídeas.
\section{Platanthera}
\begin{itemize}
\item {Grp. gram.:f.}
\end{itemize}
\begin{itemize}
\item {Proveniência:(Do gr. \textunderscore platus\textunderscore  + \textunderscore anthera\textunderscore )}
\end{itemize}
Gênero de orchídeas.
\section{Platéa}
\begin{itemize}
\item {Grp. gram.:f.}
\end{itemize}
\begin{itemize}
\item {Proveniência:(Do gr. \textunderscore plateia\textunderscore )}
\end{itemize}
Pavimento de theatro, entre a orchestra ou o palco e os camarotes.
\section{Plateia}
\begin{itemize}
\item {Grp. gram.:f.}
\end{itemize}
\begin{itemize}
\item {Proveniência:(Do gr. \textunderscore plateia\textunderscore )}
\end{itemize}
Pavimento de teatro, entre a orquestra ou o palco e os camarotes.
\section{Platelminto}
\begin{itemize}
\item {Grp. gram.:m.}
\end{itemize}
\begin{itemize}
\item {Proveniência:(Do gr. \textunderscore platus\textunderscore  + \textunderscore helmins\textunderscore )}
\end{itemize}
Animal hermafrodita, de corpo achatado, constitutivo de uma classe de vermes.
\section{Plathelmintho}
\begin{itemize}
\item {Grp. gram.:m.}
\end{itemize}
\begin{itemize}
\item {Proveniência:(Do gr. \textunderscore platus\textunderscore  + \textunderscore helmins\textunderscore )}
\end{itemize}
Animal hermaphrodita, de corpo achatado, constitutivo de uma classe de vermes.
\section{Platiasmo}
\begin{itemize}
\item {Grp. gram.:m.}
\end{itemize}
\begin{itemize}
\item {Proveniência:(Gr. \textunderscore plateiasmos\textunderscore )}
\end{itemize}
Vício de pronúncia grega, produzido por se abrir demasiadamente a boca.
\section{Platibanda}
\begin{itemize}
\item {Grp. gram.:f.}
\end{itemize}
\begin{itemize}
\item {Proveniência:(Fr. \textunderscore plate-bande\textunderscore )}
\end{itemize}
Moldura chata e unida, mais larga que saliente.
Grade de ferro, ou muro, que rodeia ou limita um terraço, um eirado ou um telhado.
Bordadura dos canteiros de jardim.
\section{Platicapno}
\begin{itemize}
\item {Grp. gram.:m.}
\end{itemize}
\begin{itemize}
\item {Proveniência:(Do gr. \textunderscore platus\textunderscore  + \textunderscore kapnos\textunderscore )}
\end{itemize}
Gênero de plantas papaveráceas.
\section{Platicarpo}
\begin{itemize}
\item {Grp. gram.:m.}
\end{itemize}
\begin{itemize}
\item {Proveniência:(Do gr. \textunderscore platus\textunderscore  + \textunderscore karpos\textunderscore )}
\end{itemize}
Gênero de plantas bignoniáceas.
\section{Platicefalia}
\begin{itemize}
\item {Grp. gram.:f.}
\end{itemize}
Qualidade ou estado de platicéfalo.
\section{Platicéfalo}
\begin{itemize}
\item {Grp. gram.:adj.}
\end{itemize}
\begin{itemize}
\item {Proveniência:(Do gr. \textunderscore platus\textunderscore  + \textunderscore kephale\textunderscore )}
\end{itemize}
Que tem cabeça chata.
Cuja parte superior é achatada.
\section{Platícero}
\begin{itemize}
\item {Grp. gram.:m.}
\end{itemize}
\begin{itemize}
\item {Proveniência:(Do gr. \textunderscore platus\textunderscore  + \textunderscore keras\textunderscore )}
\end{itemize}
Insecto coleópteros pentâmero.
\section{Platicrânia}
\begin{itemize}
\item {Grp. gram.:f.}
\end{itemize}
\begin{itemize}
\item {Proveniência:(Do gr. \textunderscore platus\textunderscore  + \textunderscore kranion\textunderscore )}
\end{itemize}
Gênero de insectos ortópteros.
\section{Platidáctilo}
\begin{itemize}
\item {Grp. gram.:adj.}
\end{itemize}
\begin{itemize}
\item {Utilização:Zool.}
\end{itemize}
\begin{itemize}
\item {Proveniência:(Do gr. \textunderscore platus\textunderscore  + \textunderscore daktulos\textunderscore )}
\end{itemize}
Que tem dedos achatados ou largos.
\section{Platiglosso}
\begin{itemize}
\item {Grp. gram.:adj.}
\end{itemize}
\begin{itemize}
\item {Utilização:Zool.}
\end{itemize}
\begin{itemize}
\item {Proveniência:(Do gr. \textunderscore platus\textunderscore  + \textunderscore glossa\textunderscore )}
\end{itemize}
Que tem língua larga.
\section{Platigónio}
\begin{itemize}
\item {Grp. gram.:m.}
\end{itemize}
Gênero de insectos coleópteros pentâmeros.
\section{Platilóbio}
\begin{itemize}
\item {Grp. gram.:m.}
\end{itemize}
Gênero de plantas leguminosas.
\section{Platilobulado}
\begin{itemize}
\item {Grp. gram.:adj.}
\end{itemize}
\begin{itemize}
\item {Utilização:Bot.}
\end{itemize}
\begin{itemize}
\item {Proveniência:(De \textunderscore platus\textunderscore  gr. + lóbulo)}
\end{itemize}
Que tem lóbulos ou segmentos largos.
\section{Platímero}
\begin{itemize}
\item {Grp. gram.:m.}
\end{itemize}
\begin{itemize}
\item {Proveniência:(Do gr. \textunderscore platus\textunderscore  + \textunderscore meros\textunderscore )}
\end{itemize}
Gênero de insectos coleópteros tetrâmeros.
\section{Platimíscio}
\begin{itemize}
\item {Grp. gram.:m.}
\end{itemize}
Gênero de plantas leguminosas.
\section{Platina}
\begin{itemize}
\item {Grp. gram.:f.}
\end{itemize}
\begin{itemize}
\item {Proveniência:(Do gr. \textunderscore platus\textunderscore )}
\end{itemize}
Peça chata, para vários usos ou instrumentos.
Presilha ou pestana, em que os soldados de infantaria seguram as correias.
Jôgo de navalhas, empregadas nas desfiladoras.
\section{Platina}
\begin{itemize}
\item {Grp. gram.:f.}
\end{itemize}
Metal branco, mais pesado que o oiro e diffícil de fundir.
(Cast. \textunderscore platina\textunderscore )
\section{Platinador}
\begin{itemize}
\item {Grp. gram.:m.}
\end{itemize}
O que platina.
\section{Platinagem}
\begin{itemize}
\item {Grp. gram.:f.}
\end{itemize}
Acto de platinar.
\section{Platinamina}
\begin{itemize}
\item {Grp. gram.:f.}
\end{itemize}
\begin{itemize}
\item {Utilização:Chím.}
\end{itemize}
\begin{itemize}
\item {Proveniência:(De \textunderscore platina\textunderscore  + \textunderscore ammoníaco\textunderscore )}
\end{itemize}
Base ammoniacal, contendo platina.
\section{Platinar}
\begin{itemize}
\item {Grp. gram.:v. t.}
\end{itemize}
\begin{itemize}
\item {Proveniência:(De \textunderscore platina\textunderscore ^2)}
\end{itemize}
Branquear com uma mistura de estanho e mercúrio.
\section{Platinato}
\begin{itemize}
\item {Grp. gram.:m.}
\end{itemize}
\begin{itemize}
\item {Utilização:Chím.}
\end{itemize}
Sal, produzido pela combinação do óxydo platínico com uma base.
\section{Platineuro}
\begin{itemize}
\item {Grp. gram.:adj.}
\end{itemize}
\begin{itemize}
\item {Utilização:Bot.}
\end{itemize}
\begin{itemize}
\item {Proveniência:(Do gr. \textunderscore platus\textunderscore  + \textunderscore neuron\textunderscore )}
\end{itemize}
Que tem nervuras largas.
\section{Platínico}
\begin{itemize}
\item {Grp. gram.:adj.}
\end{itemize}
Diz-se de um óxydo de platina.
\section{Platino}
\begin{itemize}
\item {Grp. gram.:adj.}
\end{itemize}
\begin{itemize}
\item {Grp. gram.:M.}
\end{itemize}
\begin{itemize}
\item {Proveniência:(De \textunderscore Plata\textunderscore , n. p.)}
\end{itemize}
Relativo á República Argentina.
Habitante dêsse Estado.
\section{Platinocianeto}
\begin{itemize}
\item {fónica:nê}
\end{itemize}
\begin{itemize}
\item {Grp. gram.:m.}
\end{itemize}
Cianeto duplo de platina.
\section{Platinocyaneto}
\begin{itemize}
\item {fónica:nê}
\end{itemize}
\begin{itemize}
\item {Grp. gram.:m.}
\end{itemize}
Cyaneto duplo de platina.
\section{Platinópode}
\begin{itemize}
\item {Grp. gram.:m.}
\end{itemize}
\begin{itemize}
\item {Proveniência:(Do gr. \textunderscore platuen\textunderscore  + \textunderscore pous\textunderscore , \textunderscore podos\textunderscore )}
\end{itemize}
Espécie de tôrno, com que se alarga o casco encastelado dos solípedes.
\section{Platinóptera}
\begin{itemize}
\item {Grp. gram.:f.}
\end{itemize}
\begin{itemize}
\item {Proveniência:(Do gr. \textunderscore platus\textunderscore  + \textunderscore pteron\textunderscore )}
\end{itemize}
Gênero de insectos coleópteros tetrâmeros.
\section{Platinoso}
\begin{itemize}
\item {Grp. gram.:adj.}
\end{itemize}
Diz-se de um dos óxydos da platina.
\section{Platinotipia}
\begin{itemize}
\item {Grp. gram.:f.}
\end{itemize}
\begin{itemize}
\item {Proveniência:(De \textunderscore platina\textunderscore  + \textunderscore typo\textunderscore )}
\end{itemize}
Processo de imprimir em chapas de platina.
\section{Platinotypia}
\begin{itemize}
\item {Grp. gram.:f.}
\end{itemize}
\begin{itemize}
\item {Proveniência:(De \textunderscore platina\textunderscore  + \textunderscore typo\textunderscore )}
\end{itemize}
Processo de imprimir em chapas de platina.
\section{Platioftalmo}
\begin{itemize}
\item {Grp. gram.:m.}
\end{itemize}
\begin{itemize}
\item {Proveniência:(Do gr. \textunderscore platus\textunderscore  + \textunderscore opthalmos\textunderscore )}
\end{itemize}
Espécie de pedra preciosa.
\section{Platipalpo}
\begin{itemize}
\item {Grp. gram.:m.}
\end{itemize}
Gênero de insectos dípteros.
\section{Platípede}
\begin{itemize}
\item {Grp. gram.:adj.}
\end{itemize}
\begin{itemize}
\item {Utilização:Bot.}
\end{itemize}
\begin{itemize}
\item {Proveniência:(T. hybr., do gr. \textunderscore platus\textunderscore  + lat. \textunderscore pes\textunderscore )}
\end{itemize}
Que tem o pé ou tronco dilatado na base.
\section{Platípode}
\begin{itemize}
\item {Grp. gram.:adj.}
\end{itemize}
\begin{itemize}
\item {Utilização:Zool.}
\end{itemize}
\begin{itemize}
\item {Grp. gram.:M. Pl.}
\end{itemize}
\begin{itemize}
\item {Proveniência:(Do gr. \textunderscore platus\textunderscore  + \textunderscore pous\textunderscore , \textunderscore podos\textunderscore )}
\end{itemize}
Que tem pés largos.
Divisão ornithológica, proposta por alguns naturalistas.
\section{Platíptero}
\begin{itemize}
\item {Grp. gram.:m.}
\end{itemize}
\begin{itemize}
\item {Proveniência:(Do gr. \textunderscore platus\textunderscore  + \textunderscore pteron\textunderscore )}
\end{itemize}
Gênero de insectos coleópteros pentâmeros.
\section{Platirrostro}
\begin{itemize}
\item {fónica:rós}
\end{itemize}
\begin{itemize}
\item {Grp. gram.:adj.}
\end{itemize}
\begin{itemize}
\item {Utilização:Zool.}
\end{itemize}
\begin{itemize}
\item {Grp. gram.:M. Pl.}
\end{itemize}
\begin{itemize}
\item {Proveniência:(Do gr. \textunderscore platus\textunderscore  + lat. \textunderscore rostrum\textunderscore )}
\end{itemize}
Que tem bico ou focinho largo.
Família de aves, da ordem dos pásseres.
\section{Platirrincos}
\begin{itemize}
\item {Grp. gram.:m. pl.}
\end{itemize}
Gênero de aves pequenas, que compreende duas espécies.
( Do gr. \textunderscore platus\textunderscore  + \textunderscore rhunkos\textunderscore , bico)
\section{Platirrinia}
\begin{itemize}
\item {Grp. gram.:f.}
\end{itemize}
Qualidade ou estado de platirrínico.
\section{Platirrínico}
\begin{itemize}
\item {Grp. gram.:adj.}
\end{itemize}
\begin{itemize}
\item {Proveniência:(Do gr. \textunderscore platus\textunderscore  + \textunderscore rhin\textunderscore )}
\end{itemize}
Que tem nariz achatado.
\section{Platirrínio}
\begin{itemize}
\item {Grp. gram.:adj.}
\end{itemize}
O mesmo que \textunderscore platirrínico\textunderscore .
\section{Platirrino}
\begin{itemize}
\item {Grp. gram.:m.}
\end{itemize}
\begin{itemize}
\item {Proveniência:(Do gr. \textunderscore platus\textunderscore  + \textunderscore rhin\textunderscore )}
\end{itemize}
Tríbo dos macacos americanos, cujas narinas estão muito afastadas uma da outra.
\section{Platispermo}
\begin{itemize}
\item {Grp. gram.:m.}
\end{itemize}
\begin{itemize}
\item {Proveniência:(Do gr. \textunderscore platus\textunderscore  + \textunderscore sperma\textunderscore )}
\end{itemize}
Gênero de plantas coníferas.
\section{Platissema}
\begin{itemize}
\item {fónica:se}
\end{itemize}
\begin{itemize}
\item {Grp. gram.:f.}
\end{itemize}
Gênero de plantas leguminosas.
\section{Platistigma}
\begin{itemize}
\item {Grp. gram.:f.}
\end{itemize}
\begin{itemize}
\item {Proveniência:(Do gr. \textunderscore platus\textunderscore  + \textunderscore stigma\textunderscore )}
\end{itemize}
Gênero de plantas papaveráceas.
\section{Platítomo}
\begin{itemize}
\item {Grp. gram.:m.}
\end{itemize}
Gênero de insectos coleópteros pentâmeros.
\section{Platiúra}
\begin{itemize}
\item {Grp. gram.:f.}
\end{itemize}
Gênero de insectos dípteros.
(Cp. \textunderscore platiúro\textunderscore )
\section{Platiúro}
\begin{itemize}
\item {Grp. gram.:m.}
\end{itemize}
\begin{itemize}
\item {Proveniência:(Do gr. \textunderscore platus\textunderscore  + \textunderscore oura\textunderscore )}
\end{itemize}
Gênero de serpentes venenosas, de cauda chata.
\section{Platónia}
\begin{itemize}
\item {Grp. gram.:f.}
\end{itemize}
\begin{itemize}
\item {Proveniência:(De \textunderscore Platão\textunderscore , n. p.)}
\end{itemize}
Gênero de árvores brasileiras.
\section{Platonicamente}
\begin{itemize}
\item {Grp. gram.:adv.}
\end{itemize}
De modo platónico.
Idealmente; sem mira em gozos materiaes.
\section{Platónico}
\begin{itemize}
\item {Grp. gram.:adj.}
\end{itemize}
\begin{itemize}
\item {Utilização:Ext.}
\end{itemize}
\begin{itemize}
\item {Proveniência:(Lat. \textunderscore platonicus\textunderscore )}
\end{itemize}
Relativo á philosophia de Platão.
Ideal, desligado de interesses ou gozos materiaes; casto: \textunderscore amor platónico\textunderscore .
\section{Platonismo}
\begin{itemize}
\item {Grp. gram.:m.}
\end{itemize}
\begin{itemize}
\item {Utilização:Fig.}
\end{itemize}
Philosophia de Platão.
Qualidade ou carácter do que é platónico.
\section{Platycapno}
\begin{itemize}
\item {Grp. gram.:m.}
\end{itemize}
\begin{itemize}
\item {Proveniência:(Do gr. \textunderscore platus\textunderscore  + \textunderscore kapnos\textunderscore )}
\end{itemize}
Gênero de plantas papaveráceas.
\section{Platycarpo}
\begin{itemize}
\item {Grp. gram.:m.}
\end{itemize}
\begin{itemize}
\item {Proveniência:(Do gr. \textunderscore platus\textunderscore  + \textunderscore karpos\textunderscore )}
\end{itemize}
Gênero de plantas bignoniáceas.
\section{Platycephalia}
\begin{itemize}
\item {Grp. gram.:f.}
\end{itemize}
Qualidade ou estado de platycéphalo.
\section{Platycéphalo}
\begin{itemize}
\item {Grp. gram.:adj.}
\end{itemize}
\begin{itemize}
\item {Proveniência:(Do gr. \textunderscore platus\textunderscore  + \textunderscore kephale\textunderscore )}
\end{itemize}
Que tem cabeça chata.
Cuja parte superior é achatada.
\section{Platýcero}
\begin{itemize}
\item {Grp. gram.:m.}
\end{itemize}
\begin{itemize}
\item {Proveniência:(Do gr. \textunderscore platus\textunderscore  + \textunderscore keras\textunderscore )}
\end{itemize}
Insecto coleópteros pentâmero.
\section{Platycrânia}
\begin{itemize}
\item {Grp. gram.:f.}
\end{itemize}
\begin{itemize}
\item {Proveniência:(Do gr. \textunderscore platus\textunderscore  + \textunderscore kranion\textunderscore )}
\end{itemize}
Gênero de insectos ortópteros.
\section{Platydáctylo}
\begin{itemize}
\item {Grp. gram.:adj.}
\end{itemize}
\begin{itemize}
\item {Utilização:Zool.}
\end{itemize}
\begin{itemize}
\item {Proveniência:(Do gr. \textunderscore platus\textunderscore  + \textunderscore daktulos\textunderscore )}
\end{itemize}
Que tem dedos achatados ou largos.
\section{Platyglosso}
\begin{itemize}
\item {Grp. gram.:adj.}
\end{itemize}
\begin{itemize}
\item {Utilização:Zool.}
\end{itemize}
\begin{itemize}
\item {Proveniência:(Do gr. \textunderscore platus\textunderscore  + \textunderscore glossa\textunderscore )}
\end{itemize}
Que tem língua larga.
\section{Platygónio}
\begin{itemize}
\item {Grp. gram.:m.}
\end{itemize}
Gênero de insectos coleópteros pentâmeros.
\section{Platylóbio}
\begin{itemize}
\item {Grp. gram.:m.}
\end{itemize}
Gênero de plantas leguminosas.
\section{Platylobulado}
\begin{itemize}
\item {Grp. gram.:adj.}
\end{itemize}
\begin{itemize}
\item {Utilização:Bot.}
\end{itemize}
\begin{itemize}
\item {Proveniência:(De \textunderscore platus\textunderscore  gr. + lóbulo)}
\end{itemize}
Que tem lóbulos ou segmentos largos.
\section{Platýmero}
\begin{itemize}
\item {Grp. gram.:m.}
\end{itemize}
\begin{itemize}
\item {Proveniência:(Do gr. \textunderscore platus\textunderscore  + \textunderscore meros\textunderscore )}
\end{itemize}
Gênero de insectos coleópteros tetrâmeros.
\section{Platymíscio}
\begin{itemize}
\item {Grp. gram.:m.}
\end{itemize}
Gênero de plantas leguminosas.
\section{Platyneuro}
\begin{itemize}
\item {Grp. gram.:adj.}
\end{itemize}
\begin{itemize}
\item {Utilização:Bot.}
\end{itemize}
\begin{itemize}
\item {Proveniência:(Do gr. \textunderscore platus\textunderscore  + \textunderscore neuron\textunderscore )}
\end{itemize}
Que tem nervuras largas.
\section{Platynópode}
\begin{itemize}
\item {Grp. gram.:m.}
\end{itemize}
\begin{itemize}
\item {Proveniência:(Do gr. \textunderscore platuen\textunderscore  + \textunderscore pous\textunderscore , \textunderscore podos\textunderscore )}
\end{itemize}
Espécie de tôrno, com que se alarga o casco encastellado dos solípedes.
\section{Platynóptera}
\begin{itemize}
\item {Grp. gram.:f.}
\end{itemize}
\begin{itemize}
\item {Proveniência:(Do gr. \textunderscore platus\textunderscore  + \textunderscore pteron\textunderscore )}
\end{itemize}
Gênero de insectos coleópteros tetrâmeros.
\section{Platyophtalmo}
\begin{itemize}
\item {Grp. gram.:m.}
\end{itemize}
\begin{itemize}
\item {Proveniência:(Do gr. \textunderscore platus\textunderscore  + \textunderscore opthalmos\textunderscore )}
\end{itemize}
Espécie de pedra preciosa.
\section{Platypalpo}
\begin{itemize}
\item {Grp. gram.:m.}
\end{itemize}
Gênero de insectos dípteros.
\section{Platýpede}
\begin{itemize}
\item {Grp. gram.:adj.}
\end{itemize}
\begin{itemize}
\item {Utilização:Bot.}
\end{itemize}
\begin{itemize}
\item {Proveniência:(T. hybr., do gr. \textunderscore platus\textunderscore  + lat. \textunderscore pes\textunderscore )}
\end{itemize}
Que tem o pé ou tronco dilatado na base.
\section{Platýpode}
\begin{itemize}
\item {Grp. gram.:adj.}
\end{itemize}
\begin{itemize}
\item {Utilização:Zool.}
\end{itemize}
\begin{itemize}
\item {Grp. gram.:M. Pl.}
\end{itemize}
\begin{itemize}
\item {Proveniência:(Do gr. \textunderscore platus\textunderscore  + \textunderscore pous\textunderscore , \textunderscore podos\textunderscore )}
\end{itemize}
Que tem pés largos.
Divisão ornithológica, proposta por alguns naturalistas.
\section{Platýptero}
\begin{itemize}
\item {Grp. gram.:m.}
\end{itemize}
\begin{itemize}
\item {Proveniência:(Do gr. \textunderscore platus\textunderscore  + \textunderscore pteron\textunderscore )}
\end{itemize}
Gênero de insectos coleópteros pentâmeros.
\section{Platyrostro}
\begin{itemize}
\item {fónica:rós}
\end{itemize}
\begin{itemize}
\item {Grp. gram.:adj.}
\end{itemize}
\begin{itemize}
\item {Utilização:Zool.}
\end{itemize}
\begin{itemize}
\item {Grp. gram.:M. Pl.}
\end{itemize}
\begin{itemize}
\item {Proveniência:(Do gr. \textunderscore platus\textunderscore  + lat. \textunderscore rostrum\textunderscore )}
\end{itemize}
Que tem bico ou focinho largo.
Família de aves, da ordem dos pásseres.
\section{Platyrrhynchos}
\begin{itemize}
\item {fónica:cos}
\end{itemize}
\begin{itemize}
\item {Grp. gram.:m. pl.}
\end{itemize}
Gênero de aves pequenas, que comprehende duas espécies.
( Do gr. \textunderscore platus\textunderscore  + \textunderscore rhunkos\textunderscore , bico)
\section{Platyrrhinia}
\begin{itemize}
\item {Grp. gram.:f.}
\end{itemize}
Qualidade ou estado de platyrrhínico.
\section{Platyrrhínico}
\begin{itemize}
\item {Grp. gram.:adj.}
\end{itemize}
\begin{itemize}
\item {Proveniência:(Do gr. \textunderscore platus\textunderscore  + \textunderscore rhin\textunderscore )}
\end{itemize}
Que tem nariz achatado.
\section{Platyrrhínio}
\begin{itemize}
\item {Grp. gram.:adj.}
\end{itemize}
O mesmo que \textunderscore platyrrhínico\textunderscore .
\section{Platyrrhino}
\begin{itemize}
\item {Grp. gram.:m.}
\end{itemize}
\begin{itemize}
\item {Proveniência:(Do gr. \textunderscore platus\textunderscore  + \textunderscore rhin\textunderscore )}
\end{itemize}
Tríbo dos macacos americanos, cujas narinas estão muito afastadas uma da outra.
\section{Platysema}
\begin{itemize}
\item {fónica:se}
\end{itemize}
\begin{itemize}
\item {Grp. gram.:f.}
\end{itemize}
Gênero de plantas leguminosas.
\section{Platyspermo}
\begin{itemize}
\item {Grp. gram.:m.}
\end{itemize}
\begin{itemize}
\item {Proveniência:(Do gr. \textunderscore platus\textunderscore  + \textunderscore sperma\textunderscore )}
\end{itemize}
Gênero de plantas coníferas.
\section{Platystigma}
\begin{itemize}
\item {Grp. gram.:f.}
\end{itemize}
\begin{itemize}
\item {Proveniência:(Do gr. \textunderscore platus\textunderscore  + \textunderscore stigma\textunderscore )}
\end{itemize}
Gênero de plantas papaveráceas.
\section{Platýtomo}
\begin{itemize}
\item {Grp. gram.:m.}
\end{itemize}
Gênero de insectos coleópteros pentâmeros.
\section{Platyúra}
\begin{itemize}
\item {Grp. gram.:f.}
\end{itemize}
Gênero de insectos dípteros.
(Cp. \textunderscore platyúro\textunderscore )
\section{Platyúro}
\begin{itemize}
\item {Grp. gram.:m.}
\end{itemize}
\begin{itemize}
\item {Proveniência:(Do gr. \textunderscore platus\textunderscore  + \textunderscore oura\textunderscore )}
\end{itemize}
Gênero de serpentes venenosas, de cauda chata.
\section{Plaudir}
\textunderscore v. t.\textunderscore  (e der.)
O mesmo que \textunderscore applaudir\textunderscore , etc. Cf. Filinto, IX, 188.
\section{Plausibilidade}
\begin{itemize}
\item {Grp. gram.:f.}
\end{itemize}
\begin{itemize}
\item {Proveniência:(Do lat. \textunderscore plausibilis\textunderscore )}
\end{itemize}
Qualidade do que é plausível.
\section{Plausibíllimo}
\begin{itemize}
\item {Grp. gram.:adj.}
\end{itemize}
\begin{itemize}
\item {Proveniência:(Do lat. \textunderscore plausibilis\textunderscore )}
\end{itemize}
Muito plausível.
\section{Plausível}
\begin{itemize}
\item {Grp. gram.:adj.}
\end{itemize}
\begin{itemize}
\item {Proveniência:(Lat. \textunderscore plausibilis\textunderscore )}
\end{itemize}
Que merece applauso ou approvação; razoável.
\section{Plausivelmente}
\begin{itemize}
\item {Grp. gram.:adv.}
\end{itemize}
De modo plausível.
\section{Plaustra}
\begin{itemize}
\item {Grp. gram.:f.}
\end{itemize}
\begin{itemize}
\item {Utilização:Gír.}
\end{itemize}
Capa. Cf. Camillo, \textunderscore Doze Casam.\textunderscore  195.
(Or. eslava)
\section{Plaustro}
\begin{itemize}
\item {Grp. gram.:m.}
\end{itemize}
\begin{itemize}
\item {Utilização:Poét.}
\end{itemize}
\begin{itemize}
\item {Proveniência:(Lat. \textunderscore plaustrum\textunderscore )}
\end{itemize}
Carro para transporte de fardos; carro descoberto. Cf. \textunderscore Viriato Trág.\textunderscore , 428.
\section{Plautino}
\begin{itemize}
\item {Grp. gram.:adj.}
\end{itemize}
Pertencente a Plauto; próprio de Plauto. Cf. Castilho, \textunderscore Livr. Clássica\textunderscore , XX.
\section{Plazer}
\begin{itemize}
\item {Grp. gram.:m.}
\end{itemize}
\begin{itemize}
\item {Utilização:Ant.}
\end{itemize}
O mesmo que \textunderscore prazer\textunderscore . Cf. G. Resende, \textunderscore Miscell\textunderscore .
\section{Plebano}
\begin{itemize}
\item {Grp. gram.:adj.}
\end{itemize}
\begin{itemize}
\item {Utilização:P. us.}
\end{itemize}
Relativo á plebe; plebeu.
Que é de inferior condição:«\textunderscore ...um tal fuão, bom grasnador plebano...\textunderscore »Filinto, XII, 46.
\section{Plebe}
\begin{itemize}
\item {Grp. gram.:f.}
\end{itemize}
\begin{itemize}
\item {Proveniência:(Lat. \textunderscore plebs\textunderscore )}
\end{itemize}
Ultima classe do povo, entre os Romanos.
As classes inferiores da sociedade.
Povo; populacho; ralé.
\section{Plebeiamente}
\begin{itemize}
\item {Grp. gram.:adv.}
\end{itemize}
De modo plebeu.
\section{Plebeidade}
\begin{itemize}
\item {fónica:be-i}
\end{itemize}
\begin{itemize}
\item {Grp. gram.:f.}
\end{itemize}
Qualidade do que é plebeu.
Modos, phrases ou palavras, que só usa a plebe.
\section{Plebeísmo}
\begin{itemize}
\item {Grp. gram.:m.}
\end{itemize}
Qualidade do que é plebeu.
Modos, phrases ou palavras, que só usa a plebe.
\section{Plebeu}
\begin{itemize}
\item {Grp. gram.:adj.}
\end{itemize}
\begin{itemize}
\item {Grp. gram.:M.}
\end{itemize}
\begin{itemize}
\item {Proveniência:(Lat. \textunderscore plebeius\textunderscore )}
\end{itemize}
Relativo a plebe.
Homem da plebe.
Gente plebeia, povoléu:«\textunderscore ferve a sala em plebeu\textunderscore ». Filinto, XVI, 220.
\section{Plebiscitário}
\begin{itemize}
\item {Grp. gram.:adj.}
\end{itemize}
Relativo a plebiscito. Cf. Th. Braga, \textunderscore Mod. Ideias\textunderscore , 102.
\section{Plebiscito}
\begin{itemize}
\item {Grp. gram.:m.}
\end{itemize}
\begin{itemize}
\item {Proveniência:(Lat. \textunderscore plebiscitum\textunderscore )}
\end{itemize}
Decreto da plebe romana, convocada por tríbos.
Resolução, submetida á apreciação do povo.
Voto do povo, sôbre uma proposta que lhe é apresentada.
\section{Plecópteros}
\begin{itemize}
\item {Grp. gram.:m. pl.}
\end{itemize}
\begin{itemize}
\item {Utilização:Zool.}
\end{itemize}
\begin{itemize}
\item {Proveniência:(Do gr. \textunderscore plexein\textunderscore  + \textunderscore pteron\textunderscore )}
\end{itemize}
Família de peixes cartilaginosos.
\section{Plectognathos}
\begin{itemize}
\item {Grp. gram.:m. pl.}
\end{itemize}
\begin{itemize}
\item {Utilização:Zool.}
\end{itemize}
\begin{itemize}
\item {Proveniência:(Do gr. \textunderscore plektos\textunderscore  + \textunderscore gnathos\textunderscore )}
\end{itemize}
Ordem de peixes ossosos, que tem a maxilla superior soldada ao crânio.
\section{Plectognatos}
\begin{itemize}
\item {Grp. gram.:m. pl.}
\end{itemize}
\begin{itemize}
\item {Utilização:Zool.}
\end{itemize}
\begin{itemize}
\item {Proveniência:(Do gr. \textunderscore plektos\textunderscore  + \textunderscore gnathos\textunderscore )}
\end{itemize}
Ordem de peixes ossosos, que tem a maxilla superior soldada ao crânio.
\section{Plectógnatos}
\begin{itemize}
\item {Grp. gram.:m. pl.}
\end{itemize}
\begin{itemize}
\item {Utilização:Zool.}
\end{itemize}
\begin{itemize}
\item {Proveniência:(Do gr. \textunderscore plektos\textunderscore  + \textunderscore gnathos\textunderscore )}
\end{itemize}
Ordem de peixes ossosos, que tem a maxilla superior soldada ao crânio.
\section{Plectranthos}
\begin{itemize}
\item {Grp. gram.:m.}
\end{itemize}
\begin{itemize}
\item {Proveniência:(Do gr. \textunderscore plektron\textunderscore  + \textunderscore anthos\textunderscore )}
\end{itemize}
Gênero de plantas labiadas.
\section{Plectro}
\begin{itemize}
\item {Grp. gram.:m.}
\end{itemize}
\begin{itemize}
\item {Utilização:Fig.}
\end{itemize}
\begin{itemize}
\item {Proveniência:(Gr. \textunderscore plektron\textunderscore )}
\end{itemize}
Pequena vara de marfim, com que os antigos faziam vibrar as cordas da lyra.
Inspiração poética; poesia.
\section{Plegária}
\begin{itemize}
\item {Grp. gram.:f.}
\end{itemize}
\begin{itemize}
\item {Utilização:Ant.}
\end{itemize}
\begin{itemize}
\item {Proveniência:(Do lat. \textunderscore precaria\textunderscore )}
\end{itemize}
Súpplica; oração. Cf. \textunderscore Viriato Trág.\textunderscore , 94.
\section{Pléiada}
\begin{itemize}
\item {Grp. gram.:f.}
\end{itemize}
\begin{itemize}
\item {Grp. gram.:Pl.}
\end{itemize}
\begin{itemize}
\item {Proveniência:(Do gr. \textunderscore pleias\textunderscore )}
\end{itemize}
Cada uma das estrêllas da constellação das pléiades.
Reunião de pessôas illustres ou de pessôas de certa classe.
Constellação, vulgarmente conhecida por \textunderscore sete-estrêllo\textunderscore .
\section{Pléiade}
\begin{itemize}
\item {Grp. gram.:f.}
\end{itemize}
\begin{itemize}
\item {Grp. gram.:Pl.}
\end{itemize}
\begin{itemize}
\item {Proveniência:(Do gr. \textunderscore pleias\textunderscore )}
\end{itemize}
Cada uma das estrêllas da constellação das pléiades.
Reunião de pessôas illustres ou de pessôas de certa classe.
Constellação, vulgarmente conhecida por \textunderscore sete-estrêllo\textunderscore .
\section{Pleitar}
\begin{itemize}
\item {Grp. gram.:v. t.  e  i.}
\end{itemize}
O mesmo que \textunderscore pleitear\textunderscore .
\section{Pleiteador}
\begin{itemize}
\item {Grp. gram.:m.  e  adj.}
\end{itemize}
O que pleiteia.
\section{Pleiteante}
\begin{itemize}
\item {Grp. gram.:m. ,  f.  e  adj.}
\end{itemize}
Pessôa, que pleiteia.
\section{Pleitear}
\begin{itemize}
\item {Grp. gram.:v. t.}
\end{itemize}
\begin{itemize}
\item {Grp. gram.:V. i.}
\end{itemize}
Demandar perante os tribunaes.
Litigar.
Contestar.
Discutir.
Defender: \textunderscore pleitear a própria dignidade\textunderscore .
Têr pleito, discutir.
Rivalizar.
\section{Pleito}
\begin{itemize}
\item {Grp. gram.:m.}
\end{itemize}
\begin{itemize}
\item {Utilização:Obsol.}
\end{itemize}
\begin{itemize}
\item {Proveniência:(Do lat. \textunderscore placitum\textunderscore )}
\end{itemize}
Demanda judicial; questão; discussão.
Acôrdo, combinação.
\section{Plenamente}
\begin{itemize}
\item {Grp. gram.:adv.}
\end{itemize}
De modo pleno.
Inteiramente.
\section{Plenariamente}
\begin{itemize}
\item {Grp. gram.:adv.}
\end{itemize}
De modo plenário; completamente; integralmente.
\section{Plenário}
\begin{itemize}
\item {Grp. gram.:adj.}
\end{itemize}
\begin{itemize}
\item {Grp. gram.:M.}
\end{itemize}
\begin{itemize}
\item {Utilização:Bras}
\end{itemize}
\begin{itemize}
\item {Proveniência:(Lat. \textunderscore plenarius\textunderscore )}
\end{itemize}
Pleno, completo.
Tribunal do júry: \textunderscore a questão será julgada pelo plenário\textunderscore .
\section{Plenicórneo}
\begin{itemize}
\item {Grp. gram.:adj.}
\end{itemize}
\begin{itemize}
\item {Proveniência:(Do lat. \textunderscore plenus\textunderscore  + \textunderscore cornu\textunderscore )}
\end{itemize}
Diz-se dos animaes, que têm os cornos cheios.
\section{Plenidão}
\begin{itemize}
\item {Grp. gram.:f.}
\end{itemize}
\begin{itemize}
\item {Proveniência:(Do lat. \textunderscore plenitudo\textunderscore )}
\end{itemize}
Qualidade de pleno; plenitude; abundância:«\textunderscore ...derramar a plenidão de seus pensamentos...\textunderscore »Filinto, XI, 180.
\section{Plenificar}
\begin{itemize}
\item {Grp. gram.:v. t.}
\end{itemize}
\begin{itemize}
\item {Utilização:Neol.}
\end{itemize}
\begin{itemize}
\item {Proveniência:(Do lat. \textunderscore plenus\textunderscore  + \textunderscore facere\textunderscore )}
\end{itemize}
Tornar pleno; preencher. Cf. Alv. Mendes, \textunderscore Discursos\textunderscore , 217.
\section{Plenilúnio}
\begin{itemize}
\item {Grp. gram.:m.}
\end{itemize}
\begin{itemize}
\item {Proveniência:(Lat. \textunderscore plenilunium\textunderscore )}
\end{itemize}
Lua cheia.
\section{Plenipotência}
\begin{itemize}
\item {Grp. gram.:f.}
\end{itemize}
\begin{itemize}
\item {Proveniência:(De \textunderscore pleno\textunderscore  + \textunderscore potência\textunderscore )}
\end{itemize}
Pleno poder.
\section{Plenipotenciário}
\begin{itemize}
\item {Grp. gram.:adj.}
\end{itemize}
\begin{itemize}
\item {Grp. gram.:M.}
\end{itemize}
\begin{itemize}
\item {Proveniência:(De \textunderscore plenipotência\textunderscore )}
\end{itemize}
Que tem plenos poderes.
Enviado de um Govêrno ou de um Soberano, que leva plenos poderes, para quaesquer negociações, junto de outro Govêrno ou Soberano.
\section{Plenirostro}
\begin{itemize}
\item {fónica:rós}
\end{itemize}
\begin{itemize}
\item {Grp. gram.:adj.}
\end{itemize}
\begin{itemize}
\item {Utilização:Zool.}
\end{itemize}
\begin{itemize}
\item {Grp. gram.:M. Pl.}
\end{itemize}
\begin{itemize}
\item {Proveniência:(Do lat. \textunderscore plenus\textunderscore  + \textunderscore rostrum\textunderscore )}
\end{itemize}
Que tem o bico inteiro, isto é, não denteado nem chanfrado.
Família de aves, a que pertence a ave-do-paraíso.
\section{Plenirrostro}
\begin{itemize}
\item {fónica:rós}
\end{itemize}
\begin{itemize}
\item {Grp. gram.:adj.}
\end{itemize}
\begin{itemize}
\item {Utilização:Zool.}
\end{itemize}
\begin{itemize}
\item {Grp. gram.:M. Pl.}
\end{itemize}
\begin{itemize}
\item {Proveniência:(Do lat. \textunderscore plenus\textunderscore  + \textunderscore rostrum\textunderscore )}
\end{itemize}
Que tem o bico inteiro, isto é, não denteado nem chanfrado.
Família de aves, a que pertence a ave-do-paraíso.
\section{Plenismo}
\begin{itemize}
\item {Grp. gram.:m.}
\end{itemize}
\begin{itemize}
\item {Proveniência:(Do lat. \textunderscore plenus\textunderscore )}
\end{itemize}
Systema philosóphico dos que pensam que o Universo é todo occupado pela matéria.
\section{Plenista}
\begin{itemize}
\item {Grp. gram.:m.}
\end{itemize}
\begin{itemize}
\item {Proveniência:(Do lat. \textunderscore plenus\textunderscore )}
\end{itemize}
Sectário do plenismo.
\section{Plenito}
\begin{itemize}
\item {Grp. gram.:m.}
\end{itemize}
Variedade de metal.
\section{Plenitude}
\begin{itemize}
\item {Grp. gram.:f.}
\end{itemize}
Estado do que é pleno; estado completo.
(B. lat. \textunderscore plenitudo\textunderscore )
\section{Pleno}
\begin{itemize}
\item {Grp. gram.:adj.}
\end{itemize}
\begin{itemize}
\item {Proveniência:(Lat. \textunderscore plenus\textunderscore )}
\end{itemize}
Cheio; completo.
Perfeito.
Inteiro.
\section{Pleochroismo}
\begin{itemize}
\item {fónica:cro}
\end{itemize}
\begin{itemize}
\item {Grp. gram.:m.}
\end{itemize}
\begin{itemize}
\item {Utilização:Miner.}
\end{itemize}
\begin{itemize}
\item {Proveniência:(Do gr. \textunderscore pleon\textunderscore  + \textunderscore khrousmos\textunderscore )}
\end{itemize}
Propriedade, que têm certos crystaes, de apresentar uma infinidade de côres, quando observados em differentes direcções.
\section{Pleocroismo}
\begin{itemize}
\item {fónica:cro}
\end{itemize}
\begin{itemize}
\item {Grp. gram.:m.}
\end{itemize}
\begin{itemize}
\item {Utilização:Miner.}
\end{itemize}
\begin{itemize}
\item {Proveniência:(Do gr. \textunderscore pleon\textunderscore  + \textunderscore khrousmos\textunderscore )}
\end{itemize}
Propriedade, que têm certos cristaes, de apresentar uma infinidade de côres, quando observados em diferentes direcções.
\section{Pleomazia}
\begin{itemize}
\item {Grp. gram.:f.}
\end{itemize}
\begin{itemize}
\item {Utilização:Ant.}
\end{itemize}
\begin{itemize}
\item {Proveniência:(Do gr. \textunderscore pleon\textunderscore , numeroso, e \textunderscore mazos\textunderscore , mama)}
\end{itemize}
Multiplicidade de mamas ou mamilos.
\section{Pleomorfismo}
\begin{itemize}
\item {Grp. gram.:m.}
\end{itemize}
\begin{itemize}
\item {Proveniência:(Do gr. \textunderscore pleos\textunderscore  + \textunderscore morphe\textunderscore )}
\end{itemize}
Propriedade, que certas bactérias têm, de mudar de fórma.
\section{Pleomorphismo}
\begin{itemize}
\item {Grp. gram.:m.}
\end{itemize}
\begin{itemize}
\item {Proveniência:(Do gr. \textunderscore pleos\textunderscore  + \textunderscore morphe\textunderscore )}
\end{itemize}
Propriedade, que certas bactérias têm, de mudar de fórma.
\section{Pleonasmo}
\begin{itemize}
\item {Grp. gram.:m.}
\end{itemize}
\begin{itemize}
\item {Proveniência:(Lat. \textunderscore pleonasmus\textunderscore )}
\end{itemize}
Superfluidade de termos, que ás vezes têm emprêgo legítimo, para dar mais fôrça á expressão.
Circunlóquio.
Ambages.
Superfluidade.
\section{Pleonasticamente}
\begin{itemize}
\item {Grp. gram.:adv.}
\end{itemize}
De modo pleonástico.
\section{Pleonástico}
\begin{itemize}
\item {Grp. gram.:adj.}
\end{itemize}
\begin{itemize}
\item {Proveniência:(Gr. \textunderscore pleonastikos\textunderscore )}
\end{itemize}
Em que há pleonasmo: \textunderscore phrase pleonástica\textunderscore .
\section{Pleonasto}
\begin{itemize}
\item {Grp. gram.:m.}
\end{itemize}
\begin{itemize}
\item {Utilização:Miner.}
\end{itemize}
Variedade escura de rubim, que crystalliza em dodecaédro regular.
\section{Pleonectito}
\begin{itemize}
\item {Grp. gram.:m.}
\end{itemize}
\begin{itemize}
\item {Utilização:Miner.}
\end{itemize}
\begin{itemize}
\item {Proveniência:(Do gr. \textunderscore pleonektes\textunderscore )}
\end{itemize}
Arsenio-antimoniato de chumbo.
\section{Pleonochroismo}
\begin{itemize}
\item {fónica:cro}
\end{itemize}
\begin{itemize}
\item {Grp. gram.:m.}
\end{itemize}
O mesmo ou melhor que \textunderscore pleochroísmo\textunderscore .
\section{Pleonocroismo}
\begin{itemize}
\item {Grp. gram.:m.}
\end{itemize}
O mesmo ou melhor que pleocroismo.
\section{Pleorama}
\begin{itemize}
\item {Grp. gram.:m.}
\end{itemize}
\begin{itemize}
\item {Proveniência:(Do gr. \textunderscore pleo\textunderscore , navegar, e \textunderscore orama\textunderscore , vista)}
\end{itemize}
Quadro movediço, que se desenrola aos olhos do espectador, da mesma fórma que as margens de um rio parecem desapparecer ao lado de um barco que vai singrando.
\section{Pleroma}
\begin{itemize}
\item {Grp. gram.:m.}
\end{itemize}
\begin{itemize}
\item {Proveniência:(Gr. \textunderscore pleroma\textunderscore )}
\end{itemize}
Para os Gnósticos, o deus real, o deus vivo.
Na Phýsica antiga, o conjunto de todos os seres.
\section{Plerose}
\begin{itemize}
\item {Grp. gram.:f.}
\end{itemize}
\begin{itemize}
\item {Utilização:Med.}
\end{itemize}
\begin{itemize}
\item {Proveniência:(Gr. \textunderscore plerosis\textunderscore )}
\end{itemize}
Restabelecimento da nutrição, depois de uma doença.
\section{Plerótico}
\begin{itemize}
\item {Grp. gram.:adj.}
\end{itemize}
\begin{itemize}
\item {Utilização:Med.}
\end{itemize}
\begin{itemize}
\item {Proveniência:(Gr. \textunderscore plerotikos\textunderscore )}
\end{itemize}
Que serve para substituir as chagas por tecido novo.
\section{Plésia}
\begin{itemize}
\item {Grp. gram.:f.}
\end{itemize}
\begin{itemize}
\item {Proveniência:(Do gr. \textunderscore plesios\textunderscore )}
\end{itemize}
Gênero de insectos coleópteros heterómeros.
\section{Plesiomorfismo}
\begin{itemize}
\item {Grp. gram.:m.}
\end{itemize}
\begin{itemize}
\item {Utilização:Miner.}
\end{itemize}
\begin{itemize}
\item {Proveniência:(De \textunderscore plesiomorfo\textunderscore )}
\end{itemize}
Qualidade dos mineraes que, sem possuirem constituição atómica semelhante, oferecem analogia de fórma, absolutamente comparável á das substâncias realmente isomorphas.
\section{Plesiomorfo}
\begin{itemize}
\item {Grp. gram.:adj.}
\end{itemize}
\begin{itemize}
\item {Proveniência:(Do gr. \textunderscore plesion\textunderscore  + \textunderscore morphe\textunderscore )}
\end{itemize}
Que tem o carácter do plesiomorfismo.
\section{Plesiomorphismo}
\begin{itemize}
\item {Grp. gram.:m.}
\end{itemize}
\begin{itemize}
\item {Utilização:Miner.}
\end{itemize}
\begin{itemize}
\item {Proveniência:(De \textunderscore plesiomorpho\textunderscore )}
\end{itemize}
Qualidade dos mineraes que, sem possuirem constituição atómica semelhante, offerecem analogia de fórma, absolutamente comparável á das substâncias realmente isomorphas.
\section{Plesiomorpho}
\begin{itemize}
\item {Grp. gram.:adj.}
\end{itemize}
\begin{itemize}
\item {Proveniência:(Do gr. \textunderscore plesion\textunderscore  + \textunderscore morphe\textunderscore )}
\end{itemize}
Que tem o carácter do plesiomorphismo.
\section{Plesiosáurio}
\begin{itemize}
\item {fónica:sau}
\end{itemize}
\begin{itemize}
\item {Grp. gram.:m.}
\end{itemize}
\begin{itemize}
\item {Proveniência:(Do gr. \textunderscore plesion\textunderscore  + \textunderscore sauros\textunderscore )}
\end{itemize}
Enorme reptil da fauna geológica.
\section{Plesiossáurio}
\begin{itemize}
\item {Grp. gram.:m.}
\end{itemize}
\begin{itemize}
\item {Proveniência:(Do gr. \textunderscore plesion\textunderscore  + \textunderscore sauros\textunderscore )}
\end{itemize}
Enorme reptil da fauna geológica.
\section{Plessígrafo}
\begin{itemize}
\item {Grp. gram.:m.}
\end{itemize}
\begin{itemize}
\item {Utilização:Med.}
\end{itemize}
\begin{itemize}
\item {Proveniência:(Do gr. \textunderscore plessein\textunderscore  + \textunderscore graphein\textunderscore )}
\end{itemize}
Instrumento cilíndrico e oco, terminando numa das extremidades que por uma pequena calota que se apoia no corpo, e que tem plana a outra extremidade, que se percute para mover um lápis ou ponteiro, que há dentro do aparelho e que marca um ponto, sempre que há mudança de som, dando assim a configuração do órgão que se quere observar.
\section{Plessígrapho}
\begin{itemize}
\item {Grp. gram.:m.}
\end{itemize}
\begin{itemize}
\item {Utilização:Med.}
\end{itemize}
\begin{itemize}
\item {Proveniência:(Do gr. \textunderscore plessein\textunderscore  + \textunderscore graphein\textunderscore )}
\end{itemize}
Instrumento cylíndrico e oco, terminando numa das extremidades que por uma pequena calota que se apoia no corpo, e que tem plana a outra extremidade, que se percute para mover um lápis ou ponteiro, que há dentro do apparelho e que marca um ponto, sempre que há mudança de som, dando assim a configuração do órgão que se quere observar.
\section{Plessimetria}
\begin{itemize}
\item {Grp. gram.:f.}
\end{itemize}
Emprêgo do plessímetro.
\section{Plessimétrico}
\begin{itemize}
\item {Grp. gram.:adj.}
\end{itemize}
Relativo á plessimetria.
\section{Plessímetro}
\begin{itemize}
\item {Grp. gram.:m.}
\end{itemize}
\begin{itemize}
\item {Proveniência:(Do gr. \textunderscore plessein\textunderscore  + \textunderscore metron\textunderscore )}
\end{itemize}
Instrumento de Medicina, para praticar a percussão mediata.
\section{Plessite}
\begin{itemize}
\item {Grp. gram.:f.}
\end{itemize}
\begin{itemize}
\item {Utilização:Miner.}
\end{itemize}
\begin{itemize}
\item {Proveniência:(T. mal formado, em vez de \textunderscore plethite\textunderscore , do gr. \textunderscore plethein\textunderscore , encher)}
\end{itemize}
Combinação ferruginosa, que entra no ferro meteórico e lhe preenche as cavidades.
\section{Plethóra}
\begin{itemize}
\item {Grp. gram.:f.}
\end{itemize}
\begin{itemize}
\item {Utilização:Fig.}
\end{itemize}
\begin{itemize}
\item {Proveniência:(Gr. \textunderscore plethore\textunderscore )}
\end{itemize}
Excesso de humores ou de sangue no organismo.
Excesso de seiva, que difficulta a florescência e fructificação das plantas.
Indisposição ou mal-estar de quem tem excesso de vida.
\section{Plethorico}
\begin{itemize}
\item {Grp. gram.:adj.}
\end{itemize}
\begin{itemize}
\item {Proveniência:(Gr. \textunderscore plethorikos\textunderscore )}
\end{itemize}
Relativo á plethora.
Que tem plethora.
\section{Plethro}
\begin{itemize}
\item {Grp. gram.:m.}
\end{itemize}
\begin{itemize}
\item {Proveniência:(Lat. \textunderscore plethron\textunderscore )}
\end{itemize}
Antiga medida de comprimento, correspondente a 30 metros.
Antiga medida agrária, correspondente a 9 ares.
\section{Pletóra}
\begin{itemize}
\item {Grp. gram.:f.}
\end{itemize}
\begin{itemize}
\item {Utilização:Fig.}
\end{itemize}
\begin{itemize}
\item {Proveniência:(Gr. \textunderscore plethore\textunderscore )}
\end{itemize}
Excesso de humores ou de sangue no organismo.
Excesso de seiva, que dificulta a florescência e fructificação das plantas.
Indisposição ou mal-estar de quem tem excesso de vida.
\section{Pletorico}
\begin{itemize}
\item {Grp. gram.:adj.}
\end{itemize}
\begin{itemize}
\item {Proveniência:(Gr. \textunderscore plethorikos\textunderscore )}
\end{itemize}
Relativo á pletora.
Que tem pletora.
\section{Pletro}
\begin{itemize}
\item {Grp. gram.:m.}
\end{itemize}
\begin{itemize}
\item {Proveniência:(Lat. \textunderscore plethron\textunderscore )}
\end{itemize}
Antiga medida de comprimento, correspondente a 30 metros.
Antiga medida agrária, correspondente a 9 ares.
\section{Pleura}
\begin{itemize}
\item {Grp. gram.:f.}
\end{itemize}
\begin{itemize}
\item {Utilização:Anat.}
\end{itemize}
\begin{itemize}
\item {Proveniência:(Gr. \textunderscore pleuron\textunderscore )}
\end{itemize}
Nome de duas membranas serosas, cada uma das quaes reveste interiormente um dos lados do peito, reflectindo-se sôbre os pulmões.
\section{Pleural}
\begin{itemize}
\item {Grp. gram.:adj.}
\end{itemize}
Relativo á pleura.
\section{Pleuris}
\begin{itemize}
\item {Grp. gram.:m.}
\end{itemize}
\begin{itemize}
\item {Proveniência:(Do lat. \textunderscore pleurisis\textunderscore )}
\end{itemize}
O mesmo ou melhor que \textunderscore pleurisía\textunderscore .
\section{Pleurisia}
\begin{itemize}
\item {Grp. gram.:f.}
\end{itemize}
\begin{itemize}
\item {Proveniência:(Do lat. \textunderscore pleurisis\textunderscore , sob a infl. do fr. \textunderscore pleurésie\textunderscore )}
\end{itemize}
Inflammação da pleura.
\section{Pleurite}
\begin{itemize}
\item {Grp. gram.:f.}
\end{itemize}
\begin{itemize}
\item {Proveniência:(Lat. \textunderscore pleuritis\textunderscore )}
\end{itemize}
O mesmo que \textunderscore pleurisia\textunderscore .
\section{Pleurítico}
\begin{itemize}
\item {Grp. gram.:adj.}
\end{itemize}
\begin{itemize}
\item {Grp. gram.:M.  e  adj.}
\end{itemize}
\begin{itemize}
\item {Proveniência:(Lat. \textunderscore pleuriticus\textunderscore )}
\end{itemize}
Relativo á pleurisía ou produzido por ella.
O que soffre pleurisía.
\section{Pleurocele}
\begin{itemize}
\item {Grp. gram.:m.}
\end{itemize}
\begin{itemize}
\item {Utilização:Med.}
\end{itemize}
\begin{itemize}
\item {Proveniência:(Do gr. \textunderscore pleura\textunderscore  + \textunderscore kele\textunderscore )}
\end{itemize}
Hérnia do pulmão.
\section{Pleuroclase}
\begin{itemize}
\item {Grp. gram.:f.}
\end{itemize}
\begin{itemize}
\item {Utilização:Miner.}
\end{itemize}
Phosphato de magnésia.
\section{Pleuroclásio}
\begin{itemize}
\item {Grp. gram.:m.}
\end{itemize}
O mesmo que \textunderscore pleuroclase\textunderscore .
\section{Pleurodinia}
\begin{itemize}
\item {Grp. gram.:f.}
\end{itemize}
\begin{itemize}
\item {Proveniência:(Do gr. \textunderscore pleuron\textunderscore  + \textunderscore odune\textunderscore )}
\end{itemize}
Dôr reumática nos músculos intercostaes.
\section{Pleurodínico}
\begin{itemize}
\item {Grp. gram.:adj.}
\end{itemize}
Relativo á pleurodinia.
\section{Pleurodiscal}
\begin{itemize}
\item {Grp. gram.:adj.}
\end{itemize}
\begin{itemize}
\item {Utilização:Bot.}
\end{itemize}
Diz-se da inserção, em que os estames nascem de um ponto da superfície exterior da parte que, segundo Richard, se chama disco.
\section{Pleurodynia}
\begin{itemize}
\item {Grp. gram.:f.}
\end{itemize}
\begin{itemize}
\item {Proveniência:(Do gr. \textunderscore pleuron\textunderscore  + \textunderscore odune\textunderscore )}
\end{itemize}
Dôr rheumática nos músculos intercostaes.
\section{Pleurodýnico}
\begin{itemize}
\item {Grp. gram.:adj.}
\end{itemize}
Relativo á pleurodynia.
\section{Pleuroma}
\begin{itemize}
\item {Grp. gram.:m.}
\end{itemize}
\begin{itemize}
\item {Utilização:Bot.}
\end{itemize}
\begin{itemize}
\item {Proveniência:(Gr. \textunderscore pleuroma\textunderscore )}
\end{itemize}
Uma das três partes, que resultam da primeira differenciação do meristema primitivo e corresponde ao cylindro central do eixo da planta, sendo as outras duas, successivamente, o periblema e o dermatogênio.
\section{Pleuronectos}
\begin{itemize}
\item {Grp. gram.:m. pl.}
\end{itemize}
\begin{itemize}
\item {Utilização:Zool.}
\end{itemize}
\begin{itemize}
\item {Proveniência:(Do gr. \textunderscore pleuron\textunderscore  + \textunderscore nektos\textunderscore )}
\end{itemize}
Gênero de peixes chatos, que nadam sôbre um dos lados do corpo.
\section{Pleuropathia}
\begin{itemize}
\item {Grp. gram.:f.}
\end{itemize}
\begin{itemize}
\item {Utilização:Med.}
\end{itemize}
\begin{itemize}
\item {Proveniência:(Do gr. \textunderscore pleura\textunderscore  + \textunderscore pathos\textunderscore )}
\end{itemize}
Moléstia da pleura, em geral.
\section{Pleuropatia}
\begin{itemize}
\item {Grp. gram.:f.}
\end{itemize}
\begin{itemize}
\item {Utilização:Med.}
\end{itemize}
\begin{itemize}
\item {Proveniência:(Do gr. \textunderscore pleura\textunderscore  + \textunderscore pathos\textunderscore )}
\end{itemize}
Moléstia da pleura, em geral.
\section{Pleuropericardite}
\begin{itemize}
\item {Grp. gram.:f.}
\end{itemize}
\begin{itemize}
\item {Utilização:Med.}
\end{itemize}
Inflammação simultânea da pleura e do pericárdio.
\section{Pleuropiose}
\begin{itemize}
\item {Grp. gram.:f.}
\end{itemize}
\begin{itemize}
\item {Utilização:Med.}
\end{itemize}
\begin{itemize}
\item {Proveniência:(Do gr. \textunderscore pleura\textunderscore  + \textunderscore puon\textunderscore )}
\end{itemize}
Producção de pus na pleura.
\section{Pleuropneumonia}
\begin{itemize}
\item {Grp. gram.:f.}
\end{itemize}
\begin{itemize}
\item {Proveniência:(De \textunderscore pleura\textunderscore  + \textunderscore pneumonia\textunderscore )}
\end{itemize}
Inflammação simultânea da pleura e do pulmão.
\section{Pleuropyose}
\begin{itemize}
\item {Grp. gram.:f.}
\end{itemize}
\begin{itemize}
\item {Utilização:Med.}
\end{itemize}
\begin{itemize}
\item {Proveniência:(Do gr. \textunderscore pleura\textunderscore  + \textunderscore puon\textunderscore )}
\end{itemize}
Producção de pus na pleura.
\section{Pleurosomo}
\begin{itemize}
\item {fónica:sô}
\end{itemize}
\begin{itemize}
\item {Grp. gram.:m.}
\end{itemize}
\begin{itemize}
\item {Proveniência:(Do gr. \textunderscore pleuron\textunderscore  + \textunderscore soma\textunderscore )}
\end{itemize}
Monstro, cujo ventre resai lateralmente, estendendo-se até deante do peito.
\section{Pleurospermo}
\begin{itemize}
\item {Grp. gram.:m.}
\end{itemize}
\begin{itemize}
\item {Proveniência:(Do gr. \textunderscore pleuron\textunderscore  + \textunderscore sperma\textunderscore )}
\end{itemize}
Gênero de plantas umbellíferas.
\section{Pleurossomo}
\begin{itemize}
\item {Grp. gram.:m.}
\end{itemize}
\begin{itemize}
\item {Proveniência:(Do gr. \textunderscore pleuron\textunderscore  + \textunderscore soma\textunderscore )}
\end{itemize}
Monstro, cujo ventre resai lateralmente, estendendo-se até deante do peito.
\section{Pleurothótono}
\begin{itemize}
\item {Grp. gram.:m.}
\end{itemize}
\begin{itemize}
\item {Utilização:Med.}
\end{itemize}
\begin{itemize}
\item {Proveniência:(Do gr. \textunderscore pleurothen\textunderscore )}
\end{itemize}
Tétano lateral, em que, por contracção de músculos, o doente tem de inclinar-se para um lado.
\section{Pleurotomia}
\begin{itemize}
\item {Grp. gram.:f.}
\end{itemize}
\begin{itemize}
\item {Utilização:Med.}
\end{itemize}
\begin{itemize}
\item {Proveniência:(Do gr. \textunderscore pleura\textunderscore  + \textunderscore tome\textunderscore )}
\end{itemize}
Operação do empyema.
\section{Plexo}
\begin{itemize}
\item {fónica:cso}
\end{itemize}
\begin{itemize}
\item {Grp. gram.:m.}
\end{itemize}
\begin{itemize}
\item {Utilização:Anat.}
\end{itemize}
\begin{itemize}
\item {Utilização:Fig.}
\end{itemize}
\begin{itemize}
\item {Proveniência:(Lat. \textunderscore plexus\textunderscore )}
\end{itemize}
Entrelaçamento de muitas ramificações de nervos ou de quaesquer vasos sanguíneos.
Encadeamento.
\section{Plica}
\begin{itemize}
\item {Grp. gram.:f.}
\end{itemize}
Pequeno sinal ou linha, que se põe sôbre as letras a que se quere dar accentuação aguda, e que se usa também sôbre letras algébricas.
Sinal de notação musical.
(B. lat. \textunderscore plica\textunderscore )
\section{Plica-palonica}
\begin{itemize}
\item {Grp. gram.:f.}
\end{itemize}
\begin{itemize}
\item {Utilização:Med.}
\end{itemize}
Doença, que ataca os cabellos, o mesmo que \textunderscore trichoma\textunderscore .
\section{Plicar}
\begin{itemize}
\item {Grp. gram.:v. t.}
\end{itemize}
\begin{itemize}
\item {Proveniência:(Lat. \textunderscore plicare\textunderscore )}
\end{itemize}
Pôr plicas em.
\section{Plicativo}
\begin{itemize}
\item {Grp. gram.:adj.}
\end{itemize}
\begin{itemize}
\item {Utilização:Bot.}
\end{itemize}
\begin{itemize}
\item {Proveniência:(Do lat. \textunderscore plicare\textunderscore )}
\end{itemize}
Díz-se da petaleação, em que as peças da corolla se dobram sôbre si, sem ordem apparente.
\section{Plicatura}
\begin{itemize}
\item {Grp. gram.:f.}
\end{itemize}
\begin{itemize}
\item {Proveniência:(Lat. \textunderscore plicatura\textunderscore )}
\end{itemize}
Dobra, prega.
\section{Plidar}
\begin{itemize}
\item {Grp. gram.:v. i.}
\end{itemize}
\begin{itemize}
\item {Utilização:Prov.}
\end{itemize}
\begin{itemize}
\item {Utilização:beir.}
\end{itemize}
Empenhar-se, esforçar-se, lutar a favôr de alguém ou de alguma coisa.
\section{Plínia}
\begin{itemize}
\item {Grp. gram.:f.}
\end{itemize}
\begin{itemize}
\item {Proveniência:(De \textunderscore Plínio\textunderscore , n. p.)}
\end{itemize}
Gênero de plantas, da fam. das myrtáceas.
\section{Plintho}
\begin{itemize}
\item {Grp. gram.:m.}
\end{itemize}
\begin{itemize}
\item {Proveniência:(Lat. \textunderscore plinthus\textunderscore )}
\end{itemize}
Peça quadrada, que serve de base a um pedestal ou columna.
Sócco ou pedestal de estátua.
\section{Plinto}
\begin{itemize}
\item {Grp. gram.:m.}
\end{itemize}
\begin{itemize}
\item {Proveniência:(Lat. \textunderscore plinthus\textunderscore )}
\end{itemize}
Peça quadrada, que serve de base a um pedestal ou columna.
Sócco ou pedestal de estátua.
\section{Pliocênico}
\begin{itemize}
\item {Grp. gram.:adj.}
\end{itemize}
O mesmo que \textunderscore plioceno\textunderscore .
\section{Plioceno}
\begin{itemize}
\item {Grp. gram.:adj.}
\end{itemize}
\begin{itemize}
\item {Utilização:Geol.}
\end{itemize}
\begin{itemize}
\item {Proveniência:(Do gr. \textunderscore pleion\textunderscore  + \textunderscore kainos\textunderscore )}
\end{itemize}
Diz-se do terreno terciário, em que se contêm fósseis recentes.
\section{Plistocênico}
\begin{itemize}
\item {Grp. gram.:adj.}
\end{itemize}
O mesmo que \textunderscore plistoceno\textunderscore .
\section{Plistoceno}
\begin{itemize}
\item {Grp. gram.:adj.}
\end{itemize}
\begin{itemize}
\item {Utilização:Geol.}
\end{itemize}
\begin{itemize}
\item {Proveniência:(Do gr. \textunderscore pleistos\textunderscore , muitíssimo, e \textunderscore kainos\textunderscore , novo)}
\end{itemize}
Diz-se de um dos terrenos ou de um dos períodos do systema malacênico, na série terciária, também conhecido por \textunderscore diluvial\textunderscore .
\section{Plocama}
\begin{itemize}
\item {Grp. gram.:f.}
\end{itemize}
\begin{itemize}
\item {Proveniência:(Do gr. \textunderscore plokamos\textunderscore )}
\end{itemize}
Gênero de plantas rubiáceas.
\section{Plocandra}
\begin{itemize}
\item {Grp. gram.:f.}
\end{itemize}
Gênero de plantas gencianáceas.
\section{Plócia}
\begin{itemize}
\item {Grp. gram.:f.}
\end{itemize}
Gênero de insectos coleópteros longicórneos.
\section{Ploeiro}
\begin{itemize}
\item {Grp. gram.:m.}
\end{itemize}
\begin{itemize}
\item {Utilização:Ant.}
\end{itemize}
O mesmo que \textunderscore arraes\textunderscore .
\section{Plogófora}
\begin{itemize}
\item {Grp. gram.:f.}
\end{itemize}
Insecto lepidóptero, (\textunderscore plogophora meticulosa\textunderscore , Lin.), cuja larva ataca a vegetação, perfurando as fôlhas e determinando a quéda destas e a morte da planta.
\section{Plogóphora}
\begin{itemize}
\item {Grp. gram.:f.}
\end{itemize}
Insecto lepidóptero, (\textunderscore plogophora meticulosa\textunderscore , Lin.), cuja larva ataca a vegetação, perfurando as fôlhas e determinando a quéda destas e a morte da planta.
\section{Plomo}
\begin{itemize}
\item {Grp. gram.:m.}
\end{itemize}
\begin{itemize}
\item {Utilização:Ant.}
\end{itemize}
\begin{itemize}
\item {Proveniência:(Do lat. \textunderscore plumbum\textunderscore )}
\end{itemize}
O mesmo que \textunderscore chumbo\textunderscore .
\section{Pluma}
\begin{itemize}
\item {Grp. gram.:f.}
\end{itemize}
\begin{itemize}
\item {Proveniência:(Lat. \textunderscore pluma\textunderscore )}
\end{itemize}
Penna de ave, especialmente a que é destinada a adornar chapéus, etc.
Pennacho.
Flâmmula.
Nome de vários cabos náuticos.
Penna de escrever. Cf. Filinto, I, 26.
\section{Plumaceiro}
\begin{itemize}
\item {Grp. gram.:m.}
\end{itemize}
Aquelle que prepara ou vende plumas.
O que faz plumaços.
\section{Plumacho}
\begin{itemize}
\item {Grp. gram.:m.}
\end{itemize}
\begin{itemize}
\item {Proveniência:(De \textunderscore pluma\textunderscore )}
\end{itemize}
Travesseiro, cheio de pennas.
\section{Plumaço}
\begin{itemize}
\item {Grp. gram.:m.}
\end{itemize}
\begin{itemize}
\item {Utilização:Ant.}
\end{itemize}
\begin{itemize}
\item {Proveniência:(De \textunderscore pluma\textunderscore )}
\end{itemize}
Travesseiro, cheio de pennas.
\section{Plumadada}
\begin{itemize}
\item {Grp. gram.:f.}
\end{itemize}
\begin{itemize}
\item {Proveniência:(De \textunderscore pluma\textunderscore )}
\end{itemize}
Barbas de pennas, que os tratadores de falcões davam a estas aves, misturando os fios com migalhas de carne. Cf. Fernandes, \textunderscore Caça de Altan.\textunderscore 
\section{Plumagem}
\begin{itemize}
\item {Grp. gram.:f.}
\end{itemize}
\begin{itemize}
\item {Proveniência:(De \textunderscore pluma\textunderscore )}
\end{itemize}
Conjunto das pennas de uma ave.
Pennas para adôrno.
\section{Plumário}
\begin{itemize}
\item {Grp. gram.:m.}
\end{itemize}
\begin{itemize}
\item {Proveniência:(Lat. \textunderscore plumarius\textunderscore )}
\end{itemize}
Bordador que, entre os antigos, representava em telas, por meio de agulha, várias figuras, especialmente aves.
\section{Plumbada}
\begin{itemize}
\item {Grp. gram.:f.}
\end{itemize}
\begin{itemize}
\item {Proveniência:(Do lat. \textunderscore plumbum\textunderscore )}
\end{itemize}
Péla de chumbo, com que os rapazes exercitavam fôrças.
\section{Plumbagina}
\begin{itemize}
\item {Grp. gram.:f.}
\end{itemize}
\begin{itemize}
\item {Proveniência:(Do fr. \textunderscore plombagine\textunderscore )}
\end{itemize}
Substância mineral, escura, de que se fazem lápis.
\section{Plumbagineas}
\begin{itemize}
\item {Grp. gram.:f. pl.}
\end{itemize}
Família de plantas, que tem por typo a plumbago.
(Fem. pl. de \textunderscore plumbagíneo\textunderscore )
\section{Plumbagíneo}
\begin{itemize}
\item {Grp. gram.:adj.}
\end{itemize}
Relativo ou semelhante a plumbago.
\section{Plumbago}
\begin{itemize}
\item {Grp. gram.:f.}
\end{itemize}
\begin{itemize}
\item {Proveniência:(Lat. \textunderscore plumbago\textunderscore )}
\end{itemize}
Nome scientífico de um gênero de plantas, cujas fôlhas têm côr de chumbo.
O mesmo que \textunderscore dentelária\textunderscore .
\section{Plumbaria}
\begin{itemize}
\item {Grp. gram.:f.}
\end{itemize}
\begin{itemize}
\item {Proveniência:(Do lat. \textunderscore plumbum\textunderscore )}
\end{itemize}
Arte de trabalhar em chumbo.
\section{Plumbato}
\begin{itemize}
\item {Grp. gram.:m.}
\end{itemize}
\begin{itemize}
\item {Utilização:Chím.}
\end{itemize}
\begin{itemize}
\item {Proveniência:(De \textunderscore plúmbico\textunderscore )}
\end{itemize}
Sal, produzido pela combinação do ácido plúmbico com uma base.
\section{Plumbear}
\begin{itemize}
\item {Grp. gram.:v. t.}
\end{itemize}
\begin{itemize}
\item {Proveniência:(Do lat. \textunderscore plumbum\textunderscore )}
\end{itemize}
Dar côr ou apparência de chumbo a. Cf. Dom. Vieira, \textunderscore Thes. da Líng.\textunderscore , vb. \textunderscore azerar\textunderscore .
\section{Plúmbeo}
\begin{itemize}
\item {Grp. gram.:adj.}
\end{itemize}
\begin{itemize}
\item {Proveniência:(Lat. \textunderscore plumbeus\textunderscore )}
\end{itemize}
Relativo a chumbo; que tem a côr do chumbo; feito de chumbo.
\section{Plúmbico}
\begin{itemize}
\item {Grp. gram.:adj.}
\end{itemize}
\begin{itemize}
\item {Utilização:Chím.}
\end{itemize}
\begin{itemize}
\item {Proveniência:(Do lat. \textunderscore plumbum\textunderscore )}
\end{itemize}
Relativo ao chumbo.
Diz-se de um dos óxydos do chumbo, e de diversos compostos em que entra o chumbo.
\section{Plumbífero}
\begin{itemize}
\item {Grp. gram.:adj.}
\end{itemize}
\begin{itemize}
\item {Proveniência:(Do lat. \textunderscore plumbum\textunderscore  + \textunderscore ferre\textunderscore )}
\end{itemize}
Que contém chumbo.
\section{Plumbo-argentífero}
\begin{itemize}
\item {Grp. gram.:adj.}
\end{itemize}
\begin{itemize}
\item {Proveniência:(De \textunderscore plumbum\textunderscore  lat. + \textunderscore argentífero\textunderscore )}
\end{itemize}
Que contém chumbo e prata.
\section{Plumbocalcita}
\begin{itemize}
\item {Grp. gram.:f.}
\end{itemize}
O mesmo que \textunderscore plumbocalcito\textunderscore .
\section{Plumbocalcito}
\begin{itemize}
\item {Grp. gram.:f.}
\end{itemize}
\begin{itemize}
\item {Proveniência:(De \textunderscore plumbum\textunderscore  lat. + \textunderscore cálcio\textunderscore )}
\end{itemize}
Carbonato de cal e de chumbo.
\section{Plumboso}
\begin{itemize}
\item {Grp. gram.:adj.}
\end{itemize}
\begin{itemize}
\item {Proveniência:(Lat. \textunderscore plumbosus\textunderscore )}
\end{itemize}
Diz-se de um dos óxydos do chumbo.
Que tem chumbo.
\section{Plumeiro}
\begin{itemize}
\item {Grp. gram.:m.}
\end{itemize}
\begin{itemize}
\item {Proveniência:(De \textunderscore pluma\textunderscore )}
\end{itemize}
O mesmo que \textunderscore pennacho\textunderscore .
\section{Plúmeo}
\begin{itemize}
\item {Grp. gram.:adj.}
\end{itemize}
\begin{itemize}
\item {Utilização:Poét.}
\end{itemize}
\begin{itemize}
\item {Proveniência:(Lat. \textunderscore plumeus\textunderscore )}
\end{itemize}
Relativo a plumas; que tem plumas; emplumado. Cf. G. Junqueiro, \textunderscore Musa em Férias\textunderscore , 196; Castilho, \textunderscore Primavera\textunderscore , 71.
\section{Plumetis}
\begin{itemize}
\item {Grp. gram.:m.}
\end{itemize}
\begin{itemize}
\item {Utilização:Gal}
\end{itemize}
\begin{itemize}
\item {Proveniência:(Fr. \textunderscore plumetis\textunderscore )}
\end{itemize}
Espécie de bordado de algodão pouco torcido, representando em relêvo flôres, fôlhas, escudos, etc.
\section{Plumicollo}
\begin{itemize}
\item {Grp. gram.:adj.}
\end{itemize}
\begin{itemize}
\item {Grp. gram.:M. Pl.}
\end{itemize}
\begin{itemize}
\item {Proveniência:(De \textunderscore pluma\textunderscore  + \textunderscore collo\textunderscore )}
\end{itemize}
Que tem plumas no pescoço.
Aves diurnas de rapina.
\section{Plumicolo}
\begin{itemize}
\item {Grp. gram.:adj.}
\end{itemize}
\begin{itemize}
\item {Grp. gram.:M. Pl.}
\end{itemize}
\begin{itemize}
\item {Proveniência:(De \textunderscore pluma\textunderscore  + \textunderscore colo\textunderscore )}
\end{itemize}
Que tem plumas no pescoço.
Aves diurnas de rapina.
\section{Plumicórneo}
\begin{itemize}
\item {Grp. gram.:adj.}
\end{itemize}
\begin{itemize}
\item {Proveniência:(De \textunderscore pluma\textunderscore  + \textunderscore corno\textunderscore )}
\end{itemize}
Que tem antennas em fórma de corno.
\section{Plumilha}
\begin{itemize}
\item {Grp. gram.:f.}
\end{itemize}
Pequena pluma para enfeite.
Pequeno enfeite, semelhante a uma pluma.
\section{Plumista}
\begin{itemize}
\item {Grp. gram.:m.  e  f.}
\end{itemize}
Pessôa, que faz negócio de plumas.
\section{Plumitivo}
\begin{itemize}
\item {Grp. gram.:m.}
\end{itemize}
\begin{itemize}
\item {Utilização:deprec.}
\end{itemize}
\begin{itemize}
\item {Utilização:Fam.}
\end{itemize}
\begin{itemize}
\item {Proveniência:(Fr. \textunderscore plumitif\textunderscore )}
\end{itemize}
Jornalista; escritor público. Cf. Oliv. Martins, \textunderscore Filhos de D. João I\textunderscore , 178; Ortigão, \textunderscore Praias\textunderscore , 41.
\section{Plumo}
\begin{itemize}
\item {Grp. gram.:m.}
\end{itemize}
\begin{itemize}
\item {Utilização:Ant.}
\end{itemize}
\begin{itemize}
\item {Grp. gram.:Loc. adv.}
\end{itemize}
\begin{itemize}
\item {Proveniência:(Do lat. \textunderscore plumbum\textunderscore )}
\end{itemize}
O mesmo que \textunderscore prumo\textunderscore .
\textunderscore A plumo\textunderscore , a propósito. Cf. \textunderscore Aulegrafia\textunderscore , 108 e 164.
\section{Plumoso}
\begin{itemize}
\item {Grp. gram.:adj.}
\end{itemize}
\begin{itemize}
\item {Proveniência:(Lat. \textunderscore plumosus\textunderscore )}
\end{itemize}
Que tem plumas; ornado de plumas; que tem fórma de pluma.
\section{Plum-pudim}
\begin{itemize}
\item {Grp. gram.:m.}
\end{itemize}
Variedade de pudim, em que entram passas de Corintho, farinha, pão ralado, açúcar, frutos confeitados, leite, rhum, ovos, etc.
(Do ingl.)
\section{Plúmula}
\begin{itemize}
\item {Grp. gram.:f.}
\end{itemize}
\begin{itemize}
\item {Utilização:Bot.}
\end{itemize}
\begin{itemize}
\item {Proveniência:(Lat. \textunderscore plumula\textunderscore )}
\end{itemize}
Parte do embryão vegetal, de que se há de formar a haste.
\section{Plumulária}
\begin{itemize}
\item {Grp. gram.:f.}
\end{itemize}
\begin{itemize}
\item {Proveniência:(De \textunderscore plumula\textunderscore )}
\end{itemize}
Gênero de pólypos, cujos tentáculos têm a apparência das barbas da penna.
\section{Plintérias}
\begin{itemize}
\item {Grp. gram.:f. pl.}
\end{itemize}
\begin{itemize}
\item {Proveniência:(Do gr. \textunderscore pluntheria\textunderscore )}
\end{itemize}
Ceremónia anual, em que os Atenienses lavavam a estátua de Minerva.
\section{Plumuliforme}
\begin{itemize}
\item {Grp. gram.:adj.}
\end{itemize}
\begin{itemize}
\item {Utilização:Bot.}
\end{itemize}
\begin{itemize}
\item {Proveniência:(Do lat. \textunderscore plumula\textunderscore  + \textunderscore forma\textunderscore )}
\end{itemize}
Que tem fórma de pluma ou penna.
\section{Plural}
\begin{itemize}
\item {Grp. gram.:adj.}
\end{itemize}
\begin{itemize}
\item {Utilização:Gram.}
\end{itemize}
\begin{itemize}
\item {Grp. gram.:M.}
\end{itemize}
\begin{itemize}
\item {Proveniência:(Lat. \textunderscore pluralis\textunderscore )}
\end{itemize}
Que indica mais de uma pessôa ou coisa nos nomes e nos verbos.
Flexão de um nome ou verbo, para indicar que êlles se referem a mais de uma coisa ou pessôa.
\section{Pluralidade}
\begin{itemize}
\item {Grp. gram.:f.}
\end{itemize}
\begin{itemize}
\item {Utilização:Gram.}
\end{itemize}
\begin{itemize}
\item {Proveniência:(Lat. \textunderscore pluralitas\textunderscore )}
\end{itemize}
O maior número.
Multiplicidade.
Multidão.
O geral.
Qualidade, attribuida a mais de uma coisa ou pessoa: \textunderscore o problema da pluralidade dos mundos habitáveis\textunderscore .
Carácter de um termo, que está no plural.
\section{Pluralizar}
\begin{itemize}
\item {Grp. gram.:v. t.}
\end{itemize}
Pôr no plural.
Usar no plural.
Aumentar em número, multiplicar. Cf. Galhegos, \textunderscore Virgínidos\textunderscore , III, 65.
\section{Pluri...}
\begin{itemize}
\item {Grp. gram.:pref.}
\end{itemize}
\begin{itemize}
\item {Proveniência:(Do lat. \textunderscore plus\textunderscore , \textunderscore pluris\textunderscore )}
\end{itemize}
(designativo de um número indeterminado, mas, pelo menos em Botânica, menor que o designado pelos suf. \textunderscore multi...\textunderscore  e \textunderscore poly...\textunderscore )
\section{Pluriarticulado}
\begin{itemize}
\item {Grp. gram.:adj.}
\end{itemize}
\begin{itemize}
\item {Proveniência:(De \textunderscore pluri...\textunderscore  + \textunderscore articulado\textunderscore )}
\end{itemize}
Que tem muitas articulações.
\section{Pluricellular}
\begin{itemize}
\item {Grp. gram.:adj.}
\end{itemize}
\begin{itemize}
\item {Utilização:Bot.}
\end{itemize}
\begin{itemize}
\item {Proveniência:(De \textunderscore pluri...\textunderscore  + \textunderscore cellular\textunderscore )}
\end{itemize}
Diz-se do ovário ou do fruto, que contém certo número de cellulas, não muito considerável, mas indeterminado.
\section{Pluricelular}
\begin{itemize}
\item {Grp. gram.:adj.}
\end{itemize}
\begin{itemize}
\item {Utilização:Bot.}
\end{itemize}
\begin{itemize}
\item {Proveniência:(De \textunderscore pluri...\textunderscore  + \textunderscore celular\textunderscore )}
\end{itemize}
Diz-se do ovário ou do fruto, que contém certo número de celulas, não muito considerável, mas indeterminado.
\section{Pluridentado}
\begin{itemize}
\item {Grp. gram.:adj.}
\end{itemize}
\begin{itemize}
\item {Proveniência:(De \textunderscore pluri...\textunderscore  + \textunderscore dentado\textunderscore )}
\end{itemize}
Que tem muitos dentes.
\section{Plurifloro}
\begin{itemize}
\item {Grp. gram.:adj.}
\end{itemize}
\begin{itemize}
\item {Utilização:Bot.}
\end{itemize}
\begin{itemize}
\item {Proveniência:(Do lat. \textunderscore plus\textunderscore  + \textunderscore flos\textunderscore )}
\end{itemize}
Que tem muitas flôres.
\section{Plurigamia}
\begin{itemize}
\item {Grp. gram.:f.}
\end{itemize}
O mesmo que \textunderscore polygamia\textunderscore .
\section{Plurilobulado}
\begin{itemize}
\item {Grp. gram.:adj.}
\end{itemize}
\begin{itemize}
\item {Proveniência:(De \textunderscore pluri...\textunderscore  + \textunderscore lobulado\textunderscore )}
\end{itemize}
Que tem muitos lóbulos.
\section{Plurilocular}
\begin{itemize}
\item {Grp. gram.:adj.}
\end{itemize}
\begin{itemize}
\item {Proveniência:(De \textunderscore pluri...\textunderscore  + \textunderscore locular\textunderscore )}
\end{itemize}
Que tem muitos lóculos.
\section{Pluriovulado}
\begin{itemize}
\item {Grp. gram.:adj.}
\end{itemize}
\begin{itemize}
\item {Proveniência:(De \textunderscore pluri...\textunderscore  + \textunderscore óvulo\textunderscore )}
\end{itemize}
Que tem muitos óvulos.
\section{Pluripartido}
\begin{itemize}
\item {Grp. gram.:adj.}
\end{itemize}
\begin{itemize}
\item {Utilização:Bot.}
\end{itemize}
\begin{itemize}
\item {Proveniência:(De \textunderscore pluri...\textunderscore  + \textunderscore partido\textunderscore )}
\end{itemize}
Diz-se do cálice, em que o número das divisões se não fixa.
\section{Pluripétalo}
\begin{itemize}
\item {Grp. gram.:adj.}
\end{itemize}
O mesmo que \textunderscore polypétalo\textunderscore .
\section{Pluriseriado}
\begin{itemize}
\item {fónica:se}
\end{itemize}
\begin{itemize}
\item {Grp. gram.:adj.}
\end{itemize}
\begin{itemize}
\item {Utilização:Bot.}
\end{itemize}
\begin{itemize}
\item {Proveniência:(De \textunderscore pluri...\textunderscore  + \textunderscore série\textunderscore )}
\end{itemize}
Disposto em muitas séries.
\section{Plurisseriado}
\begin{itemize}
\item {Grp. gram.:adj.}
\end{itemize}
\begin{itemize}
\item {Utilização:Bot.}
\end{itemize}
\begin{itemize}
\item {Proveniência:(De \textunderscore pluri...\textunderscore  + \textunderscore série\textunderscore )}
\end{itemize}
Disposto em muitas séries.
\section{Plurivalve}
\begin{itemize}
\item {Grp. gram.:adj.}
\end{itemize}
O mesmo que \textunderscore multivalve\textunderscore .
\section{Plúsia}
\begin{itemize}
\item {Grp. gram.:f.}
\end{itemize}
\begin{itemize}
\item {Proveniência:(Do gr. \textunderscore plusios\textunderscore )}
\end{itemize}
Gênero de insectos lepidópteros nocturnos.
\section{Plutão}
\begin{itemize}
\item {Grp. gram.:m.}
\end{itemize}
\begin{itemize}
\item {Utilização:Poét.}
\end{itemize}
\begin{itemize}
\item {Proveniência:(De \textunderscore Plutão\textunderscore , n. p.)}
\end{itemize}
O fogo.
\section{Plutarcho}
\begin{itemize}
\item {Grp. gram.:m.}
\end{itemize}
\begin{itemize}
\item {Utilização:Fig.}
\end{itemize}
\begin{itemize}
\item {Proveniência:(De \textunderscore Plutarco\textunderscore , n. p.)}
\end{itemize}
Biógrapho.
Chronista de vidas illustres.
\section{Plutarco}
\begin{itemize}
\item {Grp. gram.:m.}
\end{itemize}
\begin{itemize}
\item {Utilização:Fig.}
\end{itemize}
\begin{itemize}
\item {Proveniência:(De \textunderscore Plutarco\textunderscore , n. p.)}
\end{itemize}
Biógrafo.
Cronista de vidas ilustres.
\section{Plúteo}
\begin{itemize}
\item {Grp. gram.:m.}
\end{itemize}
\begin{itemize}
\item {Proveniência:(Lat. \textunderscore pluteus\textunderscore )}
\end{itemize}
Parede, que fecha o espaço entre duas columnas.
\section{Pluto}
\begin{itemize}
\item {Grp. gram.:m.}
\end{itemize}
\begin{itemize}
\item {Utilização:Poét.}
\end{itemize}
\begin{itemize}
\item {Proveniência:(Do gr. \textunderscore ploutos\textunderscore )}
\end{itemize}
Riqueza; o poder della.
\section{Plutocracia}
\begin{itemize}
\item {Grp. gram.:f.}
\end{itemize}
\begin{itemize}
\item {Proveniência:(Do gr. \textunderscore Ploutos\textunderscore , n. p. + \textunderscore krateia\textunderscore )}
\end{itemize}
Influência do dinheiro; preponderância dos homens ricos.
\section{Plutócrata}
\begin{itemize}
\item {Grp. gram.:m.}
\end{itemize}
Aquelle que exerce plutocracia.
\section{Plutocrático}
\begin{itemize}
\item {Grp. gram.:adj.}
\end{itemize}
Relativo a plutócrata ou á plutocracia.
\section{Plutónico}
\begin{itemize}
\item {Grp. gram.:adj.}
\end{itemize}
\begin{itemize}
\item {Utilização:Geol.}
\end{itemize}
\begin{itemize}
\item {Proveniência:(De \textunderscore Plutão\textunderscore , n. p.)}
\end{itemize}
Diz-se dos terrenos, que têm origem no fogo subterrâneo.
\section{Plutónio}
\begin{itemize}
\item {Grp. gram.:adj.}
\end{itemize}
\begin{itemize}
\item {Proveniência:(Lat. \textunderscore plutonius\textunderscore )}
\end{itemize}
Relativo a Plutão, o fabuloso rei dos infernos.
\section{Plutonismo}
\begin{itemize}
\item {Grp. gram.:m.}
\end{itemize}
\begin{itemize}
\item {Proveniência:(De \textunderscore Plutão\textunderscore , n. p.)}
\end{itemize}
Systema geológico, que attribue a formação da crosta do globo á acção do fogo interior.
\section{Plutonista}
\begin{itemize}
\item {Grp. gram.:m. ,  f.  e  adj.}
\end{itemize}
Pessôa sectária do plutonismo.
\section{Plutonografia}
\begin{itemize}
\item {Proveniência:(Do gr. \textunderscore plouton\textunderscore  + \textunderscore graphein\textunderscore )}
\end{itemize}
Descrição geográfica dos vulcões e terremotos.
\section{Plutonographia}
\begin{itemize}
\item {Proveniência:(Do gr. \textunderscore plouton\textunderscore  + \textunderscore graphein\textunderscore )}
\end{itemize}
Descrição geográfica dos vulcões e terremotos.
\section{Plutonomia}
\begin{itemize}
\item {Grp. gram.:f.}
\end{itemize}
\begin{itemize}
\item {Proveniência:(Do gr. \textunderscore ploutos\textunderscore , riqueza, e \textunderscore nomos\textunderscore , lei)}
\end{itemize}
Tratado á cêrca da riqueza.
O mesmo que Economia Política.
\section{Pluvial}
\begin{itemize}
\item {Grp. gram.:adj.}
\end{itemize}
\begin{itemize}
\item {Grp. gram.:M.}
\end{itemize}
\begin{itemize}
\item {Proveniência:(Lat. \textunderscore pluvialis\textunderscore )}
\end{itemize}
Relatívo á chuva.
Capa de asperges.
\section{Pluviátil}
\begin{itemize}
\item {Grp. gram.:adj.}
\end{itemize}
\begin{itemize}
\item {Proveniência:(Lat. \textunderscore pluviatilis\textunderscore )}
\end{itemize}
O mesmo que \textunderscore pluvial\textunderscore .
\section{Pluvímetro}
\begin{itemize}
\item {Grp. gram.:m.}
\end{itemize}
O mesmo que \textunderscore pluviómetro\textunderscore .
\section{Plúvio}
\begin{itemize}
\item {Grp. gram.:m.}
\end{itemize}
\begin{itemize}
\item {Utilização:Poét.}
\end{itemize}
\begin{itemize}
\item {Proveniência:(Do lat. \textunderscore pluvius\textunderscore )}
\end{itemize}
Céu, carregado de nuvens?:«\textunderscore ...nuvens prenhes de água... e d'entre o escuro pluvio despenham...\textunderscore »Castilho, \textunderscore Geórgicas\textunderscore , 43.
\section{Pluviometria}
\begin{itemize}
\item {Grp. gram.:f.}
\end{itemize}
\begin{itemize}
\item {Utilização:Phýs.}
\end{itemize}
Aplicação do pluviómetro.
\section{Pluviométrico}
\begin{itemize}
\item {Grp. gram.:adj.}
\end{itemize}
Relativo á pluviometria.
\section{Pluviómetro}
\begin{itemize}
\item {Grp. gram.:m.}
\end{itemize}
\begin{itemize}
\item {Proveniência:(Do lat. \textunderscore pluvia\textunderscore  + gr. \textunderscore metron\textunderscore )}
\end{itemize}
Instrumento, com que se avalia a espessura da camada de água que cai, em determinado ponto da terra, durante determinado tempo.
\section{Pluvioscópio}
\begin{itemize}
\item {Grp. gram.:m.}
\end{itemize}
\begin{itemize}
\item {Proveniência:(Do lat. \textunderscore pluvia\textunderscore  + gr. \textunderscore skopein\textunderscore )}
\end{itemize}
Instrumento, para registar a hora, quantidade e duração da chuva, em tempo e lugar determinado.
\section{Pluvioso}
\begin{itemize}
\item {Grp. gram.:adj.}
\end{itemize}
\begin{itemize}
\item {Grp. gram.:M.}
\end{itemize}
\begin{itemize}
\item {Proveniência:(Lat. \textunderscore pluviosus\textunderscore )}
\end{itemize}
Pluvial; chuvoso.
Quinto mês do calendário da primeira república francesa, (20 de Janeiro a 19 de Fevereiro).
\section{Plynthérias}
\begin{itemize}
\item {Grp. gram.:f. pl.}
\end{itemize}
\begin{itemize}
\item {Proveniência:(Do gr. \textunderscore pluntheria\textunderscore )}
\end{itemize}
Ceremónia annual, em que os Athenienses lavavam a estátua de Minerva.
\section{P. M. P.}
Abrev. da loc. \textunderscore por mão própria\textunderscore , us. no sobrescrito de algumas cartas, que não são dirigidas por via postal.
\section{P. N.}
Abrev. de \textunderscore Padre-Nosso\textunderscore , primeiras palavras da oração dominical e título da mesma oração.
\section{Pneodinâmica}
\begin{itemize}
\item {Grp. gram.:f.}
\end{itemize}
\begin{itemize}
\item {Utilização:Phýs.}
\end{itemize}
\begin{itemize}
\item {Proveniência:(Do gr. \textunderscore pneo\textunderscore  + \textunderscore dunamis\textunderscore )}
\end{itemize}
Parte mecânica da respiração.
\section{Pneodynâmica}
\begin{itemize}
\item {Grp. gram.:f.}
\end{itemize}
\begin{itemize}
\item {Utilização:Phýs.}
\end{itemize}
\begin{itemize}
\item {Proveniência:(Do gr. \textunderscore pneo\textunderscore  + \textunderscore dunamis\textunderscore )}
\end{itemize}
Parte mecânica da respiração.
\section{Pneometria}
\begin{itemize}
\item {Grp. gram.:f.}
\end{itemize}
Applicação do pneómetro.
Resultados dessa applicação.
\section{Pneométrico}
\begin{itemize}
\item {Grp. gram.:adj.}
\end{itemize}
Relativo á pneometria.
\section{Pneómetro}
\begin{itemize}
\item {Proveniência:(Do gr. \textunderscore pnein\textunderscore  + \textunderscore metron\textunderscore )}
\end{itemize}
Instrumento, para medir a quantidade de ar que, de cada vez, entra nos pulmões e a que sai delles.
\section{Pneoscópio}
\begin{itemize}
\item {Grp. gram.:m.}
\end{itemize}
\begin{itemize}
\item {Utilização:Med.}
\end{itemize}
\begin{itemize}
\item {Proveniência:(Do gr. \textunderscore pnein\textunderscore  + \textunderscore skopein\textunderscore )}
\end{itemize}
Instrumento, para se examinarem os movimentos de dilatação retracção do thórax, no acto da respiração.
\section{Pneu}
\begin{itemize}
\item {Grp. gram.:m.}
\end{itemize}
\begin{itemize}
\item {Utilização:Zool.}
\end{itemize}
Larva dos crustáceos.
\section{Pneuma}
\begin{itemize}
\item {Grp. gram.:m.}
\end{itemize}
\begin{itemize}
\item {Proveniência:(Lat. \textunderscore pneuma\textunderscore )}
\end{itemize}
Sopro ou espírito aéreo, a que alguns médicos antigos attribuíam a causa da vida e, portanto, das doenças.
Nome, que os estoicos davam a um supposto princípio espiritual, considerado como um dos cinco elementos do universo.
\section{Pneumarthrose}
\begin{itemize}
\item {Grp. gram.:f.}
\end{itemize}
\begin{itemize}
\item {Utilização:Med.}
\end{itemize}
\begin{itemize}
\item {Proveniência:(Do gr. \textunderscore pneuma\textunderscore  + \textunderscore arthron\textunderscore )}
\end{itemize}
Secreção de gases numa cavidade articular.
\section{Pneumartrose}
\begin{itemize}
\item {Grp. gram.:f.}
\end{itemize}
\begin{itemize}
\item {Utilização:Med.}
\end{itemize}
\begin{itemize}
\item {Proveniência:(Do gr. \textunderscore pneuma\textunderscore  + \textunderscore arthron\textunderscore )}
\end{itemize}
Secreção de gases numa cavidade articular.
\section{Pneumática}
\begin{itemize}
\item {Grp. gram.:f.}
\end{itemize}
Sciência, que trata das propriedades phýsicas do ar, e dos outros gases permanentes.
(Fem. de \textunderscore pneumático\textunderscore )
\section{Pneumático}
\begin{itemize}
\item {Grp. gram.:adj.}
\end{itemize}
\begin{itemize}
\item {Grp. gram.:M.}
\end{itemize}
\begin{itemize}
\item {Proveniência:(Lat. \textunderscore pneumaticus\textunderscore )}
\end{itemize}
Relativo ao ar.
Aro de borracha, para revestimento de roda de vehículo: \textunderscore os pneumáticos de uma bicycleta\textunderscore .
\section{Pneumatista}
\begin{itemize}
\item {Grp. gram.:m.}
\end{itemize}
Cada um dos membros da seita médica de Athenas, que considerava o pneuma como causa da vida e das doenças.
\section{Pneumato}
\begin{itemize}
\item {Grp. gram.:m.}
\end{itemize}
\begin{itemize}
\item {Proveniência:(De \textunderscore pnêumico\textunderscore )}
\end{itemize}
Sal, produzido pela combinação do ácido pnêumico com uma base.
\section{Pneumatocele}
\begin{itemize}
\item {Grp. gram.:m.}
\end{itemize}
\begin{itemize}
\item {Utilização:Med.}
\end{itemize}
\begin{itemize}
\item {Proveniência:(Do gr. \textunderscore pneuma\textunderscore  + \textunderscore kele\textunderscore )}
\end{itemize}
Tumor gasoso; emphysema.
\section{Pneumatochímica}
\begin{itemize}
\item {fónica:qui}
\end{itemize}
\begin{itemize}
\item {Grp. gram.:f.}
\end{itemize}
\begin{itemize}
\item {Proveniência:(De \textunderscore pneuma\textunderscore  + \textunderscore Chímica\textunderscore )}
\end{itemize}
Parte da Chímica, que trata dos gases.
\section{Pneumatofonia}
\begin{itemize}
\item {Grp. gram.:f.}
\end{itemize}
\begin{itemize}
\item {Utilização:Espir.}
\end{itemize}
\begin{itemize}
\item {Proveniência:(Do gr. \textunderscore pneuma\textunderscore  + \textunderscore phone\textunderscore )}
\end{itemize}
Voz dos espíritos; comunicação oral dos espíritos, sem auxilio da voz humana.
\section{Pneumatografia}
\begin{itemize}
\item {Grp. gram.:f.}
\end{itemize}
\begin{itemize}
\item {Utilização:Espir.}
\end{itemize}
\begin{itemize}
\item {Proveniência:(Do gr. \textunderscore pneuma\textunderscore  + \textunderscore graphein\textunderscore )}
\end{itemize}
Escrita directa dos espíritos, sem o concurso da mão do médium. Cf. Lachatre, \textunderscore Diction. Univ.\textunderscore 
\section{Pneumatográfico}
\begin{itemize}
\item {Grp. gram.:adj.}
\end{itemize}
Relativo á pneumatografia.
\section{Pneumatographia}
\begin{itemize}
\item {Grp. gram.:f.}
\end{itemize}
\begin{itemize}
\item {Utilização:Espir.}
\end{itemize}
\begin{itemize}
\item {Proveniência:(Do gr. \textunderscore pneuma\textunderscore  + \textunderscore graphein\textunderscore )}
\end{itemize}
Escrita directa dos espíritos, sem o concurso da mão do médium. Cf. Lachatre, \textunderscore Diction. Univ.\textunderscore 
\section{Pneumatográphico}
\begin{itemize}
\item {Grp. gram.:adj.}
\end{itemize}
Relativo á pneumatographia.
\section{Pneumatologia}
\begin{itemize}
\item {Grp. gram.:f.}
\end{itemize}
\begin{itemize}
\item {Proveniência:(De \textunderscore pneumatólogo\textunderscore )}
\end{itemize}
Sciência ou tratado dos espíritos, dos seres intermediários que formam a ligação entre Deus e o homem.
\section{Pneumatológico}
\begin{itemize}
\item {Grp. gram.:adj.}
\end{itemize}
Relativo á pneumatologia.
\section{Pneumatologista}
\begin{itemize}
\item {Grp. gram.:m.}
\end{itemize}
\begin{itemize}
\item {Proveniência:(Do gr. \textunderscore pneuma\textunderscore  + \textunderscore logos\textunderscore )}
\end{itemize}
Aquelle que trata de pneumatologia.
\section{Pneumatólogo}
\begin{itemize}
\item {Grp. gram.:m.}
\end{itemize}
\begin{itemize}
\item {Proveniência:(Do gr. \textunderscore pneuma\textunderscore  + \textunderscore logos\textunderscore )}
\end{itemize}
Aquelle que trata de pneumatologia.
\section{Pneumatómetro}
\begin{itemize}
\item {Grp. gram.:m.}
\end{itemize}
\begin{itemize}
\item {Proveniência:(Do gr. \textunderscore pneuma\textunderscore  + \textunderscore metron\textunderscore )}
\end{itemize}
Instrumento, para medir a quantidade do ar inspirado e expirado.
\section{Pneumatômphalo}
\begin{itemize}
\item {Grp. gram.:m.}
\end{itemize}
\begin{itemize}
\item {Proveniência:(Do gr. \textunderscore pneuma\textunderscore  + \textunderscore omphalos\textunderscore )}
\end{itemize}
Hérnia umbilical, distendida por gases.
\section{Pneumatônfalo}
\begin{itemize}
\item {Grp. gram.:m.}
\end{itemize}
\begin{itemize}
\item {Proveniência:(Do gr. \textunderscore pneuma\textunderscore  + \textunderscore omphalos\textunderscore )}
\end{itemize}
Hérnia umbilical, distendida por gases.
\section{Pneumatophonia}
\begin{itemize}
\item {Grp. gram.:f.}
\end{itemize}
\begin{itemize}
\item {Utilização:Espir.}
\end{itemize}
\begin{itemize}
\item {Proveniência:(Do gr. \textunderscore pneuma\textunderscore  + \textunderscore phone\textunderscore )}
\end{itemize}
Voz dos espíritos; communicação oral dos espíritos, sem auxilio da voz humana.
\section{Pneumatoquímica}
\begin{itemize}
\item {Grp. gram.:f.}
\end{itemize}
\begin{itemize}
\item {Proveniência:(De \textunderscore pneuma\textunderscore  + \textunderscore Química\textunderscore )}
\end{itemize}
Parte da Química, que trata dos gases.
\section{Pneumatose}
\begin{itemize}
\item {Grp. gram.:f.}
\end{itemize}
\begin{itemize}
\item {Proveniência:(Do gr. \textunderscore pneuma\textunderscore )}
\end{itemize}
Tumor, formado por accumulação de ar.
\section{Pneumectomia}
\begin{itemize}
\item {Grp. gram.:f.}
\end{itemize}
\begin{itemize}
\item {Utilização:Med.}
\end{itemize}
\begin{itemize}
\item {Proveniência:(Do gr. \textunderscore pneumon\textunderscore  + \textunderscore ektome\textunderscore )}
\end{itemize}
Resecção de uma parte ou da totalidade do pulmão.
\section{Pnêumico}
\begin{itemize}
\item {Grp. gram.:adj.}
\end{itemize}
\begin{itemize}
\item {Proveniência:(De \textunderscore pneuma\textunderscore )}
\end{itemize}
Diz-se de um ácido orgânico, encontrado nos pulmões.
\section{Pneumobrânchio}
\begin{itemize}
\item {fónica:qui}
\end{itemize}
\begin{itemize}
\item {Grp. gram.:adj.}
\end{itemize}
\begin{itemize}
\item {Proveniência:(Do gr. \textunderscore pneumon\textunderscore  + \textunderscore brankhia\textunderscore )}
\end{itemize}
Diz-se dos peixes, que respíram por guelras e pulmões.
\section{Pneumobrânquio}
\begin{itemize}
\item {Grp. gram.:adj.}
\end{itemize}
\begin{itemize}
\item {Proveniência:(Do gr. \textunderscore pneumon\textunderscore  + \textunderscore brankhia\textunderscore )}
\end{itemize}
Diz-se dos peixes, que respíram por guelras e pulmões.
\section{Pneumocele}
\begin{itemize}
\item {Grp. gram.:m.}
\end{itemize}
\begin{itemize}
\item {Proveniência:(Do gr. \textunderscore pneumon\textunderscore  + \textunderscore kele\textunderscore )}
\end{itemize}
Hérnia, produzida pela saída de uma parte do pulmão, através dos espaços intercostaes.
\section{Pneumococcia}
\begin{itemize}
\item {Grp. gram.:f.}
\end{itemize}
\begin{itemize}
\item {Utilização:Med.}
\end{itemize}
\begin{itemize}
\item {Proveniência:(De \textunderscore pneumococco\textunderscore )}
\end{itemize}
Doença complexa, uma de cujas modalidades é a pneumonia.
\section{Pneumococco}
\begin{itemize}
\item {Grp. gram.:m.}
\end{itemize}
\begin{itemize}
\item {Utilização:Med.}
\end{itemize}
Micrococco, que se acha nos productos pneumónicos.
\section{Pneumococia}
\begin{itemize}
\item {Grp. gram.:f.}
\end{itemize}
\begin{itemize}
\item {Utilização:Med.}
\end{itemize}
\begin{itemize}
\item {Proveniência:(De \textunderscore pneumococo\textunderscore )}
\end{itemize}
Doença complexa, uma de cujas modalidades é a pneumonia.
\section{Pneumococo}
\begin{itemize}
\item {Grp. gram.:m.}
\end{itemize}
\begin{itemize}
\item {Utilização:Med.}
\end{itemize}
Micrococo, que se acha nos productos pneumónicos.
\section{Pneumoconiose}
\begin{itemize}
\item {Grp. gram.:f.}
\end{itemize}
\begin{itemize}
\item {Utilização:Med.}
\end{itemize}
\begin{itemize}
\item {Proveniência:(Do gr. \textunderscore pneumon\textunderscore  + \textunderscore konis\textunderscore )}
\end{itemize}
Pneumonia chrónica, produzida por inhalação de poeiras, carvão, sílica, etc.
\section{Pneumodermo}
\begin{itemize}
\item {Grp. gram.:m.}
\end{itemize}
\begin{itemize}
\item {Proveniência:(Do gr. \textunderscore pneuma\textunderscore  + \textunderscore derma\textunderscore )}
\end{itemize}
Gênero de molluscos gasterópodes.
\section{Pneumogástrico}
\begin{itemize}
\item {Grp. gram.:adj.}
\end{itemize}
\begin{itemize}
\item {Proveniência:(Do gr. \textunderscore pneuma\textunderscore  + \textunderscore gaster\textunderscore )}
\end{itemize}
Diz-se de um nervo, que se distribue principalmente nos pulmões e no estômago.
\section{Pneumografia}
\begin{itemize}
\item {Grp. gram.:f.}
\end{itemize}
\begin{itemize}
\item {Proveniência:(Do gr. \textunderscore pneumon\textunderscore  + \textunderscore graphein\textunderscore )}
\end{itemize}
Descripção dos pulmões.
\section{Pneumográfico}
\begin{itemize}
\item {Grp. gram.:adj.}
\end{itemize}
Relativo á pneumographia.
\section{Pneumographia}
\begin{itemize}
\item {Grp. gram.:f.}
\end{itemize}
\begin{itemize}
\item {Proveniência:(Do gr. \textunderscore pneumon\textunderscore  + \textunderscore graphein\textunderscore )}
\end{itemize}
Descripção dos pulmões.
\section{Pneumográphico}
\begin{itemize}
\item {Grp. gram.:adj.}
\end{itemize}
Relativo á pneumographia.
\section{Pneumolithíase}
\begin{itemize}
\item {Grp. gram.:f.}
\end{itemize}
\begin{itemize}
\item {Utilização:Med.}
\end{itemize}
\begin{itemize}
\item {Proveniência:(Do gr. \textunderscore pneumon\textunderscore  + \textunderscore lithos\textunderscore )}
\end{itemize}
Doença, caracterizada pela formação de concreções nos pulmões.
\section{Pneumolitíase}
\begin{itemize}
\item {Grp. gram.:f.}
\end{itemize}
\begin{itemize}
\item {Utilização:Med.}
\end{itemize}
\begin{itemize}
\item {Proveniência:(Do gr. \textunderscore pneumon\textunderscore  + \textunderscore lithos\textunderscore )}
\end{itemize}
Doença, caracterizada pela formação de concreções nos pulmões.
\section{Pneumologia}
\begin{itemize}
\item {Grp. gram.:f.}
\end{itemize}
\begin{itemize}
\item {Proveniência:(Do gr. \textunderscore pneumon\textunderscore  + \textunderscore logos\textunderscore )}
\end{itemize}
Tratado á cêrca dos pulmões.
\section{Pneumológico}
\begin{itemize}
\item {Grp. gram.:adj.}
\end{itemize}
Relativo á pneumologia.
\section{Pneumonalgia}
\begin{itemize}
\item {Grp. gram.:f.}
\end{itemize}
\begin{itemize}
\item {Proveniência:(Do gr. \textunderscore pneumon\textunderscore  + \textunderscore algos\textunderscore )}
\end{itemize}
Dôr no pulmão.
\section{Pneumonálgico}
\begin{itemize}
\item {Grp. gram.:adj.}
\end{itemize}
Relativo á pneumonalgia.
\section{Pneumonia}
\begin{itemize}
\item {Grp. gram.:f.}
\end{itemize}
\begin{itemize}
\item {Proveniência:(Gr. \textunderscore pneumonia\textunderscore )}
\end{itemize}
Inflammação do parênchyma do pulmão.
\section{Pneumónico}
\begin{itemize}
\item {Grp. gram.:adj.}
\end{itemize}
\begin{itemize}
\item {Grp. gram.:M.  e  adj.}
\end{itemize}
\begin{itemize}
\item {Utilização:Ext.}
\end{itemize}
Relativo á pneumonia.
O que soffre pneumonia.
O que tem doença de peito.
\section{Pneumonite}
\begin{itemize}
\item {Grp. gram.:f.}
\end{itemize}
O mesmo que \textunderscore pneumonia\textunderscore .
\section{Pneumopathia}
\begin{itemize}
\item {Grp. gram.:f.}
\end{itemize}
\begin{itemize}
\item {Proveniência:(Do gr. \textunderscore pneumon\textunderscore  + \textunderscore pathos\textunderscore )}
\end{itemize}
Qualquer doença pulmonar.
\section{Pneumopatia}
\begin{itemize}
\item {Grp. gram.:f.}
\end{itemize}
\begin{itemize}
\item {Proveniência:(Do gr. \textunderscore pneumon\textunderscore  + \textunderscore pathos\textunderscore )}
\end{itemize}
Qualquer doença pulmonar.
\section{Pneumopericárdio}
\begin{itemize}
\item {Grp. gram.:m.}
\end{itemize}
Pneumatose no pericárdio. Cf. Macedo Pinto, \textunderscore Comp. de Veter.\textunderscore , I, 107.
\section{Pneumopléctico}
\begin{itemize}
\item {Grp. gram.:adj.}
\end{itemize}
Relativo á pneumoplegia.
\section{Pneumoplegia}
\begin{itemize}
\item {Grp. gram.:f.}
\end{itemize}
\begin{itemize}
\item {Proveniência:(Do gr. \textunderscore pneumon\textunderscore  + \textunderscore plegein\textunderscore )}
\end{itemize}
Paralysia do pulmão.
\section{Pneumopleurisía}
\begin{itemize}
\item {Grp. gram.:f.}
\end{itemize}
\begin{itemize}
\item {Proveniência:(De \textunderscore pneumon\textunderscore  gr. + \textunderscore pleurisia\textunderscore )}
\end{itemize}
Inflammação da pleura e do pulmão.
\section{Pneumopleurítico}
\begin{itemize}
\item {Grp. gram.:adj.}
\end{itemize}
\begin{itemize}
\item {Proveniência:(De \textunderscore pneumon\textunderscore  gr. + \textunderscore pleuritico\textunderscore )}
\end{itemize}
Relativo á pneumopleurisía.
\section{Pneumorragia}
\begin{itemize}
\item {Grp. gram.:f.}
\end{itemize}
\begin{itemize}
\item {Proveniência:(Do gr. \textunderscore pneumon\textunderscore  + \textunderscore rhagein\textunderscore )}
\end{itemize}
Hemorragia pulmonar.
\section{Pneumorrágico}
\begin{itemize}
\item {Grp. gram.:adj.}
\end{itemize}
Relativo á pneumorragia.
\section{Pneumorrhagia}
\begin{itemize}
\item {Grp. gram.:f.}
\end{itemize}
\begin{itemize}
\item {Proveniência:(Do gr. \textunderscore pneumon\textunderscore  + \textunderscore rhagein\textunderscore )}
\end{itemize}
Hemorrhagia pulmonar.
\section{Pneumorrhágico}
\begin{itemize}
\item {Grp. gram.:adj.}
\end{itemize}
Relativo á pneumorrhagia.
\section{Pneumothórax}
\begin{itemize}
\item {Grp. gram.:m.}
\end{itemize}
Pneumatose, no interior das pleuras. Cf. Macedo Pinto, \textunderscore Comp. de Veter.\textunderscore , I, 107.
\section{Pneumotomia}
\begin{itemize}
\item {Grp. gram.:f.}
\end{itemize}
\begin{itemize}
\item {Proveniência:(Do gr. \textunderscore pneumon\textunderscore  + \textunderscore tome\textunderscore )}
\end{itemize}
Dissecção do pulmão.
\section{Pneumotómico}
\begin{itemize}
\item {Grp. gram.:adj.}
\end{itemize}
Relativo á pneumotomia.
\section{Pneumotórax}
\begin{itemize}
\item {Grp. gram.:m.}
\end{itemize}
Pneumatose, no interior das pleuras. Cf. Macedo Pinto, \textunderscore Comp. de Veter.\textunderscore , I, 107.
\section{Pó}
\begin{itemize}
\item {Grp. gram.:m.}
\end{itemize}
\begin{itemize}
\item {Utilização:Fig.}
\end{itemize}
\begin{itemize}
\item {Utilização:Bras}
\end{itemize}
\begin{itemize}
\item {Grp. gram.:Pl.}
\end{itemize}
\begin{itemize}
\item {Utilização:Ant.}
\end{itemize}
\begin{itemize}
\item {Proveniência:(Do lat. hyp. \textunderscore pulvum\textunderscore  &lt; cast. \textunderscore polvo\textunderscore  &lt; port. ant. \textunderscore poo\textunderscore  = port. mod. \textunderscore pó\textunderscore )}
\end{itemize}
Porção de tenuíssimas partículas de terra sêca, que cobrem o solo ou se elevam no ar.
Poeira.
Partículas tenuíssimas de toda a espécie, que se depositam nos aposentos e nos móveis, e que se levantam á menor agitação.
Qualquer substância reduzida a partículas tenuíssimas.
Polvilho.
Coisa insignificante, passageira.
Espécie de esturrinho.
\textunderscore Pó de polir\textunderscore , o mesmo que \textunderscore tripoli\textunderscore .
Especiarias, adubos, temperos.
\section{Pôa}
\begin{itemize}
\item {Grp. gram.:f.}
\end{itemize}
\begin{itemize}
\item {Proveniência:(Do gr. \textunderscore poa\textunderscore , relva)}
\end{itemize}
Gênero de plantas gramíneas, a que pertence a relva vulgar.
\section{Pôa}
\begin{itemize}
\item {Grp. gram.:f.}
\end{itemize}
Um dos cabos náuticos, cujos extremos estão fixos.
\section{Poaçu}
\begin{itemize}
\item {Grp. gram.:m.}
\end{itemize}
\begin{itemize}
\item {Utilização:Bras. do N}
\end{itemize}
Espécie de tecido de algodão, em algumas tríbos do Amazonas.
\section{Poaia}
\begin{itemize}
\item {Grp. gram.:f.}
\end{itemize}
Nome de várias plantas eméticas e rubiáceas do Brasil.
\section{Poalha}
\begin{itemize}
\item {Grp. gram.:f.}
\end{itemize}
\begin{itemize}
\item {Proveniência:(De \textunderscore pó\textunderscore )}
\end{itemize}
Poeira leve na atmosphera.
\section{Poalho}
\begin{itemize}
\item {Grp. gram.:m.}
\end{itemize}
\begin{itemize}
\item {Utilização:Náut.}
\end{itemize}
Nevoeiro pouco denso, que cerra o horizonte.
Chuva miúda e passageira.
(Cp. \textunderscore poalha\textunderscore )
\section{Pobla}
\begin{itemize}
\item {Grp. gram.:f.}
\end{itemize}
\begin{itemize}
\item {Utilização:Des.}
\end{itemize}
O mesmo que \textunderscore póvoa\textunderscore .
(Cp. cast. \textunderscore puebla\textunderscore )
\section{Poblança}
\begin{itemize}
\item {Grp. gram.:f.}
\end{itemize}
\begin{itemize}
\item {Utilização:Ant.}
\end{itemize}
O mesmo que \textunderscore pobla\textunderscore .
\section{Pobra}
\begin{itemize}
\item {Grp. gram.:f.}
\end{itemize}
\begin{itemize}
\item {Utilização:Ant.}
\end{itemize}
O mesmo que \textunderscore póvoa\textunderscore . Cf. Herculano, \textunderscore Hist. de Port.\textunderscore , IV, 57 e 63.
(Cp. \textunderscore pobla\textunderscore )
\section{Pobradar}
\begin{itemize}
\item {Grp. gram.:v. t.}
\end{itemize}
\begin{itemize}
\item {Utilização:Ant.}
\end{itemize}
O mesmo que \textunderscore pobrar\textunderscore .
\section{Pobrador}
\begin{itemize}
\item {Grp. gram.:m.}
\end{itemize}
\begin{itemize}
\item {Utilização:Ant.}
\end{itemize}
O mesmo que \textunderscore povoador\textunderscore .
Procurador real, que superintendia nos lugares fortes e na população das terras, para que se não despovoassem.
\section{Pobramento}
\begin{itemize}
\item {Grp. gram.:m.}
\end{itemize}
\begin{itemize}
\item {Utilização:Ant.}
\end{itemize}
\begin{itemize}
\item {Proveniência:(De \textunderscore pobrar\textunderscore )}
\end{itemize}
Carta ou foral, que se deu para se povoar uma terra.
\section{Pobrar}
\begin{itemize}
\item {Proveniência:(De \textunderscore pobra\textunderscore )}
\end{itemize}
\textunderscore v. t.\textunderscore  (e der.)
O mesmo que \textunderscore povoar\textunderscore , etc.
\section{Pobre}
\begin{itemize}
\item {Grp. gram.:adj.}
\end{itemize}
\begin{itemize}
\item {Utilização:Ext.}
\end{itemize}
\begin{itemize}
\item {Grp. gram.:M.  e  f.}
\end{itemize}
\begin{itemize}
\item {Proveniência:(Lat. \textunderscore pauper\textunderscore )}
\end{itemize}
Que não tem o que necessita.
Que tem pouco do que lhe é necessário.
Que tem posses inferiores á sua condição social.
Que revela pobreza: \textunderscore vestuário pobre\textunderscore .
Que tem pouco.
Pouco productivo: \textunderscore terrenos pobres\textunderscore .
Digno de lástima: \textunderscore pobre homem\textunderscore !
Pessôa pobre ou que mendiga.
\section{Pobremente}
\begin{itemize}
\item {Grp. gram.:adv.}
\end{itemize}
Com pobreza: \textunderscore viver pobremente\textunderscore .
Á maneira de pobre: \textunderscore trajar pobremente\textunderscore .
\section{Pobretana}
\begin{itemize}
\item {Grp. gram.:m.}
\end{itemize}
\begin{itemize}
\item {Utilização:Prov.}
\end{itemize}
O mesmo que \textunderscore pobretão\textunderscore .
\section{Pobretão}
\begin{itemize}
\item {Grp. gram.:m.}
\end{itemize}
\begin{itemize}
\item {Proveniência:(De \textunderscore pobrete\textunderscore )}
\end{itemize}
Aquelle que é muito pobre.
O que mendiga sem necessidade.
Pobre.
\section{Pobrete}
\begin{itemize}
\item {fónica:brê}
\end{itemize}
\begin{itemize}
\item {Grp. gram.:adj.}
\end{itemize}
\begin{itemize}
\item {Grp. gram.:M.}
\end{itemize}
\begin{itemize}
\item {Proveniência:(De \textunderscore pobre\textunderscore )}
\end{itemize}
Um tanto pobre.
Homem digno de compaixão, mísero. Cf. Filinto, I, 289.
\section{Pobreza}
\begin{itemize}
\item {Grp. gram.:f.}
\end{itemize}
Estado ou qualidade de pobre.
Falta do que é necessário para viver.
Escassez; falta.
Penúria.
A classe dos pobres: \textunderscore as súpplicas da pobreza\textunderscore .
\section{Pobrinho}
\begin{itemize}
\item {Grp. gram.:m.  e  adj.}
\end{itemize}
(Dem. de \textunderscore pobre\textunderscore , empregado por Camillo, \textunderscore Ôlho de Vidro\textunderscore , 15)
\section{Pobura}
\begin{itemize}
\item {Grp. gram.:f.}
\end{itemize}
O mesmo que \textunderscore angelim\textunderscore .
\section{Pôça}
\begin{itemize}
\item {Grp. gram.:f.}
\end{itemize}
\begin{itemize}
\item {Utilização:Prov.}
\end{itemize}
\begin{itemize}
\item {Utilização:beir.}
\end{itemize}
\begin{itemize}
\item {Grp. gram.:Interj.}
\end{itemize}
\begin{itemize}
\item {Proveniência:(De \textunderscore pôço\textunderscore )}
\end{itemize}
(para alguns, póça)
Cova natural e pouco funda, com água.
Cova artificial e pouco funda, em que se represa a água nascente, com que se rega milho, hortas, etc., fazendo-a saír por um bueiro.
Irra! vai-te!
\section{Pocachím}
\begin{itemize}
\item {Grp. gram.:m.}
\end{itemize}
\begin{itemize}
\item {Utilização:Gír.}
\end{itemize}
O mesmo que \textunderscore bocanhim\textunderscore .
\section{Pocaçu}
\begin{itemize}
\item {Grp. gram.:f.}
\end{itemize}
\begin{itemize}
\item {Utilização:Bras}
\end{itemize}
Espécie de pombo.
\section{Poçada}
\begin{itemize}
\item {Grp. gram.:f.}
\end{itemize}
Porção de água, que uma poça póde conter.
\section{Poçal}
\begin{itemize}
\item {Grp. gram.:m.}
\end{itemize}
Antiga medida castelhana e portuguesa, para líquidos, correspondente a 5 almudes.
(Cast. \textunderscore pozal\textunderscore )
\section{Pocamó}
\begin{itemize}
\item {Grp. gram.:m.}
\end{itemize}
\begin{itemize}
\item {Utilização:Bras}
\end{itemize}
Peixe de água doce.
\section{Poçanangara}
\begin{itemize}
\item {Grp. gram.:m.}
\end{itemize}
\begin{itemize}
\item {Utilização:Bras. de Minas}
\end{itemize}
O mesmo que \textunderscore curandeiro\textunderscore .
\section{Poção}
\begin{itemize}
\item {Grp. gram.:f.}
\end{itemize}
\begin{itemize}
\item {Proveniência:(Lat. \textunderscore potio\textunderscore )}
\end{itemize}
Medicamento líquido, para se beber; bebida.
\section{Poceca}
\begin{itemize}
\item {Grp. gram.:f.}
\end{itemize}
\begin{itemize}
\item {Utilização:Prov.}
\end{itemize}
\begin{itemize}
\item {Utilização:trasm.}
\end{itemize}
Poça pequena.
\section{Poceira}
\begin{itemize}
\item {Grp. gram.:f.}
\end{itemize}
\begin{itemize}
\item {Utilização:T. da Bairrada}
\end{itemize}
Poça grande, com águas pluviaes.
Charco.
\section{Poceiro}
\begin{itemize}
\item {Grp. gram.:m.}
\end{itemize}
\begin{itemize}
\item {Proveniência:(De \textunderscore poço\textunderscore )}
\end{itemize}
Cesto, em que se lava lan.
Grande cesto de vime; cabano.
Indivíduo, que faz poças.
\section{Pocema}
\begin{itemize}
\item {Grp. gram.:f.}
\end{itemize}
\begin{itemize}
\item {Utilização:Bras}
\end{itemize}
Gritaria, algazarra.
\section{Pócha}
\begin{itemize}
\item {Grp. gram.:f.}
\end{itemize}
\begin{itemize}
\item {Utilização:Prov.}
\end{itemize}
\begin{itemize}
\item {Utilização:trasm.}
\end{itemize}
\begin{itemize}
\item {Utilização:minh.}
\end{itemize}
Moínha ou pequenas escamas brancas, que adherem ao grão de milho, ainda depois de tirado do carolo.
\section{Pôcha}
\begin{itemize}
\item {Grp. gram.:f.}
\end{itemize}
\begin{itemize}
\item {Utilização:Prov.}
\end{itemize}
\begin{itemize}
\item {Utilização:trasm.}
\end{itemize}
\begin{itemize}
\item {Utilização:minh.}
\end{itemize}
Moínha ou pequenas escamas brancas, que adherem ao grão de milho, ainda depois de tirado do carolo.
\section{Pôcha}
\begin{itemize}
\item {Grp. gram.:f.}
\end{itemize}
\begin{itemize}
\item {Utilização:Prov.}
\end{itemize}
\begin{itemize}
\item {Proveniência:(De \textunderscore pôcho\textunderscore )}
\end{itemize}
Cadellinha, cachorra.
\section{Poche!}
\begin{itemize}
\item {fónica:pô}
\end{itemize}
\begin{itemize}
\item {Grp. gram.:interj.}
\end{itemize}
\begin{itemize}
\item {Utilização:Prov.}
\end{itemize}
Voz, para chamar ou afagar cãesinhos.
(Cp. \textunderscore pocho\textunderscore )
\section{Pochetis}
\begin{itemize}
\item {Grp. gram.:m. pl.}
\end{itemize}
Tríbo de Índios tupinambás, entre o Araguaia e o Tocantins.
\section{Pocho}
\begin{itemize}
\item {fónica:pô}
\end{itemize}
\begin{itemize}
\item {Grp. gram.:adj.}
\end{itemize}
\begin{itemize}
\item {Utilização:Prov.}
\end{itemize}
\begin{itemize}
\item {Utilização:trasm.}
\end{itemize}
\begin{itemize}
\item {Grp. gram.:M.}
\end{itemize}
\begin{itemize}
\item {Utilização:Prov.}
\end{itemize}
\begin{itemize}
\item {Grp. gram.:Interj.}
\end{itemize}
Gordo, de gordura balofa e doentia.
Cãozinho, cachorro.
O mesmo que \textunderscore poche!\textunderscore 
\section{Pocilga}
\begin{itemize}
\item {Grp. gram.:f.}
\end{itemize}
\begin{itemize}
\item {Proveniência:(Do lat. hyp. \textunderscore porcilica\textunderscore , de \textunderscore porcile\textunderscore )}
\end{itemize}
Curral de porcos.
Casa immunda ou miserável.
\section{Pocilgo}
\begin{itemize}
\item {Grp. gram.:m.}
\end{itemize}
Alojamento de porco ou de porcos.
(Cp. \textunderscore pocilga\textunderscore )
\section{Pocilhão}
\begin{itemize}
\item {Grp. gram.:m.}
\end{itemize}
Grande pocilga.
\section{Pocima}
\begin{itemize}
\item {Grp. gram.:adv.}
\end{itemize}
\begin{itemize}
\item {Utilização:Ant.}
\end{itemize}
\begin{itemize}
\item {Proveniência:(De \textunderscore por\textunderscore  + \textunderscore cima\textunderscore )}
\end{itemize}
Finalmente.
Além de tudo; afinal de contas.
\section{Pocinheira}
\begin{itemize}
\item {Grp. gram.:f.}
\end{itemize}
\begin{itemize}
\item {Utilização:Prov.}
\end{itemize}
\begin{itemize}
\item {Utilização:trasm.}
\end{itemize}
\begin{itemize}
\item {Proveniência:(De \textunderscore poça\textunderscore )}
\end{itemize}
Pau, com maçaneta numa extremidade, para represar água numa poça, tapando o bueiro ou ôlho.
Pedra, que serve para o mesmo fim.
\section{Poço}
\begin{itemize}
\item {fónica:pô}
\end{itemize}
\begin{itemize}
\item {Grp. gram.:m.}
\end{itemize}
\begin{itemize}
\item {Utilização:Ext.}
\end{itemize}
\begin{itemize}
\item {Utilização:T. de Serpa}
\end{itemize}
\begin{itemize}
\item {Grp. gram.:Interj.}
\end{itemize}
\begin{itemize}
\item {Utilização:Prov.}
\end{itemize}
\begin{itemize}
\item {Proveniência:(Do lat. \textunderscore puteus\textunderscore )}
\end{itemize}
Cavidade funda na terra, contendo água.
Pégo.
Clarabóia de mina.
Altura de um navio, desde a aresta superior até o convés.
Abysmo.
Aquillo que é profundo.
Utensílio de barro, que se põe sôbre o fogareiro, para suster a panela. Cf. Rev. \textunderscore Tradição\textunderscore , II, 11.
O mesmo que \textunderscore pôça\textunderscore 
\section{Poculiforme}
\begin{itemize}
\item {Grp. gram.:adj.}
\end{itemize}
\begin{itemize}
\item {Utilização:Hist. Nat.}
\end{itemize}
\begin{itemize}
\item {Proveniência:(Do lat. \textunderscore poculum\textunderscore  + \textunderscore forma\textunderscore )}
\end{itemize}
Que tem fórma de copo.
\section{Poda}
\begin{itemize}
\item {Grp. gram.:f.}
\end{itemize}
\begin{itemize}
\item {Grp. gram.:Pl.}
\end{itemize}
\begin{itemize}
\item {Utilização:Prov.}
\end{itemize}
\begin{itemize}
\item {Utilização:minh.}
\end{itemize}
Acto ou effeito de podar.
Córte, deminuição.
Lenha, constituida pelas vides que se cortam na poda.
\section{Pôda}
\begin{itemize}
\item {Grp. gram.:f.}
\end{itemize}
\begin{itemize}
\item {Utilização:Prov.}
\end{itemize}
\begin{itemize}
\item {Utilização:minh.}
\end{itemize}
O mesmo que \textunderscore podôa\textunderscore .
\section{Podada}
\begin{itemize}
\item {Grp. gram.:f.}
\end{itemize}
\begin{itemize}
\item {Utilização:Prov.}
\end{itemize}
\begin{itemize}
\item {Utilização:minh.}
\end{itemize}
Confissão geral, pela Semana Santa.
\section{Podadeira}
\begin{itemize}
\item {Grp. gram.:f.}
\end{itemize}
\begin{itemize}
\item {Proveniência:(De \textunderscore podar\textunderscore )}
\end{itemize}
Foice, com que se poda.
\section{Podador}
\begin{itemize}
\item {Grp. gram.:m.  e  adj.}
\end{itemize}
O que poda.
\section{Podadura}
\begin{itemize}
\item {Grp. gram.:f.}
\end{itemize}
O mesmo que \textunderscore poda\textunderscore .
\section{Podagra}
\begin{itemize}
\item {Grp. gram.:f.}
\end{itemize}
\begin{itemize}
\item {Proveniência:(Gr. \textunderscore podagra\textunderscore )}
\end{itemize}
Doença da gota, nos pés.
\section{Podagrária}
\begin{itemize}
\item {Grp. gram.:f.}
\end{itemize}
Planta medicinal, que se applicava contra a podagra.
\section{Podágrico}
\begin{itemize}
\item {Grp. gram.:adj.}
\end{itemize}
Relativo á podagra.
\section{Podal}
\begin{itemize}
\item {Grp. gram.:adj.}
\end{itemize}
\begin{itemize}
\item {Utilização:Anat.}
\end{itemize}
\begin{itemize}
\item {Proveniência:(Do gr. \textunderscore pous\textunderscore  + \textunderscore podos\textunderscore )}
\end{itemize}
Relativo ao pé.
\section{Podalíria}
\begin{itemize}
\item {Grp. gram.:f.}
\end{itemize}
\begin{itemize}
\item {Proveniência:(De \textunderscore podalírio\textunderscore )}
\end{itemize}
Gênero de plantas leguminosas.
\section{Podalírio}
\begin{itemize}
\item {Grp. gram.:adj.}
\end{itemize}
\begin{itemize}
\item {Proveniência:(De \textunderscore Podalírio\textunderscore , n. p. de um médico myth., filho de Esculápio)}
\end{itemize}
Diz-se especialmente da arte ou sciência, que procura nas plantas o meio de curar doenças. Cf. Camões, ode 8.^a; Latino, \textunderscore Camões\textunderscore , 39.
\section{Podaliro}
\begin{itemize}
\item {Grp. gram.:m.}
\end{itemize}
\begin{itemize}
\item {Proveniência:(Do gr. \textunderscore Podaluros\textunderscore , n. p.)}
\end{itemize}
Espécie de borboleta diurna.
\section{Podalyro}
\begin{itemize}
\item {Grp. gram.:m.}
\end{itemize}
\begin{itemize}
\item {Proveniência:(Do gr. \textunderscore Podaluros\textunderscore , n. p.)}
\end{itemize}
Espécie de borboleta diurna.
\section{Podão}
\begin{itemize}
\item {Grp. gram.:m.}
\end{itemize}
\begin{itemize}
\item {Utilização:Fig.}
\end{itemize}
\begin{itemize}
\item {Proveniência:(De \textunderscore podar\textunderscore )}
\end{itemize}
O mesmo que \textunderscore podadeira\textunderscore .
Pessôa trôpega.
Pessôa, que tem pouco desembaraço.
\section{Podar}
\begin{itemize}
\item {Grp. gram.:v. t.}
\end{itemize}
\begin{itemize}
\item {Utilização:Fig.}
\end{itemize}
\begin{itemize}
\item {Proveniência:(Do lat. \textunderscore putare\textunderscore )}
\end{itemize}
Limpar ou cortar a rama ou os braços inúteis de videiras, árvores, etc.
Desbastar, cortar.
\section{Podargo}
\begin{itemize}
\item {Grp. gram.:m.}
\end{itemize}
\begin{itemize}
\item {Proveniência:(Do gr. \textunderscore pous\textunderscore , \textunderscore podos\textunderscore  + \textunderscore argos\textunderscore )}
\end{itemize}
Gênero de aves nocturnas.
\section{Podarthro}
\begin{itemize}
\item {Grp. gram.:m.}
\end{itemize}
\begin{itemize}
\item {Utilização:Zool.}
\end{itemize}
\begin{itemize}
\item {Proveniência:(Do gr. \textunderscore pous\textunderscore , \textunderscore podos\textunderscore  + \textunderscore arthron\textunderscore )}
\end{itemize}
Articulação do pé das aves com o tarso.
\section{Podartro}
\begin{itemize}
\item {Grp. gram.:m.}
\end{itemize}
\begin{itemize}
\item {Utilização:Zool.}
\end{itemize}
\begin{itemize}
\item {Proveniência:(Do gr. \textunderscore pous\textunderscore , \textunderscore podos\textunderscore  + \textunderscore arthron\textunderscore )}
\end{itemize}
Articulação do pé das aves com o tarso.
\section{Podeira}
\begin{itemize}
\item {Grp. gram.:f.}
\end{itemize}
\begin{itemize}
\item {Utilização:Prov.}
\end{itemize}
\begin{itemize}
\item {Utilização:minh.}
\end{itemize}
\begin{itemize}
\item {Proveniência:(De \textunderscore poda\textunderscore )}
\end{itemize}
Vara, que apparece na lenha chamada podas.
\section{Podencefalia}
\begin{itemize}
\item {Grp. gram.:f.}
\end{itemize}
Estado de podencéfalo.
\section{Podencéfalo}
\begin{itemize}
\item {Grp. gram.:m.}
\end{itemize}
\begin{itemize}
\item {Proveniência:(Do gr. \textunderscore pous\textunderscore , \textunderscore podos\textunderscore  + \textunderscore enkephalon\textunderscore )}
\end{itemize}
Monstro, cujo cérebro, situado fóra do crânio, se apoia num pedúnculo.
\section{Podencephalia}
\begin{itemize}
\item {Grp. gram.:f.}
\end{itemize}
Estado de podencéphalo.
\section{Podencéphalo}
\begin{itemize}
\item {Grp. gram.:m.}
\end{itemize}
\begin{itemize}
\item {Proveniência:(Do gr. \textunderscore pous\textunderscore , \textunderscore podos\textunderscore  + \textunderscore enkephalon\textunderscore )}
\end{itemize}
Monstro, cujo cérebro, situado fóra do crânio, se apoia num pedúnculo.
\section{Podenga}
\begin{itemize}
\item {Grp. gram.:f.}
\end{itemize}
A fêmea do podengo.
\section{Podengo}
\begin{itemize}
\item {Grp. gram.:m.}
\end{itemize}
\begin{itemize}
\item {Proveniência:(Do b. lat. \textunderscore potencus\textunderscore )}
\end{itemize}
Cão, próprio para a caça de coêlhos.
\section{Poder}
\begin{itemize}
\item {Grp. gram.:v. t.}
\end{itemize}
\begin{itemize}
\item {Grp. gram.:V. i.}
\end{itemize}
\begin{itemize}
\item {Grp. gram.:M.}
\end{itemize}
\begin{itemize}
\item {Grp. gram.:Loc.}
\end{itemize}
\begin{itemize}
\item {Utilização:fam.}
\end{itemize}
\begin{itemize}
\item {Proveniência:(Lat. hyp. \textunderscore potere\textunderscore )}
\end{itemize}
Têr a faculdade de: \textunderscore poder condemnar um réu\textunderscore .
Têr possibilidade ou autorização para: \textunderscore poder usar armas\textunderscore .
Estar arriscado ou exposto a: \textunderscore olha que podes escorregar\textunderscore .
Têr occasião de: \textunderscore não pude ontem falar-lhe\textunderscore .
Têr fôrça para: \textunderscore posso erguer 30 quilos\textunderscore .
Têr possibilidade.
Dispor de fôrça ou autoridade.
Faculdade, possibilidade.
Potencia.
Vigor do corpo ou da alma.
Autoridade.
Direito de mandar.
Soberania.
Influência.
Domínio.
Posse.
Govêrno de um Estado.
Fôrças militares.
Efficácia.
Procuração, mandato.
Capacidade.
Meios.
Importância.
Grande quantidade.
\textunderscore O poder do mundo\textunderscore , muita gente.
\section{Podere}
\begin{itemize}
\item {Grp. gram.:m.}
\end{itemize}
\begin{itemize}
\item {Proveniência:(Lat. \textunderscore poderis\textunderscore )}
\end{itemize}
Comprida túnica sacerdotal, entre os antigos, a qual descia até os pés.
Espécie de hábito talar.
\section{Poderio}
\begin{itemize}
\item {Grp. gram.:m.}
\end{itemize}
\begin{itemize}
\item {Utilização:Fig.}
\end{itemize}
\begin{itemize}
\item {Proveniência:(De \textunderscore poder\textunderscore )}
\end{itemize}
Grande poder.
Jurisdicção; autoridade.
Grande porção, chusma:«\textunderscore ...um poderio de setadas\textunderscore ». Filinto, \textunderscore D. Man.\textunderscore , III, 260.
\section{Poderosamente}
\begin{itemize}
\item {Grp. gram.:adv.}
\end{itemize}
De modo poderoso.
\section{Poderoso}
\begin{itemize}
\item {Grp. gram.:adj.}
\end{itemize}
\begin{itemize}
\item {Grp. gram.:M. Pl.}
\end{itemize}
Que tem poder.
Que exerce poderio ou mando.
Que produz grande effeito.
Intenso; enérgico.
Que demove, que influe.
Influente.
Indivíduos com poder ou influência, baseada na riqueza ou posição social.
\section{Pó-de-santana}
\begin{itemize}
\item {Grp. gram.:m.}
\end{itemize}
\begin{itemize}
\item {Utilização:Bras}
\end{itemize}
Planta laurácea, medicinal, (\textunderscore nectandra amara\textunderscore ).
\section{Podestade}
\begin{itemize}
\item {Grp. gram.:f.}
\end{itemize}
\begin{itemize}
\item {Utilização:Ant.}
\end{itemize}
\begin{itemize}
\item {Proveniência:(Do lat. \textunderscore protestas\textunderscore )}
\end{itemize}
Rico-homem, com autoridade suprema em certas comarcas ou provincias.
\section{Pódice}
\begin{itemize}
\item {Grp. gram.:m.}
\end{itemize}
\begin{itemize}
\item {Utilização:Poét.}
\end{itemize}
\begin{itemize}
\item {Proveniência:(Lat. \textunderscore podex\textunderscore , \textunderscore podicis\textunderscore )}
\end{itemize}
O poisadeiro, o ânus. Cf. C. Lobo, \textunderscore Sát. de Juv.\textunderscore , I, 118.
\section{Podicípede}
\begin{itemize}
\item {Grp. gram.:adj.}
\end{itemize}
\begin{itemize}
\item {Proveniência:(Do lat. \textunderscore podex\textunderscore  + \textunderscore pes\textunderscore )}
\end{itemize}
Diz-se de algumas aves, que tem os pés junto ao ânus.
\section{Pódio}
\begin{itemize}
\item {Grp. gram.:m.}
\end{itemize}
\begin{itemize}
\item {Proveniência:(Lat. \textunderscore podium\textunderscore )}
\end{itemize}
Tribuna ou varanda, em que os Imperadores Romanos e as grandes personagens assistiam aos espectáculos.
\section{Podôa}
\begin{itemize}
\item {Grp. gram.:f.}
\end{itemize}
O mesmo que \textunderscore podadeira\textunderscore .
\section{Podobrânchio}
\begin{itemize}
\item {fónica:qui}
\end{itemize}
\begin{itemize}
\item {Grp. gram.:adj.}
\end{itemize}
\begin{itemize}
\item {Utilização:Zool.}
\end{itemize}
\begin{itemize}
\item {Proveniência:(Do gr. \textunderscore pous\textunderscore , \textunderscore podos\textunderscore  + \textunderscore brankhia\textunderscore )}
\end{itemize}
Que tem as brânchias nos pés.
\section{Podobrânquio}
\begin{itemize}
\item {Grp. gram.:adj.}
\end{itemize}
\begin{itemize}
\item {Utilização:Zool.}
\end{itemize}
\begin{itemize}
\item {Proveniência:(Do gr. \textunderscore pous\textunderscore , \textunderscore podos\textunderscore  + \textunderscore brankhia\textunderscore )}
\end{itemize}
Que tem as brânquias nos pés.
\section{Podocarpato}
\begin{itemize}
\item {Grp. gram.:m.}
\end{itemize}
Sal, resultante da combinação do ácido podocárpico com uma base.
\section{Podocárpico}
\begin{itemize}
\item {Grp. gram.:adj.}
\end{itemize}
Diz-se de um ácido extrahido da resina do podocárpio.
\section{Podocárpio}
\begin{itemize}
\item {Grp. gram.:m.}
\end{itemize}
\begin{itemize}
\item {Proveniência:(Do gr. \textunderscore pous\textunderscore , \textunderscore podos\textunderscore  + \textunderscore karpos\textunderscore )}
\end{itemize}
Gênero de árvores coníferas, da tríbo das taxíneas.
\section{Podocarpo}
\begin{itemize}
\item {Grp. gram.:m.}
\end{itemize}
O mesmo que \textunderscore podocárpio\textunderscore .
\section{Podócero}
\begin{itemize}
\item {Grp. gram.:m.}
\end{itemize}
\begin{itemize}
\item {Proveniência:(Do gr. \textunderscore pous\textunderscore , \textunderscore podos\textunderscore  + \textunderscore keras\textunderscore )}
\end{itemize}
Gênero de crustáceos amphípodos.
\section{Pododigital}
\begin{itemize}
\item {Grp. gram.:adj.}
\end{itemize}
\begin{itemize}
\item {Proveniência:(De \textunderscore pous\textunderscore , \textunderscore podos\textunderscore  gr. + \textunderscore digital\textunderscore )}
\end{itemize}
Relativo aos dedos do pé.
\section{Podofalange}
\begin{itemize}
\item {Grp. gram.:f.}
\end{itemize}
\begin{itemize}
\item {Utilização:Neol.}
\end{itemize}
\begin{itemize}
\item {Proveniência:(Do gr. \textunderscore pous\textunderscore , \textunderscore podos\textunderscore  + \textunderscore phalanx\textunderscore )}
\end{itemize}
Falange dos dedos do pé, por distincção de \textunderscore falange\textunderscore , que só se referirá aos dedos da mão. Cf. J. A. Serrano, \textunderscore Osteologia Humana\textunderscore .
\section{Podofalangeta}
\begin{itemize}
\item {fónica:gê}
\end{itemize}
\begin{itemize}
\item {Grp. gram.:f.}
\end{itemize}
Falangeta do pé.
(Cp. \textunderscore podofalange\textunderscore )
\section{Podofalanginha}
\begin{itemize}
\item {Grp. gram.:f.}
\end{itemize}
Falanginha do pé.
(Cp. \textunderscore podofalange\textunderscore )
\section{Podofiláceas}
\begin{itemize}
\item {Grp. gram.:f. pl.}
\end{itemize}
Família de plantas, que tem por tipo o podofilo.
\section{Podofilina}
\begin{itemize}
\item {Grp. gram.:f.}
\end{itemize}
Substância medicinal, extraida do podofilo, e que se aplica como drástico contra a prisão habitual do ventre.
\section{Podofilino}
\begin{itemize}
\item {Grp. gram.:m.}
\end{itemize}
Substância medicinal, extraida do podofilo, e que se aplica como drástico contra a prisão habitual do ventre.
\section{Podofilo}
\begin{itemize}
\item {Grp. gram.:m.}
\end{itemize}
\begin{itemize}
\item {Proveniência:(Do gr. \textunderscore pous\textunderscore , \textunderscore podos\textunderscore  + \textunderscore phullon\textunderscore )}
\end{itemize}
Planta ranunculácea.
\section{Podofiloso}
\begin{itemize}
\item {Grp. gram.:adj.}
\end{itemize}
\begin{itemize}
\item {Proveniência:(De \textunderscore podofilo\textunderscore )}
\end{itemize}
Diz-se do tecido orgânico, que envolve o último osso do pé do cavalo. Cf. Leon, \textunderscore Arte de Ferrar\textunderscore , 32.
\section{Podogínico}
\begin{itemize}
\item {Grp. gram.:adj.}
\end{itemize}
\begin{itemize}
\item {Utilização:Bot.}
\end{itemize}
Diz-se da inserção, quando se realiza com um disco hipógino.
(Cp. \textunderscore podógino\textunderscore )
\section{Podógino}
\begin{itemize}
\item {Grp. gram.:m.}
\end{itemize}
\begin{itemize}
\item {Utilização:Bot.}
\end{itemize}
\begin{itemize}
\item {Grp. gram.:Adj.}
\end{itemize}
\begin{itemize}
\item {Proveniência:(Do gr. \textunderscore pous\textunderscore , \textunderscore podos\textunderscore  + \textunderscore gune\textunderscore )}
\end{itemize}
Pedículo, mais ou menos longo, do ovário de certos vegetaes.
Diz-se do disco, quando formado por corpo carnudo, que é distinto do receptáculo e que eleva o ovário por cima do fruto da flôr.
\section{Podogýnico}
\begin{itemize}
\item {Grp. gram.:adj.}
\end{itemize}
\begin{itemize}
\item {Utilização:Bot.}
\end{itemize}
Diz-se da inserção, quando se realiza com um disco hypógyno.
(Cp. \textunderscore podógyno\textunderscore )
\section{Podógyno}
\begin{itemize}
\item {Grp. gram.:m.}
\end{itemize}
\begin{itemize}
\item {Utilização:Bot.}
\end{itemize}
\begin{itemize}
\item {Grp. gram.:Adj.}
\end{itemize}
\begin{itemize}
\item {Proveniência:(Do gr. \textunderscore pous\textunderscore , \textunderscore podos\textunderscore  + \textunderscore gune\textunderscore )}
\end{itemize}
Pedículo, mais ou menos longo, do ovário de certos vegetaes.
Diz-se do disco, quando formado por corpo carnudo, que é distinto do receptáculo e que eleva o ovário por cima do fruto da flôr.
\section{Podologia}
\begin{itemize}
\item {Grp. gram.:f.}
\end{itemize}
\begin{itemize}
\item {Proveniência:(Do gr. \textunderscore pous\textunderscore , \textunderscore podos\textunderscore  + \textunderscore logos\textunderscore )}
\end{itemize}
Descripção do pé.
\section{Podológico}
\begin{itemize}
\item {Grp. gram.:adj.}
\end{itemize}
Relativo á podologia.
\section{Podométrico}
\begin{itemize}
\item {Grp. gram.:adj.}
\end{itemize}
Relativo ao podómetro.
\section{Podómetro}
\begin{itemize}
\item {Grp. gram.:m.}
\end{itemize}
\begin{itemize}
\item {Proveniência:(Do gr. \textunderscore pous\textunderscore , \textunderscore podos\textunderscore  + \textunderscore metron\textunderscore )}
\end{itemize}
Instrumento, com que se mede o pé, especialmente o pé dos animaes, para se fazerem ferraduras adequadas á medida do respectivo pé.
\section{Podophalange}
\begin{itemize}
\item {Grp. gram.:f.}
\end{itemize}
\begin{itemize}
\item {Utilização:Neol.}
\end{itemize}
\begin{itemize}
\item {Proveniência:(Do gr. \textunderscore pous\textunderscore , \textunderscore podos\textunderscore  + \textunderscore phalanx\textunderscore )}
\end{itemize}
Phalange dos dedos do pé, por distincção de \textunderscore phalange\textunderscore , que só se referirá aos dedos da mão. Cf. J. A. Serrano, \textunderscore Osteologia Humana\textunderscore .
\section{Podophalangeta}
\begin{itemize}
\item {fónica:gê}
\end{itemize}
\begin{itemize}
\item {Grp. gram.:f.}
\end{itemize}
Phalangeta do pé.
(Cp. \textunderscore podophalange\textunderscore )
\section{Podophalanginha}
\begin{itemize}
\item {Grp. gram.:f.}
\end{itemize}
Phalanginha do pé.
(Cp. \textunderscore podophalange\textunderscore )
\section{Podophylláceas}
\begin{itemize}
\item {Grp. gram.:f. pl.}
\end{itemize}
Família de plantas, que tem por typo o podophyllo.
\section{Podophyllina}
\begin{itemize}
\item {Grp. gram.:f.}
\end{itemize}
Substância medicinal, extrahida do podophyllo, e que se applica como drástico contra a prisão habitual do ventre.
\section{Podophyllino}
\begin{itemize}
\item {Grp. gram.:m.}
\end{itemize}
Substância medicinal, extrahida do podophyllo, e que se applica como drástico contra a prisão habitual do ventre.
\section{Podophyllo}
\begin{itemize}
\item {Grp. gram.:m.}
\end{itemize}
\begin{itemize}
\item {Proveniência:(Do gr. \textunderscore pous\textunderscore , \textunderscore podos\textunderscore  + \textunderscore phullon\textunderscore )}
\end{itemize}
Planta ranunculácea.
\section{Podophylloso}
\begin{itemize}
\item {Grp. gram.:adj.}
\end{itemize}
\begin{itemize}
\item {Proveniência:(De \textunderscore podophyllo\textunderscore )}
\end{itemize}
Diz-se do tecido orgânico, que envolve o último osso do pé do cavallo. Cf. Leon, \textunderscore Arte de Ferrar\textunderscore , 32.
\section{Podóptero}
\begin{itemize}
\item {Grp. gram.:adj.}
\end{itemize}
\begin{itemize}
\item {Utilização:Zool.}
\end{itemize}
\begin{itemize}
\item {Proveniência:(Do gr. \textunderscore pous\textunderscore , \textunderscore podos\textunderscore  + \textunderscore pteron\textunderscore )}
\end{itemize}
Que tem os pés espalmados.
\section{Podóscafo}
\begin{itemize}
\item {Grp. gram.:m.}
\end{itemize}
\begin{itemize}
\item {Proveniência:(Do gr. \textunderscore pous\textunderscore , \textunderscore podos\textunderscore --\textunderscore skaphos\textunderscore )}
\end{itemize}
Certo aparelho fluctuante:«\textunderscore Fawler, sentado num podóscafo, composto de dois dentes de comprimento de 6 metros, reunidos por varões de ferro e cuja altura acima do nível das águas é de 30 centimetros, atravessou a Mancha em onze horas e chegou são e salvo a Sandgate.\textunderscore »(Do jornal \textunderscore Primeiro de Janeiro\textunderscore )
\section{Podóscapho}
\begin{itemize}
\item {Grp. gram.:m.}
\end{itemize}
\begin{itemize}
\item {Proveniência:(Do gr. \textunderscore pous\textunderscore , \textunderscore podos\textunderscore --\textunderscore skaphos\textunderscore )}
\end{itemize}
Certo apparelho fluctuante:«\textunderscore Fawler, sentado num podóscapho, composto de dois dentes de comprimento de 6 metros, reunidos por varões de ferro e cuja altura acima do nível das águas é de 30 centimetros, atravessou a Mancha em onze horas e chegou são e salvo a Sandgate.\textunderscore »(Do jornal \textunderscore Primeiro de Janeiro\textunderscore )
\section{Podosperma}
\begin{itemize}
\item {Grp. gram.:m.}
\end{itemize}
\begin{itemize}
\item {Proveniência:(Do gr. \textunderscore pous\textunderscore  + \textunderscore perma\textunderscore )}
\end{itemize}
Um dos filamentos molles do ovário vegetal.
\section{Podospermo}
\begin{itemize}
\item {Grp. gram.:m.}
\end{itemize}
\begin{itemize}
\item {Proveniência:(Do gr. \textunderscore pous\textunderscore  + \textunderscore sperma\textunderscore )}
\end{itemize}
Gênero de plantas, da fam. das compostas.
\section{Podostemáceas}
\begin{itemize}
\item {Grp. gram.:f. pl.}
\end{itemize}
Família de plantas, que tem por typo o podostemo.
(F. pl. de \textunderscore podostemáceo\textunderscore )
\section{Podostemáceo}
\begin{itemize}
\item {Grp. gram.:adj.}
\end{itemize}
Relativo ou semelhante ao podostemo.
\section{Podostemo}
\begin{itemize}
\item {Grp. gram.:m.}
\end{itemize}
\begin{itemize}
\item {Proveniência:(Do gr. \textunderscore pous\textunderscore , \textunderscore podos\textunderscore  + \textunderscore stemon\textunderscore )}
\end{itemize}
Gênero de plantas herbáceas da América.
\section{Podostigma}
\begin{itemize}
\item {Grp. gram.:m.}
\end{itemize}
Gênero de plantas asclepiadáceas.
\section{Podoteca}
\begin{itemize}
\item {Grp. gram.:f.}
\end{itemize}
\begin{itemize}
\item {Utilização:Zool.}
\end{itemize}
\begin{itemize}
\item {Proveniência:(Do gr. \textunderscore pous\textunderscore , \textunderscore podos\textunderscore  + \textunderscore theke\textunderscore )}
\end{itemize}
Pele, que reveste o pé dos mamíferos e das aves.
\section{Podotheca}
\begin{itemize}
\item {Grp. gram.:f.}
\end{itemize}
\begin{itemize}
\item {Utilização:Zool.}
\end{itemize}
\begin{itemize}
\item {Proveniência:(Do gr. \textunderscore pous\textunderscore , \textunderscore podos\textunderscore  + \textunderscore theke\textunderscore )}
\end{itemize}
Pelle, que reveste o pé dos mammíferos e das aves.
\section{Podre}
\begin{itemize}
\item {fónica:pô}
\end{itemize}
\begin{itemize}
\item {Grp. gram.:adj.}
\end{itemize}
\begin{itemize}
\item {Utilização:Fig.}
\end{itemize}
\begin{itemize}
\item {Grp. gram.:M.}
\end{itemize}
\begin{itemize}
\item {Utilização:Fig.}
\end{itemize}
\begin{itemize}
\item {Grp. gram.:Pl.}
\end{itemize}
\begin{itemize}
\item {Proveniência:(Do lat. \textunderscore putris\textunderscore )}
\end{itemize}
Que está em decomposição.
Corrupto.
Putrefacto.
Infecto.
Deteriorado.
Pervertido.
Parte podre de alguma coisa.
O lado fraco ou condemnável.
Defeitos; vícios.
\section{Podredoiro}
\begin{itemize}
\item {Grp. gram.:m.}
\end{itemize}
\begin{itemize}
\item {Proveniência:(De \textunderscore podre\textunderscore )}
\end{itemize}
Lugar, onde apodrecem quaesquer substâncias.
Lugar onde há muita podridão.
Monturo. Cf. Camillo, \textunderscore Mar. da Fonte\textunderscore , 20.
\section{Podredouro}
\begin{itemize}
\item {Grp. gram.:m.}
\end{itemize}
\begin{itemize}
\item {Proveniência:(De \textunderscore podre\textunderscore )}
\end{itemize}
Lugar, onde apodrecem quaesquer substâncias.
Lugar onde há muita podridão.
Monturo. Cf. Camillo, \textunderscore Mar. da Fonte\textunderscore , 20.
\section{Podricalho}
\begin{itemize}
\item {Grp. gram.:adj.}
\end{itemize}
\begin{itemize}
\item {Utilização:ant.}
\end{itemize}
\begin{itemize}
\item {Utilização:Fam.}
\end{itemize}
\begin{itemize}
\item {Proveniência:(De \textunderscore podre\textunderscore )}
\end{itemize}
Mollangueiro.
Que trabalha pouco; preguiçoso.
\section{Podrida}
\begin{itemize}
\item {Grp. gram.:adj.}
\end{itemize}
Diz-se de uma ôlha ou caldo, feito de perdizes, gallinhas, carne de porco e legumes.
Aos inquisidores chamou Filinto, III, 23:«\textunderscore serpentes de mais podrida Lerna\textunderscore ».
(Cp. cast. \textunderscore olla podrida\textunderscore )
\section{Podridão}
\begin{itemize}
\item {Grp. gram.:f.}
\end{itemize}
\begin{itemize}
\item {Utilização:Fig.}
\end{itemize}
Estado do que é podre.
Desmoralização; devassidão; vício.
\section{Podrido}
\begin{itemize}
\item {Grp. gram.:adj.}
\end{itemize}
\begin{itemize}
\item {Proveniência:(De \textunderscore podre\textunderscore )}
\end{itemize}
Apodrecido; inútil:«\textunderscore ...o candil..., a espirrar, estira os podridos morrões\textunderscore ». Castilho, \textunderscore Geórgicas\textunderscore , 51.
\section{Podrura}
\begin{itemize}
\item {Grp. gram.:f.}
\end{itemize}
\begin{itemize}
\item {Utilização:Des.}
\end{itemize}
O mesmo que \textunderscore podridão\textunderscore . Cf. Usque, 17.
\section{Podura}
\begin{itemize}
\item {Grp. gram.:f.}
\end{itemize}
Gênero de insectos, que andam sôbre a cauda.
(Fem. de \textunderscore poduro\textunderscore )
\section{Podurelas}
\begin{itemize}
\item {Grp. gram.:f. pl.}
\end{itemize}
Família de insectos, que tem por typo a podura.
\section{Poduro}
\begin{itemize}
\item {Grp. gram.:adj.}
\end{itemize}
\begin{itemize}
\item {Proveniência:(Do gr. \textunderscore pous\textunderscore , \textunderscore podos\textunderscore  + \textunderscore oura\textunderscore )}
\end{itemize}
Que anda sôbre a cauda, (falando-se de insectos).
\section{Poedeira}
\begin{itemize}
\item {fónica:po-e}
\end{itemize}
\begin{itemize}
\item {Grp. gram.:adj.}
\end{itemize}
\begin{itemize}
\item {Proveniência:(De \textunderscore poêr\textunderscore )}
\end{itemize}
Diz-se da gallinha, que já põe ovos ou que põe muitos ovos.
\section{Poedoiros}
\begin{itemize}
\item {fónica:po-e}
\end{itemize}
\begin{itemize}
\item {Grp. gram.:m. pl.}
\end{itemize}
\begin{itemize}
\item {Proveniência:(De \textunderscore poêr\textunderscore )}
\end{itemize}
Trapos ou fios, que se usavam no tinteiro, para conservarem a tinta embebida nelles.
Trapos, embebidos em tintas, e de que se servem os pintores.
\section{Poedouros}
\begin{itemize}
\item {fónica:po-e}
\end{itemize}
\begin{itemize}
\item {Grp. gram.:m. pl.}
\end{itemize}
\begin{itemize}
\item {Proveniência:(De \textunderscore poêr\textunderscore )}
\end{itemize}
Trapos ou fios, que se usavam no tinteiro, para conservarem a tinta embebida nelles.
Trapos, embebidos em tintas, e de que se servem os pintores.
\section{Poeira}
\begin{itemize}
\item {Grp. gram.:f.}
\end{itemize}
\begin{itemize}
\item {Utilização:Ext.}
\end{itemize}
\begin{itemize}
\item {Utilização:Fam.}
\end{itemize}
\begin{itemize}
\item {Utilização:Fam.}
\end{itemize}
\begin{itemize}
\item {Utilização:Ant.}
\end{itemize}
\begin{itemize}
\item {Proveniência:(De \textunderscore pó\textunderscore )}
\end{itemize}
Terra sêca, reduzida a pó.
Pó.
Chão, solo.
Presumpção, vaidade, jactância.
O mesmo que \textunderscore badanal\textunderscore : \textunderscore andavam todos numa poeira\textunderscore .
Vaso com areia, empregada em secar a tinta com que se escreve. Cf. \textunderscore Peregrinação\textunderscore , CIII.
\section{Poeirada}
\begin{itemize}
\item {Grp. gram.:f.}
\end{itemize}
Grande porção de poeira; nuvem de pó.
\section{Poeirento}
\begin{itemize}
\item {Grp. gram.:adj.}
\end{itemize}
Que tem poeira; coberto de pó.
\section{Poeirinha}
\begin{itemize}
\item {Grp. gram.:f.}
\end{itemize}
Casta de uva tinta da Bairrada e do Dão.
\section{Poeiro}
\begin{itemize}
\item {Grp. gram.:m.}
\end{itemize}
\begin{itemize}
\item {Proveniência:(De \textunderscore pó\textunderscore )}
\end{itemize}
Parte da mesa, nas fábricas de telha, em que se põe o pó com que se tende o pedaço de barro para cada telha.
O mesmo que \textunderscore oídio\textunderscore .
\section{Poeiroso}
\begin{itemize}
\item {Grp. gram.:adj.}
\end{itemize}
\begin{itemize}
\item {Utilização:Bras}
\end{itemize}
Em que há poeira, que tem poeira.
\section{Poejo}
\begin{itemize}
\item {fónica:po-e}
\end{itemize}
\begin{itemize}
\item {Grp. gram.:m.}
\end{itemize}
\begin{itemize}
\item {Proveniência:(Do lat. \textunderscore pulegium\textunderscore )}
\end{itemize}
Planta labiada e medicinal.
\section{Poejo}
\begin{itemize}
\item {fónica:po-e}
\end{itemize}
\begin{itemize}
\item {Grp. gram.:m.}
\end{itemize}
\begin{itemize}
\item {Utilização:Prov.}
\end{itemize}
\begin{itemize}
\item {Utilização:beir.}
\end{itemize}
\begin{itemize}
\item {Proveniência:(De \textunderscore pó\textunderscore )}
\end{itemize}
Farinha fina ou o pó mais fino da farinha.
\section{Poema}
\begin{itemize}
\item {fónica:po-ê}
\end{itemize}
\begin{itemize}
\item {Grp. gram.:m.}
\end{itemize}
\begin{itemize}
\item {Utilização:Ext.}
\end{itemize}
\begin{itemize}
\item {Proveniência:(Lat. \textunderscore poema\textunderscore )}
\end{itemize}
Obra em verso.
Composição poética, mais ou menos extensa.
Epopeia.
Obra em prosa, em que ha ficção e estilo poético.
Assumpto ou coisa digna de sêr cantada em verso.
\section{Põe-mesa}
\begin{itemize}
\item {Grp. gram.:f.}
\end{itemize}
\begin{itemize}
\item {Utilização:Bras. do N}
\end{itemize}
O mesmo que \textunderscore louva-a-deus\textunderscore .
\section{Poemeto}
\begin{itemize}
\item {fónica:mê}
\end{itemize}
\begin{itemize}
\item {Grp. gram.:m.}
\end{itemize}
Poema curto.
\section{Poente}
\begin{itemize}
\item {Grp. gram.:m.}
\end{itemize}
\begin{itemize}
\item {Grp. gram.:Adj.}
\end{itemize}
\begin{itemize}
\item {Utilização:Des.}
\end{itemize}
\begin{itemize}
\item {Proveniência:(Lat. \textunderscore ponens\textunderscore )}
\end{itemize}
O mesmo que \textunderscore Occidente\textunderscore .
Que põe.
E diz-se do sol, quando está no occaso.
\section{Poênto}
\begin{itemize}
\item {Grp. gram.:adj.}
\end{itemize}
O mesmo que \textunderscore poeirento\textunderscore .
\section{Poêr}
\begin{itemize}
\item {Grp. gram.:v. t.}
\end{itemize}
\begin{itemize}
\item {Utilização:Ant.}
\end{itemize}
O mesmo que \textunderscore pôr\textunderscore .
\section{Poesia}
\begin{itemize}
\item {fónica:po-e}
\end{itemize}
\begin{itemize}
\item {Grp. gram.:f.}
\end{itemize}
\begin{itemize}
\item {Proveniência:(Do lat. \textunderscore poesis\textunderscore )}
\end{itemize}
Arte de escrever em verso.
Poética.
Conjunto dos differentes gêneros de poema.
Qualidade que caracteriza os bons versos: \textunderscore amar a poesia\textunderscore .
Composição poética pouco extensa: \textunderscore escrever poesias\textunderscore .
Inspiração.
Aquillo que desperta o sentimento do bello.
O que há de elevado ou commovente em qualquer coisa ou pessôa.
\section{Poéta}
\begin{itemize}
\item {Grp. gram.:m.  e  adj.}
\end{itemize}
\begin{itemize}
\item {Utilização:Bras. de Minas}
\end{itemize}
\begin{itemize}
\item {Proveniência:(Lat. \textunderscore poeta\textunderscore )}
\end{itemize}
Aquelle que se dedica á poesia.
O que tem faculdades poéticas.
O que faz versos.
O que tem inspiração poética.
O que devaneia ou tem carácter idealista.
Aquelle que tem loquacidade; que é pronóstico; que fala bem.
\section{Poetaço}
\begin{itemize}
\item {fónica:po-e}
\end{itemize}
\begin{itemize}
\item {Grp. gram.:m.}
\end{itemize}
\begin{itemize}
\item {Proveniência:(De \textunderscore poéta\textunderscore )}
\end{itemize}
O que faz maus versos.
\section{Poetagem}
\begin{itemize}
\item {fónica:po-e}
\end{itemize}
\begin{itemize}
\item {Grp. gram.:f.}
\end{itemize}
\begin{itemize}
\item {Utilização:Bras. de Minas}
\end{itemize}
Loquacidade.
(Cp. \textunderscore poéta\textunderscore )
\section{Poetar}
\begin{itemize}
\item {fónica:po-e}
\end{itemize}
\begin{itemize}
\item {Grp. gram.:v. t.}
\end{itemize}
\begin{itemize}
\item {Grp. gram.:V. i.}
\end{itemize}
\begin{itemize}
\item {Proveniência:(De \textunderscore poéta\textunderscore )}
\end{itemize}
Cantar em verso.
Fazer poesias.
\section{Poetastro}
\begin{itemize}
\item {fónica:po-e}
\end{itemize}
\begin{itemize}
\item {Grp. gram.:m.}
\end{itemize}
Mau poéta, poetaço. Cf. Filinto, VIII, 25 e 76.
\section{Poética}
\begin{itemize}
\item {Grp. gram.:f.}
\end{itemize}
Arte de fazer versos ou composições poéticas.
(Fem. de \textunderscore poético\textunderscore )
\section{Poeticamente}
\begin{itemize}
\item {Grp. gram.:adv.}
\end{itemize}
De modo poético; á maneira de poétas.
\section{Poético}
\begin{itemize}
\item {Grp. gram.:adj.}
\end{itemize}
\begin{itemize}
\item {Proveniência:(Lat. \textunderscore poeticus\textunderscore )}
\end{itemize}
Relativo á poesia: \textunderscore arte poética\textunderscore .
Em que há poesia.
Que inspira.
Digno de sêr cantado em verso.
\section{Poetificar}
\begin{itemize}
\item {fónica:po-e}
\end{itemize}
\begin{itemize}
\item {Grp. gram.:v. t.}
\end{itemize}
\begin{itemize}
\item {Proveniência:(Do lat. \textunderscore poeta\textunderscore  + \textunderscore facere\textunderscore )}
\end{itemize}
Tornar poético.
Inspirar poesia a. Cf. Castilho, \textunderscore Fastos\textunderscore , I, 178.
\section{Poetisa}
\begin{itemize}
\item {fónica:po-e}
\end{itemize}
\begin{itemize}
\item {Grp. gram.:f.}
\end{itemize}
\begin{itemize}
\item {Proveniência:(De \textunderscore poéta\textunderscore )}
\end{itemize}
Mulhér, que faz composições poéticas.
\section{Poetismo}
\begin{itemize}
\item {fónica:po-e}
\end{itemize}
\begin{itemize}
\item {Grp. gram.:m.}
\end{itemize}
\begin{itemize}
\item {Proveniência:(De \textunderscore poéta\textunderscore )}
\end{itemize}
Os poetas.
\section{Poetizar}
\begin{itemize}
\item {fónica:po-e}
\end{itemize}
\begin{itemize}
\item {Grp. gram.:v. t.}
\end{itemize}
\begin{itemize}
\item {Grp. gram.:V. i.}
\end{itemize}
\begin{itemize}
\item {Proveniência:(De \textunderscore poéta\textunderscore )}
\end{itemize}
Tornar poético.
Poetar.
\section{Pogeia}
\begin{itemize}
\item {Grp. gram.:f.}
\end{itemize}
O mesmo que \textunderscore pogeja\textunderscore .
\section{Pogeja}
\begin{itemize}
\item {Grp. gram.:f.}
\end{itemize}
Moéda antiga, o mesmo que \textunderscore mealha\textunderscore .
\section{Pogónia}
\begin{itemize}
\item {Grp. gram.:f.}
\end{itemize}
\begin{itemize}
\item {Proveniência:(Do gr. \textunderscore pogon\textunderscore )}
\end{itemize}
Gênero de orchídeas.
Gênero de peixes acanthopterýgios.
\section{Pogoníase}
\begin{itemize}
\item {Grp. gram.:f.}
\end{itemize}
\begin{itemize}
\item {Proveniência:(Do gr. \textunderscore pogon\textunderscore , barba)}
\end{itemize}
Desenvolvimento da barba em uma mulhér.
\section{Pogonóforas}
\begin{itemize}
\item {Grp. gram.:f. pl.}
\end{itemize}
\begin{itemize}
\item {Proveniência:(De \textunderscore pogonóforo\textunderscore )}
\end{itemize}
Aves trepadoras, que apresentam pêlos em volta do bico.
\section{Pogonóforo}
\begin{itemize}
\item {Grp. gram.:adj.}
\end{itemize}
\begin{itemize}
\item {Utilização:Zool.}
\end{itemize}
\begin{itemize}
\item {Proveniência:(Do gr. \textunderscore pogon\textunderscore  + \textunderscore phoros\textunderscore )}
\end{itemize}
Diz-se do animal que no focinho tem pêlos á semelhança de barbas.
\section{Pogonóphoras}
\begin{itemize}
\item {Grp. gram.:f. pl.}
\end{itemize}
\begin{itemize}
\item {Proveniência:(De \textunderscore pogonóphoro\textunderscore )}
\end{itemize}
Aves trepadoras, que apresentam pêlos em volta do bico.
\section{Pogonóphoro}
\begin{itemize}
\item {Grp. gram.:adj.}
\end{itemize}
\begin{itemize}
\item {Utilização:Zool.}
\end{itemize}
\begin{itemize}
\item {Proveniência:(Do gr. \textunderscore pogon\textunderscore  + \textunderscore phoros\textunderscore )}
\end{itemize}
Diz-se do animal que no focinho tem pêlos á semelhança de barbas.
\section{Pogonópode}
\begin{itemize}
\item {Grp. gram.:adj.}
\end{itemize}
\begin{itemize}
\item {Utilização:Zool.}
\end{itemize}
\begin{itemize}
\item {Proveniência:(Do gr. \textunderscore pogon\textunderscore  + \textunderscore pous\textunderscore , \textunderscore podos\textunderscore )}
\end{itemize}
Que tem os pés cobertos de pêlos.
\section{Pogostemo}
\begin{itemize}
\item {Grp. gram.:m.}
\end{itemize}
\begin{itemize}
\item {Proveniência:(Do gr. \textunderscore pogon\textunderscore  + \textunderscore stemon\textunderscore )}
\end{itemize}
Gênero de plantas labiadas da Índia, que comprehende muitas espécies, uma das quaes fornece o perfume, que os Franceses chamam \textunderscore patchouli\textunderscore .
\section{Pogóstoma}
\begin{itemize}
\item {Grp. gram.:f.}
\end{itemize}
Gênero de plantas escrofularíneas.
\section{Poh!}
\begin{itemize}
\item {Grp. gram.:interj.}
\end{itemize}
(designativa de espanto ou de repulsão) Cf. Garrett, \textunderscore Filippa\textunderscore , 91.
\section{Poia}
\begin{itemize}
\item {Grp. gram.:f.}
\end{itemize}
\begin{itemize}
\item {Utilização:Prov.}
\end{itemize}
\begin{itemize}
\item {Utilização:Prov.}
\end{itemize}
\begin{itemize}
\item {Utilização:beir.}
\end{itemize}
\begin{itemize}
\item {Utilização:trasm.}
\end{itemize}
\begin{itemize}
\item {Utilização:Prov.}
\end{itemize}
\begin{itemize}
\item {Utilização:beir.}
\end{itemize}
\begin{itemize}
\item {Utilização:Pop.}
\end{itemize}
Pão alto ou bôlo grande de trigo.
Bôla ou pão chato, que o dono de uma fornada dá, como retribuição, ao dono do forno onde se coze o pão.
Porção de azeite, que se dá ao dono do lagar, onde se mói azeitona, como retribuição pelo serviço da moagem.
Acervo de dejectos.
(Cp. \textunderscore poio\textunderscore )
\section{Poial}
\begin{itemize}
\item {Grp. gram.:m.}
\end{itemize}
\begin{itemize}
\item {Proveniência:(De \textunderscore poio\textunderscore )}
\end{itemize}
Lugar, onde se assenta ou colloca alguma coisa.
Banco fixo; assento de pedra.
\section{Poiar}
\begin{itemize}
\item {Grp. gram.:v. i.}
\end{itemize}
\begin{itemize}
\item {Utilização:Ant.}
\end{itemize}
\begin{itemize}
\item {Grp. gram.:V. t.}
\end{itemize}
\begin{itemize}
\item {Proveniência:(De \textunderscore poio\textunderscore )}
\end{itemize}
Apoiar-se nalguma coisa para subir.
Subir a lugar elevado.
Collocar, depôr:«\textunderscore Ganhada a peleja maritima, poiárão em terra os nossos capitães a sua gente...\textunderscore »Filinto, \textunderscore D. Man.\textunderscore , I, 319.
\section{Poiarês}
\begin{itemize}
\item {Grp. gram.:m.}
\end{itemize}
\begin{itemize}
\item {Utilização:Prov.}
\end{itemize}
\begin{itemize}
\item {Utilização:trasm.}
\end{itemize}
Aquelle que é natural de Poiares, no concelho de Freixo-de-Espada-á-Cinta.
\section{Poideira}
\begin{itemize}
\item {fónica:po-i}
\end{itemize}
\begin{itemize}
\item {Grp. gram.:f.}
\end{itemize}
Substância, com que se fricciona um objecto, para o poir.
\section{Poidoiro}
\begin{itemize}
\item {fónica:po-i}
\end{itemize}
\begin{itemize}
\item {Grp. gram.:f.}
\end{itemize}
\begin{itemize}
\item {Proveniência:(De \textunderscore poir\textunderscore )}
\end{itemize}
Trapo dobrado, por entre o qual passa o fio da meada que se doba.
\section{Poidouro}
\begin{itemize}
\item {fónica:po-i}
\end{itemize}
\begin{itemize}
\item {Grp. gram.:f.}
\end{itemize}
\begin{itemize}
\item {Proveniência:(De \textunderscore poir\textunderscore )}
\end{itemize}
Trapo dobrado, por entre o qual passa o fio da meada que se doba.
\section{Poilão}
\begin{itemize}
\item {Grp. gram.:m.}
\end{itemize}
Corpulenta árvore da Guiné, cujos frutos dão uma espécie de lan, que serve para encher cholchões; o mesmo que \textunderscore ocá\textunderscore  ou \textunderscore mafumeira\textunderscore .
\section{Poimento}
\begin{itemize}
\item {fónica:po-i}
\end{itemize}
\begin{itemize}
\item {Grp. gram.:m.}
\end{itemize}
\begin{itemize}
\item {Utilização:Ant.}
\end{itemize}
\begin{itemize}
\item {Proveniência:(De \textunderscore poêr\textunderscore )}
\end{itemize}
Acto de pôr ou depositar.
\section{Poinciana}
\begin{itemize}
\item {Grp. gram.:f.}
\end{itemize}
Nome de uma planta americana. Cf. Latino, \textunderscore Humboldt\textunderscore , 149.
\section{Poio}
\begin{itemize}
\item {Grp. gram.:m.}
\end{itemize}
\begin{itemize}
\item {Utilização:Des.}
\end{itemize}
\begin{itemize}
\item {Proveniência:(Do lat. \textunderscore podium\textunderscore )}
\end{itemize}
(outros l. \textunderscore póio\textunderscore )
O mesmo que \textunderscore poial\textunderscore .
Oiteiro, monte.
\section{Poir}
\begin{itemize}
\item {Grp. gram.:v. t.}
\end{itemize}
\begin{itemize}
\item {Utilização:Fig.}
\end{itemize}
O mesmo que \textunderscore polir\textunderscore .
Desgastar, desfazer a pouco e pouco, roçando ou friccionando.
(Contr. de \textunderscore polir\textunderscore )
\section{Pois}
\begin{itemize}
\item {Grp. gram.:conj.}
\end{itemize}
\begin{itemize}
\item {Proveniência:(Do lat. \textunderscore post\textunderscore )}
\end{itemize}
(designativa de causa ou consequência, e, algumas vezes, partícula expletiva).
Á vista disso.
Portanto.
Além disso.
\section{Poisa}
\begin{itemize}
\item {Grp. gram.:f.}
\end{itemize}
\begin{itemize}
\item {Utilização:Ant.}
\end{itemize}
\begin{itemize}
\item {Utilização:Prov.}
\end{itemize}
\begin{itemize}
\item {Utilização:minh.}
\end{itemize}
\begin{itemize}
\item {Utilização:Prov.}
\end{itemize}
\begin{itemize}
\item {Utilização:trasm.}
\end{itemize}
\begin{itemize}
\item {Utilização:Prov.}
\end{itemize}
\begin{itemize}
\item {Utilização:dur.}
\end{itemize}
\begin{itemize}
\item {Utilização:Prov.}
\end{itemize}
\begin{itemize}
\item {Utilização:minh.}
\end{itemize}
\begin{itemize}
\item {Utilização:Prov.}
\end{itemize}
\begin{itemize}
\item {Utilização:beir.}
\end{itemize}
\begin{itemize}
\item {Proveniência:(De \textunderscore poisar\textunderscore )}
\end{itemize}
Lugar ou habitação, onde o cobrador de foros reaes devia poisar e receber mantimentos.
A hora da meia noite nos trabalhos de lagareiros.
Cada um dos períodos, em que se divide o tempo de pisar o mosto.
Acto de \textunderscore poisar\textunderscore .
Lugar, onde se poisa o carrêgo, para descansar.
Quatro ou cinco feixes de pão de pragana, (trigo, centeio, etc.).
\section{Poisada}
\begin{itemize}
\item {Grp. gram.:f.}
\end{itemize}
\begin{itemize}
\item {Utilização:Prov.}
\end{itemize}
\begin{itemize}
\item {Utilização:trasm.}
\end{itemize}
\begin{itemize}
\item {Proveniência:(Do lat. \textunderscore pausata\textunderscore )}
\end{itemize}
Acto ou effeito de poisar.
Albergaria, albergue.
Lugar ou casa, em que se poisa ou se é hospedado.
Residência.
Choupana.
Conjuncto de quatro feixes de pão ceifado, devendo produzir um alqueire depois da trilha; poisa.
Casta de uva da Bairrada.
\section{Poisadeira}
\begin{itemize}
\item {Grp. gram.:f.}
\end{itemize}
O mesmo que [[nádegas|nádega]]. Cf. Castilho, \textunderscore D. Quixote\textunderscore , IX. (Também us. no pl., com o mesmo sentido).
(Cp. \textunderscore poisadeiro\textunderscore ^2)
\section{Poisadeiro}
\begin{itemize}
\item {Grp. gram.:m.}
\end{itemize}
\begin{itemize}
\item {Utilização:Ant.}
\end{itemize}
O que dá ou prepara a poisada. Cf. \textunderscore Auto de S. Antonio\textunderscore .
\section{Poisadeiro}
\begin{itemize}
\item {Grp. gram.:m.}
\end{itemize}
\begin{itemize}
\item {Utilização:Pleb.}
\end{itemize}
\begin{itemize}
\item {Proveniência:(De \textunderscore poisar\textunderscore )}
\end{itemize}
Nádegas. Cf. G. Vicente, I, 50, (ed. de M. Remédios).
\section{Poisadoiro}
\begin{itemize}
\item {Grp. gram.:m.}
\end{itemize}
\begin{itemize}
\item {Utilização:Pleb.}
\end{itemize}
\begin{itemize}
\item {Proveniência:(Do lat. \textunderscore pausatorius\textunderscore )}
\end{itemize}
O mesmo que \textunderscore poisada\textunderscore .
Poisadeiro^2.
\section{Poisafolles}
\begin{itemize}
\item {Grp. gram.:m.  e  f.}
\end{itemize}
\begin{itemize}
\item {Utilização:Ant.}
\end{itemize}
\begin{itemize}
\item {Proveniência:(De \textunderscore poisar\textunderscore  + \textunderscore folle\textunderscore )}
\end{itemize}
Pessôa muito vagarosa ou indolente, que descansa ao menor trabalho.
\section{Poisamoira}
\begin{itemize}
\item {Grp. gram.:f.}
\end{itemize}
\begin{itemize}
\item {Utilização:Prov.}
\end{itemize}
\begin{itemize}
\item {Utilização:trasm.}
\end{itemize}
Designação vulgar da borboleta.
\section{Poisar}
\begin{itemize}
\item {Grp. gram.:v. t.}
\end{itemize}
\begin{itemize}
\item {Grp. gram.:V. i.}
\end{itemize}
\begin{itemize}
\item {Proveniência:(Do lat. \textunderscore pausare\textunderscore )}
\end{itemize}
Pôr; assentar: \textunderscore poisar o pé em falso\textunderscore .
Depor: \textunderscore poisar um fardo\textunderscore .
Estabelecer-se.
Collocar-se.
Hospedar-se, albergar-se.
Empoleirar-se.
Estar assente.
Parar.
Descansar.
Residir.
Acoitar-se.
\section{Poiseiro}
\begin{itemize}
\item {Grp. gram.:m.}
\end{itemize}
\begin{itemize}
\item {Utilização:Des.}
\end{itemize}
\begin{itemize}
\item {Grp. gram.:Adj.}
\end{itemize}
\begin{itemize}
\item {Utilização:Prov.}
\end{itemize}
\begin{itemize}
\item {Utilização:minh.}
\end{itemize}
\begin{itemize}
\item {Proveniência:(De \textunderscore poisar\textunderscore )}
\end{itemize}
Nádegas, o mesmo que \textunderscore poisadeiro\textunderscore ^2.
Pacato:«\textunderscore séria, poiseira e sensaborona...\textunderscore »Camillo, \textunderscore Novellas\textunderscore , I, 216.
O mesmo que \textunderscore sedentário\textunderscore .
Costumado a poisar; que vai poisar.
\section{Poisentador}
\begin{itemize}
\item {Grp. gram.:m.}
\end{itemize}
\begin{itemize}
\item {Utilização:Ant.}
\end{itemize}
O mesmo que \textunderscore aposentador\textunderscore .
\section{Poisentar}
\begin{itemize}
\item {Grp. gram.:v. t.}
\end{itemize}
\begin{itemize}
\item {Utilização:Ant.}
\end{itemize}
O mesmo que \textunderscore aposentar\textunderscore .
\section{Poisinho}
\begin{itemize}
\item {Grp. gram.:m.}
\end{itemize}
\begin{itemize}
\item {Utilização:Prov.}
\end{itemize}
\begin{itemize}
\item {Utilização:trasm.}
\end{itemize}
\begin{itemize}
\item {Proveniência:(De \textunderscore poisar\textunderscore )}
\end{itemize}
Indivíduo, que anda pouco, que pára em qualquer parte; vagaroso.
\section{Poisio}
\begin{itemize}
\item {Grp. gram.:m.}
\end{itemize}
\begin{itemize}
\item {Grp. gram.:Adj.}
\end{itemize}
\begin{itemize}
\item {Proveniência:(De \textunderscore poiso\textunderscore )}
\end{itemize}
Interrupção da cultura de uma terra, por um ou mais annos.
Terreno, cuja cultura se interrompeu, para que elle depois se torne mais pingue.
Inculto: \textunderscore terras poisias\textunderscore .
\section{Poiso}
\begin{itemize}
\item {Grp. gram.:m.}
\end{itemize}
\begin{itemize}
\item {Utilização:Bras}
\end{itemize}
\begin{itemize}
\item {Utilização:Bras}
\end{itemize}
\begin{itemize}
\item {Grp. gram.:M. Pl.}
\end{itemize}
\begin{itemize}
\item {Proveniência:(De \textunderscore poisar\textunderscore )}
\end{itemize}
Lugar onde alguma pessôa ou coisa se poisa ou colloca.
Ancoradoiro.
Pedra, sôbre que gira a mó das azenhas.
Telheiro ou choça, á beira dos caminhos, para abrigo de viandantes.
Rancho.
O mesmo que \textunderscore pêso\textunderscore  (do lagar).
Travessa de madeira, em que assenta a quilha do navio no estaleiro.
\section{Poita}
\begin{itemize}
\item {Grp. gram.:f.}
\end{itemize}
\begin{itemize}
\item {Utilização:Pesc.}
\end{itemize}
Corpo pesado, que as pequenas embarcações de pesca usam, em vez de fateixa, para fundear.
\section{Poitada}
\begin{itemize}
\item {Grp. gram.:f.}
\end{itemize}
\begin{itemize}
\item {Utilização:Prov.}
\end{itemize}
\begin{itemize}
\item {Utilização:minh.}
\end{itemize}
O mesmo que \textunderscore poita\textunderscore .
\section{Poitão}
\begin{itemize}
\item {Grp. gram.:m.}
\end{itemize}
Poita grande.
\section{Poja}
\begin{itemize}
\item {Grp. gram.:f.}
\end{itemize}
\begin{itemize}
\item {Utilização:Náut.}
\end{itemize}
\begin{itemize}
\item {Proveniência:(De \textunderscore pojar\textunderscore )}
\end{itemize}
Parte inferior da vela do navio; corda, com que se vira a vela.
\section{Pojadoiro}
\begin{itemize}
\item {Grp. gram.:m.}
\end{itemize}
\begin{itemize}
\item {Utilização:T. de açougue}
\end{itemize}
Parte da coxa do boi, também conhecido por \textunderscore chan de dentro\textunderscore , e cuja carne é de primeira qualidade.
(Por \textunderscore bojadoiro\textunderscore , de \textunderscore bojar\textunderscore ?)
\section{Pojadouro}
\begin{itemize}
\item {Grp. gram.:m.}
\end{itemize}
\begin{itemize}
\item {Utilização:T. de açougue}
\end{itemize}
Parte da coxa do boi, também conhecido por \textunderscore chan de dentro\textunderscore , e cuja carne é de primeira qualidade.
(Por \textunderscore bojadoiro\textunderscore , de \textunderscore bojar\textunderscore ?)
\section{Pojadura}
\begin{itemize}
\item {Grp. gram.:f.}
\end{itemize}
O mesmo que \textunderscore apojadura\textunderscore .
\section{Pojal}
\begin{itemize}
\item {Grp. gram.:m.}
\end{itemize}
\begin{itemize}
\item {Utilização:Des.}
\end{itemize}
O mesmo que \textunderscore poial\textunderscore . Cf. \textunderscore Agostinheida\textunderscore , 31.
\section{Pojante}
\begin{itemize}
\item {Grp. gram.:adj.}
\end{itemize}
\begin{itemize}
\item {Proveniência:(De \textunderscore pojar\textunderscore )}
\end{itemize}
Que navega bem, ou com vento favorável.
\section{Pojar}
\begin{itemize}
\item {Grp. gram.:v. i.}
\end{itemize}
\begin{itemize}
\item {Grp. gram.:V. t.}
\end{itemize}
\begin{itemize}
\item {Proveniência:(Do lat. hyp. \textunderscore podiare\textunderscore )}
\end{itemize}
Aportar, abicar; desembarcar.
Elevar, entumecer:«\textunderscore ...e um collete..., com atacadores, que pojavam os seios...\textunderscore »Camillo, \textunderscore Eus. Macário\textunderscore , 28.
\section{Pojeira}
\begin{itemize}
\item {Grp. gram.:f.}
\end{itemize}
\begin{itemize}
\item {Utilização:Prov.}
\end{itemize}
O mesmo que \textunderscore poeira\textunderscore :«\textunderscore algumas mulheres..., sujas da pojeira das estradas...\textunderscore »Camillo, \textunderscore Brasileira\textunderscore , 339.
(Cp. [[espojar|espojar-se]])
\section{Pojo}
\begin{itemize}
\item {fónica:pô}
\end{itemize}
\begin{itemize}
\item {Grp. gram.:m.}
\end{itemize}
\begin{itemize}
\item {Proveniência:(De \textunderscore pojar\textunderscore )}
\end{itemize}
Lugar, onde se desembarca.
Poial ou lugar, onde se depõe alguma coisa; ou pequena elevação de terreno ou de pedra, para descanso ou para poisar qualquer fardo:«\textunderscore ...ao umbral da gruta, pojo musgoso jaz.\textunderscore »Filinto, XV, 181.
\section{Pojo}
\begin{itemize}
\item {fónica:pô}
\end{itemize}
\begin{itemize}
\item {Grp. gram.:m.}
\end{itemize}
\begin{itemize}
\item {Utilização:Prov.}
\end{itemize}
\begin{itemize}
\item {Utilização:trasm.}
\end{itemize}
O mesmo que \textunderscore poejo\textunderscore ^2.
\section{Pôla}
\begin{itemize}
\item {Grp. gram.:f.}
\end{itemize}
Ramo de árvore, pernada.
Rebento.
Estaca.
\section{Pola}
\begin{itemize}
\item {fónica:pula}
\end{itemize}
\begin{itemize}
\item {Utilização:Ant.}
\end{itemize}
(\textunderscore pula...\textunderscore , procliticamente)
Contr. da prep. \textunderscore por\textunderscore  e do art. \textunderscore la\textunderscore .
\section{Póla}
\begin{itemize}
\item {Grp. gram.:f.}
\end{itemize}
\begin{itemize}
\item {Proveniência:(De \textunderscore polé\textunderscore )}
\end{itemize}
Pancadaria, sova.
\section{Póla}
\begin{itemize}
\item {Grp. gram.:f.}
\end{itemize}
\begin{itemize}
\item {Utilização:Ant.}
\end{itemize}
Grandeza; aumento; sublimidade.
(Cp. \textunderscore empôla\textunderscore ^1)
\section{Polábico}
\begin{itemize}
\item {Grp. gram.:m.}
\end{itemize}
O mesmo que \textunderscore polábio\textunderscore .
\section{Polábio}
\begin{itemize}
\item {Grp. gram.:m.}
\end{itemize}
Língua morta da região do Elba.
\section{Polaca}
\begin{itemize}
\item {Grp. gram.:f.}
\end{itemize}
\begin{itemize}
\item {Utilização:Náut.}
\end{itemize}
Navio de três mastros e prôa longa e aguda.
Vela, que se emprega como estai do traquete.
(Holl. \textunderscore polaak\textunderscore )
\section{Polaca}
\begin{itemize}
\item {Grp. gram.:f.}
\end{itemize}
\begin{itemize}
\item {Proveniência:(De \textunderscore polaco\textunderscore )}
\end{itemize}
Espécie de dança e música correspondente, em movimento moderado e de carácter pomposo.
\section{Polachênio}
\begin{itemize}
\item {fónica:quê}
\end{itemize}
\begin{itemize}
\item {Grp. gram.:m.}
\end{itemize}
\begin{itemize}
\item {Proveniência:(De \textunderscore polus\textunderscore  gr. + \textunderscore achênio\textunderscore )}
\end{itemize}
Fruto sêco e indehiscente.
\section{Polaco}
\begin{itemize}
\item {Grp. gram.:adj.}
\end{itemize}
\begin{itemize}
\item {Grp. gram.:M.}
\end{itemize}
\begin{itemize}
\item {Utilização:Gír.}
\end{itemize}
Relativo á Polónia.
Habitante da Polónia.
Língua, que se fala na Polónia.
Pai.
\section{Polacra}
\begin{itemize}
\item {Grp. gram.:f.}
\end{itemize}
O mesmo que \textunderscore polaca\textunderscore ^1.
\section{Polainado}
\begin{itemize}
\item {Grp. gram.:adj.}
\end{itemize}
Que tem polainas grandes.
\section{Polainas}
\begin{itemize}
\item {Grp. gram.:f. pl.}
\end{itemize}
Peça de vestuário, que protege a parte inferior da perna e a parte superior do pé, por fóra ou por dentro das calças e por cima do calçado.
(Cp. fr. \textunderscore poulaines\textunderscore )
\section{Polanísia}
\begin{itemize}
\item {Grp. gram.:f.}
\end{itemize}
\begin{itemize}
\item {Proveniência:(Do gr. \textunderscore polus\textunderscore  + \textunderscore anisos\textunderscore )}
\end{itemize}
Gênero de plantas capparídeas das regiões quentes.
\section{Polaquênio}
\begin{itemize}
\item {Grp. gram.:m.}
\end{itemize}
\begin{itemize}
\item {Proveniência:(De \textunderscore polus\textunderscore  gr. + \textunderscore achênio\textunderscore )}
\end{itemize}
Fruto sêco e indeiscente.
\section{Polar}
\begin{itemize}
\item {Grp. gram.:adj.}
\end{itemize}
Relativo aos polos: \textunderscore gelos polares\textunderscore .
Que está junto dos polos, ou que fica na direcção de um pólo: \textunderscore estrêlla polar\textunderscore .
\section{Polaridade}
\begin{itemize}
\item {Grp. gram.:f.}
\end{itemize}
\begin{itemize}
\item {Proveniência:(De \textunderscore polar\textunderscore )}
\end{itemize}
Propriedade, que o íman ou a agulha magnética tem, de se voltar para um ponto fixo do horizonte.
\section{Polarímetro}
\begin{itemize}
\item {Grp. gram.:m.}
\end{itemize}
\begin{itemize}
\item {Proveniência:(De \textunderscore polar\textunderscore  + gr. \textunderscore metron\textunderscore )}
\end{itemize}
Instrumento, para determinar o desvio, que certas substâncias exercem sôbre os raios luminosos polarizados.
\section{Polariscópio}
\begin{itemize}
\item {Grp. gram.:m.}
\end{itemize}
\begin{itemize}
\item {Proveniência:(De \textunderscore polar\textunderscore  + gr. \textunderscore skopein\textunderscore )}
\end{itemize}
O mesmo que \textunderscore polarímetro\textunderscore .
\section{Polarização}
\begin{itemize}
\item {Grp. gram.:f.}
\end{itemize}
\begin{itemize}
\item {Proveniência:(De \textunderscore polarizar\textunderscore )}
\end{itemize}
Modificação especial dos raios luminosos, em virtude da qual, depois de reflectidos ou refractados, não pódem reflectir-se ou refractar-se novamente em certas direcções.
\section{Polarizador}
\begin{itemize}
\item {Grp. gram.:adj.}
\end{itemize}
Que polariza.
\section{Polarizar}
\begin{itemize}
\item {Grp. gram.:v. t.}
\end{itemize}
\begin{itemize}
\item {Proveniência:(De \textunderscore polar\textunderscore )}
\end{itemize}
Sujeitar á polarização.
\section{Polarizável}
\begin{itemize}
\item {Grp. gram.:adj.}
\end{itemize}
\begin{itemize}
\item {Proveniência:(De \textunderscore polar\textunderscore )}
\end{itemize}
Que se póde polarizar.
\section{Polatuco}
\begin{itemize}
\item {Grp. gram.:m.}
\end{itemize}
Gênero de mammíferos roedores, do Norte do antigo e do novo continente.
(Do russo \textunderscore polatouka\textunderscore )
\section{Polau}
\begin{itemize}
\item {Grp. gram.:m.}
\end{itemize}
\begin{itemize}
\item {Utilização:T. de Moçambique}
\end{itemize}
Espécie de môlho ou guisado, usado especialmente por Moiros.
(Cp. \textunderscore pilau\textunderscore )
\section{Polca}
\begin{itemize}
\item {Grp. gram.:f.}
\end{itemize}
\begin{itemize}
\item {Utilização:Bras}
\end{itemize}
O mesmo que \textunderscore grippe\textunderscore .
\section{Polca}
\begin{itemize}
\item {Grp. gram.:f.}
\end{itemize}
\begin{itemize}
\item {Proveniência:(Fr. \textunderscore polka\textunderscore )}
\end{itemize}
Espécie de dança a dois tempos, originária da Polónia ou da Bohêmia.
\section{Polcar}
\begin{itemize}
\item {Grp. gram.:v. i.}
\end{itemize}
Dançar a polca.
\section{Poldíngua}
\begin{itemize}
\item {Grp. gram.:f.}
\end{itemize}
Pequena e antiga moéda moscovita.
\section{Poldra}
\begin{itemize}
\item {fónica:pôl}
\end{itemize}
\begin{itemize}
\item {Grp. gram.:f.}
\end{itemize}
Égua de pouca idade.
(Cp. \textunderscore poldro\textunderscore )
\section{Poldra}
\begin{itemize}
\item {fónica:pôl}
\end{itemize}
\begin{itemize}
\item {Grp. gram.:f.}
\end{itemize}
Pernada de árvore.
(Corr. de \textunderscore pôla\textunderscore )
\section{Polaciuria}
\begin{itemize}
\item {fónica:ci-u}
\end{itemize}
\begin{itemize}
\item {Grp. gram.:f.}
\end{itemize}
\begin{itemize}
\item {Utilização:Med.}
\end{itemize}
\begin{itemize}
\item {Proveniência:(Do gr. \textunderscore pollakis\textunderscore  + \textunderscore ourein\textunderscore )}
\end{itemize}
Necessidade imperiosa e frequente de urinar.
\section{Polarda}
\begin{itemize}
\item {Grp. gram.:f.}
\end{itemize}
\begin{itemize}
\item {Utilização:Des.}
\end{itemize}
\begin{itemize}
\item {Proveniência:(Fr. \textunderscore poularde\textunderscore , de \textunderscore poule\textunderscore  = cast. \textunderscore polla\textunderscore )}
\end{itemize}
Franga crescida, quási gallinha:«\textunderscore venha a gorda polarda co'a omeleta...\textunderscore »Filinto, II, 4.
\section{Poldra-doirada}
\begin{itemize}
\item {Grp. gram.:f.}
\end{itemize}
Ave, o mesmo que \textunderscore doiradinha\textunderscore .
\section{Poldras}
\begin{itemize}
\item {fónica:pôl}
\end{itemize}
\begin{itemize}
\item {Grp. gram.:f. pl.}
\end{itemize}
(Corr. de \textunderscore alpondras\textunderscore )
\section{Poldril}
\begin{itemize}
\item {Grp. gram.:m.}
\end{itemize}
\begin{itemize}
\item {Proveniência:(De \textunderscore poldro\textunderscore )}
\end{itemize}
O mesmo que \textunderscore potril\textunderscore .
\section{Poldro}
\begin{itemize}
\item {fónica:pôl}
\end{itemize}
\begin{itemize}
\item {Grp. gram.:m.}
\end{itemize}
\begin{itemize}
\item {Proveniência:(Do b. lat. \textunderscore poltrum\textunderscore )}
\end{itemize}
Cavallo de pouca idade; potro.
\section{Polé}
\begin{itemize}
\item {Grp. gram.:f.}
\end{itemize}
\begin{itemize}
\item {Proveniência:(Do b. lat. \textunderscore polea\textunderscore )}
\end{itemize}
Roldana.
Antigo instrumento de supplício.
\section{Poleá}
\begin{itemize}
\item {Grp. gram.:m.}
\end{itemize}
Homem plebeu, no Malabar; homem de raça ínfima. Cf. \textunderscore Lusíadas\textunderscore , VII, 37.
\section{Poleadela}
\begin{itemize}
\item {Grp. gram.:f.}
\end{itemize}
\begin{itemize}
\item {Utilização:T. de Gaia}
\end{itemize}
\begin{itemize}
\item {Proveniência:(De \textunderscore polear\textunderscore )}
\end{itemize}
Muita pancada; sova, tunda.
\section{Poleame}
\begin{itemize}
\item {Grp. gram.:m.}
\end{itemize}
\begin{itemize}
\item {Utilização:Náut.}
\end{itemize}
\begin{itemize}
\item {Proveniência:(De \textunderscore polé\textunderscore )}
\end{itemize}
Conjunto de polés, cadenaes, bigotas, sapatas lisas ou dentadas, sapatilhas, patescas, andorinhas e caçoilas, que se empregam para retôrno dos cabos.
\section{Polear}
\begin{itemize}
\item {Grp. gram.:v. t.}
\end{itemize}
Maltratar com polé; maltratar.
\section{Polearia}
\begin{itemize}
\item {Grp. gram.:f.}
\end{itemize}
\begin{itemize}
\item {Proveniência:(De \textunderscore polé\textunderscore )}
\end{itemize}
Arte de poleeiro.
\section{Poleeiro}
\begin{itemize}
\item {Grp. gram.:m.}
\end{itemize}
Fabricante ou vendedor de polés.
\section{Polegada}
\begin{itemize}
\item {Grp. gram.:f.}
\end{itemize}
\begin{itemize}
\item {Proveniência:(Do b. lat. \textunderscore pollicata\textunderscore , de \textunderscore pollex\textunderscore )}
\end{itemize}
Medida, proximamente igual ao comprimento da segunda falange do polegar.
\section{Polegar}
\begin{itemize}
\item {Grp. gram.:m.  e  adj.}
\end{itemize}
\begin{itemize}
\item {Proveniência:(Lat. \textunderscore pollicaris\textunderscore )}
\end{itemize}
O dedo mais curto e grosso da mão.
O primeiro e mais grosso do pé.
Pequena vara de poda, só com um, dois ou três olhos.
\section{Poleia}
\begin{itemize}
\item {Grp. gram.:f.}
\end{itemize}
\begin{itemize}
\item {Utilização:Prov.}
\end{itemize}
\begin{itemize}
\item {Utilização:minh.}
\end{itemize}
Póla, tareia.
\section{Poleiro}
\begin{itemize}
\item {Grp. gram.:m.}
\end{itemize}
\begin{itemize}
\item {Utilização:Fig.}
\end{itemize}
\begin{itemize}
\item {Proveniência:(Do lat. \textunderscore pullarius\textunderscore )}
\end{itemize}
Vara em que as aves poisam, dentro das gaiolas ou capoeiras.
Capoeira.
Posição elevada, autoridade.
\section{Polela}
\begin{itemize}
\item {Grp. gram.:f.}
\end{itemize}
\begin{itemize}
\item {Proveniência:(Do lat. hyp. \textunderscore pullela\textunderscore , de \textunderscore pullus\textunderscore ?)}
\end{itemize}
O mesmo que \textunderscore polilha\textunderscore .
\section{Polemarcho}
\begin{itemize}
\item {Grp. gram.:m.}
\end{itemize}
\begin{itemize}
\item {Proveniência:(Gr. \textunderscore polemarkhos\textunderscore )}
\end{itemize}
Chefe superior do exército, entre os antigos Gregos.
Em Athenas, era o funccionário que geria os negócios militares, espécie de ministro da guerra, cujas funcções eram exercidas pelo terceiro archonte.
\section{Polemarco}
\begin{itemize}
\item {Grp. gram.:m.}
\end{itemize}
\begin{itemize}
\item {Proveniência:(Gr. \textunderscore polemarkhos\textunderscore )}
\end{itemize}
Chefe superior do exército, entre os antigos Gregos.
Em Athenas, era o funccionário que geria os negócios militares, espécie de ministro da guerra, cujas funções eram exercidas pelo terceiro arconte.
\section{Polemica}
\begin{itemize}
\item {Grp. gram.:f.}
\end{itemize}
Discussão oral; questão, controvérsia.
(Fem. de \textunderscore polemico\textunderscore )
\section{Polemicar}
\begin{itemize}
\item {Grp. gram.:v. i.}
\end{itemize}
Fazer polemica:«\textunderscore polemicou virulentamente.\textunderscore »Camillo, \textunderscore Cav. em Ruínas\textunderscore , 250.
\section{Polemico}
\begin{itemize}
\item {Grp. gram.:adj.}
\end{itemize}
\begin{itemize}
\item {Proveniência:(Gr. \textunderscore polemikos\textunderscore )}
\end{itemize}
Relativo a polemica.
\section{Polemista}
\begin{itemize}
\item {Grp. gram.:m. ,  f.  e  adj.}
\end{itemize}
\begin{itemize}
\item {Proveniência:(Do gr. \textunderscore polemistes\textunderscore )}
\end{itemize}
Pessôa, que faz polemica; que gosta de questionar; que discute bem.
\section{Polemístico}
\begin{itemize}
\item {Grp. gram.:adj.}
\end{itemize}
Relativo a polemista:«\textunderscore talento polemístico.\textunderscore »Romero, \textunderscore M. Assis\textunderscore , 165.
\section{Polemónia}
\begin{itemize}
\item {Grp. gram.:f.}
\end{itemize}
\begin{itemize}
\item {Proveniência:(Lat. \textunderscore polemonia\textunderscore )}
\end{itemize}
O mesmo ou melhor que \textunderscore polemónio\textunderscore .
\section{Polemoniáceas}
\begin{itemize}
\item {Grp. gram.:f. pl.}
\end{itemize}
Família de plantas, que tem por typo o polemónio.
(Fem. pl. de \textunderscore polemoniáceo\textunderscore )
\section{Polemoniáceo}
\begin{itemize}
\item {Grp. gram.:adj.}
\end{itemize}
Relativo ou semelhante ao polemónio.
\section{Polemónio}
\begin{itemize}
\item {Grp. gram.:m.}
\end{itemize}
\begin{itemize}
\item {Proveniência:(Gr. \textunderscore polemonion\textunderscore )}
\end{itemize}
Gênero de plantas herbáceas e vivazes.
\section{Pólen}
\begin{itemize}
\item {Grp. gram.:m.}
\end{itemize}
\begin{itemize}
\item {Proveniência:(Lat. \textunderscore pollen\textunderscore )}
\end{itemize}
Substância fecundante dos vegetaes, contida na anthera.
\section{Polenda}
\begin{itemize}
\item {Grp. gram.:f.}
\end{itemize}
O mesmo que \textunderscore polenta\textunderscore .
\section{Polênia}
\begin{itemize}
\item {Grp. gram.:f.}
\end{itemize}
\begin{itemize}
\item {Proveniência:(De \textunderscore pólen\textunderscore )}
\end{itemize}
Gênero de insectos dípteros.
\section{Polenta}
\begin{itemize}
\item {Grp. gram.:f.}
\end{itemize}
\begin{itemize}
\item {Proveniência:(Lat. \textunderscore polenta\textunderscore )}
\end{itemize}
Massa ou pasta de farinha de milho, com água e sal.--Parece que também se faz com farinha do castanhas, e com farinha de cevada.
\section{Pólex}
\begin{itemize}
\item {Grp. gram.:m.}
\end{itemize}
\begin{itemize}
\item {Proveniência:(Lat. \textunderscore pollex\textunderscore )}
\end{itemize}
O mesmo que \textunderscore polegar\textunderscore .
\section{Polgamio}
\begin{itemize}
\item {Grp. gram.:m.}
\end{itemize}
\begin{itemize}
\item {Utilização:Ant.}
\end{itemize}
O mesmo que \textunderscore pergaminho\textunderscore .
\section{Polgar}
\begin{itemize}
\item {Grp. gram.:m.  e  adj.}
\end{itemize}
\begin{itemize}
\item {Grp. gram.:M.}
\end{itemize}
\begin{itemize}
\item {Utilização:Prov.}
\end{itemize}
\begin{itemize}
\item {Utilização:minh.}
\end{itemize}
O mesmo que \textunderscore pollegar\textunderscore ; (cp. \textunderscore empolgar\textunderscore ).
Vara curta que, na poda, se deixa junto de uma vara grande, e que é destinada a dar varas no anno seguinte.(V.pollegar)
\section{Polha}
\begin{itemize}
\item {fónica:pô}
\end{itemize}
\begin{itemize}
\item {Grp. gram.:f.}
\end{itemize}
\begin{itemize}
\item {Utilização:Ant.}
\end{itemize}
\begin{itemize}
\item {Utilização:Fig.}
\end{itemize}
Franga.
Gallinha.
Rapariga.
(Cast. \textunderscore polla\textunderscore )
\section{Polhastro}
\begin{itemize}
\item {Grp. gram.:m.}
\end{itemize}
\begin{itemize}
\item {Utilização:Ant.}
\end{itemize}
\begin{itemize}
\item {Utilização:Fig.}
\end{itemize}
\begin{itemize}
\item {Proveniência:(Do lat. \textunderscore pullaster\textunderscore )}
\end{itemize}
Grande frango.
Rapagão.
Sujeito espertalhão. Cf. \textunderscore Eufrosina\textunderscore , 189.
\section{Polheira}
\begin{itemize}
\item {Grp. gram.:f.}
\end{itemize}
\begin{itemize}
\item {Utilização:Ant.}
\end{itemize}
Saia de mulhér, que se vestia sôbre o guardinfante.
\section{Polho}
\begin{itemize}
\item {fónica:pô}
\end{itemize}
\begin{itemize}
\item {Grp. gram.:m.}
\end{itemize}
\begin{itemize}
\item {Utilização:Ant.}
\end{itemize}
\begin{itemize}
\item {Utilização:Fig.}
\end{itemize}
Frango.
Rapaz.
Casta de uva.
(Cast. \textunderscore pollo\textunderscore )
\section{Polia}
\begin{itemize}
\item {Grp. gram.:f.}
\end{itemize}
Moléstia das plantas crucíferas, occasionada por um cogumelo, (\textunderscore cystopus candidus\textunderscore , Lev.).
\section{Polia}
\begin{itemize}
\item {Grp. gram.:f.}
\end{itemize}
\begin{itemize}
\item {Utilização:Bras}
\end{itemize}
Qualquer roda, para correia transmissoria de movimento.
(Cp. \textunderscore polé\textunderscore )
\section{Polia}
\begin{itemize}
\item {Grp. gram.:f.}
\end{itemize}
\begin{itemize}
\item {Utilização:Bras. do N}
\end{itemize}
Insecto roedor de coiro cru.
Polilha.
\section{Poliates}
\begin{itemize}
\item {Grp. gram.:m. pl.}
\end{itemize}
\begin{itemize}
\item {Utilização:Prov.}
\end{itemize}
\begin{itemize}
\item {Utilização:trasm.}
\end{itemize}
Qualquer lugar muito esconso.
\section{Polição}
\begin{itemize}
\item {Grp. gram.:f.}
\end{itemize}
\begin{itemize}
\item {Utilização:Neol.}
\end{itemize}
Acto de polir. Cf. Ortigão, \textunderscore Holanda\textunderscore , 187.
\section{Policarpo}
\begin{itemize}
\item {Grp. gram.:m.}
\end{itemize}
\begin{itemize}
\item {Proveniência:(Do gr. \textunderscore pilos\textunderscore  + \textunderscore karpos\textunderscore )}
\end{itemize}
Gênero de arbustos intertropicaes, a que pertence o jaborandi.
\section{Pólice}
\begin{itemize}
\item {Grp. gram.:f.}
\end{itemize}
\begin{itemize}
\item {Utilização:Des.}
\end{itemize}
\begin{itemize}
\item {Proveniência:(Do lat. \textunderscore pollex\textunderscore )}
\end{itemize}
O dedo pollegar. Cf. \textunderscore Viriato Trág.\textunderscore , XIV, 83.
\section{Polichinello}
\begin{itemize}
\item {Grp. gram.:m.}
\end{itemize}
\begin{itemize}
\item {Utilização:Fam.}
\end{itemize}
\begin{itemize}
\item {Proveniência:(Do fr. \textunderscore polichinelle\textunderscore )}
\end{itemize}
Boneco, títere.
Palhaço, bobo.
\section{Polichinelo}
\begin{itemize}
\item {Grp. gram.:m.}
\end{itemize}
\begin{itemize}
\item {Utilização:Fam.}
\end{itemize}
\begin{itemize}
\item {Proveniência:(Do fr. \textunderscore polichinelle\textunderscore )}
\end{itemize}
Boneco, títere.
Palhaço, bobo.
\section{Polícia}
\begin{itemize}
\item {Grp. gram.:f.}
\end{itemize}
\begin{itemize}
\item {Grp. gram.:M.}
\end{itemize}
\begin{itemize}
\item {Proveniência:(Lat. \textunderscore politia\textunderscore )}
\end{itemize}
Organização política.
Segurança pública.
Conjunto das leis que asseguram a ordem pública.
Civilização:«\textunderscore ...polícia da vossa Europa rica.\textunderscore »\textunderscore Lusíadas\textunderscore , VIII, 12.
Corporação, encarregada de fazer observar as leis concernentes á ordem pública.
Disciplina; ordem.
Indivíduo, que faz parte de uma corporação policial. Cf. Castilho, \textunderscore Misanthropo\textunderscore , 77.
\section{Policial}
\begin{itemize}
\item {Grp. gram.:adj.}
\end{itemize}
Relativo á polícia; próprio da polícia.
\section{Policiamento}
\begin{itemize}
\item {Grp. gram.:m.}
\end{itemize}
Acto ou effeito de policiar. Cf. Camillo, \textunderscore Doze Casam.\textunderscore , 155.
\section{Policiar}
\begin{itemize}
\item {Grp. gram.:v. t.}
\end{itemize}
\begin{itemize}
\item {Proveniência:(De \textunderscore polícia\textunderscore )}
\end{itemize}
Vigiar, em cumprimento de regulamentos ou leis policiaes.
Zelar.
Civilizar.
\section{Policitação}
\begin{itemize}
\item {Grp. gram.:f.}
\end{itemize}
\begin{itemize}
\item {Proveniência:(Lat. \textunderscore pollicitatio\textunderscore )}
\end{itemize}
Oferecimento; proposta; promessa.
\section{Policlínica}
\begin{itemize}
\item {Grp. gram.:f.}
\end{itemize}
\begin{itemize}
\item {Utilização:Neol.}
\end{itemize}
\begin{itemize}
\item {Proveniência:(Do gr. \textunderscore polis\textunderscore  + \textunderscore klino\textunderscore )}
\end{itemize}
Clínica, exercida nas cidades; clínica exercida fóra dos hospitaes.
\section{Polidamente}
\begin{itemize}
\item {Grp. gram.:adv.}
\end{itemize}
De modo polido.
Com cortesia.
\section{Polidez}
\begin{itemize}
\item {Grp. gram.:f.}
\end{itemize}
Qualidade do que é polido.
Delicadeza, urbanidade.
\section{Polido}
\begin{itemize}
\item {Grp. gram.:adj.}
\end{itemize}
\begin{itemize}
\item {Grp. gram.:M.}
\end{itemize}
\begin{itemize}
\item {Utilização:Veter.}
\end{itemize}
Brilhante; luzidio.
Culto; civilizado; cortês, delicado.
\textunderscore Mal de polido\textunderscore , doença dos animaes, que os faz encolerizar, deixando-os depois tristes.
\section{Polidor}
\begin{itemize}
\item {Grp. gram.:m.  e  adj.}
\end{itemize}
\begin{itemize}
\item {Proveniência:(Do lat. \textunderscore politor\textunderscore )}
\end{itemize}
O que pule.
\section{Polidura}
\begin{itemize}
\item {Grp. gram.:f.}
\end{itemize}
Acto ou effeito de polir.
\section{Polilha}
\begin{itemize}
\item {Grp. gram.:f.}
\end{itemize}
Espécie de traça.
(Cast. \textunderscore polilla\textunderscore )
\section{Polimentar}
\begin{itemize}
\item {Grp. gram.:v. i.}
\end{itemize}
\begin{itemize}
\item {Utilização:inútil}
\end{itemize}
\begin{itemize}
\item {Utilização:Neol.}
\end{itemize}
Dar polimento a; polir; illustrar. Cf. Alv. Mendes, \textunderscore Herculano\textunderscore , 38.
\section{Polimento}
\begin{itemize}
\item {Grp. gram.:m.}
\end{itemize}
\begin{itemize}
\item {Proveniência:(De \textunderscore polir\textunderscore )}
\end{itemize}
Polidura; polidez.
Coiro lustroso, de que se faz calçado.
Verniz, lustre.
\section{Polina}
\begin{itemize}
\item {Grp. gram.:f.}
\end{itemize}
\begin{itemize}
\item {Utilização:Prov.}
\end{itemize}
\begin{itemize}
\item {Utilização:beir.}
\end{itemize}
Doença, que dá nos repolhos. (Colhido no Fundão)
\section{Polinar}
\begin{itemize}
\item {Grp. gram.:adj.}
\end{itemize}
O mesmo que \textunderscore polinário\textunderscore .
\section{Polinário}
\begin{itemize}
\item {Grp. gram.:adj.}
\end{itemize}
\begin{itemize}
\item {Proveniência:(De \textunderscore pólen\textunderscore )}
\end{itemize}
Que está coberto de um pó muito fino.
\section{Polinctor}
\begin{itemize}
\item {Grp. gram.:m.}
\end{itemize}
\begin{itemize}
\item {Proveniência:(Lat. \textunderscore pollinctor\textunderscore )}
\end{itemize}
Aquele que, entre os Romanos, embalsamava cadáveres ou os lavava e preparava para a sepultura.
\section{Polinheira}
\begin{itemize}
\item {Grp. gram.:f.}
\end{itemize}
\begin{itemize}
\item {Utilização:Prov.}
\end{itemize}
\begin{itemize}
\item {Utilização:minh.}
\end{itemize}
O mesmo que \textunderscore tareia\textunderscore .
(Cp. \textunderscore póla\textunderscore ^1)
\section{Polínico}
\begin{itemize}
\item {Grp. gram.:adj.}
\end{itemize}
Relativo ao pólen; que contém pólen.
\section{Polinífero}
\begin{itemize}
\item {Grp. gram.:adj.}
\end{itemize}
\begin{itemize}
\item {Utilização:Bot.}
\end{itemize}
Que contém pólen.
\section{Polinização}
\begin{itemize}
\item {Grp. gram.:f.}
\end{itemize}
Acto ou efeito de polinizar.
\section{Polinizar}
\begin{itemize}
\item {Grp. gram.:v. t.}
\end{itemize}
Transportar o pólen, das anteras para (o estigma da planta).
\section{Polinoso}
\begin{itemize}
\item {Grp. gram.:adj.}
\end{itemize}
Coberto de um pó amarelo, semelhante ao pólen.
\section{Pólio}
\begin{itemize}
\item {Grp. gram.:m.}
\end{itemize}
Planta labiada, (\textunderscore teucrium polium\textunderscore , Lin.).
\section{Polioencefalite}
\begin{itemize}
\item {Grp. gram.:f.}
\end{itemize}
\begin{itemize}
\item {Utilização:Med.}
\end{itemize}
\begin{itemize}
\item {Proveniência:(Do gr. \textunderscore polios\textunderscore  + \textunderscore enkephalon\textunderscore )}
\end{itemize}
Lesão de certos núcleos no sistema nervoso.
\section{Polioencephalite}
\begin{itemize}
\item {Grp. gram.:f.}
\end{itemize}
\begin{itemize}
\item {Utilização:Med.}
\end{itemize}
\begin{itemize}
\item {Proveniência:(Do gr. \textunderscore polios\textunderscore  + \textunderscore enkephalon\textunderscore )}
\end{itemize}
Lesão de certos núcleos no systema nervoso.
\section{Poliomielite}
\begin{itemize}
\item {Grp. gram.:f.}
\end{itemize}
\begin{itemize}
\item {Utilização:Med.}
\end{itemize}
\begin{itemize}
\item {Proveniência:(Do gr. \textunderscore polios\textunderscore  + \textunderscore muelos\textunderscore )}
\end{itemize}
Mielite, que assenta principalmente na substância cinzenta.
\section{Poliomyelite}
\begin{itemize}
\item {Grp. gram.:f.}
\end{itemize}
\begin{itemize}
\item {Utilização:Med.}
\end{itemize}
\begin{itemize}
\item {Proveniência:(Do gr. \textunderscore polios\textunderscore  + \textunderscore muelos\textunderscore )}
\end{itemize}
Myelite, que assenta principalmente na substância cinzenta.
\section{Poliorcética}
\begin{itemize}
\item {Grp. gram.:f.}
\end{itemize}
\begin{itemize}
\item {Utilização:Ant.}
\end{itemize}
Arte de fazer cercos militares.
(Fem. de \textunderscore poliorcético\textunderscore )
\section{Poliorcético}
\begin{itemize}
\item {Grp. gram.:adj.}
\end{itemize}
\begin{itemize}
\item {Proveniência:(Gr. \textunderscore poliorketikos\textunderscore )}
\end{itemize}
Relativo á poliorcética.
\section{Poliose}
\begin{itemize}
\item {Grp. gram.:f.}
\end{itemize}
\begin{itemize}
\item {Utilização:Med.}
\end{itemize}
\begin{itemize}
\item {Proveniência:(Gr. \textunderscore poliosis\textunderscore )}
\end{itemize}
Descoramento mórbido dos pêlos.
\section{Poliquento}
\begin{itemize}
\item {Grp. gram.:adj.}
\end{itemize}
\begin{itemize}
\item {Utilização:Prov.}
\end{itemize}
Diffícil de contentar em comidas; debiqueiro.
\section{Polir}
\begin{itemize}
\item {Grp. gram.:v. t.}
\end{itemize}
\begin{itemize}
\item {Utilização:Fig.}
\end{itemize}
\begin{itemize}
\item {Proveniência:(Lat. \textunderscore polire\textunderscore )}
\end{itemize}
Tornar lustroso, friccionando.
Brunir; envernizar.
Alisar.
Engomar a polimento.
Tornar educado.
Civilizar; aperfeiçoar.
\section{Política}
\begin{itemize}
\item {Grp. gram.:f.}
\end{itemize}
Sciência do governo das nações.
Arte de dirigir as relações de um Estado com outro.
Princípios políticos.
Astúcia, artifício.
Civilidade, polidez.
(Fem. de \textunderscore político\textunderscore )
\section{Politicagem}
\begin{itemize}
\item {Grp. gram.:f.}
\end{itemize}
\begin{itemize}
\item {Utilização:Deprec.}
\end{itemize}
O mesmo que \textunderscore politiquice\textunderscore .
\section{Politicamente}
\begin{itemize}
\item {Grp. gram.:adv.}
\end{itemize}
De modo político.
Com polidez.
Maliciosamente.
\section{Politicante}
\begin{itemize}
\item {Grp. gram.:adj.}
\end{itemize}
\begin{itemize}
\item {Utilização:Deprec.}
\end{itemize}
O mesmo que \textunderscore politiqueiro\textunderscore .
\section{Politicão}
\begin{itemize}
\item {Grp. gram.:m.}
\end{itemize}
\begin{itemize}
\item {Utilização:Pop.}
\end{itemize}
Grande político.
\section{Politicar}
\begin{itemize}
\item {Grp. gram.:v. i.}
\end{itemize}
Tratar de política.
Discorrer sôbre política.
\section{Político}
\begin{itemize}
\item {Grp. gram.:adj.}
\end{itemize}
\begin{itemize}
\item {Utilização:Fig.}
\end{itemize}
\begin{itemize}
\item {Utilização:Fam.}
\end{itemize}
\begin{itemize}
\item {Grp. gram.:M.}
\end{itemize}
\begin{itemize}
\item {Proveniência:(Lat. \textunderscore politicus\textunderscore )}
\end{itemize}
Relativo aos negócios públicos.
Que se occupa de política.
Delicado, cortês.
Astuto.
Indisposto com alguém.
O que trata de política.
Estadista.
\section{Politicomania}
\begin{itemize}
\item {Grp. gram.:f.}
\end{itemize}
\begin{itemize}
\item {Proveniência:(De \textunderscore política\textunderscore  + \textunderscore mania\textunderscore )}
\end{itemize}
Mania de tratar só de política.
\section{Politicote}
\begin{itemize}
\item {Grp. gram.:m.}
\end{itemize}
\begin{itemize}
\item {Utilização:Deprec.}
\end{itemize}
Político sem valor.
Homem ordinário, que fala muito de assumptos políticos.
\section{Politiqueiro}
\begin{itemize}
\item {Grp. gram.:m.}
\end{itemize}
\begin{itemize}
\item {Proveniência:(De \textunderscore política\textunderscore )}
\end{itemize}
Designação depreciativa do indivíduo que se occupa muito da política partidária, ou que, em política, emprega processos pouco correctos. Cf. O'Neill, \textunderscore Fabul.\textunderscore , 965.
\section{Politiquete}
\begin{itemize}
\item {fónica:quê}
\end{itemize}
\begin{itemize}
\item {Grp. gram.:m.}
\end{itemize}
\begin{itemize}
\item {Utilização:Deprec.}
\end{itemize}
O mesmo que \textunderscore politiqueiro\textunderscore ; político com pouca ou nenhuma importância.
\section{Politiquice}
\begin{itemize}
\item {Grp. gram.:f.}
\end{itemize}
\begin{itemize}
\item {Utilização:Deprec.}
\end{itemize}
\begin{itemize}
\item {Proveniência:(De \textunderscore político\textunderscore )}
\end{itemize}
Acto de politiqueiro; política ordinária, mesquinha.
\section{Polito}
\begin{itemize}
\item {Grp. gram.:m.}
\end{itemize}
Planta da serra de Sintra.
\section{Pôlla}
\begin{itemize}
\item {Grp. gram.:f.}
\end{itemize}
Ramo de árvore, pernada.
Rebento.
Estaca.
\section{Pollaciuria}
\begin{itemize}
\item {fónica:ci-u}
\end{itemize}
\begin{itemize}
\item {Grp. gram.:f.}
\end{itemize}
\begin{itemize}
\item {Utilização:Med.}
\end{itemize}
\begin{itemize}
\item {Proveniência:(Do gr. \textunderscore pollakis\textunderscore  + \textunderscore ourein\textunderscore )}
\end{itemize}
Necessidade imperiosa e frequente de urinar.
\section{Pollarda}
\begin{itemize}
\item {Grp. gram.:f.}
\end{itemize}
\begin{itemize}
\item {Utilização:Des.}
\end{itemize}
\begin{itemize}
\item {Proveniência:(Fr. \textunderscore poularde\textunderscore , de \textunderscore poule\textunderscore  = cast. \textunderscore polla\textunderscore )}
\end{itemize}
Franga crescida, quási gallinha:«\textunderscore venha a gorda pollarda co'a omeleta...\textunderscore »Filinto, II, 4.
\section{Pollegada}
\begin{itemize}
\item {Grp. gram.:f.}
\end{itemize}
\begin{itemize}
\item {Proveniência:(Do b. lat. \textunderscore pollicata\textunderscore , de \textunderscore pollex\textunderscore )}
\end{itemize}
Medida, proximamente igual ao comprimento da segunda phalange do pollegar.
\section{Pollegar}
\begin{itemize}
\item {Grp. gram.:m.  e  adj.}
\end{itemize}
\begin{itemize}
\item {Proveniência:(Lat. \textunderscore pollicaris\textunderscore )}
\end{itemize}
O dedo mais curto e grosso da mão.
O primeiro e mais grosso do pé.
Pequena vara de poda, só com um, dois ou três olhos.
\section{Póllen}
\begin{itemize}
\item {Grp. gram.:m.}
\end{itemize}
\begin{itemize}
\item {Proveniência:(Lat. \textunderscore pollen\textunderscore )}
\end{itemize}
Substância fecundante dos vegetaes, contida na anthera.
\section{Pollênia}
\begin{itemize}
\item {Grp. gram.:f.}
\end{itemize}
\begin{itemize}
\item {Proveniência:(De \textunderscore póllen\textunderscore )}
\end{itemize}
Gênero de insectos dípteros.
\section{Póllex}
\begin{itemize}
\item {Grp. gram.:m.}
\end{itemize}
\begin{itemize}
\item {Proveniência:(Lat. \textunderscore pollex\textunderscore )}
\end{itemize}
O mesmo que \textunderscore pollegar\textunderscore .
\section{Póllice}
\begin{itemize}
\item {Grp. gram.:f.}
\end{itemize}
\begin{itemize}
\item {Utilização:Des.}
\end{itemize}
\begin{itemize}
\item {Proveniência:(Do lat. \textunderscore pollex\textunderscore )}
\end{itemize}
O dedo pollegar. Cf. \textunderscore Viriato Trág.\textunderscore , XIV, 83.
\section{Pollicitação}
\begin{itemize}
\item {Grp. gram.:f.}
\end{itemize}
\begin{itemize}
\item {Proveniência:(Lat. \textunderscore pollicitatio\textunderscore )}
\end{itemize}
Offerecimento; proposta; promessa.
\section{Pollinar}
\begin{itemize}
\item {Grp. gram.:adj.}
\end{itemize}
O mesmo que \textunderscore pollinário\textunderscore .
\section{Pollinário}
\begin{itemize}
\item {Grp. gram.:adj.}
\end{itemize}
\begin{itemize}
\item {Proveniência:(De \textunderscore póllen\textunderscore )}
\end{itemize}
Que está coberto de um pó muito fino.
\section{Pollinctor}
\begin{itemize}
\item {Grp. gram.:m.}
\end{itemize}
\begin{itemize}
\item {Proveniência:(Lat. \textunderscore pollinctor\textunderscore )}
\end{itemize}
Aquelle que, entre os Romanos, embalsamava cadáveres ou os lavava e preparava para a sepultura.
\section{Pollínico}
\begin{itemize}
\item {Grp. gram.:adj.}
\end{itemize}
Relativo ao póllen; que contém póllen.
\section{Pollinífero}
\begin{itemize}
\item {Grp. gram.:adj.}
\end{itemize}
\begin{itemize}
\item {Utilização:Bot.}
\end{itemize}
Que contém póllen.
\section{Pollinização}
\begin{itemize}
\item {Grp. gram.:f.}
\end{itemize}
Acto ou effeito de pollinizar.
\section{Pollinizar}
\begin{itemize}
\item {Grp. gram.:v. t.}
\end{itemize}
Transportar o póllen, das antheras para (o estigma da planta).
\section{Pollinoso}
\begin{itemize}
\item {Grp. gram.:adj.}
\end{itemize}
Coberto de um pó amarelo, semelhante ao póllen.
\section{Pôllo}
\begin{itemize}
\item {Grp. gram.:m.}
\end{itemize}
Falcão, açôr ou gavião, que ainda não tem um anno. Cf. Fernandes, \textunderscore Caça de Altan.\textunderscore 
Cp. \textunderscore polho\textunderscore .
(Cast. \textunderscore pollo\textunderscore )
\section{Pollução}
\begin{itemize}
\item {Grp. gram.:f.}
\end{itemize}
\begin{itemize}
\item {Proveniência:(Lat. \textunderscore pollutio\textunderscore )}
\end{itemize}
Acto de polluir.
Emissão involuntária de esperma.
\section{Polluição}
\begin{itemize}
\item {fónica:lu-i}
\end{itemize}
\begin{itemize}
\item {Grp. gram.:f.}
\end{itemize}
Acto ou effeito de polluir. Cf. C. Lobo, \textunderscore Sát. de Juv.\textunderscore , I, 217.
\section{Polluir}
\begin{itemize}
\item {Grp. gram.:v. t.}
\end{itemize}
\begin{itemize}
\item {Proveniência:(Lat. \textunderscore polluere\textunderscore )}
\end{itemize}
Manchar; corromper; deslustrar.
\section{Polluível}
\begin{itemize}
\item {Grp. gram.:adj.}
\end{itemize}
Que se póde polluir.
\section{Polluto}
\begin{itemize}
\item {Grp. gram.:adj.}
\end{itemize}
\begin{itemize}
\item {Proveniência:(Lat. \textunderscore pollutus\textunderscore )}
\end{itemize}
Manchado.
Corrompido.
\section{Póllux}
\begin{itemize}
\item {Grp. gram.:m.}
\end{itemize}
\begin{itemize}
\item {Proveniência:(De \textunderscore Pollux\textunderscore , n. p. myth.)}
\end{itemize}
Estrêlla de segunda grandeza.
\section{Polmão}
\begin{itemize}
\item {Grp. gram.:m.}
\end{itemize}
\begin{itemize}
\item {Utilização:Pop.}
\end{itemize}
\begin{itemize}
\item {Proveniência:(Lat. \textunderscore pulmo\textunderscore )}
\end{itemize}
Inchação; fleimão.
\section{Polme}
\begin{itemize}
\item {Grp. gram.:m.}
\end{itemize}
\begin{itemize}
\item {Proveniência:(Do rad. do lat. \textunderscore pulmentum\textunderscore )}
\end{itemize}
Massa, um pouco líquida.
\section{Polmo}
\begin{itemize}
\item {Grp. gram.:m.}
\end{itemize}
\begin{itemize}
\item {Utilização:Prov.}
\end{itemize}
\begin{itemize}
\item {Utilização:trasm.}
\end{itemize}
Turvação, produzida num líquido pela presença de corpúsculos estranhos.
O mesmo que \textunderscore oídio\textunderscore .
(Alter. de \textunderscore polme\textunderscore )
\section{Pôlo}
\begin{itemize}
\item {Grp. gram.:m.}
\end{itemize}
Falcão, açôr ou gavião, que ainda não tem um anno. Cf. Fernandes, \textunderscore Caça de Altan.\textunderscore 
Cp. \textunderscore polho\textunderscore .
(Cast. \textunderscore pollo\textunderscore )
\section{Pólo}
\begin{itemize}
\item {Grp. gram.:m.}
\end{itemize}
\begin{itemize}
\item {Proveniência:(Lat. \textunderscore polus\textunderscore )}
\end{itemize}
Cada uma das duas extremidades do eixo racional, em torno do qual parece girar a Terra.
Regiões, vizinhas dessas extremidades.
Cada um dos dois pontos oppostos de um íman ou de uma pilha.
Aquillo que dirige ou encaminha; norte, guia.
\section{Poló}
\begin{itemize}
\item {Grp. gram.:m.}
\end{itemize}
Tecido indiano, usado em ornatos de senhoras. Cf. Th. Ribeiro, \textunderscore Jornadas\textunderscore , II, 36 e 93.
\section{Polo}
\begin{itemize}
\item {fónica:pulu}
\end{itemize}
(\textunderscore pulu...\textunderscore , procliticamente)«\textunderscore Polo meo desta cidade vay ũa ribeira...\textunderscore »Tenreiro, \textunderscore Itiner.\textunderscore , XV. Cf. \textunderscore Filodemo\textunderscore , II, 3; G. Vicente, \textunderscore Auto da Feira\textunderscore ; etc.--Dizem os diccion. que se lê \textunderscore pôlo\textunderscore . É todavia uma fórma proclitica, (\textunderscore polo vêr\textunderscore , \textunderscore polos modos\textunderscore ), podendo concluír-se que a verdadeira pronúncia, ainda hoje reflectida na linguagem popular, é \textunderscore pu-lu...\textunderscore 
\section{Polografia}
\begin{itemize}
\item {Grp. gram.:f.}
\end{itemize}
\begin{itemize}
\item {Proveniência:(Do gr. \textunderscore polos\textunderscore  + \textunderscore graphein\textunderscore )}
\end{itemize}
Descrição astronómica do céu.
\section{Polographia}
\begin{itemize}
\item {Grp. gram.:f.}
\end{itemize}
\begin{itemize}
\item {Proveniência:(Do gr. \textunderscore polos\textunderscore  + \textunderscore graphein\textunderscore )}
\end{itemize}
Descripção astronómica do céu.
\section{Polonesa}
\begin{itemize}
\item {Grp. gram.:f.}
\end{itemize}
Casaco largo e comprido, para senhora.
Redingote.
(Aportuguesamento do fr. \textunderscore polonaise\textunderscore )
\section{Polónia}
\begin{itemize}
\item {Grp. gram.:f.}
\end{itemize}
Casta de uva de Azeitão. Cf. \textunderscore Rev. Agron.\textunderscore , I, 18.
\section{Polónico}
\begin{itemize}
\item {Grp. gram.:adj.}
\end{itemize}
\begin{itemize}
\item {Proveniência:(De \textunderscore Polónia\textunderscore , n. p.?)}
\end{itemize}
Diz-se de uma espécie de trigo.
\section{Polónio}
\begin{itemize}
\item {Grp. gram.:m.}
\end{itemize}
O mesmo que \textunderscore polaco\textunderscore . Cf. \textunderscore Lusíadas\textunderscore , III, 78.
\section{Polónio}
\begin{itemize}
\item {Grp. gram.:m.}
\end{itemize}
Espécie de crystal, que Curie chama bismutho activo, e que se extrai da pechurano.
\section{Polpa}
\begin{itemize}
\item {fónica:pôl}
\end{itemize}
\begin{itemize}
\item {Grp. gram.:f.}
\end{itemize}
\begin{itemize}
\item {Utilização:Fig.}
\end{itemize}
\begin{itemize}
\item {Proveniência:(Do lat. \textunderscore pulpa\textunderscore )}
\end{itemize}
Carne musculosa dos animaes, sem ossos nem gorduras.
Substância carnuda dos frutos, raízes, etc.
Importância, valimento pessoal.
\section{Polpação}
\begin{itemize}
\item {Grp. gram.:f.}
\end{itemize}
Reducção de certas substâncias a polpa.
\section{Polpão}
\begin{itemize}
\item {Grp. gram.:m.}
\end{itemize}
\begin{itemize}
\item {Utilização:Ant.}
\end{itemize}
\begin{itemize}
\item {Proveniência:(De \textunderscore polpa\textunderscore )}
\end{itemize}
A parte mais carnuda das coxas.
\section{Polpo}
\begin{itemize}
\item {fónica:pôl}
\end{itemize}
\begin{itemize}
\item {Grp. gram.:m.}
\end{itemize}
\begin{itemize}
\item {Utilização:Ant.}
\end{itemize}
O mesmo que \textunderscore polvo\textunderscore :«\textunderscore ...o contrafazer do bogio, as côres do polpo, as lágrimas do crocodilo...\textunderscore »\textunderscore Eufrosina\textunderscore , 69.
\section{Pólpoda}
\begin{itemize}
\item {Grp. gram.:f.}
\end{itemize}
Gênero de plantas portuláceas.
\section{Polposo}
\begin{itemize}
\item {Grp. gram.:adj.}
\end{itemize}
\begin{itemize}
\item {Proveniência:(Lat. \textunderscore pulposus\textunderscore )}
\end{itemize}
Que tem muita polpa; carnudo.
\section{Polpudo}
\begin{itemize}
\item {Grp. gram.:adj.}
\end{itemize}
O mesmo que \textunderscore polposo\textunderscore .
\section{Polquista}
\begin{itemize}
\item {Grp. gram.:m.  e  f.}
\end{itemize}
Pessôa, que dança a polca.
\section{Poltranaz}
\begin{itemize}
\item {Grp. gram.:m.}
\end{itemize}
\begin{itemize}
\item {Utilização:Fam.}
\end{itemize}
Grande poltrão.
\section{Polução}
\begin{itemize}
\item {Grp. gram.:f.}
\end{itemize}
\begin{itemize}
\item {Proveniência:(Lat. \textunderscore pollutio\textunderscore )}
\end{itemize}
Acto de poluir.
Emissão involuntária de esperma.
\section{Poluição}
\begin{itemize}
\item {fónica:lu-i}
\end{itemize}
\begin{itemize}
\item {Grp. gram.:f.}
\end{itemize}
Acto ou efeito de poluir. Cf. C. Lobo, \textunderscore Sát. de Juv.\textunderscore , I, 217.
\section{Poluir}
\begin{itemize}
\item {Grp. gram.:v. t.}
\end{itemize}
\begin{itemize}
\item {Proveniência:(Lat. \textunderscore polluere\textunderscore )}
\end{itemize}
Manchar; corromper; deslustrar.
\section{Poluível}
\begin{itemize}
\item {Grp. gram.:adj.}
\end{itemize}
Que se póde poluir.
\section{Poluto}
\begin{itemize}
\item {Grp. gram.:adj.}
\end{itemize}
\begin{itemize}
\item {Proveniência:(Lat. \textunderscore pollutus\textunderscore )}
\end{itemize}
Manchado.
Corrompido.
\section{Polux}
\begin{itemize}
\item {Grp. gram.:m.}
\end{itemize}
\begin{itemize}
\item {Proveniência:(De \textunderscore Pollux\textunderscore , n. p. myth.)}
\end{itemize}
Estrêla de segunda grandeza.
\section{Poli...}
\begin{itemize}
\item {Grp. gram.:pref.}
\end{itemize}
\begin{itemize}
\item {Proveniência:(Do gr. \textunderscore polus\textunderscore )}
\end{itemize}
(designativo de número indefinido e elevado)
\section{Poliacanto}
\begin{itemize}
\item {Grp. gram.:adj.}
\end{itemize}
\begin{itemize}
\item {Proveniência:(Do gr. \textunderscore polus\textunderscore  + \textunderscore akantha\textunderscore )}
\end{itemize}
Que tem muitos espinhos.
\section{Poliácido}
\begin{itemize}
\item {Grp. gram.:adj.}
\end{itemize}
\begin{itemize}
\item {Utilização:Chím.}
\end{itemize}
\begin{itemize}
\item {Proveniência:(De \textunderscore polus\textunderscore  gr. + \textunderscore ácido\textunderscore )}
\end{itemize}
Diz-se das bases, em que uma molécula satura muitas moléculas de um ácido.
\section{Poliadelfia}
\begin{itemize}
\item {Grp. gram.:f.}
\end{itemize}
\begin{itemize}
\item {Utilização:Bot.}
\end{itemize}
\begin{itemize}
\item {Proveniência:(Do gr. \textunderscore polus\textunderscore  + \textunderscore adelphos\textunderscore )}
\end{itemize}
União dos estames da flôr, formando muitos feixes.
\section{Poliadelfito}
\begin{itemize}
\item {Grp. gram.:m.}
\end{itemize}
\begin{itemize}
\item {Utilização:Miner.}
\end{itemize}
Silicato múltiplo, que forma massas compostas de grãos arredondados e diversamente coloridos.
(Cp. \textunderscore poliadelfia\textunderscore )
\section{Poliadelfo}
\begin{itemize}
\item {Grp. gram.:adj.}
\end{itemize}
\begin{itemize}
\item {Utilização:Bot.}
\end{itemize}
\begin{itemize}
\item {Proveniência:(Do gr. \textunderscore polus\textunderscore  + \textunderscore adelphos\textunderscore )}
\end{itemize}
Diz-se dos estames, quando soldados pelos seus filetes em mais de dois fascículos.
\section{Poliadênia}
\begin{itemize}
\item {Grp. gram.:f.}
\end{itemize}
\begin{itemize}
\item {Proveniência:(Do gr. \textunderscore polus\textunderscore  + \textunderscore aden\textunderscore )}
\end{itemize}
Género de árvores lauráceas da Índia.
\section{Poliaminas}
\begin{itemize}
\item {Grp. gram.:f. pl.}
\end{itemize}
\begin{itemize}
\item {Utilização:Chím.}
\end{itemize}
\begin{itemize}
\item {Proveniência:(De \textunderscore poli...\textunderscore  + \textunderscore amínas\textunderscore )}
\end{itemize}
Aminas, derivadas de muitas moléculas de ammoníaco, condensadas.
\section{Poliandra}
\begin{itemize}
\item {Grp. gram.:adj.}
\end{itemize}
\begin{itemize}
\item {Proveniência:(Do gr. \textunderscore polus\textunderscore  + \textunderscore aner\textunderscore , \textunderscore andros\textunderscore )}
\end{itemize}
Diz-se da mulhér, que tem mais de um marido ao mesmo tempo.
\section{Poliândria}
\begin{itemize}
\item {Grp. gram.:f.}
\end{itemize}
Classe botânica de Linneu, que compreende os vegetaes provídos de 20 ou mais estames, inseridos sob um pistilo simples ou múltiplo.
\section{Poliandría}
\begin{itemize}
\item {Grp. gram.:f.}
\end{itemize}
Estado ou qualidade de poliandro.
\section{Poliândrico}
\begin{itemize}
\item {Grp. gram.:adj.}
\end{itemize}
Relativo á poliândria.
\section{Poliândrio}
\begin{itemize}
\item {Grp. gram.:m.}
\end{itemize}
\begin{itemize}
\item {Proveniência:(Gr. \textunderscore poluandrion\textunderscore )}
\end{itemize}
Túmulo comum de muitos guerreiros, entre os Gregos.
\section{Poliandro}
\begin{itemize}
\item {Grp. gram.:adj.}
\end{itemize}
\begin{itemize}
\item {Utilização:Bot.}
\end{itemize}
\begin{itemize}
\item {Proveniência:(Do gr. \textunderscore polus\textunderscore  + \textunderscore aner\textunderscore , \textunderscore andros\textunderscore )}
\end{itemize}
Que tem mais de doze estames, ou muitos, todos livres entre si.
\section{Polianita}
\begin{itemize}
\item {Grp. gram.:f.}
\end{itemize}
\begin{itemize}
\item {Utilização:Miner.}
\end{itemize}
Bióxido de manganés natural.
\section{Poliantacarpia}
\begin{itemize}
\item {Grp. gram.:f.}
\end{itemize}
O mesmo que \textunderscore policladia\textunderscore .
\section{Polianto}
\begin{itemize}
\item {Grp. gram.:adj.}
\end{itemize}
\begin{itemize}
\item {Proveniência:(Do gr. \textunderscore polus\textunderscore  + \textunderscore anthos\textunderscore )}
\end{itemize}
Que tem muitas flôres.
\section{Poliarino}
\begin{itemize}
\item {Grp. gram.:adj.}
\end{itemize}
(V.poliândrico)
\section{Poliarquia}
\begin{itemize}
\item {Grp. gram.:f.}
\end{itemize}
\begin{itemize}
\item {Proveniência:(Do gr. \textunderscore polus\textunderscore  + \textunderscore arkhe\textunderscore )}
\end{itemize}
Govêrno exercido por muitos.
\section{Poliarticular}
\begin{itemize}
\item {Grp. gram.:adj.}
\end{itemize}
\begin{itemize}
\item {Proveniência:(De \textunderscore poli...\textunderscore  + \textunderscore articular\textunderscore )}
\end{itemize}
Diz-se do reumatismo, que ataca muitas articulações.
\section{Poliatómico}
\begin{itemize}
\item {Grp. gram.:adj.}
\end{itemize}
\begin{itemize}
\item {Utilização:Chím.}
\end{itemize}
\begin{itemize}
\item {Proveniência:(De \textunderscore poli...\textunderscore  + \textunderscore atómico\textunderscore )}
\end{itemize}
Cujas combinações se realizam, na relação de muitos equivalentes dos corpos que se unem.
\section{Polibásico}
\begin{itemize}
\item {Grp. gram.:adj.}
\end{itemize}
\begin{itemize}
\item {Proveniência:(De \textunderscore poli...\textunderscore  + \textunderscore básico\textunderscore )}
\end{itemize}
Diz-se dos ácidos, que contêm muitas moléculas de água básica.
\section{Polibasito}
\begin{itemize}
\item {Grp. gram.:m.}
\end{itemize}
\begin{itemize}
\item {Utilização:Miner.}
\end{itemize}
Prata sulfurada, que contém certa porção de arsênico, antimónio, cobre, etc., e procedente do México.
\section{Poliblenia}
\begin{itemize}
\item {Grp. gram.:f.}
\end{itemize}
\begin{itemize}
\item {Utilização:Med.}
\end{itemize}
\begin{itemize}
\item {Proveniência:(Do gr. \textunderscore polus\textunderscore  + \textunderscore blenna\textunderscore )}
\end{itemize}
Excessivo corrimento de mucosidades.
\section{Policárpico}
\begin{itemize}
\item {Grp. gram.:adj.}
\end{itemize}
\begin{itemize}
\item {Utilização:Bot.}
\end{itemize}
Que dá flôres e frutos por mais de uma vez.
(Cp. \textunderscore policarpo\textunderscore ^2)
\section{Policarpo}
\begin{itemize}
\item {Grp. gram.:adj.}
\end{itemize}
\begin{itemize}
\item {Proveniência:(Do gr. \textunderscore polus\textunderscore  + \textunderscore karpos\textunderscore )}
\end{itemize}
Que tem ou produz muitos frutos.
\section{Policelular}
\begin{itemize}
\item {Grp. gram.:adj.}
\end{itemize}
\begin{itemize}
\item {Proveniência:(De \textunderscore poli...\textunderscore  + \textunderscore celular\textunderscore )}
\end{itemize}
Composto de muitas células.
\section{Policêntrico}
\begin{itemize}
\item {Grp. gram.:adj.}
\end{itemize}
\begin{itemize}
\item {Utilização:Constr.}
\end{itemize}
\begin{itemize}
\item {Proveniência:(De \textunderscore poli...\textunderscore  + \textunderscore centro\textunderscore )}
\end{itemize}
Diz-se da espiral com muitos centros.
Diz-se do arco, que tem vários centros.
\section{Policolia}
\begin{itemize}
\item {Grp. gram.:f.}
\end{itemize}
\begin{itemize}
\item {Proveniência:(Do gr. \textunderscore polus\textunderscore  + \textunderscore khole\textunderscore )}
\end{itemize}
Secreção biliosa, exagerada.
\section{Policótomo}
\begin{itemize}
\item {Grp. gram.:adj.}
\end{itemize}
\begin{itemize}
\item {Proveniência:(Do gr. \textunderscore polukhos\textunderscore  + \textunderscore tome\textunderscore )}
\end{itemize}
Diz-se de um corpo, dividido em muitas articulações.
\section{Policresto}
\begin{itemize}
\item {Grp. gram.:adj.}
\end{itemize}
\begin{itemize}
\item {Proveniência:(Gr. \textunderscore polukhrestos\textunderscore )}
\end{itemize}
Que serve para muitos usos.
\section{Policromático}
\begin{itemize}
\item {Grp. gram.:adj.}
\end{itemize}
O mesmo que \textunderscore policromo\textunderscore .
\section{Policromia}
\begin{itemize}
\item {Grp. gram.:f.}
\end{itemize}
\begin{itemize}
\item {Proveniência:(De \textunderscore policromo\textunderscore )}
\end{itemize}
Estado de um corpo, em que há diferentes côres.
Conjunto de várias côres.
\section{Policrómico}
\begin{itemize}
\item {Grp. gram.:adj.}
\end{itemize}
\begin{itemize}
\item {Proveniência:(Do gr. \textunderscore polus\textunderscore  + \textunderscore khroma\textunderscore )}
\end{itemize}
Que tem muitas côres.
Em que se empregam muitas côres.
\section{Policromo}
\begin{itemize}
\item {Grp. gram.:adj.}
\end{itemize}
\begin{itemize}
\item {Proveniência:(Do gr. \textunderscore polus\textunderscore  + \textunderscore khroma\textunderscore )}
\end{itemize}
Que tem muitas côres.
Em que se empregam muitas côres.
\section{Polícrono}
\begin{itemize}
\item {Grp. gram.:adj.}
\end{itemize}
\begin{itemize}
\item {Utilização:Neol.}
\end{itemize}
\begin{itemize}
\item {Proveniência:(Do gr. \textunderscore polus\textunderscore  + \textunderscore khronos\textunderscore )}
\end{itemize}
O mesmo que \textunderscore duradoiro\textunderscore .
\section{Policitemia}
\begin{itemize}
\item {Grp. gram.:f.}
\end{itemize}
\begin{itemize}
\item {Utilização:Med.}
\end{itemize}
\begin{itemize}
\item {Proveniência:(Do gr. \textunderscore polus\textunderscore  + \textunderscore kutos\textunderscore  + \textunderscore haima\textunderscore )}
\end{itemize}
Aumento do número dos glóbulos sanguíneos.
\section{Policladia}
\begin{itemize}
\item {Grp. gram.:f.}
\end{itemize}
\begin{itemize}
\item {Proveniência:(Do gr. \textunderscore polus\textunderscore  + \textunderscore klados\textunderscore )}
\end{itemize}
Moléstia vegetal, em que a seiva, abandonando órgãos da fructificação, se dirige para os ramos, dando-lhes prodigioso desenvolvimento.
\section{Policlado}
\begin{itemize}
\item {Grp. gram.:adj.}
\end{itemize}
Diz-se do vegetal, que dá muitos ramos.
(Cp. \textunderscore policladia\textunderscore )
\section{Policladodia}
\begin{itemize}
\item {Grp. gram.:f.}
\end{itemize}
(V.policladia)
\section{Policlínica}
\begin{itemize}
\item {Grp. gram.:f.}
\end{itemize}
\begin{itemize}
\item {Proveniência:(De \textunderscore poli...\textunderscore  + \textunderscore clínica\textunderscore )}
\end{itemize}
Prática da Medicina, aplicada á generalidade das doenças.
\section{Policlínico}
\begin{itemize}
\item {Grp. gram.:m.}
\end{itemize}
Clínico, que trata das doenças em geral, ou que se não dedica especialmente a uma.
(Cp. \textunderscore policlínica\textunderscore ^2)
\section{Polícomo}
\begin{itemize}
\item {Grp. gram.:adj.}
\end{itemize}
\begin{itemize}
\item {Proveniência:(Do gr. \textunderscore polus\textunderscore  + \textunderscore kome\textunderscore )}
\end{itemize}
Que tem muitos cabelos.
\section{Policónico}
\begin{itemize}
\item {Grp. gram.:adj.}
\end{itemize}
\begin{itemize}
\item {Proveniência:(De \textunderscore poli...\textunderscore  + \textunderscore cónico\textunderscore )}
\end{itemize}
Que tem muitos cónes.
\section{Policórdio}
\begin{itemize}
\item {Grp. gram.:m.}
\end{itemize}
\begin{itemize}
\item {Proveniência:(De \textunderscore poli...\textunderscore  + \textunderscore corda\textunderscore )}
\end{itemize}
Antigo instrumento, que se tocava com arco.
\section{Policoria}
\begin{itemize}
\item {Grp. gram.:f.}
\end{itemize}
\begin{itemize}
\item {Utilização:Med.}
\end{itemize}
\begin{itemize}
\item {Proveniência:(Do gr. \textunderscore polus\textunderscore  + \textunderscore kore\textunderscore )}
\end{itemize}
Pupila múltipla.
\section{Policotiledónio}
\begin{itemize}
\item {Grp. gram.:adj.}
\end{itemize}
\begin{itemize}
\item {Proveniência:(De \textunderscore póli...\textunderscore  + \textunderscore cotiledónio\textunderscore )}
\end{itemize}
Que tem mais de dois cotilédones.
\section{Policultura}
\begin{itemize}
\item {Grp. gram.:f.}
\end{itemize}
\begin{itemize}
\item {Utilização:Agr.}
\end{itemize}
\begin{itemize}
\item {Proveniência:(De \textunderscore poli...\textunderscore  + \textunderscore cultura\textunderscore )}
\end{itemize}
Cultura variada.
\section{Polidáctilo}
\begin{itemize}
\item {Grp. gram.:adj.}
\end{itemize}
\begin{itemize}
\item {Proveniência:(Do gr. \textunderscore polus\textunderscore  + \textunderscore daktulos\textunderscore )}
\end{itemize}
Que tem muitos dedos.
\section{Polidipsia}
\begin{itemize}
\item {Grp. gram.:f.}
\end{itemize}
\begin{itemize}
\item {Utilização:Med.}
\end{itemize}
\begin{itemize}
\item {Proveniência:(Do gr. \textunderscore polus\textunderscore  + \textunderscore dipsa\textunderscore )}
\end{itemize}
Sêde excessiva.
\section{Polídimo}
\begin{itemize}
\item {Grp. gram.:m.}
\end{itemize}
\begin{itemize}
\item {Utilização:Geol.}
\end{itemize}
\begin{itemize}
\item {Proveniência:(Gr. \textunderscore poludumos\textunderscore )}
\end{itemize}
Macla, composta de quatro ou mais indivíduos. Cf. Gonç. Guimarães, \textunderscore Geol.\textunderscore , 60.
\section{Poliédrico}
\begin{itemize}
\item {Grp. gram.:adj.}
\end{itemize}
Que tem a fórma de poliedro.
\section{Poliedro}
\begin{itemize}
\item {Grp. gram.:adj.}
\end{itemize}
\begin{itemize}
\item {Grp. gram.:M.}
\end{itemize}
\begin{itemize}
\item {Proveniência:(Do gr. \textunderscore polus\textunderscore  + \textunderscore edra\textunderscore )}
\end{itemize}
Que tem muitas faces planas.
Sólido poliedro.
\section{Poliemia}
\begin{itemize}
\item {Grp. gram.:f.}
\end{itemize}
\begin{itemize}
\item {Utilização:Med.}
\end{itemize}
\begin{itemize}
\item {Proveniência:(Do gr. \textunderscore polus\textunderscore  + \textunderscore haima\textunderscore )}
\end{itemize}
Pletora sanguínea; excesso de sangue, nos vasos do corpo. Cf. Macedo Pinto, \textunderscore Comp. de Veter.\textunderscore , I, 76.
\section{Poliestesia}
\begin{itemize}
\item {Grp. gram.:f.}
\end{itemize}
\begin{itemize}
\item {Utilização:Med.}
\end{itemize}
\begin{itemize}
\item {Proveniência:(Do gr. \textunderscore polus\textunderscore  + \textunderscore esthesis\textunderscore )}
\end{itemize}
Perturbação da sensibilidade, durante a qual um excitante único produz uma sensação múltipla.
\section{Polifólia}
\begin{itemize}
\item {Grp. gram.:adj. f.}
\end{itemize}
\begin{itemize}
\item {Proveniência:(Do gr. \textunderscore polus\textunderscore  + lat. \textunderscore folium\textunderscore )}
\end{itemize}
Diz-se da charrua de mais de três aivecas.
\section{Polígala}
\begin{itemize}
\item {Grp. gram.:f.}
\end{itemize}
\begin{itemize}
\item {Proveniência:(Do gr. \textunderscore polus\textunderscore  + \textunderscore gala\textunderscore )}
\end{itemize}
Gênero de plantas, de suco leitoso, das quaes há duas espécies, empregadas em Medicina.
\section{Poligaláceas}
\begin{itemize}
\item {Grp. gram.:f. pl.}
\end{itemize}
O mesmo ou melhor que \textunderscore poligáleas\textunderscore .
\section{Poligalactia}
\begin{itemize}
\item {Grp. gram.:f.}
\end{itemize}
\begin{itemize}
\item {Utilização:Med.}
\end{itemize}
\begin{itemize}
\item {Proveniência:(Do gr. \textunderscore polus\textunderscore  + \textunderscore gala\textunderscore , \textunderscore galactos\textunderscore )}
\end{itemize}
Superabundância de leite.
\section{Poligáleas}
\begin{itemize}
\item {Grp. gram.:f. pl.}
\end{itemize}
Família de plantas, que tem por tipo a polígala.
(Fem. pl. de \textunderscore poligáleo\textunderscore )
\section{Poligáleo}
\begin{itemize}
\item {Grp. gram.:adj.}
\end{itemize}
Relativo ou semelhante á polígala.
\section{Poligálico}
\begin{itemize}
\item {Grp. gram.:adj.}
\end{itemize}
Diz-se de um ácido, achado numa especie de polígala, e em outras plantas.
\section{Poligalíneas}
\begin{itemize}
\item {Grp. gram.:f. pl.}
\end{itemize}
Ordem de plantas, que abrange as poligáleas e outras.
(Cp. \textunderscore poligáleas\textunderscore )
\section{Poligamia}
\begin{itemize}
\item {Grp. gram.:f.}
\end{itemize}
Estado de quem é polígamo.
\section{Poligâmico}
\begin{itemize}
\item {Grp. gram.:adj.}
\end{itemize}
Relativo á poligamia.
\section{Polígamo}
\begin{itemize}
\item {Grp. gram.:m.  e  adj.}
\end{itemize}
\begin{itemize}
\item {Proveniência:(Gr. \textunderscore polugamos\textunderscore )}
\end{itemize}
Indivíduo, que tem mais de um cônjuge, ao mesmo tempo.
Diz-se também de certos animaes, em que um macho tem muitas fêmeas.
E diz-se das plantas, que têm ao mesmo tempo flôres hermafroditas e unisexuaes.
\section{Poligastricidade}
\begin{itemize}
\item {Grp. gram.:f.}
\end{itemize}
\begin{itemize}
\item {Proveniência:(De \textunderscore poligástrico\textunderscore )}
\end{itemize}
Existencia de muitos estômagos, que se julgou terem-se descoberto nos infusórios.
\section{Poligástrico}
\begin{itemize}
\item {Grp. gram.:adj.}
\end{itemize}
\begin{itemize}
\item {Proveniência:(Do gr. \textunderscore polus\textunderscore  + \textunderscore gaster\textunderscore )}
\end{itemize}
Que tem muitos estômagos.
\section{Poligastro}
\begin{itemize}
\item {Grp. gram.:adj.}
\end{itemize}
O mesmo ou melhor que \textunderscore poligástrico\textunderscore .
\section{Poligenismo}
\begin{itemize}
\item {Grp. gram.:m.}
\end{itemize}
\begin{itemize}
\item {Proveniência:(De \textunderscore polígeno\textunderscore )}
\end{itemize}
Sistema dos que atribuem as raças humanas a diferentes troncos primitivos.
\section{Polígeno}
\begin{itemize}
\item {Grp. gram.:adj.}
\end{itemize}
\begin{itemize}
\item {Proveniência:(Do gr. \textunderscore polus\textunderscore  + \textunderscore genos\textunderscore )}
\end{itemize}
Que produz muito.
\section{Poliglota}
\begin{itemize}
\item {Grp. gram.:adj.}
\end{itemize}
\begin{itemize}
\item {Grp. gram.:M.  e  f.}
\end{itemize}
\begin{itemize}
\item {Proveniência:(Gr. \textunderscore poluglottos\textunderscore )}
\end{itemize}
Que está escrito em muitas línguas.
Pessôa que sabe ou fala muitas línguas.
\section{Poliglótico}
\begin{itemize}
\item {Grp. gram.:adj.}
\end{itemize}
Escrito em muitas línguas.
Relativo a poliglota.
\section{Poliglotismo}
\begin{itemize}
\item {Grp. gram.:m.}
\end{itemize}
Qualidade de quem é poliglota.
Facilidade de falar muitas línguas.
\section{Poligloto}
\begin{itemize}
\item {Grp. gram.:m.  e  adj.}
\end{itemize}
\begin{itemize}
\item {Proveniência:(Gr. \textunderscore polluglottos\textunderscore )}
\end{itemize}
O mesmo ou melhor que \textunderscore poliglota\textunderscore .
\section{Polignatos}
\begin{itemize}
\item {Grp. gram.:m. pl.}
\end{itemize}
\begin{itemize}
\item {Proveniência:(Do gr. \textunderscore polus\textunderscore  + \textunderscore gnathos\textunderscore )}
\end{itemize}
Família de insectos, sem asas, e com muitos pares de queixos.
\section{Polígnatos}
\begin{itemize}
\item {Grp. gram.:m. pl.}
\end{itemize}
\begin{itemize}
\item {Proveniência:(Do gr. \textunderscore polus\textunderscore  + \textunderscore gnathos\textunderscore )}
\end{itemize}
Família de insectos, sem asas, e com muitos pares de queixos.
\section{Poligonáceas}
\begin{itemize}
\item {Grp. gram.:f. pl.}
\end{itemize}
O mesmo ou melhor que \textunderscore poligóneas\textunderscore .
\section{Poligonal}
\begin{itemize}
\item {Grp. gram.:adj.}
\end{itemize}
Relativo ao polígono; que tem por base um polígono; que tem muitos ângulos.
\section{Poligóneas}
\begin{itemize}
\item {Grp. gram.:f. pl.}
\end{itemize}
Família de plantas, que tem por tipo o \textunderscore polígono\textunderscore  ou \textunderscore sempre-noiva\textunderscore .
(Fem. pl. de \textunderscore poligóneo\textunderscore )
\section{Poligóneo}
\begin{itemize}
\item {Grp. gram.:adj.}
\end{itemize}
\begin{itemize}
\item {Utilização:Bot.}
\end{itemize}
Relativo ou semelhante ao polígono, planta.
\section{Polígono}
\begin{itemize}
\item {Grp. gram.:m.}
\end{itemize}
\begin{itemize}
\item {Utilização:Bot.}
\end{itemize}
\begin{itemize}
\item {Proveniência:(Gr. \textunderscore polugonos\textunderscore )}
\end{itemize}
Figura geométrica, que tem muitos ângulos e lados.
Figura, que determina a fórma geral de uma praça de guerra.
Lugar, onde os artilheiros fazem exercício com bocas de fogo.
Designação moderna da \textunderscore sempre-noiva\textunderscore , tipo das poligóneas.
\section{Poligrama}
\begin{itemize}
\item {Grp. gram.:f.}
\end{itemize}
\begin{itemize}
\item {Utilização:Neol.}
\end{itemize}
\begin{itemize}
\item {Proveniência:(Do gr. \textunderscore polus\textunderscore  + \textunderscore grana\textunderscore )}
\end{itemize}
Representação vária do mesmo som.
O mesmo que \textunderscore homofonia\textunderscore .
\section{Poligrafar}
\begin{itemize}
\item {Grp. gram.:v. t.}
\end{itemize}
Escrever com o maquinismo chamado polígrafo.
Escrever sôbre matérias diversas.
\section{Poligrafia}
\begin{itemize}
\item {Grp. gram.:f.}
\end{itemize}
\begin{itemize}
\item {Utilização:Ant.}
\end{itemize}
Colecção de obras diversas, literárias ou cientificas.
Conjunto de conhecimentos vários.
Qualidade de quem é polígrafo.
Criptógrafia.
\section{Poligráfico}
\begin{itemize}
\item {Grp. gram.:adj.}
\end{itemize}
Relativo á poligrafia.
\section{Polígrafo}
\begin{itemize}
\item {Grp. gram.:m.}
\end{itemize}
\begin{itemize}
\item {Proveniência:(Gr. \textunderscore polugraphos\textunderscore )}
\end{itemize}
Aquele que escreve sôbre matérias diversas.
Maquinismo que, movendo muitas pennas, produz ao mesmo tempo muitas cópias do mesmo escrito.
\section{Poltrão}
\begin{itemize}
\item {Grp. gram.:adj.}
\end{itemize}
\begin{itemize}
\item {Grp. gram.:M.}
\end{itemize}
\begin{itemize}
\item {Proveniência:(It. \textunderscore poltrone\textunderscore )}
\end{itemize}
Que não tem coragem; cobarde.
Homem cobarde ou tímido.
\section{Poltrona}
\begin{itemize}
\item {Grp. gram.:f.}
\end{itemize}
\begin{itemize}
\item {Proveniência:(It. \textunderscore poltrona\textunderscore )}
\end{itemize}
Grande cadeira de braços, geralmente estofada.
Sella com arções baixos.
\section{Poltronaria}
\begin{itemize}
\item {Grp. gram.:f.}
\end{itemize}
Qualidade ou acto de poltrão.
\section{Poltronear}
\begin{itemize}
\item {Grp. gram.:v. i.}
\end{itemize}
Portar-se como poltrão; dar mostras de poltrão.
\section{Poltronear-se}
\begin{itemize}
\item {Grp. gram.:v. p.}
\end{itemize}
Recostar-se em poltrona; repimpar-se.
\section{Polvadeira}
\begin{itemize}
\item {Grp. gram.:f.}
\end{itemize}
\begin{itemize}
\item {Utilização:Bras}
\end{itemize}
\begin{itemize}
\item {Proveniência:(Do cast. \textunderscore polvo\textunderscore  = lat. \textunderscore pulvis\textunderscore )}
\end{itemize}
Poeirada.
\section{Polvarinho}
\begin{itemize}
\item {Grp. gram.:m.}
\end{itemize}
(Corr. de \textunderscore polvorinho\textunderscore )
\section{Polveroso}
\begin{itemize}
\item {Grp. gram.:adj.}
\end{itemize}
(Fórma, us. por Filinto, X, 21, em vez de \textunderscore pulveroso\textunderscore )
(Cp. \textunderscore polvilho\textunderscore )
\section{Polvilhação}
\begin{itemize}
\item {Grp. gram.:f.}
\end{itemize}
Acto ou effeito de polvilhar. Cf. Ortigão, \textunderscore Holanda\textunderscore , 79.
\section{Polvilhar}
\begin{itemize}
\item {Grp. gram.:v. t.}
\end{itemize}
\begin{itemize}
\item {Proveniência:(De \textunderscore polvilho\textunderscore )}
\end{itemize}
Cobrir de pó, empoar; enfarinhar.
\section{Polvilho}
\begin{itemize}
\item {Grp. gram.:m.}
\end{itemize}
\begin{itemize}
\item {Grp. gram.:Pl.}
\end{itemize}
\begin{itemize}
\item {Utilização:Bras}
\end{itemize}
Resíduo da lavagem da tapioca.
Pós, com que se branqueava o cabello, especialmente o cabello das damas, em festas carnavalescas.
Qualquer substância em pó, de applicação medicamentosa, culinária, etc.
Tapioca.
(Cast. \textunderscore polvillo\textunderscore )
\section{Polvo}
\begin{itemize}
\item {fónica:pôl}
\end{itemize}
\begin{itemize}
\item {Grp. gram.:m.}
\end{itemize}
\begin{itemize}
\item {Proveniência:(Do lat. \textunderscore polypus\textunderscore )}
\end{itemize}
Mollusco cephalópode.
\section{Pólvora}
\begin{itemize}
\item {Grp. gram.:f.}
\end{itemize}
\begin{itemize}
\item {Utilização:Bras}
\end{itemize}
\begin{itemize}
\item {Proveniência:(Do lat. \textunderscore pulvera\textunderscore )}
\end{itemize}
Substância explosiva, composta de salitre, carvão e enxôfre.
Mosquito, semelhante a um grão de pólvora.
Variedade de chá, o mesmo que \textunderscore pérola\textunderscore .
\section{Polvorada}
\begin{itemize}
\item {Grp. gram.:f.}
\end{itemize}
Explosão de pólvora; fumo de pólvora. Cf. V. de Seabra, \textunderscore Sát. e Ep.\textunderscore , II, 58.
\section{Polvoraria}
\begin{itemize}
\item {Grp. gram.:f.}
\end{itemize}
Fábrica de pólvora.
\section{Polvoreda}
\begin{itemize}
\item {fónica:vorê}
\end{itemize}
\begin{itemize}
\item {Grp. gram.:f.}
\end{itemize}
O mesmo que \textunderscore polvorada\textunderscore . Cf. Arn. Gama, \textunderscore Segr. do Abb.\textunderscore , 317.
\section{Polvorim}
\begin{itemize}
\item {Grp. gram.:m.}
\end{itemize}
Pólvora, de grão muito miúdo.
Pó, que sái da pólvora.
\section{Polvorinho}
\begin{itemize}
\item {Grp. gram.:m.}
\end{itemize}
Utensílio, em que se leva pólvora para a caça.
\section{Polvorista}
\begin{itemize}
\item {Grp. gram.:m.  e  f.}
\end{itemize}
Pessôa, que trabalha no fabríco da pólvora.
\section{Polvorós}
\begin{itemize}
\item {Grp. gram.:m.}
\end{itemize}
\begin{itemize}
\item {Utilização:Prov.}
\end{itemize}
\begin{itemize}
\item {Utilização:beir.}
\end{itemize}
O mesmo que \textunderscore polvorosa\textunderscore .
\section{Polvorosa}
\begin{itemize}
\item {Grp. gram.:f.}
\end{itemize}
\begin{itemize}
\item {Utilização:Pop.}
\end{itemize}
\begin{itemize}
\item {Utilização:Des.}
\end{itemize}
Grande actividade; azáfama.
Agitação.
Ruína, dissipação.
(Fem. de \textunderscore polvoroso\textunderscore )
\section{Polvoroso}
\begin{itemize}
\item {Grp. gram.:adj.}
\end{itemize}
O mesmo que \textunderscore pulveroso\textunderscore .
\section{Poly...}
\begin{itemize}
\item {Grp. gram.:pref.}
\end{itemize}
\begin{itemize}
\item {Proveniência:(Do gr. \textunderscore polus\textunderscore )}
\end{itemize}
(designativo de número indefinido e elevado)
\section{Polyacantho}
\begin{itemize}
\item {Grp. gram.:adj.}
\end{itemize}
\begin{itemize}
\item {Proveniência:(Do gr. \textunderscore polus\textunderscore  + \textunderscore akantha\textunderscore )}
\end{itemize}
Que tem muitos espinhos.
\section{Polyácido}
\begin{itemize}
\item {Grp. gram.:adj.}
\end{itemize}
\begin{itemize}
\item {Utilização:Chím.}
\end{itemize}
\begin{itemize}
\item {Proveniência:(De \textunderscore polus\textunderscore  gr. + \textunderscore ácido\textunderscore )}
\end{itemize}
Diz-se das bases, em que uma molécula satura muitas moléculas de um ácido.
\section{Polyadelphia}
\begin{itemize}
\item {Grp. gram.:f.}
\end{itemize}
\begin{itemize}
\item {Utilização:Bot.}
\end{itemize}
\begin{itemize}
\item {Proveniência:(Do gr. \textunderscore polus\textunderscore  + \textunderscore adelphos\textunderscore )}
\end{itemize}
União dos estames da flôr, formando muitos feixes.
\section{Polyadelphito}
\begin{itemize}
\item {Grp. gram.:m.}
\end{itemize}
\begin{itemize}
\item {Utilização:Miner.}
\end{itemize}
Silicato múltiplo, que forma massas compostas de grãos arredondados e diversamente coloridos.
(Cp. \textunderscore polyadelphia\textunderscore )
\section{Polyadelpho}
\begin{itemize}
\item {Grp. gram.:adj.}
\end{itemize}
\begin{itemize}
\item {Utilização:Bot.}
\end{itemize}
\begin{itemize}
\item {Proveniência:(Do gr. \textunderscore polus\textunderscore  + \textunderscore adelphos\textunderscore )}
\end{itemize}
Diz-se dos estames, quando soldados pelos seus filetes em mais de dois fascículos.
\section{Polyadênia}
\begin{itemize}
\item {Grp. gram.:f.}
\end{itemize}
\begin{itemize}
\item {Proveniência:(Do gr. \textunderscore polus\textunderscore  + \textunderscore aden\textunderscore )}
\end{itemize}
Género de árvores lauráceas da Índia.
\section{Polyaminas}
\begin{itemize}
\item {Grp. gram.:f. pl.}
\end{itemize}
\begin{itemize}
\item {Utilização:Chím.}
\end{itemize}
\begin{itemize}
\item {Proveniência:(De \textunderscore poly...\textunderscore  + \textunderscore amínas\textunderscore )}
\end{itemize}
Aminas, derivadas de muitas moléculas de ammoníaco, condensadas.
\section{Polyandra}
\begin{itemize}
\item {Grp. gram.:adj.}
\end{itemize}
\begin{itemize}
\item {Proveniência:(Do gr. \textunderscore polus\textunderscore  + \textunderscore aner\textunderscore , \textunderscore andros\textunderscore )}
\end{itemize}
Diz-se da mulhér, que tem mais de um marido ao mesmo tempo.
\section{Polyândria}
\begin{itemize}
\item {Grp. gram.:f.}
\end{itemize}
Classe botânica de Linneu, que comprehende os vegetaes provídos de 20 ou mais estames, inseridos sob um pistillo simples ou múltiplo.
\section{Polyandría}
\begin{itemize}
\item {Grp. gram.:f.}
\end{itemize}
Estado ou qualidade de polyandro.
\section{Polyândrico}
\begin{itemize}
\item {Grp. gram.:adj.}
\end{itemize}
Relativo á polyândria.
\section{Polyândrio}
\begin{itemize}
\item {Grp. gram.:m.}
\end{itemize}
\begin{itemize}
\item {Proveniência:(Gr. \textunderscore poluandrion\textunderscore )}
\end{itemize}
Túmulo commum de muitos guerreiros, entre os Gregos.
\section{Polyandro}
\begin{itemize}
\item {Grp. gram.:adj.}
\end{itemize}
\begin{itemize}
\item {Utilização:Bot.}
\end{itemize}
\begin{itemize}
\item {Proveniência:(Do gr. \textunderscore polus\textunderscore  + \textunderscore aner\textunderscore , \textunderscore andros\textunderscore )}
\end{itemize}
Que tem mais de doze estames, ou muitos, todos livres entre si.
\section{Polyanita}
\begin{itemize}
\item {Grp. gram.:f.}
\end{itemize}
\begin{itemize}
\item {Utilização:Miner.}
\end{itemize}
Bióxydo de manganés natural.
\section{Polyanthacarpia}
\begin{itemize}
\item {Grp. gram.:f.}
\end{itemize}
O mesmo que \textunderscore polycladia\textunderscore .
\section{Polyantho}
\begin{itemize}
\item {Grp. gram.:adj.}
\end{itemize}
\begin{itemize}
\item {Proveniência:(Do gr. \textunderscore polus\textunderscore  + \textunderscore anthos\textunderscore )}
\end{itemize}
Que tem muitas flôres.
\section{Polyarchia}
\begin{itemize}
\item {fónica:qui}
\end{itemize}
\begin{itemize}
\item {Grp. gram.:f.}
\end{itemize}
\begin{itemize}
\item {Proveniência:(Do gr. \textunderscore polus\textunderscore  + \textunderscore arkhe\textunderscore )}
\end{itemize}
Govêrno exercido por muitos.
\section{Polyarino}
\begin{itemize}
\item {Grp. gram.:adj.}
\end{itemize}
(V.polyândrico)
\section{Polyarticular}
\begin{itemize}
\item {Grp. gram.:adj.}
\end{itemize}
\begin{itemize}
\item {Proveniência:(De \textunderscore poly...\textunderscore  + \textunderscore articular\textunderscore )}
\end{itemize}
Diz-se do rheumatismo, que ataca muitas articulações.
\section{Polyatómico}
\begin{itemize}
\item {Grp. gram.:adj.}
\end{itemize}
\begin{itemize}
\item {Utilização:Chím.}
\end{itemize}
\begin{itemize}
\item {Proveniência:(De \textunderscore poly...\textunderscore  + \textunderscore atómico\textunderscore )}
\end{itemize}
Cujas combinações se realizam, na relação de muitos equivalentes dos corpos que se unem.
\section{Polybásico}
\begin{itemize}
\item {Grp. gram.:adj.}
\end{itemize}
\begin{itemize}
\item {Proveniência:(De \textunderscore poly...\textunderscore  + \textunderscore básico\textunderscore )}
\end{itemize}
Diz-se dos ácidos, que contêm muitas moléculas de água básica.
\section{Polybasito}
\begin{itemize}
\item {Grp. gram.:m.}
\end{itemize}
\begin{itemize}
\item {Utilização:Miner.}
\end{itemize}
Prata sulfurada, que contém certa porção de arsênico, antimónio, cobre, etc., e procedente do México.
\section{Polyblennia}
\begin{itemize}
\item {Grp. gram.:f.}
\end{itemize}
\begin{itemize}
\item {Utilização:Med.}
\end{itemize}
\begin{itemize}
\item {Proveniência:(Do gr. \textunderscore polus\textunderscore  + \textunderscore blenna\textunderscore )}
\end{itemize}
Excessivo corrimento de mucosidades.
\section{Polycárpico}
\begin{itemize}
\item {Grp. gram.:adj.}
\end{itemize}
\begin{itemize}
\item {Utilização:Bot.}
\end{itemize}
Que dá flôres e frutos por mais de uma vez.
(Cp. \textunderscore polycarpo\textunderscore )
\section{Polycarpo}
\begin{itemize}
\item {Grp. gram.:adj.}
\end{itemize}
\begin{itemize}
\item {Proveniência:(Do gr. \textunderscore polus\textunderscore  + \textunderscore karpos\textunderscore )}
\end{itemize}
Que tem ou produz muitos frutos.
\section{Polycellular}
\begin{itemize}
\item {Grp. gram.:adj.}
\end{itemize}
\begin{itemize}
\item {Proveniência:(De \textunderscore poly...\textunderscore  + \textunderscore cellular\textunderscore )}
\end{itemize}
Composto de muitas céllulas.
\section{Polycêntrico}
\begin{itemize}
\item {Grp. gram.:adj.}
\end{itemize}
\begin{itemize}
\item {Utilização:Constr.}
\end{itemize}
\begin{itemize}
\item {Proveniência:(De \textunderscore poly...\textunderscore  + \textunderscore centro\textunderscore )}
\end{itemize}
Diz-se da espiral com muitos centros.
Diz-se do arco, que tem vários centros.
\section{Polycholia}
\begin{itemize}
\item {fónica:co}
\end{itemize}
\begin{itemize}
\item {Grp. gram.:f.}
\end{itemize}
\begin{itemize}
\item {Proveniência:(Do gr. \textunderscore polus\textunderscore  + \textunderscore khole\textunderscore )}
\end{itemize}
Secreção biliosa, exaggerada.
\section{Polychótomo}
\begin{itemize}
\item {fónica:có}
\end{itemize}
\begin{itemize}
\item {Grp. gram.:adj.}
\end{itemize}
\begin{itemize}
\item {Proveniência:(Do gr. \textunderscore polukhos\textunderscore  + \textunderscore tome\textunderscore )}
\end{itemize}
Diz-se de um corpo, dividido em muitas articulações.
\section{Polychresto}
\begin{itemize}
\item {Grp. gram.:adj.}
\end{itemize}
\begin{itemize}
\item {Proveniência:(Gr. \textunderscore polukhrestos\textunderscore )}
\end{itemize}
Que serve para muitos usos.
\section{Polychromático}
\begin{itemize}
\item {Grp. gram.:adj.}
\end{itemize}
O mesmo que \textunderscore polychromo\textunderscore .
\section{Polychromia}
\begin{itemize}
\item {Grp. gram.:f.}
\end{itemize}
\begin{itemize}
\item {Proveniência:(De \textunderscore polychromo\textunderscore )}
\end{itemize}
Estado de um corpo, em que há differentes côres.
Conjunto de várias côres.
\section{Polychrómico}
\begin{itemize}
\item {Grp. gram.:adj.}
\end{itemize}
\begin{itemize}
\item {Proveniência:(Do gr. \textunderscore polus\textunderscore  + \textunderscore khroma\textunderscore )}
\end{itemize}
Que tem muitas côres.
Em que se empregam muitas côres.
\section{Polychromo}
\begin{itemize}
\item {Grp. gram.:adj.}
\end{itemize}
\begin{itemize}
\item {Proveniência:(Do gr. \textunderscore polus\textunderscore  + \textunderscore khroma\textunderscore )}
\end{itemize}
Que tem muitas côres.
Em que se empregam muitas côres.
\section{Polýchrono}
\begin{itemize}
\item {Grp. gram.:adj.}
\end{itemize}
\begin{itemize}
\item {Utilização:Neol.}
\end{itemize}
\begin{itemize}
\item {Proveniência:(Do gr. \textunderscore polus\textunderscore  + \textunderscore khronos\textunderscore )}
\end{itemize}
O mesmo que \textunderscore duradoiro\textunderscore .
\section{Polycladia}
\begin{itemize}
\item {Grp. gram.:f.}
\end{itemize}
\begin{itemize}
\item {Proveniência:(Do gr. \textunderscore polus\textunderscore  + \textunderscore klados\textunderscore )}
\end{itemize}
Moléstia vegetal, em que a seiva, abandonando órgãos da fructificação, se dirige para os ramos, dando-lhes prodigioso desenvolvimento.
\section{Polyclado}
\begin{itemize}
\item {Grp. gram.:adj.}
\end{itemize}
Diz-se do vegetal, que dá muitos ramos.
(Cp. \textunderscore polycladia\textunderscore )
\section{Polycladodia}
\begin{itemize}
\item {Grp. gram.:f.}
\end{itemize}
(V.polycladia)
\section{Polyclínica}
\begin{itemize}
\item {Grp. gram.:f.}
\end{itemize}
\begin{itemize}
\item {Proveniência:(De \textunderscore poly...\textunderscore  + \textunderscore clínica\textunderscore )}
\end{itemize}
Prática da Medicina, applicada á generalidade das doenças.
\section{Polyclínico}
\begin{itemize}
\item {Grp. gram.:m.}
\end{itemize}
Clínico, que trata das doenças em geral, ou que se não dedica especialmente a uma.
(Cp. \textunderscore polyclínica\textunderscore )
\section{Polýcomo}
\begin{itemize}
\item {Grp. gram.:adj.}
\end{itemize}
\begin{itemize}
\item {Proveniência:(Do gr. \textunderscore polus\textunderscore  + \textunderscore kome\textunderscore )}
\end{itemize}
Que tem muitos cabellos.
\section{Polycónico}
\begin{itemize}
\item {Grp. gram.:adj.}
\end{itemize}
\begin{itemize}
\item {Proveniência:(De \textunderscore poly...\textunderscore  + \textunderscore cónico\textunderscore )}
\end{itemize}
Que tem muitos cónes.
\section{Polycórdio}
\begin{itemize}
\item {Grp. gram.:m.}
\end{itemize}
\begin{itemize}
\item {Proveniência:(De \textunderscore poly...\textunderscore  + \textunderscore corda\textunderscore )}
\end{itemize}
Antigo instrumento, que se tocava com arco.
\section{Polycoria}
\begin{itemize}
\item {Grp. gram.:f.}
\end{itemize}
\begin{itemize}
\item {Utilização:Med.}
\end{itemize}
\begin{itemize}
\item {Proveniência:(Do gr. \textunderscore polus\textunderscore  + \textunderscore kore\textunderscore )}
\end{itemize}
Pupilla múltipla.
\section{Polycotyledónio}
\begin{itemize}
\item {Grp. gram.:adj.}
\end{itemize}
\begin{itemize}
\item {Proveniência:(De \textunderscore póly...\textunderscore  + \textunderscore cotyledónio\textunderscore )}
\end{itemize}
Que tem mais de dois cotylédones.
\section{Polycultura}
\begin{itemize}
\item {Grp. gram.:f.}
\end{itemize}
\begin{itemize}
\item {Utilização:Agr.}
\end{itemize}
\begin{itemize}
\item {Proveniência:(De \textunderscore poly...\textunderscore  + \textunderscore cultura\textunderscore )}
\end{itemize}
Cultura variada.
\section{Polycythemia}
\begin{itemize}
\item {Grp. gram.:f.}
\end{itemize}
\begin{itemize}
\item {Utilização:Med.}
\end{itemize}
\begin{itemize}
\item {Proveniência:(Do gr. \textunderscore polus\textunderscore  + \textunderscore kutos\textunderscore  + \textunderscore haima\textunderscore )}
\end{itemize}
Aumento do número dos glóbulos sanguíneos.
\section{Polydáctylo}
\begin{itemize}
\item {Grp. gram.:adj.}
\end{itemize}
\begin{itemize}
\item {Proveniência:(Do gr. \textunderscore polus\textunderscore  + \textunderscore daktulos\textunderscore )}
\end{itemize}
Que tem muitos dedos.
\section{Polydipsia}
\begin{itemize}
\item {Grp. gram.:f.}
\end{itemize}
\begin{itemize}
\item {Utilização:Med.}
\end{itemize}
\begin{itemize}
\item {Proveniência:(Do gr. \textunderscore polus\textunderscore  + \textunderscore dipsa\textunderscore )}
\end{itemize}
Sêde excessiva.
\section{Polýdymo}
\begin{itemize}
\item {Grp. gram.:m.}
\end{itemize}
\begin{itemize}
\item {Utilização:Geol.}
\end{itemize}
\begin{itemize}
\item {Proveniência:(Gr. \textunderscore poludumos\textunderscore )}
\end{itemize}
Macla, composta de quatro ou mais indivíduos. Cf. Gonç. Guimarães, \textunderscore Geol.\textunderscore , 60.
\section{Polyédrico}
\begin{itemize}
\item {Grp. gram.:adj.}
\end{itemize}
Que tem a fórma de polyedro.
\section{Polyedro}
\begin{itemize}
\item {Grp. gram.:adj.}
\end{itemize}
\begin{itemize}
\item {Grp. gram.:M.}
\end{itemize}
\begin{itemize}
\item {Proveniência:(Do gr. \textunderscore polus\textunderscore  + \textunderscore edra\textunderscore )}
\end{itemize}
Que tem muitas faces planas.
Sólido polyedro.
\section{Polyemia}
\begin{itemize}
\item {Grp. gram.:f.}
\end{itemize}
\begin{itemize}
\item {Utilização:Med.}
\end{itemize}
\begin{itemize}
\item {Proveniência:(Do gr. \textunderscore polus\textunderscore  + \textunderscore haima\textunderscore )}
\end{itemize}
Plethora sanguínea; excesso de sangue, nos vasos do corpo. Cf. Macedo Pinto, \textunderscore Comp. de Veter.\textunderscore , I, 76.
\section{Polyesthesia}
\begin{itemize}
\item {Grp. gram.:f.}
\end{itemize}
\begin{itemize}
\item {Utilização:Med.}
\end{itemize}
\begin{itemize}
\item {Proveniência:(Do gr. \textunderscore polus\textunderscore  + \textunderscore esthesis\textunderscore )}
\end{itemize}
Perturbação da sensibilidade, durante a qual um excitante único produz uma sensação múltipla.
\section{Polyfólia}
\begin{itemize}
\item {Grp. gram.:adj. f.}
\end{itemize}
\begin{itemize}
\item {Proveniência:(Do gr. \textunderscore polus\textunderscore  + lat. \textunderscore folium\textunderscore )}
\end{itemize}
Diz-se da charrua de mais de três aivecas.
\section{Polýgala}
\begin{itemize}
\item {Grp. gram.:f.}
\end{itemize}
\begin{itemize}
\item {Proveniência:(Do gr. \textunderscore polus\textunderscore  + \textunderscore gala\textunderscore )}
\end{itemize}
Gênero de plantas, de suco leitoso, das quaes há duas espécies, empregadas em Medicina.
\section{Polygaláceas}
\begin{itemize}
\item {Grp. gram.:f. pl.}
\end{itemize}
O mesmo ou melhor que \textunderscore polygáleas\textunderscore .
\section{Polygalactia}
\begin{itemize}
\item {Grp. gram.:f.}
\end{itemize}
\begin{itemize}
\item {Utilização:Med.}
\end{itemize}
\begin{itemize}
\item {Proveniência:(Do gr. \textunderscore polus\textunderscore  + \textunderscore gala\textunderscore , \textunderscore galactos\textunderscore )}
\end{itemize}
Superabundância de leite.
\section{Polygáleas}
\begin{itemize}
\item {Grp. gram.:f. pl.}
\end{itemize}
Família de plantas, que tem por typo a polýgala.
(Fem. pl. de \textunderscore polygáleo\textunderscore )
\section{Polygáleo}
\begin{itemize}
\item {Grp. gram.:adj.}
\end{itemize}
Relativo ou semelhante á polýgala.
\section{Polygálico}
\begin{itemize}
\item {Grp. gram.:adj.}
\end{itemize}
Diz-se de um ácido, achado numa especie de polýgala, e em outras plantas.
\section{Polygalíneas}
\begin{itemize}
\item {Grp. gram.:f. pl.}
\end{itemize}
Ordem de plantas, que abrange as polygáleas e outras.
(Cp. \textunderscore polygáleas\textunderscore )
\section{Polygamia}
\begin{itemize}
\item {Grp. gram.:f.}
\end{itemize}
Estado de quem é polýgamo.
\section{Polygâmico}
\begin{itemize}
\item {Grp. gram.:adj.}
\end{itemize}
Relativo á polygamia.
\section{Polýgamo}
\begin{itemize}
\item {Grp. gram.:m.  e  adj.}
\end{itemize}
\begin{itemize}
\item {Proveniência:(Gr. \textunderscore polugamos\textunderscore )}
\end{itemize}
Indivíduo, que tem mais de um cônjuge, ao mesmo tempo.
Diz-se também de certos animaes, em que um macho tem muitas fêmeas.
E diz-se das plantas, que têm ao mesmo tempo flôres hermaphroditas e unisexuaes.
\section{Polygastricidade}
\begin{itemize}
\item {Grp. gram.:f.}
\end{itemize}
\begin{itemize}
\item {Proveniência:(De \textunderscore polygástrico\textunderscore )}
\end{itemize}
Existencia de muitos estômagos, que se julgou terem-se descoberto nos infusórios.
\section{Polygástrico}
\begin{itemize}
\item {Grp. gram.:adj.}
\end{itemize}
\begin{itemize}
\item {Proveniência:(Do gr. \textunderscore polus\textunderscore  + \textunderscore gaster\textunderscore )}
\end{itemize}
Que tem muitos estômagos.
\section{Polygastro}
\begin{itemize}
\item {Grp. gram.:adj.}
\end{itemize}
O mesmo ou melhor que \textunderscore polygástrico\textunderscore .
\section{Polygenismo}
\begin{itemize}
\item {Grp. gram.:m.}
\end{itemize}
\begin{itemize}
\item {Proveniência:(De \textunderscore polýgeno\textunderscore )}
\end{itemize}
Systema dos que attribuem as raças humanas a differentes troncos primitivos.
\section{Polýgeno}
\begin{itemize}
\item {Grp. gram.:adj.}
\end{itemize}
\begin{itemize}
\item {Proveniência:(Do gr. \textunderscore polus\textunderscore  + \textunderscore genos\textunderscore )}
\end{itemize}
Que produz muito.
\section{Polyglotta}
\begin{itemize}
\item {Grp. gram.:adj.}
\end{itemize}
\begin{itemize}
\item {Grp. gram.:M.  e  f.}
\end{itemize}
\begin{itemize}
\item {Proveniência:(Gr. \textunderscore poluglottos\textunderscore )}
\end{itemize}
Que está escrito em muitas línguas.
Pessôa que sabe ou fala muitas línguas.
\section{Polyglóttico}
\begin{itemize}
\item {Grp. gram.:adj.}
\end{itemize}
Escrito em muitas línguas.
Relativo a polyglotta.
\section{Polyglottismo}
\begin{itemize}
\item {Grp. gram.:m.}
\end{itemize}
Qualidade de quem é polyglotta.
Facilidade de falar muitas línguas.
\section{Polyglotto}
\begin{itemize}
\item {Grp. gram.:m.  e  adj.}
\end{itemize}
\begin{itemize}
\item {Proveniência:(Gr. \textunderscore polluglottos\textunderscore )}
\end{itemize}
O mesmo ou melhor que \textunderscore polliglota\textunderscore .
\section{Polygnathos}
\begin{itemize}
\item {Grp. gram.:m. pl.}
\end{itemize}
\begin{itemize}
\item {Proveniência:(Do gr. \textunderscore polus\textunderscore  + \textunderscore gnathos\textunderscore )}
\end{itemize}
Família de insectos, sem asas, e com muitos pares de queixos.
\section{Polygonáceas}
\begin{itemize}
\item {Grp. gram.:f. pl.}
\end{itemize}
O mesmo ou melhor que \textunderscore polygóneas\textunderscore .
\section{Polygonal}
\begin{itemize}
\item {Grp. gram.:adj.}
\end{itemize}
Relativo ao polýgono; que tem por base um polýgono; que tem muitos ângulos.
\section{Polygóneas}
\begin{itemize}
\item {Grp. gram.:f. pl.}
\end{itemize}
Família de plantas, que tem por typo o \textunderscore polýgono\textunderscore  ou \textunderscore sempre-noiva\textunderscore .
(Fem. pl. de \textunderscore polygóneo\textunderscore )
\section{Polygóneo}
\begin{itemize}
\item {Grp. gram.:adj.}
\end{itemize}
\begin{itemize}
\item {Utilização:Bot.}
\end{itemize}
Relativo ou semelhante ao polýgono, planta.
\section{Polýgono}
\begin{itemize}
\item {Grp. gram.:m.}
\end{itemize}
\begin{itemize}
\item {Utilização:Bot.}
\end{itemize}
\begin{itemize}
\item {Proveniência:(Gr. \textunderscore polugonos\textunderscore )}
\end{itemize}
Figura geométrica, que tem muitos ângulos e lados.
Figura, que determina a fórma geral de uma praça de guerra.
Lugar, onde os artilheiros fazem exercício com bocas de fogo.
Designação moderna da \textunderscore sempre-noiva\textunderscore , typo das polygóneas.
\section{Polygramma}
\begin{itemize}
\item {Grp. gram.:f.}
\end{itemize}
\begin{itemize}
\item {Utilização:Neol.}
\end{itemize}
\begin{itemize}
\item {Proveniência:(Do gr. \textunderscore polus\textunderscore  + \textunderscore grana\textunderscore )}
\end{itemize}
Representação vária do mesmo som.
O mesmo que \textunderscore homophonia\textunderscore .
\section{Polygraphar}
\begin{itemize}
\item {Grp. gram.:v. t.}
\end{itemize}
Escrever com o maquinismo chamado polýgrapho.
Escrever sôbre matérias diversas.
\section{Polygraphia}
\begin{itemize}
\item {Grp. gram.:f.}
\end{itemize}
\begin{itemize}
\item {Utilização:Ant.}
\end{itemize}
Collecção de obras diversas, litterárias ou scientificas.
Conjunto de conhecimentos vários.
Qualidade de quem é polýgrapho.
Cryptógraphia.
\section{Polygráphico}
\begin{itemize}
\item {Grp. gram.:adj.}
\end{itemize}
Relativo á polygraphia.
\section{Polýgrapho}
\begin{itemize}
\item {Grp. gram.:m.}
\end{itemize}
\begin{itemize}
\item {Proveniência:(Gr. \textunderscore polugraphos\textunderscore )}
\end{itemize}
Aquelle que escreve sôbre matérias diversas.
Maquinismo que, movendo muitas pennas, produz ao mesmo tempo muitas cópias do mesmo escrito.
\section{Polialito}
\begin{itemize}
\item {Grp. gram.:m.}
\end{itemize}
\begin{itemize}
\item {Proveniência:(Do gr. \textunderscore polus\textunderscore  + \textunderscore halos\textunderscore )}
\end{itemize}
Sulfato natural de cálcio, magnésio e potássio.
\section{Poliginecia}
\begin{itemize}
\item {Grp. gram.:f.}
\end{itemize}
\begin{itemize}
\item {Utilização:P. us.}
\end{itemize}
\begin{itemize}
\item {Proveniência:(Do gr. \textunderscore polus\textunderscore  + \textunderscore gune\textunderscore )}
\end{itemize}
O mesmo que \textunderscore poligamia\textunderscore . Cf. Ferrer, \textunderscore Dir. Nat.\textunderscore , 170.
\section{Poliginia}
\begin{itemize}
\item {Grp. gram.:f.}
\end{itemize}
Estado ou qualidade de polígino.
\section{Poligínio}
\begin{itemize}
\item {Grp. gram.:adj.}
\end{itemize}
\begin{itemize}
\item {Utilização:Bot.}
\end{itemize}
\begin{itemize}
\item {Proveniência:(Do gr. \textunderscore polus\textunderscore  + \textunderscore gune\textunderscore )}
\end{itemize}
Que tem muitos pistilos em cada flôr.
\section{Polígino}
\begin{itemize}
\item {Grp. gram.:adj.}
\end{itemize}
O mesmo ou melhor que \textunderscore poligínio\textunderscore .
\section{Polilépido}
\begin{itemize}
\item {Grp. gram.:adj.}
\end{itemize}
\begin{itemize}
\item {Utilização:Bot.}
\end{itemize}
\begin{itemize}
\item {Proveniência:(Do gr. \textunderscore polus\textunderscore  + \textunderscore lepis\textunderscore )}
\end{itemize}
Que tem muitas escamas.
\section{Polilóbio}
\begin{itemize}
\item {Grp. gram.:m.}
\end{itemize}
Gênero de plantas leguminosas.
\section{Polilóbulo}
\begin{itemize}
\item {Grp. gram.:m.}
\end{itemize}
\begin{itemize}
\item {Proveniência:(De \textunderscore poli...\textunderscore  + \textunderscore lóbulo\textunderscore )}
\end{itemize}
Ornato arquitectónico, com muitos lóbos.
\section{Polimastia}
\begin{itemize}
\item {Grp. gram.:f.}
\end{itemize}
\begin{itemize}
\item {Utilização:Terat.}
\end{itemize}
\begin{itemize}
\item {Proveniência:(Do gr. \textunderscore polus\textunderscore  + \textunderscore mastos\textunderscore )}
\end{itemize}
Anomalia do indivíduo, que tem muitas mamas.
\section{Polímata}
\begin{itemize}
\item {Grp. gram.:m.}
\end{itemize}
O mesmo que \textunderscore polímato\textunderscore .
\section{Polimatia}
\begin{itemize}
\item {Grp. gram.:f.}
\end{itemize}
\begin{itemize}
\item {Proveniência:(De \textunderscore polímato\textunderscore )}
\end{itemize}
Instrução extensa e variada.
\section{Polimático}
\begin{itemize}
\item {Grp. gram.:adj.}
\end{itemize}
Relativo á polimatia.
\section{Polímato}
\begin{itemize}
\item {Grp. gram.:m.  e  adj.}
\end{itemize}
\begin{itemize}
\item {Proveniência:(Gr. \textunderscore polumathos\textunderscore )}
\end{itemize}
O que estudou ou sabe muitas ciências; polígrafo.
\section{Polimelia}
\begin{itemize}
\item {Grp. gram.:f.}
\end{itemize}
Qualidade de polímelo.
\section{Polimeria}
\begin{itemize}
\item {Grp. gram.:f.}
\end{itemize}
Estado do que é polímero.
\section{Polimerismo}
\begin{itemize}
\item {Grp. gram.:m.}
\end{itemize}
Estado do que é polímero.
\section{Polímero}
\begin{itemize}
\item {Grp. gram.:adj.}
\end{itemize}
\begin{itemize}
\item {Utilização:Chím.}
\end{itemize}
\begin{itemize}
\item {Proveniência:(Do gr. \textunderscore polus\textunderscore  + \textunderscore meros\textunderscore )}
\end{itemize}
Que contém os mesmos elementos na mesma quantidade relativa, mas não na mesma quantidade absoluta.
\section{Polímnia}
\begin{itemize}
\item {Grp. gram.:f.}
\end{itemize}
\begin{itemize}
\item {Proveniência:(De \textunderscore Polímnia\textunderscore , n. p.)}
\end{itemize}
Gênero de plantas, da fam. das compostas.
\section{Polímnico}
\begin{itemize}
\item {Grp. gram.:adj.}
\end{itemize}
\begin{itemize}
\item {Proveniência:(De \textunderscore Polímnia\textunderscore , n. p.)}
\end{itemize}
Relativo á Retórica.
\section{Polimorfia}
\begin{itemize}
\item {Grp. gram.:f.}
\end{itemize}
Propriedade do que é polimorfo.
\section{Polimorfismo}
\begin{itemize}
\item {Grp. gram.:m.}
\end{itemize}
Propriedade do que é polimorfo.
\section{Polimorfo}
\begin{itemize}
\item {Grp. gram.:adj.}
\end{itemize}
\begin{itemize}
\item {Proveniência:(Do gr. \textunderscore polus\textunderscore  + \textunderscore morphe\textunderscore )}
\end{itemize}
Que se apresenta sob diversas fórmas; que é sujeito a variar de fórma.
\section{Polinervado}
\begin{itemize}
\item {Grp. gram.:adj.}
\end{itemize}
\begin{itemize}
\item {Utilização:Bot.}
\end{itemize}
\begin{itemize}
\item {Proveniência:(De \textunderscore poli...\textunderscore  + \textunderscore nervo\textunderscore )}
\end{itemize}
Que tem muitas nervuras.
\section{Polinésios}
\begin{itemize}
\item {Grp. gram.:m. pl.}
\end{itemize}
Selvagens dos mares do Sul da Oceânia.
\section{Polineurite}
\begin{itemize}
\item {Grp. gram.:f.}
\end{itemize}
\begin{itemize}
\item {Proveniência:(De \textunderscore poli...\textunderscore  + \textunderscore neurite\textunderscore )}
\end{itemize}
Neurite periférica.
\section{Polineurítico}
\begin{itemize}
\item {Grp. gram.:adj.}
\end{itemize}
\begin{itemize}
\item {Proveniência:(De \textunderscore poli...\textunderscore  + \textunderscore neurítico\textunderscore )}
\end{itemize}
Relativo á polineurite.
\section{Polinómio}
\begin{itemize}
\item {Grp. gram.:m.}
\end{itemize}
\begin{itemize}
\item {Proveniência:(Do gr. \textunderscore polus\textunderscore  + \textunderscore nomos\textunderscore )}
\end{itemize}
Qualquer quantidade algébrica, composta de muitos termos, separados pelo sinal + ou -.
\section{Polinuclear}
\begin{itemize}
\item {Grp. gram.:adj.}
\end{itemize}
\begin{itemize}
\item {Proveniência:(De \textunderscore poli...\textunderscore  + \textunderscore núcleo\textunderscore )}
\end{itemize}
Que tem muitos núcleos.
\section{Poliónimo}
\begin{itemize}
\item {Grp. gram.:adj.}
\end{itemize}
\begin{itemize}
\item {Proveniência:(Do gr. \textunderscore polus\textunderscore  + \textunderscore onuma\textunderscore )}
\end{itemize}
Que tem muitos nomes; que póde nomear-se de várias fórmas.
\section{Poliope}
\begin{itemize}
\item {Grp. gram.:m.}
\end{itemize}
Aquele que sofre poliopia.
\section{Poliopia}
\begin{itemize}
\item {Grp. gram.:f.}
\end{itemize}
\begin{itemize}
\item {Proveniência:(Do gr. \textunderscore polus\textunderscore  + \textunderscore ops\textunderscore )}
\end{itemize}
Estado mórbido dos que vêem os objectos multiplicados.
\section{Polifagia}
\begin{itemize}
\item {Grp. gram.:f.}
\end{itemize}
Qualidade de polífago.
\section{Polífago}
\begin{itemize}
\item {Grp. gram.:adj.}
\end{itemize}
\begin{itemize}
\item {Proveniência:(Do gr. \textunderscore polus\textunderscore  + \textunderscore phagein\textunderscore )}
\end{itemize}
Que come muito; que tem fome canina.
\section{Polifásico}
\begin{itemize}
\item {Grp. gram.:adj.}
\end{itemize}
\begin{itemize}
\item {Utilização:Phýs.}
\end{itemize}
Diz-se das correntes, que espalham e distribuem a todas as distâncias as fôrças motrizes naturaes.
\section{Polifemo}
\begin{itemize}
\item {Grp. gram.:m.}
\end{itemize}
\begin{itemize}
\item {Proveniência:(De \textunderscore Polifemo\textunderscore , n. p.)}
\end{itemize}
Espécie de escaravelho.
Nome de um crustáceo.
\section{Polifilo}
\begin{itemize}
\item {Grp. gram.:adj.}
\end{itemize}
\begin{itemize}
\item {Utilização:Bot.}
\end{itemize}
\begin{itemize}
\item {Proveniência:(Do gr. \textunderscore polus\textunderscore  + \textunderscore phullon\textunderscore )}
\end{itemize}
Formado de muitos folíolos.
\section{Polifiodontes}
\begin{itemize}
\item {Grp. gram.:m. pl.}
\end{itemize}
\begin{itemize}
\item {Proveniência:(Do gr. \textunderscore polus\textunderscore  + \textunderscore phio\textunderscore  + \textunderscore odous\textunderscore , \textunderscore odontos\textunderscore )}
\end{itemize}
Animaes que têm mais do que uma dentição.
\section{Polifisia}
\begin{itemize}
\item {Grp. gram.:f.}
\end{itemize}
\begin{itemize}
\item {Utilização:Med.}
\end{itemize}
\begin{itemize}
\item {Proveniência:(Do gr. \textunderscore polus\textunderscore  + \textunderscore phusa\textunderscore )}
\end{itemize}
Abundância de gases ou flatuosidades.
\section{Polífito}
\begin{itemize}
\item {Grp. gram.:adj.}
\end{itemize}
\begin{itemize}
\item {Utilização:Bot.}
\end{itemize}
\begin{itemize}
\item {Proveniência:(Do gr. \textunderscore polus\textunderscore  + \textunderscore phuton\textunderscore )}
\end{itemize}
Relativo a muitas plantas.
Diz-se dos gêneros, que compreendem muitas plantas.
\section{Polifonía}
\begin{itemize}
\item {Grp. gram.:f.}
\end{itemize}
Pluralidade de sons e de articulações, relativa a um sinal vocal, na escritura dos Assírios.
(Cp. \textunderscore polifono\textunderscore )
\section{Polifónico}
\begin{itemize}
\item {Grp. gram.:adj.}
\end{itemize}
Relativo á polifonia.
\section{Polifono}
\begin{itemize}
\item {Grp. gram.:adj.}
\end{itemize}
\begin{itemize}
\item {Proveniência:(Do gr. \textunderscore polus\textunderscore  + \textunderscore phone\textunderscore )}
\end{itemize}
Que repete os sons muitas vezes.
\section{Poliorama}
\begin{itemize}
\item {Grp. gram.:m.}
\end{itemize}
\begin{itemize}
\item {Proveniência:(Do gr. \textunderscore polus\textunderscore  + \textunderscore orama\textunderscore )}
\end{itemize}
Espécie de panorama, em que os quadros móveis, penetrando-se reciprocamente, mudam de contornos e se transfiguram, aos olhos do observador.
\section{Poliorquidia}
\begin{itemize}
\item {Grp. gram.:f.}
\end{itemize}
\begin{itemize}
\item {Utilização:Anat.}
\end{itemize}
\begin{itemize}
\item {Proveniência:(Do gr. \textunderscore polus\textunderscore  + \textunderscore orkhis\textunderscore )}
\end{itemize}
Existência de mais de dois testículos num homem.
\section{Polípago}
\begin{itemize}
\item {Grp. gram.:m.}
\end{itemize}
\begin{itemize}
\item {Utilização:Terat.}
\end{itemize}
\begin{itemize}
\item {Proveniência:(Do gr. \textunderscore polus\textunderscore  + \textunderscore phageis\textunderscore )}
\end{itemize}
Monstro monocéfalo, formado de dois corpos, de eixos paralelos.
\section{Polipedia}
\begin{itemize}
\item {Grp. gram.:f.}
\end{itemize}
\begin{itemize}
\item {Utilização:Med.}
\end{itemize}
\begin{itemize}
\item {Proveniência:(Do gr. \textunderscore polus\textunderscore  + \textunderscore pais\textunderscore , \textunderscore paidos\textunderscore )}
\end{itemize}
Presença de muitos fetos, na mesma gestação.
\section{Polipeiro}
\begin{itemize}
\item {Grp. gram.:m.}
\end{itemize}
Habitação de pólipos, que vivem agrupados.
Grupo de pólipos.
\section{Polipétalo}
\begin{itemize}
\item {Grp. gram.:adj.}
\end{itemize}
\begin{itemize}
\item {Proveniência:(De \textunderscore poli...\textunderscore  + \textunderscore pétala\textunderscore )}
\end{itemize}
Que tem muitas pétalas.
\section{Polipiforme}
\begin{itemize}
\item {Grp. gram.:adj.}
\end{itemize}
\begin{itemize}
\item {Proveniência:(De \textunderscore polipo\textunderscore  + \textunderscore fórma\textunderscore )}
\end{itemize}
Que tem fórma de pólipo.
\section{Polipiose}
\begin{itemize}
\item {Grp. gram.:f.}
\end{itemize}
\begin{itemize}
\item {Utilização:Med.}
\end{itemize}
\begin{itemize}
\item {Proveniência:(Do gr. \textunderscore polus\textunderscore  + \textunderscore pion\textunderscore )}
\end{itemize}
O mesmo que \textunderscore obesidade\textunderscore .
\section{Poliplectro}
\begin{itemize}
\item {Grp. gram.:m.}
\end{itemize}
\begin{itemize}
\item {Proveniência:(Do gr. \textunderscore polus\textunderscore  + \textunderscore plektron\textunderscore )}
\end{itemize}
Instrumento, destinado á prolongação de sons num teclado, imitando instrumentos de arco.
\section{Polipneia}
\begin{itemize}
\item {Grp. gram.:f.}
\end{itemize}
\begin{itemize}
\item {Utilização:Med.}
\end{itemize}
\begin{itemize}
\item {Proveniência:(Do gr. \textunderscore polus\textunderscore  + \textunderscore pneín\textunderscore )}
\end{itemize}
Respiração rápida e superficial.
\section{Pólipo}
\begin{itemize}
\item {Grp. gram.:m.}
\end{itemize}
\begin{itemize}
\item {Grp. gram.:Pl.}
\end{itemize}
\begin{itemize}
\item {Proveniência:(Gr. \textunderscore polupous\textunderscore )}
\end{itemize}
Excrescência carnosa, fibrosa, etc., que se póde desenvolver em qualquer membrana mucosa.
Concreção sanguínea, que se fórma no coração ou nos grandes vasos.
Animaes de corpo mole e contráctil, com a cabeça rodeada de tentáculos radiados.
\section{Polipodiáceas}
\begin{itemize}
\item {Grp. gram.:f. pl.}
\end{itemize}
Família de plantas, que têm por tipo o polipódio.
\section{Polipódio}
\begin{itemize}
\item {Grp. gram.:adj.}
\end{itemize}
\begin{itemize}
\item {Grp. gram.:M.}
\end{itemize}
\begin{itemize}
\item {Proveniência:(Gr. \textunderscore polupodion\textunderscore )}
\end{itemize}
Que tem muitos pés.
Gênero de plantas parasitas da fam. dos fêtos.
\section{Políporo}
\begin{itemize}
\item {Grp. gram.:m.}
\end{itemize}
\begin{itemize}
\item {Proveniência:(Do gr. \textunderscore polus\textunderscore  + \textunderscore poros\textunderscore )}
\end{itemize}
Gênero de cogumelos.
\section{Poliposia}
\begin{itemize}
\item {Grp. gram.:f.}
\end{itemize}
\begin{itemize}
\item {Utilização:Med.}
\end{itemize}
O mesmo que \textunderscore polidipsia\textunderscore .
\section{Poliposo}
\begin{itemize}
\item {Grp. gram.:adj.}
\end{itemize}
Que tem a natureza do pólipo.
\section{Políptero}
\begin{itemize}
\item {Grp. gram.:m.}
\end{itemize}
\begin{itemize}
\item {Proveniência:(Do gr. \textunderscore polus\textunderscore  + \textunderscore pteron\textunderscore )}
\end{itemize}
Animal de cabeça alongada, espécie de lúcio, descoberto no Nilo.
\section{Poliptoto}
\begin{itemize}
\item {Grp. gram.:m.}
\end{itemize}
\begin{itemize}
\item {Utilização:Gram.}
\end{itemize}
\begin{itemize}
\item {Proveniência:(Gr. \textunderscore poluptoton\textunderscore )}
\end{itemize}
Acto de empregar, num período, uma palavra sob diversas fórmas gramaticaes.
\section{Polirrítmico}
\begin{itemize}
\item {Grp. gram.:adj.}
\end{itemize}
\begin{itemize}
\item {Utilização:Mús.}
\end{itemize}
\begin{itemize}
\item {Proveniência:(De \textunderscore poli...\textunderscore  + \textunderscore rítmico\textunderscore )}
\end{itemize}
Que se compõe de vários ritmos.
Que tem o ritmo muito variado.
\section{Polirrizo}
\begin{itemize}
\item {Grp. gram.:adj.}
\end{itemize}
\begin{itemize}
\item {Utilização:Bot.}
\end{itemize}
\begin{itemize}
\item {Proveniência:(Do gr. \textunderscore polus\textunderscore  + \textunderscore rhiza\textunderscore )}
\end{itemize}
Que tem muitas raízes.
\section{Poliscópio}
\begin{itemize}
\item {Grp. gram.:m.}
\end{itemize}
\begin{itemize}
\item {Proveniência:(Do gr. \textunderscore polus\textunderscore  + \textunderscore skopein\textunderscore )}
\end{itemize}
Óculo ou lente, que apresenta um objecto multiplicado.
\section{Polispermo}
\begin{itemize}
\item {Grp. gram.:adj.}
\end{itemize}
\begin{itemize}
\item {Proveniência:(Do gr. \textunderscore polus\textunderscore  + \textunderscore sperma\textunderscore )}
\end{itemize}
Que tem muitos grãos, (falando-se de frutos).
\section{Polísporo}
\begin{itemize}
\item {Grp. gram.:adj.}
\end{itemize}
\begin{itemize}
\item {Utilização:Bot.}
\end{itemize}
\begin{itemize}
\item {Proveniência:(Do gr. \textunderscore polus\textunderscore  + \textunderscore spora\textunderscore )}
\end{itemize}
Que contém muitos esporos.
\section{Polissarcia}
\begin{itemize}
\item {Grp. gram.:f.}
\end{itemize}
\begin{itemize}
\item {Utilização:Med.}
\end{itemize}
\begin{itemize}
\item {Proveniência:(Do gr. \textunderscore polus\textunderscore  + \textunderscore sarx\textunderscore , \textunderscore sarkhos\textunderscore )}
\end{itemize}
Aumento anormal dos músculos ou do tecido adiposo.
\section{Polissialia}
\begin{itemize}
\item {Grp. gram.:f.}
\end{itemize}
\begin{itemize}
\item {Utilização:Med.}
\end{itemize}
\begin{itemize}
\item {Proveniência:(Do gr. \textunderscore polus\textunderscore  + \textunderscore sialon\textunderscore )}
\end{itemize}
Secreção abundante de saliva.
\section{Polissilábico}
\begin{itemize}
\item {Grp. gram.:adj.}
\end{itemize}
Relativo ao polisílabo.
Que tem mais de uma sílaba.
\section{Polissílabo}
\begin{itemize}
\item {Grp. gram.:adj.}
\end{itemize}
\begin{itemize}
\item {Grp. gram.:M.}
\end{itemize}
\begin{itemize}
\item {Proveniência:(Gr. \textunderscore polusullabos\textunderscore )}
\end{itemize}
O mesmo que \textunderscore polissilábico\textunderscore .
Palavra que tem mais de uma sílaba.
\section{Polissilogístico}
\begin{itemize}
\item {Grp. gram.:adj.}
\end{itemize}
\begin{itemize}
\item {Proveniência:(De \textunderscore poli...\textunderscore  + \textunderscore sulogismo\textunderscore )}
\end{itemize}
Diz-se do raciocínio, composto de um encadeamento de silogismos.
\section{Polissíndeton}
\begin{itemize}
\item {Grp. gram.:m.}
\end{itemize}
\begin{itemize}
\item {Proveniência:(Gr. \textunderscore polusundetos\textunderscore )}
\end{itemize}
Espécie de pleonasmo, que consiste em repetir uma conjunção mais vezes do que o exige a ordem gramatical: \textunderscore entendo que assim que está bem\textunderscore .
\section{Polissíntese}
\begin{itemize}
\item {Grp. gram.:f.}
\end{itemize}
\begin{itemize}
\item {Utilização:Philol.}
\end{itemize}
\begin{itemize}
\item {Proveniência:(De \textunderscore poli...\textunderscore  + \textunderscore síntese\textunderscore )}
\end{itemize}
Fenómeno morfológico, em que uma palavra apresenta a elisão de uma ou mais sílabas: \textunderscore cangosta\textunderscore , de \textunderscore canalagosta\textunderscore ; \textunderscore quelha\textunderscore , de \textunderscore canalicula\textunderscore .
O mesmo que \textunderscore haplologia\textunderscore .
\section{Polissintético}
\begin{itemize}
\item {Grp. gram.:adj.}
\end{itemize}
\begin{itemize}
\item {Proveniência:(De \textunderscore poli...\textunderscore  + \textunderscore sintético\textunderscore )}
\end{itemize}
Em que há polissíntese.
O mesmo que \textunderscore holofrástico\textunderscore .
\section{Polissintetismo}
\begin{itemize}
\item {Grp. gram.:m.}
\end{itemize}
\begin{itemize}
\item {Proveniência:(Do gr. \textunderscore polusunthetos\textunderscore )}
\end{itemize}
Carácter de polissintético.
Carácter, que uma língua tem, de que diferentes circunstancias são expressas, não por palavras separadas, mas por modificações de uma palavra.
\section{Polistêmone}
\begin{itemize}
\item {Grp. gram.:adj.}
\end{itemize}
\begin{itemize}
\item {Utilização:Bot.}
\end{itemize}
\begin{itemize}
\item {Proveniência:(Do gr. \textunderscore polus\textunderscore  + \textunderscore stemon\textunderscore )}
\end{itemize}
Que tem muitos estames.
\section{Polistilo}
\begin{itemize}
\item {Grp. gram.:m.}
\end{itemize}
\begin{itemize}
\item {Grp. gram.:Adj.}
\end{itemize}
\begin{itemize}
\item {Proveniência:(Do gr. \textunderscore polus\textunderscore  + \textunderscore stulos\textunderscore )}
\end{itemize}
Edificio de muitas colunas; colunata.
Que constitue colunata.
\section{Politeama}
\begin{itemize}
\item {Grp. gram.:m.}
\end{itemize}
\begin{itemize}
\item {Proveniência:(Do gr. \textunderscore polus\textunderscore  + \textunderscore theama\textunderscore )}
\end{itemize}
Teatro, para vários gêneros de representação.
\section{Politécnica}
\begin{itemize}
\item {Grp. gram.:f.}
\end{itemize}
Escola polyitécnica.
(Fem. de \textunderscore politécnico\textunderscore )
\section{Politécnico}
\begin{itemize}
\item {Grp. gram.:adj.}
\end{itemize}
\begin{itemize}
\item {Proveniência:(Do gr. \textunderscore polus\textunderscore  + \textunderscore tekne\textunderscore )}
\end{itemize}
Que abrange muitas artes ou ciências.
Diz-se especialmente de alguns estabelecimentos, em que se professam ciências várias: \textunderscore escola politécnica\textunderscore ; \textunderscore academia politécnica\textunderscore .
\section{Politeico}
\begin{itemize}
\item {Grp. gram.:adj.}
\end{itemize}
\begin{itemize}
\item {Proveniência:(Do gr. \textunderscore polus\textunderscore  + \textunderscore theos\textunderscore )}
\end{itemize}
Relativo á crença em muitos deuses.
\section{Politeísmo}
\begin{itemize}
\item {Grp. gram.:m.}
\end{itemize}
\begin{itemize}
\item {Proveniência:(Do gr. \textunderscore polus\textunderscore  + \textunderscore theos\textunderscore )}
\end{itemize}
Sistema religioso, que admitte a pluralidade dos deuses.
Paganismo.
\section{Politeísta}
\begin{itemize}
\item {Grp. gram.:m. ,  f.  e  adj.}
\end{itemize}
\begin{itemize}
\item {Proveniência:(De \textunderscore politeísmo\textunderscore )}
\end{itemize}
Pessôa, que segue o politeismo: \textunderscore os Gregos eram politeístas\textunderscore .
\section{Politeístico}
\begin{itemize}
\item {Grp. gram.:adj.}
\end{itemize}
\begin{itemize}
\item {Proveniência:(De \textunderscore politeista\textunderscore )}
\end{itemize}
Relativo ao politeísmo ou aos politeístas; politeico. Cf. Castilho, \textunderscore Fastos\textunderscore , I, p. XXXI.
\section{Politelia}
\begin{itemize}
\item {Grp. gram.:f.}
\end{itemize}
\begin{itemize}
\item {Utilização:Terat.}
\end{itemize}
\begin{itemize}
\item {Proveniência:(Do gr. \textunderscore polus\textunderscore  + \textunderscore thele\textunderscore )}
\end{itemize}
Multiplicidade de mamilos numa só mama.
\section{Polítipo}
\begin{itemize}
\item {Grp. gram.:adj.}
\end{itemize}
\begin{itemize}
\item {Utilização:Bot.}
\end{itemize}
\begin{itemize}
\item {Proveniência:(Do gr. \textunderscore polus\textunderscore  + \textunderscore tupos\textunderscore )}
\end{itemize}
Diz-se dos gêneros, que contém muitas espécies.
\section{Polítrico}
\begin{itemize}
\item {Grp. gram.:m.}
\end{itemize}
\begin{itemize}
\item {Utilização:Bot.}
\end{itemize}
\begin{itemize}
\item {Proveniência:(Do gr. \textunderscore polus\textunderscore  + \textunderscore trix\textunderscore )}
\end{itemize}
Espécie de fêto, mais conhecido por \textunderscore avencão\textunderscore .
\section{Politriquia}
\begin{itemize}
\item {Grp. gram.:f.}
\end{itemize}
Superabundância de cabelos.
(Cp. \textunderscore polítrico\textunderscore )
\section{Politrofia}
\begin{itemize}
\item {Grp. gram.:f.}
\end{itemize}
\begin{itemize}
\item {Utilização:Med.}
\end{itemize}
\begin{itemize}
\item {Proveniência:(Do gr. \textunderscore polus\textunderscore  + \textunderscore trophe\textunderscore )}
\end{itemize}
Nutrição excessiva.
\section{Politropia}
\begin{itemize}
\item {Grp. gram.:f.}
\end{itemize}
Qualidade de polítropo.
\section{Polítropo}
\begin{itemize}
\item {Grp. gram.:adj.}
\end{itemize}
\begin{itemize}
\item {Utilização:Miner.}
\end{itemize}
\begin{itemize}
\item {Proveniência:(Do gr. \textunderscore polus\textunderscore  + \textunderscore trepein\textunderscore )}
\end{itemize}
Diz-se do cristal, cujas lâminas successivas têm as secções principaes inclinadas uma sôbre a outra, formando ângulos diferentes.
\section{Poliuria}
\begin{itemize}
\item {Grp. gram.:f.}
\end{itemize}
\begin{itemize}
\item {Proveniência:(Do gr. \textunderscore polus\textunderscore  + \textunderscore ouron\textunderscore )}
\end{itemize}
Secreção muito abundante de urina.
Diabete não açucarada.
\section{Poliúrico}
\begin{itemize}
\item {Grp. gram.:adj.}
\end{itemize}
Relativo á poliuria.
\section{Polixena}
\begin{itemize}
\item {fónica:csê}
\end{itemize}
\begin{itemize}
\item {Grp. gram.:f.}
\end{itemize}
O mesmo que \textunderscore polixeno\textunderscore .
\section{Polixeno}
\begin{itemize}
\item {fónica:csê}
\end{itemize}
\begin{itemize}
\item {Grp. gram.:m.}
\end{itemize}
\begin{itemize}
\item {Utilização:Miner.}
\end{itemize}
Platina natural ferrífera.
\section{Polizoicidade}
\begin{itemize}
\item {Grp. gram.:f.}
\end{itemize}
Qualidade de polizóico.
\section{Polizóico}
\begin{itemize}
\item {Grp. gram.:adj.}
\end{itemize}
\begin{itemize}
\item {Utilização:Zool.}
\end{itemize}
\begin{itemize}
\item {Proveniência:(Do gr. \textunderscore polus\textunderscore  + \textunderscore zoon\textunderscore )}
\end{itemize}
Diz-se dos animaes, que vivem em bando.
\section{Polygynecia}
\begin{itemize}
\item {Grp. gram.:f.}
\end{itemize}
\begin{itemize}
\item {Utilização:P. us.}
\end{itemize}
\begin{itemize}
\item {Proveniência:(Do gr. \textunderscore polus\textunderscore  + \textunderscore gune\textunderscore )}
\end{itemize}
O mesmo que \textunderscore polygamia\textunderscore . Cf. Ferrer, \textunderscore Dir. Nat.\textunderscore , 170.
\section{Polygynia}
\begin{itemize}
\item {Grp. gram.:f.}
\end{itemize}
Estado ou qualidade de polýgyno.
\section{Polygýnio}
\begin{itemize}
\item {Grp. gram.:adj.}
\end{itemize}
\begin{itemize}
\item {Utilização:Bot.}
\end{itemize}
\begin{itemize}
\item {Proveniência:(Do gr. \textunderscore polus\textunderscore  + \textunderscore gune\textunderscore )}
\end{itemize}
Que tem muitos pistillos em cada flôr.
\section{Polýgyno}
\begin{itemize}
\item {Grp. gram.:adj.}
\end{itemize}
O mesmo ou melhor que \textunderscore polygýnio\textunderscore .
\section{Polyhalito}
\begin{itemize}
\item {Grp. gram.:m.}
\end{itemize}
\begin{itemize}
\item {Proveniência:(Do gr. \textunderscore polus\textunderscore  + \textunderscore halos\textunderscore )}
\end{itemize}
Sulfato natural de cálcio, magnésio e potássio.
\section{Polylépido}
\begin{itemize}
\item {Grp. gram.:adj.}
\end{itemize}
\begin{itemize}
\item {Utilização:Bot.}
\end{itemize}
\begin{itemize}
\item {Proveniência:(Do gr. \textunderscore polus\textunderscore  + \textunderscore lepis\textunderscore )}
\end{itemize}
Que tem muitas escamas.
\section{Polylóbio}
\begin{itemize}
\item {Grp. gram.:m.}
\end{itemize}
Gênero de plantas leguminosas.
\section{Polylóbulo}
\begin{itemize}
\item {Grp. gram.:m.}
\end{itemize}
\begin{itemize}
\item {Proveniência:(De \textunderscore poly...\textunderscore  + \textunderscore lóbulo\textunderscore )}
\end{itemize}
Ornato architectónico, com muitos lóbos.
\section{Polymastia}
\begin{itemize}
\item {Grp. gram.:f.}
\end{itemize}
\begin{itemize}
\item {Utilização:Terat.}
\end{itemize}
\begin{itemize}
\item {Proveniência:(Do gr. \textunderscore polus\textunderscore  + \textunderscore mastos\textunderscore )}
\end{itemize}
Anomalia do indivíduo, que tem muitas mamas.
\section{Polýmatha}
\begin{itemize}
\item {Grp. gram.:m.}
\end{itemize}
O mesmo que \textunderscore polýmatho\textunderscore .
\section{Polymathia}
\begin{itemize}
\item {Grp. gram.:f.}
\end{itemize}
\begin{itemize}
\item {Proveniência:(De \textunderscore polýmatho\textunderscore )}
\end{itemize}
Instrucção extensa e variada.
\section{Polymáthico}
\begin{itemize}
\item {Grp. gram.:adj.}
\end{itemize}
Relativo á polymathia.
\section{Polýmatho}
\begin{itemize}
\item {Grp. gram.:m.  e  adj.}
\end{itemize}
\begin{itemize}
\item {Proveniência:(Gr. \textunderscore polumathos\textunderscore )}
\end{itemize}
O que estudou ou sabe muitas sciências; polýgrapho.
\section{Polymelia}
\begin{itemize}
\item {Grp. gram.:f.}
\end{itemize}
Qualidade de polýmelo.
\section{Polýmelo}
\begin{itemize}
\item {Grp. gram.:adj.}
\end{itemize}
\begin{itemize}
\item {Proveniência:(Do gr. \textunderscore polus\textunderscore  + \textunderscore melos\textunderscore )}
\end{itemize}
Diz-se do monstro, que tem membros accessórios.
\section{Polymeria}
\begin{itemize}
\item {Grp. gram.:f.}
\end{itemize}
Estado do que é polýmero.
\section{Polymerismo}
\begin{itemize}
\item {Grp. gram.:m.}
\end{itemize}
Estado do que é polýmero.
\section{Polýmero}
\begin{itemize}
\item {Grp. gram.:adj.}
\end{itemize}
\begin{itemize}
\item {Utilização:Chím.}
\end{itemize}
\begin{itemize}
\item {Proveniência:(Do gr. \textunderscore polus\textunderscore  + \textunderscore meros\textunderscore )}
\end{itemize}
Que contém os mesmos elementos na mesma quantidade relativa, mas não na mesma quantidade absoluta.
\section{Polýmnia}
\begin{itemize}
\item {Grp. gram.:f.}
\end{itemize}
\begin{itemize}
\item {Proveniência:(De \textunderscore Polýmnia\textunderscore , n. p.)}
\end{itemize}
Gênero de plantas, da fam. das compostas.
\section{Polýmnico}
\begin{itemize}
\item {Grp. gram.:adj.}
\end{itemize}
\begin{itemize}
\item {Proveniência:(De \textunderscore Polýmnia\textunderscore , n. p.)}
\end{itemize}
Relativo á Rhetórica.
\section{Polymorphia}
\begin{itemize}
\item {Grp. gram.:f.}
\end{itemize}
Propriedade do que é polymorpho.
\section{Polymorphismo}
\begin{itemize}
\item {Grp. gram.:m.}
\end{itemize}
Propriedade do que é polymorpho.
\section{Polymorpho}
\begin{itemize}
\item {Grp. gram.:adj.}
\end{itemize}
\begin{itemize}
\item {Proveniência:(Do gr. \textunderscore polus\textunderscore  + \textunderscore morphe\textunderscore )}
\end{itemize}
Que se apresenta sob diversas fórmas; que é sujeito a variar de fórma.
\section{Polynervado}
\begin{itemize}
\item {Grp. gram.:adj.}
\end{itemize}
\begin{itemize}
\item {Utilização:Bot.}
\end{itemize}
\begin{itemize}
\item {Proveniência:(De \textunderscore poly...\textunderscore  + \textunderscore nervo\textunderscore )}
\end{itemize}
Que tem muitas nervuras.
\section{Polynésios}
\begin{itemize}
\item {Grp. gram.:m. pl.}
\end{itemize}
Selvagens dos mares do Sul da Oceânia.
\section{Polyneurite}
\begin{itemize}
\item {Grp. gram.:f.}
\end{itemize}
\begin{itemize}
\item {Proveniência:(De \textunderscore poly...\textunderscore  + \textunderscore neurite\textunderscore )}
\end{itemize}
Neurite periphérica.
\section{Polyneurítico}
\begin{itemize}
\item {Grp. gram.:adj.}
\end{itemize}
\begin{itemize}
\item {Proveniência:(De \textunderscore poly...\textunderscore  + \textunderscore neurítico\textunderscore )}
\end{itemize}
Relativo á polyneurite.
\section{Polynómio}
\begin{itemize}
\item {Grp. gram.:m.}
\end{itemize}
\begin{itemize}
\item {Proveniência:(Do gr. \textunderscore polus\textunderscore  + \textunderscore nomos\textunderscore )}
\end{itemize}
Qualquer quantidade algébrica, composta de muitos termos, separados pelo sinal + ou -.
\section{Polynuclear}
\begin{itemize}
\item {Grp. gram.:adj.}
\end{itemize}
\begin{itemize}
\item {Proveniência:(De \textunderscore poly...\textunderscore  + \textunderscore núcleo\textunderscore )}
\end{itemize}
Que tem muitos núcleos.
\section{Polyónymo}
\begin{itemize}
\item {Grp. gram.:adj.}
\end{itemize}
\begin{itemize}
\item {Proveniência:(Do gr. \textunderscore polus\textunderscore  + \textunderscore onuma\textunderscore )}
\end{itemize}
Que tem muitos nomes; que póde nomear-se de várias fórmas.
\section{Polyope}
\begin{itemize}
\item {Grp. gram.:m.}
\end{itemize}
Aquelle que soffre polyopia.
\section{Polyopia}
\begin{itemize}
\item {Grp. gram.:f.}
\end{itemize}
\begin{itemize}
\item {Proveniência:(Do gr. \textunderscore polus\textunderscore  + \textunderscore ops\textunderscore )}
\end{itemize}
Estado mórbido dos que vêem os objectos multiplicados.
\section{Polyorama}
\begin{itemize}
\item {Grp. gram.:m.}
\end{itemize}
\begin{itemize}
\item {Proveniência:(Do gr. \textunderscore polus\textunderscore  + \textunderscore orama\textunderscore )}
\end{itemize}
Espécie de panorama, em que os quadros móveis, penetrando-se reciprocamente, mudam de contornos e se transfiguram, aos olhos do observador.
\section{Polyorchidia}
\begin{itemize}
\item {fónica:qui}
\end{itemize}
\begin{itemize}
\item {Grp. gram.:f.}
\end{itemize}
\begin{itemize}
\item {Utilização:Anat.}
\end{itemize}
\begin{itemize}
\item {Proveniência:(Do gr. \textunderscore polus\textunderscore  + \textunderscore orkhis\textunderscore )}
\end{itemize}
Existência de mais de dois testículos num homem.
\section{Polýpago}
\begin{itemize}
\item {Grp. gram.:m.}
\end{itemize}
\begin{itemize}
\item {Utilização:Terat.}
\end{itemize}
\begin{itemize}
\item {Proveniência:(Do gr. \textunderscore polus\textunderscore  + \textunderscore phageis\textunderscore )}
\end{itemize}
Monstro monocéphalo, formado de dois corpos, de eixos parallelos.
\section{Polypedia}
\begin{itemize}
\item {Grp. gram.:f.}
\end{itemize}
\begin{itemize}
\item {Utilização:Med.}
\end{itemize}
\begin{itemize}
\item {Proveniência:(Do gr. \textunderscore polus\textunderscore  + \textunderscore pais\textunderscore , \textunderscore paidos\textunderscore )}
\end{itemize}
Presença de muitos fetos, na mesma gestação.
\section{Polypeiro}
\begin{itemize}
\item {Grp. gram.:m.}
\end{itemize}
Habitação de pólypos, que vivem agrupados.
Grupo de pólypos.
\section{Polypétalo}
\begin{itemize}
\item {Grp. gram.:adj.}
\end{itemize}
\begin{itemize}
\item {Proveniência:(De \textunderscore poly...\textunderscore  + \textunderscore pétala\textunderscore )}
\end{itemize}
Que tem muitas pétalas.
\section{Polyphagia}
\begin{itemize}
\item {Grp. gram.:f.}
\end{itemize}
Qualidade de polýphago.
\section{Polýphago}
\begin{itemize}
\item {Grp. gram.:adj.}
\end{itemize}
\begin{itemize}
\item {Proveniência:(Do gr. \textunderscore polus\textunderscore  + \textunderscore phagein\textunderscore )}
\end{itemize}
Que come muito; que tem fome canina.
\section{Polyphásico}
\begin{itemize}
\item {Grp. gram.:adj.}
\end{itemize}
\begin{itemize}
\item {Utilização:Phýs.}
\end{itemize}
Diz-se das correntes, que espalham e distribuem a todas as distâncias as fôrças motrizes naturaes.
\section{Polyphemo}
\begin{itemize}
\item {Grp. gram.:m.}
\end{itemize}
\begin{itemize}
\item {Proveniência:(De \textunderscore Polyphemo\textunderscore , n. p.)}
\end{itemize}
Espécie de escaravelho.
Nome de um crustáceo.
\section{Polyphonía}
\begin{itemize}
\item {Grp. gram.:f.}
\end{itemize}
Pluralidade de sons e de articulações, relativa a um sinal vocal, na escritura dos Assýrios.
(Cp. \textunderscore polyphono\textunderscore )
\section{Polyphónico}
\begin{itemize}
\item {Grp. gram.:adj.}
\end{itemize}
Relativo á polyphonia.
\section{Polyphono}
\begin{itemize}
\item {Grp. gram.:adj.}
\end{itemize}
\begin{itemize}
\item {Proveniência:(Do gr. \textunderscore polus\textunderscore  + \textunderscore phone\textunderscore )}
\end{itemize}
Que repete os sons muitas vezes.
\section{Polyphyllo}
\begin{itemize}
\item {Grp. gram.:adj.}
\end{itemize}
\begin{itemize}
\item {Utilização:Bot.}
\end{itemize}
\begin{itemize}
\item {Proveniência:(Do gr. \textunderscore polus\textunderscore  + \textunderscore phullon\textunderscore )}
\end{itemize}
Formado de muitos folíolos.
\section{Polyphyodontes}
\begin{itemize}
\item {Grp. gram.:m. pl.}
\end{itemize}
\begin{itemize}
\item {Proveniência:(Do gr. \textunderscore polus\textunderscore  + \textunderscore phio\textunderscore  + \textunderscore odous\textunderscore , \textunderscore odontos\textunderscore )}
\end{itemize}
Animaes que têm mais do que uma dentição.
\section{Polyphysia}
\begin{itemize}
\item {Grp. gram.:f.}
\end{itemize}
\begin{itemize}
\item {Utilização:Med.}
\end{itemize}
\begin{itemize}
\item {Proveniência:(Do gr. \textunderscore polus\textunderscore  + \textunderscore phusa\textunderscore )}
\end{itemize}
Abundância de gases ou flatuosidades.
\section{Polýphyto}
\begin{itemize}
\item {Grp. gram.:adj.}
\end{itemize}
\begin{itemize}
\item {Utilização:Bot.}
\end{itemize}
\begin{itemize}
\item {Proveniência:(Do gr. \textunderscore polus\textunderscore  + \textunderscore phuton\textunderscore )}
\end{itemize}
Relativo a muitas plantas.
Diz-se dos gêneros, que comprehendem muitas plantas.
\section{Polypiforme}
\begin{itemize}
\item {Grp. gram.:adj.}
\end{itemize}
\begin{itemize}
\item {Proveniência:(De \textunderscore polypo\textunderscore  + \textunderscore fórma\textunderscore )}
\end{itemize}
Que tem fórma de pólypo.
\section{Polypiose}
\begin{itemize}
\item {Grp. gram.:f.}
\end{itemize}
\begin{itemize}
\item {Utilização:Med.}
\end{itemize}
\begin{itemize}
\item {Proveniência:(Do gr. \textunderscore polus\textunderscore  + \textunderscore pion\textunderscore )}
\end{itemize}
O mesmo que \textunderscore obesidade\textunderscore .
\section{Polyplectro}
\begin{itemize}
\item {Grp. gram.:m.}
\end{itemize}
\begin{itemize}
\item {Proveniência:(Do gr. \textunderscore polus\textunderscore  + \textunderscore plektron\textunderscore )}
\end{itemize}
Instrumento, destinado á prolongação de sons num teclado, imitando instrumentos de arco.
\section{Polypnéa}
\begin{itemize}
\item {Grp. gram.:f.}
\end{itemize}
\begin{itemize}
\item {Utilização:Med.}
\end{itemize}
\begin{itemize}
\item {Proveniência:(Do gr. \textunderscore polus\textunderscore  + \textunderscore pneín\textunderscore )}
\end{itemize}
Respiração rápida e superficial.
\section{Pólypo}
\begin{itemize}
\item {Grp. gram.:m.}
\end{itemize}
\begin{itemize}
\item {Grp. gram.:Pl.}
\end{itemize}
\begin{itemize}
\item {Proveniência:(Gr. \textunderscore polupous\textunderscore )}
\end{itemize}
Excrescência carnosa, fibrosa, etc., que se póde desenvolver em qualquer membrana mucosa.
Concreção sanguínea, que se fórma no coração ou nos grandes vasos.
Animaes de corpo molle e contráctil, com a cabeça rodeada de tentáculos radiados.
\section{Polypodiáceas}
\begin{itemize}
\item {Grp. gram.:f. pl.}
\end{itemize}
Família de plantas, que têm por typo o polypódio.
\section{Polypódio}
\begin{itemize}
\item {Grp. gram.:adj.}
\end{itemize}
\begin{itemize}
\item {Grp. gram.:M.}
\end{itemize}
\begin{itemize}
\item {Proveniência:(Gr. \textunderscore polupodion\textunderscore )}
\end{itemize}
Que tem muitos pés.
Gênero de plantas parasitas da fam. dos fêtos.
\section{Pólypo-hydráceo}
\begin{itemize}
\item {Grp. gram.:m.}
\end{itemize}
Reunião de várias hydras, formando só uma entidade zoológica.
\section{Polýporo}
\begin{itemize}
\item {Grp. gram.:m.}
\end{itemize}
\begin{itemize}
\item {Proveniência:(Do gr. \textunderscore polus\textunderscore  + \textunderscore poros\textunderscore )}
\end{itemize}
Gênero de cogumelos.
\section{Polyposia}
\begin{itemize}
\item {Grp. gram.:f.}
\end{itemize}
\begin{itemize}
\item {Utilização:Med.}
\end{itemize}
O mesmo que \textunderscore polydipsia\textunderscore .
\section{Polyposo}
\begin{itemize}
\item {Grp. gram.:adj.}
\end{itemize}
Que tem a natureza do pólypo.
\section{Polýptero}
\begin{itemize}
\item {Grp. gram.:m.}
\end{itemize}
\begin{itemize}
\item {Proveniência:(Do gr. \textunderscore polus\textunderscore  + \textunderscore pteron\textunderscore )}
\end{itemize}
Animal de cabeça alongada, espécie de lúcio, descoberto no Nilo.
\section{Polyptoto}
\begin{itemize}
\item {Grp. gram.:m.}
\end{itemize}
\begin{itemize}
\item {Utilização:Gram.}
\end{itemize}
\begin{itemize}
\item {Proveniência:(Gr. \textunderscore poluptoton\textunderscore )}
\end{itemize}
Acto de empregar, num período, uma palavra sob diversas fórmas grammaticaes.
\section{Polyrhizo}
\begin{itemize}
\item {Grp. gram.:adj.}
\end{itemize}
\begin{itemize}
\item {Utilização:Bot.}
\end{itemize}
\begin{itemize}
\item {Proveniência:(Do gr. \textunderscore polus\textunderscore  + \textunderscore rhiza\textunderscore )}
\end{itemize}
Que tem muitas raízes.
\section{Polyrýthmico}
\begin{itemize}
\item {fónica:ri}
\end{itemize}
\begin{itemize}
\item {Grp. gram.:adj.}
\end{itemize}
\begin{itemize}
\item {Utilização:Mús.}
\end{itemize}
\begin{itemize}
\item {Proveniência:(De \textunderscore poly...\textunderscore  + \textunderscore rýthmico\textunderscore )}
\end{itemize}
Que se compõe de vários rythmos.
Que tem o rythmo muito variado.
\section{Polysarcia}
\begin{itemize}
\item {fónica:sar}
\end{itemize}
\begin{itemize}
\item {Grp. gram.:f.}
\end{itemize}
\begin{itemize}
\item {Utilização:Med.}
\end{itemize}
\begin{itemize}
\item {Proveniência:(Do gr. \textunderscore polus\textunderscore  + \textunderscore sarx\textunderscore , \textunderscore sarkhos\textunderscore )}
\end{itemize}
Aumento anormal dos músculos ou do tecido adiposo.
\section{Polyscópio}
\begin{itemize}
\item {Grp. gram.:m.}
\end{itemize}
\begin{itemize}
\item {Proveniência:(Do gr. \textunderscore polus\textunderscore  + \textunderscore skopein\textunderscore )}
\end{itemize}
Óculo ou lente, que apresenta um objecto multiplicado.
\section{Polysialia}
\begin{itemize}
\item {fónica:si}
\end{itemize}
\begin{itemize}
\item {Grp. gram.:f.}
\end{itemize}
\begin{itemize}
\item {Utilização:Med.}
\end{itemize}
\begin{itemize}
\item {Proveniência:(Do gr. \textunderscore polus\textunderscore  + \textunderscore sialon\textunderscore )}
\end{itemize}
Secreção abundante de saliva.
\section{Polyspermo}
\begin{itemize}
\item {Grp. gram.:adj.}
\end{itemize}
\begin{itemize}
\item {Proveniência:(Do gr. \textunderscore polus\textunderscore  + \textunderscore sperma\textunderscore )}
\end{itemize}
Que tem muitos grãos, (falando-se de frutos).
\section{Polýsporo}
\begin{itemize}
\item {Grp. gram.:adj.}
\end{itemize}
\begin{itemize}
\item {Utilização:Bot.}
\end{itemize}
\begin{itemize}
\item {Proveniência:(Do gr. \textunderscore polus\textunderscore  + \textunderscore spora\textunderscore )}
\end{itemize}
Que contém muitos esporos.
\section{Polystêmone}
\begin{itemize}
\item {Grp. gram.:adj.}
\end{itemize}
\begin{itemize}
\item {Utilização:Bot.}
\end{itemize}
\begin{itemize}
\item {Proveniência:(Do gr. \textunderscore polus\textunderscore  + \textunderscore stemon\textunderscore )}
\end{itemize}
Que tem muitos estames.
\section{Polystylo}
\begin{itemize}
\item {Grp. gram.:m.}
\end{itemize}
\begin{itemize}
\item {Grp. gram.:Adj.}
\end{itemize}
\begin{itemize}
\item {Proveniência:(Do gr. \textunderscore polus\textunderscore  + \textunderscore stulos\textunderscore )}
\end{itemize}
Edificio de muitas columnas; columnata.
Que constitue columnata.
\section{Polysyllábico}
\begin{itemize}
\item {fónica:si}
\end{itemize}
\begin{itemize}
\item {Grp. gram.:adj.}
\end{itemize}
Relativo ao polysýllabo.
Que tem mais de uma sýllaba.
\section{Polysýllabo}
\begin{itemize}
\item {fónica:si}
\end{itemize}
\begin{itemize}
\item {Grp. gram.:adj.}
\end{itemize}
\begin{itemize}
\item {Grp. gram.:M.}
\end{itemize}
\begin{itemize}
\item {Proveniência:(Gr. \textunderscore polusullabos\textunderscore )}
\end{itemize}
O mesmo que \textunderscore polysyllábico\textunderscore .
Palavra que tem mais de uma sýllaba.
\section{Polysyllogístico}
\begin{itemize}
\item {fónica:si}
\end{itemize}
\begin{itemize}
\item {Grp. gram.:adj.}
\end{itemize}
\begin{itemize}
\item {Proveniência:(De \textunderscore poly...\textunderscore  + \textunderscore sullogismo\textunderscore )}
\end{itemize}
Diz-se do raciocínio, composto de um encadeamento de syllogismos.
\section{Polysýndeto}
\begin{itemize}
\item {Grp. gram.:m.}
\end{itemize}
O mesmo ou melhor que \textunderscore polysýndeton\textunderscore .
\section{Polysýndeton}
\begin{itemize}
\item {fónica:sin}
\end{itemize}
\begin{itemize}
\item {Grp. gram.:m.}
\end{itemize}
\begin{itemize}
\item {Proveniência:(Gr. \textunderscore polusundetos\textunderscore )}
\end{itemize}
Espécie de pleonasmo, que consiste em repetir uma conjuncção mais vezes do que o exige a ordem grammatical: \textunderscore entendo que assim que está bem\textunderscore .
\section{Polysýnthese}
\begin{itemize}
\item {fónica:sin}
\end{itemize}
\begin{itemize}
\item {Grp. gram.:f.}
\end{itemize}
\begin{itemize}
\item {Utilização:Philol.}
\end{itemize}
\begin{itemize}
\item {Proveniência:(De \textunderscore poly...\textunderscore  + \textunderscore sýnthese\textunderscore )}
\end{itemize}
Phenómeno morphológico, em que uma palavra apresenta a elisão de uma ou mais sýllabas: \textunderscore cangosta\textunderscore , de \textunderscore canalagosta\textunderscore ; \textunderscore quelha\textunderscore , de \textunderscore canalicula\textunderscore .
O mesmo que \textunderscore haplologia\textunderscore .
\section{Polysynthético}
\begin{itemize}
\item {fónica:sin}
\end{itemize}
\begin{itemize}
\item {Grp. gram.:adj.}
\end{itemize}
\begin{itemize}
\item {Proveniência:(De \textunderscore poly...\textunderscore  + \textunderscore synthético\textunderscore )}
\end{itemize}
Em que há polysýnthese.
O mesmo que \textunderscore holophrástico\textunderscore .
\section{Polysynthetismo}
\begin{itemize}
\item {fónica:sin}
\end{itemize}
\begin{itemize}
\item {Grp. gram.:m.}
\end{itemize}
\begin{itemize}
\item {Proveniência:(Do gr. \textunderscore polusunthetos\textunderscore )}
\end{itemize}
Carácter de polysynthético.
Carácter, que uma língua tem, de que differentes circunstancias são expressas, não por palavras separadas, mas por modificações de uma palavra.
\section{Polytéchnica}
\begin{itemize}
\item {Grp. gram.:f.}
\end{itemize}
Escola polytéchnica.
(Fem. de \textunderscore polytéchnico\textunderscore )
\section{Polytéchnico}
\begin{itemize}
\item {Grp. gram.:adj.}
\end{itemize}
\begin{itemize}
\item {Proveniência:(Do gr. \textunderscore polus\textunderscore  + \textunderscore tekne\textunderscore )}
\end{itemize}
Que abrange muitas artes ou sciências.
Diz-se especialmente de alguns estabelecimentos, em que se professam sciências várias: \textunderscore escola polytéchnica\textunderscore ; \textunderscore academia polytéchnica\textunderscore .
\section{Polytheama}
\begin{itemize}
\item {Grp. gram.:m.}
\end{itemize}
\begin{itemize}
\item {Proveniência:(Do gr. \textunderscore polus\textunderscore  + \textunderscore theama\textunderscore )}
\end{itemize}
Theatro, para vários gêneros de representação.
\section{Polytheico}
\begin{itemize}
\item {Grp. gram.:adj.}
\end{itemize}
\begin{itemize}
\item {Proveniência:(Do gr. \textunderscore polus\textunderscore  + \textunderscore theos\textunderscore )}
\end{itemize}
Relativo á crença em muitos deuses.
\section{Polytheísmo}
\begin{itemize}
\item {Grp. gram.:m.}
\end{itemize}
\begin{itemize}
\item {Proveniência:(Do gr. \textunderscore polus\textunderscore  + \textunderscore theos\textunderscore )}
\end{itemize}
Systema religioso, que admitte a pluralidade dos deuses.
Paganismo.
\section{Polytheísta}
\begin{itemize}
\item {Grp. gram.:m. ,  f.  e  adj.}
\end{itemize}
\begin{itemize}
\item {Proveniência:(De \textunderscore polytheísmo\textunderscore )}
\end{itemize}
Pessôa, que segue o polytheismo: \textunderscore os Gregos eram polytheístas\textunderscore .
\section{Polytheístico}
\begin{itemize}
\item {Grp. gram.:adj.}
\end{itemize}
\begin{itemize}
\item {Proveniência:(De \textunderscore polytheista\textunderscore )}
\end{itemize}
Relativo ao polytheísmo ou aos polytheístas; polytheico. Cf. Castilho, \textunderscore Fastos\textunderscore , I, p. XXXI.
\section{Polythelia}
\begin{itemize}
\item {Grp. gram.:f.}
\end{itemize}
\begin{itemize}
\item {Utilização:Terat.}
\end{itemize}
\begin{itemize}
\item {Proveniência:(Do gr. \textunderscore polus\textunderscore  + \textunderscore thele\textunderscore )}
\end{itemize}
Multiplicidade de mamilos numa só mama.
\section{Polytrichia}
\begin{itemize}
\item {fónica:quia}
\end{itemize}
\begin{itemize}
\item {Grp. gram.:f.}
\end{itemize}
Superabundância de cabellos.
(Cp. \textunderscore polýtricho\textunderscore )
\section{Polýtricho}
\begin{itemize}
\item {fónica:co}
\end{itemize}
\begin{itemize}
\item {Grp. gram.:m.}
\end{itemize}
\begin{itemize}
\item {Utilização:Bot.}
\end{itemize}
\begin{itemize}
\item {Proveniência:(Do gr. \textunderscore polus\textunderscore  + \textunderscore trix\textunderscore )}
\end{itemize}
Espécie de fêto, mais conhecido por \textunderscore avencão\textunderscore .
\section{Polytrophia}
\begin{itemize}
\item {Grp. gram.:f.}
\end{itemize}
\begin{itemize}
\item {Utilização:Med.}
\end{itemize}
\begin{itemize}
\item {Proveniência:(Do gr. \textunderscore polus\textunderscore  + \textunderscore trophe\textunderscore )}
\end{itemize}
Nutrição excessiva.
\section{Polytropia}
\begin{itemize}
\item {Grp. gram.:f.}
\end{itemize}
Qualidade de polýtropo.
\section{Polýtropo}
\begin{itemize}
\item {Grp. gram.:adj.}
\end{itemize}
\begin{itemize}
\item {Utilização:Miner.}
\end{itemize}
\begin{itemize}
\item {Proveniência:(Do gr. \textunderscore polus\textunderscore  + \textunderscore trepein\textunderscore )}
\end{itemize}
Diz-se do crystal, cujas lâminas successivas têm as secções principaes inclinadas uma sôbre a outra, formando ângulos differentes.
\section{Polýtypo}
\begin{itemize}
\item {Grp. gram.:adj.}
\end{itemize}
\begin{itemize}
\item {Utilização:Bot.}
\end{itemize}
\begin{itemize}
\item {Proveniência:(Do gr. \textunderscore polus\textunderscore  + \textunderscore tupos\textunderscore )}
\end{itemize}
Diz-se dos gêneros, que contém muitas espécies.
\section{Polyuria}
\begin{itemize}
\item {Grp. gram.:f.}
\end{itemize}
\begin{itemize}
\item {Proveniência:(Do gr. \textunderscore polus\textunderscore  + \textunderscore ouron\textunderscore )}
\end{itemize}
Secreção muito abundante de urina.
Diabete não açucarada.
\section{Polyúrico}
\begin{itemize}
\item {Grp. gram.:adj.}
\end{itemize}
Relativo á polyuria.
\section{Polyxena}
\begin{itemize}
\item {fónica:csê}
\end{itemize}
\begin{itemize}
\item {Grp. gram.:f.}
\end{itemize}
O mesmo que \textunderscore polyxeno\textunderscore .
\section{Polyxeno}
\begin{itemize}
\item {fónica:csê}
\end{itemize}
\begin{itemize}
\item {Grp. gram.:m.}
\end{itemize}
\begin{itemize}
\item {Utilização:Miner.}
\end{itemize}
Platina natural ferrífera.
\section{Polyzoicidade}
\begin{itemize}
\item {Grp. gram.:f.}
\end{itemize}
Qualidade de polyzóico.
\section{Polyzóico}
\begin{itemize}
\item {Grp. gram.:adj.}
\end{itemize}
\begin{itemize}
\item {Utilização:Zool.}
\end{itemize}
\begin{itemize}
\item {Proveniência:(Do gr. \textunderscore polus\textunderscore  + \textunderscore zoon\textunderscore )}
\end{itemize}
Diz-se dos animaes, que vivem em bando.
\section{Poma}
\begin{itemize}
\item {Grp. gram.:f.}
\end{itemize}
\begin{itemize}
\item {Utilização:Ant.}
\end{itemize}
\begin{itemize}
\item {Proveniência:(De \textunderscore pomo\textunderscore )}
\end{itemize}
Seio da mulhér; mama.
Esphera terrestre, para estudo. (Us. por Pedro Núnez)
Qualquer esphera ou bóla:«\textunderscore ...torres... Em todas estão liões dourados sôbre bollas ou pomas\textunderscore ». \textunderscore Peregrinação\textunderscore , XCIII.
\section{Poma-candil}
\begin{itemize}
\item {Grp. gram.:m.}
\end{itemize}
\begin{itemize}
\item {Utilização:Ant.}
\end{itemize}
Candil, em fórma de pomo, e próprio para aquecer as mãos.
\section{Pomáceas}
\begin{itemize}
\item {Grp. gram.:f. pl.}
\end{itemize}
Tríbo de plantas rosáceas, que abrange os gêneros que produzem pomos, como a macieira, a pereira, etc.
(Fem. pl. de \textunderscore pomáceo\textunderscore )
\section{Pomáceo}
\begin{itemize}
\item {Grp. gram.:adj.}
\end{itemize}
Cujos frutos são pomos.
\section{Pomada}
\begin{itemize}
\item {Grp. gram.:f.}
\end{itemize}
\begin{itemize}
\item {Utilização:Bras}
\end{itemize}
\begin{itemize}
\item {Utilização:Bras}
\end{itemize}
\begin{itemize}
\item {Proveniência:(De \textunderscore pomo\textunderscore )}
\end{itemize}
Preparado de perfumistas ou pharmacêuticos, obtido pela mistura de uma gordura animal com uma ou muitas substâncias aromáticas ou medicinaes.
Presumpção, vaidade.
Mentira, pêta.
\section{Pomadista}
\begin{itemize}
\item {Grp. gram.:m.}
\end{itemize}
\begin{itemize}
\item {Utilização:Bras}
\end{itemize}
\begin{itemize}
\item {Proveniência:(De \textunderscore pomada\textunderscore )}
\end{itemize}
Pedante; pessôa vaidosa.
Indivíduo mentiroso.
\section{Pomagem}
\begin{itemize}
\item {Grp. gram.:f.}
\end{itemize}
\begin{itemize}
\item {Utilização:Des.}
\end{itemize}
\begin{itemize}
\item {Proveniência:(De \textunderscore pomo\textunderscore )}
\end{itemize}
O mesmo que \textunderscore fruta\textunderscore . Cf. \textunderscore Menina e Moça\textunderscore , 37.
\section{Pomar}
\begin{itemize}
\item {Grp. gram.:m.}
\end{itemize}
\begin{itemize}
\item {Proveniência:(Do lat. \textunderscore pomarium\textunderscore )}
\end{itemize}
Arvoredo fructífero.
Terreno, em que há muitas árvores de fruto.
\section{Pomarada}
\begin{itemize}
\item {Grp. gram.:f.}
\end{itemize}
Reunião de pomares.
\section{Pomareiro}
\begin{itemize}
\item {Grp. gram.:adj.}
\end{itemize}
\begin{itemize}
\item {Grp. gram.:M.}
\end{itemize}
Relativo a pomar.
Acostumado a tratar de pomares:«\textunderscore ...quando ajudadas de pomareiras mãos, ellas\textunderscore  (árvores) \textunderscore produziam...\textunderscore »\textunderscore Menina e Moça\textunderscore .
Cultivador ou guarda de um pomar.
\section{Pomária}
\begin{itemize}
\item {Grp. gram.:f.}
\end{itemize}
\begin{itemize}
\item {Proveniência:(Do gr. \textunderscore poma\textunderscore )}
\end{itemize}
Gênero de arbustos leguminosos da América tropical.
\section{Pomba}
\begin{itemize}
\item {Grp. gram.:f.}
\end{itemize}
\begin{itemize}
\item {Utilização:Prov.}
\end{itemize}
\begin{itemize}
\item {Utilização:minh.}
\end{itemize}
\begin{itemize}
\item {Proveniência:(Do lat. \textunderscore palumba\textunderscore )}
\end{itemize}
Fêmea do pombo.
Vasilha de cobre, para onde se passa o caldo limpo da cana, em os engenhos de açúcar.
Peça do tear, espécie de pegadoiro, encaixado na queixa superior do pente.
\section{Pomba-do-mar}
\begin{itemize}
\item {Grp. gram.:f.}
\end{itemize}
\begin{itemize}
\item {Utilização:Mad}
\end{itemize}
Ave, o mesmo que \textunderscore anjinho\textunderscore .
\section{Pombal}
\begin{itemize}
\item {Grp. gram.:m.}
\end{itemize}
Lugar, onde se recolhem ou se criam pombos.
Variedade de uva branca.
\section{Pombalesco}
\begin{itemize}
\item {fónica:lês}
\end{itemize}
\begin{itemize}
\item {Grp. gram.:adj.}
\end{itemize}
O mesmo que \textunderscore pombalino\textunderscore . Cf. Alv. Mendes, \textunderscore Discursos\textunderscore , 95.
\section{Pombália}
\begin{itemize}
\item {Grp. gram.:f.}
\end{itemize}
\begin{itemize}
\item {Proveniência:(De \textunderscore Pombal\textunderscore , n. p. de um estadista português)}
\end{itemize}
Gênero de plantas violáceas.
\section{Pombalino}
\begin{itemize}
\item {Grp. gram.:adj.}
\end{itemize}
\begin{itemize}
\item {Proveniência:(De \textunderscore Pombal\textunderscore , n. p.)}
\end{itemize}
Relativo ao primeiro Marquês de Pombal ou ao seu tempo.
\section{Pombalista}
\begin{itemize}
\item {Grp. gram.:m.  e  adj.}
\end{itemize}
Affeiçoado ao primeiro Marquês de Pombal; que segue ou applaude o systema governativo daquelle estadista. Cf. Camillo, \textunderscore M. de Pombal\textunderscore , 104 e 195.
\section{Pombe}
\begin{itemize}
\item {Grp. gram.:m.}
\end{itemize}
Nome, com que os Angolenses designam o sertão.
\section{Pombe}
\begin{itemize}
\item {Grp. gram.:m.}
\end{itemize}
Espécie de cerveja, em Moçambique.
\section{Pombear}
\begin{itemize}
\item {Grp. gram.:v. t.}
\end{itemize}
\begin{itemize}
\item {Grp. gram.:V. i.}
\end{itemize}
Ir no encalço de.
Espionar.
Exercer a profissão de pombeiro^2.
\section{Pombeirar}
\begin{itemize}
\item {Grp. gram.:v. t.  e  i.}
\end{itemize}
O mesmo que \textunderscore pombear\textunderscore .
\section{Pombeiro}
\begin{itemize}
\item {Grp. gram.:m.}
\end{itemize}
\begin{itemize}
\item {Utilização:Bras}
\end{itemize}
\begin{itemize}
\item {Grp. gram.:Adj.}
\end{itemize}
\begin{itemize}
\item {Proveniência:(Do lat. \textunderscore palumbarius\textunderscore )}
\end{itemize}
Vendedor ambulante de gallinhas.
Diz-se de uma variedade miúda de milho branco. Cf. \textunderscore Portugal Agricola\textunderscore , 9.^o anno, 367.
\section{Pombeiro}
\begin{itemize}
\item {Grp. gram.:m.}
\end{itemize}
\begin{itemize}
\item {Proveniência:(De \textunderscore pombe\textunderscore ^1)}
\end{itemize}
Negociante ou emissário, que atravessa os sertões, commerciando com indígenas.
\section{Pombinha}
\begin{itemize}
\item {Grp. gram.:f.}
\end{itemize}
\begin{itemize}
\item {Utilização:Bras}
\end{itemize}
Carne das nádegas das reses.
Parte superior do rabo do toiro, junto do ânus. Cf. Baganha, \textunderscore Hyg. Pec.\textunderscore , 201.
Partes pudendas da mulhér.
\section{Pombinha}
\begin{itemize}
\item {Grp. gram.:f.}
\end{itemize}
O mesmo que \textunderscore pombo-da-rocha\textunderscore .
\section{Pombinho}
\begin{itemize}
\item {Grp. gram.:m.}
\end{itemize}
\begin{itemize}
\item {Utilização:Mad}
\end{itemize}
\begin{itemize}
\item {Grp. gram.:Adj.}
\end{itemize}
\begin{itemize}
\item {Utilização:Chul.}
\end{itemize}
Pequeno pombo.
A côr do pombo.
O mesmo que \textunderscore pombo-da-rocha\textunderscore .
Um tanto ébrio.
Diz-se de uma variedade de trigo.
\section{Pombo}
\begin{itemize}
\item {Grp. gram.:m.}
\end{itemize}
\begin{itemize}
\item {Utilização:T. do Fundão}
\end{itemize}
\begin{itemize}
\item {Proveniência:(Do lat. \textunderscore palumbus\textunderscore )}
\end{itemize}
Gênero de aves columbinas.
Mentira, pêta.
\section{Pombo-branco}
\begin{itemize}
\item {Grp. gram.:m.}
\end{itemize}
\begin{itemize}
\item {Utilização:Mad}
\end{itemize}
O mesmo que \textunderscore pombo-torcaz\textunderscore .
\section{Pombo-claro}
\begin{itemize}
\item {Grp. gram.:m.}
\end{itemize}
\begin{itemize}
\item {Utilização:Mad}
\end{itemize}
O mesmo que \textunderscore pombo-branco\textunderscore .
\section{Pombo-correio}
\begin{itemize}
\item {Grp. gram.:m.}
\end{itemize}
Variedade de pombo, (\textunderscore columbus livia tabellaria\textunderscore ).
\section{Pombo-da-rocha}
\begin{itemize}
\item {Grp. gram.:m.}
\end{itemize}
\begin{itemize}
\item {Utilização:Mad}
\end{itemize}
Espécie de pombo bravo, (\textunderscore columba livia\textunderscore , Lin.).
\section{Pombo-de-leque}
\begin{itemize}
\item {Grp. gram.:m.}
\end{itemize}
Variedade de pombo, (\textunderscore columbus livia laticauda\textunderscore ).
\section{Pombo-escuro-da-serra}
\begin{itemize}
\item {Grp. gram.:m.}
\end{itemize}
\begin{itemize}
\item {Utilização:Mad}
\end{itemize}
O mesmo que \textunderscore pombo-preto\textunderscore .
\section{Pombo-gravatinha}
\begin{itemize}
\item {Grp. gram.:m.}
\end{itemize}
Variedade de pombo, que tem encrespadas as pennas do pescoço.
\section{Pombo-inglês}
\begin{itemize}
\item {Grp. gram.:m.}
\end{itemize}
O mesmo que \textunderscore pombo-preto\textunderscore .
\section{Pombo-preto}
\begin{itemize}
\item {Grp. gram.:m.}
\end{itemize}
\begin{itemize}
\item {Utilização:Mad}
\end{itemize}
O mesmo que \textunderscore pombo-torcaz\textunderscore .
\section{Pombo-torcaz}
\begin{itemize}
\item {Grp. gram.:m.}
\end{itemize}
Espécie de pombo, cujo pescoço tem várias côres, (\textunderscore columba palumbus\textunderscore , Lin.)
\section{Pombo-trocal}
\begin{itemize}
\item {Grp. gram.:m.}
\end{itemize}
\begin{itemize}
\item {Utilização:Mad}
\end{itemize}
O mesmo que \textunderscore pombo-torcaz\textunderscore .
\section{Pombo-volante}
\begin{itemize}
\item {Grp. gram.:m.}
\end{itemize}
O mesmo que \textunderscore pombo-correio\textunderscore .
\section{Pomeiro}
\begin{itemize}
\item {Grp. gram.:m.}
\end{itemize}
\begin{itemize}
\item {Utilização:Ant.}
\end{itemize}
\begin{itemize}
\item {Proveniência:(De \textunderscore pomo\textunderscore )}
\end{itemize}
O mesmo que \textunderscore pomar\textunderscore .
\section{Pomério}
\begin{itemize}
\item {Grp. gram.:m.}
\end{itemize}
\begin{itemize}
\item {Utilização:Des.}
\end{itemize}
\begin{itemize}
\item {Proveniência:(De \textunderscore poma\textunderscore )}
\end{itemize}
Saliência, na base da parede, para refôrço da construcção.
\section{Pomes}
\begin{itemize}
\item {Grp. gram.:adj.}
\end{itemize}
\begin{itemize}
\item {Proveniência:(Do lat. \textunderscore pumex\textunderscore )}
\end{itemize}
Diz-se de uma variedade de pedra muito porosa, que serve para polir ou limpar.
\section{Pomfólige}
\begin{itemize}
\item {Grp. gram.:f.}
\end{itemize}
\begin{itemize}
\item {Utilização:Med.}
\end{itemize}
\begin{itemize}
\item {Utilização:Chím.}
\end{itemize}
\begin{itemize}
\item {Proveniência:(Do gr. \textunderscore pompholux\textunderscore )}
\end{itemize}
O mesmo que \textunderscore pênfigo\textunderscore .
Óxido de zinco.
\section{Pómice}
\begin{itemize}
\item {Grp. gram.:f.}
\end{itemize}
\begin{itemize}
\item {Utilização:Poét.}
\end{itemize}
\begin{itemize}
\item {Proveniência:(Lat. \textunderscore pomex\textunderscore )}
\end{itemize}
Qualquer pedra porosa. Cf. Castilho, \textunderscore Geórgicas\textunderscore , 299; \textunderscore Fastos\textunderscore , I, 111.
\section{Pomicultura}
\begin{itemize}
\item {Grp. gram.:f.}
\end{itemize}
\begin{itemize}
\item {Proveniência:(De \textunderscore pomo\textunderscore  + \textunderscore cultura\textunderscore )}
\end{itemize}
Cultura das árvores pomíferas.
\section{Pomífero}
\begin{itemize}
\item {Grp. gram.:adj.}
\end{itemize}
\begin{itemize}
\item {Proveniência:(Lat. \textunderscore pomiferus\textunderscore )}
\end{itemize}
Que tem ou produz pomos.
\section{Pomítico}
\begin{itemize}
\item {Grp. gram.:adj.}
\end{itemize}
Relativo á pedra pomes.
\section{Pomo}
\begin{itemize}
\item {Grp. gram.:m.}
\end{itemize}
\begin{itemize}
\item {Utilização:Poét.}
\end{itemize}
\begin{itemize}
\item {Proveniência:(Lat. \textunderscore pomum\textunderscore )}
\end{itemize}
Fruto carnudo, mais ou menos esphérico ou ovóide.
Seio de mulhér.
\section{Pomo-do-elephante}
\begin{itemize}
\item {Grp. gram.:m.}
\end{itemize}
Árvore rutácea da Índia, (\textunderscore feronia elephantum\textunderscore , Correia).
\section{Pô-móli}
\begin{itemize}
\item {Grp. gram.:m.}
\end{itemize}
Árvore medicinal da ilha de San-Thomé.
(Corr. de \textunderscore pau-molle\textunderscore , no dialecto santhomense)
\section{Pomologia}
\begin{itemize}
\item {Grp. gram.:f.}
\end{itemize}
Tratado á cêrca dos pomos ou das árvores pomíferas.
(Cp. \textunderscore pomólogo\textunderscore )
\section{Pomológico}
\begin{itemize}
\item {Grp. gram.:adj.}
\end{itemize}
Relativo á pomologia.
\section{Pomologista}
\begin{itemize}
\item {Grp. gram.:m.}
\end{itemize}
Aquelle que se occupa de pomologia.
\section{Pomólogo}
\begin{itemize}
\item {Grp. gram.:m.}
\end{itemize}
\begin{itemize}
\item {Proveniência:(Do lat. \textunderscore pomum\textunderscore  + gr. \textunderscore logos\textunderscore )}
\end{itemize}
Aquelle que é versado em pomologia.
\section{Pomoso}
\begin{itemize}
\item {Grp. gram.:adj.}
\end{itemize}
Em que há pomos; que tem pomos. Cf. Pato Moniz, \textunderscore Apparição\textunderscore , 41.
\section{Pompa}
\begin{itemize}
\item {Grp. gram.:f.}
\end{itemize}
\begin{itemize}
\item {Proveniência:(Lat. \textunderscore pompa\textunderscore )}
\end{itemize}
Apparato sumptuoso e magnífico; ostentação.
Grande luxo; gala.
Bizarria.
\section{Pompeante}
\begin{itemize}
\item {Grp. gram.:adj.}
\end{itemize}
Que pompeia.
\section{Pompear}
\begin{itemize}
\item {Grp. gram.:v. t.}
\end{itemize}
\begin{itemize}
\item {Grp. gram.:V. i.}
\end{itemize}
Ostentar; expôr com vaidade: \textunderscore pompear riquezas\textunderscore .
Exibir pompa.
Ostentar riqueza; brilhar; garrir; pimpar.
\section{Pompeiano}
\begin{itemize}
\item {Grp. gram.:adj.}
\end{itemize}
Relativo a Pompeu.
Partidário de Pompeu.
\section{Pompeiano}
\begin{itemize}
\item {Grp. gram.:adj.}
\end{itemize}
\begin{itemize}
\item {Grp. gram.:M.}
\end{itemize}
Relativo a Pompeios.
Habitante de Pompeios.
(É incorrecta a fórma mais usada, \textunderscore Pompeia\textunderscore )
\section{Pomphólyge}
\begin{itemize}
\item {Grp. gram.:f.}
\end{itemize}
\begin{itemize}
\item {Utilização:Med.}
\end{itemize}
\begin{itemize}
\item {Utilização:Chím.}
\end{itemize}
\begin{itemize}
\item {Proveniência:(Do gr. \textunderscore pompholux\textunderscore )}
\end{itemize}
O mesmo que \textunderscore pêmphigo\textunderscore .
Óxydo de zinco.
\section{Pompílio}
\begin{itemize}
\item {Grp. gram.:m.}
\end{itemize}
\begin{itemize}
\item {Proveniência:(Do gr. \textunderscore pompilos\textunderscore )}
\end{itemize}
Gênero de insectos hymenópteros.
Gênero de peixes, que os Franceses chamam \textunderscore pulmage\textunderscore . Cf. Fil. Simões, \textunderscore Cartas da Beiramar\textunderscore , 237.
\section{Pompom}
\begin{itemize}
\item {Grp. gram.:m.}
\end{itemize}
\begin{itemize}
\item {Proveniência:(Fr. \textunderscore pompon\textunderscore )}
\end{itemize}
Borla de fios, curtos e tosquiados em fórma esphérica.
\section{Pomposamente}
\begin{itemize}
\item {Grp. gram.:adv.}
\end{itemize}
De modo pomposo.
\section{Pomposidade}
\begin{itemize}
\item {Grp. gram.:f.}
\end{itemize}
Qualidade de pomposo. Cf. Castilho, \textunderscore D. Quixote\textunderscore , II, 350.
\section{Pomposo}
\begin{itemize}
\item {Grp. gram.:adj.}
\end{itemize}
\begin{itemize}
\item {Proveniência:(Lat. \textunderscore pomposus\textunderscore )}
\end{itemize}
Em que há pompa; que revela pompa.
\section{Pómulo}
\begin{itemize}
\item {Grp. gram.:m.}
\end{itemize}
\begin{itemize}
\item {Proveniência:(Lat. \textunderscore pomulum\textunderscore )}
\end{itemize}
Maçan do rosto.
\section{Pona}
\begin{itemize}
\item {Grp. gram.:f.}
\end{itemize}
\begin{itemize}
\item {Utilização:Prov.}
\end{itemize}
Nariz curto e achatado.
\section{Ponaca}
\begin{itemize}
\item {Grp. gram.:f.}
\end{itemize}
Árvore fructífera da Índia portuguesa.
\section{Ponção}
\textunderscore m.\textunderscore  (e der.)
O mesmo que \textunderscore puncção\textunderscore , etc.
\section{Poncategés}
\begin{itemize}
\item {Grp. gram.:m. pl.}
\end{itemize}
Tríbo de Índios das matas do Tocantins, no Brasil.
\section{Ponchaci}
\begin{itemize}
\item {Grp. gram.:m.}
\end{itemize}
\begin{itemize}
\item {Utilização:Ant.}
\end{itemize}
Homem nobre na China. Cf. \textunderscore Peregrinação\textunderscore , CVI.
\section{Ponchada}
\begin{itemize}
\item {Grp. gram.:f.}
\end{itemize}
\begin{itemize}
\item {Utilização:Bras. do S}
\end{itemize}
Grande porção de coisas, que poderiam encher um poncho.
\section{Ponche}
\begin{itemize}
\item {Grp. gram.:m.}
\end{itemize}
\begin{itemize}
\item {Proveniência:(Ingl. \textunderscore punch\textunderscore )}
\end{itemize}
Mistura de chá e aguardente ou rum, com sumo do limão, açúcar, etc.
\section{Ponche}
\begin{itemize}
\item {Grp. gram.:m.}
\end{itemize}
Capa curta, com grande roda, usada há meio século entre nós, e correspondente á que os Franceses denominavam \textunderscore talma\textunderscore .
O mesmo que \textunderscore poncho\textunderscore .
\section{Poncheira}
\begin{itemize}
\item {Grp. gram.:f.}
\end{itemize}
Vaso, em que se faz ou se serve o ponche^1.
\section{Poncho}
\begin{itemize}
\item {Grp. gram.:m.}
\end{itemize}
\begin{itemize}
\item {Utilização:Bras}
\end{itemize}
\begin{itemize}
\item {Utilização:Prolóq.}
\end{itemize}
\begin{itemize}
\item {Proveniência:(T. cast.)}
\end{itemize}
Capa de lan, de fórma quadrada, com uma abertura no meio, por onde se enfia a cabeça.
Guarda-pó, para jornadas.
\textunderscore Ferrar o poncho\textunderscore , fazer bom negócio.
\textunderscore Sacudir o poncho\textunderscore , offender sem reacção.
Poncho do pobre é o sol.
\section{Ponciana-régia}
\begin{itemize}
\item {Grp. gram.:f.}
\end{itemize}
Árvore brasileira.
\section{Ponçó}
\begin{itemize}
\item {Grp. gram.:m.}
\end{itemize}
\begin{itemize}
\item {Utilização:Ant.}
\end{itemize}
\begin{itemize}
\item {Proveniência:(Do fr. \textunderscore ponceau\textunderscore )}
\end{itemize}
Côr de fogo, muito viva.
\section{Ponderabilidade}
\begin{itemize}
\item {Grp. gram.:f.}
\end{itemize}
Qualidade do ponderável. Cf. F. Lapa, \textunderscore Phýsica\textunderscore , 6 e 8.
\section{Ponderação}
\begin{itemize}
\item {Grp. gram.:f.}
\end{itemize}
\begin{itemize}
\item {Proveniência:(Lat. \textunderscore ponderatio\textunderscore )}
\end{itemize}
Acto de ponderar; importância.
Sisudez.
\section{Ponderadamente}
\begin{itemize}
\item {Grp. gram.:adv.}
\end{itemize}
\begin{itemize}
\item {Proveniência:(De \textunderscore ponderar\textunderscore )}
\end{itemize}
Com ponderação, com sisudez.
\section{Ponderado}
\begin{itemize}
\item {Grp. gram.:adj.}
\end{itemize}
\begin{itemize}
\item {Proveniência:(De \textunderscore ponderar\textunderscore )}
\end{itemize}
Que tem ponderação ou gravidade; sisudo; prudente: \textunderscore o Henrique é muito ponderado\textunderscore .
\section{Ponderador}
\begin{itemize}
\item {Grp. gram.:m.  e  adj.}
\end{itemize}
\begin{itemize}
\item {Proveniência:(Lat. \textunderscore ponderator\textunderscore )}
\end{itemize}
O que pondera.
\section{Ponderal}
\begin{itemize}
\item {Grp. gram.:adj.}
\end{itemize}
\begin{itemize}
\item {Proveniência:(Do lat. \textunderscore pondus\textunderscore , \textunderscore ponderis\textunderscore , pêso)}
\end{itemize}
Relativo a pêso.
\section{Ponderar}
\begin{itemize}
\item {Grp. gram.:v. t.}
\end{itemize}
\begin{itemize}
\item {Grp. gram.:V. i.}
\end{itemize}
\begin{itemize}
\item {Proveniência:(Lat. \textunderscore ponderare\textunderscore )}
\end{itemize}
Pesar, apreciar, avaliar.
Allegar, observar.
Ter em consideração.
Reflectir, pensar.
\section{Ponderativo}
\begin{itemize}
\item {Grp. gram.:adj.}
\end{itemize}
Que pondera.
\section{Ponderável}
\begin{itemize}
\item {Grp. gram.:adj.}
\end{itemize}
\begin{itemize}
\item {Proveniência:(Lat. \textunderscore ponderabilis\textunderscore )}
\end{itemize}
Que se póde pesar.
Que se póde ponderar.
Digno de ponderação.
\section{Ponderosamente}
\begin{itemize}
\item {Grp. gram.:adv.}
\end{itemize}
De modo ponderoso.
De modo notável; com importância.
\section{Ponderoso}
\begin{itemize}
\item {Grp. gram.:adj.}
\end{itemize}
\begin{itemize}
\item {Proveniência:(Lat. \textunderscore ponderosus\textunderscore )}
\end{itemize}
Que é pesado.
Importante; notável; digno de attenção.
Convincente.
\section{Pondo}
\begin{itemize}
\item {Grp. gram.:m.}
\end{itemize}
Meio arrátel de calaím em Moçambique.
Vara de embarcação, em Moçambique.
\section{Pondo}
\begin{itemize}
\item {Grp. gram.:m.}
\end{itemize}
\begin{itemize}
\item {Proveniência:(Lat. \textunderscore pondo\textunderscore )}
\end{itemize}
Unidade do pêso, o mesmo que \textunderscore libra\textunderscore , entre os antigos Romanos.
\section{Pondra}
\begin{itemize}
\item {Grp. gram.:f.}
\end{itemize}
(Corr. de \textunderscore alpondra\textunderscore )
\section{Póne}
\begin{itemize}
\item {Grp. gram.:m.}
\end{itemize}
Antiga moéda da Índia portuguesa.
\section{Pónei}
\begin{itemize}
\item {Grp. gram.:m.}
\end{itemize}
\begin{itemize}
\item {Proveniência:(Fr. \textunderscore poney\textunderscore )}
\end{itemize}
Espécie de cavalo, que cresce pouco, mas que é ágil e fino.
Mosquete, cavalo-mosca.
\section{Ponente}
\begin{itemize}
\item {Grp. gram.:adj.}
\end{itemize}
\begin{itemize}
\item {Grp. gram.:M.}
\end{itemize}
\begin{itemize}
\item {Proveniência:(Lat. \textunderscore ponens\textunderscore )}
\end{itemize}
Que se põe ou se esconde, (especialmente falando-se do Sol); poente.
Vento, que sopra do Occidente.
O mesmo que \textunderscore poente\textunderscore . Cf. Filinto, D. \textunderscore Man.\textunderscore , II, 75.
\section{Ponentino}
\begin{itemize}
\item {Grp. gram.:m.}
\end{itemize}
\begin{itemize}
\item {Utilização:Ant.}
\end{itemize}
\begin{itemize}
\item {Proveniência:(De \textunderscore ponente\textunderscore )}
\end{itemize}
Homem das regiões do Poente. Cf. Pant. de Aveiro, \textunderscore Itiner.\textunderscore , 15, (2.^a ed.).
\section{Ponera}
\begin{itemize}
\item {Grp. gram.:f.}
\end{itemize}
Gênero de orchídeas.
\section{Póney}
\begin{itemize}
\item {Grp. gram.:m.}
\end{itemize}
\begin{itemize}
\item {Proveniência:(Fr. \textunderscore poney\textunderscore )}
\end{itemize}
Espécie de cavallo, que cresce pouco, mas que é ágil e fino.
Mosquete, cavallo-mosca.
\section{Ponga}
\begin{itemize}
\item {Grp. gram.:f.}
\end{itemize}
\begin{itemize}
\item {Utilização:Bras}
\end{itemize}
Espécie de jôgo, que consiste num quadrilátero de madeira, cartão ou papel, em que se traçam duas diagonaes e duas perpendiculares que se cruzam, e em que se jogam dados.
\section{Ponga}
\begin{itemize}
\item {Grp. gram.:f.}
\end{itemize}
Acto de pongar.
\section{Pongar}
\begin{itemize}
\item {Grp. gram.:v. i.}
\end{itemize}
\begin{itemize}
\item {Utilização:Bras. do N}
\end{itemize}
Subir para o bonde, sem que êste pare.
\section{Pongo}
\begin{itemize}
\item {Grp. gram.:m.}
\end{itemize}
O mesmo que \textunderscore chimpanzé\textunderscore .
\section{Pongueró}
\begin{itemize}
\item {Grp. gram.:m.}
\end{itemize}
Planta leguminosa da Índia portuguesa, (\textunderscore erythrina indica\textunderscore , Lamk.).
\section{Ponilha}
\begin{itemize}
\item {Grp. gram.:f.}
\end{itemize}
\begin{itemize}
\item {Utilização:Prov.}
\end{itemize}
Pó, que se fórma sôbre os queijos, figos passados e outras frutas.
O mesmo que \textunderscore polilha\textunderscore .
\section{Ponis}
\begin{itemize}
\item {Grp. gram.:f.}
\end{itemize}
\begin{itemize}
\item {Utilização:Gír.}
\end{itemize}
Mulhér.
\section{Ponta}
\begin{itemize}
\item {Grp. gram.:f.}
\end{itemize}
\begin{itemize}
\item {Utilização:Fig.}
\end{itemize}
\begin{itemize}
\item {Utilização:Ant.}
\end{itemize}
\begin{itemize}
\item {Utilização:Bras}
\end{itemize}
\begin{itemize}
\item {Grp. gram.:Loc. adv.}
\end{itemize}
\begin{itemize}
\item {Grp. gram.:Loc.}
\end{itemize}
\begin{itemize}
\item {Utilização:pop.}
\end{itemize}
\begin{itemize}
\item {Proveniência:(Do lat. \textunderscore puncta\textunderscore )}
\end{itemize}
Extremidade aguçada: \textunderscore a ponta do punhal\textunderscore .
Extremidade estreita ou delgada.
Extremidade de um corpo oblongo: \textunderscore a ponta de uma vara\textunderscore .
O princípio ou fim de uma série.
Esquina.
Pequena porção.
Chifre.
Extremidade: \textunderscore a ponta da unha\textunderscore .
Resto de um cigarro ou charuto, depois de fumado.
Golpe com a ponta da espada ou da lança:«\textunderscore ...na certeza das pontas e repontas.\textunderscore »Sous. Monteiro, \textunderscore Elog. de Lat.\textunderscore 
O mesmo que \textunderscore pontaria\textunderscore ^1. Cf. Usque, 46 v.^o.
\textunderscore Andar na ponta\textunderscore , têr grande nomeada; andar na berra.
Sêr muito falado.
\textunderscore Trazer de ponta\textunderscore , têr má vontade a.
\textunderscore Andar de ponta\textunderscore , andar desavindo, indisposto.
\textunderscore De ponta a ponta\textunderscore , de cabo a rabo, de uma extremidade á outra.
\textunderscore Na ponta da unha\textunderscore , com rapidez, velozmente.
\textunderscore Navalha de ponta e mola\textunderscore , espécie de navalhas, cuja lâmina, de ponta aguda, é segurada, quando aberta, por uma mola.
\section{Pontaço}
\begin{itemize}
\item {Grp. gram.:m.}
\end{itemize}
\begin{itemize}
\item {Utilização:Bras. do S}
\end{itemize}
Pancada ou golpe com a ponta de qualquer arma ou instrumento.
\section{Pontada}
\begin{itemize}
\item {Grp. gram.:f.}
\end{itemize}
\begin{itemize}
\item {Utilização:Ant.}
\end{itemize}
\begin{itemize}
\item {Proveniência:(De \textunderscore ponto\textunderscore )}
\end{itemize}
Acto de coser ou bordar. Cp. G. Vicente, \textunderscore Inês Pereira\textunderscore .
\section{Pontada}
\begin{itemize}
\item {Grp. gram.:f.}
\end{itemize}
Dôr aguda e rápida.
Ponta.
Pancada com ponta, pontuada.
\section{Ponta-de-água}
\begin{itemize}
\item {Grp. gram.:f.}
\end{itemize}
\begin{itemize}
\item {Utilização:Bras. da Baía}
\end{itemize}
Grande correnteza na volta dos rios.
\section{Pontal}
\begin{itemize}
\item {Grp. gram.:m.}
\end{itemize}
\begin{itemize}
\item {Utilização:Prov.}
\end{itemize}
\begin{itemize}
\item {Utilização:alent.}
\end{itemize}
\begin{itemize}
\item {Grp. gram.:Adj.}
\end{itemize}
\begin{itemize}
\item {Proveniência:(De \textunderscore ponta\textunderscore )}
\end{itemize}
Altura da embarcação, entre a quilha e a primeira coberta.
Ponta de terra ou de penedia, que entra um pouco no mar, acima do nível da agua.
Cada um dos pontaletes do lenho, que se serra longitudinalmente.
Prolongamento de um oiteiro.
Diz-se de uma espécie de pregos grandes.
\section{Pontaletar}
\begin{itemize}
\item {Grp. gram.:v. t.}
\end{itemize}
Segurar com pontaletes.
\section{Pontalete}
\begin{itemize}
\item {fónica:lê}
\end{itemize}
\begin{itemize}
\item {Grp. gram.:m.}
\end{itemize}
\begin{itemize}
\item {Utilização:Açor}
\end{itemize}
\begin{itemize}
\item {Proveniência:(De \textunderscore pontal\textunderscore )}
\end{itemize}
Espécie de escora de madeira; espeque.
Forquilha, em que descansa o braço dos andores, nas procissões.
Pancada com a ponta de um objecto.
\section{Pontão}
\begin{itemize}
\item {Grp. gram.:m.}
\end{itemize}
\begin{itemize}
\item {Proveniência:(De \textunderscore ponta\textunderscore )}
\end{itemize}
Espeque, escora.
\section{Pontão}
\begin{itemize}
\item {Grp. gram.:m.}
\end{itemize}
\begin{itemize}
\item {Utilização:T. da Bairrada}
\end{itemize}
\begin{itemize}
\item {Utilização:Prov.}
\end{itemize}
\begin{itemize}
\item {Utilização:trasm.}
\end{itemize}
\begin{itemize}
\item {Proveniência:(De \textunderscore ponte\textunderscore )}
\end{itemize}
Barca chata que, por si ou com outras, fórma passagem ou ponte.
Pequena ponte; pequeno viaducto, em estradas.
Excepção odiosa; aquillo que indevidamente se passa em claro.
\section{Pontapé}
\begin{itemize}
\item {Grp. gram.:m.}
\end{itemize}
\begin{itemize}
\item {Utilização:Fig.}
\end{itemize}
\begin{itemize}
\item {Proveniência:(De \textunderscore ponta\textunderscore  + \textunderscore pé\textunderscore )}
\end{itemize}
Pancada com a ponta do pé.
Offensa.
Desastre; sinistro.
\section{Pontar}
\begin{itemize}
\item {Grp. gram.:v. t.}
\end{itemize}
Guarnecer de pontes (uma embarcação).
\section{Pontar}
\textunderscore v. t.\textunderscore  (e der.)
O mesmo que \textunderscore apontar\textunderscore ^1, etc.:«\textunderscore ...alfanges pontão ao peito...\textunderscore »Filinto, VII, 115.
\section{Pontar}
\begin{itemize}
\item {Grp. gram.:v. i.}
\end{itemize}
Servir de ponto em (peças theatraes); apontar.
\section{Pontarelo}
\begin{itemize}
\item {Grp. gram.:m.}
\end{itemize}
Ponto grande e mal feito, na costura.
\section{Pontaria}
\begin{itemize}
\item {Grp. gram.:f.}
\end{itemize}
\begin{itemize}
\item {Utilização:Ext.}
\end{itemize}
\begin{itemize}
\item {Utilização:Ant.}
\end{itemize}
\begin{itemize}
\item {Proveniência:(De \textunderscore ponto\textunderscore )}
\end{itemize}
Acto de apontar.
Acto de assestar (uma arma de fogo), na direcção da linha de mira.
Alvo.
Ardil ou trapaça, usada em luta para fazer caír o adversário.
\section{Pontaria}
\begin{itemize}
\item {Grp. gram.:f.}
\end{itemize}
\begin{itemize}
\item {Utilização:Prov.}
\end{itemize}
\begin{itemize}
\item {Utilização:beir.}
\end{itemize}
\begin{itemize}
\item {Proveniência:(De \textunderscore ponte\textunderscore )}
\end{itemize}
Extensa linha de cales suspensas, por onde corre a água, de um poço ou de uma nora para um terreno que ella vai regar. (Colhido no Fundão)
\section{Pontas}
\begin{itemize}
\item {Grp. gram.:f. pl.}
\end{itemize}
\begin{itemize}
\item {Utilização:Bras. do S}
\end{itemize}
Extremidades superiores de um rio.
(Pl. de \textunderscore ponta\textunderscore )
\section{Ponta-sêca}
\begin{itemize}
\item {Grp. gram.:f.}
\end{itemize}
Utensílio, em fórma de agulha, para desenhar sôbre verniz.
\section{Pontavante}
\begin{itemize}
\item {Grp. gram.:f.}
\end{itemize}
\begin{itemize}
\item {Utilização:Náut.}
\end{itemize}
\begin{itemize}
\item {Proveniência:(De \textunderscore ponte\textunderscore  + \textunderscore avante\textunderscore )}
\end{itemize}
Ponto ou anteparo na prôa do navio:«\textunderscore ...ao impeto com que entra a água numa nau sem pontavante.\textunderscore »Camillo, \textunderscore Narcóticos\textunderscore , II, 116.
\section{Ponte}
\begin{itemize}
\item {Grp. gram.:f.}
\end{itemize}
\begin{itemize}
\item {Utilização:Prov.}
\end{itemize}
\begin{itemize}
\item {Utilização:Prov.}
\end{itemize}
\begin{itemize}
\item {Utilização:alg.}
\end{itemize}
\begin{itemize}
\item {Utilização:Prov.}
\end{itemize}
\begin{itemize}
\item {Utilização:trasm.}
\end{itemize}
\begin{itemize}
\item {Proveniência:(Lat. \textunderscore pons\textunderscore , \textunderscore pontis\textunderscore )}
\end{itemize}
Construcção, que liga dois pontos ou lugares, separados por um rio ou ribeira ou por um valle.
Sobrado do navio.
Coberta do navio.
Prancha de madeira, sôbre que trabalha o rodízio, no cavouco dos moínhos.
Travéssa de pau, que sujeita os tendaes, na traseira do carro.
Espécie de jôgo popular.
\textunderscore Fazer ponte em alguem\textunderscore , passar-lhe á porta, sem lhe falar.
\section{Ponteado}
\begin{itemize}
\item {Grp. gram.:m.}
\end{itemize}
\begin{itemize}
\item {Utilização:Bras. de Minas}
\end{itemize}
\begin{itemize}
\item {Proveniência:(De \textunderscore pontear\textunderscore )}
\end{itemize}
Desenho, notado por pontinhos.
Toque de viola.
\section{Ponteagudo}
\begin{itemize}
\item {Grp. gram.:adj.}
\end{itemize}
\begin{itemize}
\item {Proveniência:(De \textunderscore ponta\textunderscore  + \textunderscore agudo\textunderscore )}
\end{itemize}
Que termina em ponta aguda.
\section{Pontear}
\begin{itemize}
\item {Grp. gram.:v. t.}
\end{itemize}
\begin{itemize}
\item {Utilização:Mús.}
\end{itemize}
Marcar com pontos em.
Coser; alinhavar.
Collocar os dedos sôbre (cordas), no lugar onde estão os pontos ou tastos, afim de produzir as diversas entoações.
\section{Pontederiáceas}
\begin{itemize}
\item {Grp. gram.:f. pl.}
\end{itemize}
Família de plantas, vizinhas dos narcisos, e cujo principal gênero foi dedicado ao botânico Pontedera.
\section{Ponteira}
\begin{itemize}
\item {Grp. gram.:f.}
\end{itemize}
\begin{itemize}
\item {Utilização:Prov.}
\end{itemize}
\begin{itemize}
\item {Utilização:alent.}
\end{itemize}
\begin{itemize}
\item {Proveniência:(De \textunderscore ponta\textunderscore )}
\end{itemize}
Peça de metal, que reveste a extremidade inferior de bengalas, guarda-sóes, baínhas de espadas, etc.
Extremidade postiça de uma boquilha, em que se introduz a ponta do cigarro ou do charuto que se quer fumar; boquilha.
A parte mais delgada da cana de pescar, quando esta consta de duas partes.
\section{Ponteiro}
\begin{itemize}
\item {Grp. gram.:m.}
\end{itemize}
\begin{itemize}
\item {Grp. gram.:Adj.}
\end{itemize}
\begin{itemize}
\item {Proveniência:(Do lat. \textunderscore pontarius\textunderscore )}
\end{itemize}
Pequena haste, para apontar nos livros, quadros, etc.
Instrumento de canteiro, para desbastar a pedra.
Pequena peça ou lâmina com que se ferem as cordas de alguns instrumentos.
Espécie de agulha, que nos mostradores dos relógios indica as horas e as fracções destas.
Diz-se do vento que é contrário á navegação ou sopra pela prôa.
Que se desmanda, não obedecendo ao caçador, (falando-se do cão).
Diz-se da espingarda que, depois de apontada, se equilibra mal, pendendo um pouco na ponta.
\section{Ponteiro}
\begin{itemize}
\item {Grp. gram.:adj.}
\end{itemize}
\begin{itemize}
\item {Utilização:Ant.}
\end{itemize}
\begin{itemize}
\item {Proveniência:(De \textunderscore ponto\textunderscore )}
\end{itemize}
Certeiro; que tem bôa pontaria.
\section{Pontel}
\begin{itemize}
\item {Grp. gram.:m.}
\end{itemize}
\begin{itemize}
\item {Proveniência:(De \textunderscore ponta\textunderscore )}
\end{itemize}
Espécie de ponteiro, com que se segura o vidro, quando se caldeia.
\section{Ponticidade}
\begin{itemize}
\item {Grp. gram.:f.}
\end{itemize}
Azedume, qualidade do que é pontico. Cf. Garc. da Orta, \textunderscore Coloq.\textunderscore , 31.
\section{Pôntico}
\begin{itemize}
\item {Grp. gram.:adj.}
\end{itemize}
\begin{itemize}
\item {Proveniência:(Lat. \textunderscore ponticus\textunderscore )}
\end{itemize}
Relativo ao Ponto, região da Ásia-Menor, ao sul do Mar-Negro.
\section{Pontico}
\begin{itemize}
\item {Grp. gram.:adj.}
\end{itemize}
\begin{itemize}
\item {Utilização:Ant.}
\end{itemize}
O mesmo que \textunderscore azêdo\textunderscore . Cf. Orta, \textunderscore Collóq.\textunderscore , 31.
\section{Ponticular-se}
\begin{itemize}
\item {Grp. gram.:v. p.}
\end{itemize}
\begin{itemize}
\item {Utilização:Neol.}
\end{itemize}
Mostrar-se como pontículo:«\textunderscore os seus dentes ponticulavam-se de leve.\textunderscore »Eça, \textunderscore P. Amaro\textunderscore .
\section{Pontículo}
\begin{itemize}
\item {Grp. gram.:m.}
\end{itemize}
\begin{itemize}
\item {Utilização:Neol.}
\end{itemize}
Ponto pequeno.
\section{Pontificado}
\begin{itemize}
\item {Grp. gram.:m.}
\end{itemize}
\begin{itemize}
\item {Proveniência:(Lat. \textunderscore pontificatus\textunderscore )}
\end{itemize}
Dignidade de Pontífice.
Todo o tempo, em que um Pontífice exerceu a sua dignidade; papado.
\section{Pontifical}
\begin{itemize}
\item {Grp. gram.:adj.}
\end{itemize}
\begin{itemize}
\item {Grp. gram.:M.}
\end{itemize}
\begin{itemize}
\item {Proveniência:(Lat. \textunderscore pontificalis\textunderscore )}
\end{itemize}
Relativo aos Pontífices.
Relativo á dignidade de Bispo.
Livro, que contém os ritos, destinados a serem observados por Pontífices ou Bispos.
Capa comprida, que se usa na celebração de certas ceremonias religiosas.
\section{Pontificalmente}
\begin{itemize}
\item {Grp. gram.:adv.}
\end{itemize}
De modo pontifical.
\section{Pontificante}
\begin{itemize}
\item {Grp. gram.:m.  e  adj.}
\end{itemize}
O sacerdote que pontifica.
\section{Pontificar}
\begin{itemize}
\item {Grp. gram.:v. i.}
\end{itemize}
Celebrar Missa, com a capa de pontifical.
(Cp. \textunderscore pontifical\textunderscore )
\section{Pontífice}
\begin{itemize}
\item {Grp. gram.:m.}
\end{itemize}
\begin{itemize}
\item {Utilização:Fig.}
\end{itemize}
\begin{itemize}
\item {Proveniência:(Lat. \textunderscore pontifex\textunderscore )}
\end{itemize}
Dignatário ecclesiástico, ministro do culto de uma religião.
Bispo, Prelado.
Papa.
Chefe de um systema ou de uma escola.
O indivíduo mais notável de certas classes:«\textunderscore ...os pontífices literários...\textunderscore »J. de Deus, no \textunderscore Bejense\textunderscore .
\section{Pontífice}
\begin{itemize}
\item {Grp. gram.:m.}
\end{itemize}
\begin{itemize}
\item {Utilização:Gír.}
\end{itemize}
\begin{itemize}
\item {Proveniência:(De \textunderscore ponta\textunderscore )}
\end{itemize}
Ponta de cigarro.
\section{Pontifício}
\begin{itemize}
\item {Grp. gram.:adj.}
\end{itemize}
\begin{itemize}
\item {Proveniência:(Lat. \textunderscore pontificius\textunderscore )}
\end{itemize}
Relativo a Pontífice; próprio de Pontífice ou procedente delle: \textunderscore diploma pontifício\textunderscore .
\section{Pontilha}
\begin{itemize}
\item {Grp. gram.:f.}
\end{itemize}
Ponta muito aguda.
Franja de prata ou oiro, estreita e delgada; espiguilha.
\section{Pontilhão}
\begin{itemize}
\item {Grp. gram.:m.}
\end{itemize}
Pequena ponte.
\section{Pontilhar}
\begin{itemize}
\item {Grp. gram.:v. t.}
\end{itemize}
\begin{itemize}
\item {Proveniência:(De \textunderscore ponto\textunderscore )}
\end{itemize}
Pontoar; marcar com pontinhos.
Desenhar, marcando com pontos.
\section{Pontilheiro}
\begin{itemize}
\item {Grp. gram.:m.}
\end{itemize}
Aquelle que pica toiros com pontilha ou agulhão.
\section{Pontilhoso}
\begin{itemize}
\item {Grp. gram.:adj.}
\end{itemize}
\begin{itemize}
\item {Utilização:Gal}
\end{itemize}
\begin{itemize}
\item {Proveniência:(Do fr. \textunderscore pontilleux\textunderscore )}
\end{itemize}
Habituado a pôr os pontos nos \textunderscore ii\textunderscore .
Pontoso.
Que não gosta de reservas ou de procedimentos dúbios:«\textunderscore D. Theresa, muito pontilhosa em não admittir equívocos...\textunderscore »Camillo, \textunderscore Brasileira\textunderscore , 320.
\section{Pontineu}
\begin{itemize}
\item {Grp. gram.:m.}
\end{itemize}
\begin{itemize}
\item {Utilização:Ant.}
\end{itemize}
Edifício, em que certos Reis bárbaros davam audiência. Cf. \textunderscore Peregrinação\textunderscore , CXXV.
\section{Pontinha}
\begin{itemize}
\item {Grp. gram.:f.}
\end{itemize}
Pequena ponta; pouca coisa, pequena porção.
Rixa, birra.
\section{Pontinhar}
\begin{itemize}
\item {Proveniência:(De \textunderscore pontinho\textunderscore )}
\end{itemize}
\textunderscore v. t.\textunderscore  (e der.)
O mesmo que \textunderscore pontilhar\textunderscore , etc.
\section{Pontinho}
\begin{itemize}
\item {Grp. gram.:m.}
\end{itemize}
\begin{itemize}
\item {Grp. gram.:Pl.}
\end{itemize}
Pequeno ponto, especialmente o que se emprega nas luvas, etc.
Pontos de reticência.
(Dem. de \textunderscore ponto\textunderscore )
\section{Pontino}
\begin{itemize}
\item {Grp. gram.:adj.}
\end{itemize}
\begin{itemize}
\item {Proveniência:(Lat. \textunderscore pontinus\textunderscore  ou \textunderscore pomptinus\textunderscore )}
\end{itemize}
Relativo a uma antiga região do Lácio, atravessada pela Via-Áppia, e cheia de pântanos:«\textunderscore ...as lagôas pontinas...\textunderscore »Herculano, \textunderscore Opúsc.\textunderscore , III, 26.
\section{Pontizela}
\begin{itemize}
\item {Grp. gram.:f.}
\end{itemize}
\begin{itemize}
\item {Utilização:Prov.}
\end{itemize}
\begin{itemize}
\item {Utilização:minh.}
\end{itemize}
\begin{itemize}
\item {Proveniência:(De \textunderscore ponte\textunderscore )}
\end{itemize}
Pequena ponte, pontilhão.
\section{Ponto}
\begin{itemize}
\item {Grp. gram.:m.}
\end{itemize}
\begin{itemize}
\item {Utilização:Prov.}
\end{itemize}
\begin{itemize}
\item {Utilização:dur.}
\end{itemize}
\begin{itemize}
\item {Utilização:Fam.}
\end{itemize}
\begin{itemize}
\item {Utilização:Constr.}
\end{itemize}
\begin{itemize}
\item {Utilização:Prov.}
\end{itemize}
\begin{itemize}
\item {Utilização:minh.}
\end{itemize}
\begin{itemize}
\item {Utilização:Mecan.}
\end{itemize}
\begin{itemize}
\item {Grp. gram.:Loc. adv.}
\end{itemize}
\begin{itemize}
\item {Grp. gram.:Loc. adv.}
\end{itemize}
\begin{itemize}
\item {Utilização:T. da Guiné}
\end{itemize}
\begin{itemize}
\item {Grp. gram.:Loc.}
\end{itemize}
\begin{itemize}
\item {Utilização:fam.}
\end{itemize}
\begin{itemize}
\item {Grp. gram.:Loc. adv.}
\end{itemize}
\begin{itemize}
\item {Grp. gram.:Pl.}
\end{itemize}
\begin{itemize}
\item {Utilização:Carp.}
\end{itemize}
\begin{itemize}
\item {Proveniência:(Do lat. \textunderscore punctus\textunderscore )}
\end{itemize}
Furo, que se faz com a agulha enfiada em linha, seda, etc., premendo-a no fundo e sacando-a pelo lado opposto.
Pequena porção de fio, que separa dois furos da agulha, quando se cose.
Obra de costura.
Apanhado, que se faz com a agulha de meia ou agulha de coser, para se tapar um buraco em meias ou tecidos.
Pequena mancha arredondada.
Sinal, semelhante ao que deixa uma picada de agulha.
Pequena porção de emplastro, para unir a pelle num ferimento ou para vedar o sangue.
Duodécima parte de uma linha, no antigo systema de medidas.
Limite de uma linha.
Lugar infinitamente pequeno, em que duas linhas se cortam.
Cada um dos espaços iguaes, em que se divide a craveira do luveiro, do camiseiro, e do sapateiro.
Sítio, lugar.
A grandeza, considerada em abstracto, sem dimensões.
Assumpto, objecto.
Estado de uma questão.
Sinal orthográphico, que se põe no fim de um período, bem como sôbre o \textunderscore i\textunderscore  e sôbre o \textunderscore j\textunderscore .
Cada uma das pintas nas faces dos dados e nas cartas de jogar.
Encerramento de aulas, nas escolas superiores.
Fim, termo.
Sinal musical, que aumenta metade do valor de uma nota.
Cada um dos filetes de metal, no braço de alguns instrumentos de corda.
Tempo marcado, ensejo.
Conjuntura, situação.
Recife, escolho, no rio Doiro.
Sinal, com que se marca a assistência de alguém a certos actos.
Livro, em se registam as faltas ou a assiduidade de certos funccionários.
Mira.
Acção.
Pundonor.
Grau.
Empregado que, durante as representações theatraes, vai lendo em voz baixa as peças aos actores, para lhes auxiliar a memória.
Sujeito, um indivíduo qualquer.
Cada um dos parceiros, no jôgo de asar.
Relação entre a altura e a largura do vão de um telhado.
Pontada, dôr penetrante.
Cóne de aço, que entra no cabeçalho fixo do tôrno, marcando o centro do movimento.
\textunderscore A ponto\textunderscore , a propósito, opportunamente, á hora aprazada; quási, prestes: \textunderscore esteve a ponto de se afogar\textunderscore .
\textunderscore Em ponto\textunderscore , exactamente; na medida ou proporção conveniente.
\textunderscore Pôr ponto\textunderscore , dar fim a, concluir.
\textunderscore Subir de ponto\textunderscore , aumentar, crescer.
\textunderscore Tomar o ponto\textunderscore , fazer chamada, para verificar se estão presentes os que devem estar.
\textunderscore Assinar o ponto\textunderscore , inscrever o nome no livro ou relação, donde deve constar que certos indivíduos foram presentes a certos actos ou em certas repartições.
Lugar, onde nasce água potável.
\textunderscore Ponto de admiração\textunderscore  ou \textunderscore de exclamação\textunderscore , sinal gráphico, para indicar que uma locução deve lêr-se em tom de quem se admira ou exclama.
\textunderscore Ponto de interrogação\textunderscore , sinal gráphico, para indicar que uma locução deve lêr-se em tom de quem pergunta.
\textunderscore Ponto de honra\textunderscore , assumpto ou questão, que envolve a honra ou dignidade de alguém.
\textunderscore Não dar ponto sem nó\textunderscore , sêr interesseiro, não se preoccupar senão do que dá interesse.
\textunderscore Ponto de rebuçado\textunderscore , estado em que a calda do açúcar, coagulando-se, póde formar rebuçado.
\textunderscore Ponto de cabello\textunderscore , grau de consistência, em que a mesma calda cai em fio.
\textunderscore Ponto de pérola\textunderscore , outro grau em que essa calda fórma bolhas.
\textunderscore Ponto de caixa\textunderscore  e \textunderscore ponto de espadana\textunderscore , outros dois graus de consistência da mesma calda.
\textunderscore Aí é que bate o ponto\textunderscore , essa é que é a questão; a difficuldade está nisso.
\textunderscore Pontos falsos\textunderscore , tiras de adhesivo, com que se ajustam os tecidos numa ferida.
\textunderscore Pontos naturaes\textunderscore , cosedura, com agulha, dos bordos de uma ferida, um ao outro.
Ferragem, em que se movem as portas dos móveis.
\textunderscore Pôr os pontos nos ii\textunderscore , dizer tudo claramente, minuciosamente, sem omittir particularidades ou nomes.
\section{Pontoada}
\begin{itemize}
\item {Grp. gram.:f.}
\end{itemize}
Pancada com a ponta de um objecto.
\section{Pontoar}
\begin{itemize}
\item {Grp. gram.:v. t.}
\end{itemize}
O mesmo que \textunderscore apontoar\textunderscore ^1; marcar com pontos, granir.
\section{Ponto-chão}
\begin{itemize}
\item {Grp. gram.:m.}
\end{itemize}
\begin{itemize}
\item {Utilização:Ant.}
\end{itemize}
Espécie de ponto de bordadeira.
\section{Pontoneiro}
\begin{itemize}
\item {Grp. gram.:m.}
\end{itemize}
\begin{itemize}
\item {Proveniência:(De \textunderscore pontão\textunderscore ^2)}
\end{itemize}
Soldado, que trabalha na construcção de pontes militares.
Constructor de pontões.
\section{Pontoso}
\begin{itemize}
\item {Grp. gram.:adj.}
\end{itemize}
\begin{itemize}
\item {Utilização:P. us.}
\end{itemize}
\begin{itemize}
\item {Proveniência:(De \textunderscore ponto\textunderscore )}
\end{itemize}
Escrupuloso em pontos de honra; que tem pundonor; honrado, brioso.
\section{Ponto-subido}
\begin{itemize}
\item {Grp. gram.:m.}
\end{itemize}
\begin{itemize}
\item {Utilização:Carp.}
\end{itemize}
Espécie de mola, em fórma de meia lua, que se colloca nos guarda-ventos, para se conservarem fechados.
\section{Pontuação}
\begin{itemize}
\item {Grp. gram.:f.}
\end{itemize}
Acto ou effeito de pontuar.
\section{Pontuada}
\begin{itemize}
\item {Grp. gram.:f.}
\end{itemize}
O mesmo que \textunderscore pontoada\textunderscore .
\section{Pontual}
\begin{itemize}
\item {Grp. gram.:adj.}
\end{itemize}
\begin{itemize}
\item {Grp. gram.:F.}
\end{itemize}
\begin{itemize}
\item {Utilização:Mathem.}
\end{itemize}
\begin{itemize}
\item {Proveniência:(De \textunderscore ponto\textunderscore )}
\end{itemize}
Exacto no cumprimento dos seus deveres: \textunderscore funccionário pontual\textunderscore .
Feito com exactidão.
Série de pontos, dispostos em linha recta (também se chama \textunderscore pontual regrada\textunderscore ).
\section{Pontualidade}
\begin{itemize}
\item {Grp. gram.:f.}
\end{itemize}
Qualidade do que é pontual.
\section{Pontualmente}
\begin{itemize}
\item {Grp. gram.:adv.}
\end{itemize}
De modo pontual.
\section{Pontuar}
\begin{itemize}
\item {Grp. gram.:v. t.}
\end{itemize}
\begin{itemize}
\item {Proveniência:(De \textunderscore ponto\textunderscore )}
\end{itemize}
Pôr sinaes ortográphicos em.
\section{Pontudo}
\begin{itemize}
\item {Grp. gram.:adj.}
\end{itemize}
\begin{itemize}
\item {Utilização:Fig.}
\end{itemize}
\begin{itemize}
\item {Utilização:Heráld.}
\end{itemize}
Que tem ponta; agudo.
Escabroso.
Aggressivo.
Diz-se da cruz, cuja haste inferior termina em fórma aguda.
\section{Pópa}
\begin{itemize}
\item {Grp. gram.:m.}
\end{itemize}
\begin{itemize}
\item {Proveniência:(Lat. \textunderscore popa\textunderscore )}
\end{itemize}
Sacerdote, de categoria inferior, entre os Romanos, o qual, no templo, cuidava do fogo, dos vasos, do incenso, etc., e tinha também a seu cargo conduzir a víctima até o altar e feri-la.
\section{Pôpa}
\begin{itemize}
\item {Grp. gram.:f.}
\end{itemize}
\begin{itemize}
\item {Utilização:Prolóq.}
\end{itemize}
\begin{itemize}
\item {Proveniência:(Do lat. \textunderscore puppis\textunderscore )}
\end{itemize}
Parte posterior do navio.
\textunderscore Vir á pôpa\textunderscore , vir a propósito, a talho de foice. Cf. \textunderscore Filodemo\textunderscore , V, 2.
\section{Pôpa}
\begin{itemize}
\item {Grp. gram.:f.}
\end{itemize}
(V. \textunderscore poupa\textunderscore ^1)
\section{Popelina}
\begin{itemize}
\item {Grp. gram.:f.}
\end{itemize}
Tecido, para vestuário feminino.
\section{Popiá}
\begin{itemize}
\item {Grp. gram.:f.}
\end{itemize}
(V.propiá)
\section{Popiliame}
\begin{itemize}
\item {Grp. gram.:m.}
\end{itemize}
\begin{itemize}
\item {Utilização:Ant.}
\end{itemize}
(?):«\textunderscore ...e as amarras, popiliames, enxárceas, velas, e outras cousas necessárias nunca nenhumas estam tanto a propósito...\textunderscore »Couto, \textunderscore Soldado Prático\textunderscore , 84.
Talvez, por corruptela, o conjunto das peças náuticas que se chamam \textunderscore papoias\textunderscore  ou \textunderscore papoilas\textunderscore .
\section{Popina}
\begin{itemize}
\item {Grp. gram.:f.}
\end{itemize}
\begin{itemize}
\item {Proveniência:(Lat. \textunderscore popina\textunderscore )}
\end{itemize}
Nome, que os Romanos davam á casa ou lugar, onde se vendem comidas cozinhadas.
Espécie de restaurante. Cf. Castilho, \textunderscore Fastos\textunderscore , II, 564.
\section{Popinha}
\begin{itemize}
\item {Grp. gram.:f.}
\end{itemize}
\begin{itemize}
\item {Proveniência:(De \textunderscore pôpa\textunderscore ^2, por \textunderscore poupa\textunderscore ^1)}
\end{itemize}
Ave, o mesmo que \textunderscore cotovia\textunderscore . Cf. P. Moraes, \textunderscore Zool. Elem.\textunderscore , 318.
\section{Popismos}
\begin{itemize}
\item {Grp. gram.:m. pl.}
\end{itemize}
\begin{itemize}
\item {Utilização:Philol.}
\end{itemize}
\begin{itemize}
\item {Proveniência:(Do gr. \textunderscore poppuzo\textunderscore , dar estalos com a língua)}
\end{itemize}
Sons, ou elementos de linguagem, peculiares aos Hotentotes e a outros povos africanos.
Em inglês, \textunderscore clicks\textunderscore , e em port. ant. \textunderscore soluços\textunderscore .
\section{Poplíteo}
\begin{itemize}
\item {Grp. gram.:adj.}
\end{itemize}
\begin{itemize}
\item {Utilização:Anat.}
\end{itemize}
\begin{itemize}
\item {Proveniência:(Do lat. \textunderscore poples\textunderscore , \textunderscore poplitis\textunderscore )}
\end{itemize}
Relativo á curva da perna: \textunderscore músculo poplíteo\textunderscore .
\section{Popó}
\begin{itemize}
\item {Grp. gram.:m.}
\end{itemize}
Madeira da Ilha do Príncipe, talvez da árvore que em San-Thomé se chama nespéra.
\section{Popocar}
\begin{itemize}
\item {Grp. gram.:v. t.  e  i.}
\end{itemize}
\begin{itemize}
\item {Utilização:Bras. do N}
\end{itemize}
O mesmo que \textunderscore pipocar\textunderscore .
\section{Popórria}
\begin{itemize}
\item {Grp. gram.:f.}
\end{itemize}
Gênero de plantas anonáceas.
\section{Poppysmos}
\begin{itemize}
\item {Grp. gram.:m. pl.}
\end{itemize}
\begin{itemize}
\item {Utilização:Philol.}
\end{itemize}
\begin{itemize}
\item {Proveniência:(Do gr. \textunderscore poppuzo\textunderscore , dar estalos com a língua)}
\end{itemize}
Sons, ou elementos de linguagem, peculiares aos Hotentotes e a outros povos africanos.
Em inglês, \textunderscore clicks\textunderscore , e em port. ant. \textunderscore soluços\textunderscore .
\section{Populaça}
\begin{itemize}
\item {Grp. gram.:f.}
\end{itemize}
(V.populacho)
\section{População}
\begin{itemize}
\item {Grp. gram.:f.}
\end{itemize}
\begin{itemize}
\item {Utilização:Fig.}
\end{itemize}
\begin{itemize}
\item {Proveniência:(Lat. \textunderscore populatio\textunderscore )}
\end{itemize}
Habitantes; conjunto dos habitantes de um país, localidade, etc.
Classe: \textunderscore a população escolar\textunderscore .
Grande porção de animaes.
\section{Populacho}
\begin{itemize}
\item {Grp. gram.:m.}
\end{itemize}
\begin{itemize}
\item {Proveniência:(Do lat. \textunderscore populaceus\textunderscore )}
\end{itemize}
O mesmo que \textunderscore plebe\textunderscore ; as classes inferiores da sociedade.
\section{Populacional}
\begin{itemize}
\item {Grp. gram.:adj.}
\end{itemize}
\begin{itemize}
\item {Utilização:Neol.}
\end{itemize}
Relativo a população:«\textunderscore fomento populacional\textunderscore ». R. Jorge, \textunderscore Sezonismo\textunderscore , 3.
\section{Populado}
\begin{itemize}
\item {Grp. gram.:m.}
\end{itemize}
\begin{itemize}
\item {Utilização:Ant.}
\end{itemize}
\begin{itemize}
\item {Proveniência:(Do lat. \textunderscore populatus\textunderscore )}
\end{itemize}
O mesmo que \textunderscore povoado\textunderscore .
\section{Popular}
\begin{itemize}
\item {Grp. gram.:adj.}
\end{itemize}
\begin{itemize}
\item {Grp. gram.:M.}
\end{itemize}
\begin{itemize}
\item {Grp. gram.:Pl.}
\end{itemize}
\begin{itemize}
\item {Proveniência:(Lat. \textunderscore popularis\textunderscore )}
\end{itemize}
Relativo ao povo.
Próprio do povo: \textunderscore ingenuidade popular\textunderscore .
Agradável ao povo.
Estimado pelo povo; democrático: \textunderscore um orador popular\textunderscore .
Homem do povo.
Democratas; homens do povo.
\section{Popularidade}
\begin{itemize}
\item {Grp. gram.:f.}
\end{itemize}
\begin{itemize}
\item {Proveniência:(Lat. \textunderscore popularitas\textunderscore )}
\end{itemize}
Qualidade do que é popular.
Estima geral.
\section{Popularização}
\begin{itemize}
\item {Grp. gram.:f.}
\end{itemize}
Acto ou effeito de popularizar.
\section{Popularizar}
\begin{itemize}
\item {Grp. gram.:v. t.}
\end{itemize}
Tornar popular; diffundir entre o povo; divulgar.
\section{Popularmente}
\begin{itemize}
\item {Grp. gram.:adv.}
\end{itemize}
De modo popular.
\section{Populeão}
\begin{itemize}
\item {Grp. gram.:m.  e  adj.}
\end{itemize}
\begin{itemize}
\item {Proveniência:(Lat. \textunderscore populeus\textunderscore )}
\end{itemize}
Diz-se de certo unguento, em que entra belladona, folhas de papoila, etc.
\section{Popúleo}
\begin{itemize}
\item {Grp. gram.:adj.}
\end{itemize}
\begin{itemize}
\item {Utilização:Poét.}
\end{itemize}
\begin{itemize}
\item {Proveniência:(Lat. \textunderscore populeus\textunderscore )}
\end{itemize}
Relativo ao álamo ou choupo.
\section{Populina}
\begin{itemize}
\item {Grp. gram.:f.}
\end{itemize}
\begin{itemize}
\item {Utilização:Chím.}
\end{itemize}
\begin{itemize}
\item {Proveniência:(Do lat. \textunderscore populus\textunderscore )}
\end{itemize}
Substância crystallizável, que se encontra na casca e nas fôlhas do choupo.
\section{Populista}
\begin{itemize}
\item {Grp. gram.:adj.}
\end{itemize}
\begin{itemize}
\item {Utilização:Neol.}
\end{itemize}
\begin{itemize}
\item {Proveniência:(Do lat. \textunderscore populus\textunderscore )}
\end{itemize}
Amigo do povo.
\section{Popúlo}
\begin{itemize}
\item {Grp. gram.:m.}
\end{itemize}
Insecto díptero, cinzento claro e de cabeça volumosa.
\section{Populoso}
\begin{itemize}
\item {Grp. gram.:adj.}
\end{itemize}
\begin{itemize}
\item {Proveniência:(Lat. \textunderscore populosus\textunderscore )}
\end{itemize}
Que abunda em habitantes; muito povoado: \textunderscore região populosa\textunderscore .
\section{Poquéca}
\begin{itemize}
\item {Grp. gram.:f.}
\end{itemize}
\begin{itemize}
\item {Utilização:Bras. do N}
\end{itemize}
O mesmo que \textunderscore moqueca\textunderscore .
\section{Por}
\begin{itemize}
\item {fónica:pur}
\end{itemize}
\begin{itemize}
\item {Grp. gram.:prep.}
\end{itemize}
\begin{itemize}
\item {Proveniência:(Do lat. \textunderscore per\textunderscore  e \textunderscore pro\textunderscore )}
\end{itemize}
(designativa da relação de meio, qualidade, modo, estado, fórma, lugar, causa, duração de tempo, continuação, substituição, cúmulo, reciprocidade, banda ou lado, dedicação, circunstâncias e ainda outras relações, sendo a accepção determinada pela construcção da respectiva phrase)
\section{Pôr}
\begin{itemize}
\item {Grp. gram.:v. t.}
\end{itemize}
\begin{itemize}
\item {Utilização:Fam.}
\end{itemize}
\begin{itemize}
\item {Grp. gram.:Loc.}
\end{itemize}
\begin{itemize}
\item {Utilização:fam.}
\end{itemize}
\begin{itemize}
\item {Grp. gram.:V. p.}
\end{itemize}
\begin{itemize}
\item {Proveniência:(Do lat. \textunderscore ponere\textunderscore )}
\end{itemize}
Collocar: \textunderscore pôr o chá na mesa\textunderscore .
Constituír.
Firmar.
Assentar.
Situar.
Estabelecer.
Expor.
Edificar, erigir: \textunderscore pôr uma estátua\textunderscore .
Dispor, accommodar.
Produzir, causar, infundir: \textunderscore pôr mêdo\textunderscore .
Reduzir, converter, transformar.
Introduzir, meter.
Virar.
Aproximar.
Arriscar, jogar: \textunderscore pôr uma libra numa carta\textunderscore .
Lançar em almoéda.
Offerecer.
Entregar, depor.
Cifrar, fazer consistir.
Apparelhar.
Vestir; usar por adorno: \textunderscore pôr uma gravata\textunderscore .
Applicar: \textunderscore pôr cataplasmas\textunderscore .
Misturar.
Expellir ovos (a ave).
Suppor.
Prescrever, ordenar.
Impor.
Oppor: \textunderscore pôr dúvidas\textunderscore .
Afiançar.
Attribuír.
Apresentar, formular.
Fixar.
Classificar.
Tornar público.
Separar.
Destinar.
Consignar, inscrever: \textunderscore pôr o nome num papel\textunderscore .
Marcar.
Figurar.
Trasladar.
Plantar: \textunderscore pôr alfaces\textunderscore .
Depor, deixar.
Devolver, restituír.
Occultar.
Permittir.
Lançar á conta de outrem.
Dar o nome de:«\textunderscore a ti puseram-te Rosa...\textunderscore »João de Deus.
\textunderscore Pôr com dono\textunderscore , pôr fóra de casa, despedir, desfazer-se de.
Collocar-se.
Dedicar-se.
Exercitar-se.
Aventurar-se.
Converter-se, transformar-se.
Imaginar-se.
Desapparecer no occaso, (falando-se dos astros): \textunderscore pôs-se o Sol\textunderscore .
Occupar-se.--Outras accepções terá êste verbo, só determinadas pela construcção da respectiva phrase.
\section{Poracá}
\begin{itemize}
\item {Grp. gram.:m.}
\end{itemize}
\begin{itemize}
\item {Utilização:Bras. do Rio}
\end{itemize}
Cesto grande para pescaria.
\section{Poranduba}
\begin{itemize}
\item {Grp. gram.:f.}
\end{itemize}
\begin{itemize}
\item {Utilização:Bras}
\end{itemize}
\begin{itemize}
\item {Proveniência:(T. tupi)}
\end{itemize}
História; relação; notícia.
\section{Porantera}
\begin{itemize}
\item {Grp. gram.:f.}
\end{itemize}
\begin{itemize}
\item {Proveniência:(De \textunderscore poro\textunderscore  + \textunderscore anthera\textunderscore )}
\end{itemize}
Gênero de arbustos euforbiáceos da Austrália.
\section{Poranthera}
\begin{itemize}
\item {Grp. gram.:f.}
\end{itemize}
\begin{itemize}
\item {Proveniência:(De \textunderscore poro\textunderscore  + \textunderscore anthera\textunderscore )}
\end{itemize}
Gênero de arbustos euphorbiáceos da Austrália.
\section{Porão}
\begin{itemize}
\item {Grp. gram.:m.}
\end{itemize}
Espaço no interior do navio, para conter comestíveis e outros objectos, situado entre a carlinga e a ponte ou a primeira das pontes, se há mais do que uma.
(Alter. de \textunderscore prão\textunderscore )
\section{Porca}
\begin{itemize}
\item {Grp. gram.:f.}
\end{itemize}
\begin{itemize}
\item {Utilização:Prov.}
\end{itemize}
\begin{itemize}
\item {Utilização:trasm.}
\end{itemize}
\begin{itemize}
\item {Grp. gram.:Loc.}
\end{itemize}
\begin{itemize}
\item {Utilização:pop.}
\end{itemize}
\begin{itemize}
\item {Proveniência:(Lat. \textunderscore porca\textunderscore )}
\end{itemize}
Fêmea do porco.
Peça, em que se introduz a extremidade de um parafuso, para o segurar.
Peça de madeira, em que assenta a ran do moínho.
Travéssa, sôbre as tábuas que assentam no pé da uva, nos lagares.
Espécie de jôgo de rapazes.
Mulhér suja, desleixada.
\textunderscore Sair-lhe a porca mal capada\textunderscore , vêr transtornados os seus negócios ou planos.
\section{Porcada}
\begin{itemize}
\item {Grp. gram.:f.}
\end{itemize}
\begin{itemize}
\item {Utilização:Pop.}
\end{itemize}
\begin{itemize}
\item {Proveniência:(De \textunderscore porco\textunderscore )}
\end{itemize}
Vara de porcos.
Porcaria; trabalho mal feito.
\section{Porcalhão}
\begin{itemize}
\item {Grp. gram.:m.  e  adj.}
\end{itemize}
\begin{itemize}
\item {Proveniência:(De \textunderscore porcalho\textunderscore )}
\end{itemize}
O que é muito porco, immundo.
O que trabalha mal; trapalhão.
\section{Porcalho}
\begin{itemize}
\item {Grp. gram.:m.}
\end{itemize}
\begin{itemize}
\item {Utilização:Ant.}
\end{itemize}
\begin{itemize}
\item {Proveniência:(De \textunderscore porco\textunderscore )}
\end{itemize}
Porco pequeno, leitão, bácoro.
\section{Porcalhota}
\begin{itemize}
\item {Grp. gram.:f.}
\end{itemize}
\begin{itemize}
\item {Utilização:Ant.}
\end{itemize}
\begin{itemize}
\item {Proveniência:(De \textunderscore porcalho\textunderscore )}
\end{itemize}
O mesmo que \textunderscore bácora\textunderscore .
\section{Porção}
\begin{itemize}
\item {Grp. gram.:f.}
\end{itemize}
\begin{itemize}
\item {Proveniência:(Lat. \textunderscore portio\textunderscore )}
\end{itemize}
Parte de alguma coisa; bocado; parcella; fracção; dóse; ração.
\section{Porcaria}
\begin{itemize}
\item {Grp. gram.:f.}
\end{itemize}
\begin{itemize}
\item {Utilização:Fig.}
\end{itemize}
Acto ou estado do que é porco.
Immundície, sujidade.
Palavrão, obscenidade.
Coisa mal feita.
\section{Porcariço}
\begin{itemize}
\item {Grp. gram.:m.}
\end{itemize}
O mesmo que \textunderscore porqueiro\textunderscore .
(Cp. cast. \textunderscore porquerizo\textunderscore )
\section{Porceiro-do-covo}
\begin{itemize}
\item {Grp. gram.:m.}
\end{itemize}
\begin{itemize}
\item {Utilização:Pesc.}
\end{itemize}
\begin{itemize}
\item {Proveniência:(De \textunderscore porção\textunderscore )}
\end{itemize}
Amarração do extremo do segundo cordão do rabo das armações fixas de sardinha á valenciana.
\section{Porcelana}
\begin{itemize}
\item {Grp. gram.:f.}
\end{itemize}
\begin{itemize}
\item {Utilização:Prov.}
\end{itemize}
\begin{itemize}
\item {Proveniência:(It. \textunderscore porcellana\textunderscore )}
\end{itemize}
Mollusco gasterópode de concha univalve.
Nácar, extrahida desta concha, para a fabricação de vasos e de outros utensílios.
Loiça fina, dura e translúcida, feita de caulim e feldspatho.
Tigela.
\section{Porcelânico}
\begin{itemize}
\item {Grp. gram.:adj.}
\end{itemize}
Que tem a apparência da porcelana: \textunderscore jaspe porcelânico...\textunderscore 
\section{Porcelanito}
\begin{itemize}
\item {Grp. gram.:m.}
\end{itemize}
Espécie de jaspe, parecido com a porcelana.
\section{Porcélio}
\begin{itemize}
\item {Grp. gram.:m.}
\end{itemize}
\begin{itemize}
\item {Proveniência:(Do lat. \textunderscore porcellus\textunderscore , porco pequeno)}
\end{itemize}
O mesmo que \textunderscore bicho-de-conta\textunderscore .
\section{Porcéllio}
\begin{itemize}
\item {Grp. gram.:m.}
\end{itemize}
\begin{itemize}
\item {Proveniência:(Do lat. \textunderscore porcellus\textunderscore , porco pequeno)}
\end{itemize}
O mesmo que \textunderscore bicho-de-conta\textunderscore .
\section{Porcentagem}
\begin{itemize}
\item {Grp. gram.:f.}
\end{itemize}
\begin{itemize}
\item {Utilização:bras}
\end{itemize}
\begin{itemize}
\item {Utilização:Neol.}
\end{itemize}
O mesmo que \textunderscore percentagem\textunderscore . Cf. \textunderscore Diár. Official\textunderscore , do Brasil, de 17-X-900; \textunderscore Jorn. do Comm.\textunderscore , do Rio, de 15-X-900.
\section{Porcentual}
\begin{itemize}
\item {Grp. gram.:adj.}
\end{itemize}
\begin{itemize}
\item {Utilização:bras}
\end{itemize}
\begin{itemize}
\item {Utilização:Neol.}
\end{itemize}
Relativo a porcentagem.
\section{Porcino}
\begin{itemize}
\item {Grp. gram.:adj.}
\end{itemize}
\begin{itemize}
\item {Proveniência:(Lat. \textunderscore porcinus\textunderscore )}
\end{itemize}
Relativo ao porco; suíno.
\section{Porcionário}
\begin{itemize}
\item {Grp. gram.:m.}
\end{itemize}
Aquelle que tem ou recebe uma porção ou qualquer pensão ou rendas; beneficiado ecclesiastico.
(Do baixo lat. \textunderscore portionarius\textunderscore )
\section{Porcioneiro}
\begin{itemize}
\item {Grp. gram.:m.}
\end{itemize}
O mesmo que \textunderscore porcionário\textunderscore .
\section{Porcionista}
\begin{itemize}
\item {Grp. gram.:m.  e  f.}
\end{itemize}
\begin{itemize}
\item {Proveniência:(Do lat. \textunderscore portio\textunderscore )}
\end{itemize}
Alumno ou alumna, que num collégio paga a sua educação ou sustento.
\section{Porciúncula}
\begin{itemize}
\item {Grp. gram.:f.}
\end{itemize}
\begin{itemize}
\item {Proveniência:(Lat. \textunderscore portiuncula\textunderscore )}
\end{itemize}
Porção pequena.
Festa da Ordem de San-Francisco.
O primeiro convento da Ordem franciscana.
\section{Porco}
\begin{itemize}
\item {fónica:pôr}
\end{itemize}
\begin{itemize}
\item {Grp. gram.:m.}
\end{itemize}
\begin{itemize}
\item {Utilização:Ext.}
\end{itemize}
\begin{itemize}
\item {Utilização:Fig.}
\end{itemize}
\begin{itemize}
\item {Utilização:Prov.}
\end{itemize}
\begin{itemize}
\item {Utilização:minh.}
\end{itemize}
\begin{itemize}
\item {Grp. gram.:Loc.}
\end{itemize}
\begin{itemize}
\item {Utilização:bras. do N}
\end{itemize}
\begin{itemize}
\item {Grp. gram.:Loc.}
\end{itemize}
\begin{itemize}
\item {Utilização:bras. do N}
\end{itemize}
\begin{itemize}
\item {Grp. gram.:Adj.}
\end{itemize}
\begin{itemize}
\item {Proveniência:(Lat. \textunderscore porcus\textunderscore )}
\end{itemize}
Quadrúpede da classe dos mammíferos e da ordem dos pachydermes.
Carne de porco.
Indivíduo sujo, immundo.
O mesmo que \textunderscore bebedeira\textunderscore .
\textunderscore Montado no porco\textunderscore , embriagado.
\textunderscore Tomar o porco\textunderscore , embriagar-se.
Sujo; immundo.
Torpe, obsceno: \textunderscore linguagem porca\textunderscore .
Que faz tudo sem esmêro; trapalhão.
\section{Porcó}
\begin{itemize}
\item {Grp. gram.:m.}
\end{itemize}
\begin{itemize}
\item {Utilização:Gír.}
\end{itemize}
Porco.
\section{Porco-bravo}
\begin{itemize}
\item {Grp. gram.:m.}
\end{itemize}
O mesmo que \textunderscore javali\textunderscore .
\section{Porco-da-índia}
\begin{itemize}
\item {Grp. gram.:m.}
\end{itemize}
O mesmo que [[porquinho-da-índia|porquinho]].
\section{Porco-do-mar}
\begin{itemize}
\item {Grp. gram.:m.}
\end{itemize}
Crustáceo carnívoro, (\textunderscore phocaena communis\textunderscore ), semelhante ao golfinho. Cf. P. Moraes, \textunderscore Zool. Elem.\textunderscore , 259.
\section{Porco-espim}
\begin{itemize}
\item {Grp. gram.:m.}
\end{itemize}
Mammífero roedor, armado de espinhos.
\section{Porco-espinho}
\begin{itemize}
\item {Grp. gram.:m.}
\end{itemize}
Mammífero roedor, armado de espinhos.
\section{Porco-montês}
\begin{itemize}
\item {Grp. gram.:m.}
\end{itemize}
O mesmo que \textunderscore javali\textunderscore .
\section{Porco-ribeiro}
\begin{itemize}
\item {Grp. gram.:m.}
\end{itemize}
Ave, o mesmo que \textunderscore pica-peixe\textunderscore . Cf. P. Moraes, \textunderscore Zool. Elem.\textunderscore , 297.
\section{Porco-sujo}
\begin{itemize}
\item {Grp. gram.:m.}
\end{itemize}
\begin{itemize}
\item {Utilização:Pop.}
\end{itemize}
O mesmo que \textunderscore demónio\textunderscore :«\textunderscore vou-me, disse o porco-sujo.\textunderscore »Garrett, \textunderscore Fábulas\textunderscore , 85.
\section{Porco-veado}
\begin{itemize}
\item {Grp. gram.:m.}
\end{itemize}
O mesmo que \textunderscore barbirusa\textunderscore . Cf. P. Moraes, \textunderscore Zool. Elem.\textunderscore , 220.
\section{Porecamecrans}
\begin{itemize}
\item {Grp. gram.:m. pl.}
\end{itemize}
Tríbo de Índios do Brasil, entre o Tocantins e o Araguaia.
\section{Porejar}
\begin{itemize}
\item {Grp. gram.:v. t.}
\end{itemize}
\begin{itemize}
\item {Grp. gram.:V. i.}
\end{itemize}
Expellir pelos poros; destillar.
Saír pelos poros.
\section{Porém}
\begin{itemize}
\item {Grp. gram.:conj.}
\end{itemize}
\begin{itemize}
\item {Grp. gram.:Loc. pron.}
\end{itemize}
\begin{itemize}
\item {Utilização:ant.}
\end{itemize}
\begin{itemize}
\item {Proveniência:(Lat. \textunderscore proinde\textunderscore )}
\end{itemize}
(designativa de opposição ou restricção)
Todavia; apesar disso; mas.
Por isso, por tanto, porende. Cf. Fernão Lopes, \textunderscore passim\textunderscore .
\section{Porencefalia}
\begin{itemize}
\item {Grp. gram.:f.}
\end{itemize}
\begin{itemize}
\item {Utilização:Med.}
\end{itemize}
\begin{itemize}
\item {Proveniência:(De \textunderscore poro\textunderscore  + \textunderscore encéfalo\textunderscore )}
\end{itemize}
Doença, caracterizada pela formação de escavações na superfície dos hemisférios cerebraes.
\section{Porencephalia}
\begin{itemize}
\item {Grp. gram.:f.}
\end{itemize}
\begin{itemize}
\item {Utilização:Med.}
\end{itemize}
\begin{itemize}
\item {Proveniência:(De \textunderscore poro\textunderscore  + \textunderscore encéphalo\textunderscore )}
\end{itemize}
Doença, caracterizada pela formação de escavações na superfície dos hemisphérios cerebraes.
\section{Porende}
\begin{itemize}
\item {Grp. gram.:adv. Loc. pron.}
\end{itemize}
\begin{itemize}
\item {Utilização:ant.}
\end{itemize}
\begin{itemize}
\item {Proveniência:(Do lat. \textunderscore proinde\textunderscore )}
\end{itemize}
Por isso.
\section{Poreroca}
\begin{itemize}
\item {Grp. gram.:f.}
\end{itemize}
(V. \textunderscore pororoca\textunderscore ^1)
\section{Porfia}
\begin{itemize}
\item {Grp. gram.:f.}
\end{itemize}
\begin{itemize}
\item {Grp. gram.:Loc. adv.}
\end{itemize}
\begin{itemize}
\item {Proveniência:(De \textunderscore porfiar\textunderscore )}
\end{itemize}
Discussão; contenda de palavras.
Insistência; pertinácia; constância.
\textunderscore Á porfia\textunderscore , entre competidores; qual mais.
\section{Porfiada}
\begin{itemize}
\item {Grp. gram.:f.}
\end{itemize}
\begin{itemize}
\item {Utilização:Pesc.}
\end{itemize}
\begin{itemize}
\item {Proveniência:(De \textunderscore porfiar\textunderscore ^2)}
\end{itemize}
Cosedura, que une as testas das redes, umas ás outras, por meio de um fio passado nas malhas.
\section{Porfiadamente}
\begin{itemize}
\item {Grp. gram.:adv.}
\end{itemize}
De modo porfiado.
Teimosamente; com pertinácia.
\section{Porfiado}
\begin{itemize}
\item {Grp. gram.:adj.}
\end{itemize}
\begin{itemize}
\item {Proveniência:(De \textunderscore porfiar\textunderscore ^1)}
\end{itemize}
Em que houve porfia; discutido.
Pertinaz.
\section{Porfiador}
\begin{itemize}
\item {Grp. gram.:m.  e  adj.}
\end{itemize}
O que porfia; teimoso; contumaz.
\section{Porfiar}
\begin{itemize}
\item {Grp. gram.:v. i.}
\end{itemize}
Discutir, questionar com obstinação.
Sêr rival.
(Por \textunderscore perfiar\textunderscore , do lat. hyp. \textunderscore perfidiare\textunderscore )
\section{Porfiar}
\begin{itemize}
\item {Grp. gram.:v. t.}
\end{itemize}
\begin{itemize}
\item {Proveniência:(De \textunderscore por\textunderscore  + \textunderscore fio\textunderscore )}
\end{itemize}
Guarnecer com fio (um cabo ou linha).
Coser (cabos ou tralhas) com um fio ou cabo mais delgado.
\section{Porfidito}
\begin{itemize}
\item {Grp. gram.:m.}
\end{itemize}
O mesmo que \textunderscore porphryto\textunderscore .
\section{Pórfido}
\begin{itemize}
\item {Grp. gram.:m.}
\end{itemize}
\begin{itemize}
\item {Proveniência:(It. \textunderscore porfido\textunderscore )}
\end{itemize}
O mesmo que \textunderscore pórphyro\textunderscore .
\section{Porfio}
\begin{itemize}
\item {Grp. gram.:m.}
\end{itemize}
\begin{itemize}
\item {Utilização:Pesc.}
\end{itemize}
\begin{itemize}
\item {Proveniência:(De \textunderscore porfiar\textunderscore ^2)}
\end{itemize}
Amarração do canto da testinha das armações de sardinha á valenciana.
\section{Porfiosamente}
\begin{itemize}
\item {Grp. gram.:adv.}
\end{itemize}
De modo porfioso; porfiadamente.
\section{Porfioso}
\begin{itemize}
\item {Grp. gram.:adj.}
\end{itemize}
Em que há porfia.
Amigo de porfiar.
Constante, contínuo.
\section{Pórfira}
\begin{itemize}
\item {Grp. gram.:f.}
\end{itemize}
\begin{itemize}
\item {Proveniência:(Do gr. \textunderscore porphura\textunderscore )}
\end{itemize}
Edifício, destinado a realizar-se nelle o nascimento dos filhos dos Imperadores romanos:«\textunderscore o império introduziu Constantino Magno haver certo lugar, ou palácio, separado para nascerem os filhos dos Imperadores, o qual palácio se chamava pórfira\textunderscore ». Bernárdez, \textunderscore Luz e Calor\textunderscore , 347.
\section{Porfirião}
\begin{itemize}
\item {Grp. gram.:m.}
\end{itemize}
\begin{itemize}
\item {Proveniência:(Gr. \textunderscore porphurion\textunderscore )}
\end{itemize}
Espécie de galinhola.
\section{Porfírico}
\begin{itemize}
\item {Grp. gram.:adj.}
\end{itemize}
Que contém pórfiro.
\section{Porfirita}
\begin{itemize}
\item {Grp. gram.:f.}
\end{itemize}
\begin{itemize}
\item {Proveniência:(Gr. \textunderscore porphurites\textunderscore )}
\end{itemize}
Espécie de pedra dura e vermelha, procedente do Alto Egipto, o mesmo que \textunderscore pórfiro\textunderscore .
\section{Porfirítico}
\begin{itemize}
\item {Grp. gram.:adj.}
\end{itemize}
O mesmo que \textunderscore porfírico\textunderscore .
\section{Porfirito}
\begin{itemize}
\item {Grp. gram.:m.}
\end{itemize}
O mesmo ou melhor que \textunderscore porfirita\textunderscore .
\section{Porfirização}
\begin{itemize}
\item {Grp. gram.:f.}
\end{itemize}
Acto ou efeito de porfirizar.
\section{Porfirizar}
\begin{itemize}
\item {Grp. gram.:v. t.}
\end{itemize}
\begin{itemize}
\item {Proveniência:(De \textunderscore pórfiro\textunderscore )}
\end{itemize}
Reduzir a pó muito fino.
\section{Pórfiro}
\begin{itemize}
\item {Grp. gram.:m.}
\end{itemize}
\begin{itemize}
\item {Utilização:Ext.}
\end{itemize}
\begin{itemize}
\item {Utilização:Restrict.}
\end{itemize}
\begin{itemize}
\item {Proveniência:(Do gr. \textunderscore porphura\textunderscore , púrpura)}
\end{itemize}
Nome, dado pelos antigos a uma espécie de mármore de côr verde ou purpúrea, salpicado de manchas brancas ou de várias côres.
Qualquer mármore, que apresenta cristaes, cuja brancura resalta da côr do fundo.
Pedra basáltica, muito dura, vermelha ou escura, composta de feldspato, quartzo e mica.
Instrumento de farmácia, composto de uma tábua de pórfiro ou de outra pedra dura e de uma roseta, destinado a reduzir a pó muito fino substâncias sólidas.
\section{Porfirogênito}
\begin{itemize}
\item {Grp. gram.:m.}
\end{itemize}
\begin{itemize}
\item {Proveniência:(De \textunderscore pórfira\textunderscore  + \textunderscore gênito\textunderscore )}
\end{itemize}
Aquelle que nasceu na pórfira; príncipe. Cf. Bernárdez, \textunderscore Luz e Calor\textunderscore , 347.
\section{Póri}
\begin{itemize}
\item {Grp. gram.:m.}
\end{itemize}
Medida de capacidade, nas ilhas de Gôa e em Bardez, equivalente a 48 pollegadas cúbicas.
\section{Porisma}
\begin{itemize}
\item {Grp. gram.:m.}
\end{itemize}
\begin{itemize}
\item {Utilização:Mathem.}
\end{itemize}
\begin{itemize}
\item {Proveniência:(Gr. \textunderscore porisma\textunderscore )}
\end{itemize}
Problema, cuja solução consiste em tirar das condições expostas no enunciado uma verdade geométrica.
\section{Porisso}
(contr. de \textunderscore por\textunderscore  + \textunderscore isso\textunderscore )
\section{Porito}
\begin{itemize}
\item {Grp. gram.:m.}
\end{itemize}
\begin{itemize}
\item {Proveniência:(De \textunderscore poro\textunderscore )}
\end{itemize}
Gênero de polypeiros, do grupo das madréporas.
\section{Pormenór}
\begin{itemize}
\item {Grp. gram.:m.}
\end{itemize}
\begin{itemize}
\item {Proveniência:(De \textunderscore por\textunderscore  + \textunderscore menór\textunderscore )}
\end{itemize}
Particularidade; minúcia, minudência.
\section{Pormenorização}
\begin{itemize}
\item {Grp. gram.:f.}
\end{itemize}
Acto ou effeito de pormenorizar.
\section{Pormenorizar}
\begin{itemize}
\item {Grp. gram.:v. t.}
\end{itemize}
\begin{itemize}
\item {Utilização:Neol.}
\end{itemize}
Referir minuciosamente.
Expôr os pormenores de.
\section{Porneu}
\begin{itemize}
\item {Grp. gram.:m.}
\end{itemize}
\begin{itemize}
\item {Utilização:Ant.}
\end{itemize}
\begin{itemize}
\item {Proveniência:(Do gr. \textunderscore porneion\textunderscore )}
\end{itemize}
Lupanar; devassidão.
\section{Pornocracia}
\begin{itemize}
\item {Grp. gram.:f.}
\end{itemize}
\begin{itemize}
\item {Proveniência:(Do gr. \textunderscore porne\textunderscore  + \textunderscore krateia\textunderscore )}
\end{itemize}
Influência ou preponderância das cortesans na governação pública.
\section{Pornocrático}
\begin{itemize}
\item {Grp. gram.:adj.}
\end{itemize}
Relativo á pornocracia.
\section{Pornografar}
\begin{itemize}
\item {Grp. gram.:v. t.}
\end{itemize}
Descrever pornograficamente.
Descrever (actos ou episódios obscenos).
(Cp. \textunderscore pornógrafo\textunderscore )
\section{Pornografia}
\begin{itemize}
\item {Grp. gram.:f.}
\end{itemize}
\begin{itemize}
\item {Proveniência:(De \textunderscore pornógrafo\textunderscore )}
\end{itemize}
Tratado á cêrca da prostituição.
Colecção de pinturas ou gravuras obscenas.
Carácter obsceno de uma publicação.
Devassidão.
\section{Pornograficamente}
\begin{itemize}
\item {Grp. gram.:adv.}
\end{itemize}
De modo pornográfico, de modo obsceno.
\section{Pornográfico}
\begin{itemize}
\item {Grp. gram.:adj.}
\end{itemize}
Relativo a pornografia; obsceno; libidinoso.
\section{Pornografismo}
\begin{itemize}
\item {Grp. gram.:m.}
\end{itemize}
\begin{itemize}
\item {Proveniência:(De \textunderscore pornografia\textunderscore )}
\end{itemize}
Uso de descripções pornográficas ou obscenas.
\section{Pornógrafo}
\begin{itemize}
\item {Grp. gram.:m.}
\end{itemize}
\begin{itemize}
\item {Proveniência:(Do gr. \textunderscore porne\textunderscore  + \textunderscore graphein\textunderscore )}
\end{itemize}
Aquele que trata de pornografia.
O que descreve ou pinta coisas obscenas.
\section{Pornographar}
\begin{itemize}
\item {Grp. gram.:v. t.}
\end{itemize}
Descrever pornographicamente.
Descrever (actos ou episódios obscenos).
(Cp. \textunderscore pornógrapho\textunderscore )
\section{Pornographia}
\begin{itemize}
\item {Grp. gram.:f.}
\end{itemize}
\begin{itemize}
\item {Proveniência:(De \textunderscore pornógrapho\textunderscore )}
\end{itemize}
Tratado á cêrca da prostituição.
Collecção de pinturas ou gravuras obscenas.
Carácter obsceno de uma publicação.
Devassidão.
\section{Pornographicamente}
\begin{itemize}
\item {Grp. gram.:adv.}
\end{itemize}
De modo pornográphico, de modo obsceno.
\section{Pornográphico}
\begin{itemize}
\item {Grp. gram.:adj.}
\end{itemize}
Relativo a pornographia; obsceno; libidinoso.
\section{Pornographismo}
\begin{itemize}
\item {Grp. gram.:m.}
\end{itemize}
\begin{itemize}
\item {Proveniência:(De \textunderscore pornographia\textunderscore )}
\end{itemize}
Uso de descripções pornográphicas ou obscenas.
\section{Pornógrapho}
\begin{itemize}
\item {Grp. gram.:m.}
\end{itemize}
\begin{itemize}
\item {Proveniência:(Do gr. \textunderscore porne\textunderscore  + \textunderscore graphein\textunderscore )}
\end{itemize}
Aquelle que trata de pornographia.
O que descreve ou pinta coisas obscenas.
\section{Pórphyra}
\begin{itemize}
\item {Grp. gram.:f.}
\end{itemize}
\begin{itemize}
\item {Proveniência:(Do gr. \textunderscore porphura\textunderscore )}
\end{itemize}
Edifício, destinado a realizar-se nelle o nascimento dos filhos dos Imperadores romanos:«\textunderscore o império introduziu Constantino Magno haver certo lugar, ou palácio, separado para nascerem os filhos dos Imperadores, o qual palácio se chamava pórfira\textunderscore ». Bernárdez, \textunderscore Luz e Calor\textunderscore , 347.
\section{Poro}
\begin{itemize}
\item {Grp. gram.:m.}
\end{itemize}
\begin{itemize}
\item {Proveniência:(Lat. \textunderscore porus\textunderscore )}
\end{itemize}
Cada um dos pequeninos orifícios da derme.
Cada um dos interstícios hypothéticos, entre as moléculas que constituem os corpos.
Cada um dos pequenos orifícios, de que estão crivados os vegetaes.
\section{Porocele}
\begin{itemize}
\item {Grp. gram.:m.}
\end{itemize}
\begin{itemize}
\item {Proveniência:(Do gr. \textunderscore poros\textunderscore  + \textunderscore kele\textunderscore )}
\end{itemize}
Espécie de hérnia, com endurecimento do saco herniário.
\section{Poroceratose}
\begin{itemize}
\item {Grp. gram.:f.}
\end{itemize}
\begin{itemize}
\item {Utilização:Med.}
\end{itemize}
\begin{itemize}
\item {Proveniência:(Do gr. \textunderscore poros\textunderscore  + \textunderscore keras\textunderscore )}
\end{itemize}
Dermatose, caracterizada por saliências córneas superficiaes.
\section{Porócito}
\begin{itemize}
\item {Grp. gram.:m.}
\end{itemize}
\begin{itemize}
\item {Utilização:Zool.}
\end{itemize}
\begin{itemize}
\item {Proveniência:(Do gr. \textunderscore poros\textunderscore  + \textunderscore kutos\textunderscore )}
\end{itemize}
Célula cilíndrica perfurada, nas esponjas.
\section{Porócyto}
\begin{itemize}
\item {Grp. gram.:m.}
\end{itemize}
\begin{itemize}
\item {Utilização:Zool.}
\end{itemize}
\begin{itemize}
\item {Proveniência:(Do gr. \textunderscore poros\textunderscore  + \textunderscore kutos\textunderscore )}
\end{itemize}
Céllula cylíndrica perfurada, nas esponjas.
\section{Porongo}
\begin{itemize}
\item {Grp. gram.:m.}
\end{itemize}
\begin{itemize}
\item {Utilização:Bras. do S}
\end{itemize}
Pequena planta cucurbitácea, de que se fazem cuias para mate.
(Do quichua \textunderscore purunca\textunderscore )
\section{Poroplástico}
\begin{itemize}
\item {Grp. gram.:m.}
\end{itemize}
\begin{itemize}
\item {Utilização:Med.}
\end{itemize}
\begin{itemize}
\item {Proveniência:(Do gr. \textunderscore poros\textunderscore  + \textunderscore plassein\textunderscore )}
\end{itemize}
Espécie de feltro, que se applica no tratamento radioscópico das fracturas dos ossos.
\section{Poróptero}
\begin{itemize}
\item {Grp. gram.:m.}
\end{itemize}
\begin{itemize}
\item {Proveniência:(Do gr. \textunderscore poros\textunderscore  + \textunderscore pteron\textunderscore )}
\end{itemize}
Gênero de insectos coleópteros tetrâmeros.
\section{Porora}
\begin{itemize}
\item {Grp. gram.:adv.}
\end{itemize}
Por agora; ainda.
(Contr. de \textunderscore por\textunderscore  + \textunderscore hora\textunderscore )
\section{Pororoca}
\begin{itemize}
\item {Grp. gram.:f.}
\end{itemize}
\begin{itemize}
\item {Utilização:Bras}
\end{itemize}
Elevação repentina de grandes massas de água junto á foz dos grandes rios, produzida, segundo uns, por influência vulcânica, e, segundo outros, pelo encontro de marés ou de correntes oppostas. Cf. S. de Frias, \textunderscore Viagem ao Amazonas\textunderscore , 51.
(Do tupi \textunderscore poreoca\textunderscore )
\section{Pororoca}
\begin{itemize}
\item {Grp. gram.:f.}
\end{itemize}
Planta clusiácea de Pernambuco, (\textunderscore clusia volubilis\textunderscore ).
\section{Pororoca}
\begin{itemize}
\item {Grp. gram.:f.}
\end{itemize}
\begin{itemize}
\item {Utilização:Bras. do N}
\end{itemize}
O mesmo que \textunderscore pipoca\textunderscore .
\section{Pororocar}
\begin{itemize}
\item {Grp. gram.:v. i.}
\end{itemize}
\begin{itemize}
\item {Utilização:Bras}
\end{itemize}
Produzir pororoca (o rio).
\section{Pororom}
\begin{itemize}
\item {Grp. gram.:m.  e  adj.}
\end{itemize}
\begin{itemize}
\item {Utilização:Bras. do N}
\end{itemize}
Fruta atrophiada ou de má qualidade.
\section{Porose}
\begin{itemize}
\item {Grp. gram.:f.}
\end{itemize}
\begin{itemize}
\item {Utilização:Med.}
\end{itemize}
\begin{itemize}
\item {Proveniência:(Do gr. \textunderscore poros\textunderscore )}
\end{itemize}
Lesão cadavérica no encéphalo, de aspecto caseiforme.
\section{Porosidade}
\begin{itemize}
\item {Grp. gram.:f.}
\end{itemize}
Qualidade daquillo que é poroso.
\section{Poroso}
\begin{itemize}
\item {Grp. gram.:adj.}
\end{itemize}
Que tem poros.
\section{Pórphydo}
\begin{itemize}
\item {Grp. gram.:m.}
\end{itemize}
O mesmo que \textunderscore pórphyro\textunderscore .
\section{Porphyrião}
\begin{itemize}
\item {Grp. gram.:m.}
\end{itemize}
\begin{itemize}
\item {Proveniência:(Gr. \textunderscore porphurion\textunderscore )}
\end{itemize}
Espécie de gallinhola.
\section{Porphýrico}
\begin{itemize}
\item {Grp. gram.:adj.}
\end{itemize}
Que contém pórphyro.
\section{Porphyrita}
\begin{itemize}
\item {Grp. gram.:f.}
\end{itemize}
\begin{itemize}
\item {Proveniência:(Gr. \textunderscore porphurites\textunderscore )}
\end{itemize}
Espécie de pedra dura e vermelha, procedente do Alto Egypto, o mesmo que \textunderscore pórphyro\textunderscore .
\section{Porphyrítico}
\begin{itemize}
\item {Grp. gram.:adj.}
\end{itemize}
O mesmo que \textunderscore porphýrico\textunderscore .
\section{Porphyrito}
\begin{itemize}
\item {Grp. gram.:m.}
\end{itemize}
O mesmo ou melhor que \textunderscore porphyrita\textunderscore .
\section{Porphyrização}
\begin{itemize}
\item {Grp. gram.:f.}
\end{itemize}
Acto ou effeito de porphyrizar.
\section{Porphyrizar}
\begin{itemize}
\item {Grp. gram.:v. t.}
\end{itemize}
\begin{itemize}
\item {Proveniência:(De \textunderscore pórphyro\textunderscore )}
\end{itemize}
Reduzir a pó muito fino.
\section{Pórphyro}
\begin{itemize}
\item {Grp. gram.:m.}
\end{itemize}
\begin{itemize}
\item {Utilização:Ext.}
\end{itemize}
\begin{itemize}
\item {Utilização:Restrict.}
\end{itemize}
\begin{itemize}
\item {Proveniência:(Do gr. \textunderscore porphura\textunderscore , púrpura)}
\end{itemize}
Nome, dado pelos antigos a uma espécie de mármore de côr verde ou purpúrea, salpicado de manchas brancas ou de várias côres.
Qualquer mármore, que apresenta crystaes, cuja brancura resalta da côr do fundo.
Pedra basáltica, muito dura, vermelha ou escura, composta de feldspatho, quartzo e mica.
Instrumento de pharmácia, composto de uma tábua de pórphyro ou de outra pedra dura e de uma roseta, destinado a reduzir a pó muito fino substâncias sólidas.
\section{Porfiróide}
\begin{itemize}
\item {Grp. gram.:adj.}
\end{itemize}
\begin{itemize}
\item {Utilização:Geol.}
\end{itemize}
\begin{itemize}
\item {Proveniência:(De \textunderscore pórfiro\textunderscore  + gr. \textunderscore eidos\textunderscore )}
\end{itemize}
Diz-se de alguns mineraes do tipo do sienito, em que a pasta fundamental é macrocristalina.
\section{Porfirólito}
\begin{itemize}
\item {Grp. gram.:m.}
\end{itemize}
\begin{itemize}
\item {Utilização:Miner.}
\end{itemize}
\begin{itemize}
\item {Proveniência:(Do gr. \textunderscore porphura\textunderscore  + \textunderscore lithos\textunderscore )}
\end{itemize}
Pedra, da natureza do pórfiro, empregada em construcções.
\section{Porphyróide}
\begin{itemize}
\item {Grp. gram.:adj.}
\end{itemize}
\begin{itemize}
\item {Utilização:Geol.}
\end{itemize}
\begin{itemize}
\item {Proveniência:(De \textunderscore pórphyro\textunderscore  + gr. \textunderscore eidos\textunderscore )}
\end{itemize}
Diz-se de alguns mineraes do typo do syenito, em que a pasta fundamental é macrocrystallina.
\section{Porphyrólitho}
\begin{itemize}
\item {Grp. gram.:m.}
\end{itemize}
\begin{itemize}
\item {Utilização:Miner.}
\end{itemize}
\begin{itemize}
\item {Proveniência:(Do gr. \textunderscore porphura\textunderscore  + \textunderscore lithos\textunderscore )}
\end{itemize}
Pedra, da natureza do pórphyro, empregada em construcções.
\section{Porpiá}
\begin{itemize}
\item {Grp. gram.:f.}
\end{itemize}
(V.propiá)
\section{Porpianho}
\begin{itemize}
\item {Grp. gram.:m.}
\end{itemize}
\begin{itemize}
\item {Utilização:Prov.}
\end{itemize}
\begin{itemize}
\item {Utilização:trasm.}
\end{itemize}
Parede estreita de cantaria singela.
(Cp. \textunderscore perpianho\textunderscore )
\section{Porquanto}
\begin{itemize}
\item {Grp. gram.:loc. conj.}
\end{itemize}
\begin{itemize}
\item {Proveniência:(De \textunderscore por\textunderscore  + \textunderscore quanto\textunderscore )}
\end{itemize}
Por isso que; porque.
\section{Porque}
\begin{itemize}
\item {fónica:purque}
\end{itemize}
\begin{itemize}
\item {Grp. gram.:conj.}
\end{itemize}
\begin{itemize}
\item {Grp. gram.:M.}
\end{itemize}
\begin{itemize}
\item {Proveniência:(De \textunderscore por\textunderscore  + \textunderscore que\textunderscore )}
\end{itemize}
(indicativa de \textunderscore causa\textunderscore )
A fim de que; por que razão.
Causa, razão.--Insuladamente, ou no fim da phrase, lê-se \textunderscore porquê\textunderscore .
\section{Porqueira}
\begin{itemize}
\item {Grp. gram.:f.}
\end{itemize}
\begin{itemize}
\item {Proveniência:(Do lat. \textunderscore porcaria\textunderscore )}
\end{itemize}
Curral de porcos.
Casa immunda.
Porcaria.
Mulhér, que trata de porcos.
\section{Porqueirão}
\begin{itemize}
\item {Grp. gram.:m.  e  adj.}
\end{itemize}
\begin{itemize}
\item {Proveniência:(De \textunderscore porqueiro\textunderscore )}
\end{itemize}
O mesmo que \textunderscore porcalhão\textunderscore .
\section{Porqueiro}
\begin{itemize}
\item {Grp. gram.:adj.}
\end{itemize}
\begin{itemize}
\item {Grp. gram.:M.}
\end{itemize}
\begin{itemize}
\item {Proveniência:(De \textunderscore porco\textunderscore )}
\end{itemize}
Relativo a porcos.
Diz-se de uma espécie de abóbora.
Diz-se de uma espécie de couve, de caule alto e qualidade inferior.
Guardador de porcos.
\section{Porquerizo}
\begin{itemize}
\item {Grp. gram.:m.}
\end{itemize}
\begin{itemize}
\item {Utilização:Ant.}
\end{itemize}
O mesmo que \textunderscore porcariço\textunderscore . Cf. \textunderscore Eufrosina\textunderscore , 212.
(Cast. \textunderscore porquerizo\textunderscore )
\section{Porquetes}
\begin{itemize}
\item {fónica:quê}
\end{itemize}
\begin{itemize}
\item {Grp. gram.:m. pl.}
\end{itemize}
\begin{itemize}
\item {Utilização:Náut.}
\end{itemize}
\begin{itemize}
\item {Proveniência:(De \textunderscore porca\textunderscore ?)}
\end{itemize}
Paus, sôbre a carlinga de um navio, para fortalecer uma parte do casco.
\section{Porquice}
\begin{itemize}
\item {Grp. gram.:f.}
\end{itemize}
\begin{itemize}
\item {Utilização:Prov.}
\end{itemize}
O mesmo que \textunderscore porcaria\textunderscore .
\section{Porquidade}
\begin{itemize}
\item {Grp. gram.:f.}
\end{itemize}
O mesmo que \textunderscore porcaria\textunderscore .
\section{Porquidão}
\begin{itemize}
\item {Grp. gram.:f.}
\end{itemize}
(V.porquidade)
\section{Pòrquinha}
\begin{itemize}
\item {Grp. gram.:f.}
\end{itemize}
\begin{itemize}
\item {Utilização:Prov.}
\end{itemize}
\begin{itemize}
\item {Utilização:trasm.}
\end{itemize}
\begin{itemize}
\item {Proveniência:(De \textunderscore porca\textunderscore ?)}
\end{itemize}
Jôgo de rapazes, em que se emprega um nó de raíz de giesta.
O mesmo nó.
\section{Porquinha}
\begin{itemize}
\item {Grp. gram.:f.}
\end{itemize}
Porca pequena; bácora.
\section{Porquinho}
\begin{itemize}
\item {Grp. gram.:m.}
\end{itemize}
\begin{itemize}
\item {Proveniência:(De \textunderscore porco\textunderscore )}
\end{itemize}
Bácoro.
Mólho de linho.
Cobaia.
\textunderscore Porquinho da Índia\textunderscore , o mesmo que \textunderscore cobaia\textunderscore .
\section{Porráceo}
\begin{itemize}
\item {Grp. gram.:adj.}
\end{itemize}
\begin{itemize}
\item {Proveniência:(De \textunderscore pôrro\textunderscore )}
\end{itemize}
Que tem a côr esverdeada do pôrro.
\section{Porrada}
\begin{itemize}
\item {Grp. gram.:f.}
\end{itemize}
\begin{itemize}
\item {Utilização:Chul.}
\end{itemize}
Pancada com cacete; pancada. Cf. \textunderscore Filodemo\textunderscore , II, 5.
(Gall. \textunderscore porrada\textunderscore )
\section{Porrada}
\begin{itemize}
\item {Grp. gram.:f.}
\end{itemize}
\begin{itemize}
\item {Utilização:Ant.}
\end{itemize}
Qualquer guisado em que entram alhos porros.
\section{Porradaria}
\begin{itemize}
\item {Grp. gram.:f.}
\end{itemize}
\begin{itemize}
\item {Utilização:Chul.}
\end{itemize}
Muitas pancadas; pancadaria.
\section{Porral}
\begin{itemize}
\item {Grp. gram.:m.}
\end{itemize}
\begin{itemize}
\item {Proveniência:(Do b. lat. \textunderscore porralis\textunderscore )}
\end{itemize}
Terreno, em que crescem porros.
\section{Porrão}
\begin{itemize}
\item {Grp. gram.:m.}
\end{itemize}
\begin{itemize}
\item {Utilização:Fig.}
\end{itemize}
Pote, vasilha de barro, geralmente bojuda, de bôca e fundo estreitos.
Homem baixo e atarracado.
(Cast. \textunderscore porrón\textunderscore )
\section{Pôrras}
\begin{itemize}
\item {Grp. gram.:f.}
\end{itemize}
\begin{itemize}
\item {Utilização:ant.}
\end{itemize}
\begin{itemize}
\item {Utilização:Chul.}
\end{itemize}
Homem ordinário, bisbórria; bebedolas.
\section{Pórre}
\begin{itemize}
\item {Grp. gram.:m.}
\end{itemize}
\begin{itemize}
\item {Utilização:Bras}
\end{itemize}
O mesmo que \textunderscore bebedeira\textunderscore .
(Cp. \textunderscore pórrio\textunderscore )
\section{Porreger}
\begin{itemize}
\item {Grp. gram.:v. t.}
\end{itemize}
\begin{itemize}
\item {Utilização:Ant.}
\end{itemize}
\begin{itemize}
\item {Proveniência:(Lat. \textunderscore porrigere\textunderscore )}
\end{itemize}
Offerecer, apresentar.
\section{Porreta}
\begin{itemize}
\item {fónica:rê}
\end{itemize}
\begin{itemize}
\item {Grp. gram.:f.}
\end{itemize}
\begin{itemize}
\item {Utilização:Prov.}
\end{itemize}
\begin{itemize}
\item {Utilização:Prov.}
\end{itemize}
\begin{itemize}
\item {Utilização:minh.}
\end{itemize}
Maço de ferro, marreta.
Talo verde das cebolas, alhos, etc.
\section{Porretada}
\begin{itemize}
\item {Grp. gram.:f.}
\end{itemize}
\begin{itemize}
\item {Utilização:Chul.}
\end{itemize}
Pancada com porrete.
\section{Porretas}
\begin{itemize}
\item {fónica:rê}
\end{itemize}
\begin{itemize}
\item {Grp. gram.:f. pl.}
\end{itemize}
\begin{itemize}
\item {Utilização:Des.}
\end{itemize}
Alhos porros.
\section{Porrete}
\begin{itemize}
\item {fónica:rê}
\end{itemize}
\begin{itemize}
\item {Grp. gram.:f.}
\end{itemize}
\begin{itemize}
\item {Utilização:Chul.}
\end{itemize}
\begin{itemize}
\item {Grp. gram.:Loc.}
\end{itemize}
\begin{itemize}
\item {Utilização:burl.}
\end{itemize}
Cacete, com uma das extremidades arredondada; moca.
\textunderscore De porrete á esquina\textunderscore , com azedume, com desconfiança.
\section{Porreto}
\begin{itemize}
\item {fónica:rê}
\end{itemize}
\begin{itemize}
\item {Grp. gram.:m.}
\end{itemize}
\begin{itemize}
\item {Utilização:Prov.}
\end{itemize}
\begin{itemize}
\item {Utilização:trasm.}
\end{itemize}
O mesmo que \textunderscore porrete\textunderscore .
\section{Porriginoso}
\begin{itemize}
\item {Grp. gram.:adj.}
\end{itemize}
\begin{itemize}
\item {Proveniência:(Lat. \textunderscore porriginosus\textunderscore )}
\end{itemize}
Que tem tinha; semelhante á tinha.
\section{Porrilhas}
\begin{itemize}
\item {Grp. gram.:f. pl.}
\end{itemize}
Doença de solípedes.
(Cast. \textunderscore porrilla\textunderscore )
\section{Porrina}
\begin{itemize}
\item {Grp. gram.:f.}
\end{itemize}
\begin{itemize}
\item {Utilização:Ant.}
\end{itemize}
Clava curta, de cabeça de ferro ou de madeira.
\section{Porrinho}
\begin{itemize}
\item {Grp. gram.:m.}
\end{itemize}
Espécie de moca ou clava, usada pelos Luinas e outros Negros da África. Cf. Serpa Pinto, II, 36.
\section{Pórrio}
\begin{itemize}
\item {Grp. gram.:m.}
\end{itemize}
\begin{itemize}
\item {Utilização:Bras}
\end{itemize}
Bebida, servida por um copo; copázio.
(Cp. \textunderscore porrão\textunderscore )
\section{Porro}
\begin{itemize}
\item {fónica:pô}
\end{itemize}
\begin{itemize}
\item {Grp. gram.:m.}
\end{itemize}
\begin{itemize}
\item {Proveniência:(Lat. \textunderscore porrum\textunderscore )}
\end{itemize}
Alho silvestre.
Espécie de callo, formado no lugar de uma fractura.
\section{Porsoleta}
\begin{itemize}
\item {fónica:lê}
\end{itemize}
\begin{itemize}
\item {Grp. gram.:f.}
\end{itemize}
(?):«\textunderscore ...de qualquer metal que sai querem fazer sua porsoleta...\textunderscore »Soropita, \textunderscore Poes. e Prosas\textunderscore , 104.
\section{Porta}
\begin{itemize}
\item {Grp. gram.:f.}
\end{itemize}
\begin{itemize}
\item {Utilização:Ext.}
\end{itemize}
\begin{itemize}
\item {Utilização:Fig.}
\end{itemize}
\begin{itemize}
\item {Grp. gram.:Loc.}
\end{itemize}
\begin{itemize}
\item {Utilização:Loc. de jôgo.}
\end{itemize}
\begin{itemize}
\item {Utilização:Escol.}
\end{itemize}
\begin{itemize}
\item {Utilização:Fig.}
\end{itemize}
\begin{itemize}
\item {Grp. gram.:Loc.}
\end{itemize}
\begin{itemize}
\item {Utilização:fam.}
\end{itemize}
\begin{itemize}
\item {Grp. gram.:Pl.}
\end{itemize}
\begin{itemize}
\item {Proveniência:(Lat. \textunderscore porta\textunderscore )}
\end{itemize}
Abertura em parede, ao nível do chão ou de um pavimento, para dar entrada ou saída.
Obra de madeira ou metal, para fechar a entrada de uma casa, de uma cidade, de uma sala, etc.
Peça de madeira ou metal, que, girando sôbre gonzos, serve para fechar qualquer coisa á maneira de porta: \textunderscore porta de um cofre\textunderscore ; \textunderscore porta de um armário\textunderscore .
Entrada.
Lugar, por onde se entra.
Meio de entrar; accesso.
Solução, expediente.
\textunderscore Jogar de porta\textunderscore , evitar o perder, á primeira cartada de monte.
Esperar á porta da aula, que o professor chame um alumno á lição, e entrar depois.
\textunderscore Pedir por portas\textunderscore , mendigar:«\textunderscore morria a pedir por portas.\textunderscore »Camillo, \textunderscore Viuva do Enforc.\textunderscore , III, 60.
\textunderscore Falar pela porta deanteira\textunderscore , falar rudemente, e sem rodeios, nem reticências.
Passagem apertada de um rio ou caminho, entre montes.
Desfiladeiro.
Portella.
\section{Porta}
\begin{itemize}
\item {Grp. gram.:adj. f.}
\end{itemize}
\begin{itemize}
\item {Utilização:Anat.}
\end{itemize}
\begin{itemize}
\item {Proveniência:(De \textunderscore porta\textunderscore ^1)}
\end{itemize}
Diz-se da veia grossa, que recebe o sangue do estômago, do baço, do pâncreas e dos intestinos, e que se distribue no fígado.
\section{Porta...}
\begin{itemize}
\item {Grp. gram.:pref.}
\end{itemize}
\begin{itemize}
\item {Proveniência:(De \textunderscore portar\textunderscore ^2)}
\end{itemize}
(designativo \textunderscore daquelle\textunderscore  ou \textunderscore daquillo que conduz, reforça\textunderscore  ou \textunderscore sustenta\textunderscore : \textunderscore porta-bandeira\textunderscore ; \textunderscore porta-voz\textunderscore .)
\section{Porta-aljava}
\begin{itemize}
\item {Grp. gram.:adj.}
\end{itemize}
O mesmo que \textunderscore pharetrado\textunderscore . Cf. V. de Seabra, \textunderscore Tristes\textunderscore , liv. IV, 26.
\section{Porta-bandeira}
\begin{itemize}
\item {Grp. gram.:m.}
\end{itemize}
Official, que leva a bandeira do regimento.
\section{Porta-baquetas}
\begin{itemize}
\item {Grp. gram.:m.}
\end{itemize}
Chapa de metal com dois cylindros, em que os tocadores de tambor enfiam as baquetas, quando não tocam.
\section{Porta-cabos}
\begin{itemize}
\item {Grp. gram.:m.}
\end{itemize}
Apparelho, para levar cabos de soccorro a náufragos.
\section{Porta-caixa}
\begin{itemize}
\item {Grp. gram.:m.}
\end{itemize}
Peça de madeira, em que se suspende a caixa dos teares de seda.
\section{Porta-cartas}
\begin{itemize}
\item {Grp. gram.:m.}
\end{itemize}
Bolsa de carteiro, em que se transportam cartas; carteira.
\section{Porta-cáustico}
\begin{itemize}
\item {Grp. gram.:m.}
\end{itemize}
\begin{itemize}
\item {Utilização:Med.}
\end{itemize}
Instrumento, com que os médicos introduzem um cáustico no canal da uretra.
\section{Porta-cautério}
\begin{itemize}
\item {Grp. gram.:m.}
\end{itemize}
O mesmo que \textunderscore porta-cáustico\textunderscore .
\section{Porta-chapéus}
\begin{itemize}
\item {Grp. gram.:m.}
\end{itemize}
Caixa leve de papelão ou madeira, apropriada para transporte de chapéus.
\section{Porta-clava}
\begin{itemize}
\item {Grp. gram.:adj.}
\end{itemize}
Que usa clava, (falando-se de Hércules). Cf. Filinto, VI, 212.
\section{Porta-clavina}
\begin{itemize}
\item {Grp. gram.:f.}
\end{itemize}
Peça de coiro, em que se mete a clavina.
\section{Porta-cocheira}
\begin{itemize}
\item {Grp. gram.:f.}
\end{itemize}
A porta mais larga de uma casa de habitação.
\section{Portada}
\begin{itemize}
\item {Grp. gram.:f.}
\end{itemize}
\begin{itemize}
\item {Proveniência:(De \textunderscore porta\textunderscore ^1)}
\end{itemize}
Grande porta, geralmente com ornatos.
Frontispício.
\section{Porta-da-rua}
\begin{itemize}
\item {Grp. gram.:f.}
\end{itemize}
\begin{itemize}
\item {Utilização:Bras}
\end{itemize}
Espécie de pimenta amarela.
\section{Portádigo}
\begin{itemize}
\item {Grp. gram.:m.}
\end{itemize}
\begin{itemize}
\item {Utilização:Ant.}
\end{itemize}
O mesmo que \textunderscore portagem\textunderscore .
\section{Portador}
\begin{itemize}
\item {Grp. gram.:m.  e  adj.}
\end{itemize}
\begin{itemize}
\item {Utilização:Bras}
\end{itemize}
\begin{itemize}
\item {Proveniência:(Do lat. \textunderscore portator\textunderscore )}
\end{itemize}
Aquelle que porta ou conduz.
Aquelle que, em nome de outrem, leva a qualquer destino uma carta, uma encommenda, etc.
Possuidor (de titulos ou documentos, que hão de sêr pagos a quem os apresente).
O mesmo que \textunderscore carregador\textunderscore .
\section{Porta-emendas}
\begin{itemize}
\item {Grp. gram.:f.}
\end{itemize}
Instrumento de madeira com cabo, e escavado no centro em ângulo, utilizado para conduzir á máquina as emendas, que há a fazer numa prova typográphica.
\section{Porta-escôvas}
\begin{itemize}
\item {Grp. gram.:m.}
\end{itemize}
Utensílio, em que se guardam escôvas.
\section{Porta-espada}
\begin{itemize}
\item {Grp. gram.:m.}
\end{itemize}
Peça, aparafusada na costella esquerda do sellim, para suspensão e segurança da espada, nos corpos de cavallaria.
\section{Porta-estandarte}
\begin{itemize}
\item {Grp. gram.:m.}
\end{itemize}
Aquelle que conduz o estandarte.
\section{Porta-fogo}
\begin{itemize}
\item {Grp. gram.:m.}
\end{itemize}
Aquelle que traz fogo para abrasar, (falando-se de Cupido). Cf. Filinto, X, 12.
\section{Porta-fólio}
\begin{itemize}
\item {Grp. gram.:m.}
\end{itemize}
\begin{itemize}
\item {Utilização:P. us.}
\end{itemize}
\begin{itemize}
\item {Proveniência:(Fr. \textunderscore porte-feuille\textunderscore )}
\end{itemize}
Pasta de cartão, em que se guardam papéis, desenhos, etc.
Carteira de algibeira.
\section{Porta-frasco}
\begin{itemize}
\item {Grp. gram.:m.}
\end{itemize}
Cordão, em que se pendura o polvorinho, quando se vai á caça.
\section{Portagão}
\begin{itemize}
\item {Grp. gram.:m.}
\end{itemize}
O mesmo que \textunderscore comporta\textunderscore ^1.
\section{Portageiro}
\begin{itemize}
\item {Grp. gram.:m.}
\end{itemize}
\begin{itemize}
\item {Proveniência:(De \textunderscore portagem\textunderscore )}
\end{itemize}
Cobrador dos direitos de portagem.
\section{Portagem}
\begin{itemize}
\item {Grp. gram.:f.}
\end{itemize}
\begin{itemize}
\item {Proveniência:(Do b. lat. \textunderscore portago\textunderscore , \textunderscore portaginis\textunderscore )}
\end{itemize}
Tributo, que se paga por se passar numa ponte, nas portas de uma cidade, etc., ou pelos carregamentos, com que se passa nos mesmos lugares.
Lugar em que se cobra esse tributo.
\section{Porta-guião}
\begin{itemize}
\item {Grp. gram.:m.}
\end{itemize}
O mesmo que \textunderscore porta-bandeira\textunderscore . Cf. Castilho, \textunderscore Fastos\textunderscore , II, 232.
\section{Porta-jóias}
\begin{itemize}
\item {Grp. gram.:m.}
\end{itemize}
Pequeno vaso ou caixinha, em que se guardam jóias; guarda jóias.
\section{Porta-júbilos}
\begin{itemize}
\item {Grp. gram.:m.}
\end{itemize}
\begin{itemize}
\item {Utilização:Des.}
\end{itemize}
Aquillo que causa ou infunde júbilos:«\textunderscore empina êste, que enramo, porta-júbilos na alma.\textunderscore »Filinto, IV, 75.
\section{Portal}
\begin{itemize}
\item {Grp. gram.:m.}
\end{itemize}
\begin{itemize}
\item {Utilização:Prov.}
\end{itemize}
\begin{itemize}
\item {Utilização:T. de Alcanena}
\end{itemize}
\begin{itemize}
\item {Proveniência:(De \textunderscore porta\textunderscore ^1)}
\end{itemize}
Porta principal de um edifício; portada.
Abertura em muro, sebe ou vallado, que se fecha com estacas ou cancella.
Ombreira de porta ou portão.
\section{Porta-laços}
\begin{itemize}
\item {Grp. gram.:m.}
\end{itemize}
\begin{itemize}
\item {Utilização:Med.}
\end{itemize}
Instrumento, com que os parteiros introduzem no útero cordões ou laços, para dar ao féto posição natural.
\section{Porta-lanterna}
\begin{itemize}
\item {Grp. gram.:m.}
\end{itemize}
Peça metállica, ligada á haste do garfo da roda deanteira, nas bicycletas.
\section{Porta-lápis}
\begin{itemize}
\item {Grp. gram.:m.}
\end{itemize}
O mesmo que \textunderscore lapiseira\textunderscore .
\section{Portalecer}
\begin{itemize}
\item {Grp. gram.:v. i.}
\end{itemize}
\begin{itemize}
\item {Utilização:Ant.}
\end{itemize}
\begin{itemize}
\item {Proveniência:(De \textunderscore portal\textunderscore )}
\end{itemize}
Apparecer ao portal.
Surgir no alto da portela.
Assomar.
\section{Porta-leque}
\begin{itemize}
\item {Grp. gram.:m.}
\end{itemize}
Utensílio, em que se traz ou se guarda o leque.
\section{Portaló}
\begin{itemize}
\item {Grp. gram.:m.}
\end{itemize}
Lugar, por onde se entra em um navio, ou por onde se tira ou recebe a carga.
(Talvez de \textunderscore portal\textunderscore )
\section{Porta-luz}
\begin{itemize}
\item {Grp. gram.:m.}
\end{itemize}
\begin{itemize}
\item {Utilização:Ext.}
\end{itemize}
Aquelle que leva luz á frente de outros.
Guia.
\section{Porta-maça}
\begin{itemize}
\item {Grp. gram.:m.}
\end{itemize}
O mesmo que \textunderscore maceiro\textunderscore .
\section{Porta-machado}
\begin{itemize}
\item {Grp. gram.:m.}
\end{itemize}
Soldado munido de machado, para trabalhos de sapa.
\section{Porta-manta}
\begin{itemize}
\item {Grp. gram.:m.}
\end{itemize}
Espécie de mala, em que se transporta a capa ou outras peças de vestuário, especialmente em jornada.
\section{Porta-marmita}
\begin{itemize}
\item {Grp. gram.:f.}
\end{itemize}
Caixa, em que se levam as marmitas do rancho, para os soldados que estão em serviço fóra do quartel.
\section{Porta-mechas}
\begin{itemize}
\item {Grp. gram.:m.}
\end{itemize}
Instrumento, com que os cirurgiões introduzem mechas nas chagas profundas.
\section{Portamento}
\begin{itemize}
\item {Grp. gram.:m.}
\end{itemize}
\begin{itemize}
\item {Utilização:Mús.}
\end{itemize}
\begin{itemize}
\item {Proveniência:(De \textunderscore portar\textunderscore ^2)}
\end{itemize}
Maneira de ligar muito os sons, arrastando-os quási e fazendo sentir, ao passar de uns para outros, uma infinidade inapreciavel de outros sons intermediários.
\section{Porta-mítra}
\begin{itemize}
\item {Grp. gram.:m.}
\end{itemize}
Eccleciástico que, em certas solennidades, leva nas mãos a mitra do Prelado.
\section{Porta-montes}
\begin{itemize}
\item {Grp. gram.:m.}
\end{itemize}
\begin{itemize}
\item {Utilização:Ant.}
\end{itemize}
O mesmo que \textunderscore monteiro\textunderscore :«\textunderscore ...huns coxins portamontes de couro...\textunderscore »Doc. do sec. XVI.
\section{Portanário}
\begin{itemize}
\item {Grp. gram.:adj.}
\end{itemize}
\begin{itemize}
\item {Utilização:Ant.}
\end{itemize}
Dizia-se de uma parte do intestino, adjacente ao pyloro. Cf. B. Pereira, \textunderscore Prosodia\textunderscore , vb. \textunderscore pilurus\textunderscore .
\section{Porta-novas}
\begin{itemize}
\item {Grp. gram.:m.}
\end{itemize}
Aquelle que traz e leva novidades.
Bisbilhoteiro; mexeriqueiro.
\section{Portante}
\begin{itemize}
\item {Grp. gram.:adj.}
\end{itemize}
\begin{itemize}
\item {Utilização:Ant.}
\end{itemize}
\begin{itemize}
\item {Proveniência:(Lat. \textunderscore portans\textunderscore )}
\end{itemize}
Que porta ou leva.
\section{Portanto}
\begin{itemize}
\item {Grp. gram.:adv.}
\end{itemize}
\begin{itemize}
\item {Proveniência:(De \textunderscore por\textunderscore  + \textunderscore tanto\textunderscore )}
\end{itemize}
Por conseguinte, em vista disso ou disto.
\section{Portão}
\begin{itemize}
\item {Grp. gram.:m.}
\end{itemize}
Porta grande; portada.
Porta da rua.
\section{Porta-objecto}
\begin{itemize}
\item {Grp. gram.:m.}
\end{itemize}
Parte do microscópio, em quo se colloca o objecto, que se quere analysar.
\section{Porta-página}
\begin{itemize}
\item {Grp. gram.:f.}
\end{itemize}
Pedaço de papel, que se coloca por baixo de página ou chapa typográphica, para evitar que destas caiam letras.
\section{Porta-paz}
\begin{itemize}
\item {Grp. gram.:m.}
\end{itemize}
Quadro com uma cruz, que se dá a beijar por occasião de certas Missas.
\section{Porta-pedra}
\begin{itemize}
\item {Grp. gram.:m.}
\end{itemize}
\begin{itemize}
\item {Utilização:Med.}
\end{itemize}
Pequeno instrumento, em que se fixa a pedra-infernal e outras substâncias, que servem para cauterizar.
\section{Porta-pennas}
\begin{itemize}
\item {Grp. gram.:m.}
\end{itemize}
Pequena haste, em que se segura a penna ou o aparo, para se escrever com firmeza; caneta.
\section{Porta-pêso}
\begin{itemize}
\item {Grp. gram.:m.}
\end{itemize}
\begin{itemize}
\item {Utilização:Phýs.}
\end{itemize}
Ferro em que se suspende o pêso que os ímans devem sustentar.
\section{Portar}
\begin{itemize}
\item {Grp. gram.:v. i.}
\end{itemize}
\begin{itemize}
\item {Proveniência:(De \textunderscore pôrto\textunderscore )}
\end{itemize}
O mesmo que \textunderscore aportar\textunderscore . Cf. \textunderscore Port. Mon. Hist., Script.\textunderscore , 316.
\section{Portar}
\begin{itemize}
\item {Grp. gram.:v. t.}
\end{itemize}
\begin{itemize}
\item {Grp. gram.:V. i.}
\end{itemize}
\begin{itemize}
\item {Utilização:Náut.}
\end{itemize}
\begin{itemize}
\item {Grp. gram.:V. p.}
\end{itemize}
\begin{itemize}
\item {Utilização:Gal}
\end{itemize}
\begin{itemize}
\item {Proveniência:(Lat. \textunderscore portare\textunderscore )}
\end{itemize}
Levar.
\textunderscore Portar por fé\textunderscore , certificar.
\textunderscore Portar pela amarra\textunderscore , puxar por ella o navio, aproando ao vento ou á maré.
Proceder, comportar-se.
Ir passando, relativamente á saúde:«\textunderscore Portar-me-ia bem\textunderscore , (passaria bem) \textunderscore se a cada instante não viessem remorsos...\textunderscore »Filinto, XXI, 206.«\textunderscore ...se seu irmão se portava sadio e bem...\textunderscore »Filinto, \textunderscore D. Man.\textunderscore , I, 164.
\section{Portar}
\begin{itemize}
\item {Grp. gram.:v. i.}
\end{itemize}
\begin{itemize}
\item {Utilização:Bras. do S}
\end{itemize}
Passar junto á porta de alguem: \textunderscore de caminho, portei em casa de meu primo\textunderscore .
\section{Porta-rêde}
\begin{itemize}
\item {Grp. gram.:m.}
\end{itemize}
\begin{itemize}
\item {Utilização:Pesc.}
\end{itemize}
Embarcação que conduz a rêde, em pescarias do alto mar.
\section{Porta-relógio}
\begin{itemize}
\item {Grp. gram.:m.}
\end{itemize}
Utensílio, em que se accomoda o relógio, quando se não traz na algibeira.
\section{Portaria}
\begin{itemize}
\item {Grp. gram.:f.}
\end{itemize}
\begin{itemize}
\item {Utilização:Ant.}
\end{itemize}
Porta principal do convento.
Átrio de convento.
Portão.
Diploma ou documento official, assignado por um ministro em nome do rei.
Cargo ou officio de porteiro. Cf. Herculano, \textunderscore Hist. de Port.\textunderscore , IV, 227.
\section{Porta-roda}
\begin{itemize}
\item {Grp. gram.:m.}
\end{itemize}
Apparelho, annexo ao antigo carro de munições, em artilharia.
\section{Porta-saco}
\begin{itemize}
\item {Grp. gram.:m.}
\end{itemize}
Aquelle que conduz saco.
\section{Porta-serra}
\begin{itemize}
\item {Grp. gram.:m.}
\end{itemize}
Família de insectos hymenópteros.
\section{Porta-sonda}
\begin{itemize}
\item {Grp. gram.:m.}
\end{itemize}
Apparelho cirúrgico, com que se introduz a sonda no conducto nasal, quando se faz a operação da fístula lacrimal.
\section{Porta-talas}
\begin{itemize}
\item {Grp. gram.:m.}
\end{itemize}
Pedaço de pano, em que os cirurgiões lavam as talas que servem para certas fracturas.
\section{Portático}
\begin{itemize}
\item {Grp. gram.:m.}
\end{itemize}
\begin{itemize}
\item {Utilização:Ant.}
\end{itemize}
O mesmo que \textunderscore portádigo\textunderscore .
\section{Portátil}
\begin{itemize}
\item {Grp. gram.:adj.}
\end{itemize}
\begin{itemize}
\item {Proveniência:(Lat. hyp. \textunderscore portatilis\textunderscore )}
\end{itemize}
Que facilmente póde sêr transportado.
Que tem pequeno volume ou pouco pêso, podendo por isso transportar-se facilmente: \textunderscore uma pharmácia portátil\textunderscore .
Que se desarma ou se separa em muitas peças, para mais facilmente se transportar: \textunderscore uma casa portátil\textunderscore .
\section{Porta-válvula}
\begin{itemize}
\item {Grp. gram.:m.}
\end{itemize}
Peça circular de cobre, que faz parte dos corpos de bomba e serve para sustentar as válvulas.
\section{Porta-vento}
\begin{itemize}
\item {Grp. gram.:m.}
\end{itemize}
\begin{itemize}
\item {Utilização:Mús.}
\end{itemize}
Canal que, nos órgãos, conduz o vento dos folles para os someiros.
\section{Porta-voz}
\begin{itemize}
\item {Grp. gram.:m.}
\end{itemize}
Instrumento, semelhante a uma trombeta, para reforçar a voz, falando-se por elle.
\section{Portazgo}
\begin{itemize}
\item {Grp. gram.:m.}
\end{itemize}
\begin{itemize}
\item {Utilização:Ant.}
\end{itemize}
O mesmo que \textunderscore portagem\textunderscore . Cf. Pant. de Aveiro, \textunderscore Itiner\textunderscore , 54 v.^o, (2.^a ed.).
(Cp. \textunderscore portádigo\textunderscore )
\section{Porte}
\begin{itemize}
\item {Grp. gram.:m.}
\end{itemize}
\begin{itemize}
\item {Utilização:Bras}
\end{itemize}
\begin{itemize}
\item {Proveniência:(De \textunderscore portar\textunderscore ^2)}
\end{itemize}
Acto de conduzir ou trazer.
Transporte.
Carga.
Preço de um transporte.
Procedimento, comportamento.
Aspecto phýsico.
Maneira, por que alguém se apresenta.
Tonelagem; capacidade.
Correia, com que se suspendem os taroles.
\section{Portear}
\begin{itemize}
\item {Grp. gram.:v. t.}
\end{itemize}
\begin{itemize}
\item {Proveniência:(De \textunderscore porte\textunderscore )}
\end{itemize}
Franquiar ou sellar devidamente (carta ou qualquer remessa postal).
\section{Porteio}
\begin{itemize}
\item {Grp. gram.:m.}
\end{itemize}
\begin{itemize}
\item {Utilização:Ant.}
\end{itemize}
O mesmo que \textunderscore pôrto\textunderscore .
\section{Porteira}
\begin{itemize}
\item {Grp. gram.:f.}
\end{itemize}
\begin{itemize}
\item {Utilização:Prov.}
\end{itemize}
\begin{itemize}
\item {Proveniência:(De \textunderscore porteiro\textunderscore )}
\end{itemize}
Mulhér, que guarda porta ou portaria.
Mulhér de porteiro.
Cancella.
Portão.
\section{Porteiro}
\begin{itemize}
\item {Grp. gram.:m.}
\end{itemize}
\begin{itemize}
\item {Utilização:Ant.}
\end{itemize}
\begin{itemize}
\item {Proveniência:(Do b. lat. \textunderscore portarius\textunderscore )}
\end{itemize}
Aquelle que guarda uma porta ou portaria.
Guarda-portão.
O que apregôa leilões judiciaes.
Cobrador de direitos reaes.
\section{Portela}
\begin{itemize}
\item {Grp. gram.:f.}
\end{itemize}
\begin{itemize}
\item {Utilização:Des.}
\end{itemize}
\begin{itemize}
\item {Utilização:Prov.}
\end{itemize}
\begin{itemize}
\item {Utilização:alent.}
\end{itemize}
\begin{itemize}
\item {Proveniência:(Lat. \textunderscore portella\textunderscore )}
\end{itemize}
Portal.
Pequena porta.
Ângulo ou cotovelo de uma estrada ou caminho.
Passagem estreita entre montes, desfiladeiro.
Depressão de um pontal.
\section{Portelama}
\begin{itemize}
\item {Grp. gram.:f.}
\end{itemize}
\begin{itemize}
\item {Utilização:Prov.}
\end{itemize}
\begin{itemize}
\item {Utilização:alg.}
\end{itemize}
\begin{itemize}
\item {Proveniência:(Do fr. \textunderscore porte-lame\textunderscore )}
\end{itemize}
Faca, com que se fazem rolhas de cortiça.
\section{Portella}
\begin{itemize}
\item {Grp. gram.:f.}
\end{itemize}
\begin{itemize}
\item {Utilização:Des.}
\end{itemize}
\begin{itemize}
\item {Utilização:Prov.}
\end{itemize}
\begin{itemize}
\item {Utilização:alent.}
\end{itemize}
\begin{itemize}
\item {Proveniência:(Lat. \textunderscore portella\textunderscore )}
\end{itemize}
Portal.
Pequena porta.
Ângulo ou cotovelo de uma estrada ou caminho.
Passagem estreita entre montes, desfiladeiro.
Depressão de um pontal.
\section{Portello}
\begin{itemize}
\item {fónica:tê}
\end{itemize}
\begin{itemize}
\item {Grp. gram.:m.}
\end{itemize}
\begin{itemize}
\item {Utilização:Prov.}
\end{itemize}
Porta de um cercado ou terreno murado.
Cancella.
Pequena portella ou pequeno desfiladeiro.
(Cp. \textunderscore portella\textunderscore )
\section{Portelo}
\begin{itemize}
\item {fónica:tê}
\end{itemize}
\begin{itemize}
\item {Grp. gram.:m.}
\end{itemize}
\begin{itemize}
\item {Utilização:Prov.}
\end{itemize}
Porta de um cercado ou terreno murado.
Cancela.
Pequena portela ou pequeno desfiladeiro.
(Cp. \textunderscore portela\textunderscore )
\section{Portenho}
\begin{itemize}
\item {Grp. gram.:adj.}
\end{itemize}
\begin{itemize}
\item {Utilização:bras}
\end{itemize}
\begin{itemize}
\item {Utilização:Neol.}
\end{itemize}
Relativo a Buenos-Aires. Cf. \textunderscore Notícia\textunderscore , do Rio, de 10-X-900.
\section{Portento}
\begin{itemize}
\item {Grp. gram.:m.}
\end{itemize}
\begin{itemize}
\item {Proveniência:(Lat. \textunderscore portentum\textunderscore )}
\end{itemize}
Coisa ou successo prodigioso.
Maravilha; coisa rara.
\section{Portentosamente}
\begin{itemize}
\item {Grp. gram.:adv.}
\end{itemize}
De modo portentoso.
Assombrosamente.
\section{Portentoso}
\begin{itemize}
\item {Grp. gram.:adj.}
\end{itemize}
\begin{itemize}
\item {Proveniência:(Lat. \textunderscore portentosus\textunderscore )}
\end{itemize}
Em que há portento; prodigioso, maravilhoso.
Insólito, raro.
\section{Pórtico}
\begin{itemize}
\item {Grp. gram.:m.}
\end{itemize}
\begin{itemize}
\item {Proveniência:(Lat. \textunderscore porticus\textunderscore )}
\end{itemize}
Átrio amplo, com o tecto sustentado por columnas ou pilares.
Portada.
Entrada de edifício nobre.
\section{Portilha}
\begin{itemize}
\item {Grp. gram.:f.}
\end{itemize}
\begin{itemize}
\item {Proveniência:(De \textunderscore porta\textunderscore ^1)}
\end{itemize}
Seteira.
\section{Portilhão}
\begin{itemize}
\item {Grp. gram.:m.}
\end{itemize}
\begin{itemize}
\item {Utilização:Des.}
\end{itemize}
\begin{itemize}
\item {Proveniência:(De \textunderscore portilha\textunderscore )}
\end{itemize}
Grande abertura em parede.
\section{Portilho}
\begin{itemize}
\item {Grp. gram.:m.}
\end{itemize}
\begin{itemize}
\item {Proveniência:(Do b. lat. \textunderscore porticulum\textunderscore )}
\end{itemize}
Pequeno pôrto.
\section{Portinha}
\begin{itemize}
\item {Grp. gram.:f.}
\end{itemize}
Pequena porta.
\section{Portinhola}
\begin{itemize}
\item {Grp. gram.:f.}
\end{itemize}
\begin{itemize}
\item {Utilização:Prov.}
\end{itemize}
\begin{itemize}
\item {Proveniência:(De \textunderscore portinha\textunderscore )}
\end{itemize}
Pequena porta de carruagem.
Tira de pano, para resguardar a abertura da algibeira.
Braguilha.
Peça que tapa as canhoneiras dos navios.
Peça chata de madeira, que se estende no lagar sôbre o bagaço, quando êste se deve espremer e sôbre a qual se assentam os malhaes, que aguentam a vara. (Colhido na Bairrada)
\section{Portinola}
\begin{itemize}
\item {Grp. gram.:f.}
\end{itemize}
\begin{itemize}
\item {Utilização:Prov.}
\end{itemize}
\begin{itemize}
\item {Utilização:minh.}
\end{itemize}
O mesmo que \textunderscore portinhola\textunderscore .
Alçapão de tonel.
\section{Pôrto}
\begin{itemize}
\item {Grp. gram.:m.}
\end{itemize}
\begin{itemize}
\item {Utilização:Fig.}
\end{itemize}
\begin{itemize}
\item {Utilização:Prov.}
\end{itemize}
\begin{itemize}
\item {Proveniência:(Lat. \textunderscore portus\textunderscore )}
\end{itemize}
Lugar sôbre uma costa de mar ou junto á foz de um rio, para abrigo e ancoradoiro de navios.
Lugar, onde se embarca ou desembarca.
Lugar de refúgio ou de descanso.
Abertura na vedação de uma propriedade. (Colhido em Turquel)
\section{Pôrto}
\begin{itemize}
\item {Grp. gram.:m.}
\end{itemize}
Vinho, fabricado no Pôrto.
\section{Porto-franco}
\begin{itemize}
\item {Grp. gram.:m.}
\end{itemize}
Porto de livre entrada para todos os gêneros, sem pagamento de direitos.
\section{Portolano}
\begin{itemize}
\item {Grp. gram.:m.}
\end{itemize}
\begin{itemize}
\item {Proveniência:(Fr. \textunderscore portulan\textunderscore )}
\end{itemize}
Livro, em que se descrevem portos de mar, sua profundidade, marés, maneira de nelles entrar ou sair, etc. Cf. Camillo. \textunderscore Narcót.\textunderscore , I, 12.
\section{Portorriquenho}
\begin{itemize}
\item {Grp. gram.:m.  e  adj.}
\end{itemize}
O que é de Porto-Rico.
\section{Portucalense}
\begin{itemize}
\item {Grp. gram.:adj.}
\end{itemize}
O mesmo que \textunderscore portugalense\textunderscore . Cf. Herculano, \textunderscore Hist. de Port.\textunderscore , (passim).
\section{Portucha}
\begin{itemize}
\item {Grp. gram.:f.}
\end{itemize}
\begin{itemize}
\item {Utilização:Náut.}
\end{itemize}
\begin{itemize}
\item {Proveniência:(Do b. lat. \textunderscore portucula\textunderscore )}
\end{itemize}
Cada uma das ilhós, por onde se enfiam os rizes num navio.
\section{Portuchar}
\begin{itemize}
\item {Grp. gram.:v.}
\end{itemize}
\begin{itemize}
\item {Utilização:t. Náut.}
\end{itemize}
\begin{itemize}
\item {Proveniência:(De \textunderscore portucha\textunderscore )}
\end{itemize}
Encolher ou enrizar (uma vela de navio).
\section{Portuchos}
\begin{itemize}
\item {Grp. gram.:m. pl.}
\end{itemize}
Os orifícios da fieira dos ourives.
(Corr. de \textunderscore pertuchos\textunderscore )
\section{Portuense}
\begin{itemize}
\item {Grp. gram.:adj.}
\end{itemize}
\begin{itemize}
\item {Grp. gram.:M.}
\end{itemize}
Relativo á cidade do Pôrto.
Habitante do Pôrto.
\section{Portuga}
\begin{itemize}
\item {Grp. gram.:m.}
\end{itemize}
\begin{itemize}
\item {Utilização:Bras}
\end{itemize}
Português. Cf. João Ribeiro, \textunderscore Gram.\textunderscore 
\section{Portugalense}
\begin{itemize}
\item {Grp. gram.:m.  e  adj.}
\end{itemize}
\begin{itemize}
\item {Utilização:Ant.}
\end{itemize}
O mesmo que \textunderscore português\textunderscore . Cf. Herculano, \textunderscore Hist. de. Port.\textunderscore , I, 193, 194, 195 e 200.
\section{Portugalês}
\begin{itemize}
\item {Grp. gram.:m.  e  adj.}
\end{itemize}
\begin{itemize}
\item {Utilização:Ant.}
\end{itemize}
O mesmo que \textunderscore português\textunderscore .
\section{Portugal-velho}
\begin{itemize}
\item {Grp. gram.:m.}
\end{itemize}
Homem honestíssimo, franco, leal.
\section{Português}
\begin{itemize}
\item {Grp. gram.:adj.}
\end{itemize}
\begin{itemize}
\item {Utilização:Fig.}
\end{itemize}
\begin{itemize}
\item {Grp. gram.:M.}
\end{itemize}
\begin{itemize}
\item {Proveniência:(De \textunderscore Portugal\textunderscore  &gt; \textunderscore portugalense\textunderscore  &gt; \textunderscore porlugalens\textunderscore  &gt; \textunderscore portugalês\textunderscore  &gt; \textunderscore portugaês\textunderscore  &gt; \textunderscore português\textunderscore .)}
\end{itemize}
Relativo a Portugal.
Diz-se, de uma variedade de trigo molle.
Franco, apesar de rude.
Habitante de Portugal.
Língua, falada pelos Portugueses.
Antiga moéda de oiro.
\section{Portuguesa}
\begin{itemize}
\item {Grp. gram.:f.}
\end{itemize}
\begin{itemize}
\item {Utilização:Náut.}
\end{itemize}
Nó, ou amarração, feita de um cabo, para segurar as antennas da cabrilha.
Antiga moéda de oiro no Brasil.
Hymno nacional da República Portuguesa.
(Fem. de \textunderscore português\textunderscore )
\section{Portuguesar}
\textunderscore v. t.\textunderscore  (e der.)
O mesmo que \textunderscore aportuguesar\textunderscore , etc. Cf. Filinto, III, 209.
\section{Portuguesismo}
\begin{itemize}
\item {Grp. gram.:m.}
\end{itemize}
\begin{itemize}
\item {Proveniência:(De \textunderscore português\textunderscore )}
\end{itemize}
Locução ou idiotismo, peculiar á lingua portuguesa.
Modo de pensar ou sentir, próprio de portugueses.
\section{Portuguêsmente}
\begin{itemize}
\item {Grp. gram.:adv.}
\end{itemize}
\begin{itemize}
\item {Proveniência:(De português, ant. adj. uniforme, isto é, \textunderscore m.\textunderscore  e \textunderscore f.\textunderscore )}
\end{itemize}
Á maneira dos Portugueses.
Como se usa em Portugal.
\section{Portulaca}
\begin{itemize}
\item {Grp. gram.:f. pl.}
\end{itemize}
\begin{itemize}
\item {Proveniência:(Lat. \textunderscore portulaca\textunderscore )}
\end{itemize}
Nome scientífico da beldroéga.
\section{Portuláceas}
\begin{itemize}
\item {Grp. gram.:f. pl.}
\end{itemize}
Família de plantas, que tem por typo a beldroéga.
(Fem. pl. de \textunderscore portuláceo\textunderscore )
\section{Portuláceo}
\begin{itemize}
\item {Grp. gram.:adj.}
\end{itemize}
\begin{itemize}
\item {Proveniência:(De \textunderscore portulaca\textunderscore )}
\end{itemize}
Relativo ou semelhante á beldroéga.
\section{Portulano}
\begin{itemize}
\item {Grp. gram.:m.}
\end{itemize}
\begin{itemize}
\item {Proveniência:(Fr. \textunderscore portulan\textunderscore )}
\end{itemize}
Livro, em que se descrevem portos de mar, sua profundidade, marés, maneira de nelles entrar ou sair, etc. Cf. Camillo. Narcót., I, 12.
\section{Portunaes}
\begin{itemize}
\item {Grp. gram.:f. pl.}
\end{itemize}
\begin{itemize}
\item {Proveniência:(Lat. \textunderscore portunalia\textunderscore )}
\end{itemize}
Antigas festas romanas em honra de Portuno, deus dos portos.
\section{Portunais}
\begin{itemize}
\item {Grp. gram.:f. pl.}
\end{itemize}
\begin{itemize}
\item {Proveniência:(Lat. \textunderscore portunalia\textunderscore )}
\end{itemize}
Antigas festas romanas em honra de Portuno, deus dos portos.
\section{Portuoso}
\begin{itemize}
\item {Grp. gram.:adj.}
\end{itemize}
\begin{itemize}
\item {Proveniência:(Lat. \textunderscore portuosus\textunderscore )}
\end{itemize}
Que tem muitos portos.
\section{Poruca}
\begin{itemize}
\item {Grp. gram.:f.}
\end{itemize}
\begin{itemize}
\item {Utilização:Bras}
\end{itemize}
Peneira, com que se escolhe o café em grão.
\section{Porunga}
\begin{itemize}
\item {Grp. gram.:f.}
\end{itemize}
\begin{itemize}
\item {Utilização:Bras}
\end{itemize}
Vaso de coiro, espécie de borracha.
\section{Porventura}
\begin{itemize}
\item {Grp. gram.:loc. adv.}
\end{itemize}
\begin{itemize}
\item {Proveniência:(De \textunderscore por\textunderscore  + \textunderscore ventura\textunderscore )}
\end{itemize}
Por acaso.--Algumas vezes, é expressão expletiva.
\section{Porvindoiro}
\begin{itemize}
\item {Grp. gram.:adj.}
\end{itemize}
\begin{itemize}
\item {Grp. gram.:M. pl.}
\end{itemize}
\begin{itemize}
\item {Proveniência:(De \textunderscore por\textunderscore  + \textunderscore vindoiro\textunderscore )}
\end{itemize}
Que há de vir; futuro.
Vindoiros; pósteros.
\section{Porvindouro}
\begin{itemize}
\item {Grp. gram.:adj.}
\end{itemize}
\begin{itemize}
\item {Grp. gram.:M. pl.}
\end{itemize}
\begin{itemize}
\item {Proveniência:(De \textunderscore por\textunderscore  + \textunderscore vindoiro\textunderscore )}
\end{itemize}
Que há de vir; futuro.
Vindoiros; pósteros.
\section{Porvir}
\begin{itemize}
\item {Grp. gram.:m.}
\end{itemize}
\begin{itemize}
\item {Proveniência:(De \textunderscore por\textunderscore  + \textunderscore vir\textunderscore )}
\end{itemize}
Tempo que há de vir; futuro.
\section{Pós}
\begin{itemize}
\item {Grp. gram.:prep.}
\end{itemize}
\begin{itemize}
\item {Proveniência:(Do lat. \textunderscore post\textunderscore .)}
\end{itemize}
O mesmo que \textunderscore após\textunderscore .
\section{Pos...}
\begin{itemize}
\item {Grp. gram.:pref.}
\end{itemize}
\begin{itemize}
\item {Proveniência:(Lat. \textunderscore post\textunderscore )}
\end{itemize}
(que significa \textunderscore depois\textunderscore )
\section{Posalosa}
\begin{itemize}
\item {Grp. gram.:f.}
\end{itemize}
\begin{itemize}
\item {Utilização:Des.}
\end{itemize}
O mesmo que \textunderscore borboleta\textunderscore .
\section{Posboca}
\begin{itemize}
\item {Grp. gram.:f.}
\end{itemize}
\begin{itemize}
\item {Proveniência:(Do lat. \textunderscore post\textunderscore  + \textunderscore bucca\textunderscore )}
\end{itemize}
Parte posterior da boca, onde toca uma substância alimentícia, no acto da deglutição.
\section{Posca}
\begin{itemize}
\item {Grp. gram.:f.}
\end{itemize}
\begin{itemize}
\item {Proveniência:(Lat. \textunderscore posca\textunderscore )}
\end{itemize}
Bebida ácida, composta de vinagre e água, e usada na antiga milicia romana.
\section{Poscefálico}
\begin{itemize}
\item {Grp. gram.:adj.}
\end{itemize}
Relativo ao poscéfalo.
\section{Poscéfalo}
\begin{itemize}
\item {Grp. gram.:m.}
\end{itemize}
\begin{itemize}
\item {Proveniência:(Do lat. \textunderscore post\textunderscore  + gr. \textunderscore kephale\textunderscore )}
\end{itemize}
A parte posterior da cabeça.
\section{Poscênio}
\begin{itemize}
\item {Grp. gram.:m.}
\end{itemize}
\begin{itemize}
\item {Proveniência:(Do lat. \textunderscore postscenium\textunderscore )}
\end{itemize}
Parte do theatro, que fica atrás do palco; bastidores.
\section{Poscephálico}
\begin{itemize}
\item {Grp. gram.:adj.}
\end{itemize}
Relativo ao poscéphalo.
\section{Poscéphalo}
\begin{itemize}
\item {Grp. gram.:m.}
\end{itemize}
\begin{itemize}
\item {Proveniência:(Do lat. \textunderscore post\textunderscore  + gr. \textunderscore kephale\textunderscore )}
\end{itemize}
A parte posterior da cabeça.
\section{Posdata}
\begin{itemize}
\item {Grp. gram.:f.}
\end{itemize}
\begin{itemize}
\item {Proveniência:(De \textunderscore pos...\textunderscore  + \textunderscore data\textunderscore )}
\end{itemize}
Data de um documento, feita posteriormente á redacção dêste.
Data falsa, posterior á verdadeira.
\section{Posdatar}
\begin{itemize}
\item {Grp. gram.:v. t.}
\end{itemize}
Pôr posdata em.
\section{Posdiluviano}
\begin{itemize}
\item {Grp. gram.:adj.}
\end{itemize}
\begin{itemize}
\item {Proveniência:(De \textunderscore pos...\textunderscore  + \textunderscore diluviano\textunderscore )}
\end{itemize}
Posterior ao dilúvio.
\section{Posdorsal}
\begin{itemize}
\item {Grp. gram.:adj.}
\end{itemize}
\begin{itemize}
\item {Utilização:Anat.}
\end{itemize}
\begin{itemize}
\item {Proveniência:(De \textunderscore pos...\textunderscore  + \textunderscore dorsal\textunderscore )}
\end{itemize}
Situado atrás das costas.
\section{Posescrito}
\begin{itemize}
\item {Grp. gram.:adj.}
\end{itemize}
\begin{itemize}
\item {Grp. gram.:M.}
\end{itemize}
Escrito depois; escrito no fim.
Aquillo que se escreve no fim de uma carta, depois de assinada.
(Da loc. lat. \textunderscore post-scriptum\textunderscore )
\section{Posfácio}
\begin{itemize}
\item {Grp. gram.:m.}
\end{itemize}
Advertência, collocada no fim de um livro.
(Termo, mal formado, por analogia com \textunderscore prefácio\textunderscore )
\section{Posglacial}
\begin{itemize}
\item {Grp. gram.:adj.}
\end{itemize}
\begin{itemize}
\item {Utilização:Geol.}
\end{itemize}
\begin{itemize}
\item {Proveniência:(De \textunderscore pos...\textunderscore  + \textunderscore glacial\textunderscore )}
\end{itemize}
Diz-se de uma das cinco phases, que constituem o período plistoceno.
\section{Poshomérico}
\begin{itemize}
\item {Grp. gram.:adj.}
\end{itemize}
\begin{itemize}
\item {Proveniência:(De \textunderscore pos...\textunderscore  + \textunderscore homérico\textunderscore )}
\end{itemize}
Diz-se dos poétas gregos, posteriores a Homero, e das suas obras.
\section{Posição}
\begin{itemize}
\item {Grp. gram.:f.}
\end{itemize}
\begin{itemize}
\item {Proveniência:(Lat. \textunderscore positio\textunderscore )}
\end{itemize}
Lugar, onde uma pessôa ou coisa está collocada.
Modo de têr o corpo, postura.
Disposição.
Circunstâncias.
Classe.
Emprêgo público.
Situação social, moral, económica, etc.: \textunderscore homens da sua posição não se devem rebaixar assim\textunderscore .
Terreno, mais ou menos apropriado para ataque ou defesa militar.
\section{Posicionar}
\begin{itemize}
\item {Grp. gram.:v. t.}
\end{itemize}
Pôr em posição. Cf. \textunderscore Techn. Rur.\textunderscore , 285.
\section{Positivamente}
\begin{itemize}
\item {Grp. gram.:adv.}
\end{itemize}
De modo positivo.
Claramente; seguramente; terminantemente; affirmativamente.
\section{Positividade}
\begin{itemize}
\item {Grp. gram.:f.}
\end{itemize}
\begin{itemize}
\item {Utilização:Phýs.}
\end{itemize}
Estado do que é positivo.
Condição dos corpos, que apresentam os phenómenos da electricidade positiva.
\section{Positivismo}
\begin{itemize}
\item {Grp. gram.:m.}
\end{itemize}
\begin{itemize}
\item {Proveniência:(De \textunderscore positivo\textunderscore )}
\end{itemize}
Systema philosóphico, que se baseia nos factos e na experiência, e que deriva do conjunto das sciências positivas.
Modo de encarar a vida pelo lado prático.
A vida prática.
\section{Positivista}
\begin{itemize}
\item {Grp. gram.:adj.}
\end{itemize}
\begin{itemize}
\item {Grp. gram.:M.  e  f.}
\end{itemize}
\begin{itemize}
\item {Proveniência:(De \textunderscore positivo\textunderscore )}
\end{itemize}
Relativo a positivismo.
Pessôa, que segue a philosophia positivista.
\section{Positivo}
\begin{itemize}
\item {Grp. gram.:adj.}
\end{itemize}
\begin{itemize}
\item {Utilização:Phýs.}
\end{itemize}
\begin{itemize}
\item {Utilização:Gram.}
\end{itemize}
\begin{itemize}
\item {Utilização:Mathem.}
\end{itemize}
\begin{itemize}
\item {Grp. gram.:M.}
\end{itemize}
\begin{itemize}
\item {Utilização:Mús.}
\end{itemize}
\begin{itemize}
\item {Proveniência:(Lat. \textunderscore positivus\textunderscore )}
\end{itemize}
Real.
Evidente.
Indiscutível.
Que se baseia nos factos e na experiência.
Que tem carácter prático, experimental.
Que deriva da vontade e da natureza.
Diz-se de um dos dois fluidos, com que se explicam os phenómenos eléctricos.
Diz-se dos adjectivos, que se podem elevar a comparativos e a superlativos.
Diz-se das quantidades algébricas, que pódem sêr precedidas do sinal +.
Aquillo que é certo, claro, real, útil.
Parte do órgão, que contém os registos flautados.
\section{Poslúdio}
\begin{itemize}
\item {Grp. gram.:m.}
\end{itemize}
\begin{itemize}
\item {Utilização:Mús.}
\end{itemize}
Trecho, que se executa depois de uma acção ou ceremónia; o contrário de \textunderscore prelúdio\textunderscore .
\section{Posmeridiano}
\begin{itemize}
\item {Grp. gram.:adj.}
\end{itemize}
\begin{itemize}
\item {Proveniência:(De \textunderscore pos...\textunderscore  + \textunderscore meridiano\textunderscore )}
\end{itemize}
Posterior ao meio-dia; que succede depois do meio-dia.
\section{Posologia}
\begin{itemize}
\item {Grp. gram.:f.}
\end{itemize}
\begin{itemize}
\item {Proveniência:(Do gr. \textunderscore posos\textunderscore  + \textunderscore logos\textunderscore )}
\end{itemize}
Indicação das doses, em que se devem applicar os medicamentos.
\section{Pospasto}
\begin{itemize}
\item {Grp. gram.:m.}
\end{itemize}
\begin{itemize}
\item {Proveniência:(De \textunderscore pos...\textunderscore  + \textunderscore pasto\textunderscore )}
\end{itemize}
O mesmo que \textunderscore sobremesa\textunderscore .
\section{Pospelo}
\begin{itemize}
\item {fónica:pê}
\end{itemize}
\begin{itemize}
\item {Grp. gram.:m.}
\end{itemize}
\begin{itemize}
\item {Utilização:Fig.}
\end{itemize}
\begin{itemize}
\item {Proveniência:(De \textunderscore pos...\textunderscore  + \textunderscore pelo\textunderscore )}
\end{itemize}
Direcção contrária á do pêlo.
Violência.
\section{Posperna}
\begin{itemize}
\item {Grp. gram.:f.}
\end{itemize}
\begin{itemize}
\item {Proveniência:(De \textunderscore pos...\textunderscore  + \textunderscore perna\textunderscore )}
\end{itemize}
Parte superior da perna da bêsta.
\section{Pospliocênio}
\begin{itemize}
\item {Grp. gram.:adj.}
\end{itemize}
\begin{itemize}
\item {Utilização:Geol.}
\end{itemize}
\begin{itemize}
\item {Proveniência:(De \textunderscore pos...\textunderscore  + \textunderscore plioceno\textunderscore )}
\end{itemize}
Diz-se do período posterior ao plioceno.
\section{Pospoêr}
\begin{itemize}
\item {Grp. gram.:v. t.}
\end{itemize}
(Fórma ant. de \textunderscore pospor\textunderscore . Cf. Frei Fortun., \textunderscore Inéd.\textunderscore , 312)
\section{Posponto}
\begin{itemize}
\item {Grp. gram.:m.}
\end{itemize}
(V. \textunderscore pesponto\textunderscore , etc.)
\section{Pospor}
\begin{itemize}
\item {Grp. gram.:v. t.}
\end{itemize}
\begin{itemize}
\item {Proveniência:(Do lat. \textunderscore postponere\textunderscore )}
\end{itemize}
Pôr depois; preterir; postergar.
Não fazer caso de.
Adiar.
\section{Posposição}
\begin{itemize}
\item {Grp. gram.:f.}
\end{itemize}
\begin{itemize}
\item {Proveniência:(De \textunderscore pos...\textunderscore  + \textunderscore posição\textunderscore )}
\end{itemize}
Acto ou effeito de pospor.
\section{Pospositivo}
\begin{itemize}
\item {Grp. gram.:adj.}
\end{itemize}
\begin{itemize}
\item {Utilização:Gram.}
\end{itemize}
\begin{itemize}
\item {Proveniência:(Lat. \textunderscore postpositivus\textunderscore )}
\end{itemize}
Que é posposto.
Diz-se especialmente das palavras, que se não empregam no princípio da phrase.
Diz-se das particulas chamadas suffixos.
\section{Posposto}
\begin{itemize}
\item {Grp. gram.:adj.}
\end{itemize}
\begin{itemize}
\item {Proveniência:(Do lat. \textunderscore postpositus\textunderscore )}
\end{itemize}
Omittido; preterido; postergado; desprezado.
\section{Pospuerperal}
\begin{itemize}
\item {Grp. gram.:adj.}
\end{itemize}
\begin{itemize}
\item {Proveniência:(De \textunderscore pos...\textunderscore  + \textunderscore puerperal\textunderscore )}
\end{itemize}
Posterior ao parto.
\section{Posquete}
\begin{itemize}
\item {fónica:quê}
\end{itemize}
\begin{itemize}
\item {Grp. gram.:m.}
\end{itemize}
O mesmo que \textunderscore enora\textunderscore .
\section{Posromano}
\begin{itemize}
\item {Grp. gram.:adj.}
\end{itemize}
\begin{itemize}
\item {Proveniência:(De \textunderscore pos...\textunderscore  + \textunderscore romano\textunderscore )}
\end{itemize}
Posterior ao Império Romano ou á dominação dos Romanos.
\section{Possança}
\begin{itemize}
\item {Grp. gram.:f.}
\end{itemize}
\begin{itemize}
\item {Utilização:Geol.}
\end{itemize}
\begin{itemize}
\item {Proveniência:(De \textunderscore possante\textunderscore )}
\end{itemize}
Poder, valentia, vigor.
Espessura de um estrato zoológico, medida pela perpendicular ao plano da estratificação.
\section{Possanga}
\begin{itemize}
\item {Grp. gram.:f.}
\end{itemize}
\begin{itemize}
\item {Utilização:Bras}
\end{itemize}
\begin{itemize}
\item {Proveniência:(T. tupi)}
\end{itemize}
Medicamento caseiro, mèzinha.
\section{Possante}
\begin{itemize}
\item {Grp. gram.:adj.}
\end{itemize}
\begin{itemize}
\item {Proveniência:(De \textunderscore possar\textunderscore )}
\end{itemize}
Que tem possança.
Vigoroso.
Majestoso.
\section{Possar}
\begin{itemize}
\item {Grp. gram.:v. i.}
\end{itemize}
\begin{itemize}
\item {Utilização:ant.}
\end{itemize}
\begin{itemize}
\item {Utilização:Pop.}
\end{itemize}
O mesmo que \textunderscore poder\textunderscore .
Entrar na posse.
\section{Posse}
\begin{itemize}
\item {Grp. gram.:f.}
\end{itemize}
\begin{itemize}
\item {Grp. gram.:Pl.}
\end{itemize}
\begin{itemize}
\item {Proveniência:(Do lat. \textunderscore posse\textunderscore )}
\end{itemize}
Retenção ou fruição de uma coisa ou direito.
Estado de quem frue uma coisa ou a tem em seu poder.
Haveres, meios de vida.
Aptidão, capacidade.
\section{Posseír}
\begin{itemize}
\item {Grp. gram.:v. t.}
\end{itemize}
\begin{itemize}
\item {Utilização:Ant.}
\end{itemize}
O mesmo que \textunderscore possuir\textunderscore :«\textunderscore ó lingoa mortal, posseote e ofendesme.\textunderscore »Usque, 49, v.^o.
\section{Posseiro}
\begin{itemize}
\item {Grp. gram.:m.  e  adj.}
\end{itemize}
\begin{itemize}
\item {Utilização:Jur.}
\end{itemize}
\begin{itemize}
\item {Proveniência:(De \textunderscore posse\textunderscore )}
\end{itemize}
Quinhoeiro, que está na posse legal de prédio ou prédios indivisos.
\section{Possessão}
\begin{itemize}
\item {Grp. gram.:f.}
\end{itemize}
\begin{itemize}
\item {Proveniência:(Lat. \textunderscore possessio\textunderscore )}
\end{itemize}
Estado; domínio: \textunderscore as nossas possessões ultramarinas\textunderscore .
Colónia.
Estado de quem é possesso.
\section{Possessivo}
\begin{itemize}
\item {Grp. gram.:adj.}
\end{itemize}
\begin{itemize}
\item {Utilização:Gram.}
\end{itemize}
\begin{itemize}
\item {Proveniência:(Lat. \textunderscore possessivus\textunderscore )}
\end{itemize}
Que designa posse: \textunderscore pronomes possessivos\textunderscore .
\section{Possesso}
\begin{itemize}
\item {Grp. gram.:adj.}
\end{itemize}
\begin{itemize}
\item {Proveniência:(Lat. \textunderscore possessus\textunderscore )}
\end{itemize}
Possuído do demónio; endemoninhado.
\section{Possessor}
\begin{itemize}
\item {Grp. gram.:adj.}
\end{itemize}
\begin{itemize}
\item {Grp. gram.:M.}
\end{itemize}
\begin{itemize}
\item {Proveniência:(Lat. \textunderscore possessor\textunderscore )}
\end{itemize}
Que possue.
Aquelle que possue.
Entre os Romanos, cada um dos indivíduos ou colonos, pelos quaes se repartiam terras conquistadas. Cf. Herculano, \textunderscore Hist. de Port.\textunderscore , III, 318.
\section{Possessório}
\begin{itemize}
\item {Grp. gram.:adj.}
\end{itemize}
\begin{itemize}
\item {Utilização:Jur.}
\end{itemize}
\begin{itemize}
\item {Grp. gram.:M.}
\end{itemize}
\begin{itemize}
\item {Utilização:Jur.}
\end{itemize}
\begin{itemize}
\item {Proveniência:(Lat. \textunderscore possessorius\textunderscore )}
\end{itemize}
Relativo ou inherente á posse.
Diz-se do juízo, onde se movem as acções de posse.
O juiz possessório.
\section{Possíbil}
\begin{itemize}
\item {Grp. gram.:adj.}
\end{itemize}
\begin{itemize}
\item {Utilização:Ant.}
\end{itemize}
O mesmo que \textunderscore possível\textunderscore . Cf. \textunderscore Filodemo\textunderscore , I, 5.
\section{Possibilidade}
\begin{itemize}
\item {Grp. gram.:f.}
\end{itemize}
\begin{itemize}
\item {Grp. gram.:Pl.}
\end{itemize}
\begin{itemize}
\item {Proveniência:(Lat. \textunderscore possibilitas\textunderscore )}
\end{itemize}
Qualidade do que é possível.
Posses, haveres.
Capacidade.
\section{Possibilitar}
\begin{itemize}
\item {Grp. gram.:v. t.}
\end{itemize}
\begin{itemize}
\item {Proveniência:(Do lat. \textunderscore possibilis\textunderscore )}
\end{itemize}
Apresentar como possível; tornar possível.
\section{Possidónio}
\begin{itemize}
\item {Grp. gram.:m.}
\end{itemize}
\begin{itemize}
\item {Utilização:Fam.}
\end{itemize}
\begin{itemize}
\item {Proveniência:(T. muito us. há annos, e procedente do nome com que os jornaes e especialmente um folhetinista designavam certo deputado)}
\end{itemize}
Político ingênuo e sertanejo, que vê a salvação da pátria no córte profundo e incondicional de todas as despesas públicas. Cf. Camillo, \textunderscore Cancion. Al.\textunderscore , 468 e 315.
\section{Possível}
\begin{itemize}
\item {Grp. gram.:adj.}
\end{itemize}
\begin{itemize}
\item {Grp. gram.:M.}
\end{itemize}
\begin{itemize}
\item {Utilização:Fig.}
\end{itemize}
\begin{itemize}
\item {Proveniência:(Lat. \textunderscore possibilis\textunderscore )}
\end{itemize}
Que póde sêr, acontecer ou praticar-se.
Que facilmente se realiza.
Aquillo que é possível.
Empenho, esfôrço.
\section{Possoca}
\begin{itemize}
\item {Grp. gram.:f.}
\end{itemize}
\begin{itemize}
\item {Utilização:Bras}
\end{itemize}
O mesmo que \textunderscore maranduva\textunderscore .
\section{Possocrático}
\begin{itemize}
\item {Grp. gram.:adj.}
\end{itemize}
\begin{itemize}
\item {Proveniência:(De \textunderscore pos...\textunderscore  + \textunderscore socrático\textunderscore )}
\end{itemize}
Relativo aos tempos posteriaes aos de Sócrates. Cf. Latino, \textunderscore Elog.\textunderscore , 201.
\section{Possoeiro}
\begin{itemize}
\item {Grp. gram.:m.}
\end{itemize}
\begin{itemize}
\item {Utilização:Des.}
\end{itemize}
\begin{itemize}
\item {Proveniência:(De \textunderscore posse\textunderscore ; ou, antes, por \textunderscore possueiro\textunderscore , de \textunderscore possuir\textunderscore , se não é erro typográphico em vez de \textunderscore pessoeiro\textunderscore , commetido no \textunderscore Diccion.\textunderscore  de Vieira, vb. \textunderscore achega\textunderscore )}
\end{itemize}
O mesmo que \textunderscore posseiro\textunderscore ; cabecel.
\section{Possoêlo}
\begin{itemize}
\item {Grp. gram.:m.}
\end{itemize}
\begin{itemize}
\item {Utilização:Bras}
\end{itemize}
Alforge de coiro.
\section{Possuca}
\begin{itemize}
\item {Grp. gram.:m.  e  f.}
\end{itemize}
\begin{itemize}
\item {Utilização:Bras. do S}
\end{itemize}
O mesmo que \textunderscore filante\textunderscore ^1.
\section{Possuidor}
\begin{itemize}
\item {fónica:su-i}
\end{itemize}
\begin{itemize}
\item {Grp. gram.:m.  e  f.}
\end{itemize}
O que possue.
\section{Possuimento}
\begin{itemize}
\item {fónica:su-i}
\end{itemize}
\begin{itemize}
\item {Grp. gram.:m.}
\end{itemize}
Acto ou effeito de possuir.
\section{Possuinte}
\begin{itemize}
\item {Grp. gram.:adj.}
\end{itemize}
Que possue.
\section{Possuir}
\begin{itemize}
\item {Grp. gram.:v. t.}
\end{itemize}
\begin{itemize}
\item {Grp. gram.:V. p.}
\end{itemize}
\begin{itemize}
\item {Proveniência:(Lat. \textunderscore possidere\textunderscore )}
\end{itemize}
Têr em seu poder; fruír a posse de; fruír: \textunderscore possuir um prédio\textunderscore .
Exercer.
Sêr dotado de: \textunderscore possuir nobres sentimentos\textunderscore .
Gozar.
Persuadir-se, convencer-se; compenetrar-se.
\section{Posta}
\begin{itemize}
\item {Grp. gram.:f.}
\end{itemize}
\begin{itemize}
\item {Utilização:Fam.}
\end{itemize}
\begin{itemize}
\item {Utilização:Ant.}
\end{itemize}
\begin{itemize}
\item {Proveniência:(De \textunderscore pôsto\textunderscore )}
\end{itemize}
Pedaço de peixe.
Pedaço.
Talhada.
Administração de correio; correio.
Cocheira numa estrada, em que se faz a muda dos cavallos que conduzem uma diligência ou vehículo de serviço público.
Emprêgo rendoso.
O mesmo que \textunderscore aposentadoria\textunderscore .
\section{Postal}
\begin{itemize}
\item {Grp. gram.:adj.}
\end{itemize}
\begin{itemize}
\item {Proveniência:(De \textunderscore posta\textunderscore )}
\end{itemize}
Relativo ao correio: \textunderscore progressos postaes\textunderscore .
\section{Postar}
\begin{itemize}
\item {Grp. gram.:v. t.}
\end{itemize}
\begin{itemize}
\item {Utilização:Ant.}
\end{itemize}
\begin{itemize}
\item {Proveniência:(De \textunderscore posto\textunderscore )}
\end{itemize}
Pôr num lugar ou pôsto (alguém): \textunderscore fui-me postar á entrada da escola\textunderscore .
Compor, fabricar, reparar.
\section{Poste}
\begin{itemize}
\item {Grp. gram.:m.}
\end{itemize}
\begin{itemize}
\item {Proveniência:(Do lat. \textunderscore postis\textunderscore )}
\end{itemize}
Pau, fixado verticalmente no chão.
Pilar de uma portada.
Espécie de columna, a que se ligavam os criminosos, expondo-os á ignomínia pública.
\section{Posteiro}
\begin{itemize}
\item {Grp. gram.:m.}
\end{itemize}
\begin{itemize}
\item {Utilização:Bras}
\end{itemize}
Aquelle que vive no pôsto de uma fazenda.
\section{Postejar}
\begin{itemize}
\item {Grp. gram.:v. t.}
\end{itemize}
Partir em postas: \textunderscore postejar uma pescada\textunderscore .
\section{Postema}
\begin{itemize}
\item {Grp. gram.:f.}
\end{itemize}
(Corr. de \textunderscore apostema\textunderscore )
\section{Postemão}
\begin{itemize}
\item {Grp. gram.:m.}
\end{itemize}
\begin{itemize}
\item {Proveniência:(De \textunderscore postema\textunderscore )}
\end{itemize}
Navalha de alveitar, para abrir apostemas.
\section{Postemeiro}
\begin{itemize}
\item {Grp. gram.:m.}
\end{itemize}
\begin{itemize}
\item {Utilização:Fig.}
\end{itemize}
O mesmo que \textunderscore postemão\textunderscore .
Remédio; allívio. Cf. \textunderscore Anat. Joc.\textunderscore , 470.
\section{Posterciário}
\begin{itemize}
\item {Grp. gram.:adj.}
\end{itemize}
\begin{itemize}
\item {Utilização:Geol.}
\end{itemize}
\begin{itemize}
\item {Proveniência:(De \textunderscore pos...\textunderscore  + \textunderscore terciário\textunderscore )}
\end{itemize}
Posterior ao período terciário.
\section{Postergação}
\begin{itemize}
\item {Grp. gram.:f.}
\end{itemize}
Acto ou effeito de postergar.
\section{Postergamento}
\begin{itemize}
\item {Grp. gram.:m.}
\end{itemize}
O mesmo que \textunderscore postergação\textunderscore .
\section{Postergar}
\begin{itemize}
\item {Grp. gram.:v. t.}
\end{itemize}
\begin{itemize}
\item {Proveniência:(Do lat. \textunderscore post\textunderscore  + \textunderscore tergum\textunderscore )}
\end{itemize}
Deixar atrás; lançar para trás.
Preterir; desprezar.
Pospor.
Transgredir: \textunderscore postergar a lei\textunderscore .
\section{Posteridade}
\begin{itemize}
\item {Grp. gram.:f.}
\end{itemize}
\begin{itemize}
\item {Proveniência:(Lat. \textunderscore posteritas\textunderscore )}
\end{itemize}
Série de indivíduos, procedente da mesma origem; porvindoiros.
Gerações, que succederam ou succederão a uma época.
Futuro.
Celebridade ou glorificação futura.
\section{Postério}
\begin{itemize}
\item {Grp. gram.:m.}
\end{itemize}
\begin{itemize}
\item {Utilização:Ant.}
\end{itemize}
\begin{itemize}
\item {Proveniência:(Lat. \textunderscore posterius\textunderscore )}
\end{itemize}
O mesmo que \textunderscore vindoiro\textunderscore :«\textunderscore Salamão deu a esta perseguiçam grande princípio, e não piqueno azo aos posterios pera acometela.\textunderscore »Usque, 46 v.^o
\section{Posterior}
\begin{itemize}
\item {Grp. gram.:adj.}
\end{itemize}
\begin{itemize}
\item {Grp. gram.:M.}
\end{itemize}
\begin{itemize}
\item {Utilização:Pop.}
\end{itemize}
\begin{itemize}
\item {Grp. gram.:Pl.}
\end{itemize}
\begin{itemize}
\item {Proveniência:(Lat. \textunderscore posterior\textunderscore )}
\end{itemize}
Que vem ou está depois.
Ulterior.
Situado atrás; que ficou atrás.
Futuro.
Nádegas.
O mesmo que [[vindoiros|vindouro]].
\section{Posterioridade}
\begin{itemize}
\item {Grp. gram.:f.}
\end{itemize}
Carácter do que é posterior.
\section{Póstero}
\begin{itemize}
\item {Grp. gram.:adj.}
\end{itemize}
\begin{itemize}
\item {Grp. gram.:M. Pl.}
\end{itemize}
\begin{itemize}
\item {Proveniência:(Lat. \textunderscore posterus\textunderscore )}
\end{itemize}
Que há de vir depois da época actual; futuro.
Gerações, que hão de succeder á actualidade; vindoiros.
\section{Póstero...}
\begin{itemize}
\item {Grp. gram.:pref.}
\end{itemize}
\begin{itemize}
\item {Proveniência:(Lat. \textunderscore posterus\textunderscore )}
\end{itemize}
(designativo de \textunderscore posterioridade\textunderscore )
\section{Póstero-exterior}
\begin{itemize}
\item {Grp. gram.:adj.}
\end{itemize}
Que está detrás e na parte exterior.
\section{Póstero-inferior}
\begin{itemize}
\item {Grp. gram.:adj.}
\end{itemize}
Situado atrás e na parte inferior.
\section{Póstero-interior}
\begin{itemize}
\item {Grp. gram.:adj.}
\end{itemize}
Que está atrás e na parte interior.
\section{Póstero-superior}
\begin{itemize}
\item {Grp. gram.:adj.}
\end{itemize}
Que está atrás e na parte superior.
\section{Posthite}
\begin{itemize}
\item {Grp. gram.:f.}
\end{itemize}
\begin{itemize}
\item {Utilização:Med.}
\end{itemize}
\begin{itemize}
\item {Proveniência:(Do gr. \textunderscore posthe\textunderscore )}
\end{itemize}
Inflammação do prepúcio.
\section{Postiça}
\begin{itemize}
\item {Grp. gram.:f.}
\end{itemize}
\begin{itemize}
\item {Proveniência:(De \textunderscore postiço\textunderscore )}
\end{itemize}
Peça, que se accrescenta ao costado do navio, para o tornar mais alto.
\section{Postiço}
\begin{itemize}
\item {Grp. gram.:adj.}
\end{itemize}
\begin{itemize}
\item {Utilização:Prov.}
\end{itemize}
\begin{itemize}
\item {Utilização:minh.}
\end{itemize}
\begin{itemize}
\item {Grp. gram.:M.}
\end{itemize}
\begin{itemize}
\item {Utilização:Prov.}
\end{itemize}
\begin{itemize}
\item {Proveniência:(De \textunderscore posto\textunderscore )}
\end{itemize}
Accrescentado a uma obra, que já estava concluida.
Que se póde pôr ou tirar: \textunderscore dentes postiços\textunderscore .
Que não fórma corpo inteiriço com aquillo a que se accrescenta ou de que faz parte.
Collocado artificialmente no lugar de alguma coisa que era natural e que falta: \textunderscore perna postiça\textunderscore .
Fingido.
Que não é natural: \textunderscore sorriso postiço\textunderscore .
Adoptivo: \textunderscore filho postiço\textunderscore .
Criança desamparada ou abandonada pelos pais.
Exposto, enjeitado.
\section{Postico}
\begin{itemize}
\item {Grp. gram.:adj.}
\end{itemize}
\begin{itemize}
\item {Proveniência:(Lat. \textunderscore posticus\textunderscore )}
\end{itemize}
Que está atrás; posterior:«\textunderscore ...na parte postiça, a que o vulgo chama toutiço.\textunderscore »Rev. \textunderscore Portugal Med.\textunderscore , 168.
\section{Postigo}
\begin{itemize}
\item {Grp. gram.:m.}
\end{itemize}
\begin{itemize}
\item {Proveniência:(Do lat. \textunderscore posticum\textunderscore )}
\end{itemize}
Pequena porta.
Abertura quadrangular numa porta ou janela, para se vêr quem chega ou quem passa, sem se abrir a porta.
Tampa de goteiras e vigias, nos navios.
Abertura no tampo deanteiro do tonel ou da pipa, pela qual póde entrar alguém para tirar o sarro ou fazer consêrto no interior da vasilha.
\section{Postila}
\begin{itemize}
\item {Grp. gram.:f.}
\end{itemize}
\begin{itemize}
\item {Proveniência:(Lat. \textunderscore postilla\textunderscore )}
\end{itemize}
Caderno de explicações manuscritas, para uso de estudantes.
Explicação, ditada pelo professor e escrita pelo aluno.
Comentário, apostila.
\section{Postilena}
\begin{itemize}
\item {Grp. gram.:f.}
\end{itemize}
\begin{itemize}
\item {Utilização:Ant.}
\end{itemize}
\begin{itemize}
\item {Proveniência:(Do lat. \textunderscore post\textunderscore . + ...)}
\end{itemize}
Atafal; retranca.
\section{Postilhão}
\begin{itemize}
\item {Grp. gram.:m.}
\end{itemize}
\begin{itemize}
\item {Utilização:Ext.}
\end{itemize}
Homem, empregado no serviço do correio, para transportar, a cavallo e com rapidez, correspondências ou notícias entre diversas localidades.
Mensageiro.
(Cast. \textunderscore postillón\textunderscore )
\section{Postilla}
\begin{itemize}
\item {Grp. gram.:f.}
\end{itemize}
\begin{itemize}
\item {Proveniência:(Lat. \textunderscore postilla\textunderscore )}
\end{itemize}
Caderno de explicações manuscritas, para uso de estudantes.
Explicação, ditada pelo professor e escrita pelo alumno.
Commentário, apostilla.
\section{Postimeiro}
\begin{itemize}
\item {Grp. gram.:adj.}
\end{itemize}
(V.postrimeiro)
\section{Postite}
\begin{itemize}
\item {Grp. gram.:f.}
\end{itemize}
\begin{itemize}
\item {Utilização:Med.}
\end{itemize}
\begin{itemize}
\item {Proveniência:(Do gr. \textunderscore posthe\textunderscore )}
\end{itemize}
Inflamação do prepúcio.
\section{Postlimínio}
\begin{itemize}
\item {fónica:posdlim}
\end{itemize}
\begin{itemize}
\item {Grp. gram.:m.}
\end{itemize}
\begin{itemize}
\item {Proveniência:(Lat. \textunderscore postliminium\textunderscore )}
\end{itemize}
Restituição de direitos civis a quem os perdera por ausência ou cativeiro.
\section{Pôsto}
\begin{itemize}
\item {Grp. gram.:adj.}
\end{itemize}
\begin{itemize}
\item {Grp. gram.:M.}
\end{itemize}
\begin{itemize}
\item {Utilização:Bras}
\end{itemize}
\begin{itemize}
\item {Grp. gram.:Loc. conj.}
\end{itemize}
\begin{itemize}
\item {Proveniência:(Do lat. \textunderscore positus\textunderscore )}
\end{itemize}
Collocado.
Disposto.
Apresentado; patenteado: \textunderscore bem pôsto\textunderscore , que se apresenta bem; que traja distintamente.
Desapparecido, (falando-se do Sol no occaso).
Lugar, em que uma pessôa ou coisa está collocada.
Estação ou alojamento de tropas ou guardas policiaes.
Cargo, dignidade.
Graduação militar.
Lugar, que cada militar deve occupar, quando desempenha as suas funcções: \textunderscore morreu no seu pôsto\textunderscore .
Casa em fazendas, habitada por quem as vigia.
* Conj.
O mesmo que [[pôsto que|pôsto]].
Pôsto que, ainda que, se bem que, embora.
\section{Postónico}
\begin{itemize}
\item {Grp. gram.:adj.}
\end{itemize}
\begin{itemize}
\item {Utilização:Gram.}
\end{itemize}
\begin{itemize}
\item {Proveniência:(De \textunderscore pos...\textunderscore  + \textunderscore tónico\textunderscore )}
\end{itemize}
Diz-se da vogal, que está depois da vogal tónica de uma palavra.
\section{Postoiro}
\begin{itemize}
\item {Grp. gram.:m.}
\end{itemize}
\begin{itemize}
\item {Utilização:Prov.}
\end{itemize}
\begin{itemize}
\item {Utilização:minh.}
\end{itemize}
\begin{itemize}
\item {Utilização:T. de Barcelos}
\end{itemize}
\begin{itemize}
\item {Proveniência:(Do lat. \textunderscore positurus\textunderscore , de \textunderscore ponere\textunderscore , pôr)}
\end{itemize}
Lugar, onde se põe, occultando-a, a chave de uma casa.
Mólho ou feixe de erva.
\section{Postouro}
\begin{itemize}
\item {Grp. gram.:m.}
\end{itemize}
\begin{itemize}
\item {Utilização:Prov.}
\end{itemize}
\begin{itemize}
\item {Utilização:minh.}
\end{itemize}
\begin{itemize}
\item {Utilização:T. de Barcelos}
\end{itemize}
\begin{itemize}
\item {Proveniência:(Do lat. \textunderscore positurus\textunderscore , de \textunderscore ponere\textunderscore , pôr)}
\end{itemize}
Lugar, onde se põe, occultando-a, a chave de uma casa.
Mólho ou feixe de erva.
\section{Postrar}
\textunderscore v. t.\textunderscore  (e der.)
(V. \textunderscore prostrar\textunderscore , etc)
\section{Postre}
m.
O mesmo que postres:«\textunderscore ...durante o postre de um jantar.\textunderscore »Camilo, \textunderscore Regicida\textunderscore , 134.
\section{Postreiro}
\begin{itemize}
\item {Grp. gram.:adj.}
\end{itemize}
\begin{itemize}
\item {Utilização:Ant.}
\end{itemize}
O mesmo que \textunderscore postremo\textunderscore .
\section{Postremo}
\begin{itemize}
\item {Grp. gram.:adj.}
\end{itemize}
\begin{itemize}
\item {Proveniência:(Lat. \textunderscore postremus\textunderscore )}
\end{itemize}
Último, extremo.
\section{Postres}
\begin{itemize}
\item {Grp. gram.:m. pl.}
\end{itemize}
O mesmo que \textunderscore sobremesa\textunderscore .
(Cast. \textunderscore postre\textunderscore )
\section{Postrimeiro}
\begin{itemize}
\item {Grp. gram.:m.}
\end{itemize}
\begin{itemize}
\item {Utilização:P. us.}
\end{itemize}
O mesmo que \textunderscore postremo\textunderscore .
\section{Postulação}
\begin{itemize}
\item {Grp. gram.:f.}
\end{itemize}
\begin{itemize}
\item {Proveniência:(Lat. \textunderscore postulatio\textunderscore )}
\end{itemize}
Acto de postular.
\section{Postulado}
\begin{itemize}
\item {Grp. gram.:m.}
\end{itemize}
Princípio ou facto, reconhecido, mas não demonstrado.
Princípio que, em Mathemática, se admite sem discussão, mas que não é tão evidente como o axioma.
Tempo de exercícios e provações, que antecede o noviciado nas communidades religiosas.
\section{Postulante}
\begin{itemize}
\item {Grp. gram.:m. ,  f.  e  adj.}
\end{itemize}
\begin{itemize}
\item {Proveniência:(Lat. \textunderscore postulans\textunderscore )}
\end{itemize}
Pessôa, que postula.
\section{Postular}
\begin{itemize}
\item {Grp. gram.:v. t.}
\end{itemize}
\begin{itemize}
\item {Proveniência:(Lat. \textunderscore postulare\textunderscore )}
\end{itemize}
Pedir instantemente, supplicar.
Requerer, documentando a allegação.
\section{Postumamente}
\begin{itemize}
\item {Grp. gram.:adv.}
\end{itemize}
\begin{itemize}
\item {Proveniência:(De \textunderscore póstumo\textunderscore )}
\end{itemize}
Depois da morte.
\section{Postumeira}
\begin{itemize}
\item {Grp. gram.:f.}
\end{itemize}
\begin{itemize}
\item {Utilização:Ant.}
\end{itemize}
\begin{itemize}
\item {Proveniência:(De \textunderscore postumeiro\textunderscore )}
\end{itemize}
Fim, remate.
Termo da vida.
\section{Postumeiro}
\begin{itemize}
\item {Grp. gram.:adj.}
\end{itemize}
\begin{itemize}
\item {Utilização:Ant.}
\end{itemize}
\begin{itemize}
\item {Proveniência:(De \textunderscore póstumo\textunderscore )}
\end{itemize}
Último, derradeiro.
\section{Póstumo}
\begin{itemize}
\item {Grp. gram.:adj.}
\end{itemize}
\begin{itemize}
\item {Proveniência:(Lat. \textunderscore postumus\textunderscore )}
\end{itemize}
Que nasceu depois da morte do pai.
Que se realizou depois da morte de alguém.
Que se publicou depois da morte do respectivo autôr: \textunderscore livro póstumo\textunderscore .
\section{Postura}
\begin{itemize}
\item {Grp. gram.:f.}
\end{itemize}
\begin{itemize}
\item {Grp. gram.:Loc.}
\end{itemize}
\begin{itemize}
\item {Utilização:minh}
\end{itemize}
\begin{itemize}
\item {Proveniência:(Lat. \textunderscore positura\textunderscore )}
\end{itemize}
Posição do corpo.
Disposição; aspecto phýsico.
Artificios, enfeites, cosméticos, especialmente usados por algumas damas.
Deliberação municipal escrita, que obriga os munícipes ao cumprimento de certos deveres de ordem pública.
Os ovos, que uma galinha põe durante certo tempo.
\textunderscore Fazer postura\textunderscore , fazer trejeitos.
\section{Posturar}
\begin{itemize}
\item {Grp. gram.:v. i.}
\end{itemize}
(T., proposto para substituir o francesismo poser)
\section{Postureiro}
\begin{itemize}
\item {Grp. gram.:m.}
\end{itemize}
\begin{itemize}
\item {Utilização:Ant.}
\end{itemize}
\begin{itemize}
\item {Proveniência:(De \textunderscore postura\textunderscore )}
\end{itemize}
Aquelle que vende cosméticos ou arrebiques para o rosto.
\section{Posual}
\begin{itemize}
\item {Grp. gram.:m.}
\end{itemize}
\begin{itemize}
\item {Utilização:T. de Timor}
\end{itemize}
Lugar, onde se guardam coisas sagradas, zagaias, amuletos, etc.
\section{Pota}
\begin{itemize}
\item {Grp. gram.:f.}
\end{itemize}
\begin{itemize}
\item {Proveniência:(De \textunderscore pote\textunderscore ?)}
\end{itemize}
Peixe marítimo da costa de Portugal.
\section{Potaba}
\begin{itemize}
\item {Grp. gram.:f.}
\end{itemize}
\begin{itemize}
\item {Utilização:Bras}
\end{itemize}
\begin{itemize}
\item {Proveniência:(T. tupi)}
\end{itemize}
Presente; dádiva.
Legado.
\section{Potagem}
\begin{itemize}
\item {Grp. gram.:f.}
\end{itemize}
\begin{itemize}
\item {Utilização:Ant.}
\end{itemize}
\begin{itemize}
\item {Proveniência:(Fr. \textunderscore potage\textunderscore )}
\end{itemize}
Caldo, sopa.
Legumes ou qualquer hortaliça, que se mete no pote ou panela, para se tornar comestível. Cf. Rebello da Silva, \textunderscore Mocidade\textunderscore , III, 44; J. M. Grande, \textunderscore Man. do Cultivador\textunderscore , II, 47.
\section{Potamita}
\begin{itemize}
\item {Grp. gram.:adj.}
\end{itemize}
\begin{itemize}
\item {Grp. gram.:M. pl.}
\end{itemize}
\begin{itemize}
\item {Proveniência:(Do gr. \textunderscore potamos\textunderscore )}
\end{itemize}
Que vive nos rios.
Família de reptis fluviaes.
\section{Potamofobia}
\begin{itemize}
\item {Grp. gram.:f.}
\end{itemize}
\begin{itemize}
\item {Proveniência:(Do gr. \textunderscore potamos\textunderscore  + \textunderscore phobein\textunderscore )}
\end{itemize}
Mêdo mórbido dos rios.
\section{Potamófobo}
\begin{itemize}
\item {Grp. gram.:m.}
\end{itemize}
Aquelle que soffre potamophobia.
\section{Potamografia}
\begin{itemize}
\item {Grp. gram.:f.}
\end{itemize}
\begin{itemize}
\item {Proveniência:(Do gr. \textunderscore potamos\textunderscore  + \textunderscore graphein\textunderscore )}
\end{itemize}
Descrição dos rios.
\section{Potamográfico}
\begin{itemize}
\item {Grp. gram.:adj.}
\end{itemize}
Relativo á potamografia.
\section{Potamographia}
\begin{itemize}
\item {Grp. gram.:f.}
\end{itemize}
\begin{itemize}
\item {Proveniência:(Do gr. \textunderscore potamos\textunderscore  + \textunderscore graphein\textunderscore )}
\end{itemize}
Descripção dos rios.
\section{Potamográphico}
\begin{itemize}
\item {Grp. gram.:adj.}
\end{itemize}
Relativo á potamographia.
\section{Potamologia}
\begin{itemize}
\item {Grp. gram.:f.}
\end{itemize}
O mesmo que \textunderscore potamographia\textunderscore .
\section{Potamológico}
\begin{itemize}
\item {Grp. gram.:adj.}
\end{itemize}
Relativo á potamologia.
\section{Potamophobia}
\begin{itemize}
\item {Grp. gram.:f.}
\end{itemize}
\begin{itemize}
\item {Proveniência:(Do gr. \textunderscore potamos\textunderscore  + \textunderscore phobein\textunderscore )}
\end{itemize}
Mêdo mórbido dos rios.
\section{Potamóphobo}
\begin{itemize}
\item {Grp. gram.:m.}
\end{itemize}
Aquelle que soffre potamophobia.
\section{Potança}
\begin{itemize}
\item {Grp. gram.:f.}
\end{itemize}
\begin{itemize}
\item {Utilização:Prov.}
\end{itemize}
\begin{itemize}
\item {Utilização:dur.}
\end{itemize}
\begin{itemize}
\item {Proveniência:(Do fr. \textunderscore poutence\textunderscore )}
\end{itemize}
Peça de madeira ou cepo, sôbre que os chapeleiros amaciam e lustram os chapéus de seda.
\section{Potassa}
\begin{itemize}
\item {Grp. gram.:f.}
\end{itemize}
\begin{itemize}
\item {Proveniência:(Do al. \textunderscore pott\textunderscore  + \textunderscore asche\textunderscore )}
\end{itemize}
Substância, composta de oxygênio e de potássio, formando saes com os ácidos, sabão com os óleos e vidro com a sílica.
Protóxydo de potássio.
\textunderscore Potassa cáustica\textunderscore , hydrato de potássio.
\textunderscore Potassa do commércio\textunderscore , carbonato de potássio impuro.
\section{Potássico}
\begin{itemize}
\item {Grp. gram.:adj.}
\end{itemize}
Diz-se de certas combinações chímicas, em que entra o potássio.
\section{Potassímetro}
\begin{itemize}
\item {Grp. gram.:m.}
\end{itemize}
\begin{itemize}
\item {Proveniência:(De \textunderscore potassa\textunderscore  + gr. \textunderscore metron\textunderscore )}
\end{itemize}
Instrumento, para determinar as proporções de potassa e soda, contidas nas misturas alcalinas, conhecidas no commércio pelo nome de potassas.
\section{Potássio}
\begin{itemize}
\item {Grp. gram.:m.}
\end{itemize}
Metal branco, descoberto em 1807 e que, combinado com o oxygênio, dá a potassa pura.
\section{Potava}
\begin{itemize}
\item {Grp. gram.:f.}
\end{itemize}
(V.potaba)
\section{Potável}
\begin{itemize}
\item {Grp. gram.:adj.}
\end{itemize}
\begin{itemize}
\item {Proveniência:(Lat. \textunderscore potabilis\textunderscore )}
\end{itemize}
Que se póde beber, que é bom para se beber: \textunderscore nasce ali água potável\textunderscore .
\section{Pote}
\begin{itemize}
\item {Grp. gram.:m.}
\end{itemize}
\begin{itemize}
\item {Utilização:Burl.}
\end{itemize}
\begin{itemize}
\item {Utilização:Prov.}
\end{itemize}
\begin{itemize}
\item {Utilização:dur.}
\end{itemize}
Grande vaso de barro, para líquidos.
Antiga medida de seis canadas.
Pessôa atarracada.
O mesmo que penico.
(B. lat. \textunderscore potus\textunderscore )
\section{Potéa}
\begin{itemize}
\item {Grp. gram.:f.}
\end{itemize}
\begin{itemize}
\item {Proveniência:(Fr. \textunderscore potée\textunderscore )}
\end{itemize}
Nome de várias preparações de chímicos, polidores, etc., especialmente preparação de um óxydo de estanho, que serve para polir os espelhos de aço e outros objectos de igual natureza.
\section{Poteia}
\begin{itemize}
\item {Grp. gram.:f.}
\end{itemize}
\begin{itemize}
\item {Proveniência:(Fr. \textunderscore potée\textunderscore )}
\end{itemize}
Nome de várias preparações de chímicos, polidores, etc., especialmente preparação de um óxydo de estanho, que serve para polir os espelhos de aço e outros objectos de igual natureza.
\section{Potência}
\begin{itemize}
\item {Grp. gram.:f.}
\end{itemize}
\begin{itemize}
\item {Proveniência:(Lat. potentia)}
\end{itemize}
Qualidade do que é potente.
Poder; vigôr, fôrça.
Poderio.
Autoridade.
Nação soberana: \textunderscore as grandes potências europeias\textunderscore .
Pessôa muito importante.
Capacidade de realizar, de produzir.
Fôrça, com que se equilibra ou se vence uma fôrça contrária.
Ponto, em que aquella fôrça se applica.
Producto do um número, multiplicado por si uma ou mais vezes.
Faculdade (da alma)
\section{Potenciação}
\begin{itemize}
\item {Grp. gram.:f.}
\end{itemize}
Acto ou effeito de potenciar.
\section{Potencial}
\begin{itemize}
\item {Grp. gram.:adj.}
\end{itemize}
Relativo a potência; virtual.
\section{Potencialidade}
\begin{itemize}
\item {Grp. gram.:f.}
\end{itemize}
Qualidade de potencial. Cf. R. de Brito, \textunderscore Phil. do Dir.\textunderscore , 74.
\section{Potencializar}
\begin{itemize}
\item {Grp. gram.:v. t.}
\end{itemize}
\begin{itemize}
\item {Utilização:Neol.}
\end{itemize}
Tornar potente, reforçar.
\section{Potencialmente}
\begin{itemize}
\item {Grp. gram.:adv.}
\end{itemize}
De modo potencial.
\section{Potenciar}
\begin{itemize}
\item {Grp. gram.:v.}
\end{itemize}
\begin{itemize}
\item {Utilização:t. Mathem.}
\end{itemize}
\begin{itemize}
\item {Proveniência:(De \textunderscore potência\textunderscore )}
\end{itemize}
Elevar a qualquer potência (uma quantidade).
\section{Potentado}
\begin{itemize}
\item {Grp. gram.:m.}
\end{itemize}
\begin{itemize}
\item {Utilização:Ext.}
\end{itemize}
\begin{itemize}
\item {Proveniência:(Lat. \textunderscore potentatus\textunderscore )}
\end{itemize}
Príncipe soberano, de grande autoridade ou de grande poder material.
Pessôa muito influente ou poderosa.
\section{Potente}
\begin{itemize}
\item {Grp. gram.:adj.}
\end{itemize}
\begin{itemize}
\item {Proveniência:(Lat. \textunderscore potens\textunderscore )}
\end{itemize}
Que póde, que tem a faculdade de fazer ou produzir alguma coisa.
Que tem poderío, ou muita importância.
Enérgico; violento, rude.
\section{Potentéa}
\begin{itemize}
\item {Grp. gram.:f.}
\end{itemize}
\begin{itemize}
\item {Utilização:Heráld.}
\end{itemize}
Diz-se da cruz vazada, cujas hastes são rematadas por figura quadrilonga.
\section{Potenteia}
\begin{itemize}
\item {Grp. gram.:f.}
\end{itemize}
\begin{itemize}
\item {Utilização:Heráld.}
\end{itemize}
Diz-se da cruz vazada, cujas hastes são rematadas por figura quadrilonga.
\section{Potentemente}
\begin{itemize}
\item {Grp. gram.:adv.}
\end{itemize}
De modo potente.
\section{Potentila}
\begin{itemize}
\item {Grp. gram.:f.}
\end{itemize}
Planta, o mesmo que \textunderscore potentilha\textunderscore . Cf. P. Coutinho, \textunderscore Flora\textunderscore , 304.
\section{Potentilha}
\begin{itemize}
\item {Grp. gram.:f.}
\end{itemize}
(V.cinco-em-rama)
\section{Potentilla}
\begin{itemize}
\item {Grp. gram.:f.}
\end{itemize}
Planta, o mesmo que \textunderscore potentilha\textunderscore . Cf. P. Coutinho, \textunderscore Flora\textunderscore , 304.
\section{Poterantera}
\begin{itemize}
\item {Grp. gram.:f.}
\end{itemize}
\begin{itemize}
\item {Proveniência:(Do gr. \textunderscore poterion\textunderscore  + \textunderscore anthera\textunderscore )}
\end{itemize}
Gênero de plantas melastomáceas do Brasil.
\section{Poteranthera}
\begin{itemize}
\item {Grp. gram.:f.}
\end{itemize}
\begin{itemize}
\item {Proveniência:(Do gr. \textunderscore poterion\textunderscore  + \textunderscore anthera\textunderscore )}
\end{itemize}
Gênero de plantas melastomáceas do Brasil.
\section{Potério}
\begin{itemize}
\item {Grp. gram.:m.}
\end{itemize}
\begin{itemize}
\item {Proveniência:(Do gr. \textunderscore poterion\textunderscore )}
\end{itemize}
Nome scientífico da pimpinela.
\section{Poterna}
\begin{itemize}
\item {Grp. gram.:f.}
\end{itemize}
\begin{itemize}
\item {Proveniência:(Fr. \textunderscore poterne\textunderscore )}
\end{itemize}
Porta falsa ou galeria subterrânea, por onde se sai secretamente de uma praça fortificada.
\section{Potestade}
\begin{itemize}
\item {Grp. gram.:f.}
\end{itemize}
\begin{itemize}
\item {Utilização:Ext.}
\end{itemize}
\begin{itemize}
\item {Grp. gram.:Pl.}
\end{itemize}
\begin{itemize}
\item {Utilização:Theol.}
\end{itemize}
\begin{itemize}
\item {Proveniência:(Lat. \textunderscore potestas\textunderscore )}
\end{itemize}
Potência; poder; potentado.
A divindade.
Um dos nove coros de anjos.
\section{Potestativo}
\begin{itemize}
\item {Grp. gram.:adj.}
\end{itemize}
Revestido de poder. Cf. Rui Barb., \textunderscore Réplica\textunderscore , 158.
\section{Potho}
\begin{itemize}
\item {Grp. gram.:m.}
\end{itemize}
\begin{itemize}
\item {Proveniência:(Do gr. \textunderscore pothos\textunderscore )}
\end{itemize}
Planta annual, pouco conhecida.
\section{Potigares}
\begin{itemize}
\item {Grp. gram.:m. pl.}
\end{itemize}
Antiga nação de Índios do Brasil, que dominava entre o rio Paraíba e a costa do norte.
\section{Potiguarás}
\begin{itemize}
\item {Grp. gram.:m. pl.}
\end{itemize}
(V.potigares)
\section{Potiqui}
\begin{itemize}
\item {Grp. gram.:m.}
\end{itemize}
\begin{itemize}
\item {Utilização:Bras}
\end{itemize}
Cigarra do mar.
\section{Potirão}
\begin{itemize}
\item {Grp. gram.:m.}
\end{itemize}
\begin{itemize}
\item {Utilização:Bras}
\end{itemize}
O mesmo que \textunderscore potirom\textunderscore .
\section{Potirom}
\begin{itemize}
\item {Grp. gram.:m.}
\end{itemize}
\begin{itemize}
\item {Utilização:Bras. do N}
\end{itemize}
O mesmo que \textunderscore muxirão\textunderscore .
\section{Poto}
\begin{itemize}
\item {Grp. gram.:m.}
\end{itemize}
\begin{itemize}
\item {Proveniência:(Do gr. \textunderscore pothos\textunderscore )}
\end{itemize}
Planta annual, pouco conhecida.
\section{Poto}
\begin{itemize}
\item {Grp. gram.:m.}
\end{itemize}
\begin{itemize}
\item {Utilização:Poét.}
\end{itemize}
\begin{itemize}
\item {Proveniência:(Lat. \textunderscore potus\textunderscore )}
\end{itemize}
Bebida.
\section{Potó}
\begin{itemize}
\item {Grp. gram.:m.}
\end{itemize}
\begin{itemize}
\item {Utilização:Bras}
\end{itemize}
Insecto noctívago, cuja urina é cáustica.
\section{Potopoto}
\begin{itemize}
\item {Grp. gram.:m.}
\end{itemize}
Ave trepadora da África occidental.
\section{Potosi}
\begin{itemize}
\item {Grp. gram.:m.}
\end{itemize}
\begin{itemize}
\item {Utilização:Fig.}
\end{itemize}
\begin{itemize}
\item {Proveniência:(De \textunderscore Potosi\textunderscore , n. p.)}
\end{itemize}
Grande riqueza; thesoiro. Cf. Filinto, XXI, 130.
\section{Potote}
\begin{itemize}
\item {Grp. gram.:m.}
\end{itemize}
Gênero de plantígrados da América.
\section{Potra}
\begin{itemize}
\item {fónica:pô}
\end{itemize}
\begin{itemize}
\item {Grp. gram.:f.}
\end{itemize}
\begin{itemize}
\item {Proveniência:(De \textunderscore pôtro\textunderscore )}
\end{itemize}
Egua nova.
\section{Potra}
\begin{itemize}
\item {fónica:pô}
\end{itemize}
\begin{itemize}
\item {Grp. gram.:f.}
\end{itemize}
\begin{itemize}
\item {Utilização:Prov.}
\end{itemize}
\begin{itemize}
\item {Utilização:trasm.}
\end{itemize}
\begin{itemize}
\item {Proveniência:(Do lat. \textunderscore putris\textunderscore ?)}
\end{itemize}
Hérnia intestinal; quebradura.
Doença dos vegetaes, caracterizada por saliências nodosas na haste ou na raiz de algumas plantas hortenses.
Doença das gallinhas.
\section{Potranca}
\begin{itemize}
\item {Grp. gram.:f.}
\end{itemize}
\begin{itemize}
\item {Utilização:Bras}
\end{itemize}
\begin{itemize}
\item {Proveniência:(De \textunderscore potra\textunderscore )}
\end{itemize}
Potra, de menos de três annos.
\section{Potranco}
\begin{itemize}
\item {Grp. gram.:m.}
\end{itemize}
\begin{itemize}
\item {Utilização:Bras}
\end{itemize}
Potro, de menos de três annos.
\section{Potréa}
\begin{itemize}
\item {Grp. gram.:f.}
\end{itemize}
\begin{itemize}
\item {Utilização:Ext.}
\end{itemize}
\begin{itemize}
\item {Proveniência:(Do lat. \textunderscore putridus\textunderscore ?)}
\end{itemize}
Bebida desagradável ou estragada.
Coisa que não presta.
\section{Potreia}
\begin{itemize}
\item {Grp. gram.:f.}
\end{itemize}
\begin{itemize}
\item {Utilização:Ext.}
\end{itemize}
\begin{itemize}
\item {Proveniência:(Do lat. \textunderscore putridus\textunderscore ?)}
\end{itemize}
Bebida desagradável ou estragada.
Coisa que não presta.
\section{Potreiro}
\begin{itemize}
\item {Grp. gram.:m.}
\end{itemize}
Negociante de potros.
Negociante de gado cavallar.
Lugar cercado, onde se guarda gado.
\section{Potril}
\begin{itemize}
\item {Grp. gram.:m.}
\end{itemize}
Pátio ou alpendre, em que se guardam potros para adestrar.
(Cp. cast. \textunderscore potril\textunderscore )
\section{Potrilha}
\begin{itemize}
\item {Grp. gram.:m.}
\end{itemize}
\begin{itemize}
\item {Utilização:Deprec.}
\end{itemize}
Indivíduo potroso.
Farroupilha, bisbórria. Cf. F. Manuel, \textunderscore Apól. Dialogaes\textunderscore , 460.
(Cast. \textunderscore potrilla\textunderscore )
\section{Potrilho}
\begin{itemize}
\item {Grp. gram.:m.}
\end{itemize}
\begin{itemize}
\item {Utilização:Bras. do S}
\end{itemize}
Potro, de menos de um anno de idade.
\section{Potrincas}
\begin{itemize}
\item {Grp. gram.:m. pl.}
\end{itemize}
\begin{itemize}
\item {Utilização:Ant.}
\end{itemize}
\begin{itemize}
\item {Proveniência:(De \textunderscore pôtra\textunderscore ^2?)}
\end{itemize}
Homem entanguido, esgrouviado, achacadiço.
\section{Potro}
\begin{itemize}
\item {fónica:pô}
\end{itemize}
\begin{itemize}
\item {Grp. gram.:m.}
\end{itemize}
Cavallo novo, de menos de quatro annos.
Cavallo pequeno e novo.
Espécie de cavallo de madeira, em que se torturavam os accusados ou condemnados.
(Alter. de \textunderscore pôldro\textunderscore )
\section{Potroso}
\begin{itemize}
\item {Grp. gram.:adj.}
\end{itemize}
Que tem potra^2.
\section{Poucachinho}
\begin{itemize}
\item {Grp. gram.:m. ,  adj.  e  adv.}
\end{itemize}
\begin{itemize}
\item {Proveniência:(De \textunderscore pouco\textunderscore  + \textunderscore achinho\textunderscore , dem. de \textunderscore acho\textunderscore , suff.)}
\end{itemize}
Muito pouco.
\section{Poucamente}
\begin{itemize}
\item {Grp. gram.:adv.}
\end{itemize}
\begin{itemize}
\item {Proveniência:(De \textunderscore pouco\textunderscore )}
\end{itemize}
Pouco a pouco; paulatinamente; vagarosamente:«\textunderscore ião os nossos poucamente recuando...\textunderscore »Filinto, \textunderscore D. Man.\textunderscore , II, 37.
\section{Pouca-vergonha}
\begin{itemize}
\item {Grp. gram.:f.}
\end{itemize}
\begin{itemize}
\item {Utilização:Pop.}
\end{itemize}
Falta de vergonha ou pundonor.
Acto vergonhoso e immoral.
Tratantada, patifaria.
\section{Pouco}
\begin{itemize}
\item {Grp. gram.:adj.}
\end{itemize}
\begin{itemize}
\item {Grp. gram.:M.}
\end{itemize}
\begin{itemize}
\item {Grp. gram.:Adv.}
\end{itemize}
\begin{itemize}
\item {Grp. gram.:Loc. adv.}
\end{itemize}
\begin{itemize}
\item {Proveniência:(Lat. \textunderscore paucus\textunderscore )}
\end{itemize}
Que é em pequena quantidade ou número: \textunderscore pouco dinheiro\textunderscore .
Que não abunda; limitado; escasso: \textunderscore pouca prudência\textunderscore .
Pequeno.
Aquillo que é em pequena quantidade ou número.
O que tem pequeno valor.
Bagatela.
Não muito: \textunderscore soffrer pouco\textunderscore .
Em pequena quantidade; insufficientemente: \textunderscore êste terreno produz pouco\textunderscore .
\textunderscore Pouco a pouco\textunderscore , com pequenos intervallos; em pequenas porções; gradualmente.
\section{Poucochinho}
\begin{itemize}
\item {Grp. gram.:m. ,  adj.  e  adv.}
\end{itemize}
(V.poucachinho)
\section{Poupa}
\begin{itemize}
\item {Grp. gram.:f.}
\end{itemize}
\begin{itemize}
\item {Utilização:Pop.}
\end{itemize}
\begin{itemize}
\item {Proveniência:(Do lat. \textunderscore upupa\textunderscore )}
\end{itemize}
Pássaro tenuirostro, semelhante á pêga.
Tufo de pennas, que adorna a cabeça de algumas aves.
Nó de cabello, no alto da cabeça.
\section{Poupa}
\begin{itemize}
\item {Grp. gram.:f.}
\end{itemize}
O mesmo que \textunderscore poupança\textunderscore .
\section{Poupa-boubela}
\begin{itemize}
\item {Grp. gram.:f.}
\end{itemize}
Ave, o mesmo que \textunderscore poupa\textunderscore ^1. Cf. Moraes, \textunderscore Zool. Elem.\textunderscore , 298.
\section{Poupado}
\begin{itemize}
\item {Grp. gram.:adj.}
\end{itemize}
\begin{itemize}
\item {Proveniência:(De \textunderscore poupar\textunderscore )}
\end{itemize}
Que não é gastador; económico.
\section{Poupador}
\begin{itemize}
\item {Grp. gram.:m.  e  adj.}
\end{itemize}
O que poupa.
\section{Poupança}
\begin{itemize}
\item {Grp. gram.:f.}
\end{itemize}
\begin{itemize}
\item {Utilização:Fam.}
\end{itemize}
Economia demasiada; sovinice.
Acto ou effeito de poupar. Cf. Castilho, \textunderscore D. Quixote\textunderscore , I, 380.
\section{Poupanela}
\begin{itemize}
\item {Grp. gram.:f.}
\end{itemize}
\begin{itemize}
\item {Utilização:T. de Turquel}
\end{itemize}
Lírio roxo dos montes.
\section{Poupão}
\begin{itemize}
\item {Grp. gram.:m.}
\end{itemize}
\begin{itemize}
\item {Utilização:Prov.}
\end{itemize}
O mesmo que \textunderscore poupa\textunderscore ^1.
\section{Poupar}
\begin{itemize}
\item {Grp. gram.:v. t.}
\end{itemize}
\begin{itemize}
\item {Grp. gram.:V. i.}
\end{itemize}
\begin{itemize}
\item {Grp. gram.:V. p.}
\end{itemize}
\begin{itemize}
\item {Proveniência:(Lat. \textunderscore palpare\textunderscore )}
\end{itemize}
Dispender moderamente.
Preterir, não aproveitar, pôr de lado.
Não tratar de.
Não tratar mal, não offender.
Economizar; deixar de gastar.
Não dissipar.
Evitar.
Acatar; respeitar.
Administrar bem os seus haveres.
Viver com economia, sêr discreto ou moderado em despesas.
Eximir-se: \textunderscore poupar-se a esforços\textunderscore .
\section{Poupudo}
\begin{itemize}
\item {Grp. gram.:adj.}
\end{itemize}
Que tem poupa^1.
\section{Pouquetino}
\begin{itemize}
\item {Grp. gram.:adv.}
\end{itemize}
\begin{itemize}
\item {Utilização:Ant.}
\end{itemize}
Muito pouco. Cf. Frei Fortun., \textunderscore Inéd.\textunderscore , 312.
\section{Pouquidade}
\begin{itemize}
\item {Grp. gram.:f.}
\end{itemize}
\begin{itemize}
\item {Proveniência:(De \textunderscore pouco\textunderscore )}
\end{itemize}
Pequena porção ou número; pequenez.
Exiguidade; insignificância.
Pequeno mérito; deficiência de aptidão.
\section{Pouquidão}
\begin{itemize}
\item {Grp. gram.:f.}
\end{itemize}
O mesmo que \textunderscore pouquidade\textunderscore .
\section{Pouquinho}
\begin{itemize}
\item {Grp. gram.:m.}
\end{itemize}
Muito pouca coisa; quási nada. Cf. Camillo, \textunderscore Ôlho de Vidro\textunderscore , 13.
\section{Pouquíssimo}
\begin{itemize}
\item {Grp. gram.:adj.}
\end{itemize}
Muito pouco.
\section{Pousa}
\begin{itemize}
\item {Grp. gram.:f.}
\end{itemize}
\begin{itemize}
\item {Utilização:Ant.}
\end{itemize}
\begin{itemize}
\item {Utilização:Prov.}
\end{itemize}
\begin{itemize}
\item {Utilização:minh.}
\end{itemize}
\begin{itemize}
\item {Utilização:Prov.}
\end{itemize}
\begin{itemize}
\item {Utilização:trasm.}
\end{itemize}
\begin{itemize}
\item {Utilização:Prov.}
\end{itemize}
\begin{itemize}
\item {Utilização:dur.}
\end{itemize}
\begin{itemize}
\item {Utilização:Prov.}
\end{itemize}
\begin{itemize}
\item {Utilização:minh.}
\end{itemize}
\begin{itemize}
\item {Utilização:Prov.}
\end{itemize}
\begin{itemize}
\item {Utilização:beir.}
\end{itemize}
\begin{itemize}
\item {Proveniência:(De \textunderscore pousar\textunderscore )}
\end{itemize}
Lugar ou habitação, onde o cobrador de foros reaes devia pousar e receber mantimentos.
A hora da meia noite nos trabalhos de lagareiros.
Cada um dos períodos, em que se divide o tempo de pisar o mosto.
Acto de \textunderscore pousar\textunderscore .
Lugar, onde se pousa o carrêgo, para descansar.
Quatro ou cinco feixes de pão de pragana, (trigo, centeio, etc.).
\section{Pousada}
\begin{itemize}
\item {Grp. gram.:f.}
\end{itemize}
\begin{itemize}
\item {Utilização:Prov.}
\end{itemize}
\begin{itemize}
\item {Utilização:trasm.}
\end{itemize}
\begin{itemize}
\item {Proveniência:(Do lat. \textunderscore pausata\textunderscore )}
\end{itemize}
Acto ou effeito de pousar.
Albergaria, albergue.
Lugar ou casa, em que se pousa ou se é hospedado.
Residência.
Choupana.
Conjuncto de quatro feixes de pão ceifado, devendo produzir um alqueire depois da trilha; poisa.
Casta de uva da Bairrada.
\section{Pousadeira}
\begin{itemize}
\item {Grp. gram.:f.}
\end{itemize}
O mesmo que [[nádegas|nádega]]. Cf. Castilho, \textunderscore D. Quixote\textunderscore , IX. (Também us. no pl., com o mesmo sentido).
(Cp. \textunderscore poisadeiro\textunderscore ^2)
\section{Pousadouro}
\begin{itemize}
\item {Grp. gram.:m.}
\end{itemize}
\begin{itemize}
\item {Utilização:Ant.}
\end{itemize}
O que dá ou prepara a pousada. Cf. \textunderscore Auto de S. Antonio\textunderscore .
\section{Pousadouro}
\begin{itemize}
\item {Grp. gram.:m.}
\end{itemize}
\begin{itemize}
\item {Utilização:Pleb.}
\end{itemize}
\begin{itemize}
\item {Proveniência:(De \textunderscore pousar\textunderscore )}
\end{itemize}
Nádegas. Cf. G. Vicente, I, 50, (ed. de M. Remédios).
\section{Pousadouro}
\begin{itemize}
\item {Grp. gram.:m.}
\end{itemize}
\begin{itemize}
\item {Utilização:Pleb.}
\end{itemize}
\begin{itemize}
\item {Proveniência:(Do lat. \textunderscore pausatorius\textunderscore )}
\end{itemize}
O mesmo que \textunderscore pousada\textunderscore .
Pousadeiro^2.
\section{Pousafolles}
\begin{itemize}
\item {Grp. gram.:m.  e  f.}
\end{itemize}
\begin{itemize}
\item {Utilização:Ant.}
\end{itemize}
\begin{itemize}
\item {Proveniência:(De \textunderscore poisar\textunderscore  + \textunderscore folle\textunderscore )}
\end{itemize}
Pessôa muito vagarosa ou indolente, que descansa ao menor trabalho.
\section{Pousagem}
\begin{itemize}
\item {Grp. gram.:f.}
\end{itemize}
Acto ou operação de pousar pelles.
\section{Pousar}
\begin{itemize}
\item {Grp. gram.:v. t.}
\end{itemize}
\begin{itemize}
\item {Grp. gram.:V. i.}
\end{itemize}
\begin{itemize}
\item {Proveniência:(Do lat. \textunderscore pausare\textunderscore )}
\end{itemize}
Pôr; assentar: \textunderscore pousar o pé em falso\textunderscore .
Depor: \textunderscore pousar um fardo\textunderscore .
Estabelecer-se.
Collocar-se.
Hospedar-se, albergar-se.
Empoleirar-se.
Estar assente.
Parar.
Descansar.
Residir.
Acoitar-se.
\section{Pousar}
\begin{itemize}
\item {Grp. gram.:v. t.}
\end{itemize}
\begin{itemize}
\item {Utilização:T. de curtidor}
\end{itemize}
Alisar com luneta e pedra-pomes (o carnaz da pelle).
(Relaciona-se com \textunderscore poisar\textunderscore ?)
\section{Pouseiro}
\begin{itemize}
\item {Grp. gram.:m.}
\end{itemize}
\begin{itemize}
\item {Utilização:Des.}
\end{itemize}
\begin{itemize}
\item {Grp. gram.:Adj.}
\end{itemize}
\begin{itemize}
\item {Utilização:Prov.}
\end{itemize}
\begin{itemize}
\item {Utilização:minh.}
\end{itemize}
\begin{itemize}
\item {Proveniência:(De \textunderscore pousar\textunderscore )}
\end{itemize}
Nádegas, o mesmo que \textunderscore pousadeiro\textunderscore ^2.
Pacato:«\textunderscore séria, pouseira e sensaborona...\textunderscore »Camillo, \textunderscore Novellas\textunderscore , I, 216.
O mesmo que \textunderscore sedentário\textunderscore .
Costumado a pousar; que vai pousar.
\section{Pousio}
\begin{itemize}
\item {Grp. gram.:m.}
\end{itemize}
\begin{itemize}
\item {Grp. gram.:Adj.}
\end{itemize}
\begin{itemize}
\item {Proveniência:(De \textunderscore pouso\textunderscore )}
\end{itemize}
Interrupção da cultura de uma terra, por um ou mais annos.
Terreno, cuja cultura se interrompeu, para que elle depois se torne mais pingue.
Inculto: \textunderscore terras pousias\textunderscore .
\section{Pouso}
\begin{itemize}
\item {Grp. gram.:m.}
\end{itemize}
\begin{itemize}
\item {Utilização:Bras}
\end{itemize}
\begin{itemize}
\item {Utilização:Bras}
\end{itemize}
\begin{itemize}
\item {Grp. gram.:M. Pl.}
\end{itemize}
\begin{itemize}
\item {Proveniência:(De \textunderscore pousar\textunderscore )}
\end{itemize}
Lugar onde alguma pessôa ou coisa se pousa ou colloca.
Ancoradoiro.
Pedra, sôbre que gira a mó das azenhas.
Telheiro ou choça, á beira dos caminhos, para abrigo de viandantes.
Rancho.
O mesmo que \textunderscore pêso\textunderscore  (do lagar).
Travessa de madeira, em que assenta a quilha do navio no estaleiro.
\section{Pouta}
\begin{itemize}
\item {Grp. gram.:f.}
\end{itemize}
\begin{itemize}
\item {Proveniência:(De \textunderscore poutar\textunderscore )}
\end{itemize}
Objecto pesado, preso á extremidade de um cabo, e que serve de âncora aos barqueiros.
\section{Poutar}
\begin{itemize}
\item {Grp. gram.:v. t.}
\end{itemize}
\begin{itemize}
\item {Proveniência:(Do lat. \textunderscore pultare\textunderscore ?)}
\end{itemize}
Segurar com pouta, ancorar (um barco).
\section{Povaréu}
\begin{itemize}
\item {Grp. gram.:m.}
\end{itemize}
(V.povoléu)
\section{Pòveira}
\begin{itemize}
\item {Grp. gram.:f.}
\end{itemize}
\begin{itemize}
\item {Proveniência:(De \textunderscore Póvoa\textunderscore , n. p.)}
\end{itemize}
Lancha, do typo das usadas pelos pescadores da Póvoa-de-Varzim.
\section{Pòveiro}
\begin{itemize}
\item {Grp. gram.:adj.}
\end{itemize}
\begin{itemize}
\item {Grp. gram.:M.}
\end{itemize}
\begin{itemize}
\item {Proveniência:(De \textunderscore Póvoa\textunderscore , n. p.)}
\end{itemize}
Relativo aos pescadores ou habitantes da Póvoa-de-Varzim.
Pescador ou habitante da Póvoa-de-Varzim.
Barco de Ovar.
\section{Póvero}
\begin{itemize}
\item {Grp. gram.:adj.}
\end{itemize}
\begin{itemize}
\item {Utilização:Ant.}
\end{itemize}
\begin{itemize}
\item {Proveniência:(It. \textunderscore povero\textunderscore )}
\end{itemize}
Pobre.
\section{Poviléu}
\begin{itemize}
\item {Grp. gram.:m.}
\end{itemize}
O mesmo que \textunderscore povoléu\textunderscore .
\section{Povo}
\begin{itemize}
\item {fónica:pô}
\end{itemize}
\begin{itemize}
\item {Grp. gram.:m.}
\end{itemize}
\begin{itemize}
\item {Utilização:Fig.}
\end{itemize}
\begin{itemize}
\item {Proveniência:(Do lat. \textunderscore populus\textunderscore )}
\end{itemize}
Conjunto dos habitantes de um país, sujeitos ás mesmas leis.
Habitantes de uma localidade.
Pequena povoação.
Multidão de gente.
A classe inferior e mais numerosa de um país.
A classe inferior da sociedade; plebe.
O público.
Grande número.
\section{Póvoa}
\begin{itemize}
\item {Grp. gram.:f.}
\end{itemize}
\begin{itemize}
\item {Proveniência:(Do lat. \textunderscore popula\textunderscore )}
\end{itemize}
Pequena povoação; pequeno povo; casal.
\section{Povoação}
\begin{itemize}
\item {Grp. gram.:f.}
\end{itemize}
\begin{itemize}
\item {Proveniência:(Lat. \textunderscore populatio\textunderscore )}
\end{itemize}
Acto ou effeito de povoar.
Habitantes de uma região ou localidade.
Lugar povoado.
\section{Povoado}
\begin{itemize}
\item {Grp. gram.:m.}
\end{itemize}
\begin{itemize}
\item {Proveniência:(De \textunderscore povoar\textunderscore )}
\end{itemize}
Pequeno lugar povoado.
\section{Povoador}
\begin{itemize}
\item {Grp. gram.:m.  e  adj.}
\end{itemize}
O que povôa.
\section{Povoamento}
\begin{itemize}
\item {Grp. gram.:m.}
\end{itemize}
\begin{itemize}
\item {Utilização:Bras}
\end{itemize}
\begin{itemize}
\item {Proveniência:(De \textunderscore povoar\textunderscore )}
\end{itemize}
O mesmo que \textunderscore população\textunderscore : \textunderscore Directoria Geral do serviço de povoamento\textunderscore .
\section{Povoança}
\begin{itemize}
\item {Grp. gram.:f.}
\end{itemize}
\begin{itemize}
\item {Utilização:Ant.}
\end{itemize}
\begin{itemize}
\item {Proveniência:(De \textunderscore povoar\textunderscore )}
\end{itemize}
O mesmo que \textunderscore povoação\textunderscore .
\section{Povoar}
\begin{itemize}
\item {Grp. gram.:v. t.}
\end{itemize}
\begin{itemize}
\item {Proveniência:(De \textunderscore povo\textunderscore )}
\end{itemize}
Tornar habitado, dar povoação a.
Encher de habitantes.
Disseminar animaes para reproducção em.
Dispor grande quantidade de árvores em.
Prover de.
Dotar.
Infundir sentimento ou ideias em abundância em (o coração, mente, etc.): \textunderscore ...a mente povoada de illusões...\textunderscore ; \textunderscore ...povoada de sombras a imaginação...\textunderscore 
\section{Pòvoeira}
\begin{itemize}
\item {Grp. gram.:f.}
\end{itemize}
\begin{itemize}
\item {Proveniência:(De \textunderscore Póvoa-de-Varzim\textunderscore , n. p.)}
\end{itemize}
Espécie de barco de pesca.
\section{Povoléu}
\begin{itemize}
\item {Grp. gram.:m.}
\end{itemize}
\begin{itemize}
\item {Proveniência:(De \textunderscore povo\textunderscore )}
\end{itemize}
Plebe; populacho; arraia-miúda. Cf. Castilho, \textunderscore Montalverne\textunderscore .
\section{Pozolana}
\begin{itemize}
\item {Grp. gram.:f.}
\end{itemize}
\begin{itemize}
\item {Proveniência:(It. \textunderscore pozzolana\textunderscore , de \textunderscore Pozzuoli\textunderscore , n. p.)}
\end{itemize}
Terra vulcânica que, misturada com cal, serve para cimento.
\section{Pozolânico}
\begin{itemize}
\item {Grp. gram.:adj.}
\end{itemize}
Relativo á pozolana.
\section{Pozolita}
\begin{itemize}
\item {Grp. gram.:f.}
\end{itemize}
Variedade de pozolana.
\section{Pozolito}
\begin{itemize}
\item {Grp. gram.:m.}
\end{itemize}
O mesmo ou melhor que \textunderscore pozolita\textunderscore .
\section{Pozzolana}
\begin{itemize}
\item {Grp. gram.:f.}
\end{itemize}
\begin{itemize}
\item {Proveniência:(It. \textunderscore pozzolana\textunderscore , de \textunderscore Pozzuoli\textunderscore , n. p.)}
\end{itemize}
Terra vulcânica que, misturada com cal, serve para cimento.
\section{Pozzolânico}
\begin{itemize}
\item {Grp. gram.:adj.}
\end{itemize}
Relativo á pozzolana.
\section{Pozzolita}
\begin{itemize}
\item {Grp. gram.:f.}
\end{itemize}
Variedade de pozzolana.
\section{Pozzolito}
\begin{itemize}
\item {Grp. gram.:m.}
\end{itemize}
O mesmo ou melhor que \textunderscore pozzolita\textunderscore .
\section{Praça}
\begin{itemize}
\item {Grp. gram.:f.}
\end{itemize}
\begin{itemize}
\item {Utilização:Pop.}
\end{itemize}
\begin{itemize}
\item {Utilização:T. da Áfr. Or. Port}
\end{itemize}
\begin{itemize}
\item {Grp. gram.:Loc. adv.}
\end{itemize}
\begin{itemize}
\item {Utilização:ant.}
\end{itemize}
\begin{itemize}
\item {Proveniência:(Lat. \textunderscore platea\textunderscore )}
\end{itemize}
Lugar público, cercado de edifícios.
Rossio.
Largo.
Mercado.
Circo.
Terreiro.
Conjunto de negociantes de uma cidade.
Almoéda, hasta pública: \textunderscore pôr em praça\textunderscore .
Espaço de um navio, para transporte de gêneros.
Lugar público, onde estacionam trens de aluguer.
Lugar, em que se fazem exercícios militares.
Alistamento militar: \textunderscore assentar praça\textunderscore .
Soldado, sem graduação ou patente: \textunderscore chegaram 30 praças\textunderscore .
Villa ou cidade fortificada.
Fortaleza.
Cêrco.
Alarde, ostentação:«\textunderscore tinha virtudes cívicas, de que não fez praça deante do público\textunderscore ». Camillo, \textunderscore Noites de Lam.\textunderscore , 227.
Pessôa velhaca.
Fazenda ou quinta, no Transvaal.
Casta de uva branca trasmontana.
\textunderscore De praça\textunderscore , publicamente.
\section{Praçaria}
\begin{itemize}
\item {Grp. gram.:f.}
\end{itemize}
O mesmo que \textunderscore parçaria\textunderscore , (por metáth.). Cf. \textunderscore Peregrinação\textunderscore , XXV.
\section{Pracear}
\begin{itemize}
\item {Grp. gram.:v. t.}
\end{itemize}
Pôr em praça; fazer leilão de. Cf. Th. Ribeiro, \textunderscore Dom Jaime\textunderscore .
\section{Pracebo}
\begin{itemize}
\item {fónica:cê}
\end{itemize}
\begin{itemize}
\item {Grp. gram.:m.}
\end{itemize}
\begin{itemize}
\item {Utilização:Ant.}
\end{itemize}
\begin{itemize}
\item {Proveniência:(Do lat. \textunderscore placebo\textunderscore ?)}
\end{itemize}
Offício de defuntos.
\section{Praceiramente}
\begin{itemize}
\item {Grp. gram.:adv.}
\end{itemize}
\begin{itemize}
\item {Utilização:Des.}
\end{itemize}
\begin{itemize}
\item {Proveniência:(De \textunderscore praceiro\textunderscore )}
\end{itemize}
Na praça, em público, publicamente.
\section{Praceiro}
\begin{itemize}
\item {Grp. gram.:adj.}
\end{itemize}
\begin{itemize}
\item {Utilização:Des.}
\end{itemize}
Relativo a praça.
Público, á vista de todos.
\section{Pracejar}
\begin{itemize}
\item {Grp. gram.:v. t.}
\end{itemize}
Fazer praça de, ostentar, alardear.
\section{Praceta}
\begin{itemize}
\item {fónica:cê}
\end{itemize}
\begin{itemize}
\item {Grp. gram.:f.}
\end{itemize}
\begin{itemize}
\item {Utilização:P. us.}
\end{itemize}
Praça pequena.
\section{Pracista}
\begin{itemize}
\item {Grp. gram.:m.}
\end{itemize}
\begin{itemize}
\item {Utilização:Bras}
\end{itemize}
\begin{itemize}
\item {Proveniência:(De \textunderscore praça\textunderscore )}
\end{itemize}
Homem de alguma educação ou que tem vivido em cidades.
\section{Pracrítico}
\begin{itemize}
\item {Grp. gram.:adj.}
\end{itemize}
Relativo ao prácrito.
\section{Prácrito}
\begin{itemize}
\item {Grp. gram.:m.}
\end{itemize}
\begin{itemize}
\item {Proveniência:(Do sânscr. \textunderscore prâkrita\textunderscore )}
\end{itemize}
Língua popular da Índia, derivada do sânscrito.
\section{Pradaria}
\begin{itemize}
\item {Grp. gram.:f.}
\end{itemize}
Série de prados; grande planície.
\section{Prado}
\begin{itemize}
\item {Grp. gram.:m.}
\end{itemize}
\begin{itemize}
\item {Utilização:Bras. do S}
\end{itemize}
\begin{itemize}
\item {Proveniência:(Do lat. \textunderscore pratum\textunderscore )}
\end{itemize}
Campo, coberto de plantas herbáceas, que servem para pastagem; campo.
O mesmo que \textunderscore hippódromo\textunderscore .
\section{Pradoso}
\begin{itemize}
\item {Grp. gram.:adj.}
\end{itemize}
Em que há prados.
Semelhante ao prado; arrelvado.
\section{Praga}
\begin{itemize}
\item {Grp. gram.:f.}
\end{itemize}
\begin{itemize}
\item {Utilização:Ext.}
\end{itemize}
\begin{itemize}
\item {Utilização:Bras. do Maranhão}
\end{itemize}
\begin{itemize}
\item {Utilização:Ant.}
\end{itemize}
\begin{itemize}
\item {Utilização:Açor}
\end{itemize}
\begin{itemize}
\item {Proveniência:(Do lat. \textunderscore plaga\textunderscore )}
\end{itemize}
Acto de imprecar males contra alguém; maldição.
Grande desgraça; catástrophe.
Coisa ou pessôa que importuna.
Abundância de coisas prejudiciaes ou desagradáveis: \textunderscore a praga dos gafanhotos\textunderscore .
Mosquitos.
O mesmo que \textunderscore chaga\textunderscore .
Os pássaros dos campos, em geral: \textunderscore a praga deu cabo do trigo\textunderscore .
\section{Pragal}
\begin{itemize}
\item {Grp. gram.:m.}
\end{itemize}
Terreno árido, em que só crescem algumas plantas bravias.
Gândara; panascal. Cf. Camillo, \textunderscore Noites de Insómn.\textunderscore , V, 20.
\section{Pragalhar}
\begin{itemize}
\item {Grp. gram.:v. t.}
\end{itemize}
\begin{itemize}
\item {Utilização:Prov.}
\end{itemize}
\begin{itemize}
\item {Utilização:trasm.}
\end{itemize}
O mesmo que \textunderscore praguejar\textunderscore .
\section{Pragamĩo}
\begin{itemize}
\item {Grp. gram.:m.}
\end{itemize}
\begin{itemize}
\item {Utilização:Ant.}
\end{itemize}
O mesmo que \textunderscore pergaminho\textunderscore .
\section{Pragana}
\begin{itemize}
\item {Grp. gram.:f.}
\end{itemize}
\begin{itemize}
\item {Utilização:Prov.}
\end{itemize}
Barba de espiga de cereaes.
Aresta, que o linho, ao maçar-se, larga.
\section{Praganã}
\begin{itemize}
\item {Grp. gram.:m.}
\end{itemize}
Sub-divisão administrativa, espécie de província ou districto, na Índia portuguesa.
\section{Praganoso}
\begin{itemize}
\item {Grp. gram.:adj.}
\end{itemize}
Relativo a pragana; que tem pragana, (falando-se de cereaes).
\section{Pragão}
\begin{itemize}
\item {Grp. gram.:m.}
\end{itemize}
\begin{itemize}
\item {Utilização:Ant.}
\end{itemize}
\begin{itemize}
\item {Proveniência:(De \textunderscore praga\textunderscore )}
\end{itemize}
Grande chaga ou chaga incurável.
\section{Pragar}
\begin{itemize}
\item {Grp. gram.:v. i.}
\end{itemize}
\begin{itemize}
\item {Utilização:Prov.}
\end{itemize}
\begin{itemize}
\item {Utilização:trasm.}
\end{itemize}
O mesmo que \textunderscore praguejar\textunderscore .
\section{Pragmática}
\begin{itemize}
\item {Grp. gram.:f.}
\end{itemize}
\begin{itemize}
\item {Utilização:Ext.}
\end{itemize}
\begin{itemize}
\item {Proveniência:(Lat. \textunderscore pragmatica\textunderscore )}
\end{itemize}
Regulamento, que emanava do poder civil em assumptos ecclesiásticos.
Conjunto de regras ou fórmulas, para ceremónias da côrte ou da igreja.
Etiqueta, formalidades da bôa sociedade.
\section{Pragmaticamente}
\begin{itemize}
\item {Grp. gram.:adv.}
\end{itemize}
De modo pragmático; de acôrdo com a etiqueta.
\section{Pragmático}
\begin{itemize}
\item {Grp. gram.:adj.}
\end{itemize}
\begin{itemize}
\item {Proveniência:(Gr. \textunderscore pragmatikos\textunderscore )}
\end{itemize}
Relativo á pragmática; conforme á pragmática; usual.
\section{Praguedo}
\begin{itemize}
\item {fónica:guê}
\end{itemize}
\begin{itemize}
\item {Grp. gram.:m.}
\end{itemize}
Grande porção de pragas. Cf. Castilho, \textunderscore Geórgicas\textunderscore , 21.
\section{Praguejador}
\begin{itemize}
\item {Grp. gram.:m.  e  adj.}
\end{itemize}
O que pragueja.
\section{Praguejamento}
\begin{itemize}
\item {Grp. gram.:m.}
\end{itemize}
Acto ou effeito de praguejar.
\section{Praguejar}
\begin{itemize}
\item {Grp. gram.:v. i.}
\end{itemize}
\begin{itemize}
\item {Grp. gram.:V. t.}
\end{itemize}
Dizer pragas.
Amaldiçoar, maldizer.
\section{Praguento}
\begin{itemize}
\item {Grp. gram.:adj.}
\end{itemize}
Que pragueja; que costuma dizer pragas.
Maldizente.
\section{Praia}
\begin{itemize}
\item {Grp. gram.:f.}
\end{itemize}
\begin{itemize}
\item {Grp. gram.:Pl.}
\end{itemize}
\begin{itemize}
\item {Utilização:Marn.}
\end{itemize}
\begin{itemize}
\item {Proveniência:(Do lat. \textunderscore plaga\textunderscore )}
\end{itemize}
Orla de terra, geralmente coberta de areia, confinando com o mar.
Beiramar.
Região, banhada pelo mar; litoral; margem.
Depósito geral das águas que alimentam a salina, e que também se chama loiças, (cp. \textunderscore loiça\textunderscore ).
\section{Praia-mar}
\begin{itemize}
\item {Grp. gram.:f.}
\end{itemize}
(Corr. vulgar de \textunderscore preamar\textunderscore )
\section{Praiano}
\begin{itemize}
\item {Grp. gram.:m.}
\end{itemize}
\begin{itemize}
\item {Utilização:Bras}
\end{itemize}
O mesmo que \textunderscore praieiro\textunderscore .
\section{Praieiro}
\begin{itemize}
\item {Grp. gram.:m.}
\end{itemize}
\begin{itemize}
\item {Utilização:Bras}
\end{itemize}
\begin{itemize}
\item {Utilização:Ext.}
\end{itemize}
Habitante de praia ou praias: \textunderscore houve em Pernambuco a revolução dos praieiros...\textunderscore 
Homem liberal, não conservador.
\section{Praina}
\begin{itemize}
\item {Grp. gram.:f.}
\end{itemize}
\begin{itemize}
\item {Utilização:Prov.}
\end{itemize}
\begin{itemize}
\item {Utilização:trasm.}
\end{itemize}
\begin{itemize}
\item {Proveniência:(De \textunderscore praino\textunderscore )}
\end{itemize}
O mesmo que \textunderscore planície\textunderscore .
\section{Praino}
\begin{itemize}
\item {Grp. gram.:m.}
\end{itemize}
\begin{itemize}
\item {Utilização:Des.}
\end{itemize}
O mesmo que \textunderscore plaino\textunderscore ; planalto; chan.
\section{Prairial}
\begin{itemize}
\item {Grp. gram.:m.}
\end{itemize}
\begin{itemize}
\item {Proveniência:(T. fr.)}
\end{itemize}
Um dos meses do calendário da primeira república francesa, (19 de Maio a 18 de Junho).
\section{Prajá}
\begin{itemize}
\item {Grp. gram.:m.}
\end{itemize}
\begin{itemize}
\item {Utilização:Bras}
\end{itemize}
Espécie de doce, feito de melaço e ovos.
(Contr. de \textunderscore para\textunderscore  + \textunderscore já\textunderscore )
\section{Pralina}
\begin{itemize}
\item {Grp. gram.:f.}
\end{itemize}
\begin{itemize}
\item {Proveniência:(Fr. \textunderscore praline\textunderscore )}
\end{itemize}
Amêndoa coberta:«\textunderscore ...licores, café, pralinas.\textunderscore »Castilho, \textunderscore Avarento\textunderscore , III, 5.
\section{Prama}
\begin{itemize}
\item {Grp. gram.:f.}
\end{itemize}
Espécie de embarcação antiga.
\section{Prancha}
\begin{itemize}
\item {Grp. gram.:f.}
\end{itemize}
\begin{itemize}
\item {Utilização:Bras}
\end{itemize}
\begin{itemize}
\item {Proveniência:(Do lat. hyp. \textunderscore plancula\textunderscore , de \textunderscore planca\textunderscore ?)}
\end{itemize}
Tabuão, grande tábua grossa e larga.
Tábua, sôbre que se passa, de um barco para terra.
Circular, que uma loja maçónica envia ás outras.
O mesmo que \textunderscore chalana\textunderscore .
\section{Pranchada}
\begin{itemize}
\item {Grp. gram.:f.}
\end{itemize}
\begin{itemize}
\item {Proveniência:(De \textunderscore prancha\textunderscore )}
\end{itemize}
Pancada com espada ou sabre, de modo que a fôlha, na sua maior largura, assente no ponto em que bate.
Tampa de chumbo para resguardar o ouvido da peça de artilharia.
\section{Pranchão}
\begin{itemize}
\item {Grp. gram.:m.}
\end{itemize}
\begin{itemize}
\item {Utilização:Prov.}
\end{itemize}
\begin{itemize}
\item {Utilização:trasm.}
\end{itemize}
Grande prancha.
O mesmo que \textunderscore tanchão\textunderscore .
\section{Pranche}
\begin{itemize}
\item {Grp. gram.:m.}
\end{itemize}
(?): ...«\textunderscore ...jogar cos pranches pela do vento.\textunderscore »G. Vicente, \textunderscore Triunfo do Inverno\textunderscore .
\section{Pranchear}
\begin{itemize}
\item {Grp. gram.:v. i.}
\end{itemize}
\begin{itemize}
\item {Utilização:Prov.}
\end{itemize}
\begin{itemize}
\item {Utilização:extrem.}
\end{itemize}
\begin{itemize}
\item {Proveniência:(De \textunderscore prancha\textunderscore )}
\end{itemize}
Estender-se ao comprido, chapar-se, (falando-se do cavallo).
\section{Prancheta}
\begin{itemize}
\item {fónica:chê}
\end{itemize}
\begin{itemize}
\item {Grp. gram.:f.}
\end{itemize}
Prancha pequena.
Instrumento topográphico, para o levantamento de plantas.
Parche.
Grande peça rectangular de madeira, sôbre que se colloca, para secar, o telhão, quando sai da prensa.
\section{Prândio}
\begin{itemize}
\item {Grp. gram.:m.}
\end{itemize}
\begin{itemize}
\item {Utilização:Poét.}
\end{itemize}
\begin{itemize}
\item {Proveniência:(Lat. \textunderscore prandium\textunderscore )}
\end{itemize}
Jantar; banquete.
\section{Praneza}
\begin{itemize}
\item {Grp. gram.:f.}
\end{itemize}
\begin{itemize}
\item {Utilização:Prov.}
\end{itemize}
\begin{itemize}
\item {Utilização:trasm.}
\end{itemize}
O mesmo que \textunderscore planície\textunderscore .
(Por \textunderscore planeza\textunderscore )
\section{Pranheira}
\begin{itemize}
\item {Grp. gram.:f.}
\end{itemize}
\begin{itemize}
\item {Utilização:Prov.}
\end{itemize}
\begin{itemize}
\item {Utilização:beir.}
\end{itemize}
Ângulo ou parte da parede sobreposta á lareira, e que negreja pela acção do fumo.
(Cp. \textunderscore paranheira\textunderscore )
\section{Praniza}
\begin{itemize}
\item {Grp. gram.:f.}
\end{itemize}
\begin{itemize}
\item {Proveniência:(Do gr. \textunderscore pranizein\textunderscore )}
\end{itemize}
Gênero de crustáceos insópodes dos mares europeus.
\section{Pranta}
\begin{itemize}
\item {Grp. gram.:f.}
\end{itemize}
\begin{itemize}
\item {Utilização:Ant.}
\end{itemize}
\begin{itemize}
\item {Utilização:Mad}
\end{itemize}
O mesmo que \textunderscore planta\textunderscore .
Ponta da cana do açúcar, metida na terra, para reproducção.
\section{Prantar}
\begin{itemize}
\item {Grp. gram.:v. t.}
\end{itemize}
\begin{itemize}
\item {Utilização:Ant.}
\end{itemize}
Pagar, apresentar, satisfazer por direito.
(Corr. de \textunderscore preitar\textunderscore )
\section{Prantar}
\begin{itemize}
\item {Grp. gram.:v. t.}
\end{itemize}
\begin{itemize}
\item {Utilização:Pop.}
\end{itemize}
O mesmo que \textunderscore plantar\textunderscore .
Pôr, collocar; dispor: \textunderscore prantar o chapéu na cabeça\textunderscore . Cf. D. Franc. Manuel, \textunderscore Carta de Guia\textunderscore , 25; Sousa, \textunderscore Vida do Arceb.\textunderscore , I, 258.
\section{Pranteadeira}
\begin{itemize}
\item {Grp. gram.:f.  e  adj.}
\end{itemize}
\begin{itemize}
\item {Utilização:Ant.}
\end{itemize}
Mulhér que pranteia; carpideira.
\section{Pranteador}
\begin{itemize}
\item {Grp. gram.:m.  e  adj.}
\end{itemize}
O que pranteia.
\section{Pranteadura}
\begin{itemize}
\item {Grp. gram.:f.}
\end{itemize}
\begin{itemize}
\item {Utilização:Fam.}
\end{itemize}
Acto de prantear.
\section{Prantear}
\begin{itemize}
\item {Grp. gram.:v. t.}
\end{itemize}
\begin{itemize}
\item {Grp. gram.:V. i.}
\end{itemize}
Verter pranto por causa de.
Deplorar, lastimar.
Chorar.
\section{Prantivo}
\begin{itemize}
\item {Grp. gram.:adj.}
\end{itemize}
Relativo a pranto.
Plangente; lastimoso:«\textunderscore ...vida, prantiva como os threnos...\textunderscore »S. Monteiro, \textunderscore Auto dos Esquecidos\textunderscore .
(Cp. fr. \textunderscore plaintif\textunderscore )
\section{Pranto}
\begin{itemize}
\item {Grp. gram.:m.}
\end{itemize}
\begin{itemize}
\item {Proveniência:(Do lat. \textunderscore planctus\textunderscore )}
\end{itemize}
Lamentação; chôro.
Acto de lastimar.
\section{Prão}
\begin{itemize}
\item {Grp. gram.:m.  e  adj.}
\end{itemize}
\begin{itemize}
\item {Utilização:Ant.}
\end{itemize}
O mesmo que \textunderscore plano\textunderscore .
\section{Prásina}
\begin{itemize}
\item {Grp. gram.:f.}
\end{itemize}
\begin{itemize}
\item {Proveniência:(De \textunderscore prásino\textunderscore )}
\end{itemize}
Grupo de acrobatas, que nos circos romanos se apresentavam vestidos de verde.
Espécie de terra verde, de que se servem os pintores.
\section{Prásino}
\begin{itemize}
\item {Grp. gram.:adj.}
\end{itemize}
\begin{itemize}
\item {Utilização:Des.}
\end{itemize}
\begin{itemize}
\item {Grp. gram.:M.}
\end{itemize}
\begin{itemize}
\item {Proveniência:(Lat. \textunderscore prasinus\textunderscore )}
\end{itemize}
Verde, porráceo.
A esmeralda.
\section{Prásio}
\begin{itemize}
\item {Grp. gram.:m.}
\end{itemize}
(V.prásino)
\section{Prásio}
\begin{itemize}
\item {Grp. gram.:m.}
\end{itemize}
\begin{itemize}
\item {Proveniência:(Do gr. \textunderscore prasion\textunderscore )}
\end{itemize}
O mesmo que \textunderscore marroio\textunderscore .
\section{Prasmar}
\begin{itemize}
\item {Grp. gram.:v. t.}
\end{itemize}
\begin{itemize}
\item {Utilização:Ant.}
\end{itemize}
\begin{itemize}
\item {Proveniência:(Do cast. \textunderscore blasmar\textunderscore )}
\end{itemize}
Censurar; criticar; arguir.
\section{Prasme}
\begin{itemize}
\item {Grp. gram.:m.}
\end{itemize}
\begin{itemize}
\item {Utilização:Ant.}
\end{itemize}
O mesmo que \textunderscore praz-me\textunderscore .
\section{Prasmo}
\begin{itemize}
\item {Grp. gram.:m.}
\end{itemize}
O mesmo que \textunderscore prasme\textunderscore . Cf. Sim. Mach., fol. 86, v.^o.
\section{Prasmo}
\begin{itemize}
\item {Grp. gram.:m.}
\end{itemize}
\begin{itemize}
\item {Utilização:Ant.}
\end{itemize}
\begin{itemize}
\item {Proveniência:(De \textunderscore prasmar\textunderscore )}
\end{itemize}
Censura.
Mancha; vitupério.
\section{Prastrano}
\begin{itemize}
\item {Grp. gram.:m.  e  adj.}
\end{itemize}
(V.pastrano). Cf. Macedo, \textunderscore Burros\textunderscore , 309.
\section{Prata}
\begin{itemize}
\item {Grp. gram.:f.}
\end{itemize}
\begin{itemize}
\item {Utilização:Bras}
\end{itemize}
\begin{itemize}
\item {Utilização:Bras}
\end{itemize}
\begin{itemize}
\item {Proveniência:(Do lat. hyp. \textunderscore platta\textunderscore )}
\end{itemize}
Metal branco e precioso, muito dúctil e sonoro.
Moédas dêste metal.
Conjunto de objectos, feitos com êsse metal: \textunderscore a prata da casa\textunderscore .
Peixe marítimo.
Espécie de banana, muito apreciada.
\section{Pratada}
\begin{itemize}
\item {Grp. gram.:f.}
\end{itemize}
Aquillo que um prato contém: \textunderscore uma pratada de feijão\textunderscore .
O mesmo que \textunderscore pratalhada\textunderscore .
\section{Pratalhada}
\begin{itemize}
\item {Grp. gram.:f.}
\end{itemize}
\begin{itemize}
\item {Utilização:Fam.}
\end{itemize}
Porção de comida, que enche um prato.
\section{Pratalhaz}
\begin{itemize}
\item {Grp. gram.:m.}
\end{itemize}
\begin{itemize}
\item {Utilização:Pop.}
\end{itemize}
Um prato muito cheio de qualquer iguaria.
\section{Prataria}
\begin{itemize}
\item {Grp. gram.:f.}
\end{itemize}
Porção de pratos.
\section{Prataria}
\begin{itemize}
\item {Grp. gram.:f.}
\end{itemize}
Conjunto de vasos ou utensílios de prata.
\section{Prateação}
\begin{itemize}
\item {Grp. gram.:f.}
\end{itemize}
Acto ou effeito de pratear.
\section{Prateada}
\begin{itemize}
\item {Grp. gram.:f.}
\end{itemize}
O mesmo que \textunderscore erva-do-orvalho\textunderscore .
\section{Prateador}
\begin{itemize}
\item {Grp. gram.:m.  e  adj.}
\end{itemize}
O que prateia.
\section{Prateadura}
\begin{itemize}
\item {Grp. gram.:f.}
\end{itemize}
(V.prateação)
\section{Pratear}
\begin{itemize}
\item {Grp. gram.:v. t.}
\end{itemize}
Revestir de uma camada de prata.
Dar a côr e o brilho da prata a.
\section{Prateira}
\begin{itemize}
\item {Grp. gram.:f.}
\end{itemize}
\begin{itemize}
\item {Utilização:Prov.}
\end{itemize}
\begin{itemize}
\item {Utilização:minh.}
\end{itemize}
Armário ou lugar, onde se guarda baixella de prata.
Pequena bandeja de loiça ou prato covo, em que se guarda o doce corredio, para coalhar.
\section{Prateiro}
\begin{itemize}
\item {Grp. gram.:m.}
\end{itemize}
\begin{itemize}
\item {Utilização:Des.}
\end{itemize}
Aquelle que vende ou fabríca objectos de prata.
\section{Prátel}
\begin{itemize}
\item {Grp. gram.:m.}
\end{itemize}
\begin{itemize}
\item {Utilização:Ant.}
\end{itemize}
O mesmo que \textunderscore prato\textunderscore . Cf. Herculano, \textunderscore Cistér\textunderscore , 257.
\section{Prateleira}
\begin{itemize}
\item {Grp. gram.:f.}
\end{itemize}
\begin{itemize}
\item {Proveniência:(De \textunderscore prátel\textunderscore )}
\end{itemize}
Tábua ou espécie de estante, em que se collocam pratos.
Cada uma das tábuas horizontaes e interiores de um armário ou de móveis semelhantes.
Tábua, presa horizontalmente a uma parede, e em que se collocam objectos vários.
\textunderscore Salto de prateleira\textunderscore , salto de bota ou sapato, com rebordo saliente.
\textunderscore Espora de prateleira\textunderscore , espora grossa, própria para o calçado que tem salto de prateleira.
\section{Prateleiro}
\begin{itemize}
\item {Grp. gram.:m.}
\end{itemize}
O mesmo que \textunderscore prateleira\textunderscore . Cf. \textunderscore Peregrinação\textunderscore , XV.
\section{Pratense}
\begin{itemize}
\item {Grp. gram.:adj.}
\end{itemize}
\begin{itemize}
\item {Proveniência:(Lat. \textunderscore pratensis\textunderscore )}
\end{itemize}
Que nasce ou cresce nos prados.
\section{Prática}
\begin{itemize}
\item {Grp. gram.:f.}
\end{itemize}
\begin{itemize}
\item {Proveniência:(Lat. \textunderscore practica\textunderscore )}
\end{itemize}
Acto ou effeito de praticar.
Uso, praxe.
Experiência.
O saber, filho da experiência.
Rotina.
Applicação da theoria.
Licença, dada aos navegantes, para communicarem com um pôrto ou cidade.
Conversação.
Discurso; conferência.
\section{Praticabilidade}
\begin{itemize}
\item {Grp. gram.:f.}
\end{itemize}
Qualidade de praticável.
\section{Praticagem}
\begin{itemize}
\item {Grp. gram.:f.}
\end{itemize}
\begin{itemize}
\item {Proveniência:(De \textunderscore prático\textunderscore )}
\end{itemize}
O mesmo que \textunderscore pilotagem\textunderscore . Cf. M. de Aguiar, \textunderscore Dicc. de Marinha\textunderscore .
\section{Praticamente}
\begin{itemize}
\item {Grp. gram.:adv.}
\end{itemize}
De modo prático.
\section{Praticante}
\begin{itemize}
\item {Grp. gram.:m.  e  adj.}
\end{itemize}
\begin{itemize}
\item {Grp. gram.:M.}
\end{itemize}
\begin{itemize}
\item {Utilização:T. de Baião}
\end{itemize}
O que pratíca.
Aquelle que se vai exercitando nalgum mestér.
Namorado, conversado.
\section{Praticar}
\begin{itemize}
\item {Grp. gram.:v. t.}
\end{itemize}
\begin{itemize}
\item {Grp. gram.:V. i.}
\end{itemize}
\begin{itemize}
\item {Proveniência:(De \textunderscore prática\textunderscore )}
\end{itemize}
Exercer; exercitar: \textunderscore praticar um offício\textunderscore .
Realizar; levar a effeito: \textunderscore praticar um crime\textunderscore .
Proferir: \textunderscore praticar insolências\textunderscore .
Conversar, conferenciar.
\section{Praticável}
\begin{itemize}
\item {Grp. gram.:adj.}
\end{itemize}
Que se póde praticar.
Transitável: \textunderscore caminhos praticáveis\textunderscore .
\section{Prático}
\begin{itemize}
\item {Grp. gram.:adj.}
\end{itemize}
\begin{itemize}
\item {Grp. gram.:M.}
\end{itemize}
\begin{itemize}
\item {Proveniência:(Lat. \textunderscore practicus\textunderscore )}
\end{itemize}
Relativo á prática.
Que segue mais a experiência que a theoria.
Experiente.
Que encara as coisas pelo lado positivo.
Piloto, que conhece bem certas paragens maritimas.
\section{Pratícola}
\begin{itemize}
\item {Grp. gram.:adj.}
\end{itemize}
\begin{itemize}
\item {Proveniência:(Do lat. \textunderscore pratum\textunderscore  + \textunderscore colere\textunderscore )}
\end{itemize}
Que vive nos prados; relativo á cultura dos prados.
\section{Praticultor}
\begin{itemize}
\item {Grp. gram.:m.}
\end{itemize}
Aquelle que se occupa de praticultura.
\section{Praticultura}
\begin{itemize}
\item {Grp. gram.:f.}
\end{itemize}
\begin{itemize}
\item {Proveniência:(Do lat. \textunderscore pratum\textunderscore  + \textunderscore cultura\textunderscore )}
\end{itemize}
Parte da Agricultura, que trata especialmente de pastagens ou forragens.
\section{Pratilheiro}
\begin{itemize}
\item {Grp. gram.:m.}
\end{itemize}
\begin{itemize}
\item {Proveniência:(De \textunderscore prato\textunderscore )}
\end{itemize}
Aquelle que toca pratos numa orchestra ou banda.
\section{Pratinho}
\begin{itemize}
\item {Grp. gram.:m.}
\end{itemize}
\begin{itemize}
\item {Utilização:Fig.}
\end{itemize}
Pequeno prato.
O que serve para objecto de motejo ou de entretenimento; joguete.
\section{Prato}
\begin{itemize}
\item {Grp. gram.:m.}
\end{itemize}
\begin{itemize}
\item {Utilização:Ext.}
\end{itemize}
\begin{itemize}
\item {Grp. gram.:Pl.}
\end{itemize}
\begin{itemize}
\item {Utilização:Prov.}
\end{itemize}
\begin{itemize}
\item {Utilização:alent.}
\end{itemize}
\begin{itemize}
\item {Grp. gram.:Adj.}
\end{itemize}
\begin{itemize}
\item {Proveniência:(Do lat. hyp. \textunderscore plattus\textunderscore )}
\end{itemize}
Vaso pouco fundo, geralmente circular e mais ou menos largo, em que se serve o comer.
Cada uma das iguarias que entram numa refeição: \textunderscore ao jantar, bastam-lhe três pratos\textunderscore .
Cada uma das duas lâminas circulares, que servem na balança ordinária.
Peça de vários maquinismos, com a fórma semelhante á do prato.
Espécie de jôgo, o mesmo que \textunderscore tacho\textunderscore . (V. \textunderscore tacho\textunderscore )
O mesmo que \textunderscore alimentação\textunderscore : \textunderscore o tio dá-lhe 3:000 reis para prato\textunderscore .
\textunderscore Prato de meio\textunderscore , iguaria, que vai á mesa, entre a sopa ou o cozido e a sobremesa.
Instrumento musical, formado de duas peças circulares de metal.
Disco de ferro, entre as rodas e as chedas do carro.
Chato: \textunderscore queijo prato\textunderscore .
\section{Pravamente}
\begin{itemize}
\item {Grp. gram.:adv.}
\end{itemize}
\begin{itemize}
\item {Proveniência:(De um hypoth. \textunderscore pravo\textunderscore , do lat. \textunderscore pravus\textunderscore )}
\end{itemize}
Com pravidade; com perversidade Cf. Camillo, \textunderscore Ôlho de Vidro\textunderscore , 52.
\section{Pravidade}
\begin{itemize}
\item {Grp. gram.:f.}
\end{itemize}
\begin{itemize}
\item {Proveniência:(Lat. \textunderscore pravitas\textunderscore )}
\end{itemize}
Qualidade do que é mau ou perverso; ruindade.
\section{Praxe}
\begin{itemize}
\item {Grp. gram.:f.}
\end{itemize}
\begin{itemize}
\item {Proveniência:(Lat. \textunderscore praxis\textunderscore )}
\end{itemize}
O que se pratica habitualmente; prática, uso; pragmática.
\section{Praxista}
\begin{itemize}
\item {Grp. gram.:m. ,  f.  e  adj.}
\end{itemize}
\begin{itemize}
\item {Utilização:Restrict.}
\end{itemize}
Pessôa, que conhece ou segue as praxes.
O que é versado nas praxes do foro.
\section{Prazente}
\begin{itemize}
\item {Grp. gram.:adj.}
\end{itemize}
\begin{itemize}
\item {Utilização:Ant.}
\end{itemize}
\begin{itemize}
\item {Proveniência:(De \textunderscore prazer\textunderscore )}
\end{itemize}
Aprazível.
\section{Prazentearia}
\begin{itemize}
\item {Grp. gram.:f.}
\end{itemize}
\begin{itemize}
\item {Utilização:Ant.}
\end{itemize}
Acto de prazentear. Cf. Frei Fortun. \textunderscore Inéd.\textunderscore , 312.
\section{Prazentear}
\begin{itemize}
\item {Grp. gram.:v. t.}
\end{itemize}
\begin{itemize}
\item {Grp. gram.:V. i.}
\end{itemize}
\begin{itemize}
\item {Proveniência:(De \textunderscore prazente\textunderscore )}
\end{itemize}
Adular; bajular.
Gracejar.
\section{Prazenteio}
\begin{itemize}
\item {Grp. gram.:m.}
\end{itemize}
Acto de prazentear; adulação, lisonja. Cf. \textunderscore Port. Mon. Hist.\textunderscore , \textunderscore Script.\textunderscore , 278.
\section{Prazenteiramente}
\begin{itemize}
\item {Grp. gram.:adv.}
\end{itemize}
De modo prazenteiro; com affabilidade.
\section{Prazenteiro}
\begin{itemize}
\item {Grp. gram.:adj.}
\end{itemize}
\begin{itemize}
\item {Proveniência:(De \textunderscore prazente\textunderscore )}
\end{itemize}
Que revela prazer; jovial; affável; sympáthico.
\section{Prazentim}
\begin{itemize}
\item {Grp. gram.:m.}
\end{itemize}
\begin{itemize}
\item {Utilização:Ant.}
\end{itemize}
O mesmo que \textunderscore placentino\textunderscore .
\section{Prazer}
\begin{itemize}
\item {Grp. gram.:v. i.}
\end{itemize}
\begin{itemize}
\item {Grp. gram.:M.}
\end{itemize}
\begin{itemize}
\item {Proveniência:(Do lat. \textunderscore placere\textunderscore )}
\end{itemize}
Comprazer, agradar, aprazer: \textunderscore praz-me lêr versos\textunderscore .
Estado de quem se acha prazenteiro; alegria; jovialidade; satisfação; delícia; aprazimento; agrado.
Entretenimento, divertimento.
\section{Prazerosamente}
\begin{itemize}
\item {Grp. gram.:adv.}
\end{itemize}
\begin{itemize}
\item {Utilização:bras}
\end{itemize}
\begin{itemize}
\item {Utilização:Neol.}
\end{itemize}
De modo prazeroso. Cf. Boscoli, \textunderscore Gram.\textunderscore , (na dedicatória).
\section{Prazeroso}
\begin{itemize}
\item {Grp. gram.:adj.}
\end{itemize}
\begin{itemize}
\item {Utilização:bras}
\end{itemize}
\begin{itemize}
\item {Utilização:Neol.}
\end{itemize}
Em que há prazer; prazenteiro.
\section{Prazia}
\begin{itemize}
\item {Grp. gram.:f.}
\end{itemize}
\begin{itemize}
\item {Utilização:Ant.}
\end{itemize}
\begin{itemize}
\item {Utilização:Pop.}
\end{itemize}
\begin{itemize}
\item {Proveniência:(De \textunderscore prazer\textunderscore )}
\end{itemize}
Jeito, modo; ensejo.
\section{Prazimento}
\begin{itemize}
\item {Grp. gram.:m.}
\end{itemize}
\begin{itemize}
\item {Proveniência:(Do b. lat. \textunderscore placimentum\textunderscore )}
\end{itemize}
O mesmo que \textunderscore aprazimento\textunderscore .
\section{Prazível}
\begin{itemize}
\item {Grp. gram.:adj.}
\end{itemize}
\begin{itemize}
\item {Utilização:P. us.}
\end{itemize}
\begin{itemize}
\item {Proveniência:(De \textunderscore prazer\textunderscore )}
\end{itemize}
O mesmo que \textunderscore aprazível\textunderscore .
\section{Praz-me}
\begin{itemize}
\item {Grp. gram.:m.}
\end{itemize}
\begin{itemize}
\item {Utilização:Ant.}
\end{itemize}
O mesmo que \textunderscore beneplácito\textunderscore .
Despacho.
Portaria.
\section{Prazo}
\begin{itemize}
\item {Grp. gram.:m.}
\end{itemize}
\begin{itemize}
\item {Utilização:Ant.}
\end{itemize}
\begin{itemize}
\item {Utilização:Ext.}
\end{itemize}
\begin{itemize}
\item {Proveniência:(Do lat. \textunderscore placitum\textunderscore )}
\end{itemize}
Tempo determinado.
Tempo, durante o qual se deverá realizar alguma coisa: \textunderscore apresentou-se dentro do prazo legal\textunderscore .
O termo de um determinado período de tempo.
Aforamento.
Prédio emphytêutico: \textunderscore vender um prazo\textunderscore .
Esphéra de jurisdicção ou influência:«\textunderscore ...a fim de meter Moluco no seu prazo...\textunderscore »Lucena, \textunderscore San Francisco Xavier\textunderscore , l. III, c. V.
Ónus; gravame; embaraço.
\section{Pre...}
\begin{itemize}
\item {Grp. gram.:pref.}
\end{itemize}
\begin{itemize}
\item {Proveniência:(Lat. \textunderscore prae\textunderscore )}
\end{itemize}
(designativo de \textunderscore antecedência\textunderscore , \textunderscore antecipação\textunderscore , \textunderscore preferência\textunderscore )
\section{Pré}
\begin{itemize}
\item {Grp. gram.:m.}
\end{itemize}
\begin{itemize}
\item {Proveniência:(Fr. \textunderscore pret\textunderscore )}
\end{itemize}
O vencimento diário de um soldado.
\section{Preá}
\begin{itemize}
\item {Grp. gram.:f.}
\end{itemize}
\begin{itemize}
\item {Utilização:Bras. da Baía}
\end{itemize}
O mesmo que \textunderscore roedor\textunderscore .
\section{Preadamita}
\begin{itemize}
\item {Grp. gram.:m.}
\end{itemize}
\begin{itemize}
\item {Proveniência:(De \textunderscore pre...\textunderscore  + \textunderscore Adão\textunderscore , n. p.)}
\end{itemize}
Diz-se dos homens que, segundo alguns autores, existiram antes de Adão.
\section{Preadivinhar}
\begin{itemize}
\item {Grp. gram.:v. t.}
\end{itemize}
\begin{itemize}
\item {Proveniência:(De \textunderscore pre...\textunderscore  + \textunderscore adivinhar\textunderscore )}
\end{itemize}
Vêr com antecipação, prever; conhecer antecipadamente.
\section{Preagónico}
\begin{itemize}
\item {Grp. gram.:adj.}
\end{itemize}
\begin{itemize}
\item {Utilização:Neol.}
\end{itemize}
\begin{itemize}
\item {Proveniência:(De \textunderscore pre...\textunderscore  + \textunderscore agonia\textunderscore )}
\end{itemize}
Que precede a agonia ou a morte: \textunderscore o canto preagónico do cysne\textunderscore .
\section{Prealegar}
\begin{itemize}
\item {Grp. gram.:v. t.}
\end{itemize}
\begin{itemize}
\item {Proveniência:(De \textunderscore pre...\textunderscore  + \textunderscore alegar\textunderscore )}
\end{itemize}
Alegar previamente.
\section{Preallegar}
\begin{itemize}
\item {Grp. gram.:v. t.}
\end{itemize}
\begin{itemize}
\item {Proveniência:(De \textunderscore pre...\textunderscore  + \textunderscore allegar\textunderscore )}
\end{itemize}
Allegar previamente.
\section{Preamar}
\begin{itemize}
\item {Grp. gram.:f.}
\end{itemize}
\begin{itemize}
\item {Proveniência:(De \textunderscore pleno\textunderscore  + \textunderscore mar\textunderscore ? Cp. cast. \textunderscore pleamar\textunderscore )}
\end{itemize}
O ponto mais alto a que sobe a maré; maré cheia.
\section{Preambular}
\begin{itemize}
\item {Grp. gram.:adj.}
\end{itemize}
Relativo a preâmbulo.
Que serve de preâmbulo.
Que tem fórma de preâmbulo: \textunderscore discurso preambular\textunderscore .
\section{Preambular}
\begin{itemize}
\item {Grp. gram.:v. t.}
\end{itemize}
Fazer preâmbulo a; prefaciar; proemiar: \textunderscore preambular um poêma\textunderscore .
\section{Preâmbulo}
\begin{itemize}
\item {Grp. gram.:m.}
\end{itemize}
\begin{itemize}
\item {Proveniência:(Lat. \textunderscore praeambulum\textunderscore )}
\end{itemize}
Prefácio, introducção; discurso preliminar.
Relatório preliminar.
Parte preliminar.
\section{Preannunciação}
\begin{itemize}
\item {Grp. gram.:f.}
\end{itemize}
Acto de preannunciar.
\section{Preannunciador}
\begin{itemize}
\item {Grp. gram.:m.  e  adj.}
\end{itemize}
O que preannuncia.
\section{Preannunciar}
\begin{itemize}
\item {Grp. gram.:v. t.}
\end{itemize}
\begin{itemize}
\item {Proveniência:(De \textunderscore pre...\textunderscore  + \textunderscore annunciar\textunderscore )}
\end{itemize}
O mesmo que \textunderscore prenunciar\textunderscore ; annunciar prèviamente.
\section{Preantepenúltimo}
\begin{itemize}
\item {Grp. gram.:adj.}
\end{itemize}
Anterior ao antepenúltimo. Cf. Rui Barb., \textunderscore Réplica\textunderscore , 158.
\section{Preanunciação}
\begin{itemize}
\item {Grp. gram.:f.}
\end{itemize}
Acto de preanunciar.
\section{Preanunciador}
\begin{itemize}
\item {Grp. gram.:m.  e  adj.}
\end{itemize}
O que preanuncia.
\section{Preanunciar}
\begin{itemize}
\item {Grp. gram.:v. t.}
\end{itemize}
\begin{itemize}
\item {Proveniência:(De \textunderscore pre...\textunderscore  + \textunderscore anunciar\textunderscore )}
\end{itemize}
O mesmo que \textunderscore prenunciar\textunderscore ; anunciar prèviamente.
\section{Prear}
\begin{itemize}
\item {Grp. gram.:v. t.}
\end{itemize}
\begin{itemize}
\item {Grp. gram.:V. i.}
\end{itemize}
\begin{itemize}
\item {Proveniência:(Do lat. \textunderscore praedare\textunderscore )}
\end{itemize}
Prender; agarrar.
Fazer presa.
\section{Prear-se}
\begin{itemize}
\item {Grp. gram.:v. p.}
\end{itemize}
\begin{itemize}
\item {Utilização:Prov.}
\end{itemize}
\begin{itemize}
\item {Utilização:beir.}
\end{itemize}
Zangar-se, irritar-se.
(Colhido no Fundão)
\section{Prebeber}
\begin{itemize}
\item {Grp. gram.:v. t.}
\end{itemize}
\begin{itemize}
\item {Utilização:P. us.}
\end{itemize}
\begin{itemize}
\item {Proveniência:(De \textunderscore pre...\textunderscore  + \textunderscore beber\textunderscore )}
\end{itemize}
O mesmo que \textunderscore prelibar\textunderscore .
\section{Prebenda}
\begin{itemize}
\item {Grp. gram.:f.}
\end{itemize}
\begin{itemize}
\item {Utilização:Ext.}
\end{itemize}
\begin{itemize}
\item {Utilização:Fig.}
\end{itemize}
\begin{itemize}
\item {Proveniência:(Lat. \textunderscore praebenda\textunderscore )}
\end{itemize}
Rendimento de um canonicato; canonicato.
Renda ecclesiástica.
Occupação rendosa, mas pouco trabalhosa; sinecura.
\section{Prebendado}
\begin{itemize}
\item {Grp. gram.:m.  e  adj.}
\end{itemize}
Indivíduo, que tem prebenda.
\section{Prebendaria}
\begin{itemize}
\item {Grp. gram.:f.}
\end{itemize}
\begin{itemize}
\item {Proveniência:(De \textunderscore prebenda\textunderscore )}
\end{itemize}
Cargo de prebendeiro.
\section{Prebendário}
\begin{itemize}
\item {Grp. gram.:m.}
\end{itemize}
Aquelle que tinha prebenda; prebendado; o mesmo que \textunderscore prebendeiro\textunderscore . Cf. Herculano, \textunderscore Hist. de Port.\textunderscore , II, 106.
\section{Prebendeiro}
\begin{itemize}
\item {Grp. gram.:m.}
\end{itemize}
Arrematante de prebendas ou das rendas de um bispado.
\section{Prebostado}
\begin{itemize}
\item {Grp. gram.:m.}
\end{itemize}
Cargo de preboste.
\section{Prebostal}
\begin{itemize}
\item {Grp. gram.:adj.}
\end{itemize}
Relativo a preboste. Cf. Garrett, \textunderscore Port. na Balança\textunderscore , 200.
\section{Preboste}
\begin{itemize}
\item {Grp. gram.:m.}
\end{itemize}
\begin{itemize}
\item {Proveniência:(Do fr. ant. \textunderscore prevost\textunderscore )}
\end{itemize}
Antigo magistrado de justiça militar.
\section{Precação}
\begin{itemize}
\item {Grp. gram.:f.}
\end{itemize}
\begin{itemize}
\item {Proveniência:(Lat. \textunderscore precatio\textunderscore )}
\end{itemize}
Rogação, súpplica, deprecação.
\section{Precâmbrico}
\begin{itemize}
\item {Grp. gram.:adj.}
\end{itemize}
\begin{itemize}
\item {Utilização:Geol.}
\end{itemize}
\begin{itemize}
\item {Proveniência:(De \textunderscore pre...\textunderscore  + \textunderscore cámbrico\textunderscore )}
\end{itemize}
Anterior ao idioma câmbrico.
Diz-se do primeiro dos cinco períodos, em que se divide a era geológica primária; e diz-se dos terrenos dêsse período, immediatamente anteriores aos silúricos.
\section{Precantar}
\begin{itemize}
\item {Grp. gram.:v. t.}
\end{itemize}
\begin{itemize}
\item {Proveniência:(De \textunderscore pre...\textunderscore  + \textunderscore cantar\textunderscore )}
\end{itemize}
Vaticinar em verso.
\section{Preçar}
\begin{itemize}
\item {Grp. gram.:v. t.}
\end{itemize}
\begin{itemize}
\item {Utilização:Ant.}
\end{itemize}
O mesmo que \textunderscore prezar\textunderscore :«\textunderscore ...meu muito preçado e amado tio\textunderscore ». \textunderscore Doc. da T. do Tombo\textunderscore , Chancel. Aff. V, l. V, f. 18.
\section{Precária}
\begin{itemize}
\item {Grp. gram.:f.}
\end{itemize}
\begin{itemize}
\item {Utilização:Ant.}
\end{itemize}
Doação de certos bens a um estabelecimento de piedade, com a condição de serem fruídos pelo doador ou seus successores até certo tempo, pagando o mesmo doador ou seus successores uma pensão annual, até expirar aquelle prazo.
Rôgo.
Serviço que se pede.
Geira.
(Cp. \textunderscore precário\textunderscore )
\section{Precariamente}
\begin{itemize}
\item {Grp. gram.:adv.}
\end{itemize}
De modo precário; com risco; de modo incerto ou arriscado.
\section{Precário}
\begin{itemize}
\item {Grp. gram.:adj.}
\end{itemize}
\begin{itemize}
\item {Utilização:Ant.}
\end{itemize}
\begin{itemize}
\item {Proveniência:(Lat. \textunderscore precarius\textunderscore )}
\end{itemize}
Diffícil.
Minguado.
Pouco durável.
Incerto.
Vário.
Frágil.
Escasso.
Sujeito a eventualidades.
\textunderscore Conceder alguma coisa a título precário\textunderscore , concedê-la, com direito a rehavê-la, sem pagar indemnização.
Dizia-se do colono parciário.(V.parciário). Cf. Herculano, \textunderscore Hist. de Port.\textunderscore , III, 299.
\section{Preçário}
\begin{itemize}
\item {Grp. gram.:m.}
\end{itemize}
\begin{itemize}
\item {Proveniência:(De \textunderscore preço\textunderscore )}
\end{itemize}
Relação de preços, especialmente a relação dos preços autorizados para os medicamentos de pharmácia.
\section{Precatadamente}
\begin{itemize}
\item {Grp. gram.:adv.}
\end{itemize}
De modo precatado.
\section{Precatado}
\begin{itemize}
\item {Grp. gram.:adj.}
\end{itemize}
Cauteloso; que mostra precaução.
\section{Precatar}
\begin{itemize}
\item {Grp. gram.:v. t.}
\end{itemize}
\begin{itemize}
\item {Proveniência:(Do lat. \textunderscore praecautus\textunderscore )}
\end{itemize}
Acautelar, prevenir; pôr de precaução.
\section{Precate}
\begin{itemize}
\item {Grp. gram.:m.}
\end{itemize}
\begin{itemize}
\item {Utilização:P. us.}
\end{itemize}
Acto de precatar; precaução. Cf. Alex. Lobo, III, 304.
\section{Precatória}
\begin{itemize}
\item {Grp. gram.:f.  e  adj.}
\end{itemize}
Diz-se da carta, dirigida por um juiz de uma circunscrição ao de outra, para que cumpra ou faça cumprir certas diligências judiciaes.
(Fem. de \textunderscore precatório\textunderscore )
\section{Precatório}
\begin{itemize}
\item {Grp. gram.:adj.}
\end{itemize}
\begin{itemize}
\item {Grp. gram.:M.}
\end{itemize}
\begin{itemize}
\item {Proveniência:(Lat. \textunderscore precatorius\textunderscore )}
\end{itemize}
Em que se pede alguma coisa; rogatório.
Documento precatório; carta precatória.
\section{Precaução}
\begin{itemize}
\item {Grp. gram.:f.}
\end{itemize}
\begin{itemize}
\item {Proveniência:(Lat. \textunderscore praecautio\textunderscore )}
\end{itemize}
Prevenção.
Cautela antecipada.
\section{Precaucionar-se}
\begin{itemize}
\item {Grp. gram.:v. p.}
\end{itemize}
\begin{itemize}
\item {Proveniência:(Do lat. \textunderscore praecautio\textunderscore )}
\end{itemize}
Acautelar-se antecipadamente; precaver-se.
\section{Precautelar}
\begin{itemize}
\item {Grp. gram.:v. t.}
\end{itemize}
\begin{itemize}
\item {Proveniência:(De \textunderscore pre...\textunderscore  + \textunderscore cautela\textunderscore )}
\end{itemize}
O mesmo que \textunderscore precaver\textunderscore .
\section{Precautório}
\begin{itemize}
\item {Grp. gram.:adj.}
\end{itemize}
Que envolve precaução; relativo a precaução:«\textunderscore ...medicina precautória...\textunderscore »F. Manuel, \textunderscore Carta de Guia\textunderscore , 61.
\section{Precaver}
\begin{itemize}
\item {Grp. gram.:v. t.}
\end{itemize}
\begin{itemize}
\item {Proveniência:(Lat. \textunderscore praecavere\textunderscore )}
\end{itemize}
Acautelar antecipadamente; prevenir.
\section{Prece}
\begin{itemize}
\item {Grp. gram.:f.}
\end{itemize}
\begin{itemize}
\item {Utilização:Ext.}
\end{itemize}
\begin{itemize}
\item {Proveniência:(Lat. \textunderscore prex\textunderscore , \textunderscore precis\textunderscore )}
\end{itemize}
Súpplica religiosa; reza.
Pedido instante, súpplica.
\section{Precedencia}
\begin{itemize}
\item {Grp. gram.:f.}
\end{itemize}
Qualidade ou condição do que é precedente; preferência.
\section{Precedente}
\begin{itemize}
\item {Grp. gram.:adj.}
\end{itemize}
\begin{itemize}
\item {Proveniência:(Lat. \textunderscore praecedens\textunderscore )}
\end{itemize}
Que precede ou antecede; que está antes de outro.
\section{Preceder}
\begin{itemize}
\item {Grp. gram.:v. t.}
\end{itemize}
\begin{itemize}
\item {Grp. gram.:V. i.}
\end{itemize}
\begin{itemize}
\item {Proveniência:(Lat. \textunderscore praecedere\textunderscore )}
\end{itemize}
Ir ou estar adeante de; anteceder.
Pagar antes de.
Ir adeante; antepor-se.
\section{Preceito}
\begin{itemize}
\item {Grp. gram.:m.}
\end{itemize}
\begin{itemize}
\item {Grp. gram.:Loc. adv.}
\end{itemize}
\begin{itemize}
\item {Proveniência:(Do lat. \textunderscore praeceptum\textunderscore )}
\end{itemize}
Regra de proceder.
Ensinamento.
Norma.
Doutrina; prescripção; ordem.
Condição.
\textunderscore A preceito\textunderscore , com todas as regras, com todo o cuidado; minuciosamente.
\section{Preceituação}
\begin{itemize}
\item {Grp. gram.:f.}
\end{itemize}
Acto de preceituar. Cf. L. Cordeiro, \textunderscore Senh. Duq.\textunderscore , 9.
\section{Preceituar}
\begin{itemize}
\item {Grp. gram.:v. t.}
\end{itemize}
\begin{itemize}
\item {Grp. gram.:V. i.}
\end{itemize}
Estabelecer como preceito; ordenar.
Estabelecer regras; dar instrucções ou ordens.
\section{Preceituário}
\begin{itemize}
\item {Grp. gram.:m.}
\end{itemize}
\begin{itemize}
\item {Proveniência:(De \textunderscore preceito\textunderscore )}
\end{itemize}
Collecção ou reunião de preceitos; conjunto de regras.
\section{Precentor}
\begin{itemize}
\item {Grp. gram.:m.}
\end{itemize}
\begin{itemize}
\item {Proveniência:(Lat. \textunderscore praecentor\textunderscore )}
\end{itemize}
Aquelle que, entre os antigos, dirigia uma orchestra.
Corypheu; mestre de músicos.--A Igreja deu também êsse nome ao que, entre os Hebreus, cantava os psalmos deante da arca ou versos seus. Cf. Filinto, XIV, 115.
\section{Preceptivamente}
\begin{itemize}
\item {Grp. gram.:adv.}
\end{itemize}
De modo preceptivo; como quem manda; á maneira de preceito.
\section{Preceptivo}
\begin{itemize}
\item {Grp. gram.:adj.}
\end{itemize}
\begin{itemize}
\item {Proveniência:(Lat. \textunderscore praeceptivus\textunderscore )}
\end{itemize}
Em que há preceito; que tem fórma ou natureza de preceito.
\section{Precepto}
\begin{itemize}
\item {Grp. gram.:m.}
\end{itemize}
\begin{itemize}
\item {Utilização:Ant.}
\end{itemize}
O mesmo que \textunderscore preceito\textunderscore .
\section{Preceptor}
\begin{itemize}
\item {Grp. gram.:m.}
\end{itemize}
\begin{itemize}
\item {Utilização:Ant.}
\end{itemize}
\begin{itemize}
\item {Proveniência:(Lat. \textunderscore praeceptor\textunderscore )}
\end{itemize}
O que dá preceitos ou instrucções.
Mentor; mestre.
Grão-mestre; mestre ou commendador de uma Ordem militar.
\section{Preceptoria}
\begin{itemize}
\item {Grp. gram.:f.}
\end{itemize}
\begin{itemize}
\item {Proveniência:(De \textunderscore preceptor\textunderscore )}
\end{itemize}
Qualidade de mestre ou commendador de uma Ordem militar. Cf. Herculano, \textunderscore Hist. de Port.\textunderscore , II, 14 e 123.
\section{Precessão}
\begin{itemize}
\item {Grp. gram.:f.}
\end{itemize}
\begin{itemize}
\item {Proveniência:(Lat. \textunderscore praecessio\textunderscore )}
\end{itemize}
Acto ou effeito de preceder.
Precedencia.
\textunderscore Precessão dos equinócios\textunderscore , accrescentamento successivo e uniforme das longitudes das estrêllas, mantendo-se a mesma latitude.
\section{Precha}
\begin{itemize}
\item {Grp. gram.:f.}
\end{itemize}
\begin{itemize}
\item {Utilização:Prov.}
\end{itemize}
\begin{itemize}
\item {Utilização:minh.}
\end{itemize}
O mesmo que \textunderscore percha\textunderscore .
\section{Prèchristão}
\begin{itemize}
\item {Grp. gram.:adj.}
\end{itemize}
Anterior ao Christianismo.
\section{Précia}
\begin{itemize}
\item {Grp. gram.:f.}
\end{itemize}
\begin{itemize}
\item {Utilização:Prov.}
\end{itemize}
Variedade de uva temporan.
\section{Precidâneas}
\begin{itemize}
\item {Grp. gram.:f. pl.}
\end{itemize}
\begin{itemize}
\item {Proveniência:(Do lat. \textunderscore praecidaneus\textunderscore )}
\end{itemize}
As victimas que se immolavam na véspera de um sacrifício solenne, entre os antigos Romanos.
\section{Precinção}
\begin{itemize}
\item {Grp. gram.:f.}
\end{itemize}
\begin{itemize}
\item {Proveniência:(Lat. \textunderscore praecinctio\textunderscore )}
\end{itemize}
Espaço ou largo degrau, que separava as filas dos espectadores, nos anfiteatros romanos. Cf. Castilho, \textunderscore Fastos\textunderscore , II, 506.
\section{Precincção}
\begin{itemize}
\item {Grp. gram.:f.}
\end{itemize}
\begin{itemize}
\item {Proveniência:(Lat. \textunderscore praecinctio\textunderscore )}
\end{itemize}
Espaço ou largo degrau, que separava as filas dos espectadores, nos amphitheatros romanos. Cf. Castilho, \textunderscore Fastos\textunderscore , II, 506.
\section{Precingír}
\begin{itemize}
\item {Grp. gram.:v. t.}
\end{itemize}
\begin{itemize}
\item {Proveniência:(Lat. \textunderscore praecingere\textunderscore )}
\end{itemize}
Ligar com cinta; cingir; cercar; estreitar; encerrar. Cf. Alv. Mendes, \textunderscore Discursos\textunderscore , 214.
\section{Precinta}
\begin{itemize}
\item {Grp. gram.:f.}
\end{itemize}
Cinta.
Pano, de que se fazem cilhas e outros objectos.
Tira de lona, com que se forram os cabos.
(Fem. de \textunderscore precinto\textunderscore )
\section{Precintar}
\begin{itemize}
\item {Grp. gram.:v. t.}
\end{itemize}
Cingir ou atar com precintas; forrar.
\section{Precinto}
\begin{itemize}
\item {Grp. gram.:m.}
\end{itemize}
\begin{itemize}
\item {Proveniência:(Lat. \textunderscore praecinctus\textunderscore )}
\end{itemize}
O mesmo que \textunderscore precinta\textunderscore .
\section{Preciosa-de-ois}
\begin{itemize}
\item {Grp. gram.:f.}
\end{itemize}
O mesmo que \textunderscore figueirôa\textunderscore .
\section{Preciosamente}
\begin{itemize}
\item {Grp. gram.:adv.}
\end{itemize}
De modo precioso.
\section{Preciosidade}
\begin{itemize}
\item {Grp. gram.:f.}
\end{itemize}
Qualidade do que é precioso; aquillo que é precioso.
\section{Preciosismo}
\begin{itemize}
\item {Grp. gram.:m.}
\end{itemize}
\begin{itemize}
\item {Proveniência:(De \textunderscore precioso\textunderscore )}
\end{itemize}
Exaggerada delicadeza e subtileza, no falar e no escrever, usada em grêmios literários da França, no século XVII.
\section{Precioso}
\begin{itemize}
\item {Grp. gram.:adj.}
\end{itemize}
\begin{itemize}
\item {Utilização:Fig.}
\end{itemize}
\begin{itemize}
\item {Grp. gram.:M.}
\end{itemize}
\begin{itemize}
\item {Utilização:Des.}
\end{itemize}
\begin{itemize}
\item {Proveniência:(Lat. \textunderscore pretiosus\textunderscore )}
\end{itemize}
Que é de grande preço; sumptuoso; muito rico; magnífico; que tem grande importância.
Presumido, affectado.
Coisa preciosa.
\section{Precipício}
\begin{itemize}
\item {Grp. gram.:m.}
\end{itemize}
\begin{itemize}
\item {Utilização:Fig.}
\end{itemize}
\begin{itemize}
\item {Proveniência:(Lat. \textunderscore praecipitium\textunderscore )}
\end{itemize}
Lugar, donde se póde precipitar alguém ou alguma coisa.
Despenhadeiro; abysmo.
Perigo grande; ruína; perdição.
\section{Precipitação}
\begin{itemize}
\item {Grp. gram.:f.}
\end{itemize}
\begin{itemize}
\item {Proveniência:(Lat. \textunderscore praecipitatio\textunderscore )}
\end{itemize}
Acto ou effeito de precipitar ou de se precipitar.
Pressa irreflectida: \textunderscore proceder com precipitação\textunderscore .
\section{Precipitadamente}
\begin{itemize}
\item {Grp. gram.:adv.}
\end{itemize}
De modo precipitado.
Apressadamente; com imprudência.
\section{Precipitado}
\begin{itemize}
\item {Grp. gram.:adj.}
\end{itemize}
\begin{itemize}
\item {Grp. gram.:M.}
\end{itemize}
\begin{itemize}
\item {Proveniência:(De \textunderscore precipitar\textunderscore )}
\end{itemize}
Que não reflecte; temerário, imprudente; arrebatado.
Aquelle que procede sem reflexão.
Substância dissolvida, que abandonou o líquido dissolvente e se suspendeu nelle, ou se depositou no fundo do vaso.
\section{Precipitante}
\begin{itemize}
\item {Grp. gram.:adj.}
\end{itemize}
\begin{itemize}
\item {Grp. gram.:M.}
\end{itemize}
\begin{itemize}
\item {Proveniência:(Lat. \textunderscore praecipitans\textunderscore )}
\end{itemize}
Que precipita.
Reagente chímico, com que se obtém um precipitado.
\section{Precipitar}
\begin{itemize}
\item {Grp. gram.:v. t.}
\end{itemize}
\begin{itemize}
\item {Grp. gram.:V. i.}
\end{itemize}
\begin{itemize}
\item {Proveniência:(Lat. \textunderscore praecipitare\textunderscore )}
\end{itemize}
Lançar ao precipício; arremessar do cima para baixo.
Induzir ou conduzir a uma desgraça.
Apressar muito: \textunderscore precipitar uma resolução\textunderscore .
Separar-se de um liquido (uma substância), suspendendo-se nelle ou depositando-se no fundo do vaso.
\section{Precípite}
\begin{itemize}
\item {Grp. gram.:adj.}
\end{itemize}
\begin{itemize}
\item {Proveniência:(Lat. \textunderscore praeceps\textunderscore )}
\end{itemize}
Que está arriscado a precipitar-se.
Apressado, veloz.
\section{Precipitoso}
\begin{itemize}
\item {Grp. gram.:adj.}
\end{itemize}
\begin{itemize}
\item {Utilização:Fig.}
\end{itemize}
\begin{itemize}
\item {Proveniência:(De \textunderscore precipitar\textunderscore )}
\end{itemize}
Em que ha precipícios ou despenhadeiros.
Precipite.
Precipitado; temerário.
Impetuoso; impaciente.
\section{Precipuamente}
\begin{itemize}
\item {Grp. gram.:adv.}
\end{itemize}
De modo precípuo; principalmente; essencialmente.
\section{Prècristão}
\begin{itemize}
\item {Grp. gram.:adj.}
\end{itemize}
Anterior ao Cristianismo.
\section{Precípuo}
\begin{itemize}
\item {Grp. gram.:adj.}
\end{itemize}
\begin{itemize}
\item {Grp. gram.:M.}
\end{itemize}
\begin{itemize}
\item {Utilização:Jur.}
\end{itemize}
\begin{itemize}
\item {Proveniência:(Lat. \textunderscore praecipuus\textunderscore )}
\end{itemize}
Principal; essencial.
Vantagem, que um co-herdeiro tem, por disposição testamentária ou legal.
Bens, que se pódem tirar da terça para um co-herdeiro, antes que ella se divida por todos os co-herdeiros.
\section{Precisamente}
\begin{itemize}
\item {Grp. gram.:adv.}
\end{itemize}
De modo preciso; exactamente; rigorosamente.
\section{Precisão}
\begin{itemize}
\item {Grp. gram.:f.}
\end{itemize}
\begin{itemize}
\item {Utilização:Des.}
\end{itemize}
\begin{itemize}
\item {Proveniência:(Lat. \textunderscore praecisio\textunderscore )}
\end{itemize}
Carência do que é necessário ou útil: \textunderscore soffrer precisões\textunderscore .
Necessidade.
Exactidão em certos cálculos.
Pontualidade no cumprimento de um dever ou na execução de alguma coisa.
Maneira de se exprimir concisamente: \textunderscore falar com precisão\textunderscore .
Momento exacto; prazo.
\section{Precisar}
\begin{itemize}
\item {Grp. gram.:v. t.}
\end{itemize}
\begin{itemize}
\item {Grp. gram.:V. i.}
\end{itemize}
\begin{itemize}
\item {Proveniência:(De \textunderscore preciso\textunderscore )}
\end{itemize}
Ter precisão de: \textunderscore precisar vestuário\textunderscore .
Determinar ou calcular com exactidão.
Mencionar especialmente; particularizar.
Expor em resumo; proferir laconicamente.
Têr precisão: \textunderscore precisar de amparo\textunderscore .
\section{Preciso}
\begin{itemize}
\item {Grp. gram.:adj.}
\end{itemize}
\begin{itemize}
\item {Proveniência:(Lat. \textunderscore praecisus\textunderscore )}
\end{itemize}
Que é necessário.
Exacto, certo.
Claro, terminante.
Resumido, lacónico: \textunderscore linguagem precisa\textunderscore .
\section{Precito}
\begin{itemize}
\item {Grp. gram.:m.  e  adj.}
\end{itemize}
\begin{itemize}
\item {Proveniência:(Do lat. \textunderscore praescitus\textunderscore )}
\end{itemize}
Condemnado; maldito; réprobo.
\section{Precito}
\begin{itemize}
\item {Grp. gram.:adj.}
\end{itemize}
O mesmo que \textunderscore precípite\textunderscore ?:«\textunderscore precitas nascem da maior altura as águas...\textunderscore »\textunderscore Viriato Trág.\textunderscore , VII, 85.
\section{Preclaro}
\begin{itemize}
\item {Grp. gram.:adj.}
\end{itemize}
\begin{itemize}
\item {Proveniência:(Lat. \textunderscore praeclarus\textunderscore )}
\end{itemize}
Illustre; muito illustre; notável.
Brilhante; formoso.
\section{Preclávio}
\begin{itemize}
\item {Grp. gram.:m.}
\end{itemize}
\begin{itemize}
\item {Proveniência:(Lat. \textunderscore praeclavium\textunderscore )}
\end{itemize}
Parte deanteira da túnica dos senadores romanos.
\section{Preclusão}
\begin{itemize}
\item {Grp. gram.:f.}
\end{itemize}
\begin{itemize}
\item {Utilização:Philol.}
\end{itemize}
\begin{itemize}
\item {Proveniência:(Lat. \textunderscore praeclusio\textunderscore )}
\end{itemize}
Contacto prévio de dois órgãos, para a producção de phonema explosivo, como \textunderscore p\textunderscore , \textunderscore b\textunderscore , etc.
\section{Preço}
\begin{itemize}
\item {fónica:prê}
\end{itemize}
\begin{itemize}
\item {Grp. gram.:m.}
\end{itemize}
\begin{itemize}
\item {Utilização:Ant.}
\end{itemize}
\begin{itemize}
\item {Utilização:Ant.}
\end{itemize}
\begin{itemize}
\item {Proveniência:(Do lat. \textunderscore pretium\textunderscore )}
\end{itemize}
Aquillo que se pede por uma coisa vendível.
Valor pecuniário de um objecto.
O que se dá em troca de uma coisa que se comprou.
Custo.
Compensação.
Prêmio.
Castigo.
Valia.
Importância moral.
Merecimento.
Quilate.
Estimação, aprêço.
\textunderscore Juízes dos preços\textunderscore , júry dos prêmios em corridas, concursos ou jogos.
\textunderscore Mau preço\textunderscore , o mesmo que \textunderscore adultério\textunderscore . Cf. \textunderscore Port. Mon. Hist.\textunderscore , \textunderscore Script.\textunderscore , 297.
\section{Precoce}
\begin{itemize}
\item {Grp. gram.:adj.}
\end{itemize}
\begin{itemize}
\item {Grp. gram.:Adv.}
\end{itemize}
\begin{itemize}
\item {Proveniência:(Lat. \textunderscore praecox\textunderscore )}
\end{itemize}
Que amadureceu depressa ou antes do tempo próprio.
Prematuro; antecipado.
Temporão.
Que succedeu ou se desenvolveu antes do tempo, em que se costuma dar esse successo ou desenvolvimento.
Precocemente.
\section{Precocemente}
\begin{itemize}
\item {Grp. gram.:adv.}
\end{itemize}
De modo precoce.
Com precocidade; antes do tempo próprio.
\section{Precocidade}
\begin{itemize}
\item {Grp. gram.:f.}
\end{itemize}
Qualidade do que é precoce ou prematuro.
\section{Precogitar}
\begin{itemize}
\item {Grp. gram.:v. t.}
\end{itemize}
\begin{itemize}
\item {Proveniência:(De \textunderscore pre...\textunderscore  + \textunderscore cogitar\textunderscore )}
\end{itemize}
Cogitar antes, premeditar.
\section{Precógnito}
\begin{itemize}
\item {Grp. gram.:adj.}
\end{itemize}
\begin{itemize}
\item {Proveniência:(Lat. \textunderscore praecognitus\textunderscore )}
\end{itemize}
Conhecido antes; previsto.
\section{Preçolana}
\begin{itemize}
\item {Grp. gram.:f.}
\end{itemize}
\begin{itemize}
\item {Utilização:Ant.}
\end{itemize}
O mesmo que \textunderscore porcelana\textunderscore . Cf. \textunderscore Chrón. dos R. de Bisnaga\textunderscore , 70.
\section{Precolombiano}
\begin{itemize}
\item {Grp. gram.:adj.}
\end{itemize}
\begin{itemize}
\item {Proveniência:(De \textunderscore pre...\textunderscore  + \textunderscore Colombo\textunderscore , n. p.)}
\end{itemize}
Anterior a Colombo ou aos seus descobrimentos.
\section{Preconceber}
\begin{itemize}
\item {Grp. gram.:v. t.}
\end{itemize}
\begin{itemize}
\item {Proveniência:(De \textunderscore pre...\textunderscore  + \textunderscore conceber\textunderscore )}
\end{itemize}
Conceber antecipadamente; planear ou idear com antecipação.
\section{Preconcebido}
\begin{itemize}
\item {Grp. gram.:adj.}
\end{itemize}
Concebido ou planeado levianamente, sem fundamento sério: \textunderscore injustiça preconcebida\textunderscore .
\section{Preconceito}
\begin{itemize}
\item {Grp. gram.:m.}
\end{itemize}
\begin{itemize}
\item {Proveniência:(De \textunderscore pre...\textunderscore  + \textunderscore conceito\textunderscore )}
\end{itemize}
Conceito antecipado.
Opinião, formada sem reflexão.
Preoccupação; superstição.
\section{Preconceituado}
\begin{itemize}
\item {Grp. gram.:adj.}
\end{itemize}
\begin{itemize}
\item {Utilização:P. us.}
\end{itemize}
\begin{itemize}
\item {Proveniência:(De \textunderscore preconceito\textunderscore )}
\end{itemize}
O mesmo que \textunderscore preconcebido\textunderscore :«\textunderscore ...preconceituadas opiniões\textunderscore ». Filinto, XVIII, 61.
\section{Preconício}
\begin{itemize}
\item {Grp. gram.:adj.}
\end{itemize}
\begin{itemize}
\item {Utilização:bras}
\end{itemize}
\begin{itemize}
\item {Utilização:Neol.}
\end{itemize}
\begin{itemize}
\item {Proveniência:(Do rad. do lat. \textunderscore praeconium\textunderscore )}
\end{itemize}
O mesmo que \textunderscore reclamo\textunderscore . Cf. \textunderscore Notícia\textunderscore , de 19-X-900.
\section{Precónio}
\begin{itemize}
\item {Grp. gram.:m.}
\end{itemize}
\begin{itemize}
\item {Utilização:Des.}
\end{itemize}
\begin{itemize}
\item {Proveniência:(Lat. \textunderscore praeconium\textunderscore )}
\end{itemize}
Acto de apregoar.
Louvor, encómio.
\section{Preconização}
\begin{itemize}
\item {Grp. gram.:f.}
\end{itemize}
Acto ou effeito de preconizar.
Declaração, feita em consistório pontíficio, de que um ecclesiástico, nomeado pelo seu Govêrno para um bispado ou outro benefício, tem as condições exigidas ecclesiasticamente em tal caso.
\section{Preconizador}
\begin{itemize}
\item {Grp. gram.:m.  e  adj.}
\end{itemize}
O que preconiza.
\section{Preconizar}
\begin{itemize}
\item {Grp. gram.:v. t.}
\end{itemize}
\begin{itemize}
\item {Proveniência:(Do lat. \textunderscore praeconari\textunderscore )}
\end{itemize}
Declarar um Cardeal ou o Papa que há as qualidades requeridas em (um ecclesiástico nomeado pelo seu Govêrno para um bispado ou outro benefício).
Apregoar, louvando.
Louvar.
Publicar, propalar, divulgar: \textunderscore preconizar heresias\textunderscore .
\section{Precórdio}
\begin{itemize}
\item {Grp. gram.:m.}
\end{itemize}
\begin{itemize}
\item {Utilização:Anat.}
\end{itemize}
\begin{itemize}
\item {Utilização:Des.}
\end{itemize}
\begin{itemize}
\item {Proveniência:(Lat. \textunderscore praecordium\textunderscore )}
\end{itemize}
O mesmo que \textunderscore diaphragma\textunderscore .
\section{Precto}
\begin{itemize}
\item {Grp. gram.:m.}
\end{itemize}
\begin{itemize}
\item {Utilização:Ant.}
\end{itemize}
O mesmo que \textunderscore pleito\textunderscore .
\section{Prècurar}
\textunderscore v. t.\textunderscore  (e der.)
(Fórma pop. de \textunderscore procurar\textunderscore , etc.)
\section{Precursor}
\begin{itemize}
\item {Grp. gram.:m.  e  adj.}
\end{itemize}
\begin{itemize}
\item {Proveniência:(Lat. \textunderscore praecursor\textunderscore )}
\end{itemize}
O que vai adeante.
O que annuncia um successo ou a chegada de alguém.
Objecto ou facto, que preannuncia outro.
\textunderscore O precursor de Christo\textunderscore , San-João Baptista.
\section{Predatório}
\begin{itemize}
\item {Grp. gram.:adj.}
\end{itemize}
\begin{itemize}
\item {Proveniência:(Lat. \textunderscore praedatorius\textunderscore )}
\end{itemize}
Relativo a roubos ou a piratas.
Dizia-se especialmente dos navios de corsários.
\section{Predecessor}
\begin{itemize}
\item {Grp. gram.:m.}
\end{itemize}
\begin{itemize}
\item {Proveniência:(Lat. \textunderscore praedecessor\textunderscore )}
\end{itemize}
O mesmo que \textunderscore antecessor\textunderscore .
\section{Predefinição}
\begin{itemize}
\item {Grp. gram.:f.}
\end{itemize}
Acto ou effeito de predefinir.
\section{Predefinir}
\begin{itemize}
\item {Grp. gram.:v. t.}
\end{itemize}
\begin{itemize}
\item {Proveniência:(De \textunderscore pre...\textunderscore  + \textunderscore definir\textunderscore )}
\end{itemize}
Definir ou determinar antecipadamente; predestinar, prognosticar.
\section{Predefunto}
\begin{itemize}
\item {Grp. gram.:adj.}
\end{itemize}
\begin{itemize}
\item {Proveniência:(De \textunderscore pre...\textunderscore  + \textunderscore defunto\textunderscore )}
\end{itemize}
Que morreu, antes de qualquer facto: \textunderscore filho de pai predefunto\textunderscore . Cf. \textunderscore Instituto\textunderscore , XLIX, 76.
\section{Predestinação}
\begin{itemize}
\item {Grp. gram.:f.}
\end{itemize}
\begin{itemize}
\item {Proveniência:(Lat. \textunderscore praedestinatio\textunderscore )}
\end{itemize}
Acto ou effeito de predestinar.
\section{Predestinado}
\begin{itemize}
\item {Grp. gram.:m.}
\end{itemize}
\begin{itemize}
\item {Utilização:Burl.}
\end{itemize}
\begin{itemize}
\item {Proveniência:(De \textunderscore predestinar\textunderscore )}
\end{itemize}
Aquelle a quem Deus predestina á bemaventurança; santo.
Homem, que, pela sua índole, parece destinado a sêr atraiçoado por sua mulhér.
\section{Predestinar}
\begin{itemize}
\item {Grp. gram.:v. t.}
\end{itemize}
\begin{itemize}
\item {Proveniência:(Lat. \textunderscore praedestinare\textunderscore )}
\end{itemize}
Destinar antes.
Escolher desde toda a eternidade (os justos).
Destinar a grandes feitos.
\section{Predeterminação}
\begin{itemize}
\item {Grp. gram.:f.}
\end{itemize}
Acto ou effeito de predeterminar.
\section{Predeterminar}
\begin{itemize}
\item {Grp. gram.:v. t.}
\end{itemize}
\begin{itemize}
\item {Proveniência:(De \textunderscore pre...\textunderscore  + \textunderscore determinar\textunderscore )}
\end{itemize}
Determinar antecipadamente.
\section{Predial}
\begin{itemize}
\item {Grp. gram.:adj.}
\end{itemize}
Relativo a prédio ou prédios: \textunderscore contribuição predial\textunderscore .
\section{Prédica}
\begin{itemize}
\item {Grp. gram.:f.}
\end{itemize}
\begin{itemize}
\item {Proveniência:(Do lat. \textunderscore praedicare\textunderscore )}
\end{itemize}
Acto de prègar; sermão; discurso.
\section{Predicação}
\begin{itemize}
\item {Grp. gram.:f.}
\end{itemize}
\begin{itemize}
\item {Utilização:Gram.}
\end{itemize}
\begin{itemize}
\item {Proveniência:(Lat. \textunderscore praedicatio\textunderscore )}
\end{itemize}
Emprêgo ou qualidade de predicado. Cf. J. Ribeiro, \textunderscore Diccion. Gram.\textunderscore , vb. \textunderscore proposição\textunderscore .
\section{Predicado}
\begin{itemize}
\item {Grp. gram.:m.}
\end{itemize}
\begin{itemize}
\item {Utilização:Gram.}
\end{itemize}
\begin{itemize}
\item {Proveniência:(Lat. \textunderscore praedicatus\textunderscore )}
\end{itemize}
Qualidade característica.
Prenda; virtude; attributo.
Palavra ou expressão, formada por um verbo, e que enuncía alguma coisa a respeito do sujeito da proposição.
\section{Predicador}
\begin{itemize}
\item {Grp. gram.:m.  e  adj.}
\end{itemize}
\begin{itemize}
\item {Proveniência:(Lat. \textunderscore praedicator\textunderscore )}
\end{itemize}
O mesmo que \textunderscore predicante\textunderscore .
\section{Predical}
\begin{itemize}
\item {Grp. gram.:adj.}
\end{itemize}
Relativo a prédica, a sermão. Cf. Filinto, VI, 11.
\section{Predicamentar}
\begin{itemize}
\item {Grp. gram.:v. t.}
\end{itemize}
\begin{itemize}
\item {Proveniência:(De \textunderscore predicamento\textunderscore )}
\end{itemize}
Classificar; graduar.
\section{Predicamento}
\begin{itemize}
\item {Grp. gram.:m.}
\end{itemize}
\begin{itemize}
\item {Proveniência:(Lat. \textunderscore praedicamentum\textunderscore )}
\end{itemize}
Categoria, graduação.
\section{Predicante}
\begin{itemize}
\item {Grp. gram.:m.  e  adj.}
\end{itemize}
\begin{itemize}
\item {Proveniência:(Lat. \textunderscore praedicans\textunderscore )}
\end{itemize}
Prègador protestante.
\section{Predição}
\begin{itemize}
\item {Grp. gram.:f.}
\end{itemize}
\begin{itemize}
\item {Proveniência:(Lat. \textunderscore praedictio\textunderscore )}
\end{itemize}
Acto ou efeito de predizer; vaticínio, profecia.
\section{Predicar}
\begin{itemize}
\item {Grp. gram.:v. t.}
\end{itemize}
\begin{itemize}
\item {Proveniência:(Lat. \textunderscore praedicare\textunderscore )}
\end{itemize}
O mesmo que \textunderscore prègar\textunderscore ^1; aconselhar. Cf. Camillo, \textunderscore Am. de Perdição\textunderscore , 130, (ed. monum.).
\section{Predicativo}
\begin{itemize}
\item {Grp. gram.:m.  e  adj.}
\end{itemize}
\begin{itemize}
\item {Utilização:Gram.}
\end{itemize}
\begin{itemize}
\item {Proveniência:(Lat. \textunderscore praedicativus\textunderscore )}
\end{itemize}
Diz-se do nome ou pronome, que qualifica ou determina o sujeito e completa a significação do verbo.
\section{Predicatório}
\begin{itemize}
\item {Grp. gram.:adj.}
\end{itemize}
\begin{itemize}
\item {Proveniência:(Lat. \textunderscore praedicatorius\textunderscore )}
\end{itemize}
Encomiástico; lisonjeiro. Cf. Júl. Dinis, \textunderscore Morgadinha\textunderscore , 267.
\section{Predicável}
\begin{itemize}
\item {Grp. gram.:adj.}
\end{itemize}
\begin{itemize}
\item {Proveniência:(De \textunderscore predicar\textunderscore )}
\end{itemize}
Digno de se prègar ou de se aconselhar.
\section{Predicção}
\begin{itemize}
\item {Grp. gram.:f.}
\end{itemize}
\begin{itemize}
\item {Proveniência:(Lat. \textunderscore praedictio\textunderscore )}
\end{itemize}
Acto ou effeito de predizer; vaticínio, prophecia.
\section{Predicto}
\begin{itemize}
\item {Grp. gram.:adj.}
\end{itemize}
\begin{itemize}
\item {Proveniência:(Lat. \textunderscore praedictus\textunderscore )}
\end{itemize}
Que se predisse; que foi vaticinado.
\section{Predilecção}
\begin{itemize}
\item {Grp. gram.:f.}
\end{itemize}
\begin{itemize}
\item {Proveniência:(De \textunderscore pre...\textunderscore  + \textunderscore dilecção\textunderscore )}
\end{itemize}
Gôsto por alguma coisa ou amizade por alguém, com preferência a outra coisa ou a outrem.
Affeição extremosa.
\section{Predilecto}
\begin{itemize}
\item {Grp. gram.:m.  e  adj.}
\end{itemize}
\begin{itemize}
\item {Proveniência:(De \textunderscore pre...\textunderscore  + \textunderscore dilecto\textunderscore )}
\end{itemize}
O que é estimado ou querido com preferência, com extremo.
\section{Prédio}
\begin{itemize}
\item {Grp. gram.:m.}
\end{itemize}
\begin{itemize}
\item {Proveniência:(Lat. \textunderscore praedium\textunderscore )}
\end{itemize}
Propriedade immóvel.
Terreno cultivável.
Casa de habitação.
Edifício.
\section{Predisponente}
\begin{itemize}
\item {Grp. gram.:adj.}
\end{itemize}
\begin{itemize}
\item {Proveniência:(De \textunderscore pre...\textunderscore  + \textunderscore disponente\textunderscore )}
\end{itemize}
Que predispõe.
\section{Predispor}
\begin{itemize}
\item {Grp. gram.:v. t.}
\end{itemize}
\begin{itemize}
\item {Proveniência:(De \textunderscore pre...\textunderscore  + \textunderscore dispor\textunderscore )}
\end{itemize}
Dispor prèviamente; preparar de ante-mão.
\section{Predisposição}
\begin{itemize}
\item {Grp. gram.:f.}
\end{itemize}
\begin{itemize}
\item {Proveniência:(De \textunderscore pre...\textunderscore  + \textunderscore disposição\textunderscore )}
\end{itemize}
Acto de predispor ou de se predispor.
Tendência; vocação.
\section{Predito}
\begin{itemize}
\item {Grp. gram.:adj.}
\end{itemize}
\begin{itemize}
\item {Proveniência:(Lat. \textunderscore praedictus\textunderscore )}
\end{itemize}
Que se predisse; que foi vaticinado.
\section{Predizer}
\begin{itemize}
\item {Grp. gram.:v. t.}
\end{itemize}
\begin{itemize}
\item {Proveniência:(Do lat. \textunderscore praedicere\textunderscore )}
\end{itemize}
Dizer antes; prenunciar; prognosticar.
\section{Predominação}
\begin{itemize}
\item {Grp. gram.:f.}
\end{itemize}
Acto ou effeito de predominar; predomínio.
\section{Predominador}
\begin{itemize}
\item {Grp. gram.:m.  e  adj.}
\end{itemize}
O que predomina.
\section{Predominância}
\begin{itemize}
\item {Grp. gram.:f.}
\end{itemize}
\begin{itemize}
\item {Proveniência:(De \textunderscore predominar\textunderscore )}
\end{itemize}
Qualidade de predominante.
O mesmo que \textunderscore predomínio\textunderscore .
\section{Predominante}
\begin{itemize}
\item {Grp. gram.:adj.}
\end{itemize}
\begin{itemize}
\item {Utilização:Gram.}
\end{itemize}
\begin{itemize}
\item {Proveniência:(De \textunderscore predominar\textunderscore )}
\end{itemize}
Que predomina.
Diz-se do accento mais forte de uma palavra; e diz-se da sýllaba ou vogal que tem êsse accento.
\section{Predominar}
\begin{itemize}
\item {Grp. gram.:v. i.}
\end{itemize}
\begin{itemize}
\item {Proveniência:(De \textunderscore pre...\textunderscore  + \textunderscore dominar\textunderscore )}
\end{itemize}
Sêr o primeiro em domínio ou influência.
Dominar muito; prevalecer.
Sêr em maior quantidade ou intensidade.
\section{Predomínio}
\begin{itemize}
\item {Grp. gram.:m.}
\end{itemize}
\begin{itemize}
\item {Proveniência:(De \textunderscore pre...\textunderscore  + \textunderscore domínio\textunderscore )}
\end{itemize}
Domínio principal; preponderância; superioridade; predominação.
\section{Preeminência}
\begin{itemize}
\item {Grp. gram.:f.}
\end{itemize}
\begin{itemize}
\item {Proveniência:(Lat. \textunderscore praeeminentia\textunderscore )}
\end{itemize}
Qualidade do que é preeminente; superioridade; primazia.
\section{Preeminente}
\begin{itemize}
\item {Grp. gram.:adj.}
\end{itemize}
\begin{itemize}
\item {Proveniência:(Lat. \textunderscore praeeminens\textunderscore )}
\end{itemize}
Que occupa lugar mais elevado; distinto: \textunderscore orador preeminente\textunderscore .
\section{Preempção}
\begin{itemize}
\item {Grp. gram.:f.}
\end{itemize}
\begin{itemize}
\item {Proveniência:(Do lat. \textunderscore prae\textunderscore  + \textunderscore emptio\textunderscore )}
\end{itemize}
Compra antecipada.
\section{Preencher}
\begin{itemize}
\item {Grp. gram.:v. t.}
\end{itemize}
\begin{itemize}
\item {Proveniência:(De \textunderscore pre...\textunderscore  + \textunderscore encher\textunderscore )}
\end{itemize}
Encher completamente.
Occupar; completar.
Desempenhar: \textunderscore preencher um lugar de escrivão\textunderscore .
Observar, cumprir plenamente.
\section{Preenchimento}
\begin{itemize}
\item {Grp. gram.:m.}
\end{itemize}
Acto ou effeito de preencher.
\section{Preestabelecer}
\begin{itemize}
\item {Grp. gram.:v. t.}
\end{itemize}
\begin{itemize}
\item {Proveniência:(De \textunderscore pre...\textunderscore  + \textunderscore estabelecer\textunderscore )}
\end{itemize}
Estabelecer antecipadamente; determinar prèviamente.
Predispor.
\section{Préa}
\begin{itemize}
\item {Grp. gram.:f.}
\end{itemize}
\begin{itemize}
\item {Proveniência:(Do lat. \textunderscore praeda\textunderscore )}
\end{itemize}
O mesmo que \textunderscore presa\textunderscore .
\section{Preensão}
\begin{itemize}
\item {Grp. gram.:f.}
\end{itemize}
\begin{itemize}
\item {Proveniência:(Lat. \textunderscore prehensio\textunderscore )}
\end{itemize}
Acto de segurar, agarrar ou apanhar. Cf. Júl. Lour. Pinto, \textunderscore Senhor Deput.\textunderscore , 245.
\section{Preexcelência}
\begin{itemize}
\item {Grp. gram.:f.}
\end{itemize}
Qualidade do que é preexcelente.
\section{Preexcelente}
\begin{itemize}
\item {Grp. gram.:adj.}
\end{itemize}
\begin{itemize}
\item {Proveniência:(De \textunderscore pre...\textunderscore  + \textunderscore excellente\textunderscore )}
\end{itemize}
Muito excelente; magnifico.
\section{Preexcellência}
\begin{itemize}
\item {Grp. gram.:f.}
\end{itemize}
Qualidade do que é preexcellente.
\section{Preexcellente}
\begin{itemize}
\item {Grp. gram.:adj.}
\end{itemize}
\begin{itemize}
\item {Proveniência:(De \textunderscore pre...\textunderscore  + \textunderscore excellente\textunderscore )}
\end{itemize}
Muito excellente; magnifico.
\section{Preexcelso}
\begin{itemize}
\item {Grp. gram.:adj.}
\end{itemize}
\begin{itemize}
\item {Proveniência:(Do lat. \textunderscore prae\textunderscore  + \textunderscore excelsus\textunderscore )}
\end{itemize}
Muito alto; sublime. Cf. Alves Mendes, \textunderscore Discursos\textunderscore , 287.
\section{Preexistência}
\begin{itemize}
\item {Grp. gram.:f.}
\end{itemize}
Qualidade do que é preexistente.
\section{Preexistente}
\begin{itemize}
\item {Grp. gram.:adj.}
\end{itemize}
Que preexiste.
\section{Preexistir}
\begin{itemize}
\item {Grp. gram.:v. i.}
\end{itemize}
\begin{itemize}
\item {Proveniência:(De \textunderscore pre...\textunderscore  + \textunderscore existir\textunderscore )}
\end{itemize}
Existir em tempo anterior, existir antes de outro ou de outrem.
\section{Prefação}
\begin{itemize}
\item {Grp. gram.:f.}
\end{itemize}
\begin{itemize}
\item {Proveniência:(Lat. \textunderscore praefatio\textunderscore )}
\end{itemize}
Acto de falar antes.
Aquillo que se diz antes ou antecipadamente.
Prólogo, prefácio, proêmio, preâmbulo; introducção.
\section{Prefaciador}
\begin{itemize}
\item {Grp. gram.:m.}
\end{itemize}
Aquelle que prefacia.
\section{Prefaciar}
\begin{itemize}
\item {Grp. gram.:v. t.}
\end{itemize}
Fazer prefácio a (uma obra literária e scientífica).
Escrever a introducção de.
\section{Prefácio}
\begin{itemize}
\item {Grp. gram.:m.}
\end{itemize}
\begin{itemize}
\item {Proveniência:(Lat. \textunderscore praefatio\textunderscore )}
\end{itemize}
O mesmo que \textunderscore prefação\textunderscore .
Parte da Missa, antes do cânon.
\section{Prefeito}
\begin{itemize}
\item {Grp. gram.:m.}
\end{itemize}
\begin{itemize}
\item {Utilização:Bras}
\end{itemize}
\begin{itemize}
\item {Proveniência:(Do lat. \textunderscore praefectus\textunderscore )}
\end{itemize}
Chefe de uma prefeitura, no Império Romano.
Administrador geral de um departamento, em França.
Superior de um convento.
Empregado collegial, encarregado de vigiar os estudantes.
Magistrado, em alguns cantões da Suissa.
Chefe da municipalidade.
\section{Prefeitura}
\begin{itemize}
\item {Grp. gram.:f.}
\end{itemize}
\begin{itemize}
\item {Proveniência:(Do lat. \textunderscore praefectura\textunderscore )}
\end{itemize}
Divisão administrativa do Império Romano.
Cargo de prefeito.
Casa, onde estão os escritórios do prefeito departamental, ou do prefeito de policia, em França.
\section{Preferência}
\begin{itemize}
\item {Grp. gram.:f.}
\end{itemize}
\begin{itemize}
\item {Utilização:Jur.}
\end{itemize}
Acto ou effeito de preferir uma pessôa ou coisa a outra.
Predilecção; manifestação de agrado ou distincção.
Anteposição.
Direito, que certas pessoas têm, de, preço por preço, haver para si as coisas vendidas, de preferência a outras pessoas: \textunderscore naquelle processo houve concurso de preferências\textunderscore .
\section{Preferente}
\begin{itemize}
\item {Grp. gram.:m. ,  f.  e  adj.}
\end{itemize}
\begin{itemize}
\item {Proveniência:(Lat. \textunderscore praeferens\textunderscore )}
\end{itemize}
Pessôa, que prefere.
\section{Preferir}
\begin{itemize}
\item {Grp. gram.:v. t.}
\end{itemize}
\begin{itemize}
\item {Grp. gram.:V. i.}
\end{itemize}
\begin{itemize}
\item {Proveniência:(Lat. \textunderscore praeferre\textunderscore )}
\end{itemize}
Antepor, dar primazia a.
Escolher.
Querer antes.
Têr predilecção por.
Sêr preferido.
Têr primazia: \textunderscore o dever prefere á devoção\textunderscore .
\section{Preferível}
\begin{itemize}
\item {Grp. gram.:adj.}
\end{itemize}
\begin{itemize}
\item {Proveniência:(De \textunderscore preferir\textunderscore )}
\end{itemize}
Que póde ou deve sêr preferido.
\section{Préfica}
\begin{itemize}
\item {Grp. gram.:f.}
\end{itemize}
\begin{itemize}
\item {Proveniência:(Lat. \textunderscore praefica\textunderscore )}
\end{itemize}
Carpideira que, entre os Romanos, era assalariada para chorar nos cortejos fúnebres.
\section{Prefiguração}
\begin{itemize}
\item {Grp. gram.:f.}
\end{itemize}
\begin{itemize}
\item {Proveniência:(Do lat. \textunderscore praefiguratio\textunderscore )}
\end{itemize}
Acto de prefigurar.
Representação daquillo que ainda não existe, mas que ha de existir ou póde existir ou se receia que exista.
\section{Prefigurar}
\begin{itemize}
\item {Grp. gram.:v. t.}
\end{itemize}
\begin{itemize}
\item {Grp. gram.:V. i.}
\end{itemize}
\begin{itemize}
\item {Proveniência:(De \textunderscore pre...\textunderscore  + \textunderscore figurar\textunderscore )}
\end{itemize}
Figurar ou representar antecipadamente (coisa futura)
Presuppor.
Figurar, imaginando.
Antolhar-se, afigurar-se, parecer:«\textunderscore nenhuma voragem se lhes prefigura mais desastrosa.\textunderscore »Cf. Camillo, \textunderscore Caveira\textunderscore , 351.
\section{Prefinir}
\begin{itemize}
\item {Grp. gram.:v. t.}
\end{itemize}
\begin{itemize}
\item {Proveniência:(Lat. \textunderscore praefinire\textunderscore )}
\end{itemize}
Preestabelecer, determinar antecipadamente; aprazar.
\section{Prefixação}
\begin{itemize}
\item {fónica:csa}
\end{itemize}
\begin{itemize}
\item {Grp. gram.:f.}
\end{itemize}
Acto ou effeito de prefixar.
\section{Prefixadamente}
\begin{itemize}
\item {fónica:csa}
\end{itemize}
\begin{itemize}
\item {Grp. gram.:adv.}
\end{itemize}
\begin{itemize}
\item {Proveniência:(De \textunderscore prefixar\textunderscore )}
\end{itemize}
De modo prefixo.
\section{Prefixar}
\begin{itemize}
\item {fónica:csar}
\end{itemize}
\begin{itemize}
\item {Grp. gram.:v. t.}
\end{itemize}
\begin{itemize}
\item {Proveniência:(De \textunderscore pre...\textunderscore  + \textunderscore fixar\textunderscore )}
\end{itemize}
Fixar antecipadamente; prefinir; prescrever.
\section{Prefixativo}
\begin{itemize}
\item {fónica:csa}
\end{itemize}
\begin{itemize}
\item {Grp. gram.:adj.}
\end{itemize}
\begin{itemize}
\item {Proveniência:(De \textunderscore prefixo\textunderscore )}
\end{itemize}
Diz-se das línguas, em que a flexão se faz por prefixos, como no banto.
\section{Prefixo}
\begin{itemize}
\item {fónica:cso}
\end{itemize}
\begin{itemize}
\item {Grp. gram.:m.}
\end{itemize}
\begin{itemize}
\item {Utilização:Gram.}
\end{itemize}
\begin{itemize}
\item {Proveniência:(Lat. \textunderscore praefixus\textunderscore )}
\end{itemize}
Sýllaba ou sýllabas, que precedem a raíz de uma palavra, modificando o sentido desta, e formando palavra nova.
\section{Prefloração}
\begin{itemize}
\item {Grp. gram.:f.}
\end{itemize}
\begin{itemize}
\item {Utilização:Bot.}
\end{itemize}
\begin{itemize}
\item {Proveniência:(Do lat. \textunderscore prae\textunderscore  + \textunderscore floratio\textunderscore )}
\end{itemize}
Disposição das differentes partes de uma flôr, antes de esta se expandir ou desabrochar.
\section{Preflorescencia}
\begin{itemize}
\item {Grp. gram.:f.}
\end{itemize}
O mesmo que \textunderscore prefloração\textunderscore .
\section{Prefoliação}
\begin{itemize}
\item {Grp. gram.:f.}
\end{itemize}
\begin{itemize}
\item {Utilização:Bot.}
\end{itemize}
\begin{itemize}
\item {Proveniência:(Lat. \textunderscore praefoliatio\textunderscore )}
\end{itemize}
Disposição ou relação recíproca das differentes partes de uma fôlha vegetal.
\section{Prefulgência}
\begin{itemize}
\item {Grp. gram.:f.}
\end{itemize}
Qualidade de prefulgente.
\section{Prefulgente}
\begin{itemize}
\item {Grp. gram.:adj.}
\end{itemize}
\begin{itemize}
\item {Proveniência:(Lat. \textunderscore praefulgens\textunderscore )}
\end{itemize}
Que prefulge.
\section{Prefulgir}
\begin{itemize}
\item {Grp. gram.:v. i.}
\end{itemize}
\begin{itemize}
\item {Proveniência:(Lat. \textunderscore praefulgere\textunderscore )}
\end{itemize}
Fulgir muito; resplandecer.
Brilhar primeiro que outro ou outrem.
\section{Prefulguração}
\begin{itemize}
\item {Grp. gram.:f.}
\end{itemize}
Acto de prefulgurar.
\section{Prefulgurar}
\begin{itemize}
\item {Grp. gram.:v. i.}
\end{itemize}
\begin{itemize}
\item {Proveniência:(Do lat. \textunderscore praefulgurare\textunderscore )}
\end{itemize}
Brilhar muito, fulgurar intensamente.
\section{Prega}
\begin{itemize}
\item {Grp. gram.:f.}
\end{itemize}
\begin{itemize}
\item {Proveniência:(De \textunderscore pregar\textunderscore )}
\end{itemize}
Dobra, que se faz num estôfo.
Aquillo que se assemelha ás pregas.
Ruga.
Carquilha defeituosa de um estôfo, ou dobra casual.
Depressão de terreno.
\section{Prègação}
\begin{itemize}
\item {Grp. gram.:f.}
\end{itemize}
\begin{itemize}
\item {Utilização:Fam.}
\end{itemize}
\begin{itemize}
\item {Proveniência:(Do lat. \textunderscore praedicatio\textunderscore )}
\end{itemize}
Acto de prègar.
Prédica; sermão:«\textunderscore ...se diga por sua tenção hũa pregação com hũa missa rezada.\textunderscore »(De um testamento de 1694)
Discurso maçador.
Ralho, reprehensão, admoestação.
\section{Pregadeira}
\begin{itemize}
\item {Grp. gram.:f.}
\end{itemize}
\begin{itemize}
\item {Proveniência:(De \textunderscore pregar\textunderscore )}
\end{itemize}
Pequena almofada, em que se pregam agulhas, alfinetes, etc.
\section{Pregado}
\begin{itemize}
\item {Grp. gram.:m.}
\end{itemize}
O mesmo que \textunderscore rodovalho\textunderscore .
\section{Pregador}
\begin{itemize}
\item {Grp. gram.:m.  e  adj.}
\end{itemize}
\begin{itemize}
\item {Proveniência:(De \textunderscore pregar\textunderscore )}
\end{itemize}
Aquelle ou aquillo que segura com prego ou pregos.
O que abotôa.
\section{Prègador}
\begin{itemize}
\item {Grp. gram.:m.}
\end{itemize}
\begin{itemize}
\item {Utilização:Fam.}
\end{itemize}
\begin{itemize}
\item {Utilização:Ant.}
\end{itemize}
\begin{itemize}
\item {Proveniência:(Do lat. \textunderscore praedicator\textunderscore )}
\end{itemize}
O que faz prègações.
Orador sagrado.
Aquelle que ralha ou admoésta.
Religioso da Ordem de San-Domingos.
\section{Pregadura}
\begin{itemize}
\item {Grp. gram.:f.}
\end{itemize}
\begin{itemize}
\item {Proveniência:(De \textunderscore pregar\textunderscore )}
\end{itemize}
Série de pregos, segurando ou adornando; pregaria.
\section{Pregagem}
\begin{itemize}
\item {Grp. gram.:f.}
\end{itemize}
Acto de pregar.
\section{Pregalhar}
\begin{itemize}
\item {Grp. gram.:v. i.}
\end{itemize}
\begin{itemize}
\item {Utilização:Prov.}
\end{itemize}
\begin{itemize}
\item {Utilização:trasm.}
\end{itemize}
Pregar pregos.
\section{Prègalhas}
\begin{itemize}
\item {Grp. gram.:f.}
\end{itemize}
\begin{itemize}
\item {Utilização:Ant.}
\end{itemize}
\begin{itemize}
\item {Proveniência:(De \textunderscore prègar\textunderscore ^2)}
\end{itemize}
Súpplicas, rogos; preces.
\section{Pregalho}
\begin{itemize}
\item {Grp. gram.:m.}
\end{itemize}
\begin{itemize}
\item {Utilização:Náut.}
\end{itemize}
\begin{itemize}
\item {Proveniência:(De \textunderscore pregar\textunderscore )}
\end{itemize}
Cabo que serve de adriça aos toldos.
\section{Pregamento}
\begin{itemize}
\item {Grp. gram.:m.}
\end{itemize}
O mesmo que \textunderscore pregagem\textunderscore .
\section{Pregão}
\begin{itemize}
\item {Grp. gram.:m.}
\end{itemize}
\begin{itemize}
\item {Grp. gram.:Pl.}
\end{itemize}
\begin{itemize}
\item {Proveniência:(Do lat. \textunderscore praeco\textunderscore )}
\end{itemize}
Acto de apregoar.
Proclamação pública; divulgação.
Reclamo.
Proclamas de casamento.
\section{Pregar}
\begin{itemize}
\item {Grp. gram.:v. t.}
\end{itemize}
\begin{itemize}
\item {Grp. gram.:V. i.}
\end{itemize}
\begin{itemize}
\item {Proveniência:(Do lat. \textunderscore plicare\textunderscore )}
\end{itemize}
Pôr prego ou pregos em.
Fixar com pregos: \textunderscore pregar uma tábua\textunderscore .
Unir, cosendo: \textunderscore pregar folhos\textunderscore .
Applicar: \textunderscore pregou-lhe um pontapé\textunderscore .
Introduzir.
Abotoar.
Causar.
Importunar com.
O mesmo que \textunderscore preguear\textunderscore .
Arremessar alguém, arrastá-lo, conduzi-lo: \textunderscore pregou com elle no chão\textunderscore .
\section{Prègar}
\begin{itemize}
\item {Grp. gram.:v. t.}
\end{itemize}
\begin{itemize}
\item {Grp. gram.:V. i.}
\end{itemize}
\begin{itemize}
\item {Utilização:Fig.}
\end{itemize}
\begin{itemize}
\item {Proveniência:(Do lat. \textunderscore praedicare\textunderscore )}
\end{itemize}
Pronunciar, declamando.
Exaltar, preconizar.
Fazer propaganda de.
Commemorar.
Alardear.
Discursar.
Fazer sermões; evangelizar.
Clamar; vociferar; ralhar.
\section{Prègar}
\begin{itemize}
\item {Grp. gram.:v. t.}
\end{itemize}
\begin{itemize}
\item {Utilização:Ant.}
\end{itemize}
\begin{itemize}
\item {Proveniência:(Do lat. \textunderscore precari\textunderscore )}
\end{itemize}
Rogar, pedir com instância.
\section{Prègaretas}
\begin{itemize}
\item {fónica:garê}
\end{itemize}
\begin{itemize}
\item {Grp. gram.:f. pl.}
\end{itemize}
\begin{itemize}
\item {Proveniência:(De \textunderscore prègar\textunderscore ^1)}
\end{itemize}
Freiras da Ordem de San-Domingos.
\section{Pregaria}
\begin{itemize}
\item {Grp. gram.:f.}
\end{itemize}
Porção de pregos.
Fábrica de pregos.
Pregos, com que se adornam móveis.
\section{Prèglacial}
\begin{itemize}
\item {Grp. gram.:adj.}
\end{itemize}
\begin{itemize}
\item {Utilização:Geol.}
\end{itemize}
Diz-se de uma das cinco phases, que constituem o período plistoceno.
\section{Prèglaciário}
\begin{itemize}
\item {Grp. gram.:adj.}
\end{itemize}
\begin{itemize}
\item {Utilização:Geol.}
\end{itemize}
O mesmo que \textunderscore prèglacial\textunderscore .
\section{Prego}
\begin{itemize}
\item {Grp. gram.:m.}
\end{itemize}
\begin{itemize}
\item {Utilização:Pop.}
\end{itemize}
\begin{itemize}
\item {Utilização:Constr.}
\end{itemize}
\begin{itemize}
\item {Utilização:Bras}
\end{itemize}
\begin{itemize}
\item {Proveniência:(De \textunderscore pregar\textunderscore )}
\end{itemize}
Peça de metal, delgada, ponteaguda de um lado e mais grossa do outro, destinada a cravar-se num ponto ou objecto que se quere segurar ou fixar.
Espécie de grande alfinete, para segurança ou enfeite de chapéus de senhoras ou toucados.
Cravo, brocha.
Casa de penhores.
Porção de massa, com que os estucadores seguram aos fasquiados do tecto o estuque desaggregado.
Espécie de macaco do Amazonas.
\textunderscore Carta de prego\textunderscore , carta cerrada, que o commandante de um navio só abre fóra da barra, e na qual se lhe determina o que tem que fazer.
Documento análogo, para um commandante de tropas.
Pl. *
O mesmo que [[carta de prego|prego]]:«\textunderscore ...como vinha nos pregos pôr a prôa ao Oriente...\textunderscore »Filinto, \textunderscore D. Man.\textunderscore , I, 66.
\section{Pregoar}
\begin{itemize}
\item {Grp. gram.:v. t.}
\end{itemize}
\begin{itemize}
\item {Proveniência:(Do b. lat. \textunderscore pregonare\textunderscore )}
\end{itemize}
O mesmo que \textunderscore apregoar\textunderscore .
Elogiar em público; proclamar.
\section{Pregoeiro}
\begin{itemize}
\item {Grp. gram.:m.}
\end{itemize}
\begin{itemize}
\item {Proveniência:(Do b. lat. hyp. \textunderscore preconarius\textunderscore )}
\end{itemize}
Aquelle que apregôa ou lança pregão.
Leiloeiro.
\section{Pregresso}
\begin{itemize}
\item {Grp. gram.:adj.}
\end{itemize}
\begin{itemize}
\item {Proveniência:(Lat. \textunderscore praegressus\textunderscore )}
\end{itemize}
Decorrido anteriormente.
Que succedeu primeiro, (falando-se principalmente, em Medicina, da história patológica da família de um doente).
\section{Pregueadeira}
\begin{itemize}
\item {Grp. gram.:f.}
\end{itemize}
Instrumento de costureira, para preguear.
\section{Pregueador}
\begin{itemize}
\item {Grp. gram.:m.}
\end{itemize}
O mesmo que \textunderscore pregueadeira\textunderscore .
\section{Preguear}
\begin{itemize}
\item {Grp. gram.:v. t.}
\end{itemize}
Fazer pregas em.
\section{Pregueiro}
\begin{itemize}
\item {Grp. gram.:m.  e  adj.}
\end{itemize}
Fabricante ou vendedor de pregos.
\section{Preguiça}
\begin{itemize}
\item {Grp. gram.:f.}
\end{itemize}
\begin{itemize}
\item {Utilização:Serralh.}
\end{itemize}
\begin{itemize}
\item {Proveniência:(Do lat. \textunderscore pigritia\textunderscore )}
\end{itemize}
Morosidade, negligência.
Vontade de não trabalhar, desamor ao trabalho.
Pachorra.
Molleza, indolência.
Mandriice, vadiagem.
Pau, a que estão pegadas as cangalhas da canoira.
Animal do Brasil, (\textunderscore bradypus\textunderscore ).
Apparelho, para descansar ou encostar uma barra de ferro, em que se trabalha.
\section{Preguiçar}
\begin{itemize}
\item {Grp. gram.:v. t.}
\end{itemize}
Andar ou proceder com preguiça; mandriar.
\section{Preguiceira}
\begin{itemize}
\item {Grp. gram.:f.}
\end{itemize}
\begin{itemize}
\item {Grp. gram.:Pl.}
\end{itemize}
\begin{itemize}
\item {Proveniência:(De \textunderscore preguiça\textunderscore )}
\end{itemize}
O mesmo que \textunderscore preguiceiro\textunderscore .
Cadeira de recôsto.
Rôlo ou bóla, em que se embebem as barbelas das agulhas de meia, para que se não enferrugem.
\section{Preguiceiro}
\begin{itemize}
\item {Grp. gram.:adj.}
\end{itemize}
\begin{itemize}
\item {Grp. gram.:M.}
\end{itemize}
\begin{itemize}
\item {Utilização:Bras}
\end{itemize}
\begin{itemize}
\item {Utilização:Prov.}
\end{itemize}
\begin{itemize}
\item {Utilização:Pesc.}
\end{itemize}
Preguiçoso; que dá vontade de dormir.
Que convida á indolência, ao somno.
Espécie de cama, para dormir a sesta.
Banco comprido, ao lado da lareira.
Escabello, banco de encôsto.
Empregado especial das armações de atum.
(Cp. gall. \textunderscore perguizeiro\textunderscore )
\section{Preguiçosa}
\begin{itemize}
\item {Grp. gram.:f.}
\end{itemize}
\begin{itemize}
\item {Utilização:Bras}
\end{itemize}
\begin{itemize}
\item {Proveniência:(De \textunderscore preguiçoso\textunderscore )}
\end{itemize}
Pequenina abelha, que deixa que lhe tirem impunemente o mel.
\section{Preguiçosamente}
\begin{itemize}
\item {Grp. gram.:adv.}
\end{itemize}
De modo preguiçoso; com preguiça; indolentemente.
\section{Preguiçoso}
\begin{itemize}
\item {Grp. gram.:adj.}
\end{itemize}
Que revela ou tem preguiça; mandrião.
Sereno, calmo.
\section{Preguista}
\begin{itemize}
\item {Grp. gram.:m.}
\end{itemize}
\begin{itemize}
\item {Utilização:Pop.}
\end{itemize}
\begin{itemize}
\item {Proveniência:(De \textunderscore prego\textunderscore )}
\end{itemize}
Aquelle que tem casa de penhores; agiota.
\section{Pregunta}
\begin{itemize}
\item {Grp. gram.:f.}
\end{itemize}
\begin{itemize}
\item {Proveniência:(De \textunderscore preguntar\textunderscore )}
\end{itemize}
Phrase ou phrases, com que se interroga; interrogação.
Inquirição; quesito.
\section{Preguntar}
\textunderscore v. t.\textunderscore  (e der.)
(Forma preferível a \textunderscore perguntar\textunderscore , etc. Cf. \textunderscore Eufrosina\textunderscore , 71, 88 e 94; Usque, 41 v.^o Cp. \textunderscore perguntar\textunderscore )
\section{Pregustação}
\begin{itemize}
\item {Grp. gram.:f.}
\end{itemize}
Acto de pregustar.
\section{Pregustar}
\begin{itemize}
\item {Grp. gram.:v. i.}
\end{itemize}
\begin{itemize}
\item {Proveniência:(Lat. \textunderscore praegustare\textunderscore )}
\end{itemize}
Provar (comida ou bebida)
Beber antes de outrem; prelibar. Cf. Castilho, \textunderscore Fausto\textunderscore , 8.
\section{Prehensão}
\begin{itemize}
\item {Grp. gram.:f.}
\end{itemize}
\begin{itemize}
\item {Proveniência:(Lat. \textunderscore prehensio\textunderscore )}
\end{itemize}
Acto de segurar, agarrar ou apanhar. Cf. Júl. Lour. Pinto, \textunderscore Senhor Deput.\textunderscore , 245.
\section{Prehistória}
\begin{itemize}
\item {Grp. gram.:f.}
\end{itemize}
\begin{itemize}
\item {Proveniência:(De \textunderscore pre...\textunderscore  + \textunderscore história\textunderscore )}
\end{itemize}
História dos tempos que precederam os chamados tempos históricos.
\section{Prehistórico}
\begin{itemize}
\item {Grp. gram.:adj.}
\end{itemize}
\begin{itemize}
\item {Proveniência:(De \textunderscore pre...\textunderscore  + \textunderscore histórico\textunderscore )}
\end{itemize}
Anterior aos tempos históricos.
\section{Preistória}
\begin{itemize}
\item {fónica:pre-is}
\end{itemize}
\begin{itemize}
\item {Grp. gram.:f.}
\end{itemize}
\begin{itemize}
\item {Proveniência:(De \textunderscore pre...\textunderscore  + \textunderscore história\textunderscore )}
\end{itemize}
História dos tempos que precederam os chamados tempos históricos.
\section{Preistórico}
\begin{itemize}
\item {fónica:pre-is}
\end{itemize}
\begin{itemize}
\item {Grp. gram.:adj.}
\end{itemize}
\begin{itemize}
\item {Proveniência:(De \textunderscore pre...\textunderscore  + \textunderscore histórico\textunderscore )}
\end{itemize}
Anterior aos tempos históricos.
\section{Preia}
\begin{itemize}
\item {Grp. gram.:f.}
\end{itemize}
\begin{itemize}
\item {Proveniência:(Do lat. \textunderscore praeda\textunderscore )}
\end{itemize}
O mesmo que \textunderscore presa\textunderscore .
\section{Preia-mar}
\begin{itemize}
\item {Grp. gram.:f.}
\end{itemize}
(V.preamar)
\section{Preitar}
\begin{itemize}
\item {Grp. gram.:v. t.}
\end{itemize}
\begin{itemize}
\item {Utilização:Ant.}
\end{itemize}
\begin{itemize}
\item {Proveniência:(De \textunderscore preito\textunderscore )}
\end{itemize}
Pagar, satisfazer.
\section{Preitear}
\begin{itemize}
\item {Grp. gram.:v. t.}
\end{itemize}
Render preito a.
\section{Preitegar}
\begin{itemize}
\item {Grp. gram.:v. t.}
\end{itemize}
O mesmo que \textunderscore preitejar\textunderscore .
\section{Preitejar}
\begin{itemize}
\item {Grp. gram.:v. t.}
\end{itemize}
\begin{itemize}
\item {Utilização:Ant.}
\end{itemize}
\begin{itemize}
\item {Proveniência:(De \textunderscore preito\textunderscore )}
\end{itemize}
O mesmo que \textunderscore preitear\textunderscore .
Combinar, ajustar.
\section{Preitesia}
\begin{itemize}
\item {Grp. gram.:f.}
\end{itemize}
\begin{itemize}
\item {Utilização:Ant.}
\end{itemize}
Pacto, contrato, ajuste.
(Cast. \textunderscore pleitesia\textunderscore )
\section{Preito}
\begin{itemize}
\item {Grp. gram.:m.}
\end{itemize}
\begin{itemize}
\item {Utilização:Ant.}
\end{itemize}
\begin{itemize}
\item {Proveniência:(Do lat. \textunderscore placitum\textunderscore )}
\end{itemize}
Pacto.
Vassallagem.
Sujeição.
Tributo de vassalagem.
Respeito.
Homenagem.
Ajuste, convenção, contrato.
\section{P. R. e J.}
Abrev. de \textunderscore pede recebimento e justiça\textunderscore , us. no fim de algumas petições.
\section{Prejacente}
\begin{itemize}
\item {Grp. gram.:adj.}
\end{itemize}
\begin{itemize}
\item {Utilização:Ant.}
\end{itemize}
\begin{itemize}
\item {Proveniência:(De \textunderscore pre...\textunderscore  + \textunderscore jacente\textunderscore )}
\end{itemize}
Já referido, citado ou exposto.
\section{Prejereba}
\begin{itemize}
\item {Grp. gram.:f.}
\end{itemize}
\begin{itemize}
\item {Utilização:Bras}
\end{itemize}
Peixe saboroso do Rio-Grande-do-Sul.
\section{Prejudicador}
\begin{itemize}
\item {Grp. gram.:m.  e  adj.}
\end{itemize}
O que prejudica.
\section{Prejudicar}
\begin{itemize}
\item {Grp. gram.:v. t.}
\end{itemize}
\begin{itemize}
\item {Proveniência:(Lat. \textunderscore praejudicare\textunderscore )}
\end{itemize}
Fazer mal ou damno a.
Lesar; damnificar.
Embaraçar.
Desservir.
Deminuir o valor de.
\section{Prejudicial}
\begin{itemize}
\item {Grp. gram.:adj.}
\end{itemize}
\begin{itemize}
\item {Proveniência:(Lat. \textunderscore praejudicialis\textunderscore )}
\end{itemize}
Que prejudica; que causa prejuizo ou damno.
\section{Prejudicialmente}
\begin{itemize}
\item {Grp. gram.:adv.}
\end{itemize}
De modo prejudicial.
\section{Prejuízo}
\begin{itemize}
\item {Grp. gram.:m.}
\end{itemize}
\begin{itemize}
\item {Proveniência:(Lat. \textunderscore praejudicium\textunderscore )}
\end{itemize}
Acto ou effeito de prejudicar.
Preconceito, superstição.
\section{Prejulgar}
\begin{itemize}
\item {Grp. gram.:v. t.}
\end{itemize}
\begin{itemize}
\item {Utilização:P. us.}
\end{itemize}
\begin{itemize}
\item {Proveniência:(De \textunderscore pre...\textunderscore  + \textunderscore julgar\textunderscore )}
\end{itemize}
Julgar antecipadamente; avaliar com antecipação. Cf. Dom. Vieira, \textunderscore Thes. da Ling. Port.\textunderscore , vb. \textunderscore diffidencia\textunderscore .
\section{Prelação}
\begin{itemize}
\item {Grp. gram.:f.}
\end{itemize}
\begin{itemize}
\item {Proveniência:(Lat. \textunderscore praelatio\textunderscore )}
\end{itemize}
Direito, que os filhos tinham, de sêr providos, com preferência a outrem, nos cargos do seus pais.
\section{Prelacia}
\begin{itemize}
\item {Grp. gram.:f.}
\end{itemize}
\begin{itemize}
\item {Utilização:Ant.}
\end{itemize}
O mesmo que \textunderscore prelazia\textunderscore . Cf. R. Pina, \textunderscore João II\textunderscore , c. XX.
\section{Prelacial}
\begin{itemize}
\item {Grp. gram.:adj.}
\end{itemize}
\begin{itemize}
\item {Proveniência:(De \textunderscore prelacia\textunderscore )}
\end{itemize}
Relativo a Prelado; próprio de Prelado. Cf. Camillo, \textunderscore Quéda\textunderscore , 62.
\section{Prelaciar}
\begin{itemize}
\item {Grp. gram.:v. i.}
\end{itemize}
Exercer prelacia. Cf. \textunderscore Eufrosina\textunderscore , 145.
\section{Prelada}
\begin{itemize}
\item {Grp. gram.:f.}
\end{itemize}
\begin{itemize}
\item {Grp. gram.:F.  e  adj.}
\end{itemize}
\begin{itemize}
\item {Utilização:Prov.}
\end{itemize}
\begin{itemize}
\item {Utilização:trasm.}
\end{itemize}
\begin{itemize}
\item {Utilização:beir.}
\end{itemize}
\begin{itemize}
\item {Proveniência:(De \textunderscore prelado\textunderscore )}
\end{itemize}
Superiora de um convento.
Mulhér sentenciosa, doutora, pernóstica.
\section{Preladia}
\begin{itemize}
\item {Grp. gram.:f.}
\end{itemize}
\begin{itemize}
\item {Proveniência:(De \textunderscore Prelado\textunderscore )}
\end{itemize}
O mesmo que \textunderscore prelazia\textunderscore .
\section{Prelado}
\begin{itemize}
\item {Grp. gram.:m.}
\end{itemize}
\begin{itemize}
\item {Proveniência:(Do lat. \textunderscore praelatus\textunderscore )}
\end{itemize}
Título de certas dignidades ecclesiásticas.
Reitor da Universidade de Coímbra.
\section{Prelatício}
\begin{itemize}
\item {Grp. gram.:adj.}
\end{itemize}
\begin{itemize}
\item {Proveniência:(Do lat. \textunderscore praelatus\textunderscore )}
\end{itemize}
Relativo a Prelado ou a prelazia.
\section{Prelativo}
\begin{itemize}
\item {Grp. gram.:adj.}
\end{itemize}
\begin{itemize}
\item {Proveniência:(Lat. \textunderscore praelativus\textunderscore )}
\end{itemize}
Que envolve ou exprime preferência.
Relativo a prelação.
Que tem superioridade. Cf. Th. Braga, \textunderscore Mod. Ideias\textunderscore , II, 488.
\section{Prelatura}
\begin{itemize}
\item {Grp. gram.:f.}
\end{itemize}
O mesmo que \textunderscore prelazia\textunderscore .
\section{Prelazia}
\begin{itemize}
\item {Grp. gram.:f.}
\end{itemize}
Cargo, dignidade ou jurisdição de Prelado.
(Cast. \textunderscore prelacia\textunderscore )
\section{Prelecção}
\begin{itemize}
\item {Grp. gram.:f.}
\end{itemize}
\begin{itemize}
\item {Proveniência:(Lat. \textunderscore praelectio\textunderscore )}
\end{itemize}
Acto de preleccionar; lição; discurso ou conferência didáctica.
\section{Preleccionar}
\begin{itemize}
\item {Grp. gram.:v. t.}
\end{itemize}
\begin{itemize}
\item {Grp. gram.:V. i.}
\end{itemize}
\begin{itemize}
\item {Proveniência:(Do lat. \textunderscore praelectio\textunderscore )}
\end{itemize}
Dar lição a.
Dar lição sôbre.
Discorrer á cêrca de.
Dar lições.
Discursar em público.
\section{Preleccionista}
\begin{itemize}
\item {Grp. gram.:m.}
\end{itemize}
Aquelle que prelecciona.
\section{Prelector}
\begin{itemize}
\item {Grp. gram.:m.}
\end{itemize}
\begin{itemize}
\item {Proveniência:(Lat. \textunderscore praelector\textunderscore )}
\end{itemize}
Aquelle que prelecciona.
Explicador; professor. Cf. Th. Braga, \textunderscore Mod. Ideias\textunderscore , II, 190.
\section{Prelevar}
\begin{itemize}
\item {Grp. gram.:v. i.}
\end{itemize}
\begin{itemize}
\item {Grp. gram.:V. t.}
\end{itemize}
\begin{itemize}
\item {Proveniência:(Lat. \textunderscore praelevare\textunderscore )}
\end{itemize}
O mesmo que \textunderscore sobrelevar\textunderscore .
Desculpar.
\section{Prelibação}
\begin{itemize}
\item {Grp. gram.:f.}
\end{itemize}
\begin{itemize}
\item {Proveniência:(Lat. \textunderscore praelibatio\textunderscore )}
\end{itemize}
Acto ou effeito de prelibar.
\section{Prelibar}
\begin{itemize}
\item {Grp. gram.:v. t.}
\end{itemize}
\begin{itemize}
\item {Proveniência:(Lat. \textunderscore praelibare\textunderscore )}
\end{itemize}
Libar antecipadamente; provar; ante-gostar.
\section{Preliminar}
\begin{itemize}
\item {Grp. gram.:adj.}
\end{itemize}
\begin{itemize}
\item {Grp. gram.:M.}
\end{itemize}
\begin{itemize}
\item {Proveniência:(Lat. \textunderscore praeliminaris\textunderscore )}
\end{itemize}
Que antecede o assumpto principal.
Que serve de introducção; preambular: \textunderscore advertência preliminar\textunderscore .
Aquillo que antecede o assumpto principal.
Condição prévia.
Introducção, prólogo.
\section{Prélio}
\begin{itemize}
\item {Grp. gram.:m.}
\end{itemize}
\begin{itemize}
\item {Utilização:Poét.}
\end{itemize}
\begin{itemize}
\item {Proveniência:(Lat. \textunderscore praelium\textunderscore )}
\end{itemize}
Batalha.
Acto de batalhar; luta. Cf. F. Barreto, \textunderscore Eneida\textunderscore , VIII, 60.
\section{Prelo}
\begin{itemize}
\item {Grp. gram.:m.}
\end{itemize}
\begin{itemize}
\item {Proveniência:(Lat. \textunderscore prelum\textunderscore )}
\end{itemize}
Máquina typográphica, para imprimir; prensa.
\section{Prelucidação}
\begin{itemize}
\item {Grp. gram.:f.}
\end{itemize}
\begin{itemize}
\item {Proveniência:(Do lat. \textunderscore praelucidus\textunderscore )}
\end{itemize}
Esclarecimento prévio; elucidação preambular. Cf. Camillo, \textunderscore M. de Pombal\textunderscore , 128.
\section{Prelúcido}
\begin{itemize}
\item {Grp. gram.:adj.}
\end{itemize}
\begin{itemize}
\item {Proveniência:(Lat. \textunderscore praelucidus\textunderscore )}
\end{itemize}
Muito brilhante. Cf. Camillo, \textunderscore Vinho do Pôrto\textunderscore , 18.
\section{Preludiar}
\begin{itemize}
\item {Grp. gram.:v. t.}
\end{itemize}
\begin{itemize}
\item {Grp. gram.:V. i.}
\end{itemize}
\begin{itemize}
\item {Proveniência:(De \textunderscore prelúdio\textunderscore )}
\end{itemize}
Fazer prelúdios a.
Predispor.
Prefaciar.
Iniciar.
Ensaiar um instrumento ou a voz, antes de começar a tocar ou a cantar.
\section{Prelúdio}
\begin{itemize}
\item {Grp. gram.:m.}
\end{itemize}
\begin{itemize}
\item {Proveniência:(Lat. \textunderscore praeludium\textunderscore )}
\end{itemize}
Acto ou exercício preliminar.
Iniciação, preparação.
Prenúncio.
Introducção, proêmio.
Exercício musical, para ensaiar a voz ou um instrumento.
\section{Preluzente}
\begin{itemize}
\item {Grp. gram.:adj.}
\end{itemize}
\begin{itemize}
\item {Proveniência:(Do lat. \textunderscore praelucens\textunderscore )}
\end{itemize}
Que preluz; muito brilhante.
\section{Preluzir}
\begin{itemize}
\item {Grp. gram.:v. i.}
\end{itemize}
\begin{itemize}
\item {Proveniência:(Lat. \textunderscore praelucere\textunderscore )}
\end{itemize}
Brilhar muito, prefulgir.
\section{Prema}
\begin{itemize}
\item {Grp. gram.:f.}
\end{itemize}
\begin{itemize}
\item {Utilização:Des.}
\end{itemize}
Acto ou effeito de premar; oppressão; peia.
\section{Premar}
\begin{itemize}
\item {Grp. gram.:v. t.}
\end{itemize}
\begin{itemize}
\item {Utilização:Des.}
\end{itemize}
Opprimir, vexar; affligir; violentar.
(Alter. de \textunderscore premêr\textunderscore )
\section{Premática}
\begin{itemize}
\item {Grp. gram.:f.}
\end{itemize}
\begin{itemize}
\item {Utilização:Ant.}
\end{itemize}
O mesmo que \textunderscore pragmática\textunderscore . Cf. Vieira, XI, 419.
\section{Prematuramente}
\begin{itemize}
\item {Grp. gram.:adv.}
\end{itemize}
De modo prematuro.
Com precocidade; antes do tempo próprio.
\section{Prematuridade}
\begin{itemize}
\item {Grp. gram.:f.}
\end{itemize}
Qualidade ou condição do que é prematuro; precocidade.
\section{Prematuro}
\begin{itemize}
\item {Grp. gram.:adj.}
\end{itemize}
\begin{itemize}
\item {Proveniência:(Lat. \textunderscore praematurus\textunderscore )}
\end{itemize}
O mesmo que \textunderscore precoce\textunderscore ; temporão.
\section{Premedeira}
\begin{itemize}
\item {Grp. gram.:f.}
\end{itemize}
\begin{itemize}
\item {Proveniência:(De \textunderscore premer\textunderscore )}
\end{itemize}
Pedal do tear.
\section{Premeditação}
\begin{itemize}
\item {Grp. gram.:f.}
\end{itemize}
\begin{itemize}
\item {Proveniência:(Lat. \textunderscore praemeditatio\textunderscore )}
\end{itemize}
Acto ou effeito de premeditar.
\section{Premeditar}
\begin{itemize}
\item {Grp. gram.:v. t.}
\end{itemize}
\begin{itemize}
\item {Proveniência:(Lat. \textunderscore praemeditari\textunderscore )}
\end{itemize}
Meditar antecipadamente.
Planear; resolver com antecipação.
\section{Premente}
\begin{itemize}
\item {Grp. gram.:adj.}
\end{itemize}
\begin{itemize}
\item {Proveniência:(Lat. \textunderscore premens\textunderscore )}
\end{itemize}
Que preme.
\section{Premer}
\begin{itemize}
\item {Grp. gram.:v. t.}
\end{itemize}
\begin{itemize}
\item {Proveniência:(Lat. \textunderscore premere\textunderscore )}
\end{itemize}
Fazer pêso ou pressão em.
Calcar.
Opprimir.
Apertar.
Espremer.
\section{Premiador}
\begin{itemize}
\item {Grp. gram.:adj.}
\end{itemize}
\begin{itemize}
\item {Grp. gram.:M.}
\end{itemize}
\begin{itemize}
\item {Proveniência:(De \textunderscore premiar\textunderscore )}
\end{itemize}
Que dá prêmio.
Que recompensa, que galardôa.
Aquelle que dá prêmio, recompensa ou galardão.
\section{Premiar}
\begin{itemize}
\item {Grp. gram.:v. t.}
\end{itemize}
\begin{itemize}
\item {Proveniência:(Lat. \textunderscore praemiari\textunderscore )}
\end{itemize}
Dar prêmio ou galardão a; recompensar.
\section{Premido}
\begin{itemize}
\item {Grp. gram.:adj.}
\end{itemize}
\begin{itemize}
\item {Utilização:T. de Aveiro}
\end{itemize}
\begin{itemize}
\item {Proveniência:(De \textunderscore premer\textunderscore )}
\end{itemize}
Diz-se do pão, fabricado com farinha de trigo e de milho branco.
\section{Prêmio}
\begin{itemize}
\item {Grp. gram.:m.}
\end{itemize}
\begin{itemize}
\item {Proveniência:(Lat. \textunderscore praemium\textunderscore )}
\end{itemize}
Recompensa.
Distincção, conferida a quem sobresái por certos trabalhos ou por certos méritos.
\section{Premir}
\begin{itemize}
\item {Grp. gram.:v. t.}
\end{itemize}
O mesmo ou melhor que \textunderscore premer\textunderscore . Cf. Camillo, \textunderscore Estrêl. Prop.\textunderscore , 23.
\section{Premissa}
\begin{itemize}
\item {Grp. gram.:f.}
\end{itemize}
Antigo direito parochial, em virtude do qual os parochianos davam ao párocho uma parte das primeiras producções dos terrenos.
(Provavelmente, alter. de \textunderscore primícias\textunderscore )
\section{Premissa}
\begin{itemize}
\item {Grp. gram.:f.}
\end{itemize}
\begin{itemize}
\item {Proveniência:(Lat. \textunderscore praemissa\textunderscore )}
\end{itemize}
Cada uma das duas proposições, maior e menor, de um syllogismo.
\section{Premoção}
\begin{itemize}
\item {Grp. gram.:f.}
\end{itemize}
\begin{itemize}
\item {Proveniência:(Lat. \textunderscore praemotio\textunderscore )}
\end{itemize}
Acção de Deus, se influe na vontade das criaturas.
\section{Premonitório}
\begin{itemize}
\item {Grp. gram.:adj.}
\end{itemize}
\begin{itemize}
\item {Proveniência:(De \textunderscore pre...\textunderscore  + \textunderscore monitório\textunderscore )}
\end{itemize}
Que adverte com antecipação.
Que deve tomar-se como aviso.
\section{Premorso}
\begin{itemize}
\item {Grp. gram.:adj.}
\end{itemize}
\begin{itemize}
\item {Utilização:Bot.}
\end{itemize}
\begin{itemize}
\item {Proveniência:(Do lat. \textunderscore prae\textunderscore  + \textunderscore morsus\textunderscore )}
\end{itemize}
Diz-se das fôlhas, quando obtusas e terminadas em chanfraduras desiguaes, como se tivessem sido mordidas.
\section{Premunir}
\begin{itemize}
\item {Grp. gram.:v. i.}
\end{itemize}
\begin{itemize}
\item {Proveniência:(Lat. \textunderscore praemunire\textunderscore )}
\end{itemize}
Acautelar antecipadamente.
Precaver; evitar com antecipação.
\section{Prenantho}
\begin{itemize}
\item {Grp. gram.:m.}
\end{itemize}
\begin{itemize}
\item {Proveniência:(Do gr. \textunderscore prenes\textunderscore  + \textunderscore anthos\textunderscore )}
\end{itemize}
Gênero de plantas, da fam. das compostas.
\section{Prenanto}
\begin{itemize}
\item {Grp. gram.:m.}
\end{itemize}
\begin{itemize}
\item {Proveniência:(Do gr. \textunderscore prenes\textunderscore  + \textunderscore anthos\textunderscore )}
\end{itemize}
Gênero de plantas, da fam. das compostas.
\section{Prenda}
\begin{itemize}
\item {Grp. gram.:f.}
\end{itemize}
\begin{itemize}
\item {Utilização:Fam.}
\end{itemize}
Objecto, com que se brinda alguém.
Presente.
Predicado, qualidade.
Aptidão.
Habilidade.
Pessôa ruím.
\textunderscore Jogos de prendas\textunderscore , jogos, em que a pessôa, que perde, entrega objecto ou prenda, sôbre que há de recair a sentença da pena, que o dono tem de cumprir.
\section{Prendado}
\begin{itemize}
\item {Grp. gram.:adj.}
\end{itemize}
\begin{itemize}
\item {Proveniência:(De \textunderscore prendar\textunderscore )}
\end{itemize}
Que possue prendas, dotes ou qualidades apreciáveis.
\section{Prendar}
\begin{itemize}
\item {Grp. gram.:v. t.}
\end{itemize}
Dar prendas a; presentear.
Tornar hábil, destro.
Premiar.
\section{Prender}
\begin{itemize}
\item {Grp. gram.:v. t.}
\end{itemize}
\begin{itemize}
\item {Utilização:Fig.}
\end{itemize}
\begin{itemize}
\item {Grp. gram.:V. i.}
\end{itemize}
\begin{itemize}
\item {Proveniência:(Lat. \textunderscore prehendere\textunderscore )}
\end{itemize}
Segurar com a mão; agarrar; ligar; enlaçar; unir.
Embaraçar.
Pear; capturar: \textunderscore prender um ladrão\textunderscore .
Cativar; attrahir: \textunderscore encantos, que a todos prendem\textunderscore .
Pôr em contacto, pegar.
Subornar.
Enraizar-se; pegar.
Emperrar.
\section{Prendimento}
\begin{itemize}
\item {Grp. gram.:m.}
\end{itemize}
Acto ou effeito de prender; attracção.
\section{Prenhado}
\begin{itemize}
\item {Grp. gram.:adj.}
\end{itemize}
\begin{itemize}
\item {Proveniência:(Do lat. \textunderscore praegnatus\textunderscore )}
\end{itemize}
O mesmo que \textunderscore prenhe\textunderscore . Cf. G. Vicente, I, 328.
\section{Prenhe}
\begin{itemize}
\item {Grp. gram.:adj.}
\end{itemize}
\begin{itemize}
\item {Utilização:Fig.}
\end{itemize}
\begin{itemize}
\item {Proveniência:(Do lat. \textunderscore praegnare\textunderscore )}
\end{itemize}
Diz-se da fêmea grávida ou da fêmea no período da gestação.
Pleno, repleto; cheio; repassado.
\section{Prenhez}
\begin{itemize}
\item {Grp. gram.:f.}
\end{itemize}
Estado de fêmea prenhe; gravidez.
\section{Prenhidão}
\begin{itemize}
\item {Grp. gram.:f.}
\end{itemize}
O mesmo que \textunderscore prenhez\textunderscore .
\section{Prenoção}
\begin{itemize}
\item {Grp. gram.:f.}
\end{itemize}
\begin{itemize}
\item {Proveniência:(Lat. \textunderscore praenotio\textunderscore )}
\end{itemize}
Preconceito.
Noção imperfeita de alguma coisa.
\section{Prenome}
\begin{itemize}
\item {Grp. gram.:m.}
\end{itemize}
\begin{itemize}
\item {Proveniência:(Lat. \textunderscore praenomen\textunderscore )}
\end{itemize}
Nome, que precede o de família.
\section{Prenominar}
\begin{itemize}
\item {Grp. gram.:v. t.}
\end{itemize}
\begin{itemize}
\item {Proveniência:(Lat. \textunderscore praenominare\textunderscore )}
\end{itemize}
Dar prenome a; designar pelo pronome.
\section{Prenotar}
\begin{itemize}
\item {Grp. gram.:v. t.}
\end{itemize}
\begin{itemize}
\item {Proveniência:(De \textunderscore pre\textunderscore  + \textunderscore notar\textunderscore )}
\end{itemize}
Notar previamente.
\section{Prensa}
\begin{itemize}
\item {Grp. gram.:f.}
\end{itemize}
\begin{itemize}
\item {Utilização:Phot.}
\end{itemize}
\begin{itemize}
\item {Proveniência:(De \textunderscore prensar\textunderscore )}
\end{itemize}
Máquina manual para comprimir ou achatar um objecto entre as suas duas peças principaes.
Pressão ou compressão dêsse objecto.
Prelo.
Caixilho de impressão.
\section{Prensagem}
\begin{itemize}
\item {Grp. gram.:f.}
\end{itemize}
Acto de prensar.
\section{Prensar}
\begin{itemize}
\item {Grp. gram.:v. t.}
\end{itemize}
\begin{itemize}
\item {Proveniência:(Lat. \textunderscore prensare\textunderscore )}
\end{itemize}
Apertar na prensa; comprimir muito; achatar.
\section{Prensista}
\begin{itemize}
\item {Grp. gram.:m.}
\end{itemize}
\begin{itemize}
\item {Utilização:Chapel.}
\end{itemize}
Official, que prensa a massa.
\section{Prenúncia}
\begin{itemize}
\item {Grp. gram.:f.}
\end{itemize}
\begin{itemize}
\item {Utilização:Des.}
\end{itemize}
O mesmo que \textunderscore prenúncio\textunderscore .
\section{Prenunciação}
\begin{itemize}
\item {Grp. gram.:f.}
\end{itemize}
Acto ou effeito de prenunciar.
\section{Prenunciador}
\begin{itemize}
\item {Grp. gram.:m.  e  adj.}
\end{itemize}
\begin{itemize}
\item {Proveniência:(Do lat. \textunderscore praenuntiator\textunderscore )}
\end{itemize}
Aquelle ou aquillo que prenuncia.
\section{Prenunciar}
\begin{itemize}
\item {Grp. gram.:v. t.}
\end{itemize}
\begin{itemize}
\item {Proveniência:(Lat. \textunderscore praenunciare\textunderscore )}
\end{itemize}
Anunciar antecipadamente; prophetizar.
Sêr precursor de.
\section{Prenunciativo}
\begin{itemize}
\item {Grp. gram.:adj.}
\end{itemize}
Que prenuncia; que serve para prenunciar.
\section{Prenúncio}
\begin{itemize}
\item {Grp. gram.:m.}
\end{itemize}
\begin{itemize}
\item {Proveniência:(Lat. \textunderscore praenuntius\textunderscore )}
\end{itemize}
O mesmo que \textunderscore prenunciação\textunderscore .
\section{Preoccupação}
\begin{itemize}
\item {Grp. gram.:f.}
\end{itemize}
\begin{itemize}
\item {Proveniência:(Lat. \textunderscore praeocupatio\textunderscore )}
\end{itemize}
Acto ou effeito de preoccupar.
Preconceito.
Ideia antecipada; ideia fixa.
Inquietação resultante dessa ideia.
\section{Preponente}
\begin{itemize}
\item {Grp. gram.:m. ,  f.  e  adj.}
\end{itemize}
\begin{itemize}
\item {Proveniência:(Lat. \textunderscore praeponens\textunderscore )}
\end{itemize}
Pessôa, que prepõe.
\section{Prepor}
\begin{itemize}
\item {Grp. gram.:v. t.}
\end{itemize}
\begin{itemize}
\item {Proveniência:(Lat. \textunderscore praeponere\textunderscore )}
\end{itemize}
Preferir.
Pôr adeante; antepor.
Escolher.
\section{Preposição}
\begin{itemize}
\item {Grp. gram.:f.}
\end{itemize}
\begin{itemize}
\item {Utilização:Gram.}
\end{itemize}
\begin{itemize}
\item {Proveniência:(Lat. \textunderscore praepositio\textunderscore )}
\end{itemize}
Acção de prepor.
Palavra invariável, que estabelece relação de uma palavra com outra, ou partícula que se antepõe aos nomes, aos pronomes ou a palavras equivalentes, para indicar o nexo lógico que as liga a outras partes do discurso.
\section{Preposicional}
\begin{itemize}
\item {Grp. gram.:adj.}
\end{itemize}
\begin{itemize}
\item {Proveniência:(Do lat. \textunderscore praepositio\textunderscore )}
\end{itemize}
Relativo a preposição.
Em que há preposição. Cf. Aug. Freire, \textunderscore Gramm.\textunderscore , 6.^a ed., 405.
\section{Prepositivo}
\begin{itemize}
\item {Grp. gram.:adj.}
\end{itemize}
\begin{itemize}
\item {Utilização:Gram.}
\end{itemize}
\begin{itemize}
\item {Proveniência:(Lat. \textunderscore praepositivus\textunderscore )}
\end{itemize}
Que se põe adeante ou primeiro.
Relativo á preposição ou que é da natureza della.
\section{Prepósito}
\begin{itemize}
\item {Grp. gram.:m.}
\end{itemize}
\begin{itemize}
\item {Utilização:Ant.}
\end{itemize}
\begin{itemize}
\item {Proveniência:(Lat. \textunderscore praepositus\textunderscore )}
\end{itemize}
Intenção, tenção, propósito.
Antigo Prelado de certas corporações religiosas.
O mesmo que \textunderscore porta-bandeira\textunderscore .
\section{Prepositura}
\begin{itemize}
\item {Grp. gram.:f.}
\end{itemize}
\begin{itemize}
\item {Proveniência:(Lat. \textunderscore praepositura\textunderscore )}
\end{itemize}
Cargo ou dignidade de prepósito.
\section{Preposteração}
\begin{itemize}
\item {Grp. gram.:f.}
\end{itemize}
Acto ou effeito de preposterar.
\section{Preposterar}
\begin{itemize}
\item {Grp. gram.:v. t.}
\end{itemize}
\begin{itemize}
\item {Proveniência:(Lat. \textunderscore praeposterare\textunderscore )}
\end{itemize}
Inverter a ordem de.
\section{Preposteridade}
\begin{itemize}
\item {Grp. gram.:f.}
\end{itemize}
Qualidade do que é prepóstero.
\section{Prepóstero}
\begin{itemize}
\item {Grp. gram.:adj.}
\end{itemize}
\begin{itemize}
\item {Proveniência:(Lat. \textunderscore praeposterus\textunderscore )}
\end{itemize}
Invertido, transposto.
Pôsto ás avessas.
Opposto á boa ordem.
\section{Preposto}
\begin{itemize}
\item {fónica:pôs}
\end{itemize}
\begin{itemize}
\item {Grp. gram.:m.}
\end{itemize}
\begin{itemize}
\item {Proveniência:(Lat. \textunderscore praepositus\textunderscore )}
\end{itemize}
O mesmo que \textunderscore institor\textunderscore .
\section{Prepotência}
\begin{itemize}
\item {Grp. gram.:f.}
\end{itemize}
\begin{itemize}
\item {Proveniência:(Lat. \textunderscore praepotentia\textunderscore )}
\end{itemize}
Qualidade do que é prepotente.
Oppressão; tyrannia.
\section{Prepotente}
\begin{itemize}
\item {Grp. gram.:adj.}
\end{itemize}
\begin{itemize}
\item {Proveniência:(Lat. \textunderscore praepotens\textunderscore )}
\end{itemize}
Muito poderoso, muito influente.
Que tem preponderância.
Que abusa do seu poder ou autoridade.
Oppressor; despótico.
\section{Prepucial}
\begin{itemize}
\item {Grp. gram.:adj.}
\end{itemize}
Relativo ao prepúcio; que nasce ou apparece no prepúcio.
\section{Prepúcio}
\begin{itemize}
\item {Grp. gram.:m.}
\end{itemize}
\begin{itemize}
\item {Proveniência:(Lat. \textunderscore praeputium\textunderscore )}
\end{itemize}
Pelle, que cobre a glande do pênis.
\section{Prequetê}
\begin{itemize}
\item {Grp. gram.:adj.}
\end{itemize}
\begin{itemize}
\item {Utilização:Bras}
\end{itemize}
O mesmo que \textunderscore piriquitete\textunderscore .
\section{Prerafaelismo}
\begin{itemize}
\item {fónica:ra}
\end{itemize}
\begin{itemize}
\item {Grp. gram.:m.}
\end{itemize}
Cultura artística, anteriormente a Rafael.
\section{Prerafaelita}
\begin{itemize}
\item {fónica:ra,fa-e}
\end{itemize}
\begin{itemize}
\item {Grp. gram.:adj.}
\end{itemize}
\begin{itemize}
\item {Proveniência:(De \textunderscore pre...\textunderscore  + \textunderscore Rafael\textunderscore , n. p.)}
\end{itemize}
Anterior aos tempos de Rafael; que antecedeu Rafael. Cf. Camillo, \textunderscore Mar. da Fonte\textunderscore , 334.
\section{Prerogativa}
\begin{itemize}
\item {Grp. gram.:f.}
\end{itemize}
\begin{itemize}
\item {Proveniência:(Lat. \textunderscore praerogativa\textunderscore )}
\end{itemize}
Previlégio; regalia; apanágio.
\section{Prerrafaelismo}
\begin{itemize}
\item {Grp. gram.:m.}
\end{itemize}
Cultura artística, anteriormente a Rafael.
\section{Prerrafaelita}
\begin{itemize}
\item {fónica:fa-e}
\end{itemize}
\begin{itemize}
\item {Grp. gram.:adj.}
\end{itemize}
\begin{itemize}
\item {Proveniência:(De \textunderscore pre...\textunderscore  + \textunderscore Rafael\textunderscore , n. p.)}
\end{itemize}
Anterior aos tempos de Rafael; que antecedeu Rafael. Cf. Camillo, \textunderscore Mar. da Fonte\textunderscore , 334.
\section{Pre-romano}
\begin{itemize}
\item {Grp. gram.:adj.}
\end{itemize}
Anterior á dominação romana.
\section{Prés}
\begin{itemize}
\item {Grp. gram.:adv.}
\end{itemize}
\begin{itemize}
\item {Utilização:Ant.}
\end{itemize}
\begin{itemize}
\item {Proveniência:(Fr. \textunderscore près\textunderscore )}
\end{itemize}
Perto.
\section{Presa}
\begin{itemize}
\item {fónica:prê}
\end{itemize}
\begin{itemize}
\item {Grp. gram.:f.}
\end{itemize}
\begin{itemize}
\item {Utilização:Prov.}
\end{itemize}
\begin{itemize}
\item {Utilização:Prov.}
\end{itemize}
\begin{itemize}
\item {Utilização:Açor}
\end{itemize}
\begin{itemize}
\item {Proveniência:(Do lat. \textunderscore prehensa\textunderscore )}
\end{itemize}
Acto de apprehender ou de apresar; apresamento.
Objectos, apprehendidos ao inimigo.
Objectos, usurpados violentamente.
Estado de uma substância coagulada.
Cada um dos dentes caninos.
Mulhér, que está em prisão.
Peça ou cavidade larga e pouco funda, em que se ajunta água para regas, e donde se extrai, a pouco e pouco, por um bueiro, ou brecha, sendo depois dirigida pelo agricultor sôbre o terreno que se quere regar.
O mesmo que \textunderscore açude\textunderscore .
Elevação, numa estrada ou num terreno, por onde difficilmente sobem carros ou peões.
\section{Presagiador}
\begin{itemize}
\item {fónica:sa}
\end{itemize}
\begin{itemize}
\item {Grp. gram.:m.  e  adj.}
\end{itemize}
O que presagia.
\section{Presagiar}
\begin{itemize}
\item {fónica:sa}
\end{itemize}
\begin{itemize}
\item {Grp. gram.:v. t.}
\end{itemize}
\begin{itemize}
\item {Proveniência:(Lat. \textunderscore praesagiare\textunderscore )}
\end{itemize}
Prognosticar, prenunciar, prophetizar.
\section{Prescriptível}
\begin{itemize}
\item {Grp. gram.:adj.}
\end{itemize}
\begin{itemize}
\item {Utilização:Jur.}
\end{itemize}
\begin{itemize}
\item {Proveniência:(De \textunderscore prescripto\textunderscore )}
\end{itemize}
Que se póde prescrever ou ordenar.
Que é susceptível de prescripção.
\section{Prescripto}
\begin{itemize}
\item {Grp. gram.:adj.}
\end{itemize}
\begin{itemize}
\item {Utilização:Jur.}
\end{itemize}
\begin{itemize}
\item {Proveniência:(De \textunderscore prescrever\textunderscore )}
\end{itemize}
Ordenado.
Que prescreveu.
\section{Prescriptor}
\begin{itemize}
\item {Grp. gram.:m.}
\end{itemize}
\begin{itemize}
\item {Proveniência:(Do lat. \textunderscore praescriptor\textunderscore )}
\end{itemize}
Aquelle que prescreve ou preceitua.
\section{Prescritível}
\begin{itemize}
\item {Grp. gram.:adj.}
\end{itemize}
\begin{itemize}
\item {Utilização:Jur.}
\end{itemize}
\begin{itemize}
\item {Proveniência:(De \textunderscore prescrito\textunderscore )}
\end{itemize}
Que se póde prescrever ou ordenar.
Que é susceptível de prescrição.
\section{Prescrito}
\begin{itemize}
\item {Grp. gram.:adj.}
\end{itemize}
\begin{itemize}
\item {Utilização:Jur.}
\end{itemize}
\begin{itemize}
\item {Proveniência:(De \textunderscore prescrever\textunderscore )}
\end{itemize}
Ordenado.
Que prescreveu.
\section{Prescritor}
\begin{itemize}
\item {Grp. gram.:m.}
\end{itemize}
\begin{itemize}
\item {Proveniência:(Do lat. \textunderscore praescriptor\textunderscore )}
\end{itemize}
Aquele que prescreve ou preceitua.
\section{Presença}
\begin{itemize}
\item {Grp. gram.:f.}
\end{itemize}
\begin{itemize}
\item {Proveniência:(Lat. \textunderscore praesentia\textunderscore )}
\end{itemize}
Existência ou assistência de uma pessôa em determinado lugar.
Existência de um objecto em dado lugar.
Aspecto da physionomia.
Aspecto; vista.
Opinião.
Compleição phýsica.
Porte, modos.
\section{Presencear}
\begin{itemize}
\item {Grp. gram.:v. t.}
\end{itemize}
(V.presenciar)
\section{Presencia}
\begin{itemize}
\item {Grp. gram.:f.}
\end{itemize}
\begin{itemize}
\item {Utilização:Pop.}
\end{itemize}
O mesmo que \textunderscore presença\textunderscore .
\section{Presencial}
\begin{itemize}
\item {Grp. gram.:adj.}
\end{itemize}
\begin{itemize}
\item {Proveniência:(Lat. \textunderscore praesentialis\textunderscore )}
\end{itemize}
Relativo a pessôa ou coisa que está presente.
Feito á vista de alguém; que viu, que presenceou.
\section{Presencialmente}
\begin{itemize}
\item {Grp. gram.:adv.}
\end{itemize}
De modo presencial.
\section{Presenciar}
\begin{itemize}
\item {Grp. gram.:v. t.}
\end{itemize}
\begin{itemize}
\item {Proveniência:(Do lat. \textunderscore praesentia\textunderscore )}
\end{itemize}
Estar presente a.
Vêr; observar: \textunderscore presenciar um crime\textunderscore .
\section{Presentação}
\begin{itemize}
\item {Grp. gram.:f.}
\end{itemize}
\begin{itemize}
\item {Proveniência:(Lat. \textunderscore praesentatio\textunderscore )}
\end{itemize}
O mesmo que \textunderscore apresentação\textunderscore .
\section{Presentaneamente}
\begin{itemize}
\item {Grp. gram.:adv.}
\end{itemize}
De modo presentâneo; rapidamente.
\section{Presentâneo}
\begin{itemize}
\item {Grp. gram.:adj.}
\end{itemize}
\begin{itemize}
\item {Proveniência:(Lat. \textunderscore praesentaneus\textunderscore )}
\end{itemize}
Momentâneo, rápido; efficaz.
\section{Presentar}
\begin{itemize}
\item {Grp. gram.:v. t.}
\end{itemize}
\begin{itemize}
\item {Proveniência:(Lat. \textunderscore praesentare\textunderscore )}
\end{itemize}
O mesmo que \textunderscore apresentar\textunderscore .
\section{Presente}
\begin{itemize}
\item {Grp. gram.:adj.}
\end{itemize}
\begin{itemize}
\item {Utilização:Des.}
\end{itemize}
\begin{itemize}
\item {Grp. gram.:M.}
\end{itemize}
\begin{itemize}
\item {Utilização:Gram.}
\end{itemize}
\begin{itemize}
\item {Proveniência:(Lat. \textunderscore praesens\textunderscore )}
\end{itemize}
Que existe num lugar dado.
Que assiste em pessôa.
Que está á vista.
Que está prestes a realizar-se.
Patente.
Favorável.
Actualidade.
Pessôa, que comparece em dado lugar.
Tempo verbal, que indica actualidade.
Offerta, brinde, dádiva.
\section{Presenteador}
\begin{itemize}
\item {Grp. gram.:m.  e  adj.}
\end{itemize}
Aquelle que presenteia.
\section{Presentear}
\begin{itemize}
\item {Grp. gram.:v. t.}
\end{itemize}
Brindar; dar presente a.
\section{Presentemente}
\begin{itemize}
\item {Grp. gram.:adv.}
\end{itemize}
Actualmente; agora; no tempo presente.
Presencialmente.
\section{Presentido}
\begin{itemize}
\item {fónica:sen}
\end{itemize}
\begin{itemize}
\item {Grp. gram.:adj.}
\end{itemize}
\begin{itemize}
\item {Proveniência:(De \textunderscore presentir\textunderscore )}
\end{itemize}
Que percebe facilmente qualquer pequeno barulho ou rumor.
Que tem desconfianças.
\section{Presentimento}
\begin{itemize}
\item {fónica:sen}
\end{itemize}
\begin{itemize}
\item {Grp. gram.:m.}
\end{itemize}
Acto ou effeito de presentir.
\section{Presentir}
\begin{itemize}
\item {fónica:sen}
\end{itemize}
\begin{itemize}
\item {Grp. gram.:v. t.}
\end{itemize}
\begin{itemize}
\item {Proveniência:(Lat. \textunderscore praesentire\textunderscore )}
\end{itemize}
Sentir antecipadamente.
Prever; presagiar.
Têr suspeitas de.
\section{Presepada}
\begin{itemize}
\item {Grp. gram.:f.}
\end{itemize}
\begin{itemize}
\item {Utilização:Bras}
\end{itemize}
\begin{itemize}
\item {Proveniência:(De \textunderscore presepe\textunderscore ^1)}
\end{itemize}
Barulho, barafunda.
\section{Presepe}
\begin{itemize}
\item {Grp. gram.:m.}
\end{itemize}
\begin{itemize}
\item {Utilização:Bras}
\end{itemize}
O mesmo que \textunderscore mamulengos\textunderscore .
\section{Presepe}
\begin{itemize}
\item {Grp. gram.:m.}
\end{itemize}
\begin{itemize}
\item {Proveniência:(Lat. \textunderscore praesepe\textunderscore )}
\end{itemize}
O mesmo que \textunderscore presépio\textunderscore . Cf. Herculano, \textunderscore Quest. Públ.\textunderscore , II, 330.
\section{Presepeiro}
\begin{itemize}
\item {Grp. gram.:adj.}
\end{itemize}
\begin{itemize}
\item {Utilização:Bras}
\end{itemize}
Raivoso; barulhento.
(Cp. \textunderscore presepada\textunderscore )
\section{Presépio}
\begin{itemize}
\item {Grp. gram.:m.}
\end{itemize}
\begin{itemize}
\item {Utilização:Prov.}
\end{itemize}
\begin{itemize}
\item {Utilização:minh.}
\end{itemize}
\begin{itemize}
\item {Proveniência:(Lat. \textunderscore praesepium\textunderscore )}
\end{itemize}
Lugar, onde se recolhe gado.
Curral; estábulo.
Representação das figuras que, segundo o \textunderscore Evangelho\textunderscore , assistiram ao nascimento de Christo.
Qualquer armação de madeira; tenda, barraca.
\section{Preservação}
\begin{itemize}
\item {fónica:ser}
\end{itemize}
\begin{itemize}
\item {Grp. gram.:f.}
\end{itemize}
Acto ou effeito de preservar.
\section{Preservador}
\begin{itemize}
\item {fónica:ser}
\end{itemize}
\begin{itemize}
\item {Grp. gram.:m.  e  adj.}
\end{itemize}
O que preserva.
\section{Preservar}
\begin{itemize}
\item {fónica:ser}
\end{itemize}
\begin{itemize}
\item {Grp. gram.:v. t.}
\end{itemize}
\begin{itemize}
\item {Proveniência:(Lat. \textunderscore praeservare\textunderscore )}
\end{itemize}
Livrar de algum mal; resguardar, defender.
\section{Preservativo}
\begin{itemize}
\item {fónica:ser}
\end{itemize}
\begin{itemize}
\item {Grp. gram.:adj.}
\end{itemize}
\begin{itemize}
\item {Grp. gram.:M.}
\end{itemize}
Que preserva; próprio para preservar.
Aquillo que preserva.
\section{Pressagiador}
\begin{itemize}
\item {Grp. gram.:m.  e  adj.}
\end{itemize}
O que presagia.
\section{Pressagiar}
\begin{itemize}
\item {Grp. gram.:v. t.}
\end{itemize}
\begin{itemize}
\item {Proveniência:(Lat. \textunderscore praesagiare\textunderscore )}
\end{itemize}
Prognosticar, prenunciar, profetizar.
\section{Pressentido}
\begin{itemize}
\item {Grp. gram.:adj.}
\end{itemize}
\begin{itemize}
\item {Proveniência:(De \textunderscore presentir\textunderscore )}
\end{itemize}
Que percebe facilmente qualquer pequeno barulho ou rumor.
Que tem desconfianças.
\section{Pressentimento}
\begin{itemize}
\item {Grp. gram.:m.}
\end{itemize}
Acto ou effeito de presentir.
\section{Pressentir}
\begin{itemize}
\item {Grp. gram.:v. t.}
\end{itemize}
\begin{itemize}
\item {Proveniência:(Lat. \textunderscore praesentire\textunderscore )}
\end{itemize}
Sentir antecipadamente.
Prever; presagiar.
Têr suspeitas de.
\section{Presservação}
\begin{itemize}
\item {Grp. gram.:f.}
\end{itemize}
Acto ou effeito de preservar.
\section{Presservador}
\begin{itemize}
\item {Grp. gram.:m.  e  adj.}
\end{itemize}
O que preserva.
\section{Presservar}
\begin{itemize}
\item {Grp. gram.:v. t.}
\end{itemize}
\begin{itemize}
\item {Proveniência:(Lat. \textunderscore praeservare\textunderscore )}
\end{itemize}
Livrar de algum mal; resguardar, defender.
\section{Presservativo}
\begin{itemize}
\item {Grp. gram.:adj.}
\end{itemize}
\begin{itemize}
\item {Grp. gram.:M.}
\end{itemize}
Que preserva; próprio para preservar.
Aquillo que preserva.
\section{Pressuposição}
\begin{itemize}
\item {Grp. gram.:f.}
\end{itemize}
Acto ou effeito de pressuppor.
\section{Pressuposto}
\begin{itemize}
\item {fónica:pôs}
\end{itemize}
\begin{itemize}
\item {Grp. gram.:m.}
\end{itemize}
Pressupposição; desígnio, tenção.
Pretexto.
Projecto.
\section{Pressuroso}
\begin{itemize}
\item {Grp. gram.:adj.}
\end{itemize}
\begin{itemize}
\item {Proveniência:(De \textunderscore pressura\textunderscore )}
\end{itemize}
Apressado; afanoso.
Irrequieto.
\section{Prestação}
\begin{itemize}
\item {Grp. gram.:f.}
\end{itemize}
\begin{itemize}
\item {Proveniência:(Lat. \textunderscore praestatio\textunderscore )}
\end{itemize}
Acto ou effeito de prestar.
Quota; cada uma das quantias, que se há de pagar em certo prazo, para extincção de uma só dívida ou encargo.
\section{Prestacionar}
\begin{itemize}
\item {Grp. gram.:v. t.}
\end{itemize}
\begin{itemize}
\item {Utilização:P. us.}
\end{itemize}
\begin{itemize}
\item {Proveniência:(Do lat. \textunderscore praestatio\textunderscore )}
\end{itemize}
Pagar em prestações.
Dar, como prestações. Cf. Camillo, \textunderscore Livro Negro\textunderscore , 139; \textunderscore Olho de Vidro\textunderscore , 138.
\section{Prestadiamente}
\begin{itemize}
\item {Grp. gram.:adv.}
\end{itemize}
De modo prestadio; vantajosamente; com utilidade.
\section{Prestadiço}
\begin{itemize}
\item {Grp. gram.:adj.}
\end{itemize}
\begin{itemize}
\item {Utilização:Des.}
\end{itemize}
O mesmo que \textunderscore prestadio\textunderscore . Cf. Sim. Machado, fol. 46, v.^o
\section{Prestadio}
\begin{itemize}
\item {Grp. gram.:adj.}
\end{itemize}
\begin{itemize}
\item {Proveniência:(De \textunderscore prestar\textunderscore )}
\end{itemize}
Prestável; serviçal; proveitoso: \textunderscore esforços prestadíos\textunderscore .
\section{Prestador}
\begin{itemize}
\item {Grp. gram.:adj.}
\end{itemize}
Que presta; que faz bem.
O mesmo ou melhor que \textunderscore prestadio\textunderscore . Cf. Gaspar Correia, Barros, etc.
\section{Prestameiro}
\begin{itemize}
\item {Grp. gram.:m.}
\end{itemize}
\begin{itemize}
\item {Utilização:Ant.}
\end{itemize}
\begin{itemize}
\item {Proveniência:(Do b. lat. \textunderscore prestamarius\textunderscore )}
\end{itemize}
Aquelle que recebia algum préstamo; cobrador de préstamo. Cf. Herculano, \textunderscore Hist. de Port.\textunderscore , I, 398.
\section{Prestamente}
\begin{itemize}
\item {Grp. gram.:adv.}
\end{itemize}
\begin{itemize}
\item {Proveniência:(De \textunderscore presto\textunderscore )}
\end{itemize}
O mesmo que \textunderscore prestemente\textunderscore .
\section{Prestamento}
\begin{itemize}
\item {Grp. gram.:m.}
\end{itemize}
Acto ou effeito de prestar.
\section{Prestamista}
\begin{itemize}
\item {Grp. gram.:m.  e  f.}
\end{itemize}
\begin{itemize}
\item {Proveniência:(De \textunderscore préstamo\textunderscore )}
\end{itemize}
Pessôa, que empresta dinheiro a juros.
Pessôa, que recebe juros de inscripções da dívida pública.
\section{Préstamo}
\begin{itemize}
\item {Grp. gram.:m.}
\end{itemize}
\begin{itemize}
\item {Utilização:Ant.}
\end{itemize}
Consignação de certa quantidade de frutos ou dinheiro, imposta num terreno, a favor da Corôa, ou de alguma obra pia, ou de qualquer pessôa.
(Refl. do lat. \textunderscore praestare\textunderscore )
\section{Prestança}
\begin{itemize}
\item {Grp. gram.:f.}
\end{itemize}
\begin{itemize}
\item {Utilização:Ant.}
\end{itemize}
O mesmo que \textunderscore prestância\textunderscore .
\section{Prestância}
\begin{itemize}
\item {Grp. gram.:f.}
\end{itemize}
\begin{itemize}
\item {Proveniência:(Lat. \textunderscore praestantia\textunderscore )}
\end{itemize}
Qualidade do que é prestante.
Préstimo.
\section{Prestante}
\begin{itemize}
\item {Grp. gram.:adj.}
\end{itemize}
\begin{itemize}
\item {Proveniência:(Lat. \textunderscore praestans\textunderscore )}
\end{itemize}
Que presta.
Prompto a auxiliar.
Insigne, excellente: \textunderscore cidadão prestante\textunderscore .
\section{Prestar}
\begin{itemize}
\item {Grp. gram.:v. i.}
\end{itemize}
\begin{itemize}
\item {Grp. gram.:V. t.}
\end{itemize}
\begin{itemize}
\item {Proveniência:(Lat. \textunderscore praestare\textunderscore )}
\end{itemize}
Estar ao alcance de alguém para sêr útil.
Aproveitar.
Têr préstimo: \textunderscore não prestar para nada\textunderscore .
Dar, offerecer, dispensar: \textunderscore prestar soccorros\textunderscore .
Fazer, segundo certas condições.
Consagrar.
Emprestar.
Acommodar.
Exhibir.
\section{Prestativo}
\begin{itemize}
\item {Grp. gram.:adj.}
\end{itemize}
\begin{itemize}
\item {Proveniência:(De \textunderscore prestar\textunderscore )}
\end{itemize}
Prompto para servir; prestador.
\section{Prestável}
\begin{itemize}
\item {Grp. gram.:adj.}
\end{itemize}
Que presta ou póde prestar; prestante, serviçal.
\section{Preste}
\begin{itemize}
\item {Grp. gram.:m.}
\end{itemize}
\begin{itemize}
\item {Utilização:Ant.}
\end{itemize}
Sacerdote, padre, presbýtero.
(Cp. fr. \textunderscore prêtre\textunderscore )
\section{Preste}
\begin{itemize}
\item {Grp. gram.:adj.}
\end{itemize}
O mesmo que \textunderscore prestes\textunderscore . Cf. \textunderscore Chrón. dos R. de Bisnaga\textunderscore , 10.
\section{Prestemente}
\begin{itemize}
\item {Grp. gram.:adv.}
\end{itemize}
Com presteza, de modo preste.
\section{Prestes}
\begin{itemize}
\item {Grp. gram.:adj.}
\end{itemize}
\begin{itemize}
\item {Grp. gram.:Adv.}
\end{itemize}
\begin{itemize}
\item {Proveniência:(Do lat. \textunderscore praesto\textunderscore )}
\end{itemize}
Disposto, prompto.
Pròximo, imminente.
Rápido.
Com presteza.
\section{Prestesmente}
\begin{itemize}
\item {Grp. gram.:adv.}
\end{itemize}
(V.prestemente)
\section{Presteza}
\begin{itemize}
\item {Grp. gram.:f.}
\end{itemize}
Qualidade do que é prestes.
Promptidão, ligeireza, agilidade.
\section{Prestidigitação}
\begin{itemize}
\item {Grp. gram.:f.}
\end{itemize}
\begin{itemize}
\item {Proveniência:(Do lat. \textunderscore praesto\textunderscore  + \textunderscore digitus\textunderscore )}
\end{itemize}
Arte de prestidigitador.
\section{Prestidigitador}
\begin{itemize}
\item {Grp. gram.:m.}
\end{itemize}
Escamoteador que, pela rapidez do movimento das mãos, faz deslocar ou desapparecer objectos, sem o espectador saber como.
(Cp. \textunderscore prestidigitação\textunderscore )
\section{Prestigiação}
\begin{itemize}
\item {Grp. gram.:f.}
\end{itemize}
\begin{itemize}
\item {Proveniência:(De \textunderscore prestígio\textunderscore )}
\end{itemize}
Acto ou acção de prestigiador.
Bruxaria.
\section{Prestigiado}
\begin{itemize}
\item {Grp. gram.:adj.}
\end{itemize}
O mesmo que \textunderscore prestigioso\textunderscore .
\section{Prestigiador}
\begin{itemize}
\item {Grp. gram.:m.}
\end{itemize}
\begin{itemize}
\item {Proveniência:(Lat. \textunderscore praestigiator\textunderscore )}
\end{itemize}
O mesmo que \textunderscore prestidigitador\textunderscore .
Escamoteador; burlão.
\section{Presupposição}
\begin{itemize}
\item {fónica:su}
\end{itemize}
\begin{itemize}
\item {Grp. gram.:f.}
\end{itemize}
Acto ou effeito de presuppor.
\section{Presupposto}
\begin{itemize}
\item {fónica:su,pôs}
\end{itemize}
\begin{itemize}
\item {Grp. gram.:m.}
\end{itemize}
Presupposição; desígnio, tenção.
Pretexto.
Projecto.
\section{Presúria}
\begin{itemize}
\item {Grp. gram.:f.}
\end{itemize}
\begin{itemize}
\item {Proveniência:(Do lat. bárb. \textunderscore presura\textunderscore )}
\end{itemize}
Reivindicação ou reconquista, á mão armada.
Repartição de terras, reconquistadas aos Moiros.
Posse de terreno com justo título.
Açude, mota. Cf. Herculano, \textunderscore Hist. de Port.\textunderscore , III, 280.
\section{Pret}
\begin{itemize}
\item {Grp. gram.:m.}
\end{itemize}
(V.pré)
\section{Preta}
\begin{itemize}
\item {fónica:prê}
\end{itemize}
\begin{itemize}
\item {Grp. gram.:f.}
\end{itemize}
Mulhér de raça negra.
Uma das duas espécies de marcas, que designam os tentos no jôgo do bilhar.
\section{Pretalhada}
\begin{itemize}
\item {Grp. gram.:f.}
\end{itemize}
\begin{itemize}
\item {Utilização:Deprec.}
\end{itemize}
Grande número de pretos.
Os Pretos.
\section{Pretalhão}
\begin{itemize}
\item {Grp. gram.:m.}
\end{itemize}
O mesmo que \textunderscore negralhão\textunderscore . Cf. \textunderscore Agostinheida\textunderscore , 136.
\section{Preta-moira}
\begin{itemize}
\item {Grp. gram.:f.}
\end{itemize}
O mesmo que \textunderscore moreto\textunderscore ^1.
\section{Pretaria}
\begin{itemize}
\item {Grp. gram.:f.}
\end{itemize}
O mesmo que \textunderscore pretalhada\textunderscore . Cf. Castilho, \textunderscore D. Quixote\textunderscore , I, 220; Camillo, \textunderscore Estrêl. Prop.\textunderscore , 201.
\section{Pretenção}
\textunderscore f.\textunderscore  (e der.)
(V. \textunderscore pretensão\textunderscore , etc.)
\section{Pretendedor}
\begin{itemize}
\item {Grp. gram.:m.  e  adj.}
\end{itemize}
O que pretende.
\section{Pretendente}
\begin{itemize}
\item {Grp. gram.:m. ,  f.  e  adj.}
\end{itemize}
\begin{itemize}
\item {Proveniência:(Lat. \textunderscore praetendens\textunderscore )}
\end{itemize}
Pessôa, que pretende.
Pessôa, que se julga com algum direito a qualquer coisa.
Pessôa, que aspira a casar ou que já tem em vista a pessôa com quem há de casar.
\section{Pretender}
\begin{itemize}
\item {Grp. gram.:v. t.}
\end{itemize}
\begin{itemize}
\item {Utilização:Des.}
\end{itemize}
\begin{itemize}
\item {Grp. gram.:V. i.}
\end{itemize}
\begin{itemize}
\item {Proveniência:(Lat. \textunderscore praetendere\textunderscore )}
\end{itemize}
Exigir; solicitar.
Aspirar a: \textunderscore pretende sêr médico\textunderscore .
Appetecer.
Diligenciar.
Julgar, sustentar: \textunderscore pretender que a Lua é habitável\textunderscore .
Dizer respeito a.
Fazer diligência, esforçar-se.
\section{Pretendida}
\begin{itemize}
\item {Grp. gram.:f.}
\end{itemize}
\begin{itemize}
\item {Proveniência:(De \textunderscore pretender\textunderscore )}
\end{itemize}
Noiva.
Mulhér, que um homem deseja para casamento.
\section{Pretensão}
\begin{itemize}
\item {Grp. gram.:f.}
\end{itemize}
\begin{itemize}
\item {Grp. gram.:Pl.}
\end{itemize}
\begin{itemize}
\item {Proveniência:(Do lat. \textunderscore praetensus\textunderscore )}
\end{itemize}
Acto ou effeito de pretender.
Supposto direito.
Presumpção, vaidade.
Bazófia, jactância.
Vaidade exaggerada.
\section{Pretensiosa}
\begin{itemize}
\item {Grp. gram.:f.}
\end{itemize}
\begin{itemize}
\item {Proveniência:(De \textunderscore pretensioso\textunderscore )}
\end{itemize}
Mulhér presumida, vaidosa.
\section{Pretensioso}
\begin{itemize}
\item {Grp. gram.:m.  e  adj.}
\end{itemize}
\begin{itemize}
\item {Proveniência:(De \textunderscore pretensão\textunderscore )}
\end{itemize}
O que tem pretensões ou vaidade; soberbo.
\section{Pretenso}
\begin{itemize}
\item {Grp. gram.:adj.}
\end{itemize}
\begin{itemize}
\item {Proveniência:(Lat. \textunderscore praetensum\textunderscore )}
\end{itemize}
Supposto; imaginado: \textunderscore pugnar por um pretenso direito\textunderscore .
Que pretende ou que suppõe que é (qualquer coisa): \textunderscore êste pretenso critico...\textunderscore 
\section{Pretensor}
\begin{itemize}
\item {Grp. gram.:m.  e  adj.}
\end{itemize}
\begin{itemize}
\item {Proveniência:(De \textunderscore pretenso\textunderscore )}
\end{itemize}
O mesmo que \textunderscore pretendedor\textunderscore .
\section{Preter...}
\begin{itemize}
\item {Grp. gram.:pref.}
\end{itemize}
\begin{itemize}
\item {Proveniência:(Lat. \textunderscore praeter\textunderscore )}
\end{itemize}
(design. de \textunderscore além\textunderscore , \textunderscore excesso\textunderscore , etc.)
\section{Preterição}
\begin{itemize}
\item {Grp. gram.:f.}
\end{itemize}
\begin{itemize}
\item {Proveniência:(Lat. \textunderscore praeteritio\textunderscore )}
\end{itemize}
Acto ou effeito de preterir.
\section{Preterir}
\begin{itemize}
\item {Grp. gram.:v. t.}
\end{itemize}
\begin{itemize}
\item {Proveniência:(Lat. \textunderscore praeterire\textunderscore )}
\end{itemize}
Ir àlém de.
Ultrapassar.
Deixar atrás.
Passar em claro.
Desprezar.
Omittir: \textunderscore preterir pormenores\textunderscore .
Abstrahir de.
Supplantar; sêr provido indevidamente em lugar que competia a: \textunderscore preterir vários concorrentes\textunderscore .
\section{Pretérito}
\begin{itemize}
\item {Grp. gram.:adj.}
\end{itemize}
\begin{itemize}
\item {Grp. gram.:M.}
\end{itemize}
\begin{itemize}
\item {Utilização:Gram.}
\end{itemize}
\begin{itemize}
\item {Proveniência:(Lat. \textunderscore praeteritus\textunderscore )}
\end{itemize}
Que passou; passado: \textunderscore os séculos pretéritos\textunderscore .
Tempo verbal, que indica acção passada ou anterior.
\section{Preterível}
\begin{itemize}
\item {Grp. gram.:adj.}
\end{itemize}
Que se póde preterir.
\section{Pretermissão}
\begin{itemize}
\item {Grp. gram.:f.}
\end{itemize}
\begin{itemize}
\item {Proveniência:(Lat. \textunderscore praetermissio\textunderscore )}
\end{itemize}
Acto ou effeito de pretermittir.
\section{Pretermitir}
\begin{itemize}
\item {Grp. gram.:v. t.}
\end{itemize}
\begin{itemize}
\item {Proveniência:(Do lat. \textunderscore praeter\textunderscore  + \textunderscore mittere\textunderscore )}
\end{itemize}
O mesmo que \textunderscore preterir\textunderscore .
\section{Pretermittir}
\begin{itemize}
\item {Grp. gram.:v. t.}
\end{itemize}
\begin{itemize}
\item {Proveniência:(Do lat. \textunderscore praeter\textunderscore  + \textunderscore mittere\textunderscore )}
\end{itemize}
O mesmo que \textunderscore preterir\textunderscore .
\section{Preternatural}
\begin{itemize}
\item {Grp. gram.:adj.}
\end{itemize}
\begin{itemize}
\item {Proveniência:(De \textunderscore preter...\textunderscore  + \textunderscore natural\textunderscore )}
\end{itemize}
O mesmo que \textunderscore sobrenatural\textunderscore . Cf. R. Lobo, \textunderscore Côrte na Aldeia\textunderscore , I, 90.
\section{Prevenção}
\begin{itemize}
\item {Grp. gram.:f.}
\end{itemize}
\begin{itemize}
\item {Proveniência:(Lat. \textunderscore praeventio\textunderscore )}
\end{itemize}
Acto ou effeito de prevenir.
Disposição prévia; premeditação.
\textunderscore Estar de prevenção\textunderscore , diz-se das tropas que estão em quartéis, mas promptas e equipadas para qualquer serviço de defesa ou segurança pública.
\section{Prevenidamente}
\begin{itemize}
\item {Grp. gram.:adv.}
\end{itemize}
De modo prevenido; com prevenção; antecipadamente.
\section{Prevenido}
\begin{itemize}
\item {Grp. gram.:adj.}
\end{itemize}
\begin{itemize}
\item {Proveniência:(De \textunderscore prevenir\textunderscore )}
\end{itemize}
Desconfiado; receoso.
\section{Preveniente}
\begin{itemize}
\item {Grp. gram.:adj.}
\end{itemize}
\begin{itemize}
\item {Proveniência:(Lat. \textunderscore praeveniens\textunderscore )}
\end{itemize}
Que chega antes.
Que nos induz á prática do bem, (falando-se da graça divina).
\section{Prevenir}
\begin{itemize}
\item {Grp. gram.:v. t.}
\end{itemize}
\begin{itemize}
\item {Proveniência:(Lat. \textunderscore praevenire\textunderscore )}
\end{itemize}
Antecipar; dispor com antecipação.
Acautelar, precaver.
Chegar antes de.
Acautelar-se contra.
Evitar; impedir que succeda ou se execute.
Premeditar.
Prever.
\section{Preventivamente}
\begin{itemize}
\item {Grp. gram.:adv.}
\end{itemize}
De modo preventivo; com prevenção; previdentemente.
\section{Preventivo}
\begin{itemize}
\item {Grp. gram.:adj.}
\end{itemize}
\begin{itemize}
\item {Proveniência:(Do lat. \textunderscore praeventus\textunderscore )}
\end{itemize}
Que previne; próprio para prevenir; em que há prevenção.
\section{Prevento}
\begin{itemize}
\item {Grp. gram.:adj.}
\end{itemize}
\begin{itemize}
\item {Proveniência:(Lat. \textunderscore praeventus\textunderscore )}
\end{itemize}
Que está prevenido.
Acautelado.
\section{Preventor}
\begin{itemize}
\item {Grp. gram.:adj.}
\end{itemize}
\begin{itemize}
\item {Grp. gram.:M. pl.}
\end{itemize}
\begin{itemize}
\item {Proveniência:(Lat. \textunderscore praeventor\textunderscore )}
\end{itemize}
Aquelle que previne.
Soldados da guarda avançada, entre os antigos.
\section{Prever}
\begin{itemize}
\item {Grp. gram.:v. t.}
\end{itemize}
\begin{itemize}
\item {Proveniência:(Do lat. \textunderscore praevidere\textunderscore )}
\end{itemize}
Vêr antecipadamente.
Presuppor; calcular.
Predizer; prophetizar.
\section{Previamente}
\begin{itemize}
\item {Grp. gram.:adv.}
\end{itemize}
De modo prévio; antecipadamente; com prevenção.
\section{Previço}
\begin{itemize}
\item {Grp. gram.:m.}
\end{itemize}
\begin{itemize}
\item {Utilização:Ant.}
\end{itemize}
(Corr. de \textunderscore previso\textunderscore )
\section{Previdência}
\begin{itemize}
\item {Grp. gram.:f.}
\end{itemize}
\begin{itemize}
\item {Proveniência:(Lat. \textunderscore praevidentia\textunderscore )}
\end{itemize}
Qualidade ou acto do que é previdente.
Precaução.
\section{Previdente}
\begin{itemize}
\item {Grp. gram.:adj.}
\end{itemize}
\begin{itemize}
\item {Proveniência:(Lat. \textunderscore praevidens\textunderscore )}
\end{itemize}
Que prevê; acautelado; prudente.
\section{Previdentemente}
\begin{itemize}
\item {Grp. gram.:adv.}
\end{itemize}
De modo previdente.
\section{Prévio}
\begin{itemize}
\item {Grp. gram.:adj.}
\end{itemize}
\begin{itemize}
\item {Proveniência:(Lat. \textunderscore praevius\textunderscore )}
\end{itemize}
Dito ou feito antes de alguma coisa; antecipado; anterior; preliminar.
\section{Previsão}
\begin{itemize}
\item {Grp. gram.:f.}
\end{itemize}
\begin{itemize}
\item {Proveniência:(De \textunderscore previso\textunderscore )}
\end{itemize}
Acto ou effeito de prever.
Presciência; prevenção.
\section{Previso}
\begin{itemize}
\item {Grp. gram.:m.}
\end{itemize}
\begin{itemize}
\item {Utilização:Ant.}
\end{itemize}
\begin{itemize}
\item {Grp. gram.:Adj.}
\end{itemize}
\begin{itemize}
\item {Proveniência:(Lat. \textunderscore praevisus\textunderscore )}
\end{itemize}
Astrólogo; feiticeiro.
O mesmo que \textunderscore previsto\textunderscore .
\section{Previsor}
\begin{itemize}
\item {Grp. gram.:adj.}
\end{itemize}
\begin{itemize}
\item {Utilização:Bras}
\end{itemize}
\begin{itemize}
\item {Proveniência:(Do lat. \textunderscore praevisus\textunderscore )}
\end{itemize}
Que prevê, que é previdente.
\section{Previstamente}
\begin{itemize}
\item {Grp. gram.:adv.}
\end{itemize}
De modo previsto; com previsão.
\section{Previsto}
\begin{itemize}
\item {Grp. gram.:adj.}
\end{itemize}
\begin{itemize}
\item {Proveniência:(De \textunderscore prever\textunderscore )}
\end{itemize}
Conjecturado; prenunciado.
\section{Previver}
\begin{itemize}
\item {Grp. gram.:v. i.}
\end{itemize}
\begin{itemize}
\item {Utilização:Fig.}
\end{itemize}
\begin{itemize}
\item {Proveniência:(De \textunderscore pre...\textunderscore  + \textunderscore viver\textunderscore )}
\end{itemize}
Sentir existência futura.
Conhecer que se viverá no futuro; saber que os vindoiros o terão presente:«\textunderscore ...a van immortalidade, que só não é demencia nas grandes almas, e nos gênios que se sentem previver nas gerações vindouras.\textunderscore »Camillo, \textunderscore Am. de Perdição\textunderscore , (ed. monum.), 177.
\section{Prez}
\begin{itemize}
\item {Grp. gram.:m.}
\end{itemize}
\begin{itemize}
\item {Utilização:Ant.}
\end{itemize}
\begin{itemize}
\item {Proveniência:(De \textunderscore prezar\textunderscore )}
\end{itemize}
O mesmo que \textunderscore preço\textunderscore . Cf. Latino, \textunderscore Camões\textunderscore , 141.
\section{Prezado}
\begin{itemize}
\item {Grp. gram.:adj.}
\end{itemize}
\begin{itemize}
\item {Proveniência:(De \textunderscore prezar\textunderscore )}
\end{itemize}
Estimado; querido.
\section{Prezador}
\begin{itemize}
\item {Grp. gram.:m.  e  adj.}
\end{itemize}
O que preza.
\section{Prezar}
\begin{itemize}
\item {Grp. gram.:v. t.}
\end{itemize}
\begin{itemize}
\item {Proveniência:(Lat. \textunderscore pretiare\textunderscore )}
\end{itemize}
Têr em grande preço, estimar muito.
Têr em grande consideração ou respeito.
Desejar.
Amar: \textunderscore prezar a virtude\textunderscore .
\section{Prezável}
\begin{itemize}
\item {Grp. gram.:adj.}
\end{itemize}
\begin{itemize}
\item {Proveniência:(De \textunderscore prezar\textunderscore )}
\end{itemize}
Digno de sêr prezado.
\section{Priangu}
\begin{itemize}
\item {Grp. gram.:m.}
\end{itemize}
\begin{itemize}
\item {Utilização:Bras. do N}
\end{itemize}
Ave agoireira.
\section{Priapisma}
\begin{itemize}
\item {Grp. gram.:m.}
\end{itemize}
\begin{itemize}
\item {Utilização:Ant.}
\end{itemize}
O mesmo que \textunderscore priapismo\textunderscore . Cf. B. Pereira, \textunderscore Prosódia\textunderscore , vb. \textunderscore tentigo\textunderscore .
\section{Priapismo}
\begin{itemize}
\item {Grp. gram.:m.}
\end{itemize}
\begin{itemize}
\item {Proveniência:(Lat. \textunderscore priapismus\textunderscore )}
\end{itemize}
Excessivo orgasmo venéreo.
\section{Priapo}
\begin{itemize}
\item {Grp. gram.:m.}
\end{itemize}
\begin{itemize}
\item {Utilização:P. us.}
\end{itemize}
\begin{itemize}
\item {Proveniência:(Lat. \textunderscore Priapus\textunderscore , n. p.)}
\end{itemize}
O mesmo que \textunderscore phallo\textunderscore .
\section{Primeiranista}
\begin{itemize}
\item {Grp. gram.:m.}
\end{itemize}
\begin{itemize}
\item {Proveniência:(De \textunderscore primeiro\textunderscore  + \textunderscore anno\textunderscore )}
\end{itemize}
Estudante, que cursa o primeiro anno de qualquer Faculdade das universidades ou de outra escola superior.
\section{Primeirannista}
\begin{itemize}
\item {Grp. gram.:m.}
\end{itemize}
\begin{itemize}
\item {Proveniência:(De \textunderscore primeiro\textunderscore  + \textunderscore anno\textunderscore )}
\end{itemize}
Estudante, que cursa o primeiro anno de qualquer Faculdade das universidades ou de outra escola superior.
\section{Primeiríssimo}
\begin{itemize}
\item {Grp. gram.:adj.}
\end{itemize}
\begin{itemize}
\item {Utilização:Fam.}
\end{itemize}
Evidentemente primeiro. Cf. Rui Barb., \textunderscore Réplica\textunderscore , 158.
\section{Primeiro}
\begin{itemize}
\item {Grp. gram.:adj.}
\end{itemize}
\begin{itemize}
\item {Grp. gram.:M.}
\end{itemize}
\begin{itemize}
\item {Grp. gram.:Adv.}
\end{itemize}
\begin{itemize}
\item {Grp. gram.:Loc. adv.}
\end{itemize}
\begin{itemize}
\item {Proveniência:(De \textunderscore primário\textunderscore  &gt; \textunderscore primairo\textunderscore  &gt; \textunderscore primeiro\textunderscore )}
\end{itemize}
Que precede outros, relativamente ao tempo, ao lugar, á categoria.
Que é o mais antigo numa série ou numa classe: \textunderscore no primeiro século da nossa era\textunderscore .
Primário, primitivo.
Primogênito.
O mais importante ou notável: \textunderscore Herculano, o nosso primeiro historiador\textunderscore .
Que vai adeante de todos.
Fundamental; necessário: \textunderscore a hygiene é a primeira condição da saúde\textunderscore .
Elementar, rudimentar: \textunderscore estudar primeiras letras\textunderscore .
O que numa classe ou série está em primeiro lugar.
Antes, primeiramente: \textunderscore primeiro, conversemos\textunderscore .
\textunderscore De primeiro\textunderscore , no princípio, desde logo, primeiramente:«\textunderscore ...que de primeiro afugentárão Pulatecão...\textunderscore »Filinto, D. Man., II, 257.
\section{Primente}
\begin{itemize}
\item {Grp. gram.:adv.}
\end{itemize}
\begin{itemize}
\item {Utilização:Ant.}
\end{itemize}
\begin{itemize}
\item {Proveniência:(De \textunderscore primo\textunderscore ^2)}
\end{itemize}
O mesmo que \textunderscore primeiramente\textunderscore . Cf. Frei Fortun., \textunderscore Inéd.\textunderscore , 312.
\section{Primero}
\textunderscore adj.\textunderscore  (e der.) \textunderscore Ant.\textunderscore 
O mesmo que \textunderscore primeiro\textunderscore , etc. Cf. \textunderscore Eufrosina\textunderscore , 183.
\section{Primevo}
\begin{itemize}
\item {Grp. gram.:adj.}
\end{itemize}
\begin{itemize}
\item {Proveniência:(Lat. \textunderscore primaevus\textunderscore )}
\end{itemize}
Relativo aos tempos primitivos.
\section{Primi...}
\begin{itemize}
\item {Grp. gram.:pref.}
\end{itemize}
\begin{itemize}
\item {Proveniência:(Do lat. \textunderscore primus\textunderscore )}
\end{itemize}
(designativo de prioridade)
\section{Primicério}
\begin{itemize}
\item {Grp. gram.:m.}
\end{itemize}
\begin{itemize}
\item {Utilização:Ant.}
\end{itemize}
\begin{itemize}
\item {Proveniência:(Lat. \textunderscore primicerius\textunderscore )}
\end{itemize}
O mesmo que \textunderscore chantre\textunderscore .
\section{Primichica}
\begin{itemize}
\item {Grp. gram.:adj. f.}
\end{itemize}
Diz-se da fêmea dos mammíferos, que pare pela primeira vez. Cf. Baganha, \textunderscore Hyg. Pec.\textunderscore , 89.
\section{Primícias}
\begin{itemize}
\item {Grp. gram.:f. pl.}
\end{itemize}
\begin{itemize}
\item {Proveniência:(Lat. \textunderscore primitiae\textunderscore )}
\end{itemize}
Primeiros frutos, primeiras producções.
Primeiros effeitos; primeiros sentimentos; começos: \textunderscore primícias literárias\textunderscore .
\section{Primifalange}
\begin{itemize}
\item {Grp. gram.:f.}
\end{itemize}
\begin{itemize}
\item {Utilização:Anat.}
\end{itemize}
\begin{itemize}
\item {Proveniência:(De \textunderscore primi...\textunderscore  + \textunderscore phalange\textunderscore )}
\end{itemize}
Osso da raiz do dêdo grande do pé; primeira falange.
\section{Primifalangeta}
\begin{itemize}
\item {fónica:gê}
\end{itemize}
\begin{itemize}
\item {Grp. gram.:f.}
\end{itemize}
\begin{itemize}
\item {Utilização:Anat.}
\end{itemize}
Primeira falangeta.
\section{Primigênio}
\begin{itemize}
\item {Grp. gram.:adj.}
\end{itemize}
\begin{itemize}
\item {Proveniência:(Lat. \textunderscore primigenius\textunderscore )}
\end{itemize}
Primitivo; primordial; o primeiro da sua espécie.
\section{Premígeno}
\begin{itemize}
\item {Grp. gram.:adj.}
\end{itemize}
\begin{itemize}
\item {Proveniência:(Lat. \textunderscore primigenus\textunderscore )}
\end{itemize}
O mesmo que \textunderscore primigênio\textunderscore .
\section{Primimetatársico}
\begin{itemize}
\item {Grp. gram.:m.}
\end{itemize}
\begin{itemize}
\item {Utilização:Anat.}
\end{itemize}
\begin{itemize}
\item {Proveniência:(De \textunderscore primi...\textunderscore  + \textunderscore metatársico\textunderscore )}
\end{itemize}
Diz-se do primeiro dos ossos metatársicos, isto é, do osso correspondente ao dedo grande do pé.
\section{Primina}
\begin{itemize}
\item {Grp. gram.:f.}
\end{itemize}
\begin{itemize}
\item {Utilização:Bot.}
\end{itemize}
\begin{itemize}
\item {Proveniência:(Do lat. \textunderscore primus\textunderscore )}
\end{itemize}
A primeira ou mais externa das membranas, que revestem a núcula do ovário vegetal.
\section{Primípara}
\begin{itemize}
\item {Grp. gram.:adj.}
\end{itemize}
\begin{itemize}
\item {Proveniência:(Do lat. \textunderscore primus\textunderscore  + \textunderscore parere\textunderscore )}
\end{itemize}
Diz-se das fêmeas que têm o primeiro parto.
\section{Primiphalange}
\begin{itemize}
\item {Grp. gram.:f.}
\end{itemize}
\begin{itemize}
\item {Utilização:Anat.}
\end{itemize}
\begin{itemize}
\item {Proveniência:(De \textunderscore primi...\textunderscore  + \textunderscore phalange\textunderscore )}
\end{itemize}
Osso da raiz do dêdo grande do pé; primeira phalange.
\section{Primiphalangeta}
\begin{itemize}
\item {fónica:gê}
\end{itemize}
\begin{itemize}
\item {Grp. gram.:f.}
\end{itemize}
\begin{itemize}
\item {Utilização:Anat.}
\end{itemize}
Primeira phalangeta.
\section{Primipilar}
\begin{itemize}
\item {Grp. gram.:m.}
\end{itemize}
\begin{itemize}
\item {Proveniência:(Lat. \textunderscore primipilum\textunderscore )}
\end{itemize}
O primeiro centurião, em cada cohorte romana.
\section{Primipilo}
\begin{itemize}
\item {Grp. gram.:m.}
\end{itemize}
\begin{itemize}
\item {Proveniência:(Lat. \textunderscore primipilum\textunderscore )}
\end{itemize}
O primeiro centurião, em cada cohorte romana.
\section{Primitiva}
\begin{itemize}
\item {Grp. gram.:f.}
\end{itemize}
\begin{itemize}
\item {Utilização:Fam.}
\end{itemize}
Os primeiros tempos; os tempos primitivos; princípio.
(Fem. de \textunderscore primitivo\textunderscore )
\section{Primitivamente}
\begin{itemize}
\item {Grp. gram.:adv.}
\end{itemize}
\begin{itemize}
\item {Proveniência:(De \textunderscore primitivo\textunderscore )}
\end{itemize}
Nos tempos primitivos; na origem.
\section{Primitivo}
\begin{itemize}
\item {Grp. gram.:adj.}
\end{itemize}
\begin{itemize}
\item {Utilização:Ext.}
\end{itemize}
\begin{itemize}
\item {Proveniência:(Lat. \textunderscore primitivus\textunderscore )}
\end{itemize}
Que appareceu em primeiro lugar.
Relativo aos primeiros tempos.
Primevo.
Primordial.
Que precedeu; que foi dos primeiros a apparecer: \textunderscore os Christãos primitivos\textunderscore .
Antiquado, rudimentar, rude.
\section{Primo}
\begin{itemize}
\item {Grp. gram.:m.}
\end{itemize}
\begin{itemize}
\item {Proveniência:(Lat. \textunderscore primus\textunderscore )}
\end{itemize}
Diz-se do indivíduo, em relação aos filhos de seus tios.
Designação vulgar de todos os parentes, que não têm designação especial.
\section{Principelho}
\begin{itemize}
\item {fónica:pê}
\end{itemize}
\begin{itemize}
\item {Grp. gram.:m.}
\end{itemize}
\begin{itemize}
\item {Utilização:Deprec.}
\end{itemize}
Pequeno Príncipe.
Príncipe ridículo ou de pouco mérito. Cf. Th. Braga, \textunderscore Camões\textunderscore , 312.
\section{Principescamente}
\begin{itemize}
\item {Grp. gram.:adv.}
\end{itemize}
De modo principesco.
Á maneira de Príncipe; opulentamente.
\section{Principesco}
\begin{itemize}
\item {fónica:pês}
\end{itemize}
\begin{itemize}
\item {Grp. gram.:adj.}
\end{itemize}
Relativo a Príncipes.
Próprio de Príncipes; ostentoso, opulento.
\section{Principiador}
\begin{itemize}
\item {Grp. gram.:m.  e  adj.}
\end{itemize}
O que principia ou dá comêço a alguma coisa.
\section{Principiante}
\begin{itemize}
\item {Grp. gram.:adj.}
\end{itemize}
\begin{itemize}
\item {Grp. gram.:M.  e  f.}
\end{itemize}
\begin{itemize}
\item {Proveniência:(Lat. \textunderscore principians\textunderscore )}
\end{itemize}
Que principia; que está no princípio.
Pessôa, que começa a exercitar-se ou aprender alguma coisa; praticante.
\section{Principiar}
\begin{itemize}
\item {Grp. gram.:v. t.}
\end{itemize}
\begin{itemize}
\item {Proveniência:(Lat. \textunderscore principiare\textunderscore )}
\end{itemize}
Dar princípio a; começar, iniciar.
\section{Principículo}
\begin{itemize}
\item {Grp. gram.:m.}
\end{itemize}
O mesmo que \textunderscore principelho\textunderscore . Cf. Ol. Martins, \textunderscore Camões\textunderscore , 87.
\section{Princípio}
\begin{itemize}
\item {Grp. gram.:m.}
\end{itemize}
\begin{itemize}
\item {Grp. gram.:Pl.}
\end{itemize}
\begin{itemize}
\item {Proveniência:(Lat. \textunderscore principium\textunderscore )}
\end{itemize}
Momento, em que alguma coisa tem origem.
Origem.
Causa primaria.
Comêço.
Elemento, que predomina na constituição de um corpo orgânico.
Theoria, regra: \textunderscore os princípios da Philosophia\textunderscore .
Preceito moral.
Preceito.
Estreia.
Germe.
Preliminar.
Primícias; rudimento.
Antecedentes.
Primeira época da vida.
\section{Principote}
\begin{itemize}
\item {Grp. gram.:m.}
\end{itemize}
\begin{itemize}
\item {Utilização:Deprec.}
\end{itemize}
O mesmo que \textunderscore principelho\textunderscore .
Soberano de um pequeno Principado:«\textunderscore ...quer cada principote embaixadores\textunderscore ». Filinto, XII, 10.
\section{Prino}
\begin{itemize}
\item {Grp. gram.:m.}
\end{itemize}
\begin{itemize}
\item {Proveniência:(Lat. \textunderscore prinus\textunderscore )}
\end{itemize}
Gênero de plantas americanas.
\section{Príntzia}
\begin{itemize}
\item {Grp. gram.:f.}
\end{itemize}
\begin{itemize}
\item {Proveniência:(De \textunderscore Printz\textunderscore , n. p.)}
\end{itemize}
Gênero de plantas, da fam. das compostas.
\section{Prióbio}
\begin{itemize}
\item {Grp. gram.:m.}
\end{itemize}
\begin{itemize}
\item {Proveniência:(Do gr. \textunderscore prion\textunderscore  + \textunderscore bios\textunderscore )}
\end{itemize}
Gênero de insectos coleópteros pentâmeros.
\section{Priócera}
\begin{itemize}
\item {Grp. gram.:f.}
\end{itemize}
\begin{itemize}
\item {Proveniência:(Do gr. \textunderscore prion\textunderscore  + \textunderscore keras\textunderscore )}
\end{itemize}
Gênero de insectos coleópteros pentâmeros.
\section{Priol}
\begin{itemize}
\item {fónica:ôl}
\end{itemize}
\begin{itemize}
\item {Grp. gram.:m.}
\end{itemize}
\begin{itemize}
\item {Utilização:Ant.}
\end{itemize}
O mesmo que \textunderscore prior\textunderscore . Cf. R. Pina, \textunderscore Affonso V\textunderscore , c. LIX.
\section{Priólomo}
\begin{itemize}
\item {Grp. gram.:m.}
\end{itemize}
Gênero de peixes acanthopterýgios.
\section{Prionáptero}
\begin{itemize}
\item {Grp. gram.:m.}
\end{itemize}
Gênero de insectos coleópteros longicórneos.
\section{Priónia}
\begin{itemize}
\item {Grp. gram.:f.}
\end{itemize}
\begin{itemize}
\item {Proveniência:(Do gr. \textunderscore prion\textunderscore )}
\end{itemize}
Gênero de juncos, do Cabo da Bôa-Esperança.
\section{Priónio}
\begin{itemize}
\item {Grp. gram.:m.}
\end{itemize}
O mesmo que \textunderscore priónia\textunderscore .
\section{Príono}
\begin{itemize}
\item {Grp. gram.:m.}
\end{itemize}
\begin{itemize}
\item {Proveniência:(Do gr. \textunderscore prion\textunderscore )}
\end{itemize}
Gênero de insectos coleópteros tetrâmeros.
\section{Prionoderme}
\begin{itemize}
\item {Grp. gram.:m.}
\end{itemize}
\begin{itemize}
\item {Proveniência:(Do gr. \textunderscore prion\textunderscore  + \textunderscore derma\textunderscore )}
\end{itemize}
Gênero de vermes intestinaes.
\section{Prionúro}
\begin{itemize}
\item {Grp. gram.:m.}
\end{itemize}
\begin{itemize}
\item {Proveniência:(Do gr. \textunderscore prion\textunderscore  + \textunderscore oura\textunderscore )}
\end{itemize}
Gênero de peixes acanthopterýgios.
\section{Prior}
\begin{itemize}
\item {Grp. gram.:m.}
\end{itemize}
\begin{itemize}
\item {Utilização:Ant.}
\end{itemize}
\begin{itemize}
\item {Proveniência:(Lat. \textunderscore prior\textunderscore )}
\end{itemize}
Designação do párocho de certas freguesias.
Dignitário de Ordem militar.
Superior de convento.
\section{Priora}
\begin{itemize}
\item {Grp. gram.:f.}
\end{itemize}
O mesmo que \textunderscore prioresa\textunderscore .
(Cast. \textunderscore priora\textunderscore )
\section{Priorado}
\begin{itemize}
\item {Grp. gram.:m.}
\end{itemize}
\begin{itemize}
\item {Proveniência:(Do lat. \textunderscore prioratus\textunderscore )}
\end{itemize}
Cargo de prior ou de prioresa.
Tempo, durante o qual um prior ou uma prioresa exerce as suas funcções.
\section{Prioral}
\begin{itemize}
\item {Grp. gram.:adj.}
\end{itemize}
Relativo a prior ou priorado.
\section{Priorato}
\begin{itemize}
\item {Grp. gram.:m.}
\end{itemize}
\begin{itemize}
\item {Proveniência:(Lat. \textunderscore prioratus\textunderscore )}
\end{itemize}
O mesmo que \textunderscore priorado\textunderscore .
\section{Prioresa}
\begin{itemize}
\item {Grp. gram.:f.}
\end{itemize}
\begin{itemize}
\item {Proveniência:(Do b. lat. \textunderscore priorissa\textunderscore )}
\end{itemize}
Superiora de um convento de certas Ordens religiosas; abadessa.
\section{Prioridade}
\begin{itemize}
\item {Grp. gram.:f.}
\end{itemize}
\begin{itemize}
\item {Proveniência:(Do lat. \textunderscore prior\textunderscore )}
\end{itemize}
Qualidade do que está em primeiro lugar, ou do que appareceu primeiro; primazia.
\section{Prioris}
\begin{itemize}
\item {Grp. gram.:f.}
\end{itemize}
\begin{itemize}
\item {Utilização:Ant.}
\end{itemize}
O mesmo que \textunderscore pleuris\textunderscore . Cf. G. Resende, \textunderscore Cancioneiro\textunderscore .
\section{Prioste}
\begin{itemize}
\item {Grp. gram.:m.}
\end{itemize}
Antigo cobrador de rendimentos ecclesiásticos. Cf. \textunderscore Eufrosina\textunderscore , 331.
(Alter. de \textunderscore preboste\textunderscore )
\section{Prítico}
\begin{itemize}
\item {Grp. gram.:m.}
\end{itemize}
\begin{itemize}
\item {Utilização:Prov.}
\end{itemize}
\begin{itemize}
\item {Utilização:alg.}
\end{itemize}
O mesmo que \textunderscore pírtigo\textunderscore .
\section{Prítiga}
\begin{itemize}
\item {Grp. gram.:f.}
\end{itemize}
\begin{itemize}
\item {Utilização:Prov.}
\end{itemize}
O mesmo que \textunderscore prítica\textunderscore .
\section{Priul}
\begin{itemize}
\item {Grp. gram.:m.}
\end{itemize}
\begin{itemize}
\item {Utilização:Ant.}
\end{itemize}
O mesmo que \textunderscore prior\textunderscore .
\section{Priva}
\begin{itemize}
\item {Grp. gram.:f.}
\end{itemize}
Designação scientífica da verbena cheirosa.
\section{Privação}
\begin{itemize}
\item {Grp. gram.:f.}
\end{itemize}
\begin{itemize}
\item {Grp. gram.:Pl.}
\end{itemize}
\begin{itemize}
\item {Proveniência:(Lat. \textunderscore privatio\textunderscore )}
\end{itemize}
Acto ou effeito de privar.
Falta do que é necessário á vida: \textunderscore soffrer privações\textunderscore .
\section{Privada}
\begin{itemize}
\item {Grp. gram.:f.}
\end{itemize}
\begin{itemize}
\item {Proveniência:(De \textunderscore privado\textunderscore )}
\end{itemize}
O mesmo que \textunderscore latrina\textunderscore .
\section{Privadamente}
\begin{itemize}
\item {Grp. gram.:adv.}
\end{itemize}
De modo privado; particularmente.
\section{Privado}
\begin{itemize}
\item {Grp. gram.:adj.}
\end{itemize}
\begin{itemize}
\item {Grp. gram.:M.}
\end{itemize}
Que não é público; particular; íntimo: \textunderscore recólheu-se á vida privada\textunderscore .
Favorito; confidente; áulico.
Entre os Godos, cada um dos indivíduos, pelos quaes se repartiam terras conquistadas. Cf. Herculano, \textunderscore Hist. de Port.\textunderscore , III, 318.
\section{Privança}
\begin{itemize}
\item {Grp. gram.:f.}
\end{itemize}
\begin{itemize}
\item {Proveniência:(De \textunderscore privar\textunderscore )}
\end{itemize}
Estado de quem é favorito ou valído.
Intimidade: \textunderscore viver na privança de alguém\textunderscore .
\section{Privar}
\begin{itemize}
\item {Grp. gram.:v. t.}
\end{itemize}
\begin{itemize}
\item {Grp. gram.:V. i.}
\end{itemize}
\begin{itemize}
\item {Proveniência:(Lat. \textunderscore privare\textunderscore )}
\end{itemize}
Desapossar; tirar alguma coisa a.
Impedir de possuír alguma coisa.
Tirar a propriedade de alguma coisa a.
Conviver intimamente.
Têr valimento; tratar de perto.
\section{Privativamente}
\begin{itemize}
\item {Grp. gram.:adv.}
\end{itemize}
De modo privativo; particularmente.
\section{Privativo}
\begin{itemize}
\item {Grp. gram.:adj.}
\end{itemize}
\begin{itemize}
\item {Proveniência:(Lat. \textunderscore privativus\textunderscore )}
\end{itemize}
Que exprime privação.
Peculiar, próprio, particular; restricto.
\section{Privilegiado}
\begin{itemize}
\item {Grp. gram.:adj.}
\end{itemize}
\begin{itemize}
\item {Proveniência:(De \textunderscore privilegiar\textunderscore )}
\end{itemize}
Singular; distinto.
\section{Privilegiar}
\begin{itemize}
\item {Grp. gram.:v. t.}
\end{itemize}
Dar privilégio a.
Tratar com distincção; especializar.
Conceder exclusivamente alguma coisa a.
\section{Privilégio}
\begin{itemize}
\item {Grp. gram.:m.}
\end{itemize}
\begin{itemize}
\item {Proveniência:(Lat. \textunderscore privilegium\textunderscore )}
\end{itemize}
Vantagem concedida a alguém, com exclusão de outros e contra o direito commum: \textunderscore o privilégio da fabricação dos phósphoros\textunderscore .
Título, com que se consegue essa vantagem.
Permissão especial, ou só relativa a certas pessôas ou coisas.
Qualidade ou faculdade característica.
Prerogativa.
Immunidade.
Condão.
\section{Pró}
\begin{itemize}
\item {Grp. gram.:adv.}
\end{itemize}
\begin{itemize}
\item {Grp. gram.:M.}
\end{itemize}
\begin{itemize}
\item {Proveniência:(Do lat. \textunderscore pro\textunderscore )}
\end{itemize}
A favor.
Vantagem, conveniência: \textunderscore calcular prós e contras\textunderscore .
\section{Pro...}
\begin{itemize}
\item {Grp. gram.:pref.}
\end{itemize}
\begin{itemize}
\item {Proveniência:(Lat. \textunderscore pro\textunderscore )}
\end{itemize}
(design. de \textunderscore origem\textunderscore , \textunderscore anterioridade\textunderscore , \textunderscore extensão\textunderscore , \textunderscore substituição\textunderscore )
\section{Prôa}
\begin{itemize}
\item {Grp. gram.:f.}
\end{itemize}
\begin{itemize}
\item {Utilização:Fam.}
\end{itemize}
\begin{itemize}
\item {Grp. gram.:M.}
\end{itemize}
\begin{itemize}
\item {Utilização:T. da Nazaré}
\end{itemize}
\begin{itemize}
\item {Utilização:T. de Aveiro}
\end{itemize}
\begin{itemize}
\item {Proveniência:(Do lat. \textunderscore prora\textunderscore )}
\end{itemize}
Parte anterior do navio, em opposição á popa.
Frente de qualquer coisa: \textunderscore não me provoques, que me encontras pela prôa\textunderscore .
Presumpção, vaidade, bazófia: \textunderscore aquella rapariga tem muita prôa\textunderscore .
Um dos pescadores que trabalham no levantamento das rêdes.
Remeiro, que, na guiga, vai á frente dos outros.
Um dos dois remos de barco de pesca, o que fica do lado da ré, com a pá para estibordo.
\section{Proágoro}
\begin{itemize}
\item {Grp. gram.:m.}
\end{itemize}
\begin{itemize}
\item {Proveniência:(Lat. \textunderscore proagorus\textunderscore )}
\end{itemize}
Aquelle que primeiro usava da palavra, nas antigas assembleias.
Magistrado superior de algumas antigas cidades da Sicília.
\section{Proar}
\begin{itemize}
\item {Grp. gram.:v. i.}
\end{itemize}
O mesmo que \textunderscore aproar\textunderscore .
\section{Probabilidade}
\begin{itemize}
\item {Grp. gram.:f.}
\end{itemize}
\begin{itemize}
\item {Proveniência:(Lat. \textunderscore probabilitas\textunderscore )}
\end{itemize}
Qualidade do que é provável.
Motivo ou indício, que deixa presumir a verdade ou a possibilidade de um facto; verosimilhança.
\section{Probabiliorismo}
\begin{itemize}
\item {Grp. gram.:m.}
\end{itemize}
\begin{itemize}
\item {Proveniência:(Do lat. \textunderscore probabilior\textunderscore )}
\end{itemize}
Systema theológico dos que, como Concina, combatiam o probabilismo.
\section{Probabilismo}
\begin{itemize}
\item {Grp. gram.:m.}
\end{itemize}
\begin{itemize}
\item {Proveniência:(Do lat. \textunderscore probabilis\textunderscore )}
\end{itemize}
Doutrina, segundo a qual se póde seguir uma opinião provável, ainda que haja outras de mais probabilidade.
\section{Procela}
\begin{itemize}
\item {Grp. gram.:f.}
\end{itemize}
\begin{itemize}
\item {Utilização:Fig.}
\end{itemize}
\begin{itemize}
\item {Proveniência:(Lat. \textunderscore procella\textunderscore )}
\end{itemize}
Tempestade marítima; tempestade.
Agitação extraordinária.
\section{Procelária}
\begin{itemize}
\item {Grp. gram.:f.}
\end{itemize}
\begin{itemize}
\item {Proveniência:(De \textunderscore procela\textunderscore )}
\end{itemize}
Gênero de aves palmípedes, que, aparecendo em bandos sôbre as ondas, anunciam tempestade.
\section{Procella}
\begin{itemize}
\item {Grp. gram.:f.}
\end{itemize}
\begin{itemize}
\item {Utilização:Fig.}
\end{itemize}
\begin{itemize}
\item {Proveniência:(Lat. \textunderscore procella\textunderscore )}
\end{itemize}
Tempestade marítima; tempestade.
Agitação extraordinária.
\section{Procellária}
\begin{itemize}
\item {Grp. gram.:f.}
\end{itemize}
\begin{itemize}
\item {Proveniência:(De \textunderscore procella\textunderscore )}
\end{itemize}
Gênero de aves palmípedes, que, apparecendo em bandos sôbre as ondas, annunciam tempestade.
\section{Procelloso}
\begin{itemize}
\item {Grp. gram.:adj.}
\end{itemize}
\begin{itemize}
\item {Proveniência:(Lat. \textunderscore procellosus\textunderscore )}
\end{itemize}
Relativo á procella; tempestuoso; que traz procella.
\section{Proceloso}
\begin{itemize}
\item {Grp. gram.:adj.}
\end{itemize}
\begin{itemize}
\item {Proveniência:(Lat. \textunderscore procellosus\textunderscore )}
\end{itemize}
Relativo á procela; tempestuoso; que traz procela.
\section{Prócer}
\begin{itemize}
\item {Proveniência:(Lat. \textunderscore procer\textunderscore )}
\end{itemize}
\textunderscore m.\textunderscore  (p. us. no sing.)
Magnate. Cf. Latino, \textunderscore Humboldt\textunderscore , 45.
\section{Próceres}
\begin{itemize}
\item {Grp. gram.:m. pl.}
\end{itemize}
\begin{itemize}
\item {Utilização:Restrict.}
\end{itemize}
\begin{itemize}
\item {Proveniência:(Lat. \textunderscore proceres\textunderscore )}
\end{itemize}
Os homens mais importantes de uma classe ou de uma nação; magnates.
Pares do reino.
\section{Proceridade}
\begin{itemize}
\item {Grp. gram.:f.}
\end{itemize}
\begin{itemize}
\item {Utilização:Anat.}
\end{itemize}
\begin{itemize}
\item {Proveniência:(Lat. \textunderscore proceritas\textunderscore )}
\end{itemize}
Qualidade do que é alto ou corpulento; elevado.
\section{Procero}
\begin{itemize}
\item {Grp. gram.:adj.}
\end{itemize}
\begin{itemize}
\item {Utilização:Des.}
\end{itemize}
\begin{itemize}
\item {Proveniência:(Lat. \textunderscore procerus\textunderscore )}
\end{itemize}
Alto, elevado.
Importante:«\textunderscore nascem e sepultam-se as nações, por mais proceras e giganteias que sejam...\textunderscore »Latino, \textunderscore Camões\textunderscore , 5.
\section{Processamento}
\begin{itemize}
\item {Grp. gram.:m.}
\end{itemize}
Acto de processar.
\section{Processão}
\begin{itemize}
\item {Grp. gram.:f.}
\end{itemize}
\begin{itemize}
\item {Proveniência:(Lat. \textunderscore processio\textunderscore )}
\end{itemize}
O mesmo que \textunderscore procedência\textunderscore .
\section{Processar}
\begin{itemize}
\item {Grp. gram.:v. t.}
\end{itemize}
Instaurar processo contra; autuar.
Conferir, verificar.
\section{Processável}
\begin{itemize}
\item {Grp. gram.:adj.}
\end{itemize}
\begin{itemize}
\item {Proveniência:(De \textunderscore processar\textunderscore )}
\end{itemize}
Que póde sêr processado criminalmente:«\textunderscore todas as autoridades são processáveis, excepto um pai\textunderscore ». Castilho, \textunderscore Escav. Poét.\textunderscore , 52.
\section{Processional}
\begin{itemize}
\item {Grp. gram.:adj.}
\end{itemize}
\begin{itemize}
\item {Proveniência:(Do lat. \textunderscore processio\textunderscore )}
\end{itemize}
Relativo a procissão.
\section{Processionalmente}
\begin{itemize}
\item {Grp. gram.:adv.}
\end{itemize}
De modo processional; á maneira de procissão.
\section{Processionário}
\begin{itemize}
\item {Grp. gram.:m.}
\end{itemize}
\begin{itemize}
\item {Proveniência:(Do lat. \textunderscore processio\textunderscore )}
\end{itemize}
Livro de rezas, que se usa nas procissões.
\section{Processo}
\begin{itemize}
\item {Grp. gram.:m.}
\end{itemize}
\begin{itemize}
\item {Proveniência:(Lat. \textunderscore processus\textunderscore )}
\end{itemize}
Acto de proceder ou de andar.
Seguimento, procedimento.
Maneira de operar.
Actos, que determinam uma operação.
Méthodo.
Desenvolvimento das phases normaes ou mórbidas dos phenómenos orgânicos.
Demanda judicial; autos cíveis ou criminaes.
Caderno, que contém os documentos relativos a um negócio.
\section{Processologia}
\begin{itemize}
\item {Grp. gram.:f.}
\end{itemize}
\begin{itemize}
\item {Proveniência:(Do lat. \textunderscore processus\textunderscore  + gr. \textunderscore logos\textunderscore )}
\end{itemize}
Estudo ou conhecimento dos processos, applicaveis a uma arte ou a uma sciência.
\section{Processológico}
\begin{itemize}
\item {Grp. gram.:adj.}
\end{itemize}
Relativo á processologia.
\section{Processual}
\begin{itemize}
\item {Grp. gram.:adj.}
\end{itemize}
\begin{itemize}
\item {Utilização:Neol.}
\end{itemize}
Relativo a processo judicial.
\section{Processualística}
\begin{itemize}
\item {Grp. gram.:f.}
\end{itemize}
\begin{itemize}
\item {Utilização:Bras}
\end{itemize}
\begin{itemize}
\item {Utilização:Neol.}
\end{itemize}
Theoria do processo judicial. Cf. Tobias Barreto, \textunderscore Estudos de Direito\textunderscore .
\section{Prochlorithe}
\begin{itemize}
\item {Grp. gram.:f.}
\end{itemize}
\begin{itemize}
\item {Utilização:Miner.}
\end{itemize}
\begin{itemize}
\item {Proveniência:(De \textunderscore pro...\textunderscore  + \textunderscore chlorithe\textunderscore )}
\end{itemize}
Uma das três espécies principaes da chlorithe.
\section{Prochronismo}
\begin{itemize}
\item {Grp. gram.:m.}
\end{itemize}
\begin{itemize}
\item {Proveniência:(Do gr. \textunderscore pros\textunderscore  + \textunderscore khronos\textunderscore )}
\end{itemize}
Êrro de data, que consiste em collocar um acontecimento em tempo anterior àquele em que realmente se deu.
\section{Procidência}
\begin{itemize}
\item {Grp. gram.:f.}
\end{itemize}
\begin{itemize}
\item {Utilização:Med.}
\end{itemize}
\begin{itemize}
\item {Proveniência:(Lat. \textunderscore procidentia\textunderscore )}
\end{itemize}
Deslocamento ou quéda de uma parte molle do corpo.
\section{Procissão}
\begin{itemize}
\item {Grp. gram.:f.}
\end{itemize}
\begin{itemize}
\item {Proveniência:(Do lat. \textunderscore processio\textunderscore )}
\end{itemize}
Acompanhamento ou cortejo religioso.
Prestito, cortejo.
Série de indivíduos, que vão marchando uns atrás dos outros.
\section{Procissoeiro}
\begin{itemize}
\item {Grp. gram.:m.}
\end{itemize}
\begin{itemize}
\item {Utilização:Ant.}
\end{itemize}
\begin{itemize}
\item {Proveniência:(De \textunderscore procissão\textunderscore )}
\end{itemize}
Livro, que contém o que se diz ou se canta nas procissões; processionário.
\section{Proclorite}
\begin{itemize}
\item {Grp. gram.:f.}
\end{itemize}
\begin{itemize}
\item {Utilização:Miner.}
\end{itemize}
\begin{itemize}
\item {Proveniência:(De \textunderscore pro...\textunderscore  + \textunderscore clorite\textunderscore )}
\end{itemize}
Uma das três espécies principaes da clorite.
\section{Procronismo}
\begin{itemize}
\item {Grp. gram.:m.}
\end{itemize}
\begin{itemize}
\item {Proveniência:(Do gr. \textunderscore pros\textunderscore  + \textunderscore khronos\textunderscore )}
\end{itemize}
Êrro de data, que consiste em collocar um acontecimento em tempo anterior àquele em que realmente se deu.
\section{Proctoscopia}
\begin{itemize}
\item {Grp. gram.:f.}
\end{itemize}
\begin{itemize}
\item {Utilização:Med.}
\end{itemize}
\begin{itemize}
\item {Proveniência:(Do gr. \textunderscore proktos\textunderscore  + \textunderscore skopein\textunderscore )}
\end{itemize}
Observação ou exame do recto.
\section{Proctótomo}
\begin{itemize}
\item {Grp. gram.:m.}
\end{itemize}
\begin{itemize}
\item {Proveniência:(Do gr. \textunderscore proktos\textunderscore  + \textunderscore tome\textunderscore )}
\end{itemize}
Instrumento, para incisão do recto que se estreitou.
\section{Proculeiano}
\begin{itemize}
\item {Grp. gram.:adj.}
\end{itemize}
\begin{itemize}
\item {Proveniência:(Lat. \textunderscore proculeianus\textunderscore )}
\end{itemize}
Dizia-se, em Roma, dos sequazes de um célebre jurisconsulto. Cf. C. Lobo, \textunderscore Sát. de Juv.\textunderscore , I, 237.
\section{Procumbir}
\begin{itemize}
\item {Grp. gram.:v. i.}
\end{itemize}
\begin{itemize}
\item {Proveniência:(Lat. \textunderscore procumbere\textunderscore )}
\end{itemize}
Cair para deante.
Estirar-se morto ou ferido.
Prosternar-se. Cf. Castilho, \textunderscore Fastos\textunderscore , I, 77.
\section{Procura}
\begin{itemize}
\item {Grp. gram.:f.}
\end{itemize}
Acto de procurar.
Conjunto das producções ou dos serviços, que se pedem no commércio ou na indústria: \textunderscore o equilíbrio entre a offerta e a procura\textunderscore .
\section{Procuração}
\begin{itemize}
\item {Grp. gram.:f.}
\end{itemize}
\begin{itemize}
\item {Utilização:Ant.}
\end{itemize}
\begin{itemize}
\item {Utilização:Ant.}
\end{itemize}
\begin{itemize}
\item {Proveniência:(Lat. \textunderscore procuratio\textunderscore )}
\end{itemize}
Incumbência, que alguém dá a outrem, para tratar de negócios em nome daquelle.
Documento, onde se consigna legalmente essa incumbência.
Despensa ou celleiro de casas religiosas.
O mesmo que \textunderscore colheita\textunderscore  de cereaes.
\section{Procuradeira}
\begin{itemize}
\item {Grp. gram.:f.}
\end{itemize}
\begin{itemize}
\item {Proveniência:(De \textunderscore procurar\textunderscore )}
\end{itemize}
Mulhér curiosa, que gosta de procurar ou investigar.
\section{Procurador}
\begin{itemize}
\item {Grp. gram.:adj.}
\end{itemize}
\begin{itemize}
\item {Grp. gram.:M.}
\end{itemize}
\begin{itemize}
\item {Proveniência:(Lat. \textunderscore procurator\textunderscore )}
\end{itemize}
Que procura.
Indivíduo, que tem procuração para tratar negócios de outrem.
Interventor; mediador.
Administrador.
\section{Procuradoria}
\begin{itemize}
\item {Grp. gram.:f.}
\end{itemize}
Offício de procurador.
Funcções de procurador.
Escritório de procurador.
\section{Procurar}
\begin{itemize}
\item {Grp. gram.:v. t.}
\end{itemize}
\begin{itemize}
\item {Utilização:Pop.}
\end{itemize}
\begin{itemize}
\item {Proveniência:(Lat. \textunderscore procurare\textunderscore )}
\end{itemize}
Esforçar-se por achar.
Buscar; investigar.
Pretender escolher.
Dirigir-se para.
Curar, tratar de.
Proporcionar.
Perguntar.
\section{Procuratoria}
\begin{itemize}
\item {Grp. gram.:f.}
\end{itemize}
O mesmo que \textunderscore procuradoria\textunderscore .
\section{Procuratório}
\begin{itemize}
\item {Grp. gram.:adj.}
\end{itemize}
\begin{itemize}
\item {Proveniência:(Lat. \textunderscore procuratorius\textunderscore )}
\end{itemize}
Relativo a procuração ou a procurador.
\section{Procuratura}
\begin{itemize}
\item {Grp. gram.:f.}
\end{itemize}
\begin{itemize}
\item {Utilização:P. us.}
\end{itemize}
\begin{itemize}
\item {Proveniência:(Do lat. \textunderscore procuratus\textunderscore )}
\end{itemize}
O mesmo que \textunderscore procuradoria\textunderscore :«\textunderscore ...a procuratura dos negócios sínicos em Macau\textunderscore ».
\section{Procusta}
\begin{itemize}
\item {Grp. gram.:m.}
\end{itemize}
\begin{itemize}
\item {Utilização:Fig.}
\end{itemize}
\begin{itemize}
\item {Proveniência:(Do gr. \textunderscore Prokustes\textunderscore , n. p.)}
\end{itemize}
Salteador mitológico, que estendia os viajantes num leito muito curto, cortando-lhes a parte que excedia o leito.
\textunderscore Leito de Procusto\textunderscore , situação aflictiva e tirânica.
\section{Procusteano}
\begin{itemize}
\item {Grp. gram.:adj.}
\end{itemize}
\begin{itemize}
\item {Utilização:Neol.}
\end{itemize}
\begin{itemize}
\item {Utilização:Fig.}
\end{itemize}
\begin{itemize}
\item {Proveniência:(De \textunderscore Procusto\textunderscore , n. p.)}
\end{itemize}
Relativo ao leito de Procusto.
Lancinante e tyrânnico.
\section{Procusto}
\begin{itemize}
\item {Grp. gram.:m.}
\end{itemize}
\begin{itemize}
\item {Utilização:Fig.}
\end{itemize}
\begin{itemize}
\item {Proveniência:(Do gr. \textunderscore Prokustes\textunderscore , n. p.)}
\end{itemize}
Salteador mythológico, que estendia os viajantes num leito muito curto, cortando-lhes a parte que excedia o leito.
\textunderscore Leito de Procusto\textunderscore , situação afflictiva e tyrânnica.
\section{Procutos}
\begin{itemize}
\item {Grp. gram.:m. pl.}
\end{itemize}
Indígenas do norte do Brasil.
\section{Prodiagnóstico}
\begin{itemize}
\item {Grp. gram.:m.}
\end{itemize}
\begin{itemize}
\item {Proveniência:(De \textunderscore pro...\textunderscore  + \textunderscore diagnóstico\textunderscore )}
\end{itemize}
Diagnóstico antecipado.
\section{Prodigalidade}
\begin{itemize}
\item {Grp. gram.:f.}
\end{itemize}
\begin{itemize}
\item {Proveniência:(Lat. \textunderscore prodigalitas\textunderscore )}
\end{itemize}
Qualidade de pródigo.
Acto de prodigalizar.
Dissipação.
Gastos exaggeradados.
Abundância.
Bizarria, liberalidade.
\section{Prodigalíssimo}
\begin{itemize}
\item {Grp. gram.:adj.}
\end{itemize}
Muito pródigo.
(B. lat. \textunderscore prodigalissimus\textunderscore )
\section{Prodigalizador}
\begin{itemize}
\item {Grp. gram.:m.  e  adj.}
\end{itemize}
O que prodigaliza; pródigo.
\section{Prodigalizar}
\begin{itemize}
\item {Grp. gram.:v. t.}
\end{itemize}
\begin{itemize}
\item {Utilização:Fig.}
\end{itemize}
\begin{itemize}
\item {Proveniência:(De \textunderscore pródigo\textunderscore )}
\end{itemize}
Gastar excessivamente; desperdiçar, dissipar.
Arriscar.
Revelar ou empregar com profusão: \textunderscore prodigalizar carícias\textunderscore .
Despender generosamente.
\section{Prodigamente}
\begin{itemize}
\item {Grp. gram.:adv.}
\end{itemize}
De modo pródigo; com dissipação, com esbanjamento.
\section{Prodigar}
\begin{itemize}
\item {Proveniência:(Lat. \textunderscore prodigare\textunderscore )}
\end{itemize}
\textunderscore v. t.\textunderscore  (e der.)
O mesmo que \textunderscore prodigalizar\textunderscore , etc.
\section{Prodígio}
\begin{itemize}
\item {Grp. gram.:m.}
\end{itemize}
\begin{itemize}
\item {Proveniência:(Lat. \textunderscore prodigium\textunderscore )}
\end{itemize}
Coisa sobrenatural.
Coisa ou pessôa anormal.
Maravilha; milagre.
\section{Prodigiosamente}
\begin{itemize}
\item {Grp. gram.:adv.}
\end{itemize}
De modo prodigioso; assombrosamente; espantosamente.
\section{Proeza}
\begin{itemize}
\item {Grp. gram.:f.}
\end{itemize}
\begin{itemize}
\item {Proveniência:(Do fr. \textunderscore prouesse\textunderscore )}
\end{itemize}
Acção de valor; façanha.
Acto extraordinário.
Procedimento digno de censura; escândalo.
\section{Prof.}
Abrev., que precede nomes próprios e significa \textunderscore professor\textunderscore .
\section{Profaçar}
\begin{itemize}
\item {Grp. gram.:v. t.}
\end{itemize}
\begin{itemize}
\item {Utilização:Ant.}
\end{itemize}
\begin{itemize}
\item {Proveniência:(De \textunderscore pro\textunderscore  + \textunderscore face\textunderscore )}
\end{itemize}
Invectivar; lançar em rosto a.
Estorvar, impedir. Cf. \textunderscore Port. Mon. Hist. Script.\textunderscore  314.
\section{Profanação}
\begin{itemize}
\item {Grp. gram.:f.}
\end{itemize}
\begin{itemize}
\item {Proveniência:(Lat. \textunderscore profanatio\textunderscore )}
\end{itemize}
Acto ou effeito de profanar.
Sacrilégio.
\section{Profanador}
\begin{itemize}
\item {Grp. gram.:m.  e  adj.}
\end{itemize}
\begin{itemize}
\item {Proveniência:(Lat. \textunderscore profanator\textunderscore )}
\end{itemize}
O que profana.
\section{Profanamente}
\begin{itemize}
\item {Grp. gram.:adv.}
\end{itemize}
De modo profano.
\section{Profanar}
\begin{itemize}
\item {Grp. gram.:v. i.}
\end{itemize}
\begin{itemize}
\item {Proveniência:(Lat. \textunderscore profanare\textunderscore )}
\end{itemize}
Tratar com irreverência (coisas de religião).
Dar applicação profana a.
Fazer mau uso de.
Macular; aviltar.
\section{Profanete}
\begin{itemize}
\item {fónica:nê}
\end{itemize}
\begin{itemize}
\item {Grp. gram.:adj.}
\end{itemize}
Um tanto profano:«\textunderscore ...que faz ao ponto, se profanete, acaso, algum desejo na intenção se ingeriu...\textunderscore »Garrett, \textunderscore D. Branca\textunderscore , 135.
\section{Profanidade}
\begin{itemize}
\item {Grp. gram.:f.}
\end{itemize}
Acto ou dito profano; profanação.
\section{Profano}
\begin{itemize}
\item {Grp. gram.:adj.}
\end{itemize}
\begin{itemize}
\item {Utilização:Fig.}
\end{itemize}
\begin{itemize}
\item {Grp. gram.:M.}
\end{itemize}
\begin{itemize}
\item {Proveniência:(Lat. \textunderscore profanus\textunderscore )}
\end{itemize}
Que não pertence á religião.
Opposto ao respeito que se deve ás coisas sagradas.
Secular; leigo.
Alheio; que não tem ideias ou conhecimentos, relativamente a certo assumpto.
Aquelle ou aquillo que é profano ou estranho a coisas religiosas.
\section{Profectício}
\begin{itemize}
\item {Grp. gram.:adj.}
\end{itemize}
\begin{itemize}
\item {Proveniência:(Lat. \textunderscore profecticius\textunderscore )}
\end{itemize}
Diz-se dos bens, que procedem da herança de ascendentes.
\section{Profeitamento}
\begin{itemize}
\item {Grp. gram.:m.}
\end{itemize}
\begin{itemize}
\item {Utilização:Ant.}
\end{itemize}
O mesmo que \textunderscore aproveitamento\textunderscore .
\section{Profeitança}
\begin{itemize}
\item {Grp. gram.:f.}
\end{itemize}
\begin{itemize}
\item {Utilização:Ant.}
\end{itemize}
O mesmo que \textunderscore profeito\textunderscore . Cf. Frei Fortun., \textunderscore Inéd.\textunderscore , 312.
\section{Profeito}
\begin{itemize}
\item {Grp. gram.:m.}
\end{itemize}
\begin{itemize}
\item {Utilização:Ant.}
\end{itemize}
O mesmo que \textunderscore proveito\textunderscore .
\section{Proferir}
\begin{itemize}
\item {Grp. gram.:v. t.}
\end{itemize}
\begin{itemize}
\item {Proveniência:(Do lat. \textunderscore proferre\textunderscore )}
\end{itemize}
Pronunciar em voz alta e intelligivel.
Dizer.
Dizer, lendo.
Publicar em voz alta.
\section{Professante}
\begin{itemize}
\item {Grp. gram.:m.  e  f.}
\end{itemize}
Pessôa, que professa ou faz profissão. Cf. B. Pereira, \textunderscore Prosodia\textunderscore , vb. \textunderscore professor\textunderscore .
\section{Professar}
\begin{itemize}
\item {Grp. gram.:v. t.}
\end{itemize}
\begin{itemize}
\item {Grp. gram.:V. i.}
\end{itemize}
\begin{itemize}
\item {Proveniência:(De \textunderscore professo\textunderscore )}
\end{itemize}
Reconhecer publicamente; confessar: \textunderscore professar uma religião\textunderscore .
Exercer.
Ensinar: \textunderscore professar Mathemática\textunderscore .
Propagar; preconizar.
Seguir a regra de; adoptar (certa doutrina, etc.).
Fazer votos, ligando-se a uma Ordem religiosa ou equestre.
\section{Professo}
\begin{itemize}
\item {Grp. gram.:adj.}
\end{itemize}
\begin{itemize}
\item {Utilização:Fig.}
\end{itemize}
\begin{itemize}
\item {Grp. gram.:M.}
\end{itemize}
\begin{itemize}
\item {Proveniência:(Lat. \textunderscore professus\textunderscore )}
\end{itemize}
Que professou numa Ordem religiosa ou equestre.
Relativo a frades ou freiras.
Perito, adestrado.
Frade, que professou.
Indivíduo adestrado.
\section{Professor}
\begin{itemize}
\item {Grp. gram.:m.}
\end{itemize}
\begin{itemize}
\item {Utilização:Fig.}
\end{itemize}
\begin{itemize}
\item {Proveniência:(Lat. \textunderscore professor\textunderscore )}
\end{itemize}
Aquelle que professa ou ensina.
Homem adestrado ou perito.
O que professa publicamente as verdades religiosas.
\section{Professora}
\begin{itemize}
\item {Grp. gram.:f.}
\end{itemize}
Mulhér, que ensina ou que exerce o professorado; mestra.
\section{Professoraço}
\begin{itemize}
\item {Grp. gram.:m.}
\end{itemize}
\begin{itemize}
\item {Utilização:Deprec.}
\end{itemize}
Professor vaidoso e de pouco valor. Cf. Macedo, \textunderscore Motim\textunderscore , II, 287.
\section{Professorado}
\begin{itemize}
\item {Grp. gram.:m.}
\end{itemize}
Cargo ou funcções de professor; classe dos professores: \textunderscore os interesses do professorado\textunderscore .
\section{Professoral}
\begin{itemize}
\item {Grp. gram.:adj.}
\end{itemize}
Relativo ao professor ou a professora: \textunderscore vida professoral\textunderscore . Cf. Latino, \textunderscore Elog. Acad.\textunderscore , 21; T. de Queiroz, \textunderscore Comédia do Campo\textunderscore , II, 148.
\section{Professorando}
\begin{itemize}
\item {Grp. gram.:m.}
\end{itemize}
\begin{itemize}
\item {Utilização:bras}
\end{itemize}
\begin{itemize}
\item {Utilização:Neol.}
\end{itemize}
Aquelle que se habilita para professor.
\section{Profusão}
\begin{itemize}
\item {Grp. gram.:f.}
\end{itemize}
\begin{itemize}
\item {Proveniência:(Lat. \textunderscore profusio\textunderscore )}
\end{itemize}
Gasto excessivo.
Grande porção, grande abundância, exuberância.
\section{Profuso}
\begin{itemize}
\item {Grp. gram.:adj.}
\end{itemize}
\begin{itemize}
\item {Proveniência:(Lat. \textunderscore profusus\textunderscore )}
\end{itemize}
Que espalha em abundância; que dissipa ou gasta muito.
Copioso, exuberante.
\section{Progênia}
\begin{itemize}
\item {Grp. gram.:f.}
\end{itemize}
O mesmo que \textunderscore progênie\textunderscore . Cf. \textunderscore Peregrinação\textunderscore , XLIII.
\section{Progênie}
\begin{itemize}
\item {Grp. gram.:f.}
\end{itemize}
\begin{itemize}
\item {Proveniência:(Lat. \textunderscore progenies\textunderscore )}
\end{itemize}
Procedência, origem.
Ascendência.
Geração, prole.
\section{Progênito}
\begin{itemize}
\item {Grp. gram.:m.  e  adj.}
\end{itemize}
\begin{itemize}
\item {Utilização:Poét.}
\end{itemize}
\begin{itemize}
\item {Proveniência:(Lat. \textunderscore progenitus\textunderscore )}
\end{itemize}
O que é gerado ou procriado; descendente.
\section{Progenitor}
\begin{itemize}
\item {Grp. gram.:m.}
\end{itemize}
\begin{itemize}
\item {Proveniência:(Lat. \textunderscore progenitor\textunderscore )}
\end{itemize}
O que procria antes do pai; avô; ascendente.
\section{Progenitura}
\begin{itemize}
\item {Grp. gram.:f.}
\end{itemize}
O mesmo que \textunderscore progênie\textunderscore .
\section{Proglote}
\begin{itemize}
\item {Grp. gram.:f.}
\end{itemize}
\begin{itemize}
\item {Utilização:Zool.}
\end{itemize}
\begin{itemize}
\item {Proveniência:(Gr. \textunderscore proglottis\textunderscore )}
\end{itemize}
Anel ou segmento completo da tênia.
\section{Proglotte}
\begin{itemize}
\item {Grp. gram.:f.}
\end{itemize}
\begin{itemize}
\item {Utilização:Zool.}
\end{itemize}
\begin{itemize}
\item {Proveniência:(Gr. \textunderscore proglottis\textunderscore )}
\end{itemize}
Anel ou segmento completo da tênia.
\section{Prognathismo}
\begin{itemize}
\item {Grp. gram.:m.}
\end{itemize}
\begin{itemize}
\item {Proveniência:(De \textunderscore prógnatho\textunderscore )}
\end{itemize}
Conformação da face, em que as maxillas são alongadas. Cf. E. Burnay, \textunderscore Craniologia\textunderscore , 62; Capello e Ivens, I, 99.
\section{Prógnatho}
\begin{itemize}
\item {Grp. gram.:adj.}
\end{itemize}
\begin{itemize}
\item {Proveniência:(Do gr. \textunderscore pro\textunderscore  + \textunderscore gnathos\textunderscore )}
\end{itemize}
Que tem maxillas alongadas ou proeminentes.
Diz-se de certas raças africanas e australianas, caracterizadas pelo alongamento e proeminência das maxillas.--Há quem mande lêr \textunderscore prognáto\textunderscore , que é prosódia mais generalizada.
\section{Prognatismo}
\begin{itemize}
\item {Grp. gram.:m.}
\end{itemize}
\begin{itemize}
\item {Proveniência:(De \textunderscore prógnato\textunderscore )}
\end{itemize}
Conformação da face, em que as maxilas são alongadas. Cf. E. Burnay, \textunderscore Craniologia\textunderscore , 62; Capello e Ivens, I, 99.
\section{Prógnato}
\begin{itemize}
\item {Grp. gram.:adj.}
\end{itemize}
\begin{itemize}
\item {Proveniência:(Do gr. \textunderscore pro\textunderscore  + \textunderscore gnathos\textunderscore )}
\end{itemize}
Que tem maxilas alongadas ou proeminentes.
Diz-se de certas raças africanas e australianas, caracterizadas pelo alongamento e proeminência das maxilas.--Há quem mande lêr \textunderscore prognáto\textunderscore , que é prosódia mais generalizada.
\section{Progne}
\begin{itemize}
\item {Grp. gram.:f.}
\end{itemize}
\begin{itemize}
\item {Utilização:Poét.}
\end{itemize}
\begin{itemize}
\item {Proveniência:(Do gr. \textunderscore Progne\textunderscore , n. p.)}
\end{itemize}
Andorinha.
\section{Prognose}
\begin{itemize}
\item {Grp. gram.:f.}
\end{itemize}
\begin{itemize}
\item {Proveniência:(Lat. \textunderscore prognosis\textunderscore )}
\end{itemize}
Doutrina hippocrática das doenças febris, agudas, relativamente á sua marcha, indícios de accidentes, crises e soluções.
\section{Prognosticação}
\begin{itemize}
\item {Grp. gram.:f.}
\end{itemize}
O mesmo que \textunderscore prognóstico\textunderscore .
\section{Prognosticar}
\begin{itemize}
\item {Grp. gram.:v. t.}
\end{itemize}
\begin{itemize}
\item {Grp. gram.:V. i.}
\end{itemize}
\begin{itemize}
\item {Utilização:Med.}
\end{itemize}
Predizer, annunciar (successos futuros).
Presagiar; prophetizar.
Fazer o prognóstico de uma doença.
\section{Prognóstico}
\begin{itemize}
\item {Grp. gram.:m.}
\end{itemize}
\begin{itemize}
\item {Proveniência:(Lat. \textunderscore prognosticus\textunderscore )}
\end{itemize}
Acto ou effeito de prognosticar.
Conjectura.
Parecer do médico, á cêrca do seguimento e resultado de uma doença.
\section{Programa}
\begin{itemize}
\item {Grp. gram.:m.}
\end{itemize}
\begin{itemize}
\item {Proveniência:(Lat. \textunderscore programma\textunderscore )}
\end{itemize}
Projecto ou plano escrito e minucioso de uma festa pública.
Plano.
Indicação geral de um sistema político: \textunderscore o programa do partido republicano\textunderscore .
Indicação das matérias que se hão de professar numa escola, á cêrca das quaes os alumnos têm de sêr interrogados.
Prospecto.
\section{Programatizar}
\begin{itemize}
\item {Grp. gram.:v. t.}
\end{itemize}
\begin{itemize}
\item {Proveniência:(De \textunderscore programa\textunderscore )}
\end{itemize}
Fazer o programa de; planear; esboçar. Us. por Camillo.
\section{Programma}
\begin{itemize}
\item {Grp. gram.:m.}
\end{itemize}
\begin{itemize}
\item {Proveniência:(Lat. \textunderscore programma\textunderscore )}
\end{itemize}
Projecto ou plano escrito e minucioso de uma festa pública.
Plano.
Indicação geral de um systema político: \textunderscore o programma do partido republicano\textunderscore .
Indicação das matérias que se hão de professar numa escola, á cêrca das quaes os alumnos têm de sêr interrogados.
Prospecto.
\section{Programmar}
\begin{itemize}
\item {Grp. gram.:v. t.}
\end{itemize}
O mesmo que \textunderscore programmatizar\textunderscore . Cf. Camillo, \textunderscore Amor de Salv.\textunderscore , 181.
\section{Programmatizar}
\begin{itemize}
\item {Grp. gram.:v. t.}
\end{itemize}
\begin{itemize}
\item {Proveniência:(De \textunderscore programma\textunderscore )}
\end{itemize}
Fazer o programma de; planear; esboçar. Us. por Camillo.
\section{Progredimento}
\begin{itemize}
\item {Grp. gram.:m.}
\end{itemize}
Acto ou effeito de progredir.
\section{Progredir}
\begin{itemize}
\item {Grp. gram.:v. i.}
\end{itemize}
\begin{itemize}
\item {Proveniência:(Lat. \textunderscore progredi\textunderscore )}
\end{itemize}
Caminhar adeante; avançar.
Ir aumentando.
Desenvolver-se: \textunderscore a Cirurgia progride\textunderscore .
\section{Progressão}
\begin{itemize}
\item {Grp. gram.:f.}
\end{itemize}
\begin{itemize}
\item {Utilização:Mathem.}
\end{itemize}
\begin{itemize}
\item {Proveniência:(Lat. \textunderscore progressio\textunderscore )}
\end{itemize}
Progredimento; continuação.
Série de números ou quantidades, que derivam successivamente umas das outras, segundo a mesma lei.
\section{Progressista}
\begin{itemize}
\item {Grp. gram.:adj.}
\end{itemize}
\begin{itemize}
\item {Grp. gram.:M.}
\end{itemize}
Relativo ao progresso ou a progressistas.
Partidário das ideias do progresso.
Membro de um partido político, que em Portugal se denominou \textunderscore progressista\textunderscore .
\section{Progressivamente}
\begin{itemize}
\item {Grp. gram.:adv.}
\end{itemize}
De modo progressivo; com progresso.
\section{Progressivo}
\begin{itemize}
\item {Grp. gram.:adj.}
\end{itemize}
\begin{itemize}
\item {Proveniência:(De \textunderscore progresso\textunderscore )}
\end{itemize}
Que muda de lugar, andando.
Que progride.
Que se vai realizando pouco a pouco.
Que segue uma progressão.
\section{Prolator}
\begin{itemize}
\item {Grp. gram.:m.}
\end{itemize}
\begin{itemize}
\item {Utilização:Jur.}
\end{itemize}
\begin{itemize}
\item {Proveniência:(Lat. \textunderscore prolator\textunderscore )}
\end{itemize}
Aquelle que promulga (uma lei).
\section{Prole}
\begin{itemize}
\item {Grp. gram.:f.}
\end{itemize}
\begin{itemize}
\item {Proveniência:(Lat. \textunderscore proles\textunderscore )}
\end{itemize}
Progênie; geração.
Filho ou filhos.
Successão.
\section{Prolegómenos}
\begin{itemize}
\item {Grp. gram.:f. pl.}
\end{itemize}
\begin{itemize}
\item {Proveniência:(Gr. \textunderscore prolegomena\textunderscore )}
\end{itemize}
Introducção geral de uma obra sciêntifica ou artística.
Prefácio longo.
Exposição preliminar dos princípios geraes de uma sciência ou arte.
\section{Prolepse}
\begin{itemize}
\item {Grp. gram.:f.}
\end{itemize}
\begin{itemize}
\item {Proveniência:(Lat. \textunderscore prolepsis\textunderscore )}
\end{itemize}
Figura de Rhetórica, que consiste em prevenir objecções, fazendo-as a si próprio e destruíndo-as.
\section{Prolepticamente}
\begin{itemize}
\item {Grp. gram.:adv.}
\end{itemize}
De modo proléptico.
\section{Proléptico}
\begin{itemize}
\item {Grp. gram.:adj.}
\end{itemize}
\begin{itemize}
\item {Utilização:Med.}
\end{itemize}
\begin{itemize}
\item {Proveniência:(Gr. \textunderscore proleptikos\textunderscore )}
\end{itemize}
Relativo a prolepse.
Diz-se da febre, cujos accessos se antecipam, quanto á hora, em que normalmente appareceriam.
\section{Proletariado}
\begin{itemize}
\item {Grp. gram.:m.}
\end{itemize}
\begin{itemize}
\item {Proveniência:(De \textunderscore proletário\textunderscore )}
\end{itemize}
Classe dos proletários.
Estado de proletário. Cf. Herculano, \textunderscore Opúsc.\textunderscore , IV, 135.
\section{Proletário}
\begin{itemize}
\item {Grp. gram.:m.}
\end{itemize}
\begin{itemize}
\item {Proveniência:(Lat. \textunderscore proletarius\textunderscore )}
\end{itemize}
Cidadão pobre, que pertence á última classe do povo, entre os Romanos.
Membro da classe mais indigente do povo.
Indivíduo pobre, que vive de um trabalho mal retribuido.
\section{Prolfaça}
\begin{itemize}
\item {Grp. gram.:m.}
\end{itemize}
\begin{itemize}
\item {Utilização:P. us.}
\end{itemize}
\begin{itemize}
\item {Proveniência:(De \textunderscore prol\textunderscore ^1 + \textunderscore fazer\textunderscore . Entretanto, como nas antigas bodas se brindava á nubente, dizendo-se \textunderscore prolfaça\textunderscore , entende o autor do \textunderscore Elucidário\textunderscore  que o t. é contr. de \textunderscore prole\textunderscore  + \textunderscore faça\textunderscore  e exprime o voto ou desejo de que a noiva fôsse fecunda)}
\end{itemize}
O mesmo que \textunderscore parabem\textunderscore :«\textunderscore ...e a grandes vozes lhes dão os prolfaças da sua chegada\textunderscore ». Filinto, \textunderscore D. Man.\textunderscore , I, 75.
\section{Proliferação}
\begin{itemize}
\item {Grp. gram.:f.}
\end{itemize}
\begin{itemize}
\item {Proveniência:(De \textunderscore prolífero\textunderscore )}
\end{itemize}
Producção physiológica de tecidos adventícios.
\section{Proliferar}
\begin{itemize}
\item {Grp. gram.:v. i.}
\end{itemize}
\begin{itemize}
\item {Proveniência:(De \textunderscore prolífero\textunderscore )}
\end{itemize}
Têr prole; têr geração.
Reproduzir-se (o micróbio, etc.).
\section{Prolífero}
\begin{itemize}
\item {Grp. gram.:adj.}
\end{itemize}
\begin{itemize}
\item {Utilização:Bot.}
\end{itemize}
\begin{itemize}
\item {Proveniência:(Lat. \textunderscore prolifer\textunderscore )}
\end{itemize}
O mesmo que \textunderscore prolífico\textunderscore .
Diz-se de qualquer órgão vegetal, que dá origem a outro órgão igual ou anormal; e diz-se das flôres, de cujo centro parte um eixo que termina por outra flôr.
\section{Prolificação}
\begin{itemize}
\item {Grp. gram.:f.}
\end{itemize}
Acto ou effeito de prolificar.
Monstruosidade vegetal, que consiste na multiplicação de órgãos rudimentares.
\section{Prolificar}
\begin{itemize}
\item {Grp. gram.:v. i.}
\end{itemize}
\begin{itemize}
\item {Proveniência:(De \textunderscore prolífico\textunderscore )}
\end{itemize}
Têr prole; reproduzir-se.
\section{Prolífico}
\begin{itemize}
\item {Grp. gram.:adj.}
\end{itemize}
\begin{itemize}
\item {Proveniência:(Lat. \textunderscore prolificus\textunderscore )}
\end{itemize}
Que faz prole.
Que tem a faculdade de gerar.
Fecundante.
Que tem prole numerosa.
\section{Proligero}
\begin{itemize}
\item {Grp. gram.:adj.}
\end{itemize}
\begin{itemize}
\item {Proveniência:(Do lat. \textunderscore proles\textunderscore  + \textunderscore gerere\textunderscore )}
\end{itemize}
Que contém germes.
\section{Prolixamente}
\begin{itemize}
\item {fónica:csa}
\end{itemize}
\begin{itemize}
\item {Grp. gram.:adv.}
\end{itemize}
De modo prolixo.
\section{Prolixidade}
\begin{itemize}
\item {fónica:csi}
\end{itemize}
\begin{itemize}
\item {Grp. gram.:f.}
\end{itemize}
\begin{itemize}
\item {Proveniência:(Lat. \textunderscore prolixitas\textunderscore )}
\end{itemize}
Qualidade do que é prolixo.
\section{Prolixo}
\begin{itemize}
\item {fónica:cso}
\end{itemize}
\begin{itemize}
\item {Grp. gram.:adj.}
\end{itemize}
\begin{itemize}
\item {Utilização:Ext.}
\end{itemize}
\begin{itemize}
\item {Proveniência:(Lat. \textunderscore prolixus\textunderscore )}
\end{itemize}
Muito longo ou expresso em muitas palavras, faladas ou escritas.
Que se exprime em termos superabundantes.
Superabundante.
Difuso.
Duradoiro.
Fastidioso.
\section{Prologar}
\begin{itemize}
\item {Grp. gram.:v. t.}
\end{itemize}
\begin{itemize}
\item {Proveniência:(De \textunderscore prólogo\textunderscore )}
\end{itemize}
O mesmo que \textunderscore prefaciar\textunderscore .
\section{Prólogo}
\begin{itemize}
\item {Grp. gram.:m.}
\end{itemize}
\begin{itemize}
\item {Proveniência:(Gr. \textunderscore prologos\textunderscore )}
\end{itemize}
Prefácio, preâmbulo, proêmio.
Parte de um drama, que era representada por uma só personagem, antes da peça propriamente dita.
\section{Prolonga}
\begin{itemize}
\item {Grp. gram.:f.}
\end{itemize}
Acto de prolongar; prolongação.
Corda, que une o reparo ao armão, para fazer fogo, nas peças de artilharia.
\section{Prolongação}
\begin{itemize}
\item {Grp. gram.:f.}
\end{itemize}
Acto ou effeito de prolongar.
\section{Promiscuidade}
\begin{itemize}
\item {fónica:cu-i}
\end{itemize}
\begin{itemize}
\item {Grp. gram.:f.}
\end{itemize}
Qualidade do que é promíscuo.
\section{Promíscuo}
\begin{itemize}
\item {Grp. gram.:adj.}
\end{itemize}
\begin{itemize}
\item {Utilização:Gram.}
\end{itemize}
\begin{itemize}
\item {Proveniência:(Lat. \textunderscore promiscuus\textunderscore )}
\end{itemize}
Misturado; indistinto; confuso.
Diz-se dos nomes epicenos.
\section{Promissão}
\begin{itemize}
\item {Grp. gram.:f.}
\end{itemize}
\begin{itemize}
\item {Proveniência:(Lat. \textunderscore promissio\textunderscore )}
\end{itemize}
O mesmo que \textunderscore promessa\textunderscore .
\textunderscore Terra da promissão\textunderscore , a terra de Canaan, prometida por Deus a Abrahão e aos seus descendentes.
\textunderscore Uva da promissão\textunderscore , qualidade de uva branca, de grandes bagos.
\section{Promissivo}
\begin{itemize}
\item {Grp. gram.:adj.}
\end{itemize}
\begin{itemize}
\item {Proveniência:(Lat. \textunderscore promissivus\textunderscore )}
\end{itemize}
O mesmo que \textunderscore promissório\textunderscore .
\section{Promisso}
\begin{itemize}
\item {Grp. gram.:adj.}
\end{itemize}
\begin{itemize}
\item {Utilização:Des.}
\end{itemize}
\begin{itemize}
\item {Proveniência:(Lat. \textunderscore promissus\textunderscore )}
\end{itemize}
O mesmo que \textunderscore prometido\textunderscore . Cf. Filinto, XI, 7.
\section{Promissor}
\begin{itemize}
\item {Grp. gram.:adj.}
\end{itemize}
\begin{itemize}
\item {Proveniência:(Lat. \textunderscore promissor\textunderscore )}
\end{itemize}
O mesmo que \textunderscore promittente\textunderscore .
\section{Promissória}
\begin{itemize}
\item {Grp. gram.:f.}
\end{itemize}
Título, passado por um Banco ou indivíduo, que recebeu uma quantia em depósito, e no qual se reconhece e se confessa a qualidade de devedor daquella quantia.
(Fem. de \textunderscore promissório\textunderscore )
\section{Promissório}
\begin{itemize}
\item {Grp. gram.:adj.}
\end{itemize}
\begin{itemize}
\item {Proveniência:(Do lat. \textunderscore promissus\textunderscore )}
\end{itemize}
Relativo a promessa; em que há promessa.
\section{Promitente}
\begin{itemize}
\item {Grp. gram.:m. ,  f.  e  adj.}
\end{itemize}
\begin{itemize}
\item {Proveniência:(Lat. \textunderscore promittens\textunderscore )}
\end{itemize}
Pessôa, que promete.
\section{Promittente}
\begin{itemize}
\item {Grp. gram.:m. ,  f.  e  adj.}
\end{itemize}
\begin{itemize}
\item {Proveniência:(Lat. \textunderscore promittens\textunderscore )}
\end{itemize}
Pessôa, que promete.
\section{Promoção}
\begin{itemize}
\item {Grp. gram.:f.}
\end{itemize}
\begin{itemize}
\item {Proveniência:(Lat. \textunderscore promotio\textunderscore )}
\end{itemize}
Acto ou effeito de promover.
Accesso ou elevação a cargo ou categoria superior.
Acto de promotor.
Petição dos agentes do Ministério Público, na qual se propõe que se proceda a certos actos judiciaes.
\section{Promontório}
\begin{itemize}
\item {Grp. gram.:m.}
\end{itemize}
\begin{itemize}
\item {Utilização:Geogr.}
\end{itemize}
\begin{itemize}
\item {Utilização:Anat.}
\end{itemize}
\begin{itemize}
\item {Proveniência:(Lat. \textunderscore promontorium\textunderscore )}
\end{itemize}
Cabo; porção de terra elevada que, entrando pelo mar, fórma saliência acima do nível das águas.
Pequena saliência da parede interna do týmpano.
Saliência, formada pela articulação do sacro com a vértebra lombar inferior.
\section{Promorfose}
\begin{itemize}
\item {Grp. gram.:f.}
\end{itemize}
\begin{itemize}
\item {Proveniência:(Do gr. \textunderscore pro\textunderscore  + \textunderscore morphe\textunderscore )}
\end{itemize}
Propriedade, que a matéria viva tem, segundo Bouchut, de tomar uma fórma particular, em virtude de uma acção seminal variável.
\section{Promorphose}
\begin{itemize}
\item {Grp. gram.:f.}
\end{itemize}
\begin{itemize}
\item {Proveniência:(Do gr. \textunderscore pro\textunderscore  + \textunderscore morphe\textunderscore )}
\end{itemize}
Propriedade, que a matéria viva tem, segundo Bouchut, de tomar uma fórma particular, em virtude de uma acção seminal variável.
\section{Promotor}
\begin{itemize}
\item {Grp. gram.:adj.}
\end{itemize}
\begin{itemize}
\item {Grp. gram.:M.}
\end{itemize}
\begin{itemize}
\item {Proveniência:(Lat. \textunderscore promotor\textunderscore )}
\end{itemize}
Que promove, fomenta ou desenvolve.
Aquelle ou aquillo que promove ou determina.
Fautor.
Funccionário, que nalguns tribunaes promove o andamento das causas e certos actos de justiça.
\section{Promotoria}
\begin{itemize}
\item {Grp. gram.:f.}
\end{itemize}
Cargo de promotor.
Escritório ou secretaria do promotor.
\section{Promovedor}
\begin{itemize}
\item {Grp. gram.:m.  e  adj.}
\end{itemize}
\begin{itemize}
\item {Proveniência:(De \textunderscore promover\textunderscore )}
\end{itemize}
O mesmo que \textunderscore promotor\textunderscore .
\section{Promover}
\begin{itemize}
\item {Grp. gram.:v. t.}
\end{itemize}
\begin{itemize}
\item {Proveniência:(Lat. \textunderscore promovere\textunderscore )}
\end{itemize}
Fazer avançar.
Dar impulso a.
Fazer executar.
Diligenciar.
Desenvolver.
Causar: \textunderscore promover desordens\textunderscore .
Elevar a uma dignidade ou posto superior.
Requerer, propondo.
\section{Promptamente}
\begin{itemize}
\item {Grp. gram.:adv.}
\end{itemize}
\begin{itemize}
\item {Proveniência:(De \textunderscore prompto\textunderscore )}
\end{itemize}
Com promptidão; logo.
Sem embaraço, sem hesitação: \textunderscore responder promptamente\textunderscore .
\section{Promptidão}
\begin{itemize}
\item {Grp. gram.:f.}
\end{itemize}
\begin{itemize}
\item {Proveniência:(Lat. \textunderscore promptitudo\textunderscore )}
\end{itemize}
Qualidade do que é prompto.
Brevidade; desembaraço.
Facilidade de comprehensão.
\section{Promptificação}
\begin{itemize}
\item {Grp. gram.:f.}
\end{itemize}
Acto ou effeito de promptificar.
\section{Promptificar}
\begin{itemize}
\item {Grp. gram.:v. t.}
\end{itemize}
\begin{itemize}
\item {Grp. gram.:V. p.}
\end{itemize}
\begin{itemize}
\item {Proveniência:(Do lat. \textunderscore promptus\textunderscore  + \textunderscore facere\textunderscore )}
\end{itemize}
O mesmo que \textunderscore apromptar\textunderscore .
Offerecer, ministrar.
Mostrar-se prompto.
Offerecer o seu préstimo.
Declarar a alguém que está disposto para um trabalho ou encargo.
\section{Prompto}
\begin{itemize}
\item {Grp. gram.:adj.}
\end{itemize}
\begin{itemize}
\item {Grp. gram.:Adv.}
\end{itemize}
\begin{itemize}
\item {Grp. gram.:Loc. adv.}
\end{itemize}
\begin{itemize}
\item {Proveniência:(Lat. \textunderscore promptus\textunderscore )}
\end{itemize}
Que se não demora; rápido; ligeiro; ágil.
Activo.
Immediato: \textunderscore resposta prompta\textunderscore .
Que percebe facilmente.
Concluído: \textunderscore a minha obra está prompta\textunderscore .
Disposto.
Desimpedido.
Pomptamente.
\textunderscore Num prompto\textunderscore , num instante; promptamente.
\section{Prontamente}
\begin{itemize}
\item {Grp. gram.:adv.}
\end{itemize}
\begin{itemize}
\item {Proveniência:(De \textunderscore pronto\textunderscore )}
\end{itemize}
Com prontidão; logo.
Sem embaraço, sem hesitação: \textunderscore responder prontamente\textunderscore .
\section{Prontidão}
\begin{itemize}
\item {Grp. gram.:f.}
\end{itemize}
\begin{itemize}
\item {Proveniência:(Lat. \textunderscore promptitudo\textunderscore )}
\end{itemize}
Qualidade do que é pronto.
Brevidade; desembaraço.
Facilidade de compreensão.
\section{Prontificação}
\begin{itemize}
\item {Grp. gram.:f.}
\end{itemize}
Acto ou efeito de prontificar.
\section{Prontificar}
\begin{itemize}
\item {Grp. gram.:v. t.}
\end{itemize}
\begin{itemize}
\item {Grp. gram.:V. p.}
\end{itemize}
\begin{itemize}
\item {Proveniência:(Do lat. \textunderscore promptus\textunderscore  + \textunderscore facere\textunderscore )}
\end{itemize}
O mesmo que \textunderscore aprontar\textunderscore .
Oferecer, ministrar.
Mostrar-se pronto.
Oferecer o seu préstimo.
Declarar a alguém que está disposto para um trabalho ou encargo.
\section{Pronto}
\begin{itemize}
\item {Grp. gram.:adj.}
\end{itemize}
\begin{itemize}
\item {Grp. gram.:Adv.}
\end{itemize}
\begin{itemize}
\item {Grp. gram.:Loc. adv.}
\end{itemize}
\begin{itemize}
\item {Proveniência:(Lat. \textunderscore promptus\textunderscore )}
\end{itemize}
Que se não demora; rápido; ligeiro; ágil.
Activo.
Imediato: \textunderscore resposta pronta\textunderscore .
Que percebe facilmente.
Concluído: \textunderscore a minha obra está pronta\textunderscore .
Disposto.
Desimpedido.
Pomptamente.
\textunderscore Num pronto\textunderscore , num instante; prontamente.
\section{Profalange}
\begin{itemize}
\item {Grp. gram.:f.}
\end{itemize}
\begin{itemize}
\item {Utilização:Anat.}
\end{itemize}
\begin{itemize}
\item {Proveniência:(De \textunderscore pro...\textunderscore  + \textunderscore phalange\textunderscore )}
\end{itemize}
Peça proximal do dedo ou falange propriamente dita.
\section{Profalangeal}
\begin{itemize}
\item {Grp. gram.:adj.}
\end{itemize}
Relativo á profalange.
\section{Profecia}
\begin{itemize}
\item {Grp. gram.:f.}
\end{itemize}
\begin{itemize}
\item {Utilização:Fig.}
\end{itemize}
\begin{itemize}
\item {Proveniência:(Gr. \textunderscore propheteia\textunderscore )}
\end{itemize}
Acto de predizer o futuro; vaticínio; oráculo.
Preságio; conjectura.
\section{Profeta}
\begin{itemize}
\item {Grp. gram.:m.}
\end{itemize}
\begin{itemize}
\item {Proveniência:(Gr. \textunderscore prophetes\textunderscore )}
\end{itemize}
Aquele que, entre os Hebreus, predizia o futuro por inspiração divina.
Vidente; adivinho.
Título, que os muçulmanos dão a Mafoma.
Aquele que faz conjecturas sôbre o futuro.
\section{Profetante}
\begin{itemize}
\item {Grp. gram.:adj.}
\end{itemize}
\begin{itemize}
\item {Proveniência:(De \textunderscore profetar\textunderscore )}
\end{itemize}
Que profeta ou profetiza. Cf. Macedo, \textunderscore Burros\textunderscore , 8.
\section{Profetar}
\begin{itemize}
\item {Grp. gram.:v. t.}
\end{itemize}
\begin{itemize}
\item {Proveniência:(Lat. \textunderscore prophetare\textunderscore )}
\end{itemize}
(V.profetizar)
\section{Profeticamente}
\begin{itemize}
\item {fónica:fé}
\end{itemize}
\begin{itemize}
\item {Grp. gram.:adv.}
\end{itemize}
De modo profético; á maneira de profeta.
\section{Profético}
\begin{itemize}
\item {Grp. gram.:adj.}
\end{itemize}
\begin{itemize}
\item {Proveniência:(Lat. \textunderscore propheticus\textunderscore )}
\end{itemize}
Relativo a profeta ou a profecia.
\section{Profetisa}
\begin{itemize}
\item {Grp. gram.:f.}
\end{itemize}
\begin{itemize}
\item {Proveniência:(Lat. \textunderscore prophetissa\textunderscore )}
\end{itemize}
Mulhér, que faz profecias.
\section{Propagandista}
\begin{itemize}
\item {Grp. gram.:m.  e  f.}
\end{itemize}
Pessôa, que faz propaganda.
\section{Propagar}
\begin{itemize}
\item {Grp. gram.:v. t.}
\end{itemize}
\begin{itemize}
\item {Utilização:Ext.}
\end{itemize}
\begin{itemize}
\item {Grp. gram.:V. i.}
\end{itemize}
\begin{itemize}
\item {Proveniência:(Lat. \textunderscore propagare\textunderscore )}
\end{itemize}
Multiplicar, reproduzindo.
Multiplicar por geração: \textunderscore propagar a espécie\textunderscore .
Dilatar; espalhar; diffundir; aumentar o número de.
Proclamar; vulgarizar; propalar.
Têr prole.
\section{Propagem}
\begin{itemize}
\item {Grp. gram.:f.}
\end{itemize}
\begin{itemize}
\item {Utilização:Bot.}
\end{itemize}
\begin{itemize}
\item {Proveniência:(Lat. \textunderscore propago\textunderscore )}
\end{itemize}
Bolbilho dos musgos.
\section{Propagular}
\begin{itemize}
\item {Grp. gram.:adj.}
\end{itemize}
Relativo ao propágulo.
\section{Propágulo}
\begin{itemize}
\item {Grp. gram.:m.}
\end{itemize}
\begin{itemize}
\item {Utilização:Bot.}
\end{itemize}
\begin{itemize}
\item {Proveniência:(Fr. \textunderscore propagule\textunderscore )}
\end{itemize}
Gomo simples, que póde produzir um vegetal.
\section{Propalador}
\begin{itemize}
\item {Grp. gram.:m.}
\end{itemize}
Aquelle que propala. Cf. Camillo, \textunderscore Curso de Lit.\textunderscore , 69.
\section{Propalar}
\begin{itemize}
\item {Grp. gram.:v. t.}
\end{itemize}
\begin{itemize}
\item {Proveniência:(Lat. \textunderscore propalare\textunderscore )}
\end{itemize}
Tornar público; divulgar.
\section{Propano}
\begin{itemize}
\item {Grp. gram.:m.}
\end{itemize}
\begin{itemize}
\item {Utilização:Chím.}
\end{itemize}
Um dos carbonetos do grupo formênico.
\section{Proparoxítono}
\begin{itemize}
\item {fónica:csi}
\end{itemize}
\begin{itemize}
\item {Grp. gram.:adj.}
\end{itemize}
\begin{itemize}
\item {Utilização:Gram.}
\end{itemize}
\begin{itemize}
\item {Grp. gram.:M.}
\end{itemize}
\begin{itemize}
\item {Proveniência:(Gr. \textunderscore proparoxutonos\textunderscore )}
\end{itemize}
Diz-se das palavras, que tem o acento predominante na antepenúltima sílaba.
Palavra proparoxítona.
\section{Proparoxýtono}
\begin{itemize}
\item {fónica:csi}
\end{itemize}
\begin{itemize}
\item {Grp. gram.:adj.}
\end{itemize}
\begin{itemize}
\item {Utilização:Gram.}
\end{itemize}
\begin{itemize}
\item {Grp. gram.:M.}
\end{itemize}
\begin{itemize}
\item {Proveniência:(Gr. \textunderscore proparoxutonos\textunderscore )}
\end{itemize}
Diz-se das palavras, que tem o accento predominante na antepenúltima sýllaba.
Palavra proparoxýtona.
\section{Propathia}
\begin{itemize}
\item {Grp. gram.:f.}
\end{itemize}
\begin{itemize}
\item {Utilização:Med.}
\end{itemize}
\begin{itemize}
\item {Proveniência:(Do gr. \textunderscore pros\textunderscore  + \textunderscore pathos\textunderscore )}
\end{itemize}
O mesmo que \textunderscore pródromo\textunderscore  ou preliminares de uma doença.
\section{Propatia}
\begin{itemize}
\item {Grp. gram.:f.}
\end{itemize}
\begin{itemize}
\item {Utilização:Med.}
\end{itemize}
\begin{itemize}
\item {Proveniência:(Do gr. \textunderscore pros\textunderscore  + \textunderscore pathos\textunderscore )}
\end{itemize}
O mesmo que \textunderscore pródromo\textunderscore  ou preliminares de uma doença.
\section{Propedêutica}
\begin{itemize}
\item {Grp. gram.:f.}
\end{itemize}
\begin{itemize}
\item {Proveniência:(De \textunderscore propedêutico\textunderscore )}
\end{itemize}
Introducção ou prolegómenos de uma sciência.
\section{Propedêutico}
\begin{itemize}
\item {Grp. gram.:adj.}
\end{itemize}
\begin{itemize}
\item {Proveniência:(Do gr. \textunderscore propaideuein\textunderscore )}
\end{itemize}
Que serve de introducção; preliminar.
Que prepara ou habilita para receber ensino mais completo: \textunderscore a«Clínica Propedêutica», obra do Dr. Francisco de Castro...\textunderscore 
\section{Propelir}
\begin{itemize}
\item {Grp. gram.:v. t.}
\end{itemize}
\begin{itemize}
\item {Utilização:Neol.}
\end{itemize}
\begin{itemize}
\item {Proveniência:(Lat. \textunderscore propellere\textunderscore )}
\end{itemize}
Impelir para deante; arremessar. Cf. Alv. Mendes, \textunderscore Discursos\textunderscore , 40.
\section{Propellir}
\begin{itemize}
\item {Grp. gram.:v. t.}
\end{itemize}
\begin{itemize}
\item {Utilização:Neol.}
\end{itemize}
\begin{itemize}
\item {Proveniência:(Lat. \textunderscore propellere\textunderscore )}
\end{itemize}
Impellir para deante; arremessar. Cf. Alv. Mendes, \textunderscore Discursos\textunderscore , 40.
\section{Propém}
\begin{itemize}
\item {Grp. gram.:m.}
\end{itemize}
\begin{itemize}
\item {Utilização:Ant.}
\end{itemize}
O mesmo que \textunderscore perpoém\textunderscore .
\section{Propendente}
\begin{itemize}
\item {Grp. gram.:adj.}
\end{itemize}
Que propende.
\section{Propender}
\begin{itemize}
\item {Grp. gram.:v. i.}
\end{itemize}
\begin{itemize}
\item {Grp. gram.:V. t.}
\end{itemize}
\begin{itemize}
\item {Utilização:Ant.}
\end{itemize}
\begin{itemize}
\item {Proveniência:(Lat. \textunderscore propendere\textunderscore )}
\end{itemize}
Tender, inclinar-se.
Pender.
Inclinar.
\section{Propensão}
\begin{itemize}
\item {Grp. gram.:f.}
\end{itemize}
\begin{itemize}
\item {Proveniência:(Lat. \textunderscore propensio\textunderscore )}
\end{itemize}
Acto ou effeito de propender; tendência; vocação.
\section{Propenso}
\begin{itemize}
\item {Grp. gram.:adj.}
\end{itemize}
\begin{itemize}
\item {Proveniência:(Lat. \textunderscore propensus\textunderscore )}
\end{itemize}
Favorável; benévolo.
Inclinado, tendente: \textunderscore o homemzinho é propenso á luxúria\textunderscore .
\section{Prophalange}
\begin{itemize}
\item {Grp. gram.:f.}
\end{itemize}
\begin{itemize}
\item {Utilização:Anat.}
\end{itemize}
\begin{itemize}
\item {Proveniência:(De \textunderscore pro...\textunderscore  + \textunderscore phalange\textunderscore )}
\end{itemize}
Peça proximal do dedo ou phalange propriamente dita.
\section{Prophalangeal}
\begin{itemize}
\item {Grp. gram.:adj.}
\end{itemize}
Relativo á prophalange.
\section{Prophecia}
\begin{itemize}
\item {Grp. gram.:f.}
\end{itemize}
\begin{itemize}
\item {Utilização:Fig.}
\end{itemize}
\begin{itemize}
\item {Proveniência:(Gr. \textunderscore propheteia\textunderscore )}
\end{itemize}
Acto de predizer o futuro; vaticínio; oráculo.
Preságio; conjectura.
\section{Propheta}
\begin{itemize}
\item {Grp. gram.:m.}
\end{itemize}
\begin{itemize}
\item {Proveniência:(Gr. \textunderscore prophetes\textunderscore )}
\end{itemize}
Aquelle que, entre os Hebreus, predizia o futuro por inspiração divina.
Vidente; adivinho.
Título, que os muçulmanos dão a Mafoma.
Aquelle que faz conjecturas sôbre o futuro.
\section{Prophetante}
\begin{itemize}
\item {Grp. gram.:adj.}
\end{itemize}
\begin{itemize}
\item {Proveniência:(De \textunderscore prophetar\textunderscore )}
\end{itemize}
Que propheta ou prophetiza. Cf. Macedo, \textunderscore Burros\textunderscore , 8.
\section{Prophetar}
\begin{itemize}
\item {Grp. gram.:v. t.}
\end{itemize}
\begin{itemize}
\item {Proveniência:(Lat. \textunderscore prophetare\textunderscore )}
\end{itemize}
(V.prophetizar)
\section{Propheticamente}
\begin{itemize}
\item {fónica:fé}
\end{itemize}
\begin{itemize}
\item {Grp. gram.:adv.}
\end{itemize}
De modo prophético; á maneira de propheta.
\section{Prophético}
\begin{itemize}
\item {Grp. gram.:adj.}
\end{itemize}
\begin{itemize}
\item {Proveniência:(Lat. \textunderscore propheticus\textunderscore )}
\end{itemize}
Relativo a propheta ou a prophecia.
\section{Prophetisa}
\begin{itemize}
\item {Grp. gram.:f.}
\end{itemize}
\begin{itemize}
\item {Proveniência:(Lat. \textunderscore prophetissa\textunderscore )}
\end{itemize}
Mulhér, que faz prophecias.
\section{Propolisação}
\begin{itemize}
\item {Grp. gram.:f.}
\end{itemize}
Acto de propolisar.
\section{Propolisar}
\begin{itemize}
\item {Grp. gram.:v.}
\end{itemize}
\begin{itemize}
\item {Utilização:t. Apicult.}
\end{itemize}
Tapar as fendas de um cortiço com própolis.
\section{Proponente}
\begin{itemize}
\item {Grp. gram.:m. ,  f.  e  adj.}
\end{itemize}
\begin{itemize}
\item {Proveniência:(Lat. \textunderscore proponens\textunderscore )}
\end{itemize}
Pessôa, que propõe.
\section{Propor}
\begin{itemize}
\item {Grp. gram.:v. t.}
\end{itemize}
\begin{itemize}
\item {Grp. gram.:V. p.}
\end{itemize}
\begin{itemize}
\item {Proveniência:(Do lat. \textunderscore proponere\textunderscore )}
\end{itemize}
Apresentar; expor á apreciação: \textunderscore propor soluções\textunderscore .
Referir.
Alvitrar: \textunderscore propor uma viagem\textunderscore .
Prometer.
Determinar.
Tencionar.
Aventurar-se.
Seguir como regra.
Offerecer-se.
Apresentar-se como candidato ou pretendente.
Destinar-se.
\section{Proporção}
\begin{itemize}
\item {Grp. gram.:f.}
\end{itemize}
\begin{itemize}
\item {Utilização:Fig.}
\end{itemize}
\begin{itemize}
\item {Grp. gram.:Pl.}
\end{itemize}
\begin{itemize}
\item {Proveniência:(Lat. \textunderscore proportio\textunderscore )}
\end{itemize}
Relação de uma coisa com outra ou com um todo.
Comparação.
Dimensão.
Igualdade mathemática entre duas ou mais razões.
Relação chímica entre quantidades.
Conformidade: \textunderscore trajar, em proporção da idade\textunderscore .
Disposição regular.
Intensidade; importância: \textunderscore o escândalo tomou grandes proporções\textunderscore .
\section{Proporcionadamente}
\begin{itemize}
\item {Grp. gram.:adv.}
\end{itemize}
De modo proporcionado.
Com proporção; proporcionalmente.
\section{Proporcionado}
\begin{itemize}
\item {Grp. gram.:adj.}
\end{itemize}
\begin{itemize}
\item {Proveniência:(De \textunderscore proporção\textunderscore )}
\end{itemize}
Disposto regularmente; bem conformado; harmónico.
\section{Proporcionador}
\begin{itemize}
\item {Grp. gram.:m.  e  adj.}
\end{itemize}
O que proporciona.
\section{Proporcional}
\begin{itemize}
\item {Grp. gram.:adj.}
\end{itemize}
\begin{itemize}
\item {Proveniência:(Lat. \textunderscore proportionalis\textunderscore )}
\end{itemize}
Proporcionado.
Relativo a proporção mathemática.
\section{Proporcionalidade}
\begin{itemize}
\item {Grp. gram.:f.}
\end{itemize}
\begin{itemize}
\item {Proveniência:(Lat. \textunderscore proportionalitas\textunderscore )}
\end{itemize}
Qualidade do que é proporcional.
\section{Proporcionalizar}
\begin{itemize}
\item {Grp. gram.:v. t.}
\end{itemize}
Tornar proporcional. Cf. A. Candido, \textunderscore Philos. Polit.\textunderscore , 138, 177 e 180.
\section{Proporcionalmente}
\begin{itemize}
\item {Grp. gram.:adv.}
\end{itemize}
De modo proporcional; com proporção.
\section{Proporcionar}
\begin{itemize}
\item {Grp. gram.:v. t.}
\end{itemize}
\begin{itemize}
\item {Proveniência:(Lat. \textunderscore proportionare\textunderscore )}
\end{itemize}
Observar proporção entre.
Tornar proporcional; acommodar.
Dar ensejo a: \textunderscore proporcionar desgostos\textunderscore .
Prestar: \textunderscore proporcionar recursos\textunderscore .
\section{Proporcionável}
\begin{itemize}
\item {Grp. gram.:adj.}
\end{itemize}
Que se póde proporcionar.
\section{Proposição}
\begin{itemize}
\item {Grp. gram.:f.}
\end{itemize}
\begin{itemize}
\item {Proveniência:(Lat. \textunderscore propositio\textunderscore )}
\end{itemize}
Acto ou effeito de propor.
Aquillo que se propõe, para chegar a uma conclusão ou combinação.
Proposta, offerecimento.
Expressão verbal de um juizo.
Asserção; theorema; máxima; problema.
\section{Propositadamente}
\begin{itemize}
\item {Grp. gram.:adv.}
\end{itemize}
De modo propositado; de propósito.
\section{Propositado}
\begin{itemize}
\item {Grp. gram.:adj.}
\end{itemize}
Em que há propósito, intenção ou resolução prévia: \textunderscore offensa propositada\textunderscore .
\section{Propósito}
\begin{itemize}
\item {Grp. gram.:m.}
\end{itemize}
\begin{itemize}
\item {Grp. gram.:Loc. adv.}
\end{itemize}
\begin{itemize}
\item {Grp. gram.:Loc. adv.}
\end{itemize}
\begin{itemize}
\item {Proveniência:(Lat. \textunderscore propositus\textunderscore )}
\end{itemize}
Deliberação; intenção, projecto.
Bom senso; prudência: \textunderscore menina, tenha propósito\textunderscore .
Assumpto.
Relação.
\textunderscore A propósito\textunderscore , a respeito; opportunamente; na occasião própria; a tempo: \textunderscore chegou a propósito\textunderscore .
\textunderscore De propósito\textunderscore , por querer, com intenção, com premeditação, adrede.
\section{Propositura}
\begin{itemize}
\item {Grp. gram.:f.}
\end{itemize}
\begin{itemize}
\item {Utilização:P. us.}
\end{itemize}
\begin{itemize}
\item {Proveniência:(Do lat. \textunderscore propositus\textunderscore )}
\end{itemize}
Acto ou effeito de propor, (tratando-se de acções judiciaes): \textunderscore a propositura de uma demanda\textunderscore .
\section{Proposta}
\begin{itemize}
\item {Grp. gram.:f.}
\end{itemize}
Proposição.
Moção.
Promessa; offerta.
Determinação.
(Fem. de \textunderscore proposto\textunderscore )
\section{Proposto}
\begin{itemize}
\item {Grp. gram.:m.}
\end{itemize}
\begin{itemize}
\item {Proveniência:(Lat. \textunderscore propositus\textunderscore )}
\end{itemize}
Aquillo que se propôs.
Indivíduo escolhido por outro para exercer as suas funcções.
\section{Propriador}
\begin{itemize}
\item {Grp. gram.:m.}
\end{itemize}
\begin{itemize}
\item {Proveniência:(De \textunderscore apropriar\textunderscore )}
\end{itemize}
Aquelle que trabalha em propriagem.
\section{Propriagem}
\begin{itemize}
\item {Grp. gram.:f.}
\end{itemize}
\begin{itemize}
\item {Proveniência:(De \textunderscore apropriar\textunderscore )}
\end{itemize}
Trabalho, que os chapeleiros executam nos chapéus, depois de tintos.
\section{Propriamente}
\begin{itemize}
\item {Grp. gram.:adv.}
\end{itemize}
De modo próprio; com propriedade.
Exactamente.
Especialmente.
Pessoalmente.
\section{Proquestura}
\begin{itemize}
\item {Grp. gram.:f.}
\end{itemize}
Cargo ou dignidade de proquestor.
\section{Próroga}
\begin{itemize}
\item {fónica:ro}
\end{itemize}
\begin{itemize}
\item {Grp. gram.:f.}
\end{itemize}
O mesmo que \textunderscore prorogação\textunderscore .
\section{Prorogação}
\begin{itemize}
\item {fónica:ro}
\end{itemize}
\begin{itemize}
\item {Grp. gram.:f.}
\end{itemize}
\begin{itemize}
\item {Proveniência:(Lat. \textunderscore prorogatio\textunderscore )}
\end{itemize}
Acto ou effeito de prorogar.
\section{Prorogar}
\begin{itemize}
\item {fónica:ro}
\end{itemize}
\begin{itemize}
\item {Grp. gram.:v. t.}
\end{itemize}
\begin{itemize}
\item {Proveniência:(Lat. \textunderscore prorogare\textunderscore )}
\end{itemize}
Protrahir; tornar mais longo (um prazo estabelecido).
\section{Prorogativo}
\begin{itemize}
\item {fónica:ro}
\end{itemize}
\begin{itemize}
\item {Grp. gram.:adj.}
\end{itemize}
\begin{itemize}
\item {Proveniência:(Lat. \textunderscore prorogativus\textunderscore )}
\end{itemize}
Que proroga.
\section{Prorogável}
\begin{itemize}
\item {fónica:ro}
\end{itemize}
\begin{itemize}
\item {Grp. gram.:adj.}
\end{itemize}
Que se póde prorogar.
\section{Proromper}
\begin{itemize}
\item {fónica:ro}
\end{itemize}
\begin{itemize}
\item {Grp. gram.:v. i.}
\end{itemize}
\begin{itemize}
\item {Proveniência:(Lat. \textunderscore prorumpere\textunderscore )}
\end{itemize}
Irromper ou saír impetuosamente.
Manifestar-se de repente.
\section{Prorompimento}
\begin{itemize}
\item {fónica:rom}
\end{itemize}
\begin{itemize}
\item {Grp. gram.:m.}
\end{itemize}
Acto de proromper.
\section{Prórroga}
\begin{itemize}
\item {Grp. gram.:f.}
\end{itemize}
O mesmo que \textunderscore prorrogação\textunderscore .
\section{Prorrogação}
\begin{itemize}
\item {Grp. gram.:f.}
\end{itemize}
\begin{itemize}
\item {Proveniência:(Lat. \textunderscore prorogatio\textunderscore )}
\end{itemize}
Acto ou effeito de prorrogar.
\section{Prorrogar}
\begin{itemize}
\item {Grp. gram.:v. t.}
\end{itemize}
\begin{itemize}
\item {Proveniência:(Lat. \textunderscore prorogare\textunderscore )}
\end{itemize}
Protrahir; tornar mais longo (um prazo estabelecido).
\section{Prorrogativo}
\begin{itemize}
\item {Grp. gram.:adj.}
\end{itemize}
\begin{itemize}
\item {Proveniência:(Lat. \textunderscore prorogativus\textunderscore )}
\end{itemize}
Que prorroga.
\section{Prorrogável}
\begin{itemize}
\item {Grp. gram.:adj.}
\end{itemize}
Que se póde prorrogar.
\section{Prorromper}
\begin{itemize}
\item {Grp. gram.:v. i.}
\end{itemize}
\begin{itemize}
\item {Proveniência:(Lat. \textunderscore prorumpere\textunderscore )}
\end{itemize}
Irromper ou saír impetuosamente.
Manifestar-se de repente.
\section{Prorrompimento}
\begin{itemize}
\item {Grp. gram.:m.}
\end{itemize}
Acto de prorromper.
\section{Prosa}
\begin{itemize}
\item {Grp. gram.:f.}
\end{itemize}
\begin{itemize}
\item {Utilização:Fig.}
\end{itemize}
\begin{itemize}
\item {Utilização:Fam.}
\end{itemize}
\begin{itemize}
\item {Utilização:Bras. do N}
\end{itemize}
\begin{itemize}
\item {Utilização:Prov.}
\end{itemize}
\begin{itemize}
\item {Proveniência:(Lat. \textunderscore prosa\textunderscore )}
\end{itemize}
Fórma de falar ou de escrever, mais ou menos natural, sem sujeição a medida certa ou a certo número de sýllabas ou de pés.
Aquillo que é ordinário, trivial, positivo ou material: \textunderscore as prosas da vida\textunderscore .
Lábia.
Namôro.
Bazófia.
Indivíduo pedante.
\section{Prosador}
\begin{itemize}
\item {Grp. gram.:m.}
\end{itemize}
\begin{itemize}
\item {Proveniência:(De \textunderscore prosar\textunderscore )}
\end{itemize}
Aquelle que escreve em prosa.
Escritor, que faz bôa prosa.
\section{Prosaicamente}
\begin{itemize}
\item {Grp. gram.:adv.}
\end{itemize}
De modo prosaico.
De modo ordinário ou vulgar; sem elevação de ideias.
\section{Prosaico}
\begin{itemize}
\item {Grp. gram.:adj.}
\end{itemize}
\begin{itemize}
\item {Proveniência:(Lat. \textunderscore prosaicus\textunderscore )}
\end{itemize}
Relativo á prosa ou que tem a natureza della.
Trivial, vulgar.
Material, positivo.
Que não tem elevação, que não é sublime.
\section{Prosaísmo}
\begin{itemize}
\item {Grp. gram.:m.}
\end{itemize}
\begin{itemize}
\item {Proveniência:(De \textunderscore prosa\textunderscore )}
\end{itemize}
Qualidade daquillo que é prosaico.
\section{Prosaísta}
\begin{itemize}
\item {Grp. gram.:m.}
\end{itemize}
\begin{itemize}
\item {Utilização:bras}
\end{itemize}
\begin{itemize}
\item {Utilização:Neol.}
\end{itemize}
O mesmo que \textunderscore prosador\textunderscore . Cf. S. Romero, \textunderscore Assis\textunderscore , 71.
\section{Prosápia}
\begin{itemize}
\item {Grp. gram.:f.}
\end{itemize}
\begin{itemize}
\item {Proveniência:(Lat. \textunderscore prosapia\textunderscore )}
\end{itemize}
Progênie; ascendência; raça.
Jactância, orgulho; bazófia.
\section{Prosar}
\begin{itemize}
\item {Grp. gram.:v. i.}
\end{itemize}
Escrever em prosa.
\section{Proscênio}
\begin{itemize}
\item {Grp. gram.:m.}
\end{itemize}
\begin{itemize}
\item {Utilização:Ext.}
\end{itemize}
\begin{itemize}
\item {Proveniência:(Lat. \textunderscore proscenium\textunderscore )}
\end{itemize}
Frente do palco.
Parte do palco, junto á ribalta.
Scena, palco.
\section{Proscollo}
\begin{itemize}
\item {Grp. gram.:m.}
\end{itemize}
\begin{itemize}
\item {Proveniência:(Do gr. \textunderscore pros\textunderscore  + \textunderscore kolla\textunderscore )}
\end{itemize}
Tubérculo granuloso das orchídeas, que segrega um humor viscoso, a que adherem os grãos do póllen no acto da fecundação.
\section{Proscolo}
\begin{itemize}
\item {Grp. gram.:m.}
\end{itemize}
\begin{itemize}
\item {Proveniência:(Do gr. \textunderscore pros\textunderscore  + \textunderscore kolla\textunderscore )}
\end{itemize}
Tubérculo granuloso das orquídeas, que segrega um humor viscoso, a que aderem os grãos do pólen no acto da fecundação.
\section{Proscrever}
\begin{itemize}
\item {Grp. gram.:v. t.}
\end{itemize}
\begin{itemize}
\item {Proveniência:(Lat. \textunderscore proscribere\textunderscore )}
\end{itemize}
Condemnar a degrêdo, por sentença ou voto escrito.
Desterrar.
Expulsar.
Extinguir.
Abolir; prohibir.
Afastar; terminar.
\section{Proscrição}
\begin{itemize}
\item {Grp. gram.:f.}
\end{itemize}
\begin{itemize}
\item {Proveniência:(Lat. \textunderscore proscriptio\textunderscore )}
\end{itemize}
Acto ou efeito de proscrever.
Destêrro.
\section{Proscripção}
\begin{itemize}
\item {Grp. gram.:f.}
\end{itemize}
\begin{itemize}
\item {Proveniência:(Lat. \textunderscore proscriptio\textunderscore )}
\end{itemize}
Acto ou effeito de proscrever.
Destêrro.
\section{Proscrito}
\begin{itemize}
\item {Grp. gram.:m.}
\end{itemize}
\begin{itemize}
\item {Proveniência:(Lat. \textunderscore proscriptus\textunderscore )}
\end{itemize}
Aquelle que foi desterrado; emigrado.
\section{Proscritor}
\begin{itemize}
\item {Grp. gram.:m.  e  adj.}
\end{itemize}
\begin{itemize}
\item {Proveniência:(Lat. \textunderscore proscriptor\textunderscore )}
\end{itemize}
O que proscreve.
\section{Prosear}
\begin{itemize}
\item {Grp. gram.:v. i.}
\end{itemize}
\begin{itemize}
\item {Utilização:Bras}
\end{itemize}
\begin{itemize}
\item {Utilização:Bras. do N}
\end{itemize}
\begin{itemize}
\item {Proveniência:(De \textunderscore prosa\textunderscore )}
\end{itemize}
Conversar, dar á língua.
Namorar.
\section{Prosecução}
\begin{itemize}
\item {fónica:se}
\end{itemize}
\begin{itemize}
\item {Grp. gram.:f.}
\end{itemize}
\begin{itemize}
\item {Proveniência:(Lat. \textunderscore prosecutio\textunderscore )}
\end{itemize}
Acto ou effeito de proseguir.
\section{Proseguição}
\begin{itemize}
\item {fónica:se}
\end{itemize}
\begin{itemize}
\item {Grp. gram.:f.}
\end{itemize}
O mesmo que \textunderscore prosecução\textunderscore .
\section{Proseguidor}
\begin{itemize}
\item {fónica:se}
\end{itemize}
\begin{itemize}
\item {Grp. gram.:m.  e  adj.}
\end{itemize}
O que prosegue.
\section{Proseguimento}
\begin{itemize}
\item {fónica:se}
\end{itemize}
\begin{itemize}
\item {Grp. gram.:m.}
\end{itemize}
O mesmo que \textunderscore prosecução\textunderscore .
\section{Proseguir}
\begin{itemize}
\item {fónica:se}
\end{itemize}
\begin{itemize}
\item {Grp. gram.:v. t.}
\end{itemize}
\begin{itemize}
\item {Grp. gram.:V. i.}
\end{itemize}
\begin{itemize}
\item {Proveniência:(Do lat. \textunderscore prosequi\textunderscore )}
\end{itemize}
Fazer seguir; continuar: \textunderscore proseguir um discurso\textunderscore .
Dizer depois ou em seguida.
Ir por deante.
Continuar a falar, a proceder, etc.
\section{Prospectivo}
\begin{itemize}
\item {Grp. gram.:adj.}
\end{itemize}
\begin{itemize}
\item {Proveniência:(Lat. \textunderscore prospectivus\textunderscore )}
\end{itemize}
Que faz vêr adeante ou ao longe.
\section{Prospecto}
\begin{itemize}
\item {Grp. gram.:m.}
\end{itemize}
\begin{itemize}
\item {Proveniência:(Lat. \textunderscore prospectus\textunderscore )}
\end{itemize}
Acto de vêr de frente.
Aspecto.
Plano de uma obra, que ainda não está publicada, mas que o há de sêr.
Indicação impressa das particularidades ou condições de uma empresa ou negócio.
Programma.
\section{Prospector}
\begin{itemize}
\item {Grp. gram.:m.}
\end{itemize}
\begin{itemize}
\item {Utilização:T. de mineiros}
\end{itemize}
\begin{itemize}
\item {Proveniência:(Lat. \textunderscore prospector\textunderscore )}
\end{itemize}
Aquelle que conhece e indica os terrenos metallíferos.
\section{Prosperamente}
\begin{itemize}
\item {Grp. gram.:adv.}
\end{itemize}
De modo próspero.
Com prosperidade; com êxito.
\section{Prosperar}
\begin{itemize}
\item {Grp. gram.:v. t.}
\end{itemize}
\begin{itemize}
\item {Grp. gram.:V. t.}
\end{itemize}
\begin{itemize}
\item {Proveniência:(Lat. \textunderscore prosperare\textunderscore )}
\end{itemize}
Têr fortuna favorável.
Tornar-se próspero.
Enriquecer.
Melhorar; aumentar; desenvolver-se.
Mostrar-se propício.
Tornar próspero.
Tornar efficaz:«\textunderscore ...a graça prospera seus bons desejos.\textunderscore »\textunderscore Luz e Calor\textunderscore .
\section{Prosperidade}
\begin{itemize}
\item {Grp. gram.:f.}
\end{itemize}
\begin{itemize}
\item {Proveniência:(Lat. \textunderscore prosperitas\textunderscore )}
\end{itemize}
Qualidade ou estado do que é próspero; felicidade; situação próspera.
\section{Prosperina}
\begin{itemize}
\item {Grp. gram.:f.}
\end{itemize}
Espécie de droga, que se mistura no café, para lhe dar melhor sabor.
\section{Prosperna}
\begin{itemize}
\item {Grp. gram.:f.}
\end{itemize}
(Cor. ant. de \textunderscore posperna\textunderscore ). Cf. Pero Lopes, \textunderscore Diário da Naveg.\textunderscore 
\section{Próspero}
\begin{itemize}
\item {Grp. gram.:adj.}
\end{itemize}
\begin{itemize}
\item {Grp. gram.:M.}
\end{itemize}
\begin{itemize}
\item {Proveniência:(Lat. \textunderscore prosperus\textunderscore )}
\end{itemize}
Propício.
Venturoso; afortunado; que tem bom exito.
O mesmo que \textunderscore prosperidade\textunderscore :«\textunderscore ...a moderação no próspero, a serenidade no adverso.\textunderscore »Vieira.
\section{Prospérrimo}
\begin{itemize}
\item {Grp. gram.:adj.}
\end{itemize}
Muito próspero. Cf. Viriato, \textunderscore Trág.\textunderscore , VIII, 122.
\section{Prossecução}
\begin{itemize}
\item {Grp. gram.:f.}
\end{itemize}
\begin{itemize}
\item {Proveniência:(Lat. \textunderscore prosecutio\textunderscore )}
\end{itemize}
Acto ou efeito de prosseguir.
\section{Prosseguição}
\begin{itemize}
\item {Grp. gram.:f.}
\end{itemize}
O mesmo que \textunderscore prossecução\textunderscore .
\section{Prosseguidor}
\begin{itemize}
\item {Grp. gram.:m.  e  adj.}
\end{itemize}
O que prossegue.
\section{Prosseguimento}
\begin{itemize}
\item {Grp. gram.:m.}
\end{itemize}
O mesmo que \textunderscore prossecução\textunderscore .
\section{Prosseguir}
\begin{itemize}
\item {Grp. gram.:v. t.}
\end{itemize}
\begin{itemize}
\item {Grp. gram.:V. i.}
\end{itemize}
\begin{itemize}
\item {Proveniência:(Do lat. \textunderscore prosequi\textunderscore )}
\end{itemize}
Fazer seguir; continuar: \textunderscore prosseguir um discurso\textunderscore .
Dizer depois ou em seguida.
Ir por deante.
Continuar a falar, a proceder, etc.
\section{Prostadena}
\begin{itemize}
\item {Grp. gram.:f.}
\end{itemize}
Producto pharmacêutico, preparado com próstata fresca.
\section{Prostantera}
\begin{itemize}
\item {Grp. gram.:f.}
\end{itemize}
Gênero de plantas labiadas.
\section{Prostanthera}
\begin{itemize}
\item {Grp. gram.:f.}
\end{itemize}
Gênero de plantas labiadas.
\section{Próstase}
\begin{itemize}
\item {Grp. gram.:f.}
\end{itemize}
\begin{itemize}
\item {Utilização:Med.}
\end{itemize}
\begin{itemize}
\item {Proveniência:(Do gr. \textunderscore pro\textunderscore  + \textunderscore stasis\textunderscore )}
\end{itemize}
Predomínio de um humor sôbre o outro.
\section{Próstata}
\begin{itemize}
\item {Grp. gram.:f.}
\end{itemize}
\begin{itemize}
\item {Proveniência:(Gr. \textunderscore prostates\textunderscore )}
\end{itemize}
Glândula, na parte inferior do collo da bexiga.
\section{Prostatalgia}
\begin{itemize}
\item {Grp. gram.:f.}
\end{itemize}
\begin{itemize}
\item {Proveniência:(Do gr. \textunderscore prostates\textunderscore  + \textunderscore algos\textunderscore )}
\end{itemize}
Dôr na próstata.
\section{Prostatectomia}
\begin{itemize}
\item {Grp. gram.:f.}
\end{itemize}
\begin{itemize}
\item {Utilização:Med.}
\end{itemize}
Ablação da próstata.
\section{Prostático}
\begin{itemize}
\item {Grp. gram.:adj.}
\end{itemize}
Relativo á próstata.
\section{Prostatite}
\begin{itemize}
\item {Grp. gram.:f.}
\end{itemize}
Inflamação da próstata.
\section{Prostatólitho}
\begin{itemize}
\item {Grp. gram.:m.}
\end{itemize}
\begin{itemize}
\item {Utilização:Med.}
\end{itemize}
\begin{itemize}
\item {Proveniência:(Do gr. \textunderscore prostates\textunderscore  + \textunderscore lithos\textunderscore )}
\end{itemize}
Cálculo na próstata.
\section{Prostatólito}
\begin{itemize}
\item {Grp. gram.:m.}
\end{itemize}
\begin{itemize}
\item {Utilização:Med.}
\end{itemize}
\begin{itemize}
\item {Proveniência:(Do gr. \textunderscore prostates\textunderscore  + \textunderscore lithos\textunderscore )}
\end{itemize}
Cálculo na próstata.
\section{Prostatorreia}
\begin{itemize}
\item {Grp. gram.:f.}
\end{itemize}
\begin{itemize}
\item {Utilização:Med.}
\end{itemize}
\begin{itemize}
\item {Proveniência:(Do gr. \textunderscore prostates\textunderscore  + \textunderscore rhein\textunderscore )}
\end{itemize}
Derramamento mórbido, proveniente da próstata.
\section{Prostatorrhéa}
\begin{itemize}
\item {Grp. gram.:f.}
\end{itemize}
\begin{itemize}
\item {Utilização:Med.}
\end{itemize}
\begin{itemize}
\item {Proveniência:(Do gr. \textunderscore prostates\textunderscore  + \textunderscore rhein\textunderscore )}
\end{itemize}
Derramamento mórbido, proveniente da próstata.
\section{Prostatorrheia}
\begin{itemize}
\item {Grp. gram.:f.}
\end{itemize}
\begin{itemize}
\item {Utilização:Med.}
\end{itemize}
\begin{itemize}
\item {Proveniência:(Do gr. \textunderscore prostates\textunderscore  + \textunderscore rhein\textunderscore )}
\end{itemize}
Derramamento mórbido, proveniente da próstata.
\section{Prostatotomia}
\begin{itemize}
\item {Grp. gram.:f.}
\end{itemize}
\begin{itemize}
\item {Proveniência:(Do gr. \textunderscore prostates\textunderscore  + \textunderscore tome\textunderscore )}
\end{itemize}
Qualquer das várias operações cirúrgicas, que se podem realizar na próstata.
\section{Prosternação}
\begin{itemize}
\item {Grp. gram.:f.}
\end{itemize}
Acto ou effeito de prosternar.
\section{Prosternamento}
\begin{itemize}
\item {Grp. gram.:m.}
\end{itemize}
O mesmo que \textunderscore prosternação\textunderscore .
\section{Prosternar}
\begin{itemize}
\item {Grp. gram.:v. t.}
\end{itemize}
\begin{itemize}
\item {Proveniência:(Lat. \textunderscore prosternere\textunderscore )}
\end{itemize}
Prostrar; deitar por terra, em sinal de respeito ou admiração.
Humilhar.
\section{Prosterno}
\begin{itemize}
\item {Grp. gram.:m.}
\end{itemize}
\begin{itemize}
\item {Utilização:Anat.}
\end{itemize}
\begin{itemize}
\item {Proveniência:(De \textunderscore pro...\textunderscore  + \textunderscore esterno\textunderscore )}
\end{itemize}
Parte superior do esterno, á qual os antigos chamavam \textunderscore punho\textunderscore , pela semelhança do esterno com uma espada.
\section{Próstese}
\begin{itemize}
\item {Grp. gram.:f.}
\end{itemize}
O mesmo ou melhor que \textunderscore prótese\textunderscore .
\section{Prostético}
\begin{itemize}
\item {Grp. gram.:adj.}
\end{itemize}
O mesmo que \textunderscore protético\textunderscore .
\section{Prósthese}
\begin{itemize}
\item {Grp. gram.:f.}
\end{itemize}
O mesmo ou melhor que \textunderscore próthese\textunderscore .
\section{Prosthético}
\begin{itemize}
\item {Grp. gram.:adj.}
\end{itemize}
O mesmo que \textunderscore prothético\textunderscore .
\section{Prostibular}
\begin{itemize}
\item {Grp. gram.:adj.}
\end{itemize}
Relativo a prostíbulo, próprio de prostíbulo.
\section{Proteccionista}
\begin{itemize}
\item {Grp. gram.:m.  e  f.}
\end{itemize}
\begin{itemize}
\item {Grp. gram.:Adj.}
\end{itemize}
\begin{itemize}
\item {Proveniência:(Do lat. \textunderscore protectio\textunderscore )}
\end{itemize}
Pessôa, partidária do proteccionismo.
Proteccional.
\section{Protectivo}
\begin{itemize}
\item {Grp. gram.:adj.}
\end{itemize}
\begin{itemize}
\item {Proveniência:(Do lat. \textunderscore protectus\textunderscore )}
\end{itemize}
Próprio para proteger, cobrir ou resguardar: \textunderscore os invólucros protectivos das tartarugas\textunderscore .
\section{Protector}
\begin{itemize}
\item {Grp. gram.:m.  e  adj.}
\end{itemize}
\begin{itemize}
\item {Grp. gram.:M.}
\end{itemize}
\begin{itemize}
\item {Utilização:Neol.}
\end{itemize}
\begin{itemize}
\item {Utilização:Agr.}
\end{itemize}
\begin{itemize}
\item {Proveniência:(Lat. \textunderscore protector\textunderscore )}
\end{itemize}
O que protege.
O que tem a seu cuidado os interesses de outrem.
Cardeal, encarregado em Roma dos negócios consistoriaes de certas nações ou de certas Ordens religiosas.
Revestimento metállico, na superfície exterior de um navio.
Revestimento de ferro para a sola do calçado.
Vara ou estaca, a que se ata ou em que se apoia uma videira, uma roseira, etc.
\section{Protectorado}
\begin{itemize}
\item {Grp. gram.:m.}
\end{itemize}
\begin{itemize}
\item {Proveniência:(De \textunderscore protector\textunderscore )}
\end{itemize}
Apoio, dado por uma nação a outra menos poderosa.
\section{Protectoral}
\begin{itemize}
\item {Grp. gram.:adj.}
\end{itemize}
\begin{itemize}
\item {Proveniência:(De \textunderscore protector\textunderscore )}
\end{itemize}
Relativo ao protectorado.
\section{Protectorato}
\begin{itemize}
\item {Grp. gram.:m.}
\end{itemize}
O mesmo que \textunderscore protectorado\textunderscore . Cf. Garrett, \textunderscore Port. na Balança\textunderscore , 52.
\section{Protectório}
\begin{itemize}
\item {Grp. gram.:adj.}
\end{itemize}
\begin{itemize}
\item {Proveniência:(Lat. \textunderscore protectorius\textunderscore )}
\end{itemize}
Relativo a protector.
Que protege.
\section{Protegedor}
\begin{itemize}
\item {Grp. gram.:m.  e  adj.}
\end{itemize}
O mesmo que \textunderscore protector\textunderscore .
\section{Proteger}
\begin{itemize}
\item {Grp. gram.:v. t.}
\end{itemize}
\begin{itemize}
\item {Proveniência:(Lat. \textunderscore protegere\textunderscore )}
\end{itemize}
Tomar a defesa de.
Apoiar, auxiliar, soccorrer.
Têr a seu cuidado os interesses de.
Tratar de manter ou desenvolver.
Abrigar, resguardar: \textunderscore o alpendre protege contra a chuva\textunderscore .
\section{Protegida}
\begin{itemize}
\item {Grp. gram.:f.}
\end{itemize}
\begin{itemize}
\item {Proveniência:(De \textunderscore protegido\textunderscore )}
\end{itemize}
Mulhér, que recebe protecção de alguém.
\section{Protegido}
\begin{itemize}
\item {Grp. gram.:m.}
\end{itemize}
\begin{itemize}
\item {Proveniência:(De \textunderscore proteger\textunderscore )}
\end{itemize}
Indivíduo, que recebe protecção especial de alguém; valido, favorito.
\section{Proteico}
\begin{itemize}
\item {Grp. gram.:adj.}
\end{itemize}
\begin{itemize}
\item {Proveniência:(De \textunderscore prótea\textunderscore )}
\end{itemize}
O mesmo que \textunderscore albuminóide\textunderscore .
\section{Proteiforme}
\begin{itemize}
\item {Grp. gram.:adj.}
\end{itemize}
\begin{itemize}
\item {Proveniência:(De \textunderscore próteo\textunderscore  + \textunderscore fórma\textunderscore )}
\end{itemize}
Que muda de fórma frequentemente.
\section{Proteína}
\begin{itemize}
\item {Grp. gram.:f.}
\end{itemize}
\begin{itemize}
\item {Utilização:Chím.}
\end{itemize}
\begin{itemize}
\item {Proveniência:(Do gr. \textunderscore protos\textunderscore )}
\end{itemize}
Substância, resultante da acção da potassa sôbre substâncias albuminóides.
\section{Protela}
\begin{itemize}
\item {Grp. gram.:f.}
\end{itemize}
Gênero de mammíferos carnívoros; o mesmo que \textunderscore protelo\textunderscore .
\section{Protelação}
\begin{itemize}
\item {Grp. gram.:f.}
\end{itemize}
\begin{itemize}
\item {Proveniência:(Lat. \textunderscore protelatio\textunderscore )}
\end{itemize}
Acto ou effeito de protelar.
\section{Protelar}
\begin{itemize}
\item {Grp. gram.:v. t.}
\end{itemize}
\begin{itemize}
\item {Proveniência:(Lat. \textunderscore protelare\textunderscore )}
\end{itemize}
Adiar, procrastinar; prorogar.
\section{Protelo}
\begin{itemize}
\item {Grp. gram.:m.}
\end{itemize}
Animal mammífero e carnívoro, que tem alguns pontos de semelhança com a hyena.
\section{Próteo}
\begin{itemize}
\item {Grp. gram.:m.}
\end{itemize}
\begin{itemize}
\item {Utilização:Zool.}
\end{itemize}
\begin{itemize}
\item {Utilização:Fig.}
\end{itemize}
\begin{itemize}
\item {Proveniência:(Do lat. \textunderscore Proteus\textunderscore , n. p.)}
\end{itemize}
Amphíbio caudato, espécie de salamandra.
Indivíduo, que muda facilmente de opinião ou systema.
\section{Proterânteo}
\begin{itemize}
\item {Grp. gram.:adj.}
\end{itemize}
\begin{itemize}
\item {Utilização:Bot.}
\end{itemize}
Diz-se das fôlhas, que nascem antes das flôres, como de ordinário succede.
\section{Proterântheo}
\begin{itemize}
\item {Grp. gram.:adj.}
\end{itemize}
\begin{itemize}
\item {Utilização:Bot.}
\end{itemize}
Diz-se das fôlhas, que nascem antes das flôres, como de ordinário succede.
\section{Protervamente}
\begin{itemize}
\item {Grp. gram.:adv.}
\end{itemize}
De modo protervo; brutalmente.
\section{Protérvia}
\begin{itemize}
\item {Grp. gram.:f.}
\end{itemize}
\begin{itemize}
\item {Proveniência:(Lat. \textunderscore protervia\textunderscore )}
\end{itemize}
Qualidade do que é protervo.
\section{Protervo}
\begin{itemize}
\item {Grp. gram.:adj.}
\end{itemize}
\begin{itemize}
\item {Proveniência:(Lat. \textunderscore protervus\textunderscore )}
\end{itemize}
Impudente; petulante; procaz.
Descarado.
Brutal.
\section{Protestação}
\begin{itemize}
\item {Grp. gram.:f.}
\end{itemize}
\begin{itemize}
\item {Proveniência:(Lat. \textunderscore protestatio\textunderscore )}
\end{itemize}
Acto ou effeito de protestar.
\section{Protestador}
\begin{itemize}
\item {Grp. gram.:m.  e  adj.}
\end{itemize}
O que protesta.
\section{Protestante}
\begin{itemize}
\item {Grp. gram.:adj.}
\end{itemize}
\begin{itemize}
\item {Grp. gram.:M.}
\end{itemize}
\begin{itemize}
\item {Proveniência:(Lat. \textunderscore protestans\textunderscore )}
\end{itemize}
Relativo á religião dos Protestantes.
Que protesta.
O que protesta.
Sectário do Protestantismo.
\section{Protestantismo}
\begin{itemize}
\item {Grp. gram.:m.}
\end{itemize}
\begin{itemize}
\item {Proveniência:(De \textunderscore protestante\textunderscore )}
\end{itemize}
Religião dos Lutheranos, calvinistas e anglicanos.
\section{Protocolista}
\begin{itemize}
\item {Grp. gram.:m.}
\end{itemize}
\begin{itemize}
\item {Utilização:Bras}
\end{itemize}
Empregado que, nas Repartições públicas, escritura o protocolo.
\section{Protocolizar}
\begin{itemize}
\item {Grp. gram.:v. t.}
\end{itemize}
\begin{itemize}
\item {Utilização:Neol.}
\end{itemize}
Dar feição de protocolo a:«\textunderscore o dono da estalagem e o meu criado vieram protocolizar a desordem.\textunderscore »Camillo, \textunderscore Filha do Arced.\textunderscore , c. XXVII.
\section{Protocollista}
\begin{itemize}
\item {Grp. gram.:m.}
\end{itemize}
\begin{itemize}
\item {Utilização:Bras}
\end{itemize}
Empregado que, nas Repartições públicas, escritura o protocollo.
\section{Protocollizar}
\begin{itemize}
\item {Grp. gram.:v. t.}
\end{itemize}
\begin{itemize}
\item {Utilização:Neol.}
\end{itemize}
Dar feição de protocollo a:«\textunderscore o dono da estalagem e o meu criado vieram protocollizar a desordem.\textunderscore »Camillo, \textunderscore Filha do Arced.\textunderscore , c. XXVII.
\section{Protocollo}
\begin{itemize}
\item {Grp. gram.:m.}
\end{itemize}
Registo dos actos públicos, na Idade-Média.
Sêllo, que os Romanos punham no papel em que se registavam actos públicos.
Registo das audiências nos tribunaes.
Convenção internacional.
Registo de uma conferência ou deliberação diplomática.
Formulário, que regula os actos públicos.
(B. lat. \textunderscore protocullum\textunderscore )
\section{Protocolo}
\begin{itemize}
\item {Grp. gram.:m.}
\end{itemize}
Registo dos actos públicos, na Idade-Média.
Sêlo, que os Romanos punham no papel em que se registavam actos públicos.
Registo das audiências nos tribunaes.
Convenção internacional.
Registo de uma conferência ou deliberação diplomática.
Formulário, que regula os actos públicos.
(B. lat. \textunderscore protocullum\textunderscore )
\section{Protoctistas}
\begin{itemize}
\item {Grp. gram.:m. pl.}
\end{itemize}
\begin{itemize}
\item {Proveniência:(Gr. \textunderscore protoktistos\textunderscore )}
\end{itemize}
Herejes, que sustentavam que a alma é criada antes do corpo.
\section{Proto-evangelho}
\begin{itemize}
\item {Grp. gram.:m.}
\end{itemize}
Promessa, que Deus fez, da futura redempção do homem, nas palavras que dirigiu á serpente, depois da quéda de Adão.
Evangelho apócrypho de Santo Iago, em que se referem successos anteriores ao nascimento de Christo.
Evangelho que, segundo alguns críticos, serviu de fonte aos três evangelhos synópticos.
\section{Protofilo}
\begin{itemize}
\item {Grp. gram.:m.}
\end{itemize}
\begin{itemize}
\item {Utilização:Bot.}
\end{itemize}
\begin{itemize}
\item {Proveniência:(Do gr. \textunderscore protos\textunderscore  + \textunderscore phullon\textunderscore )}
\end{itemize}
A primeira fôlha de uma planta.
\section{Protofitas}
\begin{itemize}
\item {Grp. gram.:f. pl.}
\end{itemize}
\begin{itemize}
\item {Utilização:Bot.}
\end{itemize}
\begin{itemize}
\item {Proveniência:(Do gr. \textunderscore protos\textunderscore  + \textunderscore phuton\textunderscore )}
\end{itemize}
As plantas de organização mais simples.
\section{Protofonia}
\begin{itemize}
\item {Grp. gram.:f.}
\end{itemize}
\begin{itemize}
\item {Utilização:bras}
\end{itemize}
\begin{itemize}
\item {Utilização:Neol.}
\end{itemize}
\begin{itemize}
\item {Proveniência:(Do gr. \textunderscore protos\textunderscore  + \textunderscore phone\textunderscore )}
\end{itemize}
Prelúdio musical; sinfonia de abertura.
\section{Protógamo}
\begin{itemize}
\item {Grp. gram.:adj.}
\end{itemize}
\begin{itemize}
\item {Proveniência:(Do gr. \textunderscore protos\textunderscore  + \textunderscore gamos\textunderscore )}
\end{itemize}
Que casa pela primeira vez.
\section{Protogínica}
\begin{itemize}
\item {Grp. gram.:adj.}
\end{itemize}
\begin{itemize}
\item {Utilização:Bot.}
\end{itemize}
\begin{itemize}
\item {Proveniência:(Do gr. \textunderscore protos\textunderscore  + \textunderscore gune\textunderscore )}
\end{itemize}
Diz-se da dicogamia, em que os órgãos sexuaes femininos se desenvolvem primeiro que os masculinos.
\section{Protógino}
\begin{itemize}
\item {Grp. gram.:m.}
\end{itemize}
Rocha granitóide, que fórma o cimo do Monte-Branco.
\section{Protogýnica}
\begin{itemize}
\item {Grp. gram.:adj.}
\end{itemize}
\begin{itemize}
\item {Utilização:Bot.}
\end{itemize}
\begin{itemize}
\item {Proveniência:(Do gr. \textunderscore protos\textunderscore  + \textunderscore gune\textunderscore )}
\end{itemize}
Diz-se da dichogamia, em que os órgãos sexuaes femininos se desenvolvem primeiro que os masculinos.
\section{Protógyno}
\begin{itemize}
\item {Grp. gram.:m.}
\end{itemize}
Rocha granitóide, que fórma o cimo do Monte-Branco.
\section{Proto-história}
\begin{itemize}
\item {Grp. gram.:f.}
\end{itemize}
História primitiva; primeiros tempos históricos.
\section{Proto-histórico}
\begin{itemize}
\item {Grp. gram.:adj.}
\end{itemize}
Relativo á proto-história.
\section{Proto-iodeto}
\begin{itemize}
\item {fónica:dê}
\end{itemize}
\begin{itemize}
\item {Grp. gram.:m.}
\end{itemize}
\begin{itemize}
\item {Utilização:Chím.}
\end{itemize}
O primeiro grau de combinação de um corpo simples com o iodo.
\section{Protomártir}
\begin{itemize}
\item {Grp. gram.:m.}
\end{itemize}
\begin{itemize}
\item {Proveniência:(De \textunderscore proto...\textunderscore  + \textunderscore mártyr\textunderscore )}
\end{itemize}
O primeiro mártir, (designação especial de Santo-Estêvão).
\section{Protomártyr}
\begin{itemize}
\item {Grp. gram.:m.}
\end{itemize}
\begin{itemize}
\item {Proveniência:(De \textunderscore proto...\textunderscore  + \textunderscore mártyr\textunderscore )}
\end{itemize}
O primeiro mártyr, (designação especial de Santo-Estêvão).
\section{Protomedicato}
\begin{itemize}
\item {Grp. gram.:m.}
\end{itemize}
Antiga junta de médicos, que fiscalizava as boticas, fazia inspecções sanitárias, etc.
Cargo de protomédico.
\section{Protomédico}
\begin{itemize}
\item {Grp. gram.:m.}
\end{itemize}
\begin{itemize}
\item {Proveniência:(De \textunderscore proto...\textunderscore  + \textunderscore médico\textunderscore )}
\end{itemize}
Médico principal de uma côrte, de uma associação, etc.
\section{Protonauta}
\begin{itemize}
\item {Grp. gram.:m.}
\end{itemize}
\begin{itemize}
\item {Proveniência:(De \textunderscore proto...\textunderscore  + \textunderscore nauta\textunderscore )}
\end{itemize}
O primeiro nauta, o que primeiro navegou por certas paragens.
\section{Protonigromante}
\begin{itemize}
\item {Grp. gram.:m.}
\end{itemize}
O primeiro ou maior dos nigromantes. Cf. Castilho, \textunderscore D. Quixote\textunderscore , II, 272.
\section{Protonotariado}
\begin{itemize}
\item {Grp. gram.:m.}
\end{itemize}
Cargo ou dignidade de protonotário.
\section{Protonotário}
\begin{itemize}
\item {Grp. gram.:m.}
\end{itemize}
\begin{itemize}
\item {Proveniência:(De \textunderscore proto...\textunderscore  + \textunderscore notário\textunderscore )}
\end{itemize}
O principal notário, entre os Romanos.
Dignitário da Cúria Romana, que recebe e expede os actos dos consistórios.
\section{Protopatriarca}
\begin{itemize}
\item {Grp. gram.:m.}
\end{itemize}
\begin{itemize}
\item {Proveniência:(De \textunderscore proto...\textunderscore  + \textunderscore patriarca\textunderscore )}
\end{itemize}
O primeiro dos Patriarcas.
\section{Protopatriarcha}
\begin{itemize}
\item {fónica:ca}
\end{itemize}
\begin{itemize}
\item {Grp. gram.:m.}
\end{itemize}
\begin{itemize}
\item {Proveniência:(De \textunderscore proto...\textunderscore  + \textunderscore patriarcha\textunderscore )}
\end{itemize}
O primeiro dos Patriarchas.
\section{Protophonia}
\begin{itemize}
\item {Grp. gram.:f.}
\end{itemize}
\begin{itemize}
\item {Utilização:bras}
\end{itemize}
\begin{itemize}
\item {Utilização:Neol.}
\end{itemize}
\begin{itemize}
\item {Proveniência:(Do gr. \textunderscore protos\textunderscore  + \textunderscore phone\textunderscore )}
\end{itemize}
Prelúdio musical; symphonia de abertura.
\section{Protophyllo}
\begin{itemize}
\item {Grp. gram.:m.}
\end{itemize}
\begin{itemize}
\item {Utilização:Bot.}
\end{itemize}
\begin{itemize}
\item {Proveniência:(Do gr. \textunderscore protos\textunderscore  + \textunderscore phullon\textunderscore )}
\end{itemize}
A primeira fôlha de uma planta.
\section{Protóphytas}
\begin{itemize}
\item {Grp. gram.:f. pl.}
\end{itemize}
\begin{itemize}
\item {Utilização:Bot.}
\end{itemize}
\begin{itemize}
\item {Proveniência:(Do gr. \textunderscore protos\textunderscore  + \textunderscore phuton\textunderscore )}
\end{itemize}
As plantas de organização mais simples.
\section{Protoplasma}
\begin{itemize}
\item {Grp. gram.:m.}
\end{itemize}
\begin{itemize}
\item {Utilização:Hist. Nat.}
\end{itemize}
\begin{itemize}
\item {Proveniência:(Lat. \textunderscore protoplasma\textunderscore )}
\end{itemize}
Substância primordial ou matéria prima dos organismos vivos.
\section{Provação}
\begin{itemize}
\item {Grp. gram.:f.}
\end{itemize}
\begin{itemize}
\item {Proveniência:(Lat. \textunderscore probatio\textunderscore )}
\end{itemize}
Acto ou effeito de provar.
Situação afflictiva; transe.
\section{Provadamente}
\begin{itemize}
\item {Grp. gram.:adv.}
\end{itemize}
De modo provado; sem contestação possível; evidentemente.
\section{Provado}
\begin{itemize}
\item {Grp. gram.:adj.}
\end{itemize}
\begin{itemize}
\item {Proveniência:(De \textunderscore provar\textunderscore )}
\end{itemize}
Experimentado; sabido.
\section{Provador}
\begin{itemize}
\item {Grp. gram.:m.  e  adj.}
\end{itemize}
\begin{itemize}
\item {Proveniência:(Lat. \textunderscore probator\textunderscore )}
\end{itemize}
O que prova.
\section{Provadura}
\begin{itemize}
\item {Grp. gram.:f.}
\end{itemize}
\begin{itemize}
\item {Proveniência:(De \textunderscore provar\textunderscore )}
\end{itemize}
Provação.
Parte de um líquido, que serve para se verificar a qualidade dêste.
\section{Provagem}
\begin{itemize}
\item {Grp. gram.:f.}
\end{itemize}
\begin{itemize}
\item {Utilização:Prov.}
\end{itemize}
\begin{itemize}
\item {Utilização:beir.}
\end{itemize}
O mesmo que \textunderscore prumagem\textunderscore ^1.
\section{Provança}
\begin{itemize}
\item {Grp. gram.:f.}
\end{itemize}
O mesmo que \textunderscore prova\textunderscore . Cf. Garret, \textunderscore Camões\textunderscore .
\section{Provar}
\begin{itemize}
\item {Grp. gram.:v. t.}
\end{itemize}
\begin{itemize}
\item {Proveniência:(Lat. \textunderscore probare\textunderscore )}
\end{itemize}
Dar a prova ou fazer a demonstração de: \textunderscore provar o que se alega\textunderscore .
Testemunhar, patentear: \textunderscore o ferimento prova a aggressão\textunderscore .
Justificar.
Saber por experiência.
Experimentar.
Tentar.
Submeter á prova.
Fazer conhecer.
Fazer ensaio de.
Comer ou beber pequena quantidade de, para lhe verificar a qualidade ou o estado: \textunderscore provar o vinho\textunderscore .
Comer ou beber em pequena quantidade.
Experimentar, soffrendo.
Padecer: \textunderscore provar as agruras do exílio\textunderscore .
\section{Provará}
\begin{itemize}
\item {Grp. gram.:m.}
\end{itemize}
\begin{itemize}
\item {Utilização:Jur.}
\end{itemize}
\begin{itemize}
\item {Proveniência:(De \textunderscore provar\textunderscore )}
\end{itemize}
Cada um dos artigos de um libello ou requerimento judicial.
\section{Provatório}
\begin{itemize}
\item {Grp. gram.:adj.}
\end{itemize}
(V.probatório)
\section{Provável}
\begin{itemize}
\item {Grp. gram.:adj.}
\end{itemize}
\begin{itemize}
\item {Proveniência:(Lat. \textunderscore probabilis\textunderscore )}
\end{itemize}
Que se póde provar.
Que póde acontecer; verosímil.
\section{Provavelmente}
\begin{itemize}
\item {Grp. gram.:adv.}
\end{itemize}
De modo provável.
\section{Prove}
\begin{itemize}
\item {Grp. gram.:m.  e  adj.}
\end{itemize}
\begin{itemize}
\item {Utilização:pop.}
\end{itemize}
\begin{itemize}
\item {Utilização:Ant.}
\end{itemize}
O mesmo que \textunderscore pobre\textunderscore :«\textunderscore ...eis que por beber dos pais ficam proves\textunderscore ». G. Vicente, \textunderscore Maria Parda\textunderscore .
\section{Provecto}
\begin{itemize}
\item {Grp. gram.:adj.}
\end{itemize}
\begin{itemize}
\item {Utilização:Fig.}
\end{itemize}
\begin{itemize}
\item {Proveniência:(Lat. \textunderscore provectus\textunderscore )}
\end{itemize}
Que está adeantado; que tem progredido; adeantado em annos.
Experimentado; muito sabedor.
\section{Provedor}
\begin{itemize}
\item {Grp. gram.:m.}
\end{itemize}
\begin{itemize}
\item {Proveniência:(De \textunderscore prover\textunderscore )}
\end{itemize}
O que provê.
Designação especial do chefe de alguns estabelecimentos pios: \textunderscore Provedor da Misericordia\textunderscore .
\section{Provedora}
\begin{itemize}
\item {Grp. gram.:f.}
\end{itemize}
\begin{itemize}
\item {Proveniência:(De \textunderscore provedor\textunderscore )}
\end{itemize}
Mulhér, que exerce provedoria.
\section{Provedoral}
\begin{itemize}
\item {Grp. gram.:adj.}
\end{itemize}
\begin{itemize}
\item {Utilização:Neol.}
\end{itemize}
Relativo a provedor.
\section{Provedoria}
\begin{itemize}
\item {Grp. gram.:f.}
\end{itemize}
Cargo ou jurisdicção de provedor.
Escritório ou Repartição de provedor.
\section{Proveitar}
\begin{itemize}
\item {Grp. gram.:v. t.}
\end{itemize}
\begin{itemize}
\item {Proveniência:(De \textunderscore proveito\textunderscore )}
\end{itemize}
O mesmo que \textunderscore aproveitar\textunderscore :«\textunderscore ...tantas lidas proveitaste\textunderscore ». Filinto, III, 190.
\section{Proveitar}
\begin{itemize}
\item {Grp. gram.:v. t.  e  i.}
\end{itemize}
(Fórma archaica de \textunderscore prophetar\textunderscore  ou \textunderscore prophetizar\textunderscore . Cf. Frei Fortun., \textunderscore Inéd.\textunderscore , 312)
\section{Proveito}
\begin{itemize}
\item {Grp. gram.:m.}
\end{itemize}
\begin{itemize}
\item {Utilização:Prov.}
\end{itemize}
\begin{itemize}
\item {Utilização:minh.}
\end{itemize}
\begin{itemize}
\item {Proveniência:(Do lat. \textunderscore profectus\textunderscore )}
\end{itemize}
Interesse, ganho.
Utilidade, vantagem; benefício.
A bosta, que se apanha pelos caminhos.
\section{Proveitosamente}
\begin{itemize}
\item {Grp. gram.:adv.}
\end{itemize}
De modo proveitoso.
Com proveito.
\section{Proveitoso}
\begin{itemize}
\item {Grp. gram.:adj.}
\end{itemize}
Que dá proveito, que convém; profícuo; útil.
\section{Provença}
\begin{itemize}
\item {Grp. gram.:f.}
\end{itemize}
\begin{itemize}
\item {Utilização:Ant.}
\end{itemize}
\begin{itemize}
\item {Utilização:Pop.}
\end{itemize}
\begin{itemize}
\item {Proveniência:(De \textunderscore prover\textunderscore )}
\end{itemize}
O mesmo que \textunderscore provisão\textunderscore .
\section{Provença}
\begin{itemize}
\item {Grp. gram.:f.}
\end{itemize}
\begin{itemize}
\item {Utilização:Ant.}
\end{itemize}
O mesmo que \textunderscore província\textunderscore .
\section{Provença}
\begin{itemize}
\item {Grp. gram.:f.}
\end{itemize}
\begin{itemize}
\item {Utilização:Ant.}
\end{itemize}
O mesmo que \textunderscore providência\textunderscore .
\section{Provençal}
\begin{itemize}
\item {Grp. gram.:adj.}
\end{itemize}
\begin{itemize}
\item {Grp. gram.:M.}
\end{itemize}
Relativo á Provença ou aos seus habitantes.
Habitante da Provença.
Língua, falada na Provença; língua de oc.
\section{Pròvigário}
\begin{itemize}
\item {Grp. gram.:m.}
\end{itemize}
\begin{itemize}
\item {Proveniência:(De \textunderscore pro...\textunderscore  + \textunderscore vigário\textunderscore )}
\end{itemize}
Ecclesiástico, investido nas funcções de vigário.
\textunderscore Pròvigário capitular\textunderscore , ecclesiástico, que rege uma diocese por nomeação do metropolita ou do suffragâneo, na falta do que devia sêr eleito pelo cabido.
\section{Provimento}
\begin{itemize}
\item {Grp. gram.:m.}
\end{itemize}
Acto ou effeito de prover.
Providência.
Nomeação ou promoção de um funccionário.
Preenchimento de um lugar público pela nomeação ou promoção de um funccionário.
\section{Provinca}
\begin{itemize}
\item {Grp. gram.:f.}
\end{itemize}
O mesmo que \textunderscore pervinca\textunderscore ^2.
\section{Província}
\begin{itemize}
\item {Grp. gram.:f.}
\end{itemize}
\begin{itemize}
\item {Utilização:Ext.}
\end{itemize}
\begin{itemize}
\item {Proveniência:(Lat. \textunderscore provincia\textunderscore )}
\end{itemize}
Região, que faz parte de um país e que geralmente se distingue das outras regiões do mesmo país pela temperatura, producções, accidentes do solo, etc.
Cada uma das divisões administrativas de certos Estados.
Habitantes de uma província: \textunderscore a província revoltou-se\textunderscore .
Secção, divisão: \textunderscore as várias províncias do saber humano\textunderscore .
Qualquer parte do país, exceptuando a capital e suas cercanias: \textunderscore chegaram ontem da província\textunderscore .
Conjunto dos conventos de uma Ordem monástica dentro de um país.
\section{Provincial}
\begin{itemize}
\item {Grp. gram.:adj.}
\end{itemize}
\begin{itemize}
\item {Grp. gram.:M.}
\end{itemize}
\begin{itemize}
\item {Proveniência:(Lat. \textunderscore provincialis\textunderscore )}
\end{itemize}
Relativo a província.
Superior de certo número de casas religiosas.
\section{Provincialado}
\begin{itemize}
\item {Grp. gram.:m.}
\end{itemize}
Cargo de provincial.
\section{Provincialato}
\begin{itemize}
\item {Grp. gram.:m.}
\end{itemize}
O mesmo que \textunderscore provincialado\textunderscore . Cf. Camillo, \textunderscore M. de Pombal\textunderscore , 306.
\section{Provincialismo}
\begin{itemize}
\item {Grp. gram.:m.}
\end{itemize}
O mesmo ou melhor que \textunderscore provincianismo\textunderscore .
\section{Provincianamente}
\begin{itemize}
\item {Grp. gram.:adv.}
\end{itemize}
De modo provinciano.
Á maneira dos provincianos.
Ingenuamente, com simplicidade; rusticamente.
\section{Provincianismo}
\begin{itemize}
\item {Grp. gram.:m.}
\end{itemize}
\begin{itemize}
\item {Proveniência:(De \textunderscore provinciano\textunderscore )}
\end{itemize}
Accentuação ou pronúncia, peculiar a uma província.
Costume de província.
Palavra ou locução, usada especialmente em uma ou mais províncias.
\section{Provinciano}
\begin{itemize}
\item {Grp. gram.:adj.}
\end{itemize}
\begin{itemize}
\item {Grp. gram.:M.}
\end{itemize}
Provincial.
Que habita na província ou é della natural.
Que não é da capital.
Que não tem os hábitos da côrte.
Aquelle que é natural da província.
\section{Provinco}
\begin{itemize}
\item {Grp. gram.:m.}
\end{itemize}
\begin{itemize}
\item {Utilização:Ant.}
\end{itemize}
\begin{itemize}
\item {Utilização:Prov.}
\end{itemize}
\begin{itemize}
\item {Utilização:beir.}
\end{itemize}
\begin{itemize}
\item {Proveniência:(Do lat. \textunderscore propinquus\textunderscore )}
\end{itemize}
Linhagem; raça.
O mesmo que \textunderscore pervinco\textunderscore .
Parente muito próximo. Cf. Figanière, \textunderscore G. Ansures\textunderscore .
Diabo.
\section{Provindo}
\begin{itemize}
\item {Grp. gram.:adj.}
\end{itemize}
\begin{itemize}
\item {Proveniência:(De \textunderscore provir\textunderscore )}
\end{itemize}
Que proveio.
Procedente; originário; derivado.
\section{Provir}
\begin{itemize}
\item {Grp. gram.:v. i.}
\end{itemize}
\begin{itemize}
\item {Proveniência:(Lat. \textunderscore provenire\textunderscore )}
\end{itemize}
Proceder.
Originar-se; derivar; descender.
\section{Provisão}
\begin{itemize}
\item {Grp. gram.:f.}
\end{itemize}
\begin{itemize}
\item {Proveniência:(Lat. \textunderscore provisio\textunderscore )}
\end{itemize}
Acto ou effeito de prover.
Fornecimento.
Abundância de coisas úteis ou necessárias.
Designação de certos documentos officiaes, em que o Govêrno, como autoridade superior, confere cargo ou auctoriza o exercício de uma profissão, ou expede instrucções, etc.
\section{Proviscar}
\begin{itemize}
\item {Grp. gram.:v. t.}
\end{itemize}
\begin{itemize}
\item {Utilização:Prov.}
\end{itemize}
\begin{itemize}
\item {Utilização:trasm.}
\end{itemize}
Provar aos poucos.
\section{Provisional}
\begin{itemize}
\item {Grp. gram.:adj.}
\end{itemize}
\begin{itemize}
\item {Proveniência:(Do lat. \textunderscore provisio\textunderscore )}
\end{itemize}
Relativo a provisão.
Provisório.
\textunderscore Regimento provisional\textunderscore , ordenança marítima, que ainda há pouco tempo deixou de vigorar entre nós.
\section{Provisionalmente}
\begin{itemize}
\item {Grp. gram.:adv.}
\end{itemize}
De modo provisional.
\section{Provisionar}
\begin{itemize}
\item {Grp. gram.:v. t.}
\end{itemize}
O mesmo que \textunderscore aprovisionar\textunderscore .
\section{Provisioneiro}
\begin{itemize}
\item {Grp. gram.:m.}
\end{itemize}
\begin{itemize}
\item {Utilização:Des.}
\end{itemize}
\begin{itemize}
\item {Proveniência:(Do lat. \textunderscore provisio\textunderscore )}
\end{itemize}
Aquelle que faz ou fornece provisões.
\section{Provisor}
\begin{itemize}
\item {Grp. gram.:m.  e  adj.}
\end{itemize}
\begin{itemize}
\item {Proveniência:(Lat. \textunderscore provisor\textunderscore )}
\end{itemize}
O que faz provisões.
Magistrado ecclesiástico, a quem o Prelado de uma diocese incumbe de jurisdicção contenciosa.
\section{Provisoria}
\begin{itemize}
\item {Grp. gram.:f.}
\end{itemize}
Cargo ou funcções de provisor.
\section{Provisoriamente}
\begin{itemize}
\item {Grp. gram.:adv.}
\end{itemize}
De modo provisório.
Interinamente; transitoriamente.
\section{Pristo}
\begin{itemize}
\item {Grp. gram.:m.}
\end{itemize}
Antiga e pequena embarcação grega.
\section{Prítane}
\begin{itemize}
\item {Grp. gram.:m.}
\end{itemize}
\begin{itemize}
\item {Proveniência:(Gr. \textunderscore prutanis\textunderscore )}
\end{itemize}
Magistrado que, na antiga Grécia, tinha attribuições judiciaes, administrativas, etc.
\section{Pritaneu}
\begin{itemize}
\item {Grp. gram.:m.}
\end{itemize}
\begin{itemize}
\item {Proveniência:(Gr. \textunderscore prutaneion\textunderscore )}
\end{itemize}
Edifício, onde os prítanes celebravam as suas reuniões e onde, á custa do Estado, eram sustentados os cidadãos que bem merecessem da pátria, entre os Gregos.
\section{Prumar}
\begin{itemize}
\item {Grp. gram.:v. i.}
\end{itemize}
\begin{itemize}
\item {Utilização:Náut.}
\end{itemize}
Lançar o prumo para sondar.
\section{Prumbeta}
\begin{itemize}
\item {fónica:bê}
\end{itemize}
\begin{itemize}
\item {Grp. gram.:f.}
\end{itemize}
Peixe da Póvoa de Varzim, (\textunderscore brama baii\textunderscore , Schneid.). Também se encontra nos Açores.
\section{Prumo}
\begin{itemize}
\item {Grp. gram.:m.}
\end{itemize}
\begin{itemize}
\item {Utilização:Fig.}
\end{itemize}
\begin{itemize}
\item {Utilização:Agr.}
\end{itemize}
Instrumento, composto de uma peça de metal ou de pedra, suspensa por um fio e que serve para mostrar a direcção vertical.
Peça de chumbo, de fórma proximamente cónica, e em cuja extremidade se fixa a sondareza, nos navios.
Escora, esteio.
Tino, prudência.
Garfo de enxêrto. Cf. Herculano, \textunderscore Cistér\textunderscore , I, 90.
(Alter. de \textunderscore plumo\textunderscore , do lat. \textunderscore plumbum\textunderscore )
\section{Prunela}
\begin{itemize}
\item {Grp. gram.:f.}
\end{itemize}
O mesmo que \textunderscore erva-férrea\textunderscore .
(Dem. do lat. \textunderscore prunum\textunderscore )
\section{Pruniforme}
\begin{itemize}
\item {Grp. gram.:adj.}
\end{itemize}
\begin{itemize}
\item {Proveniência:(Do lat. \textunderscore prunum\textunderscore  + \textunderscore forma\textunderscore )}
\end{itemize}
Que tem fórma de ameixa.
\section{Prurido}
\begin{itemize}
\item {Grp. gram.:m.}
\end{itemize}
\begin{itemize}
\item {Utilização:Fig.}
\end{itemize}
\begin{itemize}
\item {Proveniência:(Do lat. \textunderscore pruritus\textunderscore )}
\end{itemize}
Comichão.
Tentação.
Impaciência; grande desejo.
\section{Pruriente}
\begin{itemize}
\item {Grp. gram.:adj.}
\end{itemize}
\begin{itemize}
\item {Proveniência:(Lat. \textunderscore pruriens\textunderscore )}
\end{itemize}
Que causa prurido.
\section{Prurigem}
\begin{itemize}
\item {Grp. gram.:f.}
\end{itemize}
\begin{itemize}
\item {Proveniência:(Do lat. \textunderscore prurigo\textunderscore )}
\end{itemize}
O mesmo que \textunderscore prurido\textunderscore .
Designação genérica de certas moléstias de pelle, caracterizadas por comichão e pústulas.
\section{Pruriginoso}
\begin{itemize}
\item {Grp. gram.:adj.}
\end{itemize}
\begin{itemize}
\item {Proveniência:(Lat. \textunderscore pruriginosus\textunderscore )}
\end{itemize}
Que tem prurigem; em que há prurigem.
\section{Prurigo}
\begin{itemize}
\item {Grp. gram.:m.}
\end{itemize}
(V.prurigem)
\section{Prurir}
\begin{itemize}
\item {Grp. gram.:v. t.}
\end{itemize}
\begin{itemize}
\item {Grp. gram.:V. i.}
\end{itemize}
\begin{itemize}
\item {Utilização:Fig.}
\end{itemize}
\begin{itemize}
\item {Proveniência:(Lat. \textunderscore prurire\textunderscore )}
\end{itemize}
Causar comichões a.
Estimular.
Causar prurido.
Estar inquieto, ancioso; têr grandes desejos.
\section{Prurito}
\begin{itemize}
\item {Grp. gram.:m.}
\end{itemize}
\begin{itemize}
\item {Utilização:Des.}
\end{itemize}
O mesmo que \textunderscore prurido\textunderscore :«\textunderscore ...almas que se deixão levar do prurito ou appetite...\textunderscore »\textunderscore Luz e Calor\textunderscore , 150.
\section{Prussiano}
\begin{itemize}
\item {Grp. gram.:adj.}
\end{itemize}
\begin{itemize}
\item {Grp. gram.:M.}
\end{itemize}
Relativo á Prússia.
Habitante da Prússia.
Língua, falada pelos Prussianos.
Espécie de capote ou gabão, usado há poucos annos.
\section{Prussiato}
\begin{itemize}
\item {Grp. gram.:m.}
\end{itemize}
\begin{itemize}
\item {Utilização:Chím.}
\end{itemize}
Gênero de saes, produzido pelo ácido prússico e uma base.
\section{Prússico}
\begin{itemize}
\item {Grp. gram.:adj.}
\end{itemize}
\begin{itemize}
\item {Proveniência:(De \textunderscore Prússia\textunderscore , n. p.)}
\end{itemize}
Diz-se do ácido, mais conhecido actualmente por \textunderscore ácido cyanhýdrico\textunderscore .
\section{Prussito}
\begin{itemize}
\item {Grp. gram.:m.}
\end{itemize}
\begin{itemize}
\item {Utilização:Chím.}
\end{itemize}
Prussiato de ferro.
\section{Prusso}
\begin{itemize}
\item {Grp. gram.:m.  e  adj.}
\end{itemize}
(V.prussiano). Cf. Macedo, \textunderscore Burros\textunderscore , 297.
\section{Prysto}
\begin{itemize}
\item {Grp. gram.:m.}
\end{itemize}
Antiga e pequena embarcação grega.
\section{Prýtane}
\begin{itemize}
\item {Grp. gram.:m.}
\end{itemize}
\begin{itemize}
\item {Proveniência:(Gr. \textunderscore prutanis\textunderscore )}
\end{itemize}
Magistrado que, na antiga Grécia, tinha attribuições judiciaes, administrativas, etc.
\section{Prytaneu}
\begin{itemize}
\item {Grp. gram.:m.}
\end{itemize}
\begin{itemize}
\item {Proveniência:(Gr. \textunderscore prutaneion\textunderscore )}
\end{itemize}
Edifício, onde os prýtanes celebravam as suas reuniões e onde, á custa do Estado, eram sustentados os cidadãos que bem merecessem da pátria, entre os Gregos.
\section{Psacálio}
\begin{itemize}
\item {Grp. gram.:m.}
\end{itemize}
\begin{itemize}
\item {Proveniência:(Do gr. \textunderscore psakalon\textunderscore )}
\end{itemize}
Gênero de plantas, da fam. das compostas.
\section{Psalídio}
\begin{itemize}
\item {Grp. gram.:m.}
\end{itemize}
\begin{itemize}
\item {Proveniência:(Do gr. \textunderscore psalidion\textunderscore )}
\end{itemize}
Gênero de insectos coleópteros tetrâmeros.
\section{Psalidófora}
\begin{itemize}
\item {Grp. gram.:f.}
\end{itemize}
Gênero de insectos ortópteros.
\section{Psalidóphora}
\begin{itemize}
\item {Grp. gram.:f.}
\end{itemize}
Gênero de insectos orthópteros.
\section{Psalmear}
\begin{itemize}
\item {fónica:sal}
\end{itemize}
\begin{itemize}
\item {Grp. gram.:v. t.}
\end{itemize}
\begin{itemize}
\item {Grp. gram.:V. i.}
\end{itemize}
Cantar em fórma de psalmo.
Entoar tristemente.
Entoar psalmos, sem mudar de tom.
Cantar, lêr ou recitar monotonamente.
Têr estilo monótono.
\section{Psálmico}
\begin{itemize}
\item {fónica:sál}
\end{itemize}
\begin{itemize}
\item {Grp. gram.:adj.}
\end{itemize}
Relativo a psalmo.
Semelhante a psalmo.
\section{Psalmista}
\begin{itemize}
\item {fónica:sal}
\end{itemize}
\begin{itemize}
\item {Grp. gram.:m.  e  f.}
\end{itemize}
\begin{itemize}
\item {Utilização:Restrict.}
\end{itemize}
\begin{itemize}
\item {Proveniência:(Lat. \textunderscore psalmista\textunderscore )}
\end{itemize}
Pessôa, que faz psalmos.
O rei David, autor dos psalmos bíblicos.
\section{Psalódio}
\begin{itemize}
\item {Grp. gram.:m.}
\end{itemize}
Gênero de peixes.
\section{Pseudoblepsia}
\begin{itemize}
\item {Grp. gram.:f.}
\end{itemize}
\begin{itemize}
\item {Utilização:Med.}
\end{itemize}
\begin{itemize}
\item {Proveniência:(Do gr. \textunderscore pseudos\textunderscore  + \textunderscore blepsis\textunderscore )}
\end{itemize}
Perversão do sentido da vista.
\section{Pseudocoristo}
\begin{itemize}
\item {Grp. gram.:m.}
\end{itemize}
Gênero de crustáceos decápodes.
\section{Pseudocorysto}
\begin{itemize}
\item {Grp. gram.:m.}
\end{itemize}
Gênero de crustáceos decápodes.
\section{Pseudofobia}
\begin{itemize}
\item {Grp. gram.:f.}
\end{itemize}
\begin{itemize}
\item {Proveniência:(Do gr. \textunderscore pseudos\textunderscore  + \textunderscore phobein\textunderscore )}
\end{itemize}
Mêdo mórbido de qualquer coisa que não causa dôr nem molesta, mas que só desgosta, como a agorafobía.
\section{Pseudofobo}
\begin{itemize}
\item {Grp. gram.:m.}
\end{itemize}
Aquele que tem pseudofobía.
\section{Pseudólitho}
\begin{itemize}
\item {Grp. gram.:m.}
\end{itemize}
\begin{itemize}
\item {Utilização:Miner.}
\end{itemize}
\begin{itemize}
\item {Proveniência:(Do gr. \textunderscore pseudos\textunderscore  + \textunderscore lithos\textunderscore )}
\end{itemize}
Variedade de talco, (silicato hydratado de magnésio).
\section{Pseudólito}
\begin{itemize}
\item {Grp. gram.:m.}
\end{itemize}
\begin{itemize}
\item {Utilização:Miner.}
\end{itemize}
\begin{itemize}
\item {Proveniência:(Do gr. \textunderscore pseudos\textunderscore  + \textunderscore lithos\textunderscore )}
\end{itemize}
Variedade de talco, (silicato hydratado de magnésio).
\section{Psêudomo}
\begin{itemize}
\item {Grp. gram.:m.}
\end{itemize}
Gênero de insectos coleópteros tetrâmeros.
\section{Pseudomorfismo}
\begin{itemize}
\item {Grp. gram.:m.}
\end{itemize}
\begin{itemize}
\item {Utilização:Miner.}
\end{itemize}
\begin{itemize}
\item {Proveniência:(De \textunderscore pseudomorfo\textunderscore )}
\end{itemize}
Substituição de uma substância cristalina por outra, conservando-se a fórma de um metal originário.
\section{Pseudomorfo}
\begin{itemize}
\item {Grp. gram.:adj.}
\end{itemize}
\begin{itemize}
\item {Proveniência:(Do gr. \textunderscore pseudos\textunderscore  + \textunderscore morphe\textunderscore )}
\end{itemize}
Diz-se do mineral, que tomou acidentalmente a fórma cristalina de outro.
\section{Pseudomorfose}
\begin{itemize}
\item {Grp. gram.:f.}
\end{itemize}
\begin{itemize}
\item {Proveniência:(De \textunderscore pseudomorfo\textunderscore )}
\end{itemize}
Efeito do pseudomorfismo.
Qualquer aumento anómalo de uma parte normal, em Fisiologia.
\section{Pseudomorphismo}
\begin{itemize}
\item {Grp. gram.:m.}
\end{itemize}
\begin{itemize}
\item {Utilização:Miner.}
\end{itemize}
\begin{itemize}
\item {Proveniência:(De \textunderscore pseudomorpho\textunderscore )}
\end{itemize}
Substituição de uma substância crystallina por outra, conservando-se a fórma de um metal originário.
\section{Pseudomorpho}
\begin{itemize}
\item {Grp. gram.:adj.}
\end{itemize}
\begin{itemize}
\item {Proveniência:(Do gr. \textunderscore pseudos\textunderscore  + \textunderscore morphe\textunderscore )}
\end{itemize}
Diz-se do mineral, que tomou accidentalmente a fórma crystallina de outro.
\section{Pseudomorphose}
\begin{itemize}
\item {Grp. gram.:f.}
\end{itemize}
\begin{itemize}
\item {Proveniência:(De \textunderscore pseudomorpho\textunderscore )}
\end{itemize}
Effeito do pseudomorphismo.
Qualquer aumento anómalo de uma parte normal, em Physiologia.
\section{Pseudoneurópteros}
\begin{itemize}
\item {Grp. gram.:m. pl.}
\end{itemize}
\begin{itemize}
\item {Proveniência:(Do gr. \textunderscore pseudos\textunderscore  + \textunderscore neuron\textunderscore  + \textunderscore pteron\textunderscore )}
\end{itemize}
Ordem de insectos.
\section{Pseudonímia}
\begin{itemize}
\item {Grp. gram.:f.}
\end{itemize}
Qualidade de pseudónimo.
\section{Pseudónimo}
\begin{itemize}
\item {Grp. gram.:m.}
\end{itemize}
\begin{itemize}
\item {Grp. gram.:Adj.}
\end{itemize}
\begin{itemize}
\item {Proveniência:(Gr. \textunderscore pseudonumos\textunderscore )}
\end{itemize}
Nome falso ou suposto.
Que subscreve as suas obras com um nome que lhe não pertence.
Diz-se da obra, escrita ou publicada sob um nome suposto.
\section{Pseudonýmia}
\begin{itemize}
\item {Grp. gram.:f.}
\end{itemize}
Qualidade de pseudónymo.
\section{Pseudónymo}
\begin{itemize}
\item {Grp. gram.:m.}
\end{itemize}
\begin{itemize}
\item {Grp. gram.:Adj.}
\end{itemize}
\begin{itemize}
\item {Proveniência:(Gr. \textunderscore pseudonumos\textunderscore )}
\end{itemize}
Nome falso ou supposto.
Que subscreve as suas obras com um nome que lhe não pertence.
Diz-se da obra, escrita ou publicada sob um nome supposto.
\section{Pseudophobia}
\begin{itemize}
\item {Grp. gram.:f.}
\end{itemize}
\begin{itemize}
\item {Proveniência:(Do gr. \textunderscore pseudos\textunderscore  + \textunderscore phobein\textunderscore )}
\end{itemize}
Mêdo mórbido de qualquer coisa que não causa dôr nem molesta, mas que só desgosta, como a agoraphobía.
\section{Pseudóphobo}
\begin{itemize}
\item {Grp. gram.:m.}
\end{itemize}
Aquelle que tem pseudophobía.
\section{Pseudópode}
\begin{itemize}
\item {Grp. gram.:m.}
\end{itemize}
\begin{itemize}
\item {Proveniência:(Do gr. \textunderscore pseudos\textunderscore  + \textunderscore pous\textunderscore , \textunderscore podos\textunderscore )}
\end{itemize}
Expansões protoplásmicas, que emittem vários organismos unicellulares ou certas céllulas de organismos superiores, por meio das quaes executam diversos movimentos e até o da locomoção, (\textunderscore falso pé\textunderscore )
\section{Pseudoprofeta}
\begin{itemize}
\item {Grp. gram.:m.}
\end{itemize}
\begin{itemize}
\item {Proveniência:(Lat. \textunderscore pseudopropheta\textunderscore )}
\end{itemize}
Falso profeta.
\section{Pseudopropheta}
\begin{itemize}
\item {Grp. gram.:m.}
\end{itemize}
\begin{itemize}
\item {Proveniência:(Lat. \textunderscore pseudopropheta\textunderscore )}
\end{itemize}
Falso propheta.
\section{Pseudo-sciência}
\begin{itemize}
\item {Grp. gram.:f.}
\end{itemize}
Falsa sciência, supposta sciência.
\section{Pseudoscópico}
\begin{itemize}
\item {Grp. gram.:adj.}
\end{itemize}
Relativo ao pseudoscópio.
\section{Pseudoscópio}
\begin{itemize}
\item {Grp. gram.:m.}
\end{itemize}
\begin{itemize}
\item {Proveniência:(Do gr. \textunderscore pseudos\textunderscore  + \textunderscore skopein\textunderscore )}
\end{itemize}
Espécie de estereoscópio, inventado por Wheatstone, que transforma, á vista, um espelho côncavo em convexo, um objecto ôco em saliente, etc.
\section{Pseudosophia}
\begin{itemize}
\item {fónica:so}
\end{itemize}
\begin{itemize}
\item {Grp. gram.:f.}
\end{itemize}
\begin{itemize}
\item {Proveniência:(Do gr. \textunderscore pseudos\textunderscore  + \textunderscore sophos\textunderscore )}
\end{itemize}
Falsa sciência.
\section{Pseudossofia}
\begin{itemize}
\item {Grp. gram.:f.}
\end{itemize}
\begin{itemize}
\item {Proveniência:(Do gr. \textunderscore pseudos\textunderscore  + \textunderscore sophos\textunderscore )}
\end{itemize}
Falsa ciência.
\section{Pseudospermo}
\begin{itemize}
\item {Grp. gram.:adj.}
\end{itemize}
\begin{itemize}
\item {Utilização:Bot.}
\end{itemize}
\begin{itemize}
\item {Proveniência:(Do gr. \textunderscore pseudos\textunderscore  + \textunderscore sperma\textunderscore )}
\end{itemize}
Diz-se do fruto, que tem um só grão ou um pequeno número delles, os quaes nunca se abrem espontaneamente na sua madureza, fruto em que o pericarpo é de tal maneira soldado á semente, que parece um só invólucro.
\section{Pseudo-topázio}
\begin{itemize}
\item {Grp. gram.:m.}
\end{itemize}
Quartzo, que imita o topázio.
\section{Pseudozoário}
\begin{itemize}
\item {Grp. gram.:adj.}
\end{itemize}
\begin{itemize}
\item {Proveniência:(Do gr. \textunderscore pseudos\textunderscore  + \textunderscore zoarion\textunderscore )}
\end{itemize}
Diz-se de alguns vegetaes, que têm a apparência de animaes.
\section{Psi}
\begin{itemize}
\item {Grp. gram.:m.}
\end{itemize}
Nome de letra grega, correspondente ao nosso grupo \textunderscore ps\textunderscore .
\section{Psiádia}
\begin{itemize}
\item {Grp. gram.:f.}
\end{itemize}
\begin{itemize}
\item {Proveniência:(Do gr. \textunderscore psias\textunderscore  + \textunderscore idea\textunderscore )}
\end{itemize}
Gênero de plantas, da fam. das compostas.
\section{Psicóbio}
\begin{itemize}
\item {Grp. gram.:m.}
\end{itemize}
Gênero de insectos coleópteros pentâmeros.
\section{Psicogenia}
\begin{itemize}
\item {Grp. gram.:f.}
\end{itemize}
\begin{itemize}
\item {Proveniência:(Do gr. \textunderscore psukhe\textunderscore  + \textunderscore gen\textunderscore )}
\end{itemize}
Gênese ou origem da alma.
\section{Psicogênico}
\begin{itemize}
\item {Grp. gram.:adj.}
\end{itemize}
Relativo á psicogenia.
\section{Psicognosia}
\begin{itemize}
\item {Grp. gram.:f.}
\end{itemize}
\begin{itemize}
\item {Proveniência:(Do gr. \textunderscore psukhe\textunderscore  + \textunderscore gnosis\textunderscore )}
\end{itemize}
Conhecimento profundo das faculdades da alma.
\section{Psicognóstico}
\begin{itemize}
\item {Grp. gram.:adj.}
\end{itemize}
Relativo á psicognosia.
\section{Psicografia}
\begin{itemize}
\item {Grp. gram.:f.}
\end{itemize}
\begin{itemize}
\item {Utilização:Espir.}
\end{itemize}
História ou descrição da alma ou das suas faculdades.
Escrita dos espíritos, pela mão do médium.
(Cp. \textunderscore psicógrafo\textunderscore )
\section{Psicográfico}
\begin{itemize}
\item {Grp. gram.:adj.}
\end{itemize}
Relativo á psicografia.
\section{Psicógrafo}
\begin{itemize}
\item {Grp. gram.:m.}
\end{itemize}
\begin{itemize}
\item {Utilização:Espir.}
\end{itemize}
\begin{itemize}
\item {Proveniência:(Do gr. \textunderscore psukhe\textunderscore  + \textunderscore graphein\textunderscore )}
\end{itemize}
Aquele que se ocupa de psicografia.
Médium, que escreve por sugestão ou acção dos espíritos.
\section{Psicologia}
\begin{itemize}
\item {Grp. gram.:f.}
\end{itemize}
Tratado á cêrca da alma ou das faculdades intelectuaes e moraes.
(Cp. \textunderscore psicólogo\textunderscore )
\section{Psicologicamente}
\begin{itemize}
\item {Grp. gram.:adv.}
\end{itemize}
De modo psicológico.
Segundo a psicologia.
\section{Psicológico}
\begin{itemize}
\item {Grp. gram.:adj.}
\end{itemize}
Relativo á psicologia.
\section{Psicologismo}
\begin{itemize}
\item {Grp. gram.:m.}
\end{itemize}
\begin{itemize}
\item {Proveniência:(De \textunderscore psicologia\textunderscore )}
\end{itemize}
Conjunto das teorias dos psicólogos. Cf. Camillo, \textunderscore Vinho do Pôrto\textunderscore , 19.
\section{Psicologista}
\begin{itemize}
\item {Grp. gram.:m.}
\end{itemize}
\begin{itemize}
\item {Proveniência:(De \textunderscore psicologia\textunderscore )}
\end{itemize}
Aquele que se ocupa de psicologia.
\section{Psicólogo}
\begin{itemize}
\item {Grp. gram.:m.}
\end{itemize}
\begin{itemize}
\item {Proveniência:(Do gr. \textunderscore psukhe\textunderscore  + \textunderscore logos\textunderscore )}
\end{itemize}
Aquele que é versado em psicologia.
\section{Psicomancia}
\begin{itemize}
\item {Grp. gram.:f.}
\end{itemize}
\begin{itemize}
\item {Proveniência:(Do gr. \textunderscore psukhe\textunderscore  + \textunderscore manteia\textunderscore )}
\end{itemize}
Suposta arte de adivinhar, evocando as almas dos mortos.
\section{Psicometria}
\begin{itemize}
\item {Grp. gram.:f.}
\end{itemize}
\begin{itemize}
\item {Proveniência:(Do gr. \textunderscore psukhe\textunderscore  + \textunderscore metron\textunderscore )}
\end{itemize}
Registo e medida da actividade intelectual.
\section{Psicométrico}
\begin{itemize}
\item {Grp. gram.:adj.}
\end{itemize}
Relativo a psicometria.
\section{Psicopata}
\begin{itemize}
\item {Grp. gram.:m.}
\end{itemize}
\begin{itemize}
\item {Proveniência:(Do gr. \textunderscore psukhe\textunderscore  + \textunderscore pathos\textunderscore )}
\end{itemize}
Aquele que sofre doença mental.
\section{Psiquiátrico}
\begin{itemize}
\item {Grp. gram.:adj.}
\end{itemize}
Relativo á psiquiatria.
\section{Psiquiatrista}
\begin{itemize}
\item {Grp. gram.:m.}
\end{itemize}
O mesmo que \textunderscore psiquiatro\textunderscore .
\section{Psiquiatro}
\begin{itemize}
\item {Grp. gram.:m.}
\end{itemize}
O mesmo que \textunderscore psiquiatra\textunderscore .
\section{Psíquico}
\begin{itemize}
\item {Grp. gram.:adj.}
\end{itemize}
\begin{itemize}
\item {Proveniência:(Gr. \textunderscore psukhikos\textunderscore )}
\end{itemize}
Relativo á alma ou ás faculdades intelectuaes e moraes.
\section{Psiquíneas}
\begin{itemize}
\item {Grp. gram.:f. pl.}
\end{itemize}
Tríbo de plantas crucíferas.
\section{Psiquismo}
\begin{itemize}
\item {Grp. gram.:m.}
\end{itemize}
\begin{itemize}
\item {Proveniência:(Do gr. \textunderscore psukhe\textunderscore )}
\end{itemize}
Conjunto dos fenómenos procedentes da alma ou relativos á alma.
O mesmo que \textunderscore espiritualismo\textunderscore . Cf. Th. Braga, \textunderscore Mod. Ideias\textunderscore , II, 113.
\section{Psychiátrico}
\begin{itemize}
\item {fónica:qui}
\end{itemize}
\begin{itemize}
\item {Grp. gram.:adj.}
\end{itemize}
Relativo á psychiatria.
\section{Psychiatrista}
\begin{itemize}
\item {fónica:qui}
\end{itemize}
\begin{itemize}
\item {Grp. gram.:m.}
\end{itemize}
O mesmo que \textunderscore psychiatro\textunderscore .
\section{Psychiatro}
\begin{itemize}
\item {fónica:qui}
\end{itemize}
\begin{itemize}
\item {Grp. gram.:m.}
\end{itemize}
O mesmo que \textunderscore psychiatra\textunderscore .
\section{Psýchico}
\begin{itemize}
\item {fónica:qui}
\end{itemize}
\begin{itemize}
\item {Grp. gram.:adj.}
\end{itemize}
\begin{itemize}
\item {Proveniência:(Gr. \textunderscore psukhikos\textunderscore )}
\end{itemize}
Relativo á alma ou ás faculdades intellectuaes e moraes.
\section{Psychíneas}
\begin{itemize}
\item {fónica:qui}
\end{itemize}
\begin{itemize}
\item {Grp. gram.:f. pl.}
\end{itemize}
Tríbo de plantas crucíferas.
\section{Psychismo}
\begin{itemize}
\item {fónica:qui}
\end{itemize}
\begin{itemize}
\item {Grp. gram.:m.}
\end{itemize}
\begin{itemize}
\item {Proveniência:(Do gr. \textunderscore psukhe\textunderscore )}
\end{itemize}
Conjunto dos phenómenos procedentes da alma ou relativos á alma.
O mesmo que \textunderscore espiritualismo\textunderscore . Cf. Th. Braga, \textunderscore Mod. Ideias\textunderscore , II, 113.
\section{Psychóbio}
\begin{itemize}
\item {fónica:có}
\end{itemize}
\begin{itemize}
\item {Grp. gram.:m.}
\end{itemize}
Gênero de insectos coleópteros pentâmeros.
\section{Psycho-dynâmico}
\begin{itemize}
\item {Grp. gram.:adj.}
\end{itemize}
Relativo ao psycho-dynamismo.
\section{Psycho-dynamismo}
\begin{itemize}
\item {Grp. gram.:m.}
\end{itemize}
Doutrina philosóphica dos que reduzem a uma fôrça todas as energias do universo, desde os phenómenos phýsicos elementares á ordem moral. Cf. Th. Braga, \textunderscore Mod. Ideias\textunderscore , 214.
\section{Psychogenia}
\begin{itemize}
\item {fónica:có}
\end{itemize}
\begin{itemize}
\item {Grp. gram.:f.}
\end{itemize}
\begin{itemize}
\item {Proveniência:(Do gr. \textunderscore psukhe\textunderscore  + \textunderscore gen\textunderscore )}
\end{itemize}
Gênese ou origem da alma.
\section{Psychogênico}
\begin{itemize}
\item {fónica:co}
\end{itemize}
\begin{itemize}
\item {Grp. gram.:adj.}
\end{itemize}
Relativo á psychogenia.
\section{Psychognosia}
\begin{itemize}
\item {fónica:co}
\end{itemize}
\begin{itemize}
\item {Grp. gram.:f.}
\end{itemize}
\begin{itemize}
\item {Proveniência:(Do gr. \textunderscore psukhe\textunderscore  + \textunderscore gnosis\textunderscore )}
\end{itemize}
Conhecimento profundo das faculdades da alma.
\section{Psychognóstico}
\begin{itemize}
\item {fónica:co}
\end{itemize}
\begin{itemize}
\item {Grp. gram.:adj.}
\end{itemize}
Relativo á psychognosia.
\section{Psychographia}
\begin{itemize}
\item {fónica:co}
\end{itemize}
\begin{itemize}
\item {Grp. gram.:f.}
\end{itemize}
\begin{itemize}
\item {Utilização:Espir.}
\end{itemize}
História ou descripção da alma ou das suas faculdades.
Escrita dos espíritos, pela mão do médium.
(Cp. \textunderscore psychógrapho\textunderscore )
\section{Psychográphico}
\begin{itemize}
\item {fónica:co}
\end{itemize}
\begin{itemize}
\item {Grp. gram.:adj.}
\end{itemize}
Relativo á psychographia.
\section{Psychógrapho}
\begin{itemize}
\item {fónica:co}
\end{itemize}
\begin{itemize}
\item {Grp. gram.:m.}
\end{itemize}
\begin{itemize}
\item {Utilização:Espir.}
\end{itemize}
\begin{itemize}
\item {Proveniência:(Do gr. \textunderscore psukhe\textunderscore  + \textunderscore graphein\textunderscore )}
\end{itemize}
Aquelle que se occupa de psychographia.
Médium, que escreve por suggestão ou acção dos espíritos.
\section{Psychologia}
\begin{itemize}
\item {fónica:co}
\end{itemize}
\begin{itemize}
\item {Grp. gram.:f.}
\end{itemize}
Tratado á cêrca da alma ou das faculdades intellectuaes e moraes.
(Cp. \textunderscore psychólogo\textunderscore )
\section{Psychologicamente}
\begin{itemize}
\item {fónica:co}
\end{itemize}
\begin{itemize}
\item {Grp. gram.:adv.}
\end{itemize}
De modo psychológico.
Segundo a psychologia.
\section{Psychológico}
\begin{itemize}
\item {fónica:co}
\end{itemize}
\begin{itemize}
\item {Grp. gram.:adj.}
\end{itemize}
Relativo á psychologia.
\section{Psychologismo}
\begin{itemize}
\item {fónica:co}
\end{itemize}
\begin{itemize}
\item {Grp. gram.:m.}
\end{itemize}
\begin{itemize}
\item {Proveniência:(De \textunderscore psychologia\textunderscore )}
\end{itemize}
Conjunto das theorias dos psychólogos. Cf. Camillo, \textunderscore Vinho do Pôrto\textunderscore , 19.
\section{Psychologista}
\begin{itemize}
\item {fónica:co}
\end{itemize}
\begin{itemize}
\item {Grp. gram.:m.}
\end{itemize}
\begin{itemize}
\item {Proveniência:(De \textunderscore psychologia\textunderscore )}
\end{itemize}
Aquelle que se occupa de psychologia.
\section{Psychólogo}
\begin{itemize}
\item {fónica:co}
\end{itemize}
\begin{itemize}
\item {Grp. gram.:m.}
\end{itemize}
\begin{itemize}
\item {Proveniência:(Do gr. \textunderscore psukhe\textunderscore  + \textunderscore logos\textunderscore )}
\end{itemize}
Aquelle que é versado em psychologia.
\section{Psychomancia}
\begin{itemize}
\item {fónica:co}
\end{itemize}
\begin{itemize}
\item {Grp. gram.:f.}
\end{itemize}
\begin{itemize}
\item {Proveniência:(Do gr. \textunderscore psukhe\textunderscore  + \textunderscore manteia\textunderscore )}
\end{itemize}
Supposta arte de adivinhar, evocando as almas dos mortos.
\section{Psychometria}
\begin{itemize}
\item {fónica:có}
\end{itemize}
\begin{itemize}
\item {Grp. gram.:f.}
\end{itemize}
\begin{itemize}
\item {Proveniência:(Do gr. \textunderscore psukhe\textunderscore  + \textunderscore metron\textunderscore )}
\end{itemize}
Registo e medida da actividade intellectual.
\section{Psychométrico}
\begin{itemize}
\item {fónica:có}
\end{itemize}
\begin{itemize}
\item {Grp. gram.:adj.}
\end{itemize}
Relativo a psycometria.
\section{Psychopatha}
\begin{itemize}
\item {fónica:co}
\end{itemize}
\begin{itemize}
\item {Grp. gram.:m.}
\end{itemize}
\begin{itemize}
\item {Proveniência:(Do gr. \textunderscore psukhe\textunderscore  + \textunderscore pathos\textunderscore )}
\end{itemize}
Aquelle que soffre doença mental.
\section{Ptena}
\begin{itemize}
\item {Grp. gram.:f.}
\end{itemize}
Gênero de insectos coleópteros.
\section{Pteracanta}
\begin{itemize}
\item {Grp. gram.:f.}
\end{itemize}
\begin{itemize}
\item {Proveniência:(Do gr. \textunderscore pteron\textunderscore  + \textunderscore akantha\textunderscore )}
\end{itemize}
Gênero de insectos longicórneos.
\section{Pteracantha}
\begin{itemize}
\item {Grp. gram.:f.}
\end{itemize}
\begin{itemize}
\item {Proveniência:(Do gr. \textunderscore pteron\textunderscore  + \textunderscore akantha\textunderscore )}
\end{itemize}
Gênero de insectos longicórneos.
\section{Pteracantho}
\begin{itemize}
\item {Grp. gram.:m.}
\end{itemize}
Gênero de insectos coleópteros.
(Cp. \textunderscore pteracantha\textunderscore )
\section{Pteracanto}
\begin{itemize}
\item {Grp. gram.:m.}
\end{itemize}
Gênero de insectos coleópteros.
(Cp. \textunderscore pteracanta\textunderscore )
\section{Pterantho}
\begin{itemize}
\item {Grp. gram.:m.}
\end{itemize}
\begin{itemize}
\item {Proveniência:(Do gr. \textunderscore pteron\textunderscore  + \textunderscore anthos\textunderscore )}
\end{itemize}
Gênero de plantas paronýchias.
\section{Pteranto}
\begin{itemize}
\item {Grp. gram.:m.}
\end{itemize}
\begin{itemize}
\item {Proveniência:(Do gr. \textunderscore pteron\textunderscore  + \textunderscore anthos\textunderscore )}
\end{itemize}
Gênero de plantas paroníquias.
\section{Pterela}
\begin{itemize}
\item {Grp. gram.:f.}
\end{itemize}
Gênero de crustáceos isópodes.
\section{Ptérico}
\begin{itemize}
\item {Grp. gram.:adj.}
\end{itemize}
\begin{itemize}
\item {Utilização:Anat.}
\end{itemize}
\begin{itemize}
\item {Proveniência:(Do gr. \textunderscore pteron\textunderscore )}
\end{itemize}
Diz-se do ângulo antero-inferior dos parietaes.
\section{Pteridófitas}
\begin{itemize}
\item {Grp. gram.:f. pl.}
\end{itemize}
\begin{itemize}
\item {Utilização:Bot.}
\end{itemize}
\begin{itemize}
\item {Proveniência:(Do gr. \textunderscore pteris\textunderscore  + \textunderscore phuton\textunderscore )}
\end{itemize}
Grande divisão do reino vegetal, em que estão compreendidos os fêtos, bem como todas as criptogâmicas vasculares.
\section{Pteridóphytas}
\begin{itemize}
\item {Grp. gram.:f. pl.}
\end{itemize}
\begin{itemize}
\item {Utilização:Bot.}
\end{itemize}
\begin{itemize}
\item {Proveniência:(Do gr. \textunderscore pteris\textunderscore  + \textunderscore phuton\textunderscore )}
\end{itemize}
Grande divisão do reino vegetal, em que estão comprehendidos os fêtos, bem como todas as cryptogâmicas vasculares.
\section{Pterígina}
\begin{itemize}
\item {Grp. gram.:f.}
\end{itemize}
\begin{itemize}
\item {Utilização:Bot.}
\end{itemize}
\begin{itemize}
\item {Proveniência:(Do gr. \textunderscore pteron\textunderscore  + \textunderscore gune\textunderscore )}
\end{itemize}
Apêndice membranoso de uma semente.
\section{Pterigínio}
\begin{itemize}
\item {Grp. gram.:m.}
\end{itemize}
O mesmo ou melhor que \textunderscore pterígina\textunderscore .
\section{Pterígyna}
\begin{itemize}
\item {Grp. gram.:f.}
\end{itemize}
\begin{itemize}
\item {Utilização:Bot.}
\end{itemize}
\begin{itemize}
\item {Proveniência:(Do gr. \textunderscore pteron\textunderscore  + \textunderscore gune\textunderscore )}
\end{itemize}
Appêndice membranoso de uma semente.
\section{Pterigýnio}
\begin{itemize}
\item {Grp. gram.:m.}
\end{itemize}
O mesmo ou melhor que \textunderscore pterígyna\textunderscore .
\section{Ptério}
\begin{itemize}
\item {Grp. gram.:m.}
\end{itemize}
\begin{itemize}
\item {Utilização:Anat.}
\end{itemize}
\begin{itemize}
\item {Proveniência:(Do gr. \textunderscore pterion\textunderscore )}
\end{itemize}
Região craniana, geralmente com a fórma de H, onde se articulam os ossos frontal, parietal e temporal com a asa correspondente do esphenóide.
\section{Ptérion}
\begin{itemize}
\item {Grp. gram.:m.}
\end{itemize}
\begin{itemize}
\item {Utilização:Anat.}
\end{itemize}
\begin{itemize}
\item {Proveniência:(Do gr. \textunderscore pterion\textunderscore )}
\end{itemize}
Região craniana, geralmente com a fórma de H, onde se articulam os ossos frontal, parietal e temporal com a asa correspondente do esphenóide.
\section{Ptéris}
\begin{itemize}
\item {Grp. gram.:m.}
\end{itemize}
\begin{itemize}
\item {Proveniência:(Gr. \textunderscore pteris\textunderscore )}
\end{itemize}
Gênero de fêtos polypodiáceos.
\section{Pterisantho}
\begin{itemize}
\item {Grp. gram.:m.}
\end{itemize}
\begin{itemize}
\item {Proveniência:(Do gr. \textunderscore pteris\textunderscore  + \textunderscore anthos\textunderscore )}
\end{itemize}
Gênero de videiras.
Uma das divisões da fam. das ampelídeas.
\section{Pterisanto}
\begin{itemize}
\item {Grp. gram.:m.}
\end{itemize}
\begin{itemize}
\item {Proveniência:(Do gr. \textunderscore pteris\textunderscore  + \textunderscore anthos\textunderscore )}
\end{itemize}
Gênero de videiras.
Uma das divisões da fam. das ampelídeas.
\section{Pternalgia}
\begin{itemize}
\item {Grp. gram.:f.}
\end{itemize}
\begin{itemize}
\item {Utilização:Med.}
\end{itemize}
\begin{itemize}
\item {Proveniência:(Do gr. \textunderscore pterna\textunderscore  + \textunderscore algos\textunderscore )}
\end{itemize}
Dôr, na face inferior do calcanhar.
\section{Pterocarpo}
\begin{itemize}
\item {Grp. gram.:adj.}
\end{itemize}
\begin{itemize}
\item {Utilização:Bot.}
\end{itemize}
\begin{itemize}
\item {Grp. gram.:M.}
\end{itemize}
\begin{itemize}
\item {Proveniência:(Do gr. \textunderscore pteron\textunderscore  + \textunderscore karpos\textunderscore )}
\end{itemize}
Que tem excrescências membranosas, em fórma de asas, (falando-se de frutos).
Gênero de árvores leguminosas da Ásia tropical.
\section{Pterocéfalo}
\begin{itemize}
\item {Grp. gram.:m.}
\end{itemize}
\begin{itemize}
\item {Proveniência:(Do gr. \textunderscore pteron\textunderscore  + \textunderscore kephale\textunderscore )}
\end{itemize}
Gênero de plantas dipsáceas.
\section{Pterocéphalo}
\begin{itemize}
\item {Grp. gram.:m.}
\end{itemize}
\begin{itemize}
\item {Proveniência:(Do gr. \textunderscore pteron\textunderscore  + \textunderscore kephale\textunderscore )}
\end{itemize}
Gênero de plantas dipsáceas.
\section{Pterócera}
\begin{itemize}
\item {Grp. gram.:f.}
\end{itemize}
\begin{itemize}
\item {Proveniência:(Do gr. \textunderscore pteron\textunderscore  + \textunderscore keras\textunderscore )}
\end{itemize}
Mollusco fóssil da série mesozóica.
\section{Pterodáctilo}
\begin{itemize}
\item {Grp. gram.:adj.}
\end{itemize}
\begin{itemize}
\item {Utilização:Zool.}
\end{itemize}
\begin{itemize}
\item {Proveniência:(Do gr. \textunderscore pteron\textunderscore  + \textunderscore daktulos\textunderscore )}
\end{itemize}
Que tem os dedos ligados por uma membrana.
\section{Pterodáctylo}
\begin{itemize}
\item {Grp. gram.:adj.}
\end{itemize}
\begin{itemize}
\item {Utilização:Zool.}
\end{itemize}
\begin{itemize}
\item {Proveniência:(Do gr. \textunderscore pteron\textunderscore  + \textunderscore daktulos\textunderscore )}
\end{itemize}
Que tem os dedos ligados por uma membrana.
\section{Pterodonte}
\begin{itemize}
\item {Grp. gram.:m.}
\end{itemize}
\begin{itemize}
\item {Proveniência:(Do gr. \textunderscore pteron\textunderscore  + \textunderscore odous\textunderscore )}
\end{itemize}
Gênero de plantas leguminosas.
\section{Pteróforo}
\begin{itemize}
\item {Grp. gram.:adj.}
\end{itemize}
\begin{itemize}
\item {Grp. gram.:M. pl.}
\end{itemize}
\begin{itemize}
\item {Proveniência:(Do gr. \textunderscore pteron\textunderscore  + \textunderscore phoros\textunderscore )}
\end{itemize}
Que tem asas.
Gênero de insectos lepidópteros nocturnos.
\section{Pterógono}
\begin{itemize}
\item {Grp. gram.:adj.}
\end{itemize}
\begin{itemize}
\item {Utilização:Bot.}
\end{itemize}
\begin{itemize}
\item {Proveniência:(Do gr. \textunderscore pteron\textunderscore  + \textunderscore gonos\textunderscore )}
\end{itemize}
Provido de ângulos membranosos.
\section{Pteróide}
\begin{itemize}
\item {Grp. gram.:adj.}
\end{itemize}
\begin{itemize}
\item {Proveniência:(Do gr. \textunderscore pteron\textunderscore  + \textunderscore eidos\textunderscore )}
\end{itemize}
Que tem fórma ou apparência de asa.
\section{Pteroídeo}
\begin{itemize}
\item {Grp. gram.:adj.}
\end{itemize}
\begin{itemize}
\item {Utilização:Bot.}
\end{itemize}
\begin{itemize}
\item {Proveniência:(Do gr. \textunderscore pteron\textunderscore  + \textunderscore eidos\textunderscore )}
\end{itemize}
Que tem fórma ou apparência de asa.
\section{Pterólitho}
\begin{itemize}
\item {Grp. gram.:m.}
\end{itemize}
\begin{itemize}
\item {Utilização:Miner.}
\end{itemize}
\begin{itemize}
\item {Proveniência:(Do gr. \textunderscore pteron\textunderscore  + \textunderscore lithos\textunderscore )}
\end{itemize}
Mistura de mica preta com vários mineraes.
\section{Pterólito}
\begin{itemize}
\item {Grp. gram.:m.}
\end{itemize}
\begin{itemize}
\item {Utilização:Miner.}
\end{itemize}
\begin{itemize}
\item {Proveniência:(Do gr. \textunderscore pteron\textunderscore  + \textunderscore lithos\textunderscore )}
\end{itemize}
Mistura de mica preta com vários mineraes.
\section{Pterónia}
\begin{itemize}
\item {Grp. gram.:f.}
\end{itemize}
Gênero de plantas, da fam. das compostas.
\section{Pteróphoro}
\begin{itemize}
\item {Grp. gram.:adj.}
\end{itemize}
\begin{itemize}
\item {Grp. gram.:M. pl.}
\end{itemize}
\begin{itemize}
\item {Proveniência:(Do gr. \textunderscore pteron\textunderscore  + \textunderscore phoros\textunderscore )}
\end{itemize}
Que tem asas.
Gênero de insectos lepidópteros nocturnos.
\section{Pterópode}
\begin{itemize}
\item {Grp. gram.:adj.}
\end{itemize}
\begin{itemize}
\item {Utilização:Zool.}
\end{itemize}
\begin{itemize}
\item {Proveniência:(Do gr. \textunderscore pteron\textunderscore  + \textunderscore pous\textunderscore , \textunderscore podos\textunderscore )}
\end{itemize}
Que tem pés em fórma de barbatanas.
\section{Pterosáurio}
\begin{itemize}
\item {fónica:sau}
\end{itemize}
\begin{itemize}
\item {Grp. gram.:m.}
\end{itemize}
\begin{itemize}
\item {Proveniência:(Do gr. \textunderscore pteron\textunderscore  + \textunderscore sauros\textunderscore )}
\end{itemize}
Reptil voador e marinho, fossilizado.
\section{Pterospermo}
\begin{itemize}
\item {Grp. gram.:m.}
\end{itemize}
\begin{itemize}
\item {Proveniência:(Do gr. \textunderscore pteron\textunderscore  + \textunderscore sperma\textunderscore )}
\end{itemize}
Gênero de plantas byttneriáceas.
\section{Pterósporo}
\begin{itemize}
\item {Grp. gram.:m.}
\end{itemize}
Gênero de plantas dicotyledóneas.
\section{Pterossáurio}
\begin{itemize}
\item {Grp. gram.:m.}
\end{itemize}
\begin{itemize}
\item {Proveniência:(Do gr. \textunderscore pteron\textunderscore  + \textunderscore sauros\textunderscore )}
\end{itemize}
Reptil voador e marinho, fossilizado.
\section{Pterostelma}
\begin{itemize}
\item {Grp. gram.:f.}
\end{itemize}
\begin{itemize}
\item {Proveniência:(Do gr. \textunderscore pteron\textunderscore  + \textunderscore stelma\textunderscore )}
\end{itemize}
Gênero de plantas asclepiadáceas.
\section{Pterostigma}
\begin{itemize}
\item {Grp. gram.:f.}
\end{itemize}
Gênero de plantas escrofularíneas.
\section{Ptérula}
\begin{itemize}
\item {Grp. gram.:f.}
\end{itemize}
Gênero de cogumelos.
\section{Pubescência}
\begin{itemize}
\item {Grp. gram.:f.}
\end{itemize}
Estado de pubescente.
Conjunto de pêlos finos e curtos, que revestem a epiderme de certos órgãos ou de certos frutos.
Puberdade.
\section{Pubescente}
\begin{itemize}
\item {Grp. gram.:adj.}
\end{itemize}
\begin{itemize}
\item {Proveniência:(Lat. \textunderscore pubescens\textunderscore )}
\end{itemize}
Coberto de pêlos finos e curtos.
Púbere.
\section{Pubescer}
\begin{itemize}
\item {Grp. gram.:v. i.}
\end{itemize}
\begin{itemize}
\item {Grp. gram.:M.}
\end{itemize}
\begin{itemize}
\item {Proveniência:(Lat. \textunderscore pubescere\textunderscore )}
\end{itemize}
Chegar á puberdade; tornar-se púbere.
O mesmo que \textunderscore puberdade\textunderscore . Cf. Filinto, VI, 300.
\section{Pubiano}
\begin{itemize}
\item {Grp. gram.:adj.}
\end{itemize}
O mesmo que \textunderscore púbico\textunderscore .
\section{Púbico}
\begin{itemize}
\item {Grp. gram.:adj.}
\end{itemize}
Relativo ao púbis.
\section{Pubicórneo}
\begin{itemize}
\item {Grp. gram.:adj.}
\end{itemize}
\begin{itemize}
\item {Proveniência:(Do lat. \textunderscore pubis\textunderscore  + \textunderscore cornu\textunderscore )}
\end{itemize}
Que tem os chifres revestidos de pêlos.
\section{Púbis}
\begin{itemize}
\item {Grp. gram.:m.}
\end{itemize}
\begin{itemize}
\item {Proveniência:(Lat. \textunderscore pubes\textunderscore )}
\end{itemize}
Parte ínfero-anterior do osso ilíaco.
Parte mediana e inferior da região hypogástrica.
\section{Publicação}
\begin{itemize}
\item {Grp. gram.:f.}
\end{itemize}
\begin{itemize}
\item {Proveniência:(Lat. \textunderscore publicatio\textunderscore )}
\end{itemize}
Acto ou effeito de publicar.
Aquillo que se publica.
Trabalho literário, scientífico ou artístico, que se publica pela imprensa.
Livro; folheto.
\section{Publicador}
\begin{itemize}
\item {Grp. gram.:m.  e  adj.}
\end{itemize}
\begin{itemize}
\item {Proveniência:(Lat. \textunderscore publicator\textunderscore )}
\end{itemize}
O que publica.
\section{Pública-fórma}
\begin{itemize}
\item {Grp. gram.:f.}
\end{itemize}
Cópia authêntica de um documento.
\section{Publicamente}
\begin{itemize}
\item {Grp. gram.:adv.}
\end{itemize}
De modo público; em público; á vista de todos.
\section{Publicano}
\begin{itemize}
\item {Grp. gram.:m.}
\end{itemize}
\begin{itemize}
\item {Proveniência:(Lat. \textunderscore publicanus\textunderscore )}
\end{itemize}
Cobrador de rendimentos públicos, entre os Romanos.
\section{Publicar}
\begin{itemize}
\item {Grp. gram.:v. t.}
\end{itemize}
\begin{itemize}
\item {Proveniência:(Lat. \textunderscore publicare\textunderscore )}
\end{itemize}
Tornar público; proclamar; vulgarizar.
Editar.
Annunciar, proferir.
\section{Publícia}
\begin{itemize}
\item {Grp. gram.:adj. f.}
\end{itemize}
\begin{itemize}
\item {Proveniência:(Do lat. \textunderscore Publicius\textunderscore , n. p.)}
\end{itemize}
Dizia-se de uma das collinas de Roma. Cf. Castilho, \textunderscore Fastos\textunderscore , III, 35.
\section{Publicidade}
\begin{itemize}
\item {Grp. gram.:f.}
\end{itemize}
Qualidade do que é público.
Vulgarização.
\section{Publicismo}
\begin{itemize}
\item {Grp. gram.:m.}
\end{itemize}
\begin{itemize}
\item {Utilização:Neol.}
\end{itemize}
Profissão de publicista.
Conjunto dos publicistas.
\section{Publicista}
\begin{itemize}
\item {Grp. gram.:m.  e  f.}
\end{itemize}
\begin{itemize}
\item {Proveniência:(De \textunderscore público\textunderscore )}
\end{itemize}
Pessôa, que escreve sôbre direito público ou sôbre politica.
Escritor público.
\section{Público}
\begin{itemize}
\item {Grp. gram.:adj.}
\end{itemize}
\begin{itemize}
\item {Grp. gram.:M.}
\end{itemize}
\begin{itemize}
\item {Proveniência:(Lat. \textunderscore publicus\textunderscore )}
\end{itemize}
Pertencente ou relativo a um povo ou ao povo: \textunderscore interesses públicos\textunderscore .
Que serve para uso de todos: \textunderscore fonte pública\textunderscore .
Commum.
Relativo á governação de um país.
Que é do conhecimento de todos; notório.
Manifesto.
O povo em geral.
Auditório.
\section{Publícola}
\begin{itemize}
\item {Grp. gram.:m.}
\end{itemize}
\begin{itemize}
\item {Proveniência:(Lat. \textunderscore publicola\textunderscore )}
\end{itemize}
Aquelle que ama o povo, aquelle que o lisonjeia; democrata. Cf. C. Lobo, \textunderscore Sát. de Juv.\textunderscore , I, 30.
\section{Pubo}
\begin{itemize}
\item {Grp. gram.:adj.}
\end{itemize}
\begin{itemize}
\item {Utilização:Bras. do N}
\end{itemize}
\begin{itemize}
\item {Proveniência:(De \textunderscore pubar\textunderscore )}
\end{itemize}
Fermentado; podre.
\section{Púbrico}
\textunderscore adj.\textunderscore  (e der.)
(Fórma ant. de \textunderscore público\textunderscore , etc.)
\section{Puça}
\begin{itemize}
\item {Grp. gram.:f.}
\end{itemize}
Erva-dos-muros, planta ampelídea do Brasil.
\section{Puçá}
\begin{itemize}
\item {Grp. gram.:m.}
\end{itemize}
\begin{itemize}
\item {Utilização:Bras. do N}
\end{itemize}
\begin{itemize}
\item {Grp. gram.:m.}
\end{itemize}
\begin{itemize}
\item {Utilização:Bras}
\end{itemize}
\begin{itemize}
\item {Proveniência:(T. tupi)}
\end{itemize}
Borlas de algodão, com que se enfeitam as rêdes.
Instrumento de pescar camarões, o mesmo que \textunderscore jereré\textunderscore .
\section{Pucaçu}
\begin{itemize}
\item {Grp. gram.:m.}
\end{itemize}
\begin{itemize}
\item {Utilização:Bras. do N}
\end{itemize}
Grande pomba silvestre.
\section{Púcara}
\begin{itemize}
\item {Grp. gram.:f.}
\end{itemize}
O mesmo que \textunderscore púcaro\textunderscore .
\section{Pucareiro}
\begin{itemize}
\item {Grp. gram.:adj.}
\end{itemize}
\begin{itemize}
\item {Utilização:Prov.}
\end{itemize}
Relativo a púcaro.
Que tem a apparência ou tamanho de púcaro: \textunderscore o vizinho encontrou na horta um sapo pucareiro\textunderscore .
\section{Pucarinha}
\begin{itemize}
\item {Grp. gram.:f.}
\end{itemize}
Espécie de jôgo popular.
\section{Púcaro}
\begin{itemize}
\item {Grp. gram.:m.}
\end{itemize}
\begin{itemize}
\item {Utilização:Des.}
\end{itemize}
Pequeno vaso com asa, geralmente destinado a extrahir líquidos de outros vasos maiores.
\textunderscore Púcaro de água\textunderscore , o mesmo que \textunderscore merenda\textunderscore . Cf. F. Manuel, \textunderscore Carta de Guia\textunderscore , 109. Em vez de \textunderscore púcaro de água\textunderscore , chama-se hoje \textunderscore copo de água\textunderscore  a uma ligeira refeição, depois de certas ceremónias ecclesiásticas.
\section{Puerilizar-se}
\begin{itemize}
\item {Grp. gram.:v. p.}
\end{itemize}
Tornar-se pueril. Cf. Camillo, \textunderscore Esqueleto\textunderscore , 3.^a ed., 48.
\section{Puerilmente}
\begin{itemize}
\item {Grp. gram.:adv.}
\end{itemize}
De modo pueril.
\section{Puérpera}
\begin{itemize}
\item {Grp. gram.:f.  e  adj.}
\end{itemize}
\begin{itemize}
\item {Proveniência:(Lat. \textunderscore puerpera\textunderscore )}
\end{itemize}
Diz-se da mulher parturiente.
\section{Puerperal}
\begin{itemize}
\item {Grp. gram.:adj.}
\end{itemize}
Relativo á puérpera ou ao parto: \textunderscore febre puerperal\textunderscore .
\section{Puerpério}
\begin{itemize}
\item {Grp. gram.:m.}
\end{itemize}
\begin{itemize}
\item {Proveniência:(Lat. \textunderscore puerperium\textunderscore )}
\end{itemize}
Ânsias e dôres da mulher parturiente.
\section{Puf!}
\begin{itemize}
\item {Grp. gram.:interj.}
\end{itemize}
(designativa de cansaço, enfado, etc.) Cf. Camillo, \textunderscore Doze Casam.\textunderscore , 101 e 187.
\section{Pufo}
\begin{itemize}
\item {Grp. gram.:m.}
\end{itemize}
Instrumento de ferro, a que os tanoeiros aquecem uma das extremidades, para alargar os batoques.
\section{Púgil}
\begin{itemize}
\item {Grp. gram.:m.}
\end{itemize}
\begin{itemize}
\item {Proveniência:(Lat. \textunderscore pugil\textunderscore )}
\end{itemize}
Athleta; pugilista.
\section{Pugilar}
\begin{itemize}
\item {Grp. gram.:m.}
\end{itemize}
\begin{itemize}
\item {Proveniência:(Lat. \textunderscore pugillaris\textunderscore )}
\end{itemize}
Tábua encerada, em que antigamente se escrevia. Cf Castilho, \textunderscore Fastos\textunderscore , I, 318.
\section{Pugilato}
\begin{itemize}
\item {Grp. gram.:m.}
\end{itemize}
\begin{itemize}
\item {Utilização:Fig.}
\end{itemize}
\begin{itemize}
\item {Proveniência:(Lat. \textunderscore pugilatus\textunderscore )}
\end{itemize}
Acto de lutar com os punhos.
Discussão acalorada.
\section{Pugilista}
\begin{itemize}
\item {Grp. gram.:m.  e  f.}
\end{itemize}
\begin{itemize}
\item {Proveniência:(Do lat. \textunderscore pugil\textunderscore )}
\end{itemize}
Pessôa, que briga, servindo-se dos punhos ou dando murros.
\section{Pugillar}
\begin{itemize}
\item {Grp. gram.:m.}
\end{itemize}
\begin{itemize}
\item {Proveniência:(Lat. \textunderscore pugillaris\textunderscore )}
\end{itemize}
Tábua encerada, em que antigamente se escrevia. Cf Castilho, \textunderscore Fastos\textunderscore , I, 318.
\section{Pugillo}
\begin{itemize}
\item {Grp. gram.:m.}
\end{itemize}
\begin{itemize}
\item {Utilização:P. us.}
\end{itemize}
\begin{itemize}
\item {Proveniência:(Lat. \textunderscore pugillus\textunderscore )}
\end{itemize}
Porção de qualquer coisa, que se póde abranger entre o dedo pollegar, o indicador e o maior.
Magote.
Troço: \textunderscore bradam pugillos de soldados\textunderscore .
\section{Pugilo}
\begin{itemize}
\item {Grp. gram.:m.}
\end{itemize}
\begin{itemize}
\item {Utilização:P. us.}
\end{itemize}
\begin{itemize}
\item {Proveniência:(Lat. \textunderscore pugillus\textunderscore )}
\end{itemize}
Porção de qualquer coisa, que se póde abranger entre o dedo polegar, o indicador e o maior.
Magote.
Troço: \textunderscore bradam pugilos de soldados\textunderscore .
\section{Pugilómetro}
\begin{itemize}
\item {Grp. gram.:m.}
\end{itemize}
\begin{itemize}
\item {Proveniência:(Do lat. \textunderscore pugil\textunderscore  + gr. \textunderscore metron\textunderscore )}
\end{itemize}
Instrumento, para avaliar o impulso dado pelo punho.
\section{Pugna}
\begin{itemize}
\item {Grp. gram.:f.}
\end{itemize}
\begin{itemize}
\item {Proveniência:(Lat. \textunderscore pugna\textunderscore )}
\end{itemize}
Acto de pugnar; briga; peleja.
\section{Pugnace}
\begin{itemize}
\item {Grp. gram.:adj.}
\end{itemize}
O mesmo que \textunderscore pugnaz\textunderscore . Cf. Filinto, V, 142.
\section{Pugnacidade}
\begin{itemize}
\item {Grp. gram.:f.}
\end{itemize}
\begin{itemize}
\item {Proveniência:(Lat. \textunderscore pugnacitas\textunderscore )}
\end{itemize}
Qualidade do que é pugnaz; tendência para a briga.
\section{Pugnador}
\begin{itemize}
\item {Grp. gram.:adj.}
\end{itemize}
\begin{itemize}
\item {Proveniência:(Lat. \textunderscore pugnator\textunderscore )}
\end{itemize}
Que pugna; que combate; que defende. Cf. Arn. Gama, \textunderscore Últ. Dona.\textunderscore , 92.
\section{Pugnar}
\begin{itemize}
\item {Grp. gram.:v. i.}
\end{itemize}
\begin{itemize}
\item {Proveniência:(Lat. \textunderscore pugnare\textunderscore )}
\end{itemize}
Combater.
Brigar.
Tomar a defesa.
Desavir-se; discutir acaloradamente.
\section{Pugnaz}
\begin{itemize}
\item {Grp. gram.:adj.}
\end{itemize}
\begin{itemize}
\item {Proveniência:(Lat. \textunderscore pugnax\textunderscore )}
\end{itemize}
Que pugna; que tem tendências bellicosas.
\section{Púia}
\begin{itemize}
\item {Grp. gram.:f.}
\end{itemize}
\begin{itemize}
\item {Utilização:Prov.}
\end{itemize}
\begin{itemize}
\item {Utilização:trasm.}
\end{itemize}
Pé de craveiro.
Rebento de árvore.
Tacha ou pequeno prego para os sapatos.
(Contr. de \textunderscore peúia\textunderscore , de pé? Cp. \textunderscore peúga\textunderscore )
\section{Puia}
\begin{itemize}
\item {Grp. gram.:f.}
\end{itemize}
Gênero de plantas bromeliáceas.
\section{Puicobejes}
\begin{itemize}
\item {Grp. gram.:m. pl.}
\end{itemize}
Tríbo de Índios do Brasil, nas matas do Tocantins.
\section{Puíta}
\begin{itemize}
\item {Grp. gram.:f.}
\end{itemize}
Espécie de instrumento musical dos Negros, feito de um tronco oco, tapado na parte mais larga por uma pelle. Cf. V. Pinheiro, \textunderscore Ilhas de S. Thomé\textunderscore , 31; Beaurepaire-Rohan, \textunderscore Diccion. de Vocab. Bras.\textunderscore 
\section{Pujança}
\begin{itemize}
\item {Grp. gram.:f.}
\end{itemize}
\begin{itemize}
\item {Utilização:Geol.}
\end{itemize}
Qualidade do que é pujante.
O mesmo que \textunderscore possança\textunderscore .
(Cast. \textunderscore pujanza\textunderscore )
\section{Pujante}
\begin{itemize}
\item {Grp. gram.:adj.}
\end{itemize}
\begin{itemize}
\item {Proveniência:(De \textunderscore pujar\textunderscore )}
\end{itemize}
Que tem grande fôrça.
Possante.
Que tem poderio.
Magnificente.
Altivo; denodado.
\section{Pujar}
\begin{itemize}
\item {Grp. gram.:v. t.}
\end{itemize}
\begin{itemize}
\item {Grp. gram.:V. i.}
\end{itemize}
Exceder; vencer.
Esforçar-se.
(Cast. \textunderscore pujar\textunderscore )
\section{Pul}
\begin{itemize}
\item {Grp. gram.:m.}
\end{itemize}
Nome genérico das moédas de cobre, na Pérsia.
\section{Pula}
\begin{itemize}
\item {Grp. gram.:f.}
\end{itemize}
\begin{itemize}
\item {Proveniência:(Fr. \textunderscore poule\textunderscore )}
\end{itemize}
Aquillo que se aposta ao jôgo.
Bolo, formado pelas apostas dos jogadores.
\section{Pulante}
\begin{itemize}
\item {Grp. gram.:adj.}
\end{itemize}
Que pula.
\section{Pulão}
\begin{itemize}
\item {Grp. gram.:m.}
\end{itemize}
\begin{itemize}
\item {Utilização:Ant.}
\end{itemize}
\begin{itemize}
\item {Proveniência:(Fr. \textunderscore poulain\textunderscore )}
\end{itemize}
Peão; homem da plebe.
\section{Pular}
\begin{itemize}
\item {Grp. gram.:v. i.}
\end{itemize}
\begin{itemize}
\item {Proveniência:(Lat. \textunderscore pullulare\textunderscore )}
\end{itemize}
Dar pulos; saltar.
Agitar-se; tumultuar.
Desenvolver-se com rapidez.
Pullular.
Melhorar; engrandecer-se.
\section{Pulmonária}
\begin{itemize}
\item {Grp. gram.:f.}
\end{itemize}
\begin{itemize}
\item {Proveniência:(De \textunderscore pulmão\textunderscore )}
\end{itemize}
Gênero de plantas borragíneas.
Líchen parasito de algumas árvores.
\section{Pulmonia}
\begin{itemize}
\item {Grp. gram.:f.}
\end{itemize}
(V.pneumonia)
\section{Pulmonite}
\begin{itemize}
\item {Grp. gram.:f.}
\end{itemize}
O mesmo que \textunderscore pulmonia\textunderscore .
\section{Pulmotuberculose}
\begin{itemize}
\item {Grp. gram.:f.}
\end{itemize}
\begin{itemize}
\item {Proveniência:(De \textunderscore pulmão\textunderscore  + \textunderscore tuberculose\textunderscore )}
\end{itemize}
Tuberculose dos pulmões.
\section{Pulo}
\begin{itemize}
\item {Grp. gram.:m.}
\end{itemize}
\begin{itemize}
\item {Grp. gram.:Loc.}
\end{itemize}
\begin{itemize}
\item {Utilização:fam.}
\end{itemize}
\begin{itemize}
\item {Proveniência:(De \textunderscore pular\textunderscore )}
\end{itemize}
Salto.
Agitação.
Pulsação.
\textunderscore Dar pulo de corça\textunderscore , enfurecer-se.
\section{Pulpitamente}
\begin{itemize}
\item {Grp. gram.:adv.}
\end{itemize}
\begin{itemize}
\item {Utilização:Des.}
\end{itemize}
\begin{itemize}
\item {Proveniência:(De \textunderscore púlpito\textunderscore )}
\end{itemize}
Do púlpito; á maneira do prègador:«\textunderscore ...quando não falamos academica nem pulpitamente.\textunderscore »Filinto, XIII, 115.
\section{Púlpito}
\begin{itemize}
\item {Grp. gram.:m.}
\end{itemize}
\begin{itemize}
\item {Utilização:Ant.}
\end{itemize}
\begin{itemize}
\item {Utilização:Fig.}
\end{itemize}
\begin{itemize}
\item {Proveniência:(Lat. \textunderscore pulpitum\textunderscore )}
\end{itemize}
Tríbuna ou tablado alto, donde se prega, ordinariamente dentro das igrejas.
Qualquer tribuna.
O mesmo que \textunderscore palco\textunderscore . Cf. Michaëlis, \textunderscore Notas Vicent.\textunderscore , I, 16.
Eloquência sagrada.
Armação, donde os cerieiros penduram as torcidas para fazer velas.
\section{Pulpo}
\begin{itemize}
\item {Grp. gram.:m.}
\end{itemize}
\begin{itemize}
\item {Utilização:Ant.}
\end{itemize}
\begin{itemize}
\item {Proveniência:(Lat. \textunderscore pulpus\textunderscore )}
\end{itemize}
O mesmo que \textunderscore polvo\textunderscore . Cf. Pant. de Aveiro, \textunderscore Itiner.\textunderscore , 17 v.^o, (2.^a ed.).
\section{Pulsação}
\begin{itemize}
\item {Grp. gram.:f.}
\end{itemize}
\begin{itemize}
\item {Proveniência:(Lat. \textunderscore pulsatio\textunderscore )}
\end{itemize}
Acto ou effeito de pulsar.
\section{Pulsar}
\begin{itemize}
\item {Grp. gram.:v. t.}
\end{itemize}
\begin{itemize}
\item {Grp. gram.:V. i.}
\end{itemize}
\begin{itemize}
\item {Proveniência:(Lat. \textunderscore pulsare\textunderscore )}
\end{itemize}
Impellir.
Agitar, abalar.
Tocar.
Agitar-se.
Palpitar; latejar: \textunderscore pulsava-lhe apressado o coração\textunderscore .
Repercutir-se, batendo:«\textunderscore aquelles badaladaes funebres pulsavam no coração da viuva, como rebates á penitência.\textunderscore »Camillo, \textunderscore Volcões\textunderscore , 109.
\section{Pulsátil}
\begin{itemize}
\item {Grp. gram.:adj.}
\end{itemize}
Que pulsa.
\section{Pulsatila}
\begin{itemize}
\item {Grp. gram.:f.}
\end{itemize}
Espécie de anêmona medicinal, (\textunderscore anemona pulsatilla\textunderscore ).
\section{Pulsatilha}
\begin{itemize}
\item {Grp. gram.:f.}
\end{itemize}
O mesmo que \textunderscore pulsatilla\textunderscore .
\section{Pulsatilla}
\begin{itemize}
\item {Grp. gram.:f.}
\end{itemize}
Espécie de anêmona medicinal, (\textunderscore anemona pulsatilla\textunderscore ).
\section{Pulsativo}
\begin{itemize}
\item {Grp. gram.:adj.}
\end{itemize}
Que faz pulsar.
Acompanhado ou caracterizado por pulsações.
\section{Pulsear}
\begin{itemize}
\item {Grp. gram.:v. i.}
\end{itemize}
Avaliar com outrem a fôrça do pulso, apoiando os cotovelos sôbre um ponto e travando as mãos direitas.
\section{Pulseira}
\begin{itemize}
\item {Grp. gram.:f.}
\end{itemize}
Ornato circular para os pulsos.
\section{Pulsímetro}
\begin{itemize}
\item {Grp. gram.:m.}
\end{itemize}
\begin{itemize}
\item {Proveniência:(Do lat. \textunderscore pulsus\textunderscore  + gr. \textunderscore metron\textunderscore )}
\end{itemize}
Instrumento, para avaliar o número das pulsações arteriaes no espaço de um minuto.
Esphygmógrapho.
\section{Pulso}
\begin{itemize}
\item {Grp. gram.:m.}
\end{itemize}
\begin{itemize}
\item {Utilização:Ext.}
\end{itemize}
\begin{itemize}
\item {Utilização:Fig.}
\end{itemize}
\begin{itemize}
\item {Proveniência:(Lat. \textunderscore pulsus\textunderscore )}
\end{itemize}
Parte do antebraço, junto á mão.
Pulsação arterial, que se manifesta principalmente no pulso: \textunderscore pulso agitado\textunderscore .
Mão.
Fôrça, vigor: \textunderscore aquelle homem tem pulso\textunderscore .
\section{Pulsómetro}
\begin{itemize}
\item {Grp. gram.:m.}
\end{itemize}
(V.pulsímetro)
\section{Pultáceo}
\begin{itemize}
\item {Grp. gram.:adj.}
\end{itemize}
\begin{itemize}
\item {Proveniência:(Do lat. \textunderscore puls\textunderscore , \textunderscore pultis\textunderscore )}
\end{itemize}
Semelhante a papas; que apresenta o aspecto de papas: \textunderscore há diphterías pultáceas...\textunderscore , \textunderscore Tumores pultáceos...\textunderscore 
\section{Púlvego}
\begin{itemize}
\item {Grp. gram.:adv.}
\end{itemize}
\begin{itemize}
\item {Utilização:Ant.}
\end{itemize}
O mesmo que \textunderscore público\textunderscore .
\section{Pulvéreo}
\begin{itemize}
\item {Grp. gram.:adj.}
\end{itemize}
\begin{itemize}
\item {Utilização:Poét.}
\end{itemize}
\begin{itemize}
\item {Proveniência:(Lat. \textunderscore pulvereus\textunderscore )}
\end{itemize}
Relativo a pó; reduzido a pó; que tem a natureza do pó.
\section{Pulverescência}
\begin{itemize}
\item {Grp. gram.:f.}
\end{itemize}
O mesmo que \textunderscore pulverulência\textunderscore .
\section{Pulverinho}
\begin{itemize}
\item {Grp. gram.:m.}
\end{itemize}
\begin{itemize}
\item {Utilização:Prov.}
\end{itemize}
\begin{itemize}
\item {Proveniência:(Do lat. \textunderscore pulvis\textunderscore )}
\end{itemize}
Poeira, agitada pelo vento em remoínho.
\section{Pulverização}
\begin{itemize}
\item {Grp. gram.:f.}
\end{itemize}
Acto ou effeito de pulverizar.
\section{Pulverizador}
\begin{itemize}
\item {Grp. gram.:adj.}
\end{itemize}
\begin{itemize}
\item {Grp. gram.:M.}
\end{itemize}
Que pulveriza.
Aquillo que pulveriza.
Instrumento para pulverizar.
\section{Pulverizar}
\begin{itemize}
\item {Grp. gram.:v. t.}
\end{itemize}
\begin{itemize}
\item {Utilização:Fig.}
\end{itemize}
\begin{itemize}
\item {Proveniência:(Lat. \textunderscore pulverizare\textunderscore )}
\end{itemize}
Converter em pó.
Reduzir a pequenos fragmentos.
Cobrir de pó; polvilhar.
Desbaratar inteiramente.
Destruír; refutar completamente: \textunderscore pulverizar argumentos\textunderscore .
Injectar ou diffundir (um líquido), em gotas pequeníssimas como chuva tenuíssima.
\section{Pulveroso}
\begin{itemize}
\item {Grp. gram.:adj.}
\end{itemize}
\begin{itemize}
\item {Proveniência:(Do lat. \textunderscore pulvis\textunderscore )}
\end{itemize}
Poeirento; pulverulento.
\section{Pulverulência}
\begin{itemize}
\item {Grp. gram.:f.}
\end{itemize}
Estado do que é pulverulento.
\section{Pungir}
\begin{itemize}
\item {Grp. gram.:v. t.}
\end{itemize}
\begin{itemize}
\item {Grp. gram.:V. i.}
\end{itemize}
\begin{itemize}
\item {Proveniência:(Lat. \textunderscore pungere\textunderscore )}
\end{itemize}
Ferir, picar.
Affligir.
Torturar.
Extrahir com dôr.
Estimular.
Começar a apontar (a vegetação que rompe do solo).
Começar a nascer ou a apontar (a barba):«\textunderscore tinha dezasete anos; pungia-me um buçozinho\textunderscore ». M. Assis, \textunderscore B. Cubas\textunderscore .
\section{Pungitivo}
\begin{itemize}
\item {Grp. gram.:adj.}
\end{itemize}
\begin{itemize}
\item {Proveniência:(De \textunderscore pungir\textunderscore )}
\end{itemize}
Pungente; penetrante.
\section{Pungo}
\begin{itemize}
\item {Grp. gram.:f.}
\end{itemize}
\begin{itemize}
\item {Utilização:T. da Áfr. Or. Port}
\end{itemize}
Mulhér, que ensina a um noivo ou a uma noiva os deveres conjugaes.
\section{Punguista}
\begin{itemize}
\item {Grp. gram.:m.}
\end{itemize}
\begin{itemize}
\item {Utilização:Bras}
\end{itemize}
\begin{itemize}
\item {Proveniência:(De \textunderscore punga\textunderscore )}
\end{itemize}
Boticário charlatão, sem diploma.
\section{Punguixi}
\begin{itemize}
\item {Grp. gram.:m.}
\end{itemize}
Árvore de Angola.
\section{Punhada}
\begin{itemize}
\item {Grp. gram.:f.}
\end{itemize}
\begin{itemize}
\item {Proveniência:(Do b. lat. \textunderscore pugnata\textunderscore )}
\end{itemize}
Pancada com o punho; murro.
\section{Punhado}
\begin{itemize}
\item {Grp. gram.:m.}
\end{itemize}
\begin{itemize}
\item {Utilização:Fig.}
\end{itemize}
\begin{itemize}
\item {Proveniência:(De \textunderscore punho\textunderscore )}
\end{itemize}
Porção, que se póde conter na mão fechada.
Mão-cheia.
Pequena quantidade, pequeno número.
\section{Punhal}
\begin{itemize}
\item {Grp. gram.:m.}
\end{itemize}
\begin{itemize}
\item {Utilização:Fig.}
\end{itemize}
\begin{itemize}
\item {Proveniência:(De \textunderscore punho\textunderscore )}
\end{itemize}
Pequena arma branca, constituida por uma lâmina perfurante e um cabo geralmente em cruz.
Tudo que offende gravemente.
\section{Punhalada}
\begin{itemize}
\item {Grp. gram.:f.}
\end{itemize}
\begin{itemize}
\item {Utilização:Fig.}
\end{itemize}
Golpe de punhal.
Coisa que offende muito; golpe moral.
\section{Punhar}
\begin{itemize}
\item {Grp. gram.:v. i.}
\end{itemize}
\begin{itemize}
\item {Utilização:Ant.}
\end{itemize}
O mesmo que \textunderscore punir\textunderscore ^2.
(Por \textunderscore pugnar\textunderscore )
\section{Punhete}
\begin{itemize}
\item {fónica:nhê}
\end{itemize}
\begin{itemize}
\item {Grp. gram.:m.}
\end{itemize}
\begin{itemize}
\item {Proveniência:(De \textunderscore punho\textunderscore )}
\end{itemize}
O mesmo que \textunderscore mitene\textunderscore .
\section{Punho}
\begin{itemize}
\item {Grp. gram.:m.}
\end{itemize}
\begin{itemize}
\item {Utilização:Bras. do N}
\end{itemize}
\begin{itemize}
\item {Utilização:Prov.}
\end{itemize}
\begin{itemize}
\item {Utilização:minh.}
\end{itemize}
\begin{itemize}
\item {Grp. gram.:Pl.}
\end{itemize}
\begin{itemize}
\item {Utilização:Marn.}
\end{itemize}
\begin{itemize}
\item {Proveniência:(Do lat. \textunderscore pugnus\textunderscore )}
\end{itemize}
A mão fechada.
Pulso.
Tira, mais ou menos larga, em que terminam as mangas e que cerca o pulso.
Parte por onde se empunham certos instrumentos ou utensílios: \textunderscore o punho da espada\textunderscore .
Pequeno cabo ou corda, que segura as rêdes.
Golla de certas peças de vestuario, que cercam o pescoço: \textunderscore o punho do babeiro\textunderscore .
Duas peças de madeira, uma para cada mão, com as quaes se enchem de sal as canastras que o levam á eira.
\section{Punho-punhete}
\begin{itemize}
\item {Grp. gram.:m.}
\end{itemize}
Espécie de jôgo popular. Cf. \textunderscore Eufrosina\textunderscore , 281.
\section{Punibilidade}
\begin{itemize}
\item {Grp. gram.:f.}
\end{itemize}
\begin{itemize}
\item {Utilização:Neol.}
\end{itemize}
Qualidade de punível. Cf. Deusdado, \textunderscore Ensino Carcer.\textunderscore , 274.
\section{Punicáceas}
\begin{itemize}
\item {Grp. gram.:f. pl.}
\end{itemize}
Família de plantas, que tem por typo a romeira.
(Fem. pl. de \textunderscore punicáceo\textunderscore )
\section{Punicáceo}
\begin{itemize}
\item {Grp. gram.:adj.}
\end{itemize}
Relativo ou semelhante á romeira.
(Cp. \textunderscore puníceo\textunderscore )
\section{Punição}
\begin{itemize}
\item {Grp. gram.:f.}
\end{itemize}
\begin{itemize}
\item {Proveniência:(Lat. \textunderscore punitio\textunderscore )}
\end{itemize}
Acto ou effeito de punir^1.
Pena, castigo.
\section{Puníceo}
\begin{itemize}
\item {Grp. gram.:adj.}
\end{itemize}
\begin{itemize}
\item {Utilização:Poét.}
\end{itemize}
\begin{itemize}
\item {Proveniência:(Lat. \textunderscore puniceus\textunderscore )}
\end{itemize}
Vermelho; que é da côr da roman.
\section{Punicina}
\begin{itemize}
\item {Grp. gram.:f.}
\end{itemize}
\begin{itemize}
\item {Utilização:Chím.}
\end{itemize}
\begin{itemize}
\item {Proveniência:(Do lat. bot. \textunderscore punica malus\textunderscore , romeira)}
\end{itemize}
Substância acre e incrystallizável da casca da romeira.
\section{Púnico}
\begin{itemize}
\item {Grp. gram.:adj.}
\end{itemize}
\begin{itemize}
\item {Grp. gram.:M.}
\end{itemize}
\begin{itemize}
\item {Proveniência:(Lat. \textunderscore punicus\textunderscore )}
\end{itemize}
Relativo a Carthago ou aos Carthagineses.
Pérfido, traidor.
Língua dos Carthagineses.
\section{Punida}
\begin{itemize}
\item {Grp. gram.:f.}
\end{itemize}
\begin{itemize}
\item {Utilização:Gír.}
\end{itemize}
Palha.
(Or. ind.)
\section{Punidor}
\begin{itemize}
\item {Grp. gram.:m.  e  adj.}
\end{itemize}
O que pune.
\section{Punir}
\begin{itemize}
\item {Grp. gram.:v. t.}
\end{itemize}
\begin{itemize}
\item {Proveniência:(Lat. \textunderscore punire\textunderscore )}
\end{itemize}
Infligir pena a; dar castigo a; servir de castigo a.
Reprimir.
\section{Punir}
\begin{itemize}
\item {Grp. gram.:v. i.}
\end{itemize}
\begin{itemize}
\item {Utilização:Pop.}
\end{itemize}
Lutar em defesa: \textunderscore os pais punem sempre pelos filhos\textunderscore .
Esforçar-se por vingança.
(Corr. de \textunderscore pugnar\textunderscore )
\section{Punitivo}
\begin{itemize}
\item {Grp. gram.:adj.}
\end{itemize}
Que pune.
\section{Punível}
\begin{itemize}
\item {Grp. gram.:adj.}
\end{itemize}
\begin{itemize}
\item {Proveniência:(De \textunderscore punir\textunderscore ^1)}
\end{itemize}
Que é digno de castigo.
\section{Punivelmente}
\begin{itemize}
\item {Grp. gram.:adv.}
\end{itemize}
De modo punível; merecendo castigo. Cf. Sousa Monteiro, \textunderscore Cav. Falstaff\textunderscore , pról.
\section{Pianépsias}
\begin{itemize}
\item {Grp. gram.:f. pl.}
\end{itemize}
Antigas festas de Atenas, instituídas por Teseu em honra de Apolo.
\section{Pianépsio}
\begin{itemize}
\item {Grp. gram.:m.}
\end{itemize}
Quinto mês do ano ático, mês em que se celebravam as pianépsias.
\section{Piartrose}
\begin{itemize}
\item {Grp. gram.:f.}
\end{itemize}
\begin{itemize}
\item {Utilização:Med.}
\end{itemize}
\begin{itemize}
\item {Proveniência:(Do gr. \textunderscore puon\textunderscore  + \textunderscore arthron\textunderscore )}
\end{itemize}
Artrite purulenta.
\section{Picnito}
\begin{itemize}
\item {Grp. gram.:m.}
\end{itemize}
\begin{itemize}
\item {Utilização:Miner.}
\end{itemize}
\begin{itemize}
\item {Proveniência:(Do gr. \textunderscore puknos\textunderscore )}
\end{itemize}
Variedade de topázio.
\section{Picnometria}
\begin{itemize}
\item {Grp. gram.:f.}
\end{itemize}
\begin{itemize}
\item {Proveniência:(Do gr. \textunderscore puknos\textunderscore  + \textunderscore metron\textunderscore )}
\end{itemize}
Medida da densidade dos corpos.
\section{Picnómetro}
\begin{itemize}
\item {Grp. gram.:m.}
\end{itemize}
\begin{itemize}
\item {Proveniência:(Do gr. \textunderscore puknos\textunderscore  + \textunderscore metron\textunderscore )}
\end{itemize}
Instrumento, para medir a densidade do vinho e de outros líquidos.
\section{Picnoscopia}
\begin{itemize}
\item {Grp. gram.:f.}
\end{itemize}
\begin{itemize}
\item {Proveniência:(Do gr. \textunderscore puknos\textunderscore  + \textunderscore kopein\textunderscore )}
\end{itemize}
O mesmo que \textunderscore radioscopia\textunderscore . Cf. Verg. Machado, \textunderscore Raios X\textunderscore .
\section{Picnose}
\begin{itemize}
\item {Grp. gram.:f.}
\end{itemize}
\begin{itemize}
\item {Utilização:Physiol.}
\end{itemize}
\begin{itemize}
\item {Proveniência:(Gr. \textunderscore puknosis\textunderscore )}
\end{itemize}
Condensação da cromatina na célula.
\section{Picnostilo}
\begin{itemize}
\item {Grp. gram.:m.}
\end{itemize}
\begin{itemize}
\item {Proveniência:(Do gr. \textunderscore puknos\textunderscore  + \textunderscore stulos\textunderscore )}
\end{itemize}
Pequeno intercolúnio.
Edifício, em que as colunas têm pequeno intervalo entre si.
\section{Pictácio}
\begin{itemize}
\item {Grp. gram.:m.}
\end{itemize}
\begin{itemize}
\item {Proveniência:(Gr. \textunderscore puktakion\textunderscore )}
\end{itemize}
Quadro, em que se inscreviam os nomes dos juízes do pugilato, entre os Gregos.
\section{Pielite}
\begin{itemize}
\item {Grp. gram.:f.}
\end{itemize}
\begin{itemize}
\item {Proveniência:(Do gr. \textunderscore puelos\textunderscore , bacia)}
\end{itemize}
Inflamação da membrana mucosa que reveste os bacinetes dos rins.
\section{Piemia}
\begin{itemize}
\item {Grp. gram.:f.}
\end{itemize}
\begin{itemize}
\item {Proveniência:(Do gr. \textunderscore puon\textunderscore  + \textunderscore haima\textunderscore )}
\end{itemize}
Infecção geral, produzida pelos estreptococcos. Cf. \textunderscore Jorn.-do-Comm.\textunderscore , do Rio, de 4-V-901.
\section{Piêmico}
\begin{itemize}
\item {Grp. gram.:adj.}
\end{itemize}
Relativo á piemia.
\section{Pigarga}
\begin{itemize}
\item {Grp. gram.:f.}
\end{itemize}
O mesmo que \textunderscore pigargo\textunderscore .
\section{Pigargo}
\begin{itemize}
\item {Grp. gram.:m.}
\end{itemize}
\begin{itemize}
\item {Proveniência:(Lat. \textunderscore pygargus\textunderscore )}
\end{itemize}
Espécie de grande águia aquática.
\section{Pigericu}
\begin{itemize}
\item {Grp. gram.:m.}
\end{itemize}
Planta anonácea.
\section{Pigídio}
\begin{itemize}
\item {Grp. gram.:m.}
\end{itemize}
\begin{itemize}
\item {Utilização:Geol.}
\end{itemize}
\begin{itemize}
\item {Proveniência:(Gr. \textunderscore pugidion\textunderscore )}
\end{itemize}
Peça, que limita posteriormente o corpo fóssil dos trilobitas.
\section{Pigmeia}
\begin{itemize}
\item {Grp. gram.:f.}
\end{itemize}
Flexão fem. de \textunderscore pigmeu\textunderscore :«\textunderscore vês aquela giganta? descalça-lhe os coturnos e ficará pigmeia...\textunderscore »D. Ant. da Costa, \textunderscore Três Mundos\textunderscore , 101.
\section{Pigmeu}
\begin{itemize}
\item {Grp. gram.:m.}
\end{itemize}
\begin{itemize}
\item {Utilização:Ext.}
\end{itemize}
\begin{itemize}
\item {Utilização:Fig.}
\end{itemize}
\begin{itemize}
\item {Proveniência:(Lat. \textunderscore pygmaeus\textunderscore )}
\end{itemize}
Membro de uma nação fabulosa, cujos habitantes, segundo os poétas, tinham apenas um côvado de altura.
Homem de pequena estatura.
Anão.
Homem insignificante, sem intelligência ou sem outro mérito.
\section{Pigopagia}
\begin{itemize}
\item {Grp. gram.:f.}
\end{itemize}
Estado ou qualidade de pigópago.
\section{Pigópago}
\begin{itemize}
\item {Grp. gram.:f.}
\end{itemize}
\begin{itemize}
\item {Utilização:Terat.}
\end{itemize}
\begin{itemize}
\item {Proveniência:(Do gr. \textunderscore puge\textunderscore  + \textunderscore pageis\textunderscore )}
\end{itemize}
Monstro, composto de dois indivíduos, ligados pelas nádegas.
\section{Piína}
\begin{itemize}
\item {Grp. gram.:f.}
\end{itemize}
\begin{itemize}
\item {Proveniência:(Do gr. \textunderscore puon\textunderscore )}
\end{itemize}
Um dos princípios coaguláveis do plasma do sangue.
O mesmo que \textunderscore metalbumina\textunderscore .
\section{Pilágora}
\begin{itemize}
\item {Grp. gram.:m.}
\end{itemize}
\begin{itemize}
\item {Proveniência:(Gr. \textunderscore pulagoras\textunderscore )}
\end{itemize}
Orador deputado á assembleia dos anfictiões na antiga Grécia. Cf. Latino, \textunderscore Or. da Corôa\textunderscore , 32.
\section{Puçanga}
\begin{itemize}
\item {Grp. gram.:f.}
\end{itemize}
\begin{itemize}
\item {Utilização:Bras. do N}
\end{itemize}
Medicamento caseiro; mèzinha.
(Do tupi \textunderscore poçang\textunderscore )
\section{Puçazeiro}
\begin{itemize}
\item {Grp. gram.:m.}
\end{itemize}
\begin{itemize}
\item {Utilização:Bras}
\end{itemize}
\begin{itemize}
\item {Proveniência:(De \textunderscore pussá\textunderscore ^2)}
\end{itemize}
Árvore melastomácea da América.
\section{Punaca}
\begin{itemize}
\item {Grp. gram.:f.}
\end{itemize}
Árvore fructífera da Índia portuguesa.
\section{Purgante}
\begin{itemize}
\item {Grp. gram.:adj.}
\end{itemize}
\begin{itemize}
\item {Grp. gram.:M.}
\end{itemize}
\begin{itemize}
\item {Proveniência:(Lat. \textunderscore purgans\textunderscore )}
\end{itemize}
Que faz purgar.
Purga.
\section{Purgar}
\begin{itemize}
\item {Grp. gram.:v. t.}
\end{itemize}
\begin{itemize}
\item {Grp. gram.:V. i.}
\end{itemize}
\begin{itemize}
\item {Utilização:Prov.}
\end{itemize}
\begin{itemize}
\item {Utilização:minh.}
\end{itemize}
\begin{itemize}
\item {Grp. gram.:V. p.}
\end{itemize}
\begin{itemize}
\item {Proveniência:(Lat. \textunderscore purgare\textunderscore )}
\end{itemize}
Tornar puro; purificar, limpar.
Privar de impurezas ou de substâncias nocivas.
Desembaraçar ou limpar (as vias digestivas).
Expellir maus humores ou pus.
Diz-se da videira, quando a larga a flôr.
Tomar purgante.
\section{Purgatina}
\begin{itemize}
\item {Grp. gram.:f.}
\end{itemize}
\begin{itemize}
\item {Utilização:Pharm.}
\end{itemize}
Medicamento laxativo.
\section{Purgativo}
\begin{itemize}
\item {Grp. gram.:adj.}
\end{itemize}
\begin{itemize}
\item {Grp. gram.:M.}
\end{itemize}
\begin{itemize}
\item {Proveniência:(Lat. \textunderscore purgativus\textunderscore )}
\end{itemize}
Purgante, purificativo.
Purga.
\section{Purgatol}
\begin{itemize}
\item {Grp. gram.:m.}
\end{itemize}
\begin{itemize}
\item {Utilização:Pharm.}
\end{itemize}
O mesmo que \textunderscore purgatina\textunderscore .
\section{Purgatório}
\begin{itemize}
\item {Grp. gram.:adj.}
\end{itemize}
\begin{itemize}
\item {Grp. gram.:M.}
\end{itemize}
\begin{itemize}
\item {Proveniência:(Lat. \textunderscore purgatorius\textunderscore )}
\end{itemize}
O mesmo que \textunderscore purgativo\textunderscore .
Lugar ou estado, em que as almas dos justos, saídas dêste mundo sem têr expiado completamente as suas faltas para com a justiça divina, acabam de expiá-las, para serem admittidas na bem-aventurança.
\section{Purgueira}
\begin{itemize}
\item {Grp. gram.:f.}
\end{itemize}
\begin{itemize}
\item {Proveniência:(De \textunderscore purga\textunderscore )}
\end{itemize}
Planta euphorbiácea do Brasil, de suco medicinal e semente purgativa, (\textunderscore jatropha curcas\textunderscore , Lin.).
\section{Puri}
\begin{itemize}
\item {Grp. gram.:m.}
\end{itemize}
\begin{itemize}
\item {Utilização:Bras}
\end{itemize}
Espécie de mandioca.
\section{Puridade}
\begin{itemize}
\item {Grp. gram.:f.}
\end{itemize}
\begin{itemize}
\item {Utilização:Des.}
\end{itemize}
\begin{itemize}
\item {Grp. gram.:Loc. adv.}
\end{itemize}
\begin{itemize}
\item {Proveniência:(Lat. \textunderscore puritas\textunderscore )}
\end{itemize}
O mesmo que \textunderscore pureza\textunderscore .
Segrêdo, confidência.
\textunderscore Á puridade\textunderscore , em segrêdo, particularmente.
\textunderscore Escrivão da puridade\textunderscore , antigo secretário da côrte.
\section{Purificação}
\begin{itemize}
\item {Grp. gram.:f.}
\end{itemize}
\begin{itemize}
\item {Proveniência:(Lat. \textunderscore purificatio\textunderscore )}
\end{itemize}
Acto ou effeito de purificar.
Ablução litúrgica.
Festa, que a Igreja cathólica celebra em 2 de Fevereiro.
\section{Purificador}
\begin{itemize}
\item {Grp. gram.:adj.}
\end{itemize}
\begin{itemize}
\item {Grp. gram.:M.}
\end{itemize}
Que purifica.
Aquillo que purifica.
Vaso, com que se enxagua a bôca e se lava a ponta dos dedos depois das refeições.
Pano, com que o sacerdote, na Missa, limpa o cálice depois de commungar.
\section{Purificante}
\begin{itemize}
\item {Grp. gram.:adj.}
\end{itemize}
\begin{itemize}
\item {Proveniência:(Lat. \textunderscore purificans\textunderscore )}
\end{itemize}
Que purifica.
\section{Purificar}
\begin{itemize}
\item {Grp. gram.:v. t.}
\end{itemize}
\begin{itemize}
\item {Proveniência:(Lat. \textunderscore purificare\textunderscore )}
\end{itemize}
Tornar puro.
Libertar de substâncias impuras.
Tirar mácula a.
Mundificar; santificar.
\section{Purificativo}
\begin{itemize}
\item {Grp. gram.:adj.}
\end{itemize}
O mesmo que \textunderscore purificante\textunderscore .
\section{Purificatório}
\begin{itemize}
\item {Grp. gram.:adj.}
\end{itemize}
Que serve para purificar. Cf. Júl. Dinis, \textunderscore Morgadinha\textunderscore , 173.
\section{Puriforme}
\begin{itemize}
\item {Grp. gram.:adj.}
\end{itemize}
\begin{itemize}
\item {Proveniência:(Do lat. \textunderscore pus\textunderscore , \textunderscore puris\textunderscore  + \textunderscore forma\textunderscore )}
\end{itemize}
Semelhante ao pus.
\section{Puris}
\begin{itemize}
\item {Grp. gram.:m. pl.}
\end{itemize}
Antiga nação nômade de Índios do Brasil, de que restam algumas cabildas nas matas de San-Paulo.
\section{Purismo}
\begin{itemize}
\item {Grp. gram.:m.}
\end{itemize}
\begin{itemize}
\item {Proveniência:(De \textunderscore puro\textunderscore )}
\end{itemize}
Carácter dos escritores, que se occupam excessivamente da pureza da linguagem.
\section{Purista}
\begin{itemize}
\item {Grp. gram.:m. ,  f.  e  adj.}
\end{itemize}
\begin{itemize}
\item {Proveniência:(De \textunderscore puro\textunderscore )}
\end{itemize}
Pessôa, excessivamente escrupulosa, quanto á pureza da linguagem.
\section{Puritanismo}
\begin{itemize}
\item {Grp. gram.:m.}
\end{itemize}
\begin{itemize}
\item {Utilização:Ext.}
\end{itemize}
\begin{itemize}
\item {Proveniência:(De \textunderscore puritano\textunderscore )}
\end{itemize}
Seita protestante, que suppunha interpretar melhor que ninguém a letra da \textunderscore Bíblia\textunderscore .
Qualidade da pessôa, que alardeia grande rigidez de princípios.
\section{Puritano}
\begin{itemize}
\item {Grp. gram.:m.}
\end{itemize}
\begin{itemize}
\item {Utilização:Ext.}
\end{itemize}
\begin{itemize}
\item {Grp. gram.:Adj.}
\end{itemize}
\begin{itemize}
\item {Proveniência:(Ingl. \textunderscore puritan\textunderscore )}
\end{itemize}
Sectário do puritanismo.
Homem, que alardeia grande austeridade.
Relativo ao puritanismo.
\section{Puro}
\begin{itemize}
\item {Grp. gram.:adj.}
\end{itemize}
\begin{itemize}
\item {Proveniência:(Lat. \textunderscore purus\textunderscore )}
\end{itemize}
Que não tem mistura.
Que não tem nada.
Que não soffreu alteração: \textunderscore vinho puro\textunderscore .
Transparente, límpido.
Immaculado: \textunderscore alma pura\textunderscore .
Innocente.
Virginal; casto.
Tranquillo, sereno: \textunderscore atmosphera pura\textunderscore .
Verdadeiro.
Vernáculo, castiço, (falando-se do estilo ou da linguagem).
Natural.
Sincero: \textunderscore pura verdade\textunderscore .
Mero.
Suave; mavioso: \textunderscore voz pura\textunderscore .
Que ainda não foi corrido, (falando-se do toiro).
\section{Puropurus}
\begin{itemize}
\item {Grp. gram.:m. pl.}
\end{itemize}
Indígenas do norte do Brasil.
\section{Pussanga}
\begin{itemize}
\item {Grp. gram.:f.}
\end{itemize}
\begin{itemize}
\item {Utilização:Bras. do N}
\end{itemize}
Medicamento caseiro; mèzinha.
(Do tupi \textunderscore poçang\textunderscore )
\section{Pussazeiro}
\begin{itemize}
\item {Grp. gram.:m.}
\end{itemize}
\begin{itemize}
\item {Utilização:Bras}
\end{itemize}
\begin{itemize}
\item {Proveniência:(De \textunderscore pussá\textunderscore ^2)}
\end{itemize}
Árvore melastomácea da América.
\section{Pústula}
\begin{itemize}
\item {Grp. gram.:f.}
\end{itemize}
\begin{itemize}
\item {Utilização:Fig.}
\end{itemize}
\begin{itemize}
\item {Utilização:Bot.}
\end{itemize}
\begin{itemize}
\item {Utilização:Med.}
\end{itemize}
\begin{itemize}
\item {Proveniência:(Lat. \textunderscore pustula\textunderscore )}
\end{itemize}
Pequeno tumor cutâneo, que termina por suppuração.
Chaga de mau carácter.
Corrupção; vício.
Saliência ou pequena elevação na haste ou nas fôlhas.
\textunderscore Pústula maligna\textunderscore , o mesmo que \textunderscore anthraz\textunderscore . Cf. Mac. Pinto, \textunderscore Comp. de Veter.\textunderscore , I, 138.
\section{Pustulento}
\begin{itemize}
\item {Grp. gram.:m.  e  adj.}
\end{itemize}
O que tem pústulas.
\section{Pustuloso}
\begin{itemize}
\item {Grp. gram.:adj.}
\end{itemize}
\begin{itemize}
\item {Proveniência:(Lat. \textunderscore pustulosos\textunderscore )}
\end{itemize}
Pustulento.
Que tem fórma ou natureza de pústula.
Caracterizado por pústulas.
\section{Puta}
\begin{itemize}
\item {Grp. gram.:f.}
\end{itemize}
\begin{itemize}
\item {Utilização:Pleb.}
\end{itemize}
Mulhér devassa; meretriz.
\section{Putativamente}
\begin{itemize}
\item {Grp. gram.:adv.}
\end{itemize}
De modo putativo; por supposição.
\section{Putativo}
\begin{itemize}
\item {Grp. gram.:adj.}
\end{itemize}
\begin{itemize}
\item {Proveniência:(Lat. \textunderscore putativus\textunderscore )}
\end{itemize}
Que se suppõe sêr o que não é; reputado:«\textunderscore ...pay putativo de Christo.\textunderscore »\textunderscore Luz e Calor\textunderscore , 601.
\section{Putauá}
\begin{itemize}
\item {Grp. gram.:m.}
\end{itemize}
Espécie de palmeira do Brasil.
\section{Puteal}
\begin{itemize}
\item {Grp. gram.:m.}
\end{itemize}
\begin{itemize}
\item {Proveniência:(Lat. \textunderscore putealis\textunderscore )}
\end{itemize}
Bocal de poço.
Pequeno muro de pedra á roda da bôca de um poço. Cf. Castilho, \textunderscore Fastos\textunderscore , II, 264.
\section{Pútega}
\begin{itemize}
\item {Grp. gram.:f.}
\end{itemize}
Planta cytínea, (\textunderscore cytinus hyposistis\textunderscore ).
Fruto da esteva.
\section{Puti}
\begin{itemize}
\item {Grp. gram.:m.}
\end{itemize}
Ave aquática africana, (\textunderscore cephalophus mergens\textunderscore ).
\section{Putirão}
\begin{itemize}
\item {Grp. gram.:m.}
\end{itemize}
\begin{itemize}
\item {Utilização:Bras}
\end{itemize}
O mesmo que \textunderscore muxirão\textunderscore .
\section{Putirom}
\begin{itemize}
\item {Grp. gram.:m.}
\end{itemize}
\begin{itemize}
\item {Utilização:Bras. do N}
\end{itemize}
O mesmo que \textunderscore muxirão\textunderscore .
\section{Putirum}
\begin{itemize}
\item {Grp. gram.:m.}
\end{itemize}
\begin{itemize}
\item {Utilização:Bras. do N}
\end{itemize}
O mesmo que \textunderscore muxirão\textunderscore .
\section{Puto}
\begin{itemize}
\item {Grp. gram.:adj.}
\end{itemize}
\begin{itemize}
\item {Utilização:Pleb.}
\end{itemize}
Corrompido, devasso:«\textunderscore ...o infame fez puto o coração, puto o talento.\textunderscore »Filinto, X, 131.
(Cp. \textunderscore puta\textunderscore )
\section{Putória}
\begin{itemize}
\item {Grp. gram.:f.}
\end{itemize}
Gênero de plantas rubiáceas.
\section{Putrecível}
\begin{itemize}
\item {Grp. gram.:adj.}
\end{itemize}
O mesmo que \textunderscore putrescivel\textunderscore . Cf. Camillo, \textunderscore Volcoens\textunderscore , 27.
\section{Putredinoso}
\begin{itemize}
\item {Grp. gram.:adj.}
\end{itemize}
\begin{itemize}
\item {Proveniência:(Do lat. \textunderscore putredo\textunderscore )}
\end{itemize}
Em que há podridão; corrompido. Cf. Camillo, \textunderscore Narcóticos\textunderscore , II, 286.
\section{Putrefacção}
\begin{itemize}
\item {Grp. gram.:f.}
\end{itemize}
\begin{itemize}
\item {Proveniência:(Lat. \textunderscore putrefactio\textunderscore )}
\end{itemize}
Acto ou effeito de putrefazer.
\section{Putrefaciente}
\begin{itemize}
\item {Grp. gram.:adj.}
\end{itemize}
\begin{itemize}
\item {Proveniência:(Lat. \textunderscore putrefaciens\textunderscore )}
\end{itemize}
Que putrefaz.
\section{Putrefactivo}
\begin{itemize}
\item {Grp. gram.:adj.}
\end{itemize}
O mesmo que \textunderscore putrefaciente\textunderscore .
\section{Putrefacto}
\begin{itemize}
\item {Grp. gram.:adj.}
\end{itemize}
\begin{itemize}
\item {Proveniência:(Lat. \textunderscore putrefactus\textunderscore )}
\end{itemize}
Que está em putrefacção.
Que apodreceu; corrupto.
\section{Putrefactório}
\begin{itemize}
\item {Grp. gram.:adj.}
\end{itemize}
O mesmo que \textunderscore putrefaciente\textunderscore .
\section{Putrefazer}
\begin{itemize}
\item {Grp. gram.:v. t.}
\end{itemize}
\begin{itemize}
\item {Proveniência:(Lat. \textunderscore putrefacere\textunderscore )}
\end{itemize}
Tornar podre.
Corromper.
\section{Putrefeito}
\begin{itemize}
\item {Grp. gram.:adj.}
\end{itemize}
O mesmo que \textunderscore putrefacto\textunderscore . Cf. Rui Barb., \textunderscore Réplica\textunderscore , 158.
\section{Putrescência}
\begin{itemize}
\item {Grp. gram.:f.}
\end{itemize}
Estado do que é putrescente.
\section{Putrescente}
\begin{itemize}
\item {Grp. gram.:adj.}
\end{itemize}
\begin{itemize}
\item {Proveniência:(Lat. \textunderscore putrescens\textunderscore )}
\end{itemize}
Que está apodrecendo; que começa a putrefazer-se.
\section{Putrescibilidade}
\begin{itemize}
\item {Grp. gram.:f.}
\end{itemize}
Qualidade do que é putrescivel.
\section{Putrescina}
\begin{itemize}
\item {Grp. gram.:f.}
\end{itemize}
\begin{itemize}
\item {Proveniência:(Do lat. \textunderscore putrescere\textunderscore )}
\end{itemize}
Ptomaína, que se encontra nos corpos putrefactos.
\section{Putrescivel}
\begin{itemize}
\item {Grp. gram.:adj.}
\end{itemize}
\begin{itemize}
\item {Proveniência:(Do lat. \textunderscore putrescere\textunderscore )}
\end{itemize}
Que póde apodrecer ou putrefazer-se.
\section{Pútrido}
\begin{itemize}
\item {Grp. gram.:adj.}
\end{itemize}
\begin{itemize}
\item {Proveniência:(Lat. \textunderscore putridus\textunderscore )}
\end{itemize}
Podre; corrupto.
Infectuoso, pestilencial.
\section{Putrificar}
\begin{itemize}
\item {Grp. gram.:v. t.}
\end{itemize}
O mesmo que \textunderscore putrefazer\textunderscore .
\section{Putrígeno}
\begin{itemize}
\item {Grp. gram.:adj.}
\end{itemize}
\begin{itemize}
\item {Proveniência:(Do lat. \textunderscore putris\textunderscore  + gr. \textunderscore genos\textunderscore )}
\end{itemize}
Que occasiona podridão ou putrefacção:«\textunderscore bactérias putrigenas.\textunderscore »R. Jorge, \textunderscore Boletim dos Serviços Sanit.\textunderscore , 67.
\section{Puxiri}
\begin{itemize}
\item {Grp. gram.:m.}
\end{itemize}
\begin{itemize}
\item {Utilização:Bras. do N}
\end{itemize}
O mesmo que \textunderscore puxuri\textunderscore .
\section{Puxirum}
\begin{itemize}
\item {Grp. gram.:m.}
\end{itemize}
\begin{itemize}
\item {Utilização:Bras}
\end{itemize}
O mesmo que \textunderscore muxirão\textunderscore .
\section{Puxo}
\begin{itemize}
\item {Grp. gram.:m.}
\end{itemize}
\begin{itemize}
\item {Utilização:P. us.}
\end{itemize}
\begin{itemize}
\item {Proveniência:(De \textunderscore puxar\textunderscore )}
\end{itemize}
Dôr do ânus, que precede ou acompanha uma evacuação diffícil.
Tenesmo.
Esforços, que a mulhér faz, para parir.
\section{Puxuri}
\begin{itemize}
\item {Grp. gram.:m.}
\end{itemize}
\begin{itemize}
\item {Utilização:Bras. do N}
\end{itemize}
Árvore laurácea do valle do Amazonas.
\section{Pyanépsias}
\begin{itemize}
\item {Grp. gram.:f. pl.}
\end{itemize}
Antigas festas de Athenas, instituídas por Theseu em honra de Apollo.
\section{Pyanépsio}
\begin{itemize}
\item {Grp. gram.:m.}
\end{itemize}
Quinto mês do anno áttico, mês em que se celebravam as pyanépsias.
\section{Pyarthrose}
\begin{itemize}
\item {Grp. gram.:f.}
\end{itemize}
\begin{itemize}
\item {Utilização:Med.}
\end{itemize}
\begin{itemize}
\item {Proveniência:(Do gr. \textunderscore puon\textunderscore  + \textunderscore arthron\textunderscore )}
\end{itemize}
Arthrite purulenta.
\section{Pycnito}
\begin{itemize}
\item {Grp. gram.:m.}
\end{itemize}
\begin{itemize}
\item {Utilização:Miner.}
\end{itemize}
\begin{itemize}
\item {Proveniência:(Do gr. \textunderscore puknos\textunderscore )}
\end{itemize}
Variedade de topázio.
\section{Pycnometria}
\begin{itemize}
\item {Grp. gram.:f.}
\end{itemize}
\begin{itemize}
\item {Proveniência:(Do gr. \textunderscore puknos\textunderscore  + \textunderscore metron\textunderscore )}
\end{itemize}
Medida da densidade dos corpos.
\section{Pycnómetro}
\begin{itemize}
\item {Grp. gram.:m.}
\end{itemize}
\begin{itemize}
\item {Proveniência:(Do gr. \textunderscore puknos\textunderscore  + \textunderscore metron\textunderscore )}
\end{itemize}
Instrumento, para medir a densidade do vinho e de outros líquidos.
\section{Pycnoscopia}
\begin{itemize}
\item {Grp. gram.:f.}
\end{itemize}
\begin{itemize}
\item {Proveniência:(Do gr. \textunderscore puknos\textunderscore  + \textunderscore kopein\textunderscore )}
\end{itemize}
O mesmo que \textunderscore radioscopia\textunderscore . Cf. Verg. Machado, \textunderscore Raios X\textunderscore .
\section{Pycnose}
\begin{itemize}
\item {Grp. gram.:f.}
\end{itemize}
\begin{itemize}
\item {Utilização:Physiol.}
\end{itemize}
\begin{itemize}
\item {Proveniência:(Gr. \textunderscore puknosis\textunderscore )}
\end{itemize}
Condensação da chromatina na céllula.
\section{Pycnostylo}
\begin{itemize}
\item {Grp. gram.:m.}
\end{itemize}
\begin{itemize}
\item {Proveniência:(Do gr. \textunderscore puknos\textunderscore  + \textunderscore stulos\textunderscore )}
\end{itemize}
Pequeno intercolúmnio.
Edifício, em que as columnas têm pequeno intervallo entre si.
\section{Pyctácio}
\begin{itemize}
\item {Grp. gram.:m.}
\end{itemize}
\begin{itemize}
\item {Proveniência:(Gr. \textunderscore puktakion\textunderscore )}
\end{itemize}
Quadro, em que se inscreviam os nomes dos juízes do pugilato, entre os Gregos.
\section{Pyelite}
\begin{itemize}
\item {Grp. gram.:f.}
\end{itemize}
\begin{itemize}
\item {Proveniência:(Do gr. \textunderscore puelos\textunderscore , bacia)}
\end{itemize}
Inflammação da membrana mucosa que reveste os bacinetes dos rins.
\section{Pyemia}
\begin{itemize}
\item {Grp. gram.:f.}
\end{itemize}
\begin{itemize}
\item {Proveniência:(Do gr. \textunderscore puon\textunderscore  + \textunderscore haima\textunderscore )}
\end{itemize}
Infecção geral, produzida pelos estreptococcos. Cf. \textunderscore Jorn.-do-Comm.\textunderscore , do Rio, de 4-V-901.
\section{Pyêmico}
\begin{itemize}
\item {Grp. gram.:adj.}
\end{itemize}
Relativo á pyemia.
\section{Pygarga}
\begin{itemize}
\item {Grp. gram.:f.}
\end{itemize}
O mesmo que \textunderscore pygargo\textunderscore .
\section{Pygargo}
\begin{itemize}
\item {Grp. gram.:m.}
\end{itemize}
\begin{itemize}
\item {Proveniência:(Lat. \textunderscore pygargus\textunderscore )}
\end{itemize}
Espécie de grande águia aquática.
\section{Pygericu}
\begin{itemize}
\item {Grp. gram.:m.}
\end{itemize}
Planta anonácea.
\section{Pygídio}
\begin{itemize}
\item {Grp. gram.:m.}
\end{itemize}
\begin{itemize}
\item {Utilização:Geol.}
\end{itemize}
\begin{itemize}
\item {Proveniência:(Gr. \textunderscore pugidion\textunderscore )}
\end{itemize}
Peça, que limita posteriormente o corpo fóssil dos trilobitas.
\section{Pygméa}
\begin{itemize}
\item {Grp. gram.:f.}
\end{itemize}
Flexão fem. de \textunderscore pygmeu\textunderscore :«\textunderscore vês aquella giganta? descalça-lhe os cothurnos e ficará pygmeia...\textunderscore »D. Ant. da Costa, \textunderscore Três Mundos\textunderscore , 101.
\section{Pygmeia}
\begin{itemize}
\item {Grp. gram.:f.}
\end{itemize}
Flexão fem. de \textunderscore pygmeu\textunderscore :«\textunderscore vês aquella giganta? descalça-lhe os cothurnos e ficará pygmeia...\textunderscore »D. Ant. da Costa, \textunderscore Três Mundos\textunderscore , 101.
\section{Pygmeu}
\begin{itemize}
\item {Grp. gram.:m.}
\end{itemize}
\begin{itemize}
\item {Utilização:Ext.}
\end{itemize}
\begin{itemize}
\item {Utilização:Fig.}
\end{itemize}
\begin{itemize}
\item {Proveniência:(Lat. \textunderscore pygmaeus\textunderscore )}
\end{itemize}
Membro de uma nação fabulosa, cujos habitantes, segundo os poétas, tinham apenas um côvado de altura.
Homem de pequena estatura.
Anão.
Homem insignificante, sem intelligência ou sem outro mérito.
\section{Pygopagia}
\begin{itemize}
\item {Grp. gram.:f.}
\end{itemize}
Estado ou qualidade de pygópago.
\section{Pygópago}
\begin{itemize}
\item {Grp. gram.:f.}
\end{itemize}
\begin{itemize}
\item {Utilização:Terat.}
\end{itemize}
\begin{itemize}
\item {Proveniência:(Do gr. \textunderscore puge\textunderscore  + \textunderscore pageis\textunderscore )}
\end{itemize}
Monstro, composto de dois indivíduos, ligados pelas nádegas.
\section{Pyína}
\begin{itemize}
\item {Grp. gram.:f.}
\end{itemize}
\begin{itemize}
\item {Proveniência:(Do gr. \textunderscore puon\textunderscore )}
\end{itemize}
Um dos princípios coaguláveis do plasma do sangue.
O mesmo que \textunderscore metalbumina\textunderscore .
\section{Pylágora}
\begin{itemize}
\item {Grp. gram.:m.}
\end{itemize}
\begin{itemize}
\item {Proveniência:(Gr. \textunderscore pulagoras\textunderscore )}
\end{itemize}
Orador deputado á assembleia dos amphictyões na antiga Grécia. Cf. Latino, \textunderscore Or. da Corôa\textunderscore , 32.
\section{Pirelióforo}
\begin{itemize}
\item {Grp. gram.:m.}
\end{itemize}
\begin{itemize}
\item {Proveniência:(Do gr. \textunderscore pur\textunderscore  + \textunderscore helios\textunderscore  + \textunderscore phoros\textunderscore )}
\end{itemize}
Aparelho, recentemente inventado pelo português Padre Himalaia, e que tem por fim determinar a natureza e origem do calor e da luz do Sol, estudar as propriedades da matéria, sob a influência de temperaturas muito elevadas e completar a escala das altas temperaturas.
\section{Pireliómetro}
\begin{itemize}
\item {Grp. gram.:m.}
\end{itemize}
\begin{itemize}
\item {Utilização:Phýs.}
\end{itemize}
\begin{itemize}
\item {Proveniência:(Do gr. \textunderscore pur\textunderscore  + \textunderscore helios\textunderscore  + \textunderscore metron\textunderscore )}
\end{itemize}
Instrumento, com que Pouillet conseguiu medir aproximadamente o calor emitido pelo Sol.
\section{Pireneíte}
\begin{itemize}
\item {Grp. gram.:f.}
\end{itemize}
\begin{itemize}
\item {Utilização:Miner.}
\end{itemize}
Variedade de granada dos Pirenéus.
\section{Pirenéu}
\begin{itemize}
\item {Grp. gram.:adj.}
\end{itemize}
\begin{itemize}
\item {Proveniência:(Lat. \textunderscore pyrenaeus\textunderscore )}
\end{itemize}
O mesmo que \textunderscore pirenaico\textunderscore :«\textunderscore ...as pirenéas cimas...\textunderscore »Filinto, III, 140.
\section{Pirênio}
\begin{itemize}
\item {Grp. gram.:m.}
\end{itemize}
O mesmo ou melhor que \textunderscore pireno\textunderscore .
\section{Pireno}
\begin{itemize}
\item {Grp. gram.:m.}
\end{itemize}
\begin{itemize}
\item {Proveniência:(Do gr. \textunderscore pur\textunderscore )}
\end{itemize}
Producto da destilação de madeira, que se encontra no óleo de carvão mineral.
\section{Pirenóide}
\begin{itemize}
\item {Grp. gram.:adj.}
\end{itemize}
\begin{itemize}
\item {Proveniência:(Gr. \textunderscore purenoides\textunderscore )}
\end{itemize}
Semelhante a um caroço.
\section{Pirenol}
\begin{itemize}
\item {Grp. gram.:m.}
\end{itemize}
Medicamento antiséptico, sedativo e expectorante.
\section{Pirético}
\begin{itemize}
\item {Grp. gram.:adj.}
\end{itemize}
\begin{itemize}
\item {Utilização:Med.}
\end{itemize}
\begin{itemize}
\item {Proveniência:(Do gr. \textunderscore puretos\textunderscore )}
\end{itemize}
O mesmo que \textunderscore febril\textunderscore . Cf. Camillo, \textunderscore Caveira\textunderscore , 99.
\section{Píreto}
\begin{itemize}
\item {Grp. gram.:m.}
\end{itemize}
O mesmo que \textunderscore píretro\textunderscore .
\section{Piretologia}
\begin{itemize}
\item {Grp. gram.:f.}
\end{itemize}
\begin{itemize}
\item {Proveniência:(Do gr. \textunderscore puretos\textunderscore  + \textunderscore logos\textunderscore )}
\end{itemize}
Estudo ou tratado á cêrca das febres.
\section{Piretológico}
\begin{itemize}
\item {Grp. gram.:adj.}
\end{itemize}
Relativo á piretologia.
\section{Piretologista}
\begin{itemize}
\item {Grp. gram.:m.}
\end{itemize}
Aquele que trata de piretologia.
\section{Píretro}
\begin{itemize}
\item {Grp. gram.:m.}
\end{itemize}
\begin{itemize}
\item {Proveniência:(Lat. \textunderscore pyrethrum\textunderscore )}
\end{itemize}
Gênero de plantas da fam. das compostas, herbáceas e vivazes.--R. Galvão diz \textunderscore pirétro\textunderscore , contra a prosódia etimológica.
\section{Pireu}
\begin{itemize}
\item {Grp. gram.:m.}
\end{itemize}
\begin{itemize}
\item {Proveniência:(Lat. \textunderscore pyreum\textunderscore )}
\end{itemize}
Altar de fogo, na religião dos magos.
\section{Pirexia}
\begin{itemize}
\item {fónica:csi}
\end{itemize}
\begin{itemize}
\item {Grp. gram.:f.}
\end{itemize}
\begin{itemize}
\item {Proveniência:(Gr. \textunderscore purexia\textunderscore )}
\end{itemize}
Estado febril; febre.
\section{Pirgo}
\begin{itemize}
\item {Grp. gram.:m.}
\end{itemize}
\begin{itemize}
\item {Proveniência:(Lat. \textunderscore pyrgus\textunderscore )}
\end{itemize}
Espécie de copo de madeira, com que se jogavam os dados, entre os antigos.
\section{Pirgocefalia}
\begin{itemize}
\item {Grp. gram.:f.}
\end{itemize}
Estado ou qualidade de pirgocéfalo.
\section{Pirgocéfalo}
\begin{itemize}
\item {Grp. gram.:m.}
\end{itemize}
\begin{itemize}
\item {Proveniência:(Do gr. \textunderscore purgos\textunderscore  + \textunderscore kephale\textunderscore )}
\end{itemize}
O mesmo que \textunderscore acrocéfalo\textunderscore .
\section{Pírico}
\begin{itemize}
\item {Grp. gram.:adj.}
\end{itemize}
\begin{itemize}
\item {Proveniência:(De \textunderscore pira\textunderscore )}
\end{itemize}
Relativo ao fogo.
\section{Piridina}
\begin{itemize}
\item {Grp. gram.:f.}
\end{itemize}
Producto químico, usado em inalações, no tratamento da asma e da dispneia cardíaca, e produzido pela destilação de ossos.
\section{Pirífora}
\begin{itemize}
\item {Grp. gram.:f.}
\end{itemize}
\begin{itemize}
\item {Utilização:Bras}
\end{itemize}
\begin{itemize}
\item {Proveniência:(Do gr. \textunderscore pur\textunderscore  + \textunderscore phoros\textunderscore )}
\end{itemize}
O mesmo que \textunderscore pirilampo\textunderscore .
\section{Pirilâmpico}
\begin{itemize}
\item {Grp. gram.:adj.}
\end{itemize}
Que tem luz com um pirilampo; fosforescente.
\section{Pirilampo}
\begin{itemize}
\item {Grp. gram.:m.}
\end{itemize}
\begin{itemize}
\item {Proveniência:(Gr. \textunderscore purilampis\textunderscore )}
\end{itemize}
Gênero de insectos coleópteros pentâmeros, que emitem uma luz fosforescente; vagalume.
\section{Pirite}
\begin{itemize}
\item {Grp. gram.:f.}
\end{itemize}
\begin{itemize}
\item {Proveniência:(Lat. \textunderscore pyrites\textunderscore )}
\end{itemize}
Sulfureto metálico, que tem a propriedade de se inflamar em dadas circunstâncias.
\textunderscore Pirite cúprica\textunderscore , o mesmo que \textunderscore calco-pirite\textunderscore .
\section{Piritífero}
\begin{itemize}
\item {Grp. gram.:adj.}
\end{itemize}
\begin{itemize}
\item {Proveniência:(Do lat. \textunderscore pyrites\textunderscore  + \textunderscore ferre\textunderscore )}
\end{itemize}
Que contém pirite.
\section{Piritiforme}
\begin{itemize}
\item {Grp. gram.:adj.}
\end{itemize}
\begin{itemize}
\item {Proveniência:(De \textunderscore pirite\textunderscore  + \textunderscore fórma\textunderscore )}
\end{itemize}
Que tem a fórma de pirite.
\section{Pirofobia}
\begin{itemize}
\item {Grp. gram.:f.}
\end{itemize}
\begin{itemize}
\item {Utilização:Med.}
\end{itemize}
\begin{itemize}
\item {Proveniência:(Do gr. \textunderscore pur\textunderscore  + \textunderscore phobein\textunderscore )}
\end{itemize}
Mêdo mórbido do fogo.
\section{Piróforo}
\begin{itemize}
\item {Grp. gram.:m.}
\end{itemize}
\begin{itemize}
\item {Grp. gram.:Pl.}
\end{itemize}
\begin{itemize}
\item {Proveniência:(Gr. \textunderscore purophoros\textunderscore )}
\end{itemize}
Composição química, que se inflama ao contacto do ar.
Espécie de sacerdotes que, entre os antigos Gregos, iam á frente do exército, levando vasos cheios de fogo, que êles atiravam contra o inimigo para dar sinal de combate, quando as trombetas ainda não tinham esta aplicação.
\section{Pirólatra}
\begin{itemize}
\item {Grp. gram.:m.}
\end{itemize}
\begin{itemize}
\item {Proveniência:(Do gr. \textunderscore pur\textunderscore  + \textunderscore latreuein\textunderscore )}
\end{itemize}
Adorador do fogo.
\section{Pirolatria}
\begin{itemize}
\item {Grp. gram.:f.}
\end{itemize}
Adoração do fogo.
(Cp. \textunderscore pirólatra\textunderscore )
\section{Pirolenhoso}
\begin{itemize}
\item {Grp. gram.:adj.}
\end{itemize}
\begin{itemize}
\item {Utilização:Chím.}
\end{itemize}
\begin{itemize}
\item {Proveniência:(Do gr. \textunderscore pur\textunderscore  + lat. \textunderscore lignum\textunderscore )}
\end{itemize}
Diz-se de um ácido, obtido pela destilação da madeira.
\section{Pirologia}
\begin{itemize}
\item {Grp. gram.:f.}
\end{itemize}
\begin{itemize}
\item {Proveniência:(Do gr. \textunderscore pur\textunderscore  + \textunderscore logos\textunderscore )}
\end{itemize}
Tratado á cêrca do fogo.
\section{Pirolusito}
\begin{itemize}
\item {Grp. gram.:m.}
\end{itemize}
\begin{itemize}
\item {Proveniência:(Do gr. \textunderscore pur\textunderscore  + \textunderscore lusis\textunderscore )}
\end{itemize}
Óxido de manganés.
\section{Piromaco}
\begin{itemize}
\item {Grp. gram.:adj.}
\end{itemize}
\begin{itemize}
\item {Utilização:Bot.}
\end{itemize}
\begin{itemize}
\item {Proveniência:(Do gr. \textunderscore pur\textunderscore  + \textunderscore makhe\textunderscore )}
\end{itemize}
Que produz centelhas, quando percutido com ferro.
\section{Piromancia}
\begin{itemize}
\item {Grp. gram.:f.}
\end{itemize}
\begin{itemize}
\item {Proveniência:(Lat. \textunderscore pyromantia\textunderscore )}
\end{itemize}
Previsão do futuro, por meio do fogo.
\section{Piromania}
\begin{itemize}
\item {Grp. gram.:f.}
\end{itemize}
\begin{itemize}
\item {Utilização:Med.}
\end{itemize}
\begin{itemize}
\item {Proveniência:(Do gr. \textunderscore pur\textunderscore  + \textunderscore mania\textunderscore )}
\end{itemize}
Monomania incendiária.
\section{Piromântico}
\begin{itemize}
\item {Grp. gram.:adj.}
\end{itemize}
Relativo á piromancia.
\section{Pirometria}
\begin{itemize}
\item {Grp. gram.:f.}
\end{itemize}
\begin{itemize}
\item {Proveniência:(De \textunderscore pirómetro\textunderscore )}
\end{itemize}
Arte de avaliar as altas temperaturas.
\section{Pirométrico}
\begin{itemize}
\item {Grp. gram.:adj.}
\end{itemize}
Relativo á pirometria.
\section{Pirómetro}
\begin{itemize}
\item {Grp. gram.:m.}
\end{itemize}
\begin{itemize}
\item {Proveniência:(Do gr. \textunderscore pur\textunderscore  + \textunderscore metron\textunderscore )}
\end{itemize}
Instrumento, para avaliar aproximadamente as altas temperaturas.
\section{Piromorfite}
\begin{itemize}
\item {Grp. gram.:m.}
\end{itemize}
\begin{itemize}
\item {Utilização:Miner.}
\end{itemize}
\begin{itemize}
\item {Proveniência:(Do gr. \textunderscore pur\textunderscore  + \textunderscore morphe\textunderscore )}
\end{itemize}
Um dos productos da alteração da galenite.
\section{Piromotor}
\begin{itemize}
\item {Grp. gram.:m.}
\end{itemize}
\begin{itemize}
\item {Proveniência:(Do lat. \textunderscore pyra\textunderscore  + \textunderscore motor\textunderscore )}
\end{itemize}
Aparelho agrícola, para produzir fogo no campo e conjurar o frio que danifica os sarmentos tenros.
\section{Pironomia}
\begin{itemize}
\item {Grp. gram.:f.}
\end{itemize}
\begin{itemize}
\item {Proveniência:(Do gr. \textunderscore pur\textunderscore  + \textunderscore nomos\textunderscore )}
\end{itemize}
Arte de regular a temperatura nas operações químicas.
\section{Pironomico}
\begin{itemize}
\item {Grp. gram.:adj.}
\end{itemize}
Relativo á pironomia.
\section{Piropina}
\begin{itemize}
\item {Grp. gram.:f.}
\end{itemize}
\begin{itemize}
\item {Proveniência:(De \textunderscore piropo\textunderscore )}
\end{itemize}
Substancia albuminóide e vermelha, extraida dos dentes do elefante.
\section{Piropincel}
\begin{itemize}
\item {Grp. gram.:m.}
\end{itemize}
\begin{itemize}
\item {Proveniência:(De \textunderscore piro...\textunderscore  + \textunderscore pincel\textunderscore )}
\end{itemize}
Instrumento, cuja ponta se torna candente á lâmpada, e cuja encandescência se mantém com uma corrente de ar, carregada de vapores de benzina.
\section{Piropo}
\begin{itemize}
\item {Grp. gram.:m.}
\end{itemize}
\begin{itemize}
\item {Proveniência:(Lat. \textunderscore pyropus\textunderscore )}
\end{itemize}
Liga de quatro partes de cobre e uma de oiro, usada pelos antigos.
Variedade de pedra preciosa.
\section{Piróscafo}
\begin{itemize}
\item {Grp. gram.:m.}
\end{itemize}
\begin{itemize}
\item {Proveniência:(Do gr. \textunderscore pur\textunderscore  + \textunderscore scophos\textunderscore )}
\end{itemize}
Nome, que se deu ao primeiro navio a vapor.
\section{Piroscopia}
\begin{itemize}
\item {Grp. gram.:f.}
\end{itemize}
\begin{itemize}
\item {Proveniência:(Do gr. \textunderscore pur\textunderscore  + \textunderscore skopein\textunderscore )}
\end{itemize}
Suposta arte de adivinhar, por meio das chamas dos sacrifícios antigos.
\section{Piroscópio}
\begin{itemize}
\item {Grp. gram.:m.}
\end{itemize}
\begin{itemize}
\item {Proveniência:(Do gr. \textunderscore pur\textunderscore  + \textunderscore skopein\textunderscore )}
\end{itemize}
Instrumento, para indicar que a temperatura atingiu determinado grau.
\section{Pirose}
\begin{itemize}
\item {Grp. gram.:f.}
\end{itemize}
\begin{itemize}
\item {Proveniência:(Gr. \textunderscore purosis\textunderscore )}
\end{itemize}
Sensação de ardor ou calor, que vai do estômago até á garganta; azia.
\section{Pitagorismo}
\begin{itemize}
\item {Grp. gram.:m.}
\end{itemize}
Doutrina de Pitágoras.
\section{Pitagorista}
\begin{itemize}
\item {Grp. gram.:m.}
\end{itemize}
Sectário do pitagorismo.
\section{Pitão}
\begin{itemize}
\item {Grp. gram.:m.}
\end{itemize}
\begin{itemize}
\item {Utilização:T. eccles}
\end{itemize}
\begin{itemize}
\item {Proveniência:(Lat. \textunderscore Python\textunderscore , n. p.)}
\end{itemize}
Serpente mitológica, morta por Apolo.
Mago; nigromante.
\section{Pitaula}
\begin{itemize}
\item {Grp. gram.:m.}
\end{itemize}
\begin{itemize}
\item {Proveniência:(Lat. \textunderscore pythaula\textunderscore )}
\end{itemize}
Aquele que, nos antigos teatros, acompanhava com o som da frauta o monólogo do actor.
\section{Pítia}
\begin{itemize}
\item {Grp. gram.:f.}
\end{itemize}
\begin{itemize}
\item {Proveniência:(Lat. \textunderscore pythia\textunderscore )}
\end{itemize}
Sacerdotisa de Apolo, a qual pronunciava oráculos em Delfos.
\section{Pítico}
\begin{itemize}
\item {Grp. gram.:adj.}
\end{itemize}
\begin{itemize}
\item {Proveniência:(Lat. \textunderscore pythicus\textunderscore )}
\end{itemize}
Relativo a Pítia.
Relativo aos jogos, que em Delfos se celebravam em honra de Apolo, Latona e Diana.
\section{Píton}
\begin{itemize}
\item {Grp. gram.:m.}
\end{itemize}
(V.pitão)
\section{Pitónico}
\begin{itemize}
\item {Grp. gram.:adj.}
\end{itemize}
\begin{itemize}
\item {Proveniência:(Lat. \textunderscore pythonicus\textunderscore )}
\end{itemize}
Relativo ao piton.
Nigromântico; mágico; diabólico. Cf. Arn. Gama, \textunderscore Motim\textunderscore , 226.
\section{Pitonisa}
\begin{itemize}
\item {Grp. gram.:f.}
\end{itemize}
\begin{itemize}
\item {Utilização:Ext.}
\end{itemize}
\begin{itemize}
\item {Proveniência:(Lat. \textunderscore pythonisa\textunderscore )}
\end{itemize}
Mulhér, que adivinhava.
Sacerdotisa de Apolo.
Profetisa.
\section{Pyreneíte}
\begin{itemize}
\item {Grp. gram.:f.}
\end{itemize}
\begin{itemize}
\item {Utilização:Miner.}
\end{itemize}
Variedade de granada dos Pyrenéus.
\section{Pyrenéu}
\begin{itemize}
\item {Grp. gram.:adj.}
\end{itemize}
\begin{itemize}
\item {Proveniência:(Lat. \textunderscore pyrenaeus\textunderscore )}
\end{itemize}
O mesmo que \textunderscore pyrenaico\textunderscore :«\textunderscore ...as pyrenéas cimas...\textunderscore »Filinto, III, 140.
\section{Pyrênio}
\begin{itemize}
\item {Grp. gram.:m.}
\end{itemize}
O mesmo ou melhor que \textunderscore pyreno\textunderscore .
\section{Pyreno}
\begin{itemize}
\item {Grp. gram.:m.}
\end{itemize}
\begin{itemize}
\item {Proveniência:(Do gr. \textunderscore pur\textunderscore )}
\end{itemize}
Producto da destillação de madeira, que se encontra no óleo de carvão mineral.
\section{Pyrenóide}
\begin{itemize}
\item {Grp. gram.:adj.}
\end{itemize}
\begin{itemize}
\item {Proveniência:(Gr. \textunderscore purenoides\textunderscore )}
\end{itemize}
Semelhante a um caroço.
\section{Pyrenol}
\begin{itemize}
\item {Grp. gram.:m.}
\end{itemize}
Medicamento antiséptico, sedativo e expectorante.
\section{Pýrethro}
\begin{itemize}
\item {Grp. gram.:m.}
\end{itemize}
\begin{itemize}
\item {Proveniência:(Lat. \textunderscore pyrethrum\textunderscore )}
\end{itemize}
Gênero de plantas da fam. das compostas, herbáceas e vivazes.--R. Galvão diz \textunderscore pirétro\textunderscore , contra a prosódia etymológica.
\section{Pyrético}
\begin{itemize}
\item {Grp. gram.:adj.}
\end{itemize}
\begin{itemize}
\item {Utilização:Med.}
\end{itemize}
\begin{itemize}
\item {Proveniência:(Do gr. \textunderscore puretos\textunderscore )}
\end{itemize}
O mesmo que \textunderscore febril\textunderscore . Cf. Camillo, \textunderscore Caveira\textunderscore , 99.
\section{Pýreto}
\begin{itemize}
\item {Grp. gram.:m.}
\end{itemize}
O mesmo que \textunderscore pýrethro\textunderscore .
\section{Pyretologia}
\begin{itemize}
\item {Grp. gram.:f.}
\end{itemize}
\begin{itemize}
\item {Proveniência:(Do gr. \textunderscore puretos\textunderscore  + \textunderscore logos\textunderscore )}
\end{itemize}
Estudo ou tratado á cêrca das febres.
\section{Pyretológico}
\begin{itemize}
\item {Grp. gram.:adj.}
\end{itemize}
Relativo á pyretologia.
\section{Pyretologista}
\begin{itemize}
\item {Grp. gram.:m.}
\end{itemize}
Aquelle que trata de pyretologia.
\section{Pyreu}
\begin{itemize}
\item {Grp. gram.:m.}
\end{itemize}
\begin{itemize}
\item {Proveniência:(Lat. \textunderscore pyreum\textunderscore )}
\end{itemize}
Altar de fogo, na religião dos magos.
\section{Pyrexia}
\begin{itemize}
\item {fónica:csi}
\end{itemize}
\begin{itemize}
\item {Grp. gram.:f.}
\end{itemize}
\begin{itemize}
\item {Proveniência:(Gr. \textunderscore purexia\textunderscore )}
\end{itemize}
Estado febril; febre.
\section{Pyrgo}
\begin{itemize}
\item {Grp. gram.:m.}
\end{itemize}
\begin{itemize}
\item {Proveniência:(Lat. \textunderscore pyrgus\textunderscore )}
\end{itemize}
Espécie de copo de madeira, com que se jogavam os dados, entre os antigos.
\section{Pyrgocephalia}
\begin{itemize}
\item {Grp. gram.:f.}
\end{itemize}
Estado ou qualidade de pyrgocéphalo.
\section{Pyrgocéphalo}
\begin{itemize}
\item {Grp. gram.:m.}
\end{itemize}
\begin{itemize}
\item {Proveniência:(Do gr. \textunderscore purgos\textunderscore  + \textunderscore kephale\textunderscore )}
\end{itemize}
O mesmo que \textunderscore acrocéphalo\textunderscore .
\section{Pyrheliómetro}
\begin{itemize}
\item {Grp. gram.:m.}
\end{itemize}
\begin{itemize}
\item {Utilização:Phýs.}
\end{itemize}
\begin{itemize}
\item {Proveniência:(Do gr. \textunderscore pur\textunderscore  + \textunderscore helios\textunderscore  + \textunderscore metron\textunderscore )}
\end{itemize}
Instrumento, com que Pouillet conseguiu medir aproximadamente o calor emittido pelo Sol.
\section{Pyrhelióphoro}
\begin{itemize}
\item {Grp. gram.:m.}
\end{itemize}
\begin{itemize}
\item {Proveniência:(Do gr. \textunderscore pur\textunderscore  + \textunderscore helios\textunderscore  + \textunderscore phoros\textunderscore )}
\end{itemize}
Apparelho, recentemente inventado pelo português Padre Himalaia, e que tem por fim determinar a natureza e origem do calor e da luz do Sol, estudar as propriedades da matéria, sob a influência de temperaturas muito elevadas e completar a escala das altas temperaturas.
\section{Pýrico}
\begin{itemize}
\item {Grp. gram.:adj.}
\end{itemize}
\begin{itemize}
\item {Proveniência:(De \textunderscore pyra\textunderscore )}
\end{itemize}
Relativo ao fogo.
\section{Pyridina}
\begin{itemize}
\item {Grp. gram.:f.}
\end{itemize}
Producto chímico, usado em inhalações, no tratamento da asma e da dyspneia cardíaca, e produzido pela destillação de ossos.
\section{Pyrilâmpico}
\begin{itemize}
\item {Grp. gram.:adj.}
\end{itemize}
Que tem luz com um pyrilampo; phosphorescente.
\section{Pyrilampo}
\begin{itemize}
\item {Grp. gram.:m.}
\end{itemize}
\begin{itemize}
\item {Proveniência:(Gr. \textunderscore purilampis\textunderscore )}
\end{itemize}
Gênero de insectos coleópteros pentâmeros, que emittem uma luz phosphorescente; vagalume.
\section{Pyríphora}
\begin{itemize}
\item {Grp. gram.:f.}
\end{itemize}
\begin{itemize}
\item {Utilização:Bras}
\end{itemize}
\begin{itemize}
\item {Proveniência:(Do gr. \textunderscore pur\textunderscore  + \textunderscore phoros\textunderscore )}
\end{itemize}
O mesmo que \textunderscore pyrilampo\textunderscore .
\section{Pyrite}
\begin{itemize}
\item {Grp. gram.:f.}
\end{itemize}
\begin{itemize}
\item {Proveniência:(Lat. \textunderscore pyrites\textunderscore )}
\end{itemize}
Sulfureto metállico, que tem a propriedade de se inflammar em dadas circumstâncias.
\textunderscore Pyrite cúprica\textunderscore , o mesmo que \textunderscore chalco-pyrite\textunderscore .
\section{Pyritífero}
\begin{itemize}
\item {Grp. gram.:adj.}
\end{itemize}
\begin{itemize}
\item {Proveniência:(Do lat. \textunderscore pyrites\textunderscore  + \textunderscore ferre\textunderscore )}
\end{itemize}
Que contém pyrite.
\section{Pyritiforme}
\begin{itemize}
\item {Grp. gram.:adj.}
\end{itemize}
\begin{itemize}
\item {Proveniência:(De \textunderscore pyrite\textunderscore  + \textunderscore fórma\textunderscore )}
\end{itemize}
Que tem a fórma de pyrite.
\section{Pyrólatra}
\begin{itemize}
\item {Grp. gram.:m.}
\end{itemize}
\begin{itemize}
\item {Proveniência:(Do gr. \textunderscore pur\textunderscore  + \textunderscore latreuein\textunderscore )}
\end{itemize}
Adorador do fogo.
\section{Pyrolatria}
\begin{itemize}
\item {Grp. gram.:f.}
\end{itemize}
Adoração do fogo.
(Cp. \textunderscore pyrólatra\textunderscore )
\section{Pyrolenhoso}
\begin{itemize}
\item {Grp. gram.:adj.}
\end{itemize}
\begin{itemize}
\item {Utilização:Chím.}
\end{itemize}
\begin{itemize}
\item {Proveniência:(Do gr. \textunderscore pur\textunderscore  + lat. \textunderscore lignum\textunderscore )}
\end{itemize}
Diz-se de um ácido, obtido pela destillação da madeira.
\section{Pyrologia}
\begin{itemize}
\item {Grp. gram.:f.}
\end{itemize}
\begin{itemize}
\item {Proveniência:(Do gr. \textunderscore pur\textunderscore  + \textunderscore logos\textunderscore )}
\end{itemize}
Tratado á cêrca do fogo.
\section{Pyrolusito}
\begin{itemize}
\item {Grp. gram.:m.}
\end{itemize}
\begin{itemize}
\item {Proveniência:(Do gr. \textunderscore pur\textunderscore  + \textunderscore lusis\textunderscore )}
\end{itemize}
Óxydo de manganés.
\section{Pyrómacho}
\begin{itemize}
\item {fónica:co}
\end{itemize}
\begin{itemize}
\item {Grp. gram.:adj.}
\end{itemize}
\begin{itemize}
\item {Utilização:Bot.}
\end{itemize}
\begin{itemize}
\item {Proveniência:(Do gr. \textunderscore pur\textunderscore  + \textunderscore makhe\textunderscore )}
\end{itemize}
Que produz centelhas, quando percutido com ferro.
\section{Pyromancia}
\begin{itemize}
\item {Grp. gram.:f.}
\end{itemize}
\begin{itemize}
\item {Proveniência:(Lat. \textunderscore pyromantia\textunderscore )}
\end{itemize}
Previsão do futuro, por meio do fogo.
\section{Pyromania}
\begin{itemize}
\item {Grp. gram.:f.}
\end{itemize}
\begin{itemize}
\item {Utilização:Med.}
\end{itemize}
\begin{itemize}
\item {Proveniência:(Do gr. \textunderscore pur\textunderscore  + \textunderscore mania\textunderscore )}
\end{itemize}
Monomania incendiária.
\section{Pyromântico}
\begin{itemize}
\item {Grp. gram.:adj.}
\end{itemize}
Relativo á piromancia.
\section{Pyrometria}
\begin{itemize}
\item {Grp. gram.:f.}
\end{itemize}
\begin{itemize}
\item {Proveniência:(De \textunderscore pirómetro\textunderscore )}
\end{itemize}
Arte de avaliar as altas temperaturas.
\section{Pyrométrico}
\begin{itemize}
\item {Grp. gram.:adj.}
\end{itemize}
Relativo á pyrometria.
\section{Pyrómetro}
\begin{itemize}
\item {Grp. gram.:m.}
\end{itemize}
\begin{itemize}
\item {Proveniência:(Do gr. \textunderscore pur\textunderscore  + \textunderscore metron\textunderscore )}
\end{itemize}
Instrumento, para avaliar aproximadamente as altas temperaturas.
\section{Pyromorphite}
\begin{itemize}
\item {Grp. gram.:m.}
\end{itemize}
\begin{itemize}
\item {Utilização:Miner.}
\end{itemize}
\begin{itemize}
\item {Proveniência:(Do gr. \textunderscore pur\textunderscore  + \textunderscore morphe\textunderscore )}
\end{itemize}
Um dos productos da alteração da galenite.
\section{Pyromotor}
\begin{itemize}
\item {Grp. gram.:m.}
\end{itemize}
\begin{itemize}
\item {Proveniência:(Do lat. \textunderscore pyra\textunderscore  + \textunderscore motor\textunderscore )}
\end{itemize}
Apparelho agrícola, para produzir fogo no campo e conjurar o frio que damnifica os sarmentos tenros.
\section{Pyronomia}
\begin{itemize}
\item {Grp. gram.:f.}
\end{itemize}
\begin{itemize}
\item {Proveniência:(Do gr. \textunderscore pur\textunderscore  + \textunderscore nomos\textunderscore )}
\end{itemize}
Arte de regular a temperatura nas operações chímicas.
\section{Pyronómico}
\begin{itemize}
\item {Grp. gram.:adj.}
\end{itemize}
Relativo á pyronomia.
\section{Pyrophobia}
\begin{itemize}
\item {Grp. gram.:f.}
\end{itemize}
\begin{itemize}
\item {Utilização:Med.}
\end{itemize}
\begin{itemize}
\item {Proveniência:(Do gr. \textunderscore pur\textunderscore  + \textunderscore phobein\textunderscore )}
\end{itemize}
Mêdo mórbido do fogo.
\section{Pyróphoro}
\begin{itemize}
\item {Grp. gram.:m.}
\end{itemize}
\begin{itemize}
\item {Grp. gram.:Pl.}
\end{itemize}
\begin{itemize}
\item {Proveniência:(Gr. \textunderscore purophoros\textunderscore )}
\end{itemize}
Composição chímica, que se inflamma ao contacto do ar.
Espécie de sacerdotes que, entre os antigos Gregos, iam á frente do exército, levando vasos cheios de fogo, que êlles atiravam contra o inimigo para dar signal de combate, quando as trombetas ainda não tinham esta applicação.
\section{Pyropina}
\begin{itemize}
\item {Grp. gram.:f.}
\end{itemize}
\begin{itemize}
\item {Proveniência:(De \textunderscore pyropo\textunderscore )}
\end{itemize}
Substancia albuminóide e vermelha, extrahida dos dentes do elephante.
\section{Pyropincel}
\begin{itemize}
\item {Grp. gram.:m.}
\end{itemize}
\begin{itemize}
\item {Proveniência:(De \textunderscore pyro...\textunderscore  + \textunderscore pincel\textunderscore )}
\end{itemize}
Instrumento, cuja ponta se torna candente á lâmpada, e cuja encandescência se mantém com uma corrente de ar, carregada de vapores de benzina.
\section{Pyropo}
\begin{itemize}
\item {Grp. gram.:m.}
\end{itemize}
\begin{itemize}
\item {Proveniência:(Lat. \textunderscore pyropus\textunderscore )}
\end{itemize}
Liga de quatro partes de cobre e uma de oiro, usada pelos antigos.
Variedade de pedra preciosa.
\section{Pyróscapho}
\begin{itemize}
\item {Grp. gram.:m.}
\end{itemize}
\begin{itemize}
\item {Proveniência:(Do gr. \textunderscore pur\textunderscore  + \textunderscore scophos\textunderscore )}
\end{itemize}
Nome, que se deu ao primeiro navio a vapor.
\section{Pyroscopía}
\begin{itemize}
\item {Grp. gram.:f.}
\end{itemize}
\begin{itemize}
\item {Proveniência:(Do gr. \textunderscore pur\textunderscore  + \textunderscore skopein\textunderscore )}
\end{itemize}
Supposta arte de adivinhar, por meio das chammas dos sacrifícios antigos.
\section{Pyroscópio}
\begin{itemize}
\item {Grp. gram.:m.}
\end{itemize}
\begin{itemize}
\item {Proveniência:(Do gr. \textunderscore pur\textunderscore  + \textunderscore skopein\textunderscore )}
\end{itemize}
Instrumento, para indicar que a temperatura attingiu determinado grau.
\section{Pyrose}
\begin{itemize}
\item {Grp. gram.:f.}
\end{itemize}
\begin{itemize}
\item {Proveniência:(Gr. \textunderscore purosis\textunderscore )}
\end{itemize}
Sensação de ardor ou calor, que vai do estômago até á garganta; azia.
\section{Pythagorismo}
\begin{itemize}
\item {Grp. gram.:m.}
\end{itemize}
Doutrina de Pythágoras.
\section{Pythagorista}
\begin{itemize}
\item {Grp. gram.:m.}
\end{itemize}
Sectário do pythagorismo.
\section{Pythão}
\begin{itemize}
\item {Grp. gram.:m.}
\end{itemize}
\begin{itemize}
\item {Utilização:T. eccles}
\end{itemize}
\begin{itemize}
\item {Proveniência:(Lat. \textunderscore Python\textunderscore , n. p.)}
\end{itemize}
Serpente mythológica, morta por Apollo.
Mago; nigromante.
\section{Pythaula}
\begin{itemize}
\item {Grp. gram.:m.}
\end{itemize}
\begin{itemize}
\item {Proveniência:(Lat. \textunderscore pythaula\textunderscore )}
\end{itemize}
Aquelle que, nos antigos theatros, acompanhava com o som da frauta o monólogo do actor.
\section{Pýthia}
\begin{itemize}
\item {Grp. gram.:f.}
\end{itemize}
\begin{itemize}
\item {Proveniência:(Lat. \textunderscore pythia\textunderscore )}
\end{itemize}
Sacerdotisa de Apollo, a qual pronunciava oráculos em Delphos.
\section{Pýthico}
\begin{itemize}
\item {Grp. gram.:adj.}
\end{itemize}
\begin{itemize}
\item {Proveniência:(Lat. \textunderscore pythicus\textunderscore )}
\end{itemize}
Relativo a Pýthia.
Relativo aos jogos, que em Delphos se celebravam em honra de Apollo, Latona e Diana.
\section{Pýthon}
\begin{itemize}
\item {Grp. gram.:m.}
\end{itemize}
(V.pythão)
\section{Pythónico}
\begin{itemize}
\item {Grp. gram.:adj.}
\end{itemize}
\begin{itemize}
\item {Proveniência:(Lat. \textunderscore pythonicus\textunderscore )}
\end{itemize}
Relativo ao python.
Nigromântico; mágico; diabólico. Cf. Arn. Gama, \textunderscore Motim\textunderscore , 226.
\section{Pythonisa}
\begin{itemize}
\item {Grp. gram.:f.}
\end{itemize}
\begin{itemize}
\item {Utilização:Ext.}
\end{itemize}
\begin{itemize}
\item {Proveniência:(Lat. \textunderscore pythonisa\textunderscore )}
\end{itemize}
\end{document}