
\begin{itemize}
\item {Proveniência: }
\end{itemize}\documentclass{article}
\usepackage[portuguese]{babel}
\title{E}
\begin{document}
Mergulhador.
Coleóptero amphíbio, espécie de carocha com asas, (\textunderscore dyticus marginalis\textunderscore ). Cf. P. Moraes, \textunderscore Zool. Elem.\textunderscore , 586.
Família de aves, que têm o hábito de mergulhar.
\section{Escaldo}
\begin{itemize}
\item {Grp. gram.:m.}
\end{itemize}
Cantor medieval; bardo. Cf. Garrett, \textunderscore Romanceiro\textunderscore , I, p. XVIII.
\section{Escapo}
\begin{itemize}
\item {Grp. gram.:m.}
\end{itemize}
\begin{itemize}
\item {Proveniência:(Lat. \textunderscore scapus\textunderscore )}
\end{itemize}
Aste: tronco. Cf. Castilho, \textunderscore Fastos\textunderscore , I, 312, 319 e 320.
\section{Eschizocéphalo}
\begin{itemize}
\item {fónica:qui}
\end{itemize}
\begin{itemize}
\item {Grp. gram.:adj.}
\end{itemize}
\begin{itemize}
\item {Proveniência:(Do gr. \textunderscore skhizein\textunderscore  + \textunderscore kephale\textunderscore )}
\end{itemize}
Que tem a cabeça dividida longitudinalmente, (falando-se de certos monstros).
\section{Eschizólitho}
\begin{itemize}
\item {fónica:qui}
\end{itemize}
\begin{itemize}
\item {Grp. gram.:m.}
\end{itemize}
\begin{itemize}
\item {Proveniência:(Do gr. \textunderscore skizein\textunderscore  + \textunderscore lithos\textunderscore )}
\end{itemize}
Gênero do mineraes, que comprehende a mica e outros.
\section{Eschizomycetes}
\begin{itemize}
\item {fónica:qui}
\end{itemize}
\begin{itemize}
\item {Grp. gram.:m. pl.}
\end{itemize}
\begin{itemize}
\item {Proveniência:(Do gr. \textunderscore skizein\textunderscore  + \textunderscore muke\textunderscore )}
\end{itemize}
Nome, dado ás bactérias pelos autores que as classificam como cogumelos.
\section{Eschizóphyto}
\begin{itemize}
\item {fónica:qui}
\end{itemize}
\begin{itemize}
\item {Grp. gram.:adj.}
\end{itemize}
\begin{itemize}
\item {Utilização:Bot.}
\end{itemize}
\begin{itemize}
\item {Proveniência:(Do gr. \textunderscore skizein\textunderscore  + \textunderscore phuton\textunderscore )}
\end{itemize}
Diz-se dos vegetaes, que se reproduzem por fissiparidade.
\section{Eschizópode}
\begin{itemize}
\item {fónica:qui}
\end{itemize}
\begin{itemize}
\item {Grp. gram.:adj.}
\end{itemize}
\begin{itemize}
\item {Grp. gram.:m. pl.}
\end{itemize}
\begin{itemize}
\item {Proveniência:(Gr. \textunderscore skizopous\textunderscore )}
\end{itemize}
Que tem os pés fendidos.
Gênero de crustáceos.
\section{Eschizóptero}
\begin{itemize}
\item {fónica:qui}
\end{itemize}
\begin{itemize}
\item {Grp. gram.:adj.}
\end{itemize}
\begin{itemize}
\item {Proveniência:(Gr. \textunderscore skhizopteros\textunderscore )}
\end{itemize}
Que tem asas fendidas.
\section{Eschizothórax}
\begin{itemize}
\item {fónica:qui}
\end{itemize}
\begin{itemize}
\item {Grp. gram.:m.}
\end{itemize}
\begin{itemize}
\item {Proveniência:(Do gr. \textunderscore skhizein\textunderscore  + \textunderscore thorax\textunderscore )}
\end{itemize}
Monstruosidade, caracterizada pela divisão do esterno ou das paredes thorácicas.
\section{Eschizotrichia}
\begin{itemize}
\item {fónica:qui}
\end{itemize}
\begin{itemize}
\item {Grp. gram.:f.}
\end{itemize}
\begin{itemize}
\item {Proveniência:(Do gr. \textunderscore skhizein\textunderscore  + \textunderscore strix\textunderscore , \textunderscore strikos\textunderscore )}
\end{itemize}
Qualidade de certos cabellos, que são fendidos na extremidade.
\section{Esquizocéfalo}
\begin{itemize}
\item {Grp. gram.:adj.}
\end{itemize}
\begin{itemize}
\item {Proveniência:(Do gr. \textunderscore skhizein\textunderscore  + \textunderscore kephale\textunderscore )}
\end{itemize}
Que tem a cabeça dividida longitudinalmente, (falando-se de certos monstros).
\section{Esquizófito}
\begin{itemize}
\item {Grp. gram.:adj.}
\end{itemize}
\begin{itemize}
\item {Utilização:Bot.}
\end{itemize}
\begin{itemize}
\item {Grp. gram.:M.}
\end{itemize}
\begin{itemize}
\item {Utilização:Med.}
\end{itemize}
\begin{itemize}
\item {Proveniência:(Do gr. \textunderscore skizein\textunderscore  + \textunderscore phuton\textunderscore )}
\end{itemize}
Diz-se dos vegetaes, que se reproduzem por fissiparidade.
O mesmo que \textunderscore bactéria\textunderscore .
\section{Esquizólito}
\begin{itemize}
\item {Grp. gram.:m.}
\end{itemize}
\begin{itemize}
\item {Proveniência:(Do gr. \textunderscore skizein\textunderscore  + \textunderscore lithos\textunderscore )}
\end{itemize}
Gênero do mineraes, que compreende a mica e outros.
\section{Esquizomicetes}
\begin{itemize}
\item {Grp. gram.:m. pl.}
\end{itemize}
\begin{itemize}
\item {Proveniência:(Do gr. \textunderscore skizein\textunderscore  + \textunderscore muke\textunderscore )}
\end{itemize}
Nome, dado ás bactérias pelos autores que as classificam como cogumelos.
\section{Esquizópode}
\begin{itemize}
\item {Grp. gram.:adj.}
\end{itemize}
\begin{itemize}
\item {Grp. gram.:m. pl.}
\end{itemize}
\begin{itemize}
\item {Proveniência:(Gr. \textunderscore skizopous\textunderscore )}
\end{itemize}
Que tem os pés fendidos.
Gênero de crustáceos.
\section{Esquizóptero}
\begin{itemize}
\item {Grp. gram.:adj.}
\end{itemize}
\begin{itemize}
\item {Proveniência:(Gr. \textunderscore skhizopteros\textunderscore )}
\end{itemize}
Que tem asas fendidas.
\section{Esquizotórax}
\begin{itemize}
\item {Grp. gram.:m.}
\end{itemize}
\begin{itemize}
\item {Proveniência:(Do gr. \textunderscore skhizein\textunderscore  + \textunderscore thorax\textunderscore )}
\end{itemize}
Monstruosidade, caracterizada pela divisão do esterno ou das paredes horácicas.
\section{Esquizotriquia}
\begin{itemize}
\item {Grp. gram.:f.}
\end{itemize}
\begin{itemize}
\item {Proveniência:(Do gr. \textunderscore skhizein\textunderscore  + \textunderscore strix\textunderscore , \textunderscore strikos\textunderscore )}
\end{itemize}
Qualidade de certos cabellos, que são fendidos na extremidade.
\section{Escíbalo}
\begin{itemize}
\item {Grp. gram.:m.}
\end{itemize}
\begin{itemize}
\item {Proveniência:(Do gr. \textunderscore skubalos\textunderscore )}
\end{itemize}
Excremento duro e arredondado.
\section{Escintura}
\begin{itemize}
\item {Grp. gram.:f.}
\end{itemize}
\begin{itemize}
\item {Utilização:Ant.}
\end{itemize}
Acto ou effeito de scindir:«\textunderscore com fome seram consumidos, e será por escintura desfallecimento dos membros.\textunderscore »Usque, 19 v.^o.
(Por \textunderscore scindura\textunderscore , de \textunderscore scindir\textunderscore )
\section{Espadicifloro}
\begin{itemize}
\item {Grp. gram.:adj.}
\end{itemize}
\begin{itemize}
\item {Utilização:Bot.}
\end{itemize}
\begin{itemize}
\item {Proveniência:(Do lat. \textunderscore spatha\textunderscore  + \textunderscore flos\textunderscore , \textunderscore floris\textunderscore )}
\end{itemize}
Que tem as flôres contidas em uma espata.
\section{Estalo}
\begin{itemize}
\item {Grp. gram.:m.}
\end{itemize}
\begin{itemize}
\item {Utilização:Ant.}
\end{itemize}
O mesmo que \textunderscore estau\textunderscore , Cf. Herculano, \textunderscore Bobo\textunderscore , 294, 295 e 301.
\section{Estrabónia}
\begin{itemize}
\item {Grp. gram.:f.}
\end{itemize}
\begin{itemize}
\item {Proveniência:(De \textunderscore Estrabão\textunderscore , n. p.)}
\end{itemize}
Gênero de plantas, da fam. das compostas.
\section{Ectocardia}
\begin{itemize}
\item {Grp. gram.:f.}
\end{itemize}
\begin{itemize}
\item {Utilização:Med.}
\end{itemize}
\begin{itemize}
\item {Proveniência:(Do gr. \textunderscore ektos\textunderscore  + \textunderscore kardia\textunderscore )}
\end{itemize}
Deslocamento do coração.
Situação anómala do coração.
\section{Ectocefalia}
\begin{itemize}
\item {Grp. gram.:f.}
\end{itemize}
\begin{itemize}
\item {Utilização:Med.}
\end{itemize}
\begin{itemize}
\item {Proveniência:(Do gr. \textunderscore ektos\textunderscore  + \textunderscore kephale\textunderscore )}
\end{itemize}
Monstruosidade, caracterizada pela situação anómala do cérebro.
\section{Ectocéfalo}
\begin{itemize}
\item {Grp. gram.:m.}
\end{itemize}
\begin{itemize}
\item {Utilização:Med.}
\end{itemize}
Monstro, em que há ectocefalia.
\section{Ectocephalia}
\begin{itemize}
\item {Grp. gram.:f.}
\end{itemize}
\begin{itemize}
\item {Utilização:Med.}
\end{itemize}
\begin{itemize}
\item {Proveniência:(Do gr. \textunderscore ektos\textunderscore  + \textunderscore kephale\textunderscore )}
\end{itemize}
Monstruosidade, caracterizada pela situação anómala do cérebro.
\section{Ectocéphalo}
\begin{itemize}
\item {Grp. gram.:m.}
\end{itemize}
\begin{itemize}
\item {Utilização:Med.}
\end{itemize}
Monstro, em que há ectocephalia.
\section{Ectoplacenta}
\begin{itemize}
\item {Grp. gram.:f.}
\end{itemize}
\begin{itemize}
\item {Utilização:Med.}
\end{itemize}
\begin{itemize}
\item {Proveniência:(De \textunderscore ektos\textunderscore  gr. + \textunderscore placenta\textunderscore )}
\end{itemize}
Revestimento endothelial incompleto das lacunas placentárias.
\section{Electrotropismo}
\begin{itemize}
\item {Grp. gram.:m.}
\end{itemize}
\begin{itemize}
\item {Utilização:Med.}
\end{itemize}
\begin{itemize}
\item {Proveniência:(Do gr. \textunderscore elektron\textunderscore  + \textunderscore trepein\textunderscore )}
\end{itemize}
Propriedade, que tem o protoplasma, de sêr attrahido ou repellido pela electricidade.
\section{Elitrotomia}
\begin{itemize}
\item {Grp. gram.:f.}
\end{itemize}
\begin{itemize}
\item {Utilização:Med.}
\end{itemize}
Incisão na vagina.
(Cp. \textunderscore elitrótomo\textunderscore )
\section{Elitrótomo}
\begin{itemize}
\item {Grp. gram.:m.}
\end{itemize}
\begin{itemize}
\item {Proveniência:(Do gr. \textunderscore elutros\textunderscore  + \textunderscore tome\textunderscore )}
\end{itemize}
Instrumento, com que se faz a elitrotomia.
\section{Elytrotomia}
\begin{itemize}
\item {Grp. gram.:f.}
\end{itemize}
\begin{itemize}
\item {Utilização:Med.}
\end{itemize}
Incisão na vagina.
(Cp. \textunderscore elytrótomo\textunderscore )
\section{Elytrótomo}
\begin{itemize}
\item {Grp. gram.:m.}
\end{itemize}
\begin{itemize}
\item {Proveniência:(Do gr. \textunderscore elutros\textunderscore  + \textunderscore tome\textunderscore )}
\end{itemize}
Instrumento, com que se faz a elytrotomia.
\section{Embelenar}
\begin{itemize}
\item {Grp. gram.:v. i.}
\end{itemize}
\begin{itemize}
\item {Utilização:Prov.}
\end{itemize}
\begin{itemize}
\item {Utilização:minh.}
\end{itemize}
\begin{itemize}
\item {Proveniência:(De \textunderscore beleno\textunderscore )}
\end{itemize}
Mexericar; intrigar. (Colhido em Viana)
\section{Emprestadio}
\begin{itemize}
\item {Grp. gram.:adj.}
\end{itemize}
Que se póde emprestar. Cf. Camillo, \textunderscore Bibliographia\textunderscore , II, 75.
\section{Encapelado}
\begin{itemize}
\item {Grp. gram.:adj.}
\end{itemize}
\begin{itemize}
\item {Proveniência:(De \textunderscore encapelar\textunderscore )}
\end{itemize}
Encrespado ou agitado, (falando-se do mar ou das ondas).
Que tem o encargo de capela.
\section{Encapelado}
\begin{itemize}
\item {Grp. gram.:adj.}
\end{itemize}
\begin{itemize}
\item {Utilização:Ant.}
\end{itemize}
Que traz capelo. Cf. \textunderscore Rev. Lus.\textunderscore , XVI, 6.
\section{Encapellado}
\begin{itemize}
\item {Grp. gram.:adj.}
\end{itemize}
\begin{itemize}
\item {Proveniência:(De \textunderscore encapellar\textunderscore )}
\end{itemize}
Encrespado ou agitado, (falando-se do mar ou das ondas).
Que tem o encargo de capella.
\section{Encapellado}
\begin{itemize}
\item {Grp. gram.:adj.}
\end{itemize}
\begin{itemize}
\item {Utilização:Ant.}
\end{itemize}
Que traz capello. Cf. \textunderscore Rev. Lus.\textunderscore , XVI, 6.
\section{Enceroilar}
\begin{itemize}
\item {Grp. gram.:v. t.}
\end{itemize}
Dar fórma ou applicação de ceroilas a:«\textunderscore enceroilou os seus calções...\textunderscore »Camillo, \textunderscore Noites de Insómn.\textunderscore , IV, 95.
\section{Enceroular}
\begin{itemize}
\item {Grp. gram.:v. t.}
\end{itemize}
Dar fórma ou applicação de ceroilas a:«\textunderscore enceroulou os seus calções...\textunderscore »Camillo, \textunderscore Noites de Insómn.\textunderscore , IV, 95.
\section{Encrudescer}
\begin{itemize}
\item {Grp. gram.:v. t.}
\end{itemize}
\begin{itemize}
\item {Proveniência:(Do lat. \textunderscore crudescere\textunderscore )}
\end{itemize}
Aggravar:«\textunderscore ...não mais encrudescer as tristezas de quem parecia mais feliz.\textunderscore »Camillo, \textunderscore No Bom-Jesus\textunderscore , 65.
\section{Endoparasito}
\begin{itemize}
\item {Grp. gram.:m.}
\end{itemize}
\begin{itemize}
\item {Utilização:Med.}
\end{itemize}
\begin{itemize}
\item {Proveniência:(Do gr. \textunderscore endon\textunderscore  + \textunderscore parasitos\textunderscore )}
\end{itemize}
Parasito, que vive no interior do organismo.
\section{Enfarnar}
\begin{itemize}
\item {Grp. gram.:v. t.}
\end{itemize}
\begin{itemize}
\item {Utilização:Prov.}
\end{itemize}
\begin{itemize}
\item {Utilização:beir.}
\end{itemize}
Pôr a trabalhar (o lagar de azeite), no primeiro dia da época do fabríco, por conta do dono, para que os fregueses, nos dias seguintes, não sejam lesados com a natural absorpção do óleo, feita pela vasa e outras peças, quando não servem há muito. (Colhido em Foscôa)
\section{Enfestado}
\begin{itemize}
\item {Grp. gram.:adj.}
\end{itemize}
\begin{itemize}
\item {Utilização:Ant.}
\end{itemize}
\begin{itemize}
\item {Proveniência:(De \textunderscore enfesta\textunderscore )}
\end{itemize}
Voltado para cima; empinado.
\section{Engenheiral}
\begin{itemize}
\item {Grp. gram.:adj.}
\end{itemize}
\begin{itemize}
\item {Utilização:Deprec.}
\end{itemize}
Próprio de engenheiro. Cf. R. Jorge, \textunderscore El Greco\textunderscore , 25.
\section{Engrifar}
\begin{itemize}
\item {Grp. gram.:v. t.}
\end{itemize}
Dispor como griphos:«\textunderscore engrifando os dedos para a luta.\textunderscore »Camillo, \textunderscore Cavar em Ruínas\textunderscore , 223.
\section{Enteiro}
\begin{itemize}
\item {Grp. gram.:adj.}
\end{itemize}
\begin{itemize}
\item {Utilização:Ant.}
\end{itemize}
O mesmo ou melhór que \textunderscore inteiro\textunderscore . Cf. \textunderscore Rev. Lus.\textunderscore , XVI, 6.
\section{Enteróclise}
\begin{itemize}
\item {Grp. gram.:f.}
\end{itemize}
\begin{itemize}
\item {Utilização:Med.}
\end{itemize}
\begin{itemize}
\item {Proveniência:(Do gr. \textunderscore enteron\textunderscore  + \textunderscore klusis\textunderscore )}
\end{itemize}
Lavagem do intestino pelo recto.
\section{Enteróclyse}
\begin{itemize}
\item {Grp. gram.:f.}
\end{itemize}
\begin{itemize}
\item {Utilização:Med.}
\end{itemize}
\begin{itemize}
\item {Proveniência:(Do gr. \textunderscore enteron\textunderscore  + \textunderscore klusis\textunderscore )}
\end{itemize}
Lavagem do intestino pelo recto.
\section{Entogar-se}
\begin{itemize}
\item {Grp. gram.:v. p.}
\end{itemize}
Vestir toga:«\textunderscore nem o tribunal do mundo se entoga com gravidade de juiz, por coisa tão fútil.\textunderscore »Camillo, \textunderscore Agulha\textunderscore , 189.
\section{Epícomo}
\begin{itemize}
\item {Grp. gram.:m.  e  adj.}
\end{itemize}
\begin{itemize}
\item {Utilização:Terat.}
\end{itemize}
\begin{itemize}
\item {Proveniência:(Do gr. \textunderscore epi\textunderscore  + \textunderscore kome\textunderscore )}
\end{itemize}
Monstro, que tem uma cabeça accessória, embora mal conformada.
\section{Epidemiar}
\begin{itemize}
\item {Grp. gram.:v. t.}
\end{itemize}
\begin{itemize}
\item {Utilização:Neol.}
\end{itemize}
Communicar epidemia a; contagiar:«\textunderscore epidemiava os médicos.\textunderscore »R. Jorge, \textunderscore Epid. de Lisb.\textunderscore 
\section{Epignathia}
\begin{itemize}
\item {Grp. gram.:f.}
\end{itemize}
\begin{itemize}
\item {Utilização:Terat.}
\end{itemize}
Estado de epígnatho.
\section{Epígnatho}
\begin{itemize}
\item {Grp. gram.:m.}
\end{itemize}
\begin{itemize}
\item {Utilização:Terat.}
\end{itemize}
\begin{itemize}
\item {Proveniência:(Do gr. \textunderscore epi\textunderscore  + \textunderscore gnathos\textunderscore )}
\end{itemize}
Monstro, que tem uma cabeça accessória, incompleta, presa á abóbada palatina. Cf. R. Galvão, \textunderscore Vocab.\textunderscore 
\section{Epignatia}
\begin{itemize}
\item {Grp. gram.:f.}
\end{itemize}
\begin{itemize}
\item {Utilização:Terat.}
\end{itemize}
Estado de epígnato.
\section{Epígnato}
\begin{itemize}
\item {Grp. gram.:m.}
\end{itemize}
\begin{itemize}
\item {Utilização:Terat.}
\end{itemize}
\begin{itemize}
\item {Proveniência:(Do gr. \textunderscore epi\textunderscore  + \textunderscore gnathos\textunderscore )}
\end{itemize}
Monstro, que tem uma cabeça accessória, incompleta, presa á abóbada palatina. Cf. R. Galvão, \textunderscore Vocab.\textunderscore 
\section{Eremiticamente}
\begin{itemize}
\item {Grp. gram.:adv.}
\end{itemize}
\begin{itemize}
\item {Proveniência:(De \textunderscore eremítico\textunderscore )}
\end{itemize}
Á maneira dos eremitas. Cf. Camillo, \textunderscore Bruxa\textunderscore , 208 e 248.
\section{Ergómetro}
\begin{itemize}
\item {Grp. gram.:m.}
\end{itemize}
\begin{itemize}
\item {Utilização:Physiol.}
\end{itemize}
\begin{itemize}
\item {Proveniência:(Do gr. \textunderscore ergon\textunderscore  + \textunderscore metron\textunderscore )}
\end{itemize}
Instrumento, com que se mede o trabalho executado por um músculo ou por um grupo muscular.
\section{Errada}
\begin{itemize}
\item {Grp. gram.:adj. f.}
\end{itemize}
\begin{itemize}
\item {Utilização:Pop.}
\end{itemize}
\begin{itemize}
\item {Proveniência:(De \textunderscore êrro\textunderscore )}
\end{itemize}
Diz-se da mulhér, que prevaricou contra a castidade, especialmente da que teve filhos fóra do matrimónio.
\section{Erva-belida}
\begin{itemize}
\item {Grp. gram.:f.}
\end{itemize}
Planta ranunculácea, (\textunderscore ranunculus repens\textunderscore , Lin.).
\section{Erva-besteira}
\begin{itemize}
\item {Grp. gram.:f.}
\end{itemize}
Planta ranunculácea, (\textunderscore helleborus foetidus\textunderscore , Lin.).
\section{Erva-branca}
\begin{itemize}
\item {Grp. gram.:f.}
\end{itemize}
\begin{itemize}
\item {Utilização:Bot.}
\end{itemize}
O mesmo que \textunderscore selvageira\textunderscore .
\section{Erva-canária}
\begin{itemize}
\item {Grp. gram.:f.}
\end{itemize}
\begin{itemize}
\item {Utilização:Bot.}
\end{itemize}
O mesmo que \textunderscore erva-pata\textunderscore .
\section{Erva-cavallinha}
\begin{itemize}
\item {Grp. gram.:f.}
\end{itemize}
Planta aristolochiácea, (\textunderscore aristolochia baetica\textunderscore , Lin.).
\section{Erva-coalheira}
\begin{itemize}
\item {Grp. gram.:f.}
\end{itemize}
Planta rubiácea, (\textunderscore galium verum\textunderscore , Lin.).
\section{Erva-confeiteira}
\begin{itemize}
\item {Grp. gram.:f.}
\end{itemize}
Planta rubiácea, (\textunderscore galium valantia\textunderscore , Lin.).
\section{Erva-contraveneno}
\begin{itemize}
\item {Grp. gram.:f.}
\end{itemize}
Planta asclepiadácea, (\textunderscore cynanchum vincectoxicum\textunderscore , Lin.).
\section{Erva-das-azeitonas}
\begin{itemize}
\item {Grp. gram.:f.}
\end{itemize}
\begin{itemize}
\item {Utilização:Bot.}
\end{itemize}
O mesmo que \textunderscore nêveda\textunderscore .
\section{Erva-das-dysenterias}
\begin{itemize}
\item {Grp. gram.:f.}
\end{itemize}
Planta, da fam. das compostas, (\textunderscore pulicaria dysenterica\textunderscore , Lin.).
\section{Erva-das-feridas}
\begin{itemize}
\item {Grp. gram.:f.}
\end{itemize}
\begin{itemize}
\item {Utilização:Bot.}
\end{itemize}
O mesmo que \textunderscore dentilária\textunderscore .
\section{Erva-das-verrugas}
\begin{itemize}
\item {Grp. gram.:f.}
\end{itemize}
\begin{itemize}
\item {Utilização:Bot.}
\end{itemize}
O mesmo que \textunderscore celidónia\textunderscore .
\section{Erva-de-besteiros}
\begin{itemize}
\item {Grp. gram.:f.}
\end{itemize}
\begin{itemize}
\item {Utilização:Bot.}
\end{itemize}
O mesmo que \textunderscore erva-besteira\textunderscore .
\section{Erva-de-san-roberto}
\begin{itemize}
\item {Grp. gram.:m.}
\end{itemize}
\begin{itemize}
\item {Utilização:Bot.}
\end{itemize}
O mesmo que \textunderscore erva-roberta\textunderscore .
\section{Erva-de-santa-bárbara}
\begin{itemize}
\item {Grp. gram.:f.}
\end{itemize}
Planta crucífera, (\textunderscore barbarea vulgaris\textunderscore , R. Br.).
\section{Erva-do-chá}
\begin{itemize}
\item {Grp. gram.:f.}
\end{itemize}
Planta malvácea, (\textunderscore sida rhombifolia\textunderscore , Lin.).
\section{Erva-do-homem-enforcado}
\begin{itemize}
\item {Grp. gram.:f.}
\end{itemize}
\begin{itemize}
\item {Utilização:Bot.}
\end{itemize}
O mesmo que \textunderscore rapazinhos\textunderscore .
\section{Erva-do-salepo}
\begin{itemize}
\item {Grp. gram.:f.}
\end{itemize}
\begin{itemize}
\item {Utilização:Bot.}
\end{itemize}
O mesmo que \textunderscore erva-perceveja\textunderscore .
\section{Erva-dos-cachos-da-índia}
\begin{itemize}
\item {Grp. gram.:f.}
\end{itemize}
\begin{itemize}
\item {Utilização:Bot.}
\end{itemize}
O mesmo que \textunderscore tintureira\textunderscore .
\section{Erva-dos-unheiros}
\begin{itemize}
\item {Grp. gram.:f.}
\end{itemize}
\begin{itemize}
\item {Utilização:Bot.}
\end{itemize}
O mesmo que \textunderscore erva-prata\textunderscore .
\section{Erva-dos-vasculhos}
\begin{itemize}
\item {Grp. gram.:f.}
\end{itemize}
\begin{itemize}
\item {Utilização:Bot.}
\end{itemize}
O mesmo que \textunderscore gilbardeira\textunderscore .
\section{Erva-ferradura}
\begin{itemize}
\item {Grp. gram.:f.}
\end{itemize}
Planta leguminosa, (\textunderscore hippocrepis\textunderscore , Lin.).
\section{Erva-fome}
\begin{itemize}
\item {Grp. gram.:f.}
\end{itemize}
Planta crucífera, (\textunderscore lepidium draba\textunderscore , Lin.).
\section{Erva-isqueira}
\begin{itemize}
\item {Grp. gram.:f.}
\end{itemize}
Planta onagrária, (\textunderscore cachrys laevigata\textunderscore , Lam.).
\section{Erva-lanar}
\begin{itemize}
\item {Grp. gram.:f.}
\end{itemize}
Planta gramínea, (\textunderscore holcus lanatus\textunderscore , Lin.).
\section{Erva-loira}
\begin{itemize}
\item {Grp. gram.:f.}
\end{itemize}
Planta, da fam. das compostas, (b. \textunderscore caespitosus\textunderscore , Brot.).
\section{Erva-lombrigueira}
\begin{itemize}
\item {Grp. gram.:f.}
\end{itemize}
\begin{itemize}
\item {Utilização:Bot.}
\end{itemize}
O mesmo que \textunderscore abrótono\textunderscore .
\section{Erva-mata-pulgas}
\begin{itemize}
\item {Grp. gram.:f.}
\end{itemize}
Planta leguminosa (\textunderscore dorycnium suffruticosum\textunderscore , Vill.).
\section{Erva-molarinha}
\begin{itemize}
\item {Grp. gram.:f.}
\end{itemize}
\begin{itemize}
\item {Utilização:Bot.}
\end{itemize}
O mesmo que \textunderscore erva-moleirinha\textunderscore . Cf. P. Coutinho, \textunderscore Flora\textunderscore , 245.
\section{Erva-mollar}
\begin{itemize}
\item {Grp. gram.:f.}
\end{itemize}
Planta gramínea, (\textunderscore holcus mollis\textunderscore , Lin.).
\section{Erva-prego}
\begin{itemize}
\item {Grp. gram.:f.}
\end{itemize}
Planta cariophyllácea, (\textunderscore paronychia echinata\textunderscore , Lam.).
\section{Erva-sofia}
\begin{itemize}
\item {Grp. gram.:f.}
\end{itemize}
Planta crucífera, (\textunderscore descurainia sophia\textunderscore , Lin.).
\section{Erva-ursa}
\begin{itemize}
\item {Grp. gram.:f.}
\end{itemize}
\begin{itemize}
\item {Utilização:Bot.}
\end{itemize}
O mesmo que \textunderscore erva-ussa\textunderscore ; serpão. Cf. P. Coutinho, \textunderscore Flora\textunderscore , 514.
\section{Escabrar}
\begin{itemize}
\item {Grp. gram.:v. i.}
\end{itemize}
O mesmo que \textunderscore escabrear\textunderscore . Cf. Camillo, \textunderscore Carrasco\textunderscore , 243.
\section{Escadós}
\begin{itemize}
\item {Grp. gram.:m.}
\end{itemize}
O mesmo que \textunderscore escadório\textunderscore :«\textunderscore sente-se num dos degraus do escadós principal\textunderscore ». Camillo, \textunderscore Mem. do Cárc.\textunderscore , c. LIV.
\section{Escantudo}
\begin{itemize}
\item {Grp. gram.:adj.}
\end{itemize}
Que tem grandes cantos:«\textunderscore testa escantuda\textunderscore ». Camillo, \textunderscore Bruxa\textunderscore , 11.
\section{Escandalizante}
\begin{itemize}
\item {Grp. gram.:adj.}
\end{itemize}
Que escandaliza.
\section{Escanzêlo}
\begin{itemize}
\item {Grp. gram.:m.}
\end{itemize}
Estado de escanzelado. Cf. Fialho, \textunderscore Gatos\textunderscore .
\section{Eschizomycete}
\begin{itemize}
\item {fónica:qui}
\end{itemize}
\begin{itemize}
\item {Grp. gram.:m.}
\end{itemize}
O mesmo que \textunderscore eschizomyceto\textunderscore .
\section{Eschizomyceto}
\begin{itemize}
\item {fónica:qui}
\end{itemize}
\begin{itemize}
\item {Grp. gram.:m.}
\end{itemize}
\begin{itemize}
\item {Utilização:Med.}
\end{itemize}
\begin{itemize}
\item {Proveniência:(Do gr. \textunderscore skizein\textunderscore  + \textunderscore mukes\textunderscore )}
\end{itemize}
O mesmo que \textunderscore bactéria\textunderscore .
\section{Eschyzóphito}
\begin{itemize}
\item {fónica:qui}
\end{itemize}
\begin{itemize}
\item {Grp. gram.:adj.}
\end{itemize}
\begin{itemize}
\item {Utilização:Bot.}
\end{itemize}
\begin{itemize}
\item {Grp. gram.:M.}
\end{itemize}
\begin{itemize}
\item {Utilização:Med.}
\end{itemize}
\begin{itemize}
\item {Proveniência:(Do gr. \textunderscore skizein\textunderscore  + \textunderscore phuton\textunderscore )}
\end{itemize}
Diz-se dos vegetaes, que se reproduzem por fissiparidade.
O mesmo que \textunderscore bactéria\textunderscore .
\section{Esclavos}
\begin{itemize}
\item {Grp. gram.:m. pl.}
\end{itemize}
\begin{itemize}
\item {Utilização:Ant.}
\end{itemize}
O mesmo que \textunderscore Esclavões\textunderscore .
(Cp. \textunderscore eslavo\textunderscore )
\section{Esclerectomia}
\begin{itemize}
\item {Grp. gram.:f.}
\end{itemize}
\begin{itemize}
\item {Utilização:Med.}
\end{itemize}
\begin{itemize}
\item {Proveniência:(Do gr. \textunderscore e\textunderscore  + \textunderscore skleros\textunderscore  + \textunderscore ektome\textunderscore )}
\end{itemize}
Secção da esclerótica.
\section{Escleremia}
\begin{itemize}
\item {Grp. gram.:f.}
\end{itemize}
O mesmo que \textunderscore escleroderma\textunderscore .
\section{Esclerite}
\begin{itemize}
\item {Grp. gram.:f.}
\end{itemize}
\begin{itemize}
\item {Utilização:Med.}
\end{itemize}
\begin{itemize}
\item {Proveniência:(Do gr. \textunderscore skleros\textunderscore )}
\end{itemize}
Inflammação da esclerótica.
\section{Esclerodactila}
\begin{itemize}
\item {Grp. gram.:f.}
\end{itemize}
\begin{itemize}
\item {Utilização:Med.}
\end{itemize}
\begin{itemize}
\item {Proveniência:(Do gr. \textunderscore skleros\textunderscore  + \textunderscore daktulos\textunderscore )}
\end{itemize}
Esclerodermia, limitada aos dedos.
\section{Esclerodactylia}
\begin{itemize}
\item {Grp. gram.:f.}
\end{itemize}
\begin{itemize}
\item {Utilização:Med.}
\end{itemize}
\begin{itemize}
\item {Proveniência:(Do gr. \textunderscore skleros\textunderscore  + \textunderscore daktulos\textunderscore )}
\end{itemize}
Esclerodermia, limitada aos dedos.
\section{Escleroso}
\begin{itemize}
\item {Grp. gram.:adj.}
\end{itemize}
\begin{itemize}
\item {Utilização:Med.}
\end{itemize}
\begin{itemize}
\item {Proveniência:(Do gr. \textunderscore skleros\textunderscore )}
\end{itemize}
O mesmo que \textunderscore fibroso\textunderscore .
\section{Escoliasta}
\begin{itemize}
\item {Grp. gram.:m.}
\end{itemize}
O mesmo ou melhór que \textunderscore escoliastes\textunderscore .
\section{Escrofúlida}
\begin{itemize}
\item {Grp. gram.:f.}
\end{itemize}
\begin{itemize}
\item {Utilização:Med.}
\end{itemize}
\begin{itemize}
\item {Proveniência:(De \textunderscore escrófula\textunderscore  + gr. \textunderscore eidos\textunderscore )}
\end{itemize}
Nome genérico, que se deu ás moléstias cutâneas, relacionadas etiologicamente com a escrófula.
\section{Esfenoidite}
\begin{itemize}
\item {Grp. gram.:f.}
\end{itemize}
\begin{itemize}
\item {Utilização:Med.}
\end{itemize}
\begin{itemize}
\item {Proveniência:(De \textunderscore esfenóide\textunderscore )}
\end{itemize}
Inflamação do saco esfenoidal.
\section{Esferra-cavallo}
\begin{itemize}
\item {Grp. gram.:f.}
\end{itemize}
\begin{itemize}
\item {Utilização:Bot.}
\end{itemize}
O mesmo que \textunderscore ferradurina\textunderscore .
\section{Esfincteralgia}
\begin{itemize}
\item {Grp. gram.:f.}
\end{itemize}
\begin{itemize}
\item {Utilização:Med.}
\end{itemize}
\begin{itemize}
\item {Proveniência:(Do gr. \textunderscore sphinkter\textunderscore  + \textunderscore algos\textunderscore )}
\end{itemize}
Dôr, junto do esfíncter anal.
\section{Esgaravatona}
\begin{itemize}
\item {Grp. gram.:f.}
\end{itemize}
Porta-voz?:«\textunderscore não ouvem senão por esgaravatonas.\textunderscore »M. Bernárdez, \textunderscore Excerptos\textunderscore , I, 267.
\section{Esperdiço}
\begin{itemize}
\item {Grp. gram.:m.}
\end{itemize}
\begin{itemize}
\item {Utilização:Des.}
\end{itemize}
O mesmo que \textunderscore desperdício\textunderscore . Cf. M. Bernárdez, \textunderscore Excerptos\textunderscore , II, 13.
\section{Esphenoidite}
\begin{itemize}
\item {Grp. gram.:f.}
\end{itemize}
\begin{itemize}
\item {Utilização:Med.}
\end{itemize}
\begin{itemize}
\item {Proveniência:(De \textunderscore esphenóide\textunderscore )}
\end{itemize}
Inflammação do saco esphenoidal.
\section{Esphincteralgia}
\begin{itemize}
\item {Grp. gram.:f.}
\end{itemize}
\begin{itemize}
\item {Utilização:Med.}
\end{itemize}
\begin{itemize}
\item {Proveniência:(Do gr. \textunderscore sphinkter\textunderscore  + \textunderscore algos\textunderscore )}
\end{itemize}
Dôr, junto do esphíncter anal.
\section{Espiclondrífico}
\begin{itemize}
\item {Grp. gram.:adj.}
\end{itemize}
\begin{itemize}
\item {Utilização:Burl.}
\end{itemize}
Muito esquisito, muito extravagante.
\section{Espinalgia}
\begin{itemize}
\item {Grp. gram.:f.}
\end{itemize}
\begin{itemize}
\item {Utilização:Med.}
\end{itemize}
\begin{itemize}
\item {Proveniência:(Do lat. \textunderscore spina\textunderscore  + \textunderscore algos\textunderscore )}
\end{itemize}
Sensibilidade dolorosa na pressão das apóphyses espinhaes.
\section{Espirocheta}
\begin{itemize}
\item {fónica:qué}
\end{itemize}
\begin{itemize}
\item {Grp. gram.:m.}
\end{itemize}
\begin{itemize}
\item {Utilização:Med.}
\end{itemize}
\begin{itemize}
\item {Proveniência:(Lat. scient. \textunderscore spirochoete\textunderscore )}
\end{itemize}
Designação genérica das bactérias em fórma de longos filamentos, com muitas voltas em espiral.--Há quem leia \textunderscore es-pi-ro-xé-ta\textunderscore . É êrro.
\section{Esquizomicete}
\begin{itemize}
\item {Grp. gram.:m.}
\end{itemize}
O mesmo que \textunderscore esquizomiceto\textunderscore .
\section{Esquizomiceto}
\begin{itemize}
\item {Grp. gram.:m.}
\end{itemize}
\begin{itemize}
\item {Utilização:Med.}
\end{itemize}
\begin{itemize}
\item {Proveniência:(Do gr. \textunderscore skizein\textunderscore  + \textunderscore mukes\textunderscore )}
\end{itemize}
O mesmo que \textunderscore bactéria\textunderscore .
\section{Espiroqueta}
\begin{itemize}
\item {Grp. gram.:m.}
\end{itemize}
\begin{itemize}
\item {Utilização:Med.}
\end{itemize}
\begin{itemize}
\item {Proveniência:(Lat. scient. \textunderscore spirochoete\textunderscore )}
\end{itemize}
Designação genérica das bactérias em fórma de longos filamentos, com muitas voltas em espiral.--Há quem leia \textunderscore es-pi-ro-xé-ta\textunderscore . É êrro.
\section{Espiroscópio}
\begin{itemize}
\item {Grp. gram.:m.}
\end{itemize}
\begin{itemize}
\item {Utilização:Med.}
\end{itemize}
\begin{itemize}
\item {Proveniência:(Do lat. \textunderscore spirare\textunderscore  + gr. \textunderscore skopein\textunderscore )}
\end{itemize}
Instrumento, para o estudo dos ruídos respiratórios.
\section{Esplenectomia}
\begin{itemize}
\item {Grp. gram.:f.}
\end{itemize}
\begin{itemize}
\item {Utilização:Med.}
\end{itemize}
\begin{itemize}
\item {Proveniência:(Do gr. \textunderscore splen\textunderscore  + \textunderscore ektome\textunderscore )}
\end{itemize}
O mesmo ou melhór que \textunderscore esplenotomia\textunderscore .
\section{Esplenização}
\begin{itemize}
\item {Grp. gram.:f.}
\end{itemize}
\begin{itemize}
\item {Utilização:Med.}
\end{itemize}
\begin{itemize}
\item {Proveniência:(De \textunderscore esplênico\textunderscore )}
\end{itemize}
Lesão pulmonar, caracterizada por um endurecimento do tecido, que dá o aspecto do tecido esplânico.
\section{Esplenopexia}
\begin{itemize}
\item {fónica:csi}
\end{itemize}
\begin{itemize}
\item {Grp. gram.:f.}
\end{itemize}
\begin{itemize}
\item {Utilização:Med.}
\end{itemize}
\begin{itemize}
\item {Proveniência:(Do gr. \textunderscore splen\textunderscore  + \textunderscore pexis\textunderscore )}
\end{itemize}
Operação de fixar o baço.
\section{Espois}
\begin{itemize}
\item {Grp. gram.:adv.}
\end{itemize}
\begin{itemize}
\item {Utilização:pop.}
\end{itemize}
\begin{itemize}
\item {Utilização:Ant.}
\end{itemize}
O mesmo que \textunderscore depois\textunderscore .
\section{Estadimétrico}
\begin{itemize}
\item {Grp. gram.:adj.}
\end{itemize}
\begin{itemize}
\item {Utilização:Phýs.}
\end{itemize}
\begin{itemize}
\item {Proveniência:(De \textunderscore estádia\textunderscore  + \textunderscore métrico\textunderscore )}
\end{itemize}
Diz-se coefficiente \textunderscore estadimétrico\textunderscore  de um óculo a relação entre a distância focal principal da objectiva e a extensão de uma divisão do micrómetro.
\section{Estafilococia}
\begin{itemize}
\item {Grp. gram.:f.}
\end{itemize}
\begin{itemize}
\item {Utilização:Med.}
\end{itemize}
Designação genérica dos estados mórbidos, resultantes da infecção de estaphyllococcos.
\section{Esternópago}
\begin{itemize}
\item {Grp. gram.:m.}
\end{itemize}
\begin{itemize}
\item {Utilização:Terat.}
\end{itemize}
\begin{itemize}
\item {Proveniência:(Do gr. \textunderscore sternon\textunderscore  + \textunderscore pageis\textunderscore )}
\end{itemize}
Monstro, composto de dois indivíduos, com umbigo commum, e unidos, face a face, em toda a extensão do thórax.
\section{Estaphyllococcia}
\begin{itemize}
\item {Grp. gram.:f.}
\end{itemize}
\begin{itemize}
\item {Utilização:Med.}
\end{itemize}
Designação genérica dos estados mórbidos, resultantes da infecção de estaphyllococcos.
\section{Estesiar}
\begin{itemize}
\item {Grp. gram.:v. t.}
\end{itemize}
\begin{itemize}
\item {Utilização:Neol.}
\end{itemize}
\begin{itemize}
\item {Proveniência:(De \textunderscore estesia\textunderscore )}
\end{itemize}
Produzir o sentimento do belo em.
\section{Estethógrapho}
\begin{itemize}
\item {Grp. gram.:m.}
\end{itemize}
\begin{itemize}
\item {Utilização:Med.}
\end{itemize}
\begin{itemize}
\item {Proveniência:(Do gr. \textunderscore stethos\textunderscore  + \textunderscore graphein\textunderscore )}
\end{itemize}
Apparelho, para registar os movimentos do thórax.
\section{Estetógrafo}
\begin{itemize}
\item {Grp. gram.:m.}
\end{itemize}
\begin{itemize}
\item {Utilização:Med.}
\end{itemize}
\begin{itemize}
\item {Proveniência:(Do gr. \textunderscore stethos\textunderscore  + \textunderscore graphein\textunderscore )}
\end{itemize}
Aparelho, para registar os movimentos do tórax.
\section{Esthesiar}
\begin{itemize}
\item {Grp. gram.:v. t.}
\end{itemize}
\begin{itemize}
\item {Utilização:Neol.}
\end{itemize}
\begin{itemize}
\item {Proveniência:(De \textunderscore esthesia\textunderscore )}
\end{itemize}
Produzir o sentimento do bello em.
\section{Estocar}
\begin{itemize}
\item {Grp. gram.:v. t.}
\end{itemize}
O mesmo que \textunderscore estoquear\textunderscore ; ferir ou matar com estoque:«\textunderscore se o não estocastes de lado a lado.\textunderscore »Camillo, \textunderscore Dem. do Oiro\textunderscore , II, 25.
\section{Estomatorragia}
\begin{itemize}
\item {Grp. gram.:f.}
\end{itemize}
\begin{itemize}
\item {Utilização:Med.}
\end{itemize}
\begin{itemize}
\item {Proveniência:(Do gr. \textunderscore stoma\textunderscore  + \textunderscore rhagein\textunderscore )}
\end{itemize}
Hemorragia bucal.
\section{Estomatorrhagia}
\begin{itemize}
\item {Grp. gram.:f.}
\end{itemize}
\begin{itemize}
\item {Utilização:Med.}
\end{itemize}
\begin{itemize}
\item {Proveniência:(Do gr. \textunderscore stoma\textunderscore  + \textunderscore rhagein\textunderscore )}
\end{itemize}
Hemorragia buccal.
\section{Estramelga}
\begin{itemize}
\item {Utilização:T. de Aveiro}
\end{itemize}
O mesmo que \textunderscore vassoira\textunderscore .
\section{Estrófulo}
\begin{itemize}
\item {Grp. gram.:m.}
\end{itemize}
\begin{itemize}
\item {Utilização:Med.}
\end{itemize}
\begin{itemize}
\item {Proveniência:(Lat. \textunderscore strophulus\textunderscore )}
\end{itemize}
Inflamação cutânea, especialmente nas crianças de peito.
\section{Estróphulo}
\begin{itemize}
\item {Grp. gram.:m.}
\end{itemize}
\begin{itemize}
\item {Utilização:Med.}
\end{itemize}
\begin{itemize}
\item {Proveniência:(Lat. \textunderscore strophulus\textunderscore )}
\end{itemize}
Inflammação cutânea, especialmente nas crianças de peito.
\section{Estudantesco}
\begin{itemize}
\item {fónica:tês}
\end{itemize}
\begin{itemize}
\item {Grp. gram.:adj.}
\end{itemize}
Próprio de estudante:«\textunderscore em calão estudantesco...\textunderscore »R. Jorge, na \textunderscore Luta\textunderscore , de 6-VI-913.
\section{Ethmoidite}
\begin{itemize}
\item {Grp. gram.:f.}
\end{itemize}
\begin{itemize}
\item {Utilização:Med.}
\end{itemize}
Inflammação do ethmóide.
\section{Etiologicamente}
\begin{itemize}
\item {Grp. gram.:adj.}
\end{itemize}
Relativamente a etiologia.
\section{Etmoidite}
\begin{itemize}
\item {Grp. gram.:f.}
\end{itemize}
\begin{itemize}
\item {Utilização:Med.}
\end{itemize}
Inflamação do etmóide.
\section{Exedra}
\begin{itemize}
\item {Grp. gram.:f.}
\end{itemize}
\begin{itemize}
\item {Utilização:Neol.}
\end{itemize}
\begin{itemize}
\item {Proveniência:(Lat. \textunderscore exedra\textunderscore )}
\end{itemize}
Casa de reunião; casa de recepção: \textunderscore vão construir uma exedra no Jardim Zoológico\textunderscore . (Dos jornaes de Fevereiro de 1913)
\section{Exencefalia}
\begin{itemize}
\item {Grp. gram.:f.}
\end{itemize}
Qualidade de exencéfalo.
\section{Exencéfalo}
\begin{itemize}
\item {Grp. gram.:m.}
\end{itemize}
\begin{itemize}
\item {Utilização:Terat.}
\end{itemize}
\begin{itemize}
\item {Proveniência:(Do gr. \textunderscore ex\textunderscore  + \textunderscore enkephalos\textunderscore )}
\end{itemize}
Gênero de monstros, que têm o encéfalo situado, em grande parte, fóra da caixa craniana.
\section{Exencephalia}
\begin{itemize}
\item {Grp. gram.:f.}
\end{itemize}
Qualidade de exencéphalo.
\section{Exencéphalo}
\begin{itemize}
\item {Grp. gram.:m.}
\end{itemize}
\begin{itemize}
\item {Utilização:Terat.}
\end{itemize}
\begin{itemize}
\item {Proveniência:(Do gr. \textunderscore ex\textunderscore  + \textunderscore enkephalos\textunderscore )}
\end{itemize}
Gênero de monstros, que têm o encéphalo situado, em grande parte, fóra da caixa craniana.
\section{Exometria}
\begin{itemize}
\item {Grp. gram.:f.}
\end{itemize}
O mesmo ou melhór que \textunderscore exómetra\textunderscore .
\section{Expressionista}
\begin{itemize}
\item {Grp. gram.:adj.}
\end{itemize}
Relativo a expressão:«\textunderscore energia expressionista\textunderscore ». R. Jorge, \textunderscore El Greco\textunderscore , 33.
\section{Endoutrinamento}
\begin{itemize}
\item {Grp. gram.:m.}
\end{itemize}
Acto de \textunderscore endoutrinar\textunderscore .
\section{Endoutrinar}
\begin{itemize}
\item {Grp. gram.:v. t.}
\end{itemize}
O mesmo que \textunderscore doutrinar\textunderscore . Cf. Car. Michaëlis, \textunderscore Infanta D. Maria\textunderscore , 30.
\section{Endovenoso}
\begin{itemize}
\item {Grp. gram.:adj.}
\end{itemize}
\begin{itemize}
\item {Utilização:Anat.}
\end{itemize}
Que está dentro das veias.
\section{Endrão}
\begin{itemize}
\item {Grp. gram.:m.}
\end{itemize}
Endro bravo.
\section{Endréssia}
\begin{itemize}
\item {Grp. gram.:f.}
\end{itemize}
Planta umbellífera dos Pyrenéus.
\section{Endriaco}
\begin{itemize}
\item {Grp. gram.:m.}
\end{itemize}
O mesmo que \textunderscore endriago\textunderscore .
\section{Endriago}
\begin{itemize}
\item {Grp. gram.:m.}
\end{itemize}
Monstro fabuloso, do qual se dizia que devorava as virgens:«\textunderscore D. Quixote procurava endriagos e gigantes\textunderscore ». Arn. Gama, \textunderscore Segr. do Ab.\textunderscore , 27 e 30.
\section{Endro}
\begin{itemize}
\item {Grp. gram.:m.}
\end{itemize}
Planta umbellífera, semelhante ao funcho.
(Provavelmente, de um dem. hypoth. \textunderscore anethulum\textunderscore , do lat. \textunderscore anethum\textunderscore )
\section{Endrómide}
\begin{itemize}
\item {Grp. gram.:f.}
\end{itemize}
\begin{itemize}
\item {Proveniência:(Do lat. \textunderscore endromis\textunderscore )}
\end{itemize}
Manto, com que, entre os Romanos, se cobria o corpo, depois dos exercícios phýsicos.
\section{Endrómina}
\begin{itemize}
\item {Grp. gram.:f.}
\end{itemize}
\begin{itemize}
\item {Utilização:Chul.}
\end{itemize}
Ardil.
Intrujice.
\section{Endua}
\begin{itemize}
\item {Grp. gram.:f.}
\end{itemize}
Ave africana.
\section{Enduape}
\begin{itemize}
\item {Grp. gram.:m.}
\end{itemize}
\begin{itemize}
\item {Utilização:Bras}
\end{itemize}
Fraldão de pennas, de que usavam os guerreiros índios. Cf. Gonç. Dias, \textunderscore Poesias\textunderscore , II, 17.
\section{Endumba}
\begin{itemize}
\item {Grp. gram.:f.}
\end{itemize}
Ave trepadora da África occidental.
\section{Endurar}
\textunderscore v. t.\textunderscore , \textunderscore i.\textunderscore  e \textunderscore p.\textunderscore  (e der.)
O mesmo que \textunderscore endurecer\textunderscore , etc.
\section{Endurecer}
\begin{itemize}
\item {Grp. gram.:v. t.}
\end{itemize}
\begin{itemize}
\item {Utilização:Fig.}
\end{itemize}
\begin{itemize}
\item {Grp. gram.:V. i.}
\end{itemize}
\begin{itemize}
\item {Utilização:Fig.}
\end{itemize}
\begin{itemize}
\item {Proveniência:(Do lat. \textunderscore indurescere\textunderscore )}
\end{itemize}
Tornar duro, rijo: \textunderscore a geada endureceu a terra\textunderscore .
Tornar insensível.
Tornar-se duro; enrijar: \textunderscore o pão endurece\textunderscore .
Tornar-se insensível, impassível: \textunderscore endureceu-lhe o coração\textunderscore .
\section{Endurecimento}
\begin{itemize}
\item {Grp. gram.:m.}
\end{itemize}
Acto ou effeito de endurecer.
\section{Endurentar}
\begin{itemize}
\item {Grp. gram.:v. t. ,  i.  e  p.}
\end{itemize}
O mesmo que \textunderscore endurecer\textunderscore .
\section{Enduxiquirape}
\begin{itemize}
\item {Grp. gram.:m.}
\end{itemize}
Pássaro dentirostro da África occidental.
\section{Enebriante}
\begin{itemize}
\item {Grp. gram.:adj.}
\end{itemize}
Que enebria.
\section{Enebriar}
\begin{itemize}
\item {Grp. gram.:v. t.}
\end{itemize}
\begin{itemize}
\item {Utilização:Fig.}
\end{itemize}
\begin{itemize}
\item {Proveniência:(Lat. \textunderscore inebriare\textunderscore )}
\end{itemize}
Embriagar.
Deliciar.
Enthusiasmar.
Extasiar: \textunderscore enebriam-no os encantos della\textunderscore .
\section{Eneide}
\begin{itemize}
\item {Grp. gram.:adj.}
\end{itemize}
Relativo a Eneias; descendente de Eneias:«\textunderscore mimosa Vênus, mãe da eneide Roma...\textunderscore »Lima Leitão, \textunderscore Natureza das Coisas\textunderscore , cit. por Castilho, \textunderscore Fastos\textunderscore , II, 386.
A fórma exacta seria \textunderscore enéada\textunderscore .
(Cp. lat. \textunderscore aeneadae\textunderscore )
\section{Enema}
\begin{itemize}
\item {Grp. gram.:m.}
\end{itemize}
Medicamento, com que os antigos tratavam feridas.
Clister. Cf. Júl. Ribeiro, \textunderscore Gram.\textunderscore , 156.
\section{Éneo}
\begin{itemize}
\item {Grp. gram.:adj.}
\end{itemize}
\begin{itemize}
\item {Proveniência:(Lat. \textunderscore aeneus\textunderscore )}
\end{itemize}
Relativo ao bronze.
Semelhante ao bronze na dureza.
Feito de bronze: \textunderscore cadeias êneas\textunderscore .
\section{Eneópteros}
\begin{itemize}
\item {Grp. gram.:m. pl.}
\end{itemize}
\begin{itemize}
\item {Proveniência:(Do gr. \textunderscore eneos\textunderscore  + \textunderscore pteron\textunderscore )}
\end{itemize}
Gênero de insectos orthópteros da América do Sul.
\section{Eneorema}
\begin{itemize}
\item {Grp. gram.:m.}
\end{itemize}
\begin{itemize}
\item {Proveniência:(Do gr. \textunderscore enaiorema\textunderscore )}
\end{itemize}
Substância esbranquiçada, que se observa suspensa na urina guardada por algum tempo.
\section{Enequim}
\begin{itemize}
\item {Grp. gram.:m.}
\end{itemize}
O mesmo que \textunderscore anequim\textunderscore .
\section{Enequim}
\begin{itemize}
\item {Grp. gram.:m.}
\end{itemize}
\begin{itemize}
\item {Utilização:Ant.}
\end{itemize}
Sinaes de puberdade. Cf. Camões, \textunderscore Filodemo\textunderscore , V, 3.
\section{Energética}
\begin{itemize}
\item {Grp. gram.:f.}
\end{itemize}
\begin{itemize}
\item {Utilização:Philos.}
\end{itemize}
Dynamismo puro; espiritualismo.
Sciência da energia.
\section{Energia}
\begin{itemize}
\item {Grp. gram.:f.}
\end{itemize}
\begin{itemize}
\item {Proveniência:(Lat. \textunderscore energia\textunderscore )}
\end{itemize}
Actividade.
Maneira, com que se exerce uma fôrça.
Fôrça moral.
Vigor.
Firmeza.
\section{Energicamente}
\begin{itemize}
\item {Grp. gram.:adv.}
\end{itemize}
De modo enérgico.
\section{Enérgico}
\begin{itemize}
\item {Grp. gram.:adj.}
\end{itemize}
Que tem energia: \textunderscore homem enérgico\textunderscore .
Em que há energia: \textunderscore palavras enérgicas\textunderscore .
\section{Energúmeno}
\begin{itemize}
\item {Grp. gram.:m.}
\end{itemize}
\begin{itemize}
\item {Utilização:Fig.}
\end{itemize}
\begin{itemize}
\item {Proveniência:(Gr. \textunderscore energoumenos\textunderscore )}
\end{itemize}
Indivíduo, que se suppõe possesso do demónio.
Pessôa apaixonada, desnorteada.
\section{Enervação}
\begin{itemize}
\item {Grp. gram.:f.}
\end{itemize}
\begin{itemize}
\item {Proveniência:(Lat. \textunderscore enervatio\textunderscore )}
\end{itemize}
Fraqueza; prostração de fôrças; extenuação.
Acto ou effeito de enervar.
\section{Enervador}
\begin{itemize}
\item {Grp. gram.:adj.}
\end{itemize}
Que enerva. Cf. Júl. Dinís, \textunderscore Morgadinha\textunderscore , 386.
\section{Enervamento}
\begin{itemize}
\item {Grp. gram.:m.}
\end{itemize}
O mesmo que \textunderscore enervação\textunderscore ^1.
\section{Enervante}
\begin{itemize}
\item {Grp. gram.:adj.}
\end{itemize}
\begin{itemize}
\item {Proveniência:(Lat. \textunderscore enervans\textunderscore )}
\end{itemize}
Que enerva.
\section{Enervar}
\begin{itemize}
\item {Grp. gram.:v. t.}
\end{itemize}
\begin{itemize}
\item {Grp. gram.:V. i.}
\end{itemize}
\begin{itemize}
\item {Proveniência:(Lat. \textunderscore enervare\textunderscore )}
\end{itemize}
Privar de fôrça; enfraquecer.
Effeminar.
Perder o vigor, a energia.
\section{Enerve}
\begin{itemize}
\item {Grp. gram.:adj.}
\end{itemize}
\begin{itemize}
\item {Proveniência:(De \textunderscore enervar\textunderscore )}
\end{itemize}
\begin{itemize}
\item {Proveniência:(Lat. \textunderscore innervis\textunderscore )}
\end{itemize}
Enfraquecido; debilitado.
Effeminado.
\section{Enesol}
\begin{itemize}
\item {Grp. gram.:m.}
\end{itemize}
Novo medicamento contra a sýphilis.
\section{Enfadadiço}
\begin{itemize}
\item {Grp. gram.:adj.}
\end{itemize}
Que se enfada com facilidade.
\section{Enfadamento}
\begin{itemize}
\item {Grp. gram.:m.}
\end{itemize}
O mesmo que \textunderscore enfado\textunderscore .
\section{Enfadar}
\begin{itemize}
\item {Grp. gram.:v. t.}
\end{itemize}
\begin{itemize}
\item {Proveniência:(Do lat. \textunderscore fatuus\textunderscore , seg. Körting)}
\end{itemize}
Causar cansaço a.
Produzir aborrecimento em; enfastiar.
Incommodar.
Tornar agastado.
\section{Enfado}
\begin{itemize}
\item {Grp. gram.:m.}
\end{itemize}
Acto ou effeito de enfadar.
\section{Enfadonho}
\begin{itemize}
\item {Grp. gram.:adj.}
\end{itemize}
Que enfada.
\section{Enfadosamente}
\begin{itemize}
\item {Grp. gram.:adv.}
\end{itemize}
\begin{itemize}
\item {Utilização:Des.}
\end{itemize}
De modo enfadoso.
\section{Enfadoso}
\begin{itemize}
\item {Grp. gram.:adj.}
\end{itemize}
\begin{itemize}
\item {Utilização:Des.}
\end{itemize}
O mesmo que \textunderscore enfadonho\textunderscore .
\section{Enfaixar}
\begin{itemize}
\item {Grp. gram.:v. t.}
\end{itemize}
Ligar, envolver, com faixas.
\section{Enfanicar}
\begin{itemize}
\item {Grp. gram.:v. t.}
\end{itemize}
\begin{itemize}
\item {Utilização:Prov.}
\end{itemize}
Enrolar a baraça no (pião).
(Cp. \textunderscore faniqueira\textunderscore )
\section{Enfanicar-se}
\begin{itemize}
\item {Grp. gram.:v. p.}
\end{itemize}
\begin{itemize}
\item {Utilização:Fam.}
\end{itemize}
Têr fanicos, desmaiar.
\section{Enfaramento}
\begin{itemize}
\item {Grp. gram.:m.}
\end{itemize}
Acto de enfarar.
\section{Enfarar}
\begin{itemize}
\item {Grp. gram.:v. t.  e  i.}
\end{itemize}
\begin{itemize}
\item {Proveniência:(De \textunderscore faro\textunderscore )}
\end{itemize}
Têr enjôo a. Tomar aborrecimento a (uma comida).
Sentir repugnância pelo cheiro ou sabor de.
\section{Enfardadeira}
\begin{itemize}
\item {Grp. gram.:f.}
\end{itemize}
\begin{itemize}
\item {Proveniência:(De \textunderscore enfardar\textunderscore )}
\end{itemize}
Máquina agrícola, para comprimir palha ou feno em pequenos feixes, facilitando o transporte e reduzindo o espaço, em que esses feixes se podem armazenar.
\section{Enfardadora}
\begin{itemize}
\item {Grp. gram.:f.}
\end{itemize}
O mesmo que \textunderscore enfardadeira\textunderscore .
\section{Enfardador}
\begin{itemize}
\item {Grp. gram.:adj.}
\end{itemize}
\begin{itemize}
\item {Grp. gram.:M.}
\end{itemize}
Que enfarda.
Aquelle que enfarda.
\section{Enfardamento}
\begin{itemize}
\item {Grp. gram.:m.}
\end{itemize}
Acto ou effeito de enfardar.
\section{Enfardar}
\begin{itemize}
\item {Grp. gram.:v. t.}
\end{itemize}
\begin{itemize}
\item {Proveniência:(De \textunderscore fardo\textunderscore )}
\end{itemize}
Juntar em fardo.
Entroixar; embrulhar.
\section{Enfardelar}
\begin{itemize}
\item {Grp. gram.:v. t.}
\end{itemize}
\begin{itemize}
\item {Proveniência:(De \textunderscore fardel\textunderscore )}
\end{itemize}
Meter em fardel.
O mesmo que \textunderscore enfardar\textunderscore .
\section{Enfarear}
\begin{itemize}
\item {Grp. gram.:v. t.}
\end{itemize}
\begin{itemize}
\item {Utilização:Prov.}
\end{itemize}
\begin{itemize}
\item {Utilização:trasm.}
\end{itemize}
\begin{itemize}
\item {Grp. gram.:V. i.}
\end{itemize}
Enfastiar-se pelo continuado uso de (certas comidas).
Enfastiar-se de qualquer iguaria.
(Cp. \textunderscore enfarar\textunderscore )
\section{Enfarelar}
\begin{itemize}
\item {Grp. gram.:v. t.}
\end{itemize}
Juntar farelos a.
Espalhar farelos sôbre.
\section{Enfarinhadamente}
\begin{itemize}
\item {Grp. gram.:adv.}
\end{itemize}
\begin{itemize}
\item {Utilização:Des.}
\end{itemize}
\begin{itemize}
\item {Proveniência:(De \textunderscore enfarinhar\textunderscore )}
\end{itemize}
Com dissimulação.
Sem clareza; ambiguamente.
\section{Enfarinhadela}
\begin{itemize}
\item {Grp. gram.:f.}
\end{itemize}
\begin{itemize}
\item {Utilização:Fam.}
\end{itemize}
Acto de enfarinhar.
\section{Enfarinhado}
\begin{itemize}
\item {Grp. gram.:adj.}
\end{itemize}
\begin{itemize}
\item {Grp. gram.:M.}
\end{itemize}
Polvilhado com farinha.
Empoado.
Casta de uva preta, de Azeitão.
\section{Enfarinhar}
\begin{itemize}
\item {Grp. gram.:v. t.}
\end{itemize}
\begin{itemize}
\item {Utilização:Fig.}
\end{itemize}
\begin{itemize}
\item {Proveniência:(De \textunderscore farinha\textunderscore )}
\end{itemize}
Polvilhar com farinha.
Sujar com farinha.
Empoar.
Dar noções geraes ou superficiaes a: \textunderscore enfarinhar o pequeno em Geographia\textunderscore .
\section{Enfaro}
\begin{itemize}
\item {Grp. gram.:m.}
\end{itemize}
\begin{itemize}
\item {Utilização:Des.}
\end{itemize}
Effeito de enfarar.
\section{Enfaroar}
\begin{itemize}
\item {Grp. gram.:v. t.}
\end{itemize}
\begin{itemize}
\item {Utilização:Prov.}
\end{itemize}
O mesmo que \textunderscore enfarar\textunderscore .
\section{Enfarpelar}
\begin{itemize}
\item {Grp. gram.:v. t.}
\end{itemize}
\begin{itemize}
\item {Utilização:Pop.}
\end{itemize}
Vestir; pôr farpela em.
\section{Enfarrapar}
\begin{itemize}
\item {Grp. gram.:v. t.}
\end{itemize}
Envolver em farrapos.
Vestir de farrapos.
\section{Enfarruscar}
\begin{itemize}
\item {Grp. gram.:v. t.}
\end{itemize}
Fazer farruscas em; encarvoar.
Mascarrar; sujar com carvão ou fuligem.
\section{Enfartação}
\begin{itemize}
\item {Grp. gram.:f.}
\end{itemize}
O mesmo que \textunderscore enfartamento\textunderscore .
\section{Enfartamento}
\begin{itemize}
\item {Grp. gram.:m.}
\end{itemize}
Acto ou effeito de \textunderscore enfartar\textunderscore .
\section{Enfartar}
\begin{itemize}
\item {Grp. gram.:v. t.}
\end{itemize}
\begin{itemize}
\item {Proveniência:(De \textunderscore fartar\textunderscore )}
\end{itemize}
Fartar, encher.
Encruar.
Entulhar; obstruir.
\section{Enfarte}
\begin{itemize}
\item {Grp. gram.:m.}
\end{itemize}
O mesmo que \textunderscore enfartamento\textunderscore .
\section{Ênfase}
\begin{itemize}
\item {Grp. gram.:f.}
\end{itemize}
\begin{itemize}
\item {Proveniência:(Gr. \textunderscore emphasis\textunderscore )}
\end{itemize}
Maneira empolada, exagerada ou afectada, de falar ou escrever.
Ostentação.
\section{Enfastiadamente}
\begin{itemize}
\item {Grp. gram.:adv.}
\end{itemize}
\begin{itemize}
\item {Proveniência:(De \textunderscore enfastiar\textunderscore )}
\end{itemize}
Com fastio ou tédio.
\section{Enfastiadiço}
\begin{itemize}
\item {Grp. gram.:adj.}
\end{itemize}
Que enfastia, que enfada.
Maçador.
\section{Enfastiamento}
\begin{itemize}
\item {Grp. gram.:m.}
\end{itemize}
Acto de \textunderscore enfastiar\textunderscore . Cf. Eça, \textunderscore P. Amaro\textunderscore , 250.
\section{Enfastiante}
\begin{itemize}
\item {Grp. gram.:adj.}
\end{itemize}
Que enfastia.
\section{Enfastiar}
\begin{itemize}
\item {Grp. gram.:v. t.}
\end{itemize}
Causar fastio ou aborrecimento a.
Enfadar; tornar-se aborrecido a.
\section{Enfastioso}
\begin{itemize}
\item {Grp. gram.:adj.}
\end{itemize}
Que enfastia.
\section{Enfaticamente}
\begin{itemize}
\item {Grp. gram.:adv.}
\end{itemize}
De modo enfático.
Com ênfase.
\section{Enfático}
\begin{itemize}
\item {Grp. gram.:adj.}
\end{itemize}
\begin{itemize}
\item {Proveniência:(Gr. \textunderscore emphatikos\textunderscore )}
\end{itemize}
Que tem ênfase; em que há ênfase: \textunderscore linguagem enfática\textunderscore .
\section{Enfatismo}
\begin{itemize}
\item {Grp. gram.:m.}
\end{itemize}
Qualidade daquilo ou daquele que é enfático.
Uso imoderado da ênfase.
(Cp. \textunderscore enfático\textunderscore )
\section{Enfatuamento}
\begin{itemize}
\item {Grp. gram.:m.}
\end{itemize}
Acto ou effeito de enfatuar.
\section{Enfatuar}
\begin{itemize}
\item {Grp. gram.:v. t.}
\end{itemize}
Tornar fátuo, vaidoso, suberbo.
\section{Enfear}
\begin{itemize}
\item {Grp. gram.:v. t.}
\end{itemize}
O mesmo que \textunderscore afear\textunderscore .
\section{Enfebrecer}
\begin{itemize}
\item {Grp. gram.:v. i.}
\end{itemize}
\begin{itemize}
\item {Utilização:P. us.}
\end{itemize}
\begin{itemize}
\item {Grp. gram.:V. t.}
\end{itemize}
Adquirir febre.
Causar febre a.
\section{Enfeirar}
\begin{itemize}
\item {Grp. gram.:v. t.}
\end{itemize}
\begin{itemize}
\item {Utilização:Prov.}
\end{itemize}
\begin{itemize}
\item {Utilização:minh.}
\end{itemize}
\begin{itemize}
\item {Grp. gram.:V. i.}
\end{itemize}
Comprar na feira.
Expor á venda na feira: \textunderscore enfeirar gado\textunderscore .
Fazer compras na feira.
\section{Enfeitador}
\begin{itemize}
\item {Grp. gram.:adj.}
\end{itemize}
\begin{itemize}
\item {Grp. gram.:M.}
\end{itemize}
Que enfeita.
Aquelle que enfeita.
\section{Enfeitar}
\begin{itemize}
\item {Grp. gram.:v. t.}
\end{itemize}
\begin{itemize}
\item {Utilização:Fig.}
\end{itemize}
Pôr enfeites em.
Ornar; encher de atavios.
Colorir ou disfarçar defeitos de.
Pôr farpas em (toiros).
(Cp. cast. \textunderscore afeitar\textunderscore )
\section{Enfeite}
\begin{itemize}
\item {Grp. gram.:m.}
\end{itemize}
\begin{itemize}
\item {Proveniência:(De \textunderscore enfeitar\textunderscore )}
\end{itemize}
Ornamento, adôrno, atavio.
\section{Enfeitiçar}
\begin{itemize}
\item {Grp. gram.:v. t.}
\end{itemize}
\begin{itemize}
\item {Utilização:Fig.}
\end{itemize}
Sujeitar á acção do feitiço.
Prejudicar por meio de suppostas artes diabólicas.
Atrahir, cativar, por meio de sortilégios ou bruxarias.
Atrahir irresistivelmente, encantar, seduzir.
\section{Enfeixar}
\begin{itemize}
\item {Grp. gram.:v. t.}
\end{itemize}
Juntar em feixes.
Reunir; entroixar.
\section{Enfelpado}
\begin{itemize}
\item {Grp. gram.:adj.}
\end{itemize}
\begin{itemize}
\item {Utilização:Prov.}
\end{itemize}
\begin{itemize}
\item {Utilização:alg.}
\end{itemize}
\begin{itemize}
\item {Proveniência:(De \textunderscore felpa\textunderscore ?)}
\end{itemize}
Envolvido em desordem.
\section{Enfelpar}
\begin{itemize}
\item {Grp. gram.:v. t.}
\end{itemize}
Revestir de felpa.
Tornar felpudo.
\section{Enfeltrar}
\begin{itemize}
\item {Grp. gram.:v. t.}
\end{itemize}
Converter em feltro.
Cobrir de feltro.
\section{Enfelujar}
\begin{itemize}
\item {Grp. gram.:v. t.}
\end{itemize}
Sujar com felugem; mascarrar.
\section{Enfenar}
\begin{itemize}
\item {Grp. gram.:v. i.}
\end{itemize}
\begin{itemize}
\item {Utilização:Prov.}
\end{itemize}
\begin{itemize}
\item {Utilização:alent.}
\end{itemize}
\begin{itemize}
\item {Proveniência:(De \textunderscore feno\textunderscore )}
\end{itemize}
Lançar raízes, fortificar-se, (diz-se do torrão, que communica as suas raízes á lama dos muros das salinas).
Juntar como o feno: \textunderscore enfenar os restolhos\textunderscore .
\section{Enfenecer}
\begin{itemize}
\item {Grp. gram.:v. i.}
\end{itemize}
O mesmo que \textunderscore fenecer\textunderscore . Cf. Filinto, XVIII, 150; XXI, 48 e 66.
\section{Enfermagem}
\begin{itemize}
\item {Grp. gram.:f.}
\end{itemize}
\begin{itemize}
\item {Utilização:fam.}
\end{itemize}
\begin{itemize}
\item {Utilização:Neol.}
\end{itemize}
Funcções de enfermeiro.
Tratamento de enfermos.
Os enfermeiros.
\section{Enfermar}
\begin{itemize}
\item {Grp. gram.:v. t.}
\end{itemize}
\begin{itemize}
\item {Grp. gram.:V. i.}
\end{itemize}
\begin{itemize}
\item {Utilização:Fig.}
\end{itemize}
\begin{itemize}
\item {Proveniência:(Lat. \textunderscore infirmare\textunderscore )}
\end{itemize}
Tornar enfermo.
Tornar-se enfermo; adoecer.
Têr macula ou defeito: \textunderscore a sua linguagem enferma de impropriedade\textunderscore .
\section{Enfermaria}
\begin{itemize}
\item {Grp. gram.:f.}
\end{itemize}
Lugar, onde se tratam enfermos.
Parte dos hospitaes ou de outros estabelecimentos, em que estão dispostas as camas para os enfermos.
\section{Enfermeira}
\begin{itemize}
\item {Grp. gram.:f.}
\end{itemize}
Mulher, encarregada de tratar de enfermos.
\section{Enfermeiro}
\begin{itemize}
\item {Grp. gram.:m.}
\end{itemize}
Aquelle que trata de enfermos ou de um enfermo.
\section{Enfernesar}
\begin{itemize}
\item {Grp. gram.:v. t.}
\end{itemize}
(Alter. pop. de \textunderscore enfrenesiar\textunderscore )
\section{Enferretar}
\begin{itemize}
\item {Grp. gram.:v. t.}
\end{itemize}
\begin{itemize}
\item {Utilização:Prov.}
\end{itemize}
\begin{itemize}
\item {Utilização:minh.}
\end{itemize}
Enferrujar; sujar.
\section{Enferrujar}
\begin{itemize}
\item {Grp. gram.:v. t.}
\end{itemize}
\begin{itemize}
\item {Grp. gram.:V. i.}
\end{itemize}
\begin{itemize}
\item {Utilização:Fig.}
\end{itemize}
Tornar ferrujento.
Criar ferrugem ou alforra.
Estar sem uso, sem exercício.
\section{Enfesta}
\begin{itemize}
\item {Grp. gram.:f.}
\end{itemize}
\begin{itemize}
\item {Proveniência:(Do lat. \textunderscore fastigium\textunderscore )}
\end{itemize}
Cumeada, cumeeira, picoto.
\section{Enfestado}
\begin{itemize}
\item {Grp. gram.:adj.}
\end{itemize}
\begin{itemize}
\item {Proveniência:(De \textunderscore enfestar\textunderscore )}
\end{itemize}
Dobrado em sua largura.
Largo, (falando-se de panos).
\section{Enfestar}
\begin{itemize}
\item {Grp. gram.:v. t.}
\end{itemize}
\begin{itemize}
\item {Proveniência:(De \textunderscore fêsto\textunderscore )}
\end{itemize}
Dobrar pelo meio, na sua largura.
Enrolar, assim dobrado na peça, (um pano ou fazenda).
\section{Enfèstar-se}
\begin{itemize}
\item {Grp. gram.:v. p.}
\end{itemize}
\begin{itemize}
\item {Utilização:Des.}
\end{itemize}
\begin{itemize}
\item {Proveniência:(De \textunderscore festa\textunderscore )}
\end{itemize}
Tornar-se festivo, alegre. Cf. G. Vicente. I, 265.
\section{Enfesto}
\begin{itemize}
\item {Grp. gram.:adj.}
\end{itemize}
\begin{itemize}
\item {Utilização:Des.}
\end{itemize}
Declive; escarpado.
(Da mesma or. de \textunderscore enfesta\textunderscore )
\section{Enfestoar}
\begin{itemize}
\item {Grp. gram.:v. t.}
\end{itemize}
O mesmo que \textunderscore afestoar\textunderscore .
\section{Enfeudação}
\begin{itemize}
\item {Grp. gram.:f.}
\end{itemize}
Acto de enfeudar.
\section{Enfeudar}
\begin{itemize}
\item {Grp. gram.:v. t.}
\end{itemize}
\begin{itemize}
\item {Utilização:Fig.}
\end{itemize}
Dar em feudo; converter em feudo.
Sujeitar á sua vontade, ao seu parecer.
\section{Enfezado}
\begin{itemize}
\item {fónica:fé}
\end{itemize}
\begin{itemize}
\item {Grp. gram.:adj.}
\end{itemize}
\begin{itemize}
\item {Proveniência:(De \textunderscore enfezar\textunderscore )}
\end{itemize}
Que se não desenvolveu sufficientemente.
Rachítico.
\section{Enfezamento}
\begin{itemize}
\item {fónica:fé}
\end{itemize}
\begin{itemize}
\item {Grp. gram.:m.}
\end{itemize}
Acto ou effeito de enfezar.
\section{Enfezar}
\begin{itemize}
\item {fónica:fé}
\end{itemize}
\begin{itemize}
\item {Grp. gram.:v. t.}
\end{itemize}
\begin{itemize}
\item {Utilização:Fig.}
\end{itemize}
Tolher o desenvolvimento de.
Fazer que cresça pouco.
Tornar rachítico.
Causar fezes a.
Enfastiar; agastar.
\section{Enfiação}
\begin{itemize}
\item {Grp. gram.:f.}
\end{itemize}
(V.enfiamento)
\section{Enfiada}
\begin{itemize}
\item {Grp. gram.:f.}
\end{itemize}
\begin{itemize}
\item {Proveniência:(De \textunderscore enfiar\textunderscore )}
\end{itemize}
Porção de coisas, dispostas em linha.
Fileira; série: \textunderscore uma enfiada de cadeiras\textunderscore .
\section{Enfiadura}
\begin{itemize}
\item {Grp. gram.:f.}
\end{itemize}
\begin{itemize}
\item {Proveniência:(De \textunderscore enfiar\textunderscore )}
\end{itemize}
Porção de linha, que se enfia de cada vez por uma agulha.
Enfiada de contas, pérolas, etc.
Fio.
Orifício ou fundo da agulha.
Enfiamento.
\section{Enfiamento}
\begin{itemize}
\item {Grp. gram.:m.}
\end{itemize}
Fileira.
Acto ou effeito de enfiar.
\section{Enfiar}
\begin{itemize}
\item {Grp. gram.:v. t.}
\end{itemize}
\begin{itemize}
\item {Grp. gram.:V. i.}
\end{itemize}
\begin{itemize}
\item {Proveniência:(Do lat. \textunderscore infilare\textunderscore )}
\end{itemize}
Introduzir num orifício: \textunderscore enfiar uma linha\textunderscore .
Meter fio no orifício de (uma agulha).
Pôr em fio.
Pôr em série.
Vestir: \textunderscore enfiar umas calças\textunderscore .
Calçar.
Traspassar.
Ligar o fio interrompido de (um discurso).
Perder a côr.
Desmaiar.
Entrar; encaminhar-se directamente: \textunderscore enfiou pelo bêco\textunderscore .
\section{Enfileiramento}
\begin{itemize}
\item {Grp. gram.:m.}
\end{itemize}
Acto de enfileirar.
\section{Enfileirar}
\begin{itemize}
\item {Grp. gram.:v. t.}
\end{itemize}
Dispor em fileira, em linha; alinhar.
\section{Enfinamente}
\begin{itemize}
\item {Grp. gram.:adv.}
\end{itemize}
\begin{itemize}
\item {Utilização:Prov.}
\end{itemize}
\begin{itemize}
\item {Utilização:minh.}
\end{itemize}
\begin{itemize}
\item {Proveniência:(De \textunderscore em\textunderscore  + \textunderscore fim\textunderscore )}
\end{itemize}
O mesmo que \textunderscore finalmente\textunderscore .
\section{Enficar}
\begin{itemize}
\item {Grp. gram.:v. t.}
\end{itemize}
\begin{itemize}
\item {Utilização:Prov.}
\end{itemize}
\begin{itemize}
\item {Utilização:minh.}
\end{itemize}
Encostar, arrumar.
(Cp. \textunderscore fincar\textunderscore )
\section{Enfisema}
\begin{itemize}
\item {Grp. gram.:m.}
\end{itemize}
\begin{itemize}
\item {Utilização:Med.}
\end{itemize}
\begin{itemize}
\item {Proveniência:(Gr. \textunderscore emphusema\textunderscore )}
\end{itemize}
Tumor branco elástico, causado pela infiltração do ar no tecido celular.
\section{Enfisemático}
\begin{itemize}
\item {Grp. gram.:adj.}
\end{itemize}
Relativo a enfisema.
\section{Enfisematoso}
\begin{itemize}
\item {Grp. gram.:adj.}
\end{itemize}
O mesmo que \textunderscore enfisemático\textunderscore .
\section{Enfistular}
\begin{itemize}
\item {Grp. gram.:v. t.}
\end{itemize}
\begin{itemize}
\item {Grp. gram.:V. i.  e  p.}
\end{itemize}
\begin{itemize}
\item {Proveniência:(De \textunderscore fístula\textunderscore )}
\end{itemize}
Tornar fistuloso.
Criar fístulas; ulcerar-se.
\section{Enfitar}
\begin{itemize}
\item {Grp. gram.:v. t.}
\end{itemize}
Pôr fitas em.
Adornar com fitas.
Enfeitar.
\section{Enfitar-se}
\begin{itemize}
\item {Grp. gram.:v. p.}
\end{itemize}
\begin{itemize}
\item {Proveniência:(De \textunderscore fitar\textunderscore )}
\end{itemize}
Fixar a vista:«\textunderscore enfitei-me bem nella\textunderscore ». Camillo.
\section{Enfiteuse}
\begin{itemize}
\item {Grp. gram.:f.}
\end{itemize}
\begin{itemize}
\item {Utilização:Jur.}
\end{itemize}
\begin{itemize}
\item {Proveniência:(Gr. \textunderscore emphuteusis\textunderscore )}
\end{itemize}
Convenção, pela qual o senhor de um prédio transfere para outrem o domínio útil do mesmo prédio, obrigando-se o cessionário a pagar-lhe uma pensão anual que se chama fôro.
Aforamento.
\section{Enfiteuta}
\begin{itemize}
\item {Grp. gram.:m.  e  f.}
\end{itemize}
\begin{itemize}
\item {Proveniência:(Gr. \textunderscore emphuteutes\textunderscore )}
\end{itemize}
Pêssoa, que recebe ou tem o domínio útil de um prédio, por contrato de enfiteuse.
\section{Enfiteuticação}
\begin{itemize}
\item {Grp. gram.:f.}
\end{itemize}
Acto ou efeito de enfiteuticar. Cf. Herculano, \textunderscore Opúsculos\textunderscore , IV, 71.
\section{Enfiteuticar}
\begin{itemize}
\item {Grp. gram.:v. t.}
\end{itemize}
\begin{itemize}
\item {Proveniência:(De \textunderscore enfitêutico\textunderscore )}
\end{itemize}
Aforar.
Ceder por enfiteuse.
\section{Enfiteuticário}
\begin{itemize}
\item {Grp. gram.:adj.}
\end{itemize}
\begin{itemize}
\item {Utilização:Des.}
\end{itemize}
O mesmo que \textunderscore enfitêutico\textunderscore .
\section{Enfitêutico}
\begin{itemize}
\item {Grp. gram.:adj.}
\end{itemize}
Relativo a enfiteuse.
(Cp. \textunderscore enfiteuta\textunderscore )
\section{Enfiuzado}
\begin{itemize}
\item {fónica:fi-u}
\end{itemize}
\begin{itemize}
\item {Grp. gram.:adj.}
\end{itemize}
\begin{itemize}
\item {Utilização:Prov.}
\end{itemize}
\begin{itemize}
\item {Utilização:beir.}
\end{itemize}
Magro, cadavérico.
(Cp. \textunderscore enfiado\textunderscore )
\section{Enfiuzar}
\begin{itemize}
\item {fónica:fi-u}
\end{itemize}
\textunderscore v. t.\textunderscore  (e der.)
Corr. alent. de \textunderscore enviezar\textunderscore , etc.
\section{Enfivelamento}
\begin{itemize}
\item {Grp. gram.:m.}
\end{itemize}
Acto de enfivelar.
\section{Enfivelar}
\begin{itemize}
\item {Grp. gram.:v. t.}
\end{itemize}
Pôr fivela em.
Ataviar com fivelas.
\section{Enflanelar}
\begin{itemize}
\item {Grp. gram.:v. t.}
\end{itemize}
Revestir de flanela.
\section{Enflorar}
\begin{itemize}
\item {Grp. gram.:v. t.}
\end{itemize}
\begin{itemize}
\item {Utilização:Fig.}
\end{itemize}
\begin{itemize}
\item {Proveniência:(De \textunderscore flôr\textunderscore )}
\end{itemize}
Fazer nascer flôres em: \textunderscore enflorar um canteiro\textunderscore .
Tornar florido.
Ornar de flôres: \textunderscore enflorar um altar\textunderscore .
Tornar feliz, próspero, alegre: \textunderscore enflorar a vida\textunderscore .
\section{Enflorear}
\begin{itemize}
\item {Grp. gram.:v. t.}
\end{itemize}
O mesmo que \textunderscore enflorar\textunderscore . Cf. Garrett, \textunderscore Retr. de Vênus\textunderscore , 17; Camillo, \textunderscore Myst. de Lisb.\textunderscore , II, 125.
\section{Enflorecer}
\begin{itemize}
\item {Grp. gram.:v. t.  e  i.}
\end{itemize}
O mesmo que \textunderscore florescer\textunderscore .
\section{Enflorescer}
\begin{itemize}
\item {Grp. gram.:v. t.  e  i.}
\end{itemize}
(V.florescer)
\section{Enflorestado}
\begin{itemize}
\item {Grp. gram.:adj.}
\end{itemize}
Diz-se do terreno onde há florestas.
\section{Ênfobo}
\begin{itemize}
\item {Grp. gram.:m.}
\end{itemize}
Quadrúpede selvagem, semelhante a um cavalo e mencionado na \textunderscore Etiópia Or.\textunderscore , l. I, c. I.
\section{Enfogar}
\begin{itemize}
\item {Grp. gram.:v. t.}
\end{itemize}
Pôr em fogo, tornar ardente.
Afoguear.
\section{Enfolar}
\begin{itemize}
\item {Grp. gram.:v. t.}
\end{itemize}
Dar aspecto de fole a.
Enfolechar, empolar.
Inflar.
\section{Enfolhado}
\begin{itemize}
\item {Grp. gram.:adj.}
\end{itemize}
\begin{itemize}
\item {Proveniência:(De \textunderscore enfolhar\textunderscore )}
\end{itemize}
Vestido de fôlhas:«\textunderscore revoadas de pardaes esfusiavam estridentes das sebes enfolhadas...\textunderscore ». Camillo, \textunderscore Volcões\textunderscore , 109.
\section{Enfolhamento}
\begin{itemize}
\item {Grp. gram.:m.}
\end{itemize}
Acto ou effeito de enfolhar.
\section{Enfolhar}
\begin{itemize}
\item {Grp. gram.:v. i.}
\end{itemize}
\begin{itemize}
\item {Grp. gram.:V. t.}
\end{itemize}
Criar fôlhas.
Cobrir de fôlhas.
\section{Enfolipado}
\begin{itemize}
\item {Grp. gram.:adj.}
\end{itemize}
Que fórma fole ou folipo:«\textunderscore o cavalo, com os ilhaes enfolipados...\textunderscore ». Camillo, \textunderscore Corja\textunderscore , 250.
\section{Enfolipar}
\begin{itemize}
\item {Grp. gram.:v. t.}
\end{itemize}
Produzir folipo em.
Formar seio ou fole em (peça mal costurada).
\section{Enfollar}
\begin{itemize}
\item {Grp. gram.:v. t.}
\end{itemize}
Dar aspecto de folle a.
Enfollechar, empolar.
Inflar.
\section{Enfollechar}
\begin{itemize}
\item {Grp. gram.:v. t.}
\end{itemize}
\begin{itemize}
\item {Grp. gram.:V. i.  e  p.}
\end{itemize}
Produzir follecho ou follechos em.
Criar follechos.
Empolar.
\section{Enfollipado}
\begin{itemize}
\item {Grp. gram.:adj.}
\end{itemize}
Que fórma folle ou follipo:«\textunderscore o cavallo, com os ilhaes enfollipados...\textunderscore ». Camillo, \textunderscore Corja\textunderscore , 250.
\section{Enfollipar}
\begin{itemize}
\item {Grp. gram.:v. t.}
\end{itemize}
Produzir follipo em.
Formar seio ou folle em (peça mal costurada).
\section{Enforcadinho}
\begin{itemize}
\item {Grp. gram.:m.}
\end{itemize}
Espécie de orchídea do Brasil.
\section{Enforcado}
\begin{itemize}
\item {Grp. gram.:m.}
\end{itemize}
\begin{itemize}
\item {Proveniência:(De \textunderscore enforcar\textunderscore )}
\end{itemize}
Aquelle que se enforcou.
Aquelle que se asphyxiou por suspensão.
O suppliciado na forca.
Parreira, videira, que se enrosca e empa nas árvores.
\section{Enforcamento}
\begin{itemize}
\item {Grp. gram.:m.}
\end{itemize}
Acto de enforcar.
\section{Enforcar}
\begin{itemize}
\item {Grp. gram.:v. t.}
\end{itemize}
\begin{itemize}
\item {Utilização:Fig.}
\end{itemize}
\begin{itemize}
\item {Proveniência:(Do b. lat. \textunderscore infurcare\textunderscore )}
\end{itemize}
Suppliciar, suspendendo pelo pescoço na forca, ou em lugar alto, como em trave, ramo de árvore, etc.
Estrangular.
Renunciar.
Vender barato.
Esbanjar.
\section{Enforjar}
\begin{itemize}
\item {Grp. gram.:v. t.}
\end{itemize}
Meter na forja.
\section{Enformação}
\begin{itemize}
\item {Grp. gram.:f.}
\end{itemize}
Acto de enformar.
\section{Enformadeira}
\begin{itemize}
\item {Grp. gram.:f.}
\end{itemize}
Máquina para enformar chapéus. Cf. \textunderscore Inquér. Industr\textunderscore ., 2.^a p., vol. II, 175.
\section{Enformar}
\begin{itemize}
\item {Grp. gram.:v. t.}
\end{itemize}
Meter na fôrma.
\section{Enformar}
\begin{itemize}
\item {Grp. gram.:v. i.}
\end{itemize}
\begin{itemize}
\item {Grp. gram.:V. t.}
\end{itemize}
\begin{itemize}
\item {Proveniência:(De formar)}
\end{itemize}
Crescer, tomar corpo, desenvolver-se.
Dar fórma a. Cf. Latino, \textunderscore Or. da Corôa\textunderscore , XXXVIII.
\section{Enformosar}
\begin{itemize}
\item {Grp. gram.:v. t.}
\end{itemize}
Tornar formoso, aformosear. Cf. Filinto, VIII, 266.
\section{Enforna}
\begin{itemize}
\item {Grp. gram.:f.}
\end{itemize}
\begin{itemize}
\item {Utilização:T. da Bairrada}
\end{itemize}
Acto de enfornar.
\section{Enfornar}
\begin{itemize}
\item {Grp. gram.:v. t.}
\end{itemize}
Meter no forno.
(B. lat. \textunderscore infurnare\textunderscore )
\section{Enforro}
\begin{itemize}
\item {fónica:fô}
\end{itemize}
\begin{itemize}
\item {Grp. gram.:m.}
\end{itemize}
(V. \textunderscore fôrro\textunderscore ^1)
\section{Enfortecer}
\begin{itemize}
\item {Grp. gram.:V. i.}
\end{itemize}
Tornar-se forte:«\textunderscore enfortecei mãos de froxos &amp; giolhos de fracos\textunderscore . »Usque, 38, v.
\section{Enfortir}
\begin{itemize}
\item {Grp. gram.:v. t.}
\end{itemize}
\begin{itemize}
\item {Proveniência:(De \textunderscore forte\textunderscore )}
\end{itemize}
Pisoar; dar fortaleza a (panos).
\section{Enfragado}
\begin{itemize}
\item {Grp. gram.:adj.}
\end{itemize}
\begin{itemize}
\item {Proveniência:(De \textunderscore enfragar\textunderscore )}
\end{itemize}
Formado por fragas.
\section{Enfragar}
\begin{itemize}
\item {Grp. gram.:v. i.}
\end{itemize}
Dar em fraga.
Parar em fraga, (uma mina, um caminho, etc.).
\section{Enfranque}
\begin{itemize}
\item {Grp. gram.:m.}
\end{itemize}
\begin{itemize}
\item {Utilização:Prov.}
\end{itemize}
\begin{itemize}
\item {Utilização:minh.}
\end{itemize}
\begin{itemize}
\item {Proveniência:(De \textunderscore enfranquear\textunderscore )}
\end{itemize}
Curva do calçado, correspondente ao sellado do pé.
Curva no fato, correspondente á ilharga.
Curva da meia, correspondente á barriga da perna.
\section{Enfranquear}
\begin{itemize}
\item {Grp. gram.:v. t.}
\end{itemize}
\begin{itemize}
\item {Proveniência:(De \textunderscore franco\textunderscore )}
\end{itemize}
Fazer o enfranque de.
Brunir o enfranque de (calçado).
\section{Enfraquecedor}
\begin{itemize}
\item {Grp. gram.:adj.}
\end{itemize}
Que enfraquece, que torna fraco. Cf. Eça, \textunderscore P. Basilio\textunderscore , 236.
\section{Enfraquecer}
\begin{itemize}
\item {Grp. gram.:v. t.}
\end{itemize}
\begin{itemize}
\item {Grp. gram.:V. i.  e  p.}
\end{itemize}
\begin{itemize}
\item {Proveniência:(De \textunderscore fraco\textunderscore )}
\end{itemize}
Tirar as fôrças a.
Tornar fraco.
Perder as fôrças.
Enervar-se.
\section{Enfraquecimento}
\begin{itemize}
\item {Grp. gram.:m.}
\end{itemize}
Acto ou effeito de enfraquecer.
\section{Enfraquentar}
\begin{itemize}
\item {Grp. gram.:v. t.  e  i.}
\end{itemize}
O mesmo que \textunderscore enfraquecer\textunderscore .
\section{Enfrascado}
\begin{itemize}
\item {Grp. gram.:adj.}
\end{itemize}
\begin{itemize}
\item {Utilização:Fam.}
\end{itemize}
\begin{itemize}
\item {Proveniência:(De \textunderscore enfrascar\textunderscore )}
\end{itemize}
Metido em franco.
Bêbedo, ébrio.
\section{Enfrascar}
\begin{itemize}
\item {Grp. gram.:v. t.}
\end{itemize}
\begin{itemize}
\item {Utilização:Fig.}
\end{itemize}
\begin{itemize}
\item {Grp. gram.:V. i.}
\end{itemize}
\begin{itemize}
\item {Utilização:Prov.}
\end{itemize}
\begin{itemize}
\item {Utilização:alg.}
\end{itemize}
\begin{itemize}
\item {Grp. gram.:V. p.}
\end{itemize}
\begin{itemize}
\item {Utilização:Prov.}
\end{itemize}
\begin{itemize}
\item {Utilização:trasm.}
\end{itemize}
\begin{itemize}
\item {Proveniência:(De \textunderscore frasco\textunderscore )}
\end{itemize}
Meter em frasco.
Impregnar.
Embeber, encher.
Enredar.
Tornar encarniçado.
Fartar.
Causar enjôo, aborrecimento.
Andar metido por alcoices ou dominado por mulheres.
\section{Enfrático}
\begin{itemize}
\item {Grp. gram.:adj.}
\end{itemize}
\begin{itemize}
\item {Utilização:Med.}
\end{itemize}
\begin{itemize}
\item {Proveniência:(Gr. \textunderscore emphratikos\textunderscore )}
\end{itemize}
Que obstrue.
\section{Enfraxia}
\begin{itemize}
\item {fónica:csi}
\end{itemize}
\begin{itemize}
\item {Grp. gram.:f.}
\end{itemize}
\begin{itemize}
\item {Utilização:Med.}
\end{itemize}
O mesmo que \textunderscore obstrucção\textunderscore .
(Cp. \textunderscore enfrático\textunderscore )
\section{Enfreador}
\begin{itemize}
\item {Grp. gram.:adj.}
\end{itemize}
\begin{itemize}
\item {Grp. gram.:M.}
\end{itemize}
Que enfreia.
Aquelle que enfreia.
\section{Enfreamento}
\begin{itemize}
\item {Grp. gram.:m.}
\end{itemize}
Acto ou effeito de enfrear.
\section{Enfrear}
\begin{itemize}
\item {Grp. gram.:v. t.}
\end{itemize}
\begin{itemize}
\item {Utilização:Fig.}
\end{itemize}
\begin{itemize}
\item {Proveniência:(Do Lat. \textunderscore infrenare\textunderscore )}
\end{itemize}
Pôr freio a.
Sujeitar ao freio.
Moderar.
Conter, reprimir: \textunderscore enfrear paixões\textunderscore .
Domar.
\section{Enfrechadura}
\begin{itemize}
\item {Grp. gram.:f.}
\end{itemize}
O mesmo que \textunderscore enfrechate\textunderscore .
\section{Enfrechar}
\begin{itemize}
\item {Grp. gram.:v.}
\end{itemize}
\begin{itemize}
\item {Utilização:t. Náut.}
\end{itemize}
\begin{itemize}
\item {Proveniência:(De \textunderscore frecha\textunderscore )}
\end{itemize}
Pôr enfrechates em.
\section{Enfrechate}
\begin{itemize}
\item {Grp. gram.:m.}
\end{itemize}
\begin{itemize}
\item {Utilização:Náut.}
\end{itemize}
\begin{itemize}
\item {Proveniência:(De \textunderscore enfrechar\textunderscore )}
\end{itemize}
Cada um dos cabos parallelos e horizontaes, nos ovens das enxárcias.
\section{Enfrenar}
\begin{itemize}
\item {Grp. gram.:v. t.}
\end{itemize}
\begin{itemize}
\item {Utilização:Bras. do S}
\end{itemize}
O mesmo que \textunderscore enfrear\textunderscore .
\section{Enfrenesiar}
\begin{itemize}
\item {Grp. gram.:v. t.}
\end{itemize}
Causar frenesi a.
Impacientar.
\section{Enfrentar}
\begin{itemize}
\item {Grp. gram.:v. t.}
\end{itemize}
\begin{itemize}
\item {Utilização:Bras}
\end{itemize}
\begin{itemize}
\item {Proveniência:(De \textunderscore frente\textunderscore )}
\end{itemize}
Encarar.
Atacar de frente; defrontar.
\section{Enfrestar}
\begin{itemize}
\item {Grp. gram.:v. t.}
\end{itemize}
Fazer frestas em.
Separar por frestas.
\section{Enfriar}
\begin{itemize}
\item {Grp. gram.:v. t.}
\end{itemize}
Tornar frio.
Deixar esfriar.
\section{Enfroixecer}
\begin{itemize}
\item {Grp. gram.:v. t.}
\end{itemize}
Tornar froixo.
\section{Enfrondar}
\begin{itemize}
\item {Grp. gram.:v. t.}
\end{itemize}
Vestir de frondes, tornar frondoso. Cf. Castilho, \textunderscore Fastos\textunderscore , II, 187.
\section{Enfronhado}
\begin{itemize}
\item {Grp. gram.:adj.}
\end{itemize}
\begin{itemize}
\item {Proveniência:(De \textunderscore enfronhar\textunderscore )}
\end{itemize}
Versado ou conhecedor: \textunderscore enfronhado em Phýsica\textunderscore .
\section{Enfronhar}
\begin{itemize}
\item {Grp. gram.:v. t.}
\end{itemize}
\begin{itemize}
\item {Utilização:Fig.}
\end{itemize}
\begin{itemize}
\item {Proveniência:(De \textunderscore fronha\textunderscore )}
\end{itemize}
Meter em fronha.
Revestir, encapar.
Vestir apressadamente.
Tornar versado, instruido.
Dissimular.
\section{Enfrontes}
\begin{itemize}
\item {Grp. gram.:m. pl.}
\end{itemize}
\begin{itemize}
\item {Utilização:T. de Portel}
\end{itemize}
Situação fronteira; rumo, direcção.
\section{Enfrouxecer}
\begin{itemize}
\item {Grp. gram.:v. t.}
\end{itemize}
Tornar frouxo.
\section{Enfrutecer}
\begin{itemize}
\item {Grp. gram.:v. i.}
\end{itemize}
Têr fruto, dar fruto. Cf. Camillo, \textunderscore Cancion. Al.\textunderscore , 80.
\section{Enfuar}
\begin{itemize}
\item {Grp. gram.:v. t.}
\end{itemize}
\begin{itemize}
\item {Utilização:Prov.}
\end{itemize}
\begin{itemize}
\item {Proveniência:(Do lat. \textunderscore infunare\textunderscore )}
\end{itemize}
Vestir: \textunderscore enfuar umas calças\textunderscore .
\section{Enfueirada}
\begin{itemize}
\item {Grp. gram.:f.}
\end{itemize}
\begin{itemize}
\item {Utilização:Pop.}
\end{itemize}
\begin{itemize}
\item {Proveniência:(De \textunderscore enfueirar\textunderscore )}
\end{itemize}
Grande carrada.
Carro cheio até ao cimo dos fueiros.
\section{Enfueirar}
\begin{itemize}
\item {Grp. gram.:v. t.}
\end{itemize}
\begin{itemize}
\item {Proveniência:(De \textunderscore fueiro\textunderscore )}
\end{itemize}
Pôr fueiros em.
Elevar (a carrada) até á altura dos fueiros.
Aconchegar as extremidades superiores dos fueiros, para apertar (a carrada).
\section{Enfulijar}
\begin{itemize}
\item {Grp. gram.:v. t.}
\end{itemize}
Sujar com fuligem; mascarrar.
\section{Enfumaçado}
\begin{itemize}
\item {Grp. gram.:adj.}
\end{itemize}
Envolto em fumo; toldado pelo fumo. Cf. Castilho, \textunderscore Fausto\textunderscore , 66; Camillo, \textunderscore Narcót.\textunderscore , I, 133.
\section{Enfumar}
\begin{itemize}
\item {Grp. gram.:v. t.}
\end{itemize}
O mesmo que \textunderscore enfumarar\textunderscore .
\section{Enfumarar}
\begin{itemize}
\item {Grp. gram.:v. t.}
\end{itemize}
Cobrir ou encher de fumo.
(Cp. \textunderscore fumarada\textunderscore )
\section{Enfunado}
\begin{itemize}
\item {Grp. gram.:adj.}
\end{itemize}
\begin{itemize}
\item {Utilização:Fig.}
\end{itemize}
\begin{itemize}
\item {Utilização:Prov.}
\end{itemize}
\begin{itemize}
\item {Utilização:trasm.}
\end{itemize}
\begin{itemize}
\item {Proveniência:(De \textunderscore enfunar\textunderscore )}
\end{itemize}
Retesado; pando.
Vaidoso.
Amuado, zangado.
\section{Enfunar}
\begin{itemize}
\item {Grp. gram.:v. t.}
\end{itemize}
\begin{itemize}
\item {Utilização:Fig.}
\end{itemize}
\begin{itemize}
\item {Grp. gram.:V. p.}
\end{itemize}
\begin{itemize}
\item {Utilização:Pop.}
\end{itemize}
\begin{itemize}
\item {Proveniência:(Do lat. \textunderscore funis\textunderscore )}
\end{itemize}
Retesar, tornar pando: \textunderscore enfunar velas de navio\textunderscore .
Encher.
Tornar orgulhoso, envaidar.
Amuar-se; irritar-se.
\section{Enfunilamento}
\begin{itemize}
\item {Grp. gram.:m.}
\end{itemize}
Acto ou effeito de enfunilar.
\section{Enfunilar}
\begin{itemize}
\item {Grp. gram.:v. t.}
\end{itemize}
O mesmo que \textunderscore afunilar\textunderscore .
\section{Enfuniscar}
\begin{itemize}
\item {Grp. gram.:v. i.}
\end{itemize}
\begin{itemize}
\item {Utilização:Prov.}
\end{itemize}
\begin{itemize}
\item {Utilização:minh.}
\end{itemize}
Zangar-se ou amuar, carregando o semblante, por ouvir coisa desagradável.
(Relaciona-se com \textunderscore enfunar\textunderscore ?)
\section{Enfurdar}
\begin{itemize}
\item {Grp. gram.:v. t.}
\end{itemize}
\begin{itemize}
\item {Utilização:Prov.}
\end{itemize}
Pôr barbilho a (cordeiros).
(Relaciona-se com \textunderscore furda\textunderscore ?)
\section{Enfurecer}
\begin{itemize}
\item {Grp. gram.:v. t.}
\end{itemize}
\begin{itemize}
\item {Grp. gram.:V. i.}
\end{itemize}
\begin{itemize}
\item {Grp. gram.:V. p.}
\end{itemize}
\begin{itemize}
\item {Utilização:Fig.}
\end{itemize}
\begin{itemize}
\item {Proveniência:(Do lat. \textunderscore furere\textunderscore )}
\end{itemize}
Tornar furioso.
Irar.
Irar-se.
Tornar-se furioso.
Encapellar-se: \textunderscore enfurecer-se o mar\textunderscore .
Desencadear-se: \textunderscore enfurecer-se o vento\textunderscore .
\section{Enfuriar}
\begin{itemize}
\item {Grp. gram.:v. t.}
\end{itemize}
\begin{itemize}
\item {Proveniência:(De \textunderscore fúria\textunderscore )}
\end{itemize}
O mesmo que \textunderscore enfurecer\textunderscore .
\section{Enfurnar}
\begin{itemize}
\item {Grp. gram.:v. t.}
\end{itemize}
\begin{itemize}
\item {Utilização:Náut.}
\end{itemize}
\begin{itemize}
\item {Proveniência:(De \textunderscore furna\textunderscore )}
\end{itemize}
Encafuar.
Introduzir o pé de (os mastros) no lugar próprio.
Encher de furnas. Cf. Filinto, \textunderscore D. Man.\textunderscore , I, 68.
\section{Enfuscar}
\begin{itemize}
\item {Grp. gram.:v. t.}
\end{itemize}
Tornar fusco, mascarrar.
Tornar escuro; offuscar.
\section{Enfustar}
\begin{itemize}
\item {Grp. gram.:v. i.}
\end{itemize}
\begin{itemize}
\item {Utilização:Prov.}
\end{itemize}
\begin{itemize}
\item {Utilização:alent.}
\end{itemize}
\begin{itemize}
\item {Proveniência:(De \textunderscore enfuste\textunderscore ?)}
\end{itemize}
Entrar ou saír apressadamente.
Deixar a companhia de alguém, sém se despedir.
\section{Enfuste}
\begin{itemize}
\item {Grp. gram.:m.}
\end{itemize}
\begin{itemize}
\item {Proveniência:(De \textunderscore fuste\textunderscore ?)}
\end{itemize}
Preparação, com que se entumecem as pelles.
Planta liliácea.
\section{Enga}
\begin{itemize}
\item {Grp. gram.:f.}
\end{itemize}
\begin{itemize}
\item {Utilização:Pop.}
\end{itemize}
\begin{itemize}
\item {Proveniência:(De \textunderscore engar\textunderscore )}
\end{itemize}
Pasto.
Parasitismo.
Costume; uso.
\section{Engá}
\begin{itemize}
\item {Grp. gram.:m.}
\end{itemize}
\begin{itemize}
\item {Grp. gram.:m.  ou  f.}
\end{itemize}
Planta brasileira. Cf. Crespo, \textunderscore Miniat.\textunderscore , 16.
Fruto da engazeira.
Engazeira.
(Do tupi)
\section{Engabelar}
\begin{itemize}
\item {Grp. gram.:v. t.}
\end{itemize}
\begin{itemize}
\item {Utilização:Bras}
\end{itemize}
Seduzir.
Sêr agradável a, para enganar.
(Corr. de \textunderscore engavelar\textunderscore )
\section{Engabêlo}
\begin{itemize}
\item {Grp. gram.:m.}
\end{itemize}
\begin{itemize}
\item {Utilização:Bras}
\end{itemize}
Acto de engabelar.
\section{Engaçar}
\begin{itemize}
\item {Grp. gram.:v. t.}
\end{itemize}
Esterroar com engaço ou ancinho.
Juntar com ancinho ou engaço.
\section{Engaço}
\begin{itemize}
\item {Grp. gram.:m.}
\end{itemize}
\begin{itemize}
\item {Utilização:Ant.}
\end{itemize}
\begin{itemize}
\item {Utilização:Prov.}
\end{itemize}
\begin{itemize}
\item {Utilização:trasm.}
\end{itemize}
\begin{itemize}
\item {Utilização:minh.}
\end{itemize}
A parte que fica do cacho esbagoado.
Bagaço.
Ancinho.
Instrumento, em fórma de T, com dentes de ferro ou de pau, para juntar a palha ou o feno.
\section{Engadanhado}
\begin{itemize}
\item {Grp. gram.:adj.}
\end{itemize}
\begin{itemize}
\item {Utilização:Prov.}
\end{itemize}
\begin{itemize}
\item {Utilização:beir.}
\end{itemize}
\begin{itemize}
\item {Proveniência:(De \textunderscore engadanhar-se\textunderscore )}
\end{itemize}
Muito avarento.
Sovina, miserável.
\section{Engadanhar-se}
\begin{itemize}
\item {Grp. gram.:v. p.}
\end{itemize}
\begin{itemize}
\item {Proveniência:(De \textunderscore gadanho\textunderscore )}
\end{itemize}
Têr as mãos hirtas ou tolhidas com frio.
Embaraçar-se, estar perplexo.
\section{Engadelha}
\begin{itemize}
\item {Grp. gram.:f.}
\end{itemize}
\begin{itemize}
\item {Utilização:Prov.}
\end{itemize}
\begin{itemize}
\item {Utilização:dur.}
\end{itemize}
\begin{itemize}
\item {Proveniência:(De \textunderscore engadelhar\textunderscore )}
\end{itemize}
Briga, luta, braço a braço.
\section{Engadelhar}
\begin{itemize}
\item {Grp. gram.:v. t.}
\end{itemize}
Converter em gadelha; desordenar (o cabello). Cf. Camillo, \textunderscore Narcót.\textunderscore , I, 236.
\section{Engafecer}
\begin{itemize}
\item {Grp. gram.:v. t.}
\end{itemize}
\begin{itemize}
\item {Grp. gram.:V. i.}
\end{itemize}
Tornar gafo.
Tornar-se, gafo.
\section{Engaiar}
\begin{itemize}
\item {Grp. gram.:v.}
\end{itemize}
\begin{itemize}
\item {Utilização:t. Náut.}
\end{itemize}
Introduzir linhas ou arrebens nas cóchas de (cabos).
\section{Engaio}
\begin{itemize}
\item {Grp. gram.:m.}
\end{itemize}
\begin{itemize}
\item {Proveniência:(De \textunderscore engaiar\textunderscore )}
\end{itemize}
Linha ou arrebém, que se mete nas cóchas dos cabos.
\section{Engaiolar}
\begin{itemize}
\item {Grp. gram.:v. t.}
\end{itemize}
\begin{itemize}
\item {Utilização:Pop.}
\end{itemize}
Meter em gaiola.
Meter na cadeia, prender.
\section{Engajado}
\begin{itemize}
\item {Grp. gram.:adj.}
\end{itemize}
\begin{itemize}
\item {Utilização:Prov.}
\end{itemize}
\begin{itemize}
\item {Utilização:minh.}
\end{itemize}
\begin{itemize}
\item {Grp. gram.:M.}
\end{itemize}
\begin{itemize}
\item {Proveniência:(De \textunderscore engajar\textunderscore )}
\end{itemize}
Alliciado para emigrar.
Diz-se dos instrumentos chamados engaços, quando dois travam os dentes reciprocamente, custando separá-los.
Indivíduo contratado para certos serviços.
\section{Engajador}
\begin{itemize}
\item {Grp. gram.:adj.}
\end{itemize}
\begin{itemize}
\item {Grp. gram.:M.}
\end{itemize}
Que engaja.
Aquelle que engaja.
\section{Engajamento}
\begin{itemize}
\item {Grp. gram.:m.}
\end{itemize}
Acto ou effeito de engajar.
\section{Engajar}
\begin{itemize}
\item {Grp. gram.:v. t.}
\end{itemize}
\begin{itemize}
\item {Utilização:Neol.}
\end{itemize}
\begin{itemize}
\item {Proveniência:(Fr. \textunderscore engager\textunderscore )}
\end{itemize}
Contratar para serviço pessoal.
Alliciar para emigração.
\section{Engajatado}
\begin{itemize}
\item {Grp. gram.:adj.}
\end{itemize}
\begin{itemize}
\item {Utilização:Prov.}
\end{itemize}
\begin{itemize}
\item {Utilização:trasm.}
\end{itemize}
\begin{itemize}
\item {Proveniência:(De \textunderscore gajata\textunderscore )}
\end{itemize}
Torcido, voltado.
\section{Engajavatado}
\begin{itemize}
\item {Grp. gram.:adj.}
\end{itemize}
\begin{itemize}
\item {Utilização:Prov.}
\end{itemize}
\begin{itemize}
\item {Utilização:trasm.}
\end{itemize}
O mesmo que \textunderscore engajatado\textunderscore .
\section{Engala}
\begin{itemize}
\item {Grp. gram.:f.}
\end{itemize}
\begin{itemize}
\item {Utilização:Prov.}
\end{itemize}
\begin{itemize}
\item {Utilização:beir.}
\end{itemize}
Cada uma das cavas do eixo, em que assenta o carro; romão. (Colhido na Guarda)
\section{Engalanar}
\begin{itemize}
\item {Grp. gram.:v. i.}
\end{itemize}
\begin{itemize}
\item {Proveniência:(De \textunderscore galan\textunderscore )}
\end{itemize}
Pôr galas em.
Ornamentar; ataviar.
Enflorar.
\section{Engalanear}
\begin{itemize}
\item {Grp. gram.:v. t.}
\end{itemize}
O mesmo que \textunderscore engalanar\textunderscore . Cf. Camillo, Narcót., II, 233.
\section{Engalapar}
\begin{itemize}
\item {Grp. gram.:v. i.}
\end{itemize}
\begin{itemize}
\item {Utilização:Prov.}
\end{itemize}
\begin{itemize}
\item {Utilização:trasm.}
\end{itemize}
Empenar, entortar-se (a madeira verde).
\section{Engalar}
\begin{itemize}
\item {Grp. gram.:v. t.}
\end{itemize}
\begin{itemize}
\item {Grp. gram.:V. i.}
\end{itemize}
\begin{itemize}
\item {Proveniência:(De \textunderscore galo\textunderscore )}
\end{itemize}
Levantar (o pescoço), arqueando-o, (falando-se de cavalo).
Embridar-se.
\section{Engalear-se}
\begin{itemize}
\item {Grp. gram.:v. p.}
\end{itemize}
O mesmo que \textunderscore engalinhar\textunderscore .
\section{Engaleirado}
\begin{itemize}
\item {Grp. gram.:adj.}
\end{itemize}
\begin{itemize}
\item {Utilização:Prov.}
\end{itemize}
\begin{itemize}
\item {Utilização:trasm.}
\end{itemize}
\begin{itemize}
\item {Proveniência:(De \textunderscore galleirão\textunderscore ? Ou corr. de \textunderscore engolleirado\textunderscore , de \textunderscore golleira\textunderscore , por \textunderscore colleira\textunderscore ?)}
\end{itemize}
Emproado, vaidoso.
\section{Engalfinhar-se}
\begin{itemize}
\item {Grp. gram.:v. p.}
\end{itemize}
\begin{itemize}
\item {Utilização:Pop.}
\end{itemize}
Travar-se em luta.
Brigar, corpo a corpo.
(Cp. \textunderscore engaliar-se\textunderscore )
\section{Engalgado}
\begin{itemize}
\item {Grp. gram.:adj.}
\end{itemize}
\begin{itemize}
\item {Utilização:Cyn.}
\end{itemize}
\begin{itemize}
\item {Proveniência:(De \textunderscore engalgar\textunderscore )}
\end{itemize}
Seguido por galgo ou galgos: \textunderscore uma lebre engalgada\textunderscore .
\section{Engalgar}
\begin{itemize}
\item {Grp. gram.:v.}
\end{itemize}
\begin{itemize}
\item {Utilização:t. Cyn.}
\end{itemize}
Mostrar (a lebre) aos galgos.
\section{Engalhar}
\begin{itemize}
\item {Grp. gram.:v. t.}
\end{itemize}
\begin{itemize}
\item {Utilização:Ant.}
\end{itemize}
\begin{itemize}
\item {Utilização:Prov.}
\end{itemize}
\begin{itemize}
\item {Utilização:minh.}
\end{itemize}
\begin{itemize}
\item {Utilização:Prov.}
\end{itemize}
\begin{itemize}
\item {Utilização:trasm.}
\end{itemize}
Embaraçar, impedir.
Assalariar.
Entreter, distrahir, desviar.
Agitar nos braços (a criança que chora).
(Cp. engajar)
\section{Engalhardear}
\textunderscore v. t.\textunderscore  (e der.)
O mesmo que \textunderscore engalhardetar\textunderscore , etc.
\section{Engalhardecer}
\begin{itemize}
\item {Grp. gram.:v. t.}
\end{itemize}
Tornar galhardo. Cp. Arn. Gama, \textunderscore Segr. do Abb.\textunderscore , 115.
\section{Engalhardetar}
\begin{itemize}
\item {Grp. gram.:v. t.}
\end{itemize}
Ornar de galhardetes, embandeirar.
\section{Engaliar-se}
\begin{itemize}
\item {Grp. gram.:v. p.}
\end{itemize}
\begin{itemize}
\item {Utilização:Prov.}
\end{itemize}
\begin{itemize}
\item {Utilização:trasm.}
\end{itemize}
\begin{itemize}
\item {Proveniência:(De \textunderscore galo\textunderscore  = \textunderscore gallo\textunderscore ?)}
\end{itemize}
O mesmo que \textunderscore engalfinhar-se\textunderscore .
\section{Engalicar}
\begin{itemize}
\item {Grp. gram.:v. t.}
\end{itemize}
\begin{itemize}
\item {Utilização:Prov.}
\end{itemize}
\begin{itemize}
\item {Utilização:dur.}
\end{itemize}
O mesmo que \textunderscore galicar\textunderscore .
\section{Engalinhar}
\begin{itemize}
\item {Grp. gram.:v. t.}
\end{itemize}
\begin{itemize}
\item {Utilização:Fam.}
\end{itemize}
Sêr mau agoiro para.
Encalistar.
Enguiçar.
\section{Engalispar-se}
\begin{itemize}
\item {Grp. gram.:v. p.}
\end{itemize}
\begin{itemize}
\item {Proveniência:(De \textunderscore galispo\textunderscore )}
\end{itemize}
Empavonar-se, encrespar-se como o galispo.
Entesar-se.
\section{Engallar}
\begin{itemize}
\item {Grp. gram.:v. t.}
\end{itemize}
\begin{itemize}
\item {Grp. gram.:V. i.}
\end{itemize}
\begin{itemize}
\item {Proveniência:(De gallo)}
\end{itemize}
Levantar (o pescoço), arqueando-o, (falando-se de cavallo).
Embridar-se.
\section{Engallear-se}
\begin{itemize}
\item {Grp. gram.:v. p.}
\end{itemize}
O mesmo que \textunderscore engallinhar\textunderscore .
\section{Engallinhar}
\begin{itemize}
\item {Grp. gram.:v. t.}
\end{itemize}
\begin{itemize}
\item {Utilização:Fam.}
\end{itemize}
Sêr mau agoiro para.
Encalistar.
Enguiçar.
\section{Engallispar-se}
\begin{itemize}
\item {Grp. gram.:v. p.}
\end{itemize}
\begin{itemize}
\item {Proveniência:(De \textunderscore gallispo\textunderscore )}
\end{itemize}
Empavonar-se, encrespar-se como o gallispo.
Entesar-se.
\section{Engambelar}
\textunderscore v. t.\textunderscore  (e der.)
O mesmo que \textunderscore engabelar\textunderscore , etc.
\section{Engambêlo}
\begin{itemize}
\item {Grp. gram.:m.}
\end{itemize}
\begin{itemize}
\item {Utilização:Bras}
\end{itemize}
Acto de engambelar.
\section{Engambitar}
\begin{itemize}
\item {Grp. gram.:v. t.}
\end{itemize}
\begin{itemize}
\item {Utilização:Bras}
\end{itemize}
Transpor a pé (fôsso, valla).
Atravessar.
(Cp. it. \textunderscore gambettare\textunderscore )
\section{Engamiado}
\begin{itemize}
\item {Grp. gram.:adj.}
\end{itemize}
\begin{itemize}
\item {Utilização:T. de Pare -de-Coira}
\end{itemize}
\begin{itemize}
\item {Utilização:des.}
\end{itemize}
O mesmo que \textunderscore entrèvado\textunderscore .
\section{Enganadamente}
\begin{itemize}
\item {Grp. gram.:adv.}
\end{itemize}
\begin{itemize}
\item {Proveniência:(De \textunderscore enganado\textunderscore )}
\end{itemize}
Com engano, traiçoeiramente.
\section{Enganadiço}
\begin{itemize}
\item {Grp. gram.:adj.}
\end{itemize}
Que se engana com facilidade.
\section{Enganado}
\begin{itemize}
\item {Grp. gram.:adj.}
\end{itemize}
\begin{itemize}
\item {Proveniência:(De \textunderscore enganar\textunderscore )}
\end{itemize}
Que se enganou.
Illudido.
Seduzido com promessas fallazes.
\section{Enganador}
\begin{itemize}
\item {Grp. gram.:adj.}
\end{itemize}
\begin{itemize}
\item {Grp. gram.:M.}
\end{itemize}
Que engana.
Aquelle que engana.
\section{Enganar}
\begin{itemize}
\item {Grp. gram.:v. t.}
\end{itemize}
\begin{itemize}
\item {Proveniência:(It. \textunderscore ingannare\textunderscore )}
\end{itemize}
Fazer caír em êrro.
Illudir; burlar.
Embair; seduzir.
\section{Engana-vista}
\begin{itemize}
\item {Grp. gram.:m.}
\end{itemize}
Aquillo que illude a vista, apresentando-se differente da sua realidade.
\section{Enganchar}
\begin{itemize}
\item {Grp. gram.:v. t.}
\end{itemize}
\begin{itemize}
\item {Grp. gram.:V. i.}
\end{itemize}
\begin{itemize}
\item {Grp. gram.:V. p.}
\end{itemize}
Prender com gancho.
Suspender com gancho ou com objecto análogo.
Enlaçar.
Diz-se do toiro, quando mete as hastes pelo fato do toireiro.
Enlaçar-se.
\section{Engangento}
\begin{itemize}
\item {Grp. gram.:adj.}
\end{itemize}
\begin{itemize}
\item {Utilização:Bras}
\end{itemize}
Rabugento, impertinente.
\section{Engangorrado}
\begin{itemize}
\item {Grp. gram.:adj.}
\end{itemize}
\begin{itemize}
\item {Utilização:Bras}
\end{itemize}
\begin{itemize}
\item {Proveniência:(De \textunderscore gangorra\textunderscore )}
\end{itemize}
Preso á armadilha, que se chama gangorra.
\section{Enganido}
\begin{itemize}
\item {Grp. gram.:adj.}
\end{itemize}
\begin{itemize}
\item {Utilização:Prov.}
\end{itemize}
\begin{itemize}
\item {Proveniência:(De \textunderscore enganir\textunderscore )}
\end{itemize}
Tolhido com frio; entanguido.
\section{Enganir}
\begin{itemize}
\item {Grp. gram.:v. i.}
\end{itemize}
\begin{itemize}
\item {Utilização:Ant.}
\end{itemize}
Tolher-se com frio.
\section{Engano}
\begin{itemize}
\item {Grp. gram.:m.}
\end{itemize}
Acto ou effeito de enganar.
Illusão.
Burla; lôgro.
Traição.
\section{Enganosamente}
\begin{itemize}
\item {Grp. gram.:adv.}
\end{itemize}
De modo enganoso.
\section{Enganoso}
\begin{itemize}
\item {Grp. gram.:adj.}
\end{itemize}
Que engana.
Em que há engano.
\section{Engar}
\begin{itemize}
\item {Grp. gram.:v. t.}
\end{itemize}
\begin{itemize}
\item {Grp. gram.:V. i.}
\end{itemize}
\begin{itemize}
\item {Proveniência:(Do lat. \textunderscore iniquare\textunderscore ?)}
\end{itemize}
Habituar-se a, preferir (um pasto).
Habituar-se.
Insistir, teimar.
\section{Engarampar}
\begin{itemize}
\item {Grp. gram.:v. t.}
\end{itemize}
\begin{itemize}
\item {Utilização:Prov.}
\end{itemize}
Enganar; engarapar.
(Por \textunderscore engrampar\textunderscore )
\section{Engaranhado}
\begin{itemize}
\item {Grp. gram.:adj.}
\end{itemize}
\begin{itemize}
\item {Utilização:Prov.}
\end{itemize}
\begin{itemize}
\item {Utilização:trasm.}
\end{itemize}
Enregelado; que tirita com frio.
\section{Engaranhido}
\begin{itemize}
\item {Grp. gram.:adj.}
\end{itemize}
\begin{itemize}
\item {Utilização:Prov.}
\end{itemize}
\begin{itemize}
\item {Utilização:trasm.}
\end{itemize}
O mesmo que \textunderscore engaranhado\textunderscore .
(Cp. \textunderscore engrunhido\textunderscore )
\section{Engarapar}
\begin{itemize}
\item {Grp. gram.:v. t.}
\end{itemize}
\begin{itemize}
\item {Utilização:Bras}
\end{itemize}
\begin{itemize}
\item {Utilização:Fig.}
\end{itemize}
Dar garapa a.
Embair; seduzir.
\section{Engaravitar-se}
\begin{itemize}
\item {Grp. gram.:v. p.}
\end{itemize}
Tornar-se hirto com frio.
(Por \textunderscore engaravetar\textunderscore , de \textunderscore garaveto\textunderscore )
\section{Engarbonar-se}
\begin{itemize}
\item {Grp. gram.:v. p.}
\end{itemize}
\begin{itemize}
\item {Utilização:Prov.}
\end{itemize}
\begin{itemize}
\item {Utilização:trasm.}
\end{itemize}
\begin{itemize}
\item {Proveniência:(De \textunderscore garbo\textunderscore )}
\end{itemize}
Vestir-se com o melhor fato.
\section{Engarfar}
\begin{itemize}
\item {Grp. gram.:v. i.}
\end{itemize}
\begin{itemize}
\item {Utilização:Geneal.}
\end{itemize}
Entroncar:«\textunderscore a genealogia dos Egmond engarfa na genealogia dos Nassaus\textunderscore ». Ortigão, \textunderscore Hollanda\textunderscore , 172.
\section{Engargantar}
\begin{itemize}
\item {Grp. gram.:v. t.}
\end{itemize}
\begin{itemize}
\item {Grp. gram.:V. i.}
\end{itemize}
\begin{itemize}
\item {Utilização:Bras}
\end{itemize}
\begin{itemize}
\item {Grp. gram.:V. p.}
\end{itemize}
Meter pela garganta abaixo.
Meter no estribo (o pé) até ao peito dêste.
Criar garganta ou novos gomos, perto do ôlho ou da fôlha, (falando-se da cana de açúcar).
Emperrar-se (a bala) no cano da espingarda, em vez de descer á culatra.
\section{Engarilho}
\begin{itemize}
\item {Grp. gram.:m.}
\end{itemize}
O mesmo que \textunderscore ingarilho\textunderscore .
\section{Engarrafadeira}
\begin{itemize}
\item {Grp. gram.:f.}
\end{itemize}
Máquina, para engarrafar cerveja e bebidas alcoólicas. Cf. \textunderscore Inquér. Industr.\textunderscore , P. II, l. 2.^o, 191.
\section{Engarrafado}
\begin{itemize}
\item {Grp. gram.:adj.}
\end{itemize}
\begin{itemize}
\item {Proveniência:(De \textunderscore engarrafar\textunderscore )}
\end{itemize}
Metido ou acondicionado em garrafa: \textunderscore vinho engarrafado\textunderscore .
\section{Engarrafador}
\begin{itemize}
\item {Grp. gram.:m.}
\end{itemize}
Utensílio para engarrafar; engarrafadeira.
\section{Engarrafagem}
\begin{itemize}
\item {Grp. gram.:f.}
\end{itemize}
Acto de engarrafar.
\section{Engarrafamento}
\begin{itemize}
\item {Grp. gram.:m.}
\end{itemize}
Acto de engarrafar.
\section{Engarrafar}
\begin{itemize}
\item {Grp. gram.:v. t.}
\end{itemize}
Meter ou fechar em garrafa.
\section{Engarrar}
\begin{itemize}
\item {Grp. gram.:v. i.}
\end{itemize}
\begin{itemize}
\item {Utilização:Prov.}
\end{itemize}
\begin{itemize}
\item {Utilização:trasm.}
\end{itemize}
\begin{itemize}
\item {Proveniência:(De \textunderscore garra\textunderscore )}
\end{itemize}
Trepar.
\section{Engarupar-se}
\begin{itemize}
\item {Grp. gram.:v. p.}
\end{itemize}
Montar na garupa.
\section{Engás}
\begin{itemize}
\item {Grp. gram.:m.}
\end{itemize}
O mesmo que \textunderscore engá\textunderscore .
\section{Engascado}
\begin{itemize}
\item {Grp. gram.:adj.}
\end{itemize}
\begin{itemize}
\item {Utilização:Prov.}
\end{itemize}
\begin{itemize}
\item {Utilização:trasm.}
\end{itemize}
O mesmo que [[endividado|endividar]].
(Provavelmente, por \textunderscore enrascado\textunderscore )
\section{Engasgalhar-se}
\begin{itemize}
\item {Grp. gram.:v. p.}
\end{itemize}
\begin{itemize}
\item {Utilização:Pop.}
\end{itemize}
Entalar-se.
Engasgar-se.
Ficar preso.
Brigar; lutar, braço a braço.
(Cp. \textunderscore engasgar\textunderscore )
\section{Engasgamento}
\begin{itemize}
\item {Grp. gram.:m.}
\end{itemize}
O mesmo que \textunderscore engasgo\textunderscore .
\section{Engasgar}
\begin{itemize}
\item {Grp. gram.:v. t.}
\end{itemize}
\begin{itemize}
\item {Grp. gram.:V. p.}
\end{itemize}
\begin{itemize}
\item {Utilização:Fig.}
\end{itemize}
\begin{itemize}
\item {Proveniência:(De \textunderscore engasgo\textunderscore )}
\end{itemize}
Obstruir a garganta de: \textunderscore engasgou-o um osso de gallinha\textunderscore .
Receber na garganta objecto que a obstrue.
Enlear-se.
Ficar entalado; embatucar.
\section{Engasgatar}
\begin{itemize}
\item {Grp. gram.:v. t.}
\end{itemize}
\begin{itemize}
\item {Utilização:Prov.}
\end{itemize}
\begin{itemize}
\item {Utilização:alg.}
\end{itemize}
\begin{itemize}
\item {Proveniência:(De \textunderscore engasgar\textunderscore )}
\end{itemize}
O mesmo que \textunderscore enfelpar\textunderscore .
\section{Engasgo}
\begin{itemize}
\item {Grp. gram.:m.}
\end{itemize}
\begin{itemize}
\item {Utilização:Ext.}
\end{itemize}
\begin{itemize}
\item {Proveniência:(T. onom.?)}
\end{itemize}
Acto de engasgar.
Aquillo que impede a fala; atrapalhação.
\section{Engasgue}
\begin{itemize}
\item {Grp. gram.:m.}
\end{itemize}
Acto ou effeito de engasgar.
\section{Engastador}
\begin{itemize}
\item {Grp. gram.:adj.}
\end{itemize}
\begin{itemize}
\item {Grp. gram.:M.}
\end{itemize}
Que engasta.
Aquelle que engasta.
\section{Engastalhar}
\begin{itemize}
\item {Grp. gram.:v. t.}
\end{itemize}
Apertar com gastalho; travar.
\section{Engastar}
\begin{itemize}
\item {Grp. gram.:v. t.}
\end{itemize}
\begin{itemize}
\item {Proveniência:(T. cast.)}
\end{itemize}
Embutir (pedras finas em metal).
Encaixar.
Encastoar; marchetar.
\section{Engaste}
\begin{itemize}
\item {Grp. gram.:m.}
\end{itemize}
Acto ou effeito de engastar.
Aquillo que se engastou: \textunderscore caiu-lhe o engaste\textunderscore .
\section{Engastoar}
\begin{itemize}
\item {Grp. gram.:v. i.}
\end{itemize}
O mesmo que \textunderscore engastar\textunderscore :«\textunderscore merecia tal hora engastoada em ouro\textunderscore ». F. Manuel, \textunderscore Apólogos\textunderscore .
\section{Engatadeira}
\begin{itemize}
\item {Grp. gram.:f.}
\end{itemize}
\begin{itemize}
\item {Utilização:T. do Pôrto}
\end{itemize}
\begin{itemize}
\item {Utilização:Bot.}
\end{itemize}
\begin{itemize}
\item {Proveniência:(De \textunderscore engatar\textunderscore )}
\end{itemize}
Mulher, que allicia outras para a vida das congregações religiosas.
O mesmo que \textunderscore lúpulo\textunderscore .
\section{Engatador}
\begin{itemize}
\item {Grp. gram.:m.}
\end{itemize}
Aquelle que engata.
\section{Engatanhar-se}
\begin{itemize}
\item {Grp. gram.:v. p.}
\end{itemize}
\begin{itemize}
\item {Utilização:Prov.}
\end{itemize}
O mesmo que \textunderscore engadanhar-se\textunderscore .
\section{Engatar}
\begin{itemize}
\item {Grp. gram.:v. t.}
\end{itemize}
\begin{itemize}
\item {Grp. gram.:V. i.}
\end{itemize}
\begin{itemize}
\item {Utilização:Prov.}
\end{itemize}
\begin{itemize}
\item {Utilização:trasm.}
\end{itemize}
\begin{itemize}
\item {Proveniência:(De \textunderscore gato\textunderscore )}
\end{itemize}
Ligar com gatos metállicos.
Atrelar; ligar (carruagens de caminhos de ferro), para formar combóio.
O mesmo que \textunderscore engarrar\textunderscore  ou \textunderscore trepar\textunderscore .
\section{Engate}
\begin{itemize}
\item {Grp. gram.:m.}
\end{itemize}
Acto de engatar.
Apparelho, com que se engatam os cavallos aos vehículos, ou carros a outros carros.
\section{Engatilhar}
\begin{itemize}
\item {Grp. gram.:v. t.}
\end{itemize}
\begin{itemize}
\item {Utilização:Fig.}
\end{itemize}
Armar o gatilho de.
Dispor para disparar (uma arma de fogo).
Preparar: \textunderscore engatilhar uma allocução\textunderscore .
\section{Engatinhar}
\begin{itemize}
\item {Grp. gram.:v. i.}
\end{itemize}
\begin{itemize}
\item {Utilização:Fig.}
\end{itemize}
\begin{itemize}
\item {Proveniência:(De \textunderscore gatinhas\textunderscore )}
\end{itemize}
Andar de gatinhas: \textunderscore o pequenito já engatinha\textunderscore .
Iniciar-se em alguma coisa.
\section{Engavelar}
\begin{itemize}
\item {Grp. gram.:v. t.}
\end{itemize}
Juntar ou dispor em gavelas.
Formar gavelas de.
Enfeixar.
Sobraçar.
\section{Engavetar}
\begin{itemize}
\item {Grp. gram.:v. t.}
\end{itemize}
Meter ou guardar em gaveta.
\section{Engazofilar}
\begin{itemize}
\item {Grp. gram.:v. t.}
\end{itemize}
\begin{itemize}
\item {Utilização:Pop.}
\end{itemize}
Prender, agarrar, gazofilar. Cf. Camillo, \textunderscore Myst. de Lisb.\textunderscore , II, 10.
\section{Engazupar}
\begin{itemize}
\item {Grp. gram.:v. t.}
\end{itemize}
\begin{itemize}
\item {Utilização:Chul.}
\end{itemize}
\begin{itemize}
\item {Utilização:Bras}
\end{itemize}
Embaçar, lograr.
Meter em prisão.
\section{Engegado}
\begin{itemize}
\item {fónica:gé}
\end{itemize}
\begin{itemize}
\item {Grp. gram.:adj.}
\end{itemize}
\begin{itemize}
\item {Utilização:Prov.}
\end{itemize}
Muito ordinário.
Que não presta para nada.
Reles.
Adoentado.
\section{Engeira}
\begin{itemize}
\item {Grp. gram.:f.}
\end{itemize}
\begin{itemize}
\item {Utilização:Ant.}
\end{itemize}
\begin{itemize}
\item {Proveniência:(De \textunderscore geira\textunderscore )}
\end{itemize}
Serviço obrigatório de foreiros.
\section{Engêlha}
\begin{itemize}
\item {Grp. gram.:f.}
\end{itemize}
\begin{itemize}
\item {Utilização:Prov.}
\end{itemize}
O mesmo que \textunderscore gelha\textunderscore .
\section{Engelhado}
\begin{itemize}
\item {Grp. gram.:adj.}
\end{itemize}
Que tem gêlhas.
Enrugado: \textunderscore faces engelhadas\textunderscore .
Murcho: \textunderscore rosas engelhadas\textunderscore .
\section{Engelhar}
\begin{itemize}
\item {Grp. gram.:v. t.}
\end{itemize}
\begin{itemize}
\item {Grp. gram.:V. i.}
\end{itemize}
\begin{itemize}
\item {Proveniência:(De \textunderscore gêlha\textunderscore )}
\end{itemize}
Produzir gêlhas em.
Enrugar; encarquilhar.
Contrahir; murchar: \textunderscore o calor engelha as fôlhas\textunderscore .
Tornar-se rugoso.
Murchar.
\section{Engelim}
\begin{itemize}
\item {Grp. gram.:m.}
\end{itemize}
\begin{itemize}
\item {Utilização:Ant.}
\end{itemize}
O mesmo que \textunderscore angelim\textunderscore .
\section{Engembrado}
\begin{itemize}
\item {Grp. gram.:adj.}
\end{itemize}
\begin{itemize}
\item {Utilização:Ant.}
\end{itemize}
Muito semelhante; idêntico.
(Relaciona-se com \textunderscore sembrado\textunderscore ?)
\section{Engendrar}
\begin{itemize}
\item {Grp. gram.:v. t.}
\end{itemize}
\begin{itemize}
\item {Proveniência:(Do lat. \textunderscore ingenerare\textunderscore )}
\end{itemize}
Produzir; inventar: \textunderscore engendrar calúmnias\textunderscore .
\section{Engenhador}
\begin{itemize}
\item {Grp. gram.:m.  e  adj.}
\end{itemize}
O que engenha.
\section{Engenhar}
\begin{itemize}
\item {Grp. gram.:v. t.}
\end{itemize}
Produzir ou realizar com engenho.
Inventar: \textunderscore engenhar um maquinismo\textunderscore .
Motivar.
Produzir.
Traçar, planear: \textunderscore engenhar uma conspiração\textunderscore .
(B. lat. \textunderscore ingeniare\textunderscore )
\section{Engenharia}
\begin{itemize}
\item {Grp. gram.:f.}
\end{itemize}
Sciência ou arte de construcção.
Classe dos engenheiros.
\section{Engenheira}
\begin{itemize}
\item {Grp. gram.:f.}
\end{itemize}
\begin{itemize}
\item {Utilização:Açor}
\end{itemize}
O mesmo que \textunderscore azinheira\textunderscore .
\section{Engenheiro}
\begin{itemize}
\item {Grp. gram.:m.}
\end{itemize}
\begin{itemize}
\item {Utilização:Bras}
\end{itemize}
\begin{itemize}
\item {Proveniência:(De \textunderscore engenho\textunderscore )}
\end{itemize}
Aquelle que traça ou dirige trabalhos públicos, como os de construcções, abertura de estradas, exploração de minas, etc.
Aquelle que tem diploma do curso de engenharia.
Proprietário de engenho de açúcar, senhor de engenho.
\section{Engenhim}
\begin{itemize}
\item {Grp. gram.:m.}
\end{itemize}
Peixe de Portugal.
\section{Engenho}
\begin{itemize}
\item {Grp. gram.:m.}
\end{itemize}
\begin{itemize}
\item {Utilização:Prov.}
\end{itemize}
\begin{itemize}
\item {Utilização:Ext.}
\end{itemize}
\begin{itemize}
\item {Utilização:Bras}
\end{itemize}
\begin{itemize}
\item {Utilização:Bras. de Minas}
\end{itemize}
\begin{itemize}
\item {Proveniência:(Do lat. \textunderscore ingenium\textunderscore )}
\end{itemize}
Faculdade especial; talento natural; habilidade:«\textunderscore se a tanto me ajudar o engenho\textunderscore ». \textunderscore Lusíadas\textunderscore , I, 2.
Invenção.
Máquina; maquinismo.
Estratagema.
Apparelho, o mesmo que \textunderscore nora\textunderscore .
Pessôa engenhosa, de grande talento.
Estabelecimento agrícola, destinado á cultura da cana saccharina e á fabricação de açúcar.
Espécie de dança.
\section{Engenhoca}
\begin{itemize}
\item {Grp. gram.:f.}
\end{itemize}
\begin{itemize}
\item {Utilização:deprec.}
\end{itemize}
\begin{itemize}
\item {Utilização:Pop.}
\end{itemize}
\begin{itemize}
\item {Utilização:Bras}
\end{itemize}
\begin{itemize}
\item {Proveniência:(De \textunderscore engenho\textunderscore )}
\end{itemize}
Máquina; maquinismo.
Ardil.
Armadilha.
Pequeno engenho, destinado especialmente á fabricação de aguardente.
Aguardente de cana.
\section{Engenhosamente}
\begin{itemize}
\item {Grp. gram.:adv.}
\end{itemize}
De modo engenhoso, com engenho, com grande habilidade.
Com ardil.
\section{Engenhoso}
\begin{itemize}
\item {Grp. gram.:adj.}
\end{itemize}
\begin{itemize}
\item {Grp. gram.:M.}
\end{itemize}
\begin{itemize}
\item {Proveniência:(Do lat. \textunderscore ingeniosus\textunderscore )}
\end{itemize}
Que tem engenho.
Que revela engenho: \textunderscore ideia engenhosa\textunderscore .
Artificioso.
Inventivo.
Antiga moéda de oiro portuguesa.
\section{Engenioso}
\begin{itemize}
\item {Grp. gram.:adj.}
\end{itemize}
\begin{itemize}
\item {Utilização:Ant.}
\end{itemize}
O mesmo que \textunderscore engenhoso\textunderscore . Cf. Usque, \textunderscore Tribulações\textunderscore , 46 v.
\section{Engerir-se}
\begin{itemize}
\item {Grp. gram.:v. p.}
\end{itemize}
Encolher-se com frio ou com doença.
\section{Engerizar-se}
\begin{itemize}
\item {Grp. gram.:v. p.}
\end{itemize}
\begin{itemize}
\item {Utilização:Bras}
\end{itemize}
\begin{itemize}
\item {Proveniência:(De \textunderscore geriza\textunderscore )}
\end{itemize}
Zangar-se.
\section{Engessador}
\begin{itemize}
\item {Grp. gram.:adj.}
\end{itemize}
\begin{itemize}
\item {Grp. gram.:M.}
\end{itemize}
Que engessa.
Aquelle que engessa.
\section{Engessadura}
\begin{itemize}
\item {Grp. gram.:f.}
\end{itemize}
Acto ou effeito de engessar.
\section{Engessar}
\begin{itemize}
\item {Grp. gram.:v. t.}
\end{itemize}
Cobrir com gêsso.
Branquear com gêsso.
\section{Engíbata}
\begin{itemize}
\item {Grp. gram.:f.}
\end{itemize}
\begin{itemize}
\item {Grp. gram.:Pl.}
\end{itemize}
\begin{itemize}
\item {Proveniência:(Lat. \textunderscore engibata\textunderscore )}
\end{itemize}
Antiga máquina hydráulica.
Pequenas figuras de vidro, que se moviam dentro de uma vasilha cheia de água.
\section{Engigar}
\begin{itemize}
\item {Grp. gram.:v. t.}
\end{itemize}
\begin{itemize}
\item {Proveniência:(De \textunderscore giga\textunderscore ^1)}
\end{itemize}
Meter ou acommodar na giga.
\section{Englelê}
\begin{itemize}
\item {Grp. gram.:m.}
\end{itemize}
O mesmo que \textunderscore inglelê\textunderscore .
\section{Englobadamente}
\begin{itemize}
\item {Grp. gram.:adv.}
\end{itemize}
\begin{itemize}
\item {Proveniência:(De \textunderscore englobar\textunderscore )}
\end{itemize}
Em glôbo.
\section{Englobar}
\begin{itemize}
\item {Grp. gram.:v. t.}
\end{itemize}
Dar fórma de globo a.
Reunir em um todo, juntar em globo: \textunderscore englobar accusações\textunderscore .
\section{Englobular}
\begin{itemize}
\item {Grp. gram.:v. t.}
\end{itemize}
Converter em glóbulo.
\section{Englutir}
\begin{itemize}
\item {Grp. gram.:v. t.}
\end{itemize}
\begin{itemize}
\item {Utilização:Ant.}
\end{itemize}
O mesmo que \textunderscore engulir\textunderscore . Cf. Usque, \textunderscore Tribulações\textunderscore , 13, etc.
\section{Engo}
\begin{itemize}
\item {Grp. gram.:m.}
\end{itemize}
O mesmo que \textunderscore engos\textunderscore .
\section{...engo}
\begin{itemize}
\item {Grp. gram.:suf.}
\end{itemize}
(designativo de relação ou pertença: \textunderscore mostrengo\textunderscore ; \textunderscore realengo...\textunderscore )
\section{Engobo}
\begin{itemize}
\item {Grp. gram.:m.}
\end{itemize}
(?):«\textunderscore ...productos de cerâmica, ornados pela sobreposição, por engobo, de camadas...\textunderscore »Castilho, \textunderscore Fastos\textunderscore , II, 353.
\section{Engodador}
\begin{itemize}
\item {Grp. gram.:adj.}
\end{itemize}
\begin{itemize}
\item {Grp. gram.:M.}
\end{itemize}
Que engoda.
Aquelle que engoda.
\section{Engodar}
\begin{itemize}
\item {Grp. gram.:v. t.}
\end{itemize}
Atrahir com engôdo.
Enganar ardilosamente, com vans promessas.
\section{Engodativo}
\begin{itemize}
\item {Grp. gram.:adj.}
\end{itemize}
Próprio para engodar.
\section{Engodilhar}
\begin{itemize}
\item {Grp. gram.:v. t.}
\end{itemize}
\begin{itemize}
\item {Utilização:Fig.}
\end{itemize}
\begin{itemize}
\item {Grp. gram.:V. i.}
\end{itemize}
Encher de godilhões.
Embaraçar; atrapalhar.
Criar godilhões.
\section{Engôdo}
\begin{itemize}
\item {Grp. gram.:m.}
\end{itemize}
\begin{itemize}
\item {Utilização:Fig.}
\end{itemize}
\begin{itemize}
\item {Utilização:Agr.}
\end{itemize}
\begin{itemize}
\item {Proveniência:(Do rad. do lat. \textunderscore gaudium\textunderscore ?)}
\end{itemize}
Isca para pesca.
Coisa com que se engoda ou seduz alguém.
Alliciação.
Adulação astuciosa.
A melhor substância, que as águas arrastam de terra cultivada.
\section{Engoiar-se}
\begin{itemize}
\item {Grp. gram.:v. p.}
\end{itemize}
\begin{itemize}
\item {Utilização:Pop.}
\end{itemize}
Tornar-se magro, enfezar-se.
(Por \textunderscore engrouar\textunderscore , de \textunderscore grou\textunderscore ?)
\section{Engoldrão}
\begin{itemize}
\item {Grp. gram.:m.}
\end{itemize}
\begin{itemize}
\item {Utilização:Prov.}
\end{itemize}
(Us. na loc. adv. \textunderscore de engoldrão\textunderscore , soffregamente)
\section{Engole-vento}
\begin{itemize}
\item {Grp. gram.:m.}
\end{itemize}
\begin{itemize}
\item {Utilização:Bras}
\end{itemize}
\begin{itemize}
\item {Proveniência:(Fr. \textunderscore engoule-vent\textunderscore )}
\end{itemize}
O mesmo que \textunderscore noitibó\textunderscore .
\section{Engolfar}
\begin{itemize}
\item {Grp. gram.:v. t.}
\end{itemize}
\begin{itemize}
\item {Utilização:Fig.}
\end{itemize}
\begin{itemize}
\item {Proveniência:(De \textunderscore golfo\textunderscore )}
\end{itemize}
Meter em golfo.
Abysmar, mergulhar em sorvedoiro.
Entranhar.
Tornar absorto, enlevar: \textunderscore engolfado em conjecturas\textunderscore .
\section{Engomadeira}
\begin{itemize}
\item {Grp. gram.:f.}
\end{itemize}
\begin{itemize}
\item {Proveniência:(De \textunderscore engomar\textunderscore )}
\end{itemize}
Mulher que engoma.
\section{Engomadela}
\begin{itemize}
\item {Grp. gram.:f.}
\end{itemize}
O mesmo que \textunderscore engomadura\textunderscore .
\section{Engomado}
\begin{itemize}
\item {Grp. gram.:m.}
\end{itemize}
\begin{itemize}
\item {Grp. gram.:Adj.}
\end{itemize}
\begin{itemize}
\item {Proveniência:(De \textunderscore engomar\textunderscore )}
\end{itemize}
Roupa engomada.
Serviço de engomar: \textunderscore hoje é dia de engomados, cá em casa\textunderscore .
O mesmo que gomoso, (falando-se de vinho).
\section{Engomador}
\begin{itemize}
\item {Grp. gram.:m.}
\end{itemize}
\begin{itemize}
\item {Utilização:Des.}
\end{itemize}
Aquelle que exercia a profissão de engomar roupa. (Colhido num testamento do séc. XVII)
\section{Engomadura}
\begin{itemize}
\item {Grp. gram.:f.}
\end{itemize}
Acto ou effeito de engomar.
\section{Engomagem}
\begin{itemize}
\item {Grp. gram.:f.}
\end{itemize}
Acto de engomar.
Collagem dos vinhos.
\section{Engomar}
\begin{itemize}
\item {Grp. gram.:v. t.}
\end{itemize}
\begin{itemize}
\item {Utilização:Fig.}
\end{itemize}
\begin{itemize}
\item {Utilização:Gír. lisb.}
\end{itemize}
Pôr em goma e alisar depois com ferro quente: \textunderscore engomar camisas\textunderscore .
Untar com goma.
Tornar grosso: \textunderscore engomar o vinho\textunderscore .
Ensoberbecer.
Fechar, cerrar, (porta, gaveta).
\section{Engommar}
\textunderscore v. t.\textunderscore  (e der.)
O mesmo que \textunderscore engomar\textunderscore , etc.
\section{Engonatão}
\begin{itemize}
\item {Grp. gram.:m.}
\end{itemize}
\begin{itemize}
\item {Proveniência:(Lat. \textunderscore engonaton\textunderscore )}
\end{itemize}
Espécie de antigo relógio solar.
\section{Engonçar}
\begin{itemize}
\item {Grp. gram.:v. t.}
\end{itemize}
Segurar com engonços.
Pôr engonços em.
(Por \textunderscore engonzar\textunderscore , de \textunderscore gonzo\textunderscore )
\section{Engonço}
\begin{itemize}
\item {Grp. gram.:m.}
\end{itemize}
\begin{itemize}
\item {Proveniência:(De \textunderscore engonçar\textunderscore )}
\end{itemize}
O mesmo que \textunderscore gonzo\textunderscore .
Espécie de dobradiça.
Encaixe de duas ou mais peças de uma máquina ou de um artefacto, que lhe permittem certo movimento.
\section{Engonha}
\begin{itemize}
\item {Grp. gram.:f.}
\end{itemize}
\begin{itemize}
\item {Utilização:Prov.}
\end{itemize}
\begin{itemize}
\item {Utilização:alent.}
\end{itemize}
\begin{itemize}
\item {Grp. gram.:M.}
\end{itemize}
Preguiça no trabalho.
Homem preguiçoso.
\section{Engonhar}
\begin{itemize}
\item {Grp. gram.:v. i.}
\end{itemize}
\begin{itemize}
\item {Utilização:Prov.}
\end{itemize}
\begin{itemize}
\item {Utilização:alent.}
\end{itemize}
\begin{itemize}
\item {Proveniência:(De \textunderscore engonha\textunderscore )}
\end{itemize}
Trabalhar de ma vontade.
\section{Engorda}
\begin{itemize}
\item {Grp. gram.:f.}
\end{itemize}
Acto ou effeito de engordar.
\section{Engordamento}
\begin{itemize}
\item {Grp. gram.:m.}
\end{itemize}
Acto ou effeito de engordar.
\section{Engordar}
\begin{itemize}
\item {Grp. gram.:v. t.}
\end{itemize}
\begin{itemize}
\item {Grp. gram.:V. i.}
\end{itemize}
Tornar gordo.
Dar gordura a.
Tornar-se gordo.
Alimentar-se.
Crescer.
*\textunderscore Agr.\textunderscore 
Adoecer de gordura (o vinho).
\section{Engôrdo}
\begin{itemize}
\item {Grp. gram.:m.}
\end{itemize}
\begin{itemize}
\item {Proveniência:(De \textunderscore engordar\textunderscore )}
\end{itemize}
Planta brasileira, que serve de alimento a bêstas.
\section{Engorduramento}
\begin{itemize}
\item {Grp. gram.:m.}
\end{itemize}
\begin{itemize}
\item {Proveniência:(De \textunderscore engordurar\textunderscore )}
\end{itemize}
Doença dos vinhos.
\section{Engordurar}
\begin{itemize}
\item {Grp. gram.:v. t.}
\end{itemize}
Sujar com gordura; besuntar.
\section{Engorgido}
\begin{itemize}
\item {Grp. gram.:adj.}
\end{itemize}
\begin{itemize}
\item {Utilização:Prov.}
\end{itemize}
\begin{itemize}
\item {Utilização:trasm.}
\end{itemize}
Encolhido com frio; entanguido.
(Cp. \textunderscore engrunhido\textunderscore )
\section{Engorgitar}
\begin{itemize}
\item {Grp. gram.:v. t.}
\end{itemize}
\begin{itemize}
\item {Grp. gram.:V. i.  e  p.}
\end{itemize}
\begin{itemize}
\item {Proveniência:(Lat. \textunderscore ingurgitare\textunderscore )}
\end{itemize}
\textunderscore v. t.\textunderscore  (e der.)
O mesmo ou melhor que \textunderscore ingurgitar\textunderscore , etc.
Devorar, engulir soffregamente.
Encher muito; obstruír.
Encher-se.
Soffrer obstrucção de um vaso ou ducto excretor.
Entumecer.
Encher-se muito de alimento.
Chafurdar ou atolar-se (em paixões ou vícios).
\section{Engorlar}
\textunderscore v. t.\textunderscore  (e der.)
O mesmo que \textunderscore engrolar\textunderscore , etc.
\section{Engorolar}
\begin{itemize}
\item {Grp. gram.:v. t.}
\end{itemize}
Dizer atabalhoadamente:«\textunderscore engorolava o latim das missas\textunderscore ». Camillo, \textunderscore Corja\textunderscore , 10.
(Por \textunderscore engrolar\textunderscore )
\section{Engorrar-se}
\begin{itemize}
\item {Grp. gram.:v. p.}
\end{itemize}
\begin{itemize}
\item {Proveniência:(De \textunderscore gorra\textunderscore )}
\end{itemize}
Abandear-se.
Meter-se de gorra: \textunderscore engorrar-se com fadistas\textunderscore .
\section{Engorvinhado}
\begin{itemize}
\item {Grp. gram.:adj.}
\end{itemize}
\begin{itemize}
\item {Utilização:Des.}
\end{itemize}
Cheio de dobras; amarrotado.
(Cp. \textunderscore engrouvinhado\textunderscore )
\section{Engos}
\begin{itemize}
\item {Grp. gram.:m. pl.}
\end{itemize}
Planta, semelhante ao sabugueiro, mas muito mais pequena, (\textunderscore sambucus ebulus\textunderscore ).
\section{Engoxa}
\begin{itemize}
\item {fónica:gô}
\end{itemize}
\begin{itemize}
\item {Grp. gram.:f.}
\end{itemize}
O mesmo que \textunderscore pêra-de-engonxo\textunderscore .
\section{Engra}
\begin{itemize}
\item {Grp. gram.:f.}
\end{itemize}
\begin{itemize}
\item {Utilização:Des.}
\end{itemize}
Canto, formado por duas paredes.
Curvatura.
(Corr. de \textunderscore ângulo\textunderscore )
\section{Engraçadamente}
\begin{itemize}
\item {Grp. gram.:adv.}
\end{itemize}
Com graça, de modo engraçado.
\section{Engraçado}
\begin{itemize}
\item {Grp. gram.:adj.}
\end{itemize}
Que tem graça; espirituoso.
\section{Engraçar}
\begin{itemize}
\item {Grp. gram.:v. t.}
\end{itemize}
\begin{itemize}
\item {Grp. gram.:V. i.}
\end{itemize}
Dar graça a.
Tornar gracioso.
Fazer jovial, espirituoso.
Sympathizar: \textunderscore engracei com êlle\textunderscore .
\section{Engradamento}
\begin{itemize}
\item {Grp. gram.:m.}
\end{itemize}
Acto ou effeito de engradar.
Obra engradada.
\section{Engradar}
\begin{itemize}
\item {Grp. gram.:v. t.}
\end{itemize}
Dar fórma de grade a.
Cercar de grades.
Gradear: \textunderscore engradar um jardim\textunderscore .
Pregar na grade (uma tela, em que se há de pintar).
Reunir por meio de tabuões ou falcas as peças de (reparo\textunderscore ou\textunderscore  carrêta de artilharia).
\section{Engradear}
\begin{itemize}
\item {Grp. gram.:v. t.}
\end{itemize}
O mesmo que \textunderscore engradar\textunderscore .
\section{Engrandecer}
\begin{itemize}
\item {Grp. gram.:v. i.}
\end{itemize}
Tornar-se grado.
\section{Engraecer}
\begin{itemize}
\item {Grp. gram.:v. i.}
\end{itemize}
\begin{itemize}
\item {Proveniência:(De \textunderscore grão\textunderscore )}
\end{itemize}
Formar grão ou semente.
\section{Engraixar}
\textunderscore v. t.\textunderscore  (e der.)
(V. \textunderscore engraxar\textunderscore , etc.)
\section{Engrambelar}
\textunderscore v. t.\textunderscore  (e der.)
O mesmo que \textunderscore engabelar\textunderscore .
\section{Engrampador}
\begin{itemize}
\item {Grp. gram.:m.}
\end{itemize}
Aquelle que engrampa.
\section{Engrampar}
\begin{itemize}
\item {Grp. gram.:v. t.}
\end{itemize}
\begin{itemize}
\item {Proveniência:(De \textunderscore grampo\textunderscore )}
\end{itemize}
Lograr; embaçar.
Attrahir com embustes.
\section{Engramponar-se}
\begin{itemize}
\item {Grp. gram.:v. p.}
\end{itemize}
(Corr. de \textunderscore engrimponar-se\textunderscore )
\section{Engrandecer}
\begin{itemize}
\item {Grp. gram.:v. t.}
\end{itemize}
\begin{itemize}
\item {Grp. gram.:V. i.}
\end{itemize}
\begin{itemize}
\item {Proveniência:(Do lat. \textunderscore grandescere\textunderscore )}
\end{itemize}
Tornar grande.
Aumentar.
Tornar elevado.
Illustrar: \textunderscore o talento engrandece o homem\textunderscore .
Tornar-se maior; elevar-se.
Crescer em honras.
\section{Engrandecimento}
\begin{itemize}
\item {Grp. gram.:m.}
\end{itemize}
Acto ou effeito de engrandecer.
\section{Engranzador}
\begin{itemize}
\item {Grp. gram.:adj.}
\end{itemize}
\begin{itemize}
\item {Grp. gram.:M.}
\end{itemize}
Que engranza.
Aquelle que engranza.
\section{Engranzagem}
\begin{itemize}
\item {Grp. gram.:f.}
\end{itemize}
O mesmo que \textunderscore engranzamento\textunderscore .
\section{Engranzamento}
\begin{itemize}
\item {Grp. gram.:m.}
\end{itemize}
Acto ou effeito de engranzar.
\section{Engranzar}
\begin{itemize}
\item {Grp. gram.:v. t.}
\end{itemize}
\begin{itemize}
\item {Utilização:fam.}
\end{itemize}
\begin{itemize}
\item {Utilização:Fig.}
\end{itemize}
\begin{itemize}
\item {Proveniência:(De \textunderscore grão\textunderscore )}
\end{itemize}
Enfiar.
Enganchar.
Encadear.
Endentar.
Engrenar.
Engazupar; embair.
\section{Engrassar}
\begin{itemize}
\item {Grp. gram.:v. t.}
\end{itemize}
(V.incrassar)
\section{Engravatado}
\begin{itemize}
\item {Grp. gram.:adj.}
\end{itemize}
\begin{itemize}
\item {Utilização:Fig.}
\end{itemize}
Que tem gravata.
Enfeitado, garrido:«\textunderscore Landim, uma aldeia engravatada...\textunderscore ». Camillo, \textunderscore Brasileira\textunderscore , 315.
\section{Engravatar-se}
\begin{itemize}
\item {Grp. gram.:v. p.}
\end{itemize}
\begin{itemize}
\item {Utilização:Fig.}
\end{itemize}
Pôr gravata, enfeitar-se com gravata.
Mostrar-se garrido.
\section{Engravatizar-se}
\begin{itemize}
\item {Grp. gram.:v. p.}
\end{itemize}
O mesmo que \textunderscore engravatar-se\textunderscore :«\textunderscore por isso se encasacam, se encartolam, se engravatizam\textunderscore ». Camillo, \textunderscore Serões de Scide\textunderscore , V, 66.
\section{Engravecer}
\begin{itemize}
\item {Grp. gram.:v. i.}
\end{itemize}
\begin{itemize}
\item {Proveniência:(Do lat. \textunderscore gravescere\textunderscore )}
\end{itemize}
Tornar-se grave.
Aggravar-se: \textunderscore a doença engraveceu\textunderscore .
\section{Engravescer}
\begin{itemize}
\item {Grp. gram.:v. i.}
\end{itemize}
O mesmo que engravecer.
\section{Engravitar-se}
\begin{itemize}
\item {Grp. gram.:v. p.}
\end{itemize}
Endireitar-se, impertigar-se.
Respingar; reagir.
(Cast. \textunderscore engarabitar\textunderscore )
\section{Engraxadela}
\begin{itemize}
\item {Grp. gram.:f.}
\end{itemize}
O mesmo que \textunderscore engraxamento\textunderscore .
\section{Engraxador}
\begin{itemize}
\item {Grp. gram.:m.}
\end{itemize}
Aquelle que engraxa.
\section{Engraxamento}
\begin{itemize}
\item {Grp. gram.:m.}
\end{itemize}
Acto ou effeito de engraxar.
\section{Engraxar}
\begin{itemize}
\item {Grp. gram.:v. t.}
\end{itemize}
\begin{itemize}
\item {Utilização:Fig.}
\end{itemize}
\begin{itemize}
\item {Grp. gram.:Loc.}
\end{itemize}
\begin{itemize}
\item {Utilização:fig.}
\end{itemize}
Dar graxa a, e lustrar seguidamente: \textunderscore engraxar botas\textunderscore .
Tornar preto, mascarrar.
\textunderscore Engraxar as botas a alguém\textunderscore , lisonjeá-lo, gabá-lo servilmente.
\section{Engraxate}
\begin{itemize}
\item {Grp. gram.:m.}
\end{itemize}
\begin{itemize}
\item {Utilização:Bras. da Baía}
\end{itemize}
Engraxador de calçado.
\section{Engrazar}
\begin{itemize}
\item {Grp. gram.:v. t.}
\end{itemize}
O mesmo que \textunderscore engranzar\textunderscore .
\section{Engrazular}
\begin{itemize}
\item {Grp. gram.:v. t.}
\end{itemize}
\begin{itemize}
\item {Utilização:Prov.}
\end{itemize}
\begin{itemize}
\item {Utilização:alg.}
\end{itemize}
Enganar, burlar.
\section{Engrecer}
\begin{itemize}
\item {Grp. gram.:v. i.}
\end{itemize}
\begin{itemize}
\item {Utilização:Ant.}
\end{itemize}
O mesmo que \textunderscore engraecer\textunderscore .
\section{Engrelar}
\begin{itemize}
\item {Grp. gram.:v. i.  e  p.}
\end{itemize}
\begin{itemize}
\item {Utilização:Fig.}
\end{itemize}
\begin{itemize}
\item {Proveniência:(De \textunderscore grêlo\textunderscore )}
\end{itemize}
Erguer-se viçoso, deitando grêlo, (falando-se de plantas que estavam murchas).
Pôr-se em pé.
Começar a andar.
\section{Engrenagem}
\begin{itemize}
\item {Grp. gram.:f.}
\end{itemize}
Acto ou effeito de engrenar.
\section{Engrenar}
\begin{itemize}
\item {Grp. gram.:v. t.}
\end{itemize}
\begin{itemize}
\item {Utilização:Gal}
\end{itemize}
\begin{itemize}
\item {Proveniência:(Fr. \textunderscore engrener\textunderscore )}
\end{itemize}
Engranzar.
Entrosar.
Endentar.
Relacionar e tornar reciprocamente dependentes os actos ou circunstâncias de (pessôas ou coisas distintas).
\section{Engrenhamento}
\begin{itemize}
\item {Grp. gram.:m.}
\end{itemize}
Acto ou effeito de engrenhar.
\section{Engrenhar}
\begin{itemize}
\item {Grp. gram.:v. t.}
\end{itemize}
\begin{itemize}
\item {Utilização:Des.}
\end{itemize}
\begin{itemize}
\item {Proveniência:(De \textunderscore grenha\textunderscore )}
\end{itemize}
Concertar ou atar (o cabello).
\section{Engrês}
\begin{itemize}
\item {Grp. gram.:m.}
\end{itemize}
\begin{itemize}
\item {Utilização:Ant.}
\end{itemize}
Variedade de pano. Cf. Herculano, \textunderscore Cister\textunderscore , 184.(V.ingrês)
\section{Engrideira}
\begin{itemize}
\item {Grp. gram.:f.}
\end{itemize}
\begin{itemize}
\item {Utilização:Prov.}
\end{itemize}
\begin{itemize}
\item {Utilização:trasm.}
\end{itemize}
Corda de sobrecarga.
\section{Engrifamento}
\begin{itemize}
\item {Grp. gram.:m.}
\end{itemize}
Acto de engrifar-se.
\section{Engrifar-se}
\begin{itemize}
\item {Grp. gram.:v. p.}
\end{itemize}
\begin{itemize}
\item {Utilização:Des.}
\end{itemize}
Dispor as grifas para a luta.
Preparar-se para brigar.
\section{Engrilar}
\begin{itemize}
\item {Grp. gram.:v. t.}
\end{itemize}
\begin{itemize}
\item {Utilização:Pop.}
\end{itemize}
\begin{itemize}
\item {Grp. gram.:V. i.}
\end{itemize}
\begin{itemize}
\item {Utilização:Prov.}
\end{itemize}
\begin{itemize}
\item {Utilização:dur.}
\end{itemize}
\begin{itemize}
\item {Grp. gram.:V. p.}
\end{itemize}
Engrelar.
Olhar com attenção.
Enfiar ou introduzir em buraco ou baínha.
Agastar-se.
Arrebitar-se.
Endireitar-se.
(Por \textunderscore engrelar\textunderscore )
\section{Engrimanço}
\begin{itemize}
\item {Grp. gram.:m.}
\end{itemize}
\begin{itemize}
\item {Proveniência:(Do rad. do it. \textunderscore grimo\textunderscore ?)}
\end{itemize}
Confusão no falar.
Extravagância de figuras oratórias.
Artimanha.
\section{Engrimpar-se}
\begin{itemize}
\item {Grp. gram.:v. p.}
\end{itemize}
\begin{itemize}
\item {Proveniência:(De \textunderscore grimpa\textunderscore )}
\end{itemize}
Encarrapitar-se; collocar-se na grimpa.
Elevar-se muito.
\section{Engrimpinar-se}
\begin{itemize}
\item {Grp. gram.:v. p.}
\end{itemize}
O mesmo que \textunderscore engrimponar-se\textunderscore .
\section{Engrimponar-se}
\begin{itemize}
\item {Grp. gram.:v. p.}
\end{itemize}
(V.engrimpar-se)
\section{Engrinaldar}
\begin{itemize}
\item {Grp. gram.:v. t.}
\end{itemize}
Enfeitar com grinaldas.
Coroar.
Enfeitar.
Tornar formoso.
Premiar, galardoar.
\section{Engrolador}
\begin{itemize}
\item {Grp. gram.:adj.}
\end{itemize}
\begin{itemize}
\item {Grp. gram.:M.}
\end{itemize}
Que engrola.
Aquelle que engrola.
\section{Engrolar}
\begin{itemize}
\item {Grp. gram.:v. t.}
\end{itemize}
\begin{itemize}
\item {Utilização:Fig.}
\end{itemize}
\begin{itemize}
\item {Proveniência:(Do lat. \textunderscore incrudare\textunderscore ?)}
\end{itemize}
Cozer ligeiramente.
Assar de leve.
Executar mal.
Pronunciar atrapalhadamente: \textunderscore engrolar padrenossos\textunderscore .
Não completar: \textunderscore engrolar as palavras\textunderscore .
Embair; enganar; ludibriar.
\section{Engrôlo}
\begin{itemize}
\item {Grp. gram.:m.}
\end{itemize}
\begin{itemize}
\item {Utilização:T. do Fundão}
\end{itemize}
\begin{itemize}
\item {Proveniência:(De \textunderscore engrolar\textunderscore ?)}
\end{itemize}
Criancinha de mama, de tal fórma enroupada, que parece um embrulho ou troixa.
\section{Engronhar-se}
\begin{itemize}
\item {Grp. gram.:v. p.}
\end{itemize}
\begin{itemize}
\item {Utilização:Prov.}
\end{itemize}
\begin{itemize}
\item {Utilização:trasm.}
\end{itemize}
O mesmo que [[humilhar-se|humilhar]].
\section{Engrossador}
\begin{itemize}
\item {Grp. gram.:m.}
\end{itemize}
\begin{itemize}
\item {Utilização:Bras}
\end{itemize}
\begin{itemize}
\item {Utilização:fig.}
\end{itemize}
Aquelle que engrossa.
Aquelle que, num grupo de indivíduos, faz côro com quem diffama outrem.
\section{Enarmonia}
\begin{itemize}
\item {Grp. gram.:f.}
\end{itemize}
\begin{itemize}
\item {Proveniência:(Do lat. \textunderscore enharmonius\textunderscore )}
\end{itemize}
Gênero de modulação, em que não há mudança sensível de entonação nas notas, mas sim mudança de nome.
\section{Enarmónico}
\begin{itemize}
\item {Grp. gram.:adj.}
\end{itemize}
Relativo a enarmonia.
\section{Engrossamento}
\begin{itemize}
\item {Grp. gram.:m.}
\end{itemize}
Acto de engrossar.
\section{Engrossar}
\begin{itemize}
\item {Grp. gram.:v. t.}
\end{itemize}
\begin{itemize}
\item {Utilização:Fig.}
\end{itemize}
\begin{itemize}
\item {Utilização:Bras}
\end{itemize}
\begin{itemize}
\item {Grp. gram.:V. i.}
\end{itemize}
Tornar grosso.
Aumentar a grossura de.
Encorpar.
Tornar mais numeroso.
Fortalecer, adubar: \textunderscore engrossar o caldo\textunderscore .
Adular.
Tornar-se grosso, espêsso, forte.
Tornar-se mais numeroso: \textunderscore a multidão engrossava\textunderscore .
Aumentar, crescer.
Tornar-se volumoso, tornar-se mais grave o timbre de (uma voz).
Prosperar.
\section{Engrossentar}
\begin{itemize}
\item {Grp. gram.:v. t.}
\end{itemize}
\begin{itemize}
\item {Utilização:Des.}
\end{itemize}
\begin{itemize}
\item {Proveniência:(De \textunderscore grosso\textunderscore )}
\end{itemize}
Tornar abundante.
Opulentar.
Tornar fértil.
\section{Engrotar}
\begin{itemize}
\item {Grp. gram.:v. i.}
\end{itemize}
\begin{itemize}
\item {Utilização:Des.}
\end{itemize}
\begin{itemize}
\item {Proveniência:(Do fr. \textunderscore encrouter\textunderscore ?)}
\end{itemize}
Obstruir-se o orifício da ampulheta.
\section{Engrouvinhado}
\begin{itemize}
\item {Grp. gram.:adj.}
\end{itemize}
O mesmo que \textunderscore esgrouvinhado\textunderscore .
\section{Engrumar}
\begin{itemize}
\item {Grp. gram.:v. t., i.  e  p.}
\end{itemize}
O mesmo que \textunderscore grumar\textunderscore .
\section{Engrumecer}
\begin{itemize}
\item {Grp. gram.:v. i.}
\end{itemize}
O mesmo que \textunderscore grumar\textunderscore .
\section{Engrunhar}
\begin{itemize}
\item {Grp. gram.:v. t.}
\end{itemize}
\begin{itemize}
\item {Utilização:Prov.}
\end{itemize}
\begin{itemize}
\item {Utilização:minh.}
\end{itemize}
O mesmo que \textunderscore engrunhir\textunderscore .
\section{Engrunhido}
\begin{itemize}
\item {Grp. gram.:adj.}
\end{itemize}
Preguiçoso.
Entorpecido.
Engerido.
\section{Engrunhir}
\begin{itemize}
\item {Grp. gram.:v. t.}
\end{itemize}
Entorpecer; tornar hirto ou encolhido com frio ou doença.
\section{Enguedelhar}
\begin{itemize}
\item {Grp. gram.:v. i.}
\end{itemize}
\begin{itemize}
\item {Utilização:Prov.}
\end{itemize}
\begin{itemize}
\item {Proveniência:(De \textunderscore guedelha\textunderscore )}
\end{itemize}
Brigar, arrepelando.
\section{Enguenitado}
\begin{itemize}
\item {Grp. gram.:adj.}
\end{itemize}
\begin{itemize}
\item {Utilização:Prov.}
\end{itemize}
\begin{itemize}
\item {Utilização:alent.}
\end{itemize}
Oblíquo, que vem de esguelha.
Enviesado.
\section{Enguia}
\begin{itemize}
\item {Grp. gram.:f.}
\end{itemize}
\begin{itemize}
\item {Proveniência:(Do lat. \textunderscore anguilla\textunderscore )}
\end{itemize}
Peixe de água doce, comprido e roliço, do gênero das mureias.
\section{Enguia-eléctrica}
\begin{itemize}
\item {Grp. gram.:f.}
\end{itemize}
\begin{itemize}
\item {Utilização:Bras}
\end{itemize}
Peixe, o mesmo que \textunderscore puraquê\textunderscore .
\section{Enguia-solta}
\begin{itemize}
\item {Grp. gram.:f.}
\end{itemize}
\begin{itemize}
\item {Utilização:Gír.}
\end{itemize}
O grilheta, que já cumpriu sentença.
\section{Enguiçador}
\begin{itemize}
\item {Grp. gram.:adj.}
\end{itemize}
\begin{itemize}
\item {Grp. gram.:M.}
\end{itemize}
Que enguiça.
Aquelle que enguiça.
\section{Enguiçar}
\begin{itemize}
\item {Grp. gram.:v. t.}
\end{itemize}
\begin{itemize}
\item {Utilização:Fig.}
\end{itemize}
\begin{itemize}
\item {Utilização:Prov.}
\end{itemize}
\begin{itemize}
\item {Utilização:minh.}
\end{itemize}
Causar enguiço a.
Enfezar.
Dar mau olhado a.
Fazer mal a.
Passar ou saltar por cima de: \textunderscore o garoto enguiçou a parede\textunderscore , (saltou por cima della).
\section{Enguiço}
\begin{itemize}
\item {Grp. gram.:m.}
\end{itemize}
\begin{itemize}
\item {Proveniência:(Do lat. hyp. \textunderscore iniquitium\textunderscore , seg. Cornu)}
\end{itemize}
Quebranto.
Presentimentos tristes.
Mau agoiro.
Empecilho.
\section{Enguinação}
\begin{itemize}
\item {Grp. gram.:f.}
\end{itemize}
\begin{itemize}
\item {Utilização:Pop.}
\end{itemize}
Gana; grande tentação: \textunderscore dão-me enguinações de o estrangular\textunderscore .
\section{Enguinâni}
\begin{itemize}
\item {Grp. gram.:m.}
\end{itemize}
Ave pernalta.
\section{Engulhamento}
\begin{itemize}
\item {Grp. gram.:m.}
\end{itemize}
Acto ou effeito de engulhar.
\section{Engulhar}
\begin{itemize}
\item {Grp. gram.:v. t.}
\end{itemize}
\begin{itemize}
\item {Grp. gram.:V. i.}
\end{itemize}
\begin{itemize}
\item {Proveniência:(De \textunderscore engulho\textunderscore )}
\end{itemize}
Causar náusea ou engulhos a.
Têr náusea.
Sentir nojo.
Têr ânsias, grande desejo.
\section{Engulhento}
\begin{itemize}
\item {Grp. gram.:adj.}
\end{itemize}
O mesmo que \textunderscore engulhoso\textunderscore . Cf. Camillo, \textunderscore Cav. em Ruinas\textunderscore , 133.
\section{Engulho}
\begin{itemize}
\item {Grp. gram.:m.}
\end{itemize}
Acto ou effeito de engulhar.
Pessôa, que causa nojo.
\section{Engulhoso}
\begin{itemize}
\item {Grp. gram.:adj.}
\end{itemize}
Que causa engulho:«\textunderscore o fartum engulhoso da matulagem\textunderscore ». Camillo, \textunderscore Regicida\textunderscore , 11.
\section{Engulição}
\begin{itemize}
\item {Grp. gram.:f.}
\end{itemize}
Acto de engulir.
\section{Engulideiras}
\begin{itemize}
\item {Grp. gram.:f. pl.}
\end{itemize}
\begin{itemize}
\item {Utilização:Chul.}
\end{itemize}
\begin{itemize}
\item {Proveniência:(De \textunderscore engulir\textunderscore )}
\end{itemize}
Os gorgomilos; as goélas.
\section{Engulidor}
\begin{itemize}
\item {Grp. gram.:adj.}
\end{itemize}
\begin{itemize}
\item {Grp. gram.:M.}
\end{itemize}
Que engole.
Aquelle que engole.
\section{Engulipar}
\begin{itemize}
\item {Grp. gram.:v. t.}
\end{itemize}
\begin{itemize}
\item {Utilização:Chul.}
\end{itemize}
O mesmo que \textunderscore engulir\textunderscore .
\section{Engulir}
\begin{itemize}
\item {Grp. gram.:v. t.}
\end{itemize}
\begin{itemize}
\item {Utilização:Fig.}
\end{itemize}
\begin{itemize}
\item {Utilização:Pop.}
\end{itemize}
\begin{itemize}
\item {Proveniência:(De \textunderscore gula\textunderscore , segundo alguns; do lat. \textunderscore in\textunderscore  + \textunderscore glutire\textunderscore , segundo outros. Cp. cast. \textunderscore engullir\textunderscore , fr. \textunderscore engloutir\textunderscore , it. \textunderscore inghiottire\textunderscore )}
\end{itemize}
Fazer entrar pela bôca para o estomago.
Absorver.
Tragar.
Consumir, destruir: \textunderscore enguliu os haveres do pai\textunderscore .
Soffrer occultamente: \textunderscore engulir mágoas\textunderscore .
Disfarçar.
Dar crédito a: \textunderscore engulir mentírolas\textunderscore .
Fazer desapparecer.
\section{Engulodinar}
\begin{itemize}
\item {Grp. gram.:v. t.}
\end{itemize}
\begin{itemize}
\item {Utilização:Des.}
\end{itemize}
Dar gulodices ou mimos a. Cf. Fernandes, \textunderscore Caça de Altan\textunderscore .
\section{Engulosinar}
\begin{itemize}
\item {Grp. gram.:v. t.}
\end{itemize}
\begin{itemize}
\item {Proveniência:(De \textunderscore gulosina\textunderscore )}
\end{itemize}
Tornar guloso.
Estimular o appetite a.
\section{Engurunhido}
\begin{itemize}
\item {Grp. gram.:adj.}
\end{itemize}
(V.engrunhido)
\section{Enha}
\begin{itemize}
\item {Grp. gram.:adj. f.}
\end{itemize}
\begin{itemize}
\item {Utilização:ant.}
\end{itemize}
\begin{itemize}
\item {Utilização:Fam.}
\end{itemize}
O mesmo que \textunderscore minha\textunderscore . Cf. G. Vicente, I, 128.
\section{Enhader}
\begin{itemize}
\item {Grp. gram.:v. t.}
\end{itemize}
\begin{itemize}
\item {Utilização:Ant.}
\end{itemize}
O mesmo que \textunderscore enhadir\textunderscore .
\section{Enhadir}
\begin{itemize}
\item {Grp. gram.:v. t.}
\end{itemize}
\begin{itemize}
\item {Utilização:Ant.}
\end{itemize}
O mesmo que \textunderscore anaddir\textunderscore .
(Cp. cast. \textunderscore añadir\textunderscore )
\section{Enharmonia}
\begin{itemize}
\item {fónica:nar}
\end{itemize}
\begin{itemize}
\item {Grp. gram.:f.}
\end{itemize}
\begin{itemize}
\item {Proveniência:(Do lat. \textunderscore enharmonius\textunderscore )}
\end{itemize}
Gênero de modulação, em que não há mudança sensível de entonação nas notas, mas sim mudança de nome.
\section{Enharmónico}
\begin{itemize}
\item {fónica:nar}
\end{itemize}
\begin{itemize}
\item {Grp. gram.:adj.}
\end{itemize}
Relativo a enharmonia.
\section{Enho}
\begin{itemize}
\item {Grp. gram.:m.}
\end{itemize}
Corço.
Cria da corça.
(Relaciona-se com \textunderscore anho\textunderscore ?)
\section{Enhýdride}
\begin{itemize}
\item {fónica:ni}
\end{itemize}
\begin{itemize}
\item {Grp. gram.:f.}
\end{itemize}
\begin{itemize}
\item {Proveniência:(Do gr. \textunderscore enhudris\textunderscore )}
\end{itemize}
A cobra de água.
\section{Enibu}
\begin{itemize}
\item {Grp. gram.:m.}
\end{itemize}
Planta aromática das Molucas, cuja madeira se emprega em fumigações.
\section{Enicóceros}
\begin{itemize}
\item {Grp. gram.:m. pl.}
\end{itemize}
\begin{itemize}
\item {Proveniência:(Do gr. \textunderscore enikos\textunderscore  + \textunderscore keras\textunderscore )}
\end{itemize}
Gênero de insectos coleópteros da Inglaterra.
\section{Enícola}
\begin{itemize}
\item {Grp. gram.:adj.}
\end{itemize}
\begin{itemize}
\item {Proveniência:(Do gr. \textunderscore oinos\textunderscore  + lat. \textunderscore colere\textunderscore )}
\end{itemize}
Que trata de vinhos, que faz commércio de vinhos.
\section{Eniconeta}
\begin{itemize}
\item {fónica:nê}
\end{itemize}
\begin{itemize}
\item {Grp. gram.:f.}
\end{itemize}
\begin{itemize}
\item {Proveniência:(Do gr. \textunderscore enikos\textunderscore  + \textunderscore netta\textunderscore )}
\end{itemize}
Espécie de pato.
\section{Eniconetta}
\begin{itemize}
\item {Grp. gram.:f.}
\end{itemize}
\begin{itemize}
\item {Proveniência:(Do gr. \textunderscore enikos\textunderscore  + \textunderscore netta\textunderscore )}
\end{itemize}
Espécie de pato.
\section{Ênicos}
\begin{itemize}
\item {Grp. gram.:m. pl.}
\end{itemize}
\begin{itemize}
\item {Proveniência:(Gr. \textunderscore enikos\textunderscore )}
\end{itemize}
Gênero de insectos dípteros do Cabo da Boa-Esperança.
\section{Enicotarsos}
\begin{itemize}
\item {Grp. gram.:m. pl.}
\end{itemize}
\begin{itemize}
\item {Proveniência:(Do gr. \textunderscore enikos\textunderscore  + \textunderscore tarsos\textunderscore )}
\end{itemize}
Gênero de insectos coleópteros da América do Sul.
\section{Enicuros}
\begin{itemize}
\item {Grp. gram.:m. pl.}
\end{itemize}
\begin{itemize}
\item {Proveniência:(Do gr. \textunderscore enikos\textunderscore  + \textunderscore oura\textunderscore )}
\end{itemize}
Gênero de pássaros dentirostros, na Índia e archipélagos adjacentes.
\section{Enídride}
\begin{itemize}
\item {Grp. gram.:f.}
\end{itemize}
\begin{itemize}
\item {Proveniência:(Do gr. \textunderscore enhudris\textunderscore )}
\end{itemize}
A cobra de água.
\section{Enigma}
\begin{itemize}
\item {Grp. gram.:m.}
\end{itemize}
\begin{itemize}
\item {Proveniência:(Gr. \textunderscore ainigma\textunderscore )}
\end{itemize}
Descripção metaphórica ou ambígua de uma coisa, tornando-a diffícil de sêr adivinhada.
Aquillo que difficilmente se comprehende.
Coisa obscura: \textunderscore o coração da mulher é um enigma\textunderscore .
Adivinha: \textunderscore passavam o serão a decifrar enigmas\textunderscore .
\section{Enigmar}
\begin{itemize}
\item {Grp. gram.:v. t.}
\end{itemize}
\begin{itemize}
\item {Proveniência:(De \textunderscore enigma\textunderscore )}
\end{itemize}
Tornar obscuro, enigmático.
\section{Enigmaticamente}
\begin{itemize}
\item {Grp. gram.:adv.}
\end{itemize}
De modo enigmático.
\section{Enigmático}
\begin{itemize}
\item {Grp. gram.:adj.}
\end{itemize}
Relativo a enigma.
Obscuro; incomprehensível; mysterioso.
Que difficilmente se percebe: \textunderscore palavras enigmáticas\textunderscore .
\section{Enigmatista}
\begin{itemize}
\item {Grp. gram.:m.}
\end{itemize}
Aquelle que faz ou decifra enigmas.
\section{Enigmista}
\begin{itemize}
\item {Grp. gram.:m.}
\end{itemize}
(V.enigmatista)
\section{Enilema}
\begin{itemize}
\item {Grp. gram.:m.}
\end{itemize}
\begin{itemize}
\item {Utilização:Bot.}
\end{itemize}
\begin{itemize}
\item {Proveniência:(Gr. \textunderscore eneilema\textunderscore )}
\end{itemize}
Uma das três membranas do óvulo vegetal.
\section{Enito}
\begin{itemize}
\item {Grp. gram.:m.}
\end{itemize}
\begin{itemize}
\item {Utilização:Pharm.}
\end{itemize}
\begin{itemize}
\item {Proveniência:(Do gr. \textunderscore oinos\textunderscore )}
\end{itemize}
Qualquer preparado, em que entre vinho.
\section{Enjambrar}
\begin{itemize}
\item {Grp. gram.:v. i.}
\end{itemize}
\begin{itemize}
\item {Utilização:Constr.}
\end{itemize}
Diz-se da tábua que torce ou empena.
\section{Enjangar}
\begin{itemize}
\item {Grp. gram.:v. t.}
\end{itemize}
Converter em jangada.
Ligar á maneira de jangada.
\section{Enjaular}
\begin{itemize}
\item {Grp. gram.:v. t.}
\end{itemize}
Meter em jaula.
\section{Enjeitado}
\begin{itemize}
\item {Grp. gram.:m.}
\end{itemize}
\begin{itemize}
\item {Grp. gram.:Adj.}
\end{itemize}
\begin{itemize}
\item {Utilização:Pop.}
\end{itemize}
Aquelle que foi abandonado pelos pais.
Que foi abandonado pelos pais.
Diz-se do toiro, que não tem a marca do lavrador nem se sabe a que manada pertenceu.
Diz-se do ninho ou dos ovos, que a ave enjeitou ou abandonou, por alguém lhes mexer.
\section{Enjeitador}
\begin{itemize}
\item {Grp. gram.:adj.}
\end{itemize}
\begin{itemize}
\item {Grp. gram.:M.}
\end{itemize}
Que enjeita.
Aquelle que enjeita.
\section{Enjeitamento}
\begin{itemize}
\item {Grp. gram.:m.}
\end{itemize}
Acto ou effeito de enjeitar.
\section{Enjeitar}
\begin{itemize}
\item {Grp. gram.:v. t.}
\end{itemize}
\begin{itemize}
\item {Utilização:Pop.}
\end{itemize}
\begin{itemize}
\item {Proveniência:(Lat. \textunderscore injectare\textunderscore )}
\end{itemize}
Rejeitar.
Abandonar (filhos).
Abandonar (o ninho, falando-se da ave, que não chegou a pôr ovos ou a criar filhos, no ninho que construiu).
Recusar.
Repellir: \textunderscore enjeitar offerecimentos\textunderscore .
Reprovar.
\section{Enjoadiço}
\begin{itemize}
\item {Grp. gram.:adj.}
\end{itemize}
Dado a enjôos. Cf. Castilho, \textunderscore Metam.\textunderscore , XIX.
\section{Enjoado}
\begin{itemize}
\item {Grp. gram.:adj.}
\end{itemize}
\begin{itemize}
\item {Proveniência:(De \textunderscore enjoar\textunderscore )}
\end{itemize}
Que soffre enjôo.
Diz-se do peixe, que não está bem sêco.
\section{Enjoamento}
\begin{itemize}
\item {Grp. gram.:m.}
\end{itemize}
O mesmo que \textunderscore enjôo\textunderscore .
\section{Enjoar}
\begin{itemize}
\item {Grp. gram.:v. t.}
\end{itemize}
\begin{itemize}
\item {Utilização:Fig.}
\end{itemize}
\begin{itemize}
\item {Grp. gram.:V. i.}
\end{itemize}
\begin{itemize}
\item {Utilização:Mad}
\end{itemize}
Causar enjôo a.
Sentir enjôo por.
Aborrecer, enfastiar.
Causar repugnância a.
Soffrer enjôo ou náusea.
Exalar mau cheiro.
(Metáth. de \textunderscore enojar\textunderscore )
\section{Enjoativo}
\begin{itemize}
\item {Grp. gram.:adj.}
\end{itemize}
Que enjôa, que causa enjôo.
\section{Enjòiar}
\begin{itemize}
\item {Grp. gram.:v. t.}
\end{itemize}
\begin{itemize}
\item {Utilização:Des.}
\end{itemize}
Adornar com jóias.
\section{Enjoinar}
\begin{itemize}
\item {Grp. gram.:v. t.}
\end{itemize}
\begin{itemize}
\item {Utilização:Prov.}
\end{itemize}
\begin{itemize}
\item {Utilização:alent.}
\end{itemize}
Cobrir ou tapar com joina ou mato tenro.
\section{Enjôo}
\begin{itemize}
\item {Grp. gram.:m.}
\end{itemize}
\begin{itemize}
\item {Utilização:Marn.}
\end{itemize}
\begin{itemize}
\item {Proveniência:(De \textunderscore enjoar\textunderscore )}
\end{itemize}
Náusea.
Princípio de engulho.
Soffrimento do estômago e da cabeça, peculiar a algumas pessôas que viajam em navio, carro, etc.
Náusea e entontecimento, que acompanham ás vezes o estado de gravidez.
Moléstia das salinas, produzida pelo elevado grau de concentração da água nos crystallizadores.
\section{Enjooso}
\begin{itemize}
\item {Grp. gram.:adj.}
\end{itemize}
Que causa enjôo.
Que causa aborrecimento ou enfado:«\textunderscore ...enjoosa penitencia...\textunderscore »Garrett, \textunderscore D. Branca\textunderscore , 36.
\section{Enjorcar}
\begin{itemize}
\item {Grp. gram.:v. t.}
\end{itemize}
\begin{itemize}
\item {Utilização:Prov.}
\end{itemize}
Entrajar mal ou á pressa.
Vestir atabalhoadamente: \textunderscore a mãi enjorca os filhos e manda-os para a rua\textunderscore .
(Por \textunderscore axorcar\textunderscore , de \textunderscore axorca\textunderscore ?)
\section{Enjorgar}
\begin{itemize}
\item {Grp. gram.:v. t.}
\end{itemize}
\begin{itemize}
\item {Utilização:T. da Bairrada}
\end{itemize}
O mesmo que \textunderscore enjorcar\textunderscore .
\section{Enjugamento}
\begin{itemize}
\item {Grp. gram.:m.}
\end{itemize}
Acto ou effeito de enjugar.
\section{Enjugar}
\begin{itemize}
\item {Grp. gram.:v. t.}
\end{itemize}
Pôr o jugo em (bois).
\section{Enjunçar}
\begin{itemize}
\item {Grp. gram.:v. t.  e  i.}
\end{itemize}
Humedecer (a madeira), de fórma que, ao aplainar-se ou obrar-se, o instrumento embota e o trabalho difficulta-se: \textunderscore este castanho enjunça, porque estava na adega\textunderscore .
\section{Enlabirintar}
\begin{itemize}
\item {Grp. gram.:v. t.}
\end{itemize}
Converter em labirinto; desordenar.
\section{Enlabruscar}
\begin{itemize}
\item {Grp. gram.:v. i.}
\end{itemize}
\begin{itemize}
\item {Utilização:Prov.}
\end{itemize}
\begin{itemize}
\item {Utilização:trasm.}
\end{itemize}
O mesmo que \textunderscore enlambuzar\textunderscore .
\section{Enlabuzar}
\begin{itemize}
\item {Grp. gram.:v. i. (e der.)}
\end{itemize}
O mesmo que \textunderscore enlambuzar\textunderscore .
(Us. na Beira e Trás-os-Montes)
\section{Enlabyrinthar}
\begin{itemize}
\item {Grp. gram.:v. t.}
\end{itemize}
Converter em labyrintho; desordenar.
\section{Enlaçadura}
\begin{itemize}
\item {Grp. gram.:f.}
\end{itemize}
Acto ou effeito de enlaçar.
Peça, com que se enlaça o elmo.
\section{Enlaçamento}
\begin{itemize}
\item {Grp. gram.:m.}
\end{itemize}
O mesmo que \textunderscore enlaçadura\textunderscore .
\section{Enlaçar}
\begin{itemize}
\item {Grp. gram.:v. t.}
\end{itemize}
\begin{itemize}
\item {Grp. gram.:V. i.}
\end{itemize}
Prender com laços.
Atar em fórma de laço.
Unir.
Entrelaçar; abraçar: \textunderscore enlaçou-a pela cintura\textunderscore .
Combinar.
Têr relação ou connexão.
\section{Enlace}
\begin{itemize}
\item {Grp. gram.:m}
\end{itemize}
\begin{itemize}
\item {Utilização:Fig.}
\end{itemize}
Acto ou effeito de enlaçar.
Hesitação.
Casamento.
\section{Enlacrar}
\begin{itemize}
\item {Grp. gram.:v. t.}
\end{itemize}
Dar côr de lacre a.
\section{Enladeirado}
\begin{itemize}
\item {Grp. gram.:adj.}
\end{itemize}
\begin{itemize}
\item {Proveniência:(De \textunderscore enladeirar\textunderscore )}
\end{itemize}
Inclinado, declive.
\section{Enladeirar}
\begin{itemize}
\item {Grp. gram.:v. t.}
\end{itemize}
\begin{itemize}
\item {Proveniência:(De \textunderscore ladeira\textunderscore )}
\end{itemize}
Tornar inclinado ou declive.
\section{Enlaga}
\begin{itemize}
\item {Grp. gram.:f.}
\end{itemize}
Acto de enlagar.
\section{Enlagar}
\begin{itemize}
\item {Grp. gram.:v. t.}
\end{itemize}
\begin{itemize}
\item {Utilização:Prov.}
\end{itemize}
\begin{itemize}
\item {Proveniência:(De \textunderscore lago\textunderscore )}
\end{itemize}
Alagar ou enriar (o linho).
\section{Enlaivar}
\begin{itemize}
\item {Grp. gram.:v. t.}
\end{itemize}
Cobrir de laivos.
Sujar, manchar.
\section{Enlambujar}
\begin{itemize}
\item {Grp. gram.:v. i.}
\end{itemize}
\begin{itemize}
\item {Proveniência:(De \textunderscore lambugem\textunderscore )}
\end{itemize}
Mostrar-se guloso; andar á lambujem.
\section{Enlambuzar}
\textunderscore v. t.\textunderscore  (e der.)
O mesmo que \textunderscore lambuzar\textunderscore , etc.
\section{Enlameadura}
\begin{itemize}
\item {Grp. gram.:f.}
\end{itemize}
Acto ou effeito de enlamear.
\section{Enlamear}
\begin{itemize}
\item {Grp. gram.:v. t.}
\end{itemize}
\begin{itemize}
\item {Utilização:Fig.}
\end{itemize}
Sujar com lama.
Manchar.
Ennodoar a reputação de.
Deslustrar.
\section{Enlaminar}
\begin{itemize}
\item {Grp. gram.:v. t.}
\end{itemize}
Forrar com lâminas ou chapas de metal.
\section{Enlanguescer}
\textunderscore v. i.\textunderscore  e \textunderscore p.\textunderscore  (e der.)
O mesmo que \textunderscore elanguescer\textunderscore , etc.
\section{Enlapar}
\begin{itemize}
\item {Grp. gram.:v. t.}
\end{itemize}
\begin{itemize}
\item {Utilização:Fig.}
\end{itemize}
Meter em lapa.
Alapar.
Esconder.
Fazer desapparecer, sumir.
\section{Enlatamento}
\begin{itemize}
\item {Grp. gram.:m.}
\end{itemize}
Acto de enlatar.
\section{Enlatar}
\begin{itemize}
\item {Grp. gram.:v. t.}
\end{itemize}
Meter em latas (sardinha) e soldar as mesmas latas.
\section{Enlatar}
\begin{itemize}
\item {Grp. gram.:v. t.}
\end{itemize}
\begin{itemize}
\item {Proveniência:(De lata)}
\end{itemize}
Dispor ou suster em latadas.
\section{Enleamento}
\begin{itemize}
\item {Grp. gram.:m.}
\end{itemize}
Acto de enlear; enleio.
\section{Enlear}
\begin{itemize}
\item {Grp. gram.:v. t.}
\end{itemize}
Atar com liame.
Ligar.
Tornar perplexo, hesitante.
Perturbar: \textunderscore enleou-me a pregunta\textunderscore .
Attrahir a attenção de.
Cativar, enlevar.
Meter em difficuldades, embaraçar.
(Por \textunderscore enliar\textunderscore , de \textunderscore liar\textunderscore )
\section{Enleia}
\begin{itemize}
\item {Grp. gram.:f.}
\end{itemize}
\begin{itemize}
\item {Proveniência:(De \textunderscore enlear\textunderscore )}
\end{itemize}
Corda delgada, com que se atam objectos que devem formar a carga das bêstas.
\section{Enleio}
\begin{itemize}
\item {Grp. gram.:m.}
\end{itemize}
Coisa que enleia, liame.
Acto ou effeito de enlear.
Planta trepadeira, parasita.
\section{Enleitado}
\begin{itemize}
\item {Grp. gram.:adj.}
\end{itemize}
\begin{itemize}
\item {Proveniência:(De \textunderscore leito\textunderscore )}
\end{itemize}
Que assenta bem, (falando-se de pedras para construcções).
\section{Enleivamento}
\begin{itemize}
\item {Grp. gram.:m.}
\end{itemize}
Acto ou effeito de enleivar.
\section{Enleivar}
\begin{itemize}
\item {Grp. gram.:v. t.}
\end{itemize}
\begin{itemize}
\item {Proveniência:(De \textunderscore leiva\textunderscore )}
\end{itemize}
Mover (terras para terraplanagem).
\section{Enlerdar}
\begin{itemize}
\item {Grp. gram.:v. t.}
\end{itemize}
Tornar lerdo.
\section{Enlevação}
\begin{itemize}
\item {Grp. gram.:f.}
\end{itemize}
Acto ou effeito de enlevar.
\section{Enlevador}
\begin{itemize}
\item {Grp. gram.:adj.}
\end{itemize}
Que enleva, que encanta.
\section{Enlevamento}
\begin{itemize}
\item {Grp. gram.:m.}
\end{itemize}
O mesmo que \textunderscore enlevação\textunderscore .
\section{Enlevar}
\begin{itemize}
\item {Grp. gram.:v. t.}
\end{itemize}
\begin{itemize}
\item {Proveniência:(De \textunderscore levar\textunderscore )}
\end{itemize}
Causar enlêvo a.
Extasiar, arrebatar; encantar.
\section{Enlêvo}
\begin{itemize}
\item {Grp. gram.:m.}
\end{itemize}
\begin{itemize}
\item {Proveniência:(De \textunderscore enlevar\textunderscore )}
\end{itemize}
Encanto; êxtase.
Arrobo, arrebatamento.
Deleite.
Assombro, coisa que maravilha.
Pessôa, que encanta, que cativa.
\section{Enlhear}
\begin{itemize}
\item {Grp. gram.:v. t.}
\end{itemize}
O mesmo que alhear:«\textunderscore não se póde perder nem enlhear.\textunderscore »Barros, \textunderscore Espelho de Casados\textunderscore .
\section{Enliçador}
\begin{itemize}
\item {Grp. gram.:adj.}
\end{itemize}
\begin{itemize}
\item {Grp. gram.:M.}
\end{itemize}
Que enliça.
Aquelle que enliça.
Burlão, aquelle que rouba, enganando. Cf. \textunderscore Ordenações do Reino\textunderscore .
\section{Enliçar}
\begin{itemize}
\item {Grp. gram.:v. t.}
\end{itemize}
\begin{itemize}
\item {Utilização:Fig.}
\end{itemize}
Pôr os liços em.
Fraudar, enganar.
\section{Enliço}
\begin{itemize}
\item {Grp. gram.:m.}
\end{itemize}
\begin{itemize}
\item {Utilização:Fig.}
\end{itemize}
\begin{itemize}
\item {Proveniência:(De \textunderscore enliçar\textunderscore )}
\end{itemize}
Má urdidura.
Fraude.
\section{Enlocar}
\begin{itemize}
\item {Grp. gram.:v. t.}
\end{itemize}
Meter em loca.
Encafuar.
\section{Enlodaçar}
\begin{itemize}
\item {Grp. gram.:v. t.}
\end{itemize}
Converter em lodaçal. Cf. Garrett, \textunderscore Flôres sem Fruto\textunderscore , 201.
\section{Enlodar}
\begin{itemize}
\item {Grp. gram.:v. t.}
\end{itemize}
\begin{itemize}
\item {Proveniência:(De \textunderscore lodo\textunderscore )}
\end{itemize}
O mesmo que \textunderscore enlamear\textunderscore .
Sujar; contaminar.
\section{Enloiçar}
\begin{itemize}
\item {Grp. gram.:v. t.}
\end{itemize}
Meter na loiça ou envasilhar (vinho).
\section{Enloirar}
\begin{itemize}
\item {Grp. gram.:v. t.}
\end{itemize}
Aloirar.
Enfeitar ou coroar de loiros. Cf. Garção, II, 111.
\section{Enloirecer}
\begin{itemize}
\item {Grp. gram.:v. t.}
\end{itemize}
O mesmo que \textunderscore aloirar\textunderscore .
\section{Enloisamento}
\begin{itemize}
\item {Grp. gram.:m.}
\end{itemize}
Acto ou effeito de enloisar.
\section{Enloisar}
\begin{itemize}
\item {Grp. gram.:v. t.}
\end{itemize}
Cobrir com loisa.
Forrar de loisas.
Caçar com loisa.
\section{Enlojamento}
\begin{itemize}
\item {Grp. gram.:m.}
\end{itemize}
Acto ou effeito de enlojar.
\section{Enlojar}
\begin{itemize}
\item {Grp. gram.:v. t.}
\end{itemize}
Meter em loja, armazenar.
Enloiçar, envasilhar.
\section{Enlorpecer}
\begin{itemize}
\item {Grp. gram.:v. t.}
\end{itemize}
Tornar lorpa.
\section{Enlouçar}
\begin{itemize}
\item {Grp. gram.:v. t.}
\end{itemize}
Meter na loiça ou envasilhar (vinho).
\section{Enlouquecer}
\begin{itemize}
\item {Grp. gram.:v. t.}
\end{itemize}
\begin{itemize}
\item {Grp. gram.:V. i.  e  p.}
\end{itemize}
Tirar o uso da razão a.
Tornar-se louco.
\section{Enlouquecimento}
\begin{itemize}
\item {Grp. gram.:m.}
\end{itemize}
Acto ou effeito de enlouquecer.
\section{Enlourar}
\begin{itemize}
\item {Grp. gram.:v. t.}
\end{itemize}
Alourar.
Enfeitar ou coroar de louros. Cf. Garção, II, 111.
\section{Enlousamento}
\begin{itemize}
\item {Grp. gram.:m.}
\end{itemize}
Acto ou effeito de enlousar.
\section{Enlousar}
\begin{itemize}
\item {Grp. gram.:v. t.}
\end{itemize}
Cobrir com lousa.
Forrar de lousas.
Caçar com lousa.
\section{Enluarado}
\begin{itemize}
\item {Grp. gram.:adj.}
\end{itemize}
Illuminado pelo luar: \textunderscore campo enluarado\textunderscore .
\section{Enludrar}
\begin{itemize}
\item {Grp. gram.:v. t.}
\end{itemize}
Tornar ludro, turvo, sujo: \textunderscore enludrar a água\textunderscore .
\section{Enlutar}
\begin{itemize}
\item {Grp. gram.:v. t.}
\end{itemize}
Cobrir de luto.
Causar grande mágoa a; consternar.
Envolver em trevas.
\section{Enea}
(Elemento, de origem grega, que entra na composição de alguns termos scientíficos ou artisticos, com a significação de \textunderscore nove\textunderscore )
\section{Eneaginia}
\begin{itemize}
\item {Grp. gram.:f.}
\end{itemize}
Qualidade de eneágino.
Conjunto das plantas, que têm nove esliletes \textunderscore ou\textunderscore , pelo menos, nove estigmas.
\section{Eneágino}
\begin{itemize}
\item {Grp. gram.:adj.}
\end{itemize}
\begin{itemize}
\item {Utilização:Bot.}
\end{itemize}
\begin{itemize}
\item {Proveniência:(Do gr. \textunderscore ennea\textunderscore  + \textunderscore gune\textunderscore )}
\end{itemize}
Que tem nove estiletes ou nove estigmas.
\section{Eneagonal}
\begin{itemize}
\item {Grp. gram.:adj.}
\end{itemize}
\begin{itemize}
\item {Proveniência:(De \textunderscore eneágono\textunderscore )}
\end{itemize}
Que tem nove ângulos.
\section{Eneágono}
\begin{itemize}
\item {Grp. gram.:m.}
\end{itemize}
\begin{itemize}
\item {Utilização:Geom.}
\end{itemize}
\begin{itemize}
\item {Proveniência:(Do gr. \textunderscore ennea\textunderscore  + \textunderscore gonos\textunderscore )}
\end{itemize}
Figura de nove ângulos e de nove lados.
\section{Eneandria}
\begin{itemize}
\item {Grp. gram.:f.}
\end{itemize}
\begin{itemize}
\item {Utilização:Bot.}
\end{itemize}
Qualidade de eneandro.
Classe dos vegetaes, que têm nove estames, livres entre si.
\section{Eneandro}
\begin{itemize}
\item {Grp. gram.:adj.}
\end{itemize}
\begin{itemize}
\item {Utilização:Bot.}
\end{itemize}
\begin{itemize}
\item {Proveniência:(Do gr. \textunderscore ennea\textunderscore  + \textunderscore aner\textunderscore , \textunderscore andros\textunderscore )}
\end{itemize}
Que tem nove estames, livres entre si.
\section{Eneantera}
\begin{itemize}
\item {Grp. gram.:f.}
\end{itemize}
O mesmo que \textunderscore eneandria\textunderscore .
\section{Eneapétalo}
\begin{itemize}
\item {Grp. gram.:adj.}
\end{itemize}
\begin{itemize}
\item {Utilização:Bot.}
\end{itemize}
Que tem nove pétalas.
\section{Eneaspermo}
\begin{itemize}
\item {Grp. gram.:adj.}
\end{itemize}
\begin{itemize}
\item {Utilização:Bot.}
\end{itemize}
Que tem nove sementes.
\section{Eneassépalo}
\begin{itemize}
\item {Grp. gram.:adj.}
\end{itemize}
\begin{itemize}
\item {Utilização:Bot.}
\end{itemize}
Que tem nove sépalas.
\section{Eneassílabo}
\begin{itemize}
\item {Grp. gram.:m.  e  adj.}
\end{itemize}
Palavra de nove sílabas.
\section{Enervação}
\begin{itemize}
\item {Grp. gram.:f.}
\end{itemize}
\begin{itemize}
\item {Proveniência:(De \textunderscore enervar\textunderscore )}
\end{itemize}
Modo especial de actividade, próprio dos elementos nervosos.
Conjunto dos fenómenos nervosos.
\section{Enervado}
\begin{itemize}
\item {Grp. gram.:adj.}
\end{itemize}
\begin{itemize}
\item {Proveniência:(De \textunderscore enervar\textunderscore )}
\end{itemize}
Dizia-se do instrumento, que tem cordas de nervo.
\section{Enervar}
\begin{itemize}
\item {Grp. gram.:v. t.}
\end{itemize}
\begin{itemize}
\item {Proveniência:(De \textunderscore nervo\textunderscore )}
\end{itemize}
Forrar de nervos.
Forrar de coiro.
Fazer nervuras em.
Comunicar actividade ou faculdade motriz a.
\section{Enluvado}
\begin{itemize}
\item {Grp. gram.:adj.}
\end{itemize}
Que traz luvas.
\section{Enluvar-se}
\begin{itemize}
\item {Grp. gram.:v. p.}
\end{itemize}
Calçar luvas.
\section{Ennaipar}
\begin{itemize}
\item {Grp. gram.:v. t.}
\end{itemize}
\begin{itemize}
\item {Proveniência:(De \textunderscore naipe\textunderscore )}
\end{itemize}
Juntar ou separar por ordem de naipes (um baralho ou as cartas que um parceiro vai jogar).
\section{Ennastrar}
\begin{itemize}
\item {Grp. gram.:v. t.}
\end{itemize}
Atar com fitas ou nastros.
Entrelaçar.
Ornar de nastros.
\section{Ennatar}
\begin{itemize}
\item {Grp. gram.:v. t.}
\end{itemize}
Encher de nateiros.
Cobrir de nata.
\section{Ennateiramento}
\begin{itemize}
\item {Grp. gram.:m.}
\end{itemize}
Acto ou effeito de ennateirar.
\section{Ennateirar}
\begin{itemize}
\item {Grp. gram.:v.}
\end{itemize}
\begin{itemize}
\item {Utilização:t. Agr.}
\end{itemize}
Converter em nateiro. Cf. Assis, Aguas, 164.
\section{Ennea...}
(Elemento, de origem grega, que entra na composição de alguns termos scientíficos ou artisticos, com a significação de \textunderscore nove\textunderscore )
\section{Enneagonal}
\begin{itemize}
\item {Grp. gram.:adj.}
\end{itemize}
\begin{itemize}
\item {Proveniência:(De \textunderscore enneágono\textunderscore )}
\end{itemize}
Que tem nove ângulos.
\section{Enneágono}
\begin{itemize}
\item {Grp. gram.:m.}
\end{itemize}
\begin{itemize}
\item {Utilização:Geom.}
\end{itemize}
\begin{itemize}
\item {Proveniência:(Do gr. \textunderscore ennea\textunderscore  + \textunderscore gonos\textunderscore )}
\end{itemize}
Figura de nove ângulos e de nove lados.
\section{Enneagynia}
\begin{itemize}
\item {Grp. gram.:f.}
\end{itemize}
Qualidade de enneágyno.
Conjunto das plantas, que têm nove esliletes \textunderscore ou\textunderscore , pelo menos, nove estigmas.
\section{Enneágyno}
\begin{itemize}
\item {Grp. gram.:adj.}
\end{itemize}
\begin{itemize}
\item {Utilização:Bot.}
\end{itemize}
\begin{itemize}
\item {Proveniência:(Do gr. \textunderscore ennea\textunderscore  + \textunderscore gune\textunderscore )}
\end{itemize}
Que tem nove estiletes ou nove estigmas.
\section{Enneandria}
\begin{itemize}
\item {Grp. gram.:f.}
\end{itemize}
\begin{itemize}
\item {Utilização:Bot.}
\end{itemize}
Qualidade de enneandro.
Classe dos vegetaes, que têm nove estames, livres entre si.
\section{Enneandro}
\begin{itemize}
\item {Grp. gram.:adj.}
\end{itemize}
\begin{itemize}
\item {Utilização:Bot.}
\end{itemize}
\begin{itemize}
\item {Proveniência:(Do gr. \textunderscore ennea\textunderscore  + \textunderscore aner\textunderscore , \textunderscore andros\textunderscore )}
\end{itemize}
Que tem nove estames, livres entre si.
\section{Enneanthera}
\begin{itemize}
\item {Grp. gram.:f.}
\end{itemize}
O mesmo que \textunderscore enneandria\textunderscore .
\section{Enneapétalo}
\begin{itemize}
\item {Grp. gram.:adj.}
\end{itemize}
\begin{itemize}
\item {Utilização:Bot.}
\end{itemize}
Que tem nove pétalas.
\section{Enneasépalo}
\begin{itemize}
\item {fónica:sé}
\end{itemize}
\begin{itemize}
\item {Grp. gram.:adj.}
\end{itemize}
\begin{itemize}
\item {Utilização:Bot.}
\end{itemize}
Que tem nove sépalas.
\section{Enneaspermo}
\begin{itemize}
\item {Grp. gram.:adj.}
\end{itemize}
\begin{itemize}
\item {Utilização:Bot.}
\end{itemize}
Que tem nove sementes.
\section{Enneasýllabo}
\begin{itemize}
\item {fónica:si}
\end{itemize}
\begin{itemize}
\item {Grp. gram.:m.  e  adj.}
\end{itemize}
Palavra de nove sýllabas.
\section{Ennegrecer}
\begin{itemize}
\item {Grp. gram.:v. t.}
\end{itemize}
\begin{itemize}
\item {Utilização:Fig.}
\end{itemize}
\begin{itemize}
\item {Grp. gram.:V. i.  e  p.}
\end{itemize}
\begin{itemize}
\item {Proveniência:(Do lat. \textunderscore nigrescere\textunderscore )}
\end{itemize}
Tornar negro.
Deslustrar; desacreditar.
Tornar-se negro.
\section{Ennegrecimento}
\begin{itemize}
\item {Grp. gram.:m.}
\end{itemize}
Acto ou effeito de ennegrecer.
\section{Ennervação}
\begin{itemize}
\item {Grp. gram.:f.}
\end{itemize}
\begin{itemize}
\item {Proveniência:(De \textunderscore ennervar\textunderscore )}
\end{itemize}
Modo especial de actividade, próprio dos elementos nervosos.
Conjunto dos phenómenos nervosos.
\section{Ennervado}
\begin{itemize}
\item {Grp. gram.:adj.}
\end{itemize}
\begin{itemize}
\item {Proveniência:(De \textunderscore ennervar\textunderscore )}
\end{itemize}
Dizia-se do instrumento, que tem cordas de nervo.
\section{Ennervar}
\begin{itemize}
\item {Grp. gram.:v. t.}
\end{itemize}
\begin{itemize}
\item {Proveniência:(De \textunderscore nervo\textunderscore )}
\end{itemize}
Forrar de nervos.
Forrar de coiro.
Fazer nervuras em.
Communicar actividade ou faculdade motriz a.
\section{Ennerve}
\begin{itemize}
\item {Grp. gram.:adj.}
\end{itemize}
\begin{itemize}
\item {Proveniência:(Lat. \textunderscore innervis\textunderscore )}
\end{itemize}
O mesmo que \textunderscore effeminado\textunderscore .
\section{Ennesgar}
\begin{itemize}
\item {Grp. gram.:v. t.}
\end{itemize}
\begin{itemize}
\item {Grp. gram.:V. i.}
\end{itemize}
Cortar em fórma de nesga.
Dar fórma triangular ou de nesga a: \textunderscore ennesgar uma casa\textunderscore .
Tomar fórma de nesga.
\section{Ennevoar}
\begin{itemize}
\item {Grp. gram.:v. t.}
\end{itemize}
\begin{itemize}
\item {Utilização:Fig.}
\end{itemize}
\begin{itemize}
\item {Proveniência:(Do lat. \textunderscore nebulare\textunderscore )}
\end{itemize}
Cobrir de névoa.
Obscurecer.
Anuvear, nublar.
Tornar baço, opaco.
Diffamar.
Deslustrar.
Tornar triste.
Perturbar a intelligência de.
\section{Ennobrecedor}
\begin{itemize}
\item {Grp. gram.:adj.}
\end{itemize}
\begin{itemize}
\item {Grp. gram.:M.}
\end{itemize}
Que ennobrece.
Aquelle que ennobrece.
\section{Ennobrecer}
\begin{itemize}
\item {Grp. gram.:v. t.}
\end{itemize}
\begin{itemize}
\item {Utilização:Fig.}
\end{itemize}
Tornar nobre.
Fazer illustre.
Tornar formoso, adornar.
\section{Ennobrecimento}
\begin{itemize}
\item {Grp. gram.:m.}
\end{itemize}
Acto ou effeito de ennobrecer.
\section{Ennodar}
\begin{itemize}
\item {Grp. gram.:v. t.}
\end{itemize}
\begin{itemize}
\item {Proveniência:(Do lat. \textunderscore innodare\textunderscore )}
\end{itemize}
Fazer nós em.
Tornar nodoso.
Ligar por meio de nó.
\section{Ennodoar}
\begin{itemize}
\item {Grp. gram.:v. t.}
\end{itemize}
\begin{itemize}
\item {Utilização:Fig.}
\end{itemize}
Pôr nódoas em.
Sujar.
Deslustrar; diffamar.
\section{Ennoitar}
\begin{itemize}
\item {Grp. gram.:v. t.}
\end{itemize}
O mesmo que \textunderscore ennoitecer\textunderscore .
\section{Ennoitecer}
\begin{itemize}
\item {Grp. gram.:v. t.}
\end{itemize}
\begin{itemize}
\item {Utilização:Fig.}
\end{itemize}
\begin{itemize}
\item {Grp. gram.:V. i.}
\end{itemize}
\begin{itemize}
\item {Proveniência:(De \textunderscore noite\textunderscore )}
\end{itemize}
Tornar escuro.
Cercar de trevas.
Contristar.
Affligir.
Enlutar: \textunderscore a viuvez ennoiteceu-o\textunderscore .
O mesmo que \textunderscore anoitecer\textunderscore .
\section{Ennoutar}
\begin{itemize}
\item {Grp. gram.:v. t.}
\end{itemize}
O mesmo que \textunderscore ennoutecer\textunderscore .
\section{Ennoutecer}
\begin{itemize}
\item {Grp. gram.:v. t.}
\end{itemize}
\begin{itemize}
\item {Utilização:Fig.}
\end{itemize}
\begin{itemize}
\item {Grp. gram.:V. i.}
\end{itemize}
\begin{itemize}
\item {Proveniência:(De \textunderscore noute\textunderscore )}
\end{itemize}
Tornar escuro.
Cercar de trevas.
Contristar.
Affligir.
Enlutar: \textunderscore a viuvez ennouteceu-o\textunderscore .
O mesmo que \textunderscore anoutecer\textunderscore .
\section{Ennovar}
\begin{itemize}
\item {Grp. gram.:v. t.}
\end{itemize}
O mesmo ou melhor que innovar. Cf. \textunderscore Aulegrafia\textunderscore , 26.
\section{Ennoveladeira}
\begin{itemize}
\item {Grp. gram.:f.}
\end{itemize}
Apparelho, com que se formam os novelos, nas fábricas de fiação. Cf. \textunderscore Inquér. Industr.\textunderscore , p. II, l. II, 122.
\section{Ennovelar}
\begin{itemize}
\item {Grp. gram.:v. t.}
\end{itemize}
\begin{itemize}
\item {Utilização:Fig.}
\end{itemize}
\begin{itemize}
\item {Proveniência:(De \textunderscore novelo\textunderscore )}
\end{itemize}
Converter em novelo, dobando.
Enrolar.
Enredar.
Tornar confuso.
\section{Ennublar}
\begin{itemize}
\item {Grp. gram.:v. t.}
\end{itemize}
\begin{itemize}
\item {Proveniência:(De \textunderscore nublar\textunderscore )}
\end{itemize}
O mesmo que \textunderscore anuvear\textunderscore .
\section{Ennuvear}
\begin{itemize}
\item {Grp. gram.:v. t.}
\end{itemize}
O mesmo que \textunderscore anuvear\textunderscore .
\section{Eno...}
\begin{itemize}
\item {Grp. gram.:pref.}
\end{itemize}
\begin{itemize}
\item {Proveniência:(Do gr. \textunderscore oinos\textunderscore )}
\end{itemize}
(que significa \textunderscore vinho\textunderscore )
\section{Enocarpo}
\begin{itemize}
\item {Grp. gram.:m.}
\end{itemize}
\begin{itemize}
\item {Proveniência:(Do gr. \textunderscore oinos\textunderscore  + \textunderscore karpos\textunderscore )}
\end{itemize}
Gênero de palmeiras.
\section{Enodo}
\begin{itemize}
\item {Grp. gram.:adj.}
\end{itemize}
\begin{itemize}
\item {Utilização:Bot.}
\end{itemize}
\begin{itemize}
\item {Proveniência:(Do lat. \textunderscore e\textunderscore  + \textunderscore nodus\textunderscore )}
\end{itemize}
Que não tem nós, que não é nodoso.
\section{Enófilo}
\begin{itemize}
\item {Grp. gram.:adj.}
\end{itemize}
\begin{itemize}
\item {Proveniência:(Do gr. \textunderscore oinos\textunderscore  + \textunderscore philos\textunderscore )}
\end{itemize}
Que gosta de vinho.
Que se dedica a comércio ou assuntos vinícolas.
\section{Enofobia}
\begin{itemize}
\item {Grp. gram.:f.}
\end{itemize}
\begin{itemize}
\item {Proveniência:(De \textunderscore enófobo\textunderscore )}
\end{itemize}
Aversão ou horror ao vinho.
\section{Enófora}
\begin{itemize}
\item {Grp. gram.:f.}
\end{itemize}
O mesmo que \textunderscore enóforo\textunderscore .
\section{Enóforo}
\begin{itemize}
\item {Grp. gram.:m.}
\end{itemize}
\begin{itemize}
\item {Proveniência:(Lat. \textunderscore oenophorum\textunderscore )}
\end{itemize}
Vaso para vinho, entre os Romanos.
\section{Enoira}
\begin{itemize}
\item {Grp. gram.:f.}
\end{itemize}
Árvore brasileira das regiões do Amazonas.
\section{Enoiriçar}
\begin{itemize}
\item {fónica:en-oi}
\end{itemize}
\begin{itemize}
\item {Grp. gram.:v. t.}
\end{itemize}
\begin{itemize}
\item {Utilização:Des.}
\end{itemize}
O mesmo que \textunderscore ouriçar\textunderscore .
\section{Enojadamente}
\begin{itemize}
\item {Grp. gram.:adv.}
\end{itemize}
\begin{itemize}
\item {Proveniência:(De \textunderscore enojar\textunderscore )}
\end{itemize}
Com tédio ou nojo.
\section{Enojadiço}
\begin{itemize}
\item {Grp. gram.:adj.}
\end{itemize}
Que se enoja facilmente.
\section{Enojador}
\begin{itemize}
\item {Grp. gram.:adj.}
\end{itemize}
\begin{itemize}
\item {Grp. gram.:M.}
\end{itemize}
Que enoja.
Aquelle que enoja.
\section{Enojamento}
\begin{itemize}
\item {Grp. gram.:m.}
\end{itemize}
O mesmo que \textunderscore enôjo\textunderscore .
\section{Enojar}
\begin{itemize}
\item {Grp. gram.:v. t.}
\end{itemize}
O mesmo que \textunderscore anojar\textunderscore .
\section{Enôjo}
\begin{itemize}
\item {Grp. gram.:m.}
\end{itemize}
\begin{itemize}
\item {Grp. gram.:M.}
\end{itemize}
Acto ou effeito de enojar.
Enjôo, peculiar ás mulheres grávidas. Cf. Camillo, Brasileira, 359.
\section{Enojoso}
\begin{itemize}
\item {Grp. gram.:adj.}
\end{itemize}
Que enoja.
\section{Enol}
\begin{itemize}
\item {Grp. gram.:m.}
\end{itemize}
\begin{itemize}
\item {Proveniência:(Do gr. \textunderscore oinos\textunderscore )}
\end{itemize}
Vinho, considerado como excipiente medicinal.
\section{Enolato}
\begin{itemize}
\item {Grp. gram.:m.}
\end{itemize}
\begin{itemize}
\item {Proveniência:(Do gr. \textunderscore oinos\textunderscore  + lat. \textunderscore oleum\textunderscore )}
\end{itemize}
Preparado pharmacêutico, em que entra vinho e substâncias medicamentosas.
\section{Enóleo}
\begin{itemize}
\item {Grp. gram.:m.}
\end{itemize}
\begin{itemize}
\item {Proveniência:(Do gr. \textunderscore oinos\textunderscore  + lat. \textunderscore oleum\textunderscore )}
\end{itemize}
Preparado pharmacêutico, em que entra vinho e substâncias medicamentosas.
\section{Enólico}
\begin{itemize}
\item {Grp. gram.:adj.}
\end{itemize}
Relativo a enol ou enóleo.
Que tem o vinho por excipiente.
\section{Enolina}
\begin{itemize}
\item {Grp. gram.:f.}
\end{itemize}
\begin{itemize}
\item {Proveniência:(Do gr. \textunderscore oinos\textunderscore )}
\end{itemize}
Substância còrante do vinho tinto.
\section{Enologia}
\begin{itemize}
\item {Grp. gram.:f.}
\end{itemize}
\begin{itemize}
\item {Proveniência:(Do gr. \textunderscore oinos\textunderscore  + \textunderscore logos\textunderscore )}
\end{itemize}
Tratado á cêrca dos vinhos e da sua preparação.
\section{Enológico}
\begin{itemize}
\item {Grp. gram.:adj.}
\end{itemize}
Relativo a enologia.
\section{Enologista}
\begin{itemize}
\item {Grp. gram.:m.}
\end{itemize}
Aquelle que é versado em enologia.
\section{Enólogo}
\begin{itemize}
\item {Grp. gram.:m.}
\end{itemize}
Aquelle que é versado em enologia.
\section{Enomancia}
\begin{itemize}
\item {Grp. gram.:f.}
\end{itemize}
\begin{itemize}
\item {Proveniência:(Do gr. \textunderscore oinos\textunderscore  + \textunderscore manteia\textunderscore )}
\end{itemize}
Supposta arte de adivinhar, por meio da côr e da substância do vinho.
\section{Enomania}
\begin{itemize}
\item {Grp. gram.:f.}
\end{itemize}
\begin{itemize}
\item {Proveniência:(Do gr. \textunderscore oinos\textunderscore  + \textunderscore mania\textunderscore )}
\end{itemize}
Paixão pelo vinho.
Doença, resultante do abuso do vinho.
\section{Enomaníaco}
\begin{itemize}
\item {Grp. gram.:adj.}
\end{itemize}
\begin{itemize}
\item {Grp. gram.:M.}
\end{itemize}
Relativo á enomania.
Aquelle que soffre enomania.
\section{Enomel}
\begin{itemize}
\item {Grp. gram.:m.}
\end{itemize}
\begin{itemize}
\item {Proveniência:(De \textunderscore eno...\textunderscore  + \textunderscore mel\textunderscore )}
\end{itemize}
Xarope, que tem por base o vinho, e em que o açúcar é substituído pelo mel.
\section{Enometria}
\begin{itemize}
\item {Grp. gram.:f.}
\end{itemize}
Applicação do enómetro.
\section{Enométrico}
\begin{itemize}
\item {Grp. gram.:adj.}
\end{itemize}
Relativo a enometria.
\section{Enómetro}
\begin{itemize}
\item {Grp. gram.:m.}
\end{itemize}
\begin{itemize}
\item {Proveniência:(Do gr. \textunderscore oinos\textunderscore  + \textunderscore metron\textunderscore )}
\end{itemize}
Instrumento, com que se avalia o pêso especifico dos vinhos e, em geral, a riqueza alcoólica de outros líquidos.
\section{Enóphilo}
\begin{itemize}
\item {Grp. gram.:adj.}
\end{itemize}
\begin{itemize}
\item {Proveniência:(Do gr. \textunderscore oinos\textunderscore  + \textunderscore philos\textunderscore )}
\end{itemize}
Que gosta de vinho.
Que se dedica a commércio ou assumptos vinícolas.
\section{Enophobia}
\begin{itemize}
\item {Grp. gram.:f.}
\end{itemize}
\begin{itemize}
\item {Proveniência:(De \textunderscore enóphobo\textunderscore )}
\end{itemize}
Aversão ou horror ao vinho.
\section{Enóphobo}
\begin{itemize}
\item {Grp. gram.:adj.}
\end{itemize}
\begin{itemize}
\item {Proveniência:(Do gr. \textunderscore oinos\textunderscore  + \textunderscore phobos\textunderscore )}
\end{itemize}
Que tem horror ao vinho.
\section{Enóphora}
\begin{itemize}
\item {Grp. gram.:f.}
\end{itemize}
O mesmo que \textunderscore enóphoro\textunderscore .
\section{Enóphoro}
\begin{itemize}
\item {Grp. gram.:m.}
\end{itemize}
\begin{itemize}
\item {Proveniência:(Lat. \textunderscore oenophorum\textunderscore )}
\end{itemize}
Vaso para vinho, entre os Romanos.
\section{Enóplios}
\begin{itemize}
\item {Grp. gram.:m. pl.}
\end{itemize}
\begin{itemize}
\item {Proveniência:(Do gr. \textunderscore enoplos\textunderscore )}
\end{itemize}
Gênero de insectos longicórneos da América do Norte.
\section{Enoque}
\begin{itemize}
\item {Grp. gram.:m.}
\end{itemize}
\begin{itemize}
\item {Utilização:Des.}
\end{itemize}
\begin{itemize}
\item {Utilização:Prov.}
\end{itemize}
\begin{itemize}
\item {Utilização:trasm.}
\end{itemize}
O mesmo que \textunderscore anoque\textunderscore . Cf. \textunderscore Boletim da Socied. de Geogr.\textunderscore , XVI, 168.
Fábrica de curtumes.
\section{Enora}
\begin{itemize}
\item {Grp. gram.:f.}
\end{itemize}
\begin{itemize}
\item {Utilização:Náut.}
\end{itemize}
\begin{itemize}
\item {Proveniência:(Do lat. \textunderscore ora\textunderscore )}
\end{itemize}
Abertura, por onde os mastros vão assentar na carlinga.
Peça de madeira, com que se atocha o mastro.
\section{Enorme}
\begin{itemize}
\item {Grp. gram.:adj.}
\end{itemize}
\begin{itemize}
\item {Proveniência:(Lat. \textunderscore enormis\textunderscore )}
\end{itemize}
Desviado da norma.
Que sái da regra.
Desmarcado.
Muito grande; extraordinário: \textunderscore enorme quantidade de gente\textunderscore .
\section{Enorme}
\begin{itemize}
\item {Grp. gram.:adj.}
\end{itemize}
\begin{itemize}
\item {Utilização:Prov.}
\end{itemize}
\begin{itemize}
\item {Utilização:dur.}
\end{itemize}
Inoffensivo.
Simples; simplório.
(Colhido em Penafiel)
\section{Enormemente}
\begin{itemize}
\item {Grp. gram.:adv.}
\end{itemize}
\begin{itemize}
\item {Proveniência:(De \textunderscore enorme\textunderscore )}
\end{itemize}
Excessivamente.
\section{Enormidade}
\begin{itemize}
\item {Grp. gram.:f.}
\end{itemize}
\begin{itemize}
\item {Proveniência:(Lat. \textunderscore enormitas\textunderscore )}
\end{itemize}
Irregularidade.
Desproporção na grandeza.
Qualidade daquillo que é enorme.
\section{Enostose}
\begin{itemize}
\item {Grp. gram.:f.}
\end{itemize}
\begin{itemize}
\item {Utilização:Med.}
\end{itemize}
\begin{itemize}
\item {Proveniência:(Do gr. \textunderscore en\textunderscore  + \textunderscore osteon\textunderscore )}
\end{itemize}
Tumor, desenvolvido na medulla de um osso.
\section{Enotermo}
\begin{itemize}
\item {Grp. gram.:m.}
\end{itemize}
\begin{itemize}
\item {Proveniência:(Do gr. \textunderscore oinos\textunderscore  + \textunderscore therme\textunderscore )}
\end{itemize}
Apparelho, para aquecimento dos vinhos.
\section{Enothermo}
\begin{itemize}
\item {Grp. gram.:m.}
\end{itemize}
\begin{itemize}
\item {Proveniência:(Do gr. \textunderscore oinos\textunderscore  + \textunderscore therme\textunderscore )}
\end{itemize}
Apparelho, para aquecimento dos vinhos.
\section{Enouriçar}
\begin{itemize}
\item {Grp. gram.:v. t.}
\end{itemize}
\begin{itemize}
\item {Utilização:Des.}
\end{itemize}
O mesmo que \textunderscore ouriçar\textunderscore .
\section{Enquadernar}
\begin{itemize}
\item {Grp. gram.:v. t.}
\end{itemize}
O mesmo que \textunderscore encadernar\textunderscore .
\section{Enquadramento}
\begin{itemize}
\item {Grp. gram.:m.}
\end{itemize}
Acto ou effeito de enquadrar.
\section{Enquadrar}
\begin{itemize}
\item {Grp. gram.:v. t.}
\end{itemize}
Pôr em quadro.
Emmoldurar; encaixilhar.
Tornar quadrado: \textunderscore enquadrar cortiça\textunderscore . Cf. \textunderscore Inquér. Industr.\textunderscore , p. II, l. III, 26.
\section{Enquadrilhar}
\begin{itemize}
\item {Grp. gram.:v. t.}
\end{itemize}
\begin{itemize}
\item {Utilização:Bras. do S}
\end{itemize}
\begin{itemize}
\item {Grp. gram.:V. p.}
\end{itemize}
\begin{itemize}
\item {Proveniência:(De \textunderscore quadrilha\textunderscore )}
\end{itemize}
Reunir (muitos cavallos).
Reunir-se (muita gente).
\section{Enquanto}
\begin{itemize}
\item {Grp. gram.:conj.}
\end{itemize}
\begin{itemize}
\item {Proveniência:(De \textunderscore em\textunderscore  + \textunderscore quanto\textunderscore )}
\end{itemize}
No tempo em que.
Ao passo que.
\section{Enquartar}
\begin{itemize}
\item {Grp. gram.:v. t.}
\end{itemize}
\begin{itemize}
\item {Utilização:Prov.}
\end{itemize}
\begin{itemize}
\item {Utilização:alent.}
\end{itemize}
Dividir em quartos, esquartejar.
\section{Enque}
\begin{itemize}
\item {Grp. gram.:m.}
\end{itemize}
Cabo de embarcação, para reforçar o estai do traquete.
\section{Enqueijar}
\begin{itemize}
\item {Grp. gram.:v. t.}
\end{itemize}
Coalhar, preparar, para converter em queijo: \textunderscore enqueijar leite\textunderscore .
\section{Enquerida}
\begin{itemize}
\item {Grp. gram.:f.}
\end{itemize}
\begin{itemize}
\item {Utilização:Prov.}
\end{itemize}
\begin{itemize}
\item {Utilização:alent.}
\end{itemize}
Acto de enquerir.
Cada um dos dois sacos ou feixes que, ligados por cordame, se carregam na cavalgadura, ficando um a um lado e outro a outro, para equilíbrio da carga.
\section{Enquerir}
\begin{itemize}
\item {Grp. gram.:v. t.}
\end{itemize}
\begin{itemize}
\item {Utilização:Prov.}
\end{itemize}
Dispor (a carga) em dois lotes ou volumes, correspondentes aos dois lados, esquerdo e direito, da cavalgadura.
(Relaciona-se com \textunderscore enque\textunderscore , cabo, por sêr indispensável a corda na formação da enquerida?)
\section{Enquilhar}
\begin{itemize}
\item {Grp. gram.:v. t.}
\end{itemize}
Pregar a quilha em.
\section{Enquimose}
\begin{itemize}
\item {Grp. gram.:f.}
\end{itemize}
O mesmo que \textunderscore equimose\textunderscore .
\section{Enquirídio}
\begin{itemize}
\item {Grp. gram.:m.}
\end{itemize}
\begin{itemize}
\item {Proveniência:(Gr. \textunderscore enkheiridion\textunderscore )}
\end{itemize}
Manual ou epítome, organizado por autor antigo.
\section{Enquisa}
\begin{itemize}
\item {Grp. gram.:f.}
\end{itemize}
Investigação, inquirição. Cf. Herculano, \textunderscore Hist. de Port.\textunderscore , IV, 360 e 361.
\section{Enquisilar}
\begin{itemize}
\item {Grp. gram.:v. t.}
\end{itemize}
O mesmo que \textunderscore quezilar\textunderscore . Cf. Cast., \textunderscore Sabichonas\textunderscore , 74.
\section{Enrabadoiro}
\begin{itemize}
\item {Grp. gram.:m.}
\end{itemize}
\begin{itemize}
\item {Utilização:Prov.}
\end{itemize}
\begin{itemize}
\item {Proveniência:(De \textunderscore enrabar\textunderscore )}
\end{itemize}
Lugar, onde está suspensa por um eixo a vara do lagar de vinho.
\section{Enrabadouro}
\begin{itemize}
\item {Grp. gram.:m.}
\end{itemize}
\begin{itemize}
\item {Utilização:Prov.}
\end{itemize}
\begin{itemize}
\item {Proveniência:(De \textunderscore enrabar\textunderscore )}
\end{itemize}
Lugar, onde está suspensa por um eixo a vara do lagar de vinho.
\section{Enrabar}
\begin{itemize}
\item {Grp. gram.:v. t.}
\end{itemize}
\begin{itemize}
\item {Utilização:Bras. do S}
\end{itemize}
Segurar pelo rabo.
\section{Enrabeirar}
\begin{itemize}
\item {Grp. gram.:v. t.}
\end{itemize}
\begin{itemize}
\item {Proveniência:(De \textunderscore rabeira\textunderscore )}
\end{itemize}
Sujar a cauda de (um vestido).
Enlamear a parte inferior de (vestido).
\section{Enrabichado}
\begin{itemize}
\item {Grp. gram.:adj.}
\end{itemize}
\begin{itemize}
\item {Proveniência:(De \textunderscore enrabichar\textunderscore )}
\end{itemize}
Enamorado, apaixonado.
\section{Enrabichar}
\begin{itemize}
\item {Grp. gram.:v. t.}
\end{itemize}
\begin{itemize}
\item {Utilização:Pop.}
\end{itemize}
\begin{itemize}
\item {Grp. gram.:V. p.}
\end{itemize}
Dar fórma de rabicho a (o cabello).
Meter em difficuldades, encalacrar.
Apaixonar-se amorosamente:«\textunderscore não sabiam por que carga de água o brasileiro se enrabichara com aquella trapalhona\textunderscore ». Camillo, \textunderscore Corja\textunderscore , 262.
\section{Enrabitar}
\begin{itemize}
\item {Grp. gram.:v. t.}
\end{itemize}
\begin{itemize}
\item {Utilização:T. da Bairrada}
\end{itemize}
O mesmo que \textunderscore arrebitar\textunderscore .
\section{Enradicado}
\begin{itemize}
\item {Grp. gram.:adj.}
\end{itemize}
O mesmo que \textunderscore arraigado\textunderscore .
\section{Enraiar}
\begin{itemize}
\item {Grp. gram.:v. t.}
\end{itemize}
\begin{itemize}
\item {Utilização:Gal}
\end{itemize}
\begin{itemize}
\item {Proveniência:(Fr. \textunderscore enrayer\textunderscore )}
\end{itemize}
Pôr os raios a (uma roda).
Travar (rodas).
\section{Enraivar}
\begin{itemize}
\item {Grp. gram.:v. t. ,  i.  e  p.}
\end{itemize}
O mesmo que \textunderscore enraivecer\textunderscore .
\section{Enraivecer}
\begin{itemize}
\item {Grp. gram.:v. i.}
\end{itemize}
\begin{itemize}
\item {Grp. gram.:V. i.  e  p.}
\end{itemize}
Causar raiva a.
Tornar-se raivoso; possuir-se de cólera.
\section{Enraizar}
\begin{itemize}
\item {fónica:ra-i}
\end{itemize}
\begin{itemize}
\item {Grp. gram.:v. t. ,  i.  e  p.}
\end{itemize}
O mesmo que \textunderscore arraigar\textunderscore .
\section{Enramada}
\begin{itemize}
\item {Grp. gram.:f.}
\end{itemize}
\begin{itemize}
\item {Proveniência:(De \textunderscore enramar\textunderscore )}
\end{itemize}
Ornato ou cobertura de ramos; ramada.
\section{Enramalhar}
\begin{itemize}
\item {Grp. gram.:v. t.}
\end{itemize}
Ornar com ramos.
\section{Enramalhetar}
\begin{itemize}
\item {Grp. gram.:v. t.}
\end{itemize}
\begin{itemize}
\item {Utilização:Ext.}
\end{itemize}
Juntar em ramalhete.
Adornar com ramalhetes.
Enflorar, adornar.
\section{Enramamento}
\begin{itemize}
\item {Grp. gram.:m.}
\end{itemize}
Acto de enramar.
\section{Enramar}
\begin{itemize}
\item {Grp. gram.:v. t.}
\end{itemize}
Adornar ou cobrir com ramos.
Juntar em ramo; enramalhetar.
\section{Enramilhetar}
\begin{itemize}
\item {Grp. gram.:v. t.}
\end{itemize}
O mesmo que \textunderscore enramalhetar\textunderscore :«\textunderscore enramilhetei flores para quem as quisesse\textunderscore ». Castilho, nota aos \textunderscore Amores de Ovídio\textunderscore .
\section{Enrançar}
\begin{itemize}
\item {Grp. gram.:v. t.}
\end{itemize}
\begin{itemize}
\item {Utilização:Des.}
\end{itemize}
\begin{itemize}
\item {Grp. gram.:V. i.  e  p.}
\end{itemize}
\begin{itemize}
\item {Proveniência:(De \textunderscore ranço\textunderscore )}
\end{itemize}
Tornar rançoso.
Tornar-se râncido.
\section{Enranchar}
\begin{itemize}
\item {Grp. gram.:v. t.}
\end{itemize}
\begin{itemize}
\item {Grp. gram.:V. p.}
\end{itemize}
Juntar ao rancho.
Abandear.
Agrupar-se, juntar-se ao rancho.
\section{Enrarecer}
\begin{itemize}
\item {Grp. gram.:v. t.}
\end{itemize}
\begin{itemize}
\item {Grp. gram.:V. i.}
\end{itemize}
Tornar raro.
Tornar-se raro.
\section{Enrascadela}
\begin{itemize}
\item {Grp. gram.:f.}
\end{itemize}
\begin{itemize}
\item {Utilização:Pop.}
\end{itemize}
\begin{itemize}
\item {Proveniência:(De \textunderscore enrascar\textunderscore )}
\end{itemize}
Atrapalhação; arriosca.
\section{Enrascadura}
\begin{itemize}
\item {Grp. gram.:f.}
\end{itemize}
Acto ou effeito de enrascar.
\section{Enrascar}
\begin{itemize}
\item {Grp. gram.:v. t.}
\end{itemize}
\begin{itemize}
\item {Utilização:Pop.}
\end{itemize}
\begin{itemize}
\item {Proveniência:(De \textunderscore rasca\textunderscore )}
\end{itemize}
Enredar, embaraçar (cabos, velas ou bandeiras).
Lograr, enganar.
Fazer cair em cilada.
Criar difficuldades a (alguém).
\section{Enreda}
\begin{itemize}
\item {Grp. gram.:m.  e  adj.}
\end{itemize}
\begin{itemize}
\item {Utilização:Fam.}
\end{itemize}
O mesmo que \textunderscore enredador\textunderscore .
\section{Enredadeira}
\begin{itemize}
\item {Grp. gram.:f.}
\end{itemize}
Mulher, que faz enredos.
Intriguista. Cf. Castilho, \textunderscore Misant.\textunderscore , 105.
\section{Enredadeiro}
\begin{itemize}
\item {Grp. gram.:adj.}
\end{itemize}
Que faz enredos; enredador. Cf. Pato, \textunderscore Ciprestes,\textunderscore  173.
\section{Enredadela}
\begin{itemize}
\item {Grp. gram.:f.}
\end{itemize}
\begin{itemize}
\item {Utilização:Fam.}
\end{itemize}
Enrêdo, intriga.
\section{Enredador}
\begin{itemize}
\item {Grp. gram.:adj.}
\end{itemize}
\begin{itemize}
\item {Grp. gram.:M.}
\end{itemize}
Que enreda.
Aquelle que enreda.
\section{Enredamento}
\begin{itemize}
\item {Grp. gram.:m.}
\end{itemize}
Acto ou effeito de enredar.
\section{Enredar}
\begin{itemize}
\item {Grp. gram.:v. t.}
\end{itemize}
\begin{itemize}
\item {Utilização:Fig.}
\end{itemize}
\begin{itemize}
\item {Proveniência:(De \textunderscore rêde\textunderscore )}
\end{itemize}
Colher na rêde.
Emmaranhar, confundir: \textunderscore enredar questões\textunderscore .
Intrigar.
Misturar, entretecer.
Ligar, travar (as partes de uma narrativa).
Formar ou dispor o enrêdo de.
\section{Enredear}
\begin{itemize}
\item {Grp. gram.:v. t.}
\end{itemize}
O mesmo que \textunderscore enredar\textunderscore .
\section{Enrediça}
\begin{itemize}
\item {Grp. gram.:f.}
\end{itemize}
\begin{itemize}
\item {Proveniência:(De \textunderscore enredar\textunderscore )}
\end{itemize}
Designação genérica das plantas trepadeiras ou sarmentosas.
\section{Enrediço}
\begin{itemize}
\item {Grp. gram.:adj.}
\end{itemize}
Que costuma fazer enredos. Cf. Filinto, XVI, 184.
\section{Enrêdo}
\begin{itemize}
\item {Grp. gram.:m.}
\end{itemize}
\begin{itemize}
\item {Utilização:Fig.}
\end{itemize}
\begin{itemize}
\item {Utilização:Prov.}
\end{itemize}
\begin{itemize}
\item {Utilização:trasm.}
\end{itemize}
Acto ou effeito de enredar.
Intriga.
Ardil.
Mexerico.
Mentira, que produz inimizades.
Confusão.
Entrecho: \textunderscore o enrêdo de um drama\textunderscore .
Mau trabalhador, que estorva os outros e não faz quási nada.
\section{Enredoiçar}
\begin{itemize}
\item {Grp. gram.:v. t.}
\end{itemize}
\begin{itemize}
\item {Proveniência:(De \textunderscore redoiçar\textunderscore )}
\end{itemize}
Embalançar na redoiça.
\section{Enredoso}
\begin{itemize}
\item {Grp. gram.:adj.}
\end{itemize}
Que enreda.
Em que há enredos.
\section{Encenação}
\begin{itemize}
\item {Grp. gram.:f.}
\end{itemize}
\begin{itemize}
\item {Utilização:Neol.}
\end{itemize}
Acto ou efeito de encenar.--Propôs-se êste t., em substituição do francesismo \textunderscore mis-en-scène\textunderscore .
\section{Encenar}
\begin{itemize}
\item {Grp. gram.:v. t.}
\end{itemize}
\begin{itemize}
\item {Utilização:Neol.}
\end{itemize}
Pôr em cena.
Fazer representar no teatro.
\section{Enregar}
\begin{itemize}
\item {Grp. gram.:v. i.}
\end{itemize}
\begin{itemize}
\item {Utilização:Prov.}
\end{itemize}
\begin{itemize}
\item {Grp. gram.:V. t.}
\end{itemize}
\begin{itemize}
\item {Proveniência:(De \textunderscore rêgo\textunderscore )}
\end{itemize}
Começar qualquer trabalho: \textunderscore temos de enregar amanhan\textunderscore .
Abrir (a terra) em regos.
\section{Enregelado}
\begin{itemize}
\item {Grp. gram.:adj.}
\end{itemize}
\begin{itemize}
\item {Proveniência:(De \textunderscore enregelar\textunderscore )}
\end{itemize}
Muito frio; resfriado; congelado.
\section{Enregelamento}
\begin{itemize}
\item {Grp. gram.:m.}
\end{itemize}
Acto ou effeito de enregelar.
\section{Enregelar}
\begin{itemize}
\item {Grp. gram.:v. t.}
\end{itemize}
\begin{itemize}
\item {Utilização:Fig.}
\end{itemize}
\begin{itemize}
\item {Grp. gram.:V. i.}
\end{itemize}
\begin{itemize}
\item {Proveniência:(De \textunderscore regelar\textunderscore )}
\end{itemize}
Regelar.
Tornar muito frio.
Desanimar, desalentar.
Assustar muito.
Tornar-se gelado, muito frio.
\section{Enreixar}
\begin{itemize}
\item {Grp. gram.:v. i.}
\end{itemize}
Andar de reixa; inimizar-se.
\section{Enrelhado}
\begin{itemize}
\item {Grp. gram.:adj.}
\end{itemize}
\begin{itemize}
\item {Utilização:Prov.}
\end{itemize}
\begin{itemize}
\item {Utilização:trasm.}
\end{itemize}
Que coxeia, que não póde andar bem, por causa de doença suspeita.
\section{Enrelhar}
\begin{itemize}
\item {Grp. gram.:v. t.}
\end{itemize}
\begin{itemize}
\item {Utilização:Prov.}
\end{itemize}
\begin{itemize}
\item {Utilização:trasm.}
\end{itemize}
\begin{itemize}
\item {Proveniência:(De \textunderscore relha\textunderscore )}
\end{itemize}
Ferir com a relha (os bois, na lavra).
\section{Enrelheirar}
\begin{itemize}
\item {Grp. gram.:v. t.}
\end{itemize}
\begin{itemize}
\item {Utilização:Prov.}
\end{itemize}
\begin{itemize}
\item {Utilização:trasm.}
\end{itemize}
Dispor (cereaes) em relheiro.
\section{Enrelvar}
\begin{itemize}
\item {Grp. gram.:v. t.}
\end{itemize}
Cobrir de relva, arrelvar.
\section{Enremelado}
\begin{itemize}
\item {Grp. gram.:adj.}
\end{itemize}
Que tem remela. Cf. Filinto, V, 109.
\section{Enremissar}
\begin{itemize}
\item {Grp. gram.:v. t.}
\end{itemize}
Deixar de remissa.
Demorar com remissas (o jôgo do voltarete).
\section{Enresinagem}
\begin{itemize}
\item {Grp. gram.:f.}
\end{itemize}
Acto de enresinar. Cf. \textunderscore Techn. Rur.\textunderscore , 286.
\section{Enresinar}
\begin{itemize}
\item {Grp. gram.:v. t.}
\end{itemize}
\begin{itemize}
\item {Grp. gram.:V. i.  e  p.}
\end{itemize}
Untar com resina.
Misturar com resina.
Tornar consistente como a resina, endurecer.
Tomar a consistência da resina; tornar-se duro: \textunderscore o pão enresinou\textunderscore .
\section{Enresmamento}
\begin{itemize}
\item {Grp. gram.:m.}
\end{itemize}
Acto ou effeito de enresmar.
\section{Enresmar}
\begin{itemize}
\item {Grp. gram.:v. t.}
\end{itemize}
Dispor (papel) em resmas.
\section{Enrestiar}
\begin{itemize}
\item {Grp. gram.:v. t.}
\end{itemize}
Pôr em réstia (cebolas, alhos, etc.).
\section{Enrevesar}
\begin{itemize}
\item {Grp. gram.:v. t.}
\end{itemize}
Pôr de revés.
Confundir:«\textunderscore pescou o sentido enrevesado\textunderscore ». Filinto. Cf. Garrett, \textunderscore Camões\textunderscore , 120.
\section{Enriar}
\begin{itemize}
\item {Grp. gram.:v. t.}
\end{itemize}
\begin{itemize}
\item {Utilização:Prov.}
\end{itemize}
\begin{itemize}
\item {Utilização:beir.}
\end{itemize}
\begin{itemize}
\item {Proveniência:(De \textunderscore rio\textunderscore )}
\end{itemize}
Meter na água do rio (o linho para se curtir).
\section{Enriçado}
\begin{itemize}
\item {Grp. gram.:adj.}
\end{itemize}
\begin{itemize}
\item {Utilização:Prov.}
\end{itemize}
\begin{itemize}
\item {Utilização:trasm.}
\end{itemize}
\begin{itemize}
\item {Utilização:Prov.}
\end{itemize}
\begin{itemize}
\item {Proveniência:(De \textunderscore enriçar\textunderscore )}
\end{itemize}
Enfrenesiado, encarniçado, pertinaz.
Desordenado, emmaranhado: \textunderscore traz o cabello enriçado\textunderscore .
\section{Enricar}
\begin{itemize}
\item {Grp. gram.:v. t. ,  i.  e  p.}
\end{itemize}
(V.enriquecer)
\section{Enriçar}
\begin{itemize}
\item {Grp. gram.:v. t.}
\end{itemize}
O mesmo que \textunderscore riçar\textunderscore , \textunderscore emmaranhar\textunderscore .
\section{Enriço}
\begin{itemize}
\item {Grp. gram.:m.}
\end{itemize}
Coisa enriçada. Cf. Castilho, \textunderscore Geórg.\textunderscore , 191.
\section{Enrijamento}
\begin{itemize}
\item {Grp. gram.:m.}
\end{itemize}
Acto ou effeito de enrijar.
\section{Enrijar}
\begin{itemize}
\item {Grp. gram.:v. t.}
\end{itemize}
\begin{itemize}
\item {Grp. gram.:V. i.}
\end{itemize}
Tornar rijo.
Fazer-se rijo.
Robustecer-se.
Readquirir saúde: \textunderscore você enrijou\textunderscore .
\section{Enrijecer}
\begin{itemize}
\item {Grp. gram.:v. t.  e  i.}
\end{itemize}
(V.enrijar)
\section{Enrila}
\begin{itemize}
\item {Grp. gram.:f.}
\end{itemize}
Planta trepadeira das Filippinas.
\section{Enrilhar}
\begin{itemize}
\item {Grp. gram.:v. t.}
\end{itemize}
\begin{itemize}
\item {Utilização:Prov.}
\end{itemize}
O mesmo que \textunderscore enrijar\textunderscore  (a carne).
Destemperar (o ventre).
\section{Enrilheirar}
\begin{itemize}
\item {Grp. gram.:v. t.}
\end{itemize}
(V.enrelheirar)
\section{Enrimar}
\begin{itemize}
\item {Grp. gram.:v. t.}
\end{itemize}
\begin{itemize}
\item {Utilização:Prov.}
\end{itemize}
\begin{itemize}
\item {Utilização:minh.}
\end{itemize}
\begin{itemize}
\item {Proveniência:(De \textunderscore rima\textunderscore ^3)}
\end{itemize}
Pôr em rimas, amontoar.
\section{Enripar}
\begin{itemize}
\item {Grp. gram.:v. t.}
\end{itemize}
\begin{itemize}
\item {Utilização:Bras}
\end{itemize}
Pregar as ripas sôbre os caibros de (um prédio).
\section{Enrique}
\begin{itemize}
\item {Grp. gram.:m.}
\end{itemize}
\begin{itemize}
\item {Proveniência:(De \textunderscore Enrique\textunderscore , n. p.)}
\end{itemize}
Antiga moéda espanhola.
\section{Enriquecer}
\begin{itemize}
\item {Grp. gram.:v. t.}
\end{itemize}
\begin{itemize}
\item {Utilização:Fig.}
\end{itemize}
\begin{itemize}
\item {Grp. gram.:V. i.}
\end{itemize}
Fazer rico: \textunderscore o trabalho enriqueceu-te\textunderscore .
Melhorar.
Brindar, dotar: \textunderscore a natureza enriqueceu-a de virtudes\textunderscore .
Engrandecer.
Ornar.
Tornar-se rico, abundante.
\section{Enriquecimento}
\begin{itemize}
\item {Grp. gram.:m.}
\end{itemize}
Acto ou effeito de enriquecer.
\section{Enriquentar}
\begin{itemize}
\item {Grp. gram.:v. t.}
\end{itemize}
\begin{itemize}
\item {Utilização:Ant.}
\end{itemize}
Tornar rico, encher de riquezas.
\section{Enristar}
\begin{itemize}
\item {Grp. gram.:v. t.}
\end{itemize}
\begin{itemize}
\item {Grp. gram.:V. i.}
\end{itemize}
\begin{itemize}
\item {Proveniência:(De \textunderscore riste\textunderscore )}
\end{itemize}
Pôr em riste (a lança).
Preparar-se para acommeter alguém.
\section{Enriste}
\begin{itemize}
\item {Grp. gram.:m.}
\end{itemize}
Acto de enristar.
\section{Enrizamento}
\begin{itemize}
\item {Grp. gram.:m.}
\end{itemize}
Acto ou effeito de enrizar.
\section{Enrizar}
\begin{itemize}
\item {Grp. gram.:v.}
\end{itemize}
\begin{itemize}
\item {Utilização:t. Náut.}
\end{itemize}
Meter nos rizes.
\section{Enrobustecer}
\textunderscore v. t.\textunderscore , \textunderscore i.\textunderscore  e \textunderscore p.\textunderscore  (e der.)
O mesmo que \textunderscore robustecer\textunderscore , etc.
\section{Enrocado}
\begin{itemize}
\item {Grp. gram.:adj.}
\end{itemize}
\begin{itemize}
\item {Proveniência:(De \textunderscore roca\textunderscore ^2)}
\end{itemize}
Coberto de penhascos.
\section{Enrocamento}
\begin{itemize}
\item {Grp. gram.:m.}
\end{itemize}
\begin{itemize}
\item {Proveniência:(De \textunderscore roca\textunderscore ^2)}
\end{itemize}
Conjunto de grandes pedras tôscas, que servem de alicerces nas obras hydráulicas.
\section{Enrocar}
\begin{itemize}
\item {Grp. gram.:v. t.}
\end{itemize}
\begin{itemize}
\item {Utilização:Náut.}
\end{itemize}
\begin{itemize}
\item {Proveniência:(De \textunderscore roca\textunderscore ^1)}
\end{itemize}
Dar fórma de roca a.
Fazer prégas em, encanudar.
Rodear e segurar com talas (um mastro rendido).
\section{Enrodelar}
\begin{itemize}
\item {Grp. gram.:v. t.}
\end{itemize}
\begin{itemize}
\item {Grp. gram.:V. p.}
\end{itemize}
Abroquelar ou armar com dela.
O mesmo que [[encaracolar-se|encaracolar]]. Cf. Júl. Castilho, \textunderscore Ermitério\textunderscore , 127.
\section{Enrodilha}
\begin{itemize}
\item {Grp. gram.:f.}
\end{itemize}
\begin{itemize}
\item {Utilização:Prov.}
\end{itemize}
\begin{itemize}
\item {Utilização:beir.}
\end{itemize}
\begin{itemize}
\item {Proveniência:(De \textunderscore enrodilhar\textunderscore )}
\end{itemize}
O mesmo que \textunderscore enrêdo\textunderscore .
\section{Enrodilhar}
\begin{itemize}
\item {Grp. gram.:v. t.}
\end{itemize}
Torcer.
Embrulhar, Enredar.
Entalar.
Dar de rodilha a: \textunderscore enrodilhar um lenço\textunderscore .
\section{Enrodrigar}
\begin{itemize}
\item {Grp. gram.:v. t.}
\end{itemize}
\begin{itemize}
\item {Utilização:Prov.}
\end{itemize}
\begin{itemize}
\item {Utilização:trasm.}
\end{itemize}
\begin{itemize}
\item {Proveniência:(De \textunderscore rodriga\textunderscore )}
\end{itemize}
Pôr espeques ou estacas a (videiras, feijoeiros, etc.).
\section{Enroladamente}
\begin{itemize}
\item {Grp. gram.:adv.}
\end{itemize}
\begin{itemize}
\item {Utilização:Ant.}
\end{itemize}
\begin{itemize}
\item {Proveniência:(De \textunderscore enrolar\textunderscore )}
\end{itemize}
Ás escondidas; pela calada.
\section{Enroladeira}
\begin{itemize}
\item {Grp. gram.:f.}
\end{itemize}
Maquínismo, nas fábricas de tecidos, para enrolar os productos da tecelagem.
\section{Enroladoiro}
\begin{itemize}
\item {Grp. gram.:m.}
\end{itemize}
Caroço de novelo, ou qualquer coisa, em que se enrola o fio para formar o novelo.
\section{Enroladouro}
\begin{itemize}
\item {Grp. gram.:m.}
\end{itemize}
Caroço de novelo, ou qualquer coisa, em que se enrola o fio para formar o novelo.
\section{Enrolamento}
\begin{itemize}
\item {Grp. gram.:m.}
\end{itemize}
Acto de enrolar.
\section{Enrolar}
\begin{itemize}
\item {Grp. gram.:v. t.}
\end{itemize}
\begin{itemize}
\item {Utilização:Fig.}
\end{itemize}
\begin{itemize}
\item {Grp. gram.:V. p.}
\end{itemize}
\begin{itemize}
\item {Proveniência:(De \textunderscore rôlo\textunderscore )}
\end{itemize}
Tornar roliço.
Dobrar, fazendo rolo ou espiral: \textunderscore enrolar papéis\textunderscore .
Contornar em espiral.
Embrulhar.
Esconder.
Mover-se em rôlos; encapelar-se (o mar).
\section{Enrôlár}
\begin{itemize}
\item {Grp. gram.:v. t.}
\end{itemize}
\begin{itemize}
\item {Utilização:Prov.}
\end{itemize}
\begin{itemize}
\item {Utilização:trasm.}
\end{itemize}
\begin{itemize}
\item {Proveniência:(De \textunderscore rôla\textunderscore )}
\end{itemize}
Afagar (crianças).
\section{Enrolhar}
\begin{itemize}
\item {Grp. gram.:v. t.}
\end{itemize}
O mesmo que \textunderscore arrolhar\textunderscore .
\section{Enroquetado}
\begin{itemize}
\item {Grp. gram.:adj.}
\end{itemize}
Que tem roquete: \textunderscore sobrepeliz enroquetada\textunderscore .
\section{Enroscadura}
\begin{itemize}
\item {Grp. gram.:f.}
\end{itemize}
Acto ou effeito de enroscar.
\section{Enroscamento}
\begin{itemize}
\item {Grp. gram.:m.}
\end{itemize}
Acto ou effeito de enroscar.
\section{Enroscar}
\begin{itemize}
\item {Grp. gram.:v. t.}
\end{itemize}
\begin{itemize}
\item {Grp. gram.:V. p.}
\end{itemize}
\begin{itemize}
\item {Proveniência:(De \textunderscore rosca\textunderscore )}
\end{itemize}
Mover em fórma de rosca.
Dobrar formando roscas.
Mover-se em espiral.
Formar rosca ou rôlo.
Dobrar-se, encolher-se: \textunderscore o gato enroscou-se no chão\textunderscore .
\section{Enrostar}
\begin{itemize}
\item {Grp. gram.:v. t.}
\end{itemize}
\begin{itemize}
\item {Utilização:Ant.}
\end{itemize}
Lançar em rosto de alguém.
\section{Enrotar}
\begin{itemize}
\item {Grp. gram.:v. t.}
\end{itemize}
\begin{itemize}
\item {Utilização:T. de Lisbôa}
\end{itemize}
Pôr a rota ou palhinha em (cadeiras, canapés). Cf. Th. Ribeiro, \textunderscore Jornadas\textunderscore , II, 208.
\section{Enroupar}
\begin{itemize}
\item {Grp. gram.:v. t.}
\end{itemize}
Cobrir de roupa.
Fazer roupa a.
Agasalhar.
\section{Enrouquecer}
\begin{itemize}
\item {Grp. gram.:v. t.}
\end{itemize}
\begin{itemize}
\item {Grp. gram.:V. i.}
\end{itemize}
Fazer rouco.
Tornar-se rouco.
\section{Enrouquecimento}
\begin{itemize}
\item {Grp. gram.:m.}
\end{itemize}
Acto ou effeito de enrouquecer.
\section{Enroxar-se}
\begin{itemize}
\item {Grp. gram.:v. p.}
\end{itemize}
Tornar-se roxo.
\section{Enrubecer}
\textunderscore v. t.\textunderscore  e \textunderscore i.\textunderscore  (e der.)
O mesmo que \textunderscore enrubescer\textunderscore .
\section{Enrubescer}
\begin{itemize}
\item {Grp. gram.:v. t.}
\end{itemize}
\begin{itemize}
\item {Grp. gram.:V. i.}
\end{itemize}
\begin{itemize}
\item {Proveniência:(Do lat. \textunderscore rubescere\textunderscore )}
\end{itemize}
Tornar vermelho, còrado.
Tornar-se vermelho, còrar.
\section{Enruçar}
\begin{itemize}
\item {Grp. gram.:v. t.}
\end{itemize}
\begin{itemize}
\item {Grp. gram.:V. i.}
\end{itemize}
Tornar ruço.
Tornar-se ruço.
\section{Enrudecer}
\begin{itemize}
\item {Grp. gram.:v. t.}
\end{itemize}
\begin{itemize}
\item {Grp. gram.:V. i.}
\end{itemize}
Tornar rude.
Tornar-se rude.
\section{Enrufar-se}
\begin{itemize}
\item {Grp. gram.:v. p.}
\end{itemize}
O mesmo que [[arrufar-se|arrufar]]. Cf. Macedo, \textunderscore Burros\textunderscore , 376.
\section{Enrugador}
\begin{itemize}
\item {Grp. gram.:adj.}
\end{itemize}
Que enruga ou encrespa.
\section{Enrugamento}
\begin{itemize}
\item {Grp. gram.:m.}
\end{itemize}
Acto ou effeito de enrugar.
\section{Enrugar}
\begin{itemize}
\item {Grp. gram.:v. t.}
\end{itemize}
Fazer rugas em.
Encrespar: \textunderscore enrugar as ondas\textunderscore .
Encarquilhar: \textunderscore enrugar a fruta\textunderscore .
\section{Enruminar-se}
\begin{itemize}
\item {Grp. gram.:v. p.}
\end{itemize}
\begin{itemize}
\item {Utilização:Prov.}
\end{itemize}
\begin{itemize}
\item {Utilização:trasm.}
\end{itemize}
Aprumar-se com prosápia.
Pôr-se em evidência.
\section{Ensabanado}
\begin{itemize}
\item {Grp. gram.:adj.}
\end{itemize}
\begin{itemize}
\item {Proveniência:(Do cast. \textunderscore sábana\textunderscore , lençol)}
\end{itemize}
Diz-se do toiro que tem o pêlo todo branco.
\section{Ensaboadela}
\begin{itemize}
\item {Grp. gram.:f.}
\end{itemize}
\begin{itemize}
\item {Utilização:Fam.}
\end{itemize}
Acto ou effeito de ensaboar.
Reprehensão.
Noções rudimentares.
Acquisição de ligeiros conhecimentos.
\section{Ensaboado}
\begin{itemize}
\item {Grp. gram.:m.}
\end{itemize}
\begin{itemize}
\item {Proveniência:(De \textunderscore ensaboar\textunderscore )}
\end{itemize}
Lavagem de roupa, com sabão: \textunderscore hoje cá em casa, é dia de ensaboado\textunderscore .
\section{Ensaboadura}
\begin{itemize}
\item {Grp. gram.:f.}
\end{itemize}
Roupa ensaboada de uma vez.
Água, em que há sabão desfeito.
Acto de ensaboar.
\section{Ensaboamento}
\begin{itemize}
\item {Grp. gram.:m.}
\end{itemize}
\begin{itemize}
\item {Utilização:Prov.}
\end{itemize}
\begin{itemize}
\item {Utilização:minh.}
\end{itemize}
O mesmo que \textunderscore ensaboadela\textunderscore .
Veio de barro nas minas, pelo qual o terreno facilmente se fende, podendo resultar desabamento de terra.
\section{Ensaboar}
\begin{itemize}
\item {Grp. gram.:v. t.}
\end{itemize}
\begin{itemize}
\item {Utilização:Fig.}
\end{itemize}
\begin{itemize}
\item {Grp. gram.:Loc.}
\end{itemize}
\begin{itemize}
\item {Utilização:fam.}
\end{itemize}
\begin{itemize}
\item {Proveniência:(De \textunderscore sabão\textunderscore )}
\end{itemize}
Untar ou lavar com sabão desfeito em água.
Reprehender, castigar.
\textunderscore Ensaboar o juízo a\textunderscore , apoquentar.
\section{Ensaburrar}
\begin{itemize}
\item {Grp. gram.:v. t.}
\end{itemize}
Encher de saburra; lastrar (embarcações).
Emporcalhar.
\section{Ensaca}
\begin{itemize}
\item {Grp. gram.:f.}
\end{itemize}
\begin{itemize}
\item {Utilização:Des.}
\end{itemize}
O mesmo que \textunderscore ensacamento\textunderscore .
\section{Ensaca}
\begin{itemize}
\item {Grp. gram.:f.}
\end{itemize}
\begin{itemize}
\item {Utilização:Prov.}
\end{itemize}
\begin{itemize}
\item {Utilização:dur.}
\end{itemize}
Tira estreita de coiro, que prende o pírtigo ao cisoiro e este á mangueira.
\section{Ensacador}
\begin{itemize}
\item {Grp. gram.:m.}
\end{itemize}
\begin{itemize}
\item {Utilização:Bras. do N}
\end{itemize}
Aquelle que ensaca.
Negociante, que compra café de várias qualidades, para o vender depois de ensacado.
O mesmo que \textunderscore armazenário\textunderscore .
\section{Ensacamento}
\begin{itemize}
\item {Grp. gram.:m.}
\end{itemize}
Acto ou effeito de ensacar.
\section{Ensacar}
\begin{itemize}
\item {Grp. gram.:v. t.}
\end{itemize}
\begin{itemize}
\item {Utilização:Ant.}
\end{itemize}
\begin{itemize}
\item {Utilização:Prov.}
\end{itemize}
\begin{itemize}
\item {Utilização:dur.}
\end{itemize}
Meter em saco.
Guardar.
Meter (carne), fazendo paios, chouriços, etc.
Encantoar, encurralar, meter em lugar estreito, sem saída.
Apertar ou cingir (a saia) com cinta ou faixa, formando dobra ou verdugo.
\section{Ensaiada}
\begin{itemize}
\item {Grp. gram.:f.}
\end{itemize}
\begin{itemize}
\item {Utilização:Prov.}
\end{itemize}
\begin{itemize}
\item {Utilização:alent.}
\end{itemize}
\begin{itemize}
\item {Proveniência:(De \textunderscore ensaiar\textunderscore ^2)}
\end{itemize}
O mesmo que \textunderscore mascarada\textunderscore .
\section{Ensaiador}
\begin{itemize}
\item {Grp. gram.:adj.}
\end{itemize}
\begin{itemize}
\item {Grp. gram.:M.}
\end{itemize}
Que ensaia.
Aquelle que ensaia.
\section{Ensaiamento}
\begin{itemize}
\item {Grp. gram.:m.}
\end{itemize}
O mesmo que \textunderscore ensaio\textunderscore .
\section{Ensaiar}
\begin{itemize}
\item {Grp. gram.:v. t.}
\end{itemize}
\begin{itemize}
\item {Proveniência:(De \textunderscore ensaio\textunderscore )}
\end{itemize}
Examinar o pêso, os quilates, o valor de.
Examinar o estado ou as qualidades de.
Exercitar, experimentar.
Adestrar; preparar: \textunderscore ensaiar uma representação\textunderscore .
Estudar.
\section{Ensaiar}
\begin{itemize}
\item {Grp. gram.:v. t.}
\end{itemize}
\begin{itemize}
\item {Utilização:Prov.}
\end{itemize}
\begin{itemize}
\item {Proveniência:(De \textunderscore saia\textunderscore )}
\end{itemize}
Arregaçar (a saia), apertando-a com cinta por baixo dos quadris.
Mascarar.
\section{Ensaibramento}
\begin{itemize}
\item {Grp. gram.:m.}
\end{itemize}
Acto ou effeito de ensaibrar.
\section{Ensaibrar}
\begin{itemize}
\item {Grp. gram.:v. t.}
\end{itemize}
Cobrir com saibro (a caixa de estradas macadamizadas, para a cylindrar depois), as ruas dos jardins, etc.
Cobrir de areia, arear em demonstração de festa: \textunderscore ensaibrar o adro\textunderscore .
\section{Ensainhar}
\begin{itemize}
\item {fónica:sa-i}
\end{itemize}
\begin{itemize}
\item {Grp. gram.:v. i.}
\end{itemize}
\begin{itemize}
\item {Utilização:T. da Bairrada}
\end{itemize}
Diz-se do milho, quando é atacado pela saínha.
\section{Ensaio}
\begin{itemize}
\item {Grp. gram.:m.}
\end{itemize}
\begin{itemize}
\item {Utilização:Prov.}
\end{itemize}
\begin{itemize}
\item {Utilização:dur.}
\end{itemize}
\begin{itemize}
\item {Proveniência:(Do lat. \textunderscore exagium\textunderscore )}
\end{itemize}
Acto de ensaiar.
Tratado succinto, esbôço, resumo.
Vão intermédio das duas cavernas, nos barcos rabelos.
\section{Ensais}
\begin{itemize}
\item {Grp. gram.:m. pl.}
\end{itemize}
\begin{itemize}
\item {Utilização:Náut.}
\end{itemize}
Peças, que se pregam na quilha do navio.
\section{Ensalada}
\begin{itemize}
\item {Grp. gram.:f.}
\end{itemize}
\begin{itemize}
\item {Utilização:Ant.}
\end{itemize}
O mesmo que \textunderscore salada\textunderscore .
Composição poética, em que entravam versos de differente metro e, até em linguas diversas, e destinada ao canto.
(Cast. \textunderscore ensalada\textunderscore )
\section{Ensalmador}
\begin{itemize}
\item {Grp. gram.:adj.}
\end{itemize}
\begin{itemize}
\item {Grp. gram.:M.}
\end{itemize}
Que ensalma.
Aquelle que ensalma.
\section{Ensalmar}
\begin{itemize}
\item {Grp. gram.:v. t.}
\end{itemize}
\begin{itemize}
\item {Proveniência:(De \textunderscore salmo\textunderscore )}
\end{itemize}
Curar ou tratar com ensalmos.
Exorcizar.
\section{Ensalmeiro}
\begin{itemize}
\item {Grp. gram.:m.}
\end{itemize}
(V.ensalmador)
\section{Ensalmo}
\begin{itemize}
\item {Grp. gram.:m.}
\end{itemize}
\begin{itemize}
\item {Proveniência:(De \textunderscore ensalmar\textunderscore )}
\end{itemize}
Oração supersticiosa, para curar males ou fazer malefícios.
Charlatanismo.
Bruxaria.
\section{Ensalmoirar}
\begin{itemize}
\item {Grp. gram.:v. t.}
\end{itemize}
Pôr ou têr em salmoira.
\section{Ensalmourar}
\begin{itemize}
\item {Grp. gram.:v. t.}
\end{itemize}
Pôr ou têr em salmoura.
\section{Ensalsada}
\begin{itemize}
\item {Grp. gram.:f.}
\end{itemize}
(V.salsada)
\section{Ensamarrar}
\begin{itemize}
\item {Grp. gram.:v. t.}
\end{itemize}
Vestir de samarra.
\section{Ensambenitar}
\begin{itemize}
\item {Grp. gram.:v. t.}
\end{itemize}
Pôr sambenito a.
\section{Ensamblador}
\begin{itemize}
\item {Grp. gram.:m.}
\end{itemize}
\begin{itemize}
\item {Grp. gram.:Adj.}
\end{itemize}
Entalhador.
Marceneiro.
Que ensambla.
\section{Ensambladura}
\begin{itemize}
\item {Grp. gram.:f.}
\end{itemize}
Acto ou effeito de ensamblar.
\section{Ensamblagem}
\begin{itemize}
\item {Grp. gram.:f.}
\end{itemize}
O mesmo que \textunderscore ensambladura\textunderscore .
\section{Ensamblamento}
\begin{itemize}
\item {Grp. gram.:m.}
\end{itemize}
O mesmo que \textunderscore ensambladura\textunderscore .
\section{Ensamblar}
\begin{itemize}
\item {Grp. gram.:v. t.}
\end{itemize}
Reunir ou fazer lavores em.
Embutir (peças de madeira).
Entalhar.
(Por \textunderscore ensembrar\textunderscore , de \textunderscore ensembra\textunderscore )
\section{Ensampação}
\begin{itemize}
\item {Grp. gram.:f.}
\end{itemize}
Nome, que, nas margens do Sado, se dá ao enjôo das marinhas.(V.enjôo)
\section{Ensampar}
\begin{itemize}
\item {Grp. gram.:v. i.}
\end{itemize}
Soffrer ensampação (falando-se de salinas).
\section{Ensancha}
\begin{itemize}
\item {Grp. gram.:f.}
\end{itemize}
\begin{itemize}
\item {Utilização:Fig.}
\end{itemize}
\begin{itemize}
\item {Proveniência:(T. cast.)}
\end{itemize}
Parte do vestuário, que se deixa a mais, embebida na costura, para que elle se possa alargar quando seja preciso.
Ampliação.
\section{Ensanchar}
\begin{itemize}
\item {Grp. gram.:v. t.}
\end{itemize}
Dar ensanchas a.
Ampliar; dilatar.
\section{Ensandalar}
\begin{itemize}
\item {Grp. gram.:v. t.}
\end{itemize}
Perfumar ou untar com sândalo.
\section{Ensandecer}
\begin{itemize}
\item {Grp. gram.:v. t.}
\end{itemize}
\begin{itemize}
\item {Grp. gram.:V. i.}
\end{itemize}
Tornar sandeu.
Tornar-se sandeu.
\section{Ensanefar}
\begin{itemize}
\item {Grp. gram.:v. t.}
\end{itemize}
Ornar com sanefa. Cf. Herculano, \textunderscore Quest. Públ.\textunderscore , I, 111.
\section{Ensanguentar}
\begin{itemize}
\item {fónica:gu-en}
\end{itemize}
\begin{itemize}
\item {Grp. gram.:v. t.}
\end{itemize}
\begin{itemize}
\item {Utilização:Fig.}
\end{itemize}
Manchar de sangue.
Macular; denegrir.
\section{Ensanguinhar}
\begin{itemize}
\item {fónica:gu-i}
\end{itemize}
\begin{itemize}
\item {Grp. gram.:v. t.}
\end{itemize}
\begin{itemize}
\item {Grp. gram.:V. p.}
\end{itemize}
\begin{itemize}
\item {Utilização:Des.}
\end{itemize}
Ensanguentar.
Criar sangue novo (o animal).
\section{Ensanhar}
\begin{itemize}
\item {Grp. gram.:v. t.}
\end{itemize}
\begin{itemize}
\item {Utilização:Ant.}
\end{itemize}
O mesmo que \textunderscore assanhar\textunderscore .
\section{Ensaque}
\begin{itemize}
\item {Grp. gram.:m.}
\end{itemize}
Acto de ensacar.
\section{Ensarilhar}
\begin{itemize}
\item {Grp. gram.:v. t.}
\end{itemize}
\begin{itemize}
\item {Grp. gram.:V. i.}
\end{itemize}
\begin{itemize}
\item {Utilização:Prov.}
\end{itemize}
Dobar em sarilho: \textunderscore ensarilhar meadas\textunderscore .
Formar sarilho com.
Apoiar no chão as coronhas de (espingardas), unindo-as na parte superior.
Andar continuamente de um lado para outro.
\section{Ensarnecer}
\begin{itemize}
\item {Grp. gram.:v. i.}
\end{itemize}
\begin{itemize}
\item {Proveniência:(De \textunderscore sarna\textunderscore )}
\end{itemize}
Tornar-se sarnento.
\section{Ensarranhar}
\begin{itemize}
\item {Grp. gram.:v. t.}
\end{itemize}
\begin{itemize}
\item {Utilização:Prov.}
\end{itemize}
\begin{itemize}
\item {Utilização:minh.}
\end{itemize}
\begin{itemize}
\item {Proveniência:(De \textunderscore sarranho\textunderscore )}
\end{itemize}
Mascarrar, sujar.
\section{Ensartar}
\begin{itemize}
\item {Grp. gram.:v. t.}
\end{itemize}
Enfiar (pérolas ou contas).
Engranzar, encadear:«\textunderscore neste ensartar de consoantes, como avemarias num terço...\textunderscore »Garrett, \textunderscore Arco de Santa Anna\textunderscore , II, 232.
\section{Ensaucado}
\begin{itemize}
\item {fónica:sa-u}
\end{itemize}
\begin{itemize}
\item {Grp. gram.:adj.}
\end{itemize}
Que tem saúcos.
\section{Enscenação}
\begin{itemize}
\item {Grp. gram.:f.}
\end{itemize}
\begin{itemize}
\item {Utilização:Neol.}
\end{itemize}
Acto ou effeito de enscenar.--Propôs-se êste t., em substituição do francesismo \textunderscore mis-en-scène\textunderscore .
\section{Enscenar}
\begin{itemize}
\item {Grp. gram.:v. t.}
\end{itemize}
\begin{itemize}
\item {Utilização:Neol.}
\end{itemize}
Pôr em scena.
Fazer representar no theatro.
\section{...ense}
\begin{itemize}
\item {Grp. gram.:suf. adj.}
\end{itemize}
\begin{itemize}
\item {Proveniência:(Do lat. \textunderscore ...ensis\textunderscore )}
\end{itemize}
(designativo de naturalidade, procedência, pertença: \textunderscore portuense\textunderscore , \textunderscore lisbonense...\textunderscore )
\section{Enseada}
\begin{itemize}
\item {Grp. gram.:f.}
\end{itemize}
\begin{itemize}
\item {Proveniência:(De \textunderscore enseio\textunderscore )}
\end{itemize}
Pequeno pôrto, angra.
\section{Ensebar}
\begin{itemize}
\item {Grp. gram.:v. t.}
\end{itemize}
\begin{itemize}
\item {Utilização:Ext.}
\end{itemize}
Untar ou sujar com sebo.
Engordurar.
Pôr nódoas em.
Sujar com o uso: \textunderscore um chapéu ensebado\textunderscore .
\section{Ensebar}
\begin{itemize}
\item {Grp. gram.:v. t.}
\end{itemize}
\begin{itemize}
\item {Utilização:Prov.}
\end{itemize}
\begin{itemize}
\item {Utilização:dur.}
\end{itemize}
Pôr sebe em.
Rodear de sebe.
\section{Ensecadeira}
\begin{itemize}
\item {Grp. gram.:f.}
\end{itemize}
\begin{itemize}
\item {Proveniência:(De \textunderscore ensecar\textunderscore ^1)}
\end{itemize}
Tapume, para que fique em sêco uma construcção, em que se trabalhe abaixo do nivel da água.
\section{Ensecamento}
\begin{itemize}
\item {Grp. gram.:m.}
\end{itemize}
Acto de ensecar^1.
Esgotamento.
\section{Ensecar}
\begin{itemize}
\item {Grp. gram.:v. t.}
\end{itemize}
\begin{itemize}
\item {Utilização:Ant.}
\end{itemize}
\begin{itemize}
\item {Grp. gram.:V. i.}
\end{itemize}
\begin{itemize}
\item {Proveniência:(De \textunderscore secar\textunderscore )}
\end{itemize}
Pôr em sêco.
Esgotar.
Averiguar, verificar.
Ficar em sêco.
\section{Ensecar}
\begin{itemize}
\item {Grp. gram.:v. t.}
\end{itemize}
\begin{itemize}
\item {Utilização:Des.}
\end{itemize}
\begin{itemize}
\item {Proveniência:(Do lat. \textunderscore insequi\textunderscore )}
\end{itemize}
Perseguir.
\section{Ensedadura}
\begin{itemize}
\item {Grp. gram.:f.}
\end{itemize}
\begin{itemize}
\item {Proveniência:(De \textunderscore ensedar\textunderscore )}
\end{itemize}
Porção de crinas de cavallo, tendidas no arco da rabeca e de outros instrumentos de corda.
\section{Ensedar}
\begin{itemize}
\item {Grp. gram.:v. t.}
\end{itemize}
Pôr ensedadura em (arco de instrumento).
\section{Enseio}
\begin{itemize}
\item {Grp. gram.:m.}
\end{itemize}
\begin{itemize}
\item {Utilização:Des.}
\end{itemize}
\begin{itemize}
\item {Proveniência:(De \textunderscore seio\textunderscore )}
\end{itemize}
Sinuosidade ou quebrada, entre montes.
Recôncavo, aberto pelos águas.
Enseada.
\section{Enseirador}
\begin{itemize}
\item {Grp. gram.:m.}
\end{itemize}
Operário, que enseira os figos passados, nos armazens onde estes se preparam para o commércio. Cf. \textunderscore Inquér. Industr.\textunderscore , p. II, l. III, 147.
\section{Enseiramento}
\begin{itemize}
\item {Grp. gram.:m.}
\end{itemize}
Porção de seiras.
Acto de \textunderscore enseirar\textunderscore .
\section{Enseirar}
\begin{itemize}
\item {Grp. gram.:v. t.}
\end{itemize}
Meter em seira: \textunderscore enseirar figos\textunderscore .
\section{Enseivar}
\begin{itemize}
\item {Grp. gram.:v. t.}
\end{itemize}
\begin{itemize}
\item {Utilização:Neol.}
\end{itemize}
Desenvolver a seiva de. Cf. Junqueiro, \textunderscore Pátria\textunderscore , IX.
\section{Enseja}
\begin{itemize}
\item {Grp. gram.:f.}
\end{itemize}
\begin{itemize}
\item {Utilização:Ant.}
\end{itemize}
O mesmo que \textunderscore insídia\textunderscore .
\section{Ensejar}
\begin{itemize}
\item {Grp. gram.:v. t.}
\end{itemize}
\begin{itemize}
\item {Utilização:Des.}
\end{itemize}
Esperar ou espiar a opportunidade de.
Ensaiar.
Dar ensejo a.
\section{Ensejo}
\begin{itemize}
\item {Grp. gram.:m.}
\end{itemize}
\begin{itemize}
\item {Proveniência:(Do lat. \textunderscore exagium\textunderscore )}
\end{itemize}
Occasião asada, opportuna.
Lance.
\section{Enselada}
\begin{itemize}
\item {Grp. gram.:f.}
\end{itemize}
\begin{itemize}
\item {Utilização:Ant.}
\end{itemize}
O mesmo que \textunderscore ensalada\textunderscore , composição poética ou cantilena; ensoada. Cf. Gil Vicente, I, 75.
\section{Ensembra}
\begin{itemize}
\item {Grp. gram.:adv.}
\end{itemize}
\begin{itemize}
\item {Utilização:Ant.}
\end{itemize}
Juntamente.
(Cast. \textunderscore ensembla\textunderscore )
\section{Ensenhorear}
\textunderscore v. t.\textunderscore  (e der.)
O mesmo que \textunderscore assenhorear-se\textunderscore , etc.
\section{Enserralhar-se}
\begin{itemize}
\item {Grp. gram.:v. p.}
\end{itemize}
\begin{itemize}
\item {Utilização:Prov.}
\end{itemize}
\begin{itemize}
\item {Utilização:beir.}
\end{itemize}
\begin{itemize}
\item {Proveniência:(De \textunderscore serralha\textunderscore )}
\end{itemize}
Enrijar e não crescer (a alface).
\section{Ensesgar}
\begin{itemize}
\item {Grp. gram.:v. i.}
\end{itemize}
\begin{itemize}
\item {Utilização:Prov.}
\end{itemize}
\begin{itemize}
\item {Utilização:minh.}
\end{itemize}
\begin{itemize}
\item {Proveniência:(De \textunderscore sesgo\textunderscore )}
\end{itemize}
Lavrar, não ao correr do campo, mas obliquamente.
\section{Ensífero}
\begin{itemize}
\item {Grp. gram.:adj.}
\end{itemize}
\begin{itemize}
\item {Proveniência:(Lat. \textunderscore ensifer\textunderscore )}
\end{itemize}
Que traz espada. Cf. \textunderscore Lusíadas\textunderscore , V, 85.
\section{Ensiforme}
\begin{itemize}
\item {Grp. gram.:adj.}
\end{itemize}
\begin{itemize}
\item {Proveniência:(Do lat. \textunderscore ensis\textunderscore  + \textunderscore forma\textunderscore )}
\end{itemize}
Que tem fórma de espada.
\section{Ensilagem}
\begin{itemize}
\item {Grp. gram.:f.}
\end{itemize}
Acto de ensilar.
Systema de conservar forragens em silos.
\section{Ensilamento}
\begin{itemize}
\item {Grp. gram.:m.}
\end{itemize}
Acto de \textunderscore ensilar\textunderscore .
\section{Ensilar}
\begin{itemize}
\item {Grp. gram.:v.}
\end{itemize}
\begin{itemize}
\item {Utilização:t. Agr.}
\end{itemize}
Meter ou conservar no silo (forragens).
\section{Ensilvado}
\begin{itemize}
\item {Grp. gram.:adj.}
\end{itemize}
\begin{itemize}
\item {Utilização:Prov.}
\end{itemize}
\begin{itemize}
\item {Utilização:minh.}
\end{itemize}
\begin{itemize}
\item {Utilização:Prov.}
\end{itemize}
\begin{itemize}
\item {Utilização:beir.}
\end{itemize}
Incomparável, inexcedível, (falando-se de uma festa magnífica).
Diz-se do púlpito, em que se prègou tal sermão, que ninguém o pregará melhor.
\section{Ensilvar}
\begin{itemize}
\item {Grp. gram.:v. t.}
\end{itemize}
Pôr silvas em, vedar com silvas.
\section{Ensilveirado}
\begin{itemize}
\item {Grp. gram.:adj.}
\end{itemize}
\begin{itemize}
\item {Utilização:Fig.}
\end{itemize}
Em que crescem silveiras.
Enredado. Cf. Camillo, \textunderscore Narcót.\textunderscore , II, 333.
\section{Ensimesmar-se}
\begin{itemize}
\item {Grp. gram.:v. p.}
\end{itemize}
\begin{itemize}
\item {Utilização:Des.}
\end{itemize}
Concentrar-se, meditando.
(Cast. \textunderscore ensimismar\textunderscore )
\section{Ensina}
\begin{itemize}
\item {Grp. gram.:f.}
\end{itemize}
\begin{itemize}
\item {Utilização:Pop.}
\end{itemize}
O mesmo que \textunderscore ensinadela\textunderscore .
\section{Ensinação}
\begin{itemize}
\item {Grp. gram.:f.}
\end{itemize}
O mesmo que \textunderscore ensinamento\textunderscore .
\section{Ensinadela}
\begin{itemize}
\item {Grp. gram.:f.}
\end{itemize}
\begin{itemize}
\item {Utilização:Fam.}
\end{itemize}
\begin{itemize}
\item {Proveniência:(De \textunderscore ensinar\textunderscore )}
\end{itemize}
Experiência custosa.
Reprehensão; ensaboadela.
\section{Ensinadiço}
\begin{itemize}
\item {Grp. gram.:adj.}
\end{itemize}
\begin{itemize}
\item {Proveniência:(De \textunderscore ensinar\textunderscore )}
\end{itemize}
Que anda a aprender.
\section{Ensinador}
\begin{itemize}
\item {Grp. gram.:adj.}
\end{itemize}
\begin{itemize}
\item {Grp. gram.:M.}
\end{itemize}
Que ensina.
Aquelle que ensina.
\section{Ensinamento}
\begin{itemize}
\item {Grp. gram.:m.}
\end{itemize}
Acto ou effeito de ensinar.
\section{Ensinança}
\begin{itemize}
\item {Grp. gram.:f.}
\end{itemize}
(V.ensino)
\section{Ensinar}
\begin{itemize}
\item {Grp. gram.:v. t.}
\end{itemize}
\begin{itemize}
\item {Utilização:Des.}
\end{itemize}
\begin{itemize}
\item {Utilização:Fam.}
\end{itemize}
\begin{itemize}
\item {Proveniência:(Do b. lat. \textunderscore insignare\textunderscore )}
\end{itemize}
Instruir sôbre: \textunderscore ensinar história\textunderscore .
Doutrinar.
Educar: \textunderscore ensinar meninos\textunderscore .
Mostrar, apontar: \textunderscore ensinar um caminho\textunderscore .
Demonstrar.
Adestrar: \textunderscore ensinar um cavallo\textunderscore .
Inspirar.
Castigar, bater. Cf. Junqueiro, \textunderscore Musa em Férias\textunderscore , 48.
\section{Ensino}
\begin{itemize}
\item {Grp. gram.:m.}
\end{itemize}
Acto ou effeito de ensinar.
Cortesia.
\section{Ensirostro}
\begin{itemize}
\item {fónica:rós}
\end{itemize}
\begin{itemize}
\item {Grp. gram.:adj.}
\end{itemize}
\begin{itemize}
\item {Proveniência:(Do lat. \textunderscore ensis\textunderscore  + \textunderscore rostrum\textunderscore )}
\end{itemize}
Que tem o bico arqueado como o alfange.
\section{Ensirrostro}
\begin{itemize}
\item {Grp. gram.:adj.}
\end{itemize}
\begin{itemize}
\item {Proveniência:(Do lat. \textunderscore ensis\textunderscore  + \textunderscore rostrum\textunderscore )}
\end{itemize}
Que tem o bico arqueado como o alfange.
\section{Enslénia}
\begin{itemize}
\item {Grp. gram.:f.}
\end{itemize}
\begin{itemize}
\item {Proveniência:(De \textunderscore Enslen\textunderscore , n. p.)}
\end{itemize}
Planta trepadeira do Estado da Virgínia.
\section{Ensoada}
\begin{itemize}
\item {Grp. gram.:f.}
\end{itemize}
\begin{itemize}
\item {Utilização:P. us.}
\end{itemize}
O mesmo que \textunderscore soada\textunderscore .
Acto de \textunderscore ensoar\textunderscore ^2.
\section{Ensoado}
\begin{itemize}
\item {Grp. gram.:adj.}
\end{itemize}
\begin{itemize}
\item {Proveniência:(De \textunderscore ensoar\textunderscore ^1)}
\end{itemize}
Fláccido, enfraquecido pelo calor.
Froixo, insípido.
\section{Ensoalheirado}
\begin{itemize}
\item {Grp. gram.:adj.}
\end{itemize}
\begin{itemize}
\item {Proveniência:(De \textunderscore soalheira\textunderscore )}
\end{itemize}
Cheio de sol.
Banhado de sol.
\section{Ensoamento}
\begin{itemize}
\item {Grp. gram.:m.}
\end{itemize}
Acto ou effeito de ensoar^1.
Doença das plantas, quando, por falta de água ou de humidade, desmaiam um pouco, podendo restabelecer-se com a rega ou com a chuva.
Estiolamento.
\section{Ensoar}
\begin{itemize}
\item {Grp. gram.:v. i.  e  p.}
\end{itemize}
\begin{itemize}
\item {Proveniência:(Do lat. \textunderscore insolair\textunderscore )}
\end{itemize}
Recozer-se com o calor (a fruta), antes de madura.
Murchar, com o calor do sol.
\section{Ensoar}
\begin{itemize}
\item {Grp. gram.:v. t.}
\end{itemize}
\begin{itemize}
\item {Utilização:Ant.}
\end{itemize}
\begin{itemize}
\item {Proveniência:(De \textunderscore som\textunderscore )}
\end{itemize}
Pôr em música.
Entoar.
\section{Ensoberbar-se}
\begin{itemize}
\item {Grp. gram.:v. p.}
\end{itemize}
O mesmo que [[ensoberbecer-se|ensoberbecer]].
\section{Ensoberbecer}
\begin{itemize}
\item {Grp. gram.:v. t.}
\end{itemize}
\begin{itemize}
\item {Grp. gram.:V. p.}
\end{itemize}
Tornar soberbo, orgulhoso, ufano.
Tornar-se soberbo, orgulhoso.
\section{Ensobradar}
\begin{itemize}
\item {Grp. gram.:v. t.}
\end{itemize}
O mesmo que \textunderscore assobradar\textunderscore .
\section{Ensofregar}
\begin{itemize}
\item {Grp. gram.:v. t.}
\end{itemize}
Fazer sôfrego.
\section{Ensogadura}
\begin{itemize}
\item {Grp. gram.:f.}
\end{itemize}
\begin{itemize}
\item {Utilização:Prov.}
\end{itemize}
\begin{itemize}
\item {Utilização:minh.}
\end{itemize}
Acto de ensogar.
Aselha de coiro na deanteira do jugo dos bois.
\section{Ensogar}
\begin{itemize}
\item {Grp. gram.:v. t.}
\end{itemize}
\begin{itemize}
\item {Utilização:Prov.}
\end{itemize}
\begin{itemize}
\item {Utilização:minh.}
\end{itemize}
Pôr soga em (bois).
\section{Ensoissar-se}
\begin{itemize}
\item {Grp. gram.:v. p.}
\end{itemize}
\begin{itemize}
\item {Utilização:Prov.}
\end{itemize}
\begin{itemize}
\item {Utilização:trasm.}
\end{itemize}
Atarefar-se, afanar-se.
Agoniar-se.
\section{Ensoleiramento}
\begin{itemize}
\item {Grp. gram.:m.}
\end{itemize}
Acto de ensoleirar.
Conjunto do engradamento e do sobrado que sôbre elle assenta, para servir de base aos alicerces de construcções em terrenos compressiveis.
\section{Ensoleirar}
\begin{itemize}
\item {Grp. gram.:v. t.}
\end{itemize}
Fazer a soleira de (porta).
Fazer o ensoleiramento de (uma construcção).
\section{Ensolvamento}
\begin{itemize}
\item {Grp. gram.:m.}
\end{itemize}
Acto ou effeito de ensolvar.
\section{Ensolvar}
\begin{itemize}
\item {Grp. gram.:v. t.}
\end{itemize}
Impedir de se disparar (uma peça de artilharia), humedecendo a pólvora ou atochando a bala.
(Talvez se relacione com o lat. \textunderscore solvere\textunderscore , soltar, desprender, com um pref. de negação, \textunderscore in\textunderscore )
\section{Ensombrar}
\begin{itemize}
\item {Grp. gram.:v. t.}
\end{itemize}
\begin{itemize}
\item {Utilização:Fig.}
\end{itemize}
Fazer sombra a.
Assombrar.
Tornar triste.
\section{Ensombrecer}
\begin{itemize}
\item {Grp. gram.:v. i.}
\end{itemize}
Tornar-se sombrio, cobrir-se de sombra.
\section{Ensombro}
\begin{itemize}
\item {Grp. gram.:m.}
\end{itemize}
\begin{itemize}
\item {Utilização:Fig.}
\end{itemize}
\begin{itemize}
\item {Proveniência:(De \textunderscore ensombrar\textunderscore )}
\end{itemize}
Coisa que ensombra.
Protecção.
\section{Ensopadeira}
\begin{itemize}
\item {Grp. gram.:f.}
\end{itemize}
\begin{itemize}
\item {Utilização:Bras}
\end{itemize}
\begin{itemize}
\item {Proveniência:(De \textunderscore ensopar\textunderscore )}
\end{itemize}
O mesmo que \textunderscore terrina\textunderscore .
\section{Ensopado}
\begin{itemize}
\item {Grp. gram.:m.}
\end{itemize}
\begin{itemize}
\item {Utilização:Prov.}
\end{itemize}
Guisado de carne ou peixe com sopas.
\section{Ensopar}
\begin{itemize}
\item {Grp. gram.:v. t.}
\end{itemize}
\begin{itemize}
\item {Grp. gram.:V. p.}
\end{itemize}
\begin{itemize}
\item {Utilização:Bras}
\end{itemize}
Converter em sopa.
Tornar molhado como sopa.
Embeber.
Encharcar: \textunderscore a chuva ensopou-o\textunderscore .
Guisar.
Tomar intimidade ou liberdade com alguém.
\section{Ensopear}
\begin{itemize}
\item {Grp. gram.:v. t.}
\end{itemize}
\begin{itemize}
\item {Utilização:Prov.}
\end{itemize}
\begin{itemize}
\item {Utilização:minh.}
\end{itemize}
Baptizar á pressa e em casa (uma criança), sendo um padre quem ministre o baptismo.
(Cp. \textunderscore ensopar\textunderscore )
\section{Ensorear}
\textunderscore v. t.\textunderscore  (e der.) \textunderscore Prov. dur.\textunderscore 
O mesmo que \textunderscore assorear\textunderscore , etc.
\section{Ensosso}
\begin{itemize}
\item {fónica:sô}
\end{itemize}
\textunderscore m.\textunderscore  (e der.)
O mesmo que \textunderscore insôsso\textunderscore ^1, etc.
\section{Ensovacado}
\begin{itemize}
\item {Grp. gram.:adj.}
\end{itemize}
\begin{itemize}
\item {Utilização:Prov.}
\end{itemize}
\begin{itemize}
\item {Proveniência:(De \textunderscore ensovacar\textunderscore )}
\end{itemize}
Que ensovacou.
Embuchado; impedido de falar, por surpresa ou susto.
\section{Ensovacar}
\begin{itemize}
\item {Grp. gram.:v. i.}
\end{itemize}
\begin{itemize}
\item {Utilização:Prov.}
\end{itemize}
Encavacar, embuchar; ficar sobresaltado, sem saber ou sem poder falar.
(Relaciona-se com \textunderscore suffocar\textunderscore ?)
\section{Ensujentar}
\begin{itemize}
\item {Grp. gram.:v. t.}
\end{itemize}
\begin{itemize}
\item {Utilização:Ant.}
\end{itemize}
O mesmo que \textunderscore sujar\textunderscore .
\section{Ensumagrar}
\begin{itemize}
\item {Grp. gram.:v. t.}
\end{itemize}
Preparar com sumagre.
\section{Ensurdecedor}
\begin{itemize}
\item {Grp. gram.:adj.}
\end{itemize}
Que faz ensurdecer.
Que faz grande barulho ou estrondo.
\section{Ensurdecência}
\begin{itemize}
\item {Grp. gram.:f.}
\end{itemize}
\begin{itemize}
\item {Proveniência:(De \textunderscore ensurdecer\textunderscore )}
\end{itemize}
Surdez.
\section{Ensurdecer}
\begin{itemize}
\item {Grp. gram.:v. t.}
\end{itemize}
\begin{itemize}
\item {Grp. gram.:V. i.}
\end{itemize}
\begin{itemize}
\item {Utilização:Fig.}
\end{itemize}
Tornar surdo.
Tornar-se surdo.
Não dar attenção, não fazer caso do que se diz.
\section{Ensurdecimento}
\begin{itemize}
\item {Grp. gram.:m.}
\end{itemize}
Acto ou effeito de ensurdecer.
\section{Ensurraipar}
\begin{itemize}
\item {Grp. gram.:v.}
\end{itemize}
\begin{itemize}
\item {Utilização:t. Marn.}
\end{itemize}
\begin{itemize}
\item {Proveniência:(De \textunderscore surraipa\textunderscore )}
\end{itemize}
Cobrir de sal (o solo das marinhas).
\section{Ensurroamento}
\begin{itemize}
\item {Grp. gram.:m.}
\end{itemize}
Acção de \textunderscore ensurroar\textunderscore .
\section{Ensurroar}
\begin{itemize}
\item {Grp. gram.:v. t.}
\end{itemize}
\begin{itemize}
\item {Utilização:Bras}
\end{itemize}
\begin{itemize}
\item {Proveniência:(De \textunderscore surrão\textunderscore )}
\end{itemize}
Meter no surrão.
Enrolar (tabaco) em coiros crus.
\section{Enta}
\begin{itemize}
\item {Grp. gram.:f.}
\end{itemize}
\begin{itemize}
\item {Utilização:Prov.}
\end{itemize}
\begin{itemize}
\item {Utilização:minh.}
\end{itemize}
Camada de qualquer coisa.
\section{Entabacado}
\begin{itemize}
\item {Grp. gram.:adj.}
\end{itemize}
Sujo ou manchado de tabaco em pó. Cf. Filinto, II, 133 e 262.
\section{Entablamento}
\begin{itemize}
\item {Grp. gram.:m.}
\end{itemize}
Parte de um edificío, comprehendendo architrave, friso e cornija.
(Contr. de \textunderscore entabulamento\textunderscore )
\section{Entaboar}
\textunderscore v. t.\textunderscore  (e der.)
O mesmo que \textunderscore entabuar\textunderscore , etc.
\section{Entabocar}
\begin{itemize}
\item {Grp. gram.:v. t.}
\end{itemize}
\begin{itemize}
\item {Utilização:Bras}
\end{itemize}
Entalar, apertar.
\section{Entabolar}
\textunderscore v. t.\textunderscore  (e der.)
(V. \textunderscore entabular\textunderscore , etc.)
\section{Entabuamento}
\begin{itemize}
\item {Grp. gram.:m.}
\end{itemize}
\begin{itemize}
\item {Proveniência:(Do lat. \textunderscore tabulamentum\textunderscore )}
\end{itemize}
Acto \textunderscore ou\textunderscore effeito de entabuar.
\section{Entabuar}
\begin{itemize}
\item {Grp. gram.:v. t.}
\end{itemize}
\begin{itemize}
\item {Grp. gram.:V. p.}
\end{itemize}
\begin{itemize}
\item {Proveniência:(Do lat. \textunderscore tabulare\textunderscore )}
\end{itemize}
Revestir de tábuas.
Assobradar.
Endurecer.
\section{Entabulamento}
\begin{itemize}
\item {Grp. gram.:m.}
\end{itemize}
O mesmo que \textunderscore entablamento\textunderscore .
Acto ou effeito de entabular.
Cercadura de tábuas, que se prega junto do tecto, na parede de uma sala ou quarto.
\section{Entabular}
\begin{itemize}
\item {Grp. gram.:v. t.}
\end{itemize}
\begin{itemize}
\item {Utilização:Bras. do S}
\end{itemize}
\begin{itemize}
\item {Proveniência:(De \textunderscore tábula\textunderscore )}
\end{itemize}
O mesmo que \textunderscore entabuar\textunderscore .
Ordenar, pôr em ordem, preparar.
Iniciar, inaugurar; estabelecer: \textunderscore entabular negociações\textunderscore .
Acostumar (o garanhão) a certo número de éguas, para formar a manada.
\section{Entaburrar}
\begin{itemize}
\item {Grp. gram.:v. i.}
\end{itemize}
\begin{itemize}
\item {Utilização:T. da Bairrada}
\end{itemize}
Definhar ou apodrecer (a espiga do milho).
(Relaciona-se com \textunderscore zaburro\textunderscore ?)
\section{Entada}
\begin{itemize}
\item {Grp. gram.:f.}
\end{itemize}
Planta leguminosa do Malabar, (\textunderscore mimosa entada\textunderscore ).
\section{Entaipar}
\begin{itemize}
\item {Grp. gram.:v. t.}
\end{itemize}
\begin{itemize}
\item {Utilização:Fig.}
\end{itemize}
Meter entre taipas.
Fechar ou cercar com taipas.
Emparedar.
Encarcerar.
\section{Entala}
\begin{itemize}
\item {Grp. gram.:f.}
\end{itemize}
Acto ou effeito de entalar. Cf. Camillo, \textunderscore Doze Casam.\textunderscore , 98; \textunderscore Myst. de Lisb.\textunderscore , I, 55, e II, 126.
\section{Entalação}
\begin{itemize}
\item {Grp. gram.:f.}
\end{itemize}
Acto ou effeito de entalar.
\section{Entaladamente}
\begin{itemize}
\item {Grp. gram.:adv.}
\end{itemize}
\begin{itemize}
\item {Proveniência:(De \textunderscore entalar\textunderscore )}
\end{itemize}
Com entalação.
\section{Entaladela}
\begin{itemize}
\item {Grp. gram.:f.}
\end{itemize}
O mesmo que entalação; apuros. Cf. Castilho, \textunderscore Tartufo\textunderscore , 139.
\section{Entalado}
\begin{itemize}
\item {Grp. gram.:adj.}
\end{itemize}
\begin{itemize}
\item {Utilização:Prov.}
\end{itemize}
O mesmo que \textunderscore entalicado\textunderscore .
\section{Entalador}
\begin{itemize}
\item {Grp. gram.:adj.}
\end{itemize}
Que entala, que embaraça. Cf. Filinto, II, 8.
\section{Entaladura}
\begin{itemize}
\item {Grp. gram.:f.}
\end{itemize}
(V.entalação)
\section{Entalar}
\begin{itemize}
\item {Grp. gram.:v. t.}
\end{itemize}
Apertar com talas.
Pôr entre talas.
Meter em lugar apertado ou sem saída: \textunderscore entalar um pé\textunderscore .
Pôr em difficuldades, em apertos.
Embaraçar, comprometer: \textunderscore a minha bôa fé entalou-me\textunderscore .
\section{Entalecer}
\begin{itemize}
\item {Grp. gram.:v. i.}
\end{itemize}
\begin{itemize}
\item {Utilização:Des.}
\end{itemize}
Criar talo.
\section{Entaleigar}
\begin{itemize}
\item {Grp. gram.:v. t.}
\end{itemize}
\begin{itemize}
\item {Utilização:Des.}
\end{itemize}
\begin{itemize}
\item {Utilização:Fig.}
\end{itemize}
Meter em taleiga.
Fartar; empachar.
\section{Entaleirar}
\begin{itemize}
\item {Grp. gram.:v.}
\end{itemize}
\begin{itemize}
\item {Utilização:t. Carp.}
\end{itemize}
Meter taleiras em.
\section{Entalha}
\begin{itemize}
\item {Grp. gram.:f.}
\end{itemize}
\begin{itemize}
\item {Proveniência:(De \textunderscore entalhar\textunderscore ^1)}
\end{itemize}
Córte ou chanfradura na madeira, para facilitar a entrada do machado ou de outro instrumento cortante.
\section{Entalha}
\begin{itemize}
\item {Grp. gram.:f.}
\end{itemize}
\begin{itemize}
\item {Utilização:Ant.}
\end{itemize}
Acto de entalhar^2.
\section{Entalhador}
\begin{itemize}
\item {Grp. gram.:m.}
\end{itemize}
Aquelle que entalha por offício.
Gravador em madeira.
Ensamblador.
Instrumento de espingardeiro.
\section{Entalhadura}
\begin{itemize}
\item {Grp. gram.:f.}
\end{itemize}
Acto ou effeito de entalhar.
\section{Entalhamento}
\begin{itemize}
\item {Grp. gram.:m.}
\end{itemize}
(V.entalhadura)
\section{Entalhar}
\begin{itemize}
\item {Grp. gram.:v. t.}
\end{itemize}
\begin{itemize}
\item {Grp. gram.:V. i.}
\end{itemize}
\begin{itemize}
\item {Proveniência:(Do lat. \textunderscore intaliare\textunderscore )}
\end{itemize}
Gravar.
Esculpir; cinzelar.
Abrir a meio relêvo.
Fazer obra de talha ou de meio relêvo.
\section{Entalhar}
\begin{itemize}
\item {Grp. gram.:v. t.}
\end{itemize}
\begin{itemize}
\item {Utilização:Ant.}
\end{itemize}
Meter em talha ou em talhas.
\section{Entalhe}
\begin{itemize}
\item {Grp. gram.:m.}
\end{itemize}
O mesmo que \textunderscore entalhadura\textunderscore .
\section{Entalho}
\begin{itemize}
\item {Grp. gram.:m.}
\end{itemize}
\begin{itemize}
\item {Proveniência:(De \textunderscore entalhar\textunderscore )}
\end{itemize}
Entalhadura; entalha.
\section{Entalicado}
\begin{itemize}
\item {Grp. gram.:adj.}
\end{itemize}
\begin{itemize}
\item {Utilização:Prov.}
\end{itemize}
Diz-se do alimento mal cozido ou encruado e, principalmente, da carne a que se deu uma fervura, para se assar depois, sem ficar muito dura.
(Cp. \textunderscore entaloado\textunderscore ^1)
\section{Entaliscar-se}
\begin{itemize}
\item {Grp. gram.:v. p.}
\end{itemize}
\begin{itemize}
\item {Utilização:Des.}
\end{itemize}
\begin{itemize}
\item {Proveniência:(De \textunderscore talisca\textunderscore )}
\end{itemize}
Meter-se em taliscas ou em lugar estreito; entalar-se.
\section{Entaloado}
\begin{itemize}
\item {Grp. gram.:adj.}
\end{itemize}
\begin{itemize}
\item {Utilização:Prov.}
\end{itemize}
\begin{itemize}
\item {Utilização:trasm.}
\end{itemize}
\begin{itemize}
\item {Proveniência:(De \textunderscore talo\textunderscore )}
\end{itemize}
Mal cozido, o mesmo que \textunderscore entalicado\textunderscore .
\section{Entaloado}
\begin{itemize}
\item {Grp. gram.:adj.}
\end{itemize}
\begin{itemize}
\item {Proveniência:(De \textunderscore talão\textunderscore )}
\end{itemize}
Diz-se da ferradura, que é alta no talão ou do lado de trás.
\section{Entancar}
\begin{itemize}
\item {Grp. gram.:v. i.}
\end{itemize}
\begin{itemize}
\item {Utilização:Prov.}
\end{itemize}
\begin{itemize}
\item {Utilização:beir.}
\end{itemize}
\begin{itemize}
\item {Grp. gram.:V. t.}
\end{itemize}
\begin{itemize}
\item {Proveniência:(De \textunderscore tanque\textunderscore )}
\end{itemize}
Diz-se da água pluvial ou corrente, que fórma tanque ou reprêsa.
Represar em tanques, empoçar.
\section{Entances}
\begin{itemize}
\item {Grp. gram.:adv.}
\end{itemize}
\begin{itemize}
\item {Utilização:Ant.}
\end{itemize}
O mesmo que \textunderscore então\textunderscore .
(Cp. cast. \textunderscore entonces\textunderscore )
\section{Entanguecer}
\begin{itemize}
\item {Grp. gram.:v. i.}
\end{itemize}
\begin{itemize}
\item {Utilização:Des.}
\end{itemize}
Inteiriçar-se com frio.
Encolher-se com frio.
\section{Entanguido}
\begin{itemize}
\item {Grp. gram.:adj.}
\end{itemize}
Inteiriçado com frio.
Engerido.
Acanhado.
Apoucado; enfezado.
(Por \textunderscore entanguecido\textunderscore , de \textunderscore entanguecer\textunderscore )
\section{Entanguir-se}
\begin{itemize}
\item {Grp. gram.:v. p.}
\end{itemize}
Tornar-se entanguido ou enfezado.
(Cp. \textunderscore entanguido\textunderscore )
\section{Entanguitado}
\begin{itemize}
\item {Grp. gram.:adj.}
\end{itemize}
\begin{itemize}
\item {Utilização:Bras. do N}
\end{itemize}
O mesmo que \textunderscore entanguido\textunderscore .
\section{Entaniçar}
\begin{itemize}
\item {Grp. gram.:v. t.}
\end{itemize}
\begin{itemize}
\item {Utilização:Bras. do N}
\end{itemize}
\begin{itemize}
\item {Proveniência:(De \textunderscore tani\textunderscore )}
\end{itemize}
Enrolar (fôlhas de tabaco), formando mólhos.
\section{Entanto}
\begin{itemize}
\item {Grp. gram.:adv.}
\end{itemize}
\begin{itemize}
\item {Grp. gram.:Loc. conj.}
\end{itemize}
\begin{itemize}
\item {Grp. gram.:M.}
\end{itemize}
O mesmo que \textunderscore entretanto\textunderscore .
Entrementes.
Neste meio tempo.
\textunderscore No entanto\textunderscore , todavia; não obstante.
Actualidade; tempo, que está decorrendo:«\textunderscore neste entanto...\textunderscore »Camillo, \textunderscore Carrasco\textunderscore , 148.
(Contr. de \textunderscore entretanto\textunderscore )
\section{Então}
\begin{itemize}
\item {Grp. gram.:adv.}
\end{itemize}
\begin{itemize}
\item {Proveniência:(Do lat. \textunderscore tum\textunderscore )}
\end{itemize}
Naquella occasião.
Nesse tempo.
Nesse caso.
Demais, além disso.
\section{Entapetar}
\begin{itemize}
\item {Grp. gram.:v. t.}
\end{itemize}
O mesmo que \textunderscore atapetar\textunderscore . Cf. Arn. Gama, \textunderscore Motim\textunderscore , 155.
\section{Entapizar}
\begin{itemize}
\item {Grp. gram.:v. t.}
\end{itemize}
\begin{itemize}
\item {Utilização:Ext.}
\end{itemize}
O mesmo que \textunderscore tapizar\textunderscore .
Revestir.
Adornar. Cf. Latino, \textunderscore Humboldt\textunderscore , 181.
\section{...entar}
\begin{itemize}
\item {Grp. gram.:suf.}
\end{itemize}
Sufixo de verbos frequentativos, e outras vezes equivalente aos suf. \textunderscore ...ar\textunderscore  e \textunderscore ...ecer\textunderscore .
\section{Entarambècado}
\begin{itemize}
\item {Grp. gram.:adj.}
\end{itemize}
\begin{itemize}
\item {Utilização:Prov.}
\end{itemize}
\begin{itemize}
\item {Utilização:trasm.}
\end{itemize}
Cheio de tarambecos.
\section{Entaramelar}
\begin{itemize}
\item {Grp. gram.:v. t.}
\end{itemize}
\begin{itemize}
\item {Utilização:Pop.}
\end{itemize}
\begin{itemize}
\item {Proveniência:(De \textunderscore taramela\textunderscore )}
\end{itemize}
Embaraçar, enredar.
Tornar hesitante, fazer titubear.
\section{Entardecer}
\begin{itemize}
\item {Grp. gram.:v. i.}
\end{itemize}
\begin{itemize}
\item {Proveniência:(Do lat. \textunderscore tardescere\textunderscore )}
\end{itemize}
Chegar a tarde.
Ir caíndo a tarde.
\section{Entarraxar}
\begin{itemize}
\item {Grp. gram.:v. t.}
\end{itemize}
Segurar com tarraxa; aparafusar.
\section{Entarroado}
\begin{itemize}
\item {Grp. gram.:adj.}
\end{itemize}
\begin{itemize}
\item {Utilização:Fig.}
\end{itemize}
Que tem tarro.
Que tem mucosidades a tornar a voz áspera ou rouca, (falando-se das goelas). Cf. Camillo, \textunderscore Brasileira\textunderscore , 67.
\section{Êntase}
\begin{itemize}
\item {Grp. gram.:f.}
\end{itemize}
\begin{itemize}
\item {Proveniência:(Lat. \textunderscore entasis\textunderscore )}
\end{itemize}
Engrossamento do primeiro terço das columnas, em architectura.
\section{Ente}
\begin{itemize}
\item {Grp. gram.:m.}
\end{itemize}
\begin{itemize}
\item {Proveniência:(Lat. \textunderscore ens\textunderscore , \textunderscore entis\textunderscore )}
\end{itemize}
Aquillo que existe.
Coisa; sêr.
Objecto.
Substância.
Aquillo que suppomos existir.
Cálculo, projecto:«\textunderscore formara entes de razão.\textunderscore »Camillo, \textunderscore Onde está a Fel.\textunderscore , 275.
\section{Enteada}
\textunderscore fem.\textunderscore  de \textunderscore enteado\textunderscore .
\section{Enteado}
\begin{itemize}
\item {Grp. gram.:m.}
\end{itemize}
\begin{itemize}
\item {Proveniência:(Do lat. \textunderscore antenatus\textunderscore )}
\end{itemize}
Relação de parentesco, entre um indivíduo e seu padrasto ou sua madrasta.
\section{Entear}
\begin{itemize}
\item {Grp. gram.:v. t.}
\end{itemize}
\begin{itemize}
\item {Utilização:Fig.}
\end{itemize}
\begin{itemize}
\item {Proveniência:(De \textunderscore teia\textunderscore )}
\end{itemize}
Tecer; converter em teia.
Entrelaçar.
\section{Enteato}
\begin{itemize}
\item {Grp. gram.:adj.}
\end{itemize}
O mesmo que \textunderscore enteu\textunderscore .
\section{Entecer}
\begin{itemize}
\item {Grp. gram.:v. t.}
\end{itemize}
\begin{itemize}
\item {Proveniência:(De \textunderscore tecer\textunderscore )}
\end{itemize}
O mesmo que \textunderscore entretecer\textunderscore .
\section{Entediar}
\begin{itemize}
\item {Grp. gram.:v. t.}
\end{itemize}
\begin{itemize}
\item {Utilização:Fig.}
\end{itemize}
Causar tédio a.
Tornar aborrecido; enjoar.
\section{Entejar}
\begin{itemize}
\item {Grp. gram.:v. t.}
\end{itemize}
\begin{itemize}
\item {Utilização:Des.}
\end{itemize}
\begin{itemize}
\item {Proveniência:(De \textunderscore entejo\textunderscore )}
\end{itemize}
O mesmo que \textunderscore entediar\textunderscore .
\section{Entejo}
\begin{itemize}
\item {Grp. gram.:m.}
\end{itemize}
\begin{itemize}
\item {Utilização:Des.}
\end{itemize}
\begin{itemize}
\item {Proveniência:(Do lat. \textunderscore taedium\textunderscore )}
\end{itemize}
Tédio; nojo; aversão:«\textunderscore faz-me entejo.\textunderscore »Castilho, \textunderscore Sabichonas\textunderscore , 180.
\section{Entejolar}
\begin{itemize}
\item {Grp. gram.:v. i.}
\end{itemize}
\begin{itemize}
\item {Grp. gram.:V. t.}
\end{itemize}
Adquirir a consistência do tejolo, (falando-se do solo das marinhas). Cf. \textunderscore Museu Techn.\textunderscore , 91.
Revestir de tejolo. Dar apparência de tejolo a.
\section{Entelhar}
\begin{itemize}
\item {Grp. gram.:v. t.}
\end{itemize}
\begin{itemize}
\item {Proveniência:(De \textunderscore telha\textunderscore )}
\end{itemize}
O mesmo que \textunderscore imbricar\textunderscore .
\section{Entena}
\begin{itemize}
\item {Grp. gram.:f.}
\end{itemize}
O mesmo que \textunderscore antenna\textunderscore . Cf. \textunderscore Lusíadas\textunderscore , IV; Herculano, \textunderscore Opúsc.\textunderscore , I, 80.
\section{Entenal}
\begin{itemize}
\item {Grp. gram.:m.}
\end{itemize}
Pássaro da África do Sul. Cf. Pimentel, \textunderscore Arte de Navegar\textunderscore .
(Cp. \textunderscore antennal\textunderscore )
\section{Entendedor}
\begin{itemize}
\item {Grp. gram.:m.}
\end{itemize}
\begin{itemize}
\item {Grp. gram.:Adj.}
\end{itemize}
Aquelle que entende.
Intelligente.
\section{Entender}
\begin{itemize}
\item {Grp. gram.:v. t.}
\end{itemize}
\begin{itemize}
\item {Grp. gram.:V. i.}
\end{itemize}
\begin{itemize}
\item {Utilização:Fam.}
\end{itemize}
\begin{itemize}
\item {Utilização:Ant.}
\end{itemize}
\begin{itemize}
\item {Grp. gram.:M.}
\end{itemize}
\begin{itemize}
\item {Proveniência:(Do lat. \textunderscore intendere\textunderscore )}
\end{itemize}
Comprehender; saber perfeitamente.
Julgar, interpretar: \textunderscore entendo que andas mal\textunderscore .
Ouvir.
Meditar.
Cuidar.
Sêr hábil ou perito.
Contender.
Armar rixas, fazer provocações: \textunderscore o rapaz entendeu com o outro\textunderscore .
\textunderscore Entender em\textunderscore , têr relações illícitas com. Cf. \textunderscore Port. Mon. Hist.\textunderscore , \textunderscore Script.\textunderscore , 336.
Opinião, juizo: \textunderscore no meu entender...\textunderscore 
\section{Entendidamente}
\begin{itemize}
\item {Grp. gram.:adv.}
\end{itemize}
\begin{itemize}
\item {Proveniência:(De \textunderscore entendido\textunderscore )}
\end{itemize}
Com intelligência.
\section{Entendido}
\begin{itemize}
\item {Grp. gram.:adj.}
\end{itemize}
Que entende.
Que é sabedor.
Intelligente.
\section{Entendimento}
\begin{itemize}
\item {Grp. gram.:m.}
\end{itemize}
\begin{itemize}
\item {Proveniência:(De \textunderscore entender\textunderscore )}
\end{itemize}
O mesmo que intelligência.
Percepção.
Faculdade de julgar.
\section{Entenebrar}
\begin{itemize}
\item {Grp. gram.:v. t.}
\end{itemize}
O mesmo que \textunderscore entenebrecer\textunderscore :«\textunderscore ...parecia fender-se a terra e os céos entenebrar-se.\textunderscore »Filinto, \textunderscore D. Man.\textunderscore , 83.
\section{Entenebrecer}
\begin{itemize}
\item {Grp. gram.:v. t.}
\end{itemize}
\begin{itemize}
\item {Grp. gram.:V. i.}
\end{itemize}
\begin{itemize}
\item {Proveniência:(Do lat. \textunderscore tenebrescere\textunderscore )}
\end{itemize}
Cercar de trevas.
Ennuvoar.
Entristecer, enlutar.
Encher-se de sombras.
Tornar-se escuro, escurecer.
\section{Entenrecer}
\begin{itemize}
\item {Grp. gram.:v. t.}
\end{itemize}
\begin{itemize}
\item {Grp. gram.:V. i.}
\end{itemize}
Tornar tenro.
Tornar-se tenro.
\section{Entepidecer}
\begin{itemize}
\item {fónica:té}
\end{itemize}
\begin{itemize}
\item {Grp. gram.:v. t.}
\end{itemize}
Tornar tépido. Cf. Arn. Gama, \textunderscore Motim\textunderscore , 47.
\section{Enteradenografia}
\begin{itemize}
\item {Grp. gram.:f.}
\end{itemize}
\begin{itemize}
\item {Utilização:Anat.}
\end{itemize}
Descrição dos gânglios linfáticos intestinaes.
\section{Enteradenográfico}
\begin{itemize}
\item {Grp. gram.:adj.}
\end{itemize}
Relativo á enteradenografia.
\section{Enteradenographia}
\begin{itemize}
\item {Grp. gram.:f.}
\end{itemize}
\begin{itemize}
\item {Utilização:Anat.}
\end{itemize}
Descripção dos gânglios lympháticos intestinaes.
\section{Enteradenográphico}
\begin{itemize}
\item {Grp. gram.:adj.}
\end{itemize}
Relativo á enteradenographia.
\section{Enteradenologia}
\begin{itemize}
\item {Grp. gram.:f.}
\end{itemize}
\begin{itemize}
\item {Proveniência:(Do gr. \textunderscore enteron\textunderscore  + \textunderscore adene\textunderscore  + \textunderscore logos\textunderscore )}
\end{itemize}
Tratado anatómico dos gânglios lympháticos intestinaes.
\section{Enteradenológico}
\begin{itemize}
\item {Grp. gram.:adj.}
\end{itemize}
Relativo a enteradenologia.
\section{Enteralgia}
\begin{itemize}
\item {Grp. gram.:f.}
\end{itemize}
\begin{itemize}
\item {Proveniência:(Do gr. \textunderscore enteron\textunderscore  + \textunderscore algos\textunderscore )}
\end{itemize}
Neuralgia intestinal.
\section{Enteremia}
\begin{itemize}
\item {Grp. gram.:f.}
\end{itemize}
\begin{itemize}
\item {Utilização:Med.}
\end{itemize}
\begin{itemize}
\item {Proveniência:(Do gr. \textunderscore enteron\textunderscore  + \textunderscore haima\textunderscore )}
\end{itemize}
Congestão sanguínea nos intestinos.
\section{Entérico}
\begin{itemize}
\item {Grp. gram.:adj.}
\end{itemize}
\begin{itemize}
\item {Proveniência:(Gr. \textunderscore enterikos\textunderscore )}
\end{itemize}
Relativo aos intestinos, intestinal.
\section{Enterite}
\begin{itemize}
\item {Grp. gram.:f.}
\end{itemize}
\begin{itemize}
\item {Proveniência:(Do gr. \textunderscore enteron\textunderscore )}
\end{itemize}
Inflammação nos intestinos.
\section{Enterítico}
\begin{itemize}
\item {Grp. gram.:adj.}
\end{itemize}
Relativo a enterite.
\section{Enternecedor}
\begin{itemize}
\item {Grp. gram.:adj.}
\end{itemize}
Que enternece.
\section{Enternecer}
\begin{itemize}
\item {Grp. gram.:v. t.}
\end{itemize}
Tornar terno, brando, amoroso, compassivo.
\section{Enternecidamente}
\begin{itemize}
\item {Grp. gram.:adv.}
\end{itemize}
\begin{itemize}
\item {Proveniência:(De \textunderscore enternecer\textunderscore )}
\end{itemize}
Com enternecimento.
\section{Enternecimento}
\begin{itemize}
\item {Grp. gram.:m.}
\end{itemize}
Acto ou effeito de enternecer.
\section{Entero...}
\begin{itemize}
\item {Grp. gram.:pref.}
\end{itemize}
(designativo de intestino)
\section{Enterocele}
\begin{itemize}
\item {Grp. gram.:m.}
\end{itemize}
\begin{itemize}
\item {Proveniência:(Do gr. \textunderscore enteron\textunderscore  + \textunderscore kele\textunderscore )}
\end{itemize}
Hérnia intestinal.
\section{Enterocinese}
\begin{itemize}
\item {Grp. gram.:f.}
\end{itemize}
\begin{itemize}
\item {Proveniência:(Do gr. \textunderscore enteron\textunderscore  + \textunderscore kinesis\textunderscore )}
\end{itemize}
Fermento intestinal, há pouco descoberto e necessário á trypsina do suco pancreático, para que esta se torne activa.
\section{Enterocistocele}
\begin{itemize}
\item {Grp. gram.:m.}
\end{itemize}
\begin{itemize}
\item {Proveniência:(Do gr. \textunderscore enteron\textunderscore  + \textunderscore kustis\textunderscore  + \textunderscore kele\textunderscore )}
\end{itemize}
Hérnia da bexiga, complicada com enterocele.
\section{Enterocolite}
\begin{itemize}
\item {Grp. gram.:f.}
\end{itemize}
\begin{itemize}
\item {Utilização:Med.}
\end{itemize}
\begin{itemize}
\item {Proveniência:(Do gr. \textunderscore entron\textunderscore  + \textunderscore colon\textunderscore )}
\end{itemize}
Nome, que também se dá á enterite, porque esta tem geralmente a sua séde no intestino delgado e no cólon.
\section{Enterocystocele}
\begin{itemize}
\item {Grp. gram.:m.}
\end{itemize}
\begin{itemize}
\item {Proveniência:(Do gr. \textunderscore enteron\textunderscore  + \textunderscore kustis\textunderscore  + \textunderscore kele\textunderscore )}
\end{itemize}
Hérnia da bexiga, complicada com enterocele.
\section{Enterodelo}
\begin{itemize}
\item {Grp. gram.:adj.}
\end{itemize}
\begin{itemize}
\item {Utilização:Zool.}
\end{itemize}
\begin{itemize}
\item {Proveniência:(Do gr. \textunderscore enteron\textunderscore  + \textunderscore delos\textunderscore )}
\end{itemize}
Que tem visível ou distinto um tubo intestinal.
\section{Enterodinia}
\begin{itemize}
\item {Grp. gram.:f.}
\end{itemize}
\begin{itemize}
\item {Proveniência:(Do gr. \textunderscore enteron\textunderscore  + \textunderscore odune\textunderscore )}
\end{itemize}
Dôr intestinal.
Cólica nervosa.
\section{Enterodynia}
\begin{itemize}
\item {Grp. gram.:f.}
\end{itemize}
\begin{itemize}
\item {Proveniência:(Do gr. \textunderscore enteron\textunderscore  + \textunderscore odune\textunderscore )}
\end{itemize}
Dôr intestinal.
Cólica nervosa.
\section{Enterografia}
\begin{itemize}
\item {Grp. gram.:f.}
\end{itemize}
\begin{itemize}
\item {Proveniência:(Do gr. \textunderscore enteron\textunderscore  + \textunderscore graphein\textunderscore )}
\end{itemize}
Descrição anatómica dos intestinos.
\section{Enterographia}
\begin{itemize}
\item {Grp. gram.:f.}
\end{itemize}
\begin{itemize}
\item {Proveniência:(Do gr. \textunderscore enteron\textunderscore  + \textunderscore graphein\textunderscore )}
\end{itemize}
Descripção anatómica dos intestinos.
\section{Entero-hemorrhagia}
\begin{itemize}
\item {Grp. gram.:f.}
\end{itemize}
Hemorrhagia intestinal.
\section{Enterólitho}
\begin{itemize}
\item {Grp. gram.:m.}
\end{itemize}
\begin{itemize}
\item {Proveniência:(Do gr. \textunderscore enteron\textunderscore  + \textunderscore lithos\textunderscore )}
\end{itemize}
Concreção intestinal.
\section{Enterólito}
\begin{itemize}
\item {Grp. gram.:m.}
\end{itemize}
\begin{itemize}
\item {Proveniência:(Do gr. \textunderscore enteron\textunderscore  + \textunderscore lithos\textunderscore )}
\end{itemize}
Concreção intestinal.
\section{Enterologia}
\begin{itemize}
\item {Grp. gram.:f.}
\end{itemize}
\begin{itemize}
\item {Proveniência:(Do gr. \textunderscore enteron\textunderscore  + \textunderscore logos\textunderscore )}
\end{itemize}
Tratado dos intestinos e das funcções intestinaes.
\section{Enteromesentérico}
\begin{itemize}
\item {Grp. gram.:adj.}
\end{itemize}
Relativo aos intestinos e ao mesentério.
\section{Enteropneumatose}
\begin{itemize}
\item {Grp. gram.:f.}
\end{itemize}
Desenvolvimento de considerável quantidade de gases nos intestinos.
\section{Enteropneustas}
\begin{itemize}
\item {Grp. gram.:m. pl.}
\end{itemize}
Vermes enteropneustos.
(Cp. \textunderscore enteropneusto\textunderscore )
\section{Enteropneusto}
\begin{itemize}
\item {Grp. gram.:adj.}
\end{itemize}
\begin{itemize}
\item {Proveniência:(Do gr. \textunderscore enteron\textunderscore  + \textunderscore pneuma\textunderscore )}
\end{itemize}
Diz-se dos vermes que têm respiração interior.
\section{Enteroptose}
\begin{itemize}
\item {Grp. gram.:f.}
\end{itemize}
\begin{itemize}
\item {Utilização:Med.}
\end{itemize}
\begin{itemize}
\item {Proveniência:(Do gr. \textunderscore enteron\textunderscore  + \textunderscore ptosis\textunderscore )}
\end{itemize}
Quéda do cólon transverso.
\section{Enterorrafia}
\begin{itemize}
\item {Grp. gram.:f.}
\end{itemize}
\begin{itemize}
\item {Proveniência:(Do gr. \textunderscore enteron\textunderscore  + \textunderscore rhaphe\textunderscore )}
\end{itemize}
Sutura intestinal, para se ligarem os lábios de uma chaga do intestino.
\section{Enterorragia}
\begin{itemize}
\item {Grp. gram.:f.}
\end{itemize}
O mesmo que \textunderscore melena\textunderscore ^2.
\section{Enterorrhagia}
\begin{itemize}
\item {Grp. gram.:f.}
\end{itemize}
O mesmo que \textunderscore melena\textunderscore ^2.
\section{Enterorrhaphia}
\begin{itemize}
\item {Grp. gram.:f.}
\end{itemize}
\begin{itemize}
\item {Proveniência:(Do gr. \textunderscore enteron\textunderscore  + \textunderscore rhaphe\textunderscore )}
\end{itemize}
Sutura intestinal, para se ligarem os lábios de uma chaga do intestino.
\section{Enterose}
\begin{itemize}
\item {Grp. gram.:f.}
\end{itemize}
\begin{itemize}
\item {Proveniência:(Do gr. \textunderscore enteron\textunderscore )}
\end{itemize}
Designação genérica das doenças intestinaes.
\section{Enterotomia}
\begin{itemize}
\item {Grp. gram.:f.}
\end{itemize}
\begin{itemize}
\item {Utilização:Cir.}
\end{itemize}
\begin{itemize}
\item {Proveniência:(Do gr. \textunderscore enteron\textunderscore  + \textunderscore tome\textunderscore )}
\end{itemize}
Incisão nos intestinos.
\section{Enterótomo}
\begin{itemize}
\item {Grp. gram.:m.}
\end{itemize}
Instrumento, com que se pratica a enterotomia.
\section{Enterozoário}
\begin{itemize}
\item {Grp. gram.:m.}
\end{itemize}
\begin{itemize}
\item {Proveniência:(Do gr. \textunderscore enteron\textunderscore  + \textunderscore zoon\textunderscore )}
\end{itemize}
Helmintho ou larva, que só vive nos intestinos de certos animaes.
\section{Enterração}
\begin{itemize}
\item {Grp. gram.:f.}
\end{itemize}
(V.enterramento)
\section{Enterrador}
\begin{itemize}
\item {Grp. gram.:adj.}
\end{itemize}
\begin{itemize}
\item {Grp. gram.:M.}
\end{itemize}
Que enterra.
Aquelle que enterra.
\section{Enterramento}
\begin{itemize}
\item {Grp. gram.:m.}
\end{itemize}
Acto ou effeito de enterrar.
\section{Enterrar}
\begin{itemize}
\item {Grp. gram.:v. t.}
\end{itemize}
\begin{itemize}
\item {Utilização:Fig.}
\end{itemize}
\begin{itemize}
\item {Grp. gram.:Loc.}
\end{itemize}
\begin{itemize}
\item {Utilização:escol.}
\end{itemize}
\begin{itemize}
\item {Proveniência:(De \textunderscore terra\textunderscore )}
\end{itemize}
Meter debaixo da terra.
Sepultar: \textunderscore enterrar os mortos\textunderscore .
Esconder na terra: \textunderscore enterrar dinheiro\textunderscore .
Embeber, introduzir, profundamente: \textunderscore enterrar uma estaca\textunderscore .
Espetar.
Causar a morte de.
Supplantar ou vencer, discutindo; desacreditar.
\textunderscore Enterrar o anno\textunderscore , festejar com ceia ou jantar a approvação num curso ou o acabamento do anno escolar.
\section{Enterreirar}
\begin{itemize}
\item {Grp. gram.:v. t.}
\end{itemize}
\begin{itemize}
\item {Utilização:Fig.}
\end{itemize}
\begin{itemize}
\item {Proveniência:(De \textunderscore terreiro\textunderscore )}
\end{itemize}
Converter em terreiro.
Aplanar.
Predispor favoravelmente.
\section{Enterrido}
\begin{itemize}
\item {Grp. gram.:m.}
\end{itemize}
\begin{itemize}
\item {Utilização:Prov.}
\end{itemize}
\begin{itemize}
\item {Utilização:dur.}
\end{itemize}
O mesmo que \textunderscore entêrro\textunderscore .
\section{Enterrio}
\begin{itemize}
\item {Grp. gram.:m.}
\end{itemize}
\begin{itemize}
\item {Utilização:Prov.}
\end{itemize}
O mesmo que \textunderscore entêrro\textunderscore .
\section{Entêrro}
\begin{itemize}
\item {Grp. gram.:m.}
\end{itemize}
Acto de enterrar.
Préstito fúnebre.
\section{Entesadura}
\begin{itemize}
\item {Grp. gram.:f.}
\end{itemize}
Acto ou effeito de entesar.
\section{Entesar}
\begin{itemize}
\item {Grp. gram.:v. t.}
\end{itemize}
\begin{itemize}
\item {Proveniência:(De \textunderscore teso\textunderscore )}
\end{itemize}
Fazer teso.
Tornar direito, tenso: \textunderscore entesar uma corda\textunderscore .
Fortalecer.
Tornar altivo.
Dar entono a.
\section{Entesoamento}
\begin{itemize}
\item {Grp. gram.:m.}
\end{itemize}
Arrogância. Cf. Camillo, \textunderscore Dôze Casam.\textunderscore , 99.
\section{Entesoirador}
\begin{itemize}
\item {Grp. gram.:adj.}
\end{itemize}
\begin{itemize}
\item {Grp. gram.:M.}
\end{itemize}
Que entesoira.
Aquele que entesoira.
\section{Entesoiramento}
\begin{itemize}
\item {Grp. gram.:m.}
\end{itemize}
Acto ou efeito de \textunderscore entesoirar\textunderscore .
\section{Entesoirar}
\begin{itemize}
\item {Grp. gram.:v. t.}
\end{itemize}
\begin{itemize}
\item {Proveniência:(De \textunderscore tesoiro\textunderscore )}
\end{itemize}
Converter em tesoiro.
Acumular; amontoar.
Têr em depósito (objectos de grande valor).
\section{Entestar}
\begin{itemize}
\item {Grp. gram.:v. i.}
\end{itemize}
\begin{itemize}
\item {Proveniência:(De \textunderscore testa\textunderscore )}
\end{itemize}
Defrontar.
Confinar.
Sêr limítrophe.
\section{Entesteferrar}
\begin{itemize}
\item {Grp. gram.:v. i.}
\end{itemize}
\begin{itemize}
\item {Utilização:Prov.}
\end{itemize}
\begin{itemize}
\item {Utilização:trasm.}
\end{itemize}
\begin{itemize}
\item {Proveniência:(De \textunderscore testa\textunderscore  + \textunderscore ferro\textunderscore )}
\end{itemize}
Insistir, teimar.
\section{Enteu}
\begin{itemize}
\item {Grp. gram.:adj.}
\end{itemize}
\begin{itemize}
\item {Proveniência:(Gr. \textunderscore entheos\textunderscore )}
\end{itemize}
Inspirado por Deus.
Cheio de amor divino.
\section{Entheato}
\begin{itemize}
\item {Grp. gram.:adj.}
\end{itemize}
O mesmo que \textunderscore entheu\textunderscore .
\section{Enthesoirador}
\begin{itemize}
\item {Grp. gram.:adj.}
\end{itemize}
\begin{itemize}
\item {Grp. gram.:M.}
\end{itemize}
Que enthesoira.
Aquelle que enthesoira.
\section{Enthesoiramento}
\begin{itemize}
\item {Grp. gram.:m.}
\end{itemize}
Acto ou effeito de \textunderscore enthesoirar\textunderscore .
\section{Enthesoirar}
\begin{itemize}
\item {Grp. gram.:v. t.}
\end{itemize}
\begin{itemize}
\item {Proveniência:(De \textunderscore thesoiro\textunderscore )}
\end{itemize}
Converter em thesoiro.
Acumular; amontoar.
Têr em depósito (objectos de grande valor).
\section{Enthesourador}
\begin{itemize}
\item {Grp. gram.:adj.}
\end{itemize}
\begin{itemize}
\item {Grp. gram.:M.}
\end{itemize}
Que enthesoura.
Aquelle que enthesoura.
\section{Enthesouramento}
\begin{itemize}
\item {Grp. gram.:m.}
\end{itemize}
Acto ou effeito de \textunderscore enthesourar\textunderscore .
\section{Enthesourar}
\begin{itemize}
\item {Grp. gram.:v. t.}
\end{itemize}
\begin{itemize}
\item {Proveniência:(De \textunderscore thesouro\textunderscore )}
\end{itemize}
Converter em thesouro.
Acumular; amontoar.
Têr em depósito (objectos de grande valor).
\section{Entheu}
\begin{itemize}
\item {Grp. gram.:adj.}
\end{itemize}
\begin{itemize}
\item {Proveniência:(Gr. \textunderscore entheos\textunderscore )}
\end{itemize}
Inspirado por Deus.
Cheio de amor divino.
\section{Enthronar}
\textunderscore v. t.\textunderscore  (e der.)
O mesmo que \textunderscore enthronizar\textunderscore , etc.
\section{Enthronear}
\textunderscore v. t.\textunderscore  (e der.)
O mesmo que \textunderscore enthronar\textunderscore , etc. Cf. \textunderscore Eufrosina\textunderscore , pról.
\section{Enthronização}
\begin{itemize}
\item {Grp. gram.:f.}
\end{itemize}
Acto ou effeito de \textunderscore enthronizar\textunderscore .
\section{Enthronizar}
\begin{itemize}
\item {Grp. gram.:v. t.}
\end{itemize}
\begin{itemize}
\item {Utilização:Fig.}
\end{itemize}
\begin{itemize}
\item {Proveniência:(De \textunderscore throno\textunderscore )}
\end{itemize}
Pôr no throno.
Exaltar; elevar muito.
\section{Enthusiasmado}
\begin{itemize}
\item {Grp. gram.:adj.}
\end{itemize}
\begin{itemize}
\item {Utilização:Fig.}
\end{itemize}
Que tem enthusiasmo.
Animado por bom êxito já obtido.
\section{Enthusiasmar}
\begin{itemize}
\item {Grp. gram.:v. t.}
\end{itemize}
Causar enthusiasmo ou admiração a.
\section{Enthusiasmo}
\begin{itemize}
\item {Grp. gram.:m.}
\end{itemize}
\begin{itemize}
\item {Utilização:Ant.}
\end{itemize}
\begin{itemize}
\item {Proveniência:(Gr. \textunderscore enthousiasmos\textunderscore )}
\end{itemize}
Excitação da alma, quando admira excessivamente.
Arrebatamento.
Paixão viva.
Alegria ruidosa.
Furor ou arrebatamento desordenado, que se atribuía a inspirações divinas.
\section{Enthusiasta}
\begin{itemize}
\item {Grp. gram.:m.  e  adj.}
\end{itemize}
\begin{itemize}
\item {Proveniência:(Gr. \textunderscore enthousiastes\textunderscore )}
\end{itemize}
O que se enthusiasma.
Quem se dedica vivamente: \textunderscore enthusiasta pela música\textunderscore .
Quem se exprime com enthusiasmo.
\section{Enthusiasticamente}
\begin{itemize}
\item {Grp. gram.:adv.}
\end{itemize}
De modo enthusiástico.
\section{Enthusiástico}
\begin{itemize}
\item {Grp. gram.:adj.}
\end{itemize}
\begin{itemize}
\item {Proveniência:(De \textunderscore enthusiasta\textunderscore )}
\end{itemize}
Em que há enthusiásmo: \textunderscore saudações enthusiásticas\textunderscore .
\section{Enthymema}
\begin{itemize}
\item {Grp. gram.:m.}
\end{itemize}
\begin{itemize}
\item {Proveniência:(Gr. \textunderscore enthumema\textunderscore )}
\end{itemize}
Syllogismo com duas proposições, chamada uma a antecedente, e outra a conseqüente.
\section{Enthymemático}
\begin{itemize}
\item {Grp. gram.:adj.}
\end{itemize}
Relativo a enthymema.
\section{Entibecer}
\begin{itemize}
\item {Grp. gram.:v. i.}
\end{itemize}
\begin{itemize}
\item {Proveniência:(De \textunderscore tíbio\textunderscore )}
\end{itemize}
Entibiar-se; afroixar.
\section{Entibiamento}
\begin{itemize}
\item {Grp. gram.:m.}
\end{itemize}
O mesmo que \textunderscore tibieza\textunderscore .
\section{Entibiar}
\begin{itemize}
\item {Grp. gram.:v. t.}
\end{itemize}
\begin{itemize}
\item {Utilização:Fig.}
\end{itemize}
\begin{itemize}
\item {Grp. gram.:V. i.}
\end{itemize}
Fazer tíbio, froixo, morno.
Tornar menor a energia de.
Resfriar o enthusiasmo de.
Tornar-se tíbio.
\section{Entica}
\begin{itemize}
\item {Grp. gram.:f.}
\end{itemize}
\begin{itemize}
\item {Utilização:Bras. de Minas}
\end{itemize}
\begin{itemize}
\item {Proveniência:(De \textunderscore enticar-se\textunderscore )}
\end{itemize}
Provocação.
Debique.
\section{Enticar-se}
\begin{itemize}
\item {Grp. gram.:v. p.}
\end{itemize}
\begin{itemize}
\item {Utilização:Prov.}
\end{itemize}
Travar-se de razões; altercar.
Brigar.
\section{Entidade}
\begin{itemize}
\item {Grp. gram.:f.}
\end{itemize}
\begin{itemize}
\item {Proveniência:(Lat. \textunderscore entitas\textunderscore )}
\end{itemize}
Aquillo que constitue a existência de uma coisa.
Existência, considerada independente do respectivo objecto.
Individualidade.
Ente.
Aquillo que existe, real ou imaginariamente.
Importância.
\section{Entijolar}
\begin{itemize}
\item {Grp. gram.:v. t.  e  i.}
\end{itemize}
(V.entejolar)
\section{Entijucar}
\begin{itemize}
\item {Grp. gram.:v. t.}
\end{itemize}
\begin{itemize}
\item {Utilização:Bras}
\end{itemize}
\begin{itemize}
\item {Proveniência:(De \textunderscore tijuca\textunderscore )}
\end{itemize}
Enlamear.
\section{Entimema}
\begin{itemize}
\item {Grp. gram.:m.}
\end{itemize}
\begin{itemize}
\item {Proveniência:(Gr. \textunderscore enthumema\textunderscore )}
\end{itemize}
Silogismo com duas proposições, chamada uma a antecedente, e outra a consequente.
\section{Entimemático}
\begin{itemize}
\item {Grp. gram.:adj.}
\end{itemize}
Relativo a entimema.
\section{Entisicar}
\begin{itemize}
\item {Grp. gram.:v. t.}
\end{itemize}
\begin{itemize}
\item {Utilização:Fig.}
\end{itemize}
\begin{itemize}
\item {Grp. gram.:V. i.}
\end{itemize}
Tornar tísico.
Importunar.
Tornar-se tísico.
\section{Entirrado}
\begin{itemize}
\item {Grp. gram.:adj.}
\end{itemize}
\begin{itemize}
\item {Utilização:Ant.}
\end{itemize}
Teimoso, obstinado.
\section{Entivação}
\begin{itemize}
\item {Grp. gram.:f.}
\end{itemize}
Acto de entivar.
\section{Entivar}
\begin{itemize}
\item {Grp. gram.:v.}
\end{itemize}
\begin{itemize}
\item {Utilização:t. Constr.}
\end{itemize}
Revestir de tábuas: \textunderscore entivar as paredes de um poço\textunderscore .
\section{Entivar}
\textunderscore v. t.\textunderscore  (e der.)
(V. \textunderscore estivar\textunderscore ^1, etc.)
\section{...ento}
\begin{itemize}
\item {Grp. gram.:suf. adj.}
\end{itemize}
\begin{itemize}
\item {Proveniência:(Do lat. \textunderscore ...entus\textunderscore )}
\end{itemize}
(Designativo de frequência, intensidade, etc.)
\section{Entoação}
\begin{itemize}
\item {Grp. gram.:f.}
\end{itemize}
\begin{itemize}
\item {Utilização:Ext.}
\end{itemize}
Acto de entoar.
Modulação na voz de quem fala ou recita.
\section{Entoador}
\begin{itemize}
\item {Grp. gram.:adj.}
\end{itemize}
\begin{itemize}
\item {Grp. gram.:M.}
\end{itemize}
Que entôa.
Aquelle que entôa.
\section{Entoalhado}
\begin{itemize}
\item {Grp. gram.:adj.}
\end{itemize}
Que traz ou usa toalha e véu, (falando-se de freiras). Cf. Filinto, X, 122 e 136.
\section{Entoar}
\begin{itemize}
\item {Grp. gram.:v. t.}
\end{itemize}
\begin{itemize}
\item {Utilização:Fig.}
\end{itemize}
\begin{itemize}
\item {Proveniência:(De \textunderscore tom\textunderscore )}
\end{itemize}
Dar tom a, iniciar (um canto).
Cantar afinadamente.
Encaminhar, dirigir.
\section{Entoar}
\begin{itemize}
\item {Grp. gram.:v. i.}
\end{itemize}
\begin{itemize}
\item {Utilização:Prov.}
\end{itemize}
\begin{itemize}
\item {Proveniência:(Do lat. hyp. \textunderscore intonare\textunderscore )}
\end{itemize}
Diz-se do furão que, entrando na lura ou cova de um coêlho, não sái de lá.
Parar assustado.
Embaçar, entupir.
\section{Entocar}
\begin{itemize}
\item {Grp. gram.:v. t.}
\end{itemize}
Meter em toca, em cova.
\section{Entocéfalo}
\begin{itemize}
\item {Grp. gram.:m.}
\end{itemize}
\begin{itemize}
\item {Proveniência:(Do gr. \textunderscore entos\textunderscore  + \textunderscore kephale\textunderscore )}
\end{itemize}
Uma das peças da cabeça dos hexápodes.
\section{Entocéphalo}
\begin{itemize}
\item {Grp. gram.:m.}
\end{itemize}
\begin{itemize}
\item {Proveniência:(Do gr. \textunderscore entos\textunderscore  + \textunderscore kephale\textunderscore )}
\end{itemize}
Uma das peças da cabeça dos hexápodes.
\section{Entocuneiforme}
\begin{itemize}
\item {Grp. gram.:adj.}
\end{itemize}
\begin{itemize}
\item {Utilização:Anat.}
\end{itemize}
\begin{itemize}
\item {Proveniência:(Do gr. \textunderscore entos\textunderscore  + lat. \textunderscore cuneiformis\textunderscore )}
\end{itemize}
Diz-se do mais interno dos três ossos cuneiformes, alinhados trasversalmente no peito do pé.
\section{Entoderme}
\begin{itemize}
\item {Grp. gram.:m.}
\end{itemize}
\begin{itemize}
\item {Proveniência:(Do gr. \textunderscore entos\textunderscore  + \textunderscore derma\textunderscore )}
\end{itemize}
O mesmo que \textunderscore hypoblasto\textunderscore .
\section{Entodiscal}
\begin{itemize}
\item {Grp. gram.:adj.}
\end{itemize}
\begin{itemize}
\item {Utilização:Bot.}
\end{itemize}
\begin{itemize}
\item {Proveniência:(Do gr. \textunderscore entos\textunderscore  + \textunderscore diskos\textunderscore )}
\end{itemize}
Situado dentro do disco.
\section{Entogastro}
\begin{itemize}
\item {Grp. gram.:m.}
\end{itemize}
\begin{itemize}
\item {Utilização:Zool.}
\end{itemize}
\begin{itemize}
\item {Proveniência:(Do gr. \textunderscore entos\textunderscore  + \textunderscore gaster\textunderscore )}
\end{itemize}
Uma das peças do abdome dos insectos.
\section{Entohyal}
\begin{itemize}
\item {Grp. gram.:m.}
\end{itemize}
\begin{itemize}
\item {Proveniência:(Do gr. \textunderscore entos\textunderscore , com o suf. \textunderscore ...hyal\textunderscore )}
\end{itemize}
Peça, que fica por trás do basihyal.
\section{Entoial}
\begin{itemize}
\item {fónica:to-i}
\end{itemize}
\begin{itemize}
\item {Grp. gram.:m.}
\end{itemize}
\begin{itemize}
\item {Proveniência:(Do gr. \textunderscore entos\textunderscore , com o suf. \textunderscore ...hyal\textunderscore )}
\end{itemize}
Peça, que fica por trás do basihyal.
\section{Entoiçar}
\begin{itemize}
\item {Grp. gram.:v. i.}
\end{itemize}
\begin{itemize}
\item {Utilização:Fig.}
\end{itemize}
Criar toiça.
Tornar-se robusto.
\section{Entoiceirar}
\begin{itemize}
\item {Grp. gram.:v. t.}
\end{itemize}
(V.entoiçar)
\section{Entoirar}
\begin{itemize}
\item {Grp. gram.:v. t.}
\end{itemize}
\begin{itemize}
\item {Utilização:Prov.}
\end{itemize}
\begin{itemize}
\item {Utilização:alent.}
\end{itemize}
\begin{itemize}
\item {Proveniência:(De \textunderscore toiro\textunderscore . Cp. \textunderscore embezerrar\textunderscore )}
\end{itemize}
Zangar-se, amuar.
Embezerrar-se; encamelar.
\section{Entoirido}
\begin{itemize}
\item {Grp. gram.:adj.}
\end{itemize}
\begin{itemize}
\item {Utilização:Prov.}
\end{itemize}
\begin{itemize}
\item {Proveniência:(De \textunderscore entoirir\textunderscore )}
\end{itemize}
Enfartado.
Empachado, empaturrado.
\section{Entoirido}
\begin{itemize}
\item {Grp. gram.:adj.}
\end{itemize}
\begin{itemize}
\item {Proveniência:(De \textunderscore toiro\textunderscore )}
\end{itemize}
Bravo como um toiro. Cf. Camillo, \textunderscore Cav. em Ruínas\textunderscore , 128,
\section{Entoirir}
\begin{itemize}
\item {Grp. gram.:v. t.}
\end{itemize}
\begin{itemize}
\item {Utilização:Prov.}
\end{itemize}
\begin{itemize}
\item {Grp. gram.:V. t.}
\end{itemize}
Engordar ou inchar como um toiro.
Encher.
Empaturrar?:«\textunderscore entoiriram o ânimo do principe com iguarias indigestas.\textunderscore »Camillo, \textunderscore Carrasco\textunderscore , 81.
\section{Entojar}
\begin{itemize}
\item {Grp. gram.:V. t.}
\end{itemize}
Causar nojo ou repugnância a; anojar:«\textunderscore nenhum destes manjares os entoja\textunderscore ». Filinto, \textunderscore D. Man.\textunderscore , 149.--Se no texto cit. não há êrro typogr., \textunderscore entojar\textunderscore  é corr. de \textunderscore entejar\textunderscore .
(V. \textunderscore entejar\textunderscore .)
\section{Entojo}
\begin{itemize}
\item {Grp. gram.:m.}
\end{itemize}
(V. \textunderscore antojo\textunderscore ^1)
\section{Entolecer}
\begin{itemize}
\item {Grp. gram.:v. i.}
\end{itemize}
Tornar-se tolo.
Endoidecer.
\section{Entolhar}
\begin{itemize}
\item {Grp. gram.:v. t.}
\end{itemize}
\begin{itemize}
\item {Utilização:Prov.}
\end{itemize}
\begin{itemize}
\item {Utilização:trasm.}
\end{itemize}
O mesmo que \textunderscore antojar\textunderscore .
\section{Entôlho}
\begin{itemize}
\item {Grp. gram.:m.}
\end{itemize}
O mesmo que \textunderscore antojo\textunderscore ^1.
\section{Entómico}
\begin{itemize}
\item {Grp. gram.:adj.}
\end{itemize}
\begin{itemize}
\item {Proveniência:(Do gr. \textunderscore entomon\textunderscore )}
\end{itemize}
Relativo a insectos.
\section{Entomologia}
\begin{itemize}
\item {Grp. gram.:f.}
\end{itemize}
\begin{itemize}
\item {Proveniência:(Do gr. \textunderscore entomon\textunderscore  + \textunderscore logos\textunderscore )}
\end{itemize}
Tratado dos insectos.
\section{Entomológico}
\begin{itemize}
\item {Grp. gram.:adj.}
\end{itemize}
Relativo a entomologia.
\section{Entofilocarpo}
\begin{itemize}
\item {Grp. gram.:adj.}
\end{itemize}
\begin{itemize}
\item {Utilização:Bot.}
\end{itemize}
\begin{itemize}
\item {Proveniência:(Do gr. \textunderscore entos\textunderscore  + \textunderscore phullon\textunderscore  + \textunderscore carpos\textunderscore )}
\end{itemize}
Cuja fructificação se realiza no meio das fôlhas.
\section{Entófito}
\begin{itemize}
\item {Grp. gram.:m.}
\end{itemize}
\begin{itemize}
\item {Proveniência:(Do gr. \textunderscore entus\textunderscore  + \textunderscore phuton\textunderscore )}
\end{itemize}
Vegetal, que se desenvolve no próprio tecido de uma planta vivaz.
\section{Entoftalmia}
\begin{itemize}
\item {Grp. gram.:f.}
\end{itemize}
\begin{itemize}
\item {Proveniência:(Do gr. \textunderscore entos\textunderscore  + \textunderscore phthalmos\textunderscore )}
\end{itemize}
Inflamação das partes internas do ôlho.
\section{Entomófago}
\begin{itemize}
\item {Grp. gram.:adj.}
\end{itemize}
\begin{itemize}
\item {Grp. gram.:M. pl.}
\end{itemize}
\begin{itemize}
\item {Proveniência:(Do gr. \textunderscore entomon\textunderscore  + \textunderscore phagein\textunderscore )}
\end{itemize}
Que se alimenta de insectos.
Secção dos insectos himenópteros.
\section{Entomologista}
\begin{itemize}
\item {Grp. gram.:m.}
\end{itemize}
Aquelle que trata de entomologia.
\section{Entomólogo}
\begin{itemize}
\item {Grp. gram.:m.}
\end{itemize}
Aquelle que é versado em entomologia.
\section{Entomóphago}
\begin{itemize}
\item {Grp. gram.:adj.}
\end{itemize}
\begin{itemize}
\item {Grp. gram.:M. pl.}
\end{itemize}
\begin{itemize}
\item {Proveniência:(Do gr. \textunderscore entomon\textunderscore  + \textunderscore phagein\textunderscore )}
\end{itemize}
Que se alimenta de insectos.
Secção dos insectos hymenópteros.
\section{Entomostráceos}
\begin{itemize}
\item {Grp. gram.:m. pl.}
\end{itemize}
\begin{itemize}
\item {Proveniência:(Do gr. \textunderscore entomon\textunderscore  + \textunderscore ostreon\textunderscore )}
\end{itemize}
Animaes crustáceos de corpo molle, e geralmente parasitos.
\section{Entomozoários}
\begin{itemize}
\item {Grp. gram.:m. pl.}
\end{itemize}
\begin{itemize}
\item {Utilização:Zool.}
\end{itemize}
\begin{itemize}
\item {Proveniência:(Do gr. \textunderscore entomon\textunderscore  + \textunderscore zoarion\textunderscore )}
\end{itemize}
Grupo de animaes, que correspondem proximamente aos articulados.
\section{Entonar}
\begin{itemize}
\item {Grp. gram.:v. t.}
\end{itemize}
\begin{itemize}
\item {Grp. gram.:V. p.}
\end{itemize}
\begin{itemize}
\item {Utilização:Des.}
\end{itemize}
Erguer altivamente.
Ostentar.

Amuar-se, julgar-se offendido.
\section{Entonces}
\begin{itemize}
\item {Grp. gram.:adv.}
\end{itemize}
\begin{itemize}
\item {Utilização:Ant.}
\end{itemize}
\begin{itemize}
\item {Proveniência:(T. cast.)}
\end{itemize}
O mesmo que \textunderscore então\textunderscore . Cf. G. Vicente, \textunderscore Inês Pereira\textunderscore .
\section{Entono}
\begin{itemize}
\item {Grp. gram.:m.}
\end{itemize}
\begin{itemize}
\item {Proveniência:(De \textunderscore entonar\textunderscore )}
\end{itemize}
Soberba.
Altivez.
Vaidade; orgulho.
\section{Entontecedor}
\begin{itemize}
\item {Grp. gram.:adj.}
\end{itemize}
Que faz entontecer.
\section{Entontecer}
\begin{itemize}
\item {Grp. gram.:v. t.}
\end{itemize}
\begin{itemize}
\item {Grp. gram.:V. i.}
\end{itemize}
Tornar tonto, imbecil.
Têr tonturas.
Tornar-se tonto, imbecil.
\section{Entontecimento}
\begin{itemize}
\item {Grp. gram.:m.}
\end{itemize}
Acto ou effeito de entontecer.
\section{Entoparasito}
\begin{itemize}
\item {Grp. gram.:m.}
\end{itemize}
Animal parasito, que vive no interior de um organismo.
\section{Entophtalmia}
\begin{itemize}
\item {Grp. gram.:f.}
\end{itemize}
\begin{itemize}
\item {Proveniência:(Do gr. \textunderscore entos\textunderscore  + \textunderscore phthalmos\textunderscore )}
\end{itemize}
Inflammação das partes internas do ôlho.
\section{Entophyllocarpo}
\begin{itemize}
\item {Grp. gram.:adj.}
\end{itemize}
\begin{itemize}
\item {Utilização:Bot.}
\end{itemize}
\begin{itemize}
\item {Proveniência:(Do gr. \textunderscore entos\textunderscore  + \textunderscore phullon\textunderscore  + \textunderscore carpos\textunderscore )}
\end{itemize}
Cuja fructificação se realiza no meio das fôlhas.
\section{Entóphito}
\begin{itemize}
\item {Grp. gram.:m.}
\end{itemize}
\begin{itemize}
\item {Proveniência:(Do gr. \textunderscore entus\textunderscore  + \textunderscore phuton\textunderscore )}
\end{itemize}
Vegetal, que se desenvolve no próprio tecido de uma planta vivaz.
\section{Entopógonos}
\begin{itemize}
\item {Grp. gram.:m. pl.}
\end{itemize}
\begin{itemize}
\item {Proveniência:(Do gr. \textunderscore ontos\textunderscore  + \textunderscore pogon\textunderscore )}
\end{itemize}
Família de musgos.
\section{Entornadura}
\begin{itemize}
\item {Grp. gram.:f.}
\end{itemize}
Acto ou effeito de entornar.
\section{Entornar}
\begin{itemize}
\item {Grp. gram.:v. t.}
\end{itemize}
\begin{itemize}
\item {Utilização:Fig.}
\end{itemize}
\begin{itemize}
\item {Utilização:Pop.}
\end{itemize}
\begin{itemize}
\item {Proveniência:(Do lat. \textunderscore tornare\textunderscore )}
\end{itemize}
Emborcar, despejando.
Derramar.
Extravasar, fazer trasbordar.
Diffundir.
Desperdiçar.
Beber muito.
\section{Entorneiro}
\begin{itemize}
\item {Grp. gram.:m.}
\end{itemize}
\begin{itemize}
\item {Utilização:Pop.}
\end{itemize}
\begin{itemize}
\item {Proveniência:(De \textunderscore entornar\textunderscore )}
\end{itemize}
Grande porção de água ou de outro líquido, entornada pelo chão.
\section{Entorpar}
\begin{itemize}
\item {Grp. gram.:v. t.}
\end{itemize}
\begin{itemize}
\item {Utilização:Des.}
\end{itemize}
O mesmo que \textunderscore entorpecer\textunderscore .
\section{Entorpecer}
\begin{itemize}
\item {Grp. gram.:v. t.}
\end{itemize}
\begin{itemize}
\item {Grp. gram.:V. i.}
\end{itemize}
\begin{itemize}
\item {Proveniência:(Lat. \textunderscore torpescere\textunderscore )}
\end{itemize}
Causar torpor a.
Suspender, impedir ou retardar o movimento de.
Ennervar; tirar a energia a.
Têr torpor: \textunderscore as pernas entorpecem\textunderscore .
Enfraquecer; desalentar-se.
Desfallecer.
\section{Entorpecimento}
\begin{itemize}
\item {Grp. gram.:m.}
\end{itemize}
Acto ou effeito de entorpecer.
\section{Entorroar}
\begin{itemize}
\item {Grp. gram.:v. t.}
\end{itemize}
Converter em torrões.
\section{Entorse}
\begin{itemize}
\item {Grp. gram.:f.}
\end{itemize}
\begin{itemize}
\item {Utilização:Neol.}
\end{itemize}
\begin{itemize}
\item {Proveniência:(Fr. \textunderscore entorse\textunderscore )}
\end{itemize}
Distensão repentina e violenta dos ligamentos que cercam as articulações.
\section{Entortadura}
\begin{itemize}
\item {Grp. gram.:f.}
\end{itemize}
Acto ou effeito de entortar.
\section{Entortar}
\begin{itemize}
\item {Grp. gram.:v. t.}
\end{itemize}
\begin{itemize}
\item {Grp. gram.:V. i.}
\end{itemize}
Tornar torto.
Dobrar, recurvar: \textunderscore entortar uma vara\textunderscore .
Desviar do caminho direito.
Andar torto.
\section{Entosthymênios}
\begin{itemize}
\item {Grp. gram.:m. pl.}
\end{itemize}
\begin{itemize}
\item {Utilização:Bot.}
\end{itemize}
\begin{itemize}
\item {Proveniência:(Do gr. \textunderscore entosthen\textunderscore  + \textunderscore humen\textunderscore )}
\end{itemize}
Gênero de musgos acrocarpos.
\section{Entostimênios}
\begin{itemize}
\item {Grp. gram.:m. pl.}
\end{itemize}
\begin{itemize}
\item {Utilização:Bot.}
\end{itemize}
\begin{itemize}
\item {Proveniência:(Do gr. \textunderscore entosthen\textunderscore  + \textunderscore humen\textunderscore )}
\end{itemize}
Gênero de musgos acrocarpos.
\section{Entoucado}
\begin{itemize}
\item {Grp. gram.:m.}
\end{itemize}
\begin{itemize}
\item {Utilização:Ant.}
\end{itemize}
O mesmo que \textunderscore toucado\textunderscore .
\section{Entoucar}
\begin{itemize}
\item {Grp. gram.:v.}
\end{itemize}
\begin{itemize}
\item {Utilização:t. Náut.}
\end{itemize}
\begin{itemize}
\item {Proveniência:(De \textunderscore touca\textunderscore )}
\end{itemize}
Dar a amarra voltas nos braços de (âncora).
\section{Entourada}
\begin{itemize}
\item {Grp. gram.:adj. f.}
\end{itemize}
\begin{itemize}
\item {Utilização:Prov.}
\end{itemize}
\begin{itemize}
\item {Utilização:trasm.}
\end{itemize}
Perra, diffícil de abrir, (falando-se de uma porta).
\section{Entourido}
\begin{itemize}
\item {Grp. gram.:adj.}
\end{itemize}
\begin{itemize}
\item {Utilização:Prov.}
\end{itemize}
\begin{itemize}
\item {Proveniência:(De \textunderscore entourir\textunderscore )}
\end{itemize}
Enfartado.
Empachado, empaturrado.
\section{Entourir}
\begin{itemize}
\item {Grp. gram.:v. t.}
\end{itemize}
\begin{itemize}
\item {Utilização:Prov.}
\end{itemize}
\begin{itemize}
\item {Grp. gram.:V. t.}
\end{itemize}
Engordar ou inchar como um touro.
Encher.
Empaturrar?:«\textunderscore entouriram o ânimo do principe com iguarias indigestas.\textunderscore »Camillo, \textunderscore Carrasco\textunderscore , 81.
\section{Entoxicação}
\begin{itemize}
\item {Grp. gram.:f.}
\end{itemize}
Acto de entoxicar.
\section{Entoxicante}
\begin{itemize}
\item {Grp. gram.:adj.}
\end{itemize}
Que entoxica.
\section{Entoxicar}
\begin{itemize}
\item {Grp. gram.:v. t.}
\end{itemize}
\begin{itemize}
\item {Proveniência:(De \textunderscore tóxico\textunderscore )}
\end{itemize}
O mesmo que \textunderscore envenenar\textunderscore .
\section{Entozoários}
\begin{itemize}
\item {Grp. gram.:m. pl.}
\end{itemize}
\begin{itemize}
\item {Proveniência:(Do gr. \textunderscore entos\textunderscore  + \textunderscore zoarion\textunderscore )}
\end{itemize}
Animaes, que vivem no corpo de outros.
Vermes intestinaes.
\section{Entozoologia}
\begin{itemize}
\item {Grp. gram.:f.}
\end{itemize}
\begin{itemize}
\item {Proveniência:(Do gr. \textunderscore entos\textunderscore  + \textunderscore zoon\textunderscore  + \textunderscore logos\textunderscore )}
\end{itemize}
Tratado dos animaes, que vivem no corpo de outros.
\section{Entreabrir}
\begin{itemize}
\item {Grp. gram.:v. t.}
\end{itemize}
\begin{itemize}
\item {Grp. gram.:V. i.}
\end{itemize}
\begin{itemize}
\item {Proveniência:(Lat. \textunderscore interaperire\textunderscore )}
\end{itemize}
Abrir pouco, de manso.
Desabrochar.
Aclarar-se.
\section{Entrada}
\begin{itemize}
\item {Grp. gram.:f.}
\end{itemize}
\begin{itemize}
\item {Utilização:Prov.}
\end{itemize}
Acto ou effeito de entrar: \textunderscore a sua entrada no Parlamento\textunderscore .
Lugar, por onde se entra: \textunderscore á entrada da casa\textunderscore .
Abertura, bôca.
Aquillo que se dá para entrar: \textunderscore pagou a sua entrada\textunderscore .
Começo.
Invasão: \textunderscore a entrada dos Franceses\textunderscore .
Boas relações, familiaridade.
Producto da venda dos bilhetes para entrada num espectáculo: \textunderscore hoje a entrada foi fraca\textunderscore .
Verba de débito, em escrituração.
Quantia, que, nas Companhias mercantis, se vai pagando de cada vez, até se integrar o pagamento de uma acção.
Cada um dos toques, com que o sino da igreja chama os fiéis á missa: \textunderscore o sino já deu a segunda entrada\textunderscore .
\textunderscore Dar entrada a mercadorias\textunderscore , escriturá-las, commercialmente.
\section{Entradanha}
\begin{itemize}
\item {Grp. gram.:f.}
\end{itemize}
\begin{itemize}
\item {Utilização:Ant.}
\end{itemize}
O mesmo que \textunderscore entranha\textunderscore . Cf. Frei Fortun., \textunderscore Inéd.\textunderscore , I, 306.
\section{Entrado}
\begin{itemize}
\item {Grp. gram.:adj.}
\end{itemize}
\begin{itemize}
\item {Utilização:Chul.}
\end{itemize}
\begin{itemize}
\item {Proveniência:(De \textunderscore entrar\textunderscore )}
\end{itemize}
Que entrou, que é bem acceito.
Um pouco embriagado.
\section{Entradote}
\begin{itemize}
\item {Grp. gram.:adj.}
\end{itemize}
\begin{itemize}
\item {Utilização:Fam.}
\end{itemize}
Entrado em annos; que já não é novo.
\section{Entrajar}
\begin{itemize}
\item {Grp. gram.:v. t.}
\end{itemize}
\begin{itemize}
\item {Proveniência:(De \textunderscore trajar\textunderscore )}
\end{itemize}
Pôr traje a; enroupar. Cf. Castilho, \textunderscore Fausto\textunderscore , 67.
\section{Entralhação}
\begin{itemize}
\item {Grp. gram.:f.}
\end{itemize}
\begin{itemize}
\item {Utilização:Pesc.}
\end{itemize}
\begin{itemize}
\item {Proveniência:(De \textunderscore entralhar\textunderscore )}
\end{itemize}
Conjunto de cabos á superfície da água, onde se amarram as rêdes de uma armação ou apparelho de pesca.
\section{Entralhar}
\begin{itemize}
\item {Grp. gram.:v. t.}
\end{itemize}
Tecer as tralhas de.
Enredar.
Entalar: \textunderscore entralhar um dedo na porta\textunderscore .
Prender.
Guarnecer de tralhas.
Embaraçar.
\section{Entralhe}
\begin{itemize}
\item {Grp. gram.:m.}
\end{itemize}
\begin{itemize}
\item {Utilização:T. do Ribatejo}
\end{itemize}
Acto de entralhar ou prender o boi bravo entre o manso e a charrua, para a amansia.
\section{Entralho}
\begin{itemize}
\item {Grp. gram.:m.}
\end{itemize}
\begin{itemize}
\item {Utilização:Pesc.}
\end{itemize}
\begin{itemize}
\item {Proveniência:(De \textunderscore entralhar\textunderscore )}
\end{itemize}
Fio ou cabo delgado, com que se cose ou se liga o chumbo e a cortiça ás rêdes.
\section{Entralhoada}
\begin{itemize}
\item {Grp. gram.:f.}
\end{itemize}
\begin{itemize}
\item {Utilização:Prov.}
\end{itemize}
\begin{itemize}
\item {Utilização:trasm.}
\end{itemize}
O mesmo que \textunderscore tralhoada\textunderscore .
\section{Entrambelicar}
\begin{itemize}
\item {Grp. gram.:v. t.}
\end{itemize}
\begin{itemize}
\item {Utilização:T. da Bairrada}
\end{itemize}
O mesmo que \textunderscore entrambicar\textunderscore .
\section{Entrambicar}
\begin{itemize}
\item {Grp. gram.:v. t.}
\end{itemize}
\begin{itemize}
\item {Utilização:T. da Bairrada}
\end{itemize}
Embaraçar; pear.
Fazer caír, lograr.
(Cp. \textunderscore trompicar\textunderscore )
\section{Entramento}
\begin{itemize}
\item {Grp. gram.:m.}
\end{itemize}
\begin{itemize}
\item {Utilização:Des.}
\end{itemize}
O mesmo que \textunderscore entrada\textunderscore .
\section{Entrames}
\begin{itemize}
\item {Grp. gram.:m.}
\end{itemize}
\begin{itemize}
\item {Utilização:Gír.}
\end{itemize}
\begin{itemize}
\item {Proveniência:(De \textunderscore entrar\textunderscore )}
\end{itemize}
Entrada.
Algibeira.
\section{Entrança}
\begin{itemize}
\item {Grp. gram.:f.}
\end{itemize}
\begin{itemize}
\item {Utilização:P. us.}
\end{itemize}
\begin{itemize}
\item {Proveniência:(De \textunderscore entrar\textunderscore )}
\end{itemize}
Entrada.
Princípio. Cf. Castilho, \textunderscore Fausto\textunderscore , 320.
\section{Entrançado}
\begin{itemize}
\item {Grp. gram.:m.}
\end{itemize}
\begin{itemize}
\item {Proveniência:(De \textunderscore entrançar\textunderscore )}
\end{itemize}
Entralaçamento.
\section{Entrançador}
\begin{itemize}
\item {Grp. gram.:adj.}
\end{itemize}
\begin{itemize}
\item {Grp. gram.:M.}
\end{itemize}
Que entrança.
Aquelle que entrança.
\section{Entrançadura}
\begin{itemize}
\item {Grp. gram.:f.}
\end{itemize}
Acto ou effeito de entrançar.
\section{Entrançamento}
\begin{itemize}
\item {Grp. gram.:m.}
\end{itemize}
O mesmo que \textunderscore entrançadura\textunderscore .
\section{Entrançar}
\begin{itemize}
\item {Grp. gram.:v. t.}
\end{itemize}
Dar fórma de trança a; converter em trança.
Entrelaçar.
\section{Entrância}
\begin{itemize}
\item {Grp. gram.:f.}
\end{itemize}
\begin{itemize}
\item {Utilização:Ant.}
\end{itemize}
O mesmo que \textunderscore entrança\textunderscore .
\section{Entranha}
\begin{itemize}
\item {Grp. gram.:f.}
\end{itemize}
\begin{itemize}
\item {Utilização:Ext.}
\end{itemize}
\begin{itemize}
\item {Grp. gram.:Pl.}
\end{itemize}
\begin{itemize}
\item {Utilização:Fig.}
\end{itemize}
\begin{itemize}
\item {Proveniência:(Do lat. \textunderscore intraneus\textunderscore )}
\end{itemize}
Cada uma das vísceras, contidas no abdome.
Qualquer das vísceras, contidas no abdome ou no thórax.
O conjunto dessas vísceras.
O ventre.
Carácter: \textunderscore tem más entranhas\textunderscore .
Affecto íntimo: \textunderscore filho das minhas entranhas\textunderscore .
Sentimento.
Profundidade: \textunderscore nas entranhas da terra\textunderscore .
\section{Entranhadamente}
\begin{itemize}
\item {Grp. gram.:adv.}
\end{itemize}
\begin{itemize}
\item {Utilização:Fig.}
\end{itemize}
\begin{itemize}
\item {Proveniência:(De \textunderscore entranhar\textunderscore )}
\end{itemize}
No interior.
Cordialmente: \textunderscore amava-o entranhadamente\textunderscore .
\section{Entranhar}
\begin{itemize}
\item {Grp. gram.:v. t.}
\end{itemize}
\begin{itemize}
\item {Utilização:Fig.}
\end{itemize}
\begin{itemize}
\item {Grp. gram.:V. p.}
\end{itemize}
Introduzir nas entranhas.
Cravar profundamente: \textunderscore entranhou-lhe um punhal\textunderscore .
Meter no interior.
Arraigar.
Penetrar: \textunderscore o bandido entranhou-se na floresta\textunderscore .
Dedicar-se profundamente.
\section{Entranhável}
\begin{itemize}
\item {Grp. gram.:adj.}
\end{itemize}
\begin{itemize}
\item {Proveniência:(De \textunderscore entranhar\textunderscore )}
\end{itemize}
Que penetra nas entranhas.
Que se insinua.
Que vem do coração.
\section{Entranhavelmente}
\begin{itemize}
\item {Grp. gram.:adv.}
\end{itemize}
De modo entranhável.
\section{Entranqueirar}
\begin{itemize}
\item {Grp. gram.:v. t.}
\end{itemize}
Fortificar com tranqueira; entricheirar.
Recolher em tranqueira.
\section{Entrapar}
\begin{itemize}
\item {Grp. gram.:v. t.}
\end{itemize}
Cobrir com trapos.
Emplastrar.
\section{Entrar}
\begin{itemize}
\item {Grp. gram.:v. t.}
\end{itemize}
\begin{itemize}
\item {Grp. gram.:V. i.}
\end{itemize}
\begin{itemize}
\item {Grp. gram.:Loc.}
\end{itemize}
\begin{itemize}
\item {Utilização:fam.}
\end{itemize}
\begin{itemize}
\item {Grp. gram.:V. p.}
\end{itemize}
\begin{itemize}
\item {Proveniência:(Lat. \textunderscore intrare\textunderscore )}
\end{itemize}
Introduzir-se em.
Transpor: \textunderscore entrar uma porta\textunderscore .
Passar por entre.
Passar para dentro; introduzir-se: \textunderscore entrar em casa\textunderscore .
Embeber-se: \textunderscore a faca entrou-lhe no peito\textunderscore .
Fazer parte: \textunderscore entrar em o número dos pretendentes\textunderscore .
Compartilhar.
Começar: \textunderscore entrou a chorar\textunderscore .
Envolver-se: \textunderscore entrar numa desordem\textunderscore .
Decifrar: \textunderscore não entrou com o problema\textunderscore .
Tragar.
\textunderscore Entrar com o pé direito\textunderscore , têr bôa sorte em qualquer carreira ou empresa.
\textunderscore Entrar pela pinga\textunderscore , costumar embriagar-se.
Deixar-se possuir, deixar-se dominar: \textunderscore entrar-se de receios\textunderscore .
\section{Entrasgado}
\begin{itemize}
\item {Grp. gram.:adj.}
\end{itemize}
\begin{itemize}
\item {Utilização:Prov.}
\end{itemize}
\begin{itemize}
\item {Utilização:trasm.}
\end{itemize}
\begin{itemize}
\item {Proveniência:(De \textunderscore trasga\textunderscore )}
\end{itemize}
Apertado; entalado.
\section{Entrastamento}
\begin{itemize}
\item {Grp. gram.:m.}
\end{itemize}
Acto ou effeito de entrastar.
\section{Entrastar}
\begin{itemize}
\item {Grp. gram.:v.}
\end{itemize}
\begin{itemize}
\item {Utilização:t. Mús.}
\end{itemize}
Armar contrastos (o braço da viola ou da guitarra).
\section{Entravador}
\begin{itemize}
\item {Grp. gram.:adj.}
\end{itemize}
Que entrava ou impede.
\section{Entraval}
\begin{itemize}
\item {Grp. gram.:m.}
\end{itemize}
\begin{itemize}
\item {Utilização:Marn.}
\end{itemize}
\begin{itemize}
\item {Proveniência:(De \textunderscore entrar\textunderscore  + \textunderscore valla\textunderscore ?)}
\end{itemize}
Valla, parallela ao tabuleiro do sal da marinha-velha.
\section{Entravar}
\begin{itemize}
\item {Grp. gram.:v. t.}
\end{itemize}
\begin{itemize}
\item {Proveniência:(De \textunderscore travar\textunderscore )}
\end{itemize}
Impedir; atravancar; travar.
\section{Entrave}
\begin{itemize}
\item {Grp. gram.:m.}
\end{itemize}
\begin{itemize}
\item {Utilização:Neol.}
\end{itemize}
Acto ou effeito de entravar.
\section{Entraviscar}
\textunderscore v. t.\textunderscore  (e der) \textunderscore Prov. alent.\textunderscore 
O mesmo que \textunderscore entroviscar\textunderscore , etc.
\section{Entre}
\begin{itemize}
\item {Grp. gram.:prep.}
\end{itemize}
\begin{itemize}
\item {Proveniência:(Do lat. \textunderscore inter\textunderscore )}
\end{itemize}
Designa relação de situação em meio de, ou de situação no espaço que separa.
Dentro de.
Em o número de.
\textunderscore Entre a cruz e a caldeirinha\textunderscore , ou \textunderscore entre a cruz e a água benta\textunderscore , em talas, em grande difficuldade.
Designa também o tempo médio, aquillo que está na alternativa, a differenciação de caracteres, etc.
\section{Entre...}
\begin{itemize}
\item {Grp. gram.:pref.}
\end{itemize}
\begin{itemize}
\item {Proveniência:(Do lat. \textunderscore inter\textunderscore )}
\end{itemize}
Designa intervallo, como em \textunderscore entreacto\textunderscore ; reciprocidade, como em \textunderscore entrelaçar\textunderscore ; e escassez ou pouco, como em \textunderscore entrever\textunderscore .
\section{Entreaberta}
\begin{itemize}
\item {Grp. gram.:f.}
\end{itemize}
\begin{itemize}
\item {Proveniência:(De \textunderscore entreaberto\textunderscore )}
\end{itemize}
Acto de entreabrir.
Descobrimento de uma parte do céu, em dia escuro ou ennevoado.
\section{Entreaberto}
\begin{itemize}
\item {Grp. gram.:adj.}
\end{itemize}
\begin{itemize}
\item {Proveniência:(De \textunderscore entre...\textunderscore  + \textunderscore aberto\textunderscore )}
\end{itemize}
Pouco aberto: \textunderscore uma porta entreaberta\textunderscore .
Que desabrochou: \textunderscore rosa entreaberta\textunderscore .
\section{Entreacto}
\begin{itemize}
\item {Grp. gram.:m.}
\end{itemize}
\begin{itemize}
\item {Proveniência:(De \textunderscore entre...\textunderscore  + \textunderscore acto\textunderscore )}
\end{itemize}
Intervallo, que separa dois actos de uma representação dramática.
Peça musical, que se toca nesse intervallo.
Monólogo, cançoneta ou curta representação, que se executa no mesmo intervallo.
\section{Entre-amba-las-águas}
\begin{itemize}
\item {Grp. gram.:loc. adv.}
\end{itemize}
\begin{itemize}
\item {Utilização:Prov.}
\end{itemize}
\begin{itemize}
\item {Utilização:trasm.}
\end{itemize}
\textunderscore Estar entre-amba-las-águas\textunderscore , estar indeciso, sem saber para que lado se há de voltar.
\section{Entrebanho}
\begin{itemize}
\item {Grp. gram.:m.}
\end{itemize}
Caldeirão das salinas.
\section{Entrebater}
\begin{itemize}
\item {Grp. gram.:v. t.}
\end{itemize}
\begin{itemize}
\item {Grp. gram.:V. p.}
\end{itemize}
Bater reciprocamente. Cf. Castilho, \textunderscore Fausto\textunderscore , 9, e \textunderscore Metam.\textunderscore , 149.
Digladiar.
Debater-se.
\section{Entre-branco}
\begin{itemize}
\item {Grp. gram.:adj.}
\end{itemize}
Esbranquiçado: um tanto branco.
\section{Entrecambado}
\begin{itemize}
\item {Grp. gram.:adj.}
\end{itemize}
\begin{itemize}
\item {Utilização:Des.}
\end{itemize}
\begin{itemize}
\item {Proveniência:(De \textunderscore entre...\textunderscore  + \textunderscore cambado\textunderscore )}
\end{itemize}
Diz-se das figuras heráldicas, que, por entrarem noutras, se pintam de côr diversa na parte que entra.
Enredado.
\section{Entrecana}
\begin{itemize}
\item {Grp. gram.:f.}
\end{itemize}
\begin{itemize}
\item {Proveniência:(De \textunderscore entre...\textunderscore  + \textunderscore cana\textunderscore )}
\end{itemize}
Espaço, que separa as estrias de uma columna.
\section{Entrecarga}
\begin{itemize}
\item {Grp. gram.:f.}
\end{itemize}
Espaço entre dois fardos, que constituem a carga de um animal. Cf. Camillo, \textunderscore Quéda de Um Anjo\textunderscore , 23.
\section{Entrecasa}
\begin{itemize}
\item {Grp. gram.:f.}
\end{itemize}
\begin{itemize}
\item {Utilização:Prov.}
\end{itemize}
\begin{itemize}
\item {Utilização:trasm.}
\end{itemize}
Pátio, átrio da casa.
\section{Entrecasca}
\begin{itemize}
\item {Grp. gram.:f.}
\end{itemize}
\begin{itemize}
\item {Proveniência:(De \textunderscore entre...\textunderscore  + \textunderscore casca\textunderscore )}
\end{itemize}
Lâmina interior da casca da árvore, em contacto com o lenho.
\section{Entrecasco}
\begin{itemize}
\item {Grp. gram.:m.}
\end{itemize}
\begin{itemize}
\item {Proveniência:(De \textunderscore entre...\textunderscore  + \textunderscore casco\textunderscore )}
\end{itemize}
Entrecasca.
Parte superior do casco dos animaes.
Casca tenra, que fica adherente ao sobreiro, depois de tirada a primeira camada de cortiça.
\section{Entrecerrar}
\begin{itemize}
\item {Grp. gram.:v. t.}
\end{itemize}
\begin{itemize}
\item {Proveniência:(De \textunderscore entre...\textunderscore  + \textunderscore cerrar\textunderscore )}
\end{itemize}
Cerrar incompletamente; fechar quási. Cf. B. Pato, \textunderscore Cyprestes\textunderscore , 361.
\section{Entrechado}
\begin{itemize}
\item {Grp. gram.:adj.}
\end{itemize}
Que tem entrecho.
\section{Entrechar}
\begin{itemize}
\item {Grp. gram.:v. t.}
\end{itemize}
Fazer o entrecho de; urdir. Cf. Filinto, III, 236.
\section{Entrecho}
\begin{itemize}
\item {Grp. gram.:m.}
\end{itemize}
\begin{itemize}
\item {Proveniência:(De \textunderscore trecho\textunderscore )}
\end{itemize}
Acção de uma composição dramática.
Urdidura de peça literária.
\section{Entrechocar-se}
\begin{itemize}
\item {Grp. gram.:v. p.}
\end{itemize}
\begin{itemize}
\item {Utilização:Fig.}
\end{itemize}
\begin{itemize}
\item {Proveniência:(De \textunderscore entre...\textunderscore  + \textunderscore chocar\textunderscore )}
\end{itemize}
Entrebater-se.
Contrariar-se.
Estar em contradicção.
\section{Entre-cilhas}
\begin{itemize}
\item {Grp. gram.:f. pl.}
\end{itemize}
Parte do cavallo entre as cilhas e o sovaco.
\section{Entre-coberta}
\begin{itemize}
\item {Grp. gram.:f.}
\end{itemize}
Espaço entre as cobertas de navio.
\section{Entrecolher}
\begin{itemize}
\item {Grp. gram.:v. t.}
\end{itemize}
\begin{itemize}
\item {Utilização:Prov.}
\end{itemize}
\begin{itemize}
\item {Utilização:minh.}
\end{itemize}
\begin{itemize}
\item {Proveniência:(De \textunderscore entre\textunderscore  + \textunderscore colher\textunderscore )}
\end{itemize}
Colher ou arrancar (milheiros, couves, etc., que ainda restam em seara ou plantação mais nova ou mais verde).
\section{Entrecolúmnio}
\begin{itemize}
\item {Grp. gram.:m.}
\end{itemize}
O mesmo que \textunderscore intercolúmnio\textunderscore .
\section{Entrecolúnio}
\begin{itemize}
\item {Grp. gram.:m.}
\end{itemize}
O mesmo que \textunderscore intercolúnio\textunderscore .
\section{Entreconhecer}
\begin{itemize}
\item {Grp. gram.:v. t.}
\end{itemize}
\begin{itemize}
\item {Grp. gram.:V. p.}
\end{itemize}
\begin{itemize}
\item {Proveniência:(De \textunderscore entre...\textunderscore  + \textunderscore conhecer\textunderscore )}
\end{itemize}
Conhecer um pouco, vagamente, imperfeitamente.
Lembrar-se vagamente de.
Conhecer-se reciprocamente.
\section{Entre-côro}
\begin{itemize}
\item {Grp. gram.:m.}
\end{itemize}
Espaço entre o côro da igreja e o altar-mór.
\section{Entrecorôa}
\begin{itemize}
\item {Grp. gram.:f.}
\end{itemize}
Pano de rede, entre a contra-calimba e a amalhadeira, no apparelho da xávega.
\section{Entrecorrer}
\begin{itemize}
\item {Grp. gram.:v. i.}
\end{itemize}
Correr no intervallo.
Passar entre coisas ou pessôas.
\section{Entrecortar}
\begin{itemize}
\item {Grp. gram.:v. t.}
\end{itemize}
\begin{itemize}
\item {Proveniência:(De \textunderscore entre...\textunderscore  + \textunderscore cortar\textunderscore )}
\end{itemize}
Cortar em cruz.
Interromper nervosamente, com frequência; interromper a espaços.
\section{Entrecorte}
\begin{itemize}
\item {Grp. gram.:m.}
\end{itemize}
\begin{itemize}
\item {Proveniência:(De \textunderscore entre...\textunderscore  + \textunderscore corte\textunderscore )}
\end{itemize}
Espaço, que medeia entre duas abóbadas sobrepostas.
Chanfro ou arredondamento das esquinas de um edifício, para se facilitar o transito de vehiculos.
\section{Entrecostado}
\begin{itemize}
\item {Grp. gram.:m.}
\end{itemize}
\begin{itemize}
\item {Proveniência:(De \textunderscore entre...\textunderscore  + \textunderscore costado\textunderscore )}
\end{itemize}
Refôrço de madeira, entre o costado exterior do navio e o interior.
\section{Entrecosto}
\begin{itemize}
\item {fónica:côs}
\end{itemize}
\begin{itemize}
\item {Grp. gram.:m.}
\end{itemize}
\begin{itemize}
\item {Proveniência:(De \textunderscore entre...\textunderscore  + \textunderscore costas\textunderscore )}
\end{itemize}
Espinhaço com parte das costellas de um animal, cortadas transversalmente.
Carne, entre as costellas do animal, junto do espinhaço.
\section{Entrecruzamento}
\begin{itemize}
\item {Grp. gram.:m.}
\end{itemize}
Acto ou effeito de entrecruzar-se.
\section{Entrecruzar-se}
\begin{itemize}
\item {Grp. gram.:v. p.}
\end{itemize}
Cruzar-se reciprocamente.
\section{Entrecucos}
\begin{itemize}
\item {Grp. gram.:m. pl.}
\end{itemize}
\begin{itemize}
\item {Utilização:Prov.}
\end{itemize}
\begin{itemize}
\item {Utilização:trasm.}
\end{itemize}
Pais illegitimos: \textunderscore é filho de entrecucos\textunderscore .
\section{Entrecutâneo}
\begin{itemize}
\item {Grp. gram.:adj.}
\end{itemize}
O mesmo que \textunderscore intercutâneo\textunderscore .
\section{Entredevorar-se}
\begin{itemize}
\item {Grp. gram.:v. p.}
\end{itemize}
\begin{itemize}
\item {Utilização:Fig.}
\end{itemize}
Eliminar-se reciprocamente.
\section{Entre-dia}
\begin{itemize}
\item {Grp. gram.:adv.}
\end{itemize}
\begin{itemize}
\item {Utilização:Ant.}
\end{itemize}
Durante o dia.
Fóra das horas de refeição.
\section{Entredizer}
\begin{itemize}
\item {Grp. gram.:v. t.}
\end{itemize}
Dizer para si; monologar, em voz baixa.
\section{Entre-dois}
\begin{itemize}
\item {Grp. gram.:m.}
\end{itemize}
\begin{itemize}
\item {Utilização:Serralh.}
\end{itemize}
Peça, que ajuda a apertar as cabeças dos parafusos chamados de cabeça de tremoço.
\section{Entredormido}
\begin{itemize}
\item {Grp. gram.:adj.}
\end{itemize}
Que dormita; que está meio acordado.
\section{Entre-dúvida}
\begin{itemize}
\item {Grp. gram.:f.}
\end{itemize}
Hesitação; estado do espírito entre a certeza e a dúvida. Cf. Castilho, \textunderscore Convers. Preamb.\textunderscore  do«\textunderscore D. Jaime\textunderscore ».
\section{Entre-escolher}
\begin{itemize}
\item {Grp. gram.:v. t.}
\end{itemize}
Tirar ao acaso; escolher ligeiramente.
\section{Entre-escutar}
\begin{itemize}
\item {Grp. gram.:v. t.}
\end{itemize}
Escutar a distância, com intervallos. Cf. J. Lemos, \textunderscore Canções da Tarde\textunderscore .
\section{Entrefala}
\begin{itemize}
\item {Grp. gram.:f.}
\end{itemize}
O mesmo ou melhor que \textunderscore entrevista\textunderscore :«\textunderscore ...designasse a paragem que mais arrazoada lhe parecesse para aquella entre-falla.\textunderscore »Filinto, \textunderscore D. Man.\textunderscore , I, 159.
\section{Entrefechar}
\begin{itemize}
\item {Grp. gram.:v. t.}
\end{itemize}
Fechar pouco, mansamente. Cf. Rui Barb., \textunderscore Réplica\textunderscore , 157.
\section{Entrefigurar-se}
\begin{itemize}
\item {Grp. gram.:v. p.}
\end{itemize}
\begin{itemize}
\item {Proveniência:(De \textunderscore entre...\textunderscore  + \textunderscore figurar\textunderscore )}
\end{itemize}
Parecer; dar ideia de:«\textunderscore ás vezes se me entrefigura que...\textunderscore »Castilho, \textunderscore Carta\textunderscore  a A. M. Pereira.
\section{Entrefino}
\begin{itemize}
\item {Grp. gram.:adj.}
\end{itemize}
\begin{itemize}
\item {Proveniência:(De \textunderscore entre...\textunderscore  + \textunderscore fino\textunderscore )}
\end{itemize}
Que não é fino nem grosso.
Que é de lote meão, entre o fino e o grosso ou o ordinário.
\section{Entrefolha}
\begin{itemize}
\item {fónica:fô}
\end{itemize}
\begin{itemize}
\item {Grp. gram.:f.}
\end{itemize}
\begin{itemize}
\item {Proveniência:(De \textunderscore entre...\textunderscore  + \textunderscore fôlha\textunderscore )}
\end{itemize}
Fôlha de papel, em branco ou manuscrita, intercalada nas fôlhas impressas de um livro, para annotações ou addições.
\section{Entrefolho}
\begin{itemize}
\item {fónica:fô}
\end{itemize}
\begin{itemize}
\item {Grp. gram.:m.}
\end{itemize}
\begin{itemize}
\item {Proveniência:(De \textunderscore entre...\textunderscore  + \textunderscore folho\textunderscore )}
\end{itemize}
Parte interior; escaninho.
Indigestão chrónica no folhoso dos ruminantes.
\section{Entrefoliado}
\begin{itemize}
\item {Grp. gram.:adj.}
\end{itemize}
Diz-se do livro que tem entrefolhas.
\section{Entreforro}
\begin{itemize}
\item {fónica:fô}
\end{itemize}
\begin{itemize}
\item {Grp. gram.:m.}
\end{itemize}
\begin{itemize}
\item {Proveniência:(De \textunderscore entre\textunderscore  + \textunderscore fôrro\textunderscore ^1)}
\end{itemize}
Entretela.
Fôrro do telhado ou madeiramento, entre o telhado e o fôrro da casa.
Desvão de um navio, ou lugar escuso, em que se guardam objectos de bordo.
Entrecasca.
\section{Entrega}
\begin{itemize}
\item {Grp. gram.:f.}
\end{itemize}
Acto ou effeito de entregar.
Comprometimento.
Traição.
\section{Entregadoiro}
\begin{itemize}
\item {Grp. gram.:adj.}
\end{itemize}
\begin{itemize}
\item {Utilização:Des.}
\end{itemize}
Que se deve entregar.
\section{Entregador}
\begin{itemize}
\item {Grp. gram.:m.}
\end{itemize}
Aquelle que entrega.
\section{Entregadouro}
\begin{itemize}
\item {Grp. gram.:adj.}
\end{itemize}
\begin{itemize}
\item {Utilização:Des.}
\end{itemize}
Que se deve entregar.
\section{Entregar}
\begin{itemize}
\item {Grp. gram.:v. t.}
\end{itemize}
\begin{itemize}
\item {Grp. gram.:V. p.}
\end{itemize}
\begin{itemize}
\item {Proveniência:(Do lat. \textunderscore integrare\textunderscore ?)}
\end{itemize}
Passar ás mãos de outrem.
Dar.
Arremessar.
Pagar.
Indemnizar.
Confiar: \textunderscore entregou o filho á criada\textunderscore .
Renunciar.
Trahir.
Pôr em poder de alguém.
Dedicar-se.
Confiar-se.
Render-se.
\section{Entregosto}
\begin{itemize}
\item {fónica:gôs}
\end{itemize}
\begin{itemize}
\item {Grp. gram.:m.}
\end{itemize}
\begin{itemize}
\item {Utilização:T. de Serpa}
\end{itemize}
O mesmo que \textunderscore entrecosto\textunderscore .
\section{Entregue}
\begin{itemize}
\item {Grp. gram.:adj.}
\end{itemize}
\begin{itemize}
\item {Proveniência:(De \textunderscore entregar\textunderscore )}
\end{itemize}
Dado ou confiado a alguém.
Recebido.
\section{Entrehostil}
\begin{itemize}
\item {Grp. gram.:adj.}
\end{itemize}
Um tanto hostil. Cf. Camillo, \textunderscore Narcót.\textunderscore , I, 280.
\section{Entreitória}
\begin{itemize}
\item {Grp. gram.:f.}
\end{itemize}
\begin{itemize}
\item {Utilização:Prov.}
\end{itemize}
\begin{itemize}
\item {Utilização:trasm.}
\end{itemize}
O mesmo que \textunderscore treitoeira\textunderscore .
\section{Entrelaçamento}
\begin{itemize}
\item {Grp. gram.:m.}
\end{itemize}
Acto ou effeito de entrelaçar.
\section{Entrelaçar}
\begin{itemize}
\item {Grp. gram.:v. t.}
\end{itemize}
\begin{itemize}
\item {Proveniência:(De \textunderscore entre...\textunderscore  + \textunderscore laçar\textunderscore )}
\end{itemize}
Converter em laço ou laços.
Enlaçar.
Entrançar.
Entretecer.
Ennastrar.
\section{Entreliar}
\begin{itemize}
\item {Grp. gram.:v. t.}
\end{itemize}
\begin{itemize}
\item {Utilização:Ant.}
\end{itemize}
O mesmo que \textunderscore entrelinhar\textunderscore .
\section{Entreligar}
\begin{itemize}
\item {Grp. gram.:v. t.}
\end{itemize}
Ligar reciprocamente. Cf. Camillo, \textunderscore Caveira\textunderscore , 106.
\section{Entrelinha}
\begin{itemize}
\item {Grp. gram.:f.}
\end{itemize}
\begin{itemize}
\item {Grp. gram.:Pl.}
\end{itemize}
\begin{itemize}
\item {Utilização:Fam.}
\end{itemize}
\begin{itemize}
\item {Proveniência:(De \textunderscore entre...\textunderscore  + \textunderscore linha\textunderscore )}
\end{itemize}
Espaço, entre duas linhas.
Aquillo que se escreve entre duas linhas de escrita.
Commentário.
Sentido implícito, illação mental.
\section{Entrelinhar}
\begin{itemize}
\item {Grp. gram.:v. t.}
\end{itemize}
\begin{itemize}
\item {Utilização:Typ.}
\end{itemize}
\begin{itemize}
\item {Proveniência:(De \textunderscore entrelinha\textunderscore )}
\end{itemize}
Pôr entrelinhas em.
Intervallar, espacejar, faiar.
Commentar.
\section{Entrelopo}
\begin{itemize}
\item {Grp. gram.:adj.}
\end{itemize}
\begin{itemize}
\item {Proveniência:(Do ingl. \textunderscore interloper\textunderscore )}
\end{itemize}
Relativo a contrabando.
Aventureiro.
\section{Entrelunho}
\begin{itemize}
\item {Grp. gram.:m.}
\end{itemize}
\begin{itemize}
\item {Utilização:Ant.}
\end{itemize}
O mesmo que \textunderscore interlúnio\textunderscore . Cf. \textunderscore Eufrosina\textunderscore , 265.
\section{Entrelúnio}
\begin{itemize}
\item {Grp. gram.:m.}
\end{itemize}
(V.interlúnio)
\section{Entreluo}
\begin{itemize}
\item {Grp. gram.:m.}
\end{itemize}
\begin{itemize}
\item {Utilização:T. de Turquel}
\end{itemize}
O mesmo que \textunderscore interlúnio\textunderscore .
\section{Entreluzir}
\begin{itemize}
\item {Grp. gram.:v. i.}
\end{itemize}
\begin{itemize}
\item {Grp. gram.:V. t.}
\end{itemize}
Começar a luzir; luzir froixamente.
Entremostrar-se.
Perceber. Cf. Castilho, \textunderscore Fausto\textunderscore , 41; Camillo, \textunderscore Caveira\textunderscore , 10.
\section{Entremaduro}
\begin{itemize}
\item {Grp. gram.:adj.}
\end{itemize}
Que não amadureceu inteiramente; meio maduro.
\section{Entremanhã}
\begin{itemize}
\item {Grp. gram.:f.}
\end{itemize}
O crepúsculo matinal.
\section{Entremanhan}
\begin{itemize}
\item {Grp. gram.:f.}
\end{itemize}
O crepúsculo matinal.
\section{Entremarrar-se}
\begin{itemize}
\item {Grp. gram.:v. p.}
\end{itemize}
Marrar reciprocamente.
\section{Entremear}
\begin{itemize}
\item {Grp. gram.:v. t.}
\end{itemize}
\begin{itemize}
\item {Proveniência:(De \textunderscore entremeio\textunderscore )}
\end{itemize}
Pôr de permeio. Entre-sachar; misturar.
\section{Entremecha}
\begin{itemize}
\item {Grp. gram.:f.}
\end{itemize}
\begin{itemize}
\item {Proveniência:(De \textunderscore entre...\textunderscore  + \textunderscore mecha\textunderscore )}
\end{itemize}
Trave, com que se liga um costado do navio a outro costado, quando a embarcação está alquebrada.
\section{Entremeio}
\begin{itemize}
\item {Grp. gram.:m.}
\end{itemize}
\begin{itemize}
\item {Grp. gram.:Adj.}
\end{itemize}
Espaço, coisa ou tempo, entre dois pontos ou entre dois extremos.
Renda ou tira bordada, entre espaços lisos.
O mesmo que \textunderscore intermédio\textunderscore .
(Do \textunderscore entre...\textunderscore  + \textunderscore meio\textunderscore )
\section{Entremente}
\begin{itemize}
\item {Grp. gram.:adv.}
\end{itemize}
O mesmo que \textunderscore entrementes\textunderscore .
\section{Entrementes}
\begin{itemize}
\item {Grp. gram.:adv.}
\end{itemize}
\begin{itemize}
\item {Grp. gram.:M.}
\end{itemize}
\begin{itemize}
\item {Proveniência:(De \textunderscore entre...\textunderscore  + \textunderscore mentes\textunderscore )}
\end{itemize}
Entretanto.
Naquella ou nesta occasião.
Tempo intermédio: \textunderscore neste entrementes, chegou êlle\textunderscore .
\section{Entremês}
\begin{itemize}
\item {Grp. gram.:m.}
\end{itemize}
\begin{itemize}
\item {Utilização:Prov.}
\end{itemize}
\begin{itemize}
\item {Utilização:trasm.}
\end{itemize}
Trigo tremês.
\section{Entremesa}
\begin{itemize}
\item {fónica:mê}
\end{itemize}
\begin{itemize}
\item {Grp. gram.:f.}
\end{itemize}
\begin{itemize}
\item {Grp. gram.:Adv.}
\end{itemize}
\begin{itemize}
\item {Proveniência:(De \textunderscore entre...\textunderscore  + \textunderscore mesa\textunderscore )}
\end{itemize}
Tempo de uma refeição.
Durante a refeição.
\section{Entremeter}
\begin{itemize}
\item {Grp. gram.:v. t.}
\end{itemize}
O mesmo que \textunderscore intrometer\textunderscore .
\section{Entremetimento}
\begin{itemize}
\item {Grp. gram.:m.}
\end{itemize}
Acto ou effeito de entremeter.
\section{Entremez}
\begin{itemize}
\item {Grp. gram.:m.}
\end{itemize}
\begin{itemize}
\item {Utilização:Pop.}
\end{itemize}
\begin{itemize}
\item {Proveniência:(Do it. \textunderscore intermezzo\textunderscore )}
\end{itemize}
Curta composiçacão theatral, jocosa ou burlesca; farsa.
Coisa ridícula.
\section{Entremezada}
\begin{itemize}
\item {Grp. gram.:f.}
\end{itemize}
\begin{itemize}
\item {Proveniência:(De \textunderscore entremez\textunderscore )}
\end{itemize}
Coisa ridícula, farsada.
\section{Entremezado}
\begin{itemize}
\item {Grp. gram.:adj.}
\end{itemize}
\begin{itemize}
\item {Utilização:Des.}
\end{itemize}
Que tem modos ou feições de entremez:«\textunderscore ...vilhancico entremezado...\textunderscore ». \textunderscore Anat. Joc.\textunderscore , 295.
\section{Entremezão}
\begin{itemize}
\item {Grp. gram.:m.}
\end{itemize}
Entremez grande. Cf. \textunderscore Filinto\textunderscore , I, 60.
\section{Entremezista}
\begin{itemize}
\item {Grp. gram.:m.}
\end{itemize}
Aquelle que faz entremezes ou que os representa.
\section{Entremicha}
\begin{itemize}
\item {Grp. gram.:f.}
\end{itemize}
\begin{itemize}
\item {Utilização:Ant.}
\end{itemize}
O mesmo que \textunderscore entremecha\textunderscore . Cp. \textunderscore Hist. Trág. Marít.\textunderscore , 224.
\section{Entremisturar}
\begin{itemize}
\item {Grp. gram.:v. t.}
\end{itemize}
Misturar,confundir. Cf. Castilho, \textunderscore Fastos\textunderscore , III, 143.
\section{Entremodilhão}
\begin{itemize}
\item {Grp. gram.:m.}
\end{itemize}
Espaço, entre dois modilhões.
\section{Entremontano}
\begin{itemize}
\item {Grp. gram.:adj.}
\end{itemize}
\begin{itemize}
\item {Proveniência:(De \textunderscore entre...\textunderscore  + \textunderscore monte\textunderscore )}
\end{itemize}
Que está entre montes.
\section{Entremontes}
\begin{itemize}
\item {Grp. gram.:m.}
\end{itemize}
O mesmo que \textunderscore valle\textunderscore . Cf. \textunderscore Filinto\textunderscore , XIV, 54.
\section{Entremostrar}
\begin{itemize}
\item {Grp. gram.:v. t.}
\end{itemize}
Mostrar um tanto, incompletamente.
\section{Entre-nó}
\begin{itemize}
\item {Grp. gram.:m.}
\end{itemize}
Espaço, entre dois nós de um tronco ou caule.
\section{Entrenublar-se}
\begin{itemize}
\item {Grp. gram.:v. p.}
\end{itemize}
Mostrar-se entre nuvens.
Cobrir-se de nuvens ligeiras ou transparentes.
\section{Entreolhar-se}
\begin{itemize}
\item {Grp. gram.:v. p.}
\end{itemize}
Olhar-se reciprocamente. Cf. Garrett, \textunderscore Camões\textunderscore , 187.
\section{Entreouvir}
\begin{itemize}
\item {Grp. gram.:v. t.}
\end{itemize}
Ouvir confusamente, incompletamente.
\section{Entrepano}
\begin{itemize}
\item {Grp. gram.:m.}
\end{itemize}
Tábua de estante ou armário, que divide as prateleiras de alto a baixo.
Espaço entre duas pilastras ou columnas.
\section{Entrepausa}
\begin{itemize}
\item {Grp. gram.:f.}
\end{itemize}
Pausa intermédia.
\section{Entrepelado}
\begin{itemize}
\item {Grp. gram.:adj.}
\end{itemize}
\begin{itemize}
\item {Utilização:Bras. do S}
\end{itemize}
\begin{itemize}
\item {Proveniência:(T. cast.)}
\end{itemize}
Diz-se do cavallo, que tem pêlo de três côres, preto, branco e vermelho.
\section{Entrepelejar}
\begin{itemize}
\item {Grp. gram.:v. i.}
\end{itemize}
Pelejar entre si; lutar consigo próprio. Cf. Castilho, \textunderscore Metam.\textunderscore , 284.
\section{Entrepicar}
\begin{itemize}
\item {Grp. gram.:v. i.}
\end{itemize}
O mesmo que \textunderscore entreplicar\textunderscore .
\section{Entrepilastras}
\begin{itemize}
\item {Grp. gram.:m.}
\end{itemize}
Espaço entre pilastras.
\section{Entreplicar}
\begin{itemize}
\item {Grp. gram.:v. i.}
\end{itemize}
\begin{itemize}
\item {Utilização:Prov.}
\end{itemize}
Implicar.
Intrometer-se, provocando.
Serrazinar.
\section{Entreponte}
\begin{itemize}
\item {Grp. gram.:f.}
\end{itemize}
Espaço entre duas cobertas de navio.
\section{Entrepor}
\begin{itemize}
\item {Grp. gram.:v. t.}
\end{itemize}
Pôr entre:«\textunderscore e o abbade entrepondo as cangalhas nas páginas do breviário...\textunderscore »Camillo, \textunderscore Brasileira\textunderscore , 197.
\section{Entre-portas}
\begin{itemize}
\item {Grp. gram.:loc. adv.}
\end{itemize}
Á entrada da casa; no limiar.
\section{Entrepósito}
\begin{itemize}
\item {Grp. gram.:m.}
\end{itemize}
\begin{itemize}
\item {Proveniência:(Fr. \textunderscore entrepôt\textunderscore )}
\end{itemize}
Empório; grande depósito de mercadorias.
Feitoria, armazém, onde se guardam ou vendem exclusivamente as mercadorias de um Estado ou de uma Companhia.
Depósito, em que as mercadorias podem estar certo tempo, sem pagar direitos.
Armazenagem.--«\textunderscore Cumpre adoptar uma destas duas palavras\textunderscore , (\textunderscore entreposto\textunderscore  ou \textunderscore entrepósito\textunderscore ), que o uso tem começado a introduzir por necessidade, no mesmo sentido, em que os Franceses dizem \textunderscore entrepôt.\textunderscore »F. Borges, \textunderscore Dicc. Jur.\textunderscore 
\section{Entreposto}
\begin{itemize}
\item {fónica:pôs}
\end{itemize}
\begin{itemize}
\item {Grp. gram.:m.}
\end{itemize}
(V.entrepósito)
\section{Entreprendente}
\begin{itemize}
\item {Grp. gram.:adj.}
\end{itemize}
Que entreprende; que toma de assalto. Que surprehende, acommetendo. Cf. Cenáculo, \textunderscore Pastoral\textunderscore , 5.
\section{Entreprender}
\begin{itemize}
\item {Grp. gram.:v. t.}
\end{itemize}
\begin{itemize}
\item {Utilização:Fig.}
\end{itemize}
\begin{itemize}
\item {Proveniência:(De \textunderscore entre\textunderscore  + \textunderscore prender\textunderscore )}
\end{itemize}
O mesmo que \textunderscore emprehender\textunderscore .
Assaltar de improviso:«\textunderscore até que o entreprendeo a morte\textunderscore ». \textunderscore Luz e Calor\textunderscore , 92.
\section{Entrepresa}
\begin{itemize}
\item {fónica:prê}
\end{itemize}
\begin{itemize}
\item {Grp. gram.:f.}
\end{itemize}
Acto ou effeito de \textunderscore entreprender\textunderscore .
Assalto imprevisto.
Emprehendimento:«\textunderscore são agras e agérrimas as entrepresas deste gênero.\textunderscore »Castilho, \textunderscore Amor e Melanc.\textunderscore , 339. Cf. Camillo, \textunderscore Caveira\textunderscore , 208.
\section{Entre-regar}
\begin{itemize}
\item {Grp. gram.:v. t.}
\end{itemize}
Regar interiormente:«\textunderscore ...rios..., por muitas terras, que com suas voltas entre-regam...\textunderscore »Filinto, \textunderscore D. Man.\textunderscore , I, 101.
\section{Entresacar}
\begin{itemize}
\item {fónica:sa}
\end{itemize}
\begin{itemize}
\item {Grp. gram.:v. t.}
\end{itemize}
\begin{itemize}
\item {Proveniência:(De \textunderscore entre...\textunderscore  + \textunderscore sacar\textunderscore )}
\end{itemize}
Tirar entremeadamente. Cf. Bernardez, \textunderscore Luz e Calor\textunderscore , 494.
\section{Entresachar}
\begin{itemize}
\item {fónica:sa}
\end{itemize}
\begin{itemize}
\item {Grp. gram.:v. t.}
\end{itemize}
Meter entre outras coisas; misturar; entrelaçar:«\textunderscore cobrem as paredes... entresachadas rosas.\textunderscore »Sousa.
\section{Entrescutar}
\begin{itemize}
\item {Grp. gram.:v. t.}
\end{itemize}
O mesmo que \textunderscore entre-escutar\textunderscore .
\section{Entre-seio}
\begin{itemize}
\item {Grp. gram.:m.}
\end{itemize}
O mesmo que \textunderscore enseio\textunderscore .
\section{Entresemear}
\begin{itemize}
\item {fónica:se}
\end{itemize}
\begin{itemize}
\item {Grp. gram.:v. t.}
\end{itemize}
Semear ou plantar de permeio.
Entremear.
\section{Entresilhado}
\begin{itemize}
\item {Grp. gram.:adj.}
\end{itemize}
\begin{itemize}
\item {Proveniência:(Do cast. \textunderscore trasijado\textunderscore )}
\end{itemize}
Magro, enfraquecido.
Esgrouviado.
\section{Entre-sola}
\begin{itemize}
\item {Grp. gram.:f.}
\end{itemize}
Peça, entre a sola e a palmilha do calçado.
\section{Entre-solho}
\begin{itemize}
\item {fónica:sô}
\end{itemize}
\begin{itemize}
\item {Grp. gram.:m.}
\end{itemize}
\begin{itemize}
\item {Utilização:Des.}
\end{itemize}
Espaço, entre o chão e o solho da casa.
Sobreloja.
\section{Entresonhar}
\begin{itemize}
\item {fónica:so}
\end{itemize}
\begin{itemize}
\item {Grp. gram.:v. t.}
\end{itemize}
\begin{itemize}
\item {Grp. gram.:V. i.}
\end{itemize}
\begin{itemize}
\item {Proveniência:(De \textunderscore entre...\textunderscore  + \textunderscore sonhar\textunderscore )}
\end{itemize}
Sonhar vagamente.
Imaginar.
Devanear.
\section{Entresonho}
\begin{itemize}
\item {fónica:sô}
\end{itemize}
\begin{itemize}
\item {Grp. gram.:m.}
\end{itemize}
Acto de \textunderscore entresonhar\textunderscore . Cf. Camillo, \textunderscore Esqueleto\textunderscore , 67.
\section{Entressacar}
\begin{itemize}
\item {Grp. gram.:v. t.}
\end{itemize}
\begin{itemize}
\item {Proveniência:(De \textunderscore entre...\textunderscore  + \textunderscore sacar\textunderscore )}
\end{itemize}
Tirar entremeadamente. Cf. Bernardez, \textunderscore Luz e Calor\textunderscore , 494.
\section{Entressachar}
\begin{itemize}
\item {Grp. gram.:v. t.}
\end{itemize}
Meter entre outras coisas; misturar; entrelaçar:«\textunderscore cobrem as paredes... entressachadas rosas.\textunderscore »Sousa.
\section{Entressemear}
\begin{itemize}
\item {Grp. gram.:v. t.}
\end{itemize}
Semear ou plantar de permeio.
Entremear.
\section{Entressolho}
\begin{itemize}
\item {Grp. gram.:m.}
\end{itemize}
\begin{itemize}
\item {Utilização:Des.}
\end{itemize}
Espaço, entre o chão e o solho da casa.
Sobreloja.
\section{Entressonhar}
\begin{itemize}
\item {Grp. gram.:v. t.}
\end{itemize}
\begin{itemize}
\item {Grp. gram.:V. i.}
\end{itemize}
\begin{itemize}
\item {Proveniência:(De \textunderscore entre...\textunderscore  + \textunderscore sonhar\textunderscore )}
\end{itemize}
Sonhar vagamente.
Imaginar.
Devanear.
\section{Entressonho}
\begin{itemize}
\item {Grp. gram.:m.}
\end{itemize}
Acto de \textunderscore entressonhar\textunderscore . Cf. Camillo, \textunderscore Esqueleto\textunderscore , 67.
\section{Entretalhador}
\begin{itemize}
\item {Grp. gram.:m.}
\end{itemize}
Aquelle que entretalha.
\section{Entretalhadura}
\begin{itemize}
\item {Grp. gram.:f.}
\end{itemize}
O mesmo que \textunderscore entretalho\textunderscore .
\section{Entretalhar}
\begin{itemize}
\item {Grp. gram.:v. t.}
\end{itemize}
\begin{itemize}
\item {Grp. gram.:V. i.}
\end{itemize}
Fazer entretalhos em.
Reduzir a entretalhos.
Fazer entretalhos.
\section{Entretalho}
\begin{itemize}
\item {Grp. gram.:m.}
\end{itemize}
\begin{itemize}
\item {Proveniência:(De \textunderscore entre...\textunderscore  + \textunderscore talho\textunderscore )}
\end{itemize}
Debuxo recortado.
Lavor em papel, pano, etc.
Recortes nos vestidos antigos.
Esculptura em meio relêvo ou em baixo relêvo.
\section{Entretanto}
\begin{itemize}
\item {Grp. gram.:adv.}
\end{itemize}
\begin{itemize}
\item {Grp. gram.:M.}
\end{itemize}
\begin{itemize}
\item {Proveniência:(De \textunderscore entre...\textunderscore  + \textunderscore tanto\textunderscore )}
\end{itemize}
No tempo intermédio.
Naquelle tempo.
Todavia.
Tempo intermédio: \textunderscore neste entretanto...\textunderscore 
\section{Entretecedor}
\begin{itemize}
\item {Grp. gram.:adj.}
\end{itemize}
\begin{itemize}
\item {Grp. gram.:M.}
\end{itemize}
Que entretece.
Aquelle que entretece.
\section{Entretecedura}
\begin{itemize}
\item {Grp. gram.:f.}
\end{itemize}
Acto ou effeito de entretecer.
\section{Entretecer}
\begin{itemize}
\item {Grp. gram.:v. t.}
\end{itemize}
\begin{itemize}
\item {Utilização:Fig.}
\end{itemize}
\begin{itemize}
\item {Proveniência:(De \textunderscore entre...\textunderscore  + \textunderscore tecer\textunderscore )}
\end{itemize}
Entremear, tecendo.
Entrelaçar.
Construir com laços.
Inserir num tecido.
Tecer.
Incluir.
\section{Entretecimento}
\begin{itemize}
\item {Grp. gram.:m.}
\end{itemize}
O mesmo que \textunderscore entretecedura\textunderscore .
\section{Entretela}
\begin{itemize}
\item {Grp. gram.:f.}
\end{itemize}
\begin{itemize}
\item {Utilização:Ext.}
\end{itemize}
\begin{itemize}
\item {Proveniência:(De \textunderscore entre...\textunderscore  + \textunderscore tela\textunderscore )}
\end{itemize}
Estôfo encorpado e consistente, entre a fazenda do fato e o fôrro.
Contraforte de muralha.
\section{Entretelar}
\begin{itemize}
\item {Grp. gram.:v. t.}
\end{itemize}
Pôr entretela em.
\section{Entretém}
\begin{itemize}
\item {Grp. gram.:m.}
\end{itemize}
\begin{itemize}
\item {Utilização:Pop.}
\end{itemize}
Aquillo que serve para entreter.
O mesmo que \textunderscore entretenimento\textunderscore .
\section{Entretener}
\begin{itemize}
\item {Grp. gram.:v. t.}
\end{itemize}
\begin{itemize}
\item {Utilização:Ant.}
\end{itemize}
O mesmo que \textunderscore entreter\textunderscore .
\section{Entretenga}
\begin{itemize}
\item {Grp. gram.:f.}
\end{itemize}
\begin{itemize}
\item {Utilização:Prov.}
\end{itemize}
\begin{itemize}
\item {Proveniência:(De \textunderscore entreter\textunderscore )}
\end{itemize}
Entretenimento.
\section{Entretenide}
\begin{itemize}
\item {Grp. gram.:adj.}
\end{itemize}
\begin{itemize}
\item {Proveniência:(De \textunderscore entretener\textunderscore )}
\end{itemize}
Divertido, distrahido:«\textunderscore quando vemos alguns ministros desoccupados e entretenidos...\textunderscore »Cf. Bernárdez, \textunderscore Luz e Calor\textunderscore , 65.
\section{Entretenimento}
\begin{itemize}
\item {Grp. gram.:m.}
\end{itemize}
\begin{itemize}
\item {Proveniência:(De \textunderscore entretener\textunderscore )}
\end{itemize}
Acto de entreter.
Coisa que entretém.
Brincadeira; distracção; divertimento.
\section{Entreter}
\begin{itemize}
\item {Grp. gram.:v. t.}
\end{itemize}
\begin{itemize}
\item {Proveniência:(De \textunderscore entre...\textunderscore  + \textunderscore têr\textunderscore )}
\end{itemize}
Demorar, deter.
Conservar.
Illudir.
Suavizar, alliviar.
Distrahir, divertir; servir de recreio a.
\section{Entretesta}
\begin{itemize}
\item {Grp. gram.:f.}
\end{itemize}
\begin{itemize}
\item {Proveniência:(De \textunderscore entre...\textunderscore  + \textunderscore testa\textunderscore )}
\end{itemize}
Tira de tecido, na extremidade da teia, e differente desta.
\section{Entretidamente}
\begin{itemize}
\item {Grp. gram.:adv.}
\end{itemize}
Com entretimento; distrahidamente.
\section{Entretimento}
\begin{itemize}
\item {Grp. gram.:m.}
\end{itemize}
(V.entretenimento)
\section{Entretinho}
\begin{itemize}
\item {Grp. gram.:m.}
\end{itemize}
\begin{itemize}
\item {Utilização:Prov.}
\end{itemize}
\begin{itemize}
\item {Proveniência:(Do rad. de \textunderscore entreter\textunderscore ?)}
\end{itemize}
Comida da ave.
Membrana, que envolve os intestinos do porco.
\section{Entretrópico}
\begin{itemize}
\item {Grp. gram.:adj.}
\end{itemize}
\begin{itemize}
\item {Utilização:Des.}
\end{itemize}
O mesmo que \textunderscore intertropical\textunderscore .
\section{Entreturbar}
\begin{itemize}
\item {Grp. gram.:v. t.}
\end{itemize}
Perturbar ligeiramente.
\section{Entreunir}
\begin{itemize}
\item {fónica:tre-u}
\end{itemize}
\begin{itemize}
\item {Grp. gram.:v. t.}
\end{itemize}
Unir reciprocamente. Cf. Camillo, \textunderscore Noites de Insómn.\textunderscore , III, 56.
\section{Entrèvação}
\begin{itemize}
\item {Grp. gram.:f.}
\end{itemize}
\begin{itemize}
\item {Proveniência:(De \textunderscore entrèvar\textunderscore )}
\end{itemize}
Acto ou effeito de entrèvecer.
\section{Entrèvado}
\begin{itemize}
\item {Grp. gram.:m.}
\end{itemize}
\begin{itemize}
\item {Proveniência:(De \textunderscore entrèvar\textunderscore ^2)}
\end{itemize}
Paralýtico.
Aquelle que por doença está impossibilitado de saír de casa ou de cama.
\section{Entrèvamento}
\begin{itemize}
\item {Grp. gram.:m.}
\end{itemize}
O mesmo que \textunderscore entrèvação\textunderscore .
\section{Entrèvar}
\begin{itemize}
\item {Grp. gram.:v. t.}
\end{itemize}
Entorpecer os membros de; tornar paralýtico.
\section{Entrèvar}
\begin{itemize}
\item {Grp. gram.:v. t.}
\end{itemize}
Cobrir de trevas; meter no escuro.
\section{Entreveado}
\begin{itemize}
\item {Grp. gram.:adj.}
\end{itemize}
\begin{itemize}
\item {Utilização:Prov.}
\end{itemize}
\begin{itemize}
\item {Proveniência:(De \textunderscore entre\textunderscore  + \textunderscore veio\textunderscore )}
\end{itemize}
Diz-se do toicinho entremeado de veios vermelhos ou fibrosos.
\section{Entrèvecer}
\begin{itemize}
\item {Grp. gram.:v. i.  e  p.}
\end{itemize}
Tornar-se paralýtico; perder os movimentos.
(Do \textunderscore entrèvar\textunderscore ^1)
\section{Entrèvecimento}
\begin{itemize}
\item {Grp. gram.:m.}
\end{itemize}
O mesmo que \textunderscore entrèvação\textunderscore .
\section{Entrèver}
\begin{itemize}
\item {Grp. gram.:v. t.}
\end{itemize}
\begin{itemize}
\item {Grp. gram.:V. p.}
\end{itemize}
\begin{itemize}
\item {Proveniência:(De \textunderscore entre...\textunderscore  + \textunderscore vêr\textunderscore )}
\end{itemize}
Vêr imperfeitamente.
Perceber a custo.
Antever.
Têr entrevista com alguém.
Vêr-se de passagem.
\section{Entrèverar}
\begin{itemize}
\item {Grp. gram.:v. t.}
\end{itemize}
\begin{itemize}
\item {Utilização:Bras. do S}
\end{itemize}
\begin{itemize}
\item {Proveniência:(T. cast.)}
\end{itemize}
Misturar, (falando-se de corpos militares de partido differente, que se confundem no furor do combate).
Misturar ou confundir (gados que andam pastando).
\section{Entreverde}
\begin{itemize}
\item {fónica:vêr}
\end{itemize}
\begin{itemize}
\item {Grp. gram.:m.  e  adj.}
\end{itemize}
\begin{itemize}
\item {Proveniência:(De \textunderscore entre...\textunderscore  + \textunderscore verde\textunderscore )}
\end{itemize}
Casta de uva, da região do Doiro.
Vinho, procedente dessa casta.
\section{Entrevêro}
\begin{itemize}
\item {Grp. gram.:m.}
\end{itemize}
\begin{itemize}
\item {Utilização:Bras. do S}
\end{itemize}
Acto ou effeito de entreverar.
\section{Entrevinda}
\begin{itemize}
\item {Grp. gram.:f.}
\end{itemize}
\begin{itemize}
\item {Proveniência:(De \textunderscore entre...\textunderscore  + \textunderscore vinda\textunderscore )}
\end{itemize}
Vinda repentina, inopinada.
\section{Entrevista}
\begin{itemize}
\item {Grp. gram.:f.}
\end{itemize}
\begin{itemize}
\item {Proveniência:(De \textunderscore entre...\textunderscore  + \textunderscore vista\textunderscore )}
\end{itemize}
Encontro combinado.
Conferência de duas ou mais pessôas em lugar previamente combinado.
Estôfo, entre o fôrro e a peça transparente ou golpeada do vestuário.
\section{Entrevistar}
\begin{itemize}
\item {Grp. gram.:v. t.}
\end{itemize}
Têr entrevista com.
\section{Entrezar}
\begin{itemize}
\item {Grp. gram.:v. t.}
\end{itemize}
\begin{itemize}
\item {Utilização:Des.}
\end{itemize}
\begin{itemize}
\item {Proveniência:(Do it. \textunderscore intrecciare\textunderscore )}
\end{itemize}
O mesmo que \textunderscore entretecer\textunderscore .
\section{Entriçar}
\begin{itemize}
\item {Grp. gram.:v. p.}
\end{itemize}
\begin{itemize}
\item {Utilização:Prov.}
\end{itemize}
\begin{itemize}
\item {Utilização:alg.}
\end{itemize}
Tornar encolhido; entanguir: \textunderscore estou entriçado com frio\textunderscore .
(Por \textunderscore inteiriçar\textunderscore ?)
\section{Entrincheiramento}
\begin{itemize}
\item {Grp. gram.:m.}
\end{itemize}
Acto ou effeito de entrincheirar.
Trincheira ou conjunto de trincheiras.
\section{Entrincheirar}
\begin{itemize}
\item {Grp. gram.:v. t.}
\end{itemize}
\begin{itemize}
\item {Grp. gram.:V. p.}
\end{itemize}
Fortificar com trincheira.
Barricar.
Defender-se com trincheiras.
Fortificar-se.
Apoiar-se.
Refugiar-se.
\section{Entripado}
\begin{itemize}
\item {Grp. gram.:adj.}
\end{itemize}
\begin{itemize}
\item {Utilização:Prov.}
\end{itemize}
\begin{itemize}
\item {Utilização:trasm.}
\end{itemize}
\begin{itemize}
\item {Utilização:Prov.}
\end{itemize}
\begin{itemize}
\item {Utilização:alent.}
\end{itemize}
\begin{itemize}
\item {Proveniência:(De \textunderscore tripa\textunderscore )}
\end{itemize}
Que se sente nos intestinos.
Que vai ferido no ventre, não deitando sangue, mas fraquejando na fuga e deixando excrementos no mato, (falando-se do javali ou de outra peça de caça grossa).
\section{Entristecer}
\begin{itemize}
\item {Grp. gram.:v. t.}
\end{itemize}
\begin{itemize}
\item {Grp. gram.:V. i.}
\end{itemize}
\begin{itemize}
\item {Utilização:Fig.}
\end{itemize}
\begin{itemize}
\item {Proveniência:(De \textunderscore triste\textunderscore )}
\end{itemize}
Tornar triste.
Tornar-se triste.
Estiolar-se, murchar.
\section{Entristecimento}
\begin{itemize}
\item {Grp. gram.:m.}
\end{itemize}
Acto ou effeito de entristecer.
\section{Entriteiras}
\begin{itemize}
\item {Grp. gram.:f. pl.}
\end{itemize}
\begin{itemize}
\item {Utilização:Prov.}
\end{itemize}
Peças, que, saindo dos coucilhões, como que abraçam o eixo do carro, de um e outro lado, segurando-o ao respectivo taboleiro.
(Cp. \textunderscore treitoeira\textunderscore )
\section{Entrizar-se}
\begin{itemize}
\item {Grp. gram.:v. p.}
\end{itemize}
\begin{itemize}
\item {Utilização:Prov.}
\end{itemize}
\begin{itemize}
\item {Utilização:trasm.}
\end{itemize}
Erguer-se para resistir.
(Relaciona-se com \textunderscore inteiriçar-se\textunderscore ?)
\section{Entroido}
\begin{itemize}
\item {Grp. gram.:m.}
\end{itemize}
\begin{itemize}
\item {Utilização:Ant.}
\end{itemize}
\begin{itemize}
\item {Proveniência:(Do lat. \textunderscore introitus\textunderscore )}
\end{itemize}
O mesmo que \textunderscore entrudo\textunderscore .
\section{Entroixar}
\begin{itemize}
\item {Grp. gram.:v. t.}
\end{itemize}
\begin{itemize}
\item {Proveniência:(Do cast. \textunderscore entrojar\textunderscore )}
\end{itemize}
Meter em troixa; embrulhar.
Arrumar.
Dar fórma de troixa a.
\section{Entrolhos}
\begin{itemize}
\item {Grp. gram.:m. pl.}
\end{itemize}
(V.antolhos)
\section{Entrombar-se}
\begin{itemize}
\item {Grp. gram.:v. p.}
\end{itemize}
Mostrar má cara, amuar:«\textunderscore e vai eu entrombei-me também\textunderscore ». Camillo, \textunderscore Corja\textunderscore .
\section{Entronar}
\textunderscore v. t.\textunderscore  (e der.)
O mesmo que \textunderscore entronizar\textunderscore , etc.
\section{Entroncado}
\begin{itemize}
\item {Grp. gram.:adj.}
\end{itemize}
\begin{itemize}
\item {Proveniência:(De \textunderscore entroncar\textunderscore )}
\end{itemize}
Corpulento, espadaúdo.
\section{Entroncamento}
\begin{itemize}
\item {Grp. gram.:m.}
\end{itemize}
\begin{itemize}
\item {Proveniência:(De \textunderscore entroncar\textunderscore )}
\end{itemize}
Lugar, em que entroncam dois ou mais caminhos, duas ou mais coisas.
Estação de caminho de ferro, onde se cruzam ou bifurcam duas ou mais linhas.
\section{Entroncar}
\begin{itemize}
\item {Grp. gram.:v. t.}
\end{itemize}
\begin{itemize}
\item {Grp. gram.:V. i.  e  p.}
\end{itemize}
\begin{itemize}
\item {Proveniência:(De \textunderscore tronco\textunderscore )}
\end{itemize}
Fazer reunir, fazer convergir.
Criar tronco, engrossar.
Reunir-se, convergir.
Ligar-se a um tronco de geração.
\section{Entronchado}
\begin{itemize}
\item {Grp. gram.:adj.}
\end{itemize}
\begin{itemize}
\item {Proveniência:(De \textunderscore entronchar\textunderscore )}
\end{itemize}
Que entronchou.
\section{Entronchar}
\begin{itemize}
\item {Grp. gram.:v. i.}
\end{itemize}
\begin{itemize}
\item {Proveniência:(De \textunderscore troncho\textunderscore )}
\end{itemize}
Tornar-se tronchudo.
\section{Entronear}
\textunderscore v. t.\textunderscore  (e der.)
O mesmo que \textunderscore entronar\textunderscore , etc. Cf. \textunderscore Eufrosina\textunderscore , pról.
\section{Entronização}
\begin{itemize}
\item {Grp. gram.:f.}
\end{itemize}
Acto ou efeito de \textunderscore entronizar\textunderscore .
\section{Entronizar}
\begin{itemize}
\item {Grp. gram.:v. t.}
\end{itemize}
\begin{itemize}
\item {Utilização:Fig.}
\end{itemize}
\begin{itemize}
\item {Proveniência:(De \textunderscore trono\textunderscore )}
\end{itemize}
Pôr no trono.
Exaltar; elevar muito.
\section{Entronquecer}
\begin{itemize}
\item {Grp. gram.:v. i.}
\end{itemize}
O mesmo que \textunderscore entroncar\textunderscore .
\section{Entronquecido}
\begin{itemize}
\item {Grp. gram.:adj.}
\end{itemize}
\begin{itemize}
\item {Utilização:Bot.}
\end{itemize}
\begin{itemize}
\item {Proveniência:(De \textunderscore entronquecer\textunderscore )}
\end{itemize}
Diz-se da planta, que é provida de tronco.
\section{Entropeçar}
\begin{itemize}
\item {Grp. gram.:v. i.}
\end{itemize}
(V.tropeçar)
\section{Entropêço}
\begin{itemize}
\item {Grp. gram.:m.}
\end{itemize}
(V. \textunderscore tropêço\textunderscore ^1)
\section{Entropicar}
\begin{itemize}
\item {Grp. gram.:v. i.}
\end{itemize}
\begin{itemize}
\item {Utilização:Bras. do N}
\end{itemize}
Tropeçar.
Andar com pouca firmeza.
\section{Entropilhar}
\begin{itemize}
\item {Grp. gram.:v. t.}
\end{itemize}
\begin{itemize}
\item {Utilização:Bras. do S}
\end{itemize}
Reunir (cavallos) em tropilha.
\section{Entrópio}
\begin{itemize}
\item {Grp. gram.:m.}
\end{itemize}
O mesmo ou melhor que \textunderscore entrópion\textunderscore .
\section{Entrópion}
\begin{itemize}
\item {Grp. gram.:m.}
\end{itemize}
\begin{itemize}
\item {Proveniência:(Do gr. \textunderscore en\textunderscore  + \textunderscore trepein\textunderscore )}
\end{itemize}
Reviramento do bôrdo livre da pálpebra para dentro do ôlho.
\section{Entrós}
\begin{itemize}
\item {Grp. gram.:f.}
\end{itemize}
O mesmo que \textunderscore entrosa\textunderscore .
\section{Entrosa}
\begin{itemize}
\item {Grp. gram.:f.}
\end{itemize}
\begin{itemize}
\item {Proveniência:(Do lat. \textunderscore entorsus\textunderscore ?)}
\end{itemize}
Roda dentada, que engranza noutra.
\section{Entrosação}
\begin{itemize}
\item {Grp. gram.:f.}
\end{itemize}
Acto ou effeito de entrosar.
O mesmo ou melhor que \textunderscore engrenagem\textunderscore .
\section{Entrosar}
\begin{itemize}
\item {Grp. gram.:v. t.}
\end{itemize}
\begin{itemize}
\item {Utilização:Fig.}
\end{itemize}
\begin{itemize}
\item {Proveniência:(De \textunderscore entrosa\textunderscore )}
\end{itemize}
Engranzar; engrenar.
Dispor em bôa ordem (coisas complicadas).
\section{Entrosga}
\begin{itemize}
\item {Grp. gram.:f.}
\end{itemize}
\begin{itemize}
\item {Utilização:Prov.}
\end{itemize}
\begin{itemize}
\item {Utilização:trasm.}
\end{itemize}
Vão, entre cantarias parallelas, onde gira uma roda, na azenha.
(Cp. \textunderscore entrosa\textunderscore )
\section{Entrouxar}
\begin{itemize}
\item {Grp. gram.:v. t.}
\end{itemize}
\begin{itemize}
\item {Proveniência:(Do cast. \textunderscore entrojar\textunderscore )}
\end{itemize}
Meter em trouxa; embrulhar.
Arrumar.
Dar fórma de trouxa a.
\section{Entroviscada}
\begin{itemize}
\item {Grp. gram.:f.}
\end{itemize}
\begin{itemize}
\item {Proveniência:(De \textunderscore entroviscar\textunderscore )}
\end{itemize}
Pesca de peixes, envenenando-se êstes com trovisco.
\section{Entroviscador}
\begin{itemize}
\item {Grp. gram.:m.}
\end{itemize}
Homem, que se mascara, para fazer a cresta das colmeias.
(Cp. \textunderscore entroviscar-se\textunderscore )
\section{Entroviscar}
\begin{itemize}
\item {Grp. gram.:v. t.}
\end{itemize}
\begin{itemize}
\item {Utilização:Fig.}
\end{itemize}
Espalhar raíz de trovisco em (o fundo de um pégo), para matar o peixe.
Envenenar (peixes) com trovisco.
Indispor, malquistar.
\section{Entroviscar-se}
\begin{itemize}
\item {Grp. gram.:v. p.}
\end{itemize}
\begin{itemize}
\item {Utilização:Fig.}
\end{itemize}
Ennublar-se, turvar-se (o céu).
Complicar-se.
(Por \textunderscore enturviscar-se\textunderscore , de \textunderscore enturvar\textunderscore )
\section{Entrozagem}
\begin{itemize}
\item {Grp. gram.:f.}
\end{itemize}
Acto ou effeito de entrozar.
\section{Entrozar}
\begin{itemize}
\item {Grp. gram.:v. i.}
\end{itemize}
\begin{itemize}
\item {Utilização:Bras. do Ceará}
\end{itemize}
Gabar-se.
Impor; parecer o que não é.
(Cp. \textunderscore intrujar\textunderscore )
\section{Entrudada}
\begin{itemize}
\item {Grp. gram.:f.}
\end{itemize}
\begin{itemize}
\item {Proveniência:(De \textunderscore entrudar\textunderscore )}
\end{itemize}
Folgança carnavalesca.
\section{Entrudal}
\begin{itemize}
\item {Grp. gram.:adj.}
\end{itemize}
Semelhante ou relativo ao Entrudo. Cf. Castilho, \textunderscore Metam.\textunderscore , 292.
\section{Entrudar}
\begin{itemize}
\item {Grp. gram.:v. t.}
\end{itemize}
\begin{itemize}
\item {Grp. gram.:V. i.}
\end{itemize}
\begin{itemize}
\item {Grp. gram.:V. p.}
\end{itemize}
\begin{itemize}
\item {Proveniência:(De \textunderscore entrudo\textunderscore )}
\end{itemize}
Dirigir pulhas carnavalescas a; fazer partidas ou pregar peças de Entrudo a.
Motejar de.
Jogar o Entrudo; divertir-se e gozar no Entrudo.
Causar enganos innocentes, como no Entrudo:«\textunderscore mero brinco, dêstes com que tantas vezes se entruda na república literária...\textunderscore »Castilho, \textunderscore Camões\textunderscore , II, 118.
\section{Entrudesco}
\begin{itemize}
\item {fónica:dês}
\end{itemize}
\begin{itemize}
\item {Grp. gram.:adj.}
\end{itemize}
Relativo ao Entrudo; próprio do Entrudo.
\section{Entrudo}
\begin{itemize}
\item {Grp. gram.:m.}
\end{itemize}
\begin{itemize}
\item {Utilização:Fig.}
\end{itemize}
\begin{itemize}
\item {Utilização:Prov.}
\end{itemize}
\begin{itemize}
\item {Utilização:trasm.}
\end{itemize}
\begin{itemize}
\item {Proveniência:(Do lat. \textunderscore introitus\textunderscore )}
\end{itemize}
O mesmo que \textunderscore Carnaval\textunderscore .
Indivíduo, que se apresenta ridiculamente.
Pessôa gorda.
\section{Entruido}
\begin{itemize}
\item {Grp. gram.:m.}
\end{itemize}
\begin{itemize}
\item {Utilização:Ant.}
\end{itemize}
O mesmo que \textunderscore entrudo\textunderscore .
\section{Entruito}
\begin{itemize}
\item {Grp. gram.:m.}
\end{itemize}
\begin{itemize}
\item {Utilização:Des.}
\end{itemize}
O mesmo que \textunderscore entrudo\textunderscore . Cf. J. Ribeiro, \textunderscore Frases Feitas\textunderscore , I, 160.
\section{Entrugir}
\begin{itemize}
\item {Grp. gram.:v. t.  e  i.}
\end{itemize}
O mesmo que \textunderscore intrugir\textunderscore .
\section{Entrujar}
\textunderscore v. t.\textunderscore  (e der.)
O mesmo ou melhor que \textunderscore intrujar\textunderscore , etc.
\section{Entrunfar-se}
\begin{itemize}
\item {Grp. gram.:v. p.}
\end{itemize}
\begin{itemize}
\item {Utilização:Prov.}
\end{itemize}
Amuar-se.
Zangar-se. (Colhido no Fundão)
\section{Entufar}
\begin{itemize}
\item {Grp. gram.:v. t.}
\end{itemize}
\begin{itemize}
\item {Proveniência:(De \textunderscore tufo\textunderscore )}
\end{itemize}
Tornar inchado; entumecer.
Tornar arrogante, vaidoso.
Tufar.
\section{Entujucar}
\textunderscore v. t.\textunderscore  (e der.)
O mesmo que \textunderscore entijucar\textunderscore , etc.
\section{Entulhamento}
\begin{itemize}
\item {Grp. gram.:m.}
\end{itemize}
Acto ou effeito de entulhar.
\section{Entulhar}
\begin{itemize}
\item {Grp. gram.:v. t.}
\end{itemize}
\begin{itemize}
\item {Proveniência:(De \textunderscore tulha\textunderscore )}
\end{itemize}
Meter em tulha.
Encher com entulho.
Encher.
Abarrotar.
Enfartar; empaturrar.
\section{Entulheira}
\begin{itemize}
\item {Grp. gram.:f.}
\end{itemize}
Recanto ou lugar, onde se deitam ou reúnem entulhos.
\section{Entulho}
\begin{itemize}
\item {Grp. gram.:m.}
\end{itemize}
Aquillo que enche ou entupe uma cavidade ou fôsso.
Montão de caliça, proveniente de desmoronamento.
Acto ou effeito de entulhar.
\section{Entumecência}
\begin{itemize}
\item {Grp. gram.:f.}
\end{itemize}
Acto de entumecer.
Estado de entumecente.
\section{Entumecente}
\begin{itemize}
\item {Grp. gram.:adj.}
\end{itemize}
\begin{itemize}
\item {Proveniência:(Lat. \textunderscore intumescens\textunderscore )}
\end{itemize}
Que entumeceu.
Inchado.
Em que há tumor.
\section{Entumecer}
\begin{itemize}
\item {Grp. gram.:v. i.  e  p.}
\end{itemize}
\begin{itemize}
\item {Grp. gram.:V. t.}
\end{itemize}
\begin{itemize}
\item {Utilização:Fig.}
\end{itemize}
\begin{itemize}
\item {Proveniência:(Lat. \textunderscore intumescere\textunderscore )}
\end{itemize}
Tornar-se túmido.
Inchar.
Avolumar-se.
Tornar túmido.
Tornar soberbo.
\section{Entumecimento}
\begin{itemize}
\item {Grp. gram.:m.}
\end{itemize}
O mesmo que \textunderscore entumecência\textunderscore .
\section{Entuna}
\begin{itemize}
\item {Grp. gram.:f.}
\end{itemize}
\begin{itemize}
\item {Utilização:Ant.}
\end{itemize}
\begin{itemize}
\item {Proveniência:(De \textunderscore tuna\textunderscore )}
\end{itemize}
Acto de andar a monte, caçando ou vadiando.
\section{Entunicado}
\begin{itemize}
\item {Grp. gram.:adj.}
\end{itemize}
\begin{itemize}
\item {Utilização:Bot.}
\end{itemize}
\begin{itemize}
\item {Proveniência:(De \textunderscore tunica\textunderscore )}
\end{itemize}
Que tem túnicas ou lâminas concêntricas.
\section{Entupimento}
\begin{itemize}
\item {Grp. gram.:m.}
\end{itemize}
Acto ou effeito de entupir.
\section{Entupir}
\begin{itemize}
\item {Grp. gram.:v. t.}
\end{itemize}
\begin{itemize}
\item {Utilização:Fig.}
\end{itemize}
Entulhar, obstruir.
Obstar á secreção de.
Embaraçar, embatucar.
Tornar insensível.
(Do germ., seg. Körting)
\section{Enturbar}
\begin{itemize}
\item {Grp. gram.:v. t.}
\end{itemize}
O mesmo que \textunderscore enturvar\textunderscore .
\section{Enturgecer}
\begin{itemize}
\item {Grp. gram.:v. i.}
\end{itemize}
Tornar-se túrgido.
Entumecer, inchar:«\textunderscore as veias da face enturgeciam de sangue\textunderscore ». Camillo, \textunderscore Neta do Arced.\textunderscore , 86.
\section{Enturida}
\begin{itemize}
\item {Grp. gram.:adj. f.}
\end{itemize}
\begin{itemize}
\item {Utilização:Prov.}
\end{itemize}
\begin{itemize}
\item {Utilização:trasm.}
\end{itemize}
O mesmo que \textunderscore entourada\textunderscore .
\section{Enturvação}
\begin{itemize}
\item {Grp. gram.:f.}
\end{itemize}
Acto ou effeito de enturvar.
\section{Enturvar}
\begin{itemize}
\item {Grp. gram.:v. t.}
\end{itemize}
Tornar turvo; perturbar.
Ensombrar.
Entristecer.
\section{Enturvescer}
\begin{itemize}
\item {Grp. gram.:v. i.}
\end{itemize}
Tornar-se turvo. Cf. \textunderscore Tech. Rur.\textunderscore , 24, 70 e 120.
\section{Enturviscar}
\begin{itemize}
\item {Grp. gram.:v. i.  e  p.}
\end{itemize}
Tornar-se turvo, embrulhar-se, (falando-se do tempo).
\section{Entusiasmado}
\begin{itemize}
\item {Grp. gram.:adj.}
\end{itemize}
\begin{itemize}
\item {Utilização:Fig.}
\end{itemize}
Que tem entusiasmo.
Animado por bom êxito já obtido.
\section{Entusiasmar}
\begin{itemize}
\item {Grp. gram.:v. t.}
\end{itemize}
Causar entusiasmo ou admiração a.
\section{Entusiasmo}
\begin{itemize}
\item {Grp. gram.:m.}
\end{itemize}
\begin{itemize}
\item {Utilização:Ant.}
\end{itemize}
\begin{itemize}
\item {Proveniência:(Gr. \textunderscore enthousiasmos\textunderscore )}
\end{itemize}
Excitação da alma, quando admira excessivamente.
Arrebatamento.
Paixão viva.
Alegria ruidosa.
Furor ou arrebatamento desordenado, que se atribuía a inspirações divinas.
\section{Entusiasta}
\begin{itemize}
\item {Grp. gram.:m.  e  adj.}
\end{itemize}
\begin{itemize}
\item {Proveniência:(Gr. \textunderscore enthousiastes\textunderscore )}
\end{itemize}
O que se entusiasma.
Quem se dedica vivamente: \textunderscore entusiasta pela música\textunderscore .
Quem se exprime com entusiasmo.
\section{Entusiasticamente}
\begin{itemize}
\item {Grp. gram.:adv.}
\end{itemize}
De modo entusiástico.
\section{Entusiástico}
\begin{itemize}
\item {Grp. gram.:adj.}
\end{itemize}
\begin{itemize}
\item {Proveniência:(De \textunderscore entusiasta\textunderscore )}
\end{itemize}
Em que há entusiásmo: \textunderscore saudações entusiásticas\textunderscore .
\section{Entuviada}
\begin{itemize}
\item {Grp. gram.:f.}
\end{itemize}
\begin{itemize}
\item {Utilização:Ant.}
\end{itemize}
\begin{itemize}
\item {Proveniência:(Do cast. \textunderscore antuviada\textunderscore )}
\end{itemize}
Pressa.
Desordem.
\section{Enucleação}
\begin{itemize}
\item {Grp. gram.:f.}
\end{itemize}
Acto ou effeito de enuclear.
\section{Enucleado}
\begin{itemize}
\item {Grp. gram.:adj.}
\end{itemize}
\begin{itemize}
\item {Proveniência:(De \textunderscore enuclear\textunderscore )}
\end{itemize}
Simples, corrente, claro:«\textunderscore a sua escrita era amena e original, nervosa e enucleada\textunderscore ». Latino, \textunderscore Elog.\textunderscore , I, 356.
\section{Enuclear}
\begin{itemize}
\item {Grp. gram.:v. t.}
\end{itemize}
\begin{itemize}
\item {Utilização:Fig.}
\end{itemize}
\begin{itemize}
\item {Proveniência:(Lat. \textunderscore enucleare\textunderscore )}
\end{itemize}
Extirpar, tirar o núcleo de (um tumor).
Tirar o núcleo ou caroço de (frutos).
Esclarecer, explicar:«\textunderscore ... que as occorrências enucleia ambíguas\textunderscore ». Filinto, VIII, 87.
\section{Énula}
\begin{itemize}
\item {Grp. gram.:f.}
\end{itemize}
\begin{itemize}
\item {Proveniência:(Do lat. \textunderscore inula\textunderscore )}
\end{itemize}
Erva medicinal.
\section{Énula-campana}
\begin{itemize}
\item {Grp. gram.:f.}
\end{itemize}
\begin{itemize}
\item {Proveniência:(Do lat. \textunderscore inula\textunderscore )}
\end{itemize}
Erva medicinal.
\section{Enumeração}
\begin{itemize}
\item {Grp. gram.:f.}
\end{itemize}
\begin{itemize}
\item {Proveniência:(Lat. \textunderscore enumeratio\textunderscore )}
\end{itemize}
Acto ou effeito de enumerar.
\section{Enumerador}
\begin{itemize}
\item {Grp. gram.:adj.}
\end{itemize}
\begin{itemize}
\item {Grp. gram.:M.}
\end{itemize}
Que enumera.
Aquelle que enumera.
\section{Enumerar}
\begin{itemize}
\item {Grp. gram.:v. t.}
\end{itemize}
\begin{itemize}
\item {Proveniência:(Lat. \textunderscore enumerare\textunderscore )}
\end{itemize}
Enunciar ou expor, um a um.
Numerar.
Contar por partes.
Narrar minuciosamente; especificar.
Relacionar methodicamente.
\section{Enumerável}
\begin{itemize}
\item {Grp. gram.:adj.}
\end{itemize}
Que se póde enumerar.
\section{Enunciação}
\begin{itemize}
\item {Grp. gram.:f.}
\end{itemize}
\begin{itemize}
\item {Proveniência:(Lat. \textunderscore enunciatio\textunderscore )}
\end{itemize}
Acto ou effeito de enunciar.
\section{Enunciado}
\begin{itemize}
\item {Grp. gram.:m.}
\end{itemize}
\begin{itemize}
\item {Proveniência:(De \textunderscore enunciar\textunderscore )}
\end{itemize}
Proposição; these, que se há de demonstrar.
\section{Enunciador}
\begin{itemize}
\item {Grp. gram.:adj.}
\end{itemize}
\begin{itemize}
\item {Grp. gram.:M.}
\end{itemize}
\begin{itemize}
\item {Proveniência:(Lat. \textunderscore enunciator\textunderscore )}
\end{itemize}
Que enuncia.
Aquelle que enuncia.
\section{Enunciar}
\begin{itemize}
\item {Grp. gram.:v. t.}
\end{itemize}
\begin{itemize}
\item {Proveniência:(Lat. \textunderscore enunciare\textunderscore )}
\end{itemize}
Expor, exprimir, proferir.
Manifestar: \textunderscore enunciar sentimentos\textunderscore .
\section{Enunciativa}
\begin{itemize}
\item {Grp. gram.:f.}
\end{itemize}
\begin{itemize}
\item {Utilização:Des.}
\end{itemize}
\begin{itemize}
\item {Proveniência:(De \textunderscore enunciativo\textunderscore )}
\end{itemize}
Relatório, exposição.
\section{Enunciativo}
\begin{itemize}
\item {Grp. gram.:adj.}
\end{itemize}
\begin{itemize}
\item {Proveniência:(Lat. \textunderscore enunciativus\textunderscore )}
\end{itemize}
Que enuncia.
Que serve para enunciar.
\section{Enuno}
\begin{itemize}
\item {Grp. gram.:adv.}
\end{itemize}
\begin{itemize}
\item {Utilização:Ant.}
\end{itemize}
O mesmo que \textunderscore ensembra\textunderscore .
(Da loc. lat. \textunderscore in unum\textunderscore )
\section{Enuresia}
\begin{itemize}
\item {Grp. gram.:f.}
\end{itemize}
\begin{itemize}
\item {Proveniência:(Do gr. \textunderscore en\textunderscore  + \textunderscore ouresis\textunderscore )}
\end{itemize}
Incontinencia de urina.
\section{Envaca}
\begin{itemize}
\item {Grp. gram.:f.}
\end{itemize}
\begin{itemize}
\item {Utilização:T. de Turquel}
\end{itemize}
O mesmo que \textunderscore aiveca\textunderscore .
\section{Envaginado}
\begin{itemize}
\item {Grp. gram.:adj.}
\end{itemize}
\begin{itemize}
\item {Proveniência:(Do lat. \textunderscore vaginatus\textunderscore )}
\end{itemize}
Diz-se do vegetal, que tem as fôlhas dispostas na base, por fórma que esta parece estar metida em baínha.
\section{Envaginante}
\begin{itemize}
\item {Grp. gram.:adj.}
\end{itemize}
\begin{itemize}
\item {Utilização:Bot.}
\end{itemize}
\begin{itemize}
\item {Proveniência:(Do lat. \textunderscore vagina\textunderscore )}
\end{itemize}
Que cinge o tronco ou ramo em fórma de baínha.
\section{Envaidar}
\begin{itemize}
\item {Grp. gram.:v. t.}
\end{itemize}
Tornar vaidoso; desvanecer.
Entufar.
(Por \textunderscore envaidadar\textunderscore , de \textunderscore vaidade\textunderscore )
\section{Envaidecer}
\begin{itemize}
\item {Grp. gram.:v. t.}
\end{itemize}
(V.envaidar). Cf. Camillo, \textunderscore Noites de Insómn.\textunderscore , X, 53; \textunderscore Narcót.\textunderscore , I, 202.
\section{Envalar}
\begin{itemize}
\item {Grp. gram.:v. t.}
\end{itemize}
\begin{itemize}
\item {Proveniência:(De \textunderscore vala\textunderscore )}
\end{itemize}
Fortificar com valas ou fossos; entrincheirar.
\section{Envalecer}
\begin{itemize}
\item {Grp. gram.:v. i.}
\end{itemize}
\begin{itemize}
\item {Proveniência:(Do lat. \textunderscore valescere\textunderscore )}
\end{itemize}
Tornar-se válido, restabelecer-se de saúde. Cf. Usque, \textunderscore Tribulações\textunderscore .
\section{Envallar}
\begin{itemize}
\item {Grp. gram.:v. t.}
\end{itemize}
\begin{itemize}
\item {Proveniência:(De \textunderscore valla\textunderscore )}
\end{itemize}
Fortificar com vallas ou fossos; entrincheirar.
\section{Envanecido}
\begin{itemize}
\item {Grp. gram.:adj.}
\end{itemize}
Que tem vaidade.
(Cp. lat. \textunderscore vanescere\textunderscore )
\section{Envaris}
\begin{itemize}
\item {Grp. gram.:m.}
\end{itemize}
Animal do Congo, semelhante ao veado.
\section{Envasadura}
\begin{itemize}
\item {Grp. gram.:f.}
\end{itemize}
Acto ou effeito de \textunderscore envasar\textunderscore .
\section{Envasamento}
\begin{itemize}
\item {Grp. gram.:m.}
\end{itemize}
\begin{itemize}
\item {Proveniência:(De \textunderscore envasar\textunderscore )}
\end{itemize}
Parte inferior e mais larga de um cunhal.
Base de columna.
\section{Envasar}
\begin{itemize}
\item {Grp. gram.:v. t.}
\end{itemize}
Meter em vaso; envasilhar.
Dar fórma de vaso a.
Fazer o envasamento de.
\section{Envasilhação}
\begin{itemize}
\item {Grp. gram.:f.}
\end{itemize}
O mesmo que \textunderscore envasilhagem\textunderscore . Cf. \textunderscore Techn. Rur.\textunderscore , 215 e 375.
\section{Envasilhagem}
\begin{itemize}
\item {Grp. gram.:f.}
\end{itemize}
O mesmo que \textunderscore envasilhamento\textunderscore . Cf. Aguiar, \textunderscore Proc. de Vinif.\textunderscore , 44.
\section{Envasilhamento}
\begin{itemize}
\item {Grp. gram.:m.}
\end{itemize}
Acto ou effeito de envasilhar.
\section{Envasilhar}
\begin{itemize}
\item {Grp. gram.:v. t.}
\end{itemize}
Deitar (liquidos) em vasilhas.
Meter em pipas ou tonéis.
Engarrafar.
\section{Envazadura}
\begin{itemize}
\item {Grp. gram.:f.}
\end{itemize}
\begin{itemize}
\item {Proveniência:(De \textunderscore envazar\textunderscore )}
\end{itemize}
Espeques do navio, quando se está construindo.
\section{Envazar}
\begin{itemize}
\item {Grp. gram.:v. t.}
\end{itemize}
Meter em vaza^2.
Sustentar com envazadura.
\section{Envaziado}
\begin{itemize}
\item {Grp. gram.:m.}
\end{itemize}
\begin{itemize}
\item {Utilização:Carp.}
\end{itemize}
\begin{itemize}
\item {Proveniência:(De \textunderscore vazio\textunderscore )}
\end{itemize}
Ranhura, aberta com um cantil na face estreita da coiceira, onde se encaixa a almofada da porta ou janela.
\section{Enveja}
\begin{itemize}
\item {Grp. gram.:f.}
\end{itemize}
\begin{itemize}
\item {Grp. gram.:Pl. Loc. adv.}
\end{itemize}
\begin{itemize}
\item {Proveniência:(Do lat. \textunderscore invidia\textunderscore )}
\end{itemize}
(como escreviam clássicos)
O mesmo ou melhor que \textunderscore inveja\textunderscore .
Tristeza ou desgôsto pela prosperidade ou fortuna alheia.
Desejo excessivo de possuír exclusivamente o bem de outrem.
Objecto envejado.
\textunderscore Ás envejas\textunderscore , á porfia:«\textunderscore todos os portugueses ás envejas entraram no lavor dos muros.\textunderscore »Filinto, \textunderscore D. Man.\textunderscore , II, 91.
\section{Envelhacar}
\begin{itemize}
\item {Grp. gram.:v. t.}
\end{itemize}
Tornar velhaco.
\section{Envelhecedor}
\begin{itemize}
\item {Grp. gram.:adj.}
\end{itemize}
Que faz envelhecer. Cf. \textunderscore Techn. Rur.\textunderscore , 264.
\section{Envelhecer}
\begin{itemize}
\item {Grp. gram.:v. t.}
\end{itemize}
\begin{itemize}
\item {Grp. gram.:V. i.}
\end{itemize}
Tornar velho.
Tornar-se velho.
Parecer velho.
\section{Envelhecido}
\begin{itemize}
\item {Grp. gram.:adj.}
\end{itemize}
Que envelheceu.
Decadente.
\section{Envelhecimento}
\begin{itemize}
\item {Grp. gram.:m.}
\end{itemize}
Acto ou effeito de envelhecer.
\section{Envelhentamento}
\begin{itemize}
\item {Grp. gram.:m.}
\end{itemize}
Acto ou effeito de envelhentar.
\section{Envelhentar}
\textunderscore v. t.\textunderscore  e \textunderscore p.\textunderscore  (e der)
O mesmo que \textunderscore avelhentar\textunderscore , etc.
\section{Envelhido}
\begin{itemize}
\item {Grp. gram.:adj.}
\end{itemize}
\begin{itemize}
\item {Utilização:Prov.}
\end{itemize}
\begin{itemize}
\item {Utilização:trasm.}
\end{itemize}
Envelhecido.
\section{Envencilhar}
\textunderscore v. t.\textunderscore  (e der)
O mesmo que \textunderscore envincilhar\textunderscore , etc.
\section{Envenenado}
\begin{itemize}
\item {Grp. gram.:adj.}
\end{itemize}
\begin{itemize}
\item {Utilização:Fig.}
\end{itemize}
\begin{itemize}
\item {Proveniência:(De \textunderscore envenenar\textunderscore )}
\end{itemize}
Em que se lançou veneno: \textunderscore bebida envenenada\textunderscore .
Que tomou veneno.
A quem se ministrou veneno: \textunderscore morreu envenenado\textunderscore .
Eivado de ódio ou de má vontade: \textunderscore palavras envenenadas\textunderscore .
Praguento, virulento: \textunderscore lingua envenenada\textunderscore .
\section{Envenenador}
\begin{itemize}
\item {Grp. gram.:adj.}
\end{itemize}
\begin{itemize}
\item {Grp. gram.:M.}
\end{itemize}
Que envenena.
Aquelle que envenena.
\section{Envenenamento}
\begin{itemize}
\item {Grp. gram.:m.}
\end{itemize}
Acto ou effeito de envenenar.
\section{Envenenar}
\begin{itemize}
\item {Grp. gram.:v. t.}
\end{itemize}
\begin{itemize}
\item {Utilização:Fig.}
\end{itemize}
Misturar veneno em: \textunderscore envenenar água\textunderscore .
Propinar veneno a: \textunderscore envenenar alguém\textunderscore .
Perverter, estragar.
Tornar prejudicial.
Deturpar.
Interpretar em mau sentido: \textunderscore envenenar intenções\textunderscore .
\section{Enventanar}
\begin{itemize}
\item {Grp. gram.:v. t.}
\end{itemize}
Meter na ventanilha.
(Contr. de \textunderscore enventanilhar\textunderscore )
\section{Enverar}
\begin{itemize}
\item {Grp. gram.:v. t.}
\end{itemize}
\begin{itemize}
\item {Utilização:Prov.}
\end{itemize}
\begin{itemize}
\item {Utilização:trasm.}
\end{itemize}
Fixar ou fitar bem (a vista, os olhos).
\section{Enverdecer}
\begin{itemize}
\item {Grp. gram.:v. t.}
\end{itemize}
\begin{itemize}
\item {Utilização:Fig.}
\end{itemize}
\begin{itemize}
\item {Grp. gram.:V. i.}
\end{itemize}
\begin{itemize}
\item {Utilização:Fig.}
\end{itemize}
Tornar verde.
Cobrir de verdura.
Tornar moço, remoçar.
Tornar-se verde.
Reverdecer.
Verdejar.
Remoçar.
\section{Enverdejar}
\begin{itemize}
\item {Grp. gram.:v. i.}
\end{itemize}
O mesmo que \textunderscore enverdecer\textunderscore :«\textunderscore enverdejava e ria a parda terra.\textunderscore »Filinto, VIII, 265.
\section{Envereamento}
\begin{itemize}
\item {Grp. gram.:m.}
\end{itemize}
\begin{itemize}
\item {Utilização:Ant.}
\end{itemize}
\begin{itemize}
\item {Proveniência:(De \textunderscore enverear\textunderscore )}
\end{itemize}
O mesmo que \textunderscore vereação\textunderscore .
\section{Enverear}
\begin{itemize}
\item {Grp. gram.:v. i.}
\end{itemize}
\begin{itemize}
\item {Utilização:Ant.}
\end{itemize}
Exercer o cargo de vereador.
(Cp. \textunderscore vereador\textunderscore )
\section{Enveredar}
\begin{itemize}
\item {Grp. gram.:v. i.}
\end{itemize}
\begin{itemize}
\item {Utilização:Bras}
\end{itemize}
\begin{itemize}
\item {Grp. gram.:V. t.}
\end{itemize}
Seguir por vereda.
Tomar caminho; dirigir-se.
Dirigir-se expressamente a um dado lugar.
Guiar, encaminhar.
\section{Envergadura}
\begin{itemize}
\item {Grp. gram.:f.}
\end{itemize}
\begin{itemize}
\item {Utilização:Gal}
\end{itemize}
\begin{itemize}
\item {Utilização:Fig.}
\end{itemize}
\begin{itemize}
\item {Utilização:Gír.}
\end{itemize}
\begin{itemize}
\item {Proveniência:(De \textunderscore envergar\textunderscore )}
\end{itemize}
Largura das velas.
Toda a extensão das asas abertas de uma ave, de ponta a ponta.--Em vez dêste gallicismo, temos \textunderscore encontros\textunderscore .
Capacidade, pujança.
Envergamento.
Vestuário.
\section{Envergamento}
\begin{itemize}
\item {Grp. gram.:m.}
\end{itemize}
Acto de envergar.
\section{Envergar}
\begin{itemize}
\item {Grp. gram.:v. t.}
\end{itemize}
\begin{itemize}
\item {Proveniência:(De \textunderscore vêrga\textunderscore )}
\end{itemize}
Atar (as velas) ás vêrgas ou aos estais.
Vergar.
Vestir.
Enfiar pelos braços ou pernas (uma peça de vestuário).
\section{Envergonhado}
\begin{itemize}
\item {Grp. gram.:adj.}
\end{itemize}
\begin{itemize}
\item {Utilização:Fig.}
\end{itemize}
Que tem vergonha.
Tímido, acanhado.
Humilhado, abatido.
Que se mostra a furto: \textunderscore lágrimas envergonhadas\textunderscore .
\section{Envergonhador}
\begin{itemize}
\item {Grp. gram.:adj.}
\end{itemize}
Que causa vergonha. Cf. Garrett, \textunderscore Flores sem Fruto\textunderscore , 70.
\section{Envergonhar}
\begin{itemize}
\item {Grp. gram.:v. t.}
\end{itemize}
\begin{itemize}
\item {Grp. gram.:V. p.}
\end{itemize}
Encher de vergonha.
Tornar acanhado, tímido.
Aviltar.
Deslustrar.
Têr vergonha, pejo, acanhamento.
\section{Envergues}
\begin{itemize}
\item {Grp. gram.:m. pl.}
\end{itemize}
\begin{itemize}
\item {Proveniência:(De \textunderscore envergar\textunderscore )}
\end{itemize}
Cordéis, com que se atam as velas ás vêrgas da embarcação.
\section{Envermelhar}
\begin{itemize}
\item {Grp. gram.:v. t.}
\end{itemize}
Tornar vermelho; enrubescer.
\section{Envermelhecer}
\begin{itemize}
\item {Grp. gram.:v. i.}
\end{itemize}
Tornar-se vermelho.
\section{Envernizador}
\begin{itemize}
\item {Grp. gram.:m.}
\end{itemize}
Aquelle que enverniza.
Aquelle que tem a profissão de envernizar ou polir móveis de madeira.
\section{Envernizadura}
\begin{itemize}
\item {Grp. gram.:f.}
\end{itemize}
O mesmo que \textunderscore envernizamento\textunderscore .
\section{Envernizamento}
\begin{itemize}
\item {Grp. gram.:m.}
\end{itemize}
Acto ou effeito de envernizar.
\section{Envernizar}
\begin{itemize}
\item {Grp. gram.:v. t.}
\end{itemize}
\begin{itemize}
\item {Grp. gram.:V. p.}
\end{itemize}
\begin{itemize}
\item {Utilização:Burl.}
\end{itemize}
Lustrar com verniz.
Dar polimento a.
Embebedar-se.
\section{Enverrugar}
\begin{itemize}
\item {Grp. gram.:v. t.}
\end{itemize}
\begin{itemize}
\item {Grp. gram.:V. i.}
\end{itemize}
Fazer verrugas em.
Engelhar.
Encarquilhar.
Encrespar.
Amarrotar: \textunderscore enverrugar um pano\textunderscore .
Criar verrugas.
\section{Envés}
\begin{itemize}
\item {Grp. gram.:m.}
\end{itemize}
(V.invés)
\section{Envesgar}
\begin{itemize}
\item {Grp. gram.:v. t.}
\end{itemize}
Tornar vesgo.
Torcer (os olhos).
\section{Envessadamente}
\begin{itemize}
\item {Grp. gram.:adv.}
\end{itemize}
\begin{itemize}
\item {Proveniência:(De \textunderscore envessar\textunderscore )}
\end{itemize}
Do avesso.
Ao invés.
\section{Envessar}
\begin{itemize}
\item {Grp. gram.:v. t.}
\end{itemize}
\begin{itemize}
\item {Proveniência:(De \textunderscore envêsso\textunderscore )}
\end{itemize}
Dobrar, pondo o avesso para fóra.
Enfestar.
\section{Envêsso}
\begin{itemize}
\item {Grp. gram.:m.}
\end{itemize}
O mesmo que \textunderscore avesso\textunderscore .
\section{Enviada}
\begin{itemize}
\item {Grp. gram.:f.}
\end{itemize}
Barco, que recebe de outros o producto da pesca e o leva ao pôrto, como leva mantimentos aos barcos de pesca, fóra da barra.
Enviadeira.
(Do \textunderscore enviado\textunderscore )
\section{Enviadeira}
\begin{itemize}
\item {Grp. gram.:f.}
\end{itemize}
\begin{itemize}
\item {Proveniência:(De \textunderscore enviar\textunderscore )}
\end{itemize}
Uma das embarcações que acompanham a xávega.
\section{Enviadeiro}
\begin{itemize}
\item {Grp. gram.:m.}
\end{itemize}
Tripulante de enviada.
\section{Enviado}
\begin{itemize}
\item {Grp. gram.:m.}
\end{itemize}
\begin{itemize}
\item {Proveniência:(De \textunderscore enviar\textunderscore )}
\end{itemize}
Ministro de um Estado em país estrangeiro.
Commissário, mensageiro.
\section{Enviamento}
\begin{itemize}
\item {Grp. gram.:m.}
\end{itemize}
Acto de enviar.
\section{Enviar}
\begin{itemize}
\item {Grp. gram.:v. t.}
\end{itemize}
\begin{itemize}
\item {Proveniência:(Do b. lat. \textunderscore inviare\textunderscore )}
\end{itemize}
Pôr a caminho.
Encaminhar.
Mandar alguém ou alguma coisa.
Endereçar; remeter; expedir: \textunderscore enviar uma carta\textunderscore .
\section{Enviatura}
\begin{itemize}
\item {Grp. gram.:f.}
\end{itemize}
O mesmo que \textunderscore enviamento\textunderscore . Cf. Filinto, \textunderscore D. Man.\textunderscore , I, 44.
\section{Enviçado}
\begin{itemize}
\item {Grp. gram.:adj.}
\end{itemize}
Que adquiriu viço.
\section{Enviciado}
\begin{itemize}
\item {Grp. gram.:adj.}
\end{itemize}
\begin{itemize}
\item {Utilização:Prov.}
\end{itemize}
Diz-se dos animaes aluados ou tomados de cío. (Colhido em Turquel)
\section{Envidar}
\textunderscore v. t.\textunderscore  (e der.)
O mesmo ou melhor que \textunderscore invidar\textunderscore , etc.
\section{Envide}
\begin{itemize}
\item {Grp. gram.:m.}
\end{itemize}
\begin{itemize}
\item {Utilização:Pop.}
\end{itemize}
Parte do cordão umbilical, que fica presa á placenta.
\section{Envide}
\begin{itemize}
\item {Grp. gram.:f.}
\end{itemize}
Acto de envidar.
\section{Envidilha}
\begin{itemize}
\item {Grp. gram.:f.}
\end{itemize}
Acto de envidilhar.
\section{Envidilhar}
\begin{itemize}
\item {Grp. gram.:v. t.}
\end{itemize}
\begin{itemize}
\item {Proveniência:(De \textunderscore vide\textunderscore )}
\end{itemize}
Empar, fazendo círculos com a vara da vide e metendo a ponta della por dentro das voltas.
\section{Envido}
\begin{itemize}
\item {Grp. gram.:m.}
\end{itemize}
\begin{itemize}
\item {Utilização:Prov.}
\end{itemize}
\begin{itemize}
\item {Utilização:alent.}
\end{itemize}
O mesmo que \textunderscore envide\textunderscore ^2.
\section{Envidraçamento}
\begin{itemize}
\item {Grp. gram.:m.}
\end{itemize}
Acto de envidraçar.
\section{Envidraçar}
\begin{itemize}
\item {Grp. gram.:v. t.}
\end{itemize}
Pôr vidros ou vidraças em.
Guarnecer com vidraças.
Tornar vítreo, dar a apparência de vidro a:«\textunderscore ...com as lágrimas a envidraçarem-lhe os olhos...\textunderscore »Camillo, \textunderscore Volcões\textunderscore , 155.
\section{Envieirado}
\begin{itemize}
\item {Grp. gram.:adj.}
\end{itemize}
\begin{itemize}
\item {Utilização:Prov.}
\end{itemize}
\begin{itemize}
\item {Utilização:trasm.}
\end{itemize}
Que tem fibras ou fios: \textunderscore carne envieirada\textunderscore .
(Por \textunderscore enveeirado\textunderscore , de \textunderscore veio\textunderscore . Cp. \textunderscore entreveado\textunderscore )
\section{Envieirar}
\begin{itemize}
\item {Grp. gram.:v.}
\end{itemize}
\begin{itemize}
\item {Utilização:t. Marn.}
\end{itemize}
\begin{itemize}
\item {Proveniência:(De \textunderscore vieiro\textunderscore )}
\end{itemize}
Juntar (o sal) com o ugalho, para o lado do vieiro.
\section{Enviés}
\begin{itemize}
\item {Grp. gram.:m.}
\end{itemize}
(V.viés)
\section{Enviesadamente}
\begin{itemize}
\item {Grp. gram.:adv.}
\end{itemize}
De modo enviesado.
\section{Enviesado}
\begin{itemize}
\item {Grp. gram.:adj.}
\end{itemize}
\begin{itemize}
\item {Proveniência:(De \textunderscore enviesar\textunderscore )}
\end{itemize}
Pôsto ao viés.
Oblíquo; torto.
\section{Enviesar}
\begin{itemize}
\item {Grp. gram.:v. t.}
\end{itemize}
Pôr ao viés, de esguelha, obliquamente.
Entortar, dirigir mal.
\section{Envigorante}
\begin{itemize}
\item {Grp. gram.:adj.}
\end{itemize}
Que envigora.
\section{Envigorar}
\begin{itemize}
\item {Grp. gram.:v. t.}
\end{itemize}
Dar vigor a. Cf. Lobo, \textunderscore Sátiras\textunderscore , I, 216.
\section{Envilecer}
\begin{itemize}
\item {Grp. gram.:v. t.}
\end{itemize}
\begin{itemize}
\item {Grp. gram.:V. i.}
\end{itemize}
\begin{itemize}
\item {Utilização:Fig.}
\end{itemize}
Tornar vil, aviltar.
Deslustrar.
Tornar-se vil, desprezível.
Deminuir em preço, em valor.
\section{Envilecimento}
\begin{itemize}
\item {Grp. gram.:m.}
\end{itemize}
Acto ou effeito de envilecer.
\section{Envinagrar}
\begin{itemize}
\item {Grp. gram.:v. t.}
\end{itemize}
\begin{itemize}
\item {Utilização:fam.}
\end{itemize}
\begin{itemize}
\item {Utilização:Fig.}
\end{itemize}
\begin{itemize}
\item {Grp. gram.:V. p.}
\end{itemize}
\begin{itemize}
\item {Utilização:Prov.}
\end{itemize}
\begin{itemize}
\item {Utilização:dur.}
\end{itemize}
Azedar com vinagre.
Misturar vinagre em.
Azedar.
Avinagrar.
Avermelhar ou dar apparência de vinagre a.
Fazer zangar.
Irritar.
Embebedar-se.
\section{Envincilhar}
\begin{itemize}
\item {Grp. gram.:v. t.}
\end{itemize}
Ligar com vincilho.
Enredar; confundir.
\section{Envinhado}
\begin{itemize}
\item {Grp. gram.:adj.}
\end{itemize}
Misturado com vinho. Filinto, XXII, 17.
\section{Envio}
\begin{itemize}
\item {Grp. gram.:m.}
\end{itemize}
\begin{itemize}
\item {Utilização:Neol.}
\end{itemize}
Acto de enviar.
Remessa. (T. cast.)
\section{Enviperar}
\begin{itemize}
\item {Grp. gram.:v. t.}
\end{itemize}
\begin{itemize}
\item {Proveniência:(Do lat. \textunderscore vipera\textunderscore )}
\end{itemize}
Tornar assanhado como a vibora.
Irritar.
\section{Enviscação}
\begin{itemize}
\item {Grp. gram.:f.}
\end{itemize}
Acto de enviscar.
Acto de embeber em saliva os alimentos, durante a mastigação.
\section{Enviscar}
\begin{itemize}
\item {Grp. gram.:v. t.}
\end{itemize}
\begin{itemize}
\item {Utilização:Fig.}
\end{itemize}
\begin{itemize}
\item {Grp. gram.:V. p.}
\end{itemize}
\begin{itemize}
\item {Proveniência:(De \textunderscore visco\textunderscore )}
\end{itemize}
Cobrir ou untar com visco.
Attrahir; cativar.
Embaír, ludribiar.
Prender-se no visco.
\section{Envisgar}
\begin{itemize}
\item {Grp. gram.:v. t.}
\end{itemize}
\begin{itemize}
\item {Utilização:Prov.}
\end{itemize}
\begin{itemize}
\item {Utilização:dur.}
\end{itemize}
\begin{itemize}
\item {Grp. gram.:V. p.}
\end{itemize}
\begin{itemize}
\item {Utilização:Prov.}
\end{itemize}
\begin{itemize}
\item {Utilização:dur.}
\end{itemize}
O mesmo que \textunderscore enviscar\textunderscore . Cf. Camillo, \textunderscore Regicida\textunderscore , 15.
Passar destramente (alguma coisa) a alguém, que a não deseja.
Sujar os pés ou o calçado em excrementos.
\section{Envite}
\begin{itemize}
\item {Grp. gram.:m.}
\end{itemize}
(V.invite)
\section{Enviuvar}
\begin{itemize}
\item {fónica:vi-u}
\end{itemize}
\begin{itemize}
\item {Grp. gram.:v. t.}
\end{itemize}
\begin{itemize}
\item {Grp. gram.:V. i.}
\end{itemize}
Tornar viúvo.
Tornar-se viúvo.
\section{Enviusar}
\begin{itemize}
\item {fónica:vi-u}
\end{itemize}
\textunderscore v. t.\textunderscore  (e der.)
(Corr. de \textunderscore enviesar\textunderscore , etc.)
\section{Enviveirar}
\begin{itemize}
\item {Grp. gram.:v. t.}
\end{itemize}
Recolher em viveiro, para reproducção.
Conservar em viveiro; cultivar em viveiro.
\section{Envolar-se}
\begin{itemize}
\item {Grp. gram.:v. p.}
\end{itemize}
Alar-se, evolar-se. Cf. Camillo, \textunderscore Cav. em Ruínas\textunderscore , 74.
(Cp. \textunderscore evolar-se\textunderscore )
\section{Envolta}
\begin{itemize}
\item {Grp. gram.:f.}
\end{itemize}
\begin{itemize}
\item {Utilização:Prov.}
\end{itemize}
\begin{itemize}
\item {Utilização:alg.}
\end{itemize}
Curva na estrada.
Faixa, ligadura.
\section{Envolta}
\begin{itemize}
\item {fónica:vôl}
\end{itemize}
\begin{itemize}
\item {Grp. gram.:f.}
\end{itemize}
\begin{itemize}
\item {Utilização:Ant.}
\end{itemize}
\begin{itemize}
\item {Grp. gram.:Pl.}
\end{itemize}
\begin{itemize}
\item {Proveniência:(De \textunderscore envolto\textunderscore )}
\end{itemize}
Companhia.
Confusão; mistura.
Desordem, tumulto, briga. Cf. \textunderscore Alvará\textunderscore  de D. Sebast., in \textunderscore Rev. Lus.\textunderscore , XV, 142.
Intrigas.
Enredos.
\section{Envolto}
\begin{itemize}
\item {fónica:vôl}
\end{itemize}
\begin{itemize}
\item {Grp. gram.:adj.}
\end{itemize}
\begin{itemize}
\item {Proveniência:(Do lat. \textunderscore involutus\textunderscore )}
\end{itemize}
Que se envolveu.
Misturado; amalgamado.
\section{Envoltório}
\begin{itemize}
\item {Grp. gram.:m.}
\end{itemize}
\begin{itemize}
\item {Proveniência:(De \textunderscore envolto\textunderscore )}
\end{itemize}
O mesmo que \textunderscore invólucro\textunderscore .
\section{Envoltura}
\begin{itemize}
\item {Grp. gram.:f.}
\end{itemize}
\begin{itemize}
\item {Proveniência:(De \textunderscore envolto\textunderscore )}
\end{itemize}
Acto ou effeito do envolver.
Envolvedoiro.
\section{Envólucro}
\begin{itemize}
\item {Grp. gram.:m.}
\end{itemize}
(V.invólucro)
\section{Envolvedoiro}
\begin{itemize}
\item {Grp. gram.:m.}
\end{itemize}
Faixa, em que se envolvem as crianças recém-nascidas.
\section{Envolvedor}
\begin{itemize}
\item {Grp. gram.:adj.}
\end{itemize}
\begin{itemize}
\item {Grp. gram.:M.}
\end{itemize}
Que envolve.
Aquelle que envolve.
\section{Envolvedouro}
\begin{itemize}
\item {Grp. gram.:m.}
\end{itemize}
Faixa, em que se envolvem as crianças recém-nascidas.
\section{Envolvente}
\begin{itemize}
\item {Grp. gram.:adj.}
\end{itemize}
Que envolve, que abrange.
\section{Envolver}
\begin{itemize}
\item {Grp. gram.:v. t.}
\end{itemize}
\begin{itemize}
\item {Utilização:Fig.}
\end{itemize}
\begin{itemize}
\item {Proveniência:(Lat. \textunderscore involvere\textunderscore )}
\end{itemize}
Cobrir em tôrno.
Enfaixar.
Enrolar.
Encerrar.
Abranger: \textunderscore o caso envolve dois problemas\textunderscore .
Lançar em volta.
Embrulhar.
Enredar; intrigar.
Baralhar, confundir.
\section{Envolvimento}
\begin{itemize}
\item {Grp. gram.:m.}
\end{itemize}
Acto ou effeito de envolver.
\section{Enxabeque}
\begin{itemize}
\item {Grp. gram.:m.}
\end{itemize}
Embarcação antiga; o mesmo que \textunderscore xaveco\textunderscore . Cf. Azurara, \textunderscore Crón. de P.\textunderscore  II.
\section{Enxabidez}
\begin{itemize}
\item {fónica:xá}
\end{itemize}
\begin{itemize}
\item {Grp. gram.:f.}
\end{itemize}
Qualidade de enxabido:«\textunderscore ...neste mar morto de enxabidez em que a pusemos\textunderscore  (a imprensa)...»Camillo, \textunderscore Caveira\textunderscore , I, 41.
\section{Enxabido}
\begin{itemize}
\item {fónica:xá}
\end{itemize}
\begin{itemize}
\item {Grp. gram.:adj.}
\end{itemize}
O mesmo que \textunderscore desenxabido\textunderscore .--É t. ant. e ainda hoje usado nas prov. do norte.
\section{Enxabimento}
\begin{itemize}
\item {fónica:xá}
\end{itemize}
\begin{itemize}
\item {Grp. gram.:m.}
\end{itemize}
O mesmo que \textunderscore enxabidez\textunderscore . Cf. Castilho, \textunderscore Primavera\textunderscore , 153.
\section{Enxabregano}
\begin{itemize}
\item {Grp. gram.:adj.}
\end{itemize}
Relativo a Enxabregas, (hoje Xabregas):«\textunderscore frade enxabregano.\textunderscore »Camillo, \textunderscore Rom. de Um Homem Rico\textunderscore , 122.
\section{Enxaca}
\begin{itemize}
\item {Grp. gram.:f.}
\end{itemize}
\begin{itemize}
\item {Proveniência:(Do ár. \textunderscore xaca\textunderscore )}
\end{itemize}
Um dos lados do seirão que se põe nas bêstas de carga.
\section{Enxacoco}
\begin{itemize}
\item {fónica:cô}
\end{itemize}
\begin{itemize}
\item {Grp. gram.:m.  e  adj.}
\end{itemize}
\begin{itemize}
\item {Utilização:Des.}
\end{itemize}
Aquelle, que fala mal uma língua estranha, misturando-lhe termos da sua.
Mesclado, exótico:«\textunderscore ...no cimento do Circo-bazar-theatro-restaurante gymnástico, pyrotéchnico, chamado em linguagem enxacoca Palácio-de-Cristal.\textunderscore »Camillo, \textunderscore Bruxa\textunderscore , 1.^a p., c. XII.
\section{Enxada}
\begin{itemize}
\item {Grp. gram.:f.}
\end{itemize}
\begin{itemize}
\item {Utilização:Fig.}
\end{itemize}
\begin{itemize}
\item {Utilização:Bras}
\end{itemize}
Utensílio de ferro e aço, com que se cava a terra, com que se amassa cal, etc.
Meio, com que se adquirem os meios de subsistência: \textunderscore a minha enxada é a penna\textunderscore .
Peixe marítimo.
(Cp. cast. \textunderscore azada\textunderscore )
\section{Enxadada}
\begin{itemize}
\item {Grp. gram.:f.}
\end{itemize}
Golpe de enxada.
\section{Enxada-de-cavallo}
\begin{itemize}
\item {Grp. gram.:f.}
\end{itemize}
Apparelho agrícola, o mesmo que \textunderscore sachador\textunderscore .
\section{Enxadão}
\begin{itemize}
\item {Grp. gram.:m.}
\end{itemize}
Enxada grande.
\section{Enxadar}
\begin{itemize}
\item {Grp. gram.:v. t.}
\end{itemize}
\begin{itemize}
\item {Grp. gram.:V. i.}
\end{itemize}
Cavar com a enxada.
Trabalhar com a enxada.
\section{Enxadréa}
\begin{itemize}
\item {Grp. gram.:f.}
\end{itemize}
O mesmo que \textunderscore cardamina\textunderscore .
\section{Enxadreia}
\begin{itemize}
\item {Grp. gram.:f.}
\end{itemize}
O mesmo que \textunderscore cardamina\textunderscore .
\section{Enxadrez}
\begin{itemize}
\item {Grp. gram.:m.}
\end{itemize}
\begin{itemize}
\item {Utilização:Ant.}
\end{itemize}
O mesmo que \textunderscore xadrez\textunderscore .
\section{Enxadrezar}
\begin{itemize}
\item {Grp. gram.:v. t.}
\end{itemize}
Fazer quadrados em, como os do tabuleiro de xadrez: \textunderscore enxadrezar um pavimento\textunderscore .
Dividir em quadrados.
\section{Enxadrista}
\begin{itemize}
\item {Grp. gram.:adj.}
\end{itemize}
\begin{itemize}
\item {Grp. gram.:M.}
\end{itemize}
Relativo ao enxadrez.
Jogador de xadrez. Cf. F. Manuel, \textunderscore Apólogos\textunderscore .
\section{Enxaguadela}
\begin{itemize}
\item {Grp. gram.:f.}
\end{itemize}
O mesmo que \textunderscore enxaguadura\textunderscore .
\section{Enxaguadoiro}
\begin{itemize}
\item {Grp. gram.:m.}
\end{itemize}
\begin{itemize}
\item {Utilização:Prov.}
\end{itemize}
\begin{itemize}
\item {Utilização:alent.}
\end{itemize}
\begin{itemize}
\item {Proveniência:(De \textunderscore enxaguar\textunderscore )}
\end{itemize}
Parte do leito da ribeira, em que cái a água do açude, onde há moínho.
\section{Enxaguadura}
\begin{itemize}
\item {Grp. gram.:f.}
\end{itemize}
Acto ou effeito de enxaguar.
\section{Enxaguão}
\begin{itemize}
\item {Grp. gram.:m.}
\end{itemize}
\begin{itemize}
\item {Utilização:Des.}
\end{itemize}
O mesmo que \textunderscore xaguão\textunderscore , montureira:«\textunderscore ...daquelle pestilento e pernicioso enxaguão...\textunderscore »Filinto, \textunderscore D. Man.\textunderscore , III, 102.
\section{Enxaguar}
\begin{itemize}
\item {Grp. gram.:v. t.}
\end{itemize}
Lavar ligeiramente.
Passar por água.
Lavar repetidas vezes.
(Por \textunderscore exaguar\textunderscore , de \textunderscore ex...\textunderscore  + \textunderscore aguar\textunderscore )
\section{Enxaimel}
\begin{itemize}
\item {Grp. gram.:m.}
\end{itemize}
\begin{itemize}
\item {Utilização:Carp.}
\end{itemize}
\begin{itemize}
\item {Utilização:Prov.}
\end{itemize}
Cada uma das tábuas ou tabuões, que entram na formação de uma taipa.
Pau, geralmente de castanho, mais curto e delgado que o caibro.
\section{Enxalamar}
\begin{itemize}
\item {Grp. gram.:m.}
\end{itemize}
\begin{itemize}
\item {Utilização:T. de Buarcos}
\end{itemize}
Rêde especial, que tem na bôca um arco de ferro.
\section{Enxalço}
\begin{itemize}
\item {Grp. gram.:m.}
\end{itemize}
Pequeno arco, sob a vêrga de porta ou janela.
\section{Enxalmador}
\begin{itemize}
\item {Grp. gram.:m.}
\end{itemize}
\begin{itemize}
\item {Utilização:Ant.}
\end{itemize}
\begin{itemize}
\item {Proveniência:(De \textunderscore enxalmar\textunderscore ? Ou por \textunderscore ensalmador\textunderscore , de \textunderscore salmo\textunderscore ?)}
\end{itemize}
Curandeiro, charlatão, que se serve de rezas para curar os doentes.
\section{Enxalmadura}
\begin{itemize}
\item {Grp. gram.:f.}
\end{itemize}
Acto de enxalmar.
Enxalmo. Cf. B. Pereira, \textunderscore Prosódia\textunderscore , vb. \textunderscore scordiscale\textunderscore .
\section{Enxalmar}
\begin{itemize}
\item {Grp. gram.:v. t.}
\end{itemize}
Cobrir com enxalmos.
\section{Enxalmeiro}
\begin{itemize}
\item {Grp. gram.:m.}
\end{itemize}
\begin{itemize}
\item {Proveniência:(De \textunderscore enxalmar\textunderscore )}
\end{itemize}
Aquelle que faz enxalmos.
\section{Enxalmo}
\begin{itemize}
\item {Grp. gram.:m.}
\end{itemize}
\begin{itemize}
\item {Utilização:Pop.}
\end{itemize}
\begin{itemize}
\item {Proveniência:(Do lat. \textunderscore sagma\textunderscore ? Cp. \textunderscore xalmas\textunderscore )}
\end{itemize}
Cobertor, que se põe sôbre a albarda.
Manta.
Tudo que se põe sôbre a albarda para endireitar a carga.
Pessôa inútil, papa-açorda.
Pessôa achacadiça, doente.
\section{Enxama}
\begin{itemize}
\item {Grp. gram.:f.}
\end{itemize}
Cavilha de madeira, na borda da canôa, onde joga o remo.
\section{Enxamata}
\begin{itemize}
\item {Grp. gram.:m.}
\end{itemize}
\begin{itemize}
\item {Utilização:Ant.}
\end{itemize}
\begin{itemize}
\item {Proveniência:(De \textunderscore enxame\textunderscore )}
\end{itemize}
Grande quantidade; enxame.
\section{Enxamblar}
\textunderscore v. t.\textunderscore  (e der.)
(\textunderscore Prov.\textunderscore  V. \textunderscore ensamblar\textunderscore , etc.)
\section{Enxambramento}
\begin{itemize}
\item {Grp. gram.:m.}
\end{itemize}
Acto de enxambrar.
\section{Enxambrar}
\begin{itemize}
\item {Grp. gram.:v. t.}
\end{itemize}
Enxugar ligeiramente, de ordinário á sombra.
Enxugar o bastante para engomar (a roupa).
\section{Enxame}
\begin{itemize}
\item {Grp. gram.:m.}
\end{itemize}
\begin{itemize}
\item {Utilização:Fig.}
\end{itemize}
\begin{itemize}
\item {Proveniência:(Do lat. \textunderscore examen\textunderscore )}
\end{itemize}
Conjunto das abelhas de um cortiço.
Conjunto de abelhas novas que, saindo de um cortiço, vão estabelecer-se em outro.
Multidão.
\section{Enxameal}
\begin{itemize}
\item {Grp. gram.:m.}
\end{itemize}
\begin{itemize}
\item {Utilização:P. us.}
\end{itemize}
O mesmo que \textunderscore colmeal\textunderscore .
\section{Enxamear}
\begin{itemize}
\item {Grp. gram.:v. t.}
\end{itemize}
\begin{itemize}
\item {Grp. gram.:V. i.}
\end{itemize}
\begin{itemize}
\item {Utilização:Fig.}
\end{itemize}
Reunir em cortiço (abelhas).
Formar enxame.
Apparecer em grande número: \textunderscore as asneiras enxameiam nesta obrinha do Anastácio\textunderscore .
Apinhar-se.
\section{Enxamel}
\begin{itemize}
\item {Grp. gram.:m.}
\end{itemize}
(V.enxaimel)
\section{Enxaqueca}
\begin{itemize}
\item {fónica:quê}
\end{itemize}
\begin{itemize}
\item {Grp. gram.:f.}
\end{itemize}
\begin{itemize}
\item {Proveniência:(Do cast. \textunderscore jaqueca\textunderscore )}
\end{itemize}
Dôr, em parte da cabeça; hemicrânia.
\section{Enxaquetado}
\begin{itemize}
\item {Grp. gram.:adj.}
\end{itemize}
Enxadrezado, (falando-se de brasões).
\section{Enxara}
\begin{itemize}
\item {Grp. gram.:f.}
\end{itemize}
\begin{itemize}
\item {Utilização:Ant.}
\end{itemize}
\begin{itemize}
\item {Proveniência:(Do ár. \textunderscore ex-xara\textunderscore )}
\end{itemize}
Charneca.
\section{Enxaral}
\begin{itemize}
\item {Grp. gram.:m.}
\end{itemize}
\begin{itemize}
\item {Utilização:Prov.}
\end{itemize}
\begin{itemize}
\item {Utilização:beir.}
\end{itemize}
O mesmo que \textunderscore enxara\textunderscore .
\section{Enxaraval}
\begin{itemize}
\item {Grp. gram.:m.}
\end{itemize}
\begin{itemize}
\item {Utilização:Ant.}
\end{itemize}
Véu da cabeça.
(Cp. \textunderscore enxaravia\textunderscore ^1)
\section{Enxaravia}
\begin{itemize}
\item {Grp. gram.:f.}
\end{itemize}
\begin{itemize}
\item {Utilização:Ant.}
\end{itemize}
\begin{itemize}
\item {Proveniência:(Do ár. \textunderscore al-xarbiie\textunderscore )}
\end{itemize}
Toucado de mulheres, principalmente de meretrizes e alcoviteiras.
\section{Enxaravia}
\begin{itemize}
\item {Grp. gram.:f.}
\end{itemize}
\begin{itemize}
\item {Utilização:Ant.}
\end{itemize}
Espécie de tamanco.
\section{Enxarca}
\begin{itemize}
\item {Grp. gram.:f.}
\end{itemize}
O mesmo que \textunderscore enxerca\textunderscore . Cf. Filinto, VIII, 152.
\section{Enxárcia}
\begin{itemize}
\item {Grp. gram.:f.}
\end{itemize}
\begin{itemize}
\item {Proveniência:(Do b. lat. \textunderscore sartia\textunderscore . Cp. \textunderscore sarta\textunderscore )}
\end{itemize}
Cordoalha de navio.
Cabos, que ligam os mastros e os mastaréus ás mesas de guarnição, fixando-se nas bigotas.
\section{Enxarciar}
\begin{itemize}
\item {Grp. gram.:v. t.}
\end{itemize}
Guarnecer de enxárcias.
Apparelhar (navios).
\section{Enxario}
\begin{itemize}
\item {Grp. gram.:m.}
\end{itemize}
\begin{itemize}
\item {Utilização:Prov.}
\end{itemize}
\begin{itemize}
\item {Utilização:alg.}
\end{itemize}
Certa qualidade de figo.
(Cp. \textunderscore assario\textunderscore )
\section{Enxarondo}
\begin{itemize}
\item {Grp. gram.:adj.}
\end{itemize}
\begin{itemize}
\item {Utilização:Des.}
\end{itemize}
Insosso; insípido.
\section{Enxaropar}
\begin{itemize}
\item {Grp. gram.:v. i.}
\end{itemize}
Dar xaropes a.
Mezinhar.
Tornar muito doce:«\textunderscore enxaropar o chá\textunderscore ». Filinto, XI, 195.
\section{Enxarope}
\begin{itemize}
\item {Grp. gram.:m.}
\end{itemize}
\begin{itemize}
\item {Utilização:Ant.}
\end{itemize}
O mesmo que \textunderscore xarope\textunderscore . Cf. \textunderscore Eufrosina\textunderscore , 353.
\section{Enxarroco}
\begin{itemize}
\item {Grp. gram.:m.}
\end{itemize}
(V.xarroco)
\section{Enxaugado}
\begin{itemize}
\item {Grp. gram.:adj.}
\end{itemize}
\begin{itemize}
\item {Utilização:Prov.}
\end{itemize}
\begin{itemize}
\item {Utilização:trasm.}
\end{itemize}
O mesmo que \textunderscore engegado\textunderscore .
\section{Enxaugar}
\textunderscore v. t.\textunderscore  (e der.)
(Corr. de \textunderscore enxaguar\textunderscore , etc.)
\section{Enxaugo}
\begin{itemize}
\item {Grp. gram.:m.}
\end{itemize}
\begin{itemize}
\item {Utilização:Prov.}
\end{itemize}
\begin{itemize}
\item {Utilização:trasm.}
\end{itemize}
Pessôa reles, muito ordinária.
(Cp. \textunderscore enxaugado\textunderscore , e \textunderscore enxalmo\textunderscore )
\section{Enxávega}
\begin{itemize}
\item {Grp. gram.:f.}
\end{itemize}
\begin{itemize}
\item {Utilização:Ant.}
\end{itemize}
Pesca de peixe miúdo, que se fazia com enxávegos.
\section{Enxávegos}
\begin{itemize}
\item {Grp. gram.:m. pl.}
\end{itemize}
\begin{itemize}
\item {Utilização:Ant.}
\end{itemize}
\begin{itemize}
\item {Proveniência:(Do ár. \textunderscore xabeca\textunderscore . Cp. \textunderscore chávega\textunderscore )}
\end{itemize}
Espécie de redes, para pescar peixe miúdo.
\section{Enxebre}
\begin{itemize}
\item {fónica:xê}
\end{itemize}
\begin{itemize}
\item {Grp. gram.:adj.}
\end{itemize}
\begin{itemize}
\item {Utilização:Ant.}
\end{itemize}
\begin{itemize}
\item {Grp. gram.:M.  e  adj.}
\end{itemize}
\begin{itemize}
\item {Utilização:Prov.}
\end{itemize}
\begin{itemize}
\item {Utilização:minh.}
\end{itemize}
Insípido.
Bruto, estúpido.
\section{Enxecar}
\begin{itemize}
\item {Grp. gram.:v. t.}
\end{itemize}
\begin{itemize}
\item {Utilização:Ant.}
\end{itemize}
Causar enxeco a.
Vexar.
Multar.
\section{Enxeco}
\begin{itemize}
\item {fónica:xê}
\end{itemize}
\begin{itemize}
\item {Grp. gram.:m.}
\end{itemize}
\begin{itemize}
\item {Utilização:ant.}
\end{itemize}
\begin{itemize}
\item {Utilização:Pop.}
\end{itemize}
\begin{itemize}
\item {Proveniência:(Do cast. \textunderscore enjeco\textunderscore )}
\end{itemize}
Damno; empecilho.
Multa.
Embirração, quisília.
\section{Enxelharia}
\begin{itemize}
\item {Grp. gram.:f.}
\end{itemize}
(V.enxilharia)
\section{Enxequetado}
\begin{itemize}
\item {Grp. gram.:adj.}
\end{itemize}
(V.enxaquetado)
\section{Enxemprar}
\begin{itemize}
\item {Grp. gram.:v. t.}
\end{itemize}
\begin{itemize}
\item {Utilização:Ant.}
\end{itemize}
Encher? Escandalizar?:«\textunderscore ...andastes dizendo que eu era vosso marido... e enxemprastes o reino todo\textunderscore ». Fern. Lopes, \textunderscore Chrón. de D. Fern.\textunderscore , II, 103.
(Por \textunderscore exemplar\textunderscore , do hyp. \textunderscore eximplare\textunderscore , do lat. \textunderscore ex\textunderscore  + \textunderscore implere\textunderscore ?)
\section{Enxerca}
\begin{itemize}
\item {fónica:xêr}
\end{itemize}
\begin{itemize}
\item {Grp. gram.:f.}
\end{itemize}
Acto ou effeito de enxercar.
\section{Enxercar}
\begin{itemize}
\item {Grp. gram.:v. t.}
\end{itemize}
\begin{itemize}
\item {Proveniência:(Do ár. \textunderscore xarraca\textunderscore )}
\end{itemize}
Retalhar e pôr a secar ou a defumar (a carne das reses).
\section{Enxerco}
\begin{itemize}
\item {Grp. gram.:m.}
\end{itemize}
\begin{itemize}
\item {Utilização:Bras. do N}
\end{itemize}
Espécie de erva damninha, também conhecida por \textunderscore erva de passarinho\textunderscore .
\section{Enxêrga}
\begin{itemize}
\item {Grp. gram.:f.}
\end{itemize}
\begin{itemize}
\item {Utilização:Prov.}
\end{itemize}
\begin{itemize}
\item {Utilização:minh.}
\end{itemize}
Colchão pequeno e grosseiro.
Cama pobre.
\textunderscore Gradar de enxêrga\textunderscore , o mesmo que \textunderscore enxergar\textunderscore ^2.
(Cast. \textunderscore jerga\textunderscore )
\section{Enxergadamente}
\begin{itemize}
\item {Grp. gram.:adv.}
\end{itemize}
\begin{itemize}
\item {Proveniência:(De \textunderscore enxergar\textunderscore ^1)}
\end{itemize}
Claramente. Cf. \textunderscore Aulegrafia\textunderscore , 160.
\section{Enxergão}
\begin{itemize}
\item {Grp. gram.:m.}
\end{itemize}
Espécie de saco grande, cheio ordinariamente de palha, e sôbre o qual se costuma estender o colchão.
(Cast. \textunderscore jergón\textunderscore )
\section{Enxergar}
\begin{itemize}
\item {Grp. gram.:v. t.}
\end{itemize}
Entrever.
Divisar; descortinar.
Observar; perceber.
\section{Enxergar}
\begin{itemize}
\item {Grp. gram.:v. t.}
\end{itemize}
\begin{itemize}
\item {Utilização:Prov.}
\end{itemize}
\begin{itemize}
\item {Utilização:minh.}
\end{itemize}
\begin{itemize}
\item {Proveniência:(De \textunderscore enxêrga\textunderscore )}
\end{itemize}
Alisar (terra lavrada) com a grade voltada de costas, isto é, sem se empregarem os dentes da mesma.
\section{Enxerido}
\begin{itemize}
\item {Grp. gram.:adj.}
\end{itemize}
\begin{itemize}
\item {Utilização:Bras. do N}
\end{itemize}
\begin{itemize}
\item {Proveniência:(De \textunderscore enxerir\textunderscore )}
\end{itemize}
Intrometido.
\section{Enxerir}
\begin{itemize}
\item {Grp. gram.:v. t.}
\end{itemize}
O mesmo que \textunderscore inserir\textunderscore . Cf. \textunderscore Eufrosina\textunderscore , 181; Camillo, \textunderscore Noites de Insómn.\textunderscore , X, 45.
\section{Enxermada}
\begin{itemize}
\item {Grp. gram.:adj. f.}
\end{itemize}
\begin{itemize}
\item {Utilização:Prov.}
\end{itemize}
\begin{itemize}
\item {Utilização:beir.}
\end{itemize}
Diz-se da rês, que, por desastre ou por vontade do dono, perdeu o filho e continúa a dar leite.
\section{Enxerqueira}
Fem. de \textunderscore enxerqueiro\textunderscore .
\section{Enxerqueiro}
\begin{itemize}
\item {Grp. gram.:m.}
\end{itemize}
\begin{itemize}
\item {Utilização:Ant.}
\end{itemize}
\begin{itemize}
\item {Proveniência:(De \textunderscore enxercar\textunderscore )}
\end{itemize}
Aquelle, que enxerca carne ou que vende carne enxercada.
\section{Enxerta}
\begin{itemize}
\item {Grp. gram.:f.}
\end{itemize}
Acto de enxertar.
Variedade de azeitona, também conhecida por \textunderscore passareira\textunderscore  e \textunderscore zambulha\textunderscore .
\section{Enxertadeira}
\begin{itemize}
\item {Grp. gram.:f.}
\end{itemize}
\begin{itemize}
\item {Proveniência:(De \textunderscore enxertar\textunderscore )}
\end{itemize}
Faca, própria para fazer enxertos.
\section{Enxertado}
\begin{itemize}
\item {Grp. gram.:adj.}
\end{itemize}
\begin{itemize}
\item {Utilização:Prov.}
\end{itemize}
\begin{itemize}
\item {Utilização:minh.}
\end{itemize}
\begin{itemize}
\item {Proveniência:(De \textunderscore enxertar\textunderscore )}
\end{itemize}
Diz-se do indivíduo que foi vacinado.
\section{Enxertador}
\begin{itemize}
\item {Grp. gram.:adj.}
\end{itemize}
\begin{itemize}
\item {Grp. gram.:M.}
\end{itemize}
Que enxerta.
Aquelle que enxerta.
Instrumento para enxertar.
\section{Enxertadura}
\begin{itemize}
\item {Grp. gram.:f.}
\end{itemize}
Acto ou effeito de enxertar.
\section{Enxertar}
\begin{itemize}
\item {Grp. gram.:v. t.}
\end{itemize}
\begin{itemize}
\item {Utilização:Prov.}
\end{itemize}
\begin{itemize}
\item {Proveniência:(Do lat. \textunderscore insertare\textunderscore )}
\end{itemize}
Fazer enxêrto em.
Inserir; incluir.
O mesmo que \textunderscore vacinar\textunderscore .
\section{Enxertário}
\begin{itemize}
\item {Grp. gram.:m.}
\end{itemize}
\begin{itemize}
\item {Proveniência:(De \textunderscore enxertar\textunderscore )}
\end{itemize}
Reunião de cabos de navio, que atracam as vêrgas aos mastaréus.
Argola de corda, com fôrro de coiro ou sola.
\section{Enxertas}
\begin{itemize}
\item {Grp. gram.:adj. f. pl.}
\end{itemize}
\begin{itemize}
\item {Utilização:Prov.}
\end{itemize}
\begin{itemize}
\item {Utilização:trasm.}
\end{itemize}
Diz-se das castanhas, quando longaes ou compridas.
\section{Enxertia}
\begin{itemize}
\item {Grp. gram.:f.}
\end{itemize}
O mesmo que \textunderscore enxertadura\textunderscore .
\section{Enxêrto}
\begin{itemize}
\item {Grp. gram.:m.}
\end{itemize}
\begin{itemize}
\item {Proveniência:(De \textunderscore enxertar\textunderscore )}
\end{itemize}
Operação, com que se introduz uma parte viva de um vegetal no tronco ou ramo de outro vegetal, para neste se desenvolver, como se estivesse na sua haste natural.
\section{Enxertómetro}
\begin{itemize}
\item {Grp. gram.:m.}
\end{itemize}
\begin{itemize}
\item {Proveniência:(De \textunderscore enxêrto\textunderscore  + gr. \textunderscore metron\textunderscore )}
\end{itemize}
Instrumento, para medir o dorso dos garfos de enxertia.
\section{Enxiar}
\begin{itemize}
\item {Grp. gram.:v. t.}
\end{itemize}
Ligar á amarra.
\section{Ênxido}
\begin{itemize}
\item {Grp. gram.:m.}
\end{itemize}
Pequeno vinhedo ou pomar.
Quinta. (Ouvido em Murça)
Outra fórma de \textunderscore êixido\textunderscore .
\section{Enxilhar}
\begin{itemize}
\item {Grp. gram.:v. t.}
\end{itemize}
\begin{itemize}
\item {Grp. gram.:M.}
\end{itemize}
Ajustar, assentar igualmente (pedras de cantaria).
Pedra apparelhada, que occupa grande espaço numa parede.
(Por \textunderscore ensilhar\textunderscore , de \textunderscore silhar\textunderscore )
\section{Enxilharia}
\begin{itemize}
\item {Grp. gram.:f.}
\end{itemize}
O mesmo que \textunderscore contra-arcada\textunderscore .
O mesmo que \textunderscore silharia\textunderscore .
\section{Enxó}
\begin{itemize}
\item {Grp. gram.:m.}
\end{itemize}
\begin{itemize}
\item {Proveniência:(Do lat. \textunderscore asciola\textunderscore )}
\end{itemize}
Instrumento de carpinteiro ou tanoeiro, de cabo curto e curvo, com chapa cortante, e que serve para desbastar madeira.
\section{Enxó}
\begin{itemize}
\item {Grp. gram.:f.}
\end{itemize}
\begin{itemize}
\item {Utilização:Prov.}
\end{itemize}
\begin{itemize}
\item {Utilização:alent.}
\end{itemize}
O mesmo que \textunderscore enxós\textunderscore .
\section{Enxôfar}
\begin{itemize}
\item {Grp. gram.:m.}
\end{itemize}
\begin{itemize}
\item {Utilização:Pop.}
\end{itemize}
O mesmo que \textunderscore enxôfre\textunderscore .
\section{Enxofra}
\begin{itemize}
\item {Grp. gram.:f.}
\end{itemize}
\begin{itemize}
\item {Utilização:Prov.}
\end{itemize}
\begin{itemize}
\item {Utilização:dur.}
\end{itemize}
O mesmo que \textunderscore enxofração\textunderscore .
\section{Enxofração}
\begin{itemize}
\item {Grp. gram.:f.}
\end{itemize}
\begin{itemize}
\item {Utilização:Prov.}
\end{itemize}
\begin{itemize}
\item {Utilização:dur.}
\end{itemize}
O mesmo que \textunderscore enxoframento\textunderscore .
\section{Enxofradeira}
\begin{itemize}
\item {Grp. gram.:f.}
\end{itemize}
Instrumento, para enxofrar.
O mesmo que \textunderscore enxofrador\textunderscore .
\section{Enxofrado}
\begin{itemize}
\item {Grp. gram.:adj.}
\end{itemize}
\begin{itemize}
\item {Proveniência:(De \textunderscore enxofrar\textunderscore )}
\end{itemize}
Arreliado.
Zangado.
\section{Enxofrador}
\begin{itemize}
\item {Grp. gram.:m.}
\end{itemize}
\begin{itemize}
\item {Grp. gram.:Adj.}
\end{itemize}
Instrumento, para enxofrar vinhas.
Aquelle que enxofra.
Que enxofra.
\section{Enxoframento}
\begin{itemize}
\item {Grp. gram.:m.}
\end{itemize}
Acto ou effeito de enxofrar.
\section{Enxofrante}
\begin{itemize}
\item {Grp. gram.:adj.}
\end{itemize}
Que enxofra.
\section{Enxofrar}
\begin{itemize}
\item {Grp. gram.:v. t.}
\end{itemize}
\begin{itemize}
\item {Utilização:Pop.}
\end{itemize}
\begin{itemize}
\item {Grp. gram.:V. p.}
\end{itemize}
\begin{itemize}
\item {Utilização:pop.}
\end{itemize}
\begin{itemize}
\item {Utilização:Fig.}
\end{itemize}
Impregnar ou cobrir de enxôfre.
Misturar com enxôfre.
Desinfectar com enxôfre.
Arreliar; encolerizar.
Amuar-se; agastar-se.
\section{Enxôfre}
\begin{itemize}
\item {Grp. gram.:m.}
\end{itemize}
\begin{itemize}
\item {Proveniência:(Do ár. \textunderscore axxofre\textunderscore ?)}
\end{itemize}
Corpo simples, sólido, amarelado e combustível.
\section{Enxofreira}
\begin{itemize}
\item {Grp. gram.:f.}
\end{itemize}
Vulcão, que expelle gases, impregnados de enxôfre.
\section{Enxofrento}
\begin{itemize}
\item {Grp. gram.:adj.}
\end{itemize}
Que contém enxôfre.
\section{Enxogalhar}
\begin{itemize}
\item {fónica:xó}
\end{itemize}
\begin{itemize}
\item {Grp. gram.:v. t.}
\end{itemize}
\begin{itemize}
\item {Utilização:T. da Bairrada}
\end{itemize}
O mesmo que \textunderscore enxaguar\textunderscore .
\section{Enxogar}
\begin{itemize}
\item {fónica:xó}
\end{itemize}
\textunderscore v. t.\textunderscore  (e der)
(Corr. alent. de \textunderscore enxaguar\textunderscore , etc.)
\section{Enxorar}
\begin{itemize}
\item {Grp. gram.:v. t.}
\end{itemize}
O mesmo que \textunderscore axorar\textunderscore .
\section{Enxós}
\begin{itemize}
\item {Grp. gram.:f.}
\end{itemize}
\begin{itemize}
\item {Utilização:Prov.}
\end{itemize}
\begin{itemize}
\item {Utilização:Beira-Baixa}
\end{itemize}
Armadilha para perdizes.
\section{Enxota-cães}
\begin{itemize}
\item {Grp. gram.:f.}
\end{itemize}
\begin{itemize}
\item {Utilização:Fam.}
\end{itemize}
Aquelle que, nas igrejas, enxota os cães para a rua.
\section{Enxota-diabos}
\begin{itemize}
\item {Grp. gram.:m.}
\end{itemize}
\begin{itemize}
\item {Utilização:Pop.}
\end{itemize}
Aquelle que finge de exorcísta.
\section{Enxotador}
\begin{itemize}
\item {Grp. gram.:m.  e  adj.}
\end{itemize}
O que enxota.
\section{Enxotadura}
\begin{itemize}
\item {Grp. gram.:f.}
\end{itemize}
Acto ou effeito de \textunderscore enxotar\textunderscore .
\section{Enxota-môscas}
\begin{itemize}
\item {Grp. gram.:m.}
\end{itemize}
Pau que, tendo numa das pontas uma porção de papel cortado em tiras, se agita para afugentar as môscas.
\section{Enxotar}
\begin{itemize}
\item {Grp. gram.:v. t.}
\end{itemize}
Expulsar.
Afugentar.
Fazer \textunderscore xote\textunderscore  ás aves.
\section{Enxova}
\begin{itemize}
\item {Grp. gram.:f.}
\end{itemize}
\begin{itemize}
\item {Utilização:Ant.}
\end{itemize}
O mesmo que \textunderscore enxovia\textunderscore .
\section{Enxoval}
\begin{itemize}
\item {Grp. gram.:m.}
\end{itemize}
\begin{itemize}
\item {Proveniência:(Do ár. \textunderscore a\textunderscore  + \textunderscore xuar\textunderscore )}
\end{itemize}
Conjunto de roupas e adornos de uma noiva ou de um recém-nascido.
Alfaias, roupas.
\section{Enxoval}
\begin{itemize}
\item {Grp. gram.:m.}
\end{itemize}
O mesmo que \textunderscore enxovedo\textunderscore . Cf. Simão Machado, f. 86.
\section{Enxovalhadamente}
\begin{itemize}
\item {Grp. gram.:adv.}
\end{itemize}
\begin{itemize}
\item {Proveniência:(De \textunderscore enxovalhado\textunderscore )}
\end{itemize}
Sem limpeza.
\section{Enxovalhado}
\begin{itemize}
\item {Grp. gram.:m.}
\end{itemize}
\begin{itemize}
\item {Utilização:Fig.}
\end{itemize}
\begin{itemize}
\item {Proveniência:(De \textunderscore enxovalhar\textunderscore )}
\end{itemize}
Manchado; sujo.
Desacreditado; injuriado.
\section{Enxovalhamento}
\begin{itemize}
\item {Grp. gram.:m.}
\end{itemize}
Acto de enxovalhar.
\section{Enxovalhar}
\begin{itemize}
\item {Grp. gram.:v. t.}
\end{itemize}
\begin{itemize}
\item {Proveniência:(Do cast. \textunderscore sovajar\textunderscore )}
\end{itemize}
Sujar, manchar, ennodoar.
Macular, deslustrar.
Injuriar.
\section{Enxovalho}
\begin{itemize}
\item {Grp. gram.:m.}
\end{itemize}
Acto ou effeito de enxovalhar.
\section{Enxovar}
\begin{itemize}
\item {Grp. gram.:v. t.}
\end{itemize}
\begin{itemize}
\item {Utilização:Ant.}
\end{itemize}
\begin{itemize}
\item {Proveniência:(De \textunderscore enxova\textunderscore )}
\end{itemize}
Meter em enxovia.
Encarcerar.
Prender.
\section{Enxovedo}
\begin{itemize}
\item {fónica:vê}
\end{itemize}
\begin{itemize}
\item {Grp. gram.:m.}
\end{itemize}
\begin{itemize}
\item {Utilização:Fam.}
\end{itemize}
Pacóvio, pateta. Cf. Garrett, \textunderscore Port. na Balança\textunderscore , 270.
\section{Enxovia}
\begin{itemize}
\item {Grp. gram.:f.}
\end{itemize}
Cárcere térreo ou subterrâneo, escuro e húmido.
\section{Enxu}
\begin{itemize}
\item {Grp. gram.:m.}
\end{itemize}
Casta de maribondo.
\section{Enxudreiro}
\begin{itemize}
\item {Grp. gram.:m.}
\end{itemize}
\begin{itemize}
\item {Utilização:Pop.}
\end{itemize}
Lamaçal; porcaria.
(Metáth. de \textunderscore enxurdeiro\textunderscore )
\section{Enxuga}
\begin{itemize}
\item {Grp. gram.:f.}
\end{itemize}
O mesmo que \textunderscore enxugo\textunderscore .
\section{Enxugadoiro}
\begin{itemize}
\item {Grp. gram.:m.}
\end{itemize}
\begin{itemize}
\item {Proveniência:(De \textunderscore enxugar\textunderscore )}
\end{itemize}
Lugar, onde se enxugam roupas ou cera ou outros objectos.
\section{Enxugador}
\begin{itemize}
\item {Grp. gram.:m.}
\end{itemize}
Aquelle que enxuga.
Apparelho ou estufa para enxugar roupa.
\section{Enxugadouro}
\begin{itemize}
\item {Grp. gram.:m.}
\end{itemize}
\begin{itemize}
\item {Proveniência:(De \textunderscore enxugar\textunderscore )}
\end{itemize}
Lugar, onde se enxugam roupas ou cera ou outros objectos.
\section{Enxugar}
\begin{itemize}
\item {Grp. gram.:v. t.}
\end{itemize}
\begin{itemize}
\item {Grp. gram.:V. i.}
\end{itemize}
\begin{itemize}
\item {Proveniência:(Do lat. \textunderscore exsicare\textunderscore )}
\end{itemize}
Secar a humidade de: \textunderscore enxugar o fato\textunderscore .
Despejar, bebendo: \textunderscore enxugar uma garrafa\textunderscore .
Perder a humidade, secar.
\section{Enxugo}
\begin{itemize}
\item {Grp. gram.:m.}
\end{itemize}
\begin{itemize}
\item {Utilização:T. de Alcanena}
\end{itemize}
Acto ou effeito de enxugar.
Casa, onde se põem a enxugar pelles ou coiros.
\section{Enxui}
\begin{itemize}
\item {Grp. gram.:m.}
\end{itemize}
\begin{itemize}
\item {Utilização:Bras}
\end{itemize}
O mesmo que \textunderscore enxu\textunderscore .
\section{Enxulha}
\begin{itemize}
\item {Grp. gram.:f.}
\end{itemize}
\begin{itemize}
\item {Utilização:des.}
\end{itemize}
\begin{itemize}
\item {Utilização:Pop.}
\end{itemize}
Banhas, que as aves criam depois da muda.
(Corr. de \textunderscore enxúndia\textunderscore )
\section{Enxulho}
\begin{itemize}
\item {Grp. gram.:m.}
\end{itemize}
(V.enxulha)
\section{Enxumbrar}
\textunderscore v. t.\textunderscore  (e der)
O mesmo ou melhor que \textunderscore enxambrar\textunderscore .
\section{Enxúndia}
\begin{itemize}
\item {Grp. gram.:f.}
\end{itemize}
\begin{itemize}
\item {Proveniência:(Do lat. \textunderscore axungia\textunderscore )}
\end{itemize}
Banha ou gordura das aves.
Banha de porco, unto.
\section{Enxudiar}
\begin{itemize}
\item {Grp. gram.:v. t.}
\end{itemize}
\begin{itemize}
\item {Proveniência:(De \textunderscore enxúndia\textunderscore )}
\end{itemize}
Alimentar, engordar. Cf. Camillo, \textunderscore Enjeitada\textunderscore , 28.
\section{Enxundiosamente}
\begin{itemize}
\item {Grp. gram.:adv.}
\end{itemize}
De modo enxundioso:«\textunderscore arredondando-se tão enxundiosamente...\textunderscore »Camillo, \textunderscore Novel. do Minh.\textunderscore , II, 75.
\section{Enxundioso}
\begin{itemize}
\item {Grp. gram.:adj.}
\end{itemize}
Que tem enxúndia.
Gorduroso.
Untuoso. Cf. Camillo, \textunderscore Noites de Insómn.\textunderscore , IX, 98.
\section{Enxurdar-se}
\begin{itemize}
\item {Grp. gram.:v. p.}
\end{itemize}
\begin{itemize}
\item {Proveniência:(De \textunderscore xurdo\textunderscore , por \textunderscore churdo\textunderscore )}
\end{itemize}
Revolver-se na lama.
Enlodar-se.
\section{Enxurdeiro}
\begin{itemize}
\item {Grp. gram.:m.}
\end{itemize}
\begin{itemize}
\item {Proveniência:(De \textunderscore enxurdar\textunderscore )}
\end{itemize}
Atoleiro; lamaçal.
\section{Enxurrada}
\begin{itemize}
\item {Grp. gram.:f.}
\end{itemize}
\begin{itemize}
\item {Proveniência:(De \textunderscore enxurrar\textunderscore )}
\end{itemize}
O mesmo que \textunderscore enxurro\textunderscore .
\section{Enxurrar}
\begin{itemize}
\item {Grp. gram.:v. t.}
\end{itemize}
\begin{itemize}
\item {Grp. gram.:V. i.}
\end{itemize}
Cobrir de enxurro.
Alagar com enxurro.
Produzir enxurro.
\section{Enxurreira}
\begin{itemize}
\item {Grp. gram.:f.}
\end{itemize}
O mesmo que \textunderscore enxurreiro\textunderscore .
\section{Enxurreiro}
\begin{itemize}
\item {Grp. gram.:m.}
\end{itemize}
\begin{itemize}
\item {Proveniência:(De \textunderscore enxurro\textunderscore )}
\end{itemize}
Lugar, em que passou enxurro; lamaçal.
\section{Enxurro}
\begin{itemize}
\item {Grp. gram.:m.}
\end{itemize}
\begin{itemize}
\item {Utilização:Fig.}
\end{itemize}
Corrente impetuosa de águas fluviaes.
Corrente ou jôrro de ímmundícies.
Escória; ralé.
(Corr. de \textunderscore en...\textunderscore  + \textunderscore jôrro\textunderscore )
\section{Enxuto}
\begin{itemize}
\item {Grp. gram.:adj.}
\end{itemize}
\begin{itemize}
\item {Proveniência:(Do lat. \textunderscore exsuctus\textunderscore )}
\end{itemize}
Que esteve molhado ou húmido e depois se enxugou.
Que tem pouco môlho ou caldo: \textunderscore sopa enxuta\textunderscore .
Limpo de lágrimas: \textunderscore olhos enxutos\textunderscore .
Diz-se do tempo, em que não há chuva.
\section{Enzamboado}
\begin{itemize}
\item {Grp. gram.:adj.}
\end{itemize}
\begin{itemize}
\item {Utilização:Prov.}
\end{itemize}
\begin{itemize}
\item {Utilização:alg.}
\end{itemize}
O mesmo que \textunderscore azamboado\textunderscore .
\section{Enzampa}
\begin{itemize}
\item {Grp. gram.:m.}
\end{itemize}
Indivíduo, que se torna empecilho.
Maçador. Cf. Roquete, \textunderscore Cód. do Bom Tom\textunderscore .
\section{Enzampar}
\begin{itemize}
\item {Grp. gram.:v. t.}
\end{itemize}
\begin{itemize}
\item {Utilização:Pop.}
\end{itemize}
\begin{itemize}
\item {Utilização:Prov.}
\end{itemize}
\begin{itemize}
\item {Utilização:alent.}
\end{itemize}
\begin{itemize}
\item {Proveniência:(De \textunderscore zampar\textunderscore )}
\end{itemize}
Empachar.
Embaçar.
Burlar, lograr.
Causar assombro a; assustar.
\section{Enzarel}
\begin{itemize}
\item {Grp. gram.:m.}
\end{itemize}
\begin{itemize}
\item {Utilização:Prov.}
\end{itemize}
\begin{itemize}
\item {Utilização:trasm.}
\end{itemize}
Pessôa fraca e franzina.
\section{Enzêmula}
\begin{itemize}
\item {Grp. gram.:f.}
\end{itemize}
\begin{itemize}
\item {Utilização:Prov.}
\end{itemize}
\begin{itemize}
\item {Utilização:alg.}
\end{itemize}
O mesmo que \textunderscore azêmula\textunderscore .
\section{Enzena}
\begin{itemize}
\item {Grp. gram.:f.}
\end{itemize}
\begin{itemize}
\item {Utilização:Ant.}
\end{itemize}
O mesmo que \textunderscore onzena\textunderscore .
\section{Enzima}
\begin{itemize}
\item {Grp. gram.:f.}
\end{itemize}
Fermento solúvel, de diástese e zímase.
\section{Enzinar-se}
\begin{itemize}
\item {Grp. gram.:v. p.}
\end{itemize}
\begin{itemize}
\item {Utilização:Prov.}
\end{itemize}
\begin{itemize}
\item {Utilização:trasm.}
\end{itemize}
Embeber-se, saturar-se.
\section{Enzinha}
\begin{itemize}
\item {Grp. gram.:f.}
\end{itemize}
O mesmo que \textunderscore azinho\textunderscore . Cf. Filinto, IV, 271.
\section{Enzinheira}
\begin{itemize}
\item {Grp. gram.:f.}
\end{itemize}
O mesmo que \textunderscore azinheira\textunderscore . Cf. Filinto, XV, 210.
\section{Enzóe}
\begin{itemize}
\item {Grp. gram.:m.}
\end{itemize}
Ave palmípede.
\section{Enzói}
\begin{itemize}
\item {Grp. gram.:m.}
\end{itemize}
Ave palmípede.
\section{Enzoico}
\begin{itemize}
\item {Grp. gram.:adj.}
\end{itemize}
\begin{itemize}
\item {Utilização:Geol.}
\end{itemize}
\begin{itemize}
\item {Proveniência:(Do gr. \textunderscore zoon\textunderscore )}
\end{itemize}
Diz-se do terreno, que contém animaes fósseis.
\section{Enzona}
\begin{itemize}
\item {Grp. gram.:f.}
\end{itemize}
\begin{itemize}
\item {Utilização:Pop.}
\end{itemize}
\begin{itemize}
\item {Utilização:Ant.}
\end{itemize}
\begin{itemize}
\item {Utilização:Prov.}
\end{itemize}
\begin{itemize}
\item {Utilização:trasm.}
\end{itemize}
\begin{itemize}
\item {Proveniência:(De \textunderscore enzonar\textunderscore )}
\end{itemize}
Intriga.
Mexerico.
Ódio.
Brinquedo de criança; lembrança infantil.
\section{Enzonar}
\begin{itemize}
\item {Grp. gram.:v. t.}
\end{itemize}
Enredar; mexericar.
(Corr. de \textunderscore onzenar\textunderscore )
\section{Enzoneiro}
\begin{itemize}
\item {Grp. gram.:adj.}
\end{itemize}
\begin{itemize}
\item {Utilização:Prov.}
\end{itemize}
\begin{itemize}
\item {Utilização:trasm.}
\end{itemize}
\begin{itemize}
\item {Utilização:Bras. da Baía}
\end{itemize}
O mesmo que \textunderscore onzeneiro\textunderscore .
Mentiroso.
\section{Enzonice}
\begin{itemize}
\item {Grp. gram.:f.}
\end{itemize}
\begin{itemize}
\item {Utilização:Pop.}
\end{itemize}
O mesmo que \textunderscore onzenice\textunderscore .
\section{Enzootia}
\begin{itemize}
\item {Grp. gram.:m.}
\end{itemize}
\begin{itemize}
\item {Proveniência:(Do gr. \textunderscore zoon\textunderscore )}
\end{itemize}
Doença periódica de certos animaes em determinados países.
\section{Enzoótico}
\begin{itemize}
\item {Grp. gram.:adj.}
\end{itemize}
Relativo a enzootia.
\section{Enzyma}
\begin{itemize}
\item {Grp. gram.:f.}
\end{itemize}
Fermento solúvel, de diástese e zýmase.
\section{...eo}
\begin{itemize}
\item {Grp. gram.:suf. adj.}
\end{itemize}
\begin{itemize}
\item {Proveniência:(Lat. \textunderscore ...eus\textunderscore )}
\end{itemize}
(designativo de qualidade, pertença ou relação: \textunderscore plúmbeo\textunderscore ; \textunderscore níveo\textunderscore ; \textunderscore áureo...\textunderscore )
\section{Eocene}
\begin{itemize}
\item {Grp. gram.:adj.}
\end{itemize}
(V.eoceno)
\section{Eocênico}
\begin{itemize}
\item {Grp. gram.:adj.}
\end{itemize}
O mesmo que \textunderscore eoceno\textunderscore .
\section{Eoceno}
\begin{itemize}
\item {Grp. gram.:adj.}
\end{itemize}
\begin{itemize}
\item {Proveniência:(Do gr. \textunderscore eos\textunderscore  + \textunderscore kainos\textunderscore )}
\end{itemize}
Diz-se, em Geologia, do terreno mais antigo entre os de formação recente.
\section{Eões}
\begin{itemize}
\item {Grp. gram.:m. pl.}
\end{itemize}
\begin{itemize}
\item {Proveniência:(Lat. \textunderscore aeones\textunderscore )}
\end{itemize}
Entes, imaginados pelos Gnósticos, para se preencher a distância entre Deus pai e Christo filho, e entre Christo e os homens.
\section{Eolantho}
\begin{itemize}
\item {Grp. gram.:m.}
\end{itemize}
\begin{itemize}
\item {Proveniência:(Do gr. \textunderscore aiolos\textunderscore  + \textunderscore anthos\textunderscore )}
\end{itemize}
Planta africana, da fam. das labiadas.
\section{Eolanto}
\begin{itemize}
\item {Grp. gram.:m.}
\end{itemize}
\begin{itemize}
\item {Proveniência:(Do gr. \textunderscore aiolos\textunderscore  + \textunderscore anthos\textunderscore )}
\end{itemize}
Planta africana, da fam. das labiadas.
\section{Eólico}
\begin{itemize}
\item {Grp. gram.:adj.}
\end{itemize}
\begin{itemize}
\item {Proveniência:(Lat. \textunderscore aeolicus\textunderscore )}
\end{itemize}
Relativo a Eólia.
E diz-se do verso chamado sáphico.
\section{Eólide}
\begin{itemize}
\item {Grp. gram.:f.}
\end{itemize}
O mesmo que \textunderscore eolídia\textunderscore .
\section{Eolídia}
\begin{itemize}
\item {Grp. gram.:f.}
\end{itemize}
\begin{itemize}
\item {Proveniência:(De \textunderscore Éolo\textunderscore , n. p. + gr. \textunderscore eidos\textunderscore )}
\end{itemize}
Espécie de mollusco cephalópode.
\section{Eolina}
\begin{itemize}
\item {Grp. gram.:f.}
\end{itemize}
\begin{itemize}
\item {Proveniência:(De \textunderscore éolo\textunderscore )}
\end{itemize}
Antigo e pequeno órgão de palhetas livres.
\section{Eólio}
\begin{itemize}
\item {Grp. gram.:m.}
\end{itemize}
\begin{itemize}
\item {Grp. gram.:Adj.}
\end{itemize}
\begin{itemize}
\item {Proveniência:(Lat. \textunderscore aeolius\textunderscore )}
\end{itemize}
Dialecto da Eólia, na Grécia.
Relativo ao vento; vibrado pelo vento.
\textunderscore Harpa eólia\textunderscore , harpa de seis cordas.
\section{Eolípila}
\begin{itemize}
\item {Grp. gram.:f.}
\end{itemize}
\begin{itemize}
\item {Proveniência:(Lat. \textunderscore aeolipilae\textunderscore )}
\end{itemize}
Instrumento, formado do uma bola ôca e metállica, que se faz girar por meio do vapor de água, que se aquece dentro della.
Antigo apparelho, para se conhecer a direcção do vento.
Instrumento, com que se activava a tiragem das chaminés.
Espécie de vaso com torcida, que servia de lamparina, e se collocava nas chaminés.
\section{Eólitho}
\begin{itemize}
\item {Grp. gram.:m.}
\end{itemize}
\begin{itemize}
\item {Proveniência:(Do gr. \textunderscore eos\textunderscore , aurora, e \textunderscore lithos\textunderscore , pedra)}
\end{itemize}
Peça de pedra lascada, de talho intencional.
\section{Eólito}
\begin{itemize}
\item {Grp. gram.:m.}
\end{itemize}
\begin{itemize}
\item {Proveniência:(Do gr. \textunderscore eos\textunderscore , aurora, e \textunderscore lithos\textunderscore , pedra)}
\end{itemize}
Peça de pedra lascada, de talho intencional.
\section{Efebato}
\begin{itemize}
\item {Grp. gram.:m.}
\end{itemize}
\begin{itemize}
\item {Utilização:Des.}
\end{itemize}
\begin{itemize}
\item {Proveniência:(Lat. \textunderscore ephebalus\textunderscore )}
\end{itemize}
Aquele que chegou á puberdade.
\section{Efebo}
\begin{itemize}
\item {Grp. gram.:m.}
\end{itemize}
\begin{itemize}
\item {Proveniência:(Lat. \textunderscore ephebus\textunderscore )}
\end{itemize}
Aquele que chegou á puberdade.
Homem moço.
\section{Éfedra}
\begin{itemize}
\item {Grp. gram.:f.}
\end{itemize}
\begin{itemize}
\item {Proveniência:(Lat. \textunderscore ephedra\textunderscore )}
\end{itemize}
Árvore conífera, de fruto vermelho e azedo, (\textunderscore ephedra distachya\textunderscore , Lin.).
\section{Efélide}
\begin{itemize}
\item {Grp. gram.:f.}
\end{itemize}
\begin{itemize}
\item {Proveniência:(Lat. \textunderscore ephelis\textunderscore )}
\end{itemize}
Mancha na pele.
\section{Efémeras}
\begin{itemize}
\item {Grp. gram.:f.}
\end{itemize}
\begin{itemize}
\item {Proveniência:(De \textunderscore efêmero\textunderscore )}
\end{itemize}
Insectos neurópteros, que nascem e morrem no mesmo dia.
\section{Efemeridade}
\begin{itemize}
\item {Grp. gram.:f.}
\end{itemize}
Qualidade daquilo que é efêmero.
\section{Efemérides}
\begin{itemize}
\item {Grp. gram.:f. pl.}
\end{itemize}
\begin{itemize}
\item {Proveniência:(Do lat. \textunderscore ephemeris\textunderscore )}
\end{itemize}
Tábuas astronómicas, que indicam, dia a dia, a posição dos planetas no Zodíaco.
Diário.
Relação dos factos de cada dia, sucedidos em diferentes anos e lugares.
\section{Efemerina}
\begin{itemize}
\item {Grp. gram.:f.}
\end{itemize}
\begin{itemize}
\item {Proveniência:(De \textunderscore efêmero\textunderscore )}
\end{itemize}
Espécie de junco da América e da Índia.
\section{Efemerizar}
\begin{itemize}
\item {Grp. gram.:v. t.}
\end{itemize}
\begin{itemize}
\item {Utilização:Neol.}
\end{itemize}
Tornar efêmero.
Narrar, dia a dia, a história de.
\section{Efêmero}
\begin{itemize}
\item {Grp. gram.:adj.}
\end{itemize}
\begin{itemize}
\item {Grp. gram.:M.}
\end{itemize}
\begin{itemize}
\item {Proveniência:(Gr. \textunderscore ephemeros\textunderscore )}
\end{itemize}
Que dura um só dia.
Passageiro, transitório.
O mesmo que \textunderscore efemerina\textunderscore .
\section{Efesino}
\begin{itemize}
\item {Grp. gram.:adj.}
\end{itemize}
\begin{itemize}
\item {Proveniência:(Lat. \textunderscore ephesinus\textunderscore )}
\end{itemize}
Relativo a Éfeso.
\section{Efésios}
\begin{itemize}
\item {Grp. gram.:m. pl.}
\end{itemize}
\begin{itemize}
\item {Proveniência:(Lat. \textunderscore ephasius\textunderscore )}
\end{itemize}
Habitantes de Éfeso.
\section{Éfeta}
\begin{itemize}
\item {Grp. gram.:m.}
\end{itemize}
\begin{itemize}
\item {Proveniência:(Gr. \textunderscore ephetes\textunderscore )}
\end{itemize}
Magistrado criminal em Atenas.
\section{Efialta}
\begin{itemize}
\item {Grp. gram.:f.}
\end{itemize}
\begin{itemize}
\item {Utilização:Des.}
\end{itemize}
\begin{itemize}
\item {Proveniência:(Do lat. \textunderscore ephialtes\textunderscore )}
\end{itemize}
Pesadelo.
\section{Efialtes}
\begin{itemize}
\item {Grp. gram.:m.}
\end{itemize}
O mesmo que \textunderscore efialta\textunderscore .
\section{Efidrose}
\begin{itemize}
\item {Grp. gram.:f.}
\end{itemize}
\begin{itemize}
\item {Proveniência:(Do gr. \textunderscore epi\textunderscore  + \textunderscore hudor\textunderscore )}
\end{itemize}
Doença, caracterizada por suor na parte superior do corpo.
Suor, na crise de algumas doenças.
\section{Efípia}
\begin{itemize}
\item {Grp. gram.:f.}
\end{itemize}
\begin{itemize}
\item {Proveniência:(Lat. \textunderscore ephippium\textunderscore )}
\end{itemize}
Sela de lan, imitada da cavalaria romana pelos Godos. Cf. Herculano, \textunderscore Eurico\textunderscore , 324.
\section{Efípio}
\begin{itemize}
\item {Grp. gram.:m.}
\end{itemize}
Omesmo que \textunderscore efípia\textunderscore .
\section{Efodiofobia}
\begin{itemize}
\item {Grp. gram.:f.}
\end{itemize}
\begin{itemize}
\item {Utilização:Med.}
\end{itemize}
\begin{itemize}
\item {Proveniência:(Do gr. \textunderscore ephodos\textunderscore  + \textunderscore phobein\textunderscore )}
\end{itemize}
Horror aos preparativos de viagem. Cf. Sousa Martins, \textunderscore Nosogr.\textunderscore 
\section{Eforado}
\begin{itemize}
\item {Grp. gram.:m.}
\end{itemize}
O mesmo que \textunderscore eforato\textunderscore .
\section{Eforato}
\begin{itemize}
\item {Grp. gram.:m.}
\end{itemize}
Cargo de éforo. Cf. Latino, \textunderscore Elogios\textunderscore , 125 e 221.
\section{Eforia}
\begin{itemize}
\item {Grp. gram.:f.}
\end{itemize}
O mesmo que \textunderscore eforato\textunderscore .
\section{Efórico}
\begin{itemize}
\item {Grp. gram.:adj.}
\end{itemize}
Relativo aos éforos.
\section{Éforo}
\begin{itemize}
\item {Grp. gram.:m.}
\end{itemize}
\begin{itemize}
\item {Proveniência:(Gr. \textunderscore ephoros\textunderscore )}
\end{itemize}
Cada um dos cinco magistrados espartanos, que se renovavam anualmente, e que contrabalançavam o poder dos reis e do senado.
\section{Efredina}
\begin{itemize}
\item {Grp. gram.:f.}
\end{itemize}
Espécie de colírio.
\section{Éolo}
\begin{itemize}
\item {Grp. gram.:m.}
\end{itemize}
\begin{itemize}
\item {Proveniência:(Lat. \textunderscore aeolus\textunderscore )}
\end{itemize}
Vento forte. Cf. B. do Penedo, \textunderscore Missão\textunderscore , 50.
\section{Eóo}
\begin{itemize}
\item {Grp. gram.:adj.}
\end{itemize}
\begin{itemize}
\item {Utilização:Des.}
\end{itemize}
\begin{itemize}
\item {Proveniência:(Lat. \textunderscore eous\textunderscore )}
\end{itemize}
Oriental.
\section{Epácrida}
\begin{itemize}
\item {Grp. gram.:f.}
\end{itemize}
\begin{itemize}
\item {Proveniência:(Do gr. \textunderscore epi\textunderscore  + \textunderscore akros\textunderscore )}
\end{itemize}
Arbusto da Austrália e da Nova Zelândia.
\section{Epacrídeas}
\begin{itemize}
\item {Grp. gram.:f. pl.}
\end{itemize}
Família de plantas, que têm por typo a epácrida.
\section{Epacta}
\begin{itemize}
\item {Grp. gram.:f.}
\end{itemize}
\begin{itemize}
\item {Utilização:Chron.}
\end{itemize}
\begin{itemize}
\item {Proveniência:(Lat. \textunderscore epactae\textunderscore )}
\end{itemize}
Número de dias que se accrescentam ao anno lunar, para o igualar com o anno solar.
\section{Epactal}
\begin{itemize}
\item {Grp. gram.:adj.}
\end{itemize}
\begin{itemize}
\item {Utilização:Anat.}
\end{itemize}
Relativo a epacta.
Diz-se do osso craniano, que fica na parte inferior do occipital.
\section{Epagoge}
\begin{itemize}
\item {Grp. gram.:f.}
\end{itemize}
\begin{itemize}
\item {Utilização:Philos.}
\end{itemize}
\begin{itemize}
\item {Proveniência:(Lat. \textunderscore epagoge\textunderscore )}
\end{itemize}
O mesmo que \textunderscore inducção\textunderscore . Cf. Latino, \textunderscore Or. da Corôa\textunderscore , CLV.
\section{Epagógico}
\begin{itemize}
\item {Grp. gram.:adj.}
\end{itemize}
Relativo a epagoge.
\section{Epagogo}
\begin{itemize}
\item {Grp. gram.:m.}
\end{itemize}
\begin{itemize}
\item {Proveniência:(Do gr. \textunderscore epi\textunderscore  + \textunderscore agein\textunderscore )}
\end{itemize}
Magistrado grego, que resolvia summariamente as questões de direito commercial marítimo.
\section{Epalta}
\begin{itemize}
\item {Grp. gram.:f.}
\end{itemize}
\begin{itemize}
\item {Proveniência:(Do gr. \textunderscore epaltes\textunderscore )}
\end{itemize}
Gênero de plantas, da fam. das compostas.
\section{Epanadiplose}
\begin{itemize}
\item {Grp. gram.:f.}
\end{itemize}
\begin{itemize}
\item {Utilização:Rhet.}
\end{itemize}
\begin{itemize}
\item {Proveniência:(Lat. \textunderscore epanadiplosis\textunderscore )}
\end{itemize}
Repetição de uma palavra no princípio e fim do um verso ou de uma phrase.
\section{Epanáfora}
\begin{itemize}
\item {Grp. gram.:f.}
\end{itemize}
\begin{itemize}
\item {Utilização:Rhet.}
\end{itemize}
\begin{itemize}
\item {Proveniência:(Lat. \textunderscore epanaphora\textunderscore )}
\end{itemize}
Repetição da mesma palavra no princípio de cada membro de um período, ou no princípio de cada verso.
\section{Epanalepse}
\begin{itemize}
\item {Grp. gram.:f.}
\end{itemize}
\begin{itemize}
\item {Utilização:Rhet.}
\end{itemize}
\begin{itemize}
\item {Proveniência:(Lat. \textunderscore epanalepsis\textunderscore )}
\end{itemize}
Repetição de uma palavra no meio de phrases seguidas.
\section{Epanáphora}
\begin{itemize}
\item {Grp. gram.:f.}
\end{itemize}
\begin{itemize}
\item {Utilização:Rhet.}
\end{itemize}
\begin{itemize}
\item {Proveniência:(Lat. \textunderscore epanaphora\textunderscore )}
\end{itemize}
Repetição da mesma palavra no princípio de cada membro de um período, ou no princípio de cada verso.
\section{Epanástrofe}
\begin{itemize}
\item {Grp. gram.:f.}
\end{itemize}
\begin{itemize}
\item {Utilização:Rhet.}
\end{itemize}
\begin{itemize}
\item {Proveniência:(Gr. \textunderscore panostrophe\textunderscore )}
\end{itemize}
Repetição da mesma palavra no fim de uma proposição e no princípio de outra.
\section{Epanástrophe}
\begin{itemize}
\item {Grp. gram.:f.}
\end{itemize}
\begin{itemize}
\item {Utilização:Rhet.}
\end{itemize}
\begin{itemize}
\item {Proveniência:(Gr. \textunderscore panostrophe\textunderscore )}
\end{itemize}
Repetição da mesma palavra no fim de uma proposição e no princípio de outra.
\section{Epânodos}
\begin{itemize}
\item {Grp. gram.:m.}
\end{itemize}
\begin{itemize}
\item {Proveniência:(Gr. \textunderscore epanodos\textunderscore )}
\end{itemize}
Figura de grammática, que consiste em repetir, separando-as e juntando-as com outras, palavras que se pronunciaram ou escreveram juntas.
\section{Epanorthose}
\begin{itemize}
\item {Grp. gram.:f.}
\end{itemize}
\begin{itemize}
\item {Proveniência:(Lat. \textunderscore epanodos\textunderscore )}
\end{itemize}
Correcção ou emenda de palavra ou phrase por fingido arrependimento, para se empregar outra mais expressiva.
\section{Epanortose}
\begin{itemize}
\item {Grp. gram.:f.}
\end{itemize}
\begin{itemize}
\item {Proveniência:(Lat. \textunderscore epanodos\textunderscore )}
\end{itemize}
Correcção ou emenda de palavra ou phrase por fingido arrependimento, para se empregar outra mais expressiva.
\section{Epêndima}
\begin{itemize}
\item {Grp. gram.:m.}
\end{itemize}
\begin{itemize}
\item {Utilização:Anat.}
\end{itemize}
\begin{itemize}
\item {Proveniência:(Do gr. \textunderscore epi\textunderscore  + \textunderscore enduma\textunderscore )}
\end{itemize}
Membrana dos ventrículos do cérebro, e do canal central da medula espinhal.
\section{Ependimite}
\begin{itemize}
\item {Grp. gram.:f.}
\end{itemize}
\begin{itemize}
\item {Utilização:Med.}
\end{itemize}
Inflammção do epêndima.
\section{Epêndyma}
\begin{itemize}
\item {Grp. gram.:m.}
\end{itemize}
\begin{itemize}
\item {Utilização:Anat.}
\end{itemize}
\begin{itemize}
\item {Proveniência:(Do gr. \textunderscore epi\textunderscore  + \textunderscore enduma\textunderscore )}
\end{itemize}
Membrana dos ventrículos do cérebro, e do canal central da medulla espinhal.
\section{Ependymite}
\begin{itemize}
\item {Grp. gram.:f.}
\end{itemize}
\begin{itemize}
\item {Utilização:Med.}
\end{itemize}
Inflammação do epêndyma.
\section{Epêntese}
\begin{itemize}
\item {Grp. gram.:f.}
\end{itemize}
\begin{itemize}
\item {Utilização:Gram.}
\end{itemize}
\begin{itemize}
\item {Proveniência:(Lat. \textunderscore epenthesis\textunderscore )}
\end{itemize}
Acrescentamento ou inclusão de uma letra ou de uma sílaba sem valor determinado no meio de uma palavra.
\section{Epentético}
\begin{itemize}
\item {Grp. gram.:adj.}
\end{itemize}
Acrescentado por epêntese.
Em que há epêntese.
\section{Epênthese}
\begin{itemize}
\item {Grp. gram.:f.}
\end{itemize}
\begin{itemize}
\item {Utilização:Gram.}
\end{itemize}
\begin{itemize}
\item {Proveniência:(Lat. \textunderscore epenthesis\textunderscore )}
\end{itemize}
Accrescentamento ou inclusão de uma letra ou de uma sýllaba sem valor determinado no meio de uma palavra.
\section{Epenthético}
\begin{itemize}
\item {Grp. gram.:adj.}
\end{itemize}
Accrescentado por epênthese.
Em que há epênthese.
\section{Epéolo}
\begin{itemize}
\item {Grp. gram.:m.}
\end{itemize}
\begin{itemize}
\item {Proveniência:(Do gr. \textunderscore epi\textunderscore  + \textunderscore aiolos\textunderscore )}
\end{itemize}
Insecto hymenóptero, vulgar nas vizinhanças de París.
\section{Eperlano}
\begin{itemize}
\item {Grp. gram.:m.}
\end{itemize}
\begin{itemize}
\item {Proveniência:(Fr. \textunderscore eperlan\textunderscore )}
\end{itemize}
Espécie de salmão.
\section{Epexegese}
\begin{itemize}
\item {fónica:esé}
\end{itemize}
\begin{itemize}
\item {Grp. gram.:f.}
\end{itemize}
\begin{itemize}
\item {Proveniência:(Lat. \textunderscore epexegesis\textunderscore )}
\end{itemize}
Figura grammatical, o mesmo que opposição.
\section{Ephebato}
\begin{itemize}
\item {Grp. gram.:m.}
\end{itemize}
\begin{itemize}
\item {Utilização:Des.}
\end{itemize}
\begin{itemize}
\item {Proveniência:(Lat. \textunderscore ephebalus\textunderscore )}
\end{itemize}
Aquelle que chegou á puberdade.
\section{Ephebo}
\begin{itemize}
\item {Grp. gram.:m.}
\end{itemize}
\begin{itemize}
\item {Proveniência:(Lat. \textunderscore ephebus\textunderscore )}
\end{itemize}
Aquelle que chegou á puberdade.
Homem moço.
\section{Éphedra}
\begin{itemize}
\item {Grp. gram.:f.}
\end{itemize}
\begin{itemize}
\item {Proveniência:(Lat. \textunderscore ephedra\textunderscore )}
\end{itemize}
Árvore conífera, de fruto vermelho e azedo, (\textunderscore ephedra distachya\textunderscore , Lin.).
\section{Ephélide}
\begin{itemize}
\item {Grp. gram.:f.}
\end{itemize}
\begin{itemize}
\item {Proveniência:(Lat. \textunderscore ephelis\textunderscore )}
\end{itemize}
Mancha na pelle.
\section{Ephêmeras}
\begin{itemize}
\item {Grp. gram.:f.}
\end{itemize}
\begin{itemize}
\item {Proveniência:(De \textunderscore ephêmero\textunderscore )}
\end{itemize}
Insectos neurópteros, que nascem e morrem no mesmo dia.
\section{Ephemeridade}
\begin{itemize}
\item {Grp. gram.:f.}
\end{itemize}
Qualidade daquillo que é ephêmero.
\section{Ephemérides}
\begin{itemize}
\item {Grp. gram.:f. pl.}
\end{itemize}
\begin{itemize}
\item {Proveniência:(Do lat. \textunderscore ephemeris\textunderscore )}
\end{itemize}
Tábuas astronómicas, que indicam, dia a dia, a posição dos planetas no Zodíaco.
Diário.
Relação dos factos de cada dia, succedidos em differentes annos e lugares.
\section{Ephemerina}
\begin{itemize}
\item {Grp. gram.:f.}
\end{itemize}
\begin{itemize}
\item {Proveniência:(De \textunderscore ephêmero\textunderscore )}
\end{itemize}
Espécie de junco da América e da Índia.
\section{Ephemerizar}
\begin{itemize}
\item {Grp. gram.:v. t.}
\end{itemize}
\begin{itemize}
\item {Utilização:Neol.}
\end{itemize}
Tornar ephêmero.
Narrar, dia a dia, a história de.
\section{Ephêmero}
\begin{itemize}
\item {Grp. gram.:adj.}
\end{itemize}
\begin{itemize}
\item {Grp. gram.:M.}
\end{itemize}
\begin{itemize}
\item {Proveniência:(Gr. \textunderscore ephemeros\textunderscore )}
\end{itemize}
Que dura um só dia.
Passageiro, transitório.
O mesmo que \textunderscore ephemerina\textunderscore .
\section{Ephesino}
\begin{itemize}
\item {Grp. gram.:adj.}
\end{itemize}
\begin{itemize}
\item {Proveniência:(Lat. \textunderscore ephesinus\textunderscore )}
\end{itemize}
Relativo a Épheso.
\section{Ephésios}
\begin{itemize}
\item {Grp. gram.:m. pl.}
\end{itemize}
\begin{itemize}
\item {Proveniência:(Lat. \textunderscore ephasius\textunderscore )}
\end{itemize}
Habitantes de Épheso.
\section{Épheta}
\begin{itemize}
\item {Grp. gram.:m.}
\end{itemize}
\begin{itemize}
\item {Proveniência:(Gr. \textunderscore ephetes\textunderscore )}
\end{itemize}
Magistrado criminal em Athenas.
\section{Ephialta}
\begin{itemize}
\item {Grp. gram.:f.}
\end{itemize}
\begin{itemize}
\item {Utilização:Des.}
\end{itemize}
\begin{itemize}
\item {Proveniência:(Do lat. \textunderscore ephialtes\textunderscore )}
\end{itemize}
Pesadelo.
\section{Ephialtes}
\begin{itemize}
\item {Grp. gram.:m.}
\end{itemize}
O mesmo que \textunderscore ephialta\textunderscore .
\section{Ephidrose}
\begin{itemize}
\item {Grp. gram.:f.}
\end{itemize}
\begin{itemize}
\item {Proveniência:(Do gr. \textunderscore epi\textunderscore  + \textunderscore hudor\textunderscore )}
\end{itemize}
Doença, caracterizada por suor na parte superior do corpo.
Suor, na crise de algumas doenças.
\section{Ephíppia}
\begin{itemize}
\item {Grp. gram.:f.}
\end{itemize}
\begin{itemize}
\item {Proveniência:(Lat. \textunderscore ephippium\textunderscore )}
\end{itemize}
Sella de lan, imitada da cavallaria romana pelos Godos. Cf. Herculano, \textunderscore Eurico\textunderscore , 324.
\section{Ephíppio}
\begin{itemize}
\item {Grp. gram.:m.}
\end{itemize}
Omesmo que \textunderscore ephíppia\textunderscore .
\section{Ephodiophobia}
\begin{itemize}
\item {Grp. gram.:f.}
\end{itemize}
\begin{itemize}
\item {Utilização:Med.}
\end{itemize}
\begin{itemize}
\item {Proveniência:(Do gr. \textunderscore ephodos\textunderscore  + \textunderscore phobein\textunderscore )}
\end{itemize}
Horror aos preparativos de viagem. Cf. Sousa Martins, \textunderscore Nosogr.\textunderscore 
\section{Ephorado}
\begin{itemize}
\item {Grp. gram.:m.}
\end{itemize}
O mesmo que \textunderscore ephorato\textunderscore .
\section{Ephorato}
\begin{itemize}
\item {Grp. gram.:m.}
\end{itemize}
Cargo de éphoro. Cf. Latino, \textunderscore Elogios\textunderscore , 125 e 221.
\section{Ephoria}
\begin{itemize}
\item {Grp. gram.:f.}
\end{itemize}
O mesmo que \textunderscore ephorato\textunderscore .
\section{Ephórico}
\begin{itemize}
\item {Grp. gram.:adj.}
\end{itemize}
Relativo aos éphoros.
\section{Éphoro}
\begin{itemize}
\item {Grp. gram.:m.}
\end{itemize}
\begin{itemize}
\item {Proveniência:(Gr. \textunderscore ephoros\textunderscore )}
\end{itemize}
Cada um dos cinco magistrados espartanos, que se renovavam annualmente, e que contrabalançavam o poder dos reis e do senado.
\section{Ephredina}
\begin{itemize}
\item {Grp. gram.:f.}
\end{itemize}
Espécie de colýrio.
\section{Epi...}
\begin{itemize}
\item {Grp. gram.:pref.}
\end{itemize}
\begin{itemize}
\item {Proveniência:(Gr. \textunderscore epi\textunderscore )}
\end{itemize}
(que significa \textunderscore sôbre\textunderscore , \textunderscore depois\textunderscore )
\section{Epiblasto}
\begin{itemize}
\item {Grp. gram.:m.}
\end{itemize}
\begin{itemize}
\item {Proveniência:(Do gr. \textunderscore epi\textunderscore  + \textunderscore blastos\textunderscore )}
\end{itemize}
Appêndice, no blasto de algumas plantas.
Em anatomia, folheto exterior da gástrula.
\section{Epicalícia}
\begin{itemize}
\item {Grp. gram.:f.}
\end{itemize}
\begin{itemize}
\item {Proveniência:(De \textunderscore epi...\textunderscore  + \textunderscore cálice\textunderscore )}
\end{itemize}
Classe de plantas, cujos estames são inseridos sôbre o cálice.
\section{Epicamente}
\begin{itemize}
\item {Grp. gram.:adv.}
\end{itemize}
Heroicamente, de modo épico.
\section{Epicantho}
\begin{itemize}
\item {Grp. gram.:m.}
\end{itemize}
\begin{itemize}
\item {Proveniência:(Do gr. \textunderscore epi\textunderscore  + \textunderscore canthos\textunderscore )}
\end{itemize}
Refêgo no canto interior dos olhos, por excesso de pelle na raiz do naris.
\section{Epicanto}
\begin{itemize}
\item {Grp. gram.:m.}
\end{itemize}
\begin{itemize}
\item {Proveniência:(Do gr. \textunderscore epi\textunderscore  + \textunderscore canthos\textunderscore )}
\end{itemize}
Refêgo no canto interior dos olhos, por excesso de pele na raiz do naris.
\section{Epicárpico}
\begin{itemize}
\item {Grp. gram.:adj.}
\end{itemize}
Relativo ao epicarpo.
\section{Epicarpo}
\begin{itemize}
\item {Grp. gram.:m.}
\end{itemize}
\begin{itemize}
\item {Utilização:Ant.}
\end{itemize}
\begin{itemize}
\item {Proveniência:(Do gr. \textunderscore epi\textunderscore  + \textunderscore karpos\textunderscore )}
\end{itemize}
Pellicula externa da fruta.
Emplastro febrífugo, que se applicava sôbre os pulsos.
\section{Epicaule}
\begin{itemize}
\item {Grp. gram.:adj.}
\end{itemize}
\begin{itemize}
\item {Proveniência:(De \textunderscore epi...\textunderscore  + \textunderscore caule\textunderscore )}
\end{itemize}
Diz-se do vegetal parasito, que cresce no caule de outros vegetaes.
\section{Epicauma}
\begin{itemize}
\item {Grp. gram.:f.}
\end{itemize}
\begin{itemize}
\item {Utilização:Med.}
\end{itemize}
\begin{itemize}
\item {Proveniência:(Gr. \textunderscore epikauma\textunderscore )}
\end{itemize}
Ulceração da córnea transparente.
\section{Epícea}
\begin{itemize}
\item {Grp. gram.:f.}
\end{itemize}
\begin{itemize}
\item {Proveniência:(Do lat. \textunderscore e\textunderscore  + \textunderscore piceus\textunderscore )}
\end{itemize}
Pinheiro alvar.
\section{Epicédio}
\begin{itemize}
\item {Grp. gram.:m.}
\end{itemize}
\begin{itemize}
\item {Proveniência:(Lat. \textunderscore epicedion\textunderscore )}
\end{itemize}
Composição poética ou discurso, em memória de alguém.
Elegia; nênia.
\section{Epiceno}
\begin{itemize}
\item {Grp. gram.:adj.}
\end{itemize}
\begin{itemize}
\item {Utilização:Gram.}
\end{itemize}
\begin{itemize}
\item {Proveniência:(Gr. \textunderscore epikoinos\textunderscore )}
\end{itemize}
Que designa, sob uma só fórma, ambos os sexos.
Diz-se da palavra que, com uma só desinência, designa o gênero masculino e feminino.
\section{Epicentro}
\begin{itemize}
\item {Grp. gram.:m.}
\end{itemize}
\begin{itemize}
\item {Utilização:Geol.}
\end{itemize}
\begin{itemize}
\item {Proveniência:(De \textunderscore epi...\textunderscore  + \textunderscore centro\textunderscore )}
\end{itemize}
Área superficial dos sismos, geralmente elliptica.
\section{Epicerástico}
\begin{itemize}
\item {Grp. gram.:adj.}
\end{itemize}
\begin{itemize}
\item {Proveniência:(Gr. \textunderscore epikerastikos\textunderscore )}
\end{itemize}
Dizia-se dos medicamentos ou substâncias que temperavam a acidez dos humores.
\section{Epicheia}
\begin{itemize}
\item {fónica:quei}
\end{itemize}
\begin{itemize}
\item {Grp. gram.:f.}
\end{itemize}
\begin{itemize}
\item {Utilização:Fig.}
\end{itemize}
\begin{itemize}
\item {Proveniência:(Do gr. \textunderscore epikhein\textunderscore )}
\end{itemize}
Razoável interpretação de uma lei ou preceito.
Moderação, meio termo.
\section{Epichirema}
\begin{itemize}
\item {fónica:qui}
\end{itemize}
\begin{itemize}
\item {Grp. gram.:m.}
\end{itemize}
\begin{itemize}
\item {Proveniência:(Lat. \textunderscore epichirema\textunderscore )}
\end{itemize}
Syllogismo, em que as premissas ou uma dellas são acompanhadas de prova.
\section{Epichiremático}
\begin{itemize}
\item {fónica:qui}
\end{itemize}
\begin{itemize}
\item {Grp. gram.:adj.}
\end{itemize}
Relativo a epichirema.
\section{Epichoriano}
\begin{itemize}
\item {fónica:co}
\end{itemize}
\begin{itemize}
\item {Proveniência:(Do gr. \textunderscore epikhorios\textunderscore )}
\end{itemize}
Dizia-se, entre os Gregos, dos deuses privativos de certas regiões.
\section{Epichthoniano}
\begin{itemize}
\item {Grp. gram.:adj.}
\end{itemize}
\begin{itemize}
\item {Proveniência:(Do gr. \textunderscore epikthonios\textunderscore )}
\end{itemize}
Dizia-se, na antiguidade grega, dos deuses terrestres, por opposição aos infernaes e, por vezes, aos celestes.
\section{Epiciclo}
\begin{itemize}
\item {Grp. gram.:m.}
\end{itemize}
\begin{itemize}
\item {Proveniência:(Do gr. \textunderscore epi\textunderscore  + \textunderscore kuklos\textunderscore )}
\end{itemize}
Pequeno círculo imaginário da esfera celeste, tendo o centro na circunferência de outro círculo maior.
\section{Epicicloidal}
\begin{itemize}
\item {Grp. gram.:adj.}
\end{itemize}
Relativo a epicicloide.
\section{Epicicloide}
\begin{itemize}
\item {Grp. gram.:f.}
\end{itemize}
\begin{itemize}
\item {Utilização:Mathem.}
\end{itemize}
\begin{itemize}
\item {Proveniência:(Do gr. \textunderscore epikuklos\textunderscore  + \textunderscore eidos\textunderscore )}
\end{itemize}
Curva, produzida por um ponto de uma circunferência de um círculo que gira sôbre a parte côncava ou convexa de outro círculo.
\section{Epiclino}
\begin{itemize}
\item {Grp. gram.:adj.}
\end{itemize}
\begin{itemize}
\item {Utilização:Des.}
\end{itemize}
\begin{itemize}
\item {Proveniência:(Do gr. \textunderscore epi\textunderscore  + \textunderscore kline\textunderscore )}
\end{itemize}
Diz-se do órgão, collocado sôbre o receptáculo da flôr.
\section{Epicmástico}
\begin{itemize}
\item {Grp. gram.:adj.}
\end{itemize}
\begin{itemize}
\item {Utilização:Med.}
\end{itemize}
\begin{itemize}
\item {Proveniência:(Gr. \textunderscore epikmasticos\textunderscore )}
\end{itemize}
Que aumenta gradualmente, (falando-se da febre).
\section{Épico}
\begin{itemize}
\item {Grp. gram.:adj.}
\end{itemize}
\begin{itemize}
\item {Proveniência:(Lat. \textunderscore epicus\textunderscore )}
\end{itemize}
Relativo á epopeia; heroico.
Que escreveu uma epopeia ou mais: \textunderscore poéta épico\textunderscore .
Maravilhoso.
\section{Epicombos}
\begin{itemize}
\item {Grp. gram.:m. pl.}
\end{itemize}
\begin{itemize}
\item {Utilização:Ant.}
\end{itemize}
\begin{itemize}
\item {Proveniência:(Do gr. \textunderscore epi\textunderscore  + \textunderscore kombos\textunderscore )}
\end{itemize}
Ramalhetes com moédas de oiro e de prata, que um senador atirava ao povo, quando o imperador de Constantinopla saía da Igreja.
\section{Epicomete}
\begin{itemize}
\item {Grp. gram.:m.}
\end{itemize}
\begin{itemize}
\item {Proveniência:(Do gr. \textunderscore epi\textunderscore  + \textunderscore kometes\textunderscore )}
\end{itemize}
Insecto coleóptero pentâmero.
\section{Epicometo}
\begin{itemize}
\item {Grp. gram.:m.}
\end{itemize}
\begin{itemize}
\item {Proveniência:(Do gr. \textunderscore epi\textunderscore  + \textunderscore kometes\textunderscore )}
\end{itemize}
Insecto coleóptero pentâmero.
\section{Epicondiliano}
\begin{itemize}
\item {Grp. gram.:adj.}
\end{itemize}
Relativo ao epicôndilo.
\section{Epicondiliar}
\begin{itemize}
\item {Grp. gram.:adj.}
\end{itemize}
O mesmo que \textunderscore epicondiliano\textunderscore .
\section{Epicôndilo}
\begin{itemize}
\item {Grp. gram.:m.}
\end{itemize}
\begin{itemize}
\item {Utilização:Anat.}
\end{itemize}
\begin{itemize}
\item {Proveniência:(De \textunderscore epi...\textunderscore  + \textunderscore côndylo\textunderscore )}
\end{itemize}
A saliência mais externa da extremidade inferior do húmero.
\section{Epicondyliano}
\begin{itemize}
\item {Grp. gram.:adj.}
\end{itemize}
Relativo ao epicôndylo.
\section{Epicondyliar}
\begin{itemize}
\item {Grp. gram.:adj.}
\end{itemize}
O mesmo que \textunderscore epicondyliano\textunderscore .
\section{Epicôndylo}
\begin{itemize}
\item {Grp. gram.:m.}
\end{itemize}
\begin{itemize}
\item {Utilização:Anat.}
\end{itemize}
\begin{itemize}
\item {Proveniência:(De \textunderscore epi...\textunderscore  + \textunderscore côndylo\textunderscore )}
\end{itemize}
A saliência mais externa da extremidade inferior do húmero.
\section{Epicopo}
\begin{itemize}
\item {Grp. gram.:m.}
\end{itemize}
\begin{itemize}
\item {Proveniência:(Lat. \textunderscore epicopus\textunderscore )}
\end{itemize}
Batel de ramos, usado entre os Gregos.
\section{Epicoriano}
\begin{itemize}
\item {Proveniência:(Do gr. \textunderscore epikhorios\textunderscore )}
\end{itemize}
Dizia-se, entre os Gregos, dos deuses privativos de certas regiões.
\section{Epicorólia}
\begin{itemize}
\item {Grp. gram.:f.}
\end{itemize}
\begin{itemize}
\item {Proveniência:(De \textunderscore epi...\textunderscore  + \textunderscore corolla\textunderscore )}
\end{itemize}
Designação da 10.^a e da 11.^a classe de vegetaes, segundo Desvaux e Jussieu.
\section{Epicoróllia}
\begin{itemize}
\item {Grp. gram.:f.}
\end{itemize}
\begin{itemize}
\item {Proveniência:(De \textunderscore epi...\textunderscore  + \textunderscore corolla\textunderscore )}
\end{itemize}
Designação da 10.^a e da 11.^a classe de vegetaes, segundo Desvaux e Jussieu.
\section{Epicraniano}
\begin{itemize}
\item {Grp. gram.:adj.}
\end{itemize}
O mesmo que \textunderscore epicrânico\textunderscore .
\section{Epicrânico}
\begin{itemize}
\item {Grp. gram.:adj.}
\end{itemize}
Relativo ao \textunderscore epicrânio\textunderscore .
\section{Epicrânio}
\begin{itemize}
\item {Grp. gram.:m.}
\end{itemize}
\begin{itemize}
\item {Grp. gram.:Adj.}
\end{itemize}
\begin{itemize}
\item {Proveniência:(De \textunderscore epi...\textunderscore  + \textunderscore crânio\textunderscore )}
\end{itemize}
Conjunto das partes que revestem o crânio.
Situado sôbre o crânio.
\section{Epícrase}
\begin{itemize}
\item {Grp. gram.:f.}
\end{itemize}
\begin{itemize}
\item {Proveniência:(De \textunderscore epi...\textunderscore  + \textunderscore crase\textunderscore )}
\end{itemize}
Medicação, com que se suppunha corrigirem-se os humores viciados.
\section{Epícrate}
\begin{itemize}
\item {Grp. gram.:m.}
\end{itemize}
\begin{itemize}
\item {Proveniência:(Gr. \textunderscore epikrates\textunderscore )}
\end{itemize}
Reptil, espécie de gibóia.
\section{Epicrise}
\begin{itemize}
\item {Grp. gram.:f.}
\end{itemize}
\begin{itemize}
\item {Utilização:Ant.}
\end{itemize}
\begin{itemize}
\item {Proveniência:(Gr. \textunderscore epikrisis\textunderscore )}
\end{itemize}
Apreciação crítica de uma doença, da sua origem, andamento e resultados.
\section{Epicritíco}
\begin{itemize}
\item {Grp. gram.:adj.}
\end{itemize}
Relativo a epicrise.
\section{Epictoniano}
\begin{itemize}
\item {Grp. gram.:adj.}
\end{itemize}
\begin{itemize}
\item {Proveniência:(Do gr. \textunderscore epikthonios\textunderscore )}
\end{itemize}
Dizia-se, na antiguidade grega, dos deuses terrestres, por oposição aos infernaes e, por vezes, aos celestes.
\section{Epicureísmo}
\begin{itemize}
\item {Grp. gram.:m.}
\end{itemize}
O mesmo que \textunderscore epicurismo\textunderscore . Cf. Latino, \textunderscore Elogios\textunderscore , 342.
\section{Epicureu}
\begin{itemize}
\item {Grp. gram.:adj.}
\end{itemize}
O mesmo que \textunderscore epicurio\textunderscore . Cf. \textunderscore Lusiadas\textunderscore , VII, 75.
\section{Epicurio}
\begin{itemize}
\item {Grp. gram.:adj.}
\end{itemize}
\begin{itemize}
\item {Grp. gram.:M.}
\end{itemize}
\begin{itemize}
\item {Proveniência:(Lat. \textunderscore epicurius\textunderscore , de \textunderscore Epicurus\textunderscore , n. p.)}
\end{itemize}
Relativo ao systema de Epicuro.
Libertino.
Séctârio da doutrina de Epicuro.
Homem sensual.
\section{Epicurismo}
\begin{itemize}
\item {Grp. gram.:m.}
\end{itemize}
Doutrina do Epicuro; sensualidade.
\section{Epicurista}
\begin{itemize}
\item {Grp. gram.:m.  e  adj.}
\end{itemize}
O mesmo que \textunderscore epicurio\textunderscore .
\section{Epicyclo}
\begin{itemize}
\item {Grp. gram.:m.}
\end{itemize}
\begin{itemize}
\item {Proveniência:(Do gr. \textunderscore epi\textunderscore  + \textunderscore kuklos\textunderscore )}
\end{itemize}
Pequeno círculo imaginário da esphera celeste, tendo o centro na circunferência de outro círculo maior.
\section{Epicycloidal}
\begin{itemize}
\item {Grp. gram.:adj.}
\end{itemize}
Relativo a epicycloide.
\section{Epicycloide}
\begin{itemize}
\item {Grp. gram.:f.}
\end{itemize}
\begin{itemize}
\item {Utilização:Mathem.}
\end{itemize}
\begin{itemize}
\item {Proveniência:(Do gr. \textunderscore epikuklos\textunderscore  + \textunderscore eidos\textunderscore )}
\end{itemize}
Curva, produzida por um ponto de uma circunferência de um círculo que gira sôbre a parte côncava ou convexa de outro círculo.
\section{Epidêipnides}
\begin{itemize}
\item {Grp. gram.:pl. f.}
\end{itemize}
\begin{itemize}
\item {Utilização:Des.}
\end{itemize}
\begin{itemize}
\item {Proveniência:(Gr. \textunderscore epideipnis\textunderscore )}
\end{itemize}
Frutas, ou qualquer sobremesa.
\section{Epidípnides}
\begin{itemize}
\item {Grp. gram.:pl. f.}
\end{itemize}
\begin{itemize}
\item {Utilização:Des.}
\end{itemize}
\begin{itemize}
\item {Proveniência:(Gr. \textunderscore epideipnis\textunderscore )}
\end{itemize}
Frutas, ou qualquer sobremesa.
\section{Epídema}
\begin{itemize}
\item {Grp. gram.:f.}
\end{itemize}
\begin{itemize}
\item {Utilização:Zool.}
\end{itemize}
\begin{itemize}
\item {Proveniência:(Gr. \textunderscore epidema\textunderscore )}
\end{itemize}
Prolongamento laminar, dentro do thorax dos animaes articulados.
\section{Epidemia}
\begin{itemize}
\item {Grp. gram.:f.}
\end{itemize}
\begin{itemize}
\item {Utilização:Fig.}
\end{itemize}
\begin{itemize}
\item {Proveniência:(Lat. \textunderscore epidemia\textunderscore )}
\end{itemize}
Doença, que ataca muitos indivíduos, ao mesmo tempo e na mesma terra ou região.
Ideias, systemas ou coisas, que se diffundem rapidamente, dominando os espiritos ou os costumes.
\section{Epidemicamente}
\begin{itemize}
\item {Grp. gram.:adv.}
\end{itemize}
De modo epidêmico.
\section{Epidemicidade}
\begin{itemize}
\item {Grp. gram.:f.}
\end{itemize}
Qualidade daquillo que é epidêmico.
\section{Epidêmico}
\begin{itemize}
\item {Grp. gram.:adj.}
\end{itemize}
Relativo a epidemia.
Que tem o carácter ou a natureza de epidemia.
\section{Epidêmio}
\begin{itemize}
\item {Grp. gram.:m.}
\end{itemize}
\begin{itemize}
\item {Proveniência:(Gr. \textunderscore epidemios\textunderscore )}
\end{itemize}
Planta da Índia.
\section{Epidemiologia}
\begin{itemize}
\item {Grp. gram.:f.}
\end{itemize}
\begin{itemize}
\item {Proveniência:(Do gr. \textunderscore epidemios\textunderscore  + \textunderscore logos\textunderscore )}
\end{itemize}
Tratado sôbre epidemias.
\section{Epidemiológico}
\begin{itemize}
\item {Grp. gram.:adj.}
\end{itemize}
Relativo a epidemiologia.
\section{Epidemiologista}
\begin{itemize}
\item {Grp. gram.:m.}
\end{itemize}
Aquelle que é perito em epidemiologia.
\section{Epidemiólogo}
\begin{itemize}
\item {Grp. gram.:m.}
\end{itemize}
O mesmo que \textunderscore epidemiologista\textunderscore .
\section{Epidemologia}
\begin{itemize}
\item {Grp. gram.:f.}
\end{itemize}
O mesmo que \textunderscore epidemiologia\textunderscore , Cf. Latino, \textunderscore Or. da Corôa\textunderscore , LXXVIII.
\section{Epidêndreas}
\begin{itemize}
\item {Grp. gram.:f. pl.}
\end{itemize}
\begin{itemize}
\item {Utilização:Bot.}
\end{itemize}
Tríbo de orchídeas, que tem por typo o epidendro.
\section{Epidendro}
\begin{itemize}
\item {Grp. gram.:adj.}
\end{itemize}
\begin{itemize}
\item {Grp. gram.:M.}
\end{itemize}
\begin{itemize}
\item {Proveniência:(Do gr. \textunderscore epi\textunderscore  + \textunderscore dendron\textunderscore )}
\end{itemize}
Que cresce sôbre árvores.
Gênero de orchídeas.
\section{Epiderme}
\begin{itemize}
\item {Grp. gram.:f.}
\end{itemize}
\begin{itemize}
\item {Utilização:Fig.}
\end{itemize}
\begin{itemize}
\item {Proveniência:(Lat. \textunderscore epidermis\textunderscore )}
\end{itemize}
Membrana exterior da pelle.
Pelle.
Pellícula, que envolve as plantas herbáceas e os ramos tenros.
Parte exterior, superfície.
\section{Epidérmico}
\begin{itemize}
\item {Grp. gram.:adj.}
\end{itemize}
Relativo a epiderme.
\section{Epidíctico}
\begin{itemize}
\item {Grp. gram.:adj.}
\end{itemize}
\begin{itemize}
\item {Proveniência:(Lat. \textunderscore epidicticus\textunderscore )}
\end{itemize}
Ostentoso, apparatoso, (falando-se do estilo ou de um discurso).
\section{Epididimite}
\begin{itemize}
\item {Grp. gram.:f.}
\end{itemize}
Inflamação do epidídimo.
\section{Epidídimo}
\begin{itemize}
\item {Grp. gram.:m.}
\end{itemize}
\begin{itemize}
\item {Proveniência:(Do gr. \textunderscore epi\textunderscore  + \textunderscore didumos\textunderscore )}
\end{itemize}
Pequeno corpo oblongo, na parte superior do testículo.
\section{Epididymite}
\begin{itemize}
\item {Grp. gram.:f.}
\end{itemize}
Inflammação do epidýdimo.
\section{Epidídymo}
\begin{itemize}
\item {Grp. gram.:m.}
\end{itemize}
\begin{itemize}
\item {Proveniência:(Do gr. \textunderscore epi\textunderscore  + \textunderscore didumos\textunderscore )}
\end{itemize}
Pequeno corpo oblongo, na parte superior do testículo.
\section{Epidiscal}
\begin{itemize}
\item {Grp. gram.:adj.}
\end{itemize}
\begin{itemize}
\item {Utilização:Bot.}
\end{itemize}
\begin{itemize}
\item {Proveniência:(De \textunderscore epi...\textunderscore  + \textunderscore disco\textunderscore )}
\end{itemize}
Diz-se da insersão, quando os estames se inserem na parte superior do disco.
\section{Epídoto}
\begin{itemize}
\item {Grp. gram.:m.}
\end{itemize}
\begin{itemize}
\item {Proveniência:(Do gr. \textunderscore epi\textunderscore  + \textunderscore dotos\textunderscore )}
\end{itemize}
Espécie de silicato, que se apresenta sob a fórma de agulhas achatadas.
\section{Epídromo}
\begin{itemize}
\item {Grp. gram.:m.}
\end{itemize}
\begin{itemize}
\item {Proveniência:(Lat. \textunderscore epidromus\textunderscore )}
\end{itemize}
Nome, que os Romanos davam ao mastro e vela da popa do navio.
\section{Epienomia}
\begin{itemize}
\item {Grp. gram.:f.}
\end{itemize}
\begin{itemize}
\item {Proveniência:(Do gr. \textunderscore epi\textunderscore  + \textunderscore oinos\textunderscore )}
\end{itemize}
Doença das vinhas, em geral.
\section{Epigamia}
\begin{itemize}
\item {Grp. gram.:f.}
\end{itemize}
\begin{itemize}
\item {Proveniência:(Do gr. \textunderscore epi\textunderscore  + \textunderscore gamos\textunderscore )}
\end{itemize}
Faculdade de contratar casamentos entre cidades alliadas, na antiguidade grega.
\section{Epifania}
\begin{itemize}
\item {Grp. gram.:f.}
\end{itemize}
\begin{itemize}
\item {Proveniência:(Lat. \textunderscore epiphania\textunderscore )}
\end{itemize}
Festividade religiosa, em memória da manifestação de Cristo aos gentios.
Dia de Reis.
\section{Epifaringe}
\begin{itemize}
\item {Grp. gram.:f.}
\end{itemize}
\begin{itemize}
\item {Proveniência:(De \textunderscore epi...\textunderscore  + \textunderscore faringe\textunderscore )}
\end{itemize}
Uma das peças da bôca dos insectos himenópteros.
\section{Epifenómeno}
\begin{itemize}
\item {Grp. gram.:m.}
\end{itemize}
\begin{itemize}
\item {Utilização:Med.}
\end{itemize}
\begin{itemize}
\item {Proveniência:(De \textunderscore epi...\textunderscore  + \textunderscore fenómeno\textunderscore )}
\end{itemize}
Sintoma superveniente, numa doença já declarada.
\section{Epiféria}
\begin{itemize}
\item {Grp. gram.:f.}
\end{itemize}
Espécie de escudo antigo?:«\textunderscore as (testugens) ...eram mais levantadas..., ao revés de epiférias, que hoje usamos\textunderscore ». \textunderscore Viriato Trág.\textunderscore , II, XVII. Cf. \textunderscore Idem\textunderscore , XIV, 85.
\section{Epifilo}
\begin{itemize}
\item {Grp. gram.:adj.}
\end{itemize}
\begin{itemize}
\item {Utilização:Bot.}
\end{itemize}
\begin{itemize}
\item {Proveniência:(Do gr. \textunderscore epi\textunderscore  + \textunderscore phullon\textunderscore )}
\end{itemize}
Diz-se dos órgãos vegetaes, que crescem ou estão inseridos nas fôlhas das plantas.
\section{Epifilospérmeas}
\begin{itemize}
\item {Grp. gram.:f. pl.}
\end{itemize}
\begin{itemize}
\item {Utilização:Bot.}
\end{itemize}
\begin{itemize}
\item {Proveniência:(Do gr. \textunderscore epi\textunderscore  + \textunderscore phullon\textunderscore  + \textunderscore sperma\textunderscore )}
\end{itemize}
Fêtos, cujas fructificações ficam sôbre o dorso das fôlhas.
\section{Epifisário}
\begin{itemize}
\item {Grp. gram.:adj.}
\end{itemize}
Relativo a epífise.
Que tem epífise.
\section{Epífise}
\begin{itemize}
\item {Grp. gram.:f.}
\end{itemize}
\begin{itemize}
\item {Proveniência:(Do gr. \textunderscore epi\textunderscore  + \textunderscore phusis\textunderscore )}
\end{itemize}
Saliência óssea, que, unida a um osso por uma cartilagem, se converte em apófise, pelo desenvolvimento da ossificação.
\section{Epifitia}
\begin{itemize}
\item {Grp. gram.:f.}
\end{itemize}
Doença, que ataca ao mesmo tempo grande número de plantas.
Qualidade de epífito.
\section{Epífito}
\begin{itemize}
\item {Grp. gram.:adj}
\end{itemize}
\begin{itemize}
\item {Proveniência:(Do gr. \textunderscore epi\textunderscore  + \textunderscore phuton\textunderscore )}
\end{itemize}
Diz-se das plantas que crescem sôbre outras, sem se alimentarem da substância destas.
\section{Epifleose}
\begin{itemize}
\item {Grp. gram.:f.}
\end{itemize}
\begin{itemize}
\item {Utilização:Bot.}
\end{itemize}
\begin{itemize}
\item {Proveniência:(Do gr. \textunderscore epi\textunderscore  + \textunderscore phloios\textunderscore )}
\end{itemize}
A epiderme dos vegetaes.
\section{Epifonema}
\begin{itemize}
\item {Grp. gram.:m.}
\end{itemize}
\begin{itemize}
\item {Proveniência:(Lat. \textunderscore epiphonema\textunderscore )}
\end{itemize}
Exclamação sentencíosa, com que se termina uma narrativa ou um discurso.
\section{Epifonêmico}
\begin{itemize}
\item {Grp. gram.:adj.}
\end{itemize}
Relativo a epifonema.
Em que há epifonema.
\section{Epífora}
\begin{itemize}
\item {Grp. gram.:m.  e  f.}
\end{itemize}
\begin{itemize}
\item {Utilização:Med.}
\end{itemize}
\begin{itemize}
\item {Utilização:Rhet.}
\end{itemize}
\begin{itemize}
\item {Proveniência:(Lat. \textunderscore epiphora\textunderscore )}
\end{itemize}
Fluxo de lágrimas, constante e involuntário, produzido por doença que obstruiu as vias lacrimaes.
Repetição de uma ou mais palavras, no fim de cada membro de um periódo.
\section{Epifragma}
\begin{itemize}
\item {Grp. gram.:m.}
\end{itemize}
\begin{itemize}
\item {Utilização:Zool.}
\end{itemize}
\begin{itemize}
\item {Proveniência:(Do gr. \textunderscore epi\textunderscore  + \textunderscore phragma\textunderscore )}
\end{itemize}
Opérculo temporário na concha de alguns molucos.
\section{Epifragmático}
\begin{itemize}
\item {Grp. gram.:adj.}
\end{itemize}
Relativo a epifragma.
\section{Epífrase}
\begin{itemize}
\item {Grp. gram.:f.}
\end{itemize}
\begin{itemize}
\item {Utilização:Rhet.}
\end{itemize}
\begin{itemize}
\item {Proveniência:(De \textunderscore epi...\textunderscore  + \textunderscore frase\textunderscore )}
\end{itemize}
Acrescentamento a uma frase, que parecia concluida, para se desenvolverem ideias acessórias.
\section{Epigastralgia}
\begin{itemize}
\item {Grp. gram.:f.}
\end{itemize}
\begin{itemize}
\item {Proveniência:(Do gr. \textunderscore epi\textunderscore  + \textunderscore gaster\textunderscore  + \textunderscore algos\textunderscore )}
\end{itemize}
Dôr no epigastro.
\section{Epigástrico}
\begin{itemize}
\item {Grp. gram.:adj.}
\end{itemize}
Relativo a epigastro.
\section{Epigástrio}
\begin{itemize}
\item {Grp. gram.:m.}
\end{itemize}
O mesmo que \textunderscore epigastro\textunderscore .
\section{Epigastro}
\begin{itemize}
\item {Grp. gram.:m.}
\end{itemize}
\begin{itemize}
\item {Utilização:Anat.}
\end{itemize}
\begin{itemize}
\item {Proveniência:(Do gr. \textunderscore epi\textunderscore  + \textunderscore gaster\textunderscore )}
\end{itemize}
Parte superior do abdome, entre os dois hypocôndrios.
\section{Epigeia}
\begin{itemize}
\item {Grp. gram.:f.}
\end{itemize}
\begin{itemize}
\item {Proveniência:(Do gr. \textunderscore epi\textunderscore  + \textunderscore ge\textunderscore )}
\end{itemize}
Arbusto da América do Norte, espécie de urze.
\section{Epigeico}
\begin{itemize}
\item {Grp. gram.:adj.}
\end{itemize}
\begin{itemize}
\item {Utilização:Geol.}
\end{itemize}
Diz-se dos depósitos, ormados recentemente.
(Cp. \textunderscore epígeo\textunderscore )
\section{Epigenesia}
\begin{itemize}
\item {Grp. gram.:f.}
\end{itemize}
\begin{itemize}
\item {Proveniência:(Do gr. \textunderscore epi\textunderscore  + \textunderscore genesis\textunderscore )}
\end{itemize}
Theoria da formação dos seres orgânicos por gerações graduaes.
\section{Epigenesista}
\begin{itemize}
\item {Grp. gram.:m.}
\end{itemize}
Partidário da epigenesia.
\section{Epigenia}
\begin{itemize}
\item {Grp. gram.:f.}
\end{itemize}
Phenómeno, que se dá, quando um crystal muda de natureza chimica, sem mudar de fórma.
(Do \textunderscore epígeno\textunderscore )
\section{Epígeno}
\begin{itemize}
\item {Grp. gram.:adj.}
\end{itemize}
\begin{itemize}
\item {Proveniência:(Do gr. \textunderscore epi\textunderscore  + \textunderscore genos\textunderscore )}
\end{itemize}
Que apresenta o phenómeno da epigenia.
\section{Epígeno}
\begin{itemize}
\item {Grp. gram.:adj.}
\end{itemize}
\begin{itemize}
\item {Utilização:Bot.}
\end{itemize}
\begin{itemize}
\item {Proveniência:(Do gr. \textunderscore epi\textunderscore  + \textunderscore gune\textunderscore )}
\end{itemize}
Diz-se do estame, do disco, da corola ou de qualquer órgão vegetal, que está sôbre o ovário ou acima dele.
\section{Epígeo}
\begin{itemize}
\item {Grp. gram.:adj.}
\end{itemize}
\begin{itemize}
\item {Utilização:Bot.}
\end{itemize}
\begin{itemize}
\item {Proveniência:(Do gr. \textunderscore epi\textunderscore  + \textunderscore ge\textunderscore )}
\end{itemize}
Que está sôbre a terra ou fóra della.
Diz-se dos cotylédones, quando, durante a germinação, são arrastados pelo caulículo para debaixo da terra, como succede com os do feijão.
\section{Epiginia}
\begin{itemize}
\item {Grp. gram.:f.}
\end{itemize}
\begin{itemize}
\item {Utilização:Bot.}
\end{itemize}
Qualidade do órgão vegetal que é epígino.
\section{Epiginofórico}
\begin{itemize}
\item {Grp. gram.:adj.}
\end{itemize}
\begin{itemize}
\item {Utilização:Bot.}
\end{itemize}
\begin{itemize}
\item {Proveniência:(De \textunderscore epi...\textunderscore  + \textunderscore gynóphoro\textunderscore )}
\end{itemize}
Diz-se do nectário, quando fica por baixo do ovario.
\section{Epiglossa}
\begin{itemize}
\item {Grp. gram.:f.}
\end{itemize}
\begin{itemize}
\item {Proveniência:(Do gr. \textunderscore epi\textunderscore  + \textunderscore glossa\textunderscore )}
\end{itemize}
Parte da bôca dos insectos hymenópteros.
\section{Epiglote}
\begin{itemize}
\item {Grp. gram.:f.}
\end{itemize}
\begin{itemize}
\item {Utilização:Anat.}
\end{itemize}
\begin{itemize}
\item {Proveniência:(Lat. \textunderscore epiglottis\textunderscore )}
\end{itemize}
Válvula fibro-cartilaginosa, que, tapando a abertura da glote no momento da deglutição, impede a entrada dos alimentos e das bebidas na laringe.
\section{Epiglótico}
\begin{itemize}
\item {Grp. gram.:adj.}
\end{itemize}
Relativo a epiglote.
\section{Epiglotite}
\begin{itemize}
\item {Grp. gram.:f.}
\end{itemize}
Inflamação da epiglote.
\section{Epiglotte}
\begin{itemize}
\item {Grp. gram.:f.}
\end{itemize}
\begin{itemize}
\item {Utilização:Anat.}
\end{itemize}
\begin{itemize}
\item {Proveniência:(Lat. \textunderscore epiglottis\textunderscore )}
\end{itemize}
Válvula fibro-cartilaginosa, que, tapando a abertura da glotte no momento da deglutição, impede a entrada dos alimentos e das bebidas na larynge.
\section{Epiglóttico}
\begin{itemize}
\item {Grp. gram.:adj.}
\end{itemize}
Relativo a epiglotte.
\section{Epiglottite}
\begin{itemize}
\item {Grp. gram.:f.}
\end{itemize}
Inflammação da epiglotte.
\section{Epígono}
\begin{itemize}
\item {Grp. gram.:m.  e  adj.}
\end{itemize}
\begin{itemize}
\item {Utilização:P. us.}
\end{itemize}
\begin{itemize}
\item {Proveniência:(Lat. \textunderscore epigoni\textunderscore )}
\end{itemize}
O que nasceu depois; descendente.
\section{Epigrafar}
\begin{itemize}
\item {Grp. gram.:v. t.}
\end{itemize}
Pôr epígrafe em.
Intitular; inscrever.
\section{Epígrafe}
\begin{itemize}
\item {Grp. gram.:f.}
\end{itemize}
\begin{itemize}
\item {Proveniência:(Do gr. \textunderscore epi\textunderscore  + \textunderscore graphein\textunderscore )}
\end{itemize}
Inscrição em lugar alto.
Título de um escrito.
Sentença, posta no frontispício de um capítulo ou de qualquer escrito.
\section{Epigrafia}
\begin{itemize}
\item {Grp. gram.:f.}
\end{itemize}
\begin{itemize}
\item {Proveniência:(De \textunderscore epígrafe\textunderscore )}
\end{itemize}
Ciência, que se ocupa das inscrições.
O mesmo que \textunderscore inscrição\textunderscore :«\textunderscore ...lêr-se na epigrafia do monumento.\textunderscore »Latino, \textunderscore Hist. Pol.\textunderscore , 169.
\section{Epigráfico}
\begin{itemize}
\item {Grp. gram.:adj.}
\end{itemize}
Concernente a epigrafia.
\section{Epigrafo}
\begin{itemize}
\item {Grp. gram.:m.}
\end{itemize}
\begin{itemize}
\item {Proveniência:(Gr. \textunderscore epigraphos\textunderscore )}
\end{itemize}
Funcionário, que, entre os Atenienses, tinha a seu cargo a contabilidade das contribuições.
\section{Epigrama}
\begin{itemize}
\item {Grp. gram.:m.}
\end{itemize}
\begin{itemize}
\item {Proveniência:(Lat. \textunderscore epigramma\textunderscore )}
\end{itemize}
Antigamente, qualquer composição poética de curtas dimensões.
Pequena composição poética, que termina por um pensamento engenhoso ou satírico.
Sátira; dito mordaz; referência crítica.
\section{Epigramaticamente}
\begin{itemize}
\item {Grp. gram.:adv.}
\end{itemize}
De modo epigramático.
\section{Epigramático}
\begin{itemize}
\item {Grp. gram.:adj.}
\end{itemize}
Que contém epigrama.
Relativo a epigrama.
\section{Epigramatista}
\begin{itemize}
\item {Grp. gram.:m.}
\end{itemize}
Aquele que faz epigramas.
\section{Epigramatizar}
\begin{itemize}
\item {Grp. gram.:v. t.}
\end{itemize}
\begin{itemize}
\item {Grp. gram.:V. i.}
\end{itemize}
Dirigir epigramas a.
Satirizar.
Fazer epigramas.
\section{Epigramma}
\begin{itemize}
\item {Grp. gram.:m.}
\end{itemize}
\begin{itemize}
\item {Proveniência:(Lat. \textunderscore epigramma\textunderscore )}
\end{itemize}
Antigamente, qualquer composição poética de curtas dimensões.
Pequena composição poética, que termina por um pensamento engenhoso ou satírico.
Sátira; dito mordaz; referência crítica.
\section{Epigrammaticamente}
\begin{itemize}
\item {Grp. gram.:adv.}
\end{itemize}
De modo epigrammático.
\section{Epigrammático}
\begin{itemize}
\item {Grp. gram.:adj.}
\end{itemize}
Que contém epigramma.
Relativo a epigramma.
\section{Epigrammatista}
\begin{itemize}
\item {Grp. gram.:m.}
\end{itemize}
Aquelle que faz epigrammas.
\section{Epigrammatizar}
\begin{itemize}
\item {Grp. gram.:v. t.}
\end{itemize}
\begin{itemize}
\item {Grp. gram.:V. i.}
\end{itemize}
Dirigir epigrammas a.
Satirizar.
Fazer epigrammas.
\section{Epigraphar}
\begin{itemize}
\item {Grp. gram.:v. t.}
\end{itemize}
Pôr epígraphe em.
Intitular; inscrever.
\section{Epígraphe}
\begin{itemize}
\item {Grp. gram.:f.}
\end{itemize}
\begin{itemize}
\item {Proveniência:(Do gr. \textunderscore epi\textunderscore  + \textunderscore graphein\textunderscore )}
\end{itemize}
Inscripção em lugar alto.
Título de um escrito.
Sentença, posta no frontispício de um capítulo ou de qualquer escrito.
\section{Epigraphia}
\begin{itemize}
\item {Grp. gram.:f.}
\end{itemize}
\begin{itemize}
\item {Proveniência:(De \textunderscore epígraphe\textunderscore )}
\end{itemize}
Sciência, que se occupa das inscrições.
O mesmo que \textunderscore inscripção\textunderscore :«\textunderscore ...lêr-se na epigraphia do monumento.\textunderscore »Latino, \textunderscore Hist. Pol.\textunderscore , 169.
\section{Epigráphico}
\begin{itemize}
\item {Grp. gram.:adj.}
\end{itemize}
Concernente a epigraphia.
\section{Epígrapho}
\begin{itemize}
\item {Grp. gram.:m.}
\end{itemize}
\begin{itemize}
\item {Proveniência:(Gr. \textunderscore epigraphos\textunderscore )}
\end{itemize}
Funccionário, que, entre os Athenienses, tinha a seu cargo a contabilidade das contribuições.
\section{Epigynia}
\begin{itemize}
\item {Grp. gram.:f.}
\end{itemize}
\begin{itemize}
\item {Utilização:Bot.}
\end{itemize}
Qualidade do órgão vegetal que é epígyno.
\section{Epígyno}
\begin{itemize}
\item {Grp. gram.:adj.}
\end{itemize}
\begin{itemize}
\item {Utilização:Bot.}
\end{itemize}
\begin{itemize}
\item {Proveniência:(Do gr. \textunderscore epi\textunderscore  + \textunderscore gune\textunderscore )}
\end{itemize}
Diz-se do estame, do disco, da corolla ou de qualquer órgão vegetal, que está sôbre o ovário ou acima delle.
\section{Epigynophórico}
\begin{itemize}
\item {Grp. gram.:adj.}
\end{itemize}
\begin{itemize}
\item {Utilização:Bot.}
\end{itemize}
\begin{itemize}
\item {Proveniência:(De \textunderscore epi...\textunderscore  + \textunderscore gynóphoro\textunderscore )}
\end{itemize}
Diz-se do nectário, quando fica por baixo do ovario.
\section{Epilação}
\begin{itemize}
\item {Grp. gram.:f.}
\end{itemize}
\begin{itemize}
\item {Proveniência:(Do lat. \textunderscore e\textunderscore  + \textunderscore pilare\textunderscore )}
\end{itemize}
Acto de arrancar pêlos.
Acção de arrancar os cabellos, para curar certas enfermidades do coiro cabelludo.
\section{Epilatório}
\begin{itemize}
\item {Grp. gram.:adj.}
\end{itemize}
Que faz cair o cabello.
(Cp. \textunderscore epilação\textunderscore )
\section{Epilampo}
\begin{itemize}
\item {Grp. gram.:m.}
\end{itemize}
\begin{itemize}
\item {Proveniência:(Do gr. \textunderscore epi\textunderscore  + \textunderscore lampos\textunderscore )}
\end{itemize}
Insecto coleóptero heterâmero.
\section{Epilepsia}
\begin{itemize}
\item {Grp. gram.:f.}
\end{itemize}
\begin{itemize}
\item {Proveniência:(Lat. \textunderscore epilepsia\textunderscore )}
\end{itemize}
Doença cerebral, caracterizada por convulsões e pela perda dos sentidos.
\section{Epiléptico}
\begin{itemize}
\item {Grp. gram.:adj.}
\end{itemize}
\begin{itemize}
\item {Proveniência:(Gr. \textunderscore epileptikos\textunderscore )}
\end{itemize}
Relativo a epilepsia.
Que padece epilepsia.
\section{Epileptiforme}
\begin{itemize}
\item {Grp. gram.:adj.}
\end{itemize}
\begin{itemize}
\item {Proveniência:(Do gr. \textunderscore epileptikos\textunderscore  + lat. \textunderscore forma\textunderscore )}
\end{itemize}
Semelhante á epilepsia.
\section{Epileptoide}
\begin{itemize}
\item {Grp. gram.:adj.}
\end{itemize}
O mesmo que \textunderscore epileptiforme\textunderscore .
\section{Epilóbeas}
\begin{itemize}
\item {Grp. gram.:f. pl.}
\end{itemize}
O mesmo que \textunderscore epilobeáceas\textunderscore .
\section{Epilobeáceas}
\begin{itemize}
\item {Grp. gram.:f. pl.}
\end{itemize}
O mesmo que \textunderscore onagrárias\textunderscore .
\section{Epilóbio}
\begin{itemize}
\item {Grp. gram.:m.}
\end{itemize}
\begin{itemize}
\item {Proveniência:(Do gr. \textunderscore epi\textunderscore  + \textunderscore lobos\textunderscore )}
\end{itemize}
Planta herbácea dicotyledónea e polypétala.
\section{Epílobo}
\begin{itemize}
\item {Grp. gram.:m.}
\end{itemize}
\begin{itemize}
\item {Proveniência:(Do gr. \textunderscore epi\textunderscore  + \textunderscore lobos\textunderscore )}
\end{itemize}
Planta herbácea dicotyledónea e polypétala.
\section{Epilogação}
\begin{itemize}
\item {Grp. gram.:f.}
\end{itemize}
Acto ou effeito de epilogar.
\section{Epilogador}
\begin{itemize}
\item {Grp. gram.:m.}
\end{itemize}
Aquelle que epiloga.
\section{Epilogar}
\begin{itemize}
\item {Grp. gram.:v. t.}
\end{itemize}
Reduzir a epilogo.
Resumir, recapitulando.
Concluir.
\section{Epílogo}
\begin{itemize}
\item {Grp. gram.:m.}
\end{itemize}
\begin{itemize}
\item {Proveniência:(Lat. \textunderscore epilogus\textunderscore )}
\end{itemize}
Conclusão, em que se recapitula ou se resume o que se disse.
Resumo.
Remate; fecho.
\section{Epimédio}
\begin{itemize}
\item {Grp. gram.:m.}
\end{itemize}
\begin{itemize}
\item {Proveniência:(Gr. \textunderscore epimedíon\textunderscore )}
\end{itemize}
Planta vivaz de regiões montanhosas, também conhecida por \textunderscore erva de bèsteiros\textunderscore .
\section{Epimênias}
\begin{itemize}
\item {Grp. gram.:m. pl.}
\end{itemize}
\begin{itemize}
\item {Proveniência:(Lat. \textunderscore epimenía\textunderscore )}
\end{itemize}
Presentes, que, entre os Romanos, se faziam todos os meses.
\section{Epímeno}
\begin{itemize}
\item {Grp. gram.:adj.}
\end{itemize}
O mesmo que \textunderscore epígyno\textunderscore .
\section{Epímetro}
\begin{itemize}
\item {Grp. gram.:m.}
\end{itemize}
\begin{itemize}
\item {Utilização:Ant.}
\end{itemize}
\begin{itemize}
\item {Utilização:Bot.}
\end{itemize}
\begin{itemize}
\item {Proveniência:(Gr. \textunderscore epimetron\textunderscore )}
\end{itemize}
Parte de uma carregação marítima, que se entregava ao piloto como salário, entre os Gregos.
Addição aos impostos, cobrada pelos recebedores romanos, no tempo do Império, para indemnização do seu trabalho.
Membrana, que cérca de um só lado o ovário de certas plantas.
\section{Epinema}
\begin{itemize}
\item {Grp. gram.:m.}
\end{itemize}
\begin{itemize}
\item {Utilização:Bot.}
\end{itemize}
\begin{itemize}
\item {Proveniência:(Do gr. \textunderscore epi\textunderscore  + \textunderscore nema\textunderscore , fio)}
\end{itemize}
Parte superior do filete nos estames das plantas, que dão flôres synanthéreas.
\section{Epinício}
\begin{itemize}
\item {Grp. gram.:m.}
\end{itemize}
\begin{itemize}
\item {Proveniência:(Lat. \textunderscore epinicium\textunderscore )}
\end{itemize}
Poema ou canto, em que se celebra uma victória.
\section{Epiódia}
\begin{itemize}
\item {Grp. gram.:f.}
\end{itemize}
\begin{itemize}
\item {Proveniência:(Do gr. \textunderscore epi\textunderscore  + \textunderscore oide\textunderscore )}
\end{itemize}
Marcha fúnebre, entre os Gregos.
\section{Epiodonte}
\begin{itemize}
\item {Grp. gram.:m.}
\end{itemize}
Mamifero do Mediterráneo, espécie de golfinho.
\section{Epioolíthico}
\begin{itemize}
\item {Grp. gram.:adj.}
\end{itemize}
\begin{itemize}
\item {Utilização:Geol.}
\end{itemize}
\begin{itemize}
\item {Proveniência:(De \textunderscore epi...\textunderscore  + \textunderscore oolíthico\textunderscore )}
\end{itemize}
Que é de formação posterior á do terreno oolíthico.
\section{Epioolítico}
\begin{itemize}
\item {Grp. gram.:adj.}
\end{itemize}
\begin{itemize}
\item {Utilização:Geol.}
\end{itemize}
\begin{itemize}
\item {Proveniência:(De \textunderscore epi...\textunderscore  + \textunderscore oolítico\textunderscore )}
\end{itemize}
Que é de formação posterior á do terreno oolítico.
\section{Epiórnis}
\begin{itemize}
\item {Grp. gram.:m.}
\end{itemize}
\begin{itemize}
\item {Utilização:Geol.}
\end{itemize}
\begin{itemize}
\item {Proveniência:(Do gr. \textunderscore epi\textunderscore  + \textunderscore ornis\textunderscore , ave)}
\end{itemize}
Terreno contemporâneo do homem.
Grande e antiga ave de Madagascar, de que só se descobriram ovos petrificados, que tinham cada um a capacidade de oito litros aproximadamente.
\section{Epiparoxismo}
\begin{itemize}
\item {Grp. gram.:m.}
\end{itemize}
\begin{itemize}
\item {Utilização:Med.}
\end{itemize}
\begin{itemize}
\item {Proveniência:(De \textunderscore epi...\textunderscore  + \textunderscore paroxismo\textunderscore )}
\end{itemize}
Paroxismo, que reaparece mais cedo ou mais frequentemente do que deve voltar.
\section{Epiparoxysmo}
\begin{itemize}
\item {Grp. gram.:m.}
\end{itemize}
\begin{itemize}
\item {Utilização:Med.}
\end{itemize}
\begin{itemize}
\item {Proveniência:(De \textunderscore epi...\textunderscore  + \textunderscore paroxysmo\textunderscore )}
\end{itemize}
Paroxysmo, que reapparece mais cedo ou mais frequentemente do que deve voltar.
\section{Epípedo}
\begin{itemize}
\item {Grp. gram.:m.}
\end{itemize}
\begin{itemize}
\item {Proveniência:(Do gr. \textunderscore epi\textunderscore  + \textunderscore pedos\textunderscore )}
\end{itemize}
Insecto coleóptero tetrâmero da Guiana.
\section{Epipetália}
\begin{itemize}
\item {Grp. gram.:f.}
\end{itemize}
\begin{itemize}
\item {Utilização:Bot.}
\end{itemize}
\begin{itemize}
\item {Proveniência:(De \textunderscore epipétalo\textunderscore )}
\end{itemize}
Duodêcima classe de vegetaes, no systema de Jussieu.
\section{Epipétalo}
\begin{itemize}
\item {Grp. gram.:adj.}
\end{itemize}
\begin{itemize}
\item {Utilização:Bot.}
\end{itemize}
\begin{itemize}
\item {Proveniência:(De \textunderscore epi...\textunderscore  + \textunderscore pétala\textunderscore )}
\end{itemize}
Diz-se dos estames, que nascem sôbre a corolla.
\section{Epípetro}
\begin{itemize}
\item {Grp. gram.:m.}
\end{itemize}
\begin{itemize}
\item {Proveniência:(Do gr. \textunderscore epi\textunderscore  + \textunderscore petra\textunderscore )}
\end{itemize}
Gênero de polypeiros.
\section{Epiphania}
\begin{itemize}
\item {Grp. gram.:f.}
\end{itemize}
\begin{itemize}
\item {Proveniência:(Lat. \textunderscore epiphania\textunderscore )}
\end{itemize}
Festividade religiosa, em memória da manifestação de Christo aos gentios.
Dia de Reis.
\section{Epipharynge}
\begin{itemize}
\item {Grp. gram.:f.}
\end{itemize}
\begin{itemize}
\item {Proveniência:(De \textunderscore epi...\textunderscore  + \textunderscore pharynge\textunderscore )}
\end{itemize}
Uma das peças da bôca dos insectos hymenópteros.
\section{Epiphenómeno}
\begin{itemize}
\item {Grp. gram.:m.}
\end{itemize}
\begin{itemize}
\item {Utilização:Med.}
\end{itemize}
\begin{itemize}
\item {Proveniência:(De \textunderscore epi...\textunderscore  + \textunderscore phenómeno\textunderscore )}
\end{itemize}
Symptoma superveniente, numa doença já declarada.
\section{Epiphéria}
\begin{itemize}
\item {Grp. gram.:f.}
\end{itemize}
Espécie de escudo antigo?:«\textunderscore as (testugens) ...eram mais levantadas..., ao revés de epiphérias, que hoje usamos\textunderscore ». \textunderscore Viriato Trág.\textunderscore , II, XVIII. Cf. \textunderscore Idem\textunderscore , XIV, 85.
\section{Epiphleose}
\begin{itemize}
\item {Grp. gram.:f.}
\end{itemize}
\begin{itemize}
\item {Utilização:Bot.}
\end{itemize}
\begin{itemize}
\item {Proveniência:(Do gr. \textunderscore epi\textunderscore  + \textunderscore phloios\textunderscore )}
\end{itemize}
A epiderme dos vegetaes.
\section{Epiphonema}
\begin{itemize}
\item {Grp. gram.:m.}
\end{itemize}
\begin{itemize}
\item {Proveniência:(Lat. \textunderscore epiphonema\textunderscore )}
\end{itemize}
Exclamação sentencíosa, com que se termina uma narrativa ou um discurso.
\section{Epiphonêmico}
\begin{itemize}
\item {Grp. gram.:adj.}
\end{itemize}
Relativo a epiphonema.
Em que há epiphonema.
\section{Epíphora}
\begin{itemize}
\item {Grp. gram.:m.  e  f.}
\end{itemize}
\begin{itemize}
\item {Utilização:Med.}
\end{itemize}
\begin{itemize}
\item {Utilização:Rhet.}
\end{itemize}
\begin{itemize}
\item {Proveniência:(Lat. \textunderscore epiphora\textunderscore )}
\end{itemize}
Fluxo de lágrimas, constante e involuntário, produzido por doença que obstruiu as vias lacrimaes.
Repetição de uma ou mais palavras, no fim de cada membro de um periódo.
\section{Epiphragma}
\begin{itemize}
\item {Grp. gram.:m.}
\end{itemize}
\begin{itemize}
\item {Utilização:Zool.}
\end{itemize}
\begin{itemize}
\item {Proveniência:(Do gr. \textunderscore epi\textunderscore  + \textunderscore phragma\textunderscore )}
\end{itemize}
Opérculo temporário na concha de alguns mollucos.
\section{Epiphragmático}
\begin{itemize}
\item {Grp. gram.:adj.}
\end{itemize}
Relativo a epiphragma.
\section{Epíphrase}
\begin{itemize}
\item {Grp. gram.:f.}
\end{itemize}
\begin{itemize}
\item {Utilização:Rhet.}
\end{itemize}
\begin{itemize}
\item {Proveniência:(De \textunderscore epi...\textunderscore  + \textunderscore phrase\textunderscore )}
\end{itemize}
Accrescentamento a uma phrase, que parecia concluida, para se desenvolverem ideias acessórias.
\section{Epiphyllo}
\begin{itemize}
\item {Grp. gram.:adj.}
\end{itemize}
\begin{itemize}
\item {Utilização:Bot.}
\end{itemize}
\begin{itemize}
\item {Proveniência:(Do gr. \textunderscore epi\textunderscore  + \textunderscore phullon\textunderscore )}
\end{itemize}
Diz-se dos órgãos vegetaes, que crescem ou estão inseridos nas fôlhas das plantas.
\section{Epiphyllospérmeas}
\begin{itemize}
\item {Grp. gram.:f. pl.}
\end{itemize}
\begin{itemize}
\item {Utilização:Bot.}
\end{itemize}
\begin{itemize}
\item {Proveniência:(Do gr. \textunderscore epi\textunderscore  + \textunderscore phullon\textunderscore  + \textunderscore sperma\textunderscore )}
\end{itemize}
Fêtos, cujas fructificações ficam sôbre o dorso das fôlhas.
\section{Epiphysário}
\begin{itemize}
\item {Grp. gram.:adj.}
\end{itemize}
Relativo a epíphyse.
Que tem epíphyse.
\section{Epíphyse}
\begin{itemize}
\item {Grp. gram.:f.}
\end{itemize}
\begin{itemize}
\item {Proveniência:(Do gr. \textunderscore epi\textunderscore  + \textunderscore phusis\textunderscore )}
\end{itemize}
Saliência óssea, que, unida a um osso por uma cartilagem, se converte em apóphyse, pelo desenvolvimento da ossificação.
\section{Epiphytia}
\begin{itemize}
\item {Grp. gram.:f.}
\end{itemize}
Doença, que ataca ao mesmo tempo grande número de plantas.
Qualidade de epíphyto.
\section{Epíphyto}
\begin{itemize}
\item {Grp. gram.:adj}
\end{itemize}
\begin{itemize}
\item {Proveniência:(Do gr. \textunderscore epi\textunderscore  + \textunderscore phuton\textunderscore )}
\end{itemize}
Diz-se das plantas que crescem sôbre outras, sem se alimentarem da substância destas.
\section{Epipião}
\begin{itemize}
\item {Grp. gram.:m.}
\end{itemize}
Pequena moéda portuguesa do tempo de Sancho I e equivalente a duas mealhas.
\section{Epipigma}
\begin{itemize}
\item {Grp. gram.:m.}
\end{itemize}
\begin{itemize}
\item {Proveniência:(Do gr. \textunderscore epi\textunderscore  + \textunderscore pigma\textunderscore )}
\end{itemize}
Antigo instrumento cirúrgico, para reduzir as luxações do braço.
\section{Epiplocele}
\begin{itemize}
\item {Grp. gram.:f.}
\end{itemize}
\begin{itemize}
\item {Proveniência:(Do gr. \textunderscore epiloon\textunderscore  + \textunderscore kele\textunderscore )}
\end{itemize}
Hérnia decaída do epíploon na bolsa dos testículos.
\section{Epiploíte}
\begin{itemize}
\item {Grp. gram.:f.}
\end{itemize}
Inflammação do epíploon.
\section{Epíploon}
\begin{itemize}
\item {Grp. gram.:m.}
\end{itemize}
\begin{itemize}
\item {Utilização:Anat.}
\end{itemize}
\begin{itemize}
\item {Proveniência:(Gr. \textunderscore epiploon\textunderscore )}
\end{itemize}
Dobra do peritonéu, que cobre os intestinos.
\section{Epípode}
\begin{itemize}
\item {Grp. gram.:m.}
\end{itemize}
\begin{itemize}
\item {Utilização:Bot.}
\end{itemize}
\begin{itemize}
\item {Proveniência:(Do gr. \textunderscore epi\textunderscore  + \textunderscore pous\textunderscore )}
\end{itemize}
Tubérculo, que nasce no cimo de pedúnculo de certas plantas, perto do ovário.
\section{Epipogão}
\begin{itemize}
\item {Grp. gram.:m.}
\end{itemize}
\begin{itemize}
\item {Proveniência:(Do gr. \textunderscore epi\textunderscore  + \textunderscore pogon\textunderscore )}
\end{itemize}
Espécie de orchídea, que cresce nos Alpes.
\section{Epipólase}
\begin{itemize}
\item {Grp. gram.:f.}
\end{itemize}
\begin{itemize}
\item {Proveniência:(Gr. \textunderscore epipolasis\textunderscore )}
\end{itemize}
Na Chímica ant., chamava-se assim a acção, pela qual uma substância se separa de um líquido, ficando á superfície, sem se volatizar.
\section{Epipólico}
\begin{itemize}
\item {Grp. gram.:adj.}
\end{itemize}
Relativo á epipólase.
\section{Epipolismo}
\begin{itemize}
\item {Grp. gram.:m.}
\end{itemize}
Desenvolvimento de força epipólica.
\section{Epipterado}
\begin{itemize}
\item {Grp. gram.:adj.}
\end{itemize}
\begin{itemize}
\item {Utilização:Bot.}
\end{itemize}
\begin{itemize}
\item {Proveniência:(Do gr. \textunderscore epi\textunderscore  + \textunderscore pteron\textunderscore )}
\end{itemize}
Diz-se do fruto ou do grão, quando provido de uma espécie de asa no seu ápice.
\section{Epiqueia}
\begin{itemize}
\item {Grp. gram.:f.}
\end{itemize}
\begin{itemize}
\item {Utilização:Fig.}
\end{itemize}
\begin{itemize}
\item {Proveniência:(Do gr. \textunderscore epikhein\textunderscore )}
\end{itemize}
Razoável interpretação de uma lei ou preceito.
Moderação, meio termo.
\section{Epiquirema}
\begin{itemize}
\item {Grp. gram.:m.}
\end{itemize}
\begin{itemize}
\item {Proveniência:(Lat. \textunderscore epichirema\textunderscore )}
\end{itemize}
Silogismo, em que as premissas ou uma delas são acompanhadas de prova.
\section{Epiquiremático}
\begin{itemize}
\item {Grp. gram.:adj.}
\end{itemize}
Relativo a epiquirema.
\section{Epirota}
\begin{itemize}
\item {Grp. gram.:m.}
\end{itemize}
\begin{itemize}
\item {Proveniência:(Lat. \textunderscore epirota\textunderscore )}
\end{itemize}
Habitante do Epiro.
\section{Episcênias}
\begin{itemize}
\item {Grp. gram.:f. pl.}
\end{itemize}
\begin{itemize}
\item {Proveniência:(Do gr. \textunderscore epi\textunderscore  + \textunderscore skene\textunderscore )}
\end{itemize}
Festas dos pavilhões, entre os Espartanos.
Festa judaica, mais conhecida por festa dos tabernáculos.
\section{Episcênio}
\begin{itemize}
\item {Grp. gram.:m.}
\end{itemize}
\begin{itemize}
\item {Proveniência:(Lat. \textunderscore episcenium\textunderscore )}
\end{itemize}
Primeira e segunda ordem de palanques ou varandins, nos theatros gregos.
\section{Epíscio}
\begin{itemize}
\item {Grp. gram.:m.}
\end{itemize}
\begin{itemize}
\item {Proveniência:(Gr. \textunderscore episkios\textunderscore )}
\end{itemize}
Insecto hemíptero do Brasil.
\section{Episclerite}
\begin{itemize}
\item {Grp. gram.:f.}
\end{itemize}
\begin{itemize}
\item {Utilização:Med.}
\end{itemize}
\begin{itemize}
\item {Proveniência:(Do gr. \textunderscore épi\textunderscore  + \textunderscore skleros\textunderscore )}
\end{itemize}
Inflammação do tecído sobreposto da esclerótica.
\section{Episcopado}
\begin{itemize}
\item {Grp. gram.:m.}
\end{itemize}
\begin{itemize}
\item {Proveniência:(Lat. \textunderscore episcopatus\textunderscore )}
\end{itemize}
Dignidade do Bispo; bispado.
\section{Episcopal}
\begin{itemize}
\item {Grp. gram.:adj.}
\end{itemize}
\begin{itemize}
\item {Proveniência:(Lat. \textunderscore episcopalis\textunderscore )}
\end{itemize}
Relativo a Bispo: o \textunderscore báculo episcopal\textunderscore .
\section{Episcopisa}
\begin{itemize}
\item {Grp. gram.:f.}
\end{itemize}
Mulher que, nos principios do Christianismo, desempenhava certas funcções sacerdotaes, sem jurisdicção episcopal.
(Cp. \textunderscore epíscopo\textunderscore )
\section{Epíscopo}
\begin{itemize}
\item {Grp. gram.:m.}
\end{itemize}
\begin{itemize}
\item {Proveniência:(Lat. episkopus)}
\end{itemize}
Antigo magistrado nas colónias Gregas.
Magistrado romano, inspector de uma circunscripção territorial, chamada diocese.
\section{Episemo}
\begin{itemize}
\item {Grp. gram.:m.}
\end{itemize}
\begin{itemize}
\item {Proveniência:(Gr. \textunderscore episemon\textunderscore )}
\end{itemize}
Cada um dos três caracteres, que, estranhos ao alphabeto grego, eram pelos Gregos empregados em a numeração escrita.
\section{Episépalo}
\begin{itemize}
\item {fónica:sé}
\end{itemize}
\begin{itemize}
\item {Grp. gram.:adj.}
\end{itemize}
\begin{itemize}
\item {Utilização:Bot.}
\end{itemize}
\begin{itemize}
\item {Proveniência:(De \textunderscore epi...\textunderscore  + \textunderscore sépala\textunderscore )}
\end{itemize}
Que nasce ou cresce sôbre as sépalas do cálice.
\section{Episferia}
\begin{itemize}
\item {Grp. gram.:f.}
\end{itemize}
\begin{itemize}
\item {Proveniência:(Do gr. \textunderscore epi\textunderscore  + \textunderscore sphaira\textunderscore )}
\end{itemize}
Conjunto das sinuosidades exteriores do cérebro.
\section{Episodiador}
\begin{itemize}
\item {Grp. gram.:m.}
\end{itemize}
Aquelle que episodia.
\section{Episodiar}
\begin{itemize}
\item {Grp. gram.:v. t.}
\end{itemize}
Adornar com episódios.
Tratar como episódio.
\section{Episodicamente}
\begin{itemize}
\item {Grp. gram.:adj.}
\end{itemize}
\begin{itemize}
\item {Proveniência:(De \textunderscore episódico\textunderscore )}
\end{itemize}
Em fórma de episódio.
\section{Episódico}
\begin{itemize}
\item {Grp. gram.:adj.}
\end{itemize}
Relativo a episódio.
Accessório, secundario.
Adiáphoro.
\section{Episódio}
\begin{itemize}
\item {Grp. gram.:m.}
\end{itemize}
\begin{itemize}
\item {Proveniência:(Gr. \textunderscore epeisodos\textunderscore )}
\end{itemize}
Incidente, que tem relação com a acção principal de uma narrativa ou de uma obra literária ou artística.
Accessório.
Facto notável, relacionado com uma série de outros factos também notáveis.
\section{Epispádias}
\begin{itemize}
\item {Grp. gram.:f. pl.}
\end{itemize}
\begin{itemize}
\item {Utilização:Med.}
\end{itemize}
\begin{itemize}
\item {Proveniência:(Do gr. \textunderscore spi\textunderscore  + \textunderscore spas\textunderscore )}
\end{itemize}
Deformação, produzida pela abertura da urethra, no dorso do pênis.
\section{Epíspase}
\begin{itemize}
\item {Grp. gram.:f.}
\end{itemize}
\begin{itemize}
\item {Utilização:Med.}
\end{itemize}
Erupção local, determinada por um tratamento e indicativa de uma modificação geral no organismo.
(Cp. \textunderscore epispástico\textunderscore )
\section{Epispástico}
\begin{itemize}
\item {Grp. gram.:adj.}
\end{itemize}
\begin{itemize}
\item {Proveniência:(Gr. \textunderscore epispastíkos\textunderscore )}
\end{itemize}
Que irrita a pelle, empolando a epiderme.
\section{Epispásticos}
\begin{itemize}
\item {Grp. gram.:m. pl.}
\end{itemize}
\begin{itemize}
\item {Utilização:Zool.}
\end{itemize}
\begin{itemize}
\item {Proveniência:(De \textunderscore epispástico\textunderscore )}
\end{itemize}
Família de insectos coleópteros.
\section{Episperma}
\begin{itemize}
\item {Grp. gram.:m.}
\end{itemize}
\begin{itemize}
\item {Utilização:Bot.}
\end{itemize}
\begin{itemize}
\item {Proveniência:(Do gr. \textunderscore epi\textunderscore  + \textunderscore sperma\textunderscore )}
\end{itemize}
Pellícula das sementes ou dos grãos.
\section{Epispermático}
\begin{itemize}
\item {Grp. gram.:adj.}
\end{itemize}
Relativo a episperma.
\section{Epispheria}
\begin{itemize}
\item {Grp. gram.:f.}
\end{itemize}
\begin{itemize}
\item {Proveniência:(Do gr. \textunderscore epi\textunderscore  + \textunderscore sphaira\textunderscore )}
\end{itemize}
Conjunto das sinuosidades exteriores do cérebro.
\section{Epissenagia}
\begin{itemize}
\item {Grp. gram.:f.}
\end{itemize}
Linha de 16 ecatontarchias, adjunta á phalange macedónica.
\section{Epissépalo}
\begin{itemize}
\item {Grp. gram.:adj.}
\end{itemize}
\begin{itemize}
\item {Utilização:Bot.}
\end{itemize}
\begin{itemize}
\item {Proveniência:(De \textunderscore epi...\textunderscore  + \textunderscore sépala\textunderscore )}
\end{itemize}
Que nasce ou cresce sôbre as sépalas do cálice.
\section{Epistação}
\begin{itemize}
\item {Grp. gram.:f.}
\end{itemize}
Acto de epistar.
\section{Epissilogismo}
\begin{itemize}
\item {Grp. gram.:m.}
\end{itemize}
\begin{itemize}
\item {Proveniência:(De \textunderscore epi...\textunderscore  + \textunderscore silogismo\textunderscore )}
\end{itemize}
Raciocínio, que faz parte de uma série polissilogística e que, para uma das suas premissas, toma a conclusão de um raciocínio precedente.
\section{Epissintético}
\begin{itemize}
\item {Grp. gram.:adj.}
\end{itemize}
\begin{itemize}
\item {Grp. gram.:M. pl.}
\end{itemize}
\begin{itemize}
\item {Proveniência:(Gr. \textunderscore episunthetikos\textunderscore )}
\end{itemize}
Relativo ao epissintétismo.
Seita medicinal dos que procuravam conciliar o metodismo com o empirismo e o dogmatismo.
\section{Epissintetismo}
\begin{itemize}
\item {Grp. gram.:m.}
\end{itemize}
Doutrina dos epissintéticos.
\section{Epistaminado}
\begin{itemize}
\item {Grp. gram.:adj.}
\end{itemize}
\begin{itemize}
\item {Utilização:Bot.}
\end{itemize}
\begin{itemize}
\item {Proveniência:(De \textunderscore epi...\textunderscore  + \textunderscore estaminado\textunderscore )}
\end{itemize}
Que nasce sôbre o pistillo.
\section{Epistaminal}
\begin{itemize}
\item {Grp. gram.:adj.}
\end{itemize}
\begin{itemize}
\item {Utilização:Bot.}
\end{itemize}
\begin{itemize}
\item {Proveniência:(Do gr. \textunderscore epi\textunderscore  + lat. \textunderscore stamen\textunderscore )}
\end{itemize}
Que cresce sôbre os estames.
\section{Epistaminia}
\begin{itemize}
\item {Grp. gram.:f.}
\end{itemize}
\begin{itemize}
\item {Proveniência:(Do gr. \textunderscore epi\textunderscore  + lat. \textunderscore stamen\textunderscore )}
\end{itemize}
Propriedade dos vegetaes, cujos estames são inseridos no pistillo.
\section{Epistar}
\begin{itemize}
\item {Grp. gram.:v. t.}
\end{itemize}
\begin{itemize}
\item {Proveniência:(Do lat. \textunderscore e\textunderscore  + \textunderscore pistare\textunderscore )}
\end{itemize}
Reduzir (alguma coisa) a massa, depois de pisar em almofariz.
\section{Epístase}
\begin{itemize}
\item {Grp. gram.:f.}
\end{itemize}
\begin{itemize}
\item {Proveniência:(Gr. \textunderscore epistasis\textunderscore )}
\end{itemize}
Substância, que se conserva em suspensão na urina.
\section{Epistaxe}
\begin{itemize}
\item {fónica:cse}
\end{itemize}
\begin{itemize}
\item {Grp. gram.:f.}
\end{itemize}
\begin{itemize}
\item {Proveniência:(Gr. \textunderscore epistaxis\textunderscore )}
\end{itemize}
Derramamento de sangue pelas fossas nasaes.
\section{Episternal}
\begin{itemize}
\item {Grp. gram.:adj.}
\end{itemize}
\begin{itemize}
\item {Utilização:Zool.}
\end{itemize}
\begin{itemize}
\item {Proveniência:(De \textunderscore episterno\textunderscore )}
\end{itemize}
Diz-se das apóphyses do peito de alguns insectos.
Peça do esterno das tartarugas.
\section{Episterno}
\begin{itemize}
\item {Grp. gram.:m.}
\end{itemize}
\begin{itemize}
\item {Proveniência:(De \textunderscore epi...\textunderscore  + \textunderscore esterno\textunderscore )}
\end{itemize}
Peça do thórax dos insectos hexápodes.
\section{Epistílio}
\begin{itemize}
\item {Grp. gram.:m.}
\end{itemize}
\begin{itemize}
\item {Proveniência:(Lat. \textunderscore epistylium\textunderscore )}
\end{itemize}
O mesmo que \textunderscore arquitrave\textunderscore .
\section{Epístola}
\begin{itemize}
\item {Grp. gram.:f.}
\end{itemize}
\begin{itemize}
\item {Proveniência:(Lat. \textunderscore epistola\textunderscore )}
\end{itemize}
Carta.
Composição poética, em fórma de carta.
Parte da Missa, em que o celebrante lê uma epístola extrahída da \textunderscore Biblia\textunderscore .
\textunderscore Lado da epístola\textunderscore , á direita do altar, no cruzeiro da igreja.
\section{Epistolar}
\begin{itemize}
\item {Grp. gram.:adj.}
\end{itemize}
\begin{itemize}
\item {Proveniência:(Lat. \textunderscore epistolaris\textunderscore )}
\end{itemize}
Relativo a epístola.
Próprio da epístola ou do gênero literário, cuja fórma é a carta.
\section{Epistolar}
\begin{itemize}
\item {Grp. gram.:v. t.}
\end{itemize}
\begin{itemize}
\item {Utilização:Neol.}
\end{itemize}
Narrar em epístolas ou cartas. Cf. Rui Barb., \textunderscore Réplica\textunderscore , 157.
\section{Epistolário}
\begin{itemize}
\item {Grp. gram.:m.}
\end{itemize}
Collecção de epístolas.
\section{Epistolarmente}
\begin{itemize}
\item {Grp. gram.:adv.}
\end{itemize}
\begin{itemize}
\item {Proveniência:(De epistolar^1)}
\end{itemize}
Em forma de epístola; por meio de carta.
\section{Epistoleiro}
\begin{itemize}
\item {Grp. gram.:m.}
\end{itemize}
O mesmo que \textunderscore epistolário\textunderscore .
\section{Epistólico}
\begin{itemize}
\item {Grp. gram.:adj.}
\end{itemize}
(V. \textunderscore epistolar\textunderscore ^1)
\section{Epistolizar}
\begin{itemize}
\item {Grp. gram.:v. i}
\end{itemize}
Mandar epístolas; escrever cartas. Cf. \textunderscore Fênix Renasc.\textunderscore , IV, 39.
\section{Epistolografia}
\begin{itemize}
\item {Grp. gram.:f.}
\end{itemize}
\begin{itemize}
\item {Proveniência:(De \textunderscore epistolografo\textunderscore )}
\end{itemize}
Arte de escrever cartas.
Parte da literatura, que se occupa do gênero epistolar.
\section{Epistolográfico}
\begin{itemize}
\item {Grp. gram.:adj.}
\end{itemize}
Relativo a epistolografia.
\section{Epistológrafo}
\begin{itemize}
\item {Grp. gram.:m.}
\end{itemize}
\begin{itemize}
\item {Proveniência:(Do gr. \textunderscore epistole\textunderscore  + \textunderscore graphein\textunderscore )}
\end{itemize}
Aquele que escreve cartas.
Autor de cartas notáveis, literária ou historicamente.
\section{Epistolographia}
\begin{itemize}
\item {Grp. gram.:f.}
\end{itemize}
\begin{itemize}
\item {Proveniência:(De \textunderscore epistolographo\textunderscore )}
\end{itemize}
Arte de escrever cartas.
Parte da literatura, que se occupa do gênero epistolar.
\section{Epistolographico}
\begin{itemize}
\item {Grp. gram.:adj.}
\end{itemize}
Relativo a epistolographia.
\section{Epistológrapho}
\begin{itemize}
\item {Grp. gram.:m.}
\end{itemize}
\begin{itemize}
\item {Proveniência:(Do gr. \textunderscore epistole\textunderscore  + \textunderscore graphein\textunderscore )}
\end{itemize}
Aquelle que escreve cartas.
Autor de cartas notáveis, literária ou historicamente.
\section{Epístoma}
\begin{itemize}
\item {Grp. gram.:m.}
\end{itemize}
\begin{itemize}
\item {Proveniência:(Do gr. \textunderscore epi\textunderscore  + \textunderscore stoma\textunderscore )}
\end{itemize}
O mesmo que \textunderscore opérculo\textunderscore .
\section{Epístrofe}
\begin{itemize}
\item {Grp. gram.:f.}
\end{itemize}
\begin{itemize}
\item {Utilização:Rhet.}
\end{itemize}
\begin{itemize}
\item {Proveniência:(Lat. \textunderscore epistrophe\textunderscore )}
\end{itemize}
Repetição de uma palavra no fim de frases seguidas.
\section{Epistrofeu}
\begin{itemize}
\item {Grp. gram.:m.}
\end{itemize}
\begin{itemize}
\item {Utilização:Anat.}
\end{itemize}
\begin{itemize}
\item {Proveniência:(Fr. \textunderscore epistrophée\textunderscore )}
\end{itemize}
A segunda vértebra cervical.
\section{Epístrophe}
\begin{itemize}
\item {Grp. gram.:f.}
\end{itemize}
\begin{itemize}
\item {Utilização:Rhet.}
\end{itemize}
\begin{itemize}
\item {Proveniência:(Lat. \textunderscore epistrophe\textunderscore )}
\end{itemize}
Repetição de uma palavra no fim de phrases seguidas.
\section{Epistropheu}
\begin{itemize}
\item {Grp. gram.:m.}
\end{itemize}
\begin{itemize}
\item {Utilização:Anat.}
\end{itemize}
\begin{itemize}
\item {Proveniência:(Fr. \textunderscore epistrophée\textunderscore )}
\end{itemize}
A segunda vértebra cervical.
\section{Epistýlio}
\begin{itemize}
\item {Grp. gram.:m.}
\end{itemize}
\begin{itemize}
\item {Proveniência:(Lat. \textunderscore epistylium\textunderscore )}
\end{itemize}
O mesmo que \textunderscore architrave\textunderscore .
\section{Episyllogismo}
\begin{itemize}
\item {fónica:si}
\end{itemize}
\begin{itemize}
\item {Grp. gram.:m.}
\end{itemize}
\begin{itemize}
\item {Proveniência:(De \textunderscore epi...\textunderscore  + \textunderscore syllogismo\textunderscore )}
\end{itemize}
Raciocínio, que faz parte de uma série polysyllogística e que, para uma das suas premissas, toma a conclusão de um raciocínio precedente.
\section{Episynthético}
\begin{itemize}
\item {fónica:sin}
\end{itemize}
\begin{itemize}
\item {Grp. gram.:adj.}
\end{itemize}
\begin{itemize}
\item {Grp. gram.:M. pl.}
\end{itemize}
\begin{itemize}
\item {Proveniência:(Gr. \textunderscore episunthetikos\textunderscore )}
\end{itemize}
Relativo ao episynthétismo.
Seita medicinal dos que procuravam conciliar o methodismo com o empirismo e o dogmatismo.
\section{Episynthetismo}
\begin{itemize}
\item {fónica:sin}
\end{itemize}
\begin{itemize}
\item {Grp. gram.:m.}
\end{itemize}
Doutrina dos episynthéticos.
\section{Epitáfio}
\begin{itemize}
\item {Grp. gram.:m.}
\end{itemize}
\begin{itemize}
\item {Utilização:Prov.}
\end{itemize}
\begin{itemize}
\item {Utilização:beir.}
\end{itemize}
\begin{itemize}
\item {Proveniência:(Lat. \textunderscore epitaphius\textunderscore )}
\end{itemize}
Inscripção num túmulo.
O mesmo que \textunderscore bitafe\textunderscore .
\section{Epitafista}
\begin{itemize}
\item {Grp. gram.:m.}
\end{itemize}
Indivíduo que compõe epitáfios.
\section{Epitalâmico}
\begin{itemize}
\item {Grp. gram.:adj.}
\end{itemize}
Relativo a epitalâmio.
\section{Epitalâmio}
\begin{itemize}
\item {Grp. gram.:m.}
\end{itemize}
\begin{itemize}
\item {Proveniência:(Lat. \textunderscore epithalamium\textunderscore )}
\end{itemize}
Canto, em que se celebram as núpcias de alguém.
\section{Epitáphio}
\begin{itemize}
\item {Grp. gram.:m.}
\end{itemize}
\begin{itemize}
\item {Utilização:Prov.}
\end{itemize}
\begin{itemize}
\item {Utilização:beir.}
\end{itemize}
\begin{itemize}
\item {Proveniência:(Lat. \textunderscore epitaphius\textunderscore )}
\end{itemize}
Inscripção num túmulo.
O mesmo que \textunderscore bitafe\textunderscore .
\section{Epitaphista}
\begin{itemize}
\item {Grp. gram.:m.}
\end{itemize}
Indivíduo que compõe epitáphios.
\section{Epítase}
\begin{itemize}
\item {Grp. gram.:f.}
\end{itemize}
\begin{itemize}
\item {Proveniência:(Lat. \textunderscore epitasis\textunderscore )}
\end{itemize}
Parte do poema dramático, que desenvolve os incidentes principaes e contém o enrêdo da peça.
\section{Epitelial}
\begin{itemize}
\item {Grp. gram.:adj.}
\end{itemize}
Relativo ao epitélio.
Que aparece no epitélio.
\section{Epitélio}
\begin{itemize}
\item {Grp. gram.:m.}
\end{itemize}
\begin{itemize}
\item {Utilização:Anat.}
\end{itemize}
\begin{itemize}
\item {Proveniência:(Do gr. \textunderscore epi\textunderscore  + \textunderscore thele\textunderscore )}
\end{itemize}
Película, que reveste as membranas mucosas, como a epiderme reveste a derme.
\section{Epitelioma}
\begin{itemize}
\item {Grp. gram.:m.}
\end{itemize}
Tumor epitelial.
\section{Epiteliomatoso}
\begin{itemize}
\item {Grp. gram.:adj.}
\end{itemize}
Que, sofre epitelioma.
Que é da natureza do epitelioma.
\section{Epítema}
\begin{itemize}
\item {Grp. gram.:f.}
\end{itemize}
\begin{itemize}
\item {Proveniência:(Do gr. \textunderscore epi\textunderscore  + \textunderscore thema\textunderscore )}
\end{itemize}
Qualquer medicamento tópico, não sendo emplastro ou unguento.
\section{Epítese}
\begin{itemize}
\item {Grp. gram.:f.}
\end{itemize}
\begin{itemize}
\item {Proveniência:(Lat. \textunderscore epithesis\textunderscore )}
\end{itemize}
O mesmo que \textunderscore paragoge\textunderscore .
\section{Epitetar}
\begin{itemize}
\item {Grp. gram.:v. t.}
\end{itemize}
Pôr a alguém ou a alguma coisa o epíteto de.
Intitular.
Cognominar. Cf. Arn. Gama, \textunderscore Últ. Dona\textunderscore , 409.
\section{Epitético}
\begin{itemize}
\item {Grp. gram.:adj.}
\end{itemize}
Que tem o carácter de epíteto.
\section{Epitetismo}
\begin{itemize}
\item {Grp. gram.:m.}
\end{itemize}
\begin{itemize}
\item {Proveniência:(De \textunderscore epíteto\textunderscore )}
\end{itemize}
Modificação da expressão de uma ideia principal pela expressão de uma ideia acessória.
\section{Epíteto}
\begin{itemize}
\item {Grp. gram.:m.}
\end{itemize}
\begin{itemize}
\item {Proveniência:(Gr. \textunderscore epithetos\textunderscore )}
\end{itemize}
Palavra, que qualifica um substantivo.
Qualificação; cognome.
\section{Epithalâmico}
\begin{itemize}
\item {Grp. gram.:adj.}
\end{itemize}
Relativo a epithalâmio.
\section{Epithalâmio}
\begin{itemize}
\item {Grp. gram.:m.}
\end{itemize}
\begin{itemize}
\item {Proveniência:(Lat. \textunderscore epithalamium\textunderscore )}
\end{itemize}
Canto, em que se celebram as núpcias de alguém.
\section{Epithelial}
\begin{itemize}
\item {Grp. gram.:adj.}
\end{itemize}
Relativo ao epithélio.
Que apparece no epithélio.
\section{Epithélio}
\begin{itemize}
\item {Grp. gram.:m.}
\end{itemize}
\begin{itemize}
\item {Utilização:Anat.}
\end{itemize}
\begin{itemize}
\item {Proveniência:(Do gr. \textunderscore epi\textunderscore  + \textunderscore thele\textunderscore )}
\end{itemize}
Pellícula, que reveste as membranas mucosas, como a epiderme reveste a derme.
\section{Epithelioma}
\begin{itemize}
\item {Grp. gram.:m.}
\end{itemize}
Tumor epithelial.
\section{Epitheliomatoso}
\begin{itemize}
\item {Grp. gram.:adj.}
\end{itemize}
Que, soffre epithelioma.
Que é da natureza do epithelioma.
\section{Epíthema}
\begin{itemize}
\item {Grp. gram.:f.}
\end{itemize}
\begin{itemize}
\item {Proveniência:(Do gr. \textunderscore epi\textunderscore  + \textunderscore thema\textunderscore )}
\end{itemize}
Qualquer medicamento tópico, não sendo emplastro ou unguento.
\section{Epíthese}
\begin{itemize}
\item {Grp. gram.:f.}
\end{itemize}
\begin{itemize}
\item {Proveniência:(Lat. \textunderscore epithesis\textunderscore )}
\end{itemize}
O mesmo que \textunderscore paragoge\textunderscore .
\section{Epithetar}
\begin{itemize}
\item {Grp. gram.:v. t.}
\end{itemize}
Pôr a alguém ou a alguma coisa o epítheto de.
Intitular.
Cognominar. Cf. Arn. Gama, \textunderscore Últ. Dona\textunderscore , 409.
\section{Epithético}
\begin{itemize}
\item {Grp. gram.:adj.}
\end{itemize}
Que tem o carácter de epítheto.
\section{Epithetismo}
\begin{itemize}
\item {Grp. gram.:m.}
\end{itemize}
\begin{itemize}
\item {Proveniência:(De \textunderscore epítheto\textunderscore )}
\end{itemize}
Modificação da expressão de uma ideia principal pela expressão de uma ideia accessória.
\section{Epítheto}
\begin{itemize}
\item {Grp. gram.:m.}
\end{itemize}
\begin{itemize}
\item {Proveniência:(Gr. \textunderscore epithetos\textunderscore )}
\end{itemize}
Palavra, que qualifica um substantivo.
Qualificação; cognome.
\section{Epitoga}
\begin{itemize}
\item {Grp. gram.:f.}
\end{itemize}
Capa, que os Romanos usavam sôbre a toga.
O mesmo que \textunderscore epitógio\textunderscore .
\section{Epitógio}
\begin{itemize}
\item {Grp. gram.:m.}
\end{itemize}
\begin{itemize}
\item {Utilização:Ant.}
\end{itemize}
\begin{itemize}
\item {Proveniência:(Lat. \textunderscore epitogium\textunderscore )}
\end{itemize}
O mesmo que \textunderscore tabardo\textunderscore . Cf. Herculano, \textunderscore Bobo\textunderscore , 134, 151 e 175.
\section{Epitomador}
\begin{itemize}
\item {Grp. gram.:m.}
\end{itemize}
Aquelle que epitoma.
\section{Epitomar}
\begin{itemize}
\item {Grp. gram.:v. t.}
\end{itemize}
Converter em epítome.
Resumir.
Epilogar.
\section{Epítome}
\begin{itemize}
\item {Grp. gram.:m.}
\end{itemize}
\begin{itemize}
\item {Proveniência:(Lat. \textunderscore epitome\textunderscore )}
\end{itemize}
Resumo de doutrina.
Synopse; compêndio.
\section{Epítrito}
\begin{itemize}
\item {Grp. gram.:adj.}
\end{itemize}
\begin{itemize}
\item {Proveniência:(Lat. \textunderscore epitritus\textunderscore )}
\end{itemize}
Dizia-se de um número, composto de outro e mais um terço dêste, como 4 a respeito de 3.
\section{Epitróchlea}
\begin{itemize}
\item {Grp. gram.:f}
\end{itemize}
\begin{itemize}
\item {Utilização:Anat.}
\end{itemize}
\begin{itemize}
\item {Proveniência:(De \textunderscore epi...\textunderscore  + \textunderscore tróchlea\textunderscore )}
\end{itemize}
Eminência arredondada, na parte interna da extremidade cubital do húmero, por cima tróchlea.
\section{Epitrochleano}
\begin{itemize}
\item {Grp. gram.:adj}
\end{itemize}
Relativo a epitróchlea.
\section{Epitróclea}
\begin{itemize}
\item {Grp. gram.:f}
\end{itemize}
\begin{itemize}
\item {Utilização:Anat.}
\end{itemize}
\begin{itemize}
\item {Proveniência:(De \textunderscore epi...\textunderscore  + \textunderscore tróclea\textunderscore )}
\end{itemize}
Eminência arredondada, na parte interna da extremidade cubital do húmero, por cima tróclea.
\section{Epitrocleano}
\begin{itemize}
\item {Grp. gram.:adj}
\end{itemize}
Relativo a epitróclea.
\section{Epítrope}
\begin{itemize}
\item {Grp. gram.:f.}
\end{itemize}
\begin{itemize}
\item {Proveniência:(Lat. \textunderscore epitrope\textunderscore )}
\end{itemize}
Figura de Rhetorica, que consiste em conceder alguma coisa que se poderia constestar, para dar mais autoridade ao que se procura provar.
\section{Epituitário}
\begin{itemize}
\item {fónica:tu-i}
\end{itemize}
\begin{itemize}
\item {Grp. gram.:adj.}
\end{itemize}
\begin{itemize}
\item {Proveniência:(De \textunderscore epi...\textunderscore  + \textunderscore pituíta\textunderscore )}
\end{itemize}
Situado sôbre a pituitária.
\section{Epixilo}
\begin{itemize}
\item {fónica:csi}
\end{itemize}
\begin{itemize}
\item {Grp. gram.:adj.}
\end{itemize}
\begin{itemize}
\item {Proveniência:(Do gr. \textunderscore epi\textunderscore  + \textunderscore xulon\textunderscore )}
\end{itemize}
Que cresce sôbre o lenho, (falando-se de plantas parasitas).
\section{Epixylo}
\begin{itemize}
\item {Grp. gram.:adj.}
\end{itemize}
\begin{itemize}
\item {Proveniência:(Do gr. \textunderscore epi\textunderscore  + \textunderscore xulon\textunderscore )}
\end{itemize}
Que cresce sôbre o lenho, (falando-se de plantas parasitas).
\section{Epizoário}
\begin{itemize}
\item {Grp. gram.:m.  e  adj.}
\end{itemize}
\begin{itemize}
\item {Proveniência:(Do gr. \textunderscore epi\textunderscore  + \textunderscore zoarion\textunderscore )}
\end{itemize}
Animálculo parasito, que vive sôbre a pelle do homem ou de outros animaes.
\section{Epizoico}
\begin{itemize}
\item {Grp. gram.:adj.}
\end{itemize}
\begin{itemize}
\item {Utilização:Geol.}
\end{itemize}
\begin{itemize}
\item {Proveniência:(Do gr. \textunderscore epi\textunderscore  + \textunderscore zoon\textunderscore )}
\end{itemize}
Diz-se dos terrenos superiores aos que encerram despojos orgânicos.
\section{Epizootia}
\begin{itemize}
\item {Grp. gram.:f.}
\end{itemize}
\begin{itemize}
\item {Proveniência:(Do gr. \textunderscore epi\textunderscore  + \textunderscore zoon\textunderscore )}
\end{itemize}
Doença, que ataca muitos animaes ao mesmo tempo.
\section{Epizoótico}
\begin{itemize}
\item {Grp. gram.:adj.}
\end{itemize}
Relativo a epizootia.
\section{Época}
\begin{itemize}
\item {Grp. gram.:f.}
\end{itemize}
\begin{itemize}
\item {Proveniência:(Gr. \textunderscore epokhe\textunderscore )}
\end{itemize}
Momento histórico ou espaço de tempo, assignalado por um facto importante.
Successo notável, escolhido para uma divisão do tempo: \textunderscore a época das Cruzadas\textunderscore .
Tempo, decorrido entre dois acontecimentos notáveis.
Qualquer parte do tempo, relativamente aos acontecimentos que nelle se deram.
Espaço de tempo, que se seguiu a cada uma das grandes alterações do globo terrestre: \textunderscore a época terciária\textunderscore .
\section{Épocha}
\begin{itemize}
\item {fónica:ca}
\end{itemize}
\begin{itemize}
\item {Grp. gram.:f.}
\end{itemize}
(V.época)
\section{Epódico}
\begin{itemize}
\item {Grp. gram.:adj.}
\end{itemize}
Relativo a épodo.
\section{Épodo}
\begin{itemize}
\item {Grp. gram.:m.}
\end{itemize}
\begin{itemize}
\item {Proveniência:(Lat. \textunderscore epodus\textunderscore )}
\end{itemize}
Terceira parte de um canto, dividido em estrophes, na poesia antiga.
Sentença moral.
\section{Eponímia}
\begin{itemize}
\item {Grp. gram.:f.}
\end{itemize}
\begin{itemize}
\item {Proveniência:(De \textunderscore epónimo\textunderscore )}
\end{itemize}
Nome de coisas, tirado de outras coisas ou de pessôas.
\section{Epónimo}
\begin{itemize}
\item {Grp. gram.:adj.}
\end{itemize}
\begin{itemize}
\item {Proveniência:(Gr. \textunderscore eponumos\textunderscore )}
\end{itemize}
Que dá ou empresta o seu nome a alguma coisa.
\section{Eponýmia}
\begin{itemize}
\item {Grp. gram.:f.}
\end{itemize}
\begin{itemize}
\item {Proveniência:(De \textunderscore epónymo\textunderscore )}
\end{itemize}
Nome de coisas, tirado de outras coisas ou de pessôas.
\section{Epónymo}
\begin{itemize}
\item {Grp. gram.:adj.}
\end{itemize}
\begin{itemize}
\item {Proveniência:(Gr. \textunderscore eponumos\textunderscore )}
\end{itemize}
Que dá ou empresta o seu nome a alguma coisa.
\section{Epopaico}
\begin{itemize}
\item {Grp. gram.:adj.}
\end{itemize}
(V.epopeico)
\section{Epopéa}
\begin{itemize}
\item {Grp. gram.:f.}
\end{itemize}
\begin{itemize}
\item {Utilização:Fig.}
\end{itemize}
\begin{itemize}
\item {Proveniência:(Gr. \textunderscore epopoiia\textunderscore )}
\end{itemize}
Poema, em que se narram acções heroicas.
Poema épico, baseado em elementos históricos, entrelaçados com a lenda, o maravilhoso ou a Mythologia.
Série de acções brilhantes ou heroicas, dignas de serem cantadas num poema épico.
\section{Epopeia}
\begin{itemize}
\item {Grp. gram.:f.}
\end{itemize}
\begin{itemize}
\item {Utilização:Fig.}
\end{itemize}
\begin{itemize}
\item {Proveniência:(Gr. \textunderscore epopoiia\textunderscore )}
\end{itemize}
Poema, em que se narram acções heroicas.
Poema épico, baseado em elementos históricos, entrelaçados com a lenda, o maravilhoso ou a Mitologia.
Série de acções brilhantes ou heroicas, dignas de serem cantadas num poema épico.
\section{Epopeico}
\begin{itemize}
\item {Grp. gram.:adj.}
\end{itemize}
Relativo a epopeia.
Heroico; grandioso.
\section{Epopsia}
\begin{itemize}
\item {Grp. gram.:f.}
\end{itemize}
Suprema iniciação, de noite, nos mystérios de Elêusis. Cf. Castilho, \textunderscore Fastos\textunderscore , II, 663.
O mesmo que \textunderscore epoptismo\textunderscore .
\section{Epopta}
\begin{itemize}
\item {Grp. gram.:m.}
\end{itemize}
\begin{itemize}
\item {Proveniência:(Lat. \textunderscore epopta\textunderscore )}
\end{itemize}
Indivíduo, iniciado no epoptismo.
Inspector dos mystérios de Elêusis.
\section{Epoptico}
\begin{itemize}
\item {Grp. gram.:adj.}
\end{itemize}
Relativo a epoptismo ou a epopta.
\section{Epoptismo}
\begin{itemize}
\item {Grp. gram.:m.}
\end{itemize}
\begin{itemize}
\item {Proveniência:(De \textunderscore epopta\textunderscore )}
\end{itemize}
Terceiro grau da inicíação nos mystérios de Elêusis.
\section{Epos}
\begin{itemize}
\item {Grp. gram.:m.}
\end{itemize}
\begin{itemize}
\item {Proveniência:(Lat. \textunderscore epos\textunderscore )}
\end{itemize}
Epopeia.
O gênero épico. Cf. Latino, \textunderscore Camões\textunderscore , 261 e 264.
\section{Epostracismo}
\begin{itemize}
\item {Grp. gram.:m.}
\end{itemize}
\begin{itemize}
\item {Utilização:Ant.}
\end{itemize}
\begin{itemize}
\item {Proveniência:(Gr. \textunderscore epostralismos\textunderscore )}
\end{itemize}
Jôgo de crianças, que consistia em atirar conchas ou pedras pela superfície do mar, ganhando aquelle cuja concha désse mais resaltos á tona da água.
\section{Epsomita}
\begin{itemize}
\item {Grp. gram.:f.}
\end{itemize}
\begin{itemize}
\item {Proveniência:(De \textunderscore Epsom\textunderscore , n. p.)}
\end{itemize}
Sulfato purgativo de magnésia hydratado.
\section{Epulão}
\begin{itemize}
\item {Grp. gram.:m.}
\end{itemize}
\begin{itemize}
\item {Proveniência:(Lat. \textunderscore epulo\textunderscore )}
\end{itemize}
Sacerdote, que, entre os Romanos, presidia aos festins dos sacrifícios. Cf. Castilho, \textunderscore Fastos\textunderscore , II, 615.
\section{Epular}
\begin{itemize}
\item {Grp. gram.:adj.}
\end{itemize}
\begin{itemize}
\item {Proveniência:(Lat. \textunderscore epularis\textunderscore )}
\end{itemize}
Relativo ás épulas.
\section{Epulário}
\begin{itemize}
\item {Grp. gram.:m.}
\end{itemize}
\begin{itemize}
\item {Utilização:Ant.}
\end{itemize}
\begin{itemize}
\item {Proveniência:(Do lat. \textunderscore epulum\textunderscore )}
\end{itemize}
Conviva; commensal.
\section{Épulas}
\begin{itemize}
\item {Grp. gram.:f. pl.}
\end{itemize}
\begin{itemize}
\item {Utilização:Des.}
\end{itemize}
\begin{itemize}
\item {Proveniência:(Lat. \textunderscore epulae\textunderscore )}
\end{itemize}
Iguaria.
Alimentação.
\section{Epúlida}
\begin{itemize}
\item {Grp. gram.:f.}
\end{itemize}
\begin{itemize}
\item {Proveniência:(Do gr. \textunderscore epulis\textunderscore )}
\end{itemize}
Excrescência nas gengivas.
\section{Epulótico}
\begin{itemize}
\item {Grp. gram.:adj.}
\end{itemize}
\begin{itemize}
\item {Proveniência:(Gr. \textunderscore epulotikos\textunderscore )}
\end{itemize}
Que favorece a cicatrização.
\section{Epuxa!}
\begin{itemize}
\item {Grp. gram.:interj.}
\end{itemize}
\begin{itemize}
\item {Utilização:Bras}
\end{itemize}
\begin{itemize}
\item {Utilização:chul.}
\end{itemize}
(designativa de admiração)
\section{Equabilidade}
\begin{itemize}
\item {Grp. gram.:f.}
\end{itemize}
\begin{itemize}
\item {Proveniência:(Do lat. \textunderscore aequabilitas\textunderscore )}
\end{itemize}
Uniformidade, igualdade.
\section{Equação}
\begin{itemize}
\item {Grp. gram.:f.}
\end{itemize}
\begin{itemize}
\item {Proveniência:(Lat. \textunderscore aequatio\textunderscore )}
\end{itemize}
Fórma de igualdade entre duas quantidades.
Quantidade variável, mas calculada, que é preciso ajuntar ou substrahir ao movimento médio dos planetas, para se verificar o verdadeiro movimento.
\section{Equador}
\begin{itemize}
\item {Grp. gram.:m.}
\end{itemize}
\begin{itemize}
\item {Proveniência:(Lat. \textunderscore aequator\textunderscore )}
\end{itemize}
Círculo máximo da esphera celeste, perpendicular ao eixo da Terra.
Regiões, situadas debaixo dêsse círculo.
Círculo da esphera terrestre, projecção do equador celeste.
\section{Equala}
\begin{itemize}
\item {Grp. gram.:f.}
\end{itemize}
Espécie de corvo, (\textunderscore corvus scapulatus\textunderscore ).
\section{Equalifloro}
\begin{itemize}
\item {Grp. gram.:adj.}
\end{itemize}
\begin{itemize}
\item {Utilização:Bot.}
\end{itemize}
\begin{itemize}
\item {Proveniência:(Do lat. \textunderscore aequelis\textunderscore  + \textunderscore flos\textunderscore )}
\end{itemize}
Cujas flôres são todas iguaes em comprimento.
\section{Equânime}
\begin{itemize}
\item {Grp. gram.:adj.}
\end{itemize}
\begin{itemize}
\item {Proveniência:(Lat. \textunderscore aequanimus\textunderscore )}
\end{itemize}
Que tem equanimidade.
\section{Equanimidade}
\begin{itemize}
\item {Grp. gram.:f.}
\end{itemize}
\begin{itemize}
\item {Proveniência:(Lat. \textunderscore aequanimitas\textunderscore )}
\end{itemize}
Igualdade de ânimo, deante dos perigos e deante da prosperidade.
Serenidade de espírito; imparcialidade; equidade em julgar:«\textunderscore tenha equanimidade bastante para desculpar esta offerta de saloio.\textunderscore »Herculano, \textunderscore Carta\textunderscore  a B. Pato.
\section{Equante}
\begin{itemize}
\item {Grp. gram.:m.}
\end{itemize}
\begin{itemize}
\item {Utilização:Ant.}
\end{itemize}
\begin{itemize}
\item {Proveniência:(Lat. \textunderscore aequans\textunderscore )}
\end{itemize}
Círculo excêntrico á terra, percorrido pelos planetas, segundo a theoria dos ant. astrónomos.
\section{Equatorial}
\begin{itemize}
\item {Grp. gram.:adj.}
\end{itemize}
\begin{itemize}
\item {Grp. gram.:M.}
\end{itemize}
\begin{itemize}
\item {Proveniência:(Do lat. \textunderscore aequator\textunderscore )}
\end{itemize}
Relativo ao equador.
Que está situado ou que cresce no equador: \textunderscore plantas equatoriaes\textunderscore .
Instrumento, para medir a ascensão e a declinação recta dos astros.
\section{Equatoriano}
\begin{itemize}
\item {Grp. gram.:adj.}
\end{itemize}
\begin{itemize}
\item {Grp. gram.:M.}
\end{itemize}
Relativo á república do Equador.
Habitante do Equador.
\section{Equável}
\begin{itemize}
\item {Grp. gram.:adj.}
\end{itemize}
\begin{itemize}
\item {Proveniência:(Lat. \textunderscore equabilis\textunderscore )}
\end{itemize}
Que, comparado a outro, é igual; uniforme.
Equitativo. Cf. Filinto, \textunderscore D. Man.\textunderscore , III, 288.
\section{Equeias}
\begin{itemize}
\item {Grp. gram.:f. pl.}
\end{itemize}
\begin{itemize}
\item {Proveniência:(Do gr. \textunderscore echeis\textunderscore )}
\end{itemize}
Vasos de bronze, que, colocados nos teatros gregos, tornavam êstes mais sonoros.
\section{Equestre}
\begin{itemize}
\item {fónica:cu-es}
\end{itemize}
\begin{itemize}
\item {Grp. gram.:adj.}
\end{itemize}
\begin{itemize}
\item {Proveniência:(Lat. \textunderscore equestris\textunderscore )}
\end{itemize}
Relativo a cavallaria ou a cavalleiros.
Que representa alguém a cavallo: \textunderscore estátua equestre\textunderscore .
\section{Equeus}
\begin{itemize}
\item {Grp. gram.:m. pl.}
\end{itemize}
O mesmo que \textunderscore equeias\textunderscore .
\section{Equevo}
\begin{itemize}
\item {fónica:cu-e}
\end{itemize}
\begin{itemize}
\item {Grp. gram.:adj.}
\end{itemize}
\begin{itemize}
\item {Proveniência:(Lat. \textunderscore aequaevus\textunderscore )}
\end{itemize}
Que é da mesma idade.
Contemporáneo.
\section{Equi...}
\begin{itemize}
\item {fónica:cu-i}
\end{itemize}
\begin{itemize}
\item {Grp. gram.:pref.}
\end{itemize}
\begin{itemize}
\item {Proveniência:(Do lat. \textunderscore aequus\textunderscore )}
\end{itemize}
(designativo de igualdade)
\section{Equiângulo}
\begin{itemize}
\item {fónica:cu-i}
\end{itemize}
\begin{itemize}
\item {Grp. gram.:adj.}
\end{itemize}
\begin{itemize}
\item {Utilização:Geom.}
\end{itemize}
\begin{itemize}
\item {Proveniência:(De \textunderscore equi...\textunderscore  + \textunderscore ângulo\textunderscore )}
\end{itemize}
Que tem ângulos íguaes.
\section{Equião}
\begin{itemize}
\item {Grp. gram.:m.}
\end{itemize}
\begin{itemize}
\item {Proveniência:(Gr. \textunderscore ekhion\textunderscore )}
\end{itemize}
Antigo medicamento, preparado com cinzas de víbora.
\section{Equícola}
\begin{itemize}
\item {fónica:cu-i}
\end{itemize}
\begin{itemize}
\item {Grp. gram.:m.}
\end{itemize}
\begin{itemize}
\item {Proveniência:(Do lat. \textunderscore equus\textunderscore  + \textunderscore colere\textunderscore )}
\end{itemize}
Tratador de cavallos. Cf. Barreto, \textunderscore Eneida\textunderscore , VII, 173.
\section{Equidade}
\begin{itemize}
\item {fónica:cu-i}
\end{itemize}
\begin{itemize}
\item {Grp. gram.:f.}
\end{itemize}
\begin{itemize}
\item {Proveniência:(Lat. \textunderscore aequitas\textunderscore )}
\end{itemize}
Disposição para se reconhecer imparcialmente o direito de cada qual.
Justiça natural, que póde não sêr conforme com as disposições da lei.
\section{Equídeo}
\begin{itemize}
\item {fónica:cu-í}
\end{itemize}
\begin{itemize}
\item {Grp. gram.:adj.}
\end{itemize}
\begin{itemize}
\item {Grp. gram.:M. pl.}
\end{itemize}
\begin{itemize}
\item {Proveniência:(Do lat. \textunderscore equus\textunderscore  + gr. \textunderscore eidos\textunderscore )}
\end{itemize}
Relativo ou semelhante ao cavallo.
Família de mammíferos, que têm por typo o cavallo.
\section{Equidifferença}
\begin{itemize}
\item {fónica:cu-i}
\end{itemize}
\begin{itemize}
\item {Grp. gram.:f.}
\end{itemize}
\begin{itemize}
\item {Proveniência:(De \textunderscore equi...\textunderscore  + \textunderscore differença\textunderscore )}
\end{itemize}
Proporção arithmética.
\section{Equidifferente}
\begin{itemize}
\item {fónica:cu-i}
\end{itemize}
\begin{itemize}
\item {Grp. gram.:adj.}
\end{itemize}
\begin{itemize}
\item {Proveniência:(De \textunderscore equi...\textunderscore  + \textunderscore differente\textunderscore )}
\end{itemize}
Que offerece differenças iguaes entre si.
\section{Equidilatado}
\begin{itemize}
\item {fónica:cu-i}
\end{itemize}
\begin{itemize}
\item {Grp. gram.:adj.}
\end{itemize}
\begin{itemize}
\item {Utilização:Bot.}
\end{itemize}
\begin{itemize}
\item {Proveniência:(De \textunderscore equi...\textunderscore  + \textunderscore dilatado\textunderscore )}
\end{itemize}
Diz-se de certos órgãos vegetaes, que têm a mesma largura em todo o seu comprimento.
\section{Equidistância}
\begin{itemize}
\item {fónica:cu-i}
\end{itemize}
\begin{itemize}
\item {Grp. gram.:f.}
\end{itemize}
\begin{itemize}
\item {Proveniência:(De \textunderscore equi...\textunderscore  + \textunderscore distante\textunderscore )}
\end{itemize}
Qualidade daquillo que é equidistante.
\section{Equidistante}
\begin{itemize}
\item {fónica:cu-i}
\end{itemize}
\begin{itemize}
\item {Grp. gram.:adj.}
\end{itemize}
\begin{itemize}
\item {Proveniência:(De \textunderscore equi...\textunderscore  + \textunderscore distante\textunderscore )}
\end{itemize}
Que em todas as suas partes está igualmente afastado das partes do outro corpo.
\section{Equidistar}
\begin{itemize}
\item {fónica:cu-i}
\end{itemize}
\begin{itemize}
\item {Grp. gram.:v. i.}
\end{itemize}
\begin{itemize}
\item {Proveniência:(De \textunderscore equi...\textunderscore  + \textunderscore distar\textunderscore )}
\end{itemize}
Distar igualmente de dois ou mais pontos.
\section{Equidna}
\begin{itemize}
\item {Grp. gram.:m.}
\end{itemize}
\begin{itemize}
\item {Proveniência:(Gr. \textunderscore ekhidna\textunderscore )}
\end{itemize}
Mamífero australiano coberto de espinhos, como o ouriço.
Constelação da Hidra.
\section{Equídnico}
\begin{itemize}
\item {Grp. gram.:adj.}
\end{itemize}
\begin{itemize}
\item {Proveniência:(Do gr. \textunderscore ekhidna\textunderscore )}
\end{itemize}
Relativo á víbora, próprio da víbora.
\section{Equidnina}
\begin{itemize}
\item {Grp. gram.:f.}
\end{itemize}
Substância orgânica, que é princípio activo do veneno da víbora.
(Cp. \textunderscore echídnico\textunderscore )
\section{Equidno}
\begin{itemize}
\item {Grp. gram.:m.}
\end{itemize}
O mesmo que \textunderscore equidna\textunderscore .
\section{Equífero}
\begin{itemize}
\item {fónica:cu-i}
\end{itemize}
\begin{itemize}
\item {Grp. gram.:m.}
\end{itemize}
\begin{itemize}
\item {Proveniência:(Lat. \textunderscore equiferus\textunderscore )}
\end{itemize}
Cavallo selvagem.
\section{Equilateral}
\begin{itemize}
\item {fónica:cu-i}
\end{itemize}
\begin{itemize}
\item {Grp. gram.:adj.}
\end{itemize}
\begin{itemize}
\item {Proveniência:(Lat. \textunderscore aequilaterus\textunderscore )}
\end{itemize}
Que tem os lados iguaes entre si.
\section{Equilátero}
\begin{itemize}
\item {fónica:cu-i}
\end{itemize}
\begin{itemize}
\item {Grp. gram.:adj.}
\end{itemize}
\begin{itemize}
\item {Proveniência:(Lat. \textunderscore aequilaterus\textunderscore )}
\end{itemize}
Que tem os lados iguaes entre si.
\section{Equilibracão}
\begin{itemize}
\item {Grp. gram.:f.}
\end{itemize}
Acto ou effeito de equilibrar.
Equilíbrio.
\section{Equilibrador}
\begin{itemize}
\item {fónica:cu-i}
\end{itemize}
\begin{itemize}
\item {Grp. gram.:adj.}
\end{itemize}
Que equilibra: \textunderscore pêso equilibrador\textunderscore .
\section{Equilibrante}
\begin{itemize}
\item {Grp. gram.:adj.}
\end{itemize}
Que equilibra.
\section{Equilibrar}
\begin{itemize}
\item {Grp. gram.:v. t.}
\end{itemize}
Pôr em equilíbrio.
Conservar o equilíbrio de.
Compensar, contrabalançar.
\section{Equilíbrio}
\begin{itemize}
\item {Grp. gram.:m.}
\end{itemize}
\begin{itemize}
\item {Utilização:Fig.}
\end{itemize}
\begin{itemize}
\item {Proveniência:(Lat. \textunderscore aequilibrium\textunderscore )}
\end{itemize}
Estado de um corpo, que é trahido ou solicitado por fôrças oppostas, que se anullam sôbre um ponto de resistência.
Estado de um corpo, que se mantém de pé, sem se inclinar para nenhum dos lados.
Justa proporção.
Estado dos poderes públicos, que se relacionam, sem que nenhum delles domine ou supplante outro.
Estado da política geral, em que as nações convivem de maneira, que nenhuma póde pôr outra em perigo.
\section{Equilibrista}
\begin{itemize}
\item {Grp. gram.:m.  e  f.}
\end{itemize}
\begin{itemize}
\item {Proveniência:(De \textunderscore equilibrar\textunderscore )}
\end{itemize}
Pessôa, que se mantém em equilíbrio na maroma, ou de pé sôbre um cavallo, etc.
\section{Equimosar-se}
\begin{itemize}
\item {Grp. gram.:v. p.}
\end{itemize}
Cobrir-se de equimoses.
\section{Equimose}
\begin{itemize}
\item {Grp. gram.:f.}
\end{itemize}
\begin{itemize}
\item {Proveniência:(Gr. \textunderscore ekkhumosis\textunderscore )}
\end{itemize}
Mancha avermelhada ou escura, formada na pele por sangue extravasado em consequência de contusão.
\section{Equimótico}
\begin{itemize}
\item {Grp. gram.:adj.}
\end{itemize}
\begin{itemize}
\item {Proveniência:(Gr. \textunderscore ekkhumotikos\textunderscore )}
\end{itemize}
Que tem o carácter da equimose.
\section{Equimultíplice}
\begin{itemize}
\item {fónica:cu-i}
\end{itemize}
\begin{itemize}
\item {Grp. gram.:adj.}
\end{itemize}
O mesmo que \textunderscore equimúltiplo\textunderscore .
\section{Equimúltiplo}
\begin{itemize}
\item {fónica:cu-i}
\end{itemize}
\begin{itemize}
\item {Grp. gram.:adj.}
\end{itemize}
\begin{itemize}
\item {Utilização:Arith.}
\end{itemize}
\begin{itemize}
\item {Proveniência:(De \textunderscore equi...\textunderscore  + \textunderscore múltiplo\textunderscore )}
\end{itemize}
Diz-se dos números, que contêm os seus submúltiplos tantas vezes como outro.
\section{Equínides}
\begin{itemize}
\item {Grp. gram.:m. pl.}
\end{itemize}
\begin{itemize}
\item {Utilização:Zool.}
\end{itemize}
\begin{itemize}
\item {Proveniência:(Do gr. \textunderscore ekhinos\textunderscore )}
\end{itemize}
Classe de equinodermes.
\section{Equinípede}
\begin{itemize}
\item {Grp. gram.:adj.}
\end{itemize}
\begin{itemize}
\item {Utilização:Zool.}
\end{itemize}
\begin{itemize}
\item {Proveniência:(Do lat. \textunderscore echinus\textunderscore  + \textunderscore pes\textunderscore )}
\end{itemize}
Que tem as patas revestidas de pelos ásperos.
\section{Equino}
\begin{itemize}
\item {fónica:cu-i}
\end{itemize}
\begin{itemize}
\item {Grp. gram.:adj.}
\end{itemize}
\begin{itemize}
\item {Proveniência:(Lat. \textunderscore equinus\textunderscore )}
\end{itemize}
Relativo a cavallo ou a equídeos.
\section{Équino}
\begin{itemize}
\item {Grp. gram.:m.}
\end{itemize}
\begin{itemize}
\item {Proveniência:(Gr. \textunderscore ekhinos\textunderscore )}
\end{itemize}
Moldura em quarto de círculo.
Ornato oval e convexo.
\section{Equinocarpo}
\begin{itemize}
\item {Grp. gram.:adj.}
\end{itemize}
\begin{itemize}
\item {Utilização:Bot.}
\end{itemize}
\begin{itemize}
\item {Grp. gram.:M.}
\end{itemize}
\begin{itemize}
\item {Proveniência:(Do gr. \textunderscore ekhinos\textunderscore  + \textunderscore karpos\textunderscore )}
\end{itemize}
Que produz frutos erriçados de pontas ásperas.
Grande árvore da ilha de Java.
\section{Equinoccial}
\begin{itemize}
\item {Grp. gram.:adj.}
\end{itemize}
\begin{itemize}
\item {Proveniência:(Lat. \textunderscore aequinoctialis\textunderscore )}
\end{itemize}
Relativo ao equinóccio.
\section{Equinóccio}
\begin{itemize}
\item {Grp. gram.:m.}
\end{itemize}
\begin{itemize}
\item {Proveniência:(Lat. \textunderscore aequinoctium\textunderscore )}
\end{itemize}
Ponto ou momento, em que o sol, descrevendo a eclíptica, corta o equador, fazendo que o dia seja igual á noite em toda a terra.
\section{Equinocial}
\begin{itemize}
\item {Grp. gram.:adj.}
\end{itemize}
\begin{itemize}
\item {Proveniência:(Lat. \textunderscore aequinoctialis\textunderscore )}
\end{itemize}
Relativo ao equinócio.
\section{Equinócio}
\begin{itemize}
\item {Grp. gram.:m.}
\end{itemize}
\begin{itemize}
\item {Proveniência:(Lat. \textunderscore aequinoctium\textunderscore )}
\end{itemize}
Ponto ou momento, em que o sol, descrevendo a eclíptica, corta o equador, fazendo que o dia seja igual á noite em toda a terra.
\section{Equinococo}
\begin{itemize}
\item {Grp. gram.:m.}
\end{itemize}
\begin{itemize}
\item {Proveniência:(Do gr. \textunderscore ekhinos\textunderscore  + \textunderscore kokkos\textunderscore )}
\end{itemize}
Entozoário, que se encontra nos hidátides.
\section{Equinodermes}
\begin{itemize}
\item {Grp. gram.:m. pl.}
\end{itemize}
\begin{itemize}
\item {Utilização:Zool.}
\end{itemize}
\begin{itemize}
\item {Proveniência:(Do gr. \textunderscore ekhinos\textunderscore  + \textunderscore derma\textunderscore )}
\end{itemize}
Animaes, que têm a pele coberta de tubérculos ou espinhos.
\section{Equinófora}
\begin{itemize}
\item {Grp. gram.:f.}
\end{itemize}
\begin{itemize}
\item {Proveniência:(De \textunderscore equinóforo\textunderscore )}
\end{itemize}
Espécie de molusco de concha raiada.
\section{Equinóforo}
\begin{itemize}
\item {Grp. gram.:adj.}
\end{itemize}
\begin{itemize}
\item {Utilização:Bot.}
\end{itemize}
\begin{itemize}
\item {Proveniência:(Do gr. \textunderscore ekhinos\textunderscore  + \textunderscore phoros\textunderscore )}
\end{itemize}
Que tem espinhos.
\section{Equinoftalmia}
\begin{itemize}
\item {Grp. gram.:f.}
\end{itemize}
\begin{itemize}
\item {Utilização:Med.}
\end{itemize}
\begin{itemize}
\item {Proveniência:(Do gr. \textunderscore ekhinos\textunderscore  + \textunderscore ophthalmos\textunderscore )}
\end{itemize}
Inflamação das pálpebras, na parte ocupada pelas pestanas.
\section{Equinoides}
\begin{itemize}
\item {Grp. gram.:m. pl.}
\end{itemize}
\begin{itemize}
\item {Utilização:Zool.}
\end{itemize}
\begin{itemize}
\item {Proveniência:(Do gr. \textunderscore ekhinos\textunderscore  + \textunderscore eidos\textunderscore )}
\end{itemize}
Espécie de equinoderme, a que pertence o ouriço-do-mar.
\section{Equinólitho}
\begin{itemize}
\item {Grp. gram.:m.}
\end{itemize}
Substância mineral, da América.
\section{Equinólito}
\begin{itemize}
\item {Grp. gram.:m.}
\end{itemize}
Substância mineral, da América.
\section{Equinómetra}
\begin{itemize}
\item {Grp. gram.:m.}
\end{itemize}
\begin{itemize}
\item {Proveniência:(Gr. \textunderscore ekhinometrai\textunderscore )}
\end{itemize}
Espécie de ouriço-do-mar.
\section{Equinómetro}
\begin{itemize}
\item {Grp. gram.:m.}
\end{itemize}
O mesmo que \textunderscore equinómetra\textunderscore .
\section{Equinópode}
\begin{itemize}
\item {Grp. gram.:m.}
\end{itemize}
\begin{itemize}
\item {Utilização:Bot.}
\end{itemize}
\begin{itemize}
\item {Proveniência:(Do gr. \textunderscore ekhinos\textunderscore  + \textunderscore pous\textunderscore , \textunderscore podos\textunderscore )}
\end{itemize}
Gênero de plantas vivazes das regiões quentes da Europa.
\section{Equinópseas}
\begin{itemize}
\item {Grp. gram.:m. pl.}
\end{itemize}
O mesmo que \textunderscore equinopsídeas\textunderscore .
\section{Equinopsídeas}
\begin{itemize}
\item {Grp. gram.:m. pl.}
\end{itemize}
\begin{itemize}
\item {Utilização:Bot.}
\end{itemize}
Grupo de vegetaes, da fam. das sinantéreas.
\section{Equinorrinco}
\begin{itemize}
\item {Grp. gram.:m.}
\end{itemize}
\begin{itemize}
\item {Proveniência:(Do gr. \textunderscore ekhinos\textunderscore  + \textunderscore rhunkos\textunderscore )}
\end{itemize}
Entozoário, que se encontra em alguns animaes e não no homem.
\section{Equinospermo}
\begin{itemize}
\item {Grp. gram.:adj.}
\end{itemize}
\begin{itemize}
\item {Utilização:Bot.}
\end{itemize}
\begin{itemize}
\item {Proveniência:(Do gr. \textunderscore ekhinos\textunderscore  + \textunderscore sperma\textunderscore )}
\end{itemize}
Cujos grãos são cobertos de pelos ásperos.
\section{Equioglossa}
\begin{itemize}
\item {Grp. gram.:f.}
\end{itemize}
\begin{itemize}
\item {Proveniência:(Do gr. \textunderscore ekhis\textunderscore , vibora, e \textunderscore glossa\textunderscore , língua)}
\end{itemize}
Gênero de orquídeas.
\section{Equioide}
\begin{itemize}
\item {Grp. gram.:adj.}
\end{itemize}
\begin{itemize}
\item {Grp. gram.:M.}
\end{itemize}
\begin{itemize}
\item {Proveniência:(Do gr. \textunderscore ekhis\textunderscore  + \textunderscore eidos\textunderscore )}
\end{itemize}
Semelhante á víbora ou á cabeça da víbora.
Planta, cuja semente é semelhante á cabeça da víbora.
\section{Equipagem}
\begin{itemize}
\item {Grp. gram.:f.}
\end{itemize}
\begin{itemize}
\item {Proveniência:(De \textunderscore equipar\textunderscore )}
\end{itemize}
Conjunto de pessoal, empregado na manobra e serviço de um navio.
Aprestos.
Bagagem.
Comitiva.
Trem de exército.
\section{Equipamento}
\begin{itemize}
\item {Grp. gram.:m.}
\end{itemize}
Equipagem.
Acto de \textunderscore equipar\textunderscore .
\section{Equipar}
\begin{itemize}
\item {Grp. gram.:v. t.}
\end{itemize}
\begin{itemize}
\item {Proveniência:(Fr. \textunderscore equiper\textunderscore )}
\end{itemize}
Pôr a bordo de (um navio) o que lhe é necessário para a manobra, defesa, sustentação do pessoal, etc.
Fornecer fardamento ou armamento a.
\section{Equiparação}
\begin{itemize}
\item {fónica:cu-i}
\end{itemize}
\begin{itemize}
\item {Grp. gram.:f.}
\end{itemize}
Acto ou effeito de \textunderscore equiparar\textunderscore .
\section{Equiparar}
\begin{itemize}
\item {fónica:cu-i}
\end{itemize}
\begin{itemize}
\item {Grp. gram.:v. t.}
\end{itemize}
\begin{itemize}
\item {Proveniência:(Lat. \textunderscore aequiparare\textunderscore )}
\end{itemize}
Igualar, comparando; igualar.
\section{Equiparável}
\begin{itemize}
\item {fónica:cu-i}
\end{itemize}
\begin{itemize}
\item {Grp. gram.:adj.}
\end{itemize}
Que se póde equiparar.
\section{Equiparencia}
\begin{itemize}
\item {fónica:cu-i}
\end{itemize}
\begin{itemize}
\item {Grp. gram.:f.}
\end{itemize}
O mesmo que \textunderscore equiparação\textunderscore .
\section{Equípede}
\begin{itemize}
\item {fónica:cu-i}
\end{itemize}
\begin{itemize}
\item {Grp. gram.:adj.}
\end{itemize}
\begin{itemize}
\item {Proveniência:(Do lat. \textunderscore aequus\textunderscore  + \textunderscore pes\textunderscore )}
\end{itemize}
Que tem as patas de igual comprimento.
\section{Equipendência}
\begin{itemize}
\item {fónica:cu-i}
\end{itemize}
\begin{itemize}
\item {Grp. gram.:f.}
\end{itemize}
Qualidade daquillo que é equipendente.
\section{Equipendente}
\begin{itemize}
\item {Grp. gram.:adj.}
\end{itemize}
\begin{itemize}
\item {Proveniência:(De \textunderscore equi...\textunderscore  + \textunderscore pendente\textunderscore )}
\end{itemize}
Igual, equilibrado.
\section{Equipo}
\begin{itemize}
\item {Grp. gram.:m.}
\end{itemize}
\begin{itemize}
\item {Utilização:Neol.}
\end{itemize}
\begin{itemize}
\item {Proveniência:(Fr. \textunderscore équipe\textunderscore )}
\end{itemize}
Grupo de dois ou mais indivíduos, que montam conjuntamente o mesmo velocípede.
\section{Equipolência}
\begin{itemize}
\item {fónica:cu-i}
\end{itemize}
\begin{itemize}
\item {Grp. gram.:f.}
\end{itemize}
Qualidade daquillo que é equipolente.
\section{Equipolente}
\begin{itemize}
\item {fónica:cu-i}
\end{itemize}
\begin{itemize}
\item {Grp. gram.:adj.}
\end{itemize}
\begin{itemize}
\item {Proveniência:(Lat. \textunderscore aequipollens\textunderscore )}
\end{itemize}
O mesmo que \textunderscore equivalente\textunderscore .
\section{Equipollência}
\begin{itemize}
\item {fónica:cu-i}
\end{itemize}
\begin{itemize}
\item {Grp. gram.:f.}
\end{itemize}
Qualidade daquillo que é equipolente.
\section{Equipollente}
\begin{itemize}
\item {fónica:cu-i}
\end{itemize}
\begin{itemize}
\item {Grp. gram.:adj.}
\end{itemize}
\begin{itemize}
\item {Proveniência:(Lat. \textunderscore aequipollens\textunderscore )}
\end{itemize}
O mesmo que \textunderscore equivalente\textunderscore .
\section{Equiponderância}
\begin{itemize}
\item {fónica:cu-i}
\end{itemize}
\begin{itemize}
\item {Grp. gram.:f.}
\end{itemize}
Qualidade daquelle ou daquillo que é equiponderante.
\section{Equiponderante}
\begin{itemize}
\item {fónica:cu-i}
\end{itemize}
\begin{itemize}
\item {Grp. gram.:adj.}
\end{itemize}
\begin{itemize}
\item {Proveniência:(De \textunderscore equiponderar\textunderscore )}
\end{itemize}
Que tem pêso igual.
Equilibrado.
\section{Equiponderar}
\begin{itemize}
\item {fónica:cu-i}
\end{itemize}
\begin{itemize}
\item {Grp. gram.:v. t.}
\end{itemize}
\begin{itemize}
\item {Grp. gram.:V. i.}
\end{itemize}
\begin{itemize}
\item {Proveniência:(De \textunderscore equi...\textunderscore  + \textunderscore ponderar\textunderscore )}
\end{itemize}
Contrabalançar; equilibrar.
Têr peso igual; equilibrar-se.
\section{Equírias}
\begin{itemize}
\item {fónica:cu-i}
\end{itemize}
\begin{itemize}
\item {Grp. gram.:f. pl.}
\end{itemize}
\begin{itemize}
\item {Proveniência:(Lat. \textunderscore equiria\textunderscore )}
\end{itemize}
Corridas de cavallos, com que em Roma se festejava Marte.
\section{Equírios}
\begin{itemize}
\item {fónica:cu-i}
\end{itemize}
\begin{itemize}
\item {Grp. gram.:m. pl.}
\end{itemize}
\begin{itemize}
\item {Proveniência:(Lat. \textunderscore equiria\textunderscore )}
\end{itemize}
Corridas de cavallos, com que em Roma se festejava Marte.
\section{Equisetáceas}
\begin{itemize}
\item {fónica:cu-i,se}
\end{itemize}
\begin{itemize}
\item {Grp. gram.:f. pl.}
\end{itemize}
Família de plantas acotyledóneas, que têm por typo o equiseto.
\section{Equisetíneas}
\begin{itemize}
\item {fónica:cu-i,se}
\end{itemize}
\begin{itemize}
\item {Grp. gram.:f. pl.}
\end{itemize}
(V.equisetáceas)
\section{Equiseto}
\begin{itemize}
\item {fónica:cu-i,se}
\end{itemize}
\begin{itemize}
\item {Grp. gram.:m.}
\end{itemize}
\begin{itemize}
\item {Utilização:Bot.}
\end{itemize}
\begin{itemize}
\item {Proveniência:(Lat. \textunderscore equisetum\textunderscore )}
\end{itemize}
Cavallinha, espécie de fêto.
\section{Equisonância}
\begin{itemize}
\item {fónica:cu-i,so}
\end{itemize}
\begin{itemize}
\item {Grp. gram.:f.}
\end{itemize}
\begin{itemize}
\item {Proveniência:(De \textunderscore equi...\textunderscore  + \textunderscore sonância\textunderscore )}
\end{itemize}
Consonância de dois sons semelhantes.
\section{Equisonante}
\begin{itemize}
\item {fónica:so}
\end{itemize}
\begin{itemize}
\item {Grp. gram.:adj.}
\end{itemize}
Em que há equisonância.
\section{Equísono}
\begin{itemize}
\item {fónica:so}
\end{itemize}
\begin{itemize}
\item {Grp. gram.:m.}
\end{itemize}
\begin{itemize}
\item {Utilização:Mús.}
\end{itemize}
\begin{itemize}
\item {Proveniência:(Do lat. \textunderscore aequisonus\textunderscore )}
\end{itemize}
Reunião de dois sons iguaes.
Unísono.
\section{Equissetáceas}
\begin{itemize}
\item {fónica:cu-i}
\end{itemize}
\begin{itemize}
\item {Grp. gram.:f. pl.}
\end{itemize}
Família de plantas acotiledóneas, que têm por tipo o equiseto.
\section{Equissetíneas}
\begin{itemize}
\item {fónica:cu-i}
\end{itemize}
\begin{itemize}
\item {Grp. gram.:f. pl.}
\end{itemize}
(V.equissetáceas)
\section{Equisseto}
\begin{itemize}
\item {fónica:cu-i}
\end{itemize}
\begin{itemize}
\item {Grp. gram.:m.}
\end{itemize}
\begin{itemize}
\item {Utilização:Bot.}
\end{itemize}
\begin{itemize}
\item {Proveniência:(Lat. \textunderscore equisetum\textunderscore )}
\end{itemize}
Cavalinha, espécie de fêto.
\section{Equíssimo}
\begin{itemize}
\item {fónica:cu-i}
\end{itemize}
\begin{itemize}
\item {Grp. gram.:adj.}
\end{itemize}
\begin{itemize}
\item {Utilização:Ant.}
\end{itemize}
Amantíssimo da justiça, da equidade.
(Sup. de \textunderscore équo\textunderscore )
\section{Equissonância}
\begin{itemize}
\item {fónica:cu-i}
\end{itemize}
\begin{itemize}
\item {Grp. gram.:f.}
\end{itemize}
\begin{itemize}
\item {Proveniência:(De \textunderscore equi...\textunderscore  + \textunderscore sonância\textunderscore )}
\end{itemize}
Consonância de dois sons semelhantes.
\section{Equissonante}
\begin{itemize}
\item {fónica:cu-i}
\end{itemize}
\begin{itemize}
\item {Grp. gram.:adj.}
\end{itemize}
Em que há equissonância.
\section{Equíssono}
\begin{itemize}
\item {fónica:cu-i}
\end{itemize}
\begin{itemize}
\item {Grp. gram.:m.}
\end{itemize}
\begin{itemize}
\item {Utilização:Mús.}
\end{itemize}
\begin{itemize}
\item {Proveniência:(Do lat. \textunderscore aequisonus\textunderscore )}
\end{itemize}
Reunião de dois sons iguaes.
Uníssono.
\section{Equitação}
\begin{itemize}
\item {Grp. gram.:f.}
\end{itemize}
\begin{itemize}
\item {Proveniência:(Lat. \textunderscore equitatio\textunderscore )}
\end{itemize}
Arte de cavalgar.
\section{Equitador}
\begin{itemize}
\item {Grp. gram.:m.}
\end{itemize}
\begin{itemize}
\item {Utilização:Neol.}
\end{itemize}
\begin{itemize}
\item {Proveniência:(Do lat. \textunderscore equitare\textunderscore )}
\end{itemize}
Aquelle que sabe equitação.
Bom cavalleiro.
\section{Equitativo}
\begin{itemize}
\item {fónica:cu-i}
\end{itemize}
\begin{itemize}
\item {Grp. gram.:adj.}
\end{itemize}
\begin{itemize}
\item {Proveniência:(Do lat. \textunderscore aequitas\textunderscore )}
\end{itemize}
Em que há equidade.
Que tem equidade.
\section{Equivalência}
\begin{itemize}
\item {fónica:cu-i}
\end{itemize}
\begin{itemize}
\item {Grp. gram.:f.}
\end{itemize}
Qualidade daquillo que é equivalente.
\section{Equivalente}
\begin{itemize}
\item {fónica:cu-i}
\end{itemize}
\begin{itemize}
\item {Grp. gram.:adj.}
\end{itemize}
\begin{itemize}
\item {Grp. gram.:M.}
\end{itemize}
\begin{itemize}
\item {Proveniência:(Lat. \textunderscore aequivalens\textunderscore )}
\end{itemize}
Que tem valor ou preço igual.
Aquillo que equivale.
\section{Equivaler}
\begin{itemize}
\item {fónica:cu-i}
\end{itemize}
\begin{itemize}
\item {Grp. gram.:v. i.}
\end{itemize}
\begin{itemize}
\item {Proveniência:(Lat. \textunderscore aequivalere\textunderscore )}
\end{itemize}
Sêr igual no valor, no preço, etc.
\section{Equivalve}
\begin{itemize}
\item {fónica:cu-i}
\end{itemize}
\begin{itemize}
\item {Grp. gram.:adj.}
\end{itemize}
\begin{itemize}
\item {Proveniência:(De \textunderscore equi...\textunderscore  + \textunderscore valva\textunderscore )}
\end{itemize}
Que tem duas valvas iguaes.
\section{Equivocação}
\begin{itemize}
\item {Grp. gram.:f.}
\end{itemize}
Acto ou effeito de equivocar.
Equívoco.
\section{Equivocadamente}
\begin{itemize}
\item {Grp. gram.:adj.}
\end{itemize}
\begin{itemize}
\item {Proveniência:(De \textunderscore equivocar\textunderscore )}
\end{itemize}
De modo equívoco.
\section{Equivocamente}
\begin{itemize}
\item {Grp. gram.:adv.}
\end{itemize}
O mesmo que \textunderscore equivocadamente\textunderscore .
\section{Equivocar}
\begin{itemize}
\item {Grp. gram.:v. t.}
\end{itemize}
\begin{itemize}
\item {Grp. gram.:V. p.}
\end{itemize}
\begin{itemize}
\item {Proveniência:(De \textunderscore equívoco\textunderscore )}
\end{itemize}
Induzir em engano.
Illudir.
Confundir-se.
Enganar-se.
Dizer uma coisa por outra.
\section{Equívoco}
\begin{itemize}
\item {Grp. gram.:adj.}
\end{itemize}
\begin{itemize}
\item {Grp. gram.:M.}
\end{itemize}
\begin{itemize}
\item {Proveniência:(Lat. \textunderscore aequivocus\textunderscore )}
\end{itemize}
Que dá lugar a várias interpretações.
Contestável.
Duvidoso.
Que é objecto de suspeita: \textunderscore criatura equívoca\textunderscore .
Engano; illusão.
Interpretação equívoca.
Effeito de equivocar-se; trocadilho, jôgo de palavras.
\section{Equivoquista}
\begin{itemize}
\item {Grp. gram.:m.}
\end{itemize}
\begin{itemize}
\item {Proveniência:(De \textunderscore equívoco\textunderscore )}
\end{itemize}
Aquelle que faz equívocos ou gosta de equívocos na linguagem.
\section{Équo}
\begin{itemize}
\item {Grp. gram.:adj.}
\end{itemize}
\begin{itemize}
\item {Utilização:Des.}
\end{itemize}
\begin{itemize}
\item {Proveniência:(Lat. \textunderscore aequus\textunderscore )}
\end{itemize}
Justo, bom.
\section{Equóreo}
\begin{itemize}
\item {fónica:cu-o}
\end{itemize}
\begin{itemize}
\item {Grp. gram.:adj.}
\end{itemize}
\begin{itemize}
\item {Proveniência:(Lat. \textunderscore aequoreus\textunderscore )}
\end{itemize}
Relativo ao alto mar.
\section{Équos}
\begin{itemize}
\item {Grp. gram.:m. pl.}
\end{itemize}
\begin{itemize}
\item {Proveniência:(Lat. \textunderscore aequi\textunderscore )}
\end{itemize}
Povos antigos da Itália, que habitavam as margens do Ánio e foram destruídos por Cincinnato.
\section{Equúleo}
\begin{itemize}
\item {Grp. gram.:m.}
\end{itemize}
(V.ecúleo)
\section{Er}
\begin{itemize}
\item {Grp. gram.:adv.}
\end{itemize}
\begin{itemize}
\item {Utilização:Ant.}
\end{itemize}
Também.
Aliás. Cf. Gil Vicente, I, 167 e 172.
\section{Era}
\begin{itemize}
\item {Grp. gram.:f.}
\end{itemize}
\begin{itemize}
\item {Utilização:Ext.}
\end{itemize}
\begin{itemize}
\item {Proveniência:(Lat. \textunderscore aera\textunderscore )}
\end{itemize}
Acontecimento ou época fixa, que serve de base a um systema chronológico.
Série de annos, que principia num grande acontentecimento histórico: \textunderscore a era de Christo\textunderscore .
Época notável.
Início de uma nova ordem de coisas: \textunderscore estamos em nova era\textunderscore .
\section{Eradicar}
\textunderscore v. t.\textunderscore  (e der.)
(V. \textunderscore erradicar\textunderscore , etc.)
\section{Erado}
\begin{itemize}
\item {Grp. gram.:adj.}
\end{itemize}
\begin{itemize}
\item {Utilização:Bras. do N}
\end{itemize}
\begin{itemize}
\item {Proveniência:(De \textunderscore era\textunderscore ?)}
\end{itemize}
Diz-se do boi, que completou quatro annos de idade.
\section{Eramá}
\begin{itemize}
\item {Grp. gram.:adv.}
\end{itemize}
Em má hora:«\textunderscore fartos eramá de pão.\textunderscore »G. Vicente, I, 15.
(Corr. de \textunderscore em hora má\textunderscore )
\section{Erândi}
\begin{itemize}
\item {Grp. gram.:m.}
\end{itemize}
Planta intertropical, (\textunderscore ricinus communis\textunderscore ).
\section{Erândy}
\begin{itemize}
\item {Grp. gram.:m.}
\end{itemize}
Planta intertropical, (\textunderscore ricinus communis\textunderscore ).
\section{Erântemo}
\begin{itemize}
\item {Grp. gram.:m.}
\end{itemize}
\begin{itemize}
\item {Proveniência:(Lat. \textunderscore eranthemon\textunderscore )}
\end{itemize}
Planta ranunculácea.
Designação antiga da camomila.
\section{Erânthemo}
\begin{itemize}
\item {Grp. gram.:m.}
\end{itemize}
\begin{itemize}
\item {Proveniência:(Lat. \textunderscore eranthemon\textunderscore )}
\end{itemize}
Planta ranunculácea.
Designação antiga da camomila.
\section{Erânthis}
\begin{itemize}
\item {Grp. gram.:m.}
\end{itemize}
O memso que \textunderscore erânthemo\textunderscore .
\section{Erântis}
\begin{itemize}
\item {Grp. gram.:m.}
\end{itemize}
O memso que \textunderscore erântemo\textunderscore .
\section{Erário}
\begin{itemize}
\item {Grp. gram.:m.}
\end{itemize}
\begin{itemize}
\item {Proveniência:(Lat. \textunderscore aerarium\textunderscore )}
\end{itemize}
Thesoiro público.
\section{Erarta}
\begin{itemize}
\item {Grp. gram.:f.}
\end{itemize}
Gênero de plantas gramíneas, originárias do Cabo da Bôa-Esperança.
\section{Erasmiano}
\begin{itemize}
\item {Grp. gram.:adj.}
\end{itemize}
O mesmo que \textunderscore erásmico\textunderscore .
\section{Erásmico}
\begin{itemize}
\item {Grp. gram.:adj.}
\end{itemize}
\begin{itemize}
\item {Proveniência:(De \textunderscore Erasmo\textunderscore , n. p.)}
\end{itemize}
Relativo a Erasmo.
Diz-se da pronúncia do grego, em que se distinguem as duas vogaes de um ditongo, em que o úpsilon sôa como \textunderscore u\textunderscore , etc., ao contrário da pronúncia reuchliniana.
\section{Erastianismo}
\begin{itemize}
\item {Grp. gram.:m.}
\end{itemize}
Seita dos erastianos.
\section{Erastianos}
\begin{itemize}
\item {Grp. gram.:m. pl.}
\end{itemize}
\begin{itemize}
\item {Proveniência:(De \textunderscore Eraste\textunderscore , n. p.)}
\end{itemize}
Sectários dos que sustentavam que a igreja anglicana não tem o poder de excommungar.
\section{Erbabo}
\begin{itemize}
\item {Grp. gram.:m.}
\end{itemize}
\begin{itemize}
\item {Utilização:Des.}
\end{itemize}
\begin{itemize}
\item {Proveniência:(Do ár. \textunderscore arrabab\textunderscore )}
\end{itemize}
O mesmo que \textunderscore arrabil\textunderscore .
\section{Érbio}
\begin{itemize}
\item {Grp. gram.:m.}
\end{itemize}
\begin{itemize}
\item {Proveniência:(De \textunderscore Ytterby\textunderscore , n. p.)}
\end{itemize}
Corpo mineral, raro, o mesmo que \textunderscore ýttrio\textunderscore .
\section{Ercila}
\begin{itemize}
\item {Grp. gram.:f.}
\end{itemize}
Gênero, de plantas fitoláceas.
\section{Ercilla}
\begin{itemize}
\item {Grp. gram.:f.}
\end{itemize}
Gênero, de plantas phytoláceas.
\section{Érdimo}
\begin{itemize}
\item {Grp. gram.:m.}
\end{itemize}
\begin{itemize}
\item {Utilização:Prov.}
\end{itemize}
\begin{itemize}
\item {Utilização:dur.}
\end{itemize}
O mesmo que \textunderscore herança\textunderscore .
(Cp. \textunderscore herdar\textunderscore )
\section{Érebo}
\begin{itemize}
\item {Grp. gram.:m.}
\end{itemize}
\begin{itemize}
\item {Proveniência:(Lat. \textunderscore erebus\textunderscore )}
\end{itemize}
Inferno.
\section{Erecção}
\begin{itemize}
\item {Grp. gram.:f.}
\end{itemize}
\begin{itemize}
\item {Proveniência:(Lat. \textunderscore erectio\textunderscore )}
\end{itemize}
Acto de erguer.
Inauguração.
Instituição.
Excitação do pênis; erethismo; orgasmo.
\section{Eréctil}
\begin{itemize}
\item {Grp. gram.:adj.}
\end{itemize}
\begin{itemize}
\item {Proveniência:(Do lat. \textunderscore erectus\textunderscore )}
\end{itemize}
Que é susceptível de erecção.
\section{Erectilidade}
\begin{itemize}
\item {Grp. gram.:f.}
\end{itemize}
Qualidade daquillo que é eréctil.
\section{Erecto}
\begin{itemize}
\item {Grp. gram.:adj.}
\end{itemize}
\begin{itemize}
\item {Proveniência:(Lat. \textunderscore erectus\textunderscore )}
\end{itemize}
Direito; aprumado; altivo.
\section{Erector}
\begin{itemize}
\item {Grp. gram.:adj.}
\end{itemize}
\begin{itemize}
\item {Proveniência:(Lat. \textunderscore erector\textunderscore )}
\end{itemize}
Que causa erecção.
\section{Ereita}
\begin{itemize}
\item {Grp. gram.:f.}
\end{itemize}
\begin{itemize}
\item {Utilização:Des.}
\end{itemize}
\begin{itemize}
\item {Utilização:Ant.}
\end{itemize}
\begin{itemize}
\item {Proveniência:(Do lat. \textunderscore erectus\textunderscore )}
\end{itemize}
Ardil de lutador, que levanta o adversário para o derrubar.
Altura, teso de um monte.
\section{Erêmia}
\begin{itemize}
\item {Grp. gram.:f.}
\end{itemize}
Gênero de plantas ericáceas.
\section{Eremícola}
\begin{itemize}
\item {Grp. gram.:m.  e  f.}
\end{itemize}
\begin{itemize}
\item {Utilização:Des.}
\end{itemize}
O mesmo que \textunderscore eremita\textunderscore .
\section{Eremita}
\begin{itemize}
\item {Grp. gram.:m.  e  f.}
\end{itemize}
\begin{itemize}
\item {Proveniência:(Lat. \textunderscore eremita\textunderscore )}
\end{itemize}
Pessôa, que vive no ermo, com intuitos religiosos.
Pessôa, que evita a convivência social.
\section{Eremita-bernardo}
\begin{itemize}
\item {Grp. gram.:m.}
\end{itemize}
Pequeno crustáceo, que costuma alojar-se nos búzios de outros crustáceos.
\section{Eremitão}
\begin{itemize}
\item {Grp. gram.:m.}
\end{itemize}
O mesmo que \textunderscore ermitão\textunderscore . Cf. \textunderscore Peregrinação\textunderscore , LXVI.
\section{Eremitério}
\begin{itemize}
\item {Grp. gram.:m.}
\end{itemize}
\begin{itemize}
\item {Utilização:Ext.}
\end{itemize}
\begin{itemize}
\item {Proveniência:(De \textunderscore eremita\textunderscore )}
\end{itemize}
Lugar, onde vive um eremita.
Abrigo de eremitas.
Lugar afastado de povoação.
\section{Eremítico}
\begin{itemize}
\item {Grp. gram.:adj.}
\end{itemize}
\begin{itemize}
\item {Proveniência:(De \textunderscore eremita\textunderscore )}
\end{itemize}
Relativo a eremita ou á vida do ermo.
Relativo a eremita; próprio de eremita.
\section{Erêncio}
\begin{itemize}
\item {Grp. gram.:m.}
\end{itemize}
\begin{itemize}
\item {Utilização:Ant.}
\end{itemize}
O mesmo que \textunderscore herança\textunderscore .
\section{Éreo}
\begin{itemize}
\item {Grp. gram.:adj.}
\end{itemize}
\begin{itemize}
\item {Proveniência:(Lat. \textunderscore aereus\textunderscore )}
\end{itemize}
Que é de bronze, arame ou cobre.
\section{Erepsina}
\begin{itemize}
\item {Grp. gram.:f.}
\end{itemize}
Diástese, descoberta há pouco por Cohnheim no intestino e que transforma as albuminoses ou peptonas em corpos mais simples.
\section{Eres}
\begin{itemize}
\item {Grp. gram.:m. pl.}
\end{itemize}
\begin{itemize}
\item {Utilização:Ant.}
\end{itemize}
Adornos de tartaruga e plumagens no toucado das damas. Cf. Camillo, \textunderscore Caveira\textunderscore , 75.
\section{Erethismal}
\begin{itemize}
\item {Grp. gram.:adj.}
\end{itemize}
Relativo ao erethismo.
\section{Erethismo}
\begin{itemize}
\item {Grp. gram.:m.}
\end{itemize}
\begin{itemize}
\item {Proveniência:(Gr. \textunderscore erethismos\textunderscore )}
\end{itemize}
Estado de excitação ou de irritação.
\section{Erétia}
\begin{itemize}
\item {Grp. gram.:f.}
\end{itemize}
Gênero de plantas asperifoliáceas.
\section{Eretismal}
\begin{itemize}
\item {Grp. gram.:adj.}
\end{itemize}
Relativo ao eretismo.
\section{Eretismo}
\begin{itemize}
\item {Grp. gram.:m.}
\end{itemize}
\begin{itemize}
\item {Proveniência:(Gr. \textunderscore erethismos\textunderscore )}
\end{itemize}
Estado de excitação ou de irritação.
\section{Eretria}
\begin{itemize}
\item {Grp. gram.:f.}
\end{itemize}
\begin{itemize}
\item {Proveniência:(De \textunderscore Eretria\textunderscore , n. p.)}
\end{itemize}
Espécie de alvaiade.
\section{Ergasílios}
\begin{itemize}
\item {Grp. gram.:m. pl.}
\end{itemize}
\begin{itemize}
\item {Proveniência:(Do gr. \textunderscore ergasia\textunderscore )}
\end{itemize}
Tríbo de crustáceos.
\section{Ergastulário}
\begin{itemize}
\item {Grp. gram.:m.}
\end{itemize}
\begin{itemize}
\item {Proveniência:(Lat. \textunderscore ergastularius\textunderscore )}
\end{itemize}
Carcereiro, entre os Romanos.
\section{Ergástulo}
\begin{itemize}
\item {Grp. gram.:m.}
\end{itemize}
\begin{itemize}
\item {Proveniência:(Lat. \textunderscore ergastulum\textunderscore )}
\end{itemize}
Cárcere.
Enxovia.
\section{Érgata}
\begin{itemize}
\item {Grp. gram.:f.}
\end{itemize}
\begin{itemize}
\item {Proveniência:(Lat. \textunderscore ergates\textunderscore )}
\end{itemize}
Espécie de guindaste ou máquina, para levantar pesos, entre os antigos Romanos.
\section{Ergofobia}
\begin{itemize}
\item {Grp. gram.:f.}
\end{itemize}
\begin{itemize}
\item {Proveniência:(Do gr. \textunderscore ergon\textunderscore  + \textunderscore phobos\textunderscore )}
\end{itemize}
Aversão ao trabalho.
\section{Ergógrafo}
\begin{itemize}
\item {Grp. gram.:m.}
\end{itemize}
\begin{itemize}
\item {Utilização:Med.}
\end{itemize}
\begin{itemize}
\item {Proveniência:(Do gr. \textunderscore ergon\textunderscore  + \textunderscore graphein\textunderscore )}
\end{itemize}
Aparelho, para inscrever o trabalho de um músculo.
\section{Ergógrapho}
\begin{itemize}
\item {Grp. gram.:m.}
\end{itemize}
\begin{itemize}
\item {Utilização:Med.}
\end{itemize}
\begin{itemize}
\item {Proveniência:(Do gr. \textunderscore ergon\textunderscore  + \textunderscore graphein\textunderscore )}
\end{itemize}
Apparelho, para inscrever o trabalho de um músculo.
\section{Ergophobia}
\begin{itemize}
\item {Grp. gram.:f.}
\end{itemize}
\begin{itemize}
\item {Proveniência:(Do gr. \textunderscore ergon\textunderscore  + \textunderscore phobos\textunderscore )}
\end{itemize}
Aversão ao trabalho.
\section{Ergotina}
\begin{itemize}
\item {Grp. gram.:f.}
\end{itemize}
\begin{itemize}
\item {Proveniência:(Do fr. \textunderscore ergot\textunderscore )}
\end{itemize}
Alcaloide da espiga do centeio.
\section{Ergotismo}
\begin{itemize}
\item {Grp. gram.:m.}
\end{itemize}
\begin{itemize}
\item {Proveniência:(Do lat. \textunderscore ergo\textunderscore )}
\end{itemize}
Habito ou mania de disputar por syllogismos.
\section{Ergotismo}
\begin{itemize}
\item {Grp. gram.:m.}
\end{itemize}
\begin{itemize}
\item {Proveniência:(Do fr. \textunderscore ergot\textunderscore )}
\end{itemize}
Envenenamento chrónico, produzido pela cravagem do centeio.
\section{Erguer}
\begin{itemize}
\item {Grp. gram.:v. t.}
\end{itemize}
\begin{itemize}
\item {Proveniência:(Do lat. hyp. \textunderscore ergere\textunderscore )}
\end{itemize}
Erigir, levantar: \textunderscore erguer uma estátua\textunderscore .
Construir; fundar: \textunderscore erguer um palácio\textunderscore .
Fazer soar alto: \textunderscore erguer a voz\textunderscore .
Tornar superior, elevado.
Exaltar.
\section{Erguida}
\begin{itemize}
\item {Grp. gram.:f.}
\end{itemize}
Acto de erguer e amparar as varas novas das videiras.
Empa. Cf. Villarinho, \textunderscore Viticultura e Vinicultura\textunderscore .
\section{Ergulho}
\begin{itemize}
\item {Grp. gram.:m.}
\end{itemize}
\begin{itemize}
\item {Utilização:Ant.}
\end{itemize}
O mesmo que \textunderscore orgulho\textunderscore . Cf. Frei Fortun., \textunderscore Inéd.\textunderscore , I, 306.
(Infl. de \textunderscore erguer\textunderscore )
\section{Eriantho}
\begin{itemize}
\item {Grp. gram.:m.}
\end{itemize}
\begin{itemize}
\item {Proveniência:(Do gr. \textunderscore erion\textunderscore  + \textunderscore anthos\textunderscore )}
\end{itemize}
Gênero de plantas gramíneas.
\section{Erianto}
\begin{itemize}
\item {Grp. gram.:m.}
\end{itemize}
\begin{itemize}
\item {Proveniência:(Do gr. \textunderscore erion\textunderscore  + \textunderscore anthos\textunderscore )}
\end{itemize}
Gênero de plantas gramíneas.
\section{Erica}
\begin{itemize}
\item {Grp. gram.:f.}
\end{itemize}
(V.erice)
\section{Ericáceas}
\begin{itemize}
\item {Grp. gram.:f. pl.}
\end{itemize}
Família de plantas, que têm por typo a erice.
\section{Eriçar}
\begin{itemize}
\item {Grp. gram.:v. t.}
\end{itemize}
\begin{itemize}
\item {Proveniência:(Do lat. \textunderscore ericius\textunderscore )}
\end{itemize}
Encrespar, arripiar.
Ouriçar; erriçar; riçar.
\section{Erice}
\begin{itemize}
\item {Grp. gram.:f.}
\end{itemize}
\begin{itemize}
\item {Proveniência:(Lat. \textunderscore erice\textunderscore )}
\end{itemize}
Espécie de urze.
\section{Ericíneas}
\begin{itemize}
\item {Grp. gram.:f. pl.}
\end{itemize}
(V.ericáceas)
\section{Erício}
\begin{itemize}
\item {Grp. gram.:m.}
\end{itemize}
\begin{itemize}
\item {Proveniência:(Lat. \textunderscore ericius\textunderscore )}
\end{itemize}
Trincheira eriçada de pontas de ferro, entre os antigos Romanos.
\section{Erídano}
\begin{itemize}
\item {Grp. gram.:m.}
\end{itemize}
\begin{itemize}
\item {Proveniência:(Do lat. \textunderscore Eridanus\textunderscore , n. p.)}
\end{itemize}
Constellação meridional.
\section{Erigir}
\begin{itemize}
\item {Grp. gram.:v. i.}
\end{itemize}
\begin{itemize}
\item {Proveniência:(Lat. \textunderscore erigere\textunderscore )}
\end{itemize}
Erguer a prumo.
Levantar.
Construir, instituir.
Transformar, ou dar novo carácter a.
\section{Eril}
\begin{itemize}
\item {Grp. gram.:adj.}
\end{itemize}
O mesmo que \textunderscore éreo\textunderscore .
\section{Érina}
\begin{itemize}
\item {Grp. gram.:f.}
\end{itemize}
Instrumento cirúrgico, terminado em gancho, para apprehender.
(Cp. lat. \textunderscore erinaceus\textunderscore )
\section{Erináceo}
\begin{itemize}
\item {Grp. gram.:adj.}
\end{itemize}
\begin{itemize}
\item {Proveniência:(Lat. \textunderscore erinaceus\textunderscore )}
\end{itemize}
Semelhante ao ouriço.
\section{Erinacídeos}
\begin{itemize}
\item {Grp. gram.:m. pl.}
\end{itemize}
\begin{itemize}
\item {Proveniência:(Do lat. \textunderscore erinaceus\textunderscore )}
\end{itemize}
Família de mammíferos, que têm por typo o ouriço.
\section{Erinéu}
\begin{itemize}
\item {Grp. gram.:m.}
\end{itemize}
\begin{itemize}
\item {Proveniência:(Fr. \textunderscore erinée\textunderscore )}
\end{itemize}
Verme microscópico, que pica as fôlhas da videira, originando empôlas esbranquiçadas e redondas.
Doença das videiras, determinada por aquelle insecto.
\section{Erino}
\begin{itemize}
\item {Grp. gram.:m.}
\end{itemize}
\begin{itemize}
\item {Proveniência:(Do gr. \textunderscore erinon\textunderscore )}
\end{itemize}
Gênero de plantas escrofularineas.
\section{Erinose}
\begin{itemize}
\item {Grp. gram.:f.}
\end{itemize}
Doença das videiras, caracterizada por empôlas verdes, revestidas inferiormente de uma espécie de cotão.
(Do nome do cogumelo \textunderscore erineo vitis\textunderscore . V. \textunderscore erinéu\textunderscore )
\section{Eriócomo}
\begin{itemize}
\item {Grp. gram.:adj.}
\end{itemize}
\begin{itemize}
\item {Proveniência:(Do gr. \textunderscore erion\textunderscore  + \textunderscore kome\textunderscore )}
\end{itemize}
Que tem cabello encarapinhado.
\section{Eriodendro}
\begin{itemize}
\item {Grp. gram.:m.}
\end{itemize}
\begin{itemize}
\item {Proveniência:(Do gr. \textunderscore erion\textunderscore  + \textunderscore dendron\textunderscore )}
\end{itemize}
Gênero de plantas africanas. Cf. Capello e Ivens, II, 210.
\section{Erióforo}
\begin{itemize}
\item {Grp. gram.:m.}
\end{itemize}
\begin{itemize}
\item {Proveniência:(Do gr. \textunderscore erion\textunderscore  + \textunderscore phoros\textunderscore )}
\end{itemize}
Gênero de plantas ciperáceas.
\section{Erióphoro}
\begin{itemize}
\item {Grp. gram.:m.}
\end{itemize}
\begin{itemize}
\item {Proveniência:(Do gr. \textunderscore erion\textunderscore  + \textunderscore phoros\textunderscore )}
\end{itemize}
Gênero de plantas cyperáceas.
\section{Eripo}
\begin{itemize}
\item {Grp. gram.:m.}
\end{itemize}
\begin{itemize}
\item {Proveniência:(Gr. \textunderscore eripous\textunderscore )}
\end{itemize}
Insecto coleóptero pentâmero.
\section{Eriribá}
\begin{itemize}
\item {Grp. gram.:m.}
\end{itemize}
Árvore leguminosa do Brasil.
\section{Erizar}
\begin{itemize}
\item {Grp. gram.:v. t.}
\end{itemize}
\begin{itemize}
\item {Utilização:Ant.}
\end{itemize}
O mesmo que \textunderscore eriçar\textunderscore .
\section{Ermamento}
\begin{itemize}
\item {Grp. gram.:m.}
\end{itemize}
Acto de ermar.
Despovoamento.
\section{Ermar}
\begin{itemize}
\item {Grp. gram.:v. t.}
\end{itemize}
\begin{itemize}
\item {Grp. gram.:V. i.}
\end{itemize}
Tornar ermo, deserto.
Viver no ermo.
\section{Ermelo}
\begin{itemize}
\item {fónica:mê}
\end{itemize}
\begin{itemize}
\item {Grp. gram.:m.}
\end{itemize}
\begin{itemize}
\item {Utilização:Prov.}
\end{itemize}
\begin{itemize}
\item {Utilização:trasm.}
\end{itemize}
\begin{itemize}
\item {Proveniência:(De \textunderscore Ermelo\textunderscore , n. p. de um povo do Marão)}
\end{itemize}
Homem estupido.
\section{Ermida}
\begin{itemize}
\item {Grp. gram.:f.}
\end{itemize}
\begin{itemize}
\item {Proveniência:(De \textunderscore ermo\textunderscore )}
\end{itemize}
Capella, igreja pequena, fóra do povoado.
Pequena igreja.
\section{Ermita}
\begin{itemize}
\item {Grp. gram.:f.}
\end{itemize}
O mesmo que \textunderscore eremita\textunderscore .
\section{Ermitágio}
\begin{itemize}
\item {Grp. gram.:m.}
\end{itemize}
\begin{itemize}
\item {Utilização:Ant.}
\end{itemize}
O mesmo que \textunderscore ermitério\textunderscore .
\section{Ermitania}
\begin{itemize}
\item {Grp. gram.:f.}
\end{itemize}
Cargo de ermitão.
\section{Ermitão}
\begin{itemize}
\item {Grp. gram.:m.}
\end{itemize}
\begin{itemize}
\item {Proveniência:(De \textunderscore ermita\textunderscore )}
\end{itemize}
Eremita.
Aquelle que trata de uma ermida.
\section{Ermitério}
\begin{itemize}
\item {Grp. gram.:m.}
\end{itemize}
O mesmo que \textunderscore eremitério\textunderscore .
\section{Ermitôa}
\begin{itemize}
\item {Grp. gram.:f.}
\end{itemize}
\begin{itemize}
\item {Proveniência:(De \textunderscore ermitão\textunderscore )}
\end{itemize}
Mulher, que trata da limpeza e arranjos de uma ermida.
\section{Êrmo}
\begin{itemize}
\item {Grp. gram.:m.}
\end{itemize}
\begin{itemize}
\item {Grp. gram.:Adj.}
\end{itemize}
Lugar sem habitantes; deserto.
Descampado.
Solidão.
Solitário; desacompanhado: \textunderscore em lugar ermo\textunderscore .
(B. lat. \textunderscore ermus\textunderscore )
\section{Êrmo}
\begin{itemize}
\item {Grp. gram.:m.}
\end{itemize}
\begin{itemize}
\item {Utilização:Pop.}
\end{itemize}
Crosta escamosa, que se fórma na cabeça das crianças.
(Cp. \textunderscore elmo\textunderscore )
\section{Ernéstia}
\begin{itemize}
\item {Grp. gram.:f.}
\end{itemize}
\begin{itemize}
\item {Proveniência:(De \textunderscore Ernest\textunderscore , n. p.)}
\end{itemize}
Insecto díptero.
Planta da Nova-Granada.
\section{Ero}
\begin{itemize}
\item {Grp. gram.:m.}
\end{itemize}
\begin{itemize}
\item {Utilização:Ant.}
\end{itemize}
Herdade, dividida por marcos.
(Relaciona-se provavelmente com o lat. \textunderscore haeredium\textunderscore )
\section{Eró}
\begin{itemize}
\item {Utilização:Ant.}
\end{itemize}
O mesmo que \textunderscore eiró\textunderscore ^1.
\section{Eroca}
\begin{itemize}
\item {Grp. gram.:f.}
\end{itemize}
Planta medicinal aphrodisíaca.
\section{Erodente}
\begin{itemize}
\item {Grp. gram.:adj.}
\end{itemize}
\begin{itemize}
\item {Proveniência:(Lat. \textunderscore erodens\textunderscore )}
\end{itemize}
Que corrói; corrosivo.
\section{Eródio}
\begin{itemize}
\item {Grp. gram.:m.}
\end{itemize}
\begin{itemize}
\item {Proveniência:(Gr. \textunderscore erodios\textunderscore )}
\end{itemize}
Gênero de insectos coleópteros pentâmeros.
Gênero de plantas gramíneas.
\section{Erófilo}
\begin{itemize}
\item {Grp. gram.:m.}
\end{itemize}
\begin{itemize}
\item {Proveniência:(Do gr. \textunderscore eros\textunderscore  + \textunderscore philos\textunderscore )}
\end{itemize}
Gênero de plantas crucíferas.
\section{Erogêneo}
\begin{itemize}
\item {Grp. gram.:adj.}
\end{itemize}
O mesmo que \textunderscore erotóphoro\textunderscore .
\section{Erotogêneo}
\begin{itemize}
\item {Grp. gram.:adj.}
\end{itemize}
O mesmo que \textunderscore erotóphoro\textunderscore .
\section{Eróphilo}
\begin{itemize}
\item {Grp. gram.:m.}
\end{itemize}
\begin{itemize}
\item {Proveniência:(Do gr. \textunderscore eros\textunderscore  + \textunderscore philos\textunderscore )}
\end{itemize}
Gênero de plantas crucíferas.
\section{Erosão}
\begin{itemize}
\item {Grp. gram.:f.}
\end{itemize}
\begin{itemize}
\item {Proveniência:(Lat. \textunderscore eresio\textunderscore )}
\end{itemize}
Acto ou effeito de carcomer, de corroer lentamente.
\section{Erosivo}
\begin{itemize}
\item {Grp. gram.:adj.}
\end{itemize}
\begin{itemize}
\item {Proveniência:(Do lat. \textunderscore erosus\textunderscore )}
\end{itemize}
O mesmo que \textunderscore erodente\textunderscore .
\section{Erotemática}
\begin{itemize}
\item {Grp. gram.:f.}
\end{itemize}
\begin{itemize}
\item {Proveniência:(De \textunderscore erotemático\textunderscore )}
\end{itemize}
Systema pedagógico, que consiste em interrogações, depois da exposição da doutrina.
\section{Erotemático}
\begin{itemize}
\item {Grp. gram.:adj.}
\end{itemize}
\begin{itemize}
\item {Proveniência:(Do gr. \textunderscore erotema\textunderscore , interrogação)}
\end{itemize}
Relativo a interrogações.
Que procede por meio de preguntas.
Reduzido a preguntas.
\section{Eroticamente}
\begin{itemize}
\item {Grp. gram.:adv.}
\end{itemize}
De modo erótico.
\section{Eróticas}
\begin{itemize}
\item {Grp. gram.:f. pl.}
\end{itemize}
\begin{itemize}
\item {Proveniência:(De \textunderscore erótico\textunderscore )}
\end{itemize}
Poesias relativas ao amor.
\section{Erótico}
\begin{itemize}
\item {Grp. gram.:adj.}
\end{itemize}
\begin{itemize}
\item {Proveniência:(Gr. \textunderscore erotikos\textunderscore )}
\end{itemize}
Relativo ao amor; sensual.
\section{Erótilo}
\begin{itemize}
\item {Grp. gram.:m.}
\end{itemize}
\begin{itemize}
\item {Proveniência:(Gr. \textunderscore erotulos\textunderscore )}
\end{itemize}
Insecto coleóptero herbívoro.
\section{Erotismo}
\begin{itemize}
\item {Grp. gram.:m.}
\end{itemize}
\begin{itemize}
\item {Proveniência:(De \textunderscore erótico\textunderscore )}
\end{itemize}
Paixão amorosa.
Amor lascivo.
\section{Erotóforo}
\begin{itemize}
\item {Grp. gram.:adj.}
\end{itemize}
\begin{itemize}
\item {Proveniência:(Do gr. \textunderscore eros\textunderscore  + \textunderscore phoros\textunderscore )}
\end{itemize}
Que produz ou suscita amor.
\section{Eròtomania}
\begin{itemize}
\item {Grp. gram.:f.}
\end{itemize}
\begin{itemize}
\item {Proveniência:(Do gr. \textunderscore eros\textunderscore  + \textunderscore mania\textunderscore )}
\end{itemize}
Loucura produzida pelo amor ou em que há delírio erótico.
\section{Erotomaníaco}
\begin{itemize}
\item {Grp. gram.:m.}
\end{itemize}
Aquelle que soffre erotomania.
\section{Erotómano}
\begin{itemize}
\item {Grp. gram.:m.}
\end{itemize}
Aquelle que soffre erotomania.
\section{Eretopégnio}
\begin{itemize}
\item {Grp. gram.:m.}
\end{itemize}
\begin{itemize}
\item {Proveniência:(Do gr. \textunderscore eros\textunderscore  + \textunderscore paignion\textunderscore )}
\end{itemize}
Poema erótico, entre os Gregos.
\section{Erotóphoro}
\begin{itemize}
\item {Grp. gram.:adj.}
\end{itemize}
\begin{itemize}
\item {Proveniência:(Do gr. \textunderscore eros\textunderscore  + \textunderscore phoros\textunderscore )}
\end{itemize}
Que produz ou suscita amor.
\section{Erótylo}
\begin{itemize}
\item {Grp. gram.:m.}
\end{itemize}
\begin{itemize}
\item {Proveniência:(Gr. \textunderscore erotulos\textunderscore )}
\end{itemize}
Insecto coleóptero herbívoro.
\section{Erpetão}
\begin{itemize}
\item {Grp. gram.:m.}
\end{itemize}
\begin{itemize}
\item {Proveniência:(Do gr. \textunderscore erpetos\textunderscore )}
\end{itemize}
Gênero de reptis ophídios.
\section{Erpetografia}
\begin{itemize}
\item {Grp. gram.:f.}
\end{itemize}
\begin{itemize}
\item {Proveniência:(De \textunderscore erpetografo\textunderscore )}
\end{itemize}
Tratado especial, á cêrca dos reptis.
\section{Erpetográfico}
\begin{itemize}
\item {Grp. gram.:adj.}
\end{itemize}
Relativo á \textunderscore erpetografia\textunderscore .
\section{Erpetógrafo}
\begin{itemize}
\item {Grp. gram.:m.}
\end{itemize}
\begin{itemize}
\item {Proveniência:(Do gr. \textunderscore erpetos\textunderscore  + \textunderscore graphein\textunderscore )}
\end{itemize}
Autor de trabalhos cientificos á cêrca de reptis.
\section{Erpetographia}
\begin{itemize}
\item {Grp. gram.:f.}
\end{itemize}
\begin{itemize}
\item {Proveniência:(De \textunderscore erpetographo\textunderscore )}
\end{itemize}
Tratado especial, á cêrca dos reptis.
\section{Erpetográphico}
\begin{itemize}
\item {Grp. gram.:adj.}
\end{itemize}
Relativo á \textunderscore erpetographia\textunderscore .
\section{Erpetógrapho}
\begin{itemize}
\item {Grp. gram.:m.}
\end{itemize}
\begin{itemize}
\item {Proveniência:(Do gr. \textunderscore erpetos\textunderscore  + \textunderscore graphein\textunderscore )}
\end{itemize}
Autor de trabalhos scientificos á cêrca de reptis.
\section{Erpetologia}
\begin{itemize}
\item {Grp. gram.:f.}
\end{itemize}
\begin{itemize}
\item {Proveniência:(Do gr. \textunderscore erpetos\textunderscore  + \textunderscore logos\textunderscore )}
\end{itemize}
Parte da Zoologia, que trata dos reptis.
\section{Erpetológico}
\begin{itemize}
\item {Grp. gram.:adj.}
\end{itemize}
Relativo a erpetologia.
\section{Erpetologista}
\begin{itemize}
\item {Grp. gram.:m.}
\end{itemize}
Naturalista, que trata especialmente de erpetologia.
\section{Erpetólogo}
\begin{itemize}
\item {Grp. gram.:m.}
\end{itemize}
Aquelle que é versado em erpetologia.
\section{Errabundo}
\begin{itemize}
\item {Grp. gram.:adj.}
\end{itemize}
\begin{itemize}
\item {Proveniência:(Lat. \textunderscore errabundus\textunderscore )}
\end{itemize}
Errante; vagabundo. Cf. Filinto, XIII, 253; Camillo, \textunderscore Cancion. Al.\textunderscore , 309.
\section{Errada}
\begin{itemize}
\item {Grp. gram.:f.}
\end{itemize}
\begin{itemize}
\item {Utilização:Ant.}
\end{itemize}
\begin{itemize}
\item {Proveniência:(De \textunderscore errado\textunderscore )}
\end{itemize}
Encruzilhada, que faz errar o caminho.
\section{Erradamente}
\begin{itemize}
\item {Grp. gram.:adv.}
\end{itemize}
\begin{itemize}
\item {Proveniência:(De \textunderscore errar\textunderscore )}
\end{itemize}
Com êrro.
\section{Erradicação}
\begin{itemize}
\item {Grp. gram.:f.}
\end{itemize}
\begin{itemize}
\item {Proveniência:(Lat. \textunderscore eradicatio\textunderscore )}
\end{itemize}
Acto de erradicar.
\section{Erradicante}
\begin{itemize}
\item {Grp. gram.:adj.}
\end{itemize}
\begin{itemize}
\item {Proveniência:(Lat. \textunderscore eradicans\textunderscore )}
\end{itemize}
Que erradica.
\section{Erradicar}
\begin{itemize}
\item {Grp. gram.:v. t.}
\end{itemize}
\begin{itemize}
\item {Utilização:Ant.}
\end{itemize}
\begin{itemize}
\item {Proveniência:(Lat. \textunderscore eradicare\textunderscore )}
\end{itemize}
O mesmo que \textunderscore desarraigar\textunderscore .
\section{Erradicativo}
\begin{itemize}
\item {Grp. gram.:adj.}
\end{itemize}
O mesmo que \textunderscore erradicante\textunderscore .
\section{Erradio}
\begin{itemize}
\item {Grp. gram.:adj.}
\end{itemize}
\begin{itemize}
\item {Utilização:Fig.}
\end{itemize}
\begin{itemize}
\item {Proveniência:(Do lat. \textunderscore errativus\textunderscore )}
\end{itemize}
Vagabundo.
Errante.
Que tem tendência para a vida errante.
Desvairado; desnorteado.
\section{Errante}
\begin{itemize}
\item {Grp. gram.:adj.}
\end{itemize}
\begin{itemize}
\item {Proveniência:(Lat. \textunderscore errans\textunderscore )}
\end{itemize}
Que erra.
Que vagueia.
Que se extravia.
Próprio de vagabundo: \textunderscore vida errante\textunderscore .
Que procede mal.
Que vacilla.
Que ignora.
\section{Errar}
\begin{itemize}
\item {Grp. gram.:v. t.}
\end{itemize}
\begin{itemize}
\item {Utilização:Ant.}
\end{itemize}
\begin{itemize}
\item {Grp. gram.:V. i.}
\end{itemize}
\begin{itemize}
\item {Proveniência:(Lat. \textunderscore errare\textunderscore )}
\end{itemize}
Commeter êrro em: \textunderscore errar uma conta\textunderscore .
Têr engano com; desacertar: \textunderscore errar o caminho\textunderscore .
Desencontrar-se de.
Vaguear; não têr paradeiro nem residência fixa.
Caír em êrro, em culpa: \textunderscore você agora errou\textunderscore .
Peccar.
Enganar-se.
Fluctuar: \textunderscore vão errando fôlhas de árvores na superfície da corrente\textunderscore .
\section{Errata}
\begin{itemize}
\item {Grp. gram.:f.}
\end{itemize}
\begin{itemize}
\item {Proveniência:(Lat. \textunderscore errata\textunderscore )}
\end{itemize}
Indicação ou emenda de um êrro, commetido num impresso pelo autor, pelo revisor ou pelo typógrapho.
\section{Erraticidade}
\begin{itemize}
\item {Grp. gram.:f.}
\end{itemize}
\begin{itemize}
\item {Utilização:Espir.}
\end{itemize}
Qualidade daquillo que é errático.
Estado dos espíritos não encarnados, isto é, estado dos espíritos durante os intervallos das suas encarnações.
\section{Errático}
\begin{itemize}
\item {Grp. gram.:adj.}
\end{itemize}
\begin{itemize}
\item {Proveniência:(Lat. erraticus)}
\end{itemize}
Erradio.
Vagabundo.
Que muda de lugar.
Irregular.
Transportado de longe, (falando-se dos fragmentos de rocha, que não são da natureza do terreno em que se encontram).
\section{Erratum}
\begin{itemize}
\item {fónica:rá}
\end{itemize}
\begin{itemize}
\item {Grp. gram.:m.}
\end{itemize}
O mesmo que \textunderscore errata\textunderscore . Cf. Camillo, \textunderscore Narcót.\textunderscore , II, 256.
\section{Erre}
\begin{itemize}
\item {Grp. gram.:m.}
\end{itemize}
\begin{itemize}
\item {Utilização:Ant.}
\end{itemize}
Nome da letra \textunderscore R\textunderscore .
Ponto aproximado.
Beira; risco:«\textunderscore estive em erre de levar-lhe a touca nas unhas.\textunderscore »\textunderscore Aulegrafia\textunderscore , 14 v.^o
\textunderscore Por um erre\textunderscore , por pouco, quási.
\textunderscore Levar um erre\textunderscore , têr uma reprovação, ou sêr reprovado por um dos vogaes do júry, que o approvou por maioria.
\section{Erreiro}
\begin{itemize}
\item {Grp. gram.:adj.}
\end{itemize}
\begin{itemize}
\item {Utilização:Prov.}
\end{itemize}
\begin{itemize}
\item {Utilização:alent.}
\end{itemize}
\begin{itemize}
\item {Utilização:Prov.}
\end{itemize}
\begin{itemize}
\item {Utilização:beir.}
\end{itemize}
\begin{itemize}
\item {Proveniência:(De \textunderscore errar\textunderscore )}
\end{itemize}
Diz-se do animal que, emparalhado, só trabalha bem de um lado, do esquerdo ou do direito.
Diz-se do boi, que se afasta do trilho batido para sítios pedregosos ou pouco transitáveis.
\section{Errhino}
\begin{itemize}
\item {Grp. gram.:adj.}
\end{itemize}
\begin{itemize}
\item {Proveniência:(Gr. \textunderscore errhinos\textunderscore )}
\end{itemize}
Diz-se do medicamento ou da substância, que, introduzida em o nariz, provoca o espirro.
Esternutório.
Ptármico.
\section{Erriçamento}
\begin{itemize}
\item {Grp. gram.:m.}
\end{itemize}
Acto ou effeito de erriçar.
\section{Erriçar}
\begin{itemize}
\item {Grp. gram.:v. t.}
\end{itemize}
O mesmo que \textunderscore eriçar\textunderscore .
\section{Errino}
\begin{itemize}
\item {Grp. gram.:adj.}
\end{itemize}
\begin{itemize}
\item {Proveniência:(Gr. \textunderscore errhinos\textunderscore )}
\end{itemize}
Diz-se do medicamento ou da substância, que, introduzida em o nariz, provoca o espirro.
Esternutório.
Ptármico.
\section{Êrro}
\begin{itemize}
\item {Grp. gram.:m.}
\end{itemize}
\begin{itemize}
\item {Utilização:Prov.}
\end{itemize}
\begin{itemize}
\item {Utilização:alent.}
\end{itemize}
\begin{itemize}
\item {Utilização:Pop.}
\end{itemize}
Acto ou effeito de errar.
Afastamento da honestidade ou da justiça.
Doutrina errada.
Culpa.
Desvio, volta.
Acto da mulhér, que illicitamente perdeu a virgindade: \textunderscore depois do seu êrro, todos a desprezaram\textunderscore .
\section{Errónea}
\begin{itemize}
\item {Grp. gram.:f.}
\end{itemize}
\begin{itemize}
\item {Utilização:Des.}
\end{itemize}
\begin{itemize}
\item {Proveniência:(De \textunderscore erróneo\textunderscore )}
\end{itemize}
Êrro; coisa errada. Cf. Bandarra, 122, v.^o
\section{Erroneamente}
\begin{itemize}
\item {Grp. gram.:adv.}
\end{itemize}
De modo erróneo.
\section{Erróneo}
\begin{itemize}
\item {Grp. gram.:adj.}
\end{itemize}
\begin{itemize}
\item {Proveniência:(Lat. \textunderscore erroneus\textunderscore )}
\end{itemize}
Em que há êrro.
\section{Errónia}
\begin{itemize}
\item {Grp. gram.:f.}
\end{itemize}
\begin{itemize}
\item {Utilização:Prov.}
\end{itemize}
Má índole, mau carácter.
\section{Error}
\begin{itemize}
\item {Grp. gram.:m.}
\end{itemize}
\begin{itemize}
\item {Utilização:Des.}
\end{itemize}
\begin{itemize}
\item {Proveniência:(Lat. \textunderscore error\textunderscore )}
\end{itemize}
Viagem sem rumo, indeterminada.
Êrro.
\section{Ersa}
\begin{itemize}
\item {Grp. gram.:m.}
\end{itemize}
Primitivo idioma escocês, um dos representantes do antigo céltico. Cf. Herculano, \textunderscore Hist. de Port.\textunderscore , I, 33.
\section{Erubescer}
\begin{itemize}
\item {fónica:ru}
\end{itemize}
\textunderscore v. t.\textunderscore  e \textunderscore i.\textunderscore  (e der.)
O mesmo que \textunderscore enrubescer\textunderscore , etc.
\section{Eruca}
\begin{itemize}
\item {Grp. gram.:f.}
\end{itemize}
\begin{itemize}
\item {Proveniência:(Lat. \textunderscore eruca\textunderscore )}
\end{itemize}
Lagarta da hortaliça.
Planta crucífera.
\section{Erucária}
\begin{itemize}
\item {Grp. gram.:f.}
\end{itemize}
\begin{itemize}
\item {Proveniência:(De \textunderscore eruca\textunderscore )}
\end{itemize}
Gênero de plantas crucíferas.
\section{Eruciforme}
\begin{itemize}
\item {Grp. gram.:adj.}
\end{itemize}
\begin{itemize}
\item {Proveniência:(Do lat. \textunderscore eruca\textunderscore  + \textunderscore forma\textunderscore )}
\end{itemize}
Que tem fórma de lagarta.
\section{Erucívora}
\begin{itemize}
\item {Grp. gram.:f.}
\end{itemize}
\begin{itemize}
\item {Proveniência:(Do lat. \textunderscore eruca\textunderscore  + \textunderscore vorare\textunderscore )}
\end{itemize}
Gênero de aves, de bico delgado, asas agudas e cauda comprida.
\section{Eructação}
\begin{itemize}
\item {Grp. gram.:f.}
\end{itemize}
\begin{itemize}
\item {Proveniência:(Lat. \textunderscore eructatio\textunderscore )}
\end{itemize}
Arrôto.
\section{Erudição}
\begin{itemize}
\item {Grp. gram.:f.}
\end{itemize}
\begin{itemize}
\item {Proveniência:(Lat. eruditio)}
\end{itemize}
Qualidade daquelle ou daquillo que é erudito.
Sabedoria.
\section{Erudir}
\begin{itemize}
\item {Grp. gram.:v. t.}
\end{itemize}
Instruir. Cf. Costa e Sá, \textunderscore Arte Poét\textunderscore .
\section{Eruditamente}
\begin{itemize}
\item {Grp. gram.:adv.}
\end{itemize}
De modo erudito.
Com erudição.
\section{Erudito}
\begin{itemize}
\item {Grp. gram.:adj.}
\end{itemize}
\begin{itemize}
\item {Grp. gram.:M.}
\end{itemize}
\begin{itemize}
\item {Proveniência:(Lat. \textunderscore eruditus\textunderscore )}
\end{itemize}
Que tem saber vasto e variado; que revela muito saber.
Homem muito sabedor; encyclopedista.
\section{Eruga}
\begin{itemize}
\item {Grp. gram.:f.}
\end{itemize}
O mesmo ou melhor que \textunderscore eruca\textunderscore .
\section{Eruginoso}
\begin{itemize}
\item {Grp. gram.:adj.}
\end{itemize}
\begin{itemize}
\item {Proveniência:(Lat. \textunderscore aeruginosus\textunderscore )}
\end{itemize}
Que está oxydado, enferrujado.
\section{Eruir}
\begin{itemize}
\item {Grp. gram.:v. i.}
\end{itemize}
\begin{itemize}
\item {Utilização:Des.}
\end{itemize}
\begin{itemize}
\item {Proveniência:(Lat. \textunderscore eruere\textunderscore )}
\end{itemize}
O mesmo que \textunderscore sair\textunderscore :«\textunderscore murmúrio eruído da clara fonte\textunderscore ». Usque, \textunderscore Tribulações\textunderscore , 10 v.^o
\section{Erupção}
\begin{itemize}
\item {Grp. gram.:f.}
\end{itemize}
\begin{itemize}
\item {Utilização:Med.}
\end{itemize}
\begin{itemize}
\item {Proveniência:(Lat. \textunderscore eruptio\textunderscore )}
\end{itemize}
Saída violenta e rápida.
Saida violenta dos gases vulcânicos: \textunderscore nova erupção do Vesúvio\textunderscore .
Evacuação abundante.
Apparição de quaisquer exanthemas sôbre a pelle.
\section{Eruptivo}
\begin{itemize}
\item {Grp. gram.:adj.}
\end{itemize}
\begin{itemize}
\item {Proveniência:(Do lat. \textunderscore eruptus\textunderscore )}
\end{itemize}
Relativo a erupção.
\section{Erva}
\begin{itemize}
\item {Grp. gram.:f.}
\end{itemize}
\begin{itemize}
\item {Utilização:Bras}
\end{itemize}
\begin{itemize}
\item {Grp. gram.:Pl.}
\end{itemize}
\begin{itemize}
\item {Proveniência:(Lat. \textunderscore herba\textunderscore )}
\end{itemize}
Qualquer planta, annual ou vivaz, que não é árvore nem arbusto, e que seca em frutificando.
Vegetação espontânea.
Plantas de pasto ou forragem.
Palavra, que entra na composição do nome de numerosíssimas plantas.
O mesmo que \textunderscore congonha\textunderscore .
Hortaliça; esparregado.
\section{Erva-abelha}
\begin{itemize}
\item {Grp. gram.:f.}
\end{itemize}
Nome vulgar de uma planta, (\textunderscore ophrys speculm\textunderscore , Lk.), espécie de orchídea.
\section{Erva-agulheira}
\begin{itemize}
\item {Grp. gram.:f.}
\end{itemize}
Planta umbellífera, (\textunderscore scandix pecten Veneris\textunderscore , Lin.).
\section{Erva-albiloira}
\begin{itemize}
\item {Grp. gram.:f.}
\end{itemize}
Planta da serra de Cintra.
\section{Erva-alheira}
\begin{itemize}
\item {Grp. gram.:f.}
\end{itemize}
Nome vulgar da planta \textunderscore sysimbrium alliaria\textunderscore .
\section{Erva-alleluia}
\begin{itemize}
\item {Grp. gram.:f.}
\end{itemize}
Planta da serra de Cintra.
\section{Erva-andorinha}
\begin{itemize}
\item {Grp. gram.:f.}
\end{itemize}
(V.celidónia)
\section{Erva-aranha}
\begin{itemize}
\item {Grp. gram.:f.}
\end{itemize}
Nome vulgar de uma planta (\textunderscore ophrys arachnites\textunderscore ), espécie de orchídea.
\section{Erva-armoles}
\begin{itemize}
\item {Grp. gram.:f.}
\end{itemize}
Planta chenopódea, (\textunderscore atriplex hortense\textunderscore , Lin.).
\section{Erva-babosa}
\begin{itemize}
\item {Grp. gram.:f.}
\end{itemize}
O mesmo que \textunderscore aloés\textunderscore .
\section{Erva-benta}
\begin{itemize}
\item {Grp. gram.:f.}
\end{itemize}
Nome vulgar de uma planta, (\textunderscore geumiurbano\textunderscore ).
\section{Erva-bezerra}
\begin{itemize}
\item {Grp. gram.:f.}
\end{itemize}
Nome vulgar da planta \textunderscore antirrhinum majus\textunderscore ; \textunderscore latifolium\textunderscore .
\section{Erva-bicha}
\begin{itemize}
\item {Grp. gram.:f.}
\end{itemize}
Planta aristolóchia, (\textunderscore aristolochia longa\textunderscore ).
\section{Erva-borboleta}
\begin{itemize}
\item {Grp. gram.:f.}
\end{itemize}
Variedade de orchídea portuguesa, (\textunderscore orchis papilionacea\textunderscore , Lin.).
\section{Erva-botão}
\begin{itemize}
\item {Grp. gram.:f.}
\end{itemize}
\begin{itemize}
\item {Utilização:Bras}
\end{itemize}
Planta medicinal, (\textunderscore eclypta alba\textunderscore ).
\section{Erva-brico}
\begin{itemize}
\item {Grp. gram.:f.}
\end{itemize}
Planta da serra de Cintra.
\section{Ervaçal}
\begin{itemize}
\item {Grp. gram.:m.}
\end{itemize}
\begin{itemize}
\item {Proveniência:(De \textunderscore erva\textunderscore )}
\end{itemize}
Terra, em que há muita erva; pastagem.
\section{Erva-canuda}
\begin{itemize}
\item {Grp. gram.:f.}
\end{itemize}
Planta forragínea, (\textunderscore equisetum arvense\textunderscore , Lin.).
\section{Erva-carneira}
\begin{itemize}
\item {Grp. gram.:f.}
\end{itemize}
Planta forraginosa, (\textunderscore festuca elatior\textunderscore , Brot.).
\section{Erva-castelhana}
\begin{itemize}
\item {Grp. gram.:f.}
\end{itemize}
\begin{itemize}
\item {Utilização:Prov.}
\end{itemize}
\begin{itemize}
\item {Utilização:minh.}
\end{itemize}
Planta gramínea, (\textunderscore lolium multiflorum\textunderscore , Laruk).
\section{Erva-cicutária}
\begin{itemize}
\item {Grp. gram.:f.}
\end{itemize}
Planta umbellífera, (\textunderscore anthriscus vulgaris\textunderscore ).
\section{Erva-cidreira}
\begin{itemize}
\item {Grp. gram.:f.}
\end{itemize}
Planta aromática da fam. das labiadas, (\textunderscore melissa officinalis\textunderscore ).
\section{Erva-coentrinha}
\begin{itemize}
\item {Grp. gram.:f.}
\end{itemize}
Planta daucínea, (\textunderscore daucus carota\textunderscore , Lin.).
\section{Erva-collégio}
\begin{itemize}
\item {Grp. gram.:f.}
\end{itemize}
\begin{itemize}
\item {Utilização:Bras}
\end{itemize}
O mesmo que \textunderscore erva-grossa\textunderscore .
\section{Erva-conteira}
\begin{itemize}
\item {Grp. gram.:f.}
\end{itemize}
Planta canácea, (\textunderscore canna índica\textunderscore ).
\section{Erva-crina}
\begin{itemize}
\item {Grp. gram.:f.}
\end{itemize}
Planta medicinal, (\textunderscore chamaeptus\textunderscore , \textunderscore iva orthetica\textunderscore , Dioscórides). Cf. \textunderscore Deseng. da Med.\textunderscore , 112.
\section{Erva-da-fortuna}
\begin{itemize}
\item {Grp. gram.:f.}
\end{itemize}
Designação pop. da tradescância.
\section{Erva-da-guiné}
\begin{itemize}
\item {Grp. gram.:f.}
\end{itemize}
Planta, que se propaga sem cultura e é bôa para pastos, (\textunderscore panicum altissimum\textunderscore ).
\section{Erva-da-muda}
\begin{itemize}
\item {Grp. gram.:f.}
\end{itemize}
Gênero de plantas, (\textunderscore polygonum aviculare\textunderscore , Lin.).
\section{Erva-das-sete-sangrias}
\begin{itemize}
\item {Grp. gram.:f.}
\end{itemize}
Planta, também conhecida por \textunderscore sargaço híspido\textunderscore  e \textunderscore sargacinha\textunderscore .
\section{Erva-das-sezões}
\begin{itemize}
\item {Grp. gram.:f}
\end{itemize}
Nome vulgar da \textunderscore artemisia mollis\textunderscore .
\section{Erva-da-trindade}
\begin{itemize}
\item {Grp. gram.:f.}
\end{itemize}
\begin{itemize}
\item {Utilização:Bras}
\end{itemize}
O mesmo que \textunderscore amor-perfeito\textunderscore .
\section{Erva-de-basculho}
\begin{itemize}
\item {Grp. gram.:f.}
\end{itemize}
\begin{itemize}
\item {Utilização:Prov.}
\end{itemize}
O mesmo que \textunderscore gilbardeira\textunderscore .
\section{Erva-de-bugre}
\begin{itemize}
\item {Grp. gram.:f.}
\end{itemize}
Planta medicinal, brasileira, (\textunderscore casearia sylvestris\textunderscore ).
\section{Erva-de-cobra}
\begin{itemize}
\item {Grp. gram.:f.}
\end{itemize}
\begin{itemize}
\item {Utilização:Bras}
\end{itemize}
Planta medicinal, synanthérea.
O mesmo que \textunderscore eupatório\textunderscore .
\section{Erva-dedal}
\begin{itemize}
\item {Grp. gram.:f.}
\end{itemize}
(V. \textunderscore dedaleira\textunderscore , planta)
\section{Erva-de-febra}
\begin{itemize}
\item {Grp. gram.:f.}
\end{itemize}
Planta forraginosa, (\textunderscore poa pratensis\textunderscore , Lin.).
\section{Erva-de-joão-pires}
\begin{itemize}
\item {Grp. gram.:f.}
\end{itemize}
Planta medicinal da serra de Cintra. Cf. \textunderscore Deseng. da Med.\textunderscore 
\section{Erva-de-parida}
\begin{itemize}
\item {Grp. gram.:f.}
\end{itemize}
Planta rubiácea.
\section{Erva-de-rato}
\begin{itemize}
\item {Grp. gram.:f.}
\end{itemize}
Planta cinchonácea.
\section{Erva-de-san-christóvão}
\begin{itemize}
\item {Grp. gram.:f.}
\end{itemize}
Nome vulgar do \textunderscore ébulo\textunderscore .
\section{Erva-de-san-joão}
\begin{itemize}
\item {Grp. gram.:f.}
\end{itemize}
\begin{itemize}
\item {Utilização:Bras}
\end{itemize}
Planta, o mesmo que \textunderscore mentastro\textunderscore .
\section{Erva-de-santa-helena}
\begin{itemize}
\item {Grp. gram.:f.}
\end{itemize}
\begin{itemize}
\item {Utilização:Bras}
\end{itemize}
Planta medicinal.
\section{Erva-de-santa-maria}
\begin{itemize}
\item {Grp. gram.:f.}
\end{itemize}
\begin{itemize}
\item {Utilização:Bras}
\end{itemize}
Planta medicinal, chenopódia; mastruço.
\section{Erva-diurética}
\begin{itemize}
\item {Grp. gram.:f.}
\end{itemize}
\begin{itemize}
\item {Utilização:Bras}
\end{itemize}
Planta medicinal, ericácea, (\textunderscore chimaphila umbellata\textunderscore ).
\section{Erva-divina}
\begin{itemize}
\item {Grp. gram.:f.}
\end{itemize}
Planta plumbagínea.
\section{Ervado}
\begin{itemize}
\item {Grp. gram.:m.}
\end{itemize}
\begin{itemize}
\item {Utilização:Prov.}
\end{itemize}
\begin{itemize}
\item {Utilização:trasm.}
\end{itemize}
O mesmo que \textunderscore medronho\textunderscore .
\section{Erva-do-amor}
\begin{itemize}
\item {Grp. gram.:f.}
\end{itemize}
(V.trevo)
\section{Erva-do-bicho}
\begin{itemize}
\item {Grp. gram.:m.}
\end{itemize}
\begin{itemize}
\item {Utilização:Bras}
\end{itemize}
Planta medicinal, polygonácea.
\section{Erva-doce}
\begin{itemize}
\item {Grp. gram.:f.}
\end{itemize}
Planta umbellífera, (\textunderscore pimpinella anisum\textunderscore , Lin.).
O mesmo que \textunderscore anis\textunderscore .
\section{Erva-do-grão-prior}
\begin{itemize}
\item {Grp. gram.:f.}
\end{itemize}
Primeiro nome, que em Portugal se deu ao tabaco.
\section{Erva-do-leite}
\begin{itemize}
\item {Grp. gram.:f.}
\end{itemize}
Trepadeira galactagoga, abundante em Moçambique, ao sul do Save. Cf. \textunderscore Diár. de Notìc.\textunderscore , de 27-IX-902.
\section{Erva-do-malabar}
\begin{itemize}
\item {Grp. gram.:f.}
\end{itemize}
O mesmo que \textunderscore cudô\textunderscore .
\section{Erva-do-orvalho}
\begin{itemize}
\item {Grp. gram.:f.}
\end{itemize}
Planta mesembriathêmia, também conhecida pelos nomes de \textunderscore prateada\textunderscore ,
\textunderscore gelada\textunderscore , \textunderscore orvalho-da-aurora\textunderscore , e \textunderscore erva-prata\textunderscore .
\section{Erva-do-pântano}
\begin{itemize}
\item {Grp. gram.:f.}
\end{itemize}
Planta alismácea, (\textunderscore sagittaria brasiliensis\textunderscore ).
\section{Erva-do-paraíso}
\begin{itemize}
\item {Grp. gram.:f.}
\end{itemize}
Planta da serra de Cintra.
\section{Erva-do-sapo}
\begin{itemize}
\item {Grp. gram.:f.}
\end{itemize}
(V.azedinha-do-brejo)
\section{Erva-dos-barbonos}
\begin{itemize}
\item {Grp. gram.:f.}
\end{itemize}
\begin{itemize}
\item {Utilização:Bras}
\end{itemize}
O mesmo que \textunderscore barba-de-velho\textunderscore .
\section{Erva-dos-bèsteiros}
\begin{itemize}
\item {Grp. gram.:f.}
\end{itemize}
Nome vulgar de uma planta, (\textunderscore helleborus foetidus\textunderscore ).
\section{Erva-dos-burros}
\begin{itemize}
\item {Grp. gram.:f.}
\end{itemize}
Designação vulgar da onagra.
\section{Erva-dos-callos}
\begin{itemize}
\item {Grp. gram.:f.}
\end{itemize}
Planta crassulácea, (\textunderscore sedum telephium\textunderscore , Lin.).
\section{Erva-dos-gatos}
\begin{itemize}
\item {Grp. gram.:f.}
\end{itemize}
\begin{itemize}
\item {Utilização:Bras}
\end{itemize}
Planta valerianácea, medicinal, (\textunderscore valeriana officilinalis\textunderscore ).
\section{Erva-dos-muros}
\begin{itemize}
\item {Grp. gram.:f.}
\end{itemize}
Planta, também conhecida por \textunderscore erva-dos-namorados\textunderscore , (\textunderscore cissus antiparalyticus\textunderscore ).
\section{Erva-dos-pegamaços}
\begin{itemize}
\item {Grp. gram.:f.}
\end{itemize}
Espécie de bardana, (\textunderscore canthium strumarium\textunderscore ), o mesmo que \textunderscore pega-maça\textunderscore .
\section{Erva-espirradeira}
\begin{itemize}
\item {Grp. gram.:f.}
\end{itemize}
Nome vulgar da \textunderscore achillea ptarmica\textunderscore .
\section{Erva-fedegosa}
\begin{itemize}
\item {Grp. gram.:f.}
\end{itemize}
(V.fedegosa)
\section{Erva-feiticeira}
\begin{itemize}
\item {Grp. gram.:f.}
\end{itemize}
Planta therebintácea.
\section{Erva-férrea}
\begin{itemize}
\item {Grp. gram.:f.}
\end{itemize}
\begin{itemize}
\item {Utilização:Prov.}
\end{itemize}
Planta labiada, (\textunderscore brunella vulgaris\textunderscore , Lin.).
Gênero de plantas, (\textunderscore ajuga reptans\textunderscore , Lin.).
\section{Erva-ferro}
\begin{itemize}
\item {Grp. gram.:f.}
\end{itemize}
\begin{itemize}
\item {Utilização:Bras}
\end{itemize}
Planta.
O mesmo que \textunderscore erva-férrea\textunderscore ?
\section{Erva-foicinha}
\begin{itemize}
\item {Grp. gram.:f.}
\end{itemize}
Planta leguminosa, (\textunderscore bonaveria coronilla\textunderscore ).
\section{Erva-formigueira}
\begin{itemize}
\item {Grp. gram.:f.}
\end{itemize}
Planta, também conhecida por \textunderscore cravinho-do-mato\textunderscore , (\textunderscore ambrina ambrosioide\textunderscore ).
\section{Erva-formosa}
\begin{itemize}
\item {Grp. gram.:f.}
\end{itemize}
Planta da serra de Cintra.
\section{Erva-forte}
\begin{itemize}
\item {Grp. gram.:f.}
\end{itemize}
Planta medicinal, (\textunderscore solidago sarracenica\textunderscore , nas pharmácias). Cf. \textunderscore Deseng. da Med.\textunderscore , 60.
\section{Erva-gato}
\begin{itemize}
\item {Grp. gram.:f.}
\end{itemize}
\begin{itemize}
\item {Utilização:Bras}
\end{itemize}
Planta labiada, aromática e medicinal, (\textunderscore nepeta cataria\textunderscore , Lin.).
\section{Ervagem}
\begin{itemize}
\item {Grp. gram.:f.}
\end{itemize}
\begin{itemize}
\item {Proveniência:(De \textunderscore erva\textunderscore )}
\end{itemize}
Pastagem; relvado.
Hortaliça.
\section{Erva-gigante}
\begin{itemize}
\item {Grp. gram.:f.}
\end{itemize}
Planta acanthácea, (\textunderscore acanthus mollis\textunderscore ).
\section{Erva-grossa}
\begin{itemize}
\item {Grp. gram.:f.}
\end{itemize}
\begin{itemize}
\item {Utilização:Bras}
\end{itemize}
Planta synanthérea, medicinal, (\textunderscore elephantopus cervinus\textunderscore ).
\section{Erva-herniária}
\begin{itemize}
\item {Grp. gram.:f.}
\end{itemize}
O mesmo que \textunderscore erva-turca\textunderscore .
\section{Erva-impigem}
\begin{itemize}
\item {Grp. gram.:f.}
\end{itemize}
\begin{itemize}
\item {Utilização:Bras}
\end{itemize}
Planta papaverácea medicinal, (\textunderscore sanguinaria canadensis\textunderscore ).
\section{Erva-judaca}
\begin{itemize}
\item {Grp. gram.:f.}
\end{itemize}
Planta da serra de Cintra.
\section{Erval}
\begin{itemize}
\item {Grp. gram.:m.}
\end{itemize}
\begin{itemize}
\item {Utilização:Bras. do S}
\end{itemize}
\begin{itemize}
\item {Proveniência:(De \textunderscore erva\textunderscore )}
\end{itemize}
Mato, em que predomina a congonha.
\section{Erva-lanceta}
\begin{itemize}
\item {Grp. gram.:f.}
\end{itemize}
\begin{itemize}
\item {Utilização:Bras}
\end{itemize}
Planta vulnerária, o mesmo que \textunderscore erva-botão\textunderscore .
\section{Erva-leiteira}
\begin{itemize}
\item {Grp. gram.:f.}
\end{itemize}
Planta polygálea, (\textunderscore polygala vulgaris\textunderscore , Lin.).
\section{Ervalenta}
\begin{itemize}
\item {Grp. gram.:f.}
\end{itemize}
(V.revalenta)
\section{Erva-língua}
\begin{itemize}
\item {Grp. gram.:f.}
\end{itemize}
Variedade de orchídea portuguesa, (\textunderscore serapias lingua\textunderscore , Lin.).
\section{Erva-luísa}
\begin{itemize}
\item {Grp. gram.:f.}
\end{itemize}
\begin{itemize}
\item {Utilização:Prov.}
\end{itemize}
\begin{itemize}
\item {Utilização:alent.}
\end{itemize}
O mesmo que \textunderscore bella-luísa\textunderscore , ou \textunderscore lúcia-lima\textunderscore .
\section{Erva-malabar}
\begin{itemize}
\item {Grp. gram.:f.}
\end{itemize}
O mesmo que \textunderscore coru\textunderscore . Cf. G. Orta, \textunderscore Collóq.\textunderscore 
\section{Erva-maleiteira}
\begin{itemize}
\item {Grp. gram.:f.}
\end{itemize}
Nome vulgar de uma planta, (\textunderscore euphorbia helioscopica\textunderscore ).
\section{Erva-menina}
\begin{itemize}
\item {Grp. gram.:f.}
\end{itemize}
Planta commellinácea, (\textunderscore commellina agraria\textunderscore , Kunt).
\section{Erva-moedeira}
\begin{itemize}
\item {Grp. gram.:f.}
\end{itemize}
O mesmo que \textunderscore lysimáchia\textunderscore .
\section{Erva-moira}
\begin{itemize}
\item {Grp. gram.:f.}
\end{itemize}
Planta, também conhecida por erva-do-bicho, (\textunderscore solanum nigrum\textunderscore ).
\section{Erva-moira-do-sertão}
\begin{itemize}
\item {Grp. gram.:f.}
\end{itemize}
Planta, (\textunderscore gomphrena officinalis\textunderscore ).
\section{Erva-moleirinha}
\begin{itemize}
\item {Grp. gram.:f.}
\end{itemize}
Planta fumariácea, medicinal; o mesmo que \textunderscore fumária\textunderscore .
\section{Erva-molle}
\begin{itemize}
\item {Grp. gram.:f.}
\end{itemize}
Planta amaranthácea.
\section{Erva-molle-verdadeira}
\begin{itemize}
\item {Grp. gram.:f.}
\end{itemize}
Planta, (\textunderscore cissus mollis\textunderscore ).
\section{Erva-montan}
\begin{itemize}
\item {Grp. gram.:f.}
\end{itemize}
Planta composta, (\textunderscore pulicaria odona\textunderscore , Reich.).
\section{Erva-môsca}
\begin{itemize}
\item {Grp. gram.:f.}
\end{itemize}
Designação pop. de uma espécie de orchídea portuguesa.
\section{Ervanário}
\begin{itemize}
\item {Grp. gram.:m.}
\end{itemize}
Aquelle que vende ou collecciona plantas medicinaes.
(Cp. lat. \textunderscore herbanus\textunderscore )
\section{Ervançal}
\begin{itemize}
\item {Grp. gram.:m.}
\end{itemize}
Plantação de ervanços. Cf. Castilho, \textunderscore Fastos\textunderscore , III, 77; \textunderscore Metam.\textunderscore , 47; \textunderscore Collóq. Ald.\textunderscore , 304.
\section{Ervanço}
\begin{itemize}
\item {Grp. gram.:m.}
\end{itemize}
O mesmo que \textunderscore gravanço\textunderscore ^1.
\section{Erva-noselha}
\begin{itemize}
\item {Grp. gram.:f.}
\end{itemize}
\begin{itemize}
\item {Utilização:Prov.}
\end{itemize}
\begin{itemize}
\item {Utilização:minh.}
\end{itemize}
Gênero de plantas, (\textunderscore avena elatior\textunderscore , Lin.).
\section{Ervão}
\begin{itemize}
\item {Grp. gram.:m.}
\end{itemize}
\begin{itemize}
\item {Utilização:Bras}
\end{itemize}
Planta, o mesmo que \textunderscore pataqueira\textunderscore .
\section{Erva-pata}
\begin{itemize}
\item {Grp. gram.:f.}
\end{itemize}
Planta venenosa, talvez o mesmo que \textunderscore patinha\textunderscore .
\section{Erva-perceveja}
\begin{itemize}
\item {Grp. gram.:f.}
\end{itemize}
Variedade de orchídea portuguesa.
\section{Erva-pessegueira}
\begin{itemize}
\item {Grp. gram.:f.}
\end{itemize}
Planta polygónea, (\textunderscore polygono persicaria\textunderscore ).
\section{Erva-pimenteira}
\begin{itemize}
\item {Grp. gram.:f.}
\end{itemize}
Nome vulgar da planta \textunderscore lepidium latifolium\textunderscore .
\section{Erva-pinheira}
\begin{itemize}
\item {Grp. gram.:f.}
\end{itemize}
Planta droserácea, (\textunderscore drosophyllum lusitanicum\textunderscore , Link).
\section{Erva-pinheira-enxuta}
\begin{itemize}
\item {Grp. gram.:f.}
\end{itemize}
Planta crassulácea, (\textunderscore sedum altissimum\textunderscore , Poir.).
\section{Erva-piolheira}
\begin{itemize}
\item {Grp. gram.:f.}
\end{itemize}
O mesmo que \textunderscore paparraz\textunderscore .
Gênero de plantas, (\textunderscore angelica montana\textunderscore , Lin.).
\section{Erva-pombinha}
\begin{itemize}
\item {Grp. gram.:f.}
\end{itemize}
Espécie de aquilégia, (\textunderscore aquilegia vulgaris\textunderscore , Lin.).
Nome de uma euphorbiácea brasileira, (\textunderscore phyllanthos niruri\textunderscore ).
\section{Erva-prata}
\begin{itemize}
\item {Grp. gram.:f.}
\end{itemize}
O mesmo que \textunderscore orvalhinha\textunderscore .
\section{Erva-pulgueira}
\begin{itemize}
\item {Grp. gram.:f.}
\end{itemize}
\begin{itemize}
\item {Utilização:Prov.}
\end{itemize}
\begin{itemize}
\item {Utilização:minh.}
\end{itemize}
Gênero de plantas, (\textunderscore polygno hydropiper\textunderscore , Lin.)
\section{Ervar}
\begin{itemize}
\item {Grp. gram.:v. t.}
\end{itemize}
\begin{itemize}
\item {Proveniência:(De \textunderscore erva\textunderscore )}
\end{itemize}
Humedecer com suco de erva venenosa, (falando-se de instrumentos ou armas, destinadas a ferir, envenenando).
\section{Ervário}
\begin{itemize}
\item {Grp. gram.:m.}
\end{itemize}
(V.herbário)
\section{Erva-roberta}
\begin{itemize}
\item {Grp. gram.:f.}
\end{itemize}
Planta geraniácea, (\textunderscore geranium robertianum\textunderscore , Lin.).
\section{Erva-ruiva}
\begin{itemize}
\item {Grp. gram.:f.}
\end{itemize}
O mesmo que \textunderscore açafrão\textunderscore .
\section{Erva-saboeira}
\begin{itemize}
\item {Grp. gram.:f.}
\end{itemize}
Planta da serra de Cintra.
\section{Erva-sangue}
\begin{itemize}
\item {Grp. gram.:f.}
\end{itemize}
Também conhecida por \textunderscore erva-do-figado\textunderscore  ou \textunderscore lingua-de-vaca\textunderscore , (\textunderscore anchusa italica\textunderscore ).
\section{Erva-santa}
\begin{itemize}
\item {Grp. gram.:f.}
\end{itemize}
\begin{itemize}
\item {Utilização:Pop.}
\end{itemize}
O mesmo que \textunderscore tabaco\textunderscore .
\section{Erva-sargacinha}
\begin{itemize}
\item {Grp. gram.:f.}
\end{itemize}
O mesmo que \textunderscore sargacinha\textunderscore .
\section{Ervascal}
\begin{itemize}
\item {Grp. gram.:m.}
\end{itemize}
\begin{itemize}
\item {Utilização:T. de Turquel}
\end{itemize}
Terreno, em que crescem muitas ervas inúteis.
\section{Ervascum}
\begin{itemize}
\item {Grp. gram.:m.}
\end{itemize}
\begin{itemize}
\item {Utilização:T. de Turquel}
\end{itemize}
Quantidade de ervas ruíns, que invadem terras de lavoira.
\section{Erva-serra}
\begin{itemize}
\item {Grp. gram.:f.}
\end{itemize}
(V.erva-pimenteira)
\section{Erva-silvina}
\begin{itemize}
\item {Grp. gram.:f.}
\end{itemize}
\begin{itemize}
\item {Utilização:Bras}
\end{itemize}
Planta parasita, semelhante á hera, e que vive especialmente nas mangueiras.
\section{Ervatão}
\begin{itemize}
\item {Grp. gram.:m.}
\end{itemize}
Planta medicinal, (\textunderscore petroselium macedonium\textunderscore , Diósc.).
\section{Ervateiro}
\begin{itemize}
\item {Grp. gram.:m.}
\end{itemize}
\begin{itemize}
\item {Utilização:Bras. do S}
\end{itemize}
Negociante de congonha.
\section{Erva-terrestre}
\begin{itemize}
\item {Grp. gram.:f.}
\end{itemize}
Planta da serra de Cintra.
\section{Erva-tintureira}
\begin{itemize}
\item {Grp. gram.:f.}
\end{itemize}
Planta phytolácea, também conhecida por \textunderscore erva-dos-cancros\textunderscore .
\section{Erva-toira}
\begin{itemize}
\item {Grp. gram.:f.}
\end{itemize}
Planta parasita da corriola.
\section{E}
\begin{itemize}
\item {fónica:é}
\end{itemize}
\begin{itemize}
\item {Grp. gram.:m.}
\end{itemize}
Quinta letra do alphabeto português.
Quando maiúsculo e seguido de um ponto, é abrev. de \textunderscore éste\textunderscore , e abrev. do tratamento de \textunderscore excellência\textunderscore  ou de \textunderscore eminência\textunderscore .
\section{E}
\begin{itemize}
\item {fónica:i}
\end{itemize}
\begin{itemize}
\item {Grp. gram.:conj.}
\end{itemize}
\begin{itemize}
\item {Proveniência:(Do lat. \textunderscore et\textunderscore )}
\end{itemize}
Liga as partes semelhantes de um discurso, e os nomes de hora, de medida, a uma fracção da mesma medida ou hora.
Entra na expressão dos números compostos.
Emprega-se também para dar fôrça á phrase.
No estilo bíblico emprega-se em princípio do phrase, que não tem ligação immediata com a antecedente.
\section{E...}
\begin{itemize}
\item {Grp. gram.:pref.}
\end{itemize}
(indicativo de aumento, princípio de acção ou de movimento, etc.)
\section{...e}
\begin{itemize}
\item {Grp. gram.:suf.}
\end{itemize}
(designativo de acção em alguns substantivos verbaes)
\section{...ear}
\begin{itemize}
\item {Grp. gram.:suf.}
\end{itemize}
Termina alguns verbos frequentativos, bem como os que derìvam de substantivos ou adjectivos com desinência em \textunderscore eio\textunderscore  ou \textunderscore eia\textunderscore .
\section{Earina}
\begin{itemize}
\item {Grp. gram.:f.}
\end{itemize}
Gênero de orchídeas.
\section{Eatónia}
\begin{itemize}
\item {Grp. gram.:f.}
\end{itemize}
Gênero de plantas gramíneas.
\section{E. B.}
\begin{itemize}
\item {Utilização:Náut.}
\end{itemize}
Abrev. de \textunderscore estibordo\textunderscore .
\section{Ebália}
\begin{itemize}
\item {Grp. gram.:f.}
\end{itemize}
Gênero de crustáceos, da ordem dos decápodes.
\section{Ebanáceas}
\begin{itemize}
\item {Grp. gram.:f. pl.}
\end{itemize}
Família de plantas, que têm por typo o ébano, de propriedades semelhantes ás das sapotáceas, e de que se conhecem 250 espécies.
\section{Ebande}
\begin{itemize}
\item {Grp. gram.:m.}
\end{itemize}
Peixe angolense do rio Cuito.
\section{Ebâneo}
\begin{itemize}
\item {Grp. gram.:adj.}
\end{itemize}
\begin{itemize}
\item {Utilização:Bras}
\end{itemize}
\begin{itemize}
\item {Utilização:Neol.}
\end{itemize}
Que tem côr de ébano.
\section{Ebanino}
\begin{itemize}
\item {Grp. gram.:adj.}
\end{itemize}
O mesmo que \textunderscore ebâneo\textunderscore .
\section{Ebanista}
\begin{itemize}
\item {Grp. gram.:m.}
\end{itemize}
\begin{itemize}
\item {Proveniência:(De \textunderscore ébano\textunderscore )}
\end{itemize}
Aquelle que trabalha em ébano.
Marceneiro.
Ensamblador.
\section{Ebanizar}
\begin{itemize}
\item {Grp. gram.:v. t.}
\end{itemize}
Dar apparência de ébano a.
\section{Ébano}
\begin{itemize}
\item {Grp. gram.:m.}
\end{itemize}
\begin{itemize}
\item {Utilização:Fig.}
\end{itemize}
Madeira escura e resistente.
Árvore, de que se tira essa madeira.
Aquillo que é negro como ébano: \textunderscore o ébano das tranças da Flávia\textunderscore .
(B. lat. \textunderscore ebanus\textunderscore )
\section{Ebedínio}
\begin{itemize}
\item {Grp. gram.:m.}
\end{itemize}
Gênero de plantas synanthéreas.
\section{E-bem!}
\begin{itemize}
\item {Grp. gram.:interj.}
\end{itemize}
\begin{itemize}
\item {Utilização:Ant.}
\end{itemize}
\begin{itemize}
\item {Proveniência:(It. \textunderscore ebbene\textunderscore )}
\end{itemize}
Pois bem! ainda bem! embora!
\section{Ebenáceas}
\begin{itemize}
\item {Grp. gram.:f. pl.}
\end{itemize}
O mesmo que \textunderscore ebanáceas\textunderscore .
\section{Ébeno}
\begin{itemize}
\item {Grp. gram.:m.}
\end{itemize}
O mesmo que \textunderscore ébano\textunderscore . Cf. \textunderscore Viriato Trág.\textunderscore , XIV, 49.
\section{Ebonite}
\begin{itemize}
\item {Grp. gram.:m.}
\end{itemize}
\begin{itemize}
\item {Proveniência:(Fr. \textunderscore ebonite\textunderscore )}
\end{itemize}
Bacia, de fórma variável, usada nos hospitaes de Lisbôa e feita de vidro, ferro ou borracha endurecida.
\section{Eborário}
\begin{itemize}
\item {Grp. gram.:m.}
\end{itemize}
\begin{itemize}
\item {Utilização:Des.}
\end{itemize}
\begin{itemize}
\item {Proveniência:(Lat. \textunderscore eborarius\textunderscore )}
\end{itemize}
Aquelle, que trabalha em marfim.
\section{Eborense}
\begin{itemize}
\item {Grp. gram.:adj.}
\end{itemize}
\begin{itemize}
\item {Grp. gram.:M.}
\end{itemize}
\begin{itemize}
\item {Proveniência:(Lat. \textunderscore eborensis\textunderscore )}
\end{itemize}
Relativo a Évora.
Indivíduo natural de Évora.
\section{Ebóreo}
\begin{itemize}
\item {Grp. gram.:adj.}
\end{itemize}
\begin{itemize}
\item {Proveniência:(Lat. \textunderscore eboreus\textunderscore )}
\end{itemize}
Que é feito de marfim, ou que é da côr do marfim.
Ebúrneo.
\section{Ebriático}
\begin{itemize}
\item {Grp. gram.:adj.}
\end{itemize}
\begin{itemize}
\item {Utilização:Des.}
\end{itemize}
\begin{itemize}
\item {Proveniência:(Do rad. de \textunderscore ébrio\textunderscore )}
\end{itemize}
Que causa embriaguez.
\section{Ebriativo}
\begin{itemize}
\item {Grp. gram.:adj.}
\end{itemize}
\begin{itemize}
\item {Utilização:Des.}
\end{itemize}
O mesmo que \textunderscore ebriático\textunderscore .
\section{Ebriedade}
\begin{itemize}
\item {Grp. gram.:f.}
\end{itemize}
\begin{itemize}
\item {Proveniência:(Lat. \textunderscore ebrietas\textunderscore )}
\end{itemize}
O mesmo que \textunderscore embriaguez\textunderscore .
\section{Ebriez}
\begin{itemize}
\item {Grp. gram.:f.}
\end{itemize}
\begin{itemize}
\item {Utilização:Des.}
\end{itemize}
\begin{itemize}
\item {Proveniência:(De \textunderscore ébrio\textunderscore )}
\end{itemize}
O mesmo que \textunderscore embriaguez\textunderscore .
\section{Ebrifestante}
\begin{itemize}
\item {Grp. gram.:adj.}
\end{itemize}
O mesmo que \textunderscore ebrifestivo\textunderscore . Cf. Castilho, \textunderscore Fastos\textunderscore , I, 43.
\section{Ebrifestivo}
\begin{itemize}
\item {Grp. gram.:adj.}
\end{itemize}
\begin{itemize}
\item {Proveniência:(De \textunderscore ébrio\textunderscore  + \textunderscore festivo\textunderscore )}
\end{itemize}
Alegre de embriaguez.
Que alegra, embriagando:«\textunderscore ...da ebrifestiva cepa\textunderscore ». Garção.
\section{Ébrio}
\begin{itemize}
\item {Grp. gram.:adj.}
\end{itemize}
\begin{itemize}
\item {Utilização:Fig.}
\end{itemize}
\begin{itemize}
\item {Grp. gram.:M.}
\end{itemize}
\begin{itemize}
\item {Proveniência:(Lat. \textunderscore ebrius\textunderscore )}
\end{itemize}
Embriagado, bêbedo.
Apaixonado, alucinado.
Sedento: \textunderscore ébrio de glória\textunderscore .
Indivíduo ébrio.
\section{Ebrioso}
\begin{itemize}
\item {Grp. gram.:adj.}
\end{itemize}
\begin{itemize}
\item {Proveniência:(Lat. \textunderscore ebriosus\textunderscore )}
\end{itemize}
Que se embriaga muitas vezes.
Que é effeito da embriaguez.
\section{Ebriridente}
\begin{itemize}
\item {fónica:ri}
\end{itemize}
\begin{itemize}
\item {Grp. gram.:adj.}
\end{itemize}
\begin{itemize}
\item {Proveniência:(De \textunderscore ébrio\textunderscore  + \textunderscore ridente\textunderscore )}
\end{itemize}
Que se ri, embriagado. Cf. Castilho, \textunderscore Fastos\textunderscore , I, 47.
\section{Ebrirridente}
\begin{itemize}
\item {Grp. gram.:adj.}
\end{itemize}
\begin{itemize}
\item {Proveniência:(De \textunderscore ébrio\textunderscore  + \textunderscore ridente\textunderscore )}
\end{itemize}
Que se ri, embriagado. Cf. Castilho, \textunderscore Fastos\textunderscore , I, 47.
\section{Ebulição}
\begin{itemize}
\item {Grp. gram.:f.}
\end{itemize}
\begin{itemize}
\item {Utilização:Fig.}
\end{itemize}
\begin{itemize}
\item {Proveniência:(Lat. \textunderscore ebullitio\textunderscore )}
\end{itemize}
Acto de ferver.
Desenvolvimento de bolhas de ar num líquido sujeito á acção do fogo.
Efervescência.
Fermentação.
Exaltação, agitação moral.
\section{Ebulidor}
\begin{itemize}
\item {Grp. gram.:m.}
\end{itemize}
Apparelho, annexo a algumas caldeiras de vapor. Cf. \textunderscore Inquér. Industr.\textunderscore , II, V. I, 152.
\section{Ebuliente}
\begin{itemize}
\item {Grp. gram.:adj.}
\end{itemize}
\begin{itemize}
\item {Proveniência:(Lat. \textunderscore ebulliens\textunderscore )}
\end{itemize}
Que ferve, fervente.
\section{Ebuliómetro}
\begin{itemize}
\item {Grp. gram.:m.}
\end{itemize}
\begin{itemize}
\item {Proveniência:(Do lat. \textunderscore ebullire\textunderscore  + gr. \textunderscore metron\textunderscore )}
\end{itemize}
Apparelho, para avaliar a percentagem do álcool em certos líquidos.
\section{Ebulioscópio}
\begin{itemize}
\item {Grp. gram.:f.}
\end{itemize}
\begin{itemize}
\item {Proveniência:(Do lat. \textunderscore ebullire\textunderscore  + gr. \textunderscore skopein\textunderscore )}
\end{itemize}
Instrumento, para avaliar a fôrça alcoólica de certos líquidos, sujeitos á ebulição.
\section{Ebullição}
\begin{itemize}
\item {Grp. gram.:f.}
\end{itemize}
\begin{itemize}
\item {Utilização:Fig.}
\end{itemize}
\begin{itemize}
\item {Proveniência:(Lat. \textunderscore ebullitio\textunderscore )}
\end{itemize}
Acto de ferver.
Desenvolvimento de bolhas de ar num líquido sujeito á acção do fogo.
Efervescência.
Fermentação.
Exaltação, agitação moral.
\section{Ebullidor}
\begin{itemize}
\item {Grp. gram.:m.}
\end{itemize}
Apparelho, annexo a algumas caldeiras de vapor. Cf. \textunderscore Inquér. Industr.\textunderscore , II, V. I, 152.
\section{Ebulliente}
\begin{itemize}
\item {Grp. gram.:adj.}
\end{itemize}
\begin{itemize}
\item {Proveniência:(Lat. \textunderscore ebulliens\textunderscore )}
\end{itemize}
Que ferve, fervente.
\section{Ebulliómetro}
\begin{itemize}
\item {Grp. gram.:m.}
\end{itemize}
\begin{itemize}
\item {Proveniência:(Do lat. \textunderscore ebullire\textunderscore  + gr. \textunderscore metron\textunderscore )}
\end{itemize}
Apparelho, para avaliar a percentagem do álcool em certos líquidos.
\section{Ebullioscópio}
\begin{itemize}
\item {Grp. gram.:f.}
\end{itemize}
\begin{itemize}
\item {Proveniência:(Do lat. \textunderscore ebullire\textunderscore  + gr. \textunderscore skopein\textunderscore )}
\end{itemize}
Instrumento, para avaliar a fôrça alcoólica de certos líquidos, sujeitos á ebullição.
\section{Ébulo}
\begin{itemize}
\item {Grp. gram.:m.}
\end{itemize}
\begin{itemize}
\item {Proveniência:(Lat. \textunderscore ebulum\textunderscore )}
\end{itemize}
O mesmo que \textunderscore engos\textunderscore .
\section{Eburnação}
\begin{itemize}
\item {Grp. gram.:f.}
\end{itemize}
\begin{itemize}
\item {Proveniência:(Do rad. de \textunderscore ebúrneo\textunderscore )}
\end{itemize}
Ossificação das cartilagens articulares.
\section{Ebúrneo}
\begin{itemize}
\item {Grp. gram.:adj.}
\end{itemize}
\begin{itemize}
\item {Proveniência:(Lat. \textunderscore eburneus\textunderscore )}
\end{itemize}
Feito de marfim.
Semelhante ao marfim, na côr ou na lisura: \textunderscore espáduas ebúrneas\textunderscore .
\section{Eça}
\begin{itemize}
\item {Grp. gram.:f.}
\end{itemize}
(Fórma usual, mas errónea, em vez de \textunderscore essa\textunderscore . V. \textunderscore essa\textunderscore ^2)
\section{Écano}
\begin{itemize}
\item {Grp. gram.:m.}
\end{itemize}
Gênero de insectos coleópteros.
\section{Ècar}
\begin{itemize}
\item {Grp. gram.:v. i.}
\end{itemize}
\begin{itemize}
\item {Utilização:Bras. de Minas}
\end{itemize}
Dar aviso de alguma coisa em voz alta.
(Provavelmente de \textunderscore eco\textunderscore  = \textunderscore echo\textunderscore )
\section{Ecatontarchia}
\begin{itemize}
\item {fónica:qui}
\end{itemize}
\begin{itemize}
\item {Grp. gram.:f.}
\end{itemize}
Rectângulo de 16 homens de frente, por 8 de fundo, que era a unidade táctica dos peltastos, na phalange macedónica.
\section{Ecatontarquia}
\begin{itemize}
\item {Grp. gram.:f.}
\end{itemize}
Rectângulo de 16 homens de frente, por 8 de fundo, que era a unidade táctica dos peltastos, na falange macedónica.
\section{Écbase}
\begin{itemize}
\item {Grp. gram.:f.}
\end{itemize}
\begin{itemize}
\item {Utilização:Rhet.}
\end{itemize}
\begin{itemize}
\item {Proveniência:(Lat. \textunderscore ecbasis\textunderscore )}
\end{itemize}
Digressão no discurso.
\section{Écbola}
\begin{itemize}
\item {Grp. gram.:f.}
\end{itemize}
\begin{itemize}
\item {Proveniência:(Lat. \textunderscore ecbola\textunderscore )}
\end{itemize}
Espécie de dardo ou lança, entre os antigos Romanos.
\section{Ecbólade}
\begin{itemize}
\item {Grp. gram.:f.}
\end{itemize}
\begin{itemize}
\item {Proveniência:(Lat. \textunderscore ecbolas\textunderscore , \textunderscore ecboladis\textunderscore )}
\end{itemize}
Uva egýpcia, a que se attribuía a propriedade de produzir o abôrto.
\section{Ecbólico}
\begin{itemize}
\item {Grp. gram.:adj.}
\end{itemize}
\begin{itemize}
\item {Proveniência:(Do gr. \textunderscore ekbole\textunderscore )}
\end{itemize}
Que produz abôrto.
Evacuante.
\section{Ecchymosar-se}
\begin{itemize}
\item {fónica:qui}
\end{itemize}
\begin{itemize}
\item {Grp. gram.:v. p.}
\end{itemize}
Cobrir-se de ecchymoses.
\section{Ecchymose}
\begin{itemize}
\item {fónica:qui}
\end{itemize}
\begin{itemize}
\item {Grp. gram.:f.}
\end{itemize}
\begin{itemize}
\item {Proveniência:(Gr. \textunderscore ekkhumosis\textunderscore )}
\end{itemize}
Mancha avermelhada ou escura, formada na pelle por sangue extravasado em consequência de contusão.
\section{Ecchymótico}
\begin{itemize}
\item {fónica:qui}
\end{itemize}
\begin{itemize}
\item {Grp. gram.:adj.}
\end{itemize}
\begin{itemize}
\item {Proveniência:(Gr. \textunderscore ekkhumotikos\textunderscore )}
\end{itemize}
Que tem o carácter da ecchymose.
\section{Ecclesiasticamente}
\begin{itemize}
\item {Grp. gram.:adv.}
\end{itemize}
\begin{itemize}
\item {Proveniência:(De \textunderscore ecclesiástico\textunderscore )}
\end{itemize}
Segundo as práticas da Igreja.
Á maneira dos padres.
\section{Ecclesiástico}
\begin{itemize}
\item {Grp. gram.:adj.}
\end{itemize}
\begin{itemize}
\item {Grp. gram.:M.}
\end{itemize}
\begin{itemize}
\item {Proveniência:(Lat. \textunderscore ecclesiasticus\textunderscore )}
\end{itemize}
Relativo á Igreja, ao clero.
Sacerdote, padre.
\section{Éccope}
\begin{itemize}
\item {Grp. gram.:f.}
\end{itemize}
\begin{itemize}
\item {Proveniência:(Lat. \textunderscore eccope\textunderscore )}
\end{itemize}
Incisão cirúrgica, feita com instrumento cortante, em direcção oblíqua á superfície e sem perda de substância.
\section{Eccoprótico}
\begin{itemize}
\item {Grp. gram.:adj.}
\end{itemize}
\begin{itemize}
\item {Grp. gram.:M.}
\end{itemize}
\begin{itemize}
\item {Proveniência:(Gr. \textunderscore ekkoprotikos\textunderscore )}
\end{itemize}
Laxativo.
Laxante.
\section{Écdico}
\begin{itemize}
\item {Grp. gram.:m.}
\end{itemize}
\begin{itemize}
\item {Proveniência:(Lat. \textunderscore ecdicus\textunderscore )}
\end{itemize}
Advogado ou defensor do povo, nas cidades antigas.
\section{Ecdúsias}
\begin{itemize}
\item {Grp. gram.:f. pl.}
\end{itemize}
Antigas festas de Creta, em honra de Latona.
\section{...ecer}
\begin{itemize}
\item {Proveniência:(Do lat. \textunderscore ...escere\textunderscore )}
\end{itemize}
\textunderscore suf.\textunderscore  de muitos inchoativos.
\section{Echacorvos}
\begin{itemize}
\item {Grp. gram.:m.}
\end{itemize}
\begin{itemize}
\item {Utilização:Ant.}
\end{itemize}
\begin{itemize}
\item {Utilização:Fig.}
\end{itemize}
Prègador, que andava pelas pequenas povoações, fazendo homilias e relhendo esmolas.
Embusteiro.
(Cast. \textunderscore echacuervos\textunderscore )
\section{Echalota}
\begin{itemize}
\item {Grp. gram.:f.}
\end{itemize}
\begin{itemize}
\item {Proveniência:(Fr. \textunderscore échalote\textunderscore )}
\end{itemize}
Planta hortense, bulbosa, (\textunderscore allium ascalomium\textunderscore ).
\section{Echar}
\begin{itemize}
\item {Grp. gram.:v. t.}
\end{itemize}
\begin{itemize}
\item {Utilização:Ant.}
\end{itemize}
O mesmo que \textunderscore deixar\textunderscore .
\section{Echeias}
\begin{itemize}
\item {fónica:quei}
\end{itemize}
\begin{itemize}
\item {Grp. gram.:f. pl.}
\end{itemize}
\begin{itemize}
\item {Proveniência:(Do gr. \textunderscore echeis\textunderscore )}
\end{itemize}
Vasos de bronze, que, collocados nos theatros gregos, tornavam êstes mais sonoros.
\section{Echeus}
\begin{itemize}
\item {fónica:que}
\end{itemize}
\begin{itemize}
\item {Grp. gram.:m. pl.}
\end{itemize}
O mesmo que \textunderscore echeias\textunderscore .
\section{Echevéria}
\begin{itemize}
\item {Grp. gram.:f.}
\end{itemize}
Planta suculenta do Brasil.
\section{Echião}
\begin{itemize}
\item {fónica:qui}
\end{itemize}
\begin{itemize}
\item {Grp. gram.:m.}
\end{itemize}
\begin{itemize}
\item {Proveniência:(Gr. \textunderscore ekhion\textunderscore )}
\end{itemize}
Antigo medicamento, preparado com cinzas de víbora.
\section{Echidna}
\begin{itemize}
\item {fónica:qui}
\end{itemize}
\begin{itemize}
\item {Grp. gram.:m.}
\end{itemize}
\begin{itemize}
\item {Proveniência:(Gr. \textunderscore ekhidna\textunderscore )}
\end{itemize}
Mammífero australiano coberto de espinhos, como o ouriço.
Constellação da Hydra.
\section{Echídnico}
\begin{itemize}
\item {fónica:qui}
\end{itemize}
\begin{itemize}
\item {Grp. gram.:adj.}
\end{itemize}
\begin{itemize}
\item {Proveniência:(Do gr. \textunderscore ekhidna\textunderscore )}
\end{itemize}
Relativo á víbora, próprio da víbora.
\section{Echidnina}
\begin{itemize}
\item {fónica:qui}
\end{itemize}
\begin{itemize}
\item {Grp. gram.:f.}
\end{itemize}
Substância orgânica, que é princípio activo do veneno da víbora.
(Cp. \textunderscore echídnico\textunderscore )
\section{Echidno}
\begin{itemize}
\item {fónica:qui}
\end{itemize}
\begin{itemize}
\item {Grp. gram.:m.}
\end{itemize}
O mesmo que \textunderscore echidna\textunderscore .
\section{Echínides}
\begin{itemize}
\item {fónica:qui}
\end{itemize}
\begin{itemize}
\item {Grp. gram.:m. pl.}
\end{itemize}
\begin{itemize}
\item {Utilização:Zool.}
\end{itemize}
\begin{itemize}
\item {Proveniência:(Do gr. \textunderscore ekhinos\textunderscore )}
\end{itemize}
Classe de echinodermes.
\section{Echinípede}
\begin{itemize}
\item {fónica:qui}
\end{itemize}
\begin{itemize}
\item {Grp. gram.:adj.}
\end{itemize}
\begin{itemize}
\item {Utilização:Zool.}
\end{itemize}
\begin{itemize}
\item {Proveniência:(Do lat. \textunderscore echinus\textunderscore  + \textunderscore pes\textunderscore )}
\end{itemize}
Que tem as patas revestidas de pelos ásperos.
\section{Échino}
\begin{itemize}
\item {fónica:qui}
\end{itemize}
\begin{itemize}
\item {Grp. gram.:m.}
\end{itemize}
\begin{itemize}
\item {Proveniência:(Gr. \textunderscore ekhinos\textunderscore )}
\end{itemize}
Moldura em quarto de círculo.
Ornato oval e convexo.
\section{Echinocarpo}
\begin{itemize}
\item {fónica:qui}
\end{itemize}
\begin{itemize}
\item {Grp. gram.:adj.}
\end{itemize}
\begin{itemize}
\item {Utilização:Bot.}
\end{itemize}
\begin{itemize}
\item {Grp. gram.:M.}
\end{itemize}
\begin{itemize}
\item {Proveniência:(Do gr. \textunderscore ekhinos\textunderscore  + \textunderscore karpos\textunderscore )}
\end{itemize}
Que produz frutos erriçados de pontas ásperas.
Grande árvore da ilha de Java.
\section{Echinococco}
\begin{itemize}
\item {fónica:qui}
\end{itemize}
\begin{itemize}
\item {Grp. gram.:m.}
\end{itemize}
\begin{itemize}
\item {Proveniência:(Do gr. \textunderscore ekhinos\textunderscore  + \textunderscore kokkos\textunderscore )}
\end{itemize}
Entozoário, que se encontra nos hydátides.
\section{Echinodermes}
\begin{itemize}
\item {fónica:qui}
\end{itemize}
\begin{itemize}
\item {Grp. gram.:m. pl.}
\end{itemize}
\begin{itemize}
\item {Utilização:Zool.}
\end{itemize}
\begin{itemize}
\item {Proveniência:(Do gr. \textunderscore ekhinos\textunderscore  + \textunderscore derma\textunderscore )}
\end{itemize}
Animaes, que têm a pelle coberta de tubérculos ou espinhos.
\section{Echinoides}
\begin{itemize}
\item {fónica:qui}
\end{itemize}
\begin{itemize}
\item {Grp. gram.:m. pl.}
\end{itemize}
\begin{itemize}
\item {Utilização:Zool.}
\end{itemize}
\begin{itemize}
\item {Proveniência:(Do gr. \textunderscore ekhinos\textunderscore  + \textunderscore eidos\textunderscore )}
\end{itemize}
Espécie de echinoderme, a que pertence o ouriço-do-mar.
\section{Echinómetra}
\begin{itemize}
\item {fónica:qui}
\end{itemize}
\begin{itemize}
\item {Grp. gram.:m.}
\end{itemize}
\begin{itemize}
\item {Proveniência:(Gr. \textunderscore ekhinometrai\textunderscore )}
\end{itemize}
Espécie de ouriço-do-mar.
\section{Echinómetro}
\begin{itemize}
\item {fónica:qui}
\end{itemize}
\begin{itemize}
\item {Grp. gram.:m.}
\end{itemize}
O mesmo que \textunderscore echinómetra\textunderscore .
\section{Echinóphora}
\begin{itemize}
\item {fónica:qui}
\end{itemize}
\begin{itemize}
\item {Grp. gram.:f.}
\end{itemize}
\begin{itemize}
\item {Proveniência:(De \textunderscore echinóphoro\textunderscore )}
\end{itemize}
Espécie de mollusco de concha raiada.
\section{Echinóphoro}
\begin{itemize}
\item {fónica:qui}
\end{itemize}
\begin{itemize}
\item {Grp. gram.:adj.}
\end{itemize}
\begin{itemize}
\item {Utilização:Bot.}
\end{itemize}
\begin{itemize}
\item {Proveniência:(Do gr. \textunderscore ekhinos\textunderscore  + \textunderscore phoros\textunderscore )}
\end{itemize}
Que tem espinhos.
\section{Echinophthalmia}
\begin{itemize}
\item {fónica:qui}
\end{itemize}
\begin{itemize}
\item {Grp. gram.:f.}
\end{itemize}
\begin{itemize}
\item {Utilização:Med.}
\end{itemize}
\begin{itemize}
\item {Proveniência:(Do gr. \textunderscore ekhinos\textunderscore  + \textunderscore ophthalmos\textunderscore )}
\end{itemize}
Inflammação das pálpebras, na parte occupada pelas pestanas.
\section{Echinópode}
\begin{itemize}
\item {fónica:qui}
\end{itemize}
\begin{itemize}
\item {Grp. gram.:m.}
\end{itemize}
\begin{itemize}
\item {Utilização:Bot.}
\end{itemize}
\begin{itemize}
\item {Proveniência:(Do gr. \textunderscore ekhinos\textunderscore  + \textunderscore pous\textunderscore , \textunderscore podos\textunderscore )}
\end{itemize}
Gênero de plantas vivazes das regiões quentes da Europa.
\section{Echinópseas}
\begin{itemize}
\item {fónica:qui}
\end{itemize}
\begin{itemize}
\item {Grp. gram.:m. pl.}
\end{itemize}
O mesmo que \textunderscore echinopsídeas\textunderscore .
\section{Echinopsídeas}
\begin{itemize}
\item {fónica:qui}
\end{itemize}
\begin{itemize}
\item {Grp. gram.:m. pl.}
\end{itemize}
\begin{itemize}
\item {Utilização:Bot.}
\end{itemize}
Grupo de vegetaes, da fam. das synanthéreas.
\section{Echinorrhynco}
\begin{itemize}
\item {fónica:qui}
\end{itemize}
\begin{itemize}
\item {Grp. gram.:m.}
\end{itemize}
\begin{itemize}
\item {Proveniência:(Do gr. \textunderscore ekhinos\textunderscore  + \textunderscore rhunkos\textunderscore )}
\end{itemize}
Entozoário, que se encontra em alguns animaes e não no homem.
\section{Echinospermo}
\begin{itemize}
\item {fónica:qui}
\end{itemize}
\begin{itemize}
\item {Grp. gram.:adj.}
\end{itemize}
\begin{itemize}
\item {Utilização:Bot.}
\end{itemize}
\begin{itemize}
\item {Proveniência:(Do gr. \textunderscore ekhinos\textunderscore  + \textunderscore sperma\textunderscore )}
\end{itemize}
Cujos grãos são cobertos de pelos ásperos.
\section{Echioglossa}
\begin{itemize}
\item {fónica:qui}
\end{itemize}
\begin{itemize}
\item {Grp. gram.:f.}
\end{itemize}
\begin{itemize}
\item {Proveniência:(Do gr. \textunderscore ekhis\textunderscore , vibora, e \textunderscore glossa\textunderscore , língua)}
\end{itemize}
Gênero de orchídeas.
\section{Echioide}
\begin{itemize}
\item {fónica:qui}
\end{itemize}
\begin{itemize}
\item {Grp. gram.:adj.}
\end{itemize}
\begin{itemize}
\item {Grp. gram.:M.}
\end{itemize}
\begin{itemize}
\item {Proveniência:(Do gr. \textunderscore ekhis\textunderscore  + \textunderscore eidos\textunderscore )}
\end{itemize}
Semelhante á víbora ou á cabeça da víbora.
Planta, cuja semente é semelhante á cabeça da víbora.
\section{Echo}
\begin{itemize}
\item {Grp. gram.:m.}
\end{itemize}
\begin{itemize}
\item {Utilização:Bras. do N}
\end{itemize}
\begin{itemize}
\item {Proveniência:(Gr. \textunderscore ekho\textunderscore )}
\end{itemize}
Repetição, mais ou menos clara, de um som reflectido por um corpo a que o levam as ondas sonoras.
Som repetido, repetição.
Pessôa, que repete.
Lugar, onde se produz o echo.
Bom acolhimento: \textunderscore esta doutrina teve echo nas Universidades\textunderscore .
Reflexo, impressão.
Memória.
Fama.
O mesmo que \textunderscore grito\textunderscore : \textunderscore soltar um echo\textunderscore .
\section{Ecfonema}
\begin{itemize}
\item {Grp. gram.:m.}
\end{itemize}
\begin{itemize}
\item {Proveniência:(Gr. \textunderscore ekphonema\textunderscore )}
\end{itemize}
Elevação súbita da voz, com exclamações e frases incompletas, por efeito de paixão ou facto surpreendente.
\section{Écfora}
\begin{itemize}
\item {Grp. gram.:f.}
\end{itemize}
\begin{itemize}
\item {Proveniência:(Lat. \textunderscore ecphora\textunderscore )}
\end{itemize}
Saliência da cimalha ou de outro membro arquitectónico.
\section{Ecfrático}
\begin{itemize}
\item {Grp. gram.:adj.}
\end{itemize}
\begin{itemize}
\item {Proveniência:(Gr. \textunderscore ekphratikos\textunderscore )}
\end{itemize}
O mesmo que \textunderscore aperitivo\textunderscore .
\section{...êcho}
\begin{itemize}
\item {Grp. gram.:suf.}
\end{itemize}
(designativo de depreciação ou deminuição)
\section{Echoar}
\begin{itemize}
\item {Grp. gram.:v. t.}
\end{itemize}
\begin{itemize}
\item {Grp. gram.:V. i.}
\end{itemize}
\begin{itemize}
\item {Proveniência:(De \textunderscore echo\textunderscore )}
\end{itemize}
Repetir.
Repercutir.
Fazer echo.
Reproduzir-se ao longe, no tempo ou no espaço.
Tornar-se notado ou famoso em tempos posteriores.
\section{Echocinesia}
\begin{itemize}
\item {fónica:co}
\end{itemize}
\begin{itemize}
\item {Grp. gram.:f.}
\end{itemize}
\begin{itemize}
\item {Utilização:Med.}
\end{itemize}
\begin{itemize}
\item {Proveniência:(Do gr. \textunderscore echo\textunderscore  + \textunderscore kinesis\textunderscore )}
\end{itemize}
Imitação do gesto, observada nos hystéricos e epilépticos.
\section{Echóico}
\begin{itemize}
\item {Grp. gram.:adj.}
\end{itemize}
\begin{itemize}
\item {Proveniência:(Lat. \textunderscore echoicus\textunderscore )}
\end{itemize}
Dizia-se do verso latino, cujas duas últimas palavras terminavam em vogal idêntica.
\section{Echolalia}
\begin{itemize}
\item {Grp. gram.:f.}
\end{itemize}
\begin{itemize}
\item {Utilização:Med.}
\end{itemize}
\begin{itemize}
\item {Proveniência:(Do gr. \textunderscore ekho\textunderscore  + \textunderscore lalein\textunderscore )}
\end{itemize}
Moléstia nervosa, em que o doente repete involuntariamente palavras ou phrases, que ouviu ou que êlle próprio pronunciou.
\section{Echomatismo}
\begin{itemize}
\item {fónica:co}
\end{itemize}
\begin{itemize}
\item {Grp. gram.:m.}
\end{itemize}
\begin{itemize}
\item {Utilização:Med.}
\end{itemize}
Impulsão mórbida, para repetir vozes e movimentos dos circunstantes.
\section{Echometria}
\begin{itemize}
\item {Grp. gram.:f.}
\end{itemize}
\begin{itemize}
\item {Utilização:Phýs.}
\end{itemize}
\begin{itemize}
\item {Proveniência:(Do gr. \textunderscore ekho\textunderscore  + \textunderscore metron\textunderscore )}
\end{itemize}
Arte de calcular a reflexão dos sons.
\section{Echómetro}
\begin{itemize}
\item {fónica:có}
\end{itemize}
\begin{itemize}
\item {Grp. gram.:m.}
\end{itemize}
\begin{itemize}
\item {Proveniência:(Do gr. \textunderscore ekho\textunderscore  + \textunderscore metron\textunderscore )}
\end{itemize}
Régua graduada, que se emprega em echometria.
\section{Eclampsia}
\begin{itemize}
\item {Grp. gram.:f.}
\end{itemize}
\begin{itemize}
\item {Proveniência:(Do gr. \textunderscore eklampsis\textunderscore )}
\end{itemize}
Doença convulsiva, que se manifesta nas crianças e nas puérperas, acompanhada ordinariamente da perda dos sentidos.
\section{Eclâmptico}
\begin{itemize}
\item {Grp. gram.:adj.}
\end{itemize}
Relativo a eclampsia.
\section{Eclecticamente}
\begin{itemize}
\item {Grp. gram.:adv.}
\end{itemize}
De modo ecléctico.
\section{Eclecticismo}
\begin{itemize}
\item {Grp. gram.:m.}
\end{itemize}
O mesmo que \textunderscore eclectismo\textunderscore .
\section{Ecléctico}
\begin{itemize}
\item {Grp. gram.:adj.}
\end{itemize}
\begin{itemize}
\item {Grp. gram.:M.}
\end{itemize}
\begin{itemize}
\item {Proveniência:(Gr. \textunderscore eklektikos\textunderscore )}
\end{itemize}
Relativo ao eclectismo.
Sectário de eclectismo.
Aquelle que, de vários systemas ou opiniões, adopta aquillo que lhe convém ou que julga aproveitável.
\section{Eclectismo}
\begin{itemize}
\item {Grp. gram.:m.}
\end{itemize}
\begin{itemize}
\item {Proveniência:(De \textunderscore ecléctico\textunderscore )}
\end{itemize}
Systema philosóphico, formado de elementos colhidos em diversos systemas.
Liberdade ou hábito de preferir o que se julga melhor em sciência, arte, literatura ou política, sem que se siga exclusivamente systema algum.
\section{Eclegma}
\begin{itemize}
\item {Grp. gram.:m.}
\end{itemize}
\begin{itemize}
\item {Proveniência:(Gr. \textunderscore ekleigma\textunderscore )}
\end{itemize}
Xarope espêsso, para tratamento de crianças.
\section{Eclesiasticamente}
\begin{itemize}
\item {Grp. gram.:adv.}
\end{itemize}
\begin{itemize}
\item {Proveniência:(De \textunderscore eclesiástico\textunderscore )}
\end{itemize}
Segundo as práticas da Igreja.
Á maneira dos padres.
\section{Eclesiástico}
\begin{itemize}
\item {Grp. gram.:adj.}
\end{itemize}
\begin{itemize}
\item {Grp. gram.:M.}
\end{itemize}
\begin{itemize}
\item {Proveniência:(Lat. \textunderscore ecclesiasticus\textunderscore )}
\end{itemize}
Relativo á Igreja, ao clero.
Sacerdote, padre.
\section{Eclímetro}
\begin{itemize}
\item {Grp. gram.:m.}
\end{itemize}
\begin{itemize}
\item {Proveniência:(Do gr. \textunderscore eklegein\textunderscore  + \textunderscore metron\textunderscore )}
\end{itemize}
Instrumento, que consta principalmente de um óculo e arco graduado, para medir distâncias verticaes e conhecer as differenças de nível num terreno.
\section{Eclipsar}
\begin{itemize}
\item {Grp. gram.:v. t.}
\end{itemize}
\begin{itemize}
\item {Utilização:Fig.}
\end{itemize}
\begin{itemize}
\item {Grp. gram.:V. i.}
\end{itemize}
\begin{itemize}
\item {Proveniência:(De \textunderscore eclipse\textunderscore )}
\end{itemize}
Interceptar a luz de (um astro).
Obscurecer, occultar.
Exceder; offuscar.
Soffrer eclipse, (falando-se de um astro).
Obscurecer-se. Cf. Rui Barb., \textunderscore Réplica\textunderscore , 160.
\section{Eclípse}
\begin{itemize}
\item {Grp. gram.:m.}
\end{itemize}
\begin{itemize}
\item {Utilização:Fig.}
\end{itemize}
\begin{itemize}
\item {Proveniência:(Do gr. \textunderscore ekleipsis\textunderscore )}
\end{itemize}
Desapparecimento apparente de um astro, resultante da posição de outro astro entre aquelle e o observador.
Obscurecimento moral ou intellectual.
\section{Eclíptica}
\begin{itemize}
\item {Grp. gram.:f.}
\end{itemize}
\begin{itemize}
\item {Utilização:Astron.}
\end{itemize}
\begin{itemize}
\item {Proveniência:(De \textunderscore eclíptico\textunderscore )}
\end{itemize}
Círculo imaginário, correspondente á órbita apparente do Sol em volta da Terra.
Órbita, que a Terra descreve realmente num anno em volta do Sol.
\section{Eclíptico}
\begin{itemize}
\item {Grp. gram.:adj.}
\end{itemize}
\begin{itemize}
\item {Proveniência:(Do gr. \textunderscore ekleiptikos\textunderscore )}
\end{itemize}
Relativo aos eclipses.
\section{Écloga}
\begin{itemize}
\item {Grp. gram.:f.}
\end{itemize}
\begin{itemize}
\item {Proveniência:(Lat. \textunderscore ecloga\textunderscore )}
\end{itemize}
Poesia pastoril, ordinariamente em diálogo.
\section{Eclosão}
\begin{itemize}
\item {Grp. gram.:f.}
\end{itemize}
\begin{itemize}
\item {Utilização:inútil}
\end{itemize}
\begin{itemize}
\item {Utilização:Gal}
\end{itemize}
\begin{itemize}
\item {Proveniência:(Fr. \textunderscore éclosion\textunderscore )}
\end{itemize}
Acto de sair á luz; acto de desabrochar.
Apparecimento.
Desenvolvimento. Cf. Latino, \textunderscore Or. da Corôa\textunderscore , LXXVII.
\section{Eclusa}
\begin{itemize}
\item {Grp. gram.:f.}
\end{itemize}
\begin{itemize}
\item {Utilização:Gal}
\end{itemize}
(V.esclusa)
\section{Ecnefia}
\begin{itemize}
\item {Grp. gram.:f.}
\end{itemize}
\begin{itemize}
\item {Proveniência:(Gr. \textunderscore eknephias\textunderscore )}
\end{itemize}
Tempestade sem chuva.
Furacão, formado por ventos contrários.
\section{Ecnephia}
\begin{itemize}
\item {Grp. gram.:f.}
\end{itemize}
\begin{itemize}
\item {Proveniência:(Gr. \textunderscore eknephias\textunderscore )}
\end{itemize}
Tempestade sem chuva.
Furacão, formado por ventos contrários.
\section{Eco}
\begin{itemize}
\item {Grp. gram.:m.}
\end{itemize}
\begin{itemize}
\item {Utilização:Bras. do N}
\end{itemize}
\begin{itemize}
\item {Proveniência:(Gr. \textunderscore ekho\textunderscore )}
\end{itemize}
Repetição, mais ou menos clara, de um som reflectido por um corpo a que o levam as ondas sonoras.
Som repetido, repetição.
Pessôa, que repete.
Lugar, onde se produz o echo.
Bom acolhimento: \textunderscore esta doutrina teve eco nas Universidades\textunderscore .
Reflexo, impressão.
Memória.
Fama.
O mesmo que \textunderscore grito\textunderscore : \textunderscore soltar um eco\textunderscore .
\section{Ecô}
\begin{itemize}
\item {Grp. gram.:interj.}
\end{itemize}
\begin{itemize}
\item {Utilização:Bras}
\end{itemize}
(Usada pelos caçadores, quando açulam os cães)
\section{Ecoar}
\begin{itemize}
\item {Grp. gram.:v. t.}
\end{itemize}
\begin{itemize}
\item {Grp. gram.:V. i.}
\end{itemize}
\begin{itemize}
\item {Proveniência:(De \textunderscore eco\textunderscore )}
\end{itemize}
Repetir.
Repercutir.
Fazer eco.
Reproduzir-se ao longe, no tempo ou no espaço.
Tornar-se notado ou famoso em tempos posteriores.
\section{Ecocinesia}
\begin{itemize}
\item {Grp. gram.:f.}
\end{itemize}
\begin{itemize}
\item {Utilização:Med.}
\end{itemize}
\begin{itemize}
\item {Proveniência:(Do gr. \textunderscore echo\textunderscore  + \textunderscore kinesis\textunderscore )}
\end{itemize}
Imitação do gesto, observada nos histéricos e epilépticos.
\section{Ecoico}
\begin{itemize}
\item {Grp. gram.:adj.}
\end{itemize}
\begin{itemize}
\item {Proveniência:(Lat. \textunderscore echoicus\textunderscore )}
\end{itemize}
Dizia-se do verso latino, cujas duas últimas palavras terminavam em vogal idêntica.
\section{Ecolalia}
\begin{itemize}
\item {Grp. gram.:f.}
\end{itemize}
\begin{itemize}
\item {Utilização:Med.}
\end{itemize}
\begin{itemize}
\item {Proveniência:(Do gr. \textunderscore ekho\textunderscore  + \textunderscore lalein\textunderscore )}
\end{itemize}
Moléstia nervosa, em que o doente repete involuntariamente palavras ou frases, que ouviu ou que êle próprio pronunciou.
\section{Ecomatismo}
\begin{itemize}
\item {Grp. gram.:m.}
\end{itemize}
\begin{itemize}
\item {Utilização:Med.}
\end{itemize}
Impulsão mórbida, para repetir vozes e movimentos dos circunstantes.
\section{Ecometria}
\begin{itemize}
\item {Grp. gram.:f.}
\end{itemize}
\begin{itemize}
\item {Utilização:Phýs.}
\end{itemize}
\begin{itemize}
\item {Proveniência:(Do gr. \textunderscore ekho\textunderscore  + \textunderscore metron\textunderscore )}
\end{itemize}
Arte de calcular a reflexão dos sons.
\section{Ecómetro}
\begin{itemize}
\item {Grp. gram.:m.}
\end{itemize}
\begin{itemize}
\item {Proveniência:(Do gr. \textunderscore ekho\textunderscore  + \textunderscore metron\textunderscore )}
\end{itemize}
Régua graduada, que se emprega em ecometria.
\section{Economato}
\begin{itemize}
\item {Grp. gram.:m.}
\end{itemize}
Cargo de ecónomo.
A Repartição do ecónomo.
\section{Economia}
\begin{itemize}
\item {Grp. gram.:f.}
\end{itemize}
\begin{itemize}
\item {Utilização:Fig.}
\end{itemize}
\begin{itemize}
\item {Utilização:Ant.}
\end{itemize}
\begin{itemize}
\item {Grp. gram.:Pl.}
\end{itemize}
\begin{itemize}
\item {Proveniência:(Gr. \textunderscore oikonomia\textunderscore )}
\end{itemize}
Bôa ordem, em qualquer administração particular ou pública.
Emprêgo discreto, que se faz de qualquer coisa.
Hábito de poupar, de gastar pouco.
Harmonia entre as partes de um todo.
Organismo animal ou vegetal.
Leis, que regulam êsse organismo.
O mesmo que \textunderscore economato\textunderscore .
Dinheiro, acumulado por effeito de economia.
\section{Economicamente}
\begin{itemize}
\item {Grp. gram.:adv.}
\end{itemize}
De modo económico.
\section{Económico}
\begin{itemize}
\item {Grp. gram.:adj.}
\end{itemize}
\begin{itemize}
\item {Proveniência:(Gr. \textunderscore oikonomikos\textunderscore )}
\end{itemize}
Relativo á economia.
Conforme aos preceitos da economia.
Que gasta com parcimónia.
Poupado.
Barato, que custa pouco: \textunderscore um jantar económico\textunderscore .
\section{Economista}
\begin{itemize}
\item {Grp. gram.:m.}
\end{itemize}
\begin{itemize}
\item {Proveniência:(De \textunderscore economia\textunderscore )}
\end{itemize}
Aquelle que trata especialmente de questões económicas e sociaes.
\section{Economizador}
\begin{itemize}
\item {Grp. gram.:adj.}
\end{itemize}
\begin{itemize}
\item {Grp. gram.:M.}
\end{itemize}
Que economiza.
Aquelle que economiza.
\section{Economizar}
\begin{itemize}
\item {Grp. gram.:v. t.}
\end{itemize}
\begin{itemize}
\item {Proveniência:(De \textunderscore economia\textunderscore )}
\end{itemize}
Administrar economicamente.
Despender com parcimónia.
Poupar.
Acumular, poupando.
\section{Ecónomo}
\begin{itemize}
\item {Grp. gram.:m.}
\end{itemize}
\begin{itemize}
\item {Grp. gram.:Adj.}
\end{itemize}
\begin{itemize}
\item {Proveniência:(Gr. \textunderscore oikonomos\textunderscore )}
\end{itemize}
Aquelle que dirige a administração de uma casa.
Despenseiro; mordomo.
Que administra ou dirige uma casa. Cf. Filinto, IV, 265.
\section{Écope}
\begin{itemize}
\item {Grp. gram.:f.}
\end{itemize}
\begin{itemize}
\item {Proveniência:(Lat. \textunderscore eccope\textunderscore )}
\end{itemize}
Incisão cirúrgica, feita com instrumento cortante, em direcção oblíqua á superfície e sem perda de substância.
\section{Ecopleura}
\begin{itemize}
\item {Grp. gram.:f.}
\end{itemize}
\begin{itemize}
\item {Utilização:Zool.}
\end{itemize}
Gênero de acalephos, que comprehende certos animaes marinhos, que se encontram em abundância, junto ao Cabo da Bôa-Esperança.
\section{Ecoprótico}
\begin{itemize}
\item {Grp. gram.:adj.}
\end{itemize}
\begin{itemize}
\item {Grp. gram.:M.}
\end{itemize}
\begin{itemize}
\item {Proveniência:(Gr. \textunderscore ekkoprotikos\textunderscore )}
\end{itemize}
Laxativo.
Laxante.
\section{Ecoxupé}
\begin{itemize}
\item {Grp. gram.:interj.}
\end{itemize}
\begin{itemize}
\item {Utilização:Bras. do N}
\end{itemize}
O mesmo que \textunderscore ecô\textunderscore .
\section{Ecphonema}
\begin{itemize}
\item {Grp. gram.:m.}
\end{itemize}
\begin{itemize}
\item {Proveniência:(Gr. \textunderscore ekphonema\textunderscore )}
\end{itemize}
Elevação súbita da voz, com exclamações e phrases incompletas, por effeito de paixão ou facto surprehendente.
\section{Écphora}
\begin{itemize}
\item {Grp. gram.:f.}
\end{itemize}
\begin{itemize}
\item {Proveniência:(Lat. \textunderscore ecphora\textunderscore )}
\end{itemize}
Saliência da cimalha ou de outro membro architectónico.
\section{Ecphrático}
\begin{itemize}
\item {Grp. gram.:adj.}
\end{itemize}
\begin{itemize}
\item {Proveniência:(Gr. \textunderscore ekphratikos\textunderscore )}
\end{itemize}
O mesmo que \textunderscore aperitivo\textunderscore .
\section{Ecpiesma}
\begin{itemize}
\item {Grp. gram.:m.}
\end{itemize}
\begin{itemize}
\item {Proveniência:(Gr. \textunderscore ekpiesma\textunderscore )}
\end{itemize}
Fractura de crânio, quando as esquírolas comprimem as membranas cerebraes.
\section{Ecplexia}
\begin{itemize}
\item {fónica:csi}
\end{itemize}
\begin{itemize}
\item {Grp. gram.:f.}
\end{itemize}
\begin{itemize}
\item {Utilização:Med.}
\end{itemize}
Delírio, causado por susto repentino.
\section{Ecsarcoma}
\begin{itemize}
\item {Grp. gram.:m.}
\end{itemize}
\begin{itemize}
\item {Utilização:Des.}
\end{itemize}
\begin{itemize}
\item {Proveniência:(Gr. \textunderscore eksarkoma\textunderscore )}
\end{itemize}
Excrescência carnosa.
\section{Éctase}
\begin{itemize}
\item {Grp. gram.:f.}
\end{itemize}
\begin{itemize}
\item {Proveniência:(Gr. \textunderscore ektasis\textunderscore )}
\end{itemize}
Alongamento de uma sýllaba breve, na prosódia grega.
\section{Ectasia}
\begin{itemize}
\item {Grp. gram.:f.}
\end{itemize}
\begin{itemize}
\item {Proveniência:(De \textunderscore éctase\textunderscore )}
\end{itemize}
Qualquer doença, caracterizada por inchação ou dilatação.
\section{Éctese}
\begin{itemize}
\item {Grp. gram.:f.}
\end{itemize}
\begin{itemize}
\item {Proveniência:(Gr. \textunderscore ekthesis\textunderscore )}
\end{itemize}
Profissão de fé, apresentada pelo imperador Heráclio em 639, na qual se reconhecia só uma vontade em Cristo.
\section{Écthese}
\begin{itemize}
\item {Grp. gram.:f.}
\end{itemize}
\begin{itemize}
\item {Proveniência:(Gr. \textunderscore ekthesis\textunderscore )}
\end{itemize}
Profissão de fé, apresentada pelo imperador Heráclio em 639, na qual se reconhecia só uma vontade em Christo.
\section{Ecthlipse}
\begin{itemize}
\item {Grp. gram.:f.}
\end{itemize}
\begin{itemize}
\item {Utilização:Gram.}
\end{itemize}
\begin{itemize}
\item {Proveniência:(Gr. \textunderscore ekthilípsis\textunderscore )}
\end{itemize}
Elisão do \textunderscore m\textunderscore  ou do \textunderscore s\textunderscore  no fim de uma palavra.
\section{Écthyma}
\begin{itemize}
\item {Grp. gram.:m.}
\end{itemize}
\begin{itemize}
\item {Proveniência:(Gr. \textunderscore ektuma\textunderscore )}
\end{itemize}
Phlegmasia, que ataca os follículos sebáceos.
\section{Ectillótico}
\begin{itemize}
\item {Grp. gram.:adj.}
\end{itemize}
\begin{itemize}
\item {Proveniência:(Do gr. \textunderscore ek\textunderscore  + \textunderscore tillein\textunderscore , arrancar)}
\end{itemize}
O mesmo que \textunderscore depilatório\textunderscore .
\section{Ectilótico}
\begin{itemize}
\item {Grp. gram.:adj.}
\end{itemize}
\begin{itemize}
\item {Proveniência:(Do gr. \textunderscore ek\textunderscore  + \textunderscore tillein\textunderscore , arrancar)}
\end{itemize}
O mesmo que \textunderscore depilatório\textunderscore .
\section{Ectilótico}
\begin{itemize}
\item {Grp. gram.:adj.}
\end{itemize}
\begin{itemize}
\item {Proveniência:(Do gr. \textunderscore ek\textunderscore  + \textunderscore tulos\textunderscore , callo)}
\end{itemize}
Que é próprio para desgastar os calos.
\section{Éctima}
\begin{itemize}
\item {Grp. gram.:m.}
\end{itemize}
\begin{itemize}
\item {Proveniência:(Gr. \textunderscore ektuma\textunderscore )}
\end{itemize}
Flegmasia, que ataca os folículos sebáceos.
\section{Éctipo}
\begin{itemize}
\item {Grp. gram.:m.}
\end{itemize}
\begin{itemize}
\item {Proveniência:(Gr. \textunderscore ektupos\textunderscore )}
\end{itemize}
Cópia de medalha; cunho.
\section{Ectipografia}
\begin{itemize}
\item {Grp. gram.:f.}
\end{itemize}
\begin{itemize}
\item {Proveniência:(Do gr. \textunderscore ek\textunderscore  + \textunderscore tupos\textunderscore  + \textunderscore graphein\textunderscore )}
\end{itemize}
Impressão tipográfica, que deixa os caracteres em relêvo, para leitura de cegos.
\section{Ectlipse}
\begin{itemize}
\item {Grp. gram.:f.}
\end{itemize}
\begin{itemize}
\item {Utilização:Gram.}
\end{itemize}
\begin{itemize}
\item {Proveniência:(Gr. \textunderscore ekthilípsis\textunderscore )}
\end{itemize}
Elisão do \textunderscore m\textunderscore  ou do \textunderscore s\textunderscore  no fim de uma palavra.
\section{Ectoblasta}
\begin{itemize}
\item {Grp. gram.:m.}
\end{itemize}
\begin{itemize}
\item {Proveniência:(Do gr. \textunderscore ekto\textunderscore  + \textunderscore blastos\textunderscore )}
\end{itemize}
Germinação do embryão?:«\textunderscore mera dependência do ectoblasta, o cérebro...\textunderscore »Sousa Martins, \textunderscore Nosographia\textunderscore .
\section{Ectoderme}
\begin{itemize}
\item {Grp. gram.:m.}
\end{itemize}
\begin{itemize}
\item {Proveniência:(Do gr. \textunderscore ektos\textunderscore  + \textunderscore derma\textunderscore )}
\end{itemize}
O mesmo que \textunderscore epiblasto\textunderscore .
\section{Ectópago}
\begin{itemize}
\item {Grp. gram.:m.  e  adj.}
\end{itemize}
\begin{itemize}
\item {Utilização:Terat.}
\end{itemize}
\begin{itemize}
\item {Proveniência:(Do gr. \textunderscore ektos\textunderscore  + \textunderscore pageis\textunderscore )}
\end{itemize}
Diz-se do monstro, composto de dois indivíduos, reunidos lateralmente em toda a extensão do thórax e com um umbigo commum.
\section{Ectoparasito}
\begin{itemize}
\item {Grp. gram.:m.}
\end{itemize}
\begin{itemize}
\item {Proveniência:(Do gr. \textunderscore ekto\textunderscore  + \textunderscore parasitos\textunderscore )}
\end{itemize}
Animal parasito, que vive na parte externa de um corpo ou organismo.
\section{Ectopia}
\begin{itemize}
\item {Grp. gram.:f.}
\end{itemize}
\begin{itemize}
\item {Proveniência:(Do gr. \textunderscore ek\textunderscore  + \textunderscore topos\textunderscore )}
\end{itemize}
Deslocação de um órgão.
\section{Ectopógono}
\begin{itemize}
\item {Grp. gram.:adj.}
\end{itemize}
\begin{itemize}
\item {Utilização:Bot.}
\end{itemize}
\begin{itemize}
\item {Proveniência:(Do gr. \textunderscore ektos\textunderscore , fóra, e \textunderscore pogon\textunderscore , barba)}
\end{itemize}
Diz-se dos musgos, cujas urnas são exteriormente guarnecidas de pelos.
\section{Ectospermas}
\begin{itemize}
\item {Grp. gram.:f. pl.}
\end{itemize}
Algas de água doce, sem filamentos entrelaçados.
\section{Ectozoários}
\begin{itemize}
\item {Grp. gram.:m. pl.}
\end{itemize}
\begin{itemize}
\item {Proveniência:(Do gr. \textunderscore ektos\textunderscore  + \textunderscore zoon\textunderscore )}
\end{itemize}
Insectos parasitos, que vivem á superficie do corpo do homem e de outros animaes.
\section{Ectrómelo}
\begin{itemize}
\item {Grp. gram.:m.}
\end{itemize}
\begin{itemize}
\item {Utilização:Terat.}
\end{itemize}
\begin{itemize}
\item {Proveniência:(Do gr. \textunderscore ek\textunderscore  + \textunderscore troein\textunderscore )}
\end{itemize}
Monstro privado de membros, quer thorácicos, quer abdominaes.
\section{Ectrópio}
\begin{itemize}
\item {Grp. gram.:m.}
\end{itemize}
\begin{itemize}
\item {Proveniência:(Do gr. \textunderscore ek\textunderscore  + \textunderscore trepein\textunderscore )}
\end{itemize}
Reviramento do bôrdo livre da pálpebra para fóra do ôlho.
\section{Ectrópion}
\begin{itemize}
\item {Grp. gram.:m.}
\end{itemize}
O mesmo que \textunderscore ectrópio\textunderscore .
\section{Ectrótico}
\begin{itemize}
\item {Grp. gram.:adj.}
\end{itemize}
\begin{itemize}
\item {Grp. gram.:M.}
\end{itemize}
\begin{itemize}
\item {Proveniência:(Gr. \textunderscore ektrotikos\textunderscore )}
\end{itemize}
O mesmo que \textunderscore abortivo\textunderscore .
Medicamento ectrótico.
\section{Ectylótico}
\begin{itemize}
\item {Grp. gram.:adj.}
\end{itemize}
\begin{itemize}
\item {Proveniência:(Do gr. \textunderscore ek\textunderscore  + \textunderscore tulos\textunderscore , callo)}
\end{itemize}
Que é próprio para desgastar os callos.
\section{Éctypo}
\begin{itemize}
\item {Grp. gram.:m.}
\end{itemize}
\begin{itemize}
\item {Proveniência:(Gr. \textunderscore ektupos\textunderscore )}
\end{itemize}
Cópia de medalha; cunho.
\section{Ectypographia}
\begin{itemize}
\item {Grp. gram.:f.}
\end{itemize}
\begin{itemize}
\item {Proveniência:(Do gr. \textunderscore ek\textunderscore  + \textunderscore tupos\textunderscore  + \textunderscore graphein\textunderscore )}
\end{itemize}
Impressão typográphica, que deixa os caracteres em relêvo, para leitura de cegos.
\section{Écula}
\begin{itemize}
\item {Grp. gram.:f.}
\end{itemize}
Peixe do Mar-Vermelho.
\section{Ecúleo}
\begin{itemize}
\item {Grp. gram.:m.}
\end{itemize}
\begin{itemize}
\item {Proveniência:(Lat. \textunderscore eculeus\textunderscore )}
\end{itemize}
Pôtro, instrumento de tortura.
\section{Éculo}
\begin{itemize}
\item {Grp. gram.:m.}
\end{itemize}
Espécie de mocho.
\section{Ecumenicamente}
\begin{itemize}
\item {Grp. gram.:adv.}
\end{itemize}
De modo ecumênico.
\section{Ecumenicidade}
\begin{itemize}
\item {Grp. gram.:f.}
\end{itemize}
Qualidade daquillo que é ecumênico.
\section{Ecumênico}
\begin{itemize}
\item {Grp. gram.:adj.}
\end{itemize}
\begin{itemize}
\item {Proveniência:(Gr. \textunderscore oikoumenikos\textunderscore )}
\end{itemize}
Relativo ao universo, a toda a terra habitada.
Universal.
Diz-se do concilio, para que são convocados todos os Prelados do mundo cathólico.
\section{Eczema}
\begin{itemize}
\item {Grp. gram.:m.}
\end{itemize}
\begin{itemize}
\item {Proveniência:(Gr. \textunderscore ekzema\textunderscore )}
\end{itemize}
Affecção da pelle, caracterizada pela reunião de vesículas, que produzem prurido.
\section{Eczematoso}
\begin{itemize}
\item {Grp. gram.:adj.}
\end{itemize}
Que tem a natureza de eczema.
Que padece de eczema.
\section{Edacidade}
\begin{itemize}
\item {Grp. gram.:f.}
\end{itemize}
\begin{itemize}
\item {Proveniência:(Lat. \textunderscore edacitas\textunderscore )}
\end{itemize}
O mesmo que \textunderscore voracidade\textunderscore .
\section{Edade}
\textunderscore f.\textunderscore  (e der.)
O mesmo que \textunderscore idade\textunderscore , etc.
\section{Edema}
\begin{itemize}
\item {Grp. gram.:m.}
\end{itemize}
\begin{itemize}
\item {Proveniência:(Gr. \textunderscore oidema\textunderscore )}
\end{itemize}
Inchação, formada por serosidade infiltrada no tecido cellular, sem vermelhidão nem dôr, e que desapparece com uma demorada pressão de dedos.
\section{Edemacia}
\begin{itemize}
\item {Grp. gram.:f.}
\end{itemize}
\begin{itemize}
\item {Utilização:Bras}
\end{itemize}
\begin{itemize}
\item {Utilização:Neol.}
\end{itemize}
Acto ou effeito de edemaciar.
\section{Edemaciar}
\begin{itemize}
\item {Grp. gram.:v. t.}
\end{itemize}
Produzir edema em.
\section{Edemático}
\begin{itemize}
\item {Grp. gram.:adj.}
\end{itemize}
Relativo a edema.
Edematoso.
\section{Edematoso}
\begin{itemize}
\item {Grp. gram.:adj.}
\end{itemize}
\begin{itemize}
\item {Proveniência:(Do gr. \textunderscore oidema\textunderscore , \textunderscore oidematos\textunderscore )}
\end{itemize}
Que tem edema.
Que é da natureza do edema.
\section{Edêmero}
\begin{itemize}
\item {Grp. gram.:m.}
\end{itemize}
\begin{itemize}
\item {Proveniência:(Do gr. \textunderscore oidos\textunderscore  + \textunderscore meros\textunderscore )}
\end{itemize}
Gênero de insectos coleópteros heterâmeros.
\section{Edêmono}
\begin{itemize}
\item {Grp. gram.:m.}
\end{itemize}
\begin{itemize}
\item {Proveniência:(Do gr. \textunderscore aidemon\textunderscore )}
\end{itemize}
Insecto coleóptero da África austral.
\section{Éden}
\begin{itemize}
\item {Grp. gram.:m.}
\end{itemize}
\begin{itemize}
\item {Utilização:Ext.}
\end{itemize}
\begin{itemize}
\item {Proveniência:(Do hebr. \textunderscore eden\textunderscore , prazer)}
\end{itemize}
Paraíso terreal, de que fala a \textunderscore Bíblia\textunderscore .
Lugar de delicias e de felicidade tranquilla.
\section{Edêneo}
\begin{itemize}
\item {Grp. gram.:adj.}
\end{itemize}
O mesmo que \textunderscore edênico\textunderscore .
\section{Edênico}
\begin{itemize}
\item {Grp. gram.:adj.}
\end{itemize}
Relativo a éden.
Paradisíaco.
\section{Edição}
\begin{itemize}
\item {Grp. gram.:f.}
\end{itemize}
\begin{itemize}
\item {Proveniência:(Lat. \textunderscore editis\textunderscore )}
\end{itemize}
Acto de publicar, de dar á luz.
Publicação de obra literária ou scientífica.
Conjunto dos exemplares de uma obra, impressa na mesma occasião: \textunderscore a primeira edição esgotou-se\textunderscore .
\section{Edicionar}
\begin{itemize}
\item {Grp. gram.:v. t.}
\end{itemize}
O mesmo que \textunderscore editorar\textunderscore . Cf. Camillo, \textunderscore Noites de Insóm.\textunderscore , IX, 66.
\section{Edictal}
\begin{itemize}
\item {Grp. gram.:adj.}
\end{itemize}
\begin{itemize}
\item {Proveniência:(Lat. \textunderscore edictalis\textunderscore )}
\end{itemize}
Relativo ao edicto.
\section{Edicto}
\begin{itemize}
\item {Grp. gram.:m.}
\end{itemize}
\begin{itemize}
\item {Proveniência:(Lat. \textunderscore edictum\textunderscore )}
\end{itemize}
Parte da lei, em que alguma coisa se preceitua.
Ordem.
Decreto.
\section{Edícula}
\begin{itemize}
\item {Grp. gram.:f.}
\end{itemize}
\begin{itemize}
\item {Proveniência:(Lat. \textunderscore aedicula\textunderscore )}
\end{itemize}
Pequena casa.
Nicho para imagem de santos.
Oratório.
\section{Edificação}
\begin{itemize}
\item {Grp. gram.:f.}
\end{itemize}
\begin{itemize}
\item {Proveniência:(Lat. \textunderscore aedificatio\textunderscore )}
\end{itemize}
Acto ou effeito de edificar; edifício.
\section{Edificador}
\begin{itemize}
\item {Grp. gram.:adj.}
\end{itemize}
\begin{itemize}
\item {Proveniência:(Lat. \textunderscore aedificator\textunderscore )}
\end{itemize}
Que edifica.
\section{Edificamento}
\begin{itemize}
\item {Grp. gram.:m.}
\end{itemize}
O mesmo que \textunderscore edificação\textunderscore .
\section{Edificante}
\begin{itemize}
\item {Grp. gram.:adj.}
\end{itemize}
\begin{itemize}
\item {Utilização:Deprec.}
\end{itemize}
\begin{itemize}
\item {Grp. gram.:M.}
\end{itemize}
\begin{itemize}
\item {Proveniência:(Lat. \textunderscore aedificans\textunderscore )}
\end{itemize}
Que edifica.
Instructivo.
Escandaloso.
Aquelle que edifica.
\section{Edificantemente}
\begin{itemize}
\item {Grp. gram.:adv.}
\end{itemize}
De modo edificante.
\section{Edificar}
\begin{itemize}
\item {Grp. gram.:v. t.}
\end{itemize}
\begin{itemize}
\item {Proveniência:(Lat. \textunderscore aedificare\textunderscore )}
\end{itemize}
Construir, (falando-se de edifícios).
Instituir.
Dar bons exemplos a.
Doutrinar moralmente, infundir bons sentimentos religiosos em.
Instruir.
\section{Edificativo}
\begin{itemize}
\item {Grp. gram.:adj.}
\end{itemize}
O mesmo que \textunderscore edificante\textunderscore .
\section{Edifício}
\begin{itemize}
\item {Grp. gram.:m.}
\end{itemize}
\begin{itemize}
\item {Utilização:Fig.}
\end{itemize}
\begin{itemize}
\item {Proveniência:(Lat. \textunderscore aedificium\textunderscore )}
\end{itemize}
Qualquer construcção, que póde sêr habitada, servir para alojamento ou abrigo, ou para estabelecimento de fábrica, exercício de funcções públicas, etc.
Casa.
Trabalho artístico ou literário.
Conjunto de planos ou de ideias.
\section{Edil}
\begin{itemize}
\item {Grp. gram.:m.}
\end{itemize}
\begin{itemize}
\item {Grp. gram.:Pl.}
\end{itemize}
\begin{itemize}
\item {Proveniência:(Lat. \textunderscore aedilis\textunderscore )}
\end{itemize}
Antigo magistrado administrativo, em Roma.
Em acepção moderna, o mesmo que \textunderscore vereador\textunderscore .
\textunderscore Edis\textunderscore ; raramente \textunderscore ediles\textunderscore :«\textunderscore ...nunca os ediles romanos...\textunderscore »Filinto, \textunderscore D. Man.\textunderscore , III, 45.
\section{Edílico}
\begin{itemize}
\item {Grp. gram.:adj.}
\end{itemize}
Relativo a edil. Cf. Herculano, \textunderscore Hist. de Port.\textunderscore  IV, 238.
\section{Edilidade}
\begin{itemize}
\item {Grp. gram.:f.}
\end{itemize}
\begin{itemize}
\item {Utilização:Ext.}
\end{itemize}
\begin{itemize}
\item {Proveniência:(Lat. \textunderscore aedilitas\textunderscore )}
\end{itemize}
Cargo de edil.
Vereação.
\section{Ediologia}
\begin{itemize}
\item {Grp. gram.:f.}
\end{itemize}
\begin{itemize}
\item {Proveniência:(Do gr. \textunderscore aidoia\textunderscore  + \textunderscore logos\textunderscore )}
\end{itemize}
Tratado dos órgãos da geração.
\section{Édipo}
\begin{itemize}
\item {Grp. gram.:m.}
\end{itemize}
\begin{itemize}
\item {Utilização:Fig.}
\end{itemize}
\begin{itemize}
\item {Proveniência:(De \textunderscore Édipo\textunderscore ^2 n. p.)}
\end{itemize}
Aquelle que explica um enígma, que esclarece um ponto escuro.
\section{Editação}
\begin{itemize}
\item {Grp. gram.:f.}
\end{itemize}
Acto de editar. Cf. Camillo, \textunderscore Noites de Insómn.\textunderscore , VIII, 88.
\section{Edital}
\begin{itemize}
\item {Grp. gram.:adj.}
\end{itemize}
\begin{itemize}
\item {Grp. gram.:M.}
\end{itemize}
\begin{itemize}
\item {Proveniência:(De \textunderscore édito\textunderscore )}
\end{itemize}
Relativo a éditos.
Que se faz público por affixação de editaes.
Ordem official ou traslado de édito ou postura, destinado ao conhecimento de todos, afixando-se em lugares públicos ou annunciando-se na imprensa periódica.
\section{Editar}
\begin{itemize}
\item {Grp. gram.:v. t.}
\end{itemize}
\begin{itemize}
\item {Proveniência:(Do lat. \textunderscore editus\textunderscore )}
\end{itemize}
Publicar.
Dar á luz (uma obra literária ou scientífica).
Editorar.
\section{Édito}
\begin{itemize}
\item {Grp. gram.:m.}
\end{itemize}
\begin{itemize}
\item {Proveniência:(Lat. \textunderscore editus\textunderscore )}
\end{itemize}
Ordem judicial, que se faz pública por annúncios ou editaes.
\section{Editor}
\begin{itemize}
\item {Grp. gram.:m.}
\end{itemize}
\begin{itemize}
\item {Grp. gram.:Adj.}
\end{itemize}
\begin{itemize}
\item {Proveniência:(Lat. \textunderscore editor\textunderscore )}
\end{itemize}
Aquelle que edita.
Aquelle, por conta de quem corre a composição typográphica, impressão e diffusão de qualquer publicação literária, scientífica, política, etc.
Que edita.
\section{Editoração}
\begin{itemize}
\item {Grp. gram.:f.}
\end{itemize}
Acto de editorar. Cf. Camillo, \textunderscore Noites de Insómn.\textunderscore , VII, 52.
\section{Editoral}
\begin{itemize}
\item {Grp. gram.:adj.}
\end{itemize}
O mesmo que \textunderscore editorial\textunderscore .
\section{Editorar}
\begin{itemize}
\item {Grp. gram.:v. t.}
\end{itemize}
\begin{itemize}
\item {Proveniência:(De \textunderscore editor\textunderscore )}
\end{itemize}
O mesmo que \textunderscore editar\textunderscore . Cf. Camillo, \textunderscore Quéda\textunderscore , na \textunderscore Advert.\textunderscore  da 2.^a ed.; e \textunderscore Noites de Insómn.\textunderscore  VII, 52 e 92.
\section{Editorial}
\begin{itemize}
\item {Grp. gram.:adj.}
\end{itemize}
Relativo a editor.
Diz-se do chamado artigo do fundo ou do artigo principal e inicial de um periódico.
\section{...êdo}
\begin{itemize}
\item {Grp. gram.:suf.}
\end{itemize}
(indicativo de collectividade: \textunderscore moitedo\textunderscore ; \textunderscore vinhedo\textunderscore )
\section{Edoma}
\begin{itemize}
\item {Grp. gram.:f.}
\end{itemize}
\begin{itemize}
\item {Utilização:Ant.}
\end{itemize}
O mesmo que \textunderscore semana\textunderscore . Cf. Herculano, \textunderscore Lendas\textunderscore , 281.
(Contr. de \textunderscore hebdômada\textunderscore )
\section{Edra}
\begin{itemize}
\item {Grp. gram.:f.}
\end{itemize}
\begin{itemize}
\item {Utilização:Prov.}
\end{itemize}
\begin{itemize}
\item {Utilização:trasm.}
\end{itemize}
\begin{itemize}
\item {Proveniência:(Lat. \textunderscore hedera\textunderscore . Cp. cast. \textunderscore yedra\textunderscore )}
\end{itemize}
O mesmo que \textunderscore hera\textunderscore .
\section{Edredão}
\begin{itemize}
\item {Grp. gram.:m.}
\end{itemize}
\begin{itemize}
\item {Proveniência:(Do suec. \textunderscore eider\textunderscore , espécie de pato, que se encontra especialmente na Islândia, e \textunderscore dun\textunderscore , pennugem)}
\end{itemize}
Cobertura acolchoada, que contém pennugem fina de uma ave palmípede do norte; proximamente o mesmo que \textunderscore goderim\textunderscore , \textunderscore cócedra\textunderscore , \textunderscore almadraque\textunderscore  ou \textunderscore almatricha\textunderscore .
\section{Edu}
\begin{itemize}
\item {Grp. gram.:m.}
\end{itemize}
Árvore da Índia portuguesa, o mesmo que \textunderscore aldavana\textunderscore .
\section{Educabilidade}
\begin{itemize}
\item {Grp. gram.:f.}
\end{itemize}
Qualidade daquelle que é educável.
\section{Educação}
\begin{itemize}
\item {Grp. gram.:f.}
\end{itemize}
\begin{itemize}
\item {Proveniência:(Lat. \textunderscore educatio\textunderscore )}
\end{itemize}
Acto ou effeito de educar.
Polidez, cortesia.
\section{Educacionista}
\begin{itemize}
\item {Grp. gram.:m.}
\end{itemize}
\begin{itemize}
\item {Utilização:Neol.}
\end{itemize}
\begin{itemize}
\item {Proveniência:(De \textunderscore educação\textunderscore )}
\end{itemize}
O mesmo que \textunderscore pedagogo\textunderscore .
\section{Educador}
\begin{itemize}
\item {Grp. gram.:adj.}
\end{itemize}
\begin{itemize}
\item {Grp. gram.:M.}
\end{itemize}
\begin{itemize}
\item {Proveniência:(De \textunderscore educar\textunderscore )}
\end{itemize}
Que educa.
Aquelle que educa.
Pedagogo.
\section{Educanda}
(\textunderscore fem.\textunderscore  de \textunderscore educando\textunderscore )
\section{Educandário}
\begin{itemize}
\item {Grp. gram.:m.}
\end{itemize}
\begin{itemize}
\item {Utilização:bras}
\end{itemize}
\begin{itemize}
\item {Utilização:Neol.}
\end{itemize}
\begin{itemize}
\item {Proveniência:(De \textunderscore educando\textunderscore )}
\end{itemize}
Estabelecimento de educação.
\section{Educando}
\begin{itemize}
\item {Grp. gram.:m.}
\end{itemize}
\begin{itemize}
\item {Proveniência:(Lat. \textunderscore educandus\textunderscore )}
\end{itemize}
Aquelle que está recebendo educação.
Collegial; alumno.
\section{Edução}
\begin{itemize}
\item {Grp. gram.:f.}
\end{itemize}
\begin{itemize}
\item {Proveniência:(Lat. \textunderscore eductio\textunderscore )}
\end{itemize}
Acto ou efeito de eduzir.
\section{Educar}
\begin{itemize}
\item {Grp. gram.:v. t.}
\end{itemize}
\begin{itemize}
\item {Proveniência:(Lat. \textunderscore educare\textunderscore )}
\end{itemize}
Desenvolver as faculdades phýsicas, intellectuaes e moraes de.
Instruir.
Domesticar, adestrar.
Aclimatar.
\section{Educativo}
\begin{itemize}
\item {Grp. gram.:adj.}
\end{itemize}
\begin{itemize}
\item {Proveniência:(De \textunderscore educar\textunderscore )}
\end{itemize}
Relativo a educação; que produz educação.
\section{Educável}
\begin{itemize}
\item {Grp. gram.:adj.}
\end{itemize}
Que se póde educar.
\section{Educção}
\begin{itemize}
\item {Grp. gram.:f.}
\end{itemize}
\begin{itemize}
\item {Proveniência:(Lat. \textunderscore eductio\textunderscore )}
\end{itemize}
Acto ou effeito de eduzir.
\section{Educto}
\begin{itemize}
\item {Grp. gram.:adj.}
\end{itemize}
\begin{itemize}
\item {Proveniência:(Lat. \textunderscore eductus\textunderscore )}
\end{itemize}
Extrahido.
Deduzido.
\section{Edulçar}
\begin{itemize}
\item {Grp. gram.:v. i.}
\end{itemize}
Mostrar-se doce.
Amenizar-se, suavizar-se.
(Cp. \textunderscore dulçor\textunderscore )
\section{Edulcoração}
\begin{itemize}
\item {Grp. gram.:f.}
\end{itemize}
\begin{itemize}
\item {Proveniência:(Lat. \textunderscore edulcoratio\textunderscore )}
\end{itemize}
Acto ou effeito de edulcorar.
\section{Edulcorar}
\begin{itemize}
\item {Grp. gram.:v. t.}
\end{itemize}
\begin{itemize}
\item {Proveniência:(Do b. lat. \textunderscore e\textunderscore  + \textunderscore dulcorare\textunderscore )}
\end{itemize}
Tornar doce.
\section{Édulo}
\begin{itemize}
\item {Grp. gram.:adj.}
\end{itemize}
\begin{itemize}
\item {Proveniência:(Lat. \textunderscore edulis\textunderscore )}
\end{itemize}
O mesmo que \textunderscore comestível\textunderscore .
\section{Eduzir}
\begin{itemize}
\item {Grp. gram.:v. t.}
\end{itemize}
\begin{itemize}
\item {Proveniência:(Lat. \textunderscore educere\textunderscore )}
\end{itemize}
Deduzir; extrahir.
\section{Efarado}
\begin{itemize}
\item {Grp. gram.:adj.}
\end{itemize}
\begin{itemize}
\item {Utilização:Des.}
\end{itemize}
Embravecido. Cf. \textunderscore Monarch. Lusit.\textunderscore , IV, 22.
(Cp. fr. \textunderscore effaré\textunderscore )
\section{Efectivação}
\begin{itemize}
\item {Grp. gram.:f.}
\end{itemize}
Acto de efectivar.
\section{Efectivamente}
\begin{itemize}
\item {Grp. gram.:adv.}
\end{itemize}
Com efeito; realmente.
De modo efectivo.
\section{Efectivar}
\begin{itemize}
\item {Grp. gram.:v. t.}
\end{itemize}
Tornar efectivo.
\section{Efectível}
\begin{itemize}
\item {Grp. gram.:adj.}
\end{itemize}
Que se póde efectuar.
\section{Efectividade}
\begin{itemize}
\item {Grp. gram.:f.}
\end{itemize}
Qualidade ou estado daquilo que é efectivo.
\section{Efectivo}
\begin{itemize}
\item {Grp. gram.:adj.}
\end{itemize}
\begin{itemize}
\item {Grp. gram.:M.}
\end{itemize}
\begin{itemize}
\item {Proveniência:(Lat. \textunderscore effectivus\textunderscore )}
\end{itemize}
Que tem efeito.
Real.
Permanente; que não tem interrupção: \textunderscore serviço efectivo\textunderscore .
Aquilo que existe realmente.
\section{Efectuar}
\textunderscore v. t.\textunderscore  (e der)
O mesmo que \textunderscore efeituar\textunderscore .
\section{Efectuoso}
\begin{itemize}
\item {Grp. gram.:adj.}
\end{itemize}
\begin{itemize}
\item {Utilização:P. us.}
\end{itemize}
\begin{itemize}
\item {Proveniência:(Do lat. \textunderscore effectus\textunderscore )}
\end{itemize}
Que produz efeito.
Eficaz. Cf. \textunderscore Diccion. Homophonol.\textunderscore 
\section{Efeitarrão}
\begin{itemize}
\item {Grp. gram.:m.}
\end{itemize}
Efeito grande, extraordinário.
\section{Efeito}
\begin{itemize}
\item {Grp. gram.:m.}
\end{itemize}
\begin{itemize}
\item {Proveniência:(Do lat. \textunderscore effectus\textunderscore )}
\end{itemize}
Resultado.
Acto, que procede de um agente qualquer.
Realização.
Fim, destino.
Eficácia.
Execução.
Combinação.
Valor negociável.
\section{Efeituação}
\begin{itemize}
\item {Grp. gram.:f.}
\end{itemize}
Acto de efeituar.
\section{Efeituador}
\begin{itemize}
\item {Grp. gram.:adj.}
\end{itemize}
\begin{itemize}
\item {Grp. gram.:M.}
\end{itemize}
Que efeitua.
Aquele que efeitua.
\section{Efeituar}
\begin{itemize}
\item {Grp. gram.:v. t.}
\end{itemize}
\begin{itemize}
\item {Proveniência:(De \textunderscore efeito\textunderscore )}
\end{itemize}
Levar a efeito.
Realizar: \textunderscore efeituar um plano\textunderscore .
Consumar.
Perfazer: \textunderscore efeituar certa quantia\textunderscore .
\section{Efeminação}
\begin{itemize}
\item {Grp. gram.:f.}
\end{itemize}
\begin{itemize}
\item {Proveniência:(Lat. \textunderscore effeminatio\textunderscore )}
\end{itemize}
Acto ou efeito de efeminar.
Qualidade de quem é efeminado.
\section{Efeminadamente}
\begin{itemize}
\item {Grp. gram.:adv.}
\end{itemize}
Á maneira de efeminado.
\section{Efeminado}
\begin{itemize}
\item {Grp. gram.:adj.}
\end{itemize}
\begin{itemize}
\item {Proveniência:(De \textunderscore efeminar\textunderscore )}
\end{itemize}
Femeeiro; mulherengo.
Pusilânime.
\section{Efeminar}
\begin{itemize}
\item {Grp. gram.:v. t.}
\end{itemize}
\begin{itemize}
\item {Proveniência:(Lat. \textunderscore effeminare\textunderscore )}
\end{itemize}
Tornar semelhante a uma mulher, na índole e na fraqueza.
Tornar fraco e delicado.
Enervar.
Tornar afeiçoado a mulheres.
\section{Efeminizar}
\begin{itemize}
\item {Grp. gram.:v. t.}
\end{itemize}
O mesmo que \textunderscore efeminar\textunderscore .
\section{Eferente}
\begin{itemize}
\item {Grp. gram.:adj.}
\end{itemize}
\begin{itemize}
\item {Proveniência:(Lat. \textunderscore efferens\textunderscore )}
\end{itemize}
Que conduz.
Que transporta.
\section{Efervescência}
\begin{itemize}
\item {Grp. gram.:f.}
\end{itemize}
\begin{itemize}
\item {Utilização:Fig.}
\end{itemize}
\begin{itemize}
\item {Proveniência:(Lat. \textunderscore effervescentia\textunderscore )}
\end{itemize}
Desenvolvimento de bolhas dentro de um líquido pela deminuição de pressão ou pela decomposição de um carbonato.
Fervura.
Abalo de espírito.
Movimento.
Excitação.
\section{Efervescente}
\begin{itemize}
\item {Grp. gram.:adj.}
\end{itemize}
\begin{itemize}
\item {Proveniência:(Lat. \textunderscore effervescens\textunderscore )}
\end{itemize}
Que tem efervescência.
\section{Efervescer}
\begin{itemize}
\item {Grp. gram.:v. i.}
\end{itemize}
\begin{itemize}
\item {Proveniência:(Lat. \textunderscore effervescere\textunderscore )}
\end{itemize}
Tornar-se efervescente.
\section{Effectivação}
\begin{itemize}
\item {Grp. gram.:f.}
\end{itemize}
Acto de effectivar.
\section{Effectivamente}
\begin{itemize}
\item {Grp. gram.:adv.}
\end{itemize}
Com effeito; realmente.
De modo effectivo.
\section{Effectivar}
\begin{itemize}
\item {Grp. gram.:v. t.}
\end{itemize}
Tornar effectivo.
\section{Effectível}
\begin{itemize}
\item {Grp. gram.:adj.}
\end{itemize}
Que se póde effectuar.
\section{Effectividade}
\begin{itemize}
\item {Grp. gram.:f.}
\end{itemize}
Qualidade ou estado daquillo que é effectivo.
\section{Effectivo}
\begin{itemize}
\item {Grp. gram.:adj.}
\end{itemize}
\begin{itemize}
\item {Grp. gram.:M.}
\end{itemize}
\begin{itemize}
\item {Proveniência:(Lat. \textunderscore effectivus\textunderscore )}
\end{itemize}
Que tem effeito.
Real.
Permanente; que não tem interrupção: \textunderscore serviço effectivo\textunderscore .
Aquillo que existe realmente.
\section{Effectuar}
\textunderscore v. t.\textunderscore  (e der)
O mesmo que \textunderscore effeituar\textunderscore .
\section{Effectuoso}
\begin{itemize}
\item {Grp. gram.:adj.}
\end{itemize}
\begin{itemize}
\item {Utilização:P. us.}
\end{itemize}
\begin{itemize}
\item {Proveniência:(Do lat. \textunderscore effectus\textunderscore )}
\end{itemize}
Que produz effeito.
Efficaz. Cf. \textunderscore Diccion. Homophonol.\textunderscore 
\section{Effeitarrão}
\begin{itemize}
\item {Grp. gram.:m.}
\end{itemize}
Effeito grande, extraordinário.
\section{Effeito}
\begin{itemize}
\item {Grp. gram.:m.}
\end{itemize}
\begin{itemize}
\item {Proveniência:(Do lat. \textunderscore effectus\textunderscore )}
\end{itemize}
Resultado.
Acto, que procede de um agente qualquer.
Realização.
Fim, destino.
Efficácia.
Execução.
Combinação.
Valor negociável.
\section{Effeituação}
\begin{itemize}
\item {Grp. gram.:f.}
\end{itemize}
Acto de effeituar.
\section{Effeituador}
\begin{itemize}
\item {Grp. gram.:adj.}
\end{itemize}
\begin{itemize}
\item {Grp. gram.:M.}
\end{itemize}
Que effeitua.
Aquelle que effeitua.
\section{Effeituar}
\begin{itemize}
\item {Grp. gram.:v. t.}
\end{itemize}
\begin{itemize}
\item {Proveniência:(De \textunderscore effeito\textunderscore )}
\end{itemize}
Levar a effeito.
Realizar: \textunderscore effeituar um plano\textunderscore .
Consummar.
Perfazer: \textunderscore effeituar certa quantia\textunderscore .
\section{Effeminação}
\begin{itemize}
\item {Grp. gram.:f.}
\end{itemize}
\begin{itemize}
\item {Proveniência:(Lat. \textunderscore effeminatio\textunderscore )}
\end{itemize}
Acto ou effeito de effeminar.
Qualidade de quem é effeminado.
\section{Effeminadamente}
\begin{itemize}
\item {Grp. gram.:adv.}
\end{itemize}
Á maneira de effeminado.
\section{Effeminado}
\begin{itemize}
\item {Grp. gram.:adj.}
\end{itemize}
\begin{itemize}
\item {Proveniência:(De \textunderscore effeminar\textunderscore )}
\end{itemize}
Femeeiro; mulherengo.
Pusillânime.
\section{Effeminar}
\begin{itemize}
\item {Grp. gram.:v. t.}
\end{itemize}
\begin{itemize}
\item {Proveniência:(Lat. \textunderscore effeminare\textunderscore )}
\end{itemize}
Tornar semelhante a uma mulher, na índole e na fraqueza.
Tornar fraco e delicado.
Enervar.
Tornar affeiçoado a mulheres.
\section{Effeminizar}
\begin{itemize}
\item {Grp. gram.:v. t.}
\end{itemize}
O mesmo que \textunderscore effeminar\textunderscore .
\section{Efferente}
\begin{itemize}
\item {Grp. gram.:adj.}
\end{itemize}
\begin{itemize}
\item {Proveniência:(Lat. \textunderscore efferens\textunderscore )}
\end{itemize}
Que conduz.
Que transporta.
\section{Effervescência}
\begin{itemize}
\item {Grp. gram.:f.}
\end{itemize}
\begin{itemize}
\item {Utilização:Fig.}
\end{itemize}
\begin{itemize}
\item {Proveniência:(Lat. \textunderscore effervescentia\textunderscore )}
\end{itemize}
Desenvolvimento de bolhas dentro de um líquido pela deminuição de pressão ou pela decomposição de um carbonato.
Fervura.
Abalo de espírito.
Movimento.
Excitação.
\section{Effervescente}
\begin{itemize}
\item {Grp. gram.:adj.}
\end{itemize}
\begin{itemize}
\item {Proveniência:(Lat. \textunderscore effervescens\textunderscore )}
\end{itemize}
Que tem effervescência.
\section{Effervescer}
\begin{itemize}
\item {Grp. gram.:v. i.}
\end{itemize}
\begin{itemize}
\item {Proveniência:(Lat. \textunderscore effervescere\textunderscore )}
\end{itemize}
Tornar-se effervescente.
\section{Efficácia}
\begin{itemize}
\item {Grp. gram.:f.}
\end{itemize}
\begin{itemize}
\item {Proveniência:(Lat. \textunderscore efficatia\textunderscore )}
\end{itemize}
Qualidade daquillo que é efficaz.
\section{Efficacidade}
\begin{itemize}
\item {Grp. gram.:f.}
\end{itemize}
Qualidade de efficaz.
\section{Efficacíssimo}
\begin{itemize}
\item {Grp. gram.:adj.}
\end{itemize}
Muito efficaz.
\section{Efficaz}
\begin{itemize}
\item {Grp. gram.:adj.}
\end{itemize}
\begin{itemize}
\item {Proveniência:(Lat. \textunderscore efficax\textunderscore )}
\end{itemize}
Que tem a fôrça de produzir alguma coisa.
Que produz muito.
Que tem effeito: \textunderscore recommendação efficaz\textunderscore .
\section{Efficazmente}
\begin{itemize}
\item {Grp. gram.:adv.}
\end{itemize}
De modo efficaz.
Com efficácia.
\section{Efficiência}
\begin{itemize}
\item {Grp. gram.:f.}
\end{itemize}
\begin{itemize}
\item {Proveniência:(Lat. \textunderscore efficientia\textunderscore )}
\end{itemize}
Qualidade daquillo que é efficiente.
\section{Efficiente}
\begin{itemize}
\item {Grp. gram.:adj.}
\end{itemize}
\begin{itemize}
\item {Proveniência:(Lat. \textunderscore efficiens\textunderscore )}
\end{itemize}
O mesmo que \textunderscore efficaz\textunderscore .
\section{Efficientemente}
\begin{itemize}
\item {Grp. gram.:adv.}
\end{itemize}
De modo efficiente.
\section{Effigiar}
\begin{itemize}
\item {Grp. gram.:v. t.}
\end{itemize}
\begin{itemize}
\item {Utilização:Des.}
\end{itemize}
\begin{itemize}
\item {Proveniência:(De \textunderscore effígie\textunderscore )}
\end{itemize}
Apresentar a effígie de.
Executar em effígie (condemnados).
\section{Effígie}
\begin{itemize}
\item {Grp. gram.:f.}
\end{itemize}
\begin{itemize}
\item {Proveniência:(Lat. \textunderscore effigies\textunderscore )}
\end{itemize}
Imagem.
Simulácro.
Figura.
Representação.
\section{Efflorescência}
\begin{itemize}
\item {Grp. gram.:f.}
\end{itemize}
\begin{itemize}
\item {Proveniência:(De \textunderscore efflorescente\textunderscore )}
\end{itemize}
Qualidade daquillo que é efflorescente.
Matéria purverulenta, que se fórma numa substância sólida, pela acção do ar livre.
Qualquer exanthema, que se eleva pouco acima do nível da pelle.
\section{Efflorescente}
\begin{itemize}
\item {Grp. gram.:adj.}
\end{itemize}
\begin{itemize}
\item {Proveniência:(Lat. \textunderscore efflorescens\textunderscore )}
\end{itemize}
Que effloresce.
Que tem camada pulverulenta, como as paredes salitrosas.
Coberto de óxydo metállico.
\section{Efflorescer}
\begin{itemize}
\item {Grp. gram.:v. i.}
\end{itemize}
\begin{itemize}
\item {Proveniência:(Do lat. \textunderscore efflorescere\textunderscore )}
\end{itemize}
Começar a florescer.
Apresentar efflorescência.
\section{Effluência}
\begin{itemize}
\item {Grp. gram.:f.}
\end{itemize}
Acto ou effeito de effluir.
\section{Effluente}
\begin{itemize}
\item {Grp. gram.:adj.}
\end{itemize}
\begin{itemize}
\item {Proveniência:(Lat. \textunderscore effluens\textunderscore )}
\end{itemize}
Que efflue.
\section{Effluir}
\begin{itemize}
\item {Grp. gram.:v. i.}
\end{itemize}
\begin{itemize}
\item {Proveniência:(Lat. \textunderscore effluere\textunderscore )}
\end{itemize}
Emanar; proceder.
Irradiar de um ponto.
\section{Efflúvio}
\begin{itemize}
\item {Grp. gram.:m.}
\end{itemize}
\begin{itemize}
\item {Proveniência:(Lat. \textunderscore effluvium\textunderscore )}
\end{itemize}
Fluido subtil, que efflue dos corpos organizados.
Effluência; exhalação.
Emissão de aroma.
\section{Effluvioso}
\begin{itemize}
\item {Grp. gram.:adj.}
\end{itemize}
Que lança efflúvios.
\section{Effluxão}
\begin{itemize}
\item {Grp. gram.:f.}
\end{itemize}
\begin{itemize}
\item {Proveniência:(Lat. \textunderscore effluxio\textunderscore )}
\end{itemize}
Saída do feto, nos primeiros dias da gravidez.
\section{Effracção}
\begin{itemize}
\item {Grp. gram.:f.}
\end{itemize}
O mesmo que \textunderscore effractura\textunderscore .
\section{Effractura}
\begin{itemize}
\item {Grp. gram.:f.}
\end{itemize}
\begin{itemize}
\item {Utilização:Med.}
\end{itemize}
\begin{itemize}
\item {Proveniência:(Lat. \textunderscore effractura\textunderscore )}
\end{itemize}
Arrombamento, ruptura.
\section{Effúgio}
\begin{itemize}
\item {Grp. gram.:m.}
\end{itemize}
\begin{itemize}
\item {Proveniência:(Lat. \textunderscore effugium\textunderscore )}
\end{itemize}
Subterfúgio.
Refúgio; fugida.
\section{Effundir}
\begin{itemize}
\item {Grp. gram.:v. t.}
\end{itemize}
\begin{itemize}
\item {Proveniência:(Lat. \textunderscore effundere\textunderscore )}
\end{itemize}
Tirar para fóra.
Derramar.
Entornar.
\section{Effusamente}
\begin{itemize}
\item {Grp. gram.:adv.}
\end{itemize}
Com effusão. Cf. Camillo, \textunderscore Estrêl. Fun.\textunderscore , 102.
\section{Effusão}
\begin{itemize}
\item {Grp. gram.:f.}
\end{itemize}
\begin{itemize}
\item {Utilização:Fig.}
\end{itemize}
\begin{itemize}
\item {Proveniência:(Lat. \textunderscore effusio\textunderscore )}
\end{itemize}
Acto de effundir.
Expansão.
Fervor de sentimento affectuoso.
\section{Effuso}
\begin{itemize}
\item {Grp. gram.:adj.}
\end{itemize}
\begin{itemize}
\item {Proveniência:(Lat. \textunderscore effusus\textunderscore )}
\end{itemize}
Em que ha effusão.
\section{Eficácia}
\begin{itemize}
\item {Grp. gram.:f.}
\end{itemize}
\begin{itemize}
\item {Proveniência:(Lat. \textunderscore efficatia\textunderscore )}
\end{itemize}
Qualidade daquilo que é eficaz.
\section{Eficacidade}
\begin{itemize}
\item {Grp. gram.:f.}
\end{itemize}
Qualidade de eficaz.
\section{Eficacíssimo}
\begin{itemize}
\item {Grp. gram.:adj.}
\end{itemize}
Muito eficaz.
\section{Eficaz}
\begin{itemize}
\item {Grp. gram.:adj.}
\end{itemize}
\begin{itemize}
\item {Proveniência:(Lat. \textunderscore efficax\textunderscore )}
\end{itemize}
Que tem a fôrça de produzir alguma coisa.
Que produz muito.
Que tem efeito: \textunderscore recomendação eficaz\textunderscore .
\section{Eficazmente}
\begin{itemize}
\item {Grp. gram.:adv.}
\end{itemize}
De modo eficaz.
Com eficácia.
\section{Eficiência}
\begin{itemize}
\item {Grp. gram.:f.}
\end{itemize}
\begin{itemize}
\item {Proveniência:(Lat. \textunderscore efficientia\textunderscore )}
\end{itemize}
Qualidade daquilo que é eficiente.
\section{Eficiente}
\begin{itemize}
\item {Grp. gram.:adj.}
\end{itemize}
\begin{itemize}
\item {Proveniência:(Lat. \textunderscore efficiens\textunderscore )}
\end{itemize}
O mesmo que \textunderscore eficaz\textunderscore .
\section{Eficientemente}
\begin{itemize}
\item {Grp. gram.:adv.}
\end{itemize}
De modo eficiente.
\section{Efigiar}
\begin{itemize}
\item {Grp. gram.:v. t.}
\end{itemize}
\begin{itemize}
\item {Utilização:Des.}
\end{itemize}
\begin{itemize}
\item {Proveniência:(De \textunderscore efígie\textunderscore )}
\end{itemize}
Apresentar a efígie de.
Executar em efígie (condenados).
\section{Efígie}
\begin{itemize}
\item {Grp. gram.:f.}
\end{itemize}
\begin{itemize}
\item {Proveniência:(Lat. \textunderscore effigies\textunderscore )}
\end{itemize}
Imagem.
Simulácro.
Figura.
Representação.
\section{Eflorescência}
\begin{itemize}
\item {Grp. gram.:f.}
\end{itemize}
\begin{itemize}
\item {Proveniência:(De \textunderscore eflorescente\textunderscore )}
\end{itemize}
Qualidade daquilo que é eflorescente.
Matéria purverulenta, que se fórma numa substância sólida, pela acção do ar livre.
Qualquer exantema, que se eleva pouco acima do nível da pele.
\section{Eflorescente}
\begin{itemize}
\item {Grp. gram.:adj.}
\end{itemize}
\begin{itemize}
\item {Proveniência:(Lat. \textunderscore efflorescens\textunderscore )}
\end{itemize}
Que efloresce.
Que tem camada pulverulenta, como as paredes salitrosas.
Coberto de óxido metálico.
\section{Eflorescer}
\begin{itemize}
\item {Grp. gram.:v. i.}
\end{itemize}
\begin{itemize}
\item {Proveniência:(Do lat. \textunderscore efflorescere\textunderscore )}
\end{itemize}
Começar a florescer.
Apresentar eflorescência.
\section{Efluência}
\begin{itemize}
\item {Grp. gram.:f.}
\end{itemize}
Acto ou efeito de efluir.
\section{Efluente}
\begin{itemize}
\item {Grp. gram.:adj.}
\end{itemize}
\begin{itemize}
\item {Proveniência:(Lat. \textunderscore effluens\textunderscore )}
\end{itemize}
Que eflue.
\section{Efluir}
\begin{itemize}
\item {Grp. gram.:v. i.}
\end{itemize}
\begin{itemize}
\item {Proveniência:(Lat. \textunderscore effluere\textunderscore )}
\end{itemize}
Emanar; proceder.
Irradiar de um ponto.
\section{Eflúvio}
\begin{itemize}
\item {Grp. gram.:m.}
\end{itemize}
\begin{itemize}
\item {Proveniência:(Lat. \textunderscore effluvium\textunderscore )}
\end{itemize}
Fluido subtil, que eflue dos corpos organizados.
Efluência; exalação.
Emissão de aroma.
\section{Efluvioso}
\begin{itemize}
\item {Grp. gram.:adj.}
\end{itemize}
Que lança eflúvios.
\section{Efluxão}
\begin{itemize}
\item {Grp. gram.:f.}
\end{itemize}
\begin{itemize}
\item {Proveniência:(Lat. \textunderscore effluxio\textunderscore )}
\end{itemize}
Saída do feto, nos primeiros dias da gravidez.
\section{Efó}
\begin{itemize}
\item {Grp. gram.:m.}
\end{itemize}
\begin{itemize}
\item {Utilização:Bras}
\end{itemize}
Guisado de camarões e ervas.
\section{Efracção}
\begin{itemize}
\item {Grp. gram.:f.}
\end{itemize}
O mesmo que \textunderscore efractura\textunderscore .
\section{Efractura}
\begin{itemize}
\item {Grp. gram.:f.}
\end{itemize}
\begin{itemize}
\item {Utilização:Med.}
\end{itemize}
\begin{itemize}
\item {Proveniência:(Lat. \textunderscore effractura\textunderscore )}
\end{itemize}
Arrombamento, ruptura.
\section{Efúgio}
\begin{itemize}
\item {Grp. gram.:m.}
\end{itemize}
\begin{itemize}
\item {Proveniência:(Lat. \textunderscore effugium\textunderscore )}
\end{itemize}
Subterfúgio.
Refúgio; fugida.
\section{Efumear}
\begin{itemize}
\item {Grp. gram.:v. t.}
\end{itemize}
Encher de fumo, cobrir de fumo. Cf. Filinto, XVIII, 104.
\section{Efundir}
\begin{itemize}
\item {Grp. gram.:v. t.}
\end{itemize}
\begin{itemize}
\item {Proveniência:(Lat. \textunderscore effundere\textunderscore )}
\end{itemize}
Tirar para fóra.
Derramar.
Entornar.
\section{Efusal}
\begin{itemize}
\item {Grp. gram.:m.}
\end{itemize}
Antigo pêso de linho, correspondente a dois arráteis.
\section{Efusamente}
\begin{itemize}
\item {Grp. gram.:adv.}
\end{itemize}
Com efusão. Cf. Camillo, \textunderscore Estrêl. Fun.\textunderscore , 102.
\section{Efusão}
\begin{itemize}
\item {Grp. gram.:f.}
\end{itemize}
\begin{itemize}
\item {Utilização:Fig.}
\end{itemize}
\begin{itemize}
\item {Proveniência:(Lat. \textunderscore effusio\textunderscore )}
\end{itemize}
Acto de efundir.
Expansão.
Fervor de sentimento afectuoso.
\section{Efuso}
\begin{itemize}
\item {Grp. gram.:adj.}
\end{itemize}
\begin{itemize}
\item {Proveniência:(Lat. \textunderscore effusus\textunderscore )}
\end{itemize}
Em que ha efusão.
\section{Egagropilo}
\begin{itemize}
\item {Grp. gram.:m.}
\end{itemize}
Concreção nas vias digestivas dos ruminantes, formada de pêlos que o animal enguliu, lambendo-se.
\section{Egéria}
\begin{itemize}
\item {Grp. gram.:f.}
\end{itemize}
\begin{itemize}
\item {Utilização:Fig.}
\end{itemize}
\begin{itemize}
\item {Proveniência:(De \textunderscore Egêria\textunderscore , n. p.)}
\end{itemize}
Mulher que inspíra.
Inspiração.
\section{Egerito}
\begin{itemize}
\item {Grp. gram.:m.}
\end{itemize}
\begin{itemize}
\item {Proveniência:(Do gr. \textunderscore aigeiros\textunderscore )}
\end{itemize}
Espécie de cogumelo, que nasce nas árvores.
\section{Égide}
\begin{itemize}
\item {Grp. gram.:f.}
\end{itemize}
\begin{itemize}
\item {Utilização:Fig.}
\end{itemize}
\begin{itemize}
\item {Proveniência:(Do gr. \textunderscore aigis\textunderscore )}
\end{itemize}
Escudo.
Amparo; defesa.
\section{Egídio}
\begin{itemize}
\item {Grp. gram.:m.}
\end{itemize}
\begin{itemize}
\item {Proveniência:(Do gr. \textunderscore aigidion\textunderscore )}
\end{itemize}
Insecto coleóptero pentâmero, da fam. dos lamellicórneos.
\section{Egipan}
\begin{itemize}
\item {Grp. gram.:m.}
\end{itemize}
Divindade das florestas. Cf. Filinto, I, 294; III, 294; XVI, 264.
\section{Egipano}
\begin{itemize}
\item {Grp. gram.:m.}
\end{itemize}
Divindade das florestas. Cf. Filinto, I, 294; III, 294; XVI, 264.
\section{Egipcíaco}
\begin{itemize}
\item {Grp. gram.:adj.}
\end{itemize}
\begin{itemize}
\item {Proveniência:(De \textunderscore egípcio\textunderscore )}
\end{itemize}
Relativo ao Egipto ou aos egípcios.
\section{Egipcião}
\begin{itemize}
\item {Grp. gram.:m.}
\end{itemize}
\begin{itemize}
\item {Utilização:Ant.}
\end{itemize}
O mesmo que \textunderscore egípcio\textunderscore .
\section{Egípcio}
\begin{itemize}
\item {Grp. gram.:adj.}
\end{itemize}
\begin{itemize}
\item {Grp. gram.:M.}
\end{itemize}
\begin{itemize}
\item {Proveniência:(Lat. \textunderscore aegypcius\textunderscore )}
\end{itemize}
Relativo ao Egipto.
Diz-se de uma variedade de trigo mole.
Habitante do Egipto.
Designação dos ciganos, na Inglaterra.
\section{Egiptano}
\begin{itemize}
\item {Grp. gram.:adj.}
\end{itemize}
\begin{itemize}
\item {Utilização:Des.}
\end{itemize}
O mesmo que \textunderscore egípcio\textunderscore .
\section{Egíptico}
\begin{itemize}
\item {Grp. gram.:adj.}
\end{itemize}
O mesmo que \textunderscore egipcíaco\textunderscore .
\section{Egiptologia}
\begin{itemize}
\item {Grp. gram.:f.}
\end{itemize}
\begin{itemize}
\item {Proveniência:(De \textunderscore egiptólogo\textunderscore )}
\end{itemize}
Ciência, que trata das coisas antigas do Egipto, dos seus monumentos, da sua literatura, etc.
\section{Egiptólogo}
\begin{itemize}
\item {Grp. gram.:m.}
\end{itemize}
\begin{itemize}
\item {Proveniência:(De \textunderscore Egipto\textunderscore  n. p. + gr. \textunderscore logos\textunderscore )}
\end{itemize}
Homem, versado em egiptologia.
\section{Egitanense}
\begin{itemize}
\item {Grp. gram.:m.}
\end{itemize}
\begin{itemize}
\item {Utilização:Ext.}
\end{itemize}
\begin{itemize}
\item {Grp. gram.:Adj.}
\end{itemize}
\begin{itemize}
\item {Utilização:Ext.}
\end{itemize}
\begin{itemize}
\item {Proveniência:(Do lat. \textunderscore Egitania\textunderscore , n. p.)}
\end{itemize}
Aquelle que é natural de Idanha-a-Velha.
Habitante da Guarda.
Relativo á Idanha.
Relativo á Guarda. Cf. Herculano, \textunderscore Hist. de Port.\textunderscore , II, 96 e 121.
\section{Egitaniense}
\begin{itemize}
\item {Grp. gram.:m.}
\end{itemize}
\begin{itemize}
\item {Utilização:Ext.}
\end{itemize}
\begin{itemize}
\item {Grp. gram.:Adj.}
\end{itemize}
\begin{itemize}
\item {Utilização:Ext.}
\end{itemize}
\begin{itemize}
\item {Proveniência:(Do lat. \textunderscore Egitania\textunderscore , n. p.)}
\end{itemize}
Aquelle que é natural de Idanha-a-Velha.
Habitante da Guarda.
Relativo á Idanha.
Relativo á Guarda. Cf. Herculano, \textunderscore Hist. de Port.\textunderscore , II, 96 e 121.
\section{Eglefim}
\begin{itemize}
\item {Grp. gram.:m.}
\end{itemize}
Espécie de peixe gádido, (\textunderscore gadus eglefinus\textunderscore ).
\section{Égloga}
\begin{itemize}
\item {Grp. gram.:f.}
\end{itemize}
O mesmo que \textunderscore écloga\textunderscore .
\section{Egocentrismo}
\begin{itemize}
\item {Grp. gram.:m.}
\end{itemize}
\begin{itemize}
\item {Proveniência:(Do lat. \textunderscore ego\textunderscore  + \textunderscore centrum\textunderscore )}
\end{itemize}
O mesmo que \textunderscore subjectivismo\textunderscore , segundo Lombroso.
\section{Egofonia}
\begin{itemize}
\item {Grp. gram.:f.}
\end{itemize}
\begin{itemize}
\item {Proveniência:(Do gr. \textunderscore aix\textunderscore  + \textunderscore phone\textunderscore )}
\end{itemize}
Resonância da voz que, num doente auscultado, tem semelhança com o balido trêmulo de uma cabra.
\section{Egofónico}
\begin{itemize}
\item {Grp. gram.:adj.}
\end{itemize}
Relativo a egofonia.
\section{Egoísmo}
\begin{itemize}
\item {Grp. gram.:m.}
\end{itemize}
\begin{itemize}
\item {Proveniência:(Do lat. \textunderscore ego\textunderscore )}
\end{itemize}
Qualidade de quem ou daquillo que é egoista.
\section{Egoísta}
\begin{itemize}
\item {Grp. gram.:m. f.  e  adj.}
\end{itemize}
\begin{itemize}
\item {Proveniência:(Do lat. \textunderscore ego\textunderscore )}
\end{itemize}
Pessôa, que tudo refere a si, que trata exclusivamente de si e dos seus interesses.
Que tem orgulho.
Que denota falta de sentimentos altruistas.
Em que há propensões concernentes á conservação do indivíduo.
\section{Egoistamente}
\begin{itemize}
\item {fónica:go-is}
\end{itemize}
\begin{itemize}
\item {Grp. gram.:adv.}
\end{itemize}
De modo egoista.
\section{Egoisticamente}
\begin{itemize}
\item {fónica:go-is}
\end{itemize}
\begin{itemize}
\item {Grp. gram.:adv.}
\end{itemize}
(V.egoistamente)
\section{Egoistico}
\begin{itemize}
\item {fónica:go-is}
\end{itemize}
\begin{itemize}
\item {Grp. gram.:adj.}
\end{itemize}
(V.egoísta)
\section{Egometa}
\begin{itemize}
\item {Grp. gram.:m.}
\end{itemize}
\begin{itemize}
\item {Utilização:Des.}
\end{itemize}
\begin{itemize}
\item {Proveniência:(Do lat. \textunderscore ego\textunderscore )}
\end{itemize}
Philósopho ridículo, que se considera o único sábio.
\section{Egophonia}
\begin{itemize}
\item {Grp. gram.:f.}
\end{itemize}
\begin{itemize}
\item {Proveniência:(Do gr. \textunderscore aix\textunderscore  + \textunderscore phone\textunderscore )}
\end{itemize}
Resonância da voz que, num doente auscultado, tem semelhança com o balido trêmulo de uma cabra.
\section{Egophónico}
\begin{itemize}
\item {Grp. gram.:adj.}
\end{itemize}
Relativo a egophonia.
\section{Egoséris}
\begin{itemize}
\item {Grp. gram.:f.}
\end{itemize}
\begin{itemize}
\item {Proveniência:(Do gr. \textunderscore aix\textunderscore  + \textunderscore seris\textunderscore )}
\end{itemize}
Planta, da fam. das compostas, tríbo das chicoriáceas.
\section{Egosomo}
\begin{itemize}
\item {fónica:sô}
\end{itemize}
\begin{itemize}
\item {Grp. gram.:m.}
\end{itemize}
\begin{itemize}
\item {Proveniência:(Do gr. \textunderscore aix\textunderscore  + \textunderscore soma\textunderscore )}
\end{itemize}
Insecto coleóptero tetrâmero, da fam. dos longicórneos.
\section{Egossomo}
\begin{itemize}
\item {Grp. gram.:m.}
\end{itemize}
\begin{itemize}
\item {Proveniência:(Do gr. \textunderscore aix\textunderscore  + \textunderscore soma\textunderscore )}
\end{itemize}
Insecto coleóptero tetrâmero, da fam. dos longicórneos.
\section{Egotismo}
\begin{itemize}
\item {Grp. gram.:m.}
\end{itemize}
O mesmo que \textunderscore subjectivismo\textunderscore .
(Cp. \textunderscore egotista\textunderscore )
\section{Egotista}
\begin{itemize}
\item {Grp. gram.:adj.}
\end{itemize}
\begin{itemize}
\item {Utilização:Neol.}
\end{itemize}
\begin{itemize}
\item {Grp. gram.:M.}
\end{itemize}
\begin{itemize}
\item {Proveniência:(Do lat. \textunderscore ego\textunderscore )}
\end{itemize}
Relativo a egotismo.
Sectário do egotismo.
\section{Egregiamente}
\begin{itemize}
\item {Grp. gram.:adv.}
\end{itemize}
\begin{itemize}
\item {Proveniência:(De \textunderscore egrégio\textunderscore )}
\end{itemize}
De modo insigne, nobre.
\section{Egregio}
\begin{itemize}
\item {Grp. gram.:adj.}
\end{itemize}
\begin{itemize}
\item {Proveniência:(Lat. \textunderscore egregius\textunderscore )}
\end{itemize}
Distinto.
Insigne.
Nobre, admirável.
\section{Egreja}
\textunderscore f.\textunderscore  (e der.)
(V. \textunderscore igreja\textunderscore , etc.)
\section{Egressão}
\begin{itemize}
\item {Grp. gram.:f.}
\end{itemize}
\begin{itemize}
\item {Proveniência:(Lat. \textunderscore egressio\textunderscore )}
\end{itemize}
Acto de saír, de afastar.
\section{Egresso}
\begin{itemize}
\item {Grp. gram.:adj.}
\end{itemize}
\begin{itemize}
\item {Grp. gram.:M.}
\end{itemize}
\begin{itemize}
\item {Proveniência:(Lat. \textunderscore egressus\textunderscore )}
\end{itemize}
Que saiu, que se afastou.
Que deixou de fazer parte de uma communidade.
Saída.
Indivíduo, que deixou o convento; ex-frade.
\section{Égrio}
\begin{itemize}
\item {Grp. gram.:m.}
\end{itemize}
Nome de duas plantas brasileiras.
\section{Egro}
\begin{itemize}
\item {Grp. gram.:adj.}
\end{itemize}
\begin{itemize}
\item {Utilização:Des.}
\end{itemize}
\begin{itemize}
\item {Proveniência:(Do lat. \textunderscore aeger\textunderscore )}
\end{itemize}
O mesmo que \textunderscore doente\textunderscore .
\section{Égua}
\begin{itemize}
\item {Grp. gram.:f.}
\end{itemize}
\begin{itemize}
\item {Proveniência:(Lat. \textunderscore equa\textunderscore )}
\end{itemize}
Fêmea do cavallo.
\section{Eguada}
\begin{itemize}
\item {Grp. gram.:f.}
\end{itemize}
\begin{itemize}
\item {Proveniência:(De \textunderscore égua\textunderscore )}
\end{itemize}
Manada de éguas. Us. principalmente no sul do Brasil. Cf. B. C. Rubim, \textunderscore Vocabulário Bras\textunderscore .
\section{Egual}
\textunderscore adj.\textunderscore  (e der.)
O mesmo que \textunderscore igual\textunderscore , etc.
\section{Eguariço}
\begin{itemize}
\item {Grp. gram.:m.  e  adj.}
\end{itemize}
\begin{itemize}
\item {Grp. gram.:Adj.}
\end{itemize}
\begin{itemize}
\item {Proveniência:(Lat. hyp. \textunderscore equaritius\textunderscore )}
\end{itemize}
O que trata de cavallos.
Diz-se do muar ou da muar, que procede de égua e burro.
\section{Egypcíaco}
\begin{itemize}
\item {Grp. gram.:adj.}
\end{itemize}
\begin{itemize}
\item {Proveniência:(De \textunderscore egýpcio\textunderscore )}
\end{itemize}
Relativo ao Egypto ou aos egýpcios.
\section{Egypcião}
\begin{itemize}
\item {Grp. gram.:m.}
\end{itemize}
\begin{itemize}
\item {Utilização:Ant.}
\end{itemize}
O mesmo que \textunderscore egýpcio\textunderscore .
\section{Egýpcio}
\begin{itemize}
\item {Grp. gram.:adj.}
\end{itemize}
\begin{itemize}
\item {Grp. gram.:M.}
\end{itemize}
\begin{itemize}
\item {Proveniência:(Lat. \textunderscore aegypcius\textunderscore )}
\end{itemize}
Relativo ao Egypto.
Diz-se de uma variedade de trigo molle.
Habitante do Egypto.
Designação dos ciganos, na Inglaterra.
\section{Egyptano}
\begin{itemize}
\item {Grp. gram.:adj.}
\end{itemize}
\begin{itemize}
\item {Utilização:Des.}
\end{itemize}
O mesmo que \textunderscore egýpcio\textunderscore .
\section{Egýptico}
\begin{itemize}
\item {Grp. gram.:adj.}
\end{itemize}
O mesmo que \textunderscore egypcíaco\textunderscore .
\section{Egyptologia}
\begin{itemize}
\item {Grp. gram.:f.}
\end{itemize}
\begin{itemize}
\item {Proveniência:(De \textunderscore egyptólogo\textunderscore )}
\end{itemize}
Sciência, que trata das coisas antigas do Egypto, dos seus monumentos, da sua literatura, etc.
\section{Egyptólogo}
\begin{itemize}
\item {Grp. gram.:m.}
\end{itemize}
\begin{itemize}
\item {Proveniência:(De \textunderscore Egypto\textunderscore  n. p. + gr. \textunderscore logos\textunderscore )}
\end{itemize}
Homem, versado em egyptologia.
\section{Ehamboge}
\begin{itemize}
\item {Grp. gram.:m.}
\end{itemize}
Pássaro conirostro da África occidental.
\section{Ei}
\begin{itemize}
\item {Grp. gram.:pron.}
\end{itemize}
\begin{itemize}
\item {Utilização:Ant.}
\end{itemize}
O mesmo que \textunderscore eu\textunderscore .
\section{Eia!}
\begin{itemize}
\item {Grp. gram.:interj.}
\end{itemize}
\begin{itemize}
\item {Proveniência:(Do gr. \textunderscore eia\textunderscore )}
\end{itemize}
(para estimular, excitar; para indicar admiração)
\section{Eiça!}
\begin{itemize}
\item {Grp. gram.:interj.}
\end{itemize}
\begin{itemize}
\item {Utilização:Prov.}
\end{itemize}
\begin{itemize}
\item {Utilização:trasm.}
\end{itemize}
Voz, com que se manda recuar o boi.
\section{Eiçar}
\begin{itemize}
\item {Grp. gram.:v. i.}
\end{itemize}
\begin{itemize}
\item {Utilização:Prov.}
\end{itemize}
\begin{itemize}
\item {Utilização:trasm.}
\end{itemize}
Dizer eiça ao boi, para recuar.
\section{Eider}
\begin{itemize}
\item {Grp. gram.:m.}
\end{itemize}
\begin{itemize}
\item {Proveniência:(T. sueco)}
\end{itemize}
Ave do norte, espécie de pato, de cujas penas se fazem os edredões.
\section{Eido}
\begin{itemize}
\item {Grp. gram.:m.}
\end{itemize}
\begin{itemize}
\item {Utilização:Prov.}
\end{itemize}
\begin{itemize}
\item {Utilização:Prov.}
\end{itemize}
\begin{itemize}
\item {Utilização:trasm.}
\end{itemize}
\begin{itemize}
\item {Proveniência:(Do lat. \textunderscore aditus\textunderscore )}
\end{itemize}
Pátio.
Quinchoso.
Quintal, junto a uma casa.
Lugar, occupado por uma pessôa ou coisa.
Lugar, que compete a uma pessôa ou coisa: \textunderscore senta-te ali no teu eido\textunderscore ; \textunderscore vá pendurar o capote no seu eido\textunderscore .
\section{Eigreja}
\begin{itemize}
\item {Grp. gram.:f.}
\end{itemize}
\begin{itemize}
\item {Utilização:Ant.}
\end{itemize}
O mesmo que \textunderscore igreja\textunderscore .
\section{Ei-la}
(contr. de \textunderscore eis\textunderscore  + \textunderscore la\textunderscore )
\section{Ei-las}
(contr. de \textunderscore eis\textunderscore  + \textunderscore las\textunderscore )
\section{Ei-lo}
(contr. de \textunderscore eis\textunderscore  + \textunderscore lo\textunderscore )
\section{Ei-los}
(contr. de \textunderscore eis\textunderscore  + \textunderscore los\textunderscore )
\section{...eira}
\begin{itemize}
\item {Grp. gram.:suf.}
\end{itemize}
(designativo de capacidade, collectividade, producção, aptidão, depreciação, etc.: \textunderscore figueira\textunderscore , \textunderscore pereira\textunderscore , \textunderscore bebedeira\textunderscore )
\section{Eira}
\begin{itemize}
\item {Grp. gram.:f.}
\end{itemize}
\begin{itemize}
\item {Proveniência:(Do lat. \textunderscore area\textunderscore )}
\end{itemize}
Porção de terreno liso e duro, ou lage, em que se secam cereaes e legumes, e em que se malham ou trilham e limpam.
Lugar, annexo ás fábricas de açúcar, para guardar as canas, antes de empregadas.
Terreiro, em que se junta o sal, ao lado das marinhas.
\section{Eira}
\begin{itemize}
\item {Grp. gram.:m.}
\end{itemize}
\begin{itemize}
\item {Grp. gram.:f.}
\end{itemize}
Gato do Paraguai.
Espécie de gato do Paraguai.
\section{Eirabairo}
\begin{itemize}
\item {Grp. gram.:m.}
\end{itemize}
Nome de duas espécies de aves africanas.
\section{Eirada}
\begin{itemize}
\item {Grp. gram.:f.}
\end{itemize}
\begin{itemize}
\item {Utilização:Prov.}
\end{itemize}
\begin{itemize}
\item {Utilização:minh.}
\end{itemize}
\begin{itemize}
\item {Proveniência:(De \textunderscore eira\textunderscore )}
\end{itemize}
Porção de cereaes, que se debulham por uma vez na eira.
Recinto fechado junto aos curraes e casas de lavoira, para recreio de gado.
\section{Eirádega}
\begin{itemize}
\item {Grp. gram.:f.}
\end{itemize}
\begin{itemize}
\item {Proveniência:(Do rad. de \textunderscore eira\textunderscore )}
\end{itemize}
Pensão, que antigamente pagavam os emphyteutas aos senhorios, e que variavam, segundo as cláusulas dos aforamentos e contratos.
\section{Eirádego}
\begin{itemize}
\item {Grp. gram.:m.}
\end{itemize}
\begin{itemize}
\item {Utilização:Ant.}
\end{itemize}
\begin{itemize}
\item {Proveniência:(Do rad. de \textunderscore eira\textunderscore )}
\end{itemize}
Medida para cereaes, nos campos marginaes do Tejo.
\section{Eirádiga}
\begin{itemize}
\item {Grp. gram.:f.}
\end{itemize}
\begin{itemize}
\item {Proveniência:(Do rad. de \textunderscore eira\textunderscore )}
\end{itemize}
Pensão, que antigamente pagavam os emphyteutas aos senhorios, e que variavam, segundo as cláusulas dos aforamentos e contratos.
\section{Eirado}
\begin{itemize}
\item {Grp. gram.:m.}
\end{itemize}
\begin{itemize}
\item {Utilização:Prov.}
\end{itemize}
\begin{itemize}
\item {Utilização:minh.}
\end{itemize}
\begin{itemize}
\item {Proveniência:(De \textunderscore eira\textunderscore )}
\end{itemize}
Espaço descoberto, sobre uma casa, ou ao nível de um andar della.
Terrado; terraço.
Eira.
Campo ou terreno, annexo a uma casa de habitação.
\section{Eiramá!}
\begin{itemize}
\item {Grp. gram.:interj.}
\end{itemize}
\begin{itemize}
\item {Utilização:Prov.}
\end{itemize}
\begin{itemize}
\item {Utilização:alent.}
\end{itemize}
O mesmo que \textunderscore eramá\textunderscore .
\section{Eirante}
\begin{itemize}
\item {Grp. gram.:m.}
\end{itemize}
Aquelle que trabalha nas eiras. Cf. Herculano, \textunderscore Hist. de Port.\textunderscore , IX, 145.
\section{Êiri}
\begin{itemize}
\item {Grp. gram.:adv.}
\end{itemize}
\begin{itemize}
\item {Utilização:Ant.}
\end{itemize}
\begin{itemize}
\item {Proveniência:(Lat. \textunderscore herí\textunderscore )}
\end{itemize}
Ontem.
\section{...eiro}
\begin{itemize}
\item {Grp. gram.:suf. m.  e  adj.}
\end{itemize}
(designativo de extensão, capacidade, producção, aptidão, profissão, situação: \textunderscore chapeleiro\textunderscore , \textunderscore deanteiro\textunderscore )
\section{Eiró}
\begin{itemize}
\item {Grp. gram.:f.}
\end{itemize}
\begin{itemize}
\item {Proveniência:(De \textunderscore areóla\textunderscore , de \textunderscore areia\textunderscore , por allusão á areia, com que misturam as eirós nas selhas das vendedeiras)}
\end{itemize}
Espécie de enguia, (\textunderscore anguilla marina\textunderscore ).
\section{Eiró}
\begin{itemize}
\item {Grp. gram.:m.}
\end{itemize}
\begin{itemize}
\item {Utilização:Prov.}
\end{itemize}
\begin{itemize}
\item {Utilização:trasm.}
\end{itemize}
Terra batida e calcada, semelhante a uma eira.
\section{Eirogo}
\begin{itemize}
\item {fónica:eiró}
\end{itemize}
\begin{itemize}
\item {Grp. gram.:m.}
\end{itemize}
\begin{itemize}
\item {Utilização:T. de Barcelos}
\end{itemize}
O mesmo que \textunderscore eiró\textunderscore ^1.
\section{Eirol}
\begin{itemize}
\item {Grp. gram.:f.}
\end{itemize}
\begin{itemize}
\item {Utilização:Ant.}
\end{itemize}
O mesmo que \textunderscore eiró\textunderscore ^1.
\section{Eirós}
\begin{itemize}
\item {Grp. gram.:f.}
\end{itemize}
(V. \textunderscore eiró\textunderscore ^1)
\section{Eis}
\begin{itemize}
\item {Grp. gram.:adv.}
\end{itemize}
Aqui está.
Está, estão: \textunderscore eis findo o dia\textunderscore .
(Talvez de \textunderscore heis\textunderscore , por \textunderscore haveis\textunderscore )
\section{Eitada}
\begin{itemize}
\item {Grp. gram.:f.}
\end{itemize}
\begin{itemize}
\item {Utilização:Prov.}
\end{itemize}
\begin{itemize}
\item {Utilização:minh.}
\end{itemize}
\begin{itemize}
\item {Proveniência:(De \textunderscore eito\textunderscore )}
\end{itemize}
Grande eito, ou grande porção de trabalho de uma pessôa ou série de pessôas, (na sacha, ceifa, etc.).
\section{Eito}
\begin{itemize}
\item {Grp. gram.:m.}
\end{itemize}
\begin{itemize}
\item {Utilização:Prov.}
\end{itemize}
\begin{itemize}
\item {Utilização:minh.}
\end{itemize}
\begin{itemize}
\item {Utilização:Bras}
\end{itemize}
\begin{itemize}
\item {Grp. gram.:Loc. adv.}
\end{itemize}
\begin{itemize}
\item {Proveniência:(Do lat. \textunderscore ictus\textunderscore )}
\end{itemize}
Seguimento ou successão de coisas, que estão na mesma direcção.
Porção de trabalho, realizado por uma pessôa ou série de pessôas, (na sacha, ceifa, etc.).
Roça, onde trabalhavam escravos.
\textunderscore A eito\textunderscore , seguidamente, a fio.
\section{Eiva}
\begin{itemize}
\item {Grp. gram.:f.}
\end{itemize}
\begin{itemize}
\item {Utilização:Fig.}
\end{itemize}
\begin{itemize}
\item {Proveniência:(Do lat. \textunderscore labes\textunderscore ?)}
\end{itemize}
Falha, racha.
Toque na fruta, nódoa, sinal de que a fruta começa a apodrecer.
Defeito phýsico ou \textunderscore moral\textunderscore .
Mácula.
\section{Eivar}
\begin{itemize}
\item {Grp. gram.:v. t.}
\end{itemize}
\begin{itemize}
\item {Grp. gram.:V. p.}
\end{itemize}
\begin{itemize}
\item {Proveniência:(De \textunderscore eiva\textunderscore )}
\end{itemize}
Produzir manchas em.
Viciar, contaminar.
Começar a apodrecer.
Rachar-se.
Contaminar-se.
\section{Eivigar}
\begin{itemize}
\item {Grp. gram.:v. t.}
\end{itemize}
\begin{itemize}
\item {Utilização:Ant.}
\end{itemize}
\begin{itemize}
\item {Proveniência:(Do lat. \textunderscore aedificare\textunderscore )}
\end{itemize}
O mesmo que \textunderscore edificar\textunderscore . Cf. Frei Fortun., \textunderscore Inéd.\textunderscore , I, 306.
\section{Eixa}
\begin{itemize}
\item {Grp. gram.:f.}
\end{itemize}
Planta campestre, parecida com a leituga e aproveitada para alimento de porcos.
\section{Eixe!}
\begin{itemize}
\item {Grp. gram.:interj.}
\end{itemize}
(com que os boieiros animam os bois a andar)
\section{Êixido}
\begin{itemize}
\item {Grp. gram.:m.}
\end{itemize}
\begin{itemize}
\item {Utilização:Ant.}
\end{itemize}
\begin{itemize}
\item {Proveniência:(Do lat. \textunderscore exitus\textunderscore )}
\end{itemize}
O mesmo que \textunderscore eido\textunderscore .
\section{Eixo}
\begin{itemize}
\item {Grp. gram.:m.}
\end{itemize}
\begin{itemize}
\item {Utilização:Fig.}
\end{itemize}
\begin{itemize}
\item {Proveniência:(Do lat. \textunderscore axis\textunderscore )}
\end{itemize}
Peça de metal, madeira ou outras substâncias, cujas extremidades se encaixam nas rodas de um carro ou de uma máquina.
Peça, em volta da qual gira ou se suppõe girar alguma coisa.
Linha recta ficticia ou real, que passa pelo centro de um objecto: \textunderscore o eixo da Terra\textunderscore .
Linha, que divide ao meio certas figuras geométricas, ou a que se refere a posição de um ou mais pontos fixos ou móveis.
Órgão central dos vegetaes, em volta do qual se desenvolvem os órgãos appendiculares.
Linha recta, que liga os polos de um íman.
Linha, que atravessa perpendicularmente um edifício, dividindo-o em duas partes symétricas.
Linha, que divide longitudinalmente uma rua em duas partes iguaes.
Linha recta, que passa pelo centro da câmara e da boca de uma peça de artilharia.
Essência, centro, ponto capital dos acontecimentos.
Apoio, sustentáculo.
Jôgo, em que os rapazes, collocando-se a distâncias iguaes, saltam uns por cima dos outros.
\section{Ejaculação}
\begin{itemize}
\item {Grp. gram.:f.}
\end{itemize}
Acto de ejacular.
Derramamento; jacto.
Acto de expellir abundantemente.
\section{Ejaculador}
\begin{itemize}
\item {Grp. gram.:adj.}
\end{itemize}
\begin{itemize}
\item {Grp. gram.:M.}
\end{itemize}
\begin{itemize}
\item {Proveniência:(De \textunderscore ejacular\textunderscore )}
\end{itemize}
Que ejacula.
Aquillo que serve para a ejaculação.
\section{Ejacular}
\begin{itemize}
\item {Grp. gram.:v. t.}
\end{itemize}
\begin{itemize}
\item {Proveniência:(Lat. \textunderscore ejaculare\textunderscore )}
\end{itemize}
Emittir, lançar de si.
Derramar com fôrça ou com abundância.
\section{Ejaculatório}
\begin{itemize}
\item {Grp. gram.:adj.}
\end{itemize}
\begin{itemize}
\item {Proveniência:(De \textunderscore ejacular\textunderscore )}
\end{itemize}
Próprio para a ejaculação.
Por onde se faz a ejaculação.
\section{...ejar}
\begin{itemize}
\item {Grp. gram.:suf.}
\end{itemize}
(de verbos frequentativos)
\section{Ejecção}
\begin{itemize}
\item {Grp. gram.:f.}
\end{itemize}
\begin{itemize}
\item {Proveniência:(Lat. \textunderscore ejectio\textunderscore )}
\end{itemize}
O mesmo que \textunderscore dejecção\textunderscore .
\section{Ejector}
\begin{itemize}
\item {Grp. gram.:m.}
\end{itemize}
\begin{itemize}
\item {Proveniência:(Do lat. \textunderscore ejicere\textunderscore )}
\end{itemize}
Peça nas máchinas de vapor.
Peça das armas de fogo, para fazer saltar fóra o invólucro do cartucho que detonou.
\section{...êjo}
\begin{itemize}
\item {Grp. gram.:suf.}
\end{itemize}
(designativo de deminuição)
\section{El}
\begin{itemize}
\item {Grp. gram.:art.}
\end{itemize}
O mesmo que \textunderscore o\textunderscore ^2.--Usa-se apenas na fórmula \textunderscore el-rei\textunderscore .
(Cast. \textunderscore el\textunderscore )
\section{...ela}
\begin{itemize}
\item {Grp. gram.:suf.}
\end{itemize}
(designativo de acção)
\section{Elaboração}
\begin{itemize}
\item {Grp. gram.:f.}
\end{itemize}
\begin{itemize}
\item {Proveniência:(Lat. \textunderscore elaboratio\textunderscore )}
\end{itemize}
Acto ou effeito de elaborar.
\section{Elaborador}
\begin{itemize}
\item {Grp. gram.:m.}
\end{itemize}
Aquelle que elabora.
\section{Elaborar}
\begin{itemize}
\item {Grp. gram.:v. t.}
\end{itemize}
\begin{itemize}
\item {Proveniência:(Lat. \textunderscore elaborare\textunderscore )}
\end{itemize}
Preparar, arranjar, a pouco e pouco, com trabalho.
Modificar.
Dar combinação especial a.
Formar, organizar: \textunderscore elaborar um projecto de lei\textunderscore .
Ordenar, pôr em ordem; dispor.
\section{Elação}
\begin{itemize}
\item {Grp. gram.:f.}
\end{itemize}
\begin{itemize}
\item {Utilização:Des.}
\end{itemize}
\begin{itemize}
\item {Utilização:Fig.}
\end{itemize}
\begin{itemize}
\item {Proveniência:(Lat. \textunderscore elatio\textunderscore )}
\end{itemize}
Altivez.
Elevação, sublimidade.
\section{Elafá}
\begin{itemize}
\item {Grp. gram.:m.}
\end{itemize}
\begin{itemize}
\item {Utilização:Mús.}
\end{itemize}
\begin{itemize}
\item {Utilização:Ant.}
\end{itemize}
Mi bemol.
\section{Elafebólias}
\begin{itemize}
\item {Grp. gram.:f. pl.}
\end{itemize}
Festas atenienses em honra de Latona, com ofrendas em fórma de veado.
(Cp. \textunderscore elafebólio\textunderscore )
\section{Elafebólio}
\begin{itemize}
\item {Grp. gram.:m.}
\end{itemize}
\begin{itemize}
\item {Proveniência:(Do gr. \textunderscore elaphos\textunderscore , veado, e \textunderscore ballein\textunderscore , atirar)}
\end{itemize}
Mês ateniense, em que se celebravam as elafebólias, e que primitivamente era o 3.^o e depois foi o 9.^o.
\section{Elafiano}
\begin{itemize}
\item {Grp. gram.:adj.}
\end{itemize}
\begin{itemize}
\item {Proveniência:(Do gr. \textunderscore elaphos\textunderscore )}
\end{itemize}
Relativo ou semelhante ao veado.
\section{Elafografia}
\begin{itemize}
\item {Grp. gram.:f.}
\end{itemize}
\begin{itemize}
\item {Proveniência:(Do gr. \textunderscore elaphos\textunderscore  + \textunderscore graphein\textunderscore )}
\end{itemize}
Tratado á cêrca dos veados.
\section{Elafornito}
\begin{itemize}
\item {Grp. gram.:m.}
\end{itemize}
\begin{itemize}
\item {Proveniência:(Do gr. \textunderscore elaphos\textunderscore  + \textunderscore ornis\textunderscore )}
\end{itemize}
Família de aves, que se comparavam ao veado pela velocidade do seu vôo.
\section{Élafros}
\begin{itemize}
\item {Grp. gram.:m. pl.}
\end{itemize}
\begin{itemize}
\item {Proveniência:(Gr. \textunderscore elaphros\textunderscore )}
\end{itemize}
Insectos carnívoros, da ordem dos coleópteros.
\section{Elaiágneas}
\begin{itemize}
\item {Grp. gram.:f. pl.}
\end{itemize}
Ordem de plantas, a que pertence o elaiagno.
\section{Elaiagno}
\begin{itemize}
\item {Grp. gram.:m.}
\end{itemize}
\begin{itemize}
\item {Proveniência:(Do gr. \textunderscore elaion\textunderscore )}
\end{itemize}
Árvore da Ásia tropical, semelhante á oliveira.
\section{Elaídico}
\begin{itemize}
\item {Grp. gram.:adj.}
\end{itemize}
\begin{itemize}
\item {Proveniência:(Do gr. \textunderscore elaia\textunderscore )}
\end{itemize}
Diz-se de um ácido, que se fórma na saponificação da elaidina.
\section{Elaidina}
\begin{itemize}
\item {Grp. gram.:f.}
\end{itemize}
\begin{itemize}
\item {Proveniência:(Do gr. \textunderscore elaion\textunderscore , azeite)}
\end{itemize}
Substância gorda, obtida pela combinação do ácido azótico com o azeite.
\section{Elaiuria}
\begin{itemize}
\item {fónica:lai-u}
\end{itemize}
\begin{itemize}
\item {Grp. gram.:f.}
\end{itemize}
\begin{itemize}
\item {Utilização:Med.}
\end{itemize}
\begin{itemize}
\item {Proveniência:(Do gr. \textunderscore elaion\textunderscore  + \textunderscore ouron\textunderscore )}
\end{itemize}
Alteração mórbida das urinas, caracterizada por aspecto oleaginoso.
\section{Elaiúrico}
\begin{itemize}
\item {Grp. gram.:adj.}
\end{itemize}
\begin{itemize}
\item {Grp. gram.:M.}
\end{itemize}
Relativo a elaiuria.
Aquelle que soffre elaiuria.
\section{Elami}
\begin{itemize}
\item {Grp. gram.:m.}
\end{itemize}
\begin{itemize}
\item {Utilização:Mús.}
\end{itemize}
Designação antiga da nota, que hoje se chama \textunderscore mi\textunderscore .
\section{Elance}
\begin{itemize}
\item {Grp. gram.:f.}
\end{itemize}
\begin{itemize}
\item {Utilização:Gal}
\end{itemize}
Ímpeto. Cf. Eça, \textunderscore P. Basílio\textunderscore , 522.
(Cp. fr. \textunderscore elancé\textunderscore )
\section{Elanguecer}
\textunderscore v. i.\textunderscore  e \textunderscore p.\textunderscore  (e der.)
O mesmo que \textunderscore elanguescer\textunderscore , etc.
\section{Elangueiro}
\begin{itemize}
\item {Grp. gram.:m.}
\end{itemize}
Vara, com que os pescadores do bacalhau o enfiam depois de pescado.
\section{Elanguescedor}
\begin{itemize}
\item {Grp. gram.:adj.}
\end{itemize}
O mesmo que \textunderscore elanguescente\textunderscore .
\section{Elanguescência}
\begin{itemize}
\item {Grp. gram.:f.}
\end{itemize}
Qualidade de elanguescente.
\section{Elanguescer}
\begin{itemize}
\item {Grp. gram.:v. i.}
\end{itemize}
\begin{itemize}
\item {Grp. gram.:V. p.}
\end{itemize}
\begin{itemize}
\item {Proveniência:(Lat. \textunderscore elanguescere\textunderscore )}
\end{itemize}
Tornar-se lânguido, enfraquecer a pouco e pouco.
Começar a debilitar-se.
(a mesma significação)
\section{Elanguescente}
\begin{itemize}
\item {Grp. gram.:adj.}
\end{itemize}
\begin{itemize}
\item {Proveniência:(Lat. \textunderscore elanguescens\textunderscore )}
\end{itemize}
Que elanguesce; lânguido.
\section{Elaphebólias}
\begin{itemize}
\item {Grp. gram.:f. pl.}
\end{itemize}
Festas athenienses em honra de Latona, com offrendas em fórma de veado.
(Cp. \textunderscore elaphebólio\textunderscore )
\section{Elaphebólio}
\begin{itemize}
\item {Grp. gram.:m.}
\end{itemize}
\begin{itemize}
\item {Proveniência:(Do gr. \textunderscore elaphos\textunderscore , veado, e \textunderscore ballein\textunderscore , atirar)}
\end{itemize}
Mês atheniense, em que se celebravam as elaphebólias, e que primitivamente era o 3.^o e depois foi o 9.^o.
\section{Elaphiano}
\begin{itemize}
\item {Grp. gram.:adj.}
\end{itemize}
\begin{itemize}
\item {Proveniência:(Do gr. \textunderscore elaphos\textunderscore )}
\end{itemize}
Relativo ou semelhante ao veado.
\section{Élapho}
\begin{itemize}
\item {Grp. gram.:m.}
\end{itemize}
\begin{itemize}
\item {Proveniência:(Gr. \textunderscore elaphos\textunderscore )}
\end{itemize}
Designação scientífica do veado.
\section{Elaphographia}
\begin{itemize}
\item {Grp. gram.:f.}
\end{itemize}
\begin{itemize}
\item {Proveniência:(Do gr. \textunderscore elaphos\textunderscore  + \textunderscore graphein\textunderscore )}
\end{itemize}
Tratado á cêrca dos veados.
\section{Elaphornitho}
\begin{itemize}
\item {Grp. gram.:m.}
\end{itemize}
\begin{itemize}
\item {Proveniência:(Do gr. \textunderscore elaphos\textunderscore  + \textunderscore ornis\textunderscore )}
\end{itemize}
Família de aves, que se comparavam ao veado pela velocidade do seu vôo.
\section{Élaphros}
\begin{itemize}
\item {Grp. gram.:m. pl.}
\end{itemize}
\begin{itemize}
\item {Proveniência:(Gr. \textunderscore elaphros\textunderscore )}
\end{itemize}
Insectos carnívoros, da ordem dos coleópteros.
\section{Elar}
\begin{itemize}
\item {Grp. gram.:v. i.  e  p.}
\end{itemize}
\begin{itemize}
\item {Proveniência:(De \textunderscore elo\textunderscore )}
\end{itemize}
Prender-se com elos, segurar-se com as gavinhas, (fallando-se das videiras).
\section{Elasma}
\begin{itemize}
\item {Grp. gram.:f.}
\end{itemize}
\begin{itemize}
\item {Proveniência:(Gr. \textunderscore elasma\textunderscore )}
\end{itemize}
Cada uma das placas córneas, que servem de dentes ás baleias.
\section{Elastério}
\begin{itemize}
\item {Grp. gram.:m.}
\end{itemize}
\begin{itemize}
\item {Utilização:Des.}
\end{itemize}
\begin{itemize}
\item {Utilização:Fig.}
\end{itemize}
\begin{itemize}
\item {Proveniência:(Do rad. do gr. \textunderscore elastes\textunderscore )}
\end{itemize}
Elasticidade.
Energia.
\section{Elasticamente}
\begin{itemize}
\item {Grp. gram.:adv.}
\end{itemize}
\begin{itemize}
\item {Proveniência:(De \textunderscore elástico\textunderscore )}
\end{itemize}
Com elasticidade.
\section{Elasticidade}
\begin{itemize}
\item {Grp. gram.:f.}
\end{itemize}
\begin{itemize}
\item {Utilização:Fig.}
\end{itemize}
\begin{itemize}
\item {Proveniência:(De \textunderscore elástico\textunderscore )}
\end{itemize}
Qualidade daquillo que é elástico.
Energia.
Dobrez, falta de escrúpulos.
\section{Elasticímetro}
\begin{itemize}
\item {Grp. gram.:m.}
\end{itemize}
\begin{itemize}
\item {Proveniência:(Do gr. \textunderscore elastes\textunderscore  + \textunderscore metron\textunderscore )}
\end{itemize}
Apparelho, que regista graphicamente os esforços necessários para se produzir determinado effeito em qualquer pedaço de metal.
\section{Elástico}
\begin{itemize}
\item {Grp. gram.:adj.}
\end{itemize}
\begin{itemize}
\item {Grp. gram.:M.}
\end{itemize}
\begin{itemize}
\item {Proveniência:(Do gr. \textunderscore elastes\textunderscore )}
\end{itemize}
Que tem flexibilidade, que póde curvar-se ou comprimir-se, voltando depois á sua primeira fórma.
Tecido elástico.
Mola elástica.
Cordão elástico, fita elástica.
\section{Élate}
\begin{itemize}
\item {Grp. gram.:m.}
\end{itemize}
\begin{itemize}
\item {Proveniência:(Gr. \textunderscore elates\textunderscore ?)}
\end{itemize}
Gênero de palmeiras, muito semelhante ao das que produzem tâmaras.
\section{Elaterina}
\begin{itemize}
\item {Grp. gram.:f.}
\end{itemize}
\begin{itemize}
\item {Proveniência:(De \textunderscore elatério\textunderscore )}
\end{itemize}
Substância drástica, extrahida de uma espécie de pepino.
\section{Elatério}
\begin{itemize}
\item {Grp. gram.:m.}
\end{itemize}
\begin{itemize}
\item {Proveniência:(Do gr. \textunderscore elater\textunderscore )}
\end{itemize}
Fruto dehiscente, que, abrindo-se depois de maduro, separa as suas válvulas, lançando fóra os esporos.
Pequeno tubo elástico, que se encontra nas cápsulas de algumas plantas cryptogâmicas.
Espécie de pepino, ou pepino bravo, (\textunderscore momordica elaterium\textunderscore , Lin.).
Medicamento, que se fazia com a semente dêste fruto, como vomitório.
\section{Elaterite}
\begin{itemize}
\item {Grp. gram.:f.}
\end{itemize}
\begin{itemize}
\item {Proveniência:(Do gr. elater)}
\end{itemize}
Betume elástico.
\section{Elaterómetro}
\begin{itemize}
\item {Grp. gram.:m.}
\end{itemize}
\begin{itemize}
\item {Proveniência:(Do gr. \textunderscore elater\textunderscore  + \textunderscore metron\textunderscore )}
\end{itemize}
Instrumento, para medir a elasticidade atmosphérica.
\section{Elatina}
\begin{itemize}
\item {Grp. gram.:f.}
\end{itemize}
Pimenteira aquática, (\textunderscore antirrhinon elatina\textunderscore , Lin.).
\section{Elatíneas}
\begin{itemize}
\item {Grp. gram.:f. pl.}
\end{itemize}
Família de plantas, que têm por typo a elatina.
\section{Elatobrânchios}
\begin{itemize}
\item {fónica:qui}
\end{itemize}
\begin{itemize}
\item {Grp. gram.:m. pl.}
\end{itemize}
\begin{itemize}
\item {Proveniência:(Do gr. \textunderscore elate\textunderscore  + \textunderscore brankhia\textunderscore )}
\end{itemize}
Classe de molluscos acéphalos.
\section{Elatobrânquios}
\begin{itemize}
\item {Grp. gram.:m. pl.}
\end{itemize}
\begin{itemize}
\item {Proveniência:(Do gr. \textunderscore elate\textunderscore  + \textunderscore brankhia\textunderscore )}
\end{itemize}
Classe de moluscos acéfalos.
\section{Elator}
\begin{itemize}
\item {Grp. gram.:m.  e  adj.}
\end{itemize}
\begin{itemize}
\item {Utilização:Des.}
\end{itemize}
\begin{itemize}
\item {Proveniência:(Do lat. \textunderscore elatus\textunderscore )}
\end{itemize}
Aquillo que eleva.
Erector.
\section{Elatro}
\begin{itemize}
\item {Grp. gram.:m.}
\end{itemize}
\begin{itemize}
\item {Proveniência:(Do rad. do gr. \textunderscore elater\textunderscore )}
\end{itemize}
Espécie de escaravelho.
\section{Elau}
\begin{itemize}
\item {Grp. gram.:m.}
\end{itemize}
\begin{itemize}
\item {Utilização:Ant.}
\end{itemize}
Multa, que a falsa testemunha era obrigada a pagar.
\section{Elche}
\begin{itemize}
\item {Grp. gram.:m.}
\end{itemize}
\begin{itemize}
\item {Utilização:Ant.}
\end{itemize}
Christão ou moiro renegado, na Índia portuguesa.
(Ár. \textunderscore elj\textunderscore , convertido)
\section{Eldorado}
\begin{itemize}
\item {Grp. gram.:m.}
\end{itemize}
\begin{itemize}
\item {Utilização:Fig.}
\end{itemize}
\begin{itemize}
\item {Proveniência:(Do cast. \textunderscore el\textunderscore  + \textunderscore dorado\textunderscore )}
\end{itemize}
País imaginário, que se dizia descoberto na América do Sul por um subordinado do conquistador Pizarro.
Lugar cheio de delícias e riquezas.
Espécie de videira americana.
\section{Eleagno}
\textunderscore m.\textunderscore  (e der.)
O mesmo que \textunderscore elaiagno\textunderscore , etc.
\section{Eleático}
\begin{itemize}
\item {Grp. gram.:adj.}
\end{itemize}
\begin{itemize}
\item {Grp. gram.:M.}
\end{itemize}
\begin{itemize}
\item {Proveniência:(Lat. \textunderscore eleaticus\textunderscore )}
\end{itemize}
Relativo ao eleatismo.
Sectário do eleatismo. Cf. Latino, \textunderscore Or. da Coróa, XCVI\textunderscore .
\section{Eleatismo}
\begin{itemize}
\item {Grp. gram.:m.}
\end{itemize}
\begin{itemize}
\item {Proveniência:(De \textunderscore eleático\textunderscore )}
\end{itemize}
Antigo systema philosóphico, que só admittia duas espécies de conhecimentos: os que provêm dos sentidos e são apenas illusão, e os que provêm do raciocínio e são os únicos verdadeiros.
\section{Electivamente}
\begin{itemize}
\item {Grp. gram.:adv.}
\end{itemize}
De modo electivo; por meio de eleição.
\section{Electividade}
\begin{itemize}
\item {Grp. gram.:f.}
\end{itemize}
\begin{itemize}
\item {Utilização:Med.}
\end{itemize}
\begin{itemize}
\item {Proveniência:(De \textunderscore electivo\textunderscore )}
\end{itemize}
Facto de os medicamentos actuarem num órgão ou apparelho, ficando os outros apáthicos.
\section{Electivo}
\begin{itemize}
\item {Grp. gram.:adj.}
\end{itemize}
\begin{itemize}
\item {Proveniência:(Lat. \textunderscore electivus\textunderscore )}
\end{itemize}
Relativo a eleição; feito por eleição: \textunderscore é electiva a Câmara dos Deputados\textunderscore .
\section{Electos}
\begin{itemize}
\item {Grp. gram.:m. pl.}
\end{itemize}
\begin{itemize}
\item {Utilização:Des.}
\end{itemize}
\begin{itemize}
\item {Proveniência:(Lat. \textunderscore electus\textunderscore )}
\end{itemize}
Lugares selectos de escritores.
Selecta.
\section{Electricamente}
\begin{itemize}
\item {Grp. gram.:adv.}
\end{itemize}
\begin{itemize}
\item {Proveniência:(De \textunderscore eléctrico\textunderscore )}
\end{itemize}
Por meio da electricidade.
\section{Electricidade}
\begin{itemize}
\item {Grp. gram.:f.}
\end{itemize}
\begin{itemize}
\item {Utilização:Fig.}
\end{itemize}
\begin{itemize}
\item {Proveniência:(De \textunderscore eléctrico\textunderscore )}
\end{itemize}
Propriedade, com que certos corpos friccionados, batidos, aquecidos ou comprimidos, atrahem outros, repellindo-os seguidamente e expellindo centelhas.
Flúido hypothético, a que se attribue a producção de phenómenos eléctricos.
Estado moral, comparado á tensão eléctrica.
\section{Electricista}
\begin{itemize}
\item {Grp. gram.:m.}
\end{itemize}
\begin{itemize}
\item {Proveniência:(De \textunderscore eléctrico\textunderscore )}
\end{itemize}
Indivíduo, que trata de apparelhos eléctricos.
\section{Eléctrico}
\begin{itemize}
\item {Grp. gram.:adj.}
\end{itemize}
\begin{itemize}
\item {Utilização:Fig.}
\end{itemize}
\begin{itemize}
\item {Grp. gram.:M.}
\end{itemize}
\begin{itemize}
\item {Proveniência:(Do gr. \textunderscore elektron\textunderscore )}
\end{itemize}
Relativo a electricidade, ou que é resultado della, ou movido por ella: \textunderscore carros eléctricos\textunderscore .
Que excita, que abala, como a electricidade.
Vehículo, movido por electridade sôbre carris: \textunderscore hoje andei de eléctrico duas horas\textunderscore .
\section{Electrino}
\begin{itemize}
\item {Grp. gram.:adj.}
\end{itemize}
\begin{itemize}
\item {Proveniência:(Lat. \textunderscore electrinus\textunderscore )}
\end{itemize}
Relativo a alambre.
Feito de alambre.
\section{Electriz}
\begin{itemize}
\item {Grp. gram.:f.}
\end{itemize}
\begin{itemize}
\item {Utilização:Des.}
\end{itemize}
\begin{itemize}
\item {Proveniência:(Lat. \textunderscore electrix\textunderscore )}
\end{itemize}
Mulher que elege, eleitora.
\section{Electrização}
\begin{itemize}
\item {Grp. gram.:f.}
\end{itemize}
Acto de electrizar.
\section{Electrizador}
\begin{itemize}
\item {Grp. gram.:adj.}
\end{itemize}
\begin{itemize}
\item {Grp. gram.:M.}
\end{itemize}
Que electriza.
Aquelle que electriza.
\section{Electrizar}
\begin{itemize}
\item {Grp. gram.:v. t.}
\end{itemize}
\begin{itemize}
\item {Utilização:Fig.}
\end{itemize}
\begin{itemize}
\item {Proveniência:(De \textunderscore electro\textunderscore )}
\end{itemize}
Excitar propriedades eléctricas em.
Impressionar vivamente.
Exaltar; enthusiasmar.
\section{Electro}
\begin{itemize}
\item {Grp. gram.:m.}
\end{itemize}
\begin{itemize}
\item {Proveniência:(Gr. \textunderscore elektron\textunderscore )}
\end{itemize}
Âmbar amarelo.
Liga de oiro e prata.
\section{Electro...}
\begin{itemize}
\item {Grp. gram.:pref.}
\end{itemize}
(significativo de âmbar amarelo ou de electricidade)
\section{Electro-chímica}
\begin{itemize}
\item {Grp. gram.:f.}
\end{itemize}
Parte da Chímica, que se occupa dos phenómenos eléctricos que acompanham phenómenos chímicos.
\section{Electro-chímico}
\begin{itemize}
\item {Grp. gram.:adj.}
\end{itemize}
Relativo a electro-chímica.
\section{Electrocução}
\begin{itemize}
\item {Grp. gram.:f.}
\end{itemize}
\begin{itemize}
\item {Utilização:Neol.}
\end{itemize}
Execução capital, por meio de electricidade.
Morte de animaes, pelo mesmo meio.
(Fórma arbitrária, do pref. \textunderscore electro...\textunderscore , e da desinência de \textunderscore execução\textunderscore )
\section{Electro-cultura}
\begin{itemize}
\item {Grp. gram.:f.}
\end{itemize}
Cultura vegetal por meio da electricidade.
\section{Electroda}
\begin{itemize}
\item {Grp. gram.:f.}
\end{itemize}
\begin{itemize}
\item {Proveniência:(Do gr. \textunderscore elektron\textunderscore  + \textunderscore odos\textunderscore )}
\end{itemize}
Substância, em que se da a decomposição chímica, por meio da pilha.
\section{Electródio}
\begin{itemize}
\item {Grp. gram.:m.}
\end{itemize}
(V.electrodo)
\section{Electrodo}
\begin{itemize}
\item {Grp. gram.:m.}
\end{itemize}
\begin{itemize}
\item {Proveniência:(Do gr. \textunderscore elektron\textunderscore  + \textunderscore odos\textunderscore )}
\end{itemize}
Fio conductor, que liga os polos de uma pilha.
\section{Electro-dynâmica}
\begin{itemize}
\item {Grp. gram.:f.}
\end{itemize}
Parte da Phýsica, que se occupa da acção recíproca das correntes eléctricas, e da acção das correntes eléctricas sôbre os magnetes.
\section{Electro-dynâmico}
\begin{itemize}
\item {Grp. gram.:adj.}
\end{itemize}
Que produz corrente eléctrica, ou é produzido por ella.
\section{Electro-dynamismo}
\begin{itemize}
\item {Grp. gram.:m.}
\end{itemize}
Conjunto dos effeitos da electricidade em movimento.
\section{Electrofanite}
\begin{itemize}
\item {Grp. gram.:f.}
\end{itemize}
\begin{itemize}
\item {Utilização:Med.}
\end{itemize}
\begin{itemize}
\item {Proveniência:(De \textunderscore electro...\textunderscore  + \textunderscore fan\textunderscore , t. ind., que significa ventos)}
\end{itemize}
Doença, semelhante á constipação, mas mais rebelde ao tratamento e produzida pelo resfriamento e pela infecção que os ventiladores eléctricos occasionam num compartimento de pouca extensão.
\section{Electro-galvânico}
\begin{itemize}
\item {Grp. gram.:adj.}
\end{itemize}
Relativo á pilha voltaica ou aos seus effeitos.
\section{Electro-galvanismo}
\begin{itemize}
\item {Grp. gram.:m.}
\end{itemize}
Conjunto dos phenómenos electro-galvânicos.
\section{Electrogênese}
\begin{itemize}
\item {Grp. gram.:f.}
\end{itemize}
Producção de electricidade, por meio dos tecidos vivos.
(Cp. \textunderscore electrógeno\textunderscore )
\section{Electrógeno}
\begin{itemize}
\item {Grp. gram.:adj.}
\end{itemize}
\begin{itemize}
\item {Proveniência:(Do gr. \textunderscore electron\textunderscore  + \textunderscore genes\textunderscore )}
\end{itemize}
Que produz electricidade.
\section{Electrografia}
\begin{itemize}
\item {Grp. gram.:f.}
\end{itemize}
\begin{itemize}
\item {Proveniência:(Do gr. \textunderscore elektron\textunderscore  + \textunderscore graphein\textunderscore )}
\end{itemize}
Aplicação da galvanoplastia á producção directa das lâminas, gravadas pela acção da corrente eléctrica.
\section{Electrográfico}
\begin{itemize}
\item {Grp. gram.:adj.}
\end{itemize}
Relativo á electrografia.
\section{Electrógrafo}
\begin{itemize}
\item {Grp. gram.:m.}
\end{itemize}
\begin{itemize}
\item {Proveniência:(Do gr. \textunderscore elektron\textunderscore  + \textunderscore graphein\textunderscore )}
\end{itemize}
Instrumento, que, por meio da electricidade, reproduz desenhos a distância.
\section{Electrographia}
\begin{itemize}
\item {Grp. gram.:f.}
\end{itemize}
\begin{itemize}
\item {Proveniência:(Do gr. \textunderscore elektron\textunderscore  + \textunderscore graphein\textunderscore )}
\end{itemize}
Applicação da galvanoplastia á producção directa das lâminas, gravadas pela acção da corrente eléctrica.
\section{Electrographico}
\begin{itemize}
\item {Grp. gram.:adj.}
\end{itemize}
Relativo á electrographia.
\section{Electrógrapho}
\begin{itemize}
\item {Grp. gram.:m.}
\end{itemize}
\begin{itemize}
\item {Proveniência:(Do gr. \textunderscore elektron\textunderscore  + \textunderscore graphein\textunderscore )}
\end{itemize}
Instrumento, que, por meio da electricidade, reproduz desenhos a distância.
\section{Electro-homeopatha}
\begin{itemize}
\item {Grp. gram.:m.  e  adj.}
\end{itemize}
O que trata os doentes pela electro-homeopathia.
\section{Electro-homeopathia}
\begin{itemize}
\item {Grp. gram.:f.}
\end{itemize}
Arte de curar pela electricidade e pela homeopathia conjuntamente.
\section{Electro-homeopáthico}
\begin{itemize}
\item {Grp. gram.:adj.}
\end{itemize}
Relativo á electro-homeopathia.
\section{Electro-íman}
\begin{itemize}
\item {Grp. gram.:m.}
\end{itemize}
Ferro macio, que se transforma em magnete, sob a acção de uma corrente eléctrica.
\section{Electrolisação}
\begin{itemize}
\item {Grp. gram.:f.}
\end{itemize}
O mesmo que \textunderscore electrólise\textunderscore .
\section{Electrolisar}
\begin{itemize}
\item {Grp. gram.:v. t.}
\end{itemize}
\begin{itemize}
\item {Proveniência:(De \textunderscore electrólise\textunderscore )}
\end{itemize}
Analisar ou decompor por meio de corrente eléctrica.
\section{Electrolisável}
\begin{itemize}
\item {Grp. gram.:adj.}
\end{itemize}
\begin{itemize}
\item {Proveniência:(De \textunderscore electrolisar\textunderscore )}
\end{itemize}
Susceptível de sêr electrolisado.
\section{Electrólise}
\begin{itemize}
\item {Grp. gram.:f.}
\end{itemize}
\begin{itemize}
\item {Proveniência:(Do gr. \textunderscore elektron\textunderscore  + \textunderscore lusis\textunderscore )}
\end{itemize}
Acto de electrolisar.
\section{Electrolítico}
\begin{itemize}
\item {Grp. gram.:adj.}
\end{itemize}
Relativo a electrólise.
\section{Electrólito}
\begin{itemize}
\item {Grp. gram.:m.}
\end{itemize}
\begin{itemize}
\item {Proveniência:(Do gr. \textunderscore elektron\textunderscore  + \textunderscore lutos\textunderscore )}
\end{itemize}
Corpo decomposto pela acção eléctrica.
\section{Electrologia}
\begin{itemize}
\item {Grp. gram.:f.}
\end{itemize}
O mesmo que \textunderscore electrotechnia\textunderscore .
\section{Electrológico}
\begin{itemize}
\item {Grp. gram.:adj.}
\end{itemize}
Relativo á electrologia.
\section{Electrolysação}
\begin{itemize}
\item {Grp. gram.:f.}
\end{itemize}
O mesmo que \textunderscore electrólyse\textunderscore .
\section{Electrolysar}
\begin{itemize}
\item {Grp. gram.:v. t.}
\end{itemize}
\begin{itemize}
\item {Proveniência:(De \textunderscore electrólyse\textunderscore )}
\end{itemize}
Analysar ou decompor por meio de corrente eléctrica.
\section{Electrolysável}
\begin{itemize}
\item {Grp. gram.:adj.}
\end{itemize}
\begin{itemize}
\item {Proveniência:(De \textunderscore electrolysar\textunderscore )}
\end{itemize}
Susceptível de sêr electrolysado.
\section{Electrólyse}
\begin{itemize}
\item {Grp. gram.:f.}
\end{itemize}
\begin{itemize}
\item {Proveniência:(Do gr. \textunderscore elektron\textunderscore  + \textunderscore lusis\textunderscore )}
\end{itemize}
Acto de electrolysar.
\section{Electrolýtico}
\begin{itemize}
\item {Grp. gram.:adj.}
\end{itemize}
Relativo a electrólyse.
\section{Electrólyto}
\begin{itemize}
\item {Grp. gram.:m.}
\end{itemize}
\begin{itemize}
\item {Proveniência:(Do gr. \textunderscore elektron\textunderscore  + \textunderscore lutos\textunderscore )}
\end{itemize}
Corpo decomposto pela acção eléctrica.
\section{Electro-magnete}
\begin{itemize}
\item {Grp. gram.:m.}
\end{itemize}
O mesmo que \textunderscore electro-íman\textunderscore .
\section{Electro-magnético}
\begin{itemize}
\item {Grp. gram.:adj.}
\end{itemize}
Relativo ao electro-magnetismo.
\section{Electro-magnetismo}
\begin{itemize}
\item {Grp. gram.:m.}
\end{itemize}
Phenómenos, que resultam da acção recíproca dos magnetes e de corpos electrizados.
\section{Electrometria}
\begin{itemize}
\item {Grp. gram.:f.}
\end{itemize}
\begin{itemize}
\item {Proveniência:(De \textunderscore electrómetro\textunderscore )}
\end{itemize}
Parte da Phýsica, que tem por objecto a medida da intensidade eléctrica.
\section{Electrómetro}
\begin{itemize}
\item {Grp. gram.:m.}
\end{itemize}
\begin{itemize}
\item {Proveniência:(Do gr. \textunderscore elektron\textunderscore  + \textunderscore metron\textunderscore )}
\end{itemize}
Instrumento, com que se avalia a intensidade eléctrica de um corpo ou com que se conhece a natureza da electricidade que anima êsse corpo.
\section{Electro-motor}
\begin{itemize}
\item {Grp. gram.:m.  e  adj.}
\end{itemize}
Aquillo que desenvolve electricidade.
\section{Electro-negativo}
\begin{itemize}
\item {Grp. gram.:adj.}
\end{itemize}
Relativo ao polo negativo de uma pilha.
\section{Electróphoro}
\begin{itemize}
\item {Grp. gram.:m.}
\end{itemize}
\begin{itemize}
\item {Proveniência:(Do gr. \textunderscore elektron\textunderscore  + \textunderscore phoros\textunderscore )}
\end{itemize}
Disco de resina, em que se desenvolve a electricidade por meio de fricção.
\section{Electro-physiológico}
\begin{itemize}
\item {Grp. gram.:adj.}
\end{itemize}
Relativo á acção de electricidade nos corpos vivos.
\section{Electro-positivo}
\begin{itemize}
\item {Grp. gram.:adj.}
\end{itemize}
Relativo ao polo positivo da pilha.
\section{Electro-punctura}
\begin{itemize}
\item {Grp. gram.:f.}
\end{itemize}
Combinação da electricidade e da acupunctura no tratamento de algumas enfermidades.
\section{Electroscopia}
\begin{itemize}
\item {Grp. gram.:f.}
\end{itemize}
Applicação do electroscópio.
\section{Electroscópico}
\begin{itemize}
\item {Grp. gram.:adj.}
\end{itemize}
Relativo a electroscopia.
\section{Electroscópio}
\begin{itemize}
\item {Grp. gram.:m.}
\end{itemize}
\begin{itemize}
\item {Proveniência:(Do gr. \textunderscore elektron\textunderscore  + \textunderscore skopein\textunderscore )}
\end{itemize}
Apparelho, com que se conhece a presença da electricidade.
\section{Electrostática}
\begin{itemize}
\item {Grp. gram.:f.}
\end{itemize}
Conjunto dos phenómenos eléctricos, independentes da pilha, (o contrário da electro-dynâmica).
\section{Electrostático}
\begin{itemize}
\item {Grp. gram.:adj.}
\end{itemize}
\begin{itemize}
\item {Proveniência:(Do gr. \textunderscore elektron\textunderscore  + \textunderscore statikos\textunderscore )}
\end{itemize}
Relativo á electrostática.
\section{Electrotechnia}
\begin{itemize}
\item {Grp. gram.:f.}
\end{itemize}
\begin{itemize}
\item {Proveniência:(Do gr. \textunderscore elektron\textunderscore  + \textunderscore tekhne\textunderscore )}
\end{itemize}
Tratado da electricidade e dos seus effeitos e applicações.
\section{Electrotéchnico}
\begin{itemize}
\item {Grp. gram.:adj.}
\end{itemize}
Relativo a electrotechnia.
\section{Electrotecnia}
\begin{itemize}
\item {Grp. gram.:f.}
\end{itemize}
\begin{itemize}
\item {Proveniência:(Do gr. \textunderscore elektron\textunderscore  + \textunderscore tekhne\textunderscore )}
\end{itemize}
Tratado da electricidade e dos seus efeitos e aplicações.
\section{Electrotécnico}
\begin{itemize}
\item {Grp. gram.:adj.}
\end{itemize}
Relativo a electrotecnia.
\section{Electroterapeuta}
\begin{itemize}
\item {Grp. gram.:m.}
\end{itemize}
Aquelle que exerce a electroterapêutica.
\section{Electrotherapeuta}
\begin{itemize}
\item {Grp. gram.:m.}
\end{itemize}
Aquelle que exerce a electrotherapêutica.
\section{Electro-therapêutica}
\begin{itemize}
\item {Grp. gram.:f.}
\end{itemize}
Emprêgo da electricidade no tratamento de doenças.
\section{Electro-therapêutico}
\begin{itemize}
\item {Grp. gram.:adj.}
\end{itemize}
Relativo a electro-therapêutica.
\section{Electro-therapia}
\begin{itemize}
\item {Grp. gram.:f.}
\end{itemize}
O mesmo que \textunderscore electro-therapêutica\textunderscore .
\section{Electro-therápico}
\begin{itemize}
\item {Grp. gram.:adj.}
\end{itemize}
Relativo a electro-therapia.
\section{Electrótipo}
\begin{itemize}
\item {Grp. gram.:m.}
\end{itemize}
Aparelho de electrotipia.
\section{Electrotypia}
\begin{itemize}
\item {Grp. gram.:m.}
\end{itemize}
Arte de reproduzir, por processo electro-chímico, os typos, gravuras e outros objectos.
\section{Electrótypo}
\begin{itemize}
\item {Grp. gram.:m.}
\end{itemize}
Apparelho de electrotypia.
\section{Electrovegetómetro}
\begin{itemize}
\item {Grp. gram.:m.}
\end{itemize}
\begin{itemize}
\item {Proveniência:(De \textunderscore elétrico\textunderscore  + \textunderscore vegetal\textunderscore  + gr. \textunderscore metron\textunderscore )}
\end{itemize}
Apparelho, que o Padre Bertholon inventou, para aproveitar a electricidade atmosphérica na cultura do tabaco.
\section{Electro-vital}
\begin{itemize}
\item {Grp. gram.:adj.}
\end{itemize}
Que, sendo de natureza eléctrica, se manifesta na economia animal, em resultado de actos vitaes.
\section{Electro-vitalismo}
\begin{itemize}
\item {Grp. gram.:m.}
\end{itemize}
Falso systema physiológico, que attribue á electricidade os actos do organismo \textunderscore ou\textunderscore , pelo menos, a um fluido vital análogo ao flúido eléctrico.
\section{Electrozone}
\begin{itemize}
\item {Grp. gram.:m.}
\end{itemize}
\begin{itemize}
\item {Proveniência:(De \textunderscore electro...\textunderscore  + \textunderscore ozone\textunderscore )}
\end{itemize}
Água salgada electrizada, que se emprega na desinfecção de ruas, quintaes, canalizações, etc.
\section{Electuário}
\begin{itemize}
\item {Grp. gram.:m.}
\end{itemize}
\begin{itemize}
\item {Proveniência:(Lat. \textunderscore electuarium\textunderscore )}
\end{itemize}
Medicamento, composto de pós e extractos vegetaes, misturados com mel ou açúcar.
\section{Elefanta}
\begin{itemize}
\item {Grp. gram.:f.}
\end{itemize}
\begin{itemize}
\item {Utilização:Ant.}
\end{itemize}
A fêmea do elefante.
Furacão ou tempestade no mar das Indias.
\section{Elefantário}
\begin{itemize}
\item {Grp. gram.:m.}
\end{itemize}
\begin{itemize}
\item {Utilização:Des.}
\end{itemize}
\begin{itemize}
\item {Proveniência:(Lat. \textunderscore elephantarius\textunderscore )}
\end{itemize}
O mesmo que \textunderscore cornaca\textunderscore .
\section{Elefante}
\begin{itemize}
\item {Grp. gram.:m.}
\end{itemize}
\begin{itemize}
\item {Proveniência:(Lat. \textunderscore elephas\textunderscore , \textunderscore elephantis\textunderscore )}
\end{itemize}
Mamífero corpulento, caracterizado principalmente por uma grande tromba e por longas defesas, que constituem o marfim.
\section{Elefantíaco}
\begin{itemize}
\item {Grp. gram.:adj.}
\end{itemize}
\begin{itemize}
\item {Proveniência:(Lat. \textunderscore elephantiacus\textunderscore )}
\end{itemize}
Que sofre elefantíase.
\section{Elefantíase}
\begin{itemize}
\item {Grp. gram.:f.}
\end{itemize}
\begin{itemize}
\item {Proveniência:(Gr. \textunderscore elephantiasis\textunderscore )}
\end{itemize}
Enfermidade cutânea, que produz intumescência e dureza da pele.
Doença, caracterizada principalmente por tubérculos irregulares na pele.
Morfeia.
\section{Elefântico}
\begin{itemize}
\item {Grp. gram.:adj.}
\end{itemize}
\begin{itemize}
\item {Proveniência:(Lat. \textunderscore elephanticus\textunderscore )}
\end{itemize}
O mesmo que \textunderscore elefantino\textunderscore .
\section{Elefantina}
\begin{itemize}
\item {Grp. gram.:adj. f.}
\end{itemize}
\begin{itemize}
\item {Proveniência:(De \textunderscore elefantino\textunderscore )}
\end{itemize}
Diz-se de uma espécie de tartaruga terrestre.
\section{Elefantino}
\begin{itemize}
\item {Grp. gram.:adj.}
\end{itemize}
\begin{itemize}
\item {Proveniência:(Lat. \textunderscore elephantinus\textunderscore )}
\end{itemize}
Relativo a elefante ou á elefantíase.
\section{Elefantografia}
\begin{itemize}
\item {Grp. gram.:f.}
\end{itemize}
\begin{itemize}
\item {Proveniência:(Do gr. \textunderscore elephas\textunderscore  + \textunderscore graphein\textunderscore )}
\end{itemize}
Tratado ou história dos elefantes.
\section{Elefantoide}
\begin{itemize}
\item {Grp. gram.:adj.}
\end{itemize}
\begin{itemize}
\item {Proveniência:(Do gr. \textunderscore elephas\textunderscore  + \textunderscore eidos\textunderscore )}
\end{itemize}
Semelhante ao elefante.
\section{Elefantófago}
\begin{itemize}
\item {Grp. gram.:adj.}
\end{itemize}
\begin{itemize}
\item {Proveniência:(Do gr. \textunderscore elephas\textunderscore  + \textunderscore phagein\textunderscore )}
\end{itemize}
Que come carne de elefante.
\section{Elefantópode}
\begin{itemize}
\item {Grp. gram.:adj.}
\end{itemize}
\begin{itemize}
\item {Proveniência:(Do gr. \textunderscore elephas\textunderscore  + \textunderscore pous\textunderscore , \textunderscore podos\textunderscore )}
\end{itemize}
Que tem pés comparáveis aos do elefante.
\section{Elegância}
\begin{itemize}
\item {Grp. gram.:f.}
\end{itemize}
\begin{itemize}
\item {Proveniência:(Lat. \textunderscore elegantia\textunderscore )}
\end{itemize}
Qualidade daquelle ou daquillo que é elegante.
Donaire.
Distincção.
\section{Elegante}
\begin{itemize}
\item {Grp. gram.:adj.}
\end{itemize}
\begin{itemize}
\item {Grp. gram.:M.  e  f.}
\end{itemize}
\begin{itemize}
\item {Proveniência:(Lat. \textunderscore elegans\textunderscore )}
\end{itemize}
Gracioso, que tem donaire.
Distinto, nobre.
Esbelto.
Que fala com distincção, mas desaffectadamente.
Que está bem escrito: \textunderscore páginas elegantes\textunderscore .
Em que há bôa proporção entre as ideias e a fórma: \textunderscore discurso elegante\textunderscore .
Pessôa elegante.
\section{Elegantemente}
\begin{itemize}
\item {Grp. gram.:adv.}
\end{itemize}
Com elegância; de modo elegante.
\section{Elegantizar}
\begin{itemize}
\item {Grp. gram.:v. t.}
\end{itemize}
\begin{itemize}
\item {Utilização:Neol.}
\end{itemize}
Tornar elegante.
\section{Elegendo}
\begin{itemize}
\item {Grp. gram.:m.}
\end{itemize}
\begin{itemize}
\item {Proveniência:(De \textunderscore eleger\textunderscore )}
\end{itemize}
Aquelle que há de sêr eleito.
\section{Eleger}
\begin{itemize}
\item {Grp. gram.:v. t.}
\end{itemize}
\begin{itemize}
\item {Proveniência:(Lat. \textunderscore eligere\textunderscore )}
\end{itemize}
Escolher, por meio de votos: \textunderscore eleger Deputados\textunderscore .
Escolher.
Preferir.
Optar por.
\section{Elegia}
\begin{itemize}
\item {Grp. gram.:f.}
\end{itemize}
\begin{itemize}
\item {Proveniência:(Gr. \textunderscore elegia\textunderscore )}
\end{itemize}
Composição poética, na literatura grega e latina, composta geralmente de hexâmetros e pentâmetros.
Pequena composição poética, consagrada a luto ou máguas.
\section{Elegíaco}
\begin{itemize}
\item {Grp. gram.:adj.}
\end{itemize}
\begin{itemize}
\item {Proveniência:(Lat. \textunderscore elegiacus\textunderscore )}
\end{itemize}
Relativo a elegia.
Em que há tristeza.
Que chora muito.
\section{Elegibilidade}
\begin{itemize}
\item {Grp. gram.:f.}
\end{itemize}
\begin{itemize}
\item {Proveniência:(Do lat. \textunderscore elegibilis\textunderscore )}
\end{itemize}
Qualidade de quem é elegível.
\section{Elegimento}
\begin{itemize}
\item {Grp. gram.:m.}
\end{itemize}
\begin{itemize}
\item {Utilização:Des.}
\end{itemize}
Acto de eleger.
O mesmo que \textunderscore eleição\textunderscore .
\section{Elegível}
\begin{itemize}
\item {Grp. gram.:adj.}
\end{itemize}
\begin{itemize}
\item {Proveniência:(Lat. \textunderscore eligibilis\textunderscore )}
\end{itemize}
Que póde sêr eleito.
\section{Eleição}
\begin{itemize}
\item {Grp. gram.:f.}
\end{itemize}
\begin{itemize}
\item {Proveniência:(Lat. \textunderscore electio\textunderscore )}
\end{itemize}
Acto de eleger.
Escolha.
Preferência.
\section{Eleiçoeiro}
\begin{itemize}
\item {Grp. gram.:adj.}
\end{itemize}
\begin{itemize}
\item {Utilização:deprec.}
\end{itemize}
\begin{itemize}
\item {Utilização:Neol.}
\end{itemize}
Relativo a eleições políticas: \textunderscore habilidades eleiçoeiras\textunderscore .
\section{Eleito}
\begin{itemize}
\item {Grp. gram.:adj.}
\end{itemize}
\begin{itemize}
\item {Grp. gram.:M.}
\end{itemize}
\begin{itemize}
\item {Proveniência:(Lat. \textunderscore electus\textunderscore )}
\end{itemize}
Em quem recaiu um acto eleitoral.
Escolhido; preferido.
Aquelle que foi eleito, escolhido.
\section{Eleitor}
\begin{itemize}
\item {Grp. gram.:m.}
\end{itemize}
\begin{itemize}
\item {Proveniência:(Lat. \textunderscore elector\textunderscore )}
\end{itemize}
Aquelle que elege.
Aquelle que está nas condições legaes de poder eleger ou de votar alguém para cargos electivos.
Designação de alguns Principes alemães, que elegiam o imperador.
\section{Eleitorado}
\begin{itemize}
\item {Grp. gram.:m.}
\end{itemize}
\begin{itemize}
\item {Proveniência:(De \textunderscore eleitor\textunderscore )}
\end{itemize}
Dignidade dos Principes alemães, que se denominavam eleitores.
Conjunto de eleitores.
\section{Eleitoral}
\begin{itemize}
\item {Grp. gram.:adj.}
\end{itemize}
\begin{itemize}
\item {Proveniência:(De \textunderscore eleitor\textunderscore )}
\end{itemize}
Relativo a eleições ou ao direito de eleger: \textunderscore processo eleitoral\textunderscore .
\section{Eleitriz}
\begin{itemize}
\item {Grp. gram.:f.}
\end{itemize}
Mulher de Principe eleitor. Cf. \textunderscore Hist. Geneal.\textunderscore , II, 266.
(Cp. \textunderscore electriz\textunderscore )
\section{Elelísfaco}
\begin{itemize}
\item {Grp. gram.:m.}
\end{itemize}
\begin{itemize}
\item {Proveniência:(Gr. \textunderscore elelisphakos\textunderscore )}
\end{itemize}
Planta, espécie de salva.
\section{Elelísphaco}
\begin{itemize}
\item {Grp. gram.:m.}
\end{itemize}
\begin{itemize}
\item {Proveniência:(Gr. \textunderscore elelisphakos\textunderscore )}
\end{itemize}
Planta, espécie de salva.
\section{Elementaes}
\begin{itemize}
\item {Grp. gram.:m. pl.}
\end{itemize}
\begin{itemize}
\item {Proveniência:(De \textunderscore elemento\textunderscore )}
\end{itemize}
Espíritos dos elementos, segundo o occultismo.
Agentes astraes, que fixam as imagens errantes e dão vida ás coisas imaginadas.
\section{Elemental}
\begin{itemize}
\item {Grp. gram.:adj.}
\end{itemize}
\begin{itemize}
\item {Utilização:P. us.}
\end{itemize}
O mesmo que \textunderscore elementar\textunderscore .
\section{Elementar}
\begin{itemize}
\item {Grp. gram.:adj.}
\end{itemize}
\begin{itemize}
\item {Proveniência:(De \textunderscore elemento\textunderscore )}
\end{itemize}
Que tem a natureza de elemento.
Relativo a elementos.
Rudimentar: \textunderscore tratado elementar de Algebra\textunderscore .
\section{Elementário}
\begin{itemize}
\item {Grp. gram.:adj.}
\end{itemize}
O mesmo que \textunderscore elementar\textunderscore .
\section{Elemento}
\begin{itemize}
\item {Grp. gram.:m.}
\end{itemize}
\begin{itemize}
\item {Grp. gram.:Pl.}
\end{itemize}
\begin{itemize}
\item {Proveniência:(Lat. \textunderscore elementa\textunderscore )}
\end{itemize}
Designação antiga da terra, da água, do ar e do fogo.
Substância, que se não julga decomponível.
Corpo simples.
Aquillo que entra na composição de alguma coisa, sem perder a sua organização molecular.
Tudo que entra na formação de alguma coisa.
Cada uma das partes que constituem um todo.
Matéria prima.
Ambiente ou meio, em que se vive ou se vegeta: \textunderscore a água é o elemento dos peixes\textunderscore .
Rudimentos.
Noções rudimentares.
\section{Elemi}
\begin{itemize}
\item {Grp. gram.:m.}
\end{itemize}
Substância resinosa.
\section{Elemieira}
\begin{itemize}
\item {Grp. gram.:f.}
\end{itemize}
Árvore que produz o elemi.
\section{Elemina}
\begin{itemize}
\item {Grp. gram.:f.}
\end{itemize}
Resina crystallizável do elemi.
\section{Elena}
\begin{itemize}
\item {Grp. gram.:f.}
\end{itemize}
\begin{itemize}
\item {Utilização:Des.}
\end{itemize}
O mesmo que \textunderscore santelmo\textunderscore .
\section{Elenco}
\begin{itemize}
\item {Grp. gram.:m.}
\end{itemize}
\begin{itemize}
\item {Proveniência:(Gr. \textunderscore elenkhos\textunderscore )}
\end{itemize}
Índice.
Catálogo.
Súmmula.
Relação ou conjunto dos artistas, que constituem uma Companhia theatral.
\section{Elengue}
\begin{itemize}
\item {Grp. gram.:m.}
\end{itemize}
Ave africana, (\textunderscore merops erytropterus\textunderscore ).
\section{Eleocárpeas}
\begin{itemize}
\item {Grp. gram.:f. pl.}
\end{itemize}
\begin{itemize}
\item {Proveniência:(Do gr. \textunderscore elaion\textunderscore  + \textunderscore karpos\textunderscore )}
\end{itemize}
Plantas dicotyledóneas polypétalas, que constituem uma fam. no systema de Jussieu.
\section{Eleoceróleo}
\begin{itemize}
\item {Grp. gram.:m.}
\end{itemize}
\begin{itemize}
\item {Proveniência:(Do gr. \textunderscore elaion\textunderscore  + lat. \textunderscore cera\textunderscore  + \textunderscore oleum\textunderscore )}
\end{itemize}
Emplastro, em que entra cera e óleo.
\section{Eleófago}
\begin{itemize}
\item {Grp. gram.:adj.}
\end{itemize}
\begin{itemize}
\item {Proveniência:(Do gr. \textunderscore elaia\textunderscore  + \textunderscore phagein\textunderscore )}
\end{itemize}
Que se alimenta de azeitonas.
\section{Eleóleo}
\begin{itemize}
\item {Grp. gram.:m.}
\end{itemize}
\begin{itemize}
\item {Proveniência:(Do gr. \textunderscore elaion\textunderscore )}
\end{itemize}
Preparação de óleo com substâncias medicamentosas.
\section{Eleólico}
\begin{itemize}
\item {Grp. gram.:m.}
\end{itemize}
O mesmo que \textunderscore eleóleo\textunderscore .
\section{Eleolíthico}
\begin{itemize}
\item {Grp. gram.:adj.}
\end{itemize}
Relativo a eleólitho.
Que contém eleólitho.
\section{Eleólitho}
\begin{itemize}
\item {Grp. gram.:m.}
\end{itemize}
\begin{itemize}
\item {Proveniência:(Do gr. \textunderscore elaion\textunderscore  + \textunderscore lithos\textunderscore )}
\end{itemize}
Pedra de azeite, nome vulgar de uma espécie da nephelite.
\section{Eleolítico}
\begin{itemize}
\item {Grp. gram.:adj.}
\end{itemize}
Relativo a eleólito.
Que contém eleólito.
\section{Eleólito}
\begin{itemize}
\item {Grp. gram.:m.}
\end{itemize}
\begin{itemize}
\item {Proveniência:(Do gr. \textunderscore elaion\textunderscore  + \textunderscore lithos\textunderscore )}
\end{itemize}
Pedra de azeite, nome vulgar de uma espécie da nefelite.
\section{Eleóphago}
\begin{itemize}
\item {Grp. gram.:adj.}
\end{itemize}
\begin{itemize}
\item {Proveniência:(Do gr. \textunderscore elaia\textunderscore  + \textunderscore phagein\textunderscore )}
\end{itemize}
Que se alimenta de azeitonas.
\section{Eleotésio}
\begin{itemize}
\item {Grp. gram.:m.}
\end{itemize}
\begin{itemize}
\item {Proveniência:(Do gr. \textunderscore elaion\textunderscore  + \textunderscore thesion\textunderscore )}
\end{itemize}
Lugar nos banhos públicos, onde os antigos guardavam o azeite destinado ás fricções.
\section{Eleothésio}
\begin{itemize}
\item {Grp. gram.:m.}
\end{itemize}
\begin{itemize}
\item {Proveniência:(Do gr. \textunderscore elaion\textunderscore  + \textunderscore thesion\textunderscore )}
\end{itemize}
Lugar nos banhos públicos, onde os antigos guardavam o azeite destinado ás fricções.
\section{Elephanta}
\begin{itemize}
\item {Grp. gram.:f.}
\end{itemize}
\begin{itemize}
\item {Utilização:Ant.}
\end{itemize}
A fêmea do elephante.
Furacão ou tempestade no mar das Indias.
\section{Elephantário}
\begin{itemize}
\item {Grp. gram.:m.}
\end{itemize}
\begin{itemize}
\item {Utilização:Des.}
\end{itemize}
\begin{itemize}
\item {Proveniência:(Lat. \textunderscore elephantarius\textunderscore )}
\end{itemize}
O mesmo que \textunderscore cornaca\textunderscore .
\section{Elephante}
\begin{itemize}
\item {Grp. gram.:m.}
\end{itemize}
\begin{itemize}
\item {Proveniência:(Lat. \textunderscore elephas\textunderscore , \textunderscore elephantis\textunderscore )}
\end{itemize}
Mammífero corpulento, caracterizado principalmente por uma grande tromba e por longas defesas, que constituem o marfim.
\section{Elephantíaco}
\begin{itemize}
\item {Grp. gram.:adj.}
\end{itemize}
\begin{itemize}
\item {Proveniência:(Lat. \textunderscore elephantiacus\textunderscore )}
\end{itemize}
Que soffre elephantíase.
\section{Elephantíase}
\begin{itemize}
\item {Grp. gram.:f.}
\end{itemize}
\begin{itemize}
\item {Proveniência:(Gr. \textunderscore elephantiasis\textunderscore )}
\end{itemize}
Enfermidade cutânea, que produz intumescência e dureza da pelle.
Doença, caracterizada principalmente por tubérculos irregulares na pelle.
Morpheia.
\section{Elephântico}
\begin{itemize}
\item {Grp. gram.:adj.}
\end{itemize}
\begin{itemize}
\item {Proveniência:(Lat. \textunderscore elephanticus\textunderscore )}
\end{itemize}
O mesmo que \textunderscore elephantino\textunderscore .
\section{Elephantina}
\begin{itemize}
\item {Grp. gram.:adj. f.}
\end{itemize}
\begin{itemize}
\item {Proveniência:(De \textunderscore elephantino\textunderscore )}
\end{itemize}
Diz-se de uma espécie de tartaruga terrestre.
\section{Elephantino}
\begin{itemize}
\item {Grp. gram.:adj.}
\end{itemize}
\begin{itemize}
\item {Proveniência:(Lat. \textunderscore elephantinus\textunderscore )}
\end{itemize}
Relativo a elephante ou á elephantíase.
\section{Elephantographia}
\begin{itemize}
\item {Grp. gram.:f.}
\end{itemize}
\begin{itemize}
\item {Proveniência:(Do gr. \textunderscore elephas\textunderscore  + \textunderscore graphein\textunderscore )}
\end{itemize}
Tratado ou história dos elephantes.
\section{Elephantoide}
\begin{itemize}
\item {Grp. gram.:adj.}
\end{itemize}
\begin{itemize}
\item {Proveniência:(Do gr. \textunderscore elephas\textunderscore  + \textunderscore eidos\textunderscore )}
\end{itemize}
Semelhante ao elephante.
\section{Elephantóphago}
\begin{itemize}
\item {Grp. gram.:adj.}
\end{itemize}
\begin{itemize}
\item {Proveniência:(Do gr. \textunderscore elephas\textunderscore  + \textunderscore phagein\textunderscore )}
\end{itemize}
Que come carne de elephante.
\section{Elephantópode}
\begin{itemize}
\item {Grp. gram.:adj.}
\end{itemize}
\begin{itemize}
\item {Proveniência:(Do gr. \textunderscore elephas\textunderscore  + \textunderscore pous\textunderscore , \textunderscore podos\textunderscore )}
\end{itemize}
Que tem pés comparáveis aos do elephante.
\section{Eléquis}
\begin{itemize}
\item {Grp. gram.:m. pl.}
\end{itemize}
O mesmo que \textunderscore léquios\textunderscore . Cf. Fern. Mendes Pinto, \textunderscore Peregrin.\textunderscore 
\section{Eleusinas}
\begin{itemize}
\item {Grp. gram.:f. pl.}
\end{itemize}
\begin{itemize}
\item {Proveniência:(De \textunderscore eleusino\textunderscore )}
\end{itemize}
O mesmo que \textunderscore eleusínias\textunderscore . Cf. Castilho, \textunderscore Fastos\textunderscore , II, 524.
\section{Eleusínias}
\begin{itemize}
\item {Grp. gram.:f. pl.}
\end{itemize}
\begin{itemize}
\item {Proveniência:(Lat. \textunderscore eleusinia\textunderscore )}
\end{itemize}
Festas pomposas, que se faziam a Ceres em Elêusis.
\section{Eleusino}
\begin{itemize}
\item {Grp. gram.:adj.}
\end{itemize}
\begin{itemize}
\item {Proveniência:(Lat. \textunderscore eleusinus\textunderscore )}
\end{itemize}
Relativo a Elêusis, cidade grega.
\section{Eleuterantéreo}
\begin{itemize}
\item {Grp. gram.:adj.}
\end{itemize}
\begin{itemize}
\item {Utilização:Bot.}
\end{itemize}
\begin{itemize}
\item {Proveniência:(Do gr. \textunderscore eleutheros\textunderscore  + \textunderscore antheros\textunderscore )}
\end{itemize}
Diz-se dos estames, quando as anteras são livres.
\section{Eleuterogínia}
\begin{itemize}
\item {Grp. gram.:f.}
\end{itemize}
Classe de plantas, que, segundo a \textunderscore Botânica Médica\textunderscore  de Richard, compreende as monocotiledóneas e dicotiledóneas que têm ovário livre.
(Cp. \textunderscore eleuterógino\textunderscore )
\section{Eleuterógino}
\begin{itemize}
\item {Grp. gram.:adj.}
\end{itemize}
\begin{itemize}
\item {Utilização:Bot.}
\end{itemize}
\begin{itemize}
\item {Proveniência:(Do gr. \textunderscore eleutheros\textunderscore  + \textunderscore gune\textunderscore )}
\end{itemize}
Diz-se da flôr, quando o ovário não adere ao cálice.
\section{Eleutheranthéreo}
\begin{itemize}
\item {Grp. gram.:adj.}
\end{itemize}
\begin{itemize}
\item {Utilização:Bot.}
\end{itemize}
\begin{itemize}
\item {Proveniência:(Do gr. \textunderscore eleutheros\textunderscore  + \textunderscore antheros\textunderscore )}
\end{itemize}
Diz-se dos estames, quando as antheras são livres.
\section{Eleutherogýnia}
\begin{itemize}
\item {Grp. gram.:f.}
\end{itemize}
Classe de plantas, que, segundo a \textunderscore Botânica Médica\textunderscore  de Richard, comprehende as monocotyledóneas e dicotyledóneas que têm ovário livre.
(Cp. \textunderscore eleutherógyno\textunderscore )
\section{Eleutherógyno}
\begin{itemize}
\item {Grp. gram.:adj.}
\end{itemize}
\begin{itemize}
\item {Utilização:Bot.}
\end{itemize}
\begin{itemize}
\item {Proveniência:(Do gr. \textunderscore eleutheros\textunderscore  + \textunderscore gune\textunderscore )}
\end{itemize}
Diz-se da flôr, quando o ovário não adhere ao cálice.
\section{Ela}
\begin{itemize}
\item {Proveniência:(Lat. \textunderscore illa\textunderscore )}
\end{itemize}
\textunderscore pron.\textunderscore , (\textunderscore fem.\textunderscore  de \textunderscore ele\textunderscore ).
\section{Elágico}
\begin{itemize}
\item {Grp. gram.:adj.}
\end{itemize}
Diz-se de um ácido, que se precipita, na infusão aquosa da noz de galha.
(Der. violenta do rad. do lat. \textunderscore galla\textunderscore , cujas letras se inverteram para não haver confusão com o ácido gálico)
\section{Êle}
\begin{itemize}
\item {Grp. gram.:pron. pess. m.}
\end{itemize}
\begin{itemize}
\item {Proveniência:(Lat. \textunderscore ille\textunderscore )}
\end{itemize}
O objecto ou o indivíduo, de que se fala.
\section{Eleutrópio}
\begin{itemize}
\item {Grp. gram.:m.}
\end{itemize}
\begin{itemize}
\item {Utilização:Ant.}
\end{itemize}
O mesmo que \textunderscore heliotrópio\textunderscore .
\section{Elevação}
\begin{itemize}
\item {Grp. gram.:f.}
\end{itemize}
\begin{itemize}
\item {Proveniência:(Lat. \textunderscore elevatio\textunderscore )}
\end{itemize}
Acto ou effeito de elevar.
Ponto elevado.
Aumento.
Distincção, nobreza.
Alta posição social.
Altura.
\section{Elevadamente}
\begin{itemize}
\item {Grp. gram.:adv.}
\end{itemize}
De modo elevado.
Com elevação.
\section{Elevado}
\begin{itemize}
\item {Grp. gram.:adj.}
\end{itemize}
\begin{itemize}
\item {Proveniência:(De \textunderscore elevar\textunderscore )}
\end{itemize}
Que tem elevação.
Alto.
Sublime.
Nobre: \textunderscore carácter elevado\textunderscore .
\section{Elevador}
\begin{itemize}
\item {Grp. gram.:adj.}
\end{itemize}
\begin{itemize}
\item {Grp. gram.:M.}
\end{itemize}
\begin{itemize}
\item {Proveniência:(Lat. \textunderscore elevator\textunderscore )}
\end{itemize}
Que eleva.
Apparelho, que serve para elevar, ou para transportar, subindo.
Ascensor.
\section{Elevar}
\begin{itemize}
\item {Grp. gram.:v. t.}
\end{itemize}
\begin{itemize}
\item {Proveniência:(Lat. \textunderscore elevare\textunderscore )}
\end{itemize}
Fazer subir, levantar.
Pôr em lugar alto.
Aumentar; engrandecer.
Preconizar; exaltar, louvando.
\section{Elevatório}
\begin{itemize}
\item {Grp. gram.:adj.}
\end{itemize}
Relativo a elevação; que serve para elevar. Cf. \textunderscore Techn. Rur.\textunderscore , 234.
\section{Elfa}
\begin{itemize}
\item {Grp. gram.:f.}
\end{itemize}
Cova, para plantio de bacello.
\section{Eliciação}
\begin{itemize}
\item {Grp. gram.:f.}
\end{itemize}
Acto de \textunderscore eliciar\textunderscore .
\section{Eliciar}
\begin{itemize}
\item {Grp. gram.:v. t.}
\end{itemize}
\begin{itemize}
\item {Proveniência:(Lat. \textunderscore eliciare\textunderscore )}
\end{itemize}
Fazer saír.
Expulsar.
Conjurar.
Desviar com esconjuros. Cf. Castilho, \textunderscore Fastos\textunderscore , II, 268.
\section{Elícito}
\begin{itemize}
\item {Grp. gram.:adj.}
\end{itemize}
\begin{itemize}
\item {Utilização:Des.}
\end{itemize}
\begin{itemize}
\item {Proveniência:(Lat. \textunderscore elícitus\textunderscore )}
\end{itemize}
Extrahido; attrahido.
\section{Elidir}
\begin{itemize}
\item {Grp. gram.:v. t.}
\end{itemize}
\begin{itemize}
\item {Proveniência:(Lat. \textunderscore elidere\textunderscore )}
\end{itemize}
Eliminar.
Fazer elisão de.
Expungir.
\section{Elidível}
\begin{itemize}
\item {Grp. gram.:adj.}
\end{itemize}
Que se póde elidir.
\section{Eligimento}
\begin{itemize}
\item {Grp. gram.:m.}
\end{itemize}
Acto ou effeito de \textunderscore eligir\textunderscore .
\section{Eligir}
\begin{itemize}
\item {Grp. gram.:v. t.}
\end{itemize}
(Us. em architectura, por \textunderscore erigir\textunderscore )
\section{Eliminação}
\begin{itemize}
\item {Grp. gram.:f.}
\end{itemize}
Acto ou effeito de eliminar.
\section{Eliminador}
\begin{itemize}
\item {Grp. gram.:adj.}
\end{itemize}
\begin{itemize}
\item {Grp. gram.:M.}
\end{itemize}
Que elimina.
Aquelle que elimina.
\section{Eliminar}
\begin{itemize}
\item {Grp. gram.:v. t.}
\end{itemize}
\begin{itemize}
\item {Proveniência:(Lat. \textunderscore eliminare\textunderscore )}
\end{itemize}
Fazer saír.
Expulsar.
Excluir.
Supprimir.
Fazer desapparecer.
\section{Eliminável}
\begin{itemize}
\item {Grp. gram.:adj.}
\end{itemize}
Que se póde eliminar.
\section{Elipanto}
\begin{itemize}
\item {Grp. gram.:adj.}
\end{itemize}
\begin{itemize}
\item {Utilização:Bot.}
\end{itemize}
\begin{itemize}
\item {Proveniência:(Do gr. \textunderscore ellipeis\textunderscore  + \textunderscore anthos\textunderscore )}
\end{itemize}
Que tem flôres incompletas, isto é, flôres que têm só estames ou só pistilos.
\section{Elipsar}
\begin{itemize}
\item {Grp. gram.:v. t.}
\end{itemize}
Eliminar ou fazer elipse de. Cf. Camillo, \textunderscore Narcót.\textunderscore , II, 23.
\section{Elípse}
\begin{itemize}
\item {Grp. gram.:f.}
\end{itemize}
\begin{itemize}
\item {Utilização:Gram.}
\end{itemize}
\begin{itemize}
\item {Utilização:Geom.}
\end{itemize}
\begin{itemize}
\item {Proveniência:(Gr. \textunderscore elleipsis\textunderscore )}
\end{itemize}
Omissão de uma ou mais palavras na frase, sem que esta deixe de sêr clara.
Linha curva, produzida pela secção que um plano oblíquo ao eixo fez num cóne recto.
\section{Elipsógrapho}
\begin{itemize}
\item {Grp. gram.:m.}
\end{itemize}
\begin{itemize}
\item {Proveniência:(Do gr. \textunderscore elleipsis\textunderscore  + \textunderscore graphein\textunderscore )}
\end{itemize}
Instrumento, para traçar elípses.
\section{Elipsoidal}
\begin{itemize}
\item {Grp. gram.:adj.}
\end{itemize}
O mesmo que \textunderscore elipsoide\textunderscore .
\section{Elipsoide}
\begin{itemize}
\item {Grp. gram.:adj.}
\end{itemize}
\begin{itemize}
\item {Utilização:Geom.}
\end{itemize}
\begin{itemize}
\item {Grp. gram.:M.}
\end{itemize}
\begin{itemize}
\item {Grp. gram.:F.}
\end{itemize}
\begin{itemize}
\item {Proveniência:(Do gr. \textunderscore elleipsis\textunderscore  + \textunderscore eidos\textunderscore )}
\end{itemize}
Que tem a fórma de elípse.
Sólido, produzido pela revolução de metade de uma elípse, em volta de um dos seus eixos.
Curva semelhante á elípse.
\section{Elipsologia}
\begin{itemize}
\item {Grp. gram.:f.}
\end{itemize}
\begin{itemize}
\item {Proveniência:(Do gr. \textunderscore elleipsis\textunderscore  + \textunderscore logos\textunderscore )}
\end{itemize}
Tratado á cêrca da maneira do traçar elípses.
\section{Elipsospermo}
\begin{itemize}
\item {Grp. gram.:adj.}
\end{itemize}
\begin{itemize}
\item {Utilização:Bot.}
\end{itemize}
\begin{itemize}
\item {Proveniência:(Do gr. \textunderscore elleipsos\textunderscore  + \textunderscore sperma\textunderscore )}
\end{itemize}
Que tem sementes elípticas.
\section{Elipsóstomo}
\begin{itemize}
\item {Grp. gram.:adj.}
\end{itemize}
\begin{itemize}
\item {Utilização:Zool.}
\end{itemize}
\begin{itemize}
\item {Proveniência:(Do gr. \textunderscore elleipsis\textunderscore  + \textunderscore stoma\textunderscore )}
\end{itemize}
Que tem bôca ou abertura em fórma de elípse.
\section{Elipticamente}
\begin{itemize}
\item {Grp. gram.:adv.}
\end{itemize}
Por elípse.
De modo elíptico.
\section{Elipticidade}
\begin{itemize}
\item {Grp. gram.:f.}
\end{itemize}
Qualidade daquilo que é elíptico.
\section{Elíptico}
\begin{itemize}
\item {Grp. gram.:adj.}
\end{itemize}
\begin{itemize}
\item {Proveniência:(Gr. \textunderscore elleiptikos\textunderscore )}
\end{itemize}
Relativo a elípse.
Em que há elípse.
Que é da natureza da elípse.
\section{Elisão}
\begin{itemize}
\item {Grp. gram.:f.}
\end{itemize}
\begin{itemize}
\item {Proveniência:(Lat. \textunderscore elisio\textunderscore )}
\end{itemize}
Acto ou effeito de elidir.
\section{Elísio}
\begin{itemize}
\item {Grp. gram.:m.}
\end{itemize}
\begin{itemize}
\item {Utilização:Ext.}
\end{itemize}
\begin{itemize}
\item {Grp. gram.:Adj.}
\end{itemize}
\begin{itemize}
\item {Proveniência:(Lat. \textunderscore elysium\textunderscore )}
\end{itemize}
Lugar, occupado nos infernos pelos heróes e pelos homens virtuosos, segundo a religião greco-latina.
Lugar de delicias, bem-aventurança.
Relativo ao elísio.
\section{Elitrária}
\begin{itemize}
\item {Grp. gram.:f.}
\end{itemize}
Gênero de plantas acantáceas.
\section{Elitrite}
\begin{itemize}
\item {Grp. gram.:f.}
\end{itemize}
\begin{itemize}
\item {Proveniência:(De gr. \textunderscore elutron\textunderscore )}
\end{itemize}
Inflamação da vagina.
\section{Élitro}
\begin{itemize}
\item {Grp. gram.:m.}
\end{itemize}
\begin{itemize}
\item {Proveniência:(Gr. \textunderscore elutron\textunderscore )}
\end{itemize}
Asa superior, que cobre a inferior nos coleópteros.
\section{Elitrocele}
\begin{itemize}
\item {Grp. gram.:m.}
\end{itemize}
\begin{itemize}
\item {Proveniência:(Do gr. \textunderscore elutron\textunderscore  + \textunderscore kele\textunderscore )}
\end{itemize}
Hérnia vaginal.
\section{Elitroide}
\begin{itemize}
\item {Grp. gram.:adj. f.}
\end{itemize}
\begin{itemize}
\item {Proveniência:(Do gr. \textunderscore elutron\textunderscore  + \textunderscore eidos\textunderscore )}
\end{itemize}
Diz-se da membrana, que é um prolongamento de peritonéu e acompanha o testículo, quando êste transpõe o anél inguinal.
\section{Elitroplastia}
\begin{itemize}
\item {Grp. gram.:f.}
\end{itemize}
\begin{itemize}
\item {Proveniência:(De gr. \textunderscore elutron\textunderscore  + \textunderscore plassein\textunderscore )}
\end{itemize}
Operação cirúrgica, com que se restabelece uma parte da vagina.
\section{Elitroptose}
\begin{itemize}
\item {Grp. gram.:f.}
\end{itemize}
\begin{itemize}
\item {Proveniência:(Do gr. \textunderscore elutron\textunderscore  + \textunderscore ptosis\textunderscore )}
\end{itemize}
Quéda ou reviramento da vagina.
\section{Elitrorragia}
\begin{itemize}
\item {Grp. gram.:f.}
\end{itemize}
\begin{itemize}
\item {Proveniência:(Do gr. \textunderscore elutron\textunderscore  + \textunderscore rhagein\textunderscore )}
\end{itemize}
Hemorragia vaginal.
\section{Elitrorrafia}
\begin{itemize}
\item {Grp. gram.:f.}
\end{itemize}
\begin{itemize}
\item {Proveniência:(Do gr. \textunderscore elutron\textunderscore  + \textunderscore rhaptein\textunderscore )}
\end{itemize}
Sutura na vagina, para a fechar, em caso de quéda do útero, ou para fechar um ferimento.
\section{Elixação}
\begin{itemize}
\item {Grp. gram.:f.}
\end{itemize}
Acto de \textunderscore elixar\textunderscore .
\section{Elixar}
\begin{itemize}
\item {Grp. gram.:v.}
\end{itemize}
\begin{itemize}
\item {Utilização:t. Pharm.}
\end{itemize}
\begin{itemize}
\item {Proveniência:(Lat. \textunderscore elixare\textunderscore )}
\end{itemize}
Cozer em água (uma substância), para se obter um producto líquido e outro sólido.
\section{Elixir}
\begin{itemize}
\item {Grp. gram.:m.}
\end{itemize}
\begin{itemize}
\item {Utilização:Fig.}
\end{itemize}
\begin{itemize}
\item {Utilização:Irón.}
\end{itemize}
\begin{itemize}
\item {Proveniência:(Do ár. \textunderscore al\textunderscore  + \textunderscore aksir\textunderscore )}
\end{itemize}
Preparação pharmacêutica de certos xaropes com alcoolatos.
Bebida deliciosa.
Aquillo que há de melhor.
Cura infallível; salvação pública.
\section{Ella}
\begin{itemize}
\item {Proveniência:(Lat. \textunderscore illa\textunderscore )}
\end{itemize}
\textunderscore pron.\textunderscore , (\textunderscore fem.\textunderscore  de \textunderscore elle\textunderscore ).
\section{Ellágico}
\begin{itemize}
\item {Grp. gram.:adj.}
\end{itemize}
Diz-se de um ácido, que se precipita, na infusão aquosa da noz de galha.
(Der. violenta do rad. do lat. \textunderscore galla\textunderscore , cujas letras se inverteram para não haver confusão com o ácido gállico)
\section{Êlle}
\begin{itemize}
\item {Grp. gram.:pron. pess. m.}
\end{itemize}
\begin{itemize}
\item {Proveniência:(Lat. \textunderscore ille\textunderscore )}
\end{itemize}
O objecto ou o indivíduo, de que se fala.
\section{Ellipantho}
\begin{itemize}
\item {Grp. gram.:adj.}
\end{itemize}
\begin{itemize}
\item {Utilização:Bot.}
\end{itemize}
\begin{itemize}
\item {Proveniência:(Do gr. \textunderscore ellipeis\textunderscore  + \textunderscore anthos\textunderscore )}
\end{itemize}
Que tem flôres incompletas, isto é, flôres que têm só estames ou só pistillos.
\section{Ellipsar}
\begin{itemize}
\item {Grp. gram.:v. t.}
\end{itemize}
Eliminar ou fazer ellipse de. Cf. Camillo, \textunderscore Narcót.\textunderscore , II, 23.
\section{Ellípse}
\begin{itemize}
\item {Grp. gram.:f.}
\end{itemize}
\begin{itemize}
\item {Utilização:Gram.}
\end{itemize}
\begin{itemize}
\item {Utilização:Geom.}
\end{itemize}
\begin{itemize}
\item {Proveniência:(Gr. \textunderscore elleipsis\textunderscore )}
\end{itemize}
Omissão de uma ou mais palavras na phrase, sem que esta deixe de sêr clara.
Linha curva, produzida pela secção que um plano oblíquo ao eixo fez num cóne recto.
\section{Ellipsógrapho}
\begin{itemize}
\item {Grp. gram.:m.}
\end{itemize}
\begin{itemize}
\item {Proveniência:(Do gr. \textunderscore elleipsis\textunderscore  + \textunderscore graphein\textunderscore )}
\end{itemize}
Instrumento, para traçar ellípses.
\section{Ellipsoidal}
\begin{itemize}
\item {Grp. gram.:adj.}
\end{itemize}
O mesmo que \textunderscore ellipsoide\textunderscore .
\section{Ellipsoide}
\begin{itemize}
\item {Grp. gram.:adj.}
\end{itemize}
\begin{itemize}
\item {Utilização:Geom.}
\end{itemize}
\begin{itemize}
\item {Grp. gram.:M.}
\end{itemize}
\begin{itemize}
\item {Grp. gram.:F.}
\end{itemize}
\begin{itemize}
\item {Proveniência:(Do gr. \textunderscore elleipsis\textunderscore  + \textunderscore eidos\textunderscore )}
\end{itemize}
Que tem a fórma de ellípse.
Sólido, produzido pela revolução de metade de uma ellípse, em volta de um dos seus eixos.
Curva semelhante á ellípse.
\section{Ellipsologia}
\begin{itemize}
\item {Grp. gram.:f.}
\end{itemize}
\begin{itemize}
\item {Proveniência:(Do gr. \textunderscore elleipsis\textunderscore  + \textunderscore logos\textunderscore )}
\end{itemize}
Tratado á cêrca da maneira do traçar ellípses.
\section{Ellipsospermo}
\begin{itemize}
\item {Grp. gram.:adj.}
\end{itemize}
\begin{itemize}
\item {Utilização:Bot.}
\end{itemize}
\begin{itemize}
\item {Proveniência:(Do gr. \textunderscore elleipsos\textunderscore  + \textunderscore sperma\textunderscore )}
\end{itemize}
Que tem sementes ellípticas.
\section{Ellipsóstomo}
\begin{itemize}
\item {Grp. gram.:adj.}
\end{itemize}
\begin{itemize}
\item {Utilização:Zool.}
\end{itemize}
\begin{itemize}
\item {Proveniência:(Do gr. \textunderscore elleipsis\textunderscore  + \textunderscore stoma\textunderscore )}
\end{itemize}
Que tem bôca ou abertura em fórma de ellípse.
\section{Ellipticamente}
\begin{itemize}
\item {Grp. gram.:adv.}
\end{itemize}
Por ellípse.
De modo ellíptico.
\section{Ellipticidade}
\begin{itemize}
\item {Grp. gram.:f.}
\end{itemize}
Qualidade daquillo que é ellíptico.
\section{Ellíptico}
\begin{itemize}
\item {Grp. gram.:adj.}
\end{itemize}
\begin{itemize}
\item {Proveniência:(Gr. \textunderscore elleiptikos\textunderscore )}
\end{itemize}
Relativo a ellípse.
Em que há ellípse.
Que é da natureza da ellípse.
\section{Êllo}
\begin{itemize}
\item {Grp. gram.:pron.}
\end{itemize}
\begin{itemize}
\item {Utilização:ant.}
\end{itemize}
\begin{itemize}
\item {Proveniência:(Lat. \textunderscore illum\textunderscore )}
\end{itemize}
Isto; isso; aquillo.
\section{Elmanismo}
\begin{itemize}
\item {Grp. gram.:m.}
\end{itemize}
Escola poética de Elmano (Bocage). Cf. Garrett, \textunderscore Retr. de Vênus\textunderscore , 190, 222 e 229.
\section{Elmanista}
\begin{itemize}
\item {Grp. gram.:m.}
\end{itemize}
Partidário do elmanismo.
\section{Elmete}
\begin{itemize}
\item {fónica:mê}
\end{itemize}
\begin{itemize}
\item {Grp. gram.:m.}
\end{itemize}
\begin{itemize}
\item {Utilização:Ant.}
\end{itemize}
\begin{itemize}
\item {Proveniência:(De \textunderscore elmo\textunderscore )}
\end{itemize}
Morrião.
\section{Elmo}
\begin{itemize}
\item {Grp. gram.:m.}
\end{itemize}
\begin{itemize}
\item {Utilização:Pop.}
\end{itemize}
\begin{itemize}
\item {Proveniência:(Do ant. al. \textunderscore helme\textunderscore )}
\end{itemize}
Antiga armadura para a cabeça.
Espécie de capacete.
Crosta negra, que se fórma na cabeça das crianças.
\section{Elo}
\begin{itemize}
\item {Grp. gram.:m.}
\end{itemize}
\begin{itemize}
\item {Utilização:Fig.}
\end{itemize}
\begin{itemize}
\item {Utilização:Ant.}
\end{itemize}
\begin{itemize}
\item {Proveniência:(Do lat. \textunderscore anellum\textunderscore  &lt; \textunderscore ãelo\textunderscore  &lt; \textunderscore elo\textunderscore )}
\end{itemize}
Pequena argola.
Gavinha.
Cada um dos anéis de uma cadeia.
Ligação: \textunderscore o elo do amor\textunderscore .
Seis estrigas de linho.
\section{Êlo}
\begin{itemize}
\item {Grp. gram.:pron.}
\end{itemize}
\begin{itemize}
\item {Utilização:ant.}
\end{itemize}
\begin{itemize}
\item {Proveniência:(Lat. \textunderscore illum\textunderscore )}
\end{itemize}
Isto; isso; aquilo.
\section{Elocução}
\begin{itemize}
\item {Grp. gram.:f.}
\end{itemize}
\begin{itemize}
\item {Proveniência:(Lat. \textunderscore elocutio\textunderscore )}
\end{itemize}
Fórma de exprimir por palavras.
Expressão verbal de uma ideia.
Estilo.
Escolha de palavras e phrases.
\section{Elocutória}
\begin{itemize}
\item {Grp. gram.:f.}
\end{itemize}
\begin{itemize}
\item {Utilização:Des.}
\end{itemize}
\begin{itemize}
\item {Proveniência:(De \textunderscore elocutório\textunderscore )}
\end{itemize}
O mesmo que \textunderscore retórica\textunderscore .
\section{Elocutório}
\begin{itemize}
\item {Grp. gram.:adj.}
\end{itemize}
\begin{itemize}
\item {Proveniência:(Lat. \textunderscore elocutorius\textunderscore )}
\end{itemize}
Relativo a elocução.
\section{Eloendro}
\begin{itemize}
\item {Grp. gram.:m.}
\end{itemize}
(V.loendro)
\section{Elóforo}
\begin{itemize}
\item {Grp. gram.:m.}
\end{itemize}
\begin{itemize}
\item {Proveniência:(Do gr. \textunderscore elos\textunderscore  + \textunderscore phoros\textunderscore )}
\end{itemize}
Gênero de insectos coleópteros pentâmeros.
\section{Elogíaco}
\begin{itemize}
\item {Grp. gram.:adj.}
\end{itemize}
\begin{itemize}
\item {Utilização:Des.}
\end{itemize}
Relativo a elogio.
\section{Elogiador}
\begin{itemize}
\item {Grp. gram.:adj.}
\end{itemize}
\begin{itemize}
\item {Grp. gram.:M.}
\end{itemize}
Que elogia.
Aquelle que elogia.
\section{Elogial}
\begin{itemize}
\item {Grp. gram.:adj.}
\end{itemize}
O mesmo que \textunderscore elogíaco\textunderscore . Cf. Filinto, X, 127.
\section{Elogiante}
\begin{itemize}
\item {Grp. gram.:adj.}
\end{itemize}
Que elogia.
\section{Elogiar}
\begin{itemize}
\item {Grp. gram.:v. t.}
\end{itemize}
\begin{itemize}
\item {Proveniência:(De \textunderscore elogio\textunderscore )}
\end{itemize}
Fazer o elogio de.
Louvar.
\section{Elogiável}
\begin{itemize}
\item {Grp. gram.:adj.}
\end{itemize}
Que se póde elogiar.
\section{Elogio}
\begin{itemize}
\item {Grp. gram.:m.}
\end{itemize}
\begin{itemize}
\item {Utilização:Ant.}
\end{itemize}
\begin{itemize}
\item {Utilização:Irón.}
\end{itemize}
\begin{itemize}
\item {Proveniência:(Gr. \textunderscore ellogion\textunderscore )}
\end{itemize}
Louvor.
Gabo.
Narração ou discurso em louvor de alguém.
Inscripção ou sentença, numa sepultura ou numa estátua.
Censura.
\section{Elogioso}
\begin{itemize}
\item {Grp. gram.:adj.}
\end{itemize}
Que envolve elogio; lisonjeiro.
\section{Elogista}
\begin{itemize}
\item {Grp. gram.:m.}
\end{itemize}
\begin{itemize}
\item {Utilização:P. us.}
\end{itemize}
Aquelle que faz ou escreve elogios.
\section{Elohista}
\begin{itemize}
\item {Grp. gram.:adj.}
\end{itemize}
Diz-se de alguns fragmentos do \textunderscore Pentateuco\textunderscore , em que se dá a Deus o nome de \textunderscore Elohim\textunderscore , e que alguns críticos distinguem, quanto á época e á origem, dos outros fragmentos, em que a Deus se dá o nome de \textunderscore Jehovah\textunderscore .
\section{Eloísta}
\begin{itemize}
\item {Grp. gram.:adj.}
\end{itemize}
Diz-se de alguns fragmentos do \textunderscore Pentateuco\textunderscore , em que se dá a Deus o nome de \textunderscore Elohim\textunderscore , e que alguns críticos distinguem, quanto á época e á origem, dos outros fragmentos, em que a Deus se dá o nome de \textunderscore Jehovah\textunderscore .
\section{Elongação}
\begin{itemize}
\item {Grp. gram.:f.}
\end{itemize}
\begin{itemize}
\item {Proveniência:(Do rad. do lat. \textunderscore elongare\textunderscore )}
\end{itemize}
Ângulo, formado pelos raios visuaes de quem observa o sol e um planeta.
Luxação de articulação.
\section{Elóphoro}
\begin{itemize}
\item {Grp. gram.:m.}
\end{itemize}
\begin{itemize}
\item {Proveniência:(Do gr. \textunderscore elos\textunderscore  + \textunderscore phoros\textunderscore )}
\end{itemize}
Gênero de insectos coleópteros pentâmeros.
\section{Eloquência}
\begin{itemize}
\item {fónica:cu-en}
\end{itemize}
\begin{itemize}
\item {Grp. gram.:f.}
\end{itemize}
\begin{itemize}
\item {Proveniência:(Lat. \textunderscore eloquentia\textunderscore )}
\end{itemize}
Faculdade de dizer, por fórma que se domine o ânimo de quem ouve.
Talento de convencer, deleitar ou commover, falando.
Arte de bem falar.
Um dos gêneros da elocução rhetórica.
\section{Eloquente}
\begin{itemize}
\item {fónica:cu-en}
\end{itemize}
\begin{itemize}
\item {Grp. gram.:adj.}
\end{itemize}
\begin{itemize}
\item {Proveniência:(Lat. \textunderscore eloquens\textunderscore )}
\end{itemize}
Que tem eloquência.
Em que há eloquência: \textunderscore discurso eloquente\textunderscore .
Expressivo; significativo: \textunderscore factos eloquentes\textunderscore .
\section{Eloquentemente}
\begin{itemize}
\item {fónica:cu-en}
\end{itemize}
\begin{itemize}
\item {Grp. gram.:adv.}
\end{itemize}
De modo eloquente.
Com eloquência.
\section{Elóquio}
\begin{itemize}
\item {Grp. gram.:m.}
\end{itemize}
\begin{itemize}
\item {Utilização:P. us.}
\end{itemize}
\begin{itemize}
\item {Proveniência:(Lat. \textunderscore eloquium\textunderscore )}
\end{itemize}
Fala; discurso. Cf. Vieira, X, 346 e 347.
\section{Elpiniano}
\begin{itemize}
\item {Grp. gram.:adj.}
\end{itemize}
Relativo a Elpino (António Dinis), ou á sua escola poética.
\section{Elu}
\begin{itemize}
\item {Grp. gram.:m.}
\end{itemize}
O mesmo que \textunderscore cingalá\textunderscore .
\section{Elucidação}
\begin{itemize}
\item {Grp. gram.:f.}
\end{itemize}
Acto de elucidar.
\section{Elucidador}
\begin{itemize}
\item {Grp. gram.:m.  e  adj.}
\end{itemize}
O que elucida.
\section{Elucidar}
\begin{itemize}
\item {Grp. gram.:v. t.}
\end{itemize}
\begin{itemize}
\item {Proveniência:(Lat. \textunderscore elucidare\textunderscore )}
\end{itemize}
Tornar claro.
Fazer vêr, fazer conhecer.
Explicar.
Esclarecer.
\section{Elucidário}
\begin{itemize}
\item {Grp. gram.:m.}
\end{itemize}
\begin{itemize}
\item {Proveniência:(De \textunderscore elucidar\textunderscore )}
\end{itemize}
Livro ou tratado, em que se procura esclarecer termos ou coisas obscuras.
\section{Elucidativo}
\begin{itemize}
\item {Grp. gram.:adj.}
\end{itemize}
Que elucida.
\section{Elucubração}
\begin{itemize}
\item {Grp. gram.:f.}
\end{itemize}
\begin{itemize}
\item {Proveniência:(Lat. \textunderscore elucubratio\textunderscore )}
\end{itemize}
O mesmo que \textunderscore lucubração\textunderscore .
\section{Eludórico}
\begin{itemize}
\item {Grp. gram.:adj.}
\end{itemize}
\begin{itemize}
\item {Proveniência:(Do gr. \textunderscore elaion\textunderscore  + \textunderscore udor\textunderscore )}
\end{itemize}
Diz-se de uma pintura, feita de óleo e água.
\section{Elutriação}
\begin{itemize}
\item {Grp. gram.:f.}
\end{itemize}
\begin{itemize}
\item {Utilização:Pharm.}
\end{itemize}
\begin{itemize}
\item {Proveniência:(Do lat. \textunderscore elutriare\textunderscore )}
\end{itemize}
O mesmo que \textunderscore decantação\textunderscore .
\section{Elvense}
\begin{itemize}
\item {Grp. gram.:adj.}
\end{itemize}
\begin{itemize}
\item {Grp. gram.:M.}
\end{itemize}
Relativo a Elvas.
Habitante de Elvas.
\section{Elvira}
\begin{itemize}
\item {Grp. gram.:f.}
\end{itemize}
Gênero de videiras americanas. Cf. San-Romão, \textunderscore Vitícultura e Vinicultura\textunderscore .
\section{Elxe}
\begin{itemize}
\item {Grp. gram.:m.}
\end{itemize}
\begin{itemize}
\item {Utilização:Ant.}
\end{itemize}
Christão ou moiro renegado, na Índia portuguesa.
(Ár. \textunderscore elj\textunderscore , convertido)
\section{Elýsio}
\begin{itemize}
\item {Grp. gram.:m.}
\end{itemize}
\begin{itemize}
\item {Utilização:Ext.}
\end{itemize}
\begin{itemize}
\item {Grp. gram.:Adj.}
\end{itemize}
\begin{itemize}
\item {Proveniência:(Lat. \textunderscore elysium\textunderscore )}
\end{itemize}
Lugar, occupado nos infernos pelos heróes e pelos homens virtuosos, segundo a religião greco-latina.
Lugar de delicias, bem-aventurança.
Relativo ao elýsio.
\section{Elytrária}
\begin{itemize}
\item {Grp. gram.:f.}
\end{itemize}
Gênero de plantas acantháceas.
\section{Elytrite}
\begin{itemize}
\item {Grp. gram.:f.}
\end{itemize}
\begin{itemize}
\item {Proveniência:(De gr. \textunderscore elutron\textunderscore )}
\end{itemize}
Inflammação da vagina.
\section{Élytro}
\begin{itemize}
\item {Grp. gram.:m.}
\end{itemize}
\begin{itemize}
\item {Proveniência:(Gr. \textunderscore elutron\textunderscore )}
\end{itemize}
Asa superior, que cobre a inferior nos coleópteros.
\section{Elytrocele}
\begin{itemize}
\item {Grp. gram.:m.}
\end{itemize}
\begin{itemize}
\item {Proveniência:(Do gr. \textunderscore elutron\textunderscore  + \textunderscore kele\textunderscore )}
\end{itemize}
Hérnia vaginal.
\section{Elytroide}
\begin{itemize}
\item {Grp. gram.:adj. f.}
\end{itemize}
\begin{itemize}
\item {Proveniência:(Do gr. \textunderscore elutron\textunderscore  + \textunderscore eidos\textunderscore )}
\end{itemize}
Diz-se da membrana, que é um prolongamento de peritonéu e acompanha o testículo, quando êste transpõe o anél inguinal.
\section{Elytroplastia}
\begin{itemize}
\item {Grp. gram.:f.}
\end{itemize}
\begin{itemize}
\item {Proveniência:(De gr. \textunderscore elutron\textunderscore  + \textunderscore plassein\textunderscore )}
\end{itemize}
Operação cirúrgica, com que se restabelece uma parte da vagina.
\section{Elytroptose}
\begin{itemize}
\item {Grp. gram.:f.}
\end{itemize}
\begin{itemize}
\item {Proveniência:(Do gr. \textunderscore elutron\textunderscore  + \textunderscore ptosis\textunderscore )}
\end{itemize}
Quéda ou reviramento da vagina.
\section{Elytrorrhagia}
\begin{itemize}
\item {Grp. gram.:f.}
\end{itemize}
\begin{itemize}
\item {Proveniência:(Do gr. \textunderscore elutron\textunderscore  + \textunderscore rhagein\textunderscore )}
\end{itemize}
Hemorrhagia vaginal.
\section{Elytrorrhaphia}
\begin{itemize}
\item {Grp. gram.:f.}
\end{itemize}
\begin{itemize}
\item {Proveniência:(Do gr. \textunderscore elutron\textunderscore  + \textunderscore rhaptein\textunderscore )}
\end{itemize}
Sutura na vagina, para a fechar, em caso de quéda do útero, ou para fechar um ferimento.
\section{Elzevir}
\begin{itemize}
\item {Grp. gram.:m.}
\end{itemize}
\begin{itemize}
\item {Grp. gram.:M.  e  adj.}
\end{itemize}
\begin{itemize}
\item {Utilização:Ext.}
\end{itemize}
Edição, devida a um typógrapho hollandês, de nome Elzevir.
Diz-se dos caracteres typográphicos, semelhantes aos das edições de Elzevir.
E díz-se das edições semelhantes ás de Elzevir.
\section{Elzeviriano}
\begin{itemize}
\item {Grp. gram.:adj.}
\end{itemize}
\begin{itemize}
\item {Proveniência:(De \textunderscore Elzevir\textunderscore , n. p.)}
\end{itemize}
Diz-se dos caracteres elzevires; e diz-se das edições, compostas nesses caracteres.
\section{Em}
\begin{itemize}
\item {Grp. gram.:prep.}
\end{itemize}
\begin{itemize}
\item {Proveniência:(Do lat. \textunderscore in\textunderscore )}
\end{itemize}
(indicativa de lugar onde, de modo, tempo, causa, estado, fim, divisão, etc.: \textunderscore em casa\textunderscore ; \textunderscore em tempo\textunderscore ; \textunderscore em virtude de\textunderscore ; \textunderscore em solteiro...\textunderscore )
\section{Em...}
\begin{itemize}
\item {Grp. gram.:pref.}
\end{itemize}
(correspondente á prep. \textunderscore em\textunderscore )
\section{Ema}
\begin{itemize}
\item {Grp. gram.:f.}
\end{itemize}
Áve pernalta, espécie de avestruz.
\section{Ema}
\begin{itemize}
\item {Grp. gram.:f.}
\end{itemize}
\begin{itemize}
\item {Utilização:Bras. do N}
\end{itemize}
Acto de mascar tabaco.
Bebedeira.
\section{Emaciação}
\begin{itemize}
\item {Grp. gram.:f.}
\end{itemize}
Acto de \textunderscore emaciar\textunderscore .
\section{Emaciar}
\begin{itemize}
\item {Grp. gram.:v. t.}
\end{itemize}
\begin{itemize}
\item {Grp. gram.:V. i.}
\end{itemize}
\begin{itemize}
\item {Proveniência:(Lat. \textunderscore emaciare\textunderscore )}
\end{itemize}
Tornar magro.
Emmagrecer.
Tornar-se macilento.
\section{Emanação}
\begin{itemize}
\item {Grp. gram.:f.}
\end{itemize}
\begin{itemize}
\item {Proveniência:(Lat. \textunderscore emanatio\textunderscore )}
\end{itemize}
Acto de emanar.
Proveniência, derivação.
\section{Emanante}
\begin{itemize}
\item {Grp. gram.:adj.}
\end{itemize}
\begin{itemize}
\item {Proveniência:(Lat. \textunderscore emanans\textunderscore )}
\end{itemize}
Que emana.
\section{Emanar}
\begin{itemize}
\item {Grp. gram.:v. i.}
\end{itemize}
\begin{itemize}
\item {Proveniência:(Lat. \textunderscore emanare\textunderscore )}
\end{itemize}
Provir.
Saír de.
Proceder.
Disseminar-se em partículas impalpáveis: \textunderscore do jardim emanam aromas\textunderscore .
\section{Emancipação}
\begin{itemize}
\item {Grp. gram.:f.}
\end{itemize}
\begin{itemize}
\item {Proveniência:(Lat. \textunderscore emancipatio\textunderscore )}
\end{itemize}
Acto ou effeito de emancipar.
\section{Emancipador}
\begin{itemize}
\item {Grp. gram.:adj.}
\end{itemize}
Que emancipa.
\section{Emancipar}
\begin{itemize}
\item {Grp. gram.:v. t.}
\end{itemize}
\begin{itemize}
\item {Utilização:Fig.}
\end{itemize}
\begin{itemize}
\item {Proveniência:(Lat. \textunderscore emancipare\textunderscore )}
\end{itemize}
Eximir do poder paterno ou da tutoria (uma pessôa de menor idade).
Libertar.
Isentar.
\section{Emanthioblásteo}
\begin{itemize}
\item {Grp. gram.:adj.}
\end{itemize}
\begin{itemize}
\item {Utilização:Bot.}
\end{itemize}
Que tem a radícula opposta ao hilo.
\section{Emantioblásteo}
\begin{itemize}
\item {Grp. gram.:adj.}
\end{itemize}
\begin{itemize}
\item {Utilização:Bot.}
\end{itemize}
Que tem a radícula oposta ao hilo.
\section{Emasculação}
\begin{itemize}
\item {Grp. gram.:f.}
\end{itemize}
\begin{itemize}
\item {Proveniência:(Do lat. \textunderscore e\textunderscore  + \textunderscore masculus\textunderscore )}
\end{itemize}
O mesmo que \textunderscore castração\textunderscore , no homem.
\section{Em-augamento}
\begin{itemize}
\item {Grp. gram.:m.}
\end{itemize}
\begin{itemize}
\item {Utilização:Prov.}
\end{itemize}
\begin{itemize}
\item {Utilização:trasm.}
\end{itemize}
O mesmo que \textunderscore aguamento\textunderscore .
\section{Em-augar}
\begin{itemize}
\item {Grp. gram.:v. i.}
\end{itemize}
\begin{itemize}
\item {Utilização:Prov.}
\end{itemize}
\begin{itemize}
\item {Utilização:trasm.}
\end{itemize}
Adquirir aguamento; o mesmo que \textunderscore augar\textunderscore .
\section{Embaçadela}
\begin{itemize}
\item {Grp. gram.:f.}
\end{itemize}
\begin{itemize}
\item {Utilização:Pop.}
\end{itemize}
Acto ou effeito de embaçar.
Lôgro, burla.
\section{Embaçador}
\begin{itemize}
\item {Grp. gram.:adj.}
\end{itemize}
\begin{itemize}
\item {Grp. gram.:M.}
\end{itemize}
Que embaça.
Aquelle que embaça.
\section{Embaçamento}
\begin{itemize}
\item {Grp. gram.:m.}
\end{itemize}
O mesmo que \textunderscore embaçadela\textunderscore .
\section{Embaçar}
\begin{itemize}
\item {Grp. gram.:v. t.}
\end{itemize}
\begin{itemize}
\item {Utilização:Fig.}
\end{itemize}
\begin{itemize}
\item {Grp. gram.:V. i.}
\end{itemize}
\begin{itemize}
\item {Proveniência:(De \textunderscore baço\textunderscore )}
\end{itemize}
Embaciar; tornar baço ou pállido.
Confundir.
Enganar.
Lograr, burlar.
Perder a fala ou os sentidos, por motivo de susto ou de surpresa.
Perder a fôrça, batendo em corpo molle.
\section{Embacellar}
\begin{itemize}
\item {Grp. gram.:v. t.}
\end{itemize}
O mesmo que \textunderscore abacellar\textunderscore .
\section{Embaciar}
\begin{itemize}
\item {Grp. gram.:v. t.}
\end{itemize}
\begin{itemize}
\item {Utilização:Fig.}
\end{itemize}
\begin{itemize}
\item {Grp. gram.:V. i.}
\end{itemize}
\begin{itemize}
\item {Proveniência:(De \textunderscore baço\textunderscore )}
\end{itemize}
Tornar baço.
Tirar com o bafo o brilho a: \textunderscore embaciar um espelho\textunderscore .
Deslustrar: \textunderscore embaciar o nome de alguém\textunderscore .
Tornar-se baço.
\section{Embahular}
\begin{itemize}
\item {Grp. gram.:v. t.}
\end{itemize}
\begin{itemize}
\item {Proveniência:(De \textunderscore bahul\textunderscore )}
\end{itemize}
Meter em bahu.
Esconder.
Guardar.
\section{Embaidor}
\begin{itemize}
\item {fónica:ba-i}
\end{itemize}
\begin{itemize}
\item {Grp. gram.:m.  e  adj.}
\end{itemize}
O que embái, que embaça, que engana.
\section{Embaimento}
\begin{itemize}
\item {fónica:ba-i}
\end{itemize}
\begin{itemize}
\item {Grp. gram.:m.}
\end{itemize}
Acto de embair.
\section{Embainhar}
\begin{itemize}
\item {fónica:ba-i}
\end{itemize}
\begin{itemize}
\item {Grp. gram.:v. t.}
\end{itemize}
Meter na baínha.
Impropriamente, guarnecer com baínha.
Fazer baínhas em.
\section{Embair}
\begin{itemize}
\item {Grp. gram.:v. t.}
\end{itemize}
\begin{itemize}
\item {Proveniência:(Do lat. \textunderscore invadere\textunderscore ?)}
\end{itemize}
Embaçar, enganar.
Illudir.
\section{Embaixada}
\begin{itemize}
\item {Grp. gram.:f.}
\end{itemize}
\begin{itemize}
\item {Utilização:Fig.}
\end{itemize}
\begin{itemize}
\item {Proveniência:(Do b. lat. \textunderscore ambassata\textunderscore )}
\end{itemize}
Cargo ou missão de embaixador.
Deputação a um soberano.
Residência do embaixador.
Mensagem entre particulares.
\section{Embaixador}
\begin{itemize}
\item {Grp. gram.:m.}
\end{itemize}
\begin{itemize}
\item {Utilização:Fig.}
\end{itemize}
\begin{itemize}
\item {Proveniência:(De \textunderscore embaixada\textunderscore )}
\end{itemize}
A graduação mais elevada do representante de um Govêrno ou Estado, junto de outro Estado ou Govêrno.
Espécie de jôgo popular.
Emissário.
\section{Embaixadora}
\begin{itemize}
\item {Grp. gram.:f.}
\end{itemize}
\begin{itemize}
\item {Utilização:Pop.}
\end{itemize}
\begin{itemize}
\item {Proveniência:(De \textunderscore embaixador\textunderscore )}
\end{itemize}
Mulher encarregada de missão particular.
\section{Embaixatriz}
\begin{itemize}
\item {Grp. gram.:f.}
\end{itemize}
Mulher de embaixador.
\section{Embala}
\begin{itemize}
\item {Grp. gram.:f.}
\end{itemize}
O mesmo que \textunderscore senzala\textunderscore .
\section{Embaladeira}
\begin{itemize}
\item {Grp. gram.:f.}
\end{itemize}
Mulher que embala.
\section{Embaladeiras}
\begin{itemize}
\item {Grp. gram.:f.}
\end{itemize}
\begin{itemize}
\item {Proveniência:(De \textunderscore embalar\textunderscore )}
\end{itemize}
Peças curvas, na parte inferior de um berço, para lhe facilitar o balanço.
\section{Embalado}
\begin{itemize}
\item {Grp. gram.:adj.}
\end{itemize}
\begin{itemize}
\item {Utilização:Prov.}
\end{itemize}
\begin{itemize}
\item {Utilização:alent.}
\end{itemize}
\begin{itemize}
\item {Proveniência:(De \textunderscore bala\textunderscore ^1)}
\end{itemize}
Diz-se do tiro de chumbo, cujos projécteis, partindo de perto, chegam juntos, como formando uma bala.
\section{Embalado}
\begin{itemize}
\item {Grp. gram.:adj.}
\end{itemize}
\begin{itemize}
\item {Utilização:Prov.}
\end{itemize}
\begin{itemize}
\item {Utilização:alent.}
\end{itemize}
\begin{itemize}
\item {Proveniência:(De \textunderscore bala\textunderscore ^3)}
\end{itemize}
Que tem muito dinheiro.
\section{Embalador}
\begin{itemize}
\item {Grp. gram.:m.  e  adj.}
\end{itemize}
O que embala.
\section{Embalagem}
\begin{itemize}
\item {Grp. gram.:f.}
\end{itemize}
\begin{itemize}
\item {Utilização:Gal}
\end{itemize}
\begin{itemize}
\item {Proveniência:(Fr. \textunderscore emballage\textunderscore )}
\end{itemize}
Acto de empacotar fazendas ou outros objectos.
\section{Embalançar}
\begin{itemize}
\item {Grp. gram.:v. t.}
\end{itemize}
O mesmo que \textunderscore balançar\textunderscore .
\section{Embalanço}
\begin{itemize}
\item {Grp. gram.:m.}
\end{itemize}
\begin{itemize}
\item {Utilização:Gír.}
\end{itemize}
\begin{itemize}
\item {Proveniência:(De \textunderscore embalançar\textunderscore )}
\end{itemize}
Acto de conhecer a prenhez, pelo toque da vagina. Cp. \textunderscore rechaço\textunderscore .
\section{Embalançoso}
\begin{itemize}
\item {Grp. gram.:adj.}
\end{itemize}
Que embalança. Cf. Filinto, IX, 182.
\section{Embalar}
\begin{itemize}
\item {Grp. gram.:v. t.}
\end{itemize}
\begin{itemize}
\item {Utilização:Fig.}
\end{itemize}
\begin{itemize}
\item {Proveniência:(De um rad. sanscr. \textunderscore bal\textunderscore , que se encontra em \textunderscore baloiço\textunderscore , \textunderscore abalar\textunderscore , etc.)}
\end{itemize}
Agitar o berço de: \textunderscore embalar a criança\textunderscore .
Acalentar, para adormecer.
Acarinhar.
Afagar.
Encantar: \textunderscore embalava-o a esperança\textunderscore .
Embair.
\section{Embalçamento}
\begin{itemize}
\item {Grp. gram.:m.}
\end{itemize}
Acto de embalçar.
\section{Embalçar}
\begin{itemize}
\item {Grp. gram.:v. t.}
\end{itemize}
\begin{itemize}
\item {Proveniência:(De \textunderscore balça\textunderscore )}
\end{itemize}
Meter em balça (o vinho ou mosto).
Meter nas balças, nos bosques.
\section{Embalete}
\begin{itemize}
\item {fónica:lê}
\end{itemize}
\begin{itemize}
\item {Grp. gram.:m.}
\end{itemize}
\begin{itemize}
\item {Proveniência:(Do rad. de \textunderscore embalar\textunderscore )}
\end{itemize}
Braço de um apparelho, com que se esgota água a bórdo.
\section{Embalhestado}
\begin{itemize}
\item {Grp. gram.:adj.}
\end{itemize}
Diz-se do solipede, cujos membros, desviando-se da sua verdadeira direcção, fazem que elle, estacionando, se incline mais ou menos para deante. Cf. Leon, \textunderscore Arte de ferrar\textunderscore , 152.
(Cp. \textunderscore ancado\textunderscore )
\section{Embalo}
\begin{itemize}
\item {Grp. gram.:m.}
\end{itemize}
\begin{itemize}
\item {Utilização:Prov.}
\end{itemize}
\begin{itemize}
\item {Utilização:dur.}
\end{itemize}
Acto de embalar.
Baloiço, balanço.
Rede de arrastar, com saco, a reboque de duas embarcações.
Forte ondulação ou agitação da água.
Embate da água.
\section{Embaloiçar}
\begin{itemize}
\item {Grp. gram.:v. t.}
\end{itemize}
\begin{itemize}
\item {Utilização:Pop.}
\end{itemize}
O mesmo que \textunderscore baloiçar\textunderscore . Cf. Fern. Pereira, \textunderscore Caça de Altan.\textunderscore , p. I, c. XIII.
\section{Embalouçar}
\begin{itemize}
\item {Grp. gram.:v. t.}
\end{itemize}
\begin{itemize}
\item {Utilização:Pop.}
\end{itemize}
O mesmo que \textunderscore balouçar\textunderscore . Cf. Fern. Pereira, \textunderscore Caça de Altan.\textunderscore , p. I, c. XIII.
\section{Embalsamação}
\begin{itemize}
\item {Grp. gram.:f.}
\end{itemize}
Acto ou effeito de embalsamar.
\section{Embalsamador}
\begin{itemize}
\item {Grp. gram.:m.}
\end{itemize}
Aquelle que embalsama.
\section{Embalsamamento}
\begin{itemize}
\item {Grp. gram.:m.}
\end{itemize}
Acto de embalsamar.
\section{Embalsamar}
\begin{itemize}
\item {Grp. gram.:v. t.}
\end{itemize}
\begin{itemize}
\item {Proveniência:(De \textunderscore bálsamo\textunderscore )}
\end{itemize}
Dar perfume a.
Aromatizar.
Preparar (cadáveres), para resistirem á corrupção.
\section{Embama}
\begin{itemize}
\item {Grp. gram.:f.}
\end{itemize}
\begin{itemize}
\item {Utilização:Des.}
\end{itemize}
\begin{itemize}
\item {Proveniência:(Lat. \textunderscore embamma\textunderscore )}
\end{itemize}
Tempêro de iguarias.
\section{Embamata}
\begin{itemize}
\item {Grp. gram.:f.}
\end{itemize}
\begin{itemize}
\item {Utilização:Bras}
\end{itemize}
Môlho grosso.
(Cp. \textunderscore embamma\textunderscore )
\section{Embamma}
\begin{itemize}
\item {Grp. gram.:f.}
\end{itemize}
\begin{itemize}
\item {Utilização:Des.}
\end{itemize}
\begin{itemize}
\item {Proveniência:(Lat. \textunderscore embamma\textunderscore )}
\end{itemize}
Tempêro de iguarias.
\section{Embammata}
\begin{itemize}
\item {Grp. gram.:f.}
\end{itemize}
\begin{itemize}
\item {Utilização:Bras}
\end{itemize}
Môlho grosso.
(Cp. \textunderscore embamma\textunderscore )
\section{Embanar}
\begin{itemize}
\item {Grp. gram.:v. t.}
\end{itemize}
\begin{itemize}
\item {Utilização:Prov.}
\end{itemize}
(V.abanar)
\section{Embandar}
\begin{itemize}
\item {Grp. gram.:v. t.}
\end{itemize}
\begin{itemize}
\item {Proveniência:(De \textunderscore banda\textunderscore ^1)}
\end{itemize}
Pôr bandas em.
\section{Embandar}
\begin{itemize}
\item {Grp. gram.:v. t.}
\end{itemize}
\begin{itemize}
\item {Proveniência:(De \textunderscore bando\textunderscore )}
\end{itemize}
Bandear, pôr em bando:«\textunderscore ...e se embandam, quaes pássaros\textunderscore ». Filinto, VIII, 296.
\section{Embandeirar}
\begin{itemize}
\item {Grp. gram.:v. t.}
\end{itemize}
\begin{itemize}
\item {Grp. gram.:V. i.}
\end{itemize}
Ornar com bandeiras.
Deitar bandeira (o milho).
\section{Embaraçada}
\begin{itemize}
\item {Grp. gram.:f.}
\end{itemize}
\begin{itemize}
\item {Utilização:Prov.}
\end{itemize}
\begin{itemize}
\item {Utilização:alent.}
\end{itemize}
\begin{itemize}
\item {Proveniência:(De \textunderscore embaraçar\textunderscore )}
\end{itemize}
Mulher pejada.
\section{Embaraçadamente}
\begin{itemize}
\item {Grp. gram.:adv.}
\end{itemize}
\begin{itemize}
\item {Proveniência:(De \textunderscore embaraçar\textunderscore )}
\end{itemize}
Com embaraço.
\section{Embaraçador}
\begin{itemize}
\item {Grp. gram.:adj.}
\end{itemize}
\begin{itemize}
\item {Grp. gram.:M.}
\end{itemize}
Que embaraça.
Aquelle que embaraça.
\section{Embaraçante}
\begin{itemize}
\item {Grp. gram.:adj.}
\end{itemize}
Que embaraça.
\section{Embaraçar}
\begin{itemize}
\item {Grp. gram.:v. t.}
\end{itemize}
\begin{itemize}
\item {Utilização:Prov.}
\end{itemize}
\begin{itemize}
\item {Utilização:trasm.}
\end{itemize}
\begin{itemize}
\item {Proveniência:(De \textunderscore embaraço\textunderscore )}
\end{itemize}
Pôr embaraço a.
Estorvar.
Impedir.
Complicar.
Obstruir.
Perturbar, enlear.
Tornar grávida, pejar.
\section{Embaraço}
\begin{itemize}
\item {Grp. gram.:m.}
\end{itemize}
\begin{itemize}
\item {Utilização:Pop.}
\end{itemize}
\begin{itemize}
\item {Proveniência:(Do fr. \textunderscore embarras\textunderscore )}
\end{itemize}
Aquillo que atravanca um caminho ou impede o andamento de.
Estôrvo, obstáculo.
Impedimento.
Perturbação de espirito.
Hesitação.
Doença ligeira no canal digestivo.
Catamênio.
Gravidez.
\section{Embaraçosamente}
\begin{itemize}
\item {Grp. gram.:adv.}
\end{itemize}
De modo embaraçoso.
\section{Embaraçoso}
\begin{itemize}
\item {Grp. gram.:adj.}
\end{itemize}
\begin{itemize}
\item {Proveniência:(De \textunderscore embaraço\textunderscore )}
\end{itemize}
Em que há embaraço.
Que embaraça.
Difficultoso.
\section{Embarafustar}
\begin{itemize}
\item {Grp. gram.:v. i.}
\end{itemize}
\begin{itemize}
\item {Utilização:Bras}
\end{itemize}
Entrar de tropel, desordenadamente.
(Cp. \textunderscore barafustar\textunderscore )
\section{Embaralhar}
\begin{itemize}
\item {Grp. gram.:v. t.}
\end{itemize}
(V.baralhar)
\section{Embaratecer}
\begin{itemize}
\item {Grp. gram.:v. t.}
\end{itemize}
\begin{itemize}
\item {Grp. gram.:V. p.}
\end{itemize}
Tornar barato.
Tornar-se barato.
\section{Embarbar}
\begin{itemize}
\item {Grp. gram.:v. t.}
\end{itemize}
\begin{itemize}
\item {Proveniência:(De \textunderscore barbar\textunderscore )}
\end{itemize}
O mesmo que \textunderscore encasar\textunderscore , (t. de carp.).
\section{Embarbascamento}
\begin{itemize}
\item {Grp. gram.:m.}
\end{itemize}
Acto de embarbascar.
\section{Embarbascar}
\begin{itemize}
\item {Grp. gram.:v. t.}
\end{itemize}
\begin{itemize}
\item {Grp. gram.:V. i.}
\end{itemize}
\begin{itemize}
\item {Utilização:Ant.}
\end{itemize}
\begin{itemize}
\item {Proveniência:(De \textunderscore barbasco\textunderscore )}
\end{itemize}
Fazer entontecer com barbasco (os peixes).
Entontecer.
O mesmo que \textunderscore embasbacar\textunderscore . Cf. Pant. de Aveiro, \textunderscore Itiner.\textunderscore , 198, (2.^a ed.).
\section{Embarbecer}
\begin{itemize}
\item {Grp. gram.:v. i.}
\end{itemize}
Criar barba.
\section{Embarbelar}
\begin{itemize}
\item {Grp. gram.:v. i.}
\end{itemize}
\begin{itemize}
\item {Proveniência:(De \textunderscore barbela\textunderscore )}
\end{itemize}
Encornar-se o pegador do toiro, agarrando-se â barbela.
\section{Embarbellar}
\begin{itemize}
\item {Grp. gram.:v. i.}
\end{itemize}
\begin{itemize}
\item {Proveniência:(De \textunderscore barbella\textunderscore )}
\end{itemize}
Encornar-se o pegador do toiro, agarrando-se â barbella.
\section{Embarbilhado}
\begin{itemize}
\item {Grp. gram.:adj.}
\end{itemize}
\begin{itemize}
\item {Proveniência:(De \textunderscore embarbilhar\textunderscore )}
\end{itemize}
Que traz barbilho.
\section{Embarbilhar}
\begin{itemize}
\item {Grp. gram.:v. t.}
\end{itemize}
\begin{itemize}
\item {Proveniência:(De \textunderscore barbilho\textunderscore )}
\end{itemize}
Pôr barbilho a, (especialmente a cabritos).
\section{Embarcação}
\begin{itemize}
\item {Grp. gram.:f.}
\end{itemize}
Acto de embarcar.
Navio, barco.
Qualquer construcção, destinada a navegar.
\section{Embarcadiço}
\begin{itemize}
\item {Grp. gram.:m.  e  adj.}
\end{itemize}
\begin{itemize}
\item {Proveniência:(De \textunderscore embarcar\textunderscore )}
\end{itemize}
O que anda habitualmente embarcado.
Marinheiro.
\section{Embarcadoiro}
\begin{itemize}
\item {Grp. gram.:m.}
\end{itemize}
\begin{itemize}
\item {Proveniência:(De \textunderscore embarcar\textunderscore )}
\end{itemize}
Lugar, onde se embarca.
Caes.
Lugar, onde se sobe para o combóio; gare.
\section{Embarcadouro}
\begin{itemize}
\item {Grp. gram.:m.}
\end{itemize}
\begin{itemize}
\item {Proveniência:(De \textunderscore embarcar\textunderscore )}
\end{itemize}
Lugar, onde se embarca.
Caes.
Lugar, onde se sobe para o combóio; gare.
\section{Embarcamento}
\begin{itemize}
\item {Grp. gram.:m.}
\end{itemize}
Acto de embarcar.
\section{Embarcar}
\begin{itemize}
\item {Grp. gram.:v. t.}
\end{itemize}
\begin{itemize}
\item {Utilização:Gír. lisb.}
\end{itemize}
\begin{itemize}
\item {Grp. gram.:V. i.  e  p.}
\end{itemize}
\begin{itemize}
\item {Utilização:Ext.}
\end{itemize}
\begin{itemize}
\item {Utilização:Carp.}
\end{itemize}
\begin{itemize}
\item {Proveniência:(De \textunderscore barco\textunderscore )}
\end{itemize}
Pôr ou meter numa embarcação: \textunderscore embarcar pipas de vinho\textunderscore .
Empurrar.
Entrar numa embarcação, para viajar ou recrear-se.
Entrar num combóio, para se transportar de um a outro lugar.
Collocar sobre o andaime (uma viga), para se collocar depois no lugar do costume.
\section{Embargador}
\begin{itemize}
\item {Grp. gram.:adj.}
\end{itemize}
\begin{itemize}
\item {Grp. gram.:M.}
\end{itemize}
Que embarga.
Aquelle que embarga.
\section{Embargamento}
\begin{itemize}
\item {Grp. gram.:m.}
\end{itemize}
Acto de embargar.
\section{Embargante}
\begin{itemize}
\item {Grp. gram.:m., f.  e  adj.}
\end{itemize}
Pessôa que embarga.
\section{Embargar}
\begin{itemize}
\item {Grp. gram.:v. t.}
\end{itemize}
\begin{itemize}
\item {Proveniência:(De um freq. hypoth. \textunderscore embarricar\textunderscore , do rad. de \textunderscore barra\textunderscore )}
\end{itemize}
Pôr embargo a: \textunderscore embargar uma decisão judicial\textunderscore .
Tolher.
Impedir: \textunderscore embargar a passagem\textunderscore .
Embaraçar o uso de.
Reprimir.
\section{Embargável}
\begin{itemize}
\item {Grp. gram.:adj.}
\end{itemize}
Que se póde embargar.
\section{Embargo}
\begin{itemize}
\item {Grp. gram.:m.}
\end{itemize}
\begin{itemize}
\item {Grp. gram.:Loc. adv.}
\end{itemize}
\begin{itemize}
\item {Proveniência:(De \textunderscore embargar\textunderscore )}
\end{itemize}
Obstáculo, estôrvo.
Meio judicial, com que se procura obstar ao cumprimento de uma sentença ou despacho.
Sequestro.
\textunderscore Sem embargo\textunderscore , não obstante.
\section{Embarque}
\begin{itemize}
\item {Grp. gram.:m.}
\end{itemize}
Acto de embarcar.
Lugar, onde se embarca.
\section{Embarrada}
\begin{itemize}
\item {Grp. gram.:adj. f.}
\end{itemize}
\begin{itemize}
\item {Utilização:Prov.}
\end{itemize}
\begin{itemize}
\item {Utilização:minh.}
\end{itemize}
Diz-se da mulher grávida.
(Cp. \textunderscore embarrar\textunderscore ^2)
\section{Embarrado}
\begin{itemize}
\item {Grp. gram.:m.}
\end{itemize}
\begin{itemize}
\item {Utilização:Prov.}
\end{itemize}
\begin{itemize}
\item {Utilização:alent.}
\end{itemize}
\begin{itemize}
\item {Proveniência:(De \textunderscore embarrar\textunderscore ^2)}
\end{itemize}
Videira, que se planta junto de uma árvore para trepar por ella.
\section{Embarrador}
\begin{itemize}
\item {Grp. gram.:m.}
\end{itemize}
Aquelle que embarra.
\section{Embarrancar}
\begin{itemize}
\item {Grp. gram.:v. t.}
\end{itemize}
\begin{itemize}
\item {Grp. gram.:V. i.}
\end{itemize}
\begin{itemize}
\item {Proveniência:(De \textunderscore barranco\textunderscore )}
\end{itemize}
Fazer caír em barranco.
Embaraçar.
Ficar atalhado.
Esbarrar.
\section{Embarrar}
\begin{itemize}
\item {Grp. gram.:v. t.}
\end{itemize}
\begin{itemize}
\item {Proveniência:(De \textunderscore barro\textunderscore )}
\end{itemize}
Cobrir com barro, rebocar.
\section{Embarrar}
\begin{itemize}
\item {Grp. gram.:v. i.}
\end{itemize}
\begin{itemize}
\item {Utilização:Prov.}
\end{itemize}
\begin{itemize}
\item {Utilização:trasm.}
\end{itemize}
\begin{itemize}
\item {Grp. gram.:V. t.}
\end{itemize}
\begin{itemize}
\item {Utilização:Prov.}
\end{itemize}
\begin{itemize}
\item {Utilização:dur.}
\end{itemize}
\begin{itemize}
\item {Proveniência:(De \textunderscore barra\textunderscore )}
\end{itemize}
Esbarrar, topar.
Bater contra alguma coisa, embaraçar-se nella.
O mesmo que \textunderscore dependurar\textunderscore .
Tocar de leve; roçar.
\section{Embarrear}
\begin{itemize}
\item {Grp. gram.:v. t.}
\end{itemize}
\begin{itemize}
\item {Utilização:Prov.}
\end{itemize}
O mesmo que \textunderscore embarrar\textunderscore ^1.
(Colhido em Turquel)
\section{Embarreirar}
\begin{itemize}
\item {Grp. gram.:v. i.  e  p.}
\end{itemize}
Subir a uma barreira.
Trepar.
\section{Embarrelar}
\begin{itemize}
\item {Grp. gram.:v. t.}
\end{itemize}
Dar barrela a.
Pôr em barrela.
\section{Embarretado}
\begin{itemize}
\item {Grp. gram.:adj.}
\end{itemize}
Que tem barrete na cabeça. Cf. Filinto, V, 263.
\section{Embarricar}
\begin{itemize}
\item {Grp. gram.:v. t.}
\end{itemize}
\begin{itemize}
\item {Utilização:des.}
\end{itemize}
\begin{itemize}
\item {Utilização:Fam.}
\end{itemize}
\begin{itemize}
\item {Proveniência:(De \textunderscore barrica\textunderscore )}
\end{itemize}
Meter em barrica.
Pôr barricadas em.
Enganar, embarrilar.
\section{Embarrigar}
\begin{itemize}
\item {Grp. gram.:v. i.}
\end{itemize}
\begin{itemize}
\item {Utilização:Bras. do S}
\end{itemize}
Tornar-se barrigudo ou pançudo (o cavallo).
\section{Embarrilagem}
\begin{itemize}
\item {Grp. gram.:f.}
\end{itemize}
Acto de embarrilar.
\section{Embarrilar}
\begin{itemize}
\item {Grp. gram.:v. t.}
\end{itemize}
\begin{itemize}
\item {Utilização:Pop.}
\end{itemize}
Meter em barril.
Embaraçar, enganar.
\section{Embarrilho}
\begin{itemize}
\item {Grp. gram.:m.}
\end{itemize}
\begin{itemize}
\item {Utilização:Prov.}
\end{itemize}
\begin{itemize}
\item {Utilização:minh.}
\end{itemize}
\begin{itemize}
\item {Proveniência:(De \textunderscore embarrar\textunderscore ^2)}
\end{itemize}
Embaraço, estôrvo.
Pessôa, que embaraça outras.
\section{Embarruada}
\begin{itemize}
\item {Grp. gram.:f.}
\end{itemize}
\begin{itemize}
\item {Utilização:Bras. de Minas}
\end{itemize}
Encontro.
Luta. (Cp. \textunderscore embarrar\textunderscore ^2)
\section{Embarulhar}
\begin{itemize}
\item {Grp. gram.:v. t.}
\end{itemize}
Produzir barulho em.
Desordenar, confundir. Cf. Arn. Gama, \textunderscore Motim\textunderscore , 324.
\section{Embasbacador}
\begin{itemize}
\item {Grp. gram.:adj.}
\end{itemize}
Que embasbaca.
Que causa espanto.
\section{Embasbacamento}
\begin{itemize}
\item {Grp. gram.:m.}
\end{itemize}
Acto ou effeito de embasbacar.
\section{Embasbacar}
\begin{itemize}
\item {Grp. gram.:v. i.}
\end{itemize}
\begin{itemize}
\item {Grp. gram.:V. t.}
\end{itemize}
\begin{itemize}
\item {Proveniência:(De \textunderscore basbaque\textunderscore )}
\end{itemize}
Pasmar; mostrar-se basbaque.
Admirar-se ingenuamente.
Tornar basbaque, causar espanto a.
\section{Embastar}
\begin{itemize}
\item {Grp. gram.:v. t.}
\end{itemize}
Segurar com bastas.
Acolchoar.
\section{Embastecer}
\begin{itemize}
\item {Grp. gram.:v. t.}
\end{itemize}
\begin{itemize}
\item {Proveniência:(De \textunderscore basto\textunderscore )}
\end{itemize}
Tornar basto, espêsso, grosso:«\textunderscore ...embasteceu-se novamente a escuridade\textunderscore ». Camillo, \textunderscore Caveira\textunderscore , 205.
\section{Embate}
\begin{itemize}
\item {Grp. gram.:m.}
\end{itemize}
Acto de embater.
Encontro.
Pancada recíproca de dois corpos que se encontram.
Percussão violenta.
Acommettimento impetuoso.
\section{Embater}
\begin{itemize}
\item {Grp. gram.:v. i.}
\end{itemize}
\begin{itemize}
\item {Proveniência:(De \textunderscore bater\textunderscore )}
\end{itemize}
Esbarrar.
Produzir choque, encontrando-se; arrojar-se.
\section{Embatocar}
\begin{itemize}
\item {Grp. gram.:v. t.}
\end{itemize}
\begin{itemize}
\item {Proveniência:(De \textunderscore batoque\textunderscore )}
\end{itemize}
O mesmo ou melhor que \textunderscore embatucar\textunderscore .
\section{Embatucar}
\begin{itemize}
\item {Grp. gram.:v. t.}
\end{itemize}
\begin{itemize}
\item {Utilização:Pop.}
\end{itemize}
Pôr batoque em.
Fazer calar.
Surprehender desagradavelmente.
Enlear.
(Corr. de \textunderscore embatocar\textunderscore )
\section{Embatumar}
\begin{itemize}
\item {Grp. gram.:v. t.}
\end{itemize}
\begin{itemize}
\item {Utilização:Bras}
\end{itemize}
\begin{itemize}
\item {Utilização:Pop.}
\end{itemize}
Encher de mais; accumular.
(Por \textunderscore embetumar\textunderscore , de \textunderscore betume\textunderscore )
\section{Embaucador}
\begin{itemize}
\item {fónica:ba-u}
\end{itemize}
\begin{itemize}
\item {Grp. gram.:m.  e  adj.}
\end{itemize}
Aquelle que embaúca.
\section{Embaucar}
\begin{itemize}
\item {fónica:ba-u}
\end{itemize}
\begin{itemize}
\item {Grp. gram.:v. t.}
\end{itemize}
Illudir.
Embair.
Attrahir, com artifício.
(Por \textunderscore embaiucar\textunderscore , de \textunderscore baiúca\textunderscore ? Ou por \textunderscore embiocar\textunderscore , de \textunderscore biôco\textunderscore ?)
\section{Embaular}
\begin{itemize}
\item {fónica:ba-u}
\end{itemize}
\begin{itemize}
\item {Grp. gram.:v. t.}
\end{itemize}
\begin{itemize}
\item {Proveniência:(De \textunderscore bahul\textunderscore )}
\end{itemize}
Meter em bau.
Esconder.
Guardar.
\section{Embeaxió}
\begin{itemize}
\item {Grp. gram.:m.}
\end{itemize}
\begin{itemize}
\item {Utilização:Bras. do N}
\end{itemize}
Gaita de taboca.
\section{Embebedar}
\begin{itemize}
\item {Grp. gram.:v. t.}
\end{itemize}
\begin{itemize}
\item {Utilização:Fig.}
\end{itemize}
Tornar bêbedo.
Embriagar.
Alucinar, perturbar.
\section{Embeber}
\begin{itemize}
\item {Grp. gram.:v. t.}
\end{itemize}
\begin{itemize}
\item {Proveniência:(Lat. \textunderscore embibere\textunderscore )}
\end{itemize}
Ensopar.
Sorver.
Encaixar, cravar.
Introduzir.
Insinuar, infiltrar.
\section{Embeberar}
\begin{itemize}
\item {Grp. gram.:v. t.}
\end{itemize}
\begin{itemize}
\item {Utilização:Fig.}
\end{itemize}
O mesmo que \textunderscore abeberar\textunderscore .
Impregnar; saturar.
\section{Embebição}
\begin{itemize}
\item {Grp. gram.:f.}
\end{itemize}
Acto de embeber.
\section{Embebidamente}
\begin{itemize}
\item {Grp. gram.:adv.}
\end{itemize}
\begin{itemize}
\item {Proveniência:(De \textunderscore embebido\textunderscore )}
\end{itemize}
Com embebição.
\section{Embebido}
\begin{itemize}
\item {Grp. gram.:adj.}
\end{itemize}
\begin{itemize}
\item {Proveniência:(De \textunderscore embeber\textunderscore )}
\end{itemize}
Ensopado.
Cravado.
Introduzido.
\section{Embeguacá}
\begin{itemize}
\item {Grp. gram.:f.}
\end{itemize}
Planta brasileira.
\section{Embeiçar}
\begin{itemize}
\item {Grp. gram.:v. t.}
\end{itemize}
\begin{itemize}
\item {Utilização:Pop.}
\end{itemize}
\begin{itemize}
\item {Proveniência:(De \textunderscore beiço\textunderscore )}
\end{itemize}
Prender pelo beiço, cativar.
Enlevar.
\section{Embelecador}
\begin{itemize}
\item {Grp. gram.:m.  e  adj.}
\end{itemize}
O que embeleca.
\section{Embelecar}
\begin{itemize}
\item {Grp. gram.:v. t.}
\end{itemize}
\begin{itemize}
\item {Proveniência:(Do lat. \textunderscore implicare\textunderscore ?)}
\end{itemize}
Enganar com artifícios.
Cativar.
Embair:«\textunderscore é com este cascabulho de palavrório que os poetas embellecam os incautos espíritos das crianças\textunderscore ». Camillo, \textunderscore Caveira\textunderscore , 294.
\section{Embelecer}
\begin{itemize}
\item {Grp. gram.:v. t.}
\end{itemize}
Tornar belo.
Pôr enfeites em.
Fazer formoso.
\section{Embeleco}
\begin{itemize}
\item {Grp. gram.:m.}
\end{itemize}
\begin{itemize}
\item {Utilização:Des.}
\end{itemize}
Acto ou effeito de embelecar.
O mesmo que \textunderscore empecilho\textunderscore , (ainda us. na Baía).
\section{Embelezamento}
\begin{itemize}
\item {Grp. gram.:m.}
\end{itemize}
Acto ou efeito de embelezar.
\section{Embelga}
\begin{itemize}
\item {Grp. gram.:f.}
\end{itemize}
\begin{itemize}
\item {Utilização:Prov.}
\end{itemize}
O mesmo que \textunderscore belga\textunderscore ^1.
\section{Embelgador}
\begin{itemize}
\item {Grp. gram.:m.  e  adj.}
\end{itemize}
Aquelle que sabe embelgar.
\section{Embelgar}
\begin{itemize}
\item {Grp. gram.:v. t.}
\end{itemize}
\begin{itemize}
\item {Utilização:Prov.}
\end{itemize}
\begin{itemize}
\item {Utilização:alent.}
\end{itemize}
Dividir em belgas.
Dividir em secções, por meio de regos parallelos, (um terreno), antes de lavrado.
\section{Embelinhar}
\begin{itemize}
\item {Grp. gram.:v. t.}
\end{itemize}
Embaraçar, empecer:«\textunderscore quando atirava com a barrica á água, embelinhou-se nella e foi p'ra deante\textunderscore ». Camillo, \textunderscore Myst. de Lisboa\textunderscore , I, 221.
\section{Embellecer}
\begin{itemize}
\item {Grp. gram.:v. t.}
\end{itemize}
Tornar bello.
Pôr enfeites em.
Fazer formoso.
\section{Embellezamento}
\begin{itemize}
\item {Grp. gram.:m.}
\end{itemize}
Acto ou effeito de embellezar.
\section{Embellezar}
\begin{itemize}
\item {Grp. gram.:v. t.}
\end{itemize}
\begin{itemize}
\item {Utilização:Gal}
\end{itemize}
\begin{itemize}
\item {Proveniência:(De \textunderscore belleza\textunderscore )}
\end{itemize}
Encantar.
Arrebatar os sentidos de.
Cativar.
Tornar bello, aformosear.
\section{Embellezo}
\begin{itemize}
\item {fónica:lê}
\end{itemize}
\begin{itemize}
\item {Grp. gram.:m.}
\end{itemize}
\begin{itemize}
\item {Utilização:Des.}
\end{itemize}
O mesmo que \textunderscore embellezamento\textunderscore .
\section{Emberiza}
\begin{itemize}
\item {Grp. gram.:f.}
\end{itemize}
\begin{itemize}
\item {Proveniência:(Do al. \textunderscore emmeriz\textunderscore )}
\end{itemize}
Designação scientífica do verdelhão.
\section{Embesoirado}
\begin{itemize}
\item {Grp. gram.:adj.}
\end{itemize}
Amuado; carrancudo.
Todo cheio de si. Cf. Rebello, \textunderscore Contos e Lendas\textunderscore , 59.
\section{Embesourado}
\begin{itemize}
\item {Grp. gram.:adj.}
\end{itemize}
Amuado; carrancudo.
Todo cheio de si. Cf. Rebello, \textunderscore Contos e Lendas\textunderscore , 59.
\section{Embespinhar}
\begin{itemize}
\item {Grp. gram.:v. t.  e  p.}
\end{itemize}
O mesmo que \textunderscore abespinhar-se\textunderscore .
\section{Embèstado}
\begin{itemize}
\item {Grp. gram.:adj.}
\end{itemize}
\begin{itemize}
\item {Utilização:Ant.}
\end{itemize}
\begin{itemize}
\item {Proveniência:(De \textunderscore bésta\textunderscore )}
\end{itemize}
Armado com bésta.
Disposto para a peleja.
\section{Embetesgar}
\begin{itemize}
\item {Grp. gram.:v. t.}
\end{itemize}
\begin{itemize}
\item {Grp. gram.:V. i.}
\end{itemize}
Meter em betesga.
Meter-se em lugar embaraçado, ou lugar sem saída. Cf. Barros, \textunderscore Déc.\textunderscore  II, 81.
\section{Embetumar}
\begin{itemize}
\item {Grp. gram.:v. t.}
\end{itemize}
(V.betumar)
\section{Embevecer}
\begin{itemize}
\item {Grp. gram.:v. t.}
\end{itemize}
Enlevar.
Embelecar.
Cativar.
Causar transporte a.
(Talvez do rad. de \textunderscore beber\textunderscore  ou corr. de um hypoth. \textunderscore embebedecer\textunderscore , de \textunderscore bêbedo\textunderscore )
\section{Embevecimento}
\begin{itemize}
\item {Grp. gram.:m.}
\end{itemize}
Acto ou effeito de embevecer.
\section{Embezerrar}
\begin{itemize}
\item {Grp. gram.:v. i.  e  p.}
\end{itemize}
\begin{itemize}
\item {Utilização:Pop.}
\end{itemize}
\begin{itemize}
\item {Utilização:Prov.}
\end{itemize}
\begin{itemize}
\item {Utilização:alent.}
\end{itemize}
\begin{itemize}
\item {Proveniência:(De \textunderscore bezerro\textunderscore )}
\end{itemize}
Têr amuo.
Teimar.
Pôr-se carrancudo.
Apresentar o rosto afogueado, por excesso de trabalho ou brincadeira.
\section{Embiaiendo}
\begin{itemize}
\item {Grp. gram.:m.}
\end{itemize}
O mesmo que \textunderscore tipi\textunderscore .
\section{Embiara}
\begin{itemize}
\item {Grp. gram.:f.}
\end{itemize}
\begin{itemize}
\item {Utilização:Bras. do N}
\end{itemize}
Qualquer presa; aquillo que se colheu na caça, na pesca ou na guerra.
(Do tupi \textunderscore mbiara\textunderscore )
\section{Embicadeiro}
\begin{itemize}
\item {Grp. gram.:adj.}
\end{itemize}
O mesmo que \textunderscore embicador\textunderscore .
\section{Embicador}
\begin{itemize}
\item {Grp. gram.:adj.}
\end{itemize}
\begin{itemize}
\item {Grp. gram.:M.}
\end{itemize}
Que embica.
Aquelle que embica.
\section{Embicadura}
\begin{itemize}
\item {Grp. gram.:f.}
\end{itemize}
\begin{itemize}
\item {Proveniência:(De \textunderscore embicar\textunderscore )}
\end{itemize}
Aproximação de um navio á amarra que está a pique.
\section{Embicar}
\begin{itemize}
\item {Grp. gram.:v. t.}
\end{itemize}
\begin{itemize}
\item {Grp. gram.:V. i.}
\end{itemize}
\begin{itemize}
\item {Grp. gram.:V. p.}
\end{itemize}
\begin{itemize}
\item {Utilização:Des.}
\end{itemize}
\begin{itemize}
\item {Proveniência:(De \textunderscore bico\textunderscore )}
\end{itemize}
Tornar bicudo.
Esbarrar.
Fazer reparo.
Aproximar-se o navio á amarra que está a pique.
Abicar.
Encalhar.
Dirigir-se.
\section{Embigada}
\begin{itemize}
\item {Grp. gram.:f.}
\end{itemize}
\begin{itemize}
\item {Utilização:Pop.}
\end{itemize}
\begin{itemize}
\item {Proveniência:(De \textunderscore embigo\textunderscore )}
\end{itemize}
Embate de umbigo contra umbigo.
\section{Embigo}
\begin{itemize}
\item {Grp. gram.:m.}
\end{itemize}
(Fórma pop. de \textunderscore umbigo\textunderscore . Cf. Castilho, \textunderscore Fastos\textunderscore , I, 319 e 320)
\section{Embilhar}
\begin{itemize}
\item {Grp. gram.:v. t.}
\end{itemize}
\begin{itemize}
\item {Utilização:Prov.}
\end{itemize}
\begin{itemize}
\item {Utilização:trasm.}
\end{itemize}
\begin{itemize}
\item {Grp. gram.:V. i.}
\end{itemize}
Hesitar em resolver.
Contemporizar.
Procrastinar por acanhamento.
Intrometer-se; fazer provocações; dar encontrões, brigar.
\section{Embiocar}
\begin{itemize}
\item {Grp. gram.:v. t.}
\end{itemize}
\begin{itemize}
\item {Grp. gram.:V. i.}
\end{itemize}
\begin{itemize}
\item {Utilização:Bras. do N}
\end{itemize}
\begin{itemize}
\item {Grp. gram.:V. p.}
\end{itemize}
Tapar com bioco.
Encobrir com capa, chale, etc.
Encolher-se.
Retrahir-se.
Cobrir o rosto.
Mostrar-se santarrão.
Affectar qualidades ou virtudes que não possue.
\section{Embira}
\begin{itemize}
\item {Grp. gram.:f.}
\end{itemize}
\begin{itemize}
\item {Utilização:Bras}
\end{itemize}
\begin{itemize}
\item {Proveniência:(Do guar. \textunderscore mbir\textunderscore )}
\end{itemize}
Nome de várias plantas do Brazil.
Corda de cipó ou de casca de árvore.
Fibra vegetal.
\section{Embiraçu}
\begin{itemize}
\item {Grp. gram.:m.}
\end{itemize}
\begin{itemize}
\item {Utilização:Bras}
\end{itemize}
Espécie de embira, de cujo fruto se extrai uma lanugem, com que se enchem almofadas, colxões, etc.
\section{Embirar}
\begin{itemize}
\item {Grp. gram.:v. t.}
\end{itemize}
\begin{itemize}
\item {Utilização:Bras. do N}
\end{itemize}
Atar ou ligar com embira.
\section{Embiratanha}
\begin{itemize}
\item {Grp. gram.:f.}
\end{itemize}
Árvore brasileira, de cuja casca filamentosa se fazem cordas.
\section{Embiri}
\begin{itemize}
\item {Grp. gram.:m.}
\end{itemize}
Planta medicinal do Brasil.
\section{Embiricica}
\begin{itemize}
\item {Grp. gram.:f.}
\end{itemize}
\begin{itemize}
\item {Utilização:Bras. do N}
\end{itemize}
Porção de coisas, dispostas em fileira.
Série.
\section{Embirra}
\begin{itemize}
\item {Grp. gram.:f.}
\end{itemize}
O mesmo que \textunderscore embirração\textunderscore .
\section{Embirração}
\begin{itemize}
\item {Grp. gram.:f.}
\end{itemize}
Acto de embirrar.
Teima; zanga.
Mania.
\section{Embirrante}
\begin{itemize}
\item {Grp. gram.:adj.}
\end{itemize}
\begin{itemize}
\item {Proveniência:(De \textunderscore embirrar\textunderscore )}
\end{itemize}
Que embirra.
Teimoso.
Que causa embirração.
\section{Embirrar}
\begin{itemize}
\item {Grp. gram.:v. i.}
\end{itemize}
\begin{itemize}
\item {Utilização:Pop.}
\end{itemize}
\begin{itemize}
\item {Proveniência:(De \textunderscore birra\textunderscore )}
\end{itemize}
Teimar pertinazmente, com aborrecimento, com enfado: \textunderscore embirrar numa ideia\textunderscore .
Têr aversão; antipathizar: \textunderscore embirrar com alguém\textunderscore .
Encostar com fôrça: \textunderscore embirrar os pés á parede\textunderscore .
(Figuradamente, esta expressão significa \textunderscore sêr teimoso\textunderscore ).
\section{Embirrativo}
\begin{itemize}
\item {Grp. gram.:adj.}
\end{itemize}
\begin{itemize}
\item {Proveniência:(De \textunderscore embirrar\textunderscore )}
\end{itemize}
Antipáthico.
Que produz embirração; birrento.
\section{Embirrento}
\begin{itemize}
\item {Grp. gram.:adj.}
\end{itemize}
\begin{itemize}
\item {Proveniência:(De \textunderscore embirrar\textunderscore )}
\end{itemize}
Antipáthico.
Que produz embirração; birrento.
\section{Embirruçu}
\begin{itemize}
\item {Grp. gram.:m.}
\end{itemize}
Planta brasileira, resinosa.
(Não será palavra incorrecta, por \textunderscore embiraçu\textunderscore ?)
\section{Embitesgar}
\begin{itemize}
\item {Grp. gram.:v. t.}
\end{itemize}
\begin{itemize}
\item {Grp. gram.:V. i.}
\end{itemize}
Meter em bitesga.
Meter-se em lugar embaraçado, ou lugar sem saída. Cf. Barros, \textunderscore Déc.\textunderscore  II, 81.
\section{Emblema}
\begin{itemize}
\item {Grp. gram.:m.}
\end{itemize}
\begin{itemize}
\item {Proveniência:(Gr. \textunderscore emblema\textunderscore )}
\end{itemize}
Divisa; insígnia.
Allegoria.
\section{Emblemar}
\begin{itemize}
\item {Grp. gram.:v. t.}
\end{itemize}
Mostrar por meio de emblema.
\section{Emblematicamente}
\begin{itemize}
\item {Grp. gram.:adv.}
\end{itemize}
De modo emblemático.
\section{Emblemático}
\begin{itemize}
\item {Grp. gram.:adj.}
\end{itemize}
Relativo a emblema.
\section{Emblica}
\begin{itemize}
\item {Grp. gram.:f.}
\end{itemize}
Planta euphorbiácea medicinal, (\textunderscore emblica officinalis\textunderscore , Lin.).
\section{Emblico}
\begin{itemize}
\item {Grp. gram.:m.}
\end{itemize}
O mesmo que \textunderscore emblica\textunderscore .
\section{Emboaba}
\begin{itemize}
\item {Grp. gram.:f.}
\end{itemize}
\begin{itemize}
\item {Utilização:Bras}
\end{itemize}
\begin{itemize}
\item {Grp. gram.:M.}
\end{itemize}
\begin{itemize}
\item {Utilização:deprec.}
\end{itemize}
\begin{itemize}
\item {Utilização:Fig.}
\end{itemize}
Ave, que tem pennas até aos dedos.
Indivíduo português.
O mesmo que \textunderscore boava\textunderscore .
\section{Embobar}
\begin{itemize}
\item {Grp. gram.:v. t.}
\end{itemize}
\begin{itemize}
\item {Utilização:Des.}
\end{itemize}
Converter em bobo.
Tornar aparvalhado ou ingênuo. Cf. Arn. Gama, \textunderscore Motim\textunderscore , 296, 466 e 471; Ult. Dona, 118 e 381.
\section{Emboca-bola}
\begin{itemize}
\item {Grp. gram.:m.}
\end{itemize}
Jôgo de rapazes, formado de uma bola, presa a uma haste, ponteaguda de um lado e côncava do outro.
\section{Emboçador}
\begin{itemize}
\item {Grp. gram.:m.}
\end{itemize}
Aquelle que emboça.
Operário que reboca.
Pedreiro.
\section{Embocadura}
\begin{itemize}
\item {Grp. gram.:f.}
\end{itemize}
\begin{itemize}
\item {Utilização:Fig.}
\end{itemize}
\begin{itemize}
\item {Proveniência:(De \textunderscore embocar\textunderscore )}
\end{itemize}
A parte de certos instrumentos, que se introduz na bôca, para êlles se tocarem.
Parte do freio, que entra na bôca da bêsta.
Maneira de embocar instrumentos de sopro.
Foz ou bôca de um rio.
Entrada de uma rua.
Tendência: \textunderscore tem embocadura para a estroinice\textunderscore .
\section{Emboçalar}
\begin{itemize}
\item {Grp. gram.:v. t.}
\end{itemize}
\begin{itemize}
\item {Utilização:Bras. do S}
\end{itemize}
\begin{itemize}
\item {Proveniência:(De \textunderscore boçal\textunderscore ^2)}
\end{itemize}
Pôr boçal a.
Enganar.
\section{Emboçamento}
\begin{itemize}
\item {Grp. gram.:m.}
\end{itemize}
Acto de emboçar.
\section{Embocar}
\begin{itemize}
\item {Grp. gram.:v. t.}
\end{itemize}
\begin{itemize}
\item {Proveniência:(De \textunderscore bôca\textunderscore )}
\end{itemize}
Pôr na bôca (instrumento de sôpro), para o tocar.
Chegar á bôca.
Entornar, bebendo.
Fazer entrar.
Pôr o freio a.
Entrar na foz de.
\section{Emboçar}
\begin{itemize}
\item {Grp. gram.:v. t.}
\end{itemize}
Pôr embôço em (paredes).
\section{Emboccar}
\textunderscore v. t.\textunderscore  (e der.)
O mesmo que \textunderscore embocar\textunderscore , etc.
\section{Embocetamento}
\begin{itemize}
\item {Grp. gram.:m.}
\end{itemize}
Acto ou effeito de embocetar.
\section{Embocetar}
\begin{itemize}
\item {Grp. gram.:v. t.}
\end{itemize}
Meter em boceta.
Fechar, á maneira de boceta. Cf. Camillo, \textunderscore Cav. em Ruínas\textunderscore , 107.
\section{Embocha}
\begin{itemize}
\item {Grp. gram.:f.}
\end{itemize}
\begin{itemize}
\item {Utilização:Prov.}
\end{itemize}
\begin{itemize}
\item {Utilização:trasm.}
\end{itemize}
O mesmo que \textunderscore bocha\textunderscore .
Bolha nos pés, produzida pelo calçado.
(Colhido em Chaves)
\section{Embochechar}
\begin{itemize}
\item {Grp. gram.:v. t.}
\end{itemize}
Tornar bochechudo. Cf. Filinto, III, 46; VI, 30.
\section{Embôço}
\begin{itemize}
\item {Grp. gram.:m.}
\end{itemize}
\begin{itemize}
\item {Utilização:Ant.}
\end{itemize}
Primeira camada de argamassa ou de cal, que se põe na parede.
Pedreiro.
(Cast. \textunderscore embozo\textunderscore )
\section{Embodalhar}
\begin{itemize}
\item {Grp. gram.:v. t.}
\end{itemize}
\begin{itemize}
\item {Utilização:Prov.}
\end{itemize}
\begin{itemize}
\item {Utilização:beir.}
\end{itemize}
\begin{itemize}
\item {Proveniência:(De \textunderscore bodalho\textunderscore )}
\end{itemize}
Tornar sujo; emporcalhar; embodegar.
\section{Embodegar}
\begin{itemize}
\item {Grp. gram.:v. t.}
\end{itemize}
\begin{itemize}
\item {Proveniência:(De \textunderscore bodega\textunderscore )}
\end{itemize}
Sujar.
Emporcalhar.
Tornar immundo.
\section{Embófia}
\begin{itemize}
\item {Grp. gram.:f.}
\end{itemize}
\begin{itemize}
\item {Grp. gram.:M.}
\end{itemize}
Empáfia.
Impostura.
Logração, patranha.
Pessôa presumida, vaidosa, orgulhosa.
\section{Emboitar}
\begin{itemize}
\item {Grp. gram.:v. t.}
\end{itemize}
\begin{itemize}
\item {Utilização:Prov.}
\end{itemize}
Sujar.
(Talvez do it. \textunderscore embiutare\textunderscore )
\section{Embolação}
\begin{itemize}
\item {Grp. gram.:f.}
\end{itemize}
Acto de embolar.
\section{Embolada}
\begin{itemize}
\item {Grp. gram.:f.}
\end{itemize}
\begin{itemize}
\item {Utilização:Ant.}
\end{itemize}
\begin{itemize}
\item {Proveniência:(Do rad. de \textunderscore bóla\textunderscore )}
\end{itemize}
Fatuidade.
\section{Embolado}
\begin{itemize}
\item {Grp. gram.:adj.}
\end{itemize}
\begin{itemize}
\item {Utilização:Prov.}
\end{itemize}
\begin{itemize}
\item {Utilização:trasm.}
\end{itemize}
\begin{itemize}
\item {Proveniência:(De \textunderscore embolar\textunderscore ^1)}
\end{itemize}
Diz-se do toiro, a que se embolaram os cornos.
Embostado, envolvido em trampa. (Colhido em Lagoaça)
Diz-se das couves, cujas fôlhas são enroladas, como as do repolho.
\section{Embolar}
\begin{itemize}
\item {Grp. gram.:v. t.}
\end{itemize}
\begin{itemize}
\item {Grp. gram.:V. p.}
\end{itemize}
\begin{itemize}
\item {Utilização:Prov.}
\end{itemize}
\begin{itemize}
\item {Utilização:trasm.}
\end{itemize}
\begin{itemize}
\item {Proveniência:(De \textunderscore bóla\textunderscore )}
\end{itemize}
Pôr bólas em (cornos do toiro que se há de correr).
Pôr bólas nos cornos de (um toiro).
Embostar-se, atolar-se em trampa.
\section{Embolar}
\begin{itemize}
\item {Grp. gram.:v. t.}
\end{itemize}
Reduzir a bôlo, por meio da fusão, (o oiro em pó). Cf. F. de Mendonça, \textunderscore Vocab. Techn.\textunderscore 
\section{Emboldrear}
\begin{itemize}
\item {Grp. gram.:v. t.}
\end{itemize}
\begin{itemize}
\item {Utilização:Pop.}
\end{itemize}
O mesmo que \textunderscore embodegar\textunderscore .
\section{Emboldregar-se}
\begin{itemize}
\item {Grp. gram.:v. p.}
\end{itemize}
\begin{itemize}
\item {Utilização:Prov.}
\end{itemize}
\begin{itemize}
\item {Utilização:trasm.}
\end{itemize}
Sujar-se.
(Corr. de \textunderscore embodegar-se\textunderscore )
\section{Embolha}
\begin{itemize}
\item {fónica:bô}
\end{itemize}
\begin{itemize}
\item {Grp. gram.:f.}
\end{itemize}
\begin{itemize}
\item {Utilização:Ant.}
\end{itemize}
\begin{itemize}
\item {Proveniência:(Do lat. \textunderscore ampulla\textunderscore )}
\end{itemize}
Vasilha de coiro para vinho, de maiores dimensões que as dos odres.
\section{Embolia}
\begin{itemize}
\item {Grp. gram.:f.}
\end{itemize}
\begin{itemize}
\item {Proveniência:(Do gr. \textunderscore embolion\textunderscore )}
\end{itemize}
Obstrucção, produzida por coágulos fibrinosos, que, formados numa artéria, vão obliterar artéria mais pequena.
\section{Emboligar-se}
\begin{itemize}
\item {Grp. gram.:v. p.}
\end{itemize}
\begin{itemize}
\item {Utilização:Prov.}
\end{itemize}
\begin{itemize}
\item {Utilização:trasm.}
\end{itemize}
Revolver-se no pó; espojar-se.
(Cp. \textunderscore emboldregar-se\textunderscore )
\section{Embólio}
\begin{itemize}
\item {Grp. gram.:m.}
\end{itemize}
Nome de várias aves da África occidental.
\section{Embolismal}
\begin{itemize}
\item {Grp. gram.:adj.}
\end{itemize}
\begin{itemize}
\item {Proveniência:(De \textunderscore embolismo\textunderscore )}
\end{itemize}
Diz-se do anno, que tem treze lunações.
Intercalar.
\section{Embolísmico}
\begin{itemize}
\item {Grp. gram.:adj.}
\end{itemize}
(V.embolismal)
\section{Embolismo}
\begin{itemize}
\item {Grp. gram.:m.}
\end{itemize}
\begin{itemize}
\item {Proveniência:(Gr. \textunderscore embolismos\textunderscore )}
\end{itemize}
Accrescentamento de dias ou meses ao anno lunar, para o ajustar com o anno solar.
\section{Êmbolo}
\begin{itemize}
\item {Grp. gram.:m.}
\end{itemize}
\begin{itemize}
\item {Proveniência:(Gr. \textunderscore embolos\textunderscore )}
\end{itemize}
Disco ou cylindro móvel das seringas, bombas, e de outros maquinismos.
\section{Emboloirar}
\begin{itemize}
\item {Grp. gram.:v. t.}
\end{itemize}
\begin{itemize}
\item {Utilização:Prov.}
\end{itemize}
Converter em bolo: \textunderscore o pão de centeio emboloira-se-me na bôca e não vai para baixo\textunderscore .
\section{Embolorecimento}
\begin{itemize}
\item {Grp. gram.:m.}
\end{itemize}
Acto de embolorecer.
\section{Embolorecer}
\begin{itemize}
\item {Grp. gram.:v. i.}
\end{itemize}
Criar bolor, o mesmo que \textunderscore abolorecer\textunderscore .
\section{Embolotar}
\begin{itemize}
\item {Grp. gram.:v. i.}
\end{itemize}
\begin{itemize}
\item {Utilização:Prov.}
\end{itemize}
\begin{itemize}
\item {Proveniência:(De \textunderscore bolota\textunderscore )}
\end{itemize}
Enlamear.
\section{Embolsar}
\begin{itemize}
\item {Grp. gram.:v. t.}
\end{itemize}
\begin{itemize}
\item {Proveniência:(De \textunderscore bolsa\textunderscore )}
\end{itemize}
Meter na bolsa: \textunderscore embolsar dinheiro\textunderscore .
Pagar o que se deve a: \textunderscore embolsar um credor\textunderscore .
\section{Embôlso}
\begin{itemize}
\item {Grp. gram.:m.}
\end{itemize}
Acto ou effeito de embolsar.
\section{Embonada}
\begin{itemize}
\item {Grp. gram.:f.}
\end{itemize}
Acto ou effeito de embonar.
\section{Embonar}
\begin{itemize}
\item {Grp. gram.:v. t.}
\end{itemize}
\begin{itemize}
\item {Proveniência:(De \textunderscore embono\textunderscore )}
\end{itemize}
Reforçar exteriormente o costado de (um navio).
\section{Embondeiro}
\begin{itemize}
\item {Grp. gram.:m.}
\end{itemize}
Grande árvore africana, o mesmo que \textunderscore adansónia\textunderscore .
\section{Embondo}
\begin{itemize}
\item {Grp. gram.:m.}
\end{itemize}
\begin{itemize}
\item {Utilização:Bras. do Rio}
\end{itemize}
Embaraço, difficuldade.
\section{Embonecar}
\begin{itemize}
\item {Grp. gram.:v. t.}
\end{itemize}
\begin{itemize}
\item {Grp. gram.:V. i.}
\end{itemize}
\begin{itemize}
\item {Utilização:bras. do N}
\end{itemize}
\begin{itemize}
\item {Proveniência:(De \textunderscore boneca\textunderscore )}
\end{itemize}
Tornar enfeitado, como uma boneca.
Enfeitar pretensiosamente.
Criar boneca ou espiga (o milho).
\section{Embonecrar}
\begin{itemize}
\item {Grp. gram.:v. t.}
\end{itemize}
O mesmo que \textunderscore embonecar\textunderscore . Cf. Castilho, \textunderscore Fausto\textunderscore , 59.
\section{Embonicar}
\begin{itemize}
\item {Grp. gram.:v. t.}
\end{itemize}
\begin{itemize}
\item {Utilização:Prov.}
\end{itemize}
\begin{itemize}
\item {Utilização:beir.}
\end{itemize}
Barrar com bonico (cortiços de abelhas).
\section{Embono}
\begin{itemize}
\item {Grp. gram.:m.}
\end{itemize}
O mesmo que \textunderscore embonada\textunderscore .
Peças de madeira, que escoram o navio, quando está em sêco.
Acto de embonar.
(Cast. \textunderscore embono\textunderscore )
\section{Embora}
\begin{itemize}
\item {Grp. gram.:adv.}
\end{itemize}
\begin{itemize}
\item {Grp. gram.:Conj.}
\end{itemize}
\begin{itemize}
\item {Grp. gram.:Interj.}
\end{itemize}
\begin{itemize}
\item {Grp. gram.:M. pl.}
\end{itemize}
Em bôa hora.
Não obstante.
(designativa de indifferença).
Parabens.
(Contr. de \textunderscore em\textunderscore  + \textunderscore bôa\textunderscore  + \textunderscore hora\textunderscore )
\section{Emborbetar}
\begin{itemize}
\item {Grp. gram.:v. i.}
\end{itemize}
\begin{itemize}
\item {Utilização:T. de Viana}
\end{itemize}
\begin{itemize}
\item {Proveniência:(De \textunderscore borbeto\textunderscore )}
\end{itemize}
O mesmo que \textunderscore encaroçar\textunderscore : \textunderscore a farinha emborbetou no caldo\textunderscore .
\section{Emborcação}
\begin{itemize}
\item {Grp. gram.:f.}
\end{itemize}
Derramamento de um líquido medicamentoso na parte enferma de um corpo.
Líquido medicamentoso, que se emborca na parte enferma do corpo.
Acto de emborcar.
(Cp. \textunderscore embrocação\textunderscore )
\section{Emborcadela}
\begin{itemize}
\item {Grp. gram.:f.}
\end{itemize}
Acto de emborcar. Cf. Camillo, \textunderscore Sc. da Foz\textunderscore , 94.
\section{Emborcar}
\begin{itemize}
\item {Grp. gram.:v. t.}
\end{itemize}
\begin{itemize}
\item {Proveniência:(De \textunderscore bôrco\textunderscore )}
\end{itemize}
Pôr de bôca para baixo (uma vasilha).
Despejar (copo, vasilha, etc.).
Despejar na bôca, bebendo: \textunderscore emborcar uma garrafa de vinho\textunderscore .
Diz-se do toiro que, arrancando, tem por único objecto o toireiro.
\section{Embôrco}
\begin{itemize}
\item {Grp. gram.:m.}
\end{itemize}
O mesmo que \textunderscore emborcadela\textunderscore . Cf. Filinto, VI, 23.
\section{Embornadeiro}
\begin{itemize}
\item {Grp. gram.:m.}
\end{itemize}
\begin{itemize}
\item {Utilização:Prov.}
\end{itemize}
\begin{itemize}
\item {Utilização:trasm.}
\end{itemize}
Dilatação do eixo do carro, entre a roda e o refego abraçado pelas entriteiras, para defender estas da aproximação daquellas.
(Talvez por \textunderscore embonadeiro\textunderscore , do cast. \textunderscore embono\textunderscore )
\section{Embornal}
\begin{itemize}
\item {Grp. gram.:m.}
\end{itemize}
\begin{itemize}
\item {Grp. gram.:Pl.}
\end{itemize}
Saco, que se prende ao focinho das bêstas, para estas comerem o que êlle contém.
Cevadeira.
Buracos, por onde se escôa a água que caiu na coberta do navio.
(Talvez do lat. \textunderscore ambire\textunderscore  + \textunderscore urnalis\textunderscore )
\section{Embornalar}
\begin{itemize}
\item {Grp. gram.:v. t.}
\end{itemize}
Meter no bornal.
Colher para si; arrecadar na algibeira.
Economizar.
\section{Embornecer}
\begin{itemize}
\item {Grp. gram.:v. t.}
\end{itemize}
\begin{itemize}
\item {Utilização:Prov.}
\end{itemize}
\begin{itemize}
\item {Utilização:beir.}
\end{itemize}
\begin{itemize}
\item {Proveniência:(De \textunderscore borno\textunderscore )}
\end{itemize}
O mesmo que \textunderscore amornar\textunderscore .
\section{Emborque}
\begin{itemize}
\item {Grp. gram.:m.}
\end{itemize}
Acto de emborcar.
\section{Emborrachado}
\begin{itemize}
\item {Grp. gram.:adj.}
\end{itemize}
\begin{itemize}
\item {Proveniência:(De \textunderscore emborrachar\textunderscore )}
\end{itemize}
Embriagado, bêbedo.
\section{Emborrachar}
\begin{itemize}
\item {Grp. gram.:v. t.}
\end{itemize}
\begin{itemize}
\item {Utilização:Pop.}
\end{itemize}
\begin{itemize}
\item {Grp. gram.:V. i.}
\end{itemize}
\begin{itemize}
\item {Proveniência:(De \textunderscore borracho\textunderscore )}
\end{itemize}
Embebedar.
Ir engrossando, para dar a espiga, (falando-se do trigo ou do centeio).
\section{Emborralhar}
\begin{itemize}
\item {Grp. gram.:v. t.}
\end{itemize}
Cobrir com borralho.
Enfarruscar com borralho.
\section{Emborrar}
\begin{itemize}
\item {Grp. gram.:v. t.}
\end{itemize}
\begin{itemize}
\item {Utilização:Prov.}
\end{itemize}
\begin{itemize}
\item {Utilização:Ext.}
\end{itemize}
\begin{itemize}
\item {Proveniência:(De \textunderscore bôrra\textunderscore )}
\end{itemize}
Dar a primeira carda em (a lan), depois de passada pela cardiça.
Esfregar com bagaço ou borras o interior do tonel ou pipa, antes de lhe deitar o vinho.
Preparar (cortiços) para abelhas, esfregando-os com bosta, borrifando-os com vinagre, etc.
\section{Emborrascar}
\begin{itemize}
\item {Grp. gram.:v. t.}
\end{itemize}
\begin{itemize}
\item {Grp. gram.:V. p.}
\end{itemize}
\begin{itemize}
\item {Proveniência:(De \textunderscore borrasca\textunderscore )}
\end{itemize}
Tornar borrascoso.
Agitar tempestuosamente:«\textunderscore ...emborrascava um escarcéu de lembranças\textunderscore ». Camillo, \textunderscore Freira no subterr.\textunderscore , 15.
Cobrir-se de negrume, ameaçar borrasca:«\textunderscore o céu emborrascava-se, rolando trovões\textunderscore ». Camillo, \textunderscore Mulher Fatal\textunderscore , 66.
\section{Emboscada}
\begin{itemize}
\item {Grp. gram.:f.}
\end{itemize}
\begin{itemize}
\item {Proveniência:(De \textunderscore emboscar\textunderscore )}
\end{itemize}
Cilada.
Ardil; insídia.
Lugar, em que alguém se esconde, para assaltar outrem.
\section{Emboscar}
\begin{itemize}
\item {Grp. gram.:v. t.}
\end{itemize}
\begin{itemize}
\item {Grp. gram.:V. p.}
\end{itemize}
\begin{itemize}
\item {Proveniência:(De \textunderscore bosque\textunderscore )}
\end{itemize}
Pôr de emboscada.
Esconder.
Armar cilada.
Esconder-se, para dar assalto.
\section{Embosnar}
\begin{itemize}
\item {Grp. gram.:v. t.}
\end{itemize}
\begin{itemize}
\item {Utilização:Prov.}
\end{itemize}
\begin{itemize}
\item {Utilização:alg.}
\end{itemize}
Tornar macambúzio, amuado.
\section{Embosqueirado}
\begin{itemize}
\item {Grp. gram.:adj.}
\end{itemize}
\begin{itemize}
\item {Utilização:Prov.}
\end{itemize}
\begin{itemize}
\item {Proveniência:(De \textunderscore bosque\textunderscore )}
\end{itemize}
Escondido.
Retirado em lugar escuso. (Colhido em Turquel)
\section{Embostar}
\begin{itemize}
\item {Grp. gram.:v. t.}
\end{itemize}
Sujar com bosta.
Embodegar.
\section{Embostear}
\begin{itemize}
\item {Grp. gram.:v. t.}
\end{itemize}
O mesmo que \textunderscore embostar\textunderscore .
\section{Embostelado}
\begin{itemize}
\item {Grp. gram.:adj.}
\end{itemize}
Que tem bostelas.
Mazelento.
Sujo.
\section{Embostelar}
\begin{itemize}
\item {Grp. gram.:v. t.}
\end{itemize}
Encher de bostelas.
Sujar.
\section{Embotadeira}
\begin{itemize}
\item {Grp. gram.:f.}
\end{itemize}
\begin{itemize}
\item {Proveniência:(De \textunderscore bota\textunderscore )}
\end{itemize}
Peça de vestuário, que se calça por baixo do canhão da bota, cobrindo a perna até acima do joêlho.
Espécie de meia forte e comprida.
\section{Embotado}
\begin{itemize}
\item {Grp. gram.:adj.}
\end{itemize}
\begin{itemize}
\item {Utilização:Fig.}
\end{itemize}
\begin{itemize}
\item {Proveniência:(De \textunderscore embotar\textunderscore )}
\end{itemize}
Diz-se do instrumento cortante, cujo fio se engrossou.
Bôto.
Insensível: \textunderscore coração embotado\textunderscore .
\section{Embotador}
\begin{itemize}
\item {Grp. gram.:adj.}
\end{itemize}
Que embota.
\section{Embotadura}
\begin{itemize}
\item {Grp. gram.:f.}
\end{itemize}
Acto de embotar.
\section{Embotamento}
\begin{itemize}
\item {Grp. gram.:m.}
\end{itemize}
O mesmo que \textunderscore embotadura\textunderscore .
\section{Embotar}
\begin{itemize}
\item {Grp. gram.:v. t.}
\end{itemize}
\begin{itemize}
\item {Utilização:Fig.}
\end{itemize}
Tornar bôto: \textunderscore embotar os dentes\textunderscore .
Engrossar o fio ou gume de; tirar o gume a: \textunderscore embotar uma faca\textunderscore .
Tirar a fôrça a.
Tornar insensível.
\section{Embotelhar}
\begin{itemize}
\item {Grp. gram.:v. t.}
\end{itemize}
Meter em botelha; engarrafar.
\section{Embotijar}
\begin{itemize}
\item {Grp. gram.:v. t.}
\end{itemize}
\begin{itemize}
\item {Utilização:Náut.}
\end{itemize}
Meter em botija.
Cobrir completamente de merlim ou linho (um cabo).
\section{Embotilhar}
\begin{itemize}
\item {Grp. gram.:v. t.}
\end{itemize}
\begin{itemize}
\item {Utilização:Prov.}
\end{itemize}
\begin{itemize}
\item {Utilização:trasm.}
\end{itemize}
Pôr botilho em.
Enfrear com botilho, (burros, chibos, etc.).
\section{Embraçadeira}
\begin{itemize}
\item {Grp. gram.:f.}
\end{itemize}
O mesmo que \textunderscore braçadeira\textunderscore .
\section{Embraçadura}
\begin{itemize}
\item {Grp. gram.:f.}
\end{itemize}
Acto de embraçar.
Braçadeira.
\section{Embraçamento}
\begin{itemize}
\item {Grp. gram.:m.}
\end{itemize}
Acto de embraçar.
\section{Embraçar}
\begin{itemize}
\item {Grp. gram.:v. t.}
\end{itemize}
\begin{itemize}
\item {Proveniência:(De \textunderscore braço\textunderscore )}
\end{itemize}
Suspender ou suster, metendo na braçadeira.
Suster com o braço; sobraçar.
\section{Embrace}
\begin{itemize}
\item {Grp. gram.:m.}
\end{itemize}
\begin{itemize}
\item {Proveniência:(De \textunderscore embraçar\textunderscore , se não do fr. \textunderscore embrasse\textunderscore )}
\end{itemize}
Braçadeira, para cortinas de porta ou janela.
\section{Embraceirar}
\begin{itemize}
\item {Grp. gram.:v. t.  e  i.}
\end{itemize}
\begin{itemize}
\item {Utilização:Constr.}
\end{itemize}
Fazer as braceiras de um telhado.
\section{Embragar}
\begin{itemize}
\item {Grp. gram.:v. t.}
\end{itemize}
\begin{itemize}
\item {Proveniência:(De \textunderscore braga\textunderscore )}
\end{itemize}
Encaixar (um veio de máquina) noutro, para lhe trasm.ttir movimento.
\section{Embrague}
\begin{itemize}
\item {Grp. gram.:m.}
\end{itemize}
Acto de embragar.
\section{Embrancar}
\begin{itemize}
\item {Grp. gram.:v. t.}
\end{itemize}
(V.branquear)
\section{Embrandecer}
\begin{itemize}
\item {Grp. gram.:v. t.}
\end{itemize}
\begin{itemize}
\item {Utilização:Fig.}
\end{itemize}
\begin{itemize}
\item {Grp. gram.:V. i.}
\end{itemize}
Tornar brando.
Commover.
Fazer-se brando: \textunderscore o vento embrandeceu\textunderscore .
\section{Embranquecer}
\begin{itemize}
\item {Grp. gram.:v. t.}
\end{itemize}
\begin{itemize}
\item {Grp. gram.:V.}
\end{itemize}
Tornar branco.
Tornar-se branco; encanecer.
\section{Embravear}
\begin{itemize}
\item {Grp. gram.:v. t.  e  p.}
\end{itemize}
O mesmo que \textunderscore embravecer\textunderscore .
\section{Embravecer}
\begin{itemize}
\item {Grp. gram.:v. t.}
\end{itemize}
\begin{itemize}
\item {Grp. gram.:V. i.  e  p.}
\end{itemize}
Tornar bravo.
Irritar.
Enfurecer-se.
Encapellar-se: \textunderscore o mar embravece\textunderscore .
\section{Embravecimento}
\begin{itemize}
\item {Grp. gram.:m.}
\end{itemize}
Acto ou effeito de embravecer.
\section{Embravescer}
\begin{itemize}
\item {Grp. gram.:v. t.  e  i.}
\end{itemize}
\begin{itemize}
\item {Utilização:Des.}
\end{itemize}
O mesmo que \textunderscore embravecer\textunderscore .
\section{Embrear}
\begin{itemize}
\item {Grp. gram.:v. t.}
\end{itemize}
Cobrir ou untar com breu; alcatroar.
\section{Embrechado}
\begin{itemize}
\item {Grp. gram.:m.}
\end{itemize}
\begin{itemize}
\item {Utilização:Fam.}
\end{itemize}
\begin{itemize}
\item {Utilização:T. da Maia}
\end{itemize}
\begin{itemize}
\item {Grp. gram.:Pl.}
\end{itemize}
\begin{itemize}
\item {Proveniência:(De \textunderscore embrechar\textunderscore )}
\end{itemize}
Incrustações de loiça, vidros, conchas, pedrinhas, etc., com que se enfeitam paredes e cascatas de jardins.
Pessôa importuna; empecilho.
Representação pública do auto do nascimento de Christo. Cf. Flaviense, \textunderscore Diccion Geogr.\textunderscore 
Embutidos.
\section{Embrechar}
\begin{itemize}
\item {Grp. gram.:v. t.}
\end{itemize}
\begin{itemize}
\item {Proveniência:(De \textunderscore brecha\textunderscore )}
\end{itemize}
Pôr embrechados em.
\section{Embrenhar}
\begin{itemize}
\item {Grp. gram.:v. t.}
\end{itemize}
\begin{itemize}
\item {Grp. gram.:V. p.}
\end{itemize}
Esconder em brenha; meter em bosque.
Internar-se.
\section{Embretar}
\begin{itemize}
\item {Grp. gram.:v. t.}
\end{itemize}
\begin{itemize}
\item {Utilização:Bras. do S}
\end{itemize}
Meter (gado) no brete.
\section{Embriagadamente}
\begin{itemize}
\item {Grp. gram.:adv.}
\end{itemize}
\begin{itemize}
\item {Utilização:Fig.}
\end{itemize}
\begin{itemize}
\item {Proveniência:(De \textunderscore embriagado\textunderscore )}
\end{itemize}
Com embriaguez.
Enthusiasticamente.
\section{Embriagado}
\begin{itemize}
\item {Grp. gram.:adj.}
\end{itemize}
Que se embriagou.
Ébrio, bêbedo.
\section{Embriagador}
\begin{itemize}
\item {Grp. gram.:adj.}
\end{itemize}
O mesmo que \textunderscore embriagante\textunderscore . Cf. Camillo, \textunderscore Quéda de um Anjo\textunderscore ,58.
\section{Embriagante}
\begin{itemize}
\item {Grp. gram.:adj.}
\end{itemize}
Que embriaga.
\section{Embriagar}
\begin{itemize}
\item {Grp. gram.:v. t.}
\end{itemize}
\begin{itemize}
\item {Utilização:Fig.}
\end{itemize}
\begin{itemize}
\item {Proveniência:(Do lat. \textunderscore ebriacus\textunderscore )}
\end{itemize}
O mesmo que \textunderscore embebedar\textunderscore .
Enlevar.
Enthusiasmar.
\section{Embriaguez}
\begin{itemize}
\item {Grp. gram.:f.}
\end{itemize}
\begin{itemize}
\item {Proveniência:(De \textunderscore embriagar\textunderscore )}
\end{itemize}
Estado de quem se acha embriagado.
\section{Embrião}
\begin{itemize}
\item {Grp. gram.:m.}
\end{itemize}
\begin{itemize}
\item {Utilização:Fig.}
\end{itemize}
\begin{itemize}
\item {Proveniência:(Gr. \textunderscore embruon\textunderscore )}
\end{itemize}
Germe fecundado, que adquiríu o primeiro grau de desenvolvimento no seio materno.
Germe da planta, contido na semente.
Qualquer coisa que começa.
Princípio.
Aquilo que se apresenta primordialmente em estado indefinido ou confuso.
\section{Embricar}
\textunderscore v. t.\textunderscore (e der.)
O mesmo que \textunderscore imbricar\textunderscore , etc.
\section{Embridar}
\begin{itemize}
\item {Grp. gram.:v. t.}
\end{itemize}
\begin{itemize}
\item {Grp. gram.:V. i.  e  p.}
\end{itemize}
\begin{itemize}
\item {Utilização:Fig.}
\end{itemize}
Pôr brida a (o cavallo).
Curvar o pescoço com garbo, erguendo a cabeça e inclinando a barba contra o peito, (falando-se de cavallo).
Mostrar-se arrogante.
\section{Embrincar}
\begin{itemize}
\item {Grp. gram.:v. t.}
\end{itemize}
\begin{itemize}
\item {Proveniência:(De \textunderscore brinco\textunderscore )}
\end{itemize}
Pôr enfeites a; ataviar.
Adornar, engalanar. Cf. Camillo, \textunderscore Noites de Insómn.\textunderscore , IX, 68; \textunderscore Estrêl. Prop.\textunderscore , 5.
\section{Embriogenia}
\begin{itemize}
\item {Grp. gram.:f.}
\end{itemize}
\begin{itemize}
\item {Proveniência:(Do gr. \textunderscore embruon\textunderscore  + \textunderscore genes\textunderscore )}
\end{itemize}
Formação dos seres vivos e seu desenvolvimento, até á nascença.
\section{Embriogénico}
\begin{itemize}
\item {Grp. gram.:adj.}
\end{itemize}
Relativo á embriogenia.
\section{Embriografia}
\begin{itemize}
\item {Grp. gram.:f.}
\end{itemize}
\begin{itemize}
\item {Proveniência:(Do gr. \textunderscore embruon\textunderscore  + \textunderscore graphein\textunderscore )}
\end{itemize}
Descripção do embrião.
\section{Embriologia}
\begin{itemize}
\item {Grp. gram.:f.}
\end{itemize}
\begin{itemize}
\item {Proveniência:(Do gr. \textunderscore embruon\textunderscore  + \textunderscore logos\textunderscore )}
\end{itemize}
Tratado da formação e desenvolvimento do embrião.
\section{Embriólogo}
\begin{itemize}
\item {Grp. gram.:m.}
\end{itemize}
Aquele que cultiva a embriologia.
\section{Embrionado}
\begin{itemize}
\item {Grp. gram.:adj.}
\end{itemize}
\begin{itemize}
\item {Proveniência:(Do gr. \textunderscore embruon\textunderscore )}
\end{itemize}
Que tem em bião ou embriões.
\section{Embrionário}
\begin{itemize}
\item {Grp. gram.:adj.}
\end{itemize}
\begin{itemize}
\item {Utilização:Fig.}
\end{itemize}
\begin{itemize}
\item {Proveniência:(Do gr. \textunderscore embruon\textunderscore )}
\end{itemize}
Relativo ao embrião.
Que se acha em estado nascente.
Que começa a desenvolver-se.
\section{Embrionífero}
\begin{itemize}
\item {Grp. gram.:adj.}
\end{itemize}
O mesmo que \textunderscore embrionado\textunderscore .
\section{Embriotlasto}
\begin{itemize}
\item {Grp. gram.:m.}
\end{itemize}
\begin{itemize}
\item {Proveniência:(Do gr. \textunderscore embruon\textunderscore  + \textunderscore thlaein\textunderscore )}
\end{itemize}
Antigo instrumento cirúrgico, com que se partiam os ossos do féto, para se facilitar a extracção dêste.
\section{Embriotomia}
\begin{itemize}
\item {Grp. gram.:f.}
\end{itemize}
\begin{itemize}
\item {Proveniência:(Gr. \textunderscore embruotomia\textunderscore )}
\end{itemize}
Operação cirúrgica, que consiste em cortar o féto, para o extrair da madre.
\section{Embriótomo}
\begin{itemize}
\item {Grp. gram.:m.}
\end{itemize}
Instrumento, com que se faz a embriotomia.
\section{Embriulco}
\begin{itemize}
\item {Grp. gram.:m.}
\end{itemize}
\begin{itemize}
\item {Proveniência:(Gr. \textunderscore embruulkos\textunderscore )}
\end{itemize}
Espécie de gancho, que servia para extrair do útero o féto morto.
\section{Embrocação}
\begin{itemize}
\item {Grp. gram.:f.}
\end{itemize}
\begin{itemize}
\item {Proveniência:(Do gr. \textunderscore embroke\textunderscore )}
\end{itemize}
O mesmo ou melhor que \textunderscore emborcação.\textunderscore 
\section{Embroêsa}
\begin{itemize}
\item {Grp. gram.:adj. f.}
\end{itemize}
\begin{itemize}
\item {Utilização:Prov.}
\end{itemize}
\begin{itemize}
\item {Utilização:trasm.}
\end{itemize}
Diz-se da mulher arisca, orgulhosa, brava.
(Por \textunderscore emproêsa\textunderscore , de \textunderscore prôa\textunderscore ?)
\section{Embrolamento}
\begin{itemize}
\item {Grp. gram.:m.}
\end{itemize}
\begin{itemize}
\item {Utilização:Ant.}
\end{itemize}
Bordadura.
(Por \textunderscore embroslamento\textunderscore , de \textunderscore broslar\textunderscore )
\section{Embromação}
\begin{itemize}
\item {Grp. gram.:f.}
\end{itemize}
Acto de embromar.
Mentira, embuste.
\section{Embromador}
\begin{itemize}
\item {Grp. gram.:m.  e  adj.}
\end{itemize}
Aquelle que embroma.
\section{Embromar}
\begin{itemize}
\item {Grp. gram.:v. i.}
\end{itemize}
\begin{itemize}
\item {Utilização:Bras}
\end{itemize}
Adiar com embustes a resolução de um negócio.
(Cast. \textunderscore embromar\textunderscore )
\section{Embromeiro}
\begin{itemize}
\item {Grp. gram.:m.  e  adj.}
\end{itemize}
O mesmo que \textunderscore embromador\textunderscore .
\section{Embruacar}
\begin{itemize}
\item {Grp. gram.:v. t.}
\end{itemize}
\begin{itemize}
\item {Utilização:Bras}
\end{itemize}
Arrecadar em bruaca.
\section{Embruava}
\begin{itemize}
\item {Grp. gram.:m.  e  f.}
\end{itemize}
\begin{itemize}
\item {Utilização:Bras}
\end{itemize}
(V.emboaba)
\section{Embrulhada}
\begin{itemize}
\item {Grp. gram.:f.}
\end{itemize}
\begin{itemize}
\item {Utilização:Fam.}
\end{itemize}
\begin{itemize}
\item {Proveniência:(De \textunderscore embrulhar\textunderscore )}
\end{itemize}
Confusão; mistifório; trapalhada; desordem.
\section{Embrulhadamente}
\begin{itemize}
\item {Grp. gram.:adv.}
\end{itemize}
De modo embrulhado.
\section{Embrulhado}
\begin{itemize}
\item {Grp. gram.:adj.}
\end{itemize}
Metido num invólucro.
Intrincado; confuso.
Ennevoado: \textunderscore temos o tempo embrulhado\textunderscore .
\section{Embrulhador}
\begin{itemize}
\item {Grp. gram.:adj.}
\end{itemize}
\begin{itemize}
\item {Grp. gram.:M.}
\end{itemize}
Que embrulha.
Aquelle que embrulha.
\section{Embrulhamento}
\begin{itemize}
\item {Grp. gram.:m.}
\end{itemize}
O mesmo que \textunderscore embrulhada\textunderscore .
Acto de embrulhar.
\section{Embrulhar}
\begin{itemize}
\item {Grp. gram.:v. t.}
\end{itemize}
\begin{itemize}
\item {Grp. gram.:V. p.}
\end{itemize}
\begin{itemize}
\item {Proveniência:(De \textunderscore embrulho\textunderscore )}
\end{itemize}
Empacotar; envolver em papel, pano, etc.
Perturbar.
Tornar confuso.
Embaraçar: \textunderscore embrulhar um negócio\textunderscore .
Misturar.
Complicar-se; embaraçar-se.
Toldar-se ou ennevoar-se, (falando-se do céu ou do tempo).
\section{Embrulho}
\begin{itemize}
\item {Grp. gram.:m.}
\end{itemize}
Objecto embrulhado.
Pacote.
Embaraço, embrulhada.
Indisposição de estômago.
(Cp. it. \textunderscore imbroglio\textunderscore )
\section{Embruscar}
\begin{itemize}
\item {Grp. gram.:v. t.}
\end{itemize}
\begin{itemize}
\item {Grp. gram.:V. i.  e  p.}
\end{itemize}
\begin{itemize}
\item {Utilização:Fig.}
\end{itemize}
\begin{itemize}
\item {Proveniência:(De \textunderscore brusco\textunderscore )}
\end{itemize}
Tornar escuro, brusco.
Conturbar, tornar triste:«\textunderscore a mágoa que a alma lhe embrusca\textunderscore ». Filinto, VII, 201.
Escurecer.
Assumir aspecto carregado.
\section{Embrutar}
\begin{itemize}
\item {Grp. gram.:v. t.}
\end{itemize}
O mesmo que \textunderscore embrutecer\textunderscore .
\section{Embrutecedor}
\begin{itemize}
\item {Grp. gram.:adj.}
\end{itemize}
Que embrutece.
\section{Embrutecer}
\begin{itemize}
\item {Grp. gram.:v. t.}
\end{itemize}
\begin{itemize}
\item {Grp. gram.:V. p.}
\end{itemize}
Tornar bruto ou brutal.
Mostrar-se bruto, brutal ou estúpido.
\section{Embrutecimento}
\begin{itemize}
\item {Grp. gram.:m.}
\end{itemize}
Estado de quem embruteceu.
Estupidez.
\section{Embruxar}
\begin{itemize}
\item {Grp. gram.:v. t.}
\end{itemize}
\begin{itemize}
\item {Proveniência:(De \textunderscore bruxa\textunderscore )}
\end{itemize}
Fazer bruxarias contra; fazer ou dar feitiços a.
Enfeitiçar.
\section{Embryão}
\begin{itemize}
\item {Grp. gram.:m.}
\end{itemize}
\begin{itemize}
\item {Utilização:Fig.}
\end{itemize}
\begin{itemize}
\item {Proveniência:(Gr. \textunderscore embruon\textunderscore )}
\end{itemize}
Germe fecundado, que adquiríu o primeiro grau de desenvolvimento no seio materno.
Germe da planta, contido na semente.
Qualquer coisa que começa.
Princípio.
Aquillo que se apresenta primordialmente em estado indefinido ou confuso.
\section{Embryogenia}
\begin{itemize}
\item {Grp. gram.:f.}
\end{itemize}
\begin{itemize}
\item {Proveniência:(Do gr. \textunderscore embruon\textunderscore  + \textunderscore genes\textunderscore )}
\end{itemize}
Formação dos seres vivos e seu desenvolvimento, até á nascença.
\section{Embryogénico}
\begin{itemize}
\item {Grp. gram.:adj.}
\end{itemize}
Relativo á embryogenia.
\section{Embryographia}
\begin{itemize}
\item {Grp. gram.:f.}
\end{itemize}
\begin{itemize}
\item {Proveniência:(Do gr. \textunderscore embruon\textunderscore  + \textunderscore graphein\textunderscore )}
\end{itemize}
Descripção do embryão.
\section{Embryologia}
\begin{itemize}
\item {Grp. gram.:f.}
\end{itemize}
\begin{itemize}
\item {Proveniência:(Do gr. \textunderscore embruon\textunderscore  + \textunderscore logos\textunderscore )}
\end{itemize}
Tratado da formação e desenvolvimento do embryão.
\section{Embryólogo}
\begin{itemize}
\item {Grp. gram.:m.}
\end{itemize}
Aquelle que cultiva a embryologia.
\section{Embryonado}
\begin{itemize}
\item {Grp. gram.:adj.}
\end{itemize}
\begin{itemize}
\item {Proveniência:(Do gr. \textunderscore embruon\textunderscore )}
\end{itemize}
Que tem em byão ou embryões.
\section{Embryonário}
\begin{itemize}
\item {Grp. gram.:adj.}
\end{itemize}
\begin{itemize}
\item {Utilização:Fig.}
\end{itemize}
\begin{itemize}
\item {Proveniência:(Do gr. \textunderscore embruon\textunderscore )}
\end{itemize}
Relativo ao embryão.
Que se acha em estado nascente.
Que começa a desenvolver-se.
\section{Embryonífero}
\begin{itemize}
\item {Grp. gram.:adj.}
\end{itemize}
O mesmo que \textunderscore embryonado\textunderscore .
\section{Embryothlasto}
\begin{itemize}
\item {Grp. gram.:m.}
\end{itemize}
\begin{itemize}
\item {Proveniência:(Do gr. \textunderscore embruon\textunderscore  + \textunderscore thlaein\textunderscore )}
\end{itemize}
Antigo instrumento cirúrgico, com que se partiam os ossos do féto, para se facilitar a extracção dêste.
\section{Embryotomia}
\begin{itemize}
\item {Grp. gram.:f.}
\end{itemize}
\begin{itemize}
\item {Proveniência:(Gr. \textunderscore embruotomia\textunderscore )}
\end{itemize}
Operação cirúrgica, que consiste em cortar o féto, para o extrahir da madre.
\section{Embryótomo}
\begin{itemize}
\item {Grp. gram.:m.}
\end{itemize}
Instrumento, com que se faz a embryotomia.
\section{Embryulco}
\begin{itemize}
\item {Grp. gram.:m.}
\end{itemize}
\begin{itemize}
\item {Proveniência:(Gr. \textunderscore embruulkos\textunderscore )}
\end{itemize}
Espécie de gancho, que servia para extrahir do útero o féto morto.
\section{Embuá}
\begin{itemize}
\item {Grp. gram.:m.}
\end{itemize}
Insecto brasileiro, de que se confecciona massa cáustica.
\section{Embuaba}
\begin{itemize}
\item {Grp. gram.:m.}
\end{itemize}
\begin{itemize}
\item {Utilização:Bras. do S}
\end{itemize}
O mesmo que \textunderscore embruava\textunderscore .
(Cp. \textunderscore emboaba\textunderscore  e \textunderscore boava\textunderscore )
\section{Embuava}
\begin{itemize}
\item {Grp. gram.:m.}
\end{itemize}
\begin{itemize}
\item {Utilização:Bras. do S}
\end{itemize}
O mesmo que \textunderscore embruava\textunderscore .
(Cp. \textunderscore emboaba\textunderscore  e \textunderscore boava\textunderscore )
\section{Embuçadamente}
\begin{itemize}
\item {Grp. gram.:adv.}
\end{itemize}
Dessimuladamente; de modo embuçado.
\section{Embuçado}
\begin{itemize}
\item {Grp. gram.:adj.}
\end{itemize}
\begin{itemize}
\item {Grp. gram.:M.}
\end{itemize}
\begin{itemize}
\item {Proveniência:(De \textunderscore embuçar\textunderscore )}
\end{itemize}
Que se embuçou.
Disfarçado, dissimulado.
Indivíduo embuçado.
\section{Embuçar}
\begin{itemize}
\item {Grp. gram.:v. t.}
\end{itemize}
\begin{itemize}
\item {Proveniência:(De \textunderscore buço\textunderscore ?)}
\end{itemize}
Encobrir com embuço, capa, etc.
Disfarçar.
\section{Embuchar}
\begin{itemize}
\item {Grp. gram.:v. t.}
\end{itemize}
\begin{itemize}
\item {Utilização:Fam.}
\end{itemize}
\begin{itemize}
\item {Grp. gram.:V. i.}
\end{itemize}
\begin{itemize}
\item {Utilização:Fam.}
\end{itemize}
\begin{itemize}
\item {Proveniência:(De \textunderscore bucho\textunderscore )}
\end{itemize}
Encher o bucho de; fartar.
Tornar misanthropo, secretamemente desgostoso.
Não se poder expandir o que se sente ou se pensa.
Amuar-se.
\section{Embuço}
\begin{itemize}
\item {Grp. gram.:m.}
\end{itemize}
\begin{itemize}
\item {Utilização:Fig.}
\end{itemize}
Parte da capa, com que se encobre o rosto ou parte delle.
Bioco.
Dessimulação.
(Cast. \textunderscore embozo\textunderscore )
\section{Embudado}
\begin{itemize}
\item {Grp. gram.:adj.}
\end{itemize}
\begin{itemize}
\item {Utilização:Prov.}
\end{itemize}
\begin{itemize}
\item {Utilização:trasm.}
\end{itemize}
Embezerrado, macambúzio, amuado; embuchado.
(Cp. \textunderscore embude\textunderscore ^1)
\section{Embudamento}
\begin{itemize}
\item {Grp. gram.:m.}
\end{itemize}
\begin{itemize}
\item {Proveniência:(De \textunderscore embudar\textunderscore )}
\end{itemize}
Estado do peixe embudado ou entontecido.
\section{Embudar}
\begin{itemize}
\item {Grp. gram.:v. t.}
\end{itemize}
\begin{itemize}
\item {Grp. gram.:V. i.}
\end{itemize}
\begin{itemize}
\item {Proveniência:(De \textunderscore embude\textunderscore ^2)}
\end{itemize}
Fazer entontecer (os peixes) com embude^2.
Fixar a bôca nas pedras por algum tempo, (falando-se dos peixes).
\section{Embude}
\begin{itemize}
\item {Grp. gram.:m.}
\end{itemize}
Na accepção de \textunderscore ferrolho\textunderscore , é pelos diccionaristas considerado termo antigo; mas conheço-o ainda entre o povo das províncias, na accepção de \textunderscore cadeado\textunderscore  ou \textunderscore fechadura móvel\textunderscore  de arca, mala, etc.
(Relaciona talvez com \textunderscore embude\textunderscore ^3)
\section{Embude}
\begin{itemize}
\item {Grp. gram.:m.}
\end{itemize}
Substância, com que se entontecem os peixes, para os apanhar á mão; ou seja uma espécie de \textunderscore cegude\textunderscore . Cf. \textunderscore Pharm. Port.\textunderscore 
\section{Embude}
\begin{itemize}
\item {Grp. gram.:m.}
\end{itemize}
O mesmo que \textunderscore funil\textunderscore , especialmente aquelle com que se envasilha o vinho:«\textunderscore ...o vinho do tonel por um embude\textunderscore ». Filinto, VIII, 83.--Nesta accepção, dizem os diccionários que é vocábulo antigo; póde porém ouvir-se ainda hoje no Minho, mormente na Póvoa de Lanhoso.
(Cast. \textunderscore embudo\textunderscore , do lat. \textunderscore imbutum\textunderscore )
\section{Embude}
\begin{itemize}
\item {Grp. gram.:m.}
\end{itemize}
\begin{itemize}
\item {Utilização:Prov.}
\end{itemize}
\begin{itemize}
\item {Utilização:trasm.}
\end{itemize}
Criança adoentada.
\section{Embudo}
\begin{itemize}
\item {Grp. gram.:m.}
\end{itemize}
\begin{itemize}
\item {Utilização:Prov.}
\end{itemize}
O mesmo ou melhor que \textunderscore embude\textunderscore ^3.
O mesmo que \textunderscore embude\textunderscore ^2; massa de trovisco pisado, metída num saco, em fórma de funil.--Desta definição se infere que \textunderscore embude\textunderscore ^2 e \textunderscore embude\textunderscore ^3, talvez sejam o mesmo vocábulo.
\section{Embuí}
\begin{itemize}
\item {Grp. gram.:m.}
\end{itemize}
\begin{itemize}
\item {Utilização:Bras}
\end{itemize}
Árvore silvestre, de que há duas espécies, o \textunderscore embuí branco\textunderscore  e o \textunderscore amarelo\textunderscore .
\section{Embuizar}
\begin{itemize}
\item {fónica:bu-i}
\end{itemize}
\begin{itemize}
\item {Grp. gram.:v. t.}
\end{itemize}
\begin{itemize}
\item {Utilização:Ant.}
\end{itemize}
\begin{itemize}
\item {Proveniência:(De \textunderscore buiz\textunderscore )}
\end{itemize}
Tornar curvo, como o arco da buiz.
\section{Embuizar}
\begin{itemize}
\item {fónica:bu-i}
\end{itemize}
\begin{itemize}
\item {Grp. gram.:v. t.}
\end{itemize}
\begin{itemize}
\item {Utilização:Ant.}
\end{itemize}
Embutir; imbuir.
(Por \textunderscore imbuizar\textunderscore , de \textunderscore imbuir\textunderscore )
\section{Emburana}
\begin{itemize}
\item {Grp. gram.:f.}
\end{itemize}
Nome de várias árvores brasileiras.
\section{Emburerembo}
\begin{itemize}
\item {Grp. gram.:m.}
\end{itemize}
Planta trepadeira do Pará.
\section{Emburi}
\begin{itemize}
\item {Grp. gram.:m.}
\end{itemize}
Espécie de palmeira do Brasil.
\section{Emburilhada}
\begin{itemize}
\item {Grp. gram.:f.}
\end{itemize}
O mesmo que \textunderscore embrulhada\textunderscore .
\section{Emburilhar}
\begin{itemize}
\item {Grp. gram.:v. t.}
\end{itemize}
\begin{itemize}
\item {Utilização:Ant.}
\end{itemize}
O mesmo que \textunderscore embrulhar\textunderscore . Cf. G. Vicente, \textunderscore Maria Parda\textunderscore .
\section{Emburla}
\begin{itemize}
\item {Grp. gram.:f.}
\end{itemize}
Planta leguminosa de Cabo-Verde.
\section{Emburrar}
\begin{itemize}
\item {Grp. gram.:v. t.}
\end{itemize}
\begin{itemize}
\item {Utilização:Pop.}
\end{itemize}
\begin{itemize}
\item {Utilização:T. de Turquel}
\end{itemize}
\begin{itemize}
\item {Grp. gram.:V. i.}
\end{itemize}
\begin{itemize}
\item {Proveniência:(De \textunderscore burro\textunderscore )}
\end{itemize}
O mesmo que \textunderscore embrutecer\textunderscore .
Pôr (um tronco ou toro) na burra (cavalete), para sêr serrado.
Emperrar, parar teimosamente como um burro.
Embezerrar.
\section{Emburricar}
\begin{itemize}
\item {Grp. gram.:v. t.}
\end{itemize}
\begin{itemize}
\item {Utilização:Des.}
\end{itemize}
\begin{itemize}
\item {Utilização:Prov.}
\end{itemize}
\begin{itemize}
\item {Utilização:dur.}
\end{itemize}
\begin{itemize}
\item {Proveniência:(De \textunderscore burrico\textunderscore )}
\end{itemize}
Enganar grosseiramente.
Embruxar.
Cobrir de terra (certas vergônteas de videira), para reproducção. Cp. \textunderscore burro\textunderscore ^3.
\section{Emburulhar}
\begin{itemize}
\item {Grp. gram.:v. t.}
\end{itemize}
\begin{itemize}
\item {Utilização:Ant.}
\end{itemize}
O mesmo que \textunderscore embrulhar\textunderscore . Cf. \textunderscore Peregrinação\textunderscore , CLV.
\section{Embustaria}
\begin{itemize}
\item {Grp. gram.:f.}
\end{itemize}
O mesmo que \textunderscore embustice\textunderscore .
\section{Embuste}
\begin{itemize}
\item {Grp. gram.:m.}
\end{itemize}
\begin{itemize}
\item {Proveniência:(T. cast.)}
\end{itemize}
Mentira artificiosa; ardil; lôgro; enrêdo.
\section{Embustear}
\begin{itemize}
\item {Grp. gram.:v. t.}
\end{itemize}
Enganar com embuste; lograr.
\section{Embusteiro}
\begin{itemize}
\item {Grp. gram.:m.  e  adj.}
\end{itemize}
Mentiroso; aquelle que usa de embuste.
Hypócrita.
Trapaceiro.
\section{Embustice}
\begin{itemize}
\item {Grp. gram.:f.}
\end{itemize}
O mesmo que \textunderscore embuste\textunderscore .
\section{Embutideira}
\begin{itemize}
\item {Grp. gram.:f.}
\end{itemize}
\begin{itemize}
\item {Proveniência:(De \textunderscore embutir\textunderscore )}
\end{itemize}
Utensílio de ourives, para tornar os botões relevados por dentro.
\section{Embutido}
\begin{itemize}
\item {Grp. gram.:m.}
\end{itemize}
\begin{itemize}
\item {Proveniência:(De \textunderscore embutir\textunderscore )}
\end{itemize}
Obra de várias peças embutidas; mosaico.
\section{Embutidor}
\begin{itemize}
\item {Grp. gram.:adj.}
\end{itemize}
\begin{itemize}
\item {Grp. gram.:M.}
\end{itemize}
Que embute.
Aquelle que embute.
\section{Embutidura}
\begin{itemize}
\item {Grp. gram.:f.}
\end{itemize}
Acto ou effeito de embutir.
\section{Embutimento}
\begin{itemize}
\item {Grp. gram.:m.}
\end{itemize}
O mesmo que \textunderscore embutidura\textunderscore .
\section{Embutir}
\begin{itemize}
\item {Grp. gram.:v. t.}
\end{itemize}
\begin{itemize}
\item {Utilização:Fig.}
\end{itemize}
Embeber, entalhar, introduzir, differentes peças de madeira, pedra, marfim, etc., em; marchetar.
Impingir.
(Cast. \textunderscore embutir\textunderscore , do lat. \textunderscore imbutus\textunderscore )
\section{Embuziar}
\begin{itemize}
\item {Grp. gram.:v. t.}
\end{itemize}
\begin{itemize}
\item {Utilização:des.}
\end{itemize}
\begin{itemize}
\item {Utilização:Fam.}
\end{itemize}
\begin{itemize}
\item {Grp. gram.:V. i.}
\end{itemize}
\begin{itemize}
\item {Proveniência:(De \textunderscore búzio\textunderscore . Cp. \textunderscore macambúzio\textunderscore )}
\end{itemize}
Lambuzar; emporcalhar.
Amuar.
Irar-se.
Embezerrar; fazer-se macambúzio.
\section{Embuzinar}
\begin{itemize}
\item {Grp. gram.:v. t.}
\end{itemize}
\begin{itemize}
\item {Utilização:Pop.}
\end{itemize}
\begin{itemize}
\item {Grp. gram.:V. p.}
\end{itemize}
\begin{itemize}
\item {Utilização:Prov.}
\end{itemize}
\begin{itemize}
\item {Utilização:beir.}
\end{itemize}
O mesmo que \textunderscore embuziar\textunderscore .
Empanzinar-se.
Fartar-se.
\section{Em-cima}
\begin{itemize}
\item {Grp. gram.:loc. adv.}
\end{itemize}
Na parte superior: no alto.
\section{Emenda}
\begin{itemize}
\item {Grp. gram.:f.}
\end{itemize}
\begin{itemize}
\item {Utilização:Náut.}
\end{itemize}
Acto ou effeito de emendar.
Acto de melhorar o próprio procedimento; regeneração: \textunderscore aquelle sujeito não tem emenda\textunderscore .
Objecto ou peça, que se accrescenta a outra.
Lugar onde se ligam duas peças.
Remendo.
Madeiro central dos que formam a roda da prôa.
\section{Emendação}
\begin{itemize}
\item {Grp. gram.:f.}
\end{itemize}
Acto de emendar. Cf. Castilho, \textunderscore Metam.\textunderscore  XXXI.
\section{Emendadamente}
\begin{itemize}
\item {Grp. gram.:adv.}
\end{itemize}
\begin{itemize}
\item {Proveniência:(De \textunderscore emendar\textunderscore )}
\end{itemize}
Correctamente, com emendas: \textunderscore esta obra republicou-se emendadamente\textunderscore .
\section{Emendador}
\begin{itemize}
\item {Grp. gram.:adj.}
\end{itemize}
\begin{itemize}
\item {Grp. gram.:M.}
\end{itemize}
Que emenda.
Aquelle que emenda.
\section{Emendamento}
\begin{itemize}
\item {Grp. gram.:m.}
\end{itemize}
O mesmo que \textunderscore emendação\textunderscore .
\section{Emendar}
\begin{itemize}
\item {Grp. gram.:v. t.}
\end{itemize}
\begin{itemize}
\item {Grp. gram.:V. p.}
\end{itemize}
\begin{itemize}
\item {Proveniência:(Lat. \textunderscore emendare\textunderscore )}
\end{itemize}
Tornar melhor.
Corrigir (aquillo que estava errado ou mal feito).
Castigar.
Indemnizar.
Modificar; accrescentar alguma coisa a.
Arrepender-se; corrigir-se.
\section{Emendável}
\begin{itemize}
\item {Grp. gram.:adj.}
\end{itemize}
Que se póde emendar.
\section{Emendicar}
\begin{itemize}
\item {Grp. gram.:v. t.}
\end{itemize}
\begin{itemize}
\item {Utilização:Ant.}
\end{itemize}
O mesmo que \textunderscore mendigar\textunderscore .
\section{Ementa}
\begin{itemize}
\item {Grp. gram.:f.}
\end{itemize}
\begin{itemize}
\item {Proveniência:(Do lat. \textunderscore ementum\textunderscore )}
\end{itemize}
Apontamento para lembrança; relação.
Summário.
\section{Ementar}
\begin{itemize}
\item {Grp. gram.:v. t.}
\end{itemize}
Fazer ementa de; tomar apontamento de.
Fazer menção de; relembrar.
\section{Ementário}
\begin{itemize}
\item {Grp. gram.:m.}
\end{itemize}
Livro ou caderno de ementas.
\section{Emergência}
\begin{itemize}
\item {Grp. gram.:f.}
\end{itemize}
\begin{itemize}
\item {Utilização:Fig.}
\end{itemize}
Acto de emergir.
Successão casual; incidente.
\section{Emergente}
\begin{itemize}
\item {Grp. gram.:adj.}
\end{itemize}
\begin{itemize}
\item {Proveniência:(Lat. \textunderscore emergens\textunderscore )}
\end{itemize}
Que emerge; que resulta ou procede.
\section{Emergir}
\begin{itemize}
\item {Grp. gram.:v. i.}
\end{itemize}
\begin{itemize}
\item {Proveniência:(Lat. \textunderscore emergere\textunderscore )}
\end{itemize}
Saír donde estava mergulhado.
Subir, elevar-se.
Patentear-se.
Sobresaír.
Acontecer.
Occorrer; resultar.
\section{Emérito}
\begin{itemize}
\item {Grp. gram.:adj.}
\end{itemize}
\begin{itemize}
\item {Utilização:Fig.}
\end{itemize}
\begin{itemize}
\item {Proveniência:(Lat. \textunderscore emeritus\textunderscore )}
\end{itemize}
Aposentado.
Que tem as honras de um cargo, sem o exercer.
Muito versado numa sciência, arte ou profissão: \textunderscore escritor emérito\textunderscore .
\section{Êmero}
\begin{itemize}
\item {Grp. gram.:m.}
\end{itemize}
Planta leguminosa.
\section{Emersão}
\begin{itemize}
\item {Grp. gram.:f.}
\end{itemize}
\begin{itemize}
\item {Proveniência:(Lat. \textunderscore emersio\textunderscore )}
\end{itemize}
Acto de emergir.
\section{Emerso}
\begin{itemize}
\item {Grp. gram.:adj.}
\end{itemize}
\begin{itemize}
\item {Proveniência:(Lat. \textunderscore emersus\textunderscore )}
\end{itemize}
Que emergiu; que saiu de debaixo da água ou donde estava mergulhado.
\section{Emessenos}
\begin{itemize}
\item {Grp. gram.:m. pl.}
\end{itemize}
Uma das colónias muçulmanas, que, na Idade-Média, se estabeleceram na península hispânica. Cf. Herculano. \textunderscore Hist. de Port.\textunderscore , 1^a ed., III, 201.
\section{Emético}
\begin{itemize}
\item {Grp. gram.:adj.}
\end{itemize}
\begin{itemize}
\item {Grp. gram.:M.}
\end{itemize}
\begin{itemize}
\item {Proveniência:(Gr. \textunderscore emetikos\textunderscore )}
\end{itemize}
Que provoca o vómito.
Vomitório.
\section{Emetina}
\begin{itemize}
\item {Grp. gram.:f.}
\end{itemize}
\begin{itemize}
\item {Proveniência:(De \textunderscore emético\textunderscore )}
\end{itemize}
Álcali vegetal, extrahido da ipecacuanha.
\section{Emetizar}
\begin{itemize}
\item {Grp. gram.:v. t.}
\end{itemize}
\begin{itemize}
\item {Proveniência:(De \textunderscore emético\textunderscore )}
\end{itemize}
Misturar com emético.
Applicar emético a.
\section{Emetologia}
\begin{itemize}
\item {Grp. gram.:f.}
\end{itemize}
\begin{itemize}
\item {Proveniência:(Do gr. \textunderscore emetos\textunderscore  + \textunderscore logos\textunderscore )}
\end{itemize}
Tratado scientífico do vómito e das substâncias eméticas.
\section{Emexo}
\begin{itemize}
\item {fónica:cso}
\end{itemize}
\begin{itemize}
\item {Grp. gram.:m.}
\end{itemize}
Gênero de plantas polygóneas.
\section{Emfim}
\begin{itemize}
\item {Grp. gram.:adv.}
\end{itemize}
\begin{itemize}
\item {Proveniência:(De \textunderscore em\textunderscore  + \textunderscore fim\textunderscore )}
\end{itemize}
O mesmo que \textunderscore finalmente\textunderscore .
\section{Em-fim}
\begin{itemize}
\item {Grp. gram.:adv.}
\end{itemize}
\begin{itemize}
\item {Proveniência:(De \textunderscore em\textunderscore  + \textunderscore fim\textunderscore )}
\end{itemize}
O mesmo que \textunderscore finalmente\textunderscore .
\section{Emigração}
\begin{itemize}
\item {Grp. gram.:f.}
\end{itemize}
\begin{itemize}
\item {Proveniência:(Lat. \textunderscore emigratio\textunderscore )}
\end{itemize}
Acto de emigrar.
\section{Emigrado}
\begin{itemize}
\item {Grp. gram.:m.}
\end{itemize}
\begin{itemize}
\item {Proveniência:(De \textunderscore emigrar\textunderscore )}
\end{itemize}
Aquelle que emigrou; emigrante.
\section{Emigrante}
\begin{itemize}
\item {Grp. gram.:adj.}
\end{itemize}
\begin{itemize}
\item {Grp. gram.:M.}
\end{itemize}
\begin{itemize}
\item {Proveniência:(Lat. \textunderscore emigrans\textunderscore )}
\end{itemize}
Que emigra.
Aquelle que emigra.
\section{Emigrar}
\begin{itemize}
\item {Grp. gram.:v. i.}
\end{itemize}
\begin{itemize}
\item {Proveniência:(Lat. \textunderscore emigrare\textunderscore )}
\end{itemize}
Sair da pátria, para residir noutro país.
Mudar de país.
Homisiar-se.
\section{Emilia}
\begin{itemize}
\item {Grp. gram.:f.}
\end{itemize}
Planta senecionídea.
\section{Êmina}
\begin{itemize}
\item {Grp. gram.:f.}
\end{itemize}
\begin{itemize}
\item {Proveniência:(Lat. \textunderscore hemina\textunderscore )}
\end{itemize}
Antiga medida de vinho, que, entre os Romanos, corresponderia a 7 decilitros do actual systema de medidas, e que na Idade-Média variou de nação para nação. Cf. Herculano, \textunderscore Bobo\textunderscore , 157.
\section{Eminada}
\begin{itemize}
\item {Grp. gram.:f.}
\end{itemize}
\begin{itemize}
\item {Utilização:Ant.}
\end{itemize}
\begin{itemize}
\item {Proveniência:(De \textunderscore êmina\textunderscore )}
\end{itemize}
Porção que póde conter-se numa êmina.
\section{Eminência}
\begin{itemize}
\item {Grp. gram.:f.}
\end{itemize}
\begin{itemize}
\item {Proveniência:(Lat. \textunderscore eminentia\textunderscore )}
\end{itemize}
Qualidade daquillo que é eminente.
Ponto elevado.
Superioridade.
Elevação moral.
Saliência.
Título dos Cardeaes.
\section{Eminencial}
\begin{itemize}
\item {Grp. gram.:adj.}
\end{itemize}
\begin{itemize}
\item {Proveniência:(De \textunderscore eminência\textunderscore )}
\end{itemize}
Em Philosophia, diz-se de uma faculdade ou de uma causa, que produz ou abrange um effeito com excellência.
\section{Emetrope}
\begin{itemize}
\item {Grp. gram.:adj.}
\end{itemize}
\begin{itemize}
\item {Utilização:Anat.}
\end{itemize}
\begin{itemize}
\item {Proveniência:(Do gr. \textunderscore en\textunderscore  + \textunderscore metron\textunderscore  + \textunderscore ops\textunderscore )}
\end{itemize}
Normalmente constituido, (falando-se do ôlho).
Cujos olhos são normalmente constituidos.
\section{Emetropia}
\begin{itemize}
\item {Grp. gram.:f.}
\end{itemize}
Estado de quem é emetrope.
\section{Eminencialmente}
\begin{itemize}
\item {Grp. gram.:adv.}
\end{itemize}
\begin{itemize}
\item {Proveniência:(De \textunderscore eminencial\textunderscore )}
\end{itemize}
No máximo grau.
Acima de tudo que é da mesma espécie. Cf. Bernárdez, \textunderscore Luz e calor\textunderscore , 331.
\section{Eminenciar}
\begin{itemize}
\item {Grp. gram.:v. t.}
\end{itemize}
\begin{itemize}
\item {Proveniência:(De \textunderscore eminência\textunderscore )}
\end{itemize}
Estar sobranceiro a, acima de.
\section{Eminente}
\begin{itemize}
\item {Grp. gram.:adj.}
\end{itemize}
\begin{itemize}
\item {Proveniência:(Lat. \textunderscore eminens\textunderscore )}
\end{itemize}
Alto, elevado.
Superior.
Excellente; sublime.
\section{Eminentemente}
\begin{itemize}
\item {Grp. gram.:adv.}
\end{itemize}
De modo eminente.
\section{Eminentíssimo}
\begin{itemize}
\item {Grp. gram.:adj.}
\end{itemize}
Muito eminente.
Diz-se especialmente dos Cardeaes, como qualificativo correspondente ao tratamento de \textunderscore eminência\textunderscore .
\section{Emir}
\begin{itemize}
\item {Grp. gram.:m.}
\end{itemize}
\begin{itemize}
\item {Proveniência:(T. ár.)}
\end{itemize}
Título dos descendentes de Mafoma.
Título dos chefes de algumas tríbos ou estados muçulmanos.
\section{Emirado}
\begin{itemize}
\item {Grp. gram.:m.}
\end{itemize}
\begin{itemize}
\item {Proveniência:(De \textunderscore emir\textunderscore )}
\end{itemize}
País ou estado, governado por um emir.
Dignidade de um emir. Cf. Herculano, \textunderscore Hist. de Port.\textunderscore , 49, 54 e 80.
\section{Emissão}
\begin{itemize}
\item {Grp. gram.:f.}
\end{itemize}
\begin{itemize}
\item {Proveniência:(Lat. \textunderscore emíssio\textunderscore )}
\end{itemize}
Acto de emittir.
\section{Emissário}
\begin{itemize}
\item {Grp. gram.:adj.}
\end{itemize}
\begin{itemize}
\item {Grp. gram.:M.}
\end{itemize}
\begin{itemize}
\item {Utilização:Des.}
\end{itemize}
\begin{itemize}
\item {Proveniência:(Lat. \textunderscore emissarius\textunderscore )}
\end{itemize}
Que serve para emittir.
Mensageiro.
Aquelle que é mandado a cumprir uma missão.
Espião.
Canal para escoamento.
\section{Emissivo}
\begin{itemize}
\item {Grp. gram.:adj.}
\end{itemize}
\begin{itemize}
\item {Proveniência:(Do lat. \textunderscore emissus\textunderscore )}
\end{itemize}
Que póde emittir.
\section{Emissor}
\begin{itemize}
\item {Grp. gram.:m.}
\end{itemize}
\begin{itemize}
\item {Grp. gram.:Adj.}
\end{itemize}
\begin{itemize}
\item {Proveniência:(Lat. \textunderscore emissor\textunderscore )}
\end{itemize}
Aquelle que emitte ou envia alguém ou alguma coisa.
Diz-se especialmente do Banco ou estabelecimento de crédito, que emitte moéda de papel.
\section{Emiticidade}
\begin{itemize}
\item {Grp. gram.:f.}
\end{itemize}
\begin{itemize}
\item {Proveniência:(De \textunderscore emético\textunderscore )}
\end{itemize}
Propriedade de fazer vomitar.
\section{Emitir}
\begin{itemize}
\item {Grp. gram.:v. t.}
\end{itemize}
\begin{itemize}
\item {Proveniência:(Lat. \textunderscore emittere\textunderscore )}
\end{itemize}
Mandar, enviar, para fóra.
Lançar de si.
Expedir.
Pôr em circulação monetária.
Exprimir, enunciar: \textunderscore emitir opinião\textunderscore .
\section{Emittir}
\begin{itemize}
\item {Grp. gram.:v. t.}
\end{itemize}
\begin{itemize}
\item {Proveniência:(Lat. \textunderscore emittere\textunderscore )}
\end{itemize}
Mandar, enviar, para fóra.
Lançar de si.
Expedir.
Pôr em circulação monetária.
Exprimir, enunciar: \textunderscore emittir opinião\textunderscore .
\section{Emmaçar}
\begin{itemize}
\item {Grp. gram.:v. t.}
\end{itemize}
Reunir em maço ou em maços.
\section{Emmadeirar}
\textunderscore v. t.\textunderscore  (e der.)
O mesmo que \textunderscore madeirar\textunderscore , etc.
\section{Emmadeixar}
\begin{itemize}
\item {Grp. gram.:v. t.}
\end{itemize}
Dispor em madeixas, formar madeixas com.
\section{Emmadurecer}
\begin{itemize}
\item {Grp. gram.:v. i.}
\end{itemize}
O mesmo que \textunderscore amadurecer\textunderscore .
\section{Emmagotar}
\begin{itemize}
\item {Grp. gram.:v. t.}
\end{itemize}
Reunir em magotes.
\section{Emmagrecer}
\begin{itemize}
\item {Grp. gram.:v. t.}
\end{itemize}
\begin{itemize}
\item {Grp. gram.:V. i.}
\end{itemize}
Tornar magro.
Tornar-se magro.
\section{Emmagrecimento}
\begin{itemize}
\item {Grp. gram.:m.}
\end{itemize}
Acto ou effeito de emmagrecer.
\section{Emmagrentar}
\begin{itemize}
\item {Grp. gram.:v. t.  e  i.}
\end{itemize}
O mesmo que \textunderscore emmagrecer\textunderscore .
\section{Emmalar}
\begin{itemize}
\item {Grp. gram.:v. t.}
\end{itemize}
\begin{itemize}
\item {Grp. gram.:V. i.}
\end{itemize}
\begin{itemize}
\item {Utilização:Fam.}
\end{itemize}
\begin{itemize}
\item {Proveniência:(De \textunderscore mala\textunderscore )}
\end{itemize}
Meter em mala.
Empacotar.
Enrolar em fórma de mala.
Dispor-se para viagem.
\section{Emmalhar}
\begin{itemize}
\item {Grp. gram.:v. t.}
\end{itemize}
\begin{itemize}
\item {Proveniência:(De \textunderscore malha\textunderscore )}
\end{itemize}
Fazer as malhas de (uma rede).
Prender ou colher em malhas de rede: \textunderscore emmalhar peixes\textunderscore .
\section{Emmalhetamento}
\begin{itemize}
\item {Grp. gram.:m.}
\end{itemize}
Acto de emmalhetar.
\section{Emmalhetar}
\begin{itemize}
\item {Grp. gram.:v. t.}
\end{itemize}
Juntar por meio de malhetes.
Fazer travamento de alguma coisa.
\section{Emmanchar-se}
\begin{itemize}
\item {Grp. gram.:v. p.}
\end{itemize}
\begin{itemize}
\item {Utilização:Prov.}
\end{itemize}
\begin{itemize}
\item {Utilização:alent.}
\end{itemize}
Recolher-se (a caça), na \textunderscore mancha\textunderscore .
\section{Emmandingar}
\begin{itemize}
\item {Grp. gram.:v. t.}
\end{itemize}
Dar mandinga a. Cf. Castilho, \textunderscore Fausto\textunderscore , 98.
\section{Emmanilhar}
\begin{itemize}
\item {Grp. gram.:v. t.}
\end{itemize}
Pôr manilhas em. Cf. Filinto, \textunderscore D. Man.\textunderscore , II, 282.
\section{Emmantar}
\begin{itemize}
\item {Grp. gram.:v. t.}
\end{itemize}
\begin{itemize}
\item {Utilização:Prov.}
\end{itemize}
\begin{itemize}
\item {Utilização:trasm.}
\end{itemize}
Cobrir com manta.
\section{Emmantilhar}
\begin{itemize}
\item {Grp. gram.:v. t.}
\end{itemize}
Cobrir com mantilha, pôr mantilha em. Cf. Arn. Gama, \textunderscore Bailio\textunderscore .
\section{Emmanquecer}
\begin{itemize}
\item {Grp. gram.:v. t.}
\end{itemize}
\begin{itemize}
\item {Grp. gram.:V. i.}
\end{itemize}
Tornar manco.
Tornar-se manco.
\section{Emmaranhamento}
\begin{itemize}
\item {Grp. gram.:m.}
\end{itemize}
Acto ou effeito de emmaranhar.
\section{Emmaranhar}
\begin{itemize}
\item {Grp. gram.:v. t.}
\end{itemize}
\begin{itemize}
\item {Proveniência:(De \textunderscore maranha\textunderscore )}
\end{itemize}
Enredar: \textunderscore emmaranhar meadas\textunderscore .
Complicar: \textunderscore emmaranhar uma questão\textunderscore .
Confundir.
\section{Emmarar}
\begin{itemize}
\item {Grp. gram.:v. t.  e  i.}
\end{itemize}
O mesmo que \textunderscore amarar\textunderscore .
\section{Emmareado}
\begin{itemize}
\item {Grp. gram.:adj.}
\end{itemize}
\begin{itemize}
\item {Utilização:Des.}
\end{itemize}
\begin{itemize}
\item {Proveniência:(De \textunderscore mareado\textunderscore )}
\end{itemize}
Diz-se dos mantimentos que se tornam corruptos, por andarem muito tempo no mar.
\section{Emmarelecer}
\begin{itemize}
\item {Grp. gram.:v. i.  e  t.}
\end{itemize}
(V.amarelecer)
\section{Emmarjar}
\begin{itemize}
\item {Grp. gram.:v. t.}
\end{itemize}
\begin{itemize}
\item {Utilização:Prov.}
\end{itemize}
\begin{itemize}
\item {Utilização:minh.}
\end{itemize}
\begin{itemize}
\item {Proveniência:(De \textunderscore margem\textunderscore )}
\end{itemize}
Demarcar com regos as margens ou leiras de (um campo lavrado).
Aleirar.
\section{Emmarouviado}
\begin{itemize}
\item {Grp. gram.:adj.}
\end{itemize}
\begin{itemize}
\item {Utilização:Prov.}
\end{itemize}
\begin{itemize}
\item {Utilização:alent.}
\end{itemize}
Adoentado.
Abatido por doença.
\section{Emmascarar}
\begin{itemize}
\item {Grp. gram.:v. t.}
\end{itemize}
(V.mascarar)
\section{Emmassar}
\begin{itemize}
\item {Grp. gram.:v. t.}
\end{itemize}
Converter em massa, empastar.
\section{Emmastear}
\begin{itemize}
\item {Proveniência:(De \textunderscore masto\textunderscore )}
\end{itemize}
\textunderscore v. t.\textunderscore  (e der.)
O mesmo que \textunderscore mastrear\textunderscore , etc.
\section{Emmastrar}
\begin{itemize}
\item {Grp. gram.:v. t.}
\end{itemize}
O mesmo que \textunderscore mastrear\textunderscore .
\section{Emmastrear}
\begin{itemize}
\item {Grp. gram.:v. t.}
\end{itemize}
O mesmo que \textunderscore mastrear\textunderscore .
\section{Emmechar}
\begin{itemize}
\item {Grp. gram.:v. i.}
\end{itemize}
\begin{itemize}
\item {Proveniência:(De \textunderscore mecha\textunderscore )}
\end{itemize}
Introduzir-se como espigão (na carlinga).
\section{Emmedar}
\begin{itemize}
\item {Grp. gram.:v. t.}
\end{itemize}
Dispor em medas: \textunderscore emmedar o trigo\textunderscore .
\section{Emmedoiçar}
\begin{itemize}
\item {Grp. gram.:v. t.}
\end{itemize}
\begin{itemize}
\item {Utilização:Prov.}
\end{itemize}
\begin{itemize}
\item {Utilização:trasm.}
\end{itemize}
Dispor em medoiços.
\section{Emmeio}
\begin{itemize}
\item {Grp. gram.:m.}
\end{itemize}
Comenos:«\textunderscore havia nesse emmeio frequentes recontros\textunderscore ». Filinto, \textunderscore D. Man.\textunderscore , III, 119.
\section{Emmelar}
\begin{itemize}
\item {Grp. gram.:v. t.}
\end{itemize}
O mesmo que \textunderscore melar\textunderscore ^1.
\section{Emmenagogo}
\begin{itemize}
\item {fónica:gó}
\end{itemize}
\begin{itemize}
\item {Grp. gram.:adj.}
\end{itemize}
\begin{itemize}
\item {Grp. gram.:M.}
\end{itemize}
\begin{itemize}
\item {Proveniência:(Do gr. \textunderscore emmenos\textunderscore  + \textunderscore agogos\textunderscore )}
\end{itemize}
Diz-se do medicamento, que provoca ou restabelece o mênstruo.
Êsse medicamento.
\section{Emmeninecer}
\begin{itemize}
\item {Grp. gram.:v. i.}
\end{itemize}
Voltar ao estado de menino; rejuvenescer. Cf. Camões, \textunderscore Seleuco\textunderscore .
\section{Emmensite}
\begin{itemize}
\item {Grp. gram.:f.}
\end{itemize}
Espécie de explosivo.
\section{Emmentes}
\begin{itemize}
\item {Grp. gram.:m.  e  adv.}
\end{itemize}
\begin{itemize}
\item {Utilização:Prov.}
\end{itemize}
\begin{itemize}
\item {Utilização:açor}
\end{itemize}
\begin{itemize}
\item {Utilização:trasm}
\end{itemize}
\begin{itemize}
\item {Utilização:beir.}
\end{itemize}
O mesmo que \textunderscore entrementes\textunderscore . Cf. \textunderscore Eufrosina\textunderscore , 211.
\section{Emmetrope}
\begin{itemize}
\item {Grp. gram.:adj.}
\end{itemize}
\begin{itemize}
\item {Utilização:Anat.}
\end{itemize}
\begin{itemize}
\item {Proveniência:(Do gr. \textunderscore en\textunderscore  + \textunderscore metron\textunderscore  + \textunderscore ops\textunderscore )}
\end{itemize}
Normalmente constituido, (falando-se do ôlho).
Cujos olhos são normalmente constituidos.
\section{Emmetropia}
\begin{itemize}
\item {Grp. gram.:f.}
\end{itemize}
Estado de quem é emmetrope.
\section{Emmoirar}
\begin{itemize}
\item {Grp. gram.:v. i.}
\end{itemize}
O mesmo que \textunderscore moirar\textunderscore ^1.
\section{Emmoldar}
\begin{itemize}
\item {Grp. gram.:v. t.}
\end{itemize}
(V.amoldar)
\section{Emmoldurar}
\begin{itemize}
\item {Grp. gram.:v. t.}
\end{itemize}
Cercar de moldura, encaixilhar.
Guarnecer, ornar em volta: \textunderscore emmoldurar de begónias um canteiro\textunderscore .
\section{Emmolhar}
\begin{itemize}
\item {Grp. gram.:v. t.}
\end{itemize}
Juntar em molhos.
\section{Emmonar-se}
\begin{itemize}
\item {Grp. gram.:v. p.}
\end{itemize}
\begin{itemize}
\item {Utilização:Pop.}
\end{itemize}
\begin{itemize}
\item {Proveniência:(De \textunderscore mono\textunderscore )}
\end{itemize}
Amuar-se; arrufar-se.
Embezerrar.
\section{Emmordaçar}
\begin{itemize}
\item {Grp. gram.:v. t.}
\end{itemize}
(V.amordaçar)
\section{Emmorear}
\begin{itemize}
\item {Grp. gram.:v. t.}
\end{itemize}
\begin{itemize}
\item {Utilização:Prov.}
\end{itemize}
Pôr (os cereaes) em moreia.
Emmedar.
\section{Emmoroiçar}
\begin{itemize}
\item {Grp. gram.:v. t.}
\end{itemize}
Dispor em fórma de moroiço.
Amontoar.
\section{Emmortecer}
\begin{itemize}
\item {Grp. gram.:v. t.}
\end{itemize}
\begin{itemize}
\item {Grp. gram.:V. i.}
\end{itemize}
(V.amortecer)
Definhar, extinguir-se. Cf. Filinto, I, 283.
\section{Emmosqueirar-se}
\begin{itemize}
\item {Grp. gram.:v. p.}
\end{itemize}
\begin{itemize}
\item {Utilização:Prov.}
\end{itemize}
\begin{itemize}
\item {Utilização:alent.}
\end{itemize}
Buscar sombra ou sitio asado, para descansar.
(Cp. \textunderscore mosqueiro\textunderscore )
\section{Emmostar}
\begin{itemize}
\item {Grp. gram.:v. t.}
\end{itemize}
\begin{itemize}
\item {Proveniência:(De \textunderscore mosto\textunderscore )}
\end{itemize}
Tornar doce (a uva).
Fazer sazonar (uvas).
Meter em mosto.
\section{Emmouquecer}
\begin{itemize}
\item {Grp. gram.:v. t.  e  i.}
\end{itemize}
\begin{itemize}
\item {Proveniência:(De \textunderscore mouco\textunderscore )}
\end{itemize}
O mesmo que \textunderscore ensurdecer\textunderscore .
\section{Emmudecer}
\begin{itemize}
\item {Grp. gram.:v. t.}
\end{itemize}
\begin{itemize}
\item {Grp. gram.:V. i.}
\end{itemize}
\begin{itemize}
\item {Proveniência:(Do lat. \textunderscore immutescere\textunderscore )}
\end{itemize}
Fazer calar.
Calar-se.
Tornar-se mudo.
Não têr som: \textunderscore a orchestra emmudeceu\textunderscore .
\section{Emmudecimento}
\begin{itemize}
\item {Grp. gram.:m.}
\end{itemize}
Acto de emmudecer.
\section{Emmugrecer}
\begin{itemize}
\item {Grp. gram.:v. i.}
\end{itemize}
Enferrujar-se. Cf. Júl. Castilho, \textunderscore Ermitério\textunderscore , 184.
\section{Emmuralhar}
\begin{itemize}
\item {Grp. gram.:v. t.}
\end{itemize}
Fortificar com muralhas. Cf. Filinto, I, 236.
\section{Emmurchecer}
\begin{itemize}
\item {Grp. gram.:v. t.}
\end{itemize}
\begin{itemize}
\item {Grp. gram.:V. i.}
\end{itemize}
Tornar murcho.
Fazer perder o frescor, o viço.
Murchar.
\section{Emoção}
\begin{itemize}
\item {Grp. gram.:f.}
\end{itemize}
\begin{itemize}
\item {Proveniência:(Do lat. \textunderscore emotus\textunderscore )}
\end{itemize}
Acto de deslocar.
Motim; desordem.--No sentido de commoção ou abalo moral, o termo é considerado gallicismo dispensável.
\section{Emocional}
\begin{itemize}
\item {Grp. gram.:adj.}
\end{itemize}
Que produz emoção; emotivo. Cf. Camillo, \textunderscore Carl. Rib.\textunderscore , 39.
\section{Emoliente}
\begin{itemize}
\item {Grp. gram.:m.  e  adj.}
\end{itemize}
\begin{itemize}
\item {Proveniência:(Lat. \textunderscore emolliens\textunderscore )}
\end{itemize}
Aquilo que emole (falando-se de substâncias medicamentosas).
\section{Emolir}
\begin{itemize}
\item {Grp. gram.:v. t.}
\end{itemize}
\begin{itemize}
\item {Proveniência:(Lat. \textunderscore emollire\textunderscore )}
\end{itemize}
Tornar mole, brando.
Desfazer a dureza de.
\section{Emolliente}
\begin{itemize}
\item {Grp. gram.:m.  e  adj.}
\end{itemize}
\begin{itemize}
\item {Proveniência:(Lat. \textunderscore emolliens\textunderscore )}
\end{itemize}
Aquillo que emolle (falando-se de substâncias medicamentosas).
\section{Emollir}
\begin{itemize}
\item {Grp. gram.:v. t.}
\end{itemize}
\begin{itemize}
\item {Proveniência:(Lat. \textunderscore emollire\textunderscore )}
\end{itemize}
Tornar molle, brando.
Desfazer a dureza de.
\section{Emolumento}
\begin{itemize}
\item {Grp. gram.:m.}
\end{itemize}
\begin{itemize}
\item {Proveniência:(Lat. \textunderscore emolumentum\textunderscore )}
\end{itemize}
Retribuição.
Ganho; proveito.
Gratificação.
Lucro eventual, além do rendimento habitual: \textunderscore os empregados da Alfândega recebem emolumentos\textunderscore .
\section{Emotivamente}
\begin{itemize}
\item {Grp. gram.:adv.}
\end{itemize}
\begin{itemize}
\item {Utilização:dispensável}
\end{itemize}
\begin{itemize}
\item {Utilização:Gal}
\end{itemize}
De modo emotivo.
Com emoção.
\section{Emotividade}
\begin{itemize}
\item {Grp. gram.:f.}
\end{itemize}
Qualidade de emotivo. Cf. Camillo, \textunderscore Vulcões\textunderscore , 60.
\section{Emotivo}
\begin{itemize}
\item {Grp. gram.:adj.}
\end{itemize}
\begin{itemize}
\item {Utilização:dispensável}
\end{itemize}
\begin{itemize}
\item {Utilização:Gal}
\end{itemize}
\begin{itemize}
\item {Proveniência:(Do lat. \textunderscore emotus\textunderscore )}
\end{itemize}
Que tem ou revela emoção.
\section{Empa}
\begin{itemize}
\item {Grp. gram.:f.}
\end{itemize}
\begin{itemize}
\item {Utilização:Ext.}
\end{itemize}
Acto de empar.
Estaca, a que se liga a vide ou em que se apoia a videira, o feijoeiro ou outras trepadeiras.
\section{Empacador}
\begin{itemize}
\item {Grp. gram.:adj.}
\end{itemize}
\begin{itemize}
\item {Utilização:Bras}
\end{itemize}
\begin{itemize}
\item {Utilização:Fig.}
\end{itemize}
Que empaca.
Teimoso, recalcitrante.
\section{Empacar}
\begin{itemize}
\item {Grp. gram.:v. t.}
\end{itemize}
(V.empacotar)
\section{Empacar}
\begin{itemize}
\item {Grp. gram.:v. t.}
\end{itemize}
\begin{itemize}
\item {Utilização:Bras}
\end{itemize}
Emperrar a cavalgadura.
(Cast. \textunderscore empacarse\textunderscore )
\section{Empachadamente}
\begin{itemize}
\item {Grp. gram.:adv.}
\end{itemize}
Com empacho.
\section{Empachamento}
\begin{itemize}
\item {Grp. gram.:m.}
\end{itemize}
O mesmo que \textunderscore empacho\textunderscore .
\section{Empachar}
\begin{itemize}
\item {Grp. gram.:v. t.}
\end{itemize}
\begin{itemize}
\item {Proveniência:(Do lat. hypoth. \textunderscore impactiare\textunderscore )}
\end{itemize}
Obstruir, encher muito, embaraçar, (o estômago).
Impedir, estorvar.
\section{Empacho}
\begin{itemize}
\item {Grp. gram.:m.}
\end{itemize}
Acto ou effeito de empachar.
Embaraço.
Estôrvo; obstrucção.
\section{Empachoso}
\begin{itemize}
\item {Grp. gram.:adj.}
\end{itemize}
\begin{itemize}
\item {Utilização:Fig.}
\end{itemize}
\begin{itemize}
\item {Proveniência:(De \textunderscore empachar\textunderscore )}
\end{itemize}
Que empacha.
Acanhado, timido.
\section{Empacotadeira}
\begin{itemize}
\item {Grp. gram.:f.}
\end{itemize}
O mesmo que \textunderscore empacotadôra\textunderscore .
\section{Empacotador}
\begin{itemize}
\item {Grp. gram.:adj.}
\end{itemize}
Que empacota.
\section{Empacotadôra}
\begin{itemize}
\item {Grp. gram.:f.}
\end{itemize}
\begin{itemize}
\item {Proveniência:(De \textunderscore empacotar\textunderscore )}
\end{itemize}
Máquina agrícola que, á mão ou impellida a vapor ou puxada por cavallos, serve para empacotar palha, feno, etc.
\section{Empacotamento}
\begin{itemize}
\item {Grp. gram.:m.}
\end{itemize}
Acto de empacotar.
\section{Empacotar}
\begin{itemize}
\item {Grp. gram.:v. t.}
\end{itemize}
Reunir em pacotes.
Enfardar; emmalar: \textunderscore empacotar livros\textunderscore .
\section{Empada}
\begin{itemize}
\item {Grp. gram.:f.}
\end{itemize}
\begin{itemize}
\item {Utilização:Fam.}
\end{itemize}
Pastel de massa, com recheio de carne ou peixe.
Pessôa importuna, que fatiga os outros com a sua presença ou com a sua conversa.
(Contr. de \textunderscore empanada\textunderscore ^1)
\section{Empadesar}
\begin{itemize}
\item {Grp. gram.:v. t.}
\end{itemize}
\begin{itemize}
\item {Proveniência:(De \textunderscore padês\textunderscore )}
\end{itemize}
(V.empavesar)
\section{Empador}
\begin{itemize}
\item {Grp. gram.:m.}
\end{itemize}
Aquelle que empa.
\section{Empadroar}
\begin{itemize}
\item {Grp. gram.:v. t.}
\end{itemize}
\begin{itemize}
\item {Utilização:Des.}
\end{itemize}
\begin{itemize}
\item {Proveniência:(De \textunderscore padrão\textunderscore )}
\end{itemize}
Escrever em padrão.
Incluir no rol das contribuições; alistar.
\section{Empáfia}
\begin{itemize}
\item {Grp. gram.:f.}
\end{itemize}
\begin{itemize}
\item {Grp. gram.:M.}
\end{itemize}
Altivez, soberba.
Orgulho vão.
Aquelle que tem empáfia.
\section{Empaiolar}
\begin{itemize}
\item {Grp. gram.:v. t.}
\end{itemize}
\begin{itemize}
\item {Utilização:Bras. do S}
\end{itemize}
Arrecadar em paiol.
\section{Empalação}
\begin{itemize}
\item {Grp. gram.:f.}
\end{itemize}
\begin{itemize}
\item {Proveniência:(De \textunderscore empalar\textunderscore )}
\end{itemize}
Supplício antigo, que consistia em espetar um condemnado pelo sesso em pau ou ferro agudo e deixá-lo assim exposto até morrer.
\section{Empalamado}
\begin{itemize}
\item {Grp. gram.:adj.}
\end{itemize}
\begin{itemize}
\item {Utilização:Pop.}
\end{itemize}
\begin{itemize}
\item {Utilização:Bras}
\end{itemize}
Que tem edemas.
Achacadiço.
Coberto de emplastros.
Pállido.
Que tem gordura balofa e descorada.
(Cp. \textunderscore emplasmado\textunderscore )
\section{Empalar}
\begin{itemize}
\item {Grp. gram.:v. t.}
\end{itemize}
\begin{itemize}
\item {Proveniência:(Do lat. \textunderscore palus\textunderscore )}
\end{itemize}
Applicar a empalação a.
\section{Empalecer}
\begin{itemize}
\item {Grp. gram.:v. t.}
\end{itemize}
(V.empalidecer)
\section{Empalego}
\begin{itemize}
\item {Grp. gram.:m.}
\end{itemize}
Antiga embarcação indiana.
\section{Empaletosado}
\begin{itemize}
\item {Grp. gram.:adj.}
\end{itemize}
\begin{itemize}
\item {Utilização:Bras. do N}
\end{itemize}
Vestido de paletó.
Decentemente vestido.
\section{Empalhação}
\begin{itemize}
\item {Grp. gram.:f.}
\end{itemize}
Acto de empalhar.
\section{Empalhadeira}
\begin{itemize}
\item {Grp. gram.:f.}
\end{itemize}
Mulher, que empalha cadeiras; palheireira.
\section{Empalhamento}
\begin{itemize}
\item {Grp. gram.:m.}
\end{itemize}
O mesmo que \textunderscore empalhação\textunderscore .
\section{Empalhar}
\begin{itemize}
\item {Grp. gram.:v. t.}
\end{itemize}
\begin{itemize}
\item {Utilização:Fam.}
\end{itemize}
\begin{itemize}
\item {Proveniência:(De \textunderscore palha\textunderscore )}
\end{itemize}
Meter no palheiro (a palha).
Cobrir, forrar com palhas ou vimes.
Dispor sôbre palha.
Empalheirar.
Adiar a resolução de (um negócio).
Empatar, entreter, com motivos fúteis.
Embalsamar (animaes), enchendo-os de palha ou de outra substância.
\section{Empalheirar}
\begin{itemize}
\item {Grp. gram.:v. t.}
\end{itemize}
\begin{itemize}
\item {Proveniência:(De \textunderscore palheiro\textunderscore )}
\end{itemize}
Pôr assentos de palhinha em.
Recolher em palheiro.
\section{Empalidecer}
\begin{itemize}
\item {Grp. gram.:v. i.}
\end{itemize}
\begin{itemize}
\item {Proveniência:(De \textunderscore pálido\textunderscore )}
\end{itemize}
Tornar-se pálido.
Enfiar.
Amareceler.
\section{Empallecer}
\begin{itemize}
\item {Grp. gram.:v. t.}
\end{itemize}
(V.empallidecer)
\section{Empallidecer}
\begin{itemize}
\item {Grp. gram.:v. i.}
\end{itemize}
\begin{itemize}
\item {Proveniência:(De \textunderscore pállido\textunderscore )}
\end{itemize}
Tornar-se pállido.
Enfiar.
Amareceler.
\section{Empalma}
\begin{itemize}
\item {Grp. gram.:f.}
\end{itemize}
\begin{itemize}
\item {Utilização:Prov.}
\end{itemize}
\begin{itemize}
\item {Proveniência:(De \textunderscore empalmar\textunderscore )}
\end{itemize}
Córte ou chanfradura que se faz numa tábua, para esta se ajustar no córte ou chanfradura de outra. (Colhido no Fundão)
\section{Empalmação}
\begin{itemize}
\item {Grp. gram.:f.}
\end{itemize}
Acto de empalmar.
\section{Empalmadela}
\begin{itemize}
\item {Grp. gram.:f.}
\end{itemize}
O mesmo que \textunderscore empalmação\textunderscore .
\section{Empalmador}
\begin{itemize}
\item {Grp. gram.:m.  e  adj.}
\end{itemize}
O que empalma.
\section{Empalmar}
\begin{itemize}
\item {Grp. gram.:v. t.}
\end{itemize}
\begin{itemize}
\item {Utilização:Fam.}
\end{itemize}
\begin{itemize}
\item {Utilização:Prov.}
\end{itemize}
\begin{itemize}
\item {Proveniência:(De \textunderscore palma\textunderscore )}
\end{itemize}
Esconder na palma da mão.
Furtar com destreza: \textunderscore empalmar um relógio\textunderscore .
Fazer empalma em.
\section{Empambado}
\begin{itemize}
\item {Grp. gram.:adj.}
\end{itemize}
\begin{itemize}
\item {Utilização:Bras. do N}
\end{itemize}
Pállido, anêmico.
\section{Empampanar}
\begin{itemize}
\item {Grp. gram.:v. t.}
\end{itemize}
Cobrir ou coroar de pâmpanos.
\section{Empana}
\textunderscore m.\textunderscore  (\textunderscore T. da Nazareth\textunderscore )
Um dos homens que levantam a rêde.
\section{Empanada}
\begin{itemize}
\item {Grp. gram.:f.}
\end{itemize}
Empada grande.
(Cast. \textunderscore empanada\textunderscore )
\section{Empanada}
\begin{itemize}
\item {Grp. gram.:f.}
\end{itemize}
\begin{itemize}
\item {Utilização:Bras. de Pernambuco}
\end{itemize}
\begin{itemize}
\item {Utilização:Prov.}
\end{itemize}
\begin{itemize}
\item {Utilização:trasm.}
\end{itemize}
\begin{itemize}
\item {Proveniência:(De \textunderscore pano\textunderscore )}
\end{itemize}
Caixilho de janela, tapado com pano ou papel em vez de vidro.
Estore.
Tôldo das casas commerciaes.
Qualquer coisa volumosa, que se leva tapada debaixo do braço. (Colhido em Lagoaça)
\section{Empanadilha}
\begin{itemize}
\item {Grp. gram.:f.}
\end{itemize}
Pequena empada.
(Dem. de \textunderscore empanada\textunderscore ^1)
\section{Empanamento}
\begin{itemize}
\item {Grp. gram.:m.}
\end{itemize}
Acto ou effeito de empanar.
\section{Empanar}
\begin{itemize}
\item {Grp. gram.:v. t.}
\end{itemize}
\begin{itemize}
\item {Utilização:Fig.}
\end{itemize}
\begin{itemize}
\item {Proveniência:(De \textunderscore pano\textunderscore )}
\end{itemize}
Cobrir com panos.
Embaciar; obscurecer.
Deslustrar: \textunderscore empanar a fama de alguém\textunderscore .
\section{Empancar}
\begin{itemize}
\item {Grp. gram.:v. t.}
\end{itemize}
\begin{itemize}
\item {Proveniência:(De \textunderscore panca\textunderscore )}
\end{itemize}
Segurar com panca.
Suster.
Vedar.
Represar.
Obstruír.
Empachar; enfartar.
\section{Empandeirado}
\begin{itemize}
\item {Grp. gram.:adj.}
\end{itemize}
\begin{itemize}
\item {Utilização:Gír.}
\end{itemize}
\begin{itemize}
\item {Proveniência:(De \textunderscore empandeirar\textunderscore )}
\end{itemize}
Preso.
\section{Empandeiramento}
\begin{itemize}
\item {Grp. gram.:m.}
\end{itemize}
Empacho.
Acto ou effeito de empandeirar.
\section{Empandeirar}
\begin{itemize}
\item {Grp. gram.:v. t.}
\end{itemize}
\begin{itemize}
\item {Utilização:Fam.}
\end{itemize}
\begin{itemize}
\item {Utilização:Gír.}
\end{itemize}
\begin{itemize}
\item {Proveniência:(De \textunderscore pando\textunderscore )}
\end{itemize}
Enfunar (velas de navio).
Enfartar.
Embair, ludibriar.
Esbanjar.
Matar.
\section{Empandilhar}
\begin{itemize}
\item {Grp. gram.:v. t.}
\end{itemize}
\begin{itemize}
\item {Grp. gram.:V. p.}
\end{itemize}
\begin{itemize}
\item {Proveniência:(De \textunderscore pandilhar\textunderscore )}
\end{itemize}
Conluiar-se com, para defraudar ou roubar.
Furtar com destreza.
Combinar-se com outrem, para roubar, jogando.
\section{Empandinar}
\begin{itemize}
\item {Grp. gram.:v. t.}
\end{itemize}
\begin{itemize}
\item {Proveniência:(De \textunderscore pando\textunderscore )}
\end{itemize}
Tornar pando; empanzinar.
\section{Empaneirar}
\begin{itemize}
\item {Grp. gram.:v. t.}
\end{itemize}
\begin{itemize}
\item {Utilização:Bras. do N}
\end{itemize}
Meter (cereaes) em paneiro ou cesto, forrado de fôlhas.
\section{Empanque}
\begin{itemize}
\item {Grp. gram.:m.}
\end{itemize}
\begin{itemize}
\item {Proveniência:(De \textunderscore empancar\textunderscore )}
\end{itemize}
Qualquer substância, empregada na vedação das juntas das máquinas.
\section{Empantanar}
\begin{itemize}
\item {Grp. gram.:v. t.}
\end{itemize}
Tornar pantanoso.
Meter num pântano.
\section{Empantufar-se}
\begin{itemize}
\item {Grp. gram.:v. p.}
\end{itemize}
\begin{itemize}
\item {Utilização:Fig.}
\end{itemize}
Calçar pantufos.
Mostrar-se orgulhoso.
Têr soberba.
\section{Empanturrar}
\begin{itemize}
\item {Grp. gram.:v. t.}
\end{itemize}
\begin{itemize}
\item {Proveniência:(De \textunderscore panturra\textunderscore )}
\end{itemize}
O mesmo que \textunderscore empachar\textunderscore .
\section{Empanzinador}
\begin{itemize}
\item {Grp. gram.:m.}
\end{itemize}
Aquelle que empanzina.
Aquelle que embaça ou illude.
\section{Empanzinamento}
\begin{itemize}
\item {Grp. gram.:m.}
\end{itemize}
Acto ou effeito de empanzinar.
Enfartamento.
\section{Empanzinar}
\begin{itemize}
\item {Grp. gram.:v. t.}
\end{itemize}
\begin{itemize}
\item {Utilização:Pop.}
\end{itemize}
Empachar.
Embaçar.
Empanturrar.
Enfartar.
Causar surpresa desagradável a.
Illudir.
(Talvez do rad. de \textunderscore pança\textunderscore )
\section{Empapagem}
\begin{itemize}
\item {Grp. gram.:f.}
\end{itemize}
Acto de empapar.
\section{Empapar}
\begin{itemize}
\item {Grp. gram.:v. t.}
\end{itemize}
Cobrir de papas.
Tornar molle, fazendo penetrar um liquido em.
Ensopar; encharcar.
Tornar froixo ou pouco violento o embate ou pancada de.
Tornar inoffensivo o choque de:«\textunderscore ...algodão para empapar o arrojo das bombardas\textunderscore ». Filinto, \textunderscore D. Man.\textunderscore , I, 278.
\section{Empapar}
\begin{itemize}
\item {Grp. gram.:v. i.}
\end{itemize}
Chegar a muleta ou o capote á cabeça do toiro, para que este não desvie a vista para outro objecto.
(Liga-se a \textunderscore papa\textunderscore ^4?)
\section{Empapar}
\begin{itemize}
\item {Grp. gram.:v. t.}
\end{itemize}
\begin{itemize}
\item {Utilização:Bras}
\end{itemize}
Encher o papo (a gallinha).
\section{Empapeladamente}
\begin{itemize}
\item {Grp. gram.:adv.}
\end{itemize}
Á maneira do que se empapela.
\section{Empapelador}
\begin{itemize}
\item {Grp. gram.:m.}
\end{itemize}
Operário, encarregado do empapêlo, em certas fábricas.
\section{Empapelamento}
\begin{itemize}
\item {Grp. gram.:m.}
\end{itemize}
Acto ou effeito de empapelar.
\section{Empapelar}
\begin{itemize}
\item {Grp. gram.:v. t.}
\end{itemize}
\begin{itemize}
\item {Utilização:Fig.}
\end{itemize}
Embrulhar em papel.
Resguardar com desvelo.
Agasalhar, apaparicar.
\section{Empapêlo}
\begin{itemize}
\item {Grp. gram.:m.}
\end{itemize}
Invólucro de papel.
Acto de empapelar tabaco, nas respectivas fábricas.
\section{Empapuçar-se}
\begin{itemize}
\item {Grp. gram.:v. p.}
\end{itemize}
Tornar-se opado.
Tender para hydrópico.
Inchar, tornar-se papudo.
(Talvez por \textunderscore empapoíçar-se\textunderscore , de \textunderscore papoiço\textunderscore )
\section{Empaquifado}
\begin{itemize}
\item {Grp. gram.:adj.}
\end{itemize}
Que tem paquifes. Cf. Arn. Gama, \textunderscore Últ. Dona\textunderscore , 255.
\section{Empar}
\begin{itemize}
\item {Grp. gram.:v. t.}
\end{itemize}
\begin{itemize}
\item {Proveniência:(De \textunderscore em...\textunderscore  + lat. \textunderscore palus\textunderscore , pau. Cp. \textunderscore empalar\textunderscore  &lt; \textunderscore empaar\textunderscore  &lt; \textunderscore empar\textunderscore )}
\end{itemize}
Suster e ligar ás varas, estacas ou caniçados (a parreira, os feijoeiros, etc.).
\section{Emparaisar}
\begin{itemize}
\item {fónica:para-i}
\end{itemize}
\begin{itemize}
\item {Grp. gram.:v. i.}
\end{itemize}
\begin{itemize}
\item {Utilização:Bras}
\end{itemize}
\begin{itemize}
\item {Utilização:Neol.}
\end{itemize}
\begin{itemize}
\item {Grp. gram.:V. t.}
\end{itemize}
\begin{itemize}
\item {Utilização:Fig.}
\end{itemize}
Entrar no paraíso.
Pôr no paraíso.
Tornar muito feliz.
\section{Emparamento}
\begin{itemize}
\item {Grp. gram.:m.}
\end{itemize}
\begin{itemize}
\item {Utilização:Ant.}
\end{itemize}
Enfeite, adôrno.
(Cp. \textunderscore paramento\textunderscore )
\section{Emparar}
\begin{itemize}
\item {Grp. gram.:v. t.}
\end{itemize}
\begin{itemize}
\item {Utilização:Des.}
\end{itemize}
O mesmo que \textunderscore amparar\textunderscore .
\section{Emparcar}
\begin{itemize}
\item {Grp. gram.:v. t.}
\end{itemize}
\begin{itemize}
\item {Proveniência:(De \textunderscore parque\textunderscore )}
\end{itemize}
Alojar (artilharia).
\section{Emparceirar}
\begin{itemize}
\item {Grp. gram.:v. t.}
\end{itemize}
\begin{itemize}
\item {Proveniência:(De \textunderscore parceiro\textunderscore )}
\end{itemize}
Tornar parceiro.
Unir em parçaria.
\section{Empardecer}
\begin{itemize}
\item {Grp. gram.:v. i.}
\end{itemize}
\begin{itemize}
\item {Utilização:Prov.}
\end{itemize}
\begin{itemize}
\item {Utilização:beir.}
\end{itemize}
Tornar-se pardo.
Entardecer, ir escurecendo (o dia).
\section{Empardecido}
\begin{itemize}
\item {Grp. gram.:adj.}
\end{itemize}
Que empardeceu; escuro.
\section{Emparedamento}
\begin{itemize}
\item {Grp. gram.:m.}
\end{itemize}
Acto ou effeito de emparedar.
\section{Emparedar}
\begin{itemize}
\item {Grp. gram.:v. t.}
\end{itemize}
\begin{itemize}
\item {Grp. gram.:V. p.}
\end{itemize}
Encerrar entre paredes.
Enclausurar.
Segurar com paredes.
Ladear de paredes.
Aprumar-se, como parede.
\section{Emparelhado}
\begin{itemize}
\item {Grp. gram.:adj.}
\end{itemize}
\begin{itemize}
\item {Proveniência:(De \textunderscore emparelhar\textunderscore )}
\end{itemize}
Jungido.
Igualado.
Diz-se dos versos, que rimam, dois a dois, formando parelha.
Diz-se da rima dêsses versos.
\section{Emparelhamento}
\begin{itemize}
\item {Grp. gram.:m.}
\end{itemize}
Acto ou effeito de emparelhar.
\section{Emparelhar}
\begin{itemize}
\item {Grp. gram.:v. t.}
\end{itemize}
\begin{itemize}
\item {Grp. gram.:V. i.}
\end{itemize}
\begin{itemize}
\item {Proveniência:(De \textunderscore parelha\textunderscore )}
\end{itemize}
Collocar a par.
Jungir: \textunderscore emparelhar cavallos\textunderscore .
Igualar.
Estar a par de outro.
Sêr igual: \textunderscore Camões emparelha com Vergílio\textunderscore .
Defrontar.
\section{Emparentar}
\begin{itemize}
\item {Grp. gram.:v. t.}
\end{itemize}
Tornar parente ou semelhante:«\textunderscore que emparente co'o céu tartárea alliança\textunderscore ». Filinto, XV, 208.
\section{Empariado}
\begin{itemize}
\item {Grp. gram.:adj.}
\end{itemize}
O mesmo que \textunderscore emparelhado\textunderscore , (falando-se de versos). Cf. Macedo, \textunderscore Burros\textunderscore , 134.
\section{Emparo}
\textunderscore m.\textunderscore  (e der.)
(V. \textunderscore amparo\textunderscore , etc.)
\section{Emparrar}
\begin{itemize}
\item {Grp. gram.:v. t.}
\end{itemize}
\begin{itemize}
\item {Grp. gram.:V. i.  e  p.}
\end{itemize}
Cobrir de parras.
Criar parras, cobrir-se de parras.
\section{Emparreirar}
\begin{itemize}
\item {Grp. gram.:v. t.}
\end{itemize}
Cobrir de parreiras.
Suspender em estacas ou caniçados, em fórma de parreira.
\section{Emparvecer}
\begin{itemize}
\item {Grp. gram.:v. t.  e  i.}
\end{itemize}
O mesmo que \textunderscore emparvoecer\textunderscore . Cf. Eça, \textunderscore P. Basílio\textunderscore , 285.
\section{Emparvoecer}
\begin{itemize}
\item {fónica:vo-e}
\end{itemize}
\begin{itemize}
\item {Grp. gram.:v. t.}
\end{itemize}
\begin{itemize}
\item {Grp. gram.:V. i.}
\end{itemize}
Tornar parvo.
Tornar-se parvo.
\section{Empasma}
\begin{itemize}
\item {Grp. gram.:m.}
\end{itemize}
\begin{itemize}
\item {Proveniência:(Gr. \textunderscore empasma\textunderscore )}
\end{itemize}
Pó, com que se enxuga o suor ou se lhe modifica o cheiro.
\section{Empasmar}
\begin{itemize}
\item {Grp. gram.:v. i.}
\end{itemize}
\begin{itemize}
\item {Utilização:Bras. de Minas}
\end{itemize}
O mesmo que \textunderscore pasmar\textunderscore .
\section{Empassocar}
\begin{itemize}
\item {Grp. gram.:v. t.}
\end{itemize}
\begin{itemize}
\item {Utilização:Bras}
\end{itemize}
Encher de passoca; enfartar.
\section{Empastadamente}
\begin{itemize}
\item {Grp. gram.:adv.}
\end{itemize}
Á maneira de pasta.
\section{Empastador}
\begin{itemize}
\item {Grp. gram.:m.}
\end{itemize}
Apparelho de empastar, em certas fábricas. Cf. \textunderscore Inquér. Industr.\textunderscore , p. II, v. I, 119.
\section{Empastamento}
\begin{itemize}
\item {Grp. gram.:m.}
\end{itemize}
Acto ou effeito de empastar.
\section{Empastar}
\begin{itemize}
\item {Grp. gram.:v. t.}
\end{itemize}
\begin{itemize}
\item {Proveniência:(De \textunderscore pasta\textunderscore )}
\end{itemize}
Converter em pasta.
Ligar papel ou tecidos com massa.
Carregar as côres em: \textunderscore empastar um quadro\textunderscore .
Pôr as primeiras tintas em (um quadro), para se esbaterem depois.
\section{Empaste}
\begin{itemize}
\item {Grp. gram.:m.}
\end{itemize}
Acto ou effeito de empastar.
\section{Empastelar}
\begin{itemize}
\item {Grp. gram.:v. t.}
\end{itemize}
\begin{itemize}
\item {Proveniência:(De \textunderscore pastel\textunderscore )}
\end{itemize}
Amontoar confusamente (caracteres typográphicos).
\section{Empata}
\begin{itemize}
\item {Grp. gram.:f.}
\end{itemize}
\begin{itemize}
\item {Utilização:Des.}
\end{itemize}
\begin{itemize}
\item {Grp. gram.:M.}
\end{itemize}
\begin{itemize}
\item {Utilização:Deprec.}
\end{itemize}
\begin{itemize}
\item {Proveniência:(De \textunderscore empatar\textunderscore )}
\end{itemize}
O mesmo que \textunderscore confisco\textunderscore .
Aquelle que embaraça o regular andamento de um negócio.
\section{Empatar}
\begin{itemize}
\item {Grp. gram.:v. t.}
\end{itemize}
\begin{itemize}
\item {Grp. gram.:V. i.}
\end{itemize}
\begin{itemize}
\item {Utilização:Des.}
\end{itemize}
\begin{itemize}
\item {Proveniência:(Do lat. \textunderscore in\textunderscore  + \textunderscore pactus\textunderscore )}
\end{itemize}
Sustar.
Embaraçar.
Tolher o seguimento de.
Collocar na situação de não dar lucro immediato: \textunderscore empatar capital\textunderscore .
Igualar (votações oppostas).
Tornar indeciso.
Achar obstáculo.
Fazer embaraço.
\section{Empate}
\begin{itemize}
\item {Grp. gram.:m.}
\end{itemize}
Acto ou effeito de empatar: \textunderscore houve empate na votação\textunderscore .
\section{Empaturrado}
\begin{itemize}
\item {Grp. gram.:adj.}
\end{itemize}
\begin{itemize}
\item {Utilização:Pop.}
\end{itemize}
Que tem o estômago repleto.
Enfartado.
\section{Empaturramento}
\begin{itemize}
\item {Grp. gram.:m.}
\end{itemize}
Acto ou effeito de empaturrar. Cf. Arn. Gama, \textunderscore Segr. do Abb.\textunderscore , 185; \textunderscore Últ. Dona\textunderscore , 47 e 118.
\section{Empaturrar}
\begin{itemize}
\item {Grp. gram.:v. t.}
\end{itemize}
\begin{itemize}
\item {Utilização:Pop.}
\end{itemize}
O mesmo que \textunderscore empanturrar\textunderscore .
\section{Empavear}
\begin{itemize}
\item {Grp. gram.:v. t.}
\end{itemize}
\begin{itemize}
\item {Utilização:Prov.}
\end{itemize}
\begin{itemize}
\item {Utilização:beir.}
\end{itemize}
\begin{itemize}
\item {Grp. gram.:V. i.}
\end{itemize}
Dispor ou reunir em paveias (o mato que se roça para estrume ou para lenha do forno).
Fazer paveias.
\section{Empavesar}
\begin{itemize}
\item {Grp. gram.:v. t.}
\end{itemize}
\begin{itemize}
\item {Grp. gram.:V. i.  e  p.}
\end{itemize}
\begin{itemize}
\item {Utilização:Fam.}
\end{itemize}
\begin{itemize}
\item {Proveniência:(De \textunderscore pavês\textunderscore )}
\end{itemize}
Resguardar com pavês.
Enfeitar (navios) com bandeiras, pavilhões, etc.
Cobrir-se ou guarnecer-se com paveses.
Tornar-se arrogante, fátuo.
Ensoberbecer-se.
\section{Empavonar}
\begin{itemize}
\item {Grp. gram.:v. t.}
\end{itemize}
\begin{itemize}
\item {Proveniência:(Do lat. \textunderscore pavo\textunderscore )}
\end{itemize}
Encher de vaidade.
Tornar inchado, vaidoso, como um pavão.
\section{Empavorir}
\begin{itemize}
\item {Grp. gram.:v. t.}
\end{itemize}
(V.espavorir)
\section{Empear}
\begin{itemize}
\item {Grp. gram.:v. t.}
\end{itemize}
\begin{itemize}
\item {Utilização:Des.}
\end{itemize}
\begin{itemize}
\item {Proveniência:(De \textunderscore pear\textunderscore )}
\end{itemize}
Meter na eira (os bois), para debulharem as espigas depois de separadas da palha.
\section{Empecadado}
\begin{itemize}
\item {Grp. gram.:adj.}
\end{itemize}
Que anda em pecado; contaminado pelo pecado. Cf. Camillo, \textunderscore Myst. de Fafe\textunderscore , 136.
\section{Empeçar}
\begin{itemize}
\item {Grp. gram.:v. t.}
\end{itemize}
\begin{itemize}
\item {Grp. gram.:V. i.}
\end{itemize}
\begin{itemize}
\item {Proveniência:(De \textunderscore empêço\textunderscore )}
\end{itemize}
Empecer.
Enredar.
Fazer estôrvo a.
Esbarrar.
Tropeçar; enredar-se.
\section{Empeçar}
\begin{itemize}
\item {Grp. gram.:v. t.  e  i.}
\end{itemize}
\begin{itemize}
\item {Utilização:Prov.}
\end{itemize}
\begin{itemize}
\item {Utilização:trasm.}
\end{itemize}
\begin{itemize}
\item {Utilização:minh.}
\end{itemize}
Começar.
(Cp. cast. \textunderscore empezar\textunderscore )
\section{Empeccadado}
\begin{itemize}
\item {Grp. gram.:adj.}
\end{itemize}
Que anda em peccado; contaminado pelo peccado. Cf. Camillo, \textunderscore Myst. de Fafe\textunderscore , 136.
\section{Empecer}
\begin{itemize}
\item {Grp. gram.:v. t.}
\end{itemize}
\begin{itemize}
\item {Grp. gram.:V. i.}
\end{itemize}
\begin{itemize}
\item {Proveniência:(Lat. hyp. \textunderscore impedescere\textunderscore , de \textunderscore impedire\textunderscore )}
\end{itemize}
Causar obstáculo a.
Impedir.
Estorvar.
Prejudicar.
Fazer obstáculo.
\section{Empecilhar}
\begin{itemize}
\item {Grp. gram.:v. t.}
\end{itemize}
Servir de empecilho a; embaraçar. Cf. Filinto, XVIII, 248.
\section{Empecilho}
\begin{itemize}
\item {Grp. gram.:m.}
\end{itemize}
\begin{itemize}
\item {Grp. gram.:Adj.}
\end{itemize}
Obstáculo.
Aquillo que empece.
Que empece, que embaraça. Cf. Filinto, \textunderscore D. Man.\textunderscore , II, 301.
\section{Empecimento}
\begin{itemize}
\item {Grp. gram.:m.}
\end{itemize}
\begin{itemize}
\item {Utilização:Des.}
\end{itemize}
Acção de empecer.
\section{Empecinado}
\begin{itemize}
\item {Grp. gram.:adj.}
\end{itemize}
\begin{itemize}
\item {Proveniência:(T. cast.)}
\end{itemize}
(?):«\textunderscore ...elle é seu chefe empecinado\textunderscore ». Macedo, \textunderscore Burros\textunderscore , 77.
\section{Empecível}
\begin{itemize}
\item {Grp. gram.:adj.}
\end{itemize}
\begin{itemize}
\item {Proveniência:(De \textunderscore empecer\textunderscore )}
\end{itemize}
Que empece.
\section{Empecivo}
\begin{itemize}
\item {Grp. gram.:adj.}
\end{itemize}
O mesmo que \textunderscore empecível\textunderscore .
\section{Empêço}
\begin{itemize}
\item {Grp. gram.:m.}
\end{itemize}
Aquillo que empece.
Obstáculo.
(Relaciona-se provavelmente com o lat. \textunderscore impeditio\textunderscore )
\section{Empeçonhamento}
\begin{itemize}
\item {Grp. gram.:m.}
\end{itemize}
Acto ou effeito de empeçonhar. Cf. Filinto, \textunderscore D. Man.\textunderscore , I, 149.
\section{Empeçonhar}
\begin{itemize}
\item {Grp. gram.:v. t.}
\end{itemize}
Dar peçonha a.
Envenenar.
Corromper.
Tomar em mau sentido; desvirtuar.
\section{Empeçonhentar}
\begin{itemize}
\item {Grp. gram.:v. t.}
\end{itemize}
\begin{itemize}
\item {Proveniência:(De \textunderscore peçonhento\textunderscore )}
\end{itemize}
O mesmo que \textunderscore empeçonhar\textunderscore .
\section{Empedêmia}
\begin{itemize}
\item {Grp. gram.:f.}
\end{itemize}
\begin{itemize}
\item {Utilização:Prov.}
\end{itemize}
\begin{itemize}
\item {Utilização:alg.}
\end{itemize}
O mesmo que \textunderscore epidemia\textunderscore .
\section{Empedernecer}
\begin{itemize}
\item {Grp. gram.:v. t.  e  i.}
\end{itemize}
(V.empedernir)
\section{Empedernido}
\begin{itemize}
\item {Grp. gram.:adj.}
\end{itemize}
\begin{itemize}
\item {Utilização:Fig.}
\end{itemize}
Convertido em pedra.
Endurecido.
Inflexível.
Insensível: \textunderscore coração empedernido\textunderscore .
\section{Empedernir}
\begin{itemize}
\item {Grp. gram.:v. t.}
\end{itemize}
\begin{itemize}
\item {Utilização:Fig.}
\end{itemize}
Petrificar.
Endurecer, fazer duro como a pedra.
Tornar insensível.
(Por \textunderscore empedrenir\textunderscore , de \textunderscore pedra\textunderscore , com um suf. arbitrário)
\section{Empedrado}
\begin{itemize}
\item {Grp. gram.:m.}
\end{itemize}
\begin{itemize}
\item {Grp. gram.:Adj.}
\end{itemize}
\begin{itemize}
\item {Utilização:Med.}
\end{itemize}
\begin{itemize}
\item {Proveniência:(De \textunderscore empedrar\textunderscore )}
\end{itemize}
Faixa de estrada macadamizada, limitada lateralmente pelas bermas, e composta de pedra britada e comprimida.
Chão calcetado.
Revestido ou calçado de pedras.
Resistente como a pedra.
Que tem concreções calcáreas: \textunderscore tem os rins empedrados\textunderscore .
\section{Empedrador}
\begin{itemize}
\item {Grp. gram.:m.}
\end{itemize}
\begin{itemize}
\item {Utilização:Des.}
\end{itemize}
\begin{itemize}
\item {Proveniência:(De \textunderscore empedrar\textunderscore )}
\end{itemize}
Aquelle que empedra.
Calceteiro.
\section{Empedradura}
\begin{itemize}
\item {Grp. gram.:f.}
\end{itemize}
Acto de empedrar.
Moléstia, nos cascos de cavalgadura.
\section{Empedramento}
\begin{itemize}
\item {Grp. gram.:m.}
\end{itemize}
Acto ou effeito de empedrar.
\section{Empedrar}
\begin{itemize}
\item {Grp. gram.:v. t.}
\end{itemize}
\begin{itemize}
\item {Utilização:Fig.}
\end{itemize}
\begin{itemize}
\item {Grp. gram.:V. i.  e  p.}
\end{itemize}
\begin{itemize}
\item {Proveniência:(De \textunderscore pedra\textunderscore )}
\end{itemize}
Calçar com pedras; lagear.
Revestir de pedras; calcetar.
Tornar insensivel, deshumano.
Petrificar-se, empedernir-se.
\section{Empedrenir}
\begin{itemize}
\item {Grp. gram.:v. t.}
\end{itemize}
O mesmo ou melhor que \textunderscore empedernir\textunderscore . Cf. \textunderscore Luz e Calor\textunderscore , 411.
\section{Empègar}
\begin{itemize}
\item {Grp. gram.:v. t.}
\end{itemize}
\begin{itemize}
\item {Grp. gram.:V. p.}
\end{itemize}
Meter no pégo.
Engolfar.
Fazer-se ao mar.
\section{Empeirar}
\begin{itemize}
\item {Grp. gram.:v. t.}
\end{itemize}
\begin{itemize}
\item {Utilização:Prov.}
\end{itemize}
Pôr no tear (a teia), para começar a tecedura.
(Parece-me outra fórma de \textunderscore apeirar\textunderscore )
\section{Empeiticação}
\begin{itemize}
\item {Grp. gram.:f.}
\end{itemize}
Acto de empeiticar.
\section{Empeiticar}
\begin{itemize}
\item {Grp. gram.:v. t.}
\end{itemize}
\begin{itemize}
\item {Utilização:Bras. do N}
\end{itemize}
Teimar: embirrar.
\section{Empejar}
\begin{itemize}
\item {Grp. gram.:v. t.}
\end{itemize}
\begin{itemize}
\item {Utilização:Prov.}
\end{itemize}
\begin{itemize}
\item {Utilização:minh.}
\end{itemize}
Desviar, por meio de pejeiro, a água de (rêgo, cale de moínho, etc.).
(Cp. \textunderscore pejeiro\textunderscore )
\section{Empelamar}
\begin{itemize}
\item {Grp. gram.:v. t.}
\end{itemize}
Lançar no pelame ou nos cortumes (coiros ou peles).
\section{Empelicar}
\begin{itemize}
\item {Grp. gram.:v. t.}
\end{itemize}
\begin{itemize}
\item {Proveniência:(De \textunderscore pelica\textunderscore )}
\end{itemize}
Converter em pelica, cortindo.
Cobrir com pelica.
\section{Empelicular}
\begin{itemize}
\item {Grp. gram.:v. t.}
\end{itemize}
Cobrir de pelicula.
\section{Empellamar}
\begin{itemize}
\item {Grp. gram.:v. t.}
\end{itemize}
Lançar no pellame ou nos cortumes (coiros ou pelles).
\section{Empellicar}
\begin{itemize}
\item {Grp. gram.:v. t.}
\end{itemize}
\begin{itemize}
\item {Proveniência:(De \textunderscore pellica\textunderscore )}
\end{itemize}
Converter em pellica, cortindo.
Cobrir com pellica.
\section{Empellicular}
\begin{itemize}
\item {Grp. gram.:v. t.}
\end{itemize}
Cobrir de pellicula.
\section{Empelo}
\begin{itemize}
\item {fónica:pê}
\end{itemize}
\begin{itemize}
\item {Grp. gram.:m.}
\end{itemize}
\begin{itemize}
\item {Utilização:Des.}
\end{itemize}
Pedaço de massa, antes de convertida em pão, para entrar no forno.
Porção de ervas cozidas, de que se faz o esparregado.
(Talvez de \textunderscore péla\textunderscore )
\section{Empelota}
\begin{itemize}
\item {Grp. gram.:f.}
\end{itemize}
Pequena âmbula.
Redoma.
(Por \textunderscore ampullota\textunderscore , do lat. \textunderscore ampulla\textunderscore )
\section{Empelotar}
\begin{itemize}
\item {Grp. gram.:v. t.}
\end{itemize}
\begin{itemize}
\item {Proveniência:(De \textunderscore pelota\textunderscore )}
\end{itemize}
Reduzir (cera) a pelotas ou pequenas bolas, em trabalhos de cirieiro:«\textunderscore banca de empellotar\textunderscore ». \textunderscore Inquér. Industr.\textunderscore , P. II, l. 1.^o, 298.
\section{Empena}
\begin{itemize}
\item {Grp. gram.:f.}
\end{itemize}
\begin{itemize}
\item {Utilização:Carp.}
\end{itemize}
\begin{itemize}
\item {Proveniência:(De \textunderscore empenar\textunderscore )}
\end{itemize}
Empeno.
Peça de madeira, que vai do frechal á extremidade do pau de fileira.
Paredes lateraes de um edifício, quando se prolongam até á linha culminante delle.
\section{Empenachar}
\begin{itemize}
\item {Grp. gram.:v. t.}
\end{itemize}
\begin{itemize}
\item {Utilização:Ext.}
\end{itemize}
Pôr penacho em.
Adornar com penachos.
Tornar garrido, enfeitar.
\section{Empenamento}
\begin{itemize}
\item {Grp. gram.:m.}
\end{itemize}
Acto ou effeito de empenar.
\section{Empenar}
\begin{itemize}
\item {Grp. gram.:v. t.}
\end{itemize}
\begin{itemize}
\item {Grp. gram.:V. i.}
\end{itemize}
Fazer torcer, curvar com a humidade ou o calor (a madeira).
Curvar-se, torcer-se, (a madeira), sob a acção da humidade ou do calor.
Desviar-se da direcção da linha de prumo.
(Suppõem alguns que se relaciona com o lat. \textunderscore pina\textunderscore . Não se relacionará antes com o cast. \textunderscore peinar\textunderscore ?)
\section{Empenar}
\begin{itemize}
\item {Grp. gram.:v. t.}
\end{itemize}
\begin{itemize}
\item {Grp. gram.:V. i.}
\end{itemize}
\begin{itemize}
\item {Grp. gram.:V. p.}
\end{itemize}
\begin{itemize}
\item {Proveniência:(De \textunderscore pena\textunderscore )}
\end{itemize}
Cobrir ou enfeitar de penas.
Criar penas, emplumar-se.
Enfeitar-se de penas.
Enfeitar-se.
\section{Empendículo}
\begin{itemize}
\item {Grp. gram.:m.}
\end{itemize}
\begin{itemize}
\item {Utilização:Prov.}
\end{itemize}
\begin{itemize}
\item {Utilização:alg.}
\end{itemize}
Empecilho, obstáculo.
(Cp. \textunderscore appendículo\textunderscore )
\section{Empenetrar}
\begin{itemize}
\item {Grp. gram.:v. i.}
\end{itemize}
\begin{itemize}
\item {Utilização:Prov.}
\end{itemize}
\begin{itemize}
\item {Utilização:trasm.}
\end{itemize}
\begin{itemize}
\item {Proveniência:(De \textunderscore penetra\textunderscore )}
\end{itemize}
Tornar-se rico.
Viver desafogadamente.
\section{Empenha}
\begin{itemize}
\item {Grp. gram.:f.}
\end{itemize}
\begin{itemize}
\item {Utilização:Ant.}
\end{itemize}
Coiro para sapatos.
Remendo lateral de um sapato.
(Cast. \textunderscore empeine\textunderscore )
\section{Empenhadamente}
\begin{itemize}
\item {Grp. gram.:adv.}
\end{itemize}
\begin{itemize}
\item {Proveniência:(De \textunderscore empenhar\textunderscore )}
\end{itemize}
Com empenho.
\section{Empenhador}
\begin{itemize}
\item {Grp. gram.:m.  e  adj.}
\end{itemize}
O que empenha.
\section{Empenhamento}
\begin{itemize}
\item {Grp. gram.:m.}
\end{itemize}
Acto de empenhar.
\section{Empenhar}
\begin{itemize}
\item {Grp. gram.:v. t.}
\end{itemize}
\begin{itemize}
\item {Grp. gram.:V. p.}
\end{itemize}
\begin{itemize}
\item {Proveniência:(De \textunderscore empenho\textunderscore )}
\end{itemize}
Dar em penhor: \textunderscore empenhar um piano\textunderscore .
Tornar obrigado, ou devedor de reconhecimento.
Invocar, comprometter: \textunderscore empenhar a sua palavra\textunderscore .
Expor.
Empregar com desvelo: \textunderscore empenhar esforços e sacrifícios\textunderscore .
Endividar-se.
Fazer diligência, pôr empenho em alguma coisa: \textunderscore empenhar-se em perder o vício\textunderscore .
\section{Empenho}
\begin{itemize}
\item {Grp. gram.:m.}
\end{itemize}
\begin{itemize}
\item {Proveniência:(Do lat. \textunderscore pignus\textunderscore )}
\end{itemize}
Acto ou effeito de empenhar.
A pessôa, que se empenha ou se interessa por outra: \textunderscore o seu melhor empenho era o Deputado Vaz\textunderscore .
\section{Empenhoca}
\begin{itemize}
\item {Grp. gram.:f.}
\end{itemize}
\begin{itemize}
\item {Utilização:Fam.}
\end{itemize}
\begin{itemize}
\item {Proveniência:(De \textunderscore empenho\textunderscore )}
\end{itemize}
Patronato.
Recommendação de alguém, feita por pessôa grada.
\section{Empenhorar}
\begin{itemize}
\item {Proveniência:(De \textunderscore penhor\textunderscore )}
\end{itemize}
\textunderscore v. t.\textunderscore  (e der.)
(V. \textunderscore empenhar\textunderscore , etc.)
\section{Empennachar}
\begin{itemize}
\item {Grp. gram.:v. t.}
\end{itemize}
\begin{itemize}
\item {Utilização:Ext.}
\end{itemize}
Pôr pennacho em.
Adornar com pennachos.
Tornar garrido, enfeitar.
\section{Empennar}
\begin{itemize}
\item {Grp. gram.:v. t.}
\end{itemize}
\begin{itemize}
\item {Grp. gram.:V. i.}
\end{itemize}
\begin{itemize}
\item {Grp. gram.:V. p.}
\end{itemize}
\begin{itemize}
\item {Proveniência:(De \textunderscore penna\textunderscore )}
\end{itemize}
Cobrir ou enfeitar de pennas.
Criar pennas, emplumar-se.
Enfeitar-se de pennas.
Enfeitar-se.
\section{Empeno}
\begin{itemize}
\item {fónica:pê}
\end{itemize}
\begin{itemize}
\item {Grp. gram.:m.}
\end{itemize}
\begin{itemize}
\item {Utilização:Pop.}
\end{itemize}
\begin{itemize}
\item {Utilização:Fig.}
\end{itemize}
O mesmo que \textunderscore empenamento\textunderscore .
Differença de contas.
Inexactidão num cálculo: \textunderscore nas contas do caixeiro havia um empeno\textunderscore .
Estôrvo, difficuldade: \textunderscore o negócio não teve um empeno\textunderscore .
\section{Empeolar}
\begin{itemize}
\item {Grp. gram.:v. t.}
\end{itemize}
\begin{itemize}
\item {Utilização:Prov.}
\end{itemize}
\begin{itemize}
\item {Utilização:alent.}
\end{itemize}
Preparar (a caça), para se poder pendurar.--Se é uma perdiz, atam-se duas pennas pelas extremidades, passam-se pelas narinas da ave e forma-se uma argola; se é coelho, parte-se-lhe um tarso, formando com o osso uma cruzeta, que se passa pela pelle do outro tarso.
(Cp. \textunderscore empeugar\textunderscore )
\section{Empeoramento}
\begin{itemize}
\item {Grp. gram.:m.}
\end{itemize}
Acto de empeorar:«\textunderscore empeoramento do achaque do peito\textunderscore ». Camillo, \textunderscore Judeu\textunderscore , I, 179.
\section{Empeorar}
\begin{itemize}
\item {Grp. gram.:v. t.}
\end{itemize}
\begin{itemize}
\item {Grp. gram.:V. i.}
\end{itemize}
Tornar peor.
Peorar; tornar-se peor.
\section{Empepinar}
\begin{itemize}
\item {Grp. gram.:v. t.}
\end{itemize}
\begin{itemize}
\item {Utilização:Ant.}
\end{itemize}
\begin{itemize}
\item {Grp. gram.:V. i.}
\end{itemize}
Tornar semelhante a um pepino, na consistência.
Illudir. Cf. G. Vicente.
Tornar-se duro, enresinado.
\section{Empequenitar}
\begin{itemize}
\item {Grp. gram.:v. t.}
\end{itemize}
Tornar pequeno, encurtar. Cf. Filinto, I, 53.
\section{Emperador}
\begin{itemize}
\item {Grp. gram.:m.}
\end{itemize}
(Fórma ant. e justificável de \textunderscore imperador\textunderscore ). Cf. \textunderscore Peregrinação\textunderscore , \textunderscore Déc.\textunderscore , etc.
\section{Empereirado}
\begin{itemize}
\item {Grp. gram.:adj.}
\end{itemize}
\begin{itemize}
\item {Utilização:Bras. do N}
\end{itemize}
O mesmo que \textunderscore rachítico\textunderscore .
\section{Emperlar}
\begin{itemize}
\item {Grp. gram.:v. t.}
\end{itemize}
\begin{itemize}
\item {Grp. gram.:V. p.}
\end{itemize}
\begin{itemize}
\item {Proveniência:(De \textunderscore perla\textunderscore )}
\end{itemize}
Pôr pérolas em.
Adornar com pérolas.
Dar fórma de pérola a.
Converter-se em pérolas; tomar a fórma de pérolas.
\section{Empernar}
\begin{itemize}
\item {Grp. gram.:v. i.}
\end{itemize}
Cruzar as pernas. Cf. Filinto, IX, 225.
\section{Empernear}
\begin{itemize}
\item {Grp. gram.:v. t.}
\end{itemize}
\begin{itemize}
\item {Proveniência:(De \textunderscore pernear\textunderscore )}
\end{itemize}
Meter de permeio; juntar, misturar.
\section{Emperó}
\begin{itemize}
\item {Grp. gram.:conj.}
\end{itemize}
\begin{itemize}
\item {Utilização:Ant.}
\end{itemize}
Todavia; porém.
(Cast. \textunderscore empero\textunderscore )
\section{Emperol}
\begin{itemize}
\item {Grp. gram.:conj.}
\end{itemize}
\begin{itemize}
\item {Utilização:Ant.}
\end{itemize}
O mesmo que \textunderscore emperó\textunderscore .
\section{Emperradamente}
\begin{itemize}
\item {Grp. gram.:adv.}
\end{itemize}
\begin{itemize}
\item {Proveniência:(De \textunderscore emperrar\textunderscore )}
\end{itemize}
Com teimosia.
\section{Emperramento}
\begin{itemize}
\item {Grp. gram.:m.}
\end{itemize}
Acto ou effeito de emperrar.
\section{Emperrar}
\begin{itemize}
\item {Grp. gram.:v. t.}
\end{itemize}
\begin{itemize}
\item {Grp. gram.:V. i}
\end{itemize}
Tornar perro.
Fazer obstinado, teimoso.
Tornar-se perro.
Não ter movimento: \textunderscore o cavallo emperrou\textunderscore .
Obstinar-se; enraivecer-se.
\section{Emperro}
\begin{itemize}
\item {fónica:pê}
\end{itemize}
\begin{itemize}
\item {Grp. gram.:m.}
\end{itemize}
O mesmo que \textunderscore emperramento\textunderscore .
\section{Empertigado}
\begin{itemize}
\item {Grp. gram.:adj.}
\end{itemize}
\begin{itemize}
\item {Proveniência:(De \textunderscore empertigar\textunderscore )}
\end{itemize}
Ancho.
Teso.
Orgulhoso; vaidoso.
\section{Empertigamento}
\begin{itemize}
\item {Grp. gram.:m.}
\end{itemize}
Acto ou effeito de empertigar. Cf. Camillo, \textunderscore Corja\textunderscore , 227.
\section{Empertigar}
\begin{itemize}
\item {Grp. gram.:v. t.}
\end{itemize}
\begin{itemize}
\item {Proveniência:(De \textunderscore pertiga\textunderscore )}
\end{itemize}
Tornar teso, direito, orgulhoso.
\section{Empesar}
\begin{itemize}
\item {Grp. gram.:v. t.}
\end{itemize}
\begin{itemize}
\item {Utilização:Prov.}
\end{itemize}
\begin{itemize}
\item {Utilização:trasm.}
\end{itemize}
\begin{itemize}
\item {Proveniência:(De \textunderscore empêso\textunderscore )}
\end{itemize}
Espremer (o vinho do bagaço).
\section{Empescoçado}
\begin{itemize}
\item {Grp. gram.:adj.}
\end{itemize}
\begin{itemize}
\item {Utilização:Prov.}
\end{itemize}
O mesmo que \textunderscore cachaçudo\textunderscore .
\section{Empesgado}
\begin{itemize}
\item {Grp. gram.:adj.}
\end{itemize}
\begin{itemize}
\item {Utilização:Prov.}
\end{itemize}
\begin{itemize}
\item {Utilização:trasm.}
\end{itemize}
\begin{itemize}
\item {Proveniência:(De \textunderscore empesgar\textunderscore )}
\end{itemize}
Encerrado.
Apartado.
\section{Empesgadura}
\begin{itemize}
\item {Grp. gram.:f.}
\end{itemize}
Acto ou effeito de empesgar^1.
\section{Empesgar}
\begin{itemize}
\item {Grp. gram.:v. t.}
\end{itemize}
\begin{itemize}
\item {Proveniência:(De \textunderscore pesga\textunderscore )}
\end{itemize}
Untar com pez.
\section{Empesgar}
\begin{itemize}
\item {Grp. gram.:v. t.}
\end{itemize}
\begin{itemize}
\item {Utilização:Prov.}
\end{itemize}
\begin{itemize}
\item {Proveniência:(De \textunderscore pés\textunderscore )}
\end{itemize}
O mesmo que \textunderscore empeugar\textunderscore  (a caça).
\section{Empêso}
\begin{itemize}
\item {Grp. gram.:m.}
\end{itemize}
\begin{itemize}
\item {Utilização:Prov.}
\end{itemize}
\begin{itemize}
\item {Utilização:trasm.}
\end{itemize}
Pêso ou pedra, que o fuso do lagar sustenta.
\section{Empessoar}
\begin{itemize}
\item {Proveniência:(De \textunderscore pessôa\textunderscore )}
\end{itemize}
\textunderscore v. t.\textunderscore  (e der.)
(V. \textunderscore empossar\textunderscore , etc.)
\section{Empestador}
\begin{itemize}
\item {Grp. gram.:adj.}
\end{itemize}
Que empesta.
\section{Empestamento}
\begin{itemize}
\item {Grp. gram.:m.}
\end{itemize}
Acto ou effeito de empestar.
\section{Empestar}
\begin{itemize}
\item {Grp. gram.:v. t.}
\end{itemize}
Causar peste a.
Tornar pestilento.
Communicar mau cheiro a.
Corromper; depravar.
Desmoralizar.
\section{Empetráceas}
\begin{itemize}
\item {Grp. gram.:f. pl.}
\end{itemize}
\begin{itemize}
\item {Proveniência:(Do gr. \textunderscore empetros\textunderscore )}
\end{itemize}
Fam. de plantas dicotyledóneas, a que pertence a camarinheira.
\section{Empeugar}
\begin{itemize}
\item {fónica:pe-u}
\end{itemize}
\begin{itemize}
\item {Grp. gram.:v. t.}
\end{itemize}
\begin{itemize}
\item {Utilização:Prov.}
\end{itemize}
\begin{itemize}
\item {Utilização:trasm.}
\end{itemize}
\begin{itemize}
\item {Proveniência:(De \textunderscore peúga\textunderscore )}
\end{itemize}
Prender pelos pés ao cinto (a caça).
\section{Empezar}
\begin{itemize}
\item {Grp. gram.:v. t.}
\end{itemize}
\begin{itemize}
\item {Proveniência:(De \textunderscore pez\textunderscore )}
\end{itemize}
Empesgar^1; defumar com pez.
\section{Empezinhar}
\begin{itemize}
\item {Grp. gram.:v. t.}
\end{itemize}
\begin{itemize}
\item {Proveniência:(De \textunderscore pez\textunderscore )}
\end{itemize}
Sujar com pez.
Empesgar^1.
\section{Êmphase}
\begin{itemize}
\item {Grp. gram.:f.}
\end{itemize}
\begin{itemize}
\item {Proveniência:(Gr. \textunderscore emphasis\textunderscore )}
\end{itemize}
Maneira empolada, exaggerada ou affectada, de falar ou escrever.
Ostentação.
\section{Emphaticamente}
\begin{itemize}
\item {Grp. gram.:adv.}
\end{itemize}
De modo emphático.
Com êmphase.
\section{Emphático}
\begin{itemize}
\item {Grp. gram.:adj.}
\end{itemize}
\begin{itemize}
\item {Proveniência:(Gr. \textunderscore emphatikos\textunderscore )}
\end{itemize}
Que tem êmphase; em que há êmphase: \textunderscore linguagem emphática\textunderscore .
\section{Emphatismo}
\begin{itemize}
\item {Grp. gram.:m.}
\end{itemize}
Qualidade daquillo ou daquelle que é emphático.
Uso immoderado da êmphase.
(Cp. \textunderscore emphático\textunderscore )
\section{Êmphobo}
\begin{itemize}
\item {Grp. gram.:m.}
\end{itemize}
Quadrúpede selvagem, semelhante a um cavallo e mencionado na \textunderscore Ethiópia Or.\textunderscore , l. I, c. I.
\section{Emphrático}
\begin{itemize}
\item {Grp. gram.:adj.}
\end{itemize}
\begin{itemize}
\item {Utilização:Med.}
\end{itemize}
\begin{itemize}
\item {Proveniência:(Gr. \textunderscore emphratikos\textunderscore )}
\end{itemize}
Que obstrue.
\section{Emphraxia}
\begin{itemize}
\item {fónica:csi}
\end{itemize}
\begin{itemize}
\item {Grp. gram.:f.}
\end{itemize}
\begin{itemize}
\item {Utilização:Med.}
\end{itemize}
O mesmo que \textunderscore obstrucção\textunderscore .
(Cp. \textunderscore emphrático\textunderscore )
\section{Emphysema}
\begin{itemize}
\item {Grp. gram.:m.}
\end{itemize}
\begin{itemize}
\item {Utilização:Med.}
\end{itemize}
\begin{itemize}
\item {Proveniência:(Gr. \textunderscore emphusema\textunderscore )}
\end{itemize}
Tumor branco elástico, causado pela infiltração do ar no tecido cellular.
\section{Emphysemático}
\begin{itemize}
\item {Grp. gram.:adj.}
\end{itemize}
Relativo a emphysema.
\section{Emphysematoso}
\begin{itemize}
\item {Grp. gram.:adj.}
\end{itemize}
O mesmo que \textunderscore emphysemático\textunderscore .
\section{Emphyteuse}
\begin{itemize}
\item {Grp. gram.:f.}
\end{itemize}
\begin{itemize}
\item {Utilização:Jur.}
\end{itemize}
\begin{itemize}
\item {Proveniência:(Gr. \textunderscore emphuteusis\textunderscore )}
\end{itemize}
Convenção, pela qual o senhor de um prédio transfere para outrem o domínio útil do mesmo prédio, obrigando-se o cessionário a pagar-lhe uma pensão annual que se chama fôro.
Aforamento.
\section{Emphyteuta}
\begin{itemize}
\item {Grp. gram.:m.  e  f.}
\end{itemize}
\begin{itemize}
\item {Proveniência:(Gr. \textunderscore emphuteutes\textunderscore )}
\end{itemize}
Pêssoa, que recebe ou tem o domínio útil de um prédio, por contrato de emphyteuse.
\section{Emphyteuticação}
\begin{itemize}
\item {Grp. gram.:f.}
\end{itemize}
Acto ou effeito de emphyteuticar. Cf. Herculano, \textunderscore Opúsculos\textunderscore , IV, 71.
\section{Emphyteuticar}
\begin{itemize}
\item {Grp. gram.:v. t.}
\end{itemize}
\begin{itemize}
\item {Proveniência:(De \textunderscore emphytêutico\textunderscore )}
\end{itemize}
Aforar.
Ceder por emphyteuse.
\section{Emphyteuticário}
\begin{itemize}
\item {Grp. gram.:adj.}
\end{itemize}
\begin{itemize}
\item {Utilização:Des.}
\end{itemize}
O mesmo que \textunderscore emphytêutico\textunderscore .
\section{Emphytêutico}
\begin{itemize}
\item {Grp. gram.:adj.}
\end{itemize}
Relativo a emphyteuse.
(Cp. \textunderscore emphyteuta\textunderscore )
\section{Empicotamento}
\begin{itemize}
\item {Grp. gram.:m.}
\end{itemize}
\begin{itemize}
\item {Utilização:Ant.}
\end{itemize}
Acto de empicotar.
\section{Empicotar}
\begin{itemize}
\item {Grp. gram.:v. t.}
\end{itemize}
\begin{itemize}
\item {Utilização:Fig.}
\end{itemize}
\begin{itemize}
\item {Proveniência:(De \textunderscore picoto\textunderscore  e \textunderscore picota\textunderscore )}
\end{itemize}
Pôr no pico, no picoto, no cume.
Prender, espetar na picota.
Expor á irrisão ou á vergonha.
\section{Empigem}
\begin{itemize}
\item {Grp. gram.:f.}
\end{itemize}
(V.impigem)
\section{Empilhamento}
\begin{itemize}
\item {Grp. gram.:m.}
\end{itemize}
Acto de empilhar.
\section{Empilhar}
\begin{itemize}
\item {Grp. gram.:v. t.}
\end{itemize}
Pôr em pilha.
Amontoar, acumular.
\section{Empina}
\begin{itemize}
\item {Grp. gram.:f.}
\end{itemize}
\begin{itemize}
\item {Utilização:Prov.}
\end{itemize}
\begin{itemize}
\item {Proveniência:(De \textunderscore empinar\textunderscore )}
\end{itemize}
Contenda, contestação, altercação. (Colhido na Bairrada)
\section{Empinado}
\begin{itemize}
\item {Grp. gram.:adj.}
\end{itemize}
\begin{itemize}
\item {Proveniência:(De \textunderscore empinar\textunderscore )}
\end{itemize}
Direito.
Erguido.
Alcantilado.
Que tem grande declive: \textunderscore ladeira empinada\textunderscore .
Empolado, bombástico, (falando-se de palavras ou de estilo). Cf. Bernárdez, \textunderscore Luz e Calor\textunderscore , 146.
\section{Empinar}
\begin{itemize}
\item {Grp. gram.:v. t.}
\end{itemize}
\begin{itemize}
\item {Grp. gram.:V. p.}
\end{itemize}
\begin{itemize}
\item {Proveniência:(De \textunderscore pino\textunderscore )}
\end{itemize}
Pôr a pino.
Levantar ao cume.
Erguer; emborcar (copo, garrafa, etc.).
Erguer-se sobre as patas traseiras, (falando-se das cavalgaduras).
\section{Empinheirado}
\begin{itemize}
\item {Grp. gram.:adj.}
\end{itemize}
\begin{itemize}
\item {Utilização:Prov.}
\end{itemize}
\begin{itemize}
\item {Proveniência:(De \textunderscore pinheiro\textunderscore )}
\end{itemize}
Erecto, levantado. (Colhido em Turquel)
\section{Empinhocado}
\begin{itemize}
\item {Grp. gram.:adj.}
\end{itemize}
\begin{itemize}
\item {Utilização:Prov.}
\end{itemize}
\begin{itemize}
\item {Utilização:alg.}
\end{itemize}
\begin{itemize}
\item {Proveniência:(De \textunderscore pinhoca\textunderscore )}
\end{itemize}
Que forma pinhoca com outro.
Agrupados; agarrado a outro.
\section{Empino}
\begin{itemize}
\item {Grp. gram.:m.}
\end{itemize}
\begin{itemize}
\item {Utilização:Fig.}
\end{itemize}
Acto de empinar.
Posição empinada.
Orgulho, soberba.
\section{Empinocado}
\begin{itemize}
\item {fónica:nó}
\end{itemize}
\begin{itemize}
\item {Grp. gram.:adj.}
\end{itemize}
\begin{itemize}
\item {Utilização:Prov.}
\end{itemize}
\begin{itemize}
\item {Utilização:beir.}
\end{itemize}
\begin{itemize}
\item {Proveniência:(De \textunderscore pinoca\textunderscore )}
\end{itemize}
Que subiu e se mostra em lugar elevado.
Empoleirado: \textunderscore estava uma coruja empinocada na torre\textunderscore .
\section{Empioramento}
\begin{itemize}
\item {Grp. gram.:m.}
\end{itemize}
Acto de empiorar:«\textunderscore empioramento do achaque do peito\textunderscore ». Camillo, \textunderscore Judeu\textunderscore , I, 179.
\section{Empiorar}
\begin{itemize}
\item {Grp. gram.:v. t.}
\end{itemize}
\begin{itemize}
\item {Grp. gram.:V. i.}
\end{itemize}
Tornar pior.
Piorar; tornar-se pior.
\section{Empipa}
\begin{itemize}
\item {Grp. gram.:f.}
\end{itemize}
Bebida adocicada da África occid. port.
\section{Empiricamente}
\begin{itemize}
\item {Grp. gram.:adv.}
\end{itemize}
De modo empírico.
\section{Empírico}
\begin{itemize}
\item {Grp. gram.:adj.}
\end{itemize}
\begin{itemize}
\item {Grp. gram.:M.}
\end{itemize}
\begin{itemize}
\item {Proveniência:(Gr. \textunderscore empeirikos\textunderscore )}
\end{itemize}
Relativo ao empirismo.
Que se guia só pela experiência.
Que não procede da theoria, mas de factos particulares.
Aquelle que trata doenças com remédios secretos, sem noções scientíficas sobre as doenças.
\section{Empirismo}
\begin{itemize}
\item {Grp. gram.:m.}
\end{itemize}
\begin{itemize}
\item {Utilização:Fig.}
\end{itemize}
\begin{itemize}
\item {Proveniência:(Do gr. \textunderscore en\textunderscore  + \textunderscore peira\textunderscore , experiência)}
\end{itemize}
Doutrina, baseada exclusivamente na experiência.
Conjunto de conhecimentos, resultantes da prática.
Rotina.
\section{Empiscar}
\begin{itemize}
\item {Grp. gram.:v. t.}
\end{itemize}
O mesmo que piscar:«\textunderscore trocam pai e mãe pelo primeiro perna-fina que lhes empisca o ôlho\textunderscore ». Camillo, \textunderscore Cart. Ângela\textunderscore , 23.
\section{Empiteirar}
\begin{itemize}
\item {Grp. gram.:v. t.}
\end{itemize}
\begin{itemize}
\item {Utilização:Pop.}
\end{itemize}
\begin{itemize}
\item {Grp. gram.:V. p.}
\end{itemize}
\begin{itemize}
\item {Proveniência:(De \textunderscore piteira\textunderscore )}
\end{itemize}
Embebedar.
Embebedar-se.
Encher-se de dívidas.
\section{Emplanchar}
\begin{itemize}
\item {Grp. gram.:v. i.}
\end{itemize}
\begin{itemize}
\item {Utilização:Prov.}
\end{itemize}
Restabelecer-se de uma doença.
Rehaver o que se perdeu.
(Provavelmente relaciona-se com \textunderscore implantar\textunderscore )
\section{Emplasmado}
\begin{itemize}
\item {Grp. gram.:adj.}
\end{itemize}
\begin{itemize}
\item {Utilização:Pop.}
\end{itemize}
Achacadiço.
Coberto de emplastros; cheio de feridas. Cf. Camillo, \textunderscore Corja\textunderscore , 154.
(Por \textunderscore encataplasmado\textunderscore , de \textunderscore cataplasma\textunderscore )
\section{Emplasmar}
\begin{itemize}
\item {Grp. gram.:v. t.}
\end{itemize}
Cobrir de emplastros.
Tornar acachadiço, adoentado.
(Por \textunderscore encataplasmar\textunderscore , de \textunderscore cataplasma\textunderscore )
\section{Emplastar}
\textunderscore v. t.\textunderscore  (e der.)
O mesmo que \textunderscore emplastrar\textunderscore , etc.
\section{Emplastração}
\begin{itemize}
\item {Grp. gram.:f.}
\end{itemize}
Acto ou effeito de emplastrar.
\section{Emplastrácio}
\begin{itemize}
\item {Grp. gram.:m.}
\end{itemize}
\begin{itemize}
\item {Utilização:Agr.}
\end{itemize}
Enxêrto de borbulha.
(Cp. \textunderscore emplastrar\textunderscore )
\section{Emplastrada}
\begin{itemize}
\item {Grp. gram.:f.}
\end{itemize}
Porção de emplastros. Cf. Arn. Gama, \textunderscore Motim\textunderscore , 203.
\section{Emplastragem}
\begin{itemize}
\item {Grp. gram.:f.}
\end{itemize}
Acto de emplastrar.
Revestimento tosco, á maneira de emplastro.
\section{Emplastramento}
\begin{itemize}
\item {Grp. gram.:m.}
\end{itemize}
Acto de emplastrar.
\section{Emplastrar}
\begin{itemize}
\item {Grp. gram.:v. t.}
\end{itemize}
Pôr emplastros em.
Revestir, como se cobrisse de emplastro.
\section{Emplástrico}
\begin{itemize}
\item {Grp. gram.:adj.}
\end{itemize}
\begin{itemize}
\item {Proveniência:(Gr. \textunderscore emplastrikos\textunderscore )}
\end{itemize}
Relativo a emplastro.
\section{Emplastro}
\begin{itemize}
\item {Grp. gram.:m.}
\end{itemize}
\begin{itemize}
\item {Utilização:Fig.}
\end{itemize}
\begin{itemize}
\item {Proveniência:(Gr. \textunderscore emplastron\textunderscore )}
\end{itemize}
Medicamento glutinoso, que, amollecendo com o calor, adhere á parte do corpo em que se applica.
Consêrto mal feito.
Pessôa achacadiça.
Remendo, que destôa do objecto remendado.
Pessôa indolente.
\section{Emplumação}
\begin{itemize}
\item {Grp. gram.:f.}
\end{itemize}
Acto de emplumar.
\section{Emplumar}
\begin{itemize}
\item {Grp. gram.:v. t.}
\end{itemize}
Ornar de plumas.
Empennar.
\section{Empoamento}
\begin{itemize}
\item {Grp. gram.:m.}
\end{itemize}
Acto ou effeito de empoar.
\section{Empoar}
\begin{itemize}
\item {Grp. gram.:v. t.}
\end{itemize}
\begin{itemize}
\item {Proveniência:(De \textunderscore pò\textunderscore )}
\end{itemize}
Cobrir com pó.
Polvilhar, para enfeite.
Sujar com pó.
\section{Empobrecer}
\begin{itemize}
\item {Grp. gram.:v. t.}
\end{itemize}
\begin{itemize}
\item {Grp. gram.:V. i.}
\end{itemize}
Tornar pobre.
Esgotar; depauperar: \textunderscore empobrecer o thesoiro público\textunderscore .
Tornar-se pobre.
\section{Empobrecimento}
\begin{itemize}
\item {Grp. gram.:m.}
\end{itemize}
Acto ou effeito de empobrecer.
\section{Empoçar}
\begin{itemize}
\item {Grp. gram.:v. t.}
\end{itemize}
\begin{itemize}
\item {Grp. gram.:V. i.}
\end{itemize}
\begin{itemize}
\item {Utilização:Prov.}
\end{itemize}
\begin{itemize}
\item {Utilização:trasm.}
\end{itemize}
\begin{itemize}
\item {Proveniência:(De \textunderscore poço\textunderscore  e \textunderscore poça\textunderscore )}
\end{itemize}
Meter em poço.
Formar poça ou atoleiro.
Pôr de môlho na água (o linho); enriar.
\section{Empocilgar}
\begin{itemize}
\item {Grp. gram.:v. t.}
\end{itemize}
Meter em pocilga.
Encurralar.
\section{Empoderar-se}
\begin{itemize}
\item {Grp. gram.:v. p.}
\end{itemize}
O mesmo que \textunderscore apoderar-se\textunderscore . Cf. Filinto, \textunderscore D. Man.\textunderscore , I, 44.
\section{Empoeirado}
\begin{itemize}
\item {Grp. gram.:adj.}
\end{itemize}
\begin{itemize}
\item {Utilização:Fig.}
\end{itemize}
Coberto de poeira.
Vaidoso.
Presumido.
\section{Empoeiramento}
\begin{itemize}
\item {Grp. gram.:m.}
\end{itemize}
Acto ou effeito de empoeirar.
\section{Empoeirar}
\begin{itemize}
\item {Grp. gram.:v. t.}
\end{itemize}
Cobrir de poeira, empoar.
\section{Empófia}
\begin{itemize}
\item {Grp. gram.:f.}
\end{itemize}
\begin{itemize}
\item {Utilização:Ant.}
\end{itemize}
\begin{itemize}
\item {Utilização:Prov.}
\end{itemize}
\begin{itemize}
\item {Utilização:dur.}
\end{itemize}
Pretexto para tomar conta do que é alheio.
O mesmo que \textunderscore embófia\textunderscore , \textunderscore m.\textunderscore  e \textunderscore f.\textunderscore 
\section{Empoita}
\begin{itemize}
\item {Grp. gram.:f.}
\end{itemize}
\begin{itemize}
\item {Utilização:Prov.}
\end{itemize}
\begin{itemize}
\item {Utilização:trasm.}
\end{itemize}
Burzigada, panelada de batatas com farelo, para os cevados.
\section{Empoitada}
\begin{itemize}
\item {Grp. gram.:f.}
\end{itemize}
\begin{itemize}
\item {Utilização:Prov.}
\end{itemize}
\begin{itemize}
\item {Utilização:trasm.}
\end{itemize}
Grande empoita.
\section{Empôla}
\begin{itemize}
\item {Grp. gram.:f.}
\end{itemize}
\begin{itemize}
\item {Proveniência:(Do lat. \textunderscore ampulla\textunderscore )}
\end{itemize}
Bolha, formada por derramamento de serosidade, entre a derme e a epiderme.
Corpúsculo globuloso e oco, que se observa na raiz de certas plantas.
\section{Empôla}
\begin{itemize}
\item {Grp. gram.:m.}
\end{itemize}
\begin{itemize}
\item {Utilização:Ant.}
\end{itemize}
O mesmo que \textunderscore oásis\textunderscore . Cf. Barros, \textunderscore Déc.\textunderscore  I.
\section{Empoláceo}
\begin{itemize}
\item {Grp. gram.:adj.}
\end{itemize}
\begin{itemize}
\item {Proveniência:(De \textunderscore empôla\textunderscore )}
\end{itemize}
Que tem fórma de empôla ou de vesícula.
\section{Empolado}
\begin{itemize}
\item {Grp. gram.:adj.}
\end{itemize}
\begin{itemize}
\item {Utilização:Fig.}
\end{itemize}
Que tem empolas.
Pomposo; bombástico: \textunderscore discurso empolado\textunderscore .
\section{Empoladoiras}
\begin{itemize}
\item {Grp. gram.:f. pl.}
\end{itemize}
\begin{itemize}
\item {Utilização:Prov.}
\end{itemize}
\begin{itemize}
\item {Utilização:minh.}
\end{itemize}
Estadulhos curtos e curvos, que se espetam de baixo para cima, atravessando os coucões e chedas, e abraçando o eixo do carro, para segurar o chedeiro ao eixo.
\section{Empoladouras}
\begin{itemize}
\item {Grp. gram.:f. pl.}
\end{itemize}
\begin{itemize}
\item {Utilização:Prov.}
\end{itemize}
\begin{itemize}
\item {Utilização:minh.}
\end{itemize}
Estadulhos curtos e curvos, que se espetam de baixo para cima, atravessando os coucões e chedas, e abraçando o eixo do carro, para segurar o chedeiro ao eixo.
\section{Empolamar}
\begin{itemize}
\item {Grp. gram.:v. t.}
\end{itemize}
\begin{itemize}
\item {Utilização:Prov.}
\end{itemize}
Produzir empolas em.
Fazer empolar muito.
Dilatar (a pelle).
(Cp. \textunderscore empôla\textunderscore ^1)
\section{Empolar}
\begin{itemize}
\item {Grp. gram.:v. t.}
\end{itemize}
\begin{itemize}
\item {Utilização:Fig.}
\end{itemize}
\begin{itemize}
\item {Grp. gram.:V. i.}
\end{itemize}
\begin{itemize}
\item {Proveniência:(Do lat. \textunderscore ampullare\textunderscore )}
\end{itemize}
Produzir empolas em.
Tornar pomposo, soberbo, exaggerado, ostentoso.
Criar empolas.
Avolumar-se como empôla.
\section{Empolar}
\begin{itemize}
\item {Grp. gram.:adj.}
\end{itemize}
O mesmo que \textunderscore empoláceo\textunderscore .
\section{Empolasmar}
\begin{itemize}
\item {Grp. gram.:v. t.}
\end{itemize}
(V.emplasmar)
\section{Empolear}
\begin{itemize}
\item {Grp. gram.:v. t.}
\end{itemize}
\begin{itemize}
\item {Utilização:Prov.}
\end{itemize}
\begin{itemize}
\item {Utilização:trasm.}
\end{itemize}
\begin{itemize}
\item {Proveniência:(De \textunderscore polé\textunderscore )}
\end{itemize}
Arrebatar pelos ares.
\section{Empoleirado}
\begin{itemize}
\item {Grp. gram.:adj.}
\end{itemize}
Que está no poleiro.
\section{Empoleirar}
\begin{itemize}
\item {Grp. gram.:v. t.}
\end{itemize}
\begin{itemize}
\item {Utilização:Fig.}
\end{itemize}
Pôr no poleiro.
Exaltar.
Dar elevada posição a.
\section{Empolgadeira}
\begin{itemize}
\item {Grp. gram.:f.}
\end{itemize}
\begin{itemize}
\item {Utilização:Ant.}
\end{itemize}
\begin{itemize}
\item {Proveniência:(De \textunderscore empolgar\textunderscore )}
\end{itemize}
Buraco, em que se enfiava a corda, no extremo do arco da bésta.
\section{Empolgador}
\begin{itemize}
\item {Grp. gram.:adj.}
\end{itemize}
Que empolga.
\section{Empolgadura}
\begin{itemize}
\item {Grp. gram.:f.}
\end{itemize}
O mesmo que \textunderscore empolgadeira\textunderscore .
Acto de empolgar.
\section{Empolgante}
\begin{itemize}
\item {Grp. gram.:adj.}
\end{itemize}
Que empolga.
Que domina moralmente, que arrasta.
\section{Empolgar}
\begin{itemize}
\item {Grp. gram.:v. t.}
\end{itemize}
\begin{itemize}
\item {Proveniência:(Do lat. hyp. \textunderscore impollicare\textunderscore , de \textunderscore pollex\textunderscore )}
\end{itemize}
Agarrar.
Apoderar-se de.
Adquirir com violência ou injustiça.
Estirar (a corda) para armar a bésta.
Segurar com arpéu.
Impor-se a, dominar: \textunderscore aquelle orador empolga o auditório\textunderscore .
\section{Empolgueira}
\begin{itemize}
\item {Grp. gram.:f.}
\end{itemize}
(V.empolgadeira)
\section{Empolhar}
\begin{itemize}
\item {Grp. gram.:v. t.}
\end{itemize}
\begin{itemize}
\item {Grp. gram.:V. i.}
\end{itemize}
\begin{itemize}
\item {Proveniência:(Do cast. \textunderscore pollo\textunderscore , lat. \textunderscore pullus\textunderscore , pinto, frango)}
\end{itemize}
O mesmo que \textunderscore incubar\textunderscore .
Criar pinto (o ovo).
\section{Empolmar}
\begin{itemize}
\item {Grp. gram.:v. t.}
\end{itemize}
Reduzir a polme.
\section{Emponderar}
\begin{itemize}
\item {Grp. gram.:v. t.}
\end{itemize}
\begin{itemize}
\item {Utilização:Ant.}
\end{itemize}
\begin{itemize}
\item {Proveniência:(Do lat. \textunderscore ponderare\textunderscore )}
\end{itemize}
Encarregar.
\section{Empontar}
\begin{itemize}
\item {Grp. gram.:v. t.}
\end{itemize}
\begin{itemize}
\item {Utilização:Prov.}
\end{itemize}
Despedir, mandar embora: \textunderscore empontou a criada\textunderscore .
(Provavelmente, relaciona-se com \textunderscore apontar\textunderscore )
\section{Emporcalhar}
\begin{itemize}
\item {Grp. gram.:v. t.}
\end{itemize}
\begin{itemize}
\item {Proveniência:(De \textunderscore porcalhão\textunderscore )}
\end{itemize}
Sujar.
Encher de nódoas.
\section{Emporcar}
\begin{itemize}
\item {Grp. gram.:v. t.}
\end{itemize}
O mesmo que \textunderscore emporcalhar\textunderscore .
\section{Em-porém}
\begin{itemize}
\item {Grp. gram.:conj.}
\end{itemize}
\begin{itemize}
\item {Utilização:Prov.}
\end{itemize}
\begin{itemize}
\item {Utilização:trasm.}
\end{itemize}
O mesmo que \textunderscore porém\textunderscore .
\section{Emporético}
\begin{itemize}
\item {Grp. gram.:m.}
\end{itemize}
\begin{itemize}
\item {Proveniência:(Gr. \textunderscore emporetikos\textunderscore )}
\end{itemize}
Diz-se de um papel passento, que servo para filtrar.
\section{Empório}
\begin{itemize}
\item {Grp. gram.:m.}
\end{itemize}
\begin{itemize}
\item {Proveniência:(Gr. \textunderscore emporion\textunderscore )}
\end{itemize}
Pôrto ou cidade, aonde concorrem muitos estrangeiros para commerciar.
\section{Emporisso}
\begin{itemize}
\item {Grp. gram.:conj.}
\end{itemize}
\begin{itemize}
\item {Utilização:Prov.}
\end{itemize}
\begin{itemize}
\item {Utilização:minh.}
\end{itemize}
Todavia.
\section{Em-pós}
\begin{itemize}
\item {Grp. gram.:prep.}
\end{itemize}
O mesmo que \textunderscore após\textunderscore . Cf. M. Assis, \textunderscore Poesias\textunderscore , 334.
\section{Empossar}
\begin{itemize}
\item {Grp. gram.:v. t.}
\end{itemize}
\begin{itemize}
\item {Grp. gram.:V. p.}
\end{itemize}
Investir na posse.
Dar posse a.
Assenhorear-se, apoderar-se.
\section{Emposse}
\begin{itemize}
\item {Grp. gram.:f.}
\end{itemize}
Acto ou effeito de empossar.
\section{Emposta}
\begin{itemize}
\item {Grp. gram.:f.}
\end{itemize}
(V.imposta)
\section{Empostação}
\begin{itemize}
\item {Grp. gram.:f.}
\end{itemize}
Acto de \textunderscore empostar\textunderscore .
\section{Empostar}
\begin{itemize}
\item {Grp. gram.:v.}
\end{itemize}
\begin{itemize}
\item {Utilização:t. Mús.}
\end{itemize}
\begin{itemize}
\item {Proveniência:(De \textunderscore pôsto\textunderscore , part. de \textunderscore pôr\textunderscore )}
\end{itemize}
Dar á (voz ou ás notas) a collocação apropriada na larynge.
\section{Empostigar}
\begin{itemize}
\item {Grp. gram.:v. t.}
\end{itemize}
Pôr o postigo em (vasilha de vinho).
\section{Emprasto}
\begin{itemize}
\item {Grp. gram.:m.}
\end{itemize}
\begin{itemize}
\item {Utilização:Prov.}
\end{itemize}
\begin{itemize}
\item {Utilização:dur.}
\end{itemize}
\begin{itemize}
\item {Utilização:Ant.}
\end{itemize}
O mesmo que \textunderscore emplastro\textunderscore . Cf. \textunderscore Eufrosina\textunderscore , 62.
\section{Empiema}
\begin{itemize}
\item {Grp. gram.:m.}
\end{itemize}
\begin{itemize}
\item {Utilização:Med.}
\end{itemize}
\begin{itemize}
\item {Proveniência:(Gr. \textunderscore empuema\textunderscore )}
\end{itemize}
Ajuntamento de pus ou sangue numa cavidade do corpo.
Acumulação serosa, sanguínea ou purulenta na cavidade da pleura.
Operação, para dar saída a essas matérias.
\section{Empiemático}
\begin{itemize}
\item {Grp. gram.:adj.}
\end{itemize}
Que tem empiema.
\section{Empíreo}
\begin{itemize}
\item {Grp. gram.:m.}
\end{itemize}
\begin{itemize}
\item {Grp. gram.:Adj.}
\end{itemize}
\begin{itemize}
\item {Proveniência:(Do gr. \textunderscore en\textunderscore  + \textunderscore pur\textunderscore )}
\end{itemize}
Esfera, em que a Astronomia antiga considerava fixados os astros.
Morada dos deuses, segundo o paganismo.
Lugar dos bem-aventurados e santos.
Céu.
Celeste; supremo.
\section{Empireuma}
\begin{itemize}
\item {Grp. gram.:m.}
\end{itemize}
\begin{itemize}
\item {Utilização:Chím.}
\end{itemize}
\begin{itemize}
\item {Proveniência:(Gr. \textunderscore empureuma\textunderscore )}
\end{itemize}
Sabor e cheio desagradável, procedente de substância orgânica, submetida á acção de fogo violento.
\section{Empireumático}
\begin{itemize}
\item {Grp. gram.:adj.}
\end{itemize}
Em que há empireuma: \textunderscore óleo empireumático\textunderscore .
\section{Emprazador}
\begin{itemize}
\item {Grp. gram.:adj.}
\end{itemize}
\begin{itemize}
\item {Grp. gram.:M.}
\end{itemize}
Que empraza.
Aquelle que empraza.
Aquelle que, com sua presença, impede o que se deseja.
Maçador.
\section{Emprazador}
\begin{itemize}
\item {Grp. gram.:m.}
\end{itemize}
\begin{itemize}
\item {Utilização:Prov.}
\end{itemize}
\begin{itemize}
\item {Utilização:alent.}
\end{itemize}
Indivíduo, que, pelo rasto, descobre o sitio onde está uma peça de caça grossa.
(Cp. \textunderscore emprazar-se\textunderscore )
\section{Emprazamento}
\begin{itemize}
\item {Grp. gram.:m.}
\end{itemize}
Aforamento.
Acto de emprazar.
\section{Emprazar}
\begin{itemize}
\item {Grp. gram.:v. t.}
\end{itemize}
\begin{itemize}
\item {Utilização:Fig.}
\end{itemize}
\begin{itemize}
\item {Proveniência:(De \textunderscore prazo\textunderscore )}
\end{itemize}
Intimar para comparecer em certo prazo, na presença de uma autoridade.
Marcar prazo a.
Ceder por contrato de emphyteuse.
Dar aviso ou ordem para que alguém se justifique, faça declarações ou pratique certos actos.
Empatar, fazer estôrvo a.
Cercar, para que não fuja (a caça).
\section{Emprazar-se}
\begin{itemize}
\item {Grp. gram.:v. p.}
\end{itemize}
\begin{itemize}
\item {Utilização:Ant.}
\end{itemize}
Collocar-se, postar-se. Cf. Rui de Pina, \textunderscore Chrón. de João II\textunderscore , c. 78. (Cp. fr. \textunderscore place\textunderscore , e port. \textunderscore praça\textunderscore )
\section{Empreendedor}
\begin{itemize}
\item {Grp. gram.:adj.}
\end{itemize}
\begin{itemize}
\item {Grp. gram.:M.}
\end{itemize}
\begin{itemize}
\item {Proveniência:(De \textunderscore empreender\textunderscore )}
\end{itemize}
Que empreende.
Activo.
Arrojado.
Aquelle que empreende ou toma a seu cargo uma empresa.
\section{Empreender}
\begin{itemize}
\item {Grp. gram.:v. t.}
\end{itemize}
\begin{itemize}
\item {Grp. gram.:V. i.}
\end{itemize}
\begin{itemize}
\item {Utilização:Pop.}
\end{itemize}
\begin{itemize}
\item {Proveniência:(Do lat. \textunderscore prehendere\textunderscore )}
\end{itemize}
Resolver.
Praticar.
Decidir-se a, não obstante as dificuldades.
Têr apreensões, scismas; apreender.
\section{Empreendimento}
\begin{itemize}
\item {Grp. gram.:m.}
\end{itemize}
Acto de empreender.
Empresa; tentame.
\section{Empregado}
\begin{itemize}
\item {Grp. gram.:m.}
\end{itemize}
\begin{itemize}
\item {Proveniência:(De \textunderscore empregar\textunderscore )}
\end{itemize}
Aquelle que exerce qualquer emprêgo, em repartição pública ou estabelecimento particular.
\section{Emprègado}
\begin{itemize}
\item {Grp. gram.:adj.}
\end{itemize}
\begin{itemize}
\item {Utilização:Prov.}
\end{itemize}
O mesmo que \textunderscore entrèvado\textunderscore .
\section{Empregar}
\begin{itemize}
\item {Grp. gram.:v. t.}
\end{itemize}
Dar emprêgo, uso ou applicação a: \textunderscore empregar dinheiro\textunderscore .
Occupar.
Nomear para um emprêgo.
Preencher.
(Relaciona-se com \textunderscore pregar\textunderscore ?)
\section{Emprègar}
\textunderscore v. t.\textunderscore  (e der.)
O mesmo que \textunderscore preguear\textunderscore .
\section{Emprêgo}
\begin{itemize}
\item {Grp. gram.:m.}
\end{itemize}
Acto ou effeito de empregar.
Situação ou funcções de quem faz serviço em repartição pública ou estabelecimento particular.
Collocação: \textunderscore o Felix já tem emprêgo\textunderscore .
Applicação ou exercício de quaesquer recursos ou faculdades: \textunderscore emprêgo de capitaes\textunderscore .
\section{Empregomania}
\begin{itemize}
\item {fónica:prê}
\end{itemize}
\begin{itemize}
\item {Grp. gram.:f.}
\end{itemize}
\begin{itemize}
\item {Utilização:Neol.}
\end{itemize}
\begin{itemize}
\item {Proveniência:(De \textunderscore emprêgo\textunderscore  + \textunderscore mania\textunderscore )}
\end{itemize}
Mania dos que preferem empregos públicos a qualquer outro meio de vida.
\section{Empregomaniaco}
\begin{itemize}
\item {fónica:prê}
\end{itemize}
\begin{itemize}
\item {Grp. gram.:adj.}
\end{itemize}
\begin{itemize}
\item {Grp. gram.:M.}
\end{itemize}
Relativo á empregomania.
Aquelle que tem a empregomania.
\section{Empreguiçar}
\begin{itemize}
\item {Grp. gram.:v. t.}
\end{itemize}
\begin{itemize}
\item {Proveniência:(De \textunderscore preguiça\textunderscore )}
\end{itemize}
Tornar preguiçoso.
\section{Emprehendedor}
\begin{itemize}
\item {Grp. gram.:adj.}
\end{itemize}
\begin{itemize}
\item {Grp. gram.:M.}
\end{itemize}
\begin{itemize}
\item {Proveniência:(De \textunderscore emprehender\textunderscore )}
\end{itemize}
Que emprehende.
Activo.
Arrojado.
Aquelle que emprehende ou toma a seu cargo uma empresa.
\section{Emprehender}
\begin{itemize}
\item {Grp. gram.:v. t.}
\end{itemize}
\begin{itemize}
\item {Grp. gram.:V. i.}
\end{itemize}
\begin{itemize}
\item {Utilização:Pop.}
\end{itemize}
\begin{itemize}
\item {Proveniência:(Do lat. \textunderscore prehendere\textunderscore )}
\end{itemize}
Resolver.
Praticar.
Decidir-se a, não obstante as difficuldades.
Têr apprehensões, scismas; apprehender.
\section{Emprehendimento}
\begin{itemize}
\item {Grp. gram.:m.}
\end{itemize}
Acto de emprehender.
Empresa; tentame.
\section{Empreita}
\begin{itemize}
\item {Grp. gram.:f.}
\end{itemize}
\begin{itemize}
\item {Proveniência:(Do lat. \textunderscore implicita\textunderscore )}
\end{itemize}
Tira de esparto.
Cincho.
\section{Empreita}
\begin{itemize}
\item {Grp. gram.:f.}
\end{itemize}
O mesmo que \textunderscore empreitada\textunderscore . Cf. Júl. Moreira, \textunderscore Est. da Ling. Port.\textunderscore , I, 186.
(Cp. cast. \textunderscore pleita\textunderscore )
\section{Empreitada}
\begin{itemize}
\item {Grp. gram.:f.}
\end{itemize}
\begin{itemize}
\item {Proveniência:(De \textunderscore empreitar\textunderscore )}
\end{itemize}
Obra de empreitas, trabalho de esparteiro.
Contrato, em que um ou mais indivíduos se encarregam de fazer certa obra para outrem, mediante retribuição proporcionada á quantidade do trabalho executado.
Tarefa; trabalho que se ajusta em globo e não a dias.
\section{Empreitado}
\begin{itemize}
\item {Grp. gram.:adj.}
\end{itemize}
\begin{itemize}
\item {Utilização:Des.}
\end{itemize}
Que se ajustou por empreitada.
(Cp. \textunderscore empreitada\textunderscore )
\section{Empreitar}
\begin{itemize}
\item {Grp. gram.:v. t.}
\end{itemize}
Tomar por empreitada; fazer por empreitada.
\section{Empreiteiro}
\begin{itemize}
\item {Grp. gram.:m.}
\end{itemize}
\begin{itemize}
\item {Proveniência:(De \textunderscore empreita\textunderscore )}
\end{itemize}
Aquelle que se encarrega de uma obra por empreitada.
\section{Emprender}
\textunderscore v. t.\textunderscore  (e der.)
(V. \textunderscore emprehender\textunderscore , etc.)
\section{Emprenhador}
\begin{itemize}
\item {Grp. gram.:m.}
\end{itemize}
Peixe de Portugal.
\section{Emprenhar}
\begin{itemize}
\item {Grp. gram.:v. t.}
\end{itemize}
\begin{itemize}
\item {Grp. gram.:V. i.}
\end{itemize}
Tornar prenhe (mulher ou fêmea de animal).
Tornar-se prenhe.
\section{Emprenhidão}
\begin{itemize}
\item {Grp. gram.:f.}
\end{itemize}
O mesmo que \textunderscore prenhez\textunderscore . Cf. Brito, \textunderscore Monarch. Lusit.\textunderscore , I, 62.
\section{Emprensivo}
\begin{itemize}
\item {Grp. gram.:adj.}
\end{itemize}
\begin{itemize}
\item {Utilização:Des.}
\end{itemize}
O mesmo que \textunderscore apreensivo\textunderscore . Cf. \textunderscore Alex. Lobo\textunderscore , III, 71.
\section{Empresa}
\begin{itemize}
\item {Grp. gram.:f.}
\end{itemize}
\begin{itemize}
\item {Utilização:Des.}
\end{itemize}
\begin{itemize}
\item {Proveniência:(Do lat. \textunderscore prehensus\textunderscore )}
\end{itemize}
Emprehendimento.
Negócio.
Associação, organizada para explorar uma indústria.
Aquelles que dirigem ou administram essa associação.
Symbolo, divisa.
\section{Empresador}
\begin{itemize}
\item {Grp. gram.:m.}
\end{itemize}
Aquelle que empresa ou apresa (caça grossa). Cf. Corvo, \textunderscore Anno na Côrte\textunderscore , III, 28.
\section{Empresar}
\begin{itemize}
\item {Grp. gram.:v. t.}
\end{itemize}
\begin{itemize}
\item {Proveniência:(De \textunderscore presa\textunderscore )}
\end{itemize}
O mesmo que \textunderscore represar\textunderscore .
O mesmo que \textunderscore apresar\textunderscore .
\section{Empresário}
\begin{itemize}
\item {Grp. gram.:m.}
\end{itemize}
\begin{itemize}
\item {Grp. gram.:Adj.}
\end{itemize}
\begin{itemize}
\item {Proveniência:(De \textunderscore empresa\textunderscore )}
\end{itemize}
Aquelle que emprehende negociação ou estabelecimento de uso público.
Aquelle que dirige ou administra uma empresa.
Relativo a empresa:«\textunderscore operações empresárias\textunderscore ». Camillo, \textunderscore Mulher Fatal\textunderscore , 141.
\section{Emprestador}
\begin{itemize}
\item {Grp. gram.:m.}
\end{itemize}
Aquelle que empresta.
\section{Emprestar}
\begin{itemize}
\item {Grp. gram.:v. t.}
\end{itemize}
\begin{itemize}
\item {Utilização:Bras}
\end{itemize}
\begin{itemize}
\item {Proveniência:(De \textunderscore prestar\textunderscore )}
\end{itemize}
Confiar temporariamente (qualquer coisa), com a condição de lhe sêr restituida na mesma espécie ou em coisa equivalente: \textunderscore emprestar um cavallo; emprestar 4 libras\textunderscore . Cf. \textunderscore Código Civil\textunderscore , art. 1506.
Ceder.
Communicar; dar: \textunderscore o orvalho empresta viço ás plantas\textunderscore .
Receber por empréstimo: \textunderscore quanto queres emprestar de mim\textunderscore ?
\section{Empréstimo}
\begin{itemize}
\item {Grp. gram.:m.}
\end{itemize}
\begin{itemize}
\item {Proveniência:(De \textunderscore préstimo\textunderscore )}
\end{itemize}
Acto ou effeito de emprestar.
Aquillo que se emprestou.
Contrato, em que se cede gratuitamente uma coisa, para que a pessôa, a quem é cedida, se sirva della, obrigando-se a restitui-la em espécie ou em coisa equivalente.
\section{Emprestor}
\begin{itemize}
\item {Grp. gram.:m.}
\end{itemize}
\begin{itemize}
\item {Utilização:Ant.}
\end{itemize}
O mesmo que \textunderscore emprestador\textunderscore .
\section{Emprimo}
\begin{itemize}
\item {Grp. gram.:adv.}
\end{itemize}
\begin{itemize}
\item {Utilização:Ant.}
\end{itemize}
Primeiramente.
(Da loc. lat. \textunderscore in primo\textunderscore )
\section{Emprir}
\begin{itemize}
\item {Grp. gram.:v. t.}
\end{itemize}
\begin{itemize}
\item {Utilização:Ant.}
\end{itemize}
\begin{itemize}
\item {Proveniência:(Do lat. \textunderscore implere\textunderscore )}
\end{itemize}
(V.encher)
\section{Emprisionar}
\begin{itemize}
\item {Grp. gram.:v. t.}
\end{itemize}
O mesmo que \textunderscore aprisionar\textunderscore .
\section{Emproado}
\begin{itemize}
\item {Grp. gram.:adj.}
\end{itemize}
\begin{itemize}
\item {Utilização:Fig.}
\end{itemize}
\begin{itemize}
\item {Proveniência:(De \textunderscore emproar\textunderscore )}
\end{itemize}
Orgulhoso; vaidoso.
\section{Emproar}
\begin{itemize}
\item {Grp. gram.:v. t.}
\end{itemize}
\begin{itemize}
\item {Grp. gram.:V. i.}
\end{itemize}
\begin{itemize}
\item {Grp. gram.:V. p.}
\end{itemize}
\begin{itemize}
\item {Utilização:Fig.}
\end{itemize}
\begin{itemize}
\item {Proveniência:(De \textunderscore proa\textunderscore )}
\end{itemize}
Voltar a prôa de (um navio).
Aproar. Abalroar de prôa.
Ensoberbecer-se.
\section{Emprosthótono}
\begin{itemize}
\item {Grp. gram.:m.}
\end{itemize}
\begin{itemize}
\item {Proveniência:(Do gr. \textunderscore emprosthen\textunderscore  + \textunderscore tonos\textunderscore )}
\end{itemize}
Contracção espasmódica, que obriga o corpo a curvar-se para deante.
\section{Emprostótono}
\begin{itemize}
\item {Grp. gram.:m.}
\end{itemize}
\begin{itemize}
\item {Proveniência:(Do gr. \textunderscore emprosthen\textunderscore  + \textunderscore tonos\textunderscore )}
\end{itemize}
Contracção espasmódica, que obriga o corpo a curvar-se para deante.
\section{Empubescer}
\begin{itemize}
\item {Grp. gram.:v. i.  e  p.}
\end{itemize}
\begin{itemize}
\item {Proveniência:(Do lat. \textunderscore pubescere\textunderscore )}
\end{itemize}
Torna-se púbere.
Criar pelos.
\section{Empulhação}
\begin{itemize}
\item {Grp. gram.:f.}
\end{itemize}
Acto de empulhar. Cf. M. Assis, \textunderscore B. Cubas.\textunderscore 
\section{Empulhar}
\begin{itemize}
\item {Grp. gram.:v. t.}
\end{itemize}
\begin{itemize}
\item {Utilização:Chul.}
\end{itemize}
Dirigir pulhas a.
Troçar de.
Lograr.
\section{Empunhadura}
\begin{itemize}
\item {Grp. gram.:f.}
\end{itemize}
\begin{itemize}
\item {Proveniência:(De \textunderscore empunhar\textunderscore )}
\end{itemize}
Lugar, por onde se empunha uma arma.
\section{Empunhar}
\begin{itemize}
\item {Grp. gram.:v. t.}
\end{itemize}
\begin{itemize}
\item {Proveniência:(De \textunderscore punho\textunderscore )}
\end{itemize}
Segurar pelo cabo ou pelo punho (uma arma, bengala, etc.).
Suster.
Pegar em: \textunderscore empunhar um saco\textunderscore .
Fazer o punho de (uma arma).
\section{Empunidoiros}
\begin{itemize}
\item {Grp. gram.:m. pl.}
\end{itemize}
\begin{itemize}
\item {Utilização:Náut.}
\end{itemize}
\begin{itemize}
\item {Proveniência:(De \textunderscore empunir\textunderscore )}
\end{itemize}
Garrunchos, em que se passam as empuniduras, quando as velas entram nos rizes.
\section{Empunidouros}
\begin{itemize}
\item {Grp. gram.:m. pl.}
\end{itemize}
\begin{itemize}
\item {Utilização:Náut.}
\end{itemize}
\begin{itemize}
\item {Proveniência:(De \textunderscore empunir\textunderscore )}
\end{itemize}
Garrunchos, em que se passam as empuniduras, quando as velas entram nos rizes.
\section{Empunidura}
\begin{itemize}
\item {Grp. gram.:f.}
\end{itemize}
\begin{itemize}
\item {Utilização:Náut.}
\end{itemize}
\begin{itemize}
\item {Proveniência:(De \textunderscore empunir\textunderscore )}
\end{itemize}
Cabo, com que se amarra a vela, quando esta se introduz nos rizes.
\section{Empunir}
\begin{itemize}
\item {Grp. gram.:v.}
\end{itemize}
\begin{itemize}
\item {Utilização:t. Náut.}
\end{itemize}
\begin{itemize}
\item {Proveniência:(Do rad. de \textunderscore punho\textunderscore ?)}
\end{itemize}
Amarrar (cabos) aos cunhos das vêrgas.
\section{Empurra}
\begin{itemize}
\item {Grp. gram.:f.}
\end{itemize}
\begin{itemize}
\item {Utilização:Pop.}
\end{itemize}
\begin{itemize}
\item {Proveniência:(De \textunderscore empurrar\textunderscore )}
\end{itemize}
\textunderscore Jôgo de empurra\textunderscore , acto de desviar de si qualquer incumbência, lançando-a á conta de outra.
Empate.
\section{Empurração}
\begin{itemize}
\item {Grp. gram.:f.}
\end{itemize}
O mesmo que \textunderscore empurra\textunderscore .
\section{Empurrão}
\begin{itemize}
\item {Grp. gram.:m.}
\end{itemize}
Acto de empurrar.
\section{Empurrar}
\begin{itemize}
\item {Grp. gram.:v. t.}
\end{itemize}
\begin{itemize}
\item {Utilização:pop.}
\end{itemize}
\begin{itemize}
\item {Utilização:Fig.}
\end{itemize}
Impellir com fôrça.
Dar encontrões em.
Impingir.
\section{Empuxador}
\begin{itemize}
\item {Grp. gram.:m.}
\end{itemize}
Aquelle que empuxa.
\section{Empuxamento}
\begin{itemize}
\item {Grp. gram.:m.}
\end{itemize}
O mesmo que \textunderscore empuxão\textunderscore .
\section{Empuxão}
\begin{itemize}
\item {Grp. gram.:m.}
\end{itemize}
Acto de empuxar.
\section{Empuxar}
\begin{itemize}
\item {Grp. gram.:v. t.}
\end{itemize}
\begin{itemize}
\item {Proveniência:(De \textunderscore puxar\textunderscore )}
\end{itemize}
Empurrar; impellir.
Arrastar para si.
\section{Empuxo}
\begin{itemize}
\item {Grp. gram.:m.}
\end{itemize}
Acto de empuxar.
Pressão de terra, abóbada ou arco, nos seus encontros ou supportes.
\section{Empyema}
\begin{itemize}
\item {Grp. gram.:m.}
\end{itemize}
\begin{itemize}
\item {Utilização:Med.}
\end{itemize}
\begin{itemize}
\item {Proveniência:(Gr. \textunderscore empuema\textunderscore )}
\end{itemize}
Ajuntamento de pus ou sangue numa cavidade do corpo.
Acumulação serosa, sanguínea ou purulenta na cavidade da pleura.
Operação, para dar saída a essas matérias.
\section{Empyemático}
\begin{itemize}
\item {Grp. gram.:adj.}
\end{itemize}
Que tem empyema.
\section{Empýreo}
\begin{itemize}
\item {Grp. gram.:m.}
\end{itemize}
\begin{itemize}
\item {Grp. gram.:Adj.}
\end{itemize}
\begin{itemize}
\item {Proveniência:(Do gr. \textunderscore en\textunderscore  + \textunderscore pur\textunderscore )}
\end{itemize}
Esphera, em que a Astronomia antiga considerava fixados os astros.
Morada dos deuses, segundo o paganismo.
Lugar dos bem-aventurados e santos.
Céu.
Celeste; supremo.
\section{Empyreuma}
\begin{itemize}
\item {Grp. gram.:m.}
\end{itemize}
\begin{itemize}
\item {Utilização:Chím.}
\end{itemize}
\begin{itemize}
\item {Proveniência:(Gr. \textunderscore empureuma\textunderscore )}
\end{itemize}
Sabor e cheio desagradável, procedente de substância orgânica, submetida á acção de fogo violento.
\section{Empyreumático}
\begin{itemize}
\item {Grp. gram.:adj.}
\end{itemize}
Em que há empyreuma: \textunderscore óleo empyreumático\textunderscore .
\section{Em-quanto}
\begin{itemize}
\item {Grp. gram.:conj.}
\end{itemize}
\begin{itemize}
\item {Proveniência:(De \textunderscore em\textunderscore  + \textunderscore quanto\textunderscore )}
\end{itemize}
No tempo em que.
Ao passo que.
\section{Emtanto}
\begin{itemize}
\item {Grp. gram.:adv.}
\end{itemize}
(Fórma incorrecta, por \textunderscore entanto\textunderscore )
\section{Emulação}
\begin{itemize}
\item {Grp. gram.:f.}
\end{itemize}
\begin{itemize}
\item {Proveniência:(Lat. \textunderscore aemulatio\textunderscore )}
\end{itemize}
Sentimento, que incita a imitar ou a exceder outrem.
Competência.
Estímulo; rivalidade.
\section{Emulador}
\begin{itemize}
\item {Grp. gram.:m.  e  adj.}
\end{itemize}
O mesmo que \textunderscore êmulo\textunderscore .
\section{Emular}
\begin{itemize}
\item {Grp. gram.:v. i.}
\end{itemize}
\begin{itemize}
\item {Grp. gram.:V. t.}
\end{itemize}
\begin{itemize}
\item {Proveniência:(Lat. \textunderscore aemulari\textunderscore )}
\end{itemize}
Têr emulação de alguém.
Rivalizar.
Tornar-se igual.
Rivalizar com.
Disputar preferências a.
Pôr-se a par de.
\section{Emulativo}
\begin{itemize}
\item {Grp. gram.:adj.}
\end{itemize}
Que produz emulação. Cf. Filinto, XVIII, 247.
\section{Emulgente}
\begin{itemize}
\item {Grp. gram.:adj.}
\end{itemize}
\begin{itemize}
\item {Utilização:Anat.}
\end{itemize}
\begin{itemize}
\item {Proveniência:(Lat. \textunderscore emulgens\textunderscore )}
\end{itemize}
Diz-se das artérias, que levam o sangue aos rins.
\section{Êmulo}
\begin{itemize}
\item {Grp. gram.:m.  e  adj.}
\end{itemize}
\begin{itemize}
\item {Proveniência:(Lat. \textunderscore aemulus\textunderscore )}
\end{itemize}
Aquelle que tem emulação de alguém.
Competidor.
Concorrente.
Adversário; rival.
\section{Emulsão}
\begin{itemize}
\item {Grp. gram.:f.}
\end{itemize}
\begin{itemize}
\item {Proveniência:(Lat. \textunderscore emulsio\textunderscore )}
\end{itemize}
Preparação pharmacêutica, extrahida de sementes emulsivas, e que geralmente dá a apparência do leite na côr e na consistência.
Mistura de uma substância oleosa com água.
\section{Emulsionar}
\begin{itemize}
\item {Grp. gram.:v. t.}
\end{itemize}
\begin{itemize}
\item {Proveniência:(Do lat. \textunderscore emulsio\textunderscore )}
\end{itemize}
Fazer emulsão de.
\section{Emulsivo}
\begin{itemize}
\item {Grp. gram.:adj.}
\end{itemize}
\begin{itemize}
\item {Proveniência:(Do lat. \textunderscore emulsus\textunderscore )}
\end{itemize}
De que se póde extrahir óleo, por meio de pressão.
\section{Emunctório}
\begin{itemize}
\item {Grp. gram.:m.  e  adj.}
\end{itemize}
\begin{itemize}
\item {Utilização:Med.}
\end{itemize}
\begin{itemize}
\item {Proveniência:(Lat. \textunderscore emunctorium\textunderscore )}
\end{itemize}
Diz-se de certos órgãos, que servem para a descarga de humores.
Próprio para a evacuação de humores.
\section{Emundação}
\begin{itemize}
\item {Grp. gram.:f.}
\end{itemize}
\begin{itemize}
\item {Utilização:Des.}
\end{itemize}
\begin{itemize}
\item {Proveniência:(Lat. \textunderscore emundatio\textunderscore )}
\end{itemize}
Purificação.
Limpeza.
\section{Em-vluemos}
\begin{itemize}
\item {Grp. gram.:loc. adv.}
\end{itemize}
\begin{itemize}
\item {Utilização:Prov.}
\end{itemize}
\begin{itemize}
\item {Utilização:trasm.}
\end{itemize}
Perplexamente; em dúvida.
(Corr. de \textunderscore em\textunderscore  + \textunderscore vê-lo-emos\textunderscore )
\section{En...}
\begin{itemize}
\item {Grp. gram.:pref.}
\end{itemize}
\begin{itemize}
\item {Proveniência:(Do lat. \textunderscore in\textunderscore )}
\end{itemize}
Vale o mesmo que \textunderscore em...\textunderscore 
\section{En}
\begin{itemize}
\item {Grp. gram.:prep.}
\end{itemize}
\begin{itemize}
\item {Utilização:Ant.}
\end{itemize}
O mesmo que \textunderscore em\textunderscore .
\section{Ena!}
\begin{itemize}
\item {Grp. gram.:interj.}
\end{itemize}
\begin{itemize}
\item {Utilização:Burl.}
\end{itemize}
Eia! Oh! Ápage!
\section{Enálage}
\begin{itemize}
\item {Grp. gram.:f.}
\end{itemize}
\begin{itemize}
\item {Proveniência:(Gr. \textunderscore enallage\textunderscore )}
\end{itemize}
Figura gramatical, em que, depois de se empregar um modo, se passa subitamente para outro, que não é admitido pela construcção ordinária.
\section{Enállage}
\begin{itemize}
\item {Grp. gram.:f.}
\end{itemize}
\begin{itemize}
\item {Proveniência:(Gr. \textunderscore enallage\textunderscore )}
\end{itemize}
Figura grammatical, em que, depois de se empregar um modo, se passa subitamente para outro, que não é admittido pela construcção ordinária.
\section{Enaltar}
\begin{itemize}
\item {Grp. gram.:v. t.}
\end{itemize}
O mesmo que \textunderscore enaltecer\textunderscore . Cf. Camillo, \textunderscore Narcót\textunderscore , II, 246.
\section{Enaltecer}
\begin{itemize}
\item {Grp. gram.:v. t.}
\end{itemize}
\begin{itemize}
\item {Proveniência:(De \textunderscore en...\textunderscore  + \textunderscore alto\textunderscore )}
\end{itemize}
Tornar alto, engrandecer.
Exaltar: \textunderscore enaltecer os méritos de alguém\textunderscore .
\section{Enaltecimento}
\begin{itemize}
\item {Grp. gram.:m.}
\end{itemize}
Acto de enaltecer.
\section{Enamorar}
\begin{itemize}
\item {Grp. gram.:v. t.}
\end{itemize}
\begin{itemize}
\item {Grp. gram.:V. p.}
\end{itemize}
Encantar.
Enfeitiçar.
Tornar apaixonado.
Deixar-se possuir de amor.
Apaixonar-se.
(Cast. \textunderscore enamorare\textunderscore , do lat. hyp. \textunderscore inamorare\textunderscore )
\section{Enano}
\begin{itemize}
\item {Grp. gram.:m.}
\end{itemize}
\begin{itemize}
\item {Utilização:Ant.}
\end{itemize}
O mesmo que \textunderscore enão\textunderscore .
\section{Enantal}
\begin{itemize}
\item {Grp. gram.:m.}
\end{itemize}
\begin{itemize}
\item {Proveniência:(De \textunderscore enanto\textunderscore )}
\end{itemize}
Essência, que se obtém destilando óleo de rícino.
\section{Enantema}
\begin{itemize}
\item {Grp. gram.:m.}
\end{itemize}
\begin{itemize}
\item {Utilização:Med.}
\end{itemize}
\begin{itemize}
\item {Proveniência:(Do gr. \textunderscore en\textunderscore  + \textunderscore anthein\textunderscore , florescer)}
\end{itemize}
Erupção na face interna das cavidades naturaes, (o contrário de exantema).
\section{Enantéreas}
\begin{itemize}
\item {Grp. gram.:f. pl.}
\end{itemize}
\begin{itemize}
\item {Proveniência:(De \textunderscore enanto\textunderscore )}
\end{itemize}
Família de plantas, também chamadas onagrárias, e que têm por tipo a onagra.
\section{Enanthal}
\begin{itemize}
\item {Grp. gram.:m.}
\end{itemize}
\begin{itemize}
\item {Proveniência:(De \textunderscore enantho\textunderscore )}
\end{itemize}
Essência, que se obtém destillando óleo de rícino.
\section{Enanthema}
\begin{itemize}
\item {Grp. gram.:m.}
\end{itemize}
\begin{itemize}
\item {Utilização:Med.}
\end{itemize}
\begin{itemize}
\item {Proveniência:(Do gr. \textunderscore en\textunderscore  + \textunderscore anthein\textunderscore , florescer)}
\end{itemize}
Erupção na face interna das cavidades naturaes, (o contrário de exanthema).
\section{Enanthéreas}
\begin{itemize}
\item {Grp. gram.:f. pl.}
\end{itemize}
\begin{itemize}
\item {Proveniência:(De \textunderscore enantho\textunderscore )}
\end{itemize}
Família de plantas, também chamadas onagrárias, e que têm por typo a onagra.
\section{Enânthico}
\begin{itemize}
\item {Grp. gram.:adj.}
\end{itemize}
\begin{itemize}
\item {Proveniência:(De \textunderscore enantho\textunderscore )}
\end{itemize}
Relativo ao aroma dos vinhos.
\section{Enanthina}
\begin{itemize}
\item {Grp. gram.:f.}
\end{itemize}
\begin{itemize}
\item {Proveniência:(De \textunderscore enantho\textunderscore )}
\end{itemize}
Substância viscosa, a que se attribue o aroma dos vinhos de Bordéus.
\section{Enantho}
\begin{itemize}
\item {Grp. gram.:m.}
\end{itemize}
\begin{itemize}
\item {Utilização:Ant.}
\end{itemize}
\begin{itemize}
\item {Proveniência:(Do gr. \textunderscore oinos\textunderscore  + \textunderscore anthos\textunderscore )}
\end{itemize}
Nome de várias plantas umbellíferas.
Florescência da videira brava.
Videira brava.
Filipêndula.
\section{Enanthýlico}
\begin{itemize}
\item {Grp. gram.:adj.}
\end{itemize}
O mesmo que \textunderscore heptýlico\textunderscore .
\section{Enântico}
\begin{itemize}
\item {Grp. gram.:adj.}
\end{itemize}
\begin{itemize}
\item {Proveniência:(De \textunderscore enanto\textunderscore )}
\end{itemize}
Relativo ao aroma dos vinhos.
\section{Enantílico}
\begin{itemize}
\item {Grp. gram.:adj.}
\end{itemize}
O mesmo que \textunderscore heptílico\textunderscore .
\section{Enantina}
\begin{itemize}
\item {Grp. gram.:f.}
\end{itemize}
\begin{itemize}
\item {Proveniência:(De \textunderscore enanto\textunderscore )}
\end{itemize}
Substância viscosa, a que se atribue o aroma dos vinhos de Bordéus.
\section{Enantiopathia}
\begin{itemize}
\item {Grp. gram.:f.}
\end{itemize}
\begin{itemize}
\item {Proveniência:(Do gr. \textunderscore enantios\textunderscore , contrário, e \textunderscore pathos\textunderscore , doença)}
\end{itemize}
O mesmo que \textunderscore allopathia\textunderscore .
\section{Enantiopathicamente}
\begin{itemize}
\item {Grp. gram.:adv.}
\end{itemize}
De modo enantiopáthico.
\section{Enantiopáthico}
\begin{itemize}
\item {Grp. gram.:adj.}
\end{itemize}
Relativo a enantiopathia.
Que cura uma doença, actuando no organismo em sentido inverso della.
\section{Enantiopatia}
\begin{itemize}
\item {Grp. gram.:f.}
\end{itemize}
\begin{itemize}
\item {Proveniência:(Do gr. \textunderscore enantios\textunderscore , contrário, e \textunderscore pathos\textunderscore , doença)}
\end{itemize}
O mesmo que \textunderscore alopatia\textunderscore .
\section{Enantiopaticamente}
\begin{itemize}
\item {Grp. gram.:adv.}
\end{itemize}
De modo enantiopático.
\section{Enantiopático}
\begin{itemize}
\item {Grp. gram.:adj.}
\end{itemize}
Relativo a enantiopatia.
Que cura uma doença, actuando no organismo em sentido inverso dela.
\section{Enantiose}
\begin{itemize}
\item {Grp. gram.:f.}
\end{itemize}
\begin{itemize}
\item {Utilização:Gram.}
\end{itemize}
\begin{itemize}
\item {Utilização:Philos.}
\end{itemize}
\begin{itemize}
\item {Proveniência:(Gr. \textunderscore enantiosis\textunderscore )}
\end{itemize}
Tratamento enantiopáthico das doenças.
Antithese.
Cada uma das déz opposições que, segundo a philosophia de Pythágoras, são a origem de todas as coisas.
\section{Enanto}
\begin{itemize}
\item {Grp. gram.:m.}
\end{itemize}
\begin{itemize}
\item {Utilização:Ant.}
\end{itemize}
\begin{itemize}
\item {Proveniência:(Do gr. \textunderscore oinos\textunderscore  + \textunderscore anthos\textunderscore )}
\end{itemize}
Nome de várias plantas umbelíferas.
Florescência da videira brava.
Videira brava.
Filipêndula.
\section{Enão}
\begin{itemize}
\item {Grp. gram.:m.}
\end{itemize}
\begin{itemize}
\item {Utilização:Ant.}
\end{itemize}
O mesmo que \textunderscore anão\textunderscore . Cp. \textunderscore Apólogos Dialogaes\textunderscore , 414.
\section{Enapupês}
\begin{itemize}
\item {Grp. gram.:m.}
\end{itemize}
\begin{itemize}
\item {Utilização:Bras}
\end{itemize}
Espécie de perdiz grande, de bico comprido.
\section{Enargia}
\begin{itemize}
\item {Grp. gram.:f.}
\end{itemize}
\begin{itemize}
\item {Proveniência:(Gr. \textunderscore enargeia\textunderscore )}
\end{itemize}
Representação fiel de qualquer coisa, num discurso.
\section{Enarrar}
\begin{itemize}
\item {Proveniência:(Lat. \textunderscore enarrare\textunderscore )}
\end{itemize}
\textunderscore v. t.\textunderscore (e der.)
O mesmo que \textunderscore narrar\textunderscore , etc.
\section{Enarthrose}
\begin{itemize}
\item {Grp. gram.:f.}
\end{itemize}
\begin{itemize}
\item {Utilização:Anat.}
\end{itemize}
\begin{itemize}
\item {Proveniência:(Do gr. \textunderscore en\textunderscore  + \textunderscore arthron\textunderscore )}
\end{itemize}
Articulação móvel, formada por eminência arredondada, com encaixe numa cavidade profunda.
\section{Enartrose}
\begin{itemize}
\item {Grp. gram.:f.}
\end{itemize}
\begin{itemize}
\item {Utilização:Anat.}
\end{itemize}
\begin{itemize}
\item {Proveniência:(Do gr. \textunderscore en\textunderscore  + \textunderscore arthron\textunderscore )}
\end{itemize}
Articulação móvel, formada por eminência arredondada, com encaixe numa cavidade profunda.
\section{Enase}
\begin{itemize}
\item {Grp. gram.:f.}
\end{itemize}
\begin{itemize}
\item {Utilização:Pharm.}
\end{itemize}
\begin{itemize}
\item {Proveniência:(Do gr. \textunderscore oinos\textunderscore )}
\end{itemize}
Fermento de vinho, em fórma de pastilhas.
\section{...ença}
\begin{itemize}
\item {Grp. gram.:suf.}
\end{itemize}
(Contr. do suf. \textunderscore ência\textunderscore )
\section{Encabadoiro}
\begin{itemize}
\item {Grp. gram.:m.}
\end{itemize}
\begin{itemize}
\item {Proveniência:(De \textunderscore encabar\textunderscore )}
\end{itemize}
Abertura, em que entra o cabo de qualquer instrumento de metal.
\section{Encabadouro}
\begin{itemize}
\item {Grp. gram.:m.}
\end{itemize}
\begin{itemize}
\item {Proveniência:(De \textunderscore encabar\textunderscore )}
\end{itemize}
Abertura, em que entra o cabo de qualquer instrumento de metal.
\section{Encabar}
\begin{itemize}
\item {Grp. gram.:v. t.}
\end{itemize}
\begin{itemize}
\item {Utilização:Fig.}
\end{itemize}
\begin{itemize}
\item {Utilização:Prov.}
\end{itemize}
\begin{itemize}
\item {Utilização:minh.}
\end{itemize}
\begin{itemize}
\item {Proveniência:(De \textunderscore cabo\textunderscore )}
\end{itemize}
Meter numa abertura adequada o cabo de (utensilios, instrumentos, etc.): \textunderscore encabar uma enxada\textunderscore .
Encaixar.
Lograr, enganar.
\section{Encabeçamento}
\begin{itemize}
\item {Grp. gram.:m.}
\end{itemize}
Acto ou effeito de encabeçar.
\section{Encabeçar}
\begin{itemize}
\item {Grp. gram.:v.}
\end{itemize}
\begin{itemize}
\item {Utilização:t. Jur.}
\end{itemize}
\begin{itemize}
\item {Utilização:Fam.}
\end{itemize}
\begin{itemize}
\item {Proveniência:(De \textunderscore cabeça\textunderscore )}
\end{itemize}
Tornar (um prédio) cabeça de morgado.
Deferir a herança de.
Designar (a quota de um pagamento).
Dar posse de um prédio a.
Fazer o título ou o exórdio de (um escrito).
Meter em cabeça.
Convencer de.
Unir, accrescentar.
Remendar nas extremidades: \textunderscore encabeçar meias\textunderscore .
\section{Encabeira}
\begin{itemize}
\item {Grp. gram.:f.}
\end{itemize}
\begin{itemize}
\item {Proveniência:(De \textunderscore encabeirar\textunderscore )}
\end{itemize}
Tábua que, nos soalhos, se assenta ao longo das paredes, e na qual se vão encaixar as outras transversalmente.
\section{Encabeirar}
\begin{itemize}
\item {Grp. gram.:v. t.}
\end{itemize}
\begin{itemize}
\item {Utilização:Carp.}
\end{itemize}
O mesmo que encabar.
Forrar ou assoalhar com cabeiras.
\section{Encabelado}
\begin{itemize}
\item {Grp. gram.:adj.}
\end{itemize}
\begin{itemize}
\item {Grp. gram.:M. pl.}
\end{itemize}
\begin{itemize}
\item {Proveniência:(De \textunderscore encabelar\textunderscore )}
\end{itemize}
Que encabelou.
Que tem cabelo.
Nome, que os primeiros exploradores do Brasil deram a umas tríbos das margens de alguns afluentes do Amazonas, porque essas tríbos usavam cabelo comprido, que os envolvia até á cintura.
\section{Encabeladura}
\begin{itemize}
\item {Grp. gram.:f.}
\end{itemize}
Acto ou efeito de encabelar.
Cabeladura; cabeleira.
\section{Encabelar}
\begin{itemize}
\item {Grp. gram.:v. i.}
\end{itemize}
Criar cabelos ou pelos.
\section{Encabelizar}
\begin{itemize}
\item {Grp. gram.:v. t.}
\end{itemize}
\begin{itemize}
\item {Utilização:Neol.}
\end{itemize}
Cobrir de cabelos.
Fazer nascer cabelos em. Cf. Camillo, \textunderscore Ôlho de Vidro\textunderscore , 79.
\section{Encabellado}
\begin{itemize}
\item {Grp. gram.:adj.}
\end{itemize}
\begin{itemize}
\item {Grp. gram.:M. pl.}
\end{itemize}
\begin{itemize}
\item {Proveniência:(De \textunderscore encabellar\textunderscore )}
\end{itemize}
Que encabellou.
Que tem cabello.
Nome, que os primeiros exploradores do Brasil deram a umas tríbos das margens de alguns affluentes do Amazonas, porque essas tríbos usavam cabello comprido, que os envolvia até á cintura.
\section{Encabelladura}
\begin{itemize}
\item {Grp. gram.:f.}
\end{itemize}
Acto ou effeito de encabellar.
Cabelladura; cabelleira.
\section{Encabellar}
\begin{itemize}
\item {Grp. gram.:v. i.}
\end{itemize}
Criar cabellos ou pelos.
\section{Encabellizar}
\begin{itemize}
\item {Grp. gram.:v. t.}
\end{itemize}
\begin{itemize}
\item {Utilização:Neol.}
\end{itemize}
Cobrir de cabellos.
Fazer nascer cabellos em. Cf. Camillo, \textunderscore Ôlho de Vidro\textunderscore , 79.
\section{Encabrestadura}
\begin{itemize}
\item {Grp. gram.:f.}
\end{itemize}
\begin{itemize}
\item {Proveniência:(De \textunderscore encabrestar\textunderscore )}
\end{itemize}
Ferida nas quartelas das cavalgaduras, produzida por attrito de cabrestos, cordas, etc.
\section{Encabrestamento}
\begin{itemize}
\item {Grp. gram.:m.}
\end{itemize}
Acto de encabrestar.
\section{Encabrestar}
\begin{itemize}
\item {Grp. gram.:v. t.}
\end{itemize}
\begin{itemize}
\item {Utilização:Fig.}
\end{itemize}
\begin{itemize}
\item {Grp. gram.:V. p.}
\end{itemize}
\begin{itemize}
\item {Proveniência:(Do lat. \textunderscore incapistrare\textunderscore )}
\end{itemize}
Pôr cabresto a.
Sujeitar.
Embaraçar-se no cabresto (a cavalgadura).
\section{Encabritar-se}
\begin{itemize}
\item {Grp. gram.:v. p.}
\end{itemize}
\begin{itemize}
\item {Proveniência:(De \textunderscore cabrito\textunderscore )}
\end{itemize}
Marinhar, trepar.
Alçar-se; empinar-se.
\section{Encabular}
\begin{itemize}
\item {Grp. gram.:v. t.  e  i.}
\end{itemize}
\begin{itemize}
\item {Utilização:Bras}
\end{itemize}
Encafifar; amuar.
\section{Encachaçado}
\begin{itemize}
\item {Grp. gram.:adj.}
\end{itemize}
\begin{itemize}
\item {Utilização:Bras. do N}
\end{itemize}
Embriagado com cachaça.
Ébrio.
\section{Encachapução}
\begin{itemize}
\item {Grp. gram.:m.}
\end{itemize}
O mesmo que \textunderscore cachapução\textunderscore .
\section{Encachapuçar}
\begin{itemize}
\item {Grp. gram.:v. i.}
\end{itemize}
\begin{itemize}
\item {Utilização:Prov.}
\end{itemize}
\begin{itemize}
\item {Utilização:trasm.}
\end{itemize}
Dar encachapução.
\section{Encachar}
\begin{itemize}
\item {Grp. gram.:v. t.}
\end{itemize}
Cobrir com encacho.
\section{Encachiado}
\begin{itemize}
\item {Grp. gram.:adj.}
\end{itemize}
\begin{itemize}
\item {Utilização:Fig.}
\end{itemize}
\begin{itemize}
\item {Proveniência:(De \textunderscore encachiar-se\textunderscore )}
\end{itemize}
Diz-se do peru e do pavão, quando se entufam e fazem roda com as pennas da cauda.
Vaidoso, presumido.
\section{Encachiar-se}
\begin{itemize}
\item {Grp. gram.:v. p.}
\end{itemize}
Entufar-se (o peru ou o pavão) e fazer roda com as pennas da cauda.
\section{Encacho}
\begin{itemize}
\item {Grp. gram.:m.}
\end{itemize}
\begin{itemize}
\item {Proveniência:(Do rad. de \textunderscore cacha\textunderscore ^2)}
\end{itemize}
O mesmo que \textunderscore tanga\textunderscore ^1.
\section{Encachoeirado}
\begin{itemize}
\item {Grp. gram.:adj.}
\end{itemize}
\begin{itemize}
\item {Utilização:Bras}
\end{itemize}
\begin{itemize}
\item {Utilização:Neol.}
\end{itemize}
Semelhante á cachoeira; que tem cachoeira.
\section{Encachoeiramento}
\begin{itemize}
\item {Grp. gram.:m.}
\end{itemize}
\begin{itemize}
\item {Utilização:Bras}
\end{itemize}
Formação de cachoeira; cachoeira.
\section{Encadeação}
\begin{itemize}
\item {Grp. gram.:f.}
\end{itemize}
O mesmo que \textunderscore encadeamento\textunderscore .
\section{Encadeamento}
\begin{itemize}
\item {Grp. gram.:m.}
\end{itemize}
Acto ou effeito de encadear.
Successão; série.
\section{Encadear}
\begin{itemize}
\item {Grp. gram.:v. t.}
\end{itemize}
Ligar com cadeia.
Meter em cadeia.
Lançar grilhões a.
Ligar, concatenar (coisas que seguem outras): \textunderscore encadear asneiras\textunderscore .
Ligar por affecto.
Segurar; acorrentar.
\section{Encadeirar}
\begin{itemize}
\item {Grp. gram.:v. t.}
\end{itemize}
Fazer sentar em cadeira.
\section{Encadernação}
\begin{itemize}
\item {Grp. gram.:f.}
\end{itemize}
\begin{itemize}
\item {Proveniência:(De \textunderscore encadernar\textunderscore )}
\end{itemize}
Acto de coser as fôlhas de um livro, sobrepondo-lhe capa consistente, afim de que elle melhor se conserve.
Capa de um livro encadernado.
\section{Encadernado}
\begin{itemize}
\item {Grp. gram.:adj.}
\end{itemize}
\begin{itemize}
\item {Utilização:Fam.}
\end{itemize}
\begin{itemize}
\item {Proveniência:(De \textunderscore encadernar\textunderscore )}
\end{itemize}
Coberto de encadernação: \textunderscore um livro encadernado\textunderscore .
Reunido em caderno.
Vestido, entrajado: \textunderscore vens hoje bem encadernado\textunderscore .
\section{Encadernador}
\begin{itemize}
\item {Grp. gram.:m.}
\end{itemize}
Aquelle que encaderna.
\section{Encadernar}
\begin{itemize}
\item {Grp. gram.:v. t.}
\end{itemize}
\begin{itemize}
\item {Utilização:Fam.}
\end{itemize}
\begin{itemize}
\item {Proveniência:(De \textunderscore caderno\textunderscore )}
\end{itemize}
Fazer encadernação de (livros).
Entrajar de novo.
\section{Encado}
\begin{itemize}
\item {Grp. gram.:m.}
\end{itemize}
\begin{itemize}
\item {Utilização:Prov.}
\end{itemize}
\begin{itemize}
\item {Utilização:trasm.}
\end{itemize}
O mesmo que \textunderscore estacoeiro\textunderscore .
\section{Encadro}
\begin{itemize}
\item {Grp. gram.:m.}
\end{itemize}
\begin{itemize}
\item {Utilização:Ant.}
\end{itemize}
Tanga.
(Por \textunderscore enquadro\textunderscore , de \textunderscore enquadrar?\textunderscore )
\section{Encafifar}
\begin{itemize}
\item {Grp. gram.:v. t.}
\end{itemize}
\begin{itemize}
\item {Utilização:Bras}
\end{itemize}
\begin{itemize}
\item {Grp. gram.:V. i.}
\end{itemize}
Envergonhar.
O mesmo que \textunderscore encalistrar\textunderscore .
\section{Encafuar}
\begin{itemize}
\item {Grp. gram.:v. t.}
\end{itemize}
Meter em cafua.
Encerrar; occultar.
\section{Encafurnar}
\begin{itemize}
\item {Grp. gram.:v. t.}
\end{itemize}
\begin{itemize}
\item {Proveniência:(De \textunderscore cafurna\textunderscore )}
\end{itemize}
O mesmo que \textunderscore encafuar\textunderscore .
\section{Encaibramento}
\begin{itemize}
\item {Grp. gram.:m.}
\end{itemize}
Acto ou effeito de \textunderscore encaibrar\textunderscore .
\section{Encaibrar}
\begin{itemize}
\item {Grp. gram.:v. t.}
\end{itemize}
Assentar os caibros de (um edifício).
\section{Encaiporar}
\begin{itemize}
\item {Grp. gram.:v. t.}
\end{itemize}
\begin{itemize}
\item {Utilização:Bras}
\end{itemize}
\begin{itemize}
\item {Proveniência:(De \textunderscore caipora\textunderscore )}
\end{itemize}
Tornar infeliz.
Contribuir para a infelicidade de.
Encalistar.
\section{Encaixamento}
\begin{itemize}
\item {Grp. gram.:m.}
\end{itemize}
Acto ou effeito de \textunderscore encaixar\textunderscore .
\section{Encaixar}
\begin{itemize}
\item {Grp. gram.:v. t.}
\end{itemize}
\begin{itemize}
\item {Proveniência:(Do lat. \textunderscore quassiare\textunderscore )}
\end{itemize}
Meter em caixa.
Emmalhetar.
Entalhar.
Meter em encaixe.
Introduzir.
Citar a propósito; trazer á balha: \textunderscore encaixar um texto da Bíblia\textunderscore . \textunderscore Fig.\textunderscore  Persuadir.
\section{Encaixe}
\begin{itemize}
\item {Grp. gram.:m.}
\end{itemize}
\begin{itemize}
\item {Utilização:Prov.}
\end{itemize}
\begin{itemize}
\item {Proveniência:(De \textunderscore encaixar\textunderscore )}
\end{itemize}
Cavidade, destinada a receber uma peça saliente.
Travamento do espigão de uma peça com o envasamento de outra.
Guarnição bordada ou de renda, na parte superior de uma camisa de senhora.
\section{Encaixilhar}
\begin{itemize}
\item {Grp. gram.:v. t.}
\end{itemize}
Emmoldurar.
Meter em caixilho.
Enquadrar.
\section{Encaixo}
\begin{itemize}
\item {Grp. gram.:m.}
\end{itemize}
(V.encaixe)
\section{Encaixotar}
\begin{itemize}
\item {Grp. gram.:v. t.}
\end{itemize}
\begin{itemize}
\item {Utilização:Gír.}
\end{itemize}
Meter em caixote, em caixa.
Enterrar, sepultar.
\section{Encalacração}
\begin{itemize}
\item {Grp. gram.:f.}
\end{itemize}
Acto ou effeito de encalacrar.
\section{Encalacradela}
\begin{itemize}
\item {Grp. gram.:f.}
\end{itemize}
\begin{itemize}
\item {Utilização:Fam.}
\end{itemize}
O mesmo que \textunderscore encalacração\textunderscore .
\section{Encalacrar}
\begin{itemize}
\item {Grp. gram.:v. t.}
\end{itemize}
\begin{itemize}
\item {Utilização:Pop.}
\end{itemize}
\begin{itemize}
\item {Grp. gram.:V. p.}
\end{itemize}
\begin{itemize}
\item {Proveniência:(De \textunderscore calacre\textunderscore )}
\end{itemize}
Lograr; collocar em difficuldades.
Meter em contenda judicial.
Endividar-se; collocar-se em difficuldade.
\section{Encalamento}
\begin{itemize}
\item {Grp. gram.:m.}
\end{itemize}
\begin{itemize}
\item {Utilização:Náut.}
\end{itemize}
Peça de madeira, que fortalece o navio, atravessando-lhe os braços.
\section{Encalamistrar}
\textunderscore v. t.\textunderscore  (e der.)
O mesmo que \textunderscore calamistrar\textunderscore , etc. Cf. Camillo, \textunderscore Freira no Subterr.\textunderscore , 56; \textunderscore Mulher Fatal\textunderscore , 123; \textunderscore Quéda de Um Anjo\textunderscore , 147.
\section{Encalamoucar}
\begin{itemize}
\item {Grp. gram.:v. t.}
\end{itemize}
\begin{itemize}
\item {Utilização:Pop.}
\end{itemize}
O mesmo que \textunderscore encalacrar\textunderscore .
\section{Encalcadeira}
\begin{itemize}
\item {Grp. gram.:f.}
\end{itemize}
\begin{itemize}
\item {Utilização:Serralh.}
\end{itemize}
Peça para encalcar.
\section{Encalcar}
\begin{itemize}
\item {Grp. gram.:v.}
\end{itemize}
\begin{itemize}
\item {Utilização:t. Serralh.}
\end{itemize}
Vedar as juntas de (duas peças do ferro).
\section{Encalçar}
\begin{itemize}
\item {Grp. gram.:v. t.}
\end{itemize}
Seguir as peugadas de.
Ir no encalço de.
\section{Encalço}
\begin{itemize}
\item {Grp. gram.:m.}
\end{itemize}
\begin{itemize}
\item {Proveniência:(Do lat. \textunderscore calx\textunderscore )}
\end{itemize}
Acto de seguir de perto.
Peugada.
Rasto; pista.
\section{Encaldar}
\begin{itemize}
\item {Grp. gram.:v. i.}
\end{itemize}
\begin{itemize}
\item {Utilização:Marn.}
\end{itemize}
Endurecer (o solo das marinhas). Cf. \textunderscore Museu Techn.\textunderscore , 72.
\section{Encaldeirar}
\begin{itemize}
\item {Grp. gram.:v. t.}
\end{itemize}
\begin{itemize}
\item {Proveniência:(De \textunderscore caldeira\textunderscore )}
\end{itemize}
Rodear com uma cova (árvores), para juntar água que as regue.
\section{Encalecer}
\begin{itemize}
\item {Grp. gram.:v. i.}
\end{itemize}
Criar calos.
\section{Encalecido}
\begin{itemize}
\item {Grp. gram.:adj.}
\end{itemize}
\begin{itemize}
\item {Proveniência:(De \textunderscore encalecer\textunderscore )}
\end{itemize}
Que tem calos; calejado.
\section{Encaleirar}
\begin{itemize}
\item {Grp. gram.:v. t.}
\end{itemize}
Meter em caleira; dirigir por caleira. Cf. Camillo, \textunderscore Sc. da Foz\textunderscore , 163.
\section{Encalgar}
\begin{itemize}
\item {Grp. gram.:v. t.}
\end{itemize}
\begin{itemize}
\item {Utilização:Pop.}
\end{itemize}
(V.encavalgar)
\section{Encalhação}
\begin{itemize}
\item {Grp. gram.:f.}
\end{itemize}
O mesmo que \textunderscore encalhe\textunderscore .
\section{Encalhado}
\begin{itemize}
\item {Grp. gram.:adj.}
\end{itemize}
\begin{itemize}
\item {Utilização:Bras. de Minas}
\end{itemize}
Que encalhou.
Que tem constipação de ventre.
\section{Encalhamento}
\begin{itemize}
\item {Grp. gram.:m.}
\end{itemize}
O mesmo que \textunderscore encalhe\textunderscore .
\section{Encalhar}
\begin{itemize}
\item {Grp. gram.:v. t.}
\end{itemize}
\begin{itemize}
\item {Grp. gram.:V. i.}
\end{itemize}
\begin{itemize}
\item {Utilização:Fig.}
\end{itemize}
\begin{itemize}
\item {Utilização:Gír.}
\end{itemize}
\begin{itemize}
\item {Proveniência:(De \textunderscore calha\textunderscore )}
\end{itemize}
Fazer dar em sêco (o navio).
Dar em sêco.
Achar impedimento: \textunderscore a pretensão encalhou no Ministério da Guerra\textunderscore .
Embaraçar-se.
Entrar.
\section{Encalhe}
\begin{itemize}
\item {Grp. gram.:m.}
\end{itemize}
Acto ou effeito de encalhar.
\section{Encalho}
\begin{itemize}
\item {Grp. gram.:m.}
\end{itemize}
\begin{itemize}
\item {Proveniência:(De \textunderscore encalhar\textunderscore )}
\end{itemize}
Lugar, em que o navio encalha.
Encalhe.
\section{Encaliçar}
\begin{itemize}
\item {Grp. gram.:v. t.}
\end{itemize}
Cobrir com caliça.
\section{Encalipto}
\begin{itemize}
\item {Grp. gram.:m.}
\end{itemize}
\begin{itemize}
\item {Proveniência:(Do gr. \textunderscore enkalupto\textunderscore )}
\end{itemize}
Gênero de musgos do norte.
\section{Encalir}
\begin{itemize}
\item {Grp. gram.:v. t.}
\end{itemize}
\begin{itemize}
\item {Utilização:Bras}
\end{itemize}
\begin{itemize}
\item {Proveniência:(Do lat. \textunderscore calere\textunderscore ?)}
\end{itemize}
Assar ou cozer ligeiramente, para se conservar (carne).
Sujeitar a uma fervura preparatória (os intestinos do boi, para se limparem melhor).
\section{Encalistar}
\begin{itemize}
\item {Grp. gram.:v. t.}
\end{itemize}
\begin{itemize}
\item {Utilização:Fam.}
\end{itemize}
\begin{itemize}
\item {Proveniência:(De \textunderscore calisto\textunderscore )}
\end{itemize}
Causar agoiro a.
Fazer perder no jôgo.
\section{Encalistrar}
\begin{itemize}
\item {Grp. gram.:v. i.}
\end{itemize}
\begin{itemize}
\item {Utilização:Bras. do S}
\end{itemize}
\begin{itemize}
\item {Utilização:fam.}
\end{itemize}
Encavacar.
(Relaciona-se com \textunderscore encallistar\textunderscore ?)
\section{Encallecer}
\begin{itemize}
\item {Grp. gram.:v. i.}
\end{itemize}
Criar callos.
\section{Encallecido}
\begin{itemize}
\item {Grp. gram.:adj.}
\end{itemize}
\begin{itemize}
\item {Proveniência:(De \textunderscore encallecer\textunderscore )}
\end{itemize}
Que tem callos; callejado.
\section{Encallistar}
\begin{itemize}
\item {Grp. gram.:v. t.}
\end{itemize}
\begin{itemize}
\item {Utilização:Fam.}
\end{itemize}
\begin{itemize}
\item {Proveniência:(De \textunderscore callisto\textunderscore )}
\end{itemize}
Causar agoiro a.
Fazer perder no jôgo.
\section{Encalmadiço}
\begin{itemize}
\item {Grp. gram.:adj.}
\end{itemize}
\begin{itemize}
\item {Proveniência:(De \textunderscore encalmar\textunderscore )}
\end{itemize}
Que se encalma facilmente.
\section{Encalmamento}
\begin{itemize}
\item {Grp. gram.:m.}
\end{itemize}
Acto ou effeito de encalmar.
\section{Encalmar}
\begin{itemize}
\item {Grp. gram.:v. t.}
\end{itemize}
\begin{itemize}
\item {Utilização:Fig.}
\end{itemize}
\begin{itemize}
\item {Grp. gram.:V. i.}
\end{itemize}
\begin{itemize}
\item {Proveniência:(De \textunderscore calma\textunderscore )}
\end{itemize}
Causar calor a.
Tornar quente.
Irritar.
Sentir calor.
Acalmar.
\section{Encalombar}
\begin{itemize}
\item {Grp. gram.:v. i.}
\end{itemize}
\begin{itemize}
\item {Utilização:Bras. do N}
\end{itemize}
Criar, ou têr calombos; empolar; encaroçar.
\section{Encalvecer}
\begin{itemize}
\item {Grp. gram.:v. i.}
\end{itemize}
Fazer-se calvo.
\section{Encalvecido}
\begin{itemize}
\item {Grp. gram.:adj.}
\end{itemize}
\begin{itemize}
\item {Proveniência:(De \textunderscore encalvecer\textunderscore )}
\end{itemize}
O mesmo que \textunderscore calvo\textunderscore .
\section{Encalypto}
\begin{itemize}
\item {Grp. gram.:m.}
\end{itemize}
\begin{itemize}
\item {Proveniência:(Do gr. \textunderscore enkalupto\textunderscore )}
\end{itemize}
Gênero de musgos do norte.
\section{Encamar}
\begin{itemize}
\item {Grp. gram.:v. t.}
\end{itemize}
\begin{itemize}
\item {Grp. gram.:V. i.}
\end{itemize}
\begin{itemize}
\item {Utilização:Prov.}
\end{itemize}
\begin{itemize}
\item {Utilização:dur.}
\end{itemize}
O mesmo que \textunderscore acamar\textunderscore .
Cair de cama, com doença prolongada ou incurável.
\section{Encamarado}
\begin{itemize}
\item {Grp. gram.:adj.}
\end{itemize}
\begin{itemize}
\item {Utilização:Ant.}
\end{itemize}
\begin{itemize}
\item {Proveniência:(De \textunderscore câmara\textunderscore )}
\end{itemize}
Que tem a câmara mais estreita do que a alma, (falando-se de canhões).
\section{Encamarotar}
\begin{itemize}
\item {Grp. gram.:v. t.}
\end{itemize}
Meter em camarote. Cf. Castilho, \textunderscore Fastos\textunderscore , II, 514.
\section{Encambar}
\begin{itemize}
\item {Grp. gram.:v. t.}
\end{itemize}
Enfiar num cambo.
Enfiar em cordel ou atilho.
Entrançar, formando réstia.
\section{Encambeirar}
\begin{itemize}
\item {Grp. gram.:v. t.}
\end{itemize}
\begin{itemize}
\item {Utilização:Prov.}
\end{itemize}
Atirar cambeiras a, em folguedos de Carnaval.
Enfarinhar.
\section{Encambulhada}
\begin{itemize}
\item {Grp. gram.:f.}
\end{itemize}
O mesmo que \textunderscore cambulhada\textunderscore .
\section{Encambulhar}
\begin{itemize}
\item {Grp. gram.:v. t.}
\end{itemize}
\begin{itemize}
\item {Utilização:Pop.}
\end{itemize}
Juntar como num cambo.
\section{Encame}
\begin{itemize}
\item {Grp. gram.:m.}
\end{itemize}
\begin{itemize}
\item {Proveniência:(De \textunderscore encamar\textunderscore )}
\end{itemize}
Choça, malhada, de javali.
\section{Encamelar}
\begin{itemize}
\item {Grp. gram.:v. t.}
\end{itemize}
\begin{itemize}
\item {Utilização:Prov.}
\end{itemize}
\begin{itemize}
\item {Utilização:alent.}
\end{itemize}
\begin{itemize}
\item {Proveniência:(De \textunderscore camelo\textunderscore )}
\end{itemize}
Estar zangado, macambúzio.
\section{Encaminhador}
\begin{itemize}
\item {Grp. gram.:adj.}
\end{itemize}
\begin{itemize}
\item {Grp. gram.:m.}
\end{itemize}
Que encaminha.
Aquelle que encaminha.
\section{Encaminhamento}
\begin{itemize}
\item {Grp. gram.:m.}
\end{itemize}
Acto de encaminhar.
\section{Encaminhar}
\begin{itemize}
\item {Grp. gram.:v. t.}
\end{itemize}
Mostrar o caminho a.
Guiar.
Conduzir, dirigir.
Dar bons conselhos a.
Enviar; destinar.
\section{Encaminhe-Deus!}
\begin{itemize}
\item {Grp. gram.:loc. interj.}
\end{itemize}
\begin{itemize}
\item {Utilização:Ant.}
\end{itemize}
O mesmo que \textunderscore destrenga-Deus\textunderscore 
\section{Encamisada}
\begin{itemize}
\item {Grp. gram.:f.}
\end{itemize}
\begin{itemize}
\item {Utilização:Ant.}
\end{itemize}
\begin{itemize}
\item {Proveniência:(De \textunderscore encamisar-se\textunderscore )}
\end{itemize}
Assalto nocturno, em que as tropas vestiam camisões por disfarce.
Embrulhada, difficuldade.
\section{Encamisar-se}
\begin{itemize}
\item {Grp. gram.:v. p.}
\end{itemize}
\begin{itemize}
\item {Proveniência:(De \textunderscore camisa\textunderscore )}
\end{itemize}
Vestir-se para encamisada.
\section{Encamorouçar}
\begin{itemize}
\item {Grp. gram.:v. t.}
\end{itemize}
(V.encomoroiçar)
\section{Encampação}
\begin{itemize}
\item {Grp. gram.:f.}
\end{itemize}
Acto de encampar.
\section{Encampador}
\begin{itemize}
\item {Grp. gram.:m.}
\end{itemize}
Aquelle que encampa.
\section{Encampanado}
\begin{itemize}
\item {Grp. gram.:adj.}
\end{itemize}
\begin{itemize}
\item {Proveniência:(De \textunderscore campana\textunderscore )}
\end{itemize}
Que tem fórma de sino.
\section{Encampanar}
\begin{itemize}
\item {Grp. gram.:v. i.}
\end{itemize}
\begin{itemize}
\item {Proveniência:(Do cast. \textunderscore campana\textunderscore )}
\end{itemize}
Diz-se do toiro, que de repente levanta a cabeça, fitando a vista num objecto.
\section{Encampar}
\begin{itemize}
\item {Grp. gram.:v. t.}
\end{itemize}
\begin{itemize}
\item {Proveniência:(De \textunderscore campo\textunderscore )}
\end{itemize}
Rescindir (contrato de arrendamento), voltando para o dono a coisa arrendada.
Restituir, abandonar, por motívo de lesão de interesses.
\section{Ençampar}
\begin{itemize}
\item {Grp. gram.:v. t.}
\end{itemize}
\begin{itemize}
\item {Utilização:Prov.}
\end{itemize}
\begin{itemize}
\item {Utilização:alg.}
\end{itemize}
O mesmo que \textunderscore enzampar\textunderscore .
\section{Encamurçar}
\begin{itemize}
\item {Grp. gram.:v. t.}
\end{itemize}
\begin{itemize}
\item {Grp. gram.:V. i.}
\end{itemize}
\begin{itemize}
\item {Utilização:Constr.}
\end{itemize}
Revestir de camurça ou feltro (os martelos do piano).
Empenar (a madeira).
\section{Encanamento}
\begin{itemize}
\item {Grp. gram.:m.}
\end{itemize}
Acto ou effeito de encanar.
\section{Encanar}
\begin{itemize}
\item {Grp. gram.:v. t.}
\end{itemize}
\begin{itemize}
\item {Grp. gram.:V. i.}
\end{itemize}
\begin{itemize}
\item {Utilização:Prov.}
\end{itemize}
\begin{itemize}
\item {Utilização:minh.}
\end{itemize}
\begin{itemize}
\item {Utilização:Gír.}
\end{itemize}
\begin{itemize}
\item {Proveniência:(De \textunderscore cano\textunderscore  e \textunderscore cana\textunderscore )}
\end{itemize}
Dirigir por cano ou canal.
Canalizar: \textunderscore encanar água\textunderscore .
Consertar (ossos fracturados).
Criar cana.
Sarar; curar-se.
Mandriar.
\section{Encanas}
\begin{itemize}
\item {Grp. gram.:f. pl.}
\end{itemize}
\begin{itemize}
\item {Proveniência:(Do rad. de \textunderscore cano\textunderscore )}
\end{itemize}
Água, que se junta na drainagem das marinhas-podres.
\section{Encanastrado}
\begin{itemize}
\item {Grp. gram.:m.}
\end{itemize}
\begin{itemize}
\item {Utilização:Prov.}
\end{itemize}
\begin{itemize}
\item {Utilização:minh.}
\end{itemize}
\begin{itemize}
\item {Proveniência:(De \textunderscore encanastrar\textunderscore )}
\end{itemize}
Tecido análogo ao de canastra.
O mesmo que \textunderscore canastro\textunderscore .
\section{Encanastrar}
\begin{itemize}
\item {Grp. gram.:v. t.}
\end{itemize}
\begin{itemize}
\item {Grp. gram.:V. p.}
\end{itemize}
\begin{itemize}
\item {Utilização:Prov.}
\end{itemize}
\begin{itemize}
\item {Utilização:trasm.}
\end{itemize}
Pôr dentro de canastra: \textunderscore encanastrar uvas\textunderscore .
Tecer como o entrançado da canastra.
Entretecer; entrançar.
Embebedar-se.
\section{Encancerar}
\begin{itemize}
\item {Grp. gram.:v. t.  e  p.}
\end{itemize}
(V.cancerar)
\section{Encandear}
\begin{itemize}
\item {Grp. gram.:v. t.}
\end{itemize}
\begin{itemize}
\item {Utilização:Fig.}
\end{itemize}
\begin{itemize}
\item {Proveniência:(De \textunderscore candeia\textunderscore )}
\end{itemize}
Deslumbrar.
Offuscar.
Fascinar, illudindo.
\section{Encandecer}
\begin{itemize}
\item {Grp. gram.:v. t.}
\end{itemize}
(V.incandescer)
\section{Encandilar}
\begin{itemize}
\item {Grp. gram.:v. t.}
\end{itemize}
\begin{itemize}
\item {Proveniência:(De \textunderscore cândi\textunderscore )}
\end{itemize}
Crystallizar.
Tornar cândi, (o açúcar).
\section{Encandolar}
\begin{itemize}
\item {Grp. gram.:v. i.}
\end{itemize}
\begin{itemize}
\item {Utilização:Prov.}
\end{itemize}
\begin{itemize}
\item {Utilização:trasm.}
\end{itemize}
Cobrir-se de neve.
Ficar hirto de frio.
\section{Encanecer}
\begin{itemize}
\item {Grp. gram.:v. t.}
\end{itemize}
\begin{itemize}
\item {Grp. gram.:V. i.}
\end{itemize}
\begin{itemize}
\item {Proveniência:(Do lat. \textunderscore canescere\textunderscore )}
\end{itemize}
Fazer criar cans.
Fazer branco.
Envelhecer.
\section{Encanelar}
\begin{itemize}
\item {Grp. gram.:v. t.}
\end{itemize}
Dobar em canelas.
Fazer canelas em.
Acanelar.
\section{Encangalhado}
\begin{itemize}
\item {Grp. gram.:adj.}
\end{itemize}
\begin{itemize}
\item {Utilização:T. do Fundão}
\end{itemize}
O mesmo que \textunderscore encastalhado\textunderscore .
\section{Encangalhar}
\begin{itemize}
\item {Grp. gram.:v. t.}
\end{itemize}
\begin{itemize}
\item {Utilização:Pop.}
\end{itemize}
\begin{itemize}
\item {Utilização:Bras}
\end{itemize}
\begin{itemize}
\item {Proveniência:(De \textunderscore cangalhas\textunderscore )}
\end{itemize}
Encambulhar.
Pôr cangalhas em.
\section{Encangar}
\begin{itemize}
\item {Grp. gram.:v. t.}
\end{itemize}
\begin{itemize}
\item {Utilização:Pop.}
\end{itemize}
\begin{itemize}
\item {Proveniência:(De \textunderscore canga\textunderscore )}
\end{itemize}
O mesmo que \textunderscore jungir\textunderscore .
Tornar achacado ou curvado por doença.
\section{Encanhadeira}
\begin{itemize}
\item {Grp. gram.:f.}
\end{itemize}
\begin{itemize}
\item {Utilização:Prov.}
\end{itemize}
\begin{itemize}
\item {Utilização:trasm.}
\end{itemize}
Pequena dobadoira para meadas de seda.
\section{Encanhas}
\begin{itemize}
\item {Grp. gram.:f. pl.}
\end{itemize}
\begin{itemize}
\item {Utilização:Gír.}
\end{itemize}
Meias.
\section{Encanhotado}
\begin{itemize}
\item {Grp. gram.:adj.}
\end{itemize}
\begin{itemize}
\item {Proveniência:(De \textunderscore canhoto\textunderscore )}
\end{itemize}
Que tem enguiço. Cf. Camillo, \textunderscore Sc. da Foz\textunderscore , 222.
\section{Encaniçar}
\begin{itemize}
\item {Grp. gram.:v. t.}
\end{itemize}
\begin{itemize}
\item {Proveniência:(De \textunderscore caniço\textunderscore )}
\end{itemize}
Cobrir ou cercar de canas ou de caniçado.
\section{Encanicar-se}
\begin{itemize}
\item {Grp. gram.:v. p.}
\end{itemize}
Tornar-se acanaveado, magro, abatido.
(Cp. \textunderscore acanavear\textunderscore )
\section{Encannelar}
\begin{itemize}
\item {Grp. gram.:v. t.}
\end{itemize}
Dobar em canelas.
Fazer canelas em.
Acanelar.
\section{Encanniçar}
\begin{itemize}
\item {Grp. gram.:v. t.}
\end{itemize}
\begin{itemize}
\item {Proveniência:(De \textunderscore caniço\textunderscore )}
\end{itemize}
Cobrir ou cercar de canas ou de caniçado.
\section{Encanoar}
\begin{itemize}
\item {Grp. gram.:v. i.}
\end{itemize}
\begin{itemize}
\item {Utilização:Bras. do Rio}
\end{itemize}
\begin{itemize}
\item {Proveniência:(De \textunderscore canôa\textunderscore )}
\end{itemize}
Empenar-se (uma tábua) transversalmente, imitando a fórma de uma canôa.
\section{Encantadiço}
\begin{itemize}
\item {Grp. gram.:adj.}
\end{itemize}
Que encanta facilmente. Cf. Garrett, \textunderscore D. Branca\textunderscore , 75.
\section{Encantador}
\begin{itemize}
\item {Grp. gram.:adj.}
\end{itemize}
\begin{itemize}
\item {Grp. gram.:M.}
\end{itemize}
Que encanta, que seduz.
Bellíssimo; esplêndido: \textunderscore um dia encantador\textunderscore .
Aquelle que encanta ou seduz.
\section{Encantamento}
\begin{itemize}
\item {Grp. gram.:m.}
\end{itemize}
\begin{itemize}
\item {Utilização:Fig.}
\end{itemize}
\begin{itemize}
\item {Proveniência:(Lat. \textunderscore incantamentum\textunderscore )}
\end{itemize}
Bruxedo, feitiçaria, magia.
Influência maravilhosa ou sobrenatural de feitiços, bruxas, etc.
Encanto.
Tentação.
Maravilha: \textunderscore aquella festa foi um encantamento\textunderscore .
\section{Encantar}
\begin{itemize}
\item {Grp. gram.:v. t.}
\end{itemize}
\begin{itemize}
\item {Utilização:Fig.}
\end{itemize}
\begin{itemize}
\item {Proveniência:(Lat. \textunderscore incantare\textunderscore )}
\end{itemize}
Exercer encantamento em.
Attrahir; seduzir.
Maravilhar.
Enlevar: \textunderscore encanta-me a tua voz\textunderscore .
\section{Encanteirar}
\begin{itemize}
\item {Grp. gram.:v. t.}
\end{itemize}
Pôr em canteiros.
Dividir em canteiros.
Plantar em canteiros.
\section{Encantinar}
\begin{itemize}
\item {Grp. gram.:v. t.}
\end{itemize}
\begin{itemize}
\item {Proveniência:(De \textunderscore cantina\textunderscore )}
\end{itemize}
(V.enventanar)
\section{Encantinhar}
\begin{itemize}
\item {Grp. gram.:v. t.}
\end{itemize}
\begin{itemize}
\item {Utilização:Prov.}
\end{itemize}
\begin{itemize}
\item {Utilização:beir.}
\end{itemize}
\begin{itemize}
\item {Proveniência:(De \textunderscore cantinho\textunderscore , dem. de \textunderscore canto\textunderscore ^1)}
\end{itemize}
Pôr num canto, acantoar.
\section{Encanto}
\begin{itemize}
\item {Grp. gram.:m.}
\end{itemize}
Acto ou effeito de encantar.
Coisa que encanta ou agrada extremamente: \textunderscore êste drama é um encanto\textunderscore .
Pessôa, que encanta.
Qualidade da pessôa ou coisa, que agrada em extremo.
\section{Encantoar}
\begin{itemize}
\item {Grp. gram.:v. t.}
\end{itemize}
Pôr num canto.
Retirar; desviar do trato: \textunderscore o Álvaro encantoou-se numa aldeôla da Beira\textunderscore .
\section{Encantonar}
\begin{itemize}
\item {Grp. gram.:v. t.}
\end{itemize}
O mesmo que \textunderscore encantoar\textunderscore .
\section{Encanudado}
\begin{itemize}
\item {Grp. gram.:adj.}
\end{itemize}
O mesmo que \textunderscore encastellado\textunderscore , (falando-se do cavallo).
Que tem fórma de canudo.
Engomado, formando caneluras: \textunderscore peitilho encanudado\textunderscore .
\section{Encanudamento}
\begin{itemize}
\item {Grp. gram.:m.}
\end{itemize}
Acto ou effeito de encanudar.
\section{Encanudar}
\begin{itemize}
\item {Grp. gram.:v. t.}
\end{itemize}
\begin{itemize}
\item {Proveniência:(De \textunderscore canudo\textunderscore )}
\end{itemize}
Dar a fórma de canudo ou de canela a.
Encaracolar: \textunderscore encanudar o cabello\textunderscore .
Meter em canudo.
Enfiar.
\section{Encanutado}
\begin{itemize}
\item {Grp. gram.:adj.}
\end{itemize}
(V.encanudado)
\section{Encanzinadamente}
\begin{itemize}
\item {Grp. gram.:adv.}
\end{itemize}
\begin{itemize}
\item {Proveniência:(De \textunderscore encanzinar\textunderscore )}
\end{itemize}
Com obstinação.
\section{Encanzinar-se}
\begin{itemize}
\item {Grp. gram.:v. p.}
\end{itemize}
\begin{itemize}
\item {Utilização:Fam.}
\end{itemize}
\begin{itemize}
\item {Proveniência:(Do rad. de \textunderscore canzoada\textunderscore )}
\end{itemize}
Emperrar.
Obstinar-se; enraivar-se.
\section{Encanzoar-se}
\begin{itemize}
\item {Grp. gram.:v. p.}
\end{itemize}
O mesmo que \textunderscore encanzinar-se\textunderscore .
\section{Encapachar}
\begin{itemize}
\item {Grp. gram.:v. t.}
\end{itemize}
\begin{itemize}
\item {Grp. gram.:V. p.}
\end{itemize}
\begin{itemize}
\item {Utilização:Fig.}
\end{itemize}
Meter em capacho.
Humilhar-se servilmente.
Submeter-se com ignomínia.
\section{Encapar}
\begin{itemize}
\item {Grp. gram.:v. t.}
\end{itemize}
Cobrir com capa.
Embrulhar.
Revestir.
\section{Encapeladura}
\begin{itemize}
\item {Grp. gram.:f.}
\end{itemize}
\begin{itemize}
\item {Utilização:Mús.}
\end{itemize}
\begin{itemize}
\item {Grp. gram.:Pl.}
\end{itemize}
Acto de encapelar.
Revestimento dos martelos de piano, com pele de anta, á qual se sobrepõe o feltro, que é segunda capa dos martelos.
Lugar, em que se encapelam as enxárcias.
\section{Encapelar}
\begin{itemize}
\item {Grp. gram.:v. t.}
\end{itemize}
\begin{itemize}
\item {Grp. gram.:V. i.  e  p.}
\end{itemize}
\begin{itemize}
\item {Utilização:Náut.}
\end{itemize}
\begin{itemize}
\item {Proveniência:(De \textunderscore capela\textunderscore )}
\end{itemize}
Encrespar, erguer, (o mar, as ondas): \textunderscore o vento encapela as ondas\textunderscore .
Dar o encargo de capela a.
Entumecer-se, formar ondas.
Introduzir no calcês ou lais a enxárcia, a alça, etc.
\section{Encapelladura}
\begin{itemize}
\item {Grp. gram.:f.}
\end{itemize}
\begin{itemize}
\item {Utilização:Mús.}
\end{itemize}
\begin{itemize}
\item {Grp. gram.:Pl.}
\end{itemize}
Acto de encapellar.
Revestimento dos martelos de piano, com pelle de anta, á qual se sobrepõe o feltro, que é segunda capa dos martelos.
Lugar, em que se encapellam as enxárcias.
\section{Encapellar}
\begin{itemize}
\item {Grp. gram.:v. t.}
\end{itemize}
\begin{itemize}
\item {Grp. gram.:V. i.  e  p.}
\end{itemize}
\begin{itemize}
\item {Utilização:Náut.}
\end{itemize}
\begin{itemize}
\item {Proveniência:(De \textunderscore capella\textunderscore )}
\end{itemize}
Encrespar, erguer, (o mar, as ondas): \textunderscore o vento encapella as ondas\textunderscore .
Dar o encargo de capella a.
Entumecer-se, formar ondas.
Introduzir no calcês ou lais a enxárcia, a alça, etc.
\section{Encapetado}
\begin{itemize}
\item {Grp. gram.:adj.}
\end{itemize}
\begin{itemize}
\item {Utilização:Bras. do N}
\end{itemize}
\begin{itemize}
\item {Proveniência:(De \textunderscore encapetar-se\textunderscore )}
\end{itemize}
Traquinas; endiabrado.
\section{Encapetar-se}
\begin{itemize}
\item {Grp. gram.:v. p.}
\end{itemize}
\begin{itemize}
\item {Utilização:Bras. do N}
\end{itemize}
\begin{itemize}
\item {Proveniência:(De \textunderscore capeta\textunderscore )}
\end{itemize}
Tornar-se traquinas, endiabrado.
\section{Encapoeirar}
\begin{itemize}
\item {Grp. gram.:v. t.}
\end{itemize}
Pôr em capoeira.
\section{Encapotar}
\begin{itemize}
\item {Grp. gram.:v. t.}
\end{itemize}
\begin{itemize}
\item {Grp. gram.:V. p.  e  i.}
\end{itemize}
\begin{itemize}
\item {Proveniência:(De \textunderscore capote\textunderscore )}
\end{itemize}
Encapar.
Esconder.
* Disfarçar.
Anuvear-se: \textunderscore o céu encapotou-se\textunderscore .
\section{Encaprichar-se}
\begin{itemize}
\item {Grp. gram.:v. p.}
\end{itemize}
Encher-se de brio, do capricho.
\section{Encapuchar}
\begin{itemize}
\item {Grp. gram.:v. t.}
\end{itemize}
Vestir com capucha.
Pôr capucha em. Cf. Camillo, \textunderscore Brasileira\textunderscore , 339.
\section{Encapuzar}
\begin{itemize}
\item {Grp. gram.:v. t.}
\end{itemize}
Cobrir com capuz.
\section{Encaração}
\begin{itemize}
\item {Grp. gram.:f.}
\end{itemize}
\begin{itemize}
\item {Utilização:bras}
\end{itemize}
\begin{itemize}
\item {Utilização:Neol.}
\end{itemize}
Acto de encarar. Cf. \textunderscore Jornal do Brasil\textunderscore , de 28-VIII-905.
\section{Encaracolar}
\begin{itemize}
\item {Grp. gram.:v. t.}
\end{itemize}
\begin{itemize}
\item {Grp. gram.:V. i.  e  p.}
\end{itemize}
Dar fórma de caracol, de rosca ou de espiral a: \textunderscore encaracolar o cabello\textunderscore .
Enroscar-se ou enrolar-se em espiral.
\section{Encaramanchar}
\begin{itemize}
\item {Grp. gram.:v. t.}
\end{itemize}
Converter em caramanchão. Cf. Filinto, XXI, 12.
\section{Encaramelar}
\begin{itemize}
\item {Grp. gram.:v. t.}
\end{itemize}
\begin{itemize}
\item {Grp. gram.:V. i.  e  p.}
\end{itemize}
Tornar gelado, como caramelo.
Coalhar.
Coalhar-se; tornar-se caramelo.
\section{Encaramonar}
\begin{itemize}
\item {Grp. gram.:v. t.}
\end{itemize}
\begin{itemize}
\item {Utilização:Pop.}
\end{itemize}
\begin{itemize}
\item {Proveniência:(De \textunderscore cara\textunderscore  + \textunderscore mono\textunderscore )}
\end{itemize}
Tornar tristonho.
\section{Encarangado}
\begin{itemize}
\item {Grp. gram.:adj.}
\end{itemize}
\begin{itemize}
\item {Proveniência:(De \textunderscore encarangar\textunderscore )}
\end{itemize}
Adoentado; achacadiço.
\section{Encarangar}
\begin{itemize}
\item {Grp. gram.:v. t.}
\end{itemize}
\begin{itemize}
\item {Utilização:Pop.}
\end{itemize}
\begin{itemize}
\item {Grp. gram.:V. p.}
\end{itemize}
Tolher com frio ou rheumatismo.
Encangar.
Tornar adoentado.
Tornar-se adoentado, achacadiço.
(Metáth. de \textunderscore encangarar\textunderscore , por \textunderscore encangueirar\textunderscore , de \textunderscore cangueira\textunderscore )
\section{Encarantar-se}
\begin{itemize}
\item {Grp. gram.:v. p.}
\end{itemize}
\begin{itemize}
\item {Utilização:Prov.}
\end{itemize}
\begin{itemize}
\item {Utilização:trasm.}
\end{itemize}
Acamaradar-se, arranchar.
Dar-se bem uma pessôa com outra.
(Relaciona-se com \textunderscore cara\textunderscore ?)
\section{Encarantonhado}
\begin{itemize}
\item {Grp. gram.:adj.}
\end{itemize}
\begin{itemize}
\item {Utilização:Prov.}
\end{itemize}
\begin{itemize}
\item {Utilização:alg.}
\end{itemize}
\begin{itemize}
\item {Proveniência:(De \textunderscore carantonha\textunderscore )}
\end{itemize}
Mal encarado.
\section{Encarapelar}
\begin{itemize}
\item {Grp. gram.:v. t., i.  e  p.}
\end{itemize}
\begin{itemize}
\item {Proveniência:(De \textunderscore carapela\textunderscore )}
\end{itemize}
(V.encapellar)
\section{Encarapinhar}
\begin{itemize}
\item {Grp. gram.:v. t.}
\end{itemize}
\begin{itemize}
\item {Utilização:Fig.}
\end{itemize}
Fazer carapinha em.
Encrespar (o cabello).
Encaramelar, congelar.
\section{Encarapitar}
\begin{itemize}
\item {Grp. gram.:v. t.}
\end{itemize}
(V.encarrapitar)
\section{Encarapuçar}
\begin{itemize}
\item {Grp. gram.:v. t.}
\end{itemize}
Pôr carapuça em.
\section{Encarar}
\begin{itemize}
\item {Grp. gram.:v. t.}
\end{itemize}
\begin{itemize}
\item {Utilização:Fig.}
\end{itemize}
\begin{itemize}
\item {Grp. gram.:V. i.}
\end{itemize}
Olhar de frente, de cara.
Fixar a vista em.
Arrostar.
Observar, considerar: \textunderscore encarar o futuro\textunderscore .
Fitar os olhos, olhar direito:«\textunderscore encarando no aspecto do veterano...\textunderscore »Camillo, \textunderscore Retr. de Ricard.\textunderscore , 277.«\textunderscore O pai encarou nelle.\textunderscore »Camillo, \textunderscore Estrêl. Fun.\textunderscore , pról.
\section{Encarcadar}
\begin{itemize}
\item {Grp. gram.:v. t.}
\end{itemize}
\begin{itemize}
\item {Utilização:Prov.}
\end{itemize}
\begin{itemize}
\item {Utilização:alg.}
\end{itemize}
\begin{itemize}
\item {Proveniência:(Do rad. \textunderscore carc\textunderscore , commum a muitos termos portugueses)}
\end{itemize}
Introduzir.
\section{Encarcerado}
\begin{itemize}
\item {Grp. gram.:m.}
\end{itemize}
\begin{itemize}
\item {Proveniência:(De \textunderscore encarcerar\textunderscore )}
\end{itemize}
Aquelle que está preso, metido no cárcere.
\section{Encarceramento}
\begin{itemize}
\item {Grp. gram.:m.}
\end{itemize}
Acto ou effeito de encarcerar.
\section{Encarcerar}
\begin{itemize}
\item {Grp. gram.:v. t.}
\end{itemize}
Encerrar em cárcere, prender em cárcere.
Enclausurar.
(B. lat. \textunderscore incarcerare\textunderscore )
\section{Encarchar}
\begin{itemize}
\item {Grp. gram.:v. t.}
\end{itemize}
\begin{itemize}
\item {Utilização:Prov.}
\end{itemize}
Enfeitiçar.
\section{Encardimento}
\begin{itemize}
\item {Grp. gram.:m.}
\end{itemize}
\begin{itemize}
\item {Proveniência:(De \textunderscore encardir\textunderscore )}
\end{itemize}
Estado de encardido.
Incrustação de immundície.
\section{Encardir}
\begin{itemize}
\item {Grp. gram.:v. t.}
\end{itemize}
\begin{itemize}
\item {Grp. gram.:V. i.  e  p.}
\end{itemize}
\begin{itemize}
\item {Proveniência:(De um hyp. \textunderscore cardir\textunderscore , talvez affim de \textunderscore cárdeo\textunderscore )}
\end{itemize}
Tornar enxovalhado, pouco limpo.
Lavar incompletamente.
Ficar mal lavado, conservando parte da sujidade.
\section{Encarecedor}
\begin{itemize}
\item {Grp. gram.:adj.}
\end{itemize}
\begin{itemize}
\item {Grp. gram.:M.}
\end{itemize}
Que encarece.
Aquelle que encarece.
\section{Encarecer}
\begin{itemize}
\item {Grp. gram.:v. t.}
\end{itemize}
\begin{itemize}
\item {Utilização:Fig.}
\end{itemize}
\begin{itemize}
\item {Grp. gram.:V. i}
\end{itemize}
Tornar caro: \textunderscore a falta de chuvas encareceu o pão\textunderscore .
Exaltar, louvar excessivamente.
Exaggerar verbalmente.
Tornar-se caro; subir de preço: \textunderscore o azeite encareceu\textunderscore .
\section{Encarecidamente}
\begin{itemize}
\item {Grp. gram.:adv.}
\end{itemize}
\begin{itemize}
\item {Proveniência:(De \textunderscore encarecido\textunderscore )}
\end{itemize}
Com encarecimento, com empenho: \textunderscore pedir encarecidamente\textunderscore .
\section{Encarecimento}
\begin{itemize}
\item {Grp. gram.:m.}
\end{itemize}
Acto ou effeito de encarecer.
Empenho.
Interesse moral.
\section{Encarentar}
\begin{itemize}
\item {Grp. gram.:v. t.}
\end{itemize}
(V.encarecer)
\section{Encaretar-se}
\begin{itemize}
\item {Grp. gram.:v. p.}
\end{itemize}
\begin{itemize}
\item {Proveniência:(De \textunderscore careta\textunderscore )}
\end{itemize}
Mascarar-se.
\section{Encargar}
\begin{itemize}
\item {Grp. gram.:v. t.}
\end{itemize}
\begin{itemize}
\item {Utilização:Bras}
\end{itemize}
\begin{itemize}
\item {Proveniência:(De \textunderscore cargo\textunderscore  e \textunderscore carga\textunderscore )}
\end{itemize}
Encarregar.
Encher.
Acommodar objectos de transporte dentro de. Cf. Filinto, VII, 199.
\section{Encargo}
\begin{itemize}
\item {Grp. gram.:m.}
\end{itemize}
Acto ou effeito de encarregar.
Incumbência.
Obrigação.
Gravame.
Cargo.
Pensão; legado: \textunderscore recebeu a herança com vários encargos\textunderscore .
Remorso.
(Cp. \textunderscore encargar\textunderscore )
\section{Encarna}
\begin{itemize}
\item {Grp. gram.:f.}
\end{itemize}
\begin{itemize}
\item {Proveniência:(De \textunderscore encarnar\textunderscore ^2)}
\end{itemize}
Encaixe.
Engaste, em obras de ourives.
Pequena escavação:«\textunderscore as bexigas...; os signaes... faziam saliências e encarnas\textunderscore ». M. Assis, \textunderscore B. Cubas\textunderscore .
\section{Encarnação}
\begin{itemize}
\item {Grp. gram.:f.}
\end{itemize}
Acto de \textunderscore encarnar\textunderscore ^1.
Reparação, com que se imita a côr da carne em imagens e estátuas.
Preparação especial, para collar loiça partida.
\section{Encarnação}
\begin{itemize}
\item {Grp. gram.:f.}
\end{itemize}
\begin{itemize}
\item {Proveniência:(De \textunderscore encarnar\textunderscore ^2)}
\end{itemize}
Dogma christão de que o Filho de Deus encarnou ou se fez homem.
\section{Encarnado}
\begin{itemize}
\item {Grp. gram.:adj.}
\end{itemize}
\begin{itemize}
\item {Grp. gram.:M.}
\end{itemize}
\begin{itemize}
\item {Proveniência:(De \textunderscore encarnar\textunderscore ^1)}
\end{itemize}
Que tem côr de carne.
Vermelho; escarlate.
A côr encarnada.
A encarnação de estátuas, imagens, etc.
\section{Encarnador}
\begin{itemize}
\item {Grp. gram.:m.}
\end{itemize}
\begin{itemize}
\item {Proveniência:(De \textunderscore encarnar\textunderscore ^1)}
\end{itemize}
Aquelle que dá côr de carne a imagens, estátuas, etc.
\section{Encarnar}
\begin{itemize}
\item {Grp. gram.:v. t.}
\end{itemize}
\begin{itemize}
\item {Proveniência:(De carne)}
\end{itemize}
Dar côr de carne a (imagens, estátuas ou outros objectos).
Alimentar com carne (animaes para caça)
\section{Encarnar}
\begin{itemize}
\item {Grp. gram.:v. i.}
\end{itemize}
\begin{itemize}
\item {Utilização:Fig.}
\end{itemize}
\begin{itemize}
\item {Grp. gram.:V. p.}
\end{itemize}
\begin{itemize}
\item {Proveniência:(Lat. \textunderscore incarnare\textunderscore )}
\end{itemize}
Tomar carne humana, tornar-se homem, humanar-se: \textunderscore O Filho de Deus encarnou e fez-se homem\textunderscore .
Criar carne, cicatrizar, (falando-se de ferimentos).
Tomar vulto, fórma.
Penetrar em algum corpo.
(Os mesmos significados).
\section{Encarne}
\begin{itemize}
\item {Grp. gram.:m.}
\end{itemize}
O mesmo que \textunderscore encarna\textunderscore .
\section{Encarneirar}
\begin{itemize}
\item {Grp. gram.:v. i.  e  p.}
\end{itemize}
\begin{itemize}
\item {Proveniência:(De \textunderscore carneiro\textunderscore )}
\end{itemize}
Encrespar-se (o mar) e erguer pequenas e numerosas ondas espumosas, trazendo á ideia um rebanho de carneiros brancos em movimento.
\section{Encarniçadamente}
\begin{itemize}
\item {Grp. gram.:adv.}
\end{itemize}
\begin{itemize}
\item {Proveniência:(De \textunderscore encarniçar\textunderscore )}
\end{itemize}
Com encarniçamento.
Encanzinadamente.
\section{Encarniçado}
\begin{itemize}
\item {Grp. gram.:adj.}
\end{itemize}
\begin{itemize}
\item {Proveniência:(De \textunderscore encarniçar\textunderscore )}
\end{itemize}
Assanhado, furioso.
Avermelhado, injectado:«\textunderscore olhos encarniçados\textunderscore ». Camillo, \textunderscore Corja\textunderscore , 10.
\section{Encarniçamento}
\begin{itemize}
\item {Grp. gram.:m.}
\end{itemize}
Acto ou effeito de encarniçar.
\section{Encarniçar}
\begin{itemize}
\item {Grp. gram.:v. t.}
\end{itemize}
\begin{itemize}
\item {Grp. gram.:V. p.}
\end{itemize}
\begin{itemize}
\item {Proveniência:(De \textunderscore carniça\textunderscore )}
\end{itemize}
Deitar carniça a (os cães).
Provocar para a crueldade.
Tornar bravo.
Açular.
Tornar furioso, implacável, persistente em fazer mal.
Assanhar-se ou enraivecer-se contra alguém ou contra alguma coisa.
Fazer carnagem ou carnificina.
Ter persistência ou contumácia em fazer mal.
\section{Encaroçado}
\begin{itemize}
\item {Grp. gram.:adj.}
\end{itemize}
\begin{itemize}
\item {Proveniência:(De \textunderscore encaroçar\textunderscore )}
\end{itemize}
Que tem fórma de caroço.
Entumecido, inchado.
\section{Encaroçar}
\begin{itemize}
\item {Grp. gram.:v. i.}
\end{itemize}
\begin{itemize}
\item {Utilização:Fam.}
\end{itemize}
\begin{itemize}
\item {Proveniência:(De \textunderscore caroço\textunderscore )}
\end{itemize}
Diz-se do seio da mulher que amamenta, quando nelle se formam inchaços ou tumores, produzidos pela difficuldade ou escassez com que o leite é extrahido.
\section{Encarochar}
\begin{itemize}
\item {Grp. gram.:v. t.}
\end{itemize}
\begin{itemize}
\item {Utilização:Prov.}
\end{itemize}
\begin{itemize}
\item {Utilização:minh.}
\end{itemize}
\begin{itemize}
\item {Proveniência:(De \textunderscore carocho\textunderscore )}
\end{itemize}
O mesmo que \textunderscore embruxar\textunderscore .
Dispor (o centeio) em carochos ou medas.
\section{Encarochar}
\begin{itemize}
\item {Grp. gram.:v. t.}
\end{itemize}
\begin{itemize}
\item {Utilização:Ant.}
\end{itemize}
\begin{itemize}
\item {Proveniência:(De \textunderscore carocha\textunderscore )}
\end{itemize}
Pôr mitra de papel em (condemnados da inquisição).
\section{Encaroladeira}
\begin{itemize}
\item {Grp. gram.:f.}
\end{itemize}
Um dos apparelhos das fábricas de tecidos. Cf. \textunderscore Inquér. Industr.\textunderscore , l. III, 49.
\section{Encarpo}
\begin{itemize}
\item {Grp. gram.:m.}
\end{itemize}
\begin{itemize}
\item {Proveniência:(Do gr. \textunderscore en\textunderscore  + \textunderscore karpos\textunderscore )}
\end{itemize}
Grinalda architectónica, que contém fôlhas, flôres e frutos.
\section{Encarquilhado}
\begin{itemize}
\item {Grp. gram.:adj.}
\end{itemize}
Que tem pregas.
Rugoso, enrugado: \textunderscore cara encarquilhada\textunderscore .
Resequido: \textunderscore fôlhas encarquilhadas\textunderscore .
\section{Encarquilhar}
\begin{itemize}
\item {Grp. gram.:v. t.}
\end{itemize}
Fazer carquilhas em.
Tornar rugoso ou enrugado.
\section{Encarradoiro}
\begin{itemize}
\item {Grp. gram.:m.}
\end{itemize}
\begin{itemize}
\item {Utilização:Agr.}
\end{itemize}
\begin{itemize}
\item {Proveniência:(De \textunderscore encarrar\textunderscore )}
\end{itemize}
Lugar, onde se enche o carro.
\section{Encarradouro}
\begin{itemize}
\item {Grp. gram.:m.}
\end{itemize}
\begin{itemize}
\item {Utilização:Agr.}
\end{itemize}
\begin{itemize}
\item {Proveniência:(De \textunderscore encarrar\textunderscore )}
\end{itemize}
Lugar, onde se enche o carro.
\section{Encarrancar}
\begin{itemize}
\item {Grp. gram.:v. t.}
\end{itemize}
\begin{itemize}
\item {Grp. gram.:V. i.}
\end{itemize}
Tornar carrancudo, torvo, anuveado.
Fazer carranca; anuvear-se: \textunderscore o dia encarrancou\textunderscore .
\section{Encarrapichar-se}
\begin{itemize}
\item {Grp. gram.:v. p.}
\end{itemize}
\begin{itemize}
\item {Utilização:Bras}
\end{itemize}
\begin{itemize}
\item {Utilização:Fam.}
\end{itemize}
Fazer carrapichos ou bucres.
Encher-se de capricho.
\section{Encarrapitar}
\begin{itemize}
\item {Grp. gram.:v. t.}
\end{itemize}
\begin{itemize}
\item {Utilização:Fam.}
\end{itemize}
\begin{itemize}
\item {Grp. gram.:V. p.}
\end{itemize}
Pôr no carrapito, no alto.
Empoleirar.
Pôr em sítio elevado.
Fazer caracóes ou carrapitos em (o cabello).
Pôr-se em sítio elevado; alcandorar-se.
\section{Encarrar}
\begin{itemize}
\item {Grp. gram.:v. t.}
\end{itemize}
\begin{itemize}
\item {Proveniência:(De \textunderscore carro\textunderscore )}
\end{itemize}
Pôr no carro.
Carregar o carro com: \textunderscore foram lá três homens encarrar a pedra para a obra\textunderscore .
\section{Encarrascar}
\begin{itemize}
\item {Grp. gram.:v. t.}
\end{itemize}
\begin{itemize}
\item {Grp. gram.:V. p.}
\end{itemize}
\begin{itemize}
\item {Utilização:Chul.}
\end{itemize}
\begin{itemize}
\item {Proveniência:(De \textunderscore carrascão\textunderscore )}
\end{itemize}
Tornar carrascão (o vinho). Cf. \textunderscore Techn. Rur.\textunderscore 
Embebedar-se com vinho carrascão; embebedar-se.
\section{Encarraspanar-se}
\begin{itemize}
\item {Grp. gram.:v. p.}
\end{itemize}
\begin{itemize}
\item {Utilização:Fam.}
\end{itemize}
\begin{itemize}
\item {Proveniência:(De \textunderscore carraspana\textunderscore )}
\end{itemize}
Embebedar-se.
\section{Encarregado}
\begin{itemize}
\item {Grp. gram.:m.}
\end{itemize}
\begin{itemize}
\item {Proveniência:(De \textunderscore encarregar\textunderscore )}
\end{itemize}
Aquelle, que está incumbido de um serviço ou negócio.
\section{Encarregar}
\begin{itemize}
\item {Grp. gram.:v. t.}
\end{itemize}
\begin{itemize}
\item {Utilização:Ant.}
\end{itemize}
\begin{itemize}
\item {Proveniência:(De \textunderscore carregar\textunderscore )}
\end{itemize}
Incumbir.
Pôr a cargo ou na responsabilidade de.
Dar commissão a.
Recommendar, encommendar com empenho.
Opprimir.
\section{Encarrêgo}
\begin{itemize}
\item {Grp. gram.:m.}
\end{itemize}
\begin{itemize}
\item {Utilização:Pop.}
\end{itemize}
Encargo.
Acto de encarregar.
Gravame de consciência.
Occupação, tarefa.
\section{Encarregue}
\begin{itemize}
\item {Grp. gram.:adj.}
\end{itemize}
\begin{itemize}
\item {Utilização:P. us.}
\end{itemize}
\begin{itemize}
\item {Proveniência:(De \textunderscore encarregar\textunderscore )}
\end{itemize}
Que recebeu um encargo.
\section{Encarreiramento}
\begin{itemize}
\item {Grp. gram.:m.}
\end{itemize}
Acto de encarreirar.
\section{Encarreirar}
\begin{itemize}
\item {Grp. gram.:v. t.}
\end{itemize}
\begin{itemize}
\item {Proveniência:(De \textunderscore carreira\textunderscore )}
\end{itemize}
Encaminhar, dirigir, abrir caminho a.
\section{Encarretadeira}
\begin{itemize}
\item {Grp. gram.:f.}
\end{itemize}
Um dos maquinismos das fábricas de fiação, própria para encarretar. Cf. \textunderscore Inquér. Industr.\textunderscore , p. II, l. III, 215.
\section{Encarretar}
\begin{itemize}
\item {Grp. gram.:v. t.}
\end{itemize}
\begin{itemize}
\item {Proveniência:(De \textunderscore carretar\textunderscore )}
\end{itemize}
Pôr em carrêta.
Encarrar.
\section{Encarriçada}
\begin{itemize}
\item {Grp. gram.:f. adj.}
\end{itemize}
\begin{itemize}
\item {Utilização:Prov.}
\end{itemize}
\begin{itemize}
\item {Utilização:beir.}
\end{itemize}
\begin{itemize}
\item {Proveniência:(De \textunderscore acarrado\textunderscore ?)}
\end{itemize}
Diz-se da gallinha, toda occupada em chocar os ovos.
\section{Encarrilar}
\begin{itemize}
\item {Grp. gram.:v. t.}
\end{itemize}
\begin{itemize}
\item {Utilização:Fig.}
\end{itemize}
\begin{itemize}
\item {Grp. gram.:V. i.}
\end{itemize}
\begin{itemize}
\item {Proveniência:(De \textunderscore carril\textunderscore )}
\end{itemize}
Pôr nos carris, nas calhas: \textunderscore encarrilar um vagão\textunderscore .
Encarreirar, meter em bom caminho.
Seguir caminho direito; acertar.
\section{Encarrilhado}
\begin{itemize}
\item {Grp. gram.:adj.}
\end{itemize}
\begin{itemize}
\item {Utilização:Prov.}
\end{itemize}
\begin{itemize}
\item {Utilização:minh.}
\end{itemize}
\begin{itemize}
\item {Proveniência:(De \textunderscore encarrilhar\textunderscore )}
\end{itemize}
Que anda á mercê de alguma coisa: \textunderscore barco encarrilhado nas ondas\textunderscore .
\section{Encarrilhar}
\begin{itemize}
\item {Grp. gram.:v. t.}
\end{itemize}
O mesmo que \textunderscore encarrilar\textunderscore .
\section{Encartação}
\begin{itemize}
\item {Grp. gram.:f.}
\end{itemize}
Acto de encartar.
\section{Encartadeira}
\begin{itemize}
\item {Grp. gram.:f.}
\end{itemize}
Apparelho das officinas de retorce, nas fábricas de fiação, no qual entra a urdidura, para se juntar a dois fios e entrar no torcedores.
(Relaciona-se com \textunderscore encartar\textunderscore )
\section{Encartado}
\begin{itemize}
\item {Grp. gram.:adj.}
\end{itemize}
\begin{itemize}
\item {Utilização:T. de jôgo}
\end{itemize}
Que já tem nas mãos as cartas precisas para a casca.
Que se encartou ou que tem diploma do seu emprêgo.
\section{Encartalhar}
\begin{itemize}
\item {Grp. gram.:v. t.}
\end{itemize}
Juntar (peças de madeira), nos trabalhos de carpinteiro.--Registo o vocábulo, por o vêr nos diccionários, mas creio provir de êrro typográphico, por \textunderscore encastalhar\textunderscore .
\section{Encartamento}
\begin{itemize}
\item {Grp. gram.:m.}
\end{itemize}
O mesmo que \textunderscore encartação\textunderscore .
\section{Encartar}
\begin{itemize}
\item {Grp. gram.:v. t.}
\end{itemize}
\begin{itemize}
\item {Utilização:Ant.}
\end{itemize}
\begin{itemize}
\item {Grp. gram.:V. i.}
\end{itemize}
\begin{itemize}
\item {Grp. gram.:V. p.}
\end{itemize}
\begin{itemize}
\item {Utilização:T. do jôgo}
\end{itemize}
Dar diploma de emprêgo a.
Proscrever, degredar.
Jogar, sobre uma carta, outra do mesmo naipe.
Tirar diploma do seu emprêgo, pagando os respectivos direitos.
Tomar as cartas precisas, na \textunderscore casca\textunderscore  do voltarete.
(B. lat. \textunderscore incartare\textunderscore )
\section{Encarte}
\begin{itemize}
\item {Grp. gram.:m.}
\end{itemize}
Acto de encartar ou de se encartar: \textunderscore a lei exige o encarte dos empregados públicos\textunderscore .
\section{Encartolar-se}
\begin{itemize}
\item {Grp. gram.:v. p.}
\end{itemize}
Pôr cartola ou chapéu alto. Cf. Camillo, \textunderscore Serões\textunderscore , V, 66.
\section{Encartuchar}
\begin{itemize}
\item {Grp. gram.:v. t.}
\end{itemize}
Meter em cartucho.
Dar fórma de cartucho a.
\section{Encarva}
\begin{itemize}
\item {Grp. gram.:f.}
\end{itemize}
\begin{itemize}
\item {Utilização:Ant.}
\end{itemize}
Favor ou obséquio, para alliciar ou seduzir.
\section{Encarvoador}
\begin{itemize}
\item {Grp. gram.:adj.}
\end{itemize}
Que encarvôa, que suja com carvão; que mascarra.
\section{Encarvoar}
\begin{itemize}
\item {Grp. gram.:v. t.}
\end{itemize}
Converter em carvão.
Mascarrar.
Sujar com carvão.
\section{Encarvoejar}
\begin{itemize}
\item {Grp. gram.:v. t.}
\end{itemize}
\begin{itemize}
\item {Proveniência:(De \textunderscore carvão\textunderscore )}
\end{itemize}
Encarvoar, escurecer, denegrir.
\section{Encarvoiçar}
\begin{itemize}
\item {Grp. gram.:v. t.}
\end{itemize}
\begin{itemize}
\item {Utilização:Des.}
\end{itemize}
O mesmo que \textunderscore encarvoejar\textunderscore .
\section{Encasacar-se}
\begin{itemize}
\item {Grp. gram.:v. t.}
\end{itemize}
Vestir casaca.
Pôr traje ceremonioso.
\section{Encasamento}
\begin{itemize}
\item {Grp. gram.:m.}
\end{itemize}
Entalhe, encaixe.
Acto de encasar.
\section{Encasar}
\begin{itemize}
\item {Grp. gram.:v. t.}
\end{itemize}
\begin{itemize}
\item {Utilização:Des.}
\end{itemize}
\begin{itemize}
\item {Grp. gram.:V. i.}
\end{itemize}
\begin{itemize}
\item {Proveniência:(De \textunderscore casa\textunderscore )}
\end{itemize}
Meter no encaixe.
Pôr no seu lugar.
Encaixar.
Meter em casa.
Formar encaixe, em que se embebe.
Acostumar-se.
\section{Encascalhar}
\begin{itemize}
\item {Grp. gram.:v. t.}
\end{itemize}
\begin{itemize}
\item {Utilização:Prov.}
\end{itemize}
Encher de cascalho (caixas de estrada, que se macadamiza).
Pôr cascalho em.
\section{Encascar}
\begin{itemize}
\item {Grp. gram.:v. t.}
\end{itemize}
\begin{itemize}
\item {Utilização:Prov.}
\end{itemize}
\begin{itemize}
\item {Utilização:minh.}
\end{itemize}
\begin{itemize}
\item {Grp. gram.:V. i.}
\end{itemize}
\begin{itemize}
\item {Proveniência:(De \textunderscore casco\textunderscore  e \textunderscore casca\textunderscore )}
\end{itemize}
Revestir de argamassa.
Rebocar.
Tingir (rêdes) em cozimento de casca de salgueiro.
Dar consistência a (pólvora), para depois se reduzir a grãos.
Meter em casco, envasilhar.
Criar casco.
Criar casca.
Tornar-se duro na superfície.
\section{Encasmurrar}
\begin{itemize}
\item {Grp. gram.:v. t.}
\end{itemize}
Tornar casmurro. Cf. Filinto, V, 85.
\section{Encasque}
\begin{itemize}
\item {Grp. gram.:m.}
\end{itemize}
Acto de encascar.
Operação, para dar consistência á pólvora, no fabríco della.
\section{Encasquetar}
\begin{itemize}
\item {Grp. gram.:v. t.}
\end{itemize}
\begin{itemize}
\item {Utilização:Pop.}
\end{itemize}
\begin{itemize}
\item {Utilização:Des.}
\end{itemize}
\begin{itemize}
\item {Proveniência:(De \textunderscore casquete\textunderscore )}
\end{itemize}
Persuadir.
Meter em cabeça de.
Fazer acreditar.
Envolver a cabeça com (casquete, carapuça, etc.).
\section{Encasquilhar}
\begin{itemize}
\item {Grp. gram.:v. t.}
\end{itemize}
Cobrir com casquilha de metal.
\section{Encasquilhar}
\begin{itemize}
\item {Grp. gram.:v. t.}
\end{itemize}
\begin{itemize}
\item {Grp. gram.:V. i.}
\end{itemize}
\begin{itemize}
\item {Proveniência:(De \textunderscore casquilho\textunderscore )}
\end{itemize}
Tornar casquilho, janota.
Fazer-se casquilho, aperaltar-se.
\section{Encastalhado}
\begin{itemize}
\item {Grp. gram.:adj.}
\end{itemize}
\begin{itemize}
\item {Utilização:Prov.}
\end{itemize}
\begin{itemize}
\item {Utilização:beir.}
\end{itemize}
Diz-se do cão e cadella, presos um ao outro pela cópula.
\section{Encastalhar}
\begin{itemize}
\item {Grp. gram.:v. t.}
\end{itemize}
Juntar ou ligar (peças de madeira, como tábuas apparelhadas, para constituir sobrado, etc.).
(Outra fórma de \textunderscore engastalhar\textunderscore )
\section{Encastalho}
\begin{itemize}
\item {Grp. gram.:m.}
\end{itemize}
\begin{itemize}
\item {Utilização:Artilh.}
\end{itemize}
\begin{itemize}
\item {Proveniência:(De \textunderscore encastalhar\textunderscore )}
\end{itemize}
Rebaixo ou friso numa peça que, por êsse lado, se adapta a outra.
Rebaixo no quadrado do eixo de madeira dos reparos.
\section{Encastelado}
\begin{itemize}
\item {Grp. gram.:adj.}
\end{itemize}
Diz-se do casco dos solípedes, quando a tapa e os candados não têm a obliquidade normal, apresentando-se quási verticaes.--\textunderscore Encastilado\textunderscore  e \textunderscore encastilhado\textunderscore  são fórmas espanholadas, adoptadas por Leon, na \textunderscore Arte de Ferrar\textunderscore , 167.
\section{Encasteladura}
\begin{itemize}
\item {Grp. gram.:f.}
\end{itemize}
\begin{itemize}
\item {Proveniência:(De \textunderscore encastelar\textunderscore )}
\end{itemize}
Dôr violenta nas mãos do cavalo.
\section{Encastelamento}
\begin{itemize}
\item {Grp. gram.:m.}
\end{itemize}
Acto ou efeito de encastelar.
\section{Encastelar}
\begin{itemize}
\item {Grp. gram.:v. t.}
\end{itemize}
\begin{itemize}
\item {Utilização:Fig.}
\end{itemize}
\begin{itemize}
\item {Grp. gram.:V. i.}
\end{itemize}
\begin{itemize}
\item {Proveniência:(De \textunderscore castelo\textunderscore )}
\end{itemize}
Fortificar com castelos.
Amontoar, sobrepor (vários objectos).
Dar semelhança de castelo a.
Subir a prumo e caír sem vida, (falando-se da caça ferida na cabeça ou no coração).
\section{Encastellado}
\begin{itemize}
\item {Grp. gram.:adj.}
\end{itemize}
Diz-se do casco dos solípedes, quando a tapa e os candados não têm a obliquidade normal, apresentando-se quási verticaes.--\textunderscore Encastillado\textunderscore  e \textunderscore encastilhado\textunderscore  são fórmas espanholadas, adoptadas por Leon, na \textunderscore Arte de Ferrar\textunderscore , 167.
\section{Encastelladura}
\begin{itemize}
\item {Grp. gram.:f.}
\end{itemize}
\begin{itemize}
\item {Proveniência:(De \textunderscore encastellar\textunderscore )}
\end{itemize}
Dôr violenta nas mãos do cavallo.
\section{Encastellamento}
\begin{itemize}
\item {Grp. gram.:m.}
\end{itemize}
Acto ou effeito de encastellar.
\section{Encastellar}
\begin{itemize}
\item {Grp. gram.:v. t.}
\end{itemize}
\begin{itemize}
\item {Utilização:Fig.}
\end{itemize}
\begin{itemize}
\item {Grp. gram.:V. i.}
\end{itemize}
\begin{itemize}
\item {Proveniência:(De \textunderscore castello\textunderscore )}
\end{itemize}
Fortificar com castellos.
Amontoar, sobrepor (vários objectos).
Dar semelhança de castello a.
Subir a prumo e caír sem vida, (falando-se da caça ferida na cabeça ou no coração).
\section{Encastoar}
\begin{itemize}
\item {Grp. gram.:v. t.}
\end{itemize}
Engastar, embutir.
Pôr castão em: \textunderscore encastoar bengalas\textunderscore .
\section{Encastramento}
\begin{itemize}
\item {Grp. gram.:m.}
\end{itemize}
Acto de encastrar.
\section{Encastrar}
\begin{itemize}
\item {Grp. gram.:v. t.}
\end{itemize}
\begin{itemize}
\item {Utilização:Gal}
\end{itemize}
\begin{itemize}
\item {Proveniência:(Fr. \textunderscore encastrer\textunderscore )}
\end{itemize}
Encaixar; engranzar. Cf. \textunderscore Techn. Rur.\textunderscore , 188.
\section{Encataplasmar}
\begin{itemize}
\item {Grp. gram.:v. t.}
\end{itemize}
Cobrir de cataplasmas.
Tornar achacadiço, doentio.
Encangar.
\section{Encatarrhoar-se}
\begin{itemize}
\item {Grp. gram.:v. p.}
\end{itemize}
Adoecer com catarrho.
Endefluxar-se.
\section{Encatarroar-se}
\begin{itemize}
\item {Grp. gram.:v. p.}
\end{itemize}
Adoecer com catarro.
Endefluxar-se.
\section{Encatrafiar}
\begin{itemize}
\item {Grp. gram.:v. t.}
\end{itemize}
\begin{itemize}
\item {Utilização:Prov.}
\end{itemize}
\begin{itemize}
\item {Utilização:trasm.}
\end{itemize}
Enfiar.
Encadear.
Atar.
\section{Encatramonar-se}
\begin{itemize}
\item {Grp. gram.:v. i.}
\end{itemize}
\begin{itemize}
\item {Utilização:Prov.}
\end{itemize}
\begin{itemize}
\item {Utilização:trasm.}
\end{itemize}
\begin{itemize}
\item {Utilização:pop.}
\end{itemize}
Embezerrar-se.
Pôr-se de trombas.
(Cp. \textunderscore catramonho\textunderscore )
\section{Encatravilhar}
\begin{itemize}
\item {Grp. gram.:v. t.}
\end{itemize}
\begin{itemize}
\item {Utilização:Prov.}
\end{itemize}
\begin{itemize}
\item {Utilização:trasm.}
\end{itemize}
\begin{itemize}
\item {Proveniência:(De \textunderscore travar\textunderscore )}
\end{itemize}
Cruzar (as pernas), apertando uma contra a outra na parte inferior.
\section{Encatrinar-se}
\begin{itemize}
\item {Grp. gram.:v. p.}
\end{itemize}
\begin{itemize}
\item {Utilização:Prov.}
\end{itemize}
Embriagar-se.
\section{Encauchar}
\begin{itemize}
\item {Grp. gram.:v. t.}
\end{itemize}
\begin{itemize}
\item {Utilização:Bras}
\end{itemize}
Tornar sêco e impermeável (um saco ou pano) por meio de cauchu.
\section{Encaustes}
\begin{itemize}
\item {Grp. gram.:m.}
\end{itemize}
\begin{itemize}
\item {Proveniência:(Gr. \textunderscore enkaustes\textunderscore )}
\end{itemize}
Aquelle que trabalhava em encáustica.
\section{Encáustica}
\begin{itemize}
\item {Grp. gram.:f.}
\end{itemize}
\begin{itemize}
\item {Proveniência:(De \textunderscore encáustico\textunderscore )}
\end{itemize}
Camada de cera, sôbre que se pintava.
Pintura em cera.
Preparação de cera com essência de terebenthina, para dar polimento aos móveis.
\section{Encáustico}
\begin{itemize}
\item {Grp. gram.:adj.}
\end{itemize}
\begin{itemize}
\item {Proveniência:(Gr. \textunderscore encaustikos\textunderscore )}
\end{itemize}
Relativo á pintura sôbre cera.
\section{Encausto}
\begin{itemize}
\item {Grp. gram.:m.}
\end{itemize}
\begin{itemize}
\item {Proveniência:(Lat. \textunderscore encaustum\textunderscore )}
\end{itemize}
O mesmo que \textunderscore encáustica\textunderscore .
Tinta purpúrea, de que se serviam os últimos imperadores romanos.
\section{Encava}
\begin{itemize}
\item {Grp. gram.:f.}
\end{itemize}
\begin{itemize}
\item {Proveniência:(De \textunderscore encavar\textunderscore )}
\end{itemize}
Peça, com que se unem dois corpos, em architectura.
\section{Encavacado}
\begin{itemize}
\item {Grp. gram.:adj.}
\end{itemize}
\begin{itemize}
\item {Utilização:Fam.}
\end{itemize}
\begin{itemize}
\item {Proveniência:(De \textunderscore encavacar\textunderscore )}
\end{itemize}
Que deu o cavaco.
Que se amuou.
\section{Encavacar}
\begin{itemize}
\item {Grp. gram.:v. i.}
\end{itemize}
\begin{itemize}
\item {Utilização:Fam.}
\end{itemize}
\begin{itemize}
\item {Proveniência:(De \textunderscore cavaco\textunderscore )}
\end{itemize}
Dar o cavaco.
Ficar embaraçado.
Embirrar.
\section{Encavalar}
\begin{itemize}
\item {Grp. gram.:v. t.}
\end{itemize}
O mesmo que \textunderscore acavalar\textunderscore .
\section{Encavaleirar}
\begin{itemize}
\item {Grp. gram.:v. t.}
\end{itemize}
Sobrepor:«\textunderscore encavaleiravam-se os mais negros e limosos dentes\textunderscore ». Rebello, \textunderscore Mocidade\textunderscore , I, 7.
\section{Encavalgado}
\begin{itemize}
\item {Grp. gram.:adj.}
\end{itemize}
\begin{itemize}
\item {Utilização:Ant.}
\end{itemize}
\begin{itemize}
\item {Proveniência:(De \textunderscore encavalgar\textunderscore )}
\end{itemize}
Que está ou vai a cavallo. Cf. Fern. Lopes, \textunderscore D. João I\textunderscore , p. II, c. 45.
\section{Encavalgadura}
\begin{itemize}
\item {Grp. gram.:f.}
\end{itemize}
\begin{itemize}
\item {Utilização:Ant.}
\end{itemize}
O mesmo que \textunderscore cavalgadura\textunderscore .
\section{Encavalgar}
\textunderscore v. t.\textunderscore  (e der.)
O mesmo que \textunderscore cavalgar\textunderscore ^1, etc.
\section{Encavallar}
\begin{itemize}
\item {Grp. gram.:v. t.}
\end{itemize}
O mesmo que \textunderscore acavallar\textunderscore .
\section{Encavalleirar}
\begin{itemize}
\item {Grp. gram.:v. t.}
\end{itemize}
Sobrepor:«\textunderscore encavalleiravam-se os mais negros e limosos dentes\textunderscore ». Rebello, \textunderscore Mocidade\textunderscore , I, 7.
\section{Encavar}
\begin{itemize}
\item {Grp. gram.:v. t.}
\end{itemize}
Meter na cava.
Escavar.
\section{Encavernar}
\begin{itemize}
\item {Grp. gram.:v. t.}
\end{itemize}
Meter em caverna.
Encovilar.
\section{Encavilhar}
\begin{itemize}
\item {Grp. gram.:v. t.}
\end{itemize}
Ligar com cavilhas.
\section{Encavo}
\begin{itemize}
\item {Grp. gram.:m.}
\end{itemize}
\begin{itemize}
\item {Utilização:Des.}
\end{itemize}
\begin{itemize}
\item {Proveniência:(De \textunderscore encavar\textunderscore )}
\end{itemize}
Encava, encaixe.
\section{Encebolar}
\begin{itemize}
\item {Grp. gram.:v. t.}
\end{itemize}
\begin{itemize}
\item {Utilização:Prov.}
\end{itemize}
\begin{itemize}
\item {Utilização:trasm.}
\end{itemize}
Engodar, embair.
\section{Encedoiro}
\begin{itemize}
\item {Grp. gram.:m.}
\end{itemize}
\begin{itemize}
\item {Utilização:Prov.}
\end{itemize}
\begin{itemize}
\item {Utilização:beir.}
\end{itemize}
Peça de coiro, que abraça lateralmente a cabeça do mango, no mangual.
\section{Encefalalgia}
\begin{itemize}
\item {Grp. gram.:f.}
\end{itemize}
\begin{itemize}
\item {Proveniência:(Do gr. \textunderscore enkephalon\textunderscore  + \textunderscore algos\textunderscore )}
\end{itemize}
Dôr nervosa do encéfalo.
\section{Encefalálgico}
\begin{itemize}
\item {Grp. gram.:adj.}
\end{itemize}
Que tem o carácter da encefalalgia.
\section{Encefalarto}
\begin{itemize}
\item {Grp. gram.:m.}
\end{itemize}
Gênero de plantas cicadáceas.
\section{Encefalia}
\begin{itemize}
\item {Grp. gram.:f.}
\end{itemize}
Doença do encéfalo.
\section{Encefálico}
\begin{itemize}
\item {Grp. gram.:adj.}
\end{itemize}
Relativo ao encéfalo.
\section{Encefalite}
\begin{itemize}
\item {Grp. gram.:f.}
\end{itemize}
Inflamação do encéfalo.
\section{Encefalítico}
\begin{itemize}
\item {Grp. gram.:adj.}
\end{itemize}
Relativo ou semelhante a encefalite.
\section{Encéfalo}
\begin{itemize}
\item {Grp. gram.:m.}
\end{itemize}
\begin{itemize}
\item {Proveniência:(Gr. \textunderscore enkephalon\textunderscore )}
\end{itemize}
Órgão nervoso, que, nos animaes vertebrados, é contido pelo crânio.
\section{Encefalocele}
\begin{itemize}
\item {Grp. gram.:f.}
\end{itemize}
\begin{itemize}
\item {Proveniência:(Do gr. \textunderscore enkephalon\textunderscore  + \textunderscore kele\textunderscore )}
\end{itemize}
Hérnia cerebral.
\section{Encefaloftarsia}
\begin{itemize}
\item {Grp. gram.:f.}
\end{itemize}
\begin{itemize}
\item {Utilização:Med.}
\end{itemize}
Lesão orgânica do cérebro.
\section{Encefaloide}
\begin{itemize}
\item {Grp. gram.:adj.}
\end{itemize}
\begin{itemize}
\item {Proveniência:(Do gr. \textunderscore enkephalon\textunderscore  + \textunderscore eidos\textunderscore )}
\end{itemize}
Que tem sinuosidades semelhantes ás do cérebro.
\section{Encefalólito}
\begin{itemize}
\item {Grp. gram.:m.}
\end{itemize}
\begin{itemize}
\item {Proveniência:(Do gr. \textunderscore enkephalon\textunderscore  + \textunderscore lithos\textunderscore )}
\end{itemize}
Cálculo ou concreção cerebral.
\section{Encefalologia}
\begin{itemize}
\item {Grp. gram.:f.}
\end{itemize}
\begin{itemize}
\item {Proveniência:(Do gr. \textunderscore enkephalon\textunderscore  + \textunderscore logos\textunderscore )}
\end{itemize}
Tratado á cêrca do cérebro.
\section{Encefalólogo}
\begin{itemize}
\item {Grp. gram.:m.}
\end{itemize}
Anatómico, que é perito em encefalologia.
\section{Encefalopatia}
\begin{itemize}
\item {Grp. gram.:f.}
\end{itemize}
\begin{itemize}
\item {Proveniência:(Do gr. \textunderscore enkephalon\textunderscore  + \textunderscore pathos\textunderscore )}
\end{itemize}
Designação genérica das doenças e accidentes graves do systema nervoso.
\section{Encefalorragia}
\begin{itemize}
\item {Grp. gram.:f.}
\end{itemize}
\begin{itemize}
\item {Utilização:Med.}
\end{itemize}
Hemorragia cerebral.
\section{Encefaloscopia}
\begin{itemize}
\item {Grp. gram.:f.}
\end{itemize}
\begin{itemize}
\item {Utilização:Med.}
\end{itemize}
Estudo da estructura do cérebro.
\section{Encefalotomia}
\begin{itemize}
\item {Grp. gram.:f.}
\end{itemize}
\begin{itemize}
\item {Utilização:Cir.}
\end{itemize}
Dissecção do encéfalo.
\section{Encefalozoário}
\begin{itemize}
\item {Grp. gram.:adj.}
\end{itemize}
\begin{itemize}
\item {Utilização:Zool.}
\end{itemize}
\begin{itemize}
\item {Grp. gram.:M.}
\end{itemize}
\begin{itemize}
\item {Proveniência:(Do gr. \textunderscore enkaphalon\textunderscore  + \textunderscore zoon\textunderscore )}
\end{itemize}
Diz-se dos animaes, que têm cérebro.
Animal, que tem cérebro.
\section{Encèguecer}
\begin{itemize}
\item {Grp. gram.:v. i.}
\end{itemize}
\begin{itemize}
\item {Utilização:Prov.}
\end{itemize}
Tornar-se cego.
\section{Encègueirado}
\begin{itemize}
\item {Grp. gram.:adj.}
\end{itemize}
\begin{itemize}
\item {Utilização:Prov.}
\end{itemize}
Aferrado a uma mania, a um vício, a um systema.
\section{Enceguentar}
\begin{itemize}
\item {Grp. gram.:v. t.}
\end{itemize}
\begin{itemize}
\item {Utilização:Ant.}
\end{itemize}
Tornar cego.
Deslumbrar.
Confundir.
Desnortear.
\section{Encélado}
\begin{itemize}
\item {Grp. gram.:m.}
\end{itemize}
\begin{itemize}
\item {Proveniência:(De \textunderscore Encélado\textunderscore ^2, n. p.)}
\end{itemize}
Gênero de insectos coleópteros da Guiana.
\section{Encelar}
\begin{itemize}
\item {Grp. gram.:v. t.}
\end{itemize}
Meter em cela.
Enclausurar; encerrar.
\section{Enceleirar}
\begin{itemize}
\item {Grp. gram.:v. t.}
\end{itemize}
\begin{itemize}
\item {Utilização:Fig.}
\end{itemize}
Meter no celeiro.
Fazer provisões de.
Amontoar: \textunderscore enceleirar dinheiro\textunderscore .
\section{Encélia}
\begin{itemize}
\item {Grp. gram.:f.}
\end{itemize}
\begin{itemize}
\item {Proveniência:(Do gr. \textunderscore en\textunderscore  + \textunderscore koilia\textunderscore )}
\end{itemize}
Planta da América tropical.
\section{Encelialgia}
\begin{itemize}
\item {Grp. gram.:f.}
\end{itemize}
\begin{itemize}
\item {Proveniência:(Do gr. \textunderscore enkoilia\textunderscore  + \textunderscore algos\textunderscore )}
\end{itemize}
Dôr nos intestinos.
\section{Encelite}
\begin{itemize}
\item {Grp. gram.:f.}
\end{itemize}
\begin{itemize}
\item {Proveniência:(Do gr. \textunderscore enkoilia\textunderscore )}
\end{itemize}
Inflammação de intestinos.
\section{Encellar}
\begin{itemize}
\item {Grp. gram.:v. t.}
\end{itemize}
Meter em cella.
Enclausurar; encerrar.
\section{Encelleirar}
\begin{itemize}
\item {Grp. gram.:v. t.}
\end{itemize}
\begin{itemize}
\item {Utilização:Fig.}
\end{itemize}
Meter no celleiro.
Fazer provisões de.
Amontoar: \textunderscore encelleirar dinheiro\textunderscore .
\section{Encender}
\begin{itemize}
\item {Grp. gram.:v. t.}
\end{itemize}
\begin{itemize}
\item {Utilização:Fig.}
\end{itemize}
\begin{itemize}
\item {Proveniência:(Lat. \textunderscore incendere\textunderscore )}
\end{itemize}
Acender.
Tornar inflammado.
Afoguear; avermelhar.
Tornar brilhante.
Estimular.
Exacerbar.
Enthusiasmar.
\section{Encendrar}
\begin{itemize}
\item {Grp. gram.:v. t.}
\end{itemize}
O mesmo que \textunderscore acendrar\textunderscore .
\section{Encênia}
\begin{itemize}
\item {Grp. gram.:f.}
\end{itemize}
\begin{itemize}
\item {Proveniência:(Do gr. \textunderscore en\textunderscore  + \textunderscore kainos\textunderscore )}
\end{itemize}
Festa, que os Judeus e os Gregos celebravam, quando se inaugurava um templo, quando se concluía um edifício notável, quando se iniciava uma grande empresa, etc.
\section{Encenrada}
\begin{itemize}
\item {Grp. gram.:f.}
\end{itemize}
\begin{itemize}
\item {Utilização:Prov.}
\end{itemize}
\begin{itemize}
\item {Utilização:beir.}
\end{itemize}
O mesmo que \textunderscore cenrada\textunderscore  ou \textunderscore barrela\textunderscore . Cf. Castilho, \textunderscore Fastos\textunderscore , II, 326.
\section{Encentrar}
\begin{itemize}
\item {Grp. gram.:v. t.}
\end{itemize}
\begin{itemize}
\item {Proveniência:(De \textunderscore centro\textunderscore )}
\end{itemize}
Concentrar.
\section{Encepar}
\begin{itemize}
\item {Grp. gram.:v. t.}
\end{itemize}
\begin{itemize}
\item {Grp. gram.:V. i.}
\end{itemize}
\begin{itemize}
\item {Utilização:T. da Bairrada}
\end{itemize}
Pôr em cepo.
Tropeçar, esbarrar.
\section{Encephalalgia}
\begin{itemize}
\item {Grp. gram.:f.}
\end{itemize}
\begin{itemize}
\item {Proveniência:(Do gr. \textunderscore enkephalon\textunderscore  + \textunderscore algos\textunderscore )}
\end{itemize}
Dôr nervosa do encéphalo.
\section{Encephalálgico}
\begin{itemize}
\item {Grp. gram.:adj.}
\end{itemize}
Que tem o carácter da encephalalgia.
\section{Encephalarto}
\begin{itemize}
\item {Grp. gram.:m.}
\end{itemize}
Gênero de plantas cycadáceas.
\section{Encephalia}
\begin{itemize}
\item {Grp. gram.:f.}
\end{itemize}
Doença do encéphalo.
\section{Encephálico}
\begin{itemize}
\item {Grp. gram.:adj.}
\end{itemize}
Relativo ao encéphalo.
\section{Encephalite}
\begin{itemize}
\item {Grp. gram.:f.}
\end{itemize}
Inflammação do encéphalo.
\section{Encephalítico}
\begin{itemize}
\item {Grp. gram.:adj.}
\end{itemize}
Relativo ou semelhante a encephalite.
\section{Encéphalo}
\begin{itemize}
\item {Grp. gram.:m.}
\end{itemize}
\begin{itemize}
\item {Proveniência:(Gr. \textunderscore enkephalon\textunderscore )}
\end{itemize}
Órgão nervoso, que, nos animaes vertebrados, é contido pelo crânio.
\section{Encephalocele}
\begin{itemize}
\item {Grp. gram.:f.}
\end{itemize}
\begin{itemize}
\item {Proveniência:(Do gr. \textunderscore enkephalon\textunderscore  + \textunderscore kele\textunderscore )}
\end{itemize}
Hérnia cerebral.
\section{Encephaloide}
\begin{itemize}
\item {Grp. gram.:adj.}
\end{itemize}
\begin{itemize}
\item {Proveniência:(Do gr. \textunderscore enkephalon\textunderscore  + \textunderscore eidos\textunderscore )}
\end{itemize}
Que tem sinuosidades semelhantes ás do cérebro.
\section{Encephalólitho}
\begin{itemize}
\item {Grp. gram.:m.}
\end{itemize}
\begin{itemize}
\item {Proveniência:(Do gr. \textunderscore enkephalon\textunderscore  + \textunderscore lithos\textunderscore )}
\end{itemize}
Cálculo ou concreção cerebral.
\section{Encephalologia}
\begin{itemize}
\item {Grp. gram.:f.}
\end{itemize}
\begin{itemize}
\item {Proveniência:(Do gr. \textunderscore enkephalon\textunderscore  + \textunderscore logos\textunderscore )}
\end{itemize}
Tratado á cêrca do cérebro.
\section{Encephalólogo}
\begin{itemize}
\item {Grp. gram.:m.}
\end{itemize}
Anatómico, que é perito em encephalologia.
\section{Encephalopathia}
\begin{itemize}
\item {Grp. gram.:f.}
\end{itemize}
\begin{itemize}
\item {Proveniência:(Do gr. \textunderscore enkephalon\textunderscore  + \textunderscore pathos\textunderscore )}
\end{itemize}
Designação genérica das doenças e accidentes graves do systema nervoso.
\section{Encephalophtarsia}
\begin{itemize}
\item {Grp. gram.:f.}
\end{itemize}
\begin{itemize}
\item {Utilização:Med.}
\end{itemize}
Lesão orgânica do cérebro.
\section{Encephalorrhagia}
\begin{itemize}
\item {Grp. gram.:f.}
\end{itemize}
\begin{itemize}
\item {Utilização:Med.}
\end{itemize}
Hemorrhagia cerebral.
\section{Encephaloscopia}
\begin{itemize}
\item {Grp. gram.:f.}
\end{itemize}
\begin{itemize}
\item {Utilização:Med.}
\end{itemize}
Estudo da estructura do cérebro.
\section{Encephalotomia}
\begin{itemize}
\item {Grp. gram.:f.}
\end{itemize}
\begin{itemize}
\item {Utilização:Cir.}
\end{itemize}
Dissecção do encéphalo.
\section{Encephalozoário}
\begin{itemize}
\item {Grp. gram.:adj.}
\end{itemize}
\begin{itemize}
\item {Utilização:Zool.}
\end{itemize}
\begin{itemize}
\item {Grp. gram.:M.}
\end{itemize}
\begin{itemize}
\item {Proveniência:(Do gr. \textunderscore enkaphalon\textunderscore  + \textunderscore zoon\textunderscore )}
\end{itemize}
Diz-se dos animaes, que têm cérebro.
Animal, que tem cérebro.
\section{Enceração}
\begin{itemize}
\item {Grp. gram.:f.}
\end{itemize}
\begin{itemize}
\item {Utilização:Fig.}
\end{itemize}
\begin{itemize}
\item {Proveniência:(De \textunderscore encerar\textunderscore )}
\end{itemize}
Acto de encorporar uma substância com cera.
Acto de tornar molle uma substância sêca, pela mistura de algum líquido.
\section{Encerado}
\begin{itemize}
\item {Grp. gram.:m.}
\end{itemize}
\begin{itemize}
\item {Proveniência:(De \textunderscore encerar\textunderscore )}
\end{itemize}
Pano revestido de cera, de óleo ou de breu, para se tornar impermeável.
Oleado.
\section{Enceradura}
\begin{itemize}
\item {Grp. gram.:f.}
\end{itemize}
Acto ou effeito de encerar.
\section{Encerar}
\begin{itemize}
\item {Grp. gram.:v. t.}
\end{itemize}
Cobrir com cera.
Untar com cera.
Dar côr de cera a.
Misturar com cera, ou confeccionar, misturando cera.
\section{Encerebração}
\begin{itemize}
\item {Grp. gram.:f.}
\end{itemize}
\begin{itemize}
\item {Utilização:Neol.}
\end{itemize}
\begin{itemize}
\item {Proveniência:(De \textunderscore encerebrar\textunderscore )}
\end{itemize}
Desenvolvimento intellectual.
Modo de pensar; orientação. Cf. Camillo, \textunderscore Brasileira\textunderscore , 226.
\section{Encerebrar}
\begin{itemize}
\item {Grp. gram.:v. t.}
\end{itemize}
Meter no cérebro.
Decorar, apprender.
\section{Encerra}
\begin{itemize}
\item {Grp. gram.:f.}
\end{itemize}
\begin{itemize}
\item {Utilização:Bras}
\end{itemize}
\begin{itemize}
\item {Proveniência:(De \textunderscore encerrar\textunderscore )}
\end{itemize}
Curral ao ar livre.
Malhada.
\section{Encerrador}
\begin{itemize}
\item {Grp. gram.:adj.}
\end{itemize}
\begin{itemize}
\item {Grp. gram.:M.}
\end{itemize}
Que encerra.
Aquelle que encerra.
\section{Encerramento}
\begin{itemize}
\item {Grp. gram.:m.}
\end{itemize}
Acto ou effeito de encerrar.
\section{Encerrar}
\begin{itemize}
\item {Grp. gram.:v. t.}
\end{itemize}
Fechar dentro de alguma coisa: \textunderscore encerrar dinheiro num cofre\textunderscore .
Meter em clausura.
Incluir, conter: \textunderscore livro, que encerra grandes verdades\textunderscore .
Rematar, pôr limite a: \textunderscore encerrar um discurso\textunderscore .
Estreitar, resumir.
Occultar: \textunderscore o coração encerra mystérios\textunderscore .
Cerrar, fechar a porta de.
(B. lat. \textunderscore incerrare\textunderscore )
\section{Encêrro}
\begin{itemize}
\item {Grp. gram.:m.}
\end{itemize}
Lugar, em que se encerra alguém ou alguma coisa.
Acto de encerrar.
\section{Encertar}
\begin{itemize}
\item {Grp. gram.:v. t.}
\end{itemize}
\begin{itemize}
\item {Utilização:Prov.}
\end{itemize}
Partir ou separar uma parte de.
Comer ou gastar um pedaço de: \textunderscore encertar um pão\textunderscore .
(Por \textunderscore encetar\textunderscore )
\section{Encestamento}
\begin{itemize}
\item {Grp. gram.:m.}
\end{itemize}
Acto ou effeito de encestar.
\section{Encestar}
\begin{itemize}
\item {Grp. gram.:v. t.}
\end{itemize}
\begin{itemize}
\item {Utilização:Bras}
\end{itemize}
Meter em cesto.
\section{Encetado}
\begin{itemize}
\item {Grp. gram.:adj.}
\end{itemize}
\begin{itemize}
\item {Utilização:Prov.}
\end{itemize}
\begin{itemize}
\item {Utilização:alg.}
\end{itemize}
Que se encetou.
Gretado.
\section{Encetadura}
\begin{itemize}
\item {Grp. gram.:f.}
\end{itemize}
O mesmo que \textunderscore encetamento\textunderscore .
\section{Encetamento}
\begin{itemize}
\item {Grp. gram.:m.}
\end{itemize}
Acto de encetar. Cf. Filinto, \textunderscore D. Man.\textunderscore , I, 110.
\section{Encetar}
\begin{itemize}
\item {Grp. gram.:v. t.}
\end{itemize}
\begin{itemize}
\item {Proveniência:(Do lat. \textunderscore inceptare\textunderscore )}
\end{itemize}
Principiar: \textunderscore encetar um escrito\textunderscore .
Tirar parte de (uma coisa que estava inteira).
Estrear: \textunderscore encetar uma bicycleta\textunderscore .
\section{Enchaboucado}
\begin{itemize}
\item {Grp. gram.:adj.}
\end{itemize}
Em que há chaboucos, covas ou fossos:«\textunderscore os terrenos eram lenteiros, enchaboucados\textunderscore ». B. Pato.
\section{Enchacotar}
\begin{itemize}
\item {Grp. gram.:v. t.}
\end{itemize}
Dar a primeira cozedura a (loiça), antes de sêr vidrada.
\section{Enchafurdamento}
\begin{itemize}
\item {Grp. gram.:m.}
\end{itemize}
Acto de enchafurdar.
\section{Enchafurdar}
\begin{itemize}
\item {Grp. gram.:v. t.  e  pr.}
\end{itemize}
O mesmo que \textunderscore chafurdar\textunderscore .
\section{Enchamate}
\begin{itemize}
\item {Grp. gram.:m.}
\end{itemize}
\begin{itemize}
\item {Utilização:Prov.}
\end{itemize}
Conclusão da parede de um prédio, onde assentam os frechaes, junto ao telhado.
(Colhido em Arganil)
\section{Enchamejar}
\begin{itemize}
\item {Grp. gram.:v. i.}
\end{itemize}
Lançar chamas. Cf. Filinto, XVI, 17.
\section{Enchamerdeado}
\begin{itemize}
\item {Grp. gram.:adj.}
\end{itemize}
\begin{itemize}
\item {Utilização:Pleb.}
\end{itemize}
Coberto de excrementos; immundo; emporcalhado.
\section{Enchammejar}
\begin{itemize}
\item {Grp. gram.:v. i.}
\end{itemize}
Lançar chammas. Cf. Filinto, XVI, 17.
\section{Enchamoucido}
\begin{itemize}
\item {Grp. gram.:adj.}
\end{itemize}
\begin{itemize}
\item {Utilização:T. do distr. de Castello-Branco}
\end{itemize}
\begin{itemize}
\item {Utilização:fam.}
\end{itemize}
Adoentado, encarangado.
\section{Enchanqueta}
\begin{itemize}
\item {fónica:quê}
\end{itemize}
\begin{itemize}
\item {Grp. gram.:f.}
\end{itemize}
\begin{itemize}
\item {Utilização:Prov.}
\end{itemize}
\begin{itemize}
\item {Utilização:alg.}
\end{itemize}
O mesmo que \textunderscore chanqueta\textunderscore .
\section{Enchapinado}
\begin{itemize}
\item {Grp. gram.:adj.}
\end{itemize}
\begin{itemize}
\item {Proveniência:(De \textunderscore chapim\textunderscore )}
\end{itemize}
Diz-se dos cascos muito endurecidos e defeituosos, nas cavalgaduras.
\section{Enchapotar}
\textunderscore v. t.\textunderscore  (e der.)
O mesmo que \textunderscore chapotar\textunderscore , etc.
\section{Enchapuçar}
\begin{itemize}
\item {Grp. gram.:v. t.}
\end{itemize}
\begin{itemize}
\item {Utilização:Prov.}
\end{itemize}
\begin{itemize}
\item {Utilização:beir.}
\end{itemize}
O mesmo que \textunderscore chapuçar\textunderscore .
\section{Encharcada}
\begin{itemize}
\item {Grp. gram.:f.}
\end{itemize}
\begin{itemize}
\item {Proveniência:(De \textunderscore encharcar\textunderscore )}
\end{itemize}
Espécie de pudim, feito de pão, ovos, etc.
\section{Encharcadiço}
\begin{itemize}
\item {Grp. gram.:adj.}
\end{itemize}
\begin{itemize}
\item {Proveniência:(De \textunderscore encharcar\textunderscore )}
\end{itemize}
O mesmo que \textunderscore alagadiço\textunderscore .
\section{Encharcar}
\begin{itemize}
\item {Grp. gram.:v. t.}
\end{itemize}
\begin{itemize}
\item {Grp. gram.:V. p.}
\end{itemize}
\begin{itemize}
\item {Proveniência:(De \textunderscore charco\textunderscore )}
\end{itemize}
Tornar em charco: \textunderscore a chuva encharcou o jardim\textunderscore .
Converter em pântano.
Encher de água.
Molhar muito, ensopar: \textunderscore a chuva encharcou-lhe o fato\textunderscore .
Meter em charco, em atoleiro.
Tornar-se em charco.
Meter-se em charco ou atoleiro.
Molhar-se completamente, apanhando chuva.
Estar muito molhado.
\section{Encharéu}
\begin{itemize}
\item {Grp. gram.:m.}
\end{itemize}
Peixe dos Açores.
\section{Encharolado}
\begin{itemize}
\item {Grp. gram.:adj.}
\end{itemize}
Posto em charola. Cf. Filinto, VI, 38.
\section{Encharque}
\begin{itemize}
\item {Grp. gram.:m.}
\end{itemize}
Acto de encharcar. Cf. Castilho, \textunderscore Collòq. Ald.\textunderscore , 304.
\section{Enchavetar}
\begin{itemize}
\item {Grp. gram.:v. t.}
\end{itemize}
Segurar com chaveta.
\section{Enchedeira}
\begin{itemize}
\item {Grp. gram.:f.}
\end{itemize}
\begin{itemize}
\item {Proveniência:(De \textunderscore encher\textunderscore )}
\end{itemize}
Espécie de funil pequeno, por onde se mete a carne que enche os chouriços.
\section{Enchedela}
\begin{itemize}
\item {Grp. gram.:f.}
\end{itemize}
\begin{itemize}
\item {Utilização:Fam.}
\end{itemize}
Acto de encher.
Fartadela; pançada.
\section{Enchedor}
\begin{itemize}
\item {Grp. gram.:m.  e  adj.}
\end{itemize}
O que enche.
\section{Encheio}
\begin{itemize}
\item {Grp. gram.:m.}
\end{itemize}
\begin{itemize}
\item {Utilização:Ant.}
\end{itemize}
Por encheio, completamente. Cf. \textunderscore Aulegrafia\textunderscore , 11.
\section{Enchelevar}
\begin{itemize}
\item {Grp. gram.:m.}
\end{itemize}
\begin{itemize}
\item {Proveniência:(De \textunderscore encher\textunderscore  + \textunderscore levar\textunderscore )}
\end{itemize}
Pequena rêde cylíndrica ou em fórma de saco, para transportar peixe.
\section{Enchemão}
\begin{itemize}
\item {Grp. gram.:loc. qualif.}
\end{itemize}
Que é perfeito, vistoso. Cf. Castilho, \textunderscore Fausto\textunderscore , 347 e 353.
\section{Enche-mão}
\begin{itemize}
\item {fónica:de}
\end{itemize}
\begin{itemize}
\item {Grp. gram.:loc. qualif.}
\end{itemize}
Que é perfeito, vistoso. Cf. Castilho, \textunderscore Fausto\textunderscore , 347 e 353.
\section{Enchente}
\begin{itemize}
\item {Grp. gram.:f.}
\end{itemize}
\begin{itemize}
\item {Utilização:Fam.}
\end{itemize}
Acto de encher.
Grande porção.
Inundação, cheia.
Escândalo: \textunderscore aquelle divórcio deu enchente\textunderscore .
\section{Enchentes-disso}
\begin{itemize}
\item {Grp. gram.:loc. adv.}
\end{itemize}
\begin{itemize}
\item {Utilização:Ant.}
\end{itemize}
Além disso.
\section{Encher}
\begin{itemize}
\item {Grp. gram.:v. t.}
\end{itemize}
\begin{itemize}
\item {Utilização:Bras. do N}
\end{itemize}
\begin{itemize}
\item {Grp. gram.:V. i.}
\end{itemize}
\begin{itemize}
\item {Grp. gram.:V. p.}
\end{itemize}
\begin{itemize}
\item {Proveniência:(Do lat. \textunderscore implere\textunderscore )}
\end{itemize}
Tornar cheio: \textunderscore encher um pote\textunderscore .
Occupar (aquillo que estava vazio).
Abarrotar: \textunderscore encher o estômago\textunderscore .
Diffundir-se por: \textunderscore a água encheu a rua\textunderscore .
Cobrir.
Abranger.
Cumprir.
\textunderscore Encher água\textunderscore , ir buscar água á fonte.
Crescer gradualmente, ir subindo, (falando-se das marés ou das correntes).
Fartar-se.
Enriquecer illicitamente.
Locupletar-se.
Deixar-se dominar, possuir-se: \textunderscore encher-se de razão\textunderscore .
\section{Encherca}
\begin{itemize}
\item {Grp. gram.:f.}
\end{itemize}
\begin{itemize}
\item {Utilização:Ant.}
\end{itemize}
Imposto sôbre a venda da carne.
(Cp. charque)
\section{Enchia}
\begin{itemize}
\item {Grp. gram.:f.}
\end{itemize}
\begin{itemize}
\item {Proveniência:(De \textunderscore encher\textunderscore )}
\end{itemize}
Onda, que, nas águas vivas, se alastra pela praia até maior distância que as ondas precedentes e as seguintes.
\section{Enchicharado}
\begin{itemize}
\item {Grp. gram.:adj.}
\end{itemize}
\begin{itemize}
\item {Utilização:T. de Turquel}
\end{itemize}
O mesmo que \textunderscore enchicharrado\textunderscore .
\section{Enchicharrado}
\begin{itemize}
\item {Grp. gram.:adj.}
\end{itemize}
\begin{itemize}
\item {Utilização:Prov.}
\end{itemize}
\begin{itemize}
\item {Utilização:beir.}
\end{itemize}
\begin{itemize}
\item {Proveniência:(De \textunderscore chicharro\textunderscore )}
\end{itemize}
Cheio de si.
Vaidoso.
Que faz ostentação do seu traje ou da sua pessôa.
\section{Enchicharrar-se}
\begin{itemize}
\item {Grp. gram.:v. p.}
\end{itemize}
\begin{itemize}
\item {Utilização:Prov.}
\end{itemize}
\begin{itemize}
\item {Utilização:beir.}
\end{itemize}
Tornar-se enchicharrado.
Ter vaidade ou orgulho.
\section{Enchido}
\begin{itemize}
\item {Grp. gram.:m.}
\end{itemize}
\begin{itemize}
\item {Proveniência:(De \textunderscore encher\textunderscore )}
\end{itemize}
Chumaço.
Carne ensacada.
\section{Enchimarrar}
\begin{itemize}
\item {Grp. gram.:v. t.}
\end{itemize}
Vestir de chimarra.
\section{Enchimento}
\begin{itemize}
\item {Grp. gram.:m.}
\end{itemize}
Acto ou effeito de encher.
Recheio.
\section{Enchiqueirar}
\begin{itemize}
\item {Grp. gram.:v. t.}
\end{itemize}
\begin{itemize}
\item {Utilização:Bras}
\end{itemize}
\begin{itemize}
\item {Grp. gram.:V. i.}
\end{itemize}
Meter no chiqueiro (o peixe).
Meter-se o peixe no chiqueiro.
\section{Enchirídio}
\begin{itemize}
\item {fónica:qui}
\end{itemize}
\begin{itemize}
\item {Grp. gram.:m.}
\end{itemize}
\begin{itemize}
\item {Proveniência:(Gr. \textunderscore enkheiridion\textunderscore )}
\end{itemize}
Manual ou epítome, organizado por autor antigo.
\section{Enchocalhação}
\begin{itemize}
\item {Grp. gram.:f.}
\end{itemize}
Acto de enchocalhar.
\section{Enchocalhador}
\begin{itemize}
\item {Grp. gram.:m.}
\end{itemize}
Aquelle que enchocalha.
\section{Enchocalhar}
\begin{itemize}
\item {Grp. gram.:v. i.}
\end{itemize}
\begin{itemize}
\item {Proveniência:(De \textunderscore chocalho\textunderscore )}
\end{itemize}
Pôr chocalho a (gado).
\section{Enchoçar}
\begin{itemize}
\item {Grp. gram.:v. t.}
\end{itemize}
Meter em choça; encurralar.
\section{Enchoiriçado}
\begin{itemize}
\item {Grp. gram.:adj.}
\end{itemize}
O mesmo que \textunderscore enchicharrado\textunderscore .
Altivo; arrogante.
\section{Enchoiriçar}
\begin{itemize}
\item {Grp. gram.:v. t.}
\end{itemize}
\begin{itemize}
\item {Grp. gram.:V. p.}
\end{itemize}
\begin{itemize}
\item {Utilização:Prov.}
\end{itemize}
\begin{itemize}
\item {Proveniência:(De \textunderscore chouriço\textunderscore )}
\end{itemize}
Dar fórma de chouriço a.
Ouriçar-se, encrespar-se, (um animal).
O mesmo que \textunderscore enchicharrar-se\textunderscore .
\section{Enchondroma}
\begin{itemize}
\item {fónica:con}
\end{itemize}
\begin{itemize}
\item {Grp. gram.:m.}
\end{itemize}
\begin{itemize}
\item {Utilização:Med.}
\end{itemize}
\begin{itemize}
\item {Proveniência:(Do gr. \textunderscore en\textunderscore  + \textunderscore khondros\textunderscore )}
\end{itemize}
Tumor cartilaginoso.
\section{Enchorrar}
\begin{itemize}
\item {Grp. gram.:v. i.}
\end{itemize}
\begin{itemize}
\item {Utilização:Des.}
\end{itemize}
\begin{itemize}
\item {Proveniência:(De \textunderscore chorro\textunderscore )}
\end{itemize}
Escorrer.
\section{Enchoupado}
\begin{itemize}
\item {Grp. gram.:adj.}
\end{itemize}
Diz-se do varapau, que tem choupa. Cf. Arn. Gama, \textunderscore Segr. do Abb.\textunderscore , 63.
\section{Enchouriçado}
\begin{itemize}
\item {Grp. gram.:adj.}
\end{itemize}
O mesmo que \textunderscore enchicharrado\textunderscore .
Altivo; arrogante.
\section{Enchouriçar}
\begin{itemize}
\item {Grp. gram.:v. t.}
\end{itemize}
\begin{itemize}
\item {Grp. gram.:V. p.}
\end{itemize}
\begin{itemize}
\item {Utilização:Prov.}
\end{itemize}
\begin{itemize}
\item {Proveniência:(De \textunderscore chouriço\textunderscore )}
\end{itemize}
Dar fórma de chouriço a.
Ouriçar-se, encrespar-se, (um animal).
O mesmo que \textunderscore enchicharrar-se\textunderscore .
\section{Enchova}
\begin{itemize}
\item {Grp. gram.:f.}
\end{itemize}
O mesmo que \textunderscore anchova\textunderscore .
\section{Enchumaçamento}
\begin{itemize}
\item {Grp. gram.:m.}
\end{itemize}
Acto ou effeito de enchumaçar.
\section{Enchumaçar}
\begin{itemize}
\item {Grp. gram.:v. t.}
\end{itemize}
Pôr chumaço em.
Estofar.
\section{Enchumarrar}
\begin{itemize}
\item {Grp. gram.:v. t.}
\end{itemize}
(V.enchimarrar)
\section{Enchumbado}
\begin{itemize}
\item {Grp. gram.:adj.}
\end{itemize}
\begin{itemize}
\item {Utilização:Prov.}
\end{itemize}
\begin{itemize}
\item {Utilização:trasm.}
\end{itemize}
\begin{itemize}
\item {Proveniência:(De \textunderscore enchumbar\textunderscore )}
\end{itemize}
Pesado como chumbo, por se molhar.
Que tem a elasticidade perdida, (falando-se da pelota).
\section{Enchumbar}
\begin{itemize}
\item {Grp. gram.:V. p.}
\end{itemize}
\begin{itemize}
\item {Utilização:Prov.}
\end{itemize}
\begin{itemize}
\item {Utilização:trasm.}
\end{itemize}
\textunderscore v. t.\textunderscore  (e der.)
O mesmo que \textunderscore chumbar\textunderscore , etc.
Pôr-se muito pesado, por se molhar.
Perder a elasticidade (a pelota).
\section{Enchusmar}
\begin{itemize}
\item {Grp. gram.:v. t.}
\end{itemize}
\begin{itemize}
\item {Proveniência:(De \textunderscore chusma\textunderscore )}
\end{itemize}
Encher de gente: \textunderscore os navios iam enchusmados\textunderscore .
\section{Enchymose}
\begin{itemize}
\item {fónica:qui}
\end{itemize}
\begin{itemize}
\item {Grp. gram.:f.}
\end{itemize}
O mesmo que \textunderscore ecchymose\textunderscore .
\section{Encieirado}
\begin{itemize}
\item {Grp. gram.:adj.}
\end{itemize}
\begin{itemize}
\item {Utilização:Prov.}
\end{itemize}
Diz-se do solo, recentemente lavrado, quando o calor o encrespa e greta. (Colhido em Turquel)
\section{Encilhamento}
\begin{itemize}
\item {Grp. gram.:m.}
\end{itemize}
Acto ou effeito de encilhar.
\section{Encilhar}
\begin{itemize}
\item {Grp. gram.:v. t.}
\end{itemize}
Apertar com cilha; arrear (a bêsta).
\section{Encimado}
\begin{itemize}
\item {Grp. gram.:m.}
\end{itemize}
\begin{itemize}
\item {Proveniência:(De \textunderscore encimar\textunderscore )}
\end{itemize}
Remate sôbre o escudo heráldico.
\section{Encimar}
\begin{itemize}
\item {Grp. gram.:v. t.}
\end{itemize}
\begin{itemize}
\item {Utilização:Prov.}
\end{itemize}
\begin{itemize}
\item {Utilização:alent.}
\end{itemize}
Pôr em cima: \textunderscore encimar a cruz na igreja\textunderscore .
Collocar sôbre: \textunderscore encimar a igreja com uma cruz\textunderscore .
Coroar, rematar: \textunderscore a neve encimava a serra\textunderscore .
Concluir (contrato).
\section{Encinchamento}
\begin{itemize}
\item {Grp. gram.:m.}
\end{itemize}
Acto de encinchar.
\section{Encinchar}
\begin{itemize}
\item {Grp. gram.:v. t.}
\end{itemize}
Meter no cincho (a coalhada), para o fabrico do queijo.
Meter no cincho (a massa da azeitona) para a premer e fazer azeite.
\section{Encinhar}
\begin{itemize}
\item {Grp. gram.:v. i.}
\end{itemize}
\begin{itemize}
\item {Utilização:Prov.}
\end{itemize}
\begin{itemize}
\item {Utilização:minh.}
\end{itemize}
Trabalhar com \textunderscore encinho\textunderscore .
\section{Encinho}
\begin{itemize}
\item {Grp. gram.:m.}
\end{itemize}
O mesmo que \textunderscore ancinho\textunderscore . Cf. Filinto, VII, 241.
\section{Encintar}
\begin{itemize}
\item {Grp. gram.:v. t.}
\end{itemize}
Cingir ou reforçar com cintas.
\section{Encinzar}
\begin{itemize}
\item {Grp. gram.:v. t.}
\end{itemize}
Cobrir de cinza.
Sujar com cinza.
\section{Encinzeirado}
\begin{itemize}
\item {Grp. gram.:adj.}
\end{itemize}
\begin{itemize}
\item {Proveniência:(De \textunderscore cinza\textunderscore )}
\end{itemize}
Diz-se do céu ou da atmosphera ennevoada ou pardacenta.
\section{Encioso}
\begin{itemize}
\item {Grp. gram.:adj.}
\end{itemize}
\begin{itemize}
\item {Utilização:Prov.}
\end{itemize}
\begin{itemize}
\item {Utilização:beir.}
\end{itemize}
Cheio de cio.
\section{Encirrar}
\textunderscore v. t.\textunderscore  (e der.)
O mesmo que \textunderscore acirrar\textunderscore , etc.
\section{Enciscar}
\begin{itemize}
\item {Grp. gram.:v. t.}
\end{itemize}
\begin{itemize}
\item {Utilização:Prov.}
\end{itemize}
\begin{itemize}
\item {Utilização:trasm.}
\end{itemize}
\begin{itemize}
\item {Proveniência:(De \textunderscore cisco\textunderscore )}
\end{itemize}
Borrar, sujar.
\section{Enclarear}
\begin{itemize}
\item {Grp. gram.:v. i.}
\end{itemize}
\begin{itemize}
\item {Utilização:Prov.}
\end{itemize}
\begin{itemize}
\item {Utilização:alg.}
\end{itemize}
\begin{itemize}
\item {Proveniência:(De \textunderscore clarear\textunderscore )}
\end{itemize}
Clarear (a manhan, o dia).
\section{Enclaustrar}
\begin{itemize}
\item {Grp. gram.:v. t.}
\end{itemize}
\begin{itemize}
\item {Proveniência:(De \textunderscore claustro\textunderscore )}
\end{itemize}
Meter em convento.
Enclausurar.
\section{Enclausurar}
\begin{itemize}
\item {Grp. gram.:v. t.}
\end{itemize}
\begin{itemize}
\item {Proveniência:(De \textunderscore clausura\textunderscore )}
\end{itemize}
Pôr em clausura.
Separar do trato social.
\section{Enclavinhar}
\begin{itemize}
\item {Grp. gram.:v. t.}
\end{itemize}
\begin{itemize}
\item {Proveniência:(Do rad. do lat. \textunderscore clavare\textunderscore )}
\end{itemize}
Meter uns pelos outros (os dedos).
Travá-los.
\section{Ênclise}
\begin{itemize}
\item {Grp. gram.:f.}
\end{itemize}
\begin{itemize}
\item {Utilização:Gram.}
\end{itemize}
\begin{itemize}
\item {Proveniência:(Do gr. \textunderscore enklisis\textunderscore )}
\end{itemize}
Qualidade ou emprêgo de enclítica.
Posposição de pronomes complementares aos verbos.
\section{Enclítica}
\begin{itemize}
\item {Grp. gram.:f.}
\end{itemize}
\begin{itemize}
\item {Proveniência:(De \textunderscore enclítico\textunderscore )}
\end{itemize}
Palavra que, juntando-se a outra que a precede, parece formar com ella uma só palavra, perdendo o accento próprio, como succede a \textunderscore lhes\textunderscore  em \textunderscore dirigiu-lhes\textunderscore , etc.
\section{Enclítico}
\begin{itemize}
\item {Grp. gram.:adj.}
\end{itemize}
\begin{itemize}
\item {Utilização:Gram.}
\end{itemize}
\begin{itemize}
\item {Proveniência:(Gr. \textunderscore enklitikos\textunderscore )}
\end{itemize}
Diz-se das palavras que, perdendo o accento próprio, fazem parte de outras, que as precedem.
Posposto ao verbo, (falando-se de pronomes complementares).
\section{Encoadura}
\begin{itemize}
\item {Grp. gram.:f.}
\end{itemize}
\begin{itemize}
\item {Proveniência:(De \textunderscore coar\textunderscore ?)}
\end{itemize}
Depósito provisório de peixes ainda vivos, dentro de água.
\section{Encobardar}
\begin{itemize}
\item {Grp. gram.:v. t.}
\end{itemize}
Tornar cobarde.
Intimidar. Cf. Garrett, \textunderscore Romanceiro\textunderscore , II, 253.
\section{Encoberta}
\begin{itemize}
\item {Grp. gram.:f.}
\end{itemize}
\begin{itemize}
\item {Proveniência:(De \textunderscore encoberto\textunderscore )}
\end{itemize}
Abrigo.
Escaninho.
Aquillo que encobre ou tolhe a vista.
Pretexto; subterfúgio.
\section{Encobertado}
\begin{itemize}
\item {Grp. gram.:m.}
\end{itemize}
\begin{itemize}
\item {Proveniência:(De \textunderscore encobertar\textunderscore )}
\end{itemize}
O mesmo que \textunderscore tatu\textunderscore ^1.
\section{Encobertamente}
\begin{itemize}
\item {Grp. gram.:adv.}
\end{itemize}
De modo encoberto.
\section{Encobertar}
\begin{itemize}
\item {Grp. gram.:v. t.}
\end{itemize}
(V.acobertar)
\section{Encobertas}
\begin{itemize}
\item {Grp. gram.:f. pl. Loc. adv.}
\end{itemize}
\textunderscore Ás encobertas\textunderscore , ás occultas, clandestinamente. Cf. Filinto, \textunderscore D. Man.\textunderscore , 11.
\section{Encoberto}
\begin{itemize}
\item {Grp. gram.:adj.}
\end{itemize}
\begin{itemize}
\item {Grp. gram.:M.}
\end{itemize}
\begin{itemize}
\item {Proveniência:(De \textunderscore encobrir\textunderscore )}
\end{itemize}
Que se não deixa vêr.
Escondido.
Dissimulado.
Mysterioso.
Aquelle ou aquillo que se não deixa vêr.
Aquillo que é mysterioso.
O mesmo que \textunderscore tatu\textunderscore ^1.
\section{Encobilhar}
\begin{itemize}
\item {Grp. gram.:v. t.}
\end{itemize}
\begin{itemize}
\item {Utilização:Prov.}
\end{itemize}
\begin{itemize}
\item {Utilização:trasm.}
\end{itemize}
O mesmo que \textunderscore entupir\textunderscore .
(Cp. \textunderscore encovilar\textunderscore )
\section{Encòbrado}
\begin{itemize}
\item {Grp. gram.:adj.}
\end{itemize}
\begin{itemize}
\item {Utilização:Prov.}
\end{itemize}
\begin{itemize}
\item {Utilização:alent.}
\end{itemize}
Que se move, á maneira de cobra.
\section{Encobrideira}
\begin{itemize}
\item {Grp. gram.:f.}
\end{itemize}
\begin{itemize}
\item {Proveniência:(De \textunderscore encobrir\textunderscore )}
\end{itemize}
Mulher que encobre, que recepta.
\section{Encobridor}
\begin{itemize}
\item {Grp. gram.:adj.}
\end{itemize}
\begin{itemize}
\item {Grp. gram.:M.}
\end{itemize}
\begin{itemize}
\item {Proveniência:(De \textunderscore encobrir\textunderscore )}
\end{itemize}
Que encobre.
Aquelle que encobre.
Receptador.
\section{Encobridora}
\begin{itemize}
\item {Grp. gram.:f.}
\end{itemize}
O mesmo que \textunderscore encobrideira\textunderscore .
\section{Encobrimento}
\begin{itemize}
\item {Grp. gram.:m.}
\end{itemize}
Acto ou effeito de encobrir.
\section{Encobrir}
\begin{itemize}
\item {Grp. gram.:v. t.}
\end{itemize}
\begin{itemize}
\item {Grp. gram.:V. i.}
\end{itemize}
\begin{itemize}
\item {Proveniência:(De \textunderscore cobrir\textunderscore )}
\end{itemize}
Não deixar vêr: \textunderscore as nuvens encobrem o sol\textunderscore .
Não deixar ouvir.
Esconder.
Dar acolhimento ou evasão a (criminoso ou accusado).
Receptar: \textunderscore encobrir objectos roubados\textunderscore .
Dissimular.
Tapar, toldar.
Cobrir-se de nuvens, toldar-se, (falando-se do céu ou do tempo).
\section{Encodar-se}
\begin{itemize}
\item {Grp. gram.:v. p.}
\end{itemize}
\begin{itemize}
\item {Proveniência:(De \textunderscore coda\textunderscore )}
\end{itemize}
Inclinar a popa ou metê-la debaixo de água.
\section{Encodeamento}
\begin{itemize}
\item {Grp. gram.:m.}
\end{itemize}
Acto de encodear.
\section{Encodear}
\begin{itemize}
\item {Grp. gram.:v. t.}
\end{itemize}
\begin{itemize}
\item {Grp. gram.:V. i.}
\end{itemize}
Fazer côdea em.
Cobrir de côdea.
Criar côdea.
\section{Encodoar}
\begin{itemize}
\item {Grp. gram.:v. i.}
\end{itemize}
\begin{itemize}
\item {Utilização:Prov.}
\end{itemize}
\begin{itemize}
\item {Utilização:trasm.}
\end{itemize}
Tomar ou adquirir codo.
\section{Encofar}
\begin{itemize}
\item {Grp. gram.:v. t.}
\end{itemize}
\begin{itemize}
\item {Utilização:Bras. do N}
\end{itemize}
Guardar em cofo.
Occultar.
\section{Encofrar}
\begin{itemize}
\item {Grp. gram.:v. t.}
\end{itemize}
\begin{itemize}
\item {Utilização:Neol.}
\end{itemize}
Meter em cofre. Cf. Camillo, \textunderscore Volcões\textunderscore , 166.
\section{Encoifar}
\begin{itemize}
\item {Grp. gram.:v. t.}
\end{itemize}
Pôr coifa em.
\section{Encoimação}
\begin{itemize}
\item {Grp. gram.:f.}
\end{itemize}
Acto ou effeito de encoimar.
\section{Encoimar}
\begin{itemize}
\item {Grp. gram.:v. t.}
\end{itemize}
O mesmo que \textunderscore acoimar\textunderscore .
\section{Encoiraçar}
\begin{itemize}
\item {Grp. gram.:v. t.}
\end{itemize}
(V.coiraçar)
\section{Encoirado}
\begin{itemize}
\item {Grp. gram.:adj.}
\end{itemize}
\begin{itemize}
\item {Utilização:Bras. do N}
\end{itemize}
\begin{itemize}
\item {Grp. gram.:M.}
\end{itemize}
\begin{itemize}
\item {Utilização:Bras. do N}
\end{itemize}
Vestido de coiro, como os vaqueiros do sertão.
Aquelle que usa roupa de coiro, como os vaqueiros do sertão.
\section{Encoirar}
\begin{itemize}
\item {Grp. gram.:v. t.}
\end{itemize}
\begin{itemize}
\item {Grp. gram.:V. i.  e  p.}
\end{itemize}
\begin{itemize}
\item {Proveniência:(Do b. lat. \textunderscore incoriare\textunderscore )}
\end{itemize}
Revestir de coiro.
Criar nova pelle (uma ferida).
\section{Encoitar}
\begin{itemize}
\item {Grp. gram.:v. t.}
\end{itemize}
\begin{itemize}
\item {Utilização:Ant.}
\end{itemize}
\begin{itemize}
\item {Utilização:Ant.}
\end{itemize}
\begin{itemize}
\item {Proveniência:(De \textunderscore encoito\textunderscore )}
\end{itemize}
Apprehender (objecto defeso por lei).
Avaliar ou fazer pagar o encoito a (alguém que incorreu em multa por uso de objectos prohibidos).
Dar privilégio de coito a. Cf. \textunderscore Port. Mon. Hist., Scrip.\textunderscore , 317.
\section{Encoiteiro}
\begin{itemize}
\item {Grp. gram.:m.}
\end{itemize}
\begin{itemize}
\item {Utilização:Ant.}
\end{itemize}
\begin{itemize}
\item {Proveniência:(De \textunderscore encoitar\textunderscore )}
\end{itemize}
Aquelle que encoita.
\section{Encoito}
\begin{itemize}
\item {Grp. gram.:m.}
\end{itemize}
\begin{itemize}
\item {Utilização:Ant.}
\end{itemize}
\begin{itemize}
\item {Proveniência:(De \textunderscore coito\textunderscore )}
\end{itemize}
Multa, imposta por transgressão da lei que prohibia o uso de certos objectos ou por infracção dos privilegios de coito.
\section{Encoivaramento}
\begin{itemize}
\item {Grp. gram.:m.}
\end{itemize}
Acto de encoivarar.
\section{Encoivarar}
\begin{itemize}
\item {Grp. gram.:v. t.}
\end{itemize}
O mesmo que \textunderscore coivarar\textunderscore .
\section{Encolamento}
\begin{itemize}
\item {Grp. gram.:m.}
\end{itemize}
Acto ou efeito de encolar.
\section{Encolar}
\begin{itemize}
\item {Grp. gram.:v. t.}
\end{itemize}
Pôr cola em.
Preparar, cobrindo com cola.
\section{Encolar}
\begin{itemize}
\item {Grp. gram.:v. t.}
\end{itemize}
Trazer ao colo (uma criança)
Amimar.
\section{Encoleirar}
\begin{itemize}
\item {Grp. gram.:v. t.}
\end{itemize}
Pôr coleira a.
\section{Encolerizado}
\begin{itemize}
\item {Grp. gram.:adj.}
\end{itemize}
\begin{itemize}
\item {Proveniência:(De \textunderscore encolerizar\textunderscore )}
\end{itemize}
Cheio de cólera.
Raivoso, furioso.
\section{Encolerizar}
\begin{itemize}
\item {Grp. gram.:v. t.}
\end{itemize}
Causar cólera a.
Irritar.
\section{Encoletado}
\begin{itemize}
\item {Grp. gram.:adj.}
\end{itemize}
\begin{itemize}
\item {Utilização:Bras. do N}
\end{itemize}
Que traz colete.
Vestido com apuro.
\section{Encolha}
\begin{itemize}
\item {fónica:có}
\end{itemize}
\begin{itemize}
\item {Grp. gram.:f.}
\end{itemize}
\begin{itemize}
\item {Proveniência:(De \textunderscore encolher\textunderscore )}
\end{itemize}
Acanhamento.
Timidez; encolhimento.
\section{Encolhedela}
\begin{itemize}
\item {Grp. gram.:f.}
\end{itemize}
\begin{itemize}
\item {Utilização:Fam.}
\end{itemize}
O mesmo que encolhimento. Cf. Arn. Gama, \textunderscore Segr. do Abb.\textunderscore , 74 e 126.
\section{Encolher}
\begin{itemize}
\item {Grp. gram.:v. t.}
\end{itemize}
\begin{itemize}
\item {Utilização:Fig.}
\end{itemize}
\begin{itemize}
\item {Grp. gram.:V. p.}
\end{itemize}
\begin{itemize}
\item {Proveniência:(De \textunderscore colhêr\textunderscore )}
\end{itemize}
Reduzir; deminuir: \textunderscore encolher despesas\textunderscore .
Tornar curto, contrahindo: \textunderscore encolher as garras\textunderscore .
Contrahir: \textunderscore a humidade encolheu a corda\textunderscore .
Refrear.
Conter.
Acanhar, tornar tímido.
\textunderscore Encolher os ombros\textunderscore , mostrar indifferença.
Mostrar-se resignado.
Têr timidez.
Sêr parco em despesas.
Esconder-se.
Contrahir-se.
Retrahir-se.
\section{Encolhidamente}
\begin{itemize}
\item {Grp. gram.:adv.}
\end{itemize}
Com encolhimento.
\section{Encolhido}
\begin{itemize}
\item {Grp. gram.:m.}
\end{itemize}
Aquelle que se encolhe, que é timido, que não tem energia.
\section{Encolhimento}
\begin{itemize}
\item {Grp. gram.:m.}
\end{itemize}
Acto ou effeito de encolher.
Timidez, acanhamento.
\section{Encollamento}
\begin{itemize}
\item {Grp. gram.:m.}
\end{itemize}
Acto ou effeito de encollar.
\section{Encollar}
\begin{itemize}
\item {Grp. gram.:v. t.}
\end{itemize}
Pôr colla em.
Preparar, cobrindo com colla.
\section{Encollar}
\begin{itemize}
\item {Grp. gram.:v. t.}
\end{itemize}
Trazer ao collo (uma criança).
Amimar.
\section{Encolleirar}
\begin{itemize}
\item {Grp. gram.:v. t.}
\end{itemize}
Pôr colleira a.
\section{Encolletado}
\begin{itemize}
\item {Grp. gram.:adj.}
\end{itemize}
\begin{itemize}
\item {Utilização:Bras. do N}
\end{itemize}
Que traz collete.
Vestido com apuro.
\section{Encólpio}
\begin{itemize}
\item {Grp. gram.:m.}
\end{itemize}
\begin{itemize}
\item {Utilização:Ant.}
\end{itemize}
Pequeno relicário, para se trazer ao pescoço.
\section{Encomenda}
\begin{itemize}
\item {Grp. gram.:f.}
\end{itemize}
Aquilo que se encomenda: \textunderscore entreguei-lhe a encomenda\textunderscore .
Acto de encomendar.
\section{Encomendação}
\begin{itemize}
\item {Grp. gram.:f.}
\end{itemize}
Acto de encomendar.
\section{Encomendado}
\begin{itemize}
\item {Grp. gram.:m.}
\end{itemize}
Pároco por encomendação, não colado, amovível.
\section{Encomendamento}
\begin{itemize}
\item {Grp. gram.:m.}
\end{itemize}
\begin{itemize}
\item {Utilização:Ant.}
\end{itemize}
(V.encomendação)
\section{Encomendar}
\begin{itemize}
\item {Grp. gram.:v. t.}
\end{itemize}
\begin{itemize}
\item {Proveniência:(Do lat. \textunderscore commendare\textunderscore )}
\end{itemize}
Incumbir: \textunderscore encomendei-lhe a compra de um relógio\textunderscore .
Recomendar.
Ordenar.
Confiar: \textunderscore encomendo-lhe o pequeno\textunderscore .
Fazer nomeação provisória de (um pároco).
Rezar por alma de: \textunderscore encomendar defuntos\textunderscore .
\section{Encomendeiro}
\begin{itemize}
\item {Grp. gram.:m.}
\end{itemize}
\begin{itemize}
\item {Proveniência:(De \textunderscore encomendar\textunderscore )}
\end{itemize}
Indivíduo, a quem se encomenda ou ordena que traga ou faça alguma coisa.
Recoveiro.
Comissário.
\section{Encomiador}
\begin{itemize}
\item {Grp. gram.:m.}
\end{itemize}
Aquelle que encomia.
\section{Encomiar}
\begin{itemize}
\item {Grp. gram.:v. t.}
\end{itemize}
\begin{itemize}
\item {Proveniência:(De \textunderscore encómio\textunderscore )}
\end{itemize}
Gabar.
Elogiar.
Dirigir encómios a.
\section{Encomiasta}
\begin{itemize}
\item {Grp. gram.:m.}
\end{itemize}
\begin{itemize}
\item {Proveniência:(Gr. \textunderscore enkomiastes\textunderscore )}
\end{itemize}
Aquelle que faz panegýricos.
Aquelle que escreve ou discorre, fazendo o elogio de alguém ou de alguma coisa.
\section{Encomiástico}
\begin{itemize}
\item {Grp. gram.:adj.}
\end{itemize}
\begin{itemize}
\item {Proveniência:(Gr. \textunderscore enkomiastikos\textunderscore )}
\end{itemize}
Laudatório.
Em que há encómio: \textunderscore noticia encomiástica\textunderscore .
Relativo a encómio.
\section{Encómio}
\begin{itemize}
\item {Grp. gram.:m.}
\end{itemize}
\begin{itemize}
\item {Proveniência:(Gr. \textunderscore encomion\textunderscore )}
\end{itemize}
Gabo, applauso, louvor, elogio.
\section{Encomissar}
\begin{itemize}
\item {Grp. gram.:v. i.  e  p.}
\end{itemize}
Cair em comisso.
\section{Encommenda}
\begin{itemize}
\item {Grp. gram.:f.}
\end{itemize}
Aquillo que se encommenda: \textunderscore entreguei-lhe a encommenda\textunderscore .
Acto de encommendar.
\section{Encommendação}
\begin{itemize}
\item {Grp. gram.:f.}
\end{itemize}
Acto de encommendar.
\section{Encommendado}
\begin{itemize}
\item {Grp. gram.:m.}
\end{itemize}
Párocho por encommendação, não collado, amovível.
\section{Encommendamento}
\begin{itemize}
\item {Grp. gram.:m.}
\end{itemize}
\begin{itemize}
\item {Utilização:Ant.}
\end{itemize}
(V.encommendação)
\section{Encommendar}
\begin{itemize}
\item {Grp. gram.:v. t.}
\end{itemize}
\begin{itemize}
\item {Proveniência:(Do lat. \textunderscore commendare\textunderscore )}
\end{itemize}
Incumbir: \textunderscore encommendei-lhe a compra de um relógio\textunderscore .
Recommendar.
Ordenar.
Confiar: \textunderscore encommendo-lhe o pequeno\textunderscore .
Fazer nomeação provisória de (um párocho).
Rezar por alma de: \textunderscore encommendar defuntos\textunderscore .
\section{Encommendeiro}
\begin{itemize}
\item {Grp. gram.:m.}
\end{itemize}
\begin{itemize}
\item {Proveniência:(De \textunderscore encommendar\textunderscore )}
\end{itemize}
Indivíduo, a quem se encommenda ou ordena que traga ou faça alguma coisa.
Recoveiro.
Commissário.
\section{Encommissar}
\begin{itemize}
\item {Grp. gram.:v. i.  e  p.}
\end{itemize}
Cair em commisso.
\section{Encomoroçar}
\textunderscore v. t.\textunderscore  (e der.)
O mesmo que \textunderscore encomoroiçar\textunderscore .
\section{Encomoroiçar}
\begin{itemize}
\item {Grp. gram.:v. t.}
\end{itemize}
\begin{itemize}
\item {Proveniência:(De \textunderscore cômoro\textunderscore )}
\end{itemize}
Pôr em cômoro, pôr alto.
Amontoar.--É a fórma correcta de \textunderscore encramoiçar\textunderscore .
V. \textunderscore encramoiçar\textunderscore .
\section{Encompridar}
\begin{itemize}
\item {Grp. gram.:v. t.}
\end{itemize}
\begin{itemize}
\item {Utilização:Bras}
\end{itemize}
Tornar mais comprido.
\section{Enconapar}
\begin{itemize}
\item {Grp. gram.:v. t.}
\end{itemize}
\begin{itemize}
\item {Utilização:Prov.}
\end{itemize}
\begin{itemize}
\item {Utilização:beir.}
\end{itemize}
Cerzir mal, remendar grosseiramente.
\section{Enconcar}
\begin{itemize}
\item {Grp. gram.:v. i.  e  p.}
\end{itemize}
\begin{itemize}
\item {Grp. gram.:V. t.}
\end{itemize}
\begin{itemize}
\item {Proveniência:(De \textunderscore conca\textunderscore )}
\end{itemize}
Fazer-se côncavo.
Encurvar-se.
Tomar fórma de telha ou de conca.
Dar fórma de telha ou de conca a.
\section{Enconchar}
\begin{itemize}
\item {Grp. gram.:v. t.}
\end{itemize}
\begin{itemize}
\item {Grp. gram.:V. p.}
\end{itemize}
Cobrir com concha.
Meter em concha.
Encolher-se.
Desviar-se do trato social.
\section{Enconchousado}
\begin{itemize}
\item {Grp. gram.:adj.}
\end{itemize}
\begin{itemize}
\item {Utilização:Ant.}
\end{itemize}
Dizia-se do quintal ou terreno fechado por todos os lados.
(Cp. \textunderscore chouso\textunderscore )
\section{Encondar}
\begin{itemize}
\item {Grp. gram.:v. t.}
\end{itemize}
\begin{itemize}
\item {Utilização:bras}
\end{itemize}
\begin{itemize}
\item {Utilização:Neol.}
\end{itemize}
Fazer conde.
\section{Encondensado}
\begin{itemize}
\item {Grp. gram.:adj.}
\end{itemize}
O mesmo que [[condensado|condensar]]. Cf. Filinto, VI, 169.
\section{Encondroma}
\begin{itemize}
\item {Grp. gram.:m.}
\end{itemize}
\begin{itemize}
\item {Utilização:Med.}
\end{itemize}
\begin{itemize}
\item {Proveniência:(Do gr. \textunderscore en\textunderscore  + \textunderscore khondros\textunderscore )}
\end{itemize}
Tumor cartilaginoso.
\section{Enconicar}
\begin{itemize}
\item {Grp. gram.:v. t.}
\end{itemize}
\begin{itemize}
\item {Utilização:Prov.}
\end{itemize}
\begin{itemize}
\item {Utilização:minh.}
\end{itemize}
\begin{itemize}
\item {Utilização:trasm.}
\end{itemize}
O mesmo que \textunderscore enconapar\textunderscore .
\section{Encontrada}
\begin{itemize}
\item {Grp. gram.:f.}
\end{itemize}
(V.encontrão)
\section{Encontradiço}
\begin{itemize}
\item {Grp. gram.:adj.}
\end{itemize}
Que se encontra casualmente ou frequentemente.
\textunderscore Fazer-se encontradiço\textunderscore , encontrar-se propositadamente.
\section{Encontramento}
\begin{itemize}
\item {Grp. gram.:m.}
\end{itemize}
\begin{itemize}
\item {Utilização:T. de Ceilão}
\end{itemize}
O mesmo que \textunderscore encontro\textunderscore ^1.
\section{Encontrão}
\begin{itemize}
\item {Grp. gram.:m.}
\end{itemize}
\begin{itemize}
\item {Proveniência:(De \textunderscore encontro\textunderscore )}
\end{itemize}
Embate de pessôas que se encontram.
Empurrão.
\section{Encontrar}
\begin{itemize}
\item {Grp. gram.:v. t.}
\end{itemize}
\begin{itemize}
\item {Utilização:Fig.}
\end{itemize}
\begin{itemize}
\item {Grp. gram.:V. p.}
\end{itemize}
\begin{itemize}
\item {Proveniência:(De \textunderscore contra\textunderscore )}
\end{itemize}
Ir de encontro a.
Esbarrar com.
Topar: \textunderscore encontrei o teu Carlos\textunderscore .
Achar, descobrir: \textunderscore encontrar um thesoiro\textunderscore .
Compensar.
Fazer opposição a.
\textunderscore V. i.\textunderscore  (Seguido da prep. \textunderscore com\textunderscore )
Dar de frente.
Ir ao encontro.
Embater-se, chocar-se.
Achar-se.
Ir de encontro.
Sêr idêntico.
Têr o mesmo parecer.
Bater-se em duello.
Disputar; oppor-se.
\section{Encontro}
\begin{itemize}
\item {Grp. gram.:m.}
\end{itemize}
\begin{itemize}
\item {Grp. gram.:Pl.}
\end{itemize}
\begin{itemize}
\item {Utilização:Fam.}
\end{itemize}
\begin{itemize}
\item {Utilização:Bras. do S}
\end{itemize}
Acto de encontrar.
Lugar, em que alguém se encontra com outrem.
Encontrão.
Briga, recontro.
Massiços, em que se apoiam arcos extremos de uma ponte.
Ombros.
Parte superior das asas, donde nascem as pennas maiores.
Peito do animal, entre as espáduas.
\section{Encontro}
\begin{itemize}
\item {Grp. gram.:m.}
\end{itemize}
Nome de uma ave brasileira, também chamada \textunderscore soldado\textunderscore  e \textunderscore nhapim\textunderscore .
\section{Encontroar}
\begin{itemize}
\item {Grp. gram.:v. t.}
\end{itemize}
\begin{itemize}
\item {Grp. gram.:V. p.}
\end{itemize}
\begin{itemize}
\item {Proveniência:(De \textunderscore encontrão\textunderscore )}
\end{itemize}
Dar encontrões a.
Andar aos encontrões.
Atropelar-se.
\section{Enciclia}
\begin{itemize}
\item {Grp. gram.:f.}
\end{itemize}
\begin{itemize}
\item {Proveniência:(Do gr. \textunderscore kuklos\textunderscore )}
\end{itemize}
Ondulação circular, produzida na superfície da água pela quéda de um corpo.
\section{Encíclica}
\begin{itemize}
\item {Grp. gram.:f.}
\end{itemize}
\begin{itemize}
\item {Proveniência:(De \textunderscore encíclico\textunderscore )}
\end{itemize}
Carta circular pontificia, dogmática ou doutrinária.
\section{Encíclico}
\begin{itemize}
\item {Grp. gram.:adj.}
\end{itemize}
Circular.
Diz-se das cartas circulares dos Pontífices.
(Cp. \textunderscore enciclia\textunderscore )
\section{Enciclopedia}
\begin{itemize}
\item {Grp. gram.:f.}
\end{itemize}
\begin{itemize}
\item {Proveniência:(Do gr. \textunderscore enkuklopaideia\textunderscore )}
\end{itemize}
Conjunto de conhecimentos, relativos a todas as ciências e artes, ou relativos a uma classe ou domínio de ciências e artes.
\section{Enciclopédico}
\begin{itemize}
\item {Grp. gram.:adj.}
\end{itemize}
Relativo a enciclopedia.
Que abrange todos os ramos do saber humano: \textunderscore dicionário enciclopédico\textunderscore .
Que possue conhecimentos vastos ou concernentes a todas as ciências.
\section{Enciclopedismo}
\begin{itemize}
\item {Grp. gram.:m.}
\end{itemize}
\begin{itemize}
\item {Proveniência:(De \textunderscore enciclopedia\textunderscore )}
\end{itemize}
Sistema dos enciclopedistas.
\section{Enciclopedista}
\begin{itemize}
\item {Grp. gram.:m.}
\end{itemize}
\begin{itemize}
\item {Proveniência:(De \textunderscore enciclopedia\textunderscore )}
\end{itemize}
Escritor ou colaborador de obra enciclopédica.
Partidário das doutrinas da \textunderscore Enciclopedia\textunderscore  de D'Alembert e Diderot.
\section{Enciclopedístico}
\begin{itemize}
\item {Grp. gram.:adj.}
\end{itemize}
Relativo aos enciclopedistas franceses do século XVIII.
\section{Enciprótipo}
\begin{itemize}
\item {Grp. gram.:adj.}
\end{itemize}
\begin{itemize}
\item {Proveniência:(Do gr. \textunderscore en\textunderscore  + \textunderscore cupros\textunderscore  + \textunderscore tupos\textunderscore )}
\end{itemize}
Gravado imediatamente em cobre.
\section{Encopar}
\begin{itemize}
\item {Grp. gram.:v. t.}
\end{itemize}
\begin{itemize}
\item {Proveniência:(De \textunderscore copa\textunderscore )}
\end{itemize}
O mesmo que \textunderscore copar\textunderscore  e \textunderscore enfunar\textunderscore .
\section{Encoquinar}
\begin{itemize}
\item {Grp. gram.:v. t.}
\end{itemize}
\begin{itemize}
\item {Utilização:Ext.}
\end{itemize}
\begin{itemize}
\item {Proveniência:(Do lat. \textunderscore coquina\textunderscore )}
\end{itemize}
Meter na cozinha.
Occultar; meter em esconderijo.
\section{Encoquinhar}
\begin{itemize}
\item {Grp. gram.:v. t.}
\end{itemize}
(V.encoquinar)
\section{Encora}
\begin{itemize}
\item {Grp. gram.:f.}
\end{itemize}
O mesmo que \textunderscore encoramento\textunderscore .
\section{Encorajamento}
\begin{itemize}
\item {Grp. gram.:m.}
\end{itemize}
Acto de encorajar.
\section{Encorajar}
\begin{itemize}
\item {Grp. gram.:v. t.}
\end{itemize}
Dar coragem a; animar.
Estimular.--Há quem duvide da vernaculidade do termo, mas parece-me bem derivado e legítimo.
\section{Encoramento}
\begin{itemize}
\item {Grp. gram.:m.}
\end{itemize}
Acto de encorar.
\section{Encorar}
\begin{itemize}
\item {Grp. gram.:v. t.}
\end{itemize}
\begin{itemize}
\item {Utilização:Prov.}
\end{itemize}
Empoçar ou represar (água) em tanque, valla, rêgo, etc.
\section{Encordoação}
\begin{itemize}
\item {Grp. gram.:f.}
\end{itemize}
Acto de encordoar instrumentos.
\section{Encordoadura}
\begin{itemize}
\item {Grp. gram.:f.}
\end{itemize}
\begin{itemize}
\item {Proveniência:(De \textunderscore encordoar\textunderscore )}
\end{itemize}
Conjunto das cordas, que guarnecem um instrumento.
\section{Encordoamento}
\begin{itemize}
\item {Grp. gram.:m.}
\end{itemize}
Acto de encordoar.
\section{Encordoar}
\begin{itemize}
\item {Grp. gram.:v. t.}
\end{itemize}
\begin{itemize}
\item {Utilização:Prov.}
\end{itemize}
\begin{itemize}
\item {Utilização:alg.}
\end{itemize}
\begin{itemize}
\item {Grp. gram.:V. i.}
\end{itemize}
\begin{itemize}
\item {Utilização:Fam.}
\end{itemize}
\begin{itemize}
\item {Proveniência:(De \textunderscore cordão\textunderscore )}
\end{itemize}
Pôr cordas ou cordões em: \textunderscore encordoar a viola\textunderscore .
Dispor em renques parallelos (moreias de mato).
Mostrar-se desconfiado; encavacar.
\section{Encornar-se}
\begin{itemize}
\item {Grp. gram.:v. p.}
\end{itemize}
\begin{itemize}
\item {Proveniência:(De \textunderscore côrno\textunderscore )}
\end{itemize}
Sêr colhido entre as hastes do toiro.
\section{Encornelhar}
\begin{itemize}
\item {Grp. gram.:v. t.}
\end{itemize}
\begin{itemize}
\item {Utilização:Ant.}
\end{itemize}
\begin{itemize}
\item {Proveniência:(Do rad. de \textunderscore cornilho\textunderscore ?)}
\end{itemize}
Desprezar.
Infamar.
\section{Encoro}
\begin{itemize}
\item {Grp. gram.:m.}
\end{itemize}
\begin{itemize}
\item {Utilização:Prov.}
\end{itemize}
Acto de encorar.
\section{Encoroçado}
\begin{itemize}
\item {Grp. gram.:adj.}
\end{itemize}
\begin{itemize}
\item {Utilização:Ant.}
\end{itemize}
\begin{itemize}
\item {Proveniência:(De \textunderscore coroça\textunderscore )}
\end{itemize}
Que tem jurisdicção abusiva, (falando-se do abbade, que usa báculo episcopal).
\section{Encoronhado}
\begin{itemize}
\item {Grp. gram.:adj.}
\end{itemize}
Doente ou defeituoso dos cascos, (falando-se de equídeos).
\section{Encoronhar}
\begin{itemize}
\item {Grp. gram.:v. t.}
\end{itemize}
Pôr coronha em.
\section{Encorpado}
\begin{itemize}
\item {Grp. gram.:adj.}
\end{itemize}
\begin{itemize}
\item {Proveniência:(De \textunderscore encorpar\textunderscore )}
\end{itemize}
Corpulento; bem apessoado.
Muito crescido.
Alto e grosso.
\section{Encorpadura}
\begin{itemize}
\item {Grp. gram.:f.}
\end{itemize}
Qualidade daquillo ou daquelle que é encorpado.
\section{Encorpamento}
\begin{itemize}
\item {Grp. gram.:m.}
\end{itemize}
Qualidade daquillo ou daquelle que é encorpado.
\section{Encorpar}
\begin{itemize}
\item {Grp. gram.:v. t.}
\end{itemize}
\begin{itemize}
\item {Grp. gram.:V. i.  e  p.}
\end{itemize}
Tornar grosso, fazer maior.
Dar mais corpo a.
Criar corpulência; engrossar.
Crescer.
\section{Encorporação}
\begin{itemize}
\item {Grp. gram.:f.}
\end{itemize}
Acto ou effeito de encorporar.
Qualidade ou estado de coisas encorporadas.
\section{Encorporante}
\begin{itemize}
\item {Grp. gram.:adj.}
\end{itemize}
\begin{itemize}
\item {Proveniência:(Lat. \textunderscore incorporans\textunderscore )}
\end{itemize}
Que encorpora.
\section{Encorporar}
\begin{itemize}
\item {Grp. gram.:v. t.}
\end{itemize}
\begin{itemize}
\item {Grp. gram.:V. i.}
\end{itemize}
\begin{itemize}
\item {Proveniência:(Lat. \textunderscore incorporare\textunderscore )}
\end{itemize}
Juntar num só corpo: \textunderscore encorporar duas associações\textunderscore .
Dar fórma de corpo a; dar fórma ou volume a.
Tomar corpo.
\section{Encorporativo}
\begin{itemize}
\item {Grp. gram.:adj.}
\end{itemize}
O mesmo que \textunderscore holophrástico\textunderscore .
\section{Encorquilhar}
\begin{itemize}
\item {Grp. gram.:v. t.}
\end{itemize}
\begin{itemize}
\item {Utilização:Prov.}
\end{itemize}
O mesmo que \textunderscore encarquilhar\textunderscore .
\section{Encorreadura}
\begin{itemize}
\item {Grp. gram.:f.}
\end{itemize}
\begin{itemize}
\item {Proveniência:(De \textunderscore encorrear\textunderscore )}
\end{itemize}
Armadura de correia.
Conjunto de correias para certo fim.
\section{Encorreamento}
\begin{itemize}
\item {Grp. gram.:m.}
\end{itemize}
Acto de encorrear.
\section{Encorrear}
\begin{itemize}
\item {Grp. gram.:v. t.}
\end{itemize}
\begin{itemize}
\item {Grp. gram.:V. i.}
\end{itemize}
Ligar com correia.
Tomar a apparência ou consistência de coiro.
Enrugar-se, á semelhança do coiro sujeito á acção do fogo.
\section{Encorrilha}
\begin{itemize}
\item {Grp. gram.:f.}
\end{itemize}
Acto ou effeito de encorrilhar^2.
Vinco, prega.
Dobra.
\section{Encorrilhar}
\begin{itemize}
\item {Grp. gram.:v. t.}
\end{itemize}
Incluir em corrilho.
\section{Encorrilhar}
\begin{itemize}
\item {Grp. gram.:v. t.}
\end{itemize}
\begin{itemize}
\item {Utilização:Prov.}
\end{itemize}
\begin{itemize}
\item {Utilização:minh.}
\end{itemize}
Encarquilhar, engelhar.
Tornar murcho.
\section{Encorrugido}
\begin{itemize}
\item {Grp. gram.:adj.}
\end{itemize}
\begin{itemize}
\item {Utilização:Bras}
\end{itemize}
\begin{itemize}
\item {Proveniência:(De \textunderscore encorrugir\textunderscore )}
\end{itemize}
Engelhado, encarquilhado.
Encolhido.
\section{Encorrugir}
\begin{itemize}
\item {Grp. gram.:v. t.}
\end{itemize}
\begin{itemize}
\item {Utilização:Bras}
\end{itemize}
Encarquilhar, enrugar.
Entanguir.
\section{Encortelhar}
\begin{itemize}
\item {Grp. gram.:v. t.}
\end{itemize}
Meter em cortelho; encurralar.
\section{Encortiçar}
\begin{itemize}
\item {Grp. gram.:v. t.}
\end{itemize}
\begin{itemize}
\item {Grp. gram.:V. i.  e  p.}
\end{itemize}
Meter em cortiço.
Cobrir com casca de árvores ou com cortiça.
Dar apparência de cortiça a.
Criar cortiça ou casca.
Tomar apparência de cortiça.
\section{Encortinar}
\begin{itemize}
\item {Grp. gram.:v. t.}
\end{itemize}
\begin{itemize}
\item {Utilização:P. us.}
\end{itemize}
Pôr cortinas em. Cf. \textunderscore Auto de Santo Aleixo\textunderscore , cit. por Castilho.
\section{Encorujar-se}
\begin{itemize}
\item {Grp. gram.:v. p.}
\end{itemize}
\begin{itemize}
\item {Utilização:Bras}
\end{itemize}
\begin{itemize}
\item {Proveniência:(De \textunderscore coruja\textunderscore )}
\end{itemize}
Retrahir-se, furtando-se ás vistas do público.
Embiocar-se.
Amochoir-se.
\section{Encosamentos}
\begin{itemize}
\item {Grp. gram.:m. pl.}
\end{itemize}
O mesmo que [[encalamentos|encalamento]].
\section{Encoscoramento}
\begin{itemize}
\item {Grp. gram.:m.}
\end{itemize}
Acto de encoscorar.
\section{Encoscorar}
\begin{itemize}
\item {Grp. gram.:v. t.}
\end{itemize}
\begin{itemize}
\item {Grp. gram.:V. i.  e  p.}
\end{itemize}
\begin{itemize}
\item {Proveniência:(De \textunderscore cóscoro\textunderscore )}
\end{itemize}
Encarquilhar, encrespar.
Tornar duro como o coscorão.
Encrespar-se, criar cóscoros.
\section{Encospas}
\begin{itemize}
\item {Grp. gram.:f. pl.}
\end{itemize}
\begin{itemize}
\item {Grp. gram.:Loc.}
\end{itemize}
\begin{itemize}
\item {Utilização:fam.}
\end{itemize}
\begin{itemize}
\item {Proveniência:(Do lat. \textunderscore cuspis\textunderscore )}
\end{itemize}
Fôrmas de madeira, com que os sapateiros alargam o calçado.
\textunderscore Meter-se nas encóspias\textunderscore , retrahir-se, esquivar-se á convivência; não dar satisfações.
\section{Encóspias}
\begin{itemize}
\item {Grp. gram.:f. pl.}
\end{itemize}
\begin{itemize}
\item {Grp. gram.:Loc.}
\end{itemize}
\begin{itemize}
\item {Utilização:fam.}
\end{itemize}
\begin{itemize}
\item {Proveniência:(Do lat. \textunderscore cuspis\textunderscore )}
\end{itemize}
Fôrmas de madeira, com que os sapateiros alargam o calçado
\textunderscore Meter-se nas encóspias\textunderscore , retrahir-se, esquivar-se á convivência; não dar satisfações.
\section{Encosta}
\begin{itemize}
\item {Grp. gram.:f.}
\end{itemize}
\begin{itemize}
\item {Proveniência:(De \textunderscore costa\textunderscore )}
\end{itemize}
Declive; rampa; vertente.
\section{Encostadela}
\begin{itemize}
\item {Grp. gram.:f.}
\end{itemize}
\begin{itemize}
\item {Utilização:Pop.}
\end{itemize}
\begin{itemize}
\item {Proveniência:(De \textunderscore encostar\textunderscore )}
\end{itemize}
Acto de importunar, solicitando favores ou dinheiro.
\section{Encostado}
\begin{itemize}
\item {Grp. gram.:adj.}
\end{itemize}
\begin{itemize}
\item {Utilização:Pop.}
\end{itemize}
\begin{itemize}
\item {Grp. gram.:M.}
\end{itemize}
\begin{itemize}
\item {Proveniência:(De \textunderscore encostar\textunderscore )}
\end{itemize}
Arrimado.
Apoiado.
Importunado com pedidos, de dinheiro sobretudo.
Individuo protegido ou que vive á custa de outrem:«\textunderscore ...peço-lhe que as reparta com os encostados mais pobrezinhos do convento\textunderscore ». Camillo, \textunderscore Filha do Reg.\textunderscore , 242.
\section{Encostador}
\begin{itemize}
\item {Grp. gram.:m.  e  adj.}
\end{itemize}
\begin{itemize}
\item {Utilização:Pop.}
\end{itemize}
\begin{itemize}
\item {Proveniência:(De \textunderscore encostar\textunderscore )}
\end{itemize}
O que faz encostadelas.
\section{Encostalar}
\begin{itemize}
\item {Grp. gram.:v. t.}
\end{itemize}
Pôr em costal, enfardelar.
\section{Encostamento}
\begin{itemize}
\item {Grp. gram.:m.}
\end{itemize}
Acto ou effeito de encostar.
\section{Encostar}
\begin{itemize}
\item {Grp. gram.:v. t.}
\end{itemize}
\begin{itemize}
\item {Utilização:Pop.}
\end{itemize}
\begin{itemize}
\item {Grp. gram.:V. p.}
\end{itemize}
\begin{itemize}
\item {Proveniência:(De \textunderscore costa\textunderscore )}
\end{itemize}
Arrimar.
Apoiar, firmar.
Pôr contra.
Incommodar com encostadelas; pedir dinheiro a.
Incumbir (outrem) do trabalho que nos pertence.
Apoiar as costas: \textunderscore encostar-se á parede\textunderscore .
Apoiar-se.
Deitar-se.
Amparar-se; procurar a protecção de alguém.
\section{Encostelado}
\begin{itemize}
\item {Grp. gram.:adj.}
\end{itemize}
Que tem costelas:«\textunderscore o encostelado lombo\textunderscore ». Filinto, III, 136.
\section{Encostellado}
\begin{itemize}
\item {Grp. gram.:adj.}
\end{itemize}
Que tem costellas:«\textunderscore o encostellado lombo\textunderscore ». Filinto, III, 136.
\section{Encostes}
\begin{itemize}
\item {Grp. gram.:m. pl.}
\end{itemize}
\begin{itemize}
\item {Utilização:Fig.}
\end{itemize}
\begin{itemize}
\item {Proveniência:(De \textunderscore encostar\textunderscore )}
\end{itemize}
Contrafortes, em construcções.
Sustentáculos de um arco.
Protecção.
\section{Encôsto}
\begin{itemize}
\item {Grp. gram.:m.}
\end{itemize}
\begin{itemize}
\item {Utilização:Fig.}
\end{itemize}
\begin{itemize}
\item {Proveniência:(De \textunderscore encostar\textunderscore )}
\end{itemize}
Lugar ou peça, a que alguém ou alguma coisa se encosta: \textunderscore o encôsto de uma cadeira\textunderscore .
O mesmo que \textunderscore encosta\textunderscore . Cf. Garrett, \textunderscore Romanceiro\textunderscore , 1, 35.
Protecção, arrimo; apoio.
\section{Encostrar}
\begin{itemize}
\item {Grp. gram.:v. t.}
\end{itemize}
(V.encrostar). Cf. Castilho, \textunderscore Metam.\textunderscore , 156.
\section{Encouchado}
\begin{itemize}
\item {Grp. gram.:adj.}
\end{itemize}
\begin{itemize}
\item {Utilização:Prov.}
\end{itemize}
\begin{itemize}
\item {Utilização:trasm.}
\end{itemize}
\begin{itemize}
\item {Proveniência:(De \textunderscore encouchar\textunderscore )}
\end{itemize}
Enfèzado, rachítico.
\section{Encouchar}
\begin{itemize}
\item {Grp. gram.:v. t.}
\end{itemize}
Curvar.
Tornar encolhido; acanhar.
Humilhar; deprimir.
\section{Encourado}
\begin{itemize}
\item {Grp. gram.:adj.}
\end{itemize}
\begin{itemize}
\item {Utilização:Bras. do N}
\end{itemize}
\begin{itemize}
\item {Grp. gram.:M.}
\end{itemize}
\begin{itemize}
\item {Utilização:Bras. do N}
\end{itemize}
Vestido de couro, como os vaqueiros do sertão.
Aquelle que usa roupa de couro, como os vaqueiros do sertão.
\section{Encourar}
\begin{itemize}
\item {Grp. gram.:v. t.}
\end{itemize}
\begin{itemize}
\item {Grp. gram.:V. i.  e  p.}
\end{itemize}
\begin{itemize}
\item {Proveniência:(Do b. lat. \textunderscore incoriare\textunderscore )}
\end{itemize}
Revestir de couro.
Criar nova pelle (uma ferida).
\section{Encovar}
\begin{itemize}
\item {Grp. gram.:v. t.}
\end{itemize}
\begin{itemize}
\item {Utilização:Fig.}
\end{itemize}
\begin{itemize}
\item {Grp. gram.:V. i.}
\end{itemize}
Meter em cova.
Esconder.
Enterrar.
Obrigar a dar-se por convencido.
Embatucar; não saber replicar; fugir da discussão.
\section{Encovilar}
\begin{itemize}
\item {Grp. gram.:v. t.}
\end{itemize}
Meter em covil.
\section{Encramoiçar}
\begin{itemize}
\item {Grp. gram.:v. t.}
\end{itemize}
\begin{itemize}
\item {Utilização:Prov.}
\end{itemize}
Juntar em cramoiços; acumular.
\section{Encrassar}
\begin{itemize}
\item {Grp. gram.:v. i.}
\end{itemize}
\begin{itemize}
\item {Grp. gram.:v. t.}
\end{itemize}
\begin{itemize}
\item {Proveniência:(Lat. \textunderscore incrassare\textunderscore )}
\end{itemize}
Tornar-se crasso, denso.
Tornar crasso, gordo.
Engrossar.
\section{Encrava}
\begin{itemize}
\item {Grp. gram.:f.}
\end{itemize}
O mesmo que \textunderscore encravamento\textunderscore .
\section{Encravação}
\begin{itemize}
\item {Grp. gram.:f.}
\end{itemize}
O mesmo que \textunderscore encravamento\textunderscore .
\section{Encravadoiro}
\begin{itemize}
\item {Grp. gram.:m.}
\end{itemize}
Lugar, em que se encrava uma coisa:«\textunderscore o mastro da ré parecia saltar fóra do encravadoiro\textunderscore ». Camillo, \textunderscore Myst. de Lisb.\textunderscore , II, 231.
\section{Encravadouro}
\begin{itemize}
\item {Grp. gram.:m.}
\end{itemize}
Lugar, em que se encrava uma coisa:«\textunderscore o mastro da ré parecia saltar fóra do encravadouro\textunderscore ». Camillo, \textunderscore Myst. de Lisb.\textunderscore , II, 231.
\section{Encravadura}
\begin{itemize}
\item {Grp. gram.:f.}
\end{itemize}
Cravos de ferradura.
Ferimento, produzido pelos cravos da ferradura.
Acto de encravar.
\section{Encravamento}
\begin{itemize}
\item {Grp. gram.:m.}
\end{itemize}
Acto ou effeito de encravar.
Estado daquillo que está encravado.
\section{Encravar}
\begin{itemize}
\item {Grp. gram.:v. t.}
\end{itemize}
\begin{itemize}
\item {Utilização:Fig.}
\end{itemize}
\begin{itemize}
\item {Proveniência:(De \textunderscore cravo\textunderscore )}
\end{itemize}
Segurar com prego ou cravo.
Pregar.
Espetar.
Ferir ou magoar com os cravos (o pé da bêsta).
Embutir, engastar: \textunderscore encravar pedras preciosas\textunderscore .
Embeber.
Meter prego no ouvido de (uma peça de artilharia).
Tirar ou inutilizar a substância explosiva da espoleta de (uma arma carregada).
Embair, enganar.
Encravilhar; comprometer.
Vencer na discussão.
Collocar em meio de.
\section{Encravelhação}
\begin{itemize}
\item {Grp. gram.:f.}
\end{itemize}
Acto ou effeito de encravelhar.
\section{Encravelhar}
\begin{itemize}
\item {Grp. gram.:v. t.}
\end{itemize}
\begin{itemize}
\item {Utilização:Pop.}
\end{itemize}
\begin{itemize}
\item {Proveniência:(De \textunderscore cravelha\textunderscore )}
\end{itemize}
Collocar em posição diffícil; embaraçar.
Entalar.
Encalacrar.
\section{Encravilhar}
\textunderscore v. i.\textunderscore  (e der.)
(Corr. de \textunderscore encravelhar\textunderscore , etc.)
\section{Encravo}
\begin{itemize}
\item {Grp. gram.:m.}
\end{itemize}
\begin{itemize}
\item {Proveniência:(De \textunderscore encravar\textunderscore )}
\end{itemize}
Ferimento, produzido pelo cravo da ferradura.
Encravadura; encravamento.
\section{Encrenca}
\begin{itemize}
\item {Grp. gram.:f.}
\end{itemize}
\begin{itemize}
\item {Utilização:Bras. do N}
\end{itemize}
\begin{itemize}
\item {Proveniência:(De \textunderscore encrencar\textunderscore )}
\end{itemize}
Difficuldade, embaraço.
Intriga.
\section{Encrencar}
\begin{itemize}
\item {Grp. gram.:v. t.}
\end{itemize}
\begin{itemize}
\item {Utilização:Bras. do N}
\end{itemize}
Pôr alguém em embaraços.
Intrigar.
(Cp. \textunderscore encrenque\textunderscore )
\section{Encrenque}
\begin{itemize}
\item {Grp. gram.:m.}
\end{itemize}
\begin{itemize}
\item {Utilização:Prov.}
\end{itemize}
\begin{itemize}
\item {Utilização:beir.}
\end{itemize}
Pessôa inútil ou desleixada; estafermo.
(Cast. \textunderscore enclenque\textunderscore )
\section{Encrespação}
\begin{itemize}
\item {Grp. gram.:f.}
\end{itemize}
\begin{itemize}
\item {Proveniência:(Do lat. \textunderscore incrispatio\textunderscore )}
\end{itemize}
Acto de encrespar.
\section{Encrespador}
\begin{itemize}
\item {Grp. gram.:m.}
\end{itemize}
Instrumento para encrespar.
\section{Encrespadura}
\begin{itemize}
\item {Grp. gram.:f.}
\end{itemize}
Acto ou effeito de encrespar.
\section{Encrespamento}
\begin{itemize}
\item {Grp. gram.:m.}
\end{itemize}
O mesmo que \textunderscore encrespadura\textunderscore .
\section{Encrespar}
\begin{itemize}
\item {Grp. gram.:v. t.}
\end{itemize}
\begin{itemize}
\item {Utilização:Fig.}
\end{itemize}
\begin{itemize}
\item {Proveniência:(Lat. \textunderscore incrispare\textunderscore )}
\end{itemize}
Crespir, tornar crespo.
Frisar; encaracolar: \textunderscore encrespar o cabello\textunderscore .
Enrugar.
Tornar áspero, agitado, tempestuoso: \textunderscore o vento encrespa as ondas\textunderscore .
Ouriçar.
Enfatuar.
Irritar.
\section{Encriptar}
\begin{itemize}
\item {Grp. gram.:v. t.}
\end{itemize}
\begin{itemize}
\item {Utilização:Ant.}
\end{itemize}
Meter em cripta ou túmulo.
\section{Encristar-se}
\begin{itemize}
\item {Grp. gram.:v. p.}
\end{itemize}
\begin{itemize}
\item {Utilização:Fig.}
\end{itemize}
Levantar a crista.
Mostrar-se orgulhoso.
\section{Encristinar-se}
\begin{itemize}
\item {Grp. gram.:v. t.}
\end{itemize}
\begin{itemize}
\item {Utilização:Prov.}
\end{itemize}
\begin{itemize}
\item {Utilização:trasm.}
\end{itemize}
O mesmo que \textunderscore encristar-se\textunderscore , \textunderscore orgulhar-se\textunderscore , \textunderscore insurgir-se\textunderscore .
\section{Encrostar}
\begin{itemize}
\item {Grp. gram.:v. i.  e  p.}
\end{itemize}
\begin{itemize}
\item {Grp. gram.:v. t.}
\end{itemize}
\begin{itemize}
\item {Proveniência:(Lat. \textunderscore incrustare\textunderscore )}
\end{itemize}
Criar crosta.
Cobrir de crosta.
Sobrepor uma camada a.
Embutir; tauxiar; gravar; inserir.
\section{Encruamento}
\begin{itemize}
\item {Grp. gram.:m.}
\end{itemize}
Acto ou effeito de encruar.
\section{Encruar}
\begin{itemize}
\item {Grp. gram.:v. t.}
\end{itemize}
\begin{itemize}
\item {Utilização:Fig.}
\end{itemize}
\begin{itemize}
\item {Grp. gram.:V. i.}
\end{itemize}
\begin{itemize}
\item {Proveniência:(De \textunderscore cru\textunderscore )}
\end{itemize}
Enrijar (aquillo que se estava cozendo).
Difficultar ou retardar as funcções de (o estômago).
Callejar.
Azedar o ánimo de.
Empatar, adiar (negociação que estava correndo).
Tornar-se cru.
Enrijar-se na cozedura: \textunderscore o feijão encruou\textunderscore .
\section{Encrudelecer}
\begin{itemize}
\item {Grp. gram.:v. i.  e  p.}
\end{itemize}
\begin{itemize}
\item {Proveniência:(Do lat. \textunderscore crudelis\textunderscore )}
\end{itemize}
Tornar-se cruel.
Irritar-se.
\section{Encruecer}
\begin{itemize}
\item {Grp. gram.:v. t.}
\end{itemize}
\begin{itemize}
\item {Grp. gram.:V. i.}
\end{itemize}
\begin{itemize}
\item {Proveniência:(De \textunderscore cru\textunderscore )}
\end{itemize}
Encruar.
Encarniçar, tornar feroz. Cf. Filinto, \textunderscore D. Man.\textunderscore , I, 136.
Tornar-se feroz, sanguinolento. Cf. \textunderscore Lusíadas\textunderscore , IV, 42.
\section{Encruelecer}
\begin{itemize}
\item {Grp. gram.:v. i.  e  p.}
\end{itemize}
(V.encrudelecer)
\section{Encrustar}
\begin{itemize}
\item {Grp. gram.:v. t.}
\end{itemize}
(V.encrostar). Cf. Castilho, \textunderscore Fastos\textunderscore , III, 423 e 495.
\section{Encruzada}
\begin{itemize}
\item {Grp. gram.:f.}
\end{itemize}
O mesmo que \textunderscore encruzilhada\textunderscore .
\section{Encruzado}
\begin{itemize}
\item {Grp. gram.:adj.}
\end{itemize}
Pôsto em fórma de cruz: \textunderscore pernas encruzadas\textunderscore .
\section{Encruzamento}
\begin{itemize}
\item {Grp. gram.:m.}
\end{itemize}
Acto ou effeito de encruzar.
Cruzamento.
Lugar, onde algumas coisas se cruzam.
\section{Encruzar}
\begin{itemize}
\item {Grp. gram.:v. t.}
\end{itemize}
\begin{itemize}
\item {Grp. gram.:V. p.}
\end{itemize}
Pôr em fórma de cruz.
Atravessar; cruzar.
Cruzar as pernas:«\textunderscore nunca se encruzou turco em divan de seda, com tanto gôzo...\textunderscore »Garrett, \textunderscore Viagens\textunderscore .
\section{Encruzilhada}
\begin{itemize}
\item {Grp. gram.:f.}
\end{itemize}
\begin{itemize}
\item {Proveniência:(De \textunderscore encruzilhar\textunderscore )}
\end{itemize}
Ponto, em que se cruzam caminhos.
\section{Encruzilhar}
\begin{itemize}
\item {Grp. gram.:v. t.}
\end{itemize}
O mesmo que \textunderscore encruzar\textunderscore .
\section{Encryptar}
\begin{itemize}
\item {Grp. gram.:v. t.}
\end{itemize}
\begin{itemize}
\item {Utilização:Ant.}
\end{itemize}
Meter em crypta ou túmulo.
\section{Encuba}
\begin{itemize}
\item {Grp. gram.:f.}
\end{itemize}
\begin{itemize}
\item {Utilização:Prov.}
\end{itemize}
\begin{itemize}
\item {Utilização:dur.}
\end{itemize}
Acto de encubar o vinho.
\section{Encubação}
\begin{itemize}
\item {Grp. gram.:f.}
\end{itemize}
Acto de encubar.
\section{Encubar}
\begin{itemize}
\item {Grp. gram.:v. t.}
\end{itemize}
Meter em cuba; envasilhar.
\section{Encucharrar}
\begin{itemize}
\item {Grp. gram.:v. t.}
\end{itemize}
Dar fórma de cucharra a.
\section{Encueirar}
\begin{itemize}
\item {Grp. gram.:v. t.}
\end{itemize}
Pôr cueiro a:«\textunderscore o velho ainda se encueirava num manteu sujo\textunderscore ». Filinto, XII, 181.
\section{Encumeada}
\begin{itemize}
\item {Grp. gram.:f.}
\end{itemize}
O mesmo que \textunderscore cumeada\textunderscore .
\section{Encumear}
\begin{itemize}
\item {Grp. gram.:v. t.}
\end{itemize}
Pôr no cume; encimar.
\section{Encunhar}
\begin{itemize}
\item {Grp. gram.:v. t.}
\end{itemize}
\begin{itemize}
\item {Utilização:Prov.}
\end{itemize}
Meter cunhas em (pedra fendida) para se partir.
\section{Encurralamento}
\begin{itemize}
\item {Grp. gram.:m.}
\end{itemize}
Acto de encurralar.
\section{Encurralar}
\begin{itemize}
\item {Grp. gram.:v. t.}
\end{itemize}
Meter em curral.
Encerrar em lugar estreito, sem saida.
Pôr cêrco a (os inimigos).
\section{Encurrelhado}
\begin{itemize}
\item {Grp. gram.:adj.}
\end{itemize}
\begin{itemize}
\item {Utilização:Des.}
\end{itemize}
\begin{itemize}
\item {Proveniência:(De \textunderscore encurrilhar\textunderscore )}
\end{itemize}
Afrontado, injuriado.
\section{Encurrilhado}
\begin{itemize}
\item {Grp. gram.:adj.}
\end{itemize}
\begin{itemize}
\item {Utilização:Des.}
\end{itemize}
\begin{itemize}
\item {Proveniência:(De \textunderscore encurrilhar\textunderscore )}
\end{itemize}
Afrontado, injuriado.
\section{Encurrilhar}
\begin{itemize}
\item {Grp. gram.:v. t.}
\end{itemize}
\begin{itemize}
\item {Utilização:Des.}
\end{itemize}
Encerrar, encarcerar; encantoar.
(Cp. \textunderscore encurralar\textunderscore )
\section{Encurta}
\begin{itemize}
\item {Grp. gram.:m.}
\end{itemize}
\begin{itemize}
\item {Utilização:T. de Aveiro}
\end{itemize}
\begin{itemize}
\item {Proveniência:(De \textunderscore encurtar\textunderscore ?)}
\end{itemize}
Cada um dos remadores da primeira fila no grupo de cada remo.
\section{Encurtadoiro}
\begin{itemize}
\item {Grp. gram.:m.}
\end{itemize}
\begin{itemize}
\item {Utilização:Prov.}
\end{itemize}
\begin{itemize}
\item {Utilização:alg.}
\end{itemize}
\begin{itemize}
\item {Proveniência:(De \textunderscore encurtar\textunderscore )}
\end{itemize}
Atalho.
O caminho mais curto para qualquer parte.
\section{Encurtadouro}
\begin{itemize}
\item {Grp. gram.:m.}
\end{itemize}
\begin{itemize}
\item {Utilização:Prov.}
\end{itemize}
\begin{itemize}
\item {Utilização:alg.}
\end{itemize}
\begin{itemize}
\item {Proveniência:(De \textunderscore encurtar\textunderscore )}
\end{itemize}
Atalho.
O caminho mais curto para qualquer parte.
\section{Encurtador}
\begin{itemize}
\item {Grp. gram.:adj.}
\end{itemize}
\begin{itemize}
\item {Grp. gram.:M.}
\end{itemize}
Que encurta.
Aquelle que encurta.
\section{Encurtamento}
\begin{itemize}
\item {Grp. gram.:m.}
\end{itemize}
Acto ou effeito de encurtar.
\section{Encurtar}
\begin{itemize}
\item {Grp. gram.:v. t.}
\end{itemize}
Tornar curto.
Fazer mais pequeno; deminuir; reduzir: \textunderscore encurtar o caminho\textunderscore .
Resumir: \textunderscore encurtar divagações\textunderscore .
\section{Encurvação}
\begin{itemize}
\item {Grp. gram.:f.}
\end{itemize}
\begin{itemize}
\item {Proveniência:(Lat. \textunderscore incurvatio\textunderscore )}
\end{itemize}
O mesmo que \textunderscore encurvamento\textunderscore .
\section{Encurvadura}
\begin{itemize}
\item {Grp. gram.:f.}
\end{itemize}
Acto ou effeito de encurvar.
\section{Encurvamento}
\begin{itemize}
\item {Grp. gram.:m.}
\end{itemize}
Acto ou effeito de encurvar.
\section{Encurvar}
\begin{itemize}
\item {Grp. gram.:v. t.}
\end{itemize}
\begin{itemize}
\item {Utilização:Fig.}
\end{itemize}
\begin{itemize}
\item {Proveniência:(Lat. \textunderscore incurvare\textunderscore )}
\end{itemize}
Tornar curvo; dar fórma de arco a; curvar.
Humilhar.
\section{Encyclia}
\begin{itemize}
\item {Grp. gram.:f.}
\end{itemize}
\begin{itemize}
\item {Proveniência:(Do gr. \textunderscore kuklos\textunderscore )}
\end{itemize}
Ondulação circular, produzida na superfície da água pela quéda de um corpo.
\section{Encýclica}
\begin{itemize}
\item {Grp. gram.:f.}
\end{itemize}
\begin{itemize}
\item {Proveniência:(De \textunderscore encýclico\textunderscore )}
\end{itemize}
Carta circular pontificia, dogmática ou doutrinária.
\section{Encýclico}
\begin{itemize}
\item {Grp. gram.:adj.}
\end{itemize}
Circular.
Diz-se das cartas circulares dos Pontífices.
(Cp. \textunderscore encyclia\textunderscore )
\section{Encyclopedia}
\begin{itemize}
\item {Grp. gram.:f.}
\end{itemize}
\begin{itemize}
\item {Proveniência:(Do gr. \textunderscore enkuklopaideia\textunderscore )}
\end{itemize}
Conjunto de conhecimentos, relativos a todas as sciências e artes, ou relativos a uma classe ou domínio de sciências e artes.
\section{Encyclopédico}
\begin{itemize}
\item {Grp. gram.:adj.}
\end{itemize}
Relativo a encyclopedia.
Que abrange todos os ramos do saber humano: \textunderscore diccionário encyclopédico\textunderscore .
Que possue conhecimentos vastos ou concernentes a todas as sciências.
\section{Encyclopedismo}
\begin{itemize}
\item {Grp. gram.:m.}
\end{itemize}
\begin{itemize}
\item {Proveniência:(De \textunderscore encyclopedia\textunderscore )}
\end{itemize}
Systema dos encyclopedistas.
\section{Encyclopedista}
\begin{itemize}
\item {Grp. gram.:m.}
\end{itemize}
\begin{itemize}
\item {Proveniência:(De \textunderscore encyclopedia\textunderscore )}
\end{itemize}
Escritor ou collaborador de obra encyclopédica.
Partidário das doutrinas da \textunderscore Encyclopedia\textunderscore  de D'Alembert e Diderot.
\section{Encyclopedístico}
\begin{itemize}
\item {Grp. gram.:adj.}
\end{itemize}
Relativo aos encyclopedistas franceses do século XVIII.
\section{Encyprótypo}
\begin{itemize}
\item {Grp. gram.:adj.}
\end{itemize}
\begin{itemize}
\item {Proveniência:(Do gr. \textunderscore en\textunderscore  + \textunderscore cupros\textunderscore  + \textunderscore tupos\textunderscore )}
\end{itemize}
\begin{itemize}
\item {Proveniência:Encra
}
\end{itemize}
Gravado immediatamente em cobre.
\section{Encistado}
\begin{itemize}
\item {Grp. gram.:adj.}
\end{itemize}
\begin{itemize}
\item {Proveniência:(De \textunderscore encistar\textunderscore )}
\end{itemize}
Envolto em cisto ou em membrana semelhante a cisto.
\section{Encistamento}
\begin{itemize}
\item {Grp. gram.:m.}
\end{itemize}
Acto ou efeito de encistar.
\section{Encystado}
\begin{itemize}
\item {Grp. gram.:adj.}
\end{itemize}
\begin{itemize}
\item {Proveniência:(De \textunderscore encystar\textunderscore )}
\end{itemize}
Envolto em cysto ou em membrana semelhante a cysto.
\section{Encystamento}
\begin{itemize}
\item {Grp. gram.:m.}
\end{itemize}
Acto ou effeito de encystar.
\section{Encystar}
\begin{itemize}
\item {Grp. gram.:v. i.}
\end{itemize}
Converter-se em cysto.
Envolver-se em cysto.
Cercar-se de membrana semelhante a cysto.
\section{Endavaes}
\begin{itemize}
\item {Grp. gram.:m. pl.}
\end{itemize}
\begin{itemize}
\item {Utilização:T. do Ribatejo}
\end{itemize}
O mesmo que \textunderscore andavaes\textunderscore .
\section{Endavais}
\begin{itemize}
\item {Grp. gram.:m. pl.}
\end{itemize}
\begin{itemize}
\item {Utilização:T. do Ribatejo}
\end{itemize}
O mesmo que \textunderscore andavaes\textunderscore .
\section{Ende}
\begin{itemize}
\item {Grp. gram.:adv.}
\end{itemize}
\begin{itemize}
\item {Utilização:Ant.}
\end{itemize}
\begin{itemize}
\item {Grp. gram.:Pron.}
\end{itemize}
\begin{itemize}
\item {Proveniência:(Do lat. \textunderscore inde\textunderscore )}
\end{itemize}
Juntamente.
Daí.
Isto, êste motivo.
\section{Endecandria}
\begin{itemize}
\item {Grp. gram.:f.}
\end{itemize}
\begin{itemize}
\item {Proveniência:(Do gr. \textunderscore endeka\textunderscore  + \textunderscore aner\textunderscore , \textunderscore andros\textunderscore )}
\end{itemize}
Classe de vegetaes, que comprehende as plantas que dão flôres providas de onze estames. Cf. systema de Linneu.
\section{Endecha}
\begin{itemize}
\item {Grp. gram.:f.}
\end{itemize}
Canção triste ou fúnebre.
(Cast. \textunderscore endecha\textunderscore , do lat. \textunderscore indicta\textunderscore )
\section{Endechador}
\begin{itemize}
\item {Grp. gram.:m.  e  adj.}
\end{itemize}
\begin{itemize}
\item {Utilização:Ant.}
\end{itemize}
\begin{itemize}
\item {Proveniência:(De \textunderscore endechar\textunderscore )}
\end{itemize}
Autor de endechas.
\section{Endechar}
\begin{itemize}
\item {Grp. gram.:v. i.}
\end{itemize}
\begin{itemize}
\item {Utilização:Ant.}
\end{itemize}
Fazer ou entoar endechas.
\section{Endefluxado}
\begin{itemize}
\item {Grp. gram.:adj.}
\end{itemize}
Que tem defluxo.
Constipado.
\section{Endefluxar-se}
\begin{itemize}
\item {Grp. gram.:v. p.}
\end{itemize}
Constipar-se, adquirir defluxo.
\section{Endeixa}
\begin{itemize}
\item {Grp. gram.:f.}
\end{itemize}
(V.endecha)
\section{Endejar}
\begin{itemize}
\item {Grp. gram.:v. i.}
\end{itemize}
\begin{itemize}
\item {Utilização:Prov.}
\end{itemize}
\begin{itemize}
\item {Utilização:trasm.}
\end{itemize}
Agitar-se, revolver-se, tremer: \textunderscore senti cá por dentro tudo endejar\textunderscore .
(Por \textunderscore andejar\textunderscore ?)
\section{Endemia}
\begin{itemize}
\item {Grp. gram.:f.}
\end{itemize}
\begin{itemize}
\item {Proveniência:(Gr. \textunderscore endemia\textunderscore )}
\end{itemize}
Doença, que grassa num povo ou numa região, e que depende de causas meramente locaes.
\section{Endemicidade}
\begin{itemize}
\item {Grp. gram.:f.}
\end{itemize}
Qualidade de endêmico.
\section{Endêmico}
\begin{itemize}
\item {Grp. gram.:adj.}
\end{itemize}
Relativo a endemia: \textunderscore moléstia endêmica\textunderscore .
\section{Endemoniado}
\begin{itemize}
\item {Grp. gram.:adj.}
\end{itemize}
\begin{itemize}
\item {Utilização:T. de Ceilão}
\end{itemize}
O mesmo que \textunderscore endemoninhado\textunderscore .
\section{Endemoninhado}
\begin{itemize}
\item {Grp. gram.:adj.}
\end{itemize}
\begin{itemize}
\item {Proveniência:(De \textunderscore endemoninhar\textunderscore )}
\end{itemize}
Possesso do demónio.
Furioso.
\section{Endemoninhamento}
\begin{itemize}
\item {Grp. gram.:m.}
\end{itemize}
Acto de endemoninhar. Cf. Eça, \textunderscore P. Amaro\textunderscore , 413.
\section{Endemoninhar}
\begin{itemize}
\item {Grp. gram.:v. t.}
\end{itemize}
\begin{itemize}
\item {Utilização:Fig.}
\end{itemize}
\begin{itemize}
\item {Proveniência:(De \textunderscore demónio\textunderscore )}
\end{itemize}
Meter o demónio no corpo de.
Enfurecer.
\section{Endengue}
\begin{itemize}
\item {Grp. gram.:m.}
\end{itemize}
\begin{itemize}
\item {Utilização:Pop.}
\end{itemize}
O mesmo que \textunderscore êndez\textunderscore , no sent. fig.
\section{Endentação}
\begin{itemize}
\item {Grp. gram.:f.}
\end{itemize}
Acto de endentar.
\section{Endentar}
\begin{itemize}
\item {Grp. gram.:v. t.}
\end{itemize}
\begin{itemize}
\item {Proveniência:(De \textunderscore dentar\textunderscore )}
\end{itemize}
Travar os dentes de (uma roda) com os dentes de outra; engranzar.
\section{Endentecer}
\begin{itemize}
\item {Grp. gram.:v. i.}
\end{itemize}
Começar a têr dentes.
\section{Enderêça}
\begin{itemize}
\item {Grp. gram.:f.}
\end{itemize}
\begin{itemize}
\item {Utilização:Des.}
\end{itemize}
O mesmo que enderêço:«\textunderscore deixai em casa a enderêça de Madama Birton.\textunderscore »Filinto, XIX, 108.
\section{Endereçamento}
\begin{itemize}
\item {Grp. gram.:m.}
\end{itemize}
Acto de endereçar.
\section{Endereçar}
\begin{itemize}
\item {Grp. gram.:v. t.}
\end{itemize}
\begin{itemize}
\item {Proveniência:(Do lat. hyp. \textunderscore inderectiare\textunderscore )}
\end{itemize}
Pôr sobrescríto em.
Enviar, dirigir: \textunderscore endereçar uma carta\textunderscore .
\section{Enderêço}
\begin{itemize}
\item {Grp. gram.:m.}
\end{itemize}
Acto de endereçar.
Indicação de residência.
\section{Enderencar}
\begin{itemize}
\item {Grp. gram.:v. t.}
\end{itemize}
\begin{itemize}
\item {Utilização:Ant.}
\end{itemize}
Encaminhar; o mesmo que \textunderscore endereçar\textunderscore .
\section{Endérmico}
\begin{itemize}
\item {Grp. gram.:adj.}
\end{itemize}
Que actua sôbre a derme.
\section{Êndes}
\begin{itemize}
\item {Grp. gram.:m.}
\end{itemize}
\begin{itemize}
\item {Utilização:pop.}
\end{itemize}
\begin{itemize}
\item {Utilização:Fig.}
\end{itemize}
\begin{itemize}
\item {Proveniência:(Do lat. \textunderscore index\textunderscore )}
\end{itemize}
Ovo, que se colloca no lugar, em que se deseja que uma gallinha ponha outros.
Criança ou pequena coisa, que faz empecilho.-- Há quem diga \textunderscore endêz\textunderscore , sem razão.
\section{Endeusadamente}
\begin{itemize}
\item {Grp. gram.:adv.}
\end{itemize}
\begin{itemize}
\item {Proveniência:(De \textunderscore endeusar\textunderscore )}
\end{itemize}
De modo divinal.
\section{Endeusamento}
\begin{itemize}
\item {Grp. gram.:m.}
\end{itemize}
Acto ou effeito de endeusar.
\section{Endeusar}
\begin{itemize}
\item {Grp. gram.:v. t.}
\end{itemize}
\begin{itemize}
\item {Utilização:Fig.}
\end{itemize}
\begin{itemize}
\item {Proveniência:(De \textunderscore deus\textunderscore )}
\end{itemize}
Incluír em o número dos deuses.
Divinizar.
Extasiar.
Tornar altivo, soberbo.
\section{Êndez}
\begin{itemize}
\item {Grp. gram.:m.}
\end{itemize}
\begin{itemize}
\item {Utilização:pop.}
\end{itemize}
\begin{itemize}
\item {Utilização:Fig.}
\end{itemize}
\begin{itemize}
\item {Proveniência:(Do lat. \textunderscore index\textunderscore )}
\end{itemize}
Ovo, que se colloca no lugar, em que se deseja que uma gallinha ponha outros.
Criança ou pequena coisa, que faz empecilho.--Há quem diga \textunderscore endêz\textunderscore , sem razão.
\section{Endhymenina}
\begin{itemize}
\item {Grp. gram.:f.}
\end{itemize}
\begin{itemize}
\item {Proveniência:(Do gr. \textunderscore endon\textunderscore  + \textunderscore humen\textunderscore )}
\end{itemize}
Membrana interna do póllen.
\section{Endiabradamente}
\begin{itemize}
\item {Grp. gram.:adv.}
\end{itemize}
De modo endiabrado.
\section{Endiabrado}
\begin{itemize}
\item {Grp. gram.:adj.}
\end{itemize}
Endemoninhado.
Que parece possesso do demónio.
Infernal.
Furioso: \textunderscore vento endiabrado\textunderscore .
Medonho.
(Por \textunderscore endiabado\textunderscore , de \textunderscore diabo\textunderscore . Cp. \textunderscore diabrete\textunderscore )
\section{Endiche}
\begin{itemize}
\item {Grp. gram.:m.}
\end{itemize}
Rêde vertical, que guarnece a bôca de uma armação de pesca.
\section{Endimenina}
\begin{itemize}
\item {Grp. gram.:f.}
\end{itemize}
\begin{itemize}
\item {Proveniência:(Do gr. \textunderscore endon\textunderscore  + \textunderscore humen\textunderscore )}
\end{itemize}
Membrana interna do pólen.
\section{Endinhar}
\begin{itemize}
\item {Grp. gram.:v. t.}
\end{itemize}
\begin{itemize}
\item {Utilização:Gír.}
\end{itemize}
\begin{itemize}
\item {Proveniência:(De or. ind., segundo alguns; occorre-me porém que será alter. de \textunderscore endignar\textunderscore , tornar digno, abonar)}
\end{itemize}
Abonar, afiançar.
\section{Endinheirado}
\begin{itemize}
\item {Grp. gram.:adj.}
\end{itemize}
Que tem muito dinheiro.
Em que há muito dinheiro.
Opulento.
\section{Endireita}
\begin{itemize}
\item {Grp. gram.:m.}
\end{itemize}
\begin{itemize}
\item {Utilização:Pop.}
\end{itemize}
\begin{itemize}
\item {Utilização:Irón.}
\end{itemize}
\begin{itemize}
\item {Proveniência:(De \textunderscore endireitar\textunderscore )}
\end{itemize}
Curioso ou charlatão, que encana ou compõe ossos deslocados ou fracturados.
Homem político, que procura endireitar os negócios públicos.
\section{Endireitar}
\begin{itemize}
\item {Grp. gram.:v. t.}
\end{itemize}
\begin{itemize}
\item {Utilização:Fig.}
\end{itemize}
\begin{itemize}
\item {Grp. gram.:V. i.}
\end{itemize}
Tornar direito.
Encaminhar directamente.
Pôr em pé.
Corrigir.
Dar bôa direcção a.
Seguir bôa direcção.
Caminhar directamente.
Acertar.
\section{Endireito}
\begin{itemize}
\item {Grp. gram.:m.}
\end{itemize}
\begin{itemize}
\item {Proveniência:(De \textunderscore direito\textunderscore )}
\end{itemize}
Direcção.
Encontro.
\section{Endiva}
\begin{itemize}
\item {Grp. gram.:f.}
\end{itemize}
\begin{itemize}
\item {Proveniência:(Do lat. \textunderscore entubus\textunderscore ?)}
\end{itemize}
Espécie de chicória.
\section{Endívia}
\begin{itemize}
\item {Grp. gram.:f.}
\end{itemize}
\begin{itemize}
\item {Proveniência:(Do lat. \textunderscore entubus\textunderscore ?)}
\end{itemize}
Espécie de chicória.
\section{Endividar}
\begin{itemize}
\item {Grp. gram.:v. t.}
\end{itemize}
\begin{itemize}
\item {Grp. gram.:V. p.}
\end{itemize}
Fazer contrahir dívidas; tornar devedor.
Contrahir dívidas.
\section{Endoado}
\begin{itemize}
\item {Grp. gram.:adj.}
\end{itemize}
\begin{itemize}
\item {Proveniência:(De \textunderscore dó\textunderscore )}
\end{itemize}
Enlutado.
Triste; dolorido.
\section{Endocárdio}
\begin{itemize}
\item {Grp. gram.:m.}
\end{itemize}
\begin{itemize}
\item {Utilização:Anat.}
\end{itemize}
\begin{itemize}
\item {Proveniência:(Do gr. \textunderscore endon\textunderscore  + \textunderscore kardia\textunderscore )}
\end{itemize}
Membrana interior do coração.
\section{Endocardite}
\begin{itemize}
\item {Grp. gram.:f.}
\end{itemize}
Inflammação do endocárdio.
\section{Endocarpo}
\begin{itemize}
\item {Grp. gram.:m.}
\end{itemize}
\begin{itemize}
\item {Utilização:Bot.}
\end{itemize}
\begin{itemize}
\item {Proveniência:(Do gr. \textunderscore endon\textunderscore  + \textunderscore karpos\textunderscore )}
\end{itemize}
Membrana interior do fruto, em contacto com a semente.
\section{Endocéfalo}
\begin{itemize}
\item {Grp. gram.:adj.}
\end{itemize}
\begin{itemize}
\item {Utilização:Hist. Nat.}
\end{itemize}
\begin{itemize}
\item {Proveniência:(Do gr. \textunderscore endon\textunderscore  + \textunderscore kephale\textunderscore )}
\end{itemize}
Que não tem cabeça, aparentemente.
\section{Endocéphalo}
\begin{itemize}
\item {Grp. gram.:adj.}
\end{itemize}
\begin{itemize}
\item {Utilização:Hist. Nat.}
\end{itemize}
\begin{itemize}
\item {Proveniência:(Do gr. \textunderscore endon\textunderscore  + \textunderscore kephale\textunderscore )}
\end{itemize}
Que não tem cabeça, apparentemente.
\section{Endocraniano}
\begin{itemize}
\item {Grp. gram.:adj.}
\end{itemize}
\begin{itemize}
\item {Proveniência:(De \textunderscore endocrânio\textunderscore )}
\end{itemize}
Situado na parte interior do crânio.
\section{Endocrânio}
\begin{itemize}
\item {Grp. gram.:m.}
\end{itemize}
\begin{itemize}
\item {Utilização:Anat.}
\end{itemize}
\begin{itemize}
\item {Proveniência:(Do gr. \textunderscore endon\textunderscore  + \textunderscore kranion\textunderscore )}
\end{itemize}
A parte interior do crânio.
\section{Endoderme}
\begin{itemize}
\item {Grp. gram.:m.}
\end{itemize}
\begin{itemize}
\item {Proveniência:(Do gr. \textunderscore endon\textunderscore  + \textunderscore derma\textunderscore )}
\end{itemize}
Folheto externo do blastoderme.
\section{Endodontite}
\begin{itemize}
\item {Grp. gram.:f.}
\end{itemize}
\begin{itemize}
\item {Utilização:Med.}
\end{itemize}
\begin{itemize}
\item {Proveniência:(Do gr. \textunderscore endon\textunderscore  + \textunderscore odous\textunderscore , \textunderscore odontos\textunderscore )}
\end{itemize}
Inflammação da membrana que reveste os alvéolos dos dentes.
\section{Endoenças}
\begin{itemize}
\item {Grp. gram.:f. pl.}
\end{itemize}
\begin{itemize}
\item {Proveniência:(Do lat. \textunderscore indulgentias\textunderscore , seg. Car. Michaëlis)}
\end{itemize}
Solennidade religiosa em quinta-feira santa; celebração ecclesiástica da Paixão de Christo.
\section{Endoestesia}
\begin{itemize}
\item {fónica:do-es}
\end{itemize}
\begin{itemize}
\item {Grp. gram.:f.}
\end{itemize}
Sensibilidade intrínseca, ou o sentimento mais íntimo em todas as suas gradações.
\section{Endoesthesia}
\begin{itemize}
\item {fónica:do-es}
\end{itemize}
\begin{itemize}
\item {Grp. gram.:f.}
\end{itemize}
Sensibilidade intrínseca, ou o sentimento mais íntimo em todas as suas gradações.
\section{Endogamia}
\begin{itemize}
\item {Grp. gram.:f.}
\end{itemize}
Estado de endógamo.
\section{Endógamo}
\begin{itemize}
\item {Grp. gram.:s. m.  e  adj.}
\end{itemize}
\begin{itemize}
\item {Proveniência:(Do gr. \textunderscore endon\textunderscore  + \textunderscore gamos\textunderscore )}
\end{itemize}
Selvagem que, pela organização da sua tríbo, se liga com a mulher da mesma tríbo, para conservação de sua nobreza e raça.
\section{Endogastrite}
\begin{itemize}
\item {Grp. gram.:f.}
\end{itemize}
\begin{itemize}
\item {Utilização:Med.}
\end{itemize}
\begin{itemize}
\item {Proveniência:(Do gr. \textunderscore endon\textunderscore  + \textunderscore gaster\textunderscore )}
\end{itemize}
Inflammação da membrana mucosa do estômago.
\section{Endógenas}
\begin{itemize}
\item {Grp. gram.:f. pl.}
\end{itemize}
\begin{itemize}
\item {Proveniência:(De \textunderscore endógeno\textunderscore )}
\end{itemize}
O mesmo que \textunderscore monocotyledóneas\textunderscore .
\section{Endógeno}
\begin{itemize}
\item {Grp. gram.:adj.}
\end{itemize}
\begin{itemize}
\item {Proveniência:(Do gr. \textunderscore endon\textunderscore  + \textunderscore genos\textunderscore )}
\end{itemize}
Que cresce para dentro.
\section{Endoidar}
\begin{itemize}
\item {Grp. gram.:v. t.}
\end{itemize}
Tornar doido.
Desorientar, perturbar.
\section{Endoidecer}
\begin{itemize}
\item {Grp. gram.:v. t.}
\end{itemize}
\begin{itemize}
\item {Grp. gram.:V. i.}
\end{itemize}
Tornar doido.
Tornar-se doido, enlouquecer.
\section{Endoidecimento}
\begin{itemize}
\item {Grp. gram.:m.}
\end{itemize}
Acto ou effeito de endoidecer.
\section{Endolinfa}
\begin{itemize}
\item {Grp. gram.:f.}
\end{itemize}
\begin{itemize}
\item {Utilização:Anat.}
\end{itemize}
Liquido claro e albuminoso, que enche completamente o labirinto membranoso do ouvido.
\section{Endolympha}
\begin{itemize}
\item {Grp. gram.:f.}
\end{itemize}
\begin{itemize}
\item {Utilização:Anat.}
\end{itemize}
Líquido claro e albuminoso, que enche completamente o labyrintho membranoso do ouvido.
\section{Endomingado}
\begin{itemize}
\item {Grp. gram.:adj.}
\end{itemize}
\begin{itemize}
\item {Utilização:Prov.}
\end{itemize}
\begin{itemize}
\item {Proveniência:(De \textunderscore Domingo\textunderscore )}
\end{itemize}
Vestido com o melhor fato, com o fato de ir á igreja.
Garrido, casquilho.
\section{Endopleura}
\begin{itemize}
\item {Grp. gram.:f.}
\end{itemize}
\begin{itemize}
\item {Utilização:Bot.}
\end{itemize}
\begin{itemize}
\item {Proveniência:(Do gr. \textunderscore endon\textunderscore  + \textunderscore pleuron\textunderscore )}
\end{itemize}
Pellícula, impermeável á humidade, e que é para o grão o que o endosperma é para o fruto. Cf. De-Candolle.
\section{Endóptera}
\begin{itemize}
\item {Grp. gram.:f.}
\end{itemize}
\begin{itemize}
\item {Proveniência:(Do gr. \textunderscore endon\textunderscore  + \textunderscore pteron\textunderscore )}
\end{itemize}
Planta, da fam. das compostas.
\section{Endorar}
\begin{itemize}
\item {fónica:dô-rár}
\end{itemize}
\begin{itemize}
\item {Grp. gram.:v. t.}
\end{itemize}
\begin{itemize}
\item {Utilização:Ant.}
\end{itemize}
\begin{itemize}
\item {Proveniência:(De \textunderscore dôr\textunderscore )}
\end{itemize}
Padecer, supportar.
\section{Endoscopia}
\begin{itemize}
\item {Grp. gram.:f.}
\end{itemize}
Applicação do endoscópio.
\section{Endoscópico}
\begin{itemize}
\item {Grp. gram.:adj.}
\end{itemize}
Relativo a endoscopia.
\section{Endoscópio}
\begin{itemize}
\item {Grp. gram.:m.}
\end{itemize}
\begin{itemize}
\item {Proveniência:(Do gr. \textunderscore endon\textunderscore  + \textunderscore skopein\textunderscore )}
\end{itemize}
Instrumento médico, para a observação ocular de algumas cavidades do corpo.
\section{Endosmómetro}
\begin{itemize}
\item {Grp. gram.:m.}
\end{itemize}
\begin{itemize}
\item {Proveniência:(Do gr. \textunderscore endon\textunderscore  + \textunderscore osmos\textunderscore  + \textunderscore metron\textunderscore )}
\end{itemize}
Apparelho, para apreciar os phenómenos da endosmose.
\section{Endosmose}
\begin{itemize}
\item {Grp. gram.:f.}
\end{itemize}
\begin{itemize}
\item {Utilização:Phýs.}
\end{itemize}
\begin{itemize}
\item {Proveniência:(Do gr. \textunderscore endon\textunderscore  + \textunderscore osmos\textunderscore )}
\end{itemize}
Corrente entre dois líquidos ou gases, através de uma membrana ou placa porosa que os separa.
\section{Endosmótico}
\begin{itemize}
\item {Grp. gram.:adj.}
\end{itemize}
Relativo a endosmose.
\section{Endosperma}
\begin{itemize}
\item {Grp. gram.:m.}
\end{itemize}
\begin{itemize}
\item {Utilização:Bot.}
\end{itemize}
\begin{itemize}
\item {Proveniência:(Do gr. \textunderscore endon\textunderscore  + \textunderscore sperma\textunderscore )}
\end{itemize}
Substância, que fórma ao lado do embryão um corpo accessório, sem continuidade de vasos ou tecidos, podendo por isso separar-se facilmente.
\section{Endospérmico}
\begin{itemize}
\item {Grp. gram.:adj.}
\end{itemize}
Diz-se do embryão que tem endosperma.
\section{Endósporo}
\begin{itemize}
\item {Grp. gram.:adj.}
\end{itemize}
\begin{itemize}
\item {Utilização:Bot.}
\end{itemize}
\begin{itemize}
\item {Proveniência:(Do gr. \textunderscore endon\textunderscore  + \textunderscore sporos\textunderscore )}
\end{itemize}
Provido interiormente de espórulos.
\section{Endossado}
\begin{itemize}
\item {Grp. gram.:m.}
\end{itemize}
\begin{itemize}
\item {Proveniência:(De \textunderscore endossar\textunderscore )}
\end{itemize}
Pessôa, a quem se endossa uma letra.
\section{Endossador}
\begin{itemize}
\item {Grp. gram.:m.}
\end{itemize}
Aquelle que endossa; endossante.
\section{Endossamento}
\begin{itemize}
\item {Grp. gram.:m.}
\end{itemize}
Acto de endossar.
\section{Endossante}
\begin{itemize}
\item {Grp. gram.:m.  e  f.}
\end{itemize}
Pessôa, que endossa.
\section{Endossar}
\begin{itemize}
\item {Grp. gram.:v. t.}
\end{itemize}
\begin{itemize}
\item {Utilização:Fig.}
\end{itemize}
\begin{itemize}
\item {Proveniência:(Do b. lat. \textunderscore indorsare\textunderscore )}
\end{itemize}
Escrever no reverso de (uma letra de câmbio ou documento do mesmo gênero) o nome da pessôa, a cuja ordem deve sêr paga a quantia representada por êsse documento.
Escrever no reverso de (um titulo de crédito) o pertence, com que se transfere a outrem o direito representado por êsse título.
Transferir (encargo, responsabilidade, etc.).
\section{Endossatário}
\begin{itemize}
\item {Grp. gram.:m.}
\end{itemize}
O mesmo que \textunderscore endossado\textunderscore .
\section{Endosse}
\begin{itemize}
\item {Grp. gram.:m.}
\end{itemize}
(V.endôsso)
\section{Endôsso}
\begin{itemize}
\item {Grp. gram.:m.}
\end{itemize}
Declaração, escrita no reverso de uma letra ou título de crédito, com a qual se endossa o mesmo título ou letra.
Acto de endossar.
\section{Endóstoma}
\begin{itemize}
\item {Grp. gram.:f.}
\end{itemize}
\begin{itemize}
\item {Proveniência:(Do gr. \textunderscore endon\textunderscore  + \textunderscore stoma\textunderscore )}
\end{itemize}
Bôca interior do mycrópylo.
\section{Endotelial}
\begin{itemize}
\item {Grp. gram.:adj.}
\end{itemize}
Relativo ao endotélio.
\section{Endotélio}
\begin{itemize}
\item {Grp. gram.:m.}
\end{itemize}
\begin{itemize}
\item {Utilização:Physiol.}
\end{itemize}
\begin{itemize}
\item {Proveniência:(Do gr. \textunderscore endon\textunderscore  + \textunderscore thele\textunderscore )}
\end{itemize}
Camada celular, que forra interiormente os vasos orgânicos.
\section{Endotelioma}
\begin{itemize}
\item {Grp. gram.:m.}
\end{itemize}
\begin{itemize}
\item {Utilização:Med.}
\end{itemize}
\begin{itemize}
\item {Proveniência:(De \textunderscore endotélio\textunderscore )}
\end{itemize}
Tumor, formado de células epiteliaes.
\section{Endothelial}
\begin{itemize}
\item {Grp. gram.:adj.}
\end{itemize}
Relativo ao endothélio.
\section{Endothélio}
\begin{itemize}
\item {Grp. gram.:m.}
\end{itemize}
\begin{itemize}
\item {Utilização:Physiol.}
\end{itemize}
\begin{itemize}
\item {Proveniência:(Do gr. \textunderscore endon\textunderscore  + \textunderscore thele\textunderscore )}
\end{itemize}
Camada cellular, que forra interiormente os vasos orgânicos.
\section{Endothelioma}
\begin{itemize}
\item {Grp. gram.:m.}
\end{itemize}
\begin{itemize}
\item {Utilização:Med.}
\end{itemize}
\begin{itemize}
\item {Proveniência:(De \textunderscore endothélio\textunderscore )}
\end{itemize}
Tumor, formado de céllulas epitheliaes.
\section{Endouto}
\begin{itemize}
\item {Grp. gram.:adj.}
\end{itemize}
\begin{itemize}
\item {Utilização:Ant.}
\end{itemize}
\begin{itemize}
\item {Proveniência:(Do lat. \textunderscore inductus\textunderscore ?)}
\end{itemize}
Acostumado.
\section{Erísimo}
\begin{itemize}
\item {Grp. gram.:m.}
\end{itemize}
\begin{itemize}
\item {Proveniência:(Gr. \textunderscore erusímon\textunderscore )}
\end{itemize}
Planta crucífera, rinchão, (\textunderscore erysimum officinale\textunderscore , Lin.).
\section{Erisipela}
\begin{itemize}
\item {Grp. gram.:f.}
\end{itemize}
\begin{itemize}
\item {Proveniência:(Lat. \textunderscore erysipelas\textunderscore )}
\end{itemize}
Inflamação da pele, acompanhada de pequenas vesículas e quási sempre de febre geral.
\section{Erisipelar}
\begin{itemize}
\item {Grp. gram.:v. t.}
\end{itemize}
\begin{itemize}
\item {Grp. gram.:V. i.}
\end{itemize}
Promover erisipela a.
Criar erisipela.
\section{Erisipelatoso}
\begin{itemize}
\item {Grp. gram.:adj.}
\end{itemize}
Que tem carácter de erisipela.
Que sofre erisipela.
\section{Erisipeloso}
\begin{itemize}
\item {Grp. gram.:adj.}
\end{itemize}
O mesmo que \textunderscore erisipelatoso\textunderscore .
\section{Eritemático}
\begin{itemize}
\item {Grp. gram.:adj.}
\end{itemize}
Relativo a erithema.
\section{Eritematoso}
\begin{itemize}
\item {Grp. gram.:adj.}
\end{itemize}
Que sofre erithema; que tem o carácter de erithema.
\section{Erithema}
\begin{itemize}
\item {Grp. gram.:m.}
\end{itemize}
\begin{itemize}
\item {Proveniência:(Gr. \textunderscore eruthema\textunderscore )}
\end{itemize}
Exanthema não contagioso, caracterizado pelo aparecimento de manchas avermelhadas sôbre a pele.
\section{Eritrasma}
\begin{itemize}
\item {Grp. gram.:m.}
\end{itemize}
\begin{itemize}
\item {Utilização:Med.}
\end{itemize}
Infecção cutânea da região inguino-escrotal, produzida por um parasito, \textunderscore microsporon minutissimum\textunderscore .
\section{Eritrina}
\begin{itemize}
\item {Grp. gram.:f.}
\end{itemize}
\begin{itemize}
\item {Proveniência:(Do gr. \textunderscore eruthros\textunderscore )}
\end{itemize}
Substância colorante, extraida da urzela e que toma a côr roxa, sob a influência do ar e do amoniaco.
Designação científica da árvore-de-coral.
\section{Eritrite}
\begin{itemize}
\item {Grp. gram.:f.}
\end{itemize}
\begin{itemize}
\item {Utilização:Chím.}
\end{itemize}
\begin{itemize}
\item {Proveniência:(Do gr. \textunderscore eruthros\textunderscore , vermelho)}
\end{itemize}
Composição hidro-carbonada, que se extrae da urzela.
\section{Eritro...}
\begin{itemize}
\item {Grp. gram.:pref.}
\end{itemize}
\begin{itemize}
\item {Proveniência:(Gr. \textunderscore erutros\textunderscore )}
\end{itemize}
(significativo de \textunderscore vermelho\textunderscore )
\section{Eritrocarpo}
\begin{itemize}
\item {Grp. gram.:adj.}
\end{itemize}
\begin{itemize}
\item {Utilização:Bot.}
\end{itemize}
\begin{itemize}
\item {Proveniência:(Do gr. \textunderscore eruthros\textunderscore  + \textunderscore karpos\textunderscore )}
\end{itemize}
Que tem frutos vermelhos.
\section{Eritrócero}
\begin{itemize}
\item {Grp. gram.:adj.}
\end{itemize}
\begin{itemize}
\item {Proveniência:(Do gr. \textunderscore eruthros\textunderscore  + \textunderscore keras\textunderscore )}
\end{itemize}
Que tem antenas vermelhas.
\section{Eritrocito}
\begin{itemize}
\item {Grp. gram.:m.}
\end{itemize}
Glóbulo vermelho de sangue; hematia.
\section{Eritrodermo}
\begin{itemize}
\item {Grp. gram.:adj.}
\end{itemize}
\begin{itemize}
\item {Proveniência:(Do gr. \textunderscore eruthros\textunderscore  + \textunderscore derma\textunderscore )}
\end{itemize}
Que tem pele vermelha.
\section{Eritrofila}
\begin{itemize}
\item {Grp. gram.:f.}
\end{itemize}
\begin{itemize}
\item {Proveniência:(Do gr. \textunderscore eruthros\textunderscore  + \textunderscore phullon\textunderscore )}
\end{itemize}
Substância còrante, vermelha, das fôlhas dos vegetaes.
\section{Eritrofilo}
\begin{itemize}
\item {Grp. gram.:adj.}
\end{itemize}
\begin{itemize}
\item {Proveniência:(Do gr. \textunderscore eruthros\textunderscore  + \textunderscore phullon\textunderscore )}
\end{itemize}
Que tem fôlhas vermelhas.
\section{Eritrofleína}
\begin{itemize}
\item {Grp. gram.:f.}
\end{itemize}
Alcaloide medicinal, contra as afecções cardíacas.
\section{Eritrogastro}
\begin{itemize}
\item {Grp. gram.:adj.}
\end{itemize}
\begin{itemize}
\item {Proveniência:(Do gr. \textunderscore eruthros\textunderscore  + \textunderscore gaster\textunderscore )}
\end{itemize}
Que tem o ventre vermelho.
\section{Eritroide}
\begin{itemize}
\item {Grp. gram.:adj.}
\end{itemize}
\begin{itemize}
\item {Proveniência:(Do gr. \textunderscore eruthros\textunderscore  + \textunderscore eidos\textunderscore )}
\end{itemize}
Que tem côr avermelhada.
\section{Eritrólofo}
\begin{itemize}
\item {Grp. gram.:adj.}
\end{itemize}
\begin{itemize}
\item {Utilização:Zool.}
\end{itemize}
\begin{itemize}
\item {Proveniência:(Do gr. \textunderscore eruthros\textunderscore  + \textunderscore lophos\textunderscore )}
\end{itemize}
Que tem popa vermelha.
\section{Eritrópode}
\begin{itemize}
\item {Grp. gram.:adj.}
\end{itemize}
\begin{itemize}
\item {Utilização:Zool.}
\end{itemize}
\begin{itemize}
\item {Proveniência:(Do gr. \textunderscore eruthros\textunderscore  + \textunderscore pous\textunderscore )}
\end{itemize}
Que tem os pés vermelhos.
\section{Eritropsia}
\begin{itemize}
\item {Grp. gram.:f.}
\end{itemize}
\begin{itemize}
\item {Proveniência:(Do gr. \textunderscore eruthros\textunderscore  + \textunderscore ops\textunderscore )}
\end{itemize}
Estado mórbido de quem vé tudo vermelho.
\section{Eritróptero}
\begin{itemize}
\item {Grp. gram.:adj.}
\end{itemize}
\begin{itemize}
\item {Utilização:Zool.}
\end{itemize}
\begin{itemize}
\item {Proveniência:(Do gr. \textunderscore eruthros\textunderscore  + \textunderscore pteron\textunderscore )}
\end{itemize}
Que tem asas vermelhas.
\section{Eritróptico}
\begin{itemize}
\item {Grp. gram.:adj.}
\end{itemize}
Relativo á eritropsia; que sofre eritropsia.
\section{Eritrose}
\begin{itemize}
\item {Grp. gram.:f.}
\end{itemize}
\begin{itemize}
\item {Proveniência:(Do gr. \textunderscore eruthros\textunderscore )}
\end{itemize}
Matéria còrante, extraida de várias espécies de ruibado, por meio do ácido nítrico.
\section{Eritrospermo}
\begin{itemize}
\item {Grp. gram.:adj.}
\end{itemize}
\begin{itemize}
\item {Grp. gram.:M.}
\end{itemize}
\begin{itemize}
\item {Proveniência:(Do gr. \textunderscore eruthros\textunderscore  + \textunderscore sperma\textunderscore )}
\end{itemize}
Que tem grãos vermelhos.
Gênero de plantas.
\section{Eritróstomo}
\begin{itemize}
\item {Grp. gram.:adj.}
\end{itemize}
\begin{itemize}
\item {Utilização:Zool.}
\end{itemize}
\begin{itemize}
\item {Proveniência:(Do gr. \textunderscore eruthros\textunderscore  + \textunderscore stoma\textunderscore )}
\end{itemize}
Que tem bôca vermelha.
\section{Eritrotórace}
\begin{itemize}
\item {Grp. gram.:adj.}
\end{itemize}
\begin{itemize}
\item {Utilização:Zool.}
\end{itemize}
\begin{itemize}
\item {Proveniência:(Do gr. \textunderscore eruthros\textunderscore  + \textunderscore thorax\textunderscore )}
\end{itemize}
Que tem peito vermelho.
\section{Eritroxíleas}
\begin{itemize}
\item {Grp. gram.:f. pl.}
\end{itemize}
\begin{itemize}
\item {Proveniência:(De \textunderscore ertrôxilo\textunderscore )}
\end{itemize}
Família de plantas dicotiledóneas polipétalas, de estames hipóginos.
\section{Eritróxilo}
\begin{itemize}
\item {Grp. gram.:adj.}
\end{itemize}
\begin{itemize}
\item {Proveniência:(Do gr. \textunderscore eruthron\textunderscore  + \textunderscore xulon\textunderscore )}
\end{itemize}
Cuja madeira é vermelha.
\section{Erva-tostão}
\begin{itemize}
\item {Grp. gram.:f.}
\end{itemize}
Planta nyctagínea, (\textunderscore boerhavia irsutha\textunderscore ).
\section{Erva-traqueira}
\begin{itemize}
\item {Grp. gram.:f.}
\end{itemize}
Planta dianthácea, (\textunderscore silene tenosa\textunderscore , Gilib.).
\section{Erva-turca}
\begin{itemize}
\item {Grp. gram.:f.}
\end{itemize}
Planta cariophyllácea, (\textunderscore herniaria glabra\textunderscore ).
\section{Erva-ulmeira}
\begin{itemize}
\item {Grp. gram.:f.}
\end{itemize}
Planta rosácea, (\textunderscore spiraea ulmaria\textunderscore , Lin.).
\section{Erva-ussa}
\begin{itemize}
\item {Grp. gram.:f.}
\end{itemize}
\begin{itemize}
\item {Utilização:Bras}
\end{itemize}
O mesmo que \textunderscore serpol\textunderscore .
\section{Erva-vaqueira}
\begin{itemize}
\item {Grp. gram.:f.}
\end{itemize}
Planta, da fam. das compostas, (\textunderscore calendula arvensis\textunderscore , Lin.).
\section{Erva-venenosa}
\begin{itemize}
\item {Grp. gram.:f.}
\end{itemize}
Planta apocínea, (\textunderscore echites venenosa\textunderscore ).
\section{Erva-vespa}
\begin{itemize}
\item {Grp. gram.:f.}
\end{itemize}
Nome vulgar de uma planta (\textunderscore ophys luteia\textunderscore ).
\section{Erva-viperina}
\begin{itemize}
\item {Grp. gram.:f.}
\end{itemize}
Planta asperifólia.
\section{Ervecer}
\begin{itemize}
\item {Grp. gram.:v. i.}
\end{itemize}
Produzir erva.
\section{Ervecido}
\begin{itemize}
\item {Grp. gram.:adj.}
\end{itemize}
\begin{itemize}
\item {Proveniência:(De \textunderscore ervecer\textunderscore )}
\end{itemize}
Coberto de erva.
Em que cresce erva. Cf. Camillo, \textunderscore Brasileira\textunderscore , 62.
\section{Ervedal}
\begin{itemize}
\item {Grp. gram.:m.}
\end{itemize}
Lugar onde crescem érvedos.
\section{Ervedeiro}
\begin{itemize}
\item {Grp. gram.:m.}
\end{itemize}
Planta ericácea, medronheiro.
\section{Érvedo}
\begin{itemize}
\item {Grp. gram.:m.}
\end{itemize}
O mesmo que \textunderscore ervedeiro\textunderscore  e que \textunderscore érvodo\textunderscore .(V.érvodo)
\section{Erveira}
\begin{itemize}
\item {Grp. gram.:f.}
\end{itemize}
\begin{itemize}
\item {Utilização:Prov.}
\end{itemize}
\begin{itemize}
\item {Proveniência:(De \textunderscore erva\textunderscore )}
\end{itemize}
Designação genêrica das plantas herbáceas, consideradas insuladamente.
\section{Erveiro}
\begin{itemize}
\item {Grp. gram.:adj.}
\end{itemize}
Que tem natureza da erva.
Em que cresce erva, ervecido. Cf. \textunderscore Bibl. da G. do Campo\textunderscore , 363.
\section{Erviço}
\begin{itemize}
\item {Grp. gram.:m.  e  adj.}
\end{itemize}
\begin{itemize}
\item {Utilização:Prov.}
\end{itemize}
\begin{itemize}
\item {Utilização:alent.}
\end{itemize}
\begin{itemize}
\item {Proveniência:(De \textunderscore erva\textunderscore )}
\end{itemize}
Bácoro nascido na primavera.
\section{Ervilha}
\begin{itemize}
\item {Grp. gram.:f.}
\end{itemize}
\begin{itemize}
\item {Grp. gram.:Pl.}
\end{itemize}
\begin{itemize}
\item {Utilização:Prov.}
\end{itemize}
\begin{itemize}
\item {Utilização:alent.}
\end{itemize}
\begin{itemize}
\item {Proveniência:(Lat. \textunderscore ervilia\textunderscore )}
\end{itemize}
Planta leguminosa, de que há várias especies.
Vagem e semente da ervilha.
Doce de grão de bico, envolvido numa capa de massa em fórma de ervilha.
\section{Ervilhaca}
\begin{itemize}
\item {Grp. gram.:f.}
\end{itemize}
\begin{itemize}
\item {Proveniência:(Do b. lat. \textunderscore erviliacea\textunderscore )}
\end{itemize}
Planta leguminosa que damnifica as searas, mas que serve para forragens.
\section{Ervilha-de-pomba}
\begin{itemize}
\item {Grp. gram.:f.}
\end{itemize}
Planta leguminosa, (\textunderscore ervum ervilix\textunderscore ).
\section{Ervilhado}
\begin{itemize}
\item {Grp. gram.:adj.}
\end{itemize}
\begin{itemize}
\item {Utilização:Prov.}
\end{itemize}
\begin{itemize}
\item {Utilização:beir.}
\end{itemize}
\begin{itemize}
\item {Proveniência:(De \textunderscore ervilhar\textunderscore )}
\end{itemize}
Que diz tolices, apatetado.
\section{Ervilhal}
\begin{itemize}
\item {Grp. gram.:m.}
\end{itemize}
Campo de ervilhas.
\section{Ervilhar}
\begin{itemize}
\item {Grp. gram.:v. i.}
\end{itemize}
\begin{itemize}
\item {Utilização:Des.}
\end{itemize}
Dizer despropósitos.
Fazer tolice.
Têr appetites inopportunos ou disparatados.
\section{Ervília}
\begin{itemize}
\item {Grp. gram.:f.}
\end{itemize}
Genero de molluscos.
Gênero de infusórios.
\section{Ervinha}
\begin{itemize}
\item {Grp. gram.:f.}
\end{itemize}
O mesmo que \textunderscore alfarva\textunderscore .
\section{Ervo}
\begin{itemize}
\item {Grp. gram.:m.}
\end{itemize}
Genero de plantas papilionáceas.
\section{Ervoado}
\begin{itemize}
\item {Grp. gram.:adj.}
\end{itemize}
\begin{itemize}
\item {Utilização:Ant.}
\end{itemize}
Ponto, perturbado.
(Por \textunderscore arvoado?\textunderscore )
\section{Érvodo}
\begin{itemize}
\item {Grp. gram.:m.}
\end{itemize}
\begin{itemize}
\item {Proveniência:(Lat. \textunderscore arbutus\textunderscore )}
\end{itemize}
O mesmo que \textunderscore medronheiro\textunderscore .
\section{Ervoeira}
\begin{itemize}
\item {Grp. gram.:f.}
\end{itemize}
\begin{itemize}
\item {Utilização:Ant.}
\end{itemize}
O mesmo que \textunderscore rameira\textunderscore .
\section{Ervoso}
\begin{itemize}
\item {Grp. gram.:adj.}
\end{itemize}
\begin{itemize}
\item {Proveniência:(Lat. \textunderscore herbosus\textunderscore )}
\end{itemize}
Em que cresce muita erva.
Relvoso.
\section{Erýsimo}
\begin{itemize}
\item {Grp. gram.:m.}
\end{itemize}
\begin{itemize}
\item {Proveniência:(Gr. \textunderscore erusímon\textunderscore )}
\end{itemize}
Planta crucífera, rinchão, (\textunderscore erysimum officinale\textunderscore , Lin.).
\section{Erysipela}
\begin{itemize}
\item {Grp. gram.:f.}
\end{itemize}
\begin{itemize}
\item {Proveniência:(Lat. \textunderscore erysipelas\textunderscore )}
\end{itemize}
Inflammação da pelle, acompanhada de pequenas vesículas e quási sempre de febre geral.
\section{Erysipelar}
\begin{itemize}
\item {Grp. gram.:v. t.}
\end{itemize}
\begin{itemize}
\item {Grp. gram.:V. i.}
\end{itemize}
Promover erysipela a.
Criar erysipela.
\section{Erysipelatoso}
\begin{itemize}
\item {Grp. gram.:adj.}
\end{itemize}
Que tem carácter de erysipela.
Que soffre erysipela.
\section{Erysipeloso}
\begin{itemize}
\item {Grp. gram.:adj.}
\end{itemize}
O mesmo que \textunderscore erysipelatoso\textunderscore .
\section{Erythema}
\begin{itemize}
\item {Grp. gram.:m.}
\end{itemize}
\begin{itemize}
\item {Proveniência:(Gr. \textunderscore eruthema\textunderscore )}
\end{itemize}
Exanthema não contagioso, caracterizado pelo apparecimento de manchas avermelhadas sôbre a pelle.
\section{Erythemático}
\begin{itemize}
\item {Grp. gram.:adj.}
\end{itemize}
Relativo a erythema.
\section{Erythematoso}
\begin{itemize}
\item {Grp. gram.:adj.}
\end{itemize}
Que soffre erythema; que tem o carácter de erythema.
\section{Erythrasma}
\begin{itemize}
\item {Grp. gram.:m.}
\end{itemize}
\begin{itemize}
\item {Utilização:Med.}
\end{itemize}
Infecção cutânea da região inguino-escrotal, produzida por um parasito, \textunderscore microsporon minutissimum\textunderscore .
\section{Erythrina}
\begin{itemize}
\item {Grp. gram.:f.}
\end{itemize}
\begin{itemize}
\item {Proveniência:(Do gr. \textunderscore eruthros\textunderscore )}
\end{itemize}
Substância colorante, extrahida da urzella e que toma a côr roxa, sob a influência do ar e do ammoniaco.
Designação scientífica da árvore-de-coral.
\section{Erythrite}
\begin{itemize}
\item {Grp. gram.:f.}
\end{itemize}
\begin{itemize}
\item {Utilização:Chím.}
\end{itemize}
\begin{itemize}
\item {Proveniência:(Do gr. \textunderscore eruthros\textunderscore , vermelho)}
\end{itemize}
Composição hydro-carbonada, que se extrae da urzella.
\section{Erythro...}
\begin{itemize}
\item {Grp. gram.:pref.}
\end{itemize}
\begin{itemize}
\item {Proveniência:(Gr. \textunderscore erutros\textunderscore )}
\end{itemize}
(significativo de \textunderscore vermelho\textunderscore )
\section{Erythrocarpo}
\begin{itemize}
\item {Grp. gram.:adj.}
\end{itemize}
\begin{itemize}
\item {Utilização:Bot.}
\end{itemize}
\begin{itemize}
\item {Proveniência:(Do gr. \textunderscore eruthros\textunderscore  + \textunderscore karpos\textunderscore )}
\end{itemize}
Que tem frutos vermelhos.
\section{Erythrócero}
\begin{itemize}
\item {Grp. gram.:adj.}
\end{itemize}
\begin{itemize}
\item {Proveniência:(Do gr. \textunderscore eruthros\textunderscore  + \textunderscore keras\textunderscore )}
\end{itemize}
Que tem antennas vermelhas.
\section{Erythrocyto}
\begin{itemize}
\item {Grp. gram.:m.}
\end{itemize}
Glóbulo vermelho de sangue; hematia.
\section{Erythrodermo}
\begin{itemize}
\item {Grp. gram.:adj.}
\end{itemize}
\begin{itemize}
\item {Proveniência:(Do gr. \textunderscore eruthros\textunderscore  + \textunderscore derma\textunderscore )}
\end{itemize}
Que tem pelle vermelha.
\section{Erythrogastro}
\begin{itemize}
\item {Grp. gram.:adj.}
\end{itemize}
\begin{itemize}
\item {Proveniência:(Do gr. \textunderscore eruthros\textunderscore  + \textunderscore gaster\textunderscore )}
\end{itemize}
Que tem o ventre vermelho.
\section{Erythroide}
\begin{itemize}
\item {Grp. gram.:adj.}
\end{itemize}
\begin{itemize}
\item {Proveniência:(Do gr. \textunderscore eruthros\textunderscore  + \textunderscore eidos\textunderscore )}
\end{itemize}
Que tem côr avermelhada.
\section{Erythrólopho}
\begin{itemize}
\item {Grp. gram.:adj.}
\end{itemize}
\begin{itemize}
\item {Utilização:Zool.}
\end{itemize}
\begin{itemize}
\item {Proveniência:(Do gr. \textunderscore eruthros\textunderscore  + \textunderscore lophos\textunderscore )}
\end{itemize}
Que tem popa vermelha.
\section{Erythrophleína}
\begin{itemize}
\item {Grp. gram.:f.}
\end{itemize}
Alcaloide medicinal, contra as affecções cardíacas.
\section{Erythrophylla}
\begin{itemize}
\item {Grp. gram.:f.}
\end{itemize}
\begin{itemize}
\item {Proveniência:(Do gr. \textunderscore eruthros\textunderscore  + \textunderscore phullon\textunderscore )}
\end{itemize}
Substância còrante, vermelha, das fôlhas dos vegetaes.
\section{Erythrophyllo}
\begin{itemize}
\item {Grp. gram.:adj.}
\end{itemize}
\begin{itemize}
\item {Proveniência:(Do gr. \textunderscore eruthros\textunderscore  + \textunderscore phullon\textunderscore )}
\end{itemize}
Que tem fôlhas vermelhas.
\section{Erythrópode}
\begin{itemize}
\item {Grp. gram.:adj.}
\end{itemize}
\begin{itemize}
\item {Utilização:Zool.}
\end{itemize}
\begin{itemize}
\item {Proveniência:(Do gr. \textunderscore eruthros\textunderscore  + \textunderscore pous\textunderscore )}
\end{itemize}
Que tem os pés vermelhos.
\section{Erythropsia}
\begin{itemize}
\item {Grp. gram.:f.}
\end{itemize}
\begin{itemize}
\item {Proveniência:(Do gr. \textunderscore eruthros\textunderscore  + \textunderscore ops\textunderscore )}
\end{itemize}
Estado mórbido de quem vé tudo vermelho.
\section{Erythróptero}
\begin{itemize}
\item {Grp. gram.:adj.}
\end{itemize}
\begin{itemize}
\item {Utilização:Zool.}
\end{itemize}
\begin{itemize}
\item {Proveniência:(Do gr. \textunderscore eruthros\textunderscore  + \textunderscore pteron\textunderscore )}
\end{itemize}
Que tem asas vermelhas.
\section{Erythróptico}
\begin{itemize}
\item {Grp. gram.:adj.}
\end{itemize}
Relativo á erythropsia; que soffre erythropsia.
\section{Erythrose}
\begin{itemize}
\item {Grp. gram.:f.}
\end{itemize}
\begin{itemize}
\item {Proveniência:(Do gr. \textunderscore eruthros\textunderscore )}
\end{itemize}
Matéria còrante, extrahida de várias espécies de ruibado, por meio do ácido nítrico.
\section{Erythrospermo}
\begin{itemize}
\item {Grp. gram.:adj.}
\end{itemize}
\begin{itemize}
\item {Grp. gram.:M.}
\end{itemize}
\begin{itemize}
\item {Proveniência:(Do gr. \textunderscore eruthros\textunderscore  + \textunderscore sperma\textunderscore )}
\end{itemize}
Que tem grãos vermelhos.
Gênero de plantas.
\section{Erythróstomo}
\begin{itemize}
\item {Grp. gram.:adj.}
\end{itemize}
\begin{itemize}
\item {Utilização:Zool.}
\end{itemize}
\begin{itemize}
\item {Proveniência:(Do gr. \textunderscore eruthros\textunderscore  + \textunderscore stoma\textunderscore )}
\end{itemize}
Que tem bôca vermelha.
\section{Erythrothórace}
\begin{itemize}
\item {Grp. gram.:adj.}
\end{itemize}
\begin{itemize}
\item {Utilização:Zool.}
\end{itemize}
\begin{itemize}
\item {Proveniência:(Do gr. \textunderscore eruthros\textunderscore  + \textunderscore thorax\textunderscore )}
\end{itemize}
Que tem peito vermelho.
\section{Erythroxýleas}
\begin{itemize}
\item {Grp. gram.:f. pl.}
\end{itemize}
\begin{itemize}
\item {Proveniência:(De \textunderscore erythrôxylo\textunderscore )}
\end{itemize}
Família de plantas dicotyledóneas polypétalas, de estames hypógynos.
\section{Erythróxylo}
\begin{itemize}
\item {Grp. gram.:adj.}
\end{itemize}
\begin{itemize}
\item {Proveniência:(Do gr. \textunderscore eruthron\textunderscore  + \textunderscore xulon\textunderscore )}
\end{itemize}
Cuja madeira é vermelha.
\section{Erzipela}
\begin{itemize}
\item {Grp. gram.:f.}
\end{itemize}
(Fórma pop. de erysipela)
\section{Es...}
\begin{itemize}
\item {Grp. gram.:pref.}
\end{itemize}
(indicativo de mudança, separação, estado, saída, etc.)
\section{...ês}
\begin{itemize}
\item {Grp. gram.:suf. ,  m.  e  adj.}
\end{itemize}
\begin{itemize}
\item {Proveniência:(Do lat. \textunderscore ...ensis\textunderscore )}
\end{itemize}
(designação de procedência, naturalidade, etc., como em \textunderscore português\textunderscore , \textunderscore cortês\textunderscore , etc.)
\section{Esbabacado}
\begin{itemize}
\item {Grp. gram.:m.}
\end{itemize}
\begin{itemize}
\item {Utilização:Ant.}
\end{itemize}
O mesmo que [[embasbacado|embasbacar]]. Cf. \textunderscore Eufrosina\textunderscore , 152.
\section{Esbabacar}
\begin{itemize}
\item {Grp. gram.:v. i.}
\end{itemize}
\begin{itemize}
\item {Utilização:Ant.}
\end{itemize}
O mesmo que \textunderscore embasbacar\textunderscore .
\section{Esbadanado}
\begin{itemize}
\item {Grp. gram.:adj.}
\end{itemize}
\begin{itemize}
\item {Utilização:Prov.}
\end{itemize}
\begin{itemize}
\item {Utilização:trasm.}
\end{itemize}
Diz-se do chapéu que tem a aba descaída; desabado.
\section{Esbaforiar}
\begin{itemize}
\item {Grp. gram.:v. i.}
\end{itemize}
O mesmo que \textunderscore esbaforir\textunderscore . Us. por Castilho.
\section{Esbaforir}
\begin{itemize}
\item {Grp. gram.:v. i.}
\end{itemize}
O mesmo que \textunderscore esbaforir-se\textunderscore . Cf. Rui Barb., \textunderscore Répl.\textunderscore , 160.
\section{Esbaforir-se}
\begin{itemize}
\item {Grp. gram.:v. p.}
\end{itemize}
Estar offegante.
Têr a respiração entrecortada, em consequência de cansaço.
(Talvez corr. de \textunderscore espavorir-se\textunderscore )
\section{Esbagachado}
\begin{itemize}
\item {Grp. gram.:adj.}
\end{itemize}
\begin{itemize}
\item {Utilização:Des.}
\end{itemize}
\begin{itemize}
\item {Proveniência:(Do it. \textunderscore bagascia\textunderscore )}
\end{itemize}
Muito decotado.
Descoberto no pescoço até o peito.
Impudico. Cf. Camillo, \textunderscore Corja\textunderscore , 22.
\section{Esbaganhar}
\begin{itemize}
\item {Grp. gram.:v. t.}
\end{itemize}
Tirar a baganha a (o linho).
\section{Esbagoar}
\begin{itemize}
\item {Grp. gram.:v. t.}
\end{itemize}
\begin{itemize}
\item {Grp. gram.:V. i.}
\end{itemize}
\begin{itemize}
\item {Utilização:Prov.}
\end{itemize}
\begin{itemize}
\item {Utilização:Fig.}
\end{itemize}
Tirar os bagos a: \textunderscore esbagoar uma roman\textunderscore .
Limpar dos grãos: \textunderscore esbagoar as maçarocas do milho\textunderscore .
Deixar cair o bago ou o grão.
Chorar.
\section{Esbagulhar}
\begin{itemize}
\item {Grp. gram.:v. t.}
\end{itemize}
Tirar o bagulho a.
\section{Esbaldir}
\begin{itemize}
\item {Grp. gram.:v. t.}
\end{itemize}
\begin{itemize}
\item {Utilização:Obsol.}
\end{itemize}
\begin{itemize}
\item {Proveniência:(De \textunderscore baldo\textunderscore )}
\end{itemize}
Dissipar, esbanjar.
\section{Esbalgideira}
\begin{itemize}
\item {Grp. gram.:f.  e  adj.}
\end{itemize}
Mulher ou rapariga que esbalge.
\section{Esbalgidor}
\begin{itemize}
\item {Grp. gram.:adj.}
\end{itemize}
\begin{itemize}
\item {Grp. gram.:M.}
\end{itemize}
Que esbalge.
Aquelle que esbalge.
\section{Esbalgir}
\begin{itemize}
\item {Grp. gram.:v. t.}
\end{itemize}
\begin{itemize}
\item {Utilização:Prov.}
\end{itemize}
\begin{itemize}
\item {Utilização:trasm.}
\end{itemize}
Dissipar, esbanjar.
(Corr. de \textunderscore esbaldir\textunderscore )
\section{Esbalurtado}
\begin{itemize}
\item {Grp. gram.:adj.}
\end{itemize}
\begin{itemize}
\item {Utilização:Prov.}
\end{itemize}
\begin{itemize}
\item {Utilização:trasm.}
\end{itemize}
Talado, devassado, (falando-se de terrenos).
Desguarnecido.
Assolado ou gasto por paixões, (falando-se do coração).
(Por \textunderscore esbaluartado\textunderscore , de \textunderscore baluarte\textunderscore )
\section{Esbambar}
\begin{itemize}
\item {Grp. gram.:v. t.}
\end{itemize}
\begin{itemize}
\item {Utilização:Prov.}
\end{itemize}
\begin{itemize}
\item {Utilização:trasm.}
\end{itemize}
Retesar ou esticar (pano, corda, etc).
(Por \textunderscore desbambar\textunderscore , de \textunderscore bambo\textunderscore )
\section{Esbambear}
\begin{itemize}
\item {Grp. gram.:v. t.}
\end{itemize}
O mesmo que \textunderscore bambear\textunderscore . Cf. Garrett, \textunderscore Flores sem Fruto\textunderscore , 60.
\section{Esbamboar-se}
\begin{itemize}
\item {Grp. gram.:v. p.}
\end{itemize}
Saracotear-se. Cf. Camillo, \textunderscore Corja\textunderscore , 169.
\section{Esbandalhado}
\begin{itemize}
\item {Grp. gram.:adj.}
\end{itemize}
Próprio de bandalho. Cf. Ortigão, \textunderscore Praias\textunderscore , 106.
\section{Esbandalhar}
\begin{itemize}
\item {Grp. gram.:v. t.}
\end{itemize}
Fazer em bandalhos, esfarrapar.
Desmanchar.
Destruir.
Pôr em debandada, dispersar.
Descompor: \textunderscore esbandalhar o cabello\textunderscore .
\section{Esbandeirar}
\begin{itemize}
\item {Grp. gram.:v.}
\end{itemize}
\begin{itemize}
\item {Utilização:t. Agr.}
\end{itemize}
Cortar a bandeira a (o milho).
\section{Esbandulhar}
\begin{itemize}
\item {Grp. gram.:v. t.}
\end{itemize}
Rasgar o bandulho de; esbarrigar. Cf. Camillo, \textunderscore Cancion. Al.\textunderscore , 425.
\section{Esbanjador}
\begin{itemize}
\item {Grp. gram.:adj.}
\end{itemize}
\begin{itemize}
\item {Grp. gram.:M.}
\end{itemize}
Que esbanja.
Aquelle que esbanja.
\section{Esbanjamento}
\begin{itemize}
\item {Grp. gram.:m.}
\end{itemize}
Acto ou effeito de esbanjar.
\section{Esbanjar}
\begin{itemize}
\item {Grp. gram.:v. t.}
\end{itemize}
Gastar perdulariamente.
Desbaratar; consumir á tôa.
\section{Esbanzalhado}
\begin{itemize}
\item {Grp. gram.:adj.}
\end{itemize}
\begin{itemize}
\item {Utilização:Prov.}
\end{itemize}
\begin{itemize}
\item {Utilização:trasm.}
\end{itemize}
Froixo, lasso, que tem quebreira de corpo.
Que tem alguns membros desarticulados, como esquecidos ou mortos.
(Provavelmente, corr. de \textunderscore esbandalhado\textunderscore )
\section{Esbarafustar}
\begin{itemize}
\item {Grp. gram.:v. i.}
\end{itemize}
\begin{itemize}
\item {Utilização:Des.}
\end{itemize}
O mesmo que \textunderscore barafustar\textunderscore . Cf. Macedo, \textunderscore Burros\textunderscore , 227.
\section{Esbaralhar}
\begin{itemize}
\item {Grp. gram.:v. t.}
\end{itemize}
\begin{itemize}
\item {Utilização:Des.}
\end{itemize}
(V.baralhar)
\section{Esbarar}
\begin{itemize}
\item {Grp. gram.:v. t.}
\end{itemize}
\begin{itemize}
\item {Utilização:Prov.}
\end{itemize}
\begin{itemize}
\item {Utilização:trasm.}
\end{itemize}
O mesmo que \textunderscore escorregar\textunderscore .
(Relaciona-se com \textunderscore esbarrar\textunderscore ?)
\section{Esbarbar}
\begin{itemize}
\item {Grp. gram.:v. t.}
\end{itemize}
Tirar as barbas, a crespidão, a aspereza a.
(Por \textunderscore desbarbar\textunderscore , de \textunderscore barba\textunderscore )
\section{Esbarbotar}
\begin{itemize}
\item {Grp. gram.:v. t.}
\end{itemize}
Tirar os barbotes a (pano de lan).
\section{Esbardar}
\begin{itemize}
\item {Grp. gram.:v. t.}
\end{itemize}
\begin{itemize}
\item {Utilização:Prov.}
\end{itemize}
\begin{itemize}
\item {Utilização:minh.}
\end{itemize}
Espalhar.
\section{Esbarrada}
\begin{itemize}
\item {Grp. gram.:f.}
\end{itemize}
\begin{itemize}
\item {Utilização:Prov.}
\end{itemize}
\begin{itemize}
\item {Utilização:trasm.}
\end{itemize}
\begin{itemize}
\item {Proveniência:(De \textunderscore esbarrar\textunderscore )}
\end{itemize}
Terra ou pedra caída, num desmoronamento.
Caminho escabroso.
\section{Esbarrar}
\begin{itemize}
\item {Grp. gram.:v. t.}
\end{itemize}
\begin{itemize}
\item {Grp. gram.:V. i.}
\end{itemize}
\begin{itemize}
\item {Proveniência:(De \textunderscore barra\textunderscore )}
\end{itemize}
Arremessar.
Embarrar; topar.
Ir de encontro; encontrar-se.
\section{Esbarrigado}
\begin{itemize}
\item {Grp. gram.:adj.}
\end{itemize}
\begin{itemize}
\item {Utilização:Fig.}
\end{itemize}
Bojudo.
Saliente; proeminente;«\textunderscore ...olhos esbarrigados...\textunderscore »\textunderscore Anat. Joc.\textunderscore , I, 258.
\section{Esbarrigar}
\begin{itemize}
\item {Grp. gram.:v. t.}
\end{itemize}
\begin{itemize}
\item {Utilização:Prov.}
\end{itemize}
\begin{itemize}
\item {Utilização:alent.}
\end{itemize}
Parir.
Rasgar o ventre de:«\textunderscore a última tripa de um cónego ou de um rei esbarrigado...\textunderscore »Cf. Macedo, \textunderscore Burros\textunderscore , V, 304.
\section{Esbarro}
\begin{itemize}
\item {Grp. gram.:m.}
\end{itemize}
\begin{itemize}
\item {Utilização:Bras}
\end{itemize}
Inclinação dos resaltos de uma pilastra.
Degrau inclinado, que uma parede forma, deminuindo de espessura.
Acto de esbarrar.
\section{Esbarrocamento}
\begin{itemize}
\item {Grp. gram.:m.}
\end{itemize}
Acto ou effeito de esbarrocar.
\section{Esbarrocar}
\begin{itemize}
\item {Grp. gram.:v. i.  e  p.}
\end{itemize}
\begin{itemize}
\item {Proveniência:(De \textunderscore barroca\textunderscore )}
\end{itemize}
Despenhar-se.
Caír, formando barroca.
Esbarrondar-se.
\section{Esbarrondadeiro}
\begin{itemize}
\item {Grp. gram.:m.}
\end{itemize}
\begin{itemize}
\item {Proveniência:(De \textunderscore esbarrondar\textunderscore )}
\end{itemize}
Lugar, donde é fácil despenhar-se alguém.
\section{Esbarrondamento}
\begin{itemize}
\item {Grp. gram.:m.}
\end{itemize}
Acto ou effeito de esbarrondar.
\section{Esbarrondar}
\begin{itemize}
\item {Grp. gram.:v. t.}
\end{itemize}
\begin{itemize}
\item {Grp. gram.:V. i.  e  p.}
\end{itemize}
\begin{itemize}
\item {Grp. gram.:V. p.}
\end{itemize}
\begin{itemize}
\item {Utilização:Prov.}
\end{itemize}
\begin{itemize}
\item {Proveniência:(Do rad. de \textunderscore barro\textunderscore )}
\end{itemize}
Romper, esboroar.
Caír de despenhadeiro.
Desmoronar-se, esboroar-se.
Parir. (Colhido no Fundão)
\section{Esbarrondo}
\begin{itemize}
\item {Grp. gram.:m.}
\end{itemize}
\begin{itemize}
\item {Utilização:Fam.}
\end{itemize}
\begin{itemize}
\item {Proveniência:(De \textunderscore esbarrondar\textunderscore )}
\end{itemize}
Quebra de relações, acompanhada da arguição de injúrias recebidas.
\section{Esbarrunto}
\begin{itemize}
\item {Grp. gram.:m. Loc. adv.}
\end{itemize}
\begin{itemize}
\item {Utilização:alent}
\end{itemize}
\textunderscore De esbarrunto\textunderscore , de modo extraordinário; de arromba.
\section{Esbater}
\begin{itemize}
\item {Grp. gram.:v. t.}
\end{itemize}
\begin{itemize}
\item {Grp. gram.:V. p.}
\end{itemize}
\begin{itemize}
\item {Proveniência:(De \textunderscore bater\textunderscore )}
\end{itemize}
Dar relêvo a.
Graduar as sombras e o claro-escuro de (um quadro).
Adelgaçar ou atenuar (a côr):«\textunderscore os vinhos de Carcavellos esbatem depressa a côr e ganham em perfume.\textunderscore »F. Lapa, \textunderscore Alm. do Lavr.\textunderscore , 27.
Espalhar-se, dispor-se gradualmente, resaíndo.
\section{Esbatimento}
\begin{itemize}
\item {Grp. gram.:m.}
\end{itemize}
Acto ou effeito de esbater.
\section{Esbeatar}
\begin{itemize}
\item {Grp. gram.:v. t.}
\end{itemize}
\begin{itemize}
\item {Utilização:Bras. do N}
\end{itemize}
\begin{itemize}
\item {Proveniência:(De \textunderscore beato\textunderscore ^2)}
\end{itemize}
Destramar os fios de (um tecido); desfiar.
\section{Esbeiçamento}
\begin{itemize}
\item {Grp. gram.:m.}
\end{itemize}
Acto ou effeito de esbeiçar. Cf. Camillo, \textunderscore Eus. Macário\textunderscore , 160.
\section{Esbeiçar}
\begin{itemize}
\item {Grp. gram.:v. i.}
\end{itemize}
\begin{itemize}
\item {Grp. gram.:V. t.}
\end{itemize}
\begin{itemize}
\item {Proveniência:(De \textunderscore beiço\textunderscore )}
\end{itemize}
Estender-se até certo ponto:«\textunderscore uma várzea alagada que ia esbeiçar com o rio\textunderscore ». Camillo, \textunderscore Brasileira\textunderscore , 61.
O mesmo que \textunderscore esboicelar\textunderscore . Cf. Rebello, \textunderscore Mocidade\textunderscore , I, 12 e 25.
Fazer saliência froixa e bamba (uma costura ou extremidade da peça que se cose).
\section{Esbeijadeira}
\begin{itemize}
\item {Grp. gram.:f.}
\end{itemize}
\begin{itemize}
\item {Utilização:Prov.}
\end{itemize}
\begin{itemize}
\item {Utilização:trasm.}
\end{itemize}
Peneira para esbeijar.
\section{Esbeijar}
\begin{itemize}
\item {Grp. gram.:v. t.}
\end{itemize}
\begin{itemize}
\item {Utilização:Prov.}
\end{itemize}
\begin{itemize}
\item {Utilização:trasm.}
\end{itemize}
Peneirar (farinha), separando della as sêmeas.
(Cp. \textunderscore beijinho\textunderscore )
\section{Esbeltar}
\begin{itemize}
\item {Grp. gram.:v. t.}
\end{itemize}
Tornar esbelto.
\section{Esbeltez}
\begin{itemize}
\item {Grp. gram.:f.}
\end{itemize}
O mesmo que \textunderscore esbelteza\textunderscore .
\section{Esbelteza}
\begin{itemize}
\item {Grp. gram.:f.}
\end{itemize}
Qualidade daquelle ou daquillo que é esbelto.
\section{Esbelto}
\begin{itemize}
\item {Grp. gram.:adj.}
\end{itemize}
\begin{itemize}
\item {Proveniência:(Do it. \textunderscore svelto\textunderscore )}
\end{itemize}
Elegante; gentil.
De fórmas proporcionados.
\section{Esbenairar}
\begin{itemize}
\item {Grp. gram.:v. t.}
\end{itemize}
\begin{itemize}
\item {Utilização:Prov.}
\end{itemize}
\begin{itemize}
\item {Utilização:trasm.}
\end{itemize}
Fazer em benairos, esfarrapar.
\section{Esbichar}
\begin{itemize}
\item {Grp. gram.:v. t.}
\end{itemize}
\begin{itemize}
\item {Utilização:Prov.}
\end{itemize}
\begin{itemize}
\item {Utilização:trasm.}
\end{itemize}
\begin{itemize}
\item {Utilização:minh.}
\end{itemize}
O mesmo que \textunderscore esburgar\textunderscore .
\section{Esbijar}
\begin{itemize}
\item {Grp. gram.:v. t.}
\end{itemize}
\begin{itemize}
\item {Utilização:Pop.}
\end{itemize}
Distender, retesar, esticar.
\section{Esbilhotar}
\textunderscore v. t.\textunderscore  (e der.) \textunderscore Bras. do N.\textunderscore 
(V. \textunderscore bisbilhotar\textunderscore , etc.)
\section{Esbirrar}
\begin{itemize}
\item {Grp. gram.:v.}
\end{itemize}
\begin{itemize}
\item {Utilização:t. Náut.}
\end{itemize}
\begin{itemize}
\item {Proveniência:(De \textunderscore esbirro\textunderscore )}
\end{itemize}
Segurar ou fixar (uma vêrga), por meio de uma talha forte, engatada em um estropo.
\section{Esbirro}
\begin{itemize}
\item {Grp. gram.:m.}
\end{itemize}
\begin{itemize}
\item {Grp. gram.:Pl.}
\end{itemize}
\begin{itemize}
\item {Utilização:Náut.}
\end{itemize}
\begin{itemize}
\item {Proveniência:(It. \textunderscore sbirro\textunderscore )}
\end{itemize}
Beleguim, empregado menor dos tribunaes judiciaes.
Pontaletes, que amparam a amurada do navio.
\section{Esboçar}
\begin{itemize}
\item {Grp. gram.:v. t.}
\end{itemize}
Delinear: \textunderscore esboçar uma construcção\textunderscore .
Contornar.
Fazer o esbôço de.
Entremostrar: \textunderscore esboçar um sorriso\textunderscore .
\section{Esboceto}
\begin{itemize}
\item {fónica:cê}
\end{itemize}
\begin{itemize}
\item {Grp. gram.:m.}
\end{itemize}
Pequeno esbôço.
\section{Esbôço}
\begin{itemize}
\item {Grp. gram.:m.}
\end{itemize}
\begin{itemize}
\item {Utilização:Fig.}
\end{itemize}
\begin{itemize}
\item {Proveniência:(It. \textunderscore sbozzo\textunderscore )}
\end{itemize}
Delineamento inicial de uma obra de desenho ou pintura.
Modelação inicial de uma esculptura.
Rudimentos, noções geraes.
Resumo, synopse.
\section{Esbodegação}
\begin{itemize}
\item {Grp. gram.:f.}
\end{itemize}
Acto ou effeito de esbodegar-se.
\section{Esbodegar-se}
\begin{itemize}
\item {Grp. gram.:v. p.}
\end{itemize}
\begin{itemize}
\item {Utilização:Bras}
\end{itemize}
\begin{itemize}
\item {Utilização:fam.}
\end{itemize}
\begin{itemize}
\item {Proveniência:(De \textunderscore bodega\textunderscore )}
\end{itemize}
Tornar-se froixo, molle.
Espapaçar-se.
Desmazelar-se.
Embebedar-se.
Deixar-se sujar.
\section{Esbodelar}
\begin{itemize}
\item {Grp. gram.:v. t.}
\end{itemize}
\begin{itemize}
\item {Utilização:T. de Lanhoso}
\end{itemize}
Tirar a côdea ou a casca a; esburgar.
\section{Esbofamento}
\begin{itemize}
\item {Grp. gram.:m.}
\end{itemize}
Effeito de esbofar.
Fadiga. Cf. Camillo, \textunderscore Corja\textunderscore , 206.
\section{Esbofar}
\begin{itemize}
\item {Grp. gram.:v. t.}
\end{itemize}
\begin{itemize}
\item {Proveniência:(De \textunderscore bofe\textunderscore )}
\end{itemize}
Tornar offegante.
Tornar esbaforido; fatigar.
\section{Esbofeteador}
\begin{itemize}
\item {Grp. gram.:m.}
\end{itemize}
Aquelle que esbofeteia.
\section{Esbofetear}
\begin{itemize}
\item {Grp. gram.:v. t.}
\end{itemize}
Dar bofetões em.
\section{Esboforir}
\begin{itemize}
\item {Grp. gram.:v. t.}
\end{itemize}
O mesmo ou talvez melhor que \textunderscore esbaforir\textunderscore . Cf. Filinto, VII, 8; Garrett, \textunderscore D. Branca\textunderscore , 47.
(Cp. \textunderscore esbofar\textunderscore )
\section{Esboiça}
\begin{itemize}
\item {Grp. gram.:f.}
\end{itemize}
Acto de esboiçar.
\section{Esboiçar}
\begin{itemize}
\item {Grp. gram.:v. t.}
\end{itemize}
\begin{itemize}
\item {Proveniência:(De \textunderscore boiça\textunderscore )}
\end{itemize}
Surribar, para plantação de bacellos. Cf. Júl. Moreira, \textunderscore Est. da Líng. Port.\textunderscore , I, 167.
\section{Esboicelar}
\begin{itemize}
\item {Grp. gram.:v. t.}
\end{itemize}
Fazer boicelos em.
Tirar ou partir as bordas de (um vaso de loiça).
Esborcinar.
\section{Esbolar}
\begin{itemize}
\item {Grp. gram.:v. t.}
\end{itemize}
\begin{itemize}
\item {Utilização:Prov.}
\end{itemize}
\begin{itemize}
\item {Utilização:beir.}
\end{itemize}
\begin{itemize}
\item {Utilização:Prov.}
\end{itemize}
\begin{itemize}
\item {Utilização:trasm.}
\end{itemize}
\begin{itemize}
\item {Utilização:minh.}
\end{itemize}
Descascar (frutas) com instrumento cortante.
Queimar com água a ferver (a pelle).
\section{Esbombardar}
\begin{itemize}
\item {Grp. gram.:v. t.}
\end{itemize}
\begin{itemize}
\item {Utilização:Des.}
\end{itemize}
\begin{itemize}
\item {Grp. gram.:V. i.}
\end{itemize}
Deitar fóra com enfado; atirar com repulsão.
Soar como bombarda.
Dizer coisas altisonantes:«\textunderscore calai-vos, e não esbombardeis\textunderscore ». F. Manuel, \textunderscore Apólogos\textunderscore .
(Cp. \textunderscore esbombardear\textunderscore )
\section{Esbombardear}
\begin{itemize}
\item {Grp. gram.:v. t.}
\end{itemize}
(V.bombardear)
\section{Esborbulha!}
\begin{itemize}
\item {Grp. gram.:interj.}
\end{itemize}
\begin{itemize}
\item {Utilização:Ant.}
\end{itemize}
Safa! irra! Cf. Simão Machado, 32.
\section{Esborcelar}
\begin{itemize}
\item {Grp. gram.:v. t.}
\end{itemize}
\begin{itemize}
\item {Proveniência:(De \textunderscore borcelo\textunderscore )}
\end{itemize}
O mesmo que \textunderscore esborcinar\textunderscore .
\section{Esborcinar}
\begin{itemize}
\item {Grp. gram.:v. t.}
\end{itemize}
Partir as bordas de.
Cortar pela borda.
Partir os lavores de.
Golpear.
(Corr. de \textunderscore esborcelar\textunderscore )
\section{Esbordar}
\begin{itemize}
\item {Grp. gram.:v. t.}
\end{itemize}
O mesmo que \textunderscore desbordar\textunderscore .
\section{Esbordoar}
\begin{itemize}
\item {Grp. gram.:v. t.}
\end{itemize}
Bater com bordão; dar bordoada em.
\section{Esboroamento}
\begin{itemize}
\item {Grp. gram.:m.}
\end{itemize}
Acto ou effeito de esboroar.
\section{Esboroar}
\begin{itemize}
\item {Grp. gram.:v. t.}
\end{itemize}
\begin{itemize}
\item {Proveniência:(De \textunderscore borôa\textunderscore , segundo alguns; creio porém que o t. póde sêr alter. de \textunderscore desmoronar\textunderscore  &lt; \textunderscore esmoronar\textunderscore  &lt; \textunderscore esmoroar\textunderscore  &lt; \textunderscore esboroar\textunderscore )}
\end{itemize}
Converter em pó.
Desfazer, desmoronar:«\textunderscore ...o oceano que... braceja... tentando esboroar e desfazer os continentes\textunderscore ». Herculano, \textunderscore Eurico\textunderscore , 33, 9.^a ed.
\section{Esborôo}
\begin{itemize}
\item {Grp. gram.:m.}
\end{itemize}
(V.esboroamento)
\section{Esborrachar}
\begin{itemize}
\item {Grp. gram.:v. t.}
\end{itemize}
\begin{itemize}
\item {Proveniência:(De \textunderscore borracha\textunderscore )}
\end{itemize}
Fazer estoirar ou rebentar, apertando ou pisando: \textunderscore esborrachar um sapo\textunderscore .
Ferir gravemente, pisando: \textunderscore a trave esborrachou-lhe um pé\textunderscore .
\section{Esborralha}
\begin{itemize}
\item {Grp. gram.:f.}
\end{itemize}
Acto de esborralhar.
\section{Esborralhada}
\begin{itemize}
\item {Grp. gram.:f.}
\end{itemize}
Acto ou effeito de esborralhar.
\section{Esborralhadoiro}
\begin{itemize}
\item {Grp. gram.:m.}
\end{itemize}
\begin{itemize}
\item {Utilização:Des.}
\end{itemize}
\begin{itemize}
\item {Proveniência:(De \textunderscore esborralhar\textunderscore )}
\end{itemize}
Varredoiro de borralho.
\section{Esborralhador}
\begin{itemize}
\item {Grp. gram.:m.}
\end{itemize}
Pau, para esborralhar.
\section{Esborralhadouro}
\begin{itemize}
\item {Grp. gram.:m.}
\end{itemize}
\begin{itemize}
\item {Utilização:Des.}
\end{itemize}
\begin{itemize}
\item {Proveniência:(De \textunderscore esborralhar\textunderscore )}
\end{itemize}
Varredoiro de borralho.
\section{Esborralhar}
\begin{itemize}
\item {Grp. gram.:v. t.}
\end{itemize}
\begin{itemize}
\item {Grp. gram.:V. i.}
\end{itemize}
\begin{itemize}
\item {Utilização:Prov.}
\end{itemize}
\begin{itemize}
\item {Proveniência:(De \textunderscore borralho\textunderscore )}
\end{itemize}
Desfazer ou desmanchar (borralho ou brasido que estava junto).
Esbandalhar.
Parir.
\section{Esborrar}
\begin{itemize}
\item {Grp. gram.:v. t.}
\end{itemize}
\begin{itemize}
\item {Utilização:Bras}
\end{itemize}
\begin{itemize}
\item {Grp. gram.:V. i.}
\end{itemize}
\begin{itemize}
\item {Utilização:Prov.}
\end{itemize}
Tirar as bôrras de.
Aluir-se, desabar, (falando-se de trincheiras ou socalcos, amollecidos por chuva). (Colhido em Turquel)
\section{Esborratadela}
\begin{itemize}
\item {Grp. gram.:f.}
\end{itemize}
Mancha de tinta em papel.
Acto de esborratar.
\section{Esborratar}
\begin{itemize}
\item {Grp. gram.:v. t.}
\end{itemize}
Deixar cair borrão em.
\section{Esborregar}
\begin{itemize}
\item {Grp. gram.:v. t.}
\end{itemize}
\begin{itemize}
\item {Proveniência:(De \textunderscore borrego\textunderscore )}
\end{itemize}
Sacudir ou bater, depois de enxambrar (pelles).
\section{Esborretar}
\begin{itemize}
\item {Grp. gram.:v. t.}
\end{itemize}
O mesmo que \textunderscore esborrotear\textunderscore .
\section{Esborrifar}
\begin{itemize}
\item {Grp. gram.:v. t.}
\end{itemize}
O mesmo que \textunderscore borrifar\textunderscore . Cf. \textunderscore Bibl. da G. do Campo\textunderscore , 285 e 386.
\section{Esborrifo}
\begin{itemize}
\item {Grp. gram.:m.}
\end{itemize}
Acto de esborrifar.
\section{Esborrotear}
\begin{itemize}
\item {Grp. gram.:v. i.}
\end{itemize}
\begin{itemize}
\item {Utilização:Prov.}
\end{itemize}
\begin{itemize}
\item {Utilização:alg.}
\end{itemize}
Fazer ou deitar borrões em; sujar.
(Cp. \textunderscore borrão\textunderscore )
\section{Esbotenar}
\begin{itemize}
\item {Grp. gram.:v. i.}
\end{itemize}
\begin{itemize}
\item {Utilização:Prov.}
\end{itemize}
\begin{itemize}
\item {Utilização:trasm.}
\end{itemize}
\begin{itemize}
\item {Utilização:minh.}
\end{itemize}
O mesmo que \textunderscore esboicelar\textunderscore .
\section{Esbouça}
\begin{itemize}
\item {Grp. gram.:f.}
\end{itemize}
Acto de esbouçar.
\section{Esbouçar}
\begin{itemize}
\item {Grp. gram.:v. t.}
\end{itemize}
\begin{itemize}
\item {Proveniência:(De \textunderscore bouça\textunderscore )}
\end{itemize}
Surribar, para plantação de bacellos. Cf. Júl. Moreira, \textunderscore Est. da Líng. Port.\textunderscore , I, 167.
\section{Esbraçar-se}
\begin{itemize}
\item {Grp. gram.:v. p.}
\end{itemize}
\begin{itemize}
\item {Utilização:Mad}
\end{itemize}
Formar braços; bracejar: \textunderscore as árvores esbraçam-se\textunderscore .
\section{Esbracejar}
\begin{itemize}
\item {Grp. gram.:v. i.}
\end{itemize}
O mesmo que \textunderscore bracejar\textunderscore .
\section{Esbraguilhado}
\begin{itemize}
\item {Grp. gram.:adj.}
\end{itemize}
Que tem a braguilha desabotoada.
\section{Esbranger}
\begin{itemize}
\item {Grp. gram.:v. t.}
\end{itemize}
\begin{itemize}
\item {Utilização:Prov.}
\end{itemize}
\begin{itemize}
\item {Utilização:alent.}
\end{itemize}
Esbanjar, dissipar, estragar: \textunderscore era uma boa casa, mas as partilhas esbrangeram-na\textunderscore .
(Por \textunderscore esbanjar\textunderscore , sob infl. de \textunderscore abranger\textunderscore )
\section{Esbranquiçado}
\begin{itemize}
\item {Grp. gram.:adj.}
\end{itemize}
\begin{itemize}
\item {Proveniência:(De \textunderscore branco\textunderscore )}
\end{itemize}
Quási branco; descòrado.
Alvacento.
\section{Esbraseamento}
\begin{itemize}
\item {Grp. gram.:m.}
\end{itemize}
Acto ou effeito de esbrasear.
\section{Esbrasear}
\begin{itemize}
\item {Grp. gram.:v. t.}
\end{itemize}
Pôr em brasa.
Aquecer muito.
Afoguear: \textunderscore o pudor esbraseou-lhe as faces\textunderscore .
Inflammar.
\section{Esbravear}
\begin{itemize}
\item {Grp. gram.:v. i.}
\end{itemize}
O mesmo que \textunderscore esbravecer\textunderscore .
\section{Esbravecer}
\begin{itemize}
\item {Grp. gram.:v. i.}
\end{itemize}
O mesmo que \textunderscore esbravejar\textunderscore ^1.
\section{Esbravejar}
\begin{itemize}
\item {Grp. gram.:v. t.}
\end{itemize}
\begin{itemize}
\item {Grp. gram.:V. i.}
\end{itemize}
\begin{itemize}
\item {Proveniência:(De \textunderscore bravo\textunderscore )}
\end{itemize}
Exprimir irritadamente: \textunderscore esbravejar insultos\textunderscore .
Gritar com ira; barafustar.
\section{Esbravejar}
\begin{itemize}
\item {Grp. gram.:v. t.}
\end{itemize}
\begin{itemize}
\item {Utilização:Bras. do N}
\end{itemize}
Começar a amansar, a domesticar.
(Por \textunderscore desbravejar\textunderscore , de \textunderscore des...\textunderscore  + \textunderscore bravo\textunderscore )
\section{Esbritar}
\begin{itemize}
\item {Grp. gram.:v. t.}
\end{itemize}
\begin{itemize}
\item {Utilização:Prov.}
\end{itemize}
\begin{itemize}
\item {Utilização:trasm.}
\end{itemize}
O mesmo que \textunderscore esburgar\textunderscore .
\section{Esbrizar}
\begin{itemize}
\item {Grp. gram.:v. t.}
\end{itemize}
\begin{itemize}
\item {Utilização:Ant.}
\end{itemize}
\begin{itemize}
\item {Proveniência:(Do it. \textunderscore sbrizzare\textunderscore )}
\end{itemize}
Despender; sacrificar.
\section{Esbroar}
\begin{itemize}
\item {Grp. gram.:v. t.}
\end{itemize}
O mesmo que \textunderscore esboroar\textunderscore . Cf. Filinto, VI. 121 e 213.
\section{Esbronca}
\begin{itemize}
\item {Grp. gram.:m.}
\end{itemize}
\begin{itemize}
\item {Utilização:Gír. lisb.}
\end{itemize}
Homem desconfiado.
\section{Esbroncar}
\begin{itemize}
\item {Grp. gram.:v. i.}
\end{itemize}
\begin{itemize}
\item {Utilização:Gír.}
\end{itemize}
Dizer segredos.
\section{Esbrucinar-se}
\begin{itemize}
\item {Grp. gram.:v. p.}
\end{itemize}
\begin{itemize}
\item {Utilização:Prov.}
\end{itemize}
\begin{itemize}
\item {Utilização:alent.}
\end{itemize}
O mesmo que [[debruçar-se|debruçar]].
\section{Esbrugar}
\begin{itemize}
\item {Grp. gram.:v. t.}
\end{itemize}
(V.esburgar)
\section{Esbrugo}
\begin{itemize}
\item {Grp. gram.:m.}
\end{itemize}
Acto de esbrugar. Cf. Filinto, III, 43.
\section{Esbugalhar}
\begin{itemize}
\item {Grp. gram.:v. t.}
\end{itemize}
\begin{itemize}
\item {Utilização:Fig.}
\end{itemize}
\begin{itemize}
\item {Proveniência:(De \textunderscore bugalho\textunderscore )}
\end{itemize}
Tirar o bugalho a.
Esboroar; esmigalhar.
Abrir muito (os olhos).
\section{Esbulhador}
\begin{itemize}
\item {Grp. gram.:m.  e  adj.}
\end{itemize}
\begin{itemize}
\item {Proveniência:(Do lat. \textunderscore spoliator\textunderscore )}
\end{itemize}
O que esbulha.
\section{Esbulhar}
\begin{itemize}
\item {Grp. gram.:v. t.}
\end{itemize}
\begin{itemize}
\item {Proveniência:(Do lat. \textunderscore spoliare\textunderscore )}
\end{itemize}
Despojar.
Tirar a posse de alguma coisa a; privar.
Descascar (frutas, batatas, etc.).
\section{Esbulho}
\begin{itemize}
\item {Grp. gram.:m.}
\end{itemize}
\begin{itemize}
\item {Proveniência:(Do lat. \textunderscore spolium\textunderscore )}
\end{itemize}
Acto ou effeito de esbulhar.
\section{Esburacado}
\begin{itemize}
\item {Grp. gram.:adj.}
\end{itemize}
Que tem buracos.
\section{Esburacar}
\begin{itemize}
\item {Grp. gram.:v. t.}
\end{itemize}
Fazer buracos em.
\section{Esburga!}
\begin{itemize}
\item {Grp. gram.:interj.}
\end{itemize}
\begin{itemize}
\item {Utilização:Ant.}
\end{itemize}
Apage! arreda! Cf. Simão Machado, f. 15, v.^o e 63.
\section{Esburgar}
\begin{itemize}
\item {Grp. gram.:v. t.}
\end{itemize}
\begin{itemize}
\item {Proveniência:(Do lat. \textunderscore expurgare\textunderscore )}
\end{itemize}
Tirar a casca a.
Separar da carne (os ossos).
\section{Esburnir}
\begin{itemize}
\item {Grp. gram.:v. t.}
\end{itemize}
\begin{itemize}
\item {Utilização:Bras}
\end{itemize}
Dar contra vontade ou de má vontade.
Esportular.
\section{Esbuxar}
\begin{itemize}
\item {Grp. gram.:v. t.}
\end{itemize}
\begin{itemize}
\item {Utilização:Des.}
\end{itemize}
\begin{itemize}
\item {Proveniência:(Do lat. \textunderscore luxare\textunderscore ?)}
\end{itemize}
Deslocar; desmanchar (algum membro).
\section{Escabeçar}
\begin{itemize}
\item {Grp. gram.:v. t.}
\end{itemize}
O mesmo que \textunderscore descabeçar\textunderscore . Cf. Castilho, \textunderscore Metam.\textunderscore , 216.
\section{Escabecear}
\begin{itemize}
\item {Grp. gram.:v. i.}
\end{itemize}
(V.cabecear)
\section{Escabeche}
\begin{itemize}
\item {Grp. gram.:m.}
\end{itemize}
\begin{itemize}
\item {Utilização:Fig.}
\end{itemize}
Conserva de vinagre e de outros temperos, para peixe ou carne.
Disfarce; ornato, para dissimular defeito.
\section{Escabela}
\begin{itemize}
\item {Grp. gram.:f.}
\end{itemize}
\begin{itemize}
\item {Proveniência:(De \textunderscore escabelar\textunderscore )}
\end{itemize}
Acto de arrancar os pêlos aos coiros, antes da curtimenta.
\section{Escabelado}
\begin{itemize}
\item {Grp. gram.:m.}
\end{itemize}
Casta de uva branca da Bairrada.
\section{Escabelar}
\begin{itemize}
\item {Grp. gram.:v. t.}
\end{itemize}
Soltar, desgrenhar (o cabelo)
\section{Escabeleirar}
\begin{itemize}
\item {Grp. gram.:v. t.}
\end{itemize}
\begin{itemize}
\item {Proveniência:(De \textunderscore cabeleira\textunderscore )}
\end{itemize}
O mesmo que \textunderscore escabelar\textunderscore .
\section{Escabelizar}
\begin{itemize}
\item {Grp. gram.:v. t.}
\end{itemize}
O mesmo que \textunderscore escabelar\textunderscore . Cf. Camillo, \textunderscore Cancion. Al.\textunderscore , 77.
\section{Escabella}
\begin{itemize}
\item {Grp. gram.:f.}
\end{itemize}
\begin{itemize}
\item {Proveniência:(De \textunderscore escabellar\textunderscore )}
\end{itemize}
Acto de arrancar os pêlos aos coiros, antes da curtimenta.
\section{Escabellado}
\begin{itemize}
\item {Grp. gram.:m.}
\end{itemize}
Casta de uva branca da Bairrada.
\section{Escabellar}
\begin{itemize}
\item {Grp. gram.:v. t.}
\end{itemize}
Soltar, desgrenhar (o cabello).
\section{Escabelleirar}
\begin{itemize}
\item {Grp. gram.:v. t.}
\end{itemize}
\begin{itemize}
\item {Proveniência:(De \textunderscore cabelleira\textunderscore )}
\end{itemize}
O mesmo que \textunderscore escabellar\textunderscore .
\section{Escabellizar}
\begin{itemize}
\item {Grp. gram.:v. t.}
\end{itemize}
O mesmo que \textunderscore escabellar\textunderscore . Cf. Camillo, \textunderscore Cancion. Al.\textunderscore , 77.
\section{Escabêllo}
\begin{itemize}
\item {Grp. gram.:m.}
\end{itemize}
\begin{itemize}
\item {Proveniência:(Lat. \textunderscore scabellum\textunderscore )}
\end{itemize}
Banco comprido e largo, de assento móvel, e que constitue ao mesmo tempo uma caixa, a que serve de tampa o assento.
Pequeno banco, para descanso dos pés.
\section{Escabêlo}
\begin{itemize}
\item {Grp. gram.:m.}
\end{itemize}
\begin{itemize}
\item {Proveniência:(Lat. \textunderscore scabellum\textunderscore )}
\end{itemize}
Banco comprido e largo, de assento móvel, e que constitue ao mesmo tempo uma caixa, a que serve de tampa o assento.
Pequeno banco, para descanso dos pés.
\section{Escabichador}
\begin{itemize}
\item {Grp. gram.:adj.}
\end{itemize}
\begin{itemize}
\item {Grp. gram.:M.}
\end{itemize}
Que escabicha.
Aquelle que escabicha.
\section{Escabichar}
\begin{itemize}
\item {Grp. gram.:v. t.}
\end{itemize}
\begin{itemize}
\item {Utilização:Fam.}
\end{itemize}
Investigar, examinar com curiosidade e paciência.
Limpar com o palito (os dentes).
\section{Escabicheira}
\begin{itemize}
\item {Grp. gram.:f.}
\end{itemize}
\begin{itemize}
\item {Proveniência:(De \textunderscore escabichar\textunderscore )}
\end{itemize}
Mulher, que se emprega em apanhar as algas que o mar arroja á praia, na Galliza. Cf. a rev. \textunderscore Occidente\textunderscore , XII, 176.
\section{Escabiosa}
\begin{itemize}
\item {Grp. gram.:f.}
\end{itemize}
\begin{itemize}
\item {Proveniência:(De \textunderscore escabioso\textunderscore )}
\end{itemize}
Planta dipsácea (\textunderscore scabiosa succisa\textunderscore ).
O mesmo que \textunderscore saudade\textunderscore , planta.
\section{Escabioso}
\begin{itemize}
\item {Grp. gram.:adj.}
\end{itemize}
\begin{itemize}
\item {Proveniência:(Lat. \textunderscore scabiosus\textunderscore )}
\end{itemize}
Que tem erupção semelhante á da sarna.
\section{Escabreação}
\begin{itemize}
\item {Grp. gram.:f.}
\end{itemize}
Acto ou effeito de escabrear.
\section{Escabrear}
\begin{itemize}
\item {Grp. gram.:v. t.}
\end{itemize}
\begin{itemize}
\item {Utilização:Gír.}
\end{itemize}
\begin{itemize}
\item {Grp. gram.:V. i.}
\end{itemize}
\begin{itemize}
\item {Proveniência:(De \textunderscore cabra\textunderscore )}
\end{itemize}
Levantar nos pés como as cabras.
Irritar; enfurecer.
Denunciar.
Encabritar-se.
Erguer-se com fúria.
Zangar-se.
\section{Escabro}
\begin{itemize}
\item {Grp. gram.:adj.}
\end{itemize}
\begin{itemize}
\item {Proveniência:(Lat. \textunderscore scabro\textunderscore )}
\end{itemize}
Diz-se dos dentes que têm pedra, que estão sujos ou mal tratados.
\section{Escabrosamente}
\begin{itemize}
\item {Grp. gram.:adv.}
\end{itemize}
De modo escabroso.
\section{Escabrosidade}
\begin{itemize}
\item {Grp. gram.:f.}
\end{itemize}
Qualidade daquillo que é escabroso.
\section{Escabroso}
\begin{itemize}
\item {Grp. gram.:adj.}
\end{itemize}
\begin{itemize}
\item {Proveniência:(Lat. \textunderscore scrabosus\textunderscore )}
\end{itemize}
Áspero.
Pedregoso.
Que tem accesso diffícil.
Diffícil.
Melindroso.
Opposto ás conveniências ou ao decoro: \textunderscore assumptos escabrosos\textunderscore .
\section{Escabujante}
\begin{itemize}
\item {Grp. gram.:adj.}
\end{itemize}
Que escabuja.
\section{Escabujar}
\begin{itemize}
\item {Grp. gram.:v. i.}
\end{itemize}
Esbravejar.
Debater-se; estrebuchar.
(Alter. de \textunderscore escabulhar\textunderscore ?)
\section{Escabulhar}
\begin{itemize}
\item {Grp. gram.:v. t.}
\end{itemize}
Tirar o escabulho a.
Descascar; espurgar.
\section{Escabulho}
\begin{itemize}
\item {Grp. gram.:m.}
\end{itemize}
Casco ou pellícula, que envolve as sementes ou grãos.
(Por \textunderscore escapulho\textunderscore , de \textunderscore capulho\textunderscore )
\section{Escaçado}
\begin{itemize}
\item {Grp. gram.:adj.}
\end{itemize}
\begin{itemize}
\item {Utilização:Fig.}
\end{itemize}
\begin{itemize}
\item {Proveniência:(De \textunderscore caça\textunderscore )}
\end{itemize}
Diz-se do cão, deshabituado de caçar.
Esquecido.
\section{Escacar}
\begin{itemize}
\item {Grp. gram.:v. t.}
\end{itemize}
Partir em cacos, fazer em bocados, (loiça, vidros, etc.).
\section{Escachapeirado}
\begin{itemize}
\item {Grp. gram.:adj.}
\end{itemize}
\begin{itemize}
\item {Utilização:Prov.}
\end{itemize}
\begin{itemize}
\item {Utilização:trasm.}
\end{itemize}
Alastrado ou rasteiro como a cachapeira.
\section{Escachar}
\begin{itemize}
\item {Grp. gram.:v. t.}
\end{itemize}
\begin{itemize}
\item {Grp. gram.:Loc.}
\end{itemize}
\begin{itemize}
\item {Utilização:fam.}
\end{itemize}
\begin{itemize}
\item {Proveniência:(Do lat. \textunderscore ex-quassare\textunderscore )}
\end{itemize}
Fender, partir.
Rachar ao meio.
Escanchar; alargar.
\textunderscore De escacha\textunderscore , ou \textunderscore de escacha-pessegueiro\textunderscore , de arromba; com valentia.
\section{Escachelado}
\begin{itemize}
\item {Grp. gram.:adj.}
\end{itemize}
\begin{itemize}
\item {Utilização:Bras}
\end{itemize}
Abatido, alquebrado, avelhentado.
(Por \textunderscore escacholado\textunderscore , de \textunderscore cachola\textunderscore ?)
\section{Escachoar}
\begin{itemize}
\item {Grp. gram.:v. i.}
\end{itemize}
\begin{itemize}
\item {Utilização:Prov.}
\end{itemize}
Ferver em cachão.
Rebentar em cachão (a água); borbotar.
\section{Escacholar}
\begin{itemize}
\item {Grp. gram.:v. t.}
\end{itemize}
\begin{itemize}
\item {Utilização:Pop.}
\end{itemize}
Partir a cachola de.
Quebrar a cabeça de.
\section{Escachouçar}
\begin{itemize}
\item {Grp. gram.:v. i.}
\end{itemize}
\begin{itemize}
\item {Utilização:Prov.}
\end{itemize}
\begin{itemize}
\item {Utilização:trasm.}
\end{itemize}
O mesmo que \textunderscore retoiçar\textunderscore .
\section{Escaço}
\begin{itemize}
\item {Grp. gram.:m.}
\end{itemize}
\begin{itemize}
\item {Utilização:Prov.}
\end{itemize}
\begin{itemize}
\item {Proveniência:(Do rad. de \textunderscore escacholar\textunderscore ?)}
\end{itemize}
Adubo para as terras, feito dos detritos ou resíduos da preparação de mariscos.
\section{Escaçoar}
\begin{itemize}
\item {Grp. gram.:v. t.}
\end{itemize}
\begin{itemize}
\item {Utilização:Prov.}
\end{itemize}
Preparar (uma pedra) para cação, isto é, para comêço de arco abatido sobre a torça.
\section{Escada}
\begin{itemize}
\item {Grp. gram.:f.}
\end{itemize}
Série de degraus de pedra, metal ou madeira.
Utensílio, formado de duas peças de madeira, ligadas por travessas parallelas e equidistantes, que servem de degraus.
Utensílio análogo ao antecedente e formado de cordas.
Aquillo que serve para que alguém suba ou se eleve.
(Contr. de \textunderscore escalada\textunderscore , com a quéda do \textunderscore l\textunderscore )
\section{Escadão}
\begin{itemize}
\item {Grp. gram.:m.}
\end{itemize}
\begin{itemize}
\item {Utilização:Ant.}
\end{itemize}
\begin{itemize}
\item {Proveniência:(De \textunderscore escada\textunderscore )}
\end{itemize}
Ala de pobres, que acompanhavam enterros.
\section{Escadaria}
\begin{itemize}
\item {Grp. gram.:f.}
\end{itemize}
\begin{itemize}
\item {Proveniência:(De \textunderscore escada\textunderscore )}
\end{itemize}
Série de escadas, separadas por patins e que dão accesso aos differentes andares de um prédio, ou a outro ponto ou lugar elevado.
\section{Escádea}
\begin{itemize}
\item {Grp. gram.:f.}
\end{itemize}
\begin{itemize}
\item {Utilização:Prov.}
\end{itemize}
\begin{itemize}
\item {Utilização:beir.}
\end{itemize}
Esgalho do cacho de uva.
Coisa que dá a apparência de cacho.
Pequenina lasca de madeira, que se embebe na pelle ou entre a unha e a carne.
\section{Escadeado}
\begin{itemize}
\item {Grp. gram.:adj.}
\end{itemize}
Que apresenta saliências e depressões, á semelhança de escada:«\textunderscore ...peito escadeado pela vesicacão...\textunderscore »Camillo, \textunderscore Caveira\textunderscore , XIX.
\section{Escadear}
\begin{itemize}
\item {Grp. gram.:v. t.}
\end{itemize}
Dar feição de escada a: \textunderscore olhe lá, mestre, não me escadeie o cabello\textunderscore .
\section{Escadeirado}
\begin{itemize}
\item {Grp. gram.:adj.}
\end{itemize}
Descadeirado.
\section{Escadeirão}
\begin{itemize}
\item {Grp. gram.:m.}
\end{itemize}
\begin{itemize}
\item {Utilização:Des.}
\end{itemize}
Escada grande.
\section{Escadeirar}
\begin{itemize}
\item {Grp. gram.:v. t.}
\end{itemize}
\begin{itemize}
\item {Proveniência:(De \textunderscore cadeira\textunderscore )}
\end{itemize}
Bater nas ancas de.
Desancar; bater.
\section{Escadelecer}
\begin{itemize}
\item {Grp. gram.:v. i.}
\end{itemize}
\begin{itemize}
\item {Utilização:Pop.}
\end{itemize}
Dormitar; toscanejar.
(Por \textunderscore escadecer\textunderscore , do lat. hypoth. \textunderscore cadescere\textunderscore , de \textunderscore cadere\textunderscore )
\section{Escadinha}
\begin{itemize}
\item {Grp. gram.:f.}
\end{itemize}
Arbusto erythroxýleo do Brasil.
\section{Escadório}
\begin{itemize}
\item {Grp. gram.:m.}
\end{itemize}
(V.escadaria)
\section{Escadote}
\begin{itemize}
\item {Grp. gram.:m.}
\end{itemize}
Pequena escada móvel, de madeira, com quatro pernas fixas.
\section{Escadraçar}
\begin{itemize}
\item {Grp. gram.:v. t.}
\end{itemize}
\begin{itemize}
\item {Utilização:Prov.}
\end{itemize}
\begin{itemize}
\item {Utilização:minh.}
\end{itemize}
Despedaçar, partir.
Esboroar.
(Por \textunderscore escarduçar\textunderscore ?)
\section{Escadracento}
\begin{itemize}
\item {Grp. gram.:adj.}
\end{itemize}
\begin{itemize}
\item {Utilização:Prov.}
\end{itemize}
Que se escadraça facilmente; que se parte irregularmente.
\section{Escadraçoso}
\begin{itemize}
\item {Grp. gram.:adj.}
\end{itemize}
\begin{itemize}
\item {Utilização:Prov.}
\end{itemize}
\begin{itemize}
\item {Utilização:minh.}
\end{itemize}
Desligado, solto, (falando-se de terra): \textunderscore esta terra, ao lavrar, é muito escadraçosa\textunderscore . Cf. \textunderscore Gaz. das Ald.\textunderscore , 694.
\section{Escadrilhar}
\begin{itemize}
\item {Grp. gram.:v. t.}
\end{itemize}
\begin{itemize}
\item {Utilização:Prov.}
\end{itemize}
\begin{itemize}
\item {Utilização:minh.}
\end{itemize}
O mesmo que \textunderscore escadraçar\textunderscore .
(Por \textunderscore escardilhar\textunderscore )
\section{Escafeder}
\begin{itemize}
\item {Grp. gram.:v. i.}
\end{itemize}
\begin{itemize}
\item {Grp. gram.:V. p.}
\end{itemize}
\begin{itemize}
\item {Utilização:Pop.}
\end{itemize}
\begin{itemize}
\item {Proveniência:(Do rad. do it. \textunderscore scafa\textunderscore ?)}
\end{itemize}
Fugir apressadamente. Cf. Filinto, II, 68.
Fugir apressadamente, assustado.
Esgueirar-se.
\section{Escafelo}
\begin{itemize}
\item {fónica:fé}
\end{itemize}
\begin{itemize}
\item {Grp. gram.:m.}
\end{itemize}
\begin{itemize}
\item {Utilização:Prov.}
\end{itemize}
\begin{itemize}
\item {Utilização:alg.}
\end{itemize}
Mancha de caliça na parede.
(Cp. \textunderscore acafelar\textunderscore )
\section{Escafulada}
\begin{itemize}
\item {Grp. gram.:f.}
\end{itemize}
\begin{itemize}
\item {Utilização:Prov.}
\end{itemize}
\begin{itemize}
\item {Utilização:trasm.}
\end{itemize}
Esfolhada, descamisada (do milho).
\section{Escâimbo}
\begin{itemize}
\item {Grp. gram.:m.}
\end{itemize}
Outra fórma de \textunderscore escambo\textunderscore . Cf. Filinto, \textunderscore D. Man.\textunderscore , I, 151.
\section{Escaiola}
\begin{itemize}
\item {fónica:ca-i}
\end{itemize}
\begin{itemize}
\item {Grp. gram.:f.}
\end{itemize}
\begin{itemize}
\item {Proveniência:(It. \textunderscore scagliuola\textunderscore )}
\end{itemize}
Substância, feita de gêsso e colla, para revestir estátuas, parede, etc.
\section{Escaiolador}
\begin{itemize}
\item {Grp. gram.:m.}
\end{itemize}
Aquelle que escaiola.
\section{Escaiolar}
\begin{itemize}
\item {fónica:ca-i}
\end{itemize}
\begin{itemize}
\item {Grp. gram.:v. t.}
\end{itemize}
Revestir com escaiola.
\section{Escala}
\begin{itemize}
\item {Grp. gram.:f.}
\end{itemize}
\begin{itemize}
\item {Utilização:Ant.}
\end{itemize}
\begin{itemize}
\item {Utilização:Des.}
\end{itemize}
\begin{itemize}
\item {Proveniência:(Lat. \textunderscore scala\textunderscore )}
\end{itemize}
O mesmo que \textunderscore escada\textunderscore .
Paragem ou arribada de um navio, para carga ou descarga.
Graduação de thermómetros e barómetros.
Graduação.
Categoria.
Linha graduada, que nas cartas geográphicas relacíona as distâncias reaes com as figuradas.
Petipé.
Proporção entre a medida de um desenho ou plano e a medida real do que se representa.
Série de typos das bebidas alcoólicas.
Registo, que indica a ordem de serviço para cada um de vários indivíduos; turno, vez.
Série de notas musicaes, subindo ou descendo, e representando outros tantos sons, ascendentes ou descendentes; gamma.
Assalto a uma praça ou cidade, por meio de escadas.
Fôrça, insistência.
\section{Escala}
\begin{itemize}
\item {Grp. gram.:f.}
\end{itemize}
Acto de escalar.
Escalada, escalamento. Us. por Herculano, \textunderscore Hist. de Port.\textunderscore 
\section{Escalabitano}
\begin{itemize}
\item {Grp. gram.:m.  e  adj.}
\end{itemize}
\begin{itemize}
\item {Proveniência:(Do lat. \textunderscore Scalabis\textunderscore , n. p.)}
\end{itemize}
O mesmo que \textunderscore santareno\textunderscore .
\section{Escalada}
\begin{itemize}
\item {Grp. gram.:f.}
\end{itemize}
\begin{itemize}
\item {Utilização:Prov.}
\end{itemize}
O mesmo que escalamento.
Escada de mão, escada portátil.
\section{Escalador}
\begin{itemize}
\item {Grp. gram.:adj.}
\end{itemize}
\begin{itemize}
\item {Grp. gram.:M.}
\end{itemize}
\begin{itemize}
\item {Proveniência:(De \textunderscore escalar\textunderscore ^1)}
\end{itemize}
Que escala.
Aquelle que escala.
\section{Escalafrio}
\begin{itemize}
\item {Grp. gram.:m.}
\end{itemize}
(Corr. de \textunderscore calafrio\textunderscore )
\section{Escalamão}
\begin{itemize}
\item {Grp. gram.:m.}
\end{itemize}
(V.escalmão)
\section{Escalambrar}
\begin{itemize}
\item {Grp. gram.:v. i.}
\end{itemize}
\begin{itemize}
\item {Utilização:Prov.}
\end{itemize}
\begin{itemize}
\item {Utilização:trasm.}
\end{itemize}
O mesmo que \textunderscore escambrar\textunderscore .
\section{Escalambre}
\begin{itemize}
\item {Grp. gram.:m.}
\end{itemize}
Acto de escalambrar.
\section{Escalamento}
\begin{itemize}
\item {Grp. gram.:m.}
\end{itemize}
Acto de escalar^2.
\section{Escalão}
\begin{itemize}
\item {Grp. gram.:m.}
\end{itemize}
\begin{itemize}
\item {Utilização:Prov.}
\end{itemize}
\begin{itemize}
\item {Utilização:alent.}
\end{itemize}
\begin{itemize}
\item {Proveniência:(De \textunderscore escalar\textunderscore ^2)}
\end{itemize}
Homem, que maltrata animaes.
Escalda-favaes.
\section{Escalão}
\begin{itemize}
\item {Grp. gram.:m.}
\end{itemize}
\begin{itemize}
\item {Utilização:Prov.}
\end{itemize}
\begin{itemize}
\item {Proveniência:(De \textunderscore escala\textunderscore )}
\end{itemize}
Degrau.
Passagem ou plano, por onde se sobe ou se desce.
Socalco de terreno.
\section{Escalar}
\begin{itemize}
\item {Grp. gram.:v. t.}
\end{itemize}
\begin{itemize}
\item {Proveniência:(De \textunderscore escala\textunderscore )}
\end{itemize}
Assaltar, subindo por escadas.
Saquear.
Assolar.
Subir com escada a: \textunderscore escalar um predio\textunderscore .
Trepar a.
Designar por escala (serviço).
\section{Escalar}
\begin{itemize}
\item {Grp. gram.:v. t.}
\end{itemize}
\begin{itemize}
\item {Proveniência:(De \textunderscore calar\textunderscore )}
\end{itemize}
Estripar e salgar (peixe).
O mesmo que \textunderscore escalavrar\textunderscore :«\textunderscore ...e um grumete escalou uma perna.\textunderscore »\textunderscore Ethiópia Or.\textunderscore , II, 363.
\section{Escalavardar}
\begin{itemize}
\item {Grp. gram.:v. t.}
\end{itemize}
\begin{itemize}
\item {Utilização:Prov.}
\end{itemize}
\begin{itemize}
\item {Utilização:alg.}
\end{itemize}
\begin{itemize}
\item {Utilização:beir.}
\end{itemize}
O mesmo que \textunderscore escalavrar\textunderscore .
Rasgar; retalhar com golpes.
\section{Escalavradura}
\begin{itemize}
\item {Grp. gram.:f.}
\end{itemize}
Effeito de escalavrar.
Escoriação; ferimento.
\section{Escalavramento}
\begin{itemize}
\item {Grp. gram.:m.}
\end{itemize}
Acto ou effeito de escalavrar.
\section{Escalavrão}
\begin{itemize}
\item {Grp. gram.:m.}
\end{itemize}
Grande escalavro.
\section{Escalavrar}
\begin{itemize}
\item {Grp. gram.:v. t.}
\end{itemize}
Golpear levemente.
Golpear.
Arranhar.
Esboicelar.
Damnificar o revestimento de (paredes ou tectos).
Arruinar.
(Cp. cast. \textunderscore descalabro\textunderscore )
\section{Escalavro}
\begin{itemize}
\item {Grp. gram.:m.}
\end{itemize}
O mesmo que \textunderscore escalavramento\textunderscore .
Esbanjamento de haveres; ruína.
\section{Escalda}
\begin{itemize}
\item {Grp. gram.:f.}
\end{itemize}
\begin{itemize}
\item {Utilização:Prov.}
\end{itemize}
\begin{itemize}
\item {Proveniência:(De \textunderscore escaldar\textunderscore )}
\end{itemize}
Môlho picante.
\section{Escaldação}
\begin{itemize}
\item {Grp. gram.:f.}
\end{itemize}
O mesmo que \textunderscore escaldão\textunderscore .
\section{Escaldadela}
\begin{itemize}
\item {Grp. gram.:f.}
\end{itemize}
O mesmo que \textunderscore escaldadura\textunderscore .
\section{Escaldadiço}
\begin{itemize}
\item {Grp. gram.:adj.}
\end{itemize}
\begin{itemize}
\item {Utilização:Fig.}
\end{itemize}
\begin{itemize}
\item {Proveniência:(De \textunderscore escaldar\textunderscore )}
\end{itemize}
Que se escalda facilmente.
Muito impressionável.
\section{Escaldado}
\begin{itemize}
\item {Grp. gram.:m.}
\end{itemize}
\begin{itemize}
\item {Utilização:Bras}
\end{itemize}
\begin{itemize}
\item {Proveniência:(De \textunderscore escaldar\textunderscore )}
\end{itemize}
Farinha de mandioca, escaldada com môlho de peixe ou caldo de carne.
\section{Escaldador}
\begin{itemize}
\item {Grp. gram.:adj.}
\end{itemize}
\begin{itemize}
\item {Grp. gram.:M.}
\end{itemize}
Que escalda.
Aquelle que escalda.
\section{Escaldadura}
\begin{itemize}
\item {Grp. gram.:f.}
\end{itemize}
O mesmo que \textunderscore escaldão\textunderscore .
\section{Escalda-favaes}
\begin{itemize}
\item {Grp. gram.:m.}
\end{itemize}
\begin{itemize}
\item {Utilização:Fam.}
\end{itemize}
\begin{itemize}
\item {Utilização:Prov.}
\end{itemize}
\begin{itemize}
\item {Utilização:alent.}
\end{itemize}
Pessôa irritável, que toma calor com qualquer coisa.
Homem, que maltrata animaes.
\section{Escaldante}
\begin{itemize}
\item {Grp. gram.:adj.}
\end{itemize}
Que escalda.
\section{Escaldão}
\begin{itemize}
\item {Grp. gram.:m.}
\end{itemize}
\begin{itemize}
\item {Utilização:Fig.}
\end{itemize}
\begin{itemize}
\item {Utilização:T. de Turquel}
\end{itemize}
Acto ou effeito de escaldar.
Ferimento.
Castigo; reprehensão.
Destempêro do solo arável, por se revolver no outono, antes de bem repassado de chuvas.
\section{Escaldar}
\begin{itemize}
\item {Grp. gram.:v. t.}
\end{itemize}
\begin{itemize}
\item {Utilização:Fig.}
\end{itemize}
\begin{itemize}
\item {Grp. gram.:V. i.}
\end{itemize}
\begin{itemize}
\item {Proveniência:(Lat. \textunderscore excaldare\textunderscore )}
\end{itemize}
Queimar com qualquer líquido quente.
Queimar com metal quente.
Lavar em água muito quente: \textunderscore escaldar os pés\textunderscore .
Produzir muito calor em.
Aquecer muito.
Tornar estéril.
Escarmentar.
Produzir muito calor: \textunderscore está um sol, que escalda\textunderscore .
\section{Escalda-rabo}
\begin{itemize}
\item {Grp. gram.:m.}
\end{itemize}
\begin{itemize}
\item {Utilização:Burl.}
\end{itemize}
Descompostura; reprehensão.
\section{Escaldarrapa}
\begin{itemize}
\item {Grp. gram.:f.}
\end{itemize}
\begin{itemize}
\item {Utilização:Prov.}
\end{itemize}
\begin{itemize}
\item {Utilização:beir.}
\end{itemize}
Filhó, que se faz com a calda ou massa que sobeja do ensacar de chouriços.
\section{Escaldeado}
\begin{itemize}
\item {Grp. gram.:adj.}
\end{itemize}
\begin{itemize}
\item {Utilização:P. us.}
\end{itemize}
Muito aquecido; muito cálido:«\textunderscore era escaldeada e íntima\textunderscore  (a dôr da Virgem)». Alves Mendes.
(Cp. \textunderscore escaldar\textunderscore  e \textunderscore caldear\textunderscore )
\section{Escaldeirar}
\begin{itemize}
\item {Grp. gram.:v. t.}
\end{itemize}
\begin{itemize}
\item {Utilização:Prov.}
\end{itemize}
\begin{itemize}
\item {Utilização:alent.}
\end{itemize}
Abrir caldeira ou cova, em tôrno do pé de (árvore).
\section{Escaldo}
\begin{itemize}
\item {Grp. gram.:m.}
\end{itemize}
Designação genérica dos antigos poétas escandinavos.
(Escand. \textunderscore skald\textunderscore , poéta)
\section{Escaleira}
\begin{itemize}
\item {Grp. gram.:f.}
\end{itemize}
\begin{itemize}
\item {Utilização:Des.}
\end{itemize}
\begin{itemize}
\item {Utilização:Prov.}
\end{itemize}
\begin{itemize}
\item {Utilização:minh.}
\end{itemize}
\begin{itemize}
\item {Utilização:Prov.}
\end{itemize}
\begin{itemize}
\item {Utilização:trasm.}
\end{itemize}
\begin{itemize}
\item {Proveniência:(De \textunderscore escalar\textunderscore ^1)}
\end{itemize}
Escada.
Degrau de escada.
Escada de pedra, por onde se sobe para casa.
\section{Escaleno}
\begin{itemize}
\item {Grp. gram.:adj.}
\end{itemize}
\begin{itemize}
\item {Utilização:Geom.}
\end{itemize}
\begin{itemize}
\item {Utilização:Anat.}
\end{itemize}
\begin{itemize}
\item {Utilização:Mathem.}
\end{itemize}
\begin{itemize}
\item {Proveniência:(Gr. \textunderscore skalenos\textunderscore )}
\end{itemize}
Diz-se do triângulo, que tem todos os lados desiguaes.
Diz-se dos músculos, inseridos nas apóphyses transversaes das vértebras cervicaes.
Diz-se do cóne, cujo eixo não é perpendicular á base.
\section{Escalenoedro}
\begin{itemize}
\item {fónica:no-e}
\end{itemize}
\begin{itemize}
\item {Grp. gram.:m.}
\end{itemize}
\begin{itemize}
\item {Utilização:Geom.}
\end{itemize}
\begin{itemize}
\item {Grp. gram.:Adj.}
\end{itemize}
\begin{itemize}
\item {Proveniência:(Do gr. \textunderscore skalenos\textunderscore  + \textunderscore edra\textunderscore )}
\end{itemize}
Polyedro, limitado por triângulos escalenos.
Que tem faces desiguaes.
\section{Escalér}
\begin{itemize}
\item {Grp. gram.:m.}
\end{itemize}
Pequeno barco de quilha, ordinariamente de remos ou vela, para serviço de um navio ou de uma repartição ou estação pública.
\section{Escalera}
\begin{itemize}
\item {fónica:lê}
\end{itemize}
\begin{itemize}
\item {Grp. gram.:f.}
\end{itemize}
\begin{itemize}
\item {Utilização:Prov.}
\end{itemize}
\begin{itemize}
\item {Utilização:alent.}
\end{itemize}
O mesmo que \textunderscore escada\textunderscore .
(Cp. \textunderscore escaleira\textunderscore )
\section{Escaletado}
\begin{itemize}
\item {Grp. gram.:adj.}
\end{itemize}
Semelhante a escaleto.
Magríssimo. Cf. Macedo, \textunderscore Burros\textunderscore , 237.
\section{Escaletas}
\begin{itemize}
\item {fónica:lê}
\end{itemize}
\begin{itemize}
\item {Grp. gram.:f. pl.}
\end{itemize}
\begin{itemize}
\item {Proveniência:(De \textunderscore escala\textunderscore )}
\end{itemize}
Cortaduras, em fórma de degraus, nas falcas das carrêtas de bordo.
\section{Escalete}
\begin{itemize}
\item {fónica:lê}
\end{itemize}
\begin{itemize}
\item {Grp. gram.:m.}
\end{itemize}
\begin{itemize}
\item {Utilização:Pop.}
\end{itemize}
O mesmo que \textunderscore escaleto\textunderscore .
\section{Escaleto}
\begin{itemize}
\item {fónica:lê}
\end{itemize}
\begin{itemize}
\item {Grp. gram.:m.}
\end{itemize}
\begin{itemize}
\item {Utilização:fig.}
\end{itemize}
\begin{itemize}
\item {Utilização:Pop.}
\end{itemize}
Pessôa muito magra. Cf. Camillo, \textunderscore Volcões\textunderscore , 20; Castilho, \textunderscore Sabichonas\textunderscore , 112.
(Por \textunderscore esqueleto\textunderscore )
\section{Escalfado}
\begin{itemize}
\item {Grp. gram.:adj.}
\end{itemize}
\begin{itemize}
\item {Utilização:Prov.}
\end{itemize}
\begin{itemize}
\item {Utilização:trasm.}
\end{itemize}
Vazio, desfalcado.
(Por \textunderscore descalfado\textunderscore , metáth. de \textunderscore desfalcado\textunderscore )
\section{Escalfador}
\begin{itemize}
\item {Grp. gram.:m.}
\end{itemize}
\begin{itemize}
\item {Proveniência:(De \textunderscore escalfar\textunderscore ^1)}
\end{itemize}
Vaso, em que se conserva água quente para serviço de mesa.
\section{Escalfar}
\begin{itemize}
\item {Grp. gram.:v. t.}
\end{itemize}
\begin{itemize}
\item {Proveniência:(Lat. \textunderscore excalfacere\textunderscore )}
\end{itemize}
Passar por água quente; aquecer no escalfador: \textunderscore escalfar ovos\textunderscore .
\section{Escalfar}
\begin{itemize}
\item {Grp. gram.:v. t.}
\end{itemize}
\begin{itemize}
\item {Utilização:Prov.}
\end{itemize}
\begin{itemize}
\item {Utilização:alent.}
\end{itemize}
O mesmo que \textunderscore esfalfar\textunderscore .
\section{Escalfeta}
\begin{itemize}
\item {fónica:fê}
\end{itemize}
\begin{itemize}
\item {Grp. gram.:f.}
\end{itemize}
\begin{itemize}
\item {Proveniência:(De \textunderscore escalfar\textunderscore ^1)}
\end{itemize}
Braseiro, em fórma de caixa, com tampa gradeada, sôbre que se aquecem os pés.
Utensílio de pelles ou estofos, dentro do qual se metem os pés para se aquecerem.
\section{Escalfúrnio}
\begin{itemize}
\item {Grp. gram.:adj.}
\end{itemize}
\begin{itemize}
\item {Utilização:Chul.}
\end{itemize}
Cruel; que tem má índole.
\section{Escalho}
\begin{itemize}
\item {Grp. gram.:m.}
\end{itemize}
\begin{itemize}
\item {Proveniência:(Do lat. \textunderscore squalus\textunderscore )}
\end{itemize}
Pequeno peixe do água doce, espécie de robalo.
\section{Escaliçar}
\begin{itemize}
\item {Grp. gram.:v. t.}
\end{itemize}
Tirar a cal ou a caliça de. Cf. Herculano, \textunderscore Quest. Públ.\textunderscore , II, 19; \textunderscore Lendas e Narr.\textunderscore , I, 252.
\section{Escalinata}
\begin{itemize}
\item {Grp. gram.:f.}
\end{itemize}
\begin{itemize}
\item {Proveniência:(It. \textunderscore scalinata\textunderscore )}
\end{itemize}
Lanços de escadas.
\section{Escallónia}
\begin{itemize}
\item {Grp. gram.:f.}
\end{itemize}
\begin{itemize}
\item {Proveniência:(De \textunderscore Escallon\textunderscore , n. p.)}
\end{itemize}
Planta saxifragácea, aromática.
\section{Escalmão}
\begin{itemize}
\item {Grp. gram.:m.}
\end{itemize}
\begin{itemize}
\item {Utilização:T. de Aveiro}
\end{itemize}
\begin{itemize}
\item {Proveniência:(De \textunderscore escalmo\textunderscore )}
\end{itemize}
O mesmo que \textunderscore tolete\textunderscore .
\section{Escalmo}
\begin{itemize}
\item {Grp. gram.:m.}
\end{itemize}
\begin{itemize}
\item {Utilização:Náut.}
\end{itemize}
\begin{itemize}
\item {Utilização:ant.}
\end{itemize}
\begin{itemize}
\item {Proveniência:(Lat. \textunderscore scalmus\textunderscore )}
\end{itemize}
Cavilha, a que se prende o remo.
Tolete:«\textunderscore ...se forem grandes esses tiros, nem o escalmo nem a baterola os sofrerám, que sam partes fracas\textunderscore ». Fern. de Oliv., \textunderscore Arte da Guerra do Mar\textunderscore , 47, v.^o
\section{Escalmorrado}
\begin{itemize}
\item {Grp. gram.:adj.}
\end{itemize}
\begin{itemize}
\item {Utilização:Prov.}
\end{itemize}
\begin{itemize}
\item {Utilização:alg.}
\end{itemize}
\begin{itemize}
\item {Proveniência:(De \textunderscore calma\textunderscore )}
\end{itemize}
Encalmado.
Que está suando com calor.
\section{Escalo}
\begin{itemize}
\item {Grp. gram.:m.}
\end{itemize}
O mesmo que \textunderscore escalho\textunderscore .
\section{Escalonar}
\begin{itemize}
\item {Grp. gram.:v. t.}
\end{itemize}
Dispor em escalão: \textunderscore escalonar tropas\textunderscore .
Dar fórma de escada a.
\section{Escalónia}
\begin{itemize}
\item {Grp. gram.:f.}
\end{itemize}
\begin{itemize}
\item {Proveniência:(De \textunderscore Escallon\textunderscore , n. p.)}
\end{itemize}
Planta saxifragácea, aromática.
\section{Escalpação}
\begin{itemize}
\item {Grp. gram.:f.}
\end{itemize}
O mesmo que \textunderscore escalpamento\textunderscore .
\section{Escalpamento}
\begin{itemize}
\item {Grp. gram.:m.}
\end{itemize}
Acto de \textunderscore escalpar\textunderscore .
\section{Escalpar}
\begin{itemize}
\item {Grp. gram.:v. t.}
\end{itemize}
\begin{itemize}
\item {Proveniência:(De \textunderscore escalpo\textunderscore )}
\end{itemize}
Arrancar a pelle do crânio a.
\section{Escalpelante}
\begin{itemize}
\item {Grp. gram.:adj.}
\end{itemize}
Que escalpela.
\section{Escalpelar}
\begin{itemize}
\item {Grp. gram.:v. t.}
\end{itemize}
\begin{itemize}
\item {Utilização:Fig.}
\end{itemize}
Rasgar ou dissecar com escalpelo.
Dissecar.
Analisar profundamente; criticar.
\section{Escalpelizador}
\begin{itemize}
\item {Grp. gram.:adj.}
\end{itemize}
\begin{itemize}
\item {Grp. gram.:M.}
\end{itemize}
Que escalpeliza.
Aquele que escalpeliza.
\section{Escalpelizar}
\begin{itemize}
\item {Grp. gram.:v. t.}
\end{itemize}
O mesmo que \textunderscore escalpelar\textunderscore . Cf. Camillo, \textunderscore Noites de Insómn.\textunderscore , VII, 91.
\section{Escalpellante}
\begin{itemize}
\item {Grp. gram.:adj.}
\end{itemize}
Que escalpella.
\section{Escalpellar}
\begin{itemize}
\item {Grp. gram.:v. t.}
\end{itemize}
\begin{itemize}
\item {Utilização:Fig.}
\end{itemize}
Rasgar ou dissecar com escalpello.
Dissecar.
Analysar profundamente; criticar.
\section{Escalpellizador}
\begin{itemize}
\item {Grp. gram.:adj.}
\end{itemize}
\begin{itemize}
\item {Grp. gram.:M.}
\end{itemize}
Que escalpelliza.
Aquelle que escalpelliza.
\section{Escalpellizar}
\begin{itemize}
\item {Grp. gram.:v. t.}
\end{itemize}
O mesmo que \textunderscore escalpellar\textunderscore . Cf. Camillo, \textunderscore Noites de Insómn.\textunderscore , VII, 91.
\section{Escalpêllo}
\begin{itemize}
\item {Grp. gram.:m.}
\end{itemize}
\begin{itemize}
\item {Utilização:Fig.}
\end{itemize}
\begin{itemize}
\item {Proveniência:(Lat. \textunderscore scalpellum\textunderscore )}
\end{itemize}
Instrumento cirúrgico, para dissecações anatómicas.
Crítica.
\section{Escalpêlo}
\begin{itemize}
\item {Grp. gram.:m.}
\end{itemize}
\begin{itemize}
\item {Utilização:Fig.}
\end{itemize}
\begin{itemize}
\item {Proveniência:(Lat. \textunderscore scalpellum\textunderscore )}
\end{itemize}
Instrumento cirúrgico, para dissecações anatómicas.
Crítica.
\section{Escalpo}
\begin{itemize}
\item {Grp. gram.:m.}
\end{itemize}
\begin{itemize}
\item {Proveniência:(Ingl. \textunderscore scalp\textunderscore )}
\end{itemize}
Tropheu guerreiro dos indígenas da América, formado da pelle de crânios dos inimigos.
\section{Escalrachar}
\begin{itemize}
\item {Grp. gram.:v. i.}
\end{itemize}
Tirar o escalracho da terra.
\section{Escalracho}
\begin{itemize}
\item {Grp. gram.:m.}
\end{itemize}
Planta gramínea, cujas raízes se distendem muito, damnificando as sementeiras.
Agitação, que o navio produz na água, andando.
\section{Escalrichado}
\begin{itemize}
\item {Grp. gram.:adj.}
\end{itemize}
\begin{itemize}
\item {Utilização:Pop.}
\end{itemize}
Insípido: \textunderscore sopa escalrichada\textunderscore .
\section{Escalvação}
\begin{itemize}
\item {Grp. gram.:f.}
\end{itemize}
Acto ou effeito de escalvar.
\section{Escalvar}
\begin{itemize}
\item {Grp. gram.:v. t.}
\end{itemize}
\begin{itemize}
\item {Utilização:Fig.}
\end{itemize}
Fazer calvo.
Destruir a vegetação de; esterilizar: \textunderscore o sol escalvou a cortinha\textunderscore .
\section{Escama}
\begin{itemize}
\item {Grp. gram.:f.}
\end{itemize}
\begin{itemize}
\item {Proveniência:(Do lat. \textunderscore squama\textunderscore )}
\end{itemize}
Cada uma das lâminas delgadas, que revestem o corpo de alguns peixes e reptis.
Laminazinha, que se separa da epiderme, em consequência de certas moléstias.
Qualquer lâmina, qualquer ornato, em fórma de escama.
Cascas sobrepostas ou imbricadas, que cobrem os pinhões nas pinhas.
\section{Escamação}
\begin{itemize}
\item {Grp. gram.:f.}
\end{itemize}
\begin{itemize}
\item {Utilização:Chul.}
\end{itemize}
Acto de escamar.
Doença de alguns vegetaes.
Zanga, amuo.
\section{Escamadeira}
\begin{itemize}
\item {Grp. gram.:f.}
\end{itemize}
Mulher, que escama peixe.
\section{Escamado}
\begin{itemize}
\item {Grp. gram.:adj.}
\end{itemize}
\begin{itemize}
\item {Utilização:Chul.}
\end{itemize}
Zangado; amuado.
Que fala mal de tudo.
\section{Escamador}
\begin{itemize}
\item {Grp. gram.:m.}
\end{itemize}
Aquelle que escama.
\section{Escamadura}
\begin{itemize}
\item {Grp. gram.:f.}
\end{itemize}
Acto de escamar.
\section{Escamalhar}
\begin{itemize}
\item {Grp. gram.:v. t.}
\end{itemize}
\begin{itemize}
\item {Utilização:Prov.}
\end{itemize}
\begin{itemize}
\item {Utilização:trasm.}
\end{itemize}
\begin{itemize}
\item {Proveniência:(De \textunderscore cama\textunderscore )}
\end{itemize}
O mesmo que \textunderscore escangalhar\textunderscore .
Espalhar.
\section{Escamalhoar}
\begin{itemize}
\item {Grp. gram.:v. t.}
\end{itemize}
\begin{itemize}
\item {Grp. gram.:V. i.}
\end{itemize}
Fazer camalhões em (um terreno).
Fazer camalhões.
\section{Escamanta}
\begin{itemize}
\item {Grp. gram.:f.}
\end{itemize}
\begin{itemize}
\item {Utilização:Gír.}
\end{itemize}
\begin{itemize}
\item {Proveniência:(De \textunderscore escamar\textunderscore )}
\end{itemize}
Pescada.
\section{Escamar}
\begin{itemize}
\item {Grp. gram.:v. t.}
\end{itemize}
\begin{itemize}
\item {Grp. gram.:V. p.}
\end{itemize}
\begin{itemize}
\item {Utilização:Gír.}
\end{itemize}
\begin{itemize}
\item {Utilização:Bras}
\end{itemize}
Tirar a escama a.
Zangar-se.
Fugir.
\section{Escambador}
\begin{itemize}
\item {Grp. gram.:m.}
\end{itemize}
Aquelle que escamba.
\section{Escambar}
\begin{itemize}
\item {Grp. gram.:v. t.}
\end{itemize}
\begin{itemize}
\item {Utilização:Des.}
\end{itemize}
\begin{itemize}
\item {Utilização:Prov.}
\end{itemize}
\begin{itemize}
\item {Proveniência:(De \textunderscore escambo\textunderscore )}
\end{itemize}
Trocar.
Mudar de (lugar).
\section{Escambiar}
\textunderscore v. t.\textunderscore  (e der.)
O mesmo ou melhor que \textunderscore escambar\textunderscore , etc.
\section{Escambichar}
\begin{itemize}
\item {Grp. gram.:v. t.}
\end{itemize}
\begin{itemize}
\item {Utilização:Bras. do N}
\end{itemize}
O mesmo que \textunderscore escadeirar\textunderscore .
\section{Escâmbio}
\begin{itemize}
\item {Grp. gram.:m.}
\end{itemize}
O mesmo que \textunderscore escambo\textunderscore .
\section{Escambo}
\begin{itemize}
\item {Grp. gram.:m.}
\end{itemize}
\begin{itemize}
\item {Proveniência:(Do b. lat. \textunderscore escambium\textunderscore )}
\end{itemize}
Troca, permutação; câmbio.
\section{Escambra}
\begin{itemize}
\item {Grp. gram.:f.}
\end{itemize}
Acto de escambrar.
\section{Escambrão}
\begin{itemize}
\item {Grp. gram.:m.}
\end{itemize}
\begin{itemize}
\item {Utilização:Prov.}
\end{itemize}
\begin{itemize}
\item {Utilização:trasm.}
\end{itemize}
Pessôa arisca, ríspida.
\section{Escambrar}
\begin{itemize}
\item {Grp. gram.:v. i.}
\end{itemize}
\begin{itemize}
\item {Utilização:Prov.}
\end{itemize}
\begin{itemize}
\item {Utilização:minh.}
\end{itemize}
Ennevoar-se e descobrir-se alternadamente (o céu).
Haver uma aberta em céu ennevoado.
\section{Escambro}
\begin{itemize}
\item {Grp. gram.:m.}
\end{itemize}
\begin{itemize}
\item {Proveniência:(De \textunderscore escambrar\textunderscore )}
\end{itemize}
Escambra.
Quási o mesmo que \textunderscore alvarelha\textunderscore .
\section{Escambroeiro}
\begin{itemize}
\item {Grp. gram.:m.}
\end{itemize}
Planta rhamnácea, (\textunderscore rhamnus catharticus\textunderscore ).
(Cp. \textunderscore escambrão\textunderscore )
\section{Escameado}
\begin{itemize}
\item {Grp. gram.:adj.}
\end{itemize}
Revestido de escamas.
\section{Escamechar}
\begin{itemize}
\item {Grp. gram.:v. i.}
\end{itemize}
Luzir (a mecha); chamejar.
\section{Escamédio}
\begin{itemize}
\item {Grp. gram.:m.}
\end{itemize}
\begin{itemize}
\item {Utilização:Prov.}
\end{itemize}
\begin{itemize}
\item {Utilização:alent.}
\end{itemize}
Gênero de plantas, (\textunderscore bartsia aspera\textunderscore , Lge.).
\section{Escamel}
\begin{itemize}
\item {Grp. gram.:m.}
\end{itemize}
\begin{itemize}
\item {Utilização:Fig.}
\end{itemize}
\begin{itemize}
\item {Proveniência:(Do lat. \textunderscore scamellum\textunderscore )}
\end{itemize}
Banco, sôbre que os espadeiros pulem as espadas.
Aquelle que brune.
Brunimento:«\textunderscore antes de hirdes ao escamel, deixay vaidades.\textunderscore »\textunderscore Aulegrafia\textunderscore , 145.
\section{Escamento}
\begin{itemize}
\item {Grp. gram.:adj.}
\end{itemize}
O mesmo que \textunderscore escamoso\textunderscore .
\section{Escâmeo}
\begin{itemize}
\item {Grp. gram.:adj.}
\end{itemize}
\begin{itemize}
\item {Proveniência:(Lat. \textunderscore squameus\textunderscore )}
\end{itemize}
(V.escamoso)
\section{Escameta}
\begin{itemize}
\item {fónica:mê}
\end{itemize}
\begin{itemize}
\item {Grp. gram.:f.}
\end{itemize}
\begin{itemize}
\item {Proveniência:(De \textunderscore escama\textunderscore )}
\end{itemize}
Tecido de algodão, procedente do Levante.
\section{Escamifero}
\begin{itemize}
\item {Grp. gram.:adj.}
\end{itemize}
\begin{itemize}
\item {Proveniência:(Lat. \textunderscore squamifer\textunderscore )}
\end{itemize}
O mesmo que \textunderscore escamoso\textunderscore .
\section{Escamiforme}
\begin{itemize}
\item {Grp. gram.:adj.}
\end{itemize}
\begin{itemize}
\item {Proveniência:(Do lat. \textunderscore squama\textunderscore  + \textunderscore forma\textunderscore )}
\end{itemize}
Semelhante a escama.
\section{Escamígero}
\begin{itemize}
\item {Grp. gram.:adj.}
\end{itemize}
\begin{itemize}
\item {Proveniência:(Lat. \textunderscore squamiger\textunderscore )}
\end{itemize}
O mesmo que \textunderscore escamoso\textunderscore .
\section{Escamisada}
\begin{itemize}
\item {Grp. gram.:f.}
\end{itemize}
O mesmo que \textunderscore descamisada\textunderscore .
\section{Escamisadela}
\begin{itemize}
\item {Grp. gram.:f.}
\end{itemize}
O mesmo que \textunderscore escamisada\textunderscore .
\section{Escamisar}
\begin{itemize}
\item {Grp. gram.:v. t.}
\end{itemize}
O mesmo que \textunderscore descamisar\textunderscore .
\section{Escammónea}
\begin{itemize}
\item {Grp. gram.:f.}
\end{itemize}
\begin{itemize}
\item {Proveniência:(Lat. \textunderscore scammonea\textunderscore )}
\end{itemize}
Planta trepadeira (\textunderscore convolvulus ammonea\textunderscore ).
Resina purgativa, extrahida da raíz dessa planta.
\section{Escammonina}
\begin{itemize}
\item {Grp. gram.:f.}
\end{itemize}
Princípio purgativo, contido na escammónia.
O mesmo que \textunderscore jalapina\textunderscore .
\section{Escamo}
\begin{itemize}
\item {Grp. gram.:m.}
\end{itemize}
\begin{itemize}
\item {Utilização:Ant.}
\end{itemize}
O mesmo que \textunderscore escano\textunderscore .
\section{Escamões}
\begin{itemize}
\item {Grp. gram.:m. pl.}
\end{itemize}
\begin{itemize}
\item {Utilização:Prov.}
\end{itemize}
\begin{itemize}
\item {Utilização:dur.}
\end{itemize}
Cavidades, que, nos barcos rabelos, são ligadas por uma tábua, e nas quaes os tripulantes guardam as suas grandes brôas para a jornada.
\section{Escamondar}
\begin{itemize}
\item {Grp. gram.:v. t.}
\end{itemize}
\begin{itemize}
\item {Utilização:Prov.}
\end{itemize}
Desramar.
\section{Escamónea}
\begin{itemize}
\item {Grp. gram.:f.}
\end{itemize}
\begin{itemize}
\item {Proveniência:(Lat. \textunderscore scammonea\textunderscore )}
\end{itemize}
Planta trepadeira (\textunderscore convolvulus ammonea\textunderscore ).
Resina purgativa, extraida da raíz dessa planta.
\section{Escamonear}
\begin{itemize}
\item {Grp. gram.:v. i.}
\end{itemize}
\begin{itemize}
\item {Utilização:Des.}
\end{itemize}
O mesmo que \textunderscore emmonar-se\textunderscore .
\section{Escamoso}
\begin{itemize}
\item {Grp. gram.:adj.}
\end{itemize}
\begin{itemize}
\item {Proveniência:(Lat. \textunderscore squamosus\textunderscore )}
\end{itemize}
Que tem escamas; coberto de escamas.
\section{Escamotar}
\textunderscore v. t.\textunderscore  (e der.)
O mesmo que \textunderscore escamotear\textunderscore , etc.
\section{Escamoteação}
\begin{itemize}
\item {Grp. gram.:f.}
\end{itemize}
\begin{itemize}
\item {Utilização:Fig.}
\end{itemize}
Acto de escamotear.
Furto hábil.
\section{Escamoteadela}
\begin{itemize}
\item {Grp. gram.:f.}
\end{itemize}
O mesmo que \textunderscore escamoteação\textunderscore .
\section{Escamoteador}
\begin{itemize}
\item {Grp. gram.:m.}
\end{itemize}
\begin{itemize}
\item {Utilização:Fig.}
\end{itemize}
\begin{itemize}
\item {Utilização:Phot.}
\end{itemize}
Aquelle que escamoteia.
Aquelle que furta com destreza.
Caixilho especial, para se collocarem ao abrigo da luz as chapas photográphicas.
\section{Escamotear}
\begin{itemize}
\item {Grp. gram.:v. t.}
\end{itemize}
\begin{itemize}
\item {Grp. gram.:V. i.}
\end{itemize}
Empalmar.
Furtar com destreza.
Ser prestímano, fazer sortes de prestidigitação.
(Cast. \textunderscore escamotear\textunderscore )
\section{Escamoucho}
\begin{itemize}
\item {Grp. gram.:m.}
\end{itemize}
\begin{itemize}
\item {Utilização:Ant.}
\end{itemize}
Sobejos de mesa. Cf. \textunderscore Eufrosina\textunderscore , 178.
(Cast. \textunderscore escamocho\textunderscore )
\section{Escampado}
\begin{itemize}
\item {Grp. gram.:m.}
\end{itemize}
\begin{itemize}
\item {Proveniência:(De \textunderscore escampar\textunderscore ^1)}
\end{itemize}
O mesmo que \textunderscore descampado\textunderscore .
\section{Escampar}
\begin{itemize}
\item {Grp. gram.:v. i.}
\end{itemize}
\begin{itemize}
\item {Proveniência:(De \textunderscore campo\textunderscore )}
\end{itemize}
Deixar de chover.
Limpar-se de nuvens, aclarar-se (o céu).
\section{Escampar}
\begin{itemize}
\item {Grp. gram.:v. i.}
\end{itemize}
Escapar, fugir:«\textunderscore diz que a lebre escampou\textunderscore ». Filinto, XII, 197.
(Fórma des., mas preferivel a \textunderscore escapar\textunderscore . V. \textunderscore escapar\textunderscore )
\section{Escampavia}
\begin{itemize}
\item {Grp. gram.:f.}
\end{itemize}
Barco espanhol, de pesca, que chega até á costa do Algarve.
\section{Escampo}
\begin{itemize}
\item {Grp. gram.:m.}
\end{itemize}
(V.descampado)
\section{Escamudo}
\begin{itemize}
\item {Grp. gram.:adj.}
\end{itemize}
Que tem muitas escamas.
\section{Escamugir-se}
\begin{itemize}
\item {Grp. gram.:v. p.}
\end{itemize}
Escapulir, fugir:«\textunderscore o gentil-homem e a galante menina escamugiram-se\textunderscore ». Camillo, \textunderscore Mulher Fatal\textunderscore , 151.
\section{Escâmula}
\begin{itemize}
\item {Grp. gram.:f.}
\end{itemize}
Pequena escama.
(Dem. de \textunderscore escama\textunderscore )
\section{Escanado}
\begin{itemize}
\item {Grp. gram.:adj.}
\end{itemize}
\begin{itemize}
\item {Proveniência:(De cana)}
\end{itemize}
Diz-se das aves, que já não têm matéria sanguinea nas penas grandes, como quando eram novas.
Que é adulto, velho.
\section{Escanar-se}
\begin{itemize}
\item {Grp. gram.:v. p.}
\end{itemize}
Tornar-se escanado, (falando-se de aves de rapina). Cf. Fern. Pereira, \textunderscore Caça de Altan.\textunderscore , p. II, c. 2.
\section{Escanastrado}
\begin{itemize}
\item {Grp. gram.:adj.}
\end{itemize}
\begin{itemize}
\item {Utilização:Prov.}
\end{itemize}
\begin{itemize}
\item {Utilização:trasm.}
\end{itemize}
\begin{itemize}
\item {Proveniência:(De \textunderscore canastro\textunderscore )}
\end{itemize}
Fraco; alquebrado.
Escavacado.
\section{Escanção}
\begin{itemize}
\item {Grp. gram.:m.}
\end{itemize}
\begin{itemize}
\item {Utilização:Des.}
\end{itemize}
\begin{itemize}
\item {Proveniência:(Do lat. \textunderscore scantio\textunderscore )}
\end{itemize}
Aquelle que distribue vinho pelos commensaes; copeiro.
\section{Escançar}
\begin{itemize}
\item {Grp. gram.:v. t.}
\end{itemize}
O mesmo que \textunderscore escancear\textunderscore .
\section{Escâncara}
\begin{itemize}
\item {Grp. gram.:f.}
\end{itemize}
\begin{itemize}
\item {Grp. gram.:Loc. adv.}
\end{itemize}
\begin{itemize}
\item {Proveniência:(De \textunderscore escancarar\textunderscore )}
\end{itemize}
Estado daquillo que é patente, claro, manifesto.
\textunderscore Ás escâncaras\textunderscore , claramente em público, sem rebuço.
\section{Escancaração}
\begin{itemize}
\item {Grp. gram.:f.}
\end{itemize}
\begin{itemize}
\item {Utilização:Burl.}
\end{itemize}
Acto ou effeito de escancarar.
\section{Escancarar}
\begin{itemize}
\item {Grp. gram.:v. t.}
\end{itemize}
\begin{itemize}
\item {Proveniência:(De \textunderscore câncaro\textunderscore , por \textunderscore cancro\textunderscore , Cf. it. \textunderscore sgangherare\textunderscore )}
\end{itemize}
Patentear; abrir de par em par.
Abrir inteiramente.
\section{Escançaria}
\begin{itemize}
\item {Grp. gram.:f.}
\end{itemize}
\begin{itemize}
\item {Utilização:Ant.}
\end{itemize}
\begin{itemize}
\item {Proveniência:(De \textunderscore escanção\textunderscore )}
\end{itemize}
Lugar, em que se repartia o vinho.
\section{Escancear}
\begin{itemize}
\item {Grp. gram.:v. t.}
\end{itemize}
\begin{itemize}
\item {Proveniência:(De \textunderscore escanção\textunderscore )}
\end{itemize}
Repartir (vinho) pelos convidados ou commensaes.
\section{Escancelamento}
\begin{itemize}
\item {Grp. gram.:m.}
\end{itemize}
Acto de escancelar.
\section{Escancelar}
\begin{itemize}
\item {Grp. gram.:v. t.}
\end{itemize}
\begin{itemize}
\item {Utilização:Bras}
\end{itemize}
\begin{itemize}
\item {Grp. gram.:V. p.}
\end{itemize}
\begin{itemize}
\item {Utilização:Prov.}
\end{itemize}
\begin{itemize}
\item {Utilização:minh.}
\end{itemize}
\begin{itemize}
\item {Proveniência:(De \textunderscore cancelo\textunderscore )}
\end{itemize}
Abrir muito (os olhos, a bôca, etc.).
Desconjuntar-se; escangalhar-se.
\section{Escancellamento}
\begin{itemize}
\item {Grp. gram.:m.}
\end{itemize}
Acto de escancellar.
\section{Escancellar}
\begin{itemize}
\item {Grp. gram.:v. t.}
\end{itemize}
\begin{itemize}
\item {Utilização:Bras}
\end{itemize}
\begin{itemize}
\item {Grp. gram.:V. p.}
\end{itemize}
\begin{itemize}
\item {Utilização:Prov.}
\end{itemize}
\begin{itemize}
\item {Utilização:minh.}
\end{itemize}
\begin{itemize}
\item {Proveniência:(De \textunderscore cancello\textunderscore )}
\end{itemize}
Abrir muito (os olhos, a bôca, etc.).
Desconjuntar-se; escangalhar-se.
\section{Escanchar}
\begin{itemize}
\item {Grp. gram.:v. t.}
\end{itemize}
\begin{itemize}
\item {Grp. gram.:V. p.}
\end{itemize}
Separar ao meio.
Alargar: \textunderscore escanchar as pernas\textunderscore .
Escachar.
Sentar-se sôbre um objecto, pondo uma perna de cada um dos lados delle.
Escarranchar-se.
(Corr. de \textunderscore escachar\textunderscore ? Ou por \textunderscore esganchar\textunderscore , de \textunderscore gancho\textunderscore ?)
\section{Escanchas}
\begin{itemize}
\item {Grp. gram.:f. pl.}
\end{itemize}
\begin{itemize}
\item {Utilização:Prov.}
\end{itemize}
\begin{itemize}
\item {Utilização:trasm.}
\end{itemize}
O mesmo que \textunderscore andas\textunderscore .
\section{Escandalizador}
\begin{itemize}
\item {Grp. gram.:adj.}
\end{itemize}
\begin{itemize}
\item {Grp. gram.:M.}
\end{itemize}
Que escandaliza.
Aquelle que escandaliza.
\section{Escandalizar}
\begin{itemize}
\item {Grp. gram.:v. t.}
\end{itemize}
\begin{itemize}
\item {Grp. gram.:V. i.}
\end{itemize}
\begin{itemize}
\item {Proveniência:(Lat. \textunderscore scandalizare\textunderscore )}
\end{itemize}
Causar escândalo a.
Melindrar.
Fazer offensa a.
Aggravar, maltratar.
Fazer escândalo.
Proceder mal.
\section{Escândalo}
\begin{itemize}
\item {Grp. gram.:m.}
\end{itemize}
\begin{itemize}
\item {Utilização:Fig.}
\end{itemize}
\begin{itemize}
\item {Proveniência:(Lat. \textunderscore scandalum\textunderscore )}
\end{itemize}
Aquillo que póde induzir em êrro ou peccado.
Indignação, produzida pelos maus exemplos.
Provocação ao mal, pelo mau exemplo.
Offensa, injúria.
Pessôa ou coisa que escandaliza.
\section{Escandalosamente}
\begin{itemize}
\item {Grp. gram.:adv.}
\end{itemize}
De modo escandaloso.
\section{Escandaloso}
\begin{itemize}
\item {Grp. gram.:adj.}
\end{itemize}
\begin{itemize}
\item {Proveniência:(Lat. \textunderscore scandalosus\textunderscore )}
\end{itemize}
Que causa escândalo.
Em que há escândalo: \textunderscore acto escandaloso\textunderscore .
\section{Escandar}
\begin{itemize}
\item {Grp. gram.:v. t.}
\end{itemize}
\begin{itemize}
\item {Utilização:Gal}
\end{itemize}
Indicar a quantidade ou medida dos versos, decompondo-os nas suas differentes unidades métricas ou syllábicas:«\textunderscore Bella ocupação para um homem de vinte e um annos, escandar jambos e trocheus.\textunderscore »Garrett, \textunderscore Viagens\textunderscore .
(Gal. inútil, do fr. \textunderscore scander\textunderscore , visto termos \textunderscore escandir\textunderscore , do lat. \textunderscore scandere\textunderscore )
\section{Escândea}
\begin{itemize}
\item {Grp. gram.:f.}
\end{itemize}
\begin{itemize}
\item {Proveniência:(Do lat. \textunderscore scandula\textunderscore )}
\end{itemize}
Trigo durazio.
\section{Escandecência}
\begin{itemize}
\item {Grp. gram.:f.}
\end{itemize}
\begin{itemize}
\item {Proveniência:(Lat. \textunderscore excandescentia\textunderscore )}
\end{itemize}
Acto de escandecer.
Estado daquillo que é escandecente.
\section{Escandecente}
\begin{itemize}
\item {Grp. gram.:adj.}
\end{itemize}
\begin{itemize}
\item {Proveniência:(Lat. \textunderscore excandescens\textunderscore )}
\end{itemize}
Que escandece.
\section{Escandecer}
\begin{itemize}
\item {Grp. gram.:v. i.}
\end{itemize}
\begin{itemize}
\item {Proveniência:(Lat. \textunderscore excandescere\textunderscore )}
\end{itemize}
Pôr-se em brasa.
Inflammar-se.
Tornar-se ardente.
Queimar muito.
\section{Escandecido}
\begin{itemize}
\item {Grp. gram.:adj.}
\end{itemize}
\begin{itemize}
\item {Proveniência:(De \textunderscore escandecer\textunderscore )}
\end{itemize}
Inflammado, ardente:«\textunderscore ...a cara escandecida numa congestão de júbilo\textunderscore ». Camillo, \textunderscore Brasileira\textunderscore , 128.
\section{Escandinavo}
\begin{itemize}
\item {Grp. gram.:m.}
\end{itemize}
\begin{itemize}
\item {Grp. gram.:Adj.}
\end{itemize}
Habitante da Escandinávia.
Relativo á Escandinávia.
\section{Escandir}
\begin{itemize}
\item {Grp. gram.:v. t.}
\end{itemize}
\begin{itemize}
\item {Utilização:Des.}
\end{itemize}
\begin{itemize}
\item {Utilização:Fig.}
\end{itemize}
\begin{itemize}
\item {Proveniência:(Do lat. \textunderscore scandere\textunderscore )}
\end{itemize}
Medir (versos). Cf. Castilho, \textunderscore Fastos\textunderscore , II, 583.
Examinar miudamente.
Enumerar.
\section{Escândula}
\begin{itemize}
\item {Grp. gram.:f.}
\end{itemize}
\begin{itemize}
\item {Utilização:Pop.}
\end{itemize}
Escândalo.
Motivos de queixa.
Offensa particular: \textunderscore tive delle muitas escândulas\textunderscore .
(Corr. de \textunderscore escândalo\textunderscore )
\section{Escanelado}
\begin{itemize}
\item {Grp. gram.:adj.}
\end{itemize}
\begin{itemize}
\item {Proveniência:(De \textunderscore canela\textunderscore )}
\end{itemize}
Que tem pernas esguias; magro.
\section{Escanência}
\begin{itemize}
\item {Grp. gram.:f.}
\end{itemize}
\begin{itemize}
\item {Utilização:Gír. lisb.}
\end{itemize}
Comida.
(Cp. lat. \textunderscore esca\textunderscore )
\section{Escanevada}
\begin{itemize}
\item {Grp. gram.:f.}
\end{itemize}
\begin{itemize}
\item {Utilização:Prov.}
\end{itemize}
\begin{itemize}
\item {Utilização:beir.}
\end{itemize}
O mesmo que \textunderscore esgaravanada\textunderscore .
\section{Escangadeira}
\begin{itemize}
\item {Grp. gram.:f.}
\end{itemize}
\begin{itemize}
\item {Utilização:Prov.}
\end{itemize}
\begin{itemize}
\item {Utilização:trasm.}
\end{itemize}
\begin{itemize}
\item {Proveniência:(De \textunderscore escangar\textunderscore )}
\end{itemize}
Peneira, que separa do trigo o farelo, sem tirar as sêmeas.
\section{Escangalhação}
\begin{itemize}
\item {Grp. gram.:f.}
\end{itemize}
\begin{itemize}
\item {Utilização:Burl.}
\end{itemize}
Acto de escangalhar:«\textunderscore ...dei escangalhação dos cinco reis.\textunderscore »Camillo, \textunderscore Mosaico\textunderscore , 16.
\section{Escangalhar}
\begin{itemize}
\item {Grp. gram.:v. t.}
\end{itemize}
\begin{itemize}
\item {Proveniência:(De \textunderscore cangalho\textunderscore )}
\end{itemize}
Desmanchar.
Esbarrondar.
Esbandalhar; desconjuntar.
\section{Escangalho}
\begin{itemize}
\item {Grp. gram.:m.}
\end{itemize}
\begin{itemize}
\item {Utilização:Bras. do N}
\end{itemize}
\begin{itemize}
\item {Proveniência:(De \textunderscore escangalhar\textunderscore )}
\end{itemize}
Ruína; desordem.
Confusão.
\section{Escangalho}
\begin{itemize}
\item {Grp. gram.:m.}
\end{itemize}
\begin{itemize}
\item {Utilização:Bras. do Rio}
\end{itemize}
Parede escarpada, para suster as terras de um monte.
\section{Escanganhadeira}
\begin{itemize}
\item {Grp. gram.:f.}
\end{itemize}
\begin{itemize}
\item {Utilização:Des.}
\end{itemize}
Tabuleiro com parede, para escanganhar.
\section{Escanganhar}
\begin{itemize}
\item {Grp. gram.:v. t.}
\end{itemize}
\begin{itemize}
\item {Utilização:Prov.}
\end{itemize}
\begin{itemize}
\item {Proveniência:(De \textunderscore canganho\textunderscore )}
\end{itemize}
Separar do canganho ou do engaço (os bagos da uva).
\section{Escanganho}
\begin{itemize}
\item {Grp. gram.:m.}
\end{itemize}
Acto de escanganhar.
\section{Escangar}
\begin{itemize}
\item {Grp. gram.:v. t.}
\end{itemize}
\begin{itemize}
\item {Utilização:Prov.}
\end{itemize}
\begin{itemize}
\item {Utilização:trasm.}
\end{itemize}
Peneirar, (farinha de trigo), sem separar as sêmeas.
(Relaciona-se com \textunderscore escanganhar\textunderscore )
\section{Escanho}
\begin{itemize}
\item {Grp. gram.:m.}
\end{itemize}
(V.escano)
\section{Escanhoador}
\begin{itemize}
\item {Grp. gram.:m.  e  adj.}
\end{itemize}
O que escanhôa.
\section{Escanhoamento}
\begin{itemize}
\item {Grp. gram.:m.}
\end{itemize}
Acto ou effeito de escanhoar. Cf. Camillo, \textunderscore Vingança\textunderscore , XVIII.
\section{Escanhoar}
\begin{itemize}
\item {Grp. gram.:v. t.}
\end{itemize}
Barbear com apuro.
\section{Escanhotador}
\begin{itemize}
\item {Grp. gram.:m.  e  adj.}
\end{itemize}
O que escanhota.
\section{Escanhotar}
\begin{itemize}
\item {Grp. gram.:v. t.}
\end{itemize}
\begin{itemize}
\item {Utilização:Prov.}
\end{itemize}
\begin{itemize}
\item {Grp. gram.:V. i.}
\end{itemize}
\begin{itemize}
\item {Utilização:Prov.}
\end{itemize}
\begin{itemize}
\item {Utilização:minh.}
\end{itemize}
Cortar os canhotos ou ramos grossos de.
Procurar ou cortar canhotos para lenha.
Bater.
\section{Escanifrado}
\begin{itemize}
\item {Grp. gram.:adj.}
\end{itemize}
\begin{itemize}
\item {Utilização:Pop.}
\end{itemize}
\begin{itemize}
\item {Proveniência:(Do lat. \textunderscore canis\textunderscore )}
\end{itemize}
Muito magro; escanzelado.
Que é magrizela.
\section{Escanifrar}
\begin{itemize}
\item {Grp. gram.:v. t.}
\end{itemize}
Tornar escanifrado.
Enfranquecer, entanguir. Cf. \textunderscore Techn. Rur.\textunderscore , 237; Camillo, \textunderscore Narcót.\textunderscore , I, 259.
\section{Escanifre}
\begin{itemize}
\item {Grp. gram.:m.}
\end{itemize}
\begin{itemize}
\item {Utilização:Pop.}
\end{itemize}
Homem escanifrado.
\section{Escaninho}
\begin{itemize}
\item {Grp. gram.:m.}
\end{itemize}
Pequeno compartimento, dentro de caixa, gaveta, etc.
Recanto.
Esconso.
(Dem. de \textunderscore escano\textunderscore , visto que o escano ou escabello, além de assento, serve de caixa)
\section{Escano}
\begin{itemize}
\item {Grp. gram.:m.}
\end{itemize}
\begin{itemize}
\item {Proveniência:(Lat. \textunderscore scamnum\textunderscore )}
\end{itemize}
O mesmo que \textunderscore escabêllo\textunderscore .
\section{Escantilhão}
\begin{itemize}
\item {Grp. gram.:m.}
\end{itemize}
\begin{itemize}
\item {Grp. gram.:Loc. adv.}
\end{itemize}
\begin{itemize}
\item {Utilização:Prov.}
\end{itemize}
\begin{itemize}
\item {Utilização:trasm.}
\end{itemize}
Medida, com que se regulam as distâncias em trabalhos agrícolas ou mechânicos.
Medida official, que serve de padrão no aferimento das medidas públicas.
Régua ou pau, com que os pedreiros vão aferindo a largura de uma parede, que constróem.
\textunderscore De escantilhão\textunderscore , apressadamente; em desordem, aos tombos; de roldão.
Obliquamente, de esguelha.
Em direcção aos cantos de uma superfície quadrangular.
(Talvez, de uma fórma ant., \textunderscore escantilha\textunderscore , de canto)
\section{Escanzelado}
\begin{itemize}
\item {Grp. gram.:adj.}
\end{itemize}
\begin{itemize}
\item {Utilização:Pop.}
\end{itemize}
\begin{itemize}
\item {Proveniência:(De cão. Cp. \textunderscore canzoada\textunderscore )}
\end{itemize}
Muito magro.
Magro como um cão faminto.
\section{Escanzorrado}
\begin{itemize}
\item {Grp. gram.:adj.}
\end{itemize}
\begin{itemize}
\item {Utilização:T. da Bairrada}
\end{itemize}
Disfarçado, surrateiro.
\section{Escafa}
\begin{itemize}
\item {Grp. gram.:f.}
\end{itemize}
\begin{itemize}
\item {Utilização:Ant.}
\end{itemize}
\begin{itemize}
\item {Proveniência:(Lat. \textunderscore scapha\textunderscore )}
\end{itemize}
O mesmo que \textunderscore quadrante\textunderscore .
\section{Escafandro}
\begin{itemize}
\item {Grp. gram.:m.}
\end{itemize}
\begin{itemize}
\item {Proveniência:(Do gr. \textunderscore scaphe\textunderscore  + \textunderscore aner\textunderscore , \textunderscore andros\textunderscore )}
\end{itemize}
Aparelho impermeável, que reveste os mergulhadores e lhes permite trabalhar debaixo de água.
Gênero de conchas univalves.
\section{Escáfide}
\begin{itemize}
\item {Grp. gram.:f.}
\end{itemize}
Gênero de líchens.
\section{Escafídeo}
\begin{itemize}
\item {Grp. gram.:m.}
\end{itemize}
\begin{itemize}
\item {Proveniência:(Do gr. \textunderscore skaphe\textunderscore  + \textunderscore eidos\textunderscore )}
\end{itemize}
Gênero de insectos coleópteros pentâmeros.
\section{Escafocefalia}
\begin{itemize}
\item {Grp. gram.:f.}
\end{itemize}
Qualidade de escafocéfalo.
\section{Escafocéfalo}
\begin{itemize}
\item {Grp. gram.:adj.}
\end{itemize}
\begin{itemize}
\item {Proveniência:(Do gr. \textunderscore skaphe\textunderscore  + \textunderscore kephale\textunderscore )}
\end{itemize}
Que tem a cabeça em fórma de casco de barco.
\section{Escafoide}
\begin{itemize}
\item {Grp. gram.:m.}
\end{itemize}
\begin{itemize}
\item {Utilização:Anat.}
\end{itemize}
\begin{itemize}
\item {Proveniência:(Do gr. \textunderscore skaphe\textunderscore  + \textunderscore eidos\textunderscore )}
\end{itemize}
Osso da mão, o maior da primeira série do carpo.
Osso do pé, que abrange a parte interna do tarso.
\section{Escapada}
\begin{itemize}
\item {Grp. gram.:f.}
\end{itemize}
O mesmo que \textunderscore escapadela\textunderscore .
\section{Escapadela}
\begin{itemize}
\item {Grp. gram.:f.}
\end{itemize}
\begin{itemize}
\item {Utilização:Pop.}
\end{itemize}
\begin{itemize}
\item {Proveniência:(De \textunderscore escapar\textunderscore )}
\end{itemize}
Fuga precipitada.
Acto de fugir a um dever, para se divertir: \textunderscore escapadela do collégio\textunderscore .
Escorregadella, culpa.
\section{Escapadiço}
\begin{itemize}
\item {Grp. gram.:adj.}
\end{itemize}
\begin{itemize}
\item {Proveniência:(De \textunderscore escapar\textunderscore )}
\end{itemize}
Que anda fugido; que escapou de alguma coisa; amorado^2.
\section{Escapamento}
\begin{itemize}
\item {Grp. gram.:m.}
\end{itemize}
Acto de escapar.
Escape.
\section{Escapar}
\begin{itemize}
\item {Grp. gram.:v. i.}
\end{itemize}
\begin{itemize}
\item {Grp. gram.:V. p.}
\end{itemize}
\begin{itemize}
\item {Proveniência:(It. \textunderscore scampare\textunderscore )}
\end{itemize}
Subtrahir-se.
Livrar-se; fugir de um perigo, de doença ou de coisa desagradavel: \textunderscore escapar de uma pneumonia\textunderscore .
Passar despercebido: \textunderscore escapou-me a tal nota\textunderscore .
Omittir-se.
Sobreviver.
Isentar-se: \textunderscore escapar do serviço militar\textunderscore .
Desapparecer, fugir.
Libertar-se.
\section{Escaparate}
\begin{itemize}
\item {Grp. gram.:m.}
\end{itemize}
\begin{itemize}
\item {Utilização:Fig.}
\end{itemize}
\begin{itemize}
\item {Proveniência:(Do holl. \textunderscore schaprade\textunderscore , armário)}
\end{itemize}
Redoma.
Pequeno armário.
Cantoneira.
Subterfúgio.
\section{Escapatória}
\begin{itemize}
\item {Grp. gram.:f.}
\end{itemize}
\begin{itemize}
\item {Utilização:Pop.}
\end{itemize}
Subterfúgio; escapadela.
(Cast. \textunderscore escapatoria\textunderscore )
\section{Escapatório}
\begin{itemize}
\item {Grp. gram.:m.}
\end{itemize}
\begin{itemize}
\item {Utilização:Pop.}
\end{itemize}
(V.escapatória)
\section{Escape}
\begin{itemize}
\item {Grp. gram.:m.}
\end{itemize}
\begin{itemize}
\item {Grp. gram.:Adj.}
\end{itemize}
\begin{itemize}
\item {Proveniência:(De \textunderscore escapar\textunderscore )}
\end{itemize}
Acto de escapar.
Escapadela.
O mesmo que \textunderscore escapo\textunderscore ^2. Cf. Camillo, \textunderscore Caveira\textunderscore , 111.
\section{Escape}
\begin{itemize}
\item {Grp. gram.:m.}
\end{itemize}
O mesmo que \textunderscore escapo\textunderscore ^1.
\section{Escapelada}
\begin{itemize}
\item {Grp. gram.:f.}
\end{itemize}
Acto de escapelar.
\section{Escapelar}
\begin{itemize}
\item {Grp. gram.:v. t.}
\end{itemize}
\begin{itemize}
\item {Proveniência:(De \textunderscore capela\textunderscore )}
\end{itemize}
O mesmo que \textunderscore descamisar\textunderscore .
\section{Escapellada}
\begin{itemize}
\item {Grp. gram.:f.}
\end{itemize}
Acto de escapellar.
\section{Escapellar}
\begin{itemize}
\item {Grp. gram.:v. t.}
\end{itemize}
\begin{itemize}
\item {Proveniência:(De \textunderscore capella\textunderscore )}
\end{itemize}
O mesmo que \textunderscore descamisar\textunderscore .
\section{Escapha}
\begin{itemize}
\item {Grp. gram.:f.}
\end{itemize}
\begin{itemize}
\item {Utilização:Ant.}
\end{itemize}
\begin{itemize}
\item {Proveniência:(Lat. \textunderscore scapha\textunderscore )}
\end{itemize}
O mesmo que \textunderscore quadrante\textunderscore .
\section{Escaphandro}
\begin{itemize}
\item {Grp. gram.:m.}
\end{itemize}
\begin{itemize}
\item {Proveniência:(Do gr. \textunderscore scaphe\textunderscore  + \textunderscore aner\textunderscore , \textunderscore andros\textunderscore )}
\end{itemize}
Apparelho impermeável, que reveste os mergulhadores e lhes permitte trabalhar debaixo de água.
Gênero de conchas univalves.
\section{Escáphide}
\begin{itemize}
\item {Grp. gram.:f.}
\end{itemize}
Gênero de líchens.
\section{Escaphídeo}
\begin{itemize}
\item {Grp. gram.:m.}
\end{itemize}
\begin{itemize}
\item {Proveniência:(Do gr. \textunderscore skaphe\textunderscore  + \textunderscore eidos\textunderscore )}
\end{itemize}
Gênero de insectos coleópteros pentâmeros.
\section{Escaphocephalia}
\begin{itemize}
\item {Grp. gram.:f.}
\end{itemize}
Qualidade de escaphocéphalo.
\section{Escaphocéphalo}
\begin{itemize}
\item {Grp. gram.:adj.}
\end{itemize}
\begin{itemize}
\item {Proveniência:(Do gr. \textunderscore skaphe\textunderscore  + \textunderscore kephale\textunderscore )}
\end{itemize}
Que tem a cabeça em fórma de casco de barco.
\section{Escaphoide}
\begin{itemize}
\item {Grp. gram.:m.}
\end{itemize}
\begin{itemize}
\item {Utilização:Anat.}
\end{itemize}
\begin{itemize}
\item {Proveniência:(Do gr. \textunderscore skaphe\textunderscore  + \textunderscore eidos\textunderscore )}
\end{itemize}
Osso da mão, o maior da primeira série do carpo.
Osso do pé, que abrange a parte interna do tarso.
\section{Escapo}
\begin{itemize}
\item {Grp. gram.:m.}
\end{itemize}
\begin{itemize}
\item {Utilização:Mús.}
\end{itemize}
\begin{itemize}
\item {Proveniência:(Lat. \textunderscore scapus\textunderscore )}
\end{itemize}
Mechanismo, com que se regula o movimento dos relógios.
Haste, que nas plantas acaules sái de um grupo de fôlhas e dá origem a uma flôr ou grupo de flôres.
Quadrante, que liga o fuste da columna ao capitel.
Pequena lingueta de madeira, que, no mechanismo dos pianos, serve para impellir o martello contra a corda.
\section{Escapo}
\begin{itemize}
\item {Grp. gram.:adj.}
\end{itemize}
\begin{itemize}
\item {Utilização:Pop.}
\end{itemize}
\begin{itemize}
\item {Proveniência:(De \textunderscore escapar\textunderscore )}
\end{itemize}
Que escapou.
Livre; isento.
\section{Escápole}
\begin{itemize}
\item {Grp. gram.:adj.}
\end{itemize}
\begin{itemize}
\item {Proveniência:(De \textunderscore escapar\textunderscore . Cp. \textunderscore escapulir\textunderscore )}
\end{itemize}
Livre de obrigações.
Escapo^2.
\section{Escápula}
\begin{itemize}
\item {Grp. gram.:f.}
\end{itemize}
\begin{itemize}
\item {Utilização:Fig.}
\end{itemize}
\begin{itemize}
\item {Proveniência:(Do lat. \textunderscore scapula\textunderscore )}
\end{itemize}
Prego, cuja cabeça se dobra em ângulo, para suster outro objecto.
Apoio, esteio.
\section{Escapúla}
\begin{itemize}
\item {Grp. gram.:f.}
\end{itemize}
\begin{itemize}
\item {Utilização:Pop.}
\end{itemize}
\begin{itemize}
\item {Proveniência:(Do rad. de \textunderscore escapulir\textunderscore )}
\end{itemize}
Escapadela.
Escapatória.
\section{Escapulal}
\begin{itemize}
\item {Grp. gram.:adj.}
\end{itemize}
O mesmo que \textunderscore escapular\textunderscore .
\section{Escapular}
\begin{itemize}
\item {Grp. gram.:adj.}
\end{itemize}
\begin{itemize}
\item {Utilização:Anat.}
\end{itemize}
\begin{itemize}
\item {Proveniência:(Lat. \textunderscore scapularis\textunderscore )}
\end{itemize}
Relativo ao ombro.
\section{Escapulário}
\begin{itemize}
\item {Grp. gram.:m.}
\end{itemize}
\begin{itemize}
\item {Proveniência:(Lat. \textunderscore scapularius\textunderscore )}
\end{itemize}
Tira de pano, que os frades de algumas Ordens usavam sôbre os ombros, pendente sôbre o peito.
Bentinhos.
Ligadura para comprimir emplastros ou parches.
\section{Escapulir}
\begin{itemize}
\item {Grp. gram.:v. t.}
\end{itemize}
\begin{itemize}
\item {Utilização:Pop.}
\end{itemize}
\begin{itemize}
\item {Grp. gram.:V. i.}
\end{itemize}
\begin{itemize}
\item {Proveniência:(De \textunderscore escapar\textunderscore )}
\end{itemize}
Deixar fugir.
Deixar escapar.
Fugir da prisão.
Escapar do poder de alguém.
Acontecer por descuido.
\section{Escaquear}
\begin{itemize}
\item {Grp. gram.:v. t.}
\end{itemize}
Dividir em escaques.
\section{Escaqueirar}
\begin{itemize}
\item {Grp. gram.:v. t.}
\end{itemize}
\begin{itemize}
\item {Proveniência:(De \textunderscore caqueiro\textunderscore )}
\end{itemize}
Fazer em cacos; escacar.
Despedaçar.
\section{Escaques}
\begin{itemize}
\item {Grp. gram.:m. pl.}
\end{itemize}
\begin{itemize}
\item {Utilização:Heráld.}
\end{itemize}
\begin{itemize}
\item {Proveniência:(It. \textunderscore scacchi\textunderscore , pl. de \textunderscore scacco\textunderscore )}
\end{itemize}
Divisões quadradas do escudo, mas em côres alternadas.
As casas quadradas do tabuleiro de xadrez.
\section{Escara}
\begin{itemize}
\item {Grp. gram.:f.}
\end{itemize}
\begin{itemize}
\item {Proveniência:(Do gr. \textunderscore eskhara\textunderscore )}
\end{itemize}
Crosta de ferida que resultou de gangrena ou de cáustico.
\section{Escarabeu}
\begin{itemize}
\item {Grp. gram.:m.}
\end{itemize}
O mesmo que \textunderscore escaravêlho\textunderscore .
\section{Escarabídeos}
\begin{itemize}
\item {Grp. gram.:m. pl.}
\end{itemize}
\begin{itemize}
\item {Proveniência:(Do gr. \textunderscore skarabos\textunderscore  + \textunderscore eidos\textunderscore )}
\end{itemize}
Tribo de insectos coleópteros, que têm por typo o escaravelho.
\section{Escarabocho}
\begin{itemize}
\item {fónica:bô}
\end{itemize}
\begin{itemize}
\item {Grp. gram.:m.}
\end{itemize}
\begin{itemize}
\item {Utilização:Pop.}
\end{itemize}
\begin{itemize}
\item {Proveniência:(Do it. \textunderscore scarabocchio\textunderscore )}
\end{itemize}
Esbôço imperfeito.
Delineamento tôsco.
Borrão.
\section{Escaração}
\begin{itemize}
\item {Grp. gram.:f.}
\end{itemize}
Acto ou effeito de escarar-se.
\section{Escarafolar-se}
\begin{itemize}
\item {Grp. gram.:v. p.}
\end{itemize}
\begin{itemize}
\item {Utilização:Prov.}
\end{itemize}
\begin{itemize}
\item {Utilização:trasm.}
\end{itemize}
Diz-se do pião ou da baraça do pião, quando as roscas, que ella fórma, se desfazem por falta de cuidado na sua sobreposição.
Escarpelar-se.
\section{Escarafunchador}
\begin{itemize}
\item {Grp. gram.:m.  e  adj.}
\end{itemize}
O que escarafuncha.
\section{Escarafunchar}
\begin{itemize}
\item {Grp. gram.:v. t.}
\end{itemize}
\begin{itemize}
\item {Proveniência:(Do lat. hyp. \textunderscore scaraphunculare\textunderscore )}
\end{itemize}
Esgravatar.
Remexer (a terra) como as gallinhas.
Procurar, investigar com paciência.
\section{Escarafuncho}
\begin{itemize}
\item {Grp. gram.:m.}
\end{itemize}
\begin{itemize}
\item {Utilização:Prov.}
\end{itemize}
\begin{itemize}
\item {Utilização:alent.}
\end{itemize}
\begin{itemize}
\item {Proveniência:(De \textunderscore escarafunchar\textunderscore )}
\end{itemize}
Bailarico.
\section{Escarafunchos}
\begin{itemize}
\item {Grp. gram.:m. pl.}
\end{itemize}
O mesmo que \textunderscore escarafunhas\textunderscore .
\section{Escarafunhas}
\begin{itemize}
\item {Grp. gram.:f. pl.}
\end{itemize}
O mesmo que \textunderscore garafunhos\textunderscore .
\section{Escarambada}
\begin{itemize}
\item {Grp. gram.:f.}
\end{itemize}
Acto de escarambar-se.
\section{Escarambar-se}
\begin{itemize}
\item {Grp. gram.:v. p.}
\end{itemize}
Secar-se muito e gretar (a terra), por effeito de grande calor.
\section{Escaramelado}
\begin{itemize}
\item {Grp. gram.:adj.}
\end{itemize}
\begin{itemize}
\item {Utilização:Prov.}
\end{itemize}
\begin{itemize}
\item {Utilização:minh.}
\end{itemize}
Diz-se de quem tem aspecto desagradável ou mostra têr doença de mau agoiro.
(Relaciona-se com \textunderscore cara\textunderscore )
\section{Escaramuça}
\begin{itemize}
\item {Grp. gram.:f.}
\end{itemize}
\begin{itemize}
\item {Utilização:Fig.}
\end{itemize}
\begin{itemize}
\item {Proveniência:(Do b. lat. \textunderscore scaramutia\textunderscore )}
\end{itemize}
Combate de pequena importância.
Peleja entre alguns troços de tropas contrárias.
Briga; conflicto.
\section{Escaramuçador}
\begin{itemize}
\item {Grp. gram.:adj.}
\end{itemize}
\begin{itemize}
\item {Grp. gram.:M.}
\end{itemize}
Que escaramuça.
Aquelle que escaramuça.
\section{Escaramuçar}
\begin{itemize}
\item {Grp. gram.:v. t.}
\end{itemize}
\begin{itemize}
\item {Grp. gram.:V. i.}
\end{itemize}
Obrigar (o cavallo) a dar muitas voltas.
Fazer escaramuça.
\section{Escaramuceiro}
\begin{itemize}
\item {Grp. gram.:m.}
\end{itemize}
Brigão; desordeiro. Cf. Camillo, \textunderscore Estrel. Prop.\textunderscore , 100.
\section{Escarapão}
\begin{itemize}
\item {Grp. gram.:m.}
\end{itemize}
\begin{itemize}
\item {Utilização:Prov.}
\end{itemize}
\begin{itemize}
\item {Utilização:alent.}
\end{itemize}
\begin{itemize}
\item {Utilização:Fig.}
\end{itemize}
Cobra inoffensiva, sem peçonha, de dorso escuro e barriga amarela.
Indivíduo irascível.
\section{Escarapela}
\begin{itemize}
\item {Grp. gram.:f.}
\end{itemize}
\begin{itemize}
\item {Utilização:Pop.}
\end{itemize}
\begin{itemize}
\item {Utilização:Bras}
\end{itemize}
\begin{itemize}
\item {Utilização:fam.}
\end{itemize}
Acto de escarapelar-se.
Briga.
\section{Escarapelar}
\begin{itemize}
\item {Grp. gram.:v. t.}
\end{itemize}
\begin{itemize}
\item {Grp. gram.:V. i.}
\end{itemize}
\begin{itemize}
\item {Utilização:Pop.}
\end{itemize}
\begin{itemize}
\item {Proveniência:(De \textunderscore carapela\textunderscore )}
\end{itemize}
O mesmo que \textunderscore escarpelar\textunderscore .
Brigar, arrepelando.
Armar desordem.
\section{Escarar-se}
\begin{itemize}
\item {Grp. gram.:v. p.}
\end{itemize}
\begin{itemize}
\item {Utilização:Prov.}
\end{itemize}
\begin{itemize}
\item {Utilização:alg.}
\end{itemize}
\begin{itemize}
\item {Proveniência:(De \textunderscore cara\textunderscore )}
\end{itemize}
Embebedar-se.
Tornar-se descarado.
\section{Escaravalhado}
\begin{itemize}
\item {Grp. gram.:adj.}
\end{itemize}
Que tem escaravalhos.
\section{Escaravalho}
\begin{itemize}
\item {Grp. gram.:m.}
\end{itemize}
(V.escarvalho)
\section{Escaravelha}
\begin{itemize}
\item {Grp. gram.:f.}
\end{itemize}
\begin{itemize}
\item {Utilização:Prov.}
\end{itemize}
O mesmo que \textunderscore cravelha\textunderscore  de viola ou de outro instrumento de cordas. Cf. F. Manuel, \textunderscore Visita das Fontes\textunderscore .
\section{Escaravelhar}
\begin{itemize}
\item {Grp. gram.:v. i.}
\end{itemize}
\begin{itemize}
\item {Utilização:Prov.}
\end{itemize}
\begin{itemize}
\item {Utilização:trasm.}
\end{itemize}
\begin{itemize}
\item {Utilização:beir.}
\end{itemize}
\begin{itemize}
\item {Proveniência:(De \textunderscore escaravêlho\textunderscore )}
\end{itemize}
Remexer-se como escaravelho.
Saltitar (o pião) no terreiro, por têr empenado o ferrão.
\section{Escaravêlho}
\begin{itemize}
\item {Grp. gram.:m.}
\end{itemize}
\begin{itemize}
\item {Proveniência:(Do lat. \textunderscore scarabaeus\textunderscore )}
\end{itemize}
Insecto escuro, de asas membranosas, pertencente á ordem dos coleópteros pentâmeros.
Ponta de marfim, antes de manufacturado. Cf. \textunderscore Inquér. Industr.\textunderscore , I, 252.
\section{Escarça}
\begin{itemize}
\item {Grp. gram.:f.}
\end{itemize}
\begin{itemize}
\item {Proveniência:(De \textunderscore escarçar\textunderscore )}
\end{itemize}
Doença no casco do cavallo, produzida pela introducção de qualquer corpo estranho.
\section{Escarcalhar}
\begin{itemize}
\item {Grp. gram.:v. t.}
\end{itemize}
\begin{itemize}
\item {Utilização:Prov.}
\end{itemize}
\begin{itemize}
\item {Utilização:beir.}
\end{itemize}
\begin{itemize}
\item {Grp. gram.:V. p.}
\end{itemize}
\begin{itemize}
\item {Utilização:Prov.}
\end{itemize}
Abrir muito, abrir desgraciosamente: \textunderscore o vento escarcalhou as rosas\textunderscore .
Gretar, desaggregar-se (a terra), com o calor do sol.
(Cp. fr. \textunderscore écarquiller\textunderscore )
\section{Escarção}
\begin{itemize}
\item {Grp. gram.:m.}
\end{itemize}
\begin{itemize}
\item {Utilização:Constr.}
\end{itemize}
Arco, por cima da padieira, para que esta não supporte o pêso da construcção superior.
\section{Escarçar}
\begin{itemize}
\item {Grp. gram.:v. t.}
\end{itemize}
\begin{itemize}
\item {Grp. gram.:V. p.}
\end{itemize}
Tirar das colmeias (a cera).
Esgarçar.
Soffrer escarça.
\section{Escarcavelar}
\begin{itemize}
\item {Grp. gram.:v. t.}
\end{itemize}
\begin{itemize}
\item {Utilização:Pop.}
\end{itemize}
Abrir, desconjuntar.
\section{Escarceado}
\begin{itemize}
\item {Grp. gram.:adj.}
\end{itemize}
Que tem fórma de escarcéu; que faz lembrar a agitação do mar:«\textunderscore ...das escarceadas vagas populares.\textunderscore »Camillo, \textunderscore Perfil do Marquês\textunderscore , 286.
\section{Escarcejo}
\begin{itemize}
\item {Grp. gram.:m.}
\end{itemize}
\begin{itemize}
\item {Utilização:Mús.}
\end{itemize}
\begin{itemize}
\item {Utilização:Ant.}
\end{itemize}
Notas de passagem, que procediam por meios tons.
\section{Escarcela}
\begin{itemize}
\item {Grp. gram.:f.}
\end{itemize}
Bolsa de coiro, que se usava á cintura.
Parte da armadura, desde a cintura ao joelho.
(Cast. \textunderscore escarcela\textunderscore )
\section{Escarcéo}
\begin{itemize}
\item {Grp. gram.:m.}
\end{itemize}
\begin{itemize}
\item {Utilização:Fig.}
\end{itemize}
Encapelladura das ondas.
Grande vaga em mar tempestuoso.
Acto de exaggerar ou dar importância a coisas simples ou ninharias.
Ralho.
Tormenta doméstica.
\section{Escarcéu}
\begin{itemize}
\item {Grp. gram.:m.}
\end{itemize}
\begin{itemize}
\item {Utilização:Fig.}
\end{itemize}
Encapelladura das ondas.
Grande vaga em mar tempestuoso.
Acto de exaggerar ou dar importância a coisas simples ou ninharias.
Ralho.
Tormenta doméstica.
\section{Escarcha}
\begin{itemize}
\item {Grp. gram.:f.}
\end{itemize}
\begin{itemize}
\item {Utilização:Pop.}
\end{itemize}
Acto ou effeito de escarchar.
\section{Escarchar}
\begin{itemize}
\item {Grp. gram.:v. t.}
\end{itemize}
\begin{itemize}
\item {Proveniência:(Do lat. hyp. \textunderscore exquartiare\textunderscore , dividir em quatro)}
\end{itemize}
Cobrir com flocos de neve.
Adoçar excessivamente (aguardente de anis) com açúcar, que, não podendo sêr dissolvido, crystalliza dentro da garrafa.
Tornar áspero, encrespar.
\section{Escarcina}
\begin{itemize}
\item {Grp. gram.:f.}
\end{itemize}
Alfange dos Persas.
\section{Escarço}
\begin{itemize}
\item {Grp. gram.:m.}
\end{itemize}
Acto de escarçar.
\section{Escardado}
\begin{itemize}
\item {Grp. gram.:adj.}
\end{itemize}
\begin{itemize}
\item {Proveniência:(De \textunderscore cardar\textunderscore )}
\end{itemize}
Diz-se dos chavelhos de boi, quando se desfiam, batendo de encontro a objectos resistentes.
\section{Escardar}
\begin{itemize}
\item {Grp. gram.:v. t.}
\end{itemize}
\begin{itemize}
\item {Utilização:Prov.}
\end{itemize}
Eliminar das (fôlhas de cardo hortense) a lâmina ou palma, para se utilizar em culinária a nervura principal. (Colhido em Turquel)
(Cp. \textunderscore escardear\textunderscore ^1)
\section{Escardear}
\begin{itemize}
\item {Grp. gram.:v. t.}
\end{itemize}
\begin{itemize}
\item {Utilização:Ext.}
\end{itemize}
\begin{itemize}
\item {Proveniência:(De \textunderscore cardo\textunderscore )}
\end{itemize}
Limpar de cardos.
Varrer ou cortar urzes e ervas damninhas em (sementeiras).
Limpar.
\section{Escardear}
\begin{itemize}
\item {Grp. gram.:v. i.}
\end{itemize}
\begin{itemize}
\item {Utilização:Ven.}
\end{itemize}
Diz-se do tiro, que, explodindo com muita fôrça, espalha o chumbo em vez de se concentrar no alvo.
\section{Escardear}
\begin{itemize}
\item {Grp. gram.:v. i.}
\end{itemize}
(?):«\textunderscore ...e tanto que (a nau) escardeava de hir com pressa em fim da roda, se enchia lógo de agoa\textunderscore ». \textunderscore Hist. Trág. Marit.\textunderscore , 50.
\section{Escardecer}
\begin{itemize}
\item {Grp. gram.:v. i.}
\end{itemize}
O mesmo que \textunderscore escadelecer\textunderscore .
\section{Escardecido}
\begin{itemize}
\item {Grp. gram.:adj.}
\end{itemize}
\begin{itemize}
\item {Proveniência:(De \textunderscore escardecer\textunderscore )}
\end{itemize}
Que escardeceu.
\section{Escardichar}
\begin{itemize}
\item {Grp. gram.:v. t.}
\end{itemize}
\begin{itemize}
\item {Utilização:Prov.}
\end{itemize}
\begin{itemize}
\item {Proveniência:(Do rad. de \textunderscore cardar\textunderscore )}
\end{itemize}
Remexer; catar.
\section{Escardilhar}
\begin{itemize}
\item {Grp. gram.:v. t.}
\end{itemize}
Limpar com o escardilho.
\section{Escardilho}
\begin{itemize}
\item {Grp. gram.:m.}
\end{itemize}
Instrumento para escardear^1.
(Cast. \textunderscore escardillo\textunderscore )
\section{Escarduçada}
\begin{itemize}
\item {Grp. gram.:f.}
\end{itemize}
\begin{itemize}
\item {Utilização:Pop.}
\end{itemize}
Saraivada.
\section{Escarduçador}
\begin{itemize}
\item {Grp. gram.:m.  e  adj.}
\end{itemize}
O que escarduça.
\section{Escarduçar}
\begin{itemize}
\item {Grp. gram.:v. t.}
\end{itemize}
Cardar com a carduça.
\section{Escariador}
\begin{itemize}
\item {Grp. gram.:m.}
\end{itemize}
\begin{itemize}
\item {Proveniência:(De \textunderscore escariar\textunderscore )}
\end{itemize}
Chave, com que se apertam ou alargam parafusos.
\section{Escariar}
\begin{itemize}
\item {Grp. gram.:v. t.}
\end{itemize}
\begin{itemize}
\item {Proveniência:(De \textunderscore cariar\textunderscore ? Ou relaciona-se com \textunderscore escarificar\textunderscore ?)}
\end{itemize}
Cravar (parafusos), até que as cabeças fiquem ao nível da peça em que se cravam.
Fazer escavação cónica em (madeira, pedra ou metal).
\section{Escarificação}
\begin{itemize}
\item {Grp. gram.:f.}
\end{itemize}
Acto de escarificar.
\section{Escarificador}
\begin{itemize}
\item {Grp. gram.:m.}
\end{itemize}
Instrumento para escarificar^1.
Apparelho agrícola, montado em rodas, e que corta o terreno verticalmente, sem o levantar.
\section{Escarificar}
\begin{itemize}
\item {Grp. gram.:v. t.}
\end{itemize}
\begin{itemize}
\item {Proveniência:(Lat. \textunderscore scarificare\textunderscore )}
\end{itemize}
Sarjar, golpear, para produzir escoamento de humores.
\section{Escarificar}
\begin{itemize}
\item {Grp. gram.:v. t.}
\end{itemize}
\begin{itemize}
\item {Proveniência:(De \textunderscore escara\textunderscore  + lat. \textunderscore facere\textunderscore )}
\end{itemize}
Produzir escaras em.
\section{Escarioso}
\begin{itemize}
\item {Grp. gram.:adj.}
\end{itemize}
\begin{itemize}
\item {Utilização:Bot.}
\end{itemize}
\begin{itemize}
\item {Proveniência:(T. mal derivado de \textunderscore escara\textunderscore )}
\end{itemize}
Diz-se do cálice, que tem escamas membranosas na margem, como succede na perpétua.
Que tem escaras ou escamas.
\section{Escarite}
\begin{itemize}
\item {Grp. gram.:m.}
\end{itemize}
\begin{itemize}
\item {Proveniência:(De \textunderscore escaro\textunderscore ^1)}
\end{itemize}
Gênero de insectos coleópteros pentâmeros.
\section{Escarlata}
\begin{itemize}
\item {Grp. gram.:f.}
\end{itemize}
(V. \textunderscore escarlate\textunderscore , m.)
\section{Escarlate}
\begin{itemize}
\item {Grp. gram.:m.}
\end{itemize}
\begin{itemize}
\item {Grp. gram.:Adj.}
\end{itemize}
Côr vermelha, muito viva.
Tecido de seda ou lan daquella côr.
Tinta vermelha, como o escarlate da Holanda ou o escarlate de Veneza.
Que tem côr vermelha muito viva e brilhante: \textunderscore um pano escarlate\textunderscore .
(Do \textunderscore persa\textunderscore )
\section{Escarlatim}
\begin{itemize}
\item {Grp. gram.:m.}
\end{itemize}
\begin{itemize}
\item {Proveniência:(De \textunderscore escarlate\textunderscore )}
\end{itemize}
Tecido vermelho, menos fino que o escarlate.
\section{Escarlatina}
\begin{itemize}
\item {Grp. gram.:f.}
\end{itemize}
\begin{itemize}
\item {Proveniência:(De \textunderscore escarlate\textunderscore )}
\end{itemize}
A côr de escarlate.
Febre, caracterizada por manchas vermelhas e irregulares, diffundidas pelo corpo.--Também se toma adjectivamente: \textunderscore febre escarlatina\textunderscore .
\section{Escarlatiniforme}
\begin{itemize}
\item {Grp. gram.:adj.}
\end{itemize}
\begin{itemize}
\item {Utilização:Med.}
\end{itemize}
Que tem fórma de escarlatina.
\section{Escarlatino}
\begin{itemize}
\item {Grp. gram.:adj.}
\end{itemize}
Que tem côr escarlate. Cf. Filinto, X, XIX.
\section{Escarlatinoso}
\begin{itemize}
\item {Grp. gram.:adj.}
\end{itemize}
Doente de escarlatina. Cf. Sousa Martins, \textunderscore Rev. Occid.\textunderscore , 323.
\section{Escarmenta}
\begin{itemize}
\item {Grp. gram.:f.}
\end{itemize}
O mesmo que \textunderscore escarmentação\textunderscore .
\section{Escarmentação}
\begin{itemize}
\item {Grp. gram.:f.}
\end{itemize}
(V.escarmento)
\section{Escarmentar}
\begin{itemize}
\item {Grp. gram.:v. t.}
\end{itemize}
\begin{itemize}
\item {Grp. gram.:V. i.  e  p.}
\end{itemize}
Produzir escarmento em.
Tornar experiente.
Reprehender, castigar.
Têr experiência.
Receber prevenção.
Sêr castigado.
Arrepender-se.
\section{Escarmento}
\begin{itemize}
\item {Grp. gram.:m.}
\end{itemize}
Acto ou effeito de escarmentar.
(Cast. \textunderscore escarmiento\textunderscore )
\section{Escarna}
\begin{itemize}
\item {Grp. gram.:f.}
\end{itemize}
Acto de escarnar.
\section{Escarnação}
\begin{itemize}
\item {Grp. gram.:f.}
\end{itemize}
Acto de escarnar.
\section{Escarnador}
\begin{itemize}
\item {Grp. gram.:m.}
\end{itemize}
Instrumento para escarnar^1.
\section{Escarnar}
\begin{itemize}
\item {Grp. gram.:v. t.}
\end{itemize}
\begin{itemize}
\item {Grp. gram.:V. i.}
\end{itemize}
O mesmo que \textunderscore descarnar\textunderscore .
Diz-se da maré, que descobre mais ou menos parcéis e margens.
\section{Escarnar}
\begin{itemize}
\item {Grp. gram.:v. t.}
\end{itemize}
\begin{itemize}
\item {Utilização:Bras. do Ceará}
\end{itemize}
Preparar (armas), para fazer uso dellas.
Desembainhar.
Engatilhar.
\section{Escarne}
\begin{itemize}
\item {Grp. gram.:m.}
\end{itemize}
\begin{itemize}
\item {Utilização:Pop.}
\end{itemize}
O mesmo que \textunderscore escárneo\textunderscore .
\section{Escarnear}
\begin{itemize}
\item {Grp. gram.:v. t.  e  i.}
\end{itemize}
O mesmo que \textunderscore escarnecer\textunderscore :«\textunderscore ei-lo apupado, zombado, assobiado, chasqueado, escarneado...\textunderscore »Filinto, XII, 131. Cf. \textunderscore Idem\textunderscore , XVIII, 168.
\section{Escarnecedor}
\begin{itemize}
\item {Grp. gram.:m.  e  adj.}
\end{itemize}
O que escarnece.
\section{Escarnecer}
\begin{itemize}
\item {Grp. gram.:v. t.}
\end{itemize}
\begin{itemize}
\item {Grp. gram.:V. i.}
\end{itemize}
Fazer escárneo de.
Troçar; ludibriar.
Fazer escárneo; motejar.
\section{Escarnecimento}
\begin{itemize}
\item {Grp. gram.:m.}
\end{itemize}
(V.escárneo)
\section{Escarnecível}
\begin{itemize}
\item {Grp. gram.:adj.}
\end{itemize}
\begin{itemize}
\item {Proveniência:(De \textunderscore escarnecer\textunderscore )}
\end{itemize}
Que é merecedor de escárneo.
\section{Escarnefuchão}
\begin{itemize}
\item {Grp. gram.:m.}
\end{itemize}
\begin{itemize}
\item {Utilização:Ant.}
\end{itemize}
\begin{itemize}
\item {Proveniência:(De \textunderscore escarnefuchar\textunderscore )}
\end{itemize}
Indivíduo, que é objecto de escárneo. Cf. G. Vicente, \textunderscore Inês Pereira\textunderscore .
\section{Escarnefuchar}
\begin{itemize}
\item {Grp. gram.:v. i.}
\end{itemize}
\begin{itemize}
\item {Utilização:Ant.}
\end{itemize}
\begin{itemize}
\item {Proveniência:(De \textunderscore escárneo\textunderscore )}
\end{itemize}
Escarnecer, mofar:«\textunderscore escarnefucham de vós\textunderscore ». G. Vicente, III, 134.
\section{Escárneo}
\begin{itemize}
\item {Grp. gram.:m.}
\end{itemize}
\begin{itemize}
\item {Proveniência:(Do ant. al. \textunderscore skernon\textunderscore ?)}
\end{itemize}
Zombaria.
Mofa; motejo.
Menosprêzo.
Galhofa.
\section{Escarnhida}
\begin{itemize}
\item {Grp. gram.:f.}
\end{itemize}
\begin{itemize}
\item {Utilização:Gír.}
\end{itemize}
Excremento.
\section{Escarnicação}
\begin{itemize}
\item {Grp. gram.:f.}
\end{itemize}
Acto de escarnicar.
\section{Escarnicadeira}
\textunderscore fem.\textunderscore  de \textunderscore escarnicadeiro\textunderscore .
\section{Escarnicadeiro}
\begin{itemize}
\item {Grp. gram.:m.}
\end{itemize}
O mesmo que \textunderscore escarnicador\textunderscore .
\section{Escarnicador}
\begin{itemize}
\item {Grp. gram.:m.  e  adj.}
\end{itemize}
O que escarnica.
\section{Escarnicar}
\begin{itemize}
\item {Grp. gram.:v. i.}
\end{itemize}
\begin{itemize}
\item {Proveniência:(De \textunderscore escárneo\textunderscore )}
\end{itemize}
Têr o hábito de escarnecer; troçar.
\section{Escarnido}
\begin{itemize}
\item {Grp. gram.:adj.}
\end{itemize}
\begin{itemize}
\item {Proveniência:(De \textunderscore escarnir\textunderscore )}
\end{itemize}
Que é objecto de escárneo.
\section{Escarnificação}
\begin{itemize}
\item {Grp. gram.:f.}
\end{itemize}
Acto de escarnificar.
\section{Escarnificar}
\begin{itemize}
\item {Grp. gram.:v. t.}
\end{itemize}
\begin{itemize}
\item {Proveniência:(Lat. \textunderscore excarnificare\textunderscore )}
\end{itemize}
Martyrizar, lacerando as carnes a.
\section{Escarninho}
\begin{itemize}
\item {Grp. gram.:m.}
\end{itemize}
\begin{itemize}
\item {Grp. gram.:Adj.}
\end{itemize}
(Dem. de \textunderscore escárneo\textunderscore )
Que escarnece; em que há escárneo.
\section{Escarnir}
\begin{itemize}
\item {Grp. gram.:v. i.}
\end{itemize}
\begin{itemize}
\item {Utilização:Pop.}
\end{itemize}
O mesmo que \textunderscore escarnecer\textunderscore .
\section{Escaro}
\begin{itemize}
\item {Grp. gram.:m.}
\end{itemize}
\begin{itemize}
\item {Proveniência:(Gr. \textunderscore skaros\textunderscore )}
\end{itemize}
Peixe acanthopterýgio.
\section{Escaro}
\begin{itemize}
\item {Grp. gram.:adj.}
\end{itemize}
\begin{itemize}
\item {Utilização:T. do Fundão}
\end{itemize}
Que tem mau gênio.
Intratável, descortês.
\section{Escarola}
\begin{itemize}
\item {Grp. gram.:f.}
\end{itemize}
Chicória alporcada, que se emprega em salada, (\textunderscore lactuca scariola\textunderscore ).
\section{Escarolado}
\begin{itemize}
\item {Grp. gram.:adj.}
\end{itemize}
\begin{itemize}
\item {Utilização:Pop.}
\end{itemize}
\begin{itemize}
\item {Proveniência:(De \textunderscore escarolar\textunderscore )}
\end{itemize}
Desavergonhado.
Malicioso.
Descarado.
\section{Escarolador}
\begin{itemize}
\item {Grp. gram.:m.}
\end{itemize}
\begin{itemize}
\item {Proveniência:(De \textunderscore escarolar\textunderscore )}
\end{itemize}
Apparelho agrícola, geralmente movido a braços, para a debulha do milho.
\section{Escarolar}
\begin{itemize}
\item {Grp. gram.:v. t.}
\end{itemize}
\begin{itemize}
\item {Utilização:Pop.}
\end{itemize}
\begin{itemize}
\item {Grp. gram.:V. p.}
\end{itemize}
\begin{itemize}
\item {Utilização:Pop.}
\end{itemize}
\begin{itemize}
\item {Proveniência:(De \textunderscore carolo\textunderscore )}
\end{itemize}
Esbagoar.
Limpar do grão (o carolo).
Encalvecer.
Esburgar.
Tornar apurado, catita.
Desbarretar-se; tirar o chapéu da cabeça.
\section{Escarótico}
\begin{itemize}
\item {Grp. gram.:adj.}
\end{itemize}
\begin{itemize}
\item {Grp. gram.:M.}
\end{itemize}
\begin{itemize}
\item {Proveniência:(Gr. \textunderscore eskharotikos\textunderscore )}
\end{itemize}
Que produz escaras.
Medicamento, que determina a formação de escaras.
\section{Escarpa}
\begin{itemize}
\item {Grp. gram.:f.}
\end{itemize}
\begin{itemize}
\item {Proveniência:(It. \textunderscore scarpa\textunderscore )}
\end{itemize}
Declive ou talude de um fôsso, junto a muralha.
Encosta ingreme, alcantilada.
Córte oblíquo.
\section{Escarpadura}
\begin{itemize}
\item {Grp. gram.:f.}
\end{itemize}
Acto ou effeito de escarpar.
\section{Escarpamento}
\begin{itemize}
\item {Grp. gram.:m.}
\end{itemize}
O mesmo que \textunderscore escarpadura\textunderscore .
\section{Escarpar}
\begin{itemize}
\item {Grp. gram.:v. t.}
\end{itemize}
Cortar (terrenos) em fórma de escarpa, quasi a prumo, obliquamente.
\section{Escarpelada}
\begin{itemize}
\item {Grp. gram.:f.}
\end{itemize}
(V.desfolhada)
\section{Escarpelar}
\begin{itemize}
\item {Grp. gram.:v. t.}
\end{itemize}
Tirar a carpela a, desfolhar, descamisar.
Rasgar com as unhas, arrepelar.
\section{Escarpes}
\begin{itemize}
\item {Grp. gram.:m. pl.}
\end{itemize}
\begin{itemize}
\item {Proveniência:(Do it. \textunderscore scarpa\textunderscore )}
\end{itemize}
Calçado de ferro, com que se torturavam os accusados, em antigos tribunaes.
\section{Escarpiada}
\begin{itemize}
\item {Grp. gram.:f.}
\end{itemize}
\begin{itemize}
\item {Utilização:Des.}
\end{itemize}
\begin{itemize}
\item {Proveniência:(De \textunderscore escarpiar\textunderscore )}
\end{itemize}
Pão comprido de rala, com regos ao meio.
\section{Escarpiar}
\textunderscore v. t.\textunderscore  (e der.)
O mesmo que \textunderscore carmiar\textunderscore , etc.
(Cp. \textunderscore carpiar\textunderscore )
\section{Escarpilhar}
\begin{itemize}
\item {Grp. gram.:v. t.}
\end{itemize}
\begin{itemize}
\item {Utilização:Prov.}
\end{itemize}
\begin{itemize}
\item {Utilização:beir.}
\end{itemize}
O mesmo que \textunderscore carmiar\textunderscore .
(Cp. \textunderscore escarpelar\textunderscore )
\section{Escarpim}
\begin{itemize}
\item {Grp. gram.:m.}
\end{itemize}
\begin{itemize}
\item {Utilização:Des.}
\end{itemize}
\begin{itemize}
\item {Proveniência:(Do it. \textunderscore scarpino\textunderscore . Cp. \textunderscore escarpes\textunderscore )}
\end{itemize}
Pé de meia, que se calçava por baixo das meias.
Espécie de chinela.
Sapato, que deixava o calcanhar descoberto.
\section{Escarpina}
\begin{itemize}
\item {Grp. gram.:f.}
\end{itemize}
Antiga peça de artílharia, semelhante ao arcabuz.
\section{Escarquejar}
\begin{itemize}
\item {Grp. gram.:v. t.}
\end{itemize}
\begin{itemize}
\item {Utilização:Prov.}
\end{itemize}
\begin{itemize}
\item {Utilização:beir.}
\end{itemize}
Lavar ou limpar, esfregando com carqueja.
\section{Escarrachar}
\begin{itemize}
\item {Grp. gram.:v. t.}
\end{itemize}
\begin{itemize}
\item {Utilização:Pop.}
\end{itemize}
O mesmo que \textunderscore escarranchar\textunderscore .
\section{Escarradeira}
\begin{itemize}
\item {Grp. gram.:f.}
\end{itemize}
\begin{itemize}
\item {Proveniência:(De \textunderscore escarrar\textunderscore )}
\end{itemize}
Vaso, em que se escarra.
Escarrador; cuspideira.
\section{Escarrado}
\begin{itemize}
\item {Grp. gram.:adj.}
\end{itemize}
\begin{itemize}
\item {Utilização:Pop.}
\end{itemize}
\begin{itemize}
\item {Proveniência:(De \textunderscore escarrar\textunderscore )}
\end{itemize}
Reproduzido ao natural, bem pintado, muito semelhante: \textunderscore aquelle pequeno é o pai escarrado\textunderscore .
\section{Escarrador}
\begin{itemize}
\item {Grp. gram.:m.}
\end{itemize}
\begin{itemize}
\item {Proveniência:(De \textunderscore escarrar\textunderscore )}
\end{itemize}
Aquelle que escarra muito.
Vaso, em que se escarra.
Cuspideira.
\section{Escarradura}
\begin{itemize}
\item {Grp. gram.:f.}
\end{itemize}
Acto ou effeito de escarrar.
Escarro.
\section{Escarramões}
\begin{itemize}
\item {Grp. gram.:m. pl.}
\end{itemize}
\begin{itemize}
\item {Utilização:Des.}
\end{itemize}
Guisado especial de carneiro.
\section{Escarrancha}
\begin{itemize}
\item {Grp. gram.:f.}
\end{itemize}
\begin{itemize}
\item {Utilização:Prov.}
\end{itemize}
\begin{itemize}
\item {Utilização:trasm.}
\end{itemize}
\begin{itemize}
\item {Proveniência:(De \textunderscore escarranchar\textunderscore )}
\end{itemize}
Risca, que divide o cabello da cabeça.
\section{Escarranchar}
\begin{itemize}
\item {Grp. gram.:v. t.}
\end{itemize}
\begin{itemize}
\item {Grp. gram.:V. p.}
\end{itemize}
Fazer assentar ou pôr a cavallo (alguém), abrindo-lhe muito as pernas.
Escanchar.
Alargar muito (as pernas).
Abrir muito as pernas.
(Cp. \textunderscore escanchar\textunderscore )
\section{Escarrapachar}
\begin{itemize}
\item {Grp. gram.:v. t.}
\end{itemize}
\begin{itemize}
\item {Utilização:Pop.}
\end{itemize}
\begin{itemize}
\item {Grp. gram.:V. p.}
\end{itemize}
O mesmo que \textunderscore escarranchar\textunderscore .
Pespegar.
Estatelar-se.
(Talvez por \textunderscore escarrapatar\textunderscore , de \textunderscore carrapato\textunderscore )
\section{Escarrapichar}
\begin{itemize}
\item {Grp. gram.:v. t.}
\end{itemize}
\begin{itemize}
\item {Utilização:Pop.}
\end{itemize}
\begin{itemize}
\item {Proveniência:(De \textunderscore carrapicho\textunderscore )}
\end{itemize}
Desenredar, penteando.
\section{Escarrapichar-se}
\begin{itemize}
\item {Grp. gram.:v. p.}
\end{itemize}
\begin{itemize}
\item {Utilização:T. de Turquel}
\end{itemize}
Articular as palavras com êmphase ou excessiva meticulosidade.
\section{Escarrar}
\begin{itemize}
\item {Grp. gram.:v. t.}
\end{itemize}
\begin{itemize}
\item {Utilização:chul.}
\end{itemize}
\begin{itemize}
\item {Utilização:Fig.}
\end{itemize}
\begin{itemize}
\item {Grp. gram.:V. i.}
\end{itemize}
\begin{itemize}
\item {Proveniência:(Lat. \textunderscore ex-screare\textunderscore )}
\end{itemize}
Expellir da bôca ou da garganta.
Dizer, fazer ou apresentar, com esfôrço, contrafeito: \textunderscore escarrou quanto devia\textunderscore .
Expellir escarro.
\section{Escarro}
\begin{itemize}
\item {Grp. gram.:m.}
\end{itemize}
\begin{itemize}
\item {Utilização:Pop.}
\end{itemize}
\begin{itemize}
\item {Proveniência:(De \textunderscore escarrar\textunderscore )}
\end{itemize}
Matéria, mais ou menos viscosa e purulenta, que se expelle da bôca, depois dos esforços da expectoração.
Acto de escarrar.
Coisa ou pessôa desprezível.
\section{Escarumar}
\begin{itemize}
\item {Grp. gram.:v. t.}
\end{itemize}
\begin{itemize}
\item {Grp. gram.:V. i.}
\end{itemize}
\begin{itemize}
\item {Utilização:Prov.}
\end{itemize}
Tirar a caruma a (ramos de pinheiro).
Diz-se da videira, quando está largando a flôr.
\section{Escarumba}
\begin{itemize}
\item {Grp. gram.:f.}
\end{itemize}
\begin{itemize}
\item {Utilização:Gír.}
\end{itemize}
Homem de raça negra.
(Or. afr.?)
\section{Escarva}
\begin{itemize}
\item {Grp. gram.:f.}
\end{itemize}
\begin{itemize}
\item {Proveniência:(De \textunderscore escarvar\textunderscore )}
\end{itemize}
Encaixe, em que um pau ou qualquer peça de madeira se une ou se emenda com outra peça.
\section{Escarvador}
\begin{itemize}
\item {Grp. gram.:adj.}
\end{itemize}
\begin{itemize}
\item {Grp. gram.:M.}
\end{itemize}
Que escarva.
Instrumento para escarvar.
\section{Escarvalhado}
\begin{itemize}
\item {Grp. gram.:adj.}
\end{itemize}
Que tem escarvalhos.
\section{Escarvalho}
\begin{itemize}
\item {Grp. gram.:m.}
\end{itemize}
\begin{itemize}
\item {Proveniência:(Do rad. de \textunderscore escarvar\textunderscore )}
\end{itemize}
Falha ou cavidade, na parte interior de um canhão.
\section{Escarvar}
\begin{itemize}
\item {Grp. gram.:v. t.}
\end{itemize}
\begin{itemize}
\item {Proveniência:(Do lat. hyp. \textunderscore scarifare\textunderscore )}
\end{itemize}
Escavar superficialmente; carcomer.
\section{Escarvoar}
\begin{itemize}
\item {Grp. gram.:v. t.}
\end{itemize}
Esboçar ou desenhar a carvão.
\section{Escascar}
\begin{itemize}
\item {Grp. gram.:v. t.}
\end{itemize}
(V.descascar)
\section{Escasquear}
\begin{itemize}
\item {Grp. gram.:v. t.}
\end{itemize}
\begin{itemize}
\item {Utilização:Pop.}
\end{itemize}
\begin{itemize}
\item {Proveniência:(De \textunderscore casco\textunderscore )}
\end{itemize}
Lavar ou limpar o casco ou cabeça de.
Limpar.
Aperaltar; escarolar.
\section{Escassamente}
\begin{itemize}
\item {Grp. gram.:adv.}
\end{itemize}
\begin{itemize}
\item {Proveniência:(De \textunderscore escasso\textunderscore )}
\end{itemize}
Com escassez; pouco.
\section{Escassear}
\begin{itemize}
\item {Grp. gram.:v. t.}
\end{itemize}
\begin{itemize}
\item {Grp. gram.:V. i.}
\end{itemize}
Tornar escasso, deminuto, parco, apoucado.
Ir deminuindo.
Sêr em pequena quantidade; faltar: \textunderscore escasseia-lhe o talento\textunderscore .
\section{Escassez}
\begin{itemize}
\item {Grp. gram.:f.}
\end{itemize}
Qualidade daquillo que é escasso.
\section{Escasseza}
\begin{itemize}
\item {Grp. gram.:f.}
\end{itemize}
O mesmo que \textunderscore escassez\textunderscore . Cf. Sousa, \textunderscore Vida do Arceb.\textunderscore , I, 92; III, 110.
\section{Escassilho}
\begin{itemize}
\item {Grp. gram.:m.}
\end{itemize}
\begin{itemize}
\item {Proveniência:(De \textunderscore escasso\textunderscore )}
\end{itemize}
Pedacinho de coisa partida.
\section{Escasso}
\begin{itemize}
\item {Grp. gram.:adj.}
\end{itemize}
\begin{itemize}
\item {Grp. gram.:M.}
\end{itemize}
\begin{itemize}
\item {Proveniência:(Do b. lat. \textunderscore scarpsus\textunderscore )}
\end{itemize}
De que há pequena quantidade.
Que não é abundante.
Pouco; raro.
Avaro, sovina.
Avarento.
\section{Escasular}
\begin{itemize}
\item {Grp. gram.:v. t.}
\end{itemize}
\begin{itemize}
\item {Utilização:Prov.}
\end{itemize}
\begin{itemize}
\item {Utilização:trasm.}
\end{itemize}
\begin{itemize}
\item {Proveniência:(De \textunderscore casulo\textunderscore )}
\end{itemize}
O mesmo que \textunderscore esfolhar\textunderscore .
\section{Escatel}
\begin{itemize}
\item {Grp. gram.:m.}
\end{itemize}
\begin{itemize}
\item {Utilização:Náut.}
\end{itemize}
Abertura, no extremo de uma cavilha de navio, para meter a chaveta.
\section{Escatelar}
\begin{itemize}
\item {Grp. gram.:v. t.}
\end{itemize}
\begin{itemize}
\item {Proveniência:(De \textunderscore escatel\textunderscore )}
\end{itemize}
Fechar com chaveta (a cavilha).
Formar abertura em (bocas de fogo), para dar lugar a culatra.
\section{Escatima}
\begin{itemize}
\item {Grp. gram.:f.}
\end{itemize}
\begin{itemize}
\item {Utilização:Ant.}
\end{itemize}
Escassez.
Defeito.
Fraude.
Paixão.
(Cast. \textunderscore escatima\textunderscore )
\section{Escatimar}
\begin{itemize}
\item {Grp. gram.:v. t.}
\end{itemize}
\begin{itemize}
\item {Utilização:Ant.}
\end{itemize}
\begin{itemize}
\item {Proveniência:(De \textunderscore escatima\textunderscore )}
\end{itemize}
Dar com má vontade, com escassez.
Regatear.
Apartar.
Enganar.
\section{Escatófago}
\begin{itemize}
\item {Grp. gram.:adj.}
\end{itemize}
\begin{itemize}
\item {Utilização:Zool.}
\end{itemize}
\begin{itemize}
\item {Proveniência:(Do gr. \textunderscore skatos\textunderscore  + \textunderscore phagein\textunderscore )}
\end{itemize}
Que se alimenta de excrementos.
\section{Escatófilo}
\begin{itemize}
\item {Grp. gram.:adj.}
\end{itemize}
\begin{itemize}
\item {Utilização:Zool.}
\end{itemize}
\begin{itemize}
\item {Proveniência:(Do gr. \textunderscore skatos\textunderscore  + \textunderscore philos\textunderscore )}
\end{itemize}
Que cresce ou vive nos excrementos.
\section{Escátola}
\begin{itemize}
\item {Grp. gram.:f.}
\end{itemize}
\begin{itemize}
\item {Utilização:Ant.}
\end{itemize}
(V.escátula)
\section{Escatologia}
\begin{itemize}
\item {Grp. gram.:f.}
\end{itemize}
\begin{itemize}
\item {Proveniência:(Do gr. \textunderscore eskhatos\textunderscore , último, e \textunderscore logos\textunderscore , tratado)}
\end{itemize}
Doutrina das coisas, que deverão acontecer no fim do mundo.
\section{Escatologia}
\begin{itemize}
\item {Grp. gram.:f.}
\end{itemize}
\begin{itemize}
\item {Utilização:P. us.}
\end{itemize}
\begin{itemize}
\item {Proveniência:(Do gr. \textunderscore skatos\textunderscore  + \textunderscore logos\textunderscore )}
\end{itemize}
Tratado, á cêrca dos excrementos.
\section{Escatológico}
\begin{itemize}
\item {Grp. gram.:adj.}
\end{itemize}
Relativo a escatologia.
\section{Escatológico}
\begin{itemize}
\item {Grp. gram.:adj.}
\end{itemize}
Relativo a escatologia.
Relativo a excrementos.
Nauseabundo.
\section{Escatopes}
\begin{itemize}
\item {Grp. gram.:m. pl.}
\end{itemize}
\begin{itemize}
\item {Proveniência:(Do gr. \textunderscore skatos\textunderscore  + \textunderscore ops\textunderscore )}
\end{itemize}
Gênero de insectos dípteros.
\section{Escatóphago}
\begin{itemize}
\item {Grp. gram.:adj.}
\end{itemize}
\begin{itemize}
\item {Utilização:Zool.}
\end{itemize}
\begin{itemize}
\item {Proveniência:(Do gr. \textunderscore skatos\textunderscore  + \textunderscore phagein\textunderscore )}
\end{itemize}
Que se alimenta de excrementos.
\section{Escatóphilo}
\begin{itemize}
\item {Grp. gram.:adj.}
\end{itemize}
\begin{itemize}
\item {Utilização:Zool.}
\end{itemize}
\begin{itemize}
\item {Proveniência:(Do gr. \textunderscore skatos\textunderscore  + \textunderscore philos\textunderscore )}
\end{itemize}
Que cresce ou vive nos excrementos.
\section{Escátula}
\begin{itemize}
\item {Grp. gram.:f.}
\end{itemize}
\begin{itemize}
\item {Utilização:Ant.}
\end{itemize}
Pequena caixa.
Boceta.
A omoplata, segundo Celso.
(B. lat. \textunderscore scatula\textunderscore )
\section{Escaturigem}
\begin{itemize}
\item {Grp. gram.:f.}
\end{itemize}
\begin{itemize}
\item {Utilização:Des.}
\end{itemize}
\begin{itemize}
\item {Proveniência:(Lat. \textunderscore scaturigo\textunderscore )}
\end{itemize}
Origem, nascente, de água.
\section{Escauro}
\begin{itemize}
\item {Grp. gram.:m.}
\end{itemize}
\begin{itemize}
\item {Proveniência:(Do gr. \textunderscore skauros\textunderscore )}
\end{itemize}
Gênero de insectos coleópteros heterómeros.
\section{Escava}
\begin{itemize}
\item {Grp. gram.:f.}
\end{itemize}
\begin{itemize}
\item {Utilização:Ant.}
\end{itemize}
O mesmo que \textunderscore escavação\textunderscore .
Parte do vestuário, correspondente ao sovaco; cava:«\textunderscore no pé da manga, junto da escava.\textunderscore »\textunderscore Alvará de D. Sebast.\textunderscore , in \textunderscore Rev. Lus.\textunderscore , XV, 123.
\section{Escavação}
\begin{itemize}
\item {Grp. gram.:f.}
\end{itemize}
Acto ou effeito de escavar.
\section{Escavacar}
\begin{itemize}
\item {Grp. gram.:v. t.}
\end{itemize}
\begin{itemize}
\item {Utilização:Fig.}
\end{itemize}
\begin{itemize}
\item {Grp. gram.:V. i.}
\end{itemize}
\begin{itemize}
\item {Utilização:Bras. do N}
\end{itemize}
Tirar cavacos a.
Partir em cavacos.
Despedaçar; esbandalhar.
Tornar magro, alquebrado.
Dar cavaco, zangar-se.
\section{Escavaçar}
\begin{itemize}
\item {Grp. gram.:v. t.}
\end{itemize}
\begin{itemize}
\item {Proveniência:(De \textunderscore escavação\textunderscore )}
\end{itemize}
Esterroar.
\section{Escavachar}
\begin{itemize}
\item {Grp. gram.:v. t.}
\end{itemize}
\begin{itemize}
\item {Utilização:Prov.}
\end{itemize}
Cavar, ligeira e superficialmente.
(Cp. \textunderscore escavaçar\textunderscore )
\section{Escavador}
\begin{itemize}
\item {Grp. gram.:adj.}
\end{itemize}
\begin{itemize}
\item {Grp. gram.:M.}
\end{itemize}
\begin{itemize}
\item {Utilização:Ant.}
\end{itemize}
\begin{itemize}
\item {Proveniência:(De \textunderscore escavar\textunderscore )}
\end{itemize}
Que escava.
Aquelle que escava.
Mondadentes, palito para limpar os dentes.
Gênero de crustáceos decápodes.
\section{Escavadora}
\begin{itemize}
\item {Grp. gram.:f.}
\end{itemize}
O mesmo que \textunderscore cavadora\textunderscore .
\section{Escavadura}
\begin{itemize}
\item {Grp. gram.:f.}
\end{itemize}
O mesmo que \textunderscore escavação\textunderscore .
\section{Escavar}
\begin{itemize}
\item {Grp. gram.:v. t.}
\end{itemize}
\begin{itemize}
\item {Utilização:Fig.}
\end{itemize}
\begin{itemize}
\item {Proveniência:(De \textunderscore cavar\textunderscore )}
\end{itemize}
Formar cavidade em.
Cavar em roda.
Tirar a terra de.
Cavar superficialmente.
Tornar côncavo.
Investigar.
\section{Escaveirado}
\begin{itemize}
\item {Grp. gram.:adj.}
\end{itemize}
Muito magro.
\section{Escaveirar}
\begin{itemize}
\item {Grp. gram.:v. t.}
\end{itemize}
Converter em caveira.
Descarnar a cabeça de.
Tornar magro.
\section{Escavinar}
\begin{itemize}
\item {Grp. gram.:v. t.}
\end{itemize}
\begin{itemize}
\item {Utilização:Prov.}
\end{itemize}
\begin{itemize}
\item {Utilização:beir.}
\end{itemize}
O mesmo que \textunderscore esquadrinhar\textunderscore .
(Relaciona-se com o b. lat. \textunderscore scabini\textunderscore , adjuntos de tribunal, encarregados de descobrir os criminosos?)
\section{Eschatologia}
\begin{itemize}
\item {fónica:ca}
\end{itemize}
\begin{itemize}
\item {Grp. gram.:f.}
\end{itemize}
\begin{itemize}
\item {Proveniência:(Do gr. \textunderscore eskhatos\textunderscore , último, e \textunderscore logos\textunderscore , tratado)}
\end{itemize}
Doutrina das coisas, que deverão acontecer no fim do mundo.
\section{Eschatológico}
\begin{itemize}
\item {fónica:ca}
\end{itemize}
\begin{itemize}
\item {Grp. gram.:adj.}
\end{itemize}
Relativo a eschatologia.
\section{Eschema}
\begin{itemize}
\item {fónica:quê}
\end{itemize}
\begin{itemize}
\item {Grp. gram.:m.}
\end{itemize}
\begin{itemize}
\item {Utilização:Rhet.}
\end{itemize}
\begin{itemize}
\item {Utilização:Med.}
\end{itemize}
\begin{itemize}
\item {Proveniência:(Lat. \textunderscore schema\textunderscore )}
\end{itemize}
Designação genérica de todas as fórmas de ornato, no estilo.
Representação da disposição geral de um apparelho orgânico ou da marcha de um phenómeno, abstrahindo-se de certas particularidades, que impedíriam de abranger rapidamente as noções que se pretendem.
Representação das funcções e relações de um objecto, independentemente da sua verdadeira fórma.
Proposta, submetida á deliberação de um concilio.
\section{Eschematicamente}
\begin{itemize}
\item {fónica:que}
\end{itemize}
\begin{itemize}
\item {Grp. gram.:adv.}
\end{itemize}
De modo eschemático; á maneira de eschema.
\section{Eschemático}
\begin{itemize}
\item {fónica:que}
\end{itemize}
\begin{itemize}
\item {Grp. gram.:adj.}
\end{itemize}
Relativo a eschema.
\section{Eschematismo}
\begin{itemize}
\item {fónica:que}
\end{itemize}
\begin{itemize}
\item {Grp. gram.:m.}
\end{itemize}
Systema dos que amiúde formulam eschemas.
\section{Eschyliano}
\begin{itemize}
\item {fónica:qui}
\end{itemize}
\begin{itemize}
\item {Grp. gram.:adj.}
\end{itemize}
Relativo a Éschylo; parecido com o gênio literário de Éschylo.
\section{Escindir}
\begin{itemize}
\item {Grp. gram.:v. t.}
\end{itemize}
\begin{itemize}
\item {Proveniência:(Do lat. \textunderscore scindere\textunderscore )}
\end{itemize}
Cortar.
Separar; desavir.
\section{Escirpo}
\begin{itemize}
\item {Grp. gram.:m.}
\end{itemize}
\begin{itemize}
\item {Proveniência:(Do lat. \textunderscore scirpus\textunderscore )}
\end{itemize}
O mesmo que \textunderscore junco\textunderscore ^1.
\section{Esclaréa}
\begin{itemize}
\item {Grp. gram.:f.}
\end{itemize}
Planta medicinal, da fam. das labiadas, (\textunderscore salvea sclarea\textunderscore ).
\section{Esclareia}
\begin{itemize}
\item {Grp. gram.:f.}
\end{itemize}
Planta medicinal, da fam. das labiadas, (\textunderscore salvea sclarea\textunderscore ).
\section{Esclarecer}
\begin{itemize}
\item {Grp. gram.:v. t.}
\end{itemize}
\begin{itemize}
\item {Utilização:Fig.}
\end{itemize}
\begin{itemize}
\item {Grp. gram.:V. i.}
\end{itemize}
Fazer claro.
Illuminar.
Doutrinar.
Tornar intelligível: \textunderscore esclarecer questões\textunderscore .
Tornar distinto, nobre.
Tornar-se límpido (o tempo).
Amanhecer.
\section{Esclarecido}
\begin{itemize}
\item {Grp. gram.:adj.}
\end{itemize}
\begin{itemize}
\item {Proveniência:(De \textunderscore esclarecer\textunderscore )}
\end{itemize}
Claro, que recebe luz, alumiado.
Elucidado.
Illustre, distinto, famoso: \textunderscore um esclarecido professor\textunderscore .
Ennobrecido.
\section{Esclarecimento}
\begin{itemize}
\item {Grp. gram.:m.}
\end{itemize}
Acto ou effeito de esclarecer.
\section{Esclavagem}
\begin{itemize}
\item {Grp. gram.:f.}
\end{itemize}
\begin{itemize}
\item {Utilização:Ant.}
\end{itemize}
\begin{itemize}
\item {Proveniência:(Fr. \textunderscore esclavage\textunderscore )}
\end{itemize}
Adôrno de pérolas ou jóias para o pescoço. Cf. Viterbo, \textunderscore Elucidário\textunderscore .
\section{Esclavina}
\begin{itemize}
\item {Grp. gram.:f.}
\end{itemize}
Vestuário, que os romeiros usavam sôbre a túnica.
Opa de escravo ou de cativo resgatado.
(Cast. \textunderscore esclavina\textunderscore )
\section{Esclavista}
\begin{itemize}
\item {Grp. gram.:m.}
\end{itemize}
\begin{itemize}
\item {Utilização:Gal}
\end{itemize}
(V.escravista). Cf. Latino, \textunderscore Humboldt\textunderscore , 461.
\section{Esclavões}
\begin{itemize}
\item {Grp. gram.:m. pl.}
\end{itemize}
Habitantes da Esclavónia.
\section{Esclavónico}
\begin{itemize}
\item {Grp. gram.:adj.}
\end{itemize}
Relativo a Esclavónia ou a esclavões.
\section{Escleral}
\begin{itemize}
\item {Grp. gram.:adj.}
\end{itemize}
\begin{itemize}
\item {Proveniência:(Do gr. \textunderscore skleros\textunderscore )}
\end{itemize}
Endurecido.
Fibroso, (falando-se de tecidos orgânicos).
\section{Esclerantho}
\begin{itemize}
\item {Grp. gram.:m.}
\end{itemize}
\begin{itemize}
\item {Proveniência:(Do gr. \textunderscore skleros\textunderscore  + \textunderscore anthos\textunderscore )}
\end{itemize}
Gênero de plantas caryophylláceas.
\section{Escleranto}
\begin{itemize}
\item {Grp. gram.:m.}
\end{itemize}
\begin{itemize}
\item {Proveniência:(Do gr. \textunderscore skleros\textunderscore  + \textunderscore anthos\textunderscore )}
\end{itemize}
Gênero de plantas cariofiláceas.
\section{Esclerema}
\begin{itemize}
\item {Grp. gram.:m.}
\end{itemize}
\begin{itemize}
\item {Utilização:Med.}
\end{itemize}
\begin{itemize}
\item {Proveniência:(Do gr. \textunderscore skleros\textunderscore , duro)}
\end{itemize}
Endurecimento do tecido laminoso dos recém-nascidos.
\section{Esclerênchyma}
\begin{itemize}
\item {fónica:qui}
\end{itemize}
\begin{itemize}
\item {Grp. gram.:m.}
\end{itemize}
\begin{itemize}
\item {Utilização:Bot.}
\end{itemize}
\begin{itemize}
\item {Proveniência:(Do gr. \textunderscore skleros\textunderscore  + \textunderscore enkhuma\textunderscore )}
\end{itemize}
Tecido, cujas céllulas endureceram com a liquidificação das suas paredes.
\section{Esclerênquima}
\begin{itemize}
\item {Grp. gram.:m.}
\end{itemize}
\begin{itemize}
\item {Utilização:Bot.}
\end{itemize}
\begin{itemize}
\item {Proveniência:(Do gr. \textunderscore skleros\textunderscore  + \textunderscore enkhuma\textunderscore )}
\end{itemize}
Tecido, cujas células endureceram com a liquidificação das suas paredes.
\section{Escléria}
\begin{itemize}
\item {Grp. gram.:f.}
\end{itemize}
\begin{itemize}
\item {Proveniência:(Do gr. \textunderscore skleros\textunderscore )}
\end{itemize}
Gênero de plantas cyperáceas.
\section{Esclerocarpo}
\begin{itemize}
\item {Grp. gram.:m.}
\end{itemize}
\begin{itemize}
\item {Proveniência:(Do gr. \textunderscore skleros\textunderscore  + \textunderscore karpos\textunderscore )}
\end{itemize}
Gênero de plantas synanthéreas, originárias da Guiné.
\section{Esclerócio}
\begin{itemize}
\item {Grp. gram.:m.}
\end{itemize}
Gênero de cogumelos.
\section{Escleroderma}
\begin{itemize}
\item {Grp. gram.:f.}
\end{itemize}
\begin{itemize}
\item {Proveniência:(Do gr. \textunderscore skleros\textunderscore  + \textunderscore derma\textunderscore )}
\end{itemize}
Endurecimento anormal dos tecidos orgânicos.
\section{Esclerodermia}
\begin{itemize}
\item {Grp. gram.:f.}
\end{itemize}
O mesmo que \textunderscore escleroderma\textunderscore .
\section{Esclerodermos}
\begin{itemize}
\item {Grp. gram.:m. pl.}
\end{itemize}
\begin{itemize}
\item {Proveniência:(Do gr. \textunderscore skleros\textunderscore  + \textunderscore derma\textunderscore )}
\end{itemize}
Família de peixes, que têm o corpo coberto de placas duras, que se articulam mutuamente.
\section{Esclerodonte}
\begin{itemize}
\item {Grp. gram.:m.}
\end{itemize}
\begin{itemize}
\item {Proveniência:(Do gr. \textunderscore skleros\textunderscore  + \textunderscore odous\textunderscore , \textunderscore odontos\textunderscore )}
\end{itemize}
Gênero de musgos.
\section{Esclerófito}
\begin{itemize}
\item {Grp. gram.:m.}
\end{itemize}
\begin{itemize}
\item {Proveniência:(Do gr. \textunderscore skleros\textunderscore  + \textunderscore phuton\textunderscore )}
\end{itemize}
Gênero de musgos.
\section{Escleroftalmia}
\begin{itemize}
\item {Grp. gram.:f.}
\end{itemize}
\begin{itemize}
\item {Proveniência:(Gr. \textunderscore sklerophthalmia\textunderscore )}
\end{itemize}
Inflamação da conjuntiva, sem augmento de secreção da membrana mucosa.
\section{Escleroma}
\begin{itemize}
\item {Grp. gram.:m.}
\end{itemize}
\begin{itemize}
\item {Proveniência:(Do gr. \textunderscore skleros\textunderscore )}
\end{itemize}
Tumôr duro.
\section{Esclerophtalmia}
\begin{itemize}
\item {Grp. gram.:f.}
\end{itemize}
\begin{itemize}
\item {Proveniência:(Gr. \textunderscore sklerophthalmia\textunderscore )}
\end{itemize}
Inflamação da conjuntiva, sem augmento de secreção da membrana mucosa.
\section{Escleróphyto}
\begin{itemize}
\item {Grp. gram.:m.}
\end{itemize}
\begin{itemize}
\item {Proveniência:(Do gr. \textunderscore skleros\textunderscore  + \textunderscore phuton\textunderscore )}
\end{itemize}
Gênero de musgos.
\section{Escleroptéria}
\begin{itemize}
\item {Grp. gram.:f.}
\end{itemize}
\begin{itemize}
\item {Proveniência:(Do gr. \textunderscore skleros\textunderscore  + \textunderscore pteron\textunderscore )}
\end{itemize}
Gênero de orchídeas.
\section{Esclerose}
\begin{itemize}
\item {Grp. gram.:f.}
\end{itemize}
\begin{itemize}
\item {Utilização:Med.}
\end{itemize}
\begin{itemize}
\item {Proveniência:(Do gr. \textunderscore skleros\textunderscore , duro)}
\end{itemize}
Qualquer endurecimento mórbido dos tecidos.
\section{Esclerótica}
\begin{itemize}
\item {Grp. gram.:f.}
\end{itemize}
\begin{itemize}
\item {Utilização:Anat.}
\end{itemize}
\begin{itemize}
\item {Proveniência:(Do gr. \textunderscore skleros\textunderscore , duro)}
\end{itemize}
Membrana branca e fibrosa, que fórma a maior parte da superfície do globo ocular.
\section{Escleroticotomia}
\begin{itemize}
\item {Grp. gram.:f.}
\end{itemize}
\begin{itemize}
\item {Utilização:Cir.}
\end{itemize}
Incisão da esclerótica.
\section{Esclerotínia}
\begin{itemize}
\item {Grp. gram.:f.}
\end{itemize}
\begin{itemize}
\item {Proveniência:(Do gr. \textunderscore skleros\textunderscore , duro)}
\end{itemize}
Doença dos viveiros das videiras.
\section{Esclusa}
\begin{itemize}
\item {Grp. gram.:f.}
\end{itemize}
Reprêsa em rio ou canal, para facilitar a navegação.
Comporta.
(B. lat. \textunderscore exclusa\textunderscore )
\section{...êsco}
\begin{itemize}
\item {Grp. gram.:suf.}
\end{itemize}
(designativo de qualidade, depreciação ou deminuição)
\section{Escôa}
\begin{itemize}
\item {Grp. gram.:f.}
\end{itemize}
O mesmo que \textunderscore escôas\textunderscore .
\section{Escoadeira}
\begin{itemize}
\item {Grp. gram.:f.}
\end{itemize}
\begin{itemize}
\item {Proveniência:(De \textunderscore escoar\textunderscore )}
\end{itemize}
Cano de manilhas, que da salina conduz a água para o mar.
\section{Escoadoiro}
\begin{itemize}
\item {Grp. gram.:m.}
\end{itemize}
\begin{itemize}
\item {Proveniência:(De \textunderscore escoar\textunderscore )}
\end{itemize}
Lugar ou cano, por onde se escôam águas e outros líquidos ou dejectos.
\section{Escoadouro}
\begin{itemize}
\item {Grp. gram.:m.}
\end{itemize}
\begin{itemize}
\item {Proveniência:(De \textunderscore escoar\textunderscore )}
\end{itemize}
Lugar ou cano, por onde se escôam águas e outros líquidos ou dejectos.
\section{Escoadura}
\begin{itemize}
\item {Grp. gram.:f.}
\end{itemize}
Acto de escoar.
Porção de líquido que se escoou.
\section{Escoalha}
\begin{itemize}
\item {Grp. gram.:f.}
\end{itemize}
\begin{itemize}
\item {Utilização:Fam.}
\end{itemize}
Escorralho, ralé. Cf. Camillo, \textunderscore Myst. de Lisb.\textunderscore , I, 92.
\section{Escoamento}
\begin{itemize}
\item {Grp. gram.:m.}
\end{itemize}
Acto de escoar.
Plano inclinado, por onde se escoam as águas.
\section{Escoante}
\begin{itemize}
\item {Grp. gram.:m.}
\end{itemize}
Declive ou inclinação, por onde se escôa um líquido. Cf. Lapa, \textunderscore Proc. de Vin.\textunderscore , 12 e 56.
\section{Escoar}
\begin{itemize}
\item {Grp. gram.:v. t.}
\end{itemize}
\begin{itemize}
\item {Grp. gram.:V. p.}
\end{itemize}
\begin{itemize}
\item {Proveniência:(De \textunderscore coar\textunderscore )}
\end{itemize}
Coar.
Deixar correr pouco a pouco (líquidos, ou substâncias que derivam como líquidos).
Escorrer, pouco a pouco.
Esvaziar-se.
Decorrer.
Fugir alapardadamente.
Sumir-se.
\section{Escôas}
\begin{itemize}
\item {Grp. gram.:f. pl.}
\end{itemize}
Peças, que fortificam interiormente as cavernas de uma embarcação.
\section{Escobar}
\begin{itemize}
\item {Grp. gram.:m.}
\end{itemize}
Peixe do mar dos Açores. Cf. Flaviense, \textunderscore Diccion. Geogr.\textunderscore 
\section{Escobilhar}
\begin{itemize}
\item {Grp. gram.:v. t.}
\end{itemize}
\begin{itemize}
\item {Utilização:Prov.}
\end{itemize}
\begin{itemize}
\item {Utilização:trasm.}
\end{itemize}
Mexer, revolver (terreno).
\section{Escocar}
\begin{itemize}
\item {Grp. gram.:v. t.}
\end{itemize}
\begin{itemize}
\item {Utilização:Prov.}
\end{itemize}
\begin{itemize}
\item {Utilização:trasm.}
\end{itemize}
\begin{itemize}
\item {Proveniência:(De \textunderscore cóca\textunderscore )}
\end{itemize}
Subtrahir arteiramente.
\section{Escocês}
\begin{itemize}
\item {Grp. gram.:m.}
\end{itemize}
\begin{itemize}
\item {Grp. gram.:Adj.}
\end{itemize}
Aquelle que é natural da Escócia.
Relativo a Escócia.
Diz-se dos tecidos, cuja estampagem é em riscas cruzadas e em côres vivas.
O mesmo que \textunderscore escócio\textunderscore .
\section{Escochado}
\begin{itemize}
\item {Grp. gram.:adj.}
\end{itemize}
\begin{itemize}
\item {Utilização:T. do Fundão}
\end{itemize}
Diz-se do pão mal cozido, cuja crosta superior está despegada do miolo.
(Por \textunderscore escorchado\textunderscore , de \textunderscore escorchar\textunderscore )
\section{Escochar}
\begin{itemize}
\item {Grp. gram.:v. t.}
\end{itemize}
\begin{itemize}
\item {Utilização:Des.}
\end{itemize}
\begin{itemize}
\item {Utilização:Prov.}
\end{itemize}
\begin{itemize}
\item {Utilização:minh.}
\end{itemize}
\begin{itemize}
\item {Utilização:beir.}
\end{itemize}
\begin{itemize}
\item {Utilização:T. da Bairrada}
\end{itemize}
Espatifar; matar.
Tirar a cabeça a (sardinhas), para as frigir ou guisar.
Esburgar, tirar toda a carne adherente a um osso.
(Relaciona-se com \textunderscore escorchar\textunderscore ?)
\section{Escócia}
\begin{itemize}
\item {Grp. gram.:f.}
\end{itemize}
\begin{itemize}
\item {Proveniência:(Lat. \textunderscore scotia\textunderscore )}
\end{itemize}
Moldura côncava, na base de uma columna.
\section{Escócio}
\begin{itemize}
\item {Grp. gram.:adj.}
\end{itemize}
\begin{itemize}
\item {Proveniência:(De \textunderscore Escócia\textunderscore , n. p.)}
\end{itemize}
Diz-se de uma qualidade de ferro ordinário.
\section{Escoçumelar-se}
\begin{itemize}
\item {Grp. gram.:v. p.}
\end{itemize}
\begin{itemize}
\item {Utilização:Prov.}
\end{itemize}
Mexer os ombros, sentindo nelles qualquer comichão.
Roçar-se por alguém ou por alguma coisa.
(Talvez por \textunderscore escocemegar-se\textunderscore , de \textunderscore cocemegas\textunderscore )
\section{Escoda}
\begin{itemize}
\item {Grp. gram.:f.}
\end{itemize}
\begin{itemize}
\item {Proveniência:(De \textunderscore escodar\textunderscore )}
\end{itemize}
Instrumento de canteiro, para alisar e lavrar pedras já desbastadas.
\section{Escodar}
\begin{itemize}
\item {Grp. gram.:v. t.}
\end{itemize}
\begin{itemize}
\item {Utilização:Prov.}
\end{itemize}
\begin{itemize}
\item {Utilização:beir.}
\end{itemize}
\begin{itemize}
\item {Proveniência:(Do lat. \textunderscore excudare\textunderscore ?)}
\end{itemize}
Polir, lavrar (pedra) com escoda.
Alisar o exterior de (pelles), para se poderem tingir.
Desbastar (tábuas), com enxó.
\section{Escodear}
\begin{itemize}
\item {Grp. gram.:v. t.}
\end{itemize}
Tirar a côdea a.
Descascar.
\section{Escófia}
\begin{itemize}
\item {Grp. gram.:f.}
\end{itemize}
\begin{itemize}
\item {Utilização:Ant.}
\end{itemize}
O mesmo que \textunderscore coifa\textunderscore .
\section{Escogiar}
\begin{itemize}
\item {Grp. gram.:v. t.}
\end{itemize}
\begin{itemize}
\item {Utilização:T. de Gondomar}
\end{itemize}
Vigiar, espreitar.
(Cp. \textunderscore escogitar\textunderscore )
\section{Escogita}
\begin{itemize}
\item {Grp. gram.:f.}
\end{itemize}
\begin{itemize}
\item {Utilização:Prov.}
\end{itemize}
\begin{itemize}
\item {Utilização:trasm.}
\end{itemize}
\begin{itemize}
\item {Proveniência:(De \textunderscore escogitar\textunderscore )}
\end{itemize}
Pessôa muito curiosa, que anda sempre á espreita de novidades.
\section{Escogitação}
\begin{itemize}
\item {Grp. gram.:f.}
\end{itemize}
\begin{itemize}
\item {Proveniência:(Lat. \textunderscore excogitatio\textunderscore )}
\end{itemize}
Acto de escogitar.
\section{Escogitador}
\begin{itemize}
\item {Grp. gram.:m.  e  adj.}
\end{itemize}
O que escogita.
\section{Escogitar}
\begin{itemize}
\item {Grp. gram.:v. t.}
\end{itemize}
\begin{itemize}
\item {Utilização:Prov.}
\end{itemize}
\begin{itemize}
\item {Utilização:trasm.}
\end{itemize}
\begin{itemize}
\item {Proveniência:(Lat. \textunderscore excogitare\textunderscore )}
\end{itemize}
Cogitar muito.
Meditar.
Descobrir, cogitando.
Investigar.
Espreitar.
Descobrir (alguma coisa que estava occulta).
\section{Escogitável}
\begin{itemize}
\item {Grp. gram.:adj.}
\end{itemize}
Que se póde escogitar.
\section{Escoiçar}
\begin{itemize}
\item {Grp. gram.:v. t.}
\end{itemize}
\begin{itemize}
\item {Utilização:Prov.}
\end{itemize}
\begin{itemize}
\item {Utilização:alg.}
\end{itemize}
\begin{itemize}
\item {Utilização:Prov.}
\end{itemize}
\begin{itemize}
\item {Utilização:minh.}
\end{itemize}
\begin{itemize}
\item {Proveniência:(De \textunderscore coice\textunderscore )}
\end{itemize}
Procurar com diligência.
Bater com varas (os feixes de linho enriado), para que nelles não fiquem peixes.
Esvaziar (pipas ou tonéis).
O mesmo que \textunderscore escoicear\textunderscore . Cf. Camillo, \textunderscore Cancion. Al.\textunderscore , 421.
\section{Escoiceador}
\begin{itemize}
\item {Grp. gram.:adj.}
\end{itemize}
\begin{itemize}
\item {Grp. gram.:M.}
\end{itemize}
Que escoiceia.
Aquelle que escoiceia.
\section{Escoicear}
\begin{itemize}
\item {Grp. gram.:v. t.}
\end{itemize}
\begin{itemize}
\item {Utilização:Fig.}
\end{itemize}
\begin{itemize}
\item {Grp. gram.:V. i.}
\end{itemize}
Dar coices em.
Insultar, tratar brutalmente.
Dar coices.
\section{Escoicinhador}
\begin{itemize}
\item {Grp. gram.:adj.}
\end{itemize}
\begin{itemize}
\item {Grp. gram.:M.}
\end{itemize}
Que escoicinha.
Aquelle que escoicinha.
\section{Escoicinhar}
\begin{itemize}
\item {Grp. gram.:v. t.  e  i.}
\end{itemize}
O mesmo que \textunderscore escoicear\textunderscore .
\section{Escoicinhativo}
\begin{itemize}
\item {Grp. gram.:adj.}
\end{itemize}
\begin{itemize}
\item {Utilização:Fig.}
\end{itemize}
Que escoicinha.
Grosseiro, insolente.
\section{Escoico}
\begin{itemize}
\item {Grp. gram.:m.  e  adj.}
\end{itemize}
Dizia-se do verso, cujas letras, lidas da direita para a esquerda ou da esquerda para a direita, formam as mesmas palavras.
\section{Escoiço}
\begin{itemize}
\item {Grp. gram.:m.}
\end{itemize}
\begin{itemize}
\item {Utilização:Prov.}
\end{itemize}
\begin{itemize}
\item {Utilização:trasm.}
\end{itemize}
Acto de escoiçar.
Restos.
O final.
O que resta no fundo de uma vasilha.
\section{Escoimado}
\begin{itemize}
\item {Grp. gram.:adj.}
\end{itemize}
\begin{itemize}
\item {Utilização:Fig.}
\end{itemize}
Livre de cóima.
Innocente, livre de culpa. Cf. \textunderscore Eufrosina\textunderscore , 111.
\section{Escoimar}
\begin{itemize}
\item {Grp. gram.:v. t.}
\end{itemize}
\begin{itemize}
\item {Proveniência:(De \textunderscore coimar\textunderscore )}
\end{itemize}
Livrar de cóima.
Fiscalizar, prevenir, para não haver ensejo a cóima.
Limpar.
Livrar de defeito ou censura.
\section{Escôiparo}
\begin{itemize}
\item {Grp. gram.:m.}
\end{itemize}
\begin{itemize}
\item {Utilização:Prov.}
\end{itemize}
\begin{itemize}
\item {Utilização:trasm.}
\end{itemize}
O mesmo que \textunderscore escopro\textunderscore .
\section{Escóira}
\begin{itemize}
\item {Grp. gram.:f.}
\end{itemize}
\begin{itemize}
\item {Utilização:Marn.}
\end{itemize}
\begin{itemize}
\item {Utilização:Prov.}
\end{itemize}
\begin{itemize}
\item {Utilização:trasm.}
\end{itemize}
Sulfato de cal.
O mesmo que \textunderscore escória\textunderscore , principalmente escória de ferro.
(Metáth. de \textunderscore escória\textunderscore )
\section{Escoiral}
\begin{itemize}
\item {Grp. gram.:m.}
\end{itemize}
\begin{itemize}
\item {Utilização:T. da Bairrada}
\end{itemize}
O mesmo que \textunderscore escorial\textunderscore .
\section{Escol}
\begin{itemize}
\item {Grp. gram.:m.}
\end{itemize}
\begin{itemize}
\item {Utilização:Fig.}
\end{itemize}
\begin{itemize}
\item {Proveniência:(Do rad. de \textunderscore escolher\textunderscore )}
\end{itemize}
Aquillo que há de mais distinto num grupo ou série.
A flôr, a nata: \textunderscore o escol dos literatos\textunderscore .
\section{Escola}
\begin{itemize}
\item {Grp. gram.:f.}
\end{itemize}
\begin{itemize}
\item {Utilização:Fig.}
\end{itemize}
\begin{itemize}
\item {Proveniência:(Gr. \textunderscore skhole\textunderscore )}
\end{itemize}
Casa ou estabelecimento, em que se recebe ensino de sciências, letras ou artes: \textunderscore escola polytéchnica\textunderscore .
Conjunto dos alumnos de uma escola.
Systema ou seita: \textunderscore a escola de Luthero\textunderscore .
Experiência.
Exemplo; aprendizado.
\section{Escolar}
\begin{itemize}
\item {Grp. gram.:adj.}
\end{itemize}
\begin{itemize}
\item {Grp. gram.:M.  e  f.}
\end{itemize}
\begin{itemize}
\item {Grp. gram.:M.}
\end{itemize}
Relativo a escola.
Pessôa que frequenta uma escola.
Estudante: \textunderscore os escolares do lyceu\textunderscore .
Peixe escômbrida, semelhante á pescada.
Homem douto, erudito, sábio. Cf. Garrett, \textunderscore Fr. Luis de Sousa\textunderscore , act. I, sc. 2.
\section{Escolarca}
\begin{itemize}
\item {Grp. gram.:m.}
\end{itemize}
Superintendente de escolas, na Grécia antiga.
\section{Escolarcha}
\begin{itemize}
\item {fónica:ca}
\end{itemize}
\begin{itemize}
\item {Grp. gram.:m.}
\end{itemize}
Superintendente de escolas, na Grécia antiga.
\section{Escolástica}
\begin{itemize}
\item {Grp. gram.:f.}
\end{itemize}
\begin{itemize}
\item {Proveniência:(De \textunderscore escolástico\textunderscore )}
\end{itemize}
Philosophia, que se ensinava nas escolas da Idade-Média e que se manteve em alguns estabelecimentos até fins do século XVIII.
\section{Escolasticismo}
\begin{itemize}
\item {Grp. gram.:m.}
\end{itemize}
O mesmo que \textunderscore escolástica\textunderscore .
\section{Escolástico}
\begin{itemize}
\item {Grp. gram.:adj.}
\end{itemize}
\begin{itemize}
\item {Grp. gram.:M.}
\end{itemize}
\begin{itemize}
\item {Proveniência:(Lat. \textunderscore scholasticus\textunderscore )}
\end{itemize}
Relativo a escolas: \textunderscore a vida escolástica\textunderscore .
Relativo a escolástica.
Próprio de estudantes: \textunderscore rapaziadas escolásticas\textunderscore .
Estudante.
Partidário da escolástica.
\section{Escolatício}
\begin{itemize}
\item {Grp. gram.:adj.}
\end{itemize}
\begin{itemize}
\item {Utilização:Agr.}
\end{itemize}
\begin{itemize}
\item {Proveniência:(Do lat. \textunderscore excolare\textunderscore , escoar)}
\end{itemize}
Diz-se das águas, que, infiltrando-se na terra, passam de um prédio para outro, situado inferiormente. Cf. Assis, \textunderscore Águas\textunderscore , 194 e 169.
\section{Escoldrinhar}
\begin{itemize}
\item {Grp. gram.:v. t.}
\end{itemize}
\begin{itemize}
\item {Utilização:Ant.}
\end{itemize}
(V.esquadrinhar)
\section{Escolecologia}
\begin{itemize}
\item {Grp. gram.:f.}
\end{itemize}
\begin{itemize}
\item {Proveniência:(Do gr. \textunderscore skolex\textunderscore  + \textunderscore logos\textunderscore )}
\end{itemize}
Tratado dos vermes.
\section{Escolecológico}
\begin{itemize}
\item {Grp. gram.:adj.}
\end{itemize}
Relativo a escolecologia.
\section{Escolha}
\begin{itemize}
\item {fónica:cô}
\end{itemize}
\begin{itemize}
\item {Grp. gram.:f.}
\end{itemize}
Acto ou effeito de escolher.
Gôsto, selecção.
\section{Escolhedeira}
\begin{itemize}
\item {Grp. gram.:f.}
\end{itemize}
\begin{itemize}
\item {Proveniência:(De \textunderscore escolher\textunderscore )}
\end{itemize}
Máquina, guarnecida de um ou mais tambores, formados de pentes de aço, e de cylindros de escovas, para abrir e limpar a lan, nas fábricas de lanifícios.
\section{Escolhedor}
\begin{itemize}
\item {Grp. gram.:adj.}
\end{itemize}
\begin{itemize}
\item {Grp. gram.:M.}
\end{itemize}
Que escolhe.
Aquelle que escolhe.
Apparelho agrícola, o mesmo que \textunderscore calibrador\textunderscore .
\section{Escolheita}
\begin{itemize}
\item {Grp. gram.:f.}
\end{itemize}
\begin{itemize}
\item {Utilização:Ant.}
\end{itemize}
\begin{itemize}
\item {Proveniência:(De \textunderscore escolheito\textunderscore )}
\end{itemize}
O mesmo que \textunderscore escolha\textunderscore . Cf. \textunderscore Port. Mon. Hist.\textunderscore , \textunderscore Script.\textunderscore , 281.
\section{Escolheito}
\begin{itemize}
\item {Grp. gram.:adj.}
\end{itemize}
\begin{itemize}
\item {Utilização:Ant.}
\end{itemize}
Que se escolheu; escolhido.
\section{Escolher}
\begin{itemize}
\item {Grp. gram.:v. t.}
\end{itemize}
\begin{itemize}
\item {Grp. gram.:V. i.}
\end{itemize}
\begin{itemize}
\item {Proveniência:(De \textunderscore colher\textunderscore )}
\end{itemize}
Preferir: \textunderscore escolher uma prenda\textunderscore .
Separar do que se julga mau ou menos bom: \textunderscore escolher feijões\textunderscore .
Eleger.
Fazer selecção de: \textunderscore escolher versos\textunderscore .
Fazer distincção.
Optar.
\section{Escolhidamente}
\begin{itemize}
\item {Grp. gram.:adv.}
\end{itemize}
\begin{itemize}
\item {Proveniência:(De \textunderscore escolhido\textunderscore )}
\end{itemize}
Com preferência.
\section{Escolhido}
\begin{itemize}
\item {Grp. gram.:adj.}
\end{itemize}
\begin{itemize}
\item {Proveniência:(De \textunderscore escolher\textunderscore )}
\end{itemize}
Preferido.
Separado do que é mau ou bom: \textunderscore versos escolhidos\textunderscore .
\section{Escolhimento}
\begin{itemize}
\item {Grp. gram.:m.}
\end{itemize}
O mesmo que \textunderscore escolha\textunderscore .
\section{Escolho}
\begin{itemize}
\item {fónica:cô}
\end{itemize}
\begin{itemize}
\item {Grp. gram.:m.}
\end{itemize}
\begin{itemize}
\item {Utilização:Ext.}
\end{itemize}
\begin{itemize}
\item {Utilização:Fig.}
\end{itemize}
\begin{itemize}
\item {Proveniência:(Lat. \textunderscore scopulus\textunderscore )}
\end{itemize}
Rochedo quási á flôr da água; recife.
Rochedo em costa marítima.
Obstáculo.
Perigo: \textunderscore os escolhos da vida\textunderscore .
\section{Escolhudo}
\begin{itemize}
\item {Grp. gram.:adj.}
\end{itemize}
O mesmo que \textunderscore escolhido\textunderscore .
\section{Escolhuto}
\begin{itemize}
\item {Grp. gram.:adj.}
\end{itemize}
\begin{itemize}
\item {Utilização:Ant.}
\end{itemize}
O mesmo que \textunderscore escolhido\textunderscore .
\section{Escólia}
\begin{itemize}
\item {Grp. gram.:f.}
\end{itemize}
\begin{itemize}
\item {Proveniência:(Do gr. \textunderscore skoleos\textunderscore )}
\end{itemize}
Gênero de insectos hymenópteros.
\section{Escoliador}
\begin{itemize}
\item {Grp. gram.:m.}
\end{itemize}
\begin{itemize}
\item {Proveniência:(De \textunderscore escoliar\textunderscore )}
\end{itemize}
Aquelle que faz escólios a uma obra.
\section{Escoliar}
\begin{itemize}
\item {Grp. gram.:v. i.}
\end{itemize}
Tirar escólios, formar escólios.
\section{Escoliaste}
\begin{itemize}
\item {Grp. gram.:m.}
\end{itemize}
\begin{itemize}
\item {Proveniência:(Gr. \textunderscore skoliastes\textunderscore )}
\end{itemize}
Aquelle, que fez escólios.
Commentador, explicador.
\section{Escoliastes}
\begin{itemize}
\item {Grp. gram.:m.}
\end{itemize}
\begin{itemize}
\item {Proveniência:(Gr. \textunderscore skoliastes\textunderscore )}
\end{itemize}
Aquelle, que fez escólios.
Commentador, explicador.
\section{Escólio}
\begin{itemize}
\item {Grp. gram.:m.}
\end{itemize}
\begin{itemize}
\item {Proveniência:(Gr. \textunderscore skolion\textunderscore )}
\end{itemize}
Commentário, explicação grammatical ou crítica, para tornar intelligíveis os autores clássicos.
Explicação de um texto.
\section{Escoliose}
\begin{itemize}
\item {Grp. gram.:f.}
\end{itemize}
\begin{itemize}
\item {Proveniência:(Gr. \textunderscore skoliosis\textunderscore )}
\end{itemize}
Desvio ou encurvamento lateral da espinha dorsal.
\section{Escolmar}
\begin{itemize}
\item {Grp. gram.:v. t.}
\end{itemize}
O mesmo que \textunderscore descolmar\textunderscore .
\section{Escolopendra}
\begin{itemize}
\item {Grp. gram.:f.}
\end{itemize}
\begin{itemize}
\item {Proveniência:(Do gr. \textunderscore skolopendra\textunderscore )}
\end{itemize}
Planta cryptogâmica.
Animal myriápode, que tem oitenta e tantas patas.
\section{Escolta}
\begin{itemize}
\item {Grp. gram.:f.}
\end{itemize}
\begin{itemize}
\item {Utilização:Ant.}
\end{itemize}
\begin{itemize}
\item {Proveniência:(Do b. lat. \textunderscore scorta\textunderscore )}
\end{itemize}
Troço de tropa, destacado para acompanhar pessôas ou coisas.
Navios de guerra, que acompanhavam os mercantes para os defender.
\section{Escoltar}
\begin{itemize}
\item {Grp. gram.:v. t.}
\end{itemize}
\begin{itemize}
\item {Proveniência:(De \textunderscore escolta\textunderscore )}
\end{itemize}
Acompanhar em grupo, para defender ou guardar.
\section{Escombrado}
\begin{itemize}
\item {Grp. gram.:adj.}
\end{itemize}
\begin{itemize}
\item {Utilização:Prov.}
\end{itemize}
\begin{itemize}
\item {Proveniência:(De \textunderscore escombrar\textunderscore )}
\end{itemize}
Diz-se do terreno talhado em combros ou cômoros.
\section{Escombrar}
\begin{itemize}
\item {Grp. gram.:v. t.}
\end{itemize}
\begin{itemize}
\item {Utilização:Prov.}
\end{itemize}
\begin{itemize}
\item {Utilização:Prov.}
\end{itemize}
\begin{itemize}
\item {Utilização:dur.}
\end{itemize}
\begin{itemize}
\item {Utilização:Fig.}
\end{itemize}
\begin{itemize}
\item {Proveniência:(De \textunderscore combro\textunderscore )}
\end{itemize}
Formar combro ou alambor em; desaterrar, formando talude.
Deitar abaixo o combro de (um prédio).
Bater de rijo.
\section{Escômbridas}
\begin{itemize}
\item {Grp. gram.:m. pl.}
\end{itemize}
O mesmo que \textunderscore escombroides\textunderscore .
\section{Escombro}
\begin{itemize}
\item {Grp. gram.:m.}
\end{itemize}
\begin{itemize}
\item {Proveniência:(Gr. \textunderscore skombros\textunderscore )}
\end{itemize}
Gênero de peixes, que têm por typo a cavalla.
\section{Escombroides}
\begin{itemize}
\item {Grp. gram.:m. pl.}
\end{itemize}
Fam. de peixes, a que pertence o escombro.
\section{Escombros}
\begin{itemize}
\item {Grp. gram.:m. pl.}
\end{itemize}
Entulho.
Destroços; ruínas:«\textunderscore fumegante chaos de escombros e de ruínas\textunderscore ». Latino, \textunderscore Hist. Pol.\textunderscore , I, 17.«\textunderscore ...sobre os escombros do seu throno\textunderscore ». \textunderscore Idem\textunderscore , \textunderscore Elogios\textunderscore , 346.
(Cast. \textunderscore escombros\textunderscore )
\section{Escommunal}
\begin{itemize}
\item {Grp. gram.:adj.}
\end{itemize}
(V.descommunal)
\section{Escomunal}
\begin{itemize}
\item {Grp. gram.:adj.}
\end{itemize}
(V.descomunal)
\section{Escondarelo}
\begin{itemize}
\item {fónica:darê}
\end{itemize}
\begin{itemize}
\item {Grp. gram.:m.}
\end{itemize}
(Corr. alent. de \textunderscore esconderelo\textunderscore )
\section{Escondedoiro}
\begin{itemize}
\item {Grp. gram.:m.}
\end{itemize}
\begin{itemize}
\item {Proveniência:(De \textunderscore esconder\textunderscore )}
\end{itemize}
Esconderijo.
\section{Escondedor}
\begin{itemize}
\item {Grp. gram.:m.}
\end{itemize}
Aquelle que esconde.
\section{Escondedouro}
\begin{itemize}
\item {Grp. gram.:m.}
\end{itemize}
\begin{itemize}
\item {Proveniência:(De \textunderscore esconder\textunderscore )}
\end{itemize}
Esconderijo.
\section{Escondedura}
\begin{itemize}
\item {Grp. gram.:f.}
\end{itemize}
Acto de esconder.
\section{Esconde-esconde}
\begin{itemize}
\item {Grp. gram.:m.}
\end{itemize}
Espécie de jôgo popular; o mesmo que \textunderscore escondidas\textunderscore ?
\section{Esconder}
\begin{itemize}
\item {Grp. gram.:v. t.}
\end{itemize}
\begin{itemize}
\item {Proveniência:(Do lat. \textunderscore abscondere\textunderscore )}
\end{itemize}
Pôr onde se não póde vêr.
Pôr em recato.
Resguardar.
Tapar; occultar: \textunderscore esconder a cara\textunderscore .
Disfarçar.
\section{Esconderelo}
\begin{itemize}
\item {fónica:derê}
\end{itemize}
\begin{itemize}
\item {Grp. gram.:m.}
\end{itemize}
\begin{itemize}
\item {Grp. gram.:Pl.}
\end{itemize}
\begin{itemize}
\item {Utilização:Prov.}
\end{itemize}
\begin{itemize}
\item {Utilização:alent.}
\end{itemize}
O mesmo que \textunderscore esconderijo\textunderscore . Cf. Filinto, II, 82, e XII, 159.
O mesmo que \textunderscore escondidas\textunderscore , jôgo.
\section{Esconderijeira}
\begin{itemize}
\item {Grp. gram.:f.}
\end{itemize}
\begin{itemize}
\item {Proveniência:(De \textunderscore esconderijo\textunderscore )}
\end{itemize}
(V.carriça)
\section{Esconderijo}
\begin{itemize}
\item {Grp. gram.:m.}
\end{itemize}
\begin{itemize}
\item {Proveniência:(De \textunderscore esconder\textunderscore )}
\end{itemize}
Lugar, em que se esconde alguém ou alguma coisa.
Lugar, destinado ou próprio para refúgio; recanto.
\section{Esconderilho}
\begin{itemize}
\item {Grp. gram.:m.}
\end{itemize}
\begin{itemize}
\item {Utilização:Prov.}
\end{itemize}
\begin{itemize}
\item {Utilização:beir.}
\end{itemize}
O mesmo que \textunderscore esconderijo\textunderscore .
\section{Escondidamente}
\begin{itemize}
\item {Grp. gram.:adv.}
\end{itemize}
\begin{itemize}
\item {Proveniência:(De \textunderscore escondido\textunderscore )}
\end{itemize}
Ás occultas.
\section{Escondidas}
\begin{itemize}
\item {Grp. gram.:f. pl.}
\end{itemize}
\begin{itemize}
\item {Grp. gram.:Loc. adv.}
\end{itemize}
\begin{itemize}
\item {Proveniência:(De \textunderscore escondido\textunderscore )}
\end{itemize}
Espécie de jôgo popular.
\textunderscore Ás escondidas\textunderscore , occultamente.
\section{Escondido}
\begin{itemize}
\item {Grp. gram.:adj.}
\end{itemize}
\begin{itemize}
\item {Proveniência:(De \textunderscore esconder\textunderscore )}
\end{itemize}
Occulto.
Que desappareceu: \textunderscore depois de escondido o Sol\textunderscore .
\section{Escondimento}
\begin{itemize}
\item {Grp. gram.:m.}
\end{itemize}
Acto de esconder.
\section{Escondoirelo}
\begin{itemize}
\item {Grp. gram.:m.}
\end{itemize}
\begin{itemize}
\item {Utilização:Ant.}
\end{itemize}
Jôgo de rapazes, em que um delles esconde alguma coisa, para os outros a acharem. Cf. Lobo, \textunderscore Auto do Nascimento\textunderscore .
(Cp. \textunderscore esconderelo\textunderscore )
\section{Escondrigueira}
\begin{itemize}
\item {Grp. gram.:f.}
\end{itemize}
\begin{itemize}
\item {Utilização:Prov.}
\end{itemize}
O mesmo que \textunderscore carriça\textunderscore .
(Cp. \textunderscore esconderijeira\textunderscore )
\section{Escondrijo}
\begin{itemize}
\item {Grp. gram.:m.}
\end{itemize}
(V.esconderijo)
\section{Esconjuntar}
\begin{itemize}
\item {Grp. gram.:v. t.}
\end{itemize}
(V.desconjuntar)
\section{Esconjuração}
\begin{itemize}
\item {Grp. gram.:f.}
\end{itemize}
O mesmo que \textunderscore esconjuro\textunderscore .
\section{Esconjurador}
\begin{itemize}
\item {Grp. gram.:adj.}
\end{itemize}
\begin{itemize}
\item {Grp. gram.:M.}
\end{itemize}
Que esconjura.
Aquelle que esconjura.
\section{Esconjurar}
\begin{itemize}
\item {Grp. gram.:v. t.}
\end{itemize}
\begin{itemize}
\item {Proveniência:(De \textunderscore conjurar\textunderscore )}
\end{itemize}
Fazer jurar.
Tomar juramento a.
Exorcizar.
Supplicar.
Dirigir imprecações a.
Amaldiçoar.
O mesmo que \textunderscore abjurar\textunderscore :«\textunderscore ...mui de coração se apaixonara de ver aquelles homens a esconjurar assim a sua santa fé\textunderscore ». Filinto, \textunderscore D. Man.\textunderscore , II, 266.
\section{Esconjuro}
\begin{itemize}
\item {Grp. gram.:m.}
\end{itemize}
\begin{itemize}
\item {Proveniência:(De \textunderscore esconjurar\textunderscore )}
\end{itemize}
Juramento, acompanhado de imprecações.
Exorcismo.
\section{Esconsar}
\begin{itemize}
\item {Grp. gram.:v. t.}
\end{itemize}
\begin{itemize}
\item {Utilização:T. da Bairrada}
\end{itemize}
Tornar esconso; occultar. Cf. Filinto, VII, 161.
Despejar, esgotar.
\section{Esconso}
\begin{itemize}
\item {Grp. gram.:m.}
\end{itemize}
\begin{itemize}
\item {Grp. gram.:Adj.}
\end{itemize}
\begin{itemize}
\item {Utilização:T. da Bairrada}
\end{itemize}
\begin{itemize}
\item {Proveniência:(Do lat. \textunderscore absconsus\textunderscore )}
\end{itemize}
Esconderijo, recanto.
Lugar occulto.
Esguelha, soslaio:«\textunderscore olhar de esconso...\textunderscore »Camillo, \textunderscore Esqueleto\textunderscore , 58.
Que tem escoante, que é declivoso. Cf. Filinto, VIII, 71 e 94.
Escoado.
Despejado.
\section{Escontra}
\begin{itemize}
\item {Grp. gram.:prep.}
\end{itemize}
\begin{itemize}
\item {Utilização:Ant.}
\end{itemize}
De fronte de.
Contra; em opposição a: \textunderscore marchou escontra Toledo\textunderscore .
\section{Escopa}
\begin{itemize}
\item {fónica:cô}
\end{itemize}
\begin{itemize}
\item {Grp. gram.:f.}
\end{itemize}
\begin{itemize}
\item {Utilização:Bras. de Minas}
\end{itemize}
Espécie de jôgo de cartas.
\section{Escôparo}
\begin{itemize}
\item {Grp. gram.:m.}
\end{itemize}
\begin{itemize}
\item {Utilização:Prov.}
\end{itemize}
\begin{itemize}
\item {Utilização:Ant.}
\end{itemize}
O mesmo que \textunderscore escopro\textunderscore . Cf. Gil Vicente.
\section{Escopeira}
\begin{itemize}
\item {Grp. gram.:f.}
\end{itemize}
O mesmo que \textunderscore escopeiro\textunderscore .
\section{Escopeiro}
\begin{itemize}
\item {Grp. gram.:m.}
\end{itemize}
\begin{itemize}
\item {Grp. gram.:Art.}
\end{itemize}
\begin{itemize}
\item {Proveniência:(Do lat. \textunderscore scopa\textunderscore )}
\end{itemize}
Brocha, para alcatroar navios.
Forja de campanha. Cf. Leoni, \textunderscore Diccion. de Artilh.\textunderscore 
\section{Escopelismo}
\begin{itemize}
\item {Grp. gram.:m.}
\end{itemize}
\begin{itemize}
\item {Proveniência:(Do gr. \textunderscore skopelos\textunderscore )}
\end{itemize}
\begin{itemize}
\item {Grp. gram.:m.}
\end{itemize}
\begin{itemize}
\item {Proveniência:(Do gr. \textunderscore skopelon\textunderscore )}
\end{itemize}
Acto de lançar pedras em terreno alheio, acto a que se applicava penalidade especial.
Acto de dispor pedras por certa fórma numa propriedade alheia, o que na antiguidade romana equivalia a uma ameaça de morte.
\section{Escopeta}
\begin{itemize}
\item {fónica:pê}
\end{itemize}
\begin{itemize}
\item {Grp. gram.:f.}
\end{itemize}
\begin{itemize}
\item {Utilização:Pop.}
\end{itemize}
\begin{itemize}
\item {Proveniência:(Do it. \textunderscore schiopetto\textunderscore )}
\end{itemize}
Espingarda antiga e curta.
Espingarda.
\section{Escopetada}
\begin{itemize}
\item {Grp. gram.:f.}
\end{itemize}
\begin{itemize}
\item {Utilização:Des.}
\end{itemize}
Tiro de escopeta.
\section{Escopetaria}
\begin{itemize}
\item {Grp. gram.:f.}
\end{itemize}
\begin{itemize}
\item {Utilização:Des.}
\end{itemize}
Troço de gente armada com escopeta:«\textunderscore depois de dispersarem toda a escopetaria...\textunderscore »\textunderscore Jorn. de Áfr.\textunderscore , c. VI.
\section{Escopetear}
\begin{itemize}
\item {Grp. gram.:v. t.}
\end{itemize}
\begin{itemize}
\item {Grp. gram.:V. i.}
\end{itemize}
Disparar escopeta contra.
Dar tiros de escopeta.
\section{Escopeteiro}
\begin{itemize}
\item {Grp. gram.:m.}
\end{itemize}
\begin{itemize}
\item {Grp. gram.:Adj.}
\end{itemize}
\begin{itemize}
\item {Utilização:Bras. do N}
\end{itemize}
Soldado, armado de escopeta.
Diz-se do atirador, que não erra o alvo.
\section{Escopo}
\begin{itemize}
\item {Grp. gram.:m.}
\end{itemize}
\begin{itemize}
\item {Proveniência:(Do gr. \textunderscore skopos\textunderscore )}
\end{itemize}
Alvo, mira.
Propósito; fim: \textunderscore o escopo do gazetilheiro é fazer rir\textunderscore .
\section{Escopoleína}
\begin{itemize}
\item {Grp. gram.:f.}
\end{itemize}
\begin{itemize}
\item {Utilização:Pharm.}
\end{itemize}
Medicamento mydriático, mais activo que a atropina e de effeito mais duradoiro.
\section{Escopro}
\begin{itemize}
\item {fónica:cô}
\end{itemize}
\begin{itemize}
\item {Grp. gram.:m.}
\end{itemize}
\begin{itemize}
\item {Proveniência:(Do lat. \textunderscore scalprum\textunderscore )}
\end{itemize}
Instrumento de ferro e aço, para lavrar pedra, madeira, etc.
Cinzel.
\section{Escóptula}
\begin{itemize}
\item {Grp. gram.:f.}
\end{itemize}
Designação, que os antigos anatómicos davam á omoplata.
(Cp. lat. \textunderscore scapulae\textunderscore )
\section{Escora}
\begin{itemize}
\item {Grp. gram.:f.}
\end{itemize}
\begin{itemize}
\item {Utilização:Fig.}
\end{itemize}
Peça que ampara ou sustém; espeque.
Amparo, protecção: \textunderscore teve bôas escoras o patife\textunderscore .
(Cast. \textunderscore escora\textunderscore )
\section{Escòrado}
\begin{itemize}
\item {Grp. gram.:adj.}
\end{itemize}
\begin{itemize}
\item {Utilização:Ant.}
\end{itemize}
O mesmo que descòrado.
\section{Escoramento}
\begin{itemize}
\item {Grp. gram.:m.}
\end{itemize}
Acto ou effeito de escorar.
\section{Escorar}
\begin{itemize}
\item {Grp. gram.:v. t.}
\end{itemize}
Pôr escoras a.
Amparar, suster.
\section{Escorbútico}
\begin{itemize}
\item {Grp. gram.:adj.}
\end{itemize}
Relativo a escorbuto.
Que é da natureza do escorbuto.
Que padece escorbuto.
\section{Escorbútio}
\begin{itemize}
\item {Grp. gram.:adj.}
\end{itemize}
\begin{itemize}
\item {Utilização:Ant.}
\end{itemize}
O mesmo que \textunderscore escorbútico\textunderscore . Cf. \textunderscore Âncora Med.\textunderscore , 190.
\section{Escorbuto}
\begin{itemize}
\item {Grp. gram.:m.}
\end{itemize}
Doença geral, que se manifesta principalmente por abatimento de energia, entumecimento de gengivas, mau hálito, etc.
(Or. germ.)
\section{Escorçar}
\begin{itemize}
\item {Grp. gram.:v. t.}
\end{itemize}
\begin{itemize}
\item {Proveniência:(De \textunderscore escôrço\textunderscore ^1)}
\end{itemize}
Fazer o escôrço de: \textunderscore escorçar um poema\textunderscore .
\section{Escorcemelar-se}
\begin{itemize}
\item {Grp. gram.:v. p.}
\end{itemize}
O mesmo que \textunderscore escorçomelar-se\textunderscore . Cf. Camillo, \textunderscore Sangue\textunderscore , 96.
\section{Escorchador}
\begin{itemize}
\item {Grp. gram.:adj.}
\end{itemize}
\begin{itemize}
\item {Grp. gram.:M.}
\end{itemize}
Que escorcha.
Aquelle que escorcha.
\section{Escorchamento}
\begin{itemize}
\item {Grp. gram.:m.}
\end{itemize}
Acto de escorchar.
\section{Escorchar}
\begin{itemize}
\item {Grp. gram.:v. t.}
\end{itemize}
\begin{itemize}
\item {Utilização:Ant.}
\end{itemize}
Tirar a corcha a.
Descortiçar.
Descascar.
Roubar.
\section{Escorcioneira}
\begin{itemize}
\item {Grp. gram.:f.}
\end{itemize}
Gênero de plantas compostas, (\textunderscore scorzonera hispanica\textunderscore , Lin.).
\section{Escôrço}
\begin{itemize}
\item {Grp. gram.:m.}
\end{itemize}
\begin{itemize}
\item {Utilização:Pint.}
\end{itemize}
\begin{itemize}
\item {Utilização:Ext.}
\end{itemize}
\begin{itemize}
\item {Proveniência:(Do it. \textunderscore scorzio\textunderscore )}
\end{itemize}
Reducção das dimensões de um desenho.
Effeito de perspectiva, segundo o qual os objectos, vistos de frente, apresentam dimensões reduzidas.
Arte de representar os objectos em proporções menores que a realidade.
Sýnthese.
Resumo.
Esbôço.
\section{Escôrço}
\begin{itemize}
\item {Grp. gram.:m.}
\end{itemize}
\begin{itemize}
\item {Utilização:Ant.}
\end{itemize}
O mesmo que \textunderscore corticeira\textunderscore , vasilha.
\section{Escorçomelar-se}
\begin{itemize}
\item {Grp. gram.:v. p.}
\end{itemize}
\begin{itemize}
\item {Utilização:Pop.}
\end{itemize}
Escapulir-se; esgueirar-se, surrateiramente. Cf. Camillo, \textunderscore Narcót.\textunderscore , I, 234.
(Cp. \textunderscore escoçumelar-se\textunderscore )
\section{Escorçoneira}
\begin{itemize}
\item {Grp. gram.:f.}
\end{itemize}
O mesmo ou melhor que \textunderscore escorcioneira\textunderscore . Cf. B. Pereira, \textunderscore Prosódia\textunderscore , vb. \textunderscore scorzonera\textunderscore .
\section{Escordar}
\begin{itemize}
\item {Grp. gram.:v. t.}
\end{itemize}
\begin{itemize}
\item {Utilização:Des.}
\end{itemize}
Tirar do somno, despertar.
(Cf. \textunderscore acordar\textunderscore )
\section{Escordar}
\begin{itemize}
\item {Grp. gram.:v. t.}
\end{itemize}
\begin{itemize}
\item {Utilização:Pop.}
\end{itemize}
Recordar:«\textunderscore já me não escordo\textunderscore ». Camillo, \textunderscore Brasileira\textunderscore , 12.
\section{Escórdio}
\begin{itemize}
\item {Grp. gram.:m.}
\end{itemize}
\begin{itemize}
\item {Proveniência:(Do gr. \textunderscore skordon\textunderscore )}
\end{itemize}
Planta labiada, medicinal.
\section{Escória}
\begin{itemize}
\item {Grp. gram.:f.}
\end{itemize}
\begin{itemize}
\item {Utilização:Fig.}
\end{itemize}
\begin{itemize}
\item {Proveniência:(Do gr. \textunderscore skoria\textunderscore )}
\end{itemize}
Fezes dos metaes, depois de sujeitos á fusão.
Arraia miúda, ralé.
A parte mais desprezível da sociedade.
\section{Escoriação}
\begin{itemize}
\item {Grp. gram.:f.}
\end{itemize}
Acto ou effeito de escoriar; esfoladura.
\section{Escoriáceo}
\begin{itemize}
\item {Grp. gram.:adj.}
\end{itemize}
Que é da natureza das escórias.
\section{Escorial}
\begin{itemize}
\item {Grp. gram.:m.}
\end{itemize}
\begin{itemize}
\item {Utilização:Prov.}
\end{itemize}
Terreno ou campo, onde há escórias de metaes. (Colhido na Bairrada)
\section{Escoriar}
\begin{itemize}
\item {Grp. gram.:v. t.}
\end{itemize}
\begin{itemize}
\item {Proveniência:(Do lat. \textunderscore excoriare\textunderscore )}
\end{itemize}
Esfolar, ferir superficialmente.
\section{Escoriar}
\begin{itemize}
\item {Grp. gram.:v. t.}
\end{itemize}
Tirar as escórias a.
Purificar; limpar.
Escorificar.
\section{Escorificar}
\begin{itemize}
\item {Grp. gram.:v. t.}
\end{itemize}
\begin{itemize}
\item {Proveniência:(De \textunderscore escória\textunderscore  + lat. \textunderscore facere\textunderscore )}
\end{itemize}
Purificar, tirar as escórias a (o metal).
\section{Escorificatório}
\begin{itemize}
\item {Grp. gram.:m.}
\end{itemize}
\begin{itemize}
\item {Proveniência:(De \textunderscore escorificar\textunderscore )}
\end{itemize}
Vaso, para purificar metaes, separando-os das escórias.
\section{Escorinhote}
\begin{itemize}
\item {Grp. gram.:m.}
\end{itemize}
\begin{itemize}
\item {Utilização:Bras}
\end{itemize}
\begin{itemize}
\item {Proveniência:(De \textunderscore escora\textunderscore )}
\end{itemize}
Escora, que reforça as comportas dos açudes, nos engenhos de açúcar.
\section{Escorjar}
\begin{itemize}
\item {Grp. gram.:v. t.}
\end{itemize}
\begin{itemize}
\item {Grp. gram.:V. i.}
\end{itemize}
Dar posição forçada a; constranger.
Torcer.
Confranger-se.
\section{Escornada}
\begin{itemize}
\item {Grp. gram.:f.}
\end{itemize}
(V.cornada)
\section{Escornador}
\begin{itemize}
\item {Grp. gram.:adj.}
\end{itemize}
\begin{itemize}
\item {Grp. gram.:M.}
\end{itemize}
Que escorna.
Aquelle que escorna.
\section{Escornar}
\begin{itemize}
\item {Grp. gram.:v. t.}
\end{itemize}
\begin{itemize}
\item {Utilização:Des.}
\end{itemize}
Marrar.
Maltratar com os cornos.
Investir contra.
Acommeter.
Desprezar; vituperar.
\section{Escorneador}
\begin{itemize}
\item {Grp. gram.:adj.}
\end{itemize}
Que escorneia.
\section{Escornear}
\begin{itemize}
\item {Grp. gram.:v. i.}
\end{itemize}
\begin{itemize}
\item {Proveniência:(De \textunderscore corno\textunderscore )}
\end{itemize}
Têr o hábito de escornar.
\section{Escornichar}
\begin{itemize}
\item {Grp. gram.:v. t.}
\end{itemize}
O mesmo que \textunderscore escornear\textunderscore .
\section{Escoroar}
\begin{itemize}
\item {Grp. gram.:v. t.}
\end{itemize}
O mesmo que \textunderscore descoroar\textunderscore .
\section{Escorodónia}
\begin{itemize}
\item {Grp. gram.:f.}
\end{itemize}
\begin{itemize}
\item {Proveniência:(Do gr. \textunderscore skorodon\textunderscore )}
\end{itemize}
Planta, da fam. das labiadas.
\section{Escorpena}
\begin{itemize}
\item {Grp. gram.:f.}
\end{itemize}
\begin{itemize}
\item {Proveniência:(Lat. \textunderscore scorpaena\textunderscore )}
\end{itemize}
Peixe, também chamado escorpião do mar.
\section{Escorpião}
\begin{itemize}
\item {Grp. gram.:m.}
\end{itemize}
\begin{itemize}
\item {Proveniência:(Lat. \textunderscore scorpio\textunderscore )}
\end{itemize}
O mesmo que \textunderscore lacrau\textunderscore .
Um dos signos do Zodíaco.
Antiga máquina de guerra, com que se atiravam pedras e frechas.
Espécie de dardo, usado em guerras antigas:«\textunderscore ...com escorpiões arremessados pelas manganellas\textunderscore ». Herculano, \textunderscore Bobo\textunderscore , 17.
\section{Escorpiôa}
\begin{itemize}
\item {Grp. gram.:f.}
\end{itemize}
\begin{itemize}
\item {Proveniência:(De \textunderscore escorpião\textunderscore )}
\end{itemize}
Planta leguminosa e medicinal.
\section{Escorpioide}
\begin{itemize}
\item {Grp. gram.:adj.}
\end{itemize}
\begin{itemize}
\item {Proveniência:(Do lat. \textunderscore scorpio\textunderscore  + gr. \textunderscore eidos\textunderscore )}
\end{itemize}
Semelhante á cauda do escorpião.
\section{Escorpiura}
\begin{itemize}
\item {Grp. gram.:f.}
\end{itemize}
(V.escorpiôa)
\section{Escorraçado}
\begin{itemize}
\item {Grp. gram.:adj.}
\end{itemize}
\begin{itemize}
\item {Utilização:Bras. de Minas}
\end{itemize}
\begin{itemize}
\item {Proveniência:(De \textunderscore escorraçar\textunderscore ?)}
\end{itemize}
Desconfiado, arisco.
\section{Escorraçar}
\begin{itemize}
\item {Grp. gram.:v. t.}
\end{itemize}
\begin{itemize}
\item {Proveniência:(Do it. \textunderscore scorrazziare\textunderscore )}
\end{itemize}
Expulsar ou pôr fóra com desprêzo.
Afugentar, batendo.
\section{Escorralhas}
\begin{itemize}
\item {Grp. gram.:f. pl.}
\end{itemize}
\begin{itemize}
\item {Proveniência:(De \textunderscore escorrer\textunderscore )}
\end{itemize}
Fundagens.
Resíduos de líquidos, nas bordas ou no fundo de vasilhas.
O mesmo que \textunderscore ralé\textunderscore .
\section{Escorralho}
\begin{itemize}
\item {Grp. gram.:m.}
\end{itemize}
\begin{itemize}
\item {Proveniência:(De \textunderscore escorrer\textunderscore )}
\end{itemize}
Fundagens.
Resíduos de líquidos, nas bordas ou no fundo de vasilhas.
O mesmo que \textunderscore ralé\textunderscore .
\section{Escorredoiro}
\begin{itemize}
\item {Grp. gram.:m.}
\end{itemize}
Lugar, por onde escorre água. Cf. Assis, \textunderscore Águas\textunderscore , 169.
\section{Escorredouro}
\begin{itemize}
\item {Grp. gram.:m.}
\end{itemize}
Lugar, por onde escorre água. Cf. Assis, \textunderscore Águas\textunderscore , 169.
\section{Escorredura}
\begin{itemize}
\item {Grp. gram.:f.}
\end{itemize}
O mesmo que \textunderscore escorralhas\textunderscore .
\section{Escorregadela}
\begin{itemize}
\item {Grp. gram.:f.}
\end{itemize}
\begin{itemize}
\item {Utilização:Fig.}
\end{itemize}
Acto de escorregar.
Descuido, falta.
\section{Escorregadiço}
\begin{itemize}
\item {Grp. gram.:adj.}
\end{itemize}
\begin{itemize}
\item {Utilização:Fig.}
\end{itemize}
\begin{itemize}
\item {Proveniência:(De \textunderscore escorregar\textunderscore )}
\end{itemize}
O mesmo que \textunderscore escorregadio\textunderscore .
Que tem tendência para o mal ou para o vício.
\section{Escorregadio}
\begin{itemize}
\item {Grp. gram.:adj.}
\end{itemize}
\begin{itemize}
\item {Proveniência:(De \textunderscore escorregar\textunderscore )}
\end{itemize}
Em que se escorrega ou em que se resvala facilmente.
Lúbrico.
\section{Escorregadoiro}
\begin{itemize}
\item {Grp. gram.:m.}
\end{itemize}
Sítio, em que se escorrega facilmente, por acaso ou por diversão.
\section{Escorregadouro}
\begin{itemize}
\item {Grp. gram.:m.}
\end{itemize}
Sítio, em que se escorrega facilmente, por acaso ou por diversão.
\section{Escorregadura}
\begin{itemize}
\item {Grp. gram.:f.}
\end{itemize}
O mesmo que \textunderscore escorregadela\textunderscore .
\section{Escorregamento}
\begin{itemize}
\item {Grp. gram.:m.}
\end{itemize}
O mesmo que \textunderscore escorregadela\textunderscore .
\section{Escorregão}
\begin{itemize}
\item {Grp. gram.:m.}
\end{itemize}
\begin{itemize}
\item {Utilização:Fam.}
\end{itemize}
Acto de escorregar.
\section{Escorregar}
\begin{itemize}
\item {Grp. gram.:v. i.}
\end{itemize}
\begin{itemize}
\item {Utilização:Fig.}
\end{itemize}
\begin{itemize}
\item {Proveniência:(De \textunderscore córrego\textunderscore )}
\end{itemize}
Deslisar, ir resvalando, com o pêso próprio.
Correr, passar.
Commeter faltas, erros.
\section{Escorregável}
\begin{itemize}
\item {Grp. gram.:adj.}
\end{itemize}
O mesmo que \textunderscore escorregadio\textunderscore .
\section{Escorrêgo}
\begin{itemize}
\item {Grp. gram.:m.}
\end{itemize}
O mesmo que \textunderscore escorregadela\textunderscore .
\section{Escorreito}
\begin{itemize}
\item {Grp. gram.:adj.}
\end{itemize}
\begin{itemize}
\item {Proveniência:(Do lat. \textunderscore correctus\textunderscore )}
\end{itemize}
Que não tem lesão.
Que tem bôa compleição, bôa figura: \textunderscore rapaz são e escorreito\textunderscore .
Apurado, correcto: \textunderscore linguagem escorreita\textunderscore .
\section{Escorrência}
\begin{itemize}
\item {Grp. gram.:f.}
\end{itemize}
Qualidade daquillo que escorre; facilidade e rapidez no escorrer.
Aquillo que escorre. Cf. Camillo, \textunderscore Brasileira\textunderscore , 66; \textunderscore Rattazi\textunderscore , 29.
\section{Escorrer}
\begin{itemize}
\item {Grp. gram.:v. t.}
\end{itemize}
\begin{itemize}
\item {Grp. gram.:Loc.}
\end{itemize}
\begin{itemize}
\item {Utilização:chul.}
\end{itemize}
\begin{itemize}
\item {Grp. gram.:V. i.}
\end{itemize}
\begin{itemize}
\item {Grp. gram.:V. p.}
\end{itemize}
\begin{itemize}
\item {Proveniência:(De \textunderscore correr\textunderscore )}
\end{itemize}
Separar (um líquido) embebido noutro corpo: \textunderscore escorrer um caldo\textunderscore .
Fazer correr completamente, despejando ou espremendo, (líquidos misturados com outra substância).
Fazer correr em fio ou em gotas: \textunderscore escorrer água\textunderscore .
Navegar, costeando.
Passar além de (porto ou lugar na costa, sem ancorar).
\textunderscore Escorrer água das batatas\textunderscore , ou \textunderscore das azeitonas\textunderscore , urinar.
Correr em gotas, em fio: \textunderscore a água escorria-lhe da capa\textunderscore .
Enxambrar.
Pingar; estar pendente.
A mesma sign. Cf. Castilho, \textunderscore Fastos\textunderscore , II, 123.
\section{Escorreria}
\begin{itemize}
\item {Grp. gram.:f.}
\end{itemize}
\begin{itemize}
\item {Utilização:Des.}
\end{itemize}
O mesmo que \textunderscore correria\textunderscore :«\textunderscore ...marciaes escorrerias.\textunderscore »\textunderscore Viriato Trág.\textunderscore , XIV, 84.
\section{Escorriça}
\begin{itemize}
\item {Grp. gram.:f.}
\end{itemize}
\begin{itemize}
\item {Utilização:Prov.}
\end{itemize}
\begin{itemize}
\item {Utilização:trasm.}
\end{itemize}
Corrida desenfreada a cavallo, até esfalfar o animal.
(Relaciona-se com \textunderscore correr\textunderscore )
\section{Escorrido}
\begin{itemize}
\item {Grp. gram.:adj.}
\end{itemize}
\begin{itemize}
\item {Utilização:T. da Bairrada}
\end{itemize}
\begin{itemize}
\item {Proveniência:(De \textunderscore escorrer\textunderscore )}
\end{itemize}
Elegante, bem pôsto.
\section{Escorrimento}
\begin{itemize}
\item {Grp. gram.:m.}
\end{itemize}
Acto ou effeito de escorrer.
\section{Escorripichar}
\begin{itemize}
\item {Grp. gram.:v. t.}
\end{itemize}
O mesmo que \textunderscore escorropichar\textunderscore .
\section{Escorropichadela}
\begin{itemize}
\item {Grp. gram.:f.}
\end{itemize}
Acto de escorropichar.
\section{Escorropicha-galhetas}
\begin{itemize}
\item {Grp. gram.:f. pl.}
\end{itemize}
\begin{itemize}
\item {Utilização:Pop.}
\end{itemize}
Sacristão.
\section{Escorropichar}
\begin{itemize}
\item {Grp. gram.:v. t.}
\end{itemize}
\begin{itemize}
\item {Utilização:Pop.}
\end{itemize}
\begin{itemize}
\item {Proveniência:(Do rad. de \textunderscore escorrer\textunderscore )}
\end{itemize}
Beber inteiramente.
Esgotar; beber as últimas gotas de.
\section{Escorropicho}
\begin{itemize}
\item {Grp. gram.:m.}
\end{itemize}
\begin{itemize}
\item {Utilização:Prov.}
\end{itemize}
\begin{itemize}
\item {Proveniência:(De \textunderscore escorropichar\textunderscore )}
\end{itemize}
Resíduos de um líquido.
Últimas gotas.
\section{Escorticar}
\textunderscore v. t.\textunderscore  (e der.)
O mesmo que \textunderscore decorticar\textunderscore , etc.
\section{Escortinar}
\begin{itemize}
\item {Grp. gram.:v. t.}
\end{itemize}
\begin{itemize}
\item {Utilização:Ant.}
\end{itemize}
Guarnecer de cortinas, (falando-se de fortalezas).
\section{Escortinhar}
\begin{itemize}
\item {Grp. gram.:v. t.}
\end{itemize}
\begin{itemize}
\item {Utilização:Fam.}
\end{itemize}
\begin{itemize}
\item {Proveniência:(De um hypoth. \textunderscore cortinhar\textunderscore , freq. de \textunderscore cortar\textunderscore )}
\end{itemize}
Cortar em bocadinhos.
\section{Escorva}
\begin{itemize}
\item {Grp. gram.:f.}
\end{itemize}
\begin{itemize}
\item {Grp. gram.:Loc.}
\end{itemize}
\begin{itemize}
\item {Utilização:pop.}
\end{itemize}
Parte da arma, em que se colloca a pólvora.
Porção de pólvora, que communica o fogo á arma.
A pólvora do tubo dos foguetes.
\textunderscore Mijar na escorva\textunderscore , frustrar a tentativa; illudir o propósito. Cf. Macedo, \textunderscore Burros\textunderscore , 289.
(Refl. de \textunderscore escarva\textunderscore ?)
\section{Escorvador}
\begin{itemize}
\item {Grp. gram.:m.}
\end{itemize}
Instrumento, para escorvar.
\section{Escorvar}
\begin{itemize}
\item {Grp. gram.:v. t.}
\end{itemize}
\begin{itemize}
\item {Utilização:Fig.}
\end{itemize}
Pôr pólvora na escorva de.
Preparar a escorva de.
Dispor, preparar.
\section{Escoseu}
\begin{itemize}
\item {Grp. gram.:m.}
\end{itemize}
\begin{itemize}
\item {Utilização:Des.}
\end{itemize}
Vibora peçonhenta.
\section{Escosipar}
\begin{itemize}
\item {Grp. gram.:v. t.}
\end{itemize}
\begin{itemize}
\item {Utilização:Prov.}
\end{itemize}
\begin{itemize}
\item {Utilização:minh.}
\end{itemize}
Coser mal.
Desmanchar.
\section{Escota}
\begin{itemize}
\item {Grp. gram.:f.}
\end{itemize}
\begin{itemize}
\item {Utilização:Náut.}
\end{itemize}
\begin{itemize}
\item {Proveniência:(Do hol. \textunderscore schoot\textunderscore )}
\end{itemize}
Cabo, para governar as velas do navio.
\section{Escote}
\begin{itemize}
\item {Grp. gram.:m.}
\end{itemize}
\begin{itemize}
\item {Utilização:Des.}
\end{itemize}
\begin{itemize}
\item {Proveniência:(Do germ. \textunderscore scot\textunderscore ?)}
\end{itemize}
Parte que, de uma despesa commum, deve sêr paga por cada um dos que fizeram essa despesa.--Bluteau regista o termo, declarando que não o encontrou em escritor algum. Podia lê-lo todavia na \textunderscore Pratica de Arismetica\textunderscore , de Gaspar Nicolau, 71.
\section{Escoteira}
\begin{itemize}
\item {Grp. gram.:f.}
\end{itemize}
\begin{itemize}
\item {Utilização:Náut.}
\end{itemize}
Peça, por onde passam as escotas.
\section{Escoteiro}
\begin{itemize}
\item {Grp. gram.:m.}
\end{itemize}
\begin{itemize}
\item {Utilização:Bras}
\end{itemize}
\begin{itemize}
\item {Grp. gram.:Adj.}
\end{itemize}
\begin{itemize}
\item {Proveniência:(De \textunderscore escote\textunderscore )}
\end{itemize}
Aquelle que viaja sem bagagem nem alforges, gastando por escote nas estalagens.
O mesmo que \textunderscore pioneiro\textunderscore .
Um dos déz indivíduos, que ordinariamente tripulam uma baleeira. Cf. \textunderscore Jorn. da Comm.\textunderscore , do Rio, de 29-VI-900.
Desacompanhado, só: \textunderscore o viajante escoteiro\textunderscore .
Que não traz bagagem consigo: \textunderscore o pagem escoteiro\textunderscore .--É voc. us. em Moçambique, Goiás, Minas, Pernambuco, etc.
\section{Escoteiro}
\begin{itemize}
\item {Grp. gram.:adj.}
\end{itemize}
Lépido?:«\textunderscore a fera, para tentar mais escoteira o salto...\textunderscore »Camillo, \textunderscore Nov. do Minho\textunderscore , IX, 17.
\section{Escotel}
\begin{itemize}
\item {Grp. gram.:m.}
\end{itemize}
O mesmo que \textunderscore escoteira\textunderscore .
\section{Escotiar}
\begin{itemize}
\item {Grp. gram.:v. t.}
\end{itemize}
\begin{itemize}
\item {Utilização:Prov.}
\end{itemize}
O mesmo que \textunderscore cotiar\textunderscore ^1.
\section{Escotilha}
\begin{itemize}
\item {Grp. gram.:f.}
\end{itemize}
\begin{itemize}
\item {Utilização:Náut.}
\end{itemize}
Alçapão ou abertura nas cobertas e porão do navio.
(Cast. \textunderscore escotilla\textunderscore )
\section{Escotilhão}
\begin{itemize}
\item {Grp. gram.:m.}
\end{itemize}
Pequena escotilha.
\section{Escotismo}
\begin{itemize}
\item {Grp. gram.:m.}
\end{itemize}
Seita ou doutrina de João Escoto, philósopho irlandês.
\section{Escotista}
\begin{itemize}
\item {Grp. gram.:m.}
\end{itemize}
Partidário do escotismo.
\section{Escótoma}
\begin{itemize}
\item {Grp. gram.:m.}
\end{itemize}
\begin{itemize}
\item {Utilização:Physiol.}
\end{itemize}
\begin{itemize}
\item {Proveniência:(Gr. \textunderscore skotoma\textunderscore )}
\end{itemize}
Lacuna do campo visual, dentro da sua peripheria.
\section{Escotomia}
\begin{itemize}
\item {Grp. gram.:f.}
\end{itemize}
O mesmo que \textunderscore escótoma\textunderscore .
Mancha escura e redonda, que apparece deante dos olhos, por doença da retina.
(Cp. \textunderscore escótoma\textunderscore )
\section{Escouça}
\begin{itemize}
\item {Grp. gram.:f.}
\end{itemize}
\begin{itemize}
\item {Utilização:T. da Feira}
\end{itemize}
O mesmo que \textunderscore arroteia\textunderscore .
\section{Escouçar}
\begin{itemize}
\item {Grp. gram.:v. t.}
\end{itemize}
\begin{itemize}
\item {Utilização:Prov.}
\end{itemize}
\begin{itemize}
\item {Utilização:alg.}
\end{itemize}
\begin{itemize}
\item {Utilização:Prov.}
\end{itemize}
\begin{itemize}
\item {Utilização:minh.}
\end{itemize}
\begin{itemize}
\item {Proveniência:(De \textunderscore couce\textunderscore )}
\end{itemize}
Procurar com diligência.
Bater com varas (os feixes de linho enriado), para que nelles não fiquem peixes.
Esvaziar (pipas ou tonéis).
O mesmo que \textunderscore escoucear\textunderscore . Cf. Camillo, \textunderscore Cancion. Al.\textunderscore , 421.
\section{Escouceador}
\begin{itemize}
\item {Grp. gram.:adj.}
\end{itemize}
\begin{itemize}
\item {Grp. gram.:M.}
\end{itemize}
Que escouceia.
Aquelle que escouceia.
\section{Escoucear}
\begin{itemize}
\item {Grp. gram.:v. t.}
\end{itemize}
\begin{itemize}
\item {Utilização:Fig.}
\end{itemize}
\begin{itemize}
\item {Grp. gram.:V. i.}
\end{itemize}
Dar couces em.
Insultar, tratar brutalmente.
Dar couces.
\section{Escoucinhador}
\begin{itemize}
\item {Grp. gram.:adj.}
\end{itemize}
\begin{itemize}
\item {Grp. gram.:M.}
\end{itemize}
Que escoucinha.
Aquelle que escoucinha.
\section{Escoucinhar}
\begin{itemize}
\item {Grp. gram.:v. t.  e  i.}
\end{itemize}
O mesmo que \textunderscore escoucear\textunderscore .
\section{Escoucinhativo}
\begin{itemize}
\item {Grp. gram.:adj.}
\end{itemize}
\begin{itemize}
\item {Utilização:Fig.}
\end{itemize}
Que escoucinha.
Grosseiro, insolente.
\section{Escouço}
\begin{itemize}
\item {Grp. gram.:m.}
\end{itemize}
\begin{itemize}
\item {Utilização:Prov.}
\end{itemize}
\begin{itemize}
\item {Utilização:trasm.}
\end{itemize}
Acto de escouçar.
Restos.
O final.
O que resta no fundo de uma vasilha.
\section{Escouvém}
\begin{itemize}
\item {Grp. gram.:m.}
\end{itemize}
(V.escovém)
\section{Escôva}
\begin{itemize}
\item {Grp. gram.:f.}
\end{itemize}
\begin{itemize}
\item {Utilização:Prov.}
\end{itemize}
\begin{itemize}
\item {Utilização:trasm.}
\end{itemize}
\begin{itemize}
\item {Grp. gram.:Loc.}
\end{itemize}
\begin{itemize}
\item {Utilização:Pop.}
\end{itemize}
\begin{itemize}
\item {Proveniência:(Do lat. \textunderscore scopa\textunderscore )}
\end{itemize}
Peça de madeira, metal, osso, etc., em que se fixam pelos ou fios de arame, e que serve para limpeza de vários objectos.
O mesmo que \textunderscore giesta\textunderscore .
\textunderscore Meter a escôva\textunderscore , mentir, dizer intrujices.
\section{Escóva}
\begin{itemize}
\item {Grp. gram.:f.}
\end{itemize}
O mesmo que \textunderscore escovação\textunderscore .
\section{Escovação}
\begin{itemize}
\item {Grp. gram.:f.}
\end{itemize}
O mesmo que \textunderscore escovadela\textunderscore .
\section{Escovadeira}
\begin{itemize}
\item {Grp. gram.:f.}
\end{itemize}
\begin{itemize}
\item {Proveniência:(De \textunderscore escovar\textunderscore )}
\end{itemize}
O mesmo que \textunderscore brossa\textunderscore , nas fábricas de lanifícios.
\section{Escovadela}
\begin{itemize}
\item {Grp. gram.:f.}
\end{itemize}
\begin{itemize}
\item {Utilização:Fig.}
\end{itemize}
Acto de escovar.
Reprehensão, censura.
\section{Escovadinho}
\begin{itemize}
\item {Grp. gram.:m.}
\end{itemize}
\begin{itemize}
\item {Utilização:Gír.}
\end{itemize}
\begin{itemize}
\item {Proveniência:(De \textunderscore escovar\textunderscore )}
\end{itemize}
Chapéu.
\section{Escovador}
\begin{itemize}
\item {Grp. gram.:m.}
\end{itemize}
Aquelle que escova.
Máquina, para limpar do pó o trigo.
\section{Escovalho}
\begin{itemize}
\item {Grp. gram.:m.}
\end{itemize}
\begin{itemize}
\item {Utilização:Pop.}
\end{itemize}
\begin{itemize}
\item {Proveniência:(De \textunderscore escovar\textunderscore )}
\end{itemize}
Rodilha molhada, presa na extremidade de um pau, para varrer do forno as últimas cinzas.
\section{Escovão}
\begin{itemize}
\item {Grp. gram.:m.}
\end{itemize}
Escôva grande.
\section{Escovão}
\begin{itemize}
\item {Grp. gram.:m.}
\end{itemize}
\begin{itemize}
\item {Utilização:Ant.}
\end{itemize}
Parte de embarcação?:«\textunderscore ficámos metidos debaixo da gorja dos escovões da prôa.\textunderscore »\textunderscore Peregrinação\textunderscore , 36.
\section{Escovar}
\begin{itemize}
\item {Grp. gram.:v. t.}
\end{itemize}
\begin{itemize}
\item {Utilização:Fig.}
\end{itemize}
\begin{itemize}
\item {Proveniência:(De \textunderscore escôva\textunderscore )}
\end{itemize}
Limpar com escova: \textunderscore escovar o fato\textunderscore .
Bater.
Reprehender.
\section{Escoveiro}
\begin{itemize}
\item {Grp. gram.:m.}
\end{itemize}
Negociante de escôvas.
Lugar ou objecto, em que se guardam escôvas.
\section{Escovém}
\begin{itemize}
\item {Grp. gram.:m.}
\end{itemize}
Abertura, para a passagem da amarra, no costado do navio.
(Relaciona-se com o lat. \textunderscore excubiae\textunderscore ?)
\section{Escovilha}
\begin{itemize}
\item {Grp. gram.:f.}
\end{itemize}
\begin{itemize}
\item {Proveniência:(De \textunderscore escôva\textunderscore )}
\end{itemize}
Resíduos metállicos da laboração do oiro e da prata.
Acto de escovilhar.
\section{Escovilhão}
\begin{itemize}
\item {Grp. gram.:m.}
\end{itemize}
\begin{itemize}
\item {Proveniência:(De \textunderscore escovilha\textunderscore )}
\end{itemize}
Grande escôva, em fórma de cylindro, para limpar as bocas dos canhões.
\section{Escovilhar}
\begin{itemize}
\item {Grp. gram.:v. t.}
\end{itemize}
\begin{itemize}
\item {Proveniência:(De \textunderscore escovilha\textunderscore )}
\end{itemize}
Limpar de matérias estranhas (oiro ou prata).
\section{Escovilheiro}
\begin{itemize}
\item {Grp. gram.:m.}
\end{itemize}
Aquelle que aproveita a escovilha nas officinas de ourives.
\section{Escovinha}
\begin{itemize}
\item {Grp. gram.:f.}
\end{itemize}
\begin{itemize}
\item {Grp. gram.:Loc. adv.}
\end{itemize}
\begin{itemize}
\item {Proveniência:(De \textunderscore escôva\textunderscore )}
\end{itemize}
Planta, que nasce nas searas e dá flôres azues.
Passo de dança, no fado.
\textunderscore Á escovinha\textunderscore , muito rente: \textunderscore cortar o cabello á escovinha\textunderscore .
\textunderscore Fazer escovinhas\textunderscore , lisonjear.
Desfructar.
Pôr-se a dançar deante de alguém, dispondo-se a atacá-lo de súbito.
\section{Escozer}
\begin{itemize}
\item {Grp. gram.:v. t.}
\end{itemize}
\begin{itemize}
\item {Utilização:Ant.}
\end{itemize}
Maltratar ou ferir com arma branca.
Traspassar, varar:«\textunderscore ...os cornos que me escuzerão\textunderscore ». \textunderscore Aulegrafia\textunderscore , 141.
(Cast. \textunderscore escocer\textunderscore )
\section{Escrancha}
\begin{itemize}
\item {Grp. gram.:f.}
\end{itemize}
\begin{itemize}
\item {Utilização:Prov.}
\end{itemize}
\begin{itemize}
\item {Utilização:trasm.}
\end{itemize}
O mesmo que \textunderscore escarrancha\textunderscore .
\section{Escrânquia}
\begin{itemize}
\item {Grp. gram.:f.}
\end{itemize}
Gênero de plantas leguminosas.
\section{Escrapanento}
\begin{itemize}
\item {Grp. gram.:adj.}
\end{itemize}
\begin{itemize}
\item {Utilização:Prov.}
\end{itemize}
\begin{itemize}
\item {Utilização:alg.}
\end{itemize}
O mesmo que \textunderscore escrapanoso\textunderscore .
\section{Escrapanoso}
\begin{itemize}
\item {Grp. gram.:adj.}
\end{itemize}
\begin{itemize}
\item {Utilização:Prov.}
\end{itemize}
\begin{itemize}
\item {Utilização:alg.}
\end{itemize}
Agreste, áspero.
(Talvez por \textunderscore escrabanoso\textunderscore , metáth. de \textunderscore escabranoso\textunderscore , de \textunderscore escabroso\textunderscore )
\section{Escrapiada}
\begin{itemize}
\item {Grp. gram.:f.}
\end{itemize}
\begin{itemize}
\item {Utilização:T. da Madeira}
\end{itemize}
Bolo de milho, muito chato, cozido sôbre pedras quentes.
\section{Escrava}
(\textunderscore fem.\textunderscore  de \textunderscore escravo\textunderscore )
\section{Escravagem}
\begin{itemize}
\item {Grp. gram.:f.}
\end{itemize}
(V.escravatura)
\section{Escravaria}
Grande porção de escravos; escravatura.
\section{Escravatura}
\begin{itemize}
\item {Grp. gram.:f.}
\end{itemize}
\begin{itemize}
\item {Proveniência:(De \textunderscore escravo\textunderscore )}
\end{itemize}
Escravidão; commércio de escravos.
\section{Escravelhar}
\begin{itemize}
\item {Grp. gram.:v. i.}
\end{itemize}
O mesmo que \textunderscore escaravelhar\textunderscore .
\section{Escravidão}
\begin{itemize}
\item {Grp. gram.:f.}
\end{itemize}
\begin{itemize}
\item {Utilização:Fig.}
\end{itemize}
Estado de quem é escravo.
Cativeiro.
Sujeição; falta de liberdade: \textunderscore as criadas dizem que vivem na escravidão\textunderscore .
\section{Escravista}
\begin{itemize}
\item {Grp. gram.:adj.}
\end{itemize}
\begin{itemize}
\item {Grp. gram.:M.  e  adj.}
\end{itemize}
\begin{itemize}
\item {Proveniência:(De \textunderscore escravo\textunderscore )}
\end{itemize}
Que diz respeito a escravos.
Afeiçoado á escravatura.
Partidário da escravidão ou da escravatura. Cf. Camillo, \textunderscore Perfil do Marquês\textunderscore , 284.
\section{Escravização}
\begin{itemize}
\item {Grp. gram.:f.}
\end{itemize}
Acto de escravizar.
\section{Escravizar}
\begin{itemize}
\item {Grp. gram.:v. t.}
\end{itemize}
\begin{itemize}
\item {Utilização:Fig.}
\end{itemize}
Tornar escravo.
Sujeitar, subjugar.
Encantar, enlevar.
\section{Escravo}
\begin{itemize}
\item {Grp. gram.:m.  e  adj.}
\end{itemize}
\begin{itemize}
\item {Proveniência:(Do b. lat. \textunderscore sclavus\textunderscore  = \textunderscore slavus\textunderscore )}
\end{itemize}
O que vive em absoluta sujeição a outrem.
Cativo.
Aquelle que está dominado por uma paixão ou por qualquer fôrça moral: \textunderscore sou escravo dos meus deveres\textunderscore .
\section{Escravoneta}
\begin{itemize}
\item {fónica:nê}
\end{itemize}
\begin{itemize}
\item {Grp. gram.:f.}
\end{itemize}
\begin{itemize}
\item {Utilização:Ant.}
\end{itemize}
Espécie do rubi.
(Refl. do \textunderscore carbúnculo\textunderscore )
\section{Escreder}
\begin{itemize}
\item {Grp. gram.:v. t.  e  i.}
\end{itemize}
\begin{itemize}
\item {Utilização:T. de Rioda , na Beira}
\end{itemize}
\begin{itemize}
\item {Utilização:des.}
\end{itemize}
O mesmo que \textunderscore escrever\textunderscore .
\section{Escrepever}
\begin{itemize}
\item {Grp. gram.:v. t.}
\end{itemize}
\begin{itemize}
\item {Utilização:Ant.}
\end{itemize}
O mesmo que \textunderscore escrever\textunderscore .
\section{Escrevaninha}
\begin{itemize}
\item {Grp. gram.:f.}
\end{itemize}
O mesmo que \textunderscore escrivaninha\textunderscore .
\section{Escrevedeira}
\begin{itemize}
\item {Grp. gram.:f.}
\end{itemize}
O mesmo que \textunderscore cicia\textunderscore .
\section{Escrevedor}
\begin{itemize}
\item {Grp. gram.:m.  e  adj.}
\end{itemize}
\begin{itemize}
\item {Utilização:Fam.}
\end{itemize}
\begin{itemize}
\item {Proveniência:(De \textunderscore escrever\textunderscore )}
\end{itemize}
O que escreve.
Escritor sem merecimento literário.
\section{Escrevedura}
\begin{itemize}
\item {Grp. gram.:f.}
\end{itemize}
\begin{itemize}
\item {Utilização:Fam.}
\end{itemize}
\begin{itemize}
\item {Proveniência:(De \textunderscore escrever\textunderscore )}
\end{itemize}
Escrita.
\section{Escrevente}
\begin{itemize}
\item {Grp. gram.:m.}
\end{itemize}
\begin{itemize}
\item {Proveniência:(Do lat. \textunderscore scribens\textunderscore )}
\end{itemize}
Aquelle que tem por cargo copiar o que outrem escreve, ou escrever o que outrem dita.
\section{Escrever}
\begin{itemize}
\item {Grp. gram.:v. t.}
\end{itemize}
\begin{itemize}
\item {Utilização:Fig.}
\end{itemize}
\begin{itemize}
\item {Grp. gram.:V. i.}
\end{itemize}
\begin{itemize}
\item {Proveniência:(Lat. \textunderscore scribere\textunderscore )}
\end{itemize}
Representar, por meio de caracteres ou sinaes gráphicos: \textunderscore escrever um pensamento\textunderscore .
Redigir, compor (trabalho literário ou scientífico): \textunderscore escrever um compêndio\textunderscore .
Fixar.
Dirigir carta (a alguém): \textunderscore já escrevi ao Reinaldo\textunderscore .
\section{Escrevinhadeiro}
\begin{itemize}
\item {Grp. gram.:m.  e  adj.}
\end{itemize}
O mesmo que \textunderscore escrevinhador\textunderscore .
\section{Escrevinhador}
\begin{itemize}
\item {Grp. gram.:m.  e  adj.}
\end{itemize}
O que escrevinha.
Rabiscador.
Mau escritor.
\section{Escrevinhadura}
\begin{itemize}
\item {Grp. gram.:f.}
\end{itemize}
Acto de escrevinhar. Cf. Garrett, \textunderscore Helena\textunderscore , XVIII.
\section{Escrevinhante}
\begin{itemize}
\item {Grp. gram.:m.  e  adj.}
\end{itemize}
O mesmo que \textunderscore escrevinhador\textunderscore .
\section{Escrevinhar}
\begin{itemize}
\item {Grp. gram.:v. t.}
\end{itemize}
\begin{itemize}
\item {Grp. gram.:V. i.}
\end{itemize}
\begin{itemize}
\item {Proveniência:(De \textunderscore escrever\textunderscore )}
\end{itemize}
Escrever mal; rabiscar.
Escrever coisas fúteis, sem mérito algum.
\section{Escriba}
\begin{itemize}
\item {Grp. gram.:m.}
\end{itemize}
\begin{itemize}
\item {Utilização:Ant.}
\end{itemize}
\begin{itemize}
\item {Utilização:Fam.}
\end{itemize}
\begin{itemize}
\item {Proveniência:(Lat. \textunderscore scriba\textunderscore )}
\end{itemize}
Doutor que, entre os Judeus, lia o interpretava as leis.
Escrevedor, rabiscador, mau escritor.
\section{Escrínalo}
\begin{itemize}
\item {Grp. gram.:m.}
\end{itemize}
O mesmo que \textunderscore escrino\textunderscore .
\section{Escrinha}
\begin{itemize}
\item {Grp. gram.:f.}
\end{itemize}
\begin{itemize}
\item {Utilização:Prov.}
\end{itemize}
\begin{itemize}
\item {Utilização:trasm.}
\end{itemize}
Pequenho escrinho.
\section{Escrinho}
\begin{itemize}
\item {Grp. gram.:m.}
\end{itemize}
\begin{itemize}
\item {Utilização:Prov.}
\end{itemize}
\begin{itemize}
\item {Utilização:trasm.}
\end{itemize}
Espécie de cesto, em que se guarda o pão cozido.
Cesto, em que se leveda o pão.
(Provavelmente, alter. de \textunderscore escrínio\textunderscore )
\section{Escrínio}
\begin{itemize}
\item {Grp. gram.:m.}
\end{itemize}
\begin{itemize}
\item {Proveniência:(Lat. \textunderscore scrinium\textunderscore )}
\end{itemize}
Escrivaninha.
Pequeno armário ou cofre.
Guarda-jóias.
\section{Escrino}
\begin{itemize}
\item {Grp. gram.:m.}
\end{itemize}
Baga ou drupa sêca.
\section{Escrita}
\begin{itemize}
\item {Grp. gram.:f.}
\end{itemize}
\begin{itemize}
\item {Proveniência:(De \textunderscore escrito\textunderscore )}
\end{itemize}
Aquillo que se escreve.
Calligraphia.
Arte de escrever: \textunderscore não é fácil a escrita portuguesa\textunderscore .
\section{Escrito}
\begin{itemize}
\item {Grp. gram.:m.}
\end{itemize}
\begin{itemize}
\item {Proveniência:(Lat. \textunderscore scriptus\textunderscore )}
\end{itemize}
Escrita.
Bilhete.
Título.
Composição literária.
Pedaço de papel branco, collado em portas ou janelas, para indicar que o respectivo prédio se arrenda.
\section{Escritor}
\begin{itemize}
\item {Grp. gram.:m.}
\end{itemize}
\begin{itemize}
\item {Proveniência:(Lat. \textunderscore scriptor\textunderscore )}
\end{itemize}
Autor de composições literárias ou scientíficas.
\section{Escritora}
(\textunderscore fem.\textunderscore  de \textunderscore escritor\textunderscore )
\section{Escritorinho}
\begin{itemize}
\item {Grp. gram.:m.}
\end{itemize}
\begin{itemize}
\item {Utilização:Des.}
\end{itemize}
\begin{itemize}
\item {Proveniência:(De \textunderscore escritório\textunderscore )}
\end{itemize}
Escrivaninha.
\section{Escritório}
\begin{itemize}
\item {Grp. gram.:m.}
\end{itemize}
\begin{itemize}
\item {Utilização:Ant.}
\end{itemize}
\begin{itemize}
\item {Proveniência:(Lat. \textunderscore scriptorium\textunderscore )}
\end{itemize}
Compartimento ou casa, em que se escreve.
Escrivaninha, comprehendendo só o tinteiro e o areeiro. Cf. \textunderscore Peregrinação\textunderscore , CIII.
Escrivaninha antiga, com escaninhos e tampa inclinada:«\textunderscore ...dois escritórios de charão azul\textunderscore ». Camillo, \textunderscore Caveira\textunderscore , 453.
\section{Escritura}
\begin{itemize}
\item {Grp. gram.:f.}
\end{itemize}
\begin{itemize}
\item {Proveniência:(Lat. \textunderscore scriptura\textunderscore )}
\end{itemize}
Escrita.
Calligraphia.
Documento authêntico, feito por official público.
Por excellência, o conjunto dos livros do \textunderscore Antigo e Novo Testamento\textunderscore .
\section{Escrituração}
\begin{itemize}
\item {Grp. gram.:f.}
\end{itemize}
\begin{itemize}
\item {Proveniência:(De \textunderscore escriturar\textunderscore )}
\end{itemize}
Acto de escriturar.
Escrita dos livros commerciaes; arte de os escriturar.
Escrita methódica das contas de uma casa commercial.
\section{Escriturar}
\begin{itemize}
\item {Grp. gram.:v. t.}
\end{itemize}
\begin{itemize}
\item {Proveniência:(De \textunderscore escritura\textunderscore )}
\end{itemize}
Registar methodicamente (o movimento do uma casa commercial ou de uma empresa industrial, os documentos de uma repartição pública, etc.).
Contratar por escrito (serviços de cantores, actores, acrobatas, etc.).
\section{Escriturário}
\begin{itemize}
\item {Grp. gram.:m.}
\end{itemize}
\begin{itemize}
\item {Grp. gram.:M.  e  adj.}
\end{itemize}
\begin{itemize}
\item {Proveniência:(De \textunderscore escriturar\textunderscore )}
\end{itemize}
Aquelle que faz escrituração.
Escrevente.
Categoria inferior de empregados de fazenda.
O que é sabedor ou muito lido na Escritura Sagrada. Cf. Pant. de Aveiro, \textunderscore Itiner.\textunderscore , 239 v.^o, 3.^a ed.
\section{Escrivan}
\begin{itemize}
\item {Grp. gram.:f.}
\end{itemize}
Freira, que fazia escrituração nos conventos.
(Fem. de \textunderscore escrivão\textunderscore )
\section{Escrivania}
\begin{itemize}
\item {Grp. gram.:f.}
\end{itemize}
Cargo de escrivão.
\section{Escrivaninha}
\begin{itemize}
\item {Grp. gram.:f.}
\end{itemize}
\begin{itemize}
\item {Proveniência:(Do rad. de \textunderscore escrivão\textunderscore )}
\end{itemize}
Espécie de caixa, que contém os utensílios necessários para a escritura.
Tinteiro que, além do reservatório da tinta, tem lugar para pennas, para areia, etc.
Mesa, em que se escreve; secretária.
\section{Escrivão}
\begin{itemize}
\item {Grp. gram.:m.}
\end{itemize}
\begin{itemize}
\item {Utilização:Des.}
\end{itemize}
\begin{itemize}
\item {Grp. gram.:Loc.}
\end{itemize}
\begin{itemize}
\item {Utilização:pop.}
\end{itemize}
\begin{itemize}
\item {Proveniência:(Do b. lat. \textunderscore scribanus\textunderscore ?)}
\end{itemize}
Official público, que, junto de uma autoridade, corporações ou tribunaes, escreve autos, termos de processo, actas, e outros documentos legaes.
Escrevente, escriturário.
\textunderscore Escrivão da penna grande\textunderscore , varredor das ruas.
\section{Escrivar}
\begin{itemize}
\item {Grp. gram.:v. t.}
\end{itemize}
\begin{itemize}
\item {Utilização:Prov.}
\end{itemize}
\begin{itemize}
\item {Utilização:minh.}
\end{itemize}
\begin{itemize}
\item {Proveniência:(De \textunderscore crivo\textunderscore )}
\end{itemize}
O mesmo que \textunderscore joeirar\textunderscore .
\section{Escrivedeira}
\begin{itemize}
\item {Grp. gram.:f.}
\end{itemize}
O mesmo que \textunderscore escrevedeira\textunderscore .
\section{Escrobiculado}
\begin{itemize}
\item {Grp. gram.:adj.}
\end{itemize}
Que tem escrobiculos.
\section{Escrobículo}
\begin{itemize}
\item {Grp. gram.:m.}
\end{itemize}
\begin{itemize}
\item {Proveniência:(Lat. \textunderscore scrobiculus\textunderscore )}
\end{itemize}
Pequena cavidade.
Depressão na parte anterior do peito.
\section{Escrobiculoso}
\begin{itemize}
\item {Grp. gram.:adj.}
\end{itemize}
O mesmo que \textunderscore escrobiculado\textunderscore .
\section{Escrófula}
\begin{itemize}
\item {Grp. gram.:f.}
\end{itemize}
\begin{itemize}
\item {Utilização:Med.}
\end{itemize}
\begin{itemize}
\item {Proveniência:(Lat. \textunderscore scrofulae\textunderscore )}
\end{itemize}
Doença, que se manifesta por engorgitamento dos glânglios lympháticos, alterando-se os fluidos que estes contêm, e formando-se tumores ovulares, que podem ulcerar-se.
\section{Escrofulária}
\begin{itemize}
\item {Grp. gram.:f.}
\end{itemize}
\begin{itemize}
\item {Proveniência:(De \textunderscore escrófula\textunderscore )}
\end{itemize}
Planta medicinal, (\textunderscore scrofularia aquatica\textunderscore ), que se receitava contra as escrófulas.
\section{Escrofulariáceaes}
\begin{itemize}
\item {Grp. gram.:f. pl.}
\end{itemize}
O mesmo ou melhor que \textunderscore escrofularíneas\textunderscore .
\section{Escrofularíneas}
\begin{itemize}
\item {Grp. gram.:f. pl.}
\end{itemize}
Família de plantas, que têm por typo a escrofulária.
\section{Escrofulismo}
\begin{itemize}
\item {Grp. gram.:m.}
\end{itemize}
(V.escrofulose)
\section{Escrofulose}
\begin{itemize}
\item {Grp. gram.:f.}
\end{itemize}
Doença dos que soffrem escrófulas.
\section{Escrofuloso}
\begin{itemize}
\item {Grp. gram.:adj.}
\end{itemize}
\begin{itemize}
\item {Grp. gram.:M.}
\end{itemize}
Relativo a escrófulas.
Que padece escrófulas.
Aquelle que tem escrófulas.
\section{Escrópulo}
\begin{itemize}
\item {Grp. gram.:m.}
\end{itemize}
\begin{itemize}
\item {Proveniência:(Lat. \textunderscore scrupulus\textunderscore )}
\end{itemize}
Pêso antigo, equivalente á têrça parte da oitava.
\section{Escrotal}
\begin{itemize}
\item {Grp. gram.:adj.}
\end{itemize}
Relativo ao escroto.
\section{Escroto}
\begin{itemize}
\item {fónica:crô}
\end{itemize}
\begin{itemize}
\item {Grp. gram.:m.}
\end{itemize}
\begin{itemize}
\item {Proveniência:(Lat. \textunderscore scrotum\textunderscore )}
\end{itemize}
Pelle, que envolve ambos os testículos.
\section{Escrotocele}
\begin{itemize}
\item {Grp. gram.:f.}
\end{itemize}
\begin{itemize}
\item {Proveniência:(De \textunderscore escroto\textunderscore  + gr. \textunderscore kele\textunderscore )}
\end{itemize}
Hérnia no escroto.
\section{Escrupularia}
\begin{itemize}
\item {Grp. gram.:f.}
\end{itemize}
Abundância de escrúpulos.
Escrúpulos desmedidos.
\section{Escrupulear}
\begin{itemize}
\item {Grp. gram.:v. i.}
\end{itemize}
O mesmo que \textunderscore escrupulizar\textunderscore :«\textunderscore não escrupuleava de entrar nas sinagogas\textunderscore ». Camillo, \textunderscore Judeu\textunderscore , I, 171.
\section{Escrupulizador}
\begin{itemize}
\item {Grp. gram.:adj.}
\end{itemize}
Que escrupuliza. Cf. Herculano, \textunderscore Opúsc.\textunderscore , III, 81.
\section{Escrupulizar}
\begin{itemize}
\item {Grp. gram.:v. i.}
\end{itemize}
\begin{itemize}
\item {Grp. gram.:V. t.}
\end{itemize}
Têr ou fazer escrúpulo.
Causar escrúpulos a.
\section{Escrúpulo}
\begin{itemize}
\item {Grp. gram.:m.}
\end{itemize}
\begin{itemize}
\item {Proveniência:(Lat. \textunderscore scrupulum\textunderscore )}
\end{itemize}
Hesitação sôbre a bondade ou ruindade de um acto.
Temor de consciência.
Meticulosidade.
Zêlo, muita attenção: \textunderscore veja isso com escrúpulo\textunderscore .
\section{Escrupulosamente}
\begin{itemize}
\item {Grp. gram.:adv.}
\end{itemize}
\begin{itemize}
\item {Proveniência:(De \textunderscore escrupuloso\textunderscore )}
\end{itemize}
Com escrúpulo.
\section{Escrupulosidade}
\begin{itemize}
\item {Grp. gram.:f.}
\end{itemize}
Qualidade de quem é escrupuloso.
\section{Escrupuloso}
\begin{itemize}
\item {Grp. gram.:adj.}
\end{itemize}
\begin{itemize}
\item {Proveniência:(De \textunderscore escrúpulo\textunderscore )}
\end{itemize}
Que tem escrúpulo.
Meticuloso.
Hesitante.
Cuidadoso; pontual.
\section{Escrutador}
\begin{itemize}
\item {Grp. gram.:m.  e  adj.}
\end{itemize}
O que escruta.
\section{Escrutar}
\begin{itemize}
\item {Grp. gram.:v. t.}
\end{itemize}
\begin{itemize}
\item {Proveniência:(Lat. \textunderscore scrutari\textunderscore )}
\end{itemize}
O mesmo que \textunderscore perscrutar\textunderscore .
\section{Escrutável}
\begin{itemize}
\item {Grp. gram.:adj.}
\end{itemize}
\begin{itemize}
\item {Proveniência:(De \textunderscore escrutar\textunderscore )}
\end{itemize}
Que póde sêr investigado, esquadrinhado.
\section{Escrutinação}
\begin{itemize}
\item {Grp. gram.:f.}
\end{itemize}
\begin{itemize}
\item {Proveniência:(Lat. \textunderscore scrutinatio\textunderscore )}
\end{itemize}
O mesmo que \textunderscore escrutínio\textunderscore .
\section{Escrutinador}
\begin{itemize}
\item {Grp. gram.:m.}
\end{itemize}
Aquelle que escrutina.
\section{Escrutinar}
\begin{itemize}
\item {Grp. gram.:v. i.}
\end{itemize}
\begin{itemize}
\item {Grp. gram.:V. t.}
\end{itemize}
\begin{itemize}
\item {Proveniência:(De \textunderscore escrutínio\textunderscore )}
\end{itemize}
Verificar a entrada dos votos na urna, e cotejá-la com as descargas dos votantes, feitas nos respectivos cadernos.
Contar e cotejar com as descargas dos cadernos (as listas eleitoraes)
\section{Escrutínio}
\begin{itemize}
\item {Grp. gram.:m.}
\end{itemize}
\begin{itemize}
\item {Proveniência:(Do lat. \textunderscore scrutínium\textunderscore )}
\end{itemize}
Votação por listas, ou por outros objectos convencionaes, que se lançam numa urna, ou num recipiente que a substitua.
Investigação ou contagem dos votos que entraram na urna.
\section{Escubértia}
\begin{itemize}
\item {Grp. gram.:f.}
\end{itemize}
Gênero de plantas asclepiádeas.
\section{Escubléria}
\begin{itemize}
\item {Grp. gram.:f.}
\end{itemize}
Gênero de plantas gencianáceas.
\section{Escucir-se}
\begin{itemize}
\item {Grp. gram.:v. p.}
\end{itemize}
\begin{itemize}
\item {Utilização:Prov.}
\end{itemize}
\begin{itemize}
\item {Utilização:trasm.}
\end{itemize}
Esgueirar-se; safar-se.
Retirar-se sorrateiramente.
\section{Escudar}
\begin{itemize}
\item {Grp. gram.:v. t.}
\end{itemize}
Defender com escudo.
Proteger; amparar.
\section{Escudeira}
\begin{itemize}
\item {Grp. gram.:f.}
\end{itemize}
\begin{itemize}
\item {Utilização:Prov.}
\end{itemize}
\textunderscore Andar á escudeira\textunderscore , andar á gandaia, vadiar.
(Colhido em Mourão)
\section{Escudeirar}
\begin{itemize}
\item {Grp. gram.:v. t.}
\end{itemize}
\begin{itemize}
\item {Utilização:Des.}
\end{itemize}
\begin{itemize}
\item {Grp. gram.:V. i.}
\end{itemize}
Acompanhar como escudeiro.
Fazer mesuras; desfazer-se em cortesias. Cf. \textunderscore Anat. Joc.\textunderscore , I, 6.
\section{Escudeirático}
\begin{itemize}
\item {Grp. gram.:adj.}
\end{itemize}
Próprio de escudeiro.
\section{Escudeirice}
\begin{itemize}
\item {Grp. gram.:f.}
\end{itemize}
Acção ou maneiras próprias de escudeiro.
\section{Escudeiril}
\begin{itemize}
\item {Grp. gram.:adj.}
\end{itemize}
O mesmo que \textunderscore escudeirático\textunderscore .
\section{Escudeiro}
\begin{itemize}
\item {Grp. gram.:m.}
\end{itemize}
\begin{itemize}
\item {Utilização:Pop.}
\end{itemize}
\begin{itemize}
\item {Proveniência:(Do lat. \textunderscore scutarius\textunderscore )}
\end{itemize}
Criado, que acompanhava o cavalleiro, levando-lhe o escudo, até o momento do combate.
Indivíduo, armado de lança e escudo, e que acompanhava os imperadores á guerra.
Criado particular.
Título honorífico de alguns funccionários da extincta côrte.
\section{Escudela}
\begin{itemize}
\item {Grp. gram.:f.}
\end{itemize}
\begin{itemize}
\item {Proveniência:(Do lat. \textunderscore scutella\textunderscore )}
\end{itemize}
Malga de madeira.
\section{Escudelada}
\begin{itemize}
\item {Grp. gram.:f.}
\end{itemize}
Conteúdo de uma escudela.
Aquilo que cabe numa escudela.
\section{Escudelar}
\begin{itemize}
\item {Grp. gram.:v. t.}
\end{itemize}
Distribuir por escudelas; dar em escudelas.
\section{Escudella}
\begin{itemize}
\item {Grp. gram.:f.}
\end{itemize}
\begin{itemize}
\item {Proveniência:(Do lat. \textunderscore scutella\textunderscore )}
\end{itemize}
Malga de madeira.
\section{Escudellada}
\begin{itemize}
\item {Grp. gram.:f.}
\end{itemize}
Conteúdo de uma escudella.
Aquillo que cabe numa escudella.
\section{Escudellar}
\begin{itemize}
\item {Grp. gram.:v. t.}
\end{itemize}
Distribuir por escudellas; dar em escudellas.
\section{Escudete}
\begin{itemize}
\item {fónica:dê}
\end{itemize}
\begin{itemize}
\item {Grp. gram.:m.}
\end{itemize}
\begin{itemize}
\item {Utilização:Heráld.}
\end{itemize}
\begin{itemize}
\item {Proveniência:(De \textunderscore escudo\textunderscore )}
\end{itemize}
Escudo pequeno.
Peça exterior da fechadura, por onde entra a chave, e em que ás vezes se fixam argolas.
Enxêrto de borbulha.
Escamas nos tarsos de algumas aves.
Pequeno escudo, no centro de outro e com a mesma fórma dêste.
\section{Escudilho}
\begin{itemize}
\item {Grp. gram.:m.}
\end{itemize}
\begin{itemize}
\item {Proveniência:(De \textunderscore escudo\textunderscore )}
\end{itemize}
Receptáculo redondo nos troncos e frondes dos líchens.
Tubérculo, entre as ligações das asas dos insectos.
\section{Escudilhoso}
\begin{itemize}
\item {Grp. gram.:adj.}
\end{itemize}
Que tem escudilhos.
\section{Escudo}
\begin{itemize}
\item {Grp. gram.:m.}
\end{itemize}
\begin{itemize}
\item {Utilização:Fig.}
\end{itemize}
\begin{itemize}
\item {Proveniência:(Lat. \textunderscore scutum\textunderscore )}
\end{itemize}
Peça de armadura antiga, que resguardava o corpo do guerreiro contra os golpes de lança ou de espada.
Peça, em que se representam ou se gravam as figuras das armas nobiliárias ou das armas peculiares a cada nação.
Antiga moéda de oiro portuguesa, do valor de 90 reis.
Moderna moéda portuguesa, correspondente a 1$000 reis.
Nome de moéda em varios outros países.
Prato de balança.
Borbulha de árvore para enxêrto.
Amparo, defesa.
\section{Escudo-de-brabante}
\begin{itemize}
\item {Grp. gram.:m.}
\end{itemize}
Moéda austro-húngara, de prata.
\section{Escudrinhar}
\begin{itemize}
\item {Grp. gram.:v. t.}
\end{itemize}
O mesmo que \textunderscore esquadrinhar\textunderscore . Cf. Camillo, \textunderscore Bruxa\textunderscore , 68.
\section{Escuitar}
\begin{itemize}
\item {Grp. gram.:v. t.}
\end{itemize}
\begin{itemize}
\item {Utilização:Ant.}
\end{itemize}
(V.escutar)
\section{Esculáceas}
\begin{itemize}
\item {Grp. gram.:f.}
\end{itemize}
\begin{itemize}
\item {Proveniência:(Do lat. \textunderscore aesculus\textunderscore )}
\end{itemize}
O mesmo que \textunderscore cupuláceas\textunderscore .
\section{Esculápias}
\begin{itemize}
\item {Grp. gram.:f. pl.}
\end{itemize}
Festas pagans, em honra de Esculápio.
\section{Esculápio}
\begin{itemize}
\item {Grp. gram.:m.}
\end{itemize}
\begin{itemize}
\item {Utilização:Pop.}
\end{itemize}
\begin{itemize}
\item {Proveniência:(De \textunderscore Esculápio\textunderscore , n. p.)}
\end{itemize}
Médico.
\section{Esculato}
\begin{itemize}
\item {Grp. gram.:m.}
\end{itemize}
Sal, produzido pela combinação do ácido escúlico com uma base.
\section{Esculca}
\begin{itemize}
\item {Grp. gram.:m.}
\end{itemize}
Sentinela ou guarda avançada, nos exercitos antigos.
(Do germ?)
\section{Esculento}
\begin{itemize}
\item {Grp. gram.:adj.}
\end{itemize}
\begin{itemize}
\item {Proveniência:(Lat. \textunderscore esculentus\textunderscore )}
\end{itemize}
Alimenticio; que serve de alimento; que alimenta.
\section{Escúlico}
\begin{itemize}
\item {Grp. gram.:adj.}
\end{itemize}
Diz-se de um ácido, extrahido do castanheiro-da-índia.
\section{Esculina}
\begin{itemize}
\item {Grp. gram.:f.}
\end{itemize}
Substância particular, que se encontra nos frutos do castanheiro-da-índia.
\section{Ésculo}
\begin{itemize}
\item {Grp. gram.:m.}
\end{itemize}
\begin{itemize}
\item {Proveniência:(Lat. \textunderscore aesculus\textunderscore )}
\end{itemize}
Espécie de carvalho.
\section{Esculpidor}
\begin{itemize}
\item {Grp. gram.:m.}
\end{itemize}
\begin{itemize}
\item {Utilização:Des.}
\end{itemize}
\begin{itemize}
\item {Proveniência:(De \textunderscore esculpir\textunderscore )}
\end{itemize}
O mesmo que \textunderscore escultor\textunderscore .
\section{Esculpir}
\begin{itemize}
\item {Grp. gram.:v. t.}
\end{itemize}
\begin{itemize}
\item {Utilização:Fig.}
\end{itemize}
\begin{itemize}
\item {Proveniência:(Lat. \textunderscore sculpere\textunderscore )}
\end{itemize}
Lavrar com escopro em pedra, madeira, ou noutra matéria dura.
Entalhar.
Gravar: \textunderscore esculpir uma inscripção\textunderscore .
Modelar em cera ou argila a representação de.
Deixar impresso, gravar.
\section{Esculptar}
\begin{itemize}
\item {Grp. gram.:v. t.}
\end{itemize}
\begin{itemize}
\item {Utilização:Des.}
\end{itemize}
O mesmo que \textunderscore esculturar\textunderscore . Cf. Filinto, XIV, 125.
\section{Esculptor}
\begin{itemize}
\item {Grp. gram.:m.}
\end{itemize}
(V.escultor)
\section{Esculptura}
\textunderscore f.\textunderscore  (e der.)
(V. \textunderscore escultura\textunderscore , etc.)
\section{Escultor}
\begin{itemize}
\item {Grp. gram.:m.}
\end{itemize}
\begin{itemize}
\item {Proveniência:(Lat. \textunderscore sculptor\textunderscore )}
\end{itemize}
Aquelle que esculpe por offício.
Aquelle que faz escultura.
\section{Escultório}
\begin{itemize}
\item {Grp. gram.:adj.}
\end{itemize}
\begin{itemize}
\item {Utilização:Neol.}
\end{itemize}
O mesmo que \textunderscore escultural\textunderscore . Cf. Alves Mendes, \textunderscore Itália\textunderscore .
\section{Escultura}
\begin{itemize}
\item {Grp. gram.:f.}
\end{itemize}
\begin{itemize}
\item {Proveniência:(Lat. \textunderscore sculptura\textunderscore )}
\end{itemize}
Arte de esculpir.
Obra de escultor: \textunderscore comprar uma escultura\textunderscore .
Estatuária.
\section{Escultural}
\begin{itemize}
\item {Grp. gram.:adj.}
\end{itemize}
Relativo á escultura.
Que tem fórmas correctas, podendo servir de modêlo em estatuária.
\section{Esculturar}
\begin{itemize}
\item {Grp. gram.:v. t.}
\end{itemize}
\begin{itemize}
\item {Grp. gram.:V. i.}
\end{itemize}
Fazer a escultura de: \textunderscore esculturar um herói\textunderscore .
Trabalhar em escultura.
\section{Escuma}
\begin{itemize}
\item {Grp. gram.:f.}
\end{itemize}
\begin{itemize}
\item {Utilização:Fig.}
\end{itemize}
\begin{itemize}
\item {Proveniência:(Do germ. \textunderscore skum\textunderscore )}
\end{itemize}
O mesmo que \textunderscore espuma\textunderscore .
Bolhas, cheias de ar ou gás na superfície de um líquido, que se agita ou ferve.
Bôrra ou escória, á superfície de um líquido.
Baba.
Bolhas esbranquiçadas, que o suor fórma na pelle dos cavallos.
Silicato de magnésia, de que se fazem boquilhas.
Ralé, escória social.
\section{Escumação}
\begin{itemize}
\item {Grp. gram.:f.}
\end{itemize}
Acto de escumar.
\section{Escumadeira}
\begin{itemize}
\item {Grp. gram.:f.}
\end{itemize}
\begin{itemize}
\item {Proveniência:(De \textunderscore escumar\textunderscore )}
\end{itemize}
Colhér de ralo, para tirar a escuma dos líquidos que fervem.
\section{Escumado}
\begin{itemize}
\item {Grp. gram.:m.}
\end{itemize}
\begin{itemize}
\item {Proveniência:(De \textunderscore escumar\textunderscore )}
\end{itemize}
Escuma.
\section{Escumador}
\begin{itemize}
\item {Grp. gram.:adj.}
\end{itemize}
O mesmo que \textunderscore escumoso\textunderscore .
\section{Escumalha}
\begin{itemize}
\item {Grp. gram.:f.}
\end{itemize}
\begin{itemize}
\item {Utilização:Fam.}
\end{itemize}
O mesmo que \textunderscore escumalho\textunderscore .
A escória social, a ralé.
\section{Escumalho}
\begin{itemize}
\item {Grp. gram.:m.}
\end{itemize}
\begin{itemize}
\item {Proveniência:(De \textunderscore escuma\textunderscore )}
\end{itemize}
Escória de metal em fusão.
\section{Escumante}
\begin{itemize}
\item {Grp. gram.:adj.}
\end{itemize}
Que fórma escuma, que deita escuma.
\section{Escumar}
\begin{itemize}
\item {Grp. gram.:v. t.}
\end{itemize}
\begin{itemize}
\item {Grp. gram.:V. i.}
\end{itemize}
\begin{itemize}
\item {Utilização:Fig.}
\end{itemize}
Tirar a escuma a.
Formar escuma.
Deitar escuma.
Excitar-se, ferver.
\section{Escumeada}
\begin{itemize}
\item {Grp. gram.:f.}
\end{itemize}
\begin{itemize}
\item {Utilização:Prov.}
\end{itemize}
\begin{itemize}
\item {Utilização:alg.}
\end{itemize}
Escarpa; ladeira.
Altura.
(Cp. \textunderscore cumeada\textunderscore )
\section{Escumeado}
\begin{itemize}
\item {Grp. gram.:adj.}
\end{itemize}
\begin{itemize}
\item {Utilização:Prov.}
\end{itemize}
\begin{itemize}
\item {Utilização:trasm.}
\end{itemize}
Debiqueiro, esquisito ou muito escrupuloso nas comidas.
Muito escrupuloso ou exigente na escolha da noiva.
\section{Escumilha}
\begin{itemize}
\item {Grp. gram.:f.}
\end{itemize}
\begin{itemize}
\item {Utilização:Bot.}
\end{itemize}
\begin{itemize}
\item {Proveniência:(De \textunderscore escuma\textunderscore )}
\end{itemize}
Pequenos grãos de chumbo, para a caça dos pássaros.
Tecido transparente de lan ou seda muito fina.
Designação vulgar do eupatório.
\section{Escumilhar}
\begin{itemize}
\item {Grp. gram.:v. i.}
\end{itemize}
\begin{itemize}
\item {Utilização:Bras}
\end{itemize}
Bordar sôbre escumilha. Cf. Alencar, \textunderscore Diva\textunderscore , (notas).
\section{Escumoso}
\begin{itemize}
\item {Grp. gram.:adj.}
\end{itemize}
Que deita escuma.
Que tem escuma.
\section{Escuna}
\begin{itemize}
\item {Grp. gram.:f.}
\end{itemize}
Embarcação de dois mastros, com vêrgas no da prôa e sem mastaréu de joanete.
\section{Escupila}
\begin{itemize}
\item {Grp. gram.:f.}
\end{itemize}
Grande árvore medicinal da ilha de San-Thomé.
\section{Escupir}
\begin{itemize}
\item {Grp. gram.:v. i.}
\end{itemize}
\begin{itemize}
\item {Utilização:Chul.}
\end{itemize}
O mesmo que \textunderscore cuspir\textunderscore .
(Cp. ant. fr. \textunderscore escrupir\textunderscore , provn. \textunderscore scupir\textunderscore , cast. \textunderscore escupir\textunderscore )
\section{Escuramente}
\begin{itemize}
\item {Grp. gram.:adv.}
\end{itemize}
De modo escuro.
Occultamente.
\section{Escurana}
\begin{itemize}
\item {Grp. gram.:f.}
\end{itemize}
\begin{itemize}
\item {Utilização:Ant.}
\end{itemize}
O mesmo que \textunderscore escuridão\textunderscore .
\section{Escurão}
\begin{itemize}
\item {Grp. gram.:m.}
\end{itemize}
\begin{itemize}
\item {Utilização:Bras. de Minas}
\end{itemize}
\begin{itemize}
\item {Proveniência:(De \textunderscore escuro\textunderscore )}
\end{itemize}
O fim do crepúsculo: \textunderscore daqui a pouco é escurão\textunderscore .
\section{Escurar}
\begin{itemize}
\item {Grp. gram.:v. i.}
\end{itemize}
\begin{itemize}
\item {Utilização:Des.}
\end{itemize}
O mesmo que \textunderscore escurecer\textunderscore . Cf. G. Vicente, I, 347.
\section{Escuras}
\begin{itemize}
\item {Grp. gram.:f. pl. Loc. adv.}
\end{itemize}
\begin{itemize}
\item {Proveniência:(De \textunderscore escuro\textunderscore )}
\end{itemize}
\textunderscore Ás escuras\textunderscore , occultamente.
Sem luz.
Com ignorância.
\section{Escurecedor}
\begin{itemize}
\item {Grp. gram.:m.  e  adj.}
\end{itemize}
\begin{itemize}
\item {Proveniência:(De \textunderscore escurecer\textunderscore )}
\end{itemize}
Aquelle ou aquillo que torna escuro.
\section{Escurecer}
\begin{itemize}
\item {Grp. gram.:v. t.}
\end{itemize}
\begin{itemize}
\item {Utilização:Fig.}
\end{itemize}
\begin{itemize}
\item {Grp. gram.:V. i.}
\end{itemize}
\begin{itemize}
\item {Grp. gram.:V. t.  e  i.}
\end{itemize}
\begin{itemize}
\item {Utilização:Prov.}
\end{itemize}
\begin{itemize}
\item {Utilização:Açor}
\end{itemize}
\begin{itemize}
\item {Utilização:minh.}
\end{itemize}
\begin{itemize}
\item {Proveniência:(De \textunderscore escuro\textunderscore )}
\end{itemize}
Fazer escuro:«\textunderscore uma nuvem que os ares escurece\textunderscore ». \textunderscore Lusíadas\textunderscore .
Tornar diffícil, inintelligível.
Deslustrar.
Supplantar.
Offuscar: \textunderscore a paixão escurece a intelligência\textunderscore .
Ficar escuro.
Anoitecer.
O mesmo que \textunderscore esquecer\textunderscore : \textunderscore a perda de um filho amado nunca escurece\textunderscore .
\section{Escurecível}
\begin{itemize}
\item {Grp. gram.:adj.}
\end{itemize}
\begin{itemize}
\item {Utilização:Fig.}
\end{itemize}
\begin{itemize}
\item {Proveniência:(De \textunderscore escurecer\textunderscore )}
\end{itemize}
Que se deve occultar.
\section{Escurejar}
\begin{itemize}
\item {Grp. gram.:v. i.}
\end{itemize}
Pôr escuro; mostrar-se escuro:«\textunderscore Dahi para dentro, escureja o abysmo.\textunderscore »Camillo, \textunderscore Bohémia\textunderscore , 337.
\section{Escurejar}
\begin{itemize}
\item {Grp. gram.:v. i.}
\end{itemize}
\begin{itemize}
\item {Utilização:Prov.}
\end{itemize}
\begin{itemize}
\item {Utilização:beir.}
\end{itemize}
Diz-se da criança, que olha para o que outras comem, esperando ou desejando que se lhe dê também de comer.
\section{Escurentar}
\begin{itemize}
\item {Grp. gram.:v. t.}
\end{itemize}
\begin{itemize}
\item {Proveniência:(De \textunderscore escuro\textunderscore )}
\end{itemize}
O mesmo que \textunderscore escurecer\textunderscore .
\section{Escureza}
\begin{itemize}
\item {Grp. gram.:f.}
\end{itemize}
(V.escuridão). Cf. Camillo, \textunderscore Curso de Liter.\textunderscore , 262.
\section{Escurial}
\begin{itemize}
\item {Grp. gram.:f.}
\end{itemize}
Variedade de pêra muito apreciada.
\section{Escuridade}
\begin{itemize}
\item {Grp. gram.:f.}
\end{itemize}
\begin{itemize}
\item {Utilização:Fig.}
\end{itemize}
\begin{itemize}
\item {Proveniência:(Do lat. \textunderscore obscuritas\textunderscore )}
\end{itemize}
Qualidade daquillo que é escuro.
Falta de luz.
Difficuldade; mystério.
\section{Escuridão}
\begin{itemize}
\item {Grp. gram.:f.}
\end{itemize}
\begin{itemize}
\item {Utilização:Fig.}
\end{itemize}
\begin{itemize}
\item {Proveniência:(De \textunderscore escuro\textunderscore )}
\end{itemize}
Escuridade; trevas.
Cegueira.
Mágoa profunda.
\section{Escuridez}
\begin{itemize}
\item {Grp. gram.:f.}
\end{itemize}
O mesmo que \textunderscore escuridade\textunderscore . Cf. Filinto, \textunderscore D. Man.\textunderscore , I, 18.
\section{Escuro}
\begin{itemize}
\item {Grp. gram.:adj.}
\end{itemize}
\begin{itemize}
\item {Utilização:Fig.}
\end{itemize}
\begin{itemize}
\item {Grp. gram.:M.}
\end{itemize}
\begin{itemize}
\item {Proveniência:(Lat. \textunderscore obscurus\textunderscore )}
\end{itemize}
Em que não há luz: \textunderscore casa escura\textunderscore .
Quási negro: \textunderscore tecido escuro\textunderscore .
Intrincado, diffícil: \textunderscore problema escuro\textunderscore .
Mysterioso.
Triste.
Deslustrado.
Que não é nobre: \textunderscore homem de origem escura\textunderscore .
Que se ouve mal ou se distingue mal.
Escuridão: \textunderscore tactear no escuro\textunderscore .
Noite.
Recanto escuro.
\section{Escurra}
\begin{itemize}
\item {Grp. gram.:m.}
\end{itemize}
\begin{itemize}
\item {Utilização:Ant.}
\end{itemize}
\begin{itemize}
\item {Proveniência:(Do lat. \textunderscore scurra\textunderscore )}
\end{itemize}
Bobo.
Homem chocarreiro, desprezível.
\section{Escurril}
\begin{itemize}
\item {Grp. gram.:adj.}
\end{itemize}
\begin{itemize}
\item {Proveniência:(Lat. \textunderscore scurrilis\textunderscore )}
\end{itemize}
Próprio de escurra.
Ridículo.
Reles; torpe.
\section{Escurrilidade}
\begin{itemize}
\item {Grp. gram.:f.}
\end{itemize}
Qualidade de quem ou daquillo que é escurril.
\section{Escusa}
\begin{itemize}
\item {Grp. gram.:f.}
\end{itemize}
\begin{itemize}
\item {Proveniência:(De \textunderscore escusar\textunderscore )}
\end{itemize}
Desculpa.
Acto de escusar ou dispensar.
\section{Escusação}
\begin{itemize}
\item {Grp. gram.:f.}
\end{itemize}
Escusa.
\section{Escusadamente}
\begin{itemize}
\item {Grp. gram.:adv.}
\end{itemize}
\begin{itemize}
\item {Proveniência:(De \textunderscore escusado\textunderscore )}
\end{itemize}
Inutilmente.
\section{Escusado}
\begin{itemize}
\item {Grp. gram.:adj.}
\end{itemize}
\begin{itemize}
\item {Proveniência:(De \textunderscore escusar\textunderscore )}
\end{itemize}
Dispensável.
Desnecessário: \textunderscore sacrifício escusado\textunderscore .
\section{Escusador}
\begin{itemize}
\item {Grp. gram.:m.}
\end{itemize}
Aquelle que escusa.
\section{Escusa-galés}
\begin{itemize}
\item {Grp. gram.:f.}
\end{itemize}
\begin{itemize}
\item {Utilização:Ant.}
\end{itemize}
Espécie de embarcação.
\section{Escusamente}
\begin{itemize}
\item {Grp. gram.:adv.}
\end{itemize}
\begin{itemize}
\item {Proveniência:(De \textunderscore escuso\textunderscore )}
\end{itemize}
Em segrêdo.
Ás occultas.
\section{Escusa-merenda}
\begin{itemize}
\item {Grp. gram.:f.}
\end{itemize}
O mesmo que \textunderscore quitamerendas\textunderscore .
\section{Escusar}
\begin{itemize}
\item {Grp. gram.:v. t.}
\end{itemize}
\begin{itemize}
\item {Grp. gram.:V. i.}
\end{itemize}
\begin{itemize}
\item {Proveniência:(Lat. \textunderscore excusare\textunderscore )}
\end{itemize}
Desculpar.
Justificar.
Dispensar: \textunderscore escusar alguém de comparecer\textunderscore .
Não têr precisão de: \textunderscore escusamos chorar\textunderscore .
Tornar isento.
Evitar.
Não têr necessidade.
Não lhe sêr preciso: \textunderscore escuso de o aconselhar\textunderscore .
\section{Escusatória}
\begin{itemize}
\item {Grp. gram.:f.}
\end{itemize}
\begin{itemize}
\item {Utilização:Des.}
\end{itemize}
\begin{itemize}
\item {Proveniência:(De \textunderscore escusatório\textunderscore )}
\end{itemize}
Van allegação forense.
\section{Escusatório}
\begin{itemize}
\item {Grp. gram.:adj.}
\end{itemize}
\begin{itemize}
\item {Proveniência:(De \textunderscore escusar\textunderscore )}
\end{itemize}
Que serve para escusar ou desculpar.
\section{Escusável}
\begin{itemize}
\item {Grp. gram.:adj.}
\end{itemize}
\begin{itemize}
\item {Proveniência:(Lat. \textunderscore excusabilis\textunderscore )}
\end{itemize}
Que se póde escusar ou desculpar.
\section{Escuso}
\begin{itemize}
\item {Grp. gram.:adj.}
\end{itemize}
\begin{itemize}
\item {Proveniência:(De \textunderscore escusar\textunderscore )}
\end{itemize}
Que foi objecto de escusa.
\section{Escuso}
\begin{itemize}
\item {Grp. gram.:adj.}
\end{itemize}
O mesmo que \textunderscore esconso\textunderscore .
\section{Escuta}
\begin{itemize}
\item {Grp. gram.:f.}
\end{itemize}
\begin{itemize}
\item {Utilização:Ant.}
\end{itemize}
\begin{itemize}
\item {Grp. gram.:M.}
\end{itemize}
\begin{itemize}
\item {Utilização:Ant.}
\end{itemize}
Acto de escutar.
Pessôa, que escuta.
Lugar, em que se escuta.
O mesmo que \textunderscore esculca\textunderscore .
\section{Escutador}
\begin{itemize}
\item {Grp. gram.:m.  e  adj.}
\end{itemize}
O que escuta.
\section{Escutar}
\begin{itemize}
\item {Grp. gram.:v. t.}
\end{itemize}
\begin{itemize}
\item {Grp. gram.:V. i.}
\end{itemize}
\begin{itemize}
\item {Proveniência:(Do lat. \textunderscore auscultare\textunderscore )}
\end{itemize}
Dar attenção a.
Tornar-se attento para ouvir.
Andar indagando.
Perceber.
Ouvir: \textunderscore escutar coisas indignas\textunderscore .
Prestar attenção, para ouvir alguma coisa: \textunderscore póde falar, que eu escuto\textunderscore .
\section{Escutelarina}
\begin{itemize}
\item {Grp. gram.:f.}
\end{itemize}
\begin{itemize}
\item {Utilização:Pharm.}
\end{itemize}
Medicamento tónico, estimulante.
\section{Escutellarina}
\begin{itemize}
\item {Grp. gram.:f.}
\end{itemize}
\begin{itemize}
\item {Utilização:Pharm.}
\end{itemize}
Medicamento tónico, estimulante.
\section{Esdruxular}
\begin{itemize}
\item {Grp. gram.:v. i.}
\end{itemize}
Versejar com esdrúxulos. Cf. Castilho, \textunderscore Primavera\textunderscore , 161.
\section{Esdruxularia}
\begin{itemize}
\item {Grp. gram.:f.}
\end{itemize}
\begin{itemize}
\item {Utilização:Des.}
\end{itemize}
\begin{itemize}
\item {Proveniência:(De \textunderscore esdrúxulo\textunderscore )}
\end{itemize}
Coisa extravagante, exótica.
\section{Esdrúxulo}
\begin{itemize}
\item {Grp. gram.:adj.}
\end{itemize}
\begin{itemize}
\item {Utilização:Pop.}
\end{itemize}
\begin{itemize}
\item {Grp. gram.:M.}
\end{itemize}
\begin{itemize}
\item {Proveniência:(It. \textunderscore sdrucciolo\textunderscore )}
\end{itemize}
Diz-se das palavras, que têm o accento tónico na ante-penúltima sýllaba: \textunderscore pállido\textunderscore , \textunderscore tépido\textunderscore , \textunderscore tímido\textunderscore , \textunderscore trôpego\textunderscore , \textunderscore túmulo\textunderscore .
E diz-se dos versos, que terminam em palavra esdrúxula.
Esquisito.
Excêntrico.
Extravagante.
Verso esdrúxulo.
Termo esdrúxulo.
\section{Eserina}
\begin{itemize}
\item {Grp. gram.:f.}
\end{itemize}
Alcaloide da fava do Calabar. Cf. \textunderscore Pharm. Port.\textunderscore 
\section{Esfadigado}
\begin{itemize}
\item {Grp. gram.:adj.}
\end{itemize}
O mesmo que \textunderscore afadigado\textunderscore . Cf. Camillo, \textunderscore Brasileira\textunderscore , 111.
\section{Esfaguntar}
\begin{itemize}
\item {Grp. gram.:v. t.}
\end{itemize}
(Corr. de \textunderscore esfugantar\textunderscore )
\section{Esfaimar}
\begin{itemize}
\item {Grp. gram.:v. t.}
\end{itemize}
O mesmo que \textunderscore esfomear\textunderscore :«\textunderscore certa raposa, andando muito esfaimada...\textunderscore »Bocage.
\section{Esfalcaçado}
\begin{itemize}
\item {Grp. gram.:adj.}
\end{itemize}
\begin{itemize}
\item {Utilização:Náut.}
\end{itemize}
\begin{itemize}
\item {Utilização:Pop.}
\end{itemize}
Que não tem falcaça.
Que não tem alguma coisa; que não tem dinheiro: \textunderscore andas muito esfalcaçado\textunderscore .
\section{Esfalecer}
\begin{itemize}
\item {Grp. gram.:v. i.}
\end{itemize}
\begin{itemize}
\item {Utilização:Prov.}
\end{itemize}
Morrer.
\section{Esfalfamento}
\begin{itemize}
\item {Grp. gram.:m.}
\end{itemize}
Effeito de esfalfar.
Enfraquecimento; anemia: \textunderscore trabalhando, apanhou um esfalfamento\textunderscore .
\section{Esfalfar}
\begin{itemize}
\item {Grp. gram.:v. t.}
\end{itemize}
Tornar fraco, em consequência de trabalho excessivo; fatigar.
\section{Esfallecer}
\begin{itemize}
\item {Grp. gram.:v. i.}
\end{itemize}
\begin{itemize}
\item {Utilização:Prov.}
\end{itemize}
Morrer.
\section{Esfandangado}
\begin{itemize}
\item {Grp. gram.:adj.}
\end{itemize}
\begin{itemize}
\item {Utilização:Ant.}
\end{itemize}
\begin{itemize}
\item {Proveniência:(De \textunderscore fandango\textunderscore )}
\end{itemize}
Mal soante, desafinado. Cp. G. Vicente, \textunderscore Inês Pereira\textunderscore .
\section{Esfandegado}
\begin{itemize}
\item {Grp. gram.:adj.}
\end{itemize}
Provavelmente o mesmo que \textunderscore esfadigado\textunderscore :«\textunderscore a criada appareceu então esfandegada para pôr a mesa\textunderscore ». Camillo, \textunderscore Brasileira\textunderscore , 127.
\section{Esfangoar-se}
\begin{itemize}
\item {Grp. gram.:v. p.}
\end{itemize}
\begin{itemize}
\item {Utilização:Prov.}
\end{itemize}
\begin{itemize}
\item {Utilização:trasm.}
\end{itemize}
Tornar-se fofo, deixar de sêr compacto (o conteúdo de um saco, v. g. erva, palha, etc., de maneira que o saco possa dobrar-se).
\section{Esfanicar}
\begin{itemize}
\item {Grp. gram.:v. t.}
\end{itemize}
Partir em fanicos.
Esbandalhar.
\section{Esfaqueador}
\begin{itemize}
\item {Grp. gram.:adj.}
\end{itemize}
Que esfaqueia.
\section{Esfaqueamento}
\begin{itemize}
\item {Grp. gram.:m.}
\end{itemize}
Acto ou effeito de esfaquear.
\section{Esfaquear}
\begin{itemize}
\item {Grp. gram.:v. t.}
\end{itemize}
Golpear ou matar com faca.
\section{Esfardar}
\begin{itemize}
\item {Grp. gram.:v. t.}
\end{itemize}
\begin{itemize}
\item {Utilização:Prov.}
\end{itemize}
\begin{itemize}
\item {Utilização:trasm.}
\end{itemize}
\begin{itemize}
\item {Proveniência:(De \textunderscore fardo\textunderscore )}
\end{itemize}
Despojar (alguém) do que tem nas algibeiras.
\section{Esfarelamento}
\begin{itemize}
\item {Grp. gram.:m.}
\end{itemize}
Acto ou effeito de esfarelar.
\section{Esfarelar}
\begin{itemize}
\item {Grp. gram.:v. t.}
\end{itemize}
Converter em farelo.
Reduzir a migalhas: \textunderscore esfarelar pão\textunderscore .
Esmiolar.
\section{Esfarfalhada}
\begin{itemize}
\item {Grp. gram.:adj. f.}
\end{itemize}
\begin{itemize}
\item {Proveniência:(De \textunderscore farfalha\textunderscore )}
\end{itemize}
Diz-se da flôr, que tem as folhas muito abertas e quási a cair: escarcalhada.
\section{Esfarinhar}
\begin{itemize}
\item {Grp. gram.:v. t.}
\end{itemize}
\begin{itemize}
\item {Utilização:Prov.}
\end{itemize}
\begin{itemize}
\item {Utilização:beir.}
\end{itemize}
Reduzir a farinha.
Reduzir a pó.
Esmigalhar (batatas cozidas, para as converter em pureia).
\section{Esfarpar}
\begin{itemize}
\item {Grp. gram.:v. t.}
\end{itemize}
\begin{itemize}
\item {Proveniência:(De \textunderscore farpa\textunderscore )}
\end{itemize}
Desfiar.
Fazer em lascas.
Destorcer (o murrão), para o aparar.
\section{Esfarpelar}
\begin{itemize}
\item {Grp. gram.:v. t.}
\end{itemize}
O mesmo que \textunderscore esfarpar\textunderscore , desfiar. Cf. Camillo, \textunderscore Brasileira\textunderscore , 104.
\section{Esfarrapadeira}
\begin{itemize}
\item {Grp. gram.:f.}
\end{itemize}
\begin{itemize}
\item {Proveniência:(De \textunderscore esfarrapar\textunderscore )}
\end{itemize}
Máquina, guarnecida de cylindros com dentes de aço, para desfazer os fios ou farrapos da lan, nas fábricas de lanifícios.
\section{Esfarrapador}
\begin{itemize}
\item {Grp. gram.:m.}
\end{itemize}
Apparelho, para esfarrapar. Cf. \textunderscore Inquér. Industr.\textunderscore , 2.^a p., v. I, 142.
\section{Esfarrapamento}
\begin{itemize}
\item {Grp. gram.:m.}
\end{itemize}
Acto de esfarrapar.
\section{Esfarrapar}
\begin{itemize}
\item {Grp. gram.:v. t.}
\end{itemize}
Reduzir a farrapos; rasgar.
Romper (a pelle); dilacerar.
\section{Esfarripado}
\begin{itemize}
\item {Grp. gram.:adj.}
\end{itemize}
\begin{itemize}
\item {Proveniência:(De \textunderscore esfarripar\textunderscore )}
\end{itemize}
Disposto em farripas: \textunderscore cabello esfarripado\textunderscore .
\section{Esfarripar}
\begin{itemize}
\item {Grp. gram.:v. t.}
\end{itemize}
Desfazer em farripas; desmanchar (um fio) em febras.
\section{Esfaruncar}
\begin{itemize}
\item {Grp. gram.:v. t.}
\end{itemize}
\begin{itemize}
\item {Utilização:Pop.}
\end{itemize}
O mesmo que \textunderscore esfuracar\textunderscore .
\section{Esfatiar}
\begin{itemize}
\item {Grp. gram.:v. t.}
\end{itemize}
Partir em fatias.
\section{Esfazer}
\begin{itemize}
\item {Grp. gram.:v. t.}
\end{itemize}
O mesmo que \textunderscore desfazer\textunderscore . Cf. Castilho, \textunderscore Metam.\textunderscore , 147.
\section{Esfelgar}
\begin{itemize}
\item {Grp. gram.:v. t.}
\end{itemize}
\begin{itemize}
\item {Utilização:Prov.}
\end{itemize}
\begin{itemize}
\item {Utilização:minh.}
\end{itemize}
\begin{itemize}
\item {Utilização:dur.}
\end{itemize}
Limpar de felga.
Tirar de (terreno lavrado) com ancinho a grama e outras raízes, para se fazer sementeira.
\section{Esfera}
\begin{itemize}
\item {Grp. gram.:f.}
\end{itemize}
\begin{itemize}
\item {Grp. gram.:f.}
\end{itemize}
\begin{itemize}
\item {Proveniência:(Lat. \textunderscore sphaera\textunderscore )}
\end{itemize}
Corpo, limitado em todas as direcções por uma superficie curva, cujos pontos distam igualmente de um ponto interior.
Qualquer corpo perfeitamente redondo.
Globo terrestre, o mundo.
Conjunto de círculos, com que os astrónomos representam os movimentos e as relações dos corpos do sistema planetário.
Zona.
Órbita, percorrida por um astro.
Espaço ou área da actividade de um corpo.
Extensão de atribuições, de poder, de competência: \textunderscore na esfera dos meus recursos\textunderscore .
Classe: \textunderscore homem de baixa esfera\textunderscore .
Antiga moéda portuguesa.
Antiga e pequena peça de artilharia.
Moéda de oiro, cunhada em tempo de D. Manuel.
(Cp. \textunderscore esphera\textunderscore )
\section{Esfergulhar}
\begin{itemize}
\item {Grp. gram.:v. i.}
\end{itemize}
\begin{itemize}
\item {Utilização:Prov.}
\end{itemize}
O mesmo que \textunderscore esfervilhar\textunderscore .
\section{Esfervelho}
\begin{itemize}
\item {fónica:vê}
\end{itemize}
\begin{itemize}
\item {Grp. gram.:m.}
\end{itemize}
\begin{itemize}
\item {Utilização:Fam.}
\end{itemize}
\begin{itemize}
\item {Proveniência:(De \textunderscore esfervilhar\textunderscore )}
\end{itemize}
Pessôa inquieta, travêssa.
\section{Esfervilhação}
\begin{itemize}
\item {Grp. gram.:f.}
\end{itemize}
Acto de esfervilhar.
\section{Esfervilhar}
\begin{itemize}
\item {Grp. gram.:v. i.}
\end{itemize}
\begin{itemize}
\item {Proveniência:(De \textunderscore ferver\textunderscore )}
\end{itemize}
Mexer-se muito; revolver-se.
\section{Esfiada}
\begin{itemize}
\item {Grp. gram.:f.}
\end{itemize}
Acto ou effeito de esfiar a camisa do milho para almofadas, etc.
Conjunto de pessôas, empregadas nessa tarefa. (Colhido em Torres Novas)
\section{Esfiampar}
\begin{itemize}
\item {Grp. gram.:v. t.}
\end{itemize}
Desfazer em fiapos; esfiar. Cf. Camillo, \textunderscore Corja\textunderscore , 241.
(Por \textunderscore esfiapar\textunderscore , de \textunderscore fiapa\textunderscore )
\section{Esfiapar}
\begin{itemize}
\item {Grp. gram.:v. t.}
\end{itemize}
\begin{itemize}
\item {Utilização:Bras. do N}
\end{itemize}
\begin{itemize}
\item {Proveniência:(De \textunderscore fiapo\textunderscore )}
\end{itemize}
O mesmo ou melhor que \textunderscore esfiampar\textunderscore .
\section{Esfiar}
\begin{itemize}
\item {Grp. gram.:v. t.}
\end{itemize}
O mesmo que \textunderscore desfiar\textunderscore .
\section{Esfibrar}
\begin{itemize}
\item {Grp. gram.:v. t.}
\end{itemize}
Separar as fibras de; desfazer em fibras.
\section{Esfloramento}
\begin{itemize}
\item {Grp. gram.:m.}
\end{itemize}
Acto ou effeito de esflorar.
\section{Esflorar}
\begin{itemize}
\item {Grp. gram.:v. t.}
\end{itemize}
Tirar a flôr a.
Desflorar.
Ferir a superfície de: \textunderscore o vento esflora o lago\textunderscore .
\section{Esfoguetear}
\begin{itemize}
\item {Grp. gram.:v. t.}
\end{itemize}
\begin{itemize}
\item {Utilização:Fig.}
\end{itemize}
\begin{itemize}
\item {Grp. gram.:V. i.}
\end{itemize}
Festejar com foguetes.
Censurar desabridamente.
Queimar foguetes. Cf. Eça, \textunderscore P. Amaro\textunderscore , 252, 255.
\section{Esfoirar-se}
\begin{itemize}
\item {Grp. gram.:v. p.}
\end{itemize}
\begin{itemize}
\item {Utilização:Prov.}
\end{itemize}
\begin{itemize}
\item {Utilização:trasm.}
\end{itemize}
\begin{itemize}
\item {Utilização:Prov.}
\end{itemize}
\begin{itemize}
\item {Utilização:beir.}
\end{itemize}
\begin{itemize}
\item {Proveniência:(Do lat. \textunderscore foria\textunderscore )}
\end{itemize}
Rebentar, expellindo o conteúdo, (falando-se de um saco, de uma tripa, etc.).
Têr soltura, diarreia.
\section{Esfola}
\begin{itemize}
\item {Grp. gram.:f.}
\end{itemize}
O mesmo que \textunderscore esfolamento\textunderscore .
\section{Esfola-baínha}
\begin{itemize}
\item {Grp. gram.:f.}
\end{itemize}
Planta brasileira, da fam. das anonáceas.
\section{Esfola-caras}
\begin{itemize}
\item {Grp. gram.:m.}
\end{itemize}
Brigão, desordeiro. Cf. F. Manuel, \textunderscore Apólogos\textunderscore .
\section{Esfoladela}
\begin{itemize}
\item {Grp. gram.:f.}
\end{itemize}
Acto ou effeito de esfolar.
\section{Esfolado}
\begin{itemize}
\item {Grp. gram.:adj.}
\end{itemize}
\begin{itemize}
\item {Utilização:Gír.}
\end{itemize}
\begin{itemize}
\item {Proveniência:(De \textunderscore esfolar\textunderscore )}
\end{itemize}
Escoriado.
Escamado, zangado.
\section{Esfolador}
\begin{itemize}
\item {Grp. gram.:adj.}
\end{itemize}
\begin{itemize}
\item {Grp. gram.:M.}
\end{itemize}
Que esfola.
Aquelle que esfola.
\section{Esfoladura}
\begin{itemize}
\item {Grp. gram.:f.}
\end{itemize}
O mesmo que \textunderscore esfoladela\textunderscore .
\section{Esfola-gato}
\begin{itemize}
\item {Grp. gram.:m.}
\end{itemize}
\begin{itemize}
\item {Utilização:Chul.}
\end{itemize}
\begin{itemize}
\item {Utilização:Ant.}
\end{itemize}
Censura.
Maus tratos.
\section{Esfolamento}
\begin{itemize}
\item {Grp. gram.:m.}
\end{itemize}
O mesmo que \textunderscore esfoladela\textunderscore .
\section{Esfolar}
\begin{itemize}
\item {Grp. gram.:v. t.}
\end{itemize}
\begin{itemize}
\item {Utilização:Fig.}
\end{itemize}
Tirar a pelle de: \textunderscore esfolar um cabrito\textunderscore .
Ferir superficialmente, escoriar: \textunderscore esfolar um dedo\textunderscore .
Arranhar.
Vender muito caro a.
Espoliar com usura.
Desgraçar com impostos: \textunderscore há leis que esfolam a gente\textunderscore .
Deixar na miséria.
(Cp. cast. \textunderscore desollar\textunderscore )
\section{Esfola-vaca}
\begin{itemize}
\item {Grp. gram.:m.}
\end{itemize}
\begin{itemize}
\item {Utilização:Prov.}
\end{itemize}
\begin{itemize}
\item {Utilização:alent.}
\end{itemize}
Vento noroéste, prejudicial ao gado.
\section{Esfolegar}
\begin{itemize}
\item {Grp. gram.:v. i.}
\end{itemize}
O mesmo que \textunderscore resfolegar\textunderscore .
\section{Esfólha}
\begin{itemize}
\item {Grp. gram.:f.}
\end{itemize}
Acto de esfolhar.
\section{Esfolhada}
\begin{itemize}
\item {Grp. gram.:f.}
\end{itemize}
Descamisada.
Acto de esfolhar.
\section{Esfolhadela}
\begin{itemize}
\item {Grp. gram.:f.}
\end{itemize}
O mesmo que \textunderscore esfolhada\textunderscore .
\section{Esfolhador}
\begin{itemize}
\item {Grp. gram.:adj.}
\end{itemize}
\begin{itemize}
\item {Grp. gram.:M.}
\end{itemize}
Que esfolha.
Aquelle que esfolha.
\section{Esfolhar}
\begin{itemize}
\item {Grp. gram.:v. t.}
\end{itemize}
Tirar a fôlha a.
Descamisar (milho).
\section{Esfolhear}
\begin{itemize}
\item {Grp. gram.:v. t.}
\end{itemize}
Folhear inconscientemente:«\textunderscore moços que esfolheiam livrinhos bem doirados...\textunderscore »Castilho, \textunderscore Primavera\textunderscore , 46.
\section{Esfolhoso}
\begin{itemize}
\item {Grp. gram.:adj.}
\end{itemize}
\begin{itemize}
\item {Utilização:Bot.}
\end{itemize}
\begin{itemize}
\item {Proveniência:(De \textunderscore esfolhar\textunderscore )}
\end{itemize}
Que não tem fôlhas ou estípulas.
\section{Esfoliação}
\begin{itemize}
\item {Grp. gram.:f.}
\end{itemize}
\begin{itemize}
\item {Utilização:Bot.}
\end{itemize}
\begin{itemize}
\item {Utilização:Med.}
\end{itemize}
\begin{itemize}
\item {Proveniência:(De \textunderscore esfoliar\textunderscore )}
\end{itemize}
Separação ou quéda das lâminas sêcas da casca.
Separação, em lâminas, das partes de um osso, tendão, etc.
\section{Esfoliar}
\begin{itemize}
\item {Grp. gram.:v. t.}
\end{itemize}
\begin{itemize}
\item {Proveniência:(Do lat. \textunderscore exfoliare\textunderscore )}
\end{itemize}
Separar por esfoliação a casca de, as partes de.
\section{Esfoliativo}
\begin{itemize}
\item {Grp. gram.:adj.}
\end{itemize}
\begin{itemize}
\item {Proveniência:(De \textunderscore esfoliar\textunderscore )}
\end{itemize}
Que esfolia ou causa esfoliação.
\section{Esfomear}
\begin{itemize}
\item {Grp. gram.:v. t.}
\end{itemize}
Causar fome a.
Privar de alimentação.
\section{Esforçadamente}
\begin{itemize}
\item {Grp. gram.:adv.}
\end{itemize}
\begin{itemize}
\item {Proveniência:(De \textunderscore esforçar\textunderscore )}
\end{itemize}
Com esfôrço.
\section{Esforçado}
\begin{itemize}
\item {Grp. gram.:adj.}
\end{itemize}
\begin{itemize}
\item {Proveniência:(De \textunderscore esforçar\textunderscore )}
\end{itemize}
Robusto, forte.
Rijo.
Animoso.
\section{Esforçador}
\begin{itemize}
\item {Grp. gram.:adj.}
\end{itemize}
\begin{itemize}
\item {Grp. gram.:M.}
\end{itemize}
Que esforça.
Aquelle que esforça.
\section{Esforçar}
\begin{itemize}
\item {Grp. gram.:v. t.}
\end{itemize}
\begin{itemize}
\item {Grp. gram.:V. i.}
\end{itemize}
Tornar forte.
Dar fôrça a.
Estimular.
Animar; encorajar.
Animar-se, têr coragem:«\textunderscore esforça, esforça, Israel\textunderscore ». Usque, 38, v.^o
\section{Esfôrço}
\begin{itemize}
\item {Grp. gram.:m.}
\end{itemize}
\begin{itemize}
\item {Proveniência:(De \textunderscore esforçar\textunderscore )}
\end{itemize}
Emprêgo de fôrça phýsica.
Energia.
Coragem.
Valentia.
Diligência, zêlo.
\section{Esforfalhar}
\begin{itemize}
\item {Grp. gram.:v. t.}
\end{itemize}
\begin{itemize}
\item {Utilização:Prov.}
\end{itemize}
\begin{itemize}
\item {Utilização:trasm.}
\end{itemize}
\begin{itemize}
\item {Proveniência:(De \textunderscore forfalha\textunderscore )}
\end{itemize}
Esmigalhar, esfarelar (o pão).
\section{Esfornicar}
\begin{itemize}
\item {Grp. gram.:v. t.}
\end{itemize}
O mesmo que \textunderscore fornicar\textunderscore . Cf. Fernão Lopes, \textunderscore Chrón. de D. Pedro\textunderscore .
\section{Esfornigar}
\begin{itemize}
\item {Grp. gram.:v. i.}
\end{itemize}
\begin{itemize}
\item {Utilização:Prov.}
\end{itemize}
\begin{itemize}
\item {Utilização:alent.}
\end{itemize}
Debandar, dispersar-se, (a multidão).
\section{Esforricado}
\begin{itemize}
\item {Grp. gram.:adj.}
\end{itemize}
\begin{itemize}
\item {Utilização:Ant.}
\end{itemize}
Desfeito em bocadinhos.
\section{Esfossador}
\begin{itemize}
\item {Grp. gram.:m.}
\end{itemize}
O que esfossa. Cf. Camillo, \textunderscore Noites de Insómn.\textunderscore , VII. 49.
\section{Esfossar}
\begin{itemize}
\item {Grp. gram.:v. t.}
\end{itemize}
Fossar; revolver. Cf. Camillo, \textunderscore E. Macário\textunderscore , 200.
\section{Esfossilizador}
\begin{itemize}
\item {Grp. gram.:m.}
\end{itemize}
Aquelle que esfossiliza.
\section{Esfossilizar}
\begin{itemize}
\item {Grp. gram.:v. t.}
\end{itemize}
Exhumar (coisas fósseis ou antigas); desenterrar. Cf. Camillo, \textunderscore Cav. em Ruínas\textunderscore , 7.
\section{Esfraldar}
\begin{itemize}
\item {Grp. gram.:v. t.}
\end{itemize}
\begin{itemize}
\item {Utilização:Bras}
\end{itemize}
Desfraldar, alargar, estender:«\textunderscore a figueira esfralda os ramos.\textunderscore »Alencar, \textunderscore Til.\textunderscore 
\section{Esfrançar}
\begin{itemize}
\item {Grp. gram.:v. t.}
\end{itemize}
\begin{itemize}
\item {Proveniência:(De \textunderscore frança\textunderscore )}
\end{itemize}
Cortar os ramos de.
Esgalhar.
Limpar (árvores), cortando-lhes os ramos sêcos ou mais antigos.
\section{Esfrandelhar}
\begin{itemize}
\item {Grp. gram.:v. t.}
\end{itemize}
\begin{itemize}
\item {Utilização:T. de Pare -de-Coira}
\end{itemize}
\begin{itemize}
\item {Utilização:des.}
\end{itemize}
O mesmo que \textunderscore esfrangalhar\textunderscore .
\section{Esfrangalhar}
\begin{itemize}
\item {Grp. gram.:v. t.}
\end{itemize}
Reduzir a frangalhos, a farrapos.
Rasgar.
\section{Esfrega}
\begin{itemize}
\item {Grp. gram.:f.}
\end{itemize}
\begin{itemize}
\item {Utilização:Fig.}
\end{itemize}
\begin{itemize}
\item {Utilização:Pop.}
\end{itemize}
Acto de esfregar.
Faina, grande trabalho.
Reprehensão, sova, tareia.
\section{Esfregação}
\begin{itemize}
\item {Grp. gram.:f.}
\end{itemize}
Acto de esfregar.
\section{Esfregadeira}
\begin{itemize}
\item {Grp. gram.:f.}
\end{itemize}
Mulher, que esfrega casas, para as limpar.
\section{Esfregadela}
\begin{itemize}
\item {Grp. gram.:f.}
\end{itemize}
O mesmo que \textunderscore esfrega\textunderscore .
\section{Esfregado}
\begin{itemize}
\item {Grp. gram.:m.}
\end{itemize}
\begin{itemize}
\item {Utilização:Pop.}
\end{itemize}
\begin{itemize}
\item {Proveniência:(De \textunderscore esfregar\textunderscore )}
\end{itemize}
Aquillo que se esfrega.
Serviço de esfregar: \textunderscore hoje é dia de esfregados\textunderscore .
\section{Esfregador}
\begin{itemize}
\item {Grp. gram.:m.}
\end{itemize}
Utensílio, para esfregar.
\section{Esfregadura}
\begin{itemize}
\item {Grp. gram.:f.}
\end{itemize}
O mesmo que \textunderscore esfregação\textunderscore .
\section{Esfregalho}
\begin{itemize}
\item {Grp. gram.:m.}
\end{itemize}
O mesmo que \textunderscore esfregão\textunderscore .
\section{Esfregamento}
\begin{itemize}
\item {Grp. gram.:m.}
\end{itemize}
Acto de esfregar.
\section{Esfregante}
\begin{itemize}
\item {Grp. gram.:m.}
\end{itemize}
\begin{itemize}
\item {Utilização:Prov.}
\end{itemize}
\begin{itemize}
\item {Utilização:trasm.}
\end{itemize}
\begin{itemize}
\item {Proveniência:(De \textunderscore esfregar\textunderscore )}
\end{itemize}
\textunderscore Num esfregante\textunderscore , num instante, em-quanto o diabo esfrega um ôlho.
\section{Esfregão}
\begin{itemize}
\item {Grp. gram.:m.}
\end{itemize}
\begin{itemize}
\item {Grp. gram.:Loc.}
\end{itemize}
\begin{itemize}
\item {Utilização:trasm}
\end{itemize}
\begin{itemize}
\item {Proveniência:(De \textunderscore esfregar\textunderscore )}
\end{itemize}
Rodilha, qualquer objecto com que se esfrega.
\textunderscore Dar ao esfregão\textunderscore , dar á lingua, falar muito. (Colhida em Sabrosa)
\section{Esfregar}
\begin{itemize}
\item {Grp. gram.:v. t.}
\end{itemize}
\begin{itemize}
\item {Grp. gram.:Loc.}
\end{itemize}
\begin{itemize}
\item {Utilização:fam.}
\end{itemize}
\begin{itemize}
\item {Proveniência:(Do lat. \textunderscore fricare\textunderscore )}
\end{itemize}
Mover repetidas vezes a mão ou outro objecto sôbre a superfície de, para produzir calor, limpar, etc.
Friccionar.
\textunderscore Esfregar as costas\textunderscore , bater, espancar:«\textunderscore quando os soldados portugueses lhes começaram a esfregar as costas...\textunderscore »Chagas, \textunderscore Hist. Al. de Port.\textunderscore 
\section{Esfriadoiro}
\begin{itemize}
\item {Grp. gram.:m.}
\end{itemize}
\begin{itemize}
\item {Proveniência:(De \textunderscore esfriar\textunderscore )}
\end{itemize}
Vaso, em que se esfria qualquer objecto quente.
\section{Esfriador}
\begin{itemize}
\item {Grp. gram.:m.}
\end{itemize}
\begin{itemize}
\item {Grp. gram.:Adj.}
\end{itemize}
Esfriadoiro.
Que esfria.
\section{Esfriadouro}
\begin{itemize}
\item {Grp. gram.:m.}
\end{itemize}
\begin{itemize}
\item {Proveniência:(De \textunderscore esfriar\textunderscore )}
\end{itemize}
Vaso, em que se esfria qualquer objecto quente.
\section{Esfriamento}
\begin{itemize}
\item {Grp. gram.:m.}
\end{itemize}
Acto ou effeito de esfriar.
Doença de alguns animaes.
\section{Esfriante}
\begin{itemize}
\item {Grp. gram.:adj.}
\end{itemize}
Que esfria ou faz esfriar.
\section{Esfriar}
\begin{itemize}
\item {Grp. gram.:v. t.}
\end{itemize}
\begin{itemize}
\item {Utilização:Gír.}
\end{itemize}
\begin{itemize}
\item {Utilização:Fig.}
\end{itemize}
\begin{itemize}
\item {Grp. gram.:V. i.}
\end{itemize}
\begin{itemize}
\item {Utilização:Fig.}
\end{itemize}
Tornar frio.
Tirar o calor a: \textunderscore esfriar o chá\textunderscore .
Matar.
Deminuir o enthusiasmo a.
Entibiar.
Tornar-se frio: \textunderscore a manhan esfriou\textunderscore .
Perder a paixão.
Deminuir de intensidade: \textunderscore o amor também esfria\textunderscore .
\section{Esfrolar}
\begin{itemize}
\item {Grp. gram.:v. t.}
\end{itemize}
\begin{itemize}
\item {Utilização:Bras}
\end{itemize}
\begin{itemize}
\item {Proveniência:(De \textunderscore frôl\textunderscore , por \textunderscore flôr\textunderscore )}
\end{itemize}
Escoriar, esfolar. Cf. Arn. Gama, \textunderscore Última Dona\textunderscore , 9.
\section{Esfugantar}
\begin{itemize}
\item {Grp. gram.:v. t.}
\end{itemize}
\begin{itemize}
\item {Utilização:Prov.}
\end{itemize}
\begin{itemize}
\item {Proveniência:(Do rad. de \textunderscore fuga\textunderscore )}
\end{itemize}
Tresmalhar.
Pôr em debandada; afugentar.
\section{Esfugentar}
\begin{itemize}
\item {Grp. gram.:v. t.}
\end{itemize}
O mesmo que \textunderscore afugentar\textunderscore . Cf. Filinto, \textunderscore D. Man.\textunderscore , III, 14.
\section{Esfugir}
\begin{itemize}
\item {Grp. gram.:v. t.}
\end{itemize}
O mesmo que \textunderscore esfugentar\textunderscore .
\section{Esfulinhar}
\begin{itemize}
\item {Grp. gram.:v. t.}
\end{itemize}
\begin{itemize}
\item {Utilização:Des.}
\end{itemize}
\begin{itemize}
\item {Proveniência:(Do rad. de \textunderscore fuligem\textunderscore )}
\end{itemize}
Basculhar.
Varrer, limpar, (teias de aranha, fuligem, etc.).
\section{Esfumaçamento}
\begin{itemize}
\item {Grp. gram.:m.}
\end{itemize}
\begin{itemize}
\item {Utilização:Neol.}
\end{itemize}
O mesmo que \textunderscore esfumação\textunderscore .
\section{Esfumação}
\begin{itemize}
\item {Grp. gram.:f.}
\end{itemize}
Acto ou effeito de esfumar.
\section{Esfumaçar}
\begin{itemize}
\item {Grp. gram.:v. t.}
\end{itemize}
Encher de muito fumo.
Ennegrecer com fumo. Cf. Camillo, \textunderscore Brasileira\textunderscore , 104.
\section{Esfumado}
\begin{itemize}
\item {Grp. gram.:m.}
\end{itemize}
\begin{itemize}
\item {Proveniência:(De \textunderscore esfumar\textunderscore )}
\end{itemize}
Desenho, com as sombras esbatidas a esfuminho.
\section{Esfumador}
\begin{itemize}
\item {Grp. gram.:m.}
\end{itemize}
\begin{itemize}
\item {Proveniência:(De \textunderscore esfumar\textunderscore )}
\end{itemize}
Pincel, para unir as tintas de um quadro, esbatendo-as.
\section{Esfumar}
\begin{itemize}
\item {Grp. gram.:v. t.}
\end{itemize}
\begin{itemize}
\item {Proveniência:(De \textunderscore fumo\textunderscore )}
\end{itemize}
Desenhar ou pintar a carvão.
Esbater com esfuminho (os traços a carvão num desenho).
Esboçar com o esfuminho.
Tornar escuro.
Ennegrecer com o fumo.
\section{Esfumarar}
\begin{itemize}
\item {Grp. gram.:v. t.}
\end{itemize}
Cobrir de fumo.
Tornar semelhante a fumo.
(Cp. \textunderscore fumarada\textunderscore )
\section{Esfumear}
\begin{itemize}
\item {Grp. gram.:v. i.}
\end{itemize}
O mesmo que \textunderscore fumegar\textunderscore .
\section{Esfuminhar}
\begin{itemize}
\item {Grp. gram.:v. t.}
\end{itemize}
\begin{itemize}
\item {Utilização:Neol.}
\end{itemize}
Pintar com esfuminho.
\section{Esfuminho}
\begin{itemize}
\item {Grp. gram.:m.}
\end{itemize}
\begin{itemize}
\item {Proveniência:(Do it. \textunderscore sfumino\textunderscore )}
\end{itemize}
Utensílio de pellica, para esfumar.
\section{Esfuracar}
\begin{itemize}
\item {Grp. gram.:v. t.}
\end{itemize}
Esburacar.
(Cp. cast. \textunderscore furacar\textunderscore )
\section{Esfurancar}
\begin{itemize}
\item {Grp. gram.:v. t.}
\end{itemize}
\begin{itemize}
\item {Utilização:Pop.}
\end{itemize}
O mesmo que \textunderscore esfuracar\textunderscore .
\section{Esfuziada}
\begin{itemize}
\item {Grp. gram.:f.}
\end{itemize}
\begin{itemize}
\item {Utilização:Ant.}
\end{itemize}
\begin{itemize}
\item {Proveniência:(De \textunderscore esfuziar\textunderscore )}
\end{itemize}
Descarga, tiroteio continuado.
\section{Esfuziado}
\begin{itemize}
\item {Grp. gram.:adj.}
\end{itemize}
\begin{itemize}
\item {Utilização:Prov.}
\end{itemize}
\begin{itemize}
\item {Proveniência:(De \textunderscore esfuziar\textunderscore )}
\end{itemize}
Tresloucado, desesperado. (Colhido na Bairrada)
\section{Esfuziar}
\begin{itemize}
\item {Grp. gram.:v. i.}
\end{itemize}
\begin{itemize}
\item {Grp. gram.:V. t.}
\end{itemize}
Sibilar ou zumbir como os projécteis de fuzil e de artilharia.
Facer zumbir ou sibilar. Cf. Castilho, \textunderscore Fausto\textunderscore , 169.
(Por \textunderscore esfuzilar\textunderscore , de \textunderscore fuzil\textunderscore )
\section{Esfuzilar}
\begin{itemize}
\item {Grp. gram.:v. i.}
\end{itemize}
\begin{itemize}
\item {Proveniência:(De \textunderscore fuzil\textunderscore )}
\end{itemize}
Scintillar; fuzilar.
\section{Esfuziote}
\begin{itemize}
\item {Grp. gram.:m.}
\end{itemize}
\begin{itemize}
\item {Utilização:ant.}
\end{itemize}
\begin{itemize}
\item {Utilização:Pop.}
\end{itemize}
\begin{itemize}
\item {Proveniência:(De \textunderscore esfuziar\textunderscore )}
\end{itemize}
Reprehensão; invectiva.
\section{Esgaçamento}
\begin{itemize}
\item {Grp. gram.:m.}
\end{itemize}
\begin{itemize}
\item {Utilização:Phýs.}
\end{itemize}
Acto de esgaçar ou romper, por effeito de um esfôrço de tracção perpendicular, applicado a um esfôrço de compressão.
\section{Esgaçar}
\textunderscore v. t.\textunderscore  (e der.)
O mesmo que \textunderscore esgarçar\textunderscore , etc.
\section{Esgache}
\begin{itemize}
\item {Grp. gram.:m.}
\end{itemize}
Cepo de marceneiro e carpinteiro, munido de um ferro quási vertical, e com que se moldam os bordos das peças de madeira.
\section{Esgadanhar}
\begin{itemize}
\item {Grp. gram.:v. t.}
\end{itemize}
\begin{itemize}
\item {Proveniência:(De \textunderscore gadanho\textunderscore )}
\end{itemize}
Agadanhar.
Arranhar; arrepelar.
\section{Esgadelhar}
\begin{itemize}
\item {Grp. gram.:v. t.}
\end{itemize}
O mesmo que \textunderscore esguedelhar\textunderscore .
\section{Esgadunhar}
\begin{itemize}
\item {Grp. gram.:v. t.}
\end{itemize}
\begin{itemize}
\item {Utilização:Prov.}
\end{itemize}
\begin{itemize}
\item {Utilização:minh.}
\end{itemize}
\begin{itemize}
\item {Proveniência:(De \textunderscore gadunha\textunderscore )}
\end{itemize}
O mesmo que \textunderscore esgadanhar\textunderscore .
\section{Esgaivar}
\begin{itemize}
\item {Grp. gram.:v. t.}
\end{itemize}
\begin{itemize}
\item {Proveniência:(De \textunderscore gaiva\textunderscore )}
\end{itemize}
Escavar ou abrir barrancos em: \textunderscore a chuva esgaivou as vinhas\textunderscore .
\section{Esgaivotado}
\begin{itemize}
\item {Grp. gram.:adj.}
\end{itemize}
Parecido á gaivota.
Magro, esgrouviado.
Pernalto.
\section{Esgaldripado}
\begin{itemize}
\item {Grp. gram.:adj.}
\end{itemize}
\begin{itemize}
\item {Proveniência:(De \textunderscore esgaldripar\textunderscore )}
\end{itemize}
Diz-se do cacho de uvas comprido, mas com poucos bagos.
\section{Esgaldripar}
\begin{itemize}
\item {Grp. gram.:v. t.}
\end{itemize}
Tornar esgaldripado.
(Cp. \textunderscore gualdripar\textunderscore )
\section{Esgaldrir}
\textunderscore v. t.\textunderscore  (e der.) \textunderscore Prov. beir.\textunderscore 
O mesmo que \textunderscore gualdir\textunderscore , etc.
\section{Esgalgado}
\begin{itemize}
\item {Grp. gram.:adj.}
\end{itemize}
Magro, como um galgo.
Que anda caíndo de lazeira.
\section{Esgalgar}
\begin{itemize}
\item {Grp. gram.:v. t.}
\end{itemize}
\begin{itemize}
\item {Proveniência:(De \textunderscore galgo\textunderscore )}
\end{itemize}
Tornar esgalgado, tornar magro.
Adelgaçar, alongando.
\section{Esgalgueirado}
\begin{itemize}
\item {Grp. gram.:adj.}
\end{itemize}
\begin{itemize}
\item {Utilização:Prov.}
\end{itemize}
\begin{itemize}
\item {Utilização:trasm.}
\end{itemize}
O mesmo que \textunderscore esgalgado\textunderscore .
\section{Esgalha}
\begin{itemize}
\item {Grp. gram.:f.}
\end{itemize}
O mesmo que \textunderscore esgalho\textunderscore .
Acto ou effeito de esgalhar.
Conjunto dos galhos ou ramos, que se cortaram da árvore.
\section{Esgalhada}
\begin{itemize}
\item {Grp. gram.:adj. f.}
\end{itemize}
\begin{itemize}
\item {Utilização:Prov.}
\end{itemize}
\begin{itemize}
\item {Utilização:alent.}
\end{itemize}
\begin{itemize}
\item {Proveniência:(De \textunderscore esgalho\textunderscore )}
\end{itemize}
Diz-se da rapariga airosa, esbelta.
\section{Esgalhado}
\begin{itemize}
\item {Grp. gram.:adj.}
\end{itemize}
\begin{itemize}
\item {Utilização:Prov.}
\end{itemize}
\begin{itemize}
\item {Utilização:trasm.}
\end{itemize}
O mesmo que \textunderscore desgarrado\textunderscore  das coisas da sua espécie ou natureza.
\section{Esgalhar}
\begin{itemize}
\item {Grp. gram.:v. t. ,  i.  e  p.}
\end{itemize}
\begin{itemize}
\item {Utilização:Açor}
\end{itemize}
\begin{itemize}
\item {Utilização:Prov.}
\end{itemize}
\begin{itemize}
\item {Utilização:trasm.}
\end{itemize}
O mesmo que \textunderscore desgalhar\textunderscore .
Descamisar (o milho).
Diz-se do boi, quando fere com os cornos.
\section{Esgalho}
\begin{itemize}
\item {Grp. gram.:m.}
\end{itemize}
\begin{itemize}
\item {Proveniência:(De \textunderscore galho\textunderscore )}
\end{itemize}
Renôvo vegetal, que pouco se desenvolve.
Ramificação das pontas do veado.
Escádea.
Cada uma das partes de um cacho de uvas.
Parte da vide, que o podador não corta.
Ramificação.
\section{Esgalhudo}
\begin{itemize}
\item {Grp. gram.:m.}
\end{itemize}
Peixe, o mesmo que \textunderscore galhudo\textunderscore .
\section{Esgalmido}
\begin{itemize}
\item {Grp. gram.:adj.}
\end{itemize}
\begin{itemize}
\item {Utilização:Prov.}
\end{itemize}
\begin{itemize}
\item {Utilização:trasm.}
\end{itemize}
Que não tem chorume.
Desfalcado.
\section{Esgalracho}
\begin{itemize}
\item {Grp. gram.:m.}
\end{itemize}
(V.escalracho)
\section{Esgalrichar}
\begin{itemize}
\item {Grp. gram.:v. i.}
\end{itemize}
\begin{itemize}
\item {Utilização:Prov.}
\end{itemize}
\begin{itemize}
\item {Utilização:trasm.}
\end{itemize}
O mesmo que \textunderscore galrejar\textunderscore .
\section{Esgana}
\begin{itemize}
\item {Grp. gram.:f.}
\end{itemize}
\begin{itemize}
\item {Utilização:Pop.}
\end{itemize}
Tosse convulsa.
Doença nas vias respiratórias do cão.
Espécie de uva extremenha.
Acto ou effeito de esganar.
\section{Esgana-cão}
\begin{itemize}
\item {Grp. gram.:m.}
\end{itemize}
Um dos nomes da uva cerceal.
\section{Esganação}
\begin{itemize}
\item {Grp. gram.:f.}
\end{itemize}
\begin{itemize}
\item {Utilização:Pop.}
\end{itemize}
Gana; avidez.
Acto ou effeito de esganar.
\section{Esgana-cão-preto}
\begin{itemize}
\item {Grp. gram.:m.}
\end{itemize}
Casta de uva, na região do Doiro.
\section{Esganada}
\begin{itemize}
\item {Grp. gram.:f.}
\end{itemize}
\begin{itemize}
\item {Utilização:Prov.}
\end{itemize}
\begin{itemize}
\item {Utilização:beir.}
\end{itemize}
Cabra velha.
\section{Esganado}
\begin{itemize}
\item {Grp. gram.:m.}
\end{itemize}
\begin{itemize}
\item {Utilização:Fig.}
\end{itemize}
\begin{itemize}
\item {Proveniência:(De \textunderscore esganar\textunderscore )}
\end{itemize}
Indivíduo faminto.
Sovina.
\section{Esganador}
\begin{itemize}
\item {Grp. gram.:m.}
\end{itemize}
\begin{itemize}
\item {Utilização:Gír.}
\end{itemize}
\begin{itemize}
\item {Proveniência:(De \textunderscore esganar\textunderscore )}
\end{itemize}
Gravata.
\section{Esganadura}
\begin{itemize}
\item {Grp. gram.:f.}
\end{itemize}
\begin{itemize}
\item {Utilização:Artilh.}
\end{itemize}
Acto ou effeito de esganar.
Arrebém ou cabo delgado, destinado a cada peça da coberta.
\section{Esgana-gata}
\begin{itemize}
\item {Grp. gram.:f.}
\end{itemize}
Peixe acanthopterýgio.
\section{Esganar}
\begin{itemize}
\item {Grp. gram.:v. t.}
\end{itemize}
\begin{itemize}
\item {Utilização:Gír.}
\end{itemize}
\begin{itemize}
\item {Grp. gram.:V. p.}
\end{itemize}
\begin{itemize}
\item {Utilização:Fig.}
\end{itemize}
\begin{itemize}
\item {Proveniência:(De \textunderscore gana\textunderscore )}
\end{itemize}
Estrangular.
Matar por suffocação.
Esconder.
Estrangular-se.
Enforcar-se.
Sêr avarento.
\section{Esganarelo}
\begin{itemize}
\item {Grp. gram.:m.}
\end{itemize}
\begin{itemize}
\item {Utilização:Prov.}
\end{itemize}
\begin{itemize}
\item {Utilização:beir.}
\end{itemize}
Homem muito magro e alto; homem de pescoço comprido e delgado.
\section{Esganiçado}
\begin{itemize}
\item {Grp. gram.:adj.}
\end{itemize}
Diz-se da voz muito aguda ou estrídula.
E diz-se da pessôa, que tem essa voz:«\textunderscore a Senhorinha, muito esganiçada...\textunderscore »Camillo, \textunderscore Brasileira\textunderscore , 156.
\section{Esganiçar}
\begin{itemize}
\item {Grp. gram.:v. t.}
\end{itemize}
\begin{itemize}
\item {Grp. gram.:V. p.}
\end{itemize}
\begin{itemize}
\item {Proveniência:(Do rad. de \textunderscore ganir\textunderscore )}
\end{itemize}
Tornar esganiçada ou estrídula (a voz). Cf. Filinto, V, 110.
Soltar vozes agudas como o ganir do cão.
Cantar, dando violentamente á voz um som muito agudo.
\section{Esganiço}
\begin{itemize}
\item {Grp. gram.:m.}
\end{itemize}
Acto de se esganiçar:«\textunderscore ...guinchar com um esganiço.\textunderscore »Filinto, XIII, 309.
\section{Esganifrado}
\begin{itemize}
\item {Grp. gram.:adj.}
\end{itemize}
\begin{itemize}
\item {Utilização:Prov.}
\end{itemize}
\begin{itemize}
\item {Utilização:trasm.}
\end{itemize}
O mesmo que \textunderscore escanifrado\textunderscore .
\section{Esganinho}
\begin{itemize}
\item {Grp. gram.:m.}
\end{itemize}
Um dos nomes da uva cerceal.
\section{Esganipado}
\begin{itemize}
\item {Grp. gram.:adj.}
\end{itemize}
\begin{itemize}
\item {Utilização:Prov.}
\end{itemize}
O mesmo que \textunderscore esgaldripado\textunderscore .
\section{Esganitar}
\begin{itemize}
\item {Grp. gram.:v. t.}
\end{itemize}
O mesmo que \textunderscore esganiçar\textunderscore . Cf. Camillo, \textunderscore Serões\textunderscore , III, 68.
\section{Esganosa}
\begin{itemize}
\item {Grp. gram.:f.}
\end{itemize}
O mesmo que \textunderscore esganoso\textunderscore .
\section{Esganoso}
\begin{itemize}
\item {Grp. gram.:m.}
\end{itemize}
O mesmo que \textunderscore esganinho\textunderscore .
\section{Esganzarado}
\begin{itemize}
\item {Grp. gram.:adj.}
\end{itemize}
\begin{itemize}
\item {Utilização:Prov.}
\end{itemize}
\begin{itemize}
\item {Utilização:trasm.}
\end{itemize}
Alto e desajeitado.
Que é um trangalhadanças.
E diz-se da árvore que braceja irregularmente.
\section{Esgar}
\begin{itemize}
\item {Grp. gram.:m.}
\end{itemize}
\begin{itemize}
\item {Proveniência:(Do fr. \textunderscore égard\textunderscore ?)}
\end{itemize}
Gesto do rosto.
Trejeito.
Careta de escárneo.
\section{Esgarabulhão}
\begin{itemize}
\item {Grp. gram.:m.}
\end{itemize}
\begin{itemize}
\item {Utilização:Pop.}
\end{itemize}
\begin{itemize}
\item {Proveniência:(De \textunderscore esgarabulhar\textunderscore )}
\end{itemize}
Homem desassossegado, fura-vidas.
Pião, que gira aos saltos.
\section{Esgarabulhar}
\begin{itemize}
\item {Grp. gram.:v. i.}
\end{itemize}
\begin{itemize}
\item {Utilização:Fig.}
\end{itemize}
\begin{itemize}
\item {Proveniência:(De \textunderscore garabulha\textunderscore )}
\end{itemize}
Girar aos saltos (o pião).
Pular; sêr inquieto.
\section{Esgaratujar}
\begin{itemize}
\item {Grp. gram.:v. t.}
\end{itemize}
O mesmo que \textunderscore garatujar\textunderscore .
\section{Esgaravanada}
\begin{itemize}
\item {Grp. gram.:f.}
\end{itemize}
\begin{itemize}
\item {Utilização:Prov.}
\end{itemize}
\begin{itemize}
\item {Utilização:trasm.}
\end{itemize}
Saraivada forte e de curta duração.
Grande bátega de água, com intermittências, tirada pelo vento.
(Cp. \textunderscore esgarrão\textunderscore )
\section{Esgaravatador}
\begin{itemize}
\item {Grp. gram.:adj.}
\end{itemize}
\begin{itemize}
\item {Grp. gram.:M.}
\end{itemize}
Que esgaravata.
Aquelle que esgaravata.
\section{Esgaravatamento}
\begin{itemize}
\item {Grp. gram.:m.}
\end{itemize}
Acto ou effeito de esgaravatar.
\section{Esgaravatana}
\begin{itemize}
\item {Grp. gram.:f.}
\end{itemize}
\begin{itemize}
\item {Utilização:Bras}
\end{itemize}
Canudo de madeira, formado de duas peças grudadas e liadas com correias vegetaes, pelo qual os aborígenas despedem, soprando, as setas ervadas. Cf. B. C. Rubim, \textunderscore Vocabulário Bras.\textunderscore 
\section{Esgaravatar}
\begin{itemize}
\item {Grp. gram.:v. t.}
\end{itemize}
\begin{itemize}
\item {Utilização:Fig.}
\end{itemize}
\begin{itemize}
\item {Proveniência:(De \textunderscore garavato\textunderscore )}
\end{itemize}
Remexer com as unhas (a terra).
Limpar com palito (dentes ou ouvidos).
Remexer (um brasido).
Escorvar.
Pesquisar minuciosamente: \textunderscore esgaravatar bibliothecas\textunderscore .
\section{Esgaravatil}
\begin{itemize}
\item {Grp. gram.:m.}
\end{itemize}
Instrumento, para fazer encaixes na madeira.
\section{Esgaravunchar}
\begin{itemize}
\item {Grp. gram.:v. t.}
\end{itemize}
\begin{itemize}
\item {Utilização:Pop.}
\end{itemize}
O mesmo que \textunderscore escarafunchar\textunderscore .
\section{Esgaravunhar}
\begin{itemize}
\item {Grp. gram.:v. t.}
\end{itemize}
O mesmo que \textunderscore escarafunchar\textunderscore . Cf. Arn. Gama, \textunderscore Segr. do Abb.\textunderscore , 149.
\section{Esgarçar}
\begin{itemize}
\item {Grp. gram.:v. t.}
\end{itemize}
\begin{itemize}
\item {Grp. gram.:V. i.}
\end{itemize}
Rasgar, afastando os fios de (um tecido).
Desfiar-se.
Abrir-se o tecido raro.
(Da mesma or. que \textunderscore escarchar\textunderscore )
\section{Esgardunhar}
\begin{itemize}
\item {Grp. gram.:v. t.}
\end{itemize}
Arranhar, agadanhar, á semelhança do gardunho.
\section{Esgareiro}
\begin{itemize}
\item {Grp. gram.:m.}
\end{itemize}
\begin{itemize}
\item {Utilização:P. us.}
\end{itemize}
Que faz esgares.
\section{Esgargalar}
\begin{itemize}
\item {Grp. gram.:v. t.}
\end{itemize}
\begin{itemize}
\item {Proveniência:(De \textunderscore gargalo\textunderscore )}
\end{itemize}
Decotar.
Tornar alto o pescoço de.
\section{Esgargalhar-se}
\begin{itemize}
\item {Grp. gram.:v. p.}
\end{itemize}
Rir ás gargalhadas.
\section{Esgargar}
\begin{itemize}
\item {Grp. gram.:v. t.}
\end{itemize}
\begin{itemize}
\item {Utilização:Des.}
\end{itemize}
Tirar a casca a, esburgar.
Tirar o miolo de.
(Por \textunderscore escarcar\textunderscore , do mesmo rad. de \textunderscore escarcavelar\textunderscore ?)
\section{Esgarnachado}
\begin{itemize}
\item {Grp. gram.:adj.}
\end{itemize}
\begin{itemize}
\item {Utilização:Prov.}
\end{itemize}
\begin{itemize}
\item {Utilização:trasm.}
\end{itemize}
\begin{itemize}
\item {Proveniência:(De \textunderscore garnacha\textunderscore )}
\end{itemize}
Esfrangalhado, muito roto.
Que tem desabotoado e aberto o peito da camisa.
\section{Esgarrabunhar}
\textunderscore v. t.\textunderscore  (e der.) \textunderscore Prov. trasm.\textunderscore 
O mesmo que \textunderscore esgarrafunchar\textunderscore , etc.
\section{Esgarrafunchão}
\begin{itemize}
\item {Grp. gram.:m.}
\end{itemize}
Acto ou effeito de esgarrafunchar.
\section{Esgarrafunchar}
\begin{itemize}
\item {Grp. gram.:v. t.}
\end{itemize}
\begin{itemize}
\item {Utilização:Prov.}
\end{itemize}
\begin{itemize}
\item {Utilização:trasm.}
\end{itemize}
Fazer arranhaduras extensas mas superficiaes em.
Arranhar muito.
(Cp. \textunderscore escarafunchar\textunderscore )
\section{Esgarrão}
\begin{itemize}
\item {Grp. gram.:m.}
\end{itemize}
\begin{itemize}
\item {Grp. gram.:Adj.}
\end{itemize}
\begin{itemize}
\item {Proveniência:(De \textunderscore esgarrar\textunderscore )}
\end{itemize}
Jôgo popular.
Redemoínho.
Desgarrão.
Diz-se do tempo ou do vento, que faz esgarrar embarcações.
\section{Esgarrar}
\begin{itemize}
\item {Grp. gram.:v. t.}
\end{itemize}
\begin{itemize}
\item {Grp. gram.:V. i.}
\end{itemize}
\begin{itemize}
\item {Proveniência:(De \textunderscore garrar\textunderscore . Cp. fr. \textunderscore égarer\textunderscore )}
\end{itemize}
Desviar do rumo.
Fazer garrar.
Obrigar a correr.
Afastar da companhia.
Desencaminhar: \textunderscore rebanho esgarrado\textunderscore .
Desviar-se da rota, (um navio).
Transviar-se.
Garrar.
\section{Esgastrite}
\begin{itemize}
\item {Grp. gram.:f.}
\end{itemize}
\begin{itemize}
\item {Proveniência:(De \textunderscore es...\textunderscore  + \textunderscore gastrite\textunderscore )}
\end{itemize}
Inflammação exterior do estômago.
\section{Esgatanhar}
\begin{itemize}
\item {Grp. gram.:v. t.}
\end{itemize}
O mesmo que \textunderscore agatanhar\textunderscore .
\section{Esgatiçar}
\begin{itemize}
\item {Grp. gram.:v. t.}
\end{itemize}
\begin{itemize}
\item {Utilização:T. da Bairrada}
\end{itemize}
O mesmo que \textunderscore esgatanhar\textunderscore .
\section{Esgazear}
\begin{itemize}
\item {Grp. gram.:v. t.}
\end{itemize}
\begin{itemize}
\item {Grp. gram.:V. i.}
\end{itemize}
\begin{itemize}
\item {Utilização:Prov.}
\end{itemize}
\begin{itemize}
\item {Proveniência:(De \textunderscore gázeo\textunderscore )}
\end{itemize}
Pôr em branco (os olhos).
Tornar claro ou desmaiado.
Mover ao acaso (os olhos), abrindo-os espavoridamente.
Adelgaçarem-se as nuvens, depois de chover: \textunderscore se esgazear, ainda hoje vou á quinta\textunderscore . (Colhido em Turquel)
\section{Esgoda}
\begin{itemize}
\item {Grp. gram.:f.}
\end{itemize}
\begin{itemize}
\item {Utilização:Prov.}
\end{itemize}
\begin{itemize}
\item {Utilização:trasm.}
\end{itemize}
\begin{itemize}
\item {Proveniência:(De \textunderscore esgodar-se\textunderscore )}
\end{itemize}
Sova, tunda.
Trepa.
\section{Esgodar-se}
\begin{itemize}
\item {Grp. gram.:v. p.}
\end{itemize}
\begin{itemize}
\item {Utilização:Prov.}
\end{itemize}
\begin{itemize}
\item {Utilização:trasm.}
\end{itemize}
Esfolar-se superficialmente numa parte da pelle, em consequência de attrito com uma substância dura e áspera.
(Provavelmente, relaciona-se com \textunderscore escodar\textunderscore )
\section{Esgoelar-se}
\begin{itemize}
\item {fónica:go-e}
\end{itemize}
\begin{itemize}
\item {Grp. gram.:v. p.}
\end{itemize}
Abrir muito as goélas; gritar muito.
\section{Esgoldrejar}
\begin{itemize}
\item {Grp. gram.:v. t.}
\end{itemize}
\begin{itemize}
\item {Utilização:Prov.}
\end{itemize}
\begin{itemize}
\item {Utilização:trasm.}
\end{itemize}
\begin{itemize}
\item {Proveniência:(T. onom)}
\end{itemize}
Vasculejar.
Agitar (uma vasilha, que contenha um líquido que a não enche).
Agitar (as vísceras), indo num cavallo a chouto.
\section{Esgolado}
\begin{itemize}
\item {Grp. gram.:adj.}
\end{itemize}
\begin{itemize}
\item {Utilização:Prov.}
\end{itemize}
\begin{itemize}
\item {Utilização:beir.}
\end{itemize}
Esgorjado.
Que tem o pescoço todo descoberto.
Que traz desabotoado o colarinho e o peito á mostra.
(Por \textunderscore desgollado\textunderscore , de \textunderscore golla\textunderscore )
\section{Esgoldrejão}
\begin{itemize}
\item {Grp. gram.:m.}
\end{itemize}
\begin{itemize}
\item {Utilização:Prov.}
\end{itemize}
\begin{itemize}
\item {Utilização:trasm.}
\end{itemize}
\begin{itemize}
\item {Proveniência:(De \textunderscore esgoldrejar\textunderscore )}
\end{itemize}
Sacudidela ou safanão violento.
\section{Esgollado}
\begin{itemize}
\item {Grp. gram.:adj.}
\end{itemize}
\begin{itemize}
\item {Utilização:Prov.}
\end{itemize}
\begin{itemize}
\item {Utilização:beir.}
\end{itemize}
Esgorjado.
Que tem o pescoço todo descoberto.
Que traz desabotoado o collarinho e o peito á mostra.
(Por \textunderscore desgollado\textunderscore , de \textunderscore golla\textunderscore )
\section{Esgorjar}
\begin{itemize}
\item {Grp. gram.:v. t.}
\end{itemize}
\begin{itemize}
\item {Grp. gram.:V. i.}
\end{itemize}
\begin{itemize}
\item {Utilização:Des.}
\end{itemize}
\begin{itemize}
\item {Proveniência:(De \textunderscore gorja\textunderscore )}
\end{itemize}
O mesmo que \textunderscore esgargalar\textunderscore .
Têr grande appetite, grande desejo.
\section{Esgotadoiro}
\begin{itemize}
\item {Grp. gram.:m.}
\end{itemize}
\begin{itemize}
\item {Proveniência:(De \textunderscore esgotar\textunderscore )}
\end{itemize}
Cano para esgôto.
\section{Esgotador}
\begin{itemize}
\item {Grp. gram.:adj.}
\end{itemize}
\begin{itemize}
\item {Grp. gram.:m.}
\end{itemize}
Que esgota.
Aquelle que esgota.
\section{Esgotadouro}
\begin{itemize}
\item {Grp. gram.:m.}
\end{itemize}
\begin{itemize}
\item {Proveniência:(De \textunderscore esgotar\textunderscore )}
\end{itemize}
Cano para esgôto.
\section{Esgotadura}
\begin{itemize}
\item {Grp. gram.:f.}
\end{itemize}
Acto ou effeito de esgotar.
\section{Esgotamento}
\begin{itemize}
\item {Grp. gram.:m.}
\end{itemize}
O mesmo que \textunderscore esgotadura\textunderscore .
\section{Esgotante}
\begin{itemize}
\item {Grp. gram.:adj.}
\end{itemize}
Que esgota.
\section{Esgotar}
\begin{itemize}
\item {Grp. gram.:v. t.}
\end{itemize}
\begin{itemize}
\item {Utilização:Fig.}
\end{itemize}
\begin{itemize}
\item {Grp. gram.:V. p.}
\end{itemize}
\begin{itemize}
\item {Proveniência:(De \textunderscore gota\textunderscore )}
\end{itemize}
Exhaurir.
Tirar até á última gota.
Esvaziar: \textunderscore esgotar um tonel\textunderscore .
Consumir: \textunderscore esgotar a paciência\textunderscore .
Tratar completamente (um assumpto).
Perder a fôrça.
Extenuar-se.
Dissipar tudo.
\section{Esgotável}
\begin{itemize}
\item {Grp. gram.:adj.}
\end{itemize}
Que se póde esgotar.
\section{Esgote}
\begin{itemize}
\item {Grp. gram.:m.}
\end{itemize}
(V.esgotadura)
\section{Esgoteiro}
\begin{itemize}
\item {Grp. gram.:m.}
\end{itemize}
Reservatório de água, junto de cada compartimento crystallizador, nas salinas de Rio-Maior. Cf. \textunderscore Museu Techn.\textunderscore , 124.
\section{Esgôto}
\begin{itemize}
\item {Grp. gram.:m.}
\end{itemize}
\begin{itemize}
\item {Proveniência:(De \textunderscore esgotar\textunderscore )}
\end{itemize}
O mesmo que \textunderscore esgotamento\textunderscore .
Abertura, ou cano, por onde correm ou se esgotam líquidos ou dejecções: \textunderscore os esgotos de Lisbôa\textunderscore .
\section{Esgrafiar}
\begin{itemize}
\item {Grp. gram.:v. t.}
\end{itemize}
\begin{itemize}
\item {Proveniência:(It. \textunderscore sgraffiare\textunderscore )}
\end{itemize}
Pintar ou desenhar a esgrafito.
\section{Esgrafito}
\begin{itemize}
\item {Grp. gram.:m.}
\end{itemize}
\begin{itemize}
\item {Proveniência:(It. \textunderscore sgraffito\textunderscore )}
\end{itemize}
Gênero de pintura, hoje desusada, que imita baixos relevos.
\section{Esgraminha}
\begin{itemize}
\item {Grp. gram.:f.}
\end{itemize}
Acto ou effeito de esgraminhar. Cf. Herculano, \textunderscore Opúsc.\textunderscore , IV, 143.
\section{Esgraminhador}
\begin{itemize}
\item {Grp. gram.:m.}
\end{itemize}
Utensilio de ferro, com que se esgraminha a terra lavrada.
\section{Esgraminhar}
\begin{itemize}
\item {Grp. gram.:v. t.}
\end{itemize}
Tirar a grama a (o terreno lavrado).
\section{Esgravanada}
\begin{itemize}
\item {Grp. gram.:f.}
\end{itemize}
\begin{itemize}
\item {Utilização:Prov.}
\end{itemize}
\begin{itemize}
\item {Utilização:trasm.}
\end{itemize}
O mesmo que \textunderscore esgaravanada\textunderscore .
\section{Esgravatar}
\begin{itemize}
\item {Grp. gram.:v. i.}
\end{itemize}
O mesmo que \textunderscore esgaravatar\textunderscore .
\section{Esgravatear}
\begin{itemize}
\item {Grp. gram.:v. t.}
\end{itemize}
O mesmo que \textunderscore esgravatar\textunderscore . Cf. Garret, \textunderscore Romanceiro\textunderscore , I, 338.
\section{Esgravelhar}
\begin{itemize}
\item {Grp. gram.:v. i.}
\end{itemize}
O mesmo que \textunderscore esgarabulhar\textunderscore .
\section{Esgravêlho}
\begin{itemize}
\item {Grp. gram.:m.}
\end{itemize}
\begin{itemize}
\item {Proveniência:(De \textunderscore esgravelhar\textunderscore )}
\end{itemize}
Criança traquina.
\section{Esgrilar}
\begin{itemize}
\item {Grp. gram.:v. t.}
\end{itemize}
\begin{itemize}
\item {Utilização:Pop.}
\end{itemize}
Applicar muito (a vista, os olhos), para vêr melhor ao longe.
\section{Esgrima}
\begin{itemize}
\item {Grp. gram.:f.}
\end{itemize}
Jôgo de armas brancas.
Acto de esgrimir.
\section{Esgrimaça}
\begin{itemize}
\item {Grp. gram.:f.}
\end{itemize}
\begin{itemize}
\item {Utilização:Des.}
\end{itemize}
Rodeio ou circunlóquio, para embair.
(Relaciona-se com \textunderscore esgrimir\textunderscore ?)
\section{Esgrimideiro}
\begin{itemize}
\item {Grp. gram.:m.  e  adj.}
\end{itemize}
O mesmo que \textunderscore esgrimidor\textunderscore . Cf. \textunderscore Fenix Renasc.\textunderscore , III, 303.
\section{Esgrimidor}
\begin{itemize}
\item {Grp. gram.:adj.}
\end{itemize}
\begin{itemize}
\item {Grp. gram.:M.}
\end{itemize}
Que esgrime.
Aquelle que esgrime.
Esgrimista.
\section{Esgrimidura}
\begin{itemize}
\item {Grp. gram.:f.}
\end{itemize}
Acto de esgrimir.
\section{Esgrimir}
\begin{itemize}
\item {Grp. gram.:v. t.}
\end{itemize}
\begin{itemize}
\item {Grp. gram.:V. i.}
\end{itemize}
\begin{itemize}
\item {Utilização:Fig.}
\end{itemize}
\begin{itemize}
\item {Proveniência:(Do germ. \textunderscore skirm\textunderscore )}
\end{itemize}
Manejar (armas brancas).
Vibrar.
Lutar.
Jogar armas.
Esforçar-se.
\section{Esgrimista}
\begin{itemize}
\item {Grp. gram.:m.  e  f.}
\end{itemize}
\begin{itemize}
\item {Proveniência:(De \textunderscore esgrimir\textunderscore )}
\end{itemize}
Pessôa que esgrime, ou que é perita em esgrima.
\section{Esgrouviado}
\begin{itemize}
\item {Grp. gram.:adj.}
\end{itemize}
Magro e alto como o grou.
Que tem o cabello em desalinho.
\section{Esgrouviar}
\begin{itemize}
\item {Grp. gram.:v. t.}
\end{itemize}
Desalinhar; esparralhar. Cf. Lapa, \textunderscore Proc. de Vin.\textunderscore , 37.
\section{Esgrouvinhado}
\begin{itemize}
\item {Grp. gram.:adj.}
\end{itemize}
(V.esgrouviado)
\section{Esguardar}
\begin{itemize}
\item {Grp. gram.:v. t.}
\end{itemize}
\begin{itemize}
\item {Utilização:Des.}
\end{itemize}
\begin{itemize}
\item {Proveniência:(De \textunderscore esguardo\textunderscore )}
\end{itemize}
Têr em consideração.
Respeitar, tomar em conta.
Notar.
\section{Esguardo}
\begin{itemize}
\item {Grp. gram.:m.}
\end{itemize}
Acto ou effeito de esguardar.
(Cp. it. \textunderscore sguardo\textunderscore )
\section{Esguasar}
\begin{itemize}
\item {Grp. gram.:v. t.}
\end{itemize}
\begin{itemize}
\item {Utilização:Ant.}
\end{itemize}
Vadear, passar além de (um rio).
\section{Esguedelhar}
\begin{itemize}
\item {Grp. gram.:v. t.}
\end{itemize}
\begin{itemize}
\item {Proveniência:(De \textunderscore guedelha\textunderscore )}
\end{itemize}
Enredar, desgrenhar, (cabello).
Pôr em desalinho (o cabello).
\section{Esgueirão}
\begin{itemize}
\item {Grp. gram.:m.}
\end{itemize}
\begin{itemize}
\item {Proveniência:(De \textunderscore Esgueira\textunderscore , n. p.)}
\end{itemize}
Espécie de barco, terminado em dois bicos e muito usado na ria de Aveiro.
Homem de Esgueira.
\section{Esgueirar}
\begin{itemize}
\item {Grp. gram.:v. t.}
\end{itemize}
\begin{itemize}
\item {Grp. gram.:V. p.}
\end{itemize}
Desviar.
Retirar-se sorrateiramente; safar-se.
\section{Esgueiriço}
\begin{itemize}
\item {Grp. gram.:adj.}
\end{itemize}
\begin{itemize}
\item {Utilização:Ant.}
\end{itemize}
\begin{itemize}
\item {Proveniência:(De \textunderscore esgueirar\textunderscore )}
\end{itemize}
Arredio; insociável.
\section{Esgueirôa}
\begin{itemize}
\item {Grp. gram.:f.}
\end{itemize}
\begin{itemize}
\item {Utilização:Prov.}
\end{itemize}
\begin{itemize}
\item {Utilização:dur.}
\end{itemize}
\begin{itemize}
\item {Proveniência:(De \textunderscore esgueirão\textunderscore )}
\end{itemize}
Mulher, que é natural de Esgueira:«\textunderscore raparigas de Buarcos, arredai-vos para o lado, que lá vêm as esgueirôas, c'o ramo dependurado.\textunderscore »\textunderscore Canção popular\textunderscore .
\section{Esguelha}
\begin{itemize}
\item {fónica:guê}
\end{itemize}
\begin{itemize}
\item {Grp. gram.:f.}
\end{itemize}
\begin{itemize}
\item {Proveniência:(Do gr. \textunderscore skolios\textunderscore ?)}
\end{itemize}
Través; obliquidade.
Soslaio.
\section{Esguelhadamente}
\begin{itemize}
\item {Grp. gram.:adv.}
\end{itemize}
De esguelha.
De través, de soslaio.
\section{Esguelhadas}
\begin{itemize}
\item {Grp. gram.:f. pl.}
\end{itemize}
\begin{itemize}
\item {Utilização:Ant.}
\end{itemize}
O mesmo que \textunderscore esguelha\textunderscore .
\section{Esguelhão}
\begin{itemize}
\item {Grp. gram.:m.}
\end{itemize}
\begin{itemize}
\item {Utilização:Ant.}
\end{itemize}
\begin{itemize}
\item {Proveniência:(De \textunderscore esguelha\textunderscore )}
\end{itemize}
Flanco; ilharga.
\section{Esguelhar}
\begin{itemize}
\item {Grp. gram.:v. t.}
\end{itemize}
\begin{itemize}
\item {Proveniência:(De \textunderscore esguelha\textunderscore )}
\end{itemize}
Torcer.
Pôr obliquamente.
\section{Esguião}
\begin{itemize}
\item {Grp. gram.:m.}
\end{itemize}
Tecido fino de linho ou algodão.
\section{Esguiar}
\begin{itemize}
\item {Grp. gram.:v. t.}
\end{itemize}
\begin{itemize}
\item {Proveniência:(De \textunderscore guia\textunderscore )}
\end{itemize}
Ferir (uma ave) na ponta da asa, inutilizando-lhe a penna de guia.
\section{Esguichada}
\begin{itemize}
\item {Grp. gram.:f.}
\end{itemize}
Acto ou effeito de esguichar.
\section{Esguichadela}
\begin{itemize}
\item {Grp. gram.:f.}
\end{itemize}
Acto ou effeito de esguichar.
\section{Esguichar}
\begin{itemize}
\item {Grp. gram.:v. t.}
\end{itemize}
\begin{itemize}
\item {Grp. gram.:V. i.}
\end{itemize}
\begin{itemize}
\item {Proveniência:(De \textunderscore esguicho\textunderscore )}
\end{itemize}
Expellir com fôrça por um tubo ou orifício (um liquido).
Sair com ímpeto um líquido por abertura estreita.
Sair em repuxo.
\section{Esguiche}
\begin{itemize}
\item {Grp. gram.:m.}
\end{itemize}
\begin{itemize}
\item {Utilização:Prov.}
\end{itemize}
O mesmo que \textunderscore esguicho\textunderscore .
\section{Esguicho}
\begin{itemize}
\item {Grp. gram.:m.}
\end{itemize}
\begin{itemize}
\item {Utilização:T. de Aveiro}
\end{itemize}
Jacto de um líquido.
Seringa de carnaval.
Acto de esguichar.
Bateira, o mesmo que \textunderscore chinchorra\textunderscore .
\section{Esguilhar}
\begin{itemize}
\item {Grp. gram.:v. t.}
\end{itemize}
(V.esguelhar). Cf. B. Pereira, \textunderscore Prosódia\textunderscore , vb. \textunderscore hypodra\textunderscore .
\section{Esguio}
\begin{itemize}
\item {Grp. gram.:adj.}
\end{itemize}
Alto e delgado: \textunderscore pinheiro esguio\textunderscore .
Comprido e delgado.
E diz-se do fato muito chegado ao corpo.
\section{Esguitar}
\begin{itemize}
\item {Grp. gram.:v. t.}
\end{itemize}
\begin{itemize}
\item {Utilização:Prov.}
\end{itemize}
\begin{itemize}
\item {Utilização:minh.}
\end{itemize}
\begin{itemize}
\item {Proveniência:(De \textunderscore guíta\textunderscore )}
\end{itemize}
Dividir em leiras (um campo).
\section{Esguncho}
\begin{itemize}
\item {Grp. gram.:m.}
\end{itemize}
Espécie de pá, com que se aguam exteriormente os barcos.
\section{Esgurejar}
\begin{itemize}
\item {Grp. gram.:v. i.}
\end{itemize}
\begin{itemize}
\item {Utilização:Bras. do N}
\end{itemize}
O mesmo que escurejar^1.
\section{Esgúvio}
\begin{itemize}
\item {Grp. gram.:adj.}
\end{itemize}
\begin{itemize}
\item {Utilização:Prov.}
\end{itemize}
\begin{itemize}
\item {Utilização:trasm.}
\end{itemize}
Escorregadio, como o peixe na água.
\section{Ésipo}
\begin{itemize}
\item {Grp. gram.:m.}
\end{itemize}
\begin{itemize}
\item {Proveniência:(Do gr. \textunderscore oisupos\textunderscore )}
\end{itemize}
\begin{itemize}
\item {Grp. gram.:m.}
\end{itemize}
\begin{itemize}
\item {Proveniência:(Do gr. \textunderscore oisupe\textunderscore )}
\end{itemize}
Suarda, que se extrai da lan.
Suarda ou substância gordurosa da lan das ovelhas.
Cosmético, feito com aquela substância.
\section{Eslabão}
\begin{itemize}
\item {Grp. gram.:m.}
\end{itemize}
Tumor nos joelhos das cavalgaduras, procedente de contusão.
(Cast. \textunderscore eslabon\textunderscore )
\section{Esladrôa}
\begin{itemize}
\item {Grp. gram.:f.}
\end{itemize}
O mesmo que \textunderscore esladroamento\textunderscore . Cf. \textunderscore Bibl. da G. do Campo\textunderscore , 309 e 356.
\section{Esladroamento}
\begin{itemize}
\item {Grp. gram.:m.}
\end{itemize}
Acto de esladroar.
\section{Esladroar}
\begin{itemize}
\item {Grp. gram.:v. t.}
\end{itemize}
\begin{itemize}
\item {Proveniência:(De \textunderscore ladrão\textunderscore )}
\end{itemize}
Tirar os renovos ou rebentos supérfluos a.
\section{Eslagartador}
\begin{itemize}
\item {Grp. gram.:m.}
\end{itemize}
\begin{itemize}
\item {Proveniência:(De \textunderscore eslagartar\textunderscore )}
\end{itemize}
Aquelle que eslagarta.
O mesmo que \textunderscore cotinga\textunderscore .
\section{Eslagartar}
\begin{itemize}
\item {Grp. gram.:v. t.}
\end{itemize}
Limpar de lagartas.
\section{Eslavaçado}
\begin{itemize}
\item {Grp. gram.:adj.}
\end{itemize}
\begin{itemize}
\item {Utilização:Prov.}
\end{itemize}
\begin{itemize}
\item {Utilização:beir.}
\end{itemize}
Diz-se do estômago, quando se não acha sabor á comida.
\section{Eslávico}
\begin{itemize}
\item {Grp. gram.:adj.}
\end{itemize}
Que pertence aos Eslavos.
\section{Eslavismo}
\begin{itemize}
\item {Grp. gram.:m.}
\end{itemize}
\begin{itemize}
\item {Proveniência:(De \textunderscore eslavo\textunderscore )}
\end{itemize}
Política, que tende ao agrupamento dos Eslavos numa só nação e aos progressos desta.
\section{Eslavo}
\begin{itemize}
\item {Grp. gram.:adj.}
\end{itemize}
\begin{itemize}
\item {Grp. gram.:M. pl.}
\end{itemize}
\begin{itemize}
\item {Proveniência:(Do lat. \textunderscore slavi\textunderscore )}
\end{itemize}
Relativo aos Eslavos; eslávico.
Grande familia ethnográphica, a mais oriental da Europa.
\section{Eslazeirado}
\begin{itemize}
\item {Grp. gram.:adj.}
\end{itemize}
\begin{itemize}
\item {Utilização:Prov.}
\end{itemize}
\begin{itemize}
\item {Utilização:trasm.}
\end{itemize}
Esfomeado, cheio de lazeira.
\section{Eslinga}
\begin{itemize}
\item {Grp. gram.:f.}
\end{itemize}
\begin{itemize}
\item {Proveniência:(Do ant. al. \textunderscore slinga\textunderscore )}
\end{itemize}
Cabo, com que se levantam pesos a bordo.
\section{Eslingar}
\begin{itemize}
\item {Grp. gram.:v. t.}
\end{itemize}
Levantar (fardos) por meio da eslinga.
\section{Esloana}
\begin{itemize}
\item {Grp. gram.:f.}
\end{itemize}
Gênero de plantas tiliáceas.
\section{Esloráquia}
\begin{itemize}
\item {Grp. gram.:m.}
\end{itemize}
Lingua da Bohêmia.
\section{Eslúvio}
\begin{itemize}
\item {Grp. gram.:adj.}
\end{itemize}
\begin{itemize}
\item {Utilização:Prov.}
\end{itemize}
\begin{itemize}
\item {Utilização:trasm.}
\end{itemize}
O mesmo que \textunderscore esgúvio\textunderscore .
\section{Esmadrigar}
\begin{itemize}
\item {Grp. gram.:v. t.}
\end{itemize}
\begin{itemize}
\item {Grp. gram.:V. i.}
\end{itemize}
\begin{itemize}
\item {Proveniência:(Do lat. \textunderscore matrix\textunderscore )}
\end{itemize}
Tirar do rebanho.
Tresmalhar.
Tresmalhar-se.
\section{Esmaecer}
\begin{itemize}
\item {fónica:ma-e}
\end{itemize}
\begin{itemize}
\item {Grp. gram.:v. i.}
\end{itemize}
Desmair.
Enfraquecer.
Desvanecer-se.
(Por \textunderscore desmaiecer\textunderscore , de \textunderscore desmaio\textunderscore )
\section{Esmaecimento}
\begin{itemize}
\item {fónica:ma-e}
\end{itemize}
\begin{itemize}
\item {Grp. gram.:m.}
\end{itemize}
Acto de esmaecer.
\section{Esmagação}
\begin{itemize}
\item {Grp. gram.:m.}
\end{itemize}
O mesmo que \textunderscore esmagadura\textunderscore .
\section{Esmagachar}
\begin{itemize}
\item {Grp. gram.:v. t.}
\end{itemize}
\begin{itemize}
\item {Utilização:Pop.}
\end{itemize}
\begin{itemize}
\item {Proveniência:(Do rad. de \textunderscore esmagar\textunderscore )}
\end{itemize}
Esmagar muito.
\section{Esmagador}
\begin{itemize}
\item {Grp. gram.:adj.}
\end{itemize}
\begin{itemize}
\item {Grp. gram.:M.}
\end{itemize}
Que esmaga.
Aquelle que esmaga.
Máquina, para esmagar uvas no lagar ou no balceiro.
\section{Esmagadura}
\begin{itemize}
\item {Grp. gram.:f.}
\end{itemize}
Acto ou effeito de esmagar.
\section{Esmagagem}
\begin{itemize}
\item {Grp. gram.:f.}
\end{itemize}
\begin{itemize}
\item {Utilização:inútil}
\end{itemize}
\begin{itemize}
\item {Utilização:Neol.}
\end{itemize}
O mesmo que \textunderscore esmagamento\textunderscore .
\section{Esmagamento}
\begin{itemize}
\item {Grp. gram.:m.}
\end{itemize}
O mesmo que \textunderscore esmagadura\textunderscore .
\section{Esmagar}
\begin{itemize}
\item {Grp. gram.:v. t.}
\end{itemize}
\begin{itemize}
\item {Utilização:Fig.}
\end{itemize}
\begin{itemize}
\item {Proveniência:(Do lat. hyp. \textunderscore exmaccare\textunderscore , seg. Körting)}
\end{itemize}
Calcar; comprimir muito: \textunderscore a carroça esmagou-lhe um pé\textunderscore .
Abater.
Quebrar.
Destruir completamente os argumentos de.
Vencer.
Embatucar.
Affligir; consumir de mágua.
\section{Esmaga-vides}
\begin{itemize}
\item {Grp. gram.:m.}
\end{itemize}
Apparelho agrícola, o mesmo que \textunderscore corta-vides\textunderscore .
\section{Esmagriçado}
\begin{itemize}
\item {Grp. gram.:adj.}
\end{itemize}
\begin{itemize}
\item {Proveniência:(De \textunderscore esmagriçar-se\textunderscore )}
\end{itemize}
Emmagrecido.
Esgrouviado. Cf. Camillo, \textunderscore Brasileira\textunderscore , 266.
\section{Esmagriçar-se}
\begin{itemize}
\item {Grp. gram.:v. p.}
\end{itemize}
Mostrar-se magro:«\textunderscore sob as riquezas... esmagriça-va-se a figura do asceta\textunderscore ». \textunderscore Primeiro de Janeiro\textunderscore , de 22-III-900.
\section{Esmaiar}
\begin{itemize}
\item {Grp. gram.:v. i.  e  p.}
\end{itemize}
O mesmo que \textunderscore desmaiar\textunderscore .
\section{Esmaio}
\begin{itemize}
\item {Grp. gram.:m.}
\end{itemize}
\begin{itemize}
\item {Utilização:Ant.}
\end{itemize}
(V.desmaio)
\section{Esmaleitado}
\begin{itemize}
\item {Grp. gram.:adj.}
\end{itemize}
Que padece maleitas.
\section{Esmalhar}
\begin{itemize}
\item {Grp. gram.:v. t.}
\end{itemize}
\begin{itemize}
\item {Utilização:Ant.}
\end{itemize}
Cortar as malhas de (armaduras).
\section{Esmalmado}
\begin{itemize}
\item {Grp. gram.:adj.}
\end{itemize}
\begin{itemize}
\item {Utilização:Chul.}
\end{itemize}
\begin{itemize}
\item {Proveniência:(De \textunderscore esmalmar\textunderscore )}
\end{itemize}
Indolente, molle.
\section{Esmalmar}
\begin{itemize}
\item {Grp. gram.:v. t.}
\end{itemize}
\begin{itemize}
\item {Utilização:Prov.}
\end{itemize}
\begin{itemize}
\item {Utilização:dur.}
\end{itemize}
Estafar, cansar. (Colhido na Costa-Nova)
(Por \textunderscore esalmar\textunderscore , de \textunderscore alma\textunderscore )
\section{Esmaltado}
\begin{itemize}
\item {Grp. gram.:adj.}
\end{itemize}
\begin{itemize}
\item {Proveniência:(De \textunderscore esmaltar\textunderscore )}
\end{itemize}
Que tem esmalte: \textunderscore loiça esmaltada\textunderscore .
Matizado.
\section{Esmaltador}
\begin{itemize}
\item {Grp. gram.:adj.}
\end{itemize}
\begin{itemize}
\item {Grp. gram.:M.}
\end{itemize}
Que esmalta.
Aquelle que esmalta por offício.
\section{Esmaltar}
\begin{itemize}
\item {Grp. gram.:v. t.}
\end{itemize}
\begin{itemize}
\item {Utilização:Fig.}
\end{itemize}
Applicar o esmalte a.
Matizar.
Adornar.
Tornar variegado.
\section{Esmalte}
\begin{itemize}
\item {Grp. gram.:m.}
\end{itemize}
\begin{itemize}
\item {Utilização:Fig.}
\end{itemize}
\begin{itemize}
\item {Grp. gram.:Pl.}
\end{itemize}
\begin{itemize}
\item {Utilização:Heráld.}
\end{itemize}
\begin{itemize}
\item {Proveniência:(Do b. lat. \textunderscore smaltum\textunderscore )}
\end{itemize}
Substância vitrificável, que, por meio da fusão, se applica sôbre metaes ou porcelana.
Substância polida, que reveste a corôa dos dentes.
Brilho.
Côres variegadas.
Qualquer substância ou superfície polida e branca.
Metaes, côres, arminhos e veiros, que se empregam no campo do escudo ou nas suas partes exteriôres. Cf. L. Ribeiro, \textunderscore Trat. de Armaria\textunderscore .
\section{Esmaltina}
\begin{itemize}
\item {Grp. gram.:f.}
\end{itemize}
\begin{itemize}
\item {Proveniência:(De \textunderscore esmalte\textunderscore )}
\end{itemize}
Cobalto arsenical.
Mineral pouco brilhante, que crystalliza em cubos ou octaedros regulares.
\section{Esmamaçada}
\begin{itemize}
\item {Grp. gram.:adj. f.}
\end{itemize}
O mesmo que \textunderscore esmamalhada\textunderscore . Cf. Camillo, \textunderscore Brasileira\textunderscore , 194.
\section{Esmamalhada}
\begin{itemize}
\item {Grp. gram.:adj. f.}
\end{itemize}
Diz-se da mulher desleixada, que tem grandes mamas, molles e pendentes.
\section{Esmamonar}
\begin{itemize}
\item {Grp. gram.:v. t.}
\end{itemize}
\begin{itemize}
\item {Utilização:Prov.}
\end{itemize}
\begin{itemize}
\item {Utilização:trasm.}
\end{itemize}
\begin{itemize}
\item {Proveniência:(De \textunderscore mamão\textunderscore ^1)}
\end{itemize}
Cortar os rebentos ou mamões a (pereiras ou videiras).
\section{Esmaniado}
\begin{itemize}
\item {Grp. gram.:adj.}
\end{itemize}
Que esmania.
\section{Esmaniar}
\begin{itemize}
\item {Grp. gram.:v. i.}
\end{itemize}
Têr manias.
Proceder doidamente.
\section{Esmanjar}
\textunderscore v. t.\textunderscore  (e der.)
O mesmo ou melhor que \textunderscore esbanjar\textunderscore , etc.
\section{Esmante}
\begin{itemize}
\item {Grp. gram.:adj.}
\end{itemize}
\begin{itemize}
\item {Utilização:Ant.}
\end{itemize}
\begin{itemize}
\item {Proveniência:(De \textunderscore esmar\textunderscore )}
\end{itemize}
Que calcula ou avalia.
\section{Esmar}
\begin{itemize}
\item {Grp. gram.:v. t.}
\end{itemize}
\begin{itemize}
\item {Proveniência:(Do lat. \textunderscore aestimare\textunderscore )}
\end{itemize}
Calcular, computar.
Conjecturar.
\section{Esmaranhar}
\begin{itemize}
\item {Grp. gram.:v. t.}
\end{itemize}
\begin{itemize}
\item {Utilização:Prov.}
\end{itemize}
\begin{itemize}
\item {Utilização:beir.}
\end{itemize}
Desfazer, desmanchar.
(Cp. \textunderscore desemmaranhar\textunderscore )
\section{Esmaravalhar}
\begin{itemize}
\item {Grp. gram.:v. t.}
\end{itemize}
\begin{itemize}
\item {Utilização:Prov.}
\end{itemize}
\begin{itemize}
\item {Utilização:alg.}
\end{itemize}
Espalhar pela terra (a cinza das moreias), antes da sementeira.
(Cp. \textunderscore maravalha\textunderscore )
\section{Esmarcar}
\begin{itemize}
\item {Grp. gram.:v. t.}
\end{itemize}
\begin{itemize}
\item {Utilização:Ant.}
\end{itemize}
\begin{itemize}
\item {Proveniência:(De \textunderscore marcar\textunderscore )}
\end{itemize}
Calcular; medir. Cf. Pant. de Aveiro, \textunderscore Itiner.\textunderscore , 258., v. II, (2.^a ed.).
\section{Esmarelido}
\begin{itemize}
\item {Grp. gram.:adj.}
\end{itemize}
\begin{itemize}
\item {Utilização:Des.}
\end{itemize}
O mesmo que \textunderscore amarelento\textunderscore .
\section{Esmarmoirar}
\begin{itemize}
\item {Grp. gram.:v. i.}
\end{itemize}
\begin{itemize}
\item {Utilização:Prov.}
\end{itemize}
\begin{itemize}
\item {Utilização:trasm.}
\end{itemize}
Desfallecer depressa, com fome ou sêde.
\section{Esmarrido}
\begin{itemize}
\item {Grp. gram.:adj.}
\end{itemize}
\begin{itemize}
\item {Utilização:Des.}
\end{itemize}
\begin{itemize}
\item {Proveniência:(Do it. \textunderscore smarrito\textunderscore )}
\end{itemize}
Que não tem vigor.
Que perdeu a seiva.
Estéril:«\textunderscore ...terreno escabroso e esmarrido.\textunderscore »Rui Barb., \textunderscore Réplica\textunderscore , 4.
\section{Esmarroar}
\begin{itemize}
\item {Grp. gram.:v. t.}
\end{itemize}
\begin{itemize}
\item {Utilização:Prov.}
\end{itemize}
\begin{itemize}
\item {Utilização:trasm.}
\end{itemize}
\begin{itemize}
\item {Proveniência:(De \textunderscore marrão\textunderscore ^2)}
\end{itemize}
Achatar ou partir contra uma substância dura.
Esmurrar; partir o bico a.
\section{Esmastreado}
\begin{itemize}
\item {Grp. gram.:adj.}
\end{itemize}
\begin{itemize}
\item {Utilização:Prov.}
\end{itemize}
\begin{itemize}
\item {Utilização:alg.}
\end{itemize}
Enfraquecido.
(Cp. \textunderscore desmastrear\textunderscore )
\section{Esmear}
\begin{itemize}
\item {Grp. gram.:v. t.}
\end{itemize}
Partir ou serrar longitudinalmente, pelo meio.
\section{Esmechada}
\begin{itemize}
\item {Grp. gram.:f.}
\end{itemize}
\begin{itemize}
\item {Utilização:Pop.}
\end{itemize}
\begin{itemize}
\item {Proveniência:(De \textunderscore esmechar\textunderscore )}
\end{itemize}
Ferida na cabeça.
\section{Esmechadura}
\begin{itemize}
\item {Grp. gram.:f.}
\end{itemize}
(V.esmechada)
\section{Esmechar}
\begin{itemize}
\item {Grp. gram.:v. t.}
\end{itemize}
Ferir a cabeça de.
Ferir.
(Por \textunderscore esbrechar\textunderscore , de \textunderscore brecha\textunderscore ?)
\section{Esmechar}
\begin{itemize}
\item {Grp. gram.:v. i.}
\end{itemize}
\begin{itemize}
\item {Utilização:Des.}
\end{itemize}
\begin{itemize}
\item {Proveniência:(De \textunderscore mecha\textunderscore )}
\end{itemize}
Estar muito quente, abrasar, (falando-se do sol). Cf. R. Lobo, \textunderscore Côrte na Aldeia\textunderscore , 112.
\section{Esmegma}
\begin{itemize}
\item {Grp. gram.:m.}
\end{itemize}
\begin{itemize}
\item {Proveniência:(Gr. \textunderscore smegma\textunderscore )}
\end{itemize}
Matéria esbranquiçada, com a apparência de sabão desfeito, que se junta nas dobras dos órgãos genitaes.
\section{Esmeiar}
\begin{itemize}
\item {Grp. gram.:v. t.}
\end{itemize}
Partir ou serrar longitudinalmente, pelo meio.
\section{Esmelmar}
\begin{itemize}
\item {Grp. gram.:v. i.}
\end{itemize}
\begin{itemize}
\item {Utilização:Prov.}
\end{itemize}
\begin{itemize}
\item {Utilização:trasm.}
\end{itemize}
Encolher-se (o pano), não chegar á medida.
(Talvez do lat. hyp. \textunderscore exminimare\textunderscore , do lat. \textunderscore minimus\textunderscore )
\section{Esmelodrar}
\begin{itemize}
\item {Grp. gram.:v. t.}
\end{itemize}
\begin{itemize}
\item {Utilização:Prov.}
\end{itemize}
\begin{itemize}
\item {Utilização:trasm.}
\end{itemize}
Contundir.
Ferir; escoriar.
\section{Esmenar}
\begin{itemize}
\item {Grp. gram.:v. t.}
\end{itemize}
\begin{itemize}
\item {Utilização:Prov.}
\end{itemize}
\begin{itemize}
\item {Utilização:minh.}
\end{itemize}
O mesmo que \textunderscore escarolar\textunderscore .
\section{Esmensurado}
\begin{itemize}
\item {Grp. gram.:adj.}
\end{itemize}
\begin{itemize}
\item {Utilização:Des.}
\end{itemize}
\begin{itemize}
\item {Proveniência:(Do lat. \textunderscore mensura\textunderscore )}
\end{itemize}
O mesmo que \textunderscore desmedido\textunderscore , enorme.
\section{Esmeradamente}
\begin{itemize}
\item {Grp. gram.:adv.}
\end{itemize}
\begin{itemize}
\item {Proveniência:(De \textunderscore esmerado\textunderscore )}
\end{itemize}
Com esmêro.
\section{Esmerado}
\begin{itemize}
\item {Grp. gram.:adj.}
\end{itemize}
\begin{itemize}
\item {Proveniência:(De \textunderscore esmerar\textunderscore )}
\end{itemize}
Que tem esmêro.
Correcto.
Distinto.
\section{Esmeralda}
\begin{itemize}
\item {Grp. gram.:f.}
\end{itemize}
\begin{itemize}
\item {Utilização:Ext.}
\end{itemize}
\begin{itemize}
\item {Proveniência:(Do lat. \textunderscore hyp. smaragda\textunderscore )}
\end{itemize}
Pedra preciosa, ordinariamente verde.
A côr da esmeralda.
\section{Esmeralda}
\begin{itemize}
\item {Grp. gram.:f.}
\end{itemize}
Uma das variedades da ave-do-paraíso, (\textunderscore paradisea apoda\textunderscore ).
(Cp. \textunderscore esmeralda\textunderscore ^1)
\section{Esmeraldear}
\begin{itemize}
\item {Grp. gram.:v. t.}
\end{itemize}
Dar côr de esmeralda a.
Tornar esverdeado.
\section{Esmeraldino}
\begin{itemize}
\item {Grp. gram.:adj.}
\end{itemize}
\begin{itemize}
\item {Proveniência:(De \textunderscore esmeralda\textunderscore ^1)}
\end{itemize}
Que tem côr de esmeralda.
Verde: \textunderscore paisagem esmeraldina\textunderscore .
\section{Esmerar}
\begin{itemize}
\item {Grp. gram.:v. t.}
\end{itemize}
\begin{itemize}
\item {Proveniência:(Lat. hyp. \textunderscore exmerare\textunderscore )}
\end{itemize}
Mostrar esmêro em.
Tornar perfeito, distinto, correcto.
\section{Esmeril}
\begin{itemize}
\item {Grp. gram.:m.}
\end{itemize}
\begin{itemize}
\item {Proveniência:(Lat. hyp. \textunderscore smirilis\textunderscore , seg. Körting)}
\end{itemize}
Pedra muito dura que, desfeita em pó, serve para polir o vidro, metaes, etc., e para tornar fosco o mesmo vidro.
Antiga peça de artilharia.
\section{Esmeriladeira}
\begin{itemize}
\item {Grp. gram.:f.}
\end{itemize}
Um dos maquinismos da tecelagem, nas fábricas de tecidos. Cf. \textunderscore Inquér. Industr.\textunderscore , p. II, l. III, 215.
\section{Esmerilador}
\begin{itemize}
\item {Grp. gram.:m.}
\end{itemize}
Aquelle que esmerila.
\section{Esmerilar}
\begin{itemize}
\item {Grp. gram.:v. t.}
\end{itemize}
Polir ou despolir, por meio de esmeril.
\section{Esmerilhação}
\begin{itemize}
\item {Grp. gram.:f.}
\end{itemize}
Acto de esmerilhar.
\section{Esmerilhador}
\begin{itemize}
\item {Grp. gram.:m.  e  adj.}
\end{itemize}
O que esmerilha.
\section{Esmerilhão}
\begin{itemize}
\item {Grp. gram.:m.}
\end{itemize}
\begin{itemize}
\item {Proveniência:(Do lat. \textunderscore merula\textunderscore )}
\end{itemize}
Pequena ave de rapina.
\section{Esmerilhão}
\begin{itemize}
\item {Grp. gram.:m.}
\end{itemize}
\begin{itemize}
\item {Proveniência:(De \textunderscore esmeril\textunderscore )}
\end{itemize}
Antiga peça de artilharia, maior que a chamada esmeril.
Espingarda comprida, de grande alcance:«\textunderscore ...um esmerilhão ou bacamarte\textunderscore ». Camillo, \textunderscore Caveira\textunderscore , 458.
\section{Esmerilhar}
\begin{itemize}
\item {Grp. gram.:v. t.}
\end{itemize}
\begin{itemize}
\item {Utilização:Fig.}
\end{itemize}
\begin{itemize}
\item {Proveniência:(De \textunderscore esmeril\textunderscore )}
\end{itemize}
O mesmo que \textunderscore esmerilar\textunderscore .
O mesmo que \textunderscore esmerar\textunderscore .
Esquadrinhar; procurar minuciosamente.
\section{Esmerim}
\begin{itemize}
\item {Grp. gram.:m.}
\end{itemize}
\begin{itemize}
\item {Utilização:Ant.}
\end{itemize}
O mesmo que \textunderscore esmeril\textunderscore . Cf. Fern. Oliv., \textunderscore Arte da Guerra do Mar\textunderscore , 47, v.^o
\section{Esmermar}
\begin{itemize}
\item {Grp. gram.:v. t.}
\end{itemize}
\begin{itemize}
\item {Utilização:Prov.}
\end{itemize}
\begin{itemize}
\item {Utilização:trasm.}
\end{itemize}
O mesmo que \textunderscore esmelmar\textunderscore .
\section{Esmêro}
\begin{itemize}
\item {Grp. gram.:m.}
\end{itemize}
\begin{itemize}
\item {Proveniência:(Lat. hyp. \textunderscore exmerum\textunderscore )}
\end{itemize}
Cuidado extremo, no trabalho ou no vestuário.
Primor.
Correcção.
\section{Esmetiar}
\begin{itemize}
\item {fónica:mé}
\end{itemize}
\begin{itemize}
\item {Grp. gram.:v. t.}
\end{itemize}
\begin{itemize}
\item {Utilização:Prov.}
\end{itemize}
\begin{itemize}
\item {Utilização:alg.}
\end{itemize}
Dividir ao meio.
(Por \textunderscore esmediar\textunderscore . Cp. \textunderscore médio\textunderscore )
\section{Esmicha}
\begin{itemize}
\item {Grp. gram.:f.}
\end{itemize}
\begin{itemize}
\item {Utilização:Prov.}
\end{itemize}
\begin{itemize}
\item {Utilização:trasm.}
\end{itemize}
\begin{itemize}
\item {Proveniência:(De \textunderscore esmichar\textunderscore ^1)}
\end{itemize}
A fôrça do grande calor.
\section{Esmichar}
\begin{itemize}
\item {Grp. gram.:v. i.}
\end{itemize}
\begin{itemize}
\item {Utilização:Prov.}
\end{itemize}
\begin{itemize}
\item {Utilização:trasm.}
\end{itemize}
Haver calor intensíssimo.
(Cp. \textunderscore esmechar\textunderscore ^2)
\section{Esmichar}
\begin{itemize}
\item {Grp. gram.:v. i.}
\end{itemize}
\begin{itemize}
\item {Utilização:Fam.}
\end{itemize}
\begin{itemize}
\item {Grp. gram.:V. t.}
\end{itemize}
Espichar a canela, morrer:«\textunderscore o pobre burro esmicha.\textunderscore »Filinto, XII, 80.
Fazer morrer. Cp. Filinto, XIII, 58.
(Por \textunderscore espichar\textunderscore ?)
\section{Esmidélia}
\begin{itemize}
\item {Grp. gram.:f.}
\end{itemize}
Gênero de plantas sapindáceas.
\section{Esmigalhado}
\begin{itemize}
\item {Grp. gram.:adj.}
\end{itemize}
Partido em migalhas.
Esfarelado.
Espedaçado.
Fragmentado.
\section{Esmigalhadura}
\begin{itemize}
\item {Grp. gram.:f.}
\end{itemize}
Acto de esmigalhar.
\section{Esmigalhar}
\begin{itemize}
\item {Grp. gram.:v. t.}
\end{itemize}
\begin{itemize}
\item {Utilização:Fig.}
\end{itemize}
\begin{itemize}
\item {Proveniência:(De \textunderscore migalha\textunderscore )}
\end{itemize}
Reduzir a migalhas; fragmentar.
Despedaçar.
Apertar muito.
Esmagar; opprimir: \textunderscore esmigalhar um braço\textunderscore .
\section{Esmijaçar}
\begin{itemize}
\item {Grp. gram.:v. i.}
\end{itemize}
\begin{itemize}
\item {Utilização:T. de Turquel}
\end{itemize}
\begin{itemize}
\item {Grp. gram.:V. p.}
\end{itemize}
\begin{itemize}
\item {Utilização:Prov.}
\end{itemize}
\begin{itemize}
\item {Utilização:dur.}
\end{itemize}
Mijar com intermittências.
Urinar pelas pernas (a égua).
\section{Esmilace}
\begin{itemize}
\item {Grp. gram.:f.}
\end{itemize}
\begin{itemize}
\item {Proveniência:(Do gr. \textunderscore ismilax\textunderscore )}
\end{itemize}
Gênero de plantas, que deu o nome ás esmiláceas.
\section{Esmiláceas}
\begin{itemize}
\item {Grp. gram.:f. pl.}
\end{itemize}
\begin{itemize}
\item {Proveniência:(De \textunderscore esmilace\textunderscore )}
\end{itemize}
Família de plantas americanas, a que pertence a salsaparrilha.
\section{Esmilacina}
\begin{itemize}
\item {Grp. gram.:f.}
\end{itemize}
\begin{itemize}
\item {Utilização:Chím.}
\end{itemize}
\begin{itemize}
\item {Proveniência:(De \textunderscore esmilace\textunderscore )}
\end{itemize}
Gênero de plantas esmiláceas.
Alcali, que se descobriu na medulla da salsaparrilha.
\section{Esmilacita}
\begin{itemize}
\item {Grp. gram.:f.}
\end{itemize}
Gênero de plantas fósseis.
\section{Esmiolar}
\begin{itemize}
\item {Grp. gram.:v. t.}
\end{itemize}
Tirar o miolo a; esmigalhar.
Desfazer.
\section{Esmírnio}
\begin{itemize}
\item {Grp. gram.:m.}
\end{itemize}
\begin{itemize}
\item {Proveniência:(Gr. \textunderscore smurnion\textunderscore )}
\end{itemize}
Planta umbelífera.
\section{Esmirrado}
\begin{itemize}
\item {Grp. gram.:adj.}
\end{itemize}
\begin{itemize}
\item {Proveniência:(De \textunderscore esmirrar-se\textunderscore )}
\end{itemize}
O mesmo que \textunderscore mirrado\textunderscore :«\textunderscore ...esmyrrado e resequido...\textunderscore »Garrett, \textunderscore D. Branca\textunderscore , 88.
\section{Esmirrar-se}
\begin{itemize}
\item {Grp. gram.:v. p.}
\end{itemize}
(V. \textunderscore mirrar\textunderscore ^1)
\section{Esmítia}
\begin{itemize}
\item {Grp. gram.:f.}
\end{itemize}
Gênero de plantas synanthéreas.
Gênero de plantas leguminosas.
\section{Esmiuçadamente}
\begin{itemize}
\item {fónica:mi-u}
\end{itemize}
\begin{itemize}
\item {Grp. gram.:adv.}
\end{itemize}
\begin{itemize}
\item {Proveniência:(De \textunderscore esmiuçar\textunderscore )}
\end{itemize}
Por miúdo, minuciosamente.
\section{Esmiuçador}
\begin{itemize}
\item {fónica:mi-u}
\end{itemize}
\begin{itemize}
\item {Grp. gram.:m.  e  adj.}
\end{itemize}
O que esmiúça.
\section{Esmiuçar}
\begin{itemize}
\item {fónica:mi-u}
\end{itemize}
\begin{itemize}
\item {Grp. gram.:v. t.}
\end{itemize}
\begin{itemize}
\item {Proveniência:(De \textunderscore miúça\textunderscore )}
\end{itemize}
Dividir em pequeninas partes.
Converter em pó.
Analysar, pesquisar, investigar.
Explicar miudamente, em todos os pormenores.
\section{Esmiudamento}
\begin{itemize}
\item {fónica:mi-u}
\end{itemize}
\begin{itemize}
\item {Grp. gram.:m.}
\end{itemize}
Acto ou effeito de esmiudar.
\section{Esmiudar}
\begin{itemize}
\item {fónica:mi-u}
\end{itemize}
\begin{itemize}
\item {Grp. gram.:v. t.}
\end{itemize}
\begin{itemize}
\item {Proveniência:(De \textunderscore miúdo\textunderscore )}
\end{itemize}
O mesmo que \textunderscore esmiuçar\textunderscore .
\section{Esmiunçar}
\begin{itemize}
\item {Grp. gram.:v. t.}
\end{itemize}
(V.esmiuçar)
\section{Esmo}
\begin{itemize}
\item {fónica:ês}
\end{itemize}
\begin{itemize}
\item {Grp. gram.:m.}
\end{itemize}
\begin{itemize}
\item {Grp. gram.:Loc. adv.}
\end{itemize}
Acto de esmar.
Cálculo aproximado.
\textunderscore A esmo\textunderscore , ao acaso; indistintamente; á tôa.
\section{Esmocadela}
\begin{itemize}
\item {Grp. gram.:f.}
\end{itemize}
Acto de esmocar.
\section{Esmocar}
\begin{itemize}
\item {Grp. gram.:v. t.}
\end{itemize}
\begin{itemize}
\item {Utilização:Pop.}
\end{itemize}
\begin{itemize}
\item {Utilização:Prov.}
\end{itemize}
\begin{itemize}
\item {Utilização:dur.}
\end{itemize}
Bater com moca.
Bater.
Quebrar com pancada um pedaço de: \textunderscore esmocar um cântaro\textunderscore .
Magoar com topada (os dedos dos pés).
\section{Esmochar}
\begin{itemize}
\item {Grp. gram.:v. t.}
\end{itemize}
\begin{itemize}
\item {Utilização:Prov.}
\end{itemize}
\begin{itemize}
\item {Utilização:trasm.}
\end{itemize}
Tornar mocho, descornar.
Amachucar ou esmurrar (o nariz, a testa, etc.).
\section{Esmodite}
\begin{itemize}
\item {Grp. gram.:f.}
\end{itemize}
Substância pulverulenta, expellida pelos vulcões.
\section{Esmoedor}
\begin{itemize}
\item {fónica:mo-e}
\end{itemize}
\begin{itemize}
\item {Grp. gram.:m.  e  adj.}
\end{itemize}
O que esmói.
\section{Esmoer}
\begin{itemize}
\item {Grp. gram.:v. t.}
\end{itemize}
\begin{itemize}
\item {Proveniência:(De \textunderscore moer\textunderscore )}
\end{itemize}
Moer com os dentes.
Ruminar.
Digerir: \textunderscore esmoer o jantar\textunderscore .
\section{Esmoicar-se}
\begin{itemize}
\item {Grp. gram.:v. i.}
\end{itemize}
\begin{itemize}
\item {Utilização:Prov.}
\end{itemize}
\begin{itemize}
\item {Utilização:trasm.}
\end{itemize}
\begin{itemize}
\item {Proveniência:(De \textunderscore moico\textunderscore )}
\end{itemize}
Partir o boi ou a vaca as pontas ou uma dellas.
\section{Esmoitada}
\begin{itemize}
\item {Grp. gram.:f.}
\end{itemize}
\begin{itemize}
\item {Utilização:Prov.}
\end{itemize}
\begin{itemize}
\item {Utilização:minh.}
\end{itemize}
Acto de esmoitar.
\section{Esmoitar}
\begin{itemize}
\item {Grp. gram.:v. t.}
\end{itemize}
O mesmo que \textunderscore desmoitar\textunderscore .
\section{Esmola}
\begin{itemize}
\item {Grp. gram.:f.}
\end{itemize}
\begin{itemize}
\item {Proveniência:(Do b. lat. \textunderscore eleemosyna\textunderscore )}
\end{itemize}
Aquillo que se dá aos pobres, para os beneficiar.
Benefício.
Retribuição a quem celebrou Missa, por incumbência de alguém.
\section{Esmolador}
\begin{itemize}
\item {Grp. gram.:m.  e  adj.}
\end{itemize}
O que é caritativo, que esmola.
\section{Esmolambação}
\begin{itemize}
\item {Grp. gram.:f.}
\end{itemize}
\begin{itemize}
\item {Proveniência:(De \textunderscore esmolambar\textunderscore )}
\end{itemize}
Estado de esmolambado.
\section{Esmolambado}
\begin{itemize}
\item {Grp. gram.:adj.}
\end{itemize}
\begin{itemize}
\item {Utilização:Bras}
\end{itemize}
\begin{itemize}
\item {Proveniência:(De \textunderscore molambo\textunderscore )}
\end{itemize}
Esfarrapado.
\section{Esmolambar}
\begin{itemize}
\item {Grp. gram.:v. i.}
\end{itemize}
\begin{itemize}
\item {Utilização:Bras}
\end{itemize}
Arrastar molambos.
Andar esfarrapado.
\section{Esmolante}
\begin{itemize}
\item {Grp. gram.:m.}
\end{itemize}
Aquelle que esmola. Cf. Arn. Gama, \textunderscore Últ. Dona\textunderscore , 445.
\section{Esmolar}
\begin{itemize}
\item {Grp. gram.:v. t.}
\end{itemize}
\begin{itemize}
\item {Grp. gram.:V. i.}
\end{itemize}
\begin{itemize}
\item {Utilização:Mod.}
\end{itemize}
\begin{itemize}
\item {Proveniência:(Do b. lat. \textunderscore eleemosynare\textunderscore )}
\end{itemize}
Dar como esmola.
Dar esmola a.
Pedir como esmola. Cf. Garrett, \textunderscore Camões\textunderscore , c. X.
Dar esmolas.
Pedir esmola. Cf. Garrett, \textunderscore Discursos\textunderscore , 106.
\section{Esmolaria}
\begin{itemize}
\item {Grp. gram.:f.}
\end{itemize}
\begin{itemize}
\item {Proveniência:(De \textunderscore esmolar\textunderscore )}
\end{itemize}
Offício de esmoler.
Qualidade de quem é esmoler.
\section{Esmoleira}
\begin{itemize}
\item {Grp. gram.:f.}
\end{itemize}
Saco ou bolsa, para guardar esmolas.
\section{Esmoleiro}
\begin{itemize}
\item {Grp. gram.:m.  e  adj.}
\end{itemize}
\begin{itemize}
\item {Proveniência:(Do lat. \textunderscore eleemosynarius\textunderscore )}
\end{itemize}
Frade, que pedia esmolas para o convento.
\section{Esmolento}
\begin{itemize}
\item {Grp. gram.:adj.}
\end{itemize}
\begin{itemize}
\item {Utilização:Des.}
\end{itemize}
\begin{itemize}
\item {Proveniência:(De \textunderscore esmola\textunderscore )}
\end{itemize}
Caritativo.
\section{Esmolér}
\begin{itemize}
\item {Grp. gram.:m.  e  f.}
\end{itemize}
\begin{itemize}
\item {Grp. gram.:Adj.}
\end{itemize}
\begin{itemize}
\item {Proveniência:(De \textunderscore esmoleiro\textunderscore )}
\end{itemize}
Pessôa, que distribue esmolas, por conta própria ou alheia.
Caritativo.
\section{Esmoliatório}
\begin{itemize}
\item {Grp. gram.:m.}
\end{itemize}
\begin{itemize}
\item {Utilização:Ant.}
\end{itemize}
Albergue para pobres.
Casa pia.
Hospício para enfermos e peregrinos.
(Refl. de \textunderscore esmola\textunderscore )
\section{Esmolna}
\begin{itemize}
\item {Grp. gram.:f.}
\end{itemize}
\begin{itemize}
\item {Utilização:Ant.}
\end{itemize}
O mesmo que \textunderscore esmola\textunderscore . Cf. Frei Fortun., \textunderscore Inéd.\textunderscore , I, 306.
\section{Esmoncar}
\begin{itemize}
\item {Grp. gram.:v. t.}
\end{itemize}
\begin{itemize}
\item {Utilização:Fam.}
\end{itemize}
\begin{itemize}
\item {Proveniência:(De \textunderscore monco\textunderscore )}
\end{itemize}
Assoar.
\section{Esmondar}
\textunderscore v. t.\textunderscore  (e der.)
O mesmo que \textunderscore mondar\textunderscore , etc.
\section{Esmontar}
\begin{itemize}
\item {Grp. gram.:v. t.}
\end{itemize}
\begin{itemize}
\item {Proveniência:(De \textunderscore monte\textunderscore , ou corr. de \textunderscore esmoutar\textunderscore )}
\end{itemize}
O mesmo que \textunderscore desmoitar\textunderscore .
\section{Esmorçar}
\begin{itemize}
\item {Grp. gram.:v.}
\end{itemize}
\begin{itemize}
\item {Utilização:t. Chapel.}
\end{itemize}
\begin{itemize}
\item {Proveniência:(Do it. \textunderscore smorzare\textunderscore )}
\end{itemize}
Desbastar e amaciar (o pêlo) para o fabríco dos chapéus. Cf. \textunderscore Inquér. Industr.\textunderscore , P. II, l.^o 2.^o, 176.
\section{Esmordaçar}
\begin{itemize}
\item {Grp. gram.:v. t.}
\end{itemize}
\begin{itemize}
\item {Proveniência:(De \textunderscore morder\textunderscore )}
\end{itemize}
Morder repetidas vezes.
\section{Esmordelar}
\begin{itemize}
\item {Grp. gram.:v. t.}
\end{itemize}
\begin{itemize}
\item {Utilização:Prov.}
\end{itemize}
O mesmo que \textunderscore esmordaçar\textunderscore .
\section{Esmordicar}
\begin{itemize}
\item {Grp. gram.:v. t.  e  i.}
\end{itemize}
(V.esmordaçar)
\section{Esmorecer}
\begin{itemize}
\item {Grp. gram.:v. t.}
\end{itemize}
\begin{itemize}
\item {Grp. gram.:V. i.}
\end{itemize}
Tirar o ânimo a.
Fazer desmaiar.
Entibiar.
Perder o ânimo, as fôrças.
Desmaiar; desfallecer.
Afroixar, deminuir: \textunderscore o calor vai esmorecendo\textunderscore .
(Por \textunderscore esmorrecer\textunderscore , de \textunderscore morrer\textunderscore )
\section{Esmorecidamente}
\begin{itemize}
\item {Grp. gram.:adv.}
\end{itemize}
\begin{itemize}
\item {Proveniência:(De \textunderscore esmorecer\textunderscore )}
\end{itemize}
Com desânimo.
\section{Esmorecimento}
\begin{itemize}
\item {Grp. gram.:m.}
\end{itemize}
Acto ou effeito de esmorecer.
\section{Esmormar}
\begin{itemize}
\item {Proveniência:(De \textunderscore mormo\textunderscore )}
\end{itemize}
\textunderscore v. t.\textunderscore  (e der.)
O mesmo que \textunderscore esmoncar\textunderscore , etc.
\section{Esmoronar}
\begin{itemize}
\item {Grp. gram.:v. t.}
\end{itemize}
O mesmo que \textunderscore desmoronar\textunderscore .
\section{Esmorraçar}
\begin{itemize}
\item {Grp. gram.:v. t.}
\end{itemize}
Tirar o morrão a, espevitar.
\section{Esmossadela}
\begin{itemize}
\item {Grp. gram.:f.}
\end{itemize}
Acto de esmossar.
\section{Esmossar}
\begin{itemize}
\item {Grp. gram.:v. t.}
\end{itemize}
Fazer mossas em: \textunderscore esmossar uma faca\textunderscore .
\section{Esmoucar}
\begin{itemize}
\item {Grp. gram.:v. t.}
\end{itemize}
\begin{itemize}
\item {Utilização:Prov.}
\end{itemize}
\begin{itemize}
\item {Utilização:minh.}
\end{itemize}
Damnificar.
Esboicelar.
Estragar ou damnificar com pancadas ou por attrito as bordas de (loiça, móveis, etc.).
(Cp. \textunderscore esmocar\textunderscore )
\section{Esmurar}
\textunderscore v. t.\textunderscore  (e der.)
O mesmo que \textunderscore esmurrar\textunderscore , etc. (T. de Melgaço)
\section{Esmurraçar}
\begin{itemize}
\item {Grp. gram.:v. t.}
\end{itemize}
O mesmo que \textunderscore esmurrar\textunderscore .
\section{Esmurramento}
\begin{itemize}
\item {Grp. gram.:m.}
\end{itemize}
Acto ou effeito de esmurrar.
\section{Esmurrar}
\begin{itemize}
\item {Grp. gram.:v. t.}
\end{itemize}
\begin{itemize}
\item {Utilização:Pop.}
\end{itemize}
Dar murros em.
Maltratar.
Dobrar o fio a, tornar embotado (um instrumento cortante).
\section{Esmurregar}
\begin{itemize}
\item {Grp. gram.:v. t.}
\end{itemize}
\begin{itemize}
\item {Utilização:Bras}
\end{itemize}
\begin{itemize}
\item {Proveniência:(De \textunderscore esmurrar\textunderscore )}
\end{itemize}
Esmurrar muito.
Amachucar.
\section{Esmýrnio}
\begin{itemize}
\item {Grp. gram.:m.}
\end{itemize}
\begin{itemize}
\item {Proveniência:(Gr. \textunderscore smurnion\textunderscore )}
\end{itemize}
Planta umbellífera.
\section{És-não-és}
\begin{itemize}
\item {Grp. gram.:m.}
\end{itemize}
\begin{itemize}
\item {Grp. gram.:Adv.}
\end{itemize}
Um quási nada.
Quási.
Por um triz; vai não vai.
\section{Esnocar}
\begin{itemize}
\item {Grp. gram.:v. t.}
\end{itemize}
Esgalhar, partir (ramos, troncos, etc.).
(Cp. \textunderscore desnocar\textunderscore )
\section{Esnoga}
\begin{itemize}
\item {Grp. gram.:f.}
\end{itemize}
\begin{itemize}
\item {Utilização:Ant.}
\end{itemize}
O mesmo que \textunderscore synagoga\textunderscore .
(Ainda us. pelos Judeus de Lisbôa)
\section{Esoces}
\begin{itemize}
\item {Grp. gram.:m. pl.}
\end{itemize}
\begin{itemize}
\item {Proveniência:(Do lat. \textunderscore esox\textunderscore )}
\end{itemize}
Família de peixes, que têm por typo o lúcio.
\section{Esoderma}
\begin{itemize}
\item {Grp. gram.:m.}
\end{itemize}
\begin{itemize}
\item {Proveniência:(Do gr. \textunderscore eso\textunderscore  + \textunderscore derma\textunderscore )}
\end{itemize}
Membrana interior dos insectos.
\section{Esofagiano}
\begin{itemize}
\item {Grp. gram.:adj.}
\end{itemize}
O mesmo que \textunderscore esofágico\textunderscore .
\section{Esofágico}
\begin{itemize}
\item {Grp. gram.:adj.}
\end{itemize}
Relativo a esófago.
\section{Esofagismo}
\begin{itemize}
\item {Grp. gram.:m.}
\end{itemize}
Espasmo do esófago.
\section{Esofagite}
\begin{itemize}
\item {Grp. gram.:f.}
\end{itemize}
Inflamação do esófago.
\section{Esófago}
\begin{itemize}
\item {Grp. gram.:m.}
\end{itemize}
\begin{itemize}
\item {Proveniência:(Gr. \textunderscore oisophagos\textunderscore )}
\end{itemize}
Canal, que liga a larynge ao estômago, em que introduz os alimentos.
\section{Esofagoscopia}
\begin{itemize}
\item {Grp. gram.:f.}
\end{itemize}
\begin{itemize}
\item {Utilização:Med.}
\end{itemize}
\begin{itemize}
\item {Proveniência:(Do gr. \textunderscore oisophagos\textunderscore  + \textunderscore skopein\textunderscore )}
\end{itemize}
Exame endoscópico do esóphago.
\section{Esofagotomia}
\begin{itemize}
\item {Grp. gram.:f.}
\end{itemize}
\begin{itemize}
\item {Proveniência:(Do gr. \textunderscore oisophagos\textunderscore  + \textunderscore tome\textunderscore )}
\end{itemize}
Incisão no esóphago, para a extracção de algum corpo estranho.
\section{Esophagiano}
\begin{itemize}
\item {Grp. gram.:adj.}
\end{itemize}
O mesmo que \textunderscore esophágico\textunderscore .
\section{Esophágico}
\begin{itemize}
\item {Grp. gram.:adj.}
\end{itemize}
Relativo a esóphago.
\section{Esophagismo}
\begin{itemize}
\item {Grp. gram.:m.}
\end{itemize}
Espasmo do esóphago.
\section{Esophagite}
\begin{itemize}
\item {Grp. gram.:f.}
\end{itemize}
Inflammação do esóphago.
\section{Esóphago}
\begin{itemize}
\item {Grp. gram.:m.}
\end{itemize}
\begin{itemize}
\item {Proveniência:(Gr. \textunderscore oisophagos\textunderscore )}
\end{itemize}
Canal, que liga a larynge ao estômago, em que introduz os alimentos.
\section{Esophagoscopia}
\begin{itemize}
\item {Grp. gram.:f.}
\end{itemize}
\begin{itemize}
\item {Utilização:Med.}
\end{itemize}
\begin{itemize}
\item {Proveniência:(Do gr. \textunderscore oisophagos\textunderscore  + \textunderscore skopein\textunderscore )}
\end{itemize}
Exame endoscópico do esóphago.
\section{Esophagotomia}
\begin{itemize}
\item {Grp. gram.:f.}
\end{itemize}
\begin{itemize}
\item {Proveniência:(Do gr. \textunderscore oisophagos\textunderscore  + \textunderscore tome\textunderscore )}
\end{itemize}
Incisão no esóphago, para a extracção de algum corpo estranho.
\section{Esópico}
\begin{itemize}
\item {Grp. gram.:adj.}
\end{itemize}
Relativo ao fabulista Esopo.
\section{Esotérico}
\begin{itemize}
\item {Grp. gram.:adj.}
\end{itemize}
\begin{itemize}
\item {Proveniência:(Gr. \textunderscore esoterikos\textunderscore )}
\end{itemize}
Diz-se da doutrina secreta, que alguns philósophos antigos só communicavam a alguns discípulos.
\section{Esoterismo}
\begin{itemize}
\item {Grp. gram.:m.}
\end{itemize}
\begin{itemize}
\item {Proveniência:(Do gr. \textunderscore esoteros\textunderscore )}
\end{itemize}
Conjunto de princípios, que constituíam a doutrina esotérica.
\section{Espaçadamente}
\begin{itemize}
\item {Grp. gram.:adv.}
\end{itemize}
\begin{itemize}
\item {Proveniência:(De \textunderscore espaçado\textunderscore )}
\end{itemize}
Devagar.
Intervalladamente; de espaço a espaço.
\section{Espaçado}
\begin{itemize}
\item {Grp. gram.:adj.}
\end{itemize}
Em que há intervallos de tempo: \textunderscore accessos espaçados\textunderscore .
Em que há intervallos de lugar: \textunderscore cadeiras espaçadas\textunderscore .
Duradoiro: \textunderscore discurso espaçado\textunderscore .
Adiado, prorogado: \textunderscore a festa foi espaçada\textunderscore .
\section{Espaçamento}
\begin{itemize}
\item {Grp. gram.:m.}
\end{itemize}
Acto de espaçar.
\section{Espaçar}
\begin{itemize}
\item {Grp. gram.:v. t.}
\end{itemize}
Abrir intervallos entre.
Deixar espaço entre.
Demorar.
Adiar; prorogar.
Ampliar.
\section{Espacear}
\begin{itemize}
\item {Grp. gram.:v. t.}
\end{itemize}
(V.espaçar)
\section{Espacejamento}
\begin{itemize}
\item {Grp. gram.:m.}
\end{itemize}
Acto de espacejar.
\section{Espacejar}
\begin{itemize}
\item {Grp. gram.:v. t.}
\end{itemize}
Deixar espaço entre (linhas, letras ou palavras).
\section{Espaciadamente}
\begin{itemize}
\item {Grp. gram.:adv.}
\end{itemize}
\begin{itemize}
\item {Proveniência:(De \textunderscore espaciar\textunderscore )}
\end{itemize}
O mesmo que \textunderscore espaçadamente\textunderscore :«\textunderscore ...para quem mais espaciadamente quiser histórias\textunderscore ». Camillo, \textunderscore Mosaico\textunderscore .
\section{Espaciar}
\begin{itemize}
\item {Grp. gram.:v. t.}
\end{itemize}
O mesmo que \textunderscore espaçar\textunderscore . Cf. \textunderscore Luz e Calor\textunderscore , 101; Latino, \textunderscore Humboldt\textunderscore , 224.
\section{Espácio}
\begin{itemize}
\item {Grp. gram.:adj.}
\end{itemize}
\begin{itemize}
\item {Utilização:Bras. do N}
\end{itemize}
Diz-se das reses, que têm os chifres dispostos horizontalmente.
\section{Espaciosidade}
\begin{itemize}
\item {Grp. gram.:f.}
\end{itemize}
\begin{itemize}
\item {Proveniência:(Lat. \textunderscore spatiositas\textunderscore )}
\end{itemize}
Qualidade de espacioso.
\section{Espacioso}
\begin{itemize}
\item {Grp. gram.:adj.}
\end{itemize}
\begin{itemize}
\item {Utilização:Ant.}
\end{itemize}
\begin{itemize}
\item {Proveniência:(Lat. \textunderscore spatiosus\textunderscore )}
\end{itemize}
O mesmo que \textunderscore espaçoso\textunderscore :«\textunderscore baixou o exército ao spacioso campo de Alcaçar.\textunderscore »\textunderscore Jornada da África\textunderscore , c. v.
\section{Espaço}
\begin{itemize}
\item {Grp. gram.:m.}
\end{itemize}
\begin{itemize}
\item {Grp. gram.:Loc. adv.}
\end{itemize}
\begin{itemize}
\item {Grp. gram.:Loc. adv.}
\end{itemize}
\begin{itemize}
\item {Proveniência:(Do lat. \textunderscore spatium\textunderscore )}
\end{itemize}
Extensão indefinida; firmamento: \textunderscore as estrêllas do espaço\textunderscore .
Extensão de tempo: \textunderscore por espaço de trinta annos\textunderscore .
Extensão superficial.
Intervallo.
Duração.
Demora; adiamento.
Peça, com que se formam intervallos na composição typográphica.
\textunderscore De espaço\textunderscore , com largueza:«\textunderscore conversemos de espaço\textunderscore ». Camillo, \textunderscore Retr. de Ricard.\textunderscore , 117.
\textunderscore De espaço a espaço\textunderscore , com intervallos.
\section{Espaço}
\begin{itemize}
\item {Grp. gram.:adj.}
\end{itemize}
\begin{itemize}
\item {Utilização:Bras. do N}
\end{itemize}
Diz-se das reses, que têm os chifres dispostos horizontalmente.
\section{Espaçosamente}
\begin{itemize}
\item {Grp. gram.:adv.}
\end{itemize}
Á vontade, á larga; de modo espaçoso.
\section{Espaçoso}
\begin{itemize}
\item {Grp. gram.:adj.}
\end{itemize}
Que tem espaço.
Extenso.
Amplo; largo: \textunderscore testa espaçosa\textunderscore .
\section{Espada}
\begin{itemize}
\item {Grp. gram.:f.}
\end{itemize}
\begin{itemize}
\item {Utilização:Fig.}
\end{itemize}
\begin{itemize}
\item {Grp. gram.:M.}
\end{itemize}
\begin{itemize}
\item {Grp. gram.:F. pl.}
\end{itemize}
\begin{itemize}
\item {Proveniência:(Do lat. \textunderscore spatha\textunderscore )}
\end{itemize}
Arma offensiva, mais ou menos longa e ponteaguda, que ordinariamente se traz suspensa da cintura.
A vida militar.
A fôrça armada.
Designação de vários peixes escombridas.
Planta marinha, que se aproveita para adubo de terras.
Matador de toiros, nos circos.
Um dos náipes do baralho.
\section{Espadachim}
\begin{itemize}
\item {Grp. gram.:m.  e  adj.}
\end{itemize}
\begin{itemize}
\item {Proveniência:(Do it. \textunderscore spadaccino\textunderscore )}
\end{itemize}
O que briga muitas vezes, armando-se de espada.
Brigão.
Valentão, fanfarrão.
\section{Espadada}
\begin{itemize}
\item {Grp. gram.:f.}
\end{itemize}
\begin{itemize}
\item {Utilização:Ant.}
\end{itemize}
Golpe de espada.
O mesmo que \textunderscore espadelada\textunderscore . Us. por Camillo.
\section{Espadadeira}
\begin{itemize}
\item {Grp. gram.:f.}
\end{itemize}
\begin{itemize}
\item {Utilização:Prov.}
\end{itemize}
\begin{itemize}
\item {Utilização:trasm.}
\end{itemize}
Mulher, que espada o linho.
\section{Espadado}
\begin{itemize}
\item {Grp. gram.:adj.}
\end{itemize}
\begin{itemize}
\item {Utilização:Açor}
\end{itemize}
\begin{itemize}
\item {Proveniência:(De \textunderscore espadar\textunderscore )}
\end{itemize}
Derreado.
Moído, cansado.
\section{Espadagada}
\begin{itemize}
\item {Grp. gram.:f.}
\end{itemize}
Acto de espadagar.
\section{Espadagão}
\begin{itemize}
\item {Grp. gram.:m.}
\end{itemize}
Espada grande.
(Por \textunderscore espadão\textunderscore  de \textunderscore espada\textunderscore )
\section{Espadagar}
\begin{itemize}
\item {Grp. gram.:v. t.}
\end{itemize}
\begin{itemize}
\item {Utilização:Prov.}
\end{itemize}
O mesmo que \textunderscore espadeirar\textunderscore .
\section{Espadal}
\begin{itemize}
\item {Grp. gram.:m.}
\end{itemize}
Casta de uva.
\section{Espadana}
\begin{itemize}
\item {Grp. gram.:f.}
\end{itemize}
\begin{itemize}
\item {Utilização:Prov.}
\end{itemize}
\begin{itemize}
\item {Utilização:beir.}
\end{itemize}
Planta vivaz, cujas fôlhas semelham a de uma espada.
Veia de água, repuxo ou jacto de líquido, que dá o aspecto de uma lâmina de espada.
Cauda de cometa.
Barbatana de peixe.
Instrumento, com que se espadela o linho; espadela.
Casta de uva.
(Cp. \textunderscore espadanha\textunderscore )
\section{Espadanada}
\begin{itemize}
\item {Grp. gram.:f.}
\end{itemize}
Acto de espadanar.
\section{Espadanal}
\begin{itemize}
\item {Grp. gram.:m.}
\end{itemize}
Lugar, onde crescem espadanas.
\section{Espadanar}
\begin{itemize}
\item {Grp. gram.:v. t.}
\end{itemize}
\begin{itemize}
\item {Utilização:Prov.}
\end{itemize}
\begin{itemize}
\item {Utilização:beir.}
\end{itemize}
\begin{itemize}
\item {Grp. gram.:V. i.}
\end{itemize}
Cobrir de espadanas.
Expellir em borbotões.
O mesmo que \textunderscore espadelar\textunderscore .
Sair em borbotões.
Jorrar.
\section{Espadanelo}
\begin{itemize}
\item {Grp. gram.:adj.}
\end{itemize}
\begin{itemize}
\item {Proveniência:(De \textunderscore espadana\textunderscore )}
\end{itemize}
Diz-se de uma variedade de lírio da serra de Cintra.
\section{Espadâneo}
\begin{itemize}
\item {Grp. gram.:adj.}
\end{itemize}
Semelhante á fôlha da espadana.
\section{Espadanha}
\begin{itemize}
\item {Grp. gram.:f.}
\end{itemize}
\begin{itemize}
\item {Utilização:Ant.}
\end{itemize}
O mesmo que \textunderscore espadana\textunderscore .
(Cast. \textunderscore espadaña\textunderscore )
\section{Espadão}
\begin{itemize}
\item {Grp. gram.:m.}
\end{itemize}
O mesmo que \textunderscore espadagão\textunderscore .
\section{Espadar}
\begin{itemize}
\item {Grp. gram.:v. t.}
\end{itemize}
(V.espadelar)
\section{Espadarte}
\begin{itemize}
\item {Grp. gram.:m.}
\end{itemize}
\begin{itemize}
\item {Proveniência:(Do rad. de \textunderscore espada\textunderscore )}
\end{itemize}
Cetáceo, da fam. dos delfins.
O peixe-serra do Brasil.
\section{Espadaúdo}
\begin{itemize}
\item {Grp. gram.:adj.}
\end{itemize}
Que tem largas espáduas.
Encorpado; esforçado.
(Por \textunderscore espaduado\textunderscore , de \textunderscore espádua\textunderscore )
\section{Espadeira}
\begin{itemize}
\item {Grp. gram.:f.}
\end{itemize}
Casta de uva preta, a mesma que \textunderscore espadeiro\textunderscore .
\section{Espadeirada}
\begin{itemize}
\item {Grp. gram.:f.}
\end{itemize}
\begin{itemize}
\item {Proveniência:(De \textunderscore espadeirar\textunderscore )}
\end{itemize}
Golpe ou pancada com espada.
\section{Espadeirar}
\begin{itemize}
\item {Grp. gram.:v. t.}
\end{itemize}
Ferir ou bater com a espada.
Deslombar.
Bater nas espaldas de.
(Por \textunderscore espaldeirar\textunderscore , de \textunderscore espalda\textunderscore )
\section{Espadeiro}
\begin{itemize}
\item {Grp. gram.:m.}
\end{itemize}
\begin{itemize}
\item {Proveniência:(De \textunderscore espada\textunderscore )}
\end{itemize}
Aquelle que faz ou vende espadas.
Aquelle que floreia bem a espada.
Casta de uva preta.
\section{Espadeiro-branco}
\begin{itemize}
\item {Grp. gram.:m.}
\end{itemize}
Casta de uva minhota.
\section{Espadela}
\begin{itemize}
\item {Grp. gram.:f.}
\end{itemize}
\begin{itemize}
\item {Proveniência:(Do lat. \textunderscore spatula\textunderscore  ou \textunderscore spathula\textunderscore , de \textunderscore spatha\textunderscore )}
\end{itemize}
Instrumento de madeira, com que se bate o linho para o limpar dos tomentos.
Tasquinha.
Remo da azurracha.
Esparrela ou leme provisório.
\section{Espadelada}
\begin{itemize}
\item {Grp. gram.:f.}
\end{itemize}
\begin{itemize}
\item {Utilização:Prov.}
\end{itemize}
\begin{itemize}
\item {Utilização:minh.}
\end{itemize}
Trabalho de espadelar.
Reunião ou serão, em que se espadela. Cf. Camillo, \textunderscore Narcót.\textunderscore , I, 156; Júl. Dinis, \textunderscore Fidalgos\textunderscore , I, 187.
\section{Espadeladeira}
\begin{itemize}
\item {Grp. gram.:f.}
\end{itemize}
\begin{itemize}
\item {Proveniência:(De \textunderscore espadelar\textunderscore )}
\end{itemize}
Mulher, que espadela o linho.
Tascadeira.
\section{Espadeladeiro}
\begin{itemize}
\item {Grp. gram.:m.}
\end{itemize}
\begin{itemize}
\item {Utilização:Prov.}
\end{itemize}
\begin{itemize}
\item {Utilização:trasm.}
\end{itemize}
Cortiço, em que se espadela.
\section{Espadelador}
\begin{itemize}
\item {Grp. gram.:m.}
\end{itemize}
\begin{itemize}
\item {Proveniência:(De \textunderscore espadelar\textunderscore )}
\end{itemize}
Peça de madeira ou cortiço, sôbre que se firma o linho que se espadela.
\section{Espadelagem}
\begin{itemize}
\item {Grp. gram.:f.}
\end{itemize}
Acto de espadelar.
\section{Espadelar}
\begin{itemize}
\item {Grp. gram.:v. t.}
\end{itemize}
Limpar com a espadela, batendo (o linho).
Tascar.
\section{Espadeleiro}
\begin{itemize}
\item {Grp. gram.:m.}
\end{itemize}
\begin{itemize}
\item {Proveniência:(Do b. lat. \textunderscore sptallarius\textunderscore )}
\end{itemize}
Aquelle, que governava a espadela das azurrachas.
\section{Espadeta}
\begin{itemize}
\item {fónica:dê}
\end{itemize}
\begin{itemize}
\item {Grp. gram.:f.}
\end{itemize}
\begin{itemize}
\item {Proveniência:(De \textunderscore espada\textunderscore )}
\end{itemize}
Haste de ferro, espalmada numa ponta, e com a qual os fundidores tiram as escórias do metal em fusão. Cf. F. de Mendonça, \textunderscore Vocab. Techn.\textunderscore 
\section{Espadice}
\begin{itemize}
\item {Grp. gram.:m.}
\end{itemize}
\begin{itemize}
\item {Utilização:Bot.}
\end{itemize}
\begin{itemize}
\item {Proveniência:(Lat. \textunderscore spadix\textunderscore )}
\end{itemize}
Conjunto de flósculos num receptáculo commum, envolvido por uma espatha.
\section{Espadíceo}
\begin{itemize}
\item {Grp. gram.:adj.}
\end{itemize}
Semelhante á espadice.
\section{Espadilha}
\begin{itemize}
\item {Grp. gram.:f.}
\end{itemize}
\begin{itemize}
\item {Utilização:Ichthyol.}
\end{itemize}
\begin{itemize}
\item {Utilização:Prov.}
\end{itemize}
\begin{itemize}
\item {Grp. gram.:M.}
\end{itemize}
\begin{itemize}
\item {Utilização:Fig.}
\end{itemize}
Designação do ás de espadas em alguns jogos de cartas.
O mesmo que \textunderscore petinga\textunderscore ^1.
Lâmina de madeira, com 12 orifícios, correspondentes aos fins de cada ramo, da teia que se vai urdindo.
Chefe.
(Cast. \textunderscore espadilla\textunderscore )
\section{Espadilheiro}
\begin{itemize}
\item {Grp. gram.:m.}
\end{itemize}
\begin{itemize}
\item {Utilização:Prov.}
\end{itemize}
O mesmo que \textunderscore espadeleiro\textunderscore .
\section{Espadim}
\begin{itemize}
\item {Grp. gram.:m.}
\end{itemize}
\begin{itemize}
\item {Utilização:Ichthyol.}
\end{itemize}
Pequena espada.
Antiga moéda portuguesa.
O mesmo que \textunderscore petinga\textunderscore ^1.
\section{Espadinha}
\begin{itemize}
\item {Grp. gram.:f.}
\end{itemize}
\begin{itemize}
\item {Proveniência:(De \textunderscore espada\textunderscore )}
\end{itemize}
Planta, (\textunderscore gladiolus plicatus\textunderscore ).
\section{Espadista}
\begin{itemize}
\item {Grp. gram.:m.}
\end{itemize}
\begin{itemize}
\item {Utilização:Bras}
\end{itemize}
\begin{itemize}
\item {Utilização:Gír.}
\end{itemize}
Jogador de espada.
Gatuno de môsco, que abre portas com chave falsa, introduzindo-a com o movimento de quem crava em alguém uma espada mortífera. Cf. \textunderscore Capital\textunderscore , de 22-VII-912.
\section{Espádoa}
\textunderscore f.\textunderscore  (e der.)
O mesmo que \textunderscore espádua\textunderscore , etc.
\section{Espádua}
\begin{itemize}
\item {Grp. gram.:f.}
\end{itemize}
\begin{itemize}
\item {Proveniência:(Do lat. \textunderscore spatula\textunderscore )}
\end{itemize}
Ombro.
Omoplata, com a carne que a reveste.
A parte mais elevada dos membros anteriores dos quadrúpedes.
\section{Espaduar}
\begin{itemize}
\item {Grp. gram.:v. t.}
\end{itemize}
\begin{itemize}
\item {Grp. gram.:V. i.}
\end{itemize}
Deslocar a espádua, ou distender os músculos da espádua a.
Têr deslocada a espádua.
\section{Espagíria}
\begin{itemize}
\item {Grp. gram.:f.}
\end{itemize}
\begin{itemize}
\item {Proveniência:(Do gr. \textunderscore span\textunderscore  + \textunderscore ageireín\textunderscore , se não é t. inventado por Paracelso)}
\end{itemize}
Designação antiga da Chímica ou da alchimia.
\section{Espagírica}
\begin{itemize}
\item {Grp. gram.:f.}
\end{itemize}
(V.espagíria)
\section{Espagírico}
\begin{itemize}
\item {Grp. gram.:adj.}
\end{itemize}
\begin{itemize}
\item {Utilização:Ant.}
\end{itemize}
Relativo á espagíria.
\section{Espagiristas}
\begin{itemize}
\item {Grp. gram.:m. pl.}
\end{itemize}
\begin{itemize}
\item {Proveniência:(De \textunderscore espagíria\textunderscore )}
\end{itemize}
Seita de médicos, que procuravam explicar as alterações do organismo humano, do mesmo modo que os chímicos da sua época explicavam as alterações dos seres inorgânicos.
\section{Espaireceiro}
\begin{itemize}
\item {Grp. gram.:adj.}
\end{itemize}
\begin{itemize}
\item {Utilização:Mad}
\end{itemize}
Que gosta de espairecer ou de passear ao acaso, sem destino, como quem não tem que fazer.
\section{Espairecer}
\begin{itemize}
\item {Grp. gram.:v. t.}
\end{itemize}
\begin{itemize}
\item {Grp. gram.:V. i.}
\end{itemize}
\begin{itemize}
\item {Proveniência:(De \textunderscore pairar\textunderscore )}
\end{itemize}
Distrahir, entreter.
Entreter-se.
Distrahir-se.
\section{Espairecimento}
\begin{itemize}
\item {Grp. gram.:m.}
\end{itemize}
Acto ou effeito de espairecer.
\section{Espalachado}
\begin{itemize}
\item {Grp. gram.:adj.}
\end{itemize}
\begin{itemize}
\item {Utilização:Prov.}
\end{itemize}
\begin{itemize}
\item {Utilização:trasm.}
\end{itemize}
Largo e grosseiro, (falando-se do rosto de alguém).
\section{Espalda}
\begin{itemize}
\item {Grp. gram.:f.}
\end{itemize}
\begin{itemize}
\item {Proveniência:(Do lat. \textunderscore spatula\textunderscore )}
\end{itemize}
O mesmo que \textunderscore espádua\textunderscore .
Espaldar.
Saliência no flanco de um bastião.
\section{Espaldão}
\begin{itemize}
\item {Grp. gram.:m.}
\end{itemize}
\begin{itemize}
\item {Proveniência:(De \textunderscore espalda\textunderscore )}
\end{itemize}
Anteparo de fortificação.
\section{Espaldar}
\begin{itemize}
\item {Grp. gram.:m.}
\end{itemize}
\begin{itemize}
\item {Proveniência:(De \textunderscore espalda\textunderscore )}
\end{itemize}
Costas da cadeira; espaldeira.
Peça da armadura férrea, para proteger as costas.
\section{Espaldear}
\begin{itemize}
\item {Grp. gram.:v. t.}
\end{itemize}
\begin{itemize}
\item {Proveniência:(De \textunderscore espalda\textunderscore )}
\end{itemize}
Fazer recuar, repellir (um navio).
\section{Espaldeira}
\begin{itemize}
\item {Grp. gram.:f.}
\end{itemize}
\begin{itemize}
\item {Proveniência:(De \textunderscore espalda\textunderscore )}
\end{itemize}
Pano, para cobrir o espaldar.
Renque de árvores, junto de uma parede ou casa de habitação.
\section{Espaldeirada}
\begin{itemize}
\item {Grp. gram.:f.}
\end{itemize}
\begin{itemize}
\item {Utilização:Ant.}
\end{itemize}
O mesmo que \textunderscore espadeirada\textunderscore . Cf. \textunderscore Filodemo\textunderscore , act. V, sc. II.
\section{Espaldeirar-se}
\begin{itemize}
\item {Grp. gram.:v. p.}
\end{itemize}
\begin{itemize}
\item {Utilização:Prov.}
\end{itemize}
\begin{itemize}
\item {Utilização:trasm.}
\end{itemize}
\begin{itemize}
\item {Proveniência:(De \textunderscore espalda\textunderscore )}
\end{itemize}
Partir uma espádua ou o espinhaço.
\section{Espaldeta}
\begin{itemize}
\item {fónica:dê}
\end{itemize}
\begin{itemize}
\item {Grp. gram.:f.}
\end{itemize}
\begin{itemize}
\item {Proveniência:(De \textunderscore espalda\textunderscore )}
\end{itemize}
Esguelha.
Acto de voltar o ombro, torcendo o corpo na sella.
\section{Espaldete}
\begin{itemize}
\item {fónica:dê}
\end{itemize}
\begin{itemize}
\item {Grp. gram.:m.}
\end{itemize}
Peixe de Portugal.
\section{Espaleira}
\begin{itemize}
\item {Grp. gram.:f.}
\end{itemize}
O mesmo que \textunderscore espaleiro\textunderscore .
\section{Espaleiro}
\begin{itemize}
\item {Grp. gram.:m.}
\end{itemize}
\begin{itemize}
\item {Utilização:Gal}
\end{itemize}
\begin{itemize}
\item {Proveniência:(Fr. \textunderscore espalier\textunderscore )}
\end{itemize}
Entrançamento de arbustos, ao longo das paredes.
Revestimento de paredes com trepadeiras.
\section{Espalha}
\begin{itemize}
\item {Grp. gram.:m.}
\end{itemize}
\begin{itemize}
\item {Utilização:Fam.}
\end{itemize}
\begin{itemize}
\item {Proveniência:(De \textunderscore espalhar\textunderscore )}
\end{itemize}
Homem estouvado e alegre.
Indivíduo muito falador.
\section{Espalha-brasas}
\begin{itemize}
\item {Grp. gram.:m.}
\end{itemize}
\begin{itemize}
\item {Utilização:Bras}
\end{itemize}
Sujeito estouvado, desordeiro; ferrabraz.
\section{Espalhada}
\begin{itemize}
\item {Grp. gram.:f.}
\end{itemize}
\begin{itemize}
\item {Proveniência:(De \textunderscore espalhar\textunderscore )}
\end{itemize}
Acto de espalhar.
Espalhafato.
\section{Espalhadamente}
\begin{itemize}
\item {Grp. gram.:adv.}
\end{itemize}
\begin{itemize}
\item {Proveniência:(De \textunderscore espalhar\textunderscore )}
\end{itemize}
Com diffusão.
\section{Espalhadeira}
\begin{itemize}
\item {Grp. gram.:f.}
\end{itemize}
\begin{itemize}
\item {Proveniência:(De \textunderscore espalhar\textunderscore )}
\end{itemize}
Instrumento, com que se abre e se separa a palha.
\section{Espalhado}
\begin{itemize}
\item {Grp. gram.:m.}
\end{itemize}
\begin{itemize}
\item {Proveniência:(De \textunderscore espalhar\textunderscore )}
\end{itemize}
Bulício, balbúrdia.
\section{Espalhadoira}
\begin{itemize}
\item {Grp. gram.:f.}
\end{itemize}
O mesmo que \textunderscore espalhadeira\textunderscore .
\section{Espalhador}
\begin{itemize}
\item {Grp. gram.:adj.}
\end{itemize}
\begin{itemize}
\item {Grp. gram.:M.}
\end{itemize}
Que espalha.
Aquelle ou aquillo que espalha.
\section{Espalhadoura}
\begin{itemize}
\item {Grp. gram.:f.}
\end{itemize}
O mesmo que \textunderscore espalhadeira\textunderscore .
\section{Espalhafatão}
\begin{itemize}
\item {Grp. gram.:adj.}
\end{itemize}
Que faz grande espalhafato. Cf. Camillo, \textunderscore Narcót.\textunderscore , I, 217.
\section{Espalhafatar}
\begin{itemize}
\item {Grp. gram.:v. i.}
\end{itemize}
Fazer espalhafato. Cf. Arn. Gama, \textunderscore Últ. Dona\textunderscore , 279.
\section{Espalhafato}
\begin{itemize}
\item {Grp. gram.:m.}
\end{itemize}
\begin{itemize}
\item {Utilização:Fam.}
\end{itemize}
Vozearia.
Barulho.
Balbúrdia.
Ostentação ruidosa.
Luxo exaggerado ou vaidoso.
Alarde de importáncia pessoal.
Antiga bôca de fogo:«\textunderscore ...vendo que em a casa da polvora não havia nenhuma, mandou descarregar um espalhafato e uma selvagem\textunderscore ». Lopo Coutinho, \textunderscore Hist. do Cêrco de Dio\textunderscore , l.^o II, cap. XIX, 222.
\section{Espalhafatoso}
\begin{itemize}
\item {Grp. gram.:adj.}
\end{itemize}
Que faz espalhafato.
Feito com espalhafato.
\section{Espalhagar}
\begin{itemize}
\item {Grp. gram.:v. t.}
\end{itemize}
Limpar da palha (o trigo).
\section{Espalhamento}
\begin{itemize}
\item {Grp. gram.:m.}
\end{itemize}
Acto de espalhar.
\section{Espalhar}
\begin{itemize}
\item {Grp. gram.:v. t.}
\end{itemize}
\begin{itemize}
\item {Grp. gram.:V. i.}
\end{itemize}
\begin{itemize}
\item {Proveniência:(Do lat. \textunderscore dispalare\textunderscore )}
\end{itemize}
Separar das palhas (os grãos dos cereaes).
Lançar para differentes lados.
Diffundir.
Dispensar largamente: \textunderscore espalhar favores\textunderscore .
Divulgar, tornar público: \textunderscore espalhar notícias\textunderscore .
Distrahir: \textunderscore espalhar tristezas\textunderscore .
Infiltrar.
Transmittir.
Irradiar; alargar.
Espairecer, distrahir-se.
Afastar-se: \textunderscore a trovoada espalhou\textunderscore .
\section{Espalho}
\begin{itemize}
\item {Grp. gram.:m.}
\end{itemize}
\begin{itemize}
\item {Utilização:Artilh.}
\end{itemize}
\begin{itemize}
\item {Utilização:Prov.}
\end{itemize}
\begin{itemize}
\item {Proveniência:(De \textunderscore espalhar\textunderscore )}
\end{itemize}
Espaço, que medeia entre as falcas.
Passeio, distracção.
\section{Espalitar}
\begin{itemize}
\item {Grp. gram.:v. t.}
\end{itemize}
\begin{itemize}
\item {Utilização:Prov.}
\end{itemize}
O mesmo que \textunderscore palitar\textunderscore .
\section{Espalmado}
\begin{itemize}
\item {Grp. gram.:adj.}
\end{itemize}
\begin{itemize}
\item {Utilização:Náut.}
\end{itemize}
\begin{itemize}
\item {Proveniência:(De \textunderscore espalmar\textunderscore )}
\end{itemize}
Que é plano, como a palma da mão.
Chato, raso.
Diz-se do navio que tem o casco limpo dos limos.
\section{Espalmador}
\begin{itemize}
\item {Grp. gram.:m.}
\end{itemize}
Aquelle ou aquillo que espalma.
\section{Espalmar}
\begin{itemize}
\item {Grp. gram.:v. t.}
\end{itemize}
\begin{itemize}
\item {Proveniência:(De \textunderscore palma\textunderscore )}
\end{itemize}
Tornar plano, á semelhança da palma da mão.
Achatar.
Dilatar ou estender, calcando: \textunderscore espalmar a massa\textunderscore .
Limpar o casco de (navios).
Aparar com o puxavante.
\section{Espalto}
\begin{itemize}
\item {Grp. gram.:m.}
\end{itemize}
\begin{itemize}
\item {Proveniência:(Do al. \textunderscore spalt\textunderscore )}
\end{itemize}
Côr escura, que os pintores applicavam sobre escarlate.
Pedra, empregada na fundição de metaes.
\section{Espamparar}
\begin{itemize}
\item {Grp. gram.:v. t.}
\end{itemize}
O mesmo que \textunderscore escancarar\textunderscore .
\section{Espanadela}
\begin{itemize}
\item {Grp. gram.:f.}
\end{itemize}
Acto de espanar^1. Cf. Filinto, XVII, 79.
\section{Espanador}
\begin{itemize}
\item {Grp. gram.:m.}
\end{itemize}
O mesmo que \textunderscore espanejador\textunderscore .
\section{Espanar}
\begin{itemize}
\item {Grp. gram.:v. t.}
\end{itemize}
O mesmo que \textunderscore espanejar\textunderscore .
\section{Espanar}
\begin{itemize}
\item {Grp. gram.:v.}
\end{itemize}
\begin{itemize}
\item {Utilização:t. Marn.}
\end{itemize}
\textunderscore Espanar os crystallizadores\textunderscore , passar a sua água para a andaina de cima e circiá-los depois.
\section{Espanascar}
\begin{itemize}
\item {Grp. gram.:v. t.}
\end{itemize}
\begin{itemize}
\item {Utilização:Fig.}
\end{itemize}
Limpar de panasco (terrenos).
Limpar de má gente (uma terra).
\section{Espanca-diabos}
\begin{itemize}
\item {Grp. gram.:m.}
\end{itemize}
Aquelle que afugenta os diabos com exorcismos; exorcista.
\section{Espancador}
\begin{itemize}
\item {Grp. gram.:m.  e  adj.}
\end{itemize}
O que espanca.
Homem rixoso.
\section{Espancamento}
\begin{itemize}
\item {Grp. gram.:m.}
\end{itemize}
Acto ou effeito de espancar.
\section{Espanca-numes}
\begin{itemize}
\item {Grp. gram.:m.}
\end{itemize}
Aquelle que, em poesia, combate o emprêgo de figuras mythológicas:«\textunderscore ...bandeiras triunfaes dos modernos espanca-numes.\textunderscore »Castilho, \textunderscore Primavera\textunderscore , 41.
\section{Espancar}
\begin{itemize}
\item {Grp. gram.:v. t.}
\end{itemize}
\begin{itemize}
\item {Utilização:Fig.}
\end{itemize}
Bater com panca.
Dar pancadas em.
Bater.
Afastar, dissipar: \textunderscore o sol espanca as trevas\textunderscore .
Afugentar os que andam em (o mar).
\section{Espanco}
\begin{itemize}
\item {Grp. gram.:m.}
\end{itemize}
Acto de espancar. Cf. Filinto, VI, 119.
\section{Espandongado}
\begin{itemize}
\item {Grp. gram.:adj.}
\end{itemize}
\begin{itemize}
\item {Utilização:Bras}
\end{itemize}
\begin{itemize}
\item {Proveniência:(De \textunderscore espandongar\textunderscore )}
\end{itemize}
Amachucado, esfrangalhado.
\section{Espandongar}
\begin{itemize}
\item {Grp. gram.:v. t.}
\end{itemize}
\begin{itemize}
\item {Utilização:Bras}
\end{itemize}
Esfrangalhar, amachucar.
\section{Espanéfico}
\begin{itemize}
\item {Grp. gram.:adj.}
\end{itemize}
\begin{itemize}
\item {Utilização:Pop.}
\end{itemize}
Presumido.
Janota.
Affectado em gestos ou trajes.
\section{Espanejador}
\begin{itemize}
\item {Grp. gram.:m.}
\end{itemize}
\begin{itemize}
\item {Proveniência:(De \textunderscore espanejar\textunderscore )}
\end{itemize}
Pano, escova, ou pennacho, com que se limpa ou se sacode o pó.
\section{Espanejar}
\begin{itemize}
\item {Grp. gram.:v. t.}
\end{itemize}
\begin{itemize}
\item {Proveniência:(De \textunderscore pano\textunderscore )}
\end{itemize}
Limpar ou sacudir com o espanejador.
Limpar.
\section{Espanhol}
\begin{itemize}
\item {Grp. gram.:adj.}
\end{itemize}
\begin{itemize}
\item {Grp. gram.:M.}
\end{itemize}
Relativo á Espanha.
Aquelle que é natural da Espanha.
Língua castelhana.
(Cp. lat. \textunderscore Spania\textunderscore )
\section{Espanholada}
\begin{itemize}
\item {Grp. gram.:f.}
\end{itemize}
\begin{itemize}
\item {Utilização:Fam.}
\end{itemize}
Phrase, própria de espanhol; exaggeração, hypérbole.
Fanfarronada.
\section{Espanholismo}
\begin{itemize}
\item {Grp. gram.:m.}
\end{itemize}
O mesmo que \textunderscore castelhanismo\textunderscore .
\section{Espanholizar}
\begin{itemize}
\item {Grp. gram.:v. t.}
\end{itemize}
\begin{itemize}
\item {Grp. gram.:V. p.}
\end{itemize}
Dar feição de espanhol a.
Adquirir modos, costumes ou linguagem de espanhol.
\section{Espanquear}
\begin{itemize}
\item {Grp. gram.:v. t.}
\end{itemize}
\begin{itemize}
\item {Utilização:Prov.}
\end{itemize}
\begin{itemize}
\item {Utilização:minh.}
\end{itemize}
O mesmo que \textunderscore espancar\textunderscore .
\section{Espanta-boiada}
\begin{itemize}
\item {Grp. gram.:m.}
\end{itemize}
\begin{itemize}
\item {Utilização:Bras}
\end{itemize}
Pequena ave pernalta, que dá grandes e repetidos gritos.
\section{Espantadiço}
\begin{itemize}
\item {Grp. gram.:adj.}
\end{itemize}
Que se espanta com facilidade.
\section{Espantado}
\begin{itemize}
\item {Grp. gram.:adj.}
\end{itemize}
\begin{itemize}
\item {Utilização:Heráld.}
\end{itemize}
Que se espantou.
Assustado.
Admirado, maravilhado.
Diz-se dos animaes que, nos escudos de armas, se representam empinados e não rompentes.
\section{Espantador}
\begin{itemize}
\item {Grp. gram.:adj.}
\end{itemize}
\begin{itemize}
\item {Grp. gram.:M.}
\end{itemize}
Que espanta.
Aquelle que espanta.
\section{Espantalho}
\begin{itemize}
\item {Grp. gram.:m.}
\end{itemize}
\begin{itemize}
\item {Utilização:Fam.}
\end{itemize}
\begin{itemize}
\item {Proveniência:(De \textunderscore espantar\textunderscore )}
\end{itemize}
Figura ou qualquer objecto, que se colloca nas árvores ou nos campos, para afugentar as aves.
Pessôa maltrapida e feia; pessôa inútil.
\section{Espanta-lobos}
\begin{itemize}
\item {Grp. gram.:m.}
\end{itemize}
\begin{itemize}
\item {Utilização:Pop.}
\end{itemize}
Planta leguminosa.
Pessôa muito faladora; tagarela.
\section{Espantar}
\begin{itemize}
\item {Grp. gram.:v. t.}
\end{itemize}
\begin{itemize}
\item {Grp. gram.:V. i.}
\end{itemize}
\begin{itemize}
\item {Grp. gram.:V. p.}
\end{itemize}
\begin{itemize}
\item {Utilização:Gír.}
\end{itemize}
\begin{itemize}
\item {Proveniência:(Lat. hyp. \textunderscore expaventare\textunderscore , do lat. \textunderscore expavescere\textunderscore )}
\end{itemize}
Causar espanto ou admiração a.
Assombrar.
Atemorizar.
Afugentar; enxotar: \textunderscore espantar pardaes\textunderscore .
Desviar, afastar: \textunderscore espantar o somno\textunderscore .
Sentir espanto.
Zangar-se.
\section{Espanta-ratos}
\begin{itemize}
\item {Grp. gram.:m.}
\end{itemize}
Aquelle que faz espalhafato por motivos fúteis.
Espanta-lobos.
\section{Espantável}
\begin{itemize}
\item {Grp. gram.:adj.}
\end{itemize}
\begin{itemize}
\item {Proveniência:(De \textunderscore espantar\textunderscore )}
\end{itemize}
Que causa espanto, espantoso.
\section{Espanto}
\begin{itemize}
\item {Grp. gram.:m.}
\end{itemize}
Susto ou medo excessivo.
Assombro, admiração.
Coisa imprevista; surpresa.
Acto de espantar.
\section{Espantosamente}
\begin{itemize}
\item {Grp. gram.:adv.}
\end{itemize}
De modo espantoso.
\section{Espantoso}
\begin{itemize}
\item {Grp. gram.:adj.}
\end{itemize}
\begin{itemize}
\item {Proveniência:(De \textunderscore espanto\textunderscore )}
\end{itemize}
Que causa espanto, que espanta.
\section{Espapaçado}
\begin{itemize}
\item {Grp. gram.:adj.}
\end{itemize}
\begin{itemize}
\item {Utilização:Fig.}
\end{itemize}
\begin{itemize}
\item {Proveniência:(De \textunderscore espapaçar\textunderscore .)}
\end{itemize}
Que tem o aspecto ou a consistência de papas; molle.
Desengraçado.
Froixo, indolente:«\textunderscore ...um viver espapaçado em doce molleza.\textunderscore »Camillo, \textunderscore Brasileira\textunderscore , 182.
\section{Espapaçar}
\begin{itemize}
\item {Grp. gram.:v. t.}
\end{itemize}
\begin{itemize}
\item {Utilização:Fig.}
\end{itemize}
\begin{itemize}
\item {Proveniência:(De \textunderscore papa\textunderscore ^2)}
\end{itemize}
Dar a fórma de papas a.
Tornar molle.
Alastrar como papas.
Tornar insípido, sem graça.
\section{Espapar}
\begin{itemize}
\item {Grp. gram.:v. i.  e  p.}
\end{itemize}
\begin{itemize}
\item {Grp. gram.:V. t.}
\end{itemize}
O mesmo que \textunderscore despapar\textunderscore .
Formar papo em.
\section{Esparadrapeiro}
\begin{itemize}
\item {Grp. gram.:m.}
\end{itemize}
\begin{itemize}
\item {Utilização:Bras}
\end{itemize}
Instrumento, com que se prepara o esparadrapo.
\section{Esparadrapo}
\begin{itemize}
\item {Grp. gram.:m.}
\end{itemize}
\begin{itemize}
\item {Proveniência:(De um pref. obscuro e fr. \textunderscore drap\textunderscore )}
\end{itemize}
Pano untado de medicamentos, que se applica sôbre chagas.
\section{Esparavão}
\begin{itemize}
\item {Grp. gram.:m.}
\end{itemize}
Tumor ossificado, na curva da perna do cavallo.
(Cast. \textunderscore esparaván\textunderscore )
\section{Esparavél}
\begin{itemize}
\item {Grp. gram.:m.}
\end{itemize}
\begin{itemize}
\item {Utilização:Des.}
\end{itemize}
\begin{itemize}
\item {Proveniência:(Do lat. \textunderscore sparsus\textunderscore  + \textunderscore velum\textunderscore ?)}
\end{itemize}
Tarrafa.
Dossel.
Franja de cortinado.
Tábua, com que os pedreiros põem cal e areia nos tectos.
\section{Esparavela}
\begin{itemize}
\item {Grp. gram.:f.}
\end{itemize}
\begin{itemize}
\item {Utilização:Prov.}
\end{itemize}
\begin{itemize}
\item {Utilização:alent.}
\end{itemize}
\textunderscore Estar\textunderscore  ou \textunderscore andar á esparavela\textunderscore , estar ou andar nu, em pêlo.
(Relaciona-se com \textunderscore esparavél\textunderscore ?)
\section{Esparaveleiro}
\begin{itemize}
\item {Grp. gram.:m.}
\end{itemize}
\begin{itemize}
\item {Utilização:Ant.}
\end{itemize}
\begin{itemize}
\item {Proveniência:(De \textunderscore asparavél\textunderscore )}
\end{itemize}
Aquelle que fazia esparavéis.
\section{Esparavonado}
\begin{itemize}
\item {Grp. gram.:adj.}
\end{itemize}
O mesmo que \textunderscore esparvonado\textunderscore .
\section{Esparçal}
\begin{itemize}
\item {Grp. gram.:m.}
\end{itemize}
\begin{itemize}
\item {Utilização:Des.}
\end{itemize}
\begin{itemize}
\item {Proveniência:(De \textunderscore parcel\textunderscore )}
\end{itemize}
Lugar, em que há muitos parcéis.
\section{Esparcelada}
\begin{itemize}
\item {Grp. gram.:f.}
\end{itemize}
\begin{itemize}
\item {Utilização:Ant.}
\end{itemize}
\begin{itemize}
\item {Proveniência:(De \textunderscore esparcelar\textunderscore )}
\end{itemize}
Terra baixa e plana.
\section{Esparcelar}
\begin{itemize}
\item {Grp. gram.:v. t.}
\end{itemize}
\begin{itemize}
\item {Utilização:Des.}
\end{itemize}
O mesmo que \textunderscore aparcelar\textunderscore .
\section{Esparcellada}
\begin{itemize}
\item {Grp. gram.:f.}
\end{itemize}
\begin{itemize}
\item {Utilização:Ant.}
\end{itemize}
\begin{itemize}
\item {Proveniência:(De \textunderscore esparcellar\textunderscore )}
\end{itemize}
Terra baixa e plana.
\section{Esparcellar}
\begin{itemize}
\item {Grp. gram.:v. t.}
\end{itemize}
\begin{itemize}
\item {Utilização:Des.}
\end{itemize}
O mesmo que \textunderscore aparcellar\textunderscore .
\section{Esparceta}
\begin{itemize}
\item {fónica:cê}
\end{itemize}
\begin{itemize}
\item {Grp. gram.:f.}
\end{itemize}
O mesmo que \textunderscore esparceto\textunderscore .
\section{Esparceto}
\begin{itemize}
\item {fónica:cê}
\end{itemize}
\begin{itemize}
\item {Grp. gram.:m.}
\end{itemize}
O mesmo que \textunderscore sanfeno\textunderscore . Cf. \textunderscore Bibl. da G. do Campo\textunderscore , 284.
\section{Esparecer}
\begin{itemize}
\item {Grp. gram.:v. i.}
\end{itemize}
\begin{itemize}
\item {Utilização:Des.}
\end{itemize}
O mesmo que \textunderscore espairecer\textunderscore . Cf. M. Bernárdez, \textunderscore N. Floresta\textunderscore , II, 201 e 206; IV, 37 e 38.
\section{Esparela}
\begin{itemize}
\item {Grp. gram.:f.}
\end{itemize}
O mesmo que \textunderscore pernilongo\textunderscore , ave.
\section{Espargelar}
\begin{itemize}
\item {Grp. gram.:v. t.}
\end{itemize}
(V.espargir)
\section{Esparger}
\begin{itemize}
\item {Grp. gram.:v. t.}
\end{itemize}
\begin{itemize}
\item {Utilização:Prov.}
\end{itemize}
\begin{itemize}
\item {Utilização:trasm.}
\end{itemize}
O mesmo que \textunderscore espargir\textunderscore . (Colhido em Chaves)
\section{Espargimento}
\begin{itemize}
\item {Grp. gram.:m.}
\end{itemize}
Acto ou effeito de espargir.
\section{Espargir}
\begin{itemize}
\item {Grp. gram.:v. t.}
\end{itemize}
\begin{itemize}
\item {Proveniência:(Do lat. \textunderscore spargere\textunderscore )}
\end{itemize}
Espalhar ou derramar (um líquido).
Diffundir.
Desfolhar.
Espalhar em gotas, em borrifos.
\section{Espargo}
\begin{itemize}
\item {Grp. gram.:m.}
\end{itemize}
\begin{itemize}
\item {Utilização:T. de Turquel}
\end{itemize}
\begin{itemize}
\item {Proveniência:(Do gr. \textunderscore asparagos\textunderscore )}
\end{itemize}
Planta liliácea, que serve de typo ás asparagíneas.
O mesmo que \textunderscore agraço\textunderscore .
\section{Espargueira}
\begin{itemize}
\item {Grp. gram.:f.}
\end{itemize}
Terra em que se cultivam espargos.
\section{Esparguta}
\begin{itemize}
\item {Grp. gram.:f.}
\end{itemize}
\begin{itemize}
\item {Proveniência:(Fr. \textunderscore espargoutte\textunderscore )}
\end{itemize}
Planta, que serve para forragens.
\section{Esparoides}
\begin{itemize}
\item {Grp. gram.:m. pl.}
\end{itemize}
\begin{itemize}
\item {Proveniência:(Do lat. \textunderscore sparus\textunderscore  + gr. \textunderscore eidos\textunderscore )}
\end{itemize}
Família de peixes esquamodermos, a que pertence a boga.
\section{Esparra}
\begin{itemize}
\item {Grp. gram.:f.}
\end{itemize}
Acto de esparrar. Cf. Lapa, \textunderscore Proc. de Vin.\textunderscore , 6.
\section{Esparragão}
\begin{itemize}
\item {Grp. gram.:m.}
\end{itemize}
\begin{itemize}
\item {Utilização:Ant.}
\end{itemize}
Tecido de seda, que se usava em forros de vestidos.
\section{Esparragueira}
\begin{itemize}
\item {Grp. gram.:f.}
\end{itemize}
Sulco ou valla especial, em que se semeiam ou cultivam os espargos. Cf. \textunderscore Bibl. da G. do Campo\textunderscore , 284.
(Cp. \textunderscore esparregar\textunderscore )
\section{Esparralhar}
\begin{itemize}
\item {Grp. gram.:v. t.}
\end{itemize}
\begin{itemize}
\item {Utilização:Pop.}
\end{itemize}
Espalhar ao acaso, estender numa superfície.
Derramar.
(Refl. de \textunderscore espalhar\textunderscore )
\section{Esparramar}
\begin{itemize}
\item {Grp. gram.:v. t.}
\end{itemize}
\begin{itemize}
\item {Utilização:Bras. do N}
\end{itemize}
\begin{itemize}
\item {Proveniência:(T. cast.)}
\end{itemize}
Esparralhar.
Dispersar.
Desalinhar.
Tornar imprudente, estouvado. Cf. Camillo, \textunderscore Caveira\textunderscore , 172; \textunderscore Cancion. Al.\textunderscore , 299.
Achatar.
\section{Esparramo}
\begin{itemize}
\item {Grp. gram.:m.}
\end{itemize}
\begin{itemize}
\item {Utilização:Bras}
\end{itemize}
Acto ou effeito de esparramar.
\section{Esparrar}
\begin{itemize}
\item {Grp. gram.:v. t.}
\end{itemize}
\begin{itemize}
\item {Grp. gram.:V. p.}
\end{itemize}
\begin{itemize}
\item {Utilização:Bras. do N}
\end{itemize}
O mesmo que \textunderscore desparrar\textunderscore .
Cair redondamente.
Estender-se, dizer sandices.
\section{Esparregado}
\begin{itemize}
\item {Grp. gram.:m.}
\end{itemize}
\begin{itemize}
\item {Proveniência:(De \textunderscore esparregar\textunderscore )}
\end{itemize}
Ervas guisadas, depois de cortadas miudamente, cozidas e espremidas.
\section{Esparregar}
\begin{itemize}
\item {Grp. gram.:v. t.}
\end{itemize}
Guisar (espargo, couves, nabiças, etc.) depois de as cortar, cozer e temperar.
(Por \textunderscore espargar\textunderscore , de \textunderscore espargo\textunderscore )
\section{Espárrego}
\begin{itemize}
\item {Grp. gram.:m.}
\end{itemize}
\begin{itemize}
\item {Utilização:Des.}
\end{itemize}
O mesmo que \textunderscore espargo\textunderscore .
\section{Esparrela}
\begin{itemize}
\item {Grp. gram.:f.}
\end{itemize}
\begin{itemize}
\item {Utilização:Pop.}
\end{itemize}
\begin{itemize}
\item {Utilização:Prov.}
\end{itemize}
\begin{itemize}
\item {Utilização:minh.}
\end{itemize}
Armadilha de caça.
Leme provisório.
Lôgro; arriosca.
Pessôa magra, enfezada.
\section{Esparrimar}
\textunderscore v. i.\textunderscore  (e der.)
O mesmo que \textunderscore esparramar\textunderscore  e \textunderscore esparrinhar\textunderscore .
\section{Esparrinhar}
\begin{itemize}
\item {Grp. gram.:v. t.}
\end{itemize}
\begin{itemize}
\item {Utilização:Pop.}
\end{itemize}
\begin{itemize}
\item {Grp. gram.:V. i.}
\end{itemize}
Espargir.
Sair em repuxo, em borrifos.
(Refl. de \textunderscore esparralhar\textunderscore )
\section{Esparsa}
\begin{itemize}
\item {Grp. gram.:f.}
\end{itemize}
\begin{itemize}
\item {Utilização:Ext.}
\end{itemize}
\begin{itemize}
\item {Proveniência:(De \textunderscore esparso\textunderscore )}
\end{itemize}
Antiga composição poética, em versos de seis sýllabas.
Pequena composição lýrica.
\section{Esparso}
\begin{itemize}
\item {Grp. gram.:adj.}
\end{itemize}
\begin{itemize}
\item {Proveniência:(Lat. \textunderscore sparsus\textunderscore )}
\end{itemize}
Que se espargiu.
Diffundido; espalhado.
\section{Espartal}
\begin{itemize}
\item {Grp. gram.:m.}
\end{itemize}
Lugar, onde cresce esparto.
\section{Espartano}
\begin{itemize}
\item {Grp. gram.:m.}
\end{itemize}
\begin{itemize}
\item {Grp. gram.:Adj.}
\end{itemize}
\begin{itemize}
\item {Utilização:Fig.}
\end{itemize}
\begin{itemize}
\item {Proveniência:(De \textunderscore Esparta\textunderscore , n. p.)}
\end{itemize}
Aquelle que é natural de Esparta.
Relativo a Esparta.
Austero, sóbrio.
\section{Espartão}
\begin{itemize}
\item {Grp. gram.:m.}
\end{itemize}
Tecido de esparto, que se encosta aos fueiros, para amparar dos lados a carga das carrêtas alentejanas.
(Cast. \textunderscore espartón\textunderscore )
\section{Espartaria}
\begin{itemize}
\item {Grp. gram.:f.}
\end{itemize}
Casa ou lugar, em que se fabricam ou se vendem obras de esparto.
Porção de obras de esparto.
\section{Esparteína}
\begin{itemize}
\item {Grp. gram.:f.}
\end{itemize}
Base chímica, descoberta no esparto.
\section{Esparteira}
\begin{itemize}
\item {Grp. gram.:f.}
\end{itemize}
(V.esparto)
\section{Esparteiro}
\begin{itemize}
\item {Grp. gram.:m.}
\end{itemize}
Aquelle que faz ou vende obras de esparto.
\section{Espartejar}
\begin{itemize}
\item {Grp. gram.:v. t.}
\end{itemize}
Dividir em partes; esquartejar. Cf. Filinto, XI, 221.
\section{Espartenhas}
\begin{itemize}
\item {Grp. gram.:f. pl.}
\end{itemize}
Antigo calçado de esparto.
\section{Espartéolos}
\begin{itemize}
\item {Grp. gram.:m. pl.}
\end{itemize}
\begin{itemize}
\item {Proveniência:(Lat. \textunderscore sparteoli\textunderscore )}
\end{itemize}
Corpo de bombeiros, instituido pelo imperador Augusto.
\section{Espartilhar}
\begin{itemize}
\item {Grp. gram.:v. t.}
\end{itemize}
Cingir ou apertar com espartilho.
\section{Espartilheira}
\begin{itemize}
\item {Grp. gram.:f.}
\end{itemize}
Mulher, que faz ou vende espartilhos.
\section{Espartilheiro}
\begin{itemize}
\item {Grp. gram.:m.}
\end{itemize}
Fabricante ou vendedor de espartilhos.
\section{Espartilho}
\begin{itemize}
\item {Grp. gram.:m.}
\end{itemize}
\begin{itemize}
\item {Proveniência:(De \textunderscore esparto\textunderscore , visto que os primeiros espartilhos eram de esparto)}
\end{itemize}
Collete com lâminas de aço ou barbas de baleia, usado por mulheres, para comprimir a cintura e dar elegância ao tronco.
\section{Espartir}
\begin{itemize}
\item {Grp. gram.:v. t.}
\end{itemize}
\begin{itemize}
\item {Utilização:Prov.}
\end{itemize}
Desfiar e esticar com os dentes (o linho na roca), para tornar o fio igual.
(Cp. \textunderscore partir\textunderscore )
\section{Esparto}
\begin{itemize}
\item {Grp. gram.:m.}
\end{itemize}
\begin{itemize}
\item {Proveniência:(Lat. \textunderscore spartum\textunderscore )}
\end{itemize}
Planta gramínea, de cujos caules se fabricam cordas, capachos, etc.
\section{Esparvão}
\begin{itemize}
\item {Grp. gram.:m.}
\end{itemize}
(V.esparavão)
\section{Esparvoado}
\begin{itemize}
\item {Grp. gram.:adj.}
\end{itemize}
O mesmo que \textunderscore esparvonado\textunderscore .
\section{Esparvonado}
\begin{itemize}
\item {Grp. gram.:adj.}
\end{itemize}
Que tem esparvão.
\section{Esparzir}
\textunderscore v. t.\textunderscore  e \textunderscore p.\textunderscore  (e der.)
O mesmo que \textunderscore espargir\textunderscore , etc.
\section{Espasmar}
\begin{itemize}
\item {Grp. gram.:v. t.}
\end{itemize}
\begin{itemize}
\item {Grp. gram.:V. i.}
\end{itemize}
Causar espasmo a.
Soffrer espasmo.
\section{Espasmo}
\begin{itemize}
\item {Grp. gram.:m.}
\end{itemize}
\begin{itemize}
\item {Utilização:Fig.}
\end{itemize}
\begin{itemize}
\item {Proveniência:(Gr. \textunderscore spasmos\textunderscore )}
\end{itemize}
Contracção involuntária dos músculos, mormente dos que não estão sujeitos á vontade.
Convulsão.
Êxtase.
\section{Espasmódico}
\begin{itemize}
\item {Grp. gram.:adj.}
\end{itemize}
\begin{itemize}
\item {Proveniência:(Do gr. \textunderscore spasmodes\textunderscore )}
\end{itemize}
Relativo a espasmo.
\section{Espasmologia}
\begin{itemize}
\item {Grp. gram.:f.}
\end{itemize}
\begin{itemize}
\item {Proveniência:(Do gr. \textunderscore spasmos\textunderscore  + \textunderscore logos\textunderscore )}
\end{itemize}
Tratado á cêrca dos espasmos.
\section{Espassar}
\begin{itemize}
\item {Grp. gram.:v. i.}
\end{itemize}
\begin{itemize}
\item {Utilização:Ant.}
\end{itemize}
Gastar tempo em divertimentos. Cf. Fern. Lopes, \textunderscore Chrón. de D. João I\textunderscore , p. 2.^a, c. 147.
\section{Espassaricado}
\begin{itemize}
\item {Grp. gram.:adj.}
\end{itemize}
\begin{itemize}
\item {Utilização:Prov.}
\end{itemize}
\begin{itemize}
\item {Utilização:trasm.}
\end{itemize}
Resequido, muito passado.
(Relaciona-se com \textunderscore passado\textunderscore )
\section{Espassaricar-se}
\begin{itemize}
\item {Grp. gram.:v. p.}
\end{itemize}
\begin{itemize}
\item {Utilização:Prov.}
\end{itemize}
\begin{itemize}
\item {Utilização:beir.}
\end{itemize}
\begin{itemize}
\item {Proveniência:(De \textunderscore pássaro\textunderscore )}
\end{itemize}
Remexer-se ou andar, requebrando-se com affectação.
Saracotear-se.
\section{Espassarotar}
\begin{itemize}
\item {Grp. gram.:v. i.}
\end{itemize}
\begin{itemize}
\item {Utilização:Prov.}
\end{itemize}
\begin{itemize}
\item {Utilização:trasm.}
\end{itemize}
Debandar, espalhar-se: \textunderscore o povo espassarotou\textunderscore .
\section{Espata}
\begin{itemize}
\item {Grp. gram.:f.}
\end{itemize}
\begin{itemize}
\item {Utilização:Bot.}
\end{itemize}
\begin{itemize}
\item {Utilização:Ant.}
\end{itemize}
\begin{itemize}
\item {Proveniência:(Lat. \textunderscore spatha\textunderscore )}
\end{itemize}
Espécie de cálice membranoso, que contém e protege a espiga que fórma a inflorescência.
Espada larga de dois gumes e sem ponta.
\section{Espatáceo}
\begin{itemize}
\item {Grp. gram.:adj.}
\end{itemize}
\begin{itemize}
\item {Utilização:Bot.}
\end{itemize}
Contido em uma espata.
\section{Espatala}
\begin{itemize}
\item {Grp. gram.:f.}
\end{itemize}
Gênero de arbustos do Cabo da Bôa-Esperança.
\section{Espatalla}
\begin{itemize}
\item {Grp. gram.:f.}
\end{itemize}
Gênero de arbustos do Cabo da Bôa-Esperança.
\section{Espatário}
\begin{itemize}
\item {Grp. gram.:m.}
\end{itemize}
\begin{itemize}
\item {Proveniência:(Lat. \textunderscore spatharius\textunderscore )}
\end{itemize}
Soldado, armado de espata.
Gladiador.
\section{Espatela}
\begin{itemize}
\item {Grp. gram.:f.}
\end{itemize}
\begin{itemize}
\item {Proveniência:(Do lat. \textunderscore spathula\textunderscore )}
\end{itemize}
Tabuínha inofensiva, que serve para abaixar a lingua, a fim de melhor se vêr a garganta.
\section{Espatélia}
\begin{itemize}
\item {Grp. gram.:f.}
\end{itemize}
\begin{itemize}
\item {Utilização:Bot.}
\end{itemize}
\begin{itemize}
\item {Proveniência:(Do gr. \textunderscore spathe\textunderscore )}
\end{itemize}
Cada uma das peças que envolvem a flôr das gramíneas.
\section{Espatha}
\begin{itemize}
\item {Grp. gram.:f.}
\end{itemize}
\begin{itemize}
\item {Utilização:Bot.}
\end{itemize}
\begin{itemize}
\item {Utilização:Ant.}
\end{itemize}
\begin{itemize}
\item {Proveniência:(Lat. \textunderscore spatha\textunderscore )}
\end{itemize}
Espécie de cálice membranoso, que contém e protege a espiga que fórma a inflorescência.
Espada larga de dois gumes e sem ponta.
\section{Espatháceo}
\begin{itemize}
\item {Grp. gram.:adj.}
\end{itemize}
\begin{itemize}
\item {Utilização:Bot.}
\end{itemize}
Contido em uma espatha.
\section{Espathário}
\begin{itemize}
\item {Grp. gram.:m.}
\end{itemize}
\begin{itemize}
\item {Proveniência:(Lat. \textunderscore spatharius\textunderscore )}
\end{itemize}
Soldado, armado de espatha.
Gladiador.
\section{Espathélia}
\begin{itemize}
\item {Grp. gram.:f.}
\end{itemize}
\begin{itemize}
\item {Utilização:Bot.}
\end{itemize}
\begin{itemize}
\item {Proveniência:(Do gr. \textunderscore spathe\textunderscore )}
\end{itemize}
Cada uma das peças que envolvem a flôr das gramíneas.
\section{Espathicarpa}
\begin{itemize}
\item {Grp. gram.:f.}
\end{itemize}
\begin{itemize}
\item {Proveniência:(Do gr. \textunderscore spathe\textunderscore  + \textunderscore karpos\textunderscore )}
\end{itemize}
Planta do Uruguai.
\section{Espáthico}
\begin{itemize}
\item {Grp. gram.:adj.}
\end{itemize}
Relativo ao \textunderscore espatho\textunderscore .
\section{Espatho}
\begin{itemize}
\item {Grp. gram.:m.}
\end{itemize}
\begin{itemize}
\item {Proveniência:(Do al. \textunderscore spath\textunderscore )}
\end{itemize}
Nome de diversos mineraes crystallinos e lamellosos.
Carbonato de cal crystallizado.
\section{Espaticarpa}
\begin{itemize}
\item {Grp. gram.:f.}
\end{itemize}
\begin{itemize}
\item {Proveniência:(Do gr. \textunderscore spathe\textunderscore  + \textunderscore karpos\textunderscore )}
\end{itemize}
Planta do Uruguai.
\section{Espático}
\begin{itemize}
\item {Grp. gram.:adj.}
\end{itemize}
Relativo ao \textunderscore espatho\textunderscore .
\section{Espatifante}
\begin{itemize}
\item {Grp. gram.:adj.}
\end{itemize}
Que espatifa.
\section{Espatifar}
\begin{itemize}
\item {Grp. gram.:v. t.}
\end{itemize}
\begin{itemize}
\item {Utilização:Pop.}
\end{itemize}
\begin{itemize}
\item {Utilização:Fig.}
\end{itemize}
Espedaçar.
Rasgar.
Fazer em retalhos.
Estragar, dissipar: \textunderscore espatifou quanto tinha\textunderscore .
\section{Espatilha}
\begin{itemize}
\item {Grp. gram.:f.}
\end{itemize}
\begin{itemize}
\item {Utilização:Náut.}
\end{itemize}
Cabo ou corrente, hoje em desuso, com que se aguentava para a amurada a parte superior da âncora, depois de passadas as boças.
\section{Espatilhar}
\begin{itemize}
\item {Grp. gram.:v. t.}
\end{itemize}
\begin{itemize}
\item {Proveniência:(De \textunderscore espatilha\textunderscore )}
\end{itemize}
Suspender (uma âncora), ficando os braços horizontalmente, em relação ao costado do navio.
\section{Espato}
\begin{itemize}
\item {Grp. gram.:m.}
\end{itemize}
\begin{itemize}
\item {Proveniência:(Do al. \textunderscore spath\textunderscore )}
\end{itemize}
Nome de diversos mineraes cristalinos e lamelosos.
Carbonato de cal cristalizado.
\section{Espátula}
\begin{itemize}
\item {Grp. gram.:f.}
\end{itemize}
\begin{itemize}
\item {Utilização:Prov.}
\end{itemize}
\begin{itemize}
\item {Utilização:alent.}
\end{itemize}
\begin{itemize}
\item {Utilização:Mús.}
\end{itemize}
\begin{itemize}
\item {Grp. gram.:M.}
\end{itemize}
\begin{itemize}
\item {Proveniência:(Lat. \textunderscore spathula\textunderscore  ou \textunderscore spatula\textunderscore , de \textunderscore spatha\textunderscore )}
\end{itemize}
Espécie de faca de madeira, metal, etc., com que se abrem livros ou se espalmam e amollecem substâncias medicamentosas, nas preparações pharmacêuticas.
Ferro espalmado e curvo, para os estucadores tirarem o sobrecellente da massa, com que pegam os ornatos do estuque.
Extremidade das chaves dos instrumentos de sôpro, na qual o executante apoia o dedo, quando quer pôr em acção as mesmas chaves.
Ave branca, de bico flexível e largo na extremidade, (\textunderscore leucorodios\textunderscore , Lin.).
\section{Espatulado}
\begin{itemize}
\item {Grp. gram.:adj.}
\end{itemize}
Que tem fórma de espátula.
\section{Espatulária}
\begin{itemize}
\item {Grp. gram.:f.}
\end{itemize}
\begin{itemize}
\item {Proveniência:(De \textunderscore espátula\textunderscore )}
\end{itemize}
Gênero de peixes.
\section{Espatuleta}
\begin{itemize}
\item {fónica:lê}
\end{itemize}
\begin{itemize}
\item {Grp. gram.:f.}
\end{itemize}
Pequena espátula.
\section{Espaventar}
\begin{itemize}
\item {Grp. gram.:v. t.}
\end{itemize}
\begin{itemize}
\item {Proveniência:(Do lat. \textunderscore expavere\textunderscore )}
\end{itemize}
Causar espanto a.
\section{Espavento}
\begin{itemize}
\item {Grp. gram.:m.}
\end{itemize}
\begin{itemize}
\item {Utilização:Pop.}
\end{itemize}
\begin{itemize}
\item {Proveniência:(De \textunderscore espaventar\textunderscore )}
\end{itemize}
Espanto.
Susto.
Ostentação vaidosa.
Grande pompa.
\section{Espaventoso}
\begin{itemize}
\item {Grp. gram.:adj.}
\end{itemize}
\begin{itemize}
\item {Utilização:Fig.}
\end{itemize}
\begin{itemize}
\item {Proveniência:(De \textunderscore espavento\textunderscore )}
\end{itemize}
Que espaventa.
Pomposo.
Soberbo; magnifico.
\section{Espavilado}
\begin{itemize}
\item {Grp. gram.:adj.}
\end{itemize}
\begin{itemize}
\item {Utilização:Prov.}
\end{itemize}
\begin{itemize}
\item {Utilização:trasm.}
\end{itemize}
Perspicaz, fino.
\section{Espavorecer}
\begin{itemize}
\item {Grp. gram.:v. t.  e  p.}
\end{itemize}
O mesmo que \textunderscore espavorir\textunderscore .
\section{Espavorir}
\begin{itemize}
\item {Grp. gram.:v. t.}
\end{itemize}
O mesmo que \textunderscore apavorar\textunderscore .
\section{Espavorizar}
\begin{itemize}
\item {Grp. gram.:v. t.  e  p.}
\end{itemize}
O mesmo que \textunderscore espavorir\textunderscore .
\section{Especar}
\begin{itemize}
\item {Grp. gram.:v. t.}
\end{itemize}
\begin{itemize}
\item {Grp. gram.:V. i.  e  p.}
\end{itemize}
Amparar com espeques ou escoras.
Amparar.
Parar, ficar firme como um espeque: \textunderscore o cavallo especou\textunderscore .
\section{Espeçar}
\begin{itemize}
\item {Grp. gram.:v. t.}
\end{itemize}
Tornar mais comprido, (falando-se, em marcenaria, de uma peça, a que se junta outra longitudinalmente).
(Cp. \textunderscore repeçar\textunderscore )
\section{Espécia}
\begin{itemize}
\item {Grp. gram.:f.}
\end{itemize}
(V.espécie)
\section{Especial}
\begin{itemize}
\item {Grp. gram.:adj.}
\end{itemize}
\begin{itemize}
\item {Grp. gram.:M.}
\end{itemize}
\begin{itemize}
\item {Proveniência:(Lat. \textunderscore specialis\textunderscore )}
\end{itemize}
Relativo a uma espécie.
Próprio.
Peculiar.
Particular.
Exclusivo.
Superior, distinto.
Especialista, homem entendido.
\section{Especialidade}
\begin{itemize}
\item {Grp. gram.:f.}
\end{itemize}
\begin{itemize}
\item {Proveniência:(Lat. \textunderscore specialitas\textunderscore )}
\end{itemize}
Qualidade daquillo que é especial.
Coisa superior, distinta: \textunderscore êste queijo é uma especialidade\textunderscore .
Ordem de trabalhos, a que alguém se dedica com particular cuidado: \textunderscore a obstetrícia é a sua especialidade\textunderscore .
\section{Especialista}
\begin{itemize}
\item {Grp. gram.:m.  e  adj.}
\end{itemize}
\begin{itemize}
\item {Proveniência:(De \textunderscore especial\textunderscore )}
\end{itemize}
O que se dedica com particular cuidado a certo estudo ou profissão: \textunderscore especialista em moléstias de olhos\textunderscore .
\section{Especialização}
\begin{itemize}
\item {Grp. gram.:f.}
\end{itemize}
Acto ou effeito de especializar.
\section{Especializar}
\begin{itemize}
\item {Grp. gram.:v. t.}
\end{itemize}
Tornar especial, particularizar.
Distinguir.
Preferir, pôr em primeiro lugar.
\section{Especialmente}
\begin{itemize}
\item {Grp. gram.:adv.}
\end{itemize}
De modo especial.
\section{Especiaria}
\begin{itemize}
\item {Grp. gram.:f.}
\end{itemize}
\begin{itemize}
\item {Proveniência:(De \textunderscore espécie\textunderscore )}
\end{itemize}
Qualquer droga aromática, com que se adubam iguarias.
\section{Espécie}
\begin{itemize}
\item {Grp. gram.:f.}
\end{itemize}
\begin{itemize}
\item {Utilização:Fam.}
\end{itemize}
\begin{itemize}
\item {Proveniência:(Lat. \textunderscore species\textunderscore )}
\end{itemize}
Qualidade.
Condição.
Carácter.
Apparência.
Semelhança externa.
Caso especial.
Gêneros alimentícios, que se dão em pagamento: \textunderscore emprestei-lhe dinheiro, que pagou em espécie\textunderscore .
Dinheiro.
Subdivisão de certas classificações.
Conjunto de seres que têm a mesma essência: \textunderscore espécie humana\textunderscore .
Casta.
Caso especial.
Especiaria.
Doce de amêndoa pisada.
Quantidade da mesma natureza, em arithmética.
Surpresa, intriga: \textunderscore isso faz-me espécie\textunderscore .
\section{Especieira}
\textunderscore fem.\textunderscore  de especieiro.
\section{Especieiro}
\begin{itemize}
\item {Grp. gram.:m.}
\end{itemize}
\begin{itemize}
\item {Proveniência:(Do lat. \textunderscore speciarius\textunderscore )}
\end{itemize}
Aquelle que vende especiarias.
\section{Especificação}
\begin{itemize}
\item {Grp. gram.:f.}
\end{itemize}
Acto ou effeito de especificar.
\section{Especificadamente}
\begin{itemize}
\item {Grp. gram.:adv.}
\end{itemize}
\begin{itemize}
\item {Proveniência:(De \textunderscore especificar\textunderscore )}
\end{itemize}
Minuciosamente.
\section{Especificador}
\begin{itemize}
\item {Grp. gram.:adj.}
\end{itemize}
\begin{itemize}
\item {Grp. gram.:M.}
\end{itemize}
Que especifica.
Aquelle que especifica.
\section{Especificamente}
\begin{itemize}
\item {Grp. gram.:adv.}
\end{itemize}
De modo específico.
\section{Especificar}
\begin{itemize}
\item {Grp. gram.:v. t.}
\end{itemize}
\begin{itemize}
\item {Proveniência:(Lat. \textunderscore specificare\textunderscore )}
\end{itemize}
Indicar a espécie de.
Explicar por miúdo.
Apontar individualmente.
Especializar.
\section{Especificativo}
\begin{itemize}
\item {Grp. gram.:adj.}
\end{itemize}
Que especifica.
\section{Especificidade}
\begin{itemize}
\item {Grp. gram.:f.}
\end{itemize}
\begin{itemize}
\item {Proveniência:(De \textunderscore específico\textunderscore )}
\end{itemize}
Qualidade característica de uma espécie.
\section{Específico}
\begin{itemize}
\item {Grp. gram.:adj.}
\end{itemize}
\begin{itemize}
\item {Grp. gram.:M.}
\end{itemize}
\begin{itemize}
\item {Proveniência:(Lat. \textunderscore specificus\textunderscore )}
\end{itemize}
Relativo a espécie.
Exclusivo.
Especial.
Medicamento, que tem applicação especial a determinada moléstia.
\section{Especilho}
\begin{itemize}
\item {Grp. gram.:m.}
\end{itemize}
\begin{itemize}
\item {Utilização:Ant.}
\end{itemize}
\begin{itemize}
\item {Proveniência:(Lat. \textunderscore specillum\textunderscore )}
\end{itemize}
Tenta.
\section{Espécime}
\begin{itemize}
\item {Grp. gram.:m.}
\end{itemize}
\begin{itemize}
\item {Proveniência:(Lat. \textunderscore specimen\textunderscore )}
\end{itemize}
Modêlo, amostra.
Exemplar.
\section{Espécimen}
\begin{itemize}
\item {Grp. gram.:m.}
\end{itemize}
(V.espécime)
\section{Especione}
\begin{itemize}
\item {Grp. gram.:m.}
\end{itemize}
\begin{itemize}
\item {Utilização:Fam.}
\end{itemize}
\begin{itemize}
\item {Proveniência:(It. \textunderscore spezione\textunderscore )}
\end{itemize}
Bolo tenro de farinha, ovos e açúcar.
\section{Especiosamente}
\begin{itemize}
\item {Grp. gram.:adv.}
\end{itemize}
De modo especioso.
\section{Especiosidade}
\begin{itemize}
\item {Grp. gram.:f.}
\end{itemize}
\begin{itemize}
\item {Proveniência:(Lat. \textunderscore speciositas\textunderscore )}
\end{itemize}
Qualidade daquillo que é especioso.
\section{Especioso}
\begin{itemize}
\item {Grp. gram.:adj.}
\end{itemize}
\begin{itemize}
\item {Proveniência:(Lat. \textunderscore speciosus\textunderscore )}
\end{itemize}
Que tem bôa apparência.
Illusório.
Que induz em êrro, apparentando verdade: \textunderscore argumentos especiosos\textunderscore .
Que seduz.
Bello.
Affectuoso.
\section{Espectáculo}
\begin{itemize}
\item {Grp. gram.:m.}
\end{itemize}
\begin{itemize}
\item {Utilização:Fam.}
\end{itemize}
\begin{itemize}
\item {Proveniência:(Lat. \textunderscore spectaculum\textunderscore )}
\end{itemize}
Tudo que atrái a vista.
Aquillo que prende a attenção.
Perspectiva: \textunderscore o espectáculo da natureza\textunderscore .
Contemplação.
Representação theatral.
Diversão pública em circos.
Scena ridícula; escândalo.
\section{Espectaculoso}
\begin{itemize}
\item {Grp. gram.:adj.}
\end{itemize}
\begin{itemize}
\item {Proveniência:(De \textunderscore espectáculo\textunderscore )}
\end{itemize}
Que dá muito na vista.
Ostentoso; grandioso.
\section{Espectador}
\begin{itemize}
\item {Grp. gram.:m.  e  adj.}
\end{itemize}
\begin{itemize}
\item {Proveniência:(Lat. \textunderscore spectator\textunderscore )}
\end{itemize}
O que assiste a um espectáculo.
O que observa ou que vê qualquer acto.
Testemunha de vista.
\section{Espectativa}
\begin{itemize}
\item {Grp. gram.:f.}
\end{itemize}
(V.expectativa)
\section{Espectável}
\begin{itemize}
\item {Grp. gram.:adj.}
\end{itemize}
\begin{itemize}
\item {Utilização:Des.}
\end{itemize}
\begin{itemize}
\item {Proveniência:(Lat. \textunderscore spectabilis\textunderscore )}
\end{itemize}
Digno de sêr observado.
Notável.
\section{Espectral}
\begin{itemize}
\item {Grp. gram.:adj.}
\end{itemize}
Concernente ao espectro solar.
\section{Espectro}
\begin{itemize}
\item {Grp. gram.:m.}
\end{itemize}
\begin{itemize}
\item {Utilização:Fam.}
\end{itemize}
\begin{itemize}
\item {Grp. gram.:Pl.}
\end{itemize}
\begin{itemize}
\item {Proveniência:(Lat. \textunderscore spectrum\textunderscore )}
\end{itemize}
Imagem fantástica de pessôa fallecida.
\textunderscore Espectro solar\textunderscore , imagem com as côres do arco-íris, resultante da decomposição da luz do sol, através de um prisma em câmara escura.
Pessôa muito magra.
Imagem fantástica, que persegue o criminoso.
Insectos orthópteros, de thoracete comprido.
\section{Espectrologia}
\begin{itemize}
\item {Grp. gram.:f.}
\end{itemize}
\begin{itemize}
\item {Utilização:Phýs.}
\end{itemize}
\begin{itemize}
\item {Proveniência:(Do lat. \textunderscore spectrum\textunderscore  + gr. \textunderscore logos\textunderscore )}
\end{itemize}
Tratado dos phenómenos espectraes.
\section{Espectrológico}
\begin{itemize}
\item {Grp. gram.:adj.}
\end{itemize}
Relativo á espectrologia.
\section{Espectrometria}
\begin{itemize}
\item {Grp. gram.:f.}
\end{itemize}
\begin{itemize}
\item {Proveniência:(De \textunderscore espectrómetro\textunderscore )}
\end{itemize}
Méthodo de anályse quantitativa, para se conhecer a natureza dos elementos de um foco luminoso e determinar a constituição chímica dos corpos.
\section{Espectrométrico}
\begin{itemize}
\item {Grp. gram.:adj.}
\end{itemize}
Relativo á espectrometria.
\section{Espectómetro}
\begin{itemize}
\item {Grp. gram.:m.}
\end{itemize}
\begin{itemize}
\item {Proveniência:(Do lat. \textunderscore spectrum\textunderscore  + gr. \textunderscore metron\textunderscore )}
\end{itemize}
Instrumento, que se emprega na espectrometria.
\section{Espectroscopia}
\begin{itemize}
\item {Grp. gram.:f.}
\end{itemize}
\begin{itemize}
\item {Utilização:Phýs.}
\end{itemize}
Estudo da luz, por meio do espectro fornecido pelo prisma.
(Cp. \textunderscore espectroscópio\textunderscore )
\section{Espectroscópico}
\begin{itemize}
\item {Grp. gram.:adj.}
\end{itemize}
Relativo a espectroscópio ou a espectroscopia.
\section{Espectroscópio}
\begin{itemize}
\item {Grp. gram.:m.}
\end{itemize}
\begin{itemize}
\item {Proveniência:(Do lat. \textunderscore spectrum\textunderscore  + gr. \textunderscore skopein\textunderscore )}
\end{itemize}
Instrumento análogo ao espectrómetro, mas menos perfeito.
\section{Especulação}
\begin{itemize}
\item {Grp. gram.:f.}
\end{itemize}
\begin{itemize}
\item {Proveniência:(Lat. \textunderscore speculatio\textunderscore )}
\end{itemize}
Acto de especular.
Investigação theórica.
Emprehendimento commercial.
Emprehendimento com a mira em lucro.
Exploração ardilosa.
Contracto ou negócio, em que uma das partes abusa da bôa fé da outra.
\section{Especulador}
\begin{itemize}
\item {Grp. gram.:m.  e  adj.}
\end{itemize}
\begin{itemize}
\item {Proveniência:(Lat. \textunderscore speculator\textunderscore )}
\end{itemize}
O que especula.
\section{Especular}
\begin{itemize}
\item {Grp. gram.:adj.}
\end{itemize}
\begin{itemize}
\item {Proveniência:(Lat. \textunderscore specularis\textunderscore )}
\end{itemize}
Diz-se de alguns mineraes, em que há lâminas brilhantes que reflectem a luz.
Relativo a espelho.
\section{Especular}
\begin{itemize}
\item {Grp. gram.:v. t.}
\end{itemize}
\begin{itemize}
\item {Grp. gram.:V. i.}
\end{itemize}
\begin{itemize}
\item {Proveniência:(Lat. \textunderscore speculari\textunderscore )}
\end{itemize}
Contemplar, observar.
Investigar.
Estudar theoricamente.
Explorar.
Tentar commércio ou emprêsa com a mira em lucros, que são eventuaes.
\section{Especulária}
\begin{itemize}
\item {Grp. gram.:f.}
\end{itemize}
\begin{itemize}
\item {Proveniência:(Do lat. \textunderscore specularius\textunderscore )}
\end{itemize}
Secção da Phýsica, em que se trata dos raios reflexos da luz.
\section{Especulativa}
\begin{itemize}
\item {Grp. gram.:f.}
\end{itemize}
\begin{itemize}
\item {Proveniência:(De \textunderscore especulativo\textunderscore )}
\end{itemize}
Faculdade de especular.
\section{Especulativamente}
\begin{itemize}
\item {Grp. gram.:adv.}
\end{itemize}
De modo especulativo.
\section{Especulativo}
\begin{itemize}
\item {Grp. gram.:adj.}
\end{itemize}
\begin{itemize}
\item {Proveniência:(Lat. \textunderscore speculativus\textunderscore )}
\end{itemize}
Em que há especulação.
Que é theórico: \textunderscore estudos especulativos\textunderscore .
Relativo a especulação.
\section{Espéculo}
\begin{itemize}
\item {Grp. gram.:m.}
\end{itemize}
\begin{itemize}
\item {Proveniência:(Lat. \textunderscore speculum\textunderscore )}
\end{itemize}
Instrumento cirúrgico, com que se observam certas cavidades do corpo humano.
\section{Espedaçar}
\textunderscore v. t.\textunderscore  e \textunderscore p.\textunderscore  (e der.)
O mesmo que \textunderscore despedaçar\textunderscore , etc.
\section{Espedida}
\begin{itemize}
\item {Grp. gram.:f.}
\end{itemize}
\begin{itemize}
\item {Utilização:Prov.}
\end{itemize}
\begin{itemize}
\item {Utilização:trasm.}
\end{itemize}
\begin{itemize}
\item {Proveniência:(De \textunderscore espedir\textunderscore )}
\end{itemize}
O mesmo que \textunderscore despedida\textunderscore .
\section{Espedir}
\textunderscore v. t.\textunderscore  (e der.)
O mesmo que \textunderscore despedir\textunderscore , etc.:«\textunderscore ...não me posso espedir donde vejo estar cantando\textunderscore ». Camões, \textunderscore Seleuco\textunderscore , (prol.) Cf. Barros, \textunderscore passim\textunderscore .
\section{Espedregar}
\begin{itemize}
\item {Grp. gram.:v. t.}
\end{itemize}
Limpar de pedras.
\section{Espeitamento}
\begin{itemize}
\item {Grp. gram.:m.}
\end{itemize}
Acto ou effeito de espeitar.
\section{Espeitar}
\begin{itemize}
\item {Grp. gram.:v. t.}
\end{itemize}
\begin{itemize}
\item {Utilização:Ant.}
\end{itemize}
\begin{itemize}
\item {Proveniência:(Do lat. \textunderscore spectare\textunderscore )}
\end{itemize}
Vexar, espreitando, espiando.
\section{Espelde}
\begin{itemize}
\item {Grp. gram.:m.}
\end{itemize}
\begin{itemize}
\item {Utilização:Prov.}
\end{itemize}
\begin{itemize}
\item {Utilização:trasm.}
\end{itemize}
Desembaraço.
Rasgo.
Expediente.
\section{Espeleologia}
\begin{itemize}
\item {Grp. gram.:f.}
\end{itemize}
\begin{itemize}
\item {Proveniência:(Do gr. \textunderscore spelaiou\textunderscore  + \textunderscore logos\textunderscore )}
\end{itemize}
Parte da Geologia, que se occupa das grutas e cavernas.
\section{Espeleu}
\begin{itemize}
\item {Grp. gram.:m.}
\end{itemize}
Gênero de veados fósseis.
(Cp. gr. \textunderscore spelaion\textunderscore , caverna)
\section{Espelhação}
\begin{itemize}
\item {Grp. gram.:f.}
\end{itemize}
O mesmo que \textunderscore espelhamento\textunderscore .
\section{Espelhamento}
\begin{itemize}
\item {Grp. gram.:m.}
\end{itemize}
Acto ou effeito de espelhar. Cf. Ortigão, \textunderscore Hollanda\textunderscore , 100.
\section{Espelhante}
\begin{itemize}
\item {Grp. gram.:adj.}
\end{itemize}
Que espelha: \textunderscore água espelhante\textunderscore . Cf. Filinto, XIV, 178.
\section{Espelhar}
\begin{itemize}
\item {Grp. gram.:v. t.}
\end{itemize}
\begin{itemize}
\item {Grp. gram.:V. p.}
\end{itemize}
\begin{itemize}
\item {Utilização:Fig.}
\end{itemize}
Tornar liso, polido, crystallino, como um espelho: \textunderscore espelhar metaes\textunderscore .
Reflectir como um espelho: \textunderscore o rio espelha a lua\textunderscore .
Reflectir-se.
Ver-se ao espelho.
Rever-se, têr prazer na contemplação de alguém ou de alguma coisa.
\section{Espelharia}
\begin{itemize}
\item {Grp. gram.:f.}
\end{itemize}
Lugar, onde se vendem ou se fabricam espelhos.
\section{Espelheiro}
\begin{itemize}
\item {Grp. gram.:m.}
\end{itemize}
Fabricante ou vendedor de espelhos.
\section{Espelhento}
\begin{itemize}
\item {Grp. gram.:adj.}
\end{itemize}
Polido.
Que reflecte como um espelho: \textunderscore superfície espelhenta\textunderscore .
\section{Espelhim}
\begin{itemize}
\item {Grp. gram.:m.}
\end{itemize}
\begin{itemize}
\item {Proveniência:(De \textunderscore espêlho\textunderscore )}
\end{itemize}
Gêsso branco de apparência lustrosa.
\section{Espêlho}
\begin{itemize}
\item {Grp. gram.:m.}
\end{itemize}
\begin{itemize}
\item {Utilização:Carp.}
\end{itemize}
\begin{itemize}
\item {Utilização:Carp.}
\end{itemize}
\begin{itemize}
\item {Utilização:Carp.}
\end{itemize}
\begin{itemize}
\item {Utilização:Fig.}
\end{itemize}
\begin{itemize}
\item {Proveniência:(Lat. \textunderscore speculum\textunderscore )}
\end{itemize}
Superfície polida, que reflecte a luz, ou que representa os objectos que estão ou se lhe põem deante.
Lâmina de vidro ou crystal polido, estanhada posteriormente, para adôrno de casas ou móveis, ou para nella se rever quem se veste, se barbeia ou se enfeita.
Chapa exterior de uma fechadura.
Chapa de ferro, que, collocada no divisor das máquinas de vapor, serve para regular a entrada do vapor no cylíndro.
Quartos de fazenda, que, na feitura de casacos, blusas, etc., assentam no peito ou nas costas.
Tábua, que resai, de alto a baixo, na face de uma porta.
A face anterior de uma gaveta.
Peça da frente de um degrau, também chamada \textunderscore pé\textunderscore .
Maçan de casca luzidia e encarnada.
Plano da bôca de um canhão.
Abertura no tampo de certos instrumentos de corda.
Insecto lepidóptero.
Abertura envidraçada, no frontispício de uma igreja.
Superfície tranquilla e transparente das águas.
Exemplo, ensinamento, modêlo: \textunderscore meu pai, meu nobre espêlho\textunderscore .
Tudo que reflecte ou reproduz um sentimento.
\section{Espelina}
\begin{itemize}
\item {Grp. gram.:f.}
\end{itemize}
Planta cucurbitácea do Brasil, (\textunderscore perianthopodus espelina\textunderscore ).
\section{Espeloteado}
\begin{itemize}
\item {Grp. gram.:adj.}
\end{itemize}
\begin{itemize}
\item {Utilização:Bras}
\end{itemize}
\begin{itemize}
\item {Proveniência:(De \textunderscore pelota\textunderscore , alludindo-se ao facto de um pássaro ficar ferido na cabeça, mas não morto, por pelotada)}
\end{itemize}
Tonto.
\section{Espelta}
\begin{itemize}
\item {Grp. gram.:f.}
\end{itemize}
\begin{itemize}
\item {Proveniência:(Lat. \textunderscore spelta\textunderscore )}
\end{itemize}
Espécie de trigo, de inferior qualidade, (\textunderscore titricum spelta\textunderscore , Lin.).
\section{Espelunca}
\begin{itemize}
\item {Grp. gram.:f.}
\end{itemize}
\begin{itemize}
\item {Utilização:Fig.}
\end{itemize}
\begin{itemize}
\item {Proveniência:(Lat. \textunderscore spelunca\textunderscore )}
\end{itemize}
Caverna, cova.
Casa immunda.
Lugar esconso, em que se joga.
\section{Espenda}
\begin{itemize}
\item {Grp. gram.:f.}
\end{itemize}
\begin{itemize}
\item {Utilização:Des.}
\end{itemize}
\begin{itemize}
\item {Proveniência:(De \textunderscore pender\textunderscore ?)}
\end{itemize}
Parte da sella, em que assenta a coxa.
\section{Espenejar}
\begin{itemize}
\item {Proveniência:(De \textunderscore pena\textunderscore )}
\end{itemize}
\textunderscore v. t.\textunderscore  (e der.)
O mesmo que \textunderscore espanejar\textunderscore , etc.
\section{Espenglério}
\begin{itemize}
\item {Grp. gram.:m.}
\end{itemize}
Peixe, de peito rubro, que vive nas profundidades do Oceano Índico.
\section{Espenicar}
\begin{itemize}
\item {Grp. gram.:v. t.}
\end{itemize}
\begin{itemize}
\item {Utilização:Fam.}
\end{itemize}
\begin{itemize}
\item {Utilização:Fig.}
\end{itemize}
\begin{itemize}
\item {Proveniência:(De \textunderscore pena\textunderscore )}
\end{itemize}
Depenar.
Ataviar excessivamente.
Analisar ou observar por miúdo.
\section{Espenifrar}
\begin{itemize}
\item {Grp. gram.:v. i.}
\end{itemize}
Ganhar ao espenifre.
\section{Espenifre}
\begin{itemize}
\item {Grp. gram.:m.}
\end{itemize}
Antigo jôgo de cartas, em que o dois de paus era a carta de mais valor.
\section{Espennejar}
\begin{itemize}
\item {Proveniência:(De \textunderscore penna\textunderscore )}
\end{itemize}
\textunderscore v. t.\textunderscore  (e der.)
O mesmo que \textunderscore espanejar\textunderscore , etc.
\section{Espennicar}
\begin{itemize}
\item {Grp. gram.:v. t.}
\end{itemize}
\begin{itemize}
\item {Utilização:Fam.}
\end{itemize}
\begin{itemize}
\item {Utilização:Fig.}
\end{itemize}
\begin{itemize}
\item {Proveniência:(De \textunderscore penna\textunderscore )}
\end{itemize}
Depennar.
Ataviar excessivamente.
Analysar ou observar por miúdo.
\section{Espennujar}
\begin{itemize}
\item {Grp. gram.:v. i.  e  p.}
\end{itemize}
\begin{itemize}
\item {Utilização:Prov.}
\end{itemize}
\begin{itemize}
\item {Proveniência:(De \textunderscore pennugem\textunderscore )}
\end{itemize}
Agitar ou sacudir as pennas (uma ave).
\section{Espenujar}
\begin{itemize}
\item {Grp. gram.:v. i.  e  p.}
\end{itemize}
\begin{itemize}
\item {Utilização:Prov.}
\end{itemize}
\begin{itemize}
\item {Proveniência:(De \textunderscore penugem\textunderscore )}
\end{itemize}
Agitar ou sacudir as penas (uma ave).
\section{Espeque}
\begin{itemize}
\item {Grp. gram.:m.}
\end{itemize}
\begin{itemize}
\item {Utilização:Fig.}
\end{itemize}
\begin{itemize}
\item {Proveniência:(Do germ. \textunderscore spaak\textunderscore )}
\end{itemize}
Peça de madeira, com que se escora alguma coisa; escora.
Amparo.
Palliativo.
\section{Espera}
\begin{itemize}
\item {Grp. gram.:f.}
\end{itemize}
\begin{itemize}
\item {Utilização:Prov.}
\end{itemize}
\begin{itemize}
\item {Grp. gram.:Pl.}
\end{itemize}
Acto de esperar.
Esperança.
Adiamento.
Demora.
Lugar, em que se espera.
Emboscada.
Peça de ferro, cravada numa das extremidades do banco de carpinteiro ou marceneiro, para segurar as tábuas que se aplainam e outras peças de madeira em que se trabalha.
Antiga peça de artilharia. Cf. \textunderscore Peregrinação\textunderscore , LVIII.
Espigão de ferro ou madeira, de applicação análoga á da espera de carpinteiro.
Pequena vara, que os podadores deixam em sitio anterior á vara da poda, para obstar ao alongamento das vides, e que cortam no anno seguinte, ficando assim mais curta e mais forte a vara da poda.
O mesmo que \textunderscore mata-boi\textunderscore .
Turno de caçadores, que esperam a passagem da caça, em-quanto os monteadores batem o mato.
\section{Espera}
\begin{itemize}
\item {Grp. gram.:f.}
\end{itemize}
O mesmo que \textunderscore esphera\textunderscore .
(Cp. it. ant. \textunderscore spera\textunderscore , em Dante)
\section{Éspera}
\begin{itemize}
\item {Grp. gram.:f.}
\end{itemize}
\begin{itemize}
\item {Proveniência:(De \textunderscore Esper\textunderscore , n. p.)}
\end{itemize}
Gênero de algas marinhas.
\section{Esperadamente}
\begin{itemize}
\item {Grp. gram.:adv.}
\end{itemize}
\begin{itemize}
\item {Proveniência:(De \textunderscore esperar\textunderscore )}
\end{itemize}
Com esperança.
\section{Esperadoiro}
\begin{itemize}
\item {Grp. gram.:m.}
\end{itemize}
Lugar, onde se espera.
\section{Esperador}
\begin{itemize}
\item {Grp. gram.:m.  e  adj.}
\end{itemize}
O que espera.
\section{Esperadouro}
\begin{itemize}
\item {Grp. gram.:m.}
\end{itemize}
Lugar, onde se espera.
\section{Esperagana}
\begin{itemize}
\item {Grp. gram.:f.}
\end{itemize}
Espécie de tecido antigo.
\section{Esperança}
\begin{itemize}
\item {Grp. gram.:f.}
\end{itemize}
\begin{itemize}
\item {Grp. gram.:Pl. Loc.}
\end{itemize}
\begin{itemize}
\item {Utilização:fam.}
\end{itemize}
\begin{itemize}
\item {Proveniência:(De \textunderscore esperar\textunderscore )}
\end{itemize}
Aguardamento de um bem que se deseja, e cuja realização se julga provável.
A segunda das três virtudes theologaes.
Aquillo que se espera, desejando.
Expectativa.
\textunderscore Andar de esperanças\textunderscore , diz-se da mulher que anda grávida.
\section{Esperançar}
\begin{itemize}
\item {Grp. gram.:v. t.}
\end{itemize}
\begin{itemize}
\item {Grp. gram.:V. p.}
\end{itemize}
Dar esperanças a.
Têr esperança.
\section{Esperançoso}
\begin{itemize}
\item {Grp. gram.:adj.}
\end{itemize}
Que dá ou tem esperança.
Prometedor.
\section{Esperante}
\begin{itemize}
\item {Grp. gram.:adj.}
\end{itemize}
\begin{itemize}
\item {Utilização:Des.}
\end{itemize}
\begin{itemize}
\item {Proveniência:(Lat. \textunderscore sperans\textunderscore )}
\end{itemize}
Que espera.
\section{Esperanto}
\begin{itemize}
\item {Grp. gram.:m.}
\end{itemize}
Língua internacional, fundada em 1887 pelo Dr. Zamenhof, e cuja grammática abrange apenas dezaseis regras.
\section{Esperar}
\begin{itemize}
\item {Grp. gram.:v. t.}
\end{itemize}
\begin{itemize}
\item {Grp. gram.:V. i.}
\end{itemize}
\begin{itemize}
\item {Proveniência:(Lat. \textunderscore sperare\textunderscore )}
\end{itemize}
Têr esperança em ou de.
Aguardar.
Têr como provavel.
Suppor.
Aguardar em emboscada.
Confiar: \textunderscore espero em ti\textunderscore .
Aguardar alguém.
Têr esperança:«\textunderscore ...espera de o alcançar.\textunderscore »Vieira.
Manter-se na expectativa.
\section{Esperável}
\begin{itemize}
\item {Grp. gram.:adj.}
\end{itemize}
\begin{itemize}
\item {Proveniência:(Do lat. \textunderscore sperabilis\textunderscore )}
\end{itemize}
Provável.
Que se póde esperar.
\section{Esperavél}
\begin{itemize}
\item {Grp. gram.:m.}
\end{itemize}
\begin{itemize}
\item {Utilização:Ant.}
\end{itemize}
O mesmo que \textunderscore esparavél\textunderscore .
\section{Esperdiçado}
\begin{itemize}
\item {Grp. gram.:adj.}
\end{itemize}
\begin{itemize}
\item {Utilização:Des.}
\end{itemize}
\begin{itemize}
\item {Utilização:Prov.}
\end{itemize}
Extasiado, arroubado. Cf. Filinto, VIII, 12.
Querido, estimado.
\section{Esperdiçar}
\textunderscore v. t.\textunderscore  (e der.)
O mesmo que \textunderscore desperdiçar\textunderscore .
\section{Esperdigotado}
\begin{itemize}
\item {Grp. gram.:adj.}
\end{itemize}
\begin{itemize}
\item {Utilização:Prov.}
\end{itemize}
\begin{itemize}
\item {Utilização:alg.}
\end{itemize}
\begin{itemize}
\item {Proveniência:(De \textunderscore perdigoto\textunderscore )}
\end{itemize}
Espavorido.
Lânguido.
\section{Esperdigotar}
\begin{itemize}
\item {Grp. gram.:v. i.}
\end{itemize}
\begin{itemize}
\item {Utilização:Prov.}
\end{itemize}
\begin{itemize}
\item {Utilização:trasm.}
\end{itemize}
\begin{itemize}
\item {Proveniência:(De \textunderscore perdigoto\textunderscore )}
\end{itemize}
Desenvolver-se, perder o acanhamento.
\section{Espérgula}
\begin{itemize}
\item {Grp. gram.:f.}
\end{itemize}
Planta leguminosa, (\textunderscore spergula\textunderscore ).
\section{Esperiega}
\begin{itemize}
\item {Grp. gram.:f.}
\end{itemize}
Variedade de maçan, muito apreciada. Cf. D. de Oliveira, \textunderscore Dicc. das Pêras\textunderscore .
\section{Esperma}
\begin{itemize}
\item {Grp. gram.:m.}
\end{itemize}
\begin{itemize}
\item {Proveniência:(Gr. \textunderscore sperma\textunderscore )}
\end{itemize}
Líquido seminal; semen.
\section{Espermacete}
\begin{itemize}
\item {Grp. gram.:m.}
\end{itemize}
\begin{itemize}
\item {Proveniência:(Do gr. \textunderscore sperma\textunderscore  + \textunderscore ketos\textunderscore )}
\end{itemize}
Substância branca, extrahida da cabeça dos cachalotes.
\section{Espermacéti}
\begin{itemize}
\item {Grp. gram.:m.}
\end{itemize}
O mesmo que \textunderscore espermacete\textunderscore . Cf. Fil. Simões, \textunderscore C. da Beiramar\textunderscore , 299.
\section{Espermático}
\begin{itemize}
\item {Grp. gram.:adj.}
\end{itemize}
\begin{itemize}
\item {Proveniência:(Gr. \textunderscore spermatikos\textunderscore )}
\end{itemize}
Relativo a esperma.
\section{Espermatizar}
\begin{itemize}
\item {Grp. gram.:v. t.}
\end{itemize}
\begin{itemize}
\item {Proveniência:(De \textunderscore esperma\textunderscore )}
\end{itemize}
Fecundar.
\section{Espermatocele}
\begin{itemize}
\item {Grp. gram.:m.}
\end{itemize}
\begin{itemize}
\item {Proveniência:(Do gr. \textunderscore sperma\textunderscore  + \textunderscore kele\textunderscore )}
\end{itemize}
Engorgitamento, resultante da acumulação de esperma.
\section{Espermatófago}
\begin{itemize}
\item {Grp. gram.:adj.}
\end{itemize}
\begin{itemize}
\item {Proveniência:(Do gr. \textunderscore sperma\textunderscore  + \textunderscore phagein\textunderscore )}
\end{itemize}
Que se alimenta de grãos ou sementes vegetaes. Cf. \textunderscore Ethióp. Or.\textunderscore , l. I, c. 1.
\section{Espermatóforo}
\begin{itemize}
\item {Grp. gram.:m.}
\end{itemize}
\begin{itemize}
\item {Utilização:Zool.}
\end{itemize}
\begin{itemize}
\item {Proveniência:(Do gr. \textunderscore sperma\textunderscore  + \textunderscore phoros\textunderscore )}
\end{itemize}
Corpo vermiforme, que rodeia uma massa cylíndrica de espermatozoides, próprio de certos animaes inferiores.
\section{Espermatografia}
\begin{itemize}
\item {Grp. gram.:f.}
\end{itemize}
\begin{itemize}
\item {Utilização:Bot.}
\end{itemize}
\begin{itemize}
\item {Proveniência:(Do gr. \textunderscore sperma\textunderscore  + \textunderscore graphein\textunderscore )}
\end{itemize}
Descripção das sementes.
\section{Espermatográfico}
\begin{itemize}
\item {Grp. gram.:adj.}
\end{itemize}
Relativo a espermatographia.
\section{Espermatógrafo}
\begin{itemize}
\item {Grp. gram.:m.}
\end{itemize}
Aquele que se occupa de espermatografia.
\section{Espermatographia}
\begin{itemize}
\item {Grp. gram.:f.}
\end{itemize}
\begin{itemize}
\item {Utilização:Bot.}
\end{itemize}
\begin{itemize}
\item {Proveniência:(Do gr. \textunderscore sperma\textunderscore  + \textunderscore graphein\textunderscore )}
\end{itemize}
Descripção das sementes.
\section{Espermatográphico}
\begin{itemize}
\item {Grp. gram.:adj.}
\end{itemize}
Relativo a espermatographia.
\section{Espermatógrapho}
\begin{itemize}
\item {Grp. gram.:m.}
\end{itemize}
Aquelle que se occupa de espermatographia.
\section{Espermatologia}
\begin{itemize}
\item {Grp. gram.:f.}
\end{itemize}
\begin{itemize}
\item {Proveniência:(Do gr. \textunderscore sperma\textunderscore  + \textunderscore logos\textunderscore )}
\end{itemize}
Tratado do esperma.
\section{Espermatóphago}
\begin{itemize}
\item {Grp. gram.:adj.}
\end{itemize}
\begin{itemize}
\item {Proveniência:(Do gr. \textunderscore sperma\textunderscore  + \textunderscore phagein\textunderscore )}
\end{itemize}
Que se alimenta de grãos ou sementes vegetaes. Cf. \textunderscore Ethióp. Or.\textunderscore , l. I, c. 1.
\section{Espermatóphoro}
\begin{itemize}
\item {Grp. gram.:m.}
\end{itemize}
\begin{itemize}
\item {Utilização:Zool.}
\end{itemize}
\begin{itemize}
\item {Proveniência:(Do gr. \textunderscore sperma\textunderscore  + \textunderscore phoros\textunderscore )}
\end{itemize}
Corpo vermiforme, que rodeia uma massa cylíndrica de espermatozoides, próprio de certos animaes inferiores.
\section{Espermatorreia}
\begin{itemize}
\item {Grp. gram.:f.}
\end{itemize}
\begin{itemize}
\item {Proveniência:(Do gr. \textunderscore sperma\textunderscore  + \textunderscore rhein\textunderscore )}
\end{itemize}
Derramamento involuntário de esperma.
\section{Espermatorrhéa}
\begin{itemize}
\item {Grp. gram.:f.}
\end{itemize}
\begin{itemize}
\item {Proveniência:(Do gr. \textunderscore sperma\textunderscore  + \textunderscore rhein\textunderscore )}
\end{itemize}
Derramamento involuntário de esperma.
\section{Espermatorrheia}
\begin{itemize}
\item {Grp. gram.:f.}
\end{itemize}
\begin{itemize}
\item {Proveniência:(Do gr. \textunderscore sperma\textunderscore  + \textunderscore rhein\textunderscore )}
\end{itemize}
Derramamento involuntário de esperma.
\section{Espermatose}
\begin{itemize}
\item {Grp. gram.:f.}
\end{itemize}
\begin{itemize}
\item {Proveniência:(Do gr. \textunderscore sperma\textunderscore )}
\end{itemize}
Formação do esperma.
\section{Espermatozoário}
\begin{itemize}
\item {Grp. gram.:m.}
\end{itemize}
O mesmo que \textunderscore espermatozoide\textunderscore .
\section{Espermatozoide}
\begin{itemize}
\item {Grp. gram.:m.}
\end{itemize}
\begin{itemize}
\item {Proveniência:(Do gr. \textunderscore sperma\textunderscore  + \textunderscore zoon\textunderscore )}
\end{itemize}
Elemento anatómico do corpo dos animaes e da maior parte das plantas, que é animado de movimentos próprios e exerce a funcção do corpúsculo fecundante.
\section{Espermina}
\begin{itemize}
\item {Grp. gram.:f.}
\end{itemize}
\begin{itemize}
\item {Utilização:Med.}
\end{itemize}
Base medicamentosa, extrahida do esperma.
\section{Espermólitho}
\begin{itemize}
\item {Grp. gram.:m.}
\end{itemize}
\begin{itemize}
\item {Utilização:Med.}
\end{itemize}
\begin{itemize}
\item {Proveniência:(Do gr. \textunderscore sperma\textunderscore  + \textunderscore lithos\textunderscore )}
\end{itemize}
Cálculo das vias espermáticas.
\section{Espermólito}
\begin{itemize}
\item {Grp. gram.:m.}
\end{itemize}
\begin{itemize}
\item {Utilização:Med.}
\end{itemize}
\begin{itemize}
\item {Proveniência:(Do gr. \textunderscore sperma\textunderscore  + \textunderscore lithos\textunderscore )}
\end{itemize}
Cálculo das vias espermáticas.
\section{Espernear}
\begin{itemize}
\item {Grp. gram.:v. i.}
\end{itemize}
O mesmo que \textunderscore pernear\textunderscore .
\section{Espernegar}
\begin{itemize}
\item {Grp. gram.:v. t.}
\end{itemize}
\begin{itemize}
\item {Grp. gram.:V. p.}
\end{itemize}
\begin{itemize}
\item {Proveniência:(De \textunderscore perna\textunderscore )}
\end{itemize}
Deitar de pernas ao ar.
Estender-se ao comprido, estatelar-se.
\section{Espertador}
\begin{itemize}
\item {Grp. gram.:adj.}
\end{itemize}
\begin{itemize}
\item {Grp. gram.:M.}
\end{itemize}
Que esperta.
Aquelle ou aquillo que esperta.
\section{Espertadura}
\begin{itemize}
\item {Grp. gram.:f.}
\end{itemize}
\begin{itemize}
\item {Utilização:Ant.}
\end{itemize}
\begin{itemize}
\item {Proveniência:(De \textunderscore espertar\textunderscore ? ou por \textunderscore apartadura\textunderscore , de \textunderscore apartar\textunderscore ?)}
\end{itemize}
Divisão ou rêgo no topête do cabello.
\section{Espertalhaço}
\begin{itemize}
\item {Grp. gram.:m.  e  adj.}
\end{itemize}
\begin{itemize}
\item {Utilização:Fam.}
\end{itemize}
Sujeito finório, espertalhão.
\section{Espertalhão}
\begin{itemize}
\item {Grp. gram.:m.  e  adj.}
\end{itemize}
\begin{itemize}
\item {Utilização:Fam.}
\end{itemize}
\begin{itemize}
\item {Proveniência:(De \textunderscore esperto\textunderscore )}
\end{itemize}
Homem astuto, que tem esperteza maliciosa.
\section{Espertamente}
\begin{itemize}
\item {Grp. gram.:adv.}
\end{itemize}
\begin{itemize}
\item {Proveniência:(De \textunderscore esperto\textunderscore )}
\end{itemize}
Com sagacidade, com habilidade.
\section{Espertamento}
\begin{itemize}
\item {Grp. gram.:m.}
\end{itemize}
Acto de espertar.
\section{Espertar}
\begin{itemize}
\item {Grp. gram.:v. t.}
\end{itemize}
\begin{itemize}
\item {Grp. gram.:V. i.}
\end{itemize}
\begin{itemize}
\item {Proveniência:(De \textunderscore esperto\textunderscore )}
\end{itemize}
Estimular.
Avivar: \textunderscore espertar uma luz\textunderscore .
Despertar.
Despertar, animar-se:«\textunderscore espertava sob a pressão dos beijos\textunderscore ». Camillo, \textunderscore Caveira\textunderscore , 167.
\section{Esperteza}
\begin{itemize}
\item {Grp. gram.:f.}
\end{itemize}
Qualidade de esperto.
Acção ou dito de pessôa esperta.
\section{Espertina}
\begin{itemize}
\item {Grp. gram.:f.}
\end{itemize}
\begin{itemize}
\item {Proveniência:(De \textunderscore esperto\textunderscore )}
\end{itemize}
Insômnia.
\section{Espertinar}
\begin{itemize}
\item {Grp. gram.:v. t.}
\end{itemize}
\begin{itemize}
\item {Grp. gram.:V. i.}
\end{itemize}
Causar espertina a.
Ter espertina:«\textunderscore espertinou e ficou sobresaltado.\textunderscore »Camillo, \textunderscore Corja\textunderscore , 215.
\section{Esperto}
\begin{itemize}
\item {Grp. gram.:adj.}
\end{itemize}
\begin{itemize}
\item {Utilização:Bras. do N}
\end{itemize}
\begin{itemize}
\item {Proveniência:(Do lat. \textunderscore expertus\textunderscore )}
\end{itemize}
Intelligente.
Fino; finório.
Enérgico.
Vivo.
Estimulado.
Que tem agudeza e actividade.
Despertado.
Morno, quási quente.
\section{Espescoçar}
\begin{itemize}
\item {Grp. gram.:v. t.}
\end{itemize}
\begin{itemize}
\item {Proveniência:(De \textunderscore pescoço\textunderscore )}
\end{itemize}
Cavar em tôrno da videira.
Fazer mergulhia de (videiras).
\section{Espessamente}
\begin{itemize}
\item {Grp. gram.:adv.}
\end{itemize}
De modo espêsso.
\section{Espessamento}
\begin{itemize}
\item {Grp. gram.:m.}
\end{itemize}
Acto ou effeito de espessar.
\section{Espessar}
\begin{itemize}
\item {Grp. gram.:v. t.}
\end{itemize}
\begin{itemize}
\item {Proveniência:(Lat. \textunderscore spissar\textunderscore )}
\end{itemize}
Fazer espêsso, grosso, denso.
\section{Espessidão}
\begin{itemize}
\item {Grp. gram.:f.}
\end{itemize}
Qualidade daquillo que é espêsso.
\section{Espêsso}
\begin{itemize}
\item {Grp. gram.:adj.}
\end{itemize}
\begin{itemize}
\item {Proveniência:(Lat. \textunderscore spissus\textunderscore )}
\end{itemize}
Compacto: \textunderscore multidão espêssa\textunderscore .
Condensado: \textunderscore caldo espêsso\textunderscore .
Grosso.
Basto: \textunderscore arvoredo espêsso\textunderscore ; \textunderscore cabello espêsso\textunderscore .
Consistente.
Opaco; ramoso: \textunderscore árvore espêssa\textunderscore .
\section{Espessor}
\begin{itemize}
\item {Grp. gram.:m.}
\end{itemize}
O mesmo que \textunderscore espessura\textunderscore . Cf. Castilho, \textunderscore Avarento\textunderscore , 360.
\section{Espessura}
\begin{itemize}
\item {Grp. gram.:f.}
\end{itemize}
\begin{itemize}
\item {Proveniência:(De \textunderscore espêsso\textunderscore )}
\end{itemize}
Qualidade de espêsso, espessidão.
Bosque; mata cerrada.
Densidade.
\section{Espeta}
\begin{itemize}
\item {Grp. gram.:f.}
\end{itemize}
Espécie de jôgo popular.
\section{Espetada}
\begin{itemize}
\item {Grp. gram.:f.}
\end{itemize}
\begin{itemize}
\item {Utilização:Fam.}
\end{itemize}
Golpe de espêto.
Enfiada de coisas, que se assam com o espêto:«\textunderscore ...perdizes de espetada.\textunderscore »Camillo, \textunderscore Caveira\textunderscore , 468.
\section{Espetadela}
\begin{itemize}
\item {Grp. gram.:f.}
\end{itemize}
\begin{itemize}
\item {Utilização:Fam.}
\end{itemize}
Espetada.
Acto de espetar.
Arriosca; entaladela.
\section{Espetanço}
\begin{itemize}
\item {Grp. gram.:m.}
\end{itemize}
\begin{itemize}
\item {Utilização:Chul.}
\end{itemize}
\begin{itemize}
\item {Proveniência:(De \textunderscore espetar\textunderscore )}
\end{itemize}
Espetadela.
Arriosca; lôgro.
\section{Espetão}
\begin{itemize}
\item {Grp. gram.:m.}
\end{itemize}
\begin{itemize}
\item {Proveniência:(De \textunderscore espeto\textunderscore )}
\end{itemize}
Instrumento de ferro, com que se tira da forja o cadinho.
Ferro aguçado, com que os artilheiros desmancham revestimento de argilla.
Nome de outros objectos, semelhantes a espêto, maiores que elle.
\section{Espetar}
\begin{itemize}
\item {Grp. gram.:v. t.}
\end{itemize}
\begin{itemize}
\item {Utilização:Fig.}
\end{itemize}
\begin{itemize}
\item {Utilização:Fam.}
\end{itemize}
Furar com espêto.
Traspassar.
Enfiar.
Cravar: \textunderscore espetou-lhe uma faca no peito\textunderscore .
Lograr, entalar.
Impingir.
\section{Espeteira}
\begin{itemize}
\item {Grp. gram.:f.}
\end{itemize}
\begin{itemize}
\item {Utilização:Prov.}
\end{itemize}
\begin{itemize}
\item {Utilização:trasm.}
\end{itemize}
Gancho nos tabuados das cozinhas, loiceiros, etc., para pendurar carne, vasilhas, etc.
(Cast. \textunderscore espetera\textunderscore )
\section{Esfacelamento}
\begin{itemize}
\item {Grp. gram.:m.}
\end{itemize}
O mesmo que \textunderscore esfacêlo\textunderscore . Cf. Camillo, \textunderscore Perfil do Marquês\textunderscore , 211.
\section{Esfacelar}
\begin{itemize}
\item {Grp. gram.:v. t.}
\end{itemize}
\begin{itemize}
\item {Utilização:Fig.}
\end{itemize}
Causar esfacêlo a.
Gangrenar.
Destruir.
Estragar.
Desfazer.
\section{Esfacêlo}
\begin{itemize}
\item {Grp. gram.:m.}
\end{itemize}
\begin{itemize}
\item {Utilização:Fig.}
\end{itemize}
\begin{itemize}
\item {Proveniência:(Gr. \textunderscore sphakelos\textunderscore )}
\end{itemize}
Gangrena de um membro ou de um órgão.
Estrago, destruição.
\section{Esfalerite}
\begin{itemize}
\item {Grp. gram.:f.}
\end{itemize}
Espécie de galenite.
\section{Esfena}
\begin{itemize}
\item {Grp. gram.:f.}
\end{itemize}
O mesmo que \textunderscore esfeno\textunderscore .
\section{Esfeno}
\begin{itemize}
\item {Grp. gram.:m.}
\end{itemize}
\begin{itemize}
\item {Proveniência:(Do gr. \textunderscore sphen\textunderscore )}
\end{itemize}
Espécie de mineral cristalino, translúcido.
\section{Esfenocéfalo}
\begin{itemize}
\item {Grp. gram.:adj.}
\end{itemize}
\begin{itemize}
\item {Proveniência:(Do gr. \textunderscore sphen\textunderscore , cunha, e \textunderscore kephale\textunderscore , cabeça)}
\end{itemize}
Que tem a cabeça ponteaguda.
\section{Esfenoédro}
\begin{itemize}
\item {Grp. gram.:adj.}
\end{itemize}
\begin{itemize}
\item {Utilização:Geom.}
\end{itemize}
\begin{itemize}
\item {Proveniência:(Do gr. \textunderscore sphen\textunderscore  + \textunderscore edra\textunderscore )}
\end{itemize}
Poliedro, com algum ou alguns ângulos agudos.
\section{Esfenoidal}
\begin{itemize}
\item {Grp. gram.:adj.}
\end{itemize}
Relativo a esfenoide.
\section{Esfenoide}
\begin{itemize}
\item {Grp. gram.:m.}
\end{itemize}
\begin{itemize}
\item {Utilização:Anat.}
\end{itemize}
\begin{itemize}
\item {Proveniência:(Gr. \textunderscore sphenoeides\textunderscore )}
\end{itemize}
Osso ímpar, situado no meio dos ossos da base do crânio, e que concorre para a formação das órbitas, das cavidades nasaes, etc.
\section{Esféria}
\begin{itemize}
\item {Grp. gram.:f.}
\end{itemize}
\begin{itemize}
\item {Proveniência:(Do gr. \textunderscore sphairion\textunderscore )}
\end{itemize}
Gênero de cogumelos.
Gênero de insectos dípteros.
\section{Esferiáceas}
\begin{itemize}
\item {Grp. gram.:f. pl.}
\end{itemize}
\begin{itemize}
\item {Proveniência:(De \textunderscore esferiáceo\textunderscore )}
\end{itemize}
Ordem de cogumelos.
\section{Esferiáceo}
\begin{itemize}
\item {Grp. gram.:adj.}
\end{itemize}
Relativo ou semelhante a esféria.
\section{Esfericamente}
\begin{itemize}
\item {fónica:fé}
\end{itemize}
\begin{itemize}
\item {Grp. gram.:adv.}
\end{itemize}
De modo esférico.
Á maneira ou semelhança de esfera.
\section{Esfericidade}
\begin{itemize}
\item {fónica:fé}
\end{itemize}
\begin{itemize}
\item {Grp. gram.:f.}
\end{itemize}
Qualidade daquilo que é esférico.
\section{Esférico}
\begin{itemize}
\item {Grp. gram.:adj.}
\end{itemize}
Relativo a esfera.
Que tem fórma de esfera.
\section{Esferídeos}
\begin{itemize}
\item {Grp. gram.:m. pl.}
\end{itemize}
\begin{itemize}
\item {Proveniência:(Do gr. \textunderscore sphaira\textunderscore  + \textunderscore eidos\textunderscore )}
\end{itemize}
Gênero de insectos coleópteros pentâmeros.
\section{Esferista}
\begin{itemize}
\item {Grp. gram.:m.}
\end{itemize}
\begin{itemize}
\item {Utilização:Ant.}
\end{itemize}
\begin{itemize}
\item {Proveniência:(Lat. \textunderscore sphaerista\textunderscore )}
\end{itemize}
Jogador de péla.
\section{Esferistério}
\begin{itemize}
\item {Grp. gram.:m.}
\end{itemize}
\begin{itemize}
\item {Utilização:Ant.}
\end{itemize}
\begin{itemize}
\item {Proveniência:(Gr. \textunderscore sphairisterion\textunderscore )}
\end{itemize}
Lugar, onde se joga péla.
\section{Esferística}
\begin{itemize}
\item {Grp. gram.:f.}
\end{itemize}
\begin{itemize}
\item {Proveniência:(De \textunderscore esferístico\textunderscore )}
\end{itemize}
Arte de jogar a péla.
\section{Esferístico}
\begin{itemize}
\item {Grp. gram.:adj.}
\end{itemize}
\begin{itemize}
\item {Proveniência:(De \textunderscore esferista\textunderscore )}
\end{itemize}
Relativo ao jogo da péla.
\section{Esferoidal}
\begin{itemize}
\item {Grp. gram.:adj.}
\end{itemize}
\begin{itemize}
\item {Proveniência:(De \textunderscore esferoide\textunderscore )}
\end{itemize}
Relativo a esferoide.
Que tem fórma de esferoide.
Semelhante a uma esfera.
\section{Esferoide}
\begin{itemize}
\item {Grp. gram.:m.}
\end{itemize}
\begin{itemize}
\item {Proveniência:(Do gr. \textunderscore sphaira\textunderscore  + \textunderscore eidos\textunderscore )}
\end{itemize}
Corpo, quási esférico; corpo, semelhante a uma esfera.
\section{Esferoídeo}
\begin{itemize}
\item {Grp. gram.:adj.}
\end{itemize}
O mesmo ou melhor que \textunderscore esferoidal\textunderscore .
\section{Esferométrico}
\begin{itemize}
\item {Grp. gram.:adj.}
\end{itemize}
Relativo a esferómetro.
\section{Esferómetro}
\begin{itemize}
\item {Grp. gram.:m.}
\end{itemize}
\begin{itemize}
\item {Proveniência:(Do gr. \textunderscore sphaira\textunderscore  + \textunderscore metron\textunderscore )}
\end{itemize}
Instrumento, com que se mede a curva das superfícies esféricas.
\section{Esferossiderite}
\begin{itemize}
\item {Grp. gram.:f.}
\end{itemize}
\begin{itemize}
\item {Proveniência:(Do gr. \textunderscore sphaira\textunderscore  + \textunderscore sideros\textunderscore )}
\end{itemize}
Um dos mineraes, que têm origem em massas lodosas.
\section{Esferostilbite}
\begin{itemize}
\item {Grp. gram.:f.}
\end{itemize}
\begin{itemize}
\item {Proveniência:(De \textunderscore esphera\textunderscore  + \textunderscore estilbite\textunderscore )}
\end{itemize}
Mineral, de textura fibrosa e radiada.
\section{Esférula}
\begin{itemize}
\item {Grp. gram.:f.}
\end{itemize}
\begin{itemize}
\item {Utilização:Bot.}
\end{itemize}
\begin{itemize}
\item {Proveniência:(Lat. \textunderscore sphaerula\textunderscore )}
\end{itemize}
Pequena esfera.
Gota.
Corpúsculo oblongo, na roseta de certos musgos.
\section{Esfigmografia}
\begin{itemize}
\item {Grp. gram.:f.}
\end{itemize}
\begin{itemize}
\item {Utilização:Med.}
\end{itemize}
\begin{itemize}
\item {Proveniência:(De \textunderscore esfigmografo\textunderscore )}
\end{itemize}
Estudo das pulsações arteriaes.
\section{Esfigmógrafo}
\begin{itemize}
\item {Grp. gram.:m.}
\end{itemize}
\begin{itemize}
\item {Proveniência:(Do gr. \textunderscore sphugmos\textunderscore  + \textunderscore graphein\textunderscore )}
\end{itemize}
Instrumento, que traça graphicamente as pulsações das artérias.
\section{Esfigmologia}
\begin{itemize}
\item {Grp. gram.:f.}
\end{itemize}
\begin{itemize}
\item {Proveniência:(Do gr. \textunderscore sphugmos\textunderscore  + \textunderscore logos\textunderscore )}
\end{itemize}
O mesmo que \textunderscore esfigmografia\textunderscore .
\section{Esfigmológico}
\begin{itemize}
\item {Grp. gram.:adj.}
\end{itemize}
Relativo a esfigmologia. Cf. Latino, \textunderscore Or. da Corôa\textunderscore , CLXXVI.
\section{Esfigmómetro}
\begin{itemize}
\item {Grp. gram.:m.}
\end{itemize}
\begin{itemize}
\item {Proveniência:(Do gr. \textunderscore sphugmos\textunderscore  + \textunderscore metron\textunderscore )}
\end{itemize}
Instrumento, com que se mede a velocidade ou regularidade das pulsações.
\section{Esfíncter}
\begin{itemize}
\item {Grp. gram.:m.}
\end{itemize}
\begin{itemize}
\item {Utilização:Anat.}
\end{itemize}
\begin{itemize}
\item {Proveniência:(Gr. \textunderscore sphinkter\textunderscore )}
\end{itemize}
Cada um dos músculos circulares, que, sob a acção da vontade, abrem ou fecham outras tantas cavidades do corpo humano.
\section{Esfinge}
\begin{itemize}
\item {Grp. gram.:f.}
\end{itemize}
\begin{itemize}
\item {Utilização:Fig.}
\end{itemize}
\begin{itemize}
\item {Grp. gram.:Pl.}
\end{itemize}
\begin{itemize}
\item {Proveniência:(Gr. \textunderscore sphinx\textunderscore )}
\end{itemize}
Monstro fabuloso, com corpo de cão ou de leão, e cabeça humana, que propunha enigmas ao viandante e devorava os não adivinhasse.
Estátua dêsse monstro.
Mistério, enigma.
Insectos fusicórneos, da ordem dos lepidópteros.
\section{Esfingídeos}
\begin{itemize}
\item {Grp. gram.:m. pl.}
\end{itemize}
\begin{itemize}
\item {Proveniência:(Do gr. \textunderscore sphinx\textunderscore  + \textunderscore eidos\textunderscore )}
\end{itemize}
Insectos parasitas da planta do tabaco.
\section{Esfingíneos}
\begin{itemize}
\item {Grp. gram.:m. pl.}
\end{itemize}
O mesmo que \textunderscore esfingídeos\textunderscore .
\section{Esfondílio}
\begin{itemize}
\item {Grp. gram.:m.}
\end{itemize}
\begin{itemize}
\item {Proveniência:(Do gr. \textunderscore sphondulos\textunderscore )}
\end{itemize}
O mesmo que \textunderscore canabrás\textunderscore .
\section{Espêto}
\begin{itemize}
\item {Grp. gram.:m.}
\end{itemize}
\begin{itemize}
\item {Utilização:Fig.}
\end{itemize}
\begin{itemize}
\item {Utilização:Fam.}
\end{itemize}
\begin{itemize}
\item {Proveniência:(Do al. \textunderscore spitz\textunderscore )}
\end{itemize}
Utensilio, geralmente de ferro, que serve especialmente para assar carne ou peixe.
Pau aguçado numa das extremidades.
Pessôa esguía, muito magra.
O mesmo que \textunderscore espetanço\textunderscore .
\section{Espevitadamente}
\begin{itemize}
\item {Grp. gram.:adv.}
\end{itemize}
\begin{itemize}
\item {Proveniência:(De \textunderscore espevitar\textunderscore )}
\end{itemize}
Com affectação.
\section{Espevitadeira}
\begin{itemize}
\item {Grp. gram.:f.}
\end{itemize}
Tesoira, para espevitar.
\section{Espevitador}
\begin{itemize}
\item {Grp. gram.:m.}
\end{itemize}
Aquelle ou aquillo que espevita.
\section{Espevitar}
\begin{itemize}
\item {Grp. gram.:v. t.}
\end{itemize}
\begin{itemize}
\item {Utilização:Fam.}
\end{itemize}
\begin{itemize}
\item {Utilização:Fig.}
\end{itemize}
\begin{itemize}
\item {Grp. gram.:V. p.}
\end{itemize}
\begin{itemize}
\item {Utilização:Fam.}
\end{itemize}
Aparar o morrão de (candeeiro, vela, etc.).
Repuxar a torcida de (candeia ou candeeiro de bicos), para avivar a luz.
Tornar affectado, pretensioso.
Despertar, estimular.
Mostrar-se affectado, pretensioso no falar.
(Por \textunderscore espevidar\textunderscore , de \textunderscore pevide\textunderscore ?)
\section{Espezinhar}
\begin{itemize}
\item {Grp. gram.:v. t.}
\end{itemize}
\begin{itemize}
\item {Utilização:Fig.}
\end{itemize}
\begin{itemize}
\item {Proveniência:(De \textunderscore pèzinho\textunderscore )}
\end{itemize}
Pisar, calcar com os pés.
Humilhar, vexar.
Desprezar.
\section{Esphacelamento}
\begin{itemize}
\item {Grp. gram.:m.}
\end{itemize}
O mesmo que \textunderscore esphacêlo\textunderscore . Cf. Camillo, \textunderscore Perfil do Marquês\textunderscore , 211.
\section{Esphacelar}
\begin{itemize}
\item {Grp. gram.:v. t.}
\end{itemize}
\begin{itemize}
\item {Utilização:Fig.}
\end{itemize}
Causar esphacêlo a.
Gangrenar.
Destruir.
Estragar.
Desfazer.
\section{Esphacêlo}
\begin{itemize}
\item {Grp. gram.:m.}
\end{itemize}
\begin{itemize}
\item {Utilização:Fig.}
\end{itemize}
\begin{itemize}
\item {Proveniência:(Gr. \textunderscore sphakelos\textunderscore )}
\end{itemize}
Gangrena de um membro ou de um órgão.
Estrago, destruição.
\section{Esphallerite}
\begin{itemize}
\item {Grp. gram.:f.}
\end{itemize}
Espécie de galenite.
\section{Esphena}
\begin{itemize}
\item {Grp. gram.:f.}
\end{itemize}
O mesmo que \textunderscore espheno\textunderscore .
\section{Espheno}
\begin{itemize}
\item {Grp. gram.:m.}
\end{itemize}
\begin{itemize}
\item {Proveniência:(Do gr. \textunderscore sphen\textunderscore )}
\end{itemize}
Espécie de mineral chrystallino, translúcido.
\section{Esphenocéphalo}
\begin{itemize}
\item {Grp. gram.:adj.}
\end{itemize}
\begin{itemize}
\item {Proveniência:(Do gr. \textunderscore sphen\textunderscore , cunha, e \textunderscore kephale\textunderscore , cabeça)}
\end{itemize}
Que tem a cabeça ponteaguda.
\section{Esphenoédro}
\begin{itemize}
\item {Grp. gram.:adj.}
\end{itemize}
\begin{itemize}
\item {Utilização:Geom.}
\end{itemize}
\begin{itemize}
\item {Proveniência:(Do gr. \textunderscore sphen\textunderscore  + \textunderscore edra\textunderscore )}
\end{itemize}
Polyedro, com algum ou alguns ângulos agudos.
\section{Esphenoidal}
\begin{itemize}
\item {Grp. gram.:adj.}
\end{itemize}
Relativo a esphenoide.
\section{Esphenoide}
\begin{itemize}
\item {Grp. gram.:m.}
\end{itemize}
\begin{itemize}
\item {Utilização:Anat.}
\end{itemize}
\begin{itemize}
\item {Proveniência:(Gr. \textunderscore sphenoeides\textunderscore )}
\end{itemize}
Osso ímpar, situado no meio dos ossos da base do crânio, e que concorre para a formação das órbitas, das cavidades nasaes, etc.
\section{Esphera}
\begin{itemize}
\item {Grp. gram.:f.}
\end{itemize}
\begin{itemize}
\item {Proveniência:(Lat. \textunderscore sphaera\textunderscore )}
\end{itemize}
Corpo, limitado em todas as direcções por uma superficie curva, cujos pontos distam igualmente de um ponto interior.
Qualquer corpo perfeitamente redondo.
Globo terrestre, o mundo.
Conjunto de círculos, com que os astrónomos representam os movimentos e as relações dos corpos do systema planetário.
Zona.
Órbita, percorrida por um astro.
Espaço ou área da actividade de um corpo.
Extensão de atribuições, de poder, de competência: \textunderscore na esphera dos meus recursos\textunderscore .
Classe: \textunderscore homem de baixa esphera\textunderscore .
Antiga moéda portuguesa.
Antiga e pequena peça de artilharia.
\section{Esphéria}
\begin{itemize}
\item {Grp. gram.:f.}
\end{itemize}
\begin{itemize}
\item {Proveniência:(Do gr. \textunderscore sphairion\textunderscore )}
\end{itemize}
Gênero de cogumelos.
Gênero de insectos dípteros.
\section{Espheriáceas}
\begin{itemize}
\item {Grp. gram.:f. pl.}
\end{itemize}
\begin{itemize}
\item {Proveniência:(De \textunderscore espheriáceo\textunderscore )}
\end{itemize}
Ordem de cogumelos.
\section{Espheriáceo}
\begin{itemize}
\item {Grp. gram.:adj.}
\end{itemize}
Relativo ou semelhante a esphéria.
\section{Esphericamente}
\begin{itemize}
\item {fónica:fé}
\end{itemize}
\begin{itemize}
\item {Grp. gram.:adv.}
\end{itemize}
De modo esphérico.
Á maneira ou semelhança de esphera.
\section{Esphericidade}
\begin{itemize}
\item {fónica:fé}
\end{itemize}
\begin{itemize}
\item {Grp. gram.:f.}
\end{itemize}
Qualidade daquillo que é esphérico.
\section{Esphérico}
\begin{itemize}
\item {Grp. gram.:adj.}
\end{itemize}
Relativo a esphera.
Que tem fórma de esphera.
\section{Espherídeos}
\begin{itemize}
\item {Grp. gram.:m. pl.}
\end{itemize}
\begin{itemize}
\item {Proveniência:(Do gr. \textunderscore sphaira\textunderscore  + \textunderscore eidos\textunderscore )}
\end{itemize}
Gênero de insectos coleópteros pentâmeros.
\section{Espherista}
\begin{itemize}
\item {Grp. gram.:m.}
\end{itemize}
\begin{itemize}
\item {Utilização:Ant.}
\end{itemize}
\begin{itemize}
\item {Proveniência:(Lat. \textunderscore sphaerista\textunderscore )}
\end{itemize}
Jogador de péla.
\section{Espheristério}
\begin{itemize}
\item {Grp. gram.:m.}
\end{itemize}
\begin{itemize}
\item {Utilização:Ant.}
\end{itemize}
\begin{itemize}
\item {Proveniência:(Gr. \textunderscore sphairisterion\textunderscore )}
\end{itemize}
Lugar, onde se joga péla.
\section{Espherística}
\begin{itemize}
\item {Grp. gram.:f.}
\end{itemize}
\begin{itemize}
\item {Proveniência:(De \textunderscore espherístico\textunderscore )}
\end{itemize}
Arte de jogar a péla.
\section{Espherístico}
\begin{itemize}
\item {Grp. gram.:adj.}
\end{itemize}
\begin{itemize}
\item {Proveniência:(De \textunderscore espherista\textunderscore )}
\end{itemize}
Relativo ao jogo da péla.
\section{Espheroidal}
\begin{itemize}
\item {Grp. gram.:adj.}
\end{itemize}
\begin{itemize}
\item {Proveniência:(De \textunderscore espheroide\textunderscore )}
\end{itemize}
Relativo a espheroide.
Que tem fórma de espheroide.
Semelhante a uma esphera.
\section{Espheroide}
\begin{itemize}
\item {Grp. gram.:m.}
\end{itemize}
\begin{itemize}
\item {Proveniência:(Do gr. \textunderscore sphaira\textunderscore  + \textunderscore eidos\textunderscore )}
\end{itemize}
Corpo, quási esphérico; corpo, semelhante a uma esphera.
\section{Espheroídeo}
\begin{itemize}
\item {Grp. gram.:adj.}
\end{itemize}
O mesmo ou melhor que \textunderscore espheroidal\textunderscore .
\section{Espherométrico}
\begin{itemize}
\item {Grp. gram.:adj.}
\end{itemize}
Relativo a espherómetro.
\section{Espherómetro}
\begin{itemize}
\item {Grp. gram.:m.}
\end{itemize}
\begin{itemize}
\item {Proveniência:(Do gr. \textunderscore sphaira\textunderscore  + \textunderscore metron\textunderscore )}
\end{itemize}
Instrumento, com que se mede a curva das superfícies esphéricas.
\section{Espherosiderite}
\begin{itemize}
\item {fónica:si}
\end{itemize}
\begin{itemize}
\item {Grp. gram.:f.}
\end{itemize}
\begin{itemize}
\item {Proveniência:(Do gr. \textunderscore sphaira\textunderscore  + \textunderscore sideros\textunderscore )}
\end{itemize}
Um dos mineraes, que têm origem em massas lodosas.
\section{Espherostilbite}
\begin{itemize}
\item {Grp. gram.:f.}
\end{itemize}
\begin{itemize}
\item {Proveniência:(De \textunderscore esphera\textunderscore  + \textunderscore estilbite\textunderscore )}
\end{itemize}
Mineral, de textura fibrosa e radiada.
\section{Esphérula}
\begin{itemize}
\item {Grp. gram.:f.}
\end{itemize}
\begin{itemize}
\item {Utilização:Bot.}
\end{itemize}
\begin{itemize}
\item {Proveniência:(Lat. \textunderscore sphaerula\textunderscore )}
\end{itemize}
Pequena esphera.
Gota.
Corpúsculo oblongo, na roseta de certos musgos.
\section{Esphíncter}
\begin{itemize}
\item {Grp. gram.:m.}
\end{itemize}
\begin{itemize}
\item {Utilização:Anat.}
\end{itemize}
\begin{itemize}
\item {Proveniência:(Gr. \textunderscore sphinkter\textunderscore )}
\end{itemize}
Cada um dos músculos circulares, que, sob a acção da vontade, abrem ou fecham outras tantas cavidades do corpo humano.
\section{Esphinge}
\begin{itemize}
\item {Grp. gram.:f.}
\end{itemize}
\begin{itemize}
\item {Utilização:Fig.}
\end{itemize}
\begin{itemize}
\item {Grp. gram.:Pl.}
\end{itemize}
\begin{itemize}
\item {Proveniência:(Gr. \textunderscore sphinx\textunderscore )}
\end{itemize}
Monstro fabuloso, com corpo de cão ou de leão, e cabeça humana, que propunha enigmas ao viandante e devorava os não adivinhasse.
Estátua dêsse monstro.
Mystério, enigma.
Insectos fusicórneos, da ordem dos lepidópteros.
\section{Esphingídeos}
\begin{itemize}
\item {Grp. gram.:m. pl.}
\end{itemize}
\begin{itemize}
\item {Proveniência:(Do gr. \textunderscore sphinx\textunderscore  + \textunderscore eidos\textunderscore )}
\end{itemize}
Insectos parasitas da planta do tabaco.
\section{Esphingíneos}
\begin{itemize}
\item {Grp. gram.:m. pl.}
\end{itemize}
O mesmo que \textunderscore esphingídeos\textunderscore .
\section{Esphondýlio}
\begin{itemize}
\item {Grp. gram.:m.}
\end{itemize}
\begin{itemize}
\item {Proveniência:(Do gr. \textunderscore sphondulos\textunderscore )}
\end{itemize}
O mesmo que \textunderscore canabrás\textunderscore .
\section{Esphygmographia}
\begin{itemize}
\item {Grp. gram.:f.}
\end{itemize}
\begin{itemize}
\item {Utilização:Med.}
\end{itemize}
\begin{itemize}
\item {Proveniência:(De \textunderscore esphygmographo\textunderscore )}
\end{itemize}
Estudo das pulsações arteriaes.
\section{Esphygmógrapho}
\begin{itemize}
\item {Grp. gram.:m.}
\end{itemize}
\begin{itemize}
\item {Proveniência:(Do gr. \textunderscore sphugmos\textunderscore  + \textunderscore graphein\textunderscore )}
\end{itemize}
Instrumento, que traça graphicamente as pulsações das artérias.
\section{Esphygmologia}
\begin{itemize}
\item {Grp. gram.:f.}
\end{itemize}
\begin{itemize}
\item {Proveniência:(Do gr. \textunderscore sphugmos\textunderscore  + \textunderscore logos\textunderscore )}
\end{itemize}
O mesmo que \textunderscore esphygmographia\textunderscore .
\section{Esphygmológico}
\begin{itemize}
\item {Grp. gram.:adj.}
\end{itemize}
Relativo a esphygmologia. Cf. Latino, \textunderscore Or. da Corôa\textunderscore , CLXXVI.
\section{Esphygmomanómetro}
\begin{itemize}
\item {Grp. gram.:m.}
\end{itemize}
\begin{itemize}
\item {Proveniência:(Do gr. \textunderscore sphugmos\textunderscore  + \textunderscore manos\textunderscore  + \textunderscore metron\textunderscore )}
\end{itemize}
Instrumento, para medir a pressão sanguínea.
\section{Esphygmómetro}
\begin{itemize}
\item {Grp. gram.:m.}
\end{itemize}
\begin{itemize}
\item {Proveniência:(Do gr. \textunderscore sphugmos\textunderscore  + \textunderscore metron\textunderscore )}
\end{itemize}
Instrumento, com que se mede a velocidade ou regularidade das pulsações.
\section{Espia}
\begin{itemize}
\item {Grp. gram.:m.  e  f.}
\end{itemize}
\begin{itemize}
\item {Proveniência:(Do rad. de \textunderscore espiar\textunderscore ^1)}
\end{itemize}
Pessôa, que observa escondidamente, que espreita.
Espião.
\section{Espia}
\begin{itemize}
\item {Grp. gram.:f.}
\end{itemize}
\begin{itemize}
\item {Proveniência:(Do rad. de \textunderscore espiar\textunderscore ^2)}
\end{itemize}
Corda, com que se amarram embarcações.
Corda, com que se puxa alguma coisa.
\section{Espiadoira}
\begin{itemize}
\item {Grp. gram.:f.}
\end{itemize}
\begin{itemize}
\item {Utilização:Prov.}
\end{itemize}
\begin{itemize}
\item {Utilização:dur.}
\end{itemize}
\begin{itemize}
\item {Proveniência:(De \textunderscore espiar\textunderscore ^1)}
\end{itemize}
Corda, que, prêsa ao meio da vela, nos barcos do Doiro, passa em uma roldana fixa a meio do mastro, para se formar um espaço ou abertura, por onde o homem do leme possa vêr a prôa, a fim de dirigir bem o andamento.
\section{Espiador}
\begin{itemize}
\item {Grp. gram.:adj.}
\end{itemize}
\begin{itemize}
\item {Grp. gram.:M.}
\end{itemize}
\begin{itemize}
\item {Utilização:T. de Aveiro}
\end{itemize}
Que espia.
Aquelle que espia.
Remador, immediato ao metedor.
\section{Espiadoura}
\begin{itemize}
\item {Grp. gram.:f.}
\end{itemize}
\begin{itemize}
\item {Utilização:Prov.}
\end{itemize}
\begin{itemize}
\item {Utilização:dur.}
\end{itemize}
\begin{itemize}
\item {Proveniência:(De \textunderscore espiar\textunderscore ^1)}
\end{itemize}
Corda, que, prêsa ao meio da vela, nos barcos do Douro, passa em uma roldana fixa a meio do mastro, para se formar um espaço ou abertura, por onde o homem do leme possa vêr a prôa, a fim de dirigir bem o andamento.
\section{Espiagem}
\begin{itemize}
\item {Grp. gram.:f.}
\end{itemize}
\begin{itemize}
\item {Proveniência:(De \textunderscore espiar\textunderscore ^1)}
\end{itemize}
O mesmo que \textunderscore espionagem\textunderscore .
\section{Espia-maré}
\begin{itemize}
\item {Grp. gram.:m.}
\end{itemize}
\begin{itemize}
\item {Utilização:Bras}
\end{itemize}
Crustáceo de coiraça calcária.
\section{Espião}
\begin{itemize}
\item {Grp. gram.:m.}
\end{itemize}
\begin{itemize}
\item {Proveniência:(De \textunderscore espia\textunderscore ^1)}
\end{itemize}
Indivíduo, encarregado de observar secretamente actos alheios, a qualidade ou o estado de um exército, etc.
Aquelle que espontaneamente faz essa observação.
\section{Espiar}
\begin{itemize}
\item {Grp. gram.:v. t.}
\end{itemize}
Observar secretamente, para denúncia ou simples informação.
Espreitar; seguir occultamente os passos de.
(Ant. al. \textunderscore spehon\textunderscore )
\section{Espiar}
\begin{itemize}
\item {Grp. gram.:v. t.}
\end{itemize}
\begin{itemize}
\item {Proveniência:(Do lat. \textunderscore ex-panare\textunderscore , seg. Car. Michaëlis)}
\end{itemize}
Acabar de fiar a estriga que cingia (a roca).
Segurar com espias (navio).
\section{Espicaçadela}
\begin{itemize}
\item {Grp. gram.:f.}
\end{itemize}
O mesmo que \textunderscore espicaçamento\textunderscore .
\section{Espicaçamento}
\begin{itemize}
\item {Grp. gram.:m.}
\end{itemize}
Acto de espicaçar. Cf. Arn. Gama, \textunderscore Segr. do Abb.\textunderscore , 215.
\section{Espicaçar}
\begin{itemize}
\item {Grp. gram.:v. t.}
\end{itemize}
\begin{itemize}
\item {Utilização:Fig.}
\end{itemize}
\begin{itemize}
\item {Proveniência:(De \textunderscore pico\textunderscore )}
\end{itemize}
Dar bicadas em.
Picar com instrumento aguçado.
Esburacar.
Affligir, torturar.
\section{Espicanardo}
\begin{itemize}
\item {Grp. gram.:m.}
\end{itemize}
\begin{itemize}
\item {Grp. gram.:m.}
\end{itemize}
\begin{itemize}
\item {Proveniência:(Do lat. \textunderscore spica\textunderscore  + \textunderscore nardus\textunderscore )}
\end{itemize}
Planta gramínea, aromática.
Nardo indiano.
\section{Espicha}
\begin{itemize}
\item {Grp. gram.:f.}
\end{itemize}
\begin{itemize}
\item {Utilização:Pop.}
\end{itemize}
\begin{itemize}
\item {Utilização:Prov.}
\end{itemize}
\begin{itemize}
\item {Utilização:Náut.}
\end{itemize}
\begin{itemize}
\item {Proveniência:(De \textunderscore espichar\textunderscore )}
\end{itemize}
Enfiada de peixes miúdos.
Ponta aguda do croque.
Pequena peça de osso, em fórma de ponta de seta, na extremidade da correia, que liga a estriga á roca, e que segura a mesma correia, cravando-se entre esta e a estriga.
Pequena haste de ferro ou madeira, para abrir ilhós e cochas nos cabos.
\section{Espichadela}
\begin{itemize}
\item {Grp. gram.:f.}
\end{itemize}
Acto de espichar.
\section{Espichão}
\begin{itemize}
\item {Grp. gram.:m.}
\end{itemize}
\begin{itemize}
\item {Utilização:Prov.}
\end{itemize}
\begin{itemize}
\item {Utilização:trasm.}
\end{itemize}
\begin{itemize}
\item {Proveniência:(De \textunderscore espichar\textunderscore )}
\end{itemize}
\textunderscore Ir\textunderscore  ou \textunderscore descer de espichão\textunderscore , ir ou descer em linha recta.
\section{Espichar}
\begin{itemize}
\item {Grp. gram.:v. t.}
\end{itemize}
\begin{itemize}
\item {Utilização:Bras}
\end{itemize}
\begin{itemize}
\item {Grp. gram.:V. i.}
\end{itemize}
\begin{itemize}
\item {Utilização:Pop.}
\end{itemize}
\begin{itemize}
\item {Utilização:Bras}
\end{itemize}
\begin{itemize}
\item {Utilização:fam.}
\end{itemize}
\begin{itemize}
\item {Proveniência:(De \textunderscore espicho\textunderscore )}
\end{itemize}
Enfiar pelas guelras.
Abrir furo em (barril ou pipa), para extrahir líquido.
Espetar, furar.
Esticar (coiros), para os secar.
Vencer, numa discussão.
Morrer.
Sair pelo furo do espicho.
Sair com fôrça (a água).
Fazer fiasco, estender-se.
\section{Espiche}
\begin{itemize}
\item {Grp. gram.:m.}
\end{itemize}
\begin{itemize}
\item {Utilização:Fam.}
\end{itemize}
\begin{itemize}
\item {Proveniência:(Do ingl. \textunderscore speech\textunderscore )}
\end{itemize}
Allocução; discurso.
\section{Espiche}
\begin{itemize}
\item {Grp. gram.:m.}
\end{itemize}
\begin{itemize}
\item {Utilização:Pop.}
\end{itemize}
\begin{itemize}
\item {Utilização:Bras}
\end{itemize}
\begin{itemize}
\item {Utilização:Fam.}
\end{itemize}
O mesmo que \textunderscore espicho\textunderscore .
Fiasco, êrro, estenderete.
\section{Espicho}
\begin{itemize}
\item {Grp. gram.:m.}
\end{itemize}
\begin{itemize}
\item {Utilização:Ant.}
\end{itemize}
\begin{itemize}
\item {Utilização:Pop.}
\end{itemize}
\begin{itemize}
\item {Proveniência:(Lat. \textunderscore spiculum\textunderscore )}
\end{itemize}
Pauzinho agudo, com que se tapa o furo aberto no tampo de barril ou pipa.
Pau, com que se prega o coiro, para o esticar.
Galheta.
Pessôa esguia e magra.
\section{Espicifloro}
\begin{itemize}
\item {Grp. gram.:adj.}
\end{itemize}
\begin{itemize}
\item {Utilização:Bot.}
\end{itemize}
\begin{itemize}
\item {Proveniência:(Do lat. \textunderscore spica\textunderscore  + \textunderscore flos\textunderscore , \textunderscore floris\textunderscore )}
\end{itemize}
Que tem as flores dispostas em espigas.
\section{Espiciforme}
\begin{itemize}
\item {Grp. gram.:adj.}
\end{itemize}
\begin{itemize}
\item {Proveniência:(Do lat. \textunderscore spica\textunderscore  + \textunderscore forma\textunderscore )}
\end{itemize}
Que tem fórma de espiga.
\section{Espicilégio}
\begin{itemize}
\item {Grp. gram.:m.}
\end{itemize}
\begin{itemize}
\item {Proveniência:(Lat. \textunderscore spicilegium\textunderscore )}
\end{itemize}
Collecção methódica de documentos.
Florilégio; anthologia.
\section{Espicoiçar}
\begin{itemize}
\item {Grp. gram.:v. i.}
\end{itemize}
\begin{itemize}
\item {Utilização:Prov.}
\end{itemize}
\begin{itemize}
\item {Utilização:alg.}
\end{itemize}
O mesmo que \textunderscore especular\textunderscore ^2.
\section{Espícula}
\begin{itemize}
\item {Grp. gram.:f.}
\end{itemize}
Pequena espiga. Cf. Fil. Simões, \textunderscore C. da Beiramar\textunderscore , 198.
\section{Espicular}
\begin{itemize}
\item {Grp. gram.:v. t.}
\end{itemize}
\begin{itemize}
\item {Proveniência:(De \textunderscore espículo\textunderscore )}
\end{itemize}
Dar fórma de espiga a.
Aguçar.
\section{Espicúlea}
\begin{itemize}
\item {Grp. gram.:f.}
\end{itemize}
Gênero de orchídeas.
\section{Espículo}
\begin{itemize}
\item {Grp. gram.:m.}
\end{itemize}
\begin{itemize}
\item {Proveniência:(Lat. \textunderscore spiculum\textunderscore )}
\end{itemize}
Ponta.
Aguilhão.
Ferrão.
\section{Espido}
\begin{itemize}
\item {Grp. gram.:adj.}
\end{itemize}
\begin{itemize}
\item {Utilização:Prov.}
\end{itemize}
\begin{itemize}
\item {Utilização:trasm.}
\end{itemize}
Diz-se do pão que, depois de cozido, fica poroso e leve.
\section{Espiga}
\begin{itemize}
\item {Grp. gram.:f.}
\end{itemize}
\begin{itemize}
\item {Utilização:Pop.}
\end{itemize}
\begin{itemize}
\item {Utilização:Gír.}
\end{itemize}
\begin{itemize}
\item {Proveniência:(Lat. \textunderscore spíca\textunderscore )}
\end{itemize}
Parte de plantas cerealíferas, que lhes termina superiormente a haste e contém os grãos.
Estames de várias flores.
Nome de várias plantas.
Qualquer objecto em fórma de espiga.
Parte de uma peça de madeira ou metal, que entra num furo de outra peça.
Cumeeira.
Pellícula levantada, junto á raíz das unhas.
Contratempo.
Prejuizo; maçada.
Lôgro.
Choupa.
\section{Espigado}
\begin{itemize}
\item {Grp. gram.:adj.}
\end{itemize}
\begin{itemize}
\item {Utilização:Fig.}
\end{itemize}
\begin{itemize}
\item {Utilização:Fam.}
\end{itemize}
\begin{itemize}
\item {Proveniência:(De \textunderscore espigar\textunderscore )}
\end{itemize}
Que criou espiga.
Desenvolvido, crescido: \textunderscore um rapaz espigado\textunderscore .
Prejudicado, logrado.
\section{Espigador}
\begin{itemize}
\item {Grp. gram.:adj.}
\end{itemize}
\begin{itemize}
\item {Utilização:Pop.}
\end{itemize}
Que espiga ou logra.
\section{Espigadote}
\begin{itemize}
\item {Grp. gram.:adj.}
\end{itemize}
\begin{itemize}
\item {Utilização:Fam.}
\end{itemize}
Um tanto espigado ou crescido.
\section{Espigame}
\begin{itemize}
\item {Grp. gram.:m.}
\end{itemize}
\begin{itemize}
\item {Proveniência:(De \textunderscore espiga\textunderscore )}
\end{itemize}
Grande porção de espigas; respigo.
\section{Espigão}
\begin{itemize}
\item {Grp. gram.:m.}
\end{itemize}
Espiga grande.
Peça aguçada de ferro ou madeira, para se cravar no chão, em parede, etc.
Cumeeira, aresta.
Construcção oblíqua, que corta e desvia uma corrente.
Qualquer peça ponteaguda.
Ferrão.
Espiga das unhas.
\section{Espigar}
\begin{itemize}
\item {Grp. gram.:v.}
\end{itemize}
\begin{itemize}
\item {Utilização:t. Náut.}
\end{itemize}
\begin{itemize}
\item {Utilização:Pop.}
\end{itemize}
\begin{itemize}
\item {Grp. gram.:V. i.}
\end{itemize}
\begin{itemize}
\item {Utilização:Fig.}
\end{itemize}
\begin{itemize}
\item {Utilização:Bras. do N}
\end{itemize}
\begin{itemize}
\item {Proveniência:(De \textunderscore espiga\textunderscore )}
\end{itemize}
Enfiar (mastaréus) na pêga.
Lograr.
Criar espiga.
Desenvolver-se.
Sêr já crescido, entrar na adolescência. Cf. Camillo, \textunderscore Volcões\textunderscore , 133.
Endireitar a estatura.
\section{Espigar}
\begin{itemize}
\item {Grp. gram.:v. t.}
\end{itemize}
\begin{itemize}
\item {Utilização:Prov.}
\end{itemize}
\begin{itemize}
\item {Utilização:trasm.}
\end{itemize}
Sondar astutamente (alguém).
(Corr. de \textunderscore espiar\textunderscore ?)
\section{Espiga-rodrigo!}
\begin{itemize}
\item {Grp. gram.:loc. interj.}
\end{itemize}
\begin{itemize}
\item {Utilização:Ant.}
\end{itemize}
T'arrenego!
Deus me livre. Cf. Simão Machado, 79, v.^o
\section{Espigas}
\begin{itemize}
\item {Grp. gram.:f. pl.}
\end{itemize}
\begin{itemize}
\item {Utilização:ant.}
\end{itemize}
\begin{itemize}
\item {Utilização:Gír.}
\end{itemize}
\begin{itemize}
\item {Utilização:Prov.}
\end{itemize}
\begin{itemize}
\item {Utilização:trasm.}
\end{itemize}
Bigode.
Primeiros grelos de couve.
(Cp. \textunderscore espiga\textunderscore )
\section{Espigélia}
\begin{itemize}
\item {Grp. gram.:f.}
\end{itemize}
\begin{itemize}
\item {Proveniência:(De \textunderscore Spieghel\textunderscore , n. p.)}
\end{itemize}
Planta medicinal, que serve de typo ás espigeliáceas.
\section{Espigeliáceas}
\begin{itemize}
\item {Grp. gram.:f. pl.}
\end{itemize}
\begin{itemize}
\item {Proveniência:(De \textunderscore espigélia\textunderscore )}
\end{itemize}
Família de plantas herbáceas.
\section{Espigo}
\begin{itemize}
\item {Grp. gram.:m.}
\end{itemize}
\begin{itemize}
\item {Proveniência:(De \textunderscore espiga\textunderscore , ou do lat. \textunderscore spiculum\textunderscore )}
\end{itemize}
Ponta de ferro ou madeira.
\section{Espigos}
\begin{itemize}
\item {Grp. gram.:m. pl.}
\end{itemize}
\begin{itemize}
\item {Utilização:Prov.}
\end{itemize}
\begin{itemize}
\item {Utilização:trasm.}
\end{itemize}
\begin{itemize}
\item {Proveniência:(De \textunderscore espigar\textunderscore )}
\end{itemize}
Grelos de hortaliça.
Segundos grelos da couve, de ordinário mais rijos e floridos que as espigas.
\section{Espigoso}
\begin{itemize}
\item {Grp. gram.:adj.}
\end{itemize}
Que tem espigas.
Espiciforme.
\section{Espigueiro}
\begin{itemize}
\item {Grp. gram.:m.}
\end{itemize}
\begin{itemize}
\item {Utilização:Fig.}
\end{itemize}
\begin{itemize}
\item {Proveniência:(De \textunderscore espiga\textunderscore )}
\end{itemize}
Casa ou lugar, em que se abrigam e guardam as espigas do milho.
Viveiro.
\section{Espigueta}
\begin{itemize}
\item {fónica:guê}
\end{itemize}
\begin{itemize}
\item {Grp. gram.:f.}
\end{itemize}
\begin{itemize}
\item {Proveniência:(De \textunderscore espiga\textunderscore )}
\end{itemize}
Cada uma das espigas parciaes, que constituem uma espiga composta.
\section{Espigueto}
\begin{itemize}
\item {fónica:guê}
\end{itemize}
\begin{itemize}
\item {Grp. gram.:m.}
\end{itemize}
\begin{itemize}
\item {Utilização:Des.}
\end{itemize}
\begin{itemize}
\item {Proveniência:(De \textunderscore espiga\textunderscore )}
\end{itemize}
Som agudo, em música.
\section{Espiguilha}
\begin{itemize}
\item {Grp. gram.:f.}
\end{itemize}
\begin{itemize}
\item {Proveniência:(De \textunderscore espiga\textunderscore )}
\end{itemize}
Espécie de renda estreita e denteada.
\section{Espiguilhar}
\begin{itemize}
\item {Grp. gram.:v. t.}
\end{itemize}
Guarnecer ou ornar com espiguilha.
\section{Espildrar}
\begin{itemize}
\item {Grp. gram.:v. i.}
\end{itemize}
\begin{itemize}
\item {Utilização:Prov.}
\end{itemize}
\begin{itemize}
\item {Utilização:trasm.}
\end{itemize}
Acabar.
Esgotar-se.
(Corr. de \textunderscore expirar\textunderscore ?)
\section{Espilrar}
\textunderscore v. i.\textunderscore  (e der.) \textunderscore Pop.\textunderscore 
O mesmo que \textunderscore espirrar\textunderscore .
(Por \textunderscore espirlar\textunderscore , do lat. \textunderscore expirulare\textunderscore )
\section{Espim}
\begin{itemize}
\item {Grp. gram.:adj.}
\end{itemize}
\begin{itemize}
\item {Proveniência:(Lat. \textunderscore spineus\textunderscore )}
\end{itemize}
O mesmo que \textunderscore espinhoso\textunderscore ^1.
Diz-se de uma variedade de uva.
\section{Espina}
\begin{itemize}
\item {Grp. gram.:f.}
\end{itemize}
\begin{itemize}
\item {Utilização:Ant.}
\end{itemize}
\begin{itemize}
\item {Proveniência:(Do lat. \textunderscore spina\textunderscore )}
\end{itemize}
Planta medicinal.
O mesmo que \textunderscore espinha\textunderscore  e o mesmo que \textunderscore espinho\textunderscore . Cf. Frei Fortun., \textunderscore Inéd.\textunderscore , I, 307.
\section{Espinafrado}
\begin{itemize}
\item {Grp. gram.:adj.}
\end{itemize}
\begin{itemize}
\item {Utilização:Pop.}
\end{itemize}
O mesmo que \textunderscore escanifrado\textunderscore .
\section{Espinafre}
\begin{itemize}
\item {Grp. gram.:m.}
\end{itemize}
\begin{itemize}
\item {Utilização:Pop.}
\end{itemize}
Planta hortense, annual, (\textunderscore spinacia oleracea\textunderscore ).
Pessôa magra.
\section{Espinal}
\begin{itemize}
\item {Grp. gram.:adj.}
\end{itemize}
\begin{itemize}
\item {Proveniência:(Lat. \textunderscore spinalis\textunderscore )}
\end{itemize}
Relativo á espinha.
Semelhante á espinha; espinhal: \textunderscore medulla espinal\textunderscore .
\section{Espinça}
\begin{itemize}
\item {Grp. gram.:f.}
\end{itemize}
Instrumento de espinçar.
Acto de espinçar.
\section{Espinçadeira}
\begin{itemize}
\item {Grp. gram.:f.}
\end{itemize}
\begin{itemize}
\item {Proveniência:(De \textunderscore inspinçar\textunderscore )}
\end{itemize}
O mesmo que \textunderscore espinça\textunderscore .
\section{Espinçar}
\begin{itemize}
\item {Grp. gram.:v. t.}
\end{itemize}
\begin{itemize}
\item {Proveniência:(De \textunderscore pinça\textunderscore )}
\end{itemize}
Limpar (a teia), cortando-lhe com espinça fios, nós, etc.
\section{Espinel}
\begin{itemize}
\item {Grp. gram.:m.}
\end{itemize}
O mesmo que \textunderscore espinela\textunderscore .
\section{Espinel}
\begin{itemize}
\item {Grp. gram.:m.}
\end{itemize}
\begin{itemize}
\item {Utilização:Pesc.}
\end{itemize}
Apparelho de pesca, que consta de uma linha comprida, e tem, de espaço a espaço, prêsa outra mais curta com anzol.
(Relaciona-se com o lat. \textunderscore spina\textunderscore )
\section{Espinela}
\begin{itemize}
\item {Grp. gram.:f.}
\end{itemize}
\begin{itemize}
\item {Proveniência:(Do lat. \textunderscore spina\textunderscore )}
\end{itemize}
Mineral, formado de alumina anhydra e de uma base de zinco, magnésia ou ferro.
\section{Espineleiro}
\begin{itemize}
\item {Grp. gram.:m.}
\end{itemize}
\begin{itemize}
\item {Utilização:Des.}
\end{itemize}
\begin{itemize}
\item {Proveniência:(De \textunderscore espinel\textunderscore ^2)}
\end{itemize}
Fabricante de espinéís.
\section{Espíneo}
\begin{itemize}
\item {Grp. gram.:adj.}
\end{itemize}
\begin{itemize}
\item {Proveniência:(Lat. \textunderscore spineus\textunderscore )}
\end{itemize}
Que tem espinhos, ou que é feito de espinhos.
\section{Espinescente}
\begin{itemize}
\item {Grp. gram.:adj.}
\end{itemize}
\begin{itemize}
\item {Utilização:Bot.}
\end{itemize}
\begin{itemize}
\item {Proveniência:(Lat. \textunderscore spinescens\textunderscore )}
\end{itemize}
Que se transforma em espinhos.
Que se cobre de espinhos.
\section{Espinescido}
\begin{itemize}
\item {Grp. gram.:adj.}
\end{itemize}
\begin{itemize}
\item {Utilização:Bot.}
\end{itemize}
\begin{itemize}
\item {Proveniência:(Do lat. \textunderscore spinescere\textunderscore )}
\end{itemize}
Que termina em espinhos.
\section{Espineta}
\begin{itemize}
\item {fónica:nê}
\end{itemize}
\begin{itemize}
\item {Grp. gram.:f.}
\end{itemize}
\begin{itemize}
\item {Proveniência:(It. \textunderscore spinetta\textunderscore )}
\end{itemize}
Antigo instrumento de cordas e teclas.
\section{Espinete}
\begin{itemize}
\item {fónica:nê}
\end{itemize}
\begin{itemize}
\item {Grp. gram.:m.}
\end{itemize}
Tecido antigo e fino de algodão. Cf. \textunderscore Instituto\textunderscore , LI, 447.
\section{Espingalhar}
\begin{itemize}
\item {Grp. gram.:v. i.}
\end{itemize}
\begin{itemize}
\item {Utilização:T. de Arganil}
\end{itemize}
Chuviscar.
(Cp. \textunderscore pingar\textunderscore ^1)
\section{Espingarda}
\begin{itemize}
\item {Grp. gram.:f.}
\end{itemize}
Arma de fogo, portátil e de cano comprido.
(Talvez do al. \textunderscore spinrad\textunderscore , roda de fiar, porque as primeiras espingardas tinham uma roda nos fechos)
\section{Espingardada}
\begin{itemize}
\item {Grp. gram.:f.}
\end{itemize}
Tiro de espingarda.
\section{Espingardão}
\begin{itemize}
\item {Grp. gram.:m.}
\end{itemize}
Espingarda grande.
Arcabuz.
Antiga e pequena peça de artilharia.
\section{Espingardaria}
\begin{itemize}
\item {Grp. gram.:f.}
\end{itemize}
Série de tiros de espingarda.
Grande porção de espingardas.
Trôço de gente, armada de espingardas.
\section{Espingardeamento}
\begin{itemize}
\item {Grp. gram.:m.}
\end{itemize}
Acto de espingardear. Cf. Camillo, \textunderscore Livro Negro\textunderscore , 76.
\section{Espingardear}
\begin{itemize}
\item {Grp. gram.:v. t.}
\end{itemize}
Ferir ou matar com espingarda.
\section{Espingardeira}
\begin{itemize}
\item {Grp. gram.:f.}
\end{itemize}
Cavidade na muralha, em que assenta e donde se dispara a espingarda.
\section{Espingardeiro}
\begin{itemize}
\item {Grp. gram.:m.}
\end{itemize}
\begin{itemize}
\item {Utilização:Des.}
\end{itemize}
Aquelle que vende, fabríca ou concerta espingardas.
Soldado, armado de espingarda.
\section{Espingulado}
\begin{itemize}
\item {Grp. gram.:m.}
\end{itemize}
\begin{itemize}
\item {Utilização:Pop.}
\end{itemize}
Homem alto, magrizela.
(Por \textunderscore espinhulado\textunderscore , de \textunderscore espinha\textunderscore ?)
\section{Espinha}
\begin{itemize}
\item {Grp. gram.:f.}
\end{itemize}
\begin{itemize}
\item {Utilização:Fig.}
\end{itemize}
\begin{itemize}
\item {Utilização:Prov.}
\end{itemize}
\begin{itemize}
\item {Utilização:minh.}
\end{itemize}
\begin{itemize}
\item {Utilização:Ant.}
\end{itemize}
\begin{itemize}
\item {Utilização:Gír.}
\end{itemize}
\begin{itemize}
\item {Utilização:Constr.}
\end{itemize}
\begin{itemize}
\item {Proveniência:(Lat. \textunderscore spina\textunderscore , pl. de \textunderscore spinum\textunderscore )}
\end{itemize}
Qualquer saliência óssea e alongada, no corpo humano.
Columna vertebral.
Osso de peixe.
Nome de certas borbulhas do rosto.
Instrumento, com que os fundidores dão passagem ao metal fundido.
Peça de ferro, em cujo cabo de madeira uns anilhos se ligam á cadeia de suspensão, em artilharia.
Designação de várias plantas.
Difficuldade: \textunderscore êsse negócio tem espinha\textunderscore .
Pessôa muito magra.
Parte interna da chila, a que se ligam as sementes.
Pequeno muro, que corria ao centro dos circos romanos.
Punhal.
\textunderscore Soalho de espinha\textunderscore , solho formado de tábuas, que cruzam obliquamente os barrotes.
\section{Espinhaço}
\begin{itemize}
\item {Grp. gram.:m.}
\end{itemize}
\begin{itemize}
\item {Proveniência:(De \textunderscore espinha\textunderscore )}
\end{itemize}
Columna vertebral.
Dorso; costas.
Série de montes.
\section{Espinhal}
\begin{itemize}
\item {Grp. gram.:m.}
\end{itemize}
\begin{itemize}
\item {Proveniência:(Lat. \textunderscore spinalis\textunderscore )}
\end{itemize}
Lugar, onde crescem espinheiros.
\section{Espinhal}
\begin{itemize}
\item {Grp. gram.:adj.}
\end{itemize}
\begin{itemize}
\item {Proveniência:(De \textunderscore espinha\textunderscore )}
\end{itemize}
Relativo á espinha; espinal.
\section{Espinhar}
\begin{itemize}
\item {Grp. gram.:v. t.}
\end{itemize}
\begin{itemize}
\item {Utilização:Fig.}
\end{itemize}
Picar ou ferir com espinho.
Agastar; encolerizar.
\section{Espinheira}
\begin{itemize}
\item {Grp. gram.:f.}
\end{itemize}
O mesmo que \textunderscore espinheiro\textunderscore .
\section{Espinheiral}
\begin{itemize}
\item {Grp. gram.:m.}
\end{itemize}
O mesmo que \textunderscore espinhal\textunderscore ^1.
\section{Espinheiro}
\begin{itemize}
\item {Grp. gram.:m.}
\end{itemize}
\begin{itemize}
\item {Proveniência:(De b. lat. \textunderscore spinarius\textunderscore )}
\end{itemize}
Planta espinhosa e vivaz, que serve de typo ás rhamnáceas.
Nome de várias plantas leguminosas da América.
Sarça.
\section{Espinhel}
\begin{itemize}
\item {Grp. gram.:m.}
\end{itemize}
\begin{itemize}
\item {Utilização:Bras}
\end{itemize}
O mesmo que \textunderscore espinel\textunderscore ^2.
\section{Espinhela}
\begin{itemize}
\item {Grp. gram.:f.}
\end{itemize}
\begin{itemize}
\item {Utilização:Pop.}
\end{itemize}
\begin{itemize}
\item {Proveniência:(De \textunderscore espinha\textunderscore )}
\end{itemize}
Appêndice cartilagíneo, na parte inferior do esterno: \textunderscore os curandeiros tratam a espinhela caída\textunderscore .
\section{Espinhento}
\begin{itemize}
\item {Grp. gram.:adj.}
\end{itemize}
\begin{itemize}
\item {Utilização:P. us.}
\end{itemize}
O mesmo que \textunderscore espinhoso\textunderscore ^1.
\section{Espinheta}
\begin{itemize}
\item {fónica:nhê}
\end{itemize}
\begin{itemize}
\item {Grp. gram.:f.}
\end{itemize}
O mesmo que \textunderscore espineta\textunderscore .
\section{Espinho}
\begin{itemize}
\item {Grp. gram.:m.}
\end{itemize}
\begin{itemize}
\item {Utilização:Bras. de Minas}
\end{itemize}
\begin{itemize}
\item {Utilização:Gír.}
\end{itemize}
\begin{itemize}
\item {Utilização:Fig.}
\end{itemize}
\begin{itemize}
\item {Proveniência:(Do lat. \textunderscore spinum\textunderscore )}
\end{itemize}
Saliência delgada e aguda, que nasce do lenho e faz parte delle.
Qualquer pico de um vegetal.
Planta espinhosa.
Acúleo.
Cerda rija de alguns animaes.
O mesmo que \textunderscore faca\textunderscore ^1.
Difficuldade.
\section{Espinho-de-carneiro}
\begin{itemize}
\item {Grp. gram.:m.}
\end{itemize}
\begin{itemize}
\item {Utilização:Bras}
\end{itemize}
Planta medicinal, (\textunderscore xanthium spinosum\textunderscore ).
\section{Espinhoso}
\begin{itemize}
\item {Grp. gram.:adj.}
\end{itemize}
\begin{itemize}
\item {Utilização:Fig.}
\end{itemize}
\begin{itemize}
\item {Proveniência:(Lat. \textunderscore spinosus\textunderscore )}
\end{itemize}
Que tem espinhos.
Semelhante a espinho.
Diffícil; tormentoso: \textunderscore tarefa espinhosa\textunderscore .
\section{Espinhoso}
\begin{itemize}
\item {Grp. gram.:adj.}
\end{itemize}
\begin{itemize}
\item {Utilização:Anat.}
\end{itemize}
\begin{itemize}
\item {Proveniência:(De \textunderscore espinha\textunderscore )}
\end{itemize}
Relativo á espinha ou apóphyse, que divide em duas regiões sobrepostas a face dorsal da omoplata.
\section{Espinicado}
\begin{itemize}
\item {Grp. gram.:adj.}
\end{itemize}
\begin{itemize}
\item {Utilização:Des.}
\end{itemize}
Affectado no vestir.
Pretensiosamente garrido:«\textunderscore ...basbaques... mais espinicados que um pintasilgo mimoso.\textunderscore »Seropita, \textunderscore Prosas\textunderscore , 82.
\section{Espiniforme}
\begin{itemize}
\item {Grp. gram.:adj.}
\end{itemize}
Que tem fórma de espinho.
\section{Espinilho}
\begin{itemize}
\item {Grp. gram.:m.}
\end{itemize}
\begin{itemize}
\item {Proveniência:(Do lat. \textunderscore spina\textunderscore )}
\end{itemize}
Arbusto brasileiro.
\section{Espinol}
\begin{itemize}
\item {Grp. gram.:m.}
\end{itemize}
\begin{itemize}
\item {Utilização:Pharm.}
\end{itemize}
Medicamento tónico, extrahido das folhas dos espinafres.
\section{Espinotar}
\begin{itemize}
\item {Grp. gram.:v. i.}
\end{itemize}
O mesmo que \textunderscore espinotear\textunderscore . Cf. Camillo, \textunderscore Corja\textunderscore , 202.
\section{Espinote}
\begin{itemize}
\item {Grp. gram.:m.}
\end{itemize}
Espécie de tecido, usado em tempo de D. João III.
\section{Espinotear}
\begin{itemize}
\item {Grp. gram.:v. i.}
\end{itemize}
\begin{itemize}
\item {Utilização:Fig.}
\end{itemize}
Dar pinotes.
Encolerizar-se; barafustar.
\section{Espíntria}
\begin{itemize}
\item {Grp. gram.:m.  e  f.}
\end{itemize}
\begin{itemize}
\item {Proveniência:(Lat. \textunderscore spintria\textunderscore )}
\end{itemize}
Pessôa, que inventa novos actos de luxúria, requintes de voluptuosidade. Cf. Tácito, \textunderscore Annales\textunderscore , VI, 1; Suetonio, \textunderscore Tib.\textunderscore  43, \textunderscore Calig.\textunderscore  16, \textunderscore Vitel.\textunderscore  3.
\section{Espínula}
\begin{itemize}
\item {Grp. gram.:f.}
\end{itemize}
\begin{itemize}
\item {Utilização:Des.}
\end{itemize}
\begin{itemize}
\item {Proveniência:(Lat. \textunderscore spinula\textunderscore )}
\end{itemize}
Alfinete para as vestimentas episcopaes.
\section{Espiolhar}
\begin{itemize}
\item {Grp. gram.:v. t.}
\end{itemize}
\begin{itemize}
\item {Utilização:Fig.}
\end{itemize}
Tirar piolhos de.
Investigar.
Examinar minuciosamente.
\section{Espionagem}
\begin{itemize}
\item {Grp. gram.:f.}
\end{itemize}
Acto de espionar.
Encargo de espião.
Espiões.
\section{Espionar}
\begin{itemize}
\item {Grp. gram.:v. t.}
\end{itemize}
\begin{itemize}
\item {Grp. gram.:V. i.}
\end{itemize}
\begin{itemize}
\item {Proveniência:(Do ant. al. \textunderscore spehon\textunderscore )}
\end{itemize}
Espreitar ou investigar, como espião.
Espiar.
Praticar actos de espião.
\section{Espipado}
\begin{itemize}
\item {Grp. gram.:adj.}
\end{itemize}
\begin{itemize}
\item {Proveniência:(De \textunderscore espipar\textunderscore )}
\end{itemize}
Esticado, estendido.
Saliente, esbugalhado:«\textunderscore ...com os olhos espipados...\textunderscore »Camillo, \textunderscore Volcões\textunderscore , 170.
\section{Espipar}
\begin{itemize}
\item {Grp. gram.:v. i.}
\end{itemize}
\begin{itemize}
\item {Utilização:Pop.}
\end{itemize}
\begin{itemize}
\item {Grp. gram.:V. t.}
\end{itemize}
\begin{itemize}
\item {Utilização:Prov.}
\end{itemize}
\begin{itemize}
\item {Utilização:Prov.}
\end{itemize}
\begin{itemize}
\item {Utilização:trasm.}
\end{itemize}
\begin{itemize}
\item {Proveniência:(De \textunderscore pipo\textunderscore )}
\end{itemize}
Jorrar.
Repuxar.
Rebentar, estalar.
Tirar de dentro, extrahir.
Puxar (o fiado) da roca.
\section{Espipocar}
\begin{itemize}
\item {Grp. gram.:v. t.  e  i.}
\end{itemize}
O mesmo que \textunderscore pipocar\textunderscore .
\section{Espique}
\begin{itemize}
\item {Grp. gram.:m.}
\end{itemize}
Caule lenhoso de certas plantas.
\section{Espiqueado}
\begin{itemize}
\item {Grp. gram.:adj.}
\end{itemize}
Que tem espique ou tem caule semelhante a espique.
\section{Espir}
\begin{itemize}
\item {Grp. gram.:v. t.}
\end{itemize}
\begin{itemize}
\item {Utilização:Ant.}
\end{itemize}
O mesmo que \textunderscore despir\textunderscore . Cf. Frei Fortun., \textunderscore Inéd.\textunderscore , I, 307.
\section{Espira}
\begin{itemize}
\item {Grp. gram.:f.}
\end{itemize}
\begin{itemize}
\item {Proveniência:(Lat. \textunderscore spira\textunderscore )}
\end{itemize}
Cada uma das voltas da espiral.
Circunvolução helicoide, descrita por uma parte de um vegetal.
Voltas, que algumas conchas univalves apresentam.
Rosca de parafuso.
\section{Espiração}
\begin{itemize}
\item {Grp. gram.:f.}
\end{itemize}
Acto de espirar.
Alento. Cf. Camillo, \textunderscore Volcões\textunderscore , 148.
\section{Espiráculo}
\begin{itemize}
\item {Grp. gram.:m.}
\end{itemize}
\begin{itemize}
\item {Proveniência:(Lat. \textunderscore spiraculum\textunderscore )}
\end{itemize}
Orifício, por onde sái o ar.
Respiradoiro.
Respiração.
\section{Espiral}
\begin{itemize}
\item {Grp. gram.:adj.}
\end{itemize}
\begin{itemize}
\item {Grp. gram.:F.}
\end{itemize}
\begin{itemize}
\item {Proveniência:(De \textunderscore espira\textunderscore )}
\end{itemize}
Que tem fórma de espira ou de caracol.
Curva, que, descrevendo voltas, póde sêr atravessada por uma recta em pontos indefinidos.
Mola de aço, no centro de um volante de relógio.
\section{Espiralado}
\begin{itemize}
\item {Grp. gram.:adj.}
\end{itemize}
\begin{itemize}
\item {Utilização:Bot.}
\end{itemize}
Que tem fórma de espiral.
\section{Espiralar}
\begin{itemize}
\item {Grp. gram.:v. t.}
\end{itemize}
\begin{itemize}
\item {Grp. gram.:V. p.}
\end{itemize}
Dar fórma de espiral a.
Subir em espiral, tomar a fórma de espiral:«\textunderscore ...columnas de fumo, que se espiralavam até ao tecto\textunderscore ». Camillo, \textunderscore Brasileira\textunderscore , 109.
\section{Espirante}
\begin{itemize}
\item {Grp. gram.:adj.}
\end{itemize}
\begin{itemize}
\item {Utilização:Gram.}
\end{itemize}
Que espira.
Que está ou parece vivo.
Diz-se da letra invogal, que se pronuncia com o auxilio do ar espirado.
\section{Espirar}
\begin{itemize}
\item {Grp. gram.:v. t.}
\end{itemize}
\begin{itemize}
\item {Grp. gram.:V. i.}
\end{itemize}
\begin{itemize}
\item {Proveniência:(Lat. \textunderscore spirare\textunderscore )}
\end{itemize}
Respirar.
Exhalar.
Estar ou parecer vivo.
\section{Espirar-se}
\begin{itemize}
\item {Grp. gram.:v. p.}
\end{itemize}
\begin{itemize}
\item {Utilização:Prov.}
\end{itemize}
\begin{itemize}
\item {Utilização:trasm.}
\end{itemize}
O mesmo que \textunderscore fugir\textunderscore .
\section{Espírea}
\begin{itemize}
\item {Grp. gram.:f.}
\end{itemize}
Gênero de plantas rosáceas, de ramos flexíveis e flôres miúdas e brancas, (\textunderscore spirea\textunderscore ).
\section{Espíreas}
\begin{itemize}
\item {Grp. gram.:f. pl.}
\end{itemize}
Tribo de plantas rosáceas, que tem por typo a espírea.
\section{Espirícula}
\begin{itemize}
\item {Grp. gram.:f.}
\end{itemize}
\begin{itemize}
\item {Utilização:Bot.}
\end{itemize}
\begin{itemize}
\item {Proveniência:(De \textunderscore espira\textunderscore )}
\end{itemize}
Filete espiral, nas tracheias dos vegetaes.
\section{Espiridantho}
\begin{itemize}
\item {Grp. gram.:m.}
\end{itemize}
Gênero de plantas synanthéreas.
\section{Espiridanto}
\begin{itemize}
\item {Grp. gram.:m.}
\end{itemize}
Gênero de plantas sinantéreas.
\section{Espirídia}
\begin{itemize}
\item {Grp. gram.:f.}
\end{itemize}
Gênero de algas.
\section{Espirífero}
\begin{itemize}
\item {Grp. gram.:adj.}
\end{itemize}
\begin{itemize}
\item {Utilização:Zool.}
\end{itemize}
\begin{itemize}
\item {Proveniência:(Do lat. \textunderscore spira\textunderscore  + \textunderscore ferre\textunderscore )}
\end{itemize}
Que tem espira.
\section{Espiriforme}
\begin{itemize}
\item {Grp. gram.:adj.}
\end{itemize}
Que tem fórma de espira.
\section{Espirillo}
\begin{itemize}
\item {Grp. gram.:m.}
\end{itemize}
\begin{itemize}
\item {Proveniência:(Lat. \textunderscore spirillum\textunderscore )}
\end{itemize}
Bactéria, composta de uma única e longa céllula.
\section{Espirilo}
\begin{itemize}
\item {Grp. gram.:m.}
\end{itemize}
\begin{itemize}
\item {Proveniência:(Lat. \textunderscore spirillum\textunderscore )}
\end{itemize}
Bactéria, composta de uma única e longa célula.
\section{Espirita}
\begin{itemize}
\item {Grp. gram.:m. ,  f.  e  adj.}
\end{itemize}
(V.espiritista)
\section{Espiritar}
\begin{itemize}
\item {Grp. gram.:v. t.}
\end{itemize}
\begin{itemize}
\item {Utilização:Fig.}
\end{itemize}
\begin{itemize}
\item {Proveniência:(De \textunderscore espírito\textunderscore )}
\end{itemize}
Endemoninhar.
Tornar endiabrado, travesso, inquieto.
\section{Espiriteira}
\begin{itemize}
\item {Grp. gram.:f.}
\end{itemize}
\begin{itemize}
\item {Utilização:Bras}
\end{itemize}
Vaso, para lançar espírito ou alcool.
\section{Espirítico}
\begin{itemize}
\item {Grp. gram.:adj.}
\end{itemize}
\begin{itemize}
\item {Proveniência:(De \textunderscore espirita\textunderscore )}
\end{itemize}
(V.espiritístico)
\section{Espiritístico}
\begin{itemize}
\item {Grp. gram.:adj.}
\end{itemize}
Relativo aos espiritistas ou ao espiritismo.
\section{Espiritismo}
\begin{itemize}
\item {Grp. gram.:m.}
\end{itemize}
\begin{itemize}
\item {Proveniência:(De \textunderscore espírito\textunderscore )}
\end{itemize}
Doutrina dos que suppõem estar ou poder estar em communicação com os espíritos dos mortos.
\section{Espiritista}
\begin{itemize}
\item {Grp. gram.:m.  e  f.}
\end{itemize}
\begin{itemize}
\item {Grp. gram.:Adj.}
\end{itemize}
\begin{itemize}
\item {Proveniência:(De \textunderscore espírito\textunderscore )}
\end{itemize}
Pessoa, que segue a doutrina do espiritismo.
Relativo ao espiritismo.
\section{Espírito}
\begin{itemize}
\item {Grp. gram.:m.}
\end{itemize}
\begin{itemize}
\item {Grp. gram.:Pl.}
\end{itemize}
\begin{itemize}
\item {Utilização:Espir.}
\end{itemize}
\begin{itemize}
\item {Proveniência:(Lat. \textunderscore spiritus\textunderscore )}
\end{itemize}
Substância incorpórea e intelligente.
Alma.
Sêr humano: \textunderscore Camões foi espírito superior\textunderscore .
Entidade sobrenatural, como os anjos e os demónios.
Ente imaginário, como os duendes: \textunderscore tem mêdo dos espíritos\textunderscore .
Pessôa distinta, esclarecida.
Vida.
Ânimo.
Sôpro.
Intelligência.
Finura, subtileza.
Graça, engenho: \textunderscore homem de espírito\textunderscore .
Imaginação.
Tendência: \textunderscore o espírito de revolta\textunderscore .
Essência, ideia predominante.
Opiniões.
Líquido, obtido pela destillação.
Álcool.
Corpúsculos, a que se attribuía a faculdade de levar a vida e o sentimento aos diversos pontos do organismo animal.
Seres intelligentes da criação, que povôam o universo, independentemente da vida material, e que constituem o mundo invisível.
\section{Espírito-santense}
\begin{itemize}
\item {Grp. gram.:adj.}
\end{itemize}
\begin{itemize}
\item {Utilização:Bras}
\end{itemize}
\begin{itemize}
\item {Grp. gram.:M.}
\end{itemize}
\begin{itemize}
\item {Proveniência:(De \textunderscore Espírito-Santo\textunderscore , n. p.)}
\end{itemize}
Relativo ao Estado do Espírito-Santo, no Brasil.
Habitante dêsse Estado.
\section{Espiritoso}
\begin{itemize}
\item {Grp. gram.:adj.}
\end{itemize}
\begin{itemize}
\item {Utilização:Ant.}
\end{itemize}
(V.espirituoso)
\section{Espirituado}
\begin{itemize}
\item {Grp. gram.:adj.}
\end{itemize}
\begin{itemize}
\item {Utilização:Des.}
\end{itemize}
Cheio de espírito, muito vivo.
\section{Espiritual}
\begin{itemize}
\item {Grp. gram.:adj.}
\end{itemize}
\begin{itemize}
\item {Proveniência:(Lat. \textunderscore spiritualis\textunderscore )}
\end{itemize}
Relativo ao espírito.
Que é incorpóreo: \textunderscore amor espiritual\textunderscore .
Mýstico, devoto: \textunderscore vida espiritual\textunderscore .
Allegórico.
Relativo á religião ou ao fôro ecclesiástico.
Relativo ao regime da Igreja.
Relativo á consciência.
\section{Espiritualidade}
\begin{itemize}
\item {Grp. gram.:f.}
\end{itemize}
\begin{itemize}
\item {Proveniência:(Lat. \textunderscore spiritualitas\textunderscore )}
\end{itemize}
Qualidade daquillo que é espiritual.
\section{Espiritualismo}
\begin{itemize}
\item {Grp. gram.:m.}
\end{itemize}
\begin{itemize}
\item {Proveniência:(De \textunderscore espiritual\textunderscore )}
\end{itemize}
Doutrina philosóphica, que tem por base a existência da alma e de Deus.
\section{Espiritualista}
\begin{itemize}
\item {Grp. gram.:m.  e  f.}
\end{itemize}
\begin{itemize}
\item {Grp. gram.:Adj.}
\end{itemize}
\begin{itemize}
\item {Proveniência:(De \textunderscore espiritual\textunderscore )}
\end{itemize}
Pessôa, que segue a doutrina do espiritualismo.
Relativo ao espiritualismo.
\section{Espiritualização}
\begin{itemize}
\item {Grp. gram.:f.}
\end{itemize}
Acto ou effeito de espiritualizar.
\section{Espiritualizar}
\begin{itemize}
\item {Grp. gram.:v. t.}
\end{itemize}
\begin{itemize}
\item {Utilização:Ant.}
\end{itemize}
\begin{itemize}
\item {Utilização:Fam.}
\end{itemize}
\begin{itemize}
\item {Proveniência:(De \textunderscore espiritual\textunderscore )}
\end{itemize}
Interpretar allegoricamente.
Dar feição superior ou espiritual a.
Destillar.
Animar, excitar: \textunderscore o vinho espiritualiza-o\textunderscore .
\section{Espiritualmente}
\begin{itemize}
\item {Grp. gram.:adv.}
\end{itemize}
De modo espiritual.
\section{Espirituosamente}
\begin{itemize}
\item {Grp. gram.:adv.}
\end{itemize}
De modo espirituoso.
\section{Espirituosidade}
\begin{itemize}
\item {Grp. gram.:f.}
\end{itemize}
Qualidade de espirituoso.
Fôrça espirituosa (dos vinhos). Cf. \textunderscore Techn. Rur.\textunderscore , 54 e 55.
\section{Espirituoso}
\begin{itemize}
\item {Grp. gram.:adj.}
\end{itemize}
\begin{itemize}
\item {Proveniência:(De \textunderscore espírito\textunderscore )}
\end{itemize}
Que tem espírito, que tem graça.
Conceituoso.
Alcoólico: \textunderscore bebidas espirituosas\textunderscore .
\section{Espiroide}
\begin{itemize}
\item {Grp. gram.:adj.}
\end{itemize}
\begin{itemize}
\item {Proveniência:(Do gr. \textunderscore speira\textunderscore  + \textunderscore eidos\textunderscore )}
\end{itemize}
Que tem fórma de espiral; contornado em hélice.
\section{Espirometria}
\begin{itemize}
\item {Grp. gram.:f.}
\end{itemize}
Arte de empregar o espirómetro.
\section{Espirómetro}
\begin{itemize}
\item {Grp. gram.:m.}
\end{itemize}
\begin{itemize}
\item {Proveniência:(T. hybr., do lat. \textunderscore spirare\textunderscore  + gr. \textunderscore metron\textunderscore )}
\end{itemize}
Instrumento, para medir a capacidade respiratória do pulmão.
\section{Espirra-canivetes}
\begin{itemize}
\item {Grp. gram.:m.  e  f.}
\end{itemize}
\begin{itemize}
\item {Utilização:Pop.}
\end{itemize}
Pessôa facilmente irritável.
Escalda-favaes.
\section{Espirradeira}
\begin{itemize}
\item {Grp. gram.:f.}
\end{itemize}
\begin{itemize}
\item {Proveniência:(De \textunderscore espirrar\textunderscore )}
\end{itemize}
O mesmo que \textunderscore cevadilha\textunderscore  ou \textunderscore loendro\textunderscore .
\section{Espirrador}
\begin{itemize}
\item {Grp. gram.:adj.}
\end{itemize}
\begin{itemize}
\item {Grp. gram.:M.}
\end{itemize}
Que espirra.
Aquelle que espirra.
\section{Espirrar}
\begin{itemize}
\item {Grp. gram.:v. t.}
\end{itemize}
\begin{itemize}
\item {Grp. gram.:V. i.}
\end{itemize}
\begin{itemize}
\item {Utilização:Fig.}
\end{itemize}
\begin{itemize}
\item {Grp. gram.:V. i.}
\end{itemize}
\begin{itemize}
\item {Utilização:Bras. do N}
\end{itemize}
\begin{itemize}
\item {Proveniência:(Lat. hyp. \textunderscore expirulare\textunderscore . Cp. pop. \textunderscore espilrar\textunderscore )}
\end{itemize}
Expellir, lançar fóra.
Dar espirros.
Esguichar.
Crepitar: \textunderscore há lenha que, ardendo, espirra\textunderscore .
Agastar-se.
Respingar.
Sair ou irromper, ás carreiras: \textunderscore o boi espirrou da mata\textunderscore .
\section{Espirrichar}
\begin{itemize}
\item {Grp. gram.:v. t.}
\end{itemize}
\begin{itemize}
\item {Utilização:Prov.}
\end{itemize}
\begin{itemize}
\item {Utilização:trasm.}
\end{itemize}
Fazer saltar ou jorrar (a água).
(Cp. \textunderscore espirrar\textunderscore )
\section{Espirro}
\begin{itemize}
\item {Grp. gram.:m.}
\end{itemize}
\begin{itemize}
\item {Proveniência:(De \textunderscore espirrar\textunderscore )}
\end{itemize}
Saída violenta e estrepitosa do ar pela bôca e pelo nariz, com movimento convulsivo súbito dos músculos da respiração, em consequência de comichão ou excitação na membrana pituitária.
Esternutação.
\section{Espirrote}
\begin{itemize}
\item {Grp. gram.:m.}
\end{itemize}
\begin{itemize}
\item {Utilização:Prov.}
\end{itemize}
\begin{itemize}
\item {Proveniência:(De \textunderscore espirrar\textunderscore )}
\end{itemize}
Casca de pinheiro.
Corcódea.
\section{Espiuncado}
\begin{itemize}
\item {Grp. gram.:adj.}
\end{itemize}
\begin{itemize}
\item {Utilização:Prov.}
\end{itemize}
\begin{itemize}
\item {Utilização:trasm.}
\end{itemize}
Que não contém nada; vazio, (falando-se do bolso ou da casa).
\section{Esplainada}
\begin{itemize}
\item {Grp. gram.:f.}
\end{itemize}
O mesmo que \textunderscore explanada\textunderscore . Cf. Camillo, \textunderscore Homem Rico\textunderscore , 56.
\section{Esplainar-se}
\begin{itemize}
\item {Grp. gram.:v. p.}
\end{itemize}
\begin{itemize}
\item {Proveniência:(De \textunderscore plaino\textunderscore )}
\end{itemize}
Tornar-se plano; formar planície. Cf. Camillo, \textunderscore Amor de Perd.\textunderscore , 89.
\section{Esplanada}
\begin{itemize}
\item {Grp. gram.:f.}
\end{itemize}
O mesmo que \textunderscore explanada\textunderscore .
\section{Esplânchnico}
\begin{itemize}
\item {Grp. gram.:adj.}
\end{itemize}
\begin{itemize}
\item {Utilização:Anat.}
\end{itemize}
\begin{itemize}
\item {Proveniência:(Do gr. \textunderscore splankhnon\textunderscore )}
\end{itemize}
Relativo ás vísceras.
\section{Esplanchnographia}
\begin{itemize}
\item {Grp. gram.:f.}
\end{itemize}
\begin{itemize}
\item {Proveniência:(Do gr. \textunderscore splankhnon\textunderscore  + \textunderscore graphein\textunderscore )}
\end{itemize}
Descripção das vísceras.
\section{Esplanchnográphico}
\begin{itemize}
\item {Grp. gram.:adj.}
\end{itemize}
Relativo a esplanchnographia.
\section{Esplanchnologia}
\begin{itemize}
\item {Grp. gram.:f.}
\end{itemize}
\begin{itemize}
\item {Proveniência:(Do gr. \textunderscore splankhnon\textunderscore  + \textunderscore logos\textunderscore )}
\end{itemize}
Tratado á cêrca das vísceras.
\section{Esplanchnológico}
\begin{itemize}
\item {Grp. gram.:adj.}
\end{itemize}
Relativo a esplanchnologia.
\section{Esplanchnómero}
\begin{itemize}
\item {Grp. gram.:m.}
\end{itemize}
\begin{itemize}
\item {Utilização:Anat.}
\end{itemize}
\begin{itemize}
\item {Proveniência:(Do gr. \textunderscore splankhnon\textunderscore  + \textunderscore meros\textunderscore )}
\end{itemize}
Parte visceral do metâmero.
\section{Esplanchnoptose}
\begin{itemize}
\item {Grp. gram.:f.}
\end{itemize}
\begin{itemize}
\item {Utilização:Med.}
\end{itemize}
\begin{itemize}
\item {Proveniência:(Do gr. \textunderscore splankhnon\textunderscore  + \textunderscore ptosis\textunderscore )}
\end{itemize}
Prolapso das vísceras abdominaes.
\section{Esplanchnotomia}
\begin{itemize}
\item {Grp. gram.:f.}
\end{itemize}
\begin{itemize}
\item {Proveniência:(Do gr. \textunderscore spalankhnon\textunderscore  + \textunderscore tome\textunderscore )}
\end{itemize}
Dissecção das vísceras.
\section{Esplâncnico}
\begin{itemize}
\item {Grp. gram.:adj.}
\end{itemize}
\begin{itemize}
\item {Utilização:Anat.}
\end{itemize}
\begin{itemize}
\item {Proveniência:(Do gr. \textunderscore splankhnon\textunderscore )}
\end{itemize}
Relativo ás vísceras.
\section{Esplancnografia}
\begin{itemize}
\item {Grp. gram.:f.}
\end{itemize}
\begin{itemize}
\item {Proveniência:(Do gr. \textunderscore splankhnon\textunderscore  + \textunderscore graphein\textunderscore )}
\end{itemize}
Descripção das vísceras.
\section{Esplancnográfico}
\begin{itemize}
\item {Grp. gram.:adj.}
\end{itemize}
Relativo a esplancnografia.
\section{Esplancnologia}
\begin{itemize}
\item {Grp. gram.:f.}
\end{itemize}
\begin{itemize}
\item {Proveniência:(Do gr. \textunderscore splankhnon\textunderscore  + \textunderscore logos\textunderscore )}
\end{itemize}
Tratado á cêrca das vísceras.
\section{Esplancnológico}
\begin{itemize}
\item {Grp. gram.:adj.}
\end{itemize}
Relativo a esplancnologia.
\section{Esplancnómero}
\begin{itemize}
\item {Grp. gram.:m.}
\end{itemize}
\begin{itemize}
\item {Utilização:Anat.}
\end{itemize}
\begin{itemize}
\item {Proveniência:(Do gr. \textunderscore splankhnon\textunderscore  + \textunderscore meros\textunderscore )}
\end{itemize}
Parte visceral do metâmero.
\section{Esplancnoptose}
\begin{itemize}
\item {Grp. gram.:f.}
\end{itemize}
\begin{itemize}
\item {Utilização:Med.}
\end{itemize}
\begin{itemize}
\item {Proveniência:(Do gr. \textunderscore splankhnon\textunderscore  + \textunderscore ptosis\textunderscore )}
\end{itemize}
Prolapso das vísceras abdominaes.
\section{Esplancnotomia}
\begin{itemize}
\item {Grp. gram.:f.}
\end{itemize}
\begin{itemize}
\item {Proveniência:(Do gr. \textunderscore spalankhnon\textunderscore  + \textunderscore tome\textunderscore )}
\end{itemize}
Dissecção das vísceras.
\section{Esplandecer}
\textunderscore v. i.\textunderscore  (e der.)
(Corr. de \textunderscore esplendecer\textunderscore , etc.)
\section{Esplander}
\textunderscore v. i.\textunderscore  (e der.)
(Corr. de \textunderscore esplender\textunderscore , etc.)
\section{Esplenalgia}
\begin{itemize}
\item {Grp. gram.:f.}
\end{itemize}
\begin{itemize}
\item {Proveniência:(Do gr. \textunderscore splen\textunderscore  + \textunderscore algos\textunderscore )}
\end{itemize}
Dôr no baço.
\section{Esplendecência}
\begin{itemize}
\item {Grp. gram.:f.}
\end{itemize}
Qualidade daquillo que é esplendecente.
\section{Esplendecente}
\begin{itemize}
\item {Grp. gram.:adj.}
\end{itemize}
Que esplendece.
\section{Esplendecer}
\begin{itemize}
\item {Grp. gram.:v. i.}
\end{itemize}
O mesmo que \textunderscore resplendecer\textunderscore .
\section{Esplendecimento}
\begin{itemize}
\item {Grp. gram.:m.}
\end{itemize}
Acto de esplendecer.
\section{Esplendente}
\begin{itemize}
\item {Grp. gram.:adj.}
\end{itemize}
\begin{itemize}
\item {Proveniência:(Lat. \textunderscore splendens\textunderscore )}
\end{itemize}
Que esplende.
\section{Esplender}
\begin{itemize}
\item {Grp. gram.:v. i.}
\end{itemize}
\begin{itemize}
\item {Proveniência:(Lat. \textunderscore splendere\textunderscore )}
\end{itemize}
Brilhar.
O mesmo que \textunderscore resplender\textunderscore .
\section{Esplendidamente}
\begin{itemize}
\item {Grp. gram.:adv.}
\end{itemize}
De modo esplêndido, brilhantemente.
Magnificamente.
Luxuosamente.
\section{Esplendidez}
\begin{itemize}
\item {Grp. gram.:f.}
\end{itemize}
Qualidade daquillo que é esplêndido.
\section{Esplendideza}
\begin{itemize}
\item {Grp. gram.:f.}
\end{itemize}
(V.esplendidez)
\section{Esplêndido}
\begin{itemize}
\item {Grp. gram.:adj.}
\end{itemize}
\begin{itemize}
\item {Proveniência:(Lat. \textunderscore splendidus\textunderscore )}
\end{itemize}
Que tem esplendor.
Que brilha: \textunderscore sol esplêndido\textunderscore .
Reluzente.
Magnificente: \textunderscore jóias esplêndidas\textunderscore .
Admirável.
Luxuoso.
Grandioso: \textunderscore festas esplêndidas\textunderscore .
\section{Esplendor}
\begin{itemize}
\item {Grp. gram.:m.}
\end{itemize}
\begin{itemize}
\item {Utilização:Fig.}
\end{itemize}
\begin{itemize}
\item {Proveniência:(Lat. \textunderscore splendor\textunderscore )}
\end{itemize}
Fulgor.
Brilho intenso.
Deslumbramento.
Pompa.
Grandeza; magnificência.
\section{Esplendoroso}
\begin{itemize}
\item {Grp. gram.:adj.}
\end{itemize}
\begin{itemize}
\item {Proveniência:(De \textunderscore esplendor\textunderscore )}
\end{itemize}
O mesmo que \textunderscore esplêndido\textunderscore .
\section{Esplenemphraxia}
\begin{itemize}
\item {Grp. gram.:f.}
\end{itemize}
\begin{itemize}
\item {Utilização:Med.}
\end{itemize}
\begin{itemize}
\item {Proveniência:(Do gr. \textunderscore splen\textunderscore  + \textunderscore emphraxis\textunderscore )}
\end{itemize}
Obstrucção do baço.
\section{Esplenenfraxia}
\begin{itemize}
\item {Grp. gram.:f.}
\end{itemize}
\begin{itemize}
\item {Utilização:Med.}
\end{itemize}
\begin{itemize}
\item {Proveniência:(Do gr. \textunderscore splen\textunderscore  + \textunderscore emphraxis\textunderscore )}
\end{itemize}
Obstrucção do baço.
\section{Esplenético}
\begin{itemize}
\item {Grp. gram.:m.  e  adj.}
\end{itemize}
\begin{itemize}
\item {Proveniência:(Do gr. \textunderscore splen\textunderscore )}
\end{itemize}
O que tem doença no baço.
\section{Esplenetomia}
\begin{itemize}
\item {Grp. gram.:f.}
\end{itemize}
\begin{itemize}
\item {Utilização:Med.}
\end{itemize}
\begin{itemize}
\item {Proveniência:(Do gr. \textunderscore splen\textunderscore  + \textunderscore tome\textunderscore )}
\end{itemize}
Extirpação do baço.
\section{Esplenial}
\begin{itemize}
\item {Grp. gram.:adj.}
\end{itemize}
\begin{itemize}
\item {Proveniência:(Do gr. \textunderscore splen\textunderscore )}
\end{itemize}
Relativo ao baço.
\section{Esplénico}
\begin{itemize}
\item {Grp. gram.:adj.}
\end{itemize}
\begin{itemize}
\item {Proveniência:(Do gr. \textunderscore splen\textunderscore )}
\end{itemize}
Relativo ao baço.
\section{Esplenificação}
\begin{itemize}
\item {Grp. gram.:f.}
\end{itemize}
\begin{itemize}
\item {Utilização:Med.}
\end{itemize}
\begin{itemize}
\item {Proveniência:(Do gr. \textunderscore splen\textunderscore  + lat. \textunderscore facere\textunderscore )}
\end{itemize}
Endurecimento de um tecido, á semelhança do baço.
\section{Esplênio}
\begin{itemize}
\item {Grp. gram.:m.}
\end{itemize}
\begin{itemize}
\item {Utilização:Anat.}
\end{itemize}
\begin{itemize}
\item {Proveniência:(Gr. \textunderscore splenion\textunderscore )}
\end{itemize}
Músculo alongado e achatado, na parte posterior do pescoço e superior das costas.
\section{Esplenite}
\begin{itemize}
\item {Grp. gram.:f.}
\end{itemize}
\begin{itemize}
\item {Proveniência:(Do gr. \textunderscore splen\textunderscore )}
\end{itemize}
Inflammação do baço.
\section{Esplenocele}
\begin{itemize}
\item {Grp. gram.:f.}
\end{itemize}
\begin{itemize}
\item {Proveniência:(Do gr. \textunderscore splen\textunderscore  + \textunderscore kele\textunderscore )}
\end{itemize}
Hérnia no baço.
\section{Esplenografia}
\begin{itemize}
\item {Grp. gram.:f.}
\end{itemize}
\begin{itemize}
\item {Proveniência:(Do gr. \textunderscore splen\textunderscore  + \textunderscore graphein\textunderscore )}
\end{itemize}
Descripção do baço.
\section{Esplenográfico}
\begin{itemize}
\item {Grp. gram.:adj.}
\end{itemize}
Relativo a esplenografia.
\section{Esplenógrafo}
\begin{itemize}
\item {Grp. gram.:m.}
\end{itemize}
Aquelle que se dedica á esplenografia.
\section{Esplenographia}
\begin{itemize}
\item {Grp. gram.:f.}
\end{itemize}
\begin{itemize}
\item {Proveniência:(Do gr. \textunderscore splen\textunderscore  + \textunderscore graphein\textunderscore )}
\end{itemize}
Descripção do baço.
\section{Esplenográphico}
\begin{itemize}
\item {Grp. gram.:adj.}
\end{itemize}
Relativo a esplenographia.
\section{Esplenógrapho}
\begin{itemize}
\item {Grp. gram.:m.}
\end{itemize}
Aquelle que se dedica á esplenographia.
\section{Esplenoide}
\begin{itemize}
\item {Grp. gram.:adj.}
\end{itemize}
\begin{itemize}
\item {Proveniência:(Do gr. \textunderscore splen\textunderscore  + \textunderscore eidos\textunderscore )}
\end{itemize}
Que tem a apparência do baço.
\section{Esplenologia}
\begin{itemize}
\item {Grp. gram.:f.}
\end{itemize}
\begin{itemize}
\item {Proveniência:(Do gr. \textunderscore splen\textunderscore  + \textunderscore logos\textunderscore )}
\end{itemize}
Tratado á cêrca do baço.
\section{Esplenomegalia}
\begin{itemize}
\item {Grp. gram.:f.}
\end{itemize}
\begin{itemize}
\item {Utilização:Med.}
\end{itemize}
\begin{itemize}
\item {Proveniência:(Do gr. \textunderscore splen\textunderscore  + \textunderscore megas\textunderscore )}
\end{itemize}
Hypertrophia do baço.
\section{Esplenoncia}
\begin{itemize}
\item {Grp. gram.:f.}
\end{itemize}
\begin{itemize}
\item {Proveniência:(Do gr. \textunderscore splen\textunderscore  + \textunderscore onkos\textunderscore )}
\end{itemize}
Tumefacção do baço.
\section{Esplenopathia}
\begin{itemize}
\item {Grp. gram.:f.}
\end{itemize}
\begin{itemize}
\item {Proveniência:(Do gr. \textunderscore splen\textunderscore  + \textunderscore pathos\textunderscore )}
\end{itemize}
Doença do baço.
\section{Esplenopáthico}
\begin{itemize}
\item {Grp. gram.:adj.}
\end{itemize}
Relativo a esplenopathia.
\section{Esplenopatia}
\begin{itemize}
\item {Grp. gram.:f.}
\end{itemize}
\begin{itemize}
\item {Proveniência:(Do gr. \textunderscore splen\textunderscore  + \textunderscore pathos\textunderscore )}
\end{itemize}
Doença do baço.
\section{Esplenopático}
\begin{itemize}
\item {Grp. gram.:adj.}
\end{itemize}
Relativo a esplenopatia.
\section{Esplenotomia}
\begin{itemize}
\item {Grp. gram.:f.}
\end{itemize}
\begin{itemize}
\item {Proveniência:(Do gr. \textunderscore splen\textunderscore  + \textunderscore tome\textunderscore )}
\end{itemize}
Dissecção do baço.
\section{Espoador}
\begin{itemize}
\item {Grp. gram.:m.}
\end{itemize}
Peneira ou instrumento para espoar.
\section{Espoar}
\begin{itemize}
\item {Grp. gram.:v. t.}
\end{itemize}
\begin{itemize}
\item {Proveniência:(De \textunderscore pó\textunderscore )}
\end{itemize}
Peneirar segunda vez (a farinha), para se fabricar o pão mais fino.
\section{Espocar}
\begin{itemize}
\item {Grp. gram.:v. i.}
\end{itemize}
\begin{itemize}
\item {Utilização:Bras. do N}
\end{itemize}
O mesmo que \textunderscore pipocar\textunderscore .
Estoirar, explodir.
\section{Espodita}
\begin{itemize}
\item {Grp. gram.:f.}
\end{itemize}
\begin{itemize}
\item {Proveniência:(Do gr. \textunderscore spodos\textunderscore , cinza)}
\end{itemize}
Cinza branca dos vulcões.
\section{Espodomancia}
\begin{itemize}
\item {Grp. gram.:f.}
\end{itemize}
\begin{itemize}
\item {Proveniência:(Do gr. \textunderscore spodion\textunderscore  + \textunderscore manteia\textunderscore )}
\end{itemize}
Supposta arte de adivinhar, por meio das cinzas dos antigos sacrificios.
\section{Espojadoiro}
\begin{itemize}
\item {Grp. gram.:m.}
\end{itemize}
Lugar, onde se espojam animaes.
\section{Espojadouro}
\begin{itemize}
\item {Grp. gram.:m.}
\end{itemize}
Lugar, onde se espojam animaes.
\section{Espojadura}
\begin{itemize}
\item {Grp. gram.:f.}
\end{itemize}
Acto de espojar-se.
\section{Espojar-se}
\begin{itemize}
\item {Grp. gram.:v. p.}
\end{itemize}
\begin{itemize}
\item {Utilização:Fam.}
\end{itemize}
\begin{itemize}
\item {Proveniência:(De \textunderscore pó\textunderscore ?)}
\end{itemize}
Deitar-se no chão, agitando-se e rebolando-se.
Andar aos tombos no chão.
Estender-se, fazer asneira.
\section{Espojeiro}
\begin{itemize}
\item {Grp. gram.:m.}
\end{itemize}
\begin{itemize}
\item {Utilização:Bras}
\end{itemize}
\begin{itemize}
\item {Utilização:Bras}
\end{itemize}
\begin{itemize}
\item {Utilização:Fig.}
\end{itemize}
\begin{itemize}
\item {Proveniência:(De \textunderscore espojar\textunderscore )}
\end{itemize}
O mesmo que \textunderscore espojadoiro\textunderscore .
Pequeno cercado em tòrno da casa.
Pequena roça.
\section{Espojinho}
\begin{itemize}
\item {fónica:pó}
\end{itemize}
\begin{itemize}
\item {Grp. gram.:m.}
\end{itemize}
\begin{itemize}
\item {Utilização:Prov.}
\end{itemize}
\begin{itemize}
\item {Utilização:alent.}
\end{itemize}
Pequeno remoínho de vento.
(Cp. [[espojar|espojar-se]])
\section{Espoldra}
\begin{itemize}
\item {Grp. gram.:f.}
\end{itemize}
Acto de espoldrar.
\section{Espoldrador}
\begin{itemize}
\item {Grp. gram.:m.}
\end{itemize}
\begin{itemize}
\item {Utilização:Agr.}
\end{itemize}
Instrumento, para espoldrar.
\section{Espoldrão}
\begin{itemize}
\item {Grp. gram.:m.}
\end{itemize}
\begin{itemize}
\item {Utilização:Prov.}
\end{itemize}
\begin{itemize}
\item {Utilização:dur.}
\end{itemize}
\begin{itemize}
\item {Proveniência:(De \textunderscore espoldrar\textunderscore )}
\end{itemize}
Vergôntea que, ao lume da terra, nasce do tronco da videira.
\section{Espoldrar}
\begin{itemize}
\item {Grp. gram.:v. t.}
\end{itemize}
\begin{itemize}
\item {Proveniência:(De \textunderscore poldra\textunderscore ^2)}
\end{itemize}
Podar segunda vez ou desramar depois da vindima (a videira).
\section{Espoldrinhar}
\begin{itemize}
\item {Grp. gram.:v. i.}
\end{itemize}
\begin{itemize}
\item {Utilização:T. de Turquel}
\end{itemize}
Retoiçar como um poldro.
\section{Espoleta}
\begin{itemize}
\item {fónica:lê}
\end{itemize}
\begin{itemize}
\item {Utilização:Bras}
\end{itemize}
\begin{itemize}
\item {Proveniência:(Fr. \textunderscore espolette\textunderscore )}
\end{itemize}
Artefacto de metal ou madeira, que determina a inflammação da carga dos projécteis ocos.
Espécie de pequeno funil, no extremo de uma peça de fogo, no qual se põe a escorva.
Escorva.
Fanfarrão; homem de bando, assalariado; valentão, que serve de guarda-costas a algum fazendeiro ou senhor de engenho.
\section{Espoletar}
\begin{itemize}
\item {Grp. gram.:v. t.}
\end{itemize}
Por espoleta em.
\section{Espolete}
\begin{itemize}
\item {fónica:lê}
\end{itemize}
\begin{itemize}
\item {Grp. gram.:m.}
\end{itemize}
Varinha de arame, em que gira a canela, dentro da lançadeira.
(Relaciona-se com \textunderscore espoleta\textunderscore ?)
\section{Espoletear}
\begin{itemize}
\item {Grp. gram.:v. t.}
\end{itemize}
\begin{itemize}
\item {Utilização:Bras}
\end{itemize}
Ficar tonto. Cf. J. Ribeiro, \textunderscore Diccion. Gram.\textunderscore , 71.
(Por \textunderscore espelotear\textunderscore , de \textunderscore pelota\textunderscore . Cp. \textunderscore espeloteado\textunderscore )
\section{Espoliação}
\begin{itemize}
\item {Grp. gram.:f.}
\end{itemize}
\begin{itemize}
\item {Proveniência:(Lat. \textunderscore spoliatio\textunderscore )}
\end{itemize}
Acto ou effeito de espoliar.
\section{Espoliador}
\begin{itemize}
\item {Grp. gram.:m.  e  adj.}
\end{itemize}
\begin{itemize}
\item {Proveniência:(Lat. \textunderscore spoliator\textunderscore )}
\end{itemize}
O que espolia.
\section{Espoliante}
\begin{itemize}
\item {Grp. gram.:m.  e  adj.}
\end{itemize}
\begin{itemize}
\item {Proveniência:(Lat. \textunderscore spolians\textunderscore )}
\end{itemize}
O mesmo que \textunderscore espoliador\textunderscore .
\section{Espoliar}
\begin{itemize}
\item {Grp. gram.:v. t.}
\end{itemize}
\begin{itemize}
\item {Proveniência:(Lat. \textunderscore spoliare\textunderscore )}
\end{itemize}
Privar illegalmente de alguma coisa.
Desapossar com violência ou fraude.
Despojar.
\section{Espoliário}
\begin{itemize}
\item {Grp. gram.:m.}
\end{itemize}
\begin{itemize}
\item {Proveniência:(Lat. \textunderscore spoliarium\textunderscore )}
\end{itemize}
Lugar, annexo ás arenas romanas, no qual se despojavam das vestes os gladiadores mortos na arena, e se acabavam de matar os que tinham sido feridos mortalmente.
Compartimento das thermas romanas, no qual os banhistas se despiam e vestiam.
\section{Espoliativamente}
\begin{itemize}
\item {Grp. gram.:adv.}
\end{itemize}
De modo espoliativo.
\section{Espoliativo}
\begin{itemize}
\item {Grp. gram.:adj.}
\end{itemize}
\begin{itemize}
\item {Grp. gram.:M.}
\end{itemize}
\begin{itemize}
\item {Proveniência:(De \textunderscore espoliar\textunderscore )}
\end{itemize}
Em que há espoliação.
Que espolia.
Medicamento, que, applicado sôbre a pelle, tira a epiderme.
\section{Espolim}
\begin{itemize}
\item {Grp. gram.:m.}
\end{itemize}
\begin{itemize}
\item {Proveniência:(Fr. \textunderscore espoulin\textunderscore )}
\end{itemize}
Lançadeira, para entretecer flôres nos tecidos.
\section{Espolinar}
\begin{itemize}
\item {Grp. gram.:v. t.}
\end{itemize}
Tecer com espolim.
\section{Espolinhadoiro}
\begin{itemize}
\item {Grp. gram.:m.}
\end{itemize}
Lugar, onde alguém ou alguma coisa se espolinha:«\textunderscore éramos espolinhadoiro do seu espírito.\textunderscore »Camillo, \textunderscore Nov. do Minh.\textunderscore , I, 21.
\section{Espolinhadouro}
\begin{itemize}
\item {Grp. gram.:m.}
\end{itemize}
Lugar, onde alguém ou alguma coisa se espolinha:«\textunderscore éramos espolinhadouro do seu espírito.\textunderscore »Camillo, \textunderscore Nov. do Minh.\textunderscore , I, 21.
\section{Espolinhar}
\begin{itemize}
\item {Grp. gram.:v. t.}
\end{itemize}
\begin{itemize}
\item {Utilização:Prov.}
\end{itemize}
\begin{itemize}
\item {Utilização:trasm.}
\end{itemize}
Escorraçar (uma bêsta), correndo nella a toda a brida.
\section{Espolinhar-se}
\begin{itemize}
\item {Grp. gram.:v. p.}
\end{itemize}
O mesmo que \textunderscore espojar-se\textunderscore . Cf. Camillo, \textunderscore Corja\textunderscore , 148.
\section{Espólio}
\begin{itemize}
\item {Grp. gram.:m.}
\end{itemize}
\begin{itemize}
\item {Proveniência:(Lat. \textunderscore spolium\textunderscore )}
\end{itemize}
Bens, que alguém deixou por sua morte.
Despojos de guerra.
Espoliação.
\section{Espolpar}
\begin{itemize}
\item {Grp. gram.:v. t.}
\end{itemize}
Extrahir a polpa de. Cf. \textunderscore Techn. Rur.\textunderscore , 237.
\section{Esponda}
\begin{itemize}
\item {Grp. gram.:f.}
\end{itemize}
\begin{itemize}
\item {Utilização:Des.}
\end{itemize}
\begin{itemize}
\item {Proveniência:(Lat. \textunderscore sponda\textunderscore )}
\end{itemize}
Borda ou lado do leito.
Tábua, que resguarda o colchão.
Columna do leito.
\section{Espondaico}
\begin{itemize}
\item {Grp. gram.:adj.}
\end{itemize}
\begin{itemize}
\item {Proveniência:(Gr. \textunderscore spondaikos\textunderscore  ou, antes, \textunderscore spondeiakos\textunderscore , de \textunderscore spondeios\textunderscore )}
\end{itemize}
Diz-se do verso, que é composto de espondeus.
\section{Espondeu}
\begin{itemize}
\item {Grp. gram.:m.}
\end{itemize}
\begin{itemize}
\item {Proveniência:(Lat. \textunderscore spondeus\textunderscore )}
\end{itemize}
Pé de verso grego ou latino, formado de duas sýllabas longas.
\section{Espôndil}
\begin{itemize}
\item {Grp. gram.:m.}
\end{itemize}
O mesmo que \textunderscore espôndilo\textunderscore .
\section{Espondílico}
\begin{itemize}
\item {Grp. gram.:adj.}
\end{itemize}
Relativo a espôndilo.
\section{Espondílidas}
\begin{itemize}
\item {Grp. gram.:f. pl.}
\end{itemize}
\begin{itemize}
\item {Proveniência:(Do gr. \textunderscore spondulos\textunderscore  + \textunderscore eidos\textunderscore )}
\end{itemize}
Insectos lignívoros da ordem dos coleópteros.
\section{Espondilite}
\begin{itemize}
\item {Grp. gram.:f.}
\end{itemize}
\begin{itemize}
\item {Utilização:Med.}
\end{itemize}
\begin{itemize}
\item {Proveniência:(Do gr. \textunderscore spondulos\textunderscore )}
\end{itemize}
Inflamação vertebral.
\section{Espôndilo}
\begin{itemize}
\item {Grp. gram.:m.}
\end{itemize}
\begin{itemize}
\item {Proveniência:(Gr. \textunderscore spondulos\textunderscore )}
\end{itemize}
Qualquer vértebra.
Especialmente, a segunda vértebra do pescoço.
Gênero de ostras.
\section{Espondilozoário}
\begin{itemize}
\item {Grp. gram.:m.}
\end{itemize}
\begin{itemize}
\item {Proveniência:(Do gr. \textunderscore spondulos\textunderscore  + \textunderscore zoon\textunderscore )}
\end{itemize}
Animal, provido de coluna vertebral.
\section{Espôndyl}
\begin{itemize}
\item {Grp. gram.:m.}
\end{itemize}
O mesmo que \textunderscore espôndylo\textunderscore .
\section{Espondýlico}
\begin{itemize}
\item {Grp. gram.:adj.}
\end{itemize}
Relativo a espôndylo.
\section{Espondýlidas}
\begin{itemize}
\item {Grp. gram.:f. pl.}
\end{itemize}
\begin{itemize}
\item {Proveniência:(Do gr. \textunderscore spondulos\textunderscore  + \textunderscore eidos\textunderscore )}
\end{itemize}
Insectos lignívoros da ordem dos coleópteros.
\section{Espondylite}
\begin{itemize}
\item {Grp. gram.:f.}
\end{itemize}
\begin{itemize}
\item {Utilização:Med.}
\end{itemize}
\begin{itemize}
\item {Proveniência:(Do gr. \textunderscore spondulos\textunderscore )}
\end{itemize}
Inflammação vertebral.
\section{Espôndylo}
\begin{itemize}
\item {Grp. gram.:m.}
\end{itemize}
\begin{itemize}
\item {Proveniência:(Gr. \textunderscore spondulos\textunderscore )}
\end{itemize}
Qualquer vértebra.
Especialmente, a segunda vértebra do pescoço.
Gênero de ostras.
\section{Espondylozoário}
\begin{itemize}
\item {Grp. gram.:m.}
\end{itemize}
\begin{itemize}
\item {Proveniência:(Do gr. \textunderscore spondulos\textunderscore  + \textunderscore zoon\textunderscore )}
\end{itemize}
Animal, provido de columna vertebral.
\section{Espongiários}
\begin{itemize}
\item {Grp. gram.:m. pl.}
\end{itemize}
\begin{itemize}
\item {Proveniência:(Do lat. \textunderscore spongia\textunderscore .)}
\end{itemize}
Classe de animaes rudimentares, que têm por typo a esponja.
\section{Espongíola}
\begin{itemize}
\item {Grp. gram.:f.}
\end{itemize}
O mesmo que \textunderscore espongíolo\textunderscore .
\section{Espongíolo}
\begin{itemize}
\item {Grp. gram.:m.}
\end{itemize}
\begin{itemize}
\item {Utilização:Bot.}
\end{itemize}
\begin{itemize}
\item {Proveniência:(Lat. \textunderscore spongiolus\textunderscore )}
\end{itemize}
Última ramificação das raízes, por onde estas absorvem os elementos nutritivos do solo.
\section{Espongite}
\begin{itemize}
\item {Grp. gram.:f.}
\end{itemize}
\begin{itemize}
\item {Proveniência:(Lat. \textunderscore spongites\textunderscore )}
\end{itemize}
Pedra, cheia de poros, que imita a esponja.
\section{Esponja}
\begin{itemize}
\item {Grp. gram.:f.}
\end{itemize}
\begin{itemize}
\item {Utilização:Fig.}
\end{itemize}
\begin{itemize}
\item {Proveniência:(Do lat. \textunderscore spongia\textunderscore )}
\end{itemize}
Animal invertebrado, que serve de typo aos espongiários.
Substância leve e porosa, proveniente de um espongiário marinho, e que absorve o líquido em que se mergulha.
Flôr de esponjeira.
Bêbedo.
Parasito.
\section{Esponjar}
\begin{itemize}
\item {Grp. gram.:v. t.}
\end{itemize}
\begin{itemize}
\item {Proveniência:(Do b. lat. \textunderscore spongiare\textunderscore )}
\end{itemize}
Apagar ou tirar com esponja.
Absorver, como esponja.
Eliminar:«\textunderscore ...não era esponjar a Polónia...\textunderscore »Camillo, \textunderscore Freira no Subterr.\textunderscore , 167.
Subtahir, surripiar:«\textunderscore ...que esponjara á Fazenda Nacional 50 contos.\textunderscore »Camillo, \textunderscore Mulher Fatal\textunderscore , 153. Cf. Castilho, \textunderscore Fastos\textunderscore , I, p. XXVII.
\section{Esponjeira}
\begin{itemize}
\item {Grp. gram.:f.}
\end{itemize}
Espécie de acácia mimósea, de flôres amarelas e odoríferas.
Vaso ou lugar, em que se collocam ou guardam esponjas.
\section{Esponjiforme}
\begin{itemize}
\item {Grp. gram.:adj.}
\end{itemize}
\begin{itemize}
\item {Proveniência:(Do lat. \textunderscore spongia\textunderscore  + \textunderscore forma\textunderscore )}
\end{itemize}
Que tem fórma de esponja.
\section{Esponjosidade}
\begin{itemize}
\item {Grp. gram.:f.}
\end{itemize}
Qualidade daquillo que é esponjoso.
\section{Esponjoso}
\begin{itemize}
\item {Grp. gram.:adj.}
\end{itemize}
\begin{itemize}
\item {Proveniência:(Do lat. \textunderscore spongiosus\textunderscore )}
\end{itemize}
Que tem poros, á semelhança da esponja.
\section{Esponsaes}
\begin{itemize}
\item {Grp. gram.:m. pl.}
\end{itemize}
\begin{itemize}
\item {Proveniência:(Lat. \textunderscore sponsalia\textunderscore )}
\end{itemize}
Contrato de casamento.
Ceremónia, que antecede o casamento.
\section{Esponsais}
\begin{itemize}
\item {Grp. gram.:m. pl.}
\end{itemize}
\begin{itemize}
\item {Proveniência:(Lat. \textunderscore sponsalia\textunderscore )}
\end{itemize}
Contrato de casamento.
Ceremónia, que antecede o casamento.
\section{Esponsal}
\begin{itemize}
\item {Grp. gram.:adj.}
\end{itemize}
\begin{itemize}
\item {Proveniência:(Lat. \textunderscore sponsalis\textunderscore )}
\end{itemize}
Relativo a esposos.
\section{Esponsálias}
\begin{itemize}
\item {Grp. gram.:f. pl.}
\end{itemize}
(V.esponsaes)
\section{Esponsalício}
\begin{itemize}
\item {Grp. gram.:adj.}
\end{itemize}
\begin{itemize}
\item {Proveniência:(Lat. \textunderscore sponsalicius\textunderscore )}
\end{itemize}
O mesmo que \textunderscore esponsal\textunderscore .
\section{Espontaneamente}
\begin{itemize}
\item {Grp. gram.:adv.}
\end{itemize}
De modo espontâneo.
Sem coacção.
Voluntariamente.
Facilmente.
Sem cultura: \textunderscore plantas, que apparecem espontaneamente\textunderscore .
\section{Espontaneidade}
\begin{itemize}
\item {Grp. gram.:f.}
\end{itemize}
Qualidade daquillo que é espontâneo.
\section{Espontâneo}
\begin{itemize}
\item {Grp. gram.:adj.}
\end{itemize}
\begin{itemize}
\item {Proveniência:(Lat. \textunderscore spontaneus\textunderscore )}
\end{itemize}
Que se pratica de livre vontade, de mótu-próprio: \textunderscore acto espontâneo\textunderscore .
Natural.
Independente de causa exterior apparente.
Que nasceu sem cultura: \textunderscore vegetação espontânea\textunderscore .
\section{Espontão}
\begin{itemize}
\item {Grp. gram.:m.}
\end{itemize}
\begin{itemize}
\item {Utilização:Ant.}
\end{itemize}
\begin{itemize}
\item {Proveniência:(It. \textunderscore spuntone\textunderscore )}
\end{itemize}
Albarda comprida.
\textunderscore Cortesia de espontão\textunderscore , cortesia, que os militares fazem com a espada, dirigindo a ponta para deante. Cf. Camillo, \textunderscore Cav. em Ruínas\textunderscore , 99.
\section{Espontar}
\begin{itemize}
\item {Grp. gram.:v. t.}
\end{itemize}
\begin{itemize}
\item {Proveniência:(De \textunderscore ponta\textunderscore )}
\end{itemize}
Cortar as pontas a.
Aparar.
\section{Espora}
\begin{itemize}
\item {Grp. gram.:f.}
\end{itemize}
\begin{itemize}
\item {Utilização:Fig.}
\end{itemize}
\begin{itemize}
\item {Grp. gram.:Loc. adv.}
\end{itemize}
\begin{itemize}
\item {Proveniência:(Do b. lat. \textunderscore spora\textunderscore )}
\end{itemize}
Instrumento de metal, que se adapta á parte posterior do calçado, para estimular a bêsta em que se monta.
Incitamento, estímulo.
Planta ranunculácea.
Flôr dessa planta.
\textunderscore Á espora-fita\textunderscore , á desfilada. Cf. Camillo, \textunderscore Caveira\textunderscore , 445.
\section{Esporada}
\begin{itemize}
\item {Grp. gram.:f.}
\end{itemize}
\begin{itemize}
\item {Utilização:Fig.}
\end{itemize}
\begin{itemize}
\item {Utilização:Fam.}
\end{itemize}
Picada com espora.
Incitamento.
Descompostura.
Censura picante.
\section{Esporacidade}
\begin{itemize}
\item {Grp. gram.:f.}
\end{itemize}
Qualidade de esporádico.
\section{Esporádico}
\begin{itemize}
\item {Grp. gram.:adj.}
\end{itemize}
\begin{itemize}
\item {Proveniência:(Gr. \textunderscore sporadikos\textunderscore )}
\end{itemize}
Diz-se das doenças, que, não sendo peculiares a um país, atacam accidentalmente um ou outro indivíduo, independentemente de influência epidêmica.
Disperso.
Que se produz em diversas regiões.
Casual; que acontece raramente: \textunderscore divergências esporádicas\textunderscore .
\section{Esporângio}
\begin{itemize}
\item {Grp. gram.:m.}
\end{itemize}
\begin{itemize}
\item {Proveniência:(Do gr. \textunderscore spora\textunderscore  + \textunderscore angeion\textunderscore )}
\end{itemize}
Receptáculo dos corpúsculos reproductores de muitas plantas cryptogâmicas.
\section{Esporango}
\begin{itemize}
\item {Grp. gram.:m.}
\end{itemize}
\begin{itemize}
\item {Utilização:Bot.}
\end{itemize}
\begin{itemize}
\item {Proveniência:(Do gr. \textunderscore spora\textunderscore  + \textunderscore angeion\textunderscore )}
\end{itemize}
Receptáculo dos corpúsculos reproductores de muitas plantas cryptogâmicas.
\section{Esporão}
\begin{itemize}
\item {Grp. gram.:m.}
\end{itemize}
Grande espora.
Apóphyse na parte posterior do tarso, no macho das gallináceas.
Appêndice cónico, que caracteriza certas flôres como as esporas, etc.
Contraforte de uma parede.
Dique marginal.
Parte superior da prôa de um navio, em que se póde assentar uma figura de ornato.
Espigão de metal, na prôa dos navios, para ataque ou defesa.
Doença de cereaes, o mesmo que \textunderscore cravagem\textunderscore .
\section{Esporar}
\begin{itemize}
\item {Grp. gram.:v. t.}
\end{itemize}
(V.esporear)Cf. Filinto, \textunderscore D. Man.\textunderscore , 299.
\section{Esporaúdo}
\begin{itemize}
\item {Grp. gram.:adj.}
\end{itemize}
Diz-se de certos órgãos vegetaes, que têm a fórma de esporão.
\section{Esporear}
\begin{itemize}
\item {Grp. gram.:v. t.}
\end{itemize}
\begin{itemize}
\item {Utilização:Fig.}
\end{itemize}
Ferir ou incitar com a espora.
Estimular; sacudir.
\section{Esporeira}
\begin{itemize}
\item {Grp. gram.:f.}
\end{itemize}
\begin{itemize}
\item {Proveniência:(De \textunderscore espora\textunderscore )}
\end{itemize}
Espora, (planta).
\section{Esporeiro}
\begin{itemize}
\item {Grp. gram.:m.}
\end{itemize}
Fabricante ou vendedor de esporas.
\section{Esporeto}
\begin{itemize}
\item {fónica:porê}
\end{itemize}
\begin{itemize}
\item {Grp. gram.:m.}
\end{itemize}
\begin{itemize}
\item {Proveniência:(De \textunderscore espora\textunderscore )}
\end{itemize}
Espécie de esporim, sem roseta, que serve para livrar da lama a orla inferior das calças.
\section{Esporífero}
\begin{itemize}
\item {Grp. gram.:adj.}
\end{itemize}
\begin{itemize}
\item {Proveniência:(De \textunderscore espora\textunderscore  + lat. \textunderscore ferre\textunderscore )}
\end{itemize}
Que tem esporas.
\section{Esporim}
\begin{itemize}
\item {Grp. gram.:m.}
\end{itemize}
Pequena espora.
Espécie de espora, sem roseta, para obstar a que a calçada toque no chão.
\section{Esporo}
\begin{itemize}
\item {Grp. gram.:m.}
\end{itemize}
\begin{itemize}
\item {Utilização:Bot.}
\end{itemize}
\begin{itemize}
\item {Proveniência:(Do gr. \textunderscore spora\textunderscore )}
\end{itemize}
Corpúsculo, que reproduz as plantas cryptogâmicas.
\section{Espórolo}
\begin{itemize}
\item {Grp. gram.:m.}
\end{itemize}
Pequeno esporo. Cf. \textunderscore Bibl. da G. do Campo\textunderscore , 387.
\section{Esporozoário}
\begin{itemize}
\item {Grp. gram.:m.}
\end{itemize}
\begin{itemize}
\item {Proveniência:(Do gr. \textunderscore spora\textunderscore  + \textunderscore zoon\textunderscore )}
\end{itemize}
Classe de parasitos microscópicos, a que pertence o hematozoário.
\section{Esporrinhote}
\begin{itemize}
\item {Grp. gram.:m.}
\end{itemize}
\begin{itemize}
\item {Utilização:Prov.}
\end{itemize}
\begin{itemize}
\item {Utilização:chul.}
\end{itemize}
Homem insignificante, borra-botas.
\section{Esporta}
\begin{itemize}
\item {Grp. gram.:f.}
\end{itemize}
\begin{itemize}
\item {Utilização:Prov.}
\end{itemize}
\begin{itemize}
\item {Utilização:alent.}
\end{itemize}
\begin{itemize}
\item {Proveniência:(Lat. \textunderscore sporta\textunderscore )}
\end{itemize}
Alcofa de esparto, que, rodeada de anzoes com isca, os pescadores atiram á água, deixando-a sobrenadar, á espera que o peixe toque na isca.
Alcofa; seira de esparto.
\section{Esporte}
\begin{itemize}
\item {Grp. gram.:m.}
\end{itemize}
\begin{itemize}
\item {Utilização:Neol.}
\end{itemize}
\begin{itemize}
\item {Proveniência:(Ingl. \textunderscore sport\textunderscore . Cp. port. \textunderscore desporto\textunderscore )}
\end{itemize}
Prática methódica de exercícios phýsicos.
\section{Esportela}
\begin{itemize}
\item {Grp. gram.:f.}
\end{itemize}
\begin{itemize}
\item {Proveniência:(Lat. \textunderscore sportella\textunderscore )}
\end{itemize}
Pequena esporta.
\section{Esportella}
\begin{itemize}
\item {Grp. gram.:f.}
\end{itemize}
\begin{itemize}
\item {Proveniência:(Lat. \textunderscore sportella\textunderscore )}
\end{itemize}
Pequena esporta.
\section{Espórtula}
\begin{itemize}
\item {Grp. gram.:f.}
\end{itemize}
\begin{itemize}
\item {Proveniência:(Lat. \textunderscore sportula\textunderscore )}
\end{itemize}
Gratificação pecuniária; gorgeta.
\section{Esportular}
\begin{itemize}
\item {Grp. gram.:v. t.}
\end{itemize}
\begin{itemize}
\item {Grp. gram.:V. p.}
\end{itemize}
Dar como espórtula.
Gastar.
Fazer grandes despesas.
Sêr generoso.
\section{Espós}
\begin{itemize}
\item {Grp. gram.:prep.}
\end{itemize}
\begin{itemize}
\item {Utilização:Ant.}
\end{itemize}
O mesmo que \textunderscore após\textunderscore . Cf. \textunderscore Menina e Moça\textunderscore , 133, (ed. 1852).
\section{Espôsa}
\begin{itemize}
\item {Grp. gram.:f.}
\end{itemize}
\begin{itemize}
\item {Proveniência:(Do lat. \textunderscore sponsa\textunderscore )}
\end{itemize}
Mulher, que está prometida para casamento ou que tem ajustado o seu casamento.
Mulher casada, em relação a seu marido.
A Igreja e a freira, em relação a Christo, no sentido mýstico.
\section{Esposar}
\begin{itemize}
\item {Grp. gram.:v. t.}
\end{itemize}
\begin{itemize}
\item {Grp. gram.:V. p.}
\end{itemize}
\begin{itemize}
\item {Proveniência:(De \textunderscore espôso\textunderscore  e \textunderscore espôsa\textunderscore )}
\end{itemize}
Unir em casamento.
Receber por espôso ou por espôsa.
Sêr amparo ou sustentâculo de (vegetaes que trepam ou se enroscam).
Tomar a seu cuidado.
Preconizar e defender (uma causa, um partido, certos princípios ou systemas).
Casar-se. Cf. Camillo, \textunderscore Caveira\textunderscore , 216.
\section{Espôso}
\begin{itemize}
\item {Grp. gram.:m.}
\end{itemize}
\begin{itemize}
\item {Utilização:Ext.}
\end{itemize}
\begin{itemize}
\item {Proveniência:(Do lat. \textunderscore sponsus\textunderscore )}
\end{itemize}
Aquelle que prometeu casar.
Aquelle que está para casar.
Marido.
\section{Esposoiro}
\begin{itemize}
\item {Grp. gram.:m.}
\end{itemize}
\begin{itemize}
\item {Utilização:Ant.}
\end{itemize}
O mesmo que \textunderscore esposório\textunderscore .
\section{Esposório}
\begin{itemize}
\item {Grp. gram.:m.}
\end{itemize}
\begin{itemize}
\item {Utilização:Ant.}
\end{itemize}
\begin{itemize}
\item {Proveniência:(De \textunderscore esposar\textunderscore )}
\end{itemize}
Esponsaes; festa de casamento.
Presente de núpcias.
\section{Espostejar}
\begin{itemize}
\item {Grp. gram.:v. t.}
\end{itemize}
Fazer em postas.
Esquartejar; esbandalhar.
\section{Espraiado}
\begin{itemize}
\item {Grp. gram.:m.}
\end{itemize}
\begin{itemize}
\item {Proveniência:(De \textunderscore espraiar\textunderscore )}
\end{itemize}
Espaço, que a maré descobre quando vaza.
\section{Espraiamento}
\begin{itemize}
\item {Grp. gram.:m.}
\end{itemize}
Acto de \textunderscore espraiar\textunderscore .
\section{Espraiar}
\begin{itemize}
\item {Grp. gram.:v. t.}
\end{itemize}
\begin{itemize}
\item {Grp. gram.:V. i.}
\end{itemize}
\begin{itemize}
\item {Grp. gram.:V. p.}
\end{itemize}
Lançar á praia.
Alargar; estender: \textunderscore espraiar a vista\textunderscore .
Deixar descoberta a praia, refluindo: \textunderscore as ondas espraiam\textunderscore .
Estender-se pela praia, pelas margens, (falando-se de mar ou rio).
Expandir-se; propagar-se.
Divagar sôbre um assumpto.
\section{Esprândigo}
\begin{itemize}
\item {Grp. gram.:adj.}
\end{itemize}
\begin{itemize}
\item {Utilização:Prov.}
\end{itemize}
\begin{itemize}
\item {Utilização:trasm.}
\end{itemize}
Amplo.
Desafogado. (Colhido em Lagoaça)
(Relaciona-se com \textunderscore pando\textunderscore ?)
\section{Esprangalhar}
\begin{itemize}
\item {Grp. gram.:V. p.}
\end{itemize}
\begin{itemize}
\item {Utilização:Prov.}
\end{itemize}
\begin{itemize}
\item {Utilização:trasm.}
\end{itemize}
O mesmo que \textunderscore esfrangalhar\textunderscore .
\section{Espratela}
\begin{itemize}
\item {Grp. gram.:f.}
\end{itemize}
Gênero de peixe malacopterýgios.
\section{Espravão}
\begin{itemize}
\item {Grp. gram.:m.}
\end{itemize}
O mesmo que \textunderscore esparvão\textunderscore . Cf. Jazente, II, 61.
\section{Espreguiçadeira}
\begin{itemize}
\item {Grp. gram.:f.}
\end{itemize}
O mesmo que \textunderscore espreguiçador\textunderscore .
\section{Espreguiçadoiro}
\begin{itemize}
\item {Grp. gram.:m.}
\end{itemize}
O mesmo que \textunderscore preguiceira\textunderscore . Cf. Camillo, \textunderscore Caveira\textunderscore , 494.
\section{Espreguiçadela}
\begin{itemize}
\item {Grp. gram.:f.}
\end{itemize}
\begin{itemize}
\item {Utilização:Fam.}
\end{itemize}
Acto de espreguiçar.
\section{Espreguiçador}
\begin{itemize}
\item {Grp. gram.:m.}
\end{itemize}
\begin{itemize}
\item {Proveniência:(De \textunderscore espreguiçar\textunderscore )}
\end{itemize}
Móvel, em que alguém se póde deitar, para descansar ou dormir a sésta.
Preguiceiro.
\section{Espreguiçamento}
\begin{itemize}
\item {Grp. gram.:m.}
\end{itemize}
Acto de espreguiçar.
\section{Espreguiçar}
\begin{itemize}
\item {Grp. gram.:v. t.}
\end{itemize}
\begin{itemize}
\item {Grp. gram.:V. i.  e  p.}
\end{itemize}
\begin{itemize}
\item {Utilização:Fig.}
\end{itemize}
Tirar a preguiça a.
Estirar os braços ou as pernas, por indolência ou somno.
Espraiar-se, alastrar-se.
\section{Espreguiceiro}
\begin{itemize}
\item {Grp. gram.:m.}
\end{itemize}
O mesmo que \textunderscore espreguiçador\textunderscore .
\section{Espreguntar}
\begin{itemize}
\item {Grp. gram.:v. t.}
\end{itemize}
\begin{itemize}
\item {Utilização:Ant.}
\end{itemize}
O mesmo que \textunderscore preguntar\textunderscore .
\section{Espreita}
\begin{itemize}
\item {Grp. gram.:f.}
\end{itemize}
\begin{itemize}
\item {Utilização:Ant.}
\end{itemize}
\begin{itemize}
\item {Grp. gram.:Loc. adv.}
\end{itemize}
Acto de espreitar.
Pesquisa.
Cilada, ardil, embuste.
\textunderscore Á espreita\textunderscore , debaixo de ôlho; de atalaia.
\section{Espreitadeira}
\begin{itemize}
\item {Grp. gram.:adj. f.}
\end{itemize}
\begin{itemize}
\item {Grp. gram.:F.}
\end{itemize}
\begin{itemize}
\item {Proveniência:(De \textunderscore espreitar\textunderscore )}
\end{itemize}
Diz-se da mulher que espreita, que é curiosa.
Abertura, por onde se espreita. Cf. Arn. Gama, \textunderscore Motim\textunderscore , 185.
\section{Espreitadela}
\begin{itemize}
\item {Grp. gram.:f.}
\end{itemize}
\begin{itemize}
\item {Utilização:Fam.}
\end{itemize}
Acto de espreitar. Cf. Camillo, \textunderscore Estrel. Prop.\textunderscore , 98.
\section{Espreitador}
\begin{itemize}
\item {Grp. gram.:adj.}
\end{itemize}
\begin{itemize}
\item {Grp. gram.:M.}
\end{itemize}
Que espreita.
Aquelle que espreita.
\section{Espreita-marés}
\begin{itemize}
\item {Grp. gram.:m.}
\end{itemize}
O mesmo que \textunderscore pica-peixe\textunderscore .
\section{Espreitança}
\begin{itemize}
\item {Grp. gram.:f.}
\end{itemize}
\begin{itemize}
\item {Utilização:Ant.}
\end{itemize}
O mesmo que \textunderscore espreita\textunderscore .
\section{Espreitante}
\begin{itemize}
\item {Grp. gram.:m.  e  adj.}
\end{itemize}
O que espreita.
\section{Espreitar}
\begin{itemize}
\item {Grp. gram.:v. t.}
\end{itemize}
\begin{itemize}
\item {Proveniência:(Do lat. hyp. \textunderscore explicitare\textunderscore )}
\end{itemize}
Observar occultamente (os actos de outrem, os movimentos de um animal, etc.).
Perscrutar.
Indagar.
Espiar.
Olhar com attenção, esperando.
\section{Espreiteiro}
\begin{itemize}
\item {Grp. gram.:m.  e  adj.}
\end{itemize}
O que espreita. Cf. Filinto, IV, 114.
\section{Esprém}
\begin{itemize}
\item {Grp. gram.:m.}
\end{itemize}
\begin{itemize}
\item {Utilização:Prov.}
\end{itemize}
\begin{itemize}
\item {Utilização:trasm.}
\end{itemize}
Desamparo cruel: \textunderscore morrer ao esprém\textunderscore .
\section{Espremedeira}
\begin{itemize}
\item {Grp. gram.:f.}
\end{itemize}
\begin{itemize}
\item {Utilização:Prov.}
\end{itemize}
\begin{itemize}
\item {Utilização:minh.}
\end{itemize}
\begin{itemize}
\item {Proveniência:(De \textunderscore espremer\textunderscore )}
\end{itemize}
Cada um dos pedaes do tear.
\section{Espremedor}
\begin{itemize}
\item {Grp. gram.:adj.}
\end{itemize}
\begin{itemize}
\item {Grp. gram.:M.}
\end{itemize}
Que espreme.
Aquelle que espreme.
\section{Espremedura}
\begin{itemize}
\item {Grp. gram.:f.}
\end{itemize}
Acto de espremer.
\section{Espremegado}
\begin{itemize}
\item {Grp. gram.:m.}
\end{itemize}
O mesmo que \textunderscore esparregado\textunderscore . Cf. B. C. Rubim, \textunderscore Vocabulário Bras.\textunderscore , vb. \textunderscore chuchá\textunderscore .
(Infl. de \textunderscore espremer\textunderscore )
\section{Espremer}
\begin{itemize}
\item {Grp. gram.:v. t.}
\end{itemize}
\begin{itemize}
\item {Utilização:Des.}
\end{itemize}
\begin{itemize}
\item {Utilização:Fig.}
\end{itemize}
\begin{itemize}
\item {Proveniência:(Do lat. \textunderscore premere\textunderscore )}
\end{itemize}
Comprimir ou apertar, para extrahir um líquido.
Lançar de si.
Apoquentar.
Opprimir; obrigar.
\section{Espremível}
\begin{itemize}
\item {Grp. gram.:adj.}
\end{itemize}
Que se póde espremer.
\section{Esprever}
\begin{itemize}
\item {Grp. gram.:v. t.}
\end{itemize}
\begin{itemize}
\item {Utilização:Ant.}
\end{itemize}
O mesmo que \textunderscore escrever\textunderscore .
\section{Espritado}
\begin{itemize}
\item {Grp. gram.:adj.}
\end{itemize}
\begin{itemize}
\item {Utilização:Bras. do N}
\end{itemize}
O mesmo que \textunderscore hydróphobo\textunderscore : \textunderscore cão espritado\textunderscore .
(Por \textunderscore espiritado\textunderscore , de \textunderscore espiríto\textunderscore )
\section{Esprital}
\begin{itemize}
\item {Grp. gram.:m.}
\end{itemize}
\begin{itemize}
\item {Utilização:Ant.}
\end{itemize}
O mesmo que \textunderscore hospital\textunderscore . Cf. G. Vicente, \textunderscore M. Parda\textunderscore .
\section{Esprito}
\begin{itemize}
\item {Grp. gram.:m.}
\end{itemize}
O mesmo que \textunderscore espírito\textunderscore . Cf. \textunderscore Filodemo\textunderscore , II, 5 e III, 1; Usque, (\textunderscore passim\textunderscore ).
\section{Esprovamento}
\begin{itemize}
\item {Grp. gram.:m.}
\end{itemize}
\begin{itemize}
\item {Utilização:Ant.}
\end{itemize}
O mesmo que \textunderscore experiência\textunderscore . Cf. Frei Fortun., \textunderscore Inéd.\textunderscore , I, 307.
\section{Espuição}
\begin{itemize}
\item {fónica:pu-i}
\end{itemize}
\begin{itemize}
\item {Grp. gram.:f.}
\end{itemize}
Acto ou effeito de espuir.
\section{Espuir}
\begin{itemize}
\item {Grp. gram.:v. t.  e  i.}
\end{itemize}
\begin{itemize}
\item {Proveniência:(Lat. \textunderscore spuere\textunderscore )}
\end{itemize}
Cuspir; expectorar.
\section{Espulgar}
\begin{itemize}
\item {Grp. gram.:v. t.}
\end{itemize}
Tirar as pulgas a.
\section{Espulgatório}
\begin{itemize}
\item {Grp. gram.:m.}
\end{itemize}
\begin{itemize}
\item {Proveniência:(De \textunderscore espulgar\textunderscore )}
\end{itemize}
Compartimento, de pavimento esburacado, em que os frades de alguns conventos, como o de Mafra, iam, antes de se deitar, sacudir o fato, para se livrarem das pulgas.
\section{Espuma}
\begin{itemize}
\item {Grp. gram.:f.}
\end{itemize}
\begin{itemize}
\item {Proveniência:(Lat. \textunderscore spuma\textunderscore )}
\end{itemize}
O mesmo que \textunderscore escuma\textunderscore .
Saliva escumosa, que fórma grandes bolhas entre os dentes, ou entre os lábios, ou na garganta, durante certos incômmodos nervosos.
\section{Espumante}
\begin{itemize}
\item {Grp. gram.:adj.}
\end{itemize}
Que espuma.
Excitado.
Raivoso.
\section{Espumar}
\begin{itemize}
\item {Grp. gram.:v. i.}
\end{itemize}
\begin{itemize}
\item {Grp. gram.:V. t.}
\end{itemize}
\begin{itemize}
\item {Proveniência:(De \textunderscore espuma\textunderscore )}
\end{itemize}
Tirar a escuma a; escumar.
Deitar espuma; espumejar.
\section{Espumas}
\begin{itemize}
\item {Grp. gram.:f. pl.}
\end{itemize}
Variedade de doce.
O mesmo que \textunderscore farófia\textunderscore .
(Pl. de \textunderscore espuma\textunderscore )
\section{Espumejante}
\begin{itemize}
\item {Grp. gram.:adj.}
\end{itemize}
Que espumeja. Cf. Camillo, \textunderscore Narcót.\textunderscore  I, 217.
\section{Espumejar}
\begin{itemize}
\item {Grp. gram.:v. i.}
\end{itemize}
Lançar espuma.
Irar-se.
Escumar com raiva. Cf. Camillo, \textunderscore Cancion. Al.\textunderscore , 13.
\section{Espúmeo}
\begin{itemize}
\item {Grp. gram.:adj.}
\end{itemize}
\begin{itemize}
\item {Proveniência:(Lat. \textunderscore spumifer\textunderscore )}
\end{itemize}
Que traz espuma.
\section{Espumífero}
\begin{itemize}
\item {Grp. gram.:adj.}
\end{itemize}
\begin{itemize}
\item {Proveniência:(Lat. \textunderscore spumifer\textunderscore )}
\end{itemize}
Que traz espuma.
\section{Espumígero}
\begin{itemize}
\item {Grp. gram.:adj.}
\end{itemize}
\begin{itemize}
\item {Proveniência:(Lat. \textunderscore spumiger\textunderscore )}
\end{itemize}
O mesmo que \textunderscore espumoso\textunderscore .
\section{Espumosidade}
\begin{itemize}
\item {Grp. gram.:f.}
\end{itemize}
Qualidade daquillo que é \textunderscore espumoso\textunderscore .
\section{Espumoso}
\begin{itemize}
\item {Grp. gram.:adj.}
\end{itemize}
\begin{itemize}
\item {Proveniência:(Lat. \textunderscore spumosus\textunderscore )}
\end{itemize}
Que tem espuma, que deita espuma.
\section{Espurcícia}
\begin{itemize}
\item {Grp. gram.:f.}
\end{itemize}
\begin{itemize}
\item {Utilização:Des.}
\end{itemize}
\begin{itemize}
\item {Proveniência:(Lat. \textunderscore spurcitia\textunderscore )}
\end{itemize}
Immundíce.
Impureza.
\section{Espurco}
\begin{itemize}
\item {Grp. gram.:adj.}
\end{itemize}
\begin{itemize}
\item {Utilização:Des.}
\end{itemize}
\begin{itemize}
\item {Proveniência:(Lat. \textunderscore spurcus\textunderscore )}
\end{itemize}
Porco.
Sórdido; immundo.
\section{Espurgar}
\textunderscore v. t.\textunderscore  (e der.)
O mesmo que \textunderscore expurgar\textunderscore , etc.
\section{Espuriedade}
\begin{itemize}
\item {Grp. gram.:f.}
\end{itemize}
Qualidade daquelle ou daquillo que é \textunderscore espúrio\textunderscore .
\section{Espúrio}
\begin{itemize}
\item {Grp. gram.:adj.}
\end{itemize}
\begin{itemize}
\item {Utilização:Ext.}
\end{itemize}
\begin{itemize}
\item {Utilização:Fig.}
\end{itemize}
\begin{itemize}
\item {Utilização:Des.}
\end{itemize}
\begin{itemize}
\item {Utilização:Prov.}
\end{itemize}
\begin{itemize}
\item {Utilização:trasm.}
\end{itemize}
\begin{itemize}
\item {Proveniência:(Lat. \textunderscore spurius\textunderscore )}
\end{itemize}
Diz-se dos filhos, que não são legítimos, nem podem sêr perfilhados.
Bastardo.
Adulterado.
Falsificado.
Estranho á bôa linguagem: \textunderscore phraseado espúrio\textunderscore .
Despojado.
Somítico, avaro e de cara pouco agradável.
\section{Esputação}
\begin{itemize}
\item {Grp. gram.:f.}
\end{itemize}
Acto ou effeito de esputar.
\section{Esputar}
\begin{itemize}
\item {Grp. gram.:v. i.}
\end{itemize}
\begin{itemize}
\item {Proveniência:(Lat. \textunderscore sputare\textunderscore )}
\end{itemize}
Salivar frequentemente.
\section{Esputo}
\begin{itemize}
\item {Grp. gram.:m.}
\end{itemize}
\begin{itemize}
\item {Proveniência:(Lat. \textunderscore sputus\textunderscore )}
\end{itemize}
Saliva; acto de esputar.
\section{Esquadra}
\begin{itemize}
\item {Grp. gram.:f.}
\end{itemize}
\begin{itemize}
\item {Proveniência:(It. \textunderscore squadra\textunderscore )}
\end{itemize}
Certa porção de navios de guerra, pertencentes a uma armada e commandada por um official superior.
Secção de uma companhia de infantaria.
Pôsto policial.
Instrumento, com que os artilheiros graduam a elevação dos tiros.
Esquadro.
\section{Esquadrão}
\begin{itemize}
\item {Grp. gram.:m.}
\end{itemize}
\begin{itemize}
\item {Utilização:Fig.}
\end{itemize}
Conjunto de companhias num exército.
Multidão.
(Cast. \textunderscore escuadrón\textunderscore )
\section{Esquadrar}
\begin{itemize}
\item {Grp. gram.:v. t.}
\end{itemize}
\begin{itemize}
\item {Proveniência:(De \textunderscore esquadro\textunderscore )}
\end{itemize}
Dispor ou cortar em ângulo recto.
Dispor em esquadrão.
\section{Esquadreiro}
\begin{itemize}
\item {Grp. gram.:m.}
\end{itemize}
\begin{itemize}
\item {Utilização:Prov.}
\end{itemize}
\begin{itemize}
\item {Utilização:alent.}
\end{itemize}
Aquelle que limpa a quadra ou cavallariça.
\section{Esquadrejamento}
\begin{itemize}
\item {Grp. gram.:m.}
\end{itemize}
Acto ou effeito de esquadrejar.
\section{Esquadrejar}
\begin{itemize}
\item {Grp. gram.:v. t.}
\end{itemize}
\begin{itemize}
\item {Proveniência:(De \textunderscore esquadro\textunderscore )}
\end{itemize}
Serrar ou cortar em esquadria.
\section{Esquadria}
\begin{itemize}
\item {Grp. gram.:f.}
\end{itemize}
\begin{itemize}
\item {Utilização:Fig.}
\end{itemize}
\begin{itemize}
\item {Proveniência:(De \textunderscore esquadro\textunderscore )}
\end{itemize}
Córte em ângulo recto.
Ângulo recto.
Instrumento de madeira ou metal, com que se traçam ou medem ângulos rectos.
Esquadro.
Pedra de cantaria.
Bom méthodo, regularidade.
\section{Esquadriar}
\begin{itemize}
\item {Grp. gram.:v. t.}
\end{itemize}
\begin{itemize}
\item {Proveniência:(De \textunderscore esquadria\textunderscore )}
\end{itemize}
O mesmo que \textunderscore esquadrar\textunderscore .
\section{Esquadrilha}
\begin{itemize}
\item {Grp. gram.:f.}
\end{itemize}
\begin{itemize}
\item {Proveniência:(De \textunderscore esquadra\textunderscore )}
\end{itemize}
Esquadra de pequenos navios de guerra.
\section{Esquadrilhado}
\begin{itemize}
\item {Grp. gram.:adj.}
\end{itemize}
\begin{itemize}
\item {Proveniência:(De \textunderscore esquadrilhar\textunderscore ^2)}
\end{itemize}
Que tem quadris baixos; desnalgado. Cf. Camillo, \textunderscore Vinho do Porto\textunderscore , 80.
\section{Esquadrilhar}
\begin{itemize}
\item {Grp. gram.:v. t.}
\end{itemize}
Expulsar da quadrilha.
\section{Esquadrilhar}
\begin{itemize}
\item {Grp. gram.:v. t.}
\end{itemize}
\begin{itemize}
\item {Proveniência:(De \textunderscore quadril\textunderscore )}
\end{itemize}
Partir os quadris a.
Desancar.
\section{Esquadrinhador}
\begin{itemize}
\item {Grp. gram.:m.  e  adj.}
\end{itemize}
O que esquadrinha.
\section{Esquadrinhadura}
\begin{itemize}
\item {Grp. gram.:f.}
\end{itemize}
Acto de esquadrinhar.
\section{Esquadrinhamento}
\begin{itemize}
\item {Grp. gram.:m.}
\end{itemize}
Acto de esquadrinhar.
\section{Esquadrinhar}
\begin{itemize}
\item {Grp. gram.:v. t.}
\end{itemize}
\begin{itemize}
\item {Proveniência:(Do lat. \textunderscore scrutinare\textunderscore ?)}
\end{itemize}
Perscrutar.
Investigar com cuidado, minuciosamente.
\section{Esquadro}
\begin{itemize}
\item {Grp. gram.:m.}
\end{itemize}
\begin{itemize}
\item {Proveniência:(It. \textunderscore squadro\textunderscore )}
\end{itemize}
Instrumento, com que se formam ou medem ângulos rectos, e se tiram linhas perpendiculares.
\section{Esquadronar}
\begin{itemize}
\item {Grp. gram.:v. t.}
\end{itemize}
\begin{itemize}
\item {Grp. gram.:V. i.}
\end{itemize}
\begin{itemize}
\item {Utilização:Ant.}
\end{itemize}
Dispor em esquadrão.
Formar esquadrão.
\textunderscore Arte de esquadronar\textunderscore , o mesmo que \textunderscore táctica\textunderscore . Cf. C. Ayres, \textunderscore Hist. do Ex. Port.\textunderscore 
\section{Esqualidez}
\begin{itemize}
\item {Grp. gram.:f.}
\end{itemize}
Qualidade daquelle ou daquillo que é esquálido.
\section{Esquálido}
\begin{itemize}
\item {Grp. gram.:adj.}
\end{itemize}
\begin{itemize}
\item {Proveniência:(Lat. \textunderscore squalidus\textunderscore )}
\end{itemize}
Sórdido.
Sujo.
Desalinhado:«\textunderscore a barba esquálida\textunderscore ». \textunderscore Lusíadas\textunderscore .
Carrancudo.
\section{Esqualo}
\begin{itemize}
\item {Grp. gram.:m.}
\end{itemize}
\begin{itemize}
\item {Proveniência:(Lat. \textunderscore squalus\textunderscore )}
\end{itemize}
Gênero de peixes, de esqueleto cartilaginoso, corpo alongado e pelle rugosa.
\section{Esqualor}
\begin{itemize}
\item {Grp. gram.:m.}
\end{itemize}
\begin{itemize}
\item {Proveniência:(Lat. \textunderscore squalor\textunderscore )}
\end{itemize}
O mesmo que \textunderscore esqualidez\textunderscore .
\section{Esquamodermes}
\begin{itemize}
\item {Grp. gram.:m. pl.}
\end{itemize}
\begin{itemize}
\item {Proveniência:(Do lat. \textunderscore squama\textunderscore  + gr. \textunderscore derma\textunderscore )}
\end{itemize}
Peixes, de raios duros e espiniformes nas barbatanas.
Peixes, de esqueleto ósseo, mas de barbatanas flexíveis.
\section{Esquamodermos}
\begin{itemize}
\item {Grp. gram.:m. pl.}
\end{itemize}
\begin{itemize}
\item {Proveniência:(Do lat. \textunderscore squama\textunderscore  + gr. \textunderscore derma\textunderscore )}
\end{itemize}
Peixes, de raios duros e espiniformes nas barbatanas.
Peixes, de esqueleto ósseo, mas de barbatanas flexíveis.
\section{Esquânia}
\begin{itemize}
\item {Grp. gram.:f.}
\end{itemize}
Gênero de plantas malpigiáceas.
\section{Esquarroso}
\begin{itemize}
\item {Grp. gram.:adj.}
\end{itemize}
\begin{itemize}
\item {Utilização:Des.}
\end{itemize}
\begin{itemize}
\item {Proveniência:(Lat. \textunderscore squarrosus\textunderscore )}
\end{itemize}
Aspero; que tem muitas escamas.
\section{Esquartejadoiro}
\begin{itemize}
\item {Grp. gram.:m.}
\end{itemize}
\begin{itemize}
\item {Proveniência:(De \textunderscore esquartejar\textunderscore )}
\end{itemize}
Lugar, onde se retalham os cadáveres dos animaes, para se aproveitarem nas fábricas de guano.
Fábrica de guano.
\section{Esquartejadouro}
\begin{itemize}
\item {Grp. gram.:m.}
\end{itemize}
\begin{itemize}
\item {Proveniência:(De \textunderscore esquartejar\textunderscore )}
\end{itemize}
Lugar, onde se retalham os cadáveres dos animaes, para se aproveitarem nas fábricas de guano.
Fábrica de guano.
\section{Esquartejadura}
\begin{itemize}
\item {Grp. gram.:f.}
\end{itemize}
Acto de esquartejar (reses no matadoiro).
\section{Esquartejamento}
\begin{itemize}
\item {Grp. gram.:m.}
\end{itemize}
Acto de esquartejar.
\section{Esquartejar}
\begin{itemize}
\item {Grp. gram.:v. t.}
\end{itemize}
\begin{itemize}
\item {Utilização:Fig.}
\end{itemize}
\begin{itemize}
\item {Proveniência:(De \textunderscore quarto\textunderscore )}
\end{itemize}
Partir em quartos; espostejar.
Despedaçar.
Infamar.
Desbaratar.
\section{Esquarteladura}
\begin{itemize}
\item {Grp. gram.:f.}
\end{itemize}
Effeito de esquartelar.
\section{Esquartelar}
\begin{itemize}
\item {Grp. gram.:v.}
\end{itemize}
\begin{itemize}
\item {Utilização:t. Heráld.}
\end{itemize}
\begin{itemize}
\item {Proveniência:(De \textunderscore quartel\textunderscore )}
\end{itemize}
Dividir em quatro partes ou quartéis (o escudo).
\section{Esquartilhar}
\begin{itemize}
\item {Grp. gram.:v. t.}
\end{itemize}
\begin{itemize}
\item {Utilização:Prov.}
\end{itemize}
\begin{itemize}
\item {Utilização:trasm.}
\end{itemize}
Retalhar longitudinalmente (azeitonas, antes de se curtirem).
(Cp. \textunderscore esquartejar\textunderscore )
\section{Esquatina}
\begin{itemize}
\item {Grp. gram.:f.}
\end{itemize}
Gênero de esqualos, também conhecido por \textunderscore anjo-do-mar\textunderscore  ou \textunderscore lixa\textunderscore .
\section{Esquecediço}
\begin{itemize}
\item {fónica:qué}
\end{itemize}
\begin{itemize}
\item {Grp. gram.:adj.}
\end{itemize}
Que se esquece facilmente.
\section{Esquecedor}
\begin{itemize}
\item {fónica:qué}
\end{itemize}
\begin{itemize}
\item {Grp. gram.:m.  e  adj.}
\end{itemize}
\begin{itemize}
\item {Proveniência:(De \textunderscore esquecer\textunderscore )}
\end{itemize}
O que produz esquecimento.
\section{Esquecer}
\begin{itemize}
\item {fónica:qué}
\end{itemize}
\begin{itemize}
\item {Grp. gram.:v. t.}
\end{itemize}
\begin{itemize}
\item {Grp. gram.:V. i.}
\end{itemize}
\begin{itemize}
\item {Grp. gram.:V. p.}
\end{itemize}
\begin{itemize}
\item {Proveniência:(Do lat. \textunderscore ex\textunderscore  + \textunderscore cadescere\textunderscore )}
\end{itemize}
Deixar saír da lembrança; pôr de lado.
Desprezar.
Largar.--Como v. t., reputa-se escusado gallicismo.
Saír da lembrança: \textunderscore esqueceu-me o teu nome\textunderscore .
Omittir-se.
Passar despercebido.
Perder a sensibilidade: \textunderscore esqueciam-lhe as pernas\textunderscore .
Perder a lembrança.
Deixar saír da memória conhecimentos adquiridos: \textunderscore esquecer-se da lição\textunderscore .
Deixar de attender; descuidar-se: \textunderscore esqueci-me do negócio\textunderscore .
\section{Esquecidiço}
\begin{itemize}
\item {Grp. gram.:adj.}
\end{itemize}
(V.esquecediço)
\section{Esquecido}
\begin{itemize}
\item {fónica:qué}
\end{itemize}
\begin{itemize}
\item {Grp. gram.:adj.}
\end{itemize}
\begin{itemize}
\item {Proveniência:(De \textunderscore esquecer\textunderscore )}
\end{itemize}
Que se esqueceu.
Que saiu da memória.
\section{Esquecidos}
\begin{itemize}
\item {Grp. gram.:m. pl.}
\end{itemize}
\begin{itemize}
\item {Utilização:Pop.}
\end{itemize}
Pequenos bolos, semelhantes ao especione.
\section{Esquecimento}
\begin{itemize}
\item {fónica:qué}
\end{itemize}
\begin{itemize}
\item {Grp. gram.:m.}
\end{itemize}
Acto ou effeito de esquecer.
Falta de lembrança.
Omissão.
Entorpecimento de qualquer parte do corpo.
\section{Esquefe}
\begin{itemize}
\item {Grp. gram.:m.}
\end{itemize}
\begin{itemize}
\item {Utilização:Gír. lisb.}
\end{itemize}
Blennorrhagia.
(Cp. \textunderscore esquentamento\textunderscore )
\section{Esqueiro}
\begin{itemize}
\item {Grp. gram.:m.}
\end{itemize}
\begin{itemize}
\item {Utilização:Prov.}
\end{itemize}
\begin{itemize}
\item {Utilização:minh.}
\end{itemize}
Pequena escada de mão.
\section{Esqueletal}
\begin{itemize}
\item {Grp. gram.:adj.}
\end{itemize}
\begin{itemize}
\item {Utilização:Des.}
\end{itemize}
Relativo a esqueleto.
Magro, descarnado:«\textunderscore ...a perna esqueletal cruzou...\textunderscore »Macedo, \textunderscore Burros\textunderscore , 28.
\section{Esquelético}
\begin{itemize}
\item {Grp. gram.:adj.}
\end{itemize}
\begin{itemize}
\item {Utilização:Fig.}
\end{itemize}
Relativo a esqueleto.
Próprio de esqueleto.
Extremamente magro.
\section{Esqueleto}
\begin{itemize}
\item {fónica:lê}
\end{itemize}
\begin{itemize}
\item {Grp. gram.:m.}
\end{itemize}
\begin{itemize}
\item {Utilização:Fig.}
\end{itemize}
\begin{itemize}
\item {Proveniência:(Gr. \textunderscore skeletos\textunderscore )}
\end{itemize}
Conjunto de ossos de um animal descarnado, mas em posição natural.
Madeiramento de uma casa, antes da formação das paredes.
Delineamento, esbôço.
Pessôa muito magra.
\section{Esquema}
\begin{itemize}
\item {Grp. gram.:m.}
\end{itemize}
\begin{itemize}
\item {Utilização:Rhet.}
\end{itemize}
\begin{itemize}
\item {Utilização:Med.}
\end{itemize}
\begin{itemize}
\item {Proveniência:(Lat. \textunderscore schema\textunderscore )}
\end{itemize}
Designação genérica de todas as fórmas de ornato, no estilo.
Representação da disposição geral de um aparelho orgânico ou da marcha de um fenómeno, abstraindo-se de certas particularidades, que impedíriam de abranger rapidamente as noções que se pretendem.
Representação das funções e relações de um objecto, independentemente da sua verdadeira fórma.
Proposta, submetida á deliberação de um concilio.
\section{Esquematicamente}
\begin{itemize}
\item {Grp. gram.:adv.}
\end{itemize}
De modo esquemático; á maneira de esquema.
\section{Esquemático}
\begin{itemize}
\item {Grp. gram.:adj.}
\end{itemize}
Relativo a esquema.
\section{Esquematismo}
\begin{itemize}
\item {Grp. gram.:m.}
\end{itemize}
Sistema dos que amiúde formulam esquemas.
\section{Esquença}
\begin{itemize}
\item {Grp. gram.:f.}
\end{itemize}
\begin{itemize}
\item {Utilização:Ant.}
\end{itemize}
\begin{itemize}
\item {Proveniência:(Do lat. hyp. \textunderscore excadentia\textunderscore )}
\end{itemize}
Bôa sorte, fortuna.
\section{Esquençado}
\begin{itemize}
\item {Grp. gram.:adj.}
\end{itemize}
\begin{itemize}
\item {Utilização:Ant.}
\end{itemize}
\begin{itemize}
\item {Proveniência:(De \textunderscore esquença\textunderscore )}
\end{itemize}
Venturoso, afortunado.
\section{Esquentação}
\begin{itemize}
\item {Grp. gram.:f.}
\end{itemize}
\begin{itemize}
\item {Utilização:Fig.}
\end{itemize}
Acto ou effeito de esquentar.
Calor intenso.
Rixa ou discussão azeda.
Inflammação nos pés dos animaes, resultante da falta de limpeza.
Espécie de blennorrhagia.
\section{Esquentada}
\begin{itemize}
\item {Grp. gram.:f.}
\end{itemize}
\begin{itemize}
\item {Proveniência:(De \textunderscore esquentado\textunderscore )}
\end{itemize}
A hora de maior calor.
\section{Esquentadiço}
\begin{itemize}
\item {Grp. gram.:adj.}
\end{itemize}
\begin{itemize}
\item {Utilização:Fig.}
\end{itemize}
\begin{itemize}
\item {Proveniência:(De \textunderscore esquentar\textunderscore )}
\end{itemize}
Que se irrita facilmente.
\section{Esquentado}
\begin{itemize}
\item {Grp. gram.:adj.}
\end{itemize}
Quente.
Exaltado; irritado.
\section{Esquentador}
\begin{itemize}
\item {Grp. gram.:m.}
\end{itemize}
\begin{itemize}
\item {Grp. gram.:Adj.}
\end{itemize}
\begin{itemize}
\item {Proveniência:(De \textunderscore esquentar\textunderscore )}
\end{itemize}
Utensílio de metal, com que se aquece a cama.
Apparelho, que, com o vapor de escape das máquinas, aquece a água de alimentação das caldeiras.
Que esquenta, que produz grande calor.
\section{Esquentamento}
\begin{itemize}
\item {Grp. gram.:m.}
\end{itemize}
\begin{itemize}
\item {Utilização:Pop.}
\end{itemize}
\begin{itemize}
\item {Proveniência:(De \textunderscore esquentar\textunderscore )}
\end{itemize}
Esquentação.
Gonorrheia.
\section{Esquentar}
\begin{itemize}
\item {Grp. gram.:v. t.}
\end{itemize}
\begin{itemize}
\item {Utilização:Fig.}
\end{itemize}
\begin{itemize}
\item {Proveniência:(Do lat. hyp. \textunderscore excalentare\textunderscore )}
\end{itemize}
Tornar quente.
Causar calor a.
Irritar.
\section{Esquentária}
\begin{itemize}
\item {Grp. gram.:f.}
\end{itemize}
\begin{itemize}
\item {Utilização:sertanejo}
\end{itemize}
\begin{itemize}
\item {Utilização:Bras}
\end{itemize}
O mesmo que \textunderscore canthárida\textunderscore .
\section{Esquerda}
\begin{itemize}
\item {fónica:quêr}
\end{itemize}
\begin{itemize}
\item {Grp. gram.:f.}
\end{itemize}
\begin{itemize}
\item {Grp. gram.:Interj.}
\end{itemize}
\begin{itemize}
\item {Proveniência:(De \textunderscore esquerdo\textunderscore )}
\end{itemize}
Lado esquerdo.
Parte de uma assembleia, que fica á direita do presidente.
Opposição parlamentar.
Grupo parlamentar, que tem assento á esquerda do presidente da respectiva assembleia.
Voz de commando, para se fazer executar o movimento para o lado esquerdo.
\section{Esquerdear}
\begin{itemize}
\item {Grp. gram.:v. t.}
\end{itemize}
\begin{itemize}
\item {Utilização:Des.}
\end{itemize}
\begin{itemize}
\item {Grp. gram.:V. i.}
\end{itemize}
\begin{itemize}
\item {Proveniência:(De \textunderscore esquerdo\textunderscore )}
\end{itemize}
Tornar esquerdo.
Fazer voltar para o lado esquerdo.
Desviar-se do dever.
Seguir mau rumo.
\section{Esquerdecer}
\begin{itemize}
\item {Grp. gram.:v. i.}
\end{itemize}
\begin{itemize}
\item {Utilização:Ant.}
\end{itemize}
\begin{itemize}
\item {Proveniência:(De \textunderscore esquerdo\textunderscore )}
\end{itemize}
Proceder mal; fazer tolices.
\section{Esquerdino}
\begin{itemize}
\item {Grp. gram.:adj.}
\end{itemize}
\begin{itemize}
\item {Utilização:Prov.}
\end{itemize}
\begin{itemize}
\item {Utilização:trasm.}
\end{itemize}
O mesmo que \textunderscore esquerdo\textunderscore .
\section{Esquerdo}
\begin{itemize}
\item {fónica:quêr}
\end{itemize}
\begin{itemize}
\item {Grp. gram.:adj.}
\end{itemize}
\begin{itemize}
\item {Utilização:Fig.}
\end{itemize}
\begin{itemize}
\item {Proveniência:(Do vasc. \textunderscore ezquer\textunderscore )}
\end{itemize}
Que está do lado em que bate o coração: \textunderscore o braço esquerdo\textunderscore .
Que está para o lado do poente, quando se olha para o norte.
Que fica á esquerda do observador.
Que está á esquerda do quem se volta para a foz de um rio: \textunderscore a margem esquerda\textunderscore .
Torcido.
Canhoto; desajeitado.
Vesgo.
Esquivo.
\section{Esquerdote}
\begin{itemize}
\item {Grp. gram.:adj.}
\end{itemize}
\begin{itemize}
\item {Utilização:Prov.}
\end{itemize}
\begin{itemize}
\item {Utilização:minh.}
\end{itemize}
\begin{itemize}
\item {Proveniência:(De \textunderscore esquerdo\textunderscore )}
\end{itemize}
O mesmo que \textunderscore canhoto\textunderscore .
\section{Esquibir}
\begin{itemize}
\item {Grp. gram.:v. i.}
\end{itemize}
Uma das operações de tanoaria:«\textunderscore Três máquinas de esquibir.\textunderscore »\textunderscore Inquér. Industr.\textunderscore ,P. II, l. 2.^o, 209.
\section{Esquiça}
\begin{itemize}
\item {Grp. gram.:f.}
\end{itemize}
\begin{itemize}
\item {Utilização:Ant.}
\end{itemize}
\begin{itemize}
\item {Proveniência:(Do cast. \textunderscore esquicio\textunderscore )}
\end{itemize}
Batoque.
Pau, com que se fecha o orifício, que se faz nas vasilhas para tirar vinho; espicho.
\section{Esquifado}
\begin{itemize}
\item {Grp. gram.:adj.}
\end{itemize}
Semelhante a esquife.
\section{Esquifamento}
\begin{itemize}
\item {Grp. gram.:m.}
\end{itemize}
Acto de esquifar^1.
\section{Esquifar}
\begin{itemize}
\item {Grp. gram.:v. t.}
\end{itemize}
O mesmo que \textunderscore esquipar\textunderscore .
\section{Esquifar}
\begin{itemize}
\item {Grp. gram.:v. i.}
\end{itemize}
Fazer esquifes ou tumbas.
\section{Esquife}
\begin{itemize}
\item {Grp. gram.:m.}
\end{itemize}
\begin{itemize}
\item {Utilização:Ant.}
\end{itemize}
\begin{itemize}
\item {Utilização:Des.}
\end{itemize}
Tumba.
Batel grande.
Barco.
Catre, pequeno leito.
(Cast. \textunderscore esquife\textunderscore )
\section{Esquila}
\begin{itemize}
\item {Grp. gram.:f.}
\end{itemize}
\begin{itemize}
\item {Utilização:Prov.}
\end{itemize}
\begin{itemize}
\item {Utilização:alent.}
\end{itemize}
Pequeno chocalho.
O mesmo que \textunderscore esquilha\textunderscore ^2.
\section{Esquila}
\begin{itemize}
\item {Grp. gram.:f.}
\end{itemize}
\begin{itemize}
\item {Utilização:Prov.}
\end{itemize}
\begin{itemize}
\item {Utilização:trasm}
\end{itemize}
Acto de esquilar.
\section{Esquila}
\begin{itemize}
\item {Grp. gram.:f.}
\end{itemize}
\begin{itemize}
\item {Proveniência:(Do lat. \textunderscore scilla\textunderscore )}
\end{itemize}
Planta liliácea, o mesmo que \textunderscore cebola albarran\textunderscore .
\section{Esquilão}
\begin{itemize}
\item {Grp. gram.:m.}
\end{itemize}
\begin{itemize}
\item {Utilização:Prov.}
\end{itemize}
\begin{itemize}
\item {Proveniência:(De \textunderscore esquila\textunderscore ^1)}
\end{itemize}
Chocalho grande.
\section{Esquilar}
\begin{itemize}
\item {Grp. gram.:v. t.}
\end{itemize}
\begin{itemize}
\item {Proveniência:(Do cast. \textunderscore esquilar\textunderscore )}
\end{itemize}
O mesmo que \textunderscore tosquiar\textunderscore .
(Colhido em Mogadoiro)
\section{Esquilha}
\begin{itemize}
\item {Grp. gram.:f.}
\end{itemize}
\begin{itemize}
\item {Utilização:Gír.}
\end{itemize}
Sardinha.
\section{Esquilha}
\begin{itemize}
\item {Grp. gram.:f.}
\end{itemize}
\begin{itemize}
\item {Utilização:Prov.}
\end{itemize}
\begin{itemize}
\item {Utilização:alg.}
\end{itemize}
O mesmo que \textunderscore choquilha\textunderscore .
\section{Esquiliano}
\begin{itemize}
\item {Grp. gram.:adj.}
\end{itemize}
Relativo a Ésquilo; parecido com o gênio literário de Ésquilo.
\section{Esquilla}
\begin{itemize}
\item {Grp. gram.:f.}
\end{itemize}
\begin{itemize}
\item {Proveniência:(Do lat. \textunderscore scilla\textunderscore )}
\end{itemize}
Planta liliácea, o mesmo que \textunderscore cebola albarran\textunderscore .
\section{Esquilo}
\begin{itemize}
\item {Grp. gram.:m.}
\end{itemize}
\begin{itemize}
\item {Proveniência:(Do lat. hyp. \textunderscore sciurulus\textunderscore , de \textunderscore sciurus\textunderscore ?)}
\end{itemize}
Pequeno mammífero roedor.
\section{Ésquimo}
\begin{itemize}
\item {Grp. gram.:m.}
\end{itemize}
\begin{itemize}
\item {Grp. gram.:Pl.}
\end{itemize}
Língua agglutinativa, falada pelos Ésquimos.
Habitantes da Groenlândia e de outras terras da América árctica.
\section{Esquina}
\begin{itemize}
\item {Grp. gram.:f.}
\end{itemize}
Ângulo, formado por dois planos que se cortam.
Canto exterior de edifício, caixa, etc.
Ângulo de rua: \textunderscore ao dobrar de uma esquina\textunderscore .
(Cast. \textunderscore esquina\textunderscore )
\section{Esquina}
\begin{itemize}
\item {Grp. gram.:f.}
\end{itemize}
\begin{itemize}
\item {Proveniência:(Fr. \textunderscore squine\textunderscore )}
\end{itemize}
Planta esmilácea, cuja raiz tem propriedades análogas ás da salsaparrilha.
\section{Esquinado}
\begin{itemize}
\item {Grp. gram.:adj.}
\end{itemize}
\begin{itemize}
\item {Utilização:Pop.}
\end{itemize}
\begin{itemize}
\item {Proveniência:(De \textunderscore esquina\textunderscore ^1)}
\end{itemize}
Que tem esquinas, que é anguloso.
Um tanto embriagado.
\section{Esquinal}
\begin{itemize}
\item {Grp. gram.:adj.}
\end{itemize}
Relativo a \textunderscore esquina\textunderscore ^1.
\section{Esquinante}
\begin{itemize}
\item {Grp. gram.:m.}
\end{itemize}
\begin{itemize}
\item {Utilização:Prov.}
\end{itemize}
Pau aguçado, com que os oleiros desengrossam o fundo das vasilhas.
\section{Esquinantho}
\begin{itemize}
\item {Grp. gram.:m.}
\end{itemize}
\begin{itemize}
\item {Proveniência:(Do gr. \textunderscore skoinanthos\textunderscore )}
\end{itemize}
Espécie de junco medicinal.
\section{Esquinanto}
\begin{itemize}
\item {Grp. gram.:m.}
\end{itemize}
\begin{itemize}
\item {Proveniência:(Do gr. \textunderscore skoinanthos\textunderscore )}
\end{itemize}
Espécie de junco medicinal.
\section{Esquinar}
\begin{itemize}
\item {Grp. gram.:v. t.}
\end{itemize}
\begin{itemize}
\item {Grp. gram.:V. i.}
\end{itemize}
\begin{itemize}
\item {Utilização:Prov.}
\end{itemize}
\begin{itemize}
\item {Grp. gram.:V. p.}
\end{itemize}
\begin{itemize}
\item {Utilização:Pop.}
\end{itemize}
\begin{itemize}
\item {Proveniência:(De \textunderscore esquina\textunderscore ^1)}
\end{itemize}
Cortar em ângulo.
Facetar.
Pôr obliquamente.
Dar fórma de esquina a.
Fugir, escapar-se.
Embriagar-se alguma coisa.
\section{Esquinência}
\begin{itemize}
\item {Grp. gram.:f.}
\end{itemize}
\begin{itemize}
\item {Proveniência:(Do lat. \textunderscore squinantia\textunderscore )}
\end{itemize}
O mesmo que \textunderscore amygdalite\textunderscore .
\textunderscore Esquinência maligna\textunderscore , designação antiga diphtérica.
\section{Esquineta}
\begin{itemize}
\item {fónica:nê}
\end{itemize}
\begin{itemize}
\item {Grp. gram.:f.}
\end{itemize}
O mesmo que \textunderscore lansquené\textunderscore .
(Infl. de \textunderscore esquina\textunderscore )
\section{Esquineza}
\begin{itemize}
\item {Grp. gram.:f.}
\end{itemize}
O mesmo que \textunderscore esquina\textunderscore ^1.
\section{Esquinote}
\begin{itemize}
\item {Grp. gram.:m.}
\end{itemize}
\begin{itemize}
\item {Utilização:Prov.}
\end{itemize}
Pelle ou coiro, meio preparado, que se emprega em arreios ordinários e calçado de camponeses. Cf. \textunderscore Inquér. Industr.\textunderscore , p. II, l. III, 68.
O mesmo que \textunderscore esquinante\textunderscore .
\section{Esquinudo}
\begin{itemize}
\item {Grp. gram.:adj.}
\end{itemize}
O mesmo que \textunderscore esquinado\textunderscore .
\section{Esquio}
\begin{itemize}
\item {Grp. gram.:m.}
\end{itemize}
\begin{itemize}
\item {Utilização:Ant.}
\end{itemize}
O mesmo que \textunderscore esquilo\textunderscore . Cf. G. Vicente, \textunderscore Auto das Fadas\textunderscore .
\section{Esquipação}
\begin{itemize}
\item {Grp. gram.:f.}
\end{itemize}
\begin{itemize}
\item {Utilização:Pop.}
\end{itemize}
Acto ou effeito de esquipar.
Provisões náuticas.
Conjunto de apparelhos e animaes, empregados e revezados numa lavoira.
Fato completo, andaina.
\section{Esquipado}
\begin{itemize}
\item {Grp. gram.:adj.}
\end{itemize}
\begin{itemize}
\item {Utilização:Fig.}
\end{itemize}
\begin{itemize}
\item {Grp. gram.:M.}
\end{itemize}
\begin{itemize}
\item {Utilização:Bras}
\end{itemize}
\begin{itemize}
\item {Proveniência:(De \textunderscore esquipar\textunderscore )}
\end{itemize}
Apparelhado; provido.
Enfeitado.
Aperaltado.
Ligeiro.
Andadura do cavallo, quando êste levanta ao mesmo tempo o pé e a mão do mesmo lado.
\section{Esquipador}
\begin{itemize}
\item {Grp. gram.:m.   e  adj.}
\end{itemize}
\begin{itemize}
\item {Utilização:Bras}
\end{itemize}
\begin{itemize}
\item {Proveniência:(De \textunderscore esquipar\textunderscore )}
\end{itemize}
Diz-se do cavallo que usa esquipado.
\section{Esquipamento}
\begin{itemize}
\item {Grp. gram.:m.}
\end{itemize}
\begin{itemize}
\item {Proveniência:(De \textunderscore esquipar\textunderscore )}
\end{itemize}
Aquillo com que se esquipa.
Esquipação.
\section{Esquipar}
\begin{itemize}
\item {Grp. gram.:v. t.}
\end{itemize}
\begin{itemize}
\item {Grp. gram.:V. i.}
\end{itemize}
\begin{itemize}
\item {Utilização:Bras}
\end{itemize}
O mesmo que \textunderscore equipar\textunderscore .
Correr ligeiramente (o cavallo, a embarcação).
Executar o cavallo a marcha chamada esquipado.
(Cast. \textunderscore esquipar\textunderscore )
\section{Esquipático}
\begin{itemize}
\item {Grp. gram.:adj.}
\end{itemize}
\begin{itemize}
\item {Utilização:Fam.}
\end{itemize}
Esquisito; extravagante.
\section{Esquipau}
\begin{itemize}
\item {Grp. gram.:m.}
\end{itemize}
\begin{itemize}
\item {Utilização:Prov.}
\end{itemize}
\begin{itemize}
\item {Utilização:N.}
\end{itemize}
O mesmo que \textunderscore peixe-aranha\textunderscore .
\section{Esquírola}
\begin{itemize}
\item {Grp. gram.:f.}
\end{itemize}
Lasca de osso.
Lâmina ou fragmento de objecto duro.
(Cast. \textunderscore esquirla\textunderscore )
\section{Esquisitamente}
\begin{itemize}
\item {Grp. gram.:adv.}
\end{itemize}
De modo esquisito.
\section{Esquisitice}
\begin{itemize}
\item {Grp. gram.:f.}
\end{itemize}
\begin{itemize}
\item {Utilização:Fam.}
\end{itemize}
Qualidade daquelle ou daquillo que é esquisito.
Extravagância, excentricidade.
\section{Esquisitiparla}
\begin{itemize}
\item {Grp. gram.:f.}
\end{itemize}
Aquelle que fala esquisitamente. Cf. Filinto, XIX, 267 e 269.
\section{Esquisito}
\begin{itemize}
\item {Grp. gram.:adj.}
\end{itemize}
\begin{itemize}
\item {Proveniência:(Lat. \textunderscore exquisitus\textunderscore )}
\end{itemize}
Achado com difficuldade ou raramente.
Raro; precioso: \textunderscore jóias esquisitas\textunderscore .
Excellente; primoroso.
Elegante.
Que não é vulgar.
Excêntrico; estrambótico: \textunderscore carácter esquisito\textunderscore .
Maníaco.
\section{Esquisitório}
\begin{itemize}
\item {Grp. gram.:adj.}
\end{itemize}
\begin{itemize}
\item {Utilização:Fam.}
\end{itemize}
\begin{itemize}
\item {Proveniência:(De \textunderscore esquisito\textunderscore )}
\end{itemize}
Muito excêntrico.
Ratão.
Mal feito. Cf. Camillo, \textunderscore Noites de Insómn.\textunderscore , II, 10.
\section{Esquisso}
\begin{itemize}
\item {Proveniência:(Fr. \textunderscore esquisse\textunderscore )}
\end{itemize}
\textunderscore m.\textunderscore  (e der.)
--É gal. inadmissível, em vez de \textunderscore esbôço\textunderscore , primeiros traços, delineamento, etc.
\section{Esquitar}
\begin{itemize}
\item {Grp. gram.:v. t.}
\end{itemize}
\begin{itemize}
\item {Utilização:Ant.}
\end{itemize}
\begin{itemize}
\item {Proveniência:(De \textunderscore quitar\textunderscore )}
\end{itemize}
Descontar.
\section{Esquivamente}
\begin{itemize}
\item {Grp. gram.:adv.}
\end{itemize}
\begin{itemize}
\item {Proveniência:(De \textunderscore esquivo\textunderscore )}
\end{itemize}
Com esquivança.
\section{Esquivança}
\begin{itemize}
\item {Grp. gram.:f.}
\end{itemize}
\begin{itemize}
\item {Proveniência:(De \textunderscore esquivar\textunderscore )}
\end{itemize}
Desprendimento, com desprêzo ou aversão por quem se aproxima de nós.
Desprêzo, aversão.
Trato rude.
Insulamento da sociedade.
\section{Esquivar}
\begin{itemize}
\item {Grp. gram.:v. t.}
\end{itemize}
\begin{itemize}
\item {Grp. gram.:V. i.}
\end{itemize}
\begin{itemize}
\item {Utilização:Prov.}
\end{itemize}
\begin{itemize}
\item {Grp. gram.:V. p.}
\end{itemize}
\begin{itemize}
\item {Proveniência:(De \textunderscore esquivo\textunderscore )}
\end{itemize}
Desviar de si, com desprêzo.
Evitar.
Defecar. (Colhido em Turquel)
Desviar-se; eximir-se: \textunderscore esquivei-me a responder-lhe\textunderscore .
\section{Esquivez}
\begin{itemize}
\item {Grp. gram.:f.}
\end{itemize}
(V.esquivança)
\section{Esquivo}
\begin{itemize}
\item {Grp. gram.:adj.}
\end{itemize}
\begin{itemize}
\item {Proveniência:(Do it. \textunderscore schivo\textunderscore )}
\end{itemize}
Que tem esquivança.
Intratável.
Rude.
Aborrecido.
Que rejeita affectos ou carinhos; arisco: \textunderscore rapariga esquiva\textunderscore .
Árido.
\section{Esquivoso}
\begin{itemize}
\item {Grp. gram.:adj.}
\end{itemize}
O mesmo que \textunderscore esquivo\textunderscore .
\section{Esquixa}
\begin{itemize}
\item {Grp. gram.:f.}
\end{itemize}
\begin{itemize}
\item {Utilização:T. da Villa-da-Feira}
\end{itemize}
Bocadinho, pequena porçâo.
Cigalho.
(Relaciona-se com \textunderscore esquiça\textunderscore ?)
\section{Essa}
\begin{itemize}
\item {Proveniência:(Lat. \textunderscore ersa\textunderscore , de \textunderscore erigere\textunderscore )}
\end{itemize}
Catafalco.
Estrado, erguido numa igreja, para nelle se collocar o caixão de um cadáver, quando se procede a ceremónias fúnebres.
Cenotáphio, espécie de túmulo vazio, erguido numa igreja, quando se suffraga a alma de um defunto.--É errónea a fórma usual \textunderscore eça\textunderscore .
\section{Essa}
(\textunderscore fem. de êsse\textunderscore )
\section{Essaísta}
\begin{itemize}
\item {Grp. gram.:m.}
\end{itemize}
\begin{itemize}
\item {Utilização:Angl}
\end{itemize}
\begin{itemize}
\item {Proveniência:(Do ingl. \textunderscore essay\textunderscore , ensaio)}
\end{itemize}
Escritor, que escreveu ensaios literários. Cf. S. Monteiro, \textunderscore Elog. de Lat.\textunderscore 
\section{Essaýsta}
\begin{itemize}
\item {Grp. gram.:m.}
\end{itemize}
\begin{itemize}
\item {Utilização:Angl}
\end{itemize}
\begin{itemize}
\item {Proveniência:(Do ingl. \textunderscore essay\textunderscore , ensaio)}
\end{itemize}
Escritor, que escreveu ensaios literários. Cf. S. Monteiro, \textunderscore Elog. de Lat.\textunderscore 
\section{Êsse}
\begin{itemize}
\item {Grp. gram.:pron.}
\end{itemize}
\begin{itemize}
\item {Utilização:Ant.}
\end{itemize}
\begin{itemize}
\item {Proveniência:(Do lat. \textunderscore ipse\textunderscore )}
\end{itemize}
(designativo da pessôa ou coisa que esta próxima de quem fala, ou de a quem se fala)
O mesmo que \textunderscore mesmo\textunderscore . Cf. Frei Fortun., \textunderscore Inéd.\textunderscore , I, 307.
\section{Essedários}
\begin{itemize}
\item {Grp. gram.:m. pl.}
\end{itemize}
\begin{itemize}
\item {Proveniência:(Do lat. \textunderscore essedarius\textunderscore )}
\end{itemize}
Gladiadores romanos, que combatiam sentados em carros.
\section{Éssedo}
\begin{itemize}
\item {Grp. gram.:m.}
\end{itemize}
\begin{itemize}
\item {Proveniência:(Lat. \textunderscore essedum\textunderscore )}
\end{itemize}
Carro de duas rodas, usado em campanha por Bretões e Gallos.
\section{Essência}
\begin{itemize}
\item {Grp. gram.:f.}
\end{itemize}
\begin{itemize}
\item {Proveniência:(Lat. \textunderscore essentia\textunderscore )}
\end{itemize}
Aquillo que faz que uma coisa seja o que é.
Natureza íntima das coisas.
Aquillo que constitue a natureza de um objecto.
Existência.
Líquido muito volátil e sem viscosidade, ou substância aromática, que se extrái de certos vegetaes.
Ideia principal.
Significação especial; distintivo.
\section{Essencial}
\begin{itemize}
\item {Grp. gram.:adj.}
\end{itemize}
\begin{itemize}
\item {Grp. gram.:M.}
\end{itemize}
\begin{itemize}
\item {Proveniência:(Lat. \textunderscore essentialis\textunderscore )}
\end{itemize}
Relativo á essência.
Que constitue a essência.
Indispensável.
Necessário.
Característico.
Importante.
O ponto mais importante; aquillo que é essencial.
\section{Essencialidade}
\begin{itemize}
\item {Grp. gram.:f.}
\end{itemize}
Qualidade ou estado daquillo que é essencial.
\section{Essencialismo}
\begin{itemize}
\item {Grp. gram.:m.}
\end{itemize}
\begin{itemize}
\item {Proveniência:(De \textunderscore essencial\textunderscore )}
\end{itemize}
Systema dos que consideram as doenças como independentes das funcções da economia animal.
\section{Essencialista}
\begin{itemize}
\item {Grp. gram.:m.}
\end{itemize}
\begin{itemize}
\item {Proveniência:(De \textunderscore essencial\textunderscore )}
\end{itemize}
Partidário do essencialismo.
\section{Essencialmente}
\begin{itemize}
\item {Grp. gram.:adv.}
\end{itemize}
\begin{itemize}
\item {Proveniência:(De \textunderscore essencial\textunderscore )}
\end{itemize}
Por essência, por natureza.
Indispensavelmente.
No mais alto grau.
\section{Essênios}
\begin{itemize}
\item {Grp. gram.:m. pl.}
\end{itemize}
\begin{itemize}
\item {Proveniência:(Do syr. \textunderscore asa\textunderscore )}
\end{itemize}
Seita judaica, que professava a communhão de bens, evitava todos os prazeres, bebia só água, condemnava o juramento, etc.
\section{Ésses}
\begin{itemize}
\item {Grp. gram.:m. pl.}
\end{itemize}
Biscoitos, em fórma de S.
\section{Êsso}
\begin{itemize}
\item {Grp. gram.:pron.}
\end{itemize}
\begin{itemize}
\item {Utilização:Ant.}
\end{itemize}
O mesmo que \textunderscore isso\textunderscore .
\section{Essomedes}
\begin{itemize}
\item {Grp. gram.:pron.}
\end{itemize}
\begin{itemize}
\item {Utilização:Ant.}
\end{itemize}
\begin{itemize}
\item {Grp. gram.:Adv.}
\end{itemize}
\begin{itemize}
\item {Proveniência:(Do it. \textunderscore esso\textunderscore  + \textunderscore medesimo\textunderscore , mesmo)}
\end{itemize}
Isso mesmo.
Também, da mesma sorte.
\section{Êss'outro}
\begin{itemize}
\item {Grp. gram.:pron.}
\end{itemize}
(designativo de um objecto próximo, que distinguimos de outro também próximo:«\textunderscore finalmente êss'outras almas...\textunderscore »\textunderscore Luz e Calor\textunderscore , 150)
\section{Êsse-outro}
\begin{itemize}
\item {Grp. gram.:pron.}
\end{itemize}
(designativo de um objecto próximo, que distinguimos de outro também próximo:«\textunderscore finalmente êsse-outras almas...\textunderscore »\textunderscore Luz e Calor\textunderscore , 150)
\section{Êsse outro}
\begin{itemize}
\item {Grp. gram.:pron.}
\end{itemize}
(designativo de um objecto próximo, que distinguimos de outro também próximo:«\textunderscore finalmente êsse outras almas...\textunderscore »\textunderscore Luz e Calor\textunderscore , 150)
\section{Êssoutro}
\begin{itemize}
\item {Grp. gram.:pron.}
\end{itemize}
(designativo de um objecto próximo, que distinguimos de outro também próximo:«\textunderscore finalmente êssoutras almas...\textunderscore »\textunderscore Luz e Calor\textunderscore , 150)
\section{Essudoéste}
\begin{itemize}
\item {Grp. gram.:m.}
\end{itemize}
Um dos pontos cardeaes do globo, entre Éste e Sudoéste.
\section{Essueste}
\begin{itemize}
\item {Grp. gram.:m.}
\end{itemize}
\begin{itemize}
\item {Proveniência:(De \textunderscore Éste\textunderscore  + \textunderscore Sueste\textunderscore )}
\end{itemize}
Um dos pontos cardeaes do globo, entre Éste e Suéste.
\section{Esta}
\begin{itemize}
\item {Grp. gram.:pron.}
\end{itemize}
Fem. de \textunderscore êste\textunderscore : \textunderscore esta casa\textunderscore .--Usa-se em proposições ellípticas, significando \textunderscore esta coisa\textunderscore , \textunderscore esta vez\textunderscore , etc.:--\textunderscore esta não esperava eu! Desta escapou elle!\textunderscore 
\section{Estabalhoadamente}
\begin{itemize}
\item {Grp. gram.:adv.}
\end{itemize}
O mesmo que \textunderscore estavanadamente\textunderscore . Cf. Camillo, \textunderscore General C. Ribeiro\textunderscore , 46.
\section{Estabanado}
\begin{itemize}
\item {Grp. gram.:adj.}
\end{itemize}
(V.estavanado)
\section{Estabelecedoiro}
\begin{itemize}
\item {Grp. gram.:adj.}
\end{itemize}
\begin{itemize}
\item {Utilização:Ant.}
\end{itemize}
Que se há de estabelecer.
\section{Estabelecedor}
\begin{itemize}
\item {Grp. gram.:m.  e  adj.}
\end{itemize}
O que estabelece.
\section{Estabelecedouro}
\begin{itemize}
\item {Grp. gram.:adj.}
\end{itemize}
\begin{itemize}
\item {Utilização:Ant.}
\end{itemize}
Que se há de estabelecer.
\section{Estabelecer}
\begin{itemize}
\item {Grp. gram.:v. t.}
\end{itemize}
\begin{itemize}
\item {Utilização:Des.}
\end{itemize}
\begin{itemize}
\item {Grp. gram.:V. p.}
\end{itemize}
\begin{itemize}
\item {Proveniência:(Do lat. \textunderscore stabilis\textunderscore )}
\end{itemize}
Fixar, tornar firme: \textunderscore estabelecer um contrato\textunderscore .
Instituir; fundar: \textunderscore estabelecer uma escola\textunderscore .
Fazer residir.
Alojar.
Vulgarizar.
Dar estabilidade, pôr casa, dar meios de vida, a: \textunderscore estabelecer um afilhado\textunderscore .
Ordenar.
Fixar-se.
Collocar-se.
Fixar residência.
Fundar um estabelecimento próprio: \textunderscore estabeleceu-se no Chiado\textunderscore .
Pôr casa.
\section{Estabelecimento}
\begin{itemize}
\item {Grp. gram.:m.}
\end{itemize}
\begin{itemize}
\item {Utilização:Des.}
\end{itemize}
Acto ou effeito de estabelecer.
Casa commercial, ou lugar em que se faz commércio. Instituição, instituto.
Vulgarização.
Ordem, estatuto.
\section{Estabeleçudo}
\begin{itemize}
\item {Grp. gram.:adj.}
\end{itemize}
\begin{itemize}
\item {Utilização:Ant.}
\end{itemize}
\begin{itemize}
\item {Proveniência:(De \textunderscore estabelecer\textunderscore )}
\end{itemize}
Que se estabeleceu: \textunderscore homem estabeleçudo\textunderscore .
\section{Estabeleza}
\begin{itemize}
\item {Grp. gram.:f.}
\end{itemize}
\begin{itemize}
\item {Utilização:Ant.}
\end{itemize}
Acto de estabelecer.
Estabelecimento; fundação.
(Por \textunderscore estabeleça\textunderscore , de \textunderscore estabelecer\textunderscore )
\section{Estabilidade}
\begin{itemize}
\item {Grp. gram.:f.}
\end{itemize}
\begin{itemize}
\item {Proveniência:(Lat. \textunderscore stabilitas\textunderscore )}
\end{itemize}
Qualidade daquillo que é estável.
\section{Estabilitar}
\begin{itemize}
\item {Grp. gram.:v. t.}
\end{itemize}
\begin{itemize}
\item {Utilização:Ant.}
\end{itemize}
\begin{itemize}
\item {Proveniência:(Do lat. \textunderscore stabilitas\textunderscore )}
\end{itemize}
O mesmo que \textunderscore estabelecer\textunderscore . Cf. \textunderscore Elegíada\textunderscore , (\textunderscore passim\textunderscore ).
\section{Estabilização}
\begin{itemize}
\item {Grp. gram.:f.}
\end{itemize}
Acto ou effeito de estabilizar.
\section{Estabilizar}
\begin{itemize}
\item {Grp. gram.:v. t.}
\end{itemize}
\begin{itemize}
\item {Utilização:Bras}
\end{itemize}
\begin{itemize}
\item {Proveniência:(Do lat. \textunderscore stabilis\textunderscore )}
\end{itemize}
Tornar estável; estabelecer.
\section{Estabulação}
\begin{itemize}
\item {Grp. gram.:f.}
\end{itemize}
\begin{itemize}
\item {Proveniência:(Lat. \textunderscore stabulatio\textunderscore )}
\end{itemize}
Criação ou engorda de animaes em estábulo.
\section{Estabular}
\begin{itemize}
\item {Grp. gram.:v. t.}
\end{itemize}
\begin{itemize}
\item {Proveniência:(Lat. \textunderscore stabulari\textunderscore )}
\end{itemize}
Criar ou engordar em estábulo.
Meter em estrebaria.
\section{Estabular}
\begin{itemize}
\item {Grp. gram.:adj.}
\end{itemize}
Relativo a estábulo.
\section{Estábulo}
\begin{itemize}
\item {Grp. gram.:m.}
\end{itemize}
\begin{itemize}
\item {Proveniência:(Lat. \textunderscore stabulum\textunderscore )}
\end{itemize}
Alprendre ou curral, em que se abriga o gado; malhada.
\section{Estaca}
\begin{itemize}
\item {Grp. gram.:f.}
\end{itemize}
Pau, que se crava na terra ou em qualquer lugar, para amparar ou para se lhe prender qualquer coisa.
Haste, que se crava na terra para criar raízes: \textunderscore plantar de estaca\textunderscore .
Esquírola ou pedacinho de madeira, que se embebe casualmente na pelle.
(Angl. sax. \textunderscore staca\textunderscore )
\section{Estacada}
\begin{itemize}
\item {Grp. gram.:f.}
\end{itemize}
\begin{itemize}
\item {Proveniência:(De \textunderscore estaca\textunderscore )}
\end{itemize}
Série de estacas.
Tranqueira.
Lugar, cerrado por estacas.
Lugar fechado, para brigas ou torneios.
Dique ou barreira provisória, formada de mastros, cordas, etc., para impedir a passagem de embarcação inimiga.
Estacaria.
\section{Estacado}
\begin{itemize}
\item {Grp. gram.:m.}
\end{itemize}
\begin{itemize}
\item {Proveniência:(De \textunderscore estacar\textunderscore )}
\end{itemize}
O mesmo que \textunderscore estacada\textunderscore .
\section{Estacal}
\begin{itemize}
\item {Grp. gram.:m.}
\end{itemize}
\begin{itemize}
\item {Utilização:Prov.}
\end{itemize}
\begin{itemize}
\item {Utilização:beir.}
\end{itemize}
\begin{itemize}
\item {Proveniência:(De \textunderscore estaca\textunderscore )}
\end{itemize}
Olival novo. (Colhido no Fundão)
\section{Estacão}
\begin{itemize}
\item {Grp. gram.:m.}
\end{itemize}
\begin{itemize}
\item {Utilização:Prov.}
\end{itemize}
Estaca grande.
Rasgão com estaca.
\section{Estação}
\begin{itemize}
\item {Grp. gram.:f.}
\end{itemize}
\begin{itemize}
\item {Proveniência:(Lat. \textunderscore statio\textunderscore )}
\end{itemize}
Estada ou paragem num lugar.
Lugar determinado, em que param ou suspendem a marcha combóios, carros, etc.: \textunderscore a estação de Campanhan\textunderscore .
Pôrto, em que um navio se demora por algum tempo.
Pôsto policial.
Repartição pública: \textunderscore êsse negócio está pendente das estações officiaes\textunderscore .
Cada uma das quatro partes em que os equinócios e os solstícios dividem o anno.
Cada um dos períodos, mais ou menos naturaes, da existência.
Visita da devoção ás igrejas.
Dezena de padrenossos e ave-marias.
Allocução de párocho, á missa conventual.
Paragem de procissão ou irmandade, para se rezar ou cantar alguma oração.
Medida itinerária do Oriente.
Temporada: \textunderscore a estação dos banhos\textunderscore .
Opportunidade, tempo próprio para alguma coisa.
\section{Estacar}
\begin{itemize}
\item {Grp. gram.:v. t.}
\end{itemize}
\begin{itemize}
\item {Grp. gram.:V. i.}
\end{itemize}
Segurar com estacas.
Escorar.
Amparar.
Fazer parar.
Ficar parado, parar subitamente: \textunderscore o cavallo estacou\textunderscore .
Hesitar, ficar perplexo.
\section{Estaçar}
\begin{itemize}
\item {Grp. gram.:m.}
\end{itemize}
O mesmo que \textunderscore estazar\textunderscore .
\section{Estacaria}
\begin{itemize}
\item {Grp. gram.:f.}
\end{itemize}
Grande porção de estacas.
Lugar, em que se juntam muitas estacas.
Alicerce ou dique, feito de estacas.
Estacada.
\section{Estache}
\begin{itemize}
\item {Grp. gram.:m.}
\end{itemize}
\begin{itemize}
\item {Utilização:Gír.}
\end{itemize}
\begin{itemize}
\item {Proveniência:(T. cigano de Espanha)}
\end{itemize}
Chapéu.
\section{Estacional}
\begin{itemize}
\item {Grp. gram.:adj.}
\end{itemize}
\begin{itemize}
\item {Proveniência:(Lat. \textunderscore stationalis\textunderscore )}
\end{itemize}
Relativo a estação.
Estacionário.
\section{Estacionamento}
\begin{itemize}
\item {Grp. gram.:m.}
\end{itemize}
Acto de estacionar.
\section{Estacionar}
\begin{itemize}
\item {Grp. gram.:v. i.}
\end{itemize}
\begin{itemize}
\item {Proveniência:(Do lat. \textunderscore statio\textunderscore )}
\end{itemize}
Parar.
Estacar.
Demorar-se.
Fazer estação.
\section{Estacionário}
\begin{itemize}
\item {Grp. gram.:adj.}
\end{itemize}
\begin{itemize}
\item {Utilização:Fig.}
\end{itemize}
\begin{itemize}
\item {Grp. gram.:M.}
\end{itemize}
\begin{itemize}
\item {Utilização:Bras}
\end{itemize}
\begin{itemize}
\item {Proveniência:(Lat. \textunderscore stationarius\textunderscore )}
\end{itemize}
Que estaciona.
Immóvel.
Que não progride: \textunderscore povo estacionário\textunderscore .
Que não augmenta nem deminue.
Opposto ás ideias de progresso.
Persistente, (falando-se de algumas doenças).
Dizia-se de uma espécie de sentinela entre os Romanos.
O encarregado de uma estação, em certas repartições officiaes:«\textunderscore o senhor Oscar Cesar Leal foi nomeado 1.^o estacionário da Directoria da Meteorologia da Repartição da Carta Maritima.\textunderscore »\textunderscore Jorn. do Commércio\textunderscore , do Rio, de 12-VIII-901.
\section{Estacoadela}
\begin{itemize}
\item {Grp. gram.:f.}
\end{itemize}
\begin{itemize}
\item {Utilização:Prov.}
\end{itemize}
\begin{itemize}
\item {Proveniência:(De \textunderscore estacoar\textunderscore )}
\end{itemize}
O mesmo que \textunderscore estacão\textunderscore .
\section{Estacoar}
\begin{itemize}
\item {Grp. gram.:v. t.}
\end{itemize}
\begin{itemize}
\item {Utilização:Prov.}
\end{itemize}
\begin{itemize}
\item {Proveniência:(De \textunderscore estacão\textunderscore )}
\end{itemize}
Segurar com estaca.
Ferir com estaca.
\section{Estacoeiro}
\begin{itemize}
\item {Grp. gram.:m.}
\end{itemize}
\begin{itemize}
\item {Utilização:Prov.}
\end{itemize}
\begin{itemize}
\item {Proveniência:(De \textunderscore estacão\textunderscore )}
\end{itemize}
Qualquer estaca ou tutor, que se applica á videira ou a qualquer planta, que precisa apoio.
\section{Estada}
\begin{itemize}
\item {Grp. gram.:f.}
\end{itemize}
\begin{itemize}
\item {Utilização:Ant.}
\end{itemize}
\begin{itemize}
\item {Utilização:Prov.}
\end{itemize}
\begin{itemize}
\item {Utilização:minh.}
\end{itemize}
Acto de estar.
Permanência.
Demora num lugar.
Estância.
Estábulo.
O mesmo que \textunderscore mangedoira\textunderscore .
\section{Estadão}
\begin{itemize}
\item {Grp. gram.:m.}
\end{itemize}
\begin{itemize}
\item {Utilização:Pop.}
\end{itemize}
\begin{itemize}
\item {Utilização:Prov.}
\end{itemize}
\begin{itemize}
\item {Utilização:trasm.}
\end{itemize}
\begin{itemize}
\item {Proveniência:(De \textunderscore estado\textunderscore )}
\end{itemize}
Grande luxo.
Magnificência.
Cada um dos dois estadulhos, que se põem, divergentes, em certa altura das varas do carro, para entre elles se dispor a lenha da carga, a fim de que não vá ferir os bois, assentando-lhes no dorso.
\section{Estadeador}
\begin{itemize}
\item {Grp. gram.:m.}
\end{itemize}
Aquelle que estadeia.
\section{Estadear}
\begin{itemize}
\item {Grp. gram.:v. t.}
\end{itemize}
\begin{itemize}
\item {Grp. gram.:V. i.}
\end{itemize}
\begin{itemize}
\item {Proveniência:(De \textunderscore estado\textunderscore )}
\end{itemize}
Ostentar.
Alardear: \textunderscore estadear opulência\textunderscore .
Envaidar-se.
Ensoberbecer-se.
\section{Estadeiro}
\begin{itemize}
\item {Grp. gram.:m.}
\end{itemize}
\begin{itemize}
\item {Utilização:Prov.}
\end{itemize}
\begin{itemize}
\item {Utilização:trasm.}
\end{itemize}
\begin{itemize}
\item {Utilização:Prov.}
\end{itemize}
\begin{itemize}
\item {Utilização:minh.}
\end{itemize}
\begin{itemize}
\item {Proveniência:(De \textunderscore estado\textunderscore )}
\end{itemize}
Peça de madeira, em que se prende o papagaio.
Gaiola de papagaio.
Rodela, em que se assentam os cântaros.
Grande séquito, acompanhamento pomposo.
\section{Estadela}
\begin{itemize}
\item {Grp. gram.:f.}
\end{itemize}
\begin{itemize}
\item {Utilização:Ant.}
\end{itemize}
\begin{itemize}
\item {Proveniência:(De \textunderscore estado\textunderscore )}
\end{itemize}
Espécie de throno.
Cadeira nobre, em que se sentavam Reis ou magistrados para dar audiência solemne.
\section{Estadía}
\begin{itemize}
\item {Grp. gram.:f.}
\end{itemize}
\begin{itemize}
\item {Utilização:Jur.}
\end{itemize}
Demora, que o capitão de navio fretado para o transporte de mercadorias é obrigado a fazer, no pôrto, aonde chegou, sem que por isso se lhe deva mais que o frete ajustado. Cf. F. Borges, \textunderscore Diccion. Jur.\textunderscore 
(Cast. \textunderscore estadia\textunderscore . Cp. \textunderscore estada\textunderscore )
\section{Estádia}
\begin{itemize}
\item {Grp. gram.:f.}
\end{itemize}
Instrumento, com que se avalia a distância entre o observador e um ponto distante.
(Da mesma or. que \textunderscore estádio\textunderscore )
\section{Estádio}
\begin{itemize}
\item {Grp. gram.:m.}
\end{itemize}
\begin{itemize}
\item {Proveniência:(Lat. \textunderscore stadium\textunderscore )}
\end{itemize}
Antiga arena para jogos públicos.
Antiga medida itineraria, equivalente a 41^{m},25.
Época.
Estação.
Periodo.
\section{Estadista}
\begin{itemize}
\item {Grp. gram.:m.}
\end{itemize}
\begin{itemize}
\item {Proveniência:(De \textunderscore estado\textunderscore )}
\end{itemize}
Homem de Estado.
Aquelle, que é versado em negócios de alta política, ou que tomou parte importante e profícua na governação de um país.
\section{Estadística}
\begin{itemize}
\item {Grp. gram.:f.}
\end{itemize}
\begin{itemize}
\item {Proveniência:(De \textunderscore estadístico\textunderscore )}
\end{itemize}
Política.
Sciência de governar.
O mesmo que \textunderscore estatística\textunderscore .
\section{Estadisticamente}
\begin{itemize}
\item {Grp. gram.:adv.}
\end{itemize}
\begin{itemize}
\item {Proveniência:(De \textunderscore estadístico\textunderscore )}
\end{itemize}
Segundo os preceitos da estadística.
\section{Estadístico}
\begin{itemize}
\item {Grp. gram.:adj.}
\end{itemize}
\begin{itemize}
\item {Proveniência:(De \textunderscore estadista\textunderscore )}
\end{itemize}
Relativo á estadística.
\section{Estado}
\begin{itemize}
\item {Grp. gram.:m.}
\end{itemize}
\begin{itemize}
\item {Utilização:Des.}
\end{itemize}
\begin{itemize}
\item {Utilização:Prov.}
\end{itemize}
\begin{itemize}
\item {Utilização:trasm.}
\end{itemize}
\begin{itemize}
\item {Grp. gram.:Pl.}
\end{itemize}
\begin{itemize}
\item {Proveniência:(Do lat. \textunderscore status\textunderscore )}
\end{itemize}
Situação, modo de sêr, de uma pessôa ou coisa.
Disposição, em que está uma pessôa ou coisa.
Condição.
Posição social.
Nação, politicamente organizada, e dirigida por leis próprias.
Conjunto dos poderes políticos de uma nação.
Govêrno.
Ostentação, magnificência.
Registo.
Inventário.
Representação, em Côrtes, de alguma classe social, no antigo regime: \textunderscore clero, nobreza e povo eram os três Estados do Reino\textunderscore .
Estatura ordinária de um homem.
A ceremónia funebre, chamada offício de defuntos.
\textunderscore Estado interessante\textunderscore , estado da mulher grávida.
\textunderscore Tomar estado\textunderscore  ou \textunderscore mudar de estado\textunderscore , casar-se.
Terras ou países, sujeitos á mesma autoridade ou jurisdicção.
\section{Estado}
\begin{itemize}
\item {Grp. gram.:m.}
\end{itemize}
\begin{itemize}
\item {Utilização:Des.}
\end{itemize}
O mesmo que \textunderscore estádio\textunderscore :«\textunderscore ...serrania que gira perto de dois mil estados.\textunderscore »\textunderscore Viriato Trág.\textunderscore , I, 21.
\section{Estado-maior}
\begin{itemize}
\item {Grp. gram.:m.}
\end{itemize}
Corporação militar, formada de officiaes com graduação scientífica e sem direcção immediata de tropas.
Conjunto de officiaes militares de um exército, ou de parte de um exército.
\section{Estadual}
\begin{itemize}
\item {Grp. gram.:adj.}
\end{itemize}
\begin{itemize}
\item {Utilização:bras}
\end{itemize}
\begin{itemize}
\item {Utilização:Neol.}
\end{itemize}
\begin{itemize}
\item {Proveniência:(De \textunderscore estado\textunderscore )}
\end{itemize}
Relativo a qualquer dos Estados da República brasileira.
\section{Estadulheira}
\begin{itemize}
\item {Grp. gram.:f.}
\end{itemize}
\begin{itemize}
\item {Utilização:Prov.}
\end{itemize}
\begin{itemize}
\item {Utilização:trasm.}
\end{itemize}
Estadulho, grande cacete.
\section{Estadulho}
\begin{itemize}
\item {Grp. gram.:m.}
\end{itemize}
Pau grosseiro.
Fueiro, ou pau semelhante ao fueiro.
\section{Estafa}
\begin{itemize}
\item {Grp. gram.:f.}
\end{itemize}
Acto ou effeito de estafar.
\section{Estafadeira}
\begin{itemize}
\item {Grp. gram.:f.}
\end{itemize}
O mesmo que \textunderscore estafa\textunderscore .
\section{Estafador}
\begin{itemize}
\item {Grp. gram.:adj.}
\end{itemize}
\begin{itemize}
\item {Grp. gram.:M.}
\end{itemize}
\begin{itemize}
\item {Utilização:Gír.}
\end{itemize}
Que estafa.
Aquelle que estafa.
O mesmo que \textunderscore assassino\textunderscore .
\section{Estafamento}
\begin{itemize}
\item {Grp. gram.:m.}
\end{itemize}
O mesmo que \textunderscore estafa\textunderscore .
\section{Estafar}
\begin{itemize}
\item {Grp. gram.:v. t.}
\end{itemize}
\begin{itemize}
\item {Utilização:Gír.}
\end{itemize}
\begin{itemize}
\item {Utilização:Pop.}
\end{itemize}
\begin{itemize}
\item {Grp. gram.:V. i.}
\end{itemize}
\begin{itemize}
\item {Proveniência:(Do it. \textunderscore staffilare\textunderscore , do ant. alt. al. \textunderscore staph\textunderscore )}
\end{itemize}
Cansar: \textunderscore estafar um cavallo\textunderscore .
Causar fadiga a.
Importunar; maçar.
Matar.
Gastar prodigamente: \textunderscore estafou quanto tinha\textunderscore .
Cansar-se.
\section{Estafegar}
\begin{itemize}
\item {Grp. gram.:v. t.}
\end{itemize}
\begin{itemize}
\item {Utilização:Pop.}
\end{itemize}
Apertar as goélas de.
(Cp. \textunderscore trasfegar\textunderscore )
\section{Estafeiro}
\begin{itemize}
\item {Grp. gram.:m.}
\end{itemize}
\begin{itemize}
\item {Utilização:Ant.}
\end{itemize}
\begin{itemize}
\item {Proveniência:(De \textunderscore estafar\textunderscore )}
\end{itemize}
Moço de estribeira.
Criado, que acompanhava a pé um cavalleiro.
\section{Estafermo}
\begin{itemize}
\item {fónica:fêr}
\end{itemize}
\begin{itemize}
\item {Grp. gram.:m.}
\end{itemize}
\begin{itemize}
\item {Proveniência:(It. \textunderscore staffermo\textunderscore )}
\end{itemize}
Figura de homem, movediça sôbre um eixo, e na qual deviam tocar com a lança os cavalleiros dos torneios, sem serem alcançados pelo chicote, que essa figura tinha na mão.
Espantalho.
Pessôa desmazelada.
Basbaque.
Empecilho.
\section{Estafeta}
\begin{itemize}
\item {fónica:fê}
\end{itemize}
\begin{itemize}
\item {Grp. gram.:m.}
\end{itemize}
\begin{itemize}
\item {Utilização:Fig.}
\end{itemize}
\begin{itemize}
\item {Proveniência:(It. \textunderscore staffeta\textunderscore )}
\end{itemize}
Distribuidor postal, que faz serviço a cavallo, fóra da séde do correio, ou trasm.tte a correspondência de uma estação para outra.
Mensageiro.
\section{Estafete}
\begin{itemize}
\item {fónica:fê}
\end{itemize}
\begin{itemize}
\item {Grp. gram.:m.}
\end{itemize}
\begin{itemize}
\item {Utilização:Prov.}
\end{itemize}
O mesmo que \textunderscore estafeta\textunderscore .
Moço de recados, paquete.
\section{Estafeteiro}
\begin{itemize}
\item {Grp. gram.:m.}
\end{itemize}
\begin{itemize}
\item {Proveniência:(De \textunderscore estafeta\textunderscore )}
\end{itemize}
Estafeta.
Frade, que tinha a seu cargo a gerência do correio da sua communidade.
\section{Estafim}
\begin{itemize}
\item {Grp. gram.:m.}
\end{itemize}
\begin{itemize}
\item {Utilização:Ant.}
\end{itemize}
\begin{itemize}
\item {Proveniência:(Do it. \textunderscore staffile\textunderscore )}
\end{itemize}
Chicote.
\section{Estafonar}
\begin{itemize}
\item {Grp. gram.:v. t.}
\end{itemize}
\begin{itemize}
\item {Utilização:Prov.}
\end{itemize}
Tirar a vida a.
Empandeirar.
Dar cabo de.
Chamuscar, esquartejar e salgar (porcos).
(Relaciona-se com \textunderscore estafar\textunderscore ?)
\section{Estagiário}
\begin{itemize}
\item {Grp. gram.:adj.}
\end{itemize}
\begin{itemize}
\item {Grp. gram.:M.}
\end{itemize}
\begin{itemize}
\item {Proveniência:(Fr. \textunderscore stagiaire\textunderscore )}
\end{itemize}
Relativo ao estágio.
Aquelle que está fazendo tirocínio para certas profissões.
Aquelle que está praticando, com homens peritos ou em estabelecimentos públicos, certa arte ou sciência, para depois a exercer livremente ou sob a sua exclusiva responsabilidade.
\section{Estágio}
\begin{itemize}
\item {Grp. gram.:m.}
\end{itemize}
\begin{itemize}
\item {Utilização:Ant.}
\end{itemize}
Tirocínio ou apprendizado de advogado ou médico.--Ainda hoje é us. no Hospital de San-José, entre os quintanistas de Medicina: \textunderscore amanhan estou de estágio\textunderscore , assisto aos partos.
(Ref. do lat. \textunderscore stadium\textunderscore )
\section{Estagirita}
\begin{itemize}
\item {Grp. gram.:m.}
\end{itemize}
Homem natural de Estagira; por excellência, designação de Aristóteles, natural daquella cidade.
\section{Estagnação}
\begin{itemize}
\item {Grp. gram.:f.}
\end{itemize}
\begin{itemize}
\item {Utilização:Fig.}
\end{itemize}
\begin{itemize}
\item {Proveniência:(De \textunderscore estagnar\textunderscore )}
\end{itemize}
Estado daquillo que se acha estagnado.
Inércia; paralysação.
\section{Estagnador}
\begin{itemize}
\item {Grp. gram.:adj.}
\end{itemize}
O mesmo que \textunderscore estagnante\textunderscore .
\section{Estagnante}
\begin{itemize}
\item {Grp. gram.:adj.}
\end{itemize}
Que estagna ou produz estagnação.
\section{Estagnar}
\begin{itemize}
\item {Grp. gram.:v. t.}
\end{itemize}
\begin{itemize}
\item {Grp. gram.:V. p.}
\end{itemize}
\begin{itemize}
\item {Proveniência:(Lat. \textunderscore stagnare\textunderscore )}
\end{itemize}
Impedir que corra (um líquido).
Tornar inerte, paralysar.
Perder a fluidez, não circular, não correr.
\section{Estagnícola}
\begin{itemize}
\item {Grp. gram.:adj.}
\end{itemize}
\begin{itemize}
\item {Proveniência:(Do lat. \textunderscore stagnum\textunderscore  + \textunderscore colere\textunderscore )}
\end{itemize}
Que vive na água estagnada.
\section{Estagno}
\begin{itemize}
\item {Grp. gram.:m.}
\end{itemize}
\begin{itemize}
\item {Proveniência:(Lat. \textunderscore stagnum\textunderscore )}
\end{itemize}
(V.tanque)
\section{Estai}
\begin{itemize}
\item {Grp. gram.:m.}
\end{itemize}
\begin{itemize}
\item {Utilização:Náut.}
\end{itemize}
\begin{itemize}
\item {Proveniência:(Do ingl. \textunderscore stays\textunderscore )}
\end{itemize}
Cada um dos cabos grossos, que, fixos na prôa, firmam a mastreação.
Designação de outros cabos de navio: \textunderscore estai de galope\textunderscore ; \textunderscore estai de giba\textunderscore ; \textunderscore estai de joanete\textunderscore ; \textunderscore estai de bujarrona\textunderscore ; \textunderscore estai do velacho\textunderscore ; \textunderscore estai do traquete\textunderscore .
\section{Estaiação}
\begin{itemize}
\item {Grp. gram.:f.}
\end{itemize}
\begin{itemize}
\item {Utilização:Náut.}
\end{itemize}
Collocação dos estais.
Conjunto dos estais.
\section{Estaiado}
\begin{itemize}
\item {Grp. gram.:adj.}
\end{itemize}
Que tem estai.
\section{Estai-real}
\begin{itemize}
\item {Grp. gram.:m.}
\end{itemize}
\begin{itemize}
\item {Utilização:Náut.}
\end{itemize}
Cabo engaiado e alcatroado, que volteia o calcês, passa sôbre as encapelladuras, e vai fixar o seu chicote àvante, no convés.
\section{Estala}
\begin{itemize}
\item {Grp. gram.:f.}
\end{itemize}
\begin{itemize}
\item {Utilização:Ant.}
\end{itemize}
O mesmo que \textunderscore estábulo\textunderscore .
Cadeira de monge ou de cónego, na igreja. Cf. Herculano, \textunderscore Bobo\textunderscore .
(B. lat. \textunderscore stallum\textunderscore , do ant. alt. al. \textunderscore stall\textunderscore )
\section{Estalageiro}
\begin{itemize}
\item {Grp. gram.:m.}
\end{itemize}
\begin{itemize}
\item {Proveniência:(De \textunderscore estalagem\textunderscore )}
\end{itemize}
O mesmo ou melhor que \textunderscore estalajadeiro\textunderscore .
\section{Estalagem}
\begin{itemize}
\item {Grp. gram.:f.}
\end{itemize}
\begin{itemize}
\item {Utilização:Prov.}
\end{itemize}
\begin{itemize}
\item {Utilização:trasm.}
\end{itemize}
\begin{itemize}
\item {Proveniência:(Do port. ant. \textunderscore hostalagem\textunderscore . Cp. hostalagem)}
\end{itemize}
Poisada.
Hospedaria.
Albergaria.
Casa de malta.
Atoleiro.
\section{Estalagmite}
\begin{itemize}
\item {Grp. gram.:f.}
\end{itemize}
\begin{itemize}
\item {Proveniência:(Do gr. \textunderscore stalagma\textunderscore )}
\end{itemize}
Concreção mamillosa, formada no solo das cavidades subterrâneas pela evaporação das gotas que cáem da abóbada.
\section{Estalagmítico}
\begin{itemize}
\item {Grp. gram.:adj.}
\end{itemize}
Relativo a estalagmite.
\section{Estalajadeira}
\begin{itemize}
\item {Grp. gram.:fem.}
\end{itemize}
De \textunderscore estalajadeiro\textunderscore .
\section{Estalajadeiro}
\begin{itemize}
\item {Grp. gram.:m.}
\end{itemize}
Aquelle que tem ou administra estalagem.
\section{Estalante}
\begin{itemize}
\item {Grp. gram.:adj.}
\end{itemize}
Que estala.
\section{Estalão}
\begin{itemize}
\item {Grp. gram.:m.}
\end{itemize}
\begin{itemize}
\item {Proveniência:(Do b. lat. \textunderscore stalo\textunderscore )}
\end{itemize}
Craveira, padrão.
\section{Estalar}
\begin{itemize}
\item {Grp. gram.:v. t.}
\end{itemize}
\begin{itemize}
\item {Grp. gram.:V. i.}
\end{itemize}
\begin{itemize}
\item {Utilização:Fam.}
\end{itemize}
Partir, quebrar.
Dar estalos.
Crepitar.
Rebentar com estrondo: \textunderscore a bomba estalou\textunderscore .
Partir-se com fragor.
Rachar-se; partir-se; fender-se: \textunderscore os vidros estalam\textunderscore .
Manifestar-se subitamente.
Desfallecer; acabar.
Rebentar.
Estar ansioso.
(Cast. \textunderscore estallar\textunderscore )
\section{Estalaria}
\begin{itemize}
\item {Grp. gram.:f.}
\end{itemize}
\begin{itemize}
\item {Utilização:Pop.}
\end{itemize}
O mesmo que \textunderscore estalada\textunderscore .
Ruído dos estalos dos foguetes.
\section{Estalecido}
\begin{itemize}
\item {Grp. gram.:m.}
\end{itemize}
\begin{itemize}
\item {Utilização:Prov.}
\end{itemize}
\begin{itemize}
\item {Utilização:trasm.}
\end{itemize}
Dôr, que abala todos os dentes, deixando-os aluídos.
(Corr. de \textunderscore estillicídio\textunderscore ? De \textunderscore estalar\textunderscore ?)
\section{Estalecido}
\begin{itemize}
\item {Grp. gram.:adj.}
\end{itemize}
\begin{itemize}
\item {Utilização:Bras. da Baía}
\end{itemize}
Asmático.
Doente do peito.
\section{Estaleiro}
\begin{itemize}
\item {Grp. gram.:m.}
\end{itemize}
\begin{itemize}
\item {Utilização:Bras}
\end{itemize}
\begin{itemize}
\item {Utilização:Prov.}
\end{itemize}
Lugar, onde se consertam ou constróem navios.
Leito de pano sôbre forquilhas, em que se põe a secar milho, carne, etc.
Tábua, sôbre que os carpinteiros cortam madeira pela menor espessura.
(Talvez do cast. \textunderscore astillero\textunderscore )
\section{Estalejadura}
\begin{itemize}
\item {Grp. gram.:f.}
\end{itemize}
\begin{itemize}
\item {Proveniência:(De \textunderscore estalejar\textunderscore )}
\end{itemize}
Estalido de ossos.
\section{Estalejar}
\begin{itemize}
\item {Grp. gram.:v. i.}
\end{itemize}
\begin{itemize}
\item {Proveniência:(De \textunderscore estalar\textunderscore )}
\end{itemize}
Dar estalos repetidos: \textunderscore estalejam foguetes\textunderscore .
Estalar.
\section{Estalia}
\begin{itemize}
\item {Grp. gram.:f.}
\end{itemize}
\begin{itemize}
\item {Proveniência:(It. \textunderscore stallía\textunderscore )}
\end{itemize}
O mesmo que \textunderscore estadía\textunderscore .--Os diccion. da língua dizem \textunderscore estália\textunderscore ; F. Borges, \textunderscore Díccion. Jur.\textunderscore , manda lêr \textunderscore estalía\textunderscore .
\section{Estalicado}
\begin{itemize}
\item {Grp. gram.:adj.}
\end{itemize}
\begin{itemize}
\item {Utilização:Pop.}
\end{itemize}
\begin{itemize}
\item {Proveniência:(De \textunderscore estalicar\textunderscore ^1)}
\end{itemize}
Definhado; emmagrecido.
\section{Estalicar}
\begin{itemize}
\item {Grp. gram.:v. i.}
\end{itemize}
\begin{itemize}
\item {Utilização:Pop.}
\end{itemize}
Emmagrecer, definhar.
\section{Estalicar}
\begin{itemize}
\item {Grp. gram.:v. i.}
\end{itemize}
\begin{itemize}
\item {Utilização:Pop.}
\end{itemize}
Dar estalos com os dedos.
\section{Estalicídio}
\begin{itemize}
\item {Grp. gram.:m.}
\end{itemize}
\begin{itemize}
\item {Utilização:Ant.}
\end{itemize}
(Corr. de \textunderscore estillicídio\textunderscore )
\section{Estalidante}
\begin{itemize}
\item {Grp. gram.:adj.}
\end{itemize}
\begin{itemize}
\item {Utilização:Neol.}
\end{itemize}
\begin{itemize}
\item {Proveniência:(De \textunderscore estalidar\textunderscore )}
\end{itemize}
Que dá estalidos.
\section{Estalidar}
\begin{itemize}
\item {Grp. gram.:v. i.}
\end{itemize}
\begin{itemize}
\item {Utilização:Neol.}
\end{itemize}
Dar estalidos.
\section{Estalido}
\begin{itemize}
\item {Grp. gram.:m.}
\end{itemize}
Estalos repetidos.
Ruído daquillo que estala.
Crepitação.
Estridor súbíto.
(Cp. cast. \textunderscore estallido\textunderscore )
\section{Estalla}
\begin{itemize}
\item {Grp. gram.:f.}
\end{itemize}
\begin{itemize}
\item {Utilização:Ant.}
\end{itemize}
O mesmo que \textunderscore estábulo\textunderscore .
Cadeira de monge ou de cónego, na igreja. Cf. Herculano, \textunderscore Bobo\textunderscore .
(B. lat. \textunderscore stallum\textunderscore , do ant. alt. al. \textunderscore stall\textunderscore )
\section{Estallia}
\begin{itemize}
\item {Grp. gram.:f.}
\end{itemize}
\begin{itemize}
\item {Proveniência:(It. \textunderscore stallía\textunderscore )}
\end{itemize}
O mesmo que \textunderscore estadía\textunderscore .--Os diccion. da língua dizem \textunderscore estállia\textunderscore ; F. Borges, \textunderscore Díccion. Jur.\textunderscore , manda lêr \textunderscore estallía\textunderscore .
\section{Estallo}
\begin{itemize}
\item {Grp. gram.:m.}
\end{itemize}
\begin{itemize}
\item {Utilização:Ant.}
\end{itemize}
Cadeira de monge ou de cónego.
O mesmo que \textunderscore estalla\textunderscore .
\section{Estalo}
\begin{itemize}
\item {Grp. gram.:m.}
\end{itemize}
\begin{itemize}
\item {Utilização:Pop.}
\end{itemize}
\begin{itemize}
\item {Utilização:Pop.}
\end{itemize}
\begin{itemize}
\item {Proveniência:(De \textunderscore estalar\textunderscore )}
\end{itemize}
Rumor súbito.
Crepitação.
Estoiro.
Som, produzido por objecto que estala ou que vibra repentinamente.
Bofetão.
\textunderscore Coisa de estalo\textunderscore , coisa excellente.
\section{Estalo}
\begin{itemize}
\item {Grp. gram.:m.}
\end{itemize}
\begin{itemize}
\item {Utilização:Ant.}
\end{itemize}
Cadeira de monge ou de cónego.
O mesmo que \textunderscore estala\textunderscore .
\section{Estamagado}
\begin{itemize}
\item {Grp. gram.:adj.}
\end{itemize}
\begin{itemize}
\item {Utilização:Prov.}
\end{itemize}
\begin{itemize}
\item {Utilização:trasm.}
\end{itemize}
\begin{itemize}
\item {Proveniência:(De \textunderscore estâmago\textunderscore )}
\end{itemize}
Cansado, fraco.
\section{Estâmago}
\textunderscore m.\textunderscore  (e der.)
Fórma pop. e ant. de \textunderscore estômago\textunderscore :«\textunderscore tal do rei novo o estâmago acendido...\textunderscore »\textunderscore Lusiadas\textunderscore , III, 48. Cf. \textunderscore Eufrosina\textunderscore , 171.
\section{Estamarrado}
\begin{itemize}
\item {Grp. gram.:adj.}
\end{itemize}
\begin{itemize}
\item {Utilização:T. de Serpa}
\end{itemize}
\begin{itemize}
\item {Proveniência:(De \textunderscore extra\textunderscore  + ?)}
\end{itemize}
Casual.
Imprevisto, extraordinário: \textunderscore teve uma febre estamarrada\textunderscore .
\section{Estalactífero}
\begin{itemize}
\item {Grp. gram.:adj.}
\end{itemize}
\begin{itemize}
\item {Proveniência:(Do gr. \textunderscore stalaktos\textunderscore  + lat. \textunderscore ferre\textunderscore )}
\end{itemize}
Que tem estalactites.
\section{Estalactiforme}
\begin{itemize}
\item {Grp. gram.:adj.}
\end{itemize}
Que tem fórma de estalactite.
\section{Estalactite}
\begin{itemize}
\item {Grp. gram.:f.}
\end{itemize}
\begin{itemize}
\item {Proveniência:(Do gr. \textunderscore stalaktos\textunderscore )}
\end{itemize}
Concreção alongada, formada na abóbada de cavidades subterrâneas pela infiltração de liquidos, que têm em dissolução saes calcários, silicosos, etc.
\section{Estalactítico}
\begin{itemize}
\item {Grp. gram.:adj.}
\end{itemize}
Semelhante a estalactite.
\section{Estalada}
\begin{itemize}
\item {Grp. gram.:f.}
\end{itemize}
\begin{itemize}
\item {Utilização:Fig.}
\end{itemize}
\begin{itemize}
\item {Utilização:Fam.}
\end{itemize}
\begin{itemize}
\item {Proveniência:(De \textunderscore estalar\textunderscore )}
\end{itemize}
Som, produzido por um objecto que estala: \textunderscore ouviu-se a estalada de um chicote\textunderscore .
Ruído; motim.
O mesmo que \textunderscore bofetada\textunderscore .
\section{Estaladeira}
\begin{itemize}
\item {Grp. gram.:f.}
\end{itemize}
\begin{itemize}
\item {Utilização:Prov.}
\end{itemize}
\begin{itemize}
\item {Utilização:alent.}
\end{itemize}
\begin{itemize}
\item {Proveniência:(De \textunderscore estalar\textunderscore )}
\end{itemize}
Casca do pinheiro.
Corcódea.
\section{Estaladura}
\begin{itemize}
\item {Grp. gram.:f.}
\end{itemize}
Acto ou effeito de estalar.
\section{Estala-estala}
\begin{itemize}
\item {Grp. gram.:f.}
\end{itemize}
Árvore da ilha de San-Thomé.
\section{Estafileáceas}
\begin{itemize}
\item {Grp. gram.:f.}
\end{itemize}
\begin{itemize}
\item {Proveniência:(Do gr. \textunderscore staphule\textunderscore )}
\end{itemize}
Família de plantas, formada á custa das rhamnáceas.
\section{Estafilino}
\begin{itemize}
\item {Grp. gram.:m.}
\end{itemize}
\begin{itemize}
\item {Proveniência:(Do gr. \textunderscore staphule\textunderscore )}
\end{itemize}
Gênero de insectos coleópteros.
\section{Estafilococo}
\begin{itemize}
\item {Grp. gram.:m.}
\end{itemize}
\begin{itemize}
\item {Utilização:Med.}
\end{itemize}
Bacillo da coqueluche.
Micróbio da septicemia.
\section{Estafiloma}
\begin{itemize}
\item {Grp. gram.:m.}
\end{itemize}
\begin{itemize}
\item {Utilização:Med.}
\end{itemize}
\begin{itemize}
\item {Proveniência:(Gr. \textunderscore staphuloma\textunderscore )}
\end{itemize}
Lesão na córnea.
Lesão de qualquer tecido do ôlho.
\section{Estafiságria}
\begin{itemize}
\item {Grp. gram.:f.}
\end{itemize}
\begin{itemize}
\item {Proveniência:(Gr. \textunderscore staphisagria\textunderscore )}
\end{itemize}
Designação científica do paparraz.
\section{Estambrar}
\begin{itemize}
\item {Grp. gram.:v. t.}
\end{itemize}
Converter em estambre (a lan).
\section{Estambre}
\begin{itemize}
\item {Grp. gram.:m.}
\end{itemize}
Fio de lan ou de seda.
Fio da urdidura.
Estame.
Lan, cujos filamentos têm sido dispostos parallelamente a si mesmos, e que não estão misturados como na lan cardada.
(Cast. \textunderscore estambre\textunderscore , estame)
\section{Estambreiro}
\begin{itemize}
\item {Grp. gram.:adj.}
\end{itemize}
Convertido em estambre.
\section{Estame}
\begin{itemize}
\item {Grp. gram.:m.}
\end{itemize}
\begin{itemize}
\item {Utilização:Fig.}
\end{itemize}
\begin{itemize}
\item {Proveniência:(Lat. \textunderscore stamen\textunderscore )}
\end{itemize}
Fio de urdir o tecer.
Órgão sexual masculino dos vegetaes.
Fio da vida.
\section{Estamenha}
\begin{itemize}
\item {Grp. gram.:f.}
\end{itemize}
\begin{itemize}
\item {Proveniência:(De \textunderscore estame\textunderscore )}
\end{itemize}
Tecido ordinário de lan.
\section{Estamenheiro}
\begin{itemize}
\item {Grp. gram.:m.}
\end{itemize}
Vendedor ou fabricante de estamenha.
\section{Estamento}
\begin{itemize}
\item {Grp. gram.:m.}
\end{itemize}
\begin{itemize}
\item {Utilização:Neol.}
\end{itemize}
Modo de estar.
Congresso.
(Cast. \textunderscore estamento\textunderscore )
\section{Estamete}
\begin{itemize}
\item {fónica:mê}
\end{itemize}
\begin{itemize}
\item {Grp. gram.:m.}
\end{itemize}
\begin{itemize}
\item {Proveniência:(De \textunderscore estame\textunderscore )}
\end{itemize}
Antiga espécie de estamenha.
\section{Estamináceo}
\begin{itemize}
\item {Grp. gram.:adj.}
\end{itemize}
\begin{itemize}
\item {Utilização:Bot.}
\end{itemize}
\begin{itemize}
\item {Proveniência:(Do lat. \textunderscore stamen\textunderscore )}
\end{itemize}
Relativo a estames.
\section{Estaminado}
\begin{itemize}
\item {Grp. gram.:adj.}
\end{itemize}
\begin{itemize}
\item {Utilização:Bot.}
\end{itemize}
\begin{itemize}
\item {Proveniência:(Lat. \textunderscore staminatus\textunderscore )}
\end{itemize}
Que tem estames.
\section{Estaminal}
\begin{itemize}
\item {Grp. gram.:adj.}
\end{itemize}
\begin{itemize}
\item {Utilização:Bot.}
\end{itemize}
\begin{itemize}
\item {Proveniência:(Do lat. \textunderscore stamen\textunderscore )}
\end{itemize}
Relativo a estames.
\section{Estaminário}
\begin{itemize}
\item {Grp. gram.:adj.}
\end{itemize}
\begin{itemize}
\item {Utilização:Bot.}
\end{itemize}
\begin{itemize}
\item {Proveniência:(Lat. \textunderscore staminarius\textunderscore )}
\end{itemize}
Formado pela transformação dos estames.
\section{Estaminífero}
\begin{itemize}
\item {Grp. gram.:adj.}
\end{itemize}
\begin{itemize}
\item {Utilização:Bot.}
\end{itemize}
\begin{itemize}
\item {Proveniência:(Do lat. \textunderscore stamen\textunderscore  + \textunderscore ferre\textunderscore )}
\end{itemize}
O mesmo que \textunderscore estaminado\textunderscore .
\section{Estaminoide}
\begin{itemize}
\item {Grp. gram.:adj.}
\end{itemize}
\begin{itemize}
\item {Utilização:Bot.}
\end{itemize}
\begin{itemize}
\item {Proveniência:(Do gr. \textunderscore stemon\textunderscore  + \textunderscore eidos\textunderscore )}
\end{itemize}
Semelhante a estame.
\section{Estaminoso}
\begin{itemize}
\item {Grp. gram.:adj.}
\end{itemize}
\begin{itemize}
\item {Utilização:Bot.}
\end{itemize}
\begin{itemize}
\item {Proveniência:(Do lat. \textunderscore stamen\textunderscore )}
\end{itemize}
Cujos estames são muito salientes.
\section{Estamínula}
\begin{itemize}
\item {Grp. gram.:f.}
\end{itemize}
\begin{itemize}
\item {Utilização:Bot.}
\end{itemize}
\begin{itemize}
\item {Proveniência:(Do lat. \textunderscore stamen\textunderscore )}
\end{itemize}
Estame rudimentar.
\section{Estampa}
\begin{itemize}
\item {Grp. gram.:f.}
\end{itemize}
\begin{itemize}
\item {Utilização:Fig.}
\end{itemize}
\begin{itemize}
\item {Proveniência:(It. \textunderscore stampa\textunderscore )}
\end{itemize}
Figura impressa, por meio de chapa gravada.
Impressão, imprensa: dar um livro à estampa.
Desenho.
Imagem.
Vestigio.
Coisa perfeita, formosa: \textunderscore aquelle cavallo é uma estampa\textunderscore .
\textunderscore Obrar de estampa\textunderscore , fazer sempre o mesmo, não discrepar de certa norma. Cf. Bernárdez, \textunderscore Luz e Calor\textunderscore , 276.
\section{Estampador}
\textunderscore m.\textunderscore  e \textunderscore adj\textunderscore .
O que estampa.
\section{Estampagem}
\begin{itemize}
\item {Grp. gram.:f.}
\end{itemize}
Acto ou effeito de estampar.
\section{Estampar}
\begin{itemize}
\item {Grp. gram.:v. t.}
\end{itemize}
\begin{itemize}
\item {Utilização:Fig.}
\end{itemize}
\begin{itemize}
\item {Utilização:Fam.}
\end{itemize}
\begin{itemize}
\item {Proveniência:(De \textunderscore estampar\textunderscore )}
\end{itemize}
Imprimir sôbre matriz gravada.
Converter em estampa.
Imprimir.
Desenhar, gravar.
Marcar.
Imprimir desenhos ou côres em (tecidos).
Fazer que deixe vestígio.
Patentear.
Gravar (na memória, no coração, etc.).
Assentar, com fôrça, bofetada ou pontapé.
\section{Estamparia}
\begin{itemize}
\item {Grp. gram.:f.}
\end{itemize}
Fábrica, em que se estampam tecidos.
Lugar, em que se fabricam ou se vendem estampas.
(Do \textunderscore estampar\textunderscore )
\section{Estampeiro}
\begin{itemize}
\item {Grp. gram.:m.}
\end{itemize}
Estampador.
Vendedor de estampas.
\section{Estampido}
\begin{itemize}
\item {Grp. gram.:m.}
\end{itemize}
Som repentino e forte, como o produzido pela explosão de uma arma de fogo.
Grande estrondo.
(Cast. \textunderscore estampido\textunderscore )
\section{Estampilha}
\begin{itemize}
\item {Grp. gram.:f.}
\end{itemize}
\begin{itemize}
\item {Utilização:Gír. lisb.}
\end{itemize}
Pequena estampa.
Chapa, em que se fez gravura, para estampar.
Sêllo, para franquear remessas postaes.
Bofetada.
(Cast. \textunderscore estampilla\textunderscore )
\section{Estampilhagem}
\begin{itemize}
\item {Grp. gram.:f.}
\end{itemize}
Acto de \textunderscore estampilhar\textunderscore .
\section{Estampilhar}
\begin{itemize}
\item {Grp. gram.:v. t.}
\end{itemize}
\begin{itemize}
\item {Utilização:Gír.}
\end{itemize}
Pôr estampilha ou estampilhas em.
Sellar ou franquear com estampilha.
Dar bofetadas em.
\section{Estanato}
\begin{itemize}
\item {Grp. gram.:m.}
\end{itemize}
\begin{itemize}
\item {Proveniência:(Do lat. \textunderscore stannum\textunderscore )}
\end{itemize}
Sal, produzido pela combinação do ácido estânnico com uma base.
\section{Estanca}
\begin{itemize}
\item {Grp. gram.:f.}
\end{itemize}
\begin{itemize}
\item {Utilização:Prov.}
\end{itemize}
\begin{itemize}
\item {Proveniência:(De \textunderscore estancar\textunderscore )}
\end{itemize}
Divisão na masseira.
\section{Estança}
\begin{itemize}
\item {Grp. gram.:f.}
\end{itemize}
\begin{itemize}
\item {Utilização:Des.}
\end{itemize}
\begin{itemize}
\item {Proveniência:(Do b. \textunderscore stantia\textunderscore )}
\end{itemize}
Estada.
O mesmo ou melhor que \textunderscore estância\textunderscore .
\section{Estancação}
\begin{itemize}
\item {Grp. gram.:f.}
\end{itemize}
Acto ou effeito de estancar.
\section{Estanca-cavallos}
\begin{itemize}
\item {Grp. gram.:f.}
\end{itemize}
Planta medicinal, purgativa.
\section{Estancadeira}
\begin{itemize}
\item {Grp. gram.:f.}
\end{itemize}
\begin{itemize}
\item {Proveniência:(De \textunderscore estancar\textunderscore )}
\end{itemize}
Planta plumbagínea.
\section{Estancamento}
\begin{itemize}
\item {Grp. gram.:m.}
\end{itemize}
Acto ou effeito de estancar.
\section{Estancar}
\begin{itemize}
\item {Grp. gram.:v. t.}
\end{itemize}
\begin{itemize}
\item {Grp. gram.:V. i.}
\end{itemize}
\begin{itemize}
\item {Proveniência:(Do lat. \textunderscore stagnare\textunderscore )}
\end{itemize}
Impedir que corra (um líquido).
Vedar: \textunderscore estancar o sangue\textunderscore .
Tornar estagnado: \textunderscore estancar água\textunderscore .
Esgotar.
Açambarcar, monopolizar.
Pôr fim a.
Fatigar.
Deixar de correr.
Esgotar-se; cansar-se.
\section{Estanca-rios}
\begin{itemize}
\item {Grp. gram.:m.}
\end{itemize}
Engenho de tirar água de poços, ou rios, semelhante á nora.
\section{Estanca-sangue}
\begin{itemize}
\item {Grp. gram.:m.}
\end{itemize}
Planta brasileira, medicinal.
\section{Estanca-sangues}
\begin{itemize}
\item {Grp. gram.:m.}
\end{itemize}
\begin{itemize}
\item {Utilização:Prov.}
\end{itemize}
\begin{itemize}
\item {Utilização:trasm.}
\end{itemize}
Rosário, que se põe na cabeça, para fazer estancar o sangue do nariz.
\section{Estancável}
\begin{itemize}
\item {Grp. gram.:adj.}
\end{itemize}
Que se póde estancar.
\section{Estanceiro}
\begin{itemize}
\item {Grp. gram.:m.}
\end{itemize}
\begin{itemize}
\item {Proveniência:(De \textunderscore estança\textunderscore )}
\end{itemize}
Aquelle que tem estância de madeiras.
\section{Estância}
\begin{itemize}
\item {Grp. gram.:f.}
\end{itemize}
\begin{itemize}
\item {Utilização:Ant.}
\end{itemize}
\begin{itemize}
\item {Utilização:Bras}
\end{itemize}
\begin{itemize}
\item {Utilização:Prov.}
\end{itemize}
\begin{itemize}
\item {Utilização:alg.}
\end{itemize}
\begin{itemize}
\item {Utilização:Bras. do N}
\end{itemize}
\begin{itemize}
\item {Utilização:Bras. do N}
\end{itemize}
\begin{itemize}
\item {Proveniência:(De \textunderscore estar\textunderscore )}
\end{itemize}
Lugar, onde se está ou se permanece.
Morada.
Mansão.
Recinto.
Paragem; estação.
Armazém de madeiras ou de materiaes de construcção.
Depósito de carvão, lenha, carqueja, etc.
Cada uma das divisões de uma composição poética, havendo em cada uma igual número de versos e a mesma disposição das rimas.
Pequeno baluarte.
Fazenda para criação de gados.
Tábua grande, em que os pedreiros têm a argamassa, levada pelos serventes no corcho.
Fazenda para criação de gado.
Barracão, onde vivem promiscuamente numerosas pessôas; cortiço.
\section{Estanciar}
\begin{itemize}
\item {Grp. gram.:v. i.}
\end{itemize}
Fazêr estância, residir.
Parar.
Demorar-se.
Descansar.
\section{Estancieiro}
\begin{itemize}
\item {Grp. gram.:m.}
\end{itemize}
\begin{itemize}
\item {Utilização:Bras}
\end{itemize}
\begin{itemize}
\item {Proveniência:(De \textunderscore estância\textunderscore )}
\end{itemize}
O mesmo que \textunderscore estanceiro\textunderscore .
Dono de fazenda para criação de gados.
\section{Estanciola}
\begin{itemize}
\item {Grp. gram.:f.}
\end{itemize}
\begin{itemize}
\item {Utilização:Bras. do S}
\end{itemize}
Pequena estância, chácara.
\section{Estanco}
\begin{itemize}
\item {Grp. gram.:m.}
\end{itemize}
\begin{itemize}
\item {Utilização:Ant.}
\end{itemize}
\begin{itemize}
\item {Proveniência:(De \textunderscore estancar\textunderscore )}
\end{itemize}
Loja, em que se vende tabaco.
Estanque.
Lago, tanque.
\section{Estandal}
\begin{itemize}
\item {Grp. gram.:m.}
\end{itemize}
\begin{itemize}
\item {Utilização:Ant.}
\end{itemize}
Talvez renque de velas acesas:«\textunderscore nunca tantos estandaes ardero ante o seu altar.\textunderscore »\textunderscore Cancion. da Vaticana\textunderscore .
\section{Estandarte}
\begin{itemize}
\item {Grp. gram.:m.}
\end{itemize}
\begin{itemize}
\item {Proveniência:(Do fr. \textunderscore étendard\textunderscore )}
\end{itemize}
Bandeira militar.
Insígnia de corporação.
Pequena prancha de ébano, a que se prendem as cordas, nos instrumentes de arco.
\section{Estanguido}
\begin{itemize}
\item {Grp. gram.:adj.}
\end{itemize}
\begin{itemize}
\item {Utilização:Des.}
\end{itemize}
\begin{itemize}
\item {Proveniência:(Do rad. de \textunderscore estancar\textunderscore ?)}
\end{itemize}
Estancado.
Extenuado.
\section{Estanhação}
\begin{itemize}
\item {Grp. gram.:f.}
\end{itemize}
O mesmo que \textunderscore estanhadura\textunderscore .
\section{Estanhado}
\begin{itemize}
\item {Grp. gram.:adj.}
\end{itemize}
\begin{itemize}
\item {Utilização:Fam.}
\end{itemize}
\begin{itemize}
\item {Proveniência:(De \textunderscore estanhar\textunderscore )}
\end{itemize}
Descarado, desavergonhado.
\section{Estanhador}
\begin{itemize}
\item {Grp. gram.:m.}
\end{itemize}
Aquelle que estanha.
\section{Estanhadura}
\begin{itemize}
\item {Grp. gram.:f.}
\end{itemize}
O mesmo que \textunderscore estanhagem\textunderscore .
\section{Estanhagem}
\begin{itemize}
\item {Grp. gram.:f.}
\end{itemize}
Acto ou effeito de estanhar.
\section{Estanhar}
\begin{itemize}
\item {Grp. gram.:v. t.}
\end{itemize}
Cobrir com camada de estanho ou com liga de estanho e chumbo.
\section{Estanheira}
\begin{itemize}
\item {Grp. gram.:f.}
\end{itemize}
\begin{itemize}
\item {Utilização:Prov.}
\end{itemize}
\begin{itemize}
\item {Utilização:alent.}
\end{itemize}
Estante para loiça de estanho; prateleira.
\section{Estanho}
\begin{itemize}
\item {Grp. gram.:m.}
\end{itemize}
\begin{itemize}
\item {Proveniência:(Do lat. \textunderscore stannum\textunderscore )}
\end{itemize}
Corpo metállico, branco e malleável.
\section{Estânico}
\begin{itemize}
\item {Grp. gram.:adj.}
\end{itemize}
\begin{itemize}
\item {Proveniência:(Do lat. \textunderscore stannum\textunderscore )}
\end{itemize}
Diz-se de um dos dois ácidos do estanho.
Relativo a estanho.
\section{Estanífero}
\begin{itemize}
\item {Grp. gram.:adj.}
\end{itemize}
\begin{itemize}
\item {Proveniência:(Do lat. \textunderscore stannum\textunderscore  + \textunderscore ferre\textunderscore )}
\end{itemize}
Que contém estanho. Cf. Castilho, \textunderscore Fastos\textunderscore , II, 354.
\section{Estanina}
\begin{itemize}
\item {Grp. gram.:f.}
\end{itemize}
\begin{itemize}
\item {Proveniência:(Do lat. \textunderscore stannum\textunderscore )}
\end{itemize}
Nome, que se deu a uma complexa espécie de mineraes, em que se compreendia, além de outros elementos, o sulfureto de estanho e o sulfureto de ferro.
\section{Estanite}
\begin{itemize}
\item {Grp. gram.:f.}
\end{itemize}
\begin{itemize}
\item {Utilização:Miner.}
\end{itemize}
Silicato de alumina de estanho.
\section{Estannato}
\begin{itemize}
\item {Grp. gram.:m.}
\end{itemize}
\begin{itemize}
\item {Proveniência:(Do lat. \textunderscore stannum\textunderscore )}
\end{itemize}
Sal, produzido pela combinação do ácido estânnico com uma base.
\section{Estânnico}
\begin{itemize}
\item {Grp. gram.:adj.}
\end{itemize}
\begin{itemize}
\item {Proveniência:(Do lat. \textunderscore stannum\textunderscore )}
\end{itemize}
Diz-se de um dos dois ácidos do estanho.
Relativo a estanho.
\section{Estannífero}
\begin{itemize}
\item {Grp. gram.:adj.}
\end{itemize}
\begin{itemize}
\item {Proveniência:(Do lat. \textunderscore stannum\textunderscore  + \textunderscore ferre\textunderscore )}
\end{itemize}
Que contém estanho. Cf. Castilho, \textunderscore Fastos\textunderscore , II, 354.
\section{Estannina}
\begin{itemize}
\item {Grp. gram.:f.}
\end{itemize}
\begin{itemize}
\item {Proveniência:(Do lat. \textunderscore stannum\textunderscore )}
\end{itemize}
Nome, que se deu a uma complexa espécie de mineraes, em que se comprehendia, além de outros elementos, o sulfureto de estanho e o sulfureto de ferro.
\section{Estannite}
\begin{itemize}
\item {Grp. gram.:f.}
\end{itemize}
\begin{itemize}
\item {Utilização:Miner.}
\end{itemize}
Silicato de alumina de estanho.
\section{Estannolithe}
\begin{itemize}
\item {Grp. gram.:f.}
\end{itemize}
Óxydo de estanho.
\section{Estanolite}
\begin{itemize}
\item {Grp. gram.:f.}
\end{itemize}
Óxido de estanho.
\section{Estanque}
\begin{itemize}
\item {Grp. gram.:m.}
\end{itemize}
Acto ou effeito de estancar.
Monopólio.
Estanco.
\section{Estanqueira}
\begin{itemize}
\item {Grp. gram.:f.}
\end{itemize}
\begin{itemize}
\item {Proveniência:(De \textunderscore estanqueiro\textunderscore )}
\end{itemize}
Mulher, que tem estanco ou venda de tabaco.
Mulher de estanqueiro.
\section{Estanqueiro}
\begin{itemize}
\item {Grp. gram.:m.}
\end{itemize}
\begin{itemize}
\item {Proveniência:(De \textunderscore estanco\textunderscore  e \textunderscore estanque\textunderscore )}
\end{itemize}
Aquelle que tem estanco.
Aquelle que monopolizou a venda de certas mercadorias.
\section{Estante}
\begin{itemize}
\item {Grp. gram.:m.}
\end{itemize}
\begin{itemize}
\item {Grp. gram.:Adj.}
\end{itemize}
\begin{itemize}
\item {Utilização:Des.}
\end{itemize}
\begin{itemize}
\item {Utilização:Heráld.}
\end{itemize}
\begin{itemize}
\item {Utilização:Ant.}
\end{itemize}
\begin{itemize}
\item {Proveniência:(Do lat. \textunderscore stans\textunderscore , \textunderscore stantis\textunderscore )}
\end{itemize}
Móvel, com prateleiras intervalladas, destinada especialmente para livros ou papéis.
Armário com prateleiras, para livros.
Móvel, que tem superiormente uma tábua inclinada, em que se encostam livros ou se estende o papel da música que o tocador executa.
Fixo.
Residente.
Que faz estância.
Diz-se do animal, que, no campo do escudo, se representa firme nos pés.
O mesmo que [[estando|estar]], part. ou gerúndio de \textunderscore estar\textunderscore .
\section{Estanteirola}
\begin{itemize}
\item {Grp. gram.:f.}
\end{itemize}
\begin{itemize}
\item {Utilização:Ant.}
\end{itemize}
\begin{itemize}
\item {Proveniência:(De \textunderscore estante\textunderscore )}
\end{itemize}
Columna, que sustinha o tendal.
\section{Estapafúrdico}
\begin{itemize}
\item {Grp. gram.:adj.}
\end{itemize}
O mesmo que \textunderscore estapafúrdio\textunderscore . Cf. Th. Ribeiro, \textunderscore Jornadas\textunderscore , II, 139.
\section{Estapafúrdio}
\begin{itemize}
\item {Grp. gram.:adj.}
\end{itemize}
\begin{itemize}
\item {Utilização:Pop.}
\end{itemize}
Excêntrico.
Esquisito.
Extravagante.
Estouvado.
\section{Estapafurdismo}
\begin{itemize}
\item {Grp. gram.:m.}
\end{itemize}
Qualidade de estapafúrdio.
Esquisitice.
Extravagância. Cf. Camillo, \textunderscore Serões de S. Miguel de Seide\textunderscore .
\section{Estapédico}
\begin{itemize}
\item {Grp. gram.:adj.}
\end{itemize}
\begin{itemize}
\item {Utilização:Anat.}
\end{itemize}
\begin{itemize}
\item {Proveniência:(Do b. lat. \textunderscore stapedium\textunderscore )}
\end{itemize}
Relativo ao osso que, no ouvido interno, se chama \textunderscore estribo\textunderscore ; e diz-se da articulação com êsse osso.
\section{Estapélia}
\begin{itemize}
\item {Grp. gram.:f.}
\end{itemize}
Gênero de plantas asclepiádeas.
\section{Estaphilococco}
\begin{itemize}
\item {Grp. gram.:m.}
\end{itemize}
\begin{itemize}
\item {Utilização:Med.}
\end{itemize}
Bacillo da coqueluche.
Micróbio da septicemia.
\section{Estaphiságria}
\begin{itemize}
\item {Grp. gram.:f.}
\end{itemize}
\begin{itemize}
\item {Proveniência:(Gr. \textunderscore staphisagria\textunderscore )}
\end{itemize}
Designação scientífica do paparraz.
\section{Estaphyleáceas}
\begin{itemize}
\item {Grp. gram.:f.}
\end{itemize}
\begin{itemize}
\item {Proveniência:(Do gr. \textunderscore staphule\textunderscore )}
\end{itemize}
Família de plantas, formada á custa das rhamnáceas.
\section{Estaphylino}
\begin{itemize}
\item {Grp. gram.:m.}
\end{itemize}
\begin{itemize}
\item {Proveniência:(Do gr. \textunderscore staphule\textunderscore )}
\end{itemize}
Gênero de insectos coleópteros.
\section{Estaphyloma}
\begin{itemize}
\item {Grp. gram.:m.}
\end{itemize}
\begin{itemize}
\item {Utilização:Med.}
\end{itemize}
\begin{itemize}
\item {Proveniência:(Gr. \textunderscore staphuloma\textunderscore )}
\end{itemize}
Lesão na córnea.
Lesão de qualquer tecido do ôlho.
\section{Estaqueação}
\begin{itemize}
\item {Grp. gram.:f.}
\end{itemize}
Acto de estaquear.
\section{Estaquear}
\begin{itemize}
\item {Grp. gram.:v. t.}
\end{itemize}
Segurar com estacas.
Bater com estaca.
\section{Estaqueira}
\begin{itemize}
\item {Grp. gram.:f.}
\end{itemize}
\begin{itemize}
\item {Utilização:Bras}
\end{itemize}
\begin{itemize}
\item {Proveniência:(De \textunderscore estaca\textunderscore )}
\end{itemize}
Cabide.
\section{Estaquilha}
\begin{itemize}
\item {Grp. gram.:f.}
\end{itemize}
Toro de madeira, em que os rolheiros preparam os quartos de cortiça, de que fazem as rolhas.
\section{Estar}
\begin{itemize}
\item {Grp. gram.:v. i.}
\end{itemize}
\begin{itemize}
\item {Utilização:Fam.}
\end{itemize}
\begin{itemize}
\item {Grp. gram.:V. t.}
\end{itemize}
\begin{itemize}
\item {Grp. gram.:V. p.}
\end{itemize}
\begin{itemize}
\item {Grp. gram.:M.}
\end{itemize}
\begin{itemize}
\item {Utilização:Ant.}
\end{itemize}
\begin{itemize}
\item {Proveniência:(Lat. \textunderscore stare\textunderscore )}
\end{itemize}
Sêr presente, em dado lugar ou em dado momento: \textunderscore estamos no campo\textunderscore ; \textunderscore estamos em Julho\textunderscore .
Sêr.
Achar-se em certas condições: \textunderscore o Rui está robusto\textunderscore .
Manter-se em certa posição: \textunderscore a porta está fechada\textunderscore .
Têr posição vertical.
Permanecer.
Assistir; comparecer: \textunderscore estar na festa\textunderscore .
Chegar.
Têr disposição ou tenções.
Sêr favorável: \textunderscore estou sempre pelos fracos\textunderscore .
Consistir: \textunderscore o mal está nisso\textunderscore .
Morar, residir: \textunderscore está agora no Estoril\textunderscore .
Sêr apropriado, condizer.
\textunderscore Estar de maré\textunderscore , estar bem disposto.
\textunderscore Estar em brasa\textunderscore , estar impaciente.
\textunderscore Estar na fresca ribeira\textunderscore , estar regalado.
\textunderscore Estar pela beiça\textunderscore , andar perdido de amores.
(seguido de oração integrante), sêr de opinião, entender: \textunderscore estou que não tens razão\textunderscore .
A mesma sign. que o \textunderscore v. i.\textunderscore , geralmente com um sujeito indeterminado:«\textunderscore é um estar-se preso por vontade\textunderscore ». Camões, \textunderscore Sonetos\textunderscore . Cf. Castilho, \textunderscore Fausto\textunderscore , 29; Camillo, \textunderscore Quéda\textunderscore , 164.
O mesmo que \textunderscore estau\textunderscore .--É v. auxiliar e tem accepções que só os exemplos designam.
\section{Estarcão}
\begin{itemize}
\item {Grp. gram.:m.}
\end{itemize}
\begin{itemize}
\item {Utilização:Ant.}
\end{itemize}
Cota de armas.
Chapa ou malha, com armas bordadas.
\section{Estardalhaço}
\begin{itemize}
\item {Grp. gram.:m.}
\end{itemize}
\begin{itemize}
\item {Utilização:Pop.}
\end{itemize}
\begin{itemize}
\item {Utilização:Fig.}
\end{itemize}
Grande barulho; borborinho.
Ostentação ruidosa.
\section{Estardalhante}
\begin{itemize}
\item {Grp. gram.:adj.}
\end{itemize}
Que estardalha.
\section{Estardalhar}
\begin{itemize}
\item {Grp. gram.:v. i.}
\end{itemize}
\begin{itemize}
\item {Utilização:bras}
\end{itemize}
\begin{itemize}
\item {Utilização:Neol.}
\end{itemize}
Fazer estardalhaço.
\section{Estardalho}
\begin{itemize}
\item {Grp. gram.:m.}
\end{itemize}
\begin{itemize}
\item {Utilização:Prov.}
\end{itemize}
\begin{itemize}
\item {Utilização:minh.}
\end{itemize}
\begin{itemize}
\item {Utilização:Prov.}
\end{itemize}
\begin{itemize}
\item {Utilização:beir.}
\end{itemize}
Pessôa bulhenta, inquieta, traquinas.
Mulher ou rapariga desajeitada e mal vestida.
Bilhostreira.
\section{Estardato}
\begin{itemize}
\item {Grp. gram.:m.}
\end{itemize}
\begin{itemize}
\item {Utilização:Gír.}
\end{itemize}
Estoque.
\section{Estardiota}
\begin{itemize}
\item {Grp. gram.:f.}
\end{itemize}
\begin{itemize}
\item {Utilização:Ant.}
\end{itemize}
Brida.
Processo de cavalgar, estendendo bem as pernas, processo contrário ao de gineta.
(Corr. de \textunderscore estradiota\textunderscore )
\section{Estarim}
\begin{itemize}
\item {Grp. gram.:m.}
\end{itemize}
\begin{itemize}
\item {Utilização:Gír.}
\end{itemize}
Prisão preventiva.
Calaboiço das estações policiaes.
(Cp. caló \textunderscore estaríbel\textunderscore , prisão)
\section{Estarna}
\begin{itemize}
\item {Grp. gram.:f.}
\end{itemize}
Pequena perdiz, de pés escuros.
\section{Estarola}
\begin{itemize}
\item {Grp. gram.:m.  e  f.}
\end{itemize}
\begin{itemize}
\item {Utilização:Prov.}
\end{itemize}
\begin{itemize}
\item {Utilização:beir.}
\end{itemize}
\begin{itemize}
\item {Grp. gram.:M.}
\end{itemize}
Pessôa estroina ou leviana.
Janota, casquilho.
\section{Estarosta}
\begin{itemize}
\item {Grp. gram.:m.}
\end{itemize}
\begin{itemize}
\item {Proveniência:(Do pol. \textunderscore starosta\textunderscore )}
\end{itemize}
Fidalgo polaco, que dispunha de uma estarostia.
\section{Estarostia}
\begin{itemize}
\item {Grp. gram.:f.}
\end{itemize}
\begin{itemize}
\item {Proveniência:(De \textunderscore estarosta\textunderscore )}
\end{itemize}
Feudo, que os reis da Polónia davam a certos fidalgos, para que êstes os ajudassem nas despesas da guerra.
\section{Estarrecer}
\begin{itemize}
\item {Grp. gram.:v. t.}
\end{itemize}
\begin{itemize}
\item {Grp. gram.:V. i.}
\end{itemize}
Aterrar, apavorar.
Assustar-se muito, ficar enfiado.
(Por \textunderscore esterrecer\textunderscore , do lat. \textunderscore terrere\textunderscore )
\section{Estarrecido}
\begin{itemize}
\item {Grp. gram.:adj.}
\end{itemize}
\begin{itemize}
\item {Proveniência:(De \textunderscore estarrecer\textunderscore )}
\end{itemize}
Espantado, aterrado. Cf. Camillo, \textunderscore Brasileira\textunderscore , 168.
\section{Estarrincar}
\begin{itemize}
\item {Grp. gram.:v. i.}
\end{itemize}
\begin{itemize}
\item {Utilização:Prov.}
\end{itemize}
\begin{itemize}
\item {Utilização:trasm.}
\end{itemize}
Diz-se do ranger dos dentes:«\textunderscore o estarrincar dos dentes...\textunderscore »Camillo, \textunderscore Rom. de Um Homem Rico\textunderscore , 128.
Trovejar.
(Cp. \textunderscore tarrincar\textunderscore )
\section{Estarrinco}
\begin{itemize}
\item {Grp. gram.:m.}
\end{itemize}
\begin{itemize}
\item {Utilização:Prov.}
\end{itemize}
\begin{itemize}
\item {Utilização:trasm.}
\end{itemize}
\begin{itemize}
\item {Proveniência:(De \textunderscore estarrincar\textunderscore )}
\end{itemize}
O mesmo que \textunderscore trovão\textunderscore .
\section{Estase}
\begin{itemize}
\item {Grp. gram.:f.}
\end{itemize}
\begin{itemize}
\item {Utilização:Fig.}
\end{itemize}
\begin{itemize}
\item {Proveniência:(Gr. \textunderscore stasis\textunderscore )}
\end{itemize}
Estagnação do sangue ou de outros humores do corpo.
Paralysação.
\section{Estasiado}
\begin{itemize}
\item {Grp. gram.:adj.}
\end{itemize}
\begin{itemize}
\item {Utilização:Prov.}
\end{itemize}
Resequido: \textunderscore aquellas plantas estão estasiadas\textunderscore .
Que tem muita sêde: \textunderscore dá água ao cão, que anda estasiado\textunderscore .
\section{Estatano}
\begin{itemize}
\item {Grp. gram.:m.}
\end{itemize}
\begin{itemize}
\item {Proveniência:(Lat. \textunderscore statanus\textunderscore )}
\end{itemize}
Entre os Romanos, divindade tutelar das crianças que começavam a andar.
\section{Estateladamente}
\begin{itemize}
\item {Grp. gram.:adv.}
\end{itemize}
\begin{itemize}
\item {Proveniência:(De \textunderscore estatelar\textunderscore )}
\end{itemize}
Ao comprido, no chão: \textunderscore cair estateladamente\textunderscore .
\section{Estatelar}
\begin{itemize}
\item {Grp. gram.:v. t.}
\end{itemize}
Atirar ao chão.
Deitar por terra.
Estender no solo.
(Talvez por \textunderscore estartalar\textunderscore , do cast. \textunderscore estartalado\textunderscore , descomposto)
\section{Estathmética}
\begin{itemize}
\item {Grp. gram.:f.}
\end{itemize}
\begin{itemize}
\item {Proveniência:(Gr. \textunderscore stathmekicos\textunderscore )}
\end{itemize}
Uso ou applicação de pesos e medidas.
\section{Estática}
\begin{itemize}
\item {Grp. gram.:f.}
\end{itemize}
\begin{itemize}
\item {Proveniência:(De \textunderscore estático\textunderscore )}
\end{itemize}
Parte da Mechânica, que trata do equilíbrio das fôrças.
\textunderscore Estática chímica\textunderscore , doutrina do equilíbrio das combinações chímicas.
\section{Estático}
\begin{itemize}
\item {Grp. gram.:adj.}
\end{itemize}
\begin{itemize}
\item {Proveniência:(Gr. \textunderscore statikos\textunderscore )}
\end{itemize}
Firme.
Immóvel.
Que está em repoiso.
Relativo a equilíbrio.
\section{Estatina}
\begin{itemize}
\item {Grp. gram.:f.}
\end{itemize}
\begin{itemize}
\item {Proveniência:(Lat. \textunderscore statina\textunderscore )}
\end{itemize}
Divindade feminina, tutelar das crianças que começam a andar. Cp. \textunderscore estatano\textunderscore .
\section{Estatinga}
\begin{itemize}
\item {Grp. gram.:f.}
\end{itemize}
Procissão de finados ou de almas penadas, na superstição popular.
(Por \textunderscore estantiga\textunderscore  = cast. \textunderscore estantigua\textunderscore , de \textunderscore hueste\textunderscore  + \textunderscore antigua\textunderscore )
\section{Estáo}
\begin{itemize}
\item {Grp. gram.:m.}
\end{itemize}
\begin{itemize}
\item {Utilização:Ant.}
\end{itemize}
\begin{itemize}
\item {Proveniência:(Do ant. alt. al. \textunderscore stal\textunderscore ? Por \textunderscore hostau\textunderscore , do lat. \textunderscore hospitaculum\textunderscore ?)}
\end{itemize}
Casa, em que se aposentava a côrte e os Embaixadores.
Estalagem. Cf. Herculano, \textunderscore Hist. de Port.\textunderscore , III, 54.
\section{Estatista}
\begin{itemize}
\item {Grp. gram.:m.}
\end{itemize}
Aquelle que é perito em Estatística.
\section{Estatística}
\begin{itemize}
\item {Grp. gram.:f.}
\end{itemize}
\begin{itemize}
\item {Proveniência:(Do gr. \textunderscore statizein\textunderscore )}
\end{itemize}
Sciência, que tem por objecto a extensão, população e recursos económicos de um Estado.
Estadística.
Descripção de um país, sob o ponto de vista da extensão, população, recursos económicos, etc.
Conjunto de elementos numéricos, attinentes a certa ordem de factos sociaes.
\section{Estatístico}
\begin{itemize}
\item {Grp. gram.:adj.}
\end{itemize}
\begin{itemize}
\item {Grp. gram.:M.}
\end{itemize}
Relativo a Estatística.
Aquelle, que se occupa de Estatística.
\section{Estatmética}
\begin{itemize}
\item {Grp. gram.:f.}
\end{itemize}
\begin{itemize}
\item {Proveniência:(Gr. \textunderscore stathmekicos\textunderscore )}
\end{itemize}
Uso ou applicação de pesos e medidas.
\section{Estatolatria}
\begin{itemize}
\item {Grp. gram.:f.}
\end{itemize}
\begin{itemize}
\item {Utilização:Neol.}
\end{itemize}
Systema ou doutrina dos que recorrem ao Estado, como a quem póde resolver todas as difficuldades económicas e sociaes.
(Hybr., do lat. \textunderscore status\textunderscore  + gr. \textunderscore latrein\textunderscore )
\section{Estátua}
\begin{itemize}
\item {Grp. gram.:f.}
\end{itemize}
\begin{itemize}
\item {Utilização:Fig.}
\end{itemize}
\begin{itemize}
\item {Proveniência:(Lat. \textunderscore statua\textunderscore )}
\end{itemize}
Figura inteira, em pleno relêvo, que representa um homem, uma mulher, uma divindade, um animal.
Pessôa imbecil, sem sentimentos.
Pessôa acanhada, que não sabe resolver-se.
\section{Estatuar}
\begin{itemize}
\item {Grp. gram.:v. t.}
\end{itemize}
\begin{itemize}
\item {Utilização:Ant.}
\end{itemize}
\begin{itemize}
\item {Proveniência:(Do lat. \textunderscore stalus\textunderscore )}
\end{itemize}
Pôr; collocar.
Sepultar.
\section{Estatuária}
\begin{itemize}
\item {Grp. gram.:f.}
\end{itemize}
\begin{itemize}
\item {Proveniência:(De \textunderscore estatuário\textunderscore )}
\end{itemize}
Arte de fazer estátuas.
\section{Estatuário}
\begin{itemize}
\item {Grp. gram.:m.}
\end{itemize}
\begin{itemize}
\item {Grp. gram.:Adj.}
\end{itemize}
\begin{itemize}
\item {Proveniência:(Lat. \textunderscore statuarius\textunderscore )}
\end{itemize}
Aquelle que faz estátuas.
Relativo a estátuas.
Próprio para estátuas.
\section{Estatucional}
\begin{itemize}
\item {Grp. gram.:adj.}
\end{itemize}
\begin{itemize}
\item {Utilização:Neol.}
\end{itemize}
\begin{itemize}
\item {Proveniência:(Do lat. \textunderscore statutio\textunderscore )}
\end{itemize}
Relativo a estatutos.
\section{Estatueta}
\begin{itemize}
\item {fónica:ê}
\end{itemize}
\begin{itemize}
\item {Grp. gram.:f.}
\end{itemize}
Pequena estátua.
\section{Estatuir}
\begin{itemize}
\item {Grp. gram.:v. t.}
\end{itemize}
\begin{itemize}
\item {Proveniência:(Lat. \textunderscore statuere\textunderscore )}
\end{itemize}
Determinar, por meio de estatuto.
Estabelecer.
Ordenar.
Apresentar como preceito.
\section{Estatura}
\begin{itemize}
\item {Grp. gram.:f.}
\end{itemize}
\begin{itemize}
\item {Proveniência:(Lat. \textunderscore statura\textunderscore )}
\end{itemize}
Tamanho de uma pessôa.
Altura ou grandeza de um sêr animado.
\section{Estatutário}
\begin{itemize}
\item {Grp. gram.:adj.}
\end{itemize}
\begin{itemize}
\item {Utilização:Neol.}
\end{itemize}
Relativo a estatutos: \textunderscore disposição estatutária\textunderscore .
\section{Estatuto}
\begin{itemize}
\item {Grp. gram.:m.}
\end{itemize}
\begin{itemize}
\item {Proveniência:(Lat. \textunderscore statatus\textunderscore )}
\end{itemize}
Lei, constituição ou regulamento de um Estado, de uma Associação ou de uma Companhia.
\section{Estau}
\begin{itemize}
\item {Grp. gram.:m.}
\end{itemize}
\begin{itemize}
\item {Utilização:Ant.}
\end{itemize}
\begin{itemize}
\item {Proveniência:(Do ant. alt. al. \textunderscore stal\textunderscore ? Por \textunderscore hostau\textunderscore , do lat. \textunderscore hospitaculum\textunderscore ?)}
\end{itemize}
Casa, em que se aposentava a côrte e os Embaixadores.
Estalagem. Cf. Herculano, \textunderscore Hist. de Port.\textunderscore , III, 54.
\section{Estauntónia}
\begin{itemize}
\item {Grp. gram.:f.}
\end{itemize}
Gênero de plantas melastomáceas.
\section{Estauracantho}
\begin{itemize}
\item {Grp. gram.:m.}
\end{itemize}
Gênero de plantas leguminosas.
\section{Estauracanto}
\begin{itemize}
\item {Grp. gram.:m.}
\end{itemize}
Gênero de plantas leguminosas.
\section{Estaurantera}
\begin{itemize}
\item {Grp. gram.:f.}
\end{itemize}
Gênero de plantas gesneriáceas.
\section{Estauranthera}
\begin{itemize}
\item {Grp. gram.:f.}
\end{itemize}
Gênero de plantas gesneriáceas.
\section{Estaurastro}
\begin{itemize}
\item {Grp. gram.:m.}
\end{itemize}
Gênero de algas.
\section{Estaurídia}
\begin{itemize}
\item {Grp. gram.:m.}
\end{itemize}
Gênero de pólypos.
\section{Estaurólatra}
\begin{itemize}
\item {Grp. gram.:m.}
\end{itemize}
Aquelle que adora a cruz.
Nome, dado especialmente a uns sectários armênios, que não adoravam outra imagem, além da cruz.
\section{Estaurólitho}
\begin{itemize}
\item {Grp. gram.:m.}
\end{itemize}
\begin{itemize}
\item {Proveniência:(Do gr. \textunderscore stauros\textunderscore  + \textunderscore lithos\textunderscore )}
\end{itemize}
Mineral, cujos crystaes são habitualmente cruciformes.
\section{Estaurólito}
\begin{itemize}
\item {Grp. gram.:m.}
\end{itemize}
\begin{itemize}
\item {Proveniência:(Do gr. \textunderscore stauros\textunderscore  + \textunderscore lithos\textunderscore )}
\end{itemize}
Mineral, cujos cristaes são habitualmente cruciformes.
\section{Estauroscópio}
\begin{itemize}
\item {Grp. gram.:m.}
\end{itemize}
\begin{itemize}
\item {Proveniência:(Do gr. \textunderscore stauros\textunderscore  + \textunderscore skopein\textunderscore )}
\end{itemize}
Apparelho, imaginado por Kobell, para determinar as direcções de extincção da luz polarizada nos crystaes birefringentes.
\section{Estaurospermo}
\begin{itemize}
\item {Grp. gram.:m.}
\end{itemize}
Gênero de plantas phýceas.
\section{Estavanadamente}
\begin{itemize}
\item {Grp. gram.:adv.}
\end{itemize}
De modo estavanado.
\section{Estavanado}
\begin{itemize}
\item {Grp. gram.:adj.}
\end{itemize}
\begin{itemize}
\item {Proveniência:(De \textunderscore tavão\textunderscore ?)}
\end{itemize}
Estouvado.
Inquieto.
Que é valdevinos.
\section{Estável}
\begin{itemize}
\item {Grp. gram.:adj.}
\end{itemize}
\begin{itemize}
\item {Proveniência:(Lat. \textunderscore stabilis\textunderscore )}
\end{itemize}
Firme; sólido.
Duradoiro; inalterável.
\section{Estazador}
\begin{itemize}
\item {Grp. gram.:adj.}
\end{itemize}
\begin{itemize}
\item {Grp. gram.:M.}
\end{itemize}
Que estaza.
Aquelle que estaza.
\section{Estazamento}
\begin{itemize}
\item {Grp. gram.:m.}
\end{itemize}
Acto ou effeito de estazar.
\section{Estazar}
\begin{itemize}
\item {Grp. gram.:v. t.}
\end{itemize}
Fatigar.
Esfalfar.
\section{Estaziar}
\begin{itemize}
\item {Grp. gram.:v. t.}
\end{itemize}
\begin{itemize}
\item {Utilização:Prov.}
\end{itemize}
Tornar estouvado.
Tirar o juízo a.
Tornar inquieto.
\section{Éste}
\begin{itemize}
\item {Grp. gram.:m.}
\end{itemize}
\begin{itemize}
\item {Proveniência:(Do angl. sax. \textunderscore oest\textunderscore )}
\end{itemize}
O mesmo que \textunderscore Léste\textunderscore .
\section{Êste}
\begin{itemize}
\item {Grp. gram.:pron. m.}
\end{itemize}
\begin{itemize}
\item {Proveniência:(Do lat. \textunderscore iste\textunderscore )}
\end{itemize}
(designativo de pessôa ou coisa que está presente ou muito proxima de quem fala)
\section{Estear}
\begin{itemize}
\item {Grp. gram.:v. t.}
\end{itemize}
\begin{itemize}
\item {Utilização:Fig.}
\end{itemize}
Amparar com esteios ou escoras.
Proteger, amparar.
\section{Estear}
\begin{itemize}
\item {Grp. gram.:v. i.}
\end{itemize}
\begin{itemize}
\item {Utilização:Prov.}
\end{itemize}
\begin{itemize}
\item {Utilização:trasm.}
\end{itemize}
O mesmo que \textunderscore amariçar\textunderscore .
\section{Estearato}
\begin{itemize}
\item {Grp. gram.:m.}
\end{itemize}
\begin{itemize}
\item {Proveniência:(Do gr. \textunderscore stear\textunderscore )}
\end{itemize}
Sal, que resulta da combinação do ácido esteárico com uma base.
\section{Estearia}
\begin{itemize}
\item {Grp. gram.:f.}
\end{itemize}
\begin{itemize}
\item {Utilização:Ant.}
\end{itemize}
\begin{itemize}
\item {Proveniência:(De \textunderscore estau\textunderscore )}
\end{itemize}
Palácio dos estaus, em Lisbôa.
\section{Esteárico}
\begin{itemize}
\item {Grp. gram.:adj.}
\end{itemize}
\begin{itemize}
\item {Proveniência:(Do gr. \textunderscore stear\textunderscore )}
\end{itemize}
Relativo a estearina.
Diz-se de um ácido, que resulta da saponificação das substâncias gordas, especialmente do sebo.
\section{Estearina}
\begin{itemize}
\item {Grp. gram.:f.}
\end{itemize}
\begin{itemize}
\item {Proveniência:(Do gr. \textunderscore stear\textunderscore )}
\end{itemize}
Substância sólida das gorduras do boi ou do carneiro.
\section{Estearinaria}
\begin{itemize}
\item {Grp. gram.:f.}
\end{itemize}
Fábrica de velas de estearina.
\section{Esteatite}
\begin{itemize}
\item {Grp. gram.:f.}
\end{itemize}
\begin{itemize}
\item {Proveniência:(Gr. \textunderscore steatites\textunderscore )}
\end{itemize}
Pedra molle, de côr esverdeada, que é um silicato de magnésia.
\section{Esteatoma}
\begin{itemize}
\item {Grp. gram.:m.}
\end{itemize}
\begin{itemize}
\item {Proveniência:(Gr. \textunderscore steatoma\textunderscore )}
\end{itemize}
Tumor sebáceo.
\section{Esteatomático}
\begin{itemize}
\item {Grp. gram.:adj.}
\end{itemize}
Relativo a esteatoma.
\section{Esteatose}
\begin{itemize}
\item {Grp. gram.:f.}
\end{itemize}
Producção accidental de grânulos gordurosos nos elementos anatómicos.
(Cp. \textunderscore esteatoma\textunderscore )
\section{Estefana}
\begin{itemize}
\item {Grp. gram.:f.}
\end{itemize}
\begin{itemize}
\item {Utilização:Açor}
\end{itemize}
\begin{itemize}
\item {Proveniência:(De \textunderscore Estefânia\textunderscore , n. p.?)}
\end{itemize}
Mulher corpulenta e de maus costumes.
\section{Estefânico}
\begin{itemize}
\item {Grp. gram.:adj.}
\end{itemize}
\begin{itemize}
\item {Utilização:Ant.}
\end{itemize}
Relativo ou pertencente ao estefânio.
\section{Estefânio}
\begin{itemize}
\item {Grp. gram.:m.}
\end{itemize}
\begin{itemize}
\item {Proveniência:(Do gr. \textunderscore stephanos\textunderscore )}
\end{itemize}
Ponto craniométrico, determinado pela sutura coronal e pela crista do osso temporal.
\section{Estefanómia}
\begin{itemize}
\item {Grp. gram.:f.}
\end{itemize}
Animal marinho, diáfano, com a aparência de uma grinalda animada. Cf. A. F. Simões, \textunderscore Da Beiramar\textunderscore , 15.
\section{Estefanóforo}
\begin{itemize}
\item {Grp. gram.:adj.}
\end{itemize}
\begin{itemize}
\item {Proveniência:(Gr. \textunderscore stephanophoros\textunderscore )}
\end{itemize}
Dizia-se das mesas, em que se colocavam as corôas destinadas aos vencedores dos jogos olímpicos.
\section{Esteganografia}
\begin{itemize}
\item {Grp. gram.:f.}
\end{itemize}
\begin{itemize}
\item {Proveniência:(Do gr. \textunderscore steganos\textunderscore  + \textunderscore graphein\textunderscore )}
\end{itemize}
Escrita em cifra ou em sinaes convencionaes.
\section{Esteganográfico}
\begin{itemize}
\item {Grp. gram.:adj.}
\end{itemize}
Relativo a esteganografia.
\section{Esteganógrafo}
\begin{itemize}
\item {Grp. gram.:m.}
\end{itemize}
Aquele, que é versado em esteganografia.
\section{Esteganographia}
\begin{itemize}
\item {Grp. gram.:f.}
\end{itemize}
\begin{itemize}
\item {Proveniência:(Do gr. \textunderscore steganos\textunderscore  + \textunderscore graphein\textunderscore )}
\end{itemize}
Escrita em cifra ou em sinaes convencionaes.
\section{Esteganográphico}
\begin{itemize}
\item {Grp. gram.:adj.}
\end{itemize}
Relativo a esteganographia.
\section{Esteganógrapho}
\begin{itemize}
\item {Grp. gram.:m.}
\end{itemize}
Aquelle, que é versado em esteganographia.
\section{Esteganópodes}
\begin{itemize}
\item {Grp. gram.:m. pl.}
\end{itemize}
\begin{itemize}
\item {Utilização:Zool.}
\end{itemize}
\begin{itemize}
\item {Proveniência:(Do gr. \textunderscore steganos\textunderscore  + \textunderscore pous\textunderscore )}
\end{itemize}
Ordem de aves, que tem por typo o pelicano.
\section{Estegnose}
\begin{itemize}
\item {Grp. gram.:f.}
\end{itemize}
\begin{itemize}
\item {Utilização:Med.}
\end{itemize}
\begin{itemize}
\item {Proveniência:(Gr. \textunderscore stegnosis\textunderscore )}
\end{itemize}
Constricção dos poros e dos vasos.
Suppressão das evacuações.
\section{Estegnótico}
\begin{itemize}
\item {Grp. gram.:adj.}
\end{itemize}
\begin{itemize}
\item {Proveniência:(Gr. \textunderscore stegnotikos\textunderscore )}
\end{itemize}
Relativo a estegnose.
O mesmo que \textunderscore adstringente\textunderscore .
\section{Estegómia}
\begin{itemize}
\item {Grp. gram.:m.}
\end{itemize}
Mosquito, que se considera veiculo do micróbio da febre amarela.
\section{Estegómya}
\begin{itemize}
\item {Grp. gram.:m.}
\end{itemize}
Mosquito, que se considera vehiculo do micróbio da febre amarela.
\section{Esteio}
\begin{itemize}
\item {Grp. gram.:m.}
\end{itemize}
\begin{itemize}
\item {Utilização:Fig.}
\end{itemize}
\begin{itemize}
\item {Proveniência:(Do ingl. \textunderscore stay\textunderscore ?)}
\end{itemize}
Vara, peça oblonga de madeira ou metal, com que se ampara ou sustém alguma coisa.
Amparo, sustentáculo.
\section{Esteio}
\begin{itemize}
\item {Grp. gram.:m.}
\end{itemize}
\begin{itemize}
\item {Utilização:Prov.}
\end{itemize}
\begin{itemize}
\item {Utilização:trasm.}
\end{itemize}
\begin{itemize}
\item {Utilização:Mad}
\end{itemize}
O mesmo que \textunderscore amariço\textunderscore .
O mesmo que \textunderscore estiada\textunderscore .
\section{Esteira}
\begin{itemize}
\item {Grp. gram.:f.}
\end{itemize}
\begin{itemize}
\item {Utilização:Fig.}
\end{itemize}
\begin{itemize}
\item {Utilização:Gír.}
\end{itemize}
\begin{itemize}
\item {Utilização:Náut.}
\end{itemize}
\begin{itemize}
\item {Grp. gram.:M.}
\end{itemize}
\begin{itemize}
\item {Utilização:Bras. do N}
\end{itemize}
\begin{itemize}
\item {Proveniência:(Lat. \textunderscore storea\textunderscore ?)}
\end{itemize}
Tecido de junco, tabúa, etc.
Rasto, sulco, que a embarcação deixa na água, quando navega.
Reflexo.
Vestígio.
Exemplo.
Estrada.
Parte inferior da vela.
Vaqueiro, que, na conducção do gado, segue atrás do cabeceira.
\section{Esteirada}
\begin{itemize}
\item {Grp. gram.:f.}
\end{itemize}
\begin{itemize}
\item {Utilização:Prov.}
\end{itemize}
\begin{itemize}
\item {Utilização:trasm.}
\end{itemize}
\begin{itemize}
\item {Utilização:minh.}
\end{itemize}
\begin{itemize}
\item {Proveniência:(De \textunderscore esteirar\textunderscore ? Ou corr. de \textunderscore estoirada\textunderscore ?)}
\end{itemize}
Bordoada nas costas, em cheio.
Quéda de corpo, no chão.
\section{Esteiralho}
\begin{itemize}
\item {Grp. gram.:m.}
\end{itemize}
\begin{itemize}
\item {Utilização:T. da -da-Foz}
\end{itemize}
\begin{itemize}
\item {Utilização:Fig.}
\end{itemize}
\begin{itemize}
\item {Proveniência:(De \textunderscore esteira\textunderscore )}
\end{itemize}
Apparelho para a pesca da taínha e de outros peixes saltadores, que consiste numa porção de esteiras de bunho, ligadas umas ás outras.
\section{Esteirão}
\begin{itemize}
\item {Grp. gram.:m.}
\end{itemize}
Grande esteira.
\section{Esteirar}
\begin{itemize}
\item {Grp. gram.:v. t.}
\end{itemize}
\begin{itemize}
\item {Grp. gram.:V. i.}
\end{itemize}
Cobrir ou forrar com esteira.
Atapetar.
Percorrer, navegando:«\textunderscore esteira o mar estreito de levante.\textunderscore »\textunderscore Viriato Trág.\textunderscore , VII, 26.
Navegar.
Tornar esteiro (o rio). Cf. Serpa Pinto, II, 74.
\section{Esteireiro}
\begin{itemize}
\item {Grp. gram.:m.}
\end{itemize}
Vendedor ou fabricante de esteiras.
\section{Esteiro}
\begin{itemize}
\item {Grp. gram.:f.}
\end{itemize}
\begin{itemize}
\item {Proveniência:(Do lat. \textunderscore aestuarium\textunderscore )}
\end{itemize}
Braço de rio ou de mar, que se estende pela terra.
\section{Estela}
\begin{itemize}
\item {Grp. gram.:f.}
\end{itemize}
\begin{itemize}
\item {Proveniência:(Gr. \textunderscore stele\textunderscore )}
\end{itemize}
Monólitho.
Espécie de columna, destinada a têr uma inscripção.
\section{Estelante}
\begin{itemize}
\item {Grp. gram.:adj.}
\end{itemize}
\begin{itemize}
\item {Proveniência:(Lat. \textunderscore stellans\textunderscore )}
\end{itemize}
Em que há estrêla.
Brilhante.
\section{Estelar}
\begin{itemize}
\item {Grp. gram.:adj.}
\end{itemize}
\begin{itemize}
\item {Proveniência:(Do lat. \textunderscore stellaris\textunderscore )}
\end{itemize}
Relativo a estrêlas.
\section{Estelegrafia}
\begin{itemize}
\item {Grp. gram.:f.}
\end{itemize}
\begin{itemize}
\item {Proveniência:(Do gr. \textunderscore stele\textunderscore  + \textunderscore graphein\textunderscore )}
\end{itemize}
Arte de fazer inscripções em colunas.
\section{Estelegráfico}
\begin{itemize}
\item {Grp. gram.:adj.}
\end{itemize}
Relativo a estelografia.
\section{Estelegraphia}
\begin{itemize}
\item {Grp. gram.:f.}
\end{itemize}
\begin{itemize}
\item {Proveniência:(Do gr. \textunderscore stele\textunderscore  + \textunderscore graphein\textunderscore )}
\end{itemize}
Arte de fazer inscripções em columnas.
\section{Estelegráphico}
\begin{itemize}
\item {Grp. gram.:adj.}
\end{itemize}
Relativo a estelographia.
\section{Estelerídeos}
\begin{itemize}
\item {Grp. gram.:m. pl.}
\end{itemize}
O mesmo que \textunderscore asterídeos\textunderscore .--É termo afrancesado. Melhor seria \textunderscore estalerídeos\textunderscore .
\section{Estelião}
\begin{itemize}
\item {Grp. gram.:m.}
\end{itemize}
\begin{itemize}
\item {Proveniência:(Lat. \textunderscore stellio\textunderscore , de \textunderscore stella\textunderscore )}
\end{itemize}
Espécie de lagarto, que nas costas apresenta manchas parecidas a estrêlas.
\section{Estelífero}
\begin{itemize}
\item {Grp. gram.:adj.}
\end{itemize}
\begin{itemize}
\item {Proveniência:(Lat. \textunderscore stellifer\textunderscore )}
\end{itemize}
Em que há estrêlas.
Estrelado.
Estelante.
\section{Estelígero}
\begin{itemize}
\item {Grp. gram.:adj.}
\end{itemize}
\begin{itemize}
\item {Proveniência:(Lat. stelliger)}
\end{itemize}
O mesmo que \textunderscore estelífero\textunderscore .
\section{Estelino}
\begin{itemize}
\item {Grp. gram.:m.}
\end{itemize}
Pequeno pêso antigo, usado em ourivezaria.
(Da mesma or. que \textunderscore esterlino\textunderscore )
\section{Estélio}
\begin{itemize}
\item {Grp. gram.:m.}
\end{itemize}
O mesmo que \textunderscore estelião\textunderscore .
\section{Estelionatário}
\begin{itemize}
\item {Grp. gram.:m.}
\end{itemize}
Aquele que pratíca estelionato.
\section{Estelionato}
\begin{itemize}
\item {Grp. gram.:m.}
\end{itemize}
\begin{itemize}
\item {Proveniência:(Lat. \textunderscore stellionatus\textunderscore )}
\end{itemize}
Fraude de quem cede, vende ou obriga uma coisa, ocultando que esta já estava cedida, vendida ou obrigada a outrem.
\section{Estelite}
\begin{itemize}
\item {Grp. gram.:f.}
\end{itemize}
\begin{itemize}
\item {Proveniência:(Do lat. \textunderscore stella\textunderscore )}
\end{itemize}
Mineral escocês, que cristaliza, em fórma de estrêlas.
\section{Estellante}
\begin{itemize}
\item {Grp. gram.:adj.}
\end{itemize}
\begin{itemize}
\item {Proveniência:(Lat. \textunderscore stellans\textunderscore )}
\end{itemize}
Em que há estrêlla.
Brilhante.
\section{Estellar}
\begin{itemize}
\item {Grp. gram.:adj.}
\end{itemize}
\begin{itemize}
\item {Proveniência:(Do lat. \textunderscore stellaris\textunderscore )}
\end{itemize}
Relativo a estrêllas.
\section{Estellerídeos}
\begin{itemize}
\item {Grp. gram.:m. pl.}
\end{itemize}
O mesmo que \textunderscore asterídeos\textunderscore .--É termo afrancesado. Melhor seria \textunderscore estallerídeos\textunderscore .
\section{Estellião}
\begin{itemize}
\item {Grp. gram.:m.}
\end{itemize}
\begin{itemize}
\item {Proveniência:(Lat. \textunderscore stellio\textunderscore , de \textunderscore stella\textunderscore )}
\end{itemize}
Espécie de lagarto, que nas costas apresenta manchas parecidas a estrêllas.
\section{Estellífero}
\begin{itemize}
\item {Grp. gram.:adj.}
\end{itemize}
\begin{itemize}
\item {Proveniência:(Lat. \textunderscore stellifer\textunderscore )}
\end{itemize}
Em que há estrêllas.
Estrellado.
Estellante.
\section{Estellígero}
\begin{itemize}
\item {Grp. gram.:adj.}
\end{itemize}
\begin{itemize}
\item {Proveniência:(Lat. \textunderscore stelliger\textunderscore )}
\end{itemize}
O mesmo que \textunderscore estellífero\textunderscore .
\section{Estéllio}
\begin{itemize}
\item {Grp. gram.:m.}
\end{itemize}
O mesmo que \textunderscore estellião\textunderscore .
\section{Estellionatário}
\begin{itemize}
\item {Grp. gram.:m.}
\end{itemize}
Aquelle que pratíca estellionato.
\section{Estellionato}
\begin{itemize}
\item {Grp. gram.:m.}
\end{itemize}
\begin{itemize}
\item {Proveniência:(Lat. \textunderscore stellionatus\textunderscore )}
\end{itemize}
Fraude de quem cede, vende ou obriga uma coisa, occultando que esta já estava cedida, vendida ou obrigada a outrem.
\section{Estellite}
\begin{itemize}
\item {Grp. gram.:f.}
\end{itemize}
\begin{itemize}
\item {Proveniência:(Do lat. \textunderscore stella\textunderscore )}
\end{itemize}
Mineral escocês, que crystalliza, em fórma de estrêllas.
\section{Estema}
\begin{itemize}
\item {Grp. gram.:m.}
\end{itemize}
\begin{itemize}
\item {Proveniência:(Lat. \textunderscore stemma\textunderscore )}
\end{itemize}
Grinalda.
Árvore genealógica.
\section{Estemma}
\begin{itemize}
\item {Grp. gram.:m.}
\end{itemize}
\begin{itemize}
\item {Proveniência:(Lat. \textunderscore stemma\textunderscore )}
\end{itemize}
Grinalda.
Árvore genealógica.
\section{Estendagem}
\begin{itemize}
\item {Grp. gram.:f.}
\end{itemize}
O mesmo que \textunderscore estendedoiro\textunderscore . Cf. \textunderscore Inquér. Industr.\textunderscore , 2.^a p., l. III, 229 e 234.
\section{Estendal}
\begin{itemize}
\item {Grp. gram.:m.}
\end{itemize}
\begin{itemize}
\item {Utilização:Fig.}
\end{itemize}
\begin{itemize}
\item {Proveniência:(De \textunderscore estender\textunderscore )}
\end{itemize}
Estendedoiro.
Descampado.
Larga esplanação de coisas ou de assumptos; exposição ostentosa.
\section{Estendaria}
\begin{itemize}
\item {Grp. gram.:f.}
\end{itemize}
O mesmo que \textunderscore estendedoiro\textunderscore . Cf. \textunderscore Inquér. Industr.\textunderscore , 2.^a p., l. III, 235.
\section{Estendedoiro}
\begin{itemize}
\item {Grp. gram.:m.}
\end{itemize}
\begin{itemize}
\item {Proveniência:(De \textunderscore estender\textunderscore )}
\end{itemize}
Lugar, em que se estende alguma coisa.
\section{Estendedor}
\begin{itemize}
\item {Grp. gram.:m.  e  adj.}
\end{itemize}
O que estende.
\section{Estendedouro}
\begin{itemize}
\item {Grp. gram.:m.}
\end{itemize}
\begin{itemize}
\item {Proveniência:(De \textunderscore estender\textunderscore )}
\end{itemize}
Lugar, em que se estende alguma coisa.
\section{Estendedura}
\begin{itemize}
\item {Grp. gram.:f.}
\end{itemize}
Acto de \textunderscore estender\textunderscore .
\section{Estender}
\begin{itemize}
\item {Grp. gram.:v. t.}
\end{itemize}
\begin{itemize}
\item {Utilização:Fig.}
\end{itemize}
\begin{itemize}
\item {Utilização:Fam.}
\end{itemize}
\begin{itemize}
\item {Grp. gram.:V. i.}
\end{itemize}
\begin{itemize}
\item {Grp. gram.:V. p.}
\end{itemize}
\begin{itemize}
\item {Utilização:Escol.}
\end{itemize}
\begin{itemize}
\item {Utilização:Fig.}
\end{itemize}
\begin{itemize}
\item {Proveniência:(Lat. \textunderscore extendere\textunderscore )}
\end{itemize}
Alargar.
Desenvolver.
Desdobrar: \textunderscore estender um lençol\textunderscore .
Dilatar.
Deitar ao comprido.
Amplificar.
Oferecer: \textunderscore estendeu-lhe uma cadeira\textunderscore .
Prolongar: \textunderscore estender um discurso\textunderscore .
Prostrar.
Derrotar, discutindo.
Tornar-se comprido, dilatar-se em comprimento.
Alongar-se.
Ir até.
Ramificar-se.
Pôr-se em fileira.
Divulgar-se.
Dar má lição ou fazer mau exame (o estudante).
Durar.
Sêr applicável.
Attingir.
Ficar vencido, discutindo.
\section{Estenderete}
\begin{itemize}
\item {fónica:derê}
\end{itemize}
\begin{itemize}
\item {Grp. gram.:m.}
\end{itemize}
\begin{itemize}
\item {Utilização:Fam.}
\end{itemize}
\begin{itemize}
\item {Proveniência:(De \textunderscore estender\textunderscore )}
\end{itemize}
Jôgo, em que o jogador estende as cartas, não as tendo semelhantes ás que estão na mesa.
Má lição ou mau exemplo de estudante.
Successo desairoso.
\section{Estendidamente}
\begin{itemize}
\item {Grp. gram.:adv.}
\end{itemize}
\begin{itemize}
\item {Proveniência:(De \textunderscore estender\textunderscore )}
\end{itemize}
Largamente, por extenso.
\section{Estendível}
\begin{itemize}
\item {Grp. gram.:adj.}
\end{itemize}
Que se póde estender.
\section{Estendudo}
\begin{itemize}
\item {Grp. gram.:adj.}
\end{itemize}
\begin{itemize}
\item {Utilização:Ant.}
\end{itemize}
\begin{itemize}
\item {Proveniência:(De \textunderscore estender\textunderscore )}
\end{itemize}
Que se estendeu ou se ampliou.
\section{Esteno...}
\begin{itemize}
\item {Grp. gram.:pref.}
\end{itemize}
\begin{itemize}
\item {Proveniência:(Do gr. \textunderscore stenos\textunderscore )}
\end{itemize}
(que significa \textunderscore estreito\textunderscore )
\section{Estenocardia}
\begin{itemize}
\item {Grp. gram.:f.}
\end{itemize}
\begin{itemize}
\item {Proveniência:(Do gr. \textunderscore stenos\textunderscore  + \textunderscore kardia\textunderscore )}
\end{itemize}
Angina do peito.
\section{Estenocefalia}
\begin{itemize}
\item {Grp. gram.:f.}
\end{itemize}
Estado ou qualidade de estenocéfalo.
\section{Estenocéfalo}
\begin{itemize}
\item {Grp. gram.:adj.}
\end{itemize}
\begin{itemize}
\item {Proveniência:(Do gr. \textunderscore stenos\textunderscore  + \textunderscore kepale\textunderscore )}
\end{itemize}
Que tem a cabeça estreita.
\section{Estenocephalia}
\begin{itemize}
\item {Grp. gram.:f.}
\end{itemize}
Estado ou qualidade de estenocéphalo.
\section{Estenocéphalo}
\begin{itemize}
\item {Grp. gram.:adj.}
\end{itemize}
\begin{itemize}
\item {Proveniência:(Do gr. \textunderscore stenos\textunderscore  + \textunderscore kepale\textunderscore )}
\end{itemize}
Que tem a cabeça estreita.
\section{Estenografar}
\begin{itemize}
\item {Grp. gram.:v. t.}
\end{itemize}
Escrever estenograficamente; taquigrafar.
(Cp. \textunderscore estenografia\textunderscore )
\section{Estenografia}
\begin{itemize}
\item {Grp. gram.:f.}
\end{itemize}
\begin{itemize}
\item {Proveniência:(Do gr. \textunderscore stenos\textunderscore  + \textunderscore graphein\textunderscore )}
\end{itemize}
Arte de escrever por certo processo de abreviaturas e tão rapidamente como se fala.
Taquigrafia.
\section{Estenograficamente}
\begin{itemize}
\item {Grp. gram.:adv.}
\end{itemize}
De modo estenográfico.
\section{Estenográfico}
\begin{itemize}
\item {Grp. gram.:adj.}
\end{itemize}
Relativo a estenografia.
\section{Estenógrafo}
\begin{itemize}
\item {Grp. gram.:m.}
\end{itemize}
Aquele que é versado em estenografia.
Espécie de insecto, nocivo aos arvoredos, e descoberto recentemente nas matas de Portugal.
\section{Estenographar}
\begin{itemize}
\item {Grp. gram.:v. t.}
\end{itemize}
Escrever estenographicamente; tachygraphar.
(Cp. \textunderscore estenographia\textunderscore )
\section{Estenographia}
\begin{itemize}
\item {Grp. gram.:f.}
\end{itemize}
\begin{itemize}
\item {Proveniência:(Do gr. \textunderscore stenos\textunderscore  + \textunderscore graphein\textunderscore )}
\end{itemize}
Arte de escrever por certo processo de abreviaturas e tão rapidamente como se fala.
Tachygraphia.
\section{Estenographicamente}
\begin{itemize}
\item {Grp. gram.:adv.}
\end{itemize}
De modo estenográphico.
\section{Estenográphico}
\begin{itemize}
\item {Grp. gram.:adj.}
\end{itemize}
Relativo a estenographia.
\section{Estenógrapho}
\begin{itemize}
\item {Grp. gram.:m.}
\end{itemize}
Aquelle que é versado em estenographia.
Espécie de insecto, nocivo aos arvoredos, e descoberto recentemente nas matas de Portugal.
\section{Estenos}
\begin{itemize}
\item {Grp. gram.:m. pl.}
\end{itemize}
\begin{itemize}
\item {Proveniência:(Do gr. \textunderscore stenos\textunderscore )}
\end{itemize}
Insectos brachelytros, da ordem dos coleópteros.
\section{Estenose}
\begin{itemize}
\item {Grp. gram.:f.}
\end{itemize}
\begin{itemize}
\item {Proveniência:(Do gr. \textunderscore stenos\textunderscore , apertado)}
\end{itemize}
Apêrto de qualquer canal orgânico.
\section{Estenotermes}
\begin{itemize}
\item {Grp. gram.:m. pl.}
\end{itemize}
\begin{itemize}
\item {Utilização:Zool.}
\end{itemize}
\begin{itemize}
\item {Proveniência:(Do gr. \textunderscore stenos\textunderscore  + \textunderscore therme\textunderscore )}
\end{itemize}
Animaes, a que as variações da temperatura abreviam a vida ou lh'a fazem perigar.
\section{Estentor}
\begin{itemize}
\item {Grp. gram.:m.}
\end{itemize}
\begin{itemize}
\item {Proveniência:(Do lat. \textunderscore Stentor\textunderscore , n. p.)}
\end{itemize}
Pessôa, que tem voz muito forte: \textunderscore a sua voz de estentor\textunderscore .
\section{Estentóreo}
\begin{itemize}
\item {Grp. gram.:adj.}
\end{itemize}
\begin{itemize}
\item {Proveniência:(Lat. \textunderscore stentorius\textunderscore .)}
\end{itemize}
Relativo a estentor.
Que tem voz forte.
Que é forte, (falando-se da voz).
\section{Estentórico}
\begin{itemize}
\item {Grp. gram.:adj.}
\end{itemize}
O mesmo que \textunderscore estentóreo\textunderscore .
\section{Estentorosamente}
\begin{itemize}
\item {Grp. gram.:adv.}
\end{itemize}
De modo estentoroso. Cf. Camillo, \textunderscore Estrêl. Prop.\textunderscore , 99.
\section{Estentoroso}
\begin{itemize}
\item {Grp. gram.:adj.}
\end{itemize}
O mesmo que \textunderscore estentóreo\textunderscore .
\section{Estepe}
\begin{itemize}
\item {Grp. gram.:f.}
\end{itemize}
Planície inculta e vasta, na Rússia e na América.
(Do russo \textunderscore stepi\textunderscore )
\section{Estephânico}
\begin{itemize}
\item {Grp. gram.:adj.}
\end{itemize}
\begin{itemize}
\item {Utilização:Ant.}
\end{itemize}
Relativo ou pertencente ao estephânio.
\section{Estephánio}
\begin{itemize}
\item {Grp. gram.:m.}
\end{itemize}
\begin{itemize}
\item {Proveniência:(Do gr. \textunderscore stephanos\textunderscore )}
\end{itemize}
Ponto craniométrico, determinado pela sutura coronal e pela crista do osso temporal.
\section{Estephanómia}
\begin{itemize}
\item {Grp. gram.:f.}
\end{itemize}
Animal marinho, diáphano, com a apparência de uma grinalda animada. Cf. A. F. Simões, \textunderscore Da Beiramar\textunderscore , 15.
\section{Estephanóphoro}
\begin{itemize}
\item {Grp. gram.:adj.}
\end{itemize}
\begin{itemize}
\item {Proveniência:(Gr. \textunderscore stephanophoros\textunderscore )}
\end{itemize}
Dizia-se das mesas, em que se collocavam as corôas destinadas aos vencedores dos jogos olýmpicos.
\section{Estercada}
\begin{itemize}
\item {Grp. gram.:f.}
\end{itemize}
\begin{itemize}
\item {Utilização:Prov.}
\end{itemize}
\begin{itemize}
\item {Utilização:minh.}
\end{itemize}
Acto de estercar.
Barulho, desordem.
\section{Estercador}
\begin{itemize}
\item {Grp. gram.:m.  e  adj.}
\end{itemize}
O que esterca.
\section{Estercadura}
\begin{itemize}
\item {Grp. gram.:f.}
\end{itemize}
O mesmo que \textunderscore estercada\textunderscore .
\section{Estercar}
\begin{itemize}
\item {Grp. gram.:v. t.}
\end{itemize}
\begin{itemize}
\item {Grp. gram.:V. i.}
\end{itemize}
Deitar estêrco em.
Adubar com estêrco.
Estrumar.
Defecar, (falando-se de animaes).
\section{Estêrco}
\begin{itemize}
\item {Grp. gram.:m.}
\end{itemize}
\begin{itemize}
\item {Utilização:Fig.}
\end{itemize}
\begin{itemize}
\item {Utilização:Prov.}
\end{itemize}
\begin{itemize}
\item {Utilização:minh.}
\end{itemize}
\begin{itemize}
\item {Proveniência:(Lat. \textunderscore stercus\textunderscore )}
\end{itemize}
Excremento de animal.
Adubo vegetal para os terrenos.
Estrume.
Lixo, sujidade.
Pessôa ou coisa vil.
Barulho, desordem.
\section{Estenia}
\begin{itemize}
\item {Grp. gram.:f.}
\end{itemize}
\begin{itemize}
\item {Utilização:Med.}
\end{itemize}
\begin{itemize}
\item {Proveniência:(Do gr. \textunderscore sthenos\textunderscore , fôrça)}
\end{itemize}
Excesso de fôrça.
Exaltação da acção orgânica.
\section{Estênico}
\begin{itemize}
\item {Grp. gram.:adj.}
\end{itemize}
Relativo a estenia.
\section{Estercoral}
\begin{itemize}
\item {Grp. gram.:adj.}
\end{itemize}
\begin{itemize}
\item {Proveniência:(Do lat. \textunderscore stercus\textunderscore )}
\end{itemize}
Relativo a excrementos.
\section{Estercorário}
\begin{itemize}
\item {Grp. gram.:adj.}
\end{itemize}
\begin{itemize}
\item {Grp. gram.:M. pl.}
\end{itemize}
\begin{itemize}
\item {Proveniência:(Lat. \textunderscore stercorarius\textunderscore )}
\end{itemize}
Que cresce ou vive no estêrco.
Relativo a estêrco.
Gênero de aves palmípedes.
\section{Estercoreiro}
\begin{itemize}
\item {Grp. gram.:m.}
\end{itemize}
\begin{itemize}
\item {Grp. gram.:Adj.}
\end{itemize}
\begin{itemize}
\item {Proveniência:(Do lat. \textunderscore stercorarius\textunderscore )}
\end{itemize}
Espêcie de escaravelho.
O mesmo que \textunderscore estercoral\textunderscore .
\section{Estercorosamente}
\begin{itemize}
\item {Grp. gram.:adv.}
\end{itemize}
De modo estercoroso. Cf. Camillo, \textunderscore Vulcões\textunderscore , 15.
\section{Estercoroso}
\begin{itemize}
\item {Grp. gram.:adj.}
\end{itemize}
\begin{itemize}
\item {Proveniência:(Do lat. \textunderscore stercus\textunderscore )}
\end{itemize}
Immundo, que tem estêrco.
Impróprio de gente limpa.
Grosseiro.
\section{Estercúlia}
\begin{itemize}
\item {Grp. gram.:f.}
\end{itemize}
Gênero de plantas tropicaes, a que pertence a cola.
\section{Esterculiáceas}
\begin{itemize}
\item {Grp. gram.:f. pl.}
\end{itemize}
Família de plantas que têm por typo a \textunderscore estercúlia\textunderscore .
\section{Estere}
\begin{itemize}
\item {Grp. gram.:m.}
\end{itemize}
\begin{itemize}
\item {Proveniência:(Do gr. \textunderscore stereos\textunderscore )}
\end{itemize}
Medida de volume para madeiras, equivalente a um metro cúbico.
\section{Estéreo}
\begin{itemize}
\item {Grp. gram.:m.}
\end{itemize}
O mesmo ou melhor que \textunderscore estere\textunderscore .
\section{Estereóbata}
\begin{itemize}
\item {Grp. gram.:m.}
\end{itemize}
\begin{itemize}
\item {Utilização:Archit.}
\end{itemize}
\begin{itemize}
\item {Proveniência:(Do gr. \textunderscore stereos\textunderscore  + \textunderscore bates\textunderscore )}
\end{itemize}
Sóco continuado, que não tem cornija e sustenta um edifício.
\section{Estereochímica}
\begin{itemize}
\item {fónica:qui}
\end{itemize}
\begin{itemize}
\item {Grp. gram.:f.}
\end{itemize}
\begin{itemize}
\item {Proveniência:(Do gr. \textunderscore stereos\textunderscore  + \textunderscore khumia\textunderscore )}
\end{itemize}
Processo de philosophia chímica, que considera os átomos como determinando, pela sua posição na molécula, a configuração de um sólido.
\section{Estereochromia}
\begin{itemize}
\item {Grp. gram.:f.}
\end{itemize}
\begin{itemize}
\item {Proveniência:(Do gr. \textunderscore stereos\textunderscore  + \textunderscore khroma\textunderscore )}
\end{itemize}
Méthodo de fixar côres em pinturas de paredes, recobrindo as tintas com uma solução de silicato de potássio.
\section{Estereocromia}
\begin{itemize}
\item {Grp. gram.:f.}
\end{itemize}
\begin{itemize}
\item {Proveniência:(Do gr. \textunderscore stereos\textunderscore  + \textunderscore khroma\textunderscore )}
\end{itemize}
Método de fixar côres em pinturas de paredes, recobrindo as tintas com uma solução de silicato de potássio.
\section{Estereodinâmica}
\begin{itemize}
\item {Grp. gram.:f.}
\end{itemize}
\begin{itemize}
\item {Proveniência:(Do gr. \textunderscore stereos\textunderscore  + \textunderscore dunamikos\textunderscore )}
\end{itemize}
Parte da Mecânica, que se ocupa das leis do movimento dos corpos sólidos.
\section{Estereodynâmica}
\begin{itemize}
\item {Grp. gram.:f.}
\end{itemize}
\begin{itemize}
\item {Proveniência:(Do gr. \textunderscore stereos\textunderscore  + \textunderscore dunamikos\textunderscore )}
\end{itemize}
Parte da Mechânica, que se occupa das leis do movimento dos corpos sólidos.
\section{Estereografia}
\begin{itemize}
\item {Grp. gram.:f.}
\end{itemize}
\begin{itemize}
\item {Proveniência:(De \textunderscore estereógrafo\textunderscore )}
\end{itemize}
Arte de representar os sólidos num plano.
\section{Estereográfico}
\begin{itemize}
\item {Grp. gram.:adj.}
\end{itemize}
Relativo a estereografia.
\section{Estereógrafo}
\begin{itemize}
\item {Grp. gram.:m.}
\end{itemize}
\begin{itemize}
\item {Proveniência:(Do gr. \textunderscore stereos\textunderscore  + \textunderscore graphein\textunderscore )}
\end{itemize}
Instrumento de craniografia, que da o contôrno do crânio, observado por qualquer das faces.
\section{Estereographia}
\begin{itemize}
\item {Grp. gram.:f.}
\end{itemize}
\begin{itemize}
\item {Proveniência:(De \textunderscore estereógrapho\textunderscore )}
\end{itemize}
Arte de representar os sólidos num plano.
\section{Estereográphico}
\begin{itemize}
\item {Grp. gram.:adj.}
\end{itemize}
Relativo a estereographia.
\section{Estereógrapho}
\begin{itemize}
\item {Grp. gram.:m.}
\end{itemize}
\begin{itemize}
\item {Proveniência:(Do gr. \textunderscore stereos\textunderscore  + \textunderscore graphein\textunderscore )}
\end{itemize}
Instrumento de craniographia, que da o contôrno do crânio, observado por qualquer das faces.
\section{Estereologia}
\begin{itemize}
\item {Grp. gram.:f.}
\end{itemize}
\begin{itemize}
\item {Proveniência:(Do gr. \textunderscore stereos\textunderscore  + \textunderscore logos\textunderscore )}
\end{itemize}
Estudo das partes sólidas dos corpos vivos.
\section{Estereológico}
\begin{itemize}
\item {Grp. gram.:adj.}
\end{itemize}
Relativo a estereologia.
\section{Estereometria}
\begin{itemize}
\item {Grp. gram.:f.}
\end{itemize}
\begin{itemize}
\item {Proveniência:(Do gr. \textunderscore stereos\textunderscore  + \textunderscore metron\textunderscore )}
\end{itemize}
Parte da Geometria, que ensina a medir os sólidos.
\section{Estercométrico}
\begin{itemize}
\item {Grp. gram.:adj.}
\end{itemize}
Relativo a estereometria.
\section{Estereómetro}
\begin{itemize}
\item {Grp. gram.:m.}
\end{itemize}
\begin{itemize}
\item {Proveniência:(Do gr. \textunderscore stereos\textunderscore  + \textunderscore metron\textunderscore )}
\end{itemize}
Instrumento, para medir sólidos, (em Geometria).
\section{Estereoquímica}
\begin{itemize}
\item {Grp. gram.:f.}
\end{itemize}
\begin{itemize}
\item {Proveniência:(Do gr. \textunderscore stereos\textunderscore  + \textunderscore khumia\textunderscore )}
\end{itemize}
Processo de filosofia química, que considera os átomos como determinando, pela sua posição na molécula, a configuração de um sólido.
\section{Estereorama}
\begin{itemize}
\item {Grp. gram.:m.}
\end{itemize}
\begin{itemize}
\item {Proveniência:(Do gr. \textunderscore stereos\textunderscore  + \textunderscore orama\textunderscore )}
\end{itemize}
Carta topográphica em relêvo.
\section{Estereoscópico}
\begin{itemize}
\item {Grp. gram.:adj.}
\end{itemize}
Relativo ao estereoscópio.
\section{Estereoscópio}
\begin{itemize}
\item {Grp. gram.:m.}
\end{itemize}
\begin{itemize}
\item {Proveniência:(Do gr. \textunderscore stereos\textunderscore  + \textunderscore skopein\textunderscore )}
\end{itemize}
Instrumento, que nos dá a sensação do relêvo e da perspectiva, por meio de imagens planas.
\section{Estereostática}
\begin{itemize}
\item {Grp. gram.:f.}
\end{itemize}
\begin{itemize}
\item {Proveniência:(De \textunderscore estereostático\textunderscore )}
\end{itemize}
Parte da Phýsica, que trata do equilíbrio dos corpos sólidos.
\section{Estereostático}
\begin{itemize}
\item {Grp. gram.:adj.}
\end{itemize}
\begin{itemize}
\item {Proveniência:(Do gr. \textunderscore stereos\textunderscore  + \textunderscore statikos\textunderscore )}
\end{itemize}
Relativo á estereostática.
\section{Estereotipagem}
\begin{itemize}
\item {Grp. gram.:f.}
\end{itemize}
Acto de estereotipar.
\section{Estereotipar}
\begin{itemize}
\item {Grp. gram.:v. t.}
\end{itemize}
Reduzir a estereótipo.
Imprimir pelo processo da estereotipia.
(Cp. \textunderscore estereótipo\textunderscore )
\section{Estereotipia}
\begin{itemize}
\item {Grp. gram.:f.}
\end{itemize}
\begin{itemize}
\item {Proveniência:(De \textunderscore esteriótipo\textunderscore )}
\end{itemize}
Arte de reproduzir, com o auxilio de uma liga metálica, a página, primeiramente composta em caracteres móveis.
\section{Estereotipicamente}
\begin{itemize}
\item {Grp. gram.:adv.}
\end{itemize}
Do modo estereotípico.
\section{Estereotípico}
\begin{itemize}
\item {Grp. gram.:adj.}
\end{itemize}
Relativo a estereotipia.
\section{Estereótipo}
\begin{itemize}
\item {Grp. gram.:m.}
\end{itemize}
\begin{itemize}
\item {Proveniência:(Do gr. \textunderscore stereos\textunderscore  + \textunderscore tupos\textunderscore )}
\end{itemize}
Impressão ou obra impressa numa prancha de caracteres fixos.
\section{Estereotomia}
\begin{itemize}
\item {Grp. gram.:f.}
\end{itemize}
\begin{itemize}
\item {Proveniência:(Do gr. \textunderscore stereos\textunderscore  + \textunderscore tome\textunderscore )}
\end{itemize}
Sciência, que trata do córte ou divisão scientífica dos materiaes de construcção.
\section{Estereotypagem}
\begin{itemize}
\item {Grp. gram.:f.}
\end{itemize}
Acto de estereotypar.
\section{Estereotypar}
\begin{itemize}
\item {Grp. gram.:v. t.}
\end{itemize}
Reduzir a estereótypo.
Imprimir pelo processo da estereotypia.
(Cp. \textunderscore estereótypo\textunderscore )
\section{Estereotypia}
\begin{itemize}
\item {Grp. gram.:f.}
\end{itemize}
\begin{itemize}
\item {Proveniência:(De \textunderscore esteriótypo\textunderscore )}
\end{itemize}
Arte de reproduzir, com o auxilio de uma liga metállica, a página, primeiramente composta em caracteres móveis.
\section{Estereotypicamente}
\begin{itemize}
\item {Grp. gram.:adv.}
\end{itemize}
Do modo estereotýpico.
\section{Estereotýpico}
\begin{itemize}
\item {Grp. gram.:adj.}
\end{itemize}
Relativo a estereotypia.
\section{Estereótypo}
\begin{itemize}
\item {Grp. gram.:m.}
\end{itemize}
\begin{itemize}
\item {Proveniência:(Do gr. \textunderscore stereos\textunderscore  + \textunderscore tupos\textunderscore )}
\end{itemize}
Impressão ou obra impressa numa prancha de caracteres fixos.
\section{Esterigma}
\begin{itemize}
\item {Grp. gram.:m.}
\end{itemize}
\begin{itemize}
\item {Utilização:Bot.}
\end{itemize}
\begin{itemize}
\item {Proveniência:(Gr. \textunderscore sterigma\textunderscore )}
\end{itemize}
Espécie de fruto com muitos lóculos.
\section{Estéril}
\begin{itemize}
\item {Grp. gram.:adj.}
\end{itemize}
\begin{itemize}
\item {Grp. gram.:M.}
\end{itemize}
\begin{itemize}
\item {Grp. gram.:Pl.}
\end{itemize}
\begin{itemize}
\item {Grp. gram.:Pl.}
\end{itemize}
\begin{itemize}
\item {Utilização:ant.}
\end{itemize}
\begin{itemize}
\item {Proveniência:(Lat. \textunderscore sterilis\textunderscore )}
\end{itemize}
Que não dá fruto.
Que não produz: \textunderscore campo estéril\textunderscore .
Que não póde têr filhos: \textunderscore mulher estéril\textunderscore .
De que se não tira vantagem.
Parte do minério, cujo valor não compensa as despesas de exploração.
Estéreis.
\textunderscore Estériles\textunderscore . Cf. Bern. Cruz, \textunderscore Chrón. de D. Sebast.\textunderscore , c. I.
\section{Esterilecer}
\begin{itemize}
\item {Grp. gram.:v. t.}
\end{itemize}
\begin{itemize}
\item {Grp. gram.:V. i.}
\end{itemize}
\begin{itemize}
\item {Proveniência:(Lat. \textunderscore sterilescere\textunderscore )}
\end{itemize}
Esterilizar.
Esterilizar-se.
\section{Esterilidade}
\begin{itemize}
\item {Grp. gram.:f.}
\end{itemize}
\begin{itemize}
\item {Proveniência:(Lat. \textunderscore sterilitas\textunderscore )}
\end{itemize}
Qualidade de estéril.
\section{Esterilização}
\begin{itemize}
\item {Grp. gram.:f.}
\end{itemize}
Acto de esterilizar.
\section{Esterilizador}
\begin{itemize}
\item {Grp. gram.:adj.}
\end{itemize}
\begin{itemize}
\item {Grp. gram.:M.}
\end{itemize}
Que esteriliza.
Aquelle que esteriliza.
Apparelho para destruir, pelo calor, os germes de doença, existentes no leite, no vinho, etc.
\section{Esterilizar}
\begin{itemize}
\item {Grp. gram.:v. t.}
\end{itemize}
\begin{itemize}
\item {Utilização:Fig.}
\end{itemize}
Tornar estéril.
Destruir os germes deletérios, que existem em (uma substância alimentícia): \textunderscore esterilizar\textunderscore  o leite.
Tornar inútil, improfícuo: \textunderscore esterilizar esforços\textunderscore .
\section{Esterilmente}
\begin{itemize}
\item {Grp. gram.:adv.}
\end{itemize}
\begin{itemize}
\item {Proveniência:(De \textunderscore estéril\textunderscore )}
\end{itemize}
Sem producto, sem proveito.
\section{Esterlicado}
\begin{itemize}
\item {Grp. gram.:adj.}
\end{itemize}
\begin{itemize}
\item {Proveniência:(De \textunderscore esterlicar\textunderscore )}
\end{itemize}
Tão apertado, que quási estala, (falando-se da luva, do calçado ou de qualquer peça de vestuário).
Que usa fato muito apertado.
Aperaltado, janota.
\section{Esterlicar}
\begin{itemize}
\item {Grp. gram.:v. t.}
\end{itemize}
Tornar esterlicado: \textunderscore esterlicar as luvas\textunderscore .
\section{Esterlino}
\begin{itemize}
\item {Grp. gram.:adj.}
\end{itemize}
\begin{itemize}
\item {Proveniência:(Do ingl. \textunderscore sterling\textunderscore )}
\end{itemize}
Diz-se da libra, moéda de oiro inglesa.
\section{Esterloixo}
\begin{itemize}
\item {Grp. gram.:m.}
\end{itemize}
\begin{itemize}
\item {Utilização:Prov.}
\end{itemize}
\begin{itemize}
\item {Utilização:trasm.}
\end{itemize}
Solavanco.
Acto de escabujar.
\section{Esternal}
\begin{itemize}
\item {Grp. gram.:adj.}
\end{itemize}
Relativo ao esterno.
\section{Esternalgia}
\begin{itemize}
\item {Grp. gram.:f.}
\end{itemize}
\begin{itemize}
\item {Proveniência:(Do gr. \textunderscore sternon\textunderscore  + gr. \textunderscore algos\textunderscore )}
\end{itemize}
Angina do peito.
\section{Estérnebra}
\begin{itemize}
\item {Grp. gram.:f.}
\end{itemize}
\begin{itemize}
\item {Proveniência:(De \textunderscore esterno\textunderscore , com um suff. tirado de \textunderscore vértebra\textunderscore )}
\end{itemize}
Cada um dos elementos do esterno.
\section{Esternebrado}
\begin{itemize}
\item {Grp. gram.:adj. e m.}
\end{itemize}
\begin{itemize}
\item {Proveniência:(De \textunderscore estérnebra\textunderscore )}
\end{itemize}
O mesmo que \textunderscore anelídeo\textunderscore , (para aquelles que nos anéis dos anelídios vêem o desenvolvimento de peças esternaes).
\section{Esternebral}
\begin{itemize}
\item {Grp. gram.:adj.}
\end{itemize}
Relativo a estérnebra.
\section{Esterno}
\begin{itemize}
\item {Grp. gram.:m.}
\end{itemize}
\begin{itemize}
\item {Proveniência:(Gr. \textunderscore sternon\textunderscore )}
\end{itemize}
Osso oblongo, na parte média e anterior do thórax.
\section{Esternóco-te!}
\begin{itemize}
\item {Grp. gram.:interj.}
\end{itemize}
\begin{itemize}
\item {Utilização:Prov.}
\end{itemize}
\begin{itemize}
\item {Utilização:minh.}
\end{itemize}
Eu te esconjuro! some-te!
(Cp. \textunderscore externar\textunderscore )
\section{Esternoxos}
\begin{itemize}
\item {fónica:csos}
\end{itemize}
\begin{itemize}
\item {Grp. gram.:m. pl.}
\end{itemize}
\begin{itemize}
\item {Proveniência:(Do gr. \textunderscore sternos\textunderscore  + \textunderscore oxus\textunderscore )}
\end{itemize}
Insectos coleópteros, cujo esterno resái em fórma de ponta.
\section{Esternutação}
\begin{itemize}
\item {Grp. gram.:f.}
\end{itemize}
\begin{itemize}
\item {Proveniência:(Lat. \textunderscore sternutatio\textunderscore )}
\end{itemize}
O mesmo que \textunderscore espirro\textunderscore .
\section{Esternutatório}
\begin{itemize}
\item {Grp. gram.:m.  e  adj.}
\end{itemize}
\begin{itemize}
\item {Proveniência:(Do lat. \textunderscore sternutare\textunderscore )}
\end{itemize}
Aquillo que provoca espirros; ptármico.
\section{Esterópodes}
\begin{itemize}
\item {Grp. gram.:m.}
\end{itemize}
\begin{itemize}
\item {Proveniência:(Do gr. \textunderscore stereos\textunderscore  + \textunderscore pous\textunderscore , \textunderscore podos\textunderscore )}
\end{itemize}
Classe de molluscos.
\section{Esterqueira}
\begin{itemize}
\item {Grp. gram.:f.}
\end{itemize}
\begin{itemize}
\item {Utilização:Fig.}
\end{itemize}
Estrumeira, lugar onde se junta estêrco.
Lugar immundo; immundície.
\section{Esterqueiro}
\begin{itemize}
\item {Grp. gram.:m.}
\end{itemize}
\begin{itemize}
\item {Utilização:Fig.}
\end{itemize}
Estrumeira, lugar onde se junta estêrco.
Lugar immundo; immundície.
\section{Esterquilínio}
\begin{itemize}
\item {Grp. gram.:m.}
\end{itemize}
\begin{itemize}
\item {Proveniência:(Lat. \textunderscore sterquilinium\textunderscore )}
\end{itemize}
O mesmo que \textunderscore esterqueiro\textunderscore .
\section{Esterrar}
\begin{itemize}
\item {Grp. gram.:v. t.}
\end{itemize}
\begin{itemize}
\item {Utilização:Ant.}
\end{itemize}
O mesmo que \textunderscore desterrar\textunderscore .
\section{Esterradoira}
\begin{itemize}
\item {Grp. gram.:f.}
\end{itemize}
\begin{itemize}
\item {Utilização:Prov.}
\end{itemize}
\begin{itemize}
\item {Utilização:minh.}
\end{itemize}
Espaço, entre os regos de campo lavrado.
\section{Esterradoura}
\begin{itemize}
\item {Grp. gram.:f.}
\end{itemize}
\begin{itemize}
\item {Utilização:Prov.}
\end{itemize}
\begin{itemize}
\item {Utilização:minh.}
\end{itemize}
Espaço, entre os regos de campo lavrado.
\section{Esterroada}
\begin{itemize}
\item {Grp. gram.:f.}
\end{itemize}
Acto de esterroar.
\section{Esterroador}
\begin{itemize}
\item {Grp. gram.:m.}
\end{itemize}
Instrumento, para esterroar.
\section{Esterroar}
\begin{itemize}
\item {Grp. gram.:v. t.}
\end{itemize}
\begin{itemize}
\item {Proveniência:(De \textunderscore terrão\textunderscore )}
\end{itemize}
Desfazer os terrões de.
\section{Estertor}
\begin{itemize}
\item {Grp. gram.:m.}
\end{itemize}
\begin{itemize}
\item {Proveniência:(Do rad. do lat. \textunderscore stertere\textunderscore )}
\end{itemize}
Som cavo, que catacteriza a respiração dos moribundos.
Agonia.
\section{Estertorar}
\begin{itemize}
\item {Grp. gram.:v. i.}
\end{itemize}
\begin{itemize}
\item {Utilização:Neol.}
\end{itemize}
Estar em estertor, agonizar. Cf. Júl. Ribeiro, \textunderscore Carne\textunderscore .
\section{Estertoroso}
\begin{itemize}
\item {Grp. gram.:adj.}
\end{itemize}
\begin{itemize}
\item {Proveniência:(De \textunderscore estertor\textunderscore )}
\end{itemize}
Diz-se da respiração, cujo som imita o ruído da água que ferve. Cf. Camillo, \textunderscore Filha do Regicida\textunderscore , 171; \textunderscore Mem. do Cárcere\textunderscore , etc.
\section{Estese}
\begin{itemize}
\item {Grp. gram.:f.}
\end{itemize}
\begin{itemize}
\item {Proveniência:(Gr. \textunderscore aistesis\textunderscore )}
\end{itemize}
Sentimento do belo.
\section{Estesia}
\begin{itemize}
\item {Grp. gram.:f.}
\end{itemize}
O mesmo que \textunderscore estese\textunderscore .
\section{Estesiologia}
\begin{itemize}
\item {Grp. gram.:f.}
\end{itemize}
\begin{itemize}
\item {Proveniência:(Do gr. \textunderscore aesthesis\textunderscore  + \textunderscore logos\textunderscore )}
\end{itemize}
Tratado dos órgãos dos sentidos.
\section{Estesiómetro}
\begin{itemize}
\item {Grp. gram.:m.}
\end{itemize}
\begin{itemize}
\item {Proveniência:(Do gr. \textunderscore aisthesis\textunderscore  + \textunderscore metron\textunderscore )}
\end{itemize}
Instrumento, para medir a sensibilidade táctil; compasso de Weber.
\section{Esteso}
\begin{itemize}
\item {Grp. gram.:adj.}
\end{itemize}
\begin{itemize}
\item {Utilização:Prov.}
\end{itemize}
\begin{itemize}
\item {Utilização:minh.}
\end{itemize}
Estendido.
(Cp. \textunderscore extenso\textunderscore )
\section{Estesódico}
\begin{itemize}
\item {Grp. gram.:adj.}
\end{itemize}
\begin{itemize}
\item {Proveniência:(Do gr. \textunderscore aisthesis\textunderscore  + \textunderscore odos\textunderscore )}
\end{itemize}
Que trasm.te a sensação.
\section{Esteta}
\begin{itemize}
\item {Grp. gram.:m.}
\end{itemize}
Aquele que cultiva a Estética.
Aquele que fórma da arte uma concepção elevada.
(Cp. \textunderscore estético\textunderscore )
\section{Estethómetro}
\begin{itemize}
\item {Grp. gram.:m.}
\end{itemize}
\begin{itemize}
\item {Proveniência:(Do gr. \textunderscore stethos\textunderscore  + \textunderscore metron\textunderscore )}
\end{itemize}
Instrumento, para medir as dimensões de peito.
\section{Estethoscopia}
\begin{itemize}
\item {Grp. gram.:f.}
\end{itemize}
Emprêgo do estethoscópio.
\section{Estethoscópio}
\begin{itemize}
\item {Grp. gram.:m.}
\end{itemize}
\begin{itemize}
\item {Proveniência:(Do gr. \textunderscore stethos\textunderscore  + \textunderscore skopein\textunderscore )}
\end{itemize}
Instrumento cirúrgico, para auscultação.
\section{Estética}
\begin{itemize}
\item {Grp. gram.:f.}
\end{itemize}
\begin{itemize}
\item {Proveniência:(De \textunderscore estético\textunderscore )}
\end{itemize}
Filosophia das belas-artes.
Ciência, que determina o carácter do belo nas producções naturaes e artísticas.
\section{Esteticamente}
\begin{itemize}
\item {Grp. gram.:adv.}
\end{itemize}
\begin{itemize}
\item {Proveniência:(De \textunderscore estético\textunderscore )}
\end{itemize}
Segundo os principios da estética.
Sob o ponto de vista estético.
\section{Estético}
\begin{itemize}
\item {Grp. gram.:adj.}
\end{itemize}
\begin{itemize}
\item {Proveniência:(Gr. \textunderscore aisthetikos\textunderscore )}
\end{itemize}
Relativo a Estética.
\section{Estetómetro}
\begin{itemize}
\item {Grp. gram.:m.}
\end{itemize}
\begin{itemize}
\item {Proveniência:(Do gr. \textunderscore stethos\textunderscore  + \textunderscore metron\textunderscore )}
\end{itemize}
Instrumento, para medir as dimensões de peito.
\section{Estetoscopia}
\begin{itemize}
\item {Grp. gram.:f.}
\end{itemize}
Emprêgo do estetoscópio.
\section{Estetoscópio}
\begin{itemize}
\item {Grp. gram.:m.}
\end{itemize}
\begin{itemize}
\item {Proveniência:(Do gr. \textunderscore stethos\textunderscore  + \textunderscore skopein\textunderscore )}
\end{itemize}
Instrumento cirúrgico, para auscultação.
\section{Estêva}
\begin{itemize}
\item {Grp. gram.:f.}
\end{itemize}
\begin{itemize}
\item {Proveniência:(Do lat. \textunderscore stiva\textunderscore )}
\end{itemize}
Rabiça do arado.
\section{Estêva}
\begin{itemize}
\item {Grp. gram.:f.}
\end{itemize}
\begin{itemize}
\item {Proveniência:(Do lat. \textunderscore stipa\textunderscore )}
\end{itemize}
Planta vulgar da fam. cistíneas.
\section{Estêva}
\begin{itemize}
\item {Grp. gram.:f.}
\end{itemize}
\begin{itemize}
\item {Utilização:Prov.}
\end{itemize}
\begin{itemize}
\item {Utilização:minh.}
\end{itemize}
O mesmo que \textunderscore estiva\textunderscore ^2.
\section{Estevado}
\begin{itemize}
\item {Grp. gram.:adj.}
\end{itemize}
Diz-se do cavallo, cujos cascos assentam obliquamente, voltando os lumes para dentro. Cf. Leon, \textunderscore Arte de Ferrar\textunderscore , 151.
\section{Esteval}
\begin{itemize}
\item {Grp. gram.:m.}
\end{itemize}
\begin{itemize}
\item {Proveniência:(De \textunderscore estêva\textunderscore ^2)}
\end{itemize}
Lugar, em que crescem estêvas.
\section{Estevão}
\begin{itemize}
\item {Grp. gram.:m.}
\end{itemize}
\begin{itemize}
\item {Proveniência:(De \textunderscore estêva\textunderscore ^2)}
\end{itemize}
Variedade de estêva.
\section{Estevar}
\begin{itemize}
\item {Grp. gram.:v. i.}
\end{itemize}
\begin{itemize}
\item {Proveniência:(De \textunderscore estêva\textunderscore ^1)}
\end{itemize}
Governar a estêva, a rabiça do arado.
\section{Esteveira}
\begin{itemize}
\item {Grp. gram.:f.}
\end{itemize}
\begin{itemize}
\item {Utilização:Prov.}
\end{itemize}
\begin{itemize}
\item {Utilização:alent.}
\end{itemize}
Variedade de figo.
\section{Esthenia}
\begin{itemize}
\item {Grp. gram.:f.}
\end{itemize}
\begin{itemize}
\item {Utilização:Med.}
\end{itemize}
\begin{itemize}
\item {Proveniência:(Do gr. \textunderscore sthenos\textunderscore , fôrça)}
\end{itemize}
Excesso de fôrça.
Exaltação da acção orgânica.
\section{Esthênico}
\begin{itemize}
\item {Grp. gram.:adj.}
\end{itemize}
Relativo a esthenia.
\section{Esthenothermes}
\begin{itemize}
\item {Grp. gram.:m. pl.}
\end{itemize}
\begin{itemize}
\item {Utilização:Zool.}
\end{itemize}
\begin{itemize}
\item {Proveniência:(Do gr. \textunderscore stenos\textunderscore  + \textunderscore therme\textunderscore )}
\end{itemize}
Animaes, a que as variações da temperatura abreviam a vida ou lh'a fazem perigar.
\section{Esthese}
\begin{itemize}
\item {Grp. gram.:f.}
\end{itemize}
\begin{itemize}
\item {Proveniência:(Gr. \textunderscore aistesis\textunderscore )}
\end{itemize}
Sentimento do bello.
\section{Esthesia}
\begin{itemize}
\item {Grp. gram.:f.}
\end{itemize}
O mesmo que \textunderscore esthese\textunderscore .
\section{Esthesiologia}
\begin{itemize}
\item {Grp. gram.:f.}
\end{itemize}
\begin{itemize}
\item {Proveniência:(Do gr. \textunderscore aesthesis\textunderscore  + \textunderscore logos\textunderscore )}
\end{itemize}
Tratado dos órgãos dos sentidos.
\section{Esthesiómetro}
\begin{itemize}
\item {Grp. gram.:m.}
\end{itemize}
\begin{itemize}
\item {Proveniência:(Do gr. \textunderscore aisthesis\textunderscore  + \textunderscore metron\textunderscore )}
\end{itemize}
Instrumento, para medir a sensibilidade táctil; compasso de Weber.
\section{Esthesódico}
\begin{itemize}
\item {Grp. gram.:adj.}
\end{itemize}
\begin{itemize}
\item {Proveniência:(Do gr. \textunderscore aisthesis\textunderscore  + \textunderscore odos\textunderscore )}
\end{itemize}
Que trasm.tte a sensação.
\section{Estheta}
\begin{itemize}
\item {Grp. gram.:m.}
\end{itemize}
Aquelle que cultiva a Esthética.
Aquelle que fórma da arte uma concepção elevada.
(Cp. \textunderscore esthético\textunderscore )
\section{Esthética}
\begin{itemize}
\item {Grp. gram.:f.}
\end{itemize}
\begin{itemize}
\item {Proveniência:(De \textunderscore esthético\textunderscore )}
\end{itemize}
Philosophia das bellas-artes.
Sciência, que determina o carácter do bello nas producções naturaes e artísticas.
\section{Estheticamente}
\begin{itemize}
\item {Grp. gram.:adv.}
\end{itemize}
\begin{itemize}
\item {Proveniência:(De \textunderscore esthético\textunderscore )}
\end{itemize}
Segundo os principios da esthética.
Sob o ponto de vista esthético.
\section{Esthético}
\begin{itemize}
\item {Grp. gram.:adj.}
\end{itemize}
\begin{itemize}
\item {Proveniência:(Gr. \textunderscore aisthetikos\textunderscore )}
\end{itemize}
Relativo a Esthética.
\section{Esthónico}
\begin{itemize}
\item {Grp. gram.:m.}
\end{itemize}
Língua, o mesmo que \textunderscore esthónio\textunderscore .
\section{Esthónio}
\begin{itemize}
\item {Grp. gram.:adj.}
\end{itemize}
\begin{itemize}
\item {Grp. gram.:M.}
\end{itemize}
\begin{itemize}
\item {Proveniência:(De \textunderscore Esthónia\textunderscore , n. p.)}
\end{itemize}
Pertencente á Esthónia.
Língua uralo-altaica, vernácula na Rússia.
\section{Esthopsychologia}
\begin{itemize}
\item {fónica:co}
\end{itemize}
\begin{itemize}
\item {Grp. gram.:f.}
\end{itemize}
\begin{itemize}
\item {Utilização:Neol.}
\end{itemize}
\begin{itemize}
\item {Proveniência:(De \textunderscore esthética\textunderscore  e \textunderscore psychologia\textunderscore )}
\end{itemize}
Modalidade da Psychologia, sob o ponto de vista esthético.--T. criado por Hennequin e adoptado já por Sílvio Romero.
\section{Esthopsychológico}
\begin{itemize}
\item {fónica:co}
\end{itemize}
\begin{itemize}
\item {Grp. gram.:adj.}
\end{itemize}
Relativo a esthopsychologia.
\section{Estiada}
\begin{itemize}
\item {Grp. gram.:f.}
\end{itemize}
O mesmo que \textunderscore estiagem\textunderscore .
\section{Estiado}
\begin{itemize}
\item {Grp. gram.:adj.}
\end{itemize}
\begin{itemize}
\item {Proveniência:(De \textunderscore estiar\textunderscore )}
\end{itemize}
Sereno e sêco, (falando-se do tempo).
\section{Estiagem}
\begin{itemize}
\item {Grp. gram.:f.}
\end{itemize}
\begin{itemize}
\item {Proveniência:(De \textunderscore estiar\textunderscore )}
\end{itemize}
Tempo sereno ou sêco, depois de tempo chuvoso ou tempestuoso.
\section{Estiar}
\begin{itemize}
\item {Grp. gram.:v. i.}
\end{itemize}
\begin{itemize}
\item {Utilização:Fig.}
\end{itemize}
\begin{itemize}
\item {Proveniência:(De \textunderscore estio\textunderscore )}
\end{itemize}
Tornar-se sereno ou sêco, (falando-se do tempo).
Afroixar.
\section{Estibiado}
\begin{itemize}
\item {Grp. gram.:adj.}
\end{itemize}
\begin{itemize}
\item {Proveniência:(De \textunderscore estíbio\textunderscore )}
\end{itemize}
Que tem antimónio.
\section{Estibial}
\begin{itemize}
\item {Grp. gram.:adj.}
\end{itemize}
\begin{itemize}
\item {Proveniência:(De \textunderscore estíbio\textunderscore )}
\end{itemize}
Relativo a antimónio.
\section{Estibiato}
\begin{itemize}
\item {Grp. gram.:m.}
\end{itemize}
O mesmo que \textunderscore antimonieto\textunderscore .
\section{Estibina}
\begin{itemize}
\item {Grp. gram.:f.}
\end{itemize}
\begin{itemize}
\item {Proveniência:(De \textunderscore estíbio\textunderscore )}
\end{itemize}
Sulfureto de antimónio.
\section{Estíbio}
\begin{itemize}
\item {Grp. gram.:m.}
\end{itemize}
\begin{itemize}
\item {Proveniência:(Lat. \textunderscore stibium\textunderscore )}
\end{itemize}
O mesmo que \textunderscore antimónio\textunderscore ^1.
\section{Estibiureto}
\begin{itemize}
\item {fónica:urê}
\end{itemize}
\begin{itemize}
\item {Grp. gram.:m.}
\end{itemize}
\begin{itemize}
\item {Proveniência:(De \textunderscore estíbio\textunderscore )}
\end{itemize}
O mesmo que \textunderscore antimonieto\textunderscore .
\section{Estibomar}
\begin{itemize}
\item {Grp. gram.:v. i.}
\end{itemize}
\begin{itemize}
\item {Utilização:Prov.}
\end{itemize}
\begin{itemize}
\item {Utilização:alg.}
\end{itemize}
Trasbordar.
\section{Estibordo}
\begin{itemize}
\item {Grp. gram.:m.}
\end{itemize}
\begin{itemize}
\item {Proveniência:(Do angl. sax. \textunderscore steorbord\textunderscore . Cp. \textunderscore bombordo\textunderscore )}
\end{itemize}
Lado do navio, á direita de quem olha, da popa para a prôa.
\section{Estica}
\begin{itemize}
\item {Grp. gram.:f.}
\end{itemize}
\begin{itemize}
\item {Utilização:Fam.}
\end{itemize}
\begin{itemize}
\item {Proveniência:(Do ingl. \textunderscore stick\textunderscore , vara?)}
\end{itemize}
Falta de saúde.
Magreza.
\section{Estica}
\begin{itemize}
\item {Grp. gram.:f.}
\end{itemize}
\begin{itemize}
\item {Proveniência:(Do gr. \textunderscore stikhe\textunderscore )}
\end{itemize}
Variedade de videira, que produz uva doce.
\section{Esticadela}
\begin{itemize}
\item {Grp. gram.:f.}
\end{itemize}
Acto de esticar.
\section{Esticador}
\begin{itemize}
\item {Grp. gram.:adj.}
\end{itemize}
\begin{itemize}
\item {Grp. gram.:M.}
\end{itemize}
\begin{itemize}
\item {Proveniência:(De \textunderscore esticar\textunderscore )}
\end{itemize}
Que estica.
Peça de madeira, em que se estica o papel para trabalhos de aguarela.
\section{Esticão}
\begin{itemize}
\item {Grp. gram.:m.}
\end{itemize}
Acto de esticar muito.
\section{Esticar}
\begin{itemize}
\item {Grp. gram.:v. t.}
\end{itemize}
\begin{itemize}
\item {Utilização:Fam.}
\end{itemize}
\begin{itemize}
\item {Proveniência:(De \textunderscore estica\textunderscore ^1)}
\end{itemize}
Estender, puxando.
Retesar: \textunderscore esticar uma corda\textunderscore .
\textunderscore Esticar a canela\textunderscore , morrer.
\section{Estigma}
\begin{itemize}
\item {Grp. gram.:m.}
\end{itemize}
\begin{itemize}
\item {Utilização:Bot.}
\end{itemize}
\begin{itemize}
\item {Utilização:Zool.}
\end{itemize}
\begin{itemize}
\item {Proveniência:(Gr. \textunderscore stigma\textunderscore )}
\end{itemize}
Marca.
Sinal infamante.
Ferrete.
Dilatação na parte superior do pistillo.
Órgãos da respiração nos insectos.
\section{Estigmado}
\begin{itemize}
\item {Grp. gram.:adj.}
\end{itemize}
Que tem estigma. Cf. M. Assis, \textunderscore Brás Cubas\textunderscore , 124.
\section{Estigmário}
\begin{itemize}
\item {Grp. gram.:m.}
\end{itemize}
\begin{itemize}
\item {Proveniência:(De \textunderscore estigma\textunderscore )}
\end{itemize}
Gênero de vegetaes fósseis.
\section{Estigmarrota}
\begin{itemize}
\item {Grp. gram.:f.}
\end{itemize}
\begin{itemize}
\item {Proveniência:(Do lat. \textunderscore stígma\textunderscore  + \textunderscore rota\textunderscore )}
\end{itemize}
Árvore, pouco conhecida, da Cochinchina.
\section{Estigmatário}
\begin{itemize}
\item {Grp. gram.:adj.}
\end{itemize}
\begin{itemize}
\item {Utilização:Bot.}
\end{itemize}
\begin{itemize}
\item {Proveniência:(Do lat. \textunderscore stigma\textunderscore )}
\end{itemize}
Que tem pontos cavados.
\section{Estígmate}
\begin{itemize}
\item {Grp. gram.:m.}
\end{itemize}
\begin{itemize}
\item {Utilização:Bot.}
\end{itemize}
O mesmo que \textunderscore estigma\textunderscore .
\section{Estigmático}
\begin{itemize}
\item {Grp. gram.:adj.}
\end{itemize}
Relativo ao estigma vegetal.
\section{Estigmatizar}
\begin{itemize}
\item {Grp. gram.:v. t.}
\end{itemize}
\begin{itemize}
\item {Utilização:Fig.}
\end{itemize}
\begin{itemize}
\item {Proveniência:(De \textunderscore estigma\textunderscore )}
\end{itemize}
Assignalar com estigma.
Desvirtuar.
Censurar; condemnar.
\section{Estigmatococco}
\begin{itemize}
\item {Grp. gram.:m.}
\end{itemize}
\begin{itemize}
\item {Proveniência:(Do gr. \textunderscore stigma\textunderscore  + \textunderscore kokkos\textunderscore )}
\end{itemize}
Gênero de arbustos do Brasil.
\section{Estigmatococo}
\begin{itemize}
\item {Grp. gram.:m.}
\end{itemize}
\begin{itemize}
\item {Proveniência:(Do gr. \textunderscore stigma\textunderscore  + \textunderscore kokkos\textunderscore )}
\end{itemize}
Gênero de arbustos do Brasil.
\section{Estigmatografia}
\begin{itemize}
\item {Grp. gram.:f.}
\end{itemize}
\begin{itemize}
\item {Proveniência:(Do gr. \textunderscore stigma\textunderscore  + \textunderscore graphein\textunderscore )}
\end{itemize}
Arte de escrever ou desenhar com o auxílio de pontos.
\section{Estigmatográfico}
\begin{itemize}
\item {Grp. gram.:adj.}
\end{itemize}
Relativo a estigmatografia.
\section{Estigmatographia}
\begin{itemize}
\item {Grp. gram.:f.}
\end{itemize}
\begin{itemize}
\item {Proveniência:(Do gr. \textunderscore stigma\textunderscore  + \textunderscore graphein\textunderscore )}
\end{itemize}
Arte de escrever ou desenhar com o auxílio de pontos.
\section{Estigmatográphico}
\begin{itemize}
\item {Grp. gram.:adj.}
\end{itemize}
Relativo a estigmatographia.
\section{Estónico}
\begin{itemize}
\item {Grp. gram.:m.}
\end{itemize}
Língua, o mesmo que \textunderscore estónio\textunderscore .
\section{Estónio}
\begin{itemize}
\item {Grp. gram.:adj.}
\end{itemize}
\begin{itemize}
\item {Grp. gram.:M.}
\end{itemize}
\begin{itemize}
\item {Proveniência:(De \textunderscore Estónia\textunderscore , n. p.)}
\end{itemize}
Pertencente á Estónia.
Língua uralo-altaica, vernácula na Rússia.
\section{Estopsicologia}
\begin{itemize}
\item {Grp. gram.:f.}
\end{itemize}
\begin{itemize}
\item {Utilização:Neol.}
\end{itemize}
\begin{itemize}
\item {Proveniência:(De \textunderscore estética\textunderscore  e \textunderscore psicologia\textunderscore )}
\end{itemize}
Modalidade da Psicologia, sob o ponto de vista estético.--T. criado por Hennequin e adoptado já por Sílvio Romero.
\section{Estopsicológico}
\begin{itemize}
\item {Grp. gram.:adj.}
\end{itemize}
Relativo a estopsicologia.
\section{Esthiómeno}
\begin{itemize}
\item {Grp. gram.:m.}
\end{itemize}
\begin{itemize}
\item {Grp. gram.:Adj.}
\end{itemize}
\begin{itemize}
\item {Proveniência:(Gr. \textunderscore esthiomenos\textunderscore )}
\end{itemize}
Gangrena.
Corrosivo.
Que corrói.
\section{Estigmatofilo}
\begin{itemize}
\item {Grp. gram.:m.}
\end{itemize}
\begin{itemize}
\item {Proveniência:(Do gr. \textunderscore stigma\textunderscore  + \textunderscore phullon\textunderscore )}
\end{itemize}
Gênero de arbustos da América do Sul.
\section{Estigmatóforo}
\begin{itemize}
\item {Grp. gram.:adj.}
\end{itemize}
\begin{itemize}
\item {Proveniência:(Do gr. \textunderscore stigma\textunderscore  + \textunderscore phoros\textunderscore )}
\end{itemize}
Que tem orifícios.
\section{Estigmatóphoro}
\begin{itemize}
\item {Grp. gram.:adj.}
\end{itemize}
\begin{itemize}
\item {Proveniência:(Do gr. \textunderscore stigma\textunderscore  + \textunderscore phoros\textunderscore )}
\end{itemize}
Que tem orifícios.
\section{Estigmatophyllo}
\begin{itemize}
\item {Grp. gram.:m.}
\end{itemize}
\begin{itemize}
\item {Proveniência:(Do gr. \textunderscore stigma\textunderscore  + \textunderscore phullon\textunderscore )}
\end{itemize}
Gênero de arbustos da América do Sul.
\section{Estigmatoteco}
\begin{itemize}
\item {Grp. gram.:m.}
\end{itemize}
\begin{itemize}
\item {Proveniência:(Do gr. \textunderscore stigma\textunderscore  + \textunderscore theke\textunderscore )}
\end{itemize}
Espécie de crisântemo da ilha da Madeira.
\section{Estigmatotheco}
\begin{itemize}
\item {Grp. gram.:m.}
\end{itemize}
\begin{itemize}
\item {Proveniência:(Do gr. \textunderscore stigma\textunderscore  + \textunderscore theke\textunderscore )}
\end{itemize}
Espécie de crysânthemo da ilha da Madeira.
\section{Estigmologia}
\begin{itemize}
\item {Grp. gram.:f.}
\end{itemize}
\begin{itemize}
\item {Proveniência:(Do gr. \textunderscore stigma\textunderscore , sinal, e \textunderscore logos\textunderscore , tratado)}
\end{itemize}
Tratado ou complexo dos differentes sinaes que, com as letras, se empregam na escrita, como o til, a vírgula, a cedilha, etc.
\section{Estigmológico}
\begin{itemize}
\item {Grp. gram.:adj.}
\end{itemize}
Pertencente ou relativo a estigmologia.
\section{Estigmónimo}
\begin{itemize}
\item {Grp. gram.:m.}
\end{itemize}
\begin{itemize}
\item {Proveniência:(Do gr. \textunderscore stigma\textunderscore  + \textunderscore onuma\textunderscore )}
\end{itemize}
Autor, cujo nome é substituido por pontos.
\section{Estigmónymo}
\begin{itemize}
\item {Grp. gram.:m.}
\end{itemize}
\begin{itemize}
\item {Proveniência:(Do gr. \textunderscore stigma\textunderscore  + \textunderscore onuma\textunderscore )}
\end{itemize}
Autor, cujo nome é substituido por pontos.
\section{Estígmulo}
\begin{itemize}
\item {Grp. gram.:m.}
\end{itemize}
Cada uma das divisões de um estigma vegetal.
(Dem. de \textunderscore estigma\textunderscore )
\section{Estil}
\begin{itemize}
\item {Grp. gram.:m.}
\end{itemize}
\begin{itemize}
\item {Utilização:Ant.}
\end{itemize}
O mesmo que \textunderscore hastil\textunderscore .
\section{Estila}
\begin{itemize}
\item {Grp. gram.:f.}
\end{itemize}
O mesmo que \textunderscore estilha\textunderscore .
\textunderscore Carvão de estila\textunderscore , carvão, feito de pequenos braços de árvores.
\section{Estila}
\begin{itemize}
\item {Grp. gram.:f.}
\end{itemize}
\begin{itemize}
\item {Utilização:Prov.}
\end{itemize}
\begin{itemize}
\item {Utilização:alent.}
\end{itemize}
\begin{itemize}
\item {Proveniência:(De \textunderscore estilar\textunderscore )}
\end{itemize}
Casa, onde se fabríca aguardente.
\section{Estilada}
\begin{itemize}
\item {Grp. gram.:f.}
\end{itemize}
\begin{itemize}
\item {Utilização:Neol.}
\end{itemize}
\begin{itemize}
\item {Proveniência:(De \textunderscore estilo\textunderscore )}
\end{itemize}
Período ou trecho, escrito em bom estilo.
\section{Estilado}
\begin{itemize}
\item {Grp. gram.:m.}
\end{itemize}
\begin{itemize}
\item {Utilização:Des.}
\end{itemize}
\begin{itemize}
\item {Proveniência:(De \textunderscore estilar\textunderscore )}
\end{itemize}
Ferimento penetrante; supplício.
\section{Estilar}
\begin{itemize}
\item {Grp. gram.:v. t.}
\end{itemize}
\begin{itemize}
\item {Utilização:Des.}
\end{itemize}
\begin{itemize}
\item {Proveniência:(De \textunderscore estilo\textunderscore )}
\end{itemize}
Ferir; espicaçar; torturar. Cf. \textunderscore Eufrosina\textunderscore , 116 e 128.
Usar. Cf. Latino, \textunderscore Humboldt\textunderscore , 444.
\section{Estilar}
\textunderscore v. t.\textunderscore , \textunderscore i.\textunderscore  e \textunderscore p.\textunderscore  (e der.)
(V. \textunderscore destilar\textunderscore , etc.)
\section{Estilbita}
\begin{itemize}
\item {Grp. gram.:f.}
\end{itemize}
\begin{itemize}
\item {Proveniência:(Do gr. \textunderscore stilbein\textunderscore , brilhar)}
\end{itemize}
Silicato natural de alumina, muito brilhante.
\section{Estilbite}
\begin{itemize}
\item {Grp. gram.:f.}
\end{itemize}
\begin{itemize}
\item {Proveniência:(Do gr. \textunderscore stilbein\textunderscore , brilhar)}
\end{itemize}
Silicato natural de alumina, muito brilhante.
\section{Estilete}
\begin{itemize}
\item {fónica:lê}
\end{itemize}
\begin{itemize}
\item {Grp. gram.:m.}
\end{itemize}
\begin{itemize}
\item {Utilização:Bot.}
\end{itemize}
\begin{itemize}
\item {Proveniência:(De \textunderscore estilo\textunderscore )}
\end{itemize}
Instrumento de aço, delgado e ponteagudo, espécie de punhal.
Instrumento cirúrgico, para sondagem de feridas.
Parte do pistillo, em que assenta o estigma.
\section{Estiletear}
\begin{itemize}
\item {Grp. gram.:v. t.}
\end{itemize}
\begin{itemize}
\item {Utilização:Bras}
\end{itemize}
\begin{itemize}
\item {Utilização:Neol.}
\end{itemize}
Ferir com estilete.
\section{Estilha}
\begin{itemize}
\item {Grp. gram.:f.}
\end{itemize}
Lasca de madeira.
Cavaco.
Fragmento; pedaço.
\textunderscore Carvão de estilha\textunderscore , carvão de carvalho ou de pernadas de outras árvores. Cp. \textunderscore estila\textunderscore ^1.
(Por \textunderscore hastilha\textunderscore , de \textunderscore haste\textunderscore )
\section{Estilhaçar}
\begin{itemize}
\item {Grp. gram.:v. t.}
\end{itemize}
Partir em estilhaços.
\section{Estilhaço}
\begin{itemize}
\item {Grp. gram.:m.}
\end{itemize}
\begin{itemize}
\item {Proveniência:(De \textunderscore estilha\textunderscore )}
\end{itemize}
Lasca de pedra, madeira ou metal.
Lasca.
Pedaço, fragmento.
\section{Estilhar}
\begin{itemize}
\item {Grp. gram.:v. t.}
\end{itemize}
Partir em estilhas; despedaçar.
\section{Estilheira}
\begin{itemize}
\item {Grp. gram.:f.}
\end{itemize}
\begin{itemize}
\item {Proveniência:(De \textunderscore estilha\textunderscore )}
\end{itemize}
Utensílio, em que o ourives apoia a mão e o objecto em que trabalha.
\section{Estilhial}
\begin{itemize}
\item {fónica:li}
\end{itemize}
\begin{itemize}
\item {Grp. gram.:m.}
\end{itemize}
\begin{itemize}
\item {Utilização:Anat.}
\end{itemize}
\begin{itemize}
\item {Proveniência:(De \textunderscore estil\textunderscore  + \textunderscore ...hyal\textunderscore )}
\end{itemize}
Peça superior do meio arco hioídeo.
\section{Estilhyal}
\begin{itemize}
\item {fónica:li}
\end{itemize}
\begin{itemize}
\item {Grp. gram.:m.}
\end{itemize}
\begin{itemize}
\item {Utilização:Anat.}
\end{itemize}
\begin{itemize}
\item {Proveniência:(De \textunderscore estil\textunderscore  + \textunderscore ...hyal\textunderscore )}
\end{itemize}
Peça superior do meio arco hyoídeo.
\section{Estilicidar}
\begin{itemize}
\item {Grp. gram.:v. i.}
\end{itemize}
\begin{itemize}
\item {Utilização:Neol.}
\end{itemize}
Caír em gotas. Cf. C. Neto, \textunderscore Saldunes\textunderscore .
(Cp. \textunderscore estilicídio\textunderscore )
\section{Estilicídio}
\begin{itemize}
\item {Grp. gram.:m.}
\end{itemize}
\begin{itemize}
\item {Utilização:Fig.}
\end{itemize}
\begin{itemize}
\item {Proveniência:(Lat. \textunderscore stillicidium\textunderscore )}
\end{itemize}
Cada um dos fios de água pluvial, que caem dos beirados.
O gotejar de qualquer líquido.
Coriza, defluxo.
\section{Estilicidioso}
\begin{itemize}
\item {Grp. gram.:m.  e  adj.}
\end{itemize}
O que sofre estilicídio ou coriza.
\section{Estiliforme}
\begin{itemize}
\item {Grp. gram.:adj.}
\end{itemize}
\begin{itemize}
\item {Proveniência:(Do lat. \textunderscore stilus\textunderscore  + \textunderscore forma\textunderscore )}
\end{itemize}
Que tem fórma de estilete.
\section{Estilismo}
\begin{itemize}
\item {Grp. gram.:m.}
\end{itemize}
\begin{itemize}
\item {Proveniência:(De \textunderscore estilo\textunderscore )}
\end{itemize}
Demasiado cuidado no estilo, na linguagem.
\section{Estilista}
\begin{itemize}
\item {Grp. gram.:m.  e  adj.}
\end{itemize}
\begin{itemize}
\item {Proveniência:(De \textunderscore estilo\textunderscore )}
\end{itemize}
O que escreve elegantemente.
Escritor ou orador, notável pelo vigor e elegância do seu estilo.
\section{Estilística}
\begin{itemize}
\item {Grp. gram.:f.}
\end{itemize}
\begin{itemize}
\item {Proveniência:(De \textunderscore estilístico\textunderscore )}
\end{itemize}
Tratado das differentes fórmas ou espécies de estilo e dos preceitos concernentes a cada uma.
Arte de bem escrever. Cf. Pacheco e Lameira, \textunderscore Gram. da Líng. Port.\textunderscore 
\section{Estilístico}
\begin{itemize}
\item {Grp. gram.:adj.}
\end{itemize}
\begin{itemize}
\item {Proveniência:(De \textunderscore estilista\textunderscore )}
\end{itemize}
Relativo á estilística.
\section{Estilita}
\begin{itemize}
\item {Grp. gram.:m.}
\end{itemize}
\begin{itemize}
\item {Proveniência:(Do gr. \textunderscore stulos\textunderscore , columna)}
\end{itemize}
Anachoreta, que formava a sua cella sôbre pórticos ou columnas em ruína.
\section{Estilização}
\begin{itemize}
\item {Grp. gram.:f.}
\end{itemize}
\begin{itemize}
\item {Proveniência:(De \textunderscore estilo\textunderscore )}
\end{itemize}
Processo de ornamentação, em que se aproveitam ás vezes, no delineamento geral, os elementos da flora e os da fauna.
\section{Estilla}
\begin{itemize}
\item {Grp. gram.:f.}
\end{itemize}
\begin{itemize}
\item {Utilização:Prov.}
\end{itemize}
\begin{itemize}
\item {Utilização:alent.}
\end{itemize}
\begin{itemize}
\item {Proveniência:(De \textunderscore estillar\textunderscore )}
\end{itemize}
Casa, onde se fabríca aguardente.
\section{Estillar}
\textunderscore v. t.\textunderscore , \textunderscore i.\textunderscore  e \textunderscore p.\textunderscore  (e der.)
(V. \textunderscore destillar\textunderscore , etc.)
\section{Estillicidar}
\begin{itemize}
\item {Grp. gram.:v. i.}
\end{itemize}
\begin{itemize}
\item {Utilização:Neol.}
\end{itemize}
Caír em gotas. Cf. C. Neto, \textunderscore Saldunes\textunderscore .
(Cp. \textunderscore estillicídio\textunderscore )
\section{Estillicídio}
\begin{itemize}
\item {Grp. gram.:m.}
\end{itemize}
\begin{itemize}
\item {Utilização:Fig.}
\end{itemize}
\begin{itemize}
\item {Proveniência:(Lat. \textunderscore stillicidium\textunderscore )}
\end{itemize}
Cada um dos fios de água pluvial, que caem dos beirados.
O gotejar de qualquer líquido.
Coriza, defluxo.
\section{Estillicidioso}
\begin{itemize}
\item {Grp. gram.:m.  e  adj.}
\end{itemize}
O que soffre estillicídio ou coriza.
\section{Estilo}
\begin{itemize}
\item {Grp. gram.:m.}
\end{itemize}
\begin{itemize}
\item {Utilização:Ant.}
\end{itemize}
\begin{itemize}
\item {Proveniência:(Lat. \textunderscore stilus\textunderscore )}
\end{itemize}
Maneira particular de exprimir pensamentos, falando ou escrevendo.
Demasiado apuro em falar ou escrever.
Feição especial dos trabalhos de um artista; carácter de uma producção artística.
Uso, costume.
Estilete.
Ponteiro ou pequeno instrumento, com que os antigos escreviam em tábuas enceradas.
\section{Estilóbata}
\begin{itemize}
\item {Grp. gram.:m.}
\end{itemize}
O mesmo ou melhor que \textunderscore estilóbato\textunderscore .
\section{Estilóbato}
\begin{itemize}
\item {Grp. gram.:m.}
\end{itemize}
\begin{itemize}
\item {Utilização:Archit.}
\end{itemize}
\begin{itemize}
\item {Proveniência:(Gr. \textunderscore stulobates\textunderscore )}
\end{itemize}
Envasamento, que sustenta uma ordem de columnas.
\section{Estilofaríngeo}
\begin{itemize}
\item {Grp. gram.:adj.}
\end{itemize}
\begin{itemize}
\item {Utilização:Anat.}
\end{itemize}
\begin{itemize}
\item {Proveniência:(Do gr. \textunderscore stulos\textunderscore  + \textunderscore pharunx\textunderscore )}
\end{itemize}
Diz-se do músculo, que, fixado na apófise estiloídea, se dirige para a faringe.
\section{Estiloglosso}
\begin{itemize}
\item {Grp. gram.:adj.}
\end{itemize}
\begin{itemize}
\item {Utilização:Anat.}
\end{itemize}
\begin{itemize}
\item {Proveniência:(Do gr. \textunderscore stulos\textunderscore  + \textunderscore glossa\textunderscore )}
\end{itemize}
Diz-se do músculo que, fixado na base da apóphyse estyloídea, termina na língua.
\section{Estiloide}
\begin{itemize}
\item {Grp. gram.:adj.}
\end{itemize}
\begin{itemize}
\item {Proveniência:(Do gr. \textunderscore stulos\textunderscore  + \textunderscore eidos\textunderscore )}
\end{itemize}
Semelhante a estilete.
\section{Estiloídeo}
\begin{itemize}
\item {Grp. gram.:adj.}
\end{itemize}
\begin{itemize}
\item {Proveniência:(Do gr. \textunderscore stulos\textunderscore  + \textunderscore eidos\textunderscore )}
\end{itemize}
Semelhante a estilete.
\section{Estilometria}
\begin{itemize}
\item {Grp. gram.:f.}
\end{itemize}
\begin{itemize}
\item {Proveniência:(De \textunderscore estilómetro\textunderscore )}
\end{itemize}
Arte de medir columnas.
\section{Estilómetro}
\begin{itemize}
\item {Grp. gram.:m.}
\end{itemize}
\begin{itemize}
\item {Proveniência:(Do gr. \textunderscore stulos\textunderscore  + \textunderscore metron\textunderscore )}
\end{itemize}
Instrumento, para medir columnas.
\section{Estilopharýngeo}
\begin{itemize}
\item {Grp. gram.:adj.}
\end{itemize}
\begin{itemize}
\item {Utilização:Anat.}
\end{itemize}
\begin{itemize}
\item {Proveniência:(Do gr. \textunderscore stulos\textunderscore  + \textunderscore pharunx\textunderscore )}
\end{itemize}
Diz-se do músculo, que, fixado na apóphyse estyloídea, se dirige para a pharynge.
\section{Estim}
\begin{itemize}
\item {Grp. gram.:m.}
\end{itemize}
\begin{itemize}
\item {Utilização:Ant.}
\end{itemize}
O mesmo que \textunderscore hastil\textunderscore .
\section{Estima}
\begin{itemize}
\item {Grp. gram.:f.}
\end{itemize}
Acto de estimar.
Aprêço.
Affeição.
Apreciação favorável.
Apreciação, avaliação.
\section{Estimação}
\begin{itemize}
\item {Grp. gram.:f.}
\end{itemize}
\begin{itemize}
\item {Proveniência:(Lat. \textunderscore aestimatio\textunderscore )}
\end{itemize}
O mesmo que \textunderscore estima\textunderscore .
Cálculo.
Avaliação.
Aprêço de uma coisa, independente do seu valor real.
\section{Estimadamente}
\begin{itemize}
\item {Grp. gram.:adv.}
\end{itemize}
\begin{itemize}
\item {Proveniência:(De \textunderscore estimar\textunderscore )}
\end{itemize}
Com estimação.
\section{Estimado}
\begin{itemize}
\item {Grp. gram.:adj.}
\end{itemize}
\begin{itemize}
\item {Proveniência:(De \textunderscore estimar\textunderscore )}
\end{itemize}
Que é objecto de estima.
Apreciado.
Querido.
\section{Estimador}
\begin{itemize}
\item {Grp. gram.:m.  e  adj.}
\end{itemize}
\begin{itemize}
\item {Proveniência:(Lat. \textunderscore aestimator\textunderscore )}
\end{itemize}
O que estima.
O que sabe estimar ou avaliar.
\section{Estimadura}
\begin{itemize}
\item {Grp. gram.:f.}
\end{itemize}
\begin{itemize}
\item {Utilização:Ant.}
\end{itemize}
O mesmo que \textunderscore estimação\textunderscore  ou aprêço.
\section{Estimar}
\begin{itemize}
\item {Grp. gram.:v. t.}
\end{itemize}
\begin{itemize}
\item {Utilização:Ant.}
\end{itemize}
\begin{itemize}
\item {Grp. gram.:V. p.}
\end{itemize}
\begin{itemize}
\item {Proveniência:(Lat. \textunderscore aestimare\textunderscore )}
\end{itemize}
Calcular ou saber o preço ou valor de: \textunderscore estimar uma jóia em trinta libras\textunderscore .
Dar aprêço a.
Apreciar.
Têr affeição a: \textunderscore a Lili estima muito os animaes\textunderscore .
Têr prazer em.
Determinar o valor de.
Suppor ou calcular, receando.
Têr em grande conta a sua pessôa.
Rodear-se de commodidades.
Têr consciência do próprio valimento.
\section{Estimativa}
\begin{itemize}
\item {Grp. gram.:f.}
\end{itemize}
\begin{itemize}
\item {Proveniência:(De \textunderscore estimativo\textunderscore )}
\end{itemize}
Avaliação.
Cálculo.
Apreciação.
\section{Estimativo}
\begin{itemize}
\item {Grp. gram.:adj.}
\end{itemize}
\begin{itemize}
\item {Proveniência:(De \textunderscore estimar\textunderscore )}
\end{itemize}
Que estima.
Que sabe avaliar.
Relativo a estima: \textunderscore valor estimativo\textunderscore .
\section{Estimatório}
\begin{itemize}
\item {Grp. gram.:m.}
\end{itemize}
(V.estimativo)
\section{Estimável}
\begin{itemize}
\item {Grp. gram.:adj.}
\end{itemize}
\begin{itemize}
\item {Proveniência:(Lat. \textunderscore aestimabilis\textunderscore )}
\end{itemize}
Digno de estimação.
Que se póde apreciar ou avaliar.
\section{Éstimo}
\begin{itemize}
\item {Grp. gram.:m.}
\end{itemize}
\begin{itemize}
\item {Utilização:Ant.}
\end{itemize}
\begin{itemize}
\item {Proveniência:(Lat. \textunderscore aestimo\textunderscore )}
\end{itemize}
Espécie de pensão que os lavradores do Riba-Tejo pagavam annualmente á Igreja.
O mesmo que \textunderscore estimação\textunderscore .
\section{Estimulação}
\begin{itemize}
\item {Grp. gram.:f.}
\end{itemize}
Acto ou effeito de estimular.
\section{Estimuladamente}
\begin{itemize}
\item {Grp. gram.:adv.}
\end{itemize}
\begin{itemize}
\item {Proveniência:(De \textunderscore estimular\textunderscore )}
\end{itemize}
Com estímulo.
Com ira.
\section{Estimulador}
\begin{itemize}
\item {Grp. gram.:adj.}
\end{itemize}
\begin{itemize}
\item {Grp. gram.:M.}
\end{itemize}
\begin{itemize}
\item {Proveniência:(Lat. \textunderscore stimulator\textunderscore )}
\end{itemize}
Que estimula.
Aquelle que estimula.
\section{Estimulante}
\begin{itemize}
\item {Grp. gram.:adj.}
\end{itemize}
\begin{itemize}
\item {Proveniência:(Lat. \textunderscore stimulans\textunderscore )}
\end{itemize}
Que estimula.
Excitante: \textunderscore bebidas estimulantes\textunderscore .
\section{Estimular}
\begin{itemize}
\item {Grp. gram.:v. t.}
\end{itemize}
\begin{itemize}
\item {Proveniência:(Lat. \textunderscore stimulare\textunderscore )}
\end{itemize}
Incitar; excitar: \textunderscore estimular ódios\textunderscore .
Avivar.
Animar.
Espicaçar.
Desgostar; irritar.
\section{Estimulina}
\begin{itemize}
\item {Grp. gram.:f.}
\end{itemize}
\begin{itemize}
\item {Utilização:Chím.}
\end{itemize}
Substância anti-tóxica, que se desenvolve no sangue. Cf. Roux, Borel, etc.
\section{Estímulo}
\begin{itemize}
\item {Grp. gram.:m.}
\end{itemize}
\begin{itemize}
\item {Proveniência:(Lat. \textunderscore stimulus\textunderscore )}
\end{itemize}
Aquillo que estimula.
Aguilhão.
Incentivo.
\section{Estimuloso}
\begin{itemize}
\item {Grp. gram.:adj.}
\end{itemize}
O mesmo que \textunderscore estimulante\textunderscore .
\section{Estingar}
\begin{itemize}
\item {Grp. gram.:v.}
\end{itemize}
\begin{itemize}
\item {Utilização:t. Náut.}
\end{itemize}
\begin{itemize}
\item {Proveniência:(Do lat. \textunderscore stringere\textunderscore ?)}
\end{itemize}
Colher com os estingues (as velas).
\section{Estingue}
\begin{itemize}
\item {Grp. gram.:m.}
\end{itemize}
\begin{itemize}
\item {Utilização:Náut.}
\end{itemize}
\begin{itemize}
\item {Proveniência:(De \textunderscore estingar\textunderscore )}
\end{itemize}
Cabo, que vem dos punhos inferiores das velas ao meio da vêrga e que serve para carregar as velas.
\section{Estinhar}
\begin{itemize}
\item {Grp. gram.:v. t.}
\end{itemize}
\begin{itemize}
\item {Proveniência:(Do lat. \textunderscore extenuare\textunderscore ?)}
\end{itemize}
Tirar de (colmeias) o segundo mel das abelhas.
\section{Estinhar}
\begin{itemize}
\item {Grp. gram.:v. i.}
\end{itemize}
\begin{itemize}
\item {Utilização:Prov.}
\end{itemize}
\begin{itemize}
\item {Utilização:trasm.}
\end{itemize}
\begin{itemize}
\item {Utilização:Prov.}
\end{itemize}
\begin{itemize}
\item {Utilização:minh.}
\end{itemize}
O mesmo que \textunderscore estiar\textunderscore .
Deixar de correr (a água).
\section{Estio}
\begin{itemize}
\item {Grp. gram.:m.}
\end{itemize}
\begin{itemize}
\item {Utilização:Fig.}
\end{itemize}
\begin{itemize}
\item {Proveniência:(Do lat. \textunderscore aestivus\textunderscore )}
\end{itemize}
Uma das quatro estações do anno, que principia no solstício de Junho e termina no equinóccio de Setembro.
Tempo quente e sêco.
Idade madura: \textunderscore no estio da vida\textunderscore .
\section{Estiolamento}
\begin{itemize}
\item {Grp. gram.:m.}
\end{itemize}
Acto ou effeito de estiolar.
Definhamento das plantas, por falta de luz ou de ar livre.
Estado mórbido dos indivíduos, que vivem privados de luz ou de ar livre.
\section{Estiolar}
\begin{itemize}
\item {Grp. gram.:v. t.}
\end{itemize}
\begin{itemize}
\item {Utilização:Gal}
\end{itemize}
\begin{itemize}
\item {Grp. gram.:V. i.  e  p.}
\end{itemize}
\begin{itemize}
\item {Proveniência:(Fr. \textunderscore etioler\textunderscore )}
\end{itemize}
Produzir o estiolamento de; debilitar.
Alterar-se morbidamente, descòrando ou debilitando-se pela ausência do sol ou do ar livre.
Debilitar-se; desfallecer.
\section{Estiomenar}
\begin{itemize}
\item {Grp. gram.:v. t.}
\end{itemize}
\begin{itemize}
\item {Proveniência:(De \textunderscore estiómeno\textunderscore )}
\end{itemize}
Corroer, carcomer.
\section{Estiómeno}
\begin{itemize}
\item {Grp. gram.:m.}
\end{itemize}
\begin{itemize}
\item {Grp. gram.:Adj.}
\end{itemize}
\begin{itemize}
\item {Proveniência:(Gr. \textunderscore esthiomenos\textunderscore )}
\end{itemize}
Gangrena.
Corrosivo.
Que corrói.
\section{Estipa}
\begin{itemize}
\item {Grp. gram.:f.}
\end{itemize}
\begin{itemize}
\item {Proveniência:(Lat. \textunderscore stipa\textunderscore )}
\end{itemize}
Gênero de plantas gramíneas das regiões temperadas.
\section{Estipe}
\begin{itemize}
\item {Grp. gram.:m.}
\end{itemize}
\begin{itemize}
\item {Proveniência:(Lat. \textunderscore stipes\textunderscore )}
\end{itemize}
O mesmo que \textunderscore caule\textunderscore .
\section{Estipela}
\begin{itemize}
\item {Grp. gram.:f.}
\end{itemize}
Pequena estípula. Cf. Caminhoá, \textunderscore Bot. Ger. e Med.\textunderscore 
\section{Estipella}
\begin{itemize}
\item {Grp. gram.:f.}
\end{itemize}
Pequena estípula. Cf. Caminhoá, \textunderscore Bot. Ger. e Med.\textunderscore 
\section{Estipendiar}
\begin{itemize}
\item {Grp. gram.:v. t.}
\end{itemize}
Dar estipêndio a.
Assalariar.
Assoldadar.
\section{Estipendiário}
\begin{itemize}
\item {Grp. gram.:adj.}
\end{itemize}
\begin{itemize}
\item {Proveniência:(Lat. \textunderscore stipendiarus\textunderscore )}
\end{itemize}
Que recebe estipêndio.
\section{Estipêndio}
\begin{itemize}
\item {Grp. gram.:m.}
\end{itemize}
\begin{itemize}
\item {Utilização:Ant.}
\end{itemize}
\begin{itemize}
\item {Proveniência:(Lat. \textunderscore stipendium\textunderscore )}
\end{itemize}
Retribuição.
Soldada.
Paga.
Tributo.
\section{Estipiforme}
\begin{itemize}
\item {Grp. gram.:adj.}
\end{itemize}
\begin{itemize}
\item {Proveniência:(De \textunderscore estipa\textunderscore  + \textunderscore fórma\textunderscore )}
\end{itemize}
Que tem haste como a estipa.
\section{Estípita}
\begin{itemize}
\item {Grp. gram.:f.}
\end{itemize}
Variedade de carvão mineral.
Columna abalaustrada ou invertida.
(Cp. \textunderscore estípite\textunderscore )
\section{Estipitado}
\begin{itemize}
\item {Grp. gram.:adj.}
\end{itemize}
Que tem estípite.
\section{Estípite}
\begin{itemize}
\item {Grp. gram.:m.}
\end{itemize}
\begin{itemize}
\item {Utilização:Fig.}
\end{itemize}
\begin{itemize}
\item {Proveniência:(Do lat. \textunderscore stipes\textunderscore )}
\end{itemize}
Estipe.
Caule.
Tronco de uma geração.
Raça.
\section{Estípula}
\begin{itemize}
\item {Grp. gram.:f.}
\end{itemize}
\begin{itemize}
\item {Utilização:Bot.}
\end{itemize}
\begin{itemize}
\item {Proveniência:(Lat. \textunderscore stipula\textunderscore )}
\end{itemize}
Appêndice foliáceo, no ponto em que as folhas saem do caule.
\section{Estipulação}
\begin{itemize}
\item {Grp. gram.:f.}
\end{itemize}
Acto ou effeito de estipular.
\section{Estipulado}
\begin{itemize}
\item {Grp. gram.:adj.}
\end{itemize}
Que tem estípulas.
\section{Estipulador}
\begin{itemize}
\item {Grp. gram.:m.  e  adj.}
\end{itemize}
O que estipúla.
\section{Estipulante}
\begin{itemize}
\item {Grp. gram.:adj. ,  m.  e  f.}
\end{itemize}
\begin{itemize}
\item {Proveniência:(Lat. \textunderscore stipulans\textunderscore )}
\end{itemize}
Pessôa, que estipúla.
\section{Estipular}
\begin{itemize}
\item {Grp. gram.:v. t.}
\end{itemize}
\begin{itemize}
\item {Proveniência:(Lat. \textunderscore stipulari\textunderscore )}
\end{itemize}
Contratar, com condições ou cláusulas.
Ajustar; combinar.
Estabelecer.
\section{Estipular}
\begin{itemize}
\item {Grp. gram.:adj.}
\end{itemize}
\begin{itemize}
\item {Utilização:Bot.}
\end{itemize}
Relativo a estipula.
\section{Estipuloso}
\begin{itemize}
\item {Grp. gram.:adj.}
\end{itemize}
(V.estipulado)
\section{Estique}
\begin{itemize}
\item {Grp. gram.:m.}
\end{itemize}
\begin{itemize}
\item {Utilização:Bras. de Minas}
\end{itemize}
Espécie de jôgo de cartas.
\section{Estiraçar}
\begin{itemize}
\item {Grp. gram.:v. t.}
\end{itemize}
\begin{itemize}
\item {Proveniência:(De \textunderscore estirar\textunderscore )}
\end{itemize}
Estirar.
Estender ao comprido.
Retesar; esticar.
Deitar no chão.
\section{Estiraço}
\begin{itemize}
\item {Grp. gram.:m.}
\end{itemize}
O mesmo que \textunderscore estirada\textunderscore .
\section{Estirada}
\begin{itemize}
\item {Grp. gram.:f.}
\end{itemize}
\begin{itemize}
\item {Utilização:Prov.}
\end{itemize}
\begin{itemize}
\item {Utilização:beir.}
\end{itemize}
\begin{itemize}
\item {Proveniência:(De \textunderscore estirar\textunderscore )}
\end{itemize}
Caminhada, estirão.
\section{Estirador}
\begin{itemize}
\item {Grp. gram.:f.}
\end{itemize}
\begin{itemize}
\item {Proveniência:(De \textunderscore estirar\textunderscore )}
\end{itemize}
Tábua ou mesa, em que se assenta ou se estira o papel, em que se desenha ou se pinta.
\section{Estiramento}
\begin{itemize}
\item {Grp. gram.:m.}
\end{itemize}
Acto ou effeito de estirar.
\section{Estirão}
\begin{itemize}
\item {Grp. gram.:m.}
\end{itemize}
\begin{itemize}
\item {Utilização:Pop.}
\end{itemize}
\begin{itemize}
\item {Proveniência:(De \textunderscore estirar\textunderscore )}
\end{itemize}
Estiramento.
Caminho longo; caminhada.
\section{Estirar}
\begin{itemize}
\item {Grp. gram.:v. t.}
\end{itemize}
\begin{itemize}
\item {Utilização:Fig.}
\end{itemize}
\begin{itemize}
\item {Proveniência:(De \textunderscore tirar\textunderscore )}
\end{itemize}
Estender, puxando.
Alongar.
Esticar.
Tornar longo, comprido.
Deitar no chão ao comprido.
Alargar, dilatar.
Exceder os limites de.
Forçar.
Constranger.
\section{Estirpe}
\begin{itemize}
\item {Grp. gram.:f.}
\end{itemize}
\begin{itemize}
\item {Proveniência:(Do lat. \textunderscore stirps\textunderscore , ou \textunderscore stirpes\textunderscore )}
\end{itemize}
Parte da planta, que se desenvolve debaixo da terra.
Tronco de família.
Raça; ascendência.
\section{Estirpicultura}
\begin{itemize}
\item {Grp. gram.:f.}
\end{itemize}
\begin{itemize}
\item {Utilização:Neol.}
\end{itemize}
\begin{itemize}
\item {Proveniência:(Do lat. \textunderscore stirps\textunderscore  + \textunderscore cultura\textunderscore )}
\end{itemize}
Estudos moraes e sociaes, sôbre a reproducção da espécie humana.
\section{Estiticidade}
\begin{itemize}
\item {Grp. gram.:f.}
\end{itemize}
Qualidade daquillo que é estítico.
\section{Estítico}
\begin{itemize}
\item {Grp. gram.:adj.}
\end{itemize}
\begin{itemize}
\item {Proveniência:(Do gr. \textunderscore stuptikos\textunderscore )}
\end{itemize}
O mesmo que \textunderscore adstringente\textunderscore .
\section{Estiva}
\begin{itemize}
\item {Grp. gram.:f.}
\end{itemize}
\begin{itemize}
\item {Grp. gram.:Pl.}
\end{itemize}
\begin{itemize}
\item {Utilização:Bras}
\end{itemize}
\begin{itemize}
\item {Utilização:Bras. do N}
\end{itemize}
\begin{itemize}
\item {Proveniência:(De \textunderscore estivar\textunderscore ^1)}
\end{itemize}
Lastro.
Primeira porção de carga, que se mete no navio.
Grade, em que assenta a primeira porção da carga do navio, para lhe evitar a humidade.
Fundo interno da embarcação.
Grade, que se estende no pavimento das cavallariças, para escoamento da urina das bêstas.
Traves, que formam o leito das pontes de madeira.
Registo de gêneros alimentícios, feito pelos officiaes de bordo.
Taxa ou registo, feito por algumas câmaras municipaes, dos preços de certos gêneros, que não vão á casa grande da alfândega.
Pêso ou conta dos gêneros que se despacham na alfândega.
Cada pesada, proporcional á totalidade dos volumes, que se hão verificar na alfândega.
Gêneros alimentícios: \textunderscore armazem de estivas\textunderscore .
Ponte, feita de um só madeiro, sôbre forquilhas, em terrenos alagadiços.
\section{Estiva}
\begin{itemize}
\item {Grp. gram.:f.}
\end{itemize}
\begin{itemize}
\item {Utilização:Prov.}
\end{itemize}
\begin{itemize}
\item {Utilização:minh.}
\end{itemize}
Campo, em que se segou centeio e que se lavra immediatamente para sementeira de milho serôdio.
(Relaciona-se com \textunderscore estivo\textunderscore ? Ou com \textunderscore estêva\textunderscore ^1? Cp. \textunderscore estêva\textunderscore ^3)
\section{Estivação}
\begin{itemize}
\item {Grp. gram.:f.}
\end{itemize}
\begin{itemize}
\item {Utilização:Bot.}
\end{itemize}
O mesmo que \textunderscore perfloração\textunderscore .
\section{Estivação}
\begin{itemize}
\item {Grp. gram.:f.}
\end{itemize}
Acto ou effeito de estivar^1.
\section{Estivada}
\begin{itemize}
\item {Grp. gram.:f.}
\end{itemize}
\begin{itemize}
\item {Utilização:Prov.}
\end{itemize}
\begin{itemize}
\item {Utilização:minh.}
\end{itemize}
O mesmo que \textunderscore estiva\textunderscore ^2.
\section{Estivadamente}
\begin{itemize}
\item {Grp. gram.:adv.}
\end{itemize}
\begin{itemize}
\item {Proveniência:(De \textunderscore estivar\textunderscore ^2)}
\end{itemize}
Segundo a estiva^1.
Determinadamente.
\section{Estivado}
\begin{itemize}
\item {Grp. gram.:adj.}
\end{itemize}
\begin{itemize}
\item {Utilização:Bras. de Minas}
\end{itemize}
Cheio: \textunderscore uma casa estivada de gente\textunderscore .
\section{Estivador}
\begin{itemize}
\item {Grp. gram.:m.  e  adj.}
\end{itemize}
\begin{itemize}
\item {Grp. gram.:M.}
\end{itemize}
\begin{itemize}
\item {Utilização:Bras}
\end{itemize}
O que estiva.
Carregador de navio.
Aquelle que dirige, por conta de casa commercial, descarregadores de navios.
Negociante de gêneros alimentícios.
\section{Estivagem}
\begin{itemize}
\item {Grp. gram.:f.}
\end{itemize}
Acto de estivar^1.
\section{Estival}
\begin{itemize}
\item {Grp. gram.:adj.}
\end{itemize}
\begin{itemize}
\item {Grp. gram.:M.}
\end{itemize}
\begin{itemize}
\item {Proveniência:(Lat. \textunderscore aestivalis\textunderscore )}
\end{itemize}
Relativo ao estio.
Entorpecimento de certos reptis na estação calmosa.
\section{Estivar}
\begin{itemize}
\item {Grp. gram.:v. t.}
\end{itemize}
\begin{itemize}
\item {Utilização:Techn.}
\end{itemize}
\begin{itemize}
\item {Utilização:Bras. do N}
\end{itemize}
\begin{itemize}
\item {Proveniência:(Do lat. \textunderscore stipare\textunderscore )}
\end{itemize}
Pôr estiva em.
Cobrir de estiva.
Despachar na alfândega.
Acamar. Cf. \textunderscore Inquér. Industr.\textunderscore , 2.^a p., l. III, 21.
Construir pontes sôbre (terrenos alagadiços).
\section{Estivar}
\begin{itemize}
\item {Grp. gram.:v. t.}
\end{itemize}
\begin{itemize}
\item {Utilização:T. do Ribatejo}
\end{itemize}
O mesmo que \textunderscore esticar\textunderscore : \textunderscore ó Zé, estiva a tamiça, a vêr se a manta do bacello fica direita\textunderscore .
\section{Estivo}
\begin{itemize}
\item {Grp. gram.:adj.}
\end{itemize}
\begin{itemize}
\item {Proveniência:(Lat. \textunderscore aestivus\textunderscore )}
\end{itemize}
O mesmo que \textunderscore estival\textunderscore .
\section{Ésto}
\begin{itemize}
\item {Grp. gram.:m.}
\end{itemize}
\begin{itemize}
\item {Utilização:Fig.}
\end{itemize}
\begin{itemize}
\item {Proveniência:(Do lat. \textunderscore aestus\textunderscore )}
\end{itemize}
Preamar.
Enchente.
Ondulação ruidosa.
Ruído.
Effervescência.
Calor; paixão.
\section{Êsto}
\begin{itemize}
\item {Grp. gram.:pron.}
\end{itemize}
\begin{itemize}
\item {Utilização:Ant.}
\end{itemize}
O mesmo que \textunderscore isto\textunderscore .
\section{Estocada}
\begin{itemize}
\item {Grp. gram.:f.}
\end{itemize}
\begin{itemize}
\item {Utilização:Fig.}
\end{itemize}
Ferimento com estoque ou com ponta de espada.
Surpresa desagradável ou dolorosa.
\section{Estôfa}
\begin{itemize}
\item {Grp. gram.:f.}
\end{itemize}
O mesmo que \textunderscore estôfo\textunderscore ^1.
Qualidade, classe:«\textunderscore ...homem de baixa estôfa\textunderscore ». Camillo, \textunderscore Retr. de Ricard.\textunderscore , 48.
(Cast. \textunderscore estofa\textunderscore )
\section{Estofador}
\begin{itemize}
\item {Grp. gram.:m.}
\end{itemize}
\begin{itemize}
\item {Proveniência:(De \textunderscore estofar\textunderscore )}
\end{itemize}
Aquelle que estofa por offício.
Vendedor de móveis estofados e adornos mobiliários.
\section{Estofar}
\begin{itemize}
\item {Grp. gram.:v. t.}
\end{itemize}
\begin{itemize}
\item {Proveniência:(De \textunderscore estôfo\textunderscore )}
\end{itemize}
Cobrir de estôfo.
Guarnecer ou preparar com estôfo.
Tornar encorpado.
Augmentar a consistência de, metendo estôfo entre o fôrro e o tecido de (uma peça de vestuário).
Chumaçar; acolchoar.
\section{Estôfo}
\begin{itemize}
\item {Grp. gram.:m.}
\end{itemize}
\begin{itemize}
\item {Grp. gram.:Pl.}
\end{itemize}
\begin{itemize}
\item {Utilização:Ext.}
\end{itemize}
\begin{itemize}
\item {Proveniência:(Do gr. \textunderscore stope\textunderscore )}
\end{itemize}
Tecido de seda, lan, algodão, linho, estopa, etc.
Chumaço.
Lan, crina, ou outra substância, que se mete sob o tecido que reveste sofás, cadeiras, etc.
Mobília estofada: \textunderscore adquiriu muitos estofos\textunderscore .
\section{Estôfo}
\begin{itemize}
\item {Grp. gram.:adj.}
\end{itemize}
Estagnado.
Que não sobe nem desce, (falando-se das águas do mar).
\section{Estoica}
\begin{itemize}
\item {Grp. gram.:f.}
\end{itemize}
(V.estoicismo)
\section{Estoicamente}
\begin{itemize}
\item {Grp. gram.:adv.}
\end{itemize}
\begin{itemize}
\item {Proveniência:(De \textunderscore estoico\textunderscore )}
\end{itemize}
Á maneira dos estoicos.
\section{Estoicidade}
\begin{itemize}
\item {Grp. gram.:f.}
\end{itemize}
Qualidade daquelle ou daquillo que é estoico.
\section{Estoicismo}
\begin{itemize}
\item {Grp. gram.:m.}
\end{itemize}
\begin{itemize}
\item {Utilização:Ext.}
\end{itemize}
\begin{itemize}
\item {Proveniência:(De \textunderscore estoico\textunderscore )}
\end{itemize}
Doutrina dos estoicos, segundo a qual um homem deve sêr insensível a todos os males e a todas as prosperidades, praticando só a virtude.
Austeridade.
Rigidez de princípios moraes.
\section{Estóico}
\begin{itemize}
\item {Grp. gram.:adj.}
\end{itemize}
\begin{itemize}
\item {Utilização:Ext.}
\end{itemize}
\begin{itemize}
\item {Grp. gram.:M.}
\end{itemize}
\begin{itemize}
\item {Utilização:Ext.}
\end{itemize}
\begin{itemize}
\item {Proveniência:(Gr. \textunderscore stoikos\textunderscore )}
\end{itemize}
Relativo ao estoicismo.
Austero.
Impassível.
Partidário do estoicismo.
Homem austero, impassível perante a desgraça ou a felicidade.
\section{Estoicomancia}
\begin{itemize}
\item {Grp. gram.:f.}
\end{itemize}
\begin{itemize}
\item {Proveniência:(Do gr. \textunderscore stoikos\textunderscore  + \textunderscore manteia\textunderscore )}
\end{itemize}
Arte de adivinhar, pelas primeiras palavras, que se encontram nos livros.--Vejo a palavra em Castilho, \textunderscore Fastos\textunderscore , III, 321, sob a fórma errónea de \textunderscore estoicheiomancia\textunderscore .
\section{Estóio}
\begin{itemize}
\item {Grp. gram.:m.}
\end{itemize}
O mesmo que \textunderscore catróio\textunderscore .
\section{Estoirada}
\begin{itemize}
\item {Grp. gram.:f.}
\end{itemize}
\begin{itemize}
\item {Utilização:Fam.}
\end{itemize}
\begin{itemize}
\item {Proveniência:(De \textunderscore estoirar\textunderscore )}
\end{itemize}
Ruído de muitos estoiros.
Ralhos; pancadaria.
\section{Estoiradinho}
\begin{itemize}
\item {Grp. gram.:m.}
\end{itemize}
\begin{itemize}
\item {Utilização:Fam.}
\end{itemize}
\begin{itemize}
\item {Proveniência:(De \textunderscore estoirar\textunderscore )}
\end{itemize}
Petimetre; janota.
Paparrotão. Cf. Camillo, \textunderscore Cancion. Al.\textunderscore , 342.
\section{Estoirado}
\begin{itemize}
\item {Grp. gram.:adj.}
\end{itemize}
\begin{itemize}
\item {Utilização:Fam.}
\end{itemize}
\begin{itemize}
\item {Proveniência:(De \textunderscore estoirar\textunderscore )}
\end{itemize}
Que estoirou.
Muito aberto, esbugalhado, (falando-se dos olhos).
\section{Estoira-folle}
\begin{itemize}
\item {Grp. gram.:m.}
\end{itemize}
\begin{itemize}
\item {Utilização:Prov.}
\end{itemize}
\begin{itemize}
\item {Utilização:beir.}
\end{itemize}
O mesmo que \textunderscore dedaleira\textunderscore , cujas flôres os rapazes fazem estoirar, fechando-lhes a bôca e comprimindo-as.
\section{Estoiral}
\begin{itemize}
\item {Grp. gram.:adj.}
\end{itemize}
Relativo a estoiro:«\textunderscore ...as castanhas estoiraes.\textunderscore »Filinto, II, 8.
\section{Estoirar}
\begin{itemize}
\item {Grp. gram.:v. t.}
\end{itemize}
\begin{itemize}
\item {Utilização:Fig.}
\end{itemize}
\begin{itemize}
\item {Grp. gram.:V. i.}
\end{itemize}
\begin{itemize}
\item {Utilização:Fam.}
\end{itemize}
Fazer rebentar com estrondo.
Estalar.
Irritar.
Dar estoiro.
Explodir.
Estalar.
Ribombar.
Rebentar: \textunderscore a bomba estoirou\textunderscore .
Expandir-se em brados.
\section{Estoiraria}
\begin{itemize}
\item {Grp. gram.:f.}
\end{itemize}
\begin{itemize}
\item {Utilização:Bras}
\end{itemize}
\begin{itemize}
\item {Utilização:Neol.}
\end{itemize}
Successão de estoiros; estoirada.
\section{Estoira-vêrgas}
\begin{itemize}
\item {Grp. gram.:m.}
\end{itemize}
\begin{itemize}
\item {Utilização:Pop.}
\end{itemize}
Valdevinos.
Homem estouvado, inquieto, rixoso.
\section{Estoiraz}
\begin{itemize}
\item {Grp. gram.:adj.}
\end{itemize}
\begin{itemize}
\item {Proveniência:(De \textunderscore estoiro\textunderscore )}
\end{itemize}
Que estoira; ruidoso.
\section{Estoirinhar}
\begin{itemize}
\item {Grp. gram.:v. i.}
\end{itemize}
\begin{itemize}
\item {Utilização:beir.}
\end{itemize}
\begin{itemize}
\item {Utilização:Pop.}
\end{itemize}
Saltar como o toiro.
\section{Estoiro}
\begin{itemize}
\item {Grp. gram.:m.}
\end{itemize}
\begin{itemize}
\item {Utilização:Fam.}
\end{itemize}
\begin{itemize}
\item {Proveniência:(De \textunderscore estoirar\textunderscore )}
\end{itemize}
Ruído, produzido por um corpo que rebenta.
Estampido; explosão.
Bofetada.
Pancada.
\section{Estoirote}
\begin{itemize}
\item {Grp. gram.:m.}
\end{itemize}
\begin{itemize}
\item {Utilização:Prov.}
\end{itemize}
\begin{itemize}
\item {Utilização:trasm.}
\end{itemize}
Estalo, que se dá com os dedos.
\section{Estojar}
\begin{itemize}
\item {Grp. gram.:v. t.}
\end{itemize}
\begin{itemize}
\item {Utilização:Des.}
\end{itemize}
Meter ou guardar em estôjo.
\section{Estojaria}
\begin{itemize}
\item {Grp. gram.:f.}
\end{itemize}
\begin{itemize}
\item {Utilização:T. de Lisbôa}
\end{itemize}
Officina de estojos. Cf. \textunderscore Diár. de Noticias\textunderscore , de 14-VII-900.
\section{Estojeira}
\begin{itemize}
\item {Grp. gram.:f.}
\end{itemize}
\begin{itemize}
\item {Utilização:T. de Lisbôa}
\end{itemize}
Mulher ou rapariga, que faz estojos para ourivezarias.
\section{Estôjo}
\begin{itemize}
\item {Grp. gram.:m.}
\end{itemize}
\begin{itemize}
\item {Utilização:Bot.}
\end{itemize}
\begin{itemize}
\item {Proveniência:(Do ant. alt. al. \textunderscore stuche\textunderscore )}
\end{itemize}
Pequena caixa, com divisões, para guardar apparelhos, instrumentos, etc.
Espécie de bolsa de coiro, papelão, etc., em que se guardam tesoiras, canivetes e outros objectos.
Cavidade, que contém a medulla do caule das plantas lenhosas.
\section{Estola}
\begin{itemize}
\item {Grp. gram.:f.}
\end{itemize}
\begin{itemize}
\item {Utilização:Ant.}
\end{itemize}
\begin{itemize}
\item {Proveniência:(Do lat. \textunderscore stola\textunderscore )}
\end{itemize}
Tira de seda, que se alarga nas extremidades e que os sacerdotes põem aos ombros, entre a alva e a casula, ou por cima da sobrepelliz.
Vestido talar das matronas romanas.
Craveira, igualha?:«\textunderscore ...travados com outros da sua estola...\textunderscore »Camillo, \textunderscore Regicida\textunderscore , 40. (Talvez neste exemplo houvesse êrro typogr. de \textunderscore estola\textunderscore , por \textunderscore estôfa\textunderscore )
\section{Estolão}
\begin{itemize}
\item {Grp. gram.:m.}
\end{itemize}
Estola grande.
\section{Estolho}
\begin{itemize}
\item {fónica:tô}
\end{itemize}
\begin{itemize}
\item {Grp. gram.:m.}
\end{itemize}
\begin{itemize}
\item {Proveniência:(Do lat. \textunderscore stolo\textunderscore )}
\end{itemize}
Rebento de plantas, que, fixando-se na terra, lança raízes, de espaço a espaço, como succede no morangueiro.
\section{Estolhosa}
\begin{itemize}
\item {Grp. gram.:f.}
\end{itemize}
\begin{itemize}
\item {Proveniência:(De \textunderscore estolhoso\textunderscore )}
\end{itemize}
Uma das três espécies agróstide, que se ramifica no chão. Cf. \textunderscore Gazeta das Aldeias\textunderscore , n.^o 107.
\section{Estolhoso}
\begin{itemize}
\item {Grp. gram.:adj.}
\end{itemize}
Que tem estolhos ou deita estolhos.
\section{Estolidamente}
\begin{itemize}
\item {Grp. gram.:adv.}
\end{itemize}
De modo estólido.
\section{Estolidez}
\begin{itemize}
\item {Grp. gram.:adj.}
\end{itemize}
Qualidade daquelle ou daquillo que é estólido.
\section{Estólido}
\begin{itemize}
\item {Grp. gram.:adj.}
\end{itemize}
\begin{itemize}
\item {Proveniência:(Lat. \textunderscore stolidus\textunderscore )}
\end{itemize}
Estouvado.
Parvo.
Estúpido.
Disparatado.
\section{Estomacal}
\begin{itemize}
\item {Grp. gram.:adj.}
\end{itemize}
\begin{itemize}
\item {Proveniência:(Lat. \textunderscore stomachalis\textunderscore )}
\end{itemize}
Relativo a estômago.
\section{Estomachal}
\begin{itemize}
\item {fónica:cal}
\end{itemize}
\begin{itemize}
\item {Grp. gram.:adj.}
\end{itemize}
\begin{itemize}
\item {Proveniência:(Lat. \textunderscore stomachalis\textunderscore )}
\end{itemize}
Relativo a estômago.
\section{Estomáchico}
\begin{itemize}
\item {fónica:qui}
\end{itemize}
\begin{itemize}
\item {Grp. gram.:adj.}
\end{itemize}
\begin{itemize}
\item {Proveniência:(Lat. \textunderscore stomachicus\textunderscore )}
\end{itemize}
O mesmo que \textunderscore estomacal\textunderscore .
\section{Estomagar}
\begin{itemize}
\item {Grp. gram.:v. t.}
\end{itemize}
\begin{itemize}
\item {Proveniência:(Lat. \textunderscore stomachari\textunderscore )}
\end{itemize}
Agastar.
Indignar; escandalizar.
\section{Estômago}
\begin{itemize}
\item {Grp. gram.:m.}
\end{itemize}
\begin{itemize}
\item {Utilização:Fig.}
\end{itemize}
\begin{itemize}
\item {Proveniência:(Do lat. \textunderscore stomachus\textunderscore )}
\end{itemize}
Víscera, em que se faz a digestão dos alimentos.
Parte exterior do corpo, correspondente á região estomacal: \textunderscore uma pancada no estômago\textunderscore .
Disposição, ânimo: \textunderscore não tenho estômago para o aturar\textunderscore .
Appetite; satisfação.
\section{Estomápode}
\begin{itemize}
\item {Grp. gram.:adj.}
\end{itemize}
\begin{itemize}
\item {Utilização:Zool.}
\end{itemize}
\begin{itemize}
\item {Grp. gram.:M. pl.}
\end{itemize}
\begin{itemize}
\item {Proveniência:(Do gr. \textunderscore stoma\textunderscore  + \textunderscore pous\textunderscore , \textunderscore podos\textunderscore )}
\end{itemize}
Que tem as patas ou barbatanas junto da bôca.
Ordem de crustáceos nadadores.
\section{Estomáquico}
\begin{itemize}
\item {Grp. gram.:adj.}
\end{itemize}
\begin{itemize}
\item {Proveniência:(Lat. \textunderscore stomachicus\textunderscore )}
\end{itemize}
O mesmo que \textunderscore estomacal\textunderscore .
\section{Estomático}
\begin{itemize}
\item {Grp. gram.:adj.}
\end{itemize}
\begin{itemize}
\item {Proveniência:(Do gr. \textunderscore stoma\textunderscore )}
\end{itemize}
Diz-se dos medicamentos contra as doenças da bôca.
\section{Estomatite}
\begin{itemize}
\item {Grp. gram.:f.}
\end{itemize}
\begin{itemize}
\item {Proveniência:(Do gr. \textunderscore stoma\textunderscore )}
\end{itemize}
Inflammação da mucosa da bôca.
\section{Estomatologia}
\begin{itemize}
\item {Grp. gram.:f.}
\end{itemize}
\begin{itemize}
\item {Utilização:Med.}
\end{itemize}
\begin{itemize}
\item {Proveniência:(Do gr. \textunderscore stoma\textunderscore , \textunderscore stomatos\textunderscore  + \textunderscore logos\textunderscore )}
\end{itemize}
Estudo da bôca e das suas moléstias.
\section{Estomatológico}
\begin{itemize}
\item {Grp. gram.:adj.}
\end{itemize}
Relativo a estomatologia.
\section{Estomatologista}
\begin{itemize}
\item {Grp. gram.:m.}
\end{itemize}
Tradista de estomatologia.
\section{Estómatos}
\begin{itemize}
\item {Grp. gram.:m. pl.}
\end{itemize}
\begin{itemize}
\item {Utilização:Bot.}
\end{itemize}
\begin{itemize}
\item {Proveniência:(Do gr. \textunderscore stoma\textunderscore )}
\end{itemize}
Poros microscópicos dos tecidos herbáceos.
\section{Estomatoscópico}
\begin{itemize}
\item {Grp. gram.:adj.}
\end{itemize}
Relativo ao \textunderscore estomatoscópio\textunderscore .
\section{Estomatoscópio}
\begin{itemize}
\item {Grp. gram.:m.}
\end{itemize}
\begin{itemize}
\item {Proveniência:(Do gr. \textunderscore stoma\textunderscore  + \textunderscore skopein\textunderscore )}
\end{itemize}
Instrumento cirúrgico para conservar aberta a bôca, quando dentro della se faz alguma operação ou observação.
\section{Estomegar}
\begin{itemize}
\item {Grp. gram.:v. t.}
\end{itemize}
\begin{itemize}
\item {Utilização:Prov.}
\end{itemize}
\begin{itemize}
\item {Utilização:trasm.}
\end{itemize}
Torcer, estorcegar (um pé).
\section{Estomentar}
\begin{itemize}
\item {Grp. gram.:v. t.}
\end{itemize}
\begin{itemize}
\item {Utilização:Fig.}
\end{itemize}
Tirar os tomentos a (o linho).
Limpar, depurar. Cf. \textunderscore Eufrosina\textunderscore , 189, \textunderscore Autegrafia\textunderscore , XIV.
\section{Estomoxo}
\begin{itemize}
\item {fónica:cso}
\end{itemize}
\begin{itemize}
\item {Grp. gram.:m.}
\end{itemize}
\begin{itemize}
\item {Proveniência:(Do gr. \textunderscore stoma\textunderscore  + \textunderscore oxus\textunderscore )}
\end{itemize}
Insecto díptero.
\section{Estonador}
\begin{itemize}
\item {Grp. gram.:m.}
\end{itemize}
\begin{itemize}
\item {Proveniência:(De \textunderscore estonar\textunderscore )}
\end{itemize}
Instrumento, para descascar.
\section{Estonadura}
\begin{itemize}
\item {Grp. gram.:f.}
\end{itemize}
O mesmo que \textunderscore estonamento\textunderscore .
\section{Estonamento}
\begin{itemize}
\item {Grp. gram.:m.}
\end{itemize}
Acto de estonar.
\section{Estonar}
\begin{itemize}
\item {Grp. gram.:v. t.}
\end{itemize}
\begin{itemize}
\item {Utilização:Prov.}
\end{itemize}
\begin{itemize}
\item {Utilização:T. da Bairrada}
\end{itemize}
\begin{itemize}
\item {Proveniência:(De \textunderscore tona\textunderscore )}
\end{itemize}
Tirar a tona ou a casca a.
Tirar a pelle a, por effeito de queimadura: \textunderscore estonar um pé\textunderscore .
Escaldar; chamuscar.
Tosquiar.
Tirar as espigas do (milhal).
\section{Estonce}
\begin{itemize}
\item {Grp. gram.:adv.}
\end{itemize}
\begin{itemize}
\item {Utilização:Ant.}
\end{itemize}
O mesmo que \textunderscore estonces\textunderscore .
\section{Estonces}
\begin{itemize}
\item {Grp. gram.:adv.}
\end{itemize}
\begin{itemize}
\item {Utilização:Ant.}
\end{itemize}
Então.
(Cast. \textunderscore estonces\textunderscore )
\section{Estoneiro}
\begin{itemize}
\item {Grp. gram.:m.}
\end{itemize}
\begin{itemize}
\item {Utilização:Prov.}
\end{itemize}
\begin{itemize}
\item {Utilização:minh.}
\end{itemize}
\begin{itemize}
\item {Proveniência:(De \textunderscore estonar\textunderscore )}
\end{itemize}
Cajado ou bastão, feito de uma vergôntea, a que se tirou a casca, estonando-a ao lume.
\section{Estontar}
\begin{itemize}
\item {Grp. gram.:v. t.  e  p.}
\end{itemize}
(V.estontear)
\section{Estonteador}
\begin{itemize}
\item {Grp. gram.:adj.}
\end{itemize}
Que estonteia.
\section{Estonteadoramente}
\begin{itemize}
\item {Grp. gram.:adv.}
\end{itemize}
De modo estonteador.
\section{Estonteamento}
\begin{itemize}
\item {Grp. gram.:m.}
\end{itemize}
Acto ou effeito de estontear.
\section{Estontear}
\begin{itemize}
\item {Grp. gram.:v. t.}
\end{itemize}
Tornar tonto.
Atordoar; perturbar.
\section{Estontecer}
\begin{itemize}
\item {Grp. gram.:v. i.}
\end{itemize}
O mesmo que \textunderscore entontecer\textunderscore . Cf. Arn. Gama, \textunderscore Últ. Dona\textunderscore , 375 e 383.
\section{Estopa}
\begin{itemize}
\item {Grp. gram.:f.}
\end{itemize}
\begin{itemize}
\item {Proveniência:(Do gr. \textunderscore stope\textunderscore )}
\end{itemize}
A parte mais grossa do linho, a qual se separa dêste, por meio do sedeiro.
Tecido, fabricado de estopa.
Filamento interior da noz de coco.
Fios desfeitos, com que se calafetam as embarcações.
\section{Estopa-boi}
\begin{itemize}
\item {Grp. gram.:m.}
\end{itemize}
\begin{itemize}
\item {Utilização:Bras}
\end{itemize}
Árvore silvestre, de cuja madeira se fazem frechaes e vigotas.
\section{Estopada}
\begin{itemize}
\item {Grp. gram.:f.}
\end{itemize}
\begin{itemize}
\item {Utilização:Fam.}
\end{itemize}
\begin{itemize}
\item {Utilização:T. da Bairrada}
\end{itemize}
\begin{itemize}
\item {Proveniência:(De \textunderscore estopar\textunderscore ^1)}
\end{itemize}
Estopa embebida em líquido ou num medicamento:«\textunderscore ...e lhe pus em cima duas estopadas de ovos.\textunderscore »\textunderscore Peregrinação\textunderscore , CXXXVII.
Porção de estopa para chumaço.
Coisa que enfada.
Maçada.
Brincadeira carnavalesca, em que se finge lutar, com estopa acesa.
\section{Estopador}
\begin{itemize}
\item {Grp. gram.:adj.}
\end{itemize}
\begin{itemize}
\item {Utilização:Fam.}
\end{itemize}
\begin{itemize}
\item {Proveniência:(De \textunderscore estopar\textunderscore ^1)}
\end{itemize}
Maçador, importuno. Cf. Camillo, \textunderscore Sc. da Foz\textunderscore , 93.
\section{Estopagado}
\begin{itemize}
\item {Grp. gram.:m.}
\end{itemize}
Ave aquática da costa occidental da África.
\section{Estopante}
\begin{itemize}
\item {Grp. gram.:adj.}
\end{itemize}
\begin{itemize}
\item {Utilização:Fam.}
\end{itemize}
O mesmo que \textunderscore estopador\textunderscore .
\section{Estopar}
\begin{itemize}
\item {Grp. gram.:v. t.}
\end{itemize}
\begin{itemize}
\item {Utilização:Fam.}
\end{itemize}
Encher de estopa.
Calafetar com estopa.
Enfadar, maçar.
\section{Estopar}
\begin{itemize}
\item {Grp. gram.:adj.}
\end{itemize}
Diz-se de um prego curto, de cabeça larga, usado a hordo.
\section{Estopento}
\begin{itemize}
\item {Grp. gram.:adj.}
\end{itemize}
Que tem filamentos como a estopa.
\section{Estopentudo}
\begin{itemize}
\item {Grp. gram.:adj.}
\end{itemize}
Parecido com estopa:«\textunderscore estrigas do cabello estupentudas.\textunderscore »Camillo, \textunderscore Esqueleto\textunderscore , 48.
\section{Estopetar}
\begin{itemize}
\item {Grp. gram.:v. t.}
\end{itemize}
Desmanchar o topête a; despentear.
\section{Estopim}
\begin{itemize}
\item {Grp. gram.:m.}
\end{itemize}
\begin{itemize}
\item {Proveniência:(De \textunderscore estopa\textunderscore )}
\end{itemize}
Fios, embebidos em substância inflammável, para se communicar fogo a peças pyrotéchnicas, a bombas, minas, etc.
\section{Estopinha}
\begin{itemize}
\item {Grp. gram.:f.}
\end{itemize}
\begin{itemize}
\item {Grp. gram.:Pl.}
\end{itemize}
\begin{itemize}
\item {Utilização:Fam.}
\end{itemize}
\begin{itemize}
\item {Proveniência:(De \textunderscore estopa\textunderscore )}
\end{itemize}
Os filamentos mais finos do linho, antes de fiado.

Espécie de jôgo popular.
\textunderscore Falar as estopinhas\textunderscore , falar muito, falar pelos cotovelos.
\section{Estoposo}
\begin{itemize}
\item {Grp. gram.:adj.}
\end{itemize}
\begin{itemize}
\item {Proveniência:(De \textunderscore estopa\textunderscore ?)}
\end{itemize}
Diz-se do casco dos solípedes, quando volumoso ou em desproporção com as outras partes do corpo. Cf. Leon., \textunderscore Arte de ferrar\textunderscore , 156.
\section{Estoque}
\begin{itemize}
\item {Grp. gram.:m.}
\end{itemize}
\begin{itemize}
\item {Proveniência:(Do germ. \textunderscore stock\textunderscore )}
\end{itemize}
Arma branca, comprida e direita, de forma prismática, e que só fere com a ponta.
\section{Estoqueadura}
\begin{itemize}
\item {Grp. gram.:f.}
\end{itemize}
\begin{itemize}
\item {Proveniência:(De \textunderscore estoquear\textunderscore )}
\end{itemize}
O mesmo que \textunderscore estocada\textunderscore .
\section{Estoquear}
\begin{itemize}
\item {Grp. gram.:v. t.}
\end{itemize}
Ferir com estoque.
\section{Estoraque}
\begin{itemize}
\item {Grp. gram.:m.}
\end{itemize}
\begin{itemize}
\item {Proveniência:(Do lat. \textunderscore styrax\textunderscore )}
\end{itemize}
Bálsamo, que varía de consistência e que exhala um cheiro muito agradável, semelhante ao do ácido, benzoico.
Arbusto, que produz êsse balsamo.
\section{Estoraque}
\begin{itemize}
\item {Grp. gram.:m.}
\end{itemize}
\begin{itemize}
\item {Utilização:Prov.}
\end{itemize}
\begin{itemize}
\item {Utilização:beir.}
\end{itemize}
Pessôa leviana; doidivanas.
\section{Estorcegadela}
\begin{itemize}
\item {Grp. gram.:f.}
\end{itemize}
Acto de estorcegar; torcedura.
\section{Estorcegão}
\begin{itemize}
\item {Grp. gram.:m.}
\end{itemize}
\begin{itemize}
\item {Proveniência:(De \textunderscore estorcegar\textunderscore )}
\end{itemize}
Beliscão forte.
\section{Estorcegar}
\begin{itemize}
\item {Grp. gram.:v. t.}
\end{itemize}
\begin{itemize}
\item {Proveniência:(Do rad. de \textunderscore estorcer\textunderscore )}
\end{itemize}
Torcer com fôrça.
Estorcer.
Beliscar.
\section{Estorcer}
\begin{itemize}
\item {Grp. gram.:v. t.}
\end{itemize}
\begin{itemize}
\item {Grp. gram.:V. i.}
\end{itemize}
\begin{itemize}
\item {Grp. gram.:V. p.}
\end{itemize}
\begin{itemize}
\item {Proveniência:(De \textunderscore torcer\textunderscore )}
\end{itemize}
Torcer com fôrça.
Pôr em convulsão.
Contorcer.
Mudar de direcção.
Têr convulsões.
Debater-se; escabujar.
\section{Estorcido}
\begin{itemize}
\item {Grp. gram.:adj.}
\end{itemize}
O mesmo que \textunderscore destorcido\textunderscore , direito:«\textunderscore ruas largas e bem estorcidas.\textunderscore »Filinto, \textunderscore D. Man.\textunderscore , I, 75.
\section{Estorcimento}
\begin{itemize}
\item {Grp. gram.:m.}
\end{itemize}
Acto de estorcer.
\section{Estorço}
\begin{itemize}
\item {fónica:tôr}
\end{itemize}
\begin{itemize}
\item {Grp. gram.:m.}
\end{itemize}
\begin{itemize}
\item {Utilização:Pop.}
\end{itemize}
\begin{itemize}
\item {Proveniência:(De \textunderscore estorcer\textunderscore )}
\end{itemize}
Posição violenta, contrafeita.
\section{Estore}
\begin{itemize}
\item {Grp. gram.:m.}
\end{itemize}
\begin{itemize}
\item {Proveniência:(Fr. \textunderscore store\textunderscore )}
\end{itemize}
Cortina móvel, para janelas ou carruagens, que se enrola e desenrola por meio de apparelho.
Empanada^2.
\section{Estorga}
\begin{itemize}
\item {Grp. gram.:f.}
\end{itemize}
\begin{itemize}
\item {Proveniência:(De \textunderscore torga\textunderscore )}
\end{itemize}
(V.urze)
\section{Estormo}
\begin{itemize}
\item {fónica:tôr}
\end{itemize}
\begin{itemize}
\item {Grp. gram.:m.}
\end{itemize}
\begin{itemize}
\item {Utilização:T. de Aveiro}
\end{itemize}
Planta agreste e resistente, espécie de urze.
\section{Estornar}
\begin{itemize}
\item {Grp. gram.:v. t.}
\end{itemize}
\begin{itemize}
\item {Proveniência:(De \textunderscore estôrno\textunderscore )}
\end{itemize}
Lançar em conta de crédito (aquillo que se havia lançado em débito) ou vice-versa.
Dissolver ou distractar (contrato de seguro marítimo).
\section{Estorninho}
\begin{itemize}
\item {Grp. gram.:m.}
\end{itemize}
\begin{itemize}
\item {Grp. gram.:Adj.}
\end{itemize}
\begin{itemize}
\item {Proveniência:(Do lat. \textunderscore sturnus\textunderscore )}
\end{itemize}
Pássaro conirostro, de plumagem escura, matizada de branco, verde e encarnado.

Diz-se do toiro zaino com pequenas manchas brancas.
\section{Estôrno}
\begin{itemize}
\item {Grp. gram.:m.}
\end{itemize}
\begin{itemize}
\item {Utilização:Jur.}
\end{itemize}
\begin{itemize}
\item {Proveniência:(Do lat. \textunderscore exturnare\textunderscore )}
\end{itemize}
Acto de estornar.
Dissolução de um contrato de seguro marítimo. Cf. F. Borges, \textunderscore Diccion. Jur.\textunderscore 
\section{Estorrejar}
\begin{itemize}
\item {Grp. gram.:v. t.}
\end{itemize}
\begin{itemize}
\item {Utilização:Prov.}
\end{itemize}
Torrar muito; queimar um pouco. (Colhido em Turquel)
\section{Estorricar}
\begin{itemize}
\item {Grp. gram.:v. t.}
\end{itemize}
\begin{itemize}
\item {Proveniência:(De \textunderscore torrar\textunderscore )}
\end{itemize}
Secar muito; tostar.
\section{Estorroar}
\begin{itemize}
\item {Grp. gram.:v. t.}
\end{itemize}
\begin{itemize}
\item {Proveniência:(De \textunderscore torrão\textunderscore )}
\end{itemize}
O mesmo que \textunderscore esterroar\textunderscore .
\section{Estortegada}
\begin{itemize}
\item {Grp. gram.:f.}
\end{itemize}
O mesmo que \textunderscore estortegadura\textunderscore .
\section{Estortegadela}
\begin{itemize}
\item {Grp. gram.:f.}
\end{itemize}
\begin{itemize}
\item {Utilização:Pop.}
\end{itemize}
O mesmo que \textunderscore estortegadura\textunderscore .
\section{Estortegadura}
\begin{itemize}
\item {Grp. gram.:f.}
\end{itemize}
Acto ou effeito de estortegar.
\section{Estortegar}
\begin{itemize}
\item {Grp. gram.:v. t.}
\end{itemize}
O mesmo que \textunderscore estorcegar\textunderscore . Cp. G. Vicente, I, 347.
\section{Estorturar-se}
\begin{itemize}
\item {Grp. gram.:v. p.}
\end{itemize}
Contorcer-se, soffrendo.
Têr tortura. Cf. Camillo, \textunderscore Doze Casam.\textunderscore , 56.
\section{Estorva}
\begin{itemize}
\item {Grp. gram.:f.}
\end{itemize}
Acto de estorvar.
\section{Estorvador}
\begin{itemize}
\item {Grp. gram.:adj.}
\end{itemize}
\begin{itemize}
\item {Grp. gram.:M.}
\end{itemize}
Que estorva.
Aquelle que estorva.
\section{Estorvamento}
\begin{itemize}
\item {Grp. gram.:m.}
\end{itemize}
O mesmo que \textunderscore estôrvo\textunderscore .
\section{Estorvante}
\begin{itemize}
\item {Grp. gram.:adj.}
\end{itemize}
Que estorva, que embaraça.
\section{Estorvar}
\begin{itemize}
\item {Grp. gram.:v. t.}
\end{itemize}
\begin{itemize}
\item {Proveniência:(Do lat. \textunderscore exturbare\textunderscore )}
\end{itemize}
Fazer estôrvo a.
Impedir, embaraçar.
\section{Estorvas}
\begin{itemize}
\item {Grp. gram.:f. pl.}
\end{itemize}
\begin{itemize}
\item {Proveniência:(De \textunderscore estorva\textunderscore ?)}
\end{itemize}
Costuras do costado do navio, de alto abaixo.
\section{Estorvilho}
\begin{itemize}
\item {Grp. gram.:m.}
\end{itemize}
Pequeno estôrvo.
Empecilho.
\section{Estôrvo}
\begin{itemize}
\item {Grp. gram.:m.}
\end{itemize}
\begin{itemize}
\item {Proveniência:(De \textunderscore estorvar\textunderscore )}
\end{itemize}
Impedimento; difficuldade.
Opposição.
\section{Estorvor}
\begin{itemize}
\item {Grp. gram.:m.}
\end{itemize}
\begin{itemize}
\item {Utilização:Prov.}
\end{itemize}
\begin{itemize}
\item {Utilização:alg.}
\end{itemize}
\begin{itemize}
\item {Proveniência:(De \textunderscore estorvar\textunderscore )}
\end{itemize}
Estôrvo.
\section{Estou-fraca}
\begin{itemize}
\item {Grp. gram.:f.}
\end{itemize}
O mesmo que \textunderscore gallinha-da-índia\textunderscore  ou \textunderscore pintada\textunderscore .
(Nome imitativo do canto desta ave)
\section{Estoupeirado}
\begin{itemize}
\item {Grp. gram.:adj.}
\end{itemize}
\begin{itemize}
\item {Utilização:Prov.}
\end{itemize}
\begin{itemize}
\item {Utilização:minh.}
\end{itemize}
\begin{itemize}
\item {Proveniência:(De \textunderscore toupeira\textunderscore )}
\end{itemize}
Diz-se do terreno, minado por toupeiras ou ratos.
\section{Estourinhar}
\begin{itemize}
\item {Grp. gram.:v. i.}
\end{itemize}
\begin{itemize}
\item {Utilização:beir.}
\end{itemize}
\begin{itemize}
\item {Utilização:Pop.}
\end{itemize}
Saltar como o touro.
\section{Est'outro}
\begin{itemize}
\item {Grp. gram.:pron.}
\end{itemize}
(designativo de um objecto próximo, que distinguimos de outro também próximo)
\section{Êste-outro}
\begin{itemize}
\item {Grp. gram.:pron.}
\end{itemize}
(designativo de um objecto próximo, que distinguimos de outro também próximo)
\section{Êste outro}
\begin{itemize}
\item {Grp. gram.:pron.}
\end{itemize}
(designativo de um objecto próximo, que distinguimos de outro também próximo)
\section{Estouvadamente}
\begin{itemize}
\item {Grp. gram.:adv.}
\end{itemize}
De modo estouvado.
Levianamente.
\section{Estouvado}
\begin{itemize}
\item {Grp. gram.:adj.}
\end{itemize}
Que não tem bom senso.
Que pensa pouco.
Imprudente.
Folgazão.
(Corr. de \textunderscore estavanado\textunderscore )
\section{Estouvamento}
\begin{itemize}
\item {Grp. gram.:m.}
\end{itemize}
Qualidade de estouvado.
Leviandade.
\section{Estouvanado}
\begin{itemize}
\item {Grp. gram.:adj.}
\end{itemize}
O mesmo que \textunderscore estavanado\textunderscore .
Inquieto, desenvolto. Cf. Camillo, \textunderscore Corja\textunderscore , 167.
\section{Estouvanice}
\begin{itemize}
\item {Grp. gram.:f.}
\end{itemize}
O mesmo que \textunderscore estouvamento\textunderscore . Cf. Camillo, \textunderscore Corja\textunderscore , 129; \textunderscore Caveira\textunderscore , 277.
\section{Estouvice}
\begin{itemize}
\item {Grp. gram.:f.}
\end{itemize}
O mesmo que \textunderscore estouvanice\textunderscore . Cf. Camillo, \textunderscore Judeu\textunderscore , IX; F. Alex. Lobo, II, 114.
\section{Estrabação}
\begin{itemize}
\item {Grp. gram.:f.}
\end{itemize}
O mesmo que \textunderscore estrabo\textunderscore . Cf. Camillo, \textunderscore Narcót.\textunderscore , I, 39.
\section{Estrabada}
\begin{itemize}
\item {Grp. gram.:f.}
\end{itemize}
O mesmo que \textunderscore estrabo\textunderscore .
\section{Estrabão}
\begin{itemize}
\item {Grp. gram.:m.  e  adj.}
\end{itemize}
\begin{itemize}
\item {Proveniência:(Do lat. \textunderscore strabo\textunderscore )}
\end{itemize}
Vesgo.
\section{Estrabar}
\begin{itemize}
\item {Grp. gram.:v. t.}
\end{itemize}
\begin{itemize}
\item {Proveniência:(Do lat. \textunderscore stabulare\textunderscore )}
\end{itemize}
Defecar, (falando-se de bêstas).
\section{Estrábico}
\begin{itemize}
\item {Grp. gram.:m.  e  adj.}
\end{itemize}
\begin{itemize}
\item {Proveniência:(Do gr. \textunderscore strabos\textunderscore )}
\end{itemize}
O que tem estrabismo, que é vesgo.
\section{Estrabismo}
\begin{itemize}
\item {Grp. gram.:m.}
\end{itemize}
\begin{itemize}
\item {Proveniência:(Gr. \textunderscore strabismos\textunderscore )}
\end{itemize}
Qualidade de quem é vesgo.
\section{Estrabo}
\begin{itemize}
\item {Grp. gram.:m.}
\end{itemize}
Acto ou effeito de estrabar.
Excremento de bêstas.
\section{Estrabometria}
\begin{itemize}
\item {Grp. gram.:f.}
\end{itemize}
Applicação do estrabómetro.
\section{Estrabómetro}
\begin{itemize}
\item {Grp. gram.:m.}
\end{itemize}
\begin{itemize}
\item {Proveniência:(Do gr. \textunderscore strabos\textunderscore  + \textunderscore metron\textunderscore )}
\end{itemize}
Instrumento, para medir o grau de estrabismo.
\section{Estrabotomia}
\begin{itemize}
\item {Grp. gram.:f.}
\end{itemize}
\begin{itemize}
\item {Proveniência:(Do gr. \textunderscore strabos\textunderscore  + \textunderscore tome\textunderscore )}
\end{itemize}
Córte de um ou mais músculos do ôlho, para remediar o estrabismo.
\section{Estrabotomista}
\begin{itemize}
\item {Grp. gram.:m.}
\end{itemize}
Aquelle que pratíca a estrabotomia.
\section{Estraboucha}
\begin{itemize}
\item {Grp. gram.:m.}
\end{itemize}
\begin{itemize}
\item {Utilização:T. de Turquel}
\end{itemize}
Indivíduo, que faz habitualmente estrabouchada.
\section{Estrabouchada}
\begin{itemize}
\item {Grp. gram.:f.}
\end{itemize}
Acto de estrabouchar.
\section{Estrabouchar}
\begin{itemize}
\item {Grp. gram.:v. i.}
\end{itemize}
\begin{itemize}
\item {Utilização:T. de Turquel}
\end{itemize}
Fazer estrépito, barulho, desordem.
\section{Estraçalhamento}
\begin{itemize}
\item {Grp. gram.:m.}
\end{itemize}
Acto ou effeito de estraçalhar.
\section{Estraçalhar}
\begin{itemize}
\item {Grp. gram.:v. t.}
\end{itemize}
\begin{itemize}
\item {Proveniência:(De \textunderscore traçar\textunderscore )}
\end{itemize}
Estraçoar, estracinhar.
\section{Estracinhar}
\textunderscore v. t.\textunderscore  (e der.)
O mesmo que \textunderscore estraçoar\textunderscore , etc.
\section{Estraçoar}
\begin{itemize}
\item {Grp. gram.:v. t.}
\end{itemize}
Traçar miudamente, fazer em pedaços.
Esmigalhar.
(Por \textunderscore estraçar\textunderscore , de \textunderscore traçar\textunderscore )
\section{Estrada}
\begin{itemize}
\item {Grp. gram.:f.}
\end{itemize}
\begin{itemize}
\item {Utilização:Fig.}
\end{itemize}
\begin{itemize}
\item {Utilização:Bras. do N}
\end{itemize}
\begin{itemize}
\item {Utilização:Pop.}
\end{itemize}
Caminho mais ou menos largo, em que podem transitar homens, animaes e vehículos.
Modo de proceder.
Procedimento recto: \textunderscore a estrada do dever\textunderscore .
Expediente; meio.
Certo passo ou andamento de cavalgaduras.
\textunderscore Estrada de Santiago\textunderscore , o mesmo que \textunderscore via-láctea\textunderscore .
\section{Estrada-de-ferro}
\begin{itemize}
\item {Grp. gram.:f.}
\end{itemize}
\begin{itemize}
\item {Utilização:Bras. de Minas}
\end{itemize}
Espécie de jôgo de cartas.
\section{Estradar}
\begin{itemize}
\item {Grp. gram.:v. t.}
\end{itemize}
\begin{itemize}
\item {Utilização:Fig.}
\end{itemize}
\begin{itemize}
\item {Proveniência:(De \textunderscore estrada\textunderscore )}
\end{itemize}
Abrir estradas ou caminhos em.
Encaminhar.
\section{Estradar}
\begin{itemize}
\item {Grp. gram.:v. t.}
\end{itemize}
\begin{itemize}
\item {Proveniência:(De \textunderscore estrado\textunderscore ^1)}
\end{itemize}
Pôr estrado em.
Soalhar; alcatifar: \textunderscore estradar uma casa\textunderscore .
\section{Estradeiro}
\begin{itemize}
\item {Grp. gram.:adj.}
\end{itemize}
\begin{itemize}
\item {Utilização:Bras}
\end{itemize}
\begin{itemize}
\item {Utilização:Fig.}
\end{itemize}
\begin{itemize}
\item {Proveniência:(De \textunderscore estrada\textunderscore )}
\end{itemize}
Que anda a passo de estrada; que anda bem, que tem bom passo.
Velhaco.
\section{Estradiota}
\begin{itemize}
\item {Grp. gram.:f.}
\end{itemize}
\begin{itemize}
\item {Grp. gram.:M. pl.}
\end{itemize}
\begin{itemize}
\item {Proveniência:(Do gr. \textunderscore stradiotes\textunderscore )}
\end{itemize}
Maneira de montar, firmando-se nos estribos e estendendo as pernas.
Albaneses, assoldados por Carlos VIII de França na expedição de Nápoles, os quaes constituíam cavallaria ligeira.
\section{Estradioto}
\begin{itemize}
\item {Grp. gram.:m.}
\end{itemize}
\begin{itemize}
\item {Utilização:Ant.}
\end{itemize}
Ladrão de estrada. Cf. Pant. de Aveiro, \textunderscore Itiner.\textunderscore , 38, (2.^a ed.)
\section{Estradivário}
\begin{itemize}
\item {Grp. gram.:m.}
\end{itemize}
Rabeca da fábrica Stradivarius, (antigo e célebre fabricante de rabecas), a qual é preferida nas orchestras, por ter vozes cheias.
\section{Estrado}
\begin{itemize}
\item {Grp. gram.:m.}
\end{itemize}
\begin{itemize}
\item {Utilização:Ant.}
\end{itemize}
\begin{itemize}
\item {Proveniência:(Do lat. \textunderscore stratum\textunderscore )}
\end{itemize}
Sobrado, um pouco levantado acima do chão ou de um pavimento.
Suppedâneo.
Tribunal.
\section{Estrado}
\begin{itemize}
\item {Grp. gram.:adj.}
\end{itemize}
\begin{itemize}
\item {Utilização:Ant.}
\end{itemize}
\begin{itemize}
\item {Proveniência:(Do lat. \textunderscore stratus\textunderscore )}
\end{itemize}
Alastrado.
\section{Estrafalário}
\begin{itemize}
\item {Grp. gram.:adj.}
\end{itemize}
\begin{itemize}
\item {Utilização:Fam.}
\end{itemize}
Desajeitado.
Extravagante, ridículo.
\section{Estrafega}
\begin{itemize}
\item {Grp. gram.:f.}
\end{itemize}
\begin{itemize}
\item {Utilização:Prov.}
\end{itemize}
\begin{itemize}
\item {Proveniência:(De \textunderscore estrafegar\textunderscore )}
\end{itemize}
Luta, braço a braço.
\section{Estrafegar}
\begin{itemize}
\item {Grp. gram.:v. t.}
\end{itemize}
Fazer em pedaços.
(Por \textunderscore trasfegar\textunderscore )
\section{Estraga-albardas}
\begin{itemize}
\item {Grp. gram.:m.}
\end{itemize}
\begin{itemize}
\item {Utilização:Pop.}
\end{itemize}
Homem estouvado, extravagante, dissipador.
\section{Estragação}
\begin{itemize}
\item {Grp. gram.:f.}
\end{itemize}
\begin{itemize}
\item {Utilização:Pop.}
\end{itemize}
Acto ou effeito de estragar.
\section{Estragadamente}
\begin{itemize}
\item {Grp. gram.:adv.}
\end{itemize}
\begin{itemize}
\item {Utilização:Fig.}
\end{itemize}
\begin{itemize}
\item {Proveniência:(De \textunderscore estragar\textunderscore )}
\end{itemize}
Perdulariamente.
Dissolutamente.
\section{Estragadão}
\begin{itemize}
\item {Grp. gram.:m.}
\end{itemize}
\begin{itemize}
\item {Utilização:Fam.}
\end{itemize}
Indivíduo esbanjador.
Aquelle que descura e maltrata o vestuário.
\section{Estragador}
\begin{itemize}
\item {Grp. gram.:adj.}
\end{itemize}
\begin{itemize}
\item {Grp. gram.:M.}
\end{itemize}
Que estraga.
Aquelle que estraga.
\section{Estragamento}
\begin{itemize}
\item {Grp. gram.:m.}
\end{itemize}
O mesmo que \textunderscore estrago\textunderscore .
\section{Estragão}
\begin{itemize}
\item {Grp. gram.:m.}
\end{itemize}
\begin{itemize}
\item {Proveniência:(Fr. \textunderscore estragon\textunderscore )}
\end{itemize}
Planta aromática, da fam. das compostas, (\textunderscore artemisia dracunculus\textunderscore , Lin.).
\section{Estragar}
\begin{itemize}
\item {Grp. gram.:v. t.}
\end{itemize}
\begin{itemize}
\item {Utilização:Fig.}
\end{itemize}
Fazer estrago em.
Deteriorar.
Desperdiçar, dissipar.
Tornar vicioso: \textunderscore as más companhias estragaram-no\textunderscore .
\section{Estrago}
\begin{itemize}
\item {Grp. gram.:m.}
\end{itemize}
\begin{itemize}
\item {Utilização:Fig.}
\end{itemize}
\begin{itemize}
\item {Proveniência:(Lat. hyp. \textunderscore stragus\textunderscore ; por \textunderscore strages\textunderscore )}
\end{itemize}
Deterioração.
Prejuízo.
Ruína.
Dissipação.
Definhamento, enfraquecimento phýsico: \textunderscore os estragos de uma doença\textunderscore .
\section{Estragoso}
\begin{itemize}
\item {Grp. gram.:adj.}
\end{itemize}
(V.estragador)
\section{Estrágulo}
\begin{itemize}
\item {Grp. gram.:m.}
\end{itemize}
\begin{itemize}
\item {Proveniência:(Lat. \textunderscore stragulum\textunderscore )}
\end{itemize}
Qualquer tapeçaria, reposteiro, alcatifa, tapête, colcha, etc.
\section{Estralada}
\begin{itemize}
\item {Grp. gram.:f.}
\end{itemize}
\begin{itemize}
\item {Utilização:Pop.}
\end{itemize}
Grande ruído, grande bulha.
Acto de estralar.
\section{Estralar}
\textunderscore v. t.\textunderscore  (e der.)
O mesmo que \textunderscore estalar\textunderscore , etc.
\section{Estralejar}
\begin{itemize}
\item {Grp. gram.:v. i.}
\end{itemize}
\begin{itemize}
\item {Grp. gram.:V. t.}
\end{itemize}
Dar muitos estalos.
Fazer bater com ruído. Cf. Camillo, \textunderscore Brasileira\textunderscore , 339.
(Por \textunderscore estalejar\textunderscore )
\section{Estralheira}
\begin{itemize}
\item {Grp. gram.:f.}
\end{itemize}
\begin{itemize}
\item {Proveniência:(Do rad. do it. \textunderscore straglio\textunderscore )}
\end{itemize}
Apparelho de roldanas, para suspender a bordo lanchas, âncoras, etc.
\section{Estralho}
\begin{itemize}
\item {Grp. gram.:m.}
\end{itemize}
\begin{itemize}
\item {Utilização:Prov.}
\end{itemize}
\begin{itemize}
\item {Utilização:alg.}
\end{itemize}
Fio de torçal, de linho, ou de cânhamo, de que usam os pescadores.
(Cp. it. \textunderscore straglio\textunderscore )
\section{Estraloiço}
\begin{itemize}
\item {Grp. gram.:m.}
\end{itemize}
\begin{itemize}
\item {Utilização:Açor}
\end{itemize}
Barulho forte e súbito.
\section{Estramalhar-se}
\begin{itemize}
\item {Grp. gram.:v. p.}
\end{itemize}
\begin{itemize}
\item {Utilização:Pop.}
\end{itemize}
O mesmo que [[tresmalhar-se|tresmalhar]], (por metáth.).
\section{Estra-mangueira}
\begin{itemize}
\item {Grp. gram.:f.}
\end{itemize}
Árvore de Timor.
\section{Estramazão}
\begin{itemize}
\item {Grp. gram.:m.}
\end{itemize}
\begin{itemize}
\item {Proveniência:(It. \textunderscore stramazzone\textunderscore )}
\end{itemize}
Antiga e larga espada de dois gumes.
\section{Estrambalhar}
\begin{itemize}
\item {Grp. gram.:v. t.}
\end{itemize}
\begin{itemize}
\item {Utilização:Prov.}
\end{itemize}
\begin{itemize}
\item {Utilização:beir.}
\end{itemize}
Descompor.
Desorganizar.
Esfarrapar.
(Por \textunderscore estrambelhar\textunderscore , de \textunderscore trambelho\textunderscore )
\section{Estrambelho}
\begin{itemize}
\item {fónica:bê}
\end{itemize}
\begin{itemize}
\item {Grp. gram.:m.}
\end{itemize}
Costalinho na meada. Cf. Filinto, VIII, 49.
\section{Estrambólico}
\begin{itemize}
\item {Grp. gram.:adj.}
\end{itemize}
\begin{itemize}
\item {Utilização:Fam.}
\end{itemize}
O mesmo que \textunderscore estrambótico\textunderscore .
\section{Estrambote}
\begin{itemize}
\item {Grp. gram.:m.}
\end{itemize}
\begin{itemize}
\item {Proveniência:(It. \textunderscore strambotto\textunderscore )}
\end{itemize}
Accrescentamento, ordinariamente de 3 versos, aos 14 do soneto.
\section{Estrambotice}
\begin{itemize}
\item {Grp. gram.:f.}
\end{itemize}
Qualidade de estrambótico. Cf. F. Recreio, \textunderscore Bat. de Ourique\textunderscore .
\section{Estrambótico}
\begin{itemize}
\item {Grp. gram.:adj.}
\end{itemize}
\begin{itemize}
\item {Utilização:Pop.}
\end{itemize}
\begin{itemize}
\item {Proveniência:(De \textunderscore estramboto\textunderscore )}
\end{itemize}
Esquisito; extravagante; ridículo.
\section{Estramboto}
\begin{itemize}
\item {fónica:bô}
\end{itemize}
\begin{itemize}
\item {Grp. gram.:m.}
\end{itemize}
\begin{itemize}
\item {Grp. gram.:Adj.}
\end{itemize}
\begin{itemize}
\item {Proveniência:(It. \textunderscore strambotto\textunderscore )}
\end{itemize}
Estrambote.
Antiga composição poética, amatória, entre os italianos.
Estrambótico.
\section{Estrame}
\begin{itemize}
\item {Grp. gram.:m.}
\end{itemize}
\begin{itemize}
\item {Utilização:Des.}
\end{itemize}
\begin{itemize}
\item {Proveniência:(Lat. \textunderscore stramen\textunderscore )}
\end{itemize}
Palha.
Feno.
Cama de palha ou de colmo.
\section{Estramento}
\begin{itemize}
\item {Grp. gram.:m.}
\end{itemize}
\begin{itemize}
\item {Utilização:Ant.}
\end{itemize}
\begin{itemize}
\item {Proveniência:(Lat. \textunderscore stramentum\textunderscore )}
\end{itemize}
Estrame.
Tapeçaria.
Cama para o gado, nos curraes.
Tudo que pertence a uma cama.
\section{Estramonina}
\begin{itemize}
\item {Grp. gram.:f.}
\end{itemize}
Substância, que se extrái do estramónio.
\section{Estramónio}
\begin{itemize}
\item {Grp. gram.:m.}
\end{itemize}
Planta solânea, annual, conhecida vulgarmente por \textunderscore figueira do inferno\textunderscore , (\textunderscore datura stramonium\textunderscore ).
\section{Estramontado}
\begin{itemize}
\item {Grp. gram.:adj.}
\end{itemize}
Que perdeu a tramontana.
Encolerizado; desorientado.
\section{Estrampalhar}
\begin{itemize}
\item {Grp. gram.:v. t.}
\end{itemize}
\begin{itemize}
\item {Utilização:Prov.}
\end{itemize}
\begin{itemize}
\item {Utilização:trasm.}
\end{itemize}
\begin{itemize}
\item {Utilização:minh.}
\end{itemize}
O mesmo que \textunderscore estrambalhar\textunderscore .
\section{Estrampalho}
\begin{itemize}
\item {Grp. gram.:m.}
\end{itemize}
\begin{itemize}
\item {Utilização:Prov.}
\end{itemize}
\begin{itemize}
\item {Utilização:trasm.}
\end{itemize}
O mesmo que \textunderscore espantalho\textunderscore .
(Cp. \textunderscore estrampalhar\textunderscore )
\section{Estramunhar}
\begin{itemize}
\item {Grp. gram.:v. t.}
\end{itemize}
O mesmo que \textunderscore estremunhar\textunderscore . Cf. Filinto, IV, 221; XVIII, 224.
\section{Estrançar}
\begin{itemize}
\item {Grp. gram.:v. t.}
\end{itemize}
\begin{itemize}
\item {Utilização:Prov.}
\end{itemize}
\begin{itemize}
\item {Utilização:minh.}
\end{itemize}
O mesmo que \textunderscore estraçoar\textunderscore .
(Cp. \textunderscore estrancinhar\textunderscore )
\section{Estrancilhar}
\begin{itemize}
\item {Grp. gram.:v. t.}
\end{itemize}
\begin{itemize}
\item {Utilização:Pop.}
\end{itemize}
O mesmo que \textunderscore estrancinhar\textunderscore .
\section{Estrancinhar}
\begin{itemize}
\item {Grp. gram.:v. t.}
\end{itemize}
\begin{itemize}
\item {Utilização:Prov.}
\end{itemize}
\begin{itemize}
\item {Utilização:minh.}
\end{itemize}
\begin{itemize}
\item {Utilização:Ant.}
\end{itemize}
Despedaçar com fúria.
Esforçar-se por despedaçar:«\textunderscore estrancinhada cume morta.\textunderscore »\textunderscore Cancion. da Vaticana\textunderscore .
(Cp. \textunderscore estracinhar\textunderscore )
\section{Estranfeniar}
\begin{itemize}
\item {Grp. gram.:v. t.}
\end{itemize}
\begin{itemize}
\item {Utilização:Prov.}
\end{itemize}
\begin{itemize}
\item {Utilização:trasm.}
\end{itemize}
O mesmo que \textunderscore estranfoliar\textunderscore .
\section{Estranfoliar}
\begin{itemize}
\item {Grp. gram.:v. t.}
\end{itemize}
\begin{itemize}
\item {Utilização:Prov.}
\end{itemize}
\begin{itemize}
\item {Utilização:beir.}
\end{itemize}
Dissipar rapidamente, esbanjar.
Dar cabo de.
(Relaciona-se com \textunderscore foliar\textunderscore )
\section{Estranfolinhar}
\begin{itemize}
\item {Grp. gram.:v. t.}
\end{itemize}
\begin{itemize}
\item {Utilização:Prov.}
\end{itemize}
\begin{itemize}
\item {Utilização:dur.}
\end{itemize}
O mesmo que \textunderscore estranfoliar\textunderscore .
\section{Estrangedura}
\begin{itemize}
\item {Grp. gram.:f.}
\end{itemize}
Acto de estranger.
\section{Estrangeirada}
\begin{itemize}
\item {Grp. gram.:f.}
\end{itemize}
\begin{itemize}
\item {Utilização:Fam.}
\end{itemize}
Chusma de estrangeiros.
\section{Estrangeirado}
\begin{itemize}
\item {Grp. gram.:adj.}
\end{itemize}
\begin{itemize}
\item {Proveniência:(De \textunderscore estrangeiro\textunderscore )}
\end{itemize}
Que imita ou faz lembrar coisa estrangeira.
Que tem modos ou fala de estrangeiro.
Affeiçoado a estrangeiros.
\section{Estrangeiramento}
\begin{itemize}
\item {Grp. gram.:m.}
\end{itemize}
Acto ou effeito de estrangeirar. Cf. Castilho, \textunderscore Fastos\textunderscore , I. 611.
\section{Estrangeirar}
\begin{itemize}
\item {Grp. gram.:v. t.}
\end{itemize}
Dar aspecto ou fórma estrangeira a. Cf. Filinto, XXIII, 160.
\section{Estrangeirice}
\begin{itemize}
\item {Grp. gram.:f.}
\end{itemize}
\begin{itemize}
\item {Proveniência:(De \textunderscore estrangeiro\textunderscore )}
\end{itemize}
Coisa feita, ou dita, ao gôsto ou costume de estrangeiros.
Affeição demasiada ás coisas estrangeiras.
\section{Estrangeirinha}
\begin{itemize}
\item {Grp. gram.:f.}
\end{itemize}
\begin{itemize}
\item {Utilização:Fam.}
\end{itemize}
\begin{itemize}
\item {Proveniência:(De \textunderscore estrangeiro\textunderscore )}
\end{itemize}
Ardil; tranquibérnia.
\section{Estrangeirismo}
\begin{itemize}
\item {Grp. gram.:m.}
\end{itemize}
\begin{itemize}
\item {Proveniência:(De \textunderscore estrangeiro\textunderscore )}
\end{itemize}
Emprêgo de palavra ou phrase estrangeira.
Palavra ou phrase estrangeira.
Estrangeirice.
\section{Estrangeirista}
\begin{itemize}
\item {Grp. gram.:m.}
\end{itemize}
Aquelle que usa estrangeirismos.
Amigo de estrangeirismos.
(Cp. \textunderscore estrangeirismo\textunderscore )
\section{Estrangeiro}
\begin{itemize}
\item {Grp. gram.:adj.}
\end{itemize}
\begin{itemize}
\item {Grp. gram.:M.}
\end{itemize}
\begin{itemize}
\item {Utilização:Gal}
\end{itemize}
\begin{itemize}
\item {Utilização:Ant.}
\end{itemize}
\begin{itemize}
\item {Proveniência:(Do fr. \textunderscore étranger\textunderscore )}
\end{itemize}
Que não é do país em que está.
Pessôa estrangeira.
Aquelle que é natural de pais estranho a quem se lhe refere, falando ou escrevendo: \textunderscore os estrangeiros não nos conhecem bem\textunderscore .
Nações estrangeiras.
Indivíduo que, não pertencendo a outra nação, é de terra diversa daquella em que está. Cf. \textunderscore Eufrosina\textunderscore , V, 2.
\section{Estranger}
\begin{itemize}
\item {Grp. gram.:v.}
\end{itemize}
\begin{itemize}
\item {Utilização:t. Marn.}
\end{itemize}
Tirar das marinhas (lama e algas putrefactas).
(Cp. \textunderscore estransir\textunderscore  e \textunderscore estresir\textunderscore )
\section{Estrangulação}
\begin{itemize}
\item {Grp. gram.:f.}
\end{itemize}
\begin{itemize}
\item {Utilização:Med.}
\end{itemize}
\begin{itemize}
\item {Proveniência:(Lat. \textunderscore strangulatio\textunderscore )}
\end{itemize}
Acto ou effeito de estrangular.
Constricção, esmagamento: \textunderscore a estrangulação de uma hérnia\textunderscore .
\section{Estrangulador}
\begin{itemize}
\item {Grp. gram.:adj.}
\end{itemize}
\begin{itemize}
\item {Grp. gram.:M.}
\end{itemize}
Que estrangula.
Aquelle que estrangula.
\section{Estrangulamento}
\begin{itemize}
\item {Grp. gram.:m.}
\end{itemize}
O mesmo que \textunderscore estrangulação\textunderscore .
\section{Estrangular}
\begin{itemize}
\item {Grp. gram.:v. t.}
\end{itemize}
\begin{itemize}
\item {Utilização:Ext.}
\end{itemize}
\begin{itemize}
\item {Proveniência:(Lat. \textunderscore strangulare\textunderscore )}
\end{itemize}
Interromper a respiração a, apertando-lhe o pescoço; suffocar; matar, apertando o pescoço a.
Esganar.
Apertar muito, comprimir.
Esmagar.
Impedir, abafar: \textunderscore estrangular a voz\textunderscore .
\section{Estranguria}
\begin{itemize}
\item {Grp. gram.:f.}
\end{itemize}
\begin{itemize}
\item {Proveniência:(Do gr. \textunderscore strangouria\textunderscore )}
\end{itemize}
Difficuldade de urinar, saíndo a urina ás gotas.
Apêrto de uretra.
\section{Estranhamente}
\begin{itemize}
\item {Grp. gram.:adv.}
\end{itemize}
De modo estranho.
\section{Estranhamento}
\begin{itemize}
\item {Grp. gram.:m.}
\end{itemize}
Acto de estranhar.
Admiração.
\section{Estranhão}
\begin{itemize}
\item {Grp. gram.:m.  e  adj.}
\end{itemize}
\begin{itemize}
\item {Utilização:Fam.}
\end{itemize}
\begin{itemize}
\item {Proveniência:(De \textunderscore estranho\textunderscore )}
\end{itemize}
Indivíduo esquivo, acanhado, bisonho.
\section{Estranhar}
\begin{itemize}
\item {Grp. gram.:v. t.}
\end{itemize}
\begin{itemize}
\item {Proveniência:(Do b. lat. \textunderscore straniare\textunderscore )}
\end{itemize}
Julgar estranho, opposto aos costumes, ao hábito.
Notar, censurar.
Admirar.
Achar novidade em.
Causar admiração a.
\section{Estranhável}
\begin{itemize}
\item {Grp. gram.:adj.}
\end{itemize}
Que se póde estranhar.
Censurável.
\section{Estranheiro}
\begin{itemize}
\item {Grp. gram.:m.  e  adj.}
\end{itemize}
\begin{itemize}
\item {Utilização:Ant.}
\end{itemize}
\begin{itemize}
\item {Proveniência:(De \textunderscore estranho\textunderscore )}
\end{itemize}
O mesmo que \textunderscore estrangeiro\textunderscore .
\section{Estranhez}
\begin{itemize}
\item {Grp. gram.:f.}
\end{itemize}
Qualidade daquillo que é estranho.
Admiração, impressão, produzida pelo que é estranho.
Esquivança.
\section{Estranheza}
\begin{itemize}
\item {Grp. gram.:f.}
\end{itemize}
Qualidade daquillo que é estranho.
Admiração, impressão, produzida pelo que é estranho.
Esquivança.
\section{Estranho}
\begin{itemize}
\item {Grp. gram.:adj.}
\end{itemize}
\begin{itemize}
\item {Proveniência:(Lat. \textunderscore extraneus\textunderscore )}
\end{itemize}
Externo; estrangeiro.
Que é de fora.
Descommunal.
Admirável.
Esquisito.
Mysterioso.
Impróprio.
Alheio.
Esquivo.
\section{Estranja}
\begin{itemize}
\item {Grp. gram.:f.}
\end{itemize}
\begin{itemize}
\item {Utilização:Chul.}
\end{itemize}
\begin{itemize}
\item {Proveniência:(Do rad. de \textunderscore estrangeiro\textunderscore )}
\end{itemize}
Países estrangeiros.
\section{Estransilhado}
\begin{itemize}
\item {Grp. gram.:adj.}
\end{itemize}
\begin{itemize}
\item {Utilização:Prov.}
\end{itemize}
\begin{itemize}
\item {Utilização:trasm.}
\end{itemize}
Muito acanaveado, muito magro.
(Relaciona-se com \textunderscore transir\textunderscore )
\section{Estransir}
\textunderscore v. t.\textunderscore  (e der.) \textunderscore Ant.\textunderscore 
O mesmo que \textunderscore transir\textunderscore , etc. Cp. Simão Machado, 82, v.^o
\section{Estranvésia}
\begin{itemize}
\item {Grp. gram.:f.}
\end{itemize}
Género de plantas pomáceas.
\section{Estrapada}
\begin{itemize}
\item {Grp. gram.:f.}
\end{itemize}
Antigo supplício, applicado a militares delinquentes, deslocando-se-lhes os braços por suspensão.
(Cast. \textunderscore estrapada\textunderscore )
\section{Estrapagado}
\begin{itemize}
\item {Grp. gram.:adj.}
\end{itemize}
\begin{itemize}
\item {Utilização:Mad}
\end{itemize}
Ave, o mesmo que \textunderscore patagarro\textunderscore .
\section{Estrapor}
\begin{itemize}
\item {Grp. gram.:v. t.}
\end{itemize}
\begin{itemize}
\item {Utilização:Prov.}
\end{itemize}
\begin{itemize}
\item {Utilização:alg.}
\end{itemize}
O mesmo que \textunderscore transpor\textunderscore .
(Metáth. e próthese de \textunderscore transpor\textunderscore )
\section{Estrar}
\begin{itemize}
\item {Grp. gram.:v. t.}
\end{itemize}
Estender, alastrar (palha ou mato) nos curraes de gado vacum, sôbre estrume já calcado.
(Cp. \textunderscore estrame\textunderscore )
\section{Estrasburguês}
\begin{itemize}
\item {Grp. gram.:adj.}
\end{itemize}
\begin{itemize}
\item {Grp. gram.:M.}
\end{itemize}
Relativo a Estrasburgo.
Aquelle que é natural do Estrasburgo.
\section{Estratagema}
\begin{itemize}
\item {Grp. gram.:m.}
\end{itemize}
\begin{itemize}
\item {Utilização:Ext.}
\end{itemize}
\begin{itemize}
\item {Proveniência:(Lat. \textunderscore strategema\textunderscore )}
\end{itemize}
Astúcia ou ardil, empregado por tropas contra inimigos.
Ardil, manha.
\section{Estratagemático}
\begin{itemize}
\item {Grp. gram.:adj.}
\end{itemize}
Em que há estratagema.
\section{Estratégia}
\begin{itemize}
\item {Grp. gram.:f.}
\end{itemize}
\begin{itemize}
\item {Proveniência:(Gr. \textunderscore strategia\textunderscore )}
\end{itemize}
Sciência das operações militares.
Estratagema.
Ardil, manha.
\section{Estrategicamente}
\begin{itemize}
\item {Grp. gram.:adv.}
\end{itemize}
\begin{itemize}
\item {Utilização:Ext.}
\end{itemize}
\begin{itemize}
\item {Proveniência:(De \textunderscore estratégico\textunderscore )}
\end{itemize}
Segundo os preceitos da estratégia.
Ardilosamente.
\section{Estratégico}
\begin{itemize}
\item {Grp. gram.:adj.}
\end{itemize}
\begin{itemize}
\item {Utilização:Ext.}
\end{itemize}
Relativo á estratégia.
Em que há ardil; que usa de ardil.
\section{Estrategista}
\begin{itemize}
\item {Grp. gram.:m.}
\end{itemize}
Aquelle que é versado em estratégia.
\section{Estratego}
\begin{itemize}
\item {Grp. gram.:m.}
\end{itemize}
\begin{itemize}
\item {Proveniência:(Lat. \textunderscore strategus\textunderscore )}
\end{itemize}
General superior ou generalíssimo, entre os antigos Gregos.
\section{Estratificação}
\begin{itemize}
\item {Grp. gram.:f.}
\end{itemize}
Acto ou effeito de estratificar.
\section{Estratificadamente}
\begin{itemize}
\item {Grp. gram.:adv.}
\end{itemize}
\begin{itemize}
\item {Proveniência:(De \textunderscore estratificar\textunderscore )}
\end{itemize}
Por camadas successivas.
\section{Estratificar}
\begin{itemize}
\item {Grp. gram.:v. t.}
\end{itemize}
\begin{itemize}
\item {Proveniência:(Do lat. \textunderscore stratus\textunderscore  + \textunderscore facere\textunderscore )}
\end{itemize}
Dispor em estratos, em camadas; acamar.
\section{Estratiforme}
\begin{itemize}
\item {Grp. gram.:adj.}
\end{itemize}
\begin{itemize}
\item {Proveniência:(Do lat. \textunderscore stratus\textunderscore  + \textunderscore forma\textunderscore )}
\end{itemize}
Disposto em camadas successivas e parallelas.
\section{Estratigrafia}
\begin{itemize}
\item {Grp. gram.:f.}
\end{itemize}
\begin{itemize}
\item {Proveniência:(De \textunderscore estratígrafo\textunderscore )}
\end{itemize}
Parte da Geologia, que trata da formação e disposição dos terrenos estratificados.
\section{Estratigráfico}
\begin{itemize}
\item {Grp. gram.:adj.}
\end{itemize}
Relativo a estratigrafia.
\section{Estratígrafo}
\begin{itemize}
\item {Grp. gram.:m.}
\end{itemize}
\begin{itemize}
\item {Proveniência:(Do lat. \textunderscore stratus\textunderscore  + gr. \textunderscore graphein\textunderscore )}
\end{itemize}
Aquele que é versado em estratigrafia.
\section{Estratigraphia}
\begin{itemize}
\item {Grp. gram.:f.}
\end{itemize}
\begin{itemize}
\item {Proveniência:(De \textunderscore estratígrapho\textunderscore )}
\end{itemize}
Parte da Geologia, que trata da formação e disposição dos terrenos estratificados.
\section{Estratigráphico}
\begin{itemize}
\item {Grp. gram.:adj.}
\end{itemize}
Relativo a estratigraphia.
\section{Estratígrapho}
\begin{itemize}
\item {Grp. gram.:m.}
\end{itemize}
\begin{itemize}
\item {Proveniência:(Do lat. \textunderscore stratus\textunderscore  + gr. \textunderscore graphein\textunderscore )}
\end{itemize}
Aquelle que é versado em estratigraphia.
\section{Estratiomia}
\begin{itemize}
\item {Grp. gram.:f.}
\end{itemize}
\begin{itemize}
\item {Proveniência:(Do gr. \textunderscore stratos\textunderscore  + \textunderscore muia\textunderscore )}
\end{itemize}
Gênero de insectos dípteros.
\section{Estratiomya}
\begin{itemize}
\item {Grp. gram.:f.}
\end{itemize}
\begin{itemize}
\item {Proveniência:(Do gr. \textunderscore stratos\textunderscore  + \textunderscore muia\textunderscore )}
\end{itemize}
Gênero de insectos dípteros.
\section{Estrato}
\begin{itemize}
\item {Grp. gram.:m.}
\end{itemize}
\begin{itemize}
\item {Proveniência:(Lat. \textunderscore stratus\textunderscore )}
\end{itemize}
Cada uma das camadas dos terrenos sedimentares.
Camada.
Nuvens, que formam faixas largas e horizontaes.
\section{Estratocracia}
\begin{itemize}
\item {Grp. gram.:f.}
\end{itemize}
\begin{itemize}
\item {Proveniência:(Do gr. \textunderscore stratos\textunderscore  + \textunderscore kratein\textunderscore )}
\end{itemize}
Govêrno militar.
\section{Estratografia}
\begin{itemize}
\item {Grp. gram.:f.}
\end{itemize}
\begin{itemize}
\item {Utilização:Des.}
\end{itemize}
\begin{itemize}
\item {Proveniência:(Do gr. \textunderscore stratos\textunderscore  + \textunderscore graphein\textunderscore )}
\end{itemize}
Descripção de um exército e do que lhe pertence.
\section{Estratographia}
\begin{itemize}
\item {Grp. gram.:f.}
\end{itemize}
\begin{itemize}
\item {Utilização:Des.}
\end{itemize}
\begin{itemize}
\item {Proveniência:(Do gr. \textunderscore stratos\textunderscore  + \textunderscore graphein\textunderscore )}
\end{itemize}
Descripção de um exército e do que lhe pertence.
\section{Estravar}
\textunderscore v. i.\textunderscore  (e der.)
O mesmo ou melhor que \textunderscore estrabar\textunderscore .
\section{Estravessa}
\begin{itemize}
\item {Grp. gram.:f.}
\end{itemize}
\begin{itemize}
\item {Utilização:Prov.}
\end{itemize}
\begin{itemize}
\item {Utilização:beir.}
\end{itemize}
Segunda lavra, que em Junho se dá ordinariamente ás terras que se hão de semear, e que corta ou atravessa a primeira lavra. (Colhido no Fundão)
(Cp. \textunderscore atravessar\textunderscore )
\section{Estreado}
\begin{itemize}
\item {Grp. gram.:adj.}
\end{itemize}
\begin{itemize}
\item {Utilização:Des.}
\end{itemize}
Dizia-se do indivíduo, que tem modos engraçados, que é jovial, agradável.
\section{Estrear}
\begin{itemize}
\item {Grp. gram.:v. t.}
\end{itemize}
\begin{itemize}
\item {Grp. gram.:V. p.}
\end{itemize}
\begin{itemize}
\item {Proveniência:(De \textunderscore estreia\textunderscore )}
\end{itemize}
Empregar pela primeira vez: \textunderscore estrear um casaco\textunderscore .
Iniciar, começar.
Fazer alguma coisa pela primeira vez.
\section{Estrebangar}
\begin{itemize}
\item {Grp. gram.:v. t.}
\end{itemize}
\begin{itemize}
\item {Utilização:Prov.}
\end{itemize}
\begin{itemize}
\item {Utilização:dur.}
\end{itemize}
Torcer (o pé), desnocar.
\section{Estrebaria}
\begin{itemize}
\item {Grp. gram.:f.}
\end{itemize}
\begin{itemize}
\item {Utilização:T. da Bairrada}
\end{itemize}
\begin{itemize}
\item {Proveniência:(Do lat. \textunderscore stabularía\textunderscore ? Ou por \textunderscore estribaria\textunderscore , de \textunderscore estribo\textunderscore ?)}
\end{itemize}
Curral; cavallariça.
Lugar, em que se recolhem bêstas, estribos e outros arreios.
Dito ou acto, próprio de arreeiro.
Gallegada.
\section{Estrebuchamento}
\begin{itemize}
\item {Grp. gram.:m.}
\end{itemize}
Acto de estrebuchar.
\section{Estrebuchar}
\begin{itemize}
\item {Grp. gram.:v. i.}
\end{itemize}
\begin{itemize}
\item {Proveniência:(Do fr. \textunderscore trèbucher\textunderscore )}
\end{itemize}
Agitar os braços, as pernas e a cabeça, convulsivamente.
Mexer-se muito.
\section{Estrecer}
\begin{itemize}
\item {Grp. gram.:v. t.}
\end{itemize}
\begin{itemize}
\item {Utilização:Ant.}
\end{itemize}
Estreitar, deminuir.
Esfriar.
(Cp. cast. \textunderscore estrecho\textunderscore )
\section{Estrefega}
\begin{itemize}
\item {Grp. gram.:f.}
\end{itemize}
\begin{itemize}
\item {Utilização:Prov.}
\end{itemize}
\begin{itemize}
\item {Utilização:trasm.}
\end{itemize}
Acto de estrefegar.
\section{Estrefegar}
\begin{itemize}
\item {Grp. gram.:v. t.}
\end{itemize}
\begin{itemize}
\item {Utilização:Prov.}
\end{itemize}
\begin{itemize}
\item {Utilização:trasm.}
\end{itemize}
Escorraçar (uma cavalgadura), até a esfalfar.
(Corr. de \textunderscore estrafegar\textunderscore )
\section{Estrefogueiro}
\begin{itemize}
\item {Grp. gram.:m.}
\end{itemize}
\begin{itemize}
\item {Utilização:Prov.}
\end{itemize}
\begin{itemize}
\item {Utilização:trasm.}
\end{itemize}
O mesmo que \textunderscore morilho\textunderscore .
(Corr. de \textunderscore trasfogueiro\textunderscore )
\section{Estrefura}
\begin{itemize}
\item {Grp. gram.:m.  e  f.}
\end{itemize}
\begin{itemize}
\item {Utilização:Prov.}
\end{itemize}
\begin{itemize}
\item {Utilização:trasm.}
\end{itemize}
Pessôa velhaca, cujo riso é todo manha e falsidade.
\section{Estregar}
\begin{itemize}
\item {Grp. gram.:v. t.}
\end{itemize}
Transferir para um papel, tábua, etc., com uma boneca embebida em pó de carvão, (um desenho picado).
Almofaçar.
Coçar. Cf. \textunderscore Lusíadas\textunderscore , VI, 39.
(Cp. cast. \textunderscore estregar\textunderscore , e cp. G. Viana, \textunderscore Apostilas\textunderscore )
\section{Estreia}
\begin{itemize}
\item {Grp. gram.:f.}
\end{itemize}
\begin{itemize}
\item {Utilização:Ant.}
\end{itemize}
\begin{itemize}
\item {Proveniência:(Do lat. \textunderscore strena\textunderscore )}
\end{itemize}
Acto ou effeito de estrear.
Brinde, dádiva, no primeiro dia do anno.
\section{Estreita}
\begin{itemize}
\item {Grp. gram.:f.}
\end{itemize}
\begin{itemize}
\item {Utilização:Ant.}
\end{itemize}
\begin{itemize}
\item {Proveniência:(De \textunderscore estreito\textunderscore )}
\end{itemize}
Vida miserável; circunstâncias apertadas.
\section{Estreitador}
\begin{itemize}
\item {Grp. gram.:adj.}
\end{itemize}
\begin{itemize}
\item {Grp. gram.:m.}
\end{itemize}
Que estreita.
Aquelle que estreita.
\section{Estreitamente}
\begin{itemize}
\item {Grp. gram.:adv.}
\end{itemize}
\begin{itemize}
\item {Proveniência:(De \textunderscore estreito\textunderscore )}
\end{itemize}
Com estreiteza.
Em pequeno espaço.
Escassamente.
Intimamente.
Rigorosamente; restrictamente.
\section{Estreitamento}
\begin{itemize}
\item {Grp. gram.:m.}
\end{itemize}
Acto ou effeito de estreitar.
\section{Estreitar}
\begin{itemize}
\item {Grp. gram.:v. t.}
\end{itemize}
\begin{itemize}
\item {Grp. gram.:V. i.  e  p.}
\end{itemize}
\begin{itemize}
\item {Proveniência:(Do b. lat. \textunderscore strictare\textunderscore )}
\end{itemize}
Tornar estreito.
Apertar.
Abraçar.
Unir, tornar justo.
Contratar.
Restringir.
Tornar rigoroso.
Tornar-se estreito.
\section{Estreiteza}
\begin{itemize}
\item {Grp. gram.:f.}
\end{itemize}
Qualidade daquillo que é estreito.
\section{Estreito}
\begin{itemize}
\item {Grp. gram.:adj.}
\end{itemize}
\begin{itemize}
\item {Utilização:Fig.}
\end{itemize}
\begin{itemize}
\item {Grp. gram.:M.}
\end{itemize}
\begin{itemize}
\item {Proveniência:(Do lat. \textunderscore strictos\textunderscore )}
\end{itemize}
Apertado, comprimido.
Que tem pouca largura: \textunderscore canal estreito\textunderscore .
Unido, justo.
Limitado.
Acanhado, escasso.
Parco, Rigoroso.
Intimo.
Difficultoso.
Miserável.
Ligação estreita de dois mares ou de duas partes de mar.
Desfiladeiro.
Casta de uva.
Conjuntura perigosa.
\section{Estreitoeiras}
\begin{itemize}
\item {Grp. gram.:f. pl.}
\end{itemize}
\begin{itemize}
\item {Utilização:Prov.}
\end{itemize}
\begin{itemize}
\item {Utilização:trasm.}
\end{itemize}
O mesmo que \textunderscore entriteiras\textunderscore .
\section{Estreitura}
\begin{itemize}
\item {Grp. gram.:f.}
\end{itemize}
\begin{itemize}
\item {Grp. gram.:Pl.}
\end{itemize}
\begin{itemize}
\item {Utilização:Prov.}
\end{itemize}
\begin{itemize}
\item {Utilização:beir.}
\end{itemize}
O mesmo que \textunderscore estreiteza\textunderscore .
O mesmo que \textunderscore entriteiras\textunderscore .
\section{Estrêla}
\begin{itemize}
\item {Grp. gram.:f.}
\end{itemize}
\begin{itemize}
\item {Utilização:Ext.}
\end{itemize}
\begin{itemize}
\item {Utilização:Fig.}
\end{itemize}
\begin{itemize}
\item {Proveniência:(Do lat. \textunderscore stella\textunderscore )}
\end{itemize}
Cada um dos astros que têm luz própria e não têm movimento sensível.
Astro.
Destino.
Guia.
Pessôa, que se distingue muito, especialmente nos teatros líricos.
Qualquer objecto que tem a aparência de uma estrêla.
Asterisco.
Bonina.
Planta sapotácea (\textunderscore chrysophilum Roxburgii\textunderscore , G. D.).
\section{Estreladeira}
\begin{itemize}
\item {Grp. gram.:f.}
\end{itemize}
\begin{itemize}
\item {Proveniência:(De \textunderscore estrelar\textunderscore )}
\end{itemize}
Vaso, em que se estrelam ovos.
\section{Estrelamim}
\begin{itemize}
\item {Grp. gram.:m.}
\end{itemize}
\begin{itemize}
\item {Proveniência:(Do rad. de \textunderscore estrela\textunderscore ?)}
\end{itemize}
Espécie de aristolóquia.
\section{Estrelante}
\begin{itemize}
\item {Grp. gram.:adj.}
\end{itemize}
\begin{itemize}
\item {Proveniência:(De \textunderscore estrelar\textunderscore )}
\end{itemize}
Estrelado.
Cintilante.
\section{Estrelar}
\begin{itemize}
\item {Grp. gram.:v. t.}
\end{itemize}
\begin{itemize}
\item {Utilização:Prov.}
\end{itemize}
\begin{itemize}
\item {Utilização:trasm.}
\end{itemize}
\begin{itemize}
\item {Grp. gram.:V. i.}
\end{itemize}
Encher de estrêlas.
Dar fórma de estrêla a.
Matizar.
Frigir (ovos), sem os bater.
O mesmo que \textunderscore arrestralar\textunderscore . (Colhido em Mogadoiro)
Brilhar.
\section{Estrelário}
\begin{itemize}
\item {Grp. gram.:adj.}
\end{itemize}
\begin{itemize}
\item {Proveniência:(De \textunderscore estrêla\textunderscore )}
\end{itemize}
Que tem fórma de estrêla.
\section{Estrelecer}
\begin{itemize}
\item {Grp. gram.:v. t.}
\end{itemize}
\begin{itemize}
\item {Proveniência:(De \textunderscore estrêla\textunderscore )}
\end{itemize}
Ostentar como estrêla; mostrar (coisa que brilha). Cf. Camillo, \textunderscore Serões\textunderscore , III, 84.
\section{Estreleiro}
\begin{itemize}
\item {Grp. gram.:adj.}
\end{itemize}
\begin{itemize}
\item {Proveniência:(De \textunderscore estrêla\textunderscore )}
\end{itemize}
Diz-se do cavalo, que ergue muito a cabeça.
\section{Estrelejar}
\begin{itemize}
\item {Grp. gram.:v. i.}
\end{itemize}
\begin{itemize}
\item {Grp. gram.:V. t.}
\end{itemize}
Encher-se de estrêlas; brilhar com estrêlas: o céu estreleja.
Cobrir de estrêlas; espalhar como estrêlas:«\textunderscore ...e milhões de vagalumes estrelejando o palmar\textunderscore ». Th. Ribeiro, \textunderscore Jornadas\textunderscore , II, 231.
\section{Estrelico}
\begin{itemize}
\item {Grp. gram.:m.}
\end{itemize}
\begin{itemize}
\item {Utilização:Prov.}
\end{itemize}
\begin{itemize}
\item {Utilização:trasm.}
\end{itemize}
Delíquio, chilique.
Desmaio.
\section{Estrelinha}
\begin{itemize}
\item {Grp. gram.:f.}
\end{itemize}
\begin{itemize}
\item {Proveniência:(De \textunderscore estrêla\textunderscore )}
\end{itemize}
Asterico.
Pássaro dentirostro.
Variedade de massa para sopa.
Espécie de felosa, (\textunderscore regulus ignicapillus\textunderscore , Brehm.).
\section{Estrelítzia}
\begin{itemize}
\item {Grp. gram.:f.}
\end{itemize}
Formosa flôr do Cabo da Bôa-Esperança.
(Talvez da côr do fardamento dos \textunderscore strelitsy\textunderscore , soldados que constituíram antigamente um corpo de infantaria moscovita)
\section{Estrêlla}
\begin{itemize}
\item {Grp. gram.:f.}
\end{itemize}
\begin{itemize}
\item {Utilização:Ext.}
\end{itemize}
\begin{itemize}
\item {Utilização:Fig.}
\end{itemize}
\begin{itemize}
\item {Proveniência:(Do lat. \textunderscore stella\textunderscore )}
\end{itemize}
Cada um dos astros que têm luz própria e não têm movimento sensível.
Astro.
Destino.
Guia.
Pessôa, que se distingue muito, especialmente nos theatros lýricos.
Qualquer objecto que tem a apparência de uma estrêlla.
Asterisco.
Bonina.
Planta sapotácea (\textunderscore chrysophilum Roxburgii\textunderscore , G. D.).
\section{Estrêlla-azul}
\begin{itemize}
\item {Grp. gram.:f.}
\end{itemize}
\begin{itemize}
\item {Utilização:Bras}
\end{itemize}
Planta, o mesmo que \textunderscore scilla\textunderscore .
\section{Estrellada}
\begin{itemize}
\item {Grp. gram.:f.}
\end{itemize}
Planta medicinal, espécie de hepática.
\section{Estrêlla-de-alva}
\begin{itemize}
\item {Grp. gram.:f.}
\end{itemize}
Planta esterculiácea da Índia portuguesa, (\textunderscore sterculia guttata\textunderscore , Roxb.).
\section{Estrelladeira}
\begin{itemize}
\item {Grp. gram.:f.}
\end{itemize}
\begin{itemize}
\item {Proveniência:(De \textunderscore estrellar\textunderscore )}
\end{itemize}
Vaso, em que se estrellam ovos.
\section{Estrêlla-de-jerusalém}
\begin{itemize}
\item {Grp. gram.:f.}
\end{itemize}
\begin{itemize}
\item {Utilização:Bras}
\end{itemize}
Planta, o mesmo que \textunderscore cerástio\textunderscore .
\section{Estrêlla-do-mar}
\begin{itemize}
\item {Grp. gram.:f.}
\end{itemize}
Zoóphyto echinoderme, em fórma de estrêlla com cinco pontos.
\section{Estrellamim}
\begin{itemize}
\item {Grp. gram.:m.}
\end{itemize}
\begin{itemize}
\item {Proveniência:(Do rad. de \textunderscore estrella\textunderscore ?)}
\end{itemize}
Espécie de aristolóchia.
\section{Estrellante}
\begin{itemize}
\item {Grp. gram.:adj.}
\end{itemize}
\begin{itemize}
\item {Proveniência:(De \textunderscore estrellar\textunderscore )}
\end{itemize}
Estrellado.
Scintillante.
\section{Estrellar}
\begin{itemize}
\item {Grp. gram.:v. t.}
\end{itemize}
\begin{itemize}
\item {Utilização:Prov.}
\end{itemize}
\begin{itemize}
\item {Utilização:trasm.}
\end{itemize}
\begin{itemize}
\item {Grp. gram.:V. i.}
\end{itemize}
Encher de estrêllas.
Dar fórma de estrêlla a.
Matizar.
Frigir (ovos), sem os bater.
O mesmo que \textunderscore arrestralar\textunderscore . (Colhido em Mogadoiro)
Brilhar.
\section{Estrellário}
\begin{itemize}
\item {Grp. gram.:adj.}
\end{itemize}
\begin{itemize}
\item {Proveniência:(De \textunderscore estrêlla\textunderscore )}
\end{itemize}
Que tem fórma de estrêlla.
\section{Estrellecer}
\begin{itemize}
\item {Grp. gram.:v. t.}
\end{itemize}
\begin{itemize}
\item {Proveniência:(De \textunderscore estrêlla\textunderscore )}
\end{itemize}
Ostentar como estrêlla; mostrar (coisa que brilha). Cf. Camillo, \textunderscore Serões\textunderscore , III, 84.
\section{Estrelleiro}
\begin{itemize}
\item {Grp. gram.:adj.}
\end{itemize}
\begin{itemize}
\item {Proveniência:(De \textunderscore estrêlla\textunderscore )}
\end{itemize}
Diz-se do cavallo, que ergue muito a cabeça.
\section{Estrellejar}
\begin{itemize}
\item {Grp. gram.:v. i.}
\end{itemize}
\begin{itemize}
\item {Grp. gram.:V. t.}
\end{itemize}
Encher-se de estrêllas; brilhar com estrêllas: o céu estrelleja.
Cobrir de estrêllas; espalhar como estrêllas:«\textunderscore ...e milhões de vagalumes estrellejando o palmar\textunderscore ». Th. Ribeiro, \textunderscore Jornadas\textunderscore , II, 231.
\section{Estrellinha}
\begin{itemize}
\item {Grp. gram.:f.}
\end{itemize}
\begin{itemize}
\item {Proveniência:(De \textunderscore estrêlla\textunderscore )}
\end{itemize}
Asterico.
Pássaro dentirostro.
Variedade de massa para sopa.
Espécie de felosa, (\textunderscore regulus ignicapillus\textunderscore , Brehm.).
\section{Estrêllo}
\begin{itemize}
\item {Grp. gram.:adj.}
\end{itemize}
\begin{itemize}
\item {Utilização:Bras. do N}
\end{itemize}
\begin{itemize}
\item {Proveniência:(De \textunderscore estrella\textunderscore )}
\end{itemize}
Diz-se do boi ou da vaca, que tem uma pinta na testa.
\section{Estrêlo}
\begin{itemize}
\item {Grp. gram.:adj.}
\end{itemize}
\begin{itemize}
\item {Utilização:Bras. do N}
\end{itemize}
\begin{itemize}
\item {Proveniência:(De \textunderscore estrela\textunderscore )}
\end{itemize}
Diz-se do boi ou da vaca, que tem uma pinta na testa.
\section{Estrém}
\begin{itemize}
\item {Grp. gram.:m.}
\end{itemize}
\begin{itemize}
\item {Proveniência:(Do ingl. \textunderscore string\textunderscore )}
\end{itemize}
O mesmo que \textunderscore amarra\textunderscore .
\section{Estrema}
\begin{itemize}
\item {Grp. gram.:f.}
\end{itemize}
\begin{itemize}
\item {Proveniência:(Do lat. \textunderscore extrema\textunderscore )}
\end{itemize}
Limite de terras.
Marco divisório das propriedades rústicas.
Sulco artificial, que demarca terras.
\section{Estremaça}
\begin{itemize}
\item {Grp. gram.:f.}
\end{itemize}
\begin{itemize}
\item {Utilização:Ant.}
\end{itemize}
\begin{itemize}
\item {Proveniência:(De \textunderscore estremar\textunderscore )}
\end{itemize}
O mesmo que \textunderscore estrema\textunderscore .
Divisão, demarcação.
\section{Estremadela}
\begin{itemize}
\item {Grp. gram.:f.}
\end{itemize}
\begin{itemize}
\item {Utilização:Pop.}
\end{itemize}
Acto de estremar.
\section{Estremado}
\begin{itemize}
\item {Grp. gram.:adj.}
\end{itemize}
\begin{itemize}
\item {Utilização:Fig.}
\end{itemize}
Demarcado; dividido.
O mesmo que \textunderscore extremado\textunderscore .
\section{Estremadura}
\begin{itemize}
\item {Grp. gram.:f.}
\end{itemize}
\begin{itemize}
\item {Proveniência:(De \textunderscore estremar\textunderscore )}
\end{itemize}
Estrema de uma província ou de um país.
Raia, fronteira.
\section{Estremalhar}
\begin{itemize}
\item {Grp. gram.:v. t.}
\end{itemize}
(Metáth. pop. de \textunderscore tresmalhar\textunderscore )
\section{Estremalho}
\begin{itemize}
\item {Grp. gram.:m.}
\end{itemize}
Designação, applicada na Figueira da Foz a uma rêde de um só pano, para a pesca fluvial.
(Corr. de \textunderscore tresmalho\textunderscore )
\section{Estremança}
\begin{itemize}
\item {Grp. gram.:f.}
\end{itemize}
\begin{itemize}
\item {Utilização:Ant.}
\end{itemize}
Acto de estremar.
Estrema.
\section{Estremar}
\begin{itemize}
\item {Grp. gram.:v. t.}
\end{itemize}
\begin{itemize}
\item {Proveniência:(Do b. lat. \textunderscore extremare\textunderscore )}
\end{itemize}
Pôr estremas em.
Delimitar.
Demarcar.
Escolher.
Assignalar, distinguir.
Separar.
Recopilar.
\section{Estremável}
\begin{itemize}
\item {Grp. gram.:adj.}
\end{itemize}
Que se póde estremar.
\section{Estreme}
\begin{itemize}
\item {Grp. gram.:adj.}
\end{itemize}
\begin{itemize}
\item {Grp. gram.:M.}
\end{itemize}
\begin{itemize}
\item {Utilização:Ant.}
\end{itemize}
\begin{itemize}
\item {Proveniência:(De \textunderscore estremar\textunderscore )}
\end{itemize}
Que não tem mistura.
Puro; genuíno: \textunderscore vinho estreme\textunderscore .
Quinhão; pertença.
\section{Estremeção}
\begin{itemize}
\item {Grp. gram.:f.}
\end{itemize}
Acto de estremecer subitamente.
Acto de sacudir.
\section{Estremecer}
\begin{itemize}
\item {Grp. gram.:v. t.}
\end{itemize}
\begin{itemize}
\item {Grp. gram.:V. i.}
\end{itemize}
\begin{itemize}
\item {Utilização:Fig.}
\end{itemize}
\begin{itemize}
\item {Proveniência:(Do b. lat. \textunderscore extremiscere\textunderscore )}
\end{itemize}
Fazer tremer.
Sacudir: \textunderscore o vento estremecia as árvores\textunderscore .
Assustar.
Causar abalo a.
Amar enternecidamente: \textunderscore o pai deve estremecer os filhos\textunderscore .
Tremer de súbito.
Assustar-se.
Vibrar.
\section{Estremecimento}
\begin{itemize}
\item {Grp. gram.:m.}
\end{itemize}
Acto ou effeito de estremecer.
\section{Estremenho}
\begin{itemize}
\item {Grp. gram.:adj.}
\end{itemize}
\begin{itemize}
\item {Grp. gram.:M.}
\end{itemize}
\begin{itemize}
\item {Proveniência:(De \textunderscore estrema\textunderscore )}
\end{itemize}
Relativo á estremadura, ou á provincia portuguesa dêste nome: \textunderscore costumes estremenhos\textunderscore .
Aquelle que é da estremadura, da raia.
\section{Estremunhar}
\begin{itemize}
\item {Grp. gram.:v. t.}
\end{itemize}
\begin{itemize}
\item {Grp. gram.:V. i.}
\end{itemize}
Despertar de súbito e imperfeitamente (quem dormia).
Acordar de repente, ficando ainda um tanto dominado pelo somno.
(Liga-se a \textunderscore tremer\textunderscore ?)
\section{Estrenoitar}
\textunderscore v. t.\textunderscore  (e der.)
Metáth. pop. de \textunderscore tresnoitar\textunderscore , etc. Cf. Camillo, \textunderscore O Bem e o Mal\textunderscore , 157.
\section{Estrenuamente}
\begin{itemize}
\item {Grp. gram.:adv.}
\end{itemize}
\begin{itemize}
\item {Proveniência:(De \textunderscore estrênuo\textunderscore )}
\end{itemize}
Com valor, com coragem; com denodo.
\section{Estrênuo}
\begin{itemize}
\item {Grp. gram.:adj.}
\end{itemize}
\begin{itemize}
\item {Proveniência:(Lat. \textunderscore strenuus\textunderscore )}
\end{itemize}
Valente; corajoso.
Denodado.
Activo.
\section{Estrepada}
\begin{itemize}
\item {Grp. gram.:f.}
\end{itemize}
Ferida causada por estrepe.
\section{Estrepadela}
\begin{itemize}
\item {Grp. gram.:f.}
\end{itemize}
O mesmo que \textunderscore estrepada\textunderscore .
\section{Estrepar}
\begin{itemize}
\item {Grp. gram.:v. t.}
\end{itemize}
Guarnecer de estrepes.
Ferir com estrepes.
\section{Estrepe}
\begin{itemize}
\item {Grp. gram.:m.}
\end{itemize}
\begin{itemize}
\item {Utilização:Prov.}
\end{itemize}
\begin{itemize}
\item {Proveniência:(Do it. \textunderscore sterpo\textunderscore )}
\end{itemize}
Espinho.
Pua de ferro ou madeira.
Cana de milho, que, cortada obliquamente, fere com a ponta como uma faca.
Conjunto de vidros partidos ou de puas de ferro ou de madeira, que corôam os muros, para que não sejam escalados.
\section{Estrepeiro}
\begin{itemize}
\item {Grp. gram.:m.}
\end{itemize}
\begin{itemize}
\item {Proveniência:(De \textunderscore estrepe\textunderscore )}
\end{itemize}
O mesmo que \textunderscore pilriteiro\textunderscore .
\section{Estrepitado}
\begin{itemize}
\item {Grp. gram.:adj.}
\end{itemize}
Que faz estrépito.
\section{Estrepitante}
\begin{itemize}
\item {Grp. gram.:adj.}
\end{itemize}
Que estrepita.
\section{Estrepitar}
\begin{itemize}
\item {Grp. gram.:v. i.}
\end{itemize}
Fazer estrépito.
\section{Estrépito}
\begin{itemize}
\item {Grp. gram.:m.}
\end{itemize}
\begin{itemize}
\item {Proveniência:(Lat. \textunderscore strepitus\textunderscore )}
\end{itemize}
Estrondo grande.
Ruído.
Fragor; estampido.
Tumulto.
\section{Estrepitosamente}
\begin{itemize}
\item {Grp. gram.:adv.}
\end{itemize}
\begin{itemize}
\item {Utilização:Fig.}
\end{itemize}
\begin{itemize}
\item {Proveniência:(De \textunderscore estrepitoso\textunderscore )}
\end{itemize}
Com estrépito.
Com ostentação.
\section{Estrepitoso}
\begin{itemize}
\item {Grp. gram.:adj.}
\end{itemize}
\begin{itemize}
\item {Utilização:Fig.}
\end{itemize}
\begin{itemize}
\item {Proveniência:(De \textunderscore estrépito\textunderscore )}
\end{itemize}
Que produz estrépito.
Que dá na vista, que é notório.
Ostentoso, magnificente.
\section{Estrepolia}
\begin{itemize}
\item {Grp. gram.:f.}
\end{itemize}
\begin{itemize}
\item {Utilização:Bras}
\end{itemize}
O mesmo que \textunderscore estropelia\textunderscore .
\section{Estrepontim}
\begin{itemize}
\item {Grp. gram.:adj.}
\end{itemize}
\begin{itemize}
\item {Utilização:Fam.}
\end{itemize}
Endiabrado, traquinas.
(Relaciona-se com \textunderscore estrépito\textunderscore )
\section{Estreptocóccico}
\begin{itemize}
\item {Grp. gram.:adj.}
\end{itemize}
Relativo ao estreptococco.
\section{Estreptococco}
\begin{itemize}
\item {Grp. gram.:m.}
\end{itemize}
\begin{itemize}
\item {Proveniência:(Do lat. \textunderscore streptus\textunderscore  + \textunderscore coccum\textunderscore )}
\end{itemize}
Micróbio, que produz a erysipela, o panarício, e outras doenças.
\section{Estreptocócico}
\begin{itemize}
\item {Grp. gram.:adj.}
\end{itemize}
Relativo ao estreptococo.
\section{Estreptococo}
\begin{itemize}
\item {Grp. gram.:m.}
\end{itemize}
\begin{itemize}
\item {Proveniência:(Do lat. \textunderscore streptus\textunderscore  + \textunderscore coccum\textunderscore )}
\end{itemize}
Micróbio, que produz a erisipela, o panarício, e outras doenças.
\section{Estresir}
\begin{itemize}
\item {Grp. gram.:v. t.}
\end{itemize}
Passar de um papel para outro ou de uma superfície para outra (um desenho), por meio de pó de lápis ou de carvão, ou por meio de lápis, pondo papel transparente entre as duas superfícies.
(Relaciona-se provavelmente com o lat. \textunderscore transire\textunderscore )
\section{Estretalar}
\begin{itemize}
\item {Grp. gram.:v. t.}
\end{itemize}
\begin{itemize}
\item {Utilização:Prov.}
\end{itemize}
\begin{itemize}
\item {Utilização:trasm.}
\end{itemize}
Esbugalhar (os olhos).
\section{Estrevango}
\begin{itemize}
\item {Grp. gram.:m.}
\end{itemize}
\begin{itemize}
\item {Utilização:Prov.}
\end{itemize}
\begin{itemize}
\item {Utilização:trasm.}
\end{itemize}
O mesmo que \textunderscore esterloixo\textunderscore .
\section{Estrevenga}
\begin{itemize}
\item {Grp. gram.:f.}
\end{itemize}
\begin{itemize}
\item {Utilização:Prov.}
\end{itemize}
\begin{itemize}
\item {Utilização:alent.}
\end{itemize}
(V.estrovenga)
\section{Estria}
\begin{itemize}
\item {Grp. gram.:f.}
\end{itemize}
\begin{itemize}
\item {Proveniência:(Lat. \textunderscore stria\textunderscore )}
\end{itemize}
Sulco estreitíssimo, traço ou aresta, na superfície de certos ossos, rochas, conchas, etc.
Meia cana, em columna ou pilastra.
\section{Estria}
\begin{itemize}
\item {Grp. gram.:f.}
\end{itemize}
\begin{itemize}
\item {Proveniência:(Do lat. \textunderscore striga\textunderscore )}
\end{itemize}
Vampiro.
Bruxa que, segundo as superstições populares, suga o sangue ás crianças.
\section{Estriamento}
\begin{itemize}
\item {Grp. gram.:m.}
\end{itemize}
Acto de estriar.
\section{Estriar}
\begin{itemize}
\item {Grp. gram.:v. t.}
\end{itemize}
\begin{itemize}
\item {Proveniência:(De \textunderscore estria\textunderscore ^1)}
\end{itemize}
Fazer estrias em; traçar linhas parallelas e longitudinaes em.
Abrir estrias em.
\section{Estribamento}
\begin{itemize}
\item {Grp. gram.:m.}
\end{itemize}
\begin{itemize}
\item {Utilização:Des.}
\end{itemize}
Acto de estribar.
\section{Estribar}
\begin{itemize}
\item {Grp. gram.:v. t.}
\end{itemize}
\begin{itemize}
\item {Grp. gram.:V. i.  e  p.}
\end{itemize}
\begin{itemize}
\item {Utilização:Fig.}
\end{itemize}
Segurar nos estribos.
Segurar; apoiar.
Firmar os pés nos estribos.
Fundamentar-se; apoiar-se: \textunderscore estribou-se na lei\textunderscore .
\section{Estribaria}
\begin{itemize}
\item {Grp. gram.:f.}
\end{itemize}
O mesmo ou melhor que \textunderscore estrebaria\textunderscore .
\section{Estribeira}
\begin{itemize}
\item {Grp. gram.:f.}
\end{itemize}
\begin{itemize}
\item {Grp. gram.:Loc.}
\end{itemize}
\begin{itemize}
\item {Utilização:fam.}
\end{itemize}
\begin{itemize}
\item {Proveniência:(De \textunderscore estribo\textunderscore )}
\end{itemize}
Estribo de montar á gineta.
Estribo de carruagem.
\textunderscore Perder as estribeiras\textunderscore , desnortear-se, atrapalhar-se, perder o fio do discurso.
\section{Estribeiro}
\begin{itemize}
\item {Grp. gram.:m.}
\end{itemize}
\begin{itemize}
\item {Proveniência:(Do b. lat. \textunderscore stribarius\textunderscore )}
\end{itemize}
Aquelle, que tem a seu cuidado cavallariças, coches, arreios, etc.
\section{Estribelho}
\begin{itemize}
\item {fónica:bê}
\end{itemize}
\begin{itemize}
\item {Grp. gram.:m.}
\end{itemize}
\begin{itemize}
\item {Utilização:Gír.}
\end{itemize}
Tribunal.
\section{Estribilhas}
\begin{itemize}
\item {Grp. gram.:f. pl.}
\end{itemize}
Peças de madeira, entre as quaes os encadernadores seguram os livros, para os coser.
(Cp. \textunderscore estribo\textunderscore )
\section{Estribilho}
\begin{itemize}
\item {Grp. gram.:m.}
\end{itemize}
\begin{itemize}
\item {Utilização:Fig.}
\end{itemize}
Verso ou versos, que se repetem no fim das estâncias de uma poesia lýrica.
Trecho de música, repetido na mesma peça com o mesmo intervallo.
Palavra ou phrase, que alguém emprega com frequência; bordão.
(Cast. \textunderscore estribillo\textunderscore )
\section{Estribo}
\begin{itemize}
\item {Grp. gram.:m.}
\end{itemize}
\begin{itemize}
\item {Utilização:Fig.}
\end{itemize}
Cada uma das duas peças suspensas, em que o cavalleiro firma os pés, quando cavalga.
Espécie de degrau, abaixo da entrada da carruagem, da locomotiva, etc.
Cabo nas vêrgas do navio, no qual os marinheiros firmam os pés, quando ferram o pano.
Peça de ferro, com que se seguram traves ou se ligam duas peças de madeira.
Um dos pequenos ossos interiores do ouvido.
Esteio.
Fundamento.
(B. lat. \textunderscore stribus\textunderscore )
\section{Estribordo}
\begin{itemize}
\item {Grp. gram.:m.}
\end{itemize}
\begin{itemize}
\item {Utilização:Ant.}
\end{itemize}
O mesmo que \textunderscore estibordo\textunderscore . Cf. Azurara. \textunderscore Chrón. de D. Pedro\textunderscore , L. II, c. 16.
\section{Estriçar}
\textunderscore v. t.\textunderscore  (e der.)
O mesmo que \textunderscore destrinçar\textunderscore , etc.
\section{Estricote}
\begin{itemize}
\item {Grp. gram.:m.}
\end{itemize}
\begin{itemize}
\item {Utilização:Des.}
\end{itemize}
Mistura, confusão.
Lôgro.
\section{Estrictamente}
\begin{itemize}
\item {Grp. gram.:adv.}
\end{itemize}
De modo estricto.
Precisamente, com exactidão.
\section{Estricnato}
\begin{itemize}
\item {Grp. gram.:m.}
\end{itemize}
Sal, produzido pela combinação do ácido estrícnico com uma base.
\section{Estricneas}
\begin{itemize}
\item {Grp. gram.:f. pl.}
\end{itemize}
Tribo de plantas, que têm por tipo o estricno.
\section{Estricnico}
\begin{itemize}
\item {Grp. gram.:adj.}
\end{itemize}
\begin{itemize}
\item {Proveniência:(De \textunderscore estricno\textunderscore )}
\end{itemize}
Diz-se de um ácido particular.
\section{Estricnina}
\begin{itemize}
\item {Grp. gram.:f.}
\end{itemize}
\begin{itemize}
\item {Proveniência:(De \textunderscore estricno\textunderscore )}
\end{itemize}
Alcaloide muito venenoso, extraido de uma tribo de plantas, a que pertence a nóz vómica.
\section{Estricnínico}
\begin{itemize}
\item {Grp. gram.:adj.}
\end{itemize}
Diz-se de um ácido, obtido pela acção do ácido sulfúrico quente sôbre a estricnina.
\section{Estricnismo}
\begin{itemize}
\item {Grp. gram.:m.}
\end{itemize}
Conjunto de fenómenos, resultantes do uso da estricnina.
\section{Estricno}
\begin{itemize}
\item {Grp. gram.:m.}
\end{itemize}
\begin{itemize}
\item {Proveniência:(Gr. \textunderscore struknos\textunderscore )}
\end{itemize}
Gênero de plantas loganiáceas, a que pertence aquela que dá a nóz vómica.
\section{Estricnocromina}
\begin{itemize}
\item {Grp. gram.:f.}
\end{itemize}
\begin{itemize}
\item {Proveniência:(Do gr. \textunderscore strukhnos\textunderscore  + \textunderscore khroma\textunderscore )}
\end{itemize}
Substância còrante, extraida de estricno.
\section{Estricto}
\begin{itemize}
\item {Grp. gram.:adj.}
\end{itemize}
\begin{itemize}
\item {Proveniência:(Lat. \textunderscore strictus\textunderscore )}
\end{itemize}
Exacto.
Rigoroso.
Restricto.
\section{Estrictura}
\begin{itemize}
\item {Grp. gram.:f.}
\end{itemize}
\begin{itemize}
\item {Proveniência:(De \textunderscore estricto\textunderscore )}
\end{itemize}
Compressão; estrangulação.
\section{Estridência}
\begin{itemize}
\item {Grp. gram.:f.}
\end{itemize}
Qualidade daquillo que é estridente. Cf. Eça, \textunderscore Mandarim\textunderscore , 74.
\section{Estridente}
\begin{itemize}
\item {Grp. gram.:adj.}
\end{itemize}
\begin{itemize}
\item {Proveniência:(Lat. \textunderscore stridens\textunderscore )}
\end{itemize}
Que produz ruído agudo.
Que causa estridor.
\section{Estridor}
\begin{itemize}
\item {Grp. gram.:m.}
\end{itemize}
\begin{itemize}
\item {Proveniência:(Lat. \textunderscore stridor\textunderscore )}
\end{itemize}
Som agudo e áspero.
Silvo.
\section{Estridoroso}
\begin{itemize}
\item {Grp. gram.:adj.}
\end{itemize}
Que faz estridor; estridente. Cf. Júl. Dinis, \textunderscore Fidalgos\textunderscore , I, 20; Camillo, \textunderscore Myst. de Lisb.\textunderscore , I, 138 e 198; II, 173.
\section{Estridulação}
\begin{itemize}
\item {Grp. gram.:f.}
\end{itemize}
\begin{itemize}
\item {Proveniência:(De \textunderscore estridular\textunderscore )}
\end{itemize}
Som vibrante, peculiar a certos insectos.
\section{Estridulante}
\begin{itemize}
\item {Grp. gram.:adj.}
\end{itemize}
\begin{itemize}
\item {Grp. gram.:M. pl.}
\end{itemize}
\begin{itemize}
\item {Proveniência:(De \textunderscore estridular\textunderscore )}
\end{itemize}
Que estridula.
Família de insectos, a que pertence a cigarra.
\section{Estridular}
\begin{itemize}
\item {Grp. gram.:v. i.}
\end{itemize}
\begin{itemize}
\item {Proveniência:(De \textunderscore estrídulo\textunderscore )}
\end{itemize}
Produzir estridulação.
\section{Estrídulo}
\begin{itemize}
\item {Grp. gram.:adj.}
\end{itemize}
\begin{itemize}
\item {Proveniência:(Lat. \textunderscore stridulus\textunderscore )}
\end{itemize}
O mesmo que \textunderscore estridente\textunderscore .
\section{Estriduloso}
\begin{itemize}
\item {Grp. gram.:adj.}
\end{itemize}
(V.estridente)
\section{Estriga}
\begin{itemize}
\item {Grp. gram.:f.}
\end{itemize}
\begin{itemize}
\item {Proveniência:(Lat. \textunderscore striga\textunderscore )}
\end{itemize}
Porção de linho, que se põe de cada vez na roca, para se fiar.
Conjunto de filamentos de algumas plantas.
Madeixa.
\section{Estrigada}
\begin{itemize}
\item {Grp. gram.:f.}
\end{itemize}
Acto de estrigar. Cf. Camillo, \textunderscore Narcót.\textunderscore , I, 156.
\section{Estrigar}
\begin{itemize}
\item {Grp. gram.:v. t.}
\end{itemize}
\begin{itemize}
\item {Utilização:Prov.}
\end{itemize}
\begin{itemize}
\item {Utilização:minh.}
\end{itemize}
\begin{itemize}
\item {Utilização:Prov.}
\end{itemize}
\begin{itemize}
\item {Utilização:minh.}
\end{itemize}
Separar e atar em estrigas: \textunderscore estrigar o linho\textunderscore .
Ennastrar.
Tornar assedado.
Dar segunda maçagem a (o linho).
Dar sova em, bater.
\section{Estrige}
\begin{itemize}
\item {Grp. gram.:f.}
\end{itemize}
\begin{itemize}
\item {Proveniência:(Lat. \textunderscore strix\textunderscore , \textunderscore strigis\textunderscore )}
\end{itemize}
Coruja.
Vampiro, gênio malfeitor e nocturno, segundo as crenças antigas.
Feiticeira.
O mesmo que \textunderscore estria\textunderscore ^2.
\section{Estrígil}
\begin{itemize}
\item {Grp. gram.:m.}
\end{itemize}
\begin{itemize}
\item {Proveniência:(Lat. \textunderscore strigilis\textunderscore )}
\end{itemize}
Pequena almofaça, com que se coçava o corpo, especialmente no banho.
Corpo architectónico, com a linha externa em fórma de S.
\section{Estrinca}
\begin{itemize}
\item {Grp. gram.:f.}
\end{itemize}
\begin{itemize}
\item {Proveniência:(Do ingl. \textunderscore string\textunderscore )}
\end{itemize}
Espécie de escotilha.
\section{Estrincar}
\begin{itemize}
\item {Grp. gram.:v. t.}
\end{itemize}
\begin{itemize}
\item {Utilização:Prov.}
\end{itemize}
\begin{itemize}
\item {Utilização:trasm.}
\end{itemize}
\begin{itemize}
\item {Proveniência:(Do lat. \textunderscore stringere\textunderscore ? Ou relaciona-se com \textunderscore trincar\textunderscore ?)}
\end{itemize}
Estorcer, fazendo estalar.
Partir com os dentes (qualquer coisa, sacudindo-a com frenesi).
\section{Estrincar}
\begin{itemize}
\item {Grp. gram.:v. t.}
\end{itemize}
Desfazer:«\textunderscore ...estrinçá-la com os dentes\textunderscore ». Camillo, \textunderscore Corja\textunderscore , 150.
\section{Estrinchar}
\begin{itemize}
\item {Grp. gram.:v. i.}
\end{itemize}
\begin{itemize}
\item {Utilização:Pop.}
\end{itemize}
Saltar, brincar.
\section{Estringe}
\begin{itemize}
\item {Grp. gram.:f.}
\end{itemize}
Espécie de túnica, usada entre os Godos da Espanha. Cf. Herculano, \textunderscore Eurico\textunderscore .
(B. lat. \textunderscore striga\textunderscore  e \textunderscore stringes\textunderscore , do lat. \textunderscore stringere\textunderscore )
\section{Estrinque}
\begin{itemize}
\item {Grp. gram.:m.}
\end{itemize}
O mesmo que \textunderscore estrinca\textunderscore .
\section{Estrinqueiro}
\begin{itemize}
\item {Grp. gram.:m.}
\end{itemize}
\begin{itemize}
\item {Utilização:Ant.}
\end{itemize}
Aquelle, que fazia estrinques.
Cordoeiro.
\section{Estripação}
\begin{itemize}
\item {Grp. gram.:f.}
\end{itemize}
Acto ou effeito de estripar.
\section{Estripar}
\begin{itemize}
\item {Grp. gram.:v. t.}
\end{itemize}
\begin{itemize}
\item {Utilização:Ext.}
\end{itemize}
Tirar as tripas a.
Fazer carnificina em.
\section{Estro}
\begin{itemize}
\item {Grp. gram.:m.}
\end{itemize}
\begin{itemize}
\item {Proveniência:(Lat. \textunderscore oestrus\textunderscore )}
\end{itemize}
Fantasía rica.
Engenho poético.
Inspiração.
Insecto díptero, com antennas curtas, sem sugadoiro nem tromba, e geralmente parasito do cavallo.
\section{Estro}
\begin{itemize}
\item {Grp. gram.:m.}
\end{itemize}
\begin{itemize}
\item {Utilização:Prov.}
\end{itemize}
\begin{itemize}
\item {Utilização:minh.}
\end{itemize}
Lastro, pavimento, (de um forno, por exemplo).
(Corr. de \textunderscore lastro\textunderscore ?)
\section{Estróbilo}
\begin{itemize}
\item {Grp. gram.:m.}
\end{itemize}
\begin{itemize}
\item {Proveniência:(Gr. \textunderscore strobílos\textunderscore )}
\end{itemize}
Cóne das espigas, em que as escamas protectoras das flôres são membranosas.
Fruto de planta conífera.
\section{Estroço}
\begin{itemize}
\item {fónica:trô}
\end{itemize}
\begin{itemize}
\item {Grp. gram.:m.}
\end{itemize}
O mesmo que \textunderscore destrôço\textunderscore . Cf. Filinto, VII, 114; \textunderscore Pant. de Aveiro, Itiner.\textunderscore , 37 v.^o (3.^a ed.).
\section{Estrofantina}
\begin{itemize}
\item {Grp. gram.:f.}
\end{itemize}
\begin{itemize}
\item {Utilização:Pharm.}
\end{itemize}
Alcaloide medicinal do estrofanto.
\section{Estrofanto}
\begin{itemize}
\item {Grp. gram.:m.}
\end{itemize}
Planta apocinácea, medicinal, (\textunderscore strophantus hispidus\textunderscore ).
\section{Estrofe}
\begin{itemize}
\item {Grp. gram.:f.}
\end{itemize}
\begin{itemize}
\item {Proveniência:(Lat. \textunderscore strophe\textunderscore )}
\end{itemize}
Conjunto de versos, o mesmo que \textunderscore estância\textunderscore .
No teatro antigo, a parte do canto, correspondente ao movimento dos coros, quando marchavam para a direita.
\section{Estrófico}
\begin{itemize}
\item {Grp. gram.:adj.}
\end{itemize}
Relativo a estrofe.
\section{Estroina}
\begin{itemize}
\item {Grp. gram.:adj. ,  m.  e  f.}
\end{itemize}
Pessôa extravagante, dissipadora, estouvada.
\section{Estroinar}
\begin{itemize}
\item {Grp. gram.:v. i.}
\end{itemize}
Praticar estroinices.
Viver como estroina.
\section{Estroinice}
\begin{itemize}
\item {Grp. gram.:f.}
\end{itemize}
Acção de estroina.
Qualidade de estroina.
\section{Estrói-tudo}
\begin{itemize}
\item {Grp. gram.:m.}
\end{itemize}
\begin{itemize}
\item {Utilização:Pop.}
\end{itemize}
Fanfarrão.
Homem bulhento.
\section{Estrólico}
\begin{itemize}
\item {Grp. gram.:m.}
\end{itemize}
\begin{itemize}
\item {Utilização:Ant.}
\end{itemize}
O mesmo que \textunderscore astrólogo\textunderscore . Cf. Rui de Pina, \textunderscore Chrónicas\textunderscore .
\section{Estroma}
\begin{itemize}
\item {Grp. gram.:m.}
\end{itemize}
\begin{itemize}
\item {Utilização:Bot.}
\end{itemize}
\begin{itemize}
\item {Utilização:Anat.}
\end{itemize}
\begin{itemize}
\item {Proveniência:(Lat. \textunderscore stroma\textunderscore )}
\end{itemize}
Superfície fructífera das plantas cryptogâmicas.
Parte superficial do ovário; trama de um tecido.
\section{Estromania}
\begin{itemize}
\item {Grp. gram.:f.}
\end{itemize}
Qualidade de estromaníaco.
\section{Estromaníaco}
\begin{itemize}
\item {Grp. gram.:adj.}
\end{itemize}
\begin{itemize}
\item {Proveniência:(Do lat. \textunderscore oestrum\textunderscore  + \textunderscore mania\textunderscore )}
\end{itemize}
O mesmo que \textunderscore nymphomaníaco\textunderscore .
\section{Estrômatos}
\begin{itemize}
\item {Grp. gram.:m. pl.}
\end{itemize}
\begin{itemize}
\item {Proveniência:(Do gr. \textunderscore stroma\textunderscore )}
\end{itemize}
Gênero de peixes, de corpo esguio e oval.
\section{Estromaturgia}
\begin{itemize}
\item {Grp. gram.:f.}
\end{itemize}
\begin{itemize}
\item {Utilização:Neol.}
\end{itemize}
\begin{itemize}
\item {Proveniência:(Do gr. \textunderscore stroma\textunderscore , \textunderscore stromatos\textunderscore  + \textunderscore ergon\textunderscore )}
\end{itemize}
Arte de fabricar tapêtes.
\section{Estrombo}
\begin{itemize}
\item {Grp. gram.:m.}
\end{itemize}
\begin{itemize}
\item {Proveniência:(Gr. \textunderscore strombos\textunderscore )}
\end{itemize}
Espécie de concha univalve.
Gênero de molluscos gasterópodes.
\section{Estromento}
\begin{itemize}
\item {Grp. gram.:m.}
\end{itemize}
\begin{itemize}
\item {Utilização:Ant.}
\end{itemize}
O mesmo que \textunderscore instrumento\textunderscore .
(B. lat. \textunderscore strumentum\textunderscore )
\section{Estrompar}
\begin{itemize}
\item {Grp. gram.:v. t.}
\end{itemize}
\begin{itemize}
\item {Utilização:Pop.}
\end{itemize}
Gastar.
Deteriorar.
Estragar.
Romper: \textunderscore estrompar as botas\textunderscore .
Esfalfar: \textunderscore estrompar um cavallo\textunderscore .
(Cp. \textunderscore estropear\textunderscore ^1)
\section{Estrompida}
\begin{itemize}
\item {Grp. gram.:f.}
\end{itemize}
\begin{itemize}
\item {Utilização:Prov.}
\end{itemize}
\begin{itemize}
\item {Utilização:alg.}
\end{itemize}
O mesmo que \textunderscore estrompido\textunderscore .
\section{Estrompido}
\begin{itemize}
\item {Grp. gram.:m.}
\end{itemize}
Estampido; estrépito.
(Corr. de \textunderscore estampido\textunderscore ?)
\section{Estronca}
\begin{itemize}
\item {Grp. gram.:f.}
\end{itemize}
\begin{itemize}
\item {Proveniência:(De \textunderscore estroncar\textunderscore )}
\end{itemize}
Forquilha, para levantar grandes pesos.
Escora de barreira ou de parede, para que esta não desabe.
Pau, que sustenta o cabeçalho do carro, para que êste não poise no chão, tirada a carga.
\section{Estroncada}
\begin{itemize}
\item {Grp. gram.:f.}
\end{itemize}
\begin{itemize}
\item {Utilização:Ant.}
\end{itemize}
\begin{itemize}
\item {Proveniência:(De \textunderscore estroncar\textunderscore )}
\end{itemize}
Cutilada. Cp. Lobo, \textunderscore Auto do Nascimento\textunderscore .
\section{Estroncamento}
\begin{itemize}
\item {Grp. gram.:m.}
\end{itemize}
Acto ou effeito de estroncar.
\section{Estroncar}
\begin{itemize}
\item {Grp. gram.:v. t.}
\end{itemize}
\begin{itemize}
\item {Grp. gram.:V. i.}
\end{itemize}
Destroncar.
Partir.
Derramar.
Estropear.
Desmanchar.
Fazer ruído, como de pancada num tronco.
\section{Estronçar}
\begin{itemize}
\item {Grp. gram.:v. t.}
\end{itemize}
\begin{itemize}
\item {Utilização:Prov.}
\end{itemize}
\begin{itemize}
\item {Utilização:trasm.}
\end{itemize}
Partir em troços (couves, principalmente).
(Cp. cast. \textunderscore tronzar\textunderscore )
\section{Estronciana}
\begin{itemize}
\item {Grp. gram.:f.}
\end{itemize}
\begin{itemize}
\item {Proveniência:(De \textunderscore Strontian\textunderscore , n. p.)}
\end{itemize}
Substância alcalina, descoberta em Strontian, na Escócia.
\section{Estrôncio}
\begin{itemize}
\item {Grp. gram.:m.}
\end{itemize}
\begin{itemize}
\item {Proveniência:(Do rad. de \textunderscore estronciana\textunderscore )}
\end{itemize}
Metal, que, ligado ao oxygênio, produz a estronciana.
\section{Estrondar}
\begin{itemize}
\item {Grp. gram.:v. i.}
\end{itemize}
\begin{itemize}
\item {Grp. gram.:V. p.}
\end{itemize}
\begin{itemize}
\item {Utilização:Prov.}
\end{itemize}
\begin{itemize}
\item {Utilização:minh.}
\end{itemize}
O mesmo que \textunderscore estrondear\textunderscore .
Partir-se, escangalhar-se, escancellar-se, (falando-se de carros, que se desconjuntam com os abalos em maus caminhos).
\section{Estrondeante}
\begin{itemize}
\item {Grp. gram.:adj.}
\end{itemize}
Que estrondeia.
\section{Estrondear}
\begin{itemize}
\item {Grp. gram.:v. i.}
\end{itemize}
\begin{itemize}
\item {Utilização:Fig.}
\end{itemize}
Fazer estrondo.
Sêr notório, afamado.
Clamar; vociferar.
\section{Estrondo}
\begin{itemize}
\item {Grp. gram.:m.}
\end{itemize}
\begin{itemize}
\item {Utilização:Fig.}
\end{itemize}
\begin{itemize}
\item {Proveniência:(Do lat. \textunderscore ex-tonitrus\textunderscore )}
\end{itemize}
Grande ruído.
Estampido.
Grande luxo.
Magnificência; ostentação.
\section{Estrondosamente}
\begin{itemize}
\item {Grp. gram.:adv.}
\end{itemize}
De modo estrondoso.
\section{Estrondoso}
\begin{itemize}
\item {Grp. gram.:adj.}
\end{itemize}
\begin{itemize}
\item {Utilização:Fig.}
\end{itemize}
Que faz estrondo.
Espectaculoso; magnificente.
\section{Estrongalhar}
\begin{itemize}
\item {Grp. gram.:v. i.}
\end{itemize}
\begin{itemize}
\item {Utilização:T. da Bairrada}
\end{itemize}
Fazer grande ruído, fechando portas ou batendo com ellas.
\section{Estropalho}
\begin{itemize}
\item {Grp. gram.:m.}
\end{itemize}
\begin{itemize}
\item {Utilização:Pop.}
\end{itemize}
Esfregão.
Trapo grosso.
Frangalho.
(Cp. cast. \textunderscore estropajo\textunderscore )
\section{Estropeação}
\begin{itemize}
\item {Grp. gram.:f.}
\end{itemize}
Acto ou effeito de estropear.
\section{Estropeada}
\begin{itemize}
\item {Grp. gram.:f.}
\end{itemize}
\begin{itemize}
\item {Utilização:Pop.}
\end{itemize}
\begin{itemize}
\item {Proveniência:(De \textunderscore estropear\textunderscore ^2)}
\end{itemize}
Tropel de pessôas ou animaes.
\section{Estropeadamente}
\begin{itemize}
\item {Grp. gram.:adv.}
\end{itemize}
\begin{itemize}
\item {Proveniência:(De \textunderscore estropear\textunderscore ^2)}
\end{itemize}
Com estropeamento.
\section{Estropeamento}
\begin{itemize}
\item {Grp. gram.:m.}
\end{itemize}
Acto ou effeito de estropear^1.
\section{Estropear}
\begin{itemize}
\item {Grp. gram.:v. t.}
\end{itemize}
\begin{itemize}
\item {Utilização:Fig.}
\end{itemize}
\begin{itemize}
\item {Proveniência:(It. \textunderscore stroppiare\textunderscore )}
\end{itemize}
Cortar algum membro a.
Privar do uso de algum membro.
Deformar.
Estragar.
Fatigar muito.
Interpretar mal o sentido de.
Executar mal, cantando: \textunderscore estropear uma cavatina\textunderscore .
Pronunciar mal, lêr mal: \textunderscore estropear versos\textunderscore .
\section{Estropear}
\begin{itemize}
\item {Grp. gram.:v. i.}
\end{itemize}
\begin{itemize}
\item {Utilização:Prov.}
\end{itemize}
\begin{itemize}
\item {Utilização:minh.}
\end{itemize}
\begin{itemize}
\item {Proveniência:(De \textunderscore tropear\textunderscore )}
\end{itemize}
Fazer tropel.
Bater com fôrça (a uma porta).
\section{Estropelia}
\begin{itemize}
\item {Grp. gram.:f.}
\end{itemize}
(V.tropelia)
\section{Estrophantina}
\begin{itemize}
\item {Grp. gram.:f.}
\end{itemize}
\begin{itemize}
\item {Utilização:Pharm.}
\end{itemize}
Alcaloide medicinal do estrophanto.
\section{Estrophanto}
\begin{itemize}
\item {Grp. gram.:m.}
\end{itemize}
Planta apocynácea, medicinal, (\textunderscore strophantus hispidus\textunderscore ).
\section{Estrophe}
\begin{itemize}
\item {Grp. gram.:f.}
\end{itemize}
\begin{itemize}
\item {Proveniência:(Lat. \textunderscore strophe\textunderscore )}
\end{itemize}
Conjunto de versos, o mesmo que \textunderscore estância\textunderscore .
No theatro antigo, a parte do canto, correspondente ao movimento dos coros, quando marchavam para a direita.
\section{Estróphico}
\begin{itemize}
\item {Grp. gram.:adj.}
\end{itemize}
Relativo a estrophe.
\section{Estropiar}
\begin{itemize}
\item {Grp. gram.:v. t.}
\end{itemize}
\begin{itemize}
\item {Utilização:Fig.}
\end{itemize}
\begin{itemize}
\item {Proveniência:(It. \textunderscore stroppiare\textunderscore )}
\end{itemize}
Cortar algum membro a.
Privar do uso de algum membro.
Deformar.
Estragar.
Fatigar muito.
Interpretar mal o sentido de.
Executar mal, cantando: \textunderscore estropiar uma cavatina\textunderscore .
Pronunciar mal, lêr mal: \textunderscore estropiar versos\textunderscore .
\section{Estropício}
\begin{itemize}
\item {Grp. gram.:m.}
\end{itemize}
\begin{itemize}
\item {Proveniência:(It. \textunderscore stropicio\textunderscore )}
\end{itemize}
Malefício: damno.
\section{Estropido}
\begin{itemize}
\item {Grp. gram.:m.}
\end{itemize}
\begin{itemize}
\item {Utilização:Ant.}
\end{itemize}
(V.estrupido)
\section{Estropo}
\begin{itemize}
\item {fónica:trô}
\end{itemize}
\begin{itemize}
\item {Grp. gram.:m.}
\end{itemize}
\begin{itemize}
\item {Utilização:Náut.}
\end{itemize}
\begin{itemize}
\item {Proveniência:(Do lat. \textunderscore stroppum\textunderscore )}
\end{itemize}
Cabo, que fórma um círculo e, além de outros usos, prende o remo ao tolete.
\section{Estroso}
\begin{itemize}
\item {Grp. gram.:adj.}
\end{itemize}
\begin{itemize}
\item {Utilização:Des.}
\end{itemize}
Idiota; lunático.
(Por \textunderscore astroso\textunderscore )
\section{Estrotejar}
\begin{itemize}
\item {Grp. gram.:v. i.}
\end{itemize}
Andar a trote.
\section{Estrouvinhar}
\begin{itemize}
\item {Grp. gram.:v. t.}
\end{itemize}
O mesmo que \textunderscore estrovinhar\textunderscore . Cf. Camillo, \textunderscore Corja\textunderscore , 16.
\section{Estrovadura}
\begin{itemize}
\item {Grp. gram.:f.}
\end{itemize}
O mesmo que \textunderscore estrovo\textunderscore .
\section{Estrovenga}
\begin{itemize}
\item {Grp. gram.:f.}
\end{itemize}
\begin{itemize}
\item {Utilização:Prov.}
\end{itemize}
\begin{itemize}
\item {Utilização:alent.}
\end{itemize}
\begin{itemize}
\item {Utilização:Bras. do N}
\end{itemize}
\begin{itemize}
\item {Proveniência:(De \textunderscore estrovo\textunderscore )}
\end{itemize}
Correia ou cadeia, que, nas carrêtas puxadas a quatro bois, prende a canga dos bois da deanteira á dos do coice.
Coisa complicada ou mysteriosa; aquillo, de que se não sabe a origem.
\section{Estrovinhar}
\begin{itemize}
\item {Grp. gram.:v. t.}
\end{itemize}
\begin{itemize}
\item {Utilização:Prov.}
\end{itemize}
\begin{itemize}
\item {Utilização:bras}
\end{itemize}
\begin{itemize}
\item {Utilização:trasm.}
\end{itemize}
O mesmo que \textunderscore estremunhar\textunderscore . Cf. Camillo, \textunderscore Caveira\textunderscore , 167; \textunderscore Narcót.\textunderscore , I, 248; Pacheco da Silva, \textunderscore Promptuário\textunderscore , etc.
\section{Estrovo}
\begin{itemize}
\item {fónica:trô}
\end{itemize}
\begin{itemize}
\item {Grp. gram.:m.}
\end{itemize}
\begin{itemize}
\item {Utilização:Bras. do N}
\end{itemize}
Fio, que prende o anzol á linha de pesca.
Corda, que prende o remo ao tolete.
Correia de ferro, que prende a segunda junta de bois á canga da primeira, quando um carro é puxado por mais de uma junta.
Chicote de coiro.
(Cp. gr. \textunderscore strophos\textunderscore )
\section{Estructa}
\begin{itemize}
\item {Grp. gram.:f.}
\end{itemize}
\begin{itemize}
\item {Utilização:Ant.}
\end{itemize}
Vestido de mulher, ou a parte dêsse vestido ajustada ao corpo. Cf. \textunderscore Foral de Cintra\textunderscore .
\section{Estructura}
\begin{itemize}
\item {Grp. gram.:f.}
\end{itemize}
\begin{itemize}
\item {Proveniência:(Lat. \textunderscore structura\textunderscore )}
\end{itemize}
Disposição e construcção de um edifício.
Disposição especial das partes de um todo, consideradas nas suas relações recíprocas.
Disposição e harmonia das partes de uma obra literária.
\section{Estructural}
\begin{itemize}
\item {Grp. gram.:adj.}
\end{itemize}
Relativo a estructura.
\section{Estrugido}
\begin{itemize}
\item {Grp. gram.:m.}
\end{itemize}
\begin{itemize}
\item {Utilização:Pop.}
\end{itemize}
\begin{itemize}
\item {Proveniência:(De \textunderscore estrugir\textunderscore )}
\end{itemize}
Tempêro culinário, com cebola, gordura, etc.; refogado.
\section{Estrugidor}
\begin{itemize}
\item {Grp. gram.:adj.}
\end{itemize}
Que estruge.
\section{Estrugimento}
\begin{itemize}
\item {Grp. gram.:m.}
\end{itemize}
Acto ou effeito de estrugir.
\section{Estrugir}
\begin{itemize}
\item {Grp. gram.:v. t.}
\end{itemize}
\begin{itemize}
\item {Utilização:Pop.}
\end{itemize}
\begin{itemize}
\item {Grp. gram.:V. i.}
\end{itemize}
Atroar.
Refogar.
Derreter (\textunderscore toicinho\textunderscore ).
Estrondear; vibrar com força; estralejar.
\section{Estruir}
\textunderscore v. t.\textunderscore  (e der.)
(V. \textunderscore destruir\textunderscore , etc.). Cf. \textunderscore Lusíadas\textunderscore , III, 114.
\section{Estruma}
\begin{itemize}
\item {Grp. gram.:f.}
\end{itemize}
\begin{itemize}
\item {Proveniência:(Lat. \textunderscore struma\textunderscore )}
\end{itemize}
Escrófula.
Bócio.
\section{Estrumação}
\begin{itemize}
\item {Grp. gram.:f.}
\end{itemize}
Acto de estrumar.
\section{Estrumada}
\begin{itemize}
\item {Grp. gram.:f.}
\end{itemize}
Meda de estrume. Cf. \textunderscore Bibl. da G. do Campo\textunderscore , 313.
\section{Estrumadal}
\begin{itemize}
\item {Grp. gram.:m.}
\end{itemize}
\begin{itemize}
\item {Utilização:Prov.}
\end{itemize}
\begin{itemize}
\item {Utilização:trasm.}
\end{itemize}
\begin{itemize}
\item {Utilização:Fig.}
\end{itemize}
\begin{itemize}
\item {Proveniência:(De \textunderscore estrumar\textunderscore )}
\end{itemize}
Grande porção, grande abundância.
Pessôa gorda e corpulenta.
\section{Estrumar}
\begin{itemize}
\item {Grp. gram.:v. t.}
\end{itemize}
Deitar estrume em.
Adubar (terras).
\section{Estrume}
\begin{itemize}
\item {Grp. gram.:m.}
\end{itemize}
\begin{itemize}
\item {Proveniência:(Do lat. hyp. \textunderscore strumen\textunderscore )}
\end{itemize}
Aquillo, com que se aduba a terra, para a fertilizar.
Estêrco.
Adubo vegetal, formado pelos detritos de ramos, palha, etc., misturados com os dejectos dos animaes.
\section{Estrumeira}
\begin{itemize}
\item {Grp. gram.:f.}
\end{itemize}
Lugar, onde se prepara e fermenta o estrume.
Monturo.
Esterqueira.
Lugar sujo.
\section{Estrumeiro}
\begin{itemize}
\item {Grp. gram.:m.}
\end{itemize}
Conductor de estrume para os campos.
\section{Estrumelo}
\begin{itemize}
\item {Grp. gram.:m.}
\end{itemize}
\begin{itemize}
\item {Utilização:Prov.}
\end{itemize}
\begin{itemize}
\item {Utilização:alent.}
\end{itemize}
Estalo, que o arrioz de um jogador produz, ao bater no arrioz de outro.
\section{Estrumoso}
\begin{itemize}
\item {Grp. gram.:adj.}
\end{itemize}
\begin{itemize}
\item {Proveniência:(Lat. \textunderscore strumosus\textunderscore )}
\end{itemize}
Que padece estrumas.
\section{Estrupada}
\begin{itemize}
\item {Grp. gram.:f.}
\end{itemize}
\begin{itemize}
\item {Utilização:Des.}
\end{itemize}
\begin{itemize}
\item {Proveniência:(It. \textunderscore strappata\textunderscore )}
\end{itemize}
Assalto; escaramuça.
\section{Estrupicento}
\begin{itemize}
\item {Grp. gram.:adj.}
\end{itemize}
\begin{itemize}
\item {Utilização:Bras. do N}
\end{itemize}
Que faz estrupício; desordeiro.
\section{Estrupício}
\begin{itemize}
\item {Grp. gram.:m.}
\end{itemize}
\begin{itemize}
\item {Utilização:Bras. do N}
\end{itemize}
Barulho; desordem.
(Cp. \textunderscore estrupido\textunderscore )
\section{Estrupida}
\begin{itemize}
\item {Grp. gram.:f.}
\end{itemize}
O mesmo que \textunderscore estrupido\textunderscore . Cf. Herculano, \textunderscore Hist. de Port.\textunderscore , I, 369; \textunderscore Bobo\textunderscore , 264 e 266.
\section{Estrupidante}
\begin{itemize}
\item {Grp. gram.:adj.}
\end{itemize}
\begin{itemize}
\item {Utilização:Neol.}
\end{itemize}
Que estrupida.
\section{Estrupidar}
\begin{itemize}
\item {Grp. gram.:v. i.}
\end{itemize}
\begin{itemize}
\item {Utilização:Neol.}
\end{itemize}
Fazer estrupido.
\section{Estrupido}
\begin{itemize}
\item {Grp. gram.:m.}
\end{itemize}
Grande estrondo.
Estrépito; estampido.
(Cp. \textunderscore estrompido\textunderscore )
\section{Estrupo}
\begin{itemize}
\item {Grp. gram.:m.}
\end{itemize}
\begin{itemize}
\item {Utilização:Ant.}
\end{itemize}
O mesmo que \textunderscore tropel\textunderscore ; estrondo.
(Cp. \textunderscore estrupidar\textunderscore )
\section{Estrychnato}
\begin{itemize}
\item {Grp. gram.:m.}
\end{itemize}
Sal, produzido pela combinação do ácido estrýchnico com uma base.
\section{Estrýchneas}
\begin{itemize}
\item {Grp. gram.:f. pl.}
\end{itemize}
Tribo de plantas, que têm por typo o estrychno.
\section{Estrýchnico}
\begin{itemize}
\item {Grp. gram.:adj.}
\end{itemize}
\begin{itemize}
\item {Proveniência:(De \textunderscore estrycno\textunderscore )}
\end{itemize}
Diz-se de um ácido particular.
\section{Estrychnina}
\begin{itemize}
\item {Grp. gram.:f.}
\end{itemize}
\begin{itemize}
\item {Proveniência:(De \textunderscore estrycno\textunderscore )}
\end{itemize}
Alcaloide muito venenoso, extrahido de uma tribo de plantas, a que pertence a nóz vómica.
\section{Estrychnínico}
\begin{itemize}
\item {Grp. gram.:adj.}
\end{itemize}
Diz-se de um ácido, obtido pela acção do ácido sulfúrico quente sôbre a estrychnina.
\section{Estrychnismo}
\begin{itemize}
\item {Grp. gram.:m.}
\end{itemize}
Conjunto de phenómenos, resultantes do uso da estrychnina.
\section{Estrychno}
\begin{itemize}
\item {Grp. gram.:m.}
\end{itemize}
\begin{itemize}
\item {Proveniência:(Gr. \textunderscore struknos\textunderscore )}
\end{itemize}
Gênero de plantas loganiáceas, a que pertence aquella que dá a nóz vómica.
\section{Estrychnochromina}
\begin{itemize}
\item {Grp. gram.:f.}
\end{itemize}
\begin{itemize}
\item {Proveniência:(Do gr. \textunderscore strukhnos\textunderscore  + \textunderscore khroma\textunderscore )}
\end{itemize}
Substância còrante, extrahida de estrychno.
\section{Estuação}
\begin{itemize}
\item {Grp. gram.:f.}
\end{itemize}
\begin{itemize}
\item {Proveniência:(Lat. \textunderscore aestuatio\textunderscore )}
\end{itemize}
Grande calor.
Enjôo, náuseas.
\section{Estuància}
\begin{itemize}
\item {Grp. gram.:f.}
\end{itemize}
(V.estuação)
\section{Estuante}
\begin{itemize}
\item {Grp. gram.:adj.}
\end{itemize}
\begin{itemize}
\item {Proveniência:(Lat. \textunderscore aestuans\textunderscore )}
\end{itemize}
Que estua.
\section{Estuar}
\begin{itemize}
\item {Grp. gram.:v. i.}
\end{itemize}
\begin{itemize}
\item {Proveniência:(Lat. \textunderscore aestuare\textunderscore )}
\end{itemize}
Estar ardente, effervescente.
Ferver.
Aquecer muito.
Agitar-se em ondas (o mar).
Agitar-se, como ondas:«\textunderscore de povo estuaram fúlgidos theatros.\textunderscore »Castilho, \textunderscore Escav. Poét.\textunderscore , 82.
\section{Estuário}
\begin{itemize}
\item {Grp. gram.:m.}
\end{itemize}
\begin{itemize}
\item {Proveniência:(Lat. \textunderscore aestuarium\textunderscore )}
\end{itemize}
Sinuosidade ou baía, formada por um rio, perto do mar, e na qual se confunde a água salgada com a doce.
Esteiro.
\section{Estucador}
\begin{itemize}
\item {Grp. gram.:m.  e  adj.}
\end{itemize}
\begin{itemize}
\item {Proveniência:(De \textunderscore estucar\textunderscore )}
\end{itemize}
O que trabalha em estuque por offício.
\section{Estucar}
\begin{itemize}
\item {Grp. gram.:v. t.}
\end{itemize}
\begin{itemize}
\item {Grp. gram.:V. i.}
\end{itemize}
Cobrir com estuque.
Trabalhar em estuque.
\section{Estucha}
\begin{itemize}
\item {Grp. gram.:f.}
\end{itemize}
\begin{itemize}
\item {Utilização:Pop.}
\end{itemize}
\begin{itemize}
\item {Proveniência:(De \textunderscore estuchar\textunderscore )}
\end{itemize}
Peça de ferro ou madeira, que se mete á fôrça em um orifício.
Empenho efficaz ou protecção valiosa.
O mesmo que \textunderscore estucho\textunderscore .
\section{Estuchada}
\begin{itemize}
\item {Grp. gram.:f.}
\end{itemize}
O mesmo que \textunderscore estucho\textunderscore .
\section{Estuchar}
\begin{itemize}
\item {Grp. gram.:v. t.}
\end{itemize}
\begin{itemize}
\item {Utilização:Fam.}
\end{itemize}
Meter com fôrça (peça de ferro ou madeira) em um orifício.
Obrigar com empenhos.
(Por \textunderscore estochar\textunderscore , de \textunderscore tocho\textunderscore )
\section{Estigial}
\begin{itemize}
\item {Grp. gram.:adj.}
\end{itemize}
\begin{itemize}
\item {Proveniência:(Lat. \textunderscore stygialis\textunderscore )}
\end{itemize}
O mesmo que \textunderscore estígio\textunderscore .
\section{Estígio}
\begin{itemize}
\item {Grp. gram.:adj.}
\end{itemize}
\begin{itemize}
\item {Proveniência:(Lat. \textunderscore stygius\textunderscore )}
\end{itemize}
Relativo ao rio infernal Estige.
\section{Estiráceas}
\begin{itemize}
\item {Grp. gram.:f. pl.}
\end{itemize}
\begin{itemize}
\item {Proveniência:(Do gr. \textunderscore sturax\textunderscore )}
\end{itemize}
Família de plantas, que têm por tipo o estoraque.
\section{Estuche}
\begin{itemize}
\item {Grp. gram.:m.}
\end{itemize}
\begin{itemize}
\item {Utilização:Prov.}
\end{itemize}
\begin{itemize}
\item {Utilização:trasm.}
\end{itemize}
O mesmo que \textunderscore estucha\textunderscore .
Seringa de cana, com que os rapazes se esguicham, mormente pelo Carnaval.
(Cp. cast. \textunderscore estuche\textunderscore )
\section{Estucho}
\begin{itemize}
\item {Grp. gram.:m.}
\end{itemize}
\begin{itemize}
\item {Utilização:T. da Bairrada}
\end{itemize}
O mesmo que \textunderscore estadulho\textunderscore .
Desastre, mau negócio.
Lôgo.
Maçada.
\section{Estudadamente}
\begin{itemize}
\item {Grp. gram.:adv.}
\end{itemize}
\begin{itemize}
\item {Utilização:Fig.}
\end{itemize}
\begin{itemize}
\item {Proveniência:(De \textunderscore estudar\textunderscore )}
\end{itemize}
Com estudo.
Com affectação.
Propositadamente.
\section{Estudantaço}
\begin{itemize}
\item {Grp. gram.:m.}
\end{itemize}
\begin{itemize}
\item {Utilização:Fam.}
\end{itemize}
Bom estudante.
\section{Estudantada}
\begin{itemize}
\item {Grp. gram.:f.}
\end{itemize}
Agrupamento de estudantes.
Brincadeira de estudante.
\section{Estudantado}
\begin{itemize}
\item {Grp. gram.:m.}
\end{itemize}
Estado ou vida de estudante. Cf. Sena Freitas, \textunderscore Lutas da Penna\textunderscore , II, 255.
\section{Estudantal}
\begin{itemize}
\item {Grp. gram.:adj.}
\end{itemize}
\begin{itemize}
\item {Utilização:bras}
\end{itemize}
\begin{itemize}
\item {Utilização:Neol.}
\end{itemize}
Relativo a estudantes; próprio de estudantes. Cf. \textunderscore Commércio do Amazonas\textunderscore , de 31-XII-88.
\section{Estudantão}
\begin{itemize}
\item {Grp. gram.:m.}
\end{itemize}
\begin{itemize}
\item {Utilização:Fam.}
\end{itemize}
O mesmo que \textunderscore estudantaço\textunderscore .
\section{Estudante}
\begin{itemize}
\item {Grp. gram.:m.}
\end{itemize}
Aquelle, que estuda.
Aquelle que frequenta qualquer instituto escolar.
\section{Estudanteco}
\begin{itemize}
\item {Grp. gram.:m.}
\end{itemize}
\begin{itemize}
\item {Utilização:Deprec.}
\end{itemize}
Pequeno estudante.
Estudante ordinário. Cf. Pato, \textunderscore Ciprestes\textunderscore , 273.
\section{Estudantina}
\begin{itemize}
\item {Grp. gram.:f.}
\end{itemize}
Agrupamento de estudantes, ou de indivíduos que os imitam no traje, e que cantam ou tocam em commum.
\section{Estudantório}
\begin{itemize}
\item {Grp. gram.:m.}
\end{itemize}
Mau estudante. Cf. Camillo, \textunderscore Narcót.\textunderscore , I, 176.
\section{Estudar}
\begin{itemize}
\item {Grp. gram.:v. t.}
\end{itemize}
\begin{itemize}
\item {Grp. gram.:V. i.}
\end{itemize}
\begin{itemize}
\item {Proveniência:(Lat. \textunderscore studere\textunderscore )}
\end{itemize}
Applicar a intelligência a, para apprender ou comprehender: \textunderscore estudar latim\textunderscore .
Decorar.
Analysar attentamente: \textunderscore estudar um problema\textunderscore .
Planear: \textunderscore estudar uma cilada\textunderscore .
Simular.
Applicar a intelligência ou a memória a alguma coisa.
Exercitar-se.
Sêr estudante.
Sêr estudioso.
\section{Estudaria}
\begin{itemize}
\item {Grp. gram.:f.}
\end{itemize}
\begin{itemize}
\item {Utilização:Ant.}
\end{itemize}
\begin{itemize}
\item {Proveniência:(De \textunderscore estudo\textunderscore )}
\end{itemize}
O mesmo que \textunderscore collégio\textunderscore .
\section{Estudiosamente}
\begin{itemize}
\item {Grp. gram.:adv.}
\end{itemize}
De modo estudioso.
\section{Estudiosidade}
\begin{itemize}
\item {Grp. gram.:f.}
\end{itemize}
Qualidade de quem é estudioso.
\section{Estudioso}
\begin{itemize}
\item {Grp. gram.:m.  e  adj.}
\end{itemize}
\begin{itemize}
\item {Proveniência:(Lat. \textunderscore studiosus\textunderscore )}
\end{itemize}
O que se applica ao estudo.
Aquelle que gosta de estudar.
\section{Estudo}
\begin{itemize}
\item {Grp. gram.:m.}
\end{itemize}
\begin{itemize}
\item {Grp. gram.:Pl.}
\end{itemize}
\begin{itemize}
\item {Proveniência:(Lat. \textunderscore studium\textunderscore )}
\end{itemize}
Acto de estudar.
Aquillo que se estuda.
Sala, onde se estudam ou se professam bellas-artes.
Sciência ou conhecimentos, que se adquirem, estudando.
Preparação.
Preliminar.
Investigação sôbre assumpto especial.
Composição musical, para exercicio de quem apprende música.
Esbôço.
Ensaio.
Modêlo para ensino.
Dissimulação; disfarce.
Cuidado, attenção.
Curso escolar.
Aulas.
\section{Estufa}
\begin{itemize}
\item {Grp. gram.:f.}
\end{itemize}
\begin{itemize}
\item {Utilização:Fig.}
\end{itemize}
\begin{itemize}
\item {Proveniência:(Do b. lat. \textunderscore stuba\textunderscore )}
\end{itemize}
Braseira, em fórma de caixa, para aquecer as casas.
Forno de fogão.
Galeria envidraçada, em que a temperatura se eleva artificialmente para cultura de plantas exóticas.
Qualquer recinto fechado, em que se eleva a temperatura artificialmente, para enxugar roupa ou para outros fins.
Casa ou quarto fechado e muito quente.
Pequena carruagem antiga. Cf. \textunderscore Diário de Not.\textunderscore , de 5-IX-900.
\section{Estufadeira}
\begin{itemize}
\item {Grp. gram.:f.}
\end{itemize}
\begin{itemize}
\item {Proveniência:(De \textunderscore estufar\textunderscore )}
\end{itemize}
Vaso, em que se estufa carne.
\section{Estufado}
\begin{itemize}
\item {Grp. gram.:m.}
\end{itemize}
\begin{itemize}
\item {Proveniência:(De \textunderscore estufar\textunderscore )}
\end{itemize}
Guisado de carne estufada.
\section{Estufagem}
\begin{itemize}
\item {Grp. gram.:f.}
\end{itemize}
Acto ou effeito de estufar.
\section{Estufar}
\begin{itemize}
\item {Grp. gram.:v. t.}
\end{itemize}
Meter, secar ou aquecer em estufa.
Aquecer artificialmente.
Guisar em vaso fechado.
\section{Estufeiro}
\begin{itemize}
\item {Grp. gram.:m.}
\end{itemize}
Aquelle que faz estufas.
\section{Estufilha}
\begin{itemize}
\item {Grp. gram.:f.}
\end{itemize}
\begin{itemize}
\item {Proveniência:(De \textunderscore estufa\textunderscore )}
\end{itemize}
Càrcere estreito, abafado.
\section{Estufim}
\begin{itemize}
\item {Grp. gram.:m.}
\end{itemize}
\begin{itemize}
\item {Proveniência:(De \textunderscore estufa\textunderscore )}
\end{itemize}
Manga de vidro ou caixilho envidraçado, com que se resguardam as plantas do ambiente frio.
Redoma.
\section{Estugar}
\begin{itemize}
\item {Grp. gram.:v. t.}
\end{itemize}
Apressar ou aligeirar (o passo).
(Corr. de \textunderscore instigar\textunderscore ?)
\section{Estuigar}
\begin{itemize}
\item {Grp. gram.:v. t.}
\end{itemize}
\begin{itemize}
\item {Utilização:Ant.}
\end{itemize}
O mesmo que \textunderscore estugar\textunderscore .
\section{Estulizar}
\begin{itemize}
\item {Grp. gram.:v. t.}
\end{itemize}
\begin{itemize}
\item {Utilização:Prov.}
\end{itemize}
\begin{itemize}
\item {Utilização:trasm.}
\end{itemize}
Imaginar, inventar.
\section{Estultamente}
\begin{itemize}
\item {Grp. gram.:adv.}
\end{itemize}
De modo estulto.
\section{Estultice}
\begin{itemize}
\item {Grp. gram.:f.}
\end{itemize}
O mesmo que \textunderscore estultícia\textunderscore .
\section{Estultícia}
\begin{itemize}
\item {Grp. gram.:f.}
\end{itemize}
\begin{itemize}
\item {Proveniência:(Lat. \textunderscore stultitia\textunderscore )}
\end{itemize}
Qualidade daquelle ou daquillo que é estulto.
\section{Estultificação}
\begin{itemize}
\item {Grp. gram.:f.}
\end{itemize}
Acto de estultificar.
\section{Estultificar}
\begin{itemize}
\item {Grp. gram.:v. t.}
\end{itemize}
\begin{itemize}
\item {Proveniência:(Do lat. \textunderscore stultus\textunderscore  + \textunderscore facere\textunderscore )}
\end{itemize}
Tornar estulto.
\section{Estultilóquio}
\begin{itemize}
\item {Grp. gram.:m.}
\end{itemize}
\begin{itemize}
\item {Proveniência:(Lat. \textunderscore stultiloquium\textunderscore )}
\end{itemize}
Palavras estultas.
Estultícia.
\section{Estulto}
\begin{itemize}
\item {Grp. gram.:adj.}
\end{itemize}
\begin{itemize}
\item {Proveniência:(Lat. \textunderscore stultus\textunderscore )}
\end{itemize}
Em que não há discernimento ou bom senso: \textunderscore pretensões estultas\textunderscore .
Tolo.
Imbecil; inepto.
\section{Estumar}
\begin{itemize}
\item {Grp. gram.:v. t.}
\end{itemize}
\begin{itemize}
\item {Utilização:Bras}
\end{itemize}
Açular ou estimular (cães).
(Talvez contr. de \textunderscore estimular\textunderscore )
\section{Estuoso}
\begin{itemize}
\item {Grp. gram.:adj.}
\end{itemize}
\begin{itemize}
\item {Proveniência:(Lat. \textunderscore aestuosus\textunderscore )}
\end{itemize}
Muito quente.
Tempestuoso.
Fervente.
Que está em cachão.
\section{Estupefacção}
\begin{itemize}
\item {Grp. gram.:f.}
\end{itemize}
\begin{itemize}
\item {Proveniência:(Lat. \textunderscore stupefactio\textunderscore )}
\end{itemize}
Estado de quem se acha estupefacto.
\section{Estupefaciente}
\begin{itemize}
\item {Grp. gram.:adj.}
\end{itemize}
\begin{itemize}
\item {Grp. gram.:M.}
\end{itemize}
\begin{itemize}
\item {Proveniência:(De \textunderscore estupefacto\textunderscore )}
\end{itemize}
Que produz estupefacção.
Que entorpece.
Medicamento entorpecedor; medicamento, que adormenta.
\section{Estupefactivo}
\begin{itemize}
\item {Grp. gram.:adj.}
\end{itemize}
\begin{itemize}
\item {Grp. gram.:M.}
\end{itemize}
\begin{itemize}
\item {Proveniência:(De \textunderscore estupefacto\textunderscore )}
\end{itemize}
Que produz estupefacção.
Que entorpece.
Medicamento entorpecedor; medicamento, que adormenta.
\section{Estupefacto}
\begin{itemize}
\item {Grp. gram.:adj.}
\end{itemize}
\begin{itemize}
\item {Proveniência:(Lat. \textunderscore stupefactus\textunderscore )}
\end{itemize}
Entorpecido, que não tem sensibilidade.
Assombrado, muito admirado.
Espantado.
\section{Estupefactor}
\begin{itemize}
\item {Grp. gram.:adj.}
\end{itemize}
Que estupefaz.
\section{Estupefazer}
\begin{itemize}
\item {Grp. gram.:v. t.}
\end{itemize}
\begin{itemize}
\item {Proveniência:(Do lat. \textunderscore stupefacere\textunderscore )}
\end{itemize}
Causar estupefacção a; espantar, assombrar. Cf. Filinto, XVII, 92.
\section{Estupeficador}
\begin{itemize}
\item {Grp. gram.:adj.}
\end{itemize}
Que estupefica.
\section{Estupeficante}
\begin{itemize}
\item {Grp. gram.:adj.}
\end{itemize}
\begin{itemize}
\item {Proveniência:(Do lat. \textunderscore stupefaciens\textunderscore )}
\end{itemize}
Que estupefica.
\section{Estupeficar}
\begin{itemize}
\item {Grp. gram.:v. t.}
\end{itemize}
\begin{itemize}
\item {Proveniência:(Lat. \textunderscore stupefacere\textunderscore )}
\end{itemize}
Entorpecer, fazer perder os sentidos, tirar a sensibilidade a.
Maravilhar.
Assombrar.
Causar espanto a.
\section{Estupendamente}
\begin{itemize}
\item {Grp. gram.:adv.}
\end{itemize}
De modo estupendo.
\section{Estupendo}
\begin{itemize}
\item {Grp. gram.:adj.}
\end{itemize}
\begin{itemize}
\item {Proveniência:(Lat. \textunderscore stupendus\textunderscore )}
\end{itemize}
Admirável.
Espantoso; monstruoso, extraordinário.
\section{Estupidamente}
\begin{itemize}
\item {Grp. gram.:adv.}
\end{itemize}
De modo estúpido.
\section{Estupidarrão}
\begin{itemize}
\item {Grp. gram.:m.}
\end{itemize}
\begin{itemize}
\item {Utilização:Pop.}
\end{itemize}
Homem muito estúpido.
\section{Estupidecer}
\begin{itemize}
\item {Grp. gram.:v. t.}
\end{itemize}
Tornar estúpido; embrutecer.
\section{Estupidez}
\begin{itemize}
\item {Grp. gram.:f.}
\end{itemize}
Qualidade daquelle ou daquillo que é estúpido.
Palavra ou acção própria de gente estúpida.
\section{Estupidificar}
\begin{itemize}
\item {Grp. gram.:v. t.}
\end{itemize}
\begin{itemize}
\item {Proveniência:(Do lat. \textunderscore stupidus\textunderscore  + \textunderscore facere\textunderscore )}
\end{itemize}
Bestializar, tornar estúpido.
\section{Estúpido}
\begin{itemize}
\item {Grp. gram.:adj.}
\end{itemize}
\begin{itemize}
\item {Proveniência:(Lat. \textunderscore stupidus\textunderscore )}
\end{itemize}
Que tem intelligência escassa, ou pouco juízo.
Incapaz de comprehender qualquer coisa.
Entorpecido, paralysado.
Atacado de estupor.
Que revela pouco juizo ou é prova de pouco juizo, de falta de tino: \textunderscore ideia estúpida\textunderscore .
\section{Estupor}
\begin{itemize}
\item {Grp. gram.:m.}
\end{itemize}
\begin{itemize}
\item {Utilização:Fig.}
\end{itemize}
\begin{itemize}
\item {Utilização:Pop.}
\end{itemize}
\begin{itemize}
\item {Proveniência:(Lat. \textunderscore stupor\textunderscore )}
\end{itemize}
Afroixamento das faculdades intellectuaes, por doença.
Paralysia.
Hemiplegia.
Immobilidade, produzida por surpresa.
Pessôa, de más qualidades ou muito feia.
\section{Estuporado}
\begin{itemize}
\item {Grp. gram.:adj.}
\end{itemize}
\begin{itemize}
\item {Utilização:Pop.}
\end{itemize}
\begin{itemize}
\item {Proveniência:(De \textunderscore estuporar-se\textunderscore )}
\end{itemize}
Que soffreu estupor.
Que tem más qualidades.
\section{Estuporante}
\begin{itemize}
\item {Grp. gram.:adj.}
\end{itemize}
\begin{itemize}
\item {Utilização:bras}
\end{itemize}
\begin{itemize}
\item {Utilização:Neol.}
\end{itemize}
O mesmo que \textunderscore estupendo\textunderscore . Cf. \textunderscore Jornal do Comm.\textunderscore , do Rio, de 1-I-905.
\section{Estuporar-se}
\begin{itemize}
\item {Grp. gram.:v. p.}
\end{itemize}
\begin{itemize}
\item {Utilização:Pop.}
\end{itemize}
Tornar-se estupor.
Estragar-se.
Tornar-se desprezivel, abjecto.
\section{Estuprar}
\begin{itemize}
\item {Grp. gram.:v. t.}
\end{itemize}
\begin{itemize}
\item {Proveniência:(Lat. \textunderscore stuprare\textunderscore )}
\end{itemize}
Commeter estupro contra.
Deshonestar.
Violar (mulher honesta ou donzella).
Deshonrar.
\section{Estupro}
\begin{itemize}
\item {Grp. gram.:m.}
\end{itemize}
\begin{itemize}
\item {Proveniência:(Lat. \textunderscore stuprum\textunderscore )}
\end{itemize}
Attentado contra o pudor de uma mulher.
Cóito forçado.
Desfloramento de virgem.
\section{Estuque}
\begin{itemize}
\item {Grp. gram.:m.}
\end{itemize}
\begin{itemize}
\item {Proveniência:(It. \textunderscore stucco\textunderscore )}
\end{itemize}
Mármore pulverizado, misturado com cal, gêsso, etc.
Revestimento ou ornamentos, feitos com aquella substância.
\section{Estúrdia}
\begin{itemize}
\item {Grp. gram.:f.}
\end{itemize}
\begin{itemize}
\item {Utilização:Prov.}
\end{itemize}
\begin{itemize}
\item {Utilização:minh.}
\end{itemize}
\begin{itemize}
\item {Proveniência:(De \textunderscore estúrdio\textunderscore )}
\end{itemize}
Extravagância; estroinice; travessura.
Dança de rapazes, ao som da viola.
\section{Esturdiar}
\begin{itemize}
\item {Grp. gram.:v. i.}
\end{itemize}
\begin{itemize}
\item {Proveniência:(De \textunderscore estúrdia\textunderscore )}
\end{itemize}
Fazer estúrdias.
Sêr estúrdio.
\section{Estúrdio}
\begin{itemize}
\item {Grp. gram.:m.  e  adj.}
\end{itemize}
\begin{itemize}
\item {Grp. gram.:Adj.}
\end{itemize}
\begin{itemize}
\item {Utilização:Bras}
\end{itemize}
Extravagante.
Estroina.
Valdevinos.
Indivíduo leviano ou travesso.
Esquisito, (falando-se de coisas): \textunderscore vi uma luz estúrdia na janela\textunderscore .
(Cp. it. \textunderscore stordito\textunderscore )
\section{Esturgião}
\begin{itemize}
\item {Grp. gram.:m.}
\end{itemize}
O mesmo que \textunderscore esturjão\textunderscore .
\section{Estúria}
\begin{itemize}
\item {Grp. gram.:f.}
\end{itemize}
(?):«\textunderscore ...doze remos de freixo desturias...\textunderscore »De um \textunderscore Ms.\textunderscore  do século XVII, em poder de Sousa Viterbo. É possível que o texto se refira a \textunderscore freixo das Astúrias\textunderscore . Em refôrço desta supposição, notarei que a nobiliarchia port. comprehende o appellido \textunderscore Estúrias\textunderscore , corruptela de \textunderscore Astúrias\textunderscore .
\section{Esturião}
\begin{itemize}
\item {Grp. gram.:m.}
\end{itemize}
\begin{itemize}
\item {Proveniência:(Do ant. alt. al. \textunderscore sturio\textunderscore )}
\end{itemize}
Peixe, o mesmo que \textunderscore solho\textunderscore .
\section{Esturjão}
\begin{itemize}
\item {Grp. gram.:m.}
\end{itemize}
\begin{itemize}
\item {Proveniência:(Do ant. alt. al. \textunderscore sturio\textunderscore )}
\end{itemize}
Peixe, o mesmo que \textunderscore solho\textunderscore .
\section{Esturónios}
\begin{itemize}
\item {Grp. gram.:m.}
\end{itemize}
Família de peixes, que têm por typo o esturjão.
\section{Esturrado}
\begin{itemize}
\item {Grp. gram.:adj.}
\end{itemize}
\begin{itemize}
\item {Utilização:Fig.}
\end{itemize}
\begin{itemize}
\item {Grp. gram.:M.}
\end{itemize}
Muito torrado.
Resequido.
Quási queimado.
Exaltado.
Intransigente.
Fanático.
Partidário intransigente, ferrenho.
\section{Esturrar}
\begin{itemize}
\item {Grp. gram.:v. t.}
\end{itemize}
\begin{itemize}
\item {Grp. gram.:V. p.}
\end{itemize}
\begin{itemize}
\item {Utilização:Fig.}
\end{itemize}
Torrar muito.
Queimar quási.
Tomar esturro.
Torrar-se muito.
Irritar-se; exaltar-se.
(Por \textunderscore estorrar\textunderscore , de \textunderscore torrar\textunderscore )
\section{Esturrice}
\begin{itemize}
\item {Grp. gram.:f.}
\end{itemize}
\begin{itemize}
\item {Proveniência:(De \textunderscore esturrar\textunderscore )}
\end{itemize}
Qualidade de quem se esturra ou se zanga. Cf. Rui Barb., \textunderscore Réplica\textunderscore , 100.
\section{Esturrinho}
\begin{itemize}
\item {Grp. gram.:m.}
\end{itemize}
\begin{itemize}
\item {Proveniência:(De \textunderscore esturro\textunderscore )}
\end{itemize}
Tabaco para cheirar, muito escuro, muito torrado.
\section{Esturro}
\begin{itemize}
\item {Grp. gram.:m.}
\end{itemize}
\begin{itemize}
\item {Proveniência:(De \textunderscore esturrar\textunderscore )}
\end{itemize}
Estado de coisa esturrada.
Torrefacção.
Esturrinho.
\section{Esturvinhado}
\begin{itemize}
\item {Grp. gram.:adj.}
\end{itemize}
\begin{itemize}
\item {Utilização:Pop.}
\end{itemize}
\begin{itemize}
\item {Proveniência:(Do rad. de \textunderscore turvar\textunderscore )}
\end{itemize}
Atordoado.
Adoidado.
\section{Estygial}
\begin{itemize}
\item {Grp. gram.:adj.}
\end{itemize}
\begin{itemize}
\item {Proveniência:(Lat. \textunderscore stygialis\textunderscore )}
\end{itemize}
O mesmo que \textunderscore estýgio\textunderscore .
\section{Estýgio}
\begin{itemize}
\item {Grp. gram.:adj.}
\end{itemize}
\begin{itemize}
\item {Proveniência:(Lat. \textunderscore stygius\textunderscore )}
\end{itemize}
Relativo ao rio infernal Estyge.
\section{Estypticina}
\begin{itemize}
\item {Grp. gram.:f.}
\end{itemize}
\begin{itemize}
\item {Utilização:Pharm.}
\end{itemize}
\begin{itemize}
\item {Proveniência:(De \textunderscore estýptico\textunderscore )}
\end{itemize}
Medicamento, que é um clorhydrato de cotarnina e que se applica contra as hemorrhagias uterinas.
\section{Estýptico}
\begin{itemize}
\item {Grp. gram.:adj.}
\end{itemize}
(V.estítico)
\section{Estyráceas}
\begin{itemize}
\item {Grp. gram.:f. pl.}
\end{itemize}
\begin{itemize}
\item {Proveniência:(Do gr. \textunderscore sturax\textunderscore )}
\end{itemize}
Família de plantas, que têm por typo o estoraque.
\section{Ésula}
\begin{itemize}
\item {Grp. gram.:f.}
\end{itemize}
\begin{itemize}
\item {Proveniência:(Do lat. \textunderscore esula\textunderscore )}
\end{itemize}
Planta medicinal, euphorbiácea.
\section{Esurino}
\begin{itemize}
\item {Grp. gram.:adj.}
\end{itemize}
\begin{itemize}
\item {Proveniência:(Do rad. do lat. \textunderscore esurire\textunderscore )}
\end{itemize}
Que excita a fome.
\section{Esvaecer}
\begin{itemize}
\item {fónica:va-e}
\end{itemize}
\begin{itemize}
\item {Grp. gram.:v. t.}
\end{itemize}
\begin{itemize}
\item {Grp. gram.:V. i.  e  p.}
\end{itemize}
\begin{itemize}
\item {Proveniência:(Do lat. \textunderscore vanescere\textunderscore )}
\end{itemize}
Evaporar.
Dissipar: \textunderscore o vento esvaéce o fumo\textunderscore .
Dissipar-se: \textunderscore esvaeceram-se as sombras\textunderscore .
Esvair-se; desmaiar.
\section{Esvaecimento}
\begin{itemize}
\item {fónica:va-e}
\end{itemize}
\begin{itemize}
\item {Grp. gram.:m.}
\end{itemize}
Acto ou effeito de esvaecer.
Confusão nas ideias por muito meditar. Cp. Bernárdez, \textunderscore Luz e Calor\textunderscore , 146.
\section{Esvaimento}
\begin{itemize}
\item {fónica:va-i}
\end{itemize}
\begin{itemize}
\item {Grp. gram.:m.}
\end{itemize}
Acto ou effeito de esvair.
\section{Esvair}
\begin{itemize}
\item {Grp. gram.:v. t.}
\end{itemize}
\begin{itemize}
\item {Grp. gram.:V. p.}
\end{itemize}
\begin{itemize}
\item {Proveniência:(Do lat. \textunderscore vanus\textunderscore )}
\end{itemize}
Esvaecer.
Dissipar-se; esgotar-se; desapparecer.
Desmaiar.
Têr tonturas.
Desbotar.
\section{Esvalijar}
\textunderscore v. t.\textunderscore  (e der.)
O mesmo que \textunderscore desvalijar\textunderscore , etc. Cp. Vieira, III, carta 118.
\section{Esvalteiros}
\begin{itemize}
\item {Grp. gram.:m. pl.}
\end{itemize}
\begin{itemize}
\item {Utilização:Náut.}
\end{itemize}
Paus, a que se ligam as escotas da gávea.
\section{Esvanecente}
\begin{itemize}
\item {Grp. gram.:adj.}
\end{itemize}
Que esvanece.
\section{Esvanecer}
\begin{itemize}
\item {Proveniência:(Do lat. \textunderscore vanescere\textunderscore )}
\end{itemize}
\textunderscore v. t.\textunderscore  (e der.)
O mesmo que \textunderscore esvaecer\textunderscore , etc.
Fazer desmaiar. Cf. Filinto, XXI, 65.
\section{Esvão}
\begin{itemize}
\item {Grp. gram.:m.}
\end{itemize}
O mesmo que \textunderscore desvão\textunderscore .
\section{Esvaziamento}
\begin{itemize}
\item {Grp. gram.:m.}
\end{itemize}
Acto ou effeito de esvaziar.
\section{Esvaziar}
\begin{itemize}
\item {Grp. gram.:v. t.}
\end{itemize}
\begin{itemize}
\item {Proveniência:(De \textunderscore vazio\textunderscore )}
\end{itemize}
Tornar vazio.
Despejar; esgotar: \textunderscore esvaziar um frasco\textunderscore .
\section{Esventar}
\begin{itemize}
\item {Grp. gram.:v. t.}
\end{itemize}
\begin{itemize}
\item {Proveniência:(It. \textunderscore sventare\textunderscore )}
\end{itemize}
Tirar a humidade a (peças de artilharia).
\section{Esverdados}
\begin{itemize}
\item {Grp. gram.:m. pl.}
\end{itemize}
\begin{itemize}
\item {Utilização:Ant.}
\end{itemize}
\begin{itemize}
\item {Proveniência:(De \textunderscore verde\textunderscore )}
\end{itemize}
Frutos e hortaliças, de que se pagavam dizimos.
\section{Esverdaduras}
\begin{itemize}
\item {Grp. gram.:f. pl.}
\end{itemize}
\begin{itemize}
\item {Utilização:Ant.}
\end{itemize}
O mesmo que \textunderscore esverdados\textunderscore .
\section{Esverdeado}
\begin{itemize}
\item {Grp. gram.:adj.}
\end{itemize}
\begin{itemize}
\item {Proveniência:(De \textunderscore esverdear\textunderscore )}
\end{itemize}
Que tem côr tirante a verde.
Mesclado de verde.
\section{Esverdear}
\begin{itemize}
\item {Grp. gram.:v. t.}
\end{itemize}
\begin{itemize}
\item {Utilização:Prov.}
\end{itemize}
\begin{itemize}
\item {Utilização:alg.}
\end{itemize}
\begin{itemize}
\item {Utilização:Prov.}
\end{itemize}
\begin{itemize}
\item {Utilização:alg.}
\end{itemize}
\begin{itemize}
\item {Grp. gram.:V. i.}
\end{itemize}
Tornar verde.
Dar côr verde ou tirante a verde a.
Separar (a uva que vai para o lagar) da que é verde.
Separar do figo pincre, nas esteiras, (o que é maduro).
Tornar-se verde ou tirante a verde.
\section{Esverdinhado}
\begin{itemize}
\item {Grp. gram.:adj.}
\end{itemize}
\begin{itemize}
\item {Proveniência:(De \textunderscore esverdinhar\textunderscore )}
\end{itemize}
Que tem côr verde-clara.
\section{Esverdinhar}
\begin{itemize}
\item {Grp. gram.:v. t.}
\end{itemize}
Dar côr pouco verde ou verde-clara a.
\section{Esvidar}
\textunderscore v. t.\textunderscore  (e der.)
O mesmo que \textunderscore esvidigar\textunderscore .
\section{Esvidigador}
\begin{itemize}
\item {Grp. gram.:adj.}
\end{itemize}
\begin{itemize}
\item {Grp. gram.:M.}
\end{itemize}
Que esvidiga.
Aquelle que esvidiga.
\section{Esvidigar}
\begin{itemize}
\item {Grp. gram.:v. t.}
\end{itemize}
\begin{itemize}
\item {Proveniência:(De \textunderscore vide\textunderscore )}
\end{itemize}
Limpar das vides podadas (a vinha).
\section{Esvinhar}
\begin{itemize}
\item {Grp. gram.:v. i.}
\end{itemize}
\begin{itemize}
\item {Utilização:Agr.}
\end{itemize}
\begin{itemize}
\item {Grp. gram.:V. t.}
\end{itemize}
Cair por doença (a uva).
Fazer cair (os bagos de uvas). Cf. Castilho, \textunderscore Fastos\textunderscore , II, 568.
\section{Esviscerar}
\begin{itemize}
\item {Grp. gram.:v. t.}
\end{itemize}
\begin{itemize}
\item {Utilização:Fig.}
\end{itemize}
Tirar as vísceras a; estripar.
Tornar cruel.
\section{Esvoaçante}
\begin{itemize}
\item {Grp. gram.:adj.}
\end{itemize}
Que esvoaça.
\section{Esvoaçar}
\begin{itemize}
\item {Grp. gram.:v. i.}
\end{itemize}
\begin{itemize}
\item {Utilização:Fig.}
\end{itemize}
\begin{itemize}
\item {Grp. gram.:V. p.}
\end{itemize}
\begin{itemize}
\item {Proveniência:(De \textunderscore vôo\textunderscore )}
\end{itemize}
Agitar as asas para erguer vôo.
Adejar; voejar.
Volutear.
Desfraldar-se; fluctuar: \textunderscore esvoaçavam as bandeiras...\textunderscore  Agitar-se.
O mesmo que \textunderscore esvoaçar\textunderscore :«\textunderscore ...os gaios, esvoaçando-se escorraçados...\textunderscore »Camillo, \textunderscore Volcões\textunderscore , 109. Cf. Castilho, \textunderscore Fausto\textunderscore , 285; Camillo, \textunderscore Brasileira\textunderscore , 155.
\section{Esvurmar}
\begin{itemize}
\item {Grp. gram.:v. t.}
\end{itemize}
\begin{itemize}
\item {Utilização:Fig.}
\end{itemize}
\begin{itemize}
\item {Proveniência:(De \textunderscore vurmo\textunderscore )}
\end{itemize}
Tirar ou espremer o pus a.
Descobrir ou patentear, criticando (um defeito, uma paixão): \textunderscore vamos esvurmar a sordidez daquelle homem\textunderscore .
\section{Ésypo}
\begin{itemize}
\item {Grp. gram.:m.}
\end{itemize}
\begin{itemize}
\item {Proveniência:(Do gr. \textunderscore oisupe\textunderscore )}
\end{itemize}
Suarda ou substância gordurosa da lan das ovelhas.
Cosmético, feito com aquella substância.
\section{...êta}
\begin{itemize}
\item {Grp. gram.:suf. f.}
\end{itemize}
(designativo de deminuição: \textunderscore pandeireta\textunderscore , \textunderscore corneta\textunderscore , etc.)
\section{Etá}
\begin{itemize}
\item {Grp. gram.:m.}
\end{itemize}
\begin{itemize}
\item {Utilização:Bras}
\end{itemize}
Árvore fructifera, espécie de oiti.
\section{Eta-mundo!}
\begin{itemize}
\item {Grp. gram.:interj.}
\end{itemize}
\begin{itemize}
\item {Utilização:Bras. de Minas}
\end{itemize}
(Para designar satisfação: \textunderscore eta-mundo! seja bem apparecido!\textunderscore )
\section{Etão}
\begin{itemize}
\item {Grp. gram.:m.}
\end{itemize}
\begin{itemize}
\item {Utilização:Bras}
\end{itemize}
O mesmo que \textunderscore etá\textunderscore .
\section{Etapa}
\begin{itemize}
\item {Grp. gram.:f.}
\end{itemize}
\begin{itemize}
\item {Proveniência:(Fr. \textunderscore étape\textunderscore )}
\end{itemize}
Ração diária de comida e bebida dos soldados em campanha ou em marcha.
Cada uma das paragens ou bivaques das tropas em marcha.
\section{Etas}
\begin{itemize}
\item {Grp. gram.:m. pl.}
\end{itemize}
Casta desprezível, entre os Japoneses, a qual vive em bairros separados, tem templos seus, não celebra casamentos com outras castas, etc. Cf. V. de Moraes, \textunderscore Dai-Nippon\textunderscore , 193.
\section{Etc.}
\textunderscore abrev.\textunderscore  de \textunderscore et-caetera\textunderscore .
\section{Etal}
\begin{itemize}
\item {Grp. gram.:m.}
\end{itemize}
\begin{itemize}
\item {Proveniência:(De \textunderscore éter\textunderscore  + \textunderscore álcool\textunderscore )}
\end{itemize}
Substância gorda, de composição análoga á do éter e do álcool.
\section{Etálico}
\begin{itemize}
\item {Grp. gram.:adj.}
\end{itemize}
Relativo a etal.
\section{Etano}
\begin{itemize}
\item {Grp. gram.:m.}
\end{itemize}
\begin{itemize}
\item {Utilização:Chím.}
\end{itemize}
Variedade de carboneto do grupo formênico.
\section{Et-caetera}
\begin{itemize}
\item {fónica:ed'-cétera}
\end{itemize}
\begin{itemize}
\item {Grp. gram.:loc. adv.}
\end{itemize}
Assim por deante; e ainda outras coisas; afóra o mais.
(Loc. lat.)
\section{Éte!}
\begin{itemize}
\item {Grp. gram.:interj.}
\end{itemize}
\begin{itemize}
\item {Utilização:Prov.}
\end{itemize}
\begin{itemize}
\item {Utilização:trasm.}
\end{itemize}
Voz do boieiro, quando tange os bois.
\section{...ête}
\begin{itemize}
\item {Grp. gram.:suf. m.}
\end{itemize}
(designativo de deminuição: \textunderscore clarete\textunderscore , \textunderscore collete\textunderscore , etc.)
\section{Étego}
\begin{itemize}
\item {Grp. gram.:adj.}
\end{itemize}
\begin{itemize}
\item {Utilização:Ant.}
\end{itemize}
O mesmo que \textunderscore héctico\textunderscore . Cf. \textunderscore Eufrosina\textunderscore , 126.
\section{Eter}
\begin{itemize}
\item {Grp. gram.:m.}
\end{itemize}
\begin{itemize}
\item {Utilização:Ext.}
\end{itemize}
\begin{itemize}
\item {Grp. gram.:Pl.}
\end{itemize}
\begin{itemize}
\item {Utilização:Chím.}
\end{itemize}
\begin{itemize}
\item {Proveniência:(Do lat. \textunderscore aether\textunderscore )}
\end{itemize}
Ar puro e rarefeito das regiões superiores da atmosfera.
Fluido hipotético, com que alguns físicos explicam os fenómenos da luz e do calor.
Líquido muito volátil e inflamável, produzido pela destilação de um ácido misturado com álcool.
Atmosfera.
Espaço celeste.
Corpos, derivados da combinação dos álcooes com os ácidos ou com outros álcooes, havendo eliminação de água.
\section{Eterato}
\begin{itemize}
\item {Grp. gram.:m.}
\end{itemize}
\begin{itemize}
\item {Proveniência:(De \textunderscore éter\textunderscore )}
\end{itemize}
Sal, resultante da combinação do ácido etérico com uma base.
\section{Etéreo}
\begin{itemize}
\item {Grp. gram.:adj.}
\end{itemize}
\begin{itemize}
\item {Proveniência:(Lat. \textunderscore aethereus\textunderscore )}
\end{itemize}
Relativo ao éter: \textunderscore regiões etéreas\textunderscore .
Que tem a natureza do éter.
\section{Etérico}
\begin{itemize}
\item {Grp. gram.:adj.}
\end{itemize}
\begin{itemize}
\item {Proveniência:(De \textunderscore éter\textunderscore )}
\end{itemize}
Diz-se de um ácido produzido pela combustão do álcool.
\section{Eterificação}
\begin{itemize}
\item {Grp. gram.:f.}
\end{itemize}
Acto de eterificar.
\section{Eterificar}
\begin{itemize}
\item {Grp. gram.:v. t.}
\end{itemize}
\begin{itemize}
\item {Proveniência:(Do lat. \textunderscore aether\textunderscore  + \textunderscore facere\textunderscore )}
\end{itemize}
Converter em éter.
\section{Eterismo}
\begin{itemize}
\item {Grp. gram.:m.}
\end{itemize}
\begin{itemize}
\item {Proveniência:(De \textunderscore éter\textunderscore )}
\end{itemize}
Insensibilidade, resultante da aplicação do éter ou de outra causa.
\section{Eterizador}
\begin{itemize}
\item {Grp. gram.:m.}
\end{itemize}
Instrumento, para eterizar.
\section{Eterizar}
\begin{itemize}
\item {Grp. gram.:v. t.}
\end{itemize}
Misturar com éter.
Tirar a sensibilidade a, por meio do éter.
\section{Eternal}
\begin{itemize}
\item {Grp. gram.:adj.}
\end{itemize}
\begin{itemize}
\item {Proveniência:(Lat. \textunderscore aeternalis\textunderscore )}
\end{itemize}
(V.eterno)
\section{Eternalmente}
\begin{itemize}
\item {Grp. gram.:adv.}
\end{itemize}
(V.eternamente)
\section{Eternamente}
\begin{itemize}
\item {Grp. gram.:adv.}
\end{itemize}
\begin{itemize}
\item {Proveniência:(De \textunderscore eterno\textunderscore )}
\end{itemize}
Para sempre; sempre.
Durante a eternidade.
\section{Eternar}
\begin{itemize}
\item {Grp. gram.:v. t.}
\end{itemize}
\begin{itemize}
\item {Proveniência:(Lat. \textunderscore aeternare\textunderscore )}
\end{itemize}
O mesmo que \textunderscore eternizar\textunderscore .
\section{Eternidade}
\begin{itemize}
\item {Grp. gram.:f.}
\end{itemize}
\begin{itemize}
\item {Proveniência:(Lat. \textunderscore aeternitas\textunderscore )}
\end{itemize}
Qualidade daquillo que é eterno.
Immortalidade.
A vida eterna.
Duração longa.
\section{Eternífluo}
\begin{itemize}
\item {Grp. gram.:adj.}
\end{itemize}
\begin{itemize}
\item {Utilização:Poét.}
\end{itemize}
\begin{itemize}
\item {Proveniência:(Do lat. \textunderscore aeternus\textunderscore  + \textunderscore fluere\textunderscore )}
\end{itemize}
Que flue ou corre sem fim. Cf. Filinto, VIII, 210.
\section{Eternizar}
\begin{itemize}
\item {Grp. gram.:v. t.}
\end{itemize}
\begin{itemize}
\item {Utilização:Fig.}
\end{itemize}
Tornar eterno.
Prolongar indefinidamente.
\section{Eterno}
\begin{itemize}
\item {Grp. gram.:adj.}
\end{itemize}
\begin{itemize}
\item {Proveniência:(Lat. \textunderscore aeternus\textunderscore )}
\end{itemize}
Que não tem princípio nem terá fim.
Que não tem fim; que dura sempre: \textunderscore as penas eternas\textunderscore .
Inalterável.
Que tem duração indefinida.
Que não se sabe quando finda.
Enorme, desmedido.
\section{Eterograma}
\begin{itemize}
\item {Grp. gram.:m.}
\end{itemize}
\begin{itemize}
\item {Utilização:Neol.}
\end{itemize}
\begin{itemize}
\item {Proveniência:(Do gr. \textunderscore aither\textunderscore  + \textunderscore gramma\textunderscore )}
\end{itemize}
Comunicação, trasm.tida pela telegrafia sem fios.
\section{Eterolato}
\begin{itemize}
\item {Grp. gram.:m.}
\end{itemize}
\begin{itemize}
\item {Proveniência:(Do rad. de \textunderscore éter\textunderscore )}
\end{itemize}
Producto medicamentoso, resultante da destilação do éter sulfúrico sôbre uma substância aromática.
\section{Eterolatura}
\begin{itemize}
\item {Grp. gram.:f.}
\end{itemize}
\begin{itemize}
\item {Proveniência:(De \textunderscore eterolato\textunderscore )}
\end{itemize}
Tintura de éter.
\section{Eteróleo}
\begin{itemize}
\item {Grp. gram.:m.}
\end{itemize}
\begin{itemize}
\item {Proveniência:(De \textunderscore éter\textunderscore )}
\end{itemize}
Medicamento líquido, formado de éter, tendo em dissolução princípios medicamentosos.
\section{Eterólico}
\begin{itemize}
\item {Grp. gram.:adj.}
\end{itemize}
\begin{itemize}
\item {Proveniência:(De \textunderscore eteróleo\textunderscore )}
\end{itemize}
Que tem por excipiente o éter sulfúrico.
\section{Etésios}
\begin{itemize}
\item {Grp. gram.:adj. pl.}
\end{itemize}
\begin{itemize}
\item {Proveniência:(Do lat. \textunderscore etaesiae\textunderscore )}
\end{itemize}
Diz-se dos ventos do norte, que sopram ás vezes no Mediterrâneo, modificando calores do estio.
\section{Ethal}
\begin{itemize}
\item {Grp. gram.:m.}
\end{itemize}
\begin{itemize}
\item {Proveniência:(De \textunderscore éther\textunderscore  + \textunderscore álcool\textunderscore )}
\end{itemize}
Substância gorda, de composição análoga á do éther e do álcool.
\section{Ethálico}
\begin{itemize}
\item {Grp. gram.:adj.}
\end{itemize}
Relativo a ethal.
\section{Ethano}
\begin{itemize}
\item {Grp. gram.:m.}
\end{itemize}
\begin{itemize}
\item {Utilização:Chím.}
\end{itemize}
Variedade de carboneto do grupo formênico.
\section{Éther}
\begin{itemize}
\item {Grp. gram.:m.}
\end{itemize}
\begin{itemize}
\item {Utilização:Ext.}
\end{itemize}
\begin{itemize}
\item {Grp. gram.:Pl.}
\end{itemize}
\begin{itemize}
\item {Utilização:Chím.}
\end{itemize}
\begin{itemize}
\item {Proveniência:(Do lat. \textunderscore aether\textunderscore )}
\end{itemize}
Ar puro e rarefeito das regiões superiores da atmosphera.
Fluido hypothético, com que alguns phýsicos explicam os phenómenos da luz e do calor.
Líquido muito volátil e inflammável, produzido pela destillação de um ácido misturado com álcool.
Atmosphera.
Espaço celeste.
Corpos, derivados da combinação dos álcooes com os ácidos ou com outros álcooes, havendo eliminação de água.
\section{Etherato}
\begin{itemize}
\item {Grp. gram.:m.}
\end{itemize}
\begin{itemize}
\item {Proveniência:(De \textunderscore éther\textunderscore )}
\end{itemize}
Sal, resultante da combinação do ácido ethérico com uma base.
\section{Ethéreo}
\begin{itemize}
\item {Grp. gram.:adj.}
\end{itemize}
\begin{itemize}
\item {Proveniência:(Lat. \textunderscore aethereus\textunderscore )}
\end{itemize}
Relativo ao éther: \textunderscore regiões ethéreas\textunderscore .
Que tem a natureza do éther.
\section{Ethérico}
\begin{itemize}
\item {Grp. gram.:adj.}
\end{itemize}
\begin{itemize}
\item {Proveniência:(De \textunderscore éther\textunderscore )}
\end{itemize}
Diz-se de um ácido produzido pela combustão do álcool.
\section{Etherificação}
\begin{itemize}
\item {Grp. gram.:f.}
\end{itemize}
Acto de etherificar.
\section{Etherificar}
\begin{itemize}
\item {Grp. gram.:v. t.}
\end{itemize}
\begin{itemize}
\item {Proveniência:(Do lat. \textunderscore aether\textunderscore  + \textunderscore facere\textunderscore )}
\end{itemize}
Converter em éther.
\section{Etherismo}
\begin{itemize}
\item {Grp. gram.:m.}
\end{itemize}
\begin{itemize}
\item {Proveniência:(De \textunderscore éther\textunderscore )}
\end{itemize}
Insensibilidade, resultante da applicação do éther ou de outra causa.
\section{Etherizador}
\begin{itemize}
\item {Grp. gram.:m.}
\end{itemize}
Instrumento, para etherizar.
\section{Etherizar}
\begin{itemize}
\item {Grp. gram.:v. t.}
\end{itemize}
Misturar com éther.
Tirar a sensibilidade a, por meio do éther.
\section{Etherogramma}
\begin{itemize}
\item {Grp. gram.:m.}
\end{itemize}
\begin{itemize}
\item {Utilização:Neol.}
\end{itemize}
\begin{itemize}
\item {Proveniência:(Do gr. \textunderscore aither\textunderscore  + \textunderscore gramma\textunderscore )}
\end{itemize}
Communicação, trasm.ttida pela telegraphia sem fios.
\section{Etherolato}
\begin{itemize}
\item {Grp. gram.:m.}
\end{itemize}
\begin{itemize}
\item {Proveniência:(Do rad. de \textunderscore éther\textunderscore )}
\end{itemize}
Producto medicamentoso, resultante da destillação do éther sulfúrico sôbre uma substância aromática.
\section{Etherolatura}
\begin{itemize}
\item {Grp. gram.:f.}
\end{itemize}
\begin{itemize}
\item {Proveniência:(De \textunderscore etherolato\textunderscore )}
\end{itemize}
Tintura de éther.
\section{Etheróleo}
\begin{itemize}
\item {Grp. gram.:m.}
\end{itemize}
\begin{itemize}
\item {Proveniência:(De \textunderscore éther\textunderscore )}
\end{itemize}
Medicamento líquido, formado de éther, tendo em dissolução princípios medicamentosos.
\section{Etherólico}
\begin{itemize}
\item {Grp. gram.:adj.}
\end{itemize}
\begin{itemize}
\item {Proveniência:(De \textunderscore etheróleo\textunderscore )}
\end{itemize}
Que tem por excipiente o éther sulfúrico.
\section{Ethica}
\begin{itemize}
\item {Grp. gram.:f.}
\end{itemize}
\begin{itemize}
\item {Proveniência:(De \textunderscore éthico\textunderscore )}
\end{itemize}
Sciência da moral.
\section{Éthico}
\begin{itemize}
\item {Grp. gram.:adj.}
\end{itemize}
\begin{itemize}
\item {Proveniência:(Gr. \textunderscore ethikos\textunderscore )}
\end{itemize}
Relativo á éthica ou á moral.
\section{Ethionema}
\begin{itemize}
\item {Grp. gram.:m.}
\end{itemize}
Gênero de plantas crucíferas.
\section{Ethiónico}
\begin{itemize}
\item {Grp. gram.:adj.}
\end{itemize}
Diz-se de um ácido, que se obtém pela acção do ácido sulfúrico anhydro sôbre o álcool.
\section{Ethíope}
\begin{itemize}
\item {Grp. gram.:adj.}
\end{itemize}
\begin{itemize}
\item {Grp. gram.:M.}
\end{itemize}
\begin{itemize}
\item {Proveniência:(Do lat. \textunderscore aethiops\textunderscore )}
\end{itemize}
Relativo á Ethiópia ou aos seus habitantes.
Habitante da Ethiópia.
Designação ant. de certos óxydos e sulfuretos metállicos, por causa da côr escura.
\section{Ethiópico}
\begin{itemize}
\item {Grp. gram.:adj.}
\end{itemize}
\begin{itemize}
\item {Proveniência:(De \textunderscore Ethiópia\textunderscore , n. p.)}
\end{itemize}
Relativo aos Ethíopes.
\section{Ethiópio}
\begin{itemize}
\item {Grp. gram.:m.}
\end{itemize}
\begin{itemize}
\item {Grp. gram.:Pl.}
\end{itemize}
Língua da Ethiópia.
Habitantes da Ethiópia. Cf. Esmeraldo, c. 33.
\section{Ethiopisa}
\begin{itemize}
\item {Grp. gram.:f.}
\end{itemize}
\begin{itemize}
\item {Proveniência:(Do lat. \textunderscore aethiopissa\textunderscore )}
\end{itemize}
Mulher da Ethiópia.
\section{Ethmoidal}
\begin{itemize}
\item {Grp. gram.:adj.}
\end{itemize}
Relativo ao ethmoide.
\section{Ethmoide}
\begin{itemize}
\item {Grp. gram.:m.}
\end{itemize}
\begin{itemize}
\item {Utilização:Anat.}
\end{itemize}
\begin{itemize}
\item {Proveniência:(Do gr. \textunderscore ethmos\textunderscore  + \textunderscore eidos\textunderscore )}
\end{itemize}
Osso do crânio, que contribue para a formação das cavidades nasaes.
\section{Ethmoídeo}
\begin{itemize}
\item {Grp. gram.:adj.}
\end{itemize}
(V.ethmoidal)
\section{Ethna}
\begin{itemize}
\item {Grp. gram.:m.}
\end{itemize}
\begin{itemize}
\item {Utilização:Fig.}
\end{itemize}
\begin{itemize}
\item {Proveniência:(De \textunderscore Ethna\textunderscore , n. p.)}
\end{itemize}
Vulcão. Cf. Garrett, \textunderscore Port. na Bal.\textunderscore , 75.
\section{Ethnarcha}
\begin{itemize}
\item {fónica:ca}
\end{itemize}
\begin{itemize}
\item {Grp. gram.:f.}
\end{itemize}
\begin{itemize}
\item {Proveniência:(Gr. \textunderscore ethnarkes\textunderscore )}
\end{itemize}
Governador de província, na antiguidade.
\section{Ethnarchia}
\begin{itemize}
\item {fónica:qui}
\end{itemize}
\begin{itemize}
\item {Grp. gram.:f.}
\end{itemize}
\begin{itemize}
\item {Proveniência:(De \textunderscore ethnarcha\textunderscore )}
\end{itemize}
Dignidade do ethnarca.
Território governado por elle.
\section{Ethnárchico}
\begin{itemize}
\item {fónica:qui}
\end{itemize}
\begin{itemize}
\item {Grp. gram.:adj.}
\end{itemize}
Relativo a ethnarchia.
\section{Ethnicamente}
\begin{itemize}
\item {Grp. gram.:adv.}
\end{itemize}
Á maneira dos éthnicos.
\section{Ethnicismo}
\begin{itemize}
\item {Grp. gram.:m.}
\end{itemize}
\begin{itemize}
\item {Proveniência:(De \textunderscore éthnico\textunderscore )}
\end{itemize}
O mesmo que \textunderscore paganismo\textunderscore .
\section{Éthnico}
\begin{itemize}
\item {Grp. gram.:adj.}
\end{itemize}
\begin{itemize}
\item {Grp. gram.:M.}
\end{itemize}
\begin{itemize}
\item {Proveniência:(Gr. \textunderscore ethnikos\textunderscore )}
\end{itemize}
Que pertence ao paganismo.
Característico de um país: \textunderscore as condições éthnicas de Portugal\textunderscore .
Que designa habitantes de um país ou de uma região.
Idólatra.
\section{Ethnodiceia}
\begin{itemize}
\item {Grp. gram.:f.}
\end{itemize}
\begin{itemize}
\item {Proveniência:(Do gr. \textunderscore ethnos\textunderscore  + \textunderscore dike\textunderscore )}
\end{itemize}
Direito das gentes.
\section{Ethno-genealogia}
\begin{itemize}
\item {Grp. gram.:f.}
\end{itemize}
Genealogia dos povos.
\section{Ethnogenia}
\begin{itemize}
\item {Grp. gram.:f.}
\end{itemize}
\begin{itemize}
\item {Proveniência:(Do gr. \textunderscore ethnos\textunderscore  + \textunderscore gene\textunderscore )}
\end{itemize}
Sciência, que trata da origem dos povos.
\section{Ethnographia}
\begin{itemize}
\item {Grp. gram.:f.}
\end{itemize}
\begin{itemize}
\item {Proveniência:(Do gr. \textunderscore ethnos\textunderscore  + \textunderscore graphein\textunderscore )}
\end{itemize}
Sciência, que descreve os povos, sua raça, língua, religiões, etc.
\section{Ethnographicamente}
\begin{itemize}
\item {Grp. gram.:adv.}
\end{itemize}
Sob o ponto de vista ethnográphico.
\section{Ethnográphico}
\begin{itemize}
\item {Grp. gram.:adj.}
\end{itemize}
Relativo a ethnographia.
\section{Ethnógrapho}
\begin{itemize}
\item {Grp. gram.:m.}
\end{itemize}
Aquelle que trata de ethnographia.
\section{Ethnologia}
\begin{itemize}
\item {Grp. gram.:f.}
\end{itemize}
\begin{itemize}
\item {Proveniência:(Do gr. \textunderscore ethnos\textunderscore  + \textunderscore logos\textunderscore )}
\end{itemize}
Tratado á cêrca da origem e distribuição dos povos.
\section{Ethnológico}
\begin{itemize}
\item {Grp. gram.:adj.}
\end{itemize}
Relativo a ethnologia.
\section{Ethnologista}
\begin{itemize}
\item {Grp. gram.:m.}
\end{itemize}
Aquelle que se occupa de ethnologia.
\section{Ethnológo}
\begin{itemize}
\item {Grp. gram.:m.}
\end{itemize}
Aquelle que se occupa de ethnologia.
\section{Ethnometria}
\begin{itemize}
\item {Grp. gram.:f.}
\end{itemize}
Medida da capacidade ingênita de uma raça. Cf. Oliveira Martins, \textunderscore Raças Humanas\textunderscore .
\section{Ethnométrico}
\begin{itemize}
\item {Grp. gram.:adj.}
\end{itemize}
Relativo a ethnometria.
\section{Ethocracia}
\begin{itemize}
\item {Grp. gram.:f.}
\end{itemize}
\begin{itemize}
\item {Proveniência:(Do gr. \textunderscore ethos\textunderscore  + \textunderscore krateia\textunderscore )}
\end{itemize}
Fórma de govêrno, baseada na moral.
\section{Ethogenia}
\begin{itemize}
\item {Grp. gram.:f.}
\end{itemize}
\begin{itemize}
\item {Proveniência:(Do gr. \textunderscore ethos\textunderscore  + \textunderscore genos\textunderscore )}
\end{itemize}
Sciência, que trata da origem, dos costumes e caracteres dos povos.
\section{Ethognosia}
\begin{itemize}
\item {Grp. gram.:f.}
\end{itemize}
\begin{itemize}
\item {Proveniência:(Do gr. \textunderscore ethos\textunderscore  + \textunderscore gnosis\textunderscore )}
\end{itemize}
Conhecimento dos caracteres e costumes dos povos.
\section{Ethognóstico}
\begin{itemize}
\item {Grp. gram.:adj.}
\end{itemize}
Relativo á ethognosia.
\section{Ethographia}
\begin{itemize}
\item {Grp. gram.:f.}
\end{itemize}
\begin{itemize}
\item {Proveniência:(Do gr. \textunderscore ethos\textunderscore  + \textunderscore graphein\textunderscore )}
\end{itemize}
Descripção dos costumes, carácter e paixões do homem.
\section{Ethográphico}
\begin{itemize}
\item {Grp. gram.:adj.}
\end{itemize}
Relativo a ethographia.
\section{Ethologia}
\begin{itemize}
\item {Grp. gram.:f.}
\end{itemize}
\begin{itemize}
\item {Proveniência:(De \textunderscore ethos\textunderscore  + \textunderscore logos\textunderscore )}
\end{itemize}
Tratado de costumes e caracteres.
\section{Ethologicamente}
\begin{itemize}
\item {Grp. gram.:adv.}
\end{itemize}
De módo ethológico.
\section{Ethológico}
\begin{itemize}
\item {Grp. gram.:adj.}
\end{itemize}
Relativo a ethologia.
\section{Ethólogo}
\begin{itemize}
\item {Grp. gram.:m.}
\end{itemize}
Aquelle que trata da ethologia.
\section{Ethopeia}
\begin{itemize}
\item {Grp. gram.:f.}
\end{itemize}
\begin{itemize}
\item {Proveniência:(Do gr. \textunderscore ethos\textunderscore  + \textunderscore poiein\textunderscore )}
\end{itemize}
Descripção dos costumes e paixões humanas.
\section{Ethopeu}
\begin{itemize}
\item {Grp. gram.:m.}
\end{itemize}
\begin{itemize}
\item {Proveniência:(Do rad. de \textunderscore ethopeia\textunderscore )}
\end{itemize}
Aquelle que descreve paixões e costumes.
\section{Ethrioscopia}
\begin{itemize}
\item {Grp. gram.:f.}
\end{itemize}
Applicação do ethrioscópio.
\section{Ethrioscópio}
\begin{itemize}
\item {Grp. gram.:m.}
\end{itemize}
\begin{itemize}
\item {Proveniência:(Do gr. \textunderscore aithria\textunderscore  + \textunderscore skopein\textunderscore )}
\end{itemize}
Instrumento, com que se avalia a irradiação do calor para a atmosphera sem nuvens.
\section{Ethúlia}
\begin{itemize}
\item {Grp. gram.:f.}
\end{itemize}
Gênero de plantas, da fam. das compostas.
\section{Ethusa}
\begin{itemize}
\item {Grp. gram.:f.}
\end{itemize}
\begin{itemize}
\item {Proveniência:(Lat. \textunderscore aethusa\textunderscore )}
\end{itemize}
Gênero de plantas umbellíferas:«\textunderscore a ethusa peçonhenta...\textunderscore »Bocage, \textunderscore As Plantas\textunderscore .
\section{Ethylacetemia}
\begin{itemize}
\item {Grp. gram.:f.}
\end{itemize}
\begin{itemize}
\item {Utilização:Med.}
\end{itemize}
Existencia de ácido ethylacético no sangue.
\section{Ethylacético}
\begin{itemize}
\item {Grp. gram.:adj.}
\end{itemize}
Diz-se de um ácido, o mesmo que \textunderscore diacético\textunderscore .
\section{Ethylamina}
\begin{itemize}
\item {Grp. gram.:f.}
\end{itemize}
\begin{itemize}
\item {Proveniência:(De \textunderscore ethylo\textunderscore  + \textunderscore amina\textunderscore )}
\end{itemize}
Ammoníaco composto, em que o hidrogênio é substituido pelo ethylo.
\section{Ethylênico}
\begin{itemize}
\item {Grp. gram.:adj.}
\end{itemize}
\begin{itemize}
\item {Utilização:Chím.}
\end{itemize}
Diz-se de um grupo de carbonetos.
\section{Ethyleno}
\begin{itemize}
\item {Grp. gram.:m.}
\end{itemize}
\begin{itemize}
\item {Proveniência:(De \textunderscore ethylo\textunderscore )}
\end{itemize}
Hydrocarbureto da série das olefinas.
\section{Ethýlico}
\begin{itemize}
\item {Grp. gram.:adj.}
\end{itemize}
Em que entra o ethylo.
\section{Ethylo}
\begin{itemize}
\item {Grp. gram.:m.}
\end{itemize}
\begin{itemize}
\item {Utilização:Chím.}
\end{itemize}
Radical monoatómico, que funcciona nos álcooes, nos étheres, e nos ammoníacos que dêstes derivam.
Substância, que se obtém, decompondo por meio de zinco o éther iodhýdrico.
\section{Ética}
\begin{itemize}
\item {Grp. gram.:f.}
\end{itemize}
\begin{itemize}
\item {Proveniência:(De \textunderscore ético\textunderscore )}
\end{itemize}
Ciência da moral.
\section{Ética}
\begin{itemize}
\item {Grp. gram.:f.}
\end{itemize}
O mesmo que \textunderscore héctica\textunderscore .
\section{Ético}
\begin{itemize}
\item {Grp. gram.:adj.}
\end{itemize}
\begin{itemize}
\item {Proveniência:(Gr. \textunderscore ethikos\textunderscore )}
\end{itemize}
Relativo á ética ou á moral.
\section{Ético}
\begin{itemize}
\item {Grp. gram.:m.  e  adj.}
\end{itemize}
O mesmo que \textunderscore héctico\textunderscore .
\section{Étigo}
\textunderscore m.\textunderscore  e \textunderscore adj.\textunderscore  (e der.)
Fórma ant. e pop. de \textunderscore héctico\textunderscore , etc.
\section{Etiguidade}
\begin{itemize}
\item {Grp. gram.:f.}
\end{itemize}
Qualidade de étigo. Cf. Filinto, \textunderscore D. Man.\textunderscore , III, 280.
\section{Etilacetemia}
\begin{itemize}
\item {Grp. gram.:f.}
\end{itemize}
\begin{itemize}
\item {Utilização:Med.}
\end{itemize}
Existencia de ácido etilacético no sangue.
\section{Etilacético}
\begin{itemize}
\item {Grp. gram.:adj.}
\end{itemize}
Diz-se de um ácido, o mesmo que \textunderscore diacético\textunderscore .
\section{Etilamina}
\begin{itemize}
\item {Grp. gram.:f.}
\end{itemize}
\begin{itemize}
\item {Proveniência:(De \textunderscore etilo\textunderscore  + \textunderscore amina\textunderscore )}
\end{itemize}
Amoníaco composto, em que o hidrogênio é substituido pelo etilo.
\section{Etilênico}
\begin{itemize}
\item {Grp. gram.:adj.}
\end{itemize}
\begin{itemize}
\item {Utilização:Chím.}
\end{itemize}
Diz-se de um grupo de carbonetos.
\section{Etileno}
\begin{itemize}
\item {Grp. gram.:m.}
\end{itemize}
\begin{itemize}
\item {Proveniência:(De \textunderscore etilo\textunderscore )}
\end{itemize}
Hydrocarbureto da série das olefinas.
\section{Etílico}
\begin{itemize}
\item {Grp. gram.:adj.}
\end{itemize}
Em que entra o etilo.
\section{Etilo}
\begin{itemize}
\item {Grp. gram.:m.}
\end{itemize}
\begin{itemize}
\item {Utilização:Chím.}
\end{itemize}
Radical monoatómico, que funciona nos álcooes, nos éteres, e nos amoníacos que dêstes derivam.
Substância, que se obtém, decompondo por meio de zinco o éter iodídrico.
\section{Étimo}
\begin{itemize}
\item {Grp. gram.:m.}
\end{itemize}
\begin{itemize}
\item {Proveniência:(Do lat. \textunderscore etymon\textunderscore )}
\end{itemize}
Vocábulo, que se considera origem imediata do outro.
\section{Etimologia}
\begin{itemize}
\item {Grp. gram.:f.}
\end{itemize}
\begin{itemize}
\item {Proveniência:(Lat. \textunderscore etymologia\textunderscore )}
\end{itemize}
Doutrina da derivação e composição das palavras; derivação de uma palavra.
\section{Etimologicamente}
\begin{itemize}
\item {Grp. gram.:adv.}
\end{itemize}
Sob ponto de vista etimológico; segundo as regras da etimologia.
\section{Etimológico}
\begin{itemize}
\item {Grp. gram.:adj.}
\end{itemize}
\begin{itemize}
\item {Proveniência:(Lat. \textunderscore etymologicus\textunderscore )}
\end{itemize}
Relativo a etimologia; que trata de etimologia.
\section{Etimologismo}
\begin{itemize}
\item {Grp. gram.:m.}
\end{itemize}
Processo ou maneira de determinar a etimologia das palavras.
\section{Etimologista}
\begin{itemize}
\item {Grp. gram.:m.}
\end{itemize}
Aquele que se ocupa de etimologia.
Partidário da ortografia etimológica.
\section{Etimólogo}
\begin{itemize}
\item {Grp. gram.:m.}
\end{itemize}
O mesmo que \textunderscore etimologista\textunderscore .
\section{Étimon}
\begin{itemize}
\item {Grp. gram.:m.}
\end{itemize}
O mesmo que \textunderscore étimo\textunderscore .
\section{Etiologia}
\begin{itemize}
\item {Grp. gram.:f.}
\end{itemize}
\begin{itemize}
\item {Proveniência:(Do lat. \textunderscore aetiologia\textunderscore )}
\end{itemize}
Estudo sôbre a origem das coisas.
Parte da Medicina, que trata das causas das doenças.
\section{Etiológico}
\begin{itemize}
\item {Grp. gram.:adj.}
\end{itemize}
Relativo a etiologia.
\section{Etionema}
\begin{itemize}
\item {Grp. gram.:m.}
\end{itemize}
Gênero de plantas crucíferas.
\section{Etiónico}
\begin{itemize}
\item {Grp. gram.:adj.}
\end{itemize}
Diz-se de um ácido, que se obtém pela acção do ácido sulfúrico anhidro sôbre o álcool.
\section{Etíope}
\begin{itemize}
\item {Grp. gram.:adj.}
\end{itemize}
\begin{itemize}
\item {Grp. gram.:M.}
\end{itemize}
\begin{itemize}
\item {Proveniência:(Do lat. \textunderscore aethiops\textunderscore )}
\end{itemize}
Relativo á Etiópia ou aos seus habitantes.
Habitante da Etiópia.
Designação ant. de certos óxidos e sulfuretos metálicos, por causa da côr escura.
\section{Etiópico}
\begin{itemize}
\item {Grp. gram.:adj.}
\end{itemize}
\begin{itemize}
\item {Proveniência:(De \textunderscore Etiópia\textunderscore , n. p.)}
\end{itemize}
Relativo aos Etíopes.
\section{Etiópio}
\begin{itemize}
\item {Grp. gram.:m.}
\end{itemize}
\begin{itemize}
\item {Grp. gram.:Pl.}
\end{itemize}
Língua da Etiópia.
Habitantes da Etiópia. Cf. Esmeraldo, c. 33.
\section{Etiopisa}
\begin{itemize}
\item {Grp. gram.:f.}
\end{itemize}
\begin{itemize}
\item {Proveniência:(Do lat. \textunderscore aethiopissa\textunderscore )}
\end{itemize}
Mulher da Etiópia.
\section{Etiqueta}
\begin{itemize}
\item {fónica:quê}
\end{itemize}
\begin{itemize}
\item {Grp. gram.:f.}
\end{itemize}
\begin{itemize}
\item {Utilização:Gal}
\end{itemize}
\begin{itemize}
\item {Utilização:T. da Bairrada}
\end{itemize}
\begin{itemize}
\item {Proveniência:(Fr. \textunderscore etiquette\textunderscore )}
\end{itemize}
Conjunto de ceremónias, adoptadas na alta sociedade.
Trato ceremonioso.
Letreiro, que se põe sôbre um objecto, para designar o que êste é ou o que contém; rótulo.
Frieza de relações pessoaes; desavença.
\section{Etiquetagem}
\begin{itemize}
\item {Grp. gram.:f.}
\end{itemize}
Acto de etiquetar.
\section{Etiquetar}
\begin{itemize}
\item {Grp. gram.:v. t.}
\end{itemize}
\begin{itemize}
\item {Utilização:Gal}
\end{itemize}
Pôr etiqueta ou rótulo em. Cf. Camillo, \textunderscore Narcót.\textunderscore , II, 166.
\section{Etiquetíssimo}
\begin{itemize}
\item {Grp. gram.:adj.}
\end{itemize}
Cheio de etiqueta:«\textunderscore ...em quanto durava a etiquetíssima ceremónia...\textunderscore »Filinto, VIII, 44.
\section{Etite}
\begin{itemize}
\item {Grp. gram.:f.}
\end{itemize}
\begin{itemize}
\item {Proveniência:(Do gr. \textunderscore aitos\textunderscore )}
\end{itemize}
Pedra, que se encontra em o ninho das águias e que por isso se chama também pedra de águia.
\section{Etmoidal}
\begin{itemize}
\item {Grp. gram.:adj.}
\end{itemize}
Relativo ao etmoide.
\section{Etmoide}
\begin{itemize}
\item {Grp. gram.:m.}
\end{itemize}
\begin{itemize}
\item {Utilização:Anat.}
\end{itemize}
\begin{itemize}
\item {Proveniência:(Do gr. \textunderscore ethmos\textunderscore  + \textunderscore eidos\textunderscore )}
\end{itemize}
Osso do crânio, que contribue para a formação das cavidades nasaes.
\section{Etmoídeo}
\begin{itemize}
\item {Grp. gram.:adj.}
\end{itemize}
(V.etmoidal)
\section{Etna}
\begin{itemize}
\item {Grp. gram.:m.}
\end{itemize}
\begin{itemize}
\item {Utilização:Fig.}
\end{itemize}
\begin{itemize}
\item {Proveniência:(De \textunderscore Ethna\textunderscore , n. p.)}
\end{itemize}
Vulcão. Cf. Garrett, \textunderscore Port. na Bal.\textunderscore , 75.
\section{Etnarca}
\begin{itemize}
\item {Grp. gram.:f.}
\end{itemize}
\begin{itemize}
\item {Proveniência:(Gr. \textunderscore ethnarkes\textunderscore )}
\end{itemize}
Governador de província, na antiguidade.
\section{Etnarquia}
\begin{itemize}
\item {Grp. gram.:f.}
\end{itemize}
\begin{itemize}
\item {Proveniência:(De \textunderscore etnarca\textunderscore )}
\end{itemize}
Dignidade do etnarca.
Território governado por ele.
\section{Etnárquico}
\begin{itemize}
\item {Grp. gram.:adj.}
\end{itemize}
Relativo a etnarquia.
\section{Etnicamente}
\begin{itemize}
\item {Grp. gram.:adv.}
\end{itemize}
Á maneira dos étnicos.
\section{Etnicismo}
\begin{itemize}
\item {Grp. gram.:m.}
\end{itemize}
\begin{itemize}
\item {Proveniência:(De \textunderscore étnico\textunderscore )}
\end{itemize}
O mesmo que \textunderscore paganismo\textunderscore .
\section{Étnico}
\begin{itemize}
\item {Grp. gram.:adj.}
\end{itemize}
\begin{itemize}
\item {Grp. gram.:M.}
\end{itemize}
\begin{itemize}
\item {Proveniência:(Gr. \textunderscore ethnikos\textunderscore )}
\end{itemize}
Que pertence ao paganismo.
Característico de um país: \textunderscore as condições étnicas de Portugal\textunderscore .
Que designa habitantes de um país ou de uma região.
Idólatra.
\section{Etnodicéa}
\begin{itemize}
\item {Grp. gram.:f.}
\end{itemize}
\begin{itemize}
\item {Proveniência:(Do gr. \textunderscore ethnos\textunderscore  + \textunderscore dike\textunderscore )}
\end{itemize}
Direito das gentes.
\section{Etnodiceia}
\begin{itemize}
\item {Grp. gram.:f.}
\end{itemize}
\begin{itemize}
\item {Proveniência:(Do gr. \textunderscore ethnos\textunderscore  + \textunderscore dike\textunderscore )}
\end{itemize}
Direito das gentes.
\section{Etno-genealogia}
\begin{itemize}
\item {Grp. gram.:f.}
\end{itemize}
Genealogia dos povos.
\section{Etnogenia}
\begin{itemize}
\item {Grp. gram.:f.}
\end{itemize}
\begin{itemize}
\item {Proveniência:(Do gr. \textunderscore ethnos\textunderscore  + \textunderscore gene\textunderscore )}
\end{itemize}
Ciência, que trata da origem dos povos.
\section{Etnografia}
\begin{itemize}
\item {Grp. gram.:f.}
\end{itemize}
\begin{itemize}
\item {Proveniência:(Do gr. \textunderscore ethnos\textunderscore  + \textunderscore graphein\textunderscore )}
\end{itemize}
Ciência, que descreve os povos, sua raça, língua, religiões, etc.
\section{Etnograficamente}
\begin{itemize}
\item {Grp. gram.:adv.}
\end{itemize}
Sob o ponto de vista etnográfico.
\section{Etnográfico}
\begin{itemize}
\item {Grp. gram.:adj.}
\end{itemize}
Relativo a etnografia.
\section{Etnógrafo}
\begin{itemize}
\item {Grp. gram.:m.}
\end{itemize}
Aquele que trata de etnografia.
\section{Etnologia}
\begin{itemize}
\item {Grp. gram.:f.}
\end{itemize}
\begin{itemize}
\item {Proveniência:(Do gr. \textunderscore ethnos\textunderscore  + \textunderscore logos\textunderscore )}
\end{itemize}
Tratado á cêrca da origem e distribuição dos povos.
\section{Etnológico}
\begin{itemize}
\item {Grp. gram.:adj.}
\end{itemize}
Relativo a etnologia.
\section{Etnologista}
\begin{itemize}
\item {Grp. gram.:m.}
\end{itemize}
Aquele que se ocupa de etnologia.
\section{Etnológo}
\begin{itemize}
\item {Grp. gram.:m.}
\end{itemize}
Aquele que se ocupa de etnologia.
\section{Etnometria}
\begin{itemize}
\item {Grp. gram.:f.}
\end{itemize}
Medida da capacidade ingênita de uma raça. Cf. Oliveira Martins, \textunderscore Raças Humanas\textunderscore .
\section{Etnométrico}
\begin{itemize}
\item {Grp. gram.:adj.}
\end{itemize}
Relativo a etnometria.
\section{...êto}
\begin{itemize}
\item {Grp. gram.:suf. m.}
\end{itemize}
(designativo de deminuição)
\section{Etocracia}
\begin{itemize}
\item {Grp. gram.:f.}
\end{itemize}
\begin{itemize}
\item {Proveniência:(Do gr. \textunderscore ethos\textunderscore  + \textunderscore krateia\textunderscore )}
\end{itemize}
Fórma de govêrno, baseada na moral.
\section{Etogenia}
\begin{itemize}
\item {Grp. gram.:f.}
\end{itemize}
\begin{itemize}
\item {Proveniência:(Do gr. \textunderscore ethos\textunderscore  + \textunderscore genos\textunderscore )}
\end{itemize}
Ciência, que trata da origem, dos costumes e caracteres dos povos.
\section{Etognosia}
\begin{itemize}
\item {Grp. gram.:f.}
\end{itemize}
\begin{itemize}
\item {Proveniência:(Do gr. \textunderscore ethos\textunderscore  + \textunderscore gnosis\textunderscore )}
\end{itemize}
Conhecimento dos caracteres e costumes dos povos.
\section{Etognóstico}
\begin{itemize}
\item {Grp. gram.:adj.}
\end{itemize}
Relativo á etognosia.
\section{Etografia}
\begin{itemize}
\item {Grp. gram.:f.}
\end{itemize}
\begin{itemize}
\item {Proveniência:(Do gr. \textunderscore ethos\textunderscore  + \textunderscore graphein\textunderscore )}
\end{itemize}
Descripção dos costumes, carácter e paixões do homem.
\section{Etográfico}
\begin{itemize}
\item {Grp. gram.:adj.}
\end{itemize}
Relativo a etografia.
\section{Etoliano}
\begin{itemize}
\item {Grp. gram.:adj.}
\end{itemize}
\begin{itemize}
\item {Grp. gram.:M.}
\end{itemize}
\begin{itemize}
\item {Proveniência:(Do gr. \textunderscore aitolos\textunderscore )}
\end{itemize}
Relativo á Etólia, região da Grécia.
Habitante da Etólia.
\section{Etólico}
\begin{itemize}
\item {Grp. gram.:adj.}
\end{itemize}
\begin{itemize}
\item {Grp. gram.:M.}
\end{itemize}
\begin{itemize}
\item {Proveniência:(Do gr. \textunderscore aitolos\textunderscore )}
\end{itemize}
Relativo á Etólia, região da Grécia.
Habitante da Etólia.
\section{Etólio}
\begin{itemize}
\item {Grp. gram.:adj.}
\end{itemize}
\begin{itemize}
\item {Grp. gram.:M.}
\end{itemize}
\begin{itemize}
\item {Proveniência:(Do gr. \textunderscore aitolos\textunderscore )}
\end{itemize}
Relativo á Etólia, região da Grécia.
Habitante da Etólia.
\section{Etologia}
\begin{itemize}
\item {Grp. gram.:f.}
\end{itemize}
\begin{itemize}
\item {Proveniência:(De \textunderscore ethos\textunderscore  + \textunderscore logos\textunderscore )}
\end{itemize}
Tratado de costumes e caracteres.
\section{Etologicamente}
\begin{itemize}
\item {Grp. gram.:adv.}
\end{itemize}
De módo etológico.
\section{Etológico}
\begin{itemize}
\item {Grp. gram.:adj.}
\end{itemize}
Relativo a etologia.
\section{Etólogo}
\begin{itemize}
\item {Grp. gram.:m.}
\end{itemize}
Aquele que trata da etologia.
\section{Etopéa}
\begin{itemize}
\item {Grp. gram.:f.}
\end{itemize}
\begin{itemize}
\item {Proveniência:(Do gr. \textunderscore ethos\textunderscore  + \textunderscore poiein\textunderscore )}
\end{itemize}
Descripção dos costumes e paixões humanas.
\section{Etopeia}
\begin{itemize}
\item {Grp. gram.:f.}
\end{itemize}
\begin{itemize}
\item {Proveniência:(Do gr. \textunderscore ethos\textunderscore  + \textunderscore poiein\textunderscore )}
\end{itemize}
Descripção dos costumes e paixões humanas.
\section{Etopeu}
\begin{itemize}
\item {Grp. gram.:m.}
\end{itemize}
\begin{itemize}
\item {Proveniência:(Do rad. de \textunderscore etopeia\textunderscore )}
\end{itemize}
Aquelle que descreve paixões e costumes.
\section{Etrioscopia}
\begin{itemize}
\item {Grp. gram.:f.}
\end{itemize}
Aplicação do etrioscópio.
\section{Etrioscópio}
\begin{itemize}
\item {Grp. gram.:m.}
\end{itemize}
\begin{itemize}
\item {Proveniência:(Do gr. \textunderscore aithria\textunderscore  + \textunderscore skopein\textunderscore )}
\end{itemize}
Instrumento, com que se avalia a irradiação do calor para a atmosfera sem nuvens.
\section{Etrusco}
\begin{itemize}
\item {Grp. gram.:adj.}
\end{itemize}
\begin{itemize}
\item {Proveniência:(Lat. \textunderscore etruscus\textunderscore )}
\end{itemize}
Relativo á Etrúria.
Língua dos etruscos.
\section{Etúlia}
\begin{itemize}
\item {Grp. gram.:f.}
\end{itemize}
Gênero de plantas, da fam. das compostas.
\section{Etúngula}
\begin{itemize}
\item {Grp. gram.:f.}
\end{itemize}
Espécie de falcão.
\section{Etusa}
\begin{itemize}
\item {Grp. gram.:f.}
\end{itemize}
\begin{itemize}
\item {Proveniência:(Lat. \textunderscore aethusa\textunderscore )}
\end{itemize}
Gênero de plantas umbelíferas:«\textunderscore a etusa peçonhenta...\textunderscore »Bocage, \textunderscore As Plantas\textunderscore .
\section{Étymo}
\begin{itemize}
\item {Grp. gram.:m.}
\end{itemize}
\begin{itemize}
\item {Proveniência:(Do lat. \textunderscore etymon\textunderscore )}
\end{itemize}
Vocábulo, que se considera origem immediata do outro.
\section{Etymologia}
\begin{itemize}
\item {Grp. gram.:f.}
\end{itemize}
\begin{itemize}
\item {Proveniência:(Lat. \textunderscore etymologia\textunderscore )}
\end{itemize}
Doutrina da derivação e composição das palavras; derivação de uma palavra.
\section{Etymologicamente}
\begin{itemize}
\item {Grp. gram.:adv.}
\end{itemize}
Sob ponto de vista etymológico; segundo as regras da etymologia.
\section{Etymológico}
\begin{itemize}
\item {Grp. gram.:adj.}
\end{itemize}
\begin{itemize}
\item {Proveniência:(Lat. \textunderscore etymologicus\textunderscore )}
\end{itemize}
Relativo a etymologia; que trata de etymologia.
\section{Etymologismo}
\begin{itemize}
\item {Grp. gram.:m.}
\end{itemize}
Processo ou maneira de determinar a etymologia das palavras.
\section{Etymologista}
\begin{itemize}
\item {Grp. gram.:m.}
\end{itemize}
Aquelle que se occupa de etymologia.
Partidário da orthographia etymológica.
\section{Etymólogo}
\begin{itemize}
\item {Grp. gram.:m.}
\end{itemize}
O mesmo que \textunderscore etymologista\textunderscore .
\section{Étymon}
\begin{itemize}
\item {Grp. gram.:m.}
\end{itemize}
O mesmo que \textunderscore étymo\textunderscore .
\section{Eu}
\begin{itemize}
\item {Grp. gram.:pron. pess.}
\end{itemize}
\begin{itemize}
\item {Grp. gram.:M.}
\end{itemize}
\begin{itemize}
\item {Proveniência:(Do lat. \textunderscore ego\textunderscore )}
\end{itemize}
(designativo da primeira pessôa)
A personalidade de quem fala.
O sêr humano, considerado como consciente.
A consciência.
\section{Euá-uaçu}
\begin{itemize}
\item {Grp. gram.:m.}
\end{itemize}
\begin{itemize}
\item {Utilização:Bras}
\end{itemize}
Planta gramínea, de cujos caules se fazem coberturas de casas.
\section{Eubage}
\begin{itemize}
\item {Grp. gram.:m.}
\end{itemize}
Antigo sacerdote gállio, adivinho ou astrólogo. Cf. Filinto, XV, 44, 53 e 54.
\section{Eubazo}
\begin{itemize}
\item {Grp. gram.:m.}
\end{itemize}
\begin{itemize}
\item {Proveniência:(Do gr. \textunderscore eu\textunderscore  + \textunderscore bazo\textunderscore )}
\end{itemize}
Gênero de insectos hymenópteros, cujo typo se encontra na Bélgica.
\section{Eubiótica}
\begin{itemize}
\item {Grp. gram.:f.}
\end{itemize}
\begin{itemize}
\item {Proveniência:(Do gr. \textunderscore eubiotos\textunderscore )}
\end{itemize}
Arte de bem viver.
\section{Euboico}
\begin{itemize}
\item {Grp. gram.:adj.}
\end{itemize}
\begin{itemize}
\item {Proveniência:(Lat. \textunderscore euboicus\textunderscore )}
\end{itemize}
Relativo á ilha de Eubeia, no mar Egeu.
Dizia-se da soma de dinheiro chamada talento e que differia do talento áttico.
\section{Eubolia}
\begin{itemize}
\item {Grp. gram.:f.}
\end{itemize}
\begin{itemize}
\item {Utilização:Ant.}
\end{itemize}
\begin{itemize}
\item {Proveniência:(Gr. \textunderscore eubolia\textunderscore )}
\end{itemize}
Bom conselho.
\section{Eucaína}
\begin{itemize}
\item {Grp. gram.:f.}
\end{itemize}
Substância, derivada da cocaína e com as mesmas applicações que esta.
\section{Eucalipto}
\begin{itemize}
\item {Grp. gram.:m.}
\end{itemize}
\begin{itemize}
\item {Proveniência:(Do gr. \textunderscore eu\textunderscore  + \textunderscore calupto\textunderscore )}
\end{itemize}
Gênero de árvores mirtáceas.
\section{Eucaliptol}
\begin{itemize}
\item {Grp. gram.:m.}
\end{itemize}
\begin{itemize}
\item {Proveniência:(De \textunderscore eucalipto\textunderscore  + \textunderscore óleo\textunderscore )}
\end{itemize}
Óleo ou essência, extraida das fôlhas do eucalipto.
\section{Eucalypto}
\begin{itemize}
\item {Grp. gram.:m.}
\end{itemize}
\begin{itemize}
\item {Proveniência:(Do gr. \textunderscore eu\textunderscore  + \textunderscore calupto\textunderscore )}
\end{itemize}
Gênero de árvores myrtáceas.
\section{Eucalyptol}
\begin{itemize}
\item {Grp. gram.:m.}
\end{itemize}
\begin{itemize}
\item {Proveniência:(De \textunderscore eucalypto\textunderscore  + \textunderscore óleo\textunderscore )}
\end{itemize}
Óleo ou essência, extrahida das fôlhas do eucalypto.
\section{Eucaristia}
\begin{itemize}
\item {Grp. gram.:f.}
\end{itemize}
\begin{itemize}
\item {Proveniência:(Lat. \textunderscore eucharistia\textunderscore )}
\end{itemize}
Sacramento, em que o corpo e sangue de Cristo estão representados por pão e vinho.
\section{Eucarístico}
\begin{itemize}
\item {Grp. gram.:adj.}
\end{itemize}
Relativo a \textunderscore eucaristia\textunderscore .
\section{Êucero}
\begin{itemize}
\item {Grp. gram.:m.}
\end{itemize}
\begin{itemize}
\item {Proveniência:(Do gr. \textunderscore eu\textunderscore  + \textunderscore keras\textunderscore )}
\end{itemize}
Insecto hymenóptero.
\section{Eucharia}
\begin{itemize}
\item {Grp. gram.:f.}
\end{itemize}
Variedade de figueira algarvia.
\section{Eucharistia}
\begin{itemize}
\item {fónica:ca}
\end{itemize}
\begin{itemize}
\item {Grp. gram.:f.}
\end{itemize}
\begin{itemize}
\item {Proveniência:(Lat. \textunderscore eucharistia\textunderscore )}
\end{itemize}
Sacramento, em que o corpo e sangue de Christo estão representados por pão e vinho.
\section{Eucharístico}
\begin{itemize}
\item {fónica:ca}
\end{itemize}
\begin{itemize}
\item {Grp. gram.:adj.}
\end{itemize}
Relativo a \textunderscore eucharistia\textunderscore .
\section{Euchlorina}
\begin{itemize}
\item {Grp. gram.:f.}
\end{itemize}
\begin{itemize}
\item {Utilização:Chím.}
\end{itemize}
Protóxydo de chloro.
\section{Euchológio}
\begin{itemize}
\item {fónica:co}
\end{itemize}
\begin{itemize}
\item {Grp. gram.:m.}
\end{itemize}
\begin{itemize}
\item {Proveniência:(Gr. \textunderscore eukhologion\textunderscore )}
\end{itemize}
Livro de orações, que contém especialmente o officio dos domingos e das festas principaes.
\section{Euchromo}
\begin{itemize}
\item {fónica:cro}
\end{itemize}
\begin{itemize}
\item {Grp. gram.:adj.}
\end{itemize}
\begin{itemize}
\item {Proveniência:(Do gr. \textunderscore eu\textunderscore  + \textunderscore khroma\textunderscore )}
\end{itemize}
Que tem côr bella.
\section{Euchymo}
\begin{itemize}
\item {fónica:qui}
\end{itemize}
\begin{itemize}
\item {Grp. gram.:m.}
\end{itemize}
\begin{itemize}
\item {Proveniência:(Do gr. \textunderscore eu\textunderscore  + \textunderscore khumos\textunderscore )}
\end{itemize}
Suco nutriente dos vegetaes.
\section{Eucinesia}
\begin{itemize}
\item {Grp. gram.:f.}
\end{itemize}
\begin{itemize}
\item {Proveniência:(Do gr. \textunderscore eu\textunderscore  + \textunderscore kinesis\textunderscore )}
\end{itemize}
Movimento regular orgânico.
\section{Êuclase}
\begin{itemize}
\item {Grp. gram.:f.}
\end{itemize}
\begin{itemize}
\item {Proveniência:(Do gr. \textunderscore eu\textunderscore  + \textunderscore klasis\textunderscore )}
\end{itemize}
Esmeralda prismática do Brasil.
\section{Euclídeas}
\begin{itemize}
\item {Grp. gram.:f. pl.}
\end{itemize}
Quarta tríbo de plantas crucíferas, na classificação de De-Candolle.
\section{Euclorina}
\begin{itemize}
\item {Grp. gram.:f.}
\end{itemize}
\begin{itemize}
\item {Utilização:Chím.}
\end{itemize}
Protóxido de cloro.
\section{Eucológio}
\begin{itemize}
\item {Grp. gram.:m.}
\end{itemize}
\begin{itemize}
\item {Proveniência:(Gr. \textunderscore eukhologion\textunderscore )}
\end{itemize}
Livro de orações, que contém especialmente o oficio dos domingos e das festas principaes.
\section{Eucrasia}
\begin{itemize}
\item {Grp. gram.:f.}
\end{itemize}
\begin{itemize}
\item {Utilização:Des.}
\end{itemize}
\begin{itemize}
\item {Proveniência:(Do gr. \textunderscore eu\textunderscore  + \textunderscore krasis\textunderscore )}
\end{itemize}
Bôa compleição.
Organização robusta.
\section{Eucrásico}
\begin{itemize}
\item {Grp. gram.:adj.}
\end{itemize}
Relativo a eucrasia.
\section{Eucromo}
\begin{itemize}
\item {Grp. gram.:adj.}
\end{itemize}
\begin{itemize}
\item {Proveniência:(Do gr. \textunderscore eu\textunderscore  + \textunderscore khroma\textunderscore )}
\end{itemize}
Que tem côr bela.
\section{Eudema}
\begin{itemize}
\item {Grp. gram.:f.}
\end{itemize}
\begin{itemize}
\item {Proveniência:(Do gr. \textunderscore eu\textunderscore  + \textunderscore dema\textunderscore )}
\end{itemize}
Gênero de plantas crucíferas.
\section{Eudiapneustia}
\begin{itemize}
\item {Grp. gram.:f.}
\end{itemize}
\begin{itemize}
\item {Proveniência:(Do gr. \textunderscore eu\textunderscore  + \textunderscore diapnein\textunderscore )}
\end{itemize}
Facilidade de transpiração.
\section{Eudiometria}
\begin{itemize}
\item {Grp. gram.:f.}
\end{itemize}
Applicação do eudiómetro.
\section{Eudiométrico}
\begin{itemize}
\item {Grp. gram.:adj.}
\end{itemize}
Relativo a eudiometria.
\section{Eudiómetro}
\begin{itemize}
\item {Grp. gram.:m.}
\end{itemize}
\begin{itemize}
\item {Proveniência:(Do gr. \textunderscore eudia\textunderscore  + \textunderscore metron\textunderscore )}
\end{itemize}
Instrumento, com que se determina a proporção relativa dos gases.
\section{Eudista}
\begin{itemize}
\item {fónica:e-u}
\end{itemize}
\begin{itemize}
\item {Grp. gram.:m.  e  f.}
\end{itemize}
\begin{itemize}
\item {Proveniência:(De \textunderscore Eudes\textunderscore , n. p.)}
\end{itemize}
Membro de uma corporação religiosa, fundada em França no século XVII pelo oratoriano Eudes.
\section{Euemia}
\begin{itemize}
\item {Grp. gram.:f.}
\end{itemize}
\begin{itemize}
\item {Utilização:Med.}
\end{itemize}
\begin{itemize}
\item {Grp. gram.:f.}
\end{itemize}
\begin{itemize}
\item {Proveniência:(Do gr. \textunderscore eu\textunderscore  + \textunderscore haima\textunderscore )}
\end{itemize}
Bôa qualidade do sangue.
Estado normal do sangue.
\section{Euexia}
\begin{itemize}
\item {fónica:esi}
\end{itemize}
\begin{itemize}
\item {Grp. gram.:f.}
\end{itemize}
\begin{itemize}
\item {Proveniência:(Gr. \textunderscore euexia\textunderscore )}
\end{itemize}
O mesmo que \textunderscore eucrasia\textunderscore .
\section{Eufemicamente}
\begin{itemize}
\item {Grp. gram.:adv.}
\end{itemize}
\begin{itemize}
\item {Proveniência:(De \textunderscore eufêmico\textunderscore )}
\end{itemize}
Com eufemismo.
\section{Eufêmico}
\begin{itemize}
\item {Grp. gram.:adj.}
\end{itemize}
\begin{itemize}
\item {Proveniência:(Do gr. \textunderscore eu\textunderscore  + \textunderscore phemi\textunderscore )}
\end{itemize}
Relativo ao eufemismo.
Em que há eufemismo: \textunderscore expressão eufêmica\textunderscore .
\section{Eufemismo}
\begin{itemize}
\item {Grp. gram.:m.}
\end{itemize}
\begin{itemize}
\item {Proveniência:(Gr. \textunderscore euphemismos\textunderscore )}
\end{itemize}
Suavização de uma palavra ou de uma ideia dura ou desagradável.
Acto ou maneira de disfarçar ideias tristes ou odiosas, por meio de termos que não correspondem precisamente a essas ideias.
\section{Eufonia}
\begin{itemize}
\item {Grp. gram.:f.}
\end{itemize}
\begin{itemize}
\item {Proveniência:(Do gr. \textunderscore eu\textunderscore  + \textunderscore phone\textunderscore )}
\end{itemize}
Som agradável de uma voz ou de um instrumento.
Suavidade ou elegância na pronúncia.
Aquilo que torna suave ou elegante a pronúncia.
\section{Eufonicamente}
\begin{itemize}
\item {Grp. gram.:adv.}
\end{itemize}
De modo eufónico.
\section{Eufónico}
\begin{itemize}
\item {Grp. gram.:adj.}
\end{itemize}
Em que há eufonia.
Suave, melodioso.
Que produz eufonia.
\section{Eufónio}
\begin{itemize}
\item {Grp. gram.:m.}
\end{itemize}
Espécie de harmónica, inventada em 1790 e também conhecida por \textunderscore eufono\textunderscore .
(Cp. \textunderscore eufono\textunderscore )
\section{Eufonizar}
\begin{itemize}
\item {Grp. gram.:v. t.}
\end{itemize}
\begin{itemize}
\item {Utilização:Neol.}
\end{itemize}
Tornar eufónico.
\section{Eufono}
\begin{itemize}
\item {Grp. gram.:adj.}
\end{itemize}
\begin{itemize}
\item {Grp. gram.:M.}
\end{itemize}
\begin{itemize}
\item {Proveniência:(Gr. \textunderscore euphonos\textunderscore )}
\end{itemize}
Que tem voz melodiosa.
Espécie de tangará.
Espécie de harmónica.
\section{Euforbiáceas}
\begin{itemize}
\item {Grp. gram.:f. pl.}
\end{itemize}
\begin{itemize}
\item {Proveniência:(De \textunderscore euforbiáceo\textunderscore )}
\end{itemize}
Família de plantas, que têm por tipo o eufórbio.
\section{Euforbiáceo}
\begin{itemize}
\item {Grp. gram.:adj.}
\end{itemize}
Que é da natureza do eufórbio.
Relativo ou semelhante a esta planta.
\section{Eufórbico}
\begin{itemize}
\item {Grp. gram.:adj.}
\end{itemize}
Diz-se de um ácido cristalizável, que se descobriu nas flôres e folhas do eufórbio.
\section{Euforbina}
\begin{itemize}
\item {Grp. gram.:f.}
\end{itemize}
Substância, que se extrái da raíz do eufórbio.
\section{Eufórbio}
\begin{itemize}
\item {Grp. gram.:m.}
\end{itemize}
\begin{itemize}
\item {Proveniência:(Do gr. \textunderscore eu\textunderscore  + \textunderscore phorbe\textunderscore )}
\end{itemize}
Gênero de plantas, de suco acre e cáustico.
\section{Euforia}
\begin{itemize}
\item {Grp. gram.:f.}
\end{itemize}
\begin{itemize}
\item {Utilização:Med.}
\end{itemize}
\begin{itemize}
\item {Proveniência:(Do gr. \textunderscore eu\textunderscore  + \textunderscore phoros\textunderscore )}
\end{itemize}
Sensação de bem-estar.
\section{Eufotide}
\begin{itemize}
\item {Grp. gram.:f.}
\end{itemize}
\begin{itemize}
\item {Proveniência:(Do gr. \textunderscore eu\textunderscore  + \textunderscore phos\textunderscore , \textunderscore photos\textunderscore )}
\end{itemize}
Espécie de rocha eruptiva.
\section{Eufotita}
\begin{itemize}
\item {Grp. gram.:f.}
\end{itemize}
\begin{itemize}
\item {Proveniência:(Do gr. \textunderscore eu\textunderscore  + \textunderscore phos\textunderscore , \textunderscore photos\textunderscore )}
\end{itemize}
Espécie de rocha eruptiva.
\section{Eufrásia}
\begin{itemize}
\item {Grp. gram.:f.}
\end{itemize}
\begin{itemize}
\item {Proveniência:(Gr. \textunderscore euphrasia\textunderscore )}
\end{itemize}
Planta medicinal, escrofularínea.
\section{Eufrático}
\begin{itemize}
\item {Grp. gram.:adj.}
\end{itemize}
Relativo ao Eufrates:«\textunderscore té que foi dar na eufrática ribeira.\textunderscore »Castilho, \textunderscore Fastos\textunderscore , I, 127.
\section{Euftalmina}
\begin{itemize}
\item {Grp. gram.:f.}
\end{itemize}
Composto químico, usado em Medicina, como sucedâneo da atropina.
\section{Eufuísmo}
\begin{itemize}
\item {Grp. gram.:m.}
\end{itemize}
\begin{itemize}
\item {Proveniência:(Do gr. \textunderscore euphues\textunderscore )}
\end{itemize}
Estilo afectado, que se usou em Inglaterra, semelhante ao gongorismo que dominou em Portugal e Espanha, e ao preciosismo, que dominou em França.
\section{Eufuísta}
\begin{itemize}
\item {Grp. gram.:m.}
\end{itemize}
Aquele que praticava o euphuismo.
\section{Eufuístico}
\begin{itemize}
\item {Grp. gram.:adj.}
\end{itemize}
\begin{itemize}
\item {Proveniência:(De \textunderscore eufuista\textunderscore )}
\end{itemize}
Relativo ao eufuísmo.
\section{Eugè!}
\begin{itemize}
\item {Grp. gram.:interj.}
\end{itemize}
\begin{itemize}
\item {Utilização:Des.}
\end{itemize}
\begin{itemize}
\item {Grp. gram.:M.}
\end{itemize}
\begin{itemize}
\item {Utilização:Ant.}
\end{itemize}
\begin{itemize}
\item {Proveniência:(Lat. \textunderscore euge\textunderscore )}
\end{itemize}
Bravo! viva!
Applauso; louvor.
\section{Eugenesia}
\begin{itemize}
\item {Grp. gram.:f.}
\end{itemize}
Qualidade do sêr que é eugenésico.
\section{Eugenésico}
\begin{itemize}
\item {Grp. gram.:adj.}
\end{itemize}
\begin{itemize}
\item {Proveniência:(Do gr. \textunderscore eu\textunderscore  + \textunderscore genesis\textunderscore )}
\end{itemize}
Diz-se dos indivíduos mestiços, que são directa e indefinidamente fecundos. Cf. E. Burnay, \textunderscore Cranoologia\textunderscore , 64 e 65.
\section{Eugênia}
\begin{itemize}
\item {Grp. gram.:f.}
\end{itemize}
\begin{itemize}
\item {Proveniência:(De \textunderscore Eugênio\textunderscore , n. p.)}
\end{itemize}
Gênero de plantas myrtáceas.
\section{Eugenía}
\begin{itemize}
\item {Grp. gram.:f.}
\end{itemize}
\begin{itemize}
\item {Proveniência:(Do gr. \textunderscore eu\textunderscore  + \textunderscore gene\textunderscore )}
\end{itemize}
Sciência moderna, que se occupa do aperfeiçoamento da producção humana.
\section{Eugênia-uvalha}
\begin{itemize}
\item {Grp. gram.:f.}
\end{itemize}
O mesmo que \textunderscore pitangueira\textunderscore .
\section{Eugênico}
\begin{itemize}
\item {Grp. gram.:adj.}
\end{itemize}
Diz-se de um ácido, que tem sabôr acre e ardente.
\section{Eugenina}
\begin{itemize}
\item {Grp. gram.:f.}
\end{itemize}
Substância crystallina, que se deposita espontaneamente na água destillada do cravo da Índia.
Princípio novo e medicinal, descoberto na pitangueira. Cf. \textunderscore Jorn. do Comm.\textunderscore , do Rio, de 29-VIII-902.
\section{Eugenista}
\begin{itemize}
\item {Grp. gram.:m.}
\end{itemize}
Aquelle que se dedica aos estudos da eugenia.
\section{Euglena}
\begin{itemize}
\item {Grp. gram.:f.}
\end{itemize}
\begin{itemize}
\item {Proveniência:(Do gr. \textunderscore eu\textunderscore  + \textunderscore glenè\textunderscore )}
\end{itemize}
Gênero de insectos coleópteros, cujo typo vive em o norte da Europa.
\section{Eugleno}
\begin{itemize}
\item {Grp. gram.:m.}
\end{itemize}
\begin{itemize}
\item {Proveniência:(Do gr. \textunderscore eu\textunderscore  + \textunderscore glenè\textunderscore )}
\end{itemize}
Gênero de insectos coleópteros, cujo typo vive em o norte da Europa.
\section{Êugrafo}
\begin{itemize}
\item {Grp. gram.:m.}
\end{itemize}
\begin{itemize}
\item {Proveniência:(Do gr. \textunderscore eu\textunderscore  + \textunderscore graphein\textunderscore )}
\end{itemize}
Espécie de câmara escura, em Física.
\section{Êugrapho}
\begin{itemize}
\item {Grp. gram.:m.}
\end{itemize}
\begin{itemize}
\item {Proveniência:(Do gr. \textunderscore eu\textunderscore  + \textunderscore graphein\textunderscore )}
\end{itemize}
Espécie de câmara escura, em Phýsica.
\section{Euhemia}
\begin{itemize}
\item {Grp. gram.:f.}
\end{itemize}
\begin{itemize}
\item {Proveniência:(Do gr. \textunderscore eu\textunderscore  + \textunderscore haima\textunderscore )}
\end{itemize}
Estado normal do sangue.
\section{Euíchthyos}
\begin{itemize}
\item {Grp. gram.:m. pl.}
\end{itemize}
\begin{itemize}
\item {Utilização:Zool.}
\end{itemize}
\begin{itemize}
\item {Proveniência:(Do gr. \textunderscore eu\textunderscore  + \textunderscore ikhthus\textunderscore )}
\end{itemize}
Uma das três sub-classes de peixes, segundo Clauss.
\section{Euíctios}
\begin{itemize}
\item {Grp. gram.:m. pl.}
\end{itemize}
\begin{itemize}
\item {Utilização:Zool.}
\end{itemize}
\begin{itemize}
\item {Proveniência:(Do gr. \textunderscore eu\textunderscore  + \textunderscore ikhthus\textunderscore )}
\end{itemize}
Uma das três sub-classes de peixes, segundo Clauss.
\section{Eulemo}
\begin{itemize}
\item {Grp. gram.:m.}
\end{itemize}
\begin{itemize}
\item {Proveniência:(Do gr. \textunderscore eu\textunderscore  + \textunderscore laimos\textunderscore )}
\end{itemize}
Insecto hymenóptero.
\section{Eulépia}
\begin{itemize}
\item {Grp. gram.:f.}
\end{itemize}
\begin{itemize}
\item {Proveniência:(Do gr. \textunderscore eu\textunderscore  + \textunderscore lepis\textunderscore )}
\end{itemize}
Insecto lepidóptero nocturno.
\section{Eulima}
\begin{itemize}
\item {Grp. gram.:f.}
\end{itemize}
\begin{itemize}
\item {Proveniência:(Do gr. \textunderscore eu\textunderscore  + \textunderscore limos\textunderscore )}
\end{itemize}
Gênero de molluscos gasterópodes.
\section{Eulimo}
\begin{itemize}
\item {Grp. gram.:m.}
\end{itemize}
\begin{itemize}
\item {Proveniência:(Do gr. \textunderscore eu\textunderscore  + \textunderscore limos\textunderscore )}
\end{itemize}
Gênero de molluscos gasterópodes.
\section{Eulóbio}
\begin{itemize}
\item {Grp. gram.:m.}
\end{itemize}
\begin{itemize}
\item {Proveniência:(Do gr. \textunderscore eu\textunderscore  + \textunderscore lobos\textunderscore )}
\end{itemize}
Planta synanthérea da Califórnia.
\section{Eulófio}
\begin{itemize}
\item {Grp. gram.:f.}
\end{itemize}
\begin{itemize}
\item {Proveniência:(Do gr. \textunderscore eu\textunderscore  + \textunderscore lophos\textunderscore )}
\end{itemize}
Gênero de orquídeas da Índia e da África.
\section{Êulofo}
\begin{itemize}
\item {Grp. gram.:m.}
\end{itemize}
\begin{itemize}
\item {Proveniência:(Do gr. \textunderscore eu\textunderscore  + \textunderscore lophos\textunderscore )}
\end{itemize}
Gênero de galináceas da Índia, cobertas de brilhante plumagem, e cuja cabeça é ornada de formoso penacho.
\section{Eulógia}
\begin{itemize}
\item {Grp. gram.:f.}
\end{itemize}
\begin{itemize}
\item {Utilização:Ant.}
\end{itemize}
\begin{itemize}
\item {Proveniência:(Do gr. \textunderscore eu\textunderscore  + \textunderscore logos\textunderscore )}
\end{itemize}
Pão bento.
\section{Eulóphio}
\begin{itemize}
\item {Grp. gram.:f.}
\end{itemize}
\begin{itemize}
\item {Proveniência:(Do gr. \textunderscore eu\textunderscore  + \textunderscore lophos\textunderscore )}
\end{itemize}
Gênero de orchídeas da Índia e da África.
\section{Êulopho}
\begin{itemize}
\item {Grp. gram.:m.}
\end{itemize}
\begin{itemize}
\item {Proveniência:(Do gr. \textunderscore eu\textunderscore  + \textunderscore lophos\textunderscore )}
\end{itemize}
Gênero de gallináceas da Índia, cobertas de brilhante plumagem, e cuja cabeça é ornada de formoso pennacho.
\section{Eumathia}
\begin{itemize}
\item {Grp. gram.:f.}
\end{itemize}
Facilidade de aprender.
\section{Eumatia}
\begin{itemize}
\item {Grp. gram.:f.}
\end{itemize}
Facilidade de aprender.
\section{Eumênide}
\begin{itemize}
\item {Grp. gram.:f.}
\end{itemize}
\begin{itemize}
\item {Utilização:Fig.}
\end{itemize}
\begin{itemize}
\item {Proveniência:(Gr. \textunderscore eumenis\textunderscore )}
\end{itemize}
Cada uma das três fúrias, que, segundo a Mythologia, flagellam os maus no inferno com serpentes e achas accesas.
O pungir da consciência.
O remorso. Cf. Herculano, \textunderscore Questões Pub.\textunderscore , I, IX.
\section{Éumeno}
\begin{itemize}
\item {Grp. gram.:m.}
\end{itemize}
\begin{itemize}
\item {Proveniência:(Do gr. \textunderscore eumenes\textunderscore )}
\end{itemize}
Gênero de insectos himenópteros.
\section{Eumeque}
\begin{itemize}
\item {Grp. gram.:m.}
\end{itemize}
\begin{itemize}
\item {Proveniência:(Gr. \textunderscore eumekes\textunderscore )}
\end{itemize}
Gênero de reptís das regiões tropicaes.
\section{Eumerodo}
\begin{itemize}
\item {fónica:merô}
\end{itemize}
\begin{itemize}
\item {Grp. gram.:m.}
\end{itemize}
\begin{itemize}
\item {Proveniência:(Do gr. \textunderscore eu\textunderscore  + \textunderscore meros\textunderscore )}
\end{itemize}
Gênero de reptís sáurios.
\section{Êumicro}
\begin{itemize}
\item {Grp. gram.:m.}
\end{itemize}
\begin{itemize}
\item {Proveniência:(Do gr. \textunderscore eu\textunderscore  + \textunderscore mikros\textunderscore )}
\end{itemize}
Gênero de insectos coleópteros pentâmeros.
\section{Eumolpo}
\begin{itemize}
\item {Grp. gram.:m.}
\end{itemize}
Insecto coleóptero, que ataca as folhas da videira, (\textunderscore eumolpus vitis\textunderscore ).
\section{Eunómia}
\begin{itemize}
\item {Grp. gram.:f.}
\end{itemize}
\begin{itemize}
\item {Proveniência:(Do gr. \textunderscore eu\textunderscore  + \textunderscore nomos\textunderscore )}
\end{itemize}
Pequeno planeta, descoberto em 1851.
Planta crucífera.
Gênero de polypeiros fósseis, cujo typo appareceu no terreno calcário das vizinhanças de Caêna.
\section{Eunótia}
\begin{itemize}
\item {Grp. gram.:f.}
\end{itemize}
\begin{itemize}
\item {Proveniência:(Do gr. \textunderscore eu\textunderscore  + \textunderscore notos\textunderscore )}
\end{itemize}
Gênero de algas, em que há algumas fósseis.
\section{Eunuchismo}
\begin{itemize}
\item {fónica:quis}
\end{itemize}
\begin{itemize}
\item {Grp. gram.:m.}
\end{itemize}
\begin{itemize}
\item {Utilização:bras}
\end{itemize}
\begin{itemize}
\item {Utilização:Neol.}
\end{itemize}
Qualidade do eunucho.
\section{Eunucho}
\begin{itemize}
\item {Grp. gram.:m.}
\end{itemize}
\begin{itemize}
\item {Utilização:Fig.}
\end{itemize}
\begin{itemize}
\item {Grp. gram.:Adj.}
\end{itemize}
\begin{itemize}
\item {Proveniência:(Gr. \textunderscore eunoukhos\textunderscore )}
\end{itemize}
Homem castrado, empregado na guarda dos harens orientaes.
Homem impotente.
Diz-se da flôr, cujos pistillos e estames se convertem em pétalas.
\section{Eunuco}
\begin{itemize}
\item {Grp. gram.:m.}
\end{itemize}
\begin{itemize}
\item {Utilização:Fig.}
\end{itemize}
\begin{itemize}
\item {Grp. gram.:Adj.}
\end{itemize}
\begin{itemize}
\item {Proveniência:(Gr. \textunderscore eunoukhos\textunderscore )}
\end{itemize}
Homem castrado, empregado na guarda dos harens orientaes.
Homem impotente.
Diz-se da flôr, cujos pistilos e estames se convertem em pétalas.
\section{Eunuquismo}
\begin{itemize}
\item {Grp. gram.:m.}
\end{itemize}
\begin{itemize}
\item {Utilização:bras}
\end{itemize}
\begin{itemize}
\item {Utilização:Neol.}
\end{itemize}
Qualidade do eunuco.
\section{Eupathia}
\begin{itemize}
\item {Grp. gram.:f.}
\end{itemize}
\begin{itemize}
\item {Proveniência:(Do gr. \textunderscore eu\textunderscore  + \textunderscore pathos\textunderscore )}
\end{itemize}
Paciência, resignação.
\section{Eupatia}
\begin{itemize}
\item {Grp. gram.:f.}
\end{itemize}
\begin{itemize}
\item {Proveniência:(Do gr. \textunderscore eu\textunderscore  + \textunderscore pathos\textunderscore )}
\end{itemize}
Paciência, resignação.
\section{Eupatorina}
\begin{itemize}
\item {Grp. gram.:f.}
\end{itemize}
\begin{itemize}
\item {Proveniência:(De \textunderscore eupatório\textunderscore )}
\end{itemize}
Pó branco, medicinal, que se tira do eupatório.
\section{Eupatório}
\begin{itemize}
\item {Grp. gram.:m.}
\end{itemize}
Gênero de plantas, da fam. das compostas, conhecido vulgarmente por \textunderscore escumilha\textunderscore  ou \textunderscore erva-de-cobra\textunderscore .
\section{Eupátrida}
\begin{itemize}
\item {Grp. gram.:m.}
\end{itemize}
\begin{itemize}
\item {Proveniência:(Do gr. \textunderscore eu\textunderscore , bem + \textunderscore pater\textunderscore , pai)}
\end{itemize}
Indivíduo de raça nobre, entre os antigos Gregos. Cf. Latino, \textunderscore Or. da Corôa\textunderscore , CXLIII.
\section{Eupepsia}
\begin{itemize}
\item {Grp. gram.:f.}
\end{itemize}
\begin{itemize}
\item {Proveniência:(Do gr. \textunderscore eu\textunderscore  + \textunderscore pepsis\textunderscore )}
\end{itemize}
Facilidade de digestão.
\section{Eupéptico}
\begin{itemize}
\item {Grp. gram.:adj.}
\end{itemize}
Que facilita a digestão.
(Cp. \textunderscore eupepsia\textunderscore )
\section{Eupétala}
\begin{itemize}
\item {Grp. gram.:f.}
\end{itemize}
\begin{itemize}
\item {Proveniência:(Do gr. \textunderscore eu\textunderscore  + \textunderscore petalon\textunderscore )}
\end{itemize}
O mesmo que \textunderscore opala\textunderscore .
Planta laurínea, de grandes fôlhas.
\section{Euphemicamente}
\begin{itemize}
\item {Grp. gram.:adv.}
\end{itemize}
\begin{itemize}
\item {Proveniência:(De \textunderscore euphêmico\textunderscore )}
\end{itemize}
Com euphemismo.
\section{Euphêmico}
\begin{itemize}
\item {Grp. gram.:adj.}
\end{itemize}
\begin{itemize}
\item {Proveniência:(Do gr. \textunderscore eu\textunderscore  + \textunderscore phemi\textunderscore )}
\end{itemize}
Relativo ao euphemismo.
Em que há euphemismo: \textunderscore expressão euphêmica\textunderscore .
\section{Euphemismo}
\begin{itemize}
\item {Grp. gram.:m.}
\end{itemize}
\begin{itemize}
\item {Proveniência:(Gr. \textunderscore euphemismos\textunderscore )}
\end{itemize}
Suavização de uma palavra ou de uma ideia dura ou desagradável.
Acto ou maneira de disfarçar ideias tristes ou odiosas, por meio de termos que não correspondem precisamente a essas ideias.
\section{Euphonia}
\begin{itemize}
\item {Grp. gram.:f.}
\end{itemize}
\begin{itemize}
\item {Proveniência:(Do gr. \textunderscore eu\textunderscore  + \textunderscore phone\textunderscore )}
\end{itemize}
Som agradável de uma voz ou de um instrumento.
Suavidade ou elegância na pronúncia.
Aquillo que torna suave ou elegante a pronúncia.
\section{Euphonicamente}
\begin{itemize}
\item {Grp. gram.:adv.}
\end{itemize}
De modo euphónico.
\section{Euphónico}
\begin{itemize}
\item {Grp. gram.:adj.}
\end{itemize}
Em que há euphonia.
Suave, melodioso.
Que produz euphonia.
\section{Euphónio}
\begin{itemize}
\item {Grp. gram.:m.}
\end{itemize}
Espécie de harmónica, inventada em 1790 e também conhecida por \textunderscore euphono\textunderscore .
(Cp. \textunderscore euphono\textunderscore )
\section{Euphonizar}
\begin{itemize}
\item {Grp. gram.:v. t.}
\end{itemize}
\begin{itemize}
\item {Utilização:Neol.}
\end{itemize}
Tornar euphónico.
\section{Euphono}
\begin{itemize}
\item {Grp. gram.:adj.}
\end{itemize}
\begin{itemize}
\item {Grp. gram.:M.}
\end{itemize}
\begin{itemize}
\item {Proveniência:(Gr. \textunderscore euphonos\textunderscore )}
\end{itemize}
Que tem voz melodiosa.
Espécie de tangará.
Espécie de harmónica.
\section{Euphorbiáceas}
\begin{itemize}
\item {Grp. gram.:f. pl.}
\end{itemize}
\begin{itemize}
\item {Proveniência:(De \textunderscore euphorbiáceo\textunderscore )}
\end{itemize}
Família de plantas, que têm por typo o euphórbio.
\section{Euphorbiáceo}
\begin{itemize}
\item {Grp. gram.:adj.}
\end{itemize}
Que é da natureza do euphórbio.
Relativo ou semelhante a esta planta.
\section{Euphórbico}
\begin{itemize}
\item {Grp. gram.:adj.}
\end{itemize}
Diz-se de um ácido crystallizável, que se descobriu nas flôres e folhas do euphórbio.
\section{Euphorbina}
\begin{itemize}
\item {Grp. gram.:f.}
\end{itemize}
Substância, que se extrái da raíz do euphórbio.
\section{Euphórbio}
\begin{itemize}
\item {Grp. gram.:m.}
\end{itemize}
\begin{itemize}
\item {Proveniência:(Do gr. \textunderscore eu\textunderscore  + \textunderscore phorbe\textunderscore )}
\end{itemize}
Gênero de plantas, de suco acre e cáustico.
\section{Euphoria}
\begin{itemize}
\item {Grp. gram.:f.}
\end{itemize}
\begin{itemize}
\item {Utilização:Med.}
\end{itemize}
\begin{itemize}
\item {Proveniência:(Do gr. \textunderscore eu\textunderscore  + \textunderscore phoros\textunderscore )}
\end{itemize}
Sensação de bem-estar.
\section{Euphotide}
\begin{itemize}
\item {Grp. gram.:f.}
\end{itemize}
\begin{itemize}
\item {Proveniência:(Do gr. \textunderscore eu\textunderscore  + \textunderscore phos\textunderscore , \textunderscore photos\textunderscore )}
\end{itemize}
Espécie de rocha eruptiva.
\section{Euphotita}
\begin{itemize}
\item {Grp. gram.:f.}
\end{itemize}
\begin{itemize}
\item {Proveniência:(Do gr. \textunderscore eu\textunderscore  + \textunderscore phos\textunderscore , \textunderscore photos\textunderscore )}
\end{itemize}
Espécie de rocha eruptiva.
\section{Euphrásia}
\begin{itemize}
\item {Grp. gram.:f.}
\end{itemize}
\begin{itemize}
\item {Proveniência:(Gr. \textunderscore euphrasia\textunderscore )}
\end{itemize}
Planta medicinal, escrofularínea.
\section{Euphrático}
\begin{itemize}
\item {Grp. gram.:adj.}
\end{itemize}
Relativo ao Euphrates:«\textunderscore té que foi dar na euphrática ribeira.\textunderscore »Castilho, \textunderscore Fastos\textunderscore , I, 127.
\section{Euphtalmina}
\begin{itemize}
\item {Grp. gram.:f.}
\end{itemize}
Composto chímico, usado em Medicina, como succedâneo da atropina.
\section{Euphuísmo}
\begin{itemize}
\item {Grp. gram.:m.}
\end{itemize}
\begin{itemize}
\item {Proveniência:(Do gr. \textunderscore euphues\textunderscore )}
\end{itemize}
Estilo affectado, que se usou em Inglaterra, semelhante ao gongorismo que dominou em Portugal e Espanha, e ao preciosismo, que dominou em França.
\section{Euphuísta}
\begin{itemize}
\item {Grp. gram.:m.}
\end{itemize}
Aquelle que praticava o euphuismo.
\section{Euphuístico}
\begin{itemize}
\item {Grp. gram.:adj.}
\end{itemize}
\begin{itemize}
\item {Proveniência:(De \textunderscore euphuista\textunderscore )}
\end{itemize}
Relativo ao euphuísmo.
\section{Eupistéria}
\begin{itemize}
\item {Grp. gram.:f.}
\end{itemize}
\begin{itemize}
\item {Proveniência:(Do gr. \textunderscore eu\textunderscore  + \textunderscore pisterion\textunderscore )}
\end{itemize}
Gênero de insectos lepidópteros nocturnos.
\section{Euplástico}
\begin{itemize}
\item {Grp. gram.:adj.}
\end{itemize}
\begin{itemize}
\item {Utilização:Med.}
\end{itemize}
\begin{itemize}
\item {Proveniência:(Do gr. \textunderscore eu\textunderscore  + \textunderscore plassein\textunderscore )}
\end{itemize}
Relativo ás bôas fórmas plásticas.
\section{Euplero}
\begin{itemize}
\item {Grp. gram.:m.}
\end{itemize}
\begin{itemize}
\item {Proveniência:(Do gr. \textunderscore eu\textunderscore  + \textunderscore pleros\textunderscore )}
\end{itemize}
Mammífero insectívoro de Madagascar.
\section{Euplócamo}
\begin{itemize}
\item {Grp. gram.:adj.}
\end{itemize}
\begin{itemize}
\item {Grp. gram.:M.}
\end{itemize}
\begin{itemize}
\item {Proveniência:(Do gr. \textunderscore euplokamos\textunderscore )}
\end{itemize}
Que tem cabello fino e encaracolado.
Gênero de gallináceas.
\section{Eupnéa}
\begin{itemize}
\item {Grp. gram.:f.}
\end{itemize}
\begin{itemize}
\item {Proveniência:(Do gr. \textunderscore eu\textunderscore  + \textunderscore pnein\textunderscore )}
\end{itemize}
Facilidade de respiração.
\section{Eupneia}
\begin{itemize}
\item {Grp. gram.:f.}
\end{itemize}
\begin{itemize}
\item {Proveniência:(Do gr. \textunderscore eu\textunderscore  + \textunderscore pnein\textunderscore )}
\end{itemize}
Facilidade de respiração.
\section{Eupogonia}
\begin{itemize}
\item {Grp. gram.:f.}
\end{itemize}
\begin{itemize}
\item {Proveniência:(Do gr. \textunderscore eu\textunderscore  + \textunderscore pogon\textunderscore )}
\end{itemize}
Gênero de algas do Adriático.
\section{Euquimo}
\begin{itemize}
\item {Grp. gram.:m.}
\end{itemize}
\begin{itemize}
\item {Proveniência:(Do gr. \textunderscore eu\textunderscore  + \textunderscore khumos\textunderscore )}
\end{itemize}
Suco nutriente dos vegetaes.
\section{Euquinina}
\begin{itemize}
\item {Grp. gram.:f.}
\end{itemize}
Substância medicinal, que tem as mesmas applicações que a quinina.
\section{Eurema}
\begin{itemize}
\item {Grp. gram.:m.}
\end{itemize}
\begin{itemize}
\item {Utilização:Jur.}
\end{itemize}
\begin{itemize}
\item {Proveniência:(Gr. \textunderscore eurema\textunderscore )}
\end{itemize}
Prevenção para assegurar a validade de acto jurídico.
\section{Euremático}
\begin{itemize}
\item {Grp. gram.:adj.}
\end{itemize}
Relativo a eurema.
\section{Eurhytmia}
\begin{itemize}
\item {Grp. gram.:f.}
\end{itemize}
\begin{itemize}
\item {Utilização:Med.}
\end{itemize}
\begin{itemize}
\item {Proveniência:(Do gr. \textunderscore eu\textunderscore  + \textunderscore ruthmos\textunderscore )}
\end{itemize}
Regularidade, justa proporção, entre as partes de um todo.
Regularidade das pulsações.
\section{Eurhýtmico}
\begin{itemize}
\item {Grp. gram.:adj.}
\end{itemize}
Em que há eurhytmia.
\section{Euríala}
\begin{itemize}
\item {Grp. gram.:f.}
\end{itemize}
O mesmo que \textunderscore euríalo\textunderscore .
\section{Euriáleas}
\begin{itemize}
\item {Grp. gram.:f. pl.}
\end{itemize}
Tribo de zoófitos, que têm por tipo o euríalo.
\section{Euríalo}
\begin{itemize}
\item {Grp. gram.:m.}
\end{itemize}
\begin{itemize}
\item {Proveniência:(De \textunderscore Eurýalo\textunderscore , n. p.)}
\end{itemize}
Gênero de zoófitos medusários dos mares da Oceânia.
\section{Euriângio}
\begin{itemize}
\item {Grp. gram.:m.}
\end{itemize}
\begin{itemize}
\item {Utilização:Bot.}
\end{itemize}
Espécie de musgo.
\section{Euricefalia}
\begin{itemize}
\item {Grp. gram.:f.}
\end{itemize}
Estado ou qualidade de euricéfalo.
\section{Euricéfalo}
\begin{itemize}
\item {Grp. gram.:adj.}
\end{itemize}
\begin{itemize}
\item {Proveniência:(Do gr. \textunderscore eurus\textunderscore  + \textunderscore kephale\textunderscore )}
\end{itemize}
Que tem a cabeça larga.
\section{Eurícero}
\begin{itemize}
\item {Grp. gram.:adj.}
\end{itemize}
\begin{itemize}
\item {Utilização:Zool.}
\end{itemize}
\begin{itemize}
\item {Proveniência:(Do gr. \textunderscore eurus\textunderscore  + \textunderscore keras\textunderscore )}
\end{itemize}
Que tem cornos largos.
\section{Eurícoro}
\begin{itemize}
\item {Grp. gram.:m.}
\end{itemize}
\begin{itemize}
\item {Proveniência:(Gr. \textunderscore eurukhoros\textunderscore )}
\end{itemize}
Gênero de insectos coleópteros heterómeros.
\section{Eurignato}
\begin{itemize}
\item {Grp. gram.:adj.}
\end{itemize}
\begin{itemize}
\item {Proveniência:(Do gr. \textunderscore eurus\textunderscore , largo, e \textunderscore gnathos\textunderscore , maxilla)}
\end{itemize}
Diz-se do indivíduo ou do typo humano, em que sobresái a parte média da cabeça ou a região superior do rosto.
\section{Êurino}
\begin{itemize}
\item {Grp. gram.:adj.}
\end{itemize}
\begin{itemize}
\item {Grp. gram.:M.}
\end{itemize}
\begin{itemize}
\item {Proveniência:(Lat. \textunderscore eurinus\textunderscore )}
\end{itemize}
Relativo ao euro.
O mesmo que \textunderscore euro\textunderscore .
\section{Euriopse}
\begin{itemize}
\item {Grp. gram.:adj.}
\end{itemize}
\begin{itemize}
\item {Utilização:Anthrop.}
\end{itemize}
Diz-se da face, em que o diâmetro bizigomático predomina sobre a altura, segundo Quatrefages. Cf. E. Burnay, \textunderscore Craniol.\textunderscore , 142.
\section{Euripo}
\begin{itemize}
\item {Grp. gram.:m.}
\end{itemize}
\begin{itemize}
\item {Proveniência:(Gr. \textunderscore euripos\textunderscore )}
\end{itemize}
Movimento irregular.
Parte de um estreito, onde abundam escolhos, e há sempre agitação de ondas.
Fôsso, que, nos circos romanos, impedia que as feras passassem da arena para o lugar dos espectadores.
\section{Eurística}
\begin{itemize}
\item {Grp. gram.:f.}
\end{itemize}
Emprêgo do processo eurístico.
\section{Eurístico}
\begin{itemize}
\item {Grp. gram.:adj.}
\end{itemize}
Diz-se, em pedagogia, da fórma ou processo de encaminhar o alumno, a fim de que elle descubra a verdade que se lhe quer inculcar.
\section{Euritmia}
\begin{itemize}
\item {Grp. gram.:f.}
\end{itemize}
\begin{itemize}
\item {Utilização:Med.}
\end{itemize}
\begin{itemize}
\item {Proveniência:(Do gr. \textunderscore eu\textunderscore  + \textunderscore ruthmos\textunderscore )}
\end{itemize}
Regularidade, justa proporção, entre as partes de um todo.
Regularidade das pulsações.
\section{Eurítmico}
\begin{itemize}
\item {Grp. gram.:adj.}
\end{itemize}
Em que há euritmia.
\section{Euro}
\begin{itemize}
\item {Grp. gram.:m.}
\end{itemize}
\begin{itemize}
\item {Proveniência:(Lat. \textunderscore eurus\textunderscore )}
\end{itemize}
Vento do Nascente.
\section{Eurónoto}
\begin{itemize}
\item {Grp. gram.:m.}
\end{itemize}
\begin{itemize}
\item {Proveniência:(Lat. \textunderscore euronotus\textunderscore )}
\end{itemize}
Vento de Suéste, segundo a náutica grega e romana.
\section{Europeia}
(fem. de \textunderscore europeu\textunderscore )
\section{Europeísmo}
\begin{itemize}
\item {Grp. gram.:m.}
\end{itemize}
\begin{itemize}
\item {Utilização:Neol.}
\end{itemize}
Admiração das coisas europeias.
\section{Europeísta}
\begin{itemize}
\item {Grp. gram.:m.}
\end{itemize}
\begin{itemize}
\item {Utilização:Neol.}
\end{itemize}
Admirador das coisas europeias.
\section{Europeizar}
\begin{itemize}
\item {fónica:pê-i}
\end{itemize}
\begin{itemize}
\item {Grp. gram.:v. t.}
\end{itemize}
Tornar europeu; dar feição europeia a. Cf. Th. Ribeiro, \textunderscore Jornadas\textunderscore , II, 18.
\section{Europeu}
\begin{itemize}
\item {Grp. gram.:m.}
\end{itemize}
\begin{itemize}
\item {Grp. gram.:Adj.}
\end{itemize}
\begin{itemize}
\item {Proveniência:(Lat. \textunderscore europaeus\textunderscore )}
\end{itemize}
Aquele que é natural da Europa.
Relativo á Europa.
\section{Eurotemático}
\begin{itemize}
\item {Grp. gram.:adj.}
\end{itemize}
Diz-se do processo pedagógico, em que o professor interrompe a prelecção para interrogar o aluno.
\section{Eurothemático}
\begin{itemize}
\item {Grp. gram.:adj.}
\end{itemize}
Diz-se do processo pedagógico, em que o professor interrompe a prelecção para interrogar o alumno.
\section{Eurreta}
\begin{itemize}
\item {fónica:rê}
\end{itemize}
\begin{itemize}
\item {Grp. gram.:f.}
\end{itemize}
\begin{itemize}
\item {Utilização:Prov.}
\end{itemize}
\begin{itemize}
\item {Utilização:trasm.}
\end{itemize}
Planície entre montes.
\section{Eurýala}
\begin{itemize}
\item {Grp. gram.:f.}
\end{itemize}
O mesmo que \textunderscore eurýalo\textunderscore .
\section{Euryáleas}
\begin{itemize}
\item {Grp. gram.:f. pl.}
\end{itemize}
Tribo de zoóphytos, que têm por typo o eurýalo.
\section{Eurýalo}
\begin{itemize}
\item {Grp. gram.:m.}
\end{itemize}
\begin{itemize}
\item {Proveniência:(De \textunderscore Eurýalo\textunderscore , n. p.)}
\end{itemize}
Gênero de zoóphytos medusários dos mares da Oceânia.
\section{Euryângio}
\begin{itemize}
\item {Grp. gram.:m.}
\end{itemize}
\begin{itemize}
\item {Utilização:Bot.}
\end{itemize}
Espécie de musgo.
\section{Eurycephalia}
\begin{itemize}
\item {Grp. gram.:f.}
\end{itemize}
Estado ou qualidade de eurycéphalo.
\section{Eurycéphalo}
\begin{itemize}
\item {Grp. gram.:adj.}
\end{itemize}
\begin{itemize}
\item {Proveniência:(Do gr. \textunderscore eurus\textunderscore  + \textunderscore kephale\textunderscore )}
\end{itemize}
Que tem a cabeça larga.
\section{Eurýcero}
\begin{itemize}
\item {Grp. gram.:adj.}
\end{itemize}
\begin{itemize}
\item {Utilização:Zool.}
\end{itemize}
\begin{itemize}
\item {Proveniência:(Do gr. \textunderscore eurus\textunderscore  + \textunderscore keras\textunderscore )}
\end{itemize}
Que tem cornos largos.
\section{Eurýchoro}
\begin{itemize}
\item {fónica:co}
\end{itemize}
\begin{itemize}
\item {Grp. gram.:m.}
\end{itemize}
\begin{itemize}
\item {Proveniência:(Gr. \textunderscore eurukhoros\textunderscore )}
\end{itemize}
Gênero de insectos coleópteros heterómeros.
\section{Eurýgnatho}
\begin{itemize}
\item {Grp. gram.:adj.}
\end{itemize}
\begin{itemize}
\item {Proveniência:(Do gr. \textunderscore eurus\textunderscore , largo, e \textunderscore gnathos\textunderscore , maxilla)}
\end{itemize}
Diz-se do indivíduo ou do typo humano, em que sobresái a parte média da cabeça ou a região superior do rosto.
\section{Euryopse}
\begin{itemize}
\item {Grp. gram.:adj.}
\end{itemize}
\begin{itemize}
\item {Utilização:Anthrop.}
\end{itemize}
Diz-se da face, em que o diâmetro bizygomático predomina sobre a altura, segundo Quatrefages. Cf. E. Burnay, \textunderscore Craniol.\textunderscore , 142.
\section{Euríptero}
\begin{itemize}
\item {Grp. gram.:m.}
\end{itemize}
\begin{itemize}
\item {Proveniência:(Do gr. \textunderscore eurus\textunderscore  + \textunderscore pteron\textunderscore )}
\end{itemize}
Gênero de insectos coleópteros tetrâmeros.
Gênero de crustáceos.
Gênero de plantas umbelíferas.
\section{Eurístomo}
\begin{itemize}
\item {Grp. gram.:adj.}
\end{itemize}
\begin{itemize}
\item {Utilização:Zool.}
\end{itemize}
\begin{itemize}
\item {Proveniência:(Do gr. \textunderscore eurus\textunderscore  + \textunderscore stoma\textunderscore )}
\end{itemize}
Que tem boca larga.
\section{Euritermes}
\begin{itemize}
\item {Grp. gram.:m. pl.}
\end{itemize}
\begin{itemize}
\item {Utilização:Zool.}
\end{itemize}
\begin{itemize}
\item {Proveniência:(Do gr. \textunderscore eurus\textunderscore  + \textunderscore therme\textunderscore )}
\end{itemize}
Animaes, que suportam sem sofrimento as variações da temperatura.
\section{Eurýptero}
\begin{itemize}
\item {Grp. gram.:m.}
\end{itemize}
\begin{itemize}
\item {Proveniência:(Do gr. \textunderscore eurus\textunderscore  + \textunderscore pteron\textunderscore )}
\end{itemize}
Gênero de insectos coleópteros tetrâmeros.
Gênero de crustáceos.
Gênero de plantas umbellíferas.
\section{Eurýstomo}
\begin{itemize}
\item {Grp. gram.:adj.}
\end{itemize}
\begin{itemize}
\item {Utilização:Zool.}
\end{itemize}
\begin{itemize}
\item {Proveniência:(Do gr. \textunderscore eurus\textunderscore  + \textunderscore stoma\textunderscore )}
\end{itemize}
Que tem boca larga.
\section{Eurythermes}
\begin{itemize}
\item {Grp. gram.:m. pl.}
\end{itemize}
\begin{itemize}
\item {Utilização:Zool.}
\end{itemize}
\begin{itemize}
\item {Proveniência:(Do gr. \textunderscore eurus\textunderscore  + \textunderscore therme\textunderscore )}
\end{itemize}
Animaes, que supportam sem soffrimento as variações da temperatura.
\section{Eusemia}
\begin{itemize}
\item {fónica:se}
\end{itemize}
\begin{itemize}
\item {Grp. gram.:f.}
\end{itemize}
\begin{itemize}
\item {Proveniência:(Do gr. \textunderscore eu\textunderscore  + \textunderscore sema\textunderscore )}
\end{itemize}
Bons symptomas de uma enfermidade.
\section{Eussemia}
\begin{itemize}
\item {Grp. gram.:f.}
\end{itemize}
\begin{itemize}
\item {Proveniência:(Do gr. \textunderscore eu\textunderscore  + \textunderscore sema\textunderscore )}
\end{itemize}
Bons symptomas de uma enfermidade.
\section{Eustilo}
\begin{itemize}
\item {Grp. gram.:m.}
\end{itemize}
\begin{itemize}
\item {Utilização:Archit.}
\end{itemize}
\begin{itemize}
\item {Proveniência:(Do gr. \textunderscore eu\textunderscore  + \textunderscore stulos\textunderscore )}
\end{itemize}
Espaço de dois diâmetros entre colunas.
Conjunto de colunas bem ordenadas.
\section{Eustomia}
\begin{itemize}
\item {Grp. gram.:f.}
\end{itemize}
\begin{itemize}
\item {Utilização:Gram.}
\end{itemize}
\begin{itemize}
\item {Proveniência:(Do gr. \textunderscore eu\textunderscore  + \textunderscore stoma\textunderscore )}
\end{itemize}
Facilidade em pronunciar.
\section{Eustylo}
\begin{itemize}
\item {Grp. gram.:m.}
\end{itemize}
\begin{itemize}
\item {Utilização:Archit.}
\end{itemize}
\begin{itemize}
\item {Proveniência:(Do gr. \textunderscore eu\textunderscore  + \textunderscore stulos\textunderscore )}
\end{itemize}
Espaço de dois diâmetros entre columnas.
Conjunto de columnas bem ordenadas.
\section{Eutacta}
\begin{itemize}
\item {Grp. gram.:f.}
\end{itemize}
Gênero de plantas abietíneas.
\section{Eutaxia}
\begin{itemize}
\item {fónica:csi}
\end{itemize}
\begin{itemize}
\item {Grp. gram.:f.}
\end{itemize}
\begin{itemize}
\item {Proveniência:(Do gr. \textunderscore eu\textunderscore  + \textunderscore taxis\textunderscore )}
\end{itemize}
Justa proporção entre as differentes partes do corpo ou do organismo animal.
\section{Eu-te-rógo-barqueiro}
\begin{itemize}
\item {Grp. gram.:m.}
\end{itemize}
Espécie de jôgo popular.
\section{Euterpe}
\begin{itemize}
\item {Grp. gram.:m.}
\end{itemize}
\begin{itemize}
\item {Proveniência:(Do gr. \textunderscore euterpes\textunderscore )}
\end{itemize}
Gênero de palmeiras.
\section{Euthanasia}
\begin{itemize}
\item {Grp. gram.:f.}
\end{itemize}
\begin{itemize}
\item {Proveniência:(Do gr. \textunderscore eu\textunderscore  + \textunderscore thanatos\textunderscore )}
\end{itemize}
Morte tranquilla, sem soffrimento.
\section{Euthýcomo}
\begin{itemize}
\item {Grp. gram.:adj.}
\end{itemize}
\begin{itemize}
\item {Proveniência:(Do gr. \textunderscore euthus\textunderscore , direito, e \textunderscore kome\textunderscore , cabello)}
\end{itemize}
Que tem o cabello grosso, comprido e pendente.
\section{Euthymia}
\begin{itemize}
\item {Grp. gram.:f.}
\end{itemize}
\begin{itemize}
\item {Proveniência:(Do gr. \textunderscore eu\textunderscore  + \textunderscore thumos\textunderscore )}
\end{itemize}
Tranquillidade de espirito.
\section{Eutícomo}
\begin{itemize}
\item {Grp. gram.:adj.}
\end{itemize}
\begin{itemize}
\item {Proveniência:(Do gr. \textunderscore euthus\textunderscore , direito, e \textunderscore kome\textunderscore , cabelo)}
\end{itemize}
Que tem o cabelo grosso, comprido e pendente.
\section{Eutimia}
\begin{itemize}
\item {Grp. gram.:f.}
\end{itemize}
\begin{itemize}
\item {Proveniência:(Do gr. \textunderscore eu\textunderscore  + \textunderscore thumos\textunderscore )}
\end{itemize}
Tranquillidade de espirito.
\section{Eutiquianismo}
\begin{itemize}
\item {Grp. gram.:m.}
\end{itemize}
Seita herética dos eutiquianos.
\section{Eutiquianos}
\begin{itemize}
\item {Grp. gram.:m. pl.}
\end{itemize}
Sectários do heresiarca Eutiques.
\section{Eutoca}
\begin{itemize}
\item {Grp. gram.:f.}
\end{itemize}
\begin{itemize}
\item {Proveniência:(Do gr. \textunderscore eutokos\textunderscore )}
\end{itemize}
Gênero de plantas da América boreal e da África austral.
\section{Eutocia}
\begin{itemize}
\item {Grp. gram.:f.}
\end{itemize}
\begin{itemize}
\item {Utilização:Med.}
\end{itemize}
\begin{itemize}
\item {Proveniência:(Do gr. \textunderscore eu\textunderscore  + \textunderscore tokos\textunderscore )}
\end{itemize}
Parto bom, normal.
\section{Eutrapelia}
\begin{itemize}
\item {Grp. gram.:f.}
\end{itemize}
\begin{itemize}
\item {Proveniência:(De \textunderscore eutrapelo\textunderscore )}
\end{itemize}
Qualidade daquillo que é eutrapélico.
\section{Eutrapélico}
\begin{itemize}
\item {Grp. gram.:adj.}
\end{itemize}
\begin{itemize}
\item {Utilização:Des.}
\end{itemize}
\begin{itemize}
\item {Proveniência:(De \textunderscore eutrapelia\textunderscore )}
\end{itemize}
Gracioso.
Chistoso; mordaz.
\section{Eutrapelo}
\begin{itemize}
\item {Grp. gram.:adj.}
\end{itemize}
\begin{itemize}
\item {Proveniência:(Gr. \textunderscore eutrapelos\textunderscore )}
\end{itemize}
O mesmo que \textunderscore eutrapélico\textunderscore .
\section{Eutriana}
\begin{itemize}
\item {Grp. gram.:f.}
\end{itemize}
\begin{itemize}
\item {Proveniência:(Do gr. \textunderscore eu\textunderscore  + \textunderscore triaina\textunderscore )}
\end{itemize}
Planta gramínea da América tropical.
\section{Eutriena}
\begin{itemize}
\item {Grp. gram.:f.}
\end{itemize}
\begin{itemize}
\item {Proveniência:(Do gr. \textunderscore eu\textunderscore  + \textunderscore triaina\textunderscore )}
\end{itemize}
Planta gramínea da América tropical.
\section{Eutrofia}
\begin{itemize}
\item {Grp. gram.:f.}
\end{itemize}
\begin{itemize}
\item {Proveniência:(Do gr. \textunderscore eu\textunderscore  + \textunderscore trophe\textunderscore )}
\end{itemize}
Bôa nutrição.
\section{Eutrophia}
\begin{itemize}
\item {Grp. gram.:f.}
\end{itemize}
\begin{itemize}
\item {Proveniência:(Do gr. \textunderscore eu\textunderscore  + \textunderscore trophe\textunderscore )}
\end{itemize}
Bôa nutrição.
\section{Eutychianismo}
\begin{itemize}
\item {fónica:qui}
\end{itemize}
\begin{itemize}
\item {Grp. gram.:m.}
\end{itemize}
Seita herética dos eutychianos.
\section{Eutychianos}
\begin{itemize}
\item {fónica:qui}
\end{itemize}
\begin{itemize}
\item {Grp. gram.:m. pl.}
\end{itemize}
Sectários do heresiarca Eutyches.
\section{Euvites}
\begin{itemize}
\item {Grp. gram.:f. pl.}
\end{itemize}
Uma das secções, em que, segundo alguns ampelógraphos, se dividem as videiras.
\section{Euxênia}
\begin{itemize}
\item {fónica:csê}
\end{itemize}
\begin{itemize}
\item {Grp. gram.:f.}
\end{itemize}
\begin{itemize}
\item {Proveniência:(Do gr. \textunderscore eu\textunderscore  + \textunderscore xenos\textunderscore )}
\end{itemize}
Gênero de arbustos do Chile.
\section{Evacuação}
\begin{itemize}
\item {Grp. gram.:f.}
\end{itemize}
\begin{itemize}
\item {Proveniência:(Lat. \textunderscore evacuatio\textunderscore )}
\end{itemize}
Acto de evacuar.
Acto de sair de uma praça que estava occupada militarmente.
Acto de expellir os excrementos.
Matérias evacuadas.
\section{Evacuante}
\begin{itemize}
\item {Grp. gram.:m.  e  adj.}
\end{itemize}
\begin{itemize}
\item {Proveniência:(Lat. \textunderscore evacuans\textunderscore )}
\end{itemize}
Aquillo que produz evacuação.
\section{Evacuar}
\begin{itemize}
\item {Grp. gram.:v. t.}
\end{itemize}
\begin{itemize}
\item {Grp. gram.:V. i.}
\end{itemize}
\begin{itemize}
\item {Proveniência:(Lat. \textunderscore evacuare\textunderscore )}
\end{itemize}
Sair de.
Deixar livre, vazio.
Desoccupar: \textunderscore evacuar um theatro\textunderscore .
Fazer expellir (excrementos).
Expellir os excrementos; defecar.
\section{Evacuativo}
\begin{itemize}
\item {Grp. gram.:adj.}
\end{itemize}
O mesmo que \textunderscore evacuante\textunderscore .
\section{Evacuatório}
\begin{itemize}
\item {Grp. gram.:adj.}
\end{itemize}
O mesmo que \textunderscore evacuante\textunderscore .
\section{Evadir}
\begin{itemize}
\item {Grp. gram.:v. t.}
\end{itemize}
\begin{itemize}
\item {Grp. gram.:V. p.}
\end{itemize}
\begin{itemize}
\item {Grp. gram.:V. i.}
\end{itemize}
\begin{itemize}
\item {Proveniência:(Lat. \textunderscore evadere\textunderscore )}
\end{itemize}
Desviar.
Escapar de: \textunderscore evadir perigos\textunderscore .
Fugir ás occultas; escapar-se furtivamente: \textunderscore evadir-se da cadeia\textunderscore .
Desapparecer.
(A mesma significação). Cf. Filinto, \textunderscore D. Man.\textunderscore , I, 171.
\section{Evagação}
\begin{itemize}
\item {Grp. gram.:f.}
\end{itemize}
\begin{itemize}
\item {Proveniência:(Lat. \textunderscore evagatio\textunderscore )}
\end{itemize}
Divagação, distracção.
\section{Evalvo}
\begin{itemize}
\item {Grp. gram.:adj.}
\end{itemize}
\begin{itemize}
\item {Utilização:Bot.}
\end{itemize}
\begin{itemize}
\item {Proveniência:(Do lat. \textunderscore e\textunderscore  + \textunderscore valva\textunderscore )}
\end{itemize}
Indehiscente.
\section{Evanescente}
\begin{itemize}
\item {Grp. gram.:adj.}
\end{itemize}
\begin{itemize}
\item {Proveniência:(Lat. \textunderscore evanescens\textunderscore )}
\end{itemize}
Que se esvanece.
\section{Evangelho}
\begin{itemize}
\item {Grp. gram.:m.}
\end{itemize}
\begin{itemize}
\item {Utilização:Fig.}
\end{itemize}
\begin{itemize}
\item {Proveniência:(Lat. \textunderscore evangelium\textunderscore )}
\end{itemize}
Doutrina de Christo.
Cada um dos 4 livros principaes do \textunderscore Novo Testamento\textunderscore , (o Evangelho de San-Matheus, o de San-Lucas, o de San-Marcos e o de San-João).
Parte do Evangelho, que se lê na Missa.
Coisa digna de inteiro crédito: \textunderscore acredita naquillo, como num Evangelho\textunderscore .
Conjunto de princípios, por que se regula uma seita, um partido ou um sectário.
\section{Evangeliário}
\begin{itemize}
\item {Grp. gram.:m.}
\end{itemize}
\begin{itemize}
\item {Proveniência:(Do lat. \textunderscore evangelium\textunderscore )}
\end{itemize}
Livro, que contém fragmentos dos Evangelhos para a Missa de cada dia.
\section{Evangélias}
\begin{itemize}
\item {Grp. gram.:f. pl.}
\end{itemize}
\begin{itemize}
\item {Proveniência:(Lat. \textunderscore evangelia\textunderscore )}
\end{itemize}
Antigas festas pagans, que se celebravam por occasião de bôas novas ou de notícias agradáveis ao povo.
\section{Evangelical}
\begin{itemize}
\item {Grp. gram.:adj.}
\end{itemize}
\begin{itemize}
\item {Utilização:Des.}
\end{itemize}
\begin{itemize}
\item {Proveniência:(De \textunderscore evangélico\textunderscore )}
\end{itemize}
Relativo aos Evangelhos:«\textunderscore ...o Livro da vida de Jhesu Christo... Foe tirado e ordenado segundo ha ordem da estoria evangelical.\textunderscore »Bern. de Alcob., \textunderscore Vita Christi\textunderscore .
\section{Evangelicamente}
\begin{itemize}
\item {Grp. gram.:adv.}
\end{itemize}
De modo evangélico.
\section{Evangélico}
\begin{itemize}
\item {Grp. gram.:adj.}
\end{itemize}
\begin{itemize}
\item {Proveniência:(Lat. \textunderscore evangelicus\textunderscore )}
\end{itemize}
Relativo ao Evangelho.
Conforme aos princípios do Evangelho: \textunderscore paciência evangélica\textunderscore .
\section{Evangélio}
\begin{itemize}
\item {Grp. gram.:m.}
\end{itemize}
\begin{itemize}
\item {Utilização:T. de Ceilão}
\end{itemize}
O mesmo que \textunderscore Evangelho\textunderscore .
\section{Evangelismo}
\begin{itemize}
\item {Grp. gram.:m.}
\end{itemize}
\begin{itemize}
\item {Proveniência:(Do lat. \textunderscore evangelium\textunderscore )}
\end{itemize}
Doutrina política e religiosa, baseada no Evangelho.
\section{Evangelista}
\begin{itemize}
\item {Grp. gram.:m.}
\end{itemize}
\begin{itemize}
\item {Utilização:Fig.}
\end{itemize}
\begin{itemize}
\item {Proveniência:(Do lat. \textunderscore evangelium\textunderscore )}
\end{itemize}
Autor de um dos quatro Evangelhos.
Sacerdote, que canta o Evangelho na Missa.
Preconizador de uma doutrina ou systema.
\section{Evangelização}
\begin{itemize}
\item {Grp. gram.:f.}
\end{itemize}
Acto de evangelizar.
\section{Evangelizador}
\begin{itemize}
\item {Grp. gram.:adj.}
\end{itemize}
\begin{itemize}
\item {Grp. gram.:M.}
\end{itemize}
\begin{itemize}
\item {Proveniência:(Lat. \textunderscore evangelizator\textunderscore )}
\end{itemize}
Que evangeliza.
Aquelle que evangeliza.
Evangelista.
\section{Evangelizante}
\begin{itemize}
\item {Grp. gram.:adj.}
\end{itemize}
Que evangeliza.
\section{Evangelizar}
\begin{itemize}
\item {Grp. gram.:v. t.}
\end{itemize}
\begin{itemize}
\item {Proveniência:(Do lat. \textunderscore evangelizare\textunderscore )}
\end{itemize}
Divulgar, prègando.
Missionar.
Apostolar.
Preconizar.
\section{Evânia}
\begin{itemize}
\item {Grp. gram.:f.}
\end{itemize}
\begin{itemize}
\item {Proveniência:(Do gr. \textunderscore evanios\textunderscore )}
\end{itemize}
Gênero de insectos hymenópteros.
\section{Évano}
\begin{itemize}
\item {Grp. gram.:m.}
\end{itemize}
\begin{itemize}
\item {Utilização:Des.}
\end{itemize}
O mesmo que \textunderscore ébano\textunderscore . Cf. Filinto, I, 20.
\section{Evaporação}
\begin{itemize}
\item {Grp. gram.:f.}
\end{itemize}
\begin{itemize}
\item {Proveniência:(Lat. \textunderscore evaporatio\textunderscore )}
\end{itemize}
Acto de evaporar.
Exhalação.
\section{Evaporar}
\begin{itemize}
\item {Grp. gram.:v. t.}
\end{itemize}
\begin{itemize}
\item {Utilização:Fig.}
\end{itemize}
\begin{itemize}
\item {Grp. gram.:V. p.}
\end{itemize}
\begin{itemize}
\item {Utilização:Fig.}
\end{itemize}
\begin{itemize}
\item {Proveniência:(Lat. \textunderscore evaporare\textunderscore )}
\end{itemize}
Converter em vapor: \textunderscore o calor evaporou a água\textunderscore .
Fazer desapparecer.
Dissipar.
Exhalar.
Converter-se em vapor.
Desvanecer-se; desfazer-se.
Desapparecer; evolar-se: \textunderscore evaporou-se a esperança\textunderscore .
\section{Evaporativo}
\begin{itemize}
\item {Grp. gram.:adj.}
\end{itemize}
\begin{itemize}
\item {Proveniência:(Lat. \textunderscore evaporativus\textunderscore )}
\end{itemize}
Que produz ou facilita a evaporação.
\section{Evaporatório}
\begin{itemize}
\item {Grp. gram.:adj.}
\end{itemize}
\begin{itemize}
\item {Grp. gram.:M.}
\end{itemize}
\begin{itemize}
\item {Proveniência:(De \textunderscore evaporar\textunderscore )}
\end{itemize}
Evaporativo.
Orifício, por onde sái o vapor.
Apparelho, para facilitar a evaporação.
\section{Evaporável}
\begin{itemize}
\item {Grp. gram.:adj.}
\end{itemize}
Que se póde evaporar.
\section{Evaporímetro}
\begin{itemize}
\item {Grp. gram.:m.}
\end{itemize}
(V.evaporómetro)
\section{Evaporizar}
\begin{itemize}
\item {Grp. gram.:v. t.}
\end{itemize}
(V.evaporar)
\section{Evaporómetro}
\begin{itemize}
\item {Grp. gram.:m.}
\end{itemize}
\begin{itemize}
\item {Proveniência:(T. hybr., de \textunderscore evaporar\textunderscore  + gr. \textunderscore metron\textunderscore )}
\end{itemize}
Apparelho meteorológico, que mede a evaporação pela differença de nível de uma determinada superfície de água.
\section{Evasão}
\begin{itemize}
\item {Grp. gram.:f.}
\end{itemize}
\begin{itemize}
\item {Utilização:Fig.}
\end{itemize}
\begin{itemize}
\item {Proveniência:(Lat. \textunderscore evasio\textunderscore )}
\end{itemize}
Acto de evadir-se.
Evasiva, subterfúgio.
\section{Evasiva}
\begin{itemize}
\item {Grp. gram.:f.}
\end{itemize}
\begin{itemize}
\item {Proveniência:(De \textunderscore evasivo\textunderscore )}
\end{itemize}
Subterfúgio.
Desculpa ardilosa.
Escapatória.
\section{Evasivamente}
\begin{itemize}
\item {Grp. gram.:adv.}
\end{itemize}
De modo evasivo.
\section{Evasivo}
\begin{itemize}
\item {Grp. gram.:adj.}
\end{itemize}
\begin{itemize}
\item {Proveniência:(Do lat. \textunderscore evasus\textunderscore )}
\end{itemize}
Que facilita a evasão.
Argucioso.
Que serve de subterfúgio: \textunderscore resposta evasiva\textunderscore .
\section{Evazar}
\begin{itemize}
\item {Grp. gram.:v. t.}
\end{itemize}
\begin{itemize}
\item {Proveniência:(De \textunderscore vazar\textunderscore )}
\end{itemize}
Vazar, tornar oco.
Brocar. Cf. Filinto, VIII, 81.
\section{Evecção}
\begin{itemize}
\item {Grp. gram.:f.}
\end{itemize}
\begin{itemize}
\item {Proveniência:(Lat. \textunderscore evectio\textunderscore )}
\end{itemize}
Desigualdade no movimento da Lua, em resultado da attracção solar.
\section{Evelina}
\begin{itemize}
\item {Grp. gram.:f.}
\end{itemize}
\begin{itemize}
\item {Proveniência:(De \textunderscore Evelyn\textunderscore , n. p.)}
\end{itemize}
Planta, da fam. das orchídeas, originária do Peru.
\section{Evemerismo}
\begin{itemize}
\item {Grp. gram.:m.}
\end{itemize}
\begin{itemize}
\item {Proveniência:(De \textunderscore Evêmero\textunderscore , n. p.)}
\end{itemize}
Systema philosóphico, que sustenta que os deuses mythológicos foram personagens humanas, divinizadas pelos homens.
\section{Evencer}
\begin{itemize}
\item {Grp. gram.:v. t.}
\end{itemize}
\begin{itemize}
\item {Proveniência:(Lat. \textunderscore evincere\textunderscore )}
\end{itemize}
Desapossar, despojar.
\section{Evento}
\begin{itemize}
\item {Grp. gram.:m.}
\end{itemize}
\begin{itemize}
\item {Proveniência:(Do lat. \textunderscore eventus\textunderscore )}
\end{itemize}
Acontecimento, successo.
Eventualidade.
\section{Eventração}
\begin{itemize}
\item {Grp. gram.:f.}
\end{itemize}
\begin{itemize}
\item {Utilização:Med.}
\end{itemize}
\begin{itemize}
\item {Proveniência:(Fr. \textunderscore éventration\textunderscore )}
\end{itemize}
Hérnia, nas paredes abdominaes, em resultado de uma abertura accidental.
Ferida penetrante do abdome, dando saída a uma porção de vísceras.
\section{Eventual}
\begin{itemize}
\item {Grp. gram.:adj.}
\end{itemize}
\begin{itemize}
\item {Proveniência:(De \textunderscore evento\textunderscore )}
\end{itemize}
Contingente, dependente de acontecimento incerto.
Casual, fortuito.
\section{Eventualidade}
\begin{itemize}
\item {Grp. gram.:f.}
\end{itemize}
Contingência.
Acaso.
Possibilidade.
Acontecimento incerto.
Carácter daquillo que é eventual.
\section{Eventualmente}
\begin{itemize}
\item {Grp. gram.:adv.}
\end{itemize}
De modo eventual.
\section{Eversão}
\begin{itemize}
\item {Grp. gram.:f.}
\end{itemize}
\begin{itemize}
\item {Proveniência:(Lat. \textunderscore eversio\textunderscore )}
\end{itemize}
Destruição.
Subversão; desmoronamento.
Reviramento para fóra.
\section{Eversivo}
\begin{itemize}
\item {Grp. gram.:adj.}
\end{itemize}
\begin{itemize}
\item {Proveniência:(Do lat. \textunderscore eversus\textunderscore )}
\end{itemize}
Que destrói, que subverte.
\section{Eversor}
\begin{itemize}
\item {Grp. gram.:adj.}
\end{itemize}
\begin{itemize}
\item {Utilização:Des.}
\end{itemize}
\begin{itemize}
\item {Proveniência:(Lat. \textunderscore eversor\textunderscore )}
\end{itemize}
Destruidor.
\section{Evicção}
\begin{itemize}
\item {Grp. gram.:f.}
\end{itemize}
\begin{itemize}
\item {Utilização:Jur.}
\end{itemize}
\begin{itemize}
\item {Proveniência:(Lat. \textunderscore evictio\textunderscore )}
\end{itemize}
Acto de recuperar o que outrem adquirira illegitimamente.
\section{Evicto}
\begin{itemize}
\item {Grp. gram.:m.}
\end{itemize}
\begin{itemize}
\item {Grp. gram.:Adj.}
\end{itemize}
\begin{itemize}
\item {Proveniência:(Lat. \textunderscore evictus\textunderscore )}
\end{itemize}
Aquelle que é obrigado á evicção.
Sujeito á evicção.
\section{Evictor}
\begin{itemize}
\item {Grp. gram.:m.}
\end{itemize}
\begin{itemize}
\item {Proveniência:(De \textunderscore evicto\textunderscore )}
\end{itemize}
Aquelle que faz ou intenta evicção.
\section{Evidência}
\begin{itemize}
\item {Grp. gram.:f.}
\end{itemize}
\begin{itemize}
\item {Proveniência:(Lat. \textunderscore evidentia\textunderscore )}
\end{itemize}
Qualidade daquillo que é evidente, que é incontestável, que todos vêem ou podem vêr e verificar.
\section{Evidenciar}
\begin{itemize}
\item {Grp. gram.:v. t.}
\end{itemize}
\begin{itemize}
\item {Grp. gram.:V. p.}
\end{itemize}
\begin{itemize}
\item {Proveniência:(De \textunderscore evidência\textunderscore )}
\end{itemize}
Tornar evidente; demonstrar.
Mostrar-se claramente, patentear-se.
\section{Evidente}
\begin{itemize}
\item {Grp. gram.:adj.}
\end{itemize}
\begin{itemize}
\item {Proveniência:(Lat. \textunderscore evidens\textunderscore )}
\end{itemize}
Que se comprehende sem difficuldade nenhuma.
Que não offerece dúvidas.
Claro; demonstrado.
\section{Evidentemente}
\begin{itemize}
\item {Grp. gram.:adv.}
\end{itemize}
De modo evidente.
\section{Evisceração}
\begin{itemize}
\item {Grp. gram.:f.}
\end{itemize}
\begin{itemize}
\item {Proveniência:(Lat. \textunderscore evisceratio\textunderscore )}
\end{itemize}
O mesmo que \textunderscore eventração\textunderscore ; acto de eviscerar.
\section{Eviscerar}
\begin{itemize}
\item {Grp. gram.:v. t.}
\end{itemize}
\begin{itemize}
\item {Proveniência:(Lat. \textunderscore eviscerare\textunderscore )}
\end{itemize}
Tirar as vísceras a; estripar.
\section{Evitação}
\begin{itemize}
\item {Grp. gram.:f.}
\end{itemize}
Acto de evitar.
\section{Evitamento}
\begin{itemize}
\item {Grp. gram.:m.}
\end{itemize}
O mesmo que \textunderscore evitação\textunderscore .
\section{Evitar}
\begin{itemize}
\item {Grp. gram.:v. t.}
\end{itemize}
\begin{itemize}
\item {Proveniência:(Lat. \textunderscore evitare\textunderscore )}
\end{itemize}
Evadir; desviar-se de, fugir a: \textunderscore evitar uma cilada\textunderscore .
Impedir: \textunderscore evitar um crime\textunderscore .
Escusar.
\section{Evitável}
\begin{itemize}
\item {Grp. gram.:adj.}
\end{itemize}
\begin{itemize}
\item {Proveniência:(Lat. \textunderscore evitabilis\textunderscore )}
\end{itemize}
Que se póde ou se deve evitar.
\section{Eviternidade}
\begin{itemize}
\item {Grp. gram.:f.}
\end{itemize}
Qualidade daquillo que é eviterno.
\section{Eviterno}
\begin{itemize}
\item {Grp. gram.:adj.}
\end{itemize}
\begin{itemize}
\item {Proveniência:(Lat. \textunderscore aeviternus\textunderscore )}
\end{itemize}
Que não há de têr fim.
\section{Evo}
\begin{itemize}
\item {Grp. gram.:m.}
\end{itemize}
\begin{itemize}
\item {Proveniência:(Lat. \textunderscore aevum\textunderscore )}
\end{itemize}
Duração sem fim.
Eternidade.
Século.
\section{Evocação}
\begin{itemize}
\item {Grp. gram.:f.}
\end{itemize}
\begin{itemize}
\item {Proveniência:(Lat. \textunderscore evocatio\textunderscore )}
\end{itemize}
Acto de evocar.
\section{Evocador}
\begin{itemize}
\item {Grp. gram.:m.  e  adj.}
\end{itemize}
O que evoca.
\section{Evocante}
\begin{itemize}
\item {Grp. gram.:adj.}
\end{itemize}
\begin{itemize}
\item {Proveniência:(Lat. \textunderscore evocans\textunderscore )}
\end{itemize}
Que evoca.
\section{Evocar}
\begin{itemize}
\item {Grp. gram.:v. t.}
\end{itemize}
\begin{itemize}
\item {Utilização:Fig.}
\end{itemize}
\begin{itemize}
\item {Proveniência:(Lat. \textunderscore evocare\textunderscore )}
\end{itemize}
Chamar de algum lugar.
Invocar.
Esconjurar.
Chamar, para que appareçam, (almas, demónios, etc.).
Avocar, transferir de um tribunal para outro (uma causa).
Trazer á lembrança, á imaginação: \textunderscore evocar tempos idos\textunderscore .
\section{Evocativo}
\begin{itemize}
\item {Grp. gram.:adj.}
\end{itemize}
\begin{itemize}
\item {Proveniência:(Lat. \textunderscore evocativus\textunderscore )}
\end{itemize}
O mesmo que \textunderscore evocatório\textunderscore .
\section{Evocatório}
\begin{itemize}
\item {Grp. gram.:adj.}
\end{itemize}
\begin{itemize}
\item {Proveniência:(Lat. \textunderscore evocatorius\textunderscore )}
\end{itemize}
Que serve para evocar.
\section{Evocável}
\begin{itemize}
\item {Grp. gram.:adj.}
\end{itemize}
Que se póde evocar.
\section{Evoé!}
\begin{itemize}
\item {Grp. gram.:interj.}
\end{itemize}
\begin{itemize}
\item {Proveniência:(Lat. \textunderscore evoe\textunderscore )}
\end{itemize}
Grito, que se soltava nas orgias, para invocar Baccho.
\section{Evolar-se}
\begin{itemize}
\item {Grp. gram.:v. p.}
\end{itemize}
\begin{itemize}
\item {Proveniência:(Lat. \textunderscore evolare\textunderscore )}
\end{itemize}
Elevar-se, voando.
Exhalar-se.
Desapparecer no espaço.
\section{Evolução}
\begin{itemize}
\item {Grp. gram.:f.}
\end{itemize}
\begin{itemize}
\item {Proveniência:(Lat. \textunderscore evolutio\textunderscore )}
\end{itemize}
Acto de mover-se progressivamente: \textunderscore em política, prefere-se a evolução á revolução\textunderscore .
Desenvolvimento.
Movimento regular de tropas ou de uma esquadra ou de um navio, mudando de posição.
Giros de algumas aves, voando.
\section{Evolucional}
\begin{itemize}
\item {Grp. gram.:adj.}
\end{itemize}
Relativo a evolução. Cf. A. Cândido, \textunderscore Philos.\textunderscore , 6.
\section{Evolucionar}
\begin{itemize}
\item {Grp. gram.:v. i.}
\end{itemize}
\begin{itemize}
\item {Grp. gram.:V. p.}
\end{itemize}
Executar evolução.
Passar por transformações successivas: \textunderscore o t. lat.«parabola»evolucionou para«palábora»&lt;«palabra»&lt;«palavra»\textunderscore .
(A mesma sign.)
\section{Evolucionário}
\begin{itemize}
\item {Grp. gram.:adj.}
\end{itemize}
\begin{itemize}
\item {Proveniência:(Do lat. \textunderscore evolutio\textunderscore )}
\end{itemize}
Relativo a evoluções.
\section{Evolucionismo}
\begin{itemize}
\item {Grp. gram.:m.}
\end{itemize}
\begin{itemize}
\item {Proveniência:(Do lat. \textunderscore evolutio\textunderscore )}
\end{itemize}
Systema sociológico dos que preconizam o desenvolvimento social pelo processo evolutivo, por opposição ao processo revolucionário.
\section{Evolucionista}
\begin{itemize}
\item {Grp. gram.:m. ,  f.  e  adj.}
\end{itemize}
Pessôa, que é partidaria do evolucionismo.
\section{Evoluta}
\begin{itemize}
\item {Grp. gram.:f.}
\end{itemize}
\begin{itemize}
\item {Proveniência:(De \textunderscore evoluto\textunderscore )}
\end{itemize}
Curva, em que há os centros da curvatura de outra, que se chama evolvente.
\section{Evolutivo}
\begin{itemize}
\item {Grp. gram.:adj.}
\end{itemize}
\begin{itemize}
\item {Proveniência:(De \textunderscore evoluto\textunderscore )}
\end{itemize}
Que se desenvolve.
Que se transforma espontaneamente.
Relativo a evolução: \textunderscore movimento evolutivo\textunderscore .
\section{Evoluto}
\begin{itemize}
\item {Grp. gram.:adj.}
\end{itemize}
\begin{itemize}
\item {Proveniência:(Lat. \textunderscore evolutus\textunderscore )}
\end{itemize}
Diz-se das conchas univalves, que se enrolam num plano vertical e cuja espiral é mais ou menos alongada.
\section{Evolvente}
\begin{itemize}
\item {Grp. gram.:f.}
\end{itemize}
\begin{itemize}
\item {Proveniência:(Lat. \textunderscore evolvens\textunderscore )}
\end{itemize}
Curva que deriva da chamada evoluta.
\section{Evolver}
\begin{itemize}
\item {Grp. gram.:v. i.}
\end{itemize}
O mesmo que \textunderscore evolver-se\textunderscore .
\section{Evolver-se}
\begin{itemize}
\item {Grp. gram.:v. p.}
\end{itemize}
\begin{itemize}
\item {Proveniência:(Do lat. \textunderscore evolvere\textunderscore )}
\end{itemize}
O mesmo que \textunderscore evolucionar\textunderscore .
\section{Evonímeas}
\begin{itemize}
\item {Grp. gram.:f. pl.}
\end{itemize}
\begin{itemize}
\item {Utilização:Bot.}
\end{itemize}
\begin{itemize}
\item {Proveniência:(De \textunderscore evónimo\textunderscore )}
\end{itemize}
Tríbo de plantas celastríneas, estabelecida por De-Candole.
\section{Evonimina}
\begin{itemize}
\item {Grp. gram.:f.}
\end{itemize}
Substância, extraida do evónimo.
\section{Evónimo}
\begin{itemize}
\item {Grp. gram.:m.}
\end{itemize}
\begin{itemize}
\item {Proveniência:(Gr. \textunderscore euonumòs\textunderscore )}
\end{itemize}
Designação moderna da planta, vulgarmente conhecida por zaragatôa.
\section{Evonýmeas}
\begin{itemize}
\item {Grp. gram.:f. pl.}
\end{itemize}
\begin{itemize}
\item {Utilização:Bot.}
\end{itemize}
\begin{itemize}
\item {Proveniência:(De \textunderscore evónymo\textunderscore )}
\end{itemize}
Tríbo de plantas celastríneas, estabelecida por De-Candolle.
\section{Evonymina}
\begin{itemize}
\item {Grp. gram.:f.}
\end{itemize}
Substância, extrahida do evónymo.
\section{Evónymo}
\begin{itemize}
\item {Grp. gram.:m.}
\end{itemize}
\begin{itemize}
\item {Proveniência:(Gr. \textunderscore euonumòs\textunderscore )}
\end{itemize}
Designação moderna da planta, vulgarmente conhecida por zaragatôa.
\section{Evulsão}
\begin{itemize}
\item {Grp. gram.:f.}
\end{itemize}
\begin{itemize}
\item {Proveniência:(Lat. \textunderscore evulsio\textunderscore )}
\end{itemize}
O mesmo que \textunderscore avulsão\textunderscore .
\section{Evulsivo}
\begin{itemize}
\item {Grp. gram.:adj.}
\end{itemize}
\begin{itemize}
\item {Proveniência:(Do lat. \textunderscore evulsus\textunderscore )}
\end{itemize}
Que facilita a evulsão.
\section{Ex...}
\begin{itemize}
\item {Grp. gram.:pref.}
\end{itemize}
\begin{itemize}
\item {Proveniência:(Do lat. \textunderscore ex\textunderscore )}
\end{itemize}
(designativo de saída, derivação, intensidade, afastamento, opposição, etc., e que se pronuncia \textunderscore eis\textunderscore  ou \textunderscore is\textunderscore )
\section{Exabundância}
\begin{itemize}
\item {Grp. gram.:f.}
\end{itemize}
Qualidade daquillo que é exabundante.
\section{Exabundante}
\begin{itemize}
\item {Grp. gram.:adj.}
\end{itemize}
\begin{itemize}
\item {Proveniência:(Lat. \textunderscore exabundans\textunderscore )}
\end{itemize}
Muito abundante.
\section{Exabundantemente}
\begin{itemize}
\item {Grp. gram.:adv.}
\end{itemize}
De modo exabundante.
\section{Exabundar}
\begin{itemize}
\item {Grp. gram.:v. i.}
\end{itemize}
\begin{itemize}
\item {Proveniência:(Lat. \textunderscore exabundare\textunderscore )}
\end{itemize}
Abundar muito, super-abundar.
\section{Exacção}
\begin{itemize}
\item {Grp. gram.:f.}
\end{itemize}
\begin{itemize}
\item {Proveniência:(Lat. \textunderscore exatio\textunderscore )}
\end{itemize}
Exigência.
Cobrança rigorosa de contribuições.
Exactidão; pontualidade.
\section{Exacerbação}
\begin{itemize}
\item {Grp. gram.:f.}
\end{itemize}
\begin{itemize}
\item {Proveniência:(Lat. \textunderscore exacerbatio\textunderscore )}
\end{itemize}
Acto ou effeito de exacerbar.
\section{Exacerbador}
\begin{itemize}
\item {Grp. gram.:adj.}
\end{itemize}
Que exacerba.
\section{Exacerbar}
\begin{itemize}
\item {Grp. gram.:v. t.}
\end{itemize}
\begin{itemize}
\item {Proveniência:(Lat. \textunderscore exacerbare\textunderscore )}
\end{itemize}
Tornar acerbo, áspero.
Avivar.
Irritar; aggravar: \textunderscore exacerbar ódios\textunderscore .
\section{Exactamente}
\begin{itemize}
\item {Grp. gram.:adv.}
\end{itemize}
De modo exacto.
Com exactidão, com rigor.
Nem mais nem menos; precisamente.
\section{Exactidão}
\begin{itemize}
\item {Grp. gram.:f.}
\end{itemize}
Qualidade daquillo que é exacto.
Observância rigorosa de um contrato ou de um dever.
Pontualidade.
Inteireza de carácter.
Exposição fiel ou rigorosa dos factos.
\section{Exactificar}
\begin{itemize}
\item {Grp. gram.:v. t.}
\end{itemize}
\begin{itemize}
\item {Utilização:P. us.}
\end{itemize}
\begin{itemize}
\item {Proveniência:(Do lat. \textunderscore exactus\textunderscore  + \textunderscore facere\textunderscore )}
\end{itemize}
Tornar exacto.
Verificar; deslindar. Cf. Ribeiro Saraiva, \textunderscore Narrativa de Serviços\textunderscore .
\section{Exacto}
\begin{itemize}
\item {Grp. gram.:adj.}
\end{itemize}
\begin{itemize}
\item {Proveniência:(Lat. \textunderscore exactus\textunderscore )}
\end{itemize}
Que corresponde á verdade, á justiça, ao dever.
Rigoroso.
Pontual.
Verdadeiro.
Que cumpre o seu dever.
Que dá bôas contas.
Fiel.
Em que se reproduz fielmente um original, uma physionomia, etc.: \textunderscore cópia exacta\textunderscore .
\section{Exactor}
\begin{itemize}
\item {Grp. gram.:m.}
\end{itemize}
\begin{itemize}
\item {Proveniência:(Lat. \textunderscore exactor\textunderscore )}
\end{itemize}
Aquelle que faz exacções, que cobra impostos, etc.
\section{Exactoria}
\begin{itemize}
\item {Grp. gram.:f.}
\end{itemize}
\begin{itemize}
\item {Utilização:Bras}
\end{itemize}
Cargo ou funcções de exactor.
\section{Exageração}
\begin{itemize}
\item {Grp. gram.:f.}
\end{itemize}
\begin{itemize}
\item {Proveniência:(Lat. \textunderscore exaggeratio\textunderscore )}
\end{itemize}
Acto de exagerar.
\section{Exageradamente}
\begin{itemize}
\item {Grp. gram.:adv.}
\end{itemize}
\begin{itemize}
\item {Proveniência:(De \textunderscore exagerar\textunderscore )}
\end{itemize}
Com exageração.
\section{Exagerador}
\begin{itemize}
\item {Grp. gram.:m.  e  adj.}
\end{itemize}
O que exagera.
\section{Exagerar}
\begin{itemize}
\item {Grp. gram.:v. t.}
\end{itemize}
\begin{itemize}
\item {Proveniência:(Lat. \textunderscore exaggerare\textunderscore )}
\end{itemize}
Dar proporções excessivas a: \textunderscore exagerar defeitos\textunderscore .
Exprimir com ênfase ou com demasiado encarecimento.
Encarecer.
Fazer valer.
Ampliar.
\section{Exagerativamente}
\begin{itemize}
\item {Grp. gram.:adv.}
\end{itemize}
De modo exagerativo.
\section{Exagerativo}
\begin{itemize}
\item {Grp. gram.:adj.}
\end{itemize}
\begin{itemize}
\item {Proveniência:(Do lat. \textunderscore exaggeratus\textunderscore )}
\end{itemize}
Em que há exageração.
\section{Exagêro}
\begin{itemize}
\item {Grp. gram.:m.}
\end{itemize}
O mesmo que \textunderscore exageração\textunderscore .
\section{Exaggeração}
\begin{itemize}
\item {Grp. gram.:f.}
\end{itemize}
\begin{itemize}
\item {Proveniência:(Lat. \textunderscore exaggeratio\textunderscore )}
\end{itemize}
Acto de exaggerar.
\section{Exaggeradamente}
\begin{itemize}
\item {Grp. gram.:adv.}
\end{itemize}
\begin{itemize}
\item {Proveniência:(De \textunderscore exaggerar\textunderscore )}
\end{itemize}
Com exaggeração.
\section{Exaggerador}
\begin{itemize}
\item {Grp. gram.:m.  e  adj.}
\end{itemize}
O que exaggera.
\section{Exaggerar}
\begin{itemize}
\item {Grp. gram.:v. t.}
\end{itemize}
\begin{itemize}
\item {Proveniência:(Lat. \textunderscore exaggerare\textunderscore )}
\end{itemize}
Dar proporções excessivas a: \textunderscore exaggerar defeitos\textunderscore .
Exprimir com êmphase ou com demasiado encarecimento.
Encarecer.
Fazer valer.
Ampliar.
\section{Exaggerativamente}
\begin{itemize}
\item {Grp. gram.:adv.}
\end{itemize}
De modo exaggerativo.
\section{Exaggerativo}
\begin{itemize}
\item {Grp. gram.:adj.}
\end{itemize}
\begin{itemize}
\item {Proveniência:(Do lat. \textunderscore exaggeratus\textunderscore )}
\end{itemize}
Em que há exaggeração.
\section{Exaggêro}
\begin{itemize}
\item {Grp. gram.:m.}
\end{itemize}
O mesmo que \textunderscore exaggeração\textunderscore .
\section{Exagitante}
\begin{itemize}
\item {Grp. gram.:adj.}
\end{itemize}
\begin{itemize}
\item {Proveniência:(Lat. \textunderscore exagitans\textunderscore )}
\end{itemize}
Que exagita.
\section{Exagitar}
\begin{itemize}
\item {Grp. gram.:v. t.}
\end{itemize}
\begin{itemize}
\item {Proveniência:(Lat. \textunderscore exagitare\textunderscore )}
\end{itemize}
Agitar muito.
Irritar.
\section{Exalbuminado}
\begin{itemize}
\item {Grp. gram.:adj.}
\end{itemize}
\begin{itemize}
\item {Proveniência:(De \textunderscore ex...\textunderscore  + \textunderscore albumina\textunderscore )}
\end{itemize}
Que não tem perisperma.
\section{Exalçar}
\begin{itemize}
\item {Proveniência:(De \textunderscore alçar\textunderscore )}
\end{itemize}
\textunderscore v. t.\textunderscore  (e der.)
O mesmo que \textunderscore exaltar\textunderscore , etc.
\section{Exalgina}
\begin{itemize}
\item {Grp. gram.:f.}
\end{itemize}
Medicamento antineurálgico.
\section{Exaltação}
\begin{itemize}
\item {Grp. gram.:f.}
\end{itemize}
\begin{itemize}
\item {Proveniência:(Lat. \textunderscore exaltatio\textunderscore )}
\end{itemize}
Acto ou effeito de exaltar.
Excitação.
Enthusiasmo.
Perturbação mental, acompanhada de excitação.
Louvor enthusiástico.
Sublimação chímica de uma substância.
\section{Exaltadamente}
\begin{itemize}
\item {Grp. gram.:adv.}
\end{itemize}
\begin{itemize}
\item {Proveniência:(De \textunderscore exaltado\textunderscore )}
\end{itemize}
Com exaltação.
\section{Exaltado}
\begin{itemize}
\item {Grp. gram.:adj.}
\end{itemize}
\begin{itemize}
\item {Utilização:Chím.}
\end{itemize}
\begin{itemize}
\item {Proveniência:(De \textunderscore exaltar\textunderscore )}
\end{itemize}
Levantado, elevado.
Exaggerado.
Irritável.
Rectificado, sublimado.
\section{Exaltador}
\begin{itemize}
\item {Grp. gram.:adj.}
\end{itemize}
Que exalta.
\section{Exaltamento}
\begin{itemize}
\item {Grp. gram.:m.}
\end{itemize}
(V.exaltação)
\section{Exaltar}
\begin{itemize}
\item {Grp. gram.:v. t.}
\end{itemize}
\begin{itemize}
\item {Proveniência:(Lat. \textunderscore exaltare\textunderscore )}
\end{itemize}
Tornar alto.
Levantar; engrandecer.
Elogiar muito: \textunderscore exaltar o talento de alguém\textunderscore .
Celebrar; tornar célebre: \textunderscore Camões exaltou Vasco da Gama\textunderscore .
Nobilitar.
Sublimar.
Tornar intenso.
Enthusiasmar.
Excitar; irritar: \textunderscore qualquer contrariedade o exalta\textunderscore .
\section{Exalviçado}
\begin{itemize}
\item {Grp. gram.:adj.}
\end{itemize}
\begin{itemize}
\item {Utilização:Des.}
\end{itemize}
\begin{itemize}
\item {Proveniência:(De \textunderscore ex...\textunderscore  + \textunderscore alvo\textunderscore )}
\end{itemize}
Alvacento.
\section{Exame}
\begin{itemize}
\item {Grp. gram.:m.}
\end{itemize}
\begin{itemize}
\item {Proveniência:(Lat. \textunderscore examen\textunderscore )}
\end{itemize}
Observação ou investigação minuciosa e attenta.
Anályse; revista: \textunderscore os mêdicos fizeram-lhe exame dos ferimentos\textunderscore .
Interrogatório.
Demonstração, mais ou menos solemne, de competência para o exercício de um cargo.
Prova de habilitação ou não habilitação em algum ramo de conhecimentos humanos: \textunderscore fez exame de inglês\textunderscore .
\section{Examina}
\begin{itemize}
\item {Grp. gram.:f.}
\end{itemize}
\begin{itemize}
\item {Utilização:Prov.}
\end{itemize}
\begin{itemize}
\item {Proveniência:(De \textunderscore examinar\textunderscore )}
\end{itemize}
Exame, feito pelo párocho aos seus fregueses, sôbre os conhecimentos dêstes em doutrina christan.
\section{Examinação}
\begin{itemize}
\item {Grp. gram.:f.}
\end{itemize}
\begin{itemize}
\item {Utilização:Pop.}
\end{itemize}
\begin{itemize}
\item {Proveniência:(Lat. \textunderscore examinatio\textunderscore )}
\end{itemize}
Exame.
\section{Examinador}
\begin{itemize}
\item {Grp. gram.:m.  e  adj.}
\end{itemize}
O que examina.
\section{Examinando}
\begin{itemize}
\item {Grp. gram.:m.}
\end{itemize}
\begin{itemize}
\item {Proveniência:(Lat. \textunderscore examinandus\textunderscore )}
\end{itemize}
Aquelle que vai sêr examinado.
Aquelle que se está examinando.
\section{Examinar}
\begin{itemize}
\item {Grp. gram.:v. t.}
\end{itemize}
\begin{itemize}
\item {Utilização:Fig.}
\end{itemize}
\begin{itemize}
\item {Proveniência:(Lat. \textunderscore examinare\textunderscore )}
\end{itemize}
Fazer exame de.
Observar: \textunderscore examinar os astros\textunderscore .
Investigar a aptidão ou capacidade de.
Interrogar, inquirir.
Investigar.
Estudar: \textunderscore examinar um processo\textunderscore .
Provar.
\section{Examinável}
\begin{itemize}
\item {Grp. gram.:adj.}
\end{itemize}
Que se póde examinar.
\section{Exangue}
\begin{itemize}
\item {Grp. gram.:adj.}
\end{itemize}
\begin{itemize}
\item {Utilização:Fig.}
\end{itemize}
\begin{itemize}
\item {Proveniência:(Lat. \textunderscore exanguis\textunderscore )}
\end{itemize}
Que perdeu o sangue.
Esvaído em sangue.
Débil; enfraquecido.
\section{Exania}
\begin{itemize}
\item {Grp. gram.:f.}
\end{itemize}
\begin{itemize}
\item {Proveniência:(De \textunderscore ex...\textunderscore  + \textunderscore ânus\textunderscore )}
\end{itemize}
Quéda do intestino recto para fóra do ânus.
\section{Exanimação}
\begin{itemize}
\item {Grp. gram.:f.}
\end{itemize}
\begin{itemize}
\item {Proveniência:(Lat. \textunderscore exanimatio\textunderscore )}
\end{itemize}
Desfallecimento.
Morte apparente.
\section{Exanimado}
\begin{itemize}
\item {Grp. gram.:adj.}
\end{itemize}
O mesmo que \textunderscore exâmine\textunderscore . Cf. Filinto, VII, 186.
\section{Exânime}
\begin{itemize}
\item {Grp. gram.:adj.}
\end{itemize}
\begin{itemize}
\item {Proveniência:(Lat. \textunderscore exanimis\textunderscore )}
\end{itemize}
Que perdeu a sensibilidade ou a vida.
Morto.
\section{Exantema}
\begin{itemize}
\item {Grp. gram.:m.}
\end{itemize}
\begin{itemize}
\item {Proveniência:(Gr. \textunderscore exanthema\textunderscore )}
\end{itemize}
Qualquer doença cutânea, caracterizada por uma vermelhidão mais ou menos viva, sem pústulas.
\section{Exantemático}
\begin{itemize}
\item {Grp. gram.:adj.}
\end{itemize}
Relativo a exantema.
Acompanhado ou caracterizado por exantemas: \textunderscore prurido exantemático\textunderscore .
\section{Exantematoso}
\begin{itemize}
\item {Grp. gram.:adj.}
\end{itemize}
(V.exantemático)
\section{Exanthema}
\begin{itemize}
\item {Grp. gram.:m.}
\end{itemize}
\begin{itemize}
\item {Proveniência:(Gr. \textunderscore exanthema\textunderscore )}
\end{itemize}
Qualquer doença cutânea, caracterizada por uma vermelhidão mais ou menos viva, sem pústulas.
\section{Exanthemático}
\begin{itemize}
\item {Grp. gram.:adj.}
\end{itemize}
Relativo a exanthema.
Acompanhado ou caracterizado por exanthemas: \textunderscore prurido exanthemático\textunderscore .
\section{Exanthematoso}
\begin{itemize}
\item {Grp. gram.:adj.}
\end{itemize}
(V.exanthemático)
\section{Exarar}
\begin{itemize}
\item {Grp. gram.:v. t.}
\end{itemize}
\begin{itemize}
\item {Utilização:Fig.}
\end{itemize}
\begin{itemize}
\item {Proveniência:(Lat. \textunderscore exarare\textunderscore )}
\end{itemize}
Abrir, lavrar.
Gravar: \textunderscore exarar uma inscripção\textunderscore .
Mencionar.
Consignar; escrever: \textunderscore exarar uma declaração na acta\textunderscore .
\section{Exarca}
\begin{itemize}
\item {Grp. gram.:m.}
\end{itemize}
\begin{itemize}
\item {Proveniência:(Gr. \textunderscore exarkhos\textunderscore )}
\end{itemize}
Delegado do imperador de Constantinopla, no Ocidente.
Legado do patriarcha grego.
\section{Exarcado}
\begin{itemize}
\item {Grp. gram.:m.}
\end{itemize}
\begin{itemize}
\item {Proveniência:(De \textunderscore exarca\textunderscore )}
\end{itemize}
Dignidade de exarca; território governado por exarca.
\section{Exarcha}
\begin{itemize}
\item {fónica:ca}
\end{itemize}
\begin{itemize}
\item {Grp. gram.:m.}
\end{itemize}
\begin{itemize}
\item {Proveniência:(Gr. \textunderscore exarkhos\textunderscore )}
\end{itemize}
Delegado do imperador de Constantinopla, no Occidente.
Legado do patriarcha grego.
\section{Exarchado}
\begin{itemize}
\item {fónica:ca}
\end{itemize}
\begin{itemize}
\item {Grp. gram.:m.}
\end{itemize}
\begin{itemize}
\item {Proveniência:(De \textunderscore exarcha\textunderscore )}
\end{itemize}
Dignidade de exarcha; território governado por exarcha.
\section{Exarchia}
\begin{itemize}
\item {fónica:qui}
\end{itemize}
\begin{itemize}
\item {Grp. gram.:f.}
\end{itemize}
O mesmo que \textunderscore exarcado\textunderscore . Cf. Herculano, \textunderscore Cister\textunderscore , 137.
\section{Exarcho}
\begin{itemize}
\item {fónica:co}
\end{itemize}
\begin{itemize}
\item {Grp. gram.:m.}
\end{itemize}
\begin{itemize}
\item {Utilização:Ant.}
\end{itemize}
O mesmo ou melhor que \textunderscore exarcha\textunderscore . Cf. Pant. de Aveiro, \textunderscore Etiner.\textunderscore , 91, (2.^a ed.)
\section{Exarco}
\begin{itemize}
\item {Grp. gram.:m.}
\end{itemize}
\begin{itemize}
\item {Utilização:Ant.}
\end{itemize}
O mesmo ou melhor que \textunderscore exarca\textunderscore . Cf. Pant. de Aveiro, \textunderscore Etiner.\textunderscore , 91, (2.^a ed.).
\section{Exarquia}
\begin{itemize}
\item {Grp. gram.:f.}
\end{itemize}
O mesmo que \textunderscore exarcado\textunderscore . Cf. Herculano, \textunderscore Cister\textunderscore , 137.
\section{Exarthrema}
\begin{itemize}
\item {Grp. gram.:f.}
\end{itemize}
\begin{itemize}
\item {Proveniência:(Do gr. \textunderscore ex\textunderscore  + \textunderscore arthros\textunderscore )}
\end{itemize}
O mesmo que \textunderscore luxação\textunderscore .
\section{Exarticulação}
\begin{itemize}
\item {Grp. gram.:f.}
\end{itemize}
O mesmo que \textunderscore desarticulação\textunderscore .
\section{Exartrema}
\begin{itemize}
\item {Grp. gram.:f.}
\end{itemize}
\begin{itemize}
\item {Proveniência:(Do gr. \textunderscore ex\textunderscore  + \textunderscore arthros\textunderscore )}
\end{itemize}
O mesmo que \textunderscore luxação\textunderscore .
\section{Exasperação}
\begin{itemize}
\item {Grp. gram.:f.}
\end{itemize}
\begin{itemize}
\item {Proveniência:(Lat. \textunderscore exasperatio\textunderscore )}
\end{itemize}
Acto de exasperar.
Irritação; exacerbação.
\section{Exasperador}
\begin{itemize}
\item {Grp. gram.:m.  e  adj.}
\end{itemize}
Aquelle que exaspera.
\section{Exasperar}
\begin{itemize}
\item {Grp. gram.:v. t.}
\end{itemize}
\begin{itemize}
\item {Utilização:Des.}
\end{itemize}
\begin{itemize}
\item {Proveniência:(Lat. \textunderscore exasperare\textunderscore )}
\end{itemize}
Tornar enfurecido.
Irritar muito: \textunderscore a provocação exasperou-o\textunderscore .
Exacerbar: \textunderscore a falta de hygiene exasperou-lhe a doença\textunderscore .
Tornar áspero.
\section{Exaspêro}
\begin{itemize}
\item {Grp. gram.:m.}
\end{itemize}
O mesmo que \textunderscore exasperação\textunderscore .
\section{Exautoração}
\begin{itemize}
\item {Grp. gram.:f.}
\end{itemize}
Acto ou effeito de exautorar.
\section{Exautorar}
\begin{itemize}
\item {Grp. gram.:v. t.}
\end{itemize}
\begin{itemize}
\item {Proveniência:(Lat. \textunderscore exauctorare\textunderscore )}
\end{itemize}
Tirar a autoridade a.
Privar de cargo, de insígnias, de honras: \textunderscore exautorar um capitão\textunderscore .
\section{Excandecer}
\textunderscore v. t.\textunderscore , \textunderscore i.\textunderscore  e \textunderscore p.\textunderscore  (e der.)
(V. \textunderscore escandecer\textunderscore , etc.)
\section{Excarceração}
\begin{itemize}
\item {Grp. gram.:f.}
\end{itemize}
Acto de excarcerar.
\section{Excarcerar}
\begin{itemize}
\item {Grp. gram.:v. t.}
\end{itemize}
\begin{itemize}
\item {Proveniência:(De \textunderscore ex...\textunderscore  + \textunderscore cárcere\textunderscore )}
\end{itemize}
Livrar do cárcere.
Libertar.
\section{Excarnação}
\begin{itemize}
\item {Grp. gram.:f.}
\end{itemize}
(V.escarnação)
\section{Excarnificar}
\textunderscore v. t.\textunderscore  (e der.)
(V. \textunderscore escarnificar\textunderscore , etc.)
\section{Ex-cathedra}
\begin{itemize}
\item {fónica:eis-cátedra}
\end{itemize}
\begin{itemize}
\item {Grp. gram.:loc. adv.}
\end{itemize}
Com pedantismo; doutoralmente; de cadeira.
(Loc. lat.)
\section{Excavaçar}
\textunderscore v. t.\textunderscore  (e der.)
(V. \textunderscore escavaçar\textunderscore , etc.)
\section{Excavar}
\textunderscore v. t.\textunderscore  e \textunderscore p.\textunderscore  (e der.)
(V. \textunderscore escavar\textunderscore , etc.)
\section{Excecária}
\begin{itemize}
\item {Grp. gram.:f.}
\end{itemize}
\begin{itemize}
\item {Proveniência:(Do lat. \textunderscore excaecare\textunderscore )}
\end{itemize}
Gênero de plantas tropicaes, da fam. das euphorbiáceas.
\section{Excedente}
\begin{itemize}
\item {Grp. gram.:adj.}
\end{itemize}
\begin{itemize}
\item {Grp. gram.:M.}
\end{itemize}
\begin{itemize}
\item {Proveniência:(Lat. \textunderscore excedens\textunderscore )}
\end{itemize}
Que excede.
Aquillo que excede.
Excesso; sobejo.
\section{Exceder}
\begin{itemize}
\item {Grp. gram.:v. t.}
\end{itemize}
\begin{itemize}
\item {Grp. gram.:V. i.}
\end{itemize}
\begin{itemize}
\item {Grp. gram.:V. p.}
\end{itemize}
\begin{itemize}
\item {Proveniência:(Lat. \textunderscore excedere\textunderscore )}
\end{itemize}
Ir além de.
Levar vantagem a: \textunderscore é estudante que excede os collegas\textunderscore .
Ultrapasar: \textunderscore êsse sacrifício excede as minhas fôrças\textunderscore .
Passar além, levar vantagem:«\textunderscore as bôas imitações não têm menos valia que os originaes, e casos haverá em que lhes excedam\textunderscore ». Castilho.
Demasiar-se.
Ir além do que é justo.
Enfurecer-se.
Fatigar-se.
Esmerar-se.
\section{Excedível}
\begin{itemize}
\item {Grp. gram.:adj.}
\end{itemize}
Que se póde exceder.
\section{Exceição}
\begin{itemize}
\item {Grp. gram.:f.}
\end{itemize}
\begin{itemize}
\item {Utilização:Des.}
\end{itemize}
O mesmo que \textunderscore excepção\textunderscore .
\section{Exceituar}
\begin{itemize}
\item {Grp. gram.:v. t.}
\end{itemize}
\begin{itemize}
\item {Utilização:Des.}
\end{itemize}
O mesmo que \textunderscore exceptuar\textunderscore .
\section{Excelência}
\begin{itemize}
\item {Grp. gram.:f.}
\end{itemize}
\begin{itemize}
\item {Proveniência:(Lat. \textunderscore excellentia\textunderscore )}
\end{itemize}
Qualidade daquilo que é excelente.
Tratamento, que se dá a pessôas nobres ou de elevada situação social e, geralmente, a senhoras.
\section{Excelente}
\begin{itemize}
\item {Grp. gram.:adj.}
\end{itemize}
\begin{itemize}
\item {Proveniência:(Lat. \textunderscore excellens\textunderscore )}
\end{itemize}
Que é superior ou muito bom, no seu gênero: \textunderscore café excelente\textunderscore .
Bem feito.
Primoroso.
Distinto.
Magnifico: \textunderscore um poêma excelente\textunderscore .
\section{Excelentemente}
\begin{itemize}
\item {Grp. gram.:adv.}
\end{itemize}
De modo excelente.
Distintamente.
Magnificamente.
\section{Excelentíssimo}
\begin{itemize}
\item {Grp. gram.:adj.}
\end{itemize}
Muito excelente.
Tratamento, dado a indivíduos de superior categoria social.
\section{Exceler}
\begin{itemize}
\item {Grp. gram.:v. i.}
\end{itemize}
\begin{itemize}
\item {Proveniência:(Lat. \textunderscore excellere\textunderscore )}
\end{itemize}
Sêr excelente.
\section{Excelir}
\begin{itemize}
\item {Grp. gram.:v. i.}
\end{itemize}
(V.exceler). Cf. Camillo, \textunderscore Narcót.\textunderscore , II, 228.
\section{Excellência}
\begin{itemize}
\item {Grp. gram.:f.}
\end{itemize}
\begin{itemize}
\item {Proveniência:(Lat. \textunderscore excellentia\textunderscore )}
\end{itemize}
Qualidade daquillo que é excellente.
Tratamento, que se dá a pessôas nobres ou de elevada situação social e, geralmente, a senhoras.
\section{Excellente}
\begin{itemize}
\item {Grp. gram.:adj.}
\end{itemize}
\begin{itemize}
\item {Proveniência:(Lat. \textunderscore excellens\textunderscore )}
\end{itemize}
Que é superior ou muito bom, no seu gênero: \textunderscore café excellente\textunderscore .
Bem feito.
Primoroso.
Distinto.
Magnifico: \textunderscore um poêma excellente\textunderscore .
\section{Excellentemente}
\begin{itemize}
\item {Grp. gram.:adv.}
\end{itemize}
De modo excellente.
Distintamente.
Magnificamente.
\section{Excellentíssimo}
\begin{itemize}
\item {Grp. gram.:adj.}
\end{itemize}
Muito excellente.
Tratamento, dado a indivíduos de superior categoria social.
\section{Exceller}
\begin{itemize}
\item {Grp. gram.:v. i.}
\end{itemize}
\begin{itemize}
\item {Proveniência:(Lat. \textunderscore excellere\textunderscore )}
\end{itemize}
Sêr excellente.
\section{Excellir}
\begin{itemize}
\item {Grp. gram.:v. i.}
\end{itemize}
(V.exceller)Cf. Camillo, \textunderscore Narcót.\textunderscore , II, 228.
\section{Excelsamente}
\begin{itemize}
\item {Grp. gram.:adv.}
\end{itemize}
De modo excelso.
Com sublimidade.
\section{Excelsitude}
\begin{itemize}
\item {Grp. gram.:f.}
\end{itemize}
Qualidade daquillo que é excelso.
\section{Excelso}
\begin{itemize}
\item {Grp. gram.:adj.}
\end{itemize}
\begin{itemize}
\item {Proveniência:(Lat. \textunderscore excelsus\textunderscore )}
\end{itemize}
Muito alto.
Sublime.
Illustre; magnificente.
\section{Excentricamente}
\begin{itemize}
\item {Grp. gram.:adv.}
\end{itemize}
De modo excêntrico.
Fóra do centro.
\section{Excentricidade}
\begin{itemize}
\item {Grp. gram.:f.}
\end{itemize}
Qualidade daquelle ou daquillo que é excêntrico.
\section{Excêntrico}
\begin{itemize}
\item {Grp. gram.:adj.}
\end{itemize}
\begin{itemize}
\item {Utilização:Fig.}
\end{itemize}
\begin{itemize}
\item {Grp. gram.:M.}
\end{itemize}
\begin{itemize}
\item {Proveniência:(De \textunderscore ex...\textunderscore  + \textunderscore centro\textunderscore )}
\end{itemize}
Que está fóra do centro.
Que não tem o mesmo centro.
Extravagante; esquisito: \textunderscore um sujeito excêntrico\textunderscore .
Original.
Indivíduo, que tem excentricidades:«\textunderscore os excêntricos do meu tempo\textunderscore ». L. A. Palmeirim.
\section{Excepção}
\begin{itemize}
\item {Grp. gram.:f.}
\end{itemize}
\begin{itemize}
\item {Utilização:Fig.}
\end{itemize}
\begin{itemize}
\item {Proveniência:(Lat. \textunderscore exceptio\textunderscore )}
\end{itemize}
Acto ou effeito de exceptuar.
Restricção da regra.
Pessôa, cujos actos ou ideias se afastam do procedimento ou pensar vulgar.
\section{Excepcional}
\begin{itemize}
\item {Grp. gram.:adj.}
\end{itemize}
\begin{itemize}
\item {Proveniência:(Do lat. \textunderscore exceptio\textunderscore )}
\end{itemize}
Em que há excepção.
Relativo a excepção.
Excêntrico.
\section{Excepcionalidade}
\begin{itemize}
\item {Grp. gram.:f.}
\end{itemize}
Qualidade de excepcional.
\section{Excepcionalmente}
\begin{itemize}
\item {Grp. gram.:adv.}
\end{itemize}
De modo excepcional.
\section{Excepcionar}
\begin{itemize}
\item {Grp. gram.:v. t.}
\end{itemize}
\begin{itemize}
\item {Proveniência:(Do lat. \textunderscore exceptio\textunderscore )}
\end{itemize}
Oppor excepção a.
\section{Exceptivo}
\begin{itemize}
\item {Grp. gram.:adj.}
\end{itemize}
O mesmo que \textunderscore excepcional\textunderscore .
\section{Excepto}
\begin{itemize}
\item {Grp. gram.:prep.}
\end{itemize}
\begin{itemize}
\item {Grp. gram.:Adj.}
\end{itemize}
\begin{itemize}
\item {Utilização:Des.}
\end{itemize}
\begin{itemize}
\item {Proveniência:(Lat. \textunderscore exceptus\textunderscore )}
\end{itemize}
Á excepção de.
Afóra.
Excluíndo.
O mesmo que [[exceptuado|exceptuar]]: \textunderscore exceptas as crianças\textunderscore .
\section{Exceptor}
\begin{itemize}
\item {Grp. gram.:m.}
\end{itemize}
\begin{itemize}
\item {Utilização:Ant.}
\end{itemize}
O mesmo que \textunderscore escrivão\textunderscore  ou \textunderscore notário\textunderscore . Cf. Herculano, \textunderscore Hist. de Port.\textunderscore , IV, 10.
\section{Exceptuadamente}
\begin{itemize}
\item {Grp. gram.:adv.}
\end{itemize}
\begin{itemize}
\item {Proveniência:(De \textunderscore exceptuar\textunderscore )}
\end{itemize}
Excepcionalmente.
\section{Exceptuador}
\begin{itemize}
\item {Grp. gram.:m.}
\end{itemize}
Aquelle que exceptua.
\section{Exceptuar}
\begin{itemize}
\item {Grp. gram.:v. t.}
\end{itemize}
\begin{itemize}
\item {Grp. gram.:V. i.}
\end{itemize}
\begin{itemize}
\item {Proveniência:(De \textunderscore excepto\textunderscore )}
\end{itemize}
Fazer excepção de.
Excluír; tornar isento.
Oppôr excepção em juízo.
\section{Excerpto}
\begin{itemize}
\item {Grp. gram.:m.}
\end{itemize}
\begin{itemize}
\item {Grp. gram.:Adj.}
\end{itemize}
\begin{itemize}
\item {Utilização:Ant.}
\end{itemize}
\begin{itemize}
\item {Proveniência:(Lat. \textunderscore excerptus\textunderscore )}
\end{itemize}
Fragmento.
Trecho; extracto.
Tirado, extrahido:«\textunderscore para aproveitar vários papeis já excerptos...\textunderscore »\textunderscore Luz e Calor\textunderscore , 224.
\section{Excessivamente}
\begin{itemize}
\item {Grp. gram.:adv.}
\end{itemize}
De modo excessivo.
\section{Excessivo}
\begin{itemize}
\item {Grp. gram.:adj.}
\end{itemize}
\begin{itemize}
\item {Proveniência:(De \textunderscore excesso\textunderscore )}
\end{itemize}
Que excede a justa medida.
Em que há excesso; demasiado.
Extraordinário: \textunderscore calor excessivo\textunderscore .
Muito affectuoso.
\section{Excesso}
\begin{itemize}
\item {Grp. gram.:m.}
\end{itemize}
\begin{itemize}
\item {Proveniência:(Lat. \textunderscore excessus\textunderscore )}
\end{itemize}
Differença para mais, entre duas quantidades desiguaes.
Aquillo que ultrapassa uma convenção ou um limite legal.
Desmando, desregramento: \textunderscore praticar excessos\textunderscore .
Grau elevado, cúmulo.
Falta de moderação.
\section{Excetra}
\begin{itemize}
\item {Grp. gram.:f.}
\end{itemize}
\begin{itemize}
\item {Proveniência:(Lat. \textunderscore excetra\textunderscore )}
\end{itemize}
Serpente de água doce.
Hydra.
\section{Excídio}
\begin{itemize}
\item {Grp. gram.:m.}
\end{itemize}
\begin{itemize}
\item {Proveniência:(Lat. \textunderscore excidium\textunderscore )}
\end{itemize}
Destruição; subversão.
\section{Excipiente}
\begin{itemize}
\item {Grp. gram.:m.}
\end{itemize}
\begin{itemize}
\item {Proveniência:(Lat. \textunderscore excipiens\textunderscore )}
\end{itemize}
Substância, que serve para ligar ou dissolver outras substâncias, que constituem um medicamento, ou para lhes deminuir a energia, ou para lhes disfarçar o sabor.
\section{Excisão}
\begin{itemize}
\item {Grp. gram.:f.}
\end{itemize}
\begin{itemize}
\item {Utilização:Fig.}
\end{itemize}
\begin{itemize}
\item {Proveniência:(Lat. \textunderscore excisio\textunderscore )}
\end{itemize}
Acto de cortar.
Amputação; ablação de uma parte de um todo.
Abalo ou golpe profundo. Cf. Camillo, \textunderscore Volcões\textunderscore , 70.
\section{Excisar}
\begin{itemize}
\item {Grp. gram.:v. t.}
\end{itemize}
\begin{itemize}
\item {Utilização:Neol.}
\end{itemize}
\begin{itemize}
\item {Proveniência:(Do lat. \textunderscore excisus\textunderscore )}
\end{itemize}
Cortar ou fazer excisão de.
\section{Excitabilidade}
\begin{itemize}
\item {Grp. gram.:f.}
\end{itemize}
\begin{itemize}
\item {Proveniência:(Do lat. \textunderscore excitabilis\textunderscore )}
\end{itemize}
Qualidade daquelle ou daquillo que é excitável ou irritável.
\section{Excitação}
\begin{itemize}
\item {Grp. gram.:f.}
\end{itemize}
\begin{itemize}
\item {Proveniência:(Lat. \textunderscore excitatio\textunderscore )}
\end{itemize}
Acto ou effeito de excitar.
\section{Excitador}
\begin{itemize}
\item {Grp. gram.:adj.}
\end{itemize}
\begin{itemize}
\item {Grp. gram.:M.}
\end{itemize}
Que excita.
Aquelle que excita.
\section{Excitamento}
\begin{itemize}
\item {Grp. gram.:m.}
\end{itemize}
O mesmo que \textunderscore excitação\textunderscore .
\section{Excitante}
\begin{itemize}
\item {Grp. gram.:m.  e  adj.}
\end{itemize}
\begin{itemize}
\item {Proveniência:(Lat. \textunderscore excitans\textunderscore )}
\end{itemize}
O mesmo que \textunderscore excitador\textunderscore : \textunderscore o álcool é um excitante\textunderscore .
\section{Excitar}
\begin{itemize}
\item {Grp. gram.:v. t.}
\end{itemize}
\begin{itemize}
\item {Proveniência:(Lat. \textunderscore excitare\textunderscore )}
\end{itemize}
Impellir.
Mover para algum fim.
Incitar.
Açular: \textunderscore excitar cães\textunderscore .
Animar.
Promover: \textunderscore excitar ódios\textunderscore .
Activar: \textunderscore excitar um incêndio\textunderscore .
Estimular; irritar.
\section{Excitativo}
\begin{itemize}
\item {Grp. gram.:adj.}
\end{itemize}
O mesmo que \textunderscore excitante\textunderscore .
\section{Excitatório}
\textunderscore adj.\textunderscore  (des.)(V.excitante)
\section{Excitável}
\begin{itemize}
\item {Grp. gram.:adj.}
\end{itemize}
\begin{itemize}
\item {Proveniência:(Lat. \textunderscore excitabilis\textunderscore )}
\end{itemize}
Que póde excitar-se; que é susceptível de excitação.
\section{Excito-motor}
\begin{itemize}
\item {Grp. gram.:adj.}
\end{itemize}
\begin{itemize}
\item {Proveniência:(De \textunderscore excitar\textunderscore  + \textunderscore motor\textunderscore )}
\end{itemize}
Diz-se de uma parte do systema nervoso, que é posta em acção por agentes externos, independentemente da vontade.
\section{Exclamação}
\begin{itemize}
\item {Grp. gram.:f.}
\end{itemize}
\begin{itemize}
\item {Proveniência:(Lat. \textunderscore exclamatio\textunderscore )}
\end{itemize}
Acto de exclamar.
Grito súbito de surpresa, de admiração, de prazer, de raiva, etc.
Interjeição.
Sinal gráphico, que segue uma exclamação.
\section{Exclamador}
\begin{itemize}
\item {Grp. gram.:adj.}
\end{itemize}
\begin{itemize}
\item {Grp. gram.:M.}
\end{itemize}
Que exclama.
Aquelle que exclama.
\section{Exclamar}
\begin{itemize}
\item {Grp. gram.:v. t.}
\end{itemize}
\begin{itemize}
\item {Grp. gram.:V. i.}
\end{itemize}
\begin{itemize}
\item {Proveniência:(Lat. \textunderscore exclamare\textunderscore )}
\end{itemize}
Pronunciar em voz muito alta.
Vociferar; gritar.
\section{Exclamativamente}
\begin{itemize}
\item {Grp. gram.:adv.}
\end{itemize}
De modo exclamativo.
\section{Exclamativo}
\begin{itemize}
\item {Grp. gram.:adj.}
\end{itemize}
\begin{itemize}
\item {Proveniência:(De \textunderscore exclamar\textunderscore )}
\end{itemize}
Que envolve exclamação; admirativo.
\section{Exclamatório}
\begin{itemize}
\item {Grp. gram.:adj.}
\end{itemize}
(V.exclamativo)
\section{Excluído}
\begin{itemize}
\item {Grp. gram.:adj.}
\end{itemize}
\begin{itemize}
\item {Proveniência:(De \textunderscore excluir\textunderscore )}
\end{itemize}
Pôsto fóra.
Omittido.
\section{Excluir}
\begin{itemize}
\item {Grp. gram.:v. t.}
\end{itemize}
\begin{itemize}
\item {Proveniência:(Lat. \textunderscore excludere\textunderscore )}
\end{itemize}
Pôr fóra: \textunderscore excluir alguém de uma associação\textunderscore .
Omittir: \textunderscore excluir alguém de um recenseamento eleitoral\textunderscore .
Exceptuar.
Impedir a entrada de.
Privar da posse de alguma coisa; desviar: \textunderscore excluir alguém de uma partilha\textunderscore .
\section{Exalação}
\begin{itemize}
\item {Grp. gram.:f.}
\end{itemize}
\begin{itemize}
\item {Proveniência:(Lat. \textunderscore exhalatio\textunderscore )}
\end{itemize}
Acto de exalar ou de se exalar.
Restituição, que as plantas fazem á atmosfera, dos gases que absorveram.
Emanação, imperceptível á vista, de uma substância sólida ou líquida.
Evaporação.
Vapor.
Cheiro.
Difusão de certos fluidos orgânicos sôbre certas membranas ou á superfície da pele.
\section{Exalante}
\begin{itemize}
\item {Grp. gram.:adj.}
\end{itemize}
\begin{itemize}
\item {Proveniência:(Lat. \textunderscore exhalans\textunderscore )}
\end{itemize}
Que exala.
\section{Exalar}
\begin{itemize}
\item {Grp. gram.:v. t.}
\end{itemize}
\begin{itemize}
\item {Grp. gram.:V. p.}
\end{itemize}
\begin{itemize}
\item {Proveniência:(Lat. \textunderscore exhalare\textunderscore )}
\end{itemize}
Espirar, lançar de si.
Emitir: \textunderscore exalar aroma\textunderscore .
Soltar.
Expandir.
Evaporar.
Evaporar-se; evolar-se; desaparecer.
\section{Exaurição}
\begin{itemize}
\item {Grp. gram.:f.}
\end{itemize}
Acto ou efeito de exaurir.
\section{Exaurimento}
\begin{itemize}
\item {Grp. gram.:m.}
\end{itemize}
O mesmo que \textunderscore exaurição\textunderscore .
\section{Exaurir}
\begin{itemize}
\item {Grp. gram.:v. t.}
\end{itemize}
\begin{itemize}
\item {Proveniência:(Lat. \textunderscore exhaurire\textunderscore )}
\end{itemize}
Esgotar completamente.
Dissipar completamente.
Depauperar.
\section{Exaurível}
\begin{itemize}
\item {Grp. gram.:adj.}
\end{itemize}
Que se póde exaurir.
\section{Exaustação}
\begin{itemize}
\item {Grp. gram.:f.}
\end{itemize}
\begin{itemize}
\item {Utilização:Des.}
\end{itemize}
Acto de exaustar.
\section{Exaustão}
\begin{itemize}
\item {Grp. gram.:f.}
\end{itemize}
\begin{itemize}
\item {Proveniência:(Lat. \textunderscore exhaustio\textunderscore )}
\end{itemize}
O mesmo que \textunderscore exaustação\textunderscore .
\section{Exaustar}
\begin{itemize}
\item {Grp. gram.:v. t.}
\end{itemize}
\begin{itemize}
\item {Proveniência:(Lat. \textunderscore exhaustare\textunderscore )}
\end{itemize}
(V.exaurir)
\section{Exclusão}
\begin{itemize}
\item {Grp. gram.:f.}
\end{itemize}
\begin{itemize}
\item {Proveniência:(Lat. \textunderscore exclusio\textunderscore )}
\end{itemize}
Acto ou effeito de excluír.
\section{Exclusiva}
\begin{itemize}
\item {Grp. gram.:f.}
\end{itemize}
\begin{itemize}
\item {Proveniência:(De \textunderscore exclusivo\textunderscore )}
\end{itemize}
O mesmo que \textunderscore exclusão\textunderscore .
\section{Exclusivamente}
\begin{itemize}
\item {Grp. gram.:adv.}
\end{itemize}
De modo exclusivo.
Unicamente.
\section{Exclusividade}
\begin{itemize}
\item {Grp. gram.:f.}
\end{itemize}
O mesmo que \textunderscore exclusivismo\textunderscore .
\section{Exclusivismo}
\begin{itemize}
\item {Grp. gram.:m.}
\end{itemize}
Qualidade daquillo que é exclusivo.
\section{Exclusivista}
\begin{itemize}
\item {Grp. gram.:m.  e  adj.}
\end{itemize}
Partidário do exclusivismo; intransigente.
\section{Exclusivo}
\begin{itemize}
\item {Grp. gram.:adj.}
\end{itemize}
\begin{itemize}
\item {Grp. gram.:M.}
\end{itemize}
\begin{itemize}
\item {Proveniência:(De \textunderscore excluso\textunderscore )}
\end{itemize}
Que exclue.
Direito de não têr concorrentes numa indústria ou numa empresa; monopólio: \textunderscore o exclusivo dos fósforos\textunderscore .
\section{Excluso}
\begin{itemize}
\item {Grp. gram.:adj.}
\end{itemize}
\begin{itemize}
\item {Proveniência:(Lat. \textunderscore exclusus\textunderscore )}
\end{itemize}
O mesmo que \textunderscore excluído\textunderscore . Cf. Filinto, \textunderscore D. Man.\textunderscore , II, 54.
\section{Excogitar}
\textunderscore v. t.\textunderscore  (e der.)
O mesmo que \textunderscore escogitar\textunderscore , etc.
\section{Excommungação}
\begin{itemize}
\item {Grp. gram.:f.}
\end{itemize}
(V.excommunhão)
\section{Excommungado}
\begin{itemize}
\item {Grp. gram.:m.}
\end{itemize}
\begin{itemize}
\item {Utilização:Fam.}
\end{itemize}
\begin{itemize}
\item {Proveniência:(Lat. \textunderscore excommunicatus\textunderscore )}
\end{itemize}
Indivíduo, que soffreu a excommunhão.
Indivíduo, que procede mal, que é odiado: \textunderscore não posso vêr aquelle excommungado\textunderscore .
\section{Excommungadoiro}
\begin{itemize}
\item {Grp. gram.:adj.}
\end{itemize}
\begin{itemize}
\item {Utilização:Ant.}
\end{itemize}
\begin{itemize}
\item {Proveniência:(De \textunderscore excommungar\textunderscore )}
\end{itemize}
Merecedor de excommunhão.
\section{Excommungar}
\begin{itemize}
\item {Grp. gram.:v. t.}
\end{itemize}
\begin{itemize}
\item {Proveniência:(Do lat. \textunderscore excommunicare\textunderscore )}
\end{itemize}
Separar da communicação com a Igreja.
Expulsar do grêmio dos Cathólicos.
Tornar maldito.
Exorcisar.
\section{Excommunhal}
\begin{itemize}
\item {Grp. gram.:adj.}
\end{itemize}
Relativo a excommunhão.
\section{Excommunhão}
\begin{itemize}
\item {Grp. gram.:f.}
\end{itemize}
\begin{itemize}
\item {Proveniência:(Do lat. hyp. \textunderscore excommunio\textunderscore )}
\end{itemize}
Acto de excommungar.
Pena ecclesiástica, que separa do grêmio christão quem a soffre.
\section{Excomungação}
\begin{itemize}
\item {Grp. gram.:f.}
\end{itemize}
(V.excomunhão)
\section{Excomungado}
\begin{itemize}
\item {Grp. gram.:m.}
\end{itemize}
\begin{itemize}
\item {Utilização:Fam.}
\end{itemize}
\begin{itemize}
\item {Proveniência:(Lat. \textunderscore excommunicatus\textunderscore )}
\end{itemize}
Indivíduo, que sofreu a excomunhão.
Indivíduo, que procede mal, que é odiado: \textunderscore não posso vêr aquele excomungado\textunderscore .
\section{Excomungadoiro}
\begin{itemize}
\item {Grp. gram.:adj.}
\end{itemize}
\begin{itemize}
\item {Utilização:Ant.}
\end{itemize}
\begin{itemize}
\item {Proveniência:(De \textunderscore excomungar\textunderscore )}
\end{itemize}
Merecedor de excomunhão.
\section{Excomungar}
\begin{itemize}
\item {Grp. gram.:v. t.}
\end{itemize}
\begin{itemize}
\item {Proveniência:(Do lat. \textunderscore excommunicare\textunderscore )}
\end{itemize}
Separar da comunicação com a Igreja.
Expulsar do grêmio dos Católicos.
Tornar maldito.
Exorcisar.
\section{Excomunhal}
\begin{itemize}
\item {Grp. gram.:adj.}
\end{itemize}
Relativo a excomunhão.
\section{Excomunhão}
\begin{itemize}
\item {Grp. gram.:f.}
\end{itemize}
\begin{itemize}
\item {Proveniência:(Do lat. hyp. \textunderscore excommunio\textunderscore )}
\end{itemize}
Acto de excomungar.
Pena eclesiástica, que separa do grêmio cristão quem a sofre.
\section{Excorco}
\begin{itemize}
\item {fónica:côr}
\end{itemize}
\begin{itemize}
\item {Grp. gram.:m.}
\end{itemize}
Peixe de Portugal.
\section{Excoriar}
\textunderscore v. t.\textunderscore  e \textunderscore p.\textunderscore  (e der.)
(V. \textunderscore escoriar\textunderscore ^1, etc.)
\section{Excorticar}
\textunderscore v. t.\textunderscore  (e der.)(V.escorticar)
\section{Excramar}
\textunderscore v. t. Ant.\textunderscore  (e der.)
O mesmo que \textunderscore exclamar\textunderscore , etc. Cf. \textunderscore Aulegrafia\textunderscore , 27.
\section{Excreção}
\begin{itemize}
\item {Grp. gram.:f.}
\end{itemize}
\begin{itemize}
\item {Proveniência:(Do lat. hypoth. \textunderscore excretio\textunderscore )}
\end{itemize}
Acção, com que certos órgãos expellem de si as matérias sólidas ou líquidas que contém.
Substância excrementícia.
\section{Excrementício}
\begin{itemize}
\item {Grp. gram.:adj.}
\end{itemize}
Que resulta da excreção.
Relativo a excremento.
Manchado como com excremento:«\textunderscore ...saleta caiada, muito excrementícia de porcaria.\textunderscore »Camillo, \textunderscore Brasileira\textunderscore , 104.
\section{Excremento}
\begin{itemize}
\item {Grp. gram.:m.}
\end{itemize}
\begin{itemize}
\item {Proveniência:(Lat. \textunderscore excrementum\textunderscore )}
\end{itemize}
Qualquer substância sólida ou líquida, que sai do corpo do homem ou dos animaes pelos canaes excretórios.
Matérias fecaes.
\section{Excrementoso}
\begin{itemize}
\item {Grp. gram.:adj.}
\end{itemize}
O mesmo que \textunderscore excrementício\textunderscore .
\section{Excrescência}
\begin{itemize}
\item {Grp. gram.:f.}
\end{itemize}
\begin{itemize}
\item {Proveniência:(Lat. \textunderscore excrescentia\textunderscore )}
\end{itemize}
Saliência.
Tumor, numa superfície dos órgãos.
Superfluidade.
\section{Excrescente}
\begin{itemize}
\item {Grp. gram.:adj.}
\end{itemize}
\begin{itemize}
\item {Proveniência:(Lat. \textunderscore excrescens\textunderscore )}
\end{itemize}
Que excresce.
\section{Excrescer}
\begin{itemize}
\item {Grp. gram.:v. i.}
\end{itemize}
\begin{itemize}
\item {Proveniência:(Lat. \textunderscore excrescere\textunderscore )}
\end{itemize}
Crescer muito.
Entumecer.
Fazer excrescência.
\section{Excretar}
\begin{itemize}
\item {Grp. gram.:v. t.}
\end{itemize}
\begin{itemize}
\item {Proveniência:(De \textunderscore excreto\textunderscore )}
\end{itemize}
Expellir do corpo, evacuar.
\section{Excreto}
\begin{itemize}
\item {Grp. gram.:adj.}
\end{itemize}
\begin{itemize}
\item {Grp. gram.:M.}
\end{itemize}
\begin{itemize}
\item {Proveniência:(Lat. \textunderscore excretus\textunderscore )}
\end{itemize}
Que saíu pelos canaes excretórios.
Effeito de excretar:«\textunderscore ...dos estercos dos gados, dos excretos dos homens.\textunderscore »F. Lapa, \textunderscore Alman. do Lavr.\textunderscore , 33.
\section{Excretor}
\begin{itemize}
\item {Grp. gram.:adj.}
\end{itemize}
\begin{itemize}
\item {Proveniência:(De \textunderscore excreto\textunderscore )}
\end{itemize}
Que excreta.
Que realiza a excreção: \textunderscore canal excretor\textunderscore .
\section{Excretório}
\begin{itemize}
\item {Grp. gram.:adj.}
\end{itemize}
\begin{itemize}
\item {Proveniência:(De \textunderscore excreto\textunderscore )}
\end{itemize}
Que excreta.
Que realiza a excreção: \textunderscore canal excretório\textunderscore .
\section{Excruciante}
\begin{itemize}
\item {Grp. gram.:adj.}
\end{itemize}
\begin{itemize}
\item {Proveniência:(Do lat. \textunderscore excrucians\textunderscore )}
\end{itemize}
Que excrucia.
Pungente, lancinante: \textunderscore dôres excruciantes\textunderscore .
\section{Excruciar}
\begin{itemize}
\item {Grp. gram.:v. t.}
\end{itemize}
\begin{itemize}
\item {Proveniência:(Lat. \textunderscore excruciare\textunderscore )}
\end{itemize}
Affligir muito.
Atormentar; martyrizar.
\section{Exculpação}
\begin{itemize}
\item {Grp. gram.:f.}
\end{itemize}
\begin{itemize}
\item {Proveniência:(De \textunderscore ex...\textunderscore  + \textunderscore culpar\textunderscore )}
\end{itemize}
(V.desculpa)
\section{Excursão}
\begin{itemize}
\item {Grp. gram.:f.}
\end{itemize}
\begin{itemize}
\item {Utilização:Fig.}
\end{itemize}
\begin{itemize}
\item {Proveniência:(Lat. \textunderscore excursio\textunderscore )}
\end{itemize}
Jornada ou passeio, para recreio ou instrucção, a pequena distância do lugar em que se reside.
Correria em território inimigo.
Incursão.
Divagação.
Dissertação sôbre um thema antigo.
\section{Excursar}
\begin{itemize}
\item {Grp. gram.:v. i.}
\end{itemize}
Fazer excurso; discorrer. Cf. Camillo, \textunderscore Sc. da Hora Final\textunderscore , 77.
\section{Excursionar}
\begin{itemize}
\item {Grp. gram.:v. i.}
\end{itemize}
\begin{itemize}
\item {Utilização:Neol.}
\end{itemize}
Fazer excursão. Cf. Camillo, \textunderscore Volcões\textunderscore , 212.
\section{Excursionismo}
\begin{itemize}
\item {Grp. gram.:m.}
\end{itemize}
Gôsto e prática das viagens de recreio ou de estudo.--É fórma preferível ao exótico \textunderscore turismo\textunderscore .
(Cp. \textunderscore excursionista\textunderscore )
\section{Excursionista}
\begin{itemize}
\item {Grp. gram.:m.  e  f.}
\end{itemize}
\begin{itemize}
\item {Proveniência:(Do lat. \textunderscore excursio\textunderscore )}
\end{itemize}
Pessôa que faz excursão.
\section{Excurso}
\begin{itemize}
\item {Grp. gram.:m.}
\end{itemize}
\begin{itemize}
\item {Proveniência:(Lat. \textunderscore excursus\textunderscore )}
\end{itemize}
Excursão; divagação.
\section{Excursor}
\begin{itemize}
\item {Grp. gram.:m.}
\end{itemize}
\begin{itemize}
\item {Proveniência:(Lat. \textunderscore excursor\textunderscore )}
\end{itemize}
O mesmo que \textunderscore excursionista\textunderscore .
\section{Excusar}
\textunderscore v. t.\textunderscore  e \textunderscore i.\textunderscore  (e der.)
(V. \textunderscore escusar\textunderscore , etc.)
\section{Excussão}
\begin{itemize}
\item {Grp. gram.:f.}
\end{itemize}
\begin{itemize}
\item {Proveniência:(Lat. \textunderscore excussio\textunderscore )}
\end{itemize}
Acto de excutir.
\section{Excutir}
\begin{itemize}
\item {Grp. gram.:v.}
\end{itemize}
\begin{itemize}
\item {Utilização:t. Jur.}
\end{itemize}
\begin{itemize}
\item {Proveniência:(Lat. \textunderscore excutere\textunderscore )}
\end{itemize}
Executar judicialmente os bens de (um principal devedor).
\section{Execração}
\begin{itemize}
\item {Grp. gram.:f.}
\end{itemize}
\begin{itemize}
\item {Proveniência:(Lat. \textunderscore exsecratrio\textunderscore )}
\end{itemize}
Acto de execrar.
Aversão profunda.
Ódio entranhado.
Imprecação.
Perda da qualidade de consagrado.
Aquelle ou aquillo que se execra.
\section{Execrador}
\begin{itemize}
\item {Grp. gram.:adj.}
\end{itemize}
\begin{itemize}
\item {Grp. gram.:M.}
\end{itemize}
\begin{itemize}
\item {Proveniência:(Lat. \textunderscore exsecrator\textunderscore )}
\end{itemize}
Que execra.
Aquelle que execra.
\section{Execrando}
\begin{itemize}
\item {Grp. gram.:adj.}
\end{itemize}
\begin{itemize}
\item {Proveniência:(Do lat. \textunderscore exsecrandus\textunderscore )}
\end{itemize}
O mesmo que \textunderscore execrável\textunderscore .
\section{Execrar}
\begin{itemize}
\item {Grp. gram.:v. t.}
\end{itemize}
\begin{itemize}
\item {Proveniência:(Lat. \textunderscore exsecrari\textunderscore )}
\end{itemize}
Desejar mal a.
Amaldiçoar.
Detestar.
\section{Execratório}
\begin{itemize}
\item {Grp. gram.:adj.}
\end{itemize}
\begin{itemize}
\item {Proveniência:(Do lat. \textunderscore exsecratus\textunderscore )}
\end{itemize}
Que envolve execração.
\section{Execrável}
\begin{itemize}
\item {Grp. gram.:adj.}
\end{itemize}
\begin{itemize}
\item {Proveniência:(Lat. \textunderscore exsecrabilis\textunderscore )}
\end{itemize}
Que merece execração.
\section{Execravelmente}
\begin{itemize}
\item {Grp. gram.:adv.}
\end{itemize}
De modo execrável.
\section{Execução}
\begin{itemize}
\item {Grp. gram.:f.}
\end{itemize}
\begin{itemize}
\item {Utilização:Des.}
\end{itemize}
\begin{itemize}
\item {Proveniência:(Lat. \textunderscore exsecutio\textunderscore )}
\end{itemize}
Acto, effeito ou modo de executar.
Capacidade para executar.
Cumprimento de sentença judicial.
Arrestação e venda de bens, para pagamento de dívidas.
Supplício de um condemnado.
Vexame, prepotência.
\section{Execudor}
\begin{itemize}
\item {Grp. gram.:m.}
\end{itemize}
\begin{itemize}
\item {Utilização:Ant.}
\end{itemize}
O mesmo que \textunderscore executor\textunderscore .
\section{Executado}
\begin{itemize}
\item {Grp. gram.:adj.}
\end{itemize}
\begin{itemize}
\item {Grp. gram.:M.}
\end{itemize}
Que se executou.
Realizado.
Que soffreu execução judicial.
Aquelle que é réu, numa execução judicial.
Aquelle que soffreu a pena de morte.
\section{Executante}
\begin{itemize}
\item {Grp. gram.:m.  e  adj.}
\end{itemize}
O que executa.
\section{Executar}
\begin{itemize}
\item {Grp. gram.:v. t.}
\end{itemize}
\begin{itemize}
\item {Proveniência:(Do lat. \textunderscore exsecutus\textunderscore )}
\end{itemize}
Levar a effeito.
Realizar: \textunderscore executar um plano\textunderscore .
Cumprir: \textunderscore executar os seus deveres\textunderscore .
Tocar ou cantar (uma peça de música).
Representar em scena.
Pintar, seguindo (um plano ou modêlo).
Suppliciar em cumprimento da lei.
Obrigar judicialmente ao pagamento de uma divida.
Obrigar á penhora e leilão (bens de um devedor), para pagamento de dívida.
\section{Executável}
\begin{itemize}
\item {Grp. gram.:adj.}
\end{itemize}
Que se póde executar.
\section{Executivamente}
\begin{itemize}
\item {Grp. gram.:adv.}
\end{itemize}
De modo executivo.
\section{Executivo}
\begin{itemize}
\item {Grp. gram.:adj.}
\end{itemize}
\begin{itemize}
\item {Proveniência:(De \textunderscore executar\textunderscore )}
\end{itemize}
Que executa.
Encarregado da execução: \textunderscore o poder executivo\textunderscore .
Relativo a execução.
Relativo a penhora ou execução judicial.
Enérgico; decisivo.
\section{Executor}
\begin{itemize}
\item {Grp. gram.:adj.}
\end{itemize}
\begin{itemize}
\item {Grp. gram.:M.}
\end{itemize}
\begin{itemize}
\item {Proveniência:(Lat. \textunderscore exsecutor\textunderscore )}
\end{itemize}
Que executa.
Aquelle que executa.
\section{Executória}
\begin{itemize}
\item {Grp. gram.:f.}
\end{itemize}
\begin{itemize}
\item {Proveniência:(De \textunderscore executor\textunderscore )}
\end{itemize}
Repartição, que cura da cobrança ou execução dos rendimentos e créditos de uma communidade.
\section{Executoriamente}
\begin{itemize}
\item {Grp. gram.:adv.}
\end{itemize}
De modo executório.
\section{Executório}
\begin{itemize}
\item {Grp. gram.:adj.}
\end{itemize}
Que se há de executar.
\section{Exedra}
\begin{itemize}
\item {Grp. gram.:f.}
\end{itemize}
\begin{itemize}
\item {Utilização:Ant.}
\end{itemize}
\begin{itemize}
\item {Proveniência:(Gr. \textunderscore exedra\textunderscore )}
\end{itemize}
Sala ou pórtico, em que se juntavam os philósophos para discutir.
\section{Exegese}
\begin{itemize}
\item {Grp. gram.:f.}
\end{itemize}
\begin{itemize}
\item {Proveniência:(Lat. \textunderscore exegesis\textunderscore )}
\end{itemize}
Explicação grammatical, palavra por palavra.
Interpretação grammatical e histórica da \textunderscore Biblia\textunderscore .
Explicação do texto das leis.
Interpretação histórica.
Explicação; commentário.
\section{Exegeta}
\begin{itemize}
\item {Grp. gram.:m.}
\end{itemize}
\begin{itemize}
\item {Proveniência:(Gr. \textunderscore exegetes\textunderscore )}
\end{itemize}
Aquelle que se dedica á exegese.
\section{Exegética}
\begin{itemize}
\item {Grp. gram.:f.}
\end{itemize}
\begin{itemize}
\item {Proveniência:(De \textunderscore exegético\textunderscore )}
\end{itemize}
Parte da Theologia, que trata da exegese bíblica.
\section{Exegético}
\begin{itemize}
\item {Grp. gram.:adj.}
\end{itemize}
\begin{itemize}
\item {Proveniência:(Lat. \textunderscore exegeticus\textunderscore )}
\end{itemize}
Relativo á exegese.
\section{Exempção}
\begin{itemize}
\item {Grp. gram.:f.}
\end{itemize}
\begin{itemize}
\item {Proveniência:(Lat. \textunderscore exemptio\textunderscore )}
\end{itemize}
O mesmo que \textunderscore isenção\textunderscore .
\section{Exemplador}
\begin{itemize}
\item {Grp. gram.:m.  e  adj.}
\end{itemize}
\begin{itemize}
\item {Utilização:Des.}
\end{itemize}
\begin{itemize}
\item {Proveniência:(De \textunderscore exemplar\textunderscore ^3)}
\end{itemize}
O que castiga para dar exemplo.
\section{Exemplar}
\begin{itemize}
\item {Grp. gram.:adj.}
\end{itemize}
\begin{itemize}
\item {Proveniência:(Lat. \textunderscore exemplaris\textunderscore )}
\end{itemize}
Que serve ou póde servir de exemplo: \textunderscore cidadão exemplar\textunderscore .
\section{Exemplar}
\begin{itemize}
\item {Grp. gram.:m.}
\end{itemize}
\begin{itemize}
\item {Proveniência:(Lat. \textunderscore exemplar\textunderscore )}
\end{itemize}
Modêlo para se imitar ou copiar.
Aquillo que se deve imitar.
Cópia.
Cada um dos livros, gravuras ou quaesquer objectos, que se multiplicam, segundo um typo commum.
Cada indivíduo da mesma variedade ou espécie.
\section{Exemplar}
\begin{itemize}
\item {Grp. gram.:v. t.}
\end{itemize}
\begin{itemize}
\item {Utilização:Ant.}
\end{itemize}
\begin{itemize}
\item {Proveniência:(De \textunderscore exemplo\textunderscore )}
\end{itemize}
Mostrar com ostentação.
Dar exemplo de.
\section{Exemplaridade}
\begin{itemize}
\item {Grp. gram.:f.}
\end{itemize}
Qualidade daquillo que é exemplar.
\section{Exemplário}
\begin{itemize}
\item {Grp. gram.:m.}
\end{itemize}
\begin{itemize}
\item {Proveniência:(Lat. \textunderscore exemplarium\textunderscore )}
\end{itemize}
Livro ou collecção de exemplos.
\section{Exemplarmente}
\begin{itemize}
\item {Grp. gram.:adv.}
\end{itemize}
De modo exemplar.
\section{Exemplificação}
\begin{itemize}
\item {Grp. gram.:f.}
\end{itemize}
Acto de exemplificar.
\section{Exemplificante}
\begin{itemize}
\item {Grp. gram.:adj.}
\end{itemize}
Que exemplifica.
\section{Exemplificar}
\begin{itemize}
\item {Grp. gram.:v. t.}
\end{itemize}
\begin{itemize}
\item {Proveniência:(Do lat. \textunderscore exemplum\textunderscore  + \textunderscore facere\textunderscore )}
\end{itemize}
Provar com exemplos.
Applicar como exemplo.
\section{Exemplificativo}
\begin{itemize}
\item {Grp. gram.:adj.}
\end{itemize}
Que exemplifica.
\section{Exemplo}
\begin{itemize}
\item {Grp. gram.:m.}
\end{itemize}
\begin{itemize}
\item {Proveniência:(Lat. \textunderscore exemplum\textunderscore )}
\end{itemize}
Narração de um caso particular, adduzido para explicar alguma coisa.
Aquillo que póde ou deve sêr imitado.
Facto ou palavras alheias, com que se procura fazer uma demonstração.
Caso análogo àquelle de que se trata.
Aquillo que serve de lição.
Anexim, rifão.
Modêlo, exemplar: \textunderscore Egas Moniz, exemplo de lealdade\textunderscore .
\section{Exemplo}
\begin{itemize}
\item {Grp. gram.:m.}
\end{itemize}
\begin{itemize}
\item {Utilização:Zool.}
\end{itemize}
Espécie de crustáceo, (\textunderscore inachus scorpio\textunderscore , Fabricius).
\section{Exempro}
\textunderscore m.\textunderscore  (e der.)
(Fórma ant., em vez de \textunderscore exemplo\textunderscore , etc. Cf. Aulegrafia, 4)
\section{Exemptar}
\begin{itemize}
\item {Proveniência:(Do lat. \textunderscore exemptus\textunderscore )}
\end{itemize}
\textunderscore v. t.\textunderscore  (e der.)
O mesmo que \textunderscore isentar\textunderscore , etc.
\section{Exenteração}
\begin{itemize}
\item {Grp. gram.:f.}
\end{itemize}
\begin{itemize}
\item {Proveniência:(Do gr. \textunderscore ex\textunderscore  + \textunderscore enteros\textunderscore . Cp. lat. \textunderscore exenterare\textunderscore )}
\end{itemize}
Acto de estripar.
Operação de vazar um ôlho.
\section{Exequente}
\begin{itemize}
\item {fónica:cu-en}
\end{itemize}
\begin{itemize}
\item {Grp. gram.:m.  e  adj.}
\end{itemize}
\begin{itemize}
\item {Utilização:Jur.}
\end{itemize}
\begin{itemize}
\item {Proveniência:(Lat. \textunderscore exsequens\textunderscore )}
\end{itemize}
Pessôa, que intenta processo executivo.
\section{Exequial}
\begin{itemize}
\item {Grp. gram.:adj.}
\end{itemize}
\begin{itemize}
\item {Proveniência:(Lat. \textunderscore exsequialis\textunderscore )}
\end{itemize}
Relativo a exéquias.
\section{Exéquias}
\begin{itemize}
\item {Grp. gram.:f.}
\end{itemize}
\begin{itemize}
\item {Proveniência:(Lat. \textunderscore exsequiae\textunderscore )}
\end{itemize}
Ceremónias ou honras fúnebres.
Conjunto das pessôas, que acompanham um cadáver á sepultura.
Cortejo fúnebre.
\section{Exequibilidade}
\begin{itemize}
\item {fónica:cu-i}
\end{itemize}
\begin{itemize}
\item {Grp. gram.:f.}
\end{itemize}
Qualidade daquillo que é exequivel.
\section{Exequído}
\begin{itemize}
\item {fónica:cu-i}
\end{itemize}
\begin{itemize}
\item {Grp. gram.:adj.}
\end{itemize}
\begin{itemize}
\item {Utilização:Des.}
\end{itemize}
\begin{itemize}
\item {Proveniência:(Do lat. \textunderscore exsequi\textunderscore )}
\end{itemize}
Executado.
\section{Exequível}
\begin{itemize}
\item {fónica:cu-i}
\end{itemize}
\begin{itemize}
\item {Grp. gram.:adj.}
\end{itemize}
\begin{itemize}
\item {Proveniência:(Do lat. \textunderscore exsequi\textunderscore )}
\end{itemize}
Que se póde executar: \textunderscore alvitre exequível\textunderscore .
\section{Exercer}
\begin{itemize}
\item {Grp. gram.:v. t.}
\end{itemize}
\begin{itemize}
\item {Proveniência:(Lat. \textunderscore exercere\textunderscore )}
\end{itemize}
Praticar.
Levar a effeito.
Exercitar.
Dedicar-se a: \textunderscore exercer um offício\textunderscore .
Cumprir as obrigações correspondentes a (um offício ou cargo).
\textunderscore Exercer influência\textunderscore , influir.
\section{Exercício}
\begin{itemize}
\item {Grp. gram.:m.}
\end{itemize}
\begin{itemize}
\item {Proveniência:(Lat. \textunderscore exercitium\textunderscore )}
\end{itemize}
Acto de exercer ou de exercitar.
Conjunto dos movimentos espontâneos do corpo.
Movimentos calculados e methódicos, em gymnástica.
Aquillo com que se exercitam as faculdades moraes e intellectuaes.
Recreio.
Prática, uso.
Manobras militares, para instrucção.
\section{Exercitação}
\begin{itemize}
\item {Grp. gram.:f.}
\end{itemize}
\begin{itemize}
\item {Proveniência:(Lat. \textunderscore exercitatio\textunderscore )}
\end{itemize}
Exercício.
\section{Exercitador}
\textunderscore m.\textunderscore  e \textunderscore adj.\textunderscore 
O que exercita.
\section{Exercitamento}
\begin{itemize}
\item {Grp. gram.:m.}
\end{itemize}
O mesmo que \textunderscore exercitação\textunderscore .
\section{Exercitante}
\begin{itemize}
\item {Grp. gram.:m.  e  adj.}
\end{itemize}
\begin{itemize}
\item {Utilização:Des.}
\end{itemize}
\begin{itemize}
\item {Proveniência:(Lat. \textunderscore exercitans\textunderscore )}
\end{itemize}
O que se exercita.
\section{Exercitar}
\begin{itemize}
\item {Grp. gram.:v. t.}
\end{itemize}
\begin{itemize}
\item {Proveniência:(Lat. \textunderscore exercitare\textunderscore )}
\end{itemize}
Exercer.
Tornar destro pelo exercício: \textunderscore exercitar um recruta\textunderscore .
Praticar; cultivar: \textunderscore exercitar a Medicina\textunderscore .
\section{Exército}
\begin{itemize}
\item {Grp. gram.:m.}
\end{itemize}
\begin{itemize}
\item {Proveniência:(Lat. \textunderscore exercitus\textunderscore )}
\end{itemize}
Conjunto das tropas regulares de uma nação.
Porção de tropas, dispostas para a guerra.
\section{Exercitor}
\begin{itemize}
\item {Grp. gram.:m.}
\end{itemize}
\begin{itemize}
\item {Proveniência:(Lat. \textunderscore exercitor\textunderscore )}
\end{itemize}
Aquelle que administra um navio, ou carga de um navio, por tempo determinado.
\section{Exercitório}
\begin{itemize}
\item {Grp. gram.:adj.}
\end{itemize}
\begin{itemize}
\item {Utilização:bras}
\end{itemize}
\begin{itemize}
\item {Utilização:Jur.}
\end{itemize}
\begin{itemize}
\item {Proveniência:(Lat. \textunderscore exercitorius\textunderscore )}
\end{itemize}
Relativo a exercício: \textunderscore acção exercitória\textunderscore .
\section{Exerese}
\begin{itemize}
\item {Grp. gram.:f.}
\end{itemize}
\begin{itemize}
\item {Proveniência:(Gr. \textunderscore exairesis\textunderscore )}
\end{itemize}
Operação cirúrgica, com que se tira do corpo o que lhe é nocivo ou supérfluo.
\section{Exergo}
\begin{itemize}
\item {Grp. gram.:m.}
\end{itemize}
\begin{itemize}
\item {Proveniência:(Do gr. \textunderscore ex\textunderscore  + \textunderscore ergon\textunderscore )}
\end{itemize}
Espaço numa moéda ou medalha, para uma inscripção, data, etc.
Essa data ou inscripção.
\section{Exfetação}
\begin{itemize}
\item {Grp. gram.:f.}
\end{itemize}
\begin{itemize}
\item {Proveniência:(Do lat. \textunderscore ex\textunderscore  + \textunderscore foetare\textunderscore )}
\end{itemize}
Prenhez extra-uterina.
\section{Exfoliar}
\textunderscore v. t.\textunderscore  (e der.)
(V. \textunderscore esfoliar\textunderscore , etc.)
\section{Exgregar}
\begin{itemize}
\item {Grp. gram.:v. t.}
\end{itemize}
\begin{itemize}
\item {Proveniência:(Do lat. \textunderscore ex\textunderscore  + \textunderscore gregare\textunderscore )}
\end{itemize}
O mesmo que \textunderscore segregar\textunderscore . Cf. Castilho, \textunderscore Fastos\textunderscore , II, 653.
\section{Exhalação}
\begin{itemize}
\item {Grp. gram.:f.}
\end{itemize}
\begin{itemize}
\item {Proveniência:(Lat. \textunderscore exhalatio\textunderscore )}
\end{itemize}
Acto de exhalar ou de se exhalar.
Restituição, que as plantas fazem á atmosphera, dos gases que absorveram.
Emanação, imperceptível á vista, de uma substância sólida ou líquida.
Evaporação.
Vapor.
Cheiro.
Diffusão de certos fluidos orgânicos sôbre certas membranas ou á superfície da pelle.
\section{Exhalante}
\begin{itemize}
\item {Grp. gram.:adj.}
\end{itemize}
\begin{itemize}
\item {Proveniência:(Lat. \textunderscore exhalans\textunderscore )}
\end{itemize}
Que exhala.
\section{Exhalar}
\begin{itemize}
\item {Grp. gram.:v. t.}
\end{itemize}
\begin{itemize}
\item {Grp. gram.:V. p.}
\end{itemize}
\begin{itemize}
\item {Proveniência:(Lat. \textunderscore exhalare\textunderscore )}
\end{itemize}
Espirar, lançar de si.
Emittir: \textunderscore exhalar aroma\textunderscore .
Soltar.
Expandir.
Evaporar.
Evaporar-se; evolar-se; desapparecer.
\section{Exhaurição}
\begin{itemize}
\item {Grp. gram.:f.}
\end{itemize}
Acto ou effeito de exhaurir.
\section{Exhaurimento}
\begin{itemize}
\item {Grp. gram.:m.}
\end{itemize}
O mesmo que \textunderscore exhaurição\textunderscore .
\section{Exhaurir}
\begin{itemize}
\item {Grp. gram.:v. t.}
\end{itemize}
\begin{itemize}
\item {Proveniência:(Lat. \textunderscore exhaurire\textunderscore )}
\end{itemize}
Esgotar completamente.
Dissipar completamente.
Depauperar.
\section{Exhaurível}
\begin{itemize}
\item {Grp. gram.:adj.}
\end{itemize}
Que se póde exhaurir.
\section{Exhaustação}
\begin{itemize}
\item {Grp. gram.:f.}
\end{itemize}
\begin{itemize}
\item {Utilização:Des.}
\end{itemize}
Acto de exhaustar.
\section{Exhaustão}
\begin{itemize}
\item {Grp. gram.:f.}
\end{itemize}
\begin{itemize}
\item {Proveniência:(Lat. \textunderscore exhaustio\textunderscore )}
\end{itemize}
O mesmo que \textunderscore exhaustação\textunderscore .
\section{Exhaustar}
\begin{itemize}
\item {Grp. gram.:v. t.}
\end{itemize}
\begin{itemize}
\item {Proveniência:(Lat. \textunderscore exhaustare\textunderscore )}
\end{itemize}
(V.exhaurir)
\section{Exaustivo}
\begin{itemize}
\item {Grp. gram.:adj.}
\end{itemize}
\begin{itemize}
\item {Proveniência:(De \textunderscore exausto\textunderscore )}
\end{itemize}
Que esgota ou que serve para esgotar.
Extremamente fatigante: \textunderscore trabalhos exaustivos\textunderscore .
\section{Exausto}
\begin{itemize}
\item {Grp. gram.:adj.}
\end{itemize}
\begin{itemize}
\item {Proveniência:(Lat. \textunderscore exhaustus\textunderscore )}
\end{itemize}
Que se exauriu.
Esgotado.
Acabado; extinto: \textunderscore fôrças exaustas\textunderscore .
\section{Exerdar}
\textunderscore v. t.\textunderscore  (e der.)
O mesmo que \textunderscore deserdar\textunderscore , etc.
\section{Exhaustivo}
\begin{itemize}
\item {Grp. gram.:adj.}
\end{itemize}
\begin{itemize}
\item {Proveniência:(De \textunderscore exhausto\textunderscore )}
\end{itemize}
Que esgota ou que serve para esgotar.
Extremamente fatigante: \textunderscore trabalhos exhaustivos\textunderscore .
\section{Exhausto}
\begin{itemize}
\item {Grp. gram.:adj.}
\end{itemize}
\begin{itemize}
\item {Proveniência:(Lat. \textunderscore exhaustus\textunderscore )}
\end{itemize}
Que se exhauriu.
Esgotado.
Acabado; extinto: \textunderscore fôrças exhaustas\textunderscore .
\section{Exherdar}
\textunderscore v. t.\textunderscore  (e der.)
O mesmo que \textunderscore desherdar\textunderscore , etc.
\section{Exhibição}
\begin{itemize}
\item {Grp. gram.:f.}
\end{itemize}
\begin{itemize}
\item {Proveniência:(Lat. \textunderscore exhibitio\textunderscore )}
\end{itemize}
Acto de exihibir.
\section{Exhibicionismo}
\begin{itemize}
\item {Grp. gram.:m.}
\end{itemize}
\begin{itemize}
\item {Proveniência:(Do lat. \textunderscore exhibitio\textunderscore )}
\end{itemize}
Mania da ostentação.
\section{Exhibicionista}
\begin{itemize}
\item {Grp. gram.:m.}
\end{itemize}
Aquelle que gosta ou tem a mania do exhibicionismo.
\section{Exhibir}
\begin{itemize}
\item {Grp. gram.:f.}
\end{itemize}
\begin{itemize}
\item {Proveniência:(Lat. \textunderscore exhibere\textunderscore )}
\end{itemize}
Tornar patente; apresentar; expor: \textunderscore exhibir argumentos\textunderscore .
\section{Exhibitório}
\begin{itemize}
\item {Grp. gram.:adj.}
\end{itemize}
\begin{itemize}
\item {Proveniência:(Lat. \textunderscore exhibitorius\textunderscore )}
\end{itemize}
Relativo a exhibição.
\section{Exhistórico}
\begin{itemize}
\item {Grp. gram.:adj.}
\end{itemize}
\begin{itemize}
\item {Utilização:P. us.}
\end{itemize}
\begin{itemize}
\item {Proveniência:(De \textunderscore ex...\textunderscore  + \textunderscore histórico\textunderscore )}
\end{itemize}
Prehistórico.
Estranho á história; lendário.
\section{Exhortacão}
\begin{itemize}
\item {Grp. gram.:f.}
\end{itemize}
\begin{itemize}
\item {Proveniência:(Lat. \textunderscore exhortatio\textunderscore )}
\end{itemize}
Acto de exhortar.
Conselho.
Admoestação, advertência.
Palavras, com que se procura reformar ou melhorar os actos, costumes ou opiniões de alguém.
\section{Exhortador}
\begin{itemize}
\item {Grp. gram.:m.}
\end{itemize}
\begin{itemize}
\item {Proveniência:(Lat. \textunderscore exhortator\textunderscore )}
\end{itemize}
Aquelle que exhorta.
\section{Exhortar}
\begin{itemize}
\item {Grp. gram.:v. t.}
\end{itemize}
\begin{itemize}
\item {Proveniência:(Lat. \textunderscore exhortari\textunderscore )}
\end{itemize}
Excitar á prática de alguma coisa.
Procurar convencer, por meio de palavras.
Estimular.
Advertir; aconselhar.
\section{Exhortativo}
\begin{itemize}
\item {Grp. gram.:adj.}
\end{itemize}
\begin{itemize}
\item {Proveniência:(Lat. \textunderscore exhortativus\textunderscore )}
\end{itemize}
Que exhorta.
Que é próprio para exhortar: \textunderscore palavras exhortativas\textunderscore .
\section{Exhortatória}
\begin{itemize}
\item {Grp. gram.:f.}
\end{itemize}
Exhortação.
Discurso, em que se exhorta ou aconselha. Cf. Filinto, \textunderscore D. Man.\textunderscore  I, 344; Arn. Gama, \textunderscore Motim\textunderscore , 58.
\section{Exhortatório}
\begin{itemize}
\item {Grp. gram.:adj.}
\end{itemize}
\begin{itemize}
\item {Proveniência:(Lat. \textunderscore exhortatorius\textunderscore )}
\end{itemize}
Que envolve exhortação: \textunderscore uma carta exhortatória\textunderscore .
\section{Exhumação}
\begin{itemize}
\item {Grp. gram.:f.}
\end{itemize}
Acto de exhumar.
\section{Exhumar}
\begin{itemize}
\item {Grp. gram.:v. t.}
\end{itemize}
\begin{itemize}
\item {Utilização:Fig.}
\end{itemize}
\begin{itemize}
\item {Proveniência:(Do lat. \textunderscore ex\textunderscore  + \textunderscore humus\textunderscore )}
\end{itemize}
Desenterrar.
Escavar.
Descobrir por meio de investigações.
\section{Exhymenina}
\begin{itemize}
\item {Grp. gram.:f.}
\end{itemize}
\begin{itemize}
\item {Utilização:Bot.}
\end{itemize}
\begin{itemize}
\item {Proveniência:(Do gr. \textunderscore ex\textunderscore  + \textunderscore humen\textunderscore )}
\end{itemize}
Membrana externa do póllen.
\section{Exibição}
\begin{itemize}
\item {Grp. gram.:f.}
\end{itemize}
\begin{itemize}
\item {Proveniência:(Lat. \textunderscore exhibitio\textunderscore )}
\end{itemize}
Acto de exibir.
\section{Exibicionismo}
\begin{itemize}
\item {Grp. gram.:m.}
\end{itemize}
\begin{itemize}
\item {Proveniência:(Do lat. \textunderscore exhibitio\textunderscore )}
\end{itemize}
Mania da ostentação.
\section{Exibicionista}
\begin{itemize}
\item {Grp. gram.:m.}
\end{itemize}
Aquele que gosta ou tem a mania do exibicionismo.
\section{Exibir}
\begin{itemize}
\item {Grp. gram.:f.}
\end{itemize}
\begin{itemize}
\item {Proveniência:(Lat. \textunderscore exhibere\textunderscore )}
\end{itemize}
Tornar patente; apresentar; expor: \textunderscore exibir argumentos\textunderscore .
\section{Exibitório}
\begin{itemize}
\item {Grp. gram.:adj.}
\end{itemize}
\begin{itemize}
\item {Proveniência:(Lat. \textunderscore exhibitorius\textunderscore )}
\end{itemize}
Relativo a exibição.
\section{Exicial}
\begin{itemize}
\item {Grp. gram.:adj.}
\end{itemize}
\begin{itemize}
\item {Utilização:Fig.}
\end{itemize}
\begin{itemize}
\item {Proveniência:(Lat. \textunderscore exitialis\textunderscore )}
\end{itemize}
Relativo a exício.
Que produz ruína ou morte.
Funesto.
\section{Exício}
\begin{itemize}
\item {Grp. gram.:m.}
\end{itemize}
\begin{itemize}
\item {Proveniência:(Lat. \textunderscore exitium\textunderscore )}
\end{itemize}
Perdição.
Ruína.
Morte.
\section{Exido}
\begin{itemize}
\item {fónica:xi}
\end{itemize}
\begin{itemize}
\item {Grp. gram.:m.}
\end{itemize}
\begin{itemize}
\item {Proveniência:(Do lat. \textunderscore exitus\textunderscore ?)}
\end{itemize}
Baldio ou terreno inculto, fóra de cidade ou villa, para passeio ou pastagem.
\section{Exigência}
\begin{itemize}
\item {Grp. gram.:f.}
\end{itemize}
\begin{itemize}
\item {Utilização:Fam.}
\end{itemize}
Acto de exigir.
Urgência.
Pedido importuno.
\section{Exigente}
\begin{itemize}
\item {Grp. gram.:adj.}
\end{itemize}
\begin{itemize}
\item {Proveniência:(Lat. \textunderscore exigens\textunderscore )}
\end{itemize}
Que exige.
Impertinente.
Que difficilmente se satisfaz.
\section{Exigibilidade}
\begin{itemize}
\item {Grp. gram.:f.}
\end{itemize}
Qualidade daquillo que é exigível.
\section{Exigir}
\begin{itemize}
\item {Grp. gram.:v. t.}
\end{itemize}
\begin{itemize}
\item {Proveniência:(Lat. \textunderscore exigere\textunderscore )}
\end{itemize}
Reclamar com direito ou com apparência de direito: \textunderscore exigir o pagamento de uma dívida\textunderscore .
Obrigar a, sem fundamento justo.
Intimar, ordenar: \textunderscore exijo que se vá embora\textunderscore .
Carecer de: \textunderscore êsse terreno exige adubos\textunderscore .
\section{Exigível}
\begin{itemize}
\item {Grp. gram.:adj.}
\end{itemize}
Que se póde exigir.
\section{Exiguidade}
\begin{itemize}
\item {fónica:gu-i}
\end{itemize}
\begin{itemize}
\item {Grp. gram.:f.}
\end{itemize}
\begin{itemize}
\item {Proveniência:(Lat. \textunderscore exiguitas\textunderscore )}
\end{itemize}
Qualidade daquillo que é exíguo.
Escassez.
\section{Exíguo}
\begin{itemize}
\item {Grp. gram.:adj.}
\end{itemize}
\begin{itemize}
\item {Proveniência:(Lat. \textunderscore exiguus\textunderscore )}
\end{itemize}
Pequeno; limitado: \textunderscore um rendimento exíguo\textunderscore .
Tênue.
Escasso.
\section{Exil}
\begin{itemize}
\item {Grp. gram.:adj.}
\end{itemize}
\begin{itemize}
\item {Utilização:Poét.}
\end{itemize}
\begin{itemize}
\item {Proveniência:(Lat. \textunderscore exílis\textunderscore )}
\end{itemize}
Pobre; exíguo.
Mesquinho. Cf. Filinto, V, 6.
\section{Exilado}
\begin{itemize}
\item {Grp. gram.:m.}
\end{itemize}
\begin{itemize}
\item {Proveniência:(De \textunderscore exilar\textunderscore )}
\end{itemize}
Aquelle que foi expatriado.
\section{Exilar}
\begin{itemize}
\item {Grp. gram.:v. t.}
\end{itemize}
\begin{itemize}
\item {Utilização:Fig.}
\end{itemize}
\begin{itemize}
\item {Proveniência:(De \textunderscore exílio\textunderscore )}
\end{itemize}
Expulsar da pátria; expatriar.
Desterrar.
Expulsar de casa.
Afastar da convivência social.
\section{Exilária}
\begin{itemize}
\item {Grp. gram.:f.}
\end{itemize}
\begin{itemize}
\item {Proveniência:(Do lat. \textunderscore exilis\textunderscore ?)}
\end{itemize}
Gênero de algas, que crescem na água doce e na salgada.
\section{Exile}
\begin{itemize}
\item {Grp. gram.:adj.}
\end{itemize}
\begin{itemize}
\item {Utilização:Poét.}
\end{itemize}
\begin{itemize}
\item {Proveniência:(Lat. \textunderscore exílis\textunderscore )}
\end{itemize}
Pobre; exíguo.
Mesquinho. Cf. Filinto, V, 6.
\section{Exiliar}
\begin{itemize}
\item {Proveniência:(De \textunderscore exílio\textunderscore )}
\end{itemize}
\textunderscore v. t.\textunderscore  (e der.)
O mesmo que \textunderscore exilar\textunderscore . Cf. Filinto, \textunderscore D. Man.\textunderscore , I, 40.
\section{Exílio}
\begin{itemize}
\item {Grp. gram.:m.}
\end{itemize}
\begin{itemize}
\item {Proveniência:(Lat. \textunderscore exilium\textunderscore )}
\end{itemize}
Acto ou effeito de exilar.
Degrêdo, destêrro.
\section{Eximenina}
\begin{itemize}
\item {Grp. gram.:f.}
\end{itemize}
\begin{itemize}
\item {Utilização:Bot.}
\end{itemize}
\begin{itemize}
\item {Proveniência:(Do gr. \textunderscore ex\textunderscore  + \textunderscore humen\textunderscore )}
\end{itemize}
Membrana externa do pólen.
\section{Eximiamente}
\begin{itemize}
\item {Grp. gram.:adv.}
\end{itemize}
De modo exímio.
\section{Eximição}
\begin{itemize}
\item {Grp. gram.:f.}
\end{itemize}
\begin{itemize}
\item {Utilização:Des.}
\end{itemize}
\begin{itemize}
\item {Proveniência:(De \textunderscore eximir\textunderscore )}
\end{itemize}
(V.isenção)
\section{Exímio}
\begin{itemize}
\item {Grp. gram.:adj.}
\end{itemize}
\begin{itemize}
\item {Proveniência:(Lat. \textunderscore eximius\textunderscore )}
\end{itemize}
Excellente.
Insigne.
Muito illustre: \textunderscore professor exímio\textunderscore .
Magnífico.
\section{Eximir}
\begin{itemize}
\item {Grp. gram.:v. t.}
\end{itemize}
\begin{itemize}
\item {Grp. gram.:V. p.}
\end{itemize}
\begin{itemize}
\item {Proveniência:(Lat. \textunderscore eximere\textunderscore )}
\end{itemize}
O mesmo que \textunderscore isentar\textunderscore .
Escusar-se, esquivar-se.
\section{Exina}
\begin{itemize}
\item {Grp. gram.:f.}
\end{itemize}
O mesmo ou melhor que \textunderscore exhymenina\textunderscore .
\section{Exinanição}
\begin{itemize}
\item {Grp. gram.:f.}
\end{itemize}
\begin{itemize}
\item {Proveniência:(Lat. \textunderscore exinanitio\textunderscore )}
\end{itemize}
Acto de exinanir.
Esgotamento de fôrças; prostração.
\section{Exinanir}
\begin{itemize}
\item {Grp. gram.:v. t.}
\end{itemize}
\begin{itemize}
\item {Proveniência:(Lat. \textunderscore exinanire\textunderscore )}
\end{itemize}
Tornar vazio.
Aniquilar.
Enfraquecer, por falta de alimento ou por dejecções excessivas.
\section{Exir}
\begin{itemize}
\item {Grp. gram.:v. i.}
\end{itemize}
\begin{itemize}
\item {Utilização:Ant.}
\end{itemize}
\begin{itemize}
\item {Proveniência:(Lat. \textunderscore exire\textunderscore )}
\end{itemize}
Saír; derivar.
\section{Existencia}
\begin{itemize}
\item {Grp. gram.:f.}
\end{itemize}
\begin{itemize}
\item {Proveniência:(Lat. \textunderscore existentia\textunderscore )}
\end{itemize}
Estado daquelle ou daquillo que existe.
Realidade.
Modo de vida: \textunderscore leva existência desregrada\textunderscore .
Vida: \textunderscore os baldões da existência\textunderscore .
Ente.
\section{Existencial}
\begin{itemize}
\item {Grp. gram.:adj.}
\end{itemize}
\begin{itemize}
\item {Utilização:Neol.}
\end{itemize}
\begin{itemize}
\item {Proveniência:(Lat. \textunderscore existentialis\textunderscore )}
\end{itemize}
Relativo a existência; vital.
\section{Existente}
\begin{itemize}
\item {Grp. gram.:adj.}
\end{itemize}
\begin{itemize}
\item {Grp. gram.:M.}
\end{itemize}
\begin{itemize}
\item {Proveniência:(Lat. \textunderscore existens\textunderscore )}
\end{itemize}
Que existe.
Que vive.
Que há: \textunderscore vender a porção existente das mercadorias\textunderscore .
Aquillo que existe: \textunderscore revoltou-se contra o existente\textunderscore .
\section{Existir}
\begin{itemize}
\item {Grp. gram.:v. i.}
\end{itemize}
\begin{itemize}
\item {Proveniência:(Lat. \textunderscore existere\textunderscore )}
\end{itemize}
Sêr.
Viver.
Subsistir.
Estar; exhibir-se.
\section{Êxito}
\begin{itemize}
\item {Grp. gram.:m.}
\end{itemize}
\begin{itemize}
\item {Proveniência:(Lat. \textunderscore exitus\textunderscore )}
\end{itemize}
Saída.
Resultado: \textunderscore a empresa teve bom êxito\textunderscore .
Solução; fins.
\section{Ex-libris}
\begin{itemize}
\item {fónica:eis-li-bris}
\end{itemize}
\begin{itemize}
\item {Grp. gram.:m.}
\end{itemize}
Nota, escrita ou desenhada, que indica no princípio, no frontispício ou na guarda de um livro, a livraria ou pessôa, a quem pertence ou pertenceu êsse livro.
(Loc. lat.)
\section{Exocardite}
\begin{itemize}
\item {Grp. gram.:f.}
\end{itemize}
\begin{itemize}
\item {Proveniência:(Do gr. \textunderscore exo\textunderscore  + \textunderscore kardia\textunderscore )}
\end{itemize}
Inflammação da membrana externa do coração.
\section{Exocéfalo}
\begin{itemize}
\item {Grp. gram.:m.}
\end{itemize}
\begin{itemize}
\item {Proveniência:(Do gr. \textunderscore exo\textunderscore  + \textunderscore kepale\textunderscore )}
\end{itemize}
Gênero de insectos órtopteros.
\section{Exocéphalo}
\begin{itemize}
\item {Grp. gram.:m.}
\end{itemize}
\begin{itemize}
\item {Proveniência:(Do gr. \textunderscore exo\textunderscore  + \textunderscore kepale\textunderscore )}
\end{itemize}
Gênero de insectos órthopteros.
\section{Exoceto}
\begin{itemize}
\item {fónica:cê}
\end{itemize}
\begin{itemize}
\item {Grp. gram.:m.}
\end{itemize}
\begin{itemize}
\item {Proveniência:(Gr. \textunderscore exokoitos\textunderscore )}
\end{itemize}
Espécie de peixe voador.
\section{Exocraniano}
\begin{itemize}
\item {Grp. gram.:adj.}
\end{itemize}
\begin{itemize}
\item {Proveniência:(Do gr. \textunderscore exo\textunderscore  + \textunderscore kranion\textunderscore )}
\end{itemize}
Situado fóra do crânio.
\section{Exocrânio}
\begin{itemize}
\item {Grp. gram.:m.}
\end{itemize}
\begin{itemize}
\item {Utilização:Anat.}
\end{itemize}
\begin{itemize}
\item {Proveniência:(Do gr. \textunderscore exo\textunderscore  + \textunderscore kranion\textunderscore )}
\end{itemize}
A parte externa do crânio.
\section{Exoderme}
\begin{itemize}
\item {Grp. gram.:f.}
\end{itemize}
\begin{itemize}
\item {Proveniência:(Do gr. \textunderscore exo\textunderscore  + \textunderscore derma\textunderscore )}
\end{itemize}
Camada exterior ou folheto dos plastídios.
O mesmo que \textunderscore ectoderme\textunderscore .
\section{Exódico}
\begin{itemize}
\item {Grp. gram.:adj.}
\end{itemize}
\begin{itemize}
\item {Proveniência:(Do gr. \textunderscore exo\textunderscore  + \textunderscore odos\textunderscore )}
\end{itemize}
Diz-se dos nervos, em que a acção se exerce de dentro para fóra.
\section{Êxodo}
\begin{itemize}
\item {Grp. gram.:m.}
\end{itemize}
\begin{itemize}
\item {Proveniência:(Gr. \textunderscore exodos\textunderscore )}
\end{itemize}
Saída, emigração. Cf. Herculano, \textunderscore Opúsc.\textunderscore , IV, 177.
Livro bíblico, em que se conta a saída dos Hebreus das terras do Egypto.
Fecho das tragédias gregas.
\section{Exoftalmia}
\begin{itemize}
\item {Grp. gram.:f.}
\end{itemize}
O mesmo que \textunderscore exoftalmo\textunderscore .
\section{Exoftálmico}
\begin{itemize}
\item {Grp. gram.:adj.}
\end{itemize}
Relativo ao exoftalmo.
Que se manifesta com o exoftalmo, como em geral o bócio.
\section{Exoftalmo}
\begin{itemize}
\item {Grp. gram.:m.}
\end{itemize}
\begin{itemize}
\item {Proveniência:(Do gr. \textunderscore exo\textunderscore  + \textunderscore phthalmos\textunderscore )}
\end{itemize}
Saliência exagerada do globo ocular.
\section{Exogamia}
\begin{itemize}
\item {Grp. gram.:f.}
\end{itemize}
Estado de exógamo.
\section{Exógamo}
\begin{itemize}
\item {Grp. gram.:m.  e  adj.}
\end{itemize}
\begin{itemize}
\item {Proveniência:(Do gr. \textunderscore exo\textunderscore  + \textunderscore gamos\textunderscore )}
\end{itemize}
Selvagem, que se liga com mulher roubada em tribo estranha.
\section{Exogêneo}
\begin{itemize}
\item {Grp. gram.:adj.}
\end{itemize}
\begin{itemize}
\item {Utilização:Bot.}
\end{itemize}
\begin{itemize}
\item {Proveniência:(Do gr. \textunderscore exo\textunderscore  + \textunderscore gene\textunderscore )}
\end{itemize}
Que cresce exteriormente, ou para fóra.
Que está á superfície.
O mesmo que \textunderscore dicótylo\textunderscore .
\section{Exógeno}
\begin{itemize}
\item {Grp. gram.:adj.}
\end{itemize}
\begin{itemize}
\item {Utilização:Bot.}
\end{itemize}
\begin{itemize}
\item {Proveniência:(Do gr. \textunderscore exo\textunderscore  + \textunderscore gene\textunderscore )}
\end{itemize}
Que cresce exteriormente, ou para fóra.
Que está á superfície.
O mesmo que \textunderscore dicótylo\textunderscore .
\section{Exogínio}
\begin{itemize}
\item {Grp. gram.:adj.}
\end{itemize}
\begin{itemize}
\item {Utilização:Bot.}
\end{itemize}
\begin{itemize}
\item {Proveniência:(Do gr. \textunderscore exo\textunderscore  + \textunderscore gune\textunderscore )}
\end{itemize}
Cujo estilete sai para fóra da flôr.
\section{Exogýnio}
\begin{itemize}
\item {Grp. gram.:adj.}
\end{itemize}
\begin{itemize}
\item {Utilização:Bot.}
\end{itemize}
\begin{itemize}
\item {Proveniência:(Do gr. \textunderscore exo\textunderscore  + \textunderscore gune\textunderscore )}
\end{itemize}
Cujo estilete sai para fóra da flôr.
\section{Exómetra}
\begin{itemize}
\item {Grp. gram.:f.}
\end{itemize}
\begin{itemize}
\item {Proveniência:(Do gr. \textunderscore exo\textunderscore  + \textunderscore metra\textunderscore )}
\end{itemize}
Deslocação do útero.
\section{Exómide}
\begin{itemize}
\item {Grp. gram.:f.}
\end{itemize}
\begin{itemize}
\item {Proveniência:(Lat. \textunderscore exomis\textunderscore )}
\end{itemize}
Túnica sem mangas, usada pelos antigos actores cómicos, e que lhes deixava descobertos os ombros, ou um delles e uma parte do peito.
\section{Exomologese}
\begin{itemize}
\item {Grp. gram.:f.}
\end{itemize}
\begin{itemize}
\item {Proveniência:(Do gr. \textunderscore ex\textunderscore  + \textunderscore omologesis\textunderscore )}
\end{itemize}
Penitência pública.
\section{Exoneração}
\begin{itemize}
\item {Grp. gram.:f.}
\end{itemize}
\begin{itemize}
\item {Proveniência:(Lat. \textunderscore exoneratio\textunderscore )}
\end{itemize}
Acto de exonerar.
Destituição; demissão.
\section{Exonerar}
\begin{itemize}
\item {Grp. gram.:v. t.}
\end{itemize}
\begin{itemize}
\item {Proveniência:(Lat. \textunderscore exonerare\textunderscore )}
\end{itemize}
Tirar ónus a.
Desobrigar.
Alliviar.
Desempregar, destituir: \textunderscore exonerar um amanuense\textunderscore .
\section{Exonirose}
\begin{itemize}
\item {Grp. gram.:f.}
\end{itemize}
\begin{itemize}
\item {Proveniência:(Do gr. \textunderscore exo\textunderscore  + \textunderscore oneirosis\textunderscore )}
\end{itemize}
Pollução nocturna.
\section{Exophtalmia}
\begin{itemize}
\item {Grp. gram.:f.}
\end{itemize}
O mesmo que \textunderscore exophthalmo\textunderscore .
\section{Exophthálmico}
\begin{itemize}
\item {Grp. gram.:adj.}
\end{itemize}
Relativo ao exophthalmo.
Que se manifesta com o exophthalmo, como em geral o bócio.
\section{Exophthalmo}
\begin{itemize}
\item {Grp. gram.:m.}
\end{itemize}
\begin{itemize}
\item {Proveniência:(Do gr. \textunderscore exo\textunderscore  + \textunderscore phthalmos\textunderscore )}
\end{itemize}
Saliência exaggerada do globo ocular.
\section{Exorar}
\begin{itemize}
\item {Grp. gram.:v. t.}
\end{itemize}
\begin{itemize}
\item {Proveniência:(Lat. \textunderscore exorare\textunderscore )}
\end{itemize}
Pedir com vehemência.
Supplicar instantemente.
Invocar ansiosamente.
\section{Exorável}
\begin{itemize}
\item {Grp. gram.:adj.}
\end{itemize}
\begin{itemize}
\item {Proveniência:(Lat. \textunderscore exorabilis\textunderscore )}
\end{itemize}
Que póde sêr exorado.
Que cede ás súpplicas; compassivo.
\section{Exorbitância}
\begin{itemize}
\item {Grp. gram.:f.}
\end{itemize}
Qualidade daquillo que é exorbitante.
Preço excessivo.
\section{Exorbitante}
\begin{itemize}
\item {Grp. gram.:adj.}
\end{itemize}
\begin{itemize}
\item {Proveniência:(Lat. \textunderscore exorbitans\textunderscore )}
\end{itemize}
Que sai da órbita.
Que excede os justos limites.
Excessivo.
Copioso.
\section{Exorbitantemente}
\begin{itemize}
\item {Grp. gram.:adv.}
\end{itemize}
De modo exorbitante.
\section{Exorbitar}
\begin{itemize}
\item {Grp. gram.:v. i.}
\end{itemize}
\begin{itemize}
\item {Grp. gram.:V. t.}
\end{itemize}
\begin{itemize}
\item {Proveniência:(Lat. \textunderscore exorbitare\textunderscore )}
\end{itemize}
Sair da órbita.
Exceder-se.
Passar além dos justos limites.
Abundar.
Tirar da órbita ou das órbitas:«\textunderscore ...e fazia gesticulações, exorbitando os olhos...\textunderscore »Camillo, \textunderscore Volcões\textunderscore , 56.
\section{Exorca}
\begin{itemize}
\item {fónica:xór}
\end{itemize}
\begin{itemize}
\item {Grp. gram.:f.}
\end{itemize}
O mesmo que \textunderscore axorca\textunderscore . Cf. Filinto, \textunderscore D. Man.\textunderscore , III, 250.
\section{Exorcismar}
\begin{itemize}
\item {Grp. gram.:v. t.}
\end{itemize}
Pronunciar exorcismos, para expulsar demónios ou espíritos do corpo de.
Esconjurar.
\section{Exorcismo}
\begin{itemize}
\item {Grp. gram.:m.}
\end{itemize}
\begin{itemize}
\item {Proveniência:(Lat. \textunderscore exorcismus\textunderscore )}
\end{itemize}
Oração ou ceremónia religiosa, para livrar de espíritos maus ou de coisas nocivas.
Esconjuro.
\section{Exorcista}
\begin{itemize}
\item {Grp. gram.:m.}
\end{itemize}
\begin{itemize}
\item {Proveniência:(Lat. \textunderscore exorcista\textunderscore )}
\end{itemize}
Aquelle que exorcisma.
\section{Exorcistado}
\begin{itemize}
\item {Grp. gram.:m.}
\end{itemize}
\begin{itemize}
\item {Proveniência:(De \textunderscore exorcista\textunderscore )}
\end{itemize}
Em Theologia, uma das quatro Ordens menores, que tem por matéria o livro dos exorcismos.
\section{Exorcizar}
\begin{itemize}
\item {Grp. gram.:v. t.}
\end{itemize}
\begin{itemize}
\item {Proveniência:(Lat. \textunderscore exorcizare\textunderscore )}
\end{itemize}
O mesmo que \textunderscore exorcismar\textunderscore .
Esconjurar, afastar:«\textunderscore ...exorcizar as tentações voluptuosas da sua alma e do seu corpo.\textunderscore »Camillo.
\section{Exordial}
\begin{itemize}
\item {Grp. gram.:adj.}
\end{itemize}
Relativo a exórdio: \textunderscore palavras exordiaes\textunderscore .
\section{Exordiar}
\begin{itemize}
\item {Grp. gram.:v. t.}
\end{itemize}
\begin{itemize}
\item {Grp. gram.:V. i.}
\end{itemize}
Fazer o exórdio de.
Principiar: \textunderscore exordiar um discurso\textunderscore .
Começar a falar.
\section{Exórdio}
\begin{itemize}
\item {Grp. gram.:m.}
\end{itemize}
\begin{itemize}
\item {Utilização:Fig.}
\end{itemize}
\begin{itemize}
\item {Proveniência:(Lat. \textunderscore exordium\textunderscore )}
\end{itemize}
Primeira parte de um discurso.
Introducção a um discurso.
Princípio, origem: \textunderscore no exórdio dos tempos\textunderscore .
\section{Exornação}
\begin{itemize}
\item {Grp. gram.:f.}
\end{itemize}
\begin{itemize}
\item {Proveniência:(Lat. \textunderscore exornatio\textunderscore )}
\end{itemize}
Acto ou effeito de exornar.
\section{Exornar}
\begin{itemize}
\item {Grp. gram.:v. t.}
\end{itemize}
\begin{itemize}
\item {Proveniência:(Lat. \textunderscore exornare\textunderscore )}
\end{itemize}
Ornar muito.
Enfeitar; ataviar, engalanar.
\section{Exornativo}
\begin{itemize}
\item {Grp. gram.:adj.}
\end{itemize}
Que exorna ou que serve para exornar.
\section{Exorrhizo}
\begin{itemize}
\item {Grp. gram.:adj.}
\end{itemize}
\begin{itemize}
\item {Utilização:Bot.}
\end{itemize}
\begin{itemize}
\item {Proveniência:(Do gr. \textunderscore exo\textunderscore  + \textunderscore rhiza\textunderscore , raíz)}
\end{itemize}
Diz-se das plantas, cujas radículas se alongam no eixo do embryão, na época da germinação.
\section{Exorrizo}
\begin{itemize}
\item {Grp. gram.:adj.}
\end{itemize}
\begin{itemize}
\item {Utilização:Bot.}
\end{itemize}
\begin{itemize}
\item {Proveniência:(Do gr. \textunderscore exo\textunderscore  + \textunderscore rhiza\textunderscore , raíz)}
\end{itemize}
Diz-se das plantas, cujas radículas se alongam no eixo do embrião, na época da germinação.
\section{Exortação}
\begin{itemize}
\item {Grp. gram.:f.}
\end{itemize}
\begin{itemize}
\item {Proveniência:(Lat. \textunderscore exhortatio\textunderscore )}
\end{itemize}
Acto de exortar.
Conselho.
Admoestação, advertência.
Palavras, com que se procura reformar ou melhorar os actos, costumes ou opiniões de alguém.
\section{Exortador}
\begin{itemize}
\item {Grp. gram.:m.}
\end{itemize}
\begin{itemize}
\item {Proveniência:(Lat. \textunderscore exhortator\textunderscore )}
\end{itemize}
Aquele que exorta.
\section{Exortar}
\begin{itemize}
\item {Grp. gram.:v. t.}
\end{itemize}
\begin{itemize}
\item {Proveniência:(Lat. \textunderscore exhortari\textunderscore )}
\end{itemize}
Excitar á prática de alguma coisa.
Procurar convencer, por meio de palavras.
Estimular.
Advertir; aconselhar.
\section{Exortativo}
\begin{itemize}
\item {Grp. gram.:adj.}
\end{itemize}
\begin{itemize}
\item {Proveniência:(Lat. \textunderscore exhortativus\textunderscore )}
\end{itemize}
Que exorta.
Que é próprio para exortar: \textunderscore palavras exortativas\textunderscore .
\section{Exortatória}
\begin{itemize}
\item {Grp. gram.:f.}
\end{itemize}
Exortação.
Discurso, em que se exorta ou aconselha. Cf. Filinto, \textunderscore D. Man.\textunderscore  I, 344; Arn. Gama, \textunderscore Motim\textunderscore , 58.
\section{Exortatório}
\begin{itemize}
\item {Grp. gram.:adj.}
\end{itemize}
\begin{itemize}
\item {Proveniência:(Lat. \textunderscore exhortatorius\textunderscore )}
\end{itemize}
Que envolve exortação: \textunderscore uma carta exortatória\textunderscore .
\section{Exosmose}
\begin{itemize}
\item {Grp. gram.:f.}
\end{itemize}
\begin{itemize}
\item {Utilização:Phýs.}
\end{itemize}
\begin{itemize}
\item {Proveniência:(Do gr. \textunderscore exo\textunderscore  + \textunderscore osmos\textunderscore )}
\end{itemize}
Corrente, que se estabelece ao mesmo tempo que a endosmose, mas em sentido opposto, através de uma membrana que separa dois líquidos de densidade differente.
\section{Exosmótico}
\begin{itemize}
\item {Grp. gram.:adj.}
\end{itemize}
Relativo a exosmose.
\section{Exostema}
\begin{itemize}
\item {Grp. gram.:m.}
\end{itemize}
\begin{itemize}
\item {Proveniência:(Do gr. \textunderscore exo\textunderscore  + \textunderscore stemma\textunderscore )}
\end{itemize}
Gênero de arbustos americanos.
\section{Exostemma}
\begin{itemize}
\item {Grp. gram.:m.}
\end{itemize}
\begin{itemize}
\item {Proveniência:(Do gr. \textunderscore exo\textunderscore  + \textunderscore stemma\textunderscore )}
\end{itemize}
Gênero de arbustos americanos.
\section{Exóstoma}
\begin{itemize}
\item {Grp. gram.:m.}
\end{itemize}
\begin{itemize}
\item {Utilização:Bot.}
\end{itemize}
\begin{itemize}
\item {Proveniência:(Do gr. \textunderscore exo\textunderscore  + \textunderscore stoma\textunderscore )}
\end{itemize}
Orifício da túnica do óvulo vegetal.
\section{Exostose}
\begin{itemize}
\item {Grp. gram.:f.}
\end{itemize}
\begin{itemize}
\item {Proveniência:(Gr. \textunderscore exostosis\textunderscore )}
\end{itemize}
Tumor, na superfície de um osso.
Excrescência lenhosa, no tronco de algumas árvores.
\section{Exostra}
\begin{itemize}
\item {Grp. gram.:f.}
\end{itemize}
\begin{itemize}
\item {Proveniência:(Lat. \textunderscore exostra\textunderscore )}
\end{itemize}
Ponte corrediça, que se estendia das torres móveis até ás muralhas, e pela qual passavam os soldados que iam combater uma praça.
\section{Exoteca}
\begin{itemize}
\item {Grp. gram.:f.}
\end{itemize}
\begin{itemize}
\item {Utilização:Bot.}
\end{itemize}
\begin{itemize}
\item {Proveniência:(Do gr. \textunderscore exo\textunderscore  + \textunderscore theke\textunderscore )}
\end{itemize}
Membrana exterior dos septos da antera.
\section{Exotérico}
\begin{itemize}
\item {Grp. gram.:adj.}
\end{itemize}
\begin{itemize}
\item {Proveniência:(Gr. \textunderscore exoterikos\textunderscore )}
\end{itemize}
Que se expõe em público, (tratando-se de antigas doutrinas philosóphicas).
Exterior.
Vulgar; trivial: \textunderscore rimas exotéricas\textunderscore .
\section{Exoterismo}
\begin{itemize}
\item {Grp. gram.:m.}
\end{itemize}
Qualidade de exotérico.
\section{Exotheca}
\begin{itemize}
\item {Grp. gram.:f.}
\end{itemize}
\begin{itemize}
\item {Utilização:Bot.}
\end{itemize}
\begin{itemize}
\item {Proveniência:(Do gr. \textunderscore exo\textunderscore  + \textunderscore theke\textunderscore )}
\end{itemize}
Membrana exterior dos septos da anthera.
\section{Exothyreopexia}
\begin{itemize}
\item {fónica:csi}
\end{itemize}
\begin{itemize}
\item {Grp. gram.:f.}
\end{itemize}
\begin{itemize}
\item {Proveniência:(Do gr. \textunderscore exo\textunderscore  + \textunderscore thureoeides\textunderscore )}
\end{itemize}
Operação cirúrgica, que consiste em fixar o corpo thyreoide fóra da incisão.
\section{Exoticamente}
\begin{itemize}
\item {Grp. gram.:adv.}
\end{itemize}
De modo exótico.
Excentricamente.
\section{Exoticidade}
\begin{itemize}
\item {Grp. gram.:f.}
\end{itemize}
Qualidade de exótico.
\section{Exoticismo}
\begin{itemize}
\item {Grp. gram.:m.}
\end{itemize}
O mesmo que \textunderscore exoticidade\textunderscore .
Coisa exótica.
Estrangeirismo.
\section{Exótico}
\begin{itemize}
\item {Grp. gram.:adj.}
\end{itemize}
\begin{itemize}
\item {Utilização:Fam.}
\end{itemize}
\begin{itemize}
\item {Proveniência:(Gr. \textunderscore exotikos\textunderscore )}
\end{itemize}
Que procede de um país estranho: \textunderscore plantas exóticas\textunderscore .
Que não é indígena: \textunderscore costumes exóticos\textunderscore .
Esquisito; desajeitado.
\section{Exotireopexia}
\begin{itemize}
\item {fónica:csi}
\end{itemize}
\begin{itemize}
\item {Grp. gram.:f.}
\end{itemize}
\begin{itemize}
\item {Proveniência:(Do gr. \textunderscore exo\textunderscore  + \textunderscore thureoeides\textunderscore )}
\end{itemize}
Operação cirúrgica, que consiste em fixar o corpo tireoide fóra da incisão.
\section{Exotismo}
\begin{itemize}
\item {Grp. gram.:m.}
\end{itemize}
O mesmo que \textunderscore exoticidade\textunderscore .
Coisa exótica.
Estrangeirismo.
\section{Exouvir}
\begin{itemize}
\item {Grp. gram.:v.}
\end{itemize}
\begin{itemize}
\item {Utilização:Ant.}
\end{itemize}
\begin{itemize}
\item {Proveniência:(Do lat. \textunderscore exaudire\textunderscore )}
\end{itemize}
Ouvir bem, attender.
\section{Expandidura}
\begin{itemize}
\item {Grp. gram.:f.}
\end{itemize}
\begin{itemize}
\item {Utilização:Ant.}
\end{itemize}
\begin{itemize}
\item {Proveniência:(De \textunderscore expandir\textunderscore )}
\end{itemize}
O mesmo que \textunderscore expansão\textunderscore .
Extensão, espaço.
\section{Expandir}
\begin{itemize}
\item {Grp. gram.:v. t.}
\end{itemize}
\begin{itemize}
\item {Proveniência:(Lat. \textunderscore expandere\textunderscore )}
\end{itemize}
Tornar pando.
Dilatar.
Estender.
Diffundir: \textunderscore expandir doutrinas\textunderscore .
Alargar.
Ampliar.
Desabafar, expôr com franqueza: \textunderscore expandir uma opinião\textunderscore .
\section{Expandudo}
\begin{itemize}
\item {Grp. gram.:adj.}
\end{itemize}
\begin{itemize}
\item {Utilização:Ant.}
\end{itemize}
\begin{itemize}
\item {Proveniência:(De \textunderscore expandir\textunderscore )}
\end{itemize}
Que se expandiu: \textunderscore opiniões expandudas\textunderscore .
\section{Expansão}
\begin{itemize}
\item {Grp. gram.:f.}
\end{itemize}
\begin{itemize}
\item {Proveniência:(Lat. \textunderscore expansio\textunderscore )}
\end{itemize}
Acto ou effeito de expandir.
Prolongamento.
Diffusão espontânea e communicativa de enthusiasmo, de alegria, de amizade, etc.
Manifestação enthusiastica ou impetuosa.
\section{Expansibilidade}
\begin{itemize}
\item {Grp. gram.:f.}
\end{itemize}
Qualidade daquillo que é expansível.
\section{Expansionismo}
\begin{itemize}
\item {Grp. gram.:m.}
\end{itemize}
\begin{itemize}
\item {Utilização:Neol.}
\end{itemize}
Systema ou processo de se expandirem ou dilatarem coisas ou ideias:«\textunderscore a política colonial de expansionismo, como processo...\textunderscore »\textunderscore País\textunderscore , do Rio, do 4-I-901.
\section{Expansionista}
\begin{itemize}
\item {Grp. gram.:m.}
\end{itemize}
Partidário do expansionismo.
\section{Expansível}
\begin{itemize}
\item {Grp. gram.:adj.}
\end{itemize}
\begin{itemize}
\item {Proveniência:(Do lat. \textunderscore expansus\textunderscore )}
\end{itemize}
Que se póde expandir.
\section{Expansivo}
\begin{itemize}
\item {Grp. gram.:adj.}
\end{itemize}
\begin{itemize}
\item {Utilização:Fig.}
\end{itemize}
\begin{itemize}
\item {Proveniência:(Do lat. \textunderscore expansus\textunderscore )}
\end{itemize}
Expansível.
Franco.
Enthusiasta.
Communicativo: \textunderscore a minha Dora é muito expansiva\textunderscore .
\section{Expatriação}
\begin{itemize}
\item {Grp. gram.:f.}
\end{itemize}
Acto de expatriar.
Emigração.
Destêrro.
\section{Expatriado}
\begin{itemize}
\item {Grp. gram.:m.}
\end{itemize}
\begin{itemize}
\item {Proveniência:(De \textunderscore expatriar\textunderscore )}
\end{itemize}
Aquelle que se expatriou, ou que foi condemnado a degrêdo.
\section{Esquiagra}
\begin{itemize}
\item {Grp. gram.:f.}
\end{itemize}
\begin{itemize}
\item {Utilização:Med.}
\end{itemize}
\begin{itemize}
\item {Proveniência:(Do gr. \textunderscore iskhion\textunderscore  + \textunderscore agra\textunderscore )}
\end{itemize}
Dôr fixa nos quadris.
Dôr sciática.
\section{Esquiocele}
\begin{itemize}
\item {Grp. gram.:f.}
\end{itemize}
\begin{itemize}
\item {Utilização:Med.}
\end{itemize}
\begin{itemize}
\item {Proveniência:(Do gr. \textunderscore iskhion\textunderscore  + \textunderscore kele\textunderscore )}
\end{itemize}
Hérnia, produzida através da chanfradura ischiática.
\section{Expatriar}
\begin{itemize}
\item {Grp. gram.:v. t.}
\end{itemize}
\begin{itemize}
\item {Proveniência:(De \textunderscore ex...\textunderscore  + \textunderscore pátria\textunderscore )}
\end{itemize}
Expulsar da pátria.
Exilar; desterrar.
\section{Expectação}
\begin{itemize}
\item {Grp. gram.:f.}
\end{itemize}
\begin{itemize}
\item {Proveniência:(Lat. \textunderscore expectatio\textunderscore )}
\end{itemize}
O mesmo que \textunderscore expectativa\textunderscore .
\section{Expectador}
\begin{itemize}
\item {Grp. gram.:m.}
\end{itemize}
\begin{itemize}
\item {Proveniência:(Lat. \textunderscore expectator\textunderscore )}
\end{itemize}
Aquelle que tem a expectativa.
\section{Expectante}
\begin{itemize}
\item {Grp. gram.:adj.}
\end{itemize}
\begin{itemize}
\item {Proveniência:(Lat. \textunderscore expectans\textunderscore )}
\end{itemize}
Que espera, observando.
\section{Expectantismo}
\begin{itemize}
\item {Grp. gram.:m.}
\end{itemize}
\begin{itemize}
\item {Proveniência:(De \textunderscore expectante\textunderscore )}
\end{itemize}
Medicina expectante.
\section{Expectar}
\begin{itemize}
\item {Grp. gram.:v. i.}
\end{itemize}
Estar na expectativa. Cf. Júl. Dinis, \textunderscore Serões\textunderscore , 7.
\section{Expectativa}
\begin{itemize}
\item {Grp. gram.:f.}
\end{itemize}
\begin{itemize}
\item {Proveniência:(Do lat. \textunderscore expectatus\textunderscore )}
\end{itemize}
Esperança, baseada em suppostos direitos, probabilidades ou promessas.
Esperança; probabilidade.
\section{Expectatório}
\begin{itemize}
\item {Grp. gram.:adj.}
\end{itemize}
\begin{itemize}
\item {Utilização:Ant.}
\end{itemize}
\begin{itemize}
\item {Proveniência:(Do lat. \textunderscore expectatus\textunderscore )}
\end{itemize}
Que prepara ou antecede um acto solemne.
\section{Expectável}
\begin{itemize}
\item {Grp. gram.:adj.}
\end{itemize}
\begin{itemize}
\item {Proveniência:(Lat. \textunderscore expectabilis\textunderscore )}
\end{itemize}
Que se póde esperar.
Provável.
\section{Expectoração}
\begin{itemize}
\item {Grp. gram.:f.}
\end{itemize}
\begin{itemize}
\item {Proveniência:(Lat. \textunderscore expectoratio\textunderscore )}
\end{itemize}
Acto de expectorar.
\section{Expectorante}
\begin{itemize}
\item {Grp. gram.:adj.}
\end{itemize}
\begin{itemize}
\item {Grp. gram.:M.}
\end{itemize}
\begin{itemize}
\item {Proveniência:(Lat. \textunderscore expectorans\textunderscore )}
\end{itemize}
Que facilita a expectoração: \textunderscore xarope expectorante\textunderscore .
Aquillo que promove ou facilita a expectoração.
\section{Expectorar}
\begin{itemize}
\item {Grp. gram.:v. t.}
\end{itemize}
\begin{itemize}
\item {Utilização:Fig.}
\end{itemize}
\begin{itemize}
\item {Proveniência:(Lat. \textunderscore expectorare\textunderscore )}
\end{itemize}
Expellir do peito; escarrar.
Pronunciar insensatamente, com violência: \textunderscore expectorar inconveniências\textunderscore .
\section{Expedição}
\begin{itemize}
\item {Grp. gram.:f.}
\end{itemize}
\begin{itemize}
\item {Proveniência:(Lat. \textunderscore expeditio\textunderscore )}
\end{itemize}
Acto ou effeito de expedir: \textunderscore expedição de uma encommenda\textunderscore .
Remessa de tropas, com determinado fim.
Excursão scientífica.
Desembaraço; expediente.
\section{Expedicionário}
\begin{itemize}
\item {Grp. gram.:adj.}
\end{itemize}
\begin{itemize}
\item {Grp. gram.:M.}
\end{itemize}
\begin{itemize}
\item {Proveniência:(Do lat. \textunderscore expeditio\textunderscore )}
\end{itemize}
Relativo a uma expedição.
Que faz expedição.
Aquelle que faz parte de uma expedição.
\section{Expedicioneiro}
\begin{itemize}
\item {Grp. gram.:m.}
\end{itemize}
\begin{itemize}
\item {Proveniência:(Do lat. \textunderscore expeditio\textunderscore )}
\end{itemize}
Funcionário da Cúria pontifícia, que solicita expedição de breves, bullas, etc.
\section{Expedida}
\begin{itemize}
\item {Grp. gram.:f.}
\end{itemize}
\begin{itemize}
\item {Utilização:Ant.}
\end{itemize}
\begin{itemize}
\item {Proveniência:(De \textunderscore expedir\textunderscore )}
\end{itemize}
Licença para sair ou expedir.
Despedida.
\section{Expedidamente}
\begin{itemize}
\item {Grp. gram.:adv.}
\end{itemize}
(V.expeditamente)
\section{Expedidor}
\begin{itemize}
\item {Grp. gram.:adj.}
\end{itemize}
\begin{itemize}
\item {Grp. gram.:M.}
\end{itemize}
\begin{itemize}
\item {Proveniência:(De \textunderscore expedir\textunderscore )}
\end{itemize}
Que expede.
Aquelle que expede.
Empregado de Companhia ou de Empresa de carros americanos, para fiscalizar a saída dos carros e tomar conta de quaesquer incidentes no respectivo serviço.
\section{Expediência}
\begin{itemize}
\item {Grp. gram.:f.}
\end{itemize}
\begin{itemize}
\item {Utilização:Des.}
\end{itemize}
\begin{itemize}
\item {Proveniência:(De \textunderscore expedir\textunderscore )}
\end{itemize}
Expedição.
Desembaraço, actividade.
\section{Expediente}
\begin{itemize}
\item {Grp. gram.:adj.}
\end{itemize}
\begin{itemize}
\item {Grp. gram.:M.}
\end{itemize}
\begin{itemize}
\item {Proveniência:(Lat. \textunderscore expediens\textunderscore )}
\end{itemize}
Que expede.
Expedito.
Emprêgo de meios para sair de uma difficuldade ou chegar a uma solução: \textunderscore viver de expedientes\textunderscore .
Despacho ou andamento de negócios pendentes.
Negócios pendentes de uma Repartição pública: \textunderscore demoram ali muito o expediente\textunderscore .
\section{Expedimento}
\begin{itemize}
\item {Grp. gram.:m.}
\end{itemize}
Acto de \textunderscore expedir\textunderscore .
\section{Expedir}
\begin{itemize}
\item {Grp. gram.:v. t.}
\end{itemize}
\begin{itemize}
\item {Utilização:Ant.}
\end{itemize}
\begin{itemize}
\item {Proveniência:(Lat. \textunderscore expedire\textunderscore )}
\end{itemize}
Tornar desembaraçado para poder partir.
Enviar: \textunderscore expedir mercadorias\textunderscore .
Fazer partir com determinado fim: \textunderscore expedir um emissário\textunderscore .
Resolver, despachar.
Promover activamente a solução de.
Expulsar.
Promulgar: \textunderscore expedir decretos\textunderscore .
Despedir; afastar.
\section{Expeditamente}
\begin{itemize}
\item {Grp. gram.:adv.}
\end{itemize}
De modo expedito.
\section{Expeditivo}
\begin{itemize}
\item {Grp. gram.:adj.}
\end{itemize}
\begin{itemize}
\item {Proveniência:(De \textunderscore expedito\textunderscore )}
\end{itemize}
Expedito.
\section{Expedito}
\begin{itemize}
\item {Grp. gram.:adj.}
\end{itemize}
\begin{itemize}
\item {Proveniência:(Lat. \textunderscore expeditus\textunderscore )}
\end{itemize}
Activo; diligente.
Fácil.
\section{Expeditório}
\begin{itemize}
\item {Grp. gram.:adj.}
\end{itemize}
\begin{itemize}
\item {Utilização:Neol.}
\end{itemize}
Próprio para se expedir:«\textunderscore ...mandados expeditórios\textunderscore ». \textunderscore Jornal do Comm.\textunderscore , do Rio, de 19-VI-901.
\section{Expelente}
\begin{itemize}
\item {Grp. gram.:adj.}
\end{itemize}
Que expele. Cf. Th. Ribeiro, \textunderscore Jornadas\textunderscore , I, 19.
\section{Expelir}
\begin{itemize}
\item {Grp. gram.:v. t.}
\end{itemize}
\begin{itemize}
\item {Utilização:Fig.}
\end{itemize}
\begin{itemize}
\item {Proveniência:(Lat. \textunderscore expelere\textunderscore )}
\end{itemize}
Lançar fóra.
Expulsar; atirar com ímpeto.
Expectorar: \textunderscore expelir bílis\textunderscore .
Pronunciar com violencia.
\section{Expellente}
\begin{itemize}
\item {Grp. gram.:adj.}
\end{itemize}
Que expelle. Cf. Th. Ribeiro, \textunderscore Jornadas\textunderscore , I, 19.
\section{Expellir}
\begin{itemize}
\item {Grp. gram.:v. t.}
\end{itemize}
\begin{itemize}
\item {Utilização:Fig.}
\end{itemize}
\begin{itemize}
\item {Proveniência:(Lat. \textunderscore expelere\textunderscore )}
\end{itemize}
Lançar fóra.
Expulsar; atirar com ímpeto.
Expectorar: \textunderscore expellir bílis\textunderscore .
Pronunciar com violencia.
\section{Expender}
\begin{itemize}
\item {Grp. gram.:v. t.}
\end{itemize}
\begin{itemize}
\item {Utilização:Des.}
\end{itemize}
\begin{itemize}
\item {Proveniência:(Lat. \textunderscore expendere\textunderscore )}
\end{itemize}
Expor minuciosamente: \textunderscore expender uma opínião\textunderscore .
Ponderar.
Despender.
\section{Expensas}
\begin{itemize}
\item {Grp. gram.:f. pl.}
\end{itemize}
\begin{itemize}
\item {Proveniência:(Lat. \textunderscore expensa\textunderscore )}
\end{itemize}
Despesas.
Custo; custa: \textunderscore armou um batalhão a expensas suas\textunderscore .
\section{Expensão}
\begin{itemize}
\item {Grp. gram.:f.}
\end{itemize}
\begin{itemize}
\item {Utilização:Des.}
\end{itemize}
\begin{itemize}
\item {Proveniência:(Lat. \textunderscore expensio\textunderscore )}
\end{itemize}
Acto de expender.
\section{Experiência}
\begin{itemize}
\item {Grp. gram.:f.}
\end{itemize}
\begin{itemize}
\item {Proveniência:(Lat. \textunderscore experientia\textunderscore )}
\end{itemize}
Acto ou effeito de experimentar.
Tentativa.
Experimento.
Ensaio prático, para descobrir ou determinar um phenómeno, um facto ou uma theoria.
Conhecimento das coisas, pela prática e pela observação: \textunderscore pessôa de experiência\textunderscore .
\section{Experiente}
\begin{itemize}
\item {Grp. gram.:m.  e  adj.}
\end{itemize}
\begin{itemize}
\item {Proveniência:(Lat. \textunderscore experiens\textunderscore )}
\end{itemize}
O que tem experiência; o que revela experiência.
\section{Experimenta}
\begin{itemize}
\item {Grp. gram.:f.}
\end{itemize}
(V.experimento)
\section{Experimentação}
\begin{itemize}
\item {Grp. gram.:f.}
\end{itemize}
Acto de experimentar.
\section{Experimentador}
\begin{itemize}
\item {Grp. gram.:m.  e  adj.}
\end{itemize}
O que experimenta.
\section{Experimental}
\begin{itemize}
\item {Grp. gram.:adj.}
\end{itemize}
\begin{itemize}
\item {Proveniência:(De \textunderscore experimento\textunderscore )}
\end{itemize}
Baseado na experiência: \textunderscore processo experimental\textunderscore .
Relativo a experiência.
\section{Experimentalmente}
\begin{itemize}
\item {Grp. gram.:adv.}
\end{itemize}
De modo experimental.
\section{Experimentar}
\begin{itemize}
\item {Grp. gram.:v. t.}
\end{itemize}
\begin{itemize}
\item {Proveniência:(De \textunderscore experimento\textunderscore )}
\end{itemize}
Tentar.
Sujeitar a provas: \textunderscore experimentar a fidelidade de um servo\textunderscore .
Ensaiar: \textunderscore experimentar uma espingarda\textunderscore .
Analysar praticamente.
Conhecer por observação própria.
Observar as condições de.
Praticar.
Sentir: \textunderscore experimentar dissabores\textunderscore .
Conseguir.
\section{Experimentável}
\begin{itemize}
\item {Grp. gram.:adj.}
\end{itemize}
Que se póde experimentar.
\section{Experimentavelmente}
\begin{itemize}
\item {Grp. gram.:adv.}
\end{itemize}
\begin{itemize}
\item {Proveniência:(De \textunderscore experimentável\textunderscore )}
\end{itemize}
O mesmo que \textunderscore experimentalmente\textunderscore .
\section{Experimento}
\begin{itemize}
\item {Grp. gram.:m.}
\end{itemize}
\begin{itemize}
\item {Proveniência:(Lat. \textunderscore experimentum\textunderscore )}
\end{itemize}
Experiência.
Ensaio scientífico, para a verificação de um phenómeno phýsico.
\section{Expertar}
\textunderscore v. t.\textunderscore , \textunderscore i.\textunderscore  e \textunderscore p.\textunderscore  (e der.)
(V. \textunderscore espertar\textunderscore , etc.)
\section{Expertinar}
\textunderscore v. t.\textunderscore  (e der.)
(V. \textunderscore espertinar\textunderscore , etc.)
\section{Experto}
\begin{itemize}
\item {Grp. gram.:m.  e  adj.}
\end{itemize}
\begin{itemize}
\item {Proveniência:(Lat. \textunderscore expertus\textunderscore . Cp. \textunderscore esperto\textunderscore )}
\end{itemize}
Indivíduo experimentado.
Sabedor; perito.
\section{Expiação}
\begin{itemize}
\item {Grp. gram.:f.}
\end{itemize}
\begin{itemize}
\item {Utilização:Des.}
\end{itemize}
\begin{itemize}
\item {Proveniência:(Lat. \textunderscore expiatio\textunderscore )}
\end{itemize}
Acto ou effeito de expiar.
Soffrimento de pena ou castigo, imposto a delínquente.
Penitência ou ceremónias, para abrandar a cólera divina.
\section{Expiando}
\begin{itemize}
\item {Grp. gram.:adj.}
\end{itemize}
Que envolve expiação:«\textunderscore ...desta expianda angústia.\textunderscore »Filinto.
\section{Expiar}
\begin{itemize}
\item {Grp. gram.:v. t.}
\end{itemize}
\begin{itemize}
\item {Proveniência:(Lat. \textunderscore expiare\textunderscore )}
\end{itemize}
Purificar ou rehabilitar, por meio de castigo.
Reparar (crimes ou faltas), por penitência ou pena que se cumpre.
Soffrer as consequências de.
\section{Expiatoriamente}
\begin{itemize}
\item {Grp. gram.:adv.}
\end{itemize}
De modo expiatório.
\section{Expiatório}
\begin{itemize}
\item {Grp. gram.:adj.}
\end{itemize}
\begin{itemize}
\item {Proveniência:(Lat. \textunderscore expiatorius\textunderscore )}
\end{itemize}
Próprio para expiação.
\section{Expiável}
\begin{itemize}
\item {Grp. gram.:adj.}
\end{itemize}
\begin{itemize}
\item {Proveniência:(Lat. \textunderscore expiabilis\textunderscore )}
\end{itemize}
Que se póde expiar.
\section{Expilação}
\begin{itemize}
\item {Grp. gram.:f.}
\end{itemize}
\begin{itemize}
\item {Proveniência:(Lat. \textunderscore expilatio\textunderscore )}
\end{itemize}
Acto de expilar.
\section{Expilar}
\begin{itemize}
\item {Grp. gram.:v. t.}
\end{itemize}
\begin{itemize}
\item {Proveniência:(Do lat. \textunderscore expilare\textunderscore )}
\end{itemize}
Roubar.
Subtrahir (bens de herança), antes de conhecido ou declarado o herdeiro.
Espoliar.
\section{Expiração}
\begin{itemize}
\item {Grp. gram.:f.}
\end{itemize}
\begin{itemize}
\item {Proveniência:(Lat. \textunderscore expiratio\textunderscore )}
\end{itemize}
Acto de expirar.
\section{Expirador}
\begin{itemize}
\item {Grp. gram.:adj.}
\end{itemize}
\begin{itemize}
\item {Proveniência:(Lat. \textunderscore expirans\textunderscore )}
\end{itemize}
Que expira.
Moribundo.
\section{Expirante}
\begin{itemize}
\item {Grp. gram.:adj.}
\end{itemize}
\begin{itemize}
\item {Proveniência:(Lat. \textunderscore expirans\textunderscore )}
\end{itemize}
Que expira.
Moribundo.
\section{Expirar}
\begin{itemize}
\item {Grp. gram.:v. t.}
\end{itemize}
\begin{itemize}
\item {Grp. gram.:V. i.}
\end{itemize}
\begin{itemize}
\item {Proveniência:(Lat. \textunderscore expirare\textunderscore )}
\end{itemize}
Espirar; respirar.
Deixar saír o espirito, a alma; morrer.
Extinguir-se.
Finalizar: \textunderscore expirou o prazo da moratória\textunderscore .
Acabar a pouco e pouco.
Exhalar-se.
\section{Explainada}
\begin{itemize}
\item {Grp. gram.:f.}
\end{itemize}
\begin{itemize}
\item {Proveniência:(De \textunderscore plaino\textunderscore )}
\end{itemize}
(V.explanada)
\section{Explanação}
\begin{itemize}
\item {Grp. gram.:f.}
\end{itemize}
\begin{itemize}
\item {Proveniência:(Lat. \textunderscore explanatio\textunderscore )}
\end{itemize}
Acto de explanar.
\section{Explanada}
\begin{itemize}
\item {Grp. gram.:f.}
\end{itemize}
\begin{itemize}
\item {Proveniência:(De \textunderscore explanar\textunderscore )}
\end{itemize}
Planície.
Planalto.
Chapada.
Campo largo e descoberto.
\section{Explanador}
\begin{itemize}
\item {Grp. gram.:m. e adj.}
\end{itemize}
\begin{itemize}
\item {Proveniência:(Lat. \textunderscore explanator\textunderscore )}
\end{itemize}
O que explana.
\section{Explanar}
\begin{itemize}
\item {Grp. gram.:v. t.}
\end{itemize}
\begin{itemize}
\item {Proveniência:(Lat. \textunderscore explanare\textunderscore )}
\end{itemize}
Tornar plano, claro, fácil: \textunderscore explanar um assumpto\textunderscore .
Explicar; Narrar minuciosamente: \textunderscore explanar episódios\textunderscore .
Fazer exposição verbal de.
\section{Explanatório}
\begin{itemize}
\item {Grp. gram.:adj.}
\end{itemize}
\begin{itemize}
\item {Proveniência:(Lat. \textunderscore explanatorius\textunderscore )}
\end{itemize}
Que serve para explanar.
\section{Expletivamente}
\begin{itemize}
\item {Grp. gram.:adj.}
\end{itemize}
De modo expletivo.
\section{Expletiva}
\begin{itemize}
\item {Grp. gram.:f.}
\end{itemize}
\begin{itemize}
\item {Proveniência:(De \textunderscore expletivo\textunderscore )}
\end{itemize}
Qualquer parte do discurso, usada para simples effeito decorativo da phrase.
Palavra, ou parte de palavra, que é desnecessária ao sentido de outra palavra ou de uma phrase, mas que se junta, para complemento ou por euphonia.
\section{Expletivo}
\begin{itemize}
\item {Grp. gram.:adj.}
\end{itemize}
\begin{itemize}
\item {Grp. gram.:M.}
\end{itemize}
\begin{itemize}
\item {Proveniência:(Lat. \textunderscore explectivus\textunderscore )}
\end{itemize}
Que serve para preencher ou para completar.
Diz-se das palavras ou partículas que, não sendo necessárias ao sentido de uma phrase ou palavra, lhe dão mais fôrça ou graça.
O mesmo que \textunderscore epenthético\textunderscore .
O mesmo que \textunderscore expletiva\textunderscore .
\section{Explicação}
\begin{itemize}
\item {Grp. gram.:f.}
\end{itemize}
\begin{itemize}
\item {Proveniência:(Lat. \textunderscore explicatio\textunderscore )}
\end{itemize}
Acto de explicar.
Palavras, com que se explica.
\section{Explicadamente}
\begin{itemize}
\item {Grp. gram.:adv.}
\end{itemize}
\begin{itemize}
\item {Proveniência:(De \textunderscore explicar\textunderscore )}
\end{itemize}
Minuciosamente.
\section{Explicador}
\begin{itemize}
\item {Grp. gram.:adj.}
\end{itemize}
\begin{itemize}
\item {Grp. gram.:M.}
\end{itemize}
\begin{itemize}
\item {Proveniência:(Lat. \textunderscore explicator\textunderscore )}
\end{itemize}
Que explica.
Aquelle que explica.
\section{Explicar}
\begin{itemize}
\item {Grp. gram.:v. t.}
\end{itemize}
\begin{itemize}
\item {Proveniência:(Lat. \textunderscore explicare\textunderscore )}
\end{itemize}
Explanar.
Tornar intelligivel.
Expor.
Desenvolver.
Justificar: \textunderscore explicar a razão dos seus actos\textunderscore .
Interpretar.
Leccionar á cêrca de: \textunderscore explicar Chímica\textunderscore .
Declarar.
Significar, exprimir: \textunderscore explicar o que sente\textunderscore .
\section{Explicativamente}
\begin{itemize}
\item {Grp. gram.:adv.}
\end{itemize}
De modo explicativo.
\section{Explicativo}
\begin{itemize}
\item {Grp. gram.:adj.}
\end{itemize}
Que serve para explicar.
\section{Explicável}
\begin{itemize}
\item {Grp. gram.:adj.}
\end{itemize}
\begin{itemize}
\item {Proveniência:(Lat. \textunderscore explicabilis\textunderscore )}
\end{itemize}
Que se póde explicar.
\section{Explicitamente}
\begin{itemize}
\item {Grp. gram.:adv.}
\end{itemize}
De modo explícito.
\section{Explícito}
\begin{itemize}
\item {Grp. gram.:adj.}
\end{itemize}
\begin{itemize}
\item {Proveniência:(Lat. \textunderscore explicitus\textunderscore )}
\end{itemize}
Explicado, claro: \textunderscore confissão explícita\textunderscore .
Terminante: \textunderscore resolução explícita\textunderscore .
Preciso; desenvolvido.
\section{Explodidor}
\begin{itemize}
\item {Grp. gram.:adj.}
\end{itemize}
\begin{itemize}
\item {Grp. gram.:M.}
\end{itemize}
\begin{itemize}
\item {Proveniência:(De \textunderscore explodir\textunderscore )}
\end{itemize}
Que faz explodir.
Máquina, geralmente eléctrica, que provoca a explosão das cargas, em minas, torpedos ou canhões.
\section{Explodir}
\begin{itemize}
\item {Grp. gram.:v. i.}
\end{itemize}
\begin{itemize}
\item {Utilização:Fig.}
\end{itemize}
\begin{itemize}
\item {Proveniência:(Lat. \textunderscore explodere\textunderscore )}
\end{itemize}
Rebentar com estrondo.
Fazer explosão.
Expandir-se ruidosamente.
Vociferar.
\section{Exploração}
\begin{itemize}
\item {Grp. gram.:f.}
\end{itemize}
\begin{itemize}
\item {Proveniência:(Lat. \textunderscore exploratio\textunderscore )}
\end{itemize}
Acto ou effeito de explorar.
Investigação.
Acto de analysar ou pesquisar.
Tentativa ou acto de tirar utilidade de alguma coisa.
Empresa.
Aquillo que se explora.
Abuso da bôa fé, da ignorância ou da especial situação de alguém, para auferir interesse illícito.
\section{Explorador}
\begin{itemize}
\item {Grp. gram.:adj.}
\end{itemize}
\begin{itemize}
\item {Grp. gram.:M.}
\end{itemize}
\begin{itemize}
\item {Proveniência:(Lat. \textunderscore explorator\textunderscore )}
\end{itemize}
Que explora.
Aquelle que explora.
\section{Explorar}
\begin{itemize}
\item {Grp. gram.:v. t.}
\end{itemize}
\begin{itemize}
\item {Utilização:Fig.}
\end{itemize}
\begin{itemize}
\item {Proveniência:(Lat. \textunderscore explorare\textunderscore )}
\end{itemize}
Inquirir.
Pesquisar.
Examinar.
Observar geographicamente ou commercialmente (uma região).
Percorrer, estudando ou procurando.
Estudar, para descobrir.
Emprehender.
Tirar utilidade de: \textunderscore explorar uma indústria\textunderscore .
Cultivar: \textunderscore explorar herdades\textunderscore .
Abusar, com proveito, da bôa fé de.
Desfrutar.
\section{Exploratício}
\begin{itemize}
\item {Grp. gram.:adj.}
\end{itemize}
\begin{itemize}
\item {Utilização:bras}
\end{itemize}
\begin{itemize}
\item {Utilização:Neol.}
\end{itemize}
Relativo a exploração.
\section{Exploratório}
\begin{itemize}
\item {Grp. gram.:adj.}
\end{itemize}
\begin{itemize}
\item {Grp. gram.:m.}
\end{itemize}
\begin{itemize}
\item {Proveniência:(Lat. \textunderscore exploratorius\textunderscore )}
\end{itemize}
Que serve para explorar.
Instrumento, com que se sonda a bexiga.
\section{Explorável}
\begin{itemize}
\item {Grp. gram.:adj.}
\end{itemize}
Que se póde explorar.
\section{Explosão}
\begin{itemize}
\item {Grp. gram.:f.}
\end{itemize}
\begin{itemize}
\item {Proveniência:(Lat. \textunderscore explosio\textunderscore )}
\end{itemize}
Acto de explodir.
\section{Explosir}
\begin{itemize}
\item {Grp. gram.:v. i.}
\end{itemize}
(Fórma incorrecta, em vez de \textunderscore explodir\textunderscore , embora usada por Camillo, \textunderscore Crit. do Cancion.\textunderscore , IX; \textunderscore Corja\textunderscore , 141; \textunderscore Narcót.\textunderscore , II, 149; \textunderscore Volcões\textunderscore , 10.)
\section{Explosível}
\begin{itemize}
\item {Grp. gram.:adj.}
\end{itemize}
\begin{itemize}
\item {Proveniência:(Do lat. \textunderscore explosus\textunderscore )}
\end{itemize}
Que póde explodir.
\section{Explosivo}
\begin{itemize}
\item {Grp. gram.:adj.}
\end{itemize}
\begin{itemize}
\item {Grp. gram.:m.}
\end{itemize}
\begin{itemize}
\item {Proveniência:(Do lat. \textunderscore explosus\textunderscore )}
\end{itemize}
Relativo a explosão.
Que produz explosão.
Explosível: \textunderscore matéria explosiva\textunderscore .
Qualquer substância inflammável, que póde produzir explosão.
\section{Expluir}
\begin{itemize}
\item {Grp. gram.:v. i.}
\end{itemize}
(Fórma incorrecta, embora usada por Camillo. Cf. \textunderscore Narcót.\textunderscore , I, 274. V. \textunderscore explodir\textunderscore )
\section{Expoente}
\begin{itemize}
\item {Grp. gram.:m.}
\end{itemize}
\begin{itemize}
\item {Utilização:Mathem.}
\end{itemize}
\begin{itemize}
\item {Utilização:Gram.}
\end{itemize}
\begin{itemize}
\item {Proveniência:(Lat. \textunderscore exponens\textunderscore )}
\end{itemize}
Aquelle que expõe.
Número, que se colloca á direita e um pouco acima de uma quantidade, para indicar a potência, a que essa quantidade é elevada.
Som ou letra, que caracteriza uma flexão: assim, o \textunderscore a\textunderscore , em português, é expoente do feminino, o \textunderscore s\textunderscore  é expoente do plural. Cf. J. Ribeiro, \textunderscore Diccion. Gram.\textunderscore 
\section{Expolição}
\begin{itemize}
\item {Grp. gram.:f.}
\end{itemize}
\begin{itemize}
\item {Utilização:Fig.}
\end{itemize}
\begin{itemize}
\item {Proveniência:(Lat. \textunderscore expolitio\textunderscore )}
\end{itemize}
Acto de polir, de ornar, de amplificar (um discurso).
\section{Exponencial}
\begin{itemize}
\item {Grp. gram.:adj.}
\end{itemize}
\begin{itemize}
\item {Utilização:Mathem.}
\end{itemize}
\begin{itemize}
\item {Grp. gram.:F.}
\end{itemize}
\begin{itemize}
\item {Proveniência:(De \textunderscore exponente\textunderscore )}
\end{itemize}
Que tem como expoente uma quantidade variável ou desconhecida.
Quantidade exponencial.
\section{Exponente}
\begin{itemize}
\item {Grp. gram.:m.}
\end{itemize}
O mesmo que \textunderscore expoente\textunderscore .
\section{Expor}
\begin{itemize}
\item {Grp. gram.:v. t.}
\end{itemize}
\begin{itemize}
\item {Proveniência:(Lat. \textunderscore exponere\textunderscore )}
\end{itemize}
Pôr adeante, á vista.
Apresentar: \textunderscore expor á venda mercadorias\textunderscore .
Explicar, tornar claro.
Fazer exposição de: \textunderscore expor quadros\textunderscore .
Sujeitar á acção de: \textunderscore expor ao sol\textunderscore .
Sujeitar a perigo, a desgôsto, etc.
Abandonar (uma criança).
Narrar.
\section{Exportação}
\begin{itemize}
\item {Grp. gram.:f.}
\end{itemize}
\begin{itemize}
\item {Proveniência:(Lat. \textunderscore exportatio\textunderscore )}
\end{itemize}
Acto de exportar.
\section{Exportador}
\begin{itemize}
\item {Grp. gram.:adj.}
\end{itemize}
\begin{itemize}
\item {Grp. gram.:M.}
\end{itemize}
\begin{itemize}
\item {Proveniência:(Lat. \textunderscore exportator\textunderscore )}
\end{itemize}
Que exporta.
Aquelle que exporta.
\section{Exportar}
\begin{itemize}
\item {Grp. gram.:v. t.}
\end{itemize}
\begin{itemize}
\item {Proveniência:(Lat. \textunderscore exportare\textunderscore )}
\end{itemize}
Mandar ou transportar para outro país (productos das indústrias ou artes nacionaes).
\section{Exportável}
\begin{itemize}
\item {Grp. gram.:adj.}
\end{itemize}
Que se póde exportar.
\section{Exposição}
\begin{itemize}
\item {Grp. gram.:f.}
\end{itemize}
\begin{itemize}
\item {Proveniência:(Lat. \textunderscore expositio\textunderscore )}
\end{itemize}
Acto de expor.
Conjunto de coisas expostas.
Lugar, onde se expõem coisas á vista.
Narração.
Deducção de razões.
Posição de uma coisa ou de um lugar, relativamente aos pontos cardeaes ou ao ponto donde vem a luz, o vento, etc.
\section{Expositivo}
\begin{itemize}
\item {Grp. gram.:adj.}
\end{itemize}
\begin{itemize}
\item {Proveniência:(Lat. \textunderscore expositivus\textunderscore )}
\end{itemize}
Relativo a exposição, que envolve exposição.
\section{Expositor}
\begin{itemize}
\item {Grp. gram.:m.}
\end{itemize}
\begin{itemize}
\item {Proveniência:(Lat. \textunderscore expositor\textunderscore )}
\end{itemize}
Aquelle que expõe.
Obra, que expõe ou elucida uma doutrina: \textunderscore a«Encyclopedia Británnica»é bom expositor para muitos assumptos\textunderscore .
\section{Exposto}
\begin{itemize}
\item {Grp. gram.:m.}
\end{itemize}
\begin{itemize}
\item {Proveniência:(Lat. \textunderscore expostus\textunderscore )}
\end{itemize}
Indivíduo, que foi abandonado em criança.
Enjeitado.
\section{Expostulação}
\begin{itemize}
\item {Grp. gram.:f.}
\end{itemize}
\begin{itemize}
\item {Proveniência:(Lat. \textunderscore expostulatio\textunderscore )}
\end{itemize}
Súpplica.
Reclamação ou queixa, apresentada àquelle que praticou a offensa.
\section{Expremer}
\textunderscore v. t.\textunderscore  e \textunderscore p.\textunderscore  (e der.)
(V. \textunderscore espremer\textunderscore , etc.)
\section{Expressadamente}
\begin{itemize}
\item {Grp. gram.:adv.}
\end{itemize}
O mesmo que \textunderscore expressamente\textunderscore .
\section{Expressador}
\begin{itemize}
\item {Grp. gram.:adj.}
\end{itemize}
Que expressa. Cf. Filinto, \textunderscore D. Man.\textunderscore , II, 35.
\section{Expressamente}
\begin{itemize}
\item {Grp. gram.:adv.}
\end{itemize}
De modo expresso.
Claramente.
Intencionalmente.
Propositadamente.
\section{Expressão}
\begin{itemize}
\item {Grp. gram.:f.}
\end{itemize}
\begin{itemize}
\item {Proveniência:(Lat. \textunderscore expressio\textunderscore )}
\end{itemize}
Acto ou maneira de exprimir.
Palavra; phrase, locução.
Carácter.
Representação animada de sentimentos.
Significação: \textunderscore o que elle diz é expressão da verdade\textunderscore .
Personificação.
Representação algébrica do valor de uma quantidade.
Acto de espremer.
Suco espremido.
\section{Expressar}
\begin{itemize}
\item {Grp. gram.:v. t.  e  p.}
\end{itemize}
\begin{itemize}
\item {Proveniência:(De \textunderscore expresso\textunderscore )}
\end{itemize}
O mesmo que \textunderscore exprimir\textunderscore .
\section{Expressiva}
\begin{itemize}
\item {Grp. gram.:f.}
\end{itemize}
\begin{itemize}
\item {Utilização:Des.}
\end{itemize}
\begin{itemize}
\item {Proveniência:(De \textunderscore expressivo\textunderscore )}
\end{itemize}
O mesmo que \textunderscore expressão\textunderscore .
Elocução. Cf. Sousa, \textunderscore Vida do Arceb.\textunderscore , III, 34.
\section{Expressivo}
\begin{itemize}
\item {Grp. gram.:adj.}
\end{itemize}
\begin{itemize}
\item {Proveniência:(De \textunderscore expresso\textunderscore )}
\end{itemize}
Que exprime.
Significativo: \textunderscore olhar expressivo\textunderscore .
\section{Expresso}
\begin{itemize}
\item {Grp. gram.:adj.}
\end{itemize}
\begin{itemize}
\item {Grp. gram.:M.}
\end{itemize}
\begin{itemize}
\item {Proveniência:(Lat. \textunderscore expressus\textunderscore )}
\end{itemize}
Explícito.
Concludente.
Enviado directamente.
Combóio, que vai directamente a um ponto, sem parar em todas as estações.
Mensageiro expresso ou que vai directamente a um ponto.
\section{Exprimir}
\begin{itemize}
\item {Grp. gram.:v. t.}
\end{itemize}
\begin{itemize}
\item {Proveniência:(Lat. \textunderscore exprimere\textunderscore )}
\end{itemize}
Manifestar por palavras ou gestos: \textunderscore exprimir ideias\textunderscore .
Manifestar, por meio de um trabalho do arte.
Significar.
Dar a conhecer.
Representar em obra de arte.
\section{Exprimível}
\begin{itemize}
\item {Grp. gram.:adj.}
\end{itemize}
Que se póde exprimir.
\section{Exprobração}
\begin{itemize}
\item {Grp. gram.:f.}
\end{itemize}
\begin{itemize}
\item {Proveniência:(Lat. \textunderscore exprobratio\textunderscore )}
\end{itemize}
Acto de exprobrar.
\section{Exprobrador}
\begin{itemize}
\item {Grp. gram.:adj.}
\end{itemize}
\begin{itemize}
\item {Grp. gram.:M.}
\end{itemize}
\begin{itemize}
\item {Proveniência:(Lat. \textunderscore exprobrator\textunderscore )}
\end{itemize}
Que exprobra.
Aquelle que exprobra.
\section{Exprobrante}
\begin{itemize}
\item {Grp. gram.:m.  e  adj.}
\end{itemize}
\begin{itemize}
\item {Proveniência:(Lat. \textunderscore exprobrans\textunderscore )}
\end{itemize}
Exprobrador.
\section{Exprobrar}
\begin{itemize}
\item {Grp. gram.:v. t.}
\end{itemize}
\begin{itemize}
\item {Proveniência:(Lat. \textunderscore exprobrare\textunderscore )}
\end{itemize}
Fazer censuras a.
Lançar alguma culpa em rosto a.
Inculpar; censurar.
\section{Exprobratório}
\begin{itemize}
\item {Grp. gram.:adj.}
\end{itemize}
\begin{itemize}
\item {Proveniência:(Do lat. \textunderscore exprobratus\textunderscore )}
\end{itemize}
Que envolve exprobração.
\section{Expromissor}
\begin{itemize}
\item {Grp. gram.:m.}
\end{itemize}
\begin{itemize}
\item {Utilização:Jur.}
\end{itemize}
\begin{itemize}
\item {Utilização:ant.}
\end{itemize}
\begin{itemize}
\item {Proveniência:(Lat. \textunderscore expromissor\textunderscore )}
\end{itemize}
O principal pagador.
\section{Expropriação}
\begin{itemize}
\item {Grp. gram.:f.}
\end{itemize}
Acto ou effeito do expropriar.
\section{Expropriador}
\begin{itemize}
\item {Grp. gram.:adj.}
\end{itemize}
\begin{itemize}
\item {Grp. gram.:M.}
\end{itemize}
Que expropria.
Aquelle que expropria.
\section{Expropriar}
\begin{itemize}
\item {Grp. gram.:v. t.}
\end{itemize}
\begin{itemize}
\item {Proveniência:(De \textunderscore ex...\textunderscore  + \textunderscore próprio\textunderscore )}
\end{itemize}
Tirar legalmente a alguém a posse ou a propriedade de.
\section{Exprovado}
\begin{itemize}
\item {Grp. gram.:adj.}
\end{itemize}
\begin{itemize}
\item {Utilização:Ant.}
\end{itemize}
\begin{itemize}
\item {Proveniência:(De \textunderscore ex...\textunderscore  + \textunderscore provado\textunderscore )}
\end{itemize}
Puro.
Legítimo.
Perfeito.
\section{Expugnação}
\begin{itemize}
\item {Grp. gram.:f.}
\end{itemize}
\begin{itemize}
\item {Proveniência:(Lat. \textunderscore expugnatio\textunderscore )}
\end{itemize}
Acto de expugnar.
\section{Expugnador}
\begin{itemize}
\item {Grp. gram.:adj.}
\end{itemize}
\begin{itemize}
\item {Grp. gram.:M.}
\end{itemize}
\begin{itemize}
\item {Proveniência:(Lat. \textunderscore expugnator\textunderscore )}
\end{itemize}
Que expugna.
Aquelle que expugna.
\section{Expugnando}
\begin{itemize}
\item {Grp. gram.:adj.}
\end{itemize}
\begin{itemize}
\item {Proveniência:(Lat. \textunderscore expugnandus\textunderscore )}
\end{itemize}
Que vai sêr expugnado:«\textunderscore ...surgirem deante da expugnanda cidade.\textunderscore »Filinto, \textunderscore D. Man.\textunderscore , III, 320.
\section{Expugnar}
\begin{itemize}
\item {Grp. gram.:v. t.}
\end{itemize}
\begin{itemize}
\item {Proveniência:(Lat. \textunderscore expugnare\textunderscore )}
\end{itemize}
Tomar, á força de armas.
Tomar de assalto.
Conquistar.
Vencer, pelejando.
Vencer.
\section{Expugnável}
\begin{itemize}
\item {Grp. gram.:adj.}
\end{itemize}
\begin{itemize}
\item {Proveniência:(Lat. \textunderscore expugnabilis\textunderscore )}
\end{itemize}
Que se póde expugnar.
\section{Expulsamento}
\begin{itemize}
\item {Grp. gram.:m.}
\end{itemize}
O mesmo que \textunderscore expulsão\textunderscore . Cf. Filinto, \textunderscore D. Man.\textunderscore , I, 69.
\section{Expulsão}
\begin{itemize}
\item {Grp. gram.:f.}
\end{itemize}
\begin{itemize}
\item {Proveniência:(Lat. \textunderscore expulsio\textunderscore )}
\end{itemize}
Acto ou effeito de expulsar.
Acto de expellir.
Secreção.
\section{Expulsar}
\begin{itemize}
\item {Grp. gram.:v. t.}
\end{itemize}
\begin{itemize}
\item {Proveniência:(Lat. \textunderscore expulsare\textunderscore )}
\end{itemize}
Expellir com força.
Fazer sair, por castigo.
Pôr fóra, violentamente: \textunderscore expulsar um hóspede\textunderscore .
\section{Expulsivo}
\begin{itemize}
\item {Grp. gram.:adj.}
\end{itemize}
\begin{itemize}
\item {Proveniência:(Lat. \textunderscore expulsivus\textunderscore )}
\end{itemize}
Que facilita a expulsão.
Que faz expulsar.
\section{Expulso}
\begin{itemize}
\item {Grp. gram.:adj.}
\end{itemize}
\begin{itemize}
\item {Proveniência:(Lat. \textunderscore expulsus\textunderscore )}
\end{itemize}
Que foi obrigado a sair.
Pôsto fóra, violentamente.
\section{Expulsor}
\begin{itemize}
\item {Grp. gram.:m.  e  adj.}
\end{itemize}
\begin{itemize}
\item {Proveniência:(Lat. \textunderscore expulsor\textunderscore )}
\end{itemize}
O que expulsa.
\section{Expulsório}
\begin{itemize}
\item {Grp. gram.:adj.}
\end{itemize}
\begin{itemize}
\item {Proveniência:(De \textunderscore expulso\textunderscore )}
\end{itemize}
Que envolve ordem de expulsão.
\section{Expultriz}
\begin{itemize}
\item {Grp. gram.:adj. f.}
\end{itemize}
\begin{itemize}
\item {Proveniência:(Lat. \textunderscore expultrix\textunderscore )}
\end{itemize}
Que expelle, que expulsa.
\section{Expunar}
\textunderscore v. t.\textunderscore  (e der.) \textunderscore Ant.\textunderscore 
O mesmo que \textunderscore expugnar\textunderscore , etc.:«\textunderscore ...a dita expunação do Tranto\textunderscore ». R. Pina, \textunderscore Affonso V\textunderscore , c. CCX.
\section{Expunção}
\begin{itemize}
\item {Grp. gram.:f.}
\end{itemize}
\begin{itemize}
\item {Proveniência:(Lat. \textunderscore expunctio\textunderscore )}
\end{itemize}
Acto de expungir.
\section{Expuncção}
\begin{itemize}
\item {Grp. gram.:f.}
\end{itemize}
\begin{itemize}
\item {Proveniência:(Lat. \textunderscore expunctio\textunderscore )}
\end{itemize}
Acto de expungir.
\section{Expungir}
\begin{itemize}
\item {Grp. gram.:v. t.}
\end{itemize}
\begin{itemize}
\item {Proveniência:(Lat. \textunderscore expungere\textunderscore )}
\end{itemize}
Apagar: \textunderscore o tempo expungiu a inscripcão\textunderscore .
Delir.
Fazer desapparecer (uma escrita), para a substituir por outra.
\section{Expurgação}
\begin{itemize}
\item {Grp. gram.:f.}
\end{itemize}
\begin{itemize}
\item {Proveniência:(Lat. \textunderscore expurgatio\textunderscore )}
\end{itemize}
Acto de expurgar.
Evacuação.
\section{Expurgador}
\begin{itemize}
\item {Grp. gram.:adj.}
\end{itemize}
\begin{itemize}
\item {Grp. gram.:M.}
\end{itemize}
\begin{itemize}
\item {Proveniência:(Lat. \textunderscore expurgator\textunderscore )}
\end{itemize}
Que expurga.
Aquelle que expurga.
\section{Expurgar}
\begin{itemize}
\item {Grp. gram.:v. t.}
\end{itemize}
\begin{itemize}
\item {Proveniência:(Lat. \textunderscore expurgare\textunderscore )}
\end{itemize}
Purgar completamente.
Limpar.
Corrigir: \textunderscore expurgar a reedição de uma obra\textunderscore .
Polir.
Descascar, esbrugar.
\section{Expurgatório}
\begin{itemize}
\item {Grp. gram.:adj.}
\end{itemize}
\begin{itemize}
\item {Grp. gram.:M.}
\end{itemize}
\begin{itemize}
\item {Proveniência:(Do lat. \textunderscore expurgatus\textunderscore )}
\end{itemize}
Que expurga.
Condemnatório.
Relação de livros condemnados pela Igreja.
\section{Expurgo}
\begin{itemize}
\item {Grp. gram.:m.}
\end{itemize}
\begin{itemize}
\item {Utilização:Bras}
\end{itemize}
O mesmo que \textunderscore expurgação\textunderscore .
\section{Exquisa}
\begin{itemize}
\item {Grp. gram.:f.}
\end{itemize}
\begin{itemize}
\item {Utilização:Ant.}
\end{itemize}
\begin{itemize}
\item {Proveniência:(Do rad. do lat. \textunderscore exquisitus\textunderscore )}
\end{itemize}
Inquirição.
\section{Exquisito}
\textunderscore adj.\textunderscore  (e der.)
(V. \textunderscore esquisito\textunderscore , etc.)
\section{Exsangue}
\begin{itemize}
\item {Grp. gram.:adj.}
\end{itemize}
\begin{itemize}
\item {Proveniência:(Lat. \textunderscore exsanguis\textunderscore )}
\end{itemize}
O mesmo que \textunderscore exangue\textunderscore .
\section{Exsicação}
\begin{itemize}
\item {Grp. gram.:f.}
\end{itemize}
\begin{itemize}
\item {Proveniência:(Lat. \textunderscore exsiccatio\textunderscore )}
\end{itemize}
Acto de exsicar.
\section{Exsicante}
\begin{itemize}
\item {Grp. gram.:adj.}
\end{itemize}
\begin{itemize}
\item {Utilização:Ant.}
\end{itemize}
\begin{itemize}
\item {Proveniência:(Lat. \textunderscore exsiccans\textunderscore )}
\end{itemize}
Que exsica.
\section{Exsicar}
\begin{itemize}
\item {Grp. gram.:v. t.}
\end{itemize}
\begin{itemize}
\item {Utilização:Des.}
\end{itemize}
\begin{itemize}
\item {Proveniência:(Lat. \textunderscore exsiccare\textunderscore )}
\end{itemize}
Fazer secar (drogas), para que se conservem.
Secar bem.
\section{Exsicativo}
\begin{itemize}
\item {Grp. gram.:adj.}
\end{itemize}
\begin{itemize}
\item {Proveniência:(Do lat. \textunderscore exsiccatus\textunderscore )}
\end{itemize}
Que tem a propriedade de exsicar.
\section{Exsiccação}
\begin{itemize}
\item {Grp. gram.:f.}
\end{itemize}
\begin{itemize}
\item {Proveniência:(Lat. \textunderscore exsiccatio\textunderscore )}
\end{itemize}
Acto de exsiccar.
\section{Exsiccante}
\begin{itemize}
\item {Grp. gram.:adj.}
\end{itemize}
\begin{itemize}
\item {Utilização:Ant.}
\end{itemize}
\begin{itemize}
\item {Proveniência:(Lat. \textunderscore exsiccans\textunderscore )}
\end{itemize}
Que exsicca.
\section{Exsiccar}
\begin{itemize}
\item {Grp. gram.:v. t.}
\end{itemize}
\begin{itemize}
\item {Utilização:Des.}
\end{itemize}
\begin{itemize}
\item {Proveniência:(Lat. \textunderscore exsiccare\textunderscore )}
\end{itemize}
Fazer secar (drogas), para que se conservem.
Secar bem.
\section{Exsiccativo}
\begin{itemize}
\item {Grp. gram.:adj.}
\end{itemize}
\begin{itemize}
\item {Proveniência:(Do lat. \textunderscore exsiccatus\textunderscore )}
\end{itemize}
Que tem a propriedade de exsiccar.
\section{Exsolver}
\begin{itemize}
\item {Grp. gram.:v. t.}
\end{itemize}
\begin{itemize}
\item {Proveniência:(Lat. \textunderscore exsolvere\textunderscore )}
\end{itemize}
Desligar.
Dissolver.
Pagar.
Solver. Cf. Camillo, \textunderscore Cancion. Al.\textunderscore , 36.
\section{Exspuição}
\begin{itemize}
\item {fónica:pu-i}
\end{itemize}
\begin{itemize}
\item {Grp. gram.:f.}
\end{itemize}
\begin{itemize}
\item {Proveniência:(Lat. \textunderscore exspuitio\textunderscore )}
\end{itemize}
Acto de expellir pela bôca.
\section{Exstipulado}
\begin{itemize}
\item {Grp. gram.:adj.}
\end{itemize}
\begin{itemize}
\item {Utilização:Bot.}
\end{itemize}
\begin{itemize}
\item {Proveniência:(Do lat. \textunderscore ex\textunderscore  + \textunderscore stipula\textunderscore )}
\end{itemize}
Privado de estípulas.
\section{Exsuar}
\begin{itemize}
\item {Grp. gram.:v. t.  e  i.}
\end{itemize}
O mesmo que \textunderscore exsudar\textunderscore .
\section{Exsucação}
\begin{itemize}
\item {Grp. gram.:f.}
\end{itemize}
\begin{itemize}
\item {Proveniência:(Do rad. do lat. \textunderscore exsuccare\textunderscore )}
\end{itemize}
O mesmo que \textunderscore equimose\textunderscore .
\section{Exsuccação}
\begin{itemize}
\item {Grp. gram.:f.}
\end{itemize}
\begin{itemize}
\item {Proveniência:(Do rad. do lat. \textunderscore exsuccare\textunderscore )}
\end{itemize}
O mesmo que \textunderscore ecchymose\textunderscore .
\section{Exsucção}
\begin{itemize}
\item {Grp. gram.:f.}
\end{itemize}
\begin{itemize}
\item {Proveniência:(Do rad. do lat. \textunderscore exsuctus\textunderscore )}
\end{itemize}
Acto de extrahir, sugando.
\section{Exsudação}
\begin{itemize}
\item {Grp. gram.:f.}
\end{itemize}
\begin{itemize}
\item {Proveniência:(Lat. \textunderscore exsudatio\textunderscore )}
\end{itemize}
Acto ou effeito de exsudar.
Transpiração.
Líquido, que, atravessando os poros vegetaes ou animaes, toma certa consistência ou viscosidade na superfície em que apparece.
\section{Exsudar}
\begin{itemize}
\item {Grp. gram.:v. t.}
\end{itemize}
\begin{itemize}
\item {Grp. gram.:V. i.}
\end{itemize}
\begin{itemize}
\item {Proveniência:(Lat. \textunderscore exsudare\textunderscore )}
\end{itemize}
Expellir em fórma de gotas ou de suor.
Sair ou correr em fórma de suor; sair gotejando.
\section{Exsudato}
\begin{itemize}
\item {Grp. gram.:m.}
\end{itemize}
\begin{itemize}
\item {Utilização:Med.}
\end{itemize}
\begin{itemize}
\item {Proveniência:(De \textunderscore exsudar\textunderscore )}
\end{itemize}
Producto soroso, purulento, resultante de processo inflammatório.
\section{Ex-superabundanti}
\begin{itemize}
\item {Grp. gram.:loc. adv.}
\end{itemize}
Com grande largueza; a saciedade:«\textunderscore provaram tambem ex-superabundanti...\textunderscore »Vieira, VI, 194.
(Loc. lat.)
\section{Exsurgir}
\begin{itemize}
\item {Grp. gram.:v. i.}
\end{itemize}
\begin{itemize}
\item {Grp. gram.:V. t.}
\end{itemize}
\begin{itemize}
\item {Proveniência:(Lat. \textunderscore exsurgere\textunderscore )}
\end{itemize}
Levantar-se.
Levantar. Cf. C. Lobo, \textunderscore Sát.\textunderscore , 139 e 158.
\section{Extante}
\begin{itemize}
\item {Grp. gram.:adj.}
\end{itemize}
\begin{itemize}
\item {Utilização:Des.}
\end{itemize}
\begin{itemize}
\item {Proveniência:(Lat. \textunderscore exstans\textunderscore )}
\end{itemize}
Que está patente.
\section{Extar}
\begin{itemize}
\item {Grp. gram.:v. i.}
\end{itemize}
\begin{itemize}
\item {Utilização:Des.}
\end{itemize}
\begin{itemize}
\item {Proveniência:(Lat. \textunderscore exstare\textunderscore )}
\end{itemize}
Estar patente, elevado.
O mesmo que \textunderscore subsistir\textunderscore . Cf. Vieira, II, 270.
\section{Êxtase}
\begin{itemize}
\item {Grp. gram.:m.}
\end{itemize}
\begin{itemize}
\item {Proveniência:(Lat. \textunderscore exstasis\textunderscore )}
\end{itemize}
Arrebatamento de ânimo.
Arroubo.
Enlêvo.
Contemplação íntima de coisas sobrenaturaes.
\section{Êxtasi}
\begin{itemize}
\item {Grp. gram.:m.}
\end{itemize}
(V.êxtase)
\section{Extasiante}
\begin{itemize}
\item {Grp. gram.:adj.}
\end{itemize}
Que extasia.
\section{Extasiar}
\begin{itemize}
\item {Grp. gram.:v. t.}
\end{itemize}
\begin{itemize}
\item {Proveniência:(De \textunderscore êxtasi\textunderscore )}
\end{itemize}
Tornar extático.
Enlevar; encantar.
\section{Extâsis}
\begin{itemize}
\item {Grp. gram.:m.}
\end{itemize}
(V.êxtase)
\section{Extaticamente}
\begin{itemize}
\item {Grp. gram.:adv.}
\end{itemize}
De modo extático.
\section{Extático}
\begin{itemize}
\item {Grp. gram.:adj.}
\end{itemize}
\begin{itemize}
\item {Utilização:Fam.}
\end{itemize}
\begin{itemize}
\item {Proveniência:(Gr. \textunderscore exstatikos\textunderscore )}
\end{itemize}
Que caíu em êxtase.
Enlevado.
Arroubado.
Que causa êxtase.
Causado por êxtase.
Espantado, maravilhado.
\section{Extemporaneamente}
\begin{itemize}
\item {Grp. gram.:adv.}
\end{itemize}
De modo extemporâneo.
\section{Extemporaneidade}
\begin{itemize}
\item {Grp. gram.:f.}
\end{itemize}
Qualidade daquillo que é extemporâneo.
\section{Extemporâneo}
\begin{itemize}
\item {Grp. gram.:adj.}
\end{itemize}
\begin{itemize}
\item {Proveniência:(Lat. \textunderscore extemporaneus\textunderscore )}
\end{itemize}
Que é fóra de tempo.
Inopportuno.
Improvisado.
Impróprio da occasião em que se faz ou succede.
\section{Extender}
\textunderscore v. i.\textunderscore  e \textunderscore p.\textunderscore  (e der.)
(V. \textunderscore estender\textunderscore , etc.)
\section{Extensamente}
\begin{itemize}
\item {Grp. gram.:adv.}
\end{itemize}
Largamente.
Por extenso.
\section{Extensão}
\begin{itemize}
\item {Grp. gram.:f.}
\end{itemize}
\begin{itemize}
\item {Proveniência:(Lat. \textunderscore extensio\textunderscore )}
\end{itemize}
Effeito de estender.
Qualidade daquillo que é extenso.
Dimensão.
Porção de espaço ou de tempo.
Superfície.
Operação, em que se estende a parte inferior de um osso fracturado.
Doença no tendão flexor do pé do cavallo.
Ampliação.
Desenvolvimento.
Intensidade.
Vastidão: \textunderscore a extensão do Oceano\textunderscore .
\section{Extensibilidade}
\begin{itemize}
\item {Grp. gram.:f.}
\end{itemize}
Qualidade daquillo que é extensível.
\section{Extensidade}
\begin{itemize}
\item {Grp. gram.:f.}
\end{itemize}
\begin{itemize}
\item {Utilização:Bras}
\end{itemize}
\begin{itemize}
\item {Utilização:Neol.}
\end{itemize}
O mesmo que \textunderscore extensão\textunderscore .
\section{Extensivamente}
\begin{itemize}
\item {Grp. gram.:adv.}
\end{itemize}
O mesmo que \textunderscore extensamente\textunderscore .
De modo extensivo.
Por ampliação.
\section{Extensível}
\begin{itemize}
\item {Grp. gram.:adj.}
\end{itemize}
\begin{itemize}
\item {Proveniência:(De \textunderscore extenso\textunderscore )}
\end{itemize}
O mesmo que \textunderscore estendível\textunderscore .
\section{Extensivo}
\begin{itemize}
\item {Grp. gram.:adj.}
\end{itemize}
\begin{itemize}
\item {Proveniência:(Lat. \textunderscore extensivus\textunderscore )}
\end{itemize}
Que estende.
Estendivel.
Applicável a mais de um caso.
\section{Extenso}
\begin{itemize}
\item {Grp. gram.:adj.}
\end{itemize}
\begin{itemize}
\item {Proveniência:(Do lat. \textunderscore extensus\textunderscore )}
\end{itemize}
Que tem extensão.
Largo.
Longo.
Espaçoso.
Duradoiro.
\section{Extensor}
\begin{itemize}
\item {Grp. gram.:adj.}
\end{itemize}
\begin{itemize}
\item {Proveniência:(De \textunderscore extenso\textunderscore )}
\end{itemize}
Que serve para estender: \textunderscore músculo extensor\textunderscore .
\section{Extenuação}
\begin{itemize}
\item {Grp. gram.:f.}
\end{itemize}
\begin{itemize}
\item {Proveniência:(Lat. \textunderscore extenuatio\textunderscore )}
\end{itemize}
Acto ou effeito de extenuar.
Emprêgo de uma expressão branda ou attenuada, em vez de outra mais forte.
\section{Extenuadamente}
\begin{itemize}
\item {Grp. gram.:adv.}
\end{itemize}
\begin{itemize}
\item {Proveniência:(De \textunderscore extenuar\textunderscore )}
\end{itemize}
Debilmente, com fraqueza.
\section{Extenuador}
\begin{itemize}
\item {Grp. gram.:adj.}
\end{itemize}
Que extenua.
\section{Extenuante}
\begin{itemize}
\item {Grp. gram.:adj.}
\end{itemize}
\begin{itemize}
\item {Proveniência:(Lat. \textunderscore extenuans\textunderscore )}
\end{itemize}
Que extenua.
\section{Extenuar}
\begin{itemize}
\item {Grp. gram.:v. t.}
\end{itemize}
\begin{itemize}
\item {Utilização:Fig.}
\end{itemize}
\begin{itemize}
\item {Proveniência:(Lat. \textunderscore extenuare\textunderscore )}
\end{itemize}
Tornar tênue.
Enfraquecer.
Gastar; dissipar.
\section{Extenuativo}
\begin{itemize}
\item {Grp. gram.:adj.}
\end{itemize}
\begin{itemize}
\item {Proveniência:(De \textunderscore extenuar\textunderscore )}
\end{itemize}
Extenuante.
\section{Exterior}
\begin{itemize}
\item {Grp. gram.:adj.}
\end{itemize}
\begin{itemize}
\item {Grp. gram.:M.}
\end{itemize}
\begin{itemize}
\item {Proveniência:(Lat. \textunderscore exterior\textunderscore )}
\end{itemize}
Que está da parte de fóra.
Externo.
Superficial.
Apparente.
Relativo ás nações estrangeiras.
Parte externa.
Apparência, aspecto.
As nações estrangeiras: \textunderscore notícias do exterior\textunderscore .
\section{Exterioridade}
\begin{itemize}
\item {Grp. gram.:f.}
\end{itemize}
Qualidade daquillo que é exterior.
\section{Exteriorização}
\begin{itemize}
\item {Grp. gram.:f.}
\end{itemize}
Acto ou effeito de exteriorizar.
\section{Exteriorizar}
\begin{itemize}
\item {Grp. gram.:v. t.}
\end{itemize}
\begin{itemize}
\item {Proveniência:(De \textunderscore exterior\textunderscore )}
\end{itemize}
Tornar exterior.
Manifestar (sentimentos, ideias, etc.).
\section{Exteriormente}
\begin{itemize}
\item {Grp. gram.:adv.}
\end{itemize}
\begin{itemize}
\item {Proveniência:(De \textunderscore exterior\textunderscore )}
\end{itemize}
Da parte de fóra.
Apparentemente.
\section{Exterminação}
\begin{itemize}
\item {Grp. gram.:f.}
\end{itemize}
\begin{itemize}
\item {Proveniência:(Lat. \textunderscore exterminatio\textunderscore )}
\end{itemize}
Acto ou effeito de exterminar.
\section{Exterminador}
\begin{itemize}
\item {Grp. gram.:adj.}
\end{itemize}
\begin{itemize}
\item {Grp. gram.:M.}
\end{itemize}
\begin{itemize}
\item {Proveniência:(Lat. \textunderscore exterminator\textunderscore )}
\end{itemize}
Que extermina.
Aquelle que extermina.
\section{Exterminante}
\begin{itemize}
\item {Grp. gram.:adj.}
\end{itemize}
Que extermina.
\section{Exterminar}
\begin{itemize}
\item {Grp. gram.:v. t.}
\end{itemize}
\begin{itemize}
\item {Utilização:Fig.}
\end{itemize}
\begin{itemize}
\item {Proveniência:(Lat. \textunderscore exterminare\textunderscore )}
\end{itemize}
Pôr fóra dos limites de alguma terra ou região.
Expulsar.
Banir.
Extirpar.
Destruír.
Eliminar, matando.
Aniquilar.
\section{Exterminável}
\begin{itemize}
\item {Grp. gram.:adj.}
\end{itemize}
Que se póde exterminar.
\section{Extermínio}
\begin{itemize}
\item {Grp. gram.:m.}
\end{itemize}
\begin{itemize}
\item {Proveniência:(Lat. \textunderscore exterminium\textunderscore )}
\end{itemize}
Acto ou effeito de exterminar.
Assolação.
Destruição.
Ruína total.
\section{Externação}
\begin{itemize}
\item {Grp. gram.:f.}
\end{itemize}
Acto de externar.
\section{Externamente}
\begin{itemize}
\item {Grp. gram.:adv.}
\end{itemize}
\begin{itemize}
\item {Proveniência:(De \textunderscore externo\textunderscore )}
\end{itemize}
O mesmo que \textunderscore exteriormente\textunderscore .
\section{Externar}
\begin{itemize}
\item {Grp. gram.:v. t.}
\end{itemize}
\begin{itemize}
\item {Utilização:Neol.}
\end{itemize}
\begin{itemize}
\item {Proveniência:(De \textunderscore externo\textunderscore )}
\end{itemize}
Tornar externo, manifestar exteriormente.
Exteriorizar.
\section{Externato}
\begin{itemize}
\item {Grp. gram.:m.}
\end{itemize}
\begin{itemize}
\item {Proveniência:(Lat. \textunderscore externatus\textunderscore )}
\end{itemize}
Estabelecimento de instrucção, em que só há alumnos externos.
\section{Externo}
\begin{itemize}
\item {Grp. gram.:adj.}
\end{itemize}
\begin{itemize}
\item {Grp. gram.:M.}
\end{itemize}
\begin{itemize}
\item {Proveniência:(Lat. \textunderscore externus\textunderscore )}
\end{itemize}
O mesmo que \textunderscore exterior\textunderscore .
Alumno externo de uma escola.
\section{Extero...}
\begin{itemize}
\item {Grp. gram.:pref.}
\end{itemize}
\begin{itemize}
\item {Proveniência:(De \textunderscore exterior\textunderscore )}
\end{itemize}
(designativo do que é exterior)
\section{Extero-anterior}
\begin{itemize}
\item {Grp. gram.:adj.}
\end{itemize}
Situado externamente e na parte anterior.
\section{Extero-inferior}
\begin{itemize}
\item {Grp. gram.:adj.}
\end{itemize}
Situado externamente e na parte inferior.
\section{Extero-posterior}
\begin{itemize}
\item {Grp. gram.:adj.}
\end{itemize}
Situado externamente e na parte posterior.
\section{Extero-superior}
\begin{itemize}
\item {Grp. gram.:adj.}
\end{itemize}
Situado externamente e na parte superior.
\section{Exterritorialidade}
\begin{itemize}
\item {Grp. gram.:f.}
\end{itemize}
\begin{itemize}
\item {Proveniência:(De \textunderscore ex...\textunderscore  + \textunderscore territorial\textunderscore )}
\end{itemize}
Direito, que os representantes de nações estrangeiras têm, de, fóra do seu país, se regerem pelas leis delle.
\section{Extinção}
\begin{itemize}
\item {Grp. gram.:f.}
\end{itemize}
\begin{itemize}
\item {Proveniência:(Lat. \textunderscore extinctio\textunderscore )}
\end{itemize}
Acto ou efeito de extinguir.
Acabamento.
Extirpação.
Cessação: \textunderscore extinção de privilégios\textunderscore .
Obliteração.
Extermínio.
\section{Extincção}
\begin{itemize}
\item {Grp. gram.:f.}
\end{itemize}
\begin{itemize}
\item {Proveniência:(Lat. \textunderscore extinctio\textunderscore )}
\end{itemize}
Acto ou effeito de extinguir.
Acabamento.
Extirpação.
Cessação: \textunderscore extincção de privilégios\textunderscore .
Obliteração.
Extermínio.
\section{Extinguidor}
\begin{itemize}
\item {Grp. gram.:m.}
\end{itemize}
Aquillo que extingue; extintor.
\section{Extinguir}
\begin{itemize}
\item {Grp. gram.:v. t.}
\end{itemize}
\begin{itemize}
\item {Utilização:Ext.}
\end{itemize}
\begin{itemize}
\item {Grp. gram.:V. p.}
\end{itemize}
\begin{itemize}
\item {Proveniência:(Lat. \textunderscore extinguere\textunderscore )}
\end{itemize}
Apagar (lume, fogo).
Destruir.
Dissolver.
Abolir.
Gastar; dissipar.
Arruinar.
Pagar inteiramente: \textunderscore extinguir uma dívida\textunderscore .
Amortecer.
Apagar-se.
Acabar.
Morrer.
Expungir-se.
\section{Extinguível}
\begin{itemize}
\item {Grp. gram.:adj.}
\end{itemize}
Que se póde extinguir.
\section{Extintivo}
\begin{itemize}
\item {Grp. gram.:adj.}
\end{itemize}
Relativo a extincção.
Que determina extincção. Cf. Assis, \textunderscore Águas\textunderscore , 324.
\section{Extinto}
\begin{itemize}
\item {Grp. gram.:adj.}
\end{itemize}
\begin{itemize}
\item {Grp. gram.:M.}
\end{itemize}
\begin{itemize}
\item {Proveniência:(Lat. \textunderscore extinctus\textunderscore )}
\end{itemize}
Findo, acabado.
Dissolvido.
Supprimido.
Morto.
Indivíduo, que morreu.
\section{Extintor}
\begin{itemize}
\item {Grp. gram.:adj.}
\end{itemize}
\begin{itemize}
\item {Grp. gram.:M.}
\end{itemize}
\begin{itemize}
\item {Proveniência:(Lat. \textunderscore extinctor\textunderscore )}
\end{itemize}
Que extingue.
Aquillo que extingue.
Apparelho para apagar incêndios.
\section{Extipuláceo}
\begin{itemize}
\item {Grp. gram.:adj.}
\end{itemize}
(V.exstipulado)
\section{Extirpação}
\begin{itemize}
\item {Grp. gram.:f.}
\end{itemize}
\begin{itemize}
\item {Proveniência:(Lat. \textunderscore extirpatio\textunderscore )}
\end{itemize}
Acto ou effeito de extirpar.
\section{Extirpador}
\begin{itemize}
\item {Grp. gram.:adj.}
\end{itemize}
\begin{itemize}
\item {Grp. gram.:M.}
\end{itemize}
\begin{itemize}
\item {Proveniência:(Lat. \textunderscore extirpator\textunderscore )}
\end{itemize}
Que extirpa.
Instrumento agrícola, para arrancar ervas ou raízes.
\section{Extirpamento}
\begin{itemize}
\item {Grp. gram.:m.}
\end{itemize}
O mesmo que \textunderscore extirpação\textunderscore .
\section{Extirpar}
\begin{itemize}
\item {Grp. gram.:v. t.}
\end{itemize}
\begin{itemize}
\item {Utilização:Fig.}
\end{itemize}
\begin{itemize}
\item {Proveniência:(Lat. \textunderscore extirpare\textunderscore )}
\end{itemize}
Desarraigar; arrancar pela raíz.
Extrahir cirurgicamente (um cancro, etc.).
Exterminar.
\section{Extirpável}
\begin{itemize}
\item {Grp. gram.:adj.}
\end{itemize}
Que se póde extirpar.
\section{Extíspice}
\begin{itemize}
\item {Grp. gram.:m.}
\end{itemize}
\begin{itemize}
\item {Proveniência:(Lat. \textunderscore extispex\textunderscore )}
\end{itemize}
O mesmo que \textunderscore arúspice\textunderscore .
\section{Extispicina}
\begin{itemize}
\item {Grp. gram.:f.}
\end{itemize}
O mesmo que \textunderscore extispício\textunderscore .
\section{Extispício}
\begin{itemize}
\item {Grp. gram.:m.}
\end{itemize}
\begin{itemize}
\item {Proveniência:(Lat. \textunderscore extispicium\textunderscore )}
\end{itemize}
Supposta arte de adivinhar, por meio das entranhas das víctimas dos antigos sacrifícios.
\section{Extorção}
\begin{itemize}
\item {Grp. gram.:f.}
\end{itemize}
\begin{itemize}
\item {Proveniência:(Do rad. do lat. \textunderscore extortus\textunderscore )}
\end{itemize}
Acto ou effeito de extorquir.
\section{Extorcer}
\textunderscore v. t.\textunderscore  (e der.)
(V. \textunderscore estorcer\textunderscore , etc.)
\section{Extorcionário}
\begin{itemize}
\item {Grp. gram.:adj.}
\end{itemize}
\begin{itemize}
\item {Proveniência:(De \textunderscore extorção\textunderscore )}
\end{itemize}
Que faz extorção.
\section{Extorquir}
\begin{itemize}
\item {Grp. gram.:v. t.}
\end{itemize}
\begin{itemize}
\item {Proveniência:(Lat. \textunderscore extorquere\textunderscore )}
\end{itemize}
Obter com violência.
Tirar á fôrça.
Conseguir por meio de tortura: \textunderscore extorquir declarações\textunderscore .
\section{Extorsão}
\begin{itemize}
\item {Grp. gram.:f.}
\end{itemize}
\begin{itemize}
\item {Proveniência:(Do rad. do lat. \textunderscore extorsum\textunderscore )}
\end{itemize}
O mesmo que \textunderscore extorção\textunderscore .
\section{Extorsivo}
\begin{itemize}
\item {Grp. gram.:adj.}
\end{itemize}
\begin{itemize}
\item {Proveniência:(Do lat. \textunderscore extorsus\textunderscore )}
\end{itemize}
O mesmo que \textunderscore extorcionário\textunderscore .
\section{Extorso}
\begin{itemize}
\item {Grp. gram.:m.}
\end{itemize}
\begin{itemize}
\item {Proveniência:(Lat. \textunderscore extorsus\textunderscore )}
\end{itemize}
(V.extorsão)
\section{Extortor}
\begin{itemize}
\item {Grp. gram.:adj.}
\end{itemize}
\begin{itemize}
\item {Proveniência:(Lat. \textunderscore extortor\textunderscore )}
\end{itemize}
(V.extorcionário)
\section{Extra...}
\begin{itemize}
\item {Grp. gram.:pref.}
\end{itemize}
\begin{itemize}
\item {Proveniência:(Lat. \textunderscore extra\textunderscore )}
\end{itemize}
Além; fóra.
\section{Extra-alcance}
\begin{itemize}
\item {Grp. gram.:loc. adv.}
\end{itemize}
\begin{itemize}
\item {Utilização:Des.}
\end{itemize}
Fóra do alcance.
\section{Extra-axillar}
\begin{itemize}
\item {Grp. gram.:adj.}
\end{itemize}
\begin{itemize}
\item {Utilização:Bot.}
\end{itemize}
Que nasce ao lado da axilla das folhas.
\section{Extrabarreiras}
\begin{itemize}
\item {Grp. gram.:loc. adv.}
\end{itemize}
Fóra das barreiras ou fóra de portas.
\section{Extracção}
\begin{itemize}
\item {Grp. gram.:f.}
\end{itemize}
\begin{itemize}
\item {Utilização:Gal}
\end{itemize}
\begin{itemize}
\item {Proveniência:(Lat. \textunderscore extractio\textunderscore )}
\end{itemize}
Acto ou effeito de extrahir ou extractar.
Aquillo que se extrai.
Maior ou menor procura ou venda de productos naturaes ou artísticos.
Venda.
Operação arithmética ou algébrica, para se conhecer a raíz de uma potência.
Ascendência, origem:«\textunderscore ...com o moço de baixa extracção.\textunderscore »Camillo, \textunderscore Filha do Doutor Negro\textunderscore , 35.
\section{Extracrescente}
\begin{itemize}
\item {Grp. gram.:adj.}
\end{itemize}
\begin{itemize}
\item {Proveniência:(De \textunderscore extra...\textunderscore  + \textunderscore crescente\textunderscore )}
\end{itemize}
Que se desenvolve exteriormente, (falando-se de vegetaes).
\section{Extractar}
\begin{itemize}
\item {Grp. gram.:v. t.}
\end{itemize}
\begin{itemize}
\item {Proveniência:(De \textunderscore extracto\textunderscore )}
\end{itemize}
Fazer extracto de: \textunderscore extractar um capítulo\textunderscore .
Resumir: \textunderscore extractar um processo\textunderscore .
Preparar chimicamente, por extracção.
\section{Extractivo}
\begin{itemize}
\item {Grp. gram.:adj.}
\end{itemize}
\begin{itemize}
\item {Grp. gram.:M.}
\end{itemize}
\begin{itemize}
\item {Proveniência:(De \textunderscore extracto\textunderscore )}
\end{itemize}
Que se póde extrahir chimicamente de substâncias vegetaes ou animaes.
Que indica extracção.
Princípio orgânico, solúvel, que se suppunha existir nas plantas e nos animaes, com a propriedade de se tornar espêsso pela evaporação.
\section{Extracto}
\begin{itemize}
\item {Grp. gram.:m.}
\end{itemize}
\begin{itemize}
\item {Proveniência:(Lat. \textunderscore extractus\textunderscore )}
\end{itemize}
Coisa extrahida.
Producto chímico de uma substância.
Fragmento, artigo, trecho, que se extrai de uma obra.
Synopse, resumo.
Cópia.
\section{Extractor}
\begin{itemize}
\item {Grp. gram.:m.}
\end{itemize}
\begin{itemize}
\item {Proveniência:(De \textunderscore extracto\textunderscore )}
\end{itemize}
Aquelle que extrai.
\section{Extradição}
\begin{itemize}
\item {Grp. gram.:f.}
\end{itemize}
\begin{itemize}
\item {Proveniência:(Do lat. \textunderscore ex\textunderscore  + \textunderscore traditio\textunderscore )}
\end{itemize}
Acção de enviar ou entregar um refugiado a um Govêrno estranho que o reclama.
\section{Extradicionar}
\begin{itemize}
\item {Grp. gram.:v. t.}
\end{itemize}
\begin{itemize}
\item {Utilização:Neol.}
\end{itemize}
Fazer a extradição de; extraditar.
\section{Extraditar}
\begin{itemize}
\item {Grp. gram.:v. t.}
\end{itemize}
\begin{itemize}
\item {Proveniência:(Do lat. \textunderscore ex\textunderscore  + \textunderscore traditus\textunderscore )}
\end{itemize}
Enviar ou entregar a Govêrno estrangeiro (um refugiado que elle reclama).
Fazer extradição de.
\section{Extradorsado}
\begin{itemize}
\item {Grp. gram.:adj.}
\end{itemize}
Que tem extradorso.
\section{Extradorso}
\begin{itemize}
\item {fónica:dôr}
\end{itemize}
\begin{itemize}
\item {Grp. gram.:m.}
\end{itemize}
\begin{itemize}
\item {Proveniência:(De \textunderscore extra...\textunderscore  + \textunderscore dorso\textunderscore )}
\end{itemize}
Superfície exterior e convexa de uma abóbada ou arcada regular.
\section{Extrafino}
\begin{itemize}
\item {Grp. gram.:adj.}
\end{itemize}
\begin{itemize}
\item {Proveniência:(De \textunderscore extra...\textunderscore  + \textunderscore fino\textunderscore )}
\end{itemize}
Diz-se dos artigos de commércio, que são de qualidade superior ou se apresentam como taes.
\section{Extrafolheáceo}
\begin{itemize}
\item {Grp. gram.:adj.}
\end{itemize}
\begin{itemize}
\item {Proveniência:(Do lat. \textunderscore extra\textunderscore  + \textunderscore folium\textunderscore )}
\end{itemize}
Diz-se das estípulas ou de outros órgãos vegetaes, que crescem fóra ou ao lado das fôlhas.
\section{Extrafoliáceo}
\begin{itemize}
\item {Grp. gram.:adj.}
\end{itemize}
\begin{itemize}
\item {Utilização:Bot.}
\end{itemize}
\begin{itemize}
\item {Proveniência:(Do lat. \textunderscore extra\textunderscore  + \textunderscore folium\textunderscore )}
\end{itemize}
Diz-se das estípulas ou de outros órgãos vegetaes, que crescem fóra ou ao lado das fôlhas.
\section{Extrafólio}
\begin{itemize}
\item {Grp. gram.:adj.}
\end{itemize}
\begin{itemize}
\item {Proveniência:(Do lat. \textunderscore extra\textunderscore  + \textunderscore folium\textunderscore )}
\end{itemize}
(V.extrafoliáceo)
\section{Extrahir}
\begin{itemize}
\item {Grp. gram.:v. t.}
\end{itemize}
\begin{itemize}
\item {Proveniência:(Lat. \textunderscore extrahere\textunderscore )}
\end{itemize}
Tirar para fóra; tirar de dentro: \textunderscore extrahir um féto\textunderscore .
Separar chimicamente.
Copiar.
Colher.
Sugar.
Determinar mathematicamente (a raíz de uma potência).
Extractar.
\section{Extrahivel}
\begin{itemize}
\item {Grp. gram.:adj.}
\end{itemize}
Que se póde extrahir.
\section{Extrahumano}
\begin{itemize}
\item {Grp. gram.:adj.}
\end{itemize}
\begin{itemize}
\item {Proveniência:(De \textunderscore extra...\textunderscore  + \textunderscore humano\textunderscore )}
\end{itemize}
O mesmo que \textunderscore sobrehumano\textunderscore .
\section{Extrair}
\begin{itemize}
\item {Grp. gram.:v. t.}
\end{itemize}
\begin{itemize}
\item {Proveniência:(Lat. \textunderscore extrahere\textunderscore )}
\end{itemize}
Tirar para fóra; tirar de dentro: \textunderscore extrair um féto\textunderscore .
Separar quimicamente.
Copiar.
Colher.
Sugar.
Determinar matematicamente (a raíz de uma potência).
Extractar.
\section{Extraível}
\begin{itemize}
\item {Grp. gram.:adj.}
\end{itemize}
Que se póde extrair.
\section{Extrajudicial}
\begin{itemize}
\item {Grp. gram.:adj.}
\end{itemize}
\begin{itemize}
\item {Proveniência:(De \textunderscore extra...\textunderscore  + \textunderscore judicial\textunderscore )}
\end{itemize}
Que se não refere a processo ou formalidade judicial: \textunderscore informação extrajudicial\textunderscore .
\section{Extrajudicialmente}
\begin{itemize}
\item {Grp. gram.:adv.}
\end{itemize}
De modo extrajudicial.
\section{Extrajudiciário}
\begin{itemize}
\item {Grp. gram.:adj.}
\end{itemize}
\begin{itemize}
\item {Proveniência:(De \textunderscore extra...\textunderscore  + \textunderscore judiciário\textunderscore )}
\end{itemize}
O mesmo que \textunderscore extrajudicial\textunderscore .
\section{Extralegal}
\begin{itemize}
\item {Grp. gram.:adj.}
\end{itemize}
\begin{itemize}
\item {Proveniência:(De \textunderscore extra...\textunderscore  + \textunderscore legal\textunderscore )}
\end{itemize}
O mesmo que \textunderscore illegal\textunderscore . Cf. Herculano, \textunderscore Hist. de Port.\textunderscore , I, 455.
\section{Extramarital}
\begin{itemize}
\item {Grp. gram.:adj.}
\end{itemize}
\begin{itemize}
\item {Proveniência:(De \textunderscore extra...\textunderscore  + \textunderscore marital\textunderscore )}
\end{itemize}
Estranho ao matrimónio. Cf. Val. Magalhães, \textunderscore Contos\textunderscore .
\section{Extramaritalmente}
\begin{itemize}
\item {Grp. gram.:adv.}
\end{itemize}
De modo extramarital.
\section{Extramérico}
\begin{itemize}
\item {Grp. gram.:adj.}
\end{itemize}
Relativo ao extrâmero.
\section{Extrâmero}
\begin{itemize}
\item {Grp. gram.:m.}
\end{itemize}
\begin{itemize}
\item {Utilização:Anat.}
\end{itemize}
\begin{itemize}
\item {Proveniência:(Do lat. \textunderscore extra\textunderscore  + gr. \textunderscore meros\textunderscore )}
\end{itemize}
Cada uma das partes do corpo humano, considerando-se separadas por planos parallelos ao sagittal.
\section{Extramontado}
\begin{itemize}
\item {Grp. gram.:adj.}
\end{itemize}
(V.estramontado)
\section{Extramundano}
\begin{itemize}
\item {Grp. gram.:adj.}
\end{itemize}
\begin{itemize}
\item {Utilização:Des.}
\end{itemize}
\begin{itemize}
\item {Proveniência:(De \textunderscore extra...\textunderscore  + \textunderscore mundano\textunderscore )}
\end{itemize}
Que está fóra dos limites do mundo ou das habituaes condições da existência.
\section{Extramural}
\begin{itemize}
\item {Grp. gram.:adj.}
\end{itemize}
\begin{itemize}
\item {Proveniência:(De \textunderscore extra...\textunderscore  + \textunderscore mural\textunderscore )}
\end{itemize}
Que fica fóra dos muros ou das muralhas.
\section{Extramuros}
\begin{itemize}
\item {Grp. gram.:adv.}
\end{itemize}
\begin{itemize}
\item {Proveniência:(De \textunderscore extra...\textunderscore  + \textunderscore muro\textunderscore )}
\end{itemize}
Fóra dos muros ou das muralhas.
\section{Extranatural}
\begin{itemize}
\item {Grp. gram.:adj.}
\end{itemize}
\begin{itemize}
\item {Proveniência:(De \textunderscore extra...\textunderscore  + \textunderscore natural\textunderscore )}
\end{itemize}
O mesmo que \textunderscore sobrenatural\textunderscore .
\section{Extranaturalmente}
\begin{itemize}
\item {Grp. gram.:adv.}
\end{itemize}
De modo extranatural.
\section{Extrâneo}
\begin{itemize}
\item {Grp. gram.:adj.}
\end{itemize}
\begin{itemize}
\item {Utilização:Des.}
\end{itemize}
O mesmo que \textunderscore estranho\textunderscore .
\section{Extrangeiro}
\textunderscore m.\textunderscore  e \textunderscore adj.\textunderscore  (e der.)
(V. \textunderscore estrangeiro\textunderscore , etc.)
\section{Extranho}
\textunderscore m.\textunderscore  e \textunderscore adj.\textunderscore  (e der.)
O mesmo que \textunderscore estranho\textunderscore , etc.
\section{Extranormal}
\begin{itemize}
\item {Grp. gram.:adj.}
\end{itemize}
\begin{itemize}
\item {Proveniência:(De \textunderscore extra...\textunderscore  + \textunderscore normal\textunderscore )}
\end{itemize}
Que está fóra da normalidade; anormal.
\section{Extranumeral}
\begin{itemize}
\item {Grp. gram.:adj.}
\end{itemize}
\begin{itemize}
\item {Proveniência:(De \textunderscore extra...\textunderscore  + \textunderscore numeral\textunderscore )}
\end{itemize}
Que está além de um número.
\section{Extranumerário}
\begin{itemize}
\item {Grp. gram.:adj.}
\end{itemize}
\begin{itemize}
\item {Proveniência:(De \textunderscore extra...\textunderscore  + \textunderscore numerário\textunderscore )}
\end{itemize}
Que está fóra do número certo e determinado.
\section{Extraofficial}
\begin{itemize}
\item {Grp. gram.:adj.}
\end{itemize}
\begin{itemize}
\item {Proveniência:(De \textunderscore extra...\textunderscore  + \textunderscore official\textunderscore )}
\end{itemize}
Que não tem origem official; que não proveio de autoridade ou de funccionários públicos.
Estranho a negócios públicos.
\section{Extraofficialmente}
\begin{itemize}
\item {Grp. gram.:adv.}
\end{itemize}
De modo extraofficial.
\section{Extraordinariamente}
\begin{itemize}
\item {Grp. gram.:adv.}
\end{itemize}
De modo extraordinário.
\section{Extraordinário}
\begin{itemize}
\item {Grp. gram.:adj.}
\end{itemize}
\begin{itemize}
\item {Grp. gram.:M.}
\end{itemize}
\begin{itemize}
\item {Proveniência:(Lat. \textunderscore extra...\textunderscore  + \textunderscore ordinarius\textunderscore )}
\end{itemize}
Que não é ordinário.
Anormal.
Singular.
Raro: \textunderscore animal extraordinário\textunderscore .
Grande, excessivo: \textunderscore riqueza extraordinária\textunderscore .
Muito distinto.
Admirável: \textunderscore poéta extraordinário\textunderscore .
Aquillo que se não faz de ordinário.
Aquillo que excede a despesa ordinária: \textunderscore o que se gasta em extraordinários\textunderscore .
\section{Extrapassar}
\textunderscore v. t.\textunderscore  (des.)(V.ultrapassar)
\section{Extrapor}
\begin{itemize}
\item {Grp. gram.:v. t.}
\end{itemize}
\begin{itemize}
\item {Proveniência:(De \textunderscore extra...\textunderscore  + \textunderscore pôr\textunderscore )}
\end{itemize}
Pôr fóra; pôr além.
\section{Extra-regulamentar}
\begin{itemize}
\item {Grp. gram.:adj.}
\end{itemize}
Estranho a regulamento.
\section{Extrário}
\begin{itemize}
\item {Grp. gram.:adj.}
\end{itemize}
\begin{itemize}
\item {Utilização:Bot.}
\end{itemize}
\begin{itemize}
\item {Proveniência:(Lat. \textunderscore extrarius\textunderscore )}
\end{itemize}
Diz-se do embryão, que está fóra do perisperma.
\section{Extra-sagittal}
\begin{itemize}
\item {Grp. gram.:adj.}
\end{itemize}
\begin{itemize}
\item {Utilização:Anat.}
\end{itemize}
Diz-se dos planos paralellos ao sagittal.
\section{Extra-secular}
\begin{itemize}
\item {Grp. gram.:adj.}
\end{itemize}
Que existe por mais de um século.
\section{Extraterritorial}
\begin{itemize}
\item {Grp. gram.:adj.}
\end{itemize}
Que está fóra de um território.
\section{Extrathorácico}
\begin{itemize}
\item {Grp. gram.:adj.}
\end{itemize}
Que está fóra do thórax.
\section{Extratimpânico}
\begin{itemize}
\item {Grp. gram.:adj.}
\end{itemize}
Que está fóra do tímpano.
\section{Extratympânico}
\begin{itemize}
\item {Grp. gram.:adj.}
\end{itemize}
Que está fóra do týmpano.
\section{Extrauterino}
\begin{itemize}
\item {fónica:tra-u}
\end{itemize}
\begin{itemize}
\item {Grp. gram.:adj.}
\end{itemize}
Que está fóra do útero.
Que se realizou fóra do útero: \textunderscore gravidez extrauterina\textunderscore .
\section{Extravagância}
\begin{itemize}
\item {Grp. gram.:f.}
\end{itemize}
Qualidade daquelle ou daquillo que é extravagante.
Acto ou dito próprio de quem é extravagante.
\section{Extravaganciar}
\begin{itemize}
\item {Grp. gram.:v. t.}
\end{itemize}
\begin{itemize}
\item {Grp. gram.:V. i.}
\end{itemize}
Estragar com extravagâncias.
Dissipar: \textunderscore extravaganciar uma herança\textunderscore .
Dizer ou praticar extravagâncias.
\section{Extravagante}
\begin{itemize}
\item {Grp. gram.:adj.}
\end{itemize}
\begin{itemize}
\item {Grp. gram.:M.}
\end{itemize}
\begin{itemize}
\item {Proveniência:(De \textunderscore extravagar\textunderscore )}
\end{itemize}
Que anda ou está fóra do uso.
Que não faz parte de um todo da mesma natureza:«\textunderscore soldados extravagantes...\textunderscore »\textunderscore Cêrco de Mazagão\textunderscore , 89.«\textunderscore Sacerdotes extravagantes\textunderscore ». Sousa, \textunderscore Vida do Arceb.\textunderscore , I, 191. \textunderscore As nossas leis extravagantes\textunderscore .
Vago, solto.
Afastado da razão, do bom senso: \textunderscore ideias extravagantes\textunderscore .
Extraordinário, esquisito: \textunderscore um chapéu extravagante\textunderscore .
Estroina.
Valdevinos; dissipador; esbanjador.
Aquelle, que é estroina, esbanjador, perdulário.
\section{Extravagantemente}
\begin{itemize}
\item {Grp. gram.:adv.}
\end{itemize}
\begin{itemize}
\item {Proveniência:(De \textunderscore extravagante\textunderscore )}
\end{itemize}
Com extravagância.
\section{Extravagar}
\begin{itemize}
\item {Grp. gram.:v. i.}
\end{itemize}
\begin{itemize}
\item {Proveniência:(De \textunderscore extra...\textunderscore  + \textunderscore vagar\textunderscore )}
\end{itemize}
Andar fóra de certo número, de uma espécie, de uma ordem, de uma collecção, etc.
Estar disperso.
Divagar.
\section{Extravasação}
\begin{itemize}
\item {Grp. gram.:f.}
\end{itemize}
Acto ou effeito de extravasar.
\section{Extravasamento}
\begin{itemize}
\item {Grp. gram.:m.}
\end{itemize}
O mesmo que \textunderscore extravasação\textunderscore .
\section{Extravasante}
\begin{itemize}
\item {Grp. gram.:adj.}
\end{itemize}
\begin{itemize}
\item {Grp. gram.:F.}
\end{itemize}
Que extravasa.
A água, que extravasa ou que trasborda. Cf. Camillo, \textunderscore Sc. da Foz\textunderscore , 280.
\section{Extravasão}
\begin{itemize}
\item {Grp. gram.:f.}
\end{itemize}
O mesmo que \textunderscore extravasação\textunderscore .
\section{Extravasar}
\begin{itemize}
\item {Grp. gram.:v. t.}
\end{itemize}
\begin{itemize}
\item {Grp. gram.:V. i.}
\end{itemize}
\begin{itemize}
\item {Proveniência:(De \textunderscore extra...\textunderscore  + \textunderscore vaso\textunderscore )}
\end{itemize}
Fazer trasbordar.
Trasbordar; derramar-se abundantemente.
\section{Extravenado}
\begin{itemize}
\item {Grp. gram.:adj.}
\end{itemize}
\begin{itemize}
\item {Proveniência:(Do lat. \textunderscore extra\textunderscore  + \textunderscore vena\textunderscore )}
\end{itemize}
Que está fóra das veias, (o sangue).
\section{Extravertedura}
\begin{itemize}
\item {Grp. gram.:f.}
\end{itemize}
\begin{itemize}
\item {Utilização:Prov.}
\end{itemize}
\begin{itemize}
\item {Utilização:beir.}
\end{itemize}
\begin{itemize}
\item {Proveniência:(De \textunderscore extraverter\textunderscore )}
\end{itemize}
Líquido, que se extraverteu, que trasbordou.
\section{Extraverter}
\begin{itemize}
\item {Grp. gram.:v. t.  e  i.}
\end{itemize}
\begin{itemize}
\item {Utilização:Prov.}
\end{itemize}
\begin{itemize}
\item {Utilização:beir.}
\end{itemize}
\begin{itemize}
\item {Proveniência:(De \textunderscore extra...\textunderscore  + \textunderscore verter\textunderscore )}
\end{itemize}
O mesmo que \textunderscore extravasar\textunderscore .
\section{Extraviadamente}
\begin{itemize}
\item {Grp. gram.:adv.}
\end{itemize}
\begin{itemize}
\item {Proveniência:(De \textunderscore extraviar\textunderscore )}
\end{itemize}
Com extravio.
\section{Extraviador}
\begin{itemize}
\item {Grp. gram.:adj.}
\end{itemize}
\begin{itemize}
\item {Grp. gram.:M.}
\end{itemize}
Que extravia.
Aquelle que extravia.
\section{Extraviar}
\begin{itemize}
\item {Grp. gram.:v. t.}
\end{itemize}
\begin{itemize}
\item {Proveniência:(De \textunderscore extra...\textunderscore  + \textunderscore via\textunderscore )}
\end{itemize}
Tirar do caminho, desencaminhar: \textunderscore extraviar um mensageiro\textunderscore .
Fazer desapparecer: \textunderscore extraviar uma carta\textunderscore .
Illudir; perverter: \textunderscore extraviar moços incautos\textunderscore .
\section{Extravio}
\begin{itemize}
\item {Grp. gram.:m.}
\end{itemize}
Acto ou effeito de extraviar.
\section{Extrema}
\begin{itemize}
\item {Grp. gram.:f.}
\end{itemize}
O mesmo que \textunderscore estrema\textunderscore .
\section{Extremadamente}
\begin{itemize}
\item {Grp. gram.:adv.}
\end{itemize}
\begin{itemize}
\item {Proveniência:(De \textunderscore extremado\textunderscore )}
\end{itemize}
O mesmo que \textunderscore extremamente\textunderscore .
\section{Extremadela}
\textunderscore f.\textunderscore  (pon.)(V.estremadela)
\section{Extremado}
\begin{itemize}
\item {Grp. gram.:adj.}
\end{itemize}
Extraordinário; distinto; insigne; perfeito; selecto; apropriado.
\section{Extremados}
\begin{itemize}
\item {Grp. gram.:m. pl.}
\end{itemize}
Lavor antigo.
(Pl. de \textunderscore extremado\textunderscore )
\section{Extremadura}
\begin{itemize}
\item {Grp. gram.:f.}
\end{itemize}
O mesmo que \textunderscore estremadura\textunderscore .
\section{Extremamente}
\begin{itemize}
\item {Grp. gram.:adv.}
\end{itemize}
De modo extremo.
\section{Extremança}
\begin{itemize}
\item {Grp. gram.:f.}
\end{itemize}
\begin{itemize}
\item {Utilização:Ant.}
\end{itemize}
(V.estremança)
\section{Extremar}
\begin{itemize}
\item {Grp. gram.:v. t.}
\end{itemize}
\begin{itemize}
\item {Proveniência:(De \textunderscore extremo\textunderscore )}
\end{itemize}
Assignalar.
Abalisar.
Exaltar.
Sublimar; aperfeiçoar.
O mesmo que \textunderscore estremar\textunderscore .
\section{Extrema-uncção}
\begin{itemize}
\item {Grp. gram.:f.}
\end{itemize}
Uncção do moribundo com os santos óleos.
\section{Extremável}
\begin{itemize}
\item {Grp. gram.:adj.}
\end{itemize}
Que se póde extremar.
\section{Extreme}
\begin{itemize}
\item {Grp. gram.:adj.}
\end{itemize}
O mesmo que \textunderscore estreme\textunderscore .
\section{Extremenho}
\begin{itemize}
\item {Grp. gram.:m.  e  adj.}
\end{itemize}
O mesmo que \textunderscore estremenho\textunderscore .
\section{Extremidade}
\begin{itemize}
\item {Grp. gram.:f.}
\end{itemize}
\begin{itemize}
\item {Utilização:Fig.}
\end{itemize}
\begin{itemize}
\item {Proveniência:(Lat. \textunderscore extremitas\textunderscore )}
\end{itemize}
Qualidade daquillo que é extremo.
Fim.
Limite: \textunderscore a extremidade de uma quinta\textunderscore .
Orla; ponta: \textunderscore a extremidade de um vestido\textunderscore .
Miséria ou afflicção extrema: \textunderscore chegou a tal extremidade...\textunderscore 
\section{Extremo}
\begin{itemize}
\item {Grp. gram.:adj.}
\end{itemize}
\begin{itemize}
\item {Grp. gram.:M.}
\end{itemize}
\begin{itemize}
\item {Grp. gram.:Pl.}
\end{itemize}
\begin{itemize}
\item {Proveniência:(Lat. \textunderscore extremus\textunderscore )}
\end{itemize}
Final: \textunderscore na hora extrema\textunderscore .
Distante.
Elevado.
Extraordinário: \textunderscore extrema perversidade\textunderscore .
Que está no ponto mais afastado.
Que está no último grau, em grau elevado.
Perfeito.
O ponto mais distante.
Extremidade.
Termo.
Primeiro e último termo de uma proporção arithmética ou geométrica.
Carinho excessivo.
\section{Extremosamente}
\begin{itemize}
\item {Grp. gram.:adv.}
\end{itemize}
De modo extremoso.
\section{Extremoso}
\begin{itemize}
\item {Grp. gram.:adj.}
\end{itemize}
\begin{itemize}
\item {Proveniência:(De \textunderscore extremo\textunderscore )}
\end{itemize}
Que tem extremos.
Muito affectuoso; excessivo.
\section{Extrinsecamente}
\begin{itemize}
\item {Grp. gram.:adv.}
\end{itemize}
De modo extrínseco.
\section{Extrínseco}
\begin{itemize}
\item {Grp. gram.:adj.}
\end{itemize}
\begin{itemize}
\item {Proveniência:(Lat. \textunderscore extrinsecus\textunderscore )}
\end{itemize}
Exterior.
Que não é essencial.
Diz-se do valor convencional ou legal de uma moéda.
\section{Extrofia}
\begin{itemize}
\item {Grp. gram.:f.}
\end{itemize}
\begin{itemize}
\item {Proveniência:(Do gr. \textunderscore ex\textunderscore  + \textunderscore trophe\textunderscore )}
\end{itemize}
Deslocação de alguns òrgãos do corpo humano.
\section{Extrophia}
\begin{itemize}
\item {Grp. gram.:f.}
\end{itemize}
\begin{itemize}
\item {Proveniência:(Do gr. \textunderscore ex\textunderscore  + \textunderscore trophe\textunderscore )}
\end{itemize}
Deslocação de alguns òrgãos do corpo humano.
\section{Extrorso}
\begin{itemize}
\item {Grp. gram.:adj.}
\end{itemize}
\begin{itemize}
\item {Utilização:Bot.}
\end{itemize}
\begin{itemize}
\item {Proveniência:(Lat. \textunderscore extrorsus\textunderscore )}
\end{itemize}
Que se dirige de dentro para fóra, (falando-se da dehiscência dos lóculos da anthera)
\section{Extroversão}
\begin{itemize}
\item {Grp. gram.:f.}
\end{itemize}
\begin{itemize}
\item {Proveniência:(Do lat. \textunderscore extra\textunderscore  + \textunderscore versio\textunderscore )}
\end{itemize}
O mesmo que \textunderscore extrophia\textunderscore .
\section{Extrusão}
\begin{itemize}
\item {Grp. gram.:f.}
\end{itemize}
\begin{itemize}
\item {Utilização:Des.}
\end{itemize}
\begin{itemize}
\item {Proveniência:(Do lat. \textunderscore extrusus\textunderscore )}
\end{itemize}
Expulsão.
\section{Exuberância}
\begin{itemize}
\item {Grp. gram.:f.}
\end{itemize}
\begin{itemize}
\item {Proveniência:(Lat. \textunderscore exuberantia\textunderscore )}
\end{itemize}
Qualidade daquillo que é exuberante.
\section{Exuberante}
\begin{itemize}
\item {Grp. gram.:adj.}
\end{itemize}
\begin{itemize}
\item {Utilização:Fig.}
\end{itemize}
\begin{itemize}
\item {Proveniência:(Lat. \textunderscore exuberans\textunderscore )}
\end{itemize}
Que superabunda: \textunderscore vegetação exuberante\textunderscore .
Vivo, animado: \textunderscore mocidade exuberante\textunderscore .
Repleto.
\section{Exuberantemente}
\begin{itemize}
\item {Grp. gram.:adv.}
\end{itemize}
De modo exuberante.
\section{Exuberar}
\begin{itemize}
\item {Grp. gram.:v. t.}
\end{itemize}
\begin{itemize}
\item {Grp. gram.:V. i.}
\end{itemize}
\begin{itemize}
\item {Proveniência:(Lat. \textunderscore exuberare\textunderscore )}
\end{itemize}
Têr em excesso, superabundantemente.
Superabundar.
\section{Exúbere}
\begin{itemize}
\item {Grp. gram.:adj.}
\end{itemize}
\begin{itemize}
\item {Proveniência:(Do lat. \textunderscore ex\textunderscore  + \textunderscore uber\textunderscore )}
\end{itemize}
Desmamado.
\section{Exudrado}
\begin{itemize}
\item {Grp. gram.:adj.}
\end{itemize}
\begin{itemize}
\item {Utilização:Ant.}
\end{itemize}
(?):«\textunderscore ...sobreveio o grão porco... exudrado pela gran calma que fazia.\textunderscore »Fern. Lopes, \textunderscore Chrón. de D. Fern.\textunderscore 
(Por \textunderscore exsudorado\textunderscore , do lat. \textunderscore ex\textunderscore  + \textunderscore sudor\textunderscore ?)
\section{Êxul}
\begin{itemize}
\item {Grp. gram.:m.  e  adj.}
\end{itemize}
\begin{itemize}
\item {Proveniência:(Lat. \textunderscore exul\textunderscore )}
\end{itemize}
Expatriado.
Desterrado.
\section{Exular}
\begin{itemize}
\item {Grp. gram.:v. i.}
\end{itemize}
\begin{itemize}
\item {Proveniência:(Lat. \textunderscore exulare\textunderscore )}
\end{itemize}
Expatriar-se.
Viver fóra da pátria:«\textunderscore exulando por estranhas terras, a lyra de Garrett...\textunderscore »Camillo.
\section{Exulceração}
\begin{itemize}
\item {Grp. gram.:f.}
\end{itemize}
\begin{itemize}
\item {Proveniência:(Lat. \textunderscore exulceratio\textunderscore )}
\end{itemize}
Acto ou effeito de exulcerar.
\section{Exulcerante}
\begin{itemize}
\item {Grp. gram.:adj.}
\end{itemize}
\begin{itemize}
\item {Proveniência:(Lat. \textunderscore exulcerans\textunderscore )}
\end{itemize}
Que exulcera.
\section{Exulcerar}
\begin{itemize}
\item {Grp. gram.:v. t.}
\end{itemize}
\begin{itemize}
\item {Utilização:Fig.}
\end{itemize}
\begin{itemize}
\item {Proveniência:(Lat. \textunderscore exulcerare\textunderscore )}
\end{itemize}
Ulcerar superficialmente.
Desgostar, magoar.
\section{Exulcerativo}
\begin{itemize}
\item {Grp. gram.:adj.}
\end{itemize}
\begin{itemize}
\item {Proveniência:(Do lat. \textunderscore exulceratus\textunderscore )}
\end{itemize}
Que produz úlceras.
\section{Êxule}
\begin{itemize}
\item {Grp. gram.:m.  e  adj.}
\end{itemize}
O mesmo que \textunderscore êxul\textunderscore . Cf. Latino, \textunderscore Camões\textunderscore , 100.
\section{Exultação}
\begin{itemize}
\item {Grp. gram.:f.}
\end{itemize}
\begin{itemize}
\item {Proveniência:(Lat. \textunderscore exultatio\textunderscore )}
\end{itemize}
Acto de exultar.
\section{Exultante}
\begin{itemize}
\item {Grp. gram.:adj.}
\end{itemize}
Que exulta.
\section{Exultar}
\begin{itemize}
\item {Grp. gram.:v. i.}
\end{itemize}
\begin{itemize}
\item {Proveniência:(Lat. \textunderscore exultare\textunderscore )}
\end{itemize}
Alegrar-se muito.
Têr grande júbilo.
Alvoroçar-se.
\section{Exumação}
\begin{itemize}
\item {Grp. gram.:f.}
\end{itemize}
Acto de exumar.
\section{Exumar}
\begin{itemize}
\item {Grp. gram.:v. t.}
\end{itemize}
\begin{itemize}
\item {Utilização:Fig.}
\end{itemize}
\begin{itemize}
\item {Proveniência:(Do lat. \textunderscore ex\textunderscore  + \textunderscore humus\textunderscore )}
\end{itemize}
Desenterrar.
Escavar.
Descobrir por meio de investigações.
\section{Exundação}
\begin{itemize}
\item {Grp. gram.:f.}
\end{itemize}
Acto de exundar.
Inundação.
\section{Exundar}
\begin{itemize}
\item {Grp. gram.:v. i.}
\end{itemize}
\begin{itemize}
\item {Proveniência:(Lat. \textunderscore exundare\textunderscore )}
\end{itemize}
Correr abundantemente.
Trasbordar.
Inundar.
\section{Exutório}
\begin{itemize}
\item {Grp. gram.:m.}
\end{itemize}
\begin{itemize}
\item {Proveniência:(Do lat. \textunderscore exutus\textunderscore )}
\end{itemize}
Úlcera, feita e conservada artificialmente, para manter uma supuração permanente.
\section{Exuviabilidade}
\begin{itemize}
\item {Grp. gram.:f.}
\end{itemize}
Qualidade daquillo que é exuviável.
\section{Exuviável}
\begin{itemize}
\item {Grp. gram.:adj.}
\end{itemize}
\begin{itemize}
\item {Proveniência:(Do lat. \textunderscore exuviae\textunderscore )}
\end{itemize}
Que póde mudar de pelle, mantendo a mesma fórma.
\section{Ex-voto}
\begin{itemize}
\item {Grp. gram.:m.}
\end{itemize}
Quadro, imagem, etc., que se colloca em igreja ou ermida, em cumprimento de um voto.
(Loc. lat.)
\section{Eygreja}
\begin{itemize}
\item {Grp. gram.:f.}
\end{itemize}
Fórma archaica de \textunderscore igreja\textunderscore :«\textunderscore ...a eygreja he cousa de Deos...\textunderscore ». \textunderscore Leys\textunderscore , que Sancho I mandou tomar por apontamento.
\section{Eyra}
\begin{itemize}
\item {Grp. gram.:f.}
\end{itemize}
Espécie de gato do Paraguai.
\section{...ez}
\begin{itemize}
\item {fónica:ê}
\end{itemize}
\begin{itemize}
\item {Grp. gram.:suf. f.}
\end{itemize}
\begin{itemize}
\item {Proveniência:(Lat. \textunderscore ...ita\textunderscore )}
\end{itemize}
(designativo de qualidade ou estado em abstracto: \textunderscore belleza\textunderscore , \textunderscore pureza\textunderscore , \textunderscore grandeza\textunderscore )
\section{...eza}
\begin{itemize}
\item {fónica:ê}
\end{itemize}
\begin{itemize}
\item {Grp. gram.:suf. f.}
\end{itemize}
\begin{itemize}
\item {Proveniência:(Lat. \textunderscore ...ita\textunderscore )}
\end{itemize}
(designativo de qualidade ou estado em abstracto: \textunderscore belleza\textunderscore , \textunderscore pureza\textunderscore , \textunderscore grandeza\textunderscore )
\section{Empaludar}
\begin{itemize}
\item {Grp. gram.:v. t.}
\end{itemize}
\begin{itemize}
\item {Proveniência:(Do lat. \textunderscore palus\textunderscore , \textunderscore paludis\textunderscore )}
\end{itemize}
Infeccionar ou atacar com febre palustre.
\section{Empaludismo}
\begin{itemize}
\item {Grp. gram.:m.}
\end{itemize}
\begin{itemize}
\item {Proveniência:(De \textunderscore empaludar\textunderscore )}
\end{itemize}
Malária, resultante da picada de certos insectos.
Doença duradoira ou chrónica, resultante de se têr vivido em regiões pantanosas ou na vizinhança de águas estagnadas; sezonismo.
\section{Encinerar}
\begin{itemize}
\item {Grp. gram.:v. t.}
\end{itemize}
\begin{itemize}
\item {Proveniência:(Do lat. \textunderscore cinis\textunderscore , \textunderscore cineris\textunderscore )}
\end{itemize}
Reduzir a cinzas: \textunderscore encinerar um cadáver\textunderscore .
\section{Enduração}
\begin{itemize}
\item {Grp. gram.:f.}
\end{itemize}
\begin{itemize}
\item {Utilização:Fig.}
\end{itemize}
\begin{itemize}
\item {Proveniência:(Lat. \textunderscore induratio\textunderscore )}
\end{itemize}
Acto de endurecer; endurecimento de tecidos orgânicos.
Contumácia no mal.
\section{Endurado}
\begin{itemize}
\item {Grp. gram.:adj.}
\end{itemize}
\begin{itemize}
\item {Utilização:Fig.}
\end{itemize}
\begin{itemize}
\item {Proveniência:(Lat. \textunderscore induratus\textunderscore )}
\end{itemize}
O mesmo que [[endurecido|endurecer]].
Contumaz.
\section{Enflorescência}
\begin{itemize}
\item {Grp. gram.:f.}
\end{itemize}
\begin{itemize}
\item {Utilização:Bot.}
\end{itemize}
\begin{itemize}
\item {Proveniência:(Do lat. \textunderscore inflorescens\textunderscore )}
\end{itemize}
Conjunto das flôres de uma planta.
Conjunto dos órgãos e operações, que preparam ou realizam o desenvolvimento das flôres.
\section{Enflorescente}
\begin{itemize}
\item {Grp. gram.:adj.}
\end{itemize}
\begin{itemize}
\item {Proveniência:(Lat. \textunderscore inflorescens\textunderscore )}
\end{itemize}
Relativo á enflorescência.
\section{Engazeira}
\begin{itemize}
\item {Grp. gram.:f.}
\end{itemize}
\begin{itemize}
\item {Utilização:Bras}
\end{itemize}
\begin{itemize}
\item {Proveniência:(De \textunderscore engá\textunderscore )}
\end{itemize}
Árvore leguminosa da América.
\section{Engazeiro}
\begin{itemize}
\item {Grp. gram.:m.}
\end{itemize}
O mesmo que \textunderscore engazeira\textunderscore . Cf. Crespo, \textunderscore Miniaturas\textunderscore , 116.
\section{Enviscerar}
\begin{itemize}
\item {Grp. gram.:v. t.}
\end{itemize}
\begin{itemize}
\item {Proveniência:(Lat. \textunderscore inviscerare\textunderscore )}
\end{itemize}
\end{document}